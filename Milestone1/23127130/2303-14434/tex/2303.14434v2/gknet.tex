%% ****** Start of file apsguide4-2.tex ****** %
%%
%%   This file is part of the APS files in the REVTeX 4.2 distribution.
%%   Version 4.2b of REVTeX, December 2018.
%%
%%   Copyright (c) 2019 The American Physical Society.
%%
%%   See the REVTeX 4.2 README file for restrictions and more information.
%%
\documentclass[twocolumn,secnumarabic,amssymb,nobibnotes,aps,prd,superscriptaddress,nofootinbib]{revtex4-2}
%\usepackage{acrofont}%NOTE: Comment out this line for the release version!

\newcommand{\classoption}[1]{\texttt{#1}}
\newcommand{\macro}[1]{\texttt{\textbackslash#1}}
\newcommand{\m}[1]{\macro{#1}}
\newcommand{\env}[1]{\texttt{#1}}
\setlength{\textheight}{9.5in}


\usepackage[utf8]{inputenc} % allow utf-8 input
\usepackage[T1]{fontenc}    % use 8-bit T1 fonts

\usepackage{url}            % simple URL typesetting
\usepackage{booktabs}       % professional-quality tables
\usepackage{amsfonts}       % blackboard math symbols
\usepackage{nicefrac}       % compact symbols for 1/2, etc.
\usepackage{microtype}      % microtypography
\usepackage{amsmath}        % more mathematics!
\usepackage[capitalise]{cleveref}       % more clever refs
\usepackage{chemformula}    % chemical formulas
\usepackage[separate-uncertainty=true,exponent-product=\cdot]{siunitx}        % nicely typeset units
\DeclareSIUnit\angstrom{\text{\AA}} % angstrom is not a SI unit :clown_face:
% \usepackage{subfig}
\usepackage{graphicx}
\usepackage{xcolor}
\usepackage{mathtools}
\usepackage{todonotes}
\usepackage{xspace}
% \usepackage[disable]{todonotes}
\usepackage{tikz}
\usetikzlibrary[arrows.meta,bending]
\usetikzlibrary{positioning}


\usepackage[normalem]{ulem}

% auto-substituting snippets below

% General
\newcommand{\scare}[1]{\lq#1\rq} 
\newcommand{\etal}{et~al.}
\newcommand{\el}{{\,\ldots}\xspace}
\newcommand{\yes}{\checkmark}
\newcommand{\no}{$\times$}


\newcommand{\symset}[1]{\mathcal{#1}}  % symbol of a set
% Shorthand

\newcommand{\zro}{\ch{ZrO2}}

% GK
\newcommand{\tk}{\boldsymbol{\kappa}}
% ... J
\newcommand{\J}{{\bf J}}
% CC \newcommand{\Jfull}{\boldsymbol{J}_\text{full}}
\newcommand{\Jfull}{\J_\text{Hardy}}
\newcommand{\Jgauge}{{\bf P}}
\newcommand{\Jint}{\J_\text{int}}
\newcommand{\Jdiff}{\J_\text{disp}}
\newcommand{\Jintrt}{\J_{\text{int}}^{\boldsymbol{r}(t)}}
\newcommand{\Jconv}{\J_{\text{convective}}}
\newcommand{\Jmpnn}{\J_{\text{Hardy}}^{\text{MPNN}}}
\newcommand{\Jdft}{\J_{\text{virial}}}
\newcommand{\Jfan}{\J_{\text{Fan}}}
\newcommand{\Junfolded}{\Jhardy^{\text{unfolded}}}
\newcommand{\Jhardy}{\J_{\text{Hardy}}}
\newcommand{\Jvirial}{\J_{\text{virial}}}

% ... densities
\newcommand{\densj}{{\bf j}}
\newcommand{\dense}{e}
\newcommand{\densgauge}{{\bf p}}
\newcommand{\localf}[1]{\Delta(#1)}
\newcommand{\bondf}[2]{\Lambda_{#1}(#2)}


% MD
\newcommand{\phasp}[2]{\Gamma_{#1}^{#2}} % phase-space point
\newcommand{\fullphasp}{\boldsymbol{\Gamma}} % phase-space
\newcommand{\ensemble}{\mathcal{E}}

% Graphs/MPNN
\newcommand{\dur}[3]{\frac{\partial U_{#1}}{\partial \R_{#2#3}}}
\newcommand{\indur}[3]{\partial U_{#1} / \partial \R_{#2#3}}

\newcommand{\state}[2]{\boldsymbol{s}_{#1}^{#2}}
\newcommand{\msg}[2]{\boldsymbol{m}_{#1}^{#2}}
\newcommand{\nbh}[1]{\symset{N}(#1)}
\newcommand{\msgf}[1]{\boldsymbol{M}_{#1}}
\newcommand{\updatef}[1]{\boldsymbol{F}_{#1}}
\newcommand{\edge}[2]{{\bf R}_{#1#2}}

\newcommand{\interactions}{M}
\newcommand{\cutoff}{r_{\text{c}}}
\newcommand{\effcutoff}{\cutoff^{\text{eff}}}

\newcommand{\graph}{\mathcal{G}}
\newcommand{\vertices}{\mathcal{V}}
\newcommand{\edges}{\mathcal{E}}

\newcommand{\patch}[1]{\mathcal{P}(#1)}

% Statistics
\newcommand{\sampled}{\,\sim\,}

% Math + Physics general
\newcommand{\grad}{\nabla}

\newcommand{\vR}{{\bf r}}
% \newcommand{\R}{\boldsymbol{R}}
\newcommand{\R}{{\bf R}}
\newcommand{\Rr}{{\bf R}^0}
\newcommand{\Rnorm}{\hat{\bf R}}
\newcommand{\Rrm}{{\bf R}^{0,\text{MIC}}}
\newcommand{\Rm}{{\bf R}^{\text{MIC}}}

\newcommand{\U}{\boldsymbol{u}}
% \newcommand{\V}{\boldsymbol{V}}
\newcommand{\V}{\dot{\bf R}}
\newcommand{\momentum}{{\bf p}}
\newcommand{\F}{{\bf F}}
\newcommand{\Bary}{{\bf B}}
\newcommand{\intd}{\text{d}}
\newcommand{\B}[1]{\mathbf{#1}}
\newcommand{\integral}[1]{\int_{#1} \, \text{d}}
\newcommand{\limitintegral}[2]{\int_{#1}^{#2} \, \text{d}}
\newcommand{\dirac}[1]{\delta(#1)}
\newcommand{\defas}{\coloneqq}  % needs \usepackage{mathtools}
\newcommand{\defdas}{\eqqcolon}
\newcommand{\mbeq}{\overset{!}{=}}


\newcommand{\curlyset}[2]{\{\, #1\, :\, #2 \, \}}

\newcommand{\totaldiff}[1]{\frac{\intd}{\intd #1}}
\newcommand{\wholes}{\mathbb{Z}}

\newcommand{\zerovec}{\boldsymbol{0}}

\newcommand{\bigo}[1]{O(#1)}

% \newcommand{\magnitude}[1]{\bigl|#1\bigr|}
\newcommand{\magnitude}[1]{|#1|}

% Solid State
\newcommand{\stress}{\boldsymbol{\sigma}}
\newcommand{\stressc}{\sigma}
\newcommand{\strain}{\boldsymbol{\epsilon}}
\newcommand{\strainc}{\epsilon}

\newcommand{\virial}{\boldsymbol{\Omega}}


% Periodic Systems
\newcommand{\basis}{{\bf b}}
\newcommand{\offset}{{\bf n}}

\newcommand{\Rsc}{\symset{R_{\text{sc}}}}
\newcommand{\Rrep}{\symset{R_{\text{rep}}}}
\newcommand{\Rall}{\symset{R_{\text{all}}}}
\newcommand{\Basis}{\symset{B}}

% Extras for paper
\newcommand{\Runf}{\symset{R_{\text{unf}}}}
\newcommand{\Jpot}{\J_\text{pot}}
\newcommand{\Junf}{\J_\text{pot}^{\text{unfolded}}}
\renewcommand{\Jconv}{\J_\text{conv}}
\renewcommand{\Jfull}{\J}
\renewcommand{\Jfan}{\J_\text{pot}^{\text{local}}}
\renewcommand{\Jmpnn}{\J_{\text{pot}}^{\text{semi-local}}}
\newcommand{\nbht}[1]{N^\interactions\!(#1)}
\newcommand{\Rng}{\boldsymbol{r}^{\dagger}}


% abbreviations

\newcommand{\abr}[1]{\uppercase{#1}}
\newcommand{\nn}{\abr{nn}\xspace}
\newcommand{\nns}{\abr{nn}s\xspace}
\newcommand{\mpnn}{\abr{mpnn}\xspace}
\newcommand{\mpnns}{\abr{mpnn}s\xspace}
\newcommand{\mlp}{\abr{MLP}\xspace}
\newcommand{\mlps}{\abr{MLP}s\xspace}
\newcommand{\iaps}{\abr{IAP}s\xspace}
\newcommand{\ffs}{\abr{ff}s\xspace}
\newcommand{\ff}{\abr{ff}\xspace}
\newcommand{\ad}{\abr{ad}\xspace}
\newcommand{\gk}{\abr{gk}\xspace}
\newcommand{\hfacf}{\abr{hfacf}\xspace}
\newcommand{\md}{\abr{md}\xspace}
\newcommand{\dft}{\abr{dft}\xspace}
\newcommand{\pes}{\abr{pes}\xspace}
\newcommand{\bo}{\abr{bo}\xspace}
\newcommand{\mic}{\abr{mic}\xspace}

\let\min\undefined
\DeclareMathOperator*{\min}{min}

% comments
\definecolor{yellow1}{HTML}{FADF63}
\definecolor{yellow2}{HTML}{E6AF2E}
\definecolor{light_red}{HTML}{FE404A}


\definecolor{extra_0}{HTML}{66c2a5}
\definecolor{extra_1}{HTML}{ffd92f}
\definecolor{extra_2}{HTML}{8da0cb}
\definecolor{extra_3}{HTML}{e78ac3}
\definecolor{extra_4}{HTML}{a6d854}
\definecolor{extra_5}{HTML}{fc8d62}
\definecolor{accent_2}{HTML}{006C66}

\definecolor{signalred}{HTML}{EE4B2B}


\newcommand{\FK} [1] 
{\todo[inline,backgroundcolor=yellow1,size=\small, bordercolor=white]{{\bf FK:} #1}}
\newcommand{\flocomment} [1] 
{\todo[backgroundcolor=yellow1,size=\small, bordercolor=white]{{\bf FK:} #1}}
\newcommand{\floanswer} [1] 
{\todo[inline,backgroundcolor=green,size=\small, bordercolor=white]{{\bf FK answer:} #1}}
\newcommand{\floedit}[1]{{ \color{yellow2} \textbf{#1} }}
% \newcommand{\floout}[1]{{ \color{yellow2} \sout{\textbf{{#1}}} }}
\newcommand{\flout}[1]{{ \color{signalred} #1 }}
\newcommand{\CC} [1] 
{\todo[inline,backgroundcolor=yellow1,size=\small, bordercolor=white]{{\bf CC:} #1}}
\newcommand{\mrc}[1]%
{\todo[inline,backgroundcolor=orange,size=\small, bordercolor=white]{{\bf MR:} #1}}
\newcommand{\mr}[1]{\begingroup \color{orange}#1\endgroup}

\newcommand{\marcomment} [1] 
{\todo[inline,backgroundcolor=orange,size=\small, bordercolor=white]{{\bf Marcel:} #1}}

\newcommand{\CITE} [1] 
{\todo[inline,backgroundcolor=light_red,size=\small, bordercolor=white]{{\bf CITE:} #1}}

\newcommand{\SM}{Supp. Mat.\xspace}

\usepackage{natbib}
\bibliographystyle{apsrev4-2}

\newcommand{\nomadaff}{\affiliation{The NOMAD Laboratory at the FHI of the Max-Planck-Gesellschaft and IRIS Adlershof of the Humboldt Universit{\"a}t zu Berlin, Germany}}

\begin{document}

\title{Heat flux for semi-local machine-learning potentials}

\begin{abstract}
    The Green-Kubo (GK) method is a rigorous framework for heat transport simulations in materials.
    However, it requires an accurate description of the potential-energy surface and carefully converged statistics.
    Machine-learning potentials can achieve the accuracy of first-principles simulations while allowing to reach well beyond their simulation time and length scales at a fraction of the cost.
    In this paper, we explain how to apply the GK approach to the recent class of message-passing machine-learning potentials, which iteratively consider semi-local interactions beyond the initial interaction cutoff. 
    We derive an adapted heat flux formulation that can be implemented using automatic differentiation without compromising computational efficiency.
    The approach is demonstrated and validated by calculating the thermal conductivity of zirconium dioxide across temperatures.
\end{abstract}

\author{Marcel F. Langer}
\email[Corresponding author: ]{mail@marcel.science}
\affiliation{Machine Learning Group, Technische Universit{\"a}t Berlin, 10587 Berlin, Germany}
\affiliation{BIFOLD -- Berlin Institute for the Foundations of Learning and Data, Berlin, Germany}
\nomadaff

\author{Florian Knoop}
\email[Corresponding author: ]{florian.knoop@liu.se}
\affiliation{Theoretical Physics Division, Department of Physics, Chemistry and Biology (IFM), Linköping University, SE-581 83 Linköping, Sweden}
\nomadaff

\author{Christian Carbogno}
\nomadaff

\author{Matthias Scheffler}
\nomadaff

\author{Matthias Rupp}
\nomadaff
\affiliation{Department of Computer and Information Science, University of Konstanz, 78464 Konstanz, Germany}
\affiliation{Materials Research and Technology Department, Luxembourg Institute of Science and Technology (LIST), Belvaux, Luxembourg}

\date{\today}%

    \maketitle

The thermal conductivity tensor $\tk$ describes the ability of a material to conduct heat when exposed to a temperature gradient.
Its computational prediction is of great interest for the design of novel high-performance materials which are needed, for example, as thermal barrier coatings in engines~\cite{ecl2008t}, or thermoelectrics for waste heat recovery~\cite{st2008t}.
Such materials often feature complex structure and strongly anharmonic potential-energy surfaces (\pes)~\cite{zldk2014t,kpsc2020t}. This implies the need to evaluate $\tk$ with the Green-Kubo (\gk{}) method~\cite{o1931At,o1931Bt,g1952t,k1957t,kyn1957t}.

In the \gk{} approach, $\tk$ is expressed in terms of the integral of the autocorrelation function of the instantaneous heat flux $\J(t)$ as observed in equilibrium molecular dynamics (\md{}) simulations,
\begin{align}
    \tk (T, p)
    = \frac{1}{k_{\rm B} T^2 V }
    \lim_{t \rightarrow \infty}
    \int_0^t \,
    \intd \tau\,
    \left\langle
    \J (\tau) \otimes \J (0) 
    \right\rangle_{T,p}
    ~,
    \label{eq:Kubo1}
\end{align}
where $k_{\rm B}$ is the Boltzmann constant, $V$ the simulation cell volume, and $\left\langle\cdot\right\rangle_{T,p}$ denotes an ensemble average at temperature $T$ and pressure $p$.

High-accuracy \md simulations can be performed using density-functional theory (\dft) when the exchange-correlation approximation is reliable~\cite{thxy2022p}. For the evaluation of \cref{eq:Kubo1}, this approach~\cite{mub2015t,crs2017t} suffers from its numerical costs which limits the system sizes and time scales that can be treated, and therefore requires additional denoising and extrapolation approaches~\cite{crs2017t,meb2020t,kcs2022t}.
The alternative, so far, was the use of semi-empirical force fields (\ffs)~\cite{g2011p}.
Here, the interatomic interactions are described by a physically-motivated analytical equation that includes free parameters which are fitted to experimental or \emph{ab initio} results.
This classical \ff approach has been very successful, as it enables a proper consideration of the ensemble averages needed in \cref{eq:Kubo1}. However, the restricted flexibility of \ffs may limit their generality and ability to model novel materials.

A new, more general class of \ffs is the family of machine-learning potentials (\mlps) which leverage techniques like neural networks (\nns)~\cite{wd2004q,lgs2004q,bp2007q,bpkc2010q,uctm2021q,pt2021q}.
\mlps offer, in principle, unrestricted flexibility, but are limited to the mechanisms and information that are provided by their training data.
In local \mlps, linear scaling with system size is achieved by using the short-ranged nature of chemical bonding~\cite{pk2005p} to 
decompose the total energy into contributions that only depend on local atomic environments.
However, a strict locality assumption limits the flexibility and therefore accuracy of such \mlps.
Some \ffs therefore include explicit long-range electrostatic and van der Waals interactions~\cite{rg1991p,hd2001p,cddw2009p,shgd2016p,kfgb2021Bq}.
%
Semi-local \mlps{}~\cite{gsvd2017q,sktm2017q,sstm2018q,um2019q,kgg2020q,kbg2021q,ucsm2021q,bmsk2022q,bkoc2022a,bbkc2022a,blcd2022a,fum2022a} build up longer-range correlations iteratively from local ones through message-passing mechanisms, thereby preserving linear scaling with system size. They have recently emerged as an alternative to strictly local \mlps and have shown promising performance in benchmark settings and first applications~\cite{bmsk2022q,bkoc2022a,sggm2022q,ustm2022a,co2022a}.

While local \mlps{} have been used to investigate thermal transport via \gk{}~\cite{sdbb2012q,mcld2020q,knys2019q,lqzg2021q,lll2020q,llll2020q,qpwy2019q,vkjk2021q}, more recent semi-local \mlps{} have not yet been applied, partially because a heat flux formulation that incorporates message-passing mechanisms was lacking.
In this work, we fill that gap and extend the \gk{} approach to semi-local potentials.
To this end, we derive a formulation of the heat flux that explicitly accounts for semi-local interactions, finding that the resulting thermal conductivity significantly differs from a purely local form.
While the computation of this heat flux scales quadratically with system size, we show that an alternative yet equivalent form based on an extended auxiliary system can be introduced, leading to overall linear scaling and straightforward practical implementation of the approach via automatic differentiation (\ad{})~\cite{griewank2008,bprs2017m}.
Using the SchNet message-passing neural network (\mpnn{}) \cite{sktm2017q,sstm2018q}, we demonstrate the accuracy and feasibility of large-scale semi-local \mlp{} thermal conductivity calculations for zirconia (ZrO$_2$), an oxide known 
for its strongly anharmonic \pes~\cite{fpf2001t,clws2014t}. 

For any potential function $U(\{\R_J\})$ that can be decomposed into atomic contributions $U = \sum_I U_I(\{\R_J\})$, where $\{\R_J\}$ denotes the set of all atomic positions $\R_J$, the heat flux is given by the classical equivalent of a formula by Hardy~\cite{h1963t}, which we re-derive in the \SM to explicitly account for periodic boundary conditions.

This yields the full \scare{Hardy} heat flux
\begin{align}
    \Jfull &= \sum_{\substack{I \in \Rsc \\ J \in \Rall}} \left(\R_{JI} \left(\dur{I}{J}{} \cdot \V_J\right) \right) 
        + \sum_{I \in \Rsc} E_I \V_I \label{eq:J_general}\\
        &\defdas \Jpot + \Jconv\, , \label{eq:J_pot+conv}
\end{align}
where $\V_I$ is the velocity of an atom, and $E_I = U_I + \frac{1}{2} m_I \V_I$ is the total energy per atom. For atom-pair vectors, we adopt the convention $\R_{IJ} = \R_J - \R_I$. $\Rsc$ indicate the atoms in the simulation cell, while $\Rall$ enumerates the full, infinite, bulk system.
The nomenclature for heat flux contributions and the relation of \cref{eq:J_general} to \dft formulations are further discussed in the \SM

As this work considers \ffs and \mlps that explicitly define atomic potential energies $U_I$, total atomic energies $E_I$ and consequently $\Jconv$ can be computed in a straightforward manner.
We therefore only discuss the more involved computation of $\Jpot$ in the following, whereas the heat flux used for calculating $\tk$ is always equivalent to the full flux given by \cref{eq:J_general}.

Evaluating $\Jpot$ requires disentangling the contributions of every atom, including those in the bulk, to every atomic potential energy $U_I$.
This can be challenging for non-pairwise, many-body potentials, leading to the development of specialized expressions for different \ffs~\cite{c2006t,at2011t,tno2008t,fpdh2015t,smko2019t,bbw2019t}, many of which were recently unified and shown to be equivalent to \cref{eq:J_general} by Fan~\etal~\cite{fpdh2015t}.
Their work is based on the insight that translational invariance requires that the potential is computed only from atom-pair vectors $\R_{IJ}$, which provides a convenient basis to separate the computation of each $U_I$ into distinct sets of inputs.

Combined with the introduction of an interaction cutoff radius $\cutoff$ and atomic neighborhoods $\nbh{I} = \curlyset{\R_J}{\magnitude{\R_{IJ}} \leq \cutoff, \R_J \in \Rall}$, this leads to the notion of a \emph{local} potential $U_I = U_I(\curlyset{\R_{IJ}}{J \in \nbh{I}})$ and a corresponding local formulation of $\Jpot$,
\begin{equation}
    \Jfan = \sum_{I \in \Rsc} \sum_{J \in \nbh{I}} \R_{JI} \left( \dur{I}{I}{J} \cdot \V_J \right) \, . \label{eq:J_fan}
\end{equation}
As each $\R_{IJ}$ only contributes to one $U_I$, derivatives of $U$ naturally separate into atomic contributions $\indur{I}{I}{J} = \indur{}{I}{J}$. The resulting expression can be implemented efficiently with \ad, as detailed in the \SM

While being exact for \emph{local} potentials, this formulation of the heat flux does not apply to the \emph{semi-local} case. In such potentials, longer-range interactions are introduced \emph{without} explicitly increasing the cutoff $\cutoff$ by building them up iteratively:
Neighboring atoms are allowed to exchange information for a fixed number of iterations $\interactions$~\cite{gsvd2017q}.
Neighboring environments up to an effective cutoff radius $\effcutoff = \interactions \cutoff$ therefore become correlated;
atomic potential energies $U_I$ acquire a dependence on atom-pair vectors \emph{outside} of their immediate neighborhoods $\nbh{I}$, rendering \cref{eq:J_fan} inapplicable.

To see this, we employ a description in terms of a graph $\graph$, where vertices $\vertices$ are identified with atoms $I$ in the simulation cell, and connected via edges $\edges$ labelled by atom-pair vectors $\R_{IJ}$ if they lie within $\nbh{I}$.
Semi-local \mlps{} then act on this graph by propagating information between vertices (see \SM{}). Interactions outside of the simulation cell are therefore mapped back into it, and explicit replicas are not constructed.

Assuming that $\effcutoff$ is chosen such that the minimum image convention (\mic) is applicable, $\Jpot$ in \cref{eq:J_general} can be rewritten (see \SM) as
\begin{align}
    \Jmpnn =\!\!\!\!\!\! \sum_{\substack{I \in \vertices \\ J \in \vertices \\ K \in \nbh{J}}}\!\!\!\! \Rm_{JI}\! \left( \Bigl(\dur{I}{K}{J} - \dur{I}{J}{K} \Bigr) \cdot \V_J \right) \, , \label{eq:J_graph}
\end{align}
generalizing $\Jfan$ to semi-local \mlps{}. In the case of $\interactions = 1$, this form reduces to \cref{eq:J_fan}.

\begin{figure}
  \centering
  \includegraphics[scale=0.6]{img/heat_flux_timings.pdf}
  \caption{Computation time per timestep for different system sizes $N$ for zirconia, evaluating a SchNet \mpnn{}, for different heat flux formulations on a single Tesla Volta V100 32GB GPU. To estimate the asymptotic scaling, a function proportional to $N^x$ has been fitted to the results for large $N$. Note that on this setup with limited memory, the truly asymptotic limit cannot be reached.
}
  \label{fig:heat_flux_timings}
\end{figure}

\Cref{eq:J_graph} reflects the standard construction of semi-local \mlps{}; a double sum over all atoms is required and its evaluation formally scales quadratically with system size. As shown in~\cref{fig:heat_flux_timings}, a direct implementation of this form is therefore impractical. While force predictions for a semi-local \mlp{} based on the SchNet architecture~\cite{sktm2017q,sstm2018q} remain below \SI{100}{ms} for all system sizes investigated, the unoptimized calculation of the heat flux dominates the computational cost by several orders of magnitude at the system sizes required for the \gk{} method.

If the analytical form of $U_I$ were known, a lower-scaling evaluation of the heat flux might be accessible by deriving and implementing analytical derivatives.
Modern \mlps{}, however, typically rely on \ad{}~\cite{griewank2008,bprs2017m} for efficiently computing derivatives without requiring detailed information on the functional form of the \mlp{}.

To take advantage of this, we now derive an adapted form of the heat flux that preserves the implicit treatment of interactions beyond local environments to retain the computational efficiency of semi-local \mlps{}, while explicitly attributing all contributions to $U_I$ to bulk positions for $\Jpot$ in \cref{eq:J_general}.
%
This is achieved by constructing an extended simulation cell that explicitly includes all replicas that interact with atoms in the simulation cell, inspired by previous approaches which did not consider \ad{}~\cite{tpm2009t,khc2012t}. The graph representation is then constructed \emph{without} periodic boundary conditions, yielding \scare{unfolded} vertices $\Runf$.
The potential energy obtained by summing over the original simulation cell, $U$, remains unchanged in this construction.
This allows to retain the small cutoffs needed for efficiency, while enabling \ad{} to compute the required derivatives.

With this construction, \cref{eq:J_general} can be rewritten as
\begin{align}
 \Junf &= \sum_{J \in \Runf} \frac{\partial \Bary}{\partial \R_J} \cdot \V_J \nonumber\\
  &\quad- \sum_{J \in \Runf}
    \left(\R_J \left(\dur{}{J}{} \cdot \V_J\right) \right) \, , \label{eq:J_unf}
\end{align}
introducing the energy barycenter
$\Bary = \sum_{I \in \Rsc}\nolimits \R_I U_I$, where the positions $\R_I$ are treated as pre-factors and not included in the partial derivative. The dot product is taken between denominator and velocity.
%
Writing the heat flux as the derivative of a vector $\Bary$ and a scalar $U$, as opposed to a high-dimensional Jacobian, ensures that these derivatives can be readily computed with \ad{}, incurring the same asymptotic computational cost as the calculation of $U$ and $\Bary$, which is proportional to $\magnitude{\Runf}$.
Since the number of additional positions is proportional only to the surface area of the simulation cell and the number of interactions $\interactions$, the overall asymptotic linear scaling is restored, with $\magnitude{\Runf} \propto N+N^{2/3}$ (see \cref{fig:heat_flux_timings}). 

To validate the approach, we benchmark the performance of a semi-local \mlp, in particular the SchNet~\cite{sktm2017q,sstm2018q} \mpnn architecture, for \gk calculations on zirconia (ZrO$_2$) and compare to results obtained with size-extrapolated \emph{ab initio} \gk~\cite{crs2017t}, as well as \gk with a local \mlp{}~\cite{vkjk2021q}, and experimental measurements~\cite{rwpm1998t,bflm2000t,mlld2004t}.

Training and validation data were generated using \emph{ab initio} \md in the $NpT$ ensemble, with four different trajectories heating up an initially tetragonal simulation cell with $96$ atoms to target temperatures \SI{750}{K}, \SI{1500}{K}, \SI{2250}{K} and \SI{3000}{K}.
In total, \num{10000} single-point calculations were performed using FHI-aims~\cite{FHI-aims} and FHI-vibes~\cite{FHI-vibes}, using the PBEsol~\cite{przb2008t} functional and otherwise following the computational approach of Ref.~\cite{crs2017t}.

On this data, we train a SchNet \mpnn{}, implemented in SchNetPack~\cite{sktm2018q}, with cutoff radius $\cutoff=\SI{5}{\angstrom}$. We choose an interaction depth $\interactions = 2$ leading to an effective cutoff of \SI{10}{\angstrom}. 
In line with recent findings by others~\cite{bkoc2022a}, we find this to be sufficient, as test set error does not significantly decrease for higher values of $\interactions$ or $\cutoff$.
Further details on the training procedure, choice of hyperparameters, and testing of the \mlp can be found in the \SM

We find that this simple approach yields a \mlp capable of describing the dynamics in monoclinic and tetragonal zirconia up to temperatures of approximately \SI{2000}{K}. In this temperature range, the anharmonic vibrational density of states matches that obtained from \dft. 
Beyond \SI{2000}{K}, the oxygen atoms become more mobile and different types of dynamical events are observed, in particular exchange-type oxygen diffusion. This behavior is also present in the training data in line with recent literature~\cite{tw2022t}, although slightly different diffusion events are observed given the smaller simulation cells and trajectory lengths. 
When diffusion increases at higher temperatures, the \mlp becomes unstable. This might be due to the limited amount of training data for these processes, especially for thermodynamic conditions close to the tetragonal-to-cubic phase transition. These observations suggest that an accurate description of defect formation is necessary to investigate zirconia above \SI{2000}{K}, which is beyond the scope of the current work.

\begin{figure}[t]
  \centering
  \includegraphics[scale=0.6]{img/heat_flux_variants.pdf}
  \caption{
  Comparison of the integral of the heat flux autocorrelation function for different formulations of the heat flux.
  The efficient re-formulation of the heat flux $\J^{\text{unfolded}}$ is equivalent to the full heat flux $\J^{\text{semi-local}}$, but not to
  $\J^{\text{local}}$, which neglects semi-local interactions.
  Results are given for an \mpnn{} with $\interactions{=\,}2$ and zirconia at \SI{300}{K} in the monoclinic phase (top) and \SI{1400}{K} in the tetragonal phase (bottom) for a simulation cell with 768 atoms.
  Shaded regions indicate standard error across eleven trajectories.
  }
  \label{fig:heat_flux_variants}
\end{figure}


\Cref{fig:heat_flux_variants} compares our efficient implementation with the full semi-local heat flux, as well as the purely local heat flux formulation. 
Due to the high computational cost of the unoptimized implementation, we use a small simulation cell with $N=\num{768}$ atoms, and rely on the noise reduction scheme introduced in Ref.~\cite{kcs2022t}.
The results confirm that our implementation $\Junf$ is equivalent to the semi-local heat flux $\Jmpnn$, while the local flux $\Jfan$ is not, underestimating the thermal conductivity by approximately \SI{40}{\percent} due to missing interactions beyond $\interactions = 1$.
A similar effect has been observed when formulations applicable to pairwise additive potentials are used for many-body force fields~\cite{bbw2019t,smko2019t}.

\begin{figure}[t]
  \centering
  \includegraphics[scale=0.6]{img/kappa_vs_temperature.pdf}
  \caption{Thermal conductivity across temperatures computed with an \mpnn{} using $\interactions{=\,}2$ message-passing steps
  and experimentally determined lattice parameters~\cite{ps1969t,kh1998t},
  compared with
  another MLP without extrapolation~\cite{vkjk2021q},
  size-extrapolated \emph{ab initio} \gk~\cite{crs2017t},
  and
  experimental measurements~\cite{rwpm1998t,bflm2000t,mlld2004t}.
  Error bars are shown as given in the respective publications, the ones in the present work reflect the standard error across eleven trajectories. Letters \scare{t} and \scare{m} indicate results for the tetragonal and monoclinic phase, respectively.
  }
  \label{fig:kappa_vs_temperature}
\end{figure}

Enabled by computationally efficient access to $\J$ for semi-local \mlps{}, we then predict the thermal conductivity of zirconia across temperatures. Since the focus of the present work is the heat flux, we do not treat the thermodynamics of zirconia with the \mlp{}, but use experimentally determined lattice parameters~\cite{ps1969t,kh1998t} to account for lattice expansion. At \SI{1400}{K}, both phases are investigated, as the monoclinic phase is sufficiently stable during the course of the simulations, which consist of eleven trajectories of \SI{1}{ns} each, with a $N=\num{1500}$ simulation cell. These settings yield fully size- and time-converged results (see \SM{} for details).

The results presented in \cref{fig:kappa_vs_temperature} are in good agreement with both experimental measurements in the monoclinic phase, and theoretical \mlp{} predictions in the monoclinic and tetragonal phases.
As this work uses similar lattice parameters and the same exchange-correlation functional as the work by Verdi~\etal{}~\cite{vkjk2021q}, the observed close agreement is to be expected.
Remaining differences between the \mlp results may be due to larger simulation cells used in the present work, enabled by the favorable scaling of computational cost due to the efficient heat flux implementation, and the semi-local nature of the employed \mpnn.
Compared to experiment, both \mlps are found to systematically underestimate $\kappa$ by approximately \SIrange{10}{20}{\percent}, which may be related to the intrinsic approximation of a finite-range \mlp{}, or the underlying density functional approximation.

Larger differences are observed with the size-extrapolated \emph{ab initio} \gk results reported by Carbogno~\etal~\cite{crs2017t}, which, however, were computed for the tetragonal phase at all temperatures. Additionally, due to the high computational cost of first-principles calculations, only three trajectories of \SI{60}{ps} each were used, which is reflected in the larger statistical error. 

We conclude that the adapted GK approach for semi-local \mlps{} introduced in this work can successfully and efficiently predict the thermal conductivity of zirconia across temperatures, using \num{10000} first-principles calculations in total. Despite a moderate system size of 96 atoms for training, fully size-converged results were obtained without requiring additional extrapolation schemes.

In summary, we have demonstrated the feasibility of applying \ad{}-based semi-local \mlps{} to the prediction of thermal conductivities with the \gk method.
For this, we investigated the impact of semi-local interactions on the heat flux, and derived an adapted heat flux $\Junf$ that can be efficiently implemented via \ad.
This heat flux has asymptotically linear runtime and requires no further restrictions on the form of the potential. Its formulation is independent of the body-order of the potential energy function, making no distinction between pair, angle-dependent, or many-body potentials. As it relies on explicitly constructing an extended simulation cell, it is applicable to semi-local \mlps{} with moderate effective interaction ranges.

% \vspace{2\baselineskip}
\section*{Data and Code Availability}
\noindent
Data and code required to reproduce all figures can be found at \href{https://doi.org/10.5281/zenodo.7767432}{doi:10.5281/zenodo.7767432}. First-principles calculations for the training data are additionally available on the NOMAD repository at \href{https://doi.org/10.17172/NOMAD/2023.03.24-2}{doi:10.17172/NOMAD/2023.03.24-2}. 
Further information and software can be found in the \SM{} and at \href{https://marcel.science/gknet}{https://marcel.science/gknet}.

% \vspace{2\baselineskip}
\section*{Acknowledgements}
\noindent
This work was supported by the TEC1p Project (ERC Horizon 2020 No. 740233).
M.F.L. gratefully acknowledges financial support by the German Ministry for Education and Research BIFOLD program (refs. 01IS18025A and 01IS18037A).
F.K. acknowledges support from the Swedish Research Council (VR) program 2020-04630, and the Swedish e-Science Research Centre (SeRC).
M.R. acknowledges funding from the European Union's Horizon 2020 research and innovation program under Grant Agreement 952165.
Part of this research was performed while M.F.L. and M.R. were visiting the Institute for Pure and Applied Mathematics (IPAM), which is supported by the National Science Foundation (Grant No. DMS-1925919).
M.F.L. would like to thank
Profs.~Klaus-Robert Müller and Davide Donadio,
as well as
Carla Verdi, Fabian Nagel, and Adam Norris for constructive discussions and support.

\bibliography{babel_short}

\end{document}
