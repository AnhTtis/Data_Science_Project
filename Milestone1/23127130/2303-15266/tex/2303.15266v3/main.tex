% CVPR 2022 Paper Template
% based on the CVPR template provided by Ming-Ming Cheng (https://github.com/MCG-NKU/CVPR_Template)
% modified and extended by Stefan Roth (stefan.roth@NOSPAMtu-darmstadt.de)

\documentclass[10pt,twocolumn,letterpaper]{article}

%%%%%%%%% PAPER TYPE  - PLEASE UPDATE FOR FINAL VERSION
% \usepackage[review]{cvpr}      % To produce the REVIEW version
\usepackage{cvpr}            
\usepackage[accsupp]{axessibility}


\usepackage{graphicx}
\usepackage{amsmath}
\usepackage{amssymb}
\usepackage{booktabs}
\usepackage{multirow}
\usepackage[normalem]{ulem}
\useunder{\uline}{\ul}{}
\usepackage{fancyhdr}





% Support for easy cross-referencing
\usepackage[capitalize]{cleveref}
\crefname{section}{Sec.}{Secs.}
\Crefname{section}{Section}{Sections}
\Crefname{table}{Table}{Tables}
\crefname{table}{Tab.}{Tabs.}


%%%%%%%%% PAPER ID  - PLEASE UPDATE
%\def\cvprPaperID{3212} % *** Enter the CVPR Paper ID here
\def\confName{CVPR}
\def\confYear{2023}


\begin{document}

%%%%%%%%% TITLE - PLEASE UPDATE
% \title{Multi-granularity Archaeological Dating of Chinese Bronze Dings Based on Knowledge-guided Relation Graph}
\title{Multi-Granularity Archaeological Dating of Chinese Bronze Dings Based on a Knowledge-Guided Relation Graph}

\author{
Rixin Zhou\textsuperscript{1} \qquad
Jiafu Wei\textsuperscript{1} \qquad
Qian Zhang\textsuperscript{3} \qquad
Ruihua Qi\textsuperscript{3} \qquad
Xi Yang\textsuperscript{1,2,}\textsuperscript{*} \qquad
Chuntao Li\textsuperscript{3,}\textsuperscript{*}\\
\textsuperscript{1}School of Artificial Intelligence, Jilin University\\
\textsuperscript{2}Engineering Research Center of Knowledge-Driven Human-Machine Intelligence, MoE, China \\
\textsuperscript{3}School of Archaeology, Jilin University\\
}
% \saythanks

% For a paper whose authors are all at the same institution,
% omit the following lines up until the closing ``}''.
% Additional authors and addresses can be added with ``\and'',
% just like the second author.
% To save space, use either the email address or home page, not both
% \and
% Second Author\\
% Institution2\\
% First line of institution2 address\\
% {\tt\small secondauthor@i2.org}
% \and
% 3rd Author\\
% Institution2\\
% First line of institution2 address\\
% {\tt\small secondauthor@i2.org}
% \and
% 4th Author\\
% Institution2\\
% First line of institution2 address\\
% {\tt\small secondauthor@i2.org}
% \and
% 5th Author\\
% Institution2\\
% First line of institution2 address\\
% {\tt\small secondauthor@i2.org}
% }


% \maketitle

%------------------


\twocolumn[{
\renewcommand\twocolumn[1][]{#1}
\maketitle
\begin{center}
    \captionsetup{type=figure}
    \centering
    \includegraphics[width=1.0\linewidth]{figure1.pdf}
    \captionof{figure}{Typical examples of Chinese bronze dings from 4 dynasties and 11 periods. The columns from left to right show Shang (Early, Late), Western Zhou (Early, Mid, Late), Spring and Autumn (Early, Mid, Late), and Warring States (Early, Mid, Late). The timeline under the image is the time range for the corresponding periods, where B.C. indicates Before Christ.}
   % \captionof{figure}{Typical examples of Chinese bronze Dings from 4 dynasties and 11 periods, which coloums from left to right are Shang (Early, Late), Western Zhou (Early, Mid, Late), Spring and Autumn (Eerly, Mid, Late), Warring States (Early, Mid, Late). Timeline under the image is the time ranges for corresponding periods, where B.C. indicates Before Christ.}
  \label{fig1:examples}
\end{center}


}]
%------------------
%%%%%%%%%%%%%%%%%%%%%%%%%%%%%%%%%%%%%%%%%%%%%%%%%%%%%%%%%%%%%%%%%%%%%%%%%%%%%%%%
%%%%%%%%%%%%%%%%%%%%%%%%%%%%%%%%%%%%%%%%%%%%%%%%%%%%%%%%%%%%%%%%%%%%%%%%%%%%%%%%
%%%%%%%%% ABSTRACT
\begin{abstract}
   \vspace{-0.5cm}
   The archaeological dating of bronze dings has played a critical role in the study of ancient Chinese history. Current archaeology depends on trained experts to carry out bronze dating, which is time-consuming and labor-intensive. For such dating, in this study, we propose a learning-based approach to integrate advanced deep learning techniques and archaeological knowledge. To achieve this, we first collect a large-scale image dataset of bronze dings, which contains richer attribute information than other existing fine-grained datasets. Second, we introduce a multihead classifier and a knowledge-guided relation graph to mine the relationship between attributes and the ding era. Third, we conduct comparison experiments with various existing methods, the results of which show that our dating method achieves a state-of-the-art performance. We hope that our data and applied networks will enrich fine-grained classification research relevant to other interdisciplinary areas of expertise. The dataset and source code used are included in our supplementary materials, and will be open after submission owing to the anonymity policy. Source codes and data are available at: https://github.com/zhourixin/bronze-Ding.
\end{abstract}
\thispagestyle{fancy} % IEEE模板在\maketitle后会自动声明\thispagestyle{plain},
                            % 导致第一页什么都没有。所以得把plain更改为fancy
      \fancyhead{} 
      \fancyfoot{} 
      \fancyfoot[L]{\footnotesize *Corresponding authors}
      \renewcommand{\headrulewidth}{0pt} %改为0pt即可去掉页眉下面的横线
      \renewcommand{\footrulewidth}{1pt} %改为0pt即可去掉页脚上面的横线
      \fancyfootoffset[R]{-14cm}
% \begin{abstract}
%    The archaeological dating of bronze Dings plays a critical role in the study of ancient Chinese history. Current archaeology depends on trained experts to carry out the dating of bronze Dings, which was time-consuming and labor-intensive. In this work, we propose a learning-based approach to integrate advanced deep learning techniques and archaeology knowledge in this dating task. To achieve this, we first collect a large-scale image dataset of bronze Dings, which contains richer attribute information than other existing fine-grained datasets. Second, we introduce a multi-head classifier and a knowledge-guided relation graph to mine the relation between attributes and the era of Dings. Third, we conduct comparison experiments with various existing methods, the results show that our method achieves state-of-the-art dating performance. We hope our data and 1 will enrich fine-grained classification research relevant to other interdisciplinary areas of expertise. The dataset and source code are included in our supplementary materials, and these will be open after submission due to the anonymous policy.
% \end{abstract}
%%%%%%%%%%%%%%%%%%%%%%%%%%%%%%%%%%%%%%%%%%%%%%%%%%%%%%%%%%%%%%%%%%%%%%%%%%%%%%%%
%%%%%%%%%%%%%%%%%%%%%%%%%%%%%%%%%%%%%%%%%%%%%%%%%%%%%%%%%%%%%%%%%%%%%%%%%%%%%%%%






%%%%%%%%%%%%%%%%%%%%%%%%%%%%%%%%%%%%%%%%%%%%%%%%%%%%%%%%%%%%%%%%%%%%%%%%%%%%%%%%
%%%%%%%%%%%%%%%%%%%%%%%%%%%%%%%%%%%%%%%%%%%%%%%%%%%%%%%%%%%%%%%%%%%%%%%%%%%%%%%%
\vspace{-10.5pt}
\section{Introduction}
\vspace{-4pt}
Dings are cauldrons used for cooking, storage, and ritual offerings to gods or ancestors in ancient China, and they are the most important species used in Chinese ritual bronzes~\cite{49}. The archaeological dating of dings has contributed to the study of ancient Chinese history. Although the excavated bronzes are massive, dating such artifacts depends on the long-term training and accumulation of expertise in archaeological typology~\cite{54}. In addition, some artifacts are easy to identify to a precise age and others are difficult to identify.
% Dings are cauldrons that are used for cooking, storage, and ritual offerings to the gods or to ancestors in ancient China, and it is the most important species used in Chinese ritual bronzes~\cite{49}. The archaeological dating of Dings contributes to studying ancient Chinese history. 
% Although the excavated bronzes are massive, dating these artifacts depends on the long term training and accumulation of experts currently in archaeological typology~\cite{54}. Besides, some artifacts are easy to be identified to a precise age, some are difficult. Even worse, due to the limited information, different judgments can be decided based on personal knowledge frequently. 

% \fancyfoot[LE]{Overleaf}

For the object, we focus on a ding, the features of which are similar and complicated in different eras, as shown in the columns of Figure~\ref{fig1:examples}. We therefore consider this dating task as a fine-grained classification problem. Simultaneously, research into fine-grained classification is close to that of other areas of expertise because it often requires expensive specialized data and knowledge areas, such as birds (zoology)~\cite{12,56} and flowers (botany)~\cite{55}.
% We focus on this one kind of object (Ding) whose features are similar and complicated from different eras, as shown in each column of Figure~\ref{fig1:examples}. Therefore, we deal with this dating task as a fine-grained classification problem. Simultaneously, fine-grained classification research is close to other areas of expertise, as they often require expensive specialized data and knowledge, such as birds (zoology)~\cite{12,56} and flowers (botany)~\cite{55}. 


Data features and domain knowledge, particularly in archaeology, vary in different fields. In addition to the common traits of the existing fine-grained datasets, our data are more challenging. First, our data are unbalanced and difficult to mitigate through their collection because they are determined based on an unearthed state. Second, there are more similarities between bronze dings of adjacent eras, leading to the possibility of misclassifying them into fine granularity adjacent eras beyond a coarse granularity. In other words, compared to other fine-grained classification data, our data have a larger intra-class difference and a smaller inter-class difference between adjacent eras. Third, the attributes and eras are intertwined and the relations are more complex. Each period of bronze dings has multiple shapes and characteristics, and each shape and characteristic correspond to multiple periods of bronze dings, leading to the impracticality of making simple judgments regarding the period based on the shape and characteristic. Existing fine-grained classification methods therefore struggle when applying our data.
% The data features and domain knowledge are different in different fields, especially archaeology. Besides the common traits of existing fine-grained datasets, our data is more challenging. First, Our data is more unbalanced and difficult to mitigate by collecting data, because they are determined by the unearthed situation. Second, more similarities between bronze Dings of adjacent eras, leading to the possibility of misclassifying into fine-granularity adjacent eras beyond coarse-granularity. In other words, compared to other fine-grained classification data, our data has a larger intra-class difference, and a smaller inter-class difference between adjacent eras. Thrid, attributes and eras are intertwined, and the relations are more complex. Each period of bronze Dings have multiple shape and characteristic, and each shape and characteristic correspond to multiple periods of bronze Ding, leading to the impracticality of making simple judgments about period based on shape and characteristic. Therefore, existing fine-grained classification methods struggle to perform well on our data.


To address these issues, we make the following contributions in this study:
% To deal with these issues, we make the following contributions in this paper:
\begin{itemize}
\setlength{\itemsep}{0pt}
\setlength{\parsep}{0pt}
\setlength{\parskip}{0pt}
\item We collect an image dataset of 3690 bronze dings with rich annotations made by bronze experts, including the era (4 course-grained dynasties and 11 fine-grained periods), attributes (29 shapes and 96 characteristics with bounding boxes), literature, location of excavation, and the museum where they are displayed. 
% \vspace{-4pt}
\item We build an end-to-end multihead network to solve this multi-granularity task. The two heads combine coarse- and fine-grained features in a bidirectional manner with a gradient truncated addition to improve the performance at both granularities. The outputs of other two heads, the shape and characteristic nodes, are added to a knowledge-guided relation graph to embed the domain knowledge into our network,  
\item We propose exploiting these rich attributes following archaeological knowledge by employing the focal-type probability classification loss and indicate the ineffectiveness of simply concatenating external information. 
\item We achieve the best performance in terms of the dating accuracy, outperforming other state-of-the-art (SOTA) fine-grained classification methods. 
\end{itemize}
% \begin{itemize}
% \setlength{\itemsep}{0pt}
% \setlength{\parsep}{0pt}
% \setlength{\parskip}{0pt}
% \item We collect an image dataset of 3690 bronze Dings with rich annotations by bronze experts, including eras (4 course-grained dynasties and 11 fine-grained periods), attributes (29 shapes and 96 characteristics with bounding boxes), literature, excavation, and museum. 
% % \vspace{-4pt}
% \item We build an end-to-end multi-head network to solve this multi-granularity task. Two heads combine coarse-grained and fine-grained features in a bi-directional manner with a gradient truncated addition to improve the performance at both granularities. The nodes of shape and characteristic are added in a knowledge guided graph to embed domain knowledge into our network. 
% \item We propose to exploit these rich attributes following archaeological knowledge by employing conditional probability classification loss, and indicate the ineffectiveness of simply concatenating external information. 
% \item We achieve the best performance on dating accuracy, outperforming that of other various state-of-the-art (SOTA) fine-grained classification methods. 
% \end{itemize}
%%%%%%%%%%%%%%%%%%%%%%%%%%%%%%%%%%%%%%%%%%%%%%%%%%%%%%%%%%%%%%%%%%%%%%%%%%%%%%%%
%%%%%%%%%%%%%%%%%%%%%%%%%%%%%%%%%%%%%%%%%%%%%%%%%%%%%%%%%%%%%%%%%%%%%%%%%%%%%%%%





%%%%%%%%%%%%%%%%%%%%%%%%%%%%%%%%%%%%%%%%%%%%%%%%%%%%%%%%%%%%%%%%%%%%%%%%%%%%%%%%
%%%%%%%%%%%%%%%%%%%%%%%%%%%%%%%%%%%%%%%%%%%%%%%%%%%%%%%%%%%%%%%%%%%%%%%%%%%%%%%%
\section{Related Work}
%------------------------------------------------------------------
\vspace{-4pt}
\subsection{Bronze Dings dating}
\vspace{-4pt}

In addition to manual inference~\cite{54, 28}, the chemical and physical properties of metals are also used to locate the exact year of a bronze ding~\cite{10,11,23, 24}. However, chemical and material science based techniques are time-consuming and difficult to manipulate, and they may cause irreversible damage to the bronze. Meanwhile, machine-learning-based methods have been used to explore bronze inscription recognition~\cite{15,21,22}. Although the meaning of the inscriptions on a bronze ding is important, it is insufficient for achieving an accurate dating. The automatic dating of bronze dings using artificial intelligence has been largely underexplored.
% Apart from manual inference~\cite{54, 28}, the chemical and physical properties of metals are also used to locate the exact year of the bronze Dings~\cite{10,11,23, 24}. However, the chemical and material science-based techniques are time-consuming and not easy to manipulate and may cause irreversible damage to the bronze. Meanwhile, machine learning-based methods have been used to explore bronze inscription recognition~\cite{15,21,22}. Although the meaning of inscriptions is important for bronze Dings, they are insufficient for accurate dating in themselves. The automatic dating of bronze Dings using artificial intelligence has largely been under-explored. 



\subsection{Fine-Grained Visual Classification}
\paragraph{Datasets.}
\vspace{-4pt}
Compared to a traditional image recognition task~\cite{3,4,5}, a fine-grained visual classification (FGVC) task is more challenging~\cite{44}. Although the variation between different categories of fine-grained data can be extremely small, the variation within the same category, owing to changes in pose and occlusions, is much broader, which leads to more difficulties. Several datasets have been proposed to address these challenges, including birds~\cite{12,56,57}, dogs~\cite{14}, airplanes~\cite{13}, flowers~\cite{55}, cars~\cite{58}, vegetables~\cite{59}, fruits~\cite{59}, foods~\cite{60}, fashion~\cite{61,62,41}, and retail products~\cite{63,64}.
% \paragraph{Datasets.} Compared with the traditional image recognition task~\cite{3,4,5}, the fine-grained visual classification(FGVC) task is more challenging~\cite{44}. The variation between different categories of fine-grained data can be very small, but the variation within the same category, due to pose changes and occlusions, is much broader, which leading to more difficulties. To address these challenges, there are many datasets already proposed, such as birds~\cite{12,56,57}, dogs~\cite{14}, airplanes~\cite{13}, flowers~\cite{55}, cars~\cite{58}, vegetables~\cite{59}, fruits~\cite{59}, foods~\cite{60}, fashion~\cite{61,62,41}, retail products~\cite{63,64}, etc.

\vspace{-10.5pt}


\paragraph{Single-Granularity Visual Classification.}
Single-granularity FGVC treats objects at the single-class level. Some studies~\cite{6,7,8,9,45,47} have been based on localization-classification networks to find the local features with differentiation and then combine the global features for fine-grained recognition. In addition, high-order feature interactions~\cite{29,30,31} and the design of specific loss functions~\cite{32,33,34,35,48} have resulted in significant improvements. In addition to conventional methods, to further assist fine-grained recognition tasks, some researchers leverage external information, such as attributes ~\cite{46,66,67,68,69}, web data~\cite{36,70,71}, multi-modal data~\cite{37,72,73}, or human-computer interactions~\cite{38,74}.
% Single-grained FGVC treat objects on a single class level. Some works~\cite{6,7,8,9,45,47} based on localisation-classification networks to find local features with differentiation and then combine global features for fine-grained recognition. Additionally, performing high-order feature interactions~\cite{29,30,31} and designing specific loss functions~\cite{32,33,34,35,48} have also resulted in significant improvements. Beyond the conventional methods, some researches leverage external information, such as attribute~\cite{46,66,67,68,69}, web data~\cite{36,70,71}, multi-modal data~\cite{37,72,73}, or human-computer interactions~\cite{38,74}, to further assist the fine-grained recognition task. 
\vspace{-10.5pt}

\paragraph{Multi-Granularity Visual Classification.}
Hierarchical multi-granularity structures can express richer information than single-granularity structures. We construct the network as a hierarchical structure, which has also been adapted in a number of other studies~\cite{39,40,1,20,2}. In protein function prediction, the output of each level is combined with the input of the next level, thus allowing the network to learn the features of each level jointly~\cite{39}. Parameters assigned for each task are used to encourage cross-task feature interactions~\cite{40}. Tree-structured tasks are constructed to integrate knowledge from the tree hierarchy~\cite{20,43} and conduct a feature transfer between levels~\cite{2,65}.
% Hierarchical multi-granularity structures can express richer information than single-granularity structures. We construct the network as a hierarchical structure, which has also been adapted in a number of works~\cite{39,40,1,20,2}. In protein function prediction, the output of each level is combined with the input of the next level, thus leading the network to learn the features of each level jointly~\cite{39}. Assigning parameters to each task is used to encourage cross-task feature interaction~\cite{40}. The tree-structured tasks are constructed to integrate the knowledge from tree hierarchy~\cite{20,43} and perform feature transfer between levels~\cite{2,65}. %Un
%%%%%%%%%%%%%%%%%%%%%%%%%%%%%%%%%%%%%%%%%%%%%%%%%%%%%%%%%%%%%%%%%%%%%%%%%%%%%%%%
%%%%%%%%%%%%%%%%%%%%%%%%%%%%%%%%%%%%%%%%%%%%%%%%%%%%%%%%%%%%%%%%%%%%%%%%%%%%%%%%




%%%%%%%%%%%%%%%%%%%%%%%%%%%%%%%%%%%%%%%%%%%%%%%%%%%%%%%%%%%%%%%%%%%%%%%%%%%%%%%%
%%%%%%%%%%%%%%%%%%%%%%%%%%%%%%%%%%%%%%%%%%%%%%%%%%%%%%%%%%%%%%%%%%%%%%%%%%%%%%%%
\section{Bronze Ding Dataset}
\label{section3}
\vspace{-4pt}
\paragraph{Data Collection and annotation.}
\label{sec:data Collect and Anno}
We collect more than four thousand ding images from both five published archaeology books and four websites, and sort out 3690 images as our dataset. Some of these are line graphs. Because many dating results are controversial, we re-argue the era of each artifact through discussions with three bronze experts. The collection and labelling of data were carried out by an archaeologist and eight archaeology assists, who took approximately 8 months to complete.
% We collect more than more four images of Dings from 5 published archaeology books and 4 websites, and sort out 3690 consisting of our dataset. Few of them are line graphs. Due to many dating are controversial, we re-argue the era of each artifact discussing with 3 bronze experts. The collection and labelling of data is carried out by a archaeologist and 8 archaeology assists, taking about eight months to complete.

The collected dings belong to 4 course-grained dynasties and 11 fine-grained periods, and each image is annotated with additional annotations, as shown in Figure~\ref{fig3:bbox}, including:

\begin{itemize}
\setlength{\itemsep}{0pt}
\setlength{\parsep}{0pt}
\setlength{\parskip}{0pt}
\item \textbf{Shape}: Single-category label for bronze ding shape, with 29 types in total.
\item \textbf{Characteristic}: Multi-category labels for the key components, decorations, and inscriptions, along with bounding boxes, with 96 types in total.
\item \textbf{Source}: Literature (studies to which these bronze ding images belong); excavation (location of excavation), and museum (current exhibition museums).
\end{itemize}
The detailed process used in the data annotation is described in the supplementary material.
% \begin{itemize}
% \item \textbf{Shape}: Single-category label for bronze Ding's shape, 29 types in total.
% \item \textbf{Characteristic}: Multi-category labels for key components, decorations and inscriptions, with Bounding boxes, 96 types in total.
% \item \textbf{Source}: Literature (The literatures to which these bronze Ding images belong); Excavation (Location of excavation); Museum (Current exhibition museums).
% \end{itemize}
% The detailed process of data annotation are described in the supplementary material.

%%%%%%%%%%%%%%%%%%%%%%%%%%%%%%%%%%%%%%%%%%%%%%
\begin{figure}[t]
\centering
  \includegraphics[width=0.98\linewidth]{figure3.pdf}
  \caption{Annotations of a bronze ding example. The characteristics are labeled by bounding boxes in the left figure, and the right shows the detailed information.}
  % \caption{The annotations of a bronze Ding example. The characteristics are labeled by bounding-boxes in the left figure, and the right shows the detailed information.}
  \label{fig3:bbox}
\end{figure}
%%%%%%%%%%%%%%%%%%%%%%%%%%%%%%%%%%%%%%%%%%%%%%
 \vspace{-10.5pt}
\paragraph{Statistics.}
We also count the numbers of eras, shapes, and characteristic labels from different eras, the results of which are shown in Figure~\ref{fig4:bar}. The numbers of shapes and characteristic annotations varied considerably between eras.
% We also counted the number of era, shape, and characteristic labels in different eras, and the results are shown in Figure ~\ref{fig4:bar}. The number of shape and characteristic annotations varies considerably between eras.
%%%%%%%%%%%%%%%%%%%%%%%%%%%%%%%%%%%%%%%%%%%%%%%%%%%%%%%%%%%%%%%%%%%%%%%%%%%%%%%%
%%%%%%%%%%%%%%%%%%%%%%%%%%%%%%%%%%%%%%%%%%%%%%%%%%%%%%%%%%%%%%%%%%%%%%%%%%%%%%%


%%%%%%%%%%%%%%%%%%%%%%%%%%%%%%%%%%%%%%%%%%%%%%
\begin{figure}[htbp]
\centering
  \includegraphics[width=0.98\linewidth]{figure4.pdf}
  \caption{Statistics showing the imbalance of our dataset.}
  % \caption{The statistics show the unbalance of our dataset.}
  \label{fig4:bar}
\end{figure}
%%%%%%%%%%%%%%%%%%%%%%%%%%%%%%%%%%%%%%%%%%%%%%

% %%%%%%%%%%%%%%%%%%%%%%%%%%%%%%%%%%%%%%%%%%%%%%
% \begin{table}[t]
% \centering
% \caption{Comparison of our dataset and two existing datasets in terms of entropy, conditional entropy, and information gain. The information gain shows that the shape and characteristic annotations of our dataset provide richer information.}
% % \caption{Comparison of our dataset and two existing datasets in entropy, conditional entropy and information gain. The information gains show the shape and characteristic annotations in our dataset provide richer information.}
% \label{tab2:entropy}
% \resizebox{.48\textwidth}{!}{  % Here 1/2
% \begin{tabular}{c|cc|c|c}
% \hline\hline
%                                & \multicolumn{2}{c|}{Bronze Ding}                                          & CUB\_200\_2011~\cite{14}         & Deep Fashion~\cite{41}           \\ \hline\hline
% $H(D)$                       & \multicolumn{2}{c|}{3.459}                                           & 7.644                  & 5.644                  \\ \hline
% $H(D \mid A)$                   & \multicolumn{1}{c|}{1.826(characteristic)}       & 1.717(shape)           & 6.599                  & 4.789                  \\ \hline
% \multirow{2}{*}{$g(D, A)$} & \multicolumn{1}{c|}{\multirow{2}{*}{\textbf{1.633}}} & \multirow{2}{*}{\textbf{1.742}} & \multirow{2}{*}{1.045} & \multirow{2}{*}{0.855} \\
%                                & \multicolumn{1}{c|}{}                       &                        &                        &                        \\ \hline
% \end{tabular}
% } 
% \end{table}
% %%%%%%%%%%%%%%%%%%%%%%%%%%%%%%%%%%%%%%%%%%%%%%

%%%%%%%%%%%%%%%%%%%%%%%%%%%%%%%%%%%%%%%%%%%%%%
% \setlength{\textfloatsep}{10pt}

\begin{table}[t]
\centering
\caption{Comparison of our dataset and five existing datasets in terms of category statistics, entropy, conditional entropy, and information gain. The information gain shows that the shape and characteristic annotations of our dataset provide richer information.}
\label{tab2:entropy}
\resizebox{0.48\textwidth}{!}{  % Here 1/2
\begin{tabular}{c|cc|c|cc|c|c|c}
\hline\hline
                                      & \multicolumn{2}{c|}{Bronze Ding}                                                                                                                                  & CUB\_200\_2011         & \multicolumn{2}{c|}{Deep Fashion}                                                                                                      & CompCars%~\cite{yang2015large} 
                                      & Stanford Dogs           & Food-101                \\ \hline\hline
Images                                & \multicolumn{2}{c|}{3690}                                                                                                                                         & 11788                  & \multicolumn{2}{c|}{289222}                                                                                                            & 136726                         & 20580                   & 101000                  \\ \hline
Image Categories                      & \multicolumn{2}{c|}{11}                                                                                                                                           & 200                    & \multicolumn{2}{c|}{50}                                                                                                                & 1716                           & 120                     & 101                     \\ \hline
Images per Category                   & \multicolumn{2}{c|}{$335\pm315$}                                                                                                                     & $59\pm3$  & \multicolumn{2}{c|}{$5784\pm11989$}                                                                                       & $79\pm48 $        & $171\pm23$ & $1000\pm0$ \\ \hline
\multirow{2}{*}{Attribute Categories} & \multicolumn{1}{c|}{96}                                                                        & 29                                                               & \multirow{2}{*}{312}   & \multicolumn{1}{c|}{1000}                                                     & 26                                                     & \multirow{2}{*}{218}           & \multirow{2}{*}{\textbf{$--$}}      & \multirow{2}{*}{\textbf{$--$}}      \\
                                      & \multicolumn{1}{c|}{$(characteristic)$}                                                          & $(shape)$                                                          &                        & \multicolumn{1}{c|}{$(coarse)$}                                                 & $(fine)$                                                 &                                &                         &                         \\ \hline
$H\left(D\right)$                                & \multicolumn{2}{c|}{3.459}                                                                                                                                        & 7.644                  & \multicolumn{2}{c|}{5.644}                                                                                                             & 10.745                         & \textbf{$--$}                       & \textbf{$--$}                       \\ \hline
\multirow{2}{*}{$H\left(D \mid A\right)$}        & \multicolumn{1}{c|}{1.826}                                                                     & 1.717                                                            & \multirow{2}{*}{6.599} & \multicolumn{1}{c|}{4.470}                                                    & 4.789                                                  & \multirow{2}{*}{9.570}         & \multirow{2}{*}{\textbf{$--$}}      & \multirow{2}{*}{\textbf{$--$}}      \\
                                      & \multicolumn{1}{c|}{$(characteristic)$}                                                          & $(shape)$                                                          &                        & \multicolumn{1}{c|}{$(coarse)$}                                                 & $(fine)$                                                 &                                &                         &                         \\ \hline
$g(D, A)$                             & \multicolumn{1}{c|}{\textbf{\begin{tabular}[c]{@{}c@{}}1.633\\ $(characteristic)$\end{tabular}}} & \textbf{\begin{tabular}[c]{@{}c@{}}1.742\\ $(shape)$\end{tabular}} & 1.045                  & \multicolumn{1}{c|}{\begin{tabular}[c]{@{}c@{}}1.174\\ $(coarse)$\end{tabular}} & \begin{tabular}[c]{@{}c@{}}0.855\\ $(fine)$\end{tabular} & 1.002                          & \textbf{$--$}                       & \textbf{$--$}                       \\ \hline
\end{tabular}
}
%\caption{\textcolor{red}{revise accordingly, remove caption? to save space} Comparison of our dataset and four existing datasets in terms of conditional entropy, information gain, etc.}
\end{table}
% \vspace{-20pt}
%%%%%%%%%%%%%%%%%%%%%%%%%%%%%%%%%%%%%%%%%%%%%%

\paragraph{Comparison.}
To quantify the information provided by the additional annotations, we calculate the information gain of the shape and characteristic annotations of the era judgement, as shown in Equation (\ref{eq1:entropy}).
% In order to quantify the information provided by the additional annotations, we calculate the information gain of the shape and characteristic annotations to the era judgement, as Equation (\ref{eq1:entropy}). 

\begin{equation}
g(D, A)=H(D)-H(D \mid A)
\label{eq1:entropy}
\end{equation}
where $\mathrm{H}(\mathrm{D})$ is the entropy of the fine-grained labels on dataset $D$ and $\mathrm{H}(\mathrm{D}\mid\mathrm{A})$ is the conditional entropy of attribute $A$ on dataset $D$.
% Where $\mathrm{H}(\mathrm{D})$ is the entropy of fine-grained labels on dataset $D$, and $\mathrm{H}(\mathrm{D}\mid\mathrm{A})$ is the conditional entropy of attributes $A$ on dataset $D$.

For comparison, the entropy in CUB-200-2011~\cite{14} and Deep-Fashion~\cite{41} are also calculated. The results are presented in Table~\ref{tab2:entropy}. We find that the information gain in our dataset is more significant than that of the other datasets, which means that our shape and characteristic annotations provide richer information. Such information is therefore critical for improving the network performance.
% For comparison, those in datasets CUB-200-2011~\cite{14} and Deep-Fashion~\cite{41} also be calculated. The results are shown in Table~\ref{tab2:entropy}. We can find that the information gain in our dataset is more significant than that of others, which means our shape and characteristic annotations provide richer information. Therefore, this information is critical to improve the performance of the network. 







%%%%%%%%%%%%%%%%%%%%%%%%%%%%%%%%%%%%%%%%%%%%%%
\begin{figure*}[htbp]
\centering
  \includegraphics[width=1.0\linewidth]{figure2.pdf}
  \caption{Overview of our network. We design four heads: two heads in MGM are responsible for extracting dynasty and period features at two granularities. Two other heads in KEM are responsible for extracting shape and characteristic features of the bronze dings. Then, the outputs of MGM and KEM jointly build our AKG to formulate the relationship between the eras and attributes of each ding through the graph loss. Simultaneously, the outputs are also used to compute cross-entropy and focal losses to enhance the learning of the annotations on each head.}
  \label{fig2:architecture}
\end{figure*}
%%%%%%%%%%%%%%%%%%%%%%%%%%%%%%%%%%%%%%%%%%%%%%





%%%%%%%%%%%%%%%%%%%%%%%%%%%%%%%%%%%%%%%%%%%%%%%%%%%%%%%%%%%%%%%%%%%%%%%%%%%%%%%%
%%%%%%%%%%%%%%%%%%%%%%%%%%%%%%%%%%%%%%%%%%%%%%%%%%%%%%%%%%%%%%%%%%%%%%%%%%%%%%%%
\section{Methodology}
\vspace{-4pt}
\subsection{Overview}
\vspace{-4pt}
To address the challenges to our task, we construct a multihead network for predicting the era of bronze dings on an archaeology knowledge-guided relation graph, as shown in Figure ~\ref{fig2:architecture}. The network consists of three parts: a multi-granularity module (MGM), knowledge extraction module (KEM), and archaeology knowledge-guided relation graph (AKG). Compared to HRN~\cite{2}, first, we additionally enhance the dynasty (coarse) features without affecting the period (fine) dating performance by adding the feature of period head to the dynasty head with gradient truncation, and then feeding into the next FC layer. Second, we implement KEM in our network and extend the relation graph to leverage the features of our dataset by considering attributes information (shape and characteristic) in the knowledge graph. Third, we design a focal-type probability classification loss to learn the relationship between attributes and eras from easy (shape) to difficult (characteristic), instead of just learning category information.


\subsection{Network Architecture}
\vspace{-4pt}
\paragraph{Multi-granularity Module.}

The MGM is built to combine dynasty features with period features and enhance them interactively. After the backbone network, dynasty and period features are separately extracted by two heads consisting of two convolution layers and two fully connected layers. Then, in addition to applying an element-wise addition from dynasty head to period head to enrich period information, an element-wise gradient truncation addition is applied in reverse to enhance dynasty features without affecting the period performance.

% The MGM is built to combine dynasty features with period features and enhance their features. In MGM, a bronze ding's dynasty and period features are extracted by a series of convolution layers and full connection layers. The dynasty features apply an element-wise addition to the period features to enrich the period information. Meanwhile, the period features apply an element-wise gradient truncated addition to the dynasty features, enhancing the dynasty features without undermining the performance of the period classification.

% The MGM is built to combine dynasty features with period features and enhance their granularity features. We use two heads to extract the dynasty and period features of a bronze ding and then add the dynasty features to the period features to enrich the period features. Specifically, the period features apply an element-wise gradient truncated addition to the dynasty features, enhancing the dynasty features without undermining the performance of the period classification.
% MGM is built to combine dynasty features with period features to enhance both granularity features. We use two heads to extract dynasty features and period features of the bronze Ding, and then add the dynasty features to the period features to enrich the period features. Specifically, the period features perform element-wise gradient truncated addition to the dynasty features, enhancing the dynasty features without undermining the performance of the period classification. 


For the outputs, dynasty head applies a $sigmoid$ projection, and then the result forms the dynasty node of the AKG. The period head applies $sigmoid$ and $softmax$ projections, where the $sigmoid$ output forms the period node of the AKG, and the $softmax$ output computes the cross-entropy loss $\mathcal{L}_{ce}$ to enhance the exclusive relation between the period nodes.
% The dynasty head of the MGM perform the $sigmoid$ projection and the result forms the dynasty node of the AKG. Then the period head of the MGM perform $sigmoid$ and $softmax$ projection respectively, where the $sigmoid$ output forms the period node of the AKG and the $softmax$ output computes the cross-entropy loss $\mathcal{L}_{ce}$ to enhance the exclusive relation between the period nodes.

\vspace{-10.5pt}
\paragraph{Knowledge Extraction Module.}
Because of the narrow inter-class differences and wide intra-class variations of the bronze ding, archaeologists must combine various factors to determine the era to which they belong. We therefore introduce domain knowledge to improve the performance of the network, including the shape and characteristic annotations. Specifically, we design a KEM for extracting the shape and characteristic information of a bronze ding, which consists of two separate heads for extracting knowledge-specific features.
% Because of the narrow inter-class differences and the wide intra-class variations of the bronze Ding images, even archaeologists have to combine various dimensions to determine the era to which they belong. We therefore introduce additional expert knowledge to help improve the performance of the network, including shape and characteristic annotations. Specifically, we designed the knowledge extraction module (KEM) to extract the shape and characteristic information of the bronze Ding, which consists of two separate heads for extracting the knowledge-specific features, respectively.


The shape head of the KEM applies $sigmoid$ and $softmax$ projections. From Figures~\ref{fig4:bar} (b) and (c), we can see that the shape and characteristic categories are unbalanced; therefore, we compute $\mathcal{L}_{\text {focal}}$~\cite{42} between its $softmax$ output and the shape labels to alleviate the category imbalance problem and enhance the exclusive relation between the shape nodes. Furthermore, because the classification of the characteristics is a multilabel classification task, the characteristic head of the KEM only applies a $sigmoid$ projection, and we compute the multilabel $\mathcal{L}_{\text {ml-focal}}$ between its $sigmoid$ output and the characteristic labels. The $sigmoid$ outputs of these two heads are also fed to the AKG.
% The shape head of the KEM perform $sigmoid$ and $softmax$ projection respectively. From the Figure ~\ref{fig4:bar} (b) and (c), we find that the shape and characteristic categories are unbalanced, therefore, we compute the $\mathcal{L}_{\text {focal}}$~\cite{42} between its $softmax$ output and shape labels to alleviate the category imbalance problem and model the exclusive relation between the shape nodes. As the classification of characteristic is a multi-label classification task, the characteristic head of the KEM only perform $sigmoid$ projectionwe. We compute the multi-label $\mathcal{L}_{\text {ml-focal}}$ between its $sigmoid$ output and characteristic labels. And the $sigmoid$ output of these two heads are also fed to the AKG.
%------------------------------------------------------------------


\subsection{Archaeology Knowledge Guided Relation Graph}
\vspace{-4pt}
\paragraph{The Formalism of Relation Graph.}
Inspired by the study in~\cite{2,20}, we develop an archaeological knowledge-guided relation graph embedded with domain knowledge to enable the network to synthetically learn the era, shape, and characteristic labels. The nodes of the relation graph are the types of eras and attributes from MGM and KEM, and a set of directed edges and undirected edges are defined between these nodes. A directed edge is a subsumption edge that indicates that the parent nodes subsume the child node. An undirected edge is an exclusion edge and indicates that the two nodes are mutually exclusive.
% Inspired by the work of~\cite{2,20}, we develop an archaeology knowledge guided relation graph embedded with expert knowledge to enable the network to synthetically learn the era, shape and characteristic labels. The nodes of relation graph are the types of eras and attributes from MGM and KEM, and a set of directed edges and undirected edges defined between these nodes. A directed edge is a subsumption edge, indicating that the parent nodes subsumes the child node. An undirected edge is an exclusion edge, denoting that two nodes are mutually exclusive. 




According to archaeological knowledge, we conclude that the relations of the edges and nodes are as follows:
\begin{itemize}
\setlength{\itemsep}{0pt}
\setlength{\parsep}{0pt}
\setlength{\parskip}{0pt}
\item Because a bronze ding cannot belong to two eras at the same time, any two dynasty or period nodes only have an exclusive edge between them. The relation of the era node subsumes in the dynasty node and can be expressed as a subsumption edge.
\item Although one bronze ding only contains one type of shape node, multiple dings within the same period contain multiple shape nodes. Therefore, the period and shape nodes have multiple subsumption edges. In addition, because a bronze ding cannot have two shapes, any two period nodes have exclusive edges between them.
\item A bronze ding may contain multiple characteristics, and the period and characteristic nodes therefore have multiple subsumption edges.
\end{itemize}
Based on these relations, we define an extended legal global assignment of all labels in the relation graph as binary-label vectors for an object. Beyond~\cite{2}, we consider the shape and characteristics of a node assignment. The set of all legal global assignments forms the era state space $S_{G_e} \subseteq \left\{0, 1\right\}^n$, era-shape combination state space $S_{G_{es}} \subseteq \left\{0, 1\right\}^{n+m}$, and era-characteristic combination state space $S_{G_{ec}} \subseteq \left\{0, 1\right\}^{n+k}$ of relation graph $G$, where $n$, $m$, and $k$ denote the number of nodes for the eras, shapes, and characteristics, respectively. Thus, we calculate the probabilistic classification loss on $G$, enabling the network to improve its judgement of the era by learning domain knowledge of the shapes and characteristics.
% According to archaeological knowledge, we conclude that the relations of edge-node are:
% \begin{itemize}
% \item Since a bronze Ding cannot belong to two eras at the same time, any two dynasty or period nodes only have exclusive edge between them. And the relation of era node subsumes in dynasty node can be expressed as subsumption edge.
% \item Although one bronze Ding only contains one type of shape node, multiple dings within the same period contain multiple shape nodes. Therefore, the period node and the shape node have multiple subsumption edges. Besides, since a bronze Ding cannot has two shapes, any two period nodes have exclusive edge between them.
% \item A bronze Ding may contain multiple characteristics, therefore, the period node and the characteristic node have multiple subsumption edges.
% \end{itemize}
% Based on these relations, we define a extended legal global assignment of all labels in the hierarchy as some binary label vectors for an object. Beyond~\cite{2}, we take the shape and characteristic into account for node assignment. The set of all legal global assignments forms the era state space $S_{G_e} \subseteq \left\{0, 1\right\}^n$,  the era-shape state space $S_{G_{es}} \subseteq \left\{0, 1\right\}^{n+m}$, the era-characteristic state space $S_{G_{ec}} \subseteq \left\{0, 1\right\}^{n+k}$ of relation graph $G$, where $n,m,k$ denote the number of nodes for eras, shapes and attributes, respectively. Thus, we can calculate the conditional probabilistic classification loss on $G$, enabling the network to improve its judgement of era by learning expert knowledge of shape and characteristic.

\vspace{-10.5pt}

% \paragraph{Conditional Probabilistic Classification Loss.}
\paragraph{Focal-type Probabilistic Classification Loss.} 
The two types of attributes have different effects on the dating by archaeologists. There are a few types of shapes, which are relatively easy to distinguish. However, the ding shape type is not decisive for dating. Meanwhile, some characteristic types can accurately define this era. However, the number of characteristic types is large, and they are both similar and complex. Based on this knowledge, we construct probabilistic classification losses to enable the network to learn information in the relation graph. 

During training, we obtain the predicted label in the relation graph and maximized its marginal probability in a step-by-step manner. Given an input image $\mathbf{x}$, the unnormalized era joint probability of all era nodes concerning the era label assignment $\mathbf{y_e}$ can be computed as $\tilde{P}_{e}(\mathbf{y_{e}} \!\!\mid\!\!  \mathbf{x})$.
The era joint probability is then normalized by $\operatorname{Pr_{e}}(\mathbf{y_{e}} \mid \mathbf{x})=\frac{\tilde{P_{e}}(\mathbf{y_{e}} \mid \mathbf{x})}{Z_e(\mathbf{x})}$, where $Z_e(\mathbf{x})$ is the era partition function that sums over all legal era assignments $\overline{\mathbf{y}}_e \in S_{G_e}$ in the era state space. If input image $\mathbf{x}$ has the $i$-th era label, we can obtain the era marginal probability $\operatorname{Pr_{e}}(y_{e_i}=1 \mid \mathbf{x})$ of era label $i$ by summing over all legal era assignments $\overline{\mathbf{y}}_e$ that include $\overline{y}_{e_i}$ = 1. Procedures for calculating normalized era-shape joint probability $\operatorname{Pr_{es}}(\mathbf{y}_{es} \mid \mathbf{x})$ and era-characteristic joint probability$\operatorname{Pr_{ec}}(\mathbf{y}_{ec} \mid \mathbf{x})$ are the same. The details of the calculations are described in the supplementary material.



% During training, we obtain the predicted label in the relation graph and maximized its marginal probability in a step-by-step manner. Given an input image $\mathbf{x}$, we first calculate the unnormalized era probability $\tilde{P}_{e}(\mathbf{y_{e}} \!\!\mid\!\!  \mathbf{x})$, the unnormalized era-shape combination probability $\tilde{P}_{es}(\mathbf{y_{es}} \!\!\mid\!\!  \mathbf{x})$, and the unnormalized era-characteristic combination probability $\tilde{P}_{ec}(\mathbf{y_{ec}} \!\!\mid\!\! \mathbf{x})$ of the corresponding nodes, respectively. The probability is then normalized by $\operatorname{Pr}(\mathbf{y} \mid \mathbf{x})=\frac{\tilde{P}(\mathbf{y} \mid \mathbf{x})}{Z(\mathbf{x})}$, where $Z(\mathbf{x})$ is the partition function that sums over all legal assignments $\overline{\mathbf{y}} \in \left\{ S_{G_e}, S_{G_{es}}, S_{G_{ec}}\right\}$ of $G$. After that, we can obtain the marginal probability of the labels of \mathbf{x}. The details of the calculations for the probability and partition functions are described in the supplementary material.
% Given an input image $x$, we calculate the unnormalized era probability $\tilde{P}_{e}(\mathbf{y_{e}} \!\!\mid\!\!  \mathbf{x})$, the unnormalized era-shape combination probability $\tilde{P}_{es}(\mathbf{y_{es}} \!\!\mid\!\!  \mathbf{x})$ and the unnormalized era-characteristic combination probability $\tilde{P}_{ec}(\mathbf{y_{ec}} \!\!\mid\!\! \mathbf{x})$ of corresponding nodes. The probability is then normalized by $\operatorname{Pr}(\mathbf{y} \mid \mathbf{x})=\frac{\tilde{P}(\mathbf{y} \mid \mathbf{x})}{Z(\mathbf{x})}$, where $Z(x)$ is the partition function that sums over all legal assignments $\overline{\mathbf{y}} \in \left\{ S_{G_e}, S_{G_{es}}, S_{G_{ec}}\right\}$ of $G$. Thus we can calculate the marginal probability of $x$'s labels. Details of the calculations for the probability and partition function are described in the supplementary material.


Given $m$ training samples, $\mathcal{D}=\left\{x^{l}, y_e^{l}, y_{es}^{l}, y_{ec}^{l}, g_{e}^{l}, \notag\right. \\ \left. g_{es}^{l}, g_{ec}^{l}\,\right\}$, $l=1, \ldots, m$, where $y_e^{l}$, $y_{es}^{l}$ and $y_{ec}^{l}$ are the ground-truth label vector of the era, era-shape combination, and era-characteristic combination, respectively. And $g_{e}^{l}  \in 
 \left\{1,\ldots , n\right\} $, $g_{es}^{l}\in \left\{1,\ldots , n+m\right\}$, $g_{ec}^{l}\in \left\{1,\ldots , n+k\right\}$ are the indices of the observed era, era-shape combination, and era-characteristic combination labels, respectively. Subsequently, the era probabilistic classification loss $\mathcal{L}_{\text {e}}(\mathcal{\!D\!})$ is defined as follows:
 \begin{equation}
-\frac{1}{m} \sum_{l=1}^{m} \ln (\operatorname{Pr_e}(y_{e_{g_{e} ^{l}}}^{l}=1 \mid \mathbf{x}^{l})\!)
\label{eq:Lp}
\end{equation}
 
% Given $m$ training samples, $\mathcal{D}=\left\{x^{l}, y_e^{l}, y_{es}^{l}, y_{ec}^{l}, g_{e}^{l}, \notag\right. \\ \left. g_{es}^{l}, g_{ec}^{l}\,\right\}$, $l=1, \ldots, m$, where $y_e^{l}$, $y_{es}^{l}$ and $y_{ec}^{l}$ are the ground-truth label vector of the era, era-shape combination, and era-characteristic combination, respectively. And $g_{e}^{l}$, $g_{es}^{l}$, $g_{ec}^{l}$ are the indices of the observed era, era-shape combination, and era-characteristic combination labels, respectively. Subsequently, the era probabilistic classification loss $\mathcal{L}_{\text {e}}(\mathcal{\!D\!})$ is defined as follows:
% Given $m$ training samples, $\mathcal{D}=\left\{x^{l}, y_e^{l}, y_{es}^{l}, y_{ec}^{l}, g_{e}^{l}, \notag\right. \\ \left. g_{es}^{l}, g_{ec}^{l}\,\right\}$, $l=1, \ldots, m$, where $y_e^{l}$ is the era ground-truth label vector, $y_{es}^{l}$ is the era-shape combination ground truth label vector, $y_{ec}^{l}$ is the era-characteristic combination ground truth label vector, and $g_{e}^{l}$, $g_{es}^{l}$, $g_{ec}^{l}$ are the indices of the observed era, era-shape combination, and era-characteristic combination labels, respectively. Subsequently, the era probabilistic classification loss $\mathcal{L}_{\text {e}}(\mathcal{\!D\!})$ is defined as follows:


Then, because of the different importance of attributes, we define the focal-type era-shape probabilistic classification loss $\mathcal{L}_{\text {es}}(\mathcal{\!D\!})$ and the era-shape-characteristic probabilistic classification loss $\mathcal{L}_{\text {esc}}(\mathcal{\!D\!})$ as follows:
\begin{equation}
-\frac{1}{m} \!\!\displaystyle\sum_{l=1}^{m}\!\!\left(\!\!(\!1\!-\!\operatorname{Pr_{e}}\!(y_{e_{g_{e} ^{l}}}^{l}\!\!\!=\!1 \!\!\mid \!\!\mathbf{x}^{l})\!)^{	\alpha_1} \! \ln  (\operatorname{Pr_{es}}(y_{{es}_{g_{es} ^{l}}}^{l}\!\!\!\!\!\!=\!\!1 \!\!\mid \!\!\mathbf{x}^{l})\!)\!\!\right)
\label{eq:Le-s}
\end{equation}

\vspace{-10.5pt}

\begin{equation}
-\frac{1}{m} \!\!\displaystyle\sum_{l=1}^{m}\!\!\left(\!\!(\!1\!\!-\!\!\operatorname{Pr_{es}}\!(y_{{es}_{g_{es} ^{l}}}^{l}\!\!\!\!\!\!=\!1 \!\!\mid \!\!\mathbf{x}^{l})\!)^{	\alpha_2} \! \ln  (\!\operatorname{Pr_{ec}}(y_{{ec}_{g_{ec} ^{l}}}^{l}\!\!\!\!\!=\!\!1 \!\!\mid \!\!\mathbf{x}^{l})\!)\!\!\right)
\label{eq:Le-a}
\end{equation}

Thus, when the network learns sufficient information through the era features of a given sample to determine its era, the influence of the shape and characteristics can be weakened by decay factor $\alpha_1$ and $\alpha_2$. When the network cannot learn a sufficient amount of information to determine the era of this sample, $\mathcal{L}_{\text {es}}(\mathcal{\!D\!})$ plays a supportive role. When neither the era features nor the shape features can provide sufficient information, $\mathcal{L}_{\text {esc}}(\mathcal{\!D\!})$ will contribute to determining the chronology of this sample by learning the relationship between the era and characteristics. In this manner, the network can adaptively adjust its learning of the shape and characteristics according to the amount of information possessed by different samples, thereby avoiding a disturbance of the main task.
% Given $m$ training samples $\mathcal{D}=\left\{x^{l}, y_e^{l}, y_{es}^{l}, y_{ec}^{l}, g_{e}^{l}, \notag\right. \\ \left. g_{es}^{l}, g_{ec}^{l}\,\right\}$, $l=1, \ldots, m$, where $y_e^{l}$ is the era ground truth label vector, $y_{es}^{l}$ is the era-shape combination ground truth label vector, $y_{ec}^{l}$ is the era-characteristic combination ground truth label vector, and $g_{e}^{l}$, $g_{es}^{l}$, $g_{ec}^{l}$ are the index of the observed era label, era-shape combination label, era-characteristic combination label, respectively. Then, the era probabilistic classification loss $\mathcal{L}_{\text {e}}(\mathcal{\!D\!})$ is defined as: 
%  \begin{equation}
% -\frac{1}{m} \sum_{l=1}^{m} \ln (\operatorname{Pr_e}(y_{g_{e} ^{l}}^{l}=1 \mid \mathbf{x}^{l})\!)
% \label{eq:Lp}
% \end{equation}
% The era-shape conditional probabilistic classification loss $\mathcal{L}_{\text {es}}(\mathcal{\!D\!})$ and era-characteristic conditional probabilistic classification loss $\mathcal{L}_{\text {ec}}(\mathcal{\!D\!})$ are defined as: 
% \begin{equation}
% -\frac{1}{m} \!\!\displaystyle\sum_{l=1}^{m}\!\left(\!\!(\!1\!-\!\operatorname{Pr_{e}}\!(y_{g_{e} ^{l}}^{l}\!\!\!=\!1 \!\!\mid \!\!\mathbf{x}^{l})\!)^{	\alpha} \! \ln  (\operatorname{Pr_{es}}(y_{g_{es} ^{l}}^{l}\!\!\!\!\!\!\!=\!\!1 \!\!\mid \!\!\mathbf{x}^{l})\!)\!\!\right)
% \label{eq:Le-s}
% \end{equation}
% \begin{equation}
% -\frac{1}{m} \!\!\displaystyle\sum_{l=1}^{m}\!\!\left(\!\!(\!1\!\!-\!\!\operatorname{Pr_{es}}\!(y_{g_{es} ^{l}}^{l}\!\!\!\!\!\!=\!1 \!\!\mid \!\!\mathbf{x}^{l})\!)^{	\alpha} \! \ln  (\!\operatorname{Pr_{ec}}(y_{g_{ec} ^{l}}^{l}\!\!\!\!\!=\!\!1 \!\!\mid \!\!\mathbf{x}^{l})\!)\!\!\right)
% \label{eq:Le-a}
% \end{equation}
% Therefore, when the network learns enough information through the era features of a given sample to determine its era, the influence of shape and characteristic can be weakened by a decay factor $\alpha$. When the network is not able to learn enough information to determine the era of this sample, the $\mathcal{L}_{\text {es}}(\mathcal{\!D\!})$ will play a supportive role. And when neither the era features nor the shape features can provide enough information, the $\mathcal{L}_{\text {ec}}(\mathcal{\!D\!})$ will contribute to determine the chronology of this sample by learning the relation between era and characteristic. In this way, the network can adaptively adjust its learning of shape and characteristic according to the amount of information possessed by different samples, avoiding disturbing the main task. 



Finally, the aforementioned losses are added in a linear manner to form a complete probabilistic classification loss:
\begin{equation}
\mathcal{L}_{\text {graph}}(\mathcal{D})= \mathcal{L}_{\text {e}}(\mathcal{D}) + \beta \cdot (\mathcal{L}_{\text {es}}(\mathcal{D}) + \mathcal{L}_{\text {esc}}(\mathcal{D}))
\label{eq:graph}
\end{equation}
where $\beta$ denotes the weight used to balance the influence of the loss components. 
% Finally, the aforementioned losses are added in a linear manner to form a complete probabilistic classification loss:
% \begin{equation}
% \mathcal{L}_{\text {graph }}(\mathcal{D})= \mathcal{L}_{\text {e}}(\mathcal{D}) + \beta \cdot (\mathcal{L}_{\text {es}}(\mathcal{D}) + \mathcal{L}_{\text {ec}}(\mathcal{D}))
% \label{eq:graph}
% \end{equation}
% where $\beta$ is the weight to balance the influence of loss components. 



\subsection{Total Loss}
\vspace{-4pt}
In addition to the probabilistic classification loss $\mathcal{L}_{graph}$, we also use $\mathcal{L}_{ce}$, $\mathcal{L}_{focal}$, and $\mathcal{L}_{ml-focal}$ to enable the network to learn the era, shape, and characteristic categories of bronze dings, respectively. 
In summary, the total loss is given as a summation of the aforementioned losses with a trade-off parameter, $\lambda$:
\begin{equation}
\mathcal{L}_{\text {total }}(\!\mathcal{D}\!)\!=\!\mathcal{L}_{\text {graph }}(\!\mathcal{D}\!) \!+\! \mathcal{L}_{\text {ce}}(\!\mathcal{D}\!)\! +\! \lambda \!\cdot \!(\mathcal{L}_{\text {focal}}(\!\mathcal{D}\!) \!+\! \mathcal{L}_{\text {ml-focal}}(\!\mathcal{D}\!))
\label{eq:all}
\end{equation}
% Besides probabilistic classification loss $\mathcal{L}_{graph}$, we also use $\mathcal{L}_{ce}$, $\mathcal{L}_{focal}$, and $\mathcal{L}_{ml-focal}$ to enable the network to learn the era, shape, and characteristic categories of the bronze Dings, respectively. 

% In summary, the total loss is given as a summation of aforementioned losses with one trade-off parameter $\lambda$:
% \begin{equation}
% \mathcal{L}_{\text {total }}(\!\mathcal{D}\!)\!=\!\mathcal{L}_{\text {graph }}(\!\mathcal{D}\!) \!+\! \mathcal{L}_{\text {ce}}(\!\mathcal{D}\!)\! +\! \lambda \!\cdot \!(\mathcal{L}_{\text {focal}}(\!\mathcal{D}\!) \!+\! \mathcal{L}_{\text {ml-focal}}(\!\mathcal{D}\!))
% \label{eq:all}
% \end{equation}
%%%%%%%%%%%%%%%%%%%%%%%%%%%%%%%%%%%%%%%%%%%%%%%%%%%%%%%%%%%%%%%%%%%%%%%%%%%%%%%%
%%%%%%%%%%%%%%%%%%%%%%%%%%%%%%%%%%%%%%%%%%%%%%%%%%%%%%%%%%%%%%%%%%%%%%%%%%%%%%%%






%%%%%%%%%%%%%%%%%%%%%%%%%%%%%%%%%%%%%%%%%%%%%%%%%%%%%%%%%%%%%%%%%%%%%%%%%%%%%%%%
%%%%%%%%%%%%%%%%%%%%%%%%%%%%%%%%%%%%%%%%%%%%%%%%%%%%%%%%%%%%%%%%%%%%%%%%%%%%%%%%
\section{Experiments}
%------------------------------------------------------------------
\vspace{-4pt}
\subsection{Data Preparation}
\vspace{-4pt}
We split our data into three sets: the scales of the training set, validation set, and test set are 4:1:5 (1470:363:1857), respectively, following the divisions of other fine-grained classification datasets~\cite{12,13}. Due to the data imbalance, we keep the proportions of each dynasty and period divided in the same way. For the pre-processing, we apply data augmentation to the collected images, including background removal and grayscale.
% We split our data into three sets, the scales of train set, validatation set, and test set are 4:1:5 (1470:363:1857), respectively, following the divisions of other fine-grained classification datasets~\cite{12,13}. Because of data unbalance, we keep the proportion of each dynasty and period are divided in the same way. For pre-processing, we apply data augmentation to the collected images to improve the prediction results, including background removal, gray-scale and feature line extraction. 

%------------------------------------------------------------------
\subsection{Implementation Details}
\label{ImDetails}
\vspace{-4pt}
We implement our network using PyTorch~\cite{51} and conduct experiments on a workstation equipped with an NVIDIA RTX 3090 GPU. For a fair comparison, we also adopt ResNet50 pretrained on ImageNet as our network backbone and resized the input images to 400$\times$400 throughout the experiments. We train each experiment for 64 epochs with early stopping and use Adam optimizer with a learning rate of 0.0001, adjusted using a cosine annealing strategy~\cite{27} to optimize our network. The batch size is set to 32. Besides, We set the decay factors $\alpha_1=2$ and $\alpha_2=3$ in Equations (\ref{eq:Le-s}) and (\ref{eq:Le-a}), balance weight $\beta=0.001$ in Equation (\ref{eq:graph}), and trade-off parameter $\lambda=0.1$ in Equation (\ref{eq:all}). And the parameter settings in $\mathcal{L}_{focal}$ and $\mathcal{L}_{ml-focal}$ follow~\cite{42}. The impact of hyper-parameters are also analysed in supplementary materials. Based on these implementation details, we set the HRN~\cite{2} as our baseline model.
% We implemented our network with PyTorch~\cite{51}, and conducted experiments on a workstation equipped with one NVIDIA RTX 3090 GPU. For fair comparison, we also adopt ResNet50 pre-trained on ImageNet as our network backbone and resize input images to 400$\times$400 throughout the experiments. We train each experiment for 64 epochs with early stopping and use adaptive moment estimation (Adam) with a learning rate of 0.0001 adjusted by the cosine annealing strategy~\cite{27} to optimize our network. The batch size is set to 32.  We set decay factor $\alpha=3$ in Equation (\ref{eq:Le-s}) and (\ref{eq:Le-a}), the balance weight $\beta=0.001$ in Equation (\ref{eq:graph}), and trade-off parameter $\lambda=0.1$ in Equation (\ref{eq:all}). Based on these implementation details, we set HRN~\cite{2} as our baseline model.

\vspace{-10.5pt}

\paragraph{Evaluation.} 
We use the overall accuracy ($OA$) and the area under the average precision and recall curve $AU(\overline{PRC})$ to evaluate the dating performance.
% We use the overall accuracy($OA$) and the area under the average precision and recall curve $AU(\overline{PRC})$ to evaluate the performance of dating performance. 

%%%%%%%%%%%%%%%%%%%%%%%%%%%%%%%%%%%%%%%%%%%%%%
\begin{table} 
\centering
\caption{Ablation study of each component in our network.} 
\label{tab:abalation}
\resizebox{.45\textwidth}{!}{  % Here 1/2
\begin{tabular}{c|c|cc|cccc|cc}
\hline\hline
\multirow{3}{*}{} & \multirow{3}{*}{\begin{tabular}[c]{@{}c@{}}MGM \\ w/ Truncated\end{tabular}} & \multicolumn{2}{c|}{KEM}                                 & \multicolumn{4}{c|}{AKG}                                                               & \multirow{3}{*}{Dynasty $OA$} & \multirow{3}{*}{Period $OA$} \\ \cline{3-8}
                  &                                                                              & \multirow{2}{*}{Shape} & \multirow{2}{*}{Characteristic} & \multicolumn{2}{c|}{Shape}                       & \multicolumn{2}{c|}{Characteristic} &                             &                            \\ \cline{5-8}
                  &                                                                              &                        &                                 & Concat       & \multicolumn{1}{c|}{Embed}    & Concat           & Embed        &                             &                            \\ \hline\hline
1                 &                                                                              &                        &                                 &              & \multicolumn{1}{c|}{}             &                  &                  & 85.28                       & 75.81                      \\ \hline
2                 & $\checkmark$                                                                 &                        &                                 &              & \multicolumn{1}{c|}{}             &                  &                  & 86.85                       & 77.05                      \\ \hline
3                 & $\checkmark$                                                                 & $\checkmark$           &                                 &              & \multicolumn{1}{c|}{}             &                  &                  & 84.54                       & 75.54                      \\ \hline
4                 & $\checkmark$                                                                 & $\checkmark$           &                                 & $\checkmark$ & \multicolumn{1}{c|}{}             &                  &                  & 86.53                       & 76.51                      \\ \hline
5                 & $\checkmark$                                                                 & $\checkmark$           &                                 &              & \multicolumn{1}{c|}{$\checkmark$} &                  &                  & 87.71                       & 77.86                      \\ \hline
6                 & $\checkmark$                                                                 &                        & $\checkmark$                    &              & \multicolumn{1}{c|}{}             &                  &                  & 87.23                       & 77.37                      \\ \hline
7                 & $\checkmark$                                                                 &                        & $\checkmark$                    &              & \multicolumn{1}{c|}{}             & $\checkmark$     &                  & 87.50                       & 77.10                      \\ \hline
8                 & $\checkmark$                                                                 &                        & $\checkmark$                    &              & \multicolumn{1}{c|}{}             &                  & $\checkmark$     & 87.72                       & 77.98                      \\ \hline
9                 & $\checkmark$                                                                 & $\checkmark$           & $\checkmark$                    & $\checkmark$ & \multicolumn{1}{c|}{}             & $\checkmark$     &                  & 86.80                       & 76.62                      \\ \hline
10                & $\checkmark$                                                                 & $\checkmark$           & $\checkmark$                    &              & \multicolumn{1}{c|}{$\checkmark$} &                  &                  & 87.12                       & 77.05                      \\ \hline
11                &                                                                              & $\checkmark$           & $\checkmark$                    &              & \multicolumn{1}{c|}{$\checkmark$} &                  & $\checkmark$     & 87.45                       & \textbf{78.83}             \\ \hline
12                & $\checkmark$                                                                 & $\checkmark$           & $\checkmark$                    &              & \multicolumn{1}{c|}{$\checkmark$} &                  & $\checkmark$     & \textbf{88.79}              & \textbf{78.83}             \\ \hline
\end{tabular}
} 
\end{table}
%%%%%%%%%%%%%%%%%%%%%%%%%%%%%%%%%%%%%%%%%%%%%%

\subsection{Ablation Study}
\vspace{-4pt}
We evaluate different combinations of the proposed components and followed the default parameter settings described in Section ~\ref{ImDetails}. The results are listed in Table ~\ref{tab:abalation}.
% We evaluated different combinations of the proposed components and follow the default parameter setting in Section ~\ref{ImDetails}. The results are shown in Table ~\ref{tab:abalation}.
% \vspace{-16pt}
\vspace{-10.5pt}
\paragraph{Multi-Granularity Module (MGM).} 
We apply an element-wise gradient truncated addition, from the period features to the dynasty features, forming a bi-directional interaction structure. Simple but efficient, this (1$\rightarrow$2) improves the dynasty dating accuracy of the network by $1.57\%$ and the period dating accuracy by $1.24\%$. The removal of the gradient truncated addition from the complete model (11$\rightarrow$12) leads to a $1.34\%$ decrease in the dynasty dating accuracy, while maintaining the period dating accuracy. %This result is also in line with our original intention, and strengthens the dynasty dating while not affecting the period dating.  
This result is also in line with our original intention, which strengthens the dynasty dating while not affecting the period dating.
% We perform element-wise gradient truncated addition from period features to dynasty features, forming a bi-directional interaction structure. Simple but efficient, this \textcolor{red}{(1$\rightarrow$2)} improves the network's dynasty dating accuracy by $1.57\%$ and period dating accuracy by $1.24\%$ respectively. And the removal of gradient truncated addition from the complete model (11$\rightarrow$12) leads to a $1.34\%$ dynasty dating accuracy decrease  while the period dating accuracy is maintained. This result is also in line with our original intention, which not only strengthens the dynasty dating but does not affect the period dating.
\vspace{-10.5pt}

\paragraph{Knowledge Extraction Module (KEM).}
With this module, we first verify the influence of the shape information on the dating performance. After adding the shape head of the KEM to the baseline (2$\rightarrow$3), the dynasty dating accuracy of the network is reduced by $2.31\%$, and the period dating accuracy is reduced by $1.51\%$. After concatenating the extracted shape features with the period features and using them together for period learning (3$\rightarrow$4), the accuracy of the dynasty dating increases by $1.99\%$, and the accuracy of the period dating increases by $0.97\%$.
% In this module, we first verify the influence of shape information on dating performance. After adding the shape head of KEM to the baseline (2$\rightarrow$3), the network's dynasty dating accuracy is reduced by $2.31\%$ and period dating accuracy is reduced by $1.51\%$. And after concatenating the extracted shape features with the period features and using them together for period learning (3$\rightarrow$4), the dynasty dating accuracy increases by $1.99\%$ and the period dating accuracy increases by $0.97\%$.



Second, we verify the influence of characteristic information on the dating performance. The addition of the characteristic head of the KEM (2$\rightarrow$6) improves the dating accuracy of the network by $0.38\%$ and $0.32\%$. After concatenating the extracted characteristic features with the period features and using them together for period learning (6$\rightarrow$7), the accuracy of the period dating decreases by $0.27\%$, whereas the accuracy of the dynasty dating improves by only $0.27\%$. When we simultaneously concatenate the shape and characteristic features into the features of the period for period learning (2$\rightarrow$9), but the accuracies of the dynasty dating and period dating decrease by $0.05\%$ and $0.43\%$, respectively.
% Second, we verify the influence of characteristic information on dating performance. The addition of the characteristic head of KEM (2$\rightarrow$6) improves the network's dating accuracy by $0.38\%$ and $0.32\%$. And after concatenating the extracted characteristic features with the period features and using them together for period learning (6$\rightarrow$7), the period dating accuracy decreases by $0.32\%$ while the dynasty dating accuracy improves by only $0.27\%$. When we simultaneously concatenate the features of shape and characteristic into the features of period for period learning (2$\rightarrow$9), the dynasty dating accuracy and period dating accuracy decrease by $0.05\%$ and $0.43\%$, respectively. 


These results demonstrate that concatenating the attribute information with the period information is inefficient. The external information is far from being fully utilized. In the following, we embed the shape and characteristic predictions into an archaeology knowledge-guided relation graph.
% These results demonstrate that simply concatenating the attribute information with the period information is inefficient. This external information is far from being fully utilized. In the following, we embed the shape and characteristic predictions into the archaeology knowledge-guided relation graph.

%%%%%%%%%%%%%%%%%%%%%%%%%%%%%%%%%%%%%%%%%%%%%%
\begin{table*}[htbp]
\centering
\caption{Comparison of each method on the proposed datasets. Bold indicates the best results, and underlined values are the second best results. The single- and multi-granularity methods use single- and multi-granularity era labels as supervision, respectively. In addition, our method and $A^3$M~\cite{46} use additional attribute annotation information, and Part-based R-CNN~\cite{6} uses additional bounding box annotations.}
% \caption{Comparison of each method on the proposed datasets. The bolded ones are the best results, and the underlined ones are the second best results. \textcolor{red}{The single-granularity method and the multi-granularity method use single-granularity era labels and multi-granularity era labels as supervision, respectively. In addition, our method and $A^3$M~\cite{46} use additional attribute annotations information, and Part-based R-CNN~\cite{6} uses additional bounding box annotations.} }
\label{tab3:Period ACC}
\resizebox{0.98\textwidth}{!}{  % Here 1/2
\begin{tabular}{cc|c|cc|cc|ccc|ccc|ccc}
\hline\hline
\multicolumn{2}{c|}{}                                                                                           &                                 &                      &                                              & \multicolumn{2}{c|}{Shang}                  & \multicolumn{3}{c|}{Western Zhou}                                  & \multicolumn{3}{c|}{Spring and Autumn}                                                    & \multicolumn{3}{c}{Warring States}                                                        \\
\multicolumn{2}{c|}{\multirow{-2}{*}{Method}}                                                                   & \multirow{-2}{*}{w/ Attributes} & \multirow{-2}{*}{$OA$} & \multirow{-2}{*}{$AU(\overline{PRC})$}                    & Early                & Late                 & Early                & Mid                  & Late                 & Early                & Mid                  & Late                                        & Early                                       & Mid                  & Late                 \\ \hline\hline
\multicolumn{1}{c|}{}                                     & ConvNeXt~\cite{50}                  &                                 & 76.01                & {{\ul \textit{0.8397}}} & 73.04                & 82.31                & 75.21                & 82.56                & 80.00                & 77.44                & 64.17                & { {\ul \textit{67.76}}} & { {\ul \textit{56.90}}} & 41.18                & 63.37                \\ \cline{2-16} 
\multicolumn{1}{c|}{}                                     & Part-based R-CNN~\cite{6}           & BBox                    & 69.45                & 0.7796                                       & \textbf{92.59}       & {\ul \textit{84.47}} & 62.54                & \textbf{90.78}       & 67.74                & 62.03                & 62.79                & 46.89                                       & 38.46                                       & \textbf{73.33}       & \textbf{84.91}                \\ \cline{3-3}
\multicolumn{1}{c|}{}                                     & MCL~\cite{48}                       &                                 & 70.41                & 0.7742                                       & 79.55                & 78.56                & 66.85                & 79.58                & 76.28                & 71.57                & 54.23                & 66.02                                       & 35.42                                       & 31.18                & 60.11                \\ \cline{3-3}
\multicolumn{1}{c|}{}                                     & CrossX~\cite{30}                    &                                 & 70.54                & 0.7755                                       & 70.89                & 72.97                & 75.21                & 71.74                & 82.88                & 67.67                & 62.14                & 57.20                                       & 25.00                                       & 57.14                & 69.23                \\ \cline{3-3}
\multicolumn{1}{c|}{}                                     & BCNN~\cite{29}                      &                                 & 71.59                & 0.7402                                       & 71.15                & 80.48                & 70.92                & 76.82                & 78.60                & 73.14                & 57.53                & 59.00                                       & 0.00                                        & 0.00                 & 48.11                \\ \cline{3-3}
\multicolumn{1}{c|}{}                                     & NTS-Net~\cite{47}                   &                                 & 73.06                & 0.7890                                       & 71.15                & 80.93                & 67.28                & 79.17                & 78.15                & 81.56                & 58.24                & 60.48                                       & 54.17                                       & {\ul \textit{70.83}} & 67.78                \\ \cline{3-3}
\multicolumn{1}{c|}{}                                     & $A^3$M~\cite{46}                    & $\checkmark$                    & 75.12                & 0.8002                                       & 79.59                & 78.24                & 74.11                & 85.71                & 78.42                & 78.99                & 62.35                & 63.08                                       & 44.44                                       & 50.00                & 69.32                \\ \cline{3-3}
\multicolumn{1}{c|}{}                                     & SPS~\cite{31}                       &                                 & 76.94                & 0.8245                                       & 80.39                & 83.30                & 73.19                & {\ul \textit{86.51}} & {\ul \textit{85.61}} & 81.21                & 62.38                & 66.36                                       & 42.11                                       & 50.00                & 67.78                \\ \cline{3-3}
\multicolumn{1}{c|}{\multirow{-9}{*}{\rotatebox[origin=c]{90}{Single-Granularity}}} & P2PNet~\cite{45}                    &                                 & {\ul \textit{77.32}} & 0.8370                                       & 79.25                & 78.25                & \textbf{80.39}       & 78.80                & \textbf{88.37}       & \textbf{85.11}       & 64.52                & 67.24                                       & 50.00                                       & 52.78                & 71.59                \\ \hline
\multicolumn{1}{c|}{}                                     &                                                     &                                 & 84.85                & {\ul \textit{0.9125}}                        & \multicolumn{2}{c|}{{\ul \textit{84.37}}}   & \multicolumn{3}{c|}{84.09}                                         & \multicolumn{3}{c|}{{\ul \textit{87.25}}}                                                 & \multicolumn{3}{c}{84.97}                                                                 \\
\multicolumn{1}{c|}{}                                     & \multirow{-2}{*}{YourFL~\cite{1}}   & \multirow{-2}{*}{}              & 73.92                & 0.8019                                       & 79.25                & 80.20                & 69.38                & 82.21                & 83.59                & 82.14                & 56.38                & 62.90                                       & 53.85                                       & 51.72                & 60.67                \\ \cline{3-16} 
\multicolumn{1}{c|}{}                                     &                                                     &                                 & 84.43                & 0.8934                                       & \multicolumn{2}{c|}{82.07}                  & \multicolumn{3}{c|}{84.28}                                         & \multicolumn{3}{c|}{85.91}                                                                & \multicolumn{3}{c}{{\ul \textit{90.41}}}                                                  \\
\multicolumn{1}{c|}{}                                     & \multirow{-2}{*}{C-HMCNN~\cite{43}} & \multirow{-2}{*}{}              & 74.52                & 0.7766                                       & {\ul \textit{81.25}} & 79.29                & 70.04                & 85.57                & 81.95                & 75.00                & \textbf{67.47}       & 60.45                                       & \textbf{58.06}                              & 45.24                & {\ul \textit{77.46}}       \\ \cline{3-16} 
\multicolumn{1}{c|}{}                                     &                                                     &                                 & {\ul \textit{85.28}} & 0.9124                                       & \multicolumn{2}{c|}{81.08}                  & \multicolumn{3}{c|}{{\ul \textit{88.24}}}                          & \multicolumn{3}{c|}{85.98}                                                                & \multicolumn{3}{c}{85.71}                                                                 \\
\multicolumn{1}{c|}{}                                     & \multirow{-2}{*}{HRN~\cite{2}}      & \multirow{-2}{*}{}              & 75.81                & 0.8206                                       & 75.93                & 79.66                & 75.89                & 80.63                & 85.61                & 81.70                & 62.24                & 62.71                                       & 51.06                                       & 48.15                & 68.06                \\ \cline{3-16} 
\multicolumn{1}{c|}{}                                     &                                                     &                                 & \textbf{88.79}       & \textbf{0.9380}                              & \multicolumn{2}{c|}{\textbf{86.80}}         & \multicolumn{3}{c|}{\textbf{88.62}}                                & \multicolumn{3}{c|}{\textbf{91.57}}                                                       & \multicolumn{3}{c}{\textbf{90.45}}                                                        \\
\multicolumn{1}{c|}{\multirow{-8}{*}{\rotatebox[origin=c]{90}{Multi-Granularity}}}  & \multirow{-2}{*}{Ours}                              & \multirow{-2}{*}{$\checkmark$}  & \textbf{78.83}       & \textbf{0.8550}                              & 77.36                & \textbf{84.85}       & {\ul \textit{78.30}} & 81.28                & 85.31                & {\ul \textit{83.33}} & {\ul \textit{64.84}} & \textbf{68.85}                              & 45.95                                       & 53.13                & 75.31 \\ \hline
\end{tabular}
}
\end{table*}
%%%%%%%%%%%%%%%%%%%%%%%%%%%%%%%%%%%%%%%%%%%%%%

%%%%%%%%%%%%%%%%%%%%%%%%%%%%%%%%%%%%%%%%%%%%%%
\begin{table}[]
\centering
\caption{Comparison results of the dating performance showing the effect of the order of the shape and characteristic embedding.}
% \caption{The comparison  results of dating performance show the affect of the order of shape and characteristic embedding.}
\label{tab:order}
\resizebox{.32\textwidth}{!}{  % Here 1/2
\begin{tabular}{cl|cc}
\hline\hline
\multicolumn{2}{c|}{Embedding order}          & Dynasty $OA$     & Period $OA$      \\ \hline\hline
\multicolumn{2}{c|}{Era-Characteristic-Shape} & 86.37          & 76.13          \\ \hline
\multicolumn{2}{c|}{Era-Shape-Characteristic} & \textbf{88.79} & \textbf{78.83} \\ \hline
\end{tabular}
}
\end{table}
%%%%%%%%%%%%%%%%%%%%%%%%%%%%%%%%%%%%%%%%%%%%%%


\vspace{-10.5pt}

\paragraph{Archaeology Knowledge Guided Relation Graph (AKG).}  
To make better use of the additional information, we build an AKG by adding the shape and characteristic predictions extracted from the KEM to the relation graph. Experiment results show that embedding the shape into the relation graph (2$\rightarrow$5) improves the accuracy of the dynasty dating by $0.86\%$ and the accuracy of the period dating by $0.81\%$. Embedding the characteristics into the relation graph (2$\rightarrow$8) improves the accuracy of dynasty dating by $0.87\%$ and accuracy of period dating by $0.93\%$. In the complete model, embedding both the shape and characteristics (2$\rightarrow$12) improves the accuracy of the dynasty dating by $1.94\%$ and the accuracy of the period dating by $1.78\%$.
% To make better use of the additional information, we build the archaeology knowledge guided relation graph (AKG) by adding the shape and characteristic predictions extracted from the KEM to the relation graph. Experiments show that embedding shape into the relation graph (2$\rightarrow$5) improves dynasty dating accuracy by $0.86\%$ and period dating accuracy by$0.81\%$. And embedding characteristic into relation graph (2$\rightarrow$8) improves dynasty dating accuracy by $0.87\%$ and period dating accuracy by $0.93\%$ . In the complete model, embedding both shape and characteristic (2$\rightarrow$12) can improve dynasty dating accuracy by $1.94\%$ and period dating accuracy by $1.78\%$.


It is worth noting that the embedding order of the shape and characteristics also has an impact on the dating performance. As illustrated in Equations (\ref{eq:Le-s}) and (\ref{eq:Le-a}), in our network, we use the Era-Shape-Characteristic order to incrementally learn different information. To verify the influence of the embedding order, we also tested the opposite case. In the Era-Characteristic-Shape learning order, when the model is unable to accurately determine the era of a sample based on its era feature, it first increases the influence of the characteristic features and finally considers the shape feature. The comparison results are shown in Table~\ref{tab:order}, where the embedding learning order of the Era-Characteristic-Shape is $2.42\%$ and $2.70\%$ lower than that of our applied order. This confirms that when learning labels with different distributions, the network needs to learn the labels progressively from weak to strong and from easy to complex. This also explains why the strong supervised method (Part-based R-CNN~\cite{6}) did not achieve excellent dating results in the following comparison experiment.
% It is worth to notice that the embedding order of shape and characteristic also has an impact on the dating performance. As illustrated in Equation (\ref{eq:Le-s}), (\ref{eq:Le-a}), in our network, we use the Era-Shape-Characteristic order to learn the different information incrementally. To verify the influence of embedding order, we also test the opposite way. In the Era-Characteristic-Shape learning order, when the model is unable to accurately determine the era of a sample by the its era feature, it will first increase the influences of the characteristic feature and finally consider the shape feature. \textcolor{red}{In the Era-Characteristic-Shape learning order, when the network is not able
% to learn enough information to determine the era of this
% sample, the $\mathcal{L}_{\text {ec}}(\mathcal{\!D\!})$ will play a supportive role. And when neither the era features nor the characteristic features can provide enough information, the $\mathcal{L}_{\text {es}}(\mathcal{\!D\!})$ will contribute to determine the chronology of this sample by learning the relation be-
% tween era and shape.} The comparison results are shown in Table~\ref{tab:order}, the embedding learning order of Era-Shape-Characteristic is $1.08\%$ and $2.70\%$ higher compared to the other one. We believe this confirms that when learning the labels with different distributions, the network needs to learn the labels in a progressive order from weak to strong and from easy to complex. It also explains why the strong supervised method (Part-based R-CNN~\cite{6}) has not achieved excellent dating results in the following comparison experiment.


%------------------------------------------------------------------


%%%%%%%%%%%%%%%%%%%%%%%%%%%%%%%%%%%%%%%%%%%%%%
\begin{figure*}[htbp]
\centering
  \includegraphics[width=1.0\linewidth]{figure6V2.pdf}
  \caption{Gradient-weighted class activation map of different methods on 11 period test samples. Compared to the other methods, our network is more concentrated on the discriminative regions of a bronze ding and is able to capture its key characteristics.}
  % \caption{The gradient-weighted class activation map of different method on 11 period test samples. Compared to the other methods, our network is more concentrated on the discriminative regions of bronze Ding and is able to capture the key characteristics on the bronze Ding.}
  \label{fig6:CAM}
\end{figure*}
%%%%%%%%%%%%%%%%%%%%%%%%%%%%%%%%%%%%%%%%%%%%%%

%%%%%%%%%%%%%%%%%%%%%%%%%%%%%%%%%%%%%%%%%%%%%%
\begin{figure}[htbp]
\centering
  \includegraphics[width=0.9\linewidth]{figure7.pdf}
  \caption{Visualization of learned representations of different methods on our dataset using T-SNE. Compared to the other methods, the decision boundaries of our proposed network become more separated.}
  % \caption{Visualization of learned representations of different methods on our dataset by using T-SNE. Compared to the other methods, the decision boundaries of our proposed network become more separated.}
  \label{fig7:tsne}
\end{figure}
%%%%%%%%%%%%%%%%%%%%%%%%%%%%%%%%%%%%%%%%%%%%%%

\subsection{Comparison with SOTA Methods}
\vspace{-4pt}
We compare our proposed network with other SOTA approaches under multi- and single-granularity settings. Under a multi-granularity setting, we train all multi-granularity methods with two-level labels of the bronze ding datasets. We report the $OA$ and $AU(\overline{PRC})$ results for each hierarchical level on the test set. Under a single-granularity setting, we train all single-granularity methods using fine-grained period labels of a bronze ding. Besides, to observe the classification performance of each dynasty and period independently, we calculate the precision of each approach on 4 coarse-grained dynasties as well as on 11 fine-grained periods recall is reported in our supplementary material. In addition to research related to fine-grained classification, we also compare our method with ConvNeXt~\cite{50}, which is an extremely popular approach in traditional classification tasks.
% We compare our proposed network with other SOTA approaches under multi-granularity setting and single-granularity setting respectively. In multi-granularity setting, we train all multi-granularity methods with two-level labels of the bronze Ding datasets. We report $OA$ and $AU(\overline{PRC})$ results of each hierarchical level on test sets. In single-granularity setting, we train all single-granularity methods with fine-grained period labels of the bronze Ding and report $OA$ and $AU(\overline{PRC})$ results on test sets. Additionally, in order to observe the classification performance on each dynasty and period independently, we calculate the classification accuracy of each method on 4 coarse-grained dynasties as well as on 11 fine-grained periods. In addition to the fine-grained classification related work, we also compare our method with the ConvNeXt~\cite{50}, which is very popular in traditional classification tasks.

As Table~\ref{tab3:Period ACC} shows, using the same backbone ResNet50, our method outperforms the state-of-the-art 
single-granularity method P2PNet~\cite{45} on the bronze ding dataset benchmark by more than $1.51\%$ $OA$ and $0.018$ $AU(\overline{PRC})$ for period dating. And our method outperforms the state-of-the-art 
multi-granularity method HRN~\cite{2} (our baseline) on the bronze ding dataset benchmark by more than $3.51\%$ $OA$ and $0.0256$ $AU(\overline{PRC})$ in terms of dynasty dating, and by more than $3.02\%$ $OA$ and $0.0344$ $AU(\overline{PRC})$ for period dating. Furthermore, we achieve the best performance for all 4 coarse-grained dynasties and 3 out of 11 fine-grained periods for each independent era classification.
% As Table~\ref{tab3:Period ACC} shows, using the same backbone ResNet50, our method outperforms the state-of-the-art method P2PNet~\cite{45} on the bronze ding dataset benchmark by more than $3.51\%$ $OA$ and $0.0255\%$ $AU(\overline{PRC})$ in terms of dynasty dating, and by more than $1.51\%$ $OA$ and $0.018\%$ $AU(\overline{PRC})$ for period dating. We achieved the best performance for all 4 coarse-grained dynasties and 3 out of 11 fine-grained periods for each independent period classification.

% As the Table~\ref{tab3:Period ACC} shows, with the same backbone ResNet50, our method outperforms the state-of-the-art methods P2PNet~\cite{45} on bronze Ding datasets benchmark by more than $3.51\%$ $OA$ and $0.0255\%$ $AU(\overline{PRC})$ on dynasty dating, and more than $1.51\%$ $OA$ and $0.018\%$ $AU(\overline{PRC})$ on period dating. We achieve the best performance for all four coarse-grained dynasties and three out of eleven fine-grained periods for each independent period classification.
\vspace{-10.5pt}

\paragraph{Visualization.}

To demonstrate that our network can capture important regions of interest useful for bronze ding dating, we adopt Grad-CAM~\cite{52} for an intuitive visualization. For comparison, we also conducte the same visualization for the single-granularity method P2PNet~\cite{45} and the multi-granularity methods HRN~\cite{2} and C-HMCNN~\cite{43}, which also exhibit competitive performances. As shown in Figure~\ref{fig6:CAM}, our network is more concentrated within the discriminative regions of a bronze ding. 
% Compared to the other methods, our network captures the key locations on a bronze ding when applying dating of the decorations and inscriptions, among other characteristics. 
Compared to the other methods, our network captures the key locations on a bronze ding when performing the dating, such as decorations and inscriptions.
% To demonstrate that our network can capture important regions of interests that are useful for bronze Ding dating, we also adopt Grad-CAM~\cite{52} to show intuitive visualization. For comparison, we also perform the same visualization for the single-granularity method P2PNet~\cite{45} and the multi-granularity method HRN~\cite{2} and C-HMCNN~\cite{43}, which also have competitive performance. It can be seen from Figure ~\ref{fig6:CAM} that our network is more concentrated on the discriminative regions of bronze Ding. Compared to the other methods, our network captures the key locations on the bronze Ding when performing the dating,such as decorations and inscriptions.


In addition, as shown in Figure~\ref{fig7:tsne}, we draw t-SNE~\cite{53} scatter plots from the learned high-dimensional period features of our network and some other comparison methods. For better visualization, we randomly select 40 images for each period. From the t-SNE plots, we can clearly see that our network extracts more discriminative period representations of different images.
% A shown in Figure~\ref{fig7:tsne}, we draw the t-SNE~\cite{53} scatter plot from the learned high dimensional period features of our network and some comparison methods. We randomly selecte 40 images in each period for visualization. From the t-SNE plot, we find that our network can extract more discriminative period representations of different images.
%%%%%%%%%%%%%%%%%%%%%%%%%%%%%%%%%%%%%%%%%%%%%%%%%%%%%%%%%%%%%%%%%%%%%%%%%%%%%%%%
%%%%%%%%%%%%%%%%%%%%%%%%%%%%%%%%%%%%%%%%%%%%%%%%%%%%%%%%%%%%%%%%%%%%%%%%%%%%%%%%


\vspace{-10.5pt}


\section{Conclusion and Further Work}
\vspace{-4pt}
In this study, we introduce a bronze ding dataset with rich archaeological labels. To address the challenges, we construct an end-to-end multihead network for predicting the era of bronze dings based on an AKG. Comprehensive experiments are conducted on our dataset, the results of which demonstrate the effectiveness of our proposed network in comparison to existing multi- and single-granularity FGVC methods. And, excitingly, our learning-based dating network achieve the same level of human experts. 

In a future study, we plan to collect and open up more types of Chinese bronze data to facilitate research through learning-based methods. We hope that our study will provide further contributions to both the archaeological and deep-learning communities.

% Paleography and Chinese Civilization Inheritance and Development Program
\vspace{-10.5pt}

\paragraph{Acknowledgement.} This work is supported by the "Paleography and Chinese Civilization Inheritance and Development Program" Collaborative Innovation Platform (Grant No.G3829) and Jilin University (Grant No.419021421665 and No.419021422B08).

% Thanks to the "Engineering Platform for the Inheritance and Development of Ancient Characters and Chinese Civilization" for funding the project "Construction of Artificial Intelligence Recognition System for Ancient Characters" (Grant No.G3829). This work is also supported by Jilin University (Grant No.419021421665).

% In this work, we introduce a bronze Ding dataset with rich archaeological labels. To address these challenges of our task, we construct an end-to-end multi-head network for predicting the era of bronze Dings based on an archaeology knowledge guided relation graph. Comprehensive experiments are conducted on our dataset, and the results demonstrated the effectiveness of our proposed network compared to the existing multi-granularity and single-granularity FGVC methods. Excitingly, our learning-based dating results have achieved the same level of human exerts. 

% In further work, we will collect and open up more types of Chinese bronze data to facilitate research with learning-based methods. We hope that our work will provide further contributions to both the archaeological and deep learning communities.
% This must be in the first 5 lines to tell arXiv to use pdfLaTeX, which is strongly recommended.
\pdfoutput=1
% In particular, the hyperref package requires pdfLaTeX in order to break URLs across lines.

\documentclass[11pt]{article}

% Remove the "review" option to generate the final version.
%\usepackage[review]{ACL2023}
\usepackage{ACL2023}

% Standard package includes
\usepackage{times}
\usepackage{latexsym}

% For proper rendering and hyphenation of words containing Latin characters (including in bib files)
\usepackage[T1]{fontenc}
% For Vietnamese characters
% \usepackage[T5]{fontenc}
% See https://www.latex-project.org/help/documentation/encguide.pdf for other character sets

% This assumes your files are encoded as UTF8
\usepackage[utf8]{inputenc}

% This is not strictly necessary, and may be commented out.
% However, it will improve the layout of the manuscript,
% and will typically save some space.
\usepackage{microtype}

% This is also not strictly necessary, and may be commented out.
% However, it will improve the aesthetics of text in
% the typewriter font.
\usepackage{inconsolata}


% If the title and author information does not fit in the area allocated, uncomment the following
%
%\setlength\titlebox{10cm}
%
% and set <dim> to something 5cm or larger.

%%%%%%%%%%%%%%%%%%%%%%%%%%%%%%%%%%
\usepackage{graphicx}
\usepackage{amsfonts}
\usepackage{amsmath}
\usepackage{bigdelim}
\usepackage{diagbox}
\usepackage{amsthm}
\usepackage{makecell}
\usepackage{mathtools}
\usepackage{booktabs}
\usepackage[shortlabels]{enumitem}
\graphicspath{ {figs/} }

\theoremstyle{remark}
\newtheorem*{question}{Question}

\newcommand{\tk}[1]{\textcolor{blue}{{#1}}}
\newcommand{\sy}[1]{\textcolor{red}{{#1}}}
\newcommand{\mg}[1]{\textcolor{purple}{{#1}}}
\newcommand{\lh}[1]{\textcolor{green}{{#1}}}
\newcommand{\lc}[1]{\textcolor{green}{{#1}}}

% Rounded color box
\definecolor{light_blue}{HTML}{cfdfff}
\usepackage[most]{tcolorbox}
\tcbset{on line, 
        boxsep=1pt, left=0pt,right=0pt,top=0pt,bottom=0pt,
        colframe=white,colback=light_blue,  
        highlight math style={enhanced}
        }

\newcommand{\quash}[1]{}  %Anything in \quash is ignored
\newcommand{\gpt}{\textsc{GPT-2}}
\newcommand{\bert}{\textsc{BERT}}
\newcommand{\bertlarge}{\textsc{BERT-large}}
\newcommand{\mask}{\texttt{[MASK]}}
\newcommand{\cls}{\texttt{[CLS]}}
\newcommand{\sep}{\texttt{[SEP]}}
\newcommand{\mat}{\texttt{mat}}
\newcommand{\id}{\texttt{id}}
\newcommand{\matl}{\texttt{mat}_{\ell \rightarrow \ell'}}
\newcommand{\matattnl}{\texttt{mat\_attn}_{\ell \rightarrow \ell'}}
\newcommand{\matffl}{\texttt{mat\_ffn}_{\ell \rightarrow \ell'}}
\newcommand{\matlnl}{\texttt{mat\_ln1\_ln2}_{\ell \rightarrow \ell'}}
\newcommand{\idl}{\texttt{id}_{\ell \rightarrow \ell'}}
\newcommand{\matlL}{\texttt{mat}_{\ell \rightarrow L}}
\newcommand{\matattnlL}{\texttt{mat\_attn}_{\ell \rightarrow L}}
\newcommand{\matfflL}{\texttt{mat\_ffn}_{\ell \rightarrow L}}
\newcommand{\matlnlL}{\texttt{mat\_ln1\_ln2}_{\ell \rightarrow L}}
\newcommand{\idlL}{\texttt{id}_{\ell \rightarrow L}}

\definecolor{blue(munsell)}{rgb}{0.0, 0.5, 0.69}
%%%%%%%%%%%%%%%%%%%%%%%%%%%%%%%%%%

\title{Jump to Conclusions: Short-Cutting Transformers\\With Linear Transformations}

% Author information can be set in various styles:
% For several authors from the same institution:
% \author{Author 1 \and ... \and Author n \\
%         Address line \\ ... \\ Address line}
% if the names do not fit well on one line use
%         Author 1 \\ {\bf Author 2} \\ ... \\ {\bf Author n} \\
% For authors from different institutions:
% \author{Author 1 \\ Address line \\  ... \\ Address line
%         \And  ... \And
%         Author n \\ Address line \\ ... \\ Address line}
% To start a seperate ``row'' of authors use \AND, as in
% \author{Author 1 \\ Address line \\  ... \\ Address line
%         \AND
%         Author 2 \\ Address line \\ ... \\ Address line \And
%         Author 3 \\ Address line \\ ... \\ Address line}

\author{Alexander Yom Din$^{1}$ ~~~~~ Taelin Karidi$^{1}$ ~~~~~ Leshem Choshen$^{1}$ ~~~~~
Mor Geva$^{2}$ 
\vspace{0.2cm} \\
$^1$Hebrew University of Jerusalem ~~~ $^2$Google Research \\
\small{\texttt{\{alexander.yomdin, taelin.karidi, leshem.choshen\}@mail.huji.ac.il}}, \small{\texttt{pipek@google.com}}}

\quash{
\author{Alexander Yom Din \\
  Hebrew University of Jerusalem \\ \texttt{alexander.yomdin@mail.huji.ac.il} \\\And
  Taelin Karidi \\
  Hebrew University of Jerusalem \\
  \texttt{taelin.karidi@mail.huji.ac.il} \\\And
  Leshem Choshen \\
  Hebrew University of Jerusalem \\ \texttt{leshem.choshen@mail.huji.ac.il} \\\And
  Mor Geva \\
  Google Research \\
  \texttt{pipek@google.com} \\}
}

\begin{document}
\maketitle



\begin{abstract}
% \vspace{-1em}
The diffusion-based generative models have achieved remarkable success in text-based image generation. However, since it contains enormous randomness in generation progress, it is still challenging to apply such models for real-world visual content editing, especially in videos. 
In this paper, we propose \texttt{FateZero}, a zero-shot text-based editing method on real-world videos without per-prompt training or use-specific mask. 
\RM{Specifically, different from a pipeline of two independent inversion and then generation stages, we find the intermediate attention maps during inversions store better structure and motion information. We thus reform them to temporally casual attention and replace them in the generation progress. To further reduce the unnecessary semantic leakage of source video and enhance the editing quality, we then remix the temporally casual attentions via the cross-attention features of the source prompt as the mask.}
To edit videos consistently, we propose several techniques based on the pre-trained models. Firstly, in contrast to the straightforward DDIM inversion technique, our approach captures intermediate attention maps during inversion, which effectively retain both structural and motion information. These maps are directly fused in the editing process rather than generated during denoising. To further minimize semantic leakage of the source video, we then fuse self-attentions with a blending mask obtained by cross-attention features from the source prompt. Furthermore, we have implemented a reform of the self-attention mechanism in denoising UNet by introducing spatial-temporal attention to ensure frame consistency.
Yet succinct, our method is the first one to show the ability of zero-shot text-driven video style and local attribute editing from the trained text-to-image model. We also have a better zero-shot shape-aware editing ability based on the text-to-video model~\cite{tuneavideo}. \RM{Besides video, our unified method also achieves state-of-the-art performance in zero-shot image editing.\chenyang{Need exp or remove the zero-shot image}} Extensive experiments demonstrate our superior temporal consistency and editing capability than previous works.
% The code will be released.
% \chenyang{emphasize: our observation at inversion time} \xiaodong{replacing the bold part to the actual pipeline: \textbf{Specifically, we work on replacing and mixing the attention maps between the inversion and generation since the self-attention map keeps the structure of the original natural image and the cross-attention is semantic-related, after remixing, we replace them in the corresponding generation steps for denoising.}}
% \footnote{Since there is no general video diffusion model is publicly available, we use one-shot video generation method~(Tune-A-Video~\cite{tuneavideo}) as the pretrained video diffusion model for zero-shot video editing\xiaodong{can be removed if we actually zero-shot on video}.}.
\end{abstract}
\section{Introduction}

The ability to reason about plans is critical for performing long-horizon tasks \citep{erol1996hierarchical, sohn2018hierarchical, sharma-etal-2022-skill}, compositional generalization \citep{corona-etal-2021-modular} and generalization to unseen tasks and environments \citep{shridhar2020alfred}.
Consider a simple long-horizon planning scenario where a robot is tasked with preparing a meal and serving it on the table. 
This presents a non-trivial planning problem since the agent needs to understand the sequence of operations required to perform the task and search for the relevant objects in the unfamiliar environment by interacting with various objects. %



Large language models have been recently shown to possess commonsense knowledge about the world such as object affordances and physical dynamics \citep{ouyang2022training,chowdhery2022palm}.
Early approaches considered text based environments and fine-tuned PLMs to predict actions given the history of past observations and actions \citep{jansen-2020-visually,micheli-fleuret-2021-language,yao-etal-2020-keep}.
Recent work has used this ability to reason about plans from text instructions in simulated household environments with simplifying assumptions such as text-only environment observations or feedback \citep{huang2022language,ahn2022can,li2022pre,logeswaran-etal-2022-shot}.


We focus on \emph{visually grounded planning} with PLMs --- the ability to adapt plans based on interaction and visual feedback from the environment.
While PLMs have strong planning commonsense priors, predictions from a PLM may not be directly realizable in the environment since the observation and action spaces are unknown.
This requires \emph{grounding} the PLM in the environment and adapting it to observe visual feedback, which is highly non-trivial.
Some prior works assume the availability of a pre-trained affordance function \citep{ahn2022can} or a success detector \citep{mirchandani2021ella}.
Notably, SayCan \citep{ahn2022can} completely decouples the PLM from observation information by selecting actions that have both high affordability (through a pre-trained affordance model) and high PLM likelihood.
Although this partially addresses the grounding problem, the use of visual feedback for action affordance alone is limited.
Often an agent must choose one of many affordable actions using information from observations.
For example, a driving agent should re-navigate and possibly turn around when encountering a ``road closed'' sign, but both turning around and driving forward are indistinguishable to SayCan because they are both affordable and the PLM is blind to observations.

Another workaround explored in prior work is translating the information in the visual observations to text using a pre-trained captioning system \citep{shridhar2021alfworld,huang2022language}.
However, it can be difficult to faithfully describe an image in words and information is lost in this inherently noisy process, which limits the information available to the planner.



Recent work shows that PLMs can be adapted for various natural language tasks by inserting tunable embeddings or soft prompts at the input of the PLM (also called prompt tuning or prefix tuning)~\citep{li-liang-2021-prefix,lester-etal-2021-power}.
This approach also extends to multi-modal understanding tasks such as image captioning \citep{mokady2021clipcap} and VQA \citep{tsimpoukelli2021multimodal} where images are encoded as soft prompts and finetuned for the target task.
Transformer based architectures have also been successfully applied to offline Reinforcement Learning in recent work \citep{chen2021decision,janner2021offline,li2022pre,reid2022can}.

Taking inspiration from these works, we propose the simple approach of embedding visual observations (`visual prompts') and \textit{directly inserting them as PLM input embeddings}.
The visual encoder and PLM are jointly trained for the target task, an approach we call \textbf{\oursfull}~(\ours).
By teaching the PLM to use observations for planning in an end to end manner, we remove the dependency on external data such as captions and affordability information that was used in prior work.
We show that this simple approach performs better than prior PLM-based planning approaches on two embodied planning benchmarks based on ALFWorld~\citep{shridhar2021alfworld} and Virtualhome~\cite{puig2018virtualhome}.



\section{Related Work}

%Here we summarize prior work on transfer learning and property inference.

%\shortsection{Transfer Learning}
%%Transfer learning reuses features learned by pre-trained models for new tasks, with the pretext that inherent similarities in the generic features will be useful for the downstream tasks and hence reducing their cost of downstream training. Specifically, the downstream model trainer will use a pre-trained upstream model as the starting point for the downstream training, with inclusion of (or replacement with) the task-specific classification layer/module. The downstream model is then trained by either updating all layers of the model (including ones reused from upstream model) or freezing some earlier layers of the reused parts as the ``feature extractor'' and only updating the rest. The latter approach is more popular as the reused feature extractors can already learn useful feature representations and the training cost is also much lower and affordable for individuals with limited computational resources. We study the vulnerability of the latter transfer learning approach in this paper. 


%\shortsection{Transfer Learning} 
Several works have demonstrated risks associated with transfer learning across a variety of attack goals. Wang et al.~\cite{wang2018great} and Yao et al.~\cite{yao2019latent} consider manipulating the upstream model such that the fine-tuned downstream models contain backdoors, misclassifying test inputs that contain predefined backdoor triggers. These transfer manipulations are tailored to their particular attack goals and cannot be applied for the property inference goal considered in this paper. Zou et al.~\cite{zou2020privacy} study the threat of membership inference attacks on transfer learning, but with normally trained upstream models.  
%\dnote{its clear that the goals are different for these attacks, but how similar are the methods?} \ynote{similarity of the methods? more details about the methods? do not know what is expected here}
%In contrast, we investigate the possibility of boosting the effectiveness of property inference by manipulating the upstream model training. % Schuster et al.~\cite{schuster2020humpty} show that the attacker can modify the corpus on which the word embedding is trained such that the downstream NLP models which use that embedding will behave abnormally.

%\shortsection{Property Inference}
The risk of property inference was introduced by Ateniese et al.~\cite{ateniese2015hacking}, % introduces the threat of inferring properties of the training data from pre-trained models, 
and several subsequent works have developed property inference (also known as distribution inference) attacks~\cite{Wang2022GroupPI, suri2022formalizing, Jurez2022BlackBoxAF, Hartmann2022DistributionIR}.
% Ganju et al.~\cite{ganju2018property} and Suri and Evans~\cite{suri2022formalizing} 
These works study property inference against normally trained models, and they launch attacks using a variety of black-box and white-box attacks. All the white-box attacks use meta-classifiers, which take the permutation-invariant representation~\cite{ganju2018property} of the model parameters as the features. We use the state-of-the-art white-box attack~\cite{suri2022formalizing} in our experiments.
%We will use the state-of-the-art white-box method proposed by Ganju et al.~\cite{ganju2018property} and later extended by suri et al.~\cite{suri2022formalizing} in this paper.
%\dnote{do we use these attacks?} 
Melis et al.~\cite{melis2019exploiting} and Zhang et al.~\cite{zhang2021leakage} focus on property inference in distributed training scenarios. In their settings, the attacker is a participant in the global model training and conducts property inference using meta-classifiers that are trained on model outputs or gradients. Similarly, Suri et al.~\cite{suri2022subject} focus on federated learning settings where the attacker is a participant (or the central server) that utilizes black-box attacks for inferring membership of data from particular subjects. %\dnote{if we use black-box attacks, explain which ones, or how ours are related to previous ones} 
For our experiments, We improve the black-box meta-classifier proposed by Zhang et al.~\cite{zhang2021leakage} using the ``query tuning'' technique in Xu et al.~\cite{xu2019detecting}. 

The closest works to ours are Chase et al.~\cite{saeed} and Chaudhari et al.~\cite{Chaudhari2022SNAPEE}, which both consider a scenario where the attacker can manipulate some of the training data of the model to induce a model that significantly increases property inference risk.
% \dnote{it enables precise property inference attacks?}.
These works assume an adversary with the ability to poison the victim's training data, while the adversary in our scenario has no access to the victim's training data, and therefore, their methods are not applicable.
% \dnote{example how different from ours, and why the methods are not applicable}
%Thus, their methods are not applicable to our transfer learning scenario.
%Their methods rely on inducing certain behavior correlated with the properties to be inferred, and thus are not applicable to our transfer learning scenario. \anote{Still a bit unclear why that is the case.}
%
There are also works similar to ours that leverage ``adversarial initializations'' for attack purposes.
% \cite{grosse2019adversarial, boenisch2021curious, wen2022fishing, fowl2021robbing}.
Grosse et al.~\cite{grosse2019adversarial} focus on scenarios where the attacker can control the parameter initialization of a model, and demonstrate that the attacker can use special initializations to damage the performance of the trained model. %This attack is orthogonal to ours.
Other works \cite{boenisch2021curious, wen2022fishing, fowl2021robbing} show that the malicious central server in a federated learning protocol can reconstruct some training samples via falsifying the global model in some training rounds and then analyzing the submitted gradients. These kinds of attacks do not apply to our transfer-learning scenario since the attacker cannot access the downstream gradients, and can only manipulate the upstream training.

\iffalse %%%%%%%%%%%%%%%%%%%%%%%%%%%%%%%%

In this section, we provide the background and also the summary of prior attacks on transfer learning (Section~\ref{sec:transfer_learning}) and property inference (Section~\ref{sec:property_inference}). Then, we introduce the closely related manipulation attacks against machine learning models to boost different privacy risks in Section~\ref{sec:active_inference_attacks}.

%\anote{Do we really need a dedicated section for this? It's barely 2 paragraphs right now.}

%\dnote{the most closely related work to ours are works that attempt to amplify inference attacks by poisoning models, the two most relevant I know of are \url{https://www.computer.org/csdl/proceedings-article/sp/2022/131600b569/1CIO8nmuota} and \url{https://arxiv.org/abs/2204.00032}, but need to look thoroughly for others. We should definitely be describing this and relating it to our work, probably in the introduction. Most of what is here is Background, but should be clear what this section is for (not muddling background and related work)}

\subsection{Transfer Learning} \label{sec:transfer_learning}
Transfer learning reuses features learned by pre-trained models for new tasks, with the pretext that inherent similarities in generic features can be useful for downstream tasks, thus reducing the cost of downstream training. Specifically, the downstream model trainer uses a pre-trained upstream model as the starting point for downstream training, with the inclusion (or replacement) of task-specific classification layers/modules. The downstream model is then trained by either updating all layers of the model (including ones reused from the upstream model) or freezing some earlier layers of the reused parts as the ``feature extractor'' and only updating the rest. The latter approach is more popular as the reused feature extractors can already learn useful feature representations and the training cost is also much lower and affordable for individuals with limited computational resources. We study the vulnerability of the latter transfer learning approach in this paper. 
%mainly in two ways:  1) all the layers (including ones reused from ) and tune the full model; the other one is to freeze some earlier layers of the model as the feature extractor and only tune the rest later layers. The second update strategy could achieve better efficiency since the frozen layers can already produce meaningful feature representations~\cite{wang2018great,yao2019latent}, and we will study the transfer learning using this strategy. 

Recently, various attacks have been proposed for the transfer learning setting, but with different attack goals from ours. Wang et al.~\cite{wang2018great} generate adversarial examples against black-box student models that transfer knowledge from publicly available teacher models without repeated queries. Yao et al.~\cite{yao2019latent} propose to manipulate the upstream model such that the downstream models derived from the upstream model contain backdoors, which would misclassify test inputs that contain some predefined backdoor triggers. Zou et al.~\cite{zou2020privacy} study the threat of membership inference attacks on transfer learning and the upstream models are trained normally. In contrast, we investigate the possibility of boosting the effectiveness of property inference by manipulating the upstream model training. Schuster et al.~\cite{schuster2020humpty} show that the attacker can modify the corpus on which the word embedding is trained such that the downstream NLP models which use that embedding will behave abnormally.

%This additionally allows model trainers to achieve satisfactory performance with limited training samples, leading to reduced computational costs. The most common approach reuses parameters in the earlier layers of the pre-trained model, either by fixing them as the feature extractor or just using them for initialization, to conduct downstream training.

\subsection{Property Inference} \label{sec:property_inference}

\shortsection{Property Inference Attacks} In property inference attacks, the adversary aims to infer some sensitive properties of some data, given a model trained on it. For example, the adversary may be interested in sensitive properties like the presence of people of a specific race in the dataset~\cite{ateniese2015hacking, melis2019exploiting}), or even be curious about the 
the statistics of the training set (e.g, the ratio of people with a specific gender~\cite{saeed, ganju2018property, suri2022formalizing, zhang2021leakage}).


Ateniese et al.~\cite{ateniese2015hacking} were the first to identify the threat of inferring properties of the training data from pre-trained models. Ganju et al.~\cite{ganju2018property} and Suri and Evans~\cite{suri2022formalizing} 
study property inference against normally trained models, and they launch attacks using white-box meta-classifiers, which utilize the permutation-invariance representation~\cite{ganju2018property} of the model parameters, while other works focus on distributed training~\cite{zhang2021leakage} where the attacker is a participant in the global model training and conducts property inference using meta-classifiers trained on model outputs. Similarly, Suri et al.~\cite{suri2022subject} focus on federated learning, where the attacker is a participant (or the central server) that utilizes black-box attacks for inferring membership of data from particular subjects. Chase et al.~\cite{saeed} propose an active property inference attack for data poisoning scenarios, which we will cover and compare to in Section~\ref{sec:active_inference_attacks}.

%The closest work to ours are by Chase et al.~\cite{saeed} and Tramer et al.~\cite{tramer2022truth}. In their work, the attacker can manipulate some of the training data of the model such that a model trained (from scratch) on the poisoned data has an increased inference risk. However, their methods are not applicable to the transfer learning scenario. 
%In this work, we will focus on the property inference in transfer learning scenarios in which the attacker releases the upstream model and infer sensitive properties of the downstream models tuned from that upstream model.
% 

\shortsection{Defenses}
Defending against property inference attacks is an open problem. There are no studies in the current literature on active adversaries, and only a couple on passive ones. Ma et. al.~\cite{ma2021nosnoop} propose a defense against property inference attacks on data batches in the  collaborative learning setting. However, adversaries in the transfer-learning setting do not have access to batch-wise gradients of the downstream trainer. Chen and Ohrimenko~\cite{chen2022protecting} utilize mechanisms that add carefully-crafted noise to features to provide theoretical guarantees against inference adversaries, but focus on query-based access to the underlying dataset, not a machine learning model trained on it. These existing defenses thus do not apply to our threat model.

%propose a framework that reduces property inference to Boolean functions of individual members, posing the ratio of members satisfying the given function in a dataset as the property. These property inference attacks have since then been proposed as distribution inference attacks~\cite{suri2022formalizing}, presenting such attacks as inferring properties of the distributions used to sample datasets, differentiating them from exact inference attacks like dataset inference~\cite{maini2021dataset}. Nearly all property inference attacks use meta-classifiers to perform inference: training models on versions of datasets with and without the target property, followed by training a meta-classifier on top of these classifiers's model representations. These representations can take several forms: using model weights themselves with permutation-invariance~\cite{ganju2018property}, or model activations or logits for a generated set of query points~\cite{xu2019detecting}. However, the capability of such approaches is limited: the most that these attacks have been shown to work is medium-sized convolutional networks on the CelebA dataset~\cite{suri2022formalizing}.


\subsection{Active Privacy Attacks} \label{sec:active_inference_attacks}
% Perhaps the closely related works to ours as ones that proactively enhance the effectiveness of privacy attacks by manipulating the model training process in certain ways~\cite{saeed, melis2019exploiting, nasr2019comprehensive, tramer2022truth}. 
%shown that the adversary can, by using proactive ways, achieve stronger attacks that infer private information from deep learning systems~\cite{nasr2019comprehensive, melis2019exploiting, tramer2022truth, saeed}. In this section, we introduce the ones that are close to ours.

In the decentralized federated learning training, by submitting specially crafted gradients to the central server, malicious agents can increase membership inference risk~\cite{nasr2019comprehensive} and property inference risks~\cite{melis2019exploiting} of other benign agents' training data. However, these attacks do not apply to transfer learning scenario, as the attacker cannot control model gradients of downstream training. In the centralized setting, researchers propose attacks to poison the victim's training data such that the impacts of attribute inference and membership inference~\cite{tramer2022truth} and property inference~\cite{saeed} attacks are amplified on the poisoned model.
The ability to poison the victim's data is a threat model orthogonal to ours, since we have no access to the victim's downstream data. While there is scope to combine such approaches for stronger attacks (albeit with stronger access assumptions), we choose to focus on the scenario with no read/write access to the victim's data.

\fi %%%%%%%%%%%%%%%%%%%%%%%%%%%%%%%%

\section{Linear Shortcut Across Blocks}
\label{sec:layer_jump}

To use a hidden representation from layer $\ell<L$ as a final representation, we propose to cast it using linear regression, while skipping the computation in-between these layers. More generally, this approach can be applied to cast any $\ell$-th hidden representation to any subsequent layer $\ell'>\ell$.


\subsection{Method}
\label{subsec:methodology_linear_shortcut}

Given a source layer $\ell$ and a target layer $\ell'$ such that $0 \leq \ell < \ell' \leq L$, our goal is to learn a mapping
%$A_{\ell', \ell} \in \mathbb{R}^{d_h \times d_h}$
from hidden representations at layer $\ell$ to those at layer $\ell'$. To this end, we first collect a set of corresponding hidden representation pairs $(h^\ell, h^{\ell'})$. Concretely, we run a set $\mathcal{T}$ of input sequences through the model, and for each input $s$, we extract the hidden representations $h_{i_s}^{\ell}, h_{i_s}^{\ell'}$, where $i_s$ is a random position in $s$.
Next, we learn a matrix $A_{\ell', \ell} \in \mathbb{R}^{d_h \times d_h}$ by fitting linear regression over $\mathcal{T}$, i.e., $A_{\ell', \ell}$ is a numerical minimizer for:
$$ A \mapsto \sum_{s \in \mathcal{T}} || A \cdot h_{i_s}^\ell - h_{i_s}^{\ell'} ||^2,$$ 
and define the mapping of a representation $h$ from layer $\ell$ to layer $\ell'$ as:
\begin{equation}
\label{eq:linear_jump}
    \matl{} (h) \coloneqq A_{\ell', \ell} \cdot h.
\end{equation}


\subsection{Baseline}
\label{subsec:baseline}

We evaluate 
% our method against 
the prevalent approach of ``reading'' hidden representations directly, without any transformation. 
Namely, the propagation of a hidden representation from layer $\ell$ to layer $\ell'$ is given by the identity function, dubbed \id{}:

$$ \idl{} (h) \coloneqq h.$$

% Notably, 
This baseline 
assumes that representations at different layers operate in the same linear space.

\subsection{Quality of Fit}
\label{subsec:experiments_r2}

We first evaluate our method by measuring how well the learned linear mappings approximate the representations at the target layer. To this end, we calculate the (coordinate-averaged) $r^2$-score of our mapping's outputs with respect to the representations obtained from a full inference pass, and compare to the same for the \id{} baseline.


\paragraph{Models.}

We use \gpt{} \cite{radford2019language}, a decoder-only auto-regressive LM, with $L = 48$, $d_h = 1600$, and \bert{} \cite{devlin-etal-2019-bert}, an encoder-only model trained with masked language modeling, with $L=24$, $d_h=1024$.
% \footnote{\label{footnote:hf}We use models and data from Huggingface \cite{wolf-etal-2020-transformers,lhoest-etal-2021-datasets}.}
%For masked token prediction, we use a masked LM head pre-trained for our \bert{} model.

% \footnote{Specifically, we use the Huggingface Transformers \cite{wolf-etal-2020-transformers} implementations of all these models.}

%\sy{We use \gpt{} \cite{radford2019language}, a decoder-only auto-regressive LM, coming in four scales; $\texttt{gpt2}$ ($L = 12$, $d_h = 768$), $\texttt{gpt2-medium}$ ($L = 24$, $d_h = 1024$), $\texttt{gpt2-large}$ ($L = 36$, $d_h = 1280$) and $\texttt{gpt2-xl}$ ($L = 48$, $d_h = 1600$). Also, we use \bert{} \cite{devlin-etal-2019-bert}, an encoder-only model trained with masked language modeling, coming in two scales;  \texttt{bert-base-uncased} ($L=12$, $d_h=768$) and \texttt{bert-large-uncased} ($L=24$, $d_h=1024$). For masked token prediction, we use masked LM heads pre-trained for our models. Specifically, we use the Huggingface Transformers \cite{wolf-etal-2020-transformers} implementations of all these models. The plots presented in this section are for $48$-layered \gpt{} and $24$-layered \bert{}.}

%\sy{We use \gpt{} \cite{radford2019language}, a decoder-only auto-regressive LM, in the Huggingface \cite{wolf-etal-2020-transformers} implementation\footnote{\url{https://huggingface.co/gpt2}}, coming in four scales; $\texttt{gpt2}$ ($L = 12$, $d_h = 768$), $\texttt{gpt2-medium}$ ($L = 24$, $d_h = 1024$), $\texttt{gpt2-large}$ ($L = 36$, $d_h = 1280$) and $\texttt{gpt2-xl}$ ($L = 48$, $d_h = 1600$). Also, we use \bert{} \cite{devlin-etal-2019-bert}, an encoder-only model trained with masked language modeling, in the Hugginface implementation, coming in two scales;  \texttt{bert-base-uncased}\footnote{\url{https://huggingface.co/bert-base-uncased}} ($L=12$, $d_h=768$) and \texttt{bert-large-uncased}\footnote{\url{https://huggingface.co/bert-large-uncased}} ($L=24$, $d_h=1024$). For masked token prediction, we use the \texttt{BertForMaskedLM} heads from Huggingface, pretrained for these models. The plots presented in this section are for $48$-layered \gpt{} and $24$-layered \bert{}.}

\paragraph{Data.}
We sample random sentences from Wikipedia,
% \footref{footnote:hf} 
collecting 9,000 (resp. 3,000) sentences for the training set $\mathcal{T}$ (resp. validation set $\mathcal{V}$).\footnote{We use sentences rather than full documents to simplify the analysis.}
%\sy{We use two data sources to evaluate our method. One is Wikiepdia \cite{lhoest-etal-2021-datasets}\footnote{\url{https://huggingface.co/datasets/wikipedia}}; we use \texttt{spaCy}\footnote{\url{https://spacy.io/}} to divide documents into sentences\footnote{We use sentences rather than full documents to simplify the analysis.}\footnote{We pick randomly a Wikipedia document and then pick randomly a sentence ending in a newline character in it. \sy{[maybe this footnote is not needed?]}}, collecting 9,000 (resp. 3,000) random sentences for the training set $\mathcal{T}$ (resp. validation set $\mathcal{V}$). The second is a news article sentences dataset, the 10K English 2020 news sentences corpus
% \footnote{\url{https://downloads.wortschatz-leipzig.de/corpora/eng_news_2020_10K.tar.gz}} from the Leipzig Corpora Collection \cite{goldhahn-etal-2012-building}, which we randomly divide into a training set $\mathcal{T}$ consisting of 9,000 examples and a validation set $\mathcal{V}$ consisting of 1,000 examples.
% We truncate sentences to the maximal token length allowed by the model \mg{do we ever need to truncate? a sentence has about 10 words and the max. input len is thousands} \sy{[I surely did not need to in Leipzig, but discovered (via a transformers runtime warning) that I do need to for some (probably a minority) of the Wikipedia sentences. This probably has to do with that it is not really ``sentences" necessarily, for example, I noticed that it has some listings or something like that (bulleted items)... So some minority might get very long I guess...]}.
For each example $s$, we select a random position $i_s$ and extract the hidden representations $h_{i_s}^{\ell}$ at that position from all the layers.
For \bert{}, we first replace the input token at position $i_s$ with a \mask{} token, as our motivation is interpreting predictions, which are obtained via masked tokens in \bert{} (see \S\ref{subsec:BERT}).
Thus, in this case, the hidden representations we consider
%in the case of \bert{}
are of \mask{} tokens only.
%As we observed highly similar results for the two data sources across all our experiments, throughout the paper we will mainly report results for Wikipedia (except for \S\ref{sec:robustness}, where we cross-validate).


\begin{figure}[t]
\includegraphics[scale=0.2]{figs/r2_scores_48.pdf}
% \includegraphics[width=\columnwidth]{figs/r2_scores_48.pdf}
\caption{The coordinate-averaged $r^2$-score of $\matl{}$ (left) and $\idl{}$ (right) (\gpt{}).}
\label{fig:r2_scores}
\end{figure}


\begin{figure}[t]
\setlength{\belowcaptionskip}{-10pt}
\includegraphics[scale=0.2]{figs/bertmask_r2_scores_24.pdf}
% \includegraphics[width=\columnwidth]{figs/bertmask_r2_scores_24.pdf}
\caption{The coordinate-averaged $r^2$-score of $\matl{}$ (left) and $\idl{}$ (right) (\bert{}).}
\label{fig:bertmask_r2_scores}
\end{figure}



\paragraph{Evaluation.}
For every pair of layers $\ell, \ell'$, such that $0 \leq \ell < \ell' \leq L$, we use the training set $\mathcal{T}$ to fit linear regression as described in \S\ref{subsec:methodology_linear_shortcut}, and obtain a mapping $\matl{}$. 
Next, we evaluate the quality of $\matl{}$ as well as of $\idl{}$ using the $r^2$-coefficient, uniformly averaged over all coordinates. Concretely, we compute the $r^2$-coefficient of each of the predicted representations $\matl{} (h_{i_s}^{\ell})$ and $\idl{} (h_{i_s}^{\ell})$ versus the true representations $h_{i_s}^{\ell'}$
over all $s \in \mathcal{V}$.
%as we vary $s \in \mathcal{V}$.
%for every $s \in \mathcal{V}$.



\paragraph{Results.}
Results for \gpt{} and \bert{} are presented in Figs.~\ref{fig:r2_scores} and~\ref{fig:bertmask_r2_scores}, respectively.
In both models, \mat{} consistently yields better approximations than \id{}, as it obtains higher $r^2$-scores (in blue) across the network. 
This gap between \mat{} and \id{} is especially evident in \bert{}, where \id{} completely fails to map the representations between most layers, suggesting that hidden representations are modified  substantially by every transformer block.
Overall, this highlights the shortcoming of existing practices to inspect representations in the same linear space, and the gains from using our method to approximate future layers.
% in the network.
\section{Linear Shortcut for Language Modeling}
\label{sec:prediction}

We saw that our method approximates future hidden representations substantially better than a naive propagation. 
In this section, we will show that this improvement also translates to better predictive abilities from earlier layers. Specifically, we will use our method to estimate how often intermediate representations encode the final prediction, in the context of two fundamental LM tasks; next token prediction and masked token prediction.

\paragraph{Evaluation Metrics.}
Let $h, h' \in \mathbb{R}^{d_h}$ be a final representation and a substitute final representation obtained by some mapping, and denote by $\delta (h), \delta (h') \in \mathbb{R}^{d_v}$ their corresponding output probability distributions (obtained through projection to the output vocabulary -- see details below). 
We measure the prediction quality of $h'$ with respect to $h$ using two metrics:
\begin{itemize}
[leftmargin=*,topsep=1pt,parsep=1pt]
    \item \textbf{Precision@$k$} ($\uparrow$ is better): This checks whether the token with the highest probability according to $\delta(h')$ appears in the top-$k$ tokens according to $\delta(h)$. Namely, we sort $\delta(h)$ and assign a score of $1$ if $\arg\max(\delta(h'))$ appears in the top-$k$ tokens by $\delta(h)$, and $0$ otherwise.
    
    \item \textbf{Surprisal} ($\downarrow$ is better): We measure the minus log-probability according to $\delta(h)$, of the highest-probability token according to $\delta(h')$. Intuitively, low values mean that the model sees the substitute result as probable and hence not surprising.
\end{itemize}

\noindent We report the average Precision@$k$ and Surprisal over the validation set $\mathcal{V}$.



\subsection{Next Token Prediction}
\label{subsec:next_token_prediction_task}

Auto-regressive LMs output for every position a probability distribution over the vocabulary for the next token. Specifically, the output distribution for every position $i$ is given by $\delta (h_i^L)$, where:
\begin{equation}\label{eq:output_distribution}
    \delta (h) = \texttt{softmax} ( E^\top \cdot h) \in \mathbb{R}^{d_v}
\end{equation}
For some LMs, including \gpt{}, a layer normalization $\texttt{ln\_f}$ is applied to the final layer representation before this conversion (i.e., computing $\delta (\texttt{ln\_f}(h))$ rather than $\delta (h)$).

Recall that our goal is to measure how well this distribution can be estimated from intermediate representations, i.e. estimating $\delta (h_i^L)$ from $\delta (h_i^\ell)$ where $\ell<L$. To this end, we first run examples from the validation set through the model, while extracting for each example $s$ the hidden representation of a random position $i_s$ at every layer. Next, we apply our mappings $\matlL{}$ and the $\idlL{}$ baseline to cast the hidden representations of every layer $\ell$ to final layer substitutes (see \S\ref{sec:layer_jump}). Last, for each layer, we convert its corresponding final-layer substitute to an output distribution (Eq.~\ref{eq:output_distribution}) and compute the average Precision@$k$ (for $k=1,5,10$) and Surprisal scores with respect to the final output distribution, over the validation set.

\paragraph{Results.}
Figs.~\ref{fig:pre} and~\ref{fig:surp} show the average Precision@$k$ and Surprisal scores per layer in $48$-layered \gpt{}, respectively (the plots for the other \gpt{} models are presented in \S\ref{sec:app_scale}). Across all layers, \mat{} outperforms \id{} in terms of both scores, often by a large margin (e.g. till layer $44$ the Precision@$1$ achieved by \mat{} is bigger than that of $\id{}$ by more than $0.2$). 
This shows that linear mappings enable not just better estimation of final layer representations, but also of the predictions they induce. Moreover, the relatively high Precision@$k$ scores of \mat{} in early layers ($0.62$-$0.82$ for $k=10$, $0.52$-$0.74$ for $k=5$, and $0.28$-$0.45$ for $k=1$) suggest that early representations already encode a good estimation of the final prediction. Also, the substantially lower Surprisal scores of \mat{} compared to \id{} imply that our method allows for a more representative reading into the layer-wise prediction-formation of the model than allowed through direct projection to the vocabulary.

\begin{figure}[t]
\centering
\includegraphics[scale=0.4]{figs/pre_48.pdf}
\caption{Precision@$k$ ($k = 1,5, 10$) of $\matlL{}$ and $\idlL{}$ for next token prediction in $48$-layered \gpt{}.}
\label{fig:pre}
\end{figure}

\begin{figure}[t]
\centering
\includegraphics[scale=0.35]{figs/surp_48.pdf}
\caption{Surprisal for $\matlL$ and the baseline $\idlL{}$ ($48$-layered \gpt{} next token prediction task). A 95\% confidence interval surrounds the lines.}
\label{fig:surp}
\end{figure}

\subsection{Masked Token Prediction}
\label{subsec:BERT}

We now conduct the same experiment for the task of masked language modeling, where the model predicts a probability distribution of a masked token in the input rather than the token that follows the input. Unlike next token prediction, where the output distribution is computed from representations of varying input tokens, in masked token prediction the output is always obtained from representations of the same input token (i.e. \texttt{[MASK]}).

For this experiment, we use \bert{}, on top of which we use a pretrained masked language model head $\delta$; given a token sequence $s$, a \mask{} token inside it and its final representation $h$, $\delta (h) \in \mathbb{R}^{d_v}$
 is a probability distribution over tokens giving the model's assessment
 of the likelihood of tokens to be fitting in place of the \mask{} token in $s$.


\begin{figure}[t]
\centering
\includegraphics[scale=0.4]{figs/bertmask_pre_24.pdf}
\caption{Precision@$k$ ($k = 1,5, 10$) for  $\matlL{}$ and the baseline $\idlL{}$ ($24$-layered \bert{} masked token prediction task).}
\label{fig:bertmask_pre}
\end{figure}

\begin{figure}[t]
\centering
\includegraphics[scale=0.35]{figs/bertmask_surp_24.pdf}
\caption{Surprisal for $\matlL{}$ and the baseline $\idlL{}$ ($24$-layered \bert{} masked token prediction task). A 95\% confidence interval surrounds the lines.}
\label{fig:bertmask_surp}
\end{figure}

\paragraph{Results.}
Figs.~\ref{fig:bertmask_pre} and~\ref{fig:bertmask_surp} present the average Precision@$k$ and Surprisal scores per layer in $24$-layered \bert{} (the plots for the $12$-layered \bert{} model are presented in \S\ref{sec:app_scale}), overall showing trends similar to those observed for next token prediction in \gpt{} (\S\ref{subsec:next_token_prediction_task}). This is despite the differences between the two tasks and the considerable architectural differences between \bert{} and \gpt{}.
Notably, the superiority of \mat{} over \id{} in this setting is even more prominent; 
while \mat{}'s precision is between $0.2-0.6$ in the first ten layers (Fig.~\ref{fig:bertmask_pre}), \id{}'s precision for all values of $k$ is close to zero, again strongly indicating that our method allows for better reading into early layer hidden representations. 
More generally, \mat{} improves the Precision@$1$ of \id{} by more than $17\%$ at most layers, and unveils that a substantial amount of predictions ($>25\%$ starting from layer $3$) appear already in the very first layers.
Interestingly, the (rough) divide between the first half of layers and last half of layers for $\id{}$ in Figs.~\ref{fig:bertmask_pre},~\ref{fig:bertmask_surp} seems to align with the two-hump shape of the blue region for $\mat{}$ in Fig.~\ref{fig:bertmask_r2_scores}.

\paragraph{Analysis.}
We manually compare the predictions of our mapping $\matlL{}$ with $\idlL{}$, for a $24$-layered \bert{} model.  Concretely, we select 50 random sentences from the Leipzig dataset. Next, for each layer $\ell$, we manually analyze how many of the top-$5$ tokens according to $\matlL{}$ and $\idlL{}$ fit into context. We consider a token to fit into context if it is grammatically plausible within the sentence (see Tab.~\ref{tab:manual} for concrete examples).
In the resulting $1250$ instances (i.e. $50$ sentences $\times$ $25$ representations), we observe a substantially higher plausibility rate of $85.36\%$ for \mat{} compared to $52.8\%$ for \id{}. In fact, only in less than $4.3\%$ of the instances there are more plausible tokens among the top-$5$ tokens according to \id{} than among the top-$5$ tokens according to \mat{}, further supporting the Surprisal results above.

\begin{table*}
\footnotesize
\setlength{\belowcaptionskip}{-15pt}
\begin{tabular}{p{0.3\linewidth}ccccc}
& $\texttt{id}_{4 \rightarrow 24}$ & $\texttt{mat}_{4 \rightarrow 24}$ & $\texttt{id}_{12 \rightarrow 24}$ & $\texttt{mat}_{12 \rightarrow 24}$ & $\texttt{id}_{24 \rightarrow 24}$ \\ \midrule
\multirow{5}{=}{aldridge had shoulder surgery in \mask{}.} & fellowship & \tcbox{time} & cyclist & \tcbox{2009} & \tcbox{september} \\
& employment & \tcbox{it} & emergencies & \tcbox{2008} & \tcbox{november} \\
& agreement & her & seniors & \tcbox{2010} & \tcbox{december} \\
& \#\#ostal & them & cycling & \tcbox{2006} & \tcbox{august} \\
& \#\#com & work & \tcbox{pennsylvania} & \tcbox{2007} & \tcbox{july} \\ \midrule
\multirow{5}{=}{on your next view you will be asked to \mask{} continue reading.} & \#\#com & be & be & be & \tcbox{please} \\
& accreditation & get & undergo & \tcbox{please} & \tcbox{simply} \\ 
& $	\copyright$ & go & spartans & help & \tcbox{also} \\ 
& fellowship & \tcbox{help} & seniors & \tcbox{simply} & \tcbox{again} \\ 
& summer & have & * & say & \tcbox{immediately} \\ \bottomrule
\end{tabular}
\caption{Examples of top-$5$ predictions at layers $4$, $12$ and $24$, under the mappings $\matlL{}$ and $\idlL{}$, for a $24$-layered \bert{} model. Grammatically plausible predictions (according to a human annotator) are marked in \tcbox{blue}. Note that at layer $24$ the predictions of $\matlL{}$ and $\idlL{}$ are the same (by definition).} 
\label{tab:manual}
\end{table*}

\section{Implication to Early Exiting}
\label{sec:applications}

%The fact that it is often possible to approximate
The possibility of approximating
the final prediction already in the early layers has important implications for efficiency; applying our linear mapping instead of executing transformer blocks of quadratic time complexity, could save a substantial portion of the computation. In this section, we demonstrate this in the context of early exiting.

When 
% performing transformer model inference under 
using an early exit strategy \cite{schwartz-etal-2020-right, xin-etal-2020-deebert, schuster2022confident}, one aims at deciding dynamically at which layer to stop the computation and ``read'' the prediction from the hidden representation of that layer.
More precisely, under a confidence measure paradigm, one decides to stop the computation for a position $i$ at layer $\ell$ based on a confidence criterion, that is derived from casting the hidden representation $h_i^\ell$ as a final-layer representation and converting it to an output probability distribution. Specifically, following \citet{schuster2022confident}, a decision to exit is made if the difference between the highest and the second highest probabilities is bigger than $$ 0.9 \cdot \lambda + 0.1 \cdot {\rm exp} (-4 i / N),$$
where $N$ is the average length of the input until position $i_s$ for $s \in \mathcal{V}$, and $\lambda$ is a hyper-parameter.

\begin{figure}[t]
\setlength{\belowcaptionskip}{-10pt}
\centering
\includegraphics[width=\columnwidth]{figs/ee_gpt2bert.pdf}
\caption{Precision@$1$ with early exit and ``fixed exit'', applied to the $24$-layer \gpt{} for next token prediction (left) and the $24$-layer \bert{} for masked token prediction (right). Varying the confidence parameter $\lambda$, the $x$-coordinate is the average number of layers processed before an early exit decision is reached.}
\label{fig:ee_gpt2bert}
\end{figure}

\quash{
\begin{figure}[t]
\setlength{\belowcaptionskip}{-10pt}
\centering
\includegraphics[scale=0.35]{figs/ee_pre1_24.pdf}
\caption{Precision@$1$ for the various early exit methods, and previous ``fixed exit'' methods for comparison ($24$-layer \gpt{} next token prediction task). Varying the confidence parameter $\lambda$, the $x$-coordinate is the average number of layers processed before an early exit decision is reached.}
\label{fig:ee_pre1}
\end{figure}
}

\paragraph{Experiment.}
We assess the utility of our mapping $\matlL{}$ for early exit as a plug-and-play replacement for $\idlL{}$, through which intermediate representations are cast into final-layer representations.
We use \gpt{} for the next token prediction and \bert{} for masked token prediction (both with 24 layers).
We run each of the models over the validation set examples, while varying the confidence parameter $\lambda$ and using either $\idlL{}$ or $\matlL{}$ for casting intermediate representations.
Furthermore, we compare these early exit variants to the ``fixed exit'' strategy from \S\ref{sec:prediction}, where the computation is stopped after a pre-defined number of layers rather than relying on a dynamic decision.
We evaluate each variant in terms of both prediction's accuracy, using the Precision@$1$ metric (see \S\ref{sec:prediction}), and efficiency, measured as the average number of transformer layers processed during inference.


\paragraph{Results.}
%Figs.~\ref{fig:ee_pre1} and~\ref{fig:bertmask_ee_pre1}
Fig.~\ref{fig:ee_gpt2bert}
plots the average Precision@$1$ score against the average number of layers processed, for $24$-layer \gpt{} and $24$-layer \bert{}. For both models, under an early exit strategy our mapping \mat{} again provides a substantial improvement over \id{}.
For example, aiming at $95\%$ average precision, \mat{} saves $\sim3.3$ ($13.8$\%) layers in \gpt{} compared to only $\sim1.4$ ($5.9$\%) layers by \id{}, and $\sim4.8$ ($20$\%) layers in \bert{} versus $\sim3.5$ ($14.6$\%) layers by \id{}.
These results highlight the potential gains prominent early exit methods can obtain by using our method.
Notably, in both models and for each of the mapping methods, early exit obtains better results than fixed layer exit, as expected. 

\quash{
\begin{figure}[t]
\setlength{\belowcaptionskip}{-10pt}
\centering
\includegraphics[scale=0.35]{figs/bertmask_ee_pre1_24.pdf}
\caption{Precision@$1$ for the various early exit methods, and previous ``fixed exit'' methods for comparison ($24$-layer \bert{} masked token prediction task). Varying the confidence parameter $\lambda$, the $x$-coordinate is the average number of layers processed before an early exit decision is reached.}
\label{fig:bertmask_ee_pre1}
\end{figure}
}
\section{Linear Shortcut Across Sub-Modules}
\label{sec:submodules}

% Our experiments show that
% , despite the commonly-applied simplification by interpretability works, transformer layers do not operate in the same linear space and 
% there is a major gap in approximating future representations using an identity mapping (\S\ref{sec:layer_jump}, \S\ref{sec:prediction}).
% Here, 
In this section, we investigate whether discrepancies across layers result from specific sub-modules or are a general behaviour of all sub-modules in the network.  
This is done by extending our approach to test how well particular components in transformer blocks can be linearly approximated. 


\paragraph{Method.}

Consider \gpt{} for definiteness, then:
% we have 
$$ \texttt{b}_{\ell} = \texttt{b}_{\ell}^{\texttt{ffn}} \circ \texttt{b}_{\ell}^{\texttt{attn}}$$ 
% with
\begin{equation}\label{eq:attn} \texttt{b}^{\texttt{attn}}_{\ell} (H) = \texttt{attn}_{\ell} (\texttt{ln1}_{\ell} (H)) + H,\end{equation} 
where $\texttt{attn}_{\ell}$ is
%a multi-head self-attention
a MHSA
layer and \texttt{ln1} is a layer normalization (LN), and 
$$ \texttt{b}^{\texttt{ffn}}_{\ell} (H) = \texttt{ffn}_{\ell} (\texttt{ln2}_{\ell} (H)) + H,$$  
where $\texttt{ffn}_{\ell}$ is
%a feed-forward network
an FFN
layer and $\texttt{ln2}$ is a LN.
\quash{
Given a block $\texttt{b}_\ell$ and one of its sub-modules $\texttt{ln1}_\ell, \ \texttt{attn}_\ell, \ \texttt{ln2}_\ell$, or $\texttt{ffn}_\ell$, we fit linear regression approximating the output of the sub-module given its input and then use it in order to define mappings, as we now describe.
}
Given a block $\texttt{b}_\ell$ and one of its sub-modules $\texttt{ln1}_\ell, \ \texttt{attn}_\ell, \ \texttt{ln2}_\ell$, or $\texttt{ffn}_\ell$, we fit linear regression approximating the output of the sub-module given its input, and then use it to define mappings $\matattnl{}$, $\matlnl{}$ and $\matffl{}$.
%We provide the definition of $\matattnl{}$ below, and that of the other two in App. \ref{sec:app_submodule_skip_description}.
We provide the formal definitions of these mappings in App. \ref{sec:app_submodule_skip_description}.
\iffalse
\paragraph{$\matattnl{}$.}
%Illustrating this on $\texttt{attn}_\ell$ for definiteness,
For an input $s$, let $v^\ell_{i_s}$ be the vector at position $i_s$ in the output of $\texttt{attn}_\ell (\texttt{ln1}_\ell (H^{\ell - 1}))$. We denote by $A_\ell^{\texttt{attn}} \in \mathbb{R}^{d_h \times d_h}$ the matrix numerically minimizing 
$$ A \mapsto \sum_{s \in \mathcal{T}} || A \cdot \texttt{ln1}_\ell (h^{\ell-1}_{i_s}) - v^\ell_{i_s}||^2,$$
and define an attention sub-module replacement (Eq.~\ref{eq:attn}) by $$
\texttt{b}^{\overline{\texttt{attn}}}_\ell (h) \coloneqq A_{\ell}^{\texttt{attn}} \cdot \texttt{ln1}_\ell (h) + h. $$
We then define a mapping between two layers ${\ell \rightarrow \ell'}$ by:
$$ \matattnl{} (h) \coloneqq $$
$$ \texttt{b}^{\texttt{ffn}}_{\ell'} ( \texttt{b}^{\overline{\texttt{attn}}}_{\ell'} ( \ldots (\texttt{b}^{\texttt{ffn}}_{\ell+1} ( \texttt{b}^{\overline{\texttt{attn}}}_{\ell+1} (h)))\ldots)).$$ 
Namely, when applying each $\ell''$-th block, $\ell < \ell'' \leq \ell'$, we replace its attention sub-module $\texttt{attn}_{\ell''}$ by its linear approximation.
%In an analogous way, we consider the mappings $\matffl{}$ and $\matlnl{}$, where in the latter we perform the linear shortcut both for \texttt{ln1} and for \texttt{ln2} (see~\S\ref{sec:app_submodule_skip_description} for precise descriptions).
Importantly, unlike the original attention module, the approximation $\texttt{b}^{\overline{\texttt{attn}}}_\ell$ operates on each position independently, and therefore applying $\matattnl{}$ disables any contextualization between the layers $\ell$ and $\ell'$. Note that this is not the case for $\matffl{}$ and $\matlnl{}$, which retain the self-attention sub-modules and operate contextually.
\fi

\paragraph{Evaluation.}


We analyze the $24$-layered \gpt{}, and proceed completely analogously to \S\ref{subsec:next_token_prediction_task}, evaluating the Precision@$1$ and Surprisal metrics for the mappings $\matattnlL{}$, $\matfflL{}$ and $\matlnlL{}$.

\begin{figure}[t]
\setlength{\belowcaptionskip}{-0pt}
\centering
%\includegraphics[scale=0.2]
\includegraphics[width=\columnwidth]{figs/parts_presurp_24.pdf}
\caption{Precision@$1$ and Surprisal for the various sub-module linear mappings, and $\matlL{}$ for comparison ($24$-layer \gpt{} next token prediction task). A 95\% confidence interval surrounds the Surprisal lines.}
\label{fig:parts_presurp}
\end{figure}

\quash{
\begin{figure}[t]
\centering
\includegraphics[scale=0.4]{figs/parts_pre1_24.pdf}
\caption{Precision@$1$ for the various sub-module linear shortcut mappings, and the mapping $\matlL{}$ for comparison (\gpt{} next token prediction task).}
\label{fig:parts_pre1}
\end{figure}

\begin{figure}[t]
\centering
\includegraphics[scale=0.35]{figs/parts_surp_24.pdf}
\caption{Surprisal for the various sub-module linear shortcut mappings, and the mapping $\matlL{}$ for comparison (\gpt{} next token prediction task). A 95\% confidence interval surrounds the lines.}
\label{fig:parts_surp}
\end{figure}
}

\paragraph{Results.}
Fig.~\ref{fig:parts_presurp} shows the average Precision@$1$ and Surprisal scores per layer.
From a certain layer (\textasciitilde$7$), all sub-module mappings achieve better results than the full-block mapping $\matlL{}$. Thus, it is not just the cumulative effect of all the sub-modules in the transformer block that is amenable to linear approximation, but also individual sub-modules can be linearly approximated. 
Furthermore, the linear approximation of attention sub-modules is less harmful than that of the FFN or LN sub-modules. 
% Hypothetically, 
A possible reason is that the linear replacement of FFN or LN ``erodes'' the self-attention computation after a few layers. 
Moreover, the good performance of $\matattnlL{}$ suggests that contextualization often exhausts itself in early layers; speculatively, it is only in more delicate cases that the self-attention of late layers adds important information. Last, remark the sharp ascent of the scores for layer normalization in layers $5$-$8$, for which we do not currently see a particular reason. To conclude, we see that the possibility of linear approximation permeates
%the various
transformer components.


\section{Related Work}

Recently, there was a lot of interest in utilizing intermediate representations in transformer-based LMs, both for interpretability and for efficiency.

In the direction of interpretability, one seeks to understand the prediction construction process of the model \cite{tenney-etal-2019-bert, voita-etal-2019-bottom}.

More recent works use mechanistic interpretability and view the inference pass as a residual stream of information \cite{dar2022analyzing,geva-etal-2022-transformer}. Additionally, there are works on probing, attempting to understand what features are stored in the hidden representations \cite{adi2017finegrained, conneau-etal-2018-cram,liu-etal-2019-linguistic}. Our work is different in that it attempts to convert intermediate representations into a final-layer form, which is interpretable by design.

In the direction of efficiency, there is the thread of work on early exit, where computation is cut at a dynamically-decided earlier stage \cite{schwartz-etal-2020-right,xin-etal-2020-deebert,schuster2022confident}. Other works utilize a fixed early stage network to parallelize inference \citep{leviathan2022fast, chen2023accelerating}. However, intermediate representations are directly propagated in these works, which we show is substantially worse than our approach. Moreover, our method requires training considerably less parameters than methods such as \citet{schuster-etal-2021-consistent}, that learn a different output softmax for each intermediate layer.  

More broadly, skipping transformer layers and analyzing the linearity properties of transformer components have been discussed in prior works \cite{Zhao2021of,mickus-etal-2022-dissect,wang-etal-2022-skipbert,lamparth2023analyzing}.


\section{Conclusion and Future Work}

We present a simple and effective method for enhancing utilization of hidden representations in transformer-based LMs, that uses 
pre-fitted context-free and token-uniform linear mappings.
Through a series of experiments on different data sources, model architectures and scales, we show that our method consistently outperforms the prevalent practice of interpreting representations in the final-layer space of the model, yielding better approximations of succeeding representations and the predictions they induce, thus allowing a more faithful interpretation of the model's prediction-formation.
We demonstrate the practicality of our method for improving computation efficiency, saving a substantial amount of compute on top of prominent early exiting approaches. 
Also, by extending our method to sub-modules, 
% more specifically the attention sub-modules, 
we observe that replacing a part of the transformer inference by a non-contextual linear computation often results in a small deterioration of the prediction.
This opens new research directions for improving model efficiency,
% and parallelizability.
% including breaking the computation into several parallelizable tasks.
including breaking the computation into parallel tasks.

\section*{Limitations}

Although we see in this work that there is more linear structure to transformer inference than could be explained solely by the residual connection, we do not elucidate a reason for that. We also do not try to formulate formal criteria according to which to judge, in principle, the quality of ways of short-cutting transformer inference in-between layers. In addition, our experiments cover only English data.


%\section*{Ethics Statement}
%Scientific work published at ACL 2023 must comply with the ACL Ethics Policy.\footnote{\url{https://www.aclweb.org/portal/content/acl-code-ethics}} We encourage all authors to include an explicit ethics statement on the broader impact of the work, or other ethical considerations after the conclusion but before the references. The ethics statement will not count toward the page limit (8 pages for long, 4 pages for short papers).

\section*{Acknowledgements}

We thank Tal Schuster for constructive comments.

% Entries for the entire Anthology, followed by custom entries
\bibliography{anthology,custom}
\bibliographystyle{acl_natbib}

\appendix

\section{Descriptions of $\matattn{}$, $\matff{}$ and $\matln{}$}
\label{sec:app_submodule_skip_description}

Here we detail the definitions of the mappings $\matattnl{}$, $\matffl{}$ and $\matlnl{}$ utilized in \S\ref{sec:submodules}.

\paragraph{Description of $\matattnl{}$.}
%Illustrating this on $\texttt{attn}_\ell$ for definiteness,
For an input $s$, let $v^\ell_{i_s}$ be the vector at position $i_s$ in the output of $\texttt{attn}_\ell (\texttt{ln1}_\ell (H^{\ell - 1}))$. We denote by $A_\ell^{\texttt{attn}} \in \mathbb{R}^{d_h \times d_h}$ the matrix numerically minimizing 
$$ A \mapsto \sum_{s \in \mathcal{T}} || A \cdot \texttt{ln1}_\ell (h^{\ell-1}_{i_s}) - v^\ell_{i_s}||^2,$$
and define an attention sub-module replacement (Eq.~\ref{eq:attn}) by $$
\texttt{b}^{\overline{\texttt{attn}}}_\ell (h) \coloneqq A_{\ell}^{\texttt{attn}} \cdot \texttt{ln1}_\ell (h) + h. $$
We then define a mapping between two layers ${\ell \rightarrow \ell'}$ by:
$$ \matattnl{} (h) \coloneqq $$
$$ \texttt{b}^{\texttt{ffn}}_{\ell'} ( \texttt{b}^{\overline{\texttt{attn}}}_{\ell'} ( \ldots (\texttt{b}^{\texttt{ffn}}_{\ell+1} ( \texttt{b}^{\overline{\texttt{attn}}}_{\ell+1} (h)))\ldots)).$$ 
Namely, when applying each $\ell''$-th block, $\ell < \ell'' \leq \ell'$, we replace its attention sub-module $\texttt{attn}_{\ell''}$ by its linear approximation.
%In an analogous way, we consider the mappings $\matffl{}$ and $\matlnl{}$, where in the latter we perform the linear shortcut both for \texttt{ln1} and for \texttt{ln2} (see~\S\ref{sec:app_submodule_skip_description} for precise descriptions).
Importantly, unlike the original attention module, the approximation $\texttt{b}^{\overline{\texttt{attn}}}_\ell$ operates on each position independently, and therefore applying $\matattnl{}$ disables any contextualization between the layers $\ell$ and $\ell'$. Note that this is not the case for $\matffl{}$ and $\matlnl{}$, which retain the self-attention sub-modules and operate contextually.

\paragraph{Description of $\matffl{}$.}
Let $v^\ell_{i_s}$ be the vector at position $i_s$ in the output of $\texttt{ln2}_{\ell} (\texttt{b}_\ell^{\texttt{attn}} (H^{\ell - 1}))$, for a given input $s$. We denote by $A_\ell^{\texttt{ffn}} \in \mathbb{R}^{d_h \times d_h}$ the matrix numerically minimizing 
$$ A \mapsto \sum_{s \in \mathcal{T}} || A \cdot v^{\ell}_{i_s} - \texttt{ffn}_{\ell} (v^\ell_{i_s})||^2,$$
and define a replacement of the feed-forward sub-module $\texttt{b}_{\ell}^{\texttt{ffn}}$ by $$ \texttt{b}^{\overline{\texttt{ffn}}}_\ell (H) \coloneqq A_{\ell}^{\texttt{ffn}} \cdot \texttt{ln2}_\ell (H) + H.$$
We then define a mapping between two layers ${\ell \rightarrow \ell'}$ by:
$$ \matffl{} (H) \coloneqq $$
$$ \texttt{b}^{\overline{\texttt{ffn}}}_{\ell'} ( \texttt{b}^{\texttt{attn}}_{\ell'} ( \ldots (\texttt{b}^{\overline{\texttt{ffn}}}_{\ell+1} ( \texttt{b}^{\texttt{attn}}_{\ell+1} (H))\ldots)).$$

\paragraph{Description of $\matlnl{}$.}
Let $v^\ell_{i_s}$ be the vector at position $i_s$ in the output of $\texttt{b}^{\texttt{attn}}_{\ell} (H^{\ell - 1})$, for a given input $s$. We denote by $A_\ell^{\texttt{ln1}} \in \mathbb{R}^{d_h \times d_h}$ the matrix numerically minimizing 
$$ A \mapsto \sum_{s \in \mathcal{T}} || A \cdot h^{\ell}_{i_s} - \texttt{ln1}_{\ell} (h^\ell_{i_s})||^2$$ and we denote by $A_\ell^{\texttt{ln2}} \in \mathbb{R}^{d_h \times d_h}$ the matrix numerically minimizing $$ A \mapsto \sum_{s \in \mathcal{T}} || A \cdot v^{\ell}_{i_s} - \texttt{ln2}_{\ell} (v^\ell_{i_s})||^2.$$ We define a replacement of the block $\texttt{b}^{\texttt{attn}}_{\ell}$ by \begin{equation} \texttt{b}^{\overline{\texttt{ln1}}}_\ell (H) \coloneqq \texttt{attn}_{\ell} (A_{\ell}^{\texttt{ln1}} \cdot H) + H\end{equation} and we define a replacement of the block $\texttt{b}^{\texttt{ffn}}_{\ell}$ by \begin{equation} \texttt{b}^{\overline{\texttt{ln2}}}_\ell (H) \coloneqq \texttt{ffn}_{\ell} (A_{\ell}^{\texttt{ln2}} \cdot H) + H.\end{equation}
We then define a mapping between two layers ${\ell \rightarrow \ell'}$ by:
$$ \matlnl{} (H) \coloneqq $$
$$ \texttt{b}^{\overline{\texttt{ln2}}}_{\ell'} ( \texttt{b}^{\overline{\texttt{ln1}}}_{\ell'} ( \ldots (\texttt{b}^{\overline{\texttt{ln2}}}_{\ell+1} ( \texttt{b}^{\overline{\texttt{ln1}}}_{\ell+1} (H))\ldots)).$$


\end{document}

%%%%%%%%% REFERENCES
{\small
\bibliographystyle{ieee_fullname}
\bibliography{main}

}

\end{document}
