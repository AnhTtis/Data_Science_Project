\documentclass[10pt,twocolumn,letterpaper]{article}

\usepackage{iccv}
\usepackage{times}
\usepackage{epsfig}
\usepackage{graphicx}
\usepackage{amsmath}
\usepackage{amssymb}

% Include other packages here, before hyperref.
\usepackage{xcolor,colortbl}
% \usepackage[table,x11names]{xcolor}
\usepackage{adjustbox}
\usepackage{multirow}
% \usepackage{caption}
\usepackage[font=small,labelfont=bf]{caption}
\usepackage{subcaption}
\usepackage{dirtytalk}
% Support for easy cross-referencing

% If you comment hyperref and then uncomment it, you should delete
% egpaper.aux before re-running latex.  (Or just hit 'q' on the first latex
% run, let it finish, and you should be clear).
\usepackage[pagebackref=true,breaklinks=true,letterpaper=true,colorlinks,bookmarks=false]{hyperref}
\usepackage[capitalize]{cleveref}
\usepackage{xspace}
\usepackage{mwe}

\def\eg{\emph{e.g}\onedot} \def\Eg{\emph{E.g}\onedot}
\def\@fnsymbol#1{\ensuremath{\ifcase#1\or *\or \dagger\or \ddagger\or
   \mathsection\or \mathparagraph\or \|\or **\or \dagger\dagger
   \or \ddagger\ddagger \else\@ctrerr\fi}}

\newcommand{\rulesep}{\unskip\ \vrule\ }
\newcommand{\ssymbol}[1]{^{\@fnsymbol{#1}}}
% If you comment hyperref and then uncomment it, you should delete
% egpaper.aux before re-running latex.  (Or just hit 'q' on the first latex
% run, let it finish, and you should be clear).
% \usepackage[breaklinks=true,bookmarks=false]{hyperref}

\iccvfinalcopy % *** Uncomment this line for the final submission

% \def\iccvPaperID{****} % *** Enter the ICCV Paper ID here
\def\httilde{\mbox{\tt\raisebox{-.5ex}{\symbol{126}}}}

% Pages are numbered in submission mode, and unnumbered in camera-ready
\ificcvfinal\pagestyle{empty}\fi

\begin{document}

%%%%%%%%% TITLE
\title{LD-ZNet: A Latent Diffusion Approach for Text-Based Image Segmentation}

\author{Koutilya PNVR$\ssymbol{2}$
\and
Bharat Singh$\ssymbol{3}$
\and
Pallabi Ghosh$\ssymbol{4}$
\and
Behjat Siddiquie$\ssymbol{4}$
\and
David Jacobs$\ssymbol{2}$
\and
University of Maryland College Park$\ssymbol{2}\qquad$\hspace{0.3cm}Vchar.ai$\ssymbol{3}\qquad\qquad$\hspace{1.6cm}Amazon$\ssymbol{4}\qquad\qquad$\hspace{0.2cm}\\
{\tt\small $\{$koutilya, djacobs$\}$@umiacs.umd.edu$\qquad$bharat@vchar.ai$\qquad\{$pallabi, behjats$\}$@amazon.com}
}
\maketitle
% Remove page # from the first page of camera-ready.
% \ificcvfinal\thispagestyle{empty}\fi
% \thispagestyle{empty}

%%%%%%%%% ABSTRACT
\begin{abstract}
   We present a technique for segmenting real and AI-generated images using latent diffusion models (LDMs) trained on internet-scale datasets. First, we show that the latent space of LDMs (z-space) is a better input representation compared to other feature representations like RGB images or CLIP encodings for text-based image segmentation. By training the segmentation models on the latent z-space, which creates a compressed representation across several domains like different forms of art, cartoons, illustrations, and photographs, we are also able to bridge the domain gap between real and AI-generated images. We show that the internal features of LDMs contain rich semantic information and present a technique in the form of LD-ZNet to further boost the performance of text-based segmentation. Overall, we show up to 6\% improvement over standard baselines for text-to-image segmentation on natural images. For AI-generated imagery, we show close to 20\% improvement compared to state-of-the-art techniques.
\end{abstract}

%------------------------------------------------------------------------
% Introduction
\vspace{-10pt}
%%%%%%%%% BODY TEXT

\section{Introduction}
\label{section:introduction}
%% 1. why should someone care?

%The advent of advanced interactive computer vision systems~\cite{hololens} and recent progress in vision-language and multi-modal models~\cite{} opens doors for such next generation of assistive agents. 
% We envision that the future assistive agents would build up on these visual and language reasoning capabilities of today and empower users to achieve goals in their everyday lives. In particular, such agents would be able to reason about \emph{unseen} human goals... 
% We posit that such agents would require the ability to understand user goals described in natural language at high-level i.e., without complete details about as well as unseen user goals. 

%Recent progress in augmented reality systems~\cite{hololens, magicleap}, as well as vision-language and multi-modal models~\cite{}, opens doors for the next generation of assistive agents. 
Inspired by recent progress in visual systems~\cite{MagicLeap, ungureanu2020hololens}, we consider an assistive egocentric agent capable of reasoning about daily activities. When invoked via natural language commands, for e.g., while baking a cake, the agent understands the steps involved in baking, tracks progress through the various stages of the task, detects and proactively prevents mistakes by making suggestions. Such an agent would empower users to learn new skills and accomplish tasks efficiently.
% One could envision invoking such an agent merely through natural language descriptions of tasks similar to how present day assistants such as Alexa, Siri etc.~\cite{voice_assistants} are invoked. 
%We envision such agents to empower users in daily life by  invoking them naturally through 

%% 2. Why is it challenging? 
%While recent progress in vision-language and multi-modal models~\cite{} opens doors for such next generation of assistive agents, various challenges remain in making such agents a reality. 
%To make such agents a reality, 

Developing such an egocentric agent capable of tracking and verifying everyday tasks based on their natural language specification is challenging for multiple reasons. First, such an agent must reason about various ways of doing a \emph{multi-step} task specified in natural language. This entails decomposing the task into relevant actions, state changes, object interactions as well as any necessary causal and temporal relationships between these entities. Secondly, the agent must ground these entities in egocentric observations to track progress and detect mistakes. Lastly, to truly be useful, such an agent must support tracking and verification for a combination of tasks and, ideally, even unseen tasks. These three challenges -- causal and temporal reasoning about task structure from natural language, visual grounding of sub-tasks, and compositional generalization -- form the core goals of our work.

% %% 3. What are we doing? What is our approach?
% \aks{I think this is a matter of preference, but I personally don't like related work in intro. I would make this paragraph be about EgoTV and NSG. Starting with something like - "To this end, we propose...", ie, your next paragraph.}
% \nk{+1, we should move parts of this para to lit review and delete the rest.}
% Recent research on language modeling enables decomposing tasks into multiple steps from natural language descriptions~\cite{llm_zero_shot_planning,proscript}. However, such \emph{task decompositions} cannot directly be leveraged for task tracking in egocentric agents because of lack of grounding into the visual observations or context. In parallel, the computer vision community has advanced action recognition~\cite{}, object detection and tracking~\cite{}, hand object interaction and object state change detection~\cite{ego_4d,change_it,}, step classification in procedural tasks~\cite{}, and even vision language reasoning~\cite{nsvqa,nscl,star_situated_reasoning,clevrer}, which may help with the grounding challenge. However, majority of current research on identifying actions, objects, steps, or state changes does not account for the overall task structure. Likewise, predominant research on vision language understanding~\cite{} and multi-modal grounding~\cite{} does not consider the temporal and causal constraints that emerge in task tracking and verification. We therefore focus on the order-aware visual grounding problem in our work, with an eye towards compositional generalization to scale usability of these agents. In particular, we aim to achieve visual grounding of the actions and objects corresponding to each step or sub-task obtained from the task description decomposition in an order-aware manner.

%% 4. What are our results/contributions?
As our first contribution, we propose a benchmark -- \emph{\textbf{Ego}centric \textbf{T}ask \textbf{V}erification} (\etv \inlineimg{figures/TV}) -- and a corresponding dataset in the AI2-THOR~\cite{ai2thor} simulator. % \emoji{tv}
Given a natural language (NL) task description and a corresponding egocentric video of an agent, the goal of \etv is to verify whether the task was successfully completed in the video or not.
\etv contains multi-step tasks with \emph{ordering} constraints on the steps and \emph{abstracted} NL task descriptions with omitted low-level task details inspired by the needs of real-world assistants. We also provide splits of the dataset focused on different generalization aspects, e.g., unseen visual contexts, compositions of steps, and tasks (see Figure~\ref{figure:dataset}).
% Next, we create splits of the dataset focused on different aspects of generalization, ranging from generalization to unseen visual context to unseen compositions of steps and tasks. Figure~\ref{figure:dataset} shows an example task and overview of generalization splits from \etv. Succeeding at \etv tasks requires decomposing tasks into partially-ordered steps from the NL description and order-aware visual grounding of these steps into the video. 

Our second contribution is a novel approach for order-aware visual grounding~--~\emph{\textbf{N}euro-\textbf{S}ymbolic \textbf{G}rounding} (NSG), capable of compositional reasoning and generalizing to unseen tasks owing to its ability to leverage abstract NL descriptions and compositional structure of tasks (task decomposition, ordering).~In contrast, state-of-the-art vision-language models~\cite{coca,clip,videoclip,clip_hitchiker} struggle to ground NL descriptions in egocentric videos, and do not generalize to unseen tasks.~NSG outperforms these models by~$\mathbf{33.8}\%$~on compositional generalization and~$\mathbf{32.8}\%$~on abstractly described task verification. Finally, to evaluate \nsg on real-world data, we instantiate \etv on the CrossTask~\cite{cross_task} instructional video dataset. %Specifically, we synthetically create videos with mistakes in CrossTask. 
We find that it also outperforms state-of-the-art models at task verification on CrossTask. We hope that the \etv~benchmark and dataset will enable future research on egocentric agents capable of aiding in everyday tasks.

% We experiment with many for the \etv tasks. We find that while these models generalize well to unseen visual context, they struggle to perform grounding from abstracted task descriptions and to generalize to new compositions of tasks. To deal with these challenges, we take inspiration from recent research on and develop . ~\rd{unclear why neurosymbolic models would do well on abstraction.} 

% To summarize, our main contributions are:~1)~\etv: a benchmark and synthetic dataset to systematically study egocentric task verification.
% 2)~\nsg: a novel neuro-symbolic approach to enable the core reasoning capability for \etv -- order-aware visual grounding. We demonstrate \nsg's capability on our synthetic \etv dataset as well as a real-world dataset derived from CrossTask. We will release both of these datasets and our models for future research on egocentric task tracking and verification. 


% Assistive agents require the ability to track actions and state changes from an egocentric perspective for effective assistance in day-to-day tasks. For example, an agent helping a user prepare a recipe would need to both generate the steps of the recipe (\textit{plan generation}) and track the user's actions to ensure the plan is executed correctly (\textit{plan verification}). We formulate this as a Video Entailment task~\cite{violin_dataset,9710490} \rd{should we call our task video-based goal entailment?}, wherein, given an egocentric video of an agent (or human) performing a task (\textit{premise}) and a NL task description (\textit{hypothesis}), the objective is to learn a model to track whether the given task was successfully executed in the video. 
% An ideal model should also be able to seamlessly generalize to novel compositions (of actions and objects) unseen during training. \rd{add a line about what we mean by abstraction and why is it important.} To this end, we generate a novel Vision-Language dataset on the AI2-THOR simulator~\cite{ai2thor} to study compositional and abstraction-based generalization. Our dataset provides effective evaluation measures in a controlled setting, while closely reflecting the diversity of real-world events. We implement and train a variety of end-to-end models based on existing state-of-the-art approaches. We empirically demonstrate that neural models suffer from overfitting and cannot effectively generalize to novel compositions of actions, objects, and scenes. 
% To address this problem, we propose an end-to-end Neuro-Symbolic (NeSy) framework that performs plan generation and verification. At the heart of our approach is the hypothesis that symbolic reasoning models are good at generalization and capturing compositional substructure, while neural models are good at learning representations from sensory data~\cite{10.5555/3326943.3327039,nscl,clevrer}. \rd{summarize contributions in a bulleted list.} \rd{also add a line about the main result e.g., x\% improvement as compared to end-to-end models}. 

% \rd{we also evaluate NeSy with real-world data: add briefly about CrossTask experiments.}

% % \fbox{\begin{minipage}{\linewidth}
% % \textbf{Problem Statement}

% % Given:
% % (i) Premise: Egocentric video of an agent performing a task.
% % (ii) Hypothesis: NL description of the task.

% % Learn: A model to track whether the premise entails the hypothesis. The output of the model is True if the given task is executed successfully in the video.
% % \end{minipage}}

% \textbf{Contributions:} 
% \begin{itemize}
%     \item We generate a benchmark video-language dataset to study compositional and abstraction-based generalization.
%     \item We evaluate the performance of a variety of state-of-the-art models and show that these (baseline) models cannot effectively generalize to novel compositions of actions.
%     \item We propose a novel end-to-end NeSy approach that significantly outperforms the baselines on some compositional generalization splits while performing on par with them on the rest.
%     \item We also evaluate our NeSy approach with real-world data showing similar performance improvements.
% \end{itemize}


%------------------------------------------------------------------------
% RELATED WORKS
\section{Related work}
\label{relatedworks}
\subsection{Text-based image segmentation}
Text-based image segmentation
is the general task of segmenting specific regions in an image, based on a text prompt. This is different from the referring expression segmentation (RES) task, which aims to extract instance-level segmentation of different objects through distinctive referring expressions. While RES helps applications in robotics that require localization of a {\em single} object in an image, text-based segmentation benefits image editing applications by being able to also segment 1) ``stuff" categories (clouds/ocean/beach \etc) and 2) multiple instances of an object category applicable to the text prompt. However, both these tasks have some shared literature in terms of approaches. Preliminary works \cite{hu2016segmentation, liu2017recurrent, shi2018key, li2018referring, ye2019cross} focused on the multi-modal feature fusion between the language and visual representations obtained from recurrent networks (such as LSTM) and CNNs respectively. The subsequent set of works \cite{margffoy2018dynamic, yu2018mattnet, wang2022cris, yang2022lavt} included variations of multi-modal training, attention and cross-attention networks etc. Recently, \cite{wang2022cris, luddecke2022image} used CLIP \cite{radford2021learning} to extract visual linguistic features of the image and the reference text separately. These features were then combined using a transformer based decoder to predict a binary mask. Alternately, \cite{kamath2021mdetr, zhang2022glipv2}, proposed vision-language pretraining on other text-based visual recognition tasks (object detection and phrase grounding) and later finetuned for the segmentation task. \camready{The concurrent works segment-anything (SAM) \cite{kirillov2023segment} and segment-everything-everywhere-all-at-once (SEEM) \cite{zou2023segment} allow interactive segmentation via point clicks, bounding boxes and text inputs \etc. demonstrating good zero-shot performance.} Different from all these works, we show the significance of using the latent space and the internal features from a pretrained latent diffusion model \cite{rombach2022high} for improving the more generic text-based image segmentation task.

%This task is different from the referring expression segmentation task, which is typically useful for robot localization, where the goal is to spatially localize a {\em unique object} with a distinctive referring expression (that can moreover contain complex positional references requiring dedicated training of the language encoder). This is necessary because for image editing applications, the algorithm should be able to localize ``stuff" categories like clouds/ocean/beach etc. and also be able to generate masks for multiple objects, if they are applicable to the text prompt.
\subsection{Text-to-Image synthesis}
Text-to-Image synthesis has initially been explored using GANs \cite{Xu_2018_CVPR, Zhu_2019_CVPR, tao2022df, zhang2021cross, ye2021improving, zhou2022towards} on publicly available image captioning datasets. Another line of work is by using autoregressive models \cite{ramesh2021zero, NEURIPS2021_a4d92e2c, gafni2022make} via a two stage approach. The first stage is a vector quantized autoencoder such as a VQVAE \cite{van2017neural, razavi2019generating} or a VQGAN \cite{esser2021taming} with an image reconstruction objective to convert an image into a shorter sequence of discrete tokens. This low dimensional latent space enables the training of compute intensive autoregressive models even for high resolution text-to-image synthesis. With the recent advancements in Diffusion Models (DM) \cite{nichol2021improved,NEURIPS2021_49ad23d1}, both in unconditional and class conditional settings, they have started gaining more traction compared to GANs. Their success in the text-to-image tasks \cite{saharia2022photorealistic,ramesh2022hierarchical} made them even more popular. However, the prior diffusion models worked in the high-dimensional image space that made training and inference computationally intensive. Subsequently, latent space representations \cite{nichol2021glide, gu2022vector, tang2022improved, rombach2022high} were proposed for high resolution text-to-image synthesis to reduce the heavy compute demands. More specifically, the latent diffusion model (LDM) \cite{rombach2022high} mitigates this problem by relying on a perceptually compressed latent space produced by a powerful autoencoder from the first stage. Moreover, they employ a convolutional backed UNet \cite{UNet} as the denoising architecture, allowing for different sized latent spaces as input. %
%
Recently this architecture is trained on large scale text-image data \cite{schuhmann2022laionb} from the internet and released as Stable-diffusion\footnote{\href{https://github.com/CompVis/stable-diffusion}{https://github.com/CompVis/stable-diffusion}}, which exhibited photo-realistic image generations. Subsequently, several language guided image editing applications such as inpainting \cite{couairon2022diffedit, Lugmayr_2022_CVPR, xie2022smartbrush}, text-guided image editing \cite{chen2018language, brooks2022instructpix2pix} became more popular and the usage for text-based image segmentation has surged, especially for AI generated images. We propose a solution for text-based image segmentation by leveraging the features which are already present as part of the synthesis process.

\begin{figure}[t]
    \centering
        \centering
        \includegraphics[width=\linewidth]{Images/ZSEG-VQGAN}
        \caption{Reconstructions from the first stage of the LDM. Given an input image, the latent representation $z$ generated by the encoder, can be used to reconstruct images that are perceptually indistinguishable from the inputs. The high quality of these reconstructions suggests that the latent representation $z$, preserves most of the semantic information present in the input images.}
        \vspace{-1em}
        \label{fig:vqgan}
\end{figure}

\subsection{Semantics in generative models}
Semantics in generative models
such as GANs have been studied for binary segmentation~\cite{voynov2021object,melas2021finding} as well as multi-class segmentation~\cite{zhang2021datasetgan, tritrong2021repurposing, pakhomov2021segmentation} where the intermediate features have been shown to contain semantic information for these tasks. Moreover, \cite{semanticGAN} highlighted the practical advantages of these representations, such as out-of-distribution robustness. However, prior generative models (GANs \etc) as representation learners have received less attention compared to alternative unsupervised methods \cite{pmlr-v119-chen20j}, because of the training difficulties on complex, diverse and large scale datasets. Diffusion models \cite{nichol2021improved}, on the other hand are another class of powerful generative models that recently outperformed GANs on image synthesis \cite{NEURIPS2021_49ad23d1} and are able to train on large datasets such as Imagenet \cite{deng2009imagenet} or LAION \cite{schuhmann2022laionb}. In \cite{baranchuk2021label}, the authors demonstrated that the internal features of a pre-trained diffusion model were effective at the semantic segmentation task. However, this type of analysis \cite{zhang2021datasetgan, baranchuk2021label} has mostly been done in limited settings like few shot learning \cite{fei2006one} or limited domains like faces \cite{karras2019style}, horses \cite{yu15lsun} or cars \cite{yu15lsun}. Different from these works, we analyze the visual-linguistic semantic information present in the internal features of a text-to-image LDM \cite{rombach2022high} for text based image segmentation, which is an open world visual recognition task. %
%
Furthermore, we leverage these LDM features and show performance improvements when training with full datasets instead of few-shot settings.


%------------------------------------------------------------------------
% Method
\section{Method}
Our method, {\moniker}, extends the volume rendering equation to accurately reconstruct the geometry and appearance robust to hazy conditions.
Our key idea is to introduce a series of important biases in the network architecture along with regularizers in the loss function that together underpin physically based scattering phenomena.

\subsection{Preliminary on Neural Radiance Fields}\label{sec:nerf}
Neural Radiance Fields (NeRFs)~\cite{mildenhall2020nerf} map a 3D sample point \(\p\) into a color $\mathbf{c}$ and volume density $\sigma$.
Considering only emission from classic volume rendering~\cite{kajiya1984ray,tagliasacchi2022volume}, the expected color ${C}(\r)$ of a camera ray $\r(t)=\mathbf{o} + t\mathbf{d}$ with the near and far boundary $t_n$ and $t_f$ can be written as
\begin{gather}
	{C}(\r, \mathbf{d})=\int_{t_n}^{t_f}T(t)\sigma(\r(t))c(\r(t), \mathbf{d}) \ dt \;\textrm{with} \label{eq:nerf}\\
    T(t)=\mathrm{exp}\left( - \int_{t_n}^{t}\sigma(\r(t')) \ dt'\right),
	\label{eq:occlusion}
\end{gather}
where \(T(t)\) is the accumulated transmittance between the ray section \(t_{n}\) to \(t \).
The predicted pixel value is then compared to the ground truth $\widehat{C}(\r,\d)$ for optimization.

\subsection{3D Haze Formation}\label{sec:rte_haze}
To address the 3D dehazing problem, we propose an alternative rendering equation to the image formation model.
We start from the radiative transfer equation (RTE)~\cite{chandrasekhar2013radiative,van1999multiple}, which describes the behavior of light in a medium that absorbs, scatters and emits radiation.
Assuming, a ray \(\r\left( t \right) = \mathbf{o} + t\d\) hits a surface point at \(\r\left( t_{0} \right)\), the incident radiance at the near image plane \(t_{n}\) can be divided into three parts~\cite{pharr2016physically}:
{\small
\begin{align}
C(\r, \d) &= C_{\textrm{emission}}(\r) + C_{\textrm{surface}}(\r) + C_{\textrm{in-scattering}}(\r)\nonumber\\
C_{\textrm{emission}}(\r, \d) &=
\int_{t_{n}}^{t_{0}}\epsilon\left(\r\left( t\right),\d\right)T_{\sigma_{t}}\left( t\right)dt\nonumber\\
C_{\textrm{surface}}(\r, \d) & =C_e\left(\r\left( t_{0} \right),\d\right)T_{\sigma_{t}}\left( t_{0}\right)\nonumber\\
C_{\textrm{in-scattering}}(\r, \d) &=
\int_{t_{n}}^{t_{0}}c_{\textrm{s}}\left( \r\left( t \right), \d \right)\sigma_{s}\left(\r\left( t \right)\right)T_{\sigma_{t}}\left( t \right)dt,\nonumber
\end{align}
}
where \(\epsilon\) is the emission, \(C_{e}\) is the outgoing radiance at the surface intersection, \(c_{\textrm{s}}\left(\r\left( t \right), \d \right)\) is the in-scattered light and \(\sigma_{s}\) is the scattering coefficient.
In particular, the transmittance here is computed from the attenuation coefficient \(\sigma_{t}\), \ie,
\(T_{\sigma_{t}}\left( t\right)=\exp\left( -\int_{t_{n}}^{t}\sigma_{t}(t')dt' \right)\),
where \(\sigma_{t}=\sigma_{a} + \sigma_{s}\) including the absorption and out-scattering effect.
For common haze formation, the participating particles are considered non-luminous~\cite{narasimhan2003contrast}, therefore we can drop the emission part, which leads to
{
\small
\begin{align}
\begin{split}
C(\r,\d)= {} & C_e(\r\left( t_{0} \right),\d)T_{\sigma_{t}}\left( t_{0} \right)+\\
&\int_{t_{n}}^{t_{0}}c_{\textrm{s}}\left( \r\left( t \right), \d \right)\sigma_{s}\left(\r\left( t \right)\right)T_{\sigma_{t}}\left( t \right)dt.
\end{split}
\label{eq:RTE_Haze}
\end{align}
}

Following NeRF~\cite{mildenhall2020nerf}, we represent the surface as a continuous density field with emission \(\epsilon\left(\r\left(t\right), \d\right)\coloneqq c\left(\r\left( t \right),\d\right)\sigma\left(\r\left( t \right)\right)\).
Meanwhile, the absorption part in the attenuation \(\sigma_{t}\) can be interpreted as the surface density \(\sigma\), since the volume density $\sigma$ is equal to absorption coefficient $\sigma_{a}$ in that they both determine the probability of a photon or a ray terminating at a given location.
As a result, we can write the rendering equation as
{\small
\begin{align}\begin{split}
    C(\r,\d)=&
    \underbrace{\int_{t_{n}}^{t_{0}}c(\r(t),\d)\sigma(t)T_{\sigma+\sigma_{s}}\left( t \right)dt}_{C_{\textrm{Surface}}} +\\
    &\underbrace{\int_{t_{n}}^{t_{0}}c_{s}(\r(t))\sigma_{s}(t)T_{\sigma+\sigma_{s}}\left( t \right)dt}_{C_{\textrm{Haze}}}.
    \label{eq:3D_haze_formation}
\end{split}
\end{align}
}
\cref{eq:3D_haze_formation} formally disentangles the surface and haze, represented by \(\left\{ c, \sigma \right\}\) and \(\left\{  c_{s}, \sigma_{s} \right\}\) respectively, in a principled manner.
Once successfully optimized (see the next Section), the clear-view surfaces can be recovered using \(\left\{ c, \sigma \right\}\):
\begin{equation}
C(\r,\d)=
\int_{t_{n}}^{t_{0}}c(\r(t),\d)\sigma(t)T_{\sigma}\left( t \right)dt\label{eq:clear_view}.
\end{equation}

\begin{figure}[t!]
\centering \includegraphics[width=\linewidth]{images/architecture.pdf}
\makeatother
\caption{\textbf{\moniker{} architecture.} Given a set of hazy images, our method augments the existing NeRF pipeline (gray) with a haze module (yellow), which explicitly models the scattering phenomenon using atmospheric light and scattering coefficient. During training, we render the hazy reconstruction as a composition of surface and haze, which is compared to the input hazy images to optimize the learnable parameters (in green) jointly. During inference, we use the surface module (gray) to render clear views.}
\vspace{-0.5cm}\label{fig:architecture}
\end{figure}

\subsection{Haze-aware Neural Radiance Field}\label{sec:dehaze_nerf}
Given multiple images of a hazy scene, we aim to jointly optimize for the surface appearance and geometry, \(\left\{ c, \sigma \right\}\) as well as the haze's scattering coefficient and in-scattered light (atmospheric light),  \(\left\{c_{s}, \sigma_{s} \right\}\) based on the enhanced scattering-aware rendering equation~\cref{eq:3D_haze_formation}.
However, the effects of these variables are interdependent. In order to correctly disentangle them, our model adopts suitable architecture designs and training regularizers to capture the distinct physical properties of haze and surface.
An overview of \moniker{} is illustrated in \cref{fig:architecture}.

\paragraph{Architecture.} Now we introduce inductive biases to match the physical properties of haze and surface.
For clarity, we highlight the quantities directly modeled by neural networks in \nn{green}.

\emph{Modeling a Surface.} Recall our goal is to learn the surface appearance and geometry, \(\left\{ c, \sigma \right\}\).
Similar to previous works~\cite{mildenhall2020nerf}, we model the appearance \(\cnet\left( \p, \d \right)\) with an MLP, which takes the sample location \(\p\) and viewing direction \(\d\) as inputs.
However, in order to encourage volume density \(\sigma\) to form a well-defined solid surface, instead of directly learning the volume density, we adopt the reparameterization of the volume density using signed distance function (SDF), \(\sdf\left( \r\left( t \right) \right)\in \R\), as proposed in NeuS~\cite{wang2021neus,wang2022hfs}.
The modified surface volume density \(\sigma\left( \r\left( t \right) \right) \), referred to as opaque density, can be parameterized as \(\sdf\left( \r\left( t \right) \right)\):
\begin{equation}
\sigma\left( \r\left( t \right) \right) = s\left( \Phi_{s}\left( \sdf\left( \r\left( t \right) \right)\right) -1 \right)\nabla \sdf\left( \r\left( t \right) \right)\mathbf{d},\label{eq:hfneus-sigma}
\end{equation}
where $\Phi_{s}(x)$ is the sigmoid function $\Phi_s(x) = (1 + e^{-sx})^{-1}$, whose derivative is a bell-shaped density function centered at 0 and has a learnable standard deviation of \(\nicefrac{1}{s}\).
We derive the discrete approximate following~\cite{mildenhall2020nerf,tagliasacchi2022volume}.
It samples $n$ points $\left\{ \p_{i}=\mathbf{o}+t_n\mathbf{d}|n=1,...,N,t_n<t_{n+1} \right\}$ along the ray.
The approximate pixel color of the ray is computed based on quadrature rule~\cite{max1995optical}, yielding
\begin{align}\begin{gathered}
C_{\textrm{surface}}(\r,\d) = \sum_{n=1}^{N}\frac{\sigma^{n}}{\sigma_{t}^{n}} T_{t}^{n}\alpha_{t}^{n}\nn{c}^{n} \textrm{ with } T_{t}^{n}=\prod_{m=1}^{n-1}\left(1 - \alpha_{t}^{m}\right) \label{eq:C_surface},
\end{gathered}\end{align}
where \(\alpha_{t}\) denotes the discrete \(\alpha\)-compositional weight defined as~\cite{wang2021neus,wang2022hfs}
\begin{equation}
 \resizebox{1\hsize}{!}{
 $
    \alpha_{t}^{n}=\textsc{clamp}\left( 1-\exp\left( -\sigma_{t}^{n}\delta^{n} \right),0, 1 \right) \textrm{ with } \delta^{n}=t^{n+1}-t^{n}\label{eq:alpha}\nonumber,$}
\end{equation}
where \(\sigma_{t}^{n}=\sigma^{n}+\nn{\sigma_{s}}^{n}\) denotes the total attenuation at sample \(n\), including the attenuation due to surface occlusion and the out-scattering.

\emph{Modeling Haze.} We use a low-frequency prior to compute the scattering coefficient and atmospheric light, \(\left\{c_{s}, \sigma_{s} \right\}\), since these components usually vary slowly in a common hazy scenes~\cite{li2015simultaneous}.
In practice, we use a small band-limited \textsc{MLP}~\cite{lindell2022bacon} for the scattering coefficient \(\sigma_{s}\) to capture inhomogenous haze.
Analogous to \cref{eq:C_surface}, the haze color can be approximated as
% \begin{equation}
% \begin{gathered}
% C_{\textrm{haze}}(\r) = \sum_{i=1}^{n}\frac{\nn{\sigma_{s}}^{n}}{\sigma_{t}^{n}} T_{t}^{n}\alpha_{t}^{n}\nn{c_{s}}^{n}.\label{eq:C_haze}
% \end{gathered}
% \end{equation}
\begin{equation}
\begin{gathered}
C_{\textrm{haze}}(\r) = \sum_{n=1}^{N}\frac{\nn{\sigma_{s}}^{n}}{\sigma_{t}^{n}} T_{t}^{n}\alpha_{t}^{n}\nn{c_{s}}^{n}.\label{eq:C_haze}
\end{gathered}
\end{equation}
During optimization, the color for an arbitrary input hazy image can be written as $C = C_{\textrm{surface}} + C_{\textrm{haze}}$.
At test time, we can reconstruct the clear-view color by discretizing \cref{eq:clear_view}, namely:
\begin{gather}
 C_{\textrm{clear}}\left( \r,\d \right) = \sum_{n=1}^{N}T_{\sigma}^{n}\alpha^{n} \nn{c}^{n}, \label{eq:clear_view_discrete}\\
 \resizebox{1\hsize}{!}{
 $T_{\sigma}^{n} = \prod_{j=1}^{n-1}\left(1 - \alpha^{j}\right)\, \textrm{and }\, \alpha^{n} = \textsc{clamp}\left( 1 - \exp\left( -\sigma^{n}\delta^{n} \right),0, 1 \right).\nonumber$}
\end{gather}
\paragraph{Optimization.} While the inductive biases separate the high-frequency surface appearance and geometry from the low-frequency color and density of the scattering medium, we introduce further regularizers to guide the optimization process to converge to more plausible clear-view geometry and color.

\emph{Koschmieder Consistency.}
Given an accurate depth map \(D\), assuming globally constant scattering coefficient \(\bar{\sigma}_{s}\) and airlight \(\bar{c}_{s}\), the relation between a clear-view image \(C_{\textrm{clear}}\) and the hazy image \(C\) can be described by the Koschmieder law~\cite{israel1959koschmieders} as
\begin{equation}
\resizebox{0.88\hsize}{!}{
\(C(\r)=C_{\textrm{clear}}(\r)\exp(-\bar{\sigma}_{s} D(\r))+\bar{c}_{s}(1-\exp(-\bar{\sigma}_{s} D(\r)))\).
}\label{eq:koschmieder}
\end{equation}
This model is widely adopted as the basis for image-based single and multiview dehazing.
The Koschmider model is an approximation of our rendering equation~\cref{eq:3D_haze_formation} under the assumption of
spatially-invariant (i.e., homogeneous) scattering coefficient and an ideal surface
\begin{align}
C_{\textrm{surface}}\left( \r \right) & \approx C_{\textrm{clear}}(\r)\exp(-\bar{\sigma}_{s} D(\r)) = \tilde{C}_{\textrm{surface}}\left( \r \right)\\
C_{\textrm{haze}}\left( \r \right) & \approx \bar{c}_{s}(1-\exp(-\bar{\sigma}_{s} D(\r)) = \tilde{C}_{\textrm{haze}}\left( \r \right),
\end{align}

We promote this relation with
%
\begin{align}
&\loss_{\textrm{2D}} = \left\|C_{\textrm{surface}}\left( \r \right) -  \tilde{C}_{\textrm{surface}}\left( \r \right)\right\|_{1} \\+
&\left\| C_{\textrm{haze}}\left( \r \right)\! - \!\tilde{C}_{\textrm{haze}}\left( \r \right)\right\|_{1} \!\!+\!
 \left\| C\! -\! \tilde{C}_{\textrm{surface}}\left( \r \right)\! -\! \tilde{C}_{\textrm{haze}}\left( \r \right)\right\|_{1}\!, \nonumber
\end{align}
%
where \(\bar{\sigma}_{s}\) and \(\bar{c}_{s}\) are the average over the samples on the ray, while
the depth value \(D\left( \r \right)\) is computed via the learned surface geometry~\cite{mildenhall2020nerf,yu2022monosdf} by accumulating over ray-length over all the samples on a ray:
\begin{equation}
    D\left( \r \right) = \sum_{n=1}^{N} T_{\sigma}^{n}\alpha^{n}t^{n}.
\end{equation}

\emph{Color Prior.}
Without knowing the original image, the heavily attenuated color in the hazy image can be explained by the haze but also by a dull surface color.
In order to reconstruct plausible clear-view colors, we adopt the popular 2D prior widely used in image-based dehazing methods, Dark Channel Prior (DCP)~\cite{he2010single}, which arises from the observation, that for most pixels in a natural haze-free image, the minimum of three color channels is close to zero.
We apply this prior to the estimated clear image \(C_{\textrm{clear}}\)
\begin{align}
DC(C_{\textrm{clear}})\left(\x\right)&=\underset{\y\in\Omega\left(\x\right)}{\min}\left(\underset{c\in\left\{r,g,b\right\}}{\min}C_\textrm{clear}^{c}\left(\y\right)\right),
\label{eq:DCP_definition}\\
\loss_{\textrm{dcp}}&=\frac{1}{K}\sum\limits_{k=1}^{K}\Vert DC\left(C_{\textrm{clear}}\right)\Vert_{1}.
\label{eq:loss_dcp}
\end{align}


\subsection{Implementation Details}
We adopt the same setting as that in HF-NeuS~\cite{wang2021neus} wherever possible.
This includes the MLPs for the surface SDF, \(\sdf\) and the view-dependent surface color, \(\cnet\), as well as the sampling strategy, the background composition, and learning rate schedule.

\paragraph{Loss.}
Our loss is composed of several terms:
\begin{equation}
    \loss = \loss_{\textrm{color}} + \lambda\loss_{\textrm{eikonal}} + \alpha\loss_{\textrm{dcp}} + \beta\loss_{\textrm{2D}},\label{eq:total_loss}
\end{equation}
where \(\loss_{\textrm{dcp}}\) and \(\loss_{\textrm{2D}}\) are the regularizations introduced in \cref{sec:dehaze_nerf},
while the photo-consistency loss, $\loss_{\textrm{color}}$, is the standard NeRF loss, and the eikonal loss, \(\loss_{\textrm{eikonal}}\), is commonly used to regularize SDF~\cite{gropp2020implicit},
\begin{align}
    \loss_{\textrm{color}}& = \frac{1}{K}\sum_{k=1}^{K}\left\|\widehat{C}_{k}(\r,\d) - C_{k}(\r,\d)\right\|_{1},\\
    \loss_{\textrm{eikonal}} &= \frac{1}{KN}\sum_{k}^{K}\sum_{n}^{N}(\|\nabla f({\mathbf{r}}_{k}(t_n))\|_2 - 1)^2,
\label{eq:loss_color}
\end{align}
where $\widehat{C}_{k}(\r,\d)$ is the pixel color. $N$ and $K$ denote the total sampling points on a ray and the total number of rays sampled per training batch.

Finally, because of the surface representation using SDF, we can optionally adopt the object masks for supervision~\cite{yariv2021volume,wang2021neus,wang2022hfs}.
Specifically, given the object mask, \(M\), the mask loss $\loss_{\textrm{mask}}$ for a sampled ray $k$ is defined as
\begin{equation}
    \loss_{\textrm{mask}} = \text{BCE}(M_k, \hat{O}_k),\label{eq:mask_loss}
\end{equation}
where $\hat{O}_k = \sum_{i=1}^{N}T_{\sigma}^{i}\alpha^{i}$ is the total weight for the clear-view surface color along the camera ray, and $\text{BCE}$ is the binary cross entropy loss.



%------------------------------------------------------------------------
% Experiments
\begin{table*}
\begin{center}
\caption{Comparison with \sota\ methods on the public crowd analysis benchmarks: \jhu, ShanghaiTech, UCF, and \nwpu. 
The best results are shown in \first{red}. The second-best results are shown in \second{blue}. 
}
\vspace{\tablegap}
\resizebox{0.95\textwidth}{!}{
\begin{tabular}{l c c c c c c c c c c c c c}
\toprule
 \multirow{2}{*}{Method} & \multirow{2}{*}{Venue} &\multicolumn{2}{c}{\jhu} &\multicolumn{2}{c}{\shha} &\multicolumn{2}{c}{\shhb} &\multicolumn{2}{c}{\ucf} &\multicolumn{2}{c}{\qnrf} &\multicolumn{2}{c}{\nwpu}\\[0.2ex]
 \cmidrule(lr){3-4}\cmidrule(lr){5-6}\cmidrule(lr){7-8}\cmidrule(lr){9-10}\cmidrule(lr){11-12}\cmidrule(lr){13-14}
& & MAE$\downarrow$ & MSE$\downarrow$ & MAE$\downarrow$ & MSE$\downarrow$ & MAE$\downarrow$ & MSE$\downarrow$ & MAE$\downarrow$ & MSE$\downarrow$ & MAE$\downarrow$ & MSE$\downarrow$ & MAE$\downarrow$ & MSE$\downarrow$\\[0.2ex]
\midrule\midrule
TopoCount \cite{abousamra2021localization}	& AAAI'21	& {60.9}	& {267.4}	& {61.2}	& {104.6}	& {7.8}	& {13.7}	& {184.1}	& {258.3}	& {89.0}	& {159.0}	& {107.8}	& {438.5}	\\[0.2ex]
SUA \cite{meng2021spatial}	& ICCV'21	& {80.7}	& {290.8}	& {68.5}	& {121.9}	& {14.1}	& {20.6}	& {-}	& {-}	& {130.3}	& {226.3}	& {111.7}	& {443.2}	\\[0.2ex]
ChfL \cite{shu2022crowd}	& CVPR'22	& {57.0}	& {235.7}	& {57.5}	& {94.3}	& {6.9}	& {11.0}	& {-}	& {-}	& {80.3}	& {137.6}	& {76.8}	& {343.0}	\\[0.2ex]
MAN \cite{lin2022boosting}	& CVPR'22	& {53.4}	& \second{209.9}	& {56.8}	& {90.3}	& {-}	& {-}	& {-}	& {-}	& {77.3}	& {131.5}	& {76.5}	& {323.0}	\\[0.2ex]
GauNet \cite{cheng2022rethinking}	& CVPR'22	& {58.2}	& {245.1}	& {54.8}	& {89.1}	& {6.2}	& {9.9}	& {186.3}	& {256.5}	& {81.6}	& {153.7}	& {-}	& {-}	\\[0.2ex]
CLTR \cite{liang2022end}	& ECCV'22	& {59.5}	& {240.6}	& {56.9}	& {95.2}	& {6.5}	& {10.6}	& {-}	& {-}	& {85.8}	& {141.3}	& {74.3}	& {333.8}	\\[0.2ex]
CrwodHat \cite{wu2023boosting}	& CVPR'23	& \second{52.3}	& {211.8}	& {51.2}	& {81.9}	& \first{5.7}	& {9.4}	& {-}	& {-}	& {75.1}	& \second{126.7}	& {68.7}	& \second{296.9}	\\[0.2ex]
STEERER \cite{han2023steerer}	& ICCV'23	& {54.3}	& {238.3}	& {54.5}	& {86.9}	& {5.8}	& \second{8.5}	& {-}	& {-}	& {74.3}	& {128.3}	& \second{63.7}	& {309.8}	\\[0.2ex]
PET \cite{liu2023point}	& ICCV'23	& {58.5}	& {238.0}	& \second{49.3}	& \second{78.8}	& {6.2}	& {9.7}	& {-}	& {-}	& {79.5}	& {144.3}	& {74.4}	& {328.5}	\\[0.2ex]
\rowcolor{black!10}\method\	& 	& \first{47.3}	& \first{198.9}	& \first{47.4}	& \first{75.0}	& \first{5.7}	& \first{8.2}	& \first{160.8}	& \first{225.0}	& \first{68.9}	& \first{125.6}	& \first{57.8}	& \first{221.2}	\\[0.2ex]
\bottomrule
\end{tabular}
}
\vspace{\tablegap}
\label{table: crowd counting performance}
\end{center}
\end{table*}
\begin{table*}[!t]
    \begin{center}
    
    \resizebox{\textwidth}{!}{
    \begin{tabular}{l|cc|ccccc|c}
\toprule

Model & VLM & Additional Backbone & General & Earth Monit. & Medical Sciences & Engineering & Agri. and Biology & Mean \\
\midrule\midrule
\textit{Random (LB)} & - & - & \phantom{0}\textit{1.17} & \phantom{0}\textit{7.11} & \textit{29.51} & \textit{11.71} & \phantom{0}\textit{6.14} & \textit{10.27} \\
\textit{Best supervised (UB)} & - & - & \textit{48.62} & \textit{79.12} & \textit{89.49} & \textit{67.66} & \textit{81.94} & \textit{70.99} \\
\midrule
ZSSeg~\citep{xu2022simple} & CLIP ViT-B/16 & ResNet-101 & 19.98 & 17.98 & \underline{41.82} & 14.0\phantom{0} & 22.32 & 22.73 \\
ZegFormer~\citep{ding2022decoupling} & CLIP ViT-B/16 & ResNet-101 & 13.57 & 17.25 & 17.47 & 17.92 & \underline{25.78} & 17.57 \\
X-Decoder~\citep{zou2023generalized} & UniCL-T & Focal-T & 22.01 & 18.92 & 23.28 & 15.31 & 18.17 & 19.8\phantom{0} \\
OpenSeeD~\citep{zhang2023simple} & UniCL-B & Swin-T & 22.49 & 25.11 & \textbf{44.44} & 16.5\phantom{0} & 10.35 & 24.33 \\
SAN~\citep{xu2023side} & CLIP ViT-B/16 & - & \underline{29.35} & \underline{30.64} & 29.85 & \textbf{23.58} & 15.07 & \underline{26.74} \\

\hlrow & & & \textbf{38.69} & \textbf{35.91} & 28.09 & \underline{20.34} & \textbf{32.57} & \textbf{31.96} \\
\hlrow\multirow{-2}{*}{\ours (ours)} & \multirow{-2}{*}{CLIP ViT-B/16} & \multirow{-2}{*}{-} & \textcolor{ForestGreen}{(+9.34)} & \textcolor{ForestGreen}{(+5.27)} & \color{gray}{(-16.35)} & \color{gray}{(-3.24)} & \textcolor{ForestGreen}{(+6.79)} & \textcolor{ForestGreen}{(+5.22)} \\
\midrule
OVSeg~\citep{liang2022open} & CLIP ViT-L/14 & Swin-B & 29.54 & 29.04 & \textbf{31.9\phantom{0}} & 14.16 & \underline{28.64} & 26.94 \\
SAN~\citep{xu2023side} & CLIP ViT-L/14 & - & \underline{36.18} & \underline{38.83} & \underline{30.27} & \underline{16.95} & 20.41 & \underline{30.06} \\
\hlrow & & & \textbf{44.69} & \textbf{39.99} & 24.70 & \textbf{20.20} & \textbf{38.61} & \textbf{34.70} \\
\hlrow\multirow{-2}{*}{\ours (ours)} & \multirow{-2}{*}{CLIP ViT-L/14} & \multirow{-2}{*}{-} & \textcolor{ForestGreen}{(+8.51)} & \textcolor{ForestGreen}{(+1.16)} & \color{gray}{(-7.2)} & \textcolor{ForestGreen}{(+3.25)} & \textcolor{ForestGreen}{(+9.97)} & \textcolor{ForestGreen}{(+4.64)} \\
        \bottomrule
    \end{tabular}
    }

    \vspace{-5pt}        
    \caption{\textbf{Quantitative evaluation on MESS~\citep{blumenstiel2023mess}.} MESS includes a wide range of domain-specific datasets, which pose significant challenges due to their substantial domain differences from the training dataset. We report the average score for each domain. Please refer to the supplementary material for the results of all 22 datasets. \textit{Random} is the result of uniform distributed prediction which represents the lower-bound, while \textit{Best supervised} represents the upper-bound performance for the datasets.}
    \label{tab:mess}
    \vspace{-20pt}
    \end{center}
\end{table*}


\section{Experiments}
\subsection{Datasets and Evaluation}
We train our model on the COCO-Stuff~\cite{caesar2018coco}, which has 118k densely annotated training images with 171 categories, following \cite{liang2022open}. We employ the mean Intersection-over-Union (mIoU) as the evaluation metric for all experiments. For the evaluation, we conducted experiments on two different sets of datasets~\cite{zhou2019semantic,everingham2009pascal,mottaghi2014role}: a commonly used in-domain datasets~\cite{ghiasi2022scaling}, and a multi-domain evaluation set~\cite{blumenstiel2023mess} containing domain-specific images and class labels. 

\vspace{-10pt}
\paragraph{Datasets for standard benchmarks.} 
For in-domain evaluation, we evaluate our model on ADE20K~\cite{zhou2019semantic}, PASCAL VOC~\cite{everingham2009pascal}, and PASCAL-Context~\cite{mottaghi2014role} datasets. ADE20K has 20k training and 2k validation images, with two sets of categories: A-150 with 150 frequent classes and A-847 with 847 classes~\cite{ding2022decoupling}. PASCAL-Context contains 5k training and validation images, with 459 classes in the full version (PC-459) and the most frequent 59 classes in the PC-59 version. PASCAL VOC has 20 object classes and a background class, with 1.5k training and validation images. We report PAS-20 using 20 object classes. We also report the score for PAS-$20^b$, which defines the ``background" as classes present in PC-59 but not in PAS-20, as in \citet{ghiasi2022scaling}.

\vspace{-10pt}
\paragraph{Datasets for multi-domain evaluation.}
We conducted a multi-domain evaluation on the MESS benchmark~\cite{blumenstiel2023mess}, specifically designed to stress-test the real-world applicability of open-vocabulary models with 22 datasets. The benchmark includes a wide range of domain-specific datasets from fields such as earth monitoring, medical sciences, engineering, agriculture, and biology. Additionally, the benchmark contains a diverse set of general domains, encompassing driving scenes, maritime scenes, paintings, and body parts. We report the average scores for each domain in the main text for brevity. For the complete results and details of the 22 datasets, please refer to the supplementary material.

\subsection{Implementation Details}
We train the CLIP image encoder and the cost aggregation module with per-pixel binary cross-entropy loss. We set $d_F=128$, $N_B=2$, $N_U=2$ for all of our models. We implement our work using PyTorch~\cite{paszke2019pytorch} and Detectron2~\cite{wu2019detectron2}. AdamW~\cite{loshchilov2017decoupled} 
optimizer is used with a learning rate of $2\cdot10^{-4}$ for our model  and $2\cdot10^{-6}$ for the CLIP, with weight decay set to $10^{-4}$. The batch size is set to 4. We use 4 NVIDIA RTX 3090 GPUs for training. All of the models are trained for 80k iterations. 

\subsection{Main Results}
\paragraph{Results of standard benchmarks.}
The evaluation of standard open-vocabulary semantic segmentation benchmarks is shown in Table~\ref{tab:main_table}. Overall, our method significantly outperforms all competing methods, including those~\cite{ghiasi2022scaling,liang2022open} that leverage additional datasets~\cite{chen2015microsoft,pont2020connecting} for further performance improvements. To ensure a fair comparison, we categorize the models based on the scale of the vision-language models (VLMs) they employ. First, we present results for models that use VLMs of comparable scale to ViT-B/16~\cite{dosovitskiy2020image}, and our model surpasses all previous methods, even achieving performance that matches or surpasses those using the ViT-L/14 model as their VLM~\cite{xu2023side}.
For models employing the ViT-L/14 model as their VLM, our model demonstrates remarkable results, achieving a 16.0 mIoU in the challenging A-847 dataset and a 23.8 mIoU in PC-459. These results represent a 29\% and 52\% increase, respectively, compared to the previous state-of-the-art.
We also present qualitative results of PASCAL-Context with 459 categories in Fig.~\ref{fig:qualitative}, demonstrating the efficacy of our proposed approach in comparison to the current state-of-the-art methods~\cite{ding2022decoupling, xu2022simple,liang2022open}. 


\begin{figure*}[t]
  \centering
    \subfloat[SAN]
{\includegraphics[width=0.1595\linewidth]{figures/fig4/pc_san.pdf}}\hfill
    \subfloat[\textbf{Ours}]
{\includegraphics[width=0.1595\linewidth]{figures/fig4/pc_ours.pdf}}\hfill
     \subfloat[GT]
 {\includegraphics[width=0.1595\linewidth]{figures/fig4/pc_gt.pdf}}\hfill
    \subfloat[SAN]
{\includegraphics[width=0.166\linewidth]{figures/fig4/mess_san.pdf}}\hfill
    \subfloat[\textbf{Ours}]
{\includegraphics[width=0.166\linewidth]{figures/fig4/mess_ours.pdf}}\hfill
    \subfloat[GT]
{\includegraphics[width=0.166\linewidth]{figures/fig4/mess_gt.pdf}}\hfill
\\
\vspace{-10pt}
\caption{\textbf{Qualitative comparison to SAN~\citep{xu2023side}.} We visualize the results of PC-459 dataset in (a-c). For (d-f), we visualize the results from the MESS benchmark~\citep{blumenstiel2023mess} across three domains: underwater (top), human parts (middle), and agriculture (bottom).} 
\label{fig:qualitative}
\vspace{-10pt}
\end{figure*}

\begin{table}[t]
    \centering
    \resizebox{0.48\textwidth}{!}{%
    \begin{tabular}{cl|cccccc}
    \toprule
        & Methods & A-847 & PC-459 & A-150 & PC-59 & PAS-20 & $\textnormal{PAS-20}^b$
        \\
        \midrule\midrule
        \textbf{(I)} & Feature agg. + Freeze & 3.1 & 8.7 & 16.6 & 46.8 & 92.3 & 69.7\\
        \textbf{(II)} & Feature agg. + F.T. & 5.6 & {12.8} & {23.6} & \underline{58.1} & \underline{96.3} & \underline{77.7}\\
        \midrule
        \textbf{(III)} & Cost agg. + Freeze& \underline{10.0} & \underline{14.5} & \underline{26.0} & 46.9 & 94.2 & 65.1\\
        \hlrow\textbf{(IV)} & Cost agg. + F.T. & \textbf{14.7} & \textbf{23.2} & \textbf{35.3} & \textbf{60.3} & \textbf{96.7} & \textbf{78.9}\\
        \bottomrule
    \end{tabular}%
    }
    \vspace{-5pt}
    \caption{\textbf{Quantitative comparison between feature and cost aggregation.} Cost aggregation acts as an effective alternative to direct fine-tuning of CLIP image encoder. \textit{F.T.: Fine-Tuning.}
    }
    \label{tab:feature-vs-cost}
    \vspace{-15pt}

\end{table}



\vspace{-10pt}
\paragraph{Results of multi-domain evaluation.}
In Table~\ref{tab:mess}, we present the qualitative results obtained from the MESS benchmark~\cite{blumenstiel2023mess}. This benchmark assesses the real-world performance of a model across a wide range of domains. Notably, our model demonstrates a significant performance boost over other models, achieving the highest mean score. It particularly excels in the general domain as well as in agriculture and biology, showing its strong generalization ability. However, in the domains of medical sciences and engineering, the results exhibit inconsistencies with respect to the size of the VLM. Additionally, the scores for medical sciences are comparable to random predictions. We speculate that CLIP may have limited knowledge in these particular domains~\cite{radford2021learning}.


\subsection{Analysis and Ablation Study}\label{sec:ablation}

\paragraph{Comparison between feature and cost aggregation.} We provide quantitative and qualitative comparison of two aggregation baselines, feature aggregation, and cost aggregation, in Table~\ref{tab:feature-vs-cost}. For both of baseline architectures, we simply apply the upsampling decoder and note that both methods share most of the architecture, but differ in whether they aggregate the concatenated features or aggregate the cosine similarity between image and text embeddings of CLIP.
\begin{table}[t]
    \centering
    \resizebox{0.48\textwidth}{!}{%
    \begin{tabular}{cl|cccccc}
    \toprule
        & Methods & A-847 & PC-459 & A-150 & PC-59 & PAS-20 & $\textnormal{PAS-20}^b$
        \\
        \midrule\midrule
        \textbf{(I)} & Feature agg. + Freeze & 3.1 & 8.7 & 16.6 & 46.8 & 92.3 & 69.7\\
        \textbf{(II)} & Feature agg. + F.T. & 5.6 & {12.8} & {23.6} & \underline{58.1} & \underline{96.3} & \underline{77.7}\\
        \midrule
        \textbf{(III)} & Cost agg. + Freeze& \underline{10.0} & \underline{14.5} & \underline{26.0} & 46.9 & 94.2 & 65.1\\
        \hlrow\textbf{(IV)} & Cost agg. + F.T. & \textbf{14.7} & \textbf{23.2} & \textbf{35.3} & \textbf{60.3} & \textbf{96.7} & \textbf{78.9}\\
        \bottomrule
    \end{tabular}%
    }
    \vspace{-5pt}
    \caption{\textbf{Quantitative comparison between feature and cost aggregation.} Cost aggregation acts as an effective alternative to direct fine-tuning of CLIP image encoder. \textit{F.T.: Fine-Tuning.}
    }
    \label{tab:feature-vs-cost}
    \vspace{-15pt}

\end{table}


For \textbf{(I)} and \textbf{(III)}, we freeze the encoders of CLIP and only optimize the upsampling decoder. Subsequently, in \textbf{(II)} and \textbf{(IV)}, we fine-tune the encoders of CLIP on top of \textbf{(I)} and \textbf{(III)}. Our results show that feature aggregation can benefit from fine-tuning, but the gain is only marginal. On the other hand, cost aggregation benefits significantly from fine-tuning, highlighting the effectiveness of cost aggregation for adapting CLIP to the task of segmentation. 

For the qualitative results in Fig.~\ref{fig:feature_cost}, we show the prediction results from \textbf{(II)} and \textbf{(IV)}. As seen in Fig.~\ref{fig:feature_cost}(c-d), we observe that feature aggregation shows overfitting to the seen class of ``bucket," while cost aggregation successfully identifies the unseen class ``birdcage." 

\begin{table}[t!]
\centering
\resizebox{\linewidth}{!}{
\begin{tabular}{ll|cccccc}
        \toprule
        &Components & A-847 & PC-459 & A-150 & PC-59 & PAS-20 & $\textnormal{PAS-20}^b$
         \\
        \midrule\midrule
        \textbf{(I)} & Feature Agg. & 5.6 & 12.8 & 23.6 & 58.1 & 96.3 & 77.7\\
        \midrule
        \textbf{(II)} & Cost Agg.  & 14.7& \underline{23.2}& 35.3& 60.3& \underline{96.7}&78.9\\
        \textbf{(III)} &\textbf{(II)} + Spatial agg.  & 14.9& 23.1& 35.9& 60.3& \underline{96.7}&79.5\\
        \textbf{(IV)} &\textbf{(II)} + Class agg.  & 14.7& 21.5& 36.6& 60.6& 95.5&80.5\\
        \textbf{(V)} &\textbf{(II)} + Spatial and Class agg. & \underline{15.5}& \underline{23.2}& \underline{37.0}& \underline{62.3}& \underline{96.7}&\underline{81.3}\\
        \hlrow\textbf{(VI)} &\textbf{(V)} + Embedding guidance  & \textbf{16.0} & \textbf{23.8}& \textbf{37.9}& \textbf{63.3}& \textbf{97.0}&\textbf{82.5}\\
        \bottomrule
\end{tabular}}
\vspace{-5pt}
\caption{\textbf{Ablation study for \ours.} We conduct ablation study by gradually adding components to the cost aggregation baseline.}
    \vspace{-10pt}
    \label{tab:ablation}
\end{table}


\begin{table}[h!]\scriptsize	
    \centering
    \begin{tabular}{c|c|c|c|c|c}
        \multicolumn{2}{c|}{} & Baseline & w/o normals & w/o viscosity & w/o coarea \\ \hline
        \multirow{4}{*}{Anchor}
            & $d_C$ & \textbf{0.21} & 0.61 & 0.55 & 0.72 \\
            & $d_H$ & \textbf{3.00} & 7.82 & 10.83 & 10.24 \\
            & $d_C^\too$ & 0.15 & 0.37 & 0.27 & 0.36 \\
            & $d_H^\too$ & 1.07 & 7.84 & 1.44 & 9.68 \\ \hline
        \multirow{4}{*}{Daratech}
            & $d_C$ & 0.26 & 0.24 & 0.24 & \textbf{0.23} \\
            & $d_H$ & 4.06 & 4.2 & 4.3 & \textbf{2.19} \\
            & $d_C^\too$ & 0.14 & 0.13 & 0.12 & 0.13 \\
            & $d_H^\too$ & 1.76 & 2.69 & 1.77 & 1.77 \\ \hline
        \multirow{4}{*}{DC}
            & $d_C$ & \textbf{0.15} & \textbf{0.15} & \textbf{0.15} & 0.34 \\
            & $d_H$ & \textbf{2.22} & 2.24 & 2.24 & 6.58 \\
            & $d_C^\too$ & 0.09 & 0.08 & 0.08 & 0.16 \\
            & $d_H^\too$ & 2.76 & 2.76 & 2.79 & 2.82 \\ \hline
        \multirow{4}{*}{Gargoyle}
            & $d_C$ & \textbf{0.17} & 0.58 & 0.47 & 0.59 \\
            & $d_H$ & \textbf{4.40} & 6.32 & 10.38 & 6.35 \\
            & $d_C^\too$ & 0.11 & 0.07 & 0.26 & 0.38 \\
            & $d_H^\too$ & 0.96 & 2.39 & 1.34 & 1.25 \\ \hline
        \multirow{4}{*}{Lord Quas}
            & $d_C$ & \textbf{0.12} & 0.12 & 0.12 & 0.58 \\
            & $d_H$ & 1.06 & 1.38 & \textbf{1.04} & 6.05 \\
            & $d_C^\too$ & 0.07 & 0.37 & 0.06 & 0.32 \\
            & $d_H^\too$ & 0.64 & 0.69 & 0.64 & 3.73 \\ \hline %
            
    \end{tabular} \vspace{5pt}
    \caption{Ablations study. We show the contribution of each component of VisCo Grids. Baseline is the full method. The remaining columns correspond to optimizing without normal loss, viscosity loss and coarea loss, respectively. We show results for each mesh of the benchmark \cite{williams2019deep}. The results justify the use of the different components in VisCo Grids.}
    \label{tab:ablations}
\end{table}
\vspace{-10pt}
\paragraph{Component analysis.}
Table~\ref{tab:ablation} shows the effectiveness of the main components within our architecture through quantitative results. 
First, we introduce the baseline models in \textbf{(I)} and \textbf{(II)}, identical to the fine-tuned baseline models from Table~\ref{tab:feature-vs-cost}.
We first add the proposed spatial and class aggregations to the cost aggregation baseline in \textbf{(III)} and \textbf{(IV)}, respectively. In \textbf{(V)}, we interleave the spatial and class aggregations. Lastly, we add the proposed embedding guidance to \textbf{(V)}, which becomes our final model.

As shown, we stress the gap between \textbf{(I)} and  \textbf{(II)}, which supports the findings presented in Fig.~\ref{fig:feature_cost}. Given that PAS-20 shares most of its classes with the training datasets\cite{xu2022simple}, the performance gap between \textbf{(I)} and \textbf{(II)} is minor. However, for challenging datasets such as A-847 or PC-459, the difference is notably significant, validating our cost aggregation framework for its generalizability.
We also highlight that as we incorporate the proposed spatial and class aggregation techniques, our approach \textbf{(V)} outperforms \textbf{(II)}, demonstrating the effectiveness of our design.
Finally, \textbf{(VI)} shows that our embedding guidance further improves performance across all the benchmarks.
Furthermore, we provide quantitative results of adopting the upsampling decoder in Table ~\ref{tab:conv-decoder}. The results show consistent improvements across all the benchmarks.


\begin{table}[!t]
\centering
\resizebox{\linewidth}{!}{
   \begin{tabular}{ll|cccccc|cc}
        \toprule
        &\multirow{2}{*}{Methods} & \multirow{2}{*}{A-847} & \multirow{2}{*}{PC-459} & \multirow{2}{*}{A-150} & \multirow{2}{*}{PC-59}& \multirow{2}{*}{PAS-20} & \multirow{2}{*}{$\textnormal{PAS-20}^b$} &\#param.  & Memory
         \\
         &&&&&&&&(M)&(GiB)
         \\
        \midrule\midrule
        \textbf{(I)} &Freeze & 10.4& 15.0& 31.8& 52.5& 92.2& 71.3& 5.8 & 20.0\\
        \textbf{(II)} &Prompt  & 8.8& 14.3 & 30.5& 55.8 & 93.2 & 74.7 & 7.0 & 20.9\\
        \textbf{(III)} &Full F.T.  & 13.6& 22.2& 34.0& 61.1& \textbf{97.3}& 79.7 & 393.2 & 26.8\\
        \textbf{(IV)} &Attn. F.T. & 15.7& \underline{23.7}& 37.1& \underline{63.1}& \underline{97.1}& 81.5 & 134.9 & 20.9\\
        \textbf{(V)} &QK F.T. & 15.3& 23.0& 36.3& 62.0& 95.9& 81.9 & 70.3 & 20.9\\
        \textbf{(VI)} &KV F.T. & \textbf{16.1}& \textbf{23.8}& \underline{37.6}& 62.4& 96.7& \underline{82.0} & 70.3 & 20.9\\
        \midrule
        \textbf{(VII)} & QV F.T. (Img.)  & 13.9& 22.8& 35.1& 62.0& 96.3& \underline{82.0} & 56.7 & 20.9\\
        \textbf{(VIII)} & QV F.T. (Txt.)  & 14.7& 22.2& 35.1& 60.0& 95.8& 80.3 & 19.9 & 20.0\\
        \hlrow \textbf{(IX)} & QV F.T. (Both) & \underline{16.0}& \textbf{23.8}& \textbf{37.9}& \textbf{63.3}& 97.0& \textbf{82.5} & 70.3 & 20.9\\
        \bottomrule       
\end{tabular}
}
    \vspace{-5pt}
\caption{\textbf{Analysis of fine-tuning methods for CLIP.} We additionally note the number of learnable parameters of CLIP and memory consumption during training. Our method not only outperforms full fine-tuning, but also requires smaller computation.}
\label{tab:finetuning-ablation}
\vspace{-10pt}
\end{table}

\begin{figure}[t]
  \centering
    \subfloat[CLIP]
{\includegraphics[width=0.4999\linewidth]{figures/fig_embedding/tsne_clip_final.pdf}}\hfill
     \subfloat[Fine-tuned CLIP]
 {\includegraphics[width=0.4999\linewidth]{figures/fig_embedding/tsne_clip_2_v3.pdf}}\hfill\\
        
\vspace{-5pt}
\caption{\textbf{Effects of fine-tuning CLIP.} We show the t-SNE~\cite{van2008visualizing} 
visualization of CLIP image embeddings based on its predictions. In contrast to (a), we observe well-grouped clusters in (b), showing the adaptation of CLIP to segmentation for both seen and unseen classes.} 
\label{fig:embedding_space}
\vspace{-10pt}
\end{figure}


\vspace{-10pt}
\label{finetune}
\paragraph{Analysis on fine-tuning of CLIP.}
In this section, we analyze the effects and methods of fine-tuning of the encoders of CLIP. In Table~\ref{tab:finetuning-ablation}, we report the results of different approaches, which include the variant \textbf{(I)}:~without fine-tuning, \textbf{(II)}:~adopting Prompt Tuning~\cite{zhou2022learning, jia2022visual}, \textbf{(III)}:~fine-tuning the entire CLIP, \textbf{(IV)}:~fine-tuning the attention layer only~\cite{touvron2022three}, \textbf{(V)}:~fine-tuning query and key projections only, \textbf{(VI)}:~fine-tuning key and value projections only, \textbf{(VII)}:~our approach for CLIP image encoder only, \textbf{(VIII)}:~our approach for text encoder only, and  \textbf{(IX)}:~our approach for both encoders. Note that both image and text encoders are fine-tuned in \textbf{(I-VI)}. Overall, we observed that fine-tuning enhances the performance of our framework. Among the various fine-tuning methods, fine-tuning only the query and value projection yields the best performance improvement while also demonstrating high efficiency. Additionally, as can be seen in \textbf{(VII-IX)}, fine-tuning both encoders leads to better performance compared to fine-tuning only one of them in our framework.

In Fig.~\ref{fig:embedding_space}, we show the t-SNE~\cite{van2008visualizing} visualization of the dense image embeddings of CLIP within the A-150~\cite{zhou2019semantic} dataset. We color the embeddings based on the prediction with text classes. From (a), we can observe that the clusters are not well-formed for each classes, due to the image-level training of CLIP. In contrast, we observe well-formed clusters in (b) for both seen and unseen classes, showing the adaptation of CLIP for the downstream task.

\vspace{-10pt}
\paragraph{Training with various datasets.}
In this experiment, we further examine the generalization power of our method in comparison to other methods~\cite{ding2022decoupling, xu2022simple} by training our model on smaller-scale datasets, which include A-150 and PC-59, that poses additional challenges to achieve good performance.  The results are shown in Table~\ref{tab:cross-dataset-ablation}. As shown, we find that although we observe some performance drops, which seem quite natural when a smaller dataset is used, our work significantly outperforms other competitors. These results highlight the strong generalization power of our framework, a favorable characteristic that suggests the practicality of our approach.

\begin{table}[!t]
    \centering
    
    \resizebox{\linewidth}{!}{
    \begin{tabular}{l|c|ccccccc}
    \toprule
        Methods & Training dataset & A-847 & PC-459 & A-150 & PC-59 & PAS-20 & $\textnormal{PAS-20}^b$
        \\
        \midrule\midrule
        ZegFormer & COCO-Stuff & 5.6 & \underline{10.4} & 18.0 & 45.5 & \underline{89.5} & 65.5\\
        ZSseg & COCO-Stuff & \underline{7.0} & 9.0 & \underline{20.5} & \underline{47.7} & 88.4 & \underline{67.9}\\
        \hlrow \ours (ours) & COCO-Stuff & \textbf{12.0} & \textbf{19.0} & \textbf{31.8} & \textbf{57.5} & \textbf{94.6} & \textbf{77.3}\\
        \midrule
        ZegFormer & A-150 & 6.8 & \underline{7.1} & \color{gray}{33.1} & 34.7 & 77.2 & 53.6 \\
        ZSseg & A-150 & \underline{7.6} & \underline{7.1} & \color{gray}{40.3} & \underline{39.7} & \underline{80.9} & \underline{61.1}\\
        \hlrow \ours (ours) & A-150 & \textbf{14.4} & \textbf{16.2} & \color{gray}{47.7} & \textbf{49.9} & \textbf{91.1} & \textbf{73.4} \\
        \midrule
        ZegFormer & PC-59 & \underline{3.8} & \underline{8.2} & \underline{13.1} & \color{gray}{48.7} & 86.5 & 66.8 \\
        ZSseg & PC-59 & 3.0 & 7.6 & 11.9 & \color{gray}{54.7} & \underline{87.7} & \underline{71.7}\\
        \hlrow \ours (ours) & PC-59 & \textbf{9.6} & \textbf{16.7} & \textbf{27.4} & \color{gray}{63.7} & \textbf{93.5} & \textbf{79.9} \\
        \bottomrule
    \end{tabular}
    }

    
    \vspace{-5pt}
    \caption{\textbf{Training on various datasets.} CLIP with ViT-B is used for all methods. Our model demonstrates remarkable generalization capabilities even on relatively smaller datasets. The scores evaluated on the same dataset used for training are colored in \textcolor{gray}{gray}.}
    \vspace{-10pt}
    \label{tab:cross-dataset-ablation}
\end{table}

\begin{table}[t!]
    \centering
    \caption{
    \textbf{Efficiency comparison of optimization algorithms.}
    R@1 scores evaluated on MSRVTT-7k for video retrieval are recorded.
    Multi-task learning simultaneously trains all tasks with even loss weights. 
    CG and FP are abbreviations of conjugate gradient and fixed-point optimization. 
    In terms of time costs, average training time per epoch is reported. 
    $^\dagger$ refers to our optimization algorithm which approximates $\nabla^2_w \aux$ as the identity matrix $\mathrm{I}$.}
    \begin{adjustbox}{width=\linewidth}
    \begin{tabular}{l |c| c  c}
        \toprule
        \textbf{Method}  & \textbf{Opt. Scheme}  & \textbf{R@1} &  \textbf{Time} \\
        \midrule
        \midrule
        Multi-task Learning   & 
        - &  
        26.1 \scriptsize(+0.0)    & 
        547 \scriptsize(+0.0\%) \\
        
        \textbf{MELTR} + Meta-Weight Net~\cite{shu2019meta}  & 
        ITD &  
        27.3 \scriptsize(\textcolor{red}{+1.2})  & 
        1,296 \scriptsize(\textcolor{red}{+136.9\%}) \\ 
        
        \textbf{MELTR} + StocBIO~\cite{ji2021bilevel} & 
        N/A  &  
        26.8 \scriptsize(\textcolor{red}{+0.7})   &   
        686 \scriptsize(\textcolor{red}{+25.4\%})\\
        
        \textbf{MELTR} + CG & 
        AID-CG &  
        28.0 \scriptsize(\textcolor{red}{+1.9})   &   
        624 \scriptsize(\textcolor{red}{+14.1\%})\\
        
        \textbf{MELTR} + AuxiLearn~\cite{navon2020auxiliary} &  
        AID-FP    &  
        27.9 \scriptsize(\textcolor{red}{+1.8})    &
        638 \scriptsize(\textcolor{red}{+16.6\%})      \\
        
        \textbf{MELTR} + \textbf{AID-FP-Lite}$^\dagger$ & 
        AID-FP &  
        28.5 \scriptsize(\textcolor{red}{+2.4})   &   
        574 \scriptsize(\textcolor{red}{+4.9\%})\\
        \bottomrule
    \end{tabular}
    \end{adjustbox}
    \label{tab:efficiency}
    \vspace{-3mm}
\end{table}

\vspace{-10pt}
\paragraph{Efficiency comparison.}
In Table~\ref{tab:efficiency}, we thoroughly compare the efficiency of our method to recent methods~\cite{ding2022decoupling,xu2022simple,liang2022open}. We measure the number of learnable parameters, the total number of parameters, training time, inference time, and inference GFLOPs. Our model demonstrates strong efficiency in terms of both training and inference. This efficiency is achieved because our framework does not require an additional mask generator~\cite{ding2022decoupling}.


%------------------------------------------------------------------------
% Results
\section{Results}
\label{results}

\subsection{Image Segmentation Using Text Prompts}
\begin{table}[t]
    \centering
    \begin{adjustbox}{width=0.9\columnwidth}
    \begin{tabular}{|c||c|c|c|c|}
    \hline
    Method & mIoU & $IoU_{FG}$ & AP \\
    \hline
        \hline
        MDETR \cite{kamath2021mdetr} & 53.7 & - & - \\
        GLIPv2-T \cite{zhang2022glipv2} & 59.4 & - & - \\
        \hline\hline
        RMI \cite{wu2020phrasecut} & 21.1 & 42.5 & - \\
        Mask-RCNN Top \cite{wu2020phrasecut} & 39.4 & 47.4 & -\\
        HulaNet \cite{wu2020phrasecut} & 41.3 & 50.8 & - \\
        CLIPSeg (PC+)  \cite{luddecke2022image}& 43.4 & 54.7 & 76.7\\
        CLIPSeg (PC, D=128) \cite{luddecke2022image} & 48.2 & 56.5 & 78.2\\
        \hline
        RGBNet & 46.7 & 56.2 & 77.2 \\
        \rowcolor{lightgray} ZNet (Ours) & 51.3 & 59.0 & 78.7 \\
        \rowcolor{lightgray} LD-ZNet (Ours) & \textbf{52.7} & \textbf{60.0} & \textbf{78.9} \\
        % \rowcolor{lightgray} CLIP Image features only &  & 49.04 & 57.76 & 77.51 \\
        % \rowcolor{lightgray} Z + CLIP &  & 49.8 & 58.8 & 79.8 \\
        % \rowcolor{lightgray} Z + CLIP Img + LDM features&  &  &  &  \\
    \hline
    \end{tabular}
    \end{adjustbox}
    \caption{Text-based image segmentation performance on the PhraseCut testset. The performance of ZNet and LD-ZNet is highlighted in gray. Both these models outperform the baseline RGBNet on all the metrics.}
    \vspace{-0.5em}
    \label{tab:ris_results}
\end{table}
\begin{figure}[t]
    \centering
    \begin{subfigure}[t]{0.19\columnwidth}
        \centering
        % \includegraphics[width=\linewidth]{Images/visual/00047_47_a_bad_photo_of_a_white_stand._image.png}
        % \includegraphics[width=0.95\linewidth]{Images/visual/00695_16_a_photo_of_the_oblong_pastry._image.png}
        % \includegraphics[width=0.95\linewidth]{Images/visual/01364_0_a_bad_photo_of_a_license_plate._image.png}
        % \includegraphics[width=0.95\linewidth]{Images/visual/04736_43_a_photograph_of_a_riding_person._image.png}
        \includegraphics[width=\linewidth]{Images/visual/00893_36_hanging_clock._image.png}
        \includegraphics[width=\linewidth]{Images/visual/03209_0_a_photo_of_a_castle._image.png}
        \caption{Input}
        % \label{fig:architecture}
    \end{subfigure}%
    \hspace{0.05cm}%
    \begin{subfigure}[t]{0.19\columnwidth}
        \centering
        % \includegraphics[width=\linewidth]{Images/visual/00047_47_a_bad_photo_of_a_white_stand._gt.png}
        % \includegraphics[width=0.95\linewidth]{Images/visual/00695_16_a_photo_of_the_oblong_pastry._gt.png}
        % \includegraphics[width=0.95\linewidth]{Images/visual/01364_0_a_bad_photo_of_a_license_plate._gt.png}
        % \includegraphics[width=0.95\linewidth]{Images/visual/04736_43_a_photograph_of_a_riding_person._gt.png}
        \includegraphics[width=\linewidth]{Images/visual/00893_36_hanging_clock._gt.png}
        \includegraphics[width=\linewidth]{Images/visual/03209_0_a_photo_of_a_castle._gt.png}
        \caption{GT mask}
        % \label{fig:architecture}
    \end{subfigure}%
    \hspace{0.05cm}%
    \begin{subfigure}[t]{0.19\columnwidth}
        \centering
        % \includegraphics[width=\linewidth]{Images/visual/00047_47_a_bad_photo_of_a_white_stand..png}
        % \includegraphics[width=0.95\linewidth]{Images/visual/00695_16_a_photo_of_the_oblong_pastry..png}
        % \includegraphics[width=0.95\linewidth]{Images/visual/01364_0_a_bad_photo_of_a_license_plate..png}
        % \includegraphics[width=0.95\linewidth]{Images/visual/04736_43_a_photograph_of_a_riding_person..png}
        \includegraphics[width=\linewidth]{Images/visual/00893_36_hanging_clock..png}
        \includegraphics[width=\linewidth]{Images/visual/03209_0_a_photo_of_a_castle..png}
        \caption{RGBNet}
        % \label{fig:architecture}
    \end{subfigure}%
    \hspace{0.05cm}%
    \begin{subfigure}[t]{0.19\columnwidth}
        \centering
        % \includegraphics[width=\linewidth]{Images/visual/00047_64_a_photo_of_a_white_stand..png}
        % \includegraphics[width=0.95\linewidth]{Images/visual/00695_49_an_image_of_a_oblong_pastry..png}
        % \includegraphics[width=0.95\linewidth]{Images/visual/01364_22_an_image_of_a_license_plate..png}
        % \includegraphics[width=0.95\linewidth]{Images/visual/04736_67_riding_person..png}
        \includegraphics[width=\linewidth]{Images/visual/00893_47_an_image_of_a_hanging_clock..png}
        \includegraphics[width=\linewidth]{Images/visual/03209_66_a_bad_photo_of_a_castle..png}
        \caption{ZNet}
        % \label{fig:architecture}
    \end{subfigure}%
    \hspace{0.05cm}%
    \begin{subfigure}[t]{0.19\columnwidth}
        \centering
        % \includegraphics[width=\linewidth]{Images/visual/00047_85_a_photo_of_one_white_stand..png}
        % \includegraphics[width=0.95\linewidth]{Images/visual/00695_96_a_good_photo_of_a_oblong_pastry..png}
        % \includegraphics[width=0.95\linewidth]{Images/visual/01364_77_a_good_photo_of_a_license_plate..png}
        % \includegraphics[width=0.95\linewidth]{Images/visual/04736_85_a_good_photo_of_a_riding_person..png}
        \includegraphics[width=\linewidth]{Images/visual/00893_89_an_image_of_a_hanging_clock..png}
        \includegraphics[width=\linewidth]{Images/visual/03209_86_a_good_photo_of_a_castle..png}
        \caption{LD-ZNet}
        % \label{fig:architecture}
    \end{subfigure}
    
    \caption{Qualitative comparison on the PhraseCut test set. Each row contains an input image with a text prompt as an input, with the goal being to segment the image regions corresponding to the reference text. The text prompts are \emph{``hanging clock"} and \emph{``castle"} for the top and bottom rows. We show improvements using ZNet and LD-ZNet compared to the RGBNet.}
    \vspace{-1em}
    \label{fig:visual_results}
\end{figure}

% \textcolor{red}{define each approach z baseline, RGB baseline etc. separately, so that the text reads smoother}


On the PhraseCut dataset, we compare the performance of previous approaches with our ZNet and LD-ZNet for the text-based image segmentation task (Table \ref{tab:ris_results}). In order to showcase the performance improvement of our proposed networks, we create a baseline named RGBNet with the same architecture as ZNet except we use the original images as the input instead of its latent space $z$. For RGBNet, we use additional learnable convolutional layers to map the original image to match the input resolution of ZNet. From Table \ref{tab:ris_results}, we observe that our ZNet and LD-ZNet significantly outperform RGBNet. Specifically, the performance improvement from using the latent representation $z$ over the original images is clear (i.e. ZNet vs RGBNet baseline). Performance further improves upon incorporating the LDM visual-linguistic representations (LD-ZNet) - by 6\% overall on the $mIoU$ metric compared to RGBNet. We also highlight this qualitatively in Figure~\ref{fig:visual_results}. In the figure, we show the original image and the GT mask along with outputs from the RGBNet baseline followed by ZNet and LD-ZNet, where both ZNet and LD-ZNet help improve results consistently. For example in the top row, RGBNet detects light fixtures for the ``hanging clock" prompt, and although ZNet does not have as strong activations for these incorrect detections, it is LD-ZNet that correctly segments the ``clock". Similarly in the bottom row, while RGBNet completely got the ``castle" wrong, ZNet correctly has activations on the right buildings, but with lower confidence. However, LD-ZNet improves it further. 

We outperform in all the metrics when compared to previous works, other than MDETR \cite{kamath2021mdetr} and GLIPv2 \cite{zhang2022glipv2}. Notably, these works are pre-trained on detection and phrase grounding for predicting bounding boxes on huge corpus of text-image pairs across various publicly available datasets with bounding box annotations and are later fine-tuned on the Phrasecut dataset for the segmentation task. However, our work is orthogonally focused towards exploring and utilizing LDMs and its internal features for improving the text-based segmentation performance. Note that object detection datasets  have a good overlap with the visual content in PhraseCut, however, they are not representative of the diversity in images available on the internet. For example, while they could learn common concepts like sky, ocean, chair, table and their synonyms, methods like MDETR would not understand concepts like Mikey Mouse, Pikachu etc., which we will show in Section~\ref{discussion}.

\subsection{Generalization to AI Generated Images}

% \begin{figure}
%     \centering
%     \includegraphics[width=0.9\columnwidth]{Images/AI_generated_stats}
%     \caption{Generalization of the proposed LD-ZNet on our AI-generated dataset when compared with other state-of-the-art text-based segmentation methods - CLIPSeg (PC+) and MDETR.}
%     \vspace{-1em}
%     \label{fig:ai_generated_chart}
% \end{figure} 

\begin{table}[t]
    \centering
    \begin{tabular}{|c||c|c|}
    \hline
         Method & mIoU & AP\\
         \hline
         MDETR \cite{kamath2021mdetr} & 53.4 & 63.8 \\
         CLIPSeg (PC+) \cite{luddecke2022image} & 56.4 & 79.0 \\
         \hline
         RGBNet & 63.4 & 84.1\\
         \rowcolor{lightgray} ZNet (Ours) & 68.4 & 85.0\\
         \rowcolor{lightgray} LD-ZNet (Ours) & \textbf{74.1} & \textbf{89.6}\\
    \end{tabular}
    \caption{Generalization of the proposed LD-ZNet on our AIGI dataset when compared with other state-of-the-art text-based segmentation methods.}
    \vspace{-1em}
    \label{tab:ai_generated_chart}
\end{table}

\begin{figure} [!t]
    \centering
    \begin{subfigure}[t]{0.24\columnwidth}
        \centering
        \includegraphics[width=\linewidth]{Images/AI_gen_final/Input/Mickey_mouse_as_an_evil_dictator.jpg}
        \includegraphics[width=\linewidth]{Images/AI_gen_final/Input/medieval_goblin_eating_cakes_painted_by_hieronymus.jpg}
        \includegraphics[width=\linewidth]{Images/AI_gen_final/Input/cute_cartoon_cat_eating_ramen_wearing_sunglasses.jpg}
        \includegraphics[width=\linewidth]{Images/AI_gen_final/Input/voxel_art_of_animals_in_a_forest.jpg}
        \caption{Input}
    \end{subfigure}%
    \hspace{0.05cm}%
    \begin{subfigure}[t]{0.24\columnwidth}
        \centering
        \includegraphics[width=\linewidth]{Images/AI_gen_final/MDETR/Mickey_mouse_as_an_evil_dictator_mickey_mouse.png}
        \includegraphics[width=\linewidth]{Images/AI_gen_final/MDETR/medieval_goblin_eating_cakes_painted_by_hieronymus_goblin.png}
        \includegraphics[width=\linewidth]{Images/AI_gen_final/MDETR/cute_cartoon_cat_eating_ramen_wearing_sunglasses_ramen.png}
        \includegraphics[width=\linewidth]{Images/AI_gen_final/MDETR/voxel_art_of_animals_in_a_forest_animals.png}
        \caption{MDETR \cite{kamath2021mdetr}}
    \end{subfigure}%
    \hspace{0.05cm}%
    \begin{subfigure}[t]{0.24\columnwidth}
        \centering
        \includegraphics[width=\linewidth]{Images/AI_gen_final/CLIPSeg/Mickey_mouse_as_an_evil_dictator_mickey_mouse.png}
        \includegraphics[width=\linewidth]{Images/AI_gen_final/CLIPSeg/medieval_goblin_eating_cakes_painted_by_hieronymus_goblin.png}
        \includegraphics[width=\linewidth]{Images/AI_gen_final/CLIPSeg/cute_cartoon_cat_eating_ramen_wearing_sunglasses_ramen.png}
        \includegraphics[width=\linewidth]{Images/AI_gen_final/CLIPSeg/voxel_art_of_animals_in_a_forest_animals.png}
        \caption{CLIPSeg \cite{luddecke2022image}}
    \end{subfigure}%
    \hspace{0.05cm}%
    \begin{subfigure}[t]{0.24\columnwidth}
        \centering
        \includegraphics[width=\linewidth]{Images/AI_gen_final/LD-ZNet/Mickey_mouse_as_an_evil_dictator_mickey_mouse.png}
        \includegraphics[width=\linewidth]{Images/AI_gen_final/LD-ZNet/medieval_goblin_eating_cakes_painted_by_hieronymus_goblin.png}
        \includegraphics[width=\linewidth]{Images/AI_gen_final/LD-ZNet/cute_cartoon_cat_eating_ramen_wearing_sunglasses_ramen.png}
        \includegraphics[width=\linewidth]{Images/AI_gen_final/LD-ZNet/voxel_art_of_animals_in_a_forest_animals.png}
        \caption{LD-ZNet}
    \end{subfigure}
    
    \caption{Qualitative comparison on the AI-generated images for text-based segmentation. The text prompts are \emph{``Mickey mouse"}, \emph{``Goblin"}, \emph{``Ramen"} and \emph{``animals"} respectively.}
    \vspace{-1em}
    \label{fig:ai_generated_qualitative}
\end{figure}

% We also study the generalization ability of our proposed method to AI-generated images
With the growing popularity of AI generated images, text-based image segmentation is extensively being used by content creators in their daily workflows. Many public libraries \footnote{\href{https://github.com/brycedrennan/imaginAIry}{imaginAIry}, \href{https://github.com/AUTOMATIC1111/stable-diffusion-webui}{stable-diffusion-webui}} widely employ methods such as CLIPSeg \cite{luddecke2022image} for performing segmentation in AI-generated images. So we study the generalization ability of our proposed segmentation approach on AI-generated images. To this extent, we first prepare a dataset of 100 AI-generated images from lexica.art and manually annotate them using 214 text-prompts. We name this dataset AIGI and plan to release it for future research. Next, we evaluate our approaches ZNet and LD-ZNet along with our RGBNet baseline and other text-based segmentation methods - CLIPSeg (PC+) \cite{luddecke2022image}, MDETR \cite{kamath2021mdetr}. Glipv2 was not publicly available for us to evaluate at the time of this submission. All these methods are trained on the Phrasecut dataset and we measure the IoU metric as shown in Table \ref{tab:ai_generated_chart}. It can be seen that RGBNet outperforms CLIPSeg and MDETR because its built on the UNet architecture initialized from the LDM weights that contains semantic information for good generalization. Our methods ZNet and LD-ZNet further improve the generalization to these AI-generated images by more than 20\% compared to MDETR. This is largely due to the robust $z$-space of the LDM that resulted from a VQGAN pre-training on a variety of domains like art, cartoons, illustrations \etc. Furthermore, the latent diffusion features that contain useful semantic information for the synthesis task, also help in segmenting the AI-generated images. We show the qualitative comparison of these methods in Figure \ref{fig:ai_generated_qualitative} for four AI-generated images from our dataset. While CLIPSeg can estimate most distinctive regions such as face of the \emph{Mickey mouse} or rough locations of \emph{Goblin}, \emph{Ramen} and \emph{animals}, MDETR incorrectly segments them because these concepts are unknown to it and because of the domain gap between Phrasecut and AIGI images respectively. In both such cases, our proposed LD-ZNet estimates accurate segmentation. More qualitative results for LD-ZNet on images from the AIGI dataset are shown in Figure \ref{fig:ldznet_ai_generated}.


%\begin{figure}
%\centering
%\includegraphics[width=\linewidth]%{Images/Categories_chart_latest}
%\caption{Category-level comparison of RGBNet and LD-ZNet on the text-based image segmentation task on the 25 most frequent classes of the PhraseCut testset. LD-ZNet outperforms RBGNet for all ``object" classes and most ``stuff" classes.}%, based on the mIoU metric.}
%\vspace{-1em}
%\label{fig:category}
%\end{figure}



% \begin{figure}[!t]
%     \captionsetup[subfigure]{labelformat=empty}
%     \centering
%     \begin{subfigure}{0.3\columnwidth}
%         \centering
%         \begin{subfigure}{\columnwidth}
%             \centering
%             \caption{}
%             \includegraphics[width=\columnwidth]{Images/visual/zseg_SD_features/seattle_vangough_a photo of one Ice Cap Mountains._image.png}
%         \end{subfigure}
%         \begin{subfigure}{\columnwidth}
%             \centering
%             \caption{}
%             \includegraphics[width=\columnwidth]{Images/visual/zseg_SD_features/astronaut_Horse._image.png}
%         \end{subfigure}
%         \begin{subfigure}{\columnwidth}
%             \centering
%             \caption{}
%             \includegraphics[width=\columnwidth]{Images/visual/zseg_SD_features/umbrella_art_a photo of one woman._image.png}
%         \end{subfigure}
%         % \caption{Input image}
%         % \label{fig:architecture}
%         \vspace{-1em}
%     \end{subfigure}%
%     \hspace{0.05cm}%
%     \begin{subfigure}{0.3\columnwidth}
%         \begin{subfigure}{\columnwidth}
%             \centering
%             \caption{``Space Needle"}
%             \includegraphics[width=\columnwidth]{Images/visual/zseg_SD_features/seattle_vangough_a photo of one space needle._mask.png}
%         \end{subfigure}
%         \begin{subfigure}{\columnwidth}
%             \centering
%             \caption{``Horse"}
%             \includegraphics[width=\columnwidth]{Images/visual/zseg_SD_features/astronaut_Horse._mask.png}
%         \end{subfigure}
%         \begin{subfigure}{\columnwidth}
%             \centering
%             \caption{``Umbrella"}
%             \includegraphics[width=\columnwidth]{Images/visual/zseg_SD_features/umbrella_art_a photo of one Umbrella._mask.png}
%         \end{subfigure}
%         \vspace{-1em}
%     \end{subfigure}%
%     \hspace{0.05cm}%
%     \begin{subfigure}{0.3\columnwidth}
%         \begin{subfigure}{\columnwidth}
%             \centering
%             \caption{``Ice cap mountains"}
%             \includegraphics[width=\columnwidth]{Images/visual/zseg_SD_features/seattle_vangough_a photo of one Ice Cap Mountains._mask.png}
%         \end{subfigure}
%         \begin{subfigure}{\columnwidth}
%             \centering
%             \caption{``Astronaut"}
%             \includegraphics[width=\columnwidth]{Images/visual/zseg_SD_features/astronaut_Astronaut._mask.png}
%         \end{subfigure}
%         \begin{subfigure}{\columnwidth}
%             \centering
%             \caption{``Woman"}
%             \includegraphics[width=\columnwidth]{Images/visual/zseg_SD_features/umbrella_art_a photo of one woman._mask.png}
%         \end{subfigure}
%         \vspace{-1em}
%     \end{subfigure}
%     \caption{Qualitative results on a diverse variety of images show that LD-ZNet can correctly segment multiple text prompts given the same image. Images used from LDM generations at \href{https://lexica.art/}{lexica.art} (top two rows) and google (bottom row).}
%     \vspace{-1em}
%     \label{fig:visual_results2}
% \end{figure}

\begin{figure*}[!t]
    \captionsetup[subfigure]{labelformat=empty,font=small,labelfont={bf,sf},skip=0pt}
    \centering
    \begin{subfigure}{\columnwidth}
        \centering
        \begin{subfigure}{0.32\textwidth}
            \centering
            \begin{subfigure}{\columnwidth}
                \centering
                \caption{}
                \includegraphics[width=\columnwidth]{Supp_Images/visual/zseg_SD_features/More_analysis/indoor_an_image_of_a_Books._image.png}
                % \smallskip
            \end{subfigure}
            \begin{subfigure}{\columnwidth}
                \centering
                \caption{``Books"}
                \includegraphics[width=\columnwidth]{Supp_Images/visual/zseg_SD_features/More_analysis/indoor_an_image_of_a_Books._mask.png}
            \end{subfigure}
        \end{subfigure}%
        \hspace{0.05cm}%
        \begin{subfigure}{0.32\textwidth}
            \centering
            \begin{subfigure}{\columnwidth}
                \centering
                \caption{``Flowers"}
                \includegraphics[width=\columnwidth]{Supp_Images/visual/zseg_SD_features/More_analysis/indoor_an_image_of_a_Flowers._mask.png}
            \end{subfigure}
            \begin{subfigure}{\columnwidth}
                \centering
                \caption{``Sofa"}
                \includegraphics[width=\columnwidth]{Supp_Images/visual/zseg_SD_features/More_analysis/indoor_an_image_of_a_sofa._mask.png}
            \end{subfigure}
        \end{subfigure}%
        \hspace{0.05cm}%
        \begin{subfigure}{0.32\textwidth}
            \centering
            \begin{subfigure}{\columnwidth}
                \centering
                \caption{``Table"}
                \includegraphics[width=\columnwidth]{Supp_Images/visual/zseg_SD_features/More_analysis/indoor_an_image_of_a_Table._mask.png}
            \end{subfigure}
            \begin{subfigure}{\columnwidth}
                \centering
                \caption{``Trees"}
                \includegraphics[width=\columnwidth]{Supp_Images/visual/zseg_SD_features/More_analysis/indoor_an_image_of_a_Trees._mask.png}
            \end{subfigure}
        \end{subfigure}
    \end{subfigure}%
    \rulesep
    % \noindent\rule{\textwidth}{0.4pt}
    \begin{subfigure}{\columnwidth}
        \centering
        \begin{subfigure}{0.32\textwidth}
            \centering
            \begin{subfigure}{\columnwidth}
                \centering
                \caption{}
                \includegraphics[width=\columnwidth]{Supp_Images/visual/zseg_SD_features/More_analysis/camping_an_image_of_a_Chair._image.png}
            \end{subfigure}
            \begin{subfigure}{\columnwidth}
                \centering
                \caption{``Chair"}
                \includegraphics[width=\columnwidth]{Supp_Images/visual/zseg_SD_features/More_analysis/camping_an_image_of_a_Chair._mask.png}
            \end{subfigure}
        \end{subfigure}%
        \hspace{0.05cm}%
        \begin{subfigure}{0.32\textwidth}
            \centering
            \begin{subfigure}{\columnwidth}
                \centering
                \caption{``Clouds"}
                \includegraphics[width=\columnwidth]{Supp_Images/visual/zseg_SD_features/More_analysis/camping_an_image_of_a_Clouds._mask.png}
            \end{subfigure}
            \begin{subfigure}{\columnwidth}
                \centering
                \caption{``Grass"}
                \includegraphics[width=\columnwidth]{Supp_Images/visual/zseg_SD_features/More_analysis/camping_an_image_of_a_Grass._mask.png}
            \end{subfigure}
        \end{subfigure}%
        \hspace{0.05cm}%
        \begin{subfigure}{0.32\textwidth}
            \centering
            \begin{subfigure}{\columnwidth}
                \centering
                \caption{``Mountains"}
                \includegraphics[width=\columnwidth]{Supp_Images/visual/zseg_SD_features/More_analysis/camping_an_image_of_a_Mountains._mask.png}
            \end{subfigure}
            \begin{subfigure}{\columnwidth}
                \centering
                \caption{``River"}
                \includegraphics[width=\columnwidth]{Supp_Images/visual/zseg_SD_features/More_analysis/camping_an_image_of_a_River._mask.png}
            \end{subfigure}
        \end{subfigure}
    \end{subfigure}
    \caption{LD-ZNet text-based image segmentation results for a real image and a cartoon on diverse set of things and stuff classes. High quality segmentation across multiple classes suggests that LD-ZNet has a good understanding of the overall scene.}
    % Images used from google and \href{https://www.freepik.com}{freepik}.}
    \label{fig:scene_understanding}
\end{figure*}


\begin{table}[t]
    \centering
    \begin{adjustbox}{width=\columnwidth}
    \begin{tabular}{|c||c|c|c|c|c|c|}
    \hline
    \multirow{2}{*}{Method} & \multicolumn{2}{|c|}{RefCOCO} & \multicolumn{2}{|c|}{RefCOCO+} & \multicolumn{2}{|c|}{G-Ref}\\
    \cline{2-7}
    & IoU & AP & IoU & AP & IoU & AP\\
    \hline
            CLIPSeg (PC+) \cite{luddecke2022image}& 30.1 & 14.1 & 30.3 & 15.5 & 33.8 & 23.7 \\
            \hline\hline
            RGBNet & 36.3 & 15.7 & 37.1 & 16.7 & 41.9 & 27.8 \\
            \rowcolor{lightgray}ZNet (Ours) & 40.1 & 16.8 & 40.9 & 17.8 & 47.1 & 29.2 \\
            \rowcolor{lightgray}LD-ZNet (Ours) & \textbf{41.0} & \textbf{17.2} & \textbf{42.5} & \textbf{18.6} & \textbf{47.8} & \textbf{30.8} \\
    \hline
    \end{tabular}
    \end{adjustbox}
    \caption{Generalization of our proposed approaches to different types of expressions from other datasets. Z-Net and LD-ZNet outperform both the RGBNet baseline and CLIPSeg on the generalization across all datasets.}
    \vspace{-0.5em}
    \label{tab:ris_generalization}
\end{table}


\begin{table}
    \centering
    \begin{adjustbox}{width=0.95\columnwidth}
    \begin{tabular}{|c||c|c|c|c|}
    \hline
    Diffusion features via & mIoU & $IoU_{FG}$ & AP \\
    \hline
        LD-ZNet with concatenation & 50.2 & 59.0 & 78.1 \\
        LD-ZNet with cross-attention & \textbf{52.7} & \textbf{60.0} & \textbf{78.9} \\
    \hline
    \end{tabular}
    \end{adjustbox}
    \caption{Incorporating LDM features into ZNet via cross-attention (LD-ZNet) leverages the visual-linguistic information present in them, compared to concatenation, leading to better performance on the text-based image segmentation task.}
    \label{tab:ris_concat_vs_crossattn}
\end{table}

\subsection{Generalization to Referring Expressions}
Reference expression segmentation task is aimed for robot-localization kind of applications, where segmenting at instance-level is performed through distinctive referring expression. Many works such as \cite{yang2022lavt, wang2022cris} also train the text encoder to learn the complex positional references in the text. However, we are focused on generic text-based segmentation that has support for stuff categories as well as for multiple instances. We study the generalization ability of the proposed approach - using LDM features, to this complex task. Specifically, we use the models trained on the PhraseCut dataset and evaluate them on the RefCOCO \cite{kazemzadeh2014referitgame}, RefCOCO+ \cite{kazemzadeh2014referitgame} and G-Ref \cite{nagaraja16refexp} datasets whose complex referring expressions are for single-instance localization and segmentation. We also evaluated the generalization of CLIPSeg (PC+) \cite{luddecke2022image} model that was trained on an extended version of the PhraseCut dataset (PC+), to further demonstrate the generalization capability of our methods. Table \ref{tab:ris_generalization} summarizes the performance for our models along with the RGBNet baseline. We observe a similar trend in performance improvements across RGBNet $< Z$Net $<$ LD-ZNet. These experiments demonstrate that the LDM features enhance the generalization power of the LD-ZNet model even on complex referring expressions.



\begin{figure}[!t]
    \captionsetup[subfigure]{labelformat=empty,font=small,labelfont={bf,sf},skip=0pt}
    \centering
    \begin{subfigure}{0.24\columnwidth}
        \centering
        \begin{subfigure}{\columnwidth}
            \centering
            \caption{``Hoodie"}
            \includegraphics[width=\columnwidth]{Images/AI_gen_final/LD-ZNet/plush_rabbit_coding_late_and_sunglasses_hoodie.png}
        % \vspace{0.005cm}
        \end{subfigure}
        \begin{subfigure}{\columnwidth}
            \centering
            \caption{``Spiderman"}
            \includegraphics[width=\columnwidth]{Images/AI_gen_final/LD-ZNet/spider-man_as_a_robot_serving_pizza_spider-man.png}
        \end{subfigure}
    \end{subfigure}
    \hspace{0.05cm}%
    \begin{subfigure}{0.24\columnwidth}
        \centering
        \begin{subfigure}{\columnwidth}
            \centering
            \caption{``Owl"}
            \includegraphics[width=\columnwidth]{Images/AI_gen_final/LD-ZNet/highly_detailed_owl.png}
        % \vspace{0.005cm}
        \end{subfigure}
        \begin{subfigure}{\columnwidth}
            \centering
            \caption{``Trump"}
            \includegraphics[width=\columnwidth]{Images/AI_gen_final/LD-ZNet/Polaroid_portrait_of_Barack_Obama_shaking_hands_wi_Donald_Trump.png}
        \end{subfigure}
    \end{subfigure}%
    \hspace{0.05cm}%
    \begin{subfigure}{0.24\columnwidth}
        \centering
        \begin{subfigure}{\columnwidth}
            \centering
            \caption{``Pikachu"}
            \includegraphics[width=\columnwidth]{Images/AI_gen_final/LD-ZNet/pikachu._comfy_pajamas_pikachu.png}
        % \vspace{0.005cm}
        \end{subfigure}
        \begin{subfigure}{\columnwidth}
            \centering
            \caption{``Joker"}
            \includegraphics[width=\columnwidth]{Images/AI_gen_final/LD-ZNet/the_joker_walking_through_streets_of_new_york_joker.png}
        \end{subfigure}
    \end{subfigure}%
    \hspace{0.05cm}%
    \begin{subfigure}{0.24\columnwidth}
        \centering
        \begin{subfigure}{\columnwidth}
            \centering
            \caption{``Godzilla"}
            \includegraphics[width=\columnwidth]{Images/AI_gen_final/LD-ZNet/godzilla_in_mexico_godzilla.png}
        % \vspace{0.005cm}
        \end{subfigure}
        \begin{subfigure}{\columnwidth}
            \centering
            \caption{``Eiffel"}
            \includegraphics[width=\columnwidth]{Images/AI_gen_final/LD-ZNet/france_en_super_vilain_eiffel_tower.png}
        \end{subfigure}
    \end{subfigure}
    \caption{More qualitative results of LD-ZNet from AIGI dataset.}
    \label{fig:ldznet_ai_generated}
    \vspace{-1em}
\end{figure}


% \begin{figure}
% \centering
% \includegraphics[width=\linewidth]{Images/Categories_chart}
% \caption{Category level comparison of RBGNet and LD-ZNet on the referring image segmentation task on the 25 most frequent classes of the PhraseCut test dataset. LD-ZNet outperforms RBGNet for all ``object" classes and most ``stuff" classes, based on the mIoU metric.}
% \vspace{-1em}
% \label{fig:category}
% \end{figure}

% \subsection{Category level improvements from LDM}
% We study the class-wise performance on the test dataset of PhraseCut for a more fine grained analysis of the quantitative improvement we see in Table~\ref{tab:ris_results}. 
% %We show the advantage of using LDM features by analyzing the category level improvements over predictions from the test dataset of PhraseCut in Figure \ref{fig:category}. 
% This dataset consists of 1180 categories. In order to draw meaningful conclusions, we discard the categories with less than 100 samples, leaving us with 25 categories. We compute the mIoU metric for the predictions averaged over each category and compare the RGBNet with our LD-ZNet model. The results of our analysis is in Figure \ref{fig:category}. 

% Note that the categories on the x-axis are listed in the decreasing order of relative performance gain upon using LD-ZNet. We observe that LD-ZNet has larger performance gains for object classes such as fence, person, sign, table, chair \etc compared to stuff classes such as floor, sky, street \etc. We believe this is because the large scale LDMs are typically good at generating salient objects which are predominantly in the foreground.


% \subsection{Generalization to referring object detection}
% \review{
% We show the use of LDM features for referring object detection (ROD), an instance level visual understanding task, in Table \ref{tab:rod_results_maskrcnn}. Specifically, we use the RGBNet, ZNet and LD-ZNet as the backbones and trained three Mask-RCNN object detection heads and report mAP on the handheld validation set of Phrasecut. We see a significant improvement going from RGBNet $ < Z$Net $<$ LD-ZNet confirming the usefulness of LDM features.
% }
% \begin{table}[!h]
%     \centering
%     \begin{adjustbox}{scale=0.7}
%     \begin{tabular}{|c||c|c|c|c|c|c|}
%     \hline
%     \multirow{Method} & \multicolumn{3}{|c|}{Object Detection} & \multicolumn{3}{|c|}{Instance Segmentation}\\
%     \cline{2-7}
%     & mAP & mAP50 & mAP75 & mAP & mAP50 & mAP75\\
%     \hline
%             RGBNet & 23.9 & 52.3 & 18.9 & 21.2 & 47.7 & 15.8 \\
%             \rowcolor{lightgray}ZNet (Ours) & 25.9 & 55.0 & 21.4 & 23.2 & 51.4 & 18.1 \\
%             \rowcolor{lightgray}$CA$-ZNet (Ours) & \textbf{27.9} & \textbf{58.6} & \textbf{23.0} & \textbf{25.6} & \textbf{55.2} & \textbf{20.7} \\
%     \hline
%     \end{tabular}
%     \end{adjustbox}
%     \caption{ROD performance on the PhraseCut validation dataset.}
%     \vspace{-1em}
%     \label{tab:rod_results_maskrcnn}
% \end{table}



% \begin{figure*}
% \centering
% \includegraphics[width=\linewidth]{Images/Categories_chart}
% \caption{Category level comparison between RBG baseline and our technique of using the LDM latent space intermediate features (Z +LDM features). The metric used for comparison is mIoU and the classes are from the PhraseCut test dataset. The performance of our technique is better for all ``object" classes and most ``stuff" classes.}
% \label{fig:category}
% \end{figure*}

% \subsection{Query segmentation for novel concepts}
% \subsection{Finer masks for Image editing}
%------------------------------------------------------------------------


%------------------------------------------------------------------------
% Ablation studies
\subsection{Cross-attention vs Concat for LDM features}
\label{ablations}


% \begin{figure}
% \centering
% \includegraphics[width=\linewidth]{Images/Categories_chart_latest}
% \caption{Category-level comparison of RGBNet and LD-ZNet on the text-based image segmentation task on the 25 most frequent classes of the PhraseCut testset. LD-ZNet outperforms RBGNet for all ``object" classes and most ``stuff" classes.}%, based on the mIoU metric.}
% \vspace{-1em}
% \label{fig:category}
% \end{figure}

%\textbf{Category level improvements with LDM features}
%We study the class-wise performance on the testset of PhraseCut for a more fine grained analysis of the quantitative improvement we see in Table~\ref{tab:ris_results}. 
%We show the advantage of using LDM features by analyzing the category level improvements over predictions from the test dataset of PhraseCut in Figure \ref{fig:category}. 
%This dataset consists of 1180 categories. In order to draw meaningful conclusions, we discard the categories with less than 100 samples, leaving us with 25 categories. We compute the mIoU metric for the predictions averaged over each category and compare the RGBNet baseline with our LD-ZNet model. The results of our analysis is in Figure \ref{fig:category}. 
%Note that the categories on the x-axis are listed in the decreasing order of relative performance gain upon using LD-ZNet. We observe that LD-ZNet has larger performance gains for \emph{object classes} such as fence, person, sign, table, chair \etc compared to \emph{stuff classes} such as floor, sky, street \etc. We believe this is because the large scale LDMs are typically good at generating salient objects which are predominantly in the foreground.

% We showcase the importance of each of the components in the proposed ZSEG: Latent diffusion features, CLIP Image features.
%\subsection{Z space vs RGB Image space}

%\subsection{Limitations of LDM on negative samples}
%In continuation to the proposed models from Table \ref{tab:ris_results}, we also trained two separate models that additionally includes negative samples constructed from the PhraseCut dataset. We do this following \cite{luddecke2022image} with a probability of $q_{neg}=0.2$ where no object from the image matches the prompt and the corresponding ground-truth masks are all zeros. Specifically, the sample’s phrase is replaced by a different phrase randomly chosen from the dataset. We report the test performance of these models on PhraseCut test dataset at Table \ref{tab:ris_negative_samples} and we observe that the gain in incorporating LDM features is not as much as we observed in Table \ref{tab:ris_results}. We believe this is because LDM was trained on positive examples only in the first place, as is the case with the training of any diffusion model. Thus when a negative sample appears in the training data, there is little help from the LDM features, yielding diminished gains overall.
%\begin{table}
%    \centering
%    \begin{adjustbox}{width=\columnwidth}
%    \begin{tabular}{|c||c|c|c|c|}
%    \hline
%    Method & mIoU & $IoU_{FG}$ & AP \\
%    \hline
%        Z (+ neg)& 50.24 & 58.54 & 79.11 \\
%        Z + LDM features (+ neg)& \textbf{51.47} & \textbf{59.29} & \textbf{79.6} \\
%    \hline
%    \end{tabular}
%    \end{adjustbox}
%    \caption{Having negative samples in the dataset results in diminished gains with LDM features}
    %\label{tab:ris_negative_samples}
%\end{table}

In LD-ZNet, we inject LDM features into the ZNet model using cross-attention (Figure \ref{fig:spatial-attention}). In order to understand the importance of the cross-attention layer, we also train and evaluate another model where the LDM features are concatenated with the features of the ZNet right before the spatial-attention layer. The results are summarized in Table \ref{tab:ris_concat_vs_crossattn} and it shows that concatenating the LDM features yields inferior results compared to the proposed method. This is because of the \emph{attention pool} layer which serves as a learnable layer and also encodes positional information into the LDM features for setting up the cross-attention. Moreover, the cross-attention layer learns how feature pixels from the ZNet attend to feature pixels from the LDM, thereby leveraging context and correlations from the entire image. With concatenation however, we only fuse the corresponding features of LDM and ZNet which is sub-optimal. 

% \begin{table}
%     \centering
%     \begin{adjustbox}{width=\columnwidth}
%     \begin{tabular}{|c||c|c|c|c|}
%     \hline
%     Diffusion features via & mIoU & $IoU_{FG}$ & AP \\
%     \hline
%         Concatenation & 50.22 & 58.04 & 78.19 \\
%         Cross-attention & \textbf{52.7} & \textbf{59.12} & \textbf{78.93} \\
%     \hline
%     \end{tabular}
%     \end{adjustbox}
%     \caption{Cross-attention of the diffusion features into the ZNet results in better usage of visual-linguistic information compared to concatenation}
%     \label{tab:ris_concat_vs_crossattn}
% \end{table}

%------------------------------------------------------------------------


% Discussion
\section{Summary and Discussion}
\label{sec:Discussion}

In this work, we have extended the proposal of the Love symmetry resolution of the seemingly unnatural values of the black hole Love numbers~\cite{Charalambous:2021kcz,Charalambous:2022rre} to higher-dimensional rotating black holes in General Relativity. Namely, we have explored in full the case of static scalar responses of the $5$-dimensional Myers-Perry black hole.

Compared to the examples of Kerr-Newman black holes in $d=4$ spacetime dimensions~\cite{Charalambous:2021mea} and Schwarzschild black holes in $d=5$ spacetime dimensions~\cite{Kol:2011vg}, we find some interesting exact results. To start with, static scalar Love numbers do not in general vanish in $d=5$ for generic spin parameters, not even when $\hat{\ell}\in\mathbb{N}$, in contrast to the Schwarzschild case~\cite{Kol:2011vg}. Beyond vanishing for ``axisymmetric'' perturbations~\cite{Landry:2015zfa,Pani:2015hfa,Gurlebeck:2015xpa}, we also find that the static Love numbers vanish
for equi-rotating black holes,
which does not have a counterpart in $d=4$. 
We remark here that the current results can be straightforwardly extended to include the case of $5$-d electrically charged Myers-Perry black holes, mainly due to the fact that the discriminant function remains a quadratic polynomial in $\rho$~\cite{Chong:2005hr,Chong:2006zx}. Scalar Love numbers for $5$-d charged Myers-Perry black holes were also considered in~\cite{Consoli:2022eey}, who however focused on their slowly-rotating limits thus missing the classical RG flow feature, which we study in detail here.

It appears that the vanishing of static Love numbers for rotating black holes in $d=4$ is an exception rather than the norm. Indeed, as we have demonstrated in this work, Love numbers for rotating black holes in $d=5$ are in general non-zero and exhibit running, in agreement with
Wilsonian 
naturalness arguments. Regardless, we were still able to find near-zone truncations acquiring $\SL$ Love symmetries just like in $d=4$ Kerr-Newman black holes and $d\ge4$ Reissner-Nordstr\"{o}m black holes~\cite{Bertini:2011ga,Kim:2012mh,Charalambous:2021kcz,Charalambous:2022rre}. In the special situations where Love numbers do vanish, however, it is the highest-property of the corresponding Love symmetry that outputs this vanishing as a selection rule. We see therefore that the existence of near-zone Love symmetries appears to be routed in black holes in General Relativity, rather than only with background geometries and perturbations with vanishing Love numbers.

At the same time, we have demonstrated here that the highest-weight representation of the near-zone $\SL$'s, along with its full extension into the representation of type ``$\circ[\circ[\circ$'', plays a special role in the scalar response problem: it is entirely spanned by near-zone solutions with vanishing/non-running Love numbers. These properties are in fact \'{a} posteriori seen to be shared with the Love symmetry presented in~\cite{Charalambous:2021kcz,Charalambous:2022rre} for the $d=4$ Kerr-Newman black hole. These two features, the existence of near-zone $\SL$ symmetries and the vanishing/non-running of Love numbers, appear therefore to be mutually inclusive, with the solutions of vanishing Love numbers furnishing a quotient representation of the highest-weight Verma module of the near-zone $\SL$. We remind here, though, that only the static results can be trusted within the near-zone regime.

On that account, it is interesting to further study this hypothesis. On the one hand, it is instructive to extend the analysis to other general-relativistic black holes. The obvious next candidate to analyze is the higher-dimensional Myers-Perry black holes whose scalar field perturbations are still separable~\cite{Frolov:2006pe,Frolov:2008jr}. A technical obstacle in this approach, however, is the fact that the angular eigenvalues in $d>5$ are not known in closed form, but can be obtained as an expansion in spin parameters ratios, see e.g.~\cite{Cho:2011yp}. It would be interesting, in particular, to analyze the fate of scalar Love numbers for equi-rotating Myers-Perry black holes in odd spacetime dimensions which have the enhanced isometry subgroup $U\left(1\right)^{N}\rightarrow U\left(N\right)$. Moreover, it only deems appropriate to extend to higher-spin fields, namely, electromagnetic and gravitational perturbations. At least for spin-$1$ perturbations, this should be very similar to the work done here thanks to the separability of electromagnetic perturbations in the background of Myers-Perry black holes~\cite{Lunin:2017drx}.

On the other hand, it is still an open question whether Love symmetry exists in theories of gravity beyond General Relativity. A preliminary analysis around this was done in~\cite{Charalambous:2022rre}, where a sufficient geometric condition was extracted for spherically symmetric black holes. It would be interesting to supplement that analysis with sufficient \text{and} necessary constraints, investigate what type of theories of gravity support such geometries and whether the corresponding Love symmetries live up to their names, i.e. whether they can address the potential vanishing of Love numbers. As a counterexample, it was shown in~\cite{Charalambous:2022rre} that Love symmetry does not exist for the case of Riemann-cubed modifications of general relativity, see also~\cite{Cai:2019npx,Cardoso:2018ptl}. This nicely fitted with the corresponding computation of static scalar Love numbers which were found to be non-zero and exhibit the expected RG flow.

The nature of the Love symmetry is still not fully understood. 
From arguments stated here and in~\cite{Charalambous:2022rre}, 
the approximate Love symmetry for non-extremal black holes can be 
interpreted to be a remnant of enhanced isometries of extremal black holes. 
This could be further supported by studying the perturbations of black objects with non-spherical horizons, namely, black $p$-branes~\cite{Duff:1993ye}. One would then attempt to identify near-zone truncations admitting an $SO\left(p+1,2\right)$ symmetry. We leave this for future work. If such an analysis turns out to yield affirmative results, then the Love symmetry may shine more light on the potential holographic descriptions of asymptotically flat general-relativistic black holes~\cite{Bardeen:1999px,Guica:2008mu,Castro:2010fd,Lu:2008jk}.

There are some unconventional features of the Love symmetries regarding their property to offer IR selection rules. In particular, they have the feature of mixing IR and UV modes as can be seen from the fact that representations of the near-zone $\SL$ have non-zero frequencies and, thus, they are not directly manifested at the level of the worldline EFT. For the highest-weight representation, the corresponding frequencies have the same form as the ``near-horizon'' modes presented in~\cite{Zimmerman:2011dx}. More interestingly, the purely imaginary spacing is precisely equal to the universal QNMs level spacing as extracted from Padmanabhan's argument~\cite{Padmanabhan:2003fx}. Another possible connection to the QNM spectrum has been suggested in~\cite{Charalambous:2022rre}, where the complex frequencies of highest-weight elements were contrasted to highly dumped QNMs and total transmission modes~\cite{Cook:2016fge,Cook:2016ngj}. A more direct way to reveal such ``beyond-near-zone'' connections would be to identify symmetry-breaking parameters that depart from Love symmetric near-zone configurations and apply a spurion analysis to extract relations similar to the Gell-Mann-Okubo mass formulas~\cite{Gell-Mann:1961omu,Okubo:1961jc}.
A potential approach along these lines would be to identify the effective black hole geometries, for which Love symmetries are isometries, as scaling limits of the full asymptotically flat Myers-Perry black hole solution, see e.g.~\cite{Cvetic:2012tr}.

\paragraph{Acknowledgments}
We are grateful to Sergei Dubovsky for many 
insightful comments on the draft of this paper. 
We also thank Barak Kol 
and Zihan Zhou for useful discussions. PC is partially supported by the
2022-2023 Dean's Dissertation Fellowship.

\section{Conclusion}
\label{conclusion}

We presented a novel approach for text-based image segmentation using large scale latent diffusion models. By training the segmentation models on the latent z-space, we were able to improve the generalization of segmentation models to new domains, like AI generated images. We also showed that this z-space is a better representation for text-to-image tasks in natural images. By utilizing the internal features of the LDM at appropriate time-steps, we were able to tap into the semantic information hidden inside the image synthesis pipeline using a cross-attention mechanism, which further improved the segmentation performance both on natural and AI generated images. This was experimentally validated on several publicly available datasets and on a new dataset of AI generated images, which we will make publicly available. 

% \section{Final copy}

% You must include your signed IEEE copyright release form when you submit
% your finished paper. We MUST have this form before your paper can be
% published in the proceedings.

{\small
\bibliographystyle{ieee_fullname}
\bibliography{ms}
}

% %%%%%%%%%% Merge with supplemental materials %%%%%%%%%%
% Supplementary
% \twocolumngrid
\clearpage
% \twocolumngrid
% \begin{center}
% \large\textbf{Supplementary Material}
% \end{center}
\twocolumn[{%
 \centering
 \LARGE \textbf{Supplementary Material}\\[1em]
 % \large Author: Anton van der Vegt\\[1em]
}]
%%%%%%%%%% Merge with supplemental materials %%%%%%%%%%
%%%%%%%%%% Prefix a "S" to all equations, figures, tables and reset the counter %%%%%%%%%%

\setcounter{equation}{0}
\setcounter{figure}{0}
\setcounter{table}{0}
\setcounter{page}{1}
\setcounter{section}{0}
\makeatletter
\renewcommand{\theequation}{S\arabic{equation}}
\renewcommand{\thefigure}{S\arabic{figure}}
% \renewcommand{\bibnumfmt}[1]{[S#1]}
% \renewcommand{\citenumfont}[1]{S#1}
\renewcommand{\thesection}{S\arabic{section}}
%%%%%%%%%% Prefix a "S" to all equations, figures, tables and reset the counter %%%%%%%%%%

\begin{figure}[!h]
    \centering
    \begin{subfigure}[t]{0.24\columnwidth}
        \centering
        \includegraphics[width=\linewidth]{Images/AI_gen_final/Input/spider-man_as_a_robot_serving_pizza.jpg}
        \includegraphics[width=\linewidth]{Images/AI_gen_final/Input/A_road-runner_riding_a_tortoise_in_Joshua_Tree.jpg}
        \includegraphics[width=\linewidth]{Images/AI_gen_final/Input/A_steampunk_fox_fursona_with_boots_sitting_on_a_ve.jpg}
        \includegraphics[width=\linewidth]{Images/AI_gen_final/Input/A_cute_robot_artist_painting_on_an_easel._Concept.jpg}
        % \includegraphics[width=\linewidth]{Images/AI_gen_final/Input/generate an image of a female hacker in a bra at a.jpg}
        \caption{Input}
    \end{subfigure}%
    \hspace{0.05cm}%
    \begin{subfigure}[t]{0.24\columnwidth}
        \centering
        \includegraphics[width=\linewidth]{Images/AI_gen_final/MDETR/spider-man_as_a_robot_serving_pizza_spider-man.png}
        \includegraphics[width=\linewidth]{Images/AI_gen_final/MDETR/A_road-runner_riding_a_tortoise_in_Joshua_Tree_tortoise.png}
        \includegraphics[width=\linewidth]{Images/AI_gen_final/MDETR/A_steampunk_fox_fursona_with_boots_sitting_on_a_ve_vespa.png}
        \includegraphics[width=\linewidth]{Images/AI_gen_final/MDETR/A_cute_robot_artist_painting_on_an_easel._Concept__robot.png}
        % \includegraphics[width=\linewidth]{Images/AI_gen_final/MDETR/generate an image of a female hacker in a bra at a_bra.png}
        \caption{MDETR}
    \end{subfigure}%
    \hspace{0.05cm}%
    \begin{subfigure}[t]{0.24\columnwidth}
        \centering
        \includegraphics[width=\linewidth]{Images/AI_gen_final/CLIPSegRefined/spider-man_as_a_robot_serving_pizza_spider-man.png}
        \includegraphics[width=\linewidth]{Images/AI_gen_final/CLIPSegRefined/A_road-runner_riding_a_tortoise_in_Joshua_Tree_tortoise.png}
        \includegraphics[width=\linewidth]{Images/AI_gen_final/CLIPSegRefined/A_steampunk_fox_fursona_with_boots_sitting_on_a_ve_vespa.png}
        \includegraphics[width=\linewidth]{Images/AI_gen_final/CLIPSegRefined/A_cute_robot_artist_painting_on_an_easel._Concept__robot.png}
        % \includegraphics[width=\linewidth]{Images/AI_gen_final/CLIPSegRefined/generate an image of a female hacker in a bra at a_bra.png}
        \caption{CLIPSeg}
    \end{subfigure}%
    \hspace{0.05cm}%
    \begin{subfigure}[t]{0.24\columnwidth}
        \centering
        \includegraphics[width=\linewidth]{Images/AI_gen_final/LD-ZNet/spider-man_as_a_robot_serving_pizza_spider-man.png}
        \includegraphics[width=\linewidth]{Images/AI_gen_final/LD-ZNet/A_road-runner_riding_a_tortoise_in_Joshua_Tree_tortoise.png}
        \includegraphics[width=\linewidth]{Images/AI_gen_final/LD-ZNet/A_steampunk_fox_fursona_with_boots_sitting_on_a_ve_vespa.png}
        \includegraphics[width=\linewidth]{Images/AI_gen_final/LD-ZNet/A_cute_robot_artist_painting_on_an_easel._Concept__robot.png}
        % \includegraphics[width=\linewidth]{Images/AI_gen_final/LD-ZNet/generate an image of a female hacker in a bra at a_bra.png}
        \caption{LD-ZNet}
    \end{subfigure}
    
    \caption{Qualitative comparison on the AI-generated images from AIGI dataset for text-based segmentation. The text prompts are \emph{``Spiderman"}, \emph{``tortoise"}, \emph{``vespa"} and \emph{``robot"} respectively.}
    \vspace{-1em}
    \label{fig:ai_generated_qualitative_supplementary}
\end{figure}

\section{Text-Based Image Segmentation}
\label{supplementary}
In this supplementary work, we illustrate some more qualitative text-based image segmentation results using the proposed LD-$Z$Net model on a diverse set of images. Specifically, we focus on segmenting and localizing 1) objects in AI-generated images from the AIGI dataset 2) objects described by their attributes and 3) multiple different things and stuff in a scene. We also perform a visual comparison with the RGBNet baseline on a diverse set of images, including examples from the PhraseCut test dataset.


\subsection{AI-Generated Images}

% On our AIGI dataset, we also evaluate the newly released CLIPSeg model with fine-grained predictions and found it to be 3.5\% better (mIoU=59.9) than the originally released CLIPSeg version. Furthermore, 
We show more qualitative comparisons on our AIGI dataset in Figure \ref{fig:ai_generated_qualitative_supplementary}. We observe similar trend. While MDETR fails to segment the text prompts \emph{``Spiderman"}, \emph{``tortoise"}, \emph{``vespa"} and \emph{``robot"} due to novel concepts and domain gap, CLIPSeg estimates a rough segmentation on the most discriminative regions with lower confidence. However, LD-ZNet accurately segments in all the cases.

\subsection{Attributes}

\begin{figure}
    \centering
    \captionsetup[subfigure]{labelformat=empty,font=small,labelfont={bf,sf},skip=0pt}
    \begin{subfigure}{\columnwidth}
        \begin{subfigure}{0.49\columnwidth}
            \centering
            \caption{``Blue car"}
            \includegraphics[width=\columnwidth]{Supp_Images/visual/zseg_SD_features/cars_blue_car._mask.png}
        \end{subfigure}%
        \hspace{0.05cm}%
        \begin{subfigure}{0.49\columnwidth}
            \centering
            \caption{``Yellow car"}
            \includegraphics[width=\columnwidth]{Supp_Images/visual/zseg_SD_features/cars_yellow_car._mask.png}
        \end{subfigure}%
    \end{subfigure}
    % \noindent\rule{\columnwidth}{0.4pt}
    \begin{subfigure}{\columnwidth}
        \begin{subfigure}{0.49\columnwidth}
            \centering
            \caption{``Tallest person"}
            \includegraphics[width=\columnwidth]{Supp_Images/visual/zseg_SD_features/heights_a_photo_of_a_Tallest_person._mask_resized.png}
        \end{subfigure}%
        \hspace{0.05cm}%
        \begin{subfigure}{0.49\columnwidth}
            \centering
            \caption{``Player swinging the bat"}
            \includegraphics[width=\columnwidth]{Supp_Images/visual/zseg_SD_features/baseball_an_image_of_a_Player_swinging_the_bat._mask_resized.png}
        \end{subfigure}%
    \end{subfigure}
    \caption{Qualitative results showing LD-$Z$Net correctly segmenting attribute based queries. Text prompts for objects based on color attribute (top row), relative property and action attributes (bottom row) are well segmented by LD-$Z$Net.}
    \vspace{-1em}
    \label{fig:visual_results_supp}
\end{figure}

Figure \ref{fig:visual_results_supp} depicts attribute-based segmentation. Specifically, objects described by attributes based on color or relative properties (such as height) or actions are well segmented by LD-ZNet.

\subsection{Qualitative Comparisons on Diverse Domains}

Figure \ref{fig:visual_results3_supp} demonstrates some specific cases where RGBNet fails to segment or poorly segments the object being referred to, where as LD-ZNet segments the objects better.

\begin{figure*}[!t]
    \captionsetup[subfigure]{labelformat=empty,skip=0pt}
    \centering
    \begin{subfigure}{0.85\linewidth}
        \centering
        \begin{subfigure}{0.32\columnwidth}
            \centering
            \caption{}
            \includegraphics[width=\columnwidth]{Images/visual/zseg_SD_features/coco_a_photo_of_one_Guitar._image.png}
            \includegraphics[width=\columnwidth]{Supp_Images/visual/RGBNet/panda_a_photograph_of_a_Panda._image.png}
        \end{subfigure}%
        \hspace{0.05cm}%
        \begin{subfigure}{0.32\columnwidth}
            \centering
            \caption{RGBNet}
            \includegraphics[width=\columnwidth]{Images/visual/RGBNet/coco_a_photograph_of_a_Guitar._mask.png}
            \includegraphics[width=\columnwidth]{Supp_Images/visual/RGBNet/panda_a_photograph_of_a_Panda._mask.png}
        \end{subfigure}%
        \hspace{0.05cm}%
        \begin{subfigure}{0.32\columnwidth}
            \centering
            \caption{LD-ZNet}
            \includegraphics[width=\columnwidth]{Images/visual/zseg_SD_features/coco_a_photo_of_one_Guitar._mask.png}
            \includegraphics[width=\columnwidth]{Supp_Images/visual/zseg_SD_features/panda_a_good_photo_of_a_Panda._mask.png}
        \end{subfigure}
        \vspace{0.75em}
    \end{subfigure}
    \noindent\rule{0.9\linewidth}{0.4pt}
    \begin{subfigure}{0.85\linewidth}
        % \vspace{-0.75em}
        \centering
        \begin{subfigure}{0.32\columnwidth}
            \centering
            \caption{}
            \includegraphics[width=\columnwidth]{Supp_Images/visual/zseg_SD_features/avengers4_Scarlett_Johansson._image.png}
            \includegraphics[width=\columnwidth]{Supp_Images/visual/zseg_SD_features/PRINCE-GEORGE-PRINCESS-CHARLOTTE-PRINCE-LOUIS-3-te-220907-f27ee7_a_photo_of_one_Kate_Middleton._image.png}
        \end{subfigure}%
        \hspace{0.05cm}%
        \begin{subfigure}{0.32\columnwidth}
            \centering
            \caption{}
            \includegraphics[width=\columnwidth]{Supp_Images/visual/RGBNet/avengers4_Scarlett_Johansson._mask.png}
            \includegraphics[width=\columnwidth]{Supp_Images/visual/RGBNet/PRINCE-GEORGE-PRINCESS-CHARLOTTE-PRINCE-LOUIS-3-te-220907-f27ee7_a_photo_of_one_Kate_Middleton._mask.png}
        \end{subfigure}%
        \hspace{0.05cm}%
        \begin{subfigure}{0.32\columnwidth}
            \centering
            \caption{}
            \includegraphics[width=\columnwidth]{Supp_Images/visual/zseg_SD_features/avengers4_Scarlett_Johansson._mask.png}
            \includegraphics[width=\columnwidth]{Supp_Images/visual/zseg_SD_features/PRINCE-GEORGE-PRINCESS-CHARLOTTE-PRINCE-LOUIS-3-te-220907-f27ee7_a_photo_of_one_Kate_Middleton._mask.png}
        \end{subfigure}
        \vspace{0.75em}
    \end{subfigure}
    \noindent\rule{0.9\linewidth}{0.4pt}
    \begin{subfigure}{0.85\linewidth}
        % \vspace{-0.75em}
        \centering
        \begin{subfigure}{0.32\columnwidth}
            \centering
            \caption{}
            \includegraphics[width=\columnwidth]{Images/visual/zseg_SD_features/tired-mom_a_photo_of_a_Table_lamp._image.png}
            \includegraphics[width=\columnwidth]{Supp_Images/visual/RGBNet/camping_a_photograph_of_a_Trees._image.png}
        \end{subfigure}%
        \hspace{0.05cm}%
        \begin{subfigure}{0.32\columnwidth}
            \centering
            \caption{}
            \includegraphics[width=\columnwidth]{Images/visual/RGBNet/tired-mom_a_photograph_of_a_Table_lamp._mask.png}
            \includegraphics[width=\columnwidth]{Supp_Images/visual/RGBNet/camping_a_photograph_of_a_Trees._mask.png}
        \end{subfigure}%
        \hspace{0.05cm}%
        \begin{subfigure}{0.32\columnwidth}
            \centering
            \caption{}
            \includegraphics[width=\columnwidth]{Images/visual/zseg_SD_features/tired-mom_a_photo_of_a_Table_lamp._mask.png}
            \includegraphics[width=\columnwidth]{Supp_Images/visual/zseg_SD_features/camping_an_image_of_a_Trees._mask.png}
        \end{subfigure}
    \end{subfigure}
    \caption{More qualitative examples where RGBNet fails to localize \emph{``Guitar", ``Panda"} from animation images (top row), famous celebrities \emph{``Scarlett Johansson", ``Kate Middleton"} (second row) and objects such as \emph{``Lamp", ``Trees"} from illustrations (bottom row). LD-$Z$Net benefits from using $z$ combined with the internal LDM features to correctly segment these text prompts.}
    \vspace{-1em}
    \label{fig:visual_results3_supp}
\end{figure*}

\subsection{Scene understanding}
\begin{figure*}[p]
    \captionsetup[subfigure]{labelformat=empty,font=small,labelfont={bf,sf},skip=0pt}
    \centering
    \begin{subfigure}{\textwidth}
        \centering
        \begin{subfigure}{0.32\textwidth}
            \centering
            \begin{subfigure}{\columnwidth}
                \centering
                \caption{}
                \includegraphics[width=\columnwidth]{Supp_Images/visual/zseg_SD_features/More_analysis/indoor_an_image_of_a_Books._image.png}
            \end{subfigure}
            \begin{subfigure}{\columnwidth}
                \centering
                \caption{``Books"}
                \includegraphics[width=\columnwidth]{Supp_Images/visual/zseg_SD_features/More_analysis/indoor_an_image_of_a_Books._mask.png}
            \end{subfigure}
        \end{subfigure}%
        \hspace{0.05cm}%
        \begin{subfigure}{0.32\textwidth}
            \centering
            \begin{subfigure}{\columnwidth}
                \centering
                \caption{``Flowers"}
                \includegraphics[width=\columnwidth]{Supp_Images/visual/zseg_SD_features/More_analysis/indoor_an_image_of_a_Flowers._mask.png}
            \end{subfigure}
            \begin{subfigure}{\columnwidth}
                \centering
                \caption{``Sofa"}
                \includegraphics[width=\columnwidth]{Supp_Images/visual/zseg_SD_features/More_analysis/indoor_an_image_of_a_sofa._mask.png}
            \end{subfigure}
        \end{subfigure}%
        \hspace{0.05cm}%
        \begin{subfigure}{0.32\textwidth}
            \centering
            \begin{subfigure}{\columnwidth}
                \centering
                \caption{``Table"}
                \includegraphics[width=\columnwidth]{Supp_Images/visual/zseg_SD_features/More_analysis/indoor_an_image_of_a_Table._mask.png}
            \end{subfigure}
            \begin{subfigure}{\columnwidth}
                \centering
                \caption{``Trees"}
                \includegraphics[width=\columnwidth]{Supp_Images/visual/zseg_SD_features/More_analysis/indoor_an_image_of_a_Trees._mask.png}
            \end{subfigure}
        \end{subfigure}
    \vspace{0.5em}
    \end{subfigure}
    \noindent\rule{\textwidth}{0.4pt}
    \begin{subfigure}{\textwidth}
        \vspace{0.25em}
        \centering
        \begin{subfigure}{0.32\textwidth}
            \centering
            \begin{subfigure}{\columnwidth}
                \centering
                \caption{}
                \includegraphics[width=\columnwidth]{Supp_Images/visual/zseg_SD_features/More_analysis/camping_an_image_of_a_Chair._image.png}
            \end{subfigure}
            \begin{subfigure}{\columnwidth}
                \centering
                \caption{``Chair"}
                \includegraphics[width=\columnwidth]{Supp_Images/visual/zseg_SD_features/More_analysis/camping_an_image_of_a_Chair._mask.png}
            \end{subfigure}
        \end{subfigure}%
        \hspace{0.05cm}%
        \begin{subfigure}{0.32\textwidth}
            \centering
            \begin{subfigure}{\columnwidth}
                \centering
                \caption{``Clouds"}
                \includegraphics[width=\columnwidth]{Supp_Images/visual/zseg_SD_features/More_analysis/camping_an_image_of_a_Clouds._mask.png}
            \end{subfigure}
            \begin{subfigure}{\columnwidth}
                \centering
                \caption{``Grass"}
                \includegraphics[width=\columnwidth]{Supp_Images/visual/zseg_SD_features/More_analysis/camping_an_image_of_a_Grass._mask.png}
            \end{subfigure}
        \end{subfigure}%
        \hspace{0.05cm}%
        \begin{subfigure}{0.32\textwidth}
            \centering
            \begin{subfigure}{\columnwidth}
                \centering
                \caption{``Mountains"}
                \includegraphics[width=\columnwidth]{Supp_Images/visual/zseg_SD_features/More_analysis/camping_an_image_of_a_Mountains._mask.png}
            \end{subfigure}
            \begin{subfigure}{\columnwidth}
                \centering
                \caption{``River"}
                \includegraphics[width=\columnwidth]{Supp_Images/visual/zseg_SD_features/More_analysis/camping_an_image_of_a_River._mask.png}
            \end{subfigure}
        \end{subfigure}
    \vspace{0.5em}
    \end{subfigure}
    \noindent\rule{\textwidth}{0.4pt}
    \begin{subfigure}{\textwidth}
        \vspace{0.25em}
        \centering
        \begin{subfigure}{0.32\textwidth}
            \centering
            \begin{subfigure}{\columnwidth}
                \centering
                \caption{}
                \includegraphics[width=\columnwidth]{Supp_Images/visual/zseg_SD_features/More_analysis/illustration2_an_image_of_a_Bicycle._image.png}
            \end{subfigure}
            \begin{subfigure}{\columnwidth}
                \centering
                \caption{``Buildings"}
                \includegraphics[width=\columnwidth]{Supp_Images/visual/zseg_SD_features/More_analysis/illustration2_an_image_of_a_Buildings._mask.png}
            \end{subfigure}
        \end{subfigure}%
        \hspace{0.05cm}%
        \begin{subfigure}{0.32\textwidth}
            \centering
            \begin{subfigure}{\columnwidth}
                \centering
                \caption{``Trees"}
                \includegraphics[width=\columnwidth]{Supp_Images/visual/zseg_SD_features/More_analysis/illustration2_an_image_of_a_Trees._mask.png}
            \end{subfigure}
            \begin{subfigure}{\columnwidth}
                \centering
                \caption{``Crosswalk"}
                \includegraphics[width=\columnwidth]{Supp_Images/visual/zseg_SD_features/More_analysis/illustration2_an_image_of_a_Crosswalk._mask.png}
            \end{subfigure}
        \end{subfigure}%
        \hspace{0.05cm}%
        \begin{subfigure}{0.32\textwidth}
            \centering
            \begin{subfigure}{\columnwidth}
                \centering
                \caption{``Bicycle"}
                \includegraphics[width=\columnwidth]{Supp_Images/visual/zseg_SD_features/More_analysis/illustration2_an_image_of_a_Bicycle._mask.png}
            \end{subfigure}
            \begin{subfigure}{\columnwidth}
                \centering
                \caption{``Bridge"}
                \includegraphics[width=\columnwidth]{Supp_Images/visual/zseg_SD_features/More_analysis/illustration2_an_image_of_a_Bridge._mask.png}
            \end{subfigure}
        \end{subfigure}
    \end{subfigure}
    \caption{LD-ZNet text-based segmentation results for a diverse set of things and stuff classes across  both real images (top row) and illustrations (middle and bottom rows). High-quality segmentation across multiple object classes suggests that LD-ZNet has a good understanding of the overall scene. Images used from google and \href{https://www.freepik.com}{freepik}.}
    \vspace{-1em}
    \label{fig:scene_understanding_supplementary}
\end{figure*}

Figure \ref{fig:scene_understanding_supplementary} shows the segmentation performance of LD-ZNet for several objects and regions in an image. Specifically, we show the segmentation for stuff classes such as ``Clouds", ``Mountains", ``Chair", ``Grass", ``River" \etc and thing classes such as ``Trees", ``Bicycle", ``Sofa", ``Books" \etc. The quality of the segmentation across multiple object classes suggests that LD-ZNet has a good understanding of the overall scene. 


\begin{figure*}
    \centering
    \begin{subfigure}[t]{0.19\textwidth}
        \centering
        \includegraphics[width=\linewidth]{Supp_Images/00683_25_an_image_of_a_grey_steps._image.png}
        \includegraphics[width=\linewidth]{Supp_Images/01223_0_glass_windshield._image.png}
        \includegraphics[width=\linewidth]{Supp_Images/03478_0_a_photo_of_a_tallest_building._image.png}
        \includegraphics[width=\linewidth]{Images/visual/04736_43_a_photograph_of_a_riding_person._image.png}
        \includegraphics[width=\linewidth]{Images/visual/00047_47_a_bad_photo_of_a_white_stand._image.png}
        \caption{Input image}
        % \label{fig:architecture}
    \end{subfigure}%
    \hspace{0.05cm}%
    \begin{subfigure}[t]{0.19\textwidth}
        \centering
        \includegraphics[width=\linewidth]{Supp_Images/00683_25_an_image_of_a_grey_steps._gt.png}
        \includegraphics[width=\linewidth]{Supp_Images/01223_0_glass_windshield._gt.png}
        \includegraphics[width=\linewidth]{Supp_Images/03478_0_a_photo_of_a_tallest_building._gt.png}
        \includegraphics[width=\linewidth]{Images/visual/04736_43_a_photograph_of_a_riding_person._gt.png}
        \includegraphics[width=\linewidth]{Images/visual/00047_47_a_bad_photo_of_a_white_stand._gt.png}
        \caption{GT mask}
        % \label{fig:architecture}
    \end{subfigure}%
    \hspace{0.05cm}%
    \begin{subfigure}[t]{0.19\textwidth}
        \centering
        \includegraphics[width=\linewidth]{Supp_Images/00683_25_an_image_of_a_grey_steps..png}
        \includegraphics[width=\linewidth]{Supp_Images/01223_0_glass_windshield..png}
        \includegraphics[width=\linewidth]{Supp_Images/03478_0_a_photo_of_a_tallest_building..png}
        \includegraphics[width=\linewidth]{Images/visual/04736_43_a_photograph_of_a_riding_person..png}
        \includegraphics[width=\linewidth]{Images/visual/00047_47_a_bad_photo_of_a_white_stand..png}
        \caption{RGBNet}
        % \label{fig:architecture}
    \end{subfigure}%
    \hspace{0.05cm}%
    \begin{subfigure}[t]{0.19\textwidth}
        \centering
        \includegraphics[width=\linewidth]{Supp_Images/00683_38_a_photo_of_one_grey_steps..png}
        \includegraphics[width=\linewidth]{Supp_Images/01223_51_a_photograph_of_a_glass_windshield..png}
        \includegraphics[width=\linewidth]{Supp_Images/03478_5_tallest_building..png}
        \includegraphics[width=\linewidth]{Images/visual/04736_63_riding_person..png}
        \includegraphics[width=\linewidth]{Images/visual/00047_64_a_photo_of_a_white_stand..png}
        \caption{$Z$Net}
        % \label{fig:architecture}
    \end{subfigure}%
    \hspace{0.05cm}%
    \begin{subfigure}[t]{0.19\textwidth}
        \centering
        \includegraphics[width=\linewidth]{Supp_Images/00683_76_a_bad_photo_of_a_grey_steps..png}
        \includegraphics[width=\linewidth]{Supp_Images/01223_92_a_cropped_photo_of_a_glass_windshield..png}
        \includegraphics[width=\linewidth]{Supp_Images/03478_79_a_bad_photo_of_a_tallest_building..png}
        \includegraphics[width=\linewidth]{Images/visual/04736_85_a_good_photo_of_a_riding_person..png}
        \includegraphics[width=\linewidth]{Images/visual/00047_85_a_photo_of_one_white_stand..png}
        \caption{LD-ZNet}
        % \label{fig:architecture}
    \end{subfigure}
    
    \caption{Qualitative comparisons on the PhraseCut test dataset. Each row contains an RGB image along with a reference text as an input, with the goal being to segment out the image regions corresponding to the reference text. The reference texts are \emph{``grey steps"}, \emph{``glass windshield"}, \emph{``the tallest building"}, \emph{``riding person"}, \emph{``white stand"} for rows 1, 2, 3, 4 and 5 respectively. We show improvements using ZNet and LD-ZNet compared to the RGBNet.}
    \vspace*{3in}
    \label{fig:visual_results2_supp}
\end{figure*}

\subsection{Qualitative Comparisons on Phrasecut}
Figure \ref{fig:visual_results2_supp} shows qualitative comparisons of $Z$Net and LD-ZNet with the RGBNet baseline on the test dataset of PhraseCut. Attributes such as ``grey", ``glass", ``tallest", and ``riding" are well understood and localized in LD-ZNet.


\end{document}
