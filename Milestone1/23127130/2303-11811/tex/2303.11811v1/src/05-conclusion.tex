\section{Conclusion}\label{conclusion}
% Repeat what has been done
In this paper, we have introduced a hybrid coupled fluid-particle implementation with geometrically resolved particles. We use \gls{gpu}s for the fluid dynamics, whereas the particle simulation runs on \gls{cpu}s.
% Restate problem
We have addressed the issue that in multiphysics simulations, different methodologies can have distinctly different computational properties, implying that the best-suited hardware architecture may differ between the simulation components.
% Summarize arguments and findings
The paper has examined the performance of this approach for two cases of a fluidized bed simulation that differ significantly in terms of the number of particles per volume.
The particle methodology scales poorly, reaching a parallel efficiency of about 45\% when using only six \gls{cpu} cores.
This scaling result supports our initial assumption that the particle simulation part of such a coupled simulation would not benefit from a \gls{gpu} parallelization.
The penalty introduced by the hybrid implementation (i.e., \gls{cpu}-\gls{gpu} communication) is negligible because we are transferring only a small amount of data per particle but no fluid cells.
The performance of the fluid simulation is close to utilizing the whole memory bandwidth of the A100, implying that the \gls{gpu} is the best choice for the fluid simulation.
In both cases, the \gls{gpu} routines take most of the run time.
In a weak scaling benchmark, the hybrid fluid-particle implementation reaches a parallel efficiency of 71\% in the dilute case and 53\% in the dense case when using 1024 \gls{cpu}-\gls{gpu} pairs.
The \gls{pd} methodology requires 32 \gls{cpu}-\gls{cpu} communications per time step which is the driving force for the decrease of the overall parallel efficiency. Our results are limited insofar as different numbers of particle sub-cycles, fluid cells per diameter, etc., will result in different performance results.
% Key takeaways from the paper
We have formulated four criteria that a hybrid implementation must meet to be suitable for the responsible use of heterogeneous supercomputers.
The performance results have shown that our hybrid implementation fulfills all criteria making it suitable for large-scale simulations on heterogeneous supercomputers.
% Future work / Outlook
In the future, we plan to investigate the particle communications in more detail regarding the bottleneck and optimization possibilities. We have shown the acceleration potential of hybrid implementations. Therefore, we plan to run coupled fluid-particle simulations of unprecedented sizes to better understand complex physical phenomena occurring in sea and river beds.
