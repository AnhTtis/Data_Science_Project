\documentclass[a4paper,11pt]{article}
\pdfoutput=1
\usepackage{jcappub} 
%\usepackage[T1]{fontenc}

\usepackage[utf8]{inputenc}%for some citations


% -------------------------------------------------------------------------------

\def\d{\mathop{}\!\mathrm{d}}

% -------------------------------------------------------------------------------

\title{\boldmath Dark Energy on Astrophysical Scales and Its Detection in the Milky Way}
\author[a]{Rui Zhang}
\author[b,1]{Zhen Zhang,\note{Corresponding author.}}

\affiliation[a]{Theoretical Physics Division, Institute of High Energy Physics, Chinese Academy of Sciences, 
19B Yuquan Road, Beijing 100049, People’s Republic of China}
\affiliation[b]{Key Laboratory of Particle Astrophysics, Institute of High Energy Physics, Chinese Academy of Sciences, 
19B Yuquan Road, Beijing 100049, People’s Republic of China}

\emailAdd{zhangzhen@ihep.ac.cn}

\abstract{
Understanding the origin and nature of dark energy is one of the greatest challenges of modern science. 
In this work, we investigate dark energy on astrophysical scales and provide a cosmology-dependent method to measure its equation-of-state parameter. 
First we globally introduce the concept of a perfect fluid in any static, curved spacetime and reexpress its energy-momentum tensor in a general isotropic form, 
by which the equation-of-state parameter can be defined physically. 
Within this theoretical framework, the energy-momentum tensor of dark energy can take the general isotropic form. 
Moreover, we explore the SdS$_{w}$ spacetime and establish its connection with dark energy in cosmology through the equation-of-state parameter. 
Inevitably a repulsive dark force can be induced by dark energy in the SdS$_{w}$ spacetime. 
Then we apply the concept of the dark force to realistic astrophysical systems via the Poisson equation. 
Finally, we find that the anomaly in the galactic rotation curve can be interpreted by the dark force, 
and by fitting the Milky Way curve we model-independently obtain the equation-of-state parameter of dark energy. 
}


\begin{document}
\maketitle
\flushbottom

%%%%%%%%%%%%%%%%%%%%
%%%%%%%%%%%%%%%%%%%%
%%%   Introduction
%%%%%%%%%%%%%%%%%%%%
%%%%%%%%%%%%%%%%%%%%

\section{Introduction}
The accelerating cosmic expansion implies that the universe is possibly dominated by some component termed as dark energy (DE)~\cite{Riess:1998cb, Perlmutter:1998np}. 
There are kinds of ways proposed to detect DE~\cite{PhysRevD.76.043006,Ishak:2007ea,He:2017alg,Zhang:2021ygh,Ho:2015nsa,Vagnozzi:2021quy,Ferlito:2022mok,Nunes:2022bhn}. 
So far the existing DE detections are mostly cosmology-dependent~\cite{Weinberg:2013raj,Weinberg:2013agg}. 
Actually, DE has remained mysterious in its origin and nature. 
What we know about the DE is that it can be characterized by an equation-of-state (EoS) parameter~\cite{Cai:2009zp,Li:2011sd,Wang:2016och}
\begin{equation}
w\equiv p_w/\rho_w\,, 
\end{equation}
where $\rho_w$ ($p_w$) is the DE's energy density (pressure). 


Until today, various DE models have been proposed to interpret current astronomical observations. 
These models can be described by $w$ when it takes their corresponding values~\cite{He:2017alg,Zhang:2021ygh}. 
Take several typical models for example.
In the cosmological constant (CC) model, $\,w=-1$~\cite{1917SPAW.......142E}. 
In the quintessence model, $-1<w <-\frac{1}{3}$~\cite{Wetterich:1987fm,PhysRevLett.82.896}. 
In the phantom model, $\,w<-1$~\cite{Caldwell:1999ew}. 
Clearly, $w$ can be viewed as a good probe to identify the proper DE model. 



If the EoS parameter $w$ is assumed to be a constant,
one can get $w=-1.028\pm0.031$ by fitting the $w$CDM model to the cosmological data~\cite{Aghanim:2018eyx}. 
However, there is no reason why $w$ should be a constant. 
In fact, \cite{Zhao:2017cud} shows that $w$CDM model is not preferred at a $3.5 \sigma$ significance level. 
This means that $w$ may evolve with the cosmological redshift $z_{r}$, namely $w=w\left(z_{r}\right)$.
Nevertheless, $w$ can be regarded as a constant on astrophysical scales, like the Milky Way (MW) galaxy.
Consequently, in the ideal case of a point-like mass, local DE effects can be described well by the SdS$_{w}$ metric~\cite{He:2017alg}. 


\begin{figure}
\centerline{
\includegraphics[width=0.68\columnwidth]{figure1.pdf}
}
\caption{
The MW rotation curve. The black and orange points are obtained from the Huang et al. 2016 data~\cite{2016MNRAS.463.2623H} and the Sofue 2013 data~\cite{Sofue:2013kja}, respectively. The black line represents the best fitting curve to the Huang et al. 2016 data~\cite{2016MNRAS.463.2623H}. 
We extrapolate the black line to the outer MW region with radius $r\sim100-200$ kpc, where $r$ is the galactocentric radius.
Here we use the gray dashed line to mark this extrapolation. 
Obviously the RC shows an anomalous drop in the outer MW region,
indicating possibly the existence of a repulsive force in that region. 
}
\label{sample}
\end{figure}


Interestingly a dark force can be induced by DE~\cite{Ho:2015nsa,Zhang:2021ygh}. 
In the SdS$_{w}$ spacetime, the specific form of the dark force was derived in a way independent of any specific DE models~\cite{Zhang:2021ygh}. 
For the CC case, it reduces to the one shown in~\cite{Ho:2015nsa}. 
The dark force is locally repulsive and its strength depends on the DE parameter $w$~\cite{Zhang:2021ygh}.
Its existence can lead to deviations from the purely Newtonian force.
Detecting any deviation from the inverse square law behavior
can help to measure DE parameter $w$ on astrophysical scales. 



Recently, we notice that in the MW galaxy, there seems to be an anomalous drop in the outer rotation curve (RC), 
far more than we expected, as illustrated in Fig.~\ref{sample}.
Here, we suggest that this abnormal phenomenon is related with the repulsive dark force that tends to be strengthened greatly in the outer MW region,
although the exact form of the dark force needs further derivations when dealing with a realistic astrophysical system. 


In this work, we further investigate the dark force acting on astrophysical scales and explore its cosmological origin through the EoS parameter $w$. 
To fully recognize the role of $w$ in the dark force, 
we generalize the Poisson equation to the SdS$_{w}$ spacetime,
and derive the most general gravitational potential including both the matter and DE contributions.
To well understand what role $w$ can play in the universe, 
we deduce some basic cosmological equations, like the generalized Friedmann equations, in an extended Robertson-Walker spacetime. 
Besides, by comparison, we find that the EoS parameter in the SdS$_{w}$ spacetime can be defined in the same way as that in the universe, 
establishing a connection between DE in cosmology and the counterpart in astrophysics.
Then, based on the dark force derived from the gradient of the general potential, we propose a method to measure the EoS parameter $w$ through fitting the galactic RCs. 
Finally, we summarize our findings and conclusions.


%%%%%%%%%%%%%%%%%%%%
%%%%%%%%%%%%%%%%%%%%
%%%   Theoretical analysis
%%%%%%%%%%%%%%%%%%%%
%%%%%%%%%%%%%%%%%%%%

\section{Theoretical analysis}\label{ThA} 


\subsection{Generalized Friedmann equations in an extended RW spacetime}\label{ExtRW}

The Robertson-Walker (RW) metric is used to describe the cosmic expansion. In reality, the universe is not perfectly homogeneous. Thus, the RW metric should be modified to be~\cite{Bardeen:1980kt,Ma:1995ey,Esposito-Farese:2000pbo}
\begin{eqnarray}
\label{eq:ERW}
\begin{array}{rcl}
\displaystyle
%\nonumber
\d s^{2}=\big[Z\left(x,y,z\right)\big]^{2}\d t^{2}-\big[a\left(t\right)R\left(x,y,z\right)\big]^{2}\left(\d x^{\,2}+\d y^{\,2}+\d z^{\,2}\right),
\end{array}
\end{eqnarray} 
where $t$ is the time-like coordinate, $a=a\left(t\right)$ is the cosmological expansion factor, as well as $Z\left(x,y,z\right)$ and $R\left(x,y,z\right)$ are functions of the space-like coordinates $x,~y,~z$. In the following this metic will be called the {\it extended Robertson-Walker} (ERW) metric. 
The inhomogeneity of the universe may come from the small perturbations excited by primordial quantum fluctuations or catastrophic astrophysical events. 
The ERW metric~\eqref{eq:ERW} can also be treated as a background field. 

{\iffalse
Then various forms of metric perturbations can be excited around the background field.  
Some of these perturbations may manifest themselves as the fluctuations of the physical density and pressure, %denoted as $\delta\rho$ and $\delta p$, respectively,
and they grow via some instability mechanisms into the large structures that cannot be treated as perturbations, such as the galaxies and clusters we see today.
Besides, when coming to the physics on much shorter scales than the cosmological scale, we need to include additional contributions, like that from matter, into the spacetime metric. 
When taking all these factors into consideration, we need to introduce the ERW metric rather than that written as the standard RW metric plus a perturbation.
Clearly, the ERW metric can describe a more real universe. 
\fi}



According to the law of energy-momentum conservation in an expanding universe, one also gets 
\begin{align}
\label{eq:EPConservation}
\nabla_{\mu}\,T^{\mu\nu}=\partial_{\mu}\,T^{\mu\nu}+\Gamma^{\mu}_{\,\,\,\mu\rho}\,T^{\rho\nu}+\Gamma^{\nu}_{\,\,\,\mu\rho}\,T^{\mu\rho}=0, 
\end{align}
with $\mu,\,\nu,\,\rho=0,\,1,\,2,\,3$, where $\nabla_{\mu}$ is the covariant derivative, 
$T^{\mu\nu}$ denotes the component of the energy-momentum tensor, and $\Gamma^\mu_{\,\,\,\nu\rho}$ represents the Christoffel symbol.
Considering the zero component of the energy conservation equation, as seen from the comoving observer, it should be written as (appendix~\ref{App:A})
\begin{align}
\label{eq:EConservation}
\frac{\d T^{0}_{\,\,\,0}}{\d t}+3\,\frac{\dot{a}}{a}\left(T^0_{\,\,\,0}-\frac{1}{3}T^{i}_{\,\,\,i}\right)=0, 
\end{align}
with $\dot{a}=\d a/ \d t$, where $T^{\mu}_{\,\,\,\nu}$ is the mixed component of the energy-momentum tensor. More general forms of the conservation equations of energy and momentum are presented in appendix~\ref{App:A}.


From the Einstein equation, 
\begin{align}
G_{\mu\nu}=R_{\mu\nu}-\frac{1}{2}\,g_{\mu\nu}\,R=T_{\mu\nu},
\end{align}
we can derive the generalized Friedmann equations: 
\begin{align}
\label{eq:Friedmann2M}
\frac{1}{~Z^2~}\frac{\ddot{a}}{a}=&-\frac{4}{3}\pi \left(T^{0}_{\,\,\,0}-T^{i}_{\,\,\,i}\right),\\[1.5mm]
\label{eq:Friedmann1M}
\frac{1}{~Z^2~}\left(\frac{\dot{a}}{a}\right)^2=&-\frac{K\left(x,y,z\right)}{a^2}+\frac{8}{3}\pi \,T^{0}_{\,\,\,0},
\end{align}
where $K=K\left(x,y,z\right)$ is actually a generalization of the constant curvature of the standard RW space.
As demonstrated in appendix~\ref{App:C}, we prove 
\begin{eqnarray}
\label{eq:ERW-Kterm0}
%\begin{array}{rcl}
%\displaystyle%
\nonumber
K=\frac{~\sum K^i_{p}~}{~3~}=\frac{1}{3}\,\bigg[-\frac{2}{R^2}\frac{\partial^2_i R}{R}+\frac{1}{R^2}\left(\frac{\partial_i R}{R}\right)^2\bigg], 
%\end{array}
\end{eqnarray}
where $K^i_{p}$ is some sectional curvature. 
Therefore, $K$ is intrinsically an average sectional curvature. 

The expansion rate of the universe is usually characterized by the Hubble parameter,
\begin{align*}
H=\frac{\dot{a}}{a},
\end{align*}
whose present value $H_{0}$ is called the Hubble constant.
Then, the generalized Friedmann equation~\eqref{eq:Friedmann1M} can be reexpressed in a more familiar form,
\begin{align*}
\frac{1}{~Z^2~}H^2=&-\frac{K\left(x,y,z\right)}{a^2}+\frac{8}{3}\pi \,\rho,
\end{align*}
which takes a much more general form than the traditional Friedmann equation that corresponds to the standard RW metric.




In fluid mechanics, under the incompressibility condition, a fluid can be characterized by the \textit{stress tensor}~\cite{landau2013fluid}, 
\begin{eqnarray}
\label{eq:FM}
%\begin{array}{rcl}
%\displaystyle%
\nonumber
T^{i}_{\,\,\,j}\equiv-p\,\delta^{i}_{\,\,\,j}+\tau^{i}_{\,\,\,j},
%\end{array}
\end{eqnarray}
where the trace of $\tau^{i}_{\,\,\,j}$ is zero. Then, we find 
\begin{eqnarray}
\label{eq:mpreasure}
%\begin{array}{rcl}
%\displaystyle%
\nonumber
\mathrm{Tr}\big[\,T^{i}_{\,\,\,j}\,\big]=-\mathrm{Tr}\big[\,p\,\delta^{i}_{\,\,\,j}\,\big]+\mathrm{Tr}\big[\,\tau^{i}_{\,\,\,j}\,\big]=-3\,p,
%\end{array}
\end{eqnarray}
where the trace of the stress tensor is actually equivalent to the mean of the principal stresses of the incompressible fluid~\cite{landau2013fluid}.
In fact, the pressure can be generally defined as negative one-third the trace of the stress tensor for any fluid~\cite{kundu04}, which is sometimes named as the dynamical pressure in fluid mechanics.
Accordingly, we write the energy density and the dynamical pressure for matter or DE in the ERW spacetime as 
\begin{align}
\label{eq:rho-pressure}
\rho\equiv T^{0}_{\,\,\,0}\,, \quad p\equiv-\frac{~\mathrm{Tr}\big[\,T^{i}_{\,\,\,j}\,\big]~}{3}=-\frac{~\sum\,T^{i}_{\,\,\,i}~}{3},
\end{align}
where $\rho$ is the same as the usual energy density, while $p$ differs somewhat from the usual pressure defined in classical fluid mechanics. 
In our case, the energy density defined in~\eqref{eq:rho-pressure} is just the measured one by a local static observer,
and its value does not depend the choice of coordinates. 
Generally, the trace of all the components of the energy-momentum tensor is invariant under coordinate transformations, and thus it is physical.
Once the energy density is given, the dynamical pressure will be independent of specific coordinates and thus physically measurable, 
as a result of its equivalence to the trace of all the spatial components of the energy-momentum tensor or the sum of  $T^{i}_{\,\,\,i}$s that has already been introduced in the conservation equations and the Friedmann equations.




In general relativity (GR), 
both the diagonal and off-diagonal components of the stress tensor are highly dependent of the coordinate system chosen [See, for example, after Eq.~\eqref{eq:fullT} below]. 
So the EoS parameter cannot be simply defined as the ratio of some diagonal component of the stress tensor to the energy density.
However, the trace of the stress tensor is a scalar independent of spatial coordinates. 
Accordingly, we obtain an EoS parameter
\begin{align}
\label{eq:EOS}
w=w\left(t, x, y, z\right)=\frac{~p~}{\rho},
\end{align}
which is actually a measurable parameter, independent of the basis referred and the coordinate system chosen.


The specific derivations of the conservation equations and the Friedmann equations, associated with the ERW metric~\eqref{eq:ERW}, 
are presented in appendies~\ref{App:A} and~\ref{App:B}, respectively. 
If we carefully go though the derivations of these basic equations in cosmology, 
we find that the off-diagonal components of the energy-momentum tensor do not contribute to the cosmological evolution. 
Indeed, the evolution of the universe is determined by both the $00$ component (namely $T^{0}_{\,\,\,0}$) and the trace of the components of the stress tensor (or the average value of the diagonal components) rather than one of these components, which has long been ignored. 
More exactly, it is the EoS parameter, defined in Eq.~\eqref{eq:EOS} via Eq.~\eqref{eq:rho-pressure}, that determines the evolution of the universe, as illustrated in Eqs.~\eqref{eq:EConservation}, \eqref{eq:Friedmann2M} and~\eqref{eq:Friedmann1M}.
The ERW universe can contain various forms of components, like DE, matter, and radiation.
The DE component contributes to the total energy density $\rho$~\eqref{eq:rho-pressure}, the total pressure $p$~\eqref{eq:rho-pressure}, and the cosmological EoS parameter $w$~\eqref{eq:EOS} only as a part.  As shown in Eq.~\eqref{eq:EOS}, there may be a spatial dependence for these three quantities involved in those cosmological equations associated with the ERW metric.
This dependence may be partially resulted from the contributions of matter and radiation.
Compared with the two components, DE tends to affect the three quantities on much larger scales. 




According to the ERW metric, one may propose the corresponding cosmological models 
and test the inhomogeneity of the universe on a much larger scale, like that of the large-scale structure. 
But when coming to a much shorter scale, we need to focus on the physics in a realistic astrophysical system like the Milk Way galaxy. 
Compared with the whole universe, the galaxy can only be treated as a ``dust".
Thus we need to develop a cosmology-independent method to probe the local background structures surrounding the ``dust" at astrophysical scales.
Especially for the DE background, its local properties remain mysterious on such scales. 
At present, the only knowledge about DE is that its EoS parameter may evolve with the cosmological redshift $z_{r}$, namely, $w=w\left(z_{r}\right)$~\cite{Zhao:2017cud}. 
So far there has not been any method to differentiate the DE background of the ``dust" from the DE counterpart in the standard RW universe.
Conversely, this does indicate that the local DE background can be described by the DE candidate in the standard RW universe to a good approximation.
Therefore, in the following, our analysis will be only performed in the standard RW universe. 



In the standard RW universe, the expansion factor evolves with time.
Thus, the EoS parameter may evolve as a function of time. Then, Eq.~\eqref{eq:EConservation} becomes
\begin{align}
\frac{\dot{\rho}}{\rho}=-3\left(1+w\right)\frac{\dot{a}}{a},
\end{align}
which is derived directly from the ERW metric. If $w$ is a global constant, this equation can be solved to yield 
\begin{align}
\rho\propto a^{-3\,\left(1+w\right)}\,. 
\end{align}
In general, the EoS parameter may evolve with the cosmological redshift $z_{r}$, namely, $w=w\left(z_{r}\right)$. 
However, it takes an approximately constant value on astrophysical scales~\cite{He:2017alg}. 



Similarly, for the RW case, the expansion factor is constant on astrophysical scales~\cite{He:2017alg}.
thus it can be absorbed by re-defining the spatial coordinates $\left(x,~y,~z\right)$.
Let us write $x^{\mu}~\left(\mu=0,~1,~2,~3\right)$ for $\left(t,~x,~y,~z\right)$.
Therefore, DE can be thought to be in a locally static state. Accordingly, its energy-momentum tensor can be described by 
\begin{equation}
\label{eq:fullT}
\widetilde{T}=T_{\mu\nu}\,\d x^{\mu}\otimes\d x^{\nu},
\end{equation}
where all the tensor components $T_{\mu\nu}$ are independent of the coordinate time. 
It is perhaps noteworthy that in GR, the components of the energy-momentum tensor are closely related with the referred basis $\{\d x^{\mu}\}$.
The referred basis are highly dependent of the coordinate system chosen and hence also both the diagonal and off-diagonal components.
Therefore, judging if a fluid is isotropic based on the components of the energy-momentum tensor of the fluid makes no sense.




%%%%%%%%%%%%%%%%%%%%
%%%   The ideal SdS$_{w}$ spacetime
%%%%%%%%%%%%%%%%%%%%

\subsection{The ideal SdS$_{w}$ spacetime}\label{SdSw}



As mentioned above, the EoS parameter $w$ can be reasonably assumed to be a constant in an astrophysical system at redshift $z_{r}$;
its value is inherited directly from that of the cosmological DE at the same redshift. 
Basing on this, we find that, under gravitational fields of matter, the energy-momentum tensor of DE may be no longer the same as before. 
For instance, in the quintessence model\,\cite{Wetterich:1987fm,PhysRevLett.82.896}, to keep $w$ constant on astrophysical scales, 
the quintessence as a DE candidate may redistribute itself and its stress tensor may deviate from the traditional isotropic form of $T^{i}_{\,\,\,j}\propto\delta^{i}_{\,\,\,j}$.
However, the energy-momentum tensor should obey the Einstein equation.
To obtain an appropriate expression for the energy-momentum tensor of isotropic DE, we need to derive it by solving the Einstein equation for an astrophysical system.



In current models, the DE's energy density and pressure are believed to be spatially homogeneous and isotropic on the cosmological scale. 
When coming to the physics at a smaller scale, we have to take the gravity like that of matter into our consideration. 
However, in the gravitational system of a point-like mass, it can be expected that the tensor $\widetilde{T}$ can be expressed in a static and spherically-symmetric form.
Denote $\delta_{ij}$ as the Kronecker delta and set $x_{i}=\delta_{ij}\,x^{j}$. Then, let $\vec{x}\cdot\vec{x}=x_{i}\,x^{i}=r^{2}$. 
By the assumption of staticity and spherical symmetry, the general energy-momentum tensor for DE can be therefore given in some Cartesian coordinate system by 
\begin{align}
\label{eq:Tcomponets}
\begin{array}{rcl}
\displaystyle
T^t_{\,\,\,t}&=&A(r),\quad T^t_{\,\,\,i}=0,\quad T^i_{\,\,\,j}=B(r)\,\delta^{i}_{\,\,j}+C(r)\,x^{i}\,x_{j},
\end{array}
\end{align}
which was shown by Kiselev in~\cite{Kiselev:2002dx},
with the metric
\begin{equation}
\label{eq:sphcartesian}
\d s^{2}=g_{tt}\,\d t^{2}-\bigg[\left(g_{rr}-1\right)\frac{x^{i}\,x^{j}}{r^2}+\delta^{ij}\bigg]\d x_{i}\d x_{j},
\end{equation}
where $A(r)$, $B(r)$, $C(r)$, $g_{tt}=g_{tt}\left(r\right)$ and $g_{rr}=g_{rr}\left(r\right)$ are functions of radius $r$. 
As we will show below, the energy-momentum and metric tensors are both of isotropy. 

Then consider the transformation from the Cartesian coordinates $\{x^i\}$ to the polar coordinates $\{r,\theta,\varphi\}$:
\begin{eqnarray}
\label{eq:Pcoordinates}
\begin{array}{rcl}
\displaystyle
x=r\sin\theta \cos\varphi, \quad\,y=r\sin\theta \sin\varphi,~~~z=r\cos\theta.
\end{array}
\end{eqnarray}
This implies (see Weinberg's book~\cite{Weinberg:1972kfs} for details) 
\begin{eqnarray}
\label{eq:xidxi}
\begin{array}{rcl}
\displaystyle
x_{i}\,\d x^{i}\!&=&\!r\,\d r,\quad\d x_{i}\otimes\d x^{i}\!=\!\d r\otimes\d r\!+\!r^2 \d\theta\otimes\d\theta\!+\!r^2\sin^2\theta\,\d\varphi\otimes\d\varphi, 
\end{array}
\end{eqnarray}
with $\otimes$ being the tensor product, which are both rotational invariants. 
Think of isotropy as invariance under rotations, suitably generalized in GR~\cite{Carroll2014}.
Thus, the most general form of the static, isotropic tensor of rank two can be definitively expressed in terms of these two rotational invariants as bases. 

Combining Eqs.~\eqref{eq:Tcomponets} and~\eqref{eq:sphcartesian}, we derive
\begin{eqnarray}
\label{eq:Tlk}
\begin{array}{rcl}
\displaystyle
T_{tt}=&+&g_{tt}\left(r\right)\,A(r),~~~
T_{ti}=0,\\[2mm]
%\nonumber
T_{ij}=&-&\bigg[g_{rr}\left(r\right)C(r)+\left(\frac{g_{rr}\left(r\right)-1}{r^2}\right)B\left(r\right)\bigg]x_{i}\,x_{j}-\,B\left(r\right)\delta_{ij}, 
\end{array}
\end{eqnarray} 
by which we define the energy-momentum tensor~\eqref{eq:fullT} and find that the tensor does not depend on $t$, and depends on $x^i$ and $\d x^{i}$ only through the two rotational invariants in~\eqref{eq:xidxi}. 
Actually, we can further rewrite the energy-momentum tensor in the following form, 
\begin{eqnarray}
\label{eq:Tfull}
%\begin{array}{rcl}
%\displaystyle
\nonumber
\widetilde{T}&=&g_{tt}\left(r\right)\,A(r)\,\d t\otimes\d t-g_{rr}\left(r\right)\bigg[B\left(r\right)+C(r)\,r^{2}\bigg]\d r\otimes\d r \\[2mm]
&-&\,B\left(r\right)\left(r^2 \d\theta\otimes\d\theta+r^2\sin^2\theta\,\d\varphi\otimes\d\varphi\right),
%\end{array}
\end{eqnarray}
which is already the most general form in terms of the bases given by~\eqref{eq:xidxi} that $\widetilde{T}$ can take in the polar coordinates. 
Correspondingly, the metric~\eqref{eq:sphcartesian} can be expressed as 
\begin{eqnarray}
\label{eq:gsph}
\begin{array}{rcl}
\displaystyle
\d s^{2}\!=\!g_{tt}\left(r\right)\d t^{2}\!-\!g_{rr}\left(r\right)\d r^{2}\!-\!r^2 \left(\d\theta^2\!+\!\sin^2\theta\,\d\varphi^2\right),
\end{array}
\end{eqnarray}
where $t$ is the time-like coordinate while others are space-like coordinates. This metric is consistent with that shown by Kiselev in~\cite{Kiselev:2002dx}. 
Thus, both the energy-momentum and metric tensors defined by Eqs.~\eqref{eq:Tcomponets} and~\eqref{eq:sphcartesian} are spherically symmetric. 
Actually they have the same form as the general static isotropic tensor of rank two, 
which is referred to as the {\it standard form} of the isotropic tensor by Weinberg in~\cite{Weinberg:1972kfs}.
Therefore, the energy-momentum tensor~\eqref{eq:Tcomponets} already takes the most general isotropic form in a static, isotropic spacetime.




In GR, a perfect fluid is defined by the condition that, at any spacetime point $P$, there is always a locally inertial frame, 
which is a sufficiently small region around $P$, comoving with the fluid element, 
and in which the fluid is isotropic for the comoving observer at $P$. 
More exactly, as seen from the comoving observer, 
the energy–momentum tensor of the fluid at the point $P$ should satisfy $T^{t}_{\,\,\,i}=0$ and $T^{i}_{\,\,\,j}\propto\,\delta^{i}_{\,\,\,j}$. 
Note that the tetrad of the comoving observer depends on the coordinate system chosen, and hence also the measured values by the same observer.
By this definition, the fluid described by~\eqref{eq:Tcomponets} can be treated to be perfect.
In fact, following the same steps as outlined in~\cite{Weinberg:1972kfs}, the energy-momentum tensor~\eqref{eq:Tcomponets} can always be expressed in the special form of $T^{t}_{\,\,\,i}=0$ and $T^{i}_{\,\,\,j}\propto\,\delta^{i}_{\,\,\,j}$ under coordinate transformations. Thus, this fluid is isotropic for the comoving observer in some coordinate system. 
In this coordinate system, the energy-momentum tensor~\eqref{eq:Tcomponets} can directly associate with the solution of a DE model\footnote{Nevertheless the two solutions cannot be combined directly through coordinate quantities. Usually, the coordinates used in one solution differs in the physical meaning from those used in another one, although the two systems of coordinates can be expressed using the same set of symbols. In GR, the two solutions should be combined into one through their physical quantities, like the dynamical pressure~\eqref{eq:rho-pressure} and the EoS parameter~\eqref{eq:EOS}, rather than their coordinate quantities.} where $\widetilde{T}$ of the DE candidate takes the form of ${\rm diag} (\rho, -p, -p, -p)$ in the same coordinate system. 


Generally, the specific form of the energy-momentum tensor changes from one coordinate system to another.
In classical fluid mechanics, the perfect fluid has the form of an identity matrix.
However, in GR, the stress tensor of the perfect fluid can not always take that form, especially under general coordinate transformations.
%However, in GR, the stress tensor of the perfect fluid can not always take the same form as that of an identity matrix, especially under general coordinate transformations.
In physics, the concept of a perfect fluid should be introduced independently of the coordinate system chosen.
To arrive at this concept, we need not restrict the stress tensor to the form of the identity matrix in a curved spacetime. 
Additionally, the energy-momentum tensor of the perfect fluid should be form-invariant under coordinate transformations,
Thus, it needs to take a more general form.
Furthermore, it should satisfy the isotropy condition, somewhat generalized in GR~\cite{Rindler2006book,Carroll2014}. 
In fact, the energy-momentum tensor given by~\eqref{eq:Tcomponets} has already taken the most general isotropic form and meets all these requirements. 
So the fluid with its energy-momentum tensor satisfying~\eqref{eq:Tcomponets} should be defined as the $\textit{perfect fluid}$ in GR.



Now come back to the DE in a real astrophysical system. Its EoS parameter $w$ can be treated as a constant.
If we impose the condition of $w={\rm constant}$, the energy-momentum tensor~\eqref{eq:Tcomponets} reduces to 
\begin{align}
\label{eq:Tcomponets1} 
T^{0}_{\,\,\,0}=\rho_{w}=A(r),~~~T^{i}_{\,\,\,j}=\lambda\,\rho_{w}\bigg[\widetilde{B}\,\delta^{i}_{\,j}-\left(1+3\,\widetilde{B}\right)\frac{x^{i}x_{j}}{r^2}\bigg],
\end{align}
where $\rho_{w}$ is the DE's energy density, $\lambda$ is a constant parameter,
and $\widetilde{B}$ is an arbitrary parameter, depending on local properties of the DE in the astrophysical system. 
As we will see, only if $\widetilde{T}$ is constructed in this way, we can keep the EoS parameter to be a constant. Thus, 
\begin{align}
\label{eq:Tcomponets2}
B\left(r\right)=\lambda\,\rho_{w}\,\widetilde{B},~~~C\left(r\right)=-\lambda\,\rho_{w}\bigg[1+3\,\widetilde{B}\bigg]\frac{1}{r^2},
\end{align}
which takes a more general form than the Kiselev's form~\cite{Kiselev:2002dx}. 
By the definition~\eqref{eq:rho-pressure} and from Eqs.~\eqref{eq:Tcomponets}, we obtain the energy density and the dynamical pressure of DE:
\begin{eqnarray}
%\begin{array}{rcl}
%\displaystyle
\label{eq:edensity}
\rho_{w}&\equiv&A\left(r\right),\\
\label{eq:pdensity}
p_{w}&\equiv&-\frac{~\mathrm{Tr}\big[\,T^{i}_{\,\,\,j}\,\big]~}{3}=-\bigg[B\left(r\right)+\frac{1}{3}\,C\left(r\right)r^2\bigg],
%\end{array}
\end{eqnarray}
which are defined in the same way as that in the ERW spacetime. 
Notice that the trace of all the components of the energy-momentum tensor is physical as it is independent of the choice of coordinates.
Physically the energy density can be measured by the local observer.
In our case, $\rho_{w}$ is given by a DE model and it is unchanged in different coordinates.
Thus, the coordinate transformation is only on the components of the stress tensor.
Under such a transformation, the dynamical pressure is invariant as it is equivalent to the trace of the components of the stress tensor.   

Substituting~\eqref{eq:Tcomponets2} into~\eqref{eq:edensity} and~\eqref{eq:pdensity}, one derive the equation of state 
\begin{eqnarray}
%\begin{array}{rcl}
%\displaystyle
\label{eq:pdensityw}
p_{w}=\frac{\lambda}{3}\,\rho_{w}, 
%\end{array}
\end{eqnarray}
which determines the influence of DE on what happens in an astrophysical system. 
Thus, one gets the EoS parameter, namely $w=\frac{\lambda}{3}$, which is a constant parameter.
Then, repeating the same derivations as in~\cite{Kiselev:2002dx,He:2017alg}, we can obtain the SdS$_{w}$ metric~\cite{Zhang:2021ygh} 
for the isolated system of a point-like object with mass $M$ from solving the Einstein equation,
\begin{eqnarray}
\label{eq:dSS2}
\begin{array}{rcl}
\displaystyle
\d S_{w}^2 \!=\!&+& \!\!\big[1\!-\!2\,\frac{\,M\,}{r}\!-\!2\left( \frac{\,r_{\!o}^{}\,}{r} \right)^{\!3w+1}\big]\d t^{2}-\!\!\frac{1}{\, \big[1\!-\!2\,\frac{\,M\,}{r}\!-\!2\left( \frac{\,r_{\!o}^{}\,}{r} \right)^{\!3w+1}\big] \,}\d r^2- r^{2}\left(\d \theta^2+\sin^{2}\theta\,\d\varphi^2\right),~~~
\end{array}
\end{eqnarray}
where $\,r_o^{}\,$ is a model parameter characterizing the size of the universe at a given redshift $z_{r}$. 
Here, the geometrized unit system ($G=c=1$) is adopted throughout. As shown in~\cite{Zhang:2021ygh},
there is only one $w$-term responsible for the contributions of various forms of DE candidates.




The SdS$_{w}$ metric carries a standard form of the static isotropic tensor of rank two, which is exactly the same as that shown in~\cite{Weinberg:1972kfs}.
Thus, within the framework of GR, it is also isotropic,
which is also a direct indication of the isotropy of the associated energy-momentum tensor $\widetilde{T}$. 
As mentioned previously, for a fluid, the isotropic form~\eqref{eq:Tcomponets} of its $\widetilde{T}$ clearly indicates that the fluid is intrinsically a perfect one.
But the isotropy of the energy-momentum tensor does not mean that the coordinate form of the stress tensor can aways be that of an identity matrix. 
For instance, in the SdS$_{w}$ case of $w\neq-1$, if we require $T^{r}_{\,\,\,r}=T^{\theta}_{\,\,\,\theta}=T^{\varphi}_{\,\,\,\varphi}$,  
there will be no any static solution for the corresponding Einstein equation.  
Conversely, in the polar coordinates, $T^{i}_{\,\,\,i}$ needs to vary with $i$ so as to satisfy the Einstein equation. 
To some extent, DE needs to redistribute itself under the gravity of some mass distribution.
However, it can remain perfect; that is, its energy-momentum tensor can still take the standard Weinberg's isotropic form.

Usually, we cannot define the EoS parameter of DE in analogy to what we have done for the traditional perfect fluid in fluid mechanism.
Nevertheless, in the SdS$_{w}$ spacetime, the EoS parameter $w$ can be defined via Eqs.~\eqref{eq:rho-pressure} and~\eqref{eq:EOS}, independently of coordinates.
Here the energy density $\rho_{w}$ and the dynamical pressure $p_{w}$ are invariant under a change of coordinate system. 
More exactly, once $\rho_{w}$ is physically given by a DE model, the stress tensor can be transformed into the one of the form $\delta_{ij}$ under spatial coordinate transformations 
while $p_{w}$ defined by Eq.~\eqref{eq:rho-pressure} remains unchanged. In the coordinate system corresponding to the tensor form $\delta_{ij}$, 
we can directly relate the SdS$_{w}$ metric with the solution for the DE model. 
Moreover, $\rho_{w}$ and $p_{w}$ are still well-defined regardless of any form.
However, it is improper to define an EoS parameter directly based on one of the components of the stress tensor in any coordinate system, even in which the stress tensor is already expressed in the form of an identity matrix\footnote{In general, the components of the stress tensor vary between different coordinate systems. Under some coordinate transformations, these components can be reexpressed to be different while the form of the stress tensor kept to be invariant.}, 
as this may bring the coordinate dependence into the definition of the EoS parameter.
Anyway, the EoS parameter~\eqref{eq:EOS} is defined physically in this work and it can be directly inherited from the DE parameter $w$ in cosmology. 




In summary, DE can be characterized by an evolving EoS parameter with the cosmological redshift, namely $w = w\left(z_{r}\right)$. 
Generally, the EoS parameter $w$ varies in different DE models~\cite{He:2017alg,Zhang:2021ygh}. 
The DE candidate in each model can be described by the parameter $w$ when it takes the corresponding value, 
such as the cosmological constant with $w=-1$, the quintessence with $-1<w <-\frac{1}{3}$, and the phantom with $\,w<-1$. 
However, the EoS parameter $w$ can be treated as a constant on astrophysical scales, like the Milk Way galaxy. 
Thus, starting from the assumption that the EoS parameter $w$ is locally a constant, we can always obtain the SdS$_{w}$ metric as a solution to the Einstein equation
and use it to describe various spacetime effects of DE on astrophysical scales.




%%%%%%%%%%%%%%%%%%%%
%%%   The Newtonian analogy
%%%%%%%%%%%%%%%%%%%%

\subsection{The Newtonian analogy}\label{NAnalogy}

If a static spacetime is spherically symmetric, its metric can be written in the following form,
\begin{eqnarray}
\begin{array}{rcl}
\displaystyle
\nonumber
\d s^2\!=\!\left(1\!+\!2\,\Phi\right)\d t^2\!-\!\left(1\!+\!2\,\Phi\right)^{-1}\d r^2\!-\!r^2\left(\d\theta^2\!+\!\sin^2\theta\d \varphi^2\right).
\end{array}
\end{eqnarray}
In the special SdS$_{w}$ case, one has
\begin{equation}\label{eq:potential}
\Phi=-\frac{M}{r}-\left(\frac{r_{\mbox{o}}}{r}\right)^{3w+1}\,,
\end{equation}
where the first term is the Newtonian term with mass $M\,$, and the second term arises from DE. 
Note that $3w+1<0\,$, as required by the accelerated cosmic expansion. 
Clearly, the DE term dominates at large radii. 
However, this is obtained by solving the Einstein equation for the isolated system of a single point-like mass. 
So it is only true in an isolated astrophysical system. In reality, it will begin to fail to describe the gravitational system at too large radii due to tidal effects from any other astrophysical object.


The Einstein equation can be reexpressed as
\begin{equation}
\label{eq:EinsteinEqs}
R^{\mu}_{\,\,\,\nu}=8\pi \,(T^{\mu}_{\,\,\,\nu}-\frac{1}{2}\delta^{\mu}_{\,\,\,\nu} \,T)\,, 
\end{equation}
which consists of two parts, namely 
\begin{equation}
\label{eq:Tmade}
T^{\mu}_{\,\,\,\nu}=T^\mu_{\,\,\,\nu,\rm{m}}+T^\mu_{\,\,\,\nu,w},
\end{equation}
where $T^\mu_{\,\,\,\nu,\rm{m}}\,$ and $T^\mu_{\,\,\,\nu,w}$ represent the contributions of matter and DE, respectively.


Then we consider the matter contribution in general, 
\begin{equation}
T^\mu_{\,\,\,\nu,{\mathrm m}}=\left(\rho_{{\mathrm m}},-p_{\mathrm m},-p_{\mathrm m},-p_{\mathrm m}\right), 
\end{equation}
with $p_{\mathrm m}\ll\rho_{{\mathrm m}}$, where $\rho_{{\mathrm m}}$ and $p_{\mathrm m}$ is the mass density and pressure of matter, respectively. 
Generally, the matter part obeys the Poisson equation,
\begin{eqnarray}
\label{eq:MaPoissonEqs}
\nabla^2 \Phi_\mathrm{m}&=&4\pi\,\left(\rho_\mathrm{m}+3\,p_\mathrm{m}\right)\\ 
\label{eq:MaPoissonEqs1}
&\simeq&4\pi\,\rho_\mathrm{m},
\end{eqnarray}
by which~\eqref{eq:MaPoissonEqs1} we can derive the Newtonian potential,
\begin{equation}
\label{eq:NewPotetical0}
\Phi_m(\vec{r})=-\int\frac{\rho_\mathrm{m}(\vec{r^\prime})}{\mid \vec{r}-\vec{r^\prime}\mid}\d \vec{r^\prime}\,, 
\end{equation}
where the mass density $\rho_\mathrm{m}(\vec{r^\prime})$ distributed over a space region. 
Note that in a realistic astrophysical system, the matter distribution can be approximated well within Newtonian gravity.
Compared to its energy density, its pressure can be ignored;
that is, the EoS parameter is nearly zero.
The total mass of matter is $M=\int\rho_\mathrm{m}(\vec{r})\d \vec{r}$.
If the mass distribution is point-like or spherically symmetric, the potential reduces to
\begin{equation}
\quad \Phi_\mathrm{m}=-\dfrac{M}{r}\,,
\end{equation}
with its mass density $\rho_\mathrm{m}(\vec{r})$ distributed over the space region inside radius $r$, which is in accordance with the Newtonian term in Eq.\,\eqref{eq:potential}.



Now come back to the DE part. According to the time-time component of the Einstein equation\,\eqref{eq:EinsteinEqs}, 
\begin{equation}
\label{eq:00EinsteinEqs}
R^{0}_{\,\,\,0}=8\pi \left(T^{0}_{\,\,\,0}-\frac{1}{2}T^{\mu}_{\,\,\,\mu}\right)\,, 
\end{equation}
one finds
\begin{equation}
\partial_r\partial_r\Phi_w+\frac{2}{r}\partial_r\Phi_w=8\pi\left(\frac{1+3w}{2}\rho_w\right),
\end{equation}
which is equivalent to the following Poisson equation,
\begin{eqnarray}
\label{eq:DEPoissonEqs}
\nabla^2 \Phi_w&=&4\pi\,\left(\rho_w+3\,p_w\right)\,.
\end{eqnarray}
where $\rho_{w}$ and $p_{w}$ are presented in Eqs.~\eqref{eq:edensity} and~\eqref{eq:pdensity}.
Note that this equation holds exactly in GR. 


The total potential $\Phi$ can be divided into the matter and DE parts, namely 
\begin{eqnarray}
\Phi&\equiv&\Phi_\mathrm{m}+\Phi_w\\
&=&-\int\frac{\rho_\mathrm{m}(\vec{r^\prime})}{\mid \vec{r}-\vec{r^\prime}\mid}\d \vec{r^\prime}-\left(\frac{r_{\mbox{o}}}{r}\right)^{3w+1},
\label{eq:totalPotential}
\end{eqnarray}
where each part takes a model-independent form.
Correspondingly, we have 
\begin{equation}
\rho=\rho_\mathrm{m}+\rho_{w},\,\quad\,p=p_\mathrm{m}+p_{w},
\end{equation}
where $\rho$ and $p$ are the total density and pressure of both matter and DE, respectively.
Accordingly, we can derive the Poisson equation,
\begin{equation}
\label{PoEtotal}
\nabla^2 \Phi=4\pi(\rho+3\,p)\,,
\end{equation}
which include both the contributions from matter and DE.
Thus, according to the additivity and linearity of the Poisson equation, we derive the most general form of the total potential, $\Phi=\Phi(\vec{r})$, exactly as shown in Eq.~\eqref{eq:totalPotential}.
This form can be applied directly to a real situation, without any further restrictions.
As it shows, the matter and DE terms are on the same footing;
both of them obey the Poisson equation~\eqref{PoEtotal}. 
However, there exist differences between them. For instance,
the matter term will induce an attractive force, whereas the DE term will generate a repulsive force, referred to as the {\it dark force} in the literature~\cite{Zhang:2021ygh}.





%%%%%%%%%%%%%%%%%%%%
%%%   Local repulsive effect
%%%%%%%%%%%%%%%%%%%%

\subsection{Local repulsive effect}\label{effect}

In the following, we will discuss the dark force and its astrophysical effects on galactic scales. 
Usually, the typical galactic scale is about $\sim100$ kpc. For example, the MW has a virial radius of $r_{\mbox{\scriptsize vir}}=256~\mbox{kpc}$~\cite{2016MNRAS.463.2623H}, 
which is often used to characterize the MW size. Given that rotation velocities (RVs) of stars or gas clouds bound to the MW is around $200~\mbox{km/s}\ll c\,$, the relativistic effect can be neglected. 
The potential is very weak, namely $\Phi\ll\,1$, so the weak field approximation holds well. 
In the MW region with $r\lesssim\,r_{\mbox{\scriptsize vir}}$, this is indeed a good approximation. Exactly, one has
\begin{equation}\label{eq:weak}
\mid\!\Phi_w\!\mid\,\lesssim\,\mid\!\Phi_\mathrm{m}\!\mid\,\sim\frac{M}{r}\sim V^2\lesssim10^{-6}\ll1\,. 
\end{equation}
In fact, this is supported by the fact that the outer MW part is not torn apart by the dark force. 
Denote $a=a(z)$ as the cosmological expansion factor.
For the MW, we have $\frac{|\Delta a|}{a}\sim\frac{|\Delta z|}{1+z}\sim Hr_{\mbox{\scriptsize vir}}=6\times10^{-5}$.
Thus, the evolution of $w$ with $z_{r}$ is negligible; in other words, $w$ can be regarded as a constant on the MW scale. 
So we can neglect the cosmic expansion effect on $\Phi$. 
As a result, under the weak field approximation, $\Phi$ can be treated as a traditional gravitational potential in analogy to what we have done in Newtonian gravity. 
Clearly, DE modifies the form of gravitational potential.




DE gives rise to a correction term $\Delta\Phi$ in the gravitational potential, namely $\Delta\Phi=\Phi_w$,
Now we consider its induced force. Generally, one has 
\begin{equation}\label{eq:force}
\vec{F}=-\vec{\nabla} \Phi=-\vec{\nabla} \Phi_\mathrm{m}-(3w+1)\,\frac{1}{r}\,\left(\frac{r_{\mbox{o}}}{r}\right)^{3w+1}\hat{e}_r,
\end{equation}
which holds well in the weak field approximation. 
For a point-like mass $M$ or for regions outside a
spherically symmetric mass-distribution,
the first term in Eq.\,\eqref{eq:force} is closely related to the total mass $M=M(r)$ within a radius of $r$ from the mass center by
\begin{equation}
-\vec{\nabla} \Phi_\mathrm{m}=-\frac{M}{r^2}\,\hat{e}_r\,,
\end{equation}
which is just the attractive Newtonian force.
The second term in Eq.~\eqref{eq:force} comes directly from the DE contribution.
The repulsive dark force can be exerted by this term.
In the weak field approximation, the dark force shown by~\cite{Zhang:2021ygh} takes the same form as the second term in Eq.~\eqref{eq:force}.
Eq.~\eqref{eq:force} shows that the dark force takes a model-independent form;
it can describe different DE models characterized by the EoS parameter: $w=-1$ for the cosmological constant model, $-1<w<-\frac{1}{3}$ for the quintessence model, and $w<-1$ for the phantom model.


For an isolated astrophysical system, the appearance of the dark force is unavoidable, and the repulsive force must be included into the total gravitational force, 
together with the matter contribution.
Interestingly, the dark force has a negative contribution to the RV, namely 
\begin{equation}
\Delta_{w} V^2(r)\equiv-\mid3w+1\mid\left(\frac{r_{\mbox{o}}}{r}\right)^{3w+1}\,,
\end{equation}
where the DE contribution increases significantly with $w$. 
We hereby can investigate the contribution of the dark force to various RCs on astrophysical scales.


Requiring the cancellation between the two forces, one can derive the critical radius. Letting $M_{\mbox{\scriptsize cri}}=M(r_{\mbox{\scriptsize cri}})$, one gets
\begin{equation}
\quad r_{\mbox{\scriptsize cri}}= r_{\mbox{o}}\left(\mid3w+1\mid\frac{r_{\mbox{o}}}{M_{\mbox{\scriptsize cri}}}\right)^{\frac{1}{3w}}\,. 
\end{equation}
which exactly coincides with~\cite{Zhang:2021ygh}.
In the special CC case, it reduces to
\begin{equation}
r_{\mbox{\scriptsize cri}}\,\Big{|}_{\Lambda}=\left(\frac{3GM_{\mbox{\scriptsize cri}}}{\Lambda}\right)^{\frac{1}{3}}\,,
\end{equation}
which agrees with~\cite{Ho:2015nsa}. The critical radius $r_{\mbox{\scriptsize cri}}$ is the typical scale of the dark force; usually, it decreases with $\mid\!w\!\mid$.
For the MW, $r_{\mbox{\scriptsize cri}}$ is about $\sim500~\mbox{kpc}$, estimated by using the CC model~\cite{Ho:2015nsa}.

For a point-like mass distribution, the DE effect on a RC can be characterized by
\begin{equation}\label{eq:scale}
\frac{\mid\Delta_{w} V^2(r)\mid}{V_{\mbox{\scriptsize N}}^2}=\left(\frac{r}{r_{\mbox{\scriptsize cri}}}\right)^{-3w}\,. 
\end{equation}
where $V_{\mbox{\scriptsize N}}$ is the RV obtained from the purely Newtonian force. 
The DE correction to the RV is $\frac{\mid\Delta_{w} V(r)\mid}{V_{\mbox{\scriptsize N}}}=\frac{1}{2}\frac{\mid\Delta_{w} V^2(r)\mid}{V_{\mbox{\scriptsize N}}^2}$.
By setting $w=-1\,$, one gets $\frac{\mid\Delta_{w} V(r)\mid}{V_{\mbox{\scriptsize N}}}\ge3\%$ when $r=0.4\,r_{\mbox{\scriptsize cri}}\,$. 
Here we assume that all the matter distributes in the region within $r=0.4\,r_{\mbox{\scriptsize cri}}$.
This ignores the mass distribution in the outer region with $r\ge0.4\,r_{\mbox{\scriptsize cri}}$.
As a result, via Eq.~\eqref{eq:scale}, we therefore set a lower limit on the matter contribution to the RC. 

For a spherically symmetric mass distribution, under the assumption of $M(r)\propto r$ in the outer region with $r\ge0.4\,r_{\mbox{\scriptsize cri}}$, one finds that $V_{\mbox{\scriptsize N}}$ does not change with $r$.
In fact, the RC in the outer region is not as flat as we assumed. Usually, it deceases slightly with $r$, as illustrated in Fig.~\ref{sample}. This means that we set an upper limit on the matter contribution to the RC. 
In this case, one has
\begin{equation}\label{eq:scale2}
\frac{\mid\Delta_{w} V^2(r)\mid}{V_{\mbox{\scriptsize N}}^2}
=\left(\frac{r}{r_{\mbox{\scriptsize cri}}}\right)^{-3w-1}\ge16\%\,
\end{equation}
Clearly, this DE effect gets enhanced significantly as $w$ becomes larger. 
Come back to the DE correction $\frac{\mid\Delta_{w} V(r)\mid}{V_{\mbox{\scriptsize N}}}$ to the RV. 
According to the CC model, it's about $\sim2\%$ for $r=0.2\,r_{\mbox{\scriptsize cri}}$ and about $\sim8\%$ for $r=0.4\,r_{\mbox{\scriptsize cri}}$. 
Note that the present precision of the RC is about $\sim3\%-8\%$ within $r=0.2\,r_{\mbox{\scriptsize cri}}\sim100~\mbox{kpc}$~\cite{2016MNRAS.463.2623H}.
In the MW galaxy, the influence of dark force on the RC is large enough to be detected at this precision level.



Besides, the boundary of a gravitationally self-bounded system can be characterized by the critical radius $r_{\mbox{\scriptsize cri}}$.
Outside this boundary, no objects can be treated as being gravitationally bounded to the system, and will be influenced significantly by nearby gravitational systems. 
Actually, the influence of neighboring systems may be already non-ignorable at $r\lesssim\!\,r_{\mbox{\scriptsize cri}}$.
Thus we need to introduce an effective radius, denoted by $r_{\mbox{\scriptsize eff}}$, 
to define an ideal isolated region where the gravitational effect from any other gravitational system can be ignored.
Let $n_{\mbox{\scriptsize eff}}=r_{\mbox{\scriptsize eff}}/r_{\mbox{\scriptsize cri}}$. In general, it ranges from $\sim0.2$ to $\sim1.7$~\cite{He:2017alg}.
For an ideal isolated system, the upper limit of $n_{\mbox{\scriptsize eff}}$ is $\sim1.7$ in the CC model~\cite{He:2017alg}. 
The index $n_{\mbox{\scriptsize eff}}$ can be used to judge the extent to which the chosen region is gravitationally isolated from surrounding regions or systems.
Its value can be chosen properly to avoid tidal effects from nearby astrophysical systems, so that the region with $r\lesssim\!\,r_{\mbox{\scriptsize eff}}$ behaves as an isolated system.


%%%%%%%%%%%%%%%%%%%%
%%%%%%%%%%%%%%%%%%%%
%%%   Data and fits
%%%%%%%%%%%%%%%%%%%%
%%%%%%%%%%%%%%%%%%%%


\section{Data and fits}
\label{Danalysis}


To obtain the RCs in the Milk Way, we use two sets of data: one for the inner MW region with $r\sim4.5-100~\mbox{kpc}$ and the other one for the outer MW region with $r\sim100-200~\mbox{kpc}$. The data for $r<100~\mbox{kpc}$ come from~\cite{2016MNRAS.463.2623H}, 
while for $r>100~\mbox{kpc}$ from~\cite{1992MNRAS.255..105K,Sawa:2004tx,Sofue:2013kja}. 
At larger radii than $\sim200~\mbox{kpc}$, there seems to be an increase in the RV value. 
Especially, beyond $r\sim r_{\mbox{\scriptsize vir}}$,  the RC shows an abnormal rise, 
indicating that the influence of nearby galaxies become strong.
This is also the reason why we often choose the viral radius to characterize the outer MW boundary.
To minimize the influence of nearby galaxies, we choose $n_{\mbox{\scriptsize eff}}=0.4$ to define an isolated region. 
Accordingly, we only use the data points between $4.5-200~\mbox{kpc}$ to carry out a data analysis.


For constructing the RC in the outer MW region, beyond the Galactic disc, we have to rely on non-disc tracers like satellite galaxies that do not exhibit systematic motion~\cite{1992MNRAS.255..105K,Sawa:2004tx}.
For instance, it is the dwarf galaxies of the MW that allow us to extend the RC to $\sim 200$ kpc~\cite{2012AJ....144....4M}. 
These non-disc tracers move in various non-circular orbits around the MW center. This may yield systematic uncertainties in the RV measurements. 
However, it may be sufficient to describe the galactic structure for a first approximation~\cite{Sofue:2013kja}. 
Further dynamical studies on non-disc tracers will help us to understand these orbits and develop effective approaches 
for obtaining accurate RVs with reliable uncertainties at large radii~\cite{Jenkins:2020blc,Oakes:2022xcv}.
In addition we can also start long pointed observations of the non-disc tracers like satellite galaxies using current or future telescopes,
and make precise measurements for these non-circular orbits.
Anyway, to illustrate how to measure $w$ through RCs, we only use the published RC data. 


\subsection{Benchmark}\label{benchmark}
As shown by the second term in Eq.\,\eqref{eq:force}, the dark force takes a model-independent form 
\begin{equation}
\vec{F}_w=-\vec{\nabla} \Phi_w=-(3w+1)\frac{r_{\mbox{o}}^{3w+1}}{r^{3w+2}}\hat{e}_r\,,
\end{equation}
where the parameter $r_{\mbox{o}}$ may vary in the different models~\cite{Zhang:2021ygh}.
In the CC model, one has $\Lambda=4.24\times10^{-66}~\mbox{eV}^2$~\cite{Aghanim:2018eyx}.
Thus, $r_{\mbox{o}}=\sqrt{\frac{6}{\Lambda}}=7.71\times10^6~\mbox{kpc}$,
which is taken as a benchmark point in this work.


Before a specific data fit, 
there is no reason to believe that we already know the mass distribution in the outer MW region very well.
Thus we cannot estimate the critical radius directly if there is not any assumption made for the dark force. 
However, we can make an order-of-magnitude estimate of the critical radius. 
To do this, we need to assume that in the outer MW region the RV value keeps to be of the same order of magnitude as the Newtonian RV value at $r=r_{0}$; that is, $V(r)\sim\,V_{0}=V_{\textrm{N}}(r_{0})$.
Here, $V_{\mbox{\scriptsize N}}$ is the RV value derived from the purely Newtonian force.
In this case, by definition, the critical radius can be roughly estimated as
\begin{equation}\label{eq:critical radius}
V_{0}^2\sim\left|3w+1\right|\left(\frac{r_{\mbox{o}}}{r_{\mbox{\scriptsize cri}}}\right)^{3w+1}\,. 
\end{equation}
For the MW galaxy, set $V_{0}=200~\mbox{km/s}$. Then, the critical radius is about $\sim3600~\mbox{kpc}$ for $w=-1.0$, and $\sim176~\mbox{kpc}$ for $w=-0.8$. 
This means that if $w$ is as small as $-1.0$, the dark force will have little influence on the MW galaxy, whereas if $w$ is as large as $-0.8$, the outer MW part would be torn apart by the dark force before the MW galaxy could form. 
So $w$ cannot be too large, namely $w\lesssim-0.8$; otherwise, the matter in the outer region would overcome the net force and escape from the MW galaxy directly.




\begin{figure}
\centerline{
\includegraphics[width=0.68\columnwidth]{figure2.pdf}
}
\caption{Fitting to the MW rotation curve. 
Here we adopt data from~\cite{2016MNRAS.463.2623H} (black dots), and~\cite{Sofue:2013kja} (orange dots). 
The bulge, disc, halo, and DE components are shown as the dotted lines in grey, magenta, purple and red, respectively. 
The unshaded region represents the outer MW region with $r\gtrsim100~\mbox{kpc}$. In particular the red line stands for the negative contribution to the MW rotation curve from the dark force. 
}
\label{fig:component}
\end{figure}





\subsection{Galactic mass model}


Following~\cite{Calcino:2018mwh}, we assume that the MW mass distribution consists of three components, i. e., a bulge, a disc and a dark matter halo, and it extends to $\sim100-200~\mbox{kpc}$ continuously. 
Note that there are two localized dips at $r\sim11~\textrm{and}~19$ kpc in the MW rotation curve, respectively. 
Extra components could be included to interpret these small-scale structures in the innermost WM region with $r\lesssim20$ kpc.
However, our aim is to detect the dark force by the RC fitting.
To the current precision of $\sim3\%-8\%$, the dark force begins to be detectable at $r\sim100\,\textrm{kpc}$,
which is much larger than the size of each localized dip or that of the innermost region. 
Therefore, the dark force is insensitive to these structures in the innermost region. 
So we can ignore their influence on $w$ and then parametrize the RC as the three-component mass model with the bulge, disc and dark matter halo for the RC fitting.
Then the Newtonian RVs contributed these three components can be given by
\begin{equation}
V_{\mbox{\scriptsize N}}^2=V_{\mbox{\scriptsize b}}^2+V_{\mbox{\scriptsize d}}^2+V_{\mbox{\scriptsize h}}^2\,. 
\end{equation}

There are three main MW components, which are described briefly below:

(a)~\textrm{The bulge}.
As shown in~\cite{McMillan:2011wd}, the bulge is close to axisymmetric. Note that its scale radius is about $r_{\mbox{\scriptsize cut}}\sim2.1~\mbox{kpc}$, 
which is much smaller than the characteristic radius that we are interested in, i.e., about $\sim100\,\textrm{kpc}$ . 
And as the radius increases, the internal structure of the bulge becomes quite unimportant. So we can approximately treat the bulge as a point-like mass of $M_{\mbox{{\scriptsize bulge}}}=8.9\times 10^9M_{\odot}\,$. Then,
\begin{equation}\label{eq:bulge}
\frac{V_{\mbox{\scriptsize b}}(r)}{\mbox{km/s}}=196\times\left(\frac{r}{\mbox{kpc}}\right)^{{-1/2}}\,. 
\end{equation}
In this way, $V_{\mbox{\scriptsize b}}$ may be overestimated a bit. 
However, it cannot lead to a deviation $\Delta\,V_{\mbox{\scriptsize b}}$ of more than $\sim2\%$ from the estimated value of $V_{\mbox{\scriptsize b}}$ from the density profile presented in~\cite{2016MNRAS.463.2623H}.
Namely, we have $\frac{\Delta\,V_{\mbox{\scriptsize b}}}{V_{\mbox{\scriptsize b}}}\lesssim2\%$ in the outer MW region with $r\geq100\,\textrm{kpc}$, 
which can be verified by strict numerical calculations. Given that $V_{\mbox{\scriptsize b}}$ is about one order of magnitude smaller than $V_{\mbox{\scriptsize N}}$ in this region, we can safely draw the conclusion that the deviation from the total RV value is less than $0.1\%$. Thus, for the sake of simplicity, we choose to characterize the bulge with a point-like mass model.



(b)~~\textrm{The disc}.
The disc component can be described by the surface-density profile~\cite{Dehnen:1996fa}
\begin{equation}
\Sigma_{\mbox{\scriptsize d}}(r)=\Sigma_{\mbox{\scriptsize d,0}}\exp(-r/r_{\mbox{\scriptsize d}})\,, 
\end{equation}
with a central surface density $\Sigma_{\mbox{\scriptsize d,0}}$ and a scale length $r_{\mbox{\scriptsize d}}\,$. 
Following~\cite{2016MNRAS.463.2623H}, we fix the local surface density of disc to be $54.4~M_{\odot}\,\mbox{pc}^{-2}$ at $r=8.34~\mbox{kpc}$ so as to match with the observations.
Then the circular velocity is given by
\begin{equation}\label{eq:disc}
V_{\mbox{\small c}}^ {2}(r) = 4\pi G\Sigma_{\mbox{\scriptsize d,0}}r_{\mbox{\scriptsize d}}y^{2}\Big[I_{0}(y)K_{0}(y)-I_{1}(y)K_1(y)\Big]\,,
\end{equation}
with $y=r/(2r_{\mbox{\scriptsize d}})\,$, where $I_n$ and $K_n$ ($n=0, 1$) are modified Bessel functions of the first and second kind, respectively. 


(c)~~\textrm{The dark matter halo}.
We adopt the NFW density profile~\cite{Navarro:1996gj} to describe the dark matter halo: 
\begin{equation}\label{eq:halo}
\rho_{\mbox{\scriptsize h}}(r)=\rho_{\mbox{\scriptsize h,0}}(r/r_{\mbox{\scriptsize h}})^{-1}(1+r/r_{\mbox{\scriptsize h}})^{-2}\,, 
\end{equation}
Thus, the contribution to the circular velocity of the halo can be computed using
\begin{equation}
V_{\mbox{\scriptsize h}}^2=\frac{4\pi\rho_{\mbox{\scriptsize h,0}}r_{\mbox{\scriptsize h}}^3}{r}\left(\ln\frac{r_{\mbox{\scriptsize h}}+r}{r_{\mbox{\scriptsize h}}}-\frac{r}{r+r_{\mbox{\scriptsize h}}}\right)\,. 
\end{equation}
The actual profile of the dark matter density may differ from the NFW profile, but the asymptotic behavior of $r^{-3}$ is widely accepted~\cite{Ho:2015nsa}. 
Indeed, our results below will be insensitive to the density profile at small $r$.




\subsection{Circular velocities}
Finally, we also need to add to the circular velocity the DE correction term. Consequently, the total circular velocity $V_{\mbox{\small c}}$ can be rewritten as
\begin{equation}
V_{\mbox{\small c}}^2=V_{\mbox{\scriptsize N}}^2+\Delta_{w}V^2\,. 
\end{equation}


Up to now, we have parametrized the MW rotation curve.
In total, there are four free parameters used for fitting: one for the disc ($r_{\mbox{\scriptsize d}}$), two for the NFW halo ($\rho_{\mbox{\scriptsize h,0}}\,$, $r_{\mbox{\scriptsize h}}$), and one for the dark force ($w$). 
For the fitting procedure, we use the Levenberg-Marquardt algorithm to find the best-fitting values of the four parameters. 
Furthermore, a Markov chain Monte Carlo technique is used to sample the likelihood of the data~\cite{2010CAMCS...5...65G}.
The likelihood function is defined as
\begin{equation}
\mathcal{L}=\prod\limits_{i=1}^{N}\frac{1}{\sqrt{2\pi}\sigma_{V_{\mbox{\small c}, r_i}^{\mbox{\scriptsize obs}}}}\exp\frac{-[V_{\mbox{\small c}, r_i}^{\mbox{\scriptsize obs}}-V_{\mbox{\small c}, r_i}^{\mbox{\scriptsize model}}(\hat{\theta})]^2}{2\sigma_{V_{\mbox{\small c}, r_i}^{\mbox{\scriptsize obs}}}^2}\,, 
\end{equation}
where $N$ is the number of data points used in our fit, $\sigma_{V_{\mbox{\small c}, r_i}^{\mbox{\scriptsize obs}}}$ is the uncertainty of the observed RV $V_{\mbox{\small c}, r_i}^{\mbox{\scriptsize obs}}$ and $\hat{\theta}$ represents the four fitting parameters that we want to determine. Actually, we make use of the PYTHON package LMFIT~\cite{Foreman-Mackey:2012any} to find the best-fitting values and compute the confidence intervals as well as estimate the upper and lower bounds on the RC. 




\begin{table}
\centerline{
\begin{tabular}{|c|cccc|c|}
\hline
range&$r_{\rm{\scriptsize d}} {\rm(kpc)}$&$\rho_{\mbox{\scriptsize h,0}}$ ($M_{\odot}\,{\rm pc}^{-3}$)&$r_{\mbox{\scriptsize h}} {\rm(kpc)}$&$w$&$\chi^2_{\mbox{\scriptsize red}}$\\
\hline
4.5-200~kpc&$2.9_{-0.1}^{+0.2}$&$0.011_{-0.003}^{+0.002}$&$18_{-3}^{+1}$&$-0.82_{-0.01}^{+0.01}$&$0.85$\\
\hline
4.5-100~kpc&$2.8_{-0.1}^{+0.1}$&$0.006_{-0.002}^{+0.004}$&$24_{-6}^{+8}$&$-0.79_{-0.02}^{+0.01}$&$0.85$\\
\hline
\end{tabular}
}
\caption{Best-fit parameters of the disc, halo, and dark energy components, obtained from the RCs from $r\sim4.5$ to $200~\mbox{kpc}$ and to $100~\mbox{kpc}$, respectively. 
The errors are at 1$\sigma$, and the reduced $\chi^2\,$ are shown in the last column. }
\label{tab:fit}
\end{table}


%%%%%%%%%%%%%%%%%%%%
%%%%%%%%%%%%%%%%%%%%
%%%   Results and Discussion
%%%%%%%%%%%%%%%%%%%%
%%%%%%%%%%%%%%%%%%%%

\section{Results and Discussion}\label{result}


\begin{figure}
\centerline{
\includegraphics[width=0.68\columnwidth]{figure3.pdf}
}
\caption{The best-fit results of $w$ at different $\alpha$ values, where the uncertainties at $r\sim100-200~\mbox{kpc}$ are assumed to be $\alpha$ times the observed ones, with $\alpha$ ranging from 0.3 to 1. The black line represents the best-fit line. The 1$\sigma$ and 2$\sigma$ intervals are shown in dark and light red, respectively. }
\label{fig:NFW2sigma}
\end{figure}




By fitting the Huang et al. 2016 and Sofue 2013 data, we can obtain the parameters of the galactic mass model as well as the EoS parameter $w$.
The best-fitting values of these parameters and their uncertainties are listed in Table~\ref{tab:fit}. 
We also illustrate the contributions from different MW components in Fig.~\ref{fig:component}. 

For the disc, we adopt the same disc model as in~\cite{Calcino:2018mwh}. 
As expected, the fitted value of the disc parameter $r_{\mbox{\scriptsize d}}$ in this model agrees with that shown in~\cite{Calcino:2018mwh} within 1$\sigma$ errors.
For the dark matter halo, the best-fit halo parameters are determined to be $\rho_{\mbox{\scriptsize h,0}}=0.015_{-0.004}^{+0.007}~M_{\odot}\,\mbox{pc}^{-3}$ and $r_{\mbox{\scriptsize h}}=15_{-3}^{+4}$ kpc.
Note that the NFW model~\eqref{eq:halo} is widely used in previous works~\cite{2016MNRAS.463.2623H,Calcino:2018mwh}.
Within 1$\sigma$ errors, our values of the halo parameters are also consistent with~\cite{2016MNRAS.463.2623H} and~\cite{Calcino:2018mwh}.

The local dark matter density $\rho_{\odot,\,\mbox{\scriptsize dm}}$ at the Sun’s galactocentric radius $r=8.34~\mbox{kpc}$ is obtained without considering the contribution of the dark force. Traditionally, it have been shown that $\rho_{\odot,\,\mbox{\scriptsize dm}}=0.32\pm0.02~\mbox{GeV cm}^{-3}$ in~\cite{2016MNRAS.463.2623H} 
and $\rho_{\odot,\,\mbox{\scriptsize dm}}=0.24^{+0.10}_{-0.09}~\mbox{GeV cm}^{-3}$ in~\cite{Calcino:2018mwh}, respectively.
From $\rho_{\mbox{\scriptsize h,0}}$ and $r_{\mbox{\scriptsize h}}$, we can get the local dark matter density $\rho_{\odot,\,\mbox{\scriptsize dm}}$ directly via Eq.\,\eqref{eq:halo}. 
Exactly, one has $\rho_{\odot,\,\mbox{\scriptsize dm}}=0.40_{-0.15}^{+0.07}~\mbox{GeV cm}^{-3}$. Clearly, it coincides with the traditional result within 1$\sigma$ errors.



The observational data between $r=4.5$ and $200~\mbox{kpc}$ sets a tight constraint on the DE's EoS parameter $w$. As shown in Table~\ref{tab:fit}, $w=-0.82_{-0.01}^{+0.01}$.
It is noteworthy that $w$ is obtained independently of any specific DE models.
When $w\gtrsim-0.85$, the repulsive dark force increases much more significantly in the force strength with $w$ than the attractive Newtonian force with other parameters.
In this case, the fitting results are more sensitive to $w$ than the other parameters. 
Thus, compared to other parameters, there is a much tighter constraint on the DE parameter $w$.


In addition, there have been cosmological measurements on the EoS parameter for the dynamical DE within $w(z_{r})$CDM, namely $w=-1.07_{-0.20}^{+0.21}$ at $z_{r}=0$~\cite{Zhao:2017cud},
where the evolution history of the EoS parameter, namely $w=w(z_{r})$, is reconstructed from a collection of cosmological data
by using a non-parametric Bayesian method based on applying a correlated prior~\cite{2012PhRvL.109q1301Z}. 
At 1$\sigma$ confidence level, it shows a little disagreement with our fitting value.
However, the two values coincide with each other within $\sim1.2\sigma$ errors. 
To some extent, we cross-check and confirm the cosmological measurements through measuring the effects of the dark force in the MW galaxy.
Anyway, we have proposed a new method to measure the DE parameter $w$, independently of cosmological observations.


The dark force can lead to many effects in galaxies, and become strong at large galactocentric distances. 
As Eq.~\eqref{eq:scale2} shows, the dark force effect on the RC increases with $r$ in the form of $\,r^{|-3w-1|}$. 
By choosing $r$ properly, it can be enhanced by several times within $r=r_{{\rm eff}}$. 
However, the measurements of RVs at large distances are usually less precise than those at small distances. 
Denote $\Delta V_{\mbox{\scriptsize exp}}=\Delta V_{\mbox{\scriptsize exp}}(r)$ as the uncertainty of RVs at radius $r$.
As shown in Fig.~\ref{fig:component}, the uncertainty $\Delta V_{\mbox{\scriptsize exp}}$ is $\sim20~\mbox{km/s}$ at radius $r\sim20-100~\mbox{kpc}$,  
and $\sim60~\mbox{km/s}$ at radius $r\sim100-200~\mbox{kpc}$. The large uncertainties of the RVs at large galactocentric distances make it difficult to detect the dark force effect.


More measurements will be done with current telescopes, like James Webb Space Telescope~\cite{Boccaletti_2015} and Very Large Telescope Interferometer~\cite{2017A&A...602A..94G}, as well as by future telescopes, such as European Extremely Large Telescope~\cite{2021JATIS...7c5005R}, Thirty Meter Telescope~\cite{2020SPIE11447E..2ZR}, Giant Magellan Telescope~\cite{2016SPIE.9908E..1UJ}, and Legacy Survey of Space and Time~\cite{LSST:2008ijt}, in the outer MW region with $r\sim100-200$~kpc. 
It can be expected that better data and more data can be obtained by these telescopes, and the accuracy of measurements will be improved significantly in the near future. 

Assume the expected accuracy to be improved by a factor $\alpha$, namely $\Delta V_{\mbox{\scriptsize th}}(r)=\alpha\,\Delta V_{\mbox{\scriptsize exp}}$. 
Let us change the value of $\alpha$ and show how the dark force exerts its influence on the MW rotation curve to different accuracies.
If the factor $\alpha$ decreases to 0.3 from 1, $\Delta V_{\mbox{\scriptsize th}}$ at radius $r\sim100-200$~kpc will become compared with the observed uncertainties at $r\lesssim 100~\mbox{kpc}$.
Indeed, for each $\alpha$ value, we fit the MW rotation curve and obtain the fitting result.
As shown in Fig.~\ref{fig:NFW2sigma}, the best-fit value of $w$ shows little change with $\alpha$, whereas its uncertainty seems to decrease significantly with $\alpha$.






For comparison, we also perform a fit to the MW rotation curve only using the galactic mass model.
In this case, the contribution from the dark force is completely neglected. 
Fig.~\ref{fig:emcee} shows the difference between the two cases; the RC in the case with the contribution of the dark force included in the fitted model tends to drop faster than that without including the contribution of the dark force into the fitted model, especially in the outer MW region with $r\sim100-200$~kpc.
When $\alpha=15\%$, the RCs in the two cases will show a significant difference at the $\sim2.4\sigma$ confidence level in the outer MW region. 
As demonstrated in Fig.~\ref{fig:emcee}, the difference clearly implies that the anomalous drop of the RC in the outer MW region can be explained as the dark force effect. 
This difference become rather significant at large distances, i.e. larger than 150~kpc. 
So we need to do more high-accuracy measurements in the outer MW region to provide robust evidence for DE in the future.



To well understand the uncertainty in the RC fitting, 
we also only fit the RV data from $r=4.5$ to $100~\mbox{kpc}$. 
The corresponding result is also included in Table~\ref{tab:fit}. 
As the table shows, $w\sim-0.79^{+0.01}_{-0.02}$. 
This value is slightly different from the result obtained between $r=4.5$ and $200~\mbox{kpc}$ at the 1$\sigma$ confidence level.
However, they coincide with each other at the 2$\sigma$ confidence level.
In either fit, the value of the EoS parameter deviates significantly from $w=-1.028\pm0.031$ measured by fitting the $w$CDM model to the cosmological data~\cite{Aghanim:2018eyx}, 
although it still agrees with~\cite{Zhao:2017cud} within $\sim1.2\sigma$ errors.
This deviation can be a hint for the existence of the DE beyond the CC model, 
otherwise it clearly indicates some mistakes that have been made in determining the dark matter profile in the Milk Way galaxy.



\begin{figure}
\centerline{
\includegraphics[width=0.68\columnwidth]{figure4.pdf}
}
\caption{Fits to the MW rotation curve, with and without the DE contribution included in the fitted models, respectively. In either case, the solid lines represent 1$\sigma$ bounds, and the dashed lines 2$\sigma$ bounds. In the fitting procedure, the errors of RVs at $r\sim100-200~\mbox{kpc}$ are ideally assumed to be $\alpha=15\%$ times their presently measured ones. }
\label{fig:emcee}
\end{figure}



%%%%%%%%%%%%%%%%%%%%
%%%%%%%%%%%%%%%%%%%%
%%%   Conclusions
%%%%%%%%%%%%%%%%%%%%
%%%%%%%%%%%%%%%%%%%%


\section{Conclusions}\label{discussion} 




The mysterious DE poses a great challenge to modern science. 
So far various efforts have been devoted to explore the DE's origin and nature.
In this work, we compared the role of DE in an expanding universe and that in a gravitationally self-bounded system, investigated the dark force induced by DE,
and made our attempts at determing its EoS parameter $w$ on astrophysical scales through measuring the effects of the dark force on RCs. 




First, we extended the standard RW metric to the ERW one that includes the contributions from matter and DE. 
Then, based on the Einstein equation, we derived the generalized Friedmann equation in the ERW spacetime, 
with its curvature term expressed as the average of the sectional curvatures, as well as presented the generalized conservation equations,
and found from these basic equations in cosmology that the cosmological evolution is determined by the dynamical pressure, 
where the dynamical pressure is defined as negative one-third the trace of the stress tensor components.





Second, we further investigated the SdS$_{w}$ spacetime and explored the origin and nature of the dark force.
Under gravitational fields, DE may redistribute itself, especially when $w\neq-1$ in the isolated gravitational system of a point-like mass,
otherwise it will violate the constraint from the Einstein equation. 
However, in this general case, the energy-momentum tensor for DE can still take the standard Weinberg’s isotropic form, 
which is form-invariant under general spatial rotations. 
In a curved spacetime, the EoS parameter $w$ cannot be defined in the same way as that of the traditional perfect fluid in fluid mechanism.
Nevertheless it can be defined as the ratio between the dynamical pressure and the energy density, which is compatible with current theories and cosmological observations.
Accordingly we established a connection in the EoS parameter between the DE in the ERW universe and its counterpart in the SdS$_{w}$ spacetime,
laying a theoretical foundation for exploring the origin and nature of the dark force. 



Third, we investigated the dark force in analogy to what we have done in Newtonian gravity.
Generally, the matter potential obeys the Poisson equation. 
In fact, the most general expression for the matter potential can be obtained by integrating the Poisson equation. 
Additionally, by solving the Einstein equation, we also demonstrated that the gravitational potential of the DE with a general EoS parameter $w$ satisfies the Poisson equation.
Then, according to the additivity and linearity, we derived a model-independent form for the total gravitational potential including both the matter and DE terms.
The analytical form of the dark force can therefore be obtained from the gradient of the total gravitational potential.
Then we further studied the repulsion of the dark force, 
and found that, for a galaxy, the RVs of any astrophysical objects bound to this galaxy can be affected by the dark force:
the observed RVs should be smaller than those predicted by the purely Newtonian force.


Then, by the repulsion of the dark force, we newly proposed a method to detect DE on astrophysical scales, 
independently of specific DE models, and found that the RCs can be used to measure or constrain the DE parameter $w$.
With high accuracy, we obtained $w\sim-0.8$ by fitting the MW rotation curve, 
although there are large errors in the observed RV values, especially at galactocentric distances larger than 100~kpc.
This mothod is also cosmology-independent. 
Within $\sim1.2\sigma$ errors, the measured value using our method is consistent with that measured from the fit to the cosmological data within $w(z_{r})$CDM~\cite{Zhao:2017cud},
However, we found its deviation from $w=-1$.
This can be interpreted as evidence for the deviation of DE from the cosmological constant,
unless some serious mistakes have been made in understanding the halo profile of the dark matter in the Milk Way.
%This can be interpreted as evidence for a non-constant dynamical DE, or else it signals our mistakes in understanding the DM halo in the Milk Way. %



Finally, we discussed the prospects for using our method to probe DE in the MW galaxy. 
As more measurements on RVs take place at large radii, the DE parameter $w$ will be measured to a higher accuracy.
When the accuracy is increased to a certain degree, the best-fit RC which includes the dark force contribution will deviate from that does not include the dark force contribution, especially in the outer MW region with $r\sim100-200$~kpc. 
For example, once the uncertainties of RVs in the outer MW region decrease to $\sim15\%$ of the present-day ones, the deviation can be confirmed at $\sim2.4\sigma$ confidence level. It clearly indicates that future prospects for the DE detection through the effects of the dark force on astrophysical scales are quite bright.




%%%%%%%%%%%%%%%%%%%%
%%%%%%%%%%%%%%%%%%%%
%%%   Acknowledgements
%%%%%%%%%%%%%%%%%%%%
%%%%%%%%%%%%%%%%%%%%



\section{Acknowledgements}
This work is partially supported by the National Program on Key Research and Development Project (Grant No. 2021YFA0718500) from the Minister of Science and Technology of China and the International Partnership Program of the Chinese Academy of Sciences (Grant No.113111KYSB20190020). RZ acknowledges the support by the National Science Foundation of China (grant No. 12075257) as well as the funding from the Institute of High Energy Physics, Chinese Academy of Sciences (grant No. Y6515580U1) and the funding from Chinese Academy of Sciences (grant No. Y8291120K2). ZZ acknowledges the support by the Institute of High Energy Physics (grant No. E25155U1) and the support by the Strategic Priority Research Program on Space Science of the Chinese Academy of Sciences (grant No. XDA15052700).
\bibliography{draft}
\bibliographystyle{JHEP}


%%%%%%%%%%%%%%%%%%%%
%%%%%%%%%%%%%%%%%%%%
%%%   Appendix
%%%%%%%%%%%%%%%%%%%%
%%%%%%%%%%%%%%%%%%%%

\appendix

%\onecolumngrid

\newpage
%\noindent



\section{Conservation of energy and momentum}
\label{App:A}


Generally, a spacetime metic can be written as
\begin{eqnarray}
\label{eq:metric}
%\begin{array}{rcl}
%\displaystyle
%\nonumber
\d s^2=g_{\mu\nu}\,\d x^{\mu}\d x^{\nu},
%\end{array}
\end{eqnarray}
with $\mu,\nu=0,1,2,3$, where $g_{\mu\nu}$ is the covariant component of the metric tensor. 
As shown in Eq.~\eqref{eq:ERW}, the nonzero components of the ERW metric and the inverse metric are:
\begin{eqnarray}
\label{eq:ERWmetricC}
%\begin{array}{rcl}
%\displaystyle
%\nonumber
g_{00}&\!=\!&Z^2\left(x,y,z\right),~ g_{ii}=-a^2\left(t\right)\,R^2\left(x,y,z\right),~~~\\
g^{00}&\!=\!&\frac{1}{Z^2\left(x,y,z\right)},~ g^{ii}=-\frac{1}{a^{2}\left(t\right)R^{2}\left(x,y,z\right)}.
%\end{array}
\end{eqnarray}
From these we can get the Christoffel symbols, given by 
\begin{eqnarray}
\label{eq:Christoffel}
%\begin{array}{rcl}
%\displaystyle
%\nonumber
\Gamma^{\lambda}_{\,\,\,\mu\nu}=\frac{1}{2}\,g^{\lambda\rho}\left(\partial_{\mu}\,g_{\rho\nu}+\partial_{\nu}\,g_{\rho\mu}-\partial_{\rho}\,g_{\mu\nu}\right).
%\end{array}
\end{eqnarray}
Exactly, they are
\begin{eqnarray}
\label{eq:CgristoffelSs}
%\begin{array}{rcl}
%\displaystyle
\nonumber
\Gamma^0_{\,\,\,ii}&=&-\frac{1}{2}\,g^{00}\,\partial_0\left(g_{ii}\right),\quad \Gamma^i_{\,\,\,0i}=+\frac{1}{2}\,g^{ii}\,\partial_0\left(g_{ii}\right)=\Gamma^i_{\,\,\,i0}\,, \nonumber\\
\nonumber
\Gamma^i_{\,\,\,ii}&=&+\frac{1}{2}\,g^{ii}\,\partial_i\left(g_{ii}\right),~\quad \Gamma^i_{\,\,\,jj}=-\frac{1}{2}\,g^{ii}\,\partial_i\left(g_{jj}\right)\,, \nonumber\\
%\nonumber
\Gamma^i_{\,\,\,ij}&=&+\frac{1}{2}\,g^{ii}\,\partial_j\left(g_{ii}\right)=\Gamma^i_{\,\,\,ji}. 
%\end{array}
\end{eqnarray}

As a consequence of conservation of energy and momentum, $T^{\mu\nu}$ satisfies the exact conservation equation:
\begin{align}
\label{eq:S1:EPConservation}
0=\nabla_{\mu}\,T^{\mu\nu}=\partial_{\mu}\,T^{\mu\nu}+\Gamma^{\mu}_{\,\,\,\mu\rho}\,T^{\rho\nu}+\Gamma^{\nu}_{\,\,\,\mu\rho}\,T^{\mu\rho}.
\end{align}
For $\nu=0$, we therefore have
\begin{align}\label{eq:S1:EConservation0}
0=&\partial_\nu T^{0\nu}+\Gamma^0_{\,\,\,\nu\rho}T^{\rho\nu}+\Gamma^\nu_{\,\,\,\nu0}T^{00}\nonumber\\
=&\partial_0 T^{00}+\partial_i T^{0i}-\frac{1}{2}\partial^0(g_{ii})T^{ii}+\frac{1}{2}g^{ii}\partial_0(g_{ii})T^{00}\nonumber\\
=&\partial_0 (g^{00}T^{0}_{\,\,\,0})+\partial_i (g^{00}T^{i}_0)-\frac{1}{2}g^{00}\partial_0(g_{ii})g^{ii}T^{i}_{\,\,\,i}+\frac{1}{2}g^{ii}\partial_0(g_{ii})g^{00}T^{0}_{\,\,\,0}\nonumber\\
=&g^{00}\bigg[\partial_0 T^{0}_{\,\,\,0}+\frac{1}{g^{00}}\partial_i (g^{00}T^{i}_{\,\,\,0})-\frac{1}{2}\partial_0(g_{ii})g^{ii}T^{i}_{\,\,\,i}+\frac{1}{2}g^{ii}\partial_0(g_{ii})T^{0}_{\,\,\,0}\bigg],
\end{align}
where $i=1,2,3$. Substituting Eqs.~\eqref{eq:ERWmetricC} and~\eqref{eq:CgristoffelSs} into the last line gives
\begin{align}\label{eq:S1:EConservation1}
0=&\frac{\d T^{0}_{\,\,\,0}}{\d t}+\frac{1}{g^{00}}\,\partial_i\left(g^{00}T^{i}_{\,\,\,0}\right)-\frac{1}{2\left(R^2a^2\right)}\frac{\d \left(R^2a^2\right)}{\d t}T^{i}_{\,\,\,i}+\frac{3}{2\left(R^2a^2\right)}\frac{\d \left(R^2a^2\right)}{\d t}T^{0}_{\,\,\,0}\nonumber\\
=&\frac{\d T^{0}_{\,\,\,0}}{\d t}+\frac{3\dot{a}}{a}\left(T^0_{\,\,\,0}-\frac{1}{3}T^{i}_{\,\,\,i}\right)+\frac{1}{g^{00}}\partial_i \left(g^{00}T^{i}_{\,\,\,0}\right)\nonumber\\
=&\frac{\d T^{0}_{\,\,\,0}}{\d t}+3\,\frac{\dot{a}}{a}\left(T^0_{\,\,\,0}-\frac{1}{3}T^{i}_{\,\,\,i}\right)+\frac{2\partial_i Z}{Z}T^{i}_{\,\,\,0}+\partial_i T^{i}_{\,\,\,0}. %\\
\end{align}
For $\nu=i$, we obtain
\begin{align}\label{eq:S1:PConservation0}
0=&\partial_\nu T^{i\nu}+\Gamma^i_{\,\,\,\nu\rho}T^{\rho\nu}+\Gamma^\nu_{\,\,\,\nu\rho}T^{i\rho}\nonumber\\
=&(\partial_0 T^{i0}+\partial_j T^{ij})+(2\Gamma^i_{\,\,\,i0}T^{0i}+\Gamma^i_{\,\,\,ii}T^{ii}+\Gamma^i_{\,\,\,ll}T^{ll}+\Gamma^i_{\,\,\,il}T^{li})+(\Gamma^j_{\,\,\,j0}T^{i0}+\Gamma^k_{\,\,\,kj}T^{ij})\nonumber\\
=&\partial_j T^{ij}+\Gamma^i_{\,\,\,ii}T^{ii}+\Gamma^i_{\,\,\,ll}T^{ll}+\Gamma^i_{\,\,\,il}T^{li}+\Gamma^k_{\,\,\,kj}T^{ij}\nonumber\\
&+\partial_0 T^{i0}+\Gamma^j_{\,\,\,j0}T^{i0}+2\Gamma^i_{\,\,\,i0}T^{0i}\nonumber\\
=&\partial_j T^{ij}+\frac{1}{2}g^{ii}(\partial_ig_{ii})T^{ii}
-\frac{1}{2}g^{ii}(\partial_ig_{ll})T^{ll}+\frac{1}{2}g^{ii}(\partial_lg_{ii})T^{il}
+\frac{1}{2}g^{kk}(\partial_jg_{kk})T^{ij}\nonumber\\
&+\partial_0 T^{i0}+\frac{1}{2}g^{jj}(\partial_0g_{jj})T^{i0}+g^{ii}(\partial_0g_{ii})T^{0i}\nonumber\\
=&\partial_j T^{ij}+\frac{1}{2}g^{ii}(\partial_ig_{ii})T^{ii}
-\frac{1}{2}g^{ii}(\partial_ig_{ii})T^{jj}+\frac{1}{2}g^{ii}(\partial_jg_{ii})T^{ij}
+\frac{3}{2}g^{ii}(\partial_jg_{ii})T^{ij}\nonumber\\
&+\partial_0 T^{i0}+\frac{3}{2}g^{ii}(\partial_0g_{ii})T^{0i}+g^{ii}(\partial_0g_{ii})T^{0i}\nonumber\\
=&\partial_j T^{ij}+\frac{1}{2}g^{ii}(\partial_ig_{ii})(T^{ii}-T^{jj})+2g^{ii}(\partial_jg_{ii})T^{ij}\nonumber\\
&+\partial_0 T^{i0}+\frac{5}{2}g^{ii}(\partial_0g_{ii})T^{0i}\nonumber\\
=&\partial_j (g^{ii}T^{i}_{\,\,\,j})+\frac{1}{2}(g^{ii})^2(\partial_ig_{ii})(T^{i}_{\,\,\,i}-T^{j}_{\,\,\,j})+2(g^{ii})^2(\partial_jg_{ii})T^{i}_{\,\,\,j}\nonumber\\
&+\partial_0 (g^{ii}T^{0}_{\,\,\,i})+\frac{5}{2}(g^{ii})^2(\partial_0g_{ii})T^{0}_{\,\,\,i}\nonumber\\
=&g^{ii}\Big[\partial_j T^{i}_{\,\,\,j}+g_{ii}\partial_j (g^{ii})T^{i}_{\,\,\,j}+\frac{1}{2}g^{ii}(\partial_ig_{ii})(T^{i}_{\,\,\,i}-T^{j}_{\,\,\,j})
+2g^{ii}(\partial_jg_{ii})T^{i}_{\,\,\,j}\nonumber\\
&+\partial_0 T^{0}_{\,\,\,i}+g_{ii}\partial_0 (g^{ii})T^{0}_{\,\,\,i}+\frac{5}{2}g^{ii}(\partial_0g_{ii})T^{0}_{\,\,\,i}\Big]\nonumber\\
=&g^{ii}\Big[\partial_j T^{i}_{\,\,\,j}+\frac{1}{2}g^{ii}(\partial_ig_{ii})(T^{i}_{\,\,\,i}-T^{j}_{\,\,\,j})+g^{ii}(\partial_jg_{ii})T^{i}_{\,\,\,j}\nonumber\\
&+\partial_0 T^{0}_{\,\,\,i}+\frac{3}{2}\,g^{ii}(\partial_0g_{ii})T^{0}_{\,\,\,i}\Big],
\end{align}
where $i,j,k\in\{1,2,3\}$ and $l\in\{1,2,3\}-\{i\}$. Combining Eqs.~\eqref{eq:ERWmetricC} and~\eqref{eq:CgristoffelSs} with this equation~\eqref{eq:S1:PConservation0} yields
\begin{align}\label{eq:S1:PConservation1}
0=&\partial_j T^{i}_{\,\,\,j}+\frac{1}{2}g^{ii}(\partial_ig_{ii})(T^{i}_{\,\,\,i}-T^{j}_{\,\,\,j})+g^{ii}(\partial_jg_{ii})T^{i}_{\,\,\,j}
+\partial_t T^{0}_{\,\,\,i}+\frac{3}{2}g^{ii}(\partial_0g_{ii})T^{0}_{\,\,\,i}\nonumber\\
=&\partial_j T^{i}_{\,\,\,j}+\frac{1}{R}\frac{\d R}{\d x^i}(T^{i}_{\,\,\,i}-T^{j}_{\,\,\,j})+\frac{2}{R}\frac{\d R}{\d x^j}T^{i}_{\,\,\,j}
+\partial_t T^{0}_{\,\,\,i}+3\,\frac{\dot{a}}{a}T^{0}_{\,\,\,i}.
\end{align}


At any point $P$ in curved spacetime, there is a locally inertial frame around the point $P$, comoving with the fluid element. 
As seen by the comoving observer in this inertial frame, the $0i$ and $i0$ components of the energy-momentum tensor satisfy $T_{0i}=T_{i0}=0$~\cite{Maggiore:2018}. Thus, $T^{0}_{\,\,\,i}=T^{i}_{\,\,\,0}=0$. We therefore obtain
\begin{align}
\label{eq:S1:EConservation2}
0=&\frac{\d T^{0}_{\,\,\,0}}{\d t}+3\,\frac{\dot{a}}{a}\left(T^0_{\,\,\,0}-\frac{1}{3}T^{i}_{\,\,\,i}\right),\\
\label{eq:S1:PConservation2}
0=&\partial_j T^{i}_{\,\,\,j}+\frac{1}{R}\frac{\d R}{\d x^i}(T^{i}_{\,\,\,i}-T^{j}_{\,\,\,j})+\frac{2}{R}\frac{\d R}{\d x^j}T^{i}_{\,\,\,j},
\end{align}
directly from Eqs.~\eqref{eq:S1:EConservation1} and~\eqref{eq:S1:PConservation1}, respectively. In GR, the former expresses conservation of energy in the ERW universe, while the later corresponds to conservation of the $i$th component of the momentum. 


In the standard RW spacetime, the stress tensor of a perfect fluid is given by
\begin{align*}
T^{i}_{\,\,\,j}=\hat{\pi}\,\delta^{i}_{\,\,\,j}.
\end{align*} 
Plugging this expression into Eq.\,\eqref{eq:S1:PConservation2} gives
\begin{align*}
\partial_i\,\hat{\pi}=0,
\end{align*}
which is the well-known conservation equation that corresponds to conservation of momentum in the standard RW cosmology (see Appendix~\ref{App:D} for more details).




\section{The Friedmann equations}
\label{App:B}


Here we present the steps of deriving the Friedmann equations.
The first step we would take is to calculate the Ricci tensor from the formula
\begin{align*}
R_{\mu\sigma}&=\Gamma^\nu_{\,\,\,\mu\sigma,\nu}-\Gamma^\nu_{\,\,\,\sigma\nu,\mu}+\Gamma^\lambda_{\,\,\,\mu\sigma}\Gamma^\nu_{\,\,\,\lambda\nu}-\Gamma^\nu_{\,\,\,\mu\lambda}\Gamma^\lambda_{\,\,\,\nu\sigma}.
\end{align*}
Then, for $\left(\mu,\nu\right)=\left(0,0\right)$, one gets
\begin{align*}
R_{00}=&\Gamma^\nu_{\,\,\,00,\nu}-\Gamma^\nu_{\,\,\,0\nu,0}+\Gamma^\lambda_{\,\,\,00}\Gamma^\nu_{\,\,\,\lambda\nu}-\Gamma^\nu_{\,\,\,0\lambda}\Gamma^\lambda_{\,\,\,\nu0}\\
=&-\frac{1}{2}\partial_0(g^{ii}\partial_0g_{ii})-\frac{1}{4}(g^{ii}\partial_0g_{ii})^2\\
=&-3\,\frac{\ddot{a}}{a},
\end{align*}
while for $\left(\mu,\nu\right)=\left(i,i\right)$, one has
\begin{align*}
R_{ii}=&\Gamma^\nu_{\,\,\,ii,\nu}-\Gamma^\nu_{\,\,\,i\nu,i}+\Gamma^\lambda_{\,\,\,ii}\Gamma^\nu_{\,\,\,\lambda\nu}-\Gamma^\lambda_{\,\,\,i\nu}\Gamma^\nu_{\,\,\,\lambda i}\\%(
=&-\frac{1}{2}\partial_0(g^{00}\partial_0 g_{ii})+\frac{1}{2}\partial_i(g^{ii}\partial_i g_{ii})-\frac{1}{2}\partial_\ell(g^{\ell\ell}\partial_\ell g_{ii})\\%)
&-\frac{1}{2}\partial_i(g^{jj}\partial_ig_{jj})\\%
&-\frac{1}{4}g^{00}g^{jj}(\partial_0 g_{ii})(\partial_0 g_{jj})+\frac{1}{4}g^{ii}g^{jj}(\partial_ig_{ii})(\partial_ig_{jj})
-\frac{1}{4}g^{\ell\ell}g^{kk}(\partial_\ell g_{ii})(\partial_\ell g_{kk})\\%)
&+\frac{1}{2}g^{00}g^{ii}(\partial_0g_{ii})^2-\frac{1}{4}(g^{jj}\partial_ig_{jj})^2
+\frac{1}{2}g^{ii}g^{\ell\ell}(\partial_\ell g_{ii})^2\\%)
=&-\frac{1}{2}\partial_0(g^{00}\partial_0 g_{ii})+\partial_i(g^{ii}\partial_i g_{ii})-\frac{1}{2}\partial_j(g^{jj}\partial_j g_{ii})\\%()
&-\frac{1}{2}\partial_i(g^{jj}\partial_ig_{jj})\\
&-\frac{1}{4}g^{00}g^{jj}(\partial_0 g_{ii})(\partial_0 g_{jj})+\frac{1}{2}g^{ii}g^{jj}(\partial_ig_{ii})(\partial_ig_{jj})
-\frac{1}{4}g^{jj}g^{kk}(\partial_jg_{ii})(\partial_jg_{kk})\\%)
&+\frac{1}{2}g^{00}g^{ii}(\partial_0g_{ii})^2-\frac{1}{4}(g^{jj}\partial_ig_{jj})^2
-\frac{1}{2}(g^{ii})^2(\partial_ig_{ii})^2+\frac{1}{2}g^{ii}g^{jj}(\partial_jg_{ii})^2\\%)
=&-\frac{1}{2}g^{00}\partial^2_0 g_{ii}+(\partial_ig^{ii})(\partial_i g_{ii})+g^{ii}\partial^2_i g_{ii}
-\frac{1}{2}g^{ii}\partial^2_j g_{ii}-\frac{1}{2}(\partial_jg^{ii})(\partial_j g_{ii})\\%)
&-\frac{3}{2}g^{ii}\partial^2_i g_{ii}-\frac{3}{2}(\partial_ig^{ii})(\partial_ig_{ii})\\
&-\frac{3}{4}g^{00}g^{ii}(\partial_0 g_{ii})(\partial_0 g_{ii})+\frac{3}{2}(g^{ii})^2(\partial_ig_{ii})(\partial_ig_{ii})
-\frac{3}{4}(g^{ii})^2(\partial_jg_{ii})(\partial_jg_{ii})\\%)
&+\frac{1}{2}g^{00}g^{ii}(\partial_0g_{ii})^2-\frac{3}{4}(g^{ii}\partial_ig_{ii})^2
-\frac{1}{2}(g^{ii})^2(\partial_ig_{ii})^2+\frac{1}{2}(g^{ii})^2(\partial_jg_{ii})^2\\%)
=&-\frac{1}{2}g^{00}\partial^2_0 g_{ii}-\frac{1}{4}g^{00}g^{ii}(\partial_0g_{ii})^2\\
&-\frac{1}{2}(\partial_ig^{ii})(\partial_ig_{ii})-\frac{1}{2}g^{ii}\partial^2_i g_{ii}
-\frac{1}{2}(\partial_jg^{ii})(\partial_j g_{ii})-\frac{1}{2}g^{ii}\partial^2_j g_{ii}\\%
&+\frac{1}{4}(g^{ii})^2(\partial_ig_{ii})^2-\frac{1}{4}(g^{ii})^2(\partial_jg_{ii})^2\\[1.5mm]
%=&-g^{00}\bigg[\left(\frac{\dot{a}}{a}\right)^2+\frac{\ddot{a}}{a}\bigg]g_{ii}-g^{00}\left(\frac{\dot{a}}{a}\right)^2g_{ii}\\[1.5mm]
%&+2\left(\frac{\partial_i R}{R}\right)^2-\bigg[\left(\frac{\partial_i R}{R}\right)^2+\frac{\partial^2_i R}{R}\bigg]\\[1.5mm]
%&+2\left(\frac{\partial_j R}{R}\right)^2-\bigg[\left(\frac{\partial_j R}{R}\right)^2+\frac{\partial^2_j R}{R}\bigg]\\[1.5mm]
%&+\left(\frac{\partial_i R}{R}\right)^2-\left(\frac{\partial_j R}{R}\right)^2\\
=&-g^{00}g_{ii}\bigg[\frac{\ddot{a}}{a}+2\left(\frac{\dot{a}}{a}\right)^2\bigg]-\frac{\partial^2_i R}{R}-\frac{\partial^2_j R}{R}+2\,\left(\frac{\partial_i R}{R}\right)^2, 
\end{align*}
where $i,j,k\in\{1,2,3\}$ and $l\in\{1,2,3\}-\{i\}$. Accordingly, one writes the Ricci scalar as 
\begin{align*}
R=& g^{00}R_{00}+g^{ii}R_{ii}\\
=&-3g^{00}\frac{\ddot{a}}{a}-3g^{00}\bigg[\frac{\ddot{a}}{a}+2\left(\frac{\dot{a}}{a}\right)^2\bigg]-g^{ii}\frac{\partial^2_i R}{R}-g^{ii}\frac{\partial^2_j R}{R}+2g^{ii}(\frac{\partial_i R}{R})^2\\[1.5mm]
=&-6\,g^{00}\bigg[\frac{\ddot{a}}{a}+\left(\frac{\dot{a}}{a}\right)^2\bigg]-4\,g^{ii}\frac{\partial^2_i R}{R}+2\,g^{ii}(\frac{\partial_i R}{R})^2,
\end{align*}
where $i,j\in\{1,2,3\}$. 

We now turn to the Einstein equation. Recall that it can be written in the form:
\begin{align*}
G^{\nu}_{\,\,\,\mu}=R^{\nu}_{\,\,\,\mu}-\frac{1}{2}\,\delta^{\nu}_{\,\,\,\mu}\,R=T^{\nu}_{\,\,\,\mu}.
\end{align*}
The $\mu\nu=00$ equation gives 
\begin{align}
\nonumber
8\pi \,T^{0}_{\,\,\,0}=&R^{0}_{\,\,\,0}-\frac{1}{2}R\\
\nonumber
=&3\,g^{00}\left(\frac{\dot{a}}{a}\right)^2+2\,g^{ii}\frac{\partial^2_i R}{R}-g^{ii}\left(\frac{\partial_i R}{R}\right)^2\\
\label{eq:Frie2d0}
=&3\left(\frac{\dot{a}}{a}\right)^2-\frac{2}{a^2 R^2}\frac{\partial^2_i R}{R}+\frac{1}{a^2 R^2}\left(\frac{\partial_i R}{R}\right)^2,
\end{align}
and the $\mu\nu=ii$ equations is
\begin{align}\label{eq:Frie1d0}
8\pi \,T^{i}_{\,\,\,i}=&R^{i}_{\,\,\,i}-\frac{1}{2}R\nonumber\\
=&g^{00}\bigg[2\frac{\ddot{a}}{a}+\left(\frac{\dot{a}}{a}\right)^2\bigg]+g^{ll}\frac{\partial^2_l R}{R}-g^{ll}\left(\frac{\partial_l R}{R}\right)^2+g^{ii}\left(\frac{\partial_i R}{R}\right)^2\nonumber\\
=&\bigg[2\frac{\ddot{a}}{a}+\left(\frac{\dot{a}}{a}\right)^2\bigg]-\frac{1}{a^2R^2}\frac{\partial^2_l R}{R}+\frac{1}{a^2R^2}\left(\frac{\partial_l R}{R}\right)^2-\frac{1}{a^2R^2}\left(\frac{\partial_i R}{R}\right)^2,
\end{align}
where $i,j,k\in\{1,2,3\}$ and $l\in\{1,2,3\}-\{i\}$. 
Using~\eqref{eq:Frie2d0} to eliminate the first derivative in~\eqref{eq:Frie1d0}, we can obtain the generalized Friedmann equations:
\begin{align}
\label{eq:Friedmann2g}
g^{00}\,\frac{\ddot{a}}{a}=&-\frac{4}{3}\pi \left(T^{0}_{\,\,\,0}-T^{i}_{\,\,\,i}\right),\\[1.5mm]
\label{eq:Friedmann1g}
g^{00}\,\left(\frac{\dot{a}}{a}\right)^2=&\frac{8}{3}\pi T^{0}_{\,\,\,0}-\frac{K\left(x,y,z\right)}{a^2},
\end{align}
with (see Appendix~\ref{App:C} for more details)
\begin{align*}
K\left(x,y,z\right)=\frac{1}{3}\,\bigg[-\frac{2}{R^2}\frac{\partial^2_i R}{R}+\frac{1}{R^2}\left(\frac{\partial_i R}{R}\right)^2\bigg],
\end{align*}
where $i,j,k\in\{1,2,3\}$ and $l\in\{1,2,3\}-\{i\}$.
As we will show bellow, $K=K\left(x,y,z\right)$ is intrinsically an effective sectional curvature. 
When $Z^2\left(x,y,z\right)\equiv1$, it reduces to
\begin{align}
\label{eq:Friedmann2}
\frac{\ddot{a}}{a}=&-\frac{4}{3}\pi \left(T^{0}_{\,\,\,0}-T^{i}_{\,\,\,i}\right),\\[1.5mm]
\label{eq:Friedmann1}
\left(\frac{\dot{a}}{a}\right)^2=&-\frac{K\left(x,y,z\right)}{a^2}+\frac{8}{3}\pi \,T^{0}_{\,\,\,0},
\end{align}
which reduce to the standard Friedmann equations when $K$ is a constant.



\section{Sectional curvature}
\label{App:C}

Denote $\d\sigma^{2}$ as the spatial part of the ERW metric~\eqref{eq:ERW} with $a=1$, namely
\begin{eqnarray}
\label{eq:SpaceMetric}
\d\sigma^{2}=\gamma_{ij}\,\d x^{i} \d x^{j}=R\left(x,y,z\right)^{2}\left(\d x^{\,2}\!+\!\d y^{\,2}\!+\!\d z^{\,2}\right).
\end{eqnarray}
whose fully covariant version of the curvature tensor of type $\left(0,4\right)$ is given by
\begin{eqnarray}
\label{eq:SMetric}
%\begin{array}{rcl}
%\displaystyle%
^{\left(3\right)}\!\!R_{ijkm}=\gamma_{is}\,^{\left(3\right)}\!\!R^{s}_{jkm}=\gamma_{is}\,\bigg[{}^{\left(3\right)}\!\Gamma^s_{\,\,\,jm,k}-{}^{\left(3\right)}\!\Gamma^s_{\,\,\,jk,m}+{}^{\left(3\right)}\!\Gamma^p_{\,\,\,jm}{}^{\left(3\right)}\!\Gamma^s_{\,\,\,pk}-{}^{\left(3\right)}\!\Gamma^p_{\,\,\,jk}{}^{\left(3\right)}\!\Gamma^s_{\,\,\,pm}\bigg],
%\end{array}
\end{eqnarray}
where $i,j,k,m\in\{1,2,3\}$. Here we use the superscript $^{\left(3\right)}$ to indicate that it is associated with the 3-metric~\eqref{eq:SpaceMetric}.

Then the {\it sectional curvature} of a given surface at point $p$ can be described by
\begin{eqnarray}
\label{eq:SMetric}
%\begin{array}{rcl}
%\displaystyle%
\nonumber
K_{p}=K_{p}[i,j,k,m]=-\frac{^{\left(3\right)}\!\!R_{ijkl}}{\gamma_{ik}\gamma_{jm}-\gamma_{im}\gamma_{jk}},
%\end{array}
\end{eqnarray}
which is also named as the Gaussian curvature~\cite{Zhang:2021ygh}.
It has a clear meaning. For instance, in the special case of the standard RW spacetime~\eqref{eq:RWMetric3}, one has
\begin{eqnarray}
\label{eq:RWSpaceMetric}
%\begin{array}{rcl}
%\displaystyle%
%\nonumber
\d \sigma^{2}&=&-\frac{\d x^2+\d y^2+\d z^2}{\big[1+\frac{1}{4}\kappa \left(x^{2}+y^{2}+z^{2}\right)\big]^2}.
%\end{array}
\end{eqnarray}
where $\kappa$ is a constant parameter.
In this case, for $\forall~i,j,k,m\in\{1,2,3\}$, $K_{p}=\kappa={\rm Constant}$ (see appendix~\ref{App:D} for the derivations). It means that the 3-space defined by~\eqref{eq:RWSpaceMetric} is actually a constant curvature space. 

Our task is clear. We need to understand the ERW spacetime~\eqref{eq:ERW}. Especially, we need to fully understand the $K-$term in the generalized Friedmann equations. 
In the general case of the ERW spacetime, direct calculation reveals that
\begin{eqnarray}
\label{eq:Kterm}
%\begin{array}{rcl}
%\displaystyle%
\nonumber
K_{p}^{i}&=&K_{p}[j,k,j,k]\\[1.5mm]
\nonumber
&=&-\frac{1}{R^2}\bigg[\left(\frac{\partial_j R}{R}\right)^2-\frac{\partial^2_j R}{R}\!+\!\left(\frac{\partial_k R}{R}\right)^2-\frac{\partial^2_k R}{R}-\left(\frac{\partial_i R}{R}\right)^2\!\bigg],
%\end{array}
\end{eqnarray}
where $i\in\{1,2,3\}$, $j\in\{1,2,3\}-\{i\}$ and $k\in\{1,2,3\}-\{i,j\}$.
Define %$$
\begin{eqnarray}
\label{eq:Kterm}
%\begin{array}{rcl}
%\displaystyle%
\nonumber
K\left(x,y,z\right)=\frac{~\sum K^i_{p}~}{~3~}.
%\end{array}
\end{eqnarray}
Then we have 
\begin{eqnarray}
\label{eq:ERW-Kterm}
%\begin{array}{rcl}
%\displaystyle%
\nonumber
K\left(x,y,z\right)=\frac{1}{3}\,\bigg[-\frac{2}{R^2}\frac{\partial^2_i R}{R}+\frac{1}{R^2}\left(\frac{\partial_i R}{R}\right)^2\bigg],
%\end{array}
\end{eqnarray}
which is obviously a generalization of the constant curvature $k$ in Eq.~\eqref{eq:RWSpaceMetric}. 




\section{The standard RW spacetime}
\label{App:D}

%Rinder P364

A specific example of the ERW metric is the standard RW metric. It can be written in the following form,
\begin{eqnarray}
\label{eq:RWMetric}
%\begin{array}{rcl}
%\displaystyle%
%\nonumber
\d s^{2}&=&\d t^2-a^2\left(t\right)\bigg[\frac{\d \chi^2}{1-\kappa\, \chi^2}+\chi^2 \d\Omega^2\bigg],
%\end{array}
\end{eqnarray}
where $\d\Omega^2=\d\theta^2\!+\!\sin^2\theta\,\d\varphi^2$ is the metric of the unit 2-sphere, and $\chi$ is a radial coordinate.

A second form of the standard RW metric is obtained from~\eqref{eq:RWMetric} via the relation
\begin{align}
\chi=\frac{r}{1+\frac{1}{4}\kappa r^{2}},
\end{align}
namely:
\begin{eqnarray}
\label{eq:RWMetric2}
%\begin{array}{rcl}
%\displaystyle%
%\nonumber
\d s^{2}&=&\d t^2-\bigg[a\left(t\right)R\left(r\right)\bigg]^2\left(\d r^2+r^2 \d\Omega^2\right),
%\end{array}
\end{eqnarray}
with 
\begin{eqnarray}
\label{eq:Rfactor}
%\begin{array}{rcl}
%\displaystyle%
%\nonumber
R\left(r\right)=\frac{1}{1+\frac{1}{4}\kappa r^{2}},
%\end{array}
\end{eqnarray}
where $\kappa$ is a real number.
In the same coordinate system as the ERW metric, the standard RW metric can be expressed as
\begin{eqnarray}
\label{eq:RWMetric3}
%\begin{array}{rcl}
%\displaystyle%
%\nonumber
\d s^{2}&=&\d t^2-\bigg[a\left(t\right)R\left(r\right)\bigg]^2\left(\d x^2+\d y^2+\d z^2\right),
%\end{array}
\end{eqnarray}
with $r^2=x^2+y^2+z^2$. Its spatial part with $a=1$ then reads 
\begin{eqnarray}
\label{eq:RWspaceMetric3}
%\begin{array}{rcl}
%\displaystyle%
%\nonumber
\d \sigma^{2}&=&-\frac{\d x^2+\d y^2+\d z^2}{\big[1+\frac{1}{4}\kappa \left(x^{2}+y^{2}+z^{2}\right)\big]^2}.
%\end{array}
\end{eqnarray}
Thus, the space with this metric form is a constant curvature space, and its sectional curvature is $\kappa$.



\end{document}




















