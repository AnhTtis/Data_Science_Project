 \documentclass[reqno,11pt]{article}
\usepackage{mathrsfs}
\usepackage{amssymb}
%\usepackage[usenames,dvipsnames]{xcolor}

\setlength{\oddsidemargin}{0mm} \setlength{\evensidemargin}{0mm}
\setlength{\topmargin}{-10mm} \setlength{\textheight}{220mm}
\setlength{\textwidth}{155mm}
\usepackage{amsthm}
\usepackage{amsmath}
\usepackage{amsfonts}
\usepackage{enumerate}
%\usepackage{txfonts}
\usepackage{bm}

\usepackage{multicol}
\DeclareMathOperator{\X}{X}
\usepackage{subfigure}
\usepackage{graphicx}
\usepackage{setspace,color,wrapfig}
\usepackage[colorlinks,linkcolor=blue,citecolor=cyan]{hyperref}
\newcommand{\myref}[1]{(\ref{#1})}
\numberwithin{equation}{section}
%\usepackage[showrefs]{refcheck}
\newtheorem{theorem}{Theorem}[section]
\newtheorem{lemma}[theorem]{Lemma}
\newtheorem{corollary}[theorem]{Corollary}
\newtheorem{proposition}[theorem]{Proposition}
\newtheorem{problem}[theorem]{Problem}
\theoremstyle{definition}
\newtheorem{definition}[theorem]{Definition}
\newtheorem{assumption}[theorem]{Assumption}
\newtheorem{example}[theorem]{Example}

\theoremstyle{remark}
\newtheorem{remark}[theorem]{Remark}

\begin{document}
\title{  Subsonic  flows with a contact discontinuity in a two-dimensional  finitely long curved nozzle }
\author{Shangkun Weng\thanks{School of mathematics and statistics, Wuhan University, Wuhan, Hubei Province, 430072, People's Republic of China. Email: skweng@whu.edu.cn}\and Zihao Zhang\thanks{School of mathematics and statistics, Wuhan University, Wuhan, Hubei Province, 430072, People's Republic of China. Email: zhangzihao@whu.edu.cn}}
\date{}
\maketitle
\newcommand{\de}{{\mathrm{d}}}
\def\div{{\rm div\,}}
\def\curl{{\rm curl\,}}
\newcommand{\ro}{{\rm rot}}
\newcommand{\sr}{{\rm supp}}
\newcommand{\sa}{{\rm sup}}
\newcommand{\va}{{\varphi}}
\newcommand{\me}{\mathcal{M}}
\newcommand{\ml}{\mathcal{V}}
\newcommand{\md}{\mathcal{D}}
\newcommand{\mg}{\mathcal{G}}
\newcommand{\mh}{\mathcal{H}}
\newcommand{\mf}{\mathcal{F}}
\newcommand{\ms}{\mathcal{S}}
\newcommand{\mt}{\mathcal{T}}
\newcommand{\mn}{\mathcal{N}}
\newcommand{\mb}{\mathcal{P}}
\newcommand{\mm}{\mathcal{B}}
\newcommand{\mj}{\mathcal{J}}
\newcommand{\mk}{\mathcal{K}}
\newcommand{\my}{\mathcal{U}}
\newcommand{\mw}{\mathcal{W}}
\newcommand{\mq}{\mathcal{Q}}
\newcommand{\ma}{\mathcal{L}}
\newcommand{\mc}{\mathcal{C}}
\newcommand{\mi}{\mathcal{I}}
\newcommand{\n}{\nabla}
\newcommand{\e}{\tilde}
\newcommand{\m}{\omega}
%\begin{abstract}
 \newcommand{\q}{{\rm R}}
%\end{abstract}
\newcommand{\p}{{\partial}}
\newcommand{\z}{{\varepsilon}}
\renewcommand\figurename{\scriptsize Fig}
\begin{abstract}
 This paper concerns the structural stability of subsonic flows with a contact discontinuity in a two-dimensional  finitely long slightly curved nozzle. We establish the existence and uniqueness of subsonic flows with a contact discontinuity by prescribing the entropy function, the Bernoulli quantity and  the horizontal mass flux distribution at the entrance and the flow angle at the exit.  The problem can be formulated as a free boundary problem for the  hyperbolic-elliptic coupled  system. To deal with the free boundary value problem,  the Lagrangian transformation is employed  to straighten  the contact discontinuity.  The Euler system is reduced to a nonlinear second-order equation for the stream function.  Inspired by \cite{CXZ22}, we use the implicit function theorem to locate the contact discontinuity. We also need to develop an iteration scheme to solve a nonlinear elliptic boundary value problem with nonlinear boundary conditions in a  weighted H\"{o}lder space.
\end{abstract}
\begin{center}
\begin{minipage}{5.5in}
Mathematics Subject Classifications 2010: 35L65, 35L67, 76H05, 76N15.\\
Key words: steady Euler system, contact discontinuity, free boundary, structural stability, Lagrangian transformation.
\end{minipage}
\end{center}
\section{Introduction  }\noindent
\par In this paper, we are concerned with the structural stability of  subsonic flows with a contact discontinuity governed by the two-dimensional steady  full Euler system in a finitely long slightly curved  nozzle. The  two-dimensional steady  full Euler system for compressible  inviscid gas  is of the  following form:
\begin{align}\label{1-1}
\begin{cases}
\p_{x_1}(\rho u_1)+\p_{x_2}(\rho u_2)=0,\\
\p_{x_1}(\rho u_1^2)+\p_{x_2}(\rho u_1 u_2)+\p_{x_1} P=0,\\
\p_{x_1}(\rho u_1u_2)+\p_{x_2}(\rho u_2^2)+\p_{x_2} P=0,\\
\p_{x_1}(\rho u_1 E+u_1 P)+\p_{x_2}(\rho u_2 E+u_2 P)=0,\\
\end{cases}
\end{align}
where $\bm u=(u_1,u_2)$ is the velocity, $\rho$ is the density, $P$ is the pressure, $E$ is the energy, respectively. For polytropic gas,
$P= A(S)\rho^{\gamma} $,   where  $A(S)= R e^{S}$ and $\gamma\in (1,+\infty)$, $R$ are positive constants.  Denote the Bernoulli's function and the local sonic speed by $B=\frac12|{\textbf{u}}|^2+\frac{\gamma P}{(\gamma-1)\rho}$  and $ c(\rho,A)=\sqrt{A\gamma} \rho^{\frac{\gamma-1}{2}}$, respectively. Then  the system \eqref{1-1} is hyperbolic for supersonic flows ($ |\textbf{u}|>c(\rho,A) $), and hyperbolic-elliptic coupled for subsonic flows ($ |\textbf{u}|<c(\rho,A) $).
\par   To understand  conditions on the contact discontinuity curve, we introduce the notation of weak solution of the full Euler system \eqref{1-1}.  Suppose that a domain $\md $ in $ \mathbb{R}^2 $
is divided by a $C^1$-curve  $ \Gamma$
 into sub-domains $\md^+ $ and $\md^- $ such that $ \md= \md^- \cup\Gamma \cup \md^+ $.
Assume that $ \bm U = (\rho, u_1, u_2,P)$ is a $C^1$ solution of Euler equation \eqref{1-1} in each domain $ \md^+ $ and $ \md^- $ and is continuous up to the boundary $ \Gamma $ in each sub-domain. If   $ \bm U $ is a weak solution for \eqref{1-1} in domain $ \md $, by using integration by parts, we see that  the following Rankine-Hugoniot conditions hold along $\Gamma$  almost everywhere:
\begin{equation}\label{1-2}
\begin{cases}
n_1[\rho u_1]+n_2[\rho u_2]=0,\\
n_1[\rho u_1^2]+n_1[P]+n_2[\rho u_1u_2]=0,\\
n_1[\rho u_1u_2]+n_2[\rho u_2^2]+n_2[P]=0,\\
n_1[ \rho u_1(E+\frac{P}{\rho})]+n_2[\rho u_2( E+\frac{P}{\rho})]=0,\\
\end{cases}
\end{equation}
where $ \mathbf{n} = (n_1, n_2) $  is the unit normal vector to $\Gamma$, and $[F](\mathbf{x}) = F_+(\mathbf{x})- F_-(\mathbf{x})$
denotes the jump across the  curve $\Gamma$ for a piecewise smooth function $ F $.
\par Let $ \bm \tau= (\tau_{1},\tau_{2})$ as the unit tangential vector to $\Gamma$, which means that $ \mathbf{n}\cdot{\bm \tau}= 0$. Taking the dot product of $(\eqref{1-2}_2,\eqref{1-2}_3)$ with $ \mathbf{n}$  and $ \bm \tau $ respectively, we have
\begin{equation}\label{1-3}
\rho(\mathbf{u}\cdot\mathbf{n})[\mathbf{u}\cdot \bm \tau]_{\Gamma}= 0, \quad
[\rho(\mathbf{u}\cdot\mathbf{n})^2+P]_{\Gamma}= 0.
\end{equation}
 Assume that $ \rho> 0 $  in $ \bar \md $, \eqref{1-3} implies either $ \mathbf{u}\cdot\mathbf{n} = 0 $
on $\Gamma$  or $[\mathbf{u}\cdot \bm \tau]_{\Gamma}=0$.
If $ \mathbf{u}\cdot\mathbf{n} \neq 0 $ and $[\mathbf{u}\cdot \bm \tau]_{\Gamma}=0 $ hold on $ \Gamma $, the curve $ \Gamma $ is called a shock; if the flow moves along both sides of $ \Gamma $ such that  $\mathbf{u}\cdot\mathbf{n} = 0 $ on $ \Gamma $, the curve $ \Gamma $ is called a contact discontinuity. In the latter case, $\mathbf{u}\cdot\mathbf{n} = 0 $ and the second equation in\eqref{1-3} give $ [P]=0 $. Then we get the R-H conditions corresponding to a contact discontinuity  as follows:
\begin{equation}\label{1-4}
 \mathbf{u}\cdot\mathbf{n}=0 \quad {\rm{and}} \quad [P]=0 \quad  {\rm{on}} \ \Gamma.
 \end{equation}
\par Admissible shocks, centered rarefaction waves and
contact discontinuities are fundamental wave patterns in quasi-linear hyperbolic conservation laws.  The stability analysis of these elementary waves is very important both in physics and mathematics.  In the book \cite{CF48},  Courant and Friedrichs had presented some  nonlinear phenomenas for steady compressible flows, such as Mach reflection,  jet
flows, and their interactions. All these phenomenas are formulated by  elementary waves including contact discontinuities. So to understand the stability of contact discontinuity is an important step in studying these phenomenas.
 \par In recently years, there have been many
interesting results on contact discontinuity problems  for various
physically situations.   The stability  of  contact discontinuity for subsonic flows in infinite  nozzles was established in  \cite{BM09} and \cite{ BP19,PB19} with a Helmholtz decomposition.    The global existence and uniqueness  of the subsonic contact discontinuity with large vorticity in infinity long  nozzles were obtained  in \cite{CHWX19} by the theory of compensated compactness, which is not a perturbation around piecewise constant solutions.   The stability of supersonic flat contact discontinuity and transonic flat contact discontinuity for 2-D steady Euler flows in a finite nozzle was established in \cite{HFWX19,HFWX21}.  The stability  of two-dimensional  transonic contact discontinuity over a solid wedge  and three-dimensional supersonic and  transonic contact discontinuity were established in \cite{CYK13,CKY13} and \cite{WY15,WY13,WF15}.  The contact discontinuity in the Mach
reflection was also studied in \cite{CF07,CHF13}.  Recently,  the existence, uniqueness and stability of
subsonic flows past an airfoil with a vortex line were established in \cite{CXZ22}. The  well-posedness theory
for the steady subsonic jet flow
with a shock and a contact discontinuity was established in \cite{PW22}. % For other studies related to steady subsonic flows in various important situations, such as the subsonic flow past a body and in nozzles; see \cite{BDX14,BW18,WS14,WZ22} and the references therein.
\par In this paper,   we study the  structural stability of  subsonic flows with a contact discontinuity in a two-dimensional finitely long slightly curved nozzle.  The contact discontinuity is part of the  solution  and is unknown, thus  this is a free boundary that separates the subsonic flow in the upper and lower layers of the nozzle. Thanks to the fundamental feature of the contact discontinuity, namely,  the tangent of the  contact discontinuity is parallel to the  velocity of the flow  on its both sides, we can employ  the Lagrangian transformation to straighten  the free boundary. Then by introducing a stream function and solving the
hyperbolic equations for the entropy and the Bernoulli's function, the Euler system can be reduced to a nonlinear second-order  equation for the stream function in   upper and lower domains, respectively. Then we can develop an iteration scheme to solve a nonlinear elliptic boundary value problem with nonlinear boundary conditions
in a weighted H\"{o}lder space.
\par The other key ingredient in our mathematical analysis is to use the implicit function theorem  to locate the contact discontinuity. This idea is motivated by the discussion of airfoil problem in \cite{CXZ22}. For the problem of subsonic flows past an airfoil with a
vortex line at the trailing edge, the vortex line attached to the trailing edge is a contact discontinuity and is treated as a free boundary, the authors employed the implicit function theorem to solve this problem.   In this paper, we modified the approach in \cite{CXZ22} to study our problem of subsonic flows with a contact discontinuity in a two-dimensional finitely long  curved nozzle. We choose a suitable weighted H$\ddot{\rm{o}}$lder space and design a proper  map to verify the conditions in the implicit function theorem. Then we can locate the contact discontinuity by the implicit function theorem. The detailed analysis will be given in Section 5.
\par This paper will be arranged as follows. In Section 2, we formulate the problem of subsonic flows with a contact discontinuity  in  a  two-dimensional finitely long curved nozzle and state the main result. In Section 3, we first employ  the Lagrangian  transformation to straighten  the contact discontinuity. Then by  introducing a stream function and solving the
hyperbolic equations for the entropy and the Bernoulli's function, the Euler system can be reduced to a nonlinear second-order equation for the stream function in   upper and lower domains, respectively. Finally, we state the implicit function theorem and  the main steps to solve the free boundary problem 3.1.
In Section 4, we first linearize the nonlinear equation and solve the linear equation in a suitable weighted H$\ddot{\rm{o}}$lder space. Then  the framework of the contraction mapping theorem can be used  to find the solution of the nonlinear equation.
 In Section 5, we choose a suitable weighted H$\ddot{\rm{o}}$lder space and design a proper  map to verify the conditions in the implicit function theorem. Then by using the   implicit function theorem, we  locate the contact discontinuity. In Section 6, we finish the proof of the main theorem.
   \section{Mathematical problem and the main result}\noindent
 \par In this section, we give a detailed formulation of subsonic flows with a contact discontinuity  in  a  two-dimensional finitely long curved nozzle and state the main result.   First, we consider a special class of  subsonic flows with a straight contact discontinuity in a finitely long flat nozzle.
\par The  flat nozzle (Fig 1) of the length $ L $ can be described by
\begin{equation*}
\Omega_b=\{(x_1,x_2):0<x_1<L,\ -1<x_2<1\}.
\end{equation*}
\begin{figure}
  \centering
  % Requires \usepackage{graphicx}
  \includegraphics[width=11cm,height=5cm]{sub-fig4}
  \caption{Subsonic flows with a contact discontinuity in a finitely long flat nozzle}
\end{figure}
Consider two layers of steady smooth Euler flows separated by the line $ x_2=0 $ satisfying the following properties:
     \begin{enumerate}[(i)]
\item The velocity and density of the top and bottom layers are given by $ (u_b^{+},0),\rho_b^{+} $ and $ (u_b^{-},0),\rho_b^{-} $, where $ u_b^\pm>0 $ and   $\rho_b^\pm>0$;
  \item the pressure of both the top and bottom layers   are given by the same positive constant $ P_b $;
\item the  flow in the   top and bottom layers are subsonic, i.e.,
 \begin{equation*}
 (u_b^\pm)^2>\frac{\gamma P_b}{\rho_b^\pm}.
\end{equation*}
\end{enumerate}
Then
\begin{equation}\label{2-1}
 \bm{U}_b=
  \begin{cases}
   \bm{U}_b^{+}:=(\rho_b^{+},u_{b}^{+},0,P_b),  \quad {\rm{for}}\quad (x_1,x_2)\in (0,L)\times (0,1),\\
   \bm{U}_b^{-}:=(\rho_b^{-},u_{b}^{-},0,P_b),  \quad {\rm{for}}\quad (x_1,x_2)\in (0,L)\times (-1,0),\\
  \end{cases}
 \end{equation}
 is a weak solution of  the steady Euler system \eqref{1-1} with a contact discontinuity on the line $ x_2=0 $,
  which will be called the background solution in this paper.
  This paper is going to establish the structural
stability of this background  solution  under the perturbations of suitable boundary conditions on the entrance and exit  and the upper and lower nozzle walls.
\par  The two-dimensional finitely long curved  nozzle $ \Omega $ (Fig 2) is  described by
\begin{equation}\label{2-2}
\Omega=\{(x_1,x_2):0<x_1<L, g^-(x_1)<x_2<g^+(x_1)\},
\end{equation}
  with
  \begin{equation*}
  g^\pm(x_1)\in C^{2,\alpha}([0,L]) \quad {\rm{and}}\quad g^\pm(0)=\pm1.
  \end{equation*}
  The  upper and lower  boundaries of the nozzle are denoted by $ \Gamma_w^+ $ and $ \Gamma_w^- $, i.e;
\begin{equation}\label{2-3}
\Gamma_w^\pm:=\{(x_1,x_2):  0<x_1<L,\ x_2=g^\pm(x_1)\}.
\end{equation}
\begin{figure}
  \centering
  % Requires \usepackage{graphicx}
  \includegraphics[width=12cm,height=5cm]{sub-fig5}
  \caption{Subsonic flows with a contact discontinuity in a finitely long  curved nozzle}
\end{figure}
The exit of the nozzle is denoted by
\begin{equation}\label{2-4}
\Gamma_{L}:=\{(x_1,x_2):   x_1=L,\ g^-(L)<x_2<g^+(L)\}.
\end{equation}
The entrance of the nozzle  is separated into two parts:
\begin{equation}\label{2-5}
\begin{aligned}
\Gamma_{0}^+:=\{(x_1,x_2):   x_1=0,\ 0<x_2<1\},   \quad \Gamma_{0}^-:=\{(x_1,x_2):  x_1=0,\ -1<x_2<0\}.
\end{aligned}
\end{equation}
At the entrance $ \Gamma_0^\pm $, we prescribe the boundary data   for  the entropy function $A $,  the Bernoulli quantity $ B $ and  the horizontal mass distribution  $ J=\rho u_1 $:
\begin{equation}\label{2-6}
(A,B,J)(0,x_2)=
\begin{cases}
 \bm {U}_{en}^+(x_2): =(A_{en}^+,B_{en}^+,J_{en}^+)(x_2),\quad {\rm{on}}\quad \Gamma_0^+,\\
  \bm {U}_{en}^-(x_2):=(A_{en}^-,B_{en}^-,J_{en}^-)(x_2),\quad {\rm{on}}\quad \Gamma_0^-.
  \end{cases}
  \end{equation}
  At the exit, the flow angle $ \omega=\frac{u_2}{u_1}$ is prescribed by
 \begin{equation}\label{2-7}
  \omega(L,x_2)= \omega_{ex}(x_2), \quad {\rm{on}}\quad \Gamma_{L}.
  \end{equation}
  Here $ (\bm {U}_{en}^+,\bm {U}_{en}^-)(x_2)\in \left(C^{1,\alpha}([0,1])\right)^3\times \left(C^{1,\alpha}([-1,0])\right)^3 $ and $\omega_{ex}(x_2)\in C^{2,\alpha}([g^-(L),g^+(L)]) $ are close to the background solution in some sense that will be described later.
\par  We expect the flow in the nozzle  will be separated by a  contact discontinuity   $ \Gamma:=\{x_2=g_{cd}(x_1),0<x_1<L\} $ with $ g_{cd}(0)=0 $, and we denote
\begin{equation*}
\Omega^+:=\Omega\cap\{g_{cd}(x_1)<x_2<g^+(x_1)\}, \ \Omega^-:=\Omega\cap\{ g^-(x_1)<x_2<g_{cd}(x_1)\}.
\end{equation*}
 Let
\begin{equation}\label{2-8}
{\bm {U}}(x_1,x_2)=
  \begin{cases}
   {\bm {U}}^+(x_1,x_2):=(\rho^+,u_1^+,u_{2}^+,P^+)(x_1,x_2)\quad {\rm{in}}\quad   \Omega^+,\\
   {\bm {U}}^-(x_1,x_2):=(\rho^-,u_1^-,u_{2}^-,P^-)(x_1,x_2)  \quad {\rm{in}}\quad  \Omega^-.\\
  \end{cases}
 \end{equation}
  Along the contact discontinuity $ x_2=g_{cd}(x_1) $, the following Rankine-Hugoniot conditions hold:
\begin{equation}\label{2-9}
\frac{u_2^+}{u_1^+}=\frac{u_2^-}{u_1^-}=g_{cd}^\prime(x_1), \quad P^+=P^-, \quad {\rm{on}}\quad \Gamma.
\end{equation}
On the nozzle walls $ \Gamma_w^+$  and $ \Gamma_w^- $, the usual slip boundary condition is imposed:
\begin{equation}\label{2-10}
  \frac{u_2^+}{u_1^+}=(g^+)^\prime(x_1), \ {\rm{on}}\ \Gamma_w^+, \quad
  \frac{u_2^-}{u_1^-}=(g^-)^\prime(x_1), \ {\rm{on}}\ \Gamma_w^-.
\end{equation}

  %satisfying
 % \begin{equation}\label{2-10}
 %\omega_{ex}(g^+(L))=(g^+)^\prime(L) \quad {\rm{and}} \quad \omega_{ex}(g^-(L))=(g^-)^\prime(L).
 % \end{equation}
 \par In summary, we will investigate the following problem:
 \begin{problem}
  Given  functions $ (J_{en}^\pm, A_{en}^\pm, B_{en}^\pm)(x_2)$ at the entrance and function $ \omega_{ex}(x_2)$ at the exit, find a unique piecewise smooth subsonic solution $ (\bm{U}^+ ,\bm{U}^-) $  defined on $ \Omega^+ $ and $ \Omega^- $ respectively,  with the contact discontinuity $ \Gamma: x_2=g_{cd}(x_1) $ satisfying the Euler system \eqref{1-1} in the sense of weak solution and  the Rankine-Hugoniot conditions in \eqref{2-9} and the slip boundary conditions in \eqref{2-10}.
  \end{problem}
  \par
The main theorem of this paper can be stated as follows.
   \begin{theorem}
    Let $  \bm U_{b}^\pm=(A_b^\pm,B_b^\pm,J_b^\pm)$, where
\begin{equation*}
  A_b^\pm=\frac{P_b^\pm}{(\rho_b^\pm)^\gamma}, \quad B_b^\pm=\frac12|u_b^\pm|^2+\frac{\gamma P_b^\pm}{(\gamma-1)\rho_b^\pm}, \quad J_b^\pm=\rho_b^\pm u_b^\pm.
\end{equation*}
Then given functions $(J_{en}^\pm, A_{en}^\pm, B_{en}^\pm,\omega_{ex}) $, we
define
 \begin{equation}\label{2-11}
   \begin{aligned}
   \sigma(\bm U_{en}^+,\bm U_{en}^-,\omega_{ex},g^+,g^-):&=
   \|\bm U_{en}^+ -\bm U_b^+\|_{1,\alpha;[0,1]}+\|\bm U_{en}^- -\bm U_b^-\|_{1,\alpha;[-1,0]} \\
   &\quad+\| \omega_{ex}\|_{2,\alpha;[g^-(L),g^+(L)]}+
   \| g^+ - 1\|_{2,\alpha;[0,L]}\\
   &\quad+
   \| g^- + 1\|_{2,\alpha;[0,L]}.
   \end{aligned}
   \end{equation}
  There exist positive constants $\sigma_{cd} $ and $ \mc $ depending only on  $ (A_b^\pm,B_b^\pm,J_b^\pm,L,\alpha) $ such that  if
   \begin{equation}\label{2-12}
   \begin{aligned}
   \sigma(\bm U_{en}^+,\bm U_{en}^-,\omega_{ex},g^+,g^-)\leq \sigma_{cd},
   \end{aligned}
   \end{equation}
   \rm{Problem 2.1} has a unique piecewise smooth subsonic flow  $ (\bm{U}^+,\bm{U}^-) $   with the contact discontinuity $ \Gamma: x_2=g_{cd}(x_1) $ satisfying the following properties:
 \begin{enumerate}[\rm(i)]
\item The piecewise smooth subsonic flow  $ (\bm{U}^+,\bm{U}^-)\in ( C^{1,\alpha}(\Omega^+)\cap C^{\alpha}(\overline{\Omega^+}))\times (C^{1,\alpha}(\Omega^-)\cap C^{\alpha}(\overline{\Omega^-}))$  satisfies the following estimate:
 \begin{equation}\label{2-13}
\|\bm{U}^+ -\bm{U}_b^+\|_{C^{\alpha}(\overline{\Omega^+})}+\|\bm{U}^- -\bm{U}_b^-\|_{C^{\alpha}(\overline{\Omega^-})}\leq \mc\sigma(\bm U_{en}^+,\bm U_{en}^-,\omega_{ex},g^+,g^-).
\end{equation}
\item The contact discontinuity curve $ g_{cd}(x_1)\in C^{2,\alpha}\left((0,L)\right)\cap C^{1,\alpha}([0,L])  $ satisfies
    $g_{cd}(0)=0 $. Furthermore, it holds that
\begin{equation}\label{2-14}
 \|g_{cd} \|_{1,\alpha;[0,L]}\leq \mc\sigma(\bm U_{en}^+,\bm U_{en}^-,\omega_{ex},g^+,g^-).
 \end{equation}
 \end{enumerate}
 \end{theorem}
 \section{The reformulation of  Problem 2.1}\noindent
\par In this section, we first employ the Lagrangian transformation to straighten  the contact discontinuity, and  reformulate the free boundary value    problem 2.1. Then  by introducing a stream function and solving the hyperbolic
equations for the entropy and the Bernoulli's function, the Euler system can be reduced to a
nonlinear second-order  equation for the stream function in   upper and lower domains, respectively. Furthermore,  all the boundary conditions at the  entrance, the nozzle walls and the exit become local.  Finally, we state the implicit function theorem and  the main steps to solve the free boundary problem 3.1.
\subsection{Reformulation by the Lagrangian transformation}\noindent
 \par To overcome the difficulty of the
free boundary  $ \Gamma $,  we will  apply the Lagrangian transformation
to straighten the contact discontinuity.
\par  Let $ (\bm{U}^+(x_1,x_2),\bm{U}^-(x_1,x_2), g_{cd}(x_1)) $ be a solution to  Problem 2.1.
 Define
 \begin{equation}\label{3-1}
 m^-=\int_{-1}^{0}J_{en}^-(s)\de s \quad {\rm{and}}\quad
  m^+=\int_{0}^{1}J_{en}^+(s)\de s.
  \end{equation}
   Then for any $x_1\in( 0,L)$, it follows from the conservation of mass equation in \eqref{1-1} that
 \begin{equation}\label{3-2}
 \int_{g^-(x_1)}^{g_{cd}(x_1)}\rho^- u_{1}^-(x_1,s)\de s=m^-, \quad
 \int_{g_{cd}(x_1)}^{g^+(x_1)}\rho^+ u_{1}^+(x_1,s)\de s =m^+.
 \end{equation}
 \par Let
  \begin{equation}\label{3-3}
  y_2(x_1,x_2)= \int_{g_{cd}(x_1)}^{x_2}\rho u_{1}(x_1,s)\de s.
  \end{equation}
    It is easy to verify that
  \begin{equation*}
  \frac{\p y_2}{\p x_1}=-\rho u_2, \quad  \frac{\p y_2}{\p x_2}=\rho u_1.
  \end{equation*}
  Thus we can introduce the Lagrangian transformation  as
  \begin{equation}\label{3-4}
   \begin{cases}
     y_1=x_1,\\
     y_2=y_2(x_1,x_2).\\
    \end{cases}
     \end{equation}
       A direct computation gives
   \begin{equation*}
   \frac{\p(y_1,y_2)}{\p(x_1,x_2)}=\left|
\begin{matrix} 1& 0 \\ -\rho u_2 & \rho u_1\end{matrix}\right|=\rho u_1.
 \end{equation*}
 So  if $ (\rho^\pm,u_1^\pm, u_2^\pm,P^\pm) $ are close to the background solution
$ (\rho_b^\pm,u_b^\pm,0,P_b) $, we have $ \rho^\pm u_1^\pm \geq \mc_b> 0 $, where $ \mc_b $ depends only on the  background solution.  Hence the  Lagrangian transformation is invertible.
\par Under this transformation,  the domain $ \Omega $ becomes
\begin{equation*}
\mn=\{(y_1,y_2): 0<y_1<L,\ -m^-<y_2<m^+\}.
\end{equation*}
 The  lower wall $ \Gamma_w^-$ and  upper wall $ \Gamma_w^+$ of the nozzle are  transformed  into
\begin{equation*}
\begin{aligned}
\Sigma_w^-:=\{(y_1,y_2): 0<y_1<L,\ y_2=-m^-\}, \quad
  \Sigma_w^+:=\{(y_1,y_2): 0<y_1<L,\ y_2=m^+\}.
\end{aligned}
\end{equation*}
Moreover, on $ \Gamma $, we have
\begin{equation*}
y_2(x_1,g_{cd}(x_1))=\int_{g_{cd}(x_1)}^{g_{cd}(x_1)}\rho u_{1}(x_1,s)\de s=0.
\end{equation*}
Hence the free boundary $\Gamma $ becomes the following  straight line:
\begin{equation}\label{3-5}
\Sigma=\{(y_1,y_2):0<y_1<L,\ y_2=0\}.
\end{equation}
\par Define
\begin{equation}\label{3-6}
\mn^-:=\mn\cap \{-m^-<y_2<0\}, \quad \mn^+:=\mn\cap \{0<y_2<m^+\}.
\end{equation}
Then the entrance and exit  of $ \mn^\pm $ are defined as
\begin{equation*}
\begin{aligned}
\Sigma_0^+:=\{(y_1,y_2):  y_1=0,\ 0<y_2<m^+\}, \quad \Sigma_0^-:=\{(y_1,y_2):  y_1=0,\ -m^-<y_2<0\},\\
\end{aligned}
\end{equation*}
and
\begin{equation*}
\begin{aligned}
\Sigma_L^+:=\{(y_1,y_2):  y_1=L,\ 0<y_2<m^+\}, \quad \Sigma_L^-:=\{(y_1,y_2):  y_1=L,\ -m^-<y_2<0\}.
\end{aligned}
\end{equation*}
 Let
\begin{equation*}
{\bm{ U}}(y_1,y_2)=
  \begin{cases}
 {\bm{U}}^+(y_1,y_2):=( \rho^+,u_1^+, u_{2}^+, P^+)(\mathbf{x}(y_1,y_2))\quad {\rm{in}}\quad   \mn^+,\\
  {\bm{U}}^-(y_1,y_2):=( \rho^-,u_1^-,u_{2}^-, P^-)(\mathbf{x}(y_1,y_2))  \quad {\rm{in}}\quad  \mn^-.\\
  \end{cases}
 \end{equation*}
Then the system \eqref{1-1} in the new coordinates  can be rewritten as
\begin{equation}\label{3-7}
\begin{cases}
\p_{y_1}\left(\frac{1}{\rho  u_1}\right)-\p_{y_2}\left(\frac{ u_2}{  u_1}\right)
=0,\\
\p_{y_1}  u_2+\p_{y_2}P=0,\\
\p_{y_1}  u_1+\frac{1}{\rho  u_1}(\p_{y_1}-\rho  u_2\p_{y_2}) P=0,\\
\p_{y_1}  A=0.\\
\end{cases}
\end{equation}
The background solution in the  Lagrange coordinates is
\begin{equation}\label{3-8}
 {\bm{U}}_b=
  \begin{cases}
  {\bm{U}}_b^{+}:=(\rho_b^{+},u_b^+,0,P_b),  \quad  {\rm{in}}\quad \mn_b^+,\\
  {\bm{U}}_b^{-}:=(\rho_b^{-},u_b^-,0,P_b),  \quad {\rm{in}}\quad \mn_b^-,\\
  \end{cases}
 \end{equation}
  where
  \begin{equation*}
\mn_b^+:=\{(y_1,y_2):  0<y_1<L,\ 0<y_2<m_b^+\}, \quad
\mn_b^-:=\{(y_1,y_2): 0<y_1<L,\ -m_b^-<y_2<0\},
\end{equation*}
and $ m_b^\pm=\rho_b^\pm u_b^\pm $. Without loss of generality, we assume that $m_b^\pm= m^\pm $.
\par
In the new coordinates,  the boundary data \eqref{2-6} at the entrance is given by
 \begin{equation}\label{3-9}
 {\bm {U}}_{en}(y_2)=
\begin{cases}
  {\bm {U}}_{en}^+(y_2)=(\e A_{en}^+,\e B_{en}^+,\e J_{en}^+)(y_2),\quad {\rm{on}}\quad \Sigma_0^+,\\
  {\bm {U}}_{en}^-(y_2)=(\e A_{en}^-,\e B_{en}^-,\e J_{en}^-)(y_2),\quad {\rm{on}}\quad \Sigma_0^-,\\
  \end{cases}
  \end{equation}
  where
  \begin{equation*}
  (\e A_{en}^+,\e B_{en}^+,\e J_{en}^+)(y_2)=( A_{en}^+, B_{en}^+,J_{en}^+)\left(\int_{0}^{y_2}\frac{1}{J_{en}^+}(s)\de s\right),
  \end{equation*}
  and
  \begin{equation*}
   (\e A_{en}^-,\e B_{en}^-,\e J_{en}^-)(y_2)=( A_{en}^-, B_{en}^-,J_{en}^-)\left(-\int_{y_2}^{0}\frac{1}{J_{en}^-}(s)\de s\right).
\end{equation*}
  The boundary condition \eqref{2-7} at the exit become
  \begin{equation}\label{3-10}
 \omega(L,y_2)=\begin{cases}
   \omega_{ex}(x_2^+(L,y_2)),\quad {\rm{on}}\quad \Sigma_L^+,\\
   \omega_{ex}(x_2^-(L,y_2)),\quad {\rm{on}}\quad \Sigma_L^-,\\
  \end{cases}
  \end{equation}
where
   %\begin{equation*}
%   \e \omega_{ex}^\pm(y_2)=\omega_{ex}(x_2^\pm(L,y_2)),
%  \end{equation*}
% and
  \begin{equation*}
% x_2^\pm(0,y_2)=\int_{0}^{y_2}\frac{1}{J_{en}^\pm}(s)\de s,\quad {\rm{and}}\quad
\begin{aligned}
x_2^+(L,y_2)=g_{cd}(L)+\int_{0}^{y_2}\frac{1}{\rho^+  u_{1}^+}(L,s)\de s, \quad
x_2^-(L,y_2)=g_{cd}(L)-\int_{y_2}^{0}\frac{1}{\rho^- u_{1}^-}(L,s)\de s.
\end{aligned}
\end{equation*}
 Hence it should be emphasized that the flow angle at the exit  becomes non-local and nonlinear in the Lagrange coordinates.
 %\par Moreover,  the compatibility conditions \eqref{2-11}  becomes
%   \begin{equation}\label{3-12}
  %\mn_{ex}^+(m^+)=\omega_{ex}(x_2^+(L,m^+)))=(g^+)^\prime(L),\quad
 %  \mn_{ex}^-(-m^-)=\omega_{ex}(x_2^-(L,-m^-)))=(g^-)^\prime(L).
%\end{equation}
 \par
 The Rankine-Hugoniot conditions in \eqref{2-9}  can be rewritten
\begin{equation}\label{3-11}
\frac{ u_2^+(y_1,0)}{ u_1^+(y_1,0)}=\frac{ u_2^-(y_1,0)}{ u_1^-(y_1,0)}=g_{cd}^{\prime}(y_1),
\end{equation}
and
\begin{equation}\label{3-12}
 P^+(y_1,0)=P^-(y_1,0).
\end{equation}
The slip boundary conditions in \eqref{2-10}  become
\begin{equation}\label{3-13}
 \frac{u_2^+}{ u_1^+}(y_1,m^+)=(g^+)^{\prime}(y_1), \ {\rm{on}}\ \Sigma_w^+, \quad \frac{ u_2^-}{u_1^-}(y_1,-m^-)=(g^-)^{\prime}(y_1), \ {\rm{on}}\ \Sigma_w^-.
\end{equation}
\subsection{Reduction of the Euler system }\noindent
   \par In this subsection, we introduce a stream function to  reformulate the Euler system \eqref{3-7} as a nonlinear second-order   equation with  suitable boundary conditions.
  \par By the first equation in \eqref{3-7},
  one can introduce a stream function
 \begin{equation}\label{3-14}
 \varphi(y_1,y_2)=
  \begin{cases}
  \varphi^-(y_1,y_2),\quad {\rm{for}}\quad (y_1,y_2)\in \mn^-,\\
    \varphi^+(y_1,y_2),\quad {\rm{for}}\quad (y_1,y_2)\in  \mn^+,\\
  \end{cases}
  \end{equation}
  such that
\begin{equation}\label{3-15}
\p_{y_1}\varphi=\frac{  u_2}{ u_1}, \quad \p_{y_2}\varphi=\frac{1}{\rho u_1}, \quad \varphi(0,0)=0.
\end{equation}
That is
\begin{equation*}
 u_1=\frac{1}{ \rho\p_{y_2}\varphi}, \quad  u_2=\frac{\p_{y_1}\varphi}{  \rho\p_{y_2}\varphi}.
\end{equation*}
\par It follows from  the fourth equation in \eqref{3-7} that one has
\begin{equation}\label{3-16}
A=\e A_{en}(y_2).
\end{equation}
Moreover,  the Bernoulli's function  $ B $  in the Lagrangian coordinates satisfies
\begin{equation*}
 \p_{y_1} B=0.
 \end{equation*}
 Thus the following equation holds:
\begin{equation}\label{3-17}
\frac{1}{2}\left(\frac{1}{( \rho\p_{y_2}\varphi)^2}+\frac{(\p_{y_1}\varphi)^2}
{( \rho\p_{y_2}\varphi)^2}\right)+\frac{\e A_{en}\gamma \rho^{\gamma-1}}{\gamma  - 1}= \e B_{en}(y_2).
\end{equation}
Define
\begin{equation*}
\mm(\rho,\n \varphi,\e A_{en},\e B_{en})=\frac{1+(\p_{y_1}\varphi)^2}{2(\p_{y_2}\varphi)^2}+\frac{\e A_{en}\gamma \rho^{\gamma+1}}{\gamma- 1}-\e B_{en}\rho^{2}.
\end{equation*}
 Since the flow is subsonic, we have
\begin{equation*}
\frac{\p }{\p\rho}\mm(\rho,\n \varphi,\e A_{en},\e B_{en})=\rho(c^2(\rho, \e A_{en})-u_{1}^2- u_{2}^2)>0.
\end{equation*}
 Hence the implicit function theorem implies that  $ \rho $ can be expressed as a function of  $ \n \varphi, \e A_{en}, \e B_{en} $, namely, $ \rho=\rho(\n \varphi, \e A_{en}, \e B_{en}) $.  Moreover, a direct computation yields that
\begin{equation*}
\begin{aligned}
\frac{\p \rho}{\p(\p_{y_1}\varphi)}&=-\frac{\p_{y_1}\varphi}{\rho(\p_{y_2}\varphi)^2
\left(c^2(\rho, \e A_{en})-\frac{1+(\p_{y_1}\varphi)^2}{( \rho\p_{y_2}\varphi)^2}\right)},\\
\frac{\p \rho}{\p(\p_{y_2}\varphi)}&=\frac{1+(\p_{y_1}\varphi)^2}
{\rho(\p_{y_2}\varphi)^3
\left(c^2(\rho, \e A_{en})-\frac{1+(\p_{y_1}\varphi)^2}{( \rho\p_{y_2}\varphi)^2}\right)},\\
\frac{\p \rho}{\p  \e A_{en}}&=-\frac{\gamma\rho^{\gamma}}{(\gamma-1)
\left(c^2(\rho, \e A_{en})-\frac{1+(\p_{y_1}\varphi)^2}{( \rho\p_{y_2}\varphi)^2}\right)},\\
\frac{\p \rho}{\p  \e B_{en}}&=\frac{\rho}{
c^2(\rho, \e A_{en})-\frac{1+(\p_{y_1}\varphi)^2}{( \rho\p_{y_2}\varphi)^2}}.\\
\end{aligned}
\end{equation*}
 Thus, the solution $ \bm  U $ can be expressed as a vector-value function of  $ \varphi, \e A_{en}, \e B_{en} $:
\begin{equation}\label{3-18}
\bm U=( \rho, u_1, u_2, P)
=\left( \rho,\frac{1}{ \rho\p_{y_2}\varphi},\frac{\p_{y_1}\varphi}{\rho \p_{y_2}\varphi}, \e A_{en} \rho^{\gamma}\right).
\end{equation}
\par Denote
\begin{equation*}
 u_2=W_1(\n \varphi, \e A_{en}, \e B_{en}), \quad  P=\e A_{en}(\rho(\n \varphi, \e A_{en}, \e B_{en}))^\gamma=W_2(\n \varphi, \e A_{en}, \e B_{en}).
\end{equation*}
 Then the Euler system \eqref{3-7} is reduced to the following nonlinear  second-order equation for $ \varphi $:
  \begin{equation}\label{3-19}
    \p_{y_1}W_1(\n \varphi^+, \e A_{en}^+, \e B_{en}^+)+
     \p_{y_2}W_2(\n \varphi^+, \e A_{en}^+, \e B_{en}^+)=0, \quad {\rm{in}} \quad \mn^+,
 \end{equation}
 and
 \begin{equation}\label{3-20}
     \p_{y_1}W_1(\n \varphi^-, \e A_{en}^-, \e B_{en}^-)+
     \p_{y_2}W_2(\n \varphi^-, \e A_{en}^-, \e B_{en}^-)=0, \quad {\rm{in}} \quad \mn^-,
 \end{equation}
 with the following boundary conditions:
 \begin{equation}\label{3-21}
   \begin{cases}
 \varphi^+(0,y_2)=\int_{0}^{y_2}\frac{1}{\e J_{en}^+}(s)\de s,\quad &{\rm{on}} \quad \Sigma_0^+,\\
\p_{y_1}\varphi^+(L,y_2)=\omega_{ex}(\varphi^+(L,y_2)), \quad  &{\rm{on}} \quad \Sigma_L^+,\\
  \varphi^+(y_1,0)=g_{cd}(y_1),  \ &{\rm{on}} \quad  \Sigma,\\
 \varphi^+(y_1,m^+)=g^+(y_1)-1+\int_{0}^{m^+}\frac{1}{\e J_{en}^+}(s)\de s, \quad &{\rm{on}} \quad \Sigma_w^+,\\
 \end{cases}
 \end{equation}
 and
  \begin{equation}\label{3-22}
 \begin{cases}
 \varphi^-(0,y_2)=-\int_{y_2}^{0}\frac{1}{\e J_{en}^-}(s)\de s,\quad &{\rm{on}} \quad \Sigma_0^-,\\
 \p_{y_1}\varphi^-(L,y_2)=\omega_{ex}(\varphi^-(L,y_2)), \quad  &{\rm{on}} \quad \Sigma_L^-,\\
  \varphi^-(y_1,0)=g_{cd}(y_1),  \ &{\rm{on}} \quad  \Sigma,\\
 \varphi^+(y_1,-m^-)=g^-(y_1)+1-\int_{-m^-}^{0}\frac{1}{\e J_{en}^-}(s)\de s, \quad &{\rm{on}} \quad \Sigma_w^-.\\
 \end{cases}
 \end{equation}
  Thus the flow angular at  the exit
  becomes local and nonlinear after introducing the stream function.
   \par  On the contact discontinuity,   \eqref{3-12} becomes
    \begin{equation}\label{3-23}
   W_2(\n \varphi^+,\e A_{en}^+,\e B_{en}^+ )(y_1,0)=
    W_2(\n \varphi^-,\e A_{en}^-,\e B_{en}^- )(y_1,0),
    \quad {\rm{on}} \quad \Sigma.\\
     \end{equation}
      \par Therefore,  Problem \rm{2.1} is reformulated as follows.
 \begin{problem}
  Given  functions $ (J_{en}^\pm, A_{en}^\pm, B_{en}^\pm)$ at the entrance and function $ \omega_{ex}$ at the exit, find a unique piecewise smooth subsonic solution $ (\varphi^+ ,\varphi^-;g_{cd}) $ separated by the straight line $ \Sigma $ such that the following  properties satisfied:
  \begin{enumerate}[\rm(i)]
\item
  $\varphi^+ $ and $ \varphi^-  $ are $ C^2 $ solutions to nonlinear second-order  equations \eqref{3-18} in $ \mn^+ $ and \eqref{3-19}  in $ \mn^- $, respectively;
  \item The   boundary conditions \eqref{3-21}-\eqref{3-23} are   satisfied.
      \end{enumerate}
 \end{problem}
 To prove Theorem 2.2, we first introduce some weight H$\ddot{\rm{o}}$lder spaces and their norms in rectangle $ \mn^\pm $. For $ \mathbf{y}=(y_1,y_2) $ and $ \e{\mathbf{y}}=(\e y_1,\e y_2)\in \mn^+ $, set
\begin{equation*}
 \delta_{\mathbf{y}}:=\min(y_2,m^+-y_2)\quad {\rm{and}} \quad
  \delta_{\textbf{y},\tilde{\textbf{y}}}:
  =\min(\delta_{\mathbf{y}},\delta_{\tilde{\mathbf{y}}}).
  \end{equation*}
  For any positive integer $ m $, $ \alpha\in(0,1) $ and $ \kappa\in \mathbb{R} $ and  a function $ u^+ $ defined on $ \mn^+ $, we define:
  \begin{equation*}
  \begin{aligned}
  {[u^+]}_{k,0;\mn^+}^{(\kappa)}\ &:=\sup_{|\beta|=k} \delta_{\mathbf{y}}^{\max(|\beta|+\kappa,0)}|D^{\beta}u^+(\textbf{y})|, \ k=0,1,\cdots,m;\\
  [u^+]_{m,\alpha;\mn^+}^{(\kappa)}\ &:=\sup_{|\beta|=m}\delta_{\mathbf{y},\tilde{\mathbf{y}}}^{\max(m+\alpha+\kappa,0)}
  \frac{|D^{\beta}u^+(\textbf{y})-D^{\beta}u^+(\tilde{\textbf{y}})|}{|\textbf{y}-\tilde{\textbf{y}} |^{\alpha}};\\
  \|u^+\|_{m,\alpha;\mn^+}^{(\kappa)}\ &:=\sum_{k=0}^{m}[u^+]_{k,0,\mn^+}^{(\kappa)}
  +[u^+]_{m,\alpha;\mn^+}^{(\kappa)}\\
  \end{aligned}
  \end{equation*}
  with the corresponding function space defined as
  \begin{equation*}
  C_{m,\alpha}^{(\kappa)}(\mn^+)=\{u^+:\|u^+\|_{m,\alpha;\mn^+}^{(\kappa)}<\infty\}.
  \end{equation*}
 \par  Similarly,  for $ \mathbf{y}=(y_1,y_2) $ and $ \e{\mathbf{y}}=(\e y_1,\e y_2)\in \mn^- $, set
\begin{equation*}
 \delta_{\mathbf{y}}:=\min(-y_2,m^-+y_2)\quad {\rm{and}} \quad
  \delta_{\textbf{y},\tilde{\textbf{y}}}:
  =\min(\delta_{\mathbf{y}},\delta_{\tilde{\mathbf{y}}}).
  \end{equation*}
Then for a function $ u^- $ defined on $ \mn^- $, one can define $   {[u^-]}_{k,0;\mn^-}^{(\kappa)} $,  $ [u^-]_{m,\alpha;\mn^-}^{(\kappa)} $, $ \|u^-\|_{m,\alpha;\mn^-}^{(\kappa)} $ and $  C_{m,\alpha}^{(\kappa)}(\mn^-) $, respectively.
  \par For $ \Sigma=(0,L) $ and a function $ v $ defined on $ \Sigma $,
   we can define $   {[v]}_{k,0;\Sigma }^{(\kappa)} $,  $ [v]_{m,\alpha;\Sigma }^{(\kappa)} $, $ \|v\|_{m,\alpha;\Sigma }^{(\kappa)} $ and $  C_{m,\alpha}^{(\kappa)}(\Sigma ) $, respectively.
\par Thus Theorem 2.2 will follows from the following theorem:
 \begin{theorem}
     Let  $ \varphi _b^\pm(y_2)=\frac{y_2}{m^\pm} $. Then
 there exist positive constants $\sigma_{cd}^\ast $ and $ \mc^\ast $ depending only on  $ ({\bm{U}}_b^+,{\bm{U}}_b^-,L,\alpha) $ such that  if
   \begin{equation}\label{3-24}
   \begin{aligned}
  \sigma(\bm U_{en}^+,\bm U_{en}^-,\omega_{ex},g^+,g^-)\leq  \sigma_{cd}^\ast ,
   \end{aligned}
   \end{equation}
  \rm{Problem 3.1} has a unique piecewise smooth subsonic solution $ (\varphi^+,\varphi^-;g_{cd}(y_1))$ satisfying properties $ \rm(i) $ and $ \rm(ii) $.  Furthermore, the following estimate holds:
    \begin{equation}\label{3-27}
  \|\varphi^+ -\varphi_b^+\|_{2,\alpha;\mn^+}^{(-1-\alpha)}
    +\|\varphi^- -\varphi_b^-\|_{2,\alpha;\mn^-}^{(-1-\alpha)}+ \|g_{cd} \|_{2,\alpha;\Sigma}^{(-1-\alpha)}
   \leq\mc^\ast \sigma(\bm U_{en}^+,\bm U_{en}^-,\omega_{ex},g^+,g^-).
   \end{equation}
 \end{theorem}
 \subsection{Solving the free boundary problem 3.1}\noindent
\par  Note that  Problem 3.1 is a free boundary problem since the function $ g_{cd} $ is unknown.  Inspired by the recent result in \cite{CXZ22}, we will use the implicit function theorem to solve this free boundary problem. To this end, we first state the implicit function theorem.
\begin{theorem}
   (Implicit function theorem \cite[Theorem 4.B]{ZE92}). Let $ X $, $Y$, $Z $ be Banach spaces and let $ U(x_0,y_0) $ be an open neighborhood of $ (x_0,y_0) $ in $ X\times Y $. Suppose that the mapping $ F:U(x_0,y_0)\rightarrow Z $ satisfies the following conditions:
   \begin{enumerate}[\rm(i)]
\item $ F(x_0,y_0)=0 $;
  \item The partial Fr\'{e}chet derivative $ D_yF $ exists on $ U(x_0,y_0) $ and $ D_yF(x_0,y_0)$: $Y\rightarrow Z  $ is bijective;
  \item $ F $ and $ D_yF $ are continuous at $ (x_0,y_0) $.
      \end{enumerate}
      Then the following hold:
    \begin{enumerate}[\rm(1)]
 \item Existence and uniqueness. There exist positive numbers $ r_0 $ and $ r $ such that,
for every $ x \in X $ satisfying $ \| x - x_0 \| < r_0 $,  $F(x,y) = 0  $ has a unique solution  $ y(x) \in Y $  with
 $ \|y(x) - y_0 \|<r $.
 \item  Continuity. If $ F $ is continuous in a neighborhood of $(x_0, y_0) $, then $ y(\cdot) $ is
continuous in a neighborhood of $ x_0 $.
\item Continuous differentiability. If $ F $ is a $ C^m $-map, $ 1 \leq m < \infty $, on a neighborhood of $ (x_0, y_0) $, then $ y(\cdot) $ is also a $ C^m $-map on a neighborhood of $x_0 $.
     \end{enumerate}
     \end{theorem}
     %For the proof  of implicit function theorem, one may refer to   in , here we omit it.
     \par Then we follow the steps below to solve Problem 3.1:
     \begin{enumerate}[ \bf (a)]
 \item
 Given any function $ g_{cd}(y_1)=\int_{0}^{y_1}\eta(s)\de s $ belonging to some suitable function classes, we  will solve the nonlinear second-order equation \eqref{3-19}(\eqref{3-20}) with nonlinear and mixed boundary condition \eqref{3-21}(\eqref{3-22}) in $ \mn^+ $( in $ \mn^- $). This will be achieved by constructing suitable barrier functions and employing Schauder estimates for second-order elliptic equation. The detailed analysis will be given in Section 4.
 \item Define the map $  \mq(\bm \zeta_0,\eta):=(W_2(\n \varphi^+,\e A_{en}^+,\e B_{en}^+ )-  W_2(\n \varphi^-,\e A_{en}^-,\e B_{en}^- ))(y_1,0) $. To employ the implicit function theorem, we need to  compute   the Fr\'{e}chet derivative $ D_\eta\mq(\bm \zeta_0,\eta)$ of the functional $ \mq(\bm \zeta_0,\eta)$ with respect to $ \eta $ and show that $D_\eta\mq(\bm\zeta_b,0) $ is an  isomorphism. This step will be achieved in  Section 5.
           \end{enumerate}
 \section{The  solution to a fixed boundary value problem in $\mn $}\noindent
\par In this section,  given any function $ g_{cd}(y_1)=\int_{0}^{y_1}\eta(s)\de s \in C_{2,\alpha}^{(-1-\alpha)}(\Sigma) $,
 we  will solve the nonlinear second-order equation \eqref{3-19}(\eqref{3-20}) with nonlinear and mixed boundary condition \eqref{3-21}(\eqref{3-22}) in $ \mn^+ $( in $ \mn^- $). For convenience, we will focus on $ \mn^+ $.
 \subsection{Linearization}\noindent
\par To solve nonlinear equation  \eqref{3-19} in the  domain $ \mn^+ $, we first linearize \eqref{3-19} and then solve
the linear equation in the  domain $ \mn^+ $.
\par Let $ \phi^+=\varphi^+-\varphi_b^+ $. Note that
 \begin{equation}\label{4-1}
\p_{y_1}W_1(\n \varphi_b^+, A_b^+, B_b^+)+
     \p_{y_2}W_2(\n \varphi_b^+,  A_b^+,  B_b^+)=0, \quad {\rm{in}} \quad \mn^+.
     \end{equation}
     Taking the difference of \eqref{3-19} and \eqref{4-1}, one has
     \begin{equation}\label{4-2}
\sum_{i,j=1,2}\p_{y_i}(a_{ij}(\n \phi^+)\p_{y_j}\phi^+)=\sum_{i=1,2}\p_{y_i}f_i^+, \quad {\rm{in}} \quad \mn^+,
     \end{equation}
     where
     \begin{equation*}
     \begin{aligned}
     a_{ij}(\n \phi^+)=\int_0^1 \p_{\p \varphi^+_j}
     W_i(\n\varphi_b^++s  \n \phi^+, \e A_{en}^+, \e B_{en}^+)\de s, \quad i,j=1,2,
     \end{aligned}
     \end{equation*}
     and
     \begin{equation*}
     \begin{aligned}
     f_i^+=W_i(\n\varphi_b^+, A_b^+,  B_b^+)-W_i(\n\varphi_b^+, \e A_{en}^+, \e B_{en}^+) , \quad i=1,2.
     \end{aligned}
     \end{equation*}
     \par   Define an iteration set
\begin{equation}\label{4-3}
\mj_{upp}(\delta_1)=\{\phi^+:\phi^+(0,0)=0, \|\phi^+\|_{2,\alpha;\mn^+}^{(-1-\alpha)}
\leq
\delta_1\},
\end{equation}
where $ \delta_1 $ is a positive constant to be determined later. Then for given $ \bar\varphi^+ $ such that $ \bar\phi^+=\bar\varphi^+-\varphi^+_b\in \mj_{upp}(\delta_1) $, find $ \varphi^+=\varphi^+_b+  \phi^+ $ by  solving the  linear equation:
 \begin{equation}\label{4-4}
\sum_{i,j=1,2}\p_{y_i}(a_{ij}(\n \bar\phi^+)\p_{y_j}\phi^+)=\sum_{i=1,2}\p_{y_i}f_i^+, \quad {\rm{in}} \quad \mn^+,
      \end{equation}
%       \par Next, we derive the boundary conditions for $ \phi^+ $.
%    It follows from   \eqref{3-21} that
with the following boundary conditions arising from \eqref{3-21}:
     \begin{equation}\label{4-5}
     \begin{cases}
\phi^+(0,y_2)=g_{1}^+(y_2), &\quad {\rm{on}} \quad
  \Sigma_0^+,\\
 \phi^+(y_1,0)=g_{cd}(y_1), &\quad {\rm{on}} \quad
  \Sigma,\\
  \phi^+(y_1,m^+)=g_{2}^+(y_1), &\quad {\rm{on}} \quad
  \Sigma_w^+,\\
  \p_{y_1}\phi^+(L,y_2)=\e\omega_{ex}^+(y_2), &\quad {\rm{on}} \quad
  \Sigma_L^+,\\
  \end{cases}
  \end{equation}
  where
  \begin{equation*}
     \begin{aligned}
    g_{1}^+(y_2)&=\int_0^{y_2}\left(\frac{1}{\e J_{en}^+(s)}-\frac{1}{ J_b^+}\right)\de s,\\
    g_{2}^+(y_1)&=g^+(y_1)-1+\int_0^{m^+}\left(\frac{1}{\e J_{en}^+(s)}-\frac{1}{ J_b^+}\right)\de s,\\
    \e\omega_{ex}^+(y_2)&= \omega_{ex}(\varphi_b^+(y_2)+\bar\phi^+(L,y_2)).
    \end{aligned}
    \end{equation*}
    \par Use the abbreviation
    \begin{equation*}
    \sigma_v=\sigma(\bm U_{en}^+,\bm U_{en}^-,\omega_{ex},g^+,g^-),
    \end{equation*}
    where $\sigma(\bm U_{en}^+,\bm U_{en}^-,\omega_{ex},g^+,g^-)$  is  defined in \eqref{2-11}.
 Then a direct computation yields that
\begin{equation}\label{4-6}
\begin{cases}
\begin{aligned}
&\sum_{i=1}^2\| f_i^+\|_{1,\alpha;\mn^+}^{(-\alpha)}
\leq\mc^+ \sigma_v,\\
  &\|g_{1}^+\|_{2,\alpha;\Sigma_0^+}^{(-1-\alpha)}
\leq \mc^+ \|\e J_{en}^+ -J_b^+\|_{1,\alpha;\Sigma_0^+}^{(-\alpha)}
\leq\mc^+\sigma_v,\\
&\|g_{2}^+\|_{2,\alpha;
{\Sigma_w^+}}^{(-1-\alpha)}
\leq \mc^+\left(
\|g^+(y_1)-1\|_{2,\alpha;{\Sigma_w^+}}^{(-1-\alpha)}+\|\e J_{en}^+ -J_b^+\|_{1,\alpha;\Sigma_0^+}^{(-\alpha)}\right)\leq\mc^+\sigma_v,\\
&\|\e\omega_{ex}^+\|_{1,\alpha;\Sigma_L^+}^{(-\alpha)}\leq \mc^+(\sigma_v^+\delta_1+ \sigma_v),
\end{aligned}
\end{cases}
  \end{equation}
  where $ \mc^+ $ is a positive constant depending only on $ \bm U_b^+ $, $L$ and $ \alpha $.
   \subsection{Solving the linear boundary value problem  }\noindent
  \par In this subsection, we consider the following linear boundary value problem:
  \begin{equation}\label{4-7}
  \begin{cases}
\sum_{i,j=1,2}\p_{y_i}(a_{ij}(\n \bar\phi^+)\p_{y_j}\phi^+)=\sum_{i=1}^2\p_{y_i}f_i^+, &\quad {\rm{in}} \quad \mn^+,\\
\phi^+(0,y_2)=g_{1}^+(y_2), &\quad {\rm{on}} \quad
  \Sigma_0^+,\\
 \phi^+(y_1,0)=g_{cd}(y_1), &\quad {\rm{on}} \quad
  \Sigma,\\
  \phi^+(y_1,m^+)=g_{2}^+(y_1), &\quad {\rm{on}} \quad
  \Sigma_w^+,\\
  \p_{y_1}\phi^+(L,y_2)=\e\omega_{ex}^+(y_2), &\quad {\rm{on}} \quad
  \Sigma_L^+.\\
  \end{cases}
      \end{equation}
      For the coefficients of \eqref{4-7}, it is easy to see that
\begin{equation}\label{4-8}
\|a_{ij}-e_i^+\delta_{ij}\|_{1,\alpha;\mn^+}^{(-\alpha)}\leq \kappa^+,
\end{equation}
  where $\kappa^+$  is positive constant which depends on $ \delta_1 $,  $ \bm U_b^+ $, $ L $ and $ \alpha $ and
   \begin{equation}\label{4-9}
 e_{1}^+=a_{11}(0 )=u_b^+>0,\quad
 e_{2}^+=a_{22}(0 )=\frac{  (c_b^+)^2(u_b^+)^3(\rho_b^+)^2}
{(c_b^+)^2-(u_b^+)^2}>0,
\end{equation}
and  $ a_{12}(0)=a_{21}(0)=0 $. Hence we have
\begin{equation*}
a_{11}(0 )a_{22}(0)-(a_{12}(0))^2= \frac{(c_b^+)^2(\rho_b^+)^2(u_b^+)^4}
   {(c_b^+)^2-(u_b^+)^2}>0.
   \end{equation*}
    If  $\kappa^+  $ is  sufficiently
small, there exists a constant $ \lambda^+>0 $ depending only on $ {\bm U}_b^+ $  such that for any $ \bm \xi \in \mathbb{R}^2 $, it holds that
\begin{equation}\label{4-10}
 \lambda^+ |\bm \xi|^2\leq a_{ij}\xi_i\xi_j\leq \frac{1}{\lambda^+}|\bm \xi|^2.
 \end{equation}
 \par Then for the   problem \eqref{4-7}, we have the following conclusion:
   \begin{lemma}
    Under the above assumptions, there exists a unique solution $ \phi^+ \in C_{2,\alpha}^{(-1-\alpha)}(\mn^+) $ to the boundary value problem \eqref{4-7}. Furthermore, the solution $  \phi^+$ satisfies
\begin{equation}\label{4-11}
\begin{aligned}
&\|\phi^+\|_{2,\alpha;\mn^+}^{(-1-\alpha)}
\leq \mc^+\left(\sum_{i=1}^2\| f_i^+\|_{1,\alpha;\mn^+}^{(-\alpha)}
+\|g_{cd}\|_{2,\alpha;\Sigma}^{(-1-\alpha)}
+\|g_{2}^+\|_{2,\alpha;{\Sigma_w^+}}^{(-1-\alpha)}\right.\\
&\qquad\qquad\qquad\qquad\left.
+\|g_{1}^+\|_{2,\alpha;\Sigma_0^+}^{(-1-\alpha)}
+\|\e\omega_{ex}^+\|_{1,\alpha;\Sigma_L^+}^{(-\alpha)}\right),\\
\end{aligned}
\end{equation}
where the constant $ \mc^+>0 $ depends on  $ \bm U_b^+ $, $L$ and $ \alpha $.
  \end{lemma}
 \par In order to prove Lemma 4.1, we first  consider an auxiliary problem:
  \begin{equation}\label{4-12}
  \begin{cases}
e_1^+\p_{y_1}^2\phi^++e_2^+\p_{y_2}^2\phi^+
=\sum_{i=1}^2\p_{y_i}h_i^+, &\quad {\rm{in}} \quad \mn^+,\\
\phi^+(0,y_2)=g_{1}^+(y_2), &\quad {\rm{on}} \quad
  \Sigma_0^+,\\
 \phi^+(y_1,0)=g_{cd}(y_1), &\quad {\rm{on}} \quad
  \Sigma,\\
 \phi^+(y_1,m^+)=g_{2}^+(y_1), &\quad {\rm{on}} \quad
  \Sigma_w^+,\\
  \p_{y_1}\phi^+(L,y_2)=\e\omega_{ex}^+(y_2), &\quad {\rm{on}} \quad
  \Sigma_L^+.\\
  \end{cases}
      \end{equation}
      \begin{lemma}
  If  $h_i^+\in C_{1,\alpha}^{(-\alpha)}(\mn^+) $ $ (i=1,2)$,
   the boundary value problem \eqref{4-12} has a unique solution $ \phi^+ \in C_{2,\alpha}^{(-1-\alpha)}(\mn^+) $ such that
\begin{equation}\label{4-13}
\begin{aligned}
&\|\phi^+\|_{2,\alpha;\mn^+}^{(-1-\alpha)}
\leq \mc^+\left(\sum_{i=1}^2\| h_i^+\|_{1,\alpha;\mn^+}^{(-\alpha)}
+\|g_{cd}\|_{2,\alpha;\Sigma}^{(-1-\alpha)}
+\|g_{2}^+\|_{2,\alpha;{\Sigma_w^+}}^{(-1-\alpha)}\right.\\
&\qquad\qquad\qquad\qquad\left.+\|g_{1}^+\|_{2,\alpha;\Sigma_0^+}^{(-1-\alpha)}+\|\e\omega_{ex}^+\|_{1,\alpha;\Sigma_L^+}^{(-\alpha)}\right),
\end{aligned}
\end{equation}
where $ \mc^+>0 $ depends on  $ e_i^+ $, $L$ and $ \alpha $.
\end{lemma}
\begin{proof}
By \cite[Theorem 1]{LG86}, the boundary value problem \eqref{4-12} has a unique solution
\begin{equation*}
\phi^+\in C^{2,\alpha}(\mn^+)\cap C^0(\overline{\mn^+}).
\end{equation*}
To obtain the  estimate of $\phi^+ $  in the weighted norm $ C_{2,\alpha;\mn^+}^{(-1-\alpha)} $, we modify the arguments developed in \cite{CCF07,LXY13}.
\par First, employing the transformation $ (\tilde y_1,\tilde y_2)=\left(\frac{y_1}{\sqrt{e_1^+}},\frac{y_2}{\sqrt{e_2^+}}\right)$, the operator $ e_1^+\p_{y_1}^2+e_2^+\p_{y_2}^2 $ is transformed into the Laplace operator. Thus, for convenience, below we will replace \eqref{4-12}  by
  \begin{equation}\label{4-14}
  \begin{cases}
\Delta \phi^+
=\sum_{i=1}^2\p_{y_i}h_i^+, &\quad {\rm{in}} \quad \mn^+,\\
\phi^+(0,y_2)=g_{1}^+(y_2), &\quad {\rm{on}} \quad
  \Sigma_0^+,\\
 \phi^+(y_1,0)=g_{cd}(y_1), &\quad {\rm{on}} \quad
  \Sigma,\\
 \phi^+(y_1,m^+)=g_{2}^+(y_1), &\quad {\rm{on}} \quad
  \Sigma_w^+,\\
  \p_{y_1}\phi^+(L,y_2)=\e\omega_{ex}^+(y_2), &\quad {\rm{on}} \quad
  \Sigma_L^+.\\
  \end{cases}
      \end{equation}
 We will divide into the following five steps   to derive the estimate of the solution $ \phi^+$ for the boundary value problem \eqref{4-14}.
\par  { \bf Step 1. Homogenize the boundary data.}
\par  We first seek a function $ \phi^+_1\in  C_{2,\alpha}^{(-1-\alpha)}(\mn^+) $ satisfying $ \p_{y_1}\phi^+_1(L,y_2)=\e\omega_{ex}^+(y_2) $. It follows from Lemma 6.38 in \cite{GT98} that $ \e\omega_{ex}^+$ can be extended outside $ \Sigma_{L}^+ $ so that $ \e\omega_{ex}^+\in C_0^{1,\alpha}(\mathbb{R}) $. Let $ \xi(t) $ be a smooth mollifier satisfying $ \xi(t)\geq 0 $ and  $ \int_{-\infty}^{\infty}\xi(t)\de t=1 $. We define
\begin{equation*}
  \phi^+_1(y_1,y_2)=(y_1-L)\int_{-\infty}^{\infty}
 \e\omega_{ex}^+(y_2-(y_1-L)t)\xi(t)\de t.
  \end{equation*}
  Then one has $ \p_{y_1}\phi_1^+(L,y_2)=\e\omega_{ex}^+(y_2) $ and
  \begin{equation}\label{4-15}
  \|\phi^+_1\|_{2,\alpha;\mn^+}^{(-1-\alpha)}\leq C\|\e\omega_{ex}^+\|_{1,\alpha;\Sigma_L^+}^{(-\alpha)}.
  \end{equation}
  Set $ \phi^+_2=\phi^+-\phi^+_1 $, then $ \phi^+_2$ satisfies
  \begin{equation}\label{4-16}
  \begin{cases}
\Delta \phi^+_2
=\sum_{i=1}^2\p_{y_i}h_{i+2}^+, &\quad {\rm{in}} \quad \mn^+,\\
\phi^+_2(0,y_2)=\psi_1^+(y_2), &\quad {\rm{on}} \quad
  \Sigma_0^+,\\
 \phi^+_2(y_1,0)=\psi_2^+(y_1), &\quad {\rm{on}} \quad
  \Sigma,\\
 \phi^+_2(y_1,m^+)=\psi_3^+(y_1), &\quad {\rm{on}} \quad
  \Sigma_w^+,\\
  \p_{y_1}\phi^+_2(L,y_2)=0, &\quad {\rm{on}} \quad
  \Sigma_L^+,\\
  \end{cases}
      \end{equation}
      where
   \begin{equation*}
  \begin{aligned}
  &h_3^+=h_1^+-\p_{y_1}\phi^+_1, \quad    h_4^+=h_2^+-\p_{y_2}\phi^+_1,\quad
  \psi_1^+(y_2)=g_{1}^+(y_2)-\phi^+_1(0,y_2),\\
  &\psi_2^+(y_1)=g_{cd}(y_1)-\phi^+_1(y_1,0),\quad
  \psi_3^+(y_1)=g_{2}^+(y_1)-\phi^+_1(y_1,m^+).
  \end{aligned}
      \end{equation*}
  Next, define
  \begin{equation*}
 \phi^+_3=\phi^+_2  -\left(\psi_1^+(y_2)+\frac{y_2}{m^+}(\psi_3^+(y_1)-\psi_3^+(0) ) +\left(1-\frac{y_2}{m^+}\right)(\psi_2^+(y_1)-\psi_2^+(0)) \right).
  \end{equation*}
  Then $ \phi^+_3$ satisfies
  \begin{equation}\label{4-17}
  \begin{cases}
\Delta \phi^+_3
=\sum_{i=1}^2\p_{y_i}h_{i+4}^+, &\quad {\rm{in}} \quad \mn^+,\\
\phi^+_3(0,y_2)=0, &\quad {\rm{on}} \quad
  \Sigma_0^+,\\
 \phi^+_3(y_1,0)=0, &\quad {\rm{on}} \quad
  \Sigma,\\
 \phi^+_3(y_1,m^+)=0, &\quad {\rm{on}} \quad
  \Sigma_w^+,\\
  \p_{y_1}\phi^+_3(L,y_2)=0, &\quad {\rm{on}} \quad
  \Sigma_L^+,\\
  \end{cases}
      \end{equation}
       where
   \begin{equation*}
  \begin{aligned}
  &h_5^+=h_3^+- \left(\frac{y_2}{m^+}(\psi_3^+)^\prime(y_1) +\left(1-\frac{y_2}{m^+}\right)(\psi_2^+)^\prime(y_1) \right), \\
  & h_6^+=h_4^+-\left((\psi_1^+)^\prime(y_2)
  +\frac{1}{m^+}(\psi_3^+(y_1)-\psi_3^+(0) ) -\frac{1}{m^+}(\psi_2^+(y_1)-\psi_2^+(0)) \right).
  \end{aligned}
      \end{equation*}
      \par  {\bf Step 2. The $ L^\infty  $-estimate of $\phi^+_3 $.}
\par To obtain \eqref{4-13}, we first need to the $ L^\infty  $-estimate of $\phi^+_3 $. To achieve this, define the following comparison function:
 \begin{equation*}
\begin{aligned}
v_1
=M\left
(1+\left(\frac{3L}{2}-y_1\right)^2+M_1(y_2^\alpha+(m^+-y_2)^\alpha)\right),
\end{aligned}
\end{equation*}
where $ M=\sum_{i=1,2}\|h_{i+4}\|_{1,\alpha;\mn^+}^{(-\alpha)}$ and $ M_1=\frac{4}{\alpha(1-\alpha)}(m^+)^{2-\alpha} $.
\par A direct  computation yields that
\begin{equation*}
\begin{aligned}
\Delta v_1&= -MM_1\alpha(1-\alpha)
(y_2^{\alpha-2}+(m^+-y_2)^{\alpha-2})+2M\\
 &=-MM_1\alpha(1-\alpha)
(y_2^{\alpha-2}+(m^+-y_2)^{\alpha-2})
+\frac{\alpha(1-\alpha)}{2}MM_1(m^+)^{\alpha-2}\\
&\leq -MM_1\alpha(1-\alpha)
(y_2^{\alpha-2}+(m^+-y_2)^{\alpha-2})
+\frac{\alpha(1-\alpha)}{2}MM_1(y_2^{\alpha-2}+(m^+-y_2)^{\alpha-2})\\
 &\leq-\frac{1}{2m^+}MM_1\alpha(1-\alpha)
(y_2^{\alpha-1}+(m^+-y_2)^{\alpha-1}).\\
\end{aligned}
\end{equation*}
On the boundary $ \Sigma_{L}^+$, one derives that
\begin{equation*}
\p_{y_1}v_1(L,y_2)=-{ML}.
\end{equation*}
 Thus, for sufficiently large positive constant $ C_1 $ independent of $ M $,  we have
\begin{equation*}
\begin{cases}
\Delta(C_1v_1\pm \phi^+_3)\leq 0,\quad &{\rm{in}} \quad
\mn^+,\\
(C_1v_1\pm \phi^+_3)(0,y_2)>0,\quad &{\rm{on}} \quad \Sigma_0^+.\\
(C_1v_1\pm \phi^+_3)(y_1,0)>0,
 &\ {\rm{on}} \quad
\Sigma,\\
(C_1v_1\pm \phi^+_3)(y_1,m^+)>0,  &{\rm{on}} \quad
\Sigma_w^+,\\
\p_{y_1}(C_1v_1\pm \phi^+_3)(L,y_2)< 0 , \ &{\rm{on}} \quad
\Sigma_L^+.\\
\end{cases}
\end{equation*}
Then it follows from the comparison principle that

\begin{equation}\label{4-18}
\|\phi^+_3\|_{L^\infty}\leq C_1M, \quad {\rm{in }} \quad \mn^+.
\end{equation}
\par {\bf Step 3. The  estimate of $ \phi_3^+ $ near the left corner points $ (0,0) $ and $ (0,m^+) $.}
\par To obtain a better maximum bound, we construct a comparison function $ v_2 $. We  focus on  the behavior of $\phi_3^+ $ near the corner point $ (0,0) $. The  corner point $ (0,m^+) $ can be treated similarly.
\par Define a comparison function $ v_2 $ in polar coordinates as follows
\begin{equation}\label{4-19}
v_2( r,\theta)=\frac{2}{\sin(\frac{1-\alpha}{5})} r^{1+\alpha}\sin(z( \theta))-r^{1+\alpha}\sin^{1+\alpha} ( \theta),\quad 0\leq \theta\leq \frac{\pi}{2},
\end{equation}
where
$  r=\sqrt{y_1^2+y_2^2} $, $ \theta=\arctan\frac{y_2}{y_1} $  and
\begin{equation*}
 z( \theta)=\frac{1-\alpha}{5}+\frac{6+4\alpha}{5} \theta, \quad \theta\in[0,\pi/2].
\end{equation*}
Note that $\frac{1-\alpha}{5}\leq z( \theta)\leq z\left(\frac{\pi}{2}\right)$ for $ \theta\in[0,\pi/2]$, and
\begin{equation*}
z\left(\frac{\pi}{2}\right)=
\frac{1-\alpha}{5}+\frac{6+4\alpha}{5}\frac{\pi}{2}=\pi-\frac{1-\alpha}{5}
-\frac{2(1-\alpha)}{5}(\pi-1)<\pi-\frac{1-\alpha}{5}.
\end{equation*}
 Thus $ \sin(z( \theta))\geq \sin(\frac{1-\alpha}{5}) $ and
\begin{equation}\label{4-20}
v_2(r, \theta)\geq r^{1+\alpha}\left(\frac{2\sin(\frac{1-\alpha}{5})}
{\sin(\frac{1-\alpha}{5})}-1\right)
\geq r^{1+\alpha}.
\end{equation}
Therefore
\begin{equation}\label{4-21}
\begin{aligned}
\Delta v_2&= (\p_{{r}}^2 + \frac{1}{{r}}\p_{{r}}+ \frac1{{r}^2}\p_{{\theta}}^2) v_2\\
&=\frac{2((5+5\alpha)^2-(6+4\alpha)^2)}{25
\sin(\frac{1-\alpha}{5})} r^{\alpha-1}\sin(z(\theta))-\alpha(1+\alpha) r^{\alpha-1}\sin^{\alpha-1} ( \theta)\\
&\leq -\alpha(1+\alpha) r^{\alpha-1}\sin^{\alpha-1} ( \theta)=-\alpha(1+\alpha)y_2^{\alpha-1}.\\
\end{aligned}
\end{equation}
\par Let  $  r_0>0 $ be small and fixed and set $ B_{ r_0}^+=B_{ r_0}(0,0)\cap \mn^+ $. Then it follows from
\eqref{4-18}-\eqref{4-21} that there exists a suitably large positive constant  $ C_2$ independent of $ M$   such that
\begin{equation*}
\begin{cases}
\Delta(C_2v_2\pm \phi_3^+)\leq 0, &\quad {\rm{in}} \quad B_{ r_0}^+,\\
C_2v_2\pm \phi_3^+>0, &\quad {\rm{on}} \quad  B_{ r_0}^+\cap
\p \mn^+,\\
C_2v_2\pm \phi_3^+>0,&\quad {\rm{on}} \quad  B_{ r_0}^+\cap \{r=r_0\}.\\
\end{cases}
  \end{equation*}
   This, together with the comparison principle, yields
\begin{equation}\label{4-22}
\|\phi_3^+\|_{L^\infty}\leq C_2 r^{1+\alpha}
M, \quad {\rm{in }} \quad B_{r_0}^+.
\end{equation}
\par {\bf Step 4. Some estimates for a related auxiliary problem.}
\par To estimate $ \phi^+_3 $ near the right corner points $ (L,0) $ and  $ (L,m^+) $ in $ \mn^+ $, we need to consider the following boundary value problem with Neumann boundary condition at $ y_1=0 $:
\begin{equation}\label{4-23}
\begin{cases}
\begin{aligned}
&\Delta v_3=\sum_{i=1}^2\p_{y_i}h_{i+4}^+, \quad {\rm{in}} \quad
\mn^+,\\
&\p_{y_1}v_3(0,y_2)= \p_{y_1}v_3(L,y_2)=0,\\
&v_3(y_1,0)=v_3(y_1,m^+)=0.
\end{aligned}
\end{cases}
\end{equation}
By the standard elliptic theory in \cite{GT98}, the boundary value problem \eqref{4-23} has a unique solution
\begin{equation*}
v_3\in C^{2,\alpha}(\mn^+)\cap C^{1,\alpha}(\overline{\mn^+}\setminus \{(0,0),(0,m^+),(L,0),(L,m^+)\})\cap C(\overline{\mn^+}).
\end{equation*}
Furthermore, one can  follow Step 1 to prove
\begin{equation}\label{4-24}
\|v_3\|_{0,0;\mn^+}\leq C\sum_{i=1}^{2}\|h_{i+4}\|_{1,\alpha;\mn^+}^{(-\alpha)}.
\end{equation}
 \par Set
\begin{equation}\label{4-25}
\begin{cases}
\e v_3(y_1,y_2)=v_3(y_1,y_2)\quad \e h_{5}^+(y_1,y_2)= h_{5}^+(y_1,y_2)-h_{5}^+(L,y_2),\\
\e h_{6}^+(y_1,y_2)= h_{6}^+(y_1,y_2),
\  {\rm{for}} \ 0\leq y_1\leq L;\\
\e v_3 (y_1,y_2)=v_3(-y_1,y_2),\quad \e h_{5}^+(y_1,y_2)=-\e h_{5}^+(-y_1,y_2),\\
\e h_{6}^+(y_1,y_2)= \e h_{6}^+(-y_1,y_2),
 \ {\rm{for}} \ -L\leq y_1\leq 0;\\
\e v_3 (y_1,y_2)=v_3(2L-y_1,y_2),\quad \e h_{5}^+(y_1,y_2)=-\e h_{5}^+(2L-y_1,y_2),\\
\e h_{6}^+(y_1,y_2)= \e h_{6}^+(2L-y_1,y_2),
 \ {\rm{for}} \ L\leq y_1\leq 2L.\\
\end{cases}
\end{equation}
Then $ \e v_3 $ solves the following problem
\begin{equation}\label{4-26}
\begin{cases}
\begin{aligned}
&\Delta \e v_3=\sum_{i=1}^2\p_{y_i} \e h_{i+4}^+
\quad {\rm{in}}\ E=(-L,2L)\times(0,m^+),\\
&\p_{y_1} \e v_3(-L,y_2)=\p_{y_1}\e v_3(2L,y_2)=0,\\
&\e v_3(y_1,0)=\e v_3(y_1,m^+)=0.
\end{aligned}
\end{cases}
\end{equation}
  It follows from the Schauder interior and boundary estimates in Theorem 8.32 and Corollary 8.36 of Chapter 8 in \cite{GT98}  that
\begin{equation}\label{4-27}
\begin{aligned}
\| v_3\|_{C^{1,\alpha}(\overline{\mn^+})}=\| \e v_3\|_{C^{1,\alpha}(E)}&
\leq C\left(\|\e v_3\|_{L^{\infty}( E)}+
\sum_{i=1}^{2}\|\e h_{i+4}^+\|_{C^{\alpha}(\overline{E})}\right)\\
&\leq C\left(\| v_3\|_{L^{\infty}(\overline{\mn^+})}+
\sum_{i=1}^{2}\|h_{i+4}^+\|_{1,\alpha;\mn^+}^{(-\alpha)}
\right).\\
\end{aligned}
\end{equation}
Combining the estimate \eqref{4-24}  yields that
\begin{equation}\label{4-28}
\|v_3\|_{C^{1,\alpha}(\overline{\mn^+})}\leq C
\sum_{i=1}^{2}\|h_{i+4}^+\|_{1,\alpha;\mn^+}^{(-\alpha)}.
\end{equation}
\par Next, we derive the $ C_{2,\alpha}^{(-1-\alpha)}(\mn^+) $ estimate  on $ v_3 $. By \eqref{4-23}, $ \p_{y_1}v_3 $ solves
\begin{equation}\label{4-29}
\begin{cases}
\begin{aligned}
&\Delta (\p_{y_1}v_3)=\p_{y_1}\left(\sum_{i=1}^2\p_{y_i}  h_{i+4}^+\right),\quad  {\rm{in}}\quad \mn^+,\\
&\p_{y_1}v_3(0,y_2)=\p_{y_1}v_3(L,y_2)=0,\\
&\p_{y_1}v_3(y_1,0)=\p_{y_1}v_3(y_1,m^+)=0.
\end{aligned}
\end{cases}
\end{equation}
Let $ \bm y^\ast=(y_{1}^\ast,y_{2}^\ast) $ be fixed point in $ \mn^+ $. Without loss of generality, we assume that $ y_{2}^\ast\leq m^+-y_{2}^\ast $ and set $ \mn_1^+=\mn^+\cap\{\bm y: \frac{y_{2}^\ast}{2}<y_2<\frac{3y_{2}^\ast}{2}\} $. By the Schauder estimate and standard scaling argument, one has
\begin{equation*}
\begin{aligned}
&\sum_{|\beta|=1}\left((y_{2}^\ast)^{|\beta|}
\|D^{\beta}\p_{y_1}v_3\|_{L^{\infty}(\mn_1^+)}+
(y_{2}^\ast)^{|\beta|+\alpha}[D^{\beta}\p_{y_1}v_3]_{\alpha;\mn_1^+}\right)\\
&\quad\quad\leq C\left(\|\p_{y_1}v_3\|_{L^{\infty}(\mn_1^+)}
+(y_{2}^\ast)^{\alpha}\sum_{i=1}^{2}\|h_{i+4}^+\|
_{1,\alpha;\mn^+}^{(-\alpha)}\right).
\end{aligned}
\end{equation*}
Combining with the boundary conditions on $ y_2=0 $ and $ y_2=m^+ $ in \eqref{4-26} and the estimate \eqref{4-28} yields
\begin{equation*}
\begin{aligned}
&\|v_3\|_{C^{1,\alpha}(\overline{\mn^+})}
+\sum_{|\beta|=1}\left((y_{2}^\ast)
^{|\beta|-\alpha}\|D^{\beta}\p_{y_1}v_3\|_{L^{\infty}(\mn_1^+)}+
(y_{2}^\ast)^{|\beta|}[D^{\beta}\p_{y_1}v_3]_{\alpha;\mn_1^+}\right)\\
&\quad\leq C\sum_{i=1}^{2}\|h_{i+4}^+\|
_{1,\alpha;\mn^+}^{(-\alpha)}.
\end{aligned}
\end{equation*}
Thus we have
\begin{equation*}
\|v_3\|_{C^{1,\alpha}(\overline{\mn^+})}+\|\p_{y_1}v_3\|_{1,\alpha;\mn^+}^{(-\alpha)}\leq C
\sum_{i=1}^{2}\|h_{i+4}^+\|
_{1,\alpha;\mn^+}^{(-\alpha)}.
\end{equation*}
This, together with the equation \eqref{4-23}, yields that
\begin{equation}\label{4-30}
\|v_3\|_{2,\alpha;\mn^+}^{(-1-\alpha)}\leq C\sum_{i=1}^{2}\|h_{i+4}^+\|
_{1,\alpha;\mn^+}^{(-\alpha)}.
\end{equation}
\par {\bf Step 5. The $ C_{2,\alpha}^{(-1-\alpha)}( \mn^+) $ estimate of $ \phi^+ $.}
\par It follows from \eqref{4-14} and \eqref{4-23} that $ \phi^+_3-v_3 $ satisfies
\begin{equation}\label{4-31}
\begin{cases}
\Delta(\phi^+_3-v_3)=0,\quad {\rm{in}}\ \mn^+,\\
(\phi^+_3-v_3)(0,y_2)=-v_3(0,y_2),\\
(\phi^+_3-v_3)(y_1,0)=(\phi^+_3-v_3)(y_1,m^+)=0,\\
\p_{y_1}(\phi^+_3-v_3)(L,y_2)=0.\\
\end{cases}
\end{equation}
\par Set $ \phi^+_4=\phi^+_3-v_3+\p_{y_2}v_3(0,0)y_2 $. Then $ \phi^+_4 $ solves
\begin{equation*}
\begin{cases}
\Delta\phi^+_4=0
,\quad {\rm{in}}  \quad B_{ r_0}^+,\\
\phi^+_4(y_1,0)=0,\\
\phi^+_4(0,y_2)=-v_3(0,y_2)+\p_{y_2}v_3(0,0)y_2.
\end{cases}
\end{equation*}
A simple calculation yields that $ |\phi^+_4(0,y_2)|\leq \|v_3\|_{C^{1,\alpha}} y_2^{1+\alpha} $.  Then one can follow Step 2 to obtain that
\begin{equation}\label{4-32}
\|\phi^+_4\|_{L^{\infty}}\leq C
\|v_3(0,y_2)\|_{C^{1,\alpha}} r^{1+\alpha} \quad {\rm{in}}\  \quad B_{ r_0}^+.
\end{equation}
\par For any fixed point $ \bm y^0=(y_{1}^0,y_{2}^0)\in B_{ r_0/2}^+$, let $ d_{\bm y^0}=\frac{1}{2}\sqrt{(y_{1}^0)^2+(y_{2}^0)^2} $ and $ B_1^+=B_{d_{\bm y^0}/2}^+(\bm y^0) $ and $ B_2^+=B_{d_{\bm y^0}}^+(\bm y^0) $.  Then it follows from Theorem 6.2 and Corollary 6.3 and Corollary 6.7 in \cite{GT98} that
\begin{equation}\label{4-33}
\begin{aligned}
&\sum_{|\beta|=0}^2d_{\bm y^0}^{|\beta|}\|D^\beta \phi^+_4\|_{L^{\infty}(B_1^+)} +\sum_{|\beta|=2}d_{\bm y^0}^{2+\alpha}[D^\beta \phi^+_4]_{\alpha;B_1^+} \\
&\leq C\left(\| \phi^+_4\|_{L^{\infty}(B_2^+)}+d_{\bm y^0}^{1+\alpha}\|v_3(0,y_2)\|_{2,\alpha;\Sigma_0^+}
^{(-1-\alpha)}\right).\\
\end{aligned}
\end{equation}
Since $ d_{\bm y^0}\geq \frac{|y_2^0|}{2} $, hence substituting \eqref{4-32} into \eqref{4-33} yields
\begin{equation}\label{4-34}
\|\phi^+_4\|_{2,\alpha;B_1^+}^{(-1-\alpha)}\leq C\|v_3(0,y_2)\|_{2,\alpha;\Sigma_0^+}^{(-1-\alpha)}.
\end{equation}
\par Away form the corner points, we can also obtain an analogous estimate as in \eqref{4-34} by the interior and boundary estimates in Chapter 6 of \cite{GT98}. This, together with the expressions of $ \phi^+_i  $ and $ h_i^+ $, yields
\begin{equation}\label{4-35}
\begin{aligned}
&\|\phi^+\|_{2,\alpha;\mn^+}^{(-1-\alpha)}
\leq C\left(\sum_{i=1}^2\| h_i^+\|_{1,\alpha;\mn^+}^{(-\alpha)}
+\|g_{cd}\|_{2,\alpha;\Sigma}^{(-1-\alpha)}
+\|g_{2}^+\|_{2,\alpha;\Sigma_w^+}^{(-1-\alpha)}\right.\\
&\qquad\qquad\qquad\qquad\left.
+\|g_{1}^+\|_{2,\alpha;\Sigma_0^+}^{(-1-\alpha)}+\|\e\omega_{ex}^+\|_{1,\alpha;\Sigma_L^+}^{(-\alpha)}\right).
\end{aligned}
\end{equation}
Therefore the proof of Lemma 4.2 is completed.
\end{proof}
\par Now, we come back to the proof of Lemma 4.1. \\
\\
{\bf Proof of Lemma 4.1.}
For the problem \eqref{4-7}, it follows from \eqref{4-8}  that the  coefficients of  \eqref{4-7} satisfy
\begin{equation*}
\|a_{ij}-e_i^+\delta_{ij}\|_{1,\alpha;\mn^+}^{(-\alpha)}\leq \kappa^+,
\end{equation*}
  where $\kappa^+>0$   depends on $ \delta_1 $,  $ \bm U_b^+ $, $ L $ and $ \alpha $.
Thus we  apply the Banach contraction mapping theorem to prove the existence and uniqueness of the solution $  \phi^+\in  C_{2,\alpha}^{(-1-\alpha)}(\mn^+) $ of the problem \eqref{4-7} for sufficiently small $ \kappa^+ $ in the space
\begin{equation*}
\mk=\{v:\|v\|_{2,\alpha;\mn^+}^{(-1-\alpha)}< \infty\}.
\end{equation*}
Define a mapping $ \ma: \mk\rightarrow \mk $. For $ v\in \mk $, we consider the problem \eqref{4-12} with functions $ h_i^+ $ on the right-hand sides defined as
\begin{equation}\label{4-36}
h_i^+=f_i^++\sum_{j=1}^2(e_i^+\delta_{ij}-a_{ij})\p_{y_j}v.
\end{equation}
Then $ h_i^+\in C_{1,\alpha}^{(-\alpha)}(\mn^+ ) $ and
\begin{equation}\label{4-37}
\sum_{i=1}^2\|h_i\|_{1,\alpha;\mn^+}^{(-\alpha)}
\leq C\left(\sum_{i=1}^2\|f_i\|_{1,\alpha;\mn^+}^{(-\alpha)}
+\|v\|_{2,\alpha;\mn^+}^{(-1-\alpha)}\right).
\end{equation}
By Lemma 4.2, there exists a unique solution $\phi^+ \in C_{2,\alpha}^{(-1-\alpha)}(\mn^+) $ to the problem \eqref{4-12}. We define the mapping $ \ma: \mk\rightarrow \mk $ by setting $ \ma v= \phi^+ $.
\par Now we show that $ \ma $ is a contraction mapping in the norm $ \|\cdot\|_{\mk} $ if $ \kappa^+>0 $ is small. Let $ v_1,v_2\in \mk  $ and $ \phi_k^+=\ma v_k $ for $ k=1,2 $. Then $ \hat \phi^+= \phi_1^+- \phi_2^+ $ satisfies
\begin{equation}\label{4-38}
  \begin{cases}
e_1^+\p_{y_1}^2\hat\phi^+
+e_2^+\p_{y_2}^2\hat\phi^+
=\sum_{i=1}^2\p_{y_i}(\sum_{j=1}^2(e_i^+\delta_{ij}-a_{ij})\p_{y_j}(v_1-v_2)), &\quad {\rm{in}} \quad \mn^+,\\
\hat\phi^+(0,y_2)=0, &\quad {\rm{on}} \quad
  \Sigma_0^+,\\
 \hat\phi^+(y_1,0)=0, &\quad {\rm{on}} \quad
  \Sigma,\\
 \hat\phi^+(y_1,m^+)=0, &\quad {\rm{on}} \quad
  \Sigma_w^+,\\
  \p_{y_1}\hat\phi^+(L,y_2)=0, &\quad {\rm{on}} \quad
  \Sigma_L^+.\\
  \end{cases}
      \end{equation}
      It follows from \eqref{4-13} that one has
     \begin{equation}\label{4-39}
     \| \phi_1^+- \phi_2^+\|_{\mk}\leq \mc^+\kappa^+
      \|v_1^+-v_2^+\|_{\mk}.
      \end{equation}
      Therefore the mapping $ \ma: \mk\rightarrow \mk $ is a contraction mapping in the norm $ \|\cdot\|_{\mk} $ if $ \kappa^+<\frac{1}{\mc^+} $.
      \par For such $ \kappa^+ $, there exists a fixed point $  \phi^+ \in \mk$ satisfying $ \ma  \phi ^+= \phi ^+ $. Then $ \phi^+ $ satisfies \eqref{4-12} with right-hand sides given by \eqref{4-36} computed with $ v=\phi^+ $. Moreover, by \eqref{4-13} and \eqref{4-37},
     there holds
        \begin{equation*}
\begin{aligned}
&\|\phi^+\|_{2,\alpha;\mn^+}^{(-1-\alpha)}
\leq \mc^+\left(\sum_{i=1}^2\| f_i^+\|_{1,\alpha;\mn^+}^{(-\alpha)}
+\|g_{cd}\|_{2,\alpha;\Sigma_D}^{(-1-\alpha)}
+\|g_{2}^+\|_{2,\alpha;{\Sigma_w^+}}^{(-1-\alpha)}\right.\\
&\qquad\qquad\qquad\qquad\left.
+\|g_{1}^+\|_{2,\alpha;\Sigma_0^+}^{(-1-\alpha)}+\|\e\omega_{ex}^+\|_{1,\alpha;\Sigma_L^+}^{(-\alpha)}\right).\\
\end{aligned}
\end{equation*}
Then Lemma 4.1 is proved.
  \subsection{Solving the nonlinear  boundary value problem by the fixed point argument}\noindent
  \par For a given function $   \bar\phi^+\in \mj_{upp}(\delta_1) $,
  it follows from Lemma  4.1 that the problem \eqref{4-7} has a unique solution $\phi^{+}\in C_{2,\alpha}^{(-1-\alpha)}(\mn^+) $ satisfying the estimate in \eqref{4-11}.  This, together with \eqref{4-6}, yields
    \begin{equation}\label{4-40}
   \|\phi^{+}\|_{2,\alpha;\mn^+}^{(-1-\alpha)}
  \leq
  \mc^+
  \left(\delta_1 \sigma_v+\sigma_v+\|g_{cd}\|_{2,\alpha;\Sigma }^{(-1-\alpha)}\right)\leq
  \mc^+
  \left(\delta_1 \sigma_v+\sigma_v+\|\eta\|_{1,\alpha;\Sigma }^{(-\alpha)}\right),
  \end{equation}
  where   $ \mc^+>0 $
  depending only  on  $ {\bm U}_b^+ $, $ L $ and $\alpha $.
 \par  Similarly,   define
 \begin{equation}\label{4-41}
 \mj_{low}(\delta_2)
 =\{\phi^-:\phi^-(0,0)=0, \|\phi^-\|_{2,\alpha; \mn^-}^{(-1-\alpha)}
 \leq
 \delta_2\},
 \end{equation}
 where $ \delta_2 $ is a positive constant to be determined later.
 Then for given $ \bar\varphi^- $ such that $ \bar\phi^-=\bar\varphi^--\varphi^-_b\in \mj_{low}(\delta_2) $, we  solve the following problem:
 \begin{equation}\label{4-42}
 \begin{cases}
   \sum_{i,j=1,2}\p_{y_i}(a_{ij}(\n
   \bar\phi^-)\p_{y_j}\phi^-)=\sum_{i=1}^2\p_{y_i}f_i^- &\quad {\rm{in }} \quad      \mn^-,\\
     \bar\phi^-(0,y_2)=g_{1}^-(y_2), &\quad {\rm{on}} \quad
   \Sigma_0^-,\\
  \phi^-(y_1,0)=g_{cd}(y_1), &\quad {\rm{on}} \quad
   \Sigma,\\
   \phi^-(y_1,-m^-)=g_{2}^-(y_1), &\quad {\rm{on}} \quad
   \Sigma_w^-,\\
   \p_{y_1}\phi^-(L,y_2)=\e \omega_{ex}^-(y_2), &\quad {\rm{on}} \quad
   \Sigma_L^-,\\
   \end{cases}
 \end{equation}
  where
   \begin{equation*}
      \begin{aligned}
      &f_i^-=W_i(\n\varphi_b^-, A_{b}^-,  B_{b}^-)-W_i(\n\varphi_b^-, \e
      A_{en}^-, \e B_{en}^-) , \quad i=1,2,\\
       &g_{1}^-(y_2)=-\int_{y_2}^0\left(\frac{1}{\e J_{en}^-(s)}-\frac{1}{ J_b^-}\right)\de s,\\
    &g_{2}^-(y_1)=g^-(y_1)+1-\int_{-m^-}^0\left(\frac{1}{\e J_{en}^-(s)}-\frac{1}{ J_b^-}\right)\de s,\\
    &\e\omega_{ex}^-(y_2)=\omega_{ex}(\varphi_b^-(y_2)+\bar\phi^-(L,y_2)).
    \end{aligned}
    \end{equation*}
     Then similar arguments as in Lemma 4.1 yield that \eqref{4-42}
 has a unique solution $  \phi^-\in  C_{2,\alpha}^{(-1-\alpha)}(\mn^-) $ satisfying
 \begin{equation}\label{4-43}
 \begin{aligned}
  \|\phi^-\|_{2,\alpha;\mn^-}
  ^{(-1-\alpha)}
  &\leq \mc^-\left(\sum_{i=1}^2\|f_i^-\|_{1,\alpha;\mn ^-}
  ^{(-\alpha)}+\|g_{1}^-\|_{2,\alpha;\Sigma _{0}^-}^{(-1-\alpha)}
  +\|g_{cd}\|_{2,\alpha;\Sigma }^{(-1-\alpha)}\right. \\ &\quad\quad\quad\left.+\|g_{2}^-\|_{2,\alpha;
  \Sigma_w ^-}^{(-1-\alpha)}+\|\e\omega_{ex}^-\|_{1,\alpha;
  \Sigma _{L}^-}^{(-\alpha)}\right)\\
  &\leq
 \mc^-\left(\delta_2\sigma_v +\sigma_v+\|\eta\|_{1,\alpha;\Sigma }^{(-\alpha)}\right),
  \end{aligned}
  \end{equation}
  where   $ \mc^->0 $
  depending only  on  $ {\bm U}_b^- $, $ L $ and $\alpha $.
\par Define an iteration set
\begin{equation*}
\mj(\delta_1,\delta_2)=\mj_{upp}(\delta_1)\times \mj_{low}(\delta_2),
\end{equation*}
and a map $ \mt $ as follows
\begin{equation}\label{4-44}
\mt( \bar\phi^+, \bar\phi^-)=( \phi^+, \phi^-),
\quad{\rm{ for \ each}} \ ( \bar\phi^+, \bar\phi^-)\in \mj(\delta_1,\delta_2).
\end{equation}
Then it follows from \eqref{4-40} and \eqref{4-43} that
$ ( \phi^+, \phi^-) $ satisfies
\begin{equation}\label{4-45}
\begin{aligned}
\|\phi^+\|_{2,\alpha;\mn^+}^{(-1-\alpha)}+
\| \phi^-\|_{2,\alpha;\mn^-}^{(-1-\alpha)}
\leq \mc_1
\left((\delta_1+\delta_2)\sigma_v+\sigma_v+
\|\eta\|_{1,\alpha;
\Sigma }^{(-\alpha)}\right),
\end{aligned}
\end{equation}
where  $ \mc_1>0 $  depends  on $ (\bm U_b^+,\bm U_b^-) $, $ L $
and $ \alpha $.   We assume that
\begin{equation}\label{4-46}
\|\eta\|_{1,\alpha;\Sigma }^{(-\alpha)}\leq \delta_3,
\end{equation}
where $ \mc_1 \delta_3\leq \frac{\delta_1+\delta_2}{2}  $
with $ \mc_1 $ given in \eqref{4-45}.  Let $\sigma_1=\frac{1}{4(1+\mc_1)}$ and choose $\delta_1=\delta_2= 2\mc_1\sigma_v$. Then if $ \sigma_v\leq \sigma_1 $, one has
\begin{equation*}
\begin{aligned}
\| \phi^+\|_{2,\alpha;\mn^+}^{(-1-\alpha)}+
\| \phi^-\|_{2,\alpha;\mn^-}^{(-1-\alpha)} \leq  \frac{1}{2}(\delta_1+\delta_2)
+\mc_1\delta_3 \leq \delta_1+\delta_2. \end{aligned}
\end{equation*}
Hence $ \mt $ maps $ \mj(\delta_1,\delta_2) $ into itself.
  \par Next, we will show that $ \mt $ is a
  contraction mapping in $ \mj(\delta_1,\delta_2) $. Let $(\bar\phi_i^+, \bar\phi_i^-)\in \mj(\delta_1,\delta_2)$,  $ i=1,2 $, we  have $ ( \phi_i^+, \phi_i^-)=\mt(\bar\phi_i^+, \bar\phi_i^-) $.
Define
\begin{equation*}
\bar\Phi^\pm= \bar\phi_1^\pm- \bar\phi_2^\pm,\
\Phi^\pm= \phi_1^\pm- \phi_2^\pm.
\end{equation*}
Then $ \Phi^\pm $ satisfy the following boundary value problems:
\begin{equation}\label{4-47}
  \begin{cases}
    \sum_{i,j=1,2}\p_{y_i}(a_{ij}(\n
   \bar\phi_1^\pm)\p_{y_j}\Phi^\pm)\\
  \quad =-\sum_{i,j=1,2}\p_{y_i}(a_{ij}(\n
   \bar\phi_1^\pm)-a_{ij}(\n
   \bar\phi_2^\pm)\p_{y_j}\phi_2^\pm), &\quad {\rm{in }} \quad      \mn^\pm,\\
     \Phi^\pm(0,y_2)=0, &\quad {\rm{on}} \quad
   \Sigma_0^\pm,\\
  \Phi^\pm(y_1,0)=0, &\quad {\rm{on}} \quad
   \Sigma,\\
   \Phi^\pm(y_1,\pm m^\pm)=0, &\quad {\rm{on}} \quad
   \Sigma_w^\pm,\\
   \p_{y_1}\Phi^\pm(L,y_2)= \omega_{ex}(\varphi_b^\pm(y_2)+\bar\phi_1^\pm(L,y_2))
   -\omega_{ex}(\varphi_b^\pm(y_2)+\bar\phi_2^\pm(L,y_2)), &\quad {\rm{on}} \quad
   \Sigma_L^\pm.\\
    \end{cases}
  \end{equation}
Therefore,  it follows from Lemma 4.1 that one has
\begin{equation}\label{4-48}
\begin{aligned}
&\sum_{I=\pm}\| \Phi^I\|_{2,\alpha;\mn^I}^{(-1-\alpha)}\\
&\leq \mc_2\sum_{I=\pm}\left(\| a_{ij}(\n
   \bar\phi_1^I)-a_{ij}(\n
   \bar\phi_2^I)\p_{y_j}\phi_2^I\|_{1,\alpha;\mn^I}^{(-\alpha)}\right.\\
   &\qquad\qquad\left.+\|  \omega_{ex}(\varphi_b^I(y_2)+\bar\phi_1^I(L,y_2))
   -\omega_{ex}(\varphi_b^I(y_2)+\bar\phi_2^I(L,y_2))\|
   _{1,\alpha;\Sigma_L^I}^{(-\alpha)}\right)\\
&\leq\mc_2 (\delta_1+\delta_2+\sigma_v)\sum_{I=\pm}\| \bar\Phi^I\|_{2,\alpha;\mn^I}^{(-1-\alpha)}.
\end{aligned}
\end{equation}
Setting
\begin{equation}\label{4-49}
 \sigma_2=\min\left\{\sigma_1,\frac{1}{
4\mc_2(4\mc_1+1)} \right\}.
\end{equation}
Then for $ \sigma_v\leq \sigma_2 $,
$ \mc_2 (\delta_1+\delta_2+\sigma_v)=\mc_2 (4\mc_1+1)\sigma_v\leq \frac{1}{4}  $, hence the mapping $ \mt $ is a contraction mapping so that $\mt $ has a unique fixed point in $ \mj(\delta_1,\delta_2)$.
  \section{The construction of the contact discontinuity curve}\noindent
  \par Up to now, for a given function $ g_{cd}(y_1)=\int_{0}^{y_1}\eta(s)\de s  $,  we have obtained the solution $ (\varphi^+,\varphi^- ) $ for the nonlinear  boundary value problem \eqref{3-19}-\eqref{3-22} in $ \mn^\pm $. To complete the proof of Theorem 3.2, we will use the implicit function theorem to find the contact discontinuity  $ g_{cd}(y_1) $ such that \eqref{3-23} is satisfied.
  \par  We first define a Banach space
\begin{equation*}
 V=\{\eta: \|\eta \|_{1,\alpha;\Sigma }^{(-\alpha)}
  < \infty\}.
  \end{equation*}
 Set
  \begin{equation}\label{5-1}
 V_{\delta_3}=\{\eta\in V:  \|\eta \|_{1,\alpha;\Sigma }^{(-\alpha)}
 \leq \delta_3\},
  \end{equation}
  where  $ \delta_3$ is defined in \eqref{4-46}. Then for any $ \eta\in  V_{\delta_3}$,
   the nonlinear boundary value problem \eqref{3-19}-\eqref{3-22} has a unique solution $ (\varphi^+,  \varphi^-) $ satisfying
 \begin{equation}\label{5-2}
   \|\varphi^+-\varphi_b^+\|_{2,\alpha;\mn^+}^{(-1-\alpha)}+
   \|\varphi^--\varphi_b^-\|_{2,\alpha;\mn^-}
   ^{(-1-\alpha)}\leq 4\mc_1\sigma_v.
   \end{equation}
  \par Let
\begin{equation*}
   \begin{aligned}
  V_0&= C^{1,\alpha}([0,1])\times C^{1,\alpha}([0,1])\times C^{1,\alpha}([0,1])\times
    C^{2,\alpha}([0,L])
   \times C^{1,\alpha}([-1,0])\\
  &\quad\quad\times
   C^{1,\alpha}([-1,0])\times C^{1,\alpha}([-1,0])
  \times
     C^{2,\alpha}([0,L])\times
  C^{2,\alpha}([g^-(L),g^+(L)]).
   \end{aligned}
    \end{equation*}
    Then we set
    \begin{equation}\label{5-3}
    V_{\delta_4}=\{
    \bm\zeta_0\in V_0:\|\bm \zeta_0-\bm \zeta_b\|_{ V_0}\leq \delta_4\},
    \end{equation}
    where \begin{equation*}
     \bm \zeta_0=( A_{en}^+, B_{en}^+, J_{en}^+, g^+,
   A_{en}^-,B_{en}^-,J_{en}^-, g^-, \omega_{ex}), \ {\rm{and}} \
     \bm \zeta_b= (A_{b}^+, B_b^+,J_b^+, 1, A_b^-, B_b^-,J_b^-, -1,0).
   \end{equation*}
     \par Define a map $ \mq:   V_{\delta_4} \times  V_{\delta_3}\rightarrow  V_{\delta_3} $ by
  \begin{equation}\label{5-4}
  \mq(\bm \zeta_0,\eta):=\left(W_2(\n \varphi_b^++\n \phi^+ ,\e A_{en}^+,\e B_{en}^+ )-  W_2(\n \varphi_b^-+\n \phi^-,\e A_{en}^-,\e B_{en}^- )\right)(y_1,0).
  \end{equation}
   Hence \eqref{3-23} can be written as the equation
  \begin{equation}\label{5-5}
  \mq(\bm \zeta_0,\eta)=0.
  \end{equation}
     To employ the implicit function theorem to solve \eqref{5-5}, it suffices to verify the conditions $ \rm(i) $, $ \rm(ii) $ and $ \rm(iii) $ in Theorem 3.3.
  \par  Obviously,
  \begin{equation*}
  \mq(\bm \zeta_b,0)=0.
  \end{equation*}
  Next, we will divided into two steps to  verify $ \rm(ii) $ and $ \rm(iii) $.
\par {\bf Step 1. Differentiability of $ \mq $.}
  \par
   Given any $ \eta,\eta_1\in V(\delta_3) $, and  $ \tau>0 $,
   %we consider the limit
  %\begin{equation*}
 % \frac{\|\mq(\eta+ \tau\eta_1)-\mq(\eta)-\tau D_\eta\mq[\eta] (\eta_1)
 % \|_{1,\alpha;\Sigma}^{(-\alpha)}}
 % {\tau\| \eta_1\|_{1,\alpha;\Sigma}^{(-\alpha)}}
 % \end{equation*}
  %as $ \tau\rightarrow 0 $.
 let $  \phi^\pm_{\e \eta} $ and
  $  \phi^\pm_{\eta} $ be the solutions of the following equations
  \begin{equation}\label{5-6}
  \begin{cases}
  \begin{aligned}
  &\sum_{i,j=1,2}\p_{y_i}(a_{ij}(\phi^\pm_{\e \eta})\p_{y_j}\phi^\pm_{\e \eta})=\sum_{i=1}^2\p_{y_i}f_i^\pm, \\
&\phi^\pm_{\e \eta}(0,y_2)=g_{1}^\pm(y_2), \\
 &\phi^\pm_{\e \eta}(y_1,0)=\int_{0}^{y_1}(\eta+\tau \eta_1)(s)\de s, \\
  &\phi^\pm_{\e \eta}(y_1,\pm m^\pm)=g_{2}^\pm(y_1),\\
  &\p_{y_1}\phi^\pm_{\e \eta}(L,y_2)=\omega_{ex}(\varphi_b^\pm(y_2)+\phi^\pm_{\e \eta}(L,y_2)),\\
  \end{aligned}
  \end{cases}
\end{equation}
and
\begin{equation}\label{5-7}
  \begin{cases}
  \begin{aligned}
&\sum_{i,j=1,2}\p_{y_i}(a_{ij}(\phi^\pm_{ \eta})\p_{y_j}\phi^\pm_{ \eta})=\sum_{i=1}^2\p_{y_i}f_i^\pm, \\
&\phi^\pm_{\eta}(0,y_2)=g_{1}^\pm(y_2), \\
 &\phi^\pm_{ \eta}(y_1,0)=\int_{0}^{y_1}\eta(s)\de s, \\
 & \phi^\pm_{ \eta}(y_1,\pm m^\pm)=g_{2}^\pm(y_1),,\\
  &\p_{y_1}\phi^\pm_{ \eta}(L,y_2)=\omega_{ex}(\varphi_b^\pm(y_2)+\phi^\pm_{ \eta}(L,y_2)).\\
\end{aligned}
  \end{cases}
\end{equation}
Denote
  \begin{equation*}
  \phi^{\pm}_{\tau}=\frac{ \phi^\pm_{\e \eta}- \phi^\pm_{\eta}}{\tau}.
  \end{equation*}
    Then it follows from \eqref{5-6} and \eqref{5-7} that $  \phi^{\pm}_{\tau}$ satisfy the following equations:
  \begin{equation}\label{5-8}
  \begin{cases}
  \begin{aligned}
  &\sum_{i,j=1,2}\p_{y_i}(a_{ij}(\n
   \phi_{\eta}^\pm)\p_{y_j}\phi^{\pm}_{\tau})
  =-\sum_{i,j=1,2}\p_{y_i}\left(\frac{a_{ij}(\n
   \phi_{\e \eta}^\pm)-a_{ij}(\n
   \phi_\eta^\pm)}{\tau}\p_{y_j}\phi_{\e \eta}^{\pm}\right), \\
   &  \phi^{\pm}_{\tau}(0,y_2)=0,\\
  &\phi^{\pm}_{\tau}(y_1,0)= \int_{0}^{y_1}\eta_1(s)\de s, \\
  &\phi^{\pm}_{\tau}(y_1,\pm m^\pm)=0, \\
   &\p_{y_1}\phi^{\pm}_{\tau}(L,y_2)= \frac{\omega_{ex}(\varphi_b^\pm(y_2)+\phi^\pm_{\e \eta}(L,y_2))-\omega_{ex}(\varphi_b^\pm(y_2)+\phi^\pm_{ \eta}(L,y_2))}{\tau}.\\
    \end{aligned}
    \end{cases}
  \end{equation}
 Similar to the proof of Lemma 4.1,   the following estimate holds:
  \begin{equation}\label{5-9}
  \begin{aligned}
 & \sum_{I=\pm}\|\phi^{I}_{\tau}\|_{2,\alpha; \mn^I}
  ^{(-1-\alpha)}\\
  &\leq \mc_3\sum_{I=\pm}\left(\left(\| \phi^I_{\e \eta}\|_{2,\alpha;\mn^I}
  ^{(-1-\alpha)}
   +\| \omega_{ex}\|_{2,\alpha;[g^-(L),g^+(L)]}\right)
   \|\phi^{I}_{\tau}\|
  _{2,\alpha;\mn^I}
  ^{(-1-\alpha)}+\|  \eta_1\|_{1,\alpha;\Sigma}^{(-\alpha)}\right)\\
 &\leq \mc_3\left(\delta_1+\delta_2+\delta_3
  +\tau\|  \eta_1\|_{1,\alpha;\Sigma}^{(-\alpha)}+\sigma_v\right)
  \sum_{I=\pm}\|\phi^{I}_{\tau}\|
  _{2,\alpha;\mn^I}^{(-1-\alpha)}
  +\mc_3\| \eta_1\|_{1,\alpha;\Sigma}^{(-\alpha)}\\
  &\leq \mc_3\left((3+4 \mc_1)\sigma_v
  +\tau\| \eta_1\|_{1,\alpha;\Sigma}^{(-\alpha)}\right)
  \sum_{I=\pm}\|\phi^{I}_{\tau}\|_{2,\alpha;\mn^I}
  ^{(-1-\alpha)}
  +\mc_3\|  \eta_1\|_{1,\alpha;\Sigma}^{(-\alpha)},\\
  \end{aligned}
  \end{equation}
  where   $ \mc_3>0 $  depends  on $ ({\bm U}_b^+,{\bm U}_b^-) $, $ L$ and $ \alpha $.
   Choosing $ \tau_1>0 $ such that  $ \mc_3\tau_1\|  \eta_1\|_{1,\alpha;\Sigma}^{(-\alpha)}\leq \frac{1}{4} $
   and setting
  \begin{equation}\label{5-10}
  \sigma_3=\min\left\{\sigma_2,\frac{1}{4\mc_3(3+4\mc_1)}\right\},
   \end{equation}
where $ \sigma_2 $ is defined in \eqref{4-49}. Thus for $ \tau \in (0,\tau_1) $ and $ \sigma_v\leq \sigma_3 $, one has
  \begin{equation}\label{5-11}
  \sum_{I=\pm}\|\phi^{I}_{\tau}\|_{2,\alpha; \mn^I}
  ^{(-1-\alpha)}\leq
  2\mc_3\|  \eta_1\|_{1,\alpha;\Sigma}^{(-\alpha)}.
  \end{equation}
  Therefore, there exists a subsequence $\{\tau_k\}_{k=1}^\infty $ such that $ \phi^{\pm}_{\tau_k} $ converge to $ \phi^{\pm}_0$  in  $ C_{2,\alpha^\prime}^{(-1-\alpha^\prime)}(\mn^\pm) $
   as $ \tau_k\rightarrow 0 $ for some $ 0<\alpha^\prime<\alpha $.
The estimate \eqref{5-11} also implies that  $ \phi^{\pm}_0\in C_{2,\alpha}^{(-1-\alpha)}(\mn^\pm) $ and
  \begin{equation}\label{5-13}
  \sum_{I=\pm}\|\phi^{I}_{0}\|
  _{2,\alpha;\mn^I}^{(-1-\alpha)}
  \leq\mc_3\|  \eta_1\|_{1,\alpha;\Sigma}^{(-\alpha)}.
  \end{equation}
  Define a map $ D_\eta\mq(\bm\zeta_0,\eta) $ by
  \begin{equation}\label{5-14}
  \begin{aligned}
 D_\eta\mq(\bm\zeta_0,\eta)(\eta_1)
  &=\sum_{i=1}^2\left(\p_{\p_{y_i}\phi_{\eta}^{+}} W_2(\n \varphi_b^{+}+\n \phi^{+}_{\eta},\e A_{en}^+,\e B_{en}^+)\p_{y_i}\phi^{+}_{0}\right.\\
  &\qquad\quad\left.-\p_{\p_{y_i}\phi^-_{\eta}}W_2(\n \varphi_b^{-}+\n \phi^{-}_{\eta},\e A_{en}^-,\e B_{en}^-)\p_{y_i}\phi^{-}_{0}\right)(y_1,0),\\
  \end{aligned}
  \end{equation}
   which is  a linear map from $   V $ to $ V $.
   \par  Next, we need to show that $ D_\eta\mq(\bm\zeta_0,\eta) $  is  the Fr\'{e}chet derivative of the functional $ \mq(\bm\zeta_0,\eta)$ with respect to $ \eta $.
To achieve this, we first consider the estimate of
  $ \phi^{\pm}_{\tau}-\phi^{\pm}_{0} $.
   It follows from \eqref{5-8} that
  $\phi^{\pm}_{0} $ satisfy the following equations:
  \begin{equation}\label{5-12}
  \begin{cases}
  \begin{aligned}
   &\sum_{i,j=1,2}\p_{y_i}(a_{ij}(\n
   \phi_{\eta}^\pm)\p_{y_j}\phi^{\pm}_{0})
  =-\sum_{i,j=1,2}\p_{y_i}\left(\sum_{k=1}^2\p_{\p_{y_k}
   \phi_\eta^\pm}a_{ij}(\n
   \phi_\eta^\pm)\p_{y_k}
   \phi_0^\pm \p_{y_j}\varphi_{ \eta}^{\pm}\right), \\
   &  \phi^{\pm}_{0}(0,y_2)=0,\\
  &\phi^{\pm}_{0}(y_1,0)=\int_{0}^{y_1}\eta_1(s)\de s, \\
  &\phi^{\pm}_{0}(y_1,\pm m^\pm)=0, \\
   &\p_{y_1}\phi^{\pm}_{0}(L,y_2)= \omega_{ex}^\prime(\varphi_b^\pm(y_2)+\phi^\pm_{ \eta}(L,y_2))\phi^{\pm}_{0}(L,y_2).\\
    \end{aligned}
    \end{cases}
  \end{equation}
    Thus it follows from \eqref{5-12} and \eqref{5-8} that $ \phi^{\pm}_{\tau}-\phi^{\pm}_{0}$
satisfy
   \begin{equation}\label{5-15}
  \begin{cases}
  \begin{aligned}
  &\sum_{i,j=1,2}\p_{y_i}(a_{ij}(\n
   \phi_{\eta}^\pm)\p_{y_j}(\phi^{\pm}_{\tau}-\phi^{\pm}_{0})\\
  &\quad=-\sum_{i,j=1,2}\p_{y_i}\left(\frac{a_{ij}(\n
   \phi_{\e \eta}^\pm)-a_{ij}(\n
   \phi_\eta^\pm)}{\tau}\p_{y_j}\phi_{\e \eta}^{\pm}-\sum_{k=1}^2\p_{\p_{y_k}
   \phi_\eta^\pm}a_{ij}(\n
   \phi_\eta^\pm)\p_{y_k}
   \phi_0^\pm\p_{y_j}\phi_{ \eta}^{\pm}\right), \\
   &  (\phi^{\pm}_{\tau}-\phi^{\pm}_{0})(0,y_2)=0,\\
  &(\phi^{\pm}_{\tau}-\phi^{\pm}_{0})(y_1,0)=0, \\
  &(\phi^{\pm}_{\tau}-\phi^{\pm}_{0})(y_1,\pm m^\pm)=0, \\
   &\p_{y_1}(\phi^{\pm}_{\tau}-\phi^{\pm}_{0})(L,y_2)=
   \frac{\omega_{ex}(\varphi_b^\pm(y_2)+\phi^\pm_{\e \eta}(L,y_2))-\omega_{ex}(\varphi_b^\pm(y_2)+\phi^\pm_{ \eta}(L,y_2))}{\tau}\\
    &\qquad\qquad\qquad\qquad\qquad- \omega_{ex}^\prime(\varphi_b^\pm(y_2)+\phi^\pm_{ \eta}(L,y_2))\phi^{\pm}_{0}(L,y_2).\\
    \end{aligned}
    \end{cases}
  \end{equation}
  By the direct computation, one gets
\begin{equation*}
\begin{aligned}
&\sum_{I=\pm}\left\|\frac{a_{ij}(\n
   \phi_{\e \eta}^I)-a_{ij}(\n
   \phi_\eta^I)}{\tau}\p_{y_j}\phi_{\e \eta}^{I}-\sum_{k=1}^2\p_{\p_{y_k}
   \phi_\eta^I}a_{ij}(\n
   \phi_\eta^I)\p_{y_k}
   \phi_0^I\p_{y_j}\phi_{ \eta}^{I}\right\|_{1,\alpha;\mn^I}^{(-\alpha)}\\
&\leq\sum_{I=\pm}\left\|\left(\frac{a_{ij}(\n
   \phi_{\e \eta}^I)-a_{ij}(\n
   \phi_\eta^I)}{\tau}-\sum_{k=1}^2\p_{\p_{y_k}
   \phi_\eta^I}a_{ij}(\n
   \phi_\eta^I)\p_{y_k}
   \phi_0^I\right)\p_{y_j}\phi_{\e \eta}^{I}\right\|_{1,\alpha;\mn^I}^{(-\alpha)}\\
   &\quad+ \sum_{I=\pm}\left\|\sum_{k=1}^2\p_{\p_{y_k}
   \phi_\eta^I}a_{ij}(\n
   \phi_\eta^I)(\p_{y_k}
   \phi_{\e \eta}^I-\p_{y_k}
   \phi_{\eta}^I)\p_{y_j}
   \phi_0^I\right\|_{1,\alpha;\mn^I}^{(-\alpha)}\\
&\leq
\mc\|\phi^{I}_{\e \eta}\|_{2,\alpha;\mn^I}^{(-1-\alpha)}
\|\phi^{I}_{\tau}-\phi^{I}_{0}\|_{2;\alpha,\mn^I}
^{(-1-\alpha)}+\mc\sum_{I=\pm}\|\phi^{I}_{0}\|
_{2,\alpha;\mn^I}^{(-1-\alpha)}
\|\phi^{I}_{\tau}\|_{2,\alpha;\mn^I}^{(-1-\alpha)}\tau,\\
&\sum_{I=\pm}\left\| \frac{\omega_{ex}(\varphi_b^I(y_2)+\phi^I_{\e \eta}(L,y_2))-\omega_{ex}(\varphi_b^I(y_2)+\phi^I_{ \eta}(L,y_2))}{\tau}\right.\\
&\qquad\quad\left.- \omega_{ex}^\prime(\varphi_b^I(y_2)+\phi^I_{ \eta}(L,y_2))\phi^{I}_{0}(L,y_2)\right\|
_{1,\alpha;\Sigma_L^I}^{(-\alpha)}\\
&\leq \sum_{I=\pm}\left\|\int_{0}^1\omega_{ex}^\prime(\varphi_b^I(y_2)+\phi^I_{ \eta}(L,y_2)+t\tau\phi^{I}_{\tau}(L,y_2))\de t(\phi^{I}_{\tau}-\phi^{I}_{0})(L,y_2)\right\|
_{1,\alpha;\Sigma_L^I}^{(-\alpha)}\\
&\quad+ \sum_{I=\pm}\left\|\int_{0}^1\left(\omega_{ex}^\prime(\varphi_b^I(y_2)+\phi^I_{ \eta}(L,y_2)+t\tau\phi^{I}_{\tau}(L,y_2))\right.\right.\\
&\qquad\qquad\quad\left.\left.-\omega_{ex}^\prime(\varphi_b^I(y_2)+\phi^I_{ \eta}(L,y_2))\right)\de t\phi^{I}_{0}(L,y_2) \right\|
_{1,\alpha;\Sigma_L^I}^{(-\alpha)}\\
&\leq \mc\| \omega_{ex}\|_{2,\alpha;[g^-(L),g^+(L)]}\sum_{I=\pm}
\left(\|\phi^{I}_{\tau}-\phi^{I}_{0}\|_{2,\alpha;\mn^I}^{(-1-\alpha)}
+\|\phi^{I}_{0}\|
_{2,\alpha;\mn^I}^{(-1-\alpha)}
\|\phi^{I}_{\tau}\|_{2,\alpha;\mn^I}^{(-1-\alpha)}\tau\right).
\end{aligned}
  \end{equation*}
 Hence the following estimate can be derived:
  \begin{equation}\label{5-16}
  \begin{aligned}
  &\sum_{I=\pm}\|\phi^{I}_{\tau}-\phi^{I}_{0}\|
  _{2,\alpha;\mn^I}^{(-1-\alpha)}\\
  &\leq \mc\sum_{I=\pm}\|\phi^{I}_{0}\|
_{2,\alpha;\mn^I}^{(-1-\alpha)}
\|\phi^{I}_{\tau}\|_{2,\alpha;\mn^I}^{(-1-\alpha)}\tau
+\mc\|\phi^{I}_{\e \eta}\|_{2,\alpha;\mn^I}^{(-1-\alpha)}
\|\phi^{I}_{\tau}-\phi^{I}_{0}\|_{2,\alpha;\mn^I}
^{(-1-\alpha)}\\
&\quad+\mc\| \omega_{ex}\|_{2,\alpha;[g^-(L),g^+(L)]}\sum_{I=\pm}
\left(\|\phi^{I}_{\tau}-\phi^{I}_{0}\|_{2,\alpha;\mn^I}^{(-1-\alpha)}
+\|\phi^{I}_{0}\|
_{2,\alpha;\mn^I}^{(-1-\alpha)}
\|\phi^{I}_{\tau}\|_{2,\alpha;\mn^I}^{(-1-\alpha)}\tau\right)\\
 &\leq  \mc_4\left(\delta_1+\delta_2+\delta_3
  +\tau\| \eta_1\|_{1,\alpha;\Sigma}^{(-\alpha)}+\sigma_v\right)
  \sum_{I=\pm}\|\phi^{I}_{\tau}-\phi^{I}_{0}\|
  _{2,\alpha;\mn^I}^{(-1-\alpha)}
  +\mc_4
  \left(\| \eta_1\|_{1,\alpha ;\Sigma}^{(-\alpha )}
  \right)^2\tau,\\
  &\leq \mc_4\left((3+4 \mc_1)\sigma_v
  +\tau\|  \eta_1\|_{1,\alpha ;\Sigma}^{(-\alpha )}\right)
  \sum_{I=\pm}\|\phi^{I}_{\tau}-\phi^{I}_{0}\|
  _{2,\alpha;\mn^I}^{(-1-\alpha)}
  +\mc_4
  \left(\| \eta_1\|_{1,\alpha ;\Sigma}^{(-\alpha )}
  \right)^2\tau,\\
  \end{aligned}
  \end{equation}
  where   $ \mc_4>0 $  depends  on $ ({\bm U}_b^+,{\bm U}_b^-) $, $ L$ and $ \alpha $. Choosing  $ \tau_2>0 $  such that  $ \mc_4\tau_2\|  \eta_1\|_{1,\alpha;\Sigma}^{(-\alpha)}\leq \frac{1}{8} $ and setting
  \begin{equation}\label{5-17}
   \sigma_4=\min\left\{\sigma_3,\frac{1}{8\mc_4(3+4\mc_1)}\right\}.
   \end{equation}
  Then for $ \tau \in (0,\tau_2) $ and $ \sigma_v\leq \sigma_4 $, one has
\begin{equation}\label{5-18}
  \sum_{I=\pm}\|\phi^{I}_{\tau}-\phi^{I}_{0}\|
  _{2,\alpha;\mn^I}^{(-1-\alpha)}\leq 4\mc_4
  \left(\|  \eta_1\|_{1,\alpha ;\Sigma}^{(-\alpha )}
  \right)^2\tau.
  \end{equation}
 \par By the definition of $ \mq(\bm\zeta_0,\eta) $ and $ D_\eta\mq(\bm\zeta_0,\eta) $, we have
  \begin{equation}\label{5-19}
  \begin{aligned}
  &\left\| \mq(\bm\zeta_0,\eta+\tau \eta_1)-\mq(\bm\zeta_0,\eta)- D_\eta\mq(\bm\zeta_0,\eta)( \tau\eta_1)\|_{1,\alpha;\Sigma}^{(-\alpha)}\right.\\
  &=\|W_2(\n \varphi_b^I+\n \phi^I_{\e \eta},\e A_{en}^I,\e B_{en}^I)
  - W_2(\n\varphi_b^I+\n \phi^I_{\eta},\e A_{en}^I,\e B_{en}^I)\\
  &\quad\left.-\sum_{i=1}^2\left(\p_{\p_{y_i}\phi_{\eta}^{+}} W_2(\n \varphi_b^{+}+\n \phi^{+}_{\eta},\e A_{en}^+,\e B_{en}^+)\p_{y_i}\phi^{+}_{0}\right.\right.\\
  &\qquad\quad\quad\left.\left.-\p_{\p_{y_i}\phi^-_{\eta}}W_2(\n \varphi_b^{-}+\n \phi^{-}_{\eta},\e A_{en}^-,\e B_{en}^-)\p_{y_i}\phi^{-}_{0}\right)
  \right\|_{1,\alpha;\Sigma}^{(-\alpha)}\\
  &\leq \sum_{I=\pm}\left\|  W_2(\n \varphi_b^I+\n \phi^I_{\e \eta},\e A_{en}^I,\e B_{en}^I)
  - W_2(\n\varphi_b^I+\n \phi^I_{\eta},\e A_{en}^I,\e B_{en}^I)\right.\\
  &\quad\quad\quad\left.-\tau\sum_{i=1}^2\p_{\p_{y_i} \phi^I_{\eta}} W_2(\n\varphi_b^I+\n \phi^I_{\eta},\e A_{en}^I,\e B_{en}^I)\p_{y_i}\phi^{I}_{\tau}\right\|
  _{1,\alpha;\Sigma}^{(-\alpha)}\\
  &\quad+\tau \sum_{I=\pm}\left\|\sum_{i=1}^2\p_{\p_{y_i} \phi^I_{\eta}} W_2(\n\varphi_b^I+\n \phi^I_{\eta},\e A_{en}^I,\e B_{en}^I)\p_{y_i}(
  \phi^{I}_{\tau}-\phi^{I}_{0})
  \right\|
  _{1,\alpha;\Sigma}^{(-\alpha)}\\
   &\leq \mc\tau^2\sum_{I=\pm}\left(\|\phi^{I}_{\tau}\|
_{2,\alpha;\mn^I}^{(-1-\alpha)}
  \right)^2+ \mc\tau\sum_{I=\pm}\|\phi^{I}_{\tau}
  -\phi^{I}_{0}\|
 _{2,\alpha;\mn^I}^{(-1-\alpha)}\\
&\leq \mc\tau^2
  \left(\| \eta_1\|_{1,\alpha;\Sigma}^{(-\alpha)}
  \right)^2,\\
  \end{aligned}
  \end{equation}
   where   $ \mc>0 $  depends  on $ ({\bm U}_b^+,{\bm U}_b^-) $, $ L $ and $ \alpha $. This implies that
  \begin{equation*}
  \frac{\| \mq(\bm\zeta_0,\eta+\tau \eta_1)-\mq(\bm\zeta_0,\eta)- D_\eta\mq(\bm\zeta_0,\eta)( \tau\eta_1)\|_{1,\alpha;\Sigma}^{(-\alpha)}}
  {\tau\|\eta_1\|_{1,\alpha;\Sigma}^{(-\alpha)}}\rightarrow 0
  \end{equation*}
  as $ \tau\rightarrow 0 $. Thus $ D_\eta\mq(\bm\zeta_0,\eta) $ is   the Fr\'{e}chet derivative of the functional $ \mq(\bm\zeta_0,\eta)$ with respect to $ \eta $.
\par It remains to prove  the continuity of the map $ \mq(\bm\zeta_0,\eta) $ and $ D_\eta\mq(\bm\zeta_0,\eta) $. For any fixed $(\bm \zeta_0, \eta)\in V_0\times V $, we assume that $ (\bm \zeta_0^k,\eta^k)\rightarrow (\bm \zeta_0, \eta) $ in $ V_0\times V $ as $ k\rightarrow \infty $. Then we first show that as $ k\rightarrow \infty $,
  \begin{equation}\label{5-20}
  \mq(\bm\zeta_0^k,\eta^k)\rightarrow  \mq(\bm\zeta_0,\eta), \quad {\rm{in}}\quad  V.
  \end{equation}
  By \eqref{5-7}, the solutions $ \phi^\pm_{ \eta^k}$ corresponding to $ (\bm\zeta_0^k,\eta^k) $ satisfy
  \begin{equation}\label{5-21}
  \begin{cases}
  \begin{aligned}
&\sum_{i,j=1,2}\p_{y_i}(a_{ij}(\n
   \phi_{\eta^k}^\pm,\e A_{en}^{\pm,k},\e B_{en}^{\pm,k})\p_{y_j}\phi^\pm_{ \eta^k})=\sum_{i=1}^2\p_{y_i}f_i^{\pm,k}, \\
&\phi^\pm_{\eta^k}(0,y_2)=g_{1}^{\pm,k}(y_2), \\
 &\phi^\pm_{ \eta^k}(y_1,0)=\int_{0}^{y_1}\eta^k(s)\de s, \\
 & \phi^\pm_{ \eta^k}(y_1,\pm m^{\pm})=g_{2}^{\pm,k}(y_1),\\
  &\p_{y_1}\phi^\pm_{ \eta^k}(L,y_2)=\omega_{ex}^k(\varphi_b^\pm(y_2)+\phi^\pm_{ \eta^k}(L,y_2)),\\
\end{aligned}
  \end{cases}
\end{equation}
where
   \begin{equation*}
      \begin{aligned}
      &f_i^{\pm,k}=W_i(\n\varphi_b^\pm, A_{b}^\pm,  B_{b}^\pm)-W_i(\n\varphi_b^\pm, \e
      A_{en}^{\pm,k}, \e B_{en}^{\pm,k}) , \quad i=1,2,\\
       &g_{1}^{\pm,k}(y_2)=\int_{0}^{y_2}\left(\frac{1}{\e J_{en}^{\pm,k}(s)}-\frac{1}{ J_b^\pm}\right)\de s,\\
       %&m^{+,k}=\int_{0}^1J_{en}^{+,k}(s)\de s,\quad m^{-,k}=\int_{-1}^0J_{en}^{-,k}(s)\de s,\\
    &g_{2}^{\pm,k}(y_1)=g^\pm(y_1)\pm 1+\int_{0}^{\pm m^{\pm}}\left(\frac{1}{\e J_{en}^{\pm,k}(s)}-\frac{1}{ J_b^\pm}\right)\de s.\\
    \end{aligned}
    \end{equation*}
Taking the difference of \eqref{5-21} and \eqref{5-7}, one can derive that
 \begin{equation}\label{5-22}
  \begin{cases}
  \begin{aligned}
  &\sum_{i,j=1,2}\p_{y_i}(a_{ij}(\n
   \phi_{\eta}^\pm,\e A_{en}^{\pm},\e B_{en}^{\pm})\p_{y_j}(\phi^\pm_{ \eta^k}-
   \phi_{\eta}^\pm)\\
  &=-\sum_{i,j=1,2}\p_{y_i}\left({a_{ij}(\n
   \phi_{\eta^k}^\pm,\e A_{en}^{\pm,k},\e B_{en}^{\pm,k})-a_{ij}(\n
   \phi_\eta^\pm,\e A_{en}^{\pm},\e B_{en}^{\pm})}\p_{y_j}\phi_{ \eta^k}^{\pm}\right)
   +\sum_{i=1}^2\p_{y_i}(f_i^{\pm,k}-f_i^{\pm}),\\
   &  (\phi^\pm_{ \eta^k}-
   \phi_{\eta}^\pm)(0,y_2)=g_{1}^{\pm,k}(y_2)-g_{1}^{\pm}(y_2),\\
  &(\phi^\pm_{ \eta^k}-
   \phi_{\eta}^\pm)(y_1,0)= \int_{0}^{y_1}(\eta^k(s)-\eta(s))\de s, \\
  &(\phi^\pm_{ \eta^k}-
   \phi_{\eta}^\pm)(y_1,\pm m^\pm)=g_{2}^{\pm,k}(y_1)-g_{2}^{\pm}(y_1), \\
   &\p_{y_1}(\phi^\pm_{ \eta^k}-
   \phi_{\eta}^\pm)(L,y_2)= \omega_{ex}^k(\varphi_b^\pm(y_2)+\phi^\pm_{ \eta^k}(L,y_2))-\omega_{ex}(\varphi_b^\pm(y_2)+\phi^\pm_{ \eta}(L,y_2)).\\
    \end{aligned}
    \end{cases}
  \end{equation}
 Then the estimate \eqref{4-11} implies  that
\begin{equation*}
\sum_{I=\pm}\|\phi_{\eta^k}^I- \phi_{\eta}^I\|_{2,\alpha;\mn^I}^{(-1-\alpha)}\leq \mc \left(\|\bm\zeta_0^k-\bm\zeta_0\|_{V_0}+\|\eta^k- {\eta}\|_{1,\alpha;\Sigma}^{(-\alpha)}\right).
  \end{equation*}
  Therefore,
  \begin{equation}\label{5-23}
  \begin{aligned}
   &\|\mq(\bm\zeta_0^k,\eta^k)-\mq(\bm\zeta_0,\eta)\|
   _{1,\alpha;\Sigma}^{(-\alpha)}\\
  &=\left.\|\left(W_2(\n \varphi_b^++\n \phi_{\eta^k}^+ ,\e A_{en}^{+,k},\e B_{en}^{+,k} )-  W_2(\n \varphi_b^-+\n \phi_{\eta^k}^-,\e A_{en}^{-,k},\e B_{en}^{-,k} )\right)\right.\\
  &\quad\quad-\left.\left(W_2(\n \varphi_b^++\n \phi_{\eta}^+ ,\e A_{en}^+,\e B_{en}^+ )-  W_2(\n \varphi_b^-+\n \phi_{\eta}^-,\e A_{en}^-,\e B_{en}^- )\right)\right\| _{1,\alpha;\Sigma}^{(-\alpha)}\\
  &\leq \mc\sum_{I=\pm}\left(\|\phi_{\eta^k}^I- \phi_{\eta}^I\|_{2,\alpha;\mn^I}^{(-1-\alpha)}
  +\|\bm\zeta_0^k-\bm\zeta_0\|_{V_0}\right)\leq  \mc \left(\|\bm\zeta_0^k-\bm\zeta_0\|_{V_0}+\|\eta^k- {\eta}\|_{1,\alpha;\Sigma}^{(-\alpha)}\right),
  \end{aligned}
  \end{equation}
  which yields \eqref{5-20}.
  \par Next, we  prove the continuity of the map $ D_\eta\mq(\bm\zeta_0,\eta) $, i.e. to show that as $ k\rightarrow \infty $,
  \begin{equation}\label{5-24}
  D_\eta\mq(\bm\zeta_0^k,\eta^k)\rightarrow   D_\eta\mq(\bm\zeta_0,\eta), \quad {\rm{in}}\quad  V.
  \end{equation}
   It follows from \eqref{5-12} that the solutions $ \phi^{\pm,k}_{0}$ corresponding to $ (\bm\zeta_0^k,\eta^k) $ satisfy
  \begin{equation}\label{5-25}
  \begin{cases}
  \begin{aligned}
   &\sum_{i,j=1,2}\p_{y_i}(a_{ij}(\n
   \phi_{\eta^k}^\pm,\e A_{en}^{\pm,k},\e B_{en}^{\pm,k})\p_{y_j}\phi^{\pm,k}_{0})\\
  &=-\sum_{i,j=1,2}\p_{y_i}\left(\sum_{l=1}^2\p_{\p_{y_l}
   \phi_{\eta^k}^\pm}a_{ij}(\n
   \phi_{\eta^k}^\pm,\e A_{en}^{\pm,k},\e B_{en}^{\pm,k})\p_{y_l}
   \phi_0^{\pm,k} \p_{y_j}\phi_{ \eta^k}^{\pm}\right), \\
   &  \phi^{\pm,k}_{0}(0,y_2)=0,\\
  &\phi^{\pm,k}_{0}(y_1,0)=\int_{0}^{y_1}\eta_1(s)\de s, \\
  &\phi^\pm_{ \eta^k}(y_1,\pm m^{\pm})=0, \\
   &\p_{y_1}\phi^{\pm,k}_{0}(L,y_2)= (\omega_{ex}^k)^\prime(\varphi_b^\pm(y_2)+\phi_{ \eta^k}^{\pm}(L,y_2))\phi^{\pm,k}_{0}(L,y_2).\\
    \end{aligned}
    \end{cases}
  \end{equation}
Taking the difference of \eqref{5-25} and \eqref{5-12}, one has
  \begin{equation*}
  \begin{aligned}
  \sum_{I=\pm}\|\phi^{I,k}_{0}- \phi^{I}_{0}\|_{2,\alpha;\mn^I}^{(-1-\alpha)}
  \leq \mc\sum_{I=\pm}\left(\|\phi_{\eta^k}^I- \phi_{\eta}^I\|_{2,\alpha;\mn^I}^{(-1-\alpha)}
  +\|\bm\zeta_0^k-\bm\zeta_0\|_{V_0}\right).
  \end{aligned}
  \end{equation*}
  Combining \eqref{5-23} yields that
  \begin{equation}\label{5-26}
  \begin{aligned}
  \sum_{I=\pm}\|\phi^{I,k}_{0}- \phi^{I}_{0}\|_{2,\alpha;\mn^I}^{(-1-\alpha)}
  \leq  \mc \left(\|\bm\zeta_0^k-\bm\zeta_0\|_{V_0}+\|\eta^k- {\eta}\|_{1,\alpha;\Sigma}^{(-\alpha)}\right).
  \end{aligned}
  \end{equation}
  Thus
  \begin{equation*}
  \begin{aligned}
  &\| D_\eta\mq(\bm\zeta_0^k,\eta^k)(\eta_1)-  D_\eta\mq(\bm\zeta_0,\eta)(\eta_1)\|_{1,\alpha;\Sigma}
  ^{(-\alpha)}\\
   &=\sum_{i=1}^2\left\|\left(\p_{\p_{y_i}\phi_{\eta^k}^{+}} W_2(\n \varphi_b^{+}+\n \phi^{+}_{\eta^k},\e A_{en}^{+,k},\e B_{en}^{+,k})\p_{y_i}\phi^{+,k}_{0}\right.\right.\\
  &\qquad\qquad\left.\left.-\p_{\p_{y_i}\phi^-_{\eta^k}}W_2(\n \varphi_b^{-}+\n \phi^{-}_{\eta^k},\e A_{en}^{-,k},\e B_{en}^{-,k})\p_{y_i}\phi^{-,k}_{0}\right)\right.\\
  \end{aligned}
  \end{equation*}
  \begin{equation}\label{5-27}
  \begin{aligned}
  &\qquad\quad-\left.\left(\p_{\p_{y_i}\phi_{\eta}^{+}} W_2(\n \varphi_b^{+}+\n \phi^{+}_{\eta},\e A_{en}^+,\e B_{en}^+)\p_{y_i}\phi^{+}_{0}\right.\right.\\
  &\qquad\qquad\left.\left.-\p_{\p_{y_i}\phi^-_{\eta}}W_2(\n \varphi_b^{-}+\n \phi^{-}_{\eta},\e A_{en}^-,\e B_{en}^-)\p_{y_i}\phi^{-}_{0}\right)\right\|
  _{1,\alpha;\Sigma}^{(-\alpha)}\\
  &\leq \mc\sum_{I=\pm}\left(\|\phi^{I,k}_{0}- \phi^{I}_{0}\|_{2,\alpha;\mn^I}^{(-1-\alpha)}+ \|\phi^{I}_{\eta^k}- \phi^{I}_{\eta}\|
  _{2,\alpha;\mn^I}^{(-1-\alpha)}\right)+\mc\|\bm\zeta_0^k-\bm\zeta_0\|_{V_0}\\
  &\leq  \mc \left(\|\bm\zeta_0^k-\bm\zeta_0\|_{V_0}+\|\eta^k- {\eta}\|_{1,\alpha;\Sigma}^{(-\alpha)}\right),
  \end{aligned}
  \end{equation}
    which implies that \eqref{5-24} holds.  %The proof of the continuity of the map $ \mq(\bm\zeta_0,\eta) $ and $ D_\eta\mq(\bm\zeta_0,\eta) $ is completed.
\par {\bf Step 2. The isomorphism of $ D_\eta\mq(\bm \zeta_b,0)$. }
\par To prove the isomorphism of $ D_\eta\mq(\bm\zeta_b,0)$, we need to show that for any given function $  P_\ast\in V $, there exists a unique $  \eta_\ast\in V  $ such that $ D_\eta\mq(\bm \zeta_b,0)(\eta_\ast)=P_\ast $, i.e.,
\begin{equation}\label{5-28}
\begin{aligned}
&P_\ast(y_1)= \frac{  (c_b^+)^2(u_b^+)^3(\rho_b^+)^2}
{(c_b^+)^2-(u_b^+)^2}\p_{y_2}\varphi^{+}_\ast(y_1,0)-
\frac{  (c_b^-)^2(u_b^-)^3(\rho_b^-)^2}
{(c_b^-)^2-(u_b^-)^2}\p_{y_2}\varphi^{-}_\ast(y_1,0).
\end{aligned}
\end{equation}
 At the background state, the solutions $  \varphi^{\pm}_\ast $ satisfy
\begin{equation}\label{5-29}
\begin{cases}
e_1^\pm\p_{y_1}^2
\varphi^{\pm}_\ast+e_2^\pm\p_{y_2}^2
 \varphi^{\pm}_\ast=0,\\
\varphi^{\pm}_\ast(0,y_2)=0,\\
\varphi^{\pm}_\ast(y_1,0)=\int_0^{y_1}\eta_\ast(s)\de s,\\
\varphi^{\pm}_\ast(y_1,\pm m^\pm)=0,\\
   \p_{y_1}\varphi^{\pm}_\ast(L,y_2)=0,\\
  \end{cases}
\end{equation}
where
 \begin{equation*}
 e_{1}^\pm=u_b^\pm,\quad
 e_{2}^\pm=\frac{  (c_b^\pm)^2(u_b^\pm)^3(\rho_b^\pm)^2}
{(c_b^\pm)^2-(u_b^\pm)^2}.
\end{equation*}
Let
\begin{equation*}
\begin{cases}
\hat y_1=y_1, \quad \hat y_2=\frac{y_2}{m^+}, \quad (y_1,y_2)\in \mn^+,\\
\hat y_1=y_1, \quad \hat y_2=\frac{y_2}{m^-}, \quad (y_1,y_2)\in \mn^-.\\
\end{cases}
\end{equation*}
Under the coordinates transformation, the domain  $ \mn^+\cup \mn^- $ is transformed into the following
 domain:
 \begin{equation*}
 \hat \mn^+\cup \hat \mn^-=\{(\hat y_1,\hat y_2): 0<\hat y_1 <L, 0<\hat y_2<1\}\cup \{(\hat y_1,\hat y_2): 0<\hat y_1 <L, -1<\hat y_2<0\}.
 \end{equation*}
The boundaries $ \Sigma_0^\pm,\Sigma_w^\pm, \Sigma_L^\pm $ become
\begin{equation*}
\begin{aligned}
 &\hat\Sigma_0^+=\{(\hat y_1,\hat y_2): \hat y_1=0, 0<\hat y_2<1\},\quad
 \hat\Sigma_0^-=\{(\hat y_1,\hat y_2): \hat y_1=0, -1<\hat y_2<0\},\\
 &\hat\Sigma_w^+=\{(\hat y_1,\hat y_2): 0<\hat y_1<L, \hat y_2=1\},\quad
 \hat\Sigma_w^-=\{(\hat y_1,\hat y_2): 0<\hat y_1<L, \hat y_2=-1\},\\
 &\hat\Sigma_L^+=\{(\hat y_1,\hat y_2): \hat y_1=L, 0<\hat y_2<1\}, \quad
 \hat\Sigma_L^-=\{(\hat y_1,\hat y_2): \hat y_1=L, -1<\hat y_2<0\}.
 \end{aligned}
 \end{equation*}
  We stretch $  \varphi^{\pm}_\ast $ in the $ \hat y_2 $ direction and also flip $  \varphi^{-}_\ast $ into $ \hat \Omega^+ $ in the following way:
\begin{equation*}
 \hat\varphi^{+}_\ast(\hat y_1,\hat y_2)=\varphi^{+}_\ast\left(\hat y_1,
\sqrt{\frac{e_2^+}{e_1^+}}m^+\hat y_2\right),\quad  \hat\varphi^{-}_\ast(\hat y_1,\hat y_2)=\varphi^{-}_\ast\left(\hat y_1,
-\sqrt{\frac{e_2^-}{e_1^-}}m^-\hat y_2\right).
\end{equation*}
Then \eqref{5-29} can be written as
\begin{equation}\label{5-30}
\begin{cases}
\p_{\hat y_1 }^2
 \hat\varphi^{\pm}_\ast+\p_{\hat y_2 }^2
 \hat\varphi^{\pm}_\ast=0,
&\quad {\rm in}\quad \hat \mn^+,\\
\hat\varphi^{\pm}_\ast(0, \hat y_2)=0, &\quad {\rm on}\quad      \hat \Sigma_0^+,\\
\hat\varphi^{\pm}_\ast( \hat y_1,1)=0, &\quad {\rm on}\quad      \hat \Sigma_w^+,\\
\hat\varphi^{\pm}_\ast( \hat y_1,0)=\int_0^{\hat y_1}\eta_\ast(s)\de s, &\quad {\rm on}\quad      \Sigma,\\
   \p_{\hat y_1}\hat\varphi^{\pm}_\ast(L,  \hat y_2)=0, &\quad {\rm on}\quad
    \hat\Sigma_L^+.\\
  \end{cases}
\end{equation}
Therefore $ \hat\varphi^{\pm}_\ast $ satisfy the Laplace equation with the same boundary conditions. By the uniqueness, we conclude that
$ \hat\varphi^{+}_\ast= \hat\varphi^{-}_\ast  $ in $ \hat \mn^+ $.
\par  Moreover, \eqref{5-28} in the new  coordinates can be rewritten as
\begin{equation}\label{5-31}
\begin{aligned}
& P_\ast( \hat y_1)= (\sqrt{e_1^+e_2^+}+ \sqrt{e_1^-e_2^-})\p_{\hat y_2}\hat\varphi^{+}_\ast( \hat y_1,0).
\end{aligned}
\end{equation}
Thus we consider the following equation:
\begin{equation}\label{5-32}
\begin{cases}
\p_{\hat y_1 }^2
 \hat\varphi^{+}_\ast+\p_{\hat y_2}^2
\hat\varphi^{+}_\ast=0,
&\quad {\rm in}\quad \hat \mn^+,\\
\hat\varphi^{+}_\ast(0, \hat y_2)=0, &\quad {\rm on}\quad      \hat \Sigma_0^+,\\
\p_{y_2}\hat\varphi^{+}_\ast( \hat y_1,0)=\frac{1}{\sqrt{e_1^+e_2^+}+ \sqrt{e_1^-e_2^-}} P_\ast( \hat y_1), &\quad {\rm on}\quad      \Sigma,\\
\hat\varphi^{+}_\ast(\hat y_1,1)=0, &\quad {\rm on}\quad      \hat \Sigma_w^+,\\
   \p_{\hat y_1}\hat\varphi^{+}_\ast(L, \hat y_2)=0, &\quad {\rm on}\quad
    \hat\Sigma_L^+.\\
  \end{cases}
\end{equation}
Then similar arguments as in Lemma 4.2 yield that \eqref{5-32} has a unique solution  $ \hat\varphi^{+}_\ast\in C_{2,\alpha ,\hat\mn^+}^{(-1-\alpha )} $ satisfying
\begin{equation}\label{5-33}
\|\hat\varphi^{+}_\ast\| _{2,\alpha ,\hat\mn^+}^{(-1-\alpha )}\leq \mc \| P_\ast\|_{1,\alpha ;\Sigma}^{(-\alpha )},
\end{equation}
where   $ \mc>0 $  depends  on $ ({\bm U}_b^+,{\bm U}_b^-) $, $ L$ and $ \alpha $.
\par Set $ \eta_\ast( \hat y_1)=\p_{\hat y_1}\hat\varphi^{+}_\ast( \hat y_1,0) $,
then \eqref{5-33} shows that $ \eta_\ast\in V  $. Hence we have shown there exists a unique $ \eta_\ast\in V  $ such that $ D_\eta\mq(\bm\zeta_b,0)(\eta_\ast)= P_\ast $. The proof of the isomorphism of $ D_\eta\mq(\bm\zeta_b,0)$ is completed.
\section{Proof of Theorem 2.2}\noindent
\par Now, by the implicit function theorem, there exist  positive constants $ \sigma_5  $ and $ \mc $ depending only on $ (\bm {U}_b^+,\bm {U}_b^-) $, $L $ and $\alpha$ such that for $ \delta_4\leq \sigma_5 $, the equation $ \mq(\bm \zeta_0,\eta)=0 $ has a unique solution $ \eta=\eta(\zeta_0) $ satisfying
\begin{equation}\label{6-1}
\|\eta\|_{1,\alpha;\Sigma}^{(-\alpha )}\leq \mc\|\bm \zeta_0-\bm \zeta_b\|_{V_0}= \mc\sigma_v.
\end{equation}
Here $ \delta_4 $ is defined in \eqref{5-3}.  Hence  the contact discontinuity curve $ g_{cd}(y_1)=\int_{0}^{y_1}\eta(s) \de s $ is determined  and  \eqref{6-1} implies that
\begin{equation}\label{6-2}
\|g_{cd}\|_{2,\alpha ;\Sigma}^{(-1-\alpha )}\leq \mc_5\|\bm \zeta_0-\bm \zeta_b\|_{V_0}= \mc_5\sigma_v,
\end{equation}
where $ \mc_5>0 $ depends only on $ (\bm {U}_b^+,\bm {U}_b^-) $, $L $ and $\alpha$.
\par We choose $  \sigma_{cd}^\ast $ and  $\mc^\ast $ as
   \begin{equation}\label{6-3}
   \sigma_{cd}^\ast=\min\{\sigma_4,\sigma_5\} \quad {\rm{ and}} \quad \mc^\ast=4\mc_1+\mc_5,
   \end{equation}
   where $ \sigma_4 $  is defined in \eqref{5-17} and $ \mc_1 $ is  defined in \eqref{4-45}.
  Then if $ \sigma_v\leq\sigma_{cd}^\ast $,  Problem \rm{3.1} has  a unique piecewise smooth subsonic solution $ (\varphi^+,\varphi^-;g_{cd})$ satisfying
 \begin{equation*}
 \|\varphi^+ -\varphi_b^+\|_{2,\alpha;\mn^+}^{(-1-\alpha)}+\|\varphi^- -\varphi_b^-\|_{2,\alpha;\mn^-}^{(-1-\alpha)}+\| g_{cd} \|_{2,\alpha;\Sigma}^{(-1-\alpha)}\leq \mc^\ast\sigma_v.
 \end{equation*}
 Therefore
 \begin{equation}\label{6-4}
 \|\bm U^+ -\bm U_b^+\|_{1,\alpha;\mn^+}^{(-\alpha)}+\|\bm U^- -\bm U_b^-\|_{1,\alpha;\mn^-}^{(-\alpha)}+\| g_{cd} \|_{2,\alpha;\Sigma}^{(-1-\alpha)}\leq \mc^\ast\sigma_v.
 \end{equation}
 \par Since the coordinates transformation \eqref{3-4} is invertible, thus the solution transformed back in x-coordinates solves the Problem \rm{2.1} and the estimate \eqref{6-4}
 implies that the estimates \eqref{2-13} and \eqref{2-14} hold in   Theorem 2.2. The proof of
  Theorem 2.2 is completed.
  \par {\bf Acknowledgement.} Weng is partially supported by National Natural Science Foundation of China  11971307, 12071359, 12221001.
  \begin{thebibliography}{1}
\bibitem{BM09}
Bae, M.: Stability of contact discontinuity for steady Euler system in the infinite duct. Z. Angew Math.
Phys. 64, 917-936 (2013).
\bibitem{BP19}
Bae, M., Park, H.: Contact discontinuity for 2-D inviscid compressible flows in infinitely long nozzles.
SIAM J. Math. Anal. 51, 1730-1760 (2019).
\bibitem{PB19}
 Bae, M., Park, H.: Contact discontinuity for 3-D axisymmetric inviscid compressible flows in infinitely
long cylinders. J. Differential Equations 267, 2824-2873 (2019).
%\bibitem{BDX14}
% Bae,M.,  Duan, B.,   Xie, C.: Subsonic solutions for steady Euler-Poisson system in two-dimensional nozzles, SIAM J. Math. Anal., 46 (2014), pp. 3455-3480.
% \bibitem{BW18}
% Bae,M.,  Weng, S.: 3-D axisymmetric subsonic flows with nonzero swirl for the compressible
%Euler-Poisson system, Ann. Inst. H. Poincar\'{e} Anal. Non Lin\'{e}aire, 35 (2018), pp. 161-186.
\bibitem{CXZ22}
   Chen, J; Xin, Z; Zang, A.: Subsonic flows past a profile with a vortex line at the trailing edge. SIAM J. Math. Anal. 54 (2022), no. 1, 912-939.
    \bibitem{CYK13}
   Chen, G.-Q., Kukreja, V., Yuan, H.:
   Well-posedness of transonic characteristic discontinuities in two-dimensional steady compressible Euler flows. Z. Angew. Math. Phys. 64 (2013), no. 6, 1711-1727.
   \bibitem{CKY13}
   Chen, G.-Q., Kukreja, V., Yuan, H.: Stability of transonic characteristic discontinuities in two-dimensional steady compressible Euler flows. J. Math. Phys. 54 (2013), no. 2, 021506, 24 pp.
   \bibitem{CHWX19}
 Chen, G.-Q., Huang, F.,Wang, T., Xiang, W.: Steady Euler flows with large vorticity and characteristic discontinuities in arbitrary infinitely long nozzles. Adv. Math. 346 (2019), 946-1008.
\bibitem{CCF07}
 Chen, G.-Q., Chen, J., Feldman, M.: Transonic shocks and free boundary problems for the full Euler
equations in infinite nozzles. J. Math. Pures Appl. 88, 191-218 (2007).
\bibitem{CF07}
 Chen, S.-X., Fang, B.: Stability of reflection and refraction of shocks on interface. J. Differential
Equations 244, 1946-1984 (2008).
\bibitem{CHF13}
 Chen, S.-X., Hu, D., Fang, B.: Stability of the E-H type regular shock refraction. J. Differential
Equations 254, 3146-3199 (2013).
 \bibitem{CF48}
    Courant, R., Friedrichs, K.O.: Supersonic Flow and Shock Waves, Interscience Publishers Inc.: New York, 1948.


%\bibitem{CDXX16}
%    Chen, C., Du, L., Xie, C., Xin, Z.: Two dimensional subsonic Euler flows past a wall or a symmetric body. Arch. Ration. Mech. Anal. 221 (2016), no. 2, 559-602.
% \bibitem{DWX14}
% Du, L., Weng, S., Xin, Z.: Subsonic irrotational flows in a finitely long nozzle with variable end
%pressure. Commun. Partial Differential Equations 39, 666-695 (2014)

 \bibitem{GT98}
 Gilbarg, D., Trudinger, N.S.: Elliptic Partial Differential Equations of Second Order, 2nd ed.
Grundlehren Math. Wiss. 224, Springer, Berlin, 1998.
 \bibitem{HFWX19}
 Huang, F., Kuang, J., Wang, D., Xiang,W.: Stability of supersonic contact discontinuity for 2-D steady
compressible Euler flows in a finitely long nozzle. J. Differential Equations 266, 4337-4376 (2019).
 \bibitem{HFWX21}
Huang, F., Kuang, J.,Wang, D., Xiang,W.: Stability of transonic contact discontinuity for two-dimensional steady compressible Euler flows in a finitely long nozzle. Ann. PDE 7 (2021), no. 2, Paper No. 23, 96 pp.

 \bibitem{LG86}
Lieberman, G.M.: Mixed boundary value problems for elliptic and parabolic differential equations of
second-order. J. Math. Anal. Appl. 113, 422-440 (1986).
 \bibitem{PW22}
Pei, Y., Xiang, W.: Global subsonic jet with strong transonic shock over a convex cornered wedge for the two-dimensional steady full Euler equations. SIAM J. Math. Anal. 54 (2022), no. 4, 4407-4451.


\bibitem{LXY13}
Li, J., Xin, Z., Yin, H.: Transonic Shocks for the Full Compressible Euler System in a General Two-Dimensional
De Laval Nozzle, Arch. Rational Mech. Anal. 207, 533-581 (2013).
\bibitem{WY15}
Wang, Y., Yuan, H.: Weak stability of transonic contact discontinuities in three dimensional steady
non-isentropic compressible Euler flows. Z. Angew. Math. Phys. 66, 341-388 (2015).
\bibitem{WY13}
Wang, Y., Yu, F.: Stability of contact discontinuities in three dimensional compressible steady flows.
J. Differential Equations 255, 1278-1356 (2013).
\bibitem{WF15}
Wang, Y., Yu, F.:
Structural stability of supersonic contact discontinuities in three-dimensional compressible steady flows. SIAM J. Math. Anal. 47 (2015), no. 2, 1291-1329.

%  \bibitem{WS14}
%   Weng, S.: Subsonic irrotational flows in a two-dimensional finitely long curved nozzle. Z. Angew. Math. Phys. 65 (2014), no. 2, 203-220.
%\bibitem{WZ22}
%Weng, S., Zhang, Z.:  Two dimensional subsonic and subsonic-sonic spiral flows outside a porous body. Acta Math. Sci. Ser. B (Engl. Ed.) 42 (2022), no. 4, 1569-1584

%\bibitem{XX07}
%Xie, C., Xin, Z.: Global subsonic and subsonic-sonic flows through infinitely long nozzles. Indiana
%Univ. Math. J. 56, 2991-3023 (2007)
%
%\bibitem{XX10}
% Xie, C., Xin, Z.: Existence of global steady subsonic Euler flows through infinitely long nozzles. SIAM
%J. Math. Anal. 42, 751-784 (2010)
\bibitem{ZE92}
Zeidler, E.: Functional Analysis and its applications I; Fixed Point Theorems,
Springer-Verlag, New York, 1992.
\end{thebibliography}
\end{document}
