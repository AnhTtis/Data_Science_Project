\documentclass{aa}
\usepackage[utf8]{inputenc}
\usepackage{comment}
\usepackage{xcolor,colortbl}
\definecolor{Grey}{rgb}{0.5,0.5,0.5} 
\definecolor{MyRed}{rgb}{0.9,0.0,0.0} 
\definecolor{MyPink}{rgb}{0.8,0.3,0.5} 
\definecolor{MyMediumBlue}{rgb}{0.7,0.72,1.0} 
\definecolor{MyGreen}{rgb}{0.0,0.9,0.5} 
\definecolor{SWGreen}{rgb}{0.0,0.9,0.5} 
\definecolor{SWRed}{rgb}{0.9,0.0,0.0} 

\newcommand{\todiscuss}[1]{{\textcolor{MyGreen}{#1}}}
\newcommand{\todo}[1]{{\textcolor{MyPink}{#1}}}
\newcommand{\comment}[1]{{\textcolor{MyRed}{\footnotesize{\textit{[#1]}}}}}
%
\usepackage{graphicx,natbib}
\usepackage[breaklinks,colorlinks,urlcolor=blue,citecolor=blue,linkcolor=blue]{hyperref}
\usepackage{multirow,afterpage,pdflscape,lipsum,capt-of}
\usepackage[dvipsnames]{xcolor}
\newcommand{\swedt}[1]{{\textcolor{SWGreen}{#1}}}
\newcommand{\swcmt}[1]{{\textcolor{SWRed}{\footnotesize{\textit{[#1]}}}}}
\newcommand{\swrm}[1]{{\textcolor{Grey}{\sout{#1}}}}

\begin{document}

   \title{Analysis of full disk solar maps}

   \subtitle{}


\author{Sneha Pandit \inst{1,2}
        \and 
       Sven Wedemeyer \inst{1,2}
}
\institute{Rosseland Centre for Solar  Physics, University of Oslo, Postboks 1029 Blindern, N-0315 Oslo, Norway
            \and
            Institute of  Theoretical Astrophysics, University of Oslo, Postboks 1029 Blindern, N-0315 Oslo, Norway \\
            \email{sneha.pandit@astro.uio.no}
             }

   \date{Received -; accepted -}
  \abstract
  % context heading (optional)
  {...solar observations ... Atacama Large Millimeter/sub-millimeter Array (ALMA) ... chromosphere ... Sun as
an essential reference for the study of other stars} %leave it empty if necessary 
  {%Why are we doing this? Background\\
   }
  % aims heading (mandatory)
  %{What do we want to do?}
    {}
  % methods heading (mandatory)
  {%How are we gonna tackle this problem?
    ALMA Band 3 data ( 3mm / 100 GHz) ... Band 6 ...}
  % results heading (mandatory)
  {%What do we end up with?
  }
   %{Vibrational instability is found to be a common phenomenon
  % at temperatures lower than the second He ionisation
   %zone. The $\kappa$-mechanism is widespread under `cool'
   %conditions.}
  % conclusions heading (optional), leave it empty if necessary 
   {}
   \keywords{sun: chromosphere - submillimetre: sun - radio continuum
               }
\maketitle

\section{Introduction}


\section{Plots from the first set of ALMA data}
the one standard deviation from the 

\begin{figure*}[!h]
    \centering
        \includegraphics[width=1\textwidth]{gaussian640_ALMA20_convolved.png}
        \caption{Gaussian signal of radius 320 pixels, same as the TP map, convolved with the TP map, here the panel shows five figures (same as before): 1. ALMA with a mask of radius greater than 320 pixels, 2. The gaussian used, 3.Convolution of 1 and 2, 4. 3 with a mask of radius a. 300 pixels; b. 320 pixels.}
        \label{Figure 16}
\end{figure*}

\section{Discussion}


\section*{Acknowledgments}
This work is supported by the Research Council of Norway through its Centres of Excellence scheme, project number 262622 (``Rosseland Centre for Solar Physics'').  
AM and SP acknowledge support from the EMISSA project funded by the Research Council of Norway (project number 286853).
SW and MS were supported  by the SolarALMA project, which has received funding from the European Research Council (ERC) under the European Union’s Horizon 2020 research and innovation programme (grant agreement No. 682462).  
%
This paper makes use of the following ALMA data: \todo{ADS/JAO.ALMA\#2xxx.} 
ALMA is a partnership of ESO (representing its member states), NSF (USA) and NINS (Japan), together with NRC(Canada), MOST and ASIAA (Taiwan), and KASI (Republic of Korea), in co-operation with the Republic of Chile. The Joint ALMA Observatory is operated by ESO, AUI/NRAO and NAOJ. 

\bibliographystyle{aa}
\bibliography{First_Paper} 

\section{Appendix}

1.1992A+A___254__258J.pdf.
2.Carlsson, M., Hansteen, V. H., Gudiksen, B. V., Leenaarts, J. & De Pontieu, B. A publicly available simulation of an enhanced network region of the Sun. A&A 585, A4 (2016).
3.Vitas, N., Viticchiè, B., Rutten, R. J. & Vögler, A. Explanation of the activity sensitivity of Mn I 5394.7 Å. A&A 499, 301–312 (2009).
4.Bergemann, M. et al. Observational constraints on the origin of the elements: I. 3D NLTE formation of Mn lines in late-type stars. A&A 631, A80 (2019).
5.Strassmeier, K. G., Hooten, J. T., Hall, D. S. & Fekel, F. C. The photometric variability of the chromospherically active binary star HD 80715. PASP 101, 107 (1989).
6.Pace, G., Pasquini, L. & Ortolani, S. The Wilson–Bappu effect: A tool to determine stellar distances. A&A 401, 997–1007 (2003).


\end{document}
