\section{Additional Results}

\begin{figure*}[t]
%\vskip 0.2in
\begin{center}
%\centerline{\includegraphics[width=\linewidth,height=12cm]{example-image-a}}
\centerline{\includegraphics[width=0.5\linewidth]{figures/Frame_1.pdf}}
\caption{\textbf{Failure case of EG3D \cite{chan2022efficient}} -- on 3D-FRONT bedroom. We notice that the tri-plane representation induces replicating the scenes that are symmetric about one of the feature planes.}
\label{fig:eg3d_failure}
\end{center}
\vskip -0.2in
\end{figure*}

\begin{figure*}[t]
%\vskip 0.2in
\begin{center}
\centerline{\includegraphics[width=1.0\linewidth]{figures/layout_scene_style_half.pdf}}
\caption{\textbf{Additional results on 3D-FRONT.} We visualize the conditional generation results on 3D-FRONT bedrooms with varying latent codes (shown in three styles). Note that changing the global latent codes results in a change of general styles.}
\label{fig:additional_beds}
\end{center}
\vskip -0.2in
\end{figure*}

In Fig.~\ref{fig:additional_beds} and Fig.~\ref{fig:additional_living}, we show additional visualizations of our conditional generation results. Note that the generated 3D scenes generally follow the input layouts. Moreover, we sample three different global latent vectors which, when applied to the generation process, synthesize scenes with different styles. In Fig.~\ref{fig:living_removal}, we demonstrate the object removal capability. Note how we can remove individual objects such as a chair and a coffee table. In Fig.~\ref{fig:kitti_layout} we visualize KITTI-360 layouts and renderings. While our model generates better image quality and view consistency than previous works, we acknowledge that the current model has difficulties closely following complex layouts.



\begin{figure*}[t]
%\vskip 0.2in
\begin{center}
\centerline{\includegraphics[width=1.0\linewidth]{figures/layout_living-1.pdf}}
\caption{\textbf{Additional results on 3D-FRONT.} We visualize the conditional generation results on 3D-FRONT living rooms with varying latent codes (shown in three styles). Note that changing the global latent codes results in a change of general styles. We notice that for living room scenes the layout conditionings are not perfectly respected. For example, the big sofa bounding box of the second example is splitted into a sofa and a side table.}
\label{fig:additional_living}
\end{center}
\vskip -0.2in
\end{figure*}

\begin{figure*}[t]
%\vskip 0.2in
\begin{center}
\centerline{\includegraphics[width=0.8\linewidth]{figures/living_edit_supp.pdf}}
\caption{\textbf{Object removal experiment.} We showcase the object removal capability of our approach. Note that from the image on the leftmost column, we can remove the green sofa chair (middle column) and the black coffee table (right column).}
\label{fig:living_removal}
\end{center}
\vskip -0.2in
\end{figure*}

\begin{figure*}[t]
%\vskip 0.2in
\begin{center}
\centerline{\includegraphics[width=0.8\linewidth]{figures/living_trans_supp.pdf}}
\caption{\textbf{Translation experiment.} We showcase the object translation capability of our approach. Note that from the image on the leftmost column, we can translate the object to the left (middle column) and right (right column).}
\label{fig:living_trans}
\end{center}
\vskip -0.2in
\end{figure*}

\begin{figure*}[t]
\begin{center}
\centerline{\includegraphics[width=1.0\linewidth]{figures/kitti_layout.pdf}}
   \caption{\textbf{KITTI-360 conditioning.} Layout inputs and generated 3D scenes for KITTI-360.}
\label{fig:kitti_layout}
\end{center}
\vskip -0.2in
\end{figure*}
