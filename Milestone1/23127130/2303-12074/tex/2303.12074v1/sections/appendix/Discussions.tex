\begin{table}[!t]
    \caption{\textbf{Quantitative ablation studies} on \threedfront living rooms. We measure the realism of generated 3D scenes without using 2D layout conditioning (i.e., unconditional version of our model) or using the layout consistency loss described in Sec.~\ref{sec: discriminator}. Moreover, we swap out our 3D extrusion representation with the ``floorplan" and tri-plane schemes, proving the advantage of our method.}
    \label{table:ablations_living}
    \vskip 0.15in
    \begin{center}
    \begin{tabularx}{1.0\linewidth}{@{}X@{}ccc}
        \toprule
        Method & FID ($\downarrow$) & KID ($\downarrow$) \\
        \midrule
        Ours & \textbf{40.3} & \textbf{34.5}\\
        %\midrule
        \hspace{1mm} w/o Layout Conditioning & {60.1} & {54.1} \\
        \hspace{1mm} w/o Layout Consistency Loss & {44.7} & {38.0} \\
        \hspace{1mm} w/ GSN's Floorplan Representation & {65.6} & {59.0} \\
        \hspace{1mm} w/ EG3D's Tri-plane Representation & {69.3} & {60.8} \\
        \bottomrule
    \end{tabularx}
    \end{center}
    \vskip -0.1in
\end{table}

\section{Discussions}
\subsection{Note on Tri-Plane Results}
As discussed in the main text, we hypothesize that the tri-plane representation is conceptually not ideal for representing large-scale scenes due to weak geometric inductive bias when generating the tri-plane jointly. Moreover, as the scene gets larger, the same plane-projected features are used to describe totally different objects in a scene, which hampers the representational power of tri-planes. Indeed, in the included video websites, one can observe that the EG3D results contain artifacts where the scene contains two bedrooms, symmetric about one of the three planes (see Fig.~\ref{fig:eg3d_failure}). 
\quad
We also note that the ablated version of our conditional model using the Tri-plane representation suffers from severe layout inconsistencies.
That is, we observe that the input layout is almost completely ignored and the output scenes have almost no resemblance to the input layouts, which clearly indicates that the tri-plane representation lacks geometric inductive bias in our use-case.

\subsection{Note on ``Floorplan" Results}
We discussed in the main text how the ``floorplan'' representation requires a larger MLP because the vertical information needs to be decoded, or ``generated'' by the MLP network. Indeed, we observe that GSN \cite{devries2021unconstrained} adopts an 11-layer MLP with 128 channels in the hidden layers. In comparison, EG3D and our representation require using a two-layer MLP with 64 channels in the hidden layer. Approximately, our MLP network size (in number of weights) is less than 20 times smaller than that of GSN's. 

In our ablation study in the main text (Tab. 2), we swap our ``extrusion'' representation with a ``floorplan'' representation. Here, to make the comparison fair, we used the same two-layer MLPs for the experiment. Note that, while increasing the size of the MLP might improve its performance, it comes at a significant cost of computational resources. 

