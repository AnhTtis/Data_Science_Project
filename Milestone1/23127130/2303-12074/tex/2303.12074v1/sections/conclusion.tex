\section{Discussion}
\label{sec:conclusions}


\paragraph{Limitations}
% 
Even with the semantic consistency loss of Sec.~\ref{sec: discriminator}, there are still missing objects in the generated scenes, especially for large living room scenes with many objects. We believe that tightly enforcing the  generator to closely follow the conditioning is a challenging but important problem that needs to be explored by our community.

Similar to previous 3D-aware GANs, our approach suffers from view-inconsistencies caused by 2D upsampling, which is mostly visible in rendered camera trajectories.
One solution could be to leverage patch-based training to discard the 2D super-resolution module as in~\cite{skorokhodov2022epigraf, son2022singraf}.

The ability to change disentangled latent codes for each object could enable more controlled scene editing, similar to GIRAFFE \cite{niemeyer2021giraffe}.
Furthermore, we observe that the global style code and the input layouts are not completely disentangled, i.e., layout changes often lead to appearance changes. 
Moreover, we rely on a manually-defined camera distribution for each dataset. 
Finally, extending our method to dynamic scenes \cite{bahmani20223d, xu2022pv3d} could enable spatio-temporal control of complex scene generation. We leave addressing the above concerns as future work.

\AT{you could say something about the use of mortom codes to further increase the correlation between CNN feature proximity and geometric proximity? this could be a nice follow up?}

\paragraph{Conclusions}
In this work, we present a conditional 3D GAN, dubbed {\em CC3D}, that can compositionally synthesize complex 3D scenes, supervised only from unstructured image collections and scene layouts. We show that our 2D-conditioned 3D generation technique, along with our novel 3D field representation, enables high-quality generation of multi-object scenes. With CC3D, we can set the layouts of realistic 3D scenes that can be rendered from arbitrary camera trajectories, opening up a research direction towards controllable and scalable 3D generative technologies.






