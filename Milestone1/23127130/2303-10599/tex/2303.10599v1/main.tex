
\documentclass{article}

\usepackage[bbgreekl]{mathbbol}
\usepackage{mathrsfs}
\usepackage{graphicx}
\usepackage{amsmath}
\usepackage{amsthm}
\usepackage{amsfonts}
\usepackage{indentfirst}
\usepackage{amssymb}
\usepackage{algorithm, algorithmic}
\usepackage{xcolor}
\usepackage{geometry}
\usepackage{appendix}
\usepackage{bm}
\usepackage{url}
\usepackage{setspace}
\usepackage{subfigure}
\usepackage{verbatim}
\usepackage[normalem]{ulem}
\usepackage{multirow}
\usepackage{lineno}
\usepackage{float}
\usepackage{array}
\usepackage{diagbox}
\usepackage{stmaryrd}
\usepackage{enumerate}
\usepackage{cleveref}
\usepackage[affil-it]{authblk}
\numberwithin{equation}{section}

\newtheorem{assumption}{Assumption}
\newtheorem{definition}{Definition}
\newtheorem{lemma}{Lemma}
\newtheorem{theorem}{Theorem}

\newcommand\parameter{\bm{\theta}}
\newcommand\bmx{\bm{x}}

%
\newcommand{\dm}{displaymath}
\newcommand{\p}{\partial}
\newcommand{\lp}{\left(}
\newcommand{\rp}{\right)}
\newcommand{\lb}{\left[}
\newcommand{\rb}{\right]}
\newcommand{\la}{\left\langle}
\newcommand{\ra}{\right\rangle}

% combination
\newcommand{\lrp}[1]{\left(#1\right)}
\newcommand{\lrb}[1]{\left[#1\right]}
\newcommand{\lrbb}[1]{\left\{#1\right\}}
\newcommand{\lra}[1]{\left\langle #1\right\rangle}
\newcommand{\norm}[1]{\left\| #1\right\|}
\newcommand{\lbb}[1]{\left\{\begin{aligned} #1\end{aligned}\right.}
\newcommand{\ldr}[1]{\left. #1\right|}
\newcommand{\bmbm}[1]{\begin{bmatrix} #1 \end{bmatrix}}
\newcommand{\pmpm}[1]{\begin{pmatrix} #1 \end{pmatrix}}
\newcommand{\bra}[1]{\left\langle #1\right|}
\newcommand{\ket}[1]{\left| #1\right\rangle}
\newcommand{\flr}[1]{\lfloor #1\rfloor}

% 上标
\newcommand{\T}{\mathrm{T}}


% text names in equation
\newcommand{\ac}{\operatorname{ac}}
\newcommand{\co}{\operatorname{co}}
\newcommand{\diag}{\operatorname{diag}}
\newcommand{\tdist}{\operatorname{dist}}
\newcommand{\dist}{\operatorname{dis}}
\newcommand{\dom}{\operatorname{dom}}
\newcommand{\epi}{\operatorname{epi}}
\newcommand{\gl}{\operatorname{gl}}
\newcommand{\grad}{\operatorname{grad}}
\newcommand{\im}{\operatorname{im}}
\newcommand{\prox}{\operatorname{prox}}
\newcommand{\tsign}{\operatorname{sign}}
\newcommand{\tr}{\operatorname{tr}}
\newcommand{\vvec}{\operatorname{vec}}
\newcommand{\Aut}{\operatorname{Aut}}
\newcommand{\Dens}{\operatorname{Dens}}
\newcommand{\Diff}{\operatorname{Diff}}
\newcommand{\Exp}{\operatorname{Exp}}
\newcommand{\GL}{\operatorname{GL}}
\newcommand{\Hess}{\operatorname{Hess}}
\newcommand{\Hom}{\operatorname{Hom}}
\newcommand{\Hor}{\operatorname{Hor}}
\newcommand{\Id}{\operatorname{Id}}
\newcommand{\KL}{\operatorname{KL}}
\newcommand{\MMD}{\operatorname{MMD}}
\newcommand{\OB}{\mathcal{OB}}
\newcommand{\Prox}{\operatorname{Prox}}
\newcommand{\Proj}{\operatorname{Proj}}
\newcommand{\Var}{\operatorname{Var}}
\newcommand{\Ver}{\operatorname{Ver}}
\newcommand{\der}{\operatorname{d}}
% mathbf
\newcommand{\bfa}{\mathbf{a}}
\newcommand{\bfb}{\mathbf{b}}
\newcommand{\bfc}{\mathbf{c}}
\newcommand{\bfd}{\mathbf{d}}
\newcommand{\bfe}{\mathbf{e}}
\newcommand{\bff}{\mathbf{f}}
\newcommand{\bfg}{\mathbf{g}}
\newcommand{\bfh}{\mathbf{h}}
\newcommand{\bfi}{\mathbf{i}}
\newcommand{\bfj}{\mathbf{j}}
\newcommand{\bfk}{\mathbf{k}}
\newcommand{\bfl}{\mathbf{l}}
\newcommand{\bfm}{\mathbf{m}}
\newcommand{\bfn}{\mathbf{n}}
\newcommand{\bfo}{\mathbf{o}}
\newcommand{\bfp}{\mathbf{p}}
\newcommand{\bfq}{\mathbf{q}}
\newcommand{\bfr}{\mathbf{r}}
\newcommand{\bfs}{\mathbf{s}}
\newcommand{\bft}{\mathbf{t}}
\newcommand{\bfu}{\mathbf{u}}
\newcommand{\bfv}{\mathbf{v}}
\newcommand{\bfw}{\mathbf{w}}
\newcommand{\bfx}{\mathbf{x}}
\newcommand{\bfy}{\mathbf{y}}
\newcommand{\bfz}{\mathbf{z}}
\newcommand{\bfA}{\mathbf{A}}
\newcommand{\bfB}{\mathbf{B}}
\newcommand{\bfC}{\mathbf{C}}
\newcommand{\bfD}{\mathbf{D}}
\newcommand{\bfE}{\mathbf{E}}
\newcommand{\bfF}{\mathbf{F}}
\newcommand{\bfG}{\mathbf{G}}
\newcommand{\bfH}{\mathbf{H}}
\newcommand{\bfI}{\mathbf{I}}
\newcommand{\bfJ}{\mathbf{J}}
\newcommand{\bfK}{\mathbf{K}}
\newcommand{\bfL}{\mathbf{L}}
\newcommand{\bfM}{\mathbf{M}}
\newcommand{\bfN}{\mathbf{N}}
\newcommand{\bfO}{\mathbf{O}}
\newcommand{\bfP}{\mathbf{P}}
\newcommand{\bfQ}{\mathbf{Q}}
\newcommand{\bfR}{\mathbf{R}}
\newcommand{\bfS}{\mathbf{S}}
\newcommand{\bfT}{\mathbf{T}}
\newcommand{\bfU}{\mathbf{U}}
\newcommand{\bfV}{\mathbf{V}}
\newcommand{\bfW}{\mathbf{W}}
\newcommand{\bfX}{\mathbf{X}}
\newcommand{\bfY}{\mathbf{Y}}
\newcommand{\bfZ}{\mathbf{Z}}
% mathbb
\newcommand{\Abb}{\mathbb{A}}
%\newcommand{\Bbb}{\mathbb{B}}
\newcommand{\Cbb}{\mathbb{C}}
\newcommand{\Dbb}{\mathbb{D}}
\newcommand{\Ebb}{\mathbb{E}}
\newcommand{\Fbb}{\mathbb{F}}
\newcommand{\Gbb}{\mathbb{G}}
\newcommand{\Hbb}{\mathbb{H}}
\newcommand{\Ibb}{\mathbb{I}}
\newcommand{\Jbb}{\mathbb{J}}
\newcommand{\Kbb}{\mathbb{K}}
\newcommand{\Lbb}{\mathbb{L}}
\newcommand{\Mbb}{\mathbb{M}}
\newcommand{\Nbb}{\mathbb{N}}
\newcommand{\Obb}{\mathbb{O}}
\newcommand{\Pbb}{\mathbb{P}}
\newcommand{\Qbb}{\mathbb{Q}}
\newcommand{\Rbb}{\mathbb{R}}
\newcommand{\Sbb}{\mathbb{S}}
\newcommand{\Tbb}{\mathbb{T}}
\newcommand{\Ubb}{\mathbb{U}}
\newcommand{\Vbb}{\mathbb{V}}
\newcommand{\Wbb}{\mathbb{W}}
\newcommand{\Xbb}{\mathbb{X}}
\newcommand{\Ybb}{\mathbb{Y}}
\newcommand{\Zbb}{\mathbb{Z}}
% mathcal
\newcommand{\Acal}{\mathcal{A}}
\newcommand{\Bcal}{\mathcal{B}}
\newcommand{\Ccal}{\mathcal{C}}
\newcommand{\Dcal}{\mathcal{D}}
\newcommand{\Ecal}{\mathcal{E}}
\newcommand{\Fcal}{\mathcal{F}}
\newcommand{\Gcal}{\mathcal{G}}
\newcommand{\Hcal}{\mathcal{H}}
\newcommand{\Ical}{\mathcal{I}}
\newcommand{\Jcal}{\mathcal{J}}
\newcommand{\Kcal}{\mathcal{K}}
\newcommand{\Lcal}{\mathcal{L}}
\newcommand{\Mcal}{\mathcal{M}}
\newcommand{\Ncal}{\mathcal{N}}
\newcommand{\Ocal}{\mathcal{O}}
\newcommand{\Pcal}{\mathcal{P}}
\newcommand{\Qcal}{\mathcal{Q}}
\newcommand{\Rcal}{\mathcal{R}}
\newcommand{\Scal}{\mathcal{S}}
\newcommand{\Tcal}{\mathcal{T}}
\newcommand{\Ucal}{\mathcal{U}}
\newcommand{\Vcal}{\mathcal{V}}
\newcommand{\Wcal}{\mathcal{W}}
\newcommand{\Xcal}{\mathcal{X}}
\newcommand{\Ycal}{\mathcal{Y}}
\newcommand{\Zcal}{\mathcal{Z}}
\newcommand{\Hscr}{\mathscr{H}}
\newcommand{\II}{\mathbb{I}}
\newcommand{\EE}{\mathbb{E}}
\newcommand{\PP}{\mathbb{P}}
\newcommand{\VV}{\mathbb{V}}
\newcommand{\leqsim}{\lesssim}
\newcommand{\geqsim}{\gtrsim}
\newcommand{\minimize}{\mathop{\textrm{minimize}}}
\newcommand{\maximize}{\mathop{\textrm{maximize}}}
\newcommand{\iprod}[2]{\left\langle #1, #2 \right\rangle}
\newcommand{\nrm}[1]{\left\|#1\right\|}
\newcommand{\abs}[1]{\left|#1\right|}
\newcommand{\minop}[1]{\min\left\{#1\right\}}
\newcommand{\sqr}[1]{\left\|#1\right\|^2}
\newcommand{\prob}[1]{\mathbb{P}\left(#1\right)}
\newcommand{\epct}[1]{\mathbb{E}\left[#1\right]}
\newcommand{\cond}[2]{\mathbb{E}\left[\left.#1\right|#2\right]}
\newcommand{\condP}[2]{\mathbb{P}\left(\left.#1\right|#2\right)}
\newcommand{\condV}[2]{\mathbb{V}\left(\left.#1\right|#2\right)}
\newcommand{\bigO}[1]{\mathcal{O}\left(#1\right)}
\newcommand{\tbO}[1]{\tilde{\mathcal{O}}\left(#1\right)}
\newcommand{\tbOm}[1]{\tilde{\Omega}\left(#1\right)}
\newcommand{\Om}[1]{\Omega\left(#1\right)}
\newcommand{\ThO}[1]{\Theta\left(#1\right)}
\newcommand{\ceil}[1]{\left\lceil #1\right\rceil}
\newcommand{\floor}[1]{\left\lfloor #1\right\rfloor}
\newcommand{\Log}[1]{\log\left(#1\right)}
\newcommand{\bmy}{\bm{y}}
% \newcommand{\minop}[1]{\min\left\{#1\right\}}
\newcommand{\pos}[1]{\left[#1\right]_{+}}

\newcommand{\Lipt}{L_{\Theta}}
\newcommand{\Lipe}{L_e}
\newcommand{\oE}{\bar{E}}
\newcommand{\oP}{\operatorname{P}}
\DeclareRobustCommand{\rchi}{{\mathpalette\irchi\relax}}
\newcommand{\irchi}[2]{\raisebox{\depth}{$#1\chi$}}
\newcommand{\Lthre}{\mathcal{D}}
\newcommand{\tparameter}{\tilde{\parameter}}
\newcommand{\thre}{\mathrm{T}}


\newcommand{\cfn}[1]{{(\color{blue} cf:\,\,#1)}}
\title{Provable Convergence of Variational Monte Carlo Methods%for Solving Many-body Quantum Problems
}

\author{Tianyou Li\thanks{School of Mathematical Sciences, Peking University, CHINA (tianyouli@stu.pku.edu.cn).}
,\quad Fan Chen\thanks{School of Mathematical Sciences, Peking University, CHINA (chern@pku.edu.cn).}
,\quad  Huajie Chen\thanks{School of Mathematical Sciences, Beijing Normal University, CHINA  (chen.huajie@bnu.edu.cn).}
,\quad  Zaiwen Wen\thanks{Beijing International Center for Mathematical Research, Peking University, CHINA (wenzw@pku.edu.cn).}
% 
}


\begin{document}

\maketitle

\begin{abstract}
    The Variational Monte Carlo (VMC) is a promising approach for computing the ground state energy of many-body quantum problems and attracts more and more interests due to the development of machine learning.  The recent paradigms in VMC construct neural networks as trial wave functions, sample quantum configurations using Markov chain Monte Carlo (MCMC) and train neural networks with stochastic gradient descent (SGD) method.  However,  the theoretical convergence of VMC is still unknown when SGD interacts with MCMC sampling given a well-designed trial wave function.  Since MCMC reduces the difficulty of estimating gradients, it has inevitable bias in practice.  Moreover, the local energy may be unbounded, which makes it harder to analyze the error of MCMC sampling.  Therefore, we assume that the local energy is sub-exponential and use the Bernstein inequality for non-stationary Markov chains to derive error bounds of the MCMC estimator.  Consequently, VMC is proven to have a first order convergence rate $O(\log K/\sqrt{n K})$ with $K$ iterations and a sample size $n$.  It partially explains how MCMC influences the behavior of SGD.  Furthermore, we verify the so-called correlated negative curvature condition  and relate it to the zero-variance phenomena in solving eigenvalue functions.  It is shown that VMC escapes from saddle points and reaches $(\epsilon,\epsilon^{1/4})$ approximate second order stationary points or $\epsilon^{1/2}$-variance points in at least $O(\epsilon^{-11/2}\log^{2}(1/\epsilon) )$ steps with high probability. Our analysis enriches the understanding of how VMC converges efficiently and can be applied to general variational methods in physics and statistics. 
\end{abstract}


% \section{Introduction}
\label{sec:introduction}
% \begin{itemize}
%     % Diffusion of FL
%     \item {\st{Diffusion of FL}}
%     % Security threats to FL
%     \item {\st{Security threats to FL with particular focus on model poisoning}}
%     % Limitations of existing countermeasures
%     \item {\st{Current countermeasures (e.g., KRUM) and their limitations}}
%     % Proposed method and its advantages
%     \item {\st{Intuitive description of the proposed method and its difference (i.e., advantages) w.r.t. state of the art}}
%     % Main contributions
%     \item {\st{Summary of the main contributions of this work}}
%     % Paper's structure and organization
%     \item {\st{Paper's structure and organization}}
% \end{itemize}

% Diffusion of FL
Recently, {\em federated learning} (FL) has emerged as the leading paradigm for training distributed, large-scale, and privacy-preserving machine learning (ML) systems~\cite{mcmahan2017googleai,mcmahan2017aistats}. 
The core idea of FL is to allow multiple edge clients to collaboratively train a shared, global model without disclosing their local private training data.
%Specifically, an FL system consists of a central server and many edge clients; 
A typical FL round involves the following steps: {\em(i)} the server randomly picks some clients and sends them the current, global model; {\em(ii)} each selected client locally trains its model with its own private data; then, it sends the resulting local model to the server;\footnote{Whenever we refer to global/local model, we mean global/local model {\em parameters}.} {\em(iii)} the server updates the global model by computing an \emph{aggregation function}, usually the average (FedAvg), on the local models received from clients.
% \begin{enumerate}
%     \item[{\em(i)}] the server sends the current, global model to the clients and appoints some of them for training;
%     \item[{\em(ii)}] each selected client locally trains its copy of the global model with its own private data; then, it sends the resulting local model back to the server;\footnote{Whenever we refer to global/local model, we mean global/local model {\em parameters}.}
%     \item[{\em(iii)}] the server updates the global model by computing an \emph{aggregation function} on the local models received from clients (by default, the average, also referred to as FedAvg~\cite{mcmahan2017aistats}).
% \end{enumerate}
This process goes on until the global model converges. %(e.g., after a certain number of rounds or other similar stopping criteria).
%\\
% The advantages of FL over the traditional, centralized learning paradigm are undoubtedly clear in terms of flexibility/scalability (clients can join/disconnect from the FL network dynamically), network communications (only model weights\footnote{We will use \textit{parameters} and \textit{weights} interchangeably.} are exchanged between clients and server), and privacy (each client's private training data is kept local at the client's end and not uploaded to the server).
\\
% Security threats to FL
%However, the growing adoption of FL also raises security concerns~\cite{costa2022covert}, particularly about its confidentiality, integrity, and availability.
Although its advantages over standard ML, FL also raises security concerns~\cite{costa2022covert}. %, particularly about its confidentiality, integrity, and availability~\cite{costa2022covert}.
% OLD, LONG VERSION
% Indeed, some work deals with privacy leakage that may expose the local data of some clients~\cite{melis2019sp}. 
% A large body of work, instead, investigates attacks that usually aim to detriment the predictive accuracy of the learned global model. For instance, \emph{data poisoning} attacks achieve this goal by letting an adversary pollute the training set of some corrupt FL clients with maliciously crafted examples~\cite{jagielski2018sp}.
% Similarly, in \emph{model poisoning} the attacker attempts to tweak the global model weights~\cite{bhagoji2019pmlr} by directly perturbing the local model's weights of some infected FL clients before these are sent to the central server for aggregation, usually via so-called Byzantine attacks. 
% It turns out that Byzantine model poisoning attacks severely impact standard FedAvg; therefore, more robust aggregation functions must be designed to make FL systems secure.
Here, we focus on \emph{untargeted model poisoning} attacks~\cite{bhagoji2019pmlr}, where an adversary attempts to tweak the global model weights %\footnote{We will use the terms \textit{parameters} and \textit{weights} interchangeably.} 
by directly perturbing the local model's parameters of some infected clients before these are sent to the central server for aggregation.
In doing so, the adversary aims to jeopardize the global model \textit{indiscriminately} at inference time.
Such model poisoning attacks severely impact standard FedAvg; therefore, more robust aggregation functions must be designed to secure FL systems.
\\
% In this paper, we focus on designing a novel robust aggregation scheme at the server's end to contrast the effect of Byzantine model poisoning attacks.
%
% Current countermeasures and their limitations
%Several countermeasures have been proposed in the literature to combat model poisoning attacks on FL systems.
% Some methods use simple statistics more robust than plain average to smooth the impact of malicious updates (e.g., Trimmed Mean and FedMedian~\cite{yin2018icml}). 
% Other defenses implement outlier detection techniques to discard malicious updates from the aggregation performed at the server's end. Those are either based on heuristics (e.g., Krum/Multi-Krum~\cite{blanchard2017nips} and Bulyan~\cite{mhamdi2018pmlr}) or data-driven approaches (e.g., K-means clustering~\cite{shen2016acm} or DnC via spectral analysis~\cite{shejwalkar2021ndss}). 
% Finally, some strategies rely on a centralized ``source of trust'' to spot potential malicious updates (e.g., FLTrust~\cite{cao2020fltrust}).
% Several countermeasures have been proposed in the literature to combat model poisoning attacks on FL systems, i.e., to discard possible malicious local updates from the aggregation performed at the server's end. 
% These techniques range from simple statistics more robust than plain average (e.g., Trimmed Mean and FedMedian~\cite{yin2018icml}) to outlier detection heuristics (e.g., Krum/Multi-Krum~\cite{blanchard2017nips} and Bulyan~\cite{mhamdi2018pmlr}) or data-driven approaches (e.g., spectral analysis via K-means clustering~\cite{shen2016acm} or spectral analysis), or methods based on ``source of trust'' (e.g., FLTrust~\cite{cao2020fltrust}).
% OLD, LONG VERSION
%Several countermeasures have been proposed in the literature to combat Byzantine model poisoning attacks on FL systems.
% Descriptive statistics
% For example, Trimmed Mean and FedMedian aggregate local model updates using more robust statistics than standard average~\cite{yin2018icml}.
%
% % Heuristics for outlier detection
% Many existing Byzantine-resilient strategies implement some outlier detection heuristics to discard the model updates sent by potentially malicious clients from the input of the aggregation function.
% One of the most popular heuristics is Krum~\cite{blanchard2017nips}.
% This strategy tries to mitigate the impact of Byzantine attacks by selecting as a global model the local model with the smallest sum of Euclidean distances to {\em all} the other local models.
% Although powerful, Krum requires the server to know (or, at least, estimate) the number of malicious FL clients upfront, which is generally impossible in a realistic attack scenario. %
% Moreover, Krum may become ineffective for complex, high-dimensional model parameter spaces due to the curse of dimensionality.
% Bulyan~\cite{mhamdi2018pmlr} tries to overcome this issue by combining Krum with a variant of Trimmed Mean.
% % Data-driven outlier detection
% Other strategies use data-driven outlier detection techniques -- e.g., via K-means clustering~\cite{shen2016acm} -- to spot potential malicious local model updates. 
% %For instance, Shen et al. propose to cluster local model updates with K-means and thus identify outliers.
%
% % Other techniques
% As far as the server is concerned, any local model received can be from a potential malicious client. 
% FLTrust~\cite{cao2020fltrust} assumes the server acts as a client, i.e., trains a local model on an additional {\em trustworthy} dataset at the server's end and compares it against all the local models from other clients. 
% This way, the server can rely on some ``source of trust'' when discarding potentially malicious clients.
%\\
% Limitations of existing Byzantine-resilient strategies
Unfortunately, existing defense mechanisms either rely on simple heuristics (e.g., Trimmed Mean and FedMedian by~\cite{yin2018icml}) or need strong and unrealistic assumptions to work effectively (e.g., foreknowledge or estimation of the number of malicious clients in the FL system, as for Krum/Multi-Krum~\cite{blanchard2017nips} and Bulyan~\cite{mhamdi2018pmlr}, which, however, cannot exceed a fixed threshold).
Furthermore, outlier detection methods using K-means clustering~\cite{shen2016acm} or spectral analysis like DnC~\cite{shejwalkar2021ndss} do not directly consider the temporal evolution of local model updates received.
Finally, strategies like FLTrust~\cite{cao2020fltrust} require the server to collect its own dataset and act as a proper client, thereby altering the standard FL protocol.
\\
% OLD, LONG VERSION
% Overall, existing Byzantine-resilient strategies are either simple heuristics (e.g., FedMedian) or, if they are more complex, they rely on strong and unrealistic assumptions to work effectively (e.g., knowing the number of malicious clients in the FL system in advance, as for Krum and alike).
% Furthermore, data-driven outlier detection methods do not consider the temporary evolution of local model updates received (e.g., K-means clustering). 
% Finally, strategies like FLTrust requires the server to collect its own dataset and act as a proper client, thereby altering the standard FL protocol.
%
% Description of the proposed method
This work introduces a novel pre-aggregation \textit{filter} robust to untargeted model poisoning attacks. Notably, this filter $(i)$ operates without requiring prior knowledge or constraints on the number of malicious clients and $(ii)$ inherently integrates temporal dependencies. 
The FL server can employ this filter as a preprocessing step before applying \textit{any} aggregation function, be it standard like FedAvg or robust like Krum or Bulyan.
Specifically, we formulate the problem of identifying corrupted updates as a multidimensional (i.e., matrix-valued) time series anomaly detection task. 
The key idea is that legitimate local updates, resulting from well-calibrated iterative procedures like stochastic gradient descent (SGD) with an appropriate learning rate, show \textit{higher predictability} compared to malicious updates. This hypothesis stems from the fact that the sequence of gradients (thus, model parameters) observed during legitimate training exhibit regular patterns, as validated in Section~\ref{subsec:intuition}. %until convergence. 
%This regularity may be more pronounced for smooth convex loss functions, but it can still be captured within an appropriate time window, even for more complex and convoluted loss surfaces. 
%We provide evidence of this claim in Appendix~B, where we show that the average mutual information (i.e., ``predictability''), calculated over pairs of legitimate model updates sent at different FL rounds, is significantly higher than the corresponding computation for a malicious client.
\\
Inspired by the matrix autoregressive (MAR) framework for multidimensional time series forecasting~\cite{chen2021je}, we propose the FLANDERS ({\em \textbf{F}ederated \textbf{L}earning meets \textbf{AN}omaly \textbf{DE}tection for a \textbf{R}obust and \textbf{S}ecure}) filter.
The main advantages of FLANDERS over existing strategies like FLDetector~\cite{zhao2020multivariate} are its resilience to large-scale attacks, where $50\%$ or more FL participants are hostile, and the capability of working under realistic non-iid scenarios.
We attribute such a capability to two key factors: $(i)$ FLANDERS works without knowing a priori the ratio of corrupted clients, and $(ii)$ it embodies temporal dependencies between intra- and inter-client updates, quickly recognizing local model drifts caused by evil players. Below, we summarize our main contributions:

\begin{itemize}
\item[{\em(i)}]
We provide empirical evidence that the sequence of models sent by legitimate clients is more predictable than those of malicious participants performing untargeted model poisoning attacks.
\\
\item[{\em(ii)}] 
We introduce FLANDERS, the first pre-aggregation filter for FL robust to untargeted model poisoning based on multidimensional time series anomaly detection.
\\
\item[{\em(iii)}] 
We integrate FLANDERS into Flower,\footnote{\scriptsize{\url{https://flower.dev/}}} a popular FL simulation framework for reproducibility.
\\
\item[{\em(iv)}] 
We show that FLANDERS improves the robustness of the existing aggregation methods under multiple settings: different datasets, client's data distribution (non-iid), models, and attack scenarios.
\\
\item[{\em(v)}] 
We publicly release all the implementation code of FLANDERS along with our experiments.\footnote{\scriptsize{\url{https://anonymous.4open.science/r/flanders_exp-7EEB}}}
\end{itemize}

% Paper's structure and organization
The remainder of the paper is structured as follows. %some related work and the current state-of-the-art solutions to security issues that FL entails. 
Section~\ref{sec:background} covers background and preliminaries. 
In Section~\ref{sec:related}, we discuss related work.
Section~\ref{sec:problem} and Section~\ref{sec:method} describe the problem formulation and the method proposed. % to tackle it. 
Section~\ref{sec:experiments} gathers experimental results. %, and Section~\ref{sec:limitations} discusses some limitations of this work.
Finally, we conclude in Section~\ref{sec:conclusion}.
 %discusses the limitations of this work and draws future research directions.
%reports conclusions and draws perspectives for future research directions.

%%%%%%% OLD %%%%%%%
%to overcome the resilience of Byzantine failures in distributed Stochastic Gradient Descent computations. 
% The strength of Krum is its time complexity, which is linear in the gradient dimension. 
% However, the robustness of the approach is guaranteed for gradient-based learning applications only when the majority of the clients are not compromised. 
% Besides, the aggregation mechanism of Krum, as well as that of similar methods, is robust from a coarse-grained perspective and does not provide solutions to errors and perturbations that may occur at inference time.
%A related approach to~\cite{blanchard2017nips} is the work of Su et al.~\cite{su2016dc}. Here, the authors propose an iterated approximate agreement to tackle a multi-layer scenario attacked by Byzantine agents. 
%However, the method works efficiently on the sole discrete context and it is inapplicable to continuous state environments.
%\gabri{Maybe, we should just talk about the main limitations of existing countermeasures without digging into their details (or, we can just mention Krum as this is the most popular one). I will move the description of all these methods to the Related Work section.}
% \section{Proposed Framework: {\ourmodel}}
\label{model}


In this section, we introduce a novel self-supervised co-training framework {\ourmodel}.
The proposed framework is illustrated in Figure~\ref{fig:intro_model} and works in three phases.
Phase one automatically generates two sets of pseudo labels.
We use a combination of off-the-shelf pre-trained POS and NER taggers, knowledge graph, and GPT-2 scorer for generating the first set of pseudo labels automatically without any hand-crafted rules for matching the slot values.
The other set of pseudo labels is acquired through a zero-shot slot filling model~\cite{liu2020coach}, trained on the out-of-domain dataset.
It is critical to emphasize that both sets of labels are noisy and incomplete which poses serious challenges to training effective models for the task of open-domain slot filling.
Phase two fine-tunes the pre-trained BERT to the slot filling task that effectively transfers the knowledge from the pre-trained language model~(LM) to overcome the issue of label incompleteness to some extent. 
Further, we employ the early stopping technique to minimize the noise in the labels.
The output of this phase is two BERT models that can generate soft labels for self-supervision during co-training in phase three.
Phase three leverages the fine-tuned models and further trains them in an iterative fashion.
Specifically, the proposed peer training approach facilitates high-confidence soft label selection for the other peer to perform training. This phase progressively reduces the noise in the labels and enables effective model fitting. 



\subsection{Phase One: Automatic Label Generation}
To acquire the first set of labels, we perform the following steps.
First of all, off-the-shelf trained POS and NER taggers are used to predict initial estimates of the slot values irrespective of the slot types. Then, the type information of the slot values is queried from the KG and the slot value is tagged for the most appropriate slot in the target domain.
This approach, however, produces low recall. 
To expand the candidate slot values, we generate n-grams of the natural language text and employ a partial matching scheme to query the KG for type information (e.g., \myspecial{Jason} \myspecial{Aldean} = \myspecial{American} \myspecial{singer}) of the n-grams if the entry exists.
This process generates multiple overlapping hypotheses about the slot values.
We replace a span of text that corresponds to a slot value by its type information and a GPT-2 based scorer (see Section~\ref{sec:nlpmodels}) is used to select the best candidate based on the fluency of the text.
Naturally, if a token (or span of tokens) is replaced by its type, the sentence should score higher as compared to the case where an inappropriate substitution is performed. 
We select the best hypothesis if the score is greater than the threshold.
Intuitively, the candidate selection threshold can automatically be searched based on a small validation set from the target domain, making the label generation process fully automatic. 
The other set of noisy labels is acquired by the zero-shot slot filling model~\cite{liu2020coach} that has been trained using an out-of-domain dataset. It is important to highlight that the zero-shot slot filling model does not require any labeled in-domain training example. 
To summarize the automatic label generation phase, both sets of labels are acquired in a fully automatic fashion without any hand-crafting.


In contrast to previous work in weak supervision~\cite{ren2015clustype,he2017autoentity,fries2017swellshark,giannakopoulos2017unsupervised} that obtains a single set of noisy labels and then propose techniques to overcome the challenge of fitting an effective model to the noisy labels, we acquire two sets of complementary labels.
The choice of these two sets of labels is guided by the intuition that they should be complementary and the models trained on these sets of labels should be able to share complementary information with the other to improve the performance in the later phases of the framework.
Essentially, the first set of labels carries information from external knowledge sources, whereas the labels generated through the pre-trained zero-shot slot filling model capture how the slot values are mentioned in other domains.
%
To further elaborate on the motivation and our process for the first set of labels (i.e., labels using KG and other NLP models), the pre-trained LMs have been shown to have a great deal of knowledge~\cite{petroni2019language}, thus should be capable of generating automatic labels with no need of external KG. 
To the best of our knowledge, there exists no work that shows that accurate token-level automatic labeling (e.g., slot filling task) is possible with pre-trained LMs. 
Moreover, such approaches would require heavy prompting in each new target domain, whereas our label generation process is fully automatic and only relies on the readily-available pre-trained NLP models and external KG.

\subsection{Phase Two: LM-assisted Weak Supervision}
Since we do not have access to dataset $\{(\mathbf{X}_n,\mathbf{Y}_n)\}_{n=1}^N$ with true ground-truth labels.
We use pseudo labels generated in phase one, $\{(\mathbf{X}_n,\mathbf{D}_n)\}_{n=1}^N$, to learn 
$f_{m,c}(\cdot; \cdot)$ that outputs the probability of the $m$-th token to take on class $c$. 
We learn $f_{m,c}(\cdot; \cdot)$ by minimizing the following loss over the noisy dataset $\{(\mathbf{X}_n,\mathbf{D}_n)\}_{n=1}^N$: 
$$
\hat\theta = \argmin_{\theta}\frac{1}{N}\sum_{n=1}^{N} \ell(\mathbf{D}_n, f(\mathbf{X}_{n}; \theta)),
\label{eq:stage1}
$$
where $\ell(\mathbf{D}_n, f(\mathbf{X}_{n}; \theta)) = \frac{1}{M} \sum_{m=1}^{M} -\log{f_{m,d_{n, m}}(\mathbf{X}_{n}; \theta)}$. 
We employ the pre-trained multilingual BERT with token-level classification head that uses Adam optimizer \cite{kingma2014adam,Liu2019} with early stopping and multiple random initializations. 


Since slot filling task is similar to the MLM training objective of the BERT, we employ pre-trained BERT as the backbone model.
That is, MLM's goal is to predict the masked tokens using bidirectional contexts. Similarly, slot filling tries to predict the label for a token leveraging both left and right contexts simultaneously, which makes the pre-trained BERT an ideal model of choice that greatly facilitates minimizing incomplete labels.
It is important to highlight that our automatically generated labels are not only incomplete but also potentially wrong.
The training strategies employed in this phase minimize the noise in the label to some extent. 
Specifically, early stopping can provide a strong regularization and would not let the model overfit to the noisy labels, especially wrong labels. 
Moreover, early stopping does not let the model forget the knowledge in the pre-trained model.
Similarly, multiple random initializations enforce robustness. 
Since the model is fine-tuned on the noisy labels, averaging the predictions of multiple models for each token ensures that wrong labels end up with low probabilities and true labels consistently achieve high probabilities.
Using the above-mentioned strategies, we train two slot filling models, which we call the peers. The peer one is trained on the first set of pseudo labels that were generated using POS and NER taggers, KG, and the GPT-2 scorer in phase one. Similarly, peer two is trained using the predictions of the zero-shot slot filling model~\cite{liu2020coach}.
Both models have the same architecture and follow the same training procedures.

\begin{table*}[t!]
\centering
\caption{Dataset statistics.}
\vspace{-7pt}
\label{tab:dataset}
\begin{tabular}{lccccc}
\toprule
\textbf{Dataset}  & \textbf{Dataset Size} & \textbf{Vocab. Size} & \textbf{Avg. Length} & \textbf{\# of Domains} & \textbf{\# of Slots} \\ \hline
\textbf{SGD}      & 188K                  & 33.6K                & 13.8                 & 20                     & 240                  \\
\textbf{MultiWoZ} & 67.4K                 & 10.5K                & 13.3                 & 8                      & 61 \\
\bottomrule
\end{tabular}
\vspace{-7pt}
\end{table*}

\subsection{Phase Three: Self-supervised Co-training}
We introduce an iterative peer training algorithm where both peers generate high-confidence soft labels for training the other peer in the next iteration. 
Theoretically, these peers can be anything, but in this work, 
we explore two of the most promising directions that have shown the promise to minimize the need for manual labeling for the task: zero-shot learning and distant supervision.
This phase uses a self-supervised co-training scheme to exploit the patterns of slot values from other domains through the labels generated by the zero-shot filling model (i.e., peer two)~\cite{liu2020coach} as well as utilize the knowledge in external KGs and pre-trained models via labels provided by the peer one.
Specifically, we initialize the peers trained in phase two and use their pseudo labels to kick-start training in this phase.
Specifically, peer one $f_{m,c}(\cdot; \theta_{\textrm{p1}})$ would generate labels $\{\tilde{\mathbf{Y}}^{(t)}_n = [\tilde{y}_{n,1}^{(t)}, ..., \tilde{y}_{n,m}^{(t)}]\}_{n=1}^{N}$ for peer two $f_{m,c}(\cdot; \theta_{\textrm{p2}})$ at the $t$-th iteration by:
$$
\tilde{y}_{n,m}^{(t)} = \argmax_{c}{f_{m,c}(\mathbf{X}_n; \theta_{\textrm{p1}}^{(t)})}. 
\label{eq:pseudo}
$$

Based on these labels, the peer two can be fine-tuned by: 
$$
\hat\theta_{\textrm{p2}}^{(t+1)} = \argmin_{\theta}\frac{1}{N}\sum_{n=1}^N \ell(\tilde{\mathbf{Y}}_n^{(t)}, f(\mathbf{X}_{n}; \theta)).
\label{eq:self_train1}
$$

Similarly, peer two $f_{m,c}(\cdot; \theta_{\textrm{p2}})$ would generate pseudo labels for peer one $f_{m,c}(\cdot; \theta_{\textrm{p1}})$ that are used to fine-tune peer one. 
We also notice that it is beneficial to stop early during this phase as well, to improve the model fitting and gradually reduce the noise associated with the automatically generated labels.
Since pseudo labels are refined gradually in an iterative way, both peers can benefit from the knowledge contained within the labels of the other while avoiding overfitting.
Furthermore, as an alternative to pseudo labels, we also generate soft labels that are used for confidence re-weighting. 
The high-confidence soft label selection strategy enables better model fitting and efficient learning via better quality of the automatic labels.
Specifically, for the given $m$-th token in the $n$-th training example, the probability for all classes $C$ is $[f_{m,1}(\mathbf{X}_n;\theta),...,f_{m,C}(\mathbf{X}_n;\theta)]$. 
Following ~\cite{xie2016unsupervised}, at $t$-th iteration, peer one generates soft labels, $\{\mathbf{S}_n^{(t)} = [\mathbf{s}_{n,m}^{(t)}]_{m=1}^M \}_{n=1}^N$, as given below:
$$
\mathbf{s}_{n,m}^{(t)} = [s_{n,m,c}^{(t)}]_{c=1}^{C} = \Bigg[  \frac{f_{m,c}^2(\mathbf{X}_n;\theta_{\textrm{peer1}}^{(t)})/p_{c}}{\sum_{c'=1}^C f_{m,c'}^2(\mathbf{X}_n;\theta_{\textrm{peer1}}^{(t)})/p_{c'}}\Bigg]_{c=1}^{C}
\label{eq:soft}
$$ 
where $p_{c} = \sum_{n=1}^N \sum_{m=1}^M f_{m,c}(\mathbf{X}_n;\theta_{\textrm{p1}}^{(t)})$ computes the frequency of the tokens for the $c$-th class. 
Then, peer two $f(\cdot; \theta_{\textrm{p2}}^{(t+1)})$ is fine-tuned by:
$$
\theta_{\textrm{p2}}^{(t+1)} = \argmin_{\theta} \frac{1}{N} \sum_{n=1}^{N} \ell_{\rm KL}(\mathbf{S}_n^{(t)}, f(\mathbf{X}_{n}; \theta)),
$$
where $\ell_{\rm KL}(\cdot,\cdot)$ is the KL-divergence-based loss:
$$
\ell_{\rm KL}(\mathbf{S}_n^{(t)}, f(\mathbf{X}_{n}; \theta))=\frac{1}{M}\sum_{m=1}^M\sum_{c=1}^C - s_{n,m,c}^{(t)} \log f_{m,c}(\mathbf{X}_{n}; \theta).
\label{eq:klloss}
$$

Moreover, we also investigate selecting tokens that have high confidence. 
For instance, we pick high-confidence tokens from the $m$-th input example at the $t$-th iteration by  
$
H^{(t)}_n = \{m : \max_{c} s_{n,m,c}^{(t)} > \epsilon \},
$
where $\epsilon\in [0,1]$ is a threshold that can be searched based on a small validation set. 
Then, peer two $f(\cdot; \theta_{\textrm{p2}}^{(t+1)})$ is fine-tuned by:
$$
\theta_{\textrm{p2}}^{(t+1)} %&= \argmin_{\theta} \frac{1}{N} \sum_{n=1}^{N} \ell_{\rm S-KL}(\bS_n^{(t)}, f(\bX_{n}; \theta)) \\
= \argmin_{\theta} \frac{1}{N|H^{(t)}_n|}\sum_{n=1}^{N} \sum_{m\in H^{(t)}_n}\sum_{c=1}^C - s_{n,m,c}^{(t)} \log f_{m,c}(\mathbf{X}_{n}; \theta).
$$

This phase improves the robustness to effectively fit the model for tokens with high confidence. 
Both peers keep sharing information and their confidence by producing soft labels for their counterparts until they approximate to the true labels while employing early stopping and scheduled learning rates.
It is important to remind that phase three is the most important phase that progressively reduces noise from the labels to a great extent and enables superior performance for the task of open-domain slot filling.
% \input{vmc.tex}
% In Section~\ref{subsec:rgd_algo}, we describe the RGD algorithm for solving the alignment problem in Eq.~(\ref{eq:GPOP}). Then in \revdel{Section~\ref{subsec:loc_sub_conv} we prove the local sublinear convergence of RGD to a critical point of \revdel{$F$}\revadd{$\widetilde{F}$}. In} Section~\ref{subsec:loc_lin_conv}, using the theory of Morse\revdel{-Bott} functions \revadd{ and Proposition~\ref{prop:HessVicinity}}, we \revdel{extend the result to}\revadd{show} the local linear convergence of RGD to a non-degenerate alignment (see Section~\ref{subsec:non_deg_gen_setting}). \revadd{TODO: add one more line about contrasting conditions and exact recovery.}

\subsection{RGD Algorithm}
\label{subsec:rgd_algo}
A standard way to find a local minimum of Eq.~(\ref{eq:GPOP}) is to use RGD with a suitable initial point, step size and retraction strategy.
%In this work, our choice of retraction is based on \revdel{QR decomposition}\revadd{polar decomposition},
\revadd{In this work we use retraction based on the exponential map on $\mathbb{O}(d)$. Define,}
\revdel{
\begin{align}
    R_{\QR }: \cup_{\mathbf{S} \in \mathbb{O}(d)^m}(\{\mathbf{S}\} \times T_\mathbf{S}\mathbb{O}(d)^m) &\mapsto \mathbb{O}(d)^m\\
    R_{\QR }\left(\begin{bmatrix}\mathbf{S}_1\\\vdots\\\mathbf{S}_m\end{bmatrix}, \begin{bmatrix}\boldsymbol{\xi}_1\\\vdots\\\boldsymbol{\xi}_m\end{bmatrix}\right) &= \begin{bmatrix}\qf (\mathbf{S}_1+\boldsymbol{\xi}_1)\\\vdots\\\qf (\mathbf{S}_m+\boldsymbol{\xi}_m)\end{bmatrix}. \label{eq:R_QR}
\end{align}
}
\revadd{
\begin{align}
    R_{\EXP }: \cup_{\mathbf{S} \in \mathbb{O}(d)^m}(\{\mathbf{S}\} \times T_\mathbf{S}\mathbb{O}(d)^m) &\mapsto \mathbb{O}(d)^m\\
    R_{\EXP }\left([\mathbf{S}_i]_1^m, [\boldsymbol{\xi}_i]_1^m\right) &= [\mathbf{S}_i\exp(\mathbf{S}_i^T\boldsymbol{\xi}_i)]_1^m. \label{eq:R_PF}
\end{align}
}\revdel{where $\qf (\mathbf{A})$ denotes the $\mathbf{Q}$ factor in the thin QR decomposition of $\mathbf{A}$ \citea{van1996matrix, absil2009optimization}}\revadd{where $\exp (\mathbf{A})$ denotes the matrix exponential of $\mathbf{A}$ \citea{van1996matrix, absil2009optimization}. Then the following lemma provides a consistent definition of retraction on the quotient manifold $\mathbb{O}(d)^m/_{\sim}$.}
\revadd{
\begin{lem}
\label{lem:retraction}
Let $\widetilde{\mathbf{S}} \in \mathbb{O}(d)^{m}/_{\sim}$ and $\mathbf{S}^a, \mathbf{S}^b \in \pi^{-1}(\widetilde{\mathbf{S}})$. If $\mathbf{Z}^a \in T_{\mathbf{S}^a}\mathbb{O}(d)^m$ and $\mathbf{Z}^b \in T_{\mathbf{S}^b}\mathbb{O}(d)^m$ are the horizontal lifts of $\widetilde{\mathbf{Z}} \in T_{\widetilde{\mathbf{S}}}\mathbb{O}(d)^{m}/_{\sim}$ then $\pi(R_{\EXP }(\mathbf{S}^a, \mathbf{Z}^a)) = \pi(R_{\EXP }(\mathbf{S}^b, \mathbf{Z}^b))$. As a result, the retraction
\begin{align}
    \widetilde{R}_{\EXP }: \cup_{\widetilde{\mathbf{S}} \in \mathbb{O}(d)^m/_{\sim} }(\{\widetilde{\mathbf{S}}\} \times T_{\widetilde{\mathbf{S}}}\mathbb{O}(d)^m/_{\sim}) &\mapsto \mathbb{O}(d)^m/_{\sim}\\
    \widetilde{R}_{\EXP }\left([\widetilde{\mathbf{S}}_i]_1^m, [\widetilde{\mathbf{Z}}_i]_1^m\right) &=  \pi(R_{\EXP }(\mathbf{S}, \mathbf{Z}))\label{eq:Rtilde_PF}
\end{align}
is well defined for any $\mathbf{S} \in \pi^{-1}(\widetilde{\mathbf{S}})$ and $\mathbf{Z}$ being the horizontal lift of $\widetilde{\mathbf{Z}}$ at $\mathbf{S}$.
\end{lem}
}

\revadd{The step direction will always be the horizontal lift of $-\grad \widetilde{F}(\widetilde{\mathbf{S}})$ at some $\mathbf{S} \in \pi^{-1}(\widetilde{\mathbf{S}})$. Consequently, due to Proposition~\ref{prop:gradFS},} the step direction is $\boldsymbol{\xi} = -\grad F(\mathbf{S})$ which is the projection of the antigradient $-\nabla F(\mathbf{S})$ onto $T_\mathbf{S}\mathbb{O}(d)^m$. Recall (from the proof of Proposition~\ref{prop:gradFS}) that $\grad F(\mathbf{S}) = [[\mathbf{C}\mathbf{S}]_i - \mathbf{S}_i[\mathbf{C}\mathbf{S}]_i^T\mathbf{S}_i]_1^m$. Then the step size $\alpha$ is calculated using the Armijo-type rule with parameters $\beta,\gamma \in (0,1)$ (here $g$ is the canonical metric on $\mathbb{O}(d)^m$ as in Eq.~(\ref{eq:g_Z_W})),
\begin{equation}
    \alpha = \max_{l \geq 0}\{\beta^l\ \vertbar\ F(R_{\EXP }(\mathbf{S}, -\beta^l\grad F(\mathbf{S}))) - F(\mathbf{S}) \leq -\gamma \beta^l g(\nabla F(\mathbf{S}),  \grad F(\mathbf{S})) \}. \label{eq:armijo_step}
\end{equation}
\revadd{Since $F$ extends to a continuously differentiable non-negative function on $\mathbb{R}^{md \times d}$ containing $\mathbb{O}(d)^m$, it follows from \citeb[Proposition 2.8]{schneider2015convergence} that $\alpha$ is well-defined.}

\revdel{
\begin{algorithm}
\caption{Riemannian gradient descent for solving GPOP \label{algo:rgd_old}}
\revdel{
\begin{algorithmic}[1]
\REQUIRE $\mathbf{S}^0 \in \mathbb{O}(d)^m$, $\Gamma$, $\{\mathbf{x}_{k,i}: (k,i) \in E(\Gamma)\}$, $\beta, \gamma \in (0,1)$
\STATE Construct $\mathbf{C}$ as in Eq.~(\ref{eq:GPOP}).
\REPEAT
    \STATE calculate the descent direction $-\grad F(\mathbf{S}^k)$ at $\mathbf{S}^k$ using Eq.~(\ref{eq:gradFS}).
    \STATE calculate the step size $\alpha_k$ according to the Armijo-type rule (see Eq.~(\ref{eq:armijo_step})).
    \STATE set $\mathbf{S}^{k+1} = R_{\QR }(\mathbf{S}^k, -\alpha_k \grad F(\mathbf{S}^k))$ using Eq.~(\ref{eq:R_PF}).
    \STATE $k \leftarrow k + 1$.
\UNTIL{convergence.}
\end{algorithmic}
}
\end{algorithm}
}

\begin{algorithm}
\caption{Riemannian gradient descent for solving GPOP \label{algo:rgd}}
\revadd{
\begin{algorithmic}[1]
\REQUIRE $\widetilde{\mathbf{S}}^0 \in \mathbb{O}(d)^{m-1}$, $\Gamma$, $\{\mathbf{x}_{k,i}: (k,i) \in E(\Gamma)\}$, $\beta, \gamma \in (0,1)$
\STATE Construct $\mathbf{C}$ as in Eq.~(\ref{eq:GPOP}).
\REPEAT
    \STATE set $\mathbf{S}^k = [\mathbf{I}_d; \widetilde{\mathbf{S}}^k] \in \pi^{-1}(\widetilde{\mathbf{S}}^k) \subset \mathbb{O}(d)^m$ (Eq.~\ref{eq:pi_inv_wtS}).
    \STATE calculate the descent direction $-\grad F(\mathbf{S}^k)$ at $\mathbf{S}^k$ using Eq.~(\ref{eq:gradFS}).
    \STATE calculate the step size $\alpha_k$ according to the Armijo-type rule (see Eq.~(\ref{eq:armijo_step})).
    \STATE set $\widetilde{\mathbf{S}}^{k+1} = \widetilde{R}_{\EXP}(\mathbf{S}^k, -\alpha_k \grad F(\mathbf{S}^k))$ using Eq.~(\ref{eq:R_PF}, \ref{eq:pi}).
    \STATE $k \leftarrow k + 1$.
\UNTIL{convergence.}
\end{algorithmic}
}
\end{algorithm}

\revdel{
\subsection{Local Sublinear Convergence of RGD}
\label{subsec:loc_sub_conv}
}
\revadd{
\subsection{Local linear Convergence of RGD}
\label{subsec:loc}
}
We proceed to show the local sublinear convergence of Algorithm~\ref{algo:rgd} to a non-degenerate alignment (see Definition~\ref{def:non_deg_alignment0}). Our main tool will be the convergence analysis framework presented in \citeb[Section 2.3]{schneider2015convergence} as used in \citea{liu2019quadratic}.
\revdel{To this end, we first note that $F$ is a real-analytic function bounded from below by zero and $\mathbb{O}(d)^m$ is a compact submanifold of $\mathbb{R}^{md \times d}$. Thus, the Lojasiewicz gradient inequality \citea{lojasiewicz1965ensembles}, \citeb[Section 2.2]{schneider2015convergence} holds at every $\mathbf{S}^* \in \mathbb{O}(d)^m$ and in particular for every $\mathbf{S}^* \in \mathcal{C}$ (see Eq.~(\ref{eq:crit_pts2})) i.e. there exist $\delta, \eta > 0$ and $\theta \in (0,1/2]$ (generally dependent on $\mathbf{S}^*$) such that}
\revadd{To this end, we first note that $\widetilde{F}$ is a real-analytic function bounded from below by zero and $\mathbb{O}(d)^m/_{\sim}$ (whose elements are identified with $\mathbb{O}(d)^{m-1}$ here) is a compact submanifold of $\mathbb{R}^{(m-1)d \times d}$. Thus, the Lojasiewicz gradient inequality \citea{lojasiewicz1965ensembles}, \citeb[Section 2.2]{schneider2015convergence} holds at every $\widetilde{\mathbf{S}}^* \in \mathbb{O}(d)^m/_{\sim}$ and in particular for every $\widetilde{\mathbf{S}}^* \in \widetilde{\mathcal{C}}$ (see Eq.~(\ref{eq:crit_pts})) i.e. there exist $\delta, \eta > 0$ and $\theta \in (0,1/2]$ (generally dependent on $\widetilde{\mathbf{S}}^*$) such that}
\revdel{
\begin{equation}
    |F(\mathbf{S}) - F(\mathbf{S}^*)|^{1-\theta} \leq \eta \left\| \grad F(\mathbf{S})\right\|_F. \label{eq:Lojasiewicz_gradient_ineq_old}
\end{equation}
holds for every $\mathbf{S} \in \mathbb{O}(d)^m$ satisfying $\left\|\mathbf{S}-\mathbf{S}^*\right\|_F < \delta$.
}
\revadd{
\begin{equation}
    |\widetilde{F}(\widetilde{\mathbf{S}}) - \widetilde{F}(\widetilde{\mathbf{S}}^*)|^{1-\theta} \leq \eta \left\| \grad \widetilde{F}(\widetilde{\mathbf{S}})\right\|_F. \label{eq:Lojasiewicz_gradient_ineq}
\end{equation}
holds for every $\widetilde{\mathbf{S}} \in \mathbb{O}(d)^m/_{\sim}$ satisfying $\left\|\widetilde{\mathbf{S}}-\widetilde{\mathbf{S}}^*\right\|_F < \delta$.
}

\revdel{
Then using Theorem 2.3 in \citea{schneider2015convergence} and the fact that $\mathbb{O}(d)^m$ is compact (thus every sequence on it has a cluster point), for the Algorithm~\ref{algo:rgd} to converge at least sublinearly to a non-degenerate alignment $\mathbf{S}^*$, it suffices to show that the iterates $\{\mathbf{S}^k\}_{k \geq 0}$ generated by the algorithm satisfy the following:\\
}
\revdel{
\noindent \textbf{(A1)}. \textit{(Sufficient Descent)} There exist $\kappa_0 > 0$ and $k_1 \in \mathbb{N}$ such that, the inequality $ F(\mathbf{S}^{k+1}) - F(\mathbf{S}^k) \leq - \kappa_0 \left\|\grad F(\mathbf{S}^k)\right\|_F \cdot \left\|\mathbf{S}^{k+1}-\mathbf{S}^k\right\|_F$ holds for all $k \geq k_1$.
    % \begin{equation}
    %     F(\mathbf{S}^{k+1}) - F(\mathbf{S}^k) \leq - \kappa_0 \left\|\grad F(\mathbf{S}^k)\right\|_F \cdot \left\|\mathbf{S}^{k+1}-\mathbf{S}^k\right\|_F
    % \end{equation}
\smallskip\\
}
\revdel{
\noindent \textbf{(A2)}. \textit{(Stationarity)} There exist $k_2 \in \mathbb{N}$ such that for all $k \geq k_2$, if $\left\|\grad F(\mathbf{S}^k)\right\|_F = 0$ then $\mathbf{S}^{k+1} = \mathbf{S}^k$. The sequence $\{\mathbf{S}^{k}\}_{k \geq 0}$ satisfies this trivially.
\smallskip\\
}
\revdel{
\noindent \textbf{(A3)}. \textit{(Safeguard)} There exist a constant $\mu > 0$ and $k_3 \in \mathbb{N}$ such that the inequality $\left\|\grad F(\mathbf{S}^k)\right\|_F \leq \mu \left\|\mathbf{S}^{k+1}-\mathbf{S}^k\right\|_F$ holds for all $k \geq k_3$.
    % \begin{equation}
    %     \left\|\grad F(\mathbf{S}^k)\right\|_F \leq \mu \left\|\mathbf{S}^{k+1}-\mathbf{S}^k\right\|_F.
    % \end{equation}
\smallskip\\
}
\revdel{
To prove \textbf{(A1)} and \textbf{(A3)}, we need
\begin{prop}{\citeb[Appendix E.2]{liu2019quadratic}}
\label{prop:liu_qr}
There exist $\phi, M > 0$ such that for all $\mathbf{S}_i \in \mathbb{O}(d)$ and $\boldsymbol{\xi}_i \in T_{\mathbf{S}_i}\mathbb{O}(d)$ satisfying $\left\|\boldsymbol{\xi}_i\right\|_F \leq \phi$, $\left\|\qf (\mathbf{S}_i + \boldsymbol{\xi}_i) - (\mathbf{S}_i + \boldsymbol{\xi}_i)\right\|_F \leq M \left\|\boldsymbol{\xi}_i\right\|_F^2$. In particular, $M = \sqrt{10}/4$ and $\phi = 1/2$ satisfy the above inequality.
\end{prop}
\begin{prop}
\label{prop:second_order_boundedness_of_RQR}
There exist $\phi, M > 0$ such that for all $\mathbf{S} \in \mathbb{O}(d)^{m}$ and $\boldsymbol{\xi} \in T_{\mathbf{S}}\mathbb{O}(d)^m$ satisfying $\left\|\boldsymbol{\xi}\right\|_F \leq \phi$, $\left\|R_{\QR }(\mathbf{S},\boldsymbol{\xi}) - (S + \boldsymbol{\xi})\right\|_F \leq M \left\|\boldsymbol{\xi}\right\|_F^2$.
\end{prop}
}

\revadd{Then using Theorem 2.3 in \citea{schneider2015convergence} and the fact that $\mathbb{O}(d)^m/_{\sim}$ is compact (thus every sequence on it has a cluster point), for the Algorithm~\ref{algo:rgd} to converge at least sublinearly to a non-degenerate alignment $\widetilde{\mathbf{S}}^*$, it suffices to show that the iterates $\{\widetilde{\mathbf{S}}^k\}_{k \geq 0}$ generated by the algorithm satisfy the following:}
\smallskip

\noindent \textbf{(A1)}. \revadd{\textit{(Sufficient Descent)} There exist $\kappa_0 > 0$ and $k_1 \in \mathbb{N}$ such that, the inequality $\widetilde{F}(\widetilde{\mathbf{S}}^{k+1}) - \widetilde{F}(\widetilde{\mathbf{S}}^k) \leq - \kappa_0 \left\|\grad \widetilde{F}(\widetilde{\mathbf{S}}^k)\right\|_F \cdot \left\|\widetilde{\mathbf{S}}^{k+1}-\widetilde{\mathbf{S}}^k\right\|_F$ holds for all $k \geq k_1$.}
    % \begin{equation}
    %     F(\mathbf{S}^{k+1}) - F(\mathbf{S}^k) \leq - \kappa_0 \left\|\grad F(\mathbf{S}^k)\right\|_F \cdot \left\|\mathbf{S}^{k+1}-\mathbf{S}^k\right\|_F
    % \end{equation}
\smallskip

\noindent \textbf{(A2)}. \revadd{\textit{(Stationarity)} There exist $k_2 \in \mathbb{N}$ such that for all $k \geq k_2$, if $\left\|\grad \widetilde{F}(\widetilde{\mathbf{S}}^k)\right\|_F = 0$ then $\widetilde{\mathbf{S}}^{k+1} = \widetilde{\mathbf{S}}^k$. The sequence $\{\widetilde{\mathbf{S}}^{k}\}_{k \geq 0}$ satisfies this trivially.}

\smallskip

\noindent \textbf{(A3)}. \revadd{\textit{(Safeguard)} There exist a constant $\mu > 0$ and $k_3 \in \mathbb{N}$ such that the inequality $\left\|\grad \widetilde{F}(\widetilde{\mathbf{S}}^k)\right\|_F \leq \mu \left\|\widetilde{\mathbf{S}}^{k+1}-\widetilde{\mathbf{S}}^k\right\|_F$ holds for all $k \geq k_3$.}
    % \begin{equation}
    %     \left\|\grad F(\mathbf{S}^k)\right\|_F \leq \mu \left\|\mathbf{S}^{k+1}-\mathbf{S}^k\right\|_F.
    % \end{equation}

\revadd{To prove \textbf{(A1)} and \textbf{(A3)}, we need
\begin{prop}
\label{prop:liu_pf}
For all $\mathbf{S}_i \in \mathbb{O}(d)$ and $\boldsymbol{\xi}_i \in T_{\mathbf{S}_i}\mathbb{O}(d)$ satisfying $\left\|\boldsymbol{\xi}_i\right\|_F \leq 1$, $\left\|\mathbf{S}_i\exp (\mathbf{S}_i^T\boldsymbol{\xi}_i) - (\mathbf{S}_i + \boldsymbol{\xi}_i)\right\|_F \leq (e-1)\left\|\boldsymbol{\xi}_i\right\|_F^2$.
\end{prop}
\begin{prop}
\label{prop:second_order_boundedness_of_RPF}
For $\mathbf{S} \in \mathbb{O}(d)^m$ and $\boldsymbol{\xi} \in T_{\mathbf{S}}\mathbb{O}(d)^m$ satisfying $\left\|\boldsymbol{\xi}\right\|_F \leq 1$, $\left\|R_\EXP(\mathbf{S}, \boldsymbol{\xi})(\mathbf{S}_1\exp (\mathbf{S}_1^T\boldsymbol{\xi}_1))^T - (\mathbf{S} + \boldsymbol{\xi})\right\|_F \leq 2\sqrt{m}\left\|\boldsymbol{\xi}\right\|_F^2$.
%For all $\mathbf{S}_i \in \mathbb{O}(d)$ and $\boldsymbol{\xi}_i \in T_{\mathbf{S}_i}\mathbb{O}(d)$, $i \in [1,m]$, satisfying $\left\|\boldsymbol{\xi}_i\right\|_F \leq 1/2$, $\left\|\EXP (\mathbf{S}_i + \boldsymbol{\xi}_i)\EXP (\mathbf{S}_1 + \boldsymbol{\xi}_1)^T - (\mathbf{S}_i + \boldsymbol{\xi}_i)\right\|_F \leq \left\|\boldsymbol{\xi}_i\right\|_F^2 + \left\|\boldsymbol{\xi}_1\right\|_F^2$.
\end{prop}
}\begin{prop}
\label{prop:alpha_grad}
$(\alpha_k)_{k \geq 0}$ and $(\mathbf{S}^k)_{k \geq 0}$ satisfy $\lim \alpha_k \left\|\grad F(\mathbf{S}^k)\right\|_F = 0$.
\end{prop}
\revdel{
\subsection{Local Linear Convegence of RGD}
\label{subsec:loc_lin_conv}
}
Now we extend the above result to the R-linear convergence of the sequence $(\widetilde{\mathbf{S}}^k)$ generated through Algorithm~\ref{algo:rgd} to a non-degenerate alignment. \revadd{Since $F(\mathbf{S}\mathbf{Q}) = F(\mathbf{S})$ for all $\mathbf{Q} \in \mathbb{O}(d)$, therefore every critical point of $F$ is degenerate and in particular $F$ is not a Morse-function \citea{cohen_iga_norbury_2006}. However, if $\mathbf{S}^*$ is a non-degenerate alignment then $\widetilde{\mathbf{S}}^* = \pi(\mathbf{S}^*)$ is a non-degenerate critical point of $\widetilde{F}$, as a result $\widetilde{F}$ is a Morse function at $\widetilde{\mathbf{S}}^*$.
%due to \citeb[Remark 6.6]{usevich2020approximate}. As a result, due to \citeb[Theorem 6.7 and 6.8]{usevich2020approximate} and equivalently
Consequently, due to \citeb[Proposition 4.2]{hu2018convergence}, the Lojasiewicz gradient inequality is satisfied with $\theta = 1/2$. Precisely,} \revdel{In turn, it suffices to show that $F$ is a Morse-Bott function at $\mathbf{S}^*$ \citeb[Section 6.2]{usevich2020approximate} \citeb[Definition 1.5]{feehan2021optimal}.}
\begin{prop}
\revadd{Let $\mathbf{S}^*$ be a non-degenerate alignment and define $\widetilde{\mathbf{S}}^* = \pi(\mathbf{S}^*)$. Then there exist $\delta, \eta > 0$ such that
\begin{equation}
    |\widetilde{F}(\widetilde{\mathbf{S}}) - \widetilde{F}(\widetilde{\mathbf{S}}^*)|^{1/2} \leq \eta \left\| \grad \widetilde{F}(\widetilde{\mathbf{S}})\right\|_F. \label{eq:Lojasiewicz_gradient_ineq_half}
\end{equation}
holds for every $\widetilde{\mathbf{S}} \in \mathbb{O}(d)^m/_{\sim}$ satisfying $\left\|\widetilde{\mathbf{S}}-\widetilde{\mathbf{S}}^*\right\|_F < \delta$.}
\end{prop}
\revdel{
\begin{rmk}
Since $F(\mathbf{S}\mathbf{Q}) = F(\mathbf{S})$ for all $\mathbf{Q} \in \mathbb{O}(d)$, therefore no critical point of $F$ is non-degenerate and in particular $F$ is not a Morse-function \citea{cohen_iga_norbury_2006}.
\end{rmk}
}
% \revadd{Despite the above remark, from Proposition~\ref{prop:hlift_frob_ineq}, \ref{prop:d_g_tilde} and \ref{prop:gradFS}, we obtain $d_{\widetilde{g}}(\widetilde{\mathbf{S}},\widetilde{\mathbf{S}}^*) < d_g(\mathbf{S}, \mathbf{S}^*)$ and $\left\|\grad F(\mathbf{S})\right\|_F \leq \left\| \grad \widetilde{F}(\widetilde{\mathbf{S}})\right\|_F \leq \sqrt{(m+1)}\left\|\grad F(\mathbf{S})\right\|_F$. Combining these with the above proposition and the fact that $F(\mathbf{S}) = \widetilde{F}(\widetilde{\mathbf{S}})$ for all $\mathbf{S} \in \pi^{-1}(\widetilde{\mathbf{S}})$, we obtain the Eq.~(\ref{eq:Lojasiewicz_gradient_ineq}) with $\theta=1/2$.}
\revadd{Then, due to \citeb[Theorem 2.3]{schneider2015convergence} and the fact that non-degenerate critical points are isolated due to Morse Lemma\citeb[Proposition 4.2]{hu2018convergence}, we have the following result,}
% \begin{cor}
% \revadd{Let $\mathbf{S}^*$ be a non-degenerate alignment. Then there exist $\delta, \eta > 0$ such that
% \begin{equation}
%     |F(\mathbf{S}) - F(\mathbf{S}^*)|^{1/2} \leq \eta \left\| \grad F(\mathbf{S})\right\|_F. \label{eq:Lojasiewicz_gradient_ineq}
% \end{equation}
% holds for every $\widetilde{\mathbf{S}} \in \mathbb{O}(d)^m/_{\sim}$ satisfying $\left\|\widetilde{\mathbf{S}}-\widetilde{\mathbf{S}}^*\right\|_F < \delta$.}
% \end{cor}
%\begin{prop}
%\label{prop:morse_bott_1}
\revdel{Let $\mathbf{S}^*$ be a non-degenerate alignment. Then $\widetilde{F}$ is Morse-Bott at $\pi(\mathbf{S}^*)$ and consequently $F$ is Morse-Bott at $\mathbf{S}^*$. Combining this with \citeb[Theorem 6.3]{usevich2020approximate} (equivalently \citeb[Theorem 2.3]{schneider2015convergence}) we obtain}
%\end{prop}
% Combining previous propositions with \citea{usevich2020approximate}[Proposition $6.8$, Theorem $6.7$], \citea{hu2018convergence}[proposition $4.1$, Proposition $4.2$], we have the following result
% \begin{thm}
% If $S^* \in \mathcal{C}$ satisfies the rigidity constraints then there exist $\delta, \eta > 0$ such that for every $S \in \mathbb{O}(d)^m$ with $\left\|S-S^*\right\|_F < \delta$,
% \begin{align}
%     |F(S) - F(S^*)| \leq \eta \left\|\grad F(S)\right\|_F^2.
% \end{align}
% \end{thm}
% \begin{cor}
% We conclude that $\theta = 1/2$ in Eq.~(\ref{eq:Lojasiewicz_gradient_ineq}). Then invoking the convergence theorem in \citea{schneider2015convergence}[Theorem 2.3], we conclude that the sequence $\{\mathbf{S}^k\}_{k \geq 0}$ generated by Algorithm~\ref{algo:rgd} converges linearly to a critical point in $\mathcal{C}$.
% \end{cor}
\revdel{As a consequence of the above proposition, we have the following result}
\begin{thm}
\label{thm:rgd_conv}
Let $\mathbf{S}^*$ be a non-degenerate alignment and define $\widetilde{\mathbf{S}}^* = \pi(\mathbf{S}^*)$. Then there exist $\delta > 0$ such that RGD converges to $\widetilde{\mathbf{S}}^*$ \revadd{R-}linearly when initialized with $\widetilde{\mathbf{S}}^0$ such that $\left\|\widetilde{\mathbf{S}}^0-\widetilde{\mathbf{S}}^*\right\|_F < \delta$.
\end{thm}
\begin{rmk}
\revadd{Here the radius of convergence $\delta$ depends on the size of the neighborhood on which the Morse Lemma is applicable. In fact, due to Proposition~\ref{prop:HessFSZ}, Proposition~\ref{prop:HessVicinity}, Corollary~\ref{cor:HessVicinity} and $\min_{\mathbf{Q}\in\mathbb{O}(d)}\left\|\mathbf{O}-\mathbf{S}\mathbf{Q}\right\|_F \leq \left\|\pi(\mathbf{O})-\pi(\mathbf{S})\right\|_F$,
%the choice of $\mathbf{S}^{k} = [\mathbf{I}_d; \widetilde{\mathbf{S}}^{k}]$ (consequently, $\left\|\mathbf{S}^{k+1}-\mathbf{S}^{k}\right\|_F = \left\|\widetilde{\mathbf{S}}^{k+1}-\widetilde{\mathbf{S}}^{k}\right\|_F$),
the radius $\delta$ is given by
\begin{align}
    \delta = \begin{cases}(2(\delta_1 + 2 \delta_{3}))^{-1}|\lambda|, & \text{if } \mathbf{S}^* \text{ is a perfect alignment}\\
    (2(\delta_1 + \delta_2(\mathbf{S}^*) + 2 \delta_{3}(\mathbf{S}^*))))^{-1}|\lambda|, & \text{otherwise.}\end{cases}
\end{align}
}
\end{rmk}

Combining the above with Theorem~\ref{thm:non_deg_loc_min}, Corollary~\ref{cor:suff_cond_views_non_deg} and Theorem~\ref{thm:loc_rigid}, we obtain the following corollaries.
\begin{cor}
If $\mathbf{S}^*$ is an alignment such that $\mathbf{S}^* \in \mathcal{C}$, $\mathbb{L}(\mathbf{S}^*)$ is negative semi-definite and of rank $(m-1)d(d-1)/2$, then RGD converges locally linearly to $\widetilde{\mathbf{S}}^* = \pi(\mathbf{S}^*)$.
\end{cor}

\revadd{Exact recovery:}
\revadd{\citea{chaudhury2015global} showed that under the affine rigidity condition, the algorithms based on spectral and semidefinite relaxation exactly recovers the perfect alignment. While \citea{ling2021generalized} showed exact recovery due to generalized power method in a special setting which in fact exhibits affine rigidity where each point is represented in every local view i.e. the bipartite graph $\Gamma$ is complete. Here we advocate local linear convergence of RGD to a perfect alignment under affine rigidity as well as under weaker conditions of global and local rigidity. Precisely,}
\revadd{
\begin{cor}
Suppose $\mathbf{S}^*$ is a perfect alignment. Then RGD converges locally linearly to $\widetilde{\mathbf{S}}^* = \pi(\mathbf{S}^*)$ if any of the following conditions hold:
\begin{enumerate}
    \item $\Theta(\mathbf{S}^*)$ is affinely rigid, equivalently $\mathbf{C}$ is of rank $(m-1)d$.
    \item $\Theta(\mathbf{S}^*)$ is globally rigid, equivalently $\mathbf{S}^*$ is a unique perfect alignment.
    \item $\Theta(\mathbf{S}^*)$ is locally rigid, equivalently $\mathbb{L}(\mathbf{S}^*)$ is of rank $(m-1)d(d-1)/2$.
\end{enumerate}
Note that $1 \implies 2 \implies 3$.
\end{cor}
}
% \begin{cor}
% \label{cor:G_star_1_conv}
% If $\mathbf{S}^*$ is a perfect alignment such that $|\mathbb{G}^*(\mathbf{S})| = 1$ (see Theorem~\ref{thm:G_star_1}), then RGD converges locally linearly to $\mathbf{S}^*$.
% \end{cor}
\revadd{Finally, we state a sufficient condition on the structure of the noiseless local views that enables local linear convergence of RGD. This condition is stronger than the local rigidity condition and weaker than the global rigidity condition in the above corollary.}
\begin{cor}
\label{cor:G_conv}
If the local views are noiseless and $\mathbb{G}$ is connected then RGD converges locally linearly to a perfect alignment.
\end{cor}
\revdel{
\begin{cor}
If each local view is affinely non-degenerate (see Remark~\ref{rmk:non_deg_views}) and $\mathbf{S}^*$ is a perfect alignment such that the realization $\Theta(\mathbf{S}^*)$ is locally rigid then RGD converges locally linearly to $\mathbf{S}^*$.
\end{cor}
}
% \input{saddle.tex}
% \section{Conclusion}\label{sec:conclusion}
In this work, we focus on addressing the fundamental challenge of OOD detection tasks, which is how to fully understand the semantic discrepancy between the ID/OOD samples. We reveal that the key to success in the realistic SCOOD task is to allocate as many ID samples in the unlabeled set correctly as possible. To this end, we propose a novel uncertainty-aware optimal transport scheme that introduces class-specific energy scores as guidance for effective label assignment. Experimental results show that our method achieves better performance than previous state-of-the-art methods on SCOOD benchmarks.

\textbf{Limitations.} In addition to temperature scaling, other techniques such as feature clipping applied in ReAct~\cite{sun2021react} also enhance the performance of energy score, so how to obtain an OOD score that best fits the SCOOD task can be further explored. Moreover, a setting highly related to SCOOD has been proposed in \cite{katz2022training} and formulated as a constrained optimization problem. We will also theoretically analyze these practical OOD settings in our feature work.

% \section*{Acknowledgments}
\textbf{Acknowledgments.} 
This work is supported by National Key R\&D Program of China under Grant 2020AAA0105701, National Natural Science Foundation of China (NSFC) under Grants 61872327, Major Special Science and Technology Project of Anhui, National Natural Science Foundation of China (62033012) and Ant Group through Ant Research Intern Program.

\section{Introduction}
\label{sec:introduction}
\setcounter{equation}{0}

Solving many-body quantum problems at a manageable cost is a central aspect in many fields of physics and chemistry.
A fundamental difficulty is the complexity of the problem grows exponentially fast with respect to the number of particles in the system.
An efficient numerical algorithm is the key to reducing the computational cost and promoting accuracy.
%
The variational Monte Carlo (VMC) methods in particular can produce highly accurate predictions provided that (i) a sufficiently flexible trial wavefunction is available and (ii) the variational parameters of this wavefunction can be optimized.

A rapidly developing approach for trial wavefunctions is based on machine learning techniques.
These techniques typically rely on the ability of artificial neural networks (ANNs) to represent complex high-dimensional functions, which have already been explored in many fields of physics and chemistry.
The restricted Boltzmann machine (RBM) is first proposed  by Carleo and Troyer \cite{carleo17} as a variational ansatz for many-body quantum problems. Furthermore, a large number of deep neural networks, such as feed forward neural networks \cite{saito17,cai18}, deep RBMs \cite{deng17,glasser18,nomura17,kaubruegger2018}, convolutional neural networks \cite{liang2018,choo2018}, variational autoencoders \cite{rocchetto18}, have been applied to capture the physical features and improve the accuracy of the ground state. Motivated by the traditional Slater-Jastrow-backflow ansatzes, PauliNet \cite{hermann2020deep} and FermiNet \cite{pfau2020ab} use the permutation equivariant and invariant construction of networks and many determinants to approximate a general antisymmetric wavefunction. Both of them achieve a great success in electronic structure calculations. With an efficient ansatz that has strong ability to represent high-dimensional quantum states, the VMC method provides the route to obtain the best possible solution to the problem based on the variational principle.
%
% Particularly, ANNs have proven to be a flexible tool to approximate quantum many-body states that allow people to consider a wide range of different quantum problems.
% In a series of recent works \cite{cai18,carleo17,deng17,glasser18,kaubruegger18,nomura17,saito17,rocchetto18}, deep neural networks have been developed to tackle {\it ab initio} problems within VMC, showing accuracy improvements over traditional methods used to describe correlated systems. 
% With an efficient ansatz that has good ability to represent high-dimensional quantum states, the VMC method provides the route to obtain the best possible solution to the problem based on the variational principle.
% %
% By performing an optimization of the parameters, we can reach the lowest energy state and capture the correct ground state behavior. 

A main issue in optimizing parameterized quantum states is that the total scale of configurations grows exponentially with the system size. It is prohibitive to perform direct sampling for exceptionally large configuration spaces. 
However, the VMC approach recasts the variational eigenvalue problem into a stochastic optimization problem that is solvable under the curse of dimensionality. To compute the expectation, Markov chain Monte Carlo (MCMC) is employed to sample from the unnormalized probability. In contrast to the ordinary unbiased Monte Carlo methods, the MCMC method requires some time for mixing and produces the desired samples biasedly and non-independently. There are several common MCMC algorithms, such as  Gibbs sampling, Metropolis-Hastings (MH) algorithm, Hamiltonian Monte Carlo (HMC) etc. The efficiency of the VMC algorithm relies on the error of the stochastic sampling, which has not been well investigated in the VMC literatures.

%\hc{[add the literature review for analysis of (1) stochastic optimization and (2) MCMC sampling?]}
% VMC method !
The VMC framework is solved by stochastic optimization algorithms. With the MCMC samples, we minimize the objective function using the stochastic gradient descent (SGD) method. The vanilla SGD method is to sample independently from a uniform distribution on the finite sum, and has been extensively studied. Moulines \& Bach \cite{moulines2011non} first show linear convergence of SGD non-asymptotically for strongly convex functions. Needell et al. \cite{needell2014stochastic} improve these results by removing the quadratic dependency on the condition number in the iteration complexity results. Among these convergence results, the gradient noise assumptions for i.i.d samples are of vital importance to establish an $O(1/\sqrt{nK})$ convergence rate for general non-convex cost functions, where $n$ is the sample size per iteration and $K$ is the total number of iterations. However, stochastic sampling is not always independent or unbiased.
The convergence of SGD with Markovian gradients has been studied in \cite{duchi2012ergodic, sun2018markov}, and SGD with biased gradient estimators is considered in \cite{ajalloeian2020convergence}. But the VMC method, which produces MCMC samples from varying distributions, has been seldomly considered. More practical settings of many-body quantum systems make it harder to deal with theoretical properties in sampling and optimization.

The goal of this paper is to provide a rigorous analysis of the VMC method from an aspect of stochastic optimization, which has not been studied in pervious literatures according to our limited knowledge. %The VMC method, whose gradient estimator is biased and non-independent, cannot be covered by the vanilla SGD framework. Our assumptions on the ansatzes and Markov chains are standard or verifiable for the VMC method in practice. 
MCMC constructs a Markov chain to sample quantum configurations from an unnormalized distribution. Since it reduces the computational cost of sampling, the biasness and dependency  lead to a distinct analysis of the vanilla SGD method. 
Moreover, the local energy may be unbound due to the unbound operator and untrained trial wave functions. In our settings, the local energy is assumed sub-exponential and differentiable with respect to parameters. 
By a uniform spectral gap of Markov operators, we give a Bernstein inequality for Markov chains
and  present non-asymptotic error bounds of the MCMC estimator with unbound functions.
We establish the first-order convergence of VMC depending on a few related factors, such as step sizes, sample sizes and the length of burn-in time. 
It may help us to set suitable hyperparameters and promote algorithm efficiency in practice.  
Due to the particularity of eigenvalue problems, we further discuss how VMC escapes from saddle points. A positive lower bound of MCMC variance is constructed to satisfy a correlated negative curvature condition. It is shown that the intrinsic noise contributes to make a slight perturbation to get rid of saddle points. We make a convergence analysis of VMC escaping from saddle points. The VMC method returns  $(\epsilon,\epsilon^{1/4})$ approximate second order stationary points or $\epsilon^{1/2}$-variance points after at least $O(\epsilon^{-11/2}\log^{2}(1/\epsilon) )$ steps in high probability. Our analysis explains how VMC escapes from saddle points and why it may converge to other excited states. 

%
%\hc{[shall we show a sketch of the main contribution and techniques of this paper?]}

The rest of this paper is organized as follows. In Section \ref{sec:model}, we describe the many-body quantum system and the variational optimization problem to obtain the ground state energy. In Section \ref{sec:vmc}, the MH algorithm and the VMC method are introduced. In Section \ref{sec:conv}, we first give our assumptions for the Hamiltonian and the ansatz. Then, the sampling error is analyzed asymptotically by the concentration inequality for Markov chains. We prove that the VMC method converges to stationary points and estimate the convergence rates. In Section \ref{sec:saddle}, we provide the convergence guarantee to avoid saddle points with high probability by the stochastic error of MCMC.


\section{Many-body quantum problems}
\label{sec:model}
\setcounter{equation}{0}
Consider a many-body system with $N$ particles.
We denote the $N$-particle configuration by $\bmx:=(x_1,\cdots,x_N)\in \Xcal := \Acal^N$, with $\Acal$ being the one-particle configuration space, which can be continuous or discrete. The wavefunction $\Psi:\Xcal\rightarrow\Cbb~({\rm or}~\Rbb)$ describes the quantum state of the many-body system, and is often required to satisfy some symmetric/anti-symmetric conditions.
We denote the Hilbert space for the wavefunction by $\Hscr$. The Hamiltonian $\Hcal$ is a self-adjoint operator on the Hilbert space $\Hscr$, which determines the dynamics and the ground state of the quantum system.
$\Hcal$ can always be written as the sum of local operators acting only on one or two coordinates.

\vskip 0.2cm

% The total energy of the quantum system can be written as the Rayleigh-quotient 
% \begin{eqnarray}
% \label{eq:energy}
% \Lcal(\Psi) = \frac{\lra{\Psi,\Hcal\Psi}}{\lra{\Psi,\Psi}} .
% \end{eqnarray}
% The central task of this work is to find the ground state of the system, which can be obtained by minimizing the energy \eqref{eq:energy}
% \begin{eqnarray}
% \label{pb:min}
% \min_{\Psi\in\Hscr} \Lcal(\Psi)
% \end{eqnarray}
% or solving the corresponding eigenvalue problem $\Hcal\Psi = E_{0}\Psi$ for the lowest lying eigenvalue $E_{0}$.
Our goal is to compute the ground state energy and wavefunction of the system, which corresponds to the lowest eigenvalue $E_0$ and its corresponding eigenfunction of 
\begin{eqnarray}
\label{eigen}
\Hcal\Psi_0 = E_0\Psi_0 .
\end{eqnarray}
%
To obtain the ground state solution,  one can either solve the eigenvalue problem \eqref{eigen} directly, or alternatively minimize the following Rayleigh quotient
\begin{align}
\label{min:Rayleigh}
E_0 = \min\limits_{\Psi\in \Hscr,~\Psi\neq\bm{0}} \frac{\left \langle\Psi,\Hcal\Psi\right \rangle}{\left \langle\Psi,\Psi\right \rangle} ,
\end{align}
where the bracket means
\begin{align}
\label{sum_sigma}
\left \langle\Psi,\Hcal\Psi\right \rangle = \int_{\bmx\in \Xcal} \Psi^{*}(\bmx)\cdot \big(\Hcal\Psi\big)(\bmx)d\bmx
,\quad
\left \langle\Psi,\Psi\right \rangle = \int_{\bmx\in \Xcal} \big|\Psi(\bmx)\big|^2 d\bmx .
\end{align}
When $\Xcal$ is a discrete configuration space (see Example 1), the above integrals in \eqref{sum_sigma} is regarded as summations over all configurations.

We will then give two examples of the most popular many-body problems in the following. 

\vskip 0.2cm

{\bf Example 1.} (Heisenberg model)
%
The Heisenberg model is a spin model with $\Acal=\Zbb_2:=\{-1,1\}$.
The $N$-body spin configuration is denoted by $\bmx=(x_1,\cdots,x_N)$ with $x_i\in\Zbb_2$.
The corresponding $N$-body wavefunction 
$\Psi:\big(\Zbb_2\big)^N\rightarrow \Cbb$ belongs to the discrete space $\Hscr=\big(\Cbb^2\big)^{\otimes N}$. The Hamiltonian of the spin system is more conveniently expressed as in the linear space $\big(\Cbb^2\big)^{\otimes N}$ as %$H:\big(\Cbb^2\big)^{\otimes N} \rightarrow \big(\Cbb^2\big)^{\otimes N}$ with
\begin{align}
\label{ham:tensor}
\Hcal=\sum_{(j,k)\in G}\bigg( J_x\cdot H_{jk}^{x} + J_y\cdot H_{jk}^{y} + J_z\cdot H_{jk}^{z} \bigg) ~.
\end{align}
In \eqref{ham:tensor}, $G$ denotes the given interacting pairs within the $N$ particles, $J_x,J_y,J_z\in\Rbb$ are the coupling coefficients and the local operator are defined by
% the $x,y,z$ components represents different physical interactions,
\begin{align}
\label{ham:xyz}
& H_{jk}^{\alpha} := I_2^{\otimes j-1}\otimes \sigma^\alpha \otimes I_2^{\otimes k-j-1} \otimes \sigma^\alpha \otimes I_2^{\otimes N-k-1} 
\quad\quad {\rm for} ~~ \alpha=x,y,z,
\end{align}
with $\otimes$ denoting the kronecker product, $I_2 \in \Cbb^{2\times 2}$ being an identity matrix and $\sigma^x,~\sigma^y,~\sigma^z\in \Cbb^{2\times 2}$ representing the Pauli matrices
\begin{eqnarray}
\label{pauliM}
	\sigma^x=
	\begin{pmatrix}
	0 & 1 \\
	1 & 0
	\end{pmatrix}
	, \qquad
	\sigma^y=
	\begin{pmatrix}
	0 & -i \\
	i & 0
	\end{pmatrix}
	,\qquad
	\sigma^z=
	\begin{pmatrix}
	1 & 0 \\
	0 & -1
	\end{pmatrix}. 
\end{eqnarray} 



% We can further give a more explicit form of the Hamiltonian computation for this Heisenberg model.
% For a given wavefunction $\Psi$, we only concern the value of $\big(\mathcal{H}\Psi\big)(\bmx)$ at the given spin configuration $\bmx=(x_1,\cdots,x_N)$.
% %
% Let $\delta_{jk}:\big(\Zbb_2\big)^N\rightarrow\Rbb$ be functions defined by
% \begin{align*}
% 	\delta_{jk} (\bmx) = \left\{
% \begin{array}{ll}
% 1 & {\rm if}~x_j=x_k
% \\[1ex]
% -1 & ~{\rm if}~x_j\neq x_k
% \end{array} .
% \right.
% \end{align*}
% Let $\tau_j:\big(\Zbb_2\big)^N\rightarrow\big(\Zbb_2\big)^N$ be flipping operators, defined by
% \begin{align*}
% 	\tau_j\bmx= \big(x_1,\cdots,x_{j-1},\bar{x}_j,x_{j+1},\cdots,x_N\big) 
% \end{align*}
% with $\bar{x} = 1$ if $x=-1$ and $\bar{x} = -1$ if $x=1$.
% %
% We observe from the definitions \eqref{ham:xyz} that
% % and the relation between tensor and wavefunction formulas 
% \begin{align*}
% & \big(\Hcal_{jk}^{x}\Psi\big)(\bmx) ~=~  \Psi( \tau_k\tau_j\bmx) ,
% \\[1ex]
% & \big(\Hcal_{jk}^{y}\Psi\big)(\bmx) ~=~ - \delta_{jk}(\bmx) \cdot \Psi( \tau_k\tau_j\bmx) ,
% \\[1ex]
% & \big(\Hcal_{jk}^{z}\Psi\big)(\bmx) ~=~  \delta_{jk}(\bmx) \cdot \Psi(\bmx) ,
% \end{align*}
% where $\Hcal_{jk}^{\alpha}$ corresponds to $H_{jk}^{\alpha}$ in \eqref{ham:xyz}.
% %
% Then we have the following total Hamiltonian $\Hcal$ as
% \begin{multline}
% \label{ham:Hphi}
% \quad
% \big(\Hcal\Psi\big)(\bmx) ~=~  \sum_{(j,k)\in G} 
% \Big( J_x\cdot \Psi( \tau_k\tau_j\bmx) 
% \\[1ex]
% - J_y\cdot \delta_{jk}(\bmx) \cdot  \Psi( \tau_k\tau_j\bmx)
% + J_z\cdot \delta_{jk}(\bmx) \cdot  \Psi(\bmx) \Big),
% \qquad\forall~\bmx\in\big(\Zbb_2\big)^N.
% \qquad
% \end{multline}

\vskip 0.2cm

{\bf Example 2.} (Schr\"{o}dinger equation)
%
For a many-electron system in 3 dimension, $\Acal=\Rbb^3\times \Zbb_2$, and the $N$-electron configuration is $\bmx=(x_1,\cdots,x_N)$ with $x_i=(r_i,\sigma_i)\in \Acal$.
Here $r_i$ represents the spatial coordinate and $\sigma_i$ is the spin coordinate.
%
Noting that the $N$-electron wavefunction is required to be anti-symmetric, we have the space
\begin{eqnarray*}
\Hscr = \bigwedge_{i=1}^N L^2(\Rbb^3\times\Zbb_2,\Cbb) ,
\end{eqnarray*}
where the symbol $\bigwedge$ means the usual tensorial product $\otimes$ with the additional requirement that one  keeps only the permutational anti-symmetrized products. 

The Hamiltonian of the electron system is given by
\begin{eqnarray}
\label{hamiltonian:SE}
\Hcal = -\frac{1}{2}\sum_{i=1}^N\Delta_{r_i} + \sum_{i=1}^{N} v_\mathrm{ext}(r_i) + \sum_{1\leq i<j\leq N}v_\mathrm{ee}(r_i,r_j),
\end{eqnarray}
where $v_\mathrm{ext}:\Rbb^3\rightarrow\Rbb$ is the ionic potential and $v_\mathrm{ee}:\Rbb^3\times \Rbb^3\rightarrow\Rbb$ represents the interaction between electrons
\begin{equation}
	\begin{aligned}
		v_\mathrm{ext}(r)=-\sum_{I}\frac{Z_I}{|r-R_I|},~~ v_\mathrm{ee}(r,r^{\prime})=\frac{1}{|r-r^{\prime}|},\quad r,r^{\prime}\in \Rbb^3,
	\end{aligned}
\end{equation}
with $R_I\in \Rbb^3$ and $Z_I\in \Nbb $ being the position and atomic number of the $I$-th nuclear.


\section{Variational Monte Carlo}
\label{sec:vmc}

\subsection{Problem formulation}
% By using this matrix-vector representation, we see that the Rayleigh quotient can be exactly rewritten as
% \begin{align}
% \frac{\left \langle\Psi|\Hcal|\Psi\right \rangle}{\left \langle\Psi|\Psi\right \rangle} 
% = \frac{\Psi^{\rm T}H\Psi}{\Psi^{\rm T}\Psi}
% = \frac{\sum_{j=1}^{2^N}\Psi_j \cdot (H\Psi)_j}{\sum_{j=1}^{2^N}|\Psi_j|^2} ,
% \end{align}
% where the summation over spin configurations $\{(\sigma_1,\cdots,\sigma_N)\}$ is nothing but the summation over the vector entry indices, say, $\displaystyle \sum_{\sigma_1,\cdots,\sigma_N\in\Z_2} \rightarrow \sum_{j=1}^{2^N}$~.

 The VMC method is well known in electronic structure calculations \cite{foulkes01} by evaluating the Rayleigh quotient \eqref{min:Rayleigh}. However, the complexity of many-body quantum systems grows exponentially with respect to $N$. To search for the optimum in an infinite dimensional function space, VMC approximates the wavefunction by a suitable ansatz $\Psi_{\parameter},~ \parameter\in\Rbb^d$ within a finite parameter space. There are plenty of common variational ansatzes in VMC, including Jastrow ansatz, restricted Boltzmann machine (RBM), feed forward neural networks (FFNN) and a variety of other neural networks. We solve the optimization problem:
\begin{equation}
	\min\limits_{\parameter}\Lcal(\parameter)=\frac{\left \langle\Psi_{\parameter},\Hcal\Psi_{\parameter}\right \rangle}{\left \langle\Psi_{\parameter},\Psi_{\parameter}\right \rangle}=\frac{\int_{\bmx\in \Xcal} \Psi_{\parameter}^{*}(\bmx)\cdot \big(\Hcal\Psi_{\parameter}\big)(\bmx)d\bmx}{\int_{\bmx\in \Xcal} \Psi_{\parameter}^{*}(\bmx)\cdot\Psi_{\parameter}(\bmx) d\bmx}.
\end{equation}
In the VMC framework, the Rayleigh quotient in \eqref{min:Rayleigh} can be rewritten in terms of a statistical expectation:
\begin{equation}
	\label{loss:vmc}
	\begin{aligned}
		\Lcal(\parameter)
		&= \int_{\bmx\in\Xcal}\underbrace{\frac{\Psi_{\parameter}^*(\bmx)\cdot\Psi_{\parameter}(\bmx)}{\int_{\bmx\in\Xcal}\Psi_{\parameter}^*(\bmx)\cdot\Psi_{\parameter}(\bmx)d\bmx}}_{\pi_{\parameter}(\bmx)}\cdot\underbrace{\frac{\Hcal\Psi_{\parameter}(\bmx)}{\Psi_{\parameter}(\bmx)}}_{E_{\parameter}(\bmx)}d\bmx
		\\
		&=\mathbb{E}_{\bmx\sim\pi_{\parameter}}[E_{\parameter}(\bmx)] ,
		\end{aligned}
\end{equation}
% \[
% 	\frac{\int_{\bmx\in \Xcal} \Psi_{\parameter}^{*}(\bmx)\cdot \big(\Hcal\Psi_{\parameter}\big)(\bmx)d\bmx}{\int_{\bmx\in \Xcal} \Psi_{\parameter}^{*}(\bmx)\cdot\Psi_{\parameter}(\bmx) d\bmx}=\int_{\bmx\in\Xcal}\underbrace{\frac{\Psi_{\parameter}^*(\bmx)\cdot\Psi_{\parameter}(\bmx)}{\int_{\bmx\in\Xcal}\Psi_{\parameter}^*(\bmx)\cdot\Psi_{\parameter}(\bmx)d\bmx}}_{\pi_{\parameter}(\bmx)}\cdot\underbrace{\frac{\Hcal\Psi_{\parameter}(\bmx)}{\Psi_{\parameter}(\bmx)}}_{E_{\parameter}(\bmx)}d\bmx =\mathbb{E}_{\bmx\sim\pi_{\parameter}}[E_{\parameter}(\bmx)]
% \]
where $\pi_{\parameter}(\bmx)$ and $E_{\parameter}(\bmx)$ represent the probability and the local energy at configuration $\bmx$.
%
% 

We derive the gradient and the Hessian of \eqref{loss:vmc} in the real settings for convenience. A similar statement holds true for the complex settings, according to Appendix E in \cite{lin2021explicitly}.
\begin{theorem}
	Let the map 
	$\parameter\rightarrow \Psi_{\parameter}$ from $\Rbb^{d}$ to $ L^2(\Xcal;\Rbb)$ be smooth. The gradient of $\Lcal(\parameter)$ has the following form
	\begin{equation}
		\label{eq:grad}
		g(\parameter):=\nabla_{\parameter}\Lcal(\parameter)=2\Ebb_{ \pi_{\parameter}}\lb\oE_{\parameter}(\bmx)\nabla_{\parameter}\log \Psi_{\parameter}(\bmx)\rb,
	\end{equation}
	with $\oE_{\parameter}(\bmx)=E_{\parameter}(\bmx)-\Ebb_{\bmx\sim\pi_{\parameter}}\left[E_{\parameter}(\bmx)\right]$. The Hessian of $\Lcal(\parameter)$ takes the form of 
	\begin{equation}
		\label{eq:Hess}
		\begin{aligned}
			&H(\parameter):=\nabla_{\parameter}^2\Lcal(\parameter)
		=H_1(\parameter)+H_2(\parameter)+H_3(\parameter)+H_{4}(\parameter)+H_{4}(\parameter)^{\T},
		\end{aligned}
	\end{equation}
	where 
	\begin{equation*}
		\begin{aligned}
		H_1(\parameter)&=2\Ebb_{\pi_{\parameter}}\lrb{\nabla_{\parameter}E_{\parameter}\nabla_{\parameter}\log \Psi_{\parameter}^{\T}},~~
		H_2(\parameter)=4\Ebb_{\pi_{\parameter}}\lrb{\oE_{\parameter}\nabla_{\parameter}\log \Psi_{\parameter}\nabla_{\parameter}\log \Psi_{\parameter}^{\T}},\\
		H_3(\parameter)&=2\Ebb_{\pi_{\parameter}}\lrb{\oE_{\parameter}\nabla^{2}_{\parameter}\log \Psi_{\parameter}},~~
		H_{4}(\parameter)=-4\Ebb_{\pi_{\parameter}}\lrb{\oE_{\parameter}\nabla_{\parameter}\log \Psi_{\parameter}}\Ebb_{\pi_{\parameter}}\lrb{\nabla_{\parameter}\log \Psi_{\parameter}^{\T}}.
		\end{aligned}
	\end{equation*}
	
\end{theorem}
\begin{proof}
	Using the definition of $E_{\parameter}$ and $\pi_{\parameter}$ in \eqref{loss:vmc}, we have
	\begin{equation*}
		\begin{aligned}
			g(\parameter)
					=&\frac{\int_{\bmx\in \Xcal} 2\nabla_{\parameter}\Psi_{\parameter}(\bmx)\cdot \big(\Hcal\Psi_{\parameter}\big)(\bmx)d\bmx}{\int_{\bmx\in \Xcal} \big|\Psi_{\parameter}(\bmx)\big|^2 d\bmx}
					-\frac{\int_{\bmx\in \Xcal} \Psi_{\parameter}(\bmx)\cdot \big(\Hcal\Psi_{\parameter}\big)(\bmx)d\bmx}{\int_{\bmx\in \Xcal} \big|\Psi_{\parameter}(\bmx)\big|^2 d\bmx} \frac{\int_{\bmx\in \Xcal} 2\Psi_{\parameter}(\bmx)\cdot\nabla_{\parameter}\Psi_{\parameter}(\bmx) d\bmx}{\int_{\bmx\in \Xcal} \big|\Psi_{\parameter}(\bmx)\big|^2 d\bmx}\\
					=&2\int_{\bmx\in \Xcal} \pi_{\parameter}(\bmx)E_{\parameter}(\bmx)\nabla_{\parameter}\log\Psi_{\parameter}(\bmx) d\bmx -2\int_{\bmx\in\Xcal}\pi_{\parameter}(\bmx)E_{\parameter}(\bmx) d\bmx \int_{\bmx\in \Xcal} \pi_{\parameter}(\bmx)\nabla_{\parameter}\log\Psi_{\parameter}(\bmx) d\bmx \\
					=&2\Ebb_{ \pi_{\parameter}}\lb \oE_{\parameter}(\bmx)\nabla_{\parameter}\log \Psi_{\parameter}(\bmx)\rb.
		\end{aligned}
	\end{equation*}
	Moreover, by differentiating the gradient \eqref{eq:grad}, the Hessian is calculated as follows:
	\begin{equation*}
		\begin{aligned}
			H(\parameter)=&2\nabla_{\parameter}\Ebb_{ \pi_{\parameter}}\lb \oE_{\parameter}(\bmx)\nabla_{\parameter}\log \Psi_{\parameter}(\bmx)\rb\\
			=&2\Ebb_{ \pi_{\parameter}}\lb \oE_{\parameter}(\bmx)\nabla_{\parameter}\log \Psi_{\parameter}(\bmx)\nabla_{\parameter}\log \pi_{\parameter}(\bmx)^{\T}\rb+2\Ebb_{ \pi_{\parameter}}\lb \nabla_{\parameter}\log \Psi_{\parameter}(\bmx)\nabla_{\parameter}\oE_{\parameter}(\bmx)^{\T}\rb\\
			&+2\Ebb_{ \pi_{\parameter}}\lb \oE_{\parameter}(\bmx)\nabla^2_{\parameter}\log \Psi_{\parameter}(\bmx)\rb\\
			=&4\Ebb_{\pi_{\parameter}}\lrb{\oE_{\parameter}\nabla_{\parameter}\log \Psi_{\parameter}\nabla_{\parameter}\log \Psi_{\parameter}^{\T}}-4\Ebb_{\pi_{\parameter}}\lrb{\oE_{\parameter}\nabla_{\parameter}\log \Psi_{\parameter}}\Ebb_{\pi_{\parameter}}\lrb{\nabla_{\parameter}\log \Psi_{\parameter}^{\T}}\\
			&+2\Ebb_{ \pi_{\parameter}}\lb \nabla_{\parameter}E_{\parameter} \nabla_{\parameter}\log \Psi_{\parameter}^{\T}\rb-4\Ebb_{\pi_{\parameter}}\lrb{\nabla_{\parameter}\log \Psi_{\parameter}}\Ebb_{\pi_{\parameter}}\lrb{\oE_{\parameter}\nabla_{\parameter}\log \Psi_{\parameter}^{\T}}+2\Ebb_{ \pi_{\parameter}}\lb \oE_{\parameter}\nabla^2_{\parameter}\log \Psi_{\parameter}\rb.
		\end{aligned}
	\end{equation*}
	This completes the derivation.
\end{proof}
% VMC operates an optimization algorithm to update the parameters $\parameter$, by which the energy approaches minimization within the parameter space. With the complex wavefunction, the gradient of \eqref{loss:vmc} can be obtained by the following formula. Due to the essential self-adjointness of the Hamiltonian $\Hcal$ in the Hilbert space, it holds that
% \begin{equation}
% 	\begin{aligned}
% 		\label{eq:grad}
% 		\partial_{\parameter}\Lcal(\parameter)=&\partial_{\parameter}\frac{\int_{\bmx\in \Xcal} \Psi_{\parameter}^{*}(\bmx)\cdot \big(\Hcal\Psi_{\parameter}\big)(\bmx)d\bmx}{\int_{\bmx\in \Xcal} \Psi_{\parameter}^{*}(\bmx)\cdot\Psi_{\parameter}(\bmx) d\bmx}\\
% 		=&\frac{\int_{\bmx\in \Xcal} \partial_{\parameter}\Psi_{\parameter}^{*}(\bmx)\cdot \big(\Hcal\Psi_{\parameter}\big)(\bmx)d\bmx+\int_{\bmx\in \Xcal} \Psi_{\parameter}^{*}(\bmx)\cdot \big(\Hcal\partial_{\parameter}\Psi_{\parameter}\big)(\bmx)d\bmx}{\int_{\bmx\in \Xcal} \Psi_{\parameter}^{*}(\bmx)\cdot\Psi_{\parameter}(\bmx) d\bmx}\\
% 		&-\frac{\int_{\bmx\in \Xcal} \Psi_{\parameter}^{*}(\bmx)\cdot \big(\Hcal\Psi_{\parameter}\big)(\bmx)d\bmx}{\int_{\bmx\in \Xcal} \Psi_{\parameter}^{*}(\bmx)\cdot\Psi_{\parameter}(\bmx) d\bmx} \frac{\int_{\bmx\in \Xcal} \partial_{\parameter}\Psi_{\parameter}^{*}(\bmx)\cdot\Psi_{\parameter}(\bmx) d\bmx+\int_{\bmx\in \Xcal} \Psi_{\parameter}^{*}(\bmx)\cdot\partial_{\parameter}\Psi_{\parameter}(\bmx) d\bmx}{\int_{\bmx\in \Xcal} \Psi_{\parameter}^{*}(\bmx)\cdot\Psi_{\parameter}(\bmx) d\bmx}\\
% 		=&\int_{\bmx\in \Xcal} \pi_{\parameter}(\bmx)E_{\parameter}(\bmx)\partial_{\parameter}\log\Psi_{\parameter}^{*}(\bmx) d\bmx+\int_{\bmx\in \Xcal} \pi_{\parameter}(\bmx)E^{*}_{\parameter}(\bmx)\partial_{\parameter}\log\Psi_{\parameter}(\bmx) d\bmx\\
% 		&-\int_{\bmx\in\Xcal}\pi_{\parameter}(\bmx)E_{\parameter}(\bmx) d\bmx \cdot \lrp{\int_{\bmx\in \Xcal} \pi_{\parameter}(\bmx)\partial_{\parameter}\log\Psi_{\parameter}^{*}(\bmx) d\bmx+\int_{\bmx\in \Xcal} \pi_{\parameter}(\bmx)\partial_{\parameter}\log\Psi_{\parameter}(\bmx) d\bmx }\\
% 		=&\Ebb_{ \pi_{\parameter}}\lb \lp E_{\parameter}(\bmx)-\Lcal(\parameter)\rp\partial_{\parameter}\log \Psi^{*}_{\parameter}(\bmx)\rb+\Ebb_{\pi_{\parameter}}\lb\lp E_{\parameter}^{*}(\bmx)-\Lcal(\parameter)\rp\partial_{\parameter}\log \Psi_{\parameter}(\bmx)\rb.
% 	\end{aligned}
% \end{equation}
% \begin{equation}
% 	\label{eq:grad}
% 	\begin{aligned}
% 		\nabla_{\parameter}\Lcal(\parameter)&=\sum_{\bmx,\bmx^{\prime}}\left\{\frac{\nabla_{\parameter}\left(\Psi_{\parameter}^{*}(\bmx)H_{\bmx,\bmx^{\prime}}\Psi_{\parameter}(\bmx^{\prime})\right)}{\sum_{\bmx}\Psi_{\parameter}^{*}(\bmx)\Psi_{\parameter}(\bmx)}-\frac{\nabla_{\parameter}\left(\sum_{\bmx}\Psi_{\parameter}^{*}(\bmx)\Psi_{\parameter}(\bmx)\right)\Psi_{\parameter}^{*}(\bmx)H_{\bmx,\bmx^{\prime}}\Psi_{\parameter}(\bmx^{\prime})}{(\sum_{\bmx}\Psi_{\parameter}^{*}(\bmx)\Psi_{\parameter}(\bmx))^2}\right\}\\
% 				&\overset{a)}{=}\sum_{\bmx,\bmx^{\prime}}\left\{\frac{\Psi_{\parameter}^{*}(\bmx)H_{\bmx,\bmx^{\prime}}\nabla_{\parameter}\Psi_{\parameter}(\bmx^{\prime})}{\sum_{\bmx}\Psi_{\parameter}^{*}(\bmx)\Psi_{\parameter}(\bmx)}-\frac{\sum_{\bmx}\Psi_{\parameter}^{*}(\bmx)\nabla_{\parameter}\Psi_{\parameter}(\bmx)}{\sum_{\bmx}\Psi_{\parameter}^{*}(\bmx)\Psi_{\parameter}(\bmx)}\cdot\frac{\Psi_{\parameter}^{*}(\bmx)H_{\bmx,\bmx^{\prime}}\Psi_{\parameter}(\bmx^{\prime})}{\sum_{\bmx}\Psi_{\parameter}^{*}(\bmx)\Psi_{\parameter}(\bmx)}\right\}\\
% 				&=\sum_{\bmx,\bmx^{\prime}}\left\{\frac{\Psi_{\parameter}^{*}(\bmx)H_{\bmx,\bmx^{\prime}}\Psi_{\parameter}(\bmx^{\prime})}{\sum_{\bmx}\Psi_{\parameter}^{*}(\bmx)\Psi_{\parameter}(\bmx)}\nabla_{\parameter}\log\Psi_{\parameter}(\bmx^{\prime})-\Ebb_{\bmx\sim\pi_{\parameter}}[\nabla_{\parameter}\log\Psi_{\parameter}(\bmx)]\cdot\frac{\Psi_{\parameter}^{*}(\bmx)H_{\bmx,\bmx^{\prime}}\Psi_{\parameter}(\bmx^{\prime})}{\sum_{\bmx}\Psi_{\parameter}^{*}(\bmx)\Psi_{\parameter}(\bmx)}\right\}
% 				\\
% 				&\overset{b)}{=}\sum_{\bmx,\bmx^{\prime}}\left\{\frac{\Psi_{\parameter}^{*}(\bmx^{\prime})H_{\bmx,\bmx^{\prime}}\Psi_{\parameter}(\bmx)}{\sum_{\bmx}\Psi_{\parameter}^{*}(\bmx)\Psi_{\parameter}(\bmx)}\nabla_{\parameter}\log\Psi_{\parameter}(\bmx)-\Ebb_{\bmx\sim\pi_{\parameter}}[\nabla_{\parameter}\log\Psi_{\parameter}(\bmx)]\cdot\frac{\Psi_{\parameter}^{*}(\bmx^{\prime})H_{\bmx,\bmx^{\prime}}\Psi_{\parameter}(\bmx)}{\sum_{\bmx}\Psi_{\parameter}^{*}(\bmx)\Psi_{\parameter}(\bmx)}\right\}
% 				\\
% 				&=\sum_{\bmx}\left\{\pi_{\parameter}(\bmx)E^{*}_{\parameter}(\bmx)\nabla_{\parameter}\log\Psi_{\parameter}(\bmx)-\Ebb_{\bmx\sim\pi_{\parameter}}[\nabla_{\parameter}\log\Psi_{\parameter}(\bmx)]\cdot \pi_{\parameter}(\bmx)E^{*}_{\parameter}(\bmx)\right\}\\
% 				&=\Ebb_{\bmx\sim\pi_{\parameter}}\left[E^{*}_{\parameter}(\bmx)\nabla_{\parameter}\log\Psi_{\parameter}(\bmx)\right]-\Ebb_{\bmx\sim\pi_{\parameter}}\left[E^{*}_{\parameter}(\bmx)\right]\Ebb_{\bmx\sim\pi_{\parameter}}\left[\nabla_{\parameter}\log\Psi_{\parameter}(\bmx)\right]\\
% 				&=\Ebb_{\bmx\sim\pi_{\parameter}}\left[(E^{*}_{\parameter}(\bmx)-\Ebb_{\bmx\sim\pi_{\parameter}}\left[E^{*}_{\parameter}(\bmx)\right])\nabla_{\parameter}\log\Psi_{\parameter}(\bmx)\right],
% 	\end{aligned}
% \end{equation}
% where a) holds since $\Psi_{\parameter}$ is complex analytic, and b) switches $\bmx$ and $\bmx^{\prime}$ in summation order.
% When the wavefunction is real, we may simplify this to 
% \begin{equation}
% 	\begin{aligned}
% 		\label{eq:realgrad}
% 		\nabla_{\parameter}\Lcal(\parameter)=2\Ebb_{ \pi_{\parameter}}\lb\lp E_{\parameter}(\bmx)-\Ebb_{ \pi_{\parameter}}\lb E_{\parameter}(\bmx)\rb\rp\nabla_{\parameter}\log \Psi_{\parameter}(\bmx)\rb.
% 	\end{aligned}
% \end{equation}
\subsection{Sampling algorithm}
The exact loss function and the gradient can be hardly calculated in practice as the dimension of the Hilbert space grows exponentially on the number of particles. We can hardly compute the high-dimensional integral $\int_{\bmx\in \Xcal} \Psi_{\parameter}^{*}(\bmx)\cdot\Psi_{\parameter}(\bmx) d\bmx $ to sample directly. However, the VMC method uses the MCMC sampling to estimate the expectation with a unnormalized probability. The MH algorithm is a common MCMC algorithm which constructs a Markov chain with proposal and acceptance steps.

The MH algorithm generates samples from a Markov chain whose stationary distribution is desired. As more and more samples are produced, the Markov chain converges to some equilibrium, which means that it becomes stationary over time. We can construct a Markov chain with a transition probability $Q(\bmx^{\prime}|\bmx)$, which is also called proposal distribution. For the Schr\"{o}dinger equation, the simplest proposal distribution is a normal distribution with a mean of zero. At the $i$-th step, the MH algorithm produces a Markov chain with the stationary distribution $\pi$ as follows. We first draw a sample $\bmx^{\prime}$ from $Q(\bmx^{\prime}|\bmx_{i})$, where $\bmx_{i}$ is the previous sample. Then, the acceptance probability is computed by
\begin{equation}
	\begin{aligned}
		\label{eq: accept}
		a(\bmx^{\prime}|\bmx_{i})=\min\left\{1,\frac{\pi(\bmx^{\prime})Q(\bmx_{i}|\bmx^{\prime})}{\pi(\bmx_{i})Q(\bmx^{\prime}|\bmx_{i})}\right\}=\min\left\{1,\frac{|\Psi_{\parameter}(\bmx^{\prime})|^2Q(\bmx_i|\bmx^{\prime})}{|\Psi_{\parameter}(\bmx_i)|^2Q(\bmx^{\prime}|\bmx_i)}\right\}.
	\end{aligned}
\end{equation}
We accept the new sample $\bmx_{i+1}=\bmx^{\prime}$ with probability $a(\bmx^{\prime}|\bmx_i)$ or remain $\bmx_{i+1}=\bmx_{i}$. After $n_0+n$ iterations, we discard first $n_0$ samples for a better estimate, which is called a burn-in period. Finally, $n$ samples $\mathbf{S}=\{\bmx_{n_0+1},\bmx_{n_0+2},\dots,\bmx_{n_0+n}\}$ is obtained by to estimate the expectation. We avoid computing the high-dimensional integral by the cancellation of terms in the numerator and denominator in \eqref{eq: accept}.
The MH algorithm is summarized as Algorithm \ref{alg:MH}.
\begin{algorithm} %算法开始 
	
	\caption{Metropolis-Hasting algorithm}
	\begin{algorithmic}[1]
	\label{alg:MH}
		\REQUIRE Any initial configuration $\bmx_0$, the proposal distribution $Q(\bmx^{\prime}|\bmx)$.
		\FOR{ $i=0,1,2,\dots$}
		\STATE Draw a sample $\bmx^{\prime}$ from $Q(\bmx^{\prime}|\bmx_i)$.
		\STATE Generate $u\sim U[0,1]$ and compute the acceptance probability $a(\bmx^{\prime}|\bmx_i)$ defined by \eqref{eq: accept}.
		\IF{ $u\leq a(\bmx^{\prime}|\bmx_i)$ }
		\STATE Accept the proposal sample and set $\bmx_{i+1}=\bmx^{\prime}$.
		\ELSE 
		\STATE Reject it and set $\bmx_{i+1}=\bmx_{i}$.
		\ENDIF 
		\ENDFOR
	\end{algorithmic}
\end{algorithm}


\subsection{Stochastic gradient descent using MCMC estimator}

The objective function and its gradient is estimated by the MCMC samples $\mathbf{S}$ generated by the MH Algorithm \ref{alg:MH}:
\begin{align}
	\hat{\Lcal}(\parameter;\mathbf{S})&=\frac{1}{|\mathbf{S}|}\sum_{\bmx\in \mathbf{S}}E_{\parameter}(\bmx),\\
	\hat{g}(\parameter;\mathbf{S})&=\frac{2}{|\mathbf{S}|}\sum_{\bmx\in \mathbf{S}}\big(E_{\parameter}(\bmx)-\frac{1}{|\mathbf{S}|}\sum_{\bmx\in \mathbf{S}}E_{\parameter}(\bmx)\big)\nabla_{\parameter} \log \Psi_{\parameter}(\bmx).\label{eq:approxgrad}
\end{align}
% \begin{align}
% 	\hat{\Lcal}(\parameter;\mathbf{S})&=\frac{1}{n}\sum_{i=n_0+1}^{n_0+n}E_{\parameter}(\bmx_i),\\
% 	\hat{g}(\parameter;\mathbf{S})&=\frac{2}{n}\sum_{i=n_0+1}^{n_0+n}\big(E_{\parameter}(\bmx_i)-\frac{1}{n}\sum_{i=n_0+1}^{n_0+n}E_{\parameter}(\bmx_i)\big)\nabla \log \Psi_{\parameter}(\bmx_i).\label{eq:approxgrad}
% \end{align}
 The MCMC methods provide a tractable and efficient way of approximating expectation in the objective function and its gradient. Nevertheless, MCMC estimators are not generally unbiased for a finite sample size and the samples generated by MCMC are non-independent. The existence of bias implies that the expectation of our stochastic gradient is not equal to the exact gradient. A sufficiently large sample size should be applied for reducing the bias, while the size should not be too large to lose randomness and computability. It suggests that we should pay attention to the influence of sampling methods in optimization. The MCMC method enables us to apply the SGD to update the parameters:
\begin{equation}
	\label{eq:iter}
	\parameter_{k+1}= \parameter_{k} - \alpha_k\hat{g}(\parameter_k,\mathbf{S}_{k}),
\end{equation}
where $\alpha_k$ are chosen stepsizes and $\mathbf{S}_k$ can be obtained by the MH algorithm with $\pi_{\parameter_{k}}$. 

% Additionally and not necessary, for noise reduction, we clip those outliers in local energies calculated by MCMC algorithm to get better estimate of the gradient. When we already have a batch of local energies $E_{\parameter_k}(\bmx_{i}^{(k)})$, replace those with an extreme deviation 
% \begin{equation}
% 	\label{ineq:clip}
% 	\begin{aligned}
% 		\hat{g}^{clip}(\parameter,\mathbf{S})&=\frac{2}{n}\sum_{i=n_0+1}^{n_0+n}\hat{E}_{\parameter}\nabla \log \Psi_{\parameter}(\bmx_i)\\
% 		\hat{E}_{\parameter}&=\begin{cases}~E_{\parameter_k}(\bmx_i)-\hat{\Lcal}(\parameter,\mathbf{S})&\mathrm{if}~\left|E_{\parameter_k}(\bmx_i)-\hat{\Lcal}(\parameter,\mathbf{S})\right|\leq R,\\
% 		~0 &\mathrm{else}.
% 		\end{cases}
% 	\end{aligned}
% \end{equation}
% by the mean energy $\hat{\Lcal}(\parameter,\mathbf{S})$, where $R$ is a hyperparameter. In practice, it is verified that this trick reduces the risk of abnormal gradients and improve stability.
\begin{algorithm} %算法开始 

	\setstretch{1.15}
	\caption{Variational Monte Carlo}
	\begin{algorithmic}[1]
		\label{alg:VMC}
		\REQUIRE The Hamiltonian $\Hcal$, the initialized parameter $\parameter_0$, the sample size $n$ and the length of the burn-in period $n_0$.
		\FOR{ $k=0,1,2,\dots$}
		\STATE Draw $n$ samples $\mathbf{S}_k=\{\bmx_{n_0+1}^{(k)},\dots,\bmx^{(k)}_{n_0+n}\}$ by MH algorithm after a burn-in period of $n_0$. 
		\STATE Compute the estimated gradient $\hat{g}(\parameter_k,\mathbf{S}_{k})$ by \eqref{eq:approxgrad}.
		\STATE Update the parameter by \eqref{eq:iter} with the stepsize $\alpha_k$.
		\ENDFOR
	\end{algorithmic}
\end{algorithm} 


\section{Convergence analysis of VMC}
\label{sec:conv}

In this section, we study the convergence of optimization in the VMC method. Some assumptions are imposed to ensure indispensable properties in the objective function and sampling. Then, we analyze the MCMC estimator by the concentration inequality for Markov chains, which is essentially different from the ordinary SGD. Finally, our convergence theorems for VMC are presented through the gradient norm in expectation.
\subsection{Assumptions}
\label{subsec:asm}
We first give the definition of sub-exponential random variables to introduce our assumptions.
\begin{definition}
    The sub-exponential norm of a random variable $X$ is 
    \begin{equation}
        \|X\|_{\psi_1}=\inf \left\{t > 0 : \epct{\exp\lrp{\frac{|X-\Ebb[X]|}{t}}}\leq 2\right\}.
    \end{equation}
    If $\|X\|_{\psi_1}$ is finite, we say that $X$ is sub-exponential with the parameter $\|X\|_{\psi_1}$.

    % (2) The sub-gaussian norm of a random variable $X$ is 
    % \begin{equation}
    %     \|X\|_{\psi_2}=\inf \left\{t > 0 : \epct{\exp\lrp{\frac{|X|^2}{t}}}\leq 2\right\}.
    % \end{equation}
    % If $\|X\|_{\psi_2}$ is finite, we say that $X$ is  sub-gaussian with the parameter $\|X\|_{\psi_2}$.
\end{definition}

We introduce some regularity conditions on the trial wavefunction $\Psi_{\parameter}$, in order to guarantee Lipschitz continuity of the gradient $g(\parameter)$. In practice, we usually substitute $\Psi_{\parameter}(\bmx)$ by $\log \Psi_{\parameter}(\bmx)$ to avoid potential numerical instability. The following assumptions can be satisfied in most of ansatzes.
\begin{assumption}
    \label{asm:wavefun}
    Let $\log\Psi_{\parameter}(\bmx)$ be differentiable with respect to the parameters $\parameter\in \Rbb^d$ for any $\bmx\in \Xcal$. There exist constants $B,L_1>0$ such that
    \begin{enumerate}
        \setlength{\itemsep}{2pt}
        \item[(1)] $\sup\limits_{\parameter\in \Rbb^d}\sup\limits_{\bmx \in \Xcal}\norm{\nabla_{\parameter}\log \Psi_{\parameter}(\bmx)}\leq B$,
        \item[(2)] $\sup\limits_{\parameter\in \Rbb^d}\Ebb_{\pi_{\parameter}}\lb \norm{\nabla_{\parameter}^2\log \Psi_{\parameter}(\bmx)}^2\rb\leq L_1^2$.
    \end{enumerate}
\end{assumption}


There are also some assumptions on the local energy $E_{\parameter}$. Noticing that the local energy may be unbounded for trial function, we give the sub-exponential assumption of the local energy under the distribution $\pi_{\parameter}$.
\begin{assumption}
    \label{asm:local}
    Let the local energy $E_{\parameter}(\bmx)$ defined in \eqref{loss:vmc} satisfy the following conditions. There exist constants $M,L_2>0$ such that
    \begin{enumerate}
        \setlength{\itemsep}{2pt}
        \item[(1)] $\sup\limits_{\parameter\in \Rbb^d}\norm{\oE_{\parameter}(\bmx)}_{\psi_1}\leq M$,
        \item[(2)] $\sup\limits_{\parameter\in \Rbb^d}\Ebb_{\pi_{\parameter}}\lb \norm{\nabla_{\parameter}\oE_{\parameter}(\bmx)}^2\rb\leq L_2^2$.
    \end{enumerate}
    
    % (3) The local energy is locally Lipschitz continuous, namely for $\ell_{\parameter}(\bmx):=\nrm{\nabla_{\parameter}E_{\parameter}(\bmx)}$, its expectation over $\pi_{\parameter}(\bmx)$ can be bounded by some $\Lipe$. More specifically, we assume
    % \[\begin{aligned}
    % \EE_{\bmx\sim\pi_{\parameter}}\left[\ell_{\parameter}(\bmx)\right]=
    % \EE_{\bmx\sim\pi_{\parameter}}\left[\nrm{\nabla_{\parameter}E_{\parameter}(\bmx)}\right]
    % \leq L_e.
    % \end{aligned}\]
\end{assumption}

% \begin{assumption}
%     \label{asm:wavefun}
%     Let $\Psi_{\parameter}$ be a real wavefunction parameterized by $\parameter\in\Rbb^{d}$. We assume there exist constants $B_g,L_g >0$ such that 
%     \begin{enumerate}
%         \setlength{\itemsep}{2pt}
%         % \item[(1)] $|\log \Psi_{\parameter}(\bmx)|\leq B,~\forall \parameter \in \Rbb^{d},$
%         \item[(1)] $\Ebb_{\pi_{\parameter}}[\|\nabla_{\parameter} \log \Psi_{\parameter}(\bmx)\|^4]\leq B_g^4,~\forall \parameter \in \Rbb^{d},$
%         \item[(2)] $\Ebb_{\pi_{\parameter}}[\|\nabla_{\parameter}^2 \log \Psi_{\parameter}(\bmx)\|^2]\leq L_{g}^2,~\forall \parameter \in \Rbb^{d}.$
%     \end{enumerate}
% \end{assumption}

% The regularity assumptions on the distribution are similar to those in the convergence analysis of policy gradient in reinforcement learning, according to the literature \cite{wu2020finite}. It is reasonable for neural networks to achieve the assumption.
% Obviously, $\log \Psi_{\parameter}(\bmx)$ is also Lipschitz continuous with the constant $B_g$ due to its bounded gradient.

% Based on this assumption, we show the Lipschitz continuity of the probability and the local energy. Before the lemmas, there are some facts of Lipschitz continuous functions:
% \begin{lemma}
%     \label{lem:Lips}
%     Let $f,g$ be any Lipschitz continuous functions of $\parameter$ with constants $L_1,L_2 > 0$, bounded by $B_1,B_2 > 0$, then it holds that 
%     \begin{enumerate}
%         \item  $e^{f}$ is Lipschitz continuous with the constant $e^{B_1}L_1$;
%         \item  $f+g$ is Lipschitz continuous with the constant $L_1+L_2$;
%         \item  $fg$ is Lipschitz continuous with the constant $B_1L_2+B_2L_1$.
%     \end{enumerate}
% \end{lemma}
% \begin{proof}
%     \begin{enumerate}
%         \item Using Lagrange's mean theorem, the inequality holds that 
%         \begin{align*}
%             |e^{f(\parameter_1)}-e^{f(\parameter_1)}|\leq \max_{\parameter}\{e^{f(\parameter)}\}\left|f(\parameter_1)-f(\parameter_2)\right|\leq e^{B_1}L_1\|\parameter_1-\parameter_2\|.
%         \end{align*}
%         \item Obviously.
%         \item It holds that
%         \begin{align*}
%             |f(\parameter_1)g(\parameter_1)-f(\parameter_2)g(\parameter_2)|&\leq |f(\parameter_1)g(\parameter_1)-f(\parameter_1)g(\parameter_2)|+|f(\parameter_1)g(\parameter_2)-f(\parameter_2)g(\parameter_2)|\\
%             &\leq B_1 L_1\|\parameter_1-\parameter_2\|+B_2 L_2\|\parameter_1-\parameter_2\|.
%         \end{align*}
%     \end{enumerate} 
% \end{proof}


% \begin{proof} By Lemma \ref{lem:Lips}, we have
% \begin{align*}
%     \|\pi_{\parameter_1}(\bmx)-\pi_{\parameter_2}(\bmx)\|&=\left|\frac{e^{2\log \Psi_{\parameter_1}(\bmx)}}{\sum_{\bmx^{\prime}}e^{2\log \Psi_{\parameter_1}(\bmx^{\prime})}}-\frac{e^{2\log \Psi_{\parameter_2}(\bmx)}}{\sum_{\bmx^{\prime}}e^{2\log \Psi_{\parameter_2}(\bmx^{\prime})}}\right|\\
%     &\leq \left|\frac{e^{2\log \Psi_{\parameter_1}(\bmx)}-e^{2\log \Psi_{\parameter_2}(\bmx)}}{\sum_{\bmx^{\prime}}e^{2\log \Psi_{\parameter_1}(\bmx^{\prime})}}\right|+\left|e^{2\log \Psi_{\parameter_2}(\bmx)}\right|\left|\frac{1}{\sum_{\bmx^{\prime}}e^{2\log \Psi_{\parameter_1}(\bmx^{\prime})}}-\frac{1}{\sum_{\bmx^{\prime}}e^{2\log \Psi_{\parameter_2}(\bmx^{\prime})}}\right|\\
%     &\leq \frac{1}{2^{N}e^{-2B}}e^{2B}2B_g\|\parameter_1-\parameter_2\|+e^{2B}\frac{1}{(2^{N}e^{-2B})^2}2^{N}e^{2B}2B_g\|\parameter_1-\parameter_2\|\\
%     &=C_{\pi}\|\parameter_1-\parameter_2\|.
% \end{align*}
% where we denote that $C_{\pi}=\frac{B_g}{2^{N-1}}(e^{4B}+e^{8B})$.
% \end{proof}

% In practice, the local energy is computed in the form \begin{equation}
%     \label{eq:Eloc}
%         E_{\parameter}(\bmx)=\sum_{\bmx^{\prime}}H_{\bmx,\bmx^{\prime}}e^{2\log \Psi_{\parameter}(\bmx^{\prime})-2\log \Psi_{\parameter}(\bmx)},
%     \end{equation}
%     where $H_{\bmx,\bmx^{\prime}}$ be the element of the Hamiltonian matrix defined in \eqref{ham:tensor}. If $\bmx$ is not relevant to $\bmx$ in Heisenberg model, which means that $\bmx$ can not transfer to $\bmx^{\prime}$ through flips on two neighboring spins, then $H_{\bmx,\bmx^{\prime}}=0$. It suggests that $H$ is a sparse matrix since relevant spin configurations are rare in a huge space of $2^{N}$.
% \begin{assumption}
%     \label{asm:Eloc}
%     For the local energy $E_{\parameter}(\bmx)$ defined in \eqref{loss:vmc}, there exist a bound $B_e$ and a Lipschitz constant $C_{\pi}>0$ such that for all $\bmx$, it holds that
%     \begin{enumerate}
%         \item $|E_{\parameter}(\bmx)|\leq B_e,~\forall \parameter \in \Rbb^{d},$
%         \item $|E_{\parameter_1}(\bmx)-E_{\parameter_2}(\bmx)|\leq L_e\|\parameter_1-\parameter_2\|,~\forall \parameter_1,\parameter_2 \in \Rbb^{d}.$
%     \end{enumerate}
% \end{assumption}
% \begin{proof}\begin{enumerate}
%         \item Let $H_{\max}=\max_{\bmx,\bmx^{\prime}}|H_{\bmx,\bmx^{\prime}}|$, then it holds that 
%         \begin{align*}
%             |E_{\parameter}(\bmx)|=\left|\sum_{\bmx^{\prime}}H_{\bmx,\bmx^{\prime}}e^{2\log \Psi_{\parameter}(\bmx^{\prime})-2\log \Psi_{\parameter}(\bmx)}\right|\leq \sum_{\bmx^{\prime}}H_{\max}e^{4B}=2^{N}H_{\max}e^{4B}=:B_e.    \end{align*}
%         \item By Lemma \ref{lem:Lips}, we have
%         \begin{align*}
%             |E_{\parameter_1}(\bmx)-E_{\parameter_2}(\bmx)|&=\left|\sum_{\sigma^{\prime}}H_{\bmx,\bmx^{\prime}}\left(e^{2\log \Psi_{\parameter_1}(\bmx^{\prime})-2\log \Psi_{\parameter_1}(\bmx)}-e^{2\log \Psi_{\parameter_2}(\bmx^{\prime})-2\log \Psi_{\parameter_2}(\bmx)}\right)\right|\\
%             &\leq 2^{N} H_{\max} e^{4B}\cdot 4B_g\|\parameter_1-\parameter_2\|=L_e\|\parameter_1-\parameter_2\|,
%         \end{align*}
%         since $2\log \Psi_{\parameter}(\bmx^{\prime})-2\log \Psi_{\parameter}(\bmx)$ be bounded by $4B$ and has the Lipschitz constant $4B_g$.
%     \end{enumerate}
% \end{proof}
% Beside the regularity conditions on the wavefunction, there is a general assumption on the local energy for different types of Hamiltonian. For example, the local energy in Heisenberg model has a uniform bound while that in Schr\"odinger equation \eqref{hamiltonian:SE} is unbounded. Compared to the bounded condition, there are some weaker assumptions on the local energy $E_{\parameter}$ under the distribution $\pi_{\parameter}$.

% Besides, we may have an alternative assumption that claims $\|\nabla_{\parameter} \log \Psi_{\parameter}(\bmx)\|$ and $\bar{E}_{\parameter}$ are all sub-gaussian with respect to $\pi_{\parameter}$.
% \begin{assumption}
%     \label{asm:wavefun2}
%     Let $\Psi_{\parameter}$ be the real parameterized wavefunction and $\bar{E}_{\parameter} $ is the local energy function. We assume there exist constants $M_g,M_e >0$ such that 
%     \begin{enumerate}
%         \setlength{\itemsep}{2pt}
%         \item[(1)] $ \|\nabla_{\parameter} \log \Psi_{\parameter}\|$ is sub-gaussian with the parameter $M_g$, that is 
%         \begin{equation*}
%             \Ebb_{\pi_{\parameter}}\left[\exp\lrp{\frac{\|\nabla_{\parameter} \log \Psi_{\parameter}(\bmx)\|^2}{M_g}}\right]\leq 2,~\forall \parameter \in \Rbb^{d},
%         \end{equation*}

%         \item[(2)] $ \bar{E}_{\parameter}$ is sub-gaussian with the parameter $M_e$, that is 
%         \begin{equation*}
%             \Ebb_{\pi_{\parameter}}\left[\exp\lrp{\frac{|\bar{E}_{\parameter}(\bmx)|^2}{M_e}}\right]\leq 2,~\forall \parameter \in \Rbb^{d}.
%         \end{equation*}
%     \end{enumerate}
% \end{assumption}



Under Assumptions \ref{asm:wavefun} and \ref{asm:local}, we are able to analyze the objective function $\Lcal(\parameter)$. Since the wavefunction $\Psi_{\parameter}$ appears in the denominator of the local energy expression \eqref{loss:vmc}, and there are unbounded potentials in the many-electron Hamiltonian \eqref{hamiltonian:SE}, the above two assumptions seem not easy to be satisfied.
%\color{blue} Nevertheless, the choice of the most suited ansatz for the wavefunction, as well as the optimization of its parameters, is a prerequisite in the VMC calculations.}
For instances, when solving the many-electron Schr\"{o}dinger equations, people often use the so-called Jastrow factor \cite{gubernatis16} to enable the network to efficiently capture the  cusps and decay of the wavefunction.
By exploiting physical knowledge in the construction of the wavefunction ansatz can make the local energy $E_{\parameter}$ smooth with respect to the parameters $\parameter$, such that Assumption \ref{asm:local} is satisfied in practical VMC calculations.

% The sub-exponential random variables have the following properties. 
% \begin{lemma}\label{lemma:sub-exp}
%     Assume the sub-exponential norm of a random variable $X$ is bounded by $M$. Then the following holds.

%     (1) For any $t\geq 0$, we have the tail bound of $X$
%     \[\begin{aligned}
%         \Pbb(\left|X\right|\geq t) \leq 2 \exp \left(-\frac{t}{M}\right).
%     \end{aligned}\]

%     (2) For $p=1,2,\dots$, the absolute $p$-th moment of $X$ satisfies
%     \begin{equation*}
%         \epct{\left|X\right|^p}\leq 2 \cdot p! M^{p}.
%     \end{equation*}
% \end{lemma}
% \begin{proof}
%     (1) Using Markov's inequality, it holds for any $t>0$,
%     \begin{align*}
%         \Pbb(|X|\geq t)=\Pbb(e^{\frac{|X-\epct{X}|}{M}}\geq e^{ \frac{t}{M}})\leq e^{-\frac{t}{M}}\epct{e^{\frac{|X-\epct{X}|}{M}}}\leq  2\exp \left(-\frac{t}{M}\right).
%     \end{align*}

%     (2) Due to the tail bound, the following holds for the absolute $p$-th moment of $X$,
%     \begin{equation*}
%         \begin{aligned}
%             \epct{\left|X\right|^p}&=\int_{0}^{+\infty}p s^{p-1}\Pbb(\left|X \right|\geq s)d s\\
%             &\leq 2 p \int_{0}^{+\infty}  s^{p-1}\exp \left(-\frac{t}{M}\right)d s\\
%             &=2\cdot p! M^{p}.
%         \end{aligned} 
%     \end{equation*}

% \end{proof}




% \begin{assumption}
%     {\color{blue} (This assumption makes Assumption 4.1 redundent.)} Either one of the following statements holds true:
    
%     (1) Both $\bar{E}_{\parameter}(\bmx)$ and $\nabla_{\parameter} \log \Psi_{\parameter}(\bmx)$ are subgaussian random variables under $\bmx\sim\pi_{\parameter}$.

%     (2) $\nabla_{\parameter} \log \Psi_{\parameter}(\bmx)$ is uniformly bounded, and Assumption 4.4(2) holds.
% \end{assumption}





% \begin{proof} We can obtain the gradient $g(\parameter)$ by \eqref{eq:grad}. We rewrite its real-valued version in the same way, then the gradient is 
% \begin{align*}
%     \nabla l(\parameter)&=2\Ebb_{\bmx\sim\pi_{\parameter}}\left[(E_{\parameter}(\bmx)-\Ebb_{\bmx\sim\pi_{\parameter}}\left[E_{\parameter}(\bmx)\right])\nabla_{\parameter}\log\Psi_{\parameter}(\bmx)\right],\\
%     &=2\sum_{\bmx}\pi_{\parameter}(\bmx)E_{\parameter}(\bmx)\log\Psi_{\parameter}(\bmx)-2\left(\sum_{\bmx}\pi_{\parameter}(\bmx)E_{\parameter}(\bmx)\right)\left(\sum_{\bmx}\pi_{\parameter}(\bmx)\log\Psi_{\parameter}(\bmx)\right).
% \end{align*}
% Lemma \ref{lem:Lips} suggests that the addition and multiplication of two bounded Lipschitz function is also Lipschitz continuous. Lemma \ref{lem:prob} and Lemma \ref{lem:Eloc} show that $\pi_{\parameter}(\bmx)$ and $E_{\parameter}(\bmx)$ is bounded and Lipschitz continuous for all $\bmx$. Meanwhile, $\nabla \log \Psi_{\parameter}(\bmx)$ has also boundness and Lipschitz continuity under Assumption \ref{asm:wavefun}. Hence, it implies the gradient is also Lipschitz continuous with constant $L=2^{N+1}(3C_{\pi}B_eB_g+2L_eB_g+2L_gB_e)$.
% \end{proof}

\subsection{Analysis of the MCMC error}
\label{subsec:error}
%% Markov chain definition

We introduce following notations which are used frequently throughout this paper. For any function $f:\Xcal\rightarrow \Rbb$ and any distribution $\pi$ on $\Xcal$, we write its expectation $\Ebb_\pi[f]:=\int f(x)\pi(dx)$ and $p$-th central moment $\sigma^p_p[f]:=\Ebb[(f-\Ebb_\pi[f])^p]$. The second central moment, called the variance, is denoted by $\mathrm{Var}_{\pi}[f]=\sigma_2^2[f]:=\Ebb[(f-\Ebb_\pi[f])^2]$. 

Before the analysis of MCMC methods, we provide several general definitions about Markov chains. 
% and $L_{0}^2(\pi)=\{f\in L^{2}(\pi):\Ebb_\pi[f]=0 \}$ be its subspace of $\pi$-measure zero functions. 
Within our consideration, the state space $\Xcal$ is Polish and equipped with its $\sigma$-algebra $\Bcal$. Let $\{X_i\}_{i=1}^{n}$ be a time-homogeneous Markov chain defined on $\Xcal$. The distribution of the Markov chain is uniquely determined by its initial distribution $\nu$ and its transition kernel $P$. For any Borel set $A\in \Bcal$, let 
\begin{equation*}
    \begin{aligned}
        \nu(A)=\Pbb(X_1\in A),\quad P(X_i,A)=\Pbb(X_{i+1}\in A|X_i).
    \end{aligned}
\end{equation*}
A distribution $\pi$ is called stationary with respect to a transition kernel $P$ if 
\begin{equation*}
    \pi(A)=\int P(x,A)\pi(dx), ~\forall A\in \Bcal.
\end{equation*}
When the initial distribution $\nu=\pi$, we call the Markov chain stationary.

Our analysis starts from the perspective of operator theory on Hilbert spaces. Let $\pi$ be the stationary distribution of a Markov chain and $L^{2}(\pi)=\{f:\Ebb_\pi[f^2]<\infty \}$ be the Hilbert space equipped with the norm $\norm{f}_{\pi}=(\Ebb_\pi[f^2])^{1/2}$. And for any $f\in L^2(\pi)$, we define $\norm{f}_{\psi_1}$ by $\norm{f(X)}_{\psi_1} $ with $X\sim\pi$. Each transition kernel can be viewed as a Markov operator on the Hilbert space $L^2(\pi)$. The Markov operator $\operatorname{P}:L^{2}(\pi)\rightarrow L^{2}(\pi)$ is defined by 
\begin{equation*}
    \begin{aligned}
        \operatorname{P}f(x)=\int f(y)P(x,dy),~\forall x\in \Xcal,~\forall f \in L^{2}(\pi).
    \end{aligned}
\end{equation*}
It is easy to show that $\operatorname{P}$ has the largest eigenvalue $1$. Intuitively but not strictly, the gap between $1$ and other eigenvalues matters to the Markov chain from non-stationarity towards stationarity. Hence, we introduce the definition of the absolute spectral gap.
\begin{definition}[absolute spectral gap]
    \label{def:absgap}
     A Markov operator $\operatorname{P}:L^{2}(\pi)\rightarrow L^{2}(\pi)$ admits an absolute spectral gap $\gamma$ if 
    \begin{equation*}
        \gamma(\operatorname{P})=1-\lambda(\operatorname{P}):=1-\interleave \operatorname{P}-\Pi\interleave _{\pi}>0,
    \end{equation*}
    where  $\Pi:f\in L^2(\pi)\rightarrow\Ebb_\pi[f]\mathit{1}$ is the projection operator with $\mathit{1}$ denoting the identity operator and $\interleave\cdot\interleave_{\pi}$ is the operator norm induced by $\norm{\cdot}_{\pi}$ on $L^2(\pi)$.
\end{definition}
% \begin{definition}[right spectral gap]
%     \label{def:rsgap}
%     A Markov operator $P$ admits a right spectral gap $1-\gamma$ if 
%     \begin{equation*}
%         \gamma=\gamma(R):=\sup\{\langle Rf,f\rangle:f\in \Lcal^2_{0}(\pi)\},~~ where ~ R=(P+P^*)/2.
%     \end{equation*}
% \end{definition}
% By these definitions, we have
% \begin{equation}
%     \label{eq:gap}
%     |\gamma(R)|\leq \gamma(R)\leq \frac{\gamma(P)+\gamma(P^{*})}{2}=\gamma(P).
% \end{equation}


% \begin{remark}
%     For the finite-state time-homogeneous Markov chain, $\gamma(P)=\frac{\max\{|\gamma_2|,|\gamma_{min}|\}+1}{2}$ where $\gamma_2$ is actually the second largest eigenvalue of the transition matrix $P$ and $\gamma_{min}$ is the minimum eigenvalue. 
% \end{remark}

We consider the MCMC samples generated by the MH algorithm $\mathbf{S}=\{\bmx_{i}\}_{i=n_0+1}^{n_0+n}$ from the desired distribution $\pi_{\parameter}$. The Markov operator in the Metropolis way, denoted by $\operatorname{P}_{\parameter}$, determines the convergence rate of the Markov chain. We assume that there exists a uniform lower bound of the spectral gap.  
\begin{assumption}
    \label{asm:unigap}
    Let $\operatorname{P}_{\parameter}$ be the Markov operator induced by the MH algorithm with the absolute spectral gap $\gamma(\operatorname{P}_{\parameter})$. For any $\parameter\in \Rbb^d$, there is a positive lower bound of absolute spectral gaps, that is,
    \begin{equation*}
        \gamma := \inf_{\parameter\in\Rbb^d}\gamma(\operatorname{P}_{\parameter}) >0.
    \end{equation*}
\end{assumption}

This lemma excludes the situation that $\inf_{\parameter\in\Rbb^d}\gamma(\operatorname{P}_{\parameter})=0$, which means the spectral gap might converge to zero in the iteration. The spectral gap $\gamma$ ensures the Markov chains to mix well enough, by which the gradient is approximated correctly.

The spectral gap of the MH algorithm has been studied for some specific examples. On finite-state spaces, $\operatorname{P}$ becomes a transition matrix while $1-\gamma(\operatorname{P})$ relates to its eigenvalue.  A survey of spectrums for Markov chains on discrete state-spaces is in \cite{saloff1997lectures}. This develops amounts of analytic techniques and \cite{diaconis1998we} has further applications to the MH algorithm. For the continuous spaces, there are few examples of sharp rates of convergence for the MH algorithm. In \cite{kienitz2000convergence,miclo2000trous}, it is claimed that the spectral gap $\gamma\sim O(h^2)$ for the Gaussian proposal $Q(x^{\prime}|x)\sim \exp\left(-\tfrac{1}{2h^2}\norm{x^{\prime}-x}^2\right)$.

MCMC provides an approach to estimate the expectation in VMC by averaging Markov chain samples after a burn-in period. A Bernstein inequality for general Markov chains is proposed by Jiang and Fan \cite{jiang2018bernstein} and beneficial to our analysis. The following lemma is a direct corollary of \cite[Theorem 2]{jiang2018bernstein}
%Then, Fan et al. (2021) derive a non-asymptotic error bound for MCMC estimation using Hoeffding's inequality \cite{fan2021hoeffding}. A corollary is obtained by these two references without much effort.

\begin{lemma}%[Jiang and Fan, 2018]
    \label{lem:Bern}
    Let $\{X_i\}_{i= 1}^{n}$ be a Markov chain with stationary distribution $\pi$ and absolute spectral gap $\gamma$. Suppose the initial distribution $\nu$ is absolute continuous with respect to the stationary distribution $\pi$ and its derivative $\tfrac{d\nu}{d\pi}\in L^2(\pi)$. Consider a bounded function $f:\mathcal{X}\rightarrow [-c,c]$ with $\Ebb_\pi[f]=0$ and variance $\sigma_2^2[f]$. Then, when $\nu=\pi$ , that is, $\{X_i\}_{i= 1}^{n}$ is stationary, it holds that
    \begin{equation}
        \label{eq:Bern-stat}
        \mathbb{P}_{\pi}\left(\frac{1}{n} \sum_{i=1}^{n} f\left(X_{i}\right)\geq s\right) 
    \leq \exp\left(-\frac{\gamma ns^2}{4\sigma^2+5cs}\right), \qquad \forall s\geq 0.
    \end{equation}
    % Therefore, for the general case, it holds that
    % \begin{equation}
    %     \label{eq:Bern}
    %     \mathbb{P}_{\nu}\left(\frac{1}{n} \sum_{i=1}^{n} f\left(X_{i}\right)\geq s\right) 
    % \leq (1+\rchi^2(\nu,\pi))^{\frac{1}{2}} \exp\left(-\frac{\gamma ns^2}{8\sigma^2+10cs}\right), \qquad \forall s\geq 0.
    % \end{equation}
    % where $\rchi^2(\nu,\pi):=\norm{\tfrac{d\nu}{d\pi}-1}^{2}_{\pi}$ represents the chi-squared divergence between $\nu$ and $\pi$.
\end{lemma}

% \begin{lemma}[Jiang and Fan, 2018]
%     \label{lem:Bern}
%     Let $\{X_i\}_{i= 1}^{n}$ be a non-stationary Markov chain with stationary distribution $\pi$ and absolute spectral gap $\gamma$. Suppose the initial distribution $\nu$ is absolute continuous with respect to the stationary distribution $\pi$ and its derivative $\tfrac{d\nu}{d\pi}\in L^2(\pi)$. Consider a bounded function $f:\mathcal{X}\rightarrow [-c,c]$ with variance $\sigma_2^2[f]$. Then, for any $ |t|<\frac{\gamma}{ 10 c}$, it holds that
%     % \begin{equation}
%     %     \mathbb{E}\left[e^{t \left(\sum_{i=n_0+1}^{n_0+n} f\left(X_{i}\right)-n\pi(f)\right)}\right] 
%     %     \leq C\exp \left(\frac{n \sigma^{2}}{c^{2}}\left(e^{t qc}-1-t qc\right)+\frac{n \sigma^{2} \max \left\{\gamma, 0\right\} q^2t^{2}}{1-\max \left\{\gamma, 0\right\}-5 qc t}\right). 
%     % \end{equation}
%     % Moreover, for any $\epsilon>0$,
%     % \begin{equation}
%     %     \mathbb{P}\left(\frac{1}{n} \sum_{i=n_0+1}^{n_0+n} f\left(X_{i}\right)-n\pi(f)>\epsilon\right) 
%     % \leq C\exp \left(-\frac{n \epsilon^{2} / 2}{\alpha_{1}\left(\max \left\{\gamma, 0\right\}\right) \cdot q\sigma^{2}+\alpha_{2}\left(\max \left\{\gamma, 0\right\}\right) \cdot qc \epsilon}\right),
%     % \end{equation}
%     % where $\alpha_1,\alpha_2$ are defined as
%     % \[\begin{aligned}
%     %     \alpha_{1}(\gamma)=\frac{1+\gamma}{1-\gamma}, \quad \alpha_{2}(\gamma)= \begin{cases}\frac{1}{3} & \text { if } \gamma=0 \\ \frac{5}{1-\gamma} & \text { if } \gamma \in(0,1)\end{cases}
%     % \end{aligned}\]

%     % Especially, we can simplify for $0 \leq t<\left(1-\max \left\{\gamma, 0\right\}\right) / 10 qc$
%     \begin{equation}
%         \label{eq:Bern}
%         \mathbb{E}_{\nu}\left[e^{t \left(\sum_{i=1}^{n} f\left(X_{i}\right)-n\Ebb_\pi[f]\right)}\right] 
%     \leq (1+\rchi^2(\nu,\pi))^{\frac{1}{2}}\exp\left(\frac{16n\sigma_2^2[f]  t^2}{\gamma}\right),
%     \end{equation}
%     where $\rchi^2(\nu,\pi):=\norm{\tfrac{d\nu}{d\pi}-1}^{2}_{\pi}$ represents the chi-squared divergence between $\nu$ and $\pi$.
% \end{lemma}

The Bernstein inequality \eqref{eq:Bern-stat} shows how the average of MCMC samples concentrates at the expectation. However, the Markov chain is not always non-stationary. We define $\rchi^2(\nu,\pi):=\norm{\tfrac{d\nu}{d\pi}-1}^{2}_{\pi}$ represents the chi-squared divergence between $\nu$ and $\pi$, by which we can extend the Bernstein inequality into a non-stationary one. We abbreviate $\rchi=\rchi(\nu,\pi)$ and $C=(1+\rchi^2)^{\frac{1}{2}}$. Then we give the tail bound of the MCMC estimator for sub-exponential functions in the following lemma.

\begin{lemma}
    \label{lem:Bern-exp}
    Let $\{X_i\}_{i= 1}^{n}$ be a non-stationary Markov chain with stationary distribution $\pi$ and absolute spectral gap $\gamma$. Suppose the initial distribution $\nu$ is absolute continuous with respect to the stationary distribution $\pi$ and its derivative $\tfrac{d\nu}{d\pi}\in L^2(\pi)$.  We consider a function $f\in L^2(\pi)$ satisfying $ \norm{f}_{\psi_1}\leq M$. If $s\geq \frac{20M(\log n)^2}{n}$, the following tail bound holds,  
    \begin{equation}
        \mathbb{P}_{\nu}\left(\abs{ \frac{1}{n} \sum_{i=1}^{n} f\left(X_{i}\right)-\Ebb_\pi[f] }\geq s\right) 
    \leq 2C\exp\left(-\frac{\gamma ns^2}{64\sigma^2_2[f]}\right) + 2C\exp\left(-\sqrt\frac{\gamma ns}{80M}\right).
    \end{equation}
    In other words, for $\delta>0$, with probability at least $1-\delta$, it holds that
    \begin{equation}
        \label{eq:highprobbound}
        \left|\frac{1}{n}\sum_{i=1}^{n}f(X_i)-\Ebb_\pi[f]\right|\leq 8\sigma_2[f]\sqrt{\frac{\log(4C/\delta)}{n\gamma}}+ 80M\frac{[\log(4Cn/\delta)]^2}{n\gamma}.
    \end{equation}
\end{lemma}

\begin{proof}
    Without loss of generality, we assume that $\Ebb_\pi[f]=0$ in the following proof.
    To deal with a possibly unbounded $f$, we firstly fix a $M'>0$ and consider the truncation function $\bar{f}=\max\{\min\{f,M^{\prime}\},-M^{\prime}\}$ and $\hat{f}=f-\bar{f}$. The basic property of $\hat{f}$ is that, as $\norm{f(X)}_{\psi_1}\leq M$, the Markov's inequality implies 
    \begin{equation}
        \label{eq:subexp-tail}
        \Pbb_\pi(\abs{f}>s)\leq \exp\left(-\frac{s}{M}\right)\Ebb_\pi\left[\exp\left(\frac{|f|}{M}\right)\right] \leq 2\exp\left(-\frac{s}{M}\right).
    \end{equation}
    % Then, for any $p\geq 1$, we have
    % \begin{equation}
    %     \label{eq:pmonent-hatf}
    %     \begin{aligned}
    %         \Ebb_\pi\big[|\hat{f}|^p\big]&=\int_{0}^{+\infty} ps^{p-1}\Pbb_\pi(|\hat{f}|>s)ds\leq \int_{0}^{+\infty} ps^{p-1}\Pbb_\pi(|f|>s+M^{\prime})ds\\
    %         &\leq 2p \int_{0}^{+\infty} s^{p-1}\exp\left(-\frac{s+M^{\prime}}{M}\right)ds=2\cdot p! M^{p}\exp\left(-\frac{M^{\prime}}{M}\right).
    %     \end{aligned}
    % \end{equation}
    
    % By introducing the truncation $\bar{f},\hat{f}$, we make the corresponding decomposition,
    % \begin{align*}
    %     &\Delta:=\frac{1}{n}\sum_{i=1}^{n}f(X_i)=\bar{\Delta}+\hat{\Delta},\\
    %     &\bar{\Delta}:=\frac{1}{n}\sum_{i=1}^{n}\bar{f}(X_i)-\Ebb_\pi[\bar{f}],~~
    %     \hat{\Delta}:=\frac{1}{n}\sum_{i=1}^{n}\hat{f}(X_i)-\Ebb_\pi[\hat{f}].
    % \end{align*}
    % We analyze the error from two aspects.
    % \paragraph{High probability bound} 
    Notice that for any event $A\in\sigma(X_1,\cdots,X_n)$, it holds from the Cauchy-Schwarz inequality
    \begin{align*}
        \Pbb_{\nu}(A)
        =&~ \int_{\mathcal{X}} \Pbb(A|X_1=x)\nu(dx) 
        = \int_{A} \Pbb(A|X_1=x)\frac{d\nu}{d\pi}\pi(dx)\\
        \leq&~ \sqrt{\int_{\mathcal{X}} \left(\frac{d\nu}{d\pi}\right)^2\Pbb(A|X_1=x)\pi(dx) \int_{\mathcal{X}} \Pbb(A|X_1=x) \pi(dx) } \\
        \leq&~ C\sqrt{\Pbb_{\pi}(A)}.
    \end{align*}
    Hence, we only consider the case $\nu=\pi$, i.e., the case $\{X_i\}_{i=1}^{n}$ is stationary.
    We first fix a large $M^{\prime}>0$. For any $1\leq i\leq n$, we have 
    \begin{equation}
        \label{eq:highprob1}
        \begin{aligned}
            \Pbb\left(\abs{f(X_i)}>M^{\prime} \right) \leq 2\exp\left(-\frac{M^{\prime}}{M}\right).
        \end{aligned}
    \end{equation} 
    It follows from the inclusion of events that
    \begin{equation}
        \begin{aligned}
            \mathbb{P}\left(\abs{ \frac{1}{n} \sum_{i=1}^{n} f\left(X_{i}\right) }\geq s\right)
            \leq&~\mathbb{P}\left(\abs{ \frac{1}{n} \sum_{i=1}^{n} \bar{f}\left(X_{i}\right) }\geq s\right) + \Pbb\left( \exists ~ 1\leq i\leq n,~ \abs{f(X_i)}>M^{\prime}\right)\\
            \leq&~\mathbb{P}\left(\abs{ \frac{1}{n} \sum_{i=1}^{n} \bar{f}\left(X_{i}\right) -\EE_{\pi}[\bar{f}]}\geq s-\abs{\EE_{\pi}[\bar{f}]}\right) + 2n\exp\left(-\frac{M^{\prime}}{M}\right).
        \end{aligned}
    \end{equation}
    Notice that $\abs{\EE_{\pi}[\bar{f}]}=\abs{\EE_{\pi}[\hat{f}]}\leq 2M\exp\left(-\frac{M^{\prime}}{M}\right)$. Therefore, when $s\geq 2\abs{\EE_{\pi}[\hat{f}]}$, it holds that
    \begin{equation}
        \begin{aligned}
            \mathbb{P}\left(\abs{ \frac{1}{n} \sum_{i=1}^{n} \bar{f}\left(X_{i}\right) }\geq s\right) 
            \leq&~\mathbb{P}\left(\abs{ \frac{1}{n} \sum_{i=1}^{n} \bar{f}\left(X_{i}\right) -\EE_{\pi}[\bar{f}]}\geq s-\abs{\EE_{\pi}[\bar{f}]}\right) \\
            \leq&~\mathbb{P}\left(\abs{ \frac{1}{n} \sum_{i=1}^{n} \bar{f}\left(X_{i}\right) -\EE_{\pi}[\bar{f}]}\geq \frac{s}{2}\right) \\
            \leq&~ 2\exp\left(-\frac{\gamma ns^2}{16\sigma^2_2[\bar{f}]+10M^{\prime}s}\right) 
            \leq 2\exp\left(-\frac{\gamma ns^2}{16\sigma^2_2[f]+10M^{\prime}s}\right),
        \end{aligned}
    \end{equation} 
    where the third inequality is due to \eqref{eq:Bern-stat}, and  the last inequality uses $\sigma^2_2[\bar{f}]\leq \sigma^2_2[f]$. Finally, for any fixed $s\geq \frac{20M(\log n)^2}{n}$, we take $M^{\prime}=\sqrt{\gamma nMs/5}$ and then obtain
    \begin{equation}
        \begin{aligned}
            \mathbb{P}\left(\abs{ \frac{1}{n} \sum_{i=1}^{n} f\left(X_{i}\right) }\geq s\right)
            \leq&~2\exp\left(-\frac{\gamma ns^2}{16\sigma^2_2[f]+10M^{\prime}s}\right) + 2n\exp\left(-\frac{M^{\prime}}{M}\right)\\
            =&~ 2\exp\left(-\frac{\gamma ns^2}{16\sigma^2_2[f]+10s\sqrt{\gamma nMs/5}}\right) + 2n\exp\left(-\sqrt\frac{\gamma ns}{5M}\right)\\
            \leq&~ 2\exp\left(-\frac{\gamma ns^2}{2\max\{16\sigma^2_2[f],10s\sqrt{\gamma nMs/5}\}}\right) + 2n\exp\left(-\sqrt\frac{\gamma ns}{5M}\right)\\
            \leq&~ 2\exp\left(-\frac{\gamma ns^2}{32\sigma^2_2[f]}\right) + 2\exp\left(-\sqrt\frac{\gamma ns}{20M}\right) + 2n\exp\left(-\sqrt\frac{\gamma ns}{5M}\right)\\
            \leq&~ 2\exp\left(-\frac{\gamma ns^2}{32\sigma^2_2[f]}\right) + 4\exp\left(-\sqrt\frac{\gamma ns}{20M}\right),
        \end{aligned}
    \end{equation}
    where the last inequality holds from $n\exp\left(-\sqrt\frac{\gamma ns}{5M}\right)\leq \exp\left(-\sqrt\frac{\gamma ns}{20M}\right)$ as long as $s\geq \frac{20M(\log n)^2}{n}$.
    This completes the proof.
\end{proof}

In the above lemma, we establish the tail bound of the MCMC estimator for unbounded Markov chains. To the best of our knowledge, this result has not appeared in the literatures of Bernstein inequality for Markov chains. 

\begin{lemma}
    \label{lem:BernBiasVar}
    Suppose that the condition of Lemma \ref{lem:Bern-exp} holds. Then, error bounds for the MCMC estimator are given as follows. 
    
    (1) The bias satisfies that
    \begin{equation}
        \label{eq:generalbias}
        \left|\Ebb_{\nu}\lrb{\frac{1}{n}\sum_{i=1}^{n}f(X_i)}-\Ebb_\pi[f]\right|
        \leq\frac{c_1}{n\gamma},
    \end{equation}
    where $c_1=\sigma_2[f]\minop{1,\rchi}+4M\pos{\log\rchi}^2+4M\pos{\log\rchi}$ and $\rchi=\rchi(\nu,\pi)$. In particular, when $\rchi\leq 1$, $c_1=\sigma_2[f]\rchi$.

    (2) The variance satisfies that
    \begin{equation}
        \label{eq:generalvariance}
        \Ebb_{\nu}\lrb{\left|\frac{1}{n}\sum_{i=1}^{n}f(X_i)-\Ebb_\pi[f]\right|^2} \leq\frac{c_2}{n\gamma}\sigma_2^2[f]+\frac{c_3+c_4\log^4 n}{n^2\gamma^2}M^2,
    \end{equation}
    where $c_2=64(1+\log 2C)$, $c_3=6400(4+\log 2C)^4$, $c_4=800$, and $C=(1+\rchi^2)^{\frac{1}{2}}$. 
\end{lemma}

\begin{proof}
    % Without loss of generality, we assume that $\Ebb_\pi[f]=0$ in the following proof.
    % To deal with a possibly unbounded $f$, we firstly fix a $M'>0$ and consider the truncation function $\bar{f}=\max\{\min\{f,M^{\prime}\},-M^{\prime}\}$ and $\hat{f}=f-\bar{f}$. The basic property of $\hat{f}$ is that, as $\norm{f(X)}_{\psi_1}\leq M$, the Markov's inequality implies 
    % \begin{equation}
    %     \label{eq:subexp-tail}
    %     \Pbb_\pi(\abs{f}>s)\leq \exp\left(-\frac{s}{M}\right)\Ebb_\pi\left[\exp\left(\frac{|f|}{M}\right)\right] \leq 2\exp\left(-\frac{s}{M}\right).
    % \end{equation}
    % Then, for any $p\geq 1$, we have
    % \begin{equation}
    %     \label{eq:pmonent-hatf}
    %     \begin{aligned}
    %         \Ebb_\pi\big[|\hat{f}|^p\big]&=\int_{0}^{+\infty} ps^{p-1}\Pbb_\pi(|\hat{f}|>s)ds\leq \int_{0}^{+\infty} ps^{p-1}\Pbb_\pi(|f|>s+M^{\prime})ds\\
    %         &\leq 2p \int_{0}^{+\infty} s^{p-1}\exp\left(-\frac{s+M^{\prime}}{M}\right)ds=2\cdot p! M^{p}\exp\left(-\frac{M^{\prime}}{M}\right).
    %     \end{aligned}
    % \end{equation}
    
    % By introducing the truncation $\bar{f},\hat{f}$, we make the corresponding decomposition,
    % \begin{align*}
    %     &\Delta:=\frac{1}{n}\sum_{i=1}^{n}f(X_i)=\bar{\Delta}+\hat{\Delta},\\
    %     &\bar{\Delta}:=\frac{1}{n}\sum_{i=1}^{n}\bar{f}(X_i)-\Ebb_\pi[\bar{f}],~~
    %     \hat{\Delta}:=\frac{1}{n}\sum_{i=1}^{n}\hat{f}(X_i)-\Ebb_\pi[\hat{f}].
    % \end{align*}
    % We analyze the error from two aspects.
    We follow the notations in the proof of Lemma \ref{lem:Bern-exp}.
    \paragraph{Bias bound} 
    Clearly, we only need to bound $\frac{1}{n}\sum_{i=1}^{n}\abs{\Ebb_{\nu_i}[f]-\Ebb_{\pi}[f]}$, where $\nu_i= \nu \oP^{i-1}$ is the distribution of $X_i$. 
    For the distribution $\pi^{\prime}$ whose derivative $\frac{d\pi^{\prime}}{d\pi}\in L^{2}(\pi)$, it holds from the Cauchy-Schwarz inequality that 
    \begin{align}\label{eq:change-measure}
        \abs{\Ebb_{\pi^{\prime}}[f]-\Ebb_\pi[f]}
        =\abs{ \EE_{\pi}\left[\left(\frac{d\pi^{\prime}}{d\pi}-1\right)(f-\Ebb_\pi[f])\right] }
        \leq  \left[\EE_{\pi}\left(\frac{d\pi^{\prime}}{d\pi}-1\right)^2\right]^{1/2}\sigma_2[f]
        = \rchi(\pi^{\prime},\pi)\sigma_2[f].
    \end{align}
    Besides, noticing that $\left(\Ebb_\pi \abs{\hat{f}}^2\right)^{\frac12}\leq 2M\exp\left(-\frac{M'}{2M}\right)$, we have
    \begin{align}
        \abs{\Ebb_{\pi^{\prime}}[\hat{f}]-\Ebb_{\pi}[\hat{f}]}
        \leq \rchi(\pi^{\prime},\pi) \left(\Ebb_\pi \abs{\hat{f}}^2\right)^{\frac12}
        \leq 2M\rchi(\pi^{\prime},\pi)\exp\left(-\frac{M'}{2M}\right).
    \end{align}
    % \begin{align}\label{eq:change-measure}
    %     \abs{\nu(f)-\pi(f)}
    %     =\abs{ \EE_{\pi}\left[\left(\frac{d\nu}{d\pi}-1\right)f\right] }
    %     \leq \sigma_f \left[\EE_{\pi}\left(\frac{d\nu}{d\pi}-1\right)^2\right]^{1/2}
    %     =\sigma_f\rchi(\nu,\pi).
    % \end{align}
    Then \eqref{eq:change-measure} implies that
    \begin{equation}
        \label{eq:change-function}
        \abs{\Ebb_{\nu_i}[f]-\Ebb_{\pi}[f]}
        \leq \abs{\Ebb_{\nu_i}[\bar{f}]-\Ebb_{\pi}[\bar{f}]}+\abs{\Ebb_{\nu_i}[\hat{f}]-\Ebb_{\pi}[\hat{f}]}
        \leq 2M'+2M\rchi(\nu_i,\pi)\exp\left(-\frac{M'}{2M}\right),
    \end{equation}
    and as long as $\rchi(\nu_i,\pi)\geq 1$, we can choose $M'=2M\log \rchi(\nu_i,\pi)$ in \eqref{eq:change-function} to derive $\abs{\Ebb_{\nu_i}[f]-\Ebb_{\pi}[f]}\leq 4M(1+\log \rchi(\nu_i,\pi))$.
    Combining \eqref{eq:change-measure} and \eqref{eq:change-function} yields that
    \begin{equation}
        \begin{aligned}
            \label{eq:nu-pi}
            \abs{\Ebb_{\nu_i}[f]-\Ebb_{\pi}[f]}
        \leq \begin{cases}
             \rchi(\nu_i,\pi)\sigma_2[f], & \text{always},\\
            4M\left(1+\log\rchi(\nu_i,\pi)\right), &\text{if }\rchi(\nu_i,\pi)\geq 1.
        \end{cases}
        \end{aligned}
    \end{equation}
    Now, by the definition of $\gamma$, it holds that $\rchi(\nu_i,\pi)\leq (1-\gamma)^{i-1} \rchi(\nu_1,\pi)=(1-\gamma)^{i-1} \rchi$. 
        Let us consider the smallest $k\geq 1$ such that $\rchi(\nu_{k},\pi)\leq 1$. Then clearly $k\leq 1+\ceil{\frac{[\log\rchi]_+}{\gamma}}$, and for $i\geq k$, $\rchi(\nu_{i},\pi)\leq (1-\gamma)^{i-1}\min\{1,\rchi\}$. Hence,
        \begin{align*}
            \sum_{i=1}^{n}\abs{\Ebb_{\nu_i}[f]-\Ebb_{\pi}[f]}
            \leq& \sum_{i=1}^{k-1}\abs{\Ebb_{\nu_i}[f]-\Ebb_{\pi}[f]}+\sum_{i=k}^{n}\abs{\Ebb_{\nu_i}[f]-\Ebb_{\pi}[f]}\\
            \leq& \sum_{i=1}^{k-1}4M\left(1+\log\rchi(\nu_i,\pi)\right)+\sum_{i=k}^{n} \rchi(\nu_i,\pi)\sigma_2[f]\\
            \leq& 4M(k-1)\left(1+\log\rchi\right)+ \sigma_2[f] \min\{1,\rchi\}\cdot \frac1{\gamma},
        \end{align*}
    where the first inequality is the triangle inequality and the second uses \eqref{eq:nu-pi}. Therefore, it holds that
    \begin{equation}
        \frac{1}{n}\sum_{i=1}^{n}\abs{\Ebb_{\nu_i}[f]-\Ebb_{\pi}[f]}
        \leq \frac{1}{n\gamma}\left(\sigma_2[f] \minop{
            1, \rchi}+4M\pos{\log\rchi}^2+4M\pos{\log\rchi}\right).
    \end{equation}

    \paragraph{Variance bound}
    In the following proof, we provide an upper bound on all higher-order moments of $\Delta$ simultaneously.
    Denote $A=\log(2C), R_1=8\sqrt{\frac{\sigma_2^2[f]}{\gamma n}}, R_2=\frac{80M}{\gamma n}, s_0=\frac{20M(\log n)^2}{n}$. Then, we only need to bound $\Ebb[\abs{\Delta}^m]$ for even $m$ under the condition
    \begin{align*}
        \Pbb\left(\abs{\Delta}\geq s\right)\leq \exp(A-s^2/R_1^2)+\exp(A-\sqrt{s/R_2}), \qquad \forall s\geq s_0.
    \end{align*}
    Notice that
    \begin{align*}
        \frac1m\Ebb[\abs{\Delta}^m]
        =&~\int_{0}^\infty s^{m-1}\Pbb(\abs{\Delta}\geq s)ds\\
        \leq&~\int_{0}^\infty s^{m-1}\min\{1,\exp(A-s^2/R_1^2)+\exp(A-\sqrt{s/R_2})\}ds\\
        \leq&~\underbrace{\int_{0}^\infty s^{m-1}\min\{1,\exp(A-s^2/R_1^2)\}ds}_{I_1}+\underbrace{\int_{0}^\infty s^{m-1}\min\{1,\exp(A-\sqrt{s/R_2})\}ds}_{I_2}.
    \end{align*}
    Thus, we further denote $s_1=\max\{s_0,R_1\sqrt{A}\}, s_2=\max\{s_0,R_2A^2\}$. Then
    \begin{align*}
        I_1=&~\int_{0}^{s_1} s^{m-1}ds+\int_{s_1}^\infty s^{m-1}\exp(A-s^2/R_1^2)ds\\
        =&~\frac{s_1^m}{m}+\frac{R_1^{m}}{2}\int_{s_1}^\infty \left(\frac{s}{R_1}\right)^{m-2}\exp(A-s^2/R_1^2)d\left(\frac{s^2}{R_1^2}\right)\\
        % =&~\frac{s_1^m}{m}+\frac{R_1^{m}}{2}\cdot\exp(A-s_1^2/R_1^2)\cdot \left(\frac{m-2}{2}\right)!,
        =&~\frac{s_1^m}{m}+\frac{R_1^{m}}{2}\int_{s_1^2/R_1^2}^\infty t^{(m-2)/2}\exp(A-t)dt\\
        \leq&~\frac{s_1^m}{m}+\frac{R_1^{m}}{m}\left((A+m/2)^{m/2}-A^{m/2}\right),
    \end{align*}
    where the last inequality is due to the fact $\int_{A}^{\infty} t^{k-1}\exp(A-t)dt\leq (A+k-1)^{k-1}\leq ((A+k)^k-A^k)/k$ for any integer $k\geq 1$. Similarly,
    \begin{align*}
        I_2=&~\int_{0}^\infty s^{m-1}\min\{1,\exp(A-\sqrt{s/R_2})\}ds\\
        =&~\int_{0}^{s_2} s^{m-1}ds+\int_{s_2}^\infty s^{m-1}\exp(A-\sqrt{s/R_2})ds\\
        =&~\frac{s_2^m}{m}+\int_{\sqrt{s_2/R_2}}^\infty 2R_2^{m}t^{2m-1}\exp(A-t)dt\\
        \leq&~ \frac{s_2^m}{m}+\frac{R_2^{m}}{m}\left((A+2m)^{2m}-A^{2m}\right).
    \end{align*}
    Combining these two cases, we obtain
    \begin{align*}
        \Ebb[\abs{\Delta}^m]\leq&~ mI_1+mI_2
        \leq s_0^m+s_1^m+R_1^m\left((A+m/2)^{m/2}-A^{m/2}\right) + R_2^m \left((A+2m)^{2m}-A^{2m}\right)\\
        \leq&~ 2s_0^m+R_1^mA^{m/2}+R_2^mA^{2m}+R_1^m\left((A+m/2)^{m/2}-A^{m/2}\right) + R_2^m \left((A+2m)^{2m}-A^{2m}\right)\\
        =&~2s_0^m+R_1^m(A+m/2)^{m/2} + R_2^m (A+2m)^{2m}.
    \end{align*}
    This completes the proof.
\end{proof}

% We can also consider the MCMC error of a bounded function that can be regarded as the special case of a sub-exponential variable. If $X$ is a random variable bounded by $M$, then the exponential norm of $X$ is less than $3M$.

% \begin{proof}
%     Use the Jensen inequality and Hoeffding inequality \eqref{eq:HoeffdingExp}, then for any $t>0$, it holds that 
%     \begin{equation}
%         \begin{aligned}
%             \left|\Ebb\left[\frac{1}{n}\sum_{i=n_0+1}^{n_0+n}f(X_i)\right]-\pi(f)\right|&=\frac{1}{t}\left|\Ebb\left[t\left(\frac{1}{n}\sum_{i=n_0+1}^{n_0+n}f(X_i)-\pi(f)\right)\right]\right|\\
%             &\leq \frac{1}{t} \left|\log \Ebb\left[e^{\frac{t}{n}(\sum_{i=n_0+1}^{n_0+n}f(X_i)-n\pi(f))}\right]\right|\\
%             &\leq \frac{\log C}{t} +\frac{\sigma^2 t}{2n}.
%         \end{aligned}
%     \end{equation}
%     Let $t=\sqrt{\frac{2n \log C}{\sigma}}$, the RHS is equal to $\sqrt{\frac{2\sigma^2 \log C}{n}}$. 
% \end{proof}

Through Lemma \ref{lem:BernBiasVar}, we are able to characterize the sampling error in VMC. For any fixed $\parameter\in \Rbb^d$, let the MH algorithm generate $n$ samples $\mathbf{S}=\{\bmx_i\}_{i=n_0+1}^{n_0+n}$, and the stochastic gradient calculated by \eqref{eq:approxgrad}. Then, we give the error bounds of the stochastic gradient with the MCMC estimator in the following lemma.
\begin{lemma}
    \label{lem:gbiasVar}
    Let Assumption \ref{asm:wavefun}, \ref{asm:local} and \ref{asm:unigap} hold. For a fixed parameter $\parameter\in \Rbb^d$, we generate the MCMC samples $\mathbf{S}=\{\bmx_i\}_{i=n_0+1}^{n_0+n}$ with the stationary distribution $\pi_{\parameter}$ by the MH algorithm \ref{alg:MH}. Suppose  we start the MH algorithm from the initial distribution $\nu$ and let $\rchi=\rchi(\nu,\pi_{\parameter})<+\infty$. The stochastic gradient $\hat{g}(\parameter;\mathbf{S})$, defined by \eqref{eq:approxgrad} has the following error bounds that
    \begin{equation}
        \label{eq:BV}
        \begin{aligned}
            \norm{\Ebb[\hat{g}(\parameter;\mathbf{S})]-g(\parameter)}&\leq B_{n,n_0}:=\frac{4c_1B}{n\gamma},\\
            \epct{\norm{\hat{g}(\parameter;\mathbf{S})-g(\parameter)}^2}
            &\leq V_{n,n_0}:=\frac{16c_2B^2\sigma^2_{2}[E_{\parameter}]}{n\gamma}+\frac{40(c_3+c_4\log^4 n)B^2M^2}{n^2\gamma^2},
        \end{aligned}
    \end{equation}
    where with $\rchi_{n_0}= (1-\gamma)^{n_0}\rchi$ and $C=(1+\rchi_{n_0})^{\frac{1}{2}}$, these factors are defined by $c_1=\rchi_{n_0}\sigma_{2}[E_{\parameter}] +4M[\log \rchi_{n_0}]^{2}_{+}+4M[\log \rchi_{n_0}]_{+}$, $c_2=64(1+\log 2C)$ and $c_3=6400(4+\log 2C)^4$, $c_4=800$.
    % where $c_1=(1-\gamma)^{n_0}\rchi\sigma_{2}[E_{\parameter}] +4M[\log \rchi + n_0\log( 1-\gamma)]^{2}_{+}+4M[\log \rchi + n_0\log( 1-\gamma)]_{+}$, $c_2=128(3+\log C)$, $ c_3=100(8+4\log C)^4$ and $C=(1+(1-\gamma)^{2n_0}\rchi^2)^{\frac{1}{2}}$.
\end{lemma}
\begin{proof}
    The error of the stochastic gradient can be rewritten as 
    \begin{equation*}
        \begin{aligned}
            \norm{\Ebb[\hat{g}(\parameter;\mathbf{S})]-g(\parameter)}&=\sup_{\norm{v}=1}\abs{\Ebb[v^{\T}\hat{g}(\parameter;\mathbf{S})]-v^{\T}g(\parameter)},\\
            \epct{\norm{\hat{g}(\parameter;\mathbf{S})-g(\parameter)}^2}&=\tr\left(\epct{\left(\hat{g}(\parameter;\mathbf{S})-g(\parameter)\right)\left(\hat{g}(\parameter;\mathbf{S})-g(\parameter)\right)^{\T}}\right),\\
            &\leq d \sup_{\norm{v}=1}\Ebb[\left(v^{\T}\hat{g}(\parameter;\mathbf{S})-v^{\T}g(\parameter)\right)^2].
        \end{aligned}
    \end{equation*}
    For any given $v$ such that $\norm{v}=1$, $v^{\T}g(\parameter)$ is approximated by $v^{\T}\hat{g}(\parameter;\mathbf{S})$. We denote the stationary variables $E = E_{\parameter}(\bmx)$, $Y = v^{\T}\nabla_{\parameter}\log \Psi_{\parameter}(\bmx)$, $\bar{E}= E_{\parameter}(\bmx)-\Ebb_{\bmx\sim \pi_{\parameter}} [E_{\parameter}(\bmx)]$ with $\bmx\sim \pi_{\parameter}$ and the empirical variables  $E_i = E_{\parameter}(\bmx_i)$, $Y_i = v^{\T}\nabla_{\parameter}\log \Psi_{\parameter}(\bmx_i)$, $\bar{E}_i=E_i-\Ebb_{\bmx\sim \pi_{\parameter}} [E_{\parameter}(\bmx)]$. Then it holds 
    \begin{equation}
        \label{eq:gradsplit}
        \begin{aligned}
            \frac{1}{2}\left(v^{\T}\hat{g}(\parameter;\mathbf{S})-v^{\T}g(\parameter)\right)&=\frac{1}{n}\sum_{i=n_0+1}^{n+n_0}E_i Y_i-\left(\frac{1}{n}\sum_{j=n_0+1}^{n+n_0}E_j \right)\left(\frac{1}{n}\sum_{i=n_0+1}^{n+n_0}Y_i\right) - \Ebb_{\pi_{\parameter}} [\bar{E}Y]\\
            &=\underbrace{\frac{1}{n}\sum_{i=n_0+1}^{n+n_0}\bar{E}_i Y_i-\Ebb_{\pi_{\parameter}}[\bar{E}Y]}_{I_1}-\underbrace{\left(\frac{1}{n}\sum_{j=n_0+1}^{n+n_0}\bar{E}_i\right)\left(\frac{1}{n}\sum_{i=n_0+1}^{n+n_0}Y_i\right)}_{I_2}.
        \end{aligned}
    \end{equation}
    With Assumptions \ref{asm:wavefun} and \ref{asm:local}, we have $\norm{\bar{E}}_{\psi_1}\leq M$ and $\norm{Y}\leq\norm{v}\norm{\nabla_{\parameter}\log\Psi_{\parameter}(\bmx)} \leq B$. Then the variance is bounded by $\sigma_{2}^{2}[\bar{E}Y]\leq \Ebb[(\bar{E}Y)^2]\leq \sigma_2^{2}[E]B^2$. It also holds 
    \begin{equation}
        \begin{aligned}
            \Ebb\left[\exp\left( \frac{\abs{\bar{E}Y-\Ebb[\bar{E}Y]}}{2MB}\right) \right]\leq  \Ebb\left[\exp\left( \frac{\abs{\bar{E}}B+MB }{2MB}\right) \right]\leq 2 ,
        \end{aligned}
    \end{equation}
    which implies $\norm{\bar{E}Y}_{\psi_1}\leq 2MB$. 
    Applying Lemma \ref{lem:BernBiasVar} to $\{\bar{E}_i Y_{i}\}_{ i= n_0+1}^{n_0+n}$, we have %\cfn{more details? The factor $\sigma_{\parameter}$ in $c_1$ seems to be $\sqrt{d}\sigma_{\parameter}$?}
    \begin{equation}
        \begin{aligned}
            \label{eq:Bias}
            \norm{\Ebb[I_1]}\leq \frac{c_1B}{n\gamma},\quad
            \Ebb\left[ \norm{I_1}^2\right] \leq \frac{c_2dB^2\sigma^2_{2}[E_{\parameter}]}{n\gamma}+\frac{4(c_3d+c_4d\log^4 n)M^2B^2}{n^2\gamma^2}.
        \end{aligned}
    \end{equation} 
    % where $c_1=(1-\gamma)^{n_0}\rchi\sigma_{2}[E_{\parameter}] +4M[\log \rchi + n_0\log( 1-\gamma)]^{2}_{+}+4M[\log \rchi + n_0\log( 1-\gamma)]_{+}$, $c_2=128(2+\log C)$, $ c_3=100(8+4\log C)^4$ and $C=(1+(1-\gamma)^{2n_0}\rchi^2)^{\frac{1}{2}}$.

    As $\norm{\bar{E}}_{\psi_1}\leq M$, $\Ebb_{\pi_{\parameter}}[\bar{E}]=0$ and $\norm{\frac{1}{n}\sum_{i=n_0+1}^{n+n_0}Y_i}\leq B$, Lemma \ref{lem:BernBiasVar} implies that
    \begin{equation}
        \begin{aligned}
            \label{eq:Var}
            \norm{\Ebb[I_2]}\leq \frac{c_1B}{n\gamma},\quad \Ebb\left[ \norm{I_2}^2\right] \leq \frac{c_2dB^2\sigma^2_{2}[E_{\parameter}]}{n\gamma}+\frac{(c_3d+c_4d\log^4 n)M^2B^2}{n^2\gamma^2}.
        \end{aligned}
    \end{equation}
    Combining \eqref{eq:Bias} and \eqref{eq:Var}, we finally obtain
    \begin{equation}
        \begin{aligned}
            \norm{\Ebb[\hat{g}(\parameter;\mathbf{S})]-g(\parameter)}&=2\norm{\Ebb[I_1-I_2]}\leq 2\norm{\Ebb[I_1]}+2\norm{\Ebb[I_2]}\leq B_{n,n_0},\\
            \epct{\norm{\hat{g}(\parameter;\mathbf{S})-g(\parameter)}^2}
            &=4\Ebb[\norm{I_1-I_2}^2]\leq 8\Ebb[\norm{I_1}^2]+8\Ebb[\norm{I_2}^2]\leq V_{n,n_0},
        \end{aligned}
    \end{equation}
    where $B_{n,n_0},V_{n,n_0}$ are given in \eqref{eq:BV}.
\end{proof}
%     Then the bias can be divided into two portions,
%     \begin{equation*}
%         \norm{\epct{\hat{g}(\parameter;\mathbf{S})}-\nabla_{\parameter} \Lcal(\parameter)}\leq \underbrace{\norm{\epct{\hat{g}(\parameter;\mathbf{S})-\tilde{g}(\parameter;\mathbf{S})}}}_{\mathrm{I}}+\underbrace{\norm{\epct{\tilde{g}(\parameter;\mathbf{S})}-\nabla_{\parameter} \Lcal(\parameter)}}_{\mathrm{II}}.
%     \end{equation*}

%     As $E_{\parameter}(\bmx)$ has a sub-exponential tail with the parameter $M$, using Lemma \ref{lem:BernBiasVar}, we have
%     \begin{equation*}
%         \begin{aligned}
%             \mathrm{I}&=\left\|\Ebb\left[\frac{1}{n}\sum_{i=n_0+1}^{n_0+n}\left(\frac{1}{n}\sum_{j=n_0+1}^{n_0+n}E_{\parameter}(\bmx_j)-\Ebb_{\pi_{\parameter}}[E_{\parameter}(\bmx)]\right)\nabla_{\parameter}\log \Psi_{\parameter}(\bmx_i)\right]\right\|\\
%             &\leq \frac{8\sigma_{\parameter}B\sqrt{\log C}}{\sqrt{n(1-\gamma)}}+\frac{2MB(1+C)\log C}{n(1-\gamma)}.
%         \end{aligned}
%     \end{equation*}  
%    Besides, each component of $\bar{E}_{\parameter}(\bmx)\log \Psi_{\parameter}(\bmx)$ has a sub-exponential norm less than $\frac{MB}{\sqrt{d}}$ and variance less that $\frac{4B^2\sigma^2_{\parameter}}{d}$. It follows that
%     \begin{equation*}
%         \begin{aligned}
%             \mathrm{II}&= \left\|\Ebb\left[\frac{1}{n}\sum_{i=n_0+1}^{n_0+n}\bar{E}_{\parameter}(\bmx_i)\nabla_{\parameter}\log \Psi_{\parameter}(\bmx_i)-\Ebb_{\pi_{\parameter}}\left[ \bar{E}_{\parameter}(\bmx)\nabla_{\parameter}\log \Psi_{\parameter}(\bmx)\right]\right]\right\|\\
%             &\leq \frac{16\sigma_{\parameter}B\sqrt{\log C}}{\sqrt{n(1-\gamma)}}+\frac{2MB(1+C)\log C}{n(1-\gamma)} .
%         \end{aligned}
%     \end{equation*}
%     Above all, we obtain the bias bound
%     \begin{equation*}
%         B_{n,n_0}=  \frac{24\sigma_{\parameter}B\sqrt{\log C}}{\sqrt{n(1-\gamma)}}+\frac{4MB(1+C)\log C}{n(1-\gamma)}.
%     \end{equation*}

%     Similarly, we estimate the variance through Lemma \ref{lem:BernBiasVar},
%     \begin{equation*}
%         \begin{aligned}
%             % \epct{\|\hat{g}(\parameter;\mathbf{S})-\tilde{g}(\parameter;\mathbf{S})\|^2}\leq & B^2 \cdot\Ebb\left[\left(\frac{1}{n}\sum_{j=n_0+1}^{n_0+n}E_{\parameter}(\bmx_j)-\Ebb_{\pi_{\parameter}}[E_{\parameter}(\bmx)]\right)^2\right]\\
%             % \leq& \frac{256C\sigma_{\parameter}^2B^2}{n(1-\gamma)}+\frac{8\sqrt{3}(1+C)M^2B^2}{n^2(1-\gamma)^2},\\
%             % \epct{\|\tilde{g}(\parameter;\mathbf{S})-\nabla \Lcal(\parameter)\|^2}\leq & \Ebb\left[\left\|\frac{1}{n}\sum_{i=n_0+1}^{n_0+n}\bar{E}_{\parameter}(\bmx_i)\nabla_{\parameter}\log \Psi_{\parameter}(\bmx_i)-\Ebb_{\pi_{\parameter}}\left[\bar{E}_{\parameter}(\bmx)\nabla_{\parameter}\log \Psi_{\parameter}(\bmx)\right]\right\|^2\right]\\
%             % \leq & \frac{1024C\sigma_{\parameter}^2B^2}{n(1-\gamma)}+\frac{8\sqrt{3}(1+C)M^2B^2}{n^2(1-\gamma)^2}.
%             \epct{\|\hat{g}(\parameter;\mathbf{S})-\tilde{g}(\parameter;\mathbf{S})\|^2}\leq& \frac{256C\sigma_{\parameter}^2B^2}{n(1-\gamma)}+\frac{8\sqrt{3}(1+C)M^2B^2}{n^2(1-\gamma)^2},\\
%             \epct{\|\tilde{g}(\parameter;\mathbf{S})-\nabla \Lcal(\parameter)\|^2}
%             \leq & \frac{1024C\sigma_{\parameter}^2B^2}{n(1-\gamma)}+\frac{8\sqrt{3}(1+C)M^2B^2}{n^2(1-\gamma)^2}.
%         \end{aligned}
%     \end{equation*}
%     Hence, the variance bound is shown as follows,
%     \begin{equation*}
%         \begin{aligned}
%             V_{n,n_0}= \frac{2560C\sigma_{\parameter}^2B^2}{n(1-\gamma)}+\frac{32\sqrt{3}(1+C)M^2B^2}{n^2(1-\gamma)^2}.
%         \end{aligned}
%     \end{equation*}
%     Here we complete the proof.


% \begin{lemma}
%     \label{lem:ghighprob}
%     Let Assumption \ref{asm:wavefun} and \ref{asm:unigap} hold. For a fixed parameter $\parameter$, we obtain $n$ MCMC samples $\mathbf{S}=\{\bmx_i\}_{i=n_0+1}^{n_0+n}$ with the stationary distribution $\pi_{\parameter}$ after the burn-in period of length $n_0$. For any $\delta \in (0,1)$, with probability at least $1-\delta$,
%     \begin{equation}
%         \left\|\hat{g}(\parameter,\mathbf{S})-\nabla \Lcal(\parameter)\right\|\leq O\left(\sqrt{\frac{\log(4C/\delta)}{n} }\right).
%     \end{equation}
% \end{lemma}
% \begin{proof}
%     Let the auxiliary gradient $\tilde{g}(\parameter,\mathbf{S})$ be defined by \eqref{eq:auxigrad}. Corollary \ref{crl:highprob} implies that with probability at least $1-\frac{\delta}{2}$,
%     \begin{equation*}
%         \begin{aligned}
%             \|\hat{g}(\parameter,\mathbf{S})-\tilde{g}(\parameter,\mathbf{S})\|&=\left\|\frac{1}{n}\sum_{i=n_0+1}^{n_0+n}\left(\frac{1}{n}\sum_{i=n_0+1}^{n_0+n}E_{\parameter}(\bmx_i)-\Ebb_{\pi_{\parameter}}[E_{\parameter}(\sigma)]\right)\nabla_{\parameter}\log \Psi_{\parameter}(\bmx_i)\right\|\\
%             &\leq 2B_gB_e\sqrt{\frac{1+\gamma}{1-\gamma}\cdot \frac{1}{n}\cdot \log\left(\frac{4C}{\delta}\right) }.
%         \end{aligned}
%     \end{equation*}
%     Similarly, with probability at least $1-\frac{\delta}{2}$,
%     \begin{equation*}
%         \begin{aligned}
%             \|\tilde{g}(\parameter,\mathbf{S})-\nabla \Lcal(\parameter)\|&\leq\left\|\frac{1}{n}\sum_{i=n_0+1}^{n_0+n}E_{\parameter}(\bmx_i)\nabla_{\parameter}\log \Psi_{\parameter}(\bmx_i)-\Ebb_{\pi_{\parameter}}\left[E_{\parameter}(\bmx)\nabla_{\parameter}\log \Psi_{\parameter}(\bmx)\right]\right\|\\
%             &+\left|\Ebb_{\pi_{\parameter}}[E_{\parameter}(\bmx)]\right|\left\|\frac{1}{n}\sum_{i=n_0+1}^{n_0+n}\nabla_{\parameter}\log \Psi_{\parameter}(\bmx_i)-\Ebb_{\pi_{\parameter}}[\nabla_{\parameter}\log \Psi_{\parameter}(\bmx)]\right\| \\
%             &\leq 4B_gB_e\sqrt{\frac{1+\gamma}{1-\gamma}\cdot \frac{1}{n}\cdot \log\left(\frac{4C}{\delta}\right) }.
%         \end{aligned}
%     \end{equation*}
%     Hence, we obtain high probability error bound of the gradient. With probability at least $1-\delta\leq(1-\frac{\delta}{2})^2$, it holds
%     \begin{equation*}
%         \begin{aligned}
%             \left\|\hat{g}(\parameter,\mathbf{S})-\nabla\Lcal(\parameter)\right\|&\leq  \|\hat{g}(\parameter,\mathbf{S})-\tilde{g}(\parameter,\mathbf{S})\|+\|\tilde{g}(\parameter,\mathbf{S})-\nabla \Lcal(\parameter)\|\\
%             &\leq 6B_gB_e\sqrt{\frac{1+\gamma}{1-\gamma}\cdot \frac{1}{n}\cdot \log\left(\frac{4C}{\delta}\right) } \\
%             &= O\left(\sqrt{\frac{\log(4C/\delta)}{n} }\right).
%         \end{aligned}
%     \end{equation*}
% \end{proof}

\subsection{First order convergence}
In this subsection, we will first analyze the descent for one iteration based on the error bounds. Then, our main result shows the first order convergence of the VMC method in expectation.

Under our assumptions in subsection \ref{subsec:asm}, we can establish the $L$-smoothness of the objective function $\Lcal(\parameter)$ to perform a standardized analysis in optimization. This is a common condition for functions to achieve effective descent. We prove the following lemma by giving the bound of the Hessian.
\begin{lemma}
    \label{prop:gLip}
    Let Assumption \ref{asm:wavefun} and \ref{asm:local} hold.  There exists a constant $L>0$ such that $\Lcal(\parameter)$ is $L$-smooth, that is
    \begin{equation}
        \label{eq:gLip}
        \|g(\parameter_1)-g(\parameter_2)\|\leq L\|\parameter_1-\parameter_2\|,~\forall \parameter_1,\parameter_2 \in \Rbb^{d}.
    \end{equation} 
\end{lemma}




% \begin{proof}
%     A random variable $Y$ is of sub-Gaussian type if there exists $\sigma,c>0$ such that
%     \[\begin{aligned}
%         \Ebb\left[e^{tY}\right]\leq c\exp\left(\frac{\sigma^2t^2}{2}\right).
%     \end{aligned}\]
%     The definition above implies that $\Pbb\left(|Y|>t\right)\leq 2c\exp\left(-\frac{t^2}{2\sigma^2}\right)$, and thus
%     \[\begin{aligned}
%         \Ebb\left[Y^2\right]
%         &=\int_{0}^{\infty} \Pbb\left(Y^2\geq s\right)ds\\
%         &\leq \int_{0}^{\infty}\min\left\{1,2c\exp\left(-\frac{s}{2\sigma^2}\right)\right\}ds\\
%         &=2\sigma^2(\log 2c +1).
%     \end{aligned}\]
%     Combined with \eqref{eq:HoeffdingExp}, this fact implies that
%     \[\begin{aligned}
%         \Ebb\left[\frac{1}{n}\sum_{i=n_0+1}^{n_0+n}f(X_i)-\pi(f)\right]^2
%         \leq \frac{2\sigma^2(\log 2C +1)}{n} .
%     \end{aligned}\]
% \end{proof}
% Besides, a high probability upper bound is given by the following corollary.
% \begin{corollary}
%     \label{crl:highprob}
%     Let the conditions in Lemma \ref{lem:hoeffding} be satisfied. For any $\delta \in (0,1)$, with probability at least $1-\delta$,
%     \begin{equation}
%         \left|\frac{1}{n}\sum_{i=n_0+1}^{n_0+n}f(X_i)-\pi(f)\right|\leq \sqrt{\frac{2\sigma^2\log (2C/\delta)}{n} }.
%     \end{equation}
% \end{corollary}
% \begin{proof}
%     Let the right of the inequality \eqref{eq:HoeffdingProb} be $\delta$, then $\epsilon$ become the high probability upper bound. 
% \end{proof}

\begin{proof}
    As Assumptions \ref{asm:wavefun} and \ref{asm:local} hold, we have 
    \begin{equation*}
        \Ebb_{\pi_{\parameter}}\lrb{|\oE_{\parameter}|}\leq M. 
    \end{equation*}
    The Hessian  $H(\parameter)$ defined in \eqref{eq:Hess} can be bounded by
        \begin{equation*}
    \begin{aligned}
    \norm{H(\parameter)}
    \leq &\norm{H_1}+\norm{H_2}+\norm{H_3}+2 \norm{H_{12}}\\
    \leq &2\Ebb_{\pi_{\parameter}}\lrb{\norm{\nabla_{\parameter}E_{\parameter} \nabla_{\parameter}\log \Psi_{\parameter}^{\T}}}+4\Ebb_{\pi_{\parameter}}\lrb{|\oE_{\parameter}|\norm{\nabla_{\parameter}\log \Psi_{\parameter}\nabla_{\parameter}\log \Psi_{\parameter}^{\T}}}\\
    &+2\Ebb_{\pi_{\parameter}}\lrb{|\oE_{\parameter}|\norm{\nabla^{2}_{\parameter}\log \Psi_{\parameter}}}+8\Ebb_{\pi_{\parameter}}\lrb{|\oE_{\parameter}|\norm{\nabla_{\parameter}\log \Psi_{\parameter}}}\Ebb_{\pi_{\parameter}}\lrb{\norm{\nabla_{\parameter}\log \Psi_{\parameter}^{\T}}}\\
    \leq & 2 B L_1+4M B^2+2M L_2+8 M B^2,
    \end{aligned}
    \end{equation*}
    where the last inequality is due to Assumptions \ref{asm:wavefun} and \ref{asm:local}.
    Let $L =2 B L_1+2M L_2+12 M B^2$, then $\Lcal(\parameter)$ is $L$-smooth. 
\end{proof}

According to \cite{wright1999numerical}, the $L$-smoothness \eqref{eq:gLip} is also equivalent to 
\begin{equation}
    \Lcal(\parameter_2)\leq \Lcal(\parameter_1)+\langle g(\parameter_1),\parameter_2-\parameter_1\rangle+\frac{L}{2}\|\parameter_1-\parameter_2\|^2.
\end{equation}

It is already known that the stochastic gradient $\hat{g}$ is biased, unlike the unbiased estimator in the classical SGD. However, the bias decays with increasing burn-in time $n_0$ or the sample size $n$,  which can be regarded sufficiently small. We discuss the expected descent of the function value within each iteration in the following lemma. 

\begin{lemma}
    \label{lem:decrease}
    If the stepsize $\alpha_k\leq \frac{1}{2L}$, for any given $\parameter_k$ the loss function $\Lcal(\parameter_{k+1}) $ decreases in expectation as
    \begin{equation}
        \label{eq:descent}
        \Lcal(\parameter_{k})-\epct{\Lcal(\parameter_{k+1})|\parameter_k}\geq \frac{\alpha_k}{2}\|\nabla\Lcal(\parameter_{k})\|^2-\frac{\alpha_k}{2}\cdot B_{n,n_0}^2
        -\frac{\alpha_k^2L}{2} \cdot V_{n,n_0}.
    \end{equation}
    where  $B_{n,n_0}$ and $V_{n,n_0}$ are defined in Lemma \ref{lem:gbiasVar}.
\end{lemma}
\begin{proof}
    For convenience, we simplify the notations with $\Lcal_k:= \Lcal(\parameter_{k})$, $g_k:=g(\parameter_k)$ and $\hat{g}_k:=\hat{g}(\parameter_k,\mathbf{S}_k)$ for any $k\geq 1$.
    By Lemma \ref{prop:gLip}, we perform a gradient descent analysis of the iteration \eqref{eq:iter}, 
    \begin{equation}\label{eqn:sgd-s1}
        \begin{aligned}
            \Lcal_{k+1}\leq & \Lcal_k+\langle g_k,\parameter_{k+1}-\parameter_k\rangle+\frac{L}{2}\|\parameter_{k+1}-\parameter_k\|^2\\
            =&\Lcal_k-\alpha_k\langle g_k,\hat{g}_k-g_k\rangle-\alpha_k\|g_k\|^2+\frac{\alpha_k^2 L}{2}\|\hat{g}_k\|^2\\
            =&\Lcal_k-(\alpha_k-\alpha_k^2L)\langle g_k,\hat{g}_k-g_k\rangle-\left(\alpha_k-\frac{\alpha_k^2L}{2}\right)\|g_k\|^2+\frac{\alpha_k^2 L}{2}\|\hat{g}_k-g_k\|^2.
        \end{aligned}
    \end{equation}
Taking the conditional expectation of \eqref{eqn:sgd-s1} on $\parameter_k$ gives
\begin{equation}\label{eqn:sgd-s2}
    \begin{aligned}
        \cond{\Lcal_{k+1}}{\parameter_{k}}\leq
        &\Lcal_k-\left(\alpha_k-\frac{\alpha_k^2L}{2}\right)\|g_k\|^2\\
        &-(\alpha_k-\alpha_k^2L)\langle g_k,\cond{\hat{g}_k}{\parameter_{k}}-g_k\rangle+\frac{\alpha_k^2 L}{2}\cond{\nrm{\hat{g}_k-g_k}^2}{\parameter_k}.
    \end{aligned}
\end{equation}
Through the fact $|\langle x,y\rangle|\leq (\nrm{x}^2+\nrm{y}^2)/2$, it holds
\begin{equation}\label{eqn:sgd-s3}
\begin{aligned}
    -(\alpha_k-\alpha_k^2L)\iprod{g_k}{\cond{\hat{g}_k}{\parameter_{k}}-g_k}
    \leq& \frac{\alpha_k-\alpha_k^2L}{2}\nrm{g_k}^2
    +\frac{\alpha_k}{2}\nrm{\cond{\hat{g}_k}{\parameter_{k}}-g_k}^2.
\end{aligned}
\end{equation}
Therefore, we can plug in \eqref{eqn:sgd-s3} to \eqref{eqn:sgd-s2} and rearrange to get
\begin{equation}
    \label{eq:descent2}
    2\left(\Lcal_k-\cond{\Lcal_{k+1}}{\parameter_{k}}
    \right)
    \geq \alpha_k\nrm{g_k}^2
    -\alpha_k \nrm{\cond{\hat{g}_k}{\parameter_{k}}-g_k}^2-L\alpha_k^2 \cond{\nrm{\hat{g}_k-g_k}^2}{\parameter_k}.
\end{equation}
Thus, \eqref{eq:descent} holds when we substitute the error bounds in Lemma \ref{lem:gbiasVar} into \eqref{eq:descent2}.
\end{proof}

As $n$ goes towards positive infinity, the objective function decreases as an exact gradient descent. However, $n$ and $n_0$ are not too large because of the high sampling cost in practice. The error bounds studied in Section \ref{subsec:error} are valuable for our analysis of the stochastic optimization. Since $\Lcal(\parameter)$ is non-convex, we consider the convergence rate in terms of the expected norm of the gradient $\Ebb\norm{\nabla_{\parameter}\Lcal(\parameter)}^2$. Finally, we establish our first-order convergence results as follows.  
\begin{theorem}
    \label{thm:expt}
    Let Assumptions \ref{asm:wavefun},\ref{asm:local} and \ref{asm:unigap} hold and $\{\parameter_k\}$ be generated by Algorithm \ref{alg:VMC}. If the stepsize satisfies $\alpha_k\leq \frac{1}{2L}$, 
        then for any $K$, we have
        \begin{equation}
            \label{eq:converge1}
            \min_{1\leq k\leq K}\Ebb \|g(\parameter_k)\|^2 \leq O\left(\frac{1}{\sum_{k=1}^{K}\alpha_k}\right)+O\left(\frac{\sum_{k=1}^{K}\alpha_k^2}{n\sum_{k=1}^{K}\alpha_k}\right)+O\left(\frac{(1-\gamma)^{2n_0}}{n^2}\right),
        \end{equation}
        where $O(\cdot)$ hides constants $\gamma,M,B,C$ and retains $n,n_0$ and $K$. In particular, if the stepsize is chosen as  $\alpha_k=\frac{c\sqrt{n}}{\sqrt{k}}$ where $c\leq\frac{1}{2L} $, then we have
        \begin{equation}
            \label{eq:converge2}
            \min_{1\leq k\leq K}\Ebb \|g(\parameter_k)\|^2 \leq O\left(\frac{\log K}{\sqrt{nK}}\right)+O\left(\frac{(1-\gamma)^{2n_0}}{n^2}\right).
        \end{equation}
\end{theorem}
\begin{proof}
    


Lemma \ref{lem:decrease} suggests that, for any $k\geq 1$, 
\begin{equation*}
    \begin{aligned}
        \alpha_k\nrm{g(\parameter_k)}^2
        \leq & 2\left(\Lcal(\parameter_k)-\cond{\Lcal(\parameter_{k+1})}{\parameter_{k}}\right)+\alpha_k\cdot B_{n,n_0}^2+\alpha_k^2\cdot L V_{n,n_0}.%\\
        %B_{n,n_0}\leq & O\left(\frac{\rchi^{\frac{1}{2}}\gamma^{\frac{n_0}{2}}\sigma}{\sqrt{n}}\right)+O\left(\frac{\rchi\gamma^{n_0}}{n}\right) ,\\
        %V_{n,n_0}\leq & O\left(\frac{(1+\rchi\gamma^{n_0})\sigma^2}{n}\right)+O\left(\frac{1}{n^2}\right).
    \end{aligned}
\end{equation*}
Thus, taking total expectation and summing over $k=1,\cdots,K$ yields
\begin{align*}
    \left(\sum_{k=1}^{K}\alpha_k \right)\min_{1\leq k\leq K} \epct{\nrm{g(\parameter_k)}^2} &\leq \sum_{k=1}^{K}\alpha_k \epct{\nrm{g(\parameter_k)}^2}
    \\
    &\leq 2\left(\Lcal(\parameter_1)-\epct{\Lcal(\parameter_{K+1})}
    \right)
    +\sum_{k=1}^{K}\alpha_k^2 \cdot LV_{n,n_0}
    + \sum_{k=1}^{K}\alpha_k\cdot B_{n,n_0}^2\\
    &=O(1)+\sum_{k=1}^{K}\alpha_k^2 \cdot O\left(\frac{1}{n}\right)+\sum_{k=1}^{K}\alpha_k \cdot O\left(\frac{(1-\gamma)^{2n_0}  }{n^2}\right).
\end{align*}
Divide both sides by $\sum_{k=1}^{K}\alpha_k$, then \eqref{eq:converge1} holds. If we take $\alpha_k=\frac{c\sqrt{n}}{\sqrt{k}}$, \eqref{eq:converge2} is implied by
\begin{equation*}
        \sum_{k=1}^{K}\alpha_k=\sum_{k=1}^{K}\frac{c\sqrt{n}}{\sqrt{k}}=O(\sqrt{nK}),\quad \sum_{k=1}^{K}\alpha_k^2=\sum_{k=1}^{K}\frac{c^2n}{k}=O(n\log K).
\end{equation*}
This completes the proof.
\end{proof}

Theorem \ref{thm:expt} shows the convergence rate of the VMC method with certain choices of stepsizes. The convergence rate is related to $K$ and the sample size $n$ with an additional term from the bias. The burn-in period $n_0$ influences the bias. Theoretically, when $n_0$ is sufficiently large, the Markov chain becomes stationary and the bias is reduced to zero, regardless of the sample size $n$.

\section{Escaping from saddle points}
\label{sec:saddle}
In this section, we discuss how VMC escapes from saddle points. 
First, a second moment lower-bound for non-stationary Markov chains is developed to guarantee the efficient noise. Then, we verify that the correlated negative curvature condition is satisfied by VMC. Eventually, we demonstrate our convergence analysis of Algorithm \ref{alg:VMC} to reach approximate second-order stationary points in high probability.

% We first give the definition of kurtosis. Let $\sigma_2^{2}[X]:=\Ebb[\left(X-\Ebb[X]\right)^2]$ be the variance and $\sigma_4^4[X]:=\Ebb[\left(X-\Ebb[X]\right)^4]$ be the fourth central moment for any random variable $X$. The kurtosis of $X$ is the fourth standardized moment, defined as
% \begin{equation*}
%     \kappa[X]:=\frac{\sigma_4^4[X]}{\sigma_2^{4}[X]}=\frac{\Ebb[\left(X-\Ebb[X]\right)^4]}{\big(\Ebb[\left(X-\Ebb[X]\right)^2]\big)^2}.
% \end{equation*}
% Kurtosis means the "tailedness" of the probability distribution of a random variable. This number is related to the tails of the distribution.

To deal with the second-order structure, the following assumption is needed. 
\begin{assumption}
    \label{asm:sec} 
    (1) The Hessian matrix $H(\parameter)$ defined in \eqref{eq:Hess} is $\rho$-Lipschitz continuous, i.e.,
    \begin{equation*}
       \norm{H(\parameter_1)-H(\parameter_2)}\leq \rho\norm{\parameter_1-\parameter_2}, ~\forall \parameter_1,\parameter_2\in\Rbb^{d}.
   \end{equation*}
    
   (2) Let $v$ be the unit eigenvector with respect to the minimum eigenvalue of the Hessian matrix $H(\parameter)$. Let $\sigma_2^2(\parameter)$ be the variance of the local energy in the quantum system. There exists a constant $\eta>0$ such that 
   \begin{equation*}
    \label{eq:glowerbound}
    \Ebb_{\pi_{\parameter}}\left[\left(\oE_{\parameter}(\bmx)v^{T}\nabla_{\parameter}\log \Psi_{\parameter}(\bmx)\right)^2 \right]\geq \eta \sigma_{2}^2[E_{\parameter}],~\forall \parameter \in \Rbb^{d}.
   \end{equation*}

   (3) For any $\parameter\in \Rbb^d$, the ratio of the sub-exponential norm of the local energy to its variance has an upper bound $\kappa>0$, that is,
   \begin{equation*}
    \sup_{\parameter\in \Rbb^d}\frac{\norm{E_{\parameter}}_{\psi_1}}{\sigma_2[E_{\parameter}]}\leq \kappa.
   \end{equation*}
\end{assumption}

The first assumption is common to preform a second-order convergence analysis. The second assumption shows the relationship between the gradient $g(\parameter)=2\Ebb_{ \pi_{\parameter}}\lb\oE_{\parameter}(\bmx)\nabla_{\parameter}\log \Psi_{\parameter}(\bmx)\rb$ and a negative curvature direction $v$. It guarantees the variance of the stochastic gradient will not disappear along a negative curvature direction unless the local energy has a zero variance. We assume that the absolute angle between the gradient and the eigenvector of Hessian is expected to have a positive lower bound. 
The third one guarantees that the local energy will have a respectively light-tailed distribution. It holds for most of distributions and especially near the ground state of quantum many-body problems.

Besides, over Assumption \ref{asm:wavefun} and \ref{asm:local}, we can assume an upper bound of the stochastic gradient to simply our analysis:
\begin{equation}
    \label{eq:gradientbound}
    \begin{aligned}
        l_g:=\sup_{\parameter\in \Rbb^{d}}\sup_{\bmx\in \Xcal}\norm{\oE_{\parameter}(\bmx)\nabla_{\parameter}\log \Psi_{\parameter}(\bmx)}<+\infty.
        %\Ebb\norm{\hat{g}(\parameter;\mathbf{S})}^2\leq 2\norm{g(\parameter)}^2+2\Ebb\norm{\hat{g}(\parameter;\mathbf{S})-g(\parameter)}^2\leq 2l^2+2V_{n,n_0}=:l_g^2,
    \end{aligned}
\end{equation}
This treatment is legal because $\oE_{\parameter}$ is assumed to sub-exponential. Otherwise, the proof can be also established with similar techniques in section \ref{subsec:error}. The upper bound will simplify our proof in Lemma \ref{lem:R2}.

\subsection{The Correlated Negative Curvature condition}
\label{subsec:cnc}
We first prove a general lemma about the second moment for non-stationary Markov chains. The lemma shows that the second moment of the MCMC estimation has a positive lower-bound relying on the sample size $n$, the spectral gap $\gamma$ and $\Ebb_{\pi}[f^2]$. 
\begin{lemma}
    \label{lem:lowerbound}
    Let $\{X_i\}_{i= 1}^{n_0+n}$ be the Markov chain with the stationary distribution $\pi$ and the initial distribution $\nu$. Suppose it admits a absolute spectral gap $\gamma$ and $\rchi=\rchi(\nu,\pi)<+\infty$. The function $f$ has a finite fourth central moment $\sigma_4^4[f]:=\Ebb_{\pi}[\left(f-\Ebb_{\pi}[f]\right)^4]$. If $n\geq \frac{32}{\gamma^3}$ and $n_0\geq\frac{2}{\gamma}(\log \rchi+\log(\sigma_4[f]/\sigma_2[f])+\log n)$, it holds that  
\begin{equation}
    \label{eq:lowerbound}
    \Ebb_{\nu}\left[\left(\frac{1}{n}\sum_{i=n_0+1}^{n+n_0}f(X_i)\right)^2\right]\geq \frac{\gamma}{4n}\Ebb_{\pi}[f^2].
\end{equation}

\end{lemma}
\begin{proof}
    We denote $Z=\frac{1}{n}\sum_{i=n_0+1}^{n+n_0}f(X_i)$ and $c=\Ebb_\pi[f]$. Then it holds
    \begin{equation}
        \label{eq:lb1}
        \Ebb_\nu[Z^2]=c^2+\Ebb_\nu[(Z-c)^2]+2c(\Ebb_\nu[Z]-c)
        \geq \frac{c^2}{2}+\Ebb_\nu[(Z-c)^2]-2(\Ebb_\nu[Z]-c)^2.
    \end{equation}
    Notice the difference 
    \begin{equation}
        \label{eq:diff}
        \begin{aligned}
            \left| \Ebb_{\nu}\left[(Z-c)^2\right] - \Ebb_{\pi}\left[(Z-c)^2\right]\right|
             \leq (1-\gamma)^{n_0}\rchi \cdot \left(  \Ebb_{\pi}\left[(Z-c)^4\right] \right)^{\frac{1}{2}}
        \end{aligned}
    \end{equation}
    where the inequality changes the measure as in \eqref{eq:change-measure}. It follows from the Cauchy-Schwarz inequality that
    \begin{align*}
        \Ebb_{\pi}\left[(Z-c)^4\right]
        =&~\Ebb_{\pi}\left[\left(\frac{1}{n}\sum_{i=n_0+1}^{n+n_0} f(X_i)-\Ebb_\pi[f]\right)^4\right] \\
        \leq&~ \Ebb_{\pi}\left[\frac{1}{n}\sum_{i=n_0+1}^{n+n_0} \left(f(X_i)-\Ebb_\pi[f]\right)^4\right] = \Ebb_{X\sim \pi}\left[\left(f(X)-\Ebb_\pi[f]\right)^4\right] = \sigma_4^4[f],
    \end{align*}
    where the second equality is because $\pi$ is the stationary distribution of the Markov chain.
    Therefore, we derive that
    \begin{equation}
        \label{eq:lb2}
        \Ebb_{\nu}\left[(Z-c)^2\right] \geq \Ebb_{\pi}\left[\left(\frac{1}{n}\sum_{i=n_0+1}^{n+n_0} f(X_i)-\Ebb_\pi[f]\right)^2\right] - (1-\gamma)^{n_0}\rchi\sigma_4^2[f].
    \end{equation}
    It holds from Theorem 3.1 in \cite{paulin2015concentration} that
    \begin{equation}
        \label{eq:lb3}
        \begin{aligned}
            \left|\Ebb_{\pi}\left[\left(\frac{1}{n}\sum_{i=n_0+1}^{n+n_0} f(X_i)-\Ebb_\pi[f]\right)^2 \right]-\frac{\sigma^2_{asy}[f]}{n}\right|\leq \frac{16 \sigma^2_2[f]}{\gamma^2n^2},
        \end{aligned}
    \end{equation}
    where $\sigma^2_{asy}[f]:=\langle f,[2(I-(\oP-\pi))^{-1}-I]f\rangle_{\pi}$ on page 24 of \cite{paulin2015concentration}. Using the spectral method therein, we obtain that 
    \begin{equation}
        \label{eq:lb4}
        \begin{aligned}
        \sigma^2_{asy}[f]&=\langle f,[2(I-(\oP-\pi))^{-1}-I]f\rangle_{\pi}=\norm{(I-\oP)^{-1}f}_{\pi}^2-\norm{P(I-P)^{-1}f}_{\pi}^2\\
        &\geq (1-(1-\gamma)^2)\norm{(I-\oP)^{-1}f}_{\pi}^2\geq \frac{\gamma\sigma^2_2[f]}{2}.
    \end{aligned}
    \end{equation}
    Next, we bound the term $(\Ebb_\nu[Z]-c)^2$ in \eqref{eq:lb1}. By Lemma \ref{lem:BernBiasVar}, it holds that
    \begin{equation}
        \label{eq:lb5}
        \abs{ \Ebb_\nu[Z]-c } = \abs{ \frac{1}{n}\sum_{i=n_0+1}^{n+n_0} \Ebb_{\pi} \left[f(X_i)\right]-\Ebb_\pi[f] } \leq \frac{(1-\gamma)^{n_0} \rchi \sigma_2[f] }{ \gamma n}.
    \end{equation}
    Finally, combining \eqref{eq:lb1}, \eqref{eq:lb2}, \eqref{eq:lb3}, \eqref{eq:lb4} and \eqref{eq:lb5} yields
    \begin{align*}
        \Ebb_{\nu}\left[\left(\frac{1}{n}\sum_{i=n_0+1}^{n+n_0}f(X_i)\right)^2\right]
        &\geq \frac{1}{2}\left(\Ebb_\pi[f]\right)^2 + \frac{\gamma\sigma^2_2[f]}{2n}-\frac{16\sigma^2_2[f]}{\gamma^2n^2}- (1-\gamma)^{n_0}\rchi\sigma_4^2[f] -\frac{2(1-\gamma)^{2n_0}\rchi^2\sigma^2_2[f]}{\gamma^2 n^2}\\
        &\geq \sigma^2_2[f]\left(\frac{\gamma}{n}-\frac{16}{\gamma^2n^2}-\frac{2(1-\gamma)^{2n_0}\rchi^2}{\gamma^2 n^2}-(1-\gamma)^{n_0}\rchi\cdot \frac{\sigma^2_4[f]}{\sigma^2_2[f]}\right)+ \frac{1}{2}\left(\Ebb_\pi[f]\right)^2
        \\
        &\geq \frac{\gamma}{4n}\sigma^2_2[f] + \frac{1}{2}\left(\Ebb_\pi[f]\right)^2
        \geq \frac{\gamma}{4n}\Ebb_{\pi}[f^2],
    \end{align*}
    where the third inequality holds when $n\geq \frac{32}{\gamma^3}$ and $n_0\geq\frac{2}{\gamma}(\log \rchi+\log(\sigma_4[f]/\sigma_2[f]) + \log n)$.
\end{proof}

The CNC condition \eqref{eq:cnc} in the following is of vital importance to analyze the saddle escaping property in VMC. That is one of the reason why stochastic optimization algorithms can escape from saddle points. If the CNC condition does not hold, the noise may not be strong enough to escape along the descent direction. The following lemma shows that the stochastic gradient $\hat{g}(\parameter;\mathbf{S})$ defined in \eqref{eq:approxgrad} satisfies CNC condition.
\begin{lemma}
    \label{lem:cnc}
    Let Assumptions \ref{asm:wavefun},\ref{asm:local}, \ref{asm:unigap} and  \ref{asm:sec} hold. Then, for the unit eigenvector $v$ with respect to the minimum eigenvalue of the Hessian, there exists $\mu=\frac{\eta \gamma}{16n}$ such that it holds 
    \begin{equation}
        \label{eq:cnc}
        \Ebb_{\nu}\left[\left(v^{\T}\hat{g}(\parameter;\mathbf{S})\right)^2\right]\geq \mu\sigma_2^2[E_{\parameter}],\quad \forall \parameter\in \Rbb^d,
    \end{equation}
    as long as $n\geq \Omega\left(\frac{1}{\eta\gamma^3}\right)+\tilde{\Omega}\left( \frac{\kappa B }{\sqrt{\eta}\gamma^2}\right)$  and $n_0\geq\frac{2}{\gamma}(\log \rchi+\log(2\kappa) + \log n)$ where $\Omega(\cdot)$ and $\tilde{\Omega}(\cdot)$ hide constants $M,B,C,\rchi$. 
\end{lemma}
\begin{proof}
   The MCMC estimate of the gradient $\hat{g}(\parameter,\mathbf{S})$ is computed by \eqref{eq:approxgrad}. For convenience, fix a unit vector $v$ and we denote empirical variables by $E_i=E_{\parameter}(\bmx_i),Y_i=v^{\T}\nabla_{\parameter}\log \Psi_{\parameter}(\bmx_i)$ and $\overline{E},\overline{Y}$ represents the average of $E_i,Y_i$. Let stationary variables $E=E_{\parameter}(\bmx)$ and $Y=v^{\T}\nabla_{\parameter}\log \Psi_{\parameter}(\bmx)$ with a dependent variable $\bmx\sim \pi_{\parameter}$, then it holds that
    \begin{align*}
        v^{\T}\hat{g}(\parameter,\mathbf{S})&=\frac{2}{n}\sum_{i=n_0+1}^{n_0+n}\big(E_{\parameter}(\bmx_i)-\frac{1}{n}\sum_{i=n_0+1}^{n_0+n}E_{\parameter}(\bmx_i)\big)v^{\T}\nabla \log \Psi_{\parameter}(\bmx_i)\\
        &=\frac{2}{n}\sum_{i=n_0+1}^{n_0+n}E_iY_i-2\left(\overline{E}-\Ebb E+\Ebb E\right)\left(\overline{Y}-\Ebb Y+\Ebb Y\right)\\
        &=\underbrace{\frac{2}{n}\sum_{i=n_0+1}^{n_0+n}(E_i-\Ebb E)(Y_i-\Ebb Y) }_{Z_1}-\underbrace{2\left(\overline{E}-\Ebb E\right)\left(\overline{Y}-\Ebb Y\right)}_{Z_2}.
    \end{align*}
To obtain the lower bound, we have
\begin{align*}
    \Ebb_{\nu}\left[(v^{\T}\hat{g}(\parameter,\mathbf{S}))^2\right]=\Ebb_{\nu} \left[(Z_1-Z_2)^2\right]\geq \frac{1}{2}\Ebb_{\nu} \left[Z_1^2\right]-\Ebb_{\nu} \left[Z_2^2\right].
\end{align*}
Since $e^{x}> x^4/8$ when $x\geq 0$, it holds that $\sigma_{4}[E]\leq 2 \norm{E}_{\psi_1}$. It implies the kurtosis of the local energy $\sigma_4[E]/\sigma_2[E]$ is less than $2\kappa$.
Then, when $n\geq \frac{32}{\gamma^3}$ and $n_0\geq\frac{2}{\gamma}(\log \rchi+\log(2\kappa) + \log n)$, it follows from Lemma \ref{lem:lowerbound} that 
\begin{equation}
    \label{eq:Z1}
    \begin{aligned}
        \Ebb_{\nu}\left[ Z_1^2\right]& \geq \frac{\gamma}{4n}\Ebb_{\pi}[(E-\Ebb E)^2 (Y-\Ebb Y)^2]{\geq}\frac{\eta\gamma \sigma_2^2[E_{\parameter}]}{4n}.
    \end{aligned}
\end{equation}

% We introduce the Cramer-Rao's inequality in the multivariate case. Suppose $\parameter\in \Rbb^{d}$ is a parameter with probability density function $p_{\parameter}(x)$ and its Fisher information matrix $F(\parameter)$. Let $T(X)$ be an estimator of any function of parameters and denote its expectation $h(\parameter)=\Ebb_{X\sim p_{\parameter}}[T(X)]$.Then, the Cramer-Rao's inequality states that the variance of $T(X)$ satisfies
% \begin{equation*}
%     \mathrm{Var}_{p_{\parameter}}[T(X)]\geq (\nabla_{\parameter}h(\parameter))^{\T} F(\parameter) (\nabla_{\parameter}h(\parameter)).
% \end{equation*}
% As $(X-\Ebb X)(Y-\Ebb Y)$ is an estimator of $v^{\T}\nabla_{\parameter}\Lcal(\parameter)$ whose gradient is $H(\parameter)v$, it follows from the Cramer-Rao's inequality that
% \begin{equation}
%     \label{eq:CRineq}
%     \mathrm{Var}\left(XY- \Ebb Y \cdot X-\Ebb X \cdot Y\right)\geq v^{\T}H(\parameter) F^{-1}(\parameter) H(\parameter) v.
% \end{equation}
% When we take $v$ as the unit eigenvector corresponding to the minimum eigenvalue of the Hessian $\nabla_{\parameter}^2\Lcal(\parameter)$, together with \eqref{eq:Z1}, \eqref{eq:CRineq} implies that
% \begin{equation}
%     \Ebb\left[ Z_1^2\right]\geq \frac{4\zeta\lambda_0^2}{n M_F}.
% \end{equation}

Besides, the upper bound of $ \Ebb_{\nu} \left[Z_2^2\right]$ can be obtained by the Cauchy-Schwarz inequality that
\begin{equation}
    \label{eq:z2}
    \begin{aligned}
        \Ebb_{\nu} [Z_2^2]&=4\Ebb_{\nu}[(\bar{E}-\Ebb E)^2(\bar{Y}-\Ebb Y)^2]\leq 4\sqrt{\Ebb_{\nu}[(\bar{E}-\Ebb E)^4]\cdot \Ebb_{\nu}[(\bar{Y}-\Ebb Y)^4]}.
    \end{aligned}
\end{equation}
By the result in the proof of Lemma \ref{lem:BernBiasVar}, we have
\begin{equation}
    \label{eq:Xsquare}
    \Ebb_\nu[\abs{\bar{E}-\Ebb E}^4]\leq \bigO{ \frac{\sigma_2^4[E_{\parameter}]}{n^2\gamma^2}+\frac{M^4(\log n)^8}{n^{4}\gamma^4} }.
\end{equation}
% To estimate fourth central moment, we use the high probability bound in \eqref{eq:highprobbound}. Reusing notations in Lemma \ref{lem:BernBiasVar}, we write that for any $\delta>0$, it holds that
% \begin{equation*}
%     \Pbb\left(\abs{\Delta}\leq s\right)\geq 1-\delta,\quad s\geq\max\left\{\frac{8\sigma_2[f]\sqrt{\log(4C/\delta)}}{\sqrt{n \gamma}},\frac{40M\left(\log(4Cn/\delta)\right)^2 }{n\gamma}\right\}.
% \end{equation*}
% By computing the following integral, we obtain the fourth moment bound
% \begin{equation}
%     \label{eq:fourthmoment}
%     \Ebb_\nu[\abs{\Delta}^4]=\int_{0}^{+\infty}4 s^3 \Pbb(\abs{\Delta}>s)ds\leq \frac{8^5C\sigma_2^4[f]}{n^2\gamma^2}+\frac{10080 C(80M)^4}{n^{3}\gamma^4}.
% \end{equation}
% Under Assumption \ref{asm:sec}, it follows from \eqref{eq:fourthmoment} and $n\geq \frac{4096}{\gamma^3}$ that
% \begin{equation}
%     \label{eq:Xsquare}
%     \Ebb(\bar{X}-\Ebb X)^2\leq \frac{8^5C\sigma_2^4[E_{\parameter}]}{n^2\gamma^2}+\frac{10080 C(80M)^4}{n^{3}\gamma^4}\leq \frac{40^5\kappa^4C\sigma_2^4[E_{\parameter}]}{n^2\gamma^2}.
% \end{equation}
Similarly, notice that $\abs{Y}=\abs{v^{\T}\nabla_{\parameter}\log \Psi_{\parameter}(\bmx)}\leq \norm{v}\norm{\nabla_{\parameter}\log \Psi_{\parameter}(\bmx)}\leq B$, we can apply \cite[Theorem 12]{fan2021hoeffding} to obtain 
\begin{equation*}
    \Pbb\left(|\bar{Y}-\Ebb Y|\geq t \right)\leq 2C \exp\left(-\frac{n\gamma t^2}{4B^2} \right).
\end{equation*}  
Using the similar technique in the proof of Lemma \ref{lem:Bern-exp}, we derive
\begin{equation}
    \label{eq:Ysquare}
    \Ebb_{\nu}[|\bar{Y}-\Ebb Y|^4]\leq \bigO{ \frac{B^4}{n^2\gamma^2} }.
\end{equation}
% \begin{equation*}
%     \Pbb\left(|\bar{Y}-\Ebb Y|\geq t \right)\leq 2C \exp\left(-\frac{n\gamma t^2}{4B^2} \right).
% \end{equation*}  
% It implies that
% \begin{equation}
%     \label{eq:Ysquare}
%     \Ebb_{\nu}[|\bar{Y}-\Ebb Y|^4]=\int_{0}^{+\infty}4t^3\Pbb\left(|\bar{Y}-\Ebb Y|\geq t \right)dt\leq \frac{64C B^4}{n^2\gamma^2}.
% \end{equation}
Plugging \eqref{eq:Xsquare} and \eqref{eq:Ysquare} into \eqref{eq:z2} yields
\begin{equation*}
    \Ebb_{\nu} [Z_2^2]\leq 4\sqrt{\Ebb_{\nu}[(\bar{E}-\Ebb E)^4]\cdot \Ebb_{\nu}[(\bar{Y}-\Ebb Y)^4]}
    \leq \bigO{ \frac{\sigma_2^{2}[E_{\parameter}]B^2}{n^2\gamma^2} + \frac{(\log n)^4M^2B^2}{n^3\gamma^3} }.
    % \leq \frac{409600\kappa^2CB^2\sigma_2^{2}[E_{\parameter}]}{n^2\gamma^2}\leq \frac{\eta\gamma \sigma_2^2[E_{\parameter}]}{32n},
\end{equation*}
Therefore, $\Ebb_{\nu} [Z_2^2]\leq \frac{\eta\gamma \sigma_2^2[E_{\parameter}]}{32n}$ as long as $n\geq \Omega\left(\frac{1}{\eta\gamma^3}\right)+\tilde{\Omega}\left( \frac{\kappa B }{\sqrt{\eta}\gamma^2}\right)$.
% when $n\geq \frac{c\kappa^2CB^2}{\eta\gamma^3}$ with a constant $c$. 
Finally, combining our discussion of $Z_1$ and $Z_2$, we derive that
\begin{equation}
    \Ebb_{\nu}\left[(v^{\T}\hat{g}(\parameter,\mathbf{S}))^2\right]\geq \frac{1}{2}\Ebb_{\nu} \left[Z_1^2\right]-\Ebb_{\nu} \left[Z_2^2\right]\geq \frac{\eta\gamma \sigma_2^2[E_{\parameter}]}{32n}.
\end{equation}
This completes the proof.
\end{proof}

\subsection{Second order stationary}

As discussed above, VMC performs a special SGD with the biased gradient estimator. While the convergence of SGD is well-understood for convex functions, the existence of saddle points and local minimum poses challenges for non-convex optimization. Since the ordinary GD  often stucks near the saddle points, the additional noise within SGD allows it to escape from saddle points. By this way, the simple SGD without explicit improvement is proven to have second-order convergence. 

We now give the definition of approximate second-order stationary point.
\begin{definition}[Approximate second-order stationary point]
    Given a function $\Lcal$, an $(\epsilon_g,\epsilon_h)$ approximate second-order stationary point $\parameter$ of $\Lcal$ is defined as 
    \begin{equation*}
        \norm{g(\parameter)}\leq \epsilon_g,\quad \lambda_{\min}\left(H(\parameter)\right)\leq -\epsilon_h,
    \end{equation*}
    where $g$ and $H$ denote the gradient and Hessian of $\Lcal$ respectively.
\end{definition}
    
If $\epsilon_g=\epsilon_h=0$, the point $\parameter$ is a second-order stationary point. The second order analysis contributes to understand how VMC performs well in solving eigenvalue problems.

We simultaneously consider the convergence of the VMC method from the perspective of optimization and  eigenvalue equations. When the exact wavefunction $\Psi$ satisfies $\Hcal \Psi=\lambda \Psi$, the variance of the local energy $\sigma_2^2[E]$ is equal to zero. Naturally, we define $\epsilon$-variance points as a criteria for the trial wavefunction approximation.
\begin{definition}[$\epsilon$-variance point]
    For the optimization problem \eqref{loss:vmc}, we call $\parameter$ an $\epsilon$-variance point if the local energy satisfies $\sigma_2^2[E_{\parameter}]< \epsilon$.
\end{definition}
For notational simplification, we take an abbreviation $\sigma_2^2=\sigma_2^2[E_{\parameter}]$ in this subsection.

To escape from saddle points, a new stepsize schedule is established for Algorithm \ref{alg:VMC}. Given a period $T$, note that $\alpha$ and $\beta$ are the constant stepsizes with $\beta>\alpha>0$, with values given in Table \ref{table:1}. Within one iteration period of $T$ steps, we adopt a large stepsize $\beta$ at the beginning of the period and a small one $\alpha$ at the other $T-1$ iterations, that is,
\begin{equation}
    \label{eq:schedule}
    \alpha_k=\begin{cases}
        \alpha, \quad k (\mathrm{mod}T)\not =0,\\
        \beta,  \quad k (\mathrm{mod}T) =0.
    \end{cases}
\end{equation}
It will be shown that the schedule can be suitably designed to achieve sufficient descent in one period.

Suppose the total number of iterations $K$ is a multiple of $T$ and there are $K/T$ periods. We denote $\tparameter_m=\parameter_{m\cdot T}$ for $m=0,1,\dots,K/T$ in each period. For some given $\epsilon>0$, we consider four regimes of the iterates $\{\tparameter_m\}$ as follow,
\begin{align}
    \Rcal_1&:=\left\{\parameter\Big| \norm{g(\parameter)}\geq \epsilon\right\},\label{eq:R1}\\
    \Rcal_2&:=\left\{\parameter\Big| \norm{g(\parameter)}< \epsilon,~\lambda_{\min}\left(H(\parameter)\right)\leq -\epsilon^{\frac{1}{4}} ,~~\sigma_2^2[E_{\parameter}]\geq \epsilon^{\frac{1}{2}} \right\},\label{eq:R2}\\
    \Rcal_3&:=\left\{\parameter\Big| \norm{g(\parameter)}< \epsilon,~\lambda_{\min}\left(H(\parameter)\right)>-\epsilon^{\frac{1}{4}} ,~\mathrm{or}~\sigma_2^2[E_{\parameter}]<\epsilon^{\frac{1}{2}} \right\},\label{eq:R3}
\end{align}
$\Rcal_1$ stands for the regime with a large gradient, where the stochastic gradient works effectively. When the iterate lies in $\Rcal_2$, despite being close to a first-order stationary point, the CNC condition mentioned in section \ref{subsec:cnc} guarantees a decrease after $T$ iterations under our schedule. $\Rcal_3$ is a regime of $(\epsilon,\epsilon^{1/4})$ approximate second order stationary points or $\epsilon^{1/2}$ variance points. We need to show Algorithm \ref{alg:VMC} will reach $\Rcal_3$ with high probability, that is, converge to approximate second order stationary points.

The analysis below relies on a particular choice of parameters, whose values satisfy the following lemmas and the main theorem. For ease of verification, the choice of parameters is collected in Table \ref{table:1}. 
\begin{table}[ht]
    \centering
    \renewcommand{\arraystretch}{1.8}
    \begin{tabular}{|c|c|c|c|c|c|}
    \hline
     Parameter& Value & Order & Conditions  \\ \hline
     $\beta $ & $\frac{ \delta\epsilon^2}{192 l_g\rho L V_{n,n_0}}$ & $O(\epsilon)$ &     \eqref{eq:R1conditions}, \eqref{eq:conditions_1} \\ \hline
     $\alpha$& $\frac{\beta}{\sqrt{T}}$ & $  O\left(\epsilon^{9/4}\log^{-1}\left(\frac{1}{\epsilon}\right)\right)$&  \eqref{eq:conditions_1}  \\ \hline
     $\Lthre$&$\frac{\beta \epsilon^2 }{192 \rho l_g}$ & $ O(\epsilon^{3})$ &  \eqref{eq:R1conditions},\eqref{eq:conditions_1}  \\ \hline
     $n$&  $\frac{\eta \gamma }{64\epsilon}$ & $O\left(\epsilon^{-1}\right)$ &  Lemma \ref{lem:cnc}  \\ \hline
     $n_0$&  - & $O\left(\log\left(\frac{1}{\epsilon}\right)\right)$ &  \eqref{eq:BV}, Lemma \ref{lem:cnc}  \\ \hline
     $B_{n,n_0}$ & $\sqrt{\frac{\epsilon^{2} }{96 T^{3/2} l_g\rho}} $  & $ O(\epsilon^{3})$ &  \eqref{eq:R1conditions}, \eqref{eq:conditions_1}  \\ \hline
     $T$& $\frac{1}{\beta^2\epsilon^{1/2}}
     \log^2\left(\frac{\rho l_g L V_{n,n_0}}{\mu \delta \epsilon} \right)$ &  $ O\left(\epsilon^{-5/2}\log^2\epsilon\right)$& \eqref{eq:Tconst}    \\ \hline
     $K$& $\frac{2[\Lcal(\parameter_0)-\Lcal^{*}]T}{\delta \Lthre}$ & $ O\left(\epsilon^{-11/2}\log^2\epsilon\right)$ &   \eqref{eq:Kconst}  \\ \hline
    \end{tabular}
    \caption{List of parameter values used in the convergence analysis.}
    \label{table:1}
\end{table}

With a large gradient, it is easy to show a sufficient decrease of the objective function value. We have analyzed the decrease for each iteration in Lemma \ref{lem:decrease}, and it follows a similar argument for the biased SGD in VMC.
\begin{lemma}
    \label{lem:R1}
    Suppose that $\tparameter_m$ lies in $\Rcal_1$ defined in \eqref{eq:R1} and Algorithm \ref{alg:VMC} updates with the schedule \eqref{eq:schedule} and parameters in Table \ref{table:1}, then the expected value of $\Lcal(\tparameter_{m+1})$ taken over the randomness of $\{\parameter_{k}\}_{k=m\cdot T+1}^{(m+1)\cdot T}$ decreases as 
    \begin{equation}
        \Lcal(\tparameter_{m})-\Ebb[\Lcal(\tparameter_{m+1})|\tparameter_{m}]\geq \Lthre.
    \end{equation}
\end{lemma}
\begin{proof}
    We first decompose the difference of the expected function value into each iteration,
    \begin{equation}
        \label{eq:period-decrease}
        \Lcal(\tparameter_m)-\Ebb[\Lcal(\tparameter_{m+1})|\tparameter_{m}]=\sum_{p=0}^{T -1}\epct{\Lcal(\parameter_{m\cdot T +p})-\epct{\Lcal(\parameter_{m\cdot T +p+1})}\bigg|\parameter_{m\cdot T +p}},
    \end{equation}
    where $\epct{\Lcal(\parameter_{m\cdot T})|\parameter_{m\cdot T }}=\Lcal(\tparameter_{m})$ due to the definition $\tparameter_{m}=\parameter_{m\cdot T}$. Using the choice $\beta^2= T\alpha^2$ in Table \ref{table:1}, it follows from \eqref{eq:period-decrease} that
    \begin{equation}
        \label{eq:R1ineq}
        \begin{aligned}
            2\left(\Lcal(\tparameter_m)-\Ebb[\Lcal(\tparameter_{m+1})|\tparameter_{m}]\right)\geq &\beta\norm{g(\tparameter_m)}^2-\beta B_{n,n_0}^2-\beta^2LV_{n,n_0}\\
            &\quad -(T-1)\left(\alpha B_{n,n_0}^2+\alpha^2LV_{n,n_0} \right)\\
            \geq &\beta\norm{g(\tparameter_m)}^2-2T\alpha B_{n,n_0}^2-2\beta^2LV_{n,n_0},
        \end{aligned}
    \end{equation}
    where the first inequality is by Lemma \ref{lem:decrease} for $p=0,\dots,T-1$ and the second is by the direct substitution.
    Then, by the choice of $\beta,n_0,\Lthre$ in Table \ref{table:1}, these conditions hold that
    \begin{equation}
        \label{eq:R1conditions}
        \beta\leq \frac{\epsilon^2}{8LV_{n,n_0}},\quad B_{n,n_0}^2\leq \frac{\epsilon^2}{8\sqrt{T}},\quad \Lthre \leq \frac{\beta\epsilon^2}{4}.
    \end{equation}
    As $\tparameter_m$ lies in $\Rcal_1$, which means $\norm{\nabla_{\parameter}\Lcal(\tparameter_m)}\geq \epsilon$, we plug \eqref{eq:R1conditions} into \eqref{eq:R1ineq} and obtain that
    \begin{equation}
        \begin{aligned}
            2\left(\Lcal(\tparameter_m)-\Ebb[\Lcal(\tparameter_{m+1})|\tparameter_{m}]\right)\geq &\beta \epsilon^2-2\beta\sqrt{T}B_{n,n_0}^2-2\beta LV_{n,n_0}\cdot \frac{\epsilon^2}{8LV_{n,n_0}}\\
            \geq& \beta \epsilon^2-\frac{\beta \epsilon^2}{4}-\frac{\beta \epsilon^2}{4}= \frac{\beta\epsilon^2}{2}\geq 2\Lthre.
        \end{aligned}
    \end{equation}
    This completes the proof.
\end{proof}

Near the saddle points, the classical GD gets stuck if the gradient is orthogonal to the negative curvature direction. However, the stochastic gradient with the CNC condition has inherent noise along the negative curvature direction. Under our stepsize schedule \eqref{eq:schedule}, the objective function value can have sufficient decrease after a period.
\begin{lemma}
    \label{lem:R2}
    Suppose $\tparameter_m$ lies in $\Rcal_2$ defined in \eqref{eq:R2} and Algorithm \ref{alg:VMC} updates with the schedule \eqref{eq:schedule} and parameters in Table \ref{table:1}, then the expected value of $\Lcal(\tparameter_{m+1})$ taken over the randomness of $\{\parameter_{k}\}_{k=m\cdot T+1}^{(m+1)\cdot T}$ decreases as 
    \begin{equation}
        \Lcal(\tparameter_{m})-\Ebb[\Lcal(\tparameter_{m+1})|\tparameter_{m}]\geq \Lthre.
    \end{equation}
\end{lemma}
\begin{proof}
\newcommand{\asumgh}{A_1}
    
    The proof is by contradiction. Without loss of generality, we suppose that $m=0$ in this proof and denote
    \begin{equation*}
        \Lcal_p=\Lcal\left( \parameter_p\right),~~g_{p}=\nabla_{\parameter}\Lcal\left( \parameter_p\right),~~
            \hat{g}_{p}=\hat{g}\left( \parameter_p,\mathbf{S}_{p}\right),~~ H_p=\nabla^2_{\parameter}\Lcal\left( \parameter_p\right)
    \end{equation*}
    for $p=0,\dots,T-1$. Every expectation in this proof is taken over all existed $\parameter_{p}$ in every formula. We assume the expected function value decreases by no more than $\Lthre$, i.e.,
    \begin{equation}
        \label{eq:fakedecrease}
        \Lcal_0-\epct{\Lcal_{T }}< \Lthre.
    \end{equation}
    We proceed to show that the assumption \eqref{eq:fakedecrease} is invalid.

    We start with estimating the expected distance between $\parameter_0$ and $\parameter_p$ for $p=1,\dots,T-1$. By Lemma \ref{lem:decrease}, it holds
    \begin{equation}
        \label{eq:R2dec1}
        \begin{aligned}
            2\left(\Lcal_0-\epct{\Lcal_{T }}\right)\geq & \beta\norm{g_0}^2-\beta B^2_{n,n_0}-\beta^2 L V_{n,n_0}
            +\sum_{h=1}^{T -1}\left(\alpha \Ebb\norm{g_h}^2-\alpha B^2_{n,n_0}-\alpha^2 L V_{n,n_0}\right)\\
            \geq & \beta\norm{g_0}^2 + \alpha \sum_{h=1}^{T-1}\Ebb\norm{g_h}^2- 2T\alpha B^2_{n,n_0}-2\beta^2LV_{n,n_0},
        \end{aligned}
    \end{equation}
    where the first inequality is derived similarly to \eqref{eq:R1ineq}, and the second inequality is due to $\beta=\sqrt{T}\alpha$.
    Together with the assumption \eqref{eq:fakedecrease}, it follows that
    \begin{equation}
        \label{eq:R2dec2}
        \beta\norm{g_0}^2+\alpha\sum_{h=1}^{T-1}\Ebb\norm{g_h}^2\leq 2\Lthre+2T\alpha B^2_{n,n_0}+2\beta^2 L V_{n,n_0}=:\asumgh.
    \end{equation}
    A direct implication of \eqref{eq:R2dec2} is that
    \begin{equation}
        \label{eq:R2dec3}
        \beta\norm{g_0}^2\leq \asumgh, \qquad
        \alpha\Ebb\norm{\sum_{h=1}^{p}g_h}^2\leq p\alpha\sum_{h=1}^{p}\Ebb\norm{g_h}^2\leq p\asumgh.
    \end{equation}
    
    We proceed to bound $\parameter_{p+1}-\parameter_0=\beta \hat{g}_0+\alpha\sum_{h=1}^{p}\hat{g}_h$ as follows. Firstly,
    \begin{equation}
        \label{eq:distdecomp2}
        \begin{aligned}
            \Ebb\norm{\sum_{h=1}^{p}(\hat{g}_h-g_h)}^2
            &= \sum_{h=1}^{p}\Ebb\norm{\hat{g}_h-g_h}^2+2\sum_{1\leq h<l\leq p}\Ebb\left\langle \hat{g}_h-g_h,\hat{g}_l-g_l\right\rangle\\
            &= \sum_{h=1}^{p}\Ebb\norm{\hat{g}_h-g_h}^2+2\sum_{1\leq h<l\leq p}\Ebb\left\langle \hat{g}_h-g_h,\Ebb[\hat{g}_l|\parameter_l]-g_l\right\rangle\\
            &\leq pV_{n,n_0}+p(p-1)B_{n,n_0}\sqrt{V_{n,n_0}}
            \leq 2pV_{n,n_0}+p^3B_{n,n_0}^2,
        \end{aligned}
    \end{equation}
    where the second equality is because we can take conditional expectation on $\parameter_l$ first, the following inequality holds from the result $\Ebb\norm{\hat{g}_h-g_h}^2\leq V_{n,n_0}$ and $\norm{\Ebb[\hat{g}_l|\parameter_l]-g_l}\leq B_{n,n_0}$ in Lemma \ref{lem:gbiasVar}, and the last inequality is due to AM-GM inequality. Therefore, we can bound
    \begin{equation}
        \label{eq:distbound}
        \begin{aligned}
            \Ebb\norm{\parameter_{p+1}-\parameter_0}^2
            &=\Ebb\norm{\beta\hat{g}_0+ \alpha\sum_{h=1}^{p}\hat{g}_h}^2\\
            &=\Ebb\norm{\beta g_0+\beta(\hat{g}_0-g_0)+ \alpha\sum_{h=1}^{p}g_h+\alpha\sum_{h=1}^{p}(\hat{g}_h-g_h)}^2\\
            &\leq \underbrace{ 4\beta^2\norm{g_0}^2 }_{\eqref{eq:R2dec3}}
            +\underbrace{ 4\beta^2\Ebb\norm{\hat{g}_0-g_0}^2 }_{\text{\cref{lem:gbiasVar}}}
            +\underbrace{ 4\alpha^2\Ebb\norm{\sum_{h=1}^{p}g_h}^2 }_{\eqref{eq:R2dec3}}
            +\underbrace{ 4\alpha^2\Ebb\norm{\sum_{h=1}^{p}(\hat{g}_h-g_h)}^2 }_{ \eqref{eq:distdecomp2} }\\
            &\leq 4\beta \asumgh + 4\beta^2 V_{n,n_0} + 4\alpha p \asumgh + 2\alpha^2(2pV_{n,n_0}+p^3B_{n,n_0}^2)\\
            &\leq 4\beta \asumgh + 8\beta^2 V_{n,n_0} + 4 p (\alpha\asumgh+ \alpha^2 T^2 B_{n,n_0}^2 ),
        \end{aligned}
    \end{equation}
    where the second inequality is due to \eqref{eq:R2dec3}, Lemma \ref{lem:gbiasVar} and \eqref{eq:distdecomp2}, and the final follows from $p\alpha^2\leq T\alpha^2 = \beta^2$. Thus, we can take 
    \begin{align*}
        A_2&:=8\alpha\Lthre+16\alpha^2 T^2B_{n,n_0}^2+ 8\alpha\beta^2 LV_{n,n_0},\\
        A_3&:=8\beta\Lthre+8\alpha\beta T B_{n,n_0}^2+ 16\beta^2 V_{n,n_0},
    \end{align*}
    and then
    \begin{align}\label{eq:dtheta-linear}
        \Ebb\norm{\parameter_{p}-\parameter_0}^2 \leq (p-1)A_2+A_3.
    \end{align}
    When we take $p=T-1$, \eqref{eq:distbound} shows that the expected distance between $\parameter_0$ and $\parameter_T$ is bounded by a quadratic function of $T$.

    We further prove the expected distance between $\parameter_0$ and $\parameter_T$ grows at least exponentially for $T$, leading to a contradiction. Since $\parameter_{p}$ stays close to $\parameter_0$, the quadratic Taylor approximation of the function $\Lcal$ at $\parameter_0$ is introduced as
    \begin{equation*}
        Q(\parameter):=\Lcal_0+g_0^{\T}(\parameter-\parameter_0)+\frac{1}{2}(\parameter-\parameter_0)^{\T}H_0(\parameter-\parameter_0).
    \end{equation*}
    We denote $Q_p=Q(\parameter_p)$ and $q_p=\nabla_{\parameter}Q(\parameter_p)=g_0+H_0(\parameter_p-\parameter_0)$ for $p=0,\dots,T-1$. Using the Taylor approximation is firstly proposed in \cite{ge2015escaping}. As $\Lcal$ is twice-differentiable with a $\rho$-Lipschitz Hessian, \cite[Lemma 1.2.4]{nesterov2003introductory} gives that 
    \begin{equation}
        \label{eq:Taylorapprox}
        \norm{\nabla_{\parameter}\Lcal(\parameter)-\nabla_{\parameter}Q(\parameter)}\leq \frac{\rho}{2} \norm{\parameter-\parameter_0}^2.
    \end{equation}
    Thus, $\norm{q_h-g_h}^2\leq \frac\rho2\norm{\parameter_h-\parameter_0}^2$.
    To derive the lower bound, $\parameter_{p+1}-\parameter_0$ is decomposed as
    \begin{equation*}
        \label{eq:decompTaylor}
        \begin{aligned}
            \parameter_{p+1}-\parameter_0&=\parameter_{p}-\parameter_0-\alpha\hat{g}_{p}\\
            &=\parameter_{p}-\parameter_0-\alpha q_p-\alpha(\hat{g}_p-g_p+g_p-q_p)\\
            &=(I-\alpha H_0)(\parameter_{p}-\parameter_0)-\alpha g_0-\alpha(\hat{g}_p-g_p+g_p-q_p).
        \end{aligned}
    \end{equation*}
    Let $-\lambda_0<0$ be the minimum eigenvalue of the Hessian $H_0=H(\parameter_0)$, and let $v$ be the unit eigenvector with respect to $-\lambda_0$ (which is deterministic conditional on $\parameter_0$). Then $(I-\alpha H_0)v=(1+\alpha \lambda_0) v=\kappa v$, and hence
    \begin{align*}
        \iprod{v}{\parameter_{p+1}-\parameter_0}= \kappa \iprod{v}{\parameter_{p+1}-\parameter_0} - \alpha \iprod{v}{g_0} - \alpha \iprod{v}{\hat{g}_p-q_p}.
    \end{align*}
    Recursively expanding this equality out, we finally obtain 
    \begin{align}
        &\iprod{v}{\parameter_{p+1}-\parameter_0}
        = \kappa^p \iprod{v}{\parameter_{1}-\parameter_0} - \alpha \iprod{v}{g_0}\sum_{h=1}^p \kappa^{p-h} - \alpha \sum_{h=1}^p \kappa^{p-h}\iprod{v}{\hat{g}_h-q_h}\notag\\
        &\qquad = \kappa^p \bigg[\beta  \underbrace{ \iprod{v}{-\hat{g}_0} }_{u} - \alpha \underbrace{ \frac{1-\kappa^{-p}}{\kappa-1} \iprod{v}{g_0} }_{d_p} - \alpha \underbrace{ \sum_{h=1}^p \kappa^{-h}\iprod{v}{\hat{g}_h-g_h} }_{\xi_p}
        - \alpha \underbrace{ \sum_{h=1}^p \kappa^{-h}\iprod{v}{g_h-q_h} }_{\delta_p} \bigg]. \label{eq:decompTaylor1}
    \end{align} 
    Therefore,
    \begin{equation}
        \label{eq:R2decp}
        \begin{aligned}
            \Ebb\norm{\parameter_{p+1}-\parameter_0}^2
            \geq& \Ebb\iprod{v}{\parameter_{p+1}-\parameter_0}^2
            =\kappa^{2p}\Ebb\left(\beta u-\alpha d_p-\alpha \xi_p-\alpha \delta_p\right)^2\\
            \geq& \kappa^{2p} \left(\beta^2 \Ebb [u^2] - 2\alpha\beta \epct{ud_p} - 2\alpha\beta\epct{u\xi_p} - 2\alpha\beta\epct{u\delta_p} \right).
        \end{aligned}
    \end{equation}
    
    % Therefore, the lower bound can be estimated by 
    % \begin{equation}
    %     \label{eq:R2decp}
    %     \Ebb\norm{\parameter_{p+1}-\parameter_0}^2\geq \Ebb\norm{u_p}^2-2\alpha \epct{ \iprod{u_p}{\delta_p}}-2\alpha \epct{ \iprod{u_p}{d_p}}-2\alpha  \epct{ \iprod{u_p}{\xi_p}},
    % \end{equation}
    % where we use the fact that $\norm{a+b}^2\geq \norm{a}^2+2\iprod{a}{b}$. 
    
    % Now we bound each term on the RHS of \eqref{eq:R2decp}. We introduce the notation
    % \begin{equation*}
    %     \kappa :=1+\alpha \lambda_{0}, ~~\lambda_{0}=\abs{ \lambda_{\min}\left(H_0\right)}.
    % \end{equation*}
    % Since $0<\alpha<\frac{1}{L}$, the eigenvalue of $I-\alpha H_0$ lies in $(0,\kappa]$. As $1<\kappa<2$, it holds that %\cfn{easy formula, no need to state here (instead, we can state it inline after inequalities if needed)} 
    % \begin{equation}
    %     \label{eq:summation}
    %     \sum_{h=1}^{p}\kappa^{p-h}\leq \frac{2\kappa^p}{\kappa-1},\quad \sum_{h=1}^{p}\kappa^{p-h}h\leq \frac{2\kappa^p}{(\kappa-1)^2}.
    % \end{equation}
    By the Cauchy-Schwarz inequality, Lemma \ref{lem:cnc} implies that 
    \begin{equation}
        \label{eq:right1}
        \begin{aligned}
            \Ebb u^2&= \epct{(v^{\T}\hat{g}_0)^2}\geq \mu\sigma_2^2.
        \end{aligned}
    \end{equation}
    where $\mu=\frac{\eta \gamma}{16n}$.
    Next, because $d_p$ is deterministic, the term $\epct{ud_p}$ can be bounded as
    \begin{equation}
        \label{eq:d_p}
        \begin{aligned}
            \epct{ud_p}
            =& -d_p\Ebb \iprod{v}{\hat{g}_0}
            =-d_p\iprod{v}{g_0}+d_p\Ebb \iprod{v}{g_0-\hat{g}_0}
            \leq d_p\Ebb \iprod{v}{g_0-\hat{g}_0} \leq \frac{l_gB_{n,n_0}}{\kappa-1},
        \end{aligned}
    \end{equation}
    where the first inequality is due to $-d_p\iprod{v}{g_0}=-\frac{1-\kappa^{-p}}{\kappa-1}\iprod{v}{g_0}^2\leq 0$, and the second inequality uses Lemma \ref{lem:gbiasVar}.
    
    
    We next upper bound the term $\epct{u\xi_p}$ as follows.
    \begin{equation}
        \label{eq:xi_p}
        \begin{aligned}
            \epct{u\xi_p}
            =& \Ebb u\sum_{h=1}^p \kappa^{-h}\iprod{v}{\hat{g}_h-g_h} 
            = \epct{ u\sum_{h=1}^p \kappa^{-h}\iprod{v}{\Ebb[\hat{g}_h|\parameter_h]-g_h} }\\
            \leq& \epct{ |u|\sum_{h=1}^p \kappa^{-h}\norm{\Ebb[\hat{g}_h|\parameter_h]-g_h} }
            \leq \epct{ l_g\sum_{h=1}^p \kappa^{-h}B_{n,n_0} }
            \leq \frac{l_g B_{n,n_0}}{\kappa-1}.
            % \leq& \Ebb \left(\sum_{h=1}^p \kappa^{p-h}\norm{\hat{g}_h-g_h}\right)^2 \\
            % \leq& \left(\Ebb \sum_{l=1}^p \kappa^{p-l}\right) \cdot \left(\sum_{h=1}^p \kappa^{p-h}\norm{\hat{g}_h-g_h}^2\right) \\
            % \leq& \frac{\kappa^p}{\kappa-1} \sum_{h=1}^p \kappa^{p-h} \Ebb\norm{\hat{g}_h-g_h}^2
            % \leq \frac{\kappa^{2p}}{(\kappa-1)^2} V_{n,n_0}
        \end{aligned}
    \end{equation}
    where the second inequality is due to the Cauchy-Schwarz inequality, and in the last inequality we use Lemma \ref{lem:gbiasVar}. 

    Finally, we bound the term $\epct{u\delta_p}$:
    \begin{equation}
        \label{eq:delta_p}
        \begin{aligned}
            \epct{u\delta_p}
            =& \epct{ u\sum_{h=1}^p \kappa^{-h}\iprod{v}{g_h-q_h} }
            \leq \epct{ l_g \sum_{h=1}^p \kappa^{-h}\norm{g_h-q_h} }\\
            \leq& \frac{l_g\rho}{2} \epct{ \sum_{h=1}^p \kappa^{-h}\norm{\parameter_h-\parameter_0}^2 }
            \leq \frac{l_g\rho}{2} \sum_{h=1}^p \kappa^{-h}\left(A_2(h-1)+A_3\right) \\
            \leq& \frac{l_g\rho}{2} \left[ \frac{A_2}{(\kappa-1)^2}+\frac{A_3}{\kappa-1} \right],
        \end{aligned}
    \end{equation}
    where the second inequality is due to \eqref{eq:gradientbound}, the third inequality uses \eqref{eq:dtheta-linear}, and the last inequality is because $\sum_{h=1}^p \kappa^{-h}(h-1)\leq \frac{1}{(\kappa-1)^2}$.
    
    % Finally for the last term on the RHS of \eqref{eq:R2decp}, it is derived similarly that
    % \begin{equation}
    %     \label{eq:right4}
    %     \begin{aligned}
    %         \epct{\iprod{u_p}{\xi_p}}&=-\beta \epct{\hat{g}_0^{\T}\sum_{h=1}^{p}(I-\alpha H_0)^{2p-h}(g_h-\hat{g}_h)}\\
    %         &\leq -\beta \Ebb\left[\norm{\hat{g}_0}\sum_{h=1}^{p}\kappa^{2p-h}\norm{g_h-\epct{\hat{g}_h}}\right]\\
    %         &\leq 2\beta l_g B_{n,n_0}\cdot \sum_{h=1}^{p}\kappa^{2p-h}\overset{\eqref{eq:summation}}{\leq}2\beta l_g B_{n,n_0}\cdot \frac{\kappa^{2p}}{\kappa-1},
    %     \end{aligned}
    % \end{equation}
    % where the first inequality is also due to $\norm{I-\alpha H_0}\leq \kappa$ and the second follows from Lemma \ref{lem:gbiasVar} and \eqref{eq:gradientbound}.
    % By the choice of parameters in Table \ref{table:1}, there hold that
    % \begin{equation}
    %     \label{eq:conditions}
    %     \begin{aligned}
    %         B_{n,n_0}\leq \frac{\eta \beta \lambda_0^2}{64 l_g n},~ B_{n,n_0}^{2}\leq \frac{\eta\lambda_0^4}{64(\sqrt{T}+1) \rho l_g n}&\Rightarrow\frac{8l_g B_{n,n_0}}{\lambda_0}\leq \frac{\eta \beta \lambda_0^2}{8n},~ \frac{8(\beta+2T\alpha) l_g\rho B_{n,n_0}^2}{\lambda_0^2}\leq \frac{\eta \beta \lambda_0^2}{8n},\\
    %      \beta \leq \frac{\eta\lambda_0^3}{32\rho l_g^3n},~ \beta\leq \frac{\eta \lambda_0^4}{128\rho l_g LV_{n,n_0}n}&\Rightarrow\frac{4\beta^2l_g^3\rho}{\lambda_0}\leq \frac{\eta \beta \lambda_0^2}{8n},~\frac{16\beta^2 l_g\rho LV_{n,n_0}}{\lambda_0^2}\leq \frac{\eta \beta \lambda_0^2}{8n}\\
    %      \alpha\leq \frac{\eta \beta \lambda_0^4}{64\rho l_gV_{n,n_0}n},~\Lthre \leq \frac{\eta \beta\lambda_0^4}{128\rho l_g n}&\Rightarrow\frac{8\alpha l_g\rho V_{n,n_0}}{\lambda_0^2}\leq \frac{\eta \beta \lambda_0^2}{8n},~\frac{16 l_g\rho \Lthre}{\lambda_0^2}\leq \frac{\eta \beta \lambda_0^2}{8n}.
    %     \end{aligned}
    % \end{equation}

    Substituting the four inequalities \eqref{eq:right1}, \eqref{eq:d_p}, \eqref{eq:xi_p} and \eqref{eq:delta_p} into \eqref{eq:R2decp}, we obtain the lower bound as
    \begin{equation*}
        % \label{eq:total}
        \begin{aligned}
           \Ebb\norm{\parameter_{p+1}-\parameter_0}^2&\geq \kappa^{2p} \left(\beta^2 \mu\sigma_2^2 - 2\alpha\beta \frac{l_g B_{n,n_0}}{\kappa-1}- 2\alpha\beta\frac{l_g B_{n,n_0}}{\kappa-1} - 2\alpha\beta\frac{l_g\rho}{2} \left[ \frac{A_2}{(\kappa-1)^2}+\frac{A_3}{\kappa-1}\right] \right).
        \end{aligned}
    \end{equation*}
    According to our settings in Table \ref{table:1}, these conditions are satisfied:
    \begin{align}
        \Lthre\leq \frac{\mu\sigma_2^2\lambda_0\cdot\min\{\beta\lambda_0,1\}}{192l_g\rho},~ B_{n,n_0}^2\leq\frac{\mu\beta\sigma_2^2\lambda_0\cdot\min\{\beta T\lambda_0,1\}}{384\alpha T^2l_g\rho},~ \beta\leq \frac{\mu\sigma_2^2\lambda_0^2L}{384l_g\rho LV_{n,n_0}}.\label{eq:conditions_1}
    \end{align}
    It follows that 
    \begin{equation}
        \label{eq:A2A3}
        A_2\leq \frac{\mu \alpha \beta \sigma_2^2\lambda_0^2}{8l_g\rho},\quad A_3\leq \frac{\mu \beta \sigma_2^2\lambda_0}{8l_g\rho}.
    \end{equation}
    Therefore, we can conclude that
    \begin{equation}
        \label{eq:total-2}
        \begin{aligned}
           \Ebb\norm{\parameter_{p+1}-\parameter_0}^2&\geq 
           \kappa^{2p} \left(\beta^2 \mu\sigma_2^2-\frac{1}{4}\beta^2 \mu\sigma_2^2 -\frac{1}{4}\beta^2 \mu\sigma_2^2-\frac{1}{4}\beta^2 \mu\sigma_2^2\right)=\frac{1}{4}\beta^2 \kappa^{2p}\mu\sigma_2^2.
            % &~~ -8\alpha^2 \beta  l_g \rho \cdot \frac{\kappa^{2p} }{(\kappa-1)^2}\cdot (\alpha V_{n,n_0} +2\Lthre+(\beta+2T \alpha)B^2_{n,n_0}+2\beta^2LV_{n,n_0}),\\
            % &\overset{\eqref{eq:conditions}}{\geq} \frac{\eta\beta^2\kappa^{2p}\sigma_{2}^{2}}{n}-6\cdot \frac{\eta\beta^2\kappa^{2p}\sigma_{2}^{2}}{8n}=\frac{\eta\beta^2\kappa^{2p}\sigma_{2}^{2}}{4n}.
        \end{aligned}
    \end{equation}
    In other word, \eqref{eq:total-2} shows that the expected distance between $\parameter_0$ and $\parameter_{p+1}$ grows at least exponentially for $p$. Substituting $p=T-1$, it leads to a contradiction if the lower bound of $\Ebb\norm{\parameter_T-\parameter_0}^2$ is greater than the upper bound, i.e.,
    \begin{equation}
        \label{eq:contradiction}
         \frac{1}{4}\beta^2 \kappa^{2T}\mu\sigma_2^2\geq(T-1)A_2+A_3.
        % \frac{\eta\beta^2\kappa^{2T }\sigma_{2}^{2}}{4n}> 4 \alpha T\left(\alpha V_{n,n_0}+ 2\Lthre+(\beta+2T\alpha) B^2_{n,n_0}+2\beta^2LV_{n,n_0}\right)+2\beta^2 l_g^2.
    \end{equation}
    %\cfn{the constraints spell as $\beta\leq c_0 \frac{\underline{V}_{n,n_0}}{LV_{n,n_0}}\lambda^2, B_{n,n_0}<<1, \Lthre\leq \lambda^2\beta\underline{V}_{n,n_0}$, and the $n_0,n$ can simply be taken to be $\tbO{1}$.}
    We choose 
    \begin{equation}
        \label{eq:Tconst}
        T=\frac{c}{\alpha \lambda_0}\log \left(\frac{LV_{n,n_0}n}{\eta \alpha \beta \sigma_2} \right)
    \end{equation}
     in Table \ref{table:1} with taking a sufficiently large $c$ and then \eqref{eq:contradiction} implies a contradiction. Hence, the proof is completed.
\end{proof}


Finally, we establish the main theorem in this section. It is shown that VMC returns approximate second-order points or $\epsilon$-variance points in high probability. Since an $\epsilon$-variance point is desired, we should design better neural network architectures to reduce those meaningless second order stability points. 

\begin{theorem}
    \label{thm:efsp}
    Let Assumptions \ref{asm:wavefun} ,\ref{asm:local} ,\ref{asm:unigap} and \ref{asm:sec} hold. For any $\delta\in (0,1)$, with the stepsizes \eqref{eq:schedule} and parameters in Table \ref{table:1}, Algorithm \ref{alg:VMC} returns an $(\epsilon,\epsilon^{1/4})$ approximate second-order stationary point or an $\epsilon^{1/2}$-variance point with probability at least $1-\delta$ after the following steps

    \begin{equation}
        O\left(\delta^{-4}\epsilon^{-11/2}\log^2\left(\frac{1}{\epsilon\delta}\right)\right).
    \end{equation}
    
\end{theorem}
\begin{proof}
    Suppose $\Ecal_m$ is the event
    \begin{equation*}
        \Ecal_m:=\left\{\tparameter_m\in \Rcal_1\cup \Rcal_2\right\},
    \end{equation*}
    and its complement is $\Ecal_m^c=\left\{\tparameter_m\in \Rcal_3 
   \right\}$. Let $\Pcal_m$ denote the  probability of the occurrence of the event $\Ecal_m$. 
    
    When $\Ecal_m$ occurs, by Lemmas \ref{lem:R1} and \ref{lem:R2}, we have 
    \begin{equation}
        \label{eq:Edescent}
        \epct{\Lcal(\tparameter_{m})-\Lcal(\tparameter_{m+1})\Big|\Ecal_m}\geq \Lthre.
    \end{equation} On the other hand, when $\Ecal_m^c$ occurs, it follows from \eqref{eq:R1ineq} that
    \begin{equation}
        \label{eq:Ecdescent}
        2\epct{\Lcal(\tparameter_m)-\Lcal(\tparameter_{m+1})\Big|\Ecal_m^c}\geq -2T\alpha B_{n,n_0}^2-2\beta^2LV_{n,n_0}\geq -\delta \Lthre,
    \end{equation}
    where the first inequality is by discarding positive terms in \eqref{eq:R1ineq} and the second inequality is due to the choice $\Lthre\geq \left(2T\alpha B_{n,n_0}^2+2\beta^2LV_{n,n_0}\right)/\delta$ in Table \ref{table:1}. It means that the function value may increase by no more than $\delta \Lthre/2$. When the expectation is taken overall, \eqref{eq:Edescent} and \eqref{eq:Ecdescent} imply that
    \begin{equation}
        \label{eq:totaldescent}
        \epct{\Lcal(\tparameter_{m})-\Lcal(\tparameter_{m+1})} \geq (1-\Pcal_m)\cdot \left(-\frac{\delta \Lthre}{2}\right)+\Pcal_m\cdot \Lthre.
    \end{equation}
    
    Suppose Algorithm \ref{alg:VMC} runs for $K$ steps starting from $\parameter_0$ and there are $M=K/T $ of $\tparameter_{m}$. Let $\Lcal^*$ be the global minimum of $\Lcal(\parameter)$. Summing \eqref{eq:totaldescent} for $m=1,\dots, M$ yields that
    \begin{equation*}
        \Lcal(\parameter_0)-\Lcal^{*}\geq  -\frac{\delta \Lthre M}{2}+\sum_{m=1}^{M}\Pcal_m\cdot \Lthre\Rightarrow\frac{1}{M}\sum_{m=1}^{M}\Pcal_m \leq \frac{\Lcal(\parameter_0)-\Lcal^{*}}{M\Lthre}+\frac{\delta}{2}\leq \delta,
    \end{equation*}
    where the last inequality holds if $K$ satisfies 
    \begin{equation}
        \label{eq:Kconst}
        K\geq \frac{2[\Lcal(\parameter_0)-\Lcal^{*}]T}{\delta \Lthre}=O\left(\delta^{-4}\epsilon^{-11/2}\log^2\left(\frac{1}{\epsilon\delta}\right)\right).
    \end{equation} Hence, the probability of the event $\Ecal_m^c$ occurs can be bounded by 
    \begin{equation*}
        1-\frac{1}{M}\sum_{m=1}^{M}\Pcal_m\geq 1-\delta.
    \end{equation*}
    This proves the statement in Theorem \ref{thm:efsp}.
\end{proof}
\section{Conclusions}
\label{sec:conclusion}

We explore the theoretical convergence of the VMC algorithm for solving the ground state of many-body quantum systems.
The upper bound of the bias and variance corresponding to the MCMC  is estimated without assuming that the local energy in a quantum system is bounded.
Then, we show that VMC with the biased stochastic gradient achieves first order stationary convergence. It has $O\left(\frac{\log K}{\sqrt{n K}}\right)$ convergence rate with a sufficiently large sample size $n$ after $K$ iterations. Moreover, by verifying the correlated negative curvature condition, we discuss how VMC escapes from saddle points and establish the second order convergence guarantee of $O\left(\epsilon^{-11/2}\log^{2} \left(\frac{1}{\epsilon}\right)\right)$. Our result explains the observation that VMC usually converges to eigenvalues of the quantum system.

There are some potential directions to be concerned for future works. (1) Other variational methods, such as variational inference, variational Bayesian matrix factorization, and variational annealing,  can be studied in our analytical framework. (2) Our convergence analysis suggests that improving the efficiency of sampling methods is of vital importance for the better performance of stochastic algorithms. (3) The convergence of the natural gradient method and the KFAC method for VMC is also interesting.

\bibliographystyle{plain}
\bibliography{bib.bib}

\end{document}
