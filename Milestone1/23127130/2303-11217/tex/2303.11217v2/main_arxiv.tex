\documentclass[10pt,twocolumn,letterpaper]{article}

\usepackage[accsupp]{axessibility}  %
\usepackage{iccv}
\usepackage{times}
\usepackage{epsfig}
\usepackage{graphicx}
\usepackage{amsmath}
\usepackage{amssymb}
\usepackage{array}
\usepackage{mydefs}
\usepackage{xcolor}
\usepackage{bm}
\usepackage{booktabs}
\usepackage{subcaption}
\usepackage{algorithm}
\usepackage{algpseudocode} 
\usepackage[normalem]{ulem}
\usepackage{multirow}
\usepackage{tabularx,rotating}
\usepackage{wrapfig}


\newcommand{\np}[1]{\textcolor{purple}{#1}}
\newcommand{\jean}[1]{\textcolor{red}{#1}}
\newcommand{\andres}[1]{\textcolor{blue}{#1}}
\newcommand{\ah}[1]{\textcolor{olive}{#1}}
\newcommand{\id}{\operatorname{Id}}
\newcommand{\diag}{\operatorname{diag}}

\usepackage[pagebackref=true,breaklinks=true,letterpaper=true,colorlinks,bookmarks=false]{hyperref}

\iccvfinalcopy %

\def\iccvPaperID{11126} %
\def\httilde{\mbox{\tt\raisebox{-.5ex}{\symbol{126}}}}

\ificcvfinal\pagestyle{empty}\fi

\begin{document}

\title{Inverse problem regularization with hierarchical variational autoencoders\vspace{-0.2cm}}

\author{Jean Prost\\
Univ. Bordeaux, CNRS, Bordeaux INP\\
IMB, UMR 5251, F-33400 Talence, France\\
{\tt\small jean.prost@math.u-bordeaux.fr}
\and
Antoine Houdard\\
Ubisoft La Forge\\
Bordeaux\\
{\tt\small antoine.houdard@ubisoft.com}
\and
Andrés Almansa\\
Université Paris Cité, CNRS\\
 MAP5, UMR 8145\\
{\tt\small  andres.almansa@parisdescartes.fr}
\and 
Nicolas Papadakis\\
Univ. Bordeaux, CNRS, Bordeaux INP\\
IMB, UMR 5251, F-33400 Talence, France\\
{\tt \small  nicolas.papadakis@math.u-bordeaux.fr}
}

\maketitle
\ificcvfinal\thispagestyle{empty}\fi


\begin{abstract}


Over the past few years, there has been a significant amount of research focused on studying the ReLU activation function, with the aim of achieving neural network convergence through over-parametrization. However, recent developments in the field of Large Language Models (LLMs) have sparked interest in the use of exponential activation functions, specifically in the attention mechanism.

Mathematically, we define the neural function $F: \R^{d \times m} \times  \mathbb{R}^d \rightarrow \mathbb{R}$ using an exponential activation function. Given a set of data points with labels $\{(x_1, y_1), (x_2, y_2), \dots, (x_n, y_n)\} \subset \mathbb{R}^d \times \mathbb{R}$ where $n$ denotes the number of the data. Here $F(W(t),x)$ can be expressed as $F(W(t),x) := \sum_{r=1}^m a_r \exp(\langle w_r, x \rangle)$, where $m$ represents the number of neurons, and $w_r(t)$ are weights at time $t$. It's standard in literature that $a_r$ are the fixed weights and it's never changed during the training. We initialize the weights $W(0) \in \mathbb{R}^{d \times m}$ with random Gaussian distributions, such that $w_r(0) \sim \mathcal{N}(0, I_d)$ and initialize $a_r$ from random sign distribution for each $r \in [m]$.

Using the gradient descent algorithm, we can find a weight $W(T)$ such that $\| F(W(T), X) - y \|_2 \leq \epsilon$ holds with probability $1-\delta$, where $\epsilon \in (0,0.1)$ and $m = \Omega(n^{2+o(1)}\log(n/\delta))$. To optimize the over-parametrization bound $m$, we employ several tight analysis techniques from previous studies [Song and Yang arXiv 2019, Munteanu, Omlor, Song and Woodruff ICML 2022]. 

 
\vspace{-0.2cm}
\end{abstract}

\section{Introduction}
\section{Introduction}

The increasing complexity of source code poses a key challenge to the reliability of large-scale software systems. Software bugs in these systems can lead to safety issues~\cite{bug_safety} for users around the world as well as cause non-negligible financial losses~\cite{bug_loss}. As such, developers have to spend a large amount of time and effort on bug fixing. Consequently, \aprfull (\apr), designed to automatically generate patches to fix software bugs, has attracted wide attention from both academia and industry~\cite{long2016prophet, legoues2012genprog, long2015spr, lou2020can, tufano2018empstudy}. 


To achieve \apr, one popular approach is known as Generate-and-Validate (G\&V)~\cite{qi2015gv, ghanbari2019prapr, lou2020can, le2016hdrepair, legoues2012genprog, wen2018capgen, hua2018sketchfix, martinez2016astor, koyuncu2020fixminder, liu2019tbar, liu2019avatar}, which is typically based on the following pipeline: First, fault localization techniques~\cite{wong2016fl, abreu2007ochiai, zhang2013injecting, papadakis2015metallaxis, li2019deepfl, li2017transforming} are applied to determine the suspicious locations in programs where bugs are likely to exist. Then, the buggy locations are used by the \apr tools to generate a list of patches that replace buggy lines with correct lines. Afterward, each patch is validated against the original test suite to identify any \emph{plausible patches} (i.e., passing all tests in the test suite). Finally, to determine the \emph{correct patches}, developers examine the list of plausible patches to see if any of them can correctly fix the bug. 

Traditional \apr tools can mainly be categorized into heuristic-based~\cite{legoues2012genprog, le2016hdrepair, wen2018capgen}, constraint-based~\cite{mechtaev2016angelix, le2017s3, demacro2014nopol, long2015spr} and \template~\cite{ghanbari2019prapr, hua2018sketchfix, martinez2016astor, liu2019tbar, liu2019avatar}. Among these traditional tools, \template \apr tools~\cite{ghanbari2019prapr, liu2019tbar, benton2020effectiveness} have been able to achieve state-of-the-art results. \Template \apr tools typically leverage pre-defined templates (e.g., adding a nullness check) for bug fixing. However, since these fix templates are typically handcrafted, the number and types of bugs they are able to fix can be limited. 



To address the limitations of traditional \apr, researchers have proposed various \learning \apr tools~\cite{li2020dlfix, chen2018sequencer, jiang2021cure, lutellier2020coconut, zhu2021recoder, ye2022rewardrepair} based on the \nmtfull (\nmt) architecture~\cite{sutskever2014mt} where the input is the buggy code snippets and the goal is to translate the buggy code snippets into a fixed version. To accomplish this, \learning \apr tools require supervised training datasets with pairs of both buggy and fixed code snippets in order to learn how to perform this translation step. These training data are usually obtained by mining historical bug fixes using heuristics/keywords~\cite{dallmeier2007benchmark}, which can be imprecise for identifying bug-fixing commits; even the actual bug-fixing commits can include irrelevant code changes, leading to further pollution in the dataset~\cite{xia2022alpharepair}.
% 
Moreover, it can be hard for such \apr tools to generalize and fix bug types unseen during training. 



To better leverage recent advances in \plmfull{s} (\plm{s}), researchers~\cite{xia2022alpharepair, xia2023repairstudy, kolak2022patch, prenner2021codexws} have directly applied \plm{s} to generate patches without bug-fixing datasets. These \llm-based \apr tools work by either directly generating a complete code function~\cite{prenner2021codexws, xia2023repairstudy} or predict/infill the correct code snippet given its surrounding context~\cite{xia2022alpharepair, xia2023repairstudy}. By directly using \llm{s} that are pre-trained on billions of open-source code snippets, \llm-based \apr tools can achieve state-of-the-art performance on many repair datasets~\cite{xia2022alpharepair}. 


% 
%
%

Traditional \apr tools have long used the insight of the \emph{plastic surgery hypothesis}~\cite{barr2014plastic} where it states that the code ingredients to fix a bug already exist within the same project. Traditional \apr tools have manually designed pattern-~\cite{ghanbari2019prapr, saha2017elixir} or heuristic-based~\cite{jiang2018simfix, legoues2012genprog} approaches to finding and using such relevant code ingredients to generate fixes for bugs. However, the plastic surgery hypothesis has been largely ignored in \llm-based \apr. In fact, \llm provides a unique opportunity to fully automate the plastic surgery hypothesis idea via fine-tuning (learning project-specific information via model updates from the buggy project) and prompting (directly providing relevant code ingredients to the model), and make it directly applicable to different languages (since the \llm{s} are typically multi-lingual).%
Moreover, despite the intensive manual efforts involved, traditional \apr tools still cannot fully leverage project-specific information due to large search space for leveraging/composing existing code ingredients. In contrast, the project-specific information can effectively leveraged by \llm{s} due to their power in code understanding/vectorization, e.g., even partial/imprecise information may still guide \llm{s} in correct patch generation!
 To this end, we ask the question: \emph{How useful is the plastic surgery hypothesis in the era of \plm{s}}?








\mypara{Our Work.} To answer the question, we present \ourtech{\xspace} -- a \llm-based approach that automatically utilizes the plastic surgery hypothesis by systematically combining multiple fine-tuning and prompting strategies for \apr. \ourtech fine-tunes \plm{s} using two novel domain-specific training strategies: \textbf{\epfinetune} -- we fine-tune using the original buggy project by aggressively masking out a high percentage of tokens, which allows \plm to learn project-specific code tokens and programming styles; and \textbf{\rofinetune} -- which only masks out a single continuous code sequence per training sample, allowing the model to get used to the final \csapr task of predicting a single continuous code sequence. Furthermore, we directly leverage the ability for \plm{s} to understand natural language instructions and introduce a novel prompting strategy, \textbf{\idprompting}, which uses information retrieval and static analysis to obtain a list of relevant identifiers for the buggy lines. While such relevant identifiers are critical for fixing some difficult bugs, they may not be seen by the \llm during inference due to limited context window size. Through the use of prompting, we directly tell the model to use these extracted identifiers (relevant code ingredients) to generate the correct code. Finally, to perform repair, we combine all four model variants (including the base model, both fine-tuned models and the base model with prompting) for the final repair.





While our insight of leveraging the plastic surgery hypothesis for \llm-based \apr is generalizable across different types of \plm{s}, to implement \ourtech, we choose a recent \plm{\xspace}, \ctfive~\cite{wang2021codet5}, which is pre-trained on millions of open-source code snippets. \ctfive is an encoder-decoder model trained using \mspfull (\msp) objective where a percentage of tokens are masked out and each continuous masked token sequence is referred to as a masked span. Also, although we only extract relevant identifiers from the current buggy project (since this paper focuses on the plastic surgery hypothesis), our work can be easily extended to obtain other code information (such as relevant statements or functions) from other sources, such as  the massive pre-training corpora~\cite{husain2020codesearchnet} or historical bug-fixing datasets~\cite{jiang2019infer}, which can provide more coding knowledge for \llm{s}. Besides, although we mainly focus on using traditional string comparison algorithms for information retrieval in this paper, these techniques can be easily replaced by other frequency-based retrieval~\cite{robertson2009probabilistic} and neural search (or embedding-based search)~\cite{reimers2019sentence}.
  In summary, this paper makes the following contributions:


%


\begin{itemize}[noitemsep, leftmargin=*, topsep=0pt]
    \item \textbf{Dimension.} This paper is the first to revisit the important plastic surgery hypothesis in the era of \llm{s}. It opens up a new dimension for \llm-based \apr to incorporate previously neglected information from the buggy project itself to boost \apr performance. Furthermore, it demonstrates the promising future of retrieval-based prompting for modern \llm-based \apr.
    \item \textbf{Implementation.} We implement \ourtech based on the recent \ctfive model. We augment the model using two novel fine-tuning strategies: \epfinetune and \rofinetune, along with a novel prompting strategy based on information retrieval and static analysis: \idprompting. We combine the patches generated by all four models together and perform patch ranking to speed up \apr.% 
    \item \textbf{Evaluation Study.} We conduct an extensive evaluation against state-of-the-art \apr tools. On the widely studied \dfj 1.2 and 2.0 datasets~\cite{just2014dfj}, \ourtech is able to achieve the new state-of-the-art results of 89 and 44 correct bug fixes (15 and 8 more than best baseline) respectively.  Furthermore, we perform a broad ablation study to justify our design. \ourtech demonstrates for the first time that the plastic surgery hypothesis can substantially boost \llm-based \apr and advance state-of-the-art \apr, while being fully automated and general. Moreover, even partial/imprecise code ingredients may still effectively guide \llm{s} for \apr!
\end{itemize}



\section{Related works}\label{sec:related}
\setlength{\tabcolsep}{1.6mm}{
\renewcommand\arraystretch{1.1}
\begin{table}[ht]
  \centering
  \scalebox{0.9}{
  \begin{tabular}{llcccc}
    \toprule
    &\multirow{2}*{Methods} & \multirow{2}*{Sal.} &   \multicolumn{2}{c}{VOC} & MS~COCO \\
    \cmidrule(r){4-5}\cmidrule(r){6-6}
    &&&\texttt{val}&\texttt{test}&\texttt{val}\\
    \hline
    \multirow{13}*{\rotatebox{90}{ResNet-50}}
    &IRN~\cite{irn}          \tiny{CVPR'19}     &              & 63.5       & 64.8          & 42.0  \\
    &LayerCAM~\cite{layercam}\tiny{TIP'21}      &              & 63.0       & 64.5          & -     \\
    &AdvCAM~\cite{advcam}    \tiny{CVPR'21}     &              & 68.1       & 68.0          & 44.2  \\
    &RIB~\cite{rib}          \tiny{NeurIPS'21}  &              & 68.3       & 68.6          & 44.2  \\
    &ReCAM~\cite{recam}      \tiny{CVPR'22}     &              & 68.5       & 68.4          & 42.9  \\
    % \rowcolor{Gray}
    &\cellcolor{Gray}IRN+\texttt{LPCAM}    &\cellcolor{Gray} & \cellcolor{Gray}68.6    & \cellcolor{Gray}68.7      & \cellcolor{Gray}44.5  \\
    &SIPE~\cite{sipe}        \tiny{CVPR'22}     &              & 68.8       & 69.7          & 40.6  \\
    &OOD~\cite{ood}+Adv      \tiny{CVPR'22}     &              & 69.8       & 69.9          & -     \\
    &AMN~\cite{amn}          \tiny{CVPR'22}     &              & 69.5       & 69.6          & 44.7  \\
    &\cellcolor{Gray}AMN+\texttt{LPCAM}    &\cellcolor{Gray} & \cellcolor{Gray}70.1    &\cellcolor{Gray} 70.4      & \cellcolor{Gray}45.5  \\ 
    &ESOL~\cite{esol}        \tiny{NeurIPS'22}  &              & 69.9$^*$   & 69.3$^*$      & 42.6  \\
    &CLIMS~\cite{clims}      \tiny{CVPR'22}     &              & 70.4$^*$   & 70.0$^*$      & -     \\
    &EDAM~\cite{edam}        \tiny{CVPR'21}     &\checkmark    & 70.9$^*$   & 71.8$^*$      & -     \\
    &\cellcolor{Gray}EDAM+\texttt{LPCAM}  &\cellcolor{Gray}\checkmark & \cellcolor{Gray}71.8$^*$ &\cellcolor{Gray} 72.1$^*$& \cellcolor{Gray}42.1\\
    \hline
    \multirow{9}*{\rotatebox{90}{WideResNet-38}}
    &Spatial-BCE~\cite{sbce} \tiny{ECCV'22}     &              & 70.0       & 71.3      & 35.2  \\
    &BDM~\cite{bdm}          \tiny{ACMMM'22}    &\checkmark    & 71.0       & 71.0      & 36.7  \\ 
    &RCA~\cite{rca}+OOA      \tiny{CVPR'22}     &\checkmark    & 71.1       & 71.6      & 35.7  \\
    &RCA~\cite{rca}+EPS      \tiny{CVPR'22}     &\checkmark    & 72.2       & 72.8      & 36.8  \\
    &HGNN~\cite{hgnn}        \tiny{ACMMM'22}    &\checkmark         & 70.5$^*$   & 71.0$^*$  & 34.5  \\ 
    &EPS~\cite{eps}          \tiny{CVPR'21}     &\checkmark         & 70.9$^*$   & 70.8$^*$  & -     \\
    &RPIM~\cite{rpim}        \tiny{ACMMM'22}    &\checkmark         & 71.4$^*$   & 71.4$^*$  & -     \\ 
    &L2G~\cite{l2g}          \tiny{CVPR'22}     &\checkmark         & 72.1$^*$   & 71.7$^*$  & 44.2  \\
    \hline
    \multirow{2}*{\rotatebox{90}{\small{DeiT-S}}}
    &MCTformer~\cite{mctformer}    \tiny{CVPR'22}     &                 & 71.9$^{\dag}$  & 71.6$^{\dag}$   & 42.0  \\
    &\cellcolor{Gray}MCTformer+\texttt{LPCAM}      &\cellcolor{Gray} & \cellcolor{Gray}72.6$^{\dag}$  & \cellcolor{Gray}72.4$^{\dag}$  &\cellcolor{Gray} 42.8 \\
    \bottomrule
  \end{tabular}}
  \vspace{-2mm}
  \caption{The mIoU results (\%) based on DeepLabV2 on VOC and MS~COCO. The side column shows three backbones of multi-label classification model. ``Sal.'' denotes using saliency maps. * denotes the segmentation model is pre-trained on MS~COCO. $^\dag$ denotes the segmentation model is pre-trained on VOC.
  }
  \vspace{-6mm}
  \label{table_related}
\end{table}
}

 



\section{Background on  variational autoencoders}\label{sec:back}
This work  exploits the capacity of HVAEs to model complex image distributions~\cite{child2021very,vahdat2020nvae}.
 We review the properties of the loss function used to train a VAE
(section~\ref{sec:perfect_vae}), then we detail the generalization of VAE to hierarchical VAE (section~\ref{ssec:hvae}), and present the temperature scaling approach to monitor the quality of generated images (section~\ref{ssec:temp}).

\subsection{VAE training} \label{sec:perfect_vae}

Variational autoencoders (VAE)  have been introduced in~\cite{kingma2013auto} to model complex data distributions. VAEs are trained to fit a parametric probability distribution in the form of a latent variable model:
\begin{equation}
    \pt{\x} = \int{\pt{\x|\z}\pt{\z}d\z},
\end{equation}
where $\pt{\x|\z}$ is a probabilistic decoder, and $\pt{\z}$ corresponds to the prior distribution over the model latent variable $\z$. 
A VAE is also composed of a probabilistic encoder $\qp{\z|\x}$, whose role is to approximate the posterior of the latent model $\pt{\z|\x}$, which is usually intractable.

The generative model parameters $\theta$ and the approximate posterior parameters $\phi$ of a VAE are jointly trained by maximizing the evidence lower bound (ELBO)~\cite{kingma2013auto, rezende2014stochastic} on a training data distribution $\pd{\x}$.
\begin{equation}
    \label{eq:elbo}
    \mathcal{L}\left(\x; \theta, \phi\right)\hspace{-1pt} =\hspace{-1pt} \E{\qp{\z|\x}}{\log\pt{\x|\z}} - \KL{\qp{\z|\x}}{\pt{\z}}.
\end{equation}
The ELBO expectation on $\pd{\x}$ is upper-bounded by the negative entropy of the data distribution,
and, when the upper-bound is reached we have that~\cite{zhao2017learning}:
\begin{equation}
    \label{eq:perfect_vae}
    \KL{\pd{\x}\qp{\z|\x}}{\pt{\x}\pt{\z|\x}} = 0.
\end{equation}


\subsection{Hierarchical variational autoencoders}\label{ssec:hvae}

The ability of hierarchical VAEs to model complex distributions is due to their hierarchical structure imposed in the latent space.
 The latent variable of HVAE is partitioned into $L$ subgroups $\z = (\z_0, \z_1, \cdots, \z_{L-1})$, and the prior and the encoder are respectively defined as: 
\begin{align}
    \pt{\z} &= \prod_{l=1}^{L-1} \pt{\z_l|\z_{<l}}\pt{\z_0}  \label{eq:hierarchical_prior}\\
    \qp{\z|\x} &= \prod_{l=1}^{L-1} \qp{\z_l|\z_{<l},\x}\qp{\z_0|\x}.    \label{eq:hierarchical_encoder}
\end{align}

We consider a specific class of HVAEs with gaussian conditional distributions for the encoder and the decoder
\begin{equation}
    \label{eq:gaussian_hvae}
    \hspace*{-0.25cm}\begin{cases}
        \pt{\z_l|\z_{<l}} &= \mathcal{N}\!\left(\z_l; \mutl{\z_{<l}}\hspace{-1pt}\hspace{-1pt}\Sigma_{\theta, l}(\z_{<l})\right) \\
        \qp{\z_l|\z_{<l}, \x}\mkern-18mu &= \mathcal{N}\!\left(\z_l; \mupl{\z_{<l}, \x}\hspace{-1pt},\hspace{-1pt} \Sigma_{\phi, l}(\z_{<l}, \x)\right)\hspace{-1pt},
    \end{cases}
\end{equation} 
where $\mu_{\theta, 0}$ and $\Sigma_{\theta, 0}$ can either be trainable or non-trainable constants, and the remaining mean vectors ($\mu_{\theta,l}$, and $\mu_{\phi,l}$, for $l>0$) and covariance matrices ($\Sigma_{\theta,l}$ and $\Sigma_{\phi,l}$, for $l>0$)  are parametrized by neural networks. \footnote{
	Note that for the special case $l=0$, $\z_{<l}$ is empty, meaning that 
	$\Sigma_{\theta, 0}(\z_{<0})=\Sigma_{\theta, 0}$ is actually a constant,
	$\qp{\z_0|\z_{<0},\x}=\qp{\z_0|\x}$ is only conditioned on $\x$, etc.
}
In this work, we consider models with a gaussian decoder:
\begin{equation}
    \label{eq:decoder_constant}
    \pt{\x|\z} = \N\left(\x; \mut{\z}, \gamma^2 I\right).
\end{equation}


\subsection{Temperature scaling}\label{ssec:temp}
As demonstrated in~\cite{vahdat2020nvae,child2021very}, sampling the latent variables $\z_l$ from a prior with reduced temperature improves the visual quality of the generated images from the VAE. 
In practice, this is done by multipling the covariance matrix of the gaussian distribution $\pt{\z_l|\z_{<l}}$ by a factor $\tau_l<1$.
This factor $\tau_l$ is called temperature because of its link to statistical physics.
Reducing the temperature of the priors amounts to defining the auxilliary model:
\begin{equation}
    \label{eq:hvae_joint_model_temp}
    p_{\theta, \tauvec}\left(\z_0, \cdots, \z_{L-1}, \x\right) = \prod_{l=0}^{L-1} \frac{\pt{\z_l|\z_{<l}}^{\frac{1}{\tau_l^2}}}{\tau_l^{\frac{d_l}{2}}}\pt{\x|\z_{<L}},
\end{equation}
where $\tauvec := \left(\tau_0, \cdots, \tau_{L-1}\right)$ gives the temperature for each level of the  hierarchy, and $d_l$ is the dimension of the latent variable $\z_l$.
In the following, we use this  temperature-scaled model to balance the regularization of our inverse problem.



\section{Regularization with HVAE Prior}
\label{sec:alternate}
In this section  we introduce our  Plug-and-Play method using a Hierarchical VAE  prior (PnP-HVAE) to solve generic image inverse problems. Building on top of the JPMAP framework~\cite{gonzalez2022solving}, we propose a joint model  over  the image and its latent variable that we optimize in an alternate way. By doing so, we take advantage of the HVAE encoder to avoid backpropagation through the generative network.
Although our motivation is similar to JPMAP, PnP-HVAE overcomes two of its limitations, that are the lack of control, and the limitation to simple and non-hierarchical VAEs.
We show in section~\ref{ssec:HJP} that the strength of the regularization of the tackled inverse problem can be monitored by tuning the temperature of the prior in the latent space. In section~\ref{ssec:encoder}, we  propose an approximation of the joint posterior distribution using the hierarchical VAE encoder. 
In section~\ref{ssec:optim}, we present our final algorithm that includes a new greedy scheme to optimize the latent variable of the HVAE.



\subsection{Tempered hierarchical joint posterior}\label{ssec:HJP}
The linear image degradation model~\eqref{eq:invprob} yields  
\begin{equation}
    \label{eq:f}
    p(\y|\x) \propto e^{-f(\x)}, \quad f(\x) = \frac{1}{2\sigma^2}||A\x-\y||^2.
\end{equation}
Solving the underlying image inverse problem in a bayesian framework requires an a priori distribution $p(\x)$ over clean images and studying the posterior distribution of $\x$ knowing its degraded observation $\y$. 
In this work, the image prior is given by a hierarchical VAE and we exploit the joint posterior model $p(\z,\x|\y)$. 
From the HVAE latent variable model with reduced temperature $p_{\theta, \bf{\tau}}\left(\z_0, \cdots, \z_{l-1}, \x\right)$ as defined in \eqref{eq:hvae_joint_model_temp}, 
we define the associated tempered joint model as:
\begin{equation}
    \label{eq:full_joint_model}
    p\left(\z
    , \x, \y\right) 
    := p_{\theta, \bf{\tau}}\left(\z_0, \cdots, \z_{L-1}, \x\right)p\left(\y|\x\right).
\end{equation}

Following the JPMAP idea from~\cite{gonzalez2022solving}, we aim at finding the couple $(\x,\z)$ that maximizes the joint posterior $p(\x, \z|\y)$: 
\begin{equation}
    \label{eq:joint_map}
    \min_{\x, \z} - \log p(\x, \z|\y) .
\end{equation}
Although we are only interested in finding the image $\x$,  the joint Maximum A Posteriori (MAP) criterion \eqref{eq:joint_map} makes it possible to derive an optimization scheme that only relies on forward calls of the HVAE. 
Using  Bayes' rule and the definition of the tempered HVAE joint model~\eqref{eq:full_joint_model} and~\eqref{eq:hvae_joint_model_temp}, the logarithm of the joint posterior rewrites:

\begin{align}
  &\log p(\x, \z|\y)\log p(\y) \\ =&\log p(\y|\x)  + \sum_{l=0}^{L-1} \log\frac{\pt{\z_l|\z_{<l}}^{\frac{1}{\tau_l^2}}}{\tau_l^{\frac{d_l}{2}}} + \log \pt{\x|\z_{<L}}\nonumber.
\end{align}


Since $p(\y)$ is constant,  
finding the joint MAP estimate \eqref{eq:joint_map} amounts to minimizing the following criterion:
\begin{align}
    \min_{\x,\z}J_1\left(\x, \z\right):=- \sum_{l=1}^{L-1}\frac{1}{\tau_l^2}\log \pt{\z_l|\z_{<l}}+f(\x) -\log \pt{\x|\z_{<L}}.
    \label{eq:J1}
\end{align}
Notice that the temperature of the prior over the latent space $\tau_l$ controls the weight of the regularization over the latent variable $\z_l$.
Optimizing~\eqref{eq:J1} w.r.t. $\x$ is tractable, whereas the minimization  w.r.t.$\z$ requires a backpropagation through the decoder $\log \pt{\x|\z_{<L}}$ that is impractical due to the high dimensionality and the hierarchical structure of the HVAE latent space.

\subsection{Encoder approximation of the joint posterior}\label{ssec:encoder}
Using the encoder $q_\phi$, we can reformulate the joint MAP problem~\eqref{eq:J1} in a form that is more  convenient to optimize with respect to $\z$.
Indeed, for a HVAE with enough capacity and trained to optimality, relation~\eqref{eq:perfect_vae} holds, and  we get
\begin{equation}
    \pt{\x|\z} = \frac{\qp{\z|\x}\pd{\x}}{\pt{\z}} \label{eq:decoder_approx}.
\end{equation}

Thus, by introducing the decoder expression~\eqref{eq:decoder_approx} in the full model~\eqref{eq:full_joint_model}, we have:
\begin{align}
    \nonumber
    p\left(\z, \x, \y\right)=
    \prod_{l=0}^{L-1} \frac{1}{\tau^{\frac{d_l}{2}}}\frac{\qp{\z_l|\z_{<l}, \x}}{\pt{\z_l|\z_{<l}}^{1-\tau_l^{-2}}}p(\y|\x)\pd{\x}.
\end{align} 

Denoting $\lambda_l = \frac{1}{\tau_l^2}-1$, we reformulate the joint MAP problem~\eqref{eq:joint_map} as a joint MAP problem over the encoder model:
\begin{align}
    %\nonumber
        \min_{\x,\z}\hspace{-1pt}J_2(\x, \z)\hspace{-2pt} :=  \hspace{-1pt}-\hspace{-4pt}\sum_{l=0}^{L-1} \hspace{-3pt}\left(\log\qp{\z_l|\x, \z_{<l}} \hspace{-2pt}+\hspace{-2pt} \lambda_l\hspace{-1pt}\log\pt{\z_l|\z_{<l}}\hspace{-1pt}\right)+f(\x)- \log\pd{\x}\label{eq:J2}.
\end{align}


\subsection{Alternate optimization with PnP-HVAE}\label{ssec:optim}
We introduce an alternate  scheme to minimize~\eqref{eq:joint_map} that  sequentially optimizes with respect to $\x$ and  to $\z$.
For a linear degradation model and a gaussian decoder~\eqref{eq:invprob}, the criterion $J_1(\x, \z)$ in~\eqref{eq:J1} is convex in $\x$ and its global minimum is

$\x = \left(A^tA+\frac{\sigma^2}{\gamma^2}\id\right)^{-1}
        \hspace{-4pt}\left(
            A^t\y+\frac{\sigma^2}{\gamma^2}\mut{\z}\right)$.
Next we propose to compute an approximate solution of the problem $\min_{\z} J_2(\x,\z)$ with the greedy 
 algorithm~\ref{algo:hierarchical_latent_reg}.

\begin{algorithm}
    \caption{Hierarchical encoding with latent regularization to minimize~\eqref{eq:J2} w.r.t. $\z$ for a fixed $\x$}\label{algo:hierarchical_latent_reg}
        \begin{algorithmic}\small
            \Require image $\x$; HVAE ($\phi,\theta$); temperature $\tau_l$;  $\lambda_l=\frac{1}{\tau_l^2}-1$
        \For{$0\leq l < L$}
      \State $S_q \gets \Vpil{{\z}_{<l}, \x}$; $m_q\gets\mupl{{\z}_{<l}, \x}$
        \Comment{Encoder}

        \State $S_p \gets \Vtil{{\z}_{<l}}$; $m_p\gets\mutl{{\z}_{<l}}$
        \Comment{Prior}

        \State $\z_l\gets \left(S_q + \lambda_lS_p\right)^{-1} 
        \left(S_qm_q + \lambda_lS_pm_p\right)$       

        \EndFor
            \State \textbf{return} $\enc{\x}=(\hat{\z}_{0},\hat{\z}_{1}, \cdots, \hat{\z}_{L-1})$
            \end{algorithmic}
\end{algorithm}
    
In algorithm~\ref{algo:hierarchical_latent_reg}, the latent variables $\z_l$ are determined in a hierarchical fashion starting from the coarsest to the finest one. 
As defined in relations~\eqref{eq:gaussian_hvae}, the conditionals $\qp{\z_l|\x, \z_{<l}}$ and $\pt{\z_l|\z_{<l}}$ are gaussians. Therefore, the minimization at each step can be viewed as an interpolation between the mean of the encoder $\qp{\z_l|\x, \z_{<l}}$ and the prior $\pt{\z_l|\z_{<l}}$, given some weights conditioned by the covariance matrices and the temperature $\tau_l$.
The solution of each minimization problem in $\z_l$ is given by:
\begin{align}
    \hat{\z_l} \hspace{-1pt}= &\hspace{-1pt}\left(\Vpil{\hat{\z}_{<l}, \x} + \lambda_l\Vtil{\hat{\z}_{<l}}\right)^{-1} \\
    &\hspace{-1pt}\left(\Vpil{\hat{\z}_{<l}, \x}\mupl{\hat{\z}_{<l}, \x} \hspace{-2pt}+ \hspace{-2pt}\lambda_l\Vtil{\hat{\z}_{<l}}\mutl{\hat{\z}_{<l}}\hspace{-1pt}\right)\nonumber
\end{align}
where $\lambda_l=1/{\tau_l^2}-1$. 
In the following, we denote as $\hat{\z} := \enc{\x}$ the output of the hierarchical encoding of Algorithm~\ref{algo:hierarchical_latent_reg}.
In Appendix~\ref{sec:algo1-global-min} we show that this algorithm finds the global optimum of $J_2(\x,\cdot)$ under mild assumptions.

The final PnP-HVAE procedure to solve an inverse problem with the HVAE prior is presented in Algorithm~\ref{algo:final}.
\begin{algorithm}[!h]
    \caption{PnP-HVAE - Restoration  by solving~\eqref{eq:J1}}\label{algo:final}
    \begin{algorithmic}\small
        \State $k \gets 0$; $res \gets + \infty$; 
        initialize $\x^{(0)}$
        \While{$res > tol$}
\State \textcolor{gray}{\% $\min_{\z} J_2(\x^{(k)},\z)$}
	\Comment{Optimize~\eqref{eq:J2} w.r.t. $\z$ using Alg. \ref{algo:hierarchical_latent_reg}}
	\State $\z^{(k+1)} = E_{\tauvec} (\x^{(k)})$
        \State \textcolor{gray}{\% $\min_{\x} J_1(\x,\z^{(k+1)})$}
        \Comment{Optimize~\eqref{eq:J1} w.r.t. $\x$}
        \State $\x^{(k+1)} = \left(A^tA+\frac{\sigma^2}{\gamma^2}\id\right)^{-1}
        \hspace{-4pt}\left(
            A^t\y+\frac{\sigma^2}{\gamma^2}\mut{\z^{(k+1)}}
        \right)$ 
        \State $res \gets ||\x^{(k+1)}-\x^{(k)}||$;  $k \gets k+1$
        \EndWhile
        \State \Return $\x^{(k)}$
    \end{algorithmic}
\end{algorithm}

 

\section{Convergence analysis}\label{sec:convergence}
We now analyse the convergence of Algorithm~\ref{algo:final}. Following the work of~\cite{attouch2010proximal},  the alternate optimization scheme converges if  $\qp{\z|x}=\pt{\z|\x}$ and the greedy optimization scheme in Algorithm~\ref{algo:hierarchical_latent_reg} actually solves $\min_{\z} %
J_2(\x, \z)$. %
In practice, it is difficult to verify if these hypotheses hold. 
We propose to theoretically study algorithm~\ref{algo:final}, and next verify empirically that the assumptions are met.

In section~\ref{ssec:pnp}, we reformulate Algorithm~\ref{algo:final} as a Plug-and-Play algorithm, where the HVAE reconstruction takes the role of the denoiser. 
Then we study in section~\ref{ssec:fp_conv} the fixed-point convergence of the algorithm. %
Finally, section~\ref{ssec:num_conv} contains numerical experiments with the patch HVAE architecture proposed in section~\ref{ssec:patchVDVAE}. We empirically show that the patch architecture satisfies the aforementioned technical  assumptions and then illustrate the numerical  convergence and the stability of our alternate algorithm. 


\subsection{Plug-and-Play HVAE}\label{ssec:pnp}
In this section we  make the assumption that the HVAE decoder is Gaussian with a constant variance on its diagonal~\eqref{eq:decoder_constant}.
We rely on the proximal operator of a convex function~$f$ that is defined as $\prox_f(\x) = \arg\min_{\bm{u}} f(\bm{u}) + \frac{1}{2}||\x - \bm{u}||^2$.

\begin{proposition}
    \label{prop:pnp_vae}
    Assume the decoder is defined as in~\eqref{eq:decoder_constant}.
    Denote $\HVAE(\x, \tauvec) := \mut{\enc{\x}}$. Then the alternate scheme described in Algorithm~\ref{algo:final} writes
    \begin{equation}
        \label{eq:pnp-vae}
        \x_{k+1} = \prox_{\gamma^2f}\left(\HVAE\left(\x_k, \tauvec\right)\right).
    \end{equation}
\end{proposition}
From relation~\eqref{eq:pnp-vae}, algorithm~\ref{algo:final} is a Plug-and-Play Half-Quadratic Splitting method~\cite{ryu2019plug} where the role of the denoiser is played by the reconstruction $\HVAE\left(\x_k, \tauvec\right)$.
We now derive from relation~\eqref{eq:pnp-vae} sufficient conditions to establish the convergence of the iterations. %
\subsection{Fixed-point convergence}\label{ssec:fp_conv}
Let us denote $\T$ the operator corresponding to one iteration of \eqref{eq:pnp-vae}:
   $ \T(\x) = \prox_{\gamma^2}f\left(\HVAE\left(\x, \tauvec\right)\right)$.
The Lipschitz constant of $\T$ can then be expressed as a function of $f$ and the HVAE reconstruction operator $\HVAE\left(\x_k, \tauvec\right)$.
\begin{proposition}[Proof in supplementary]
    \label{eq:prop_lipschitz}
   Assume that the decoder has a constant variance $\Vti{\z} = \frac{1}{\gamma^2}\id$ for all $\z$; and  the autoencoder with latent regularization is $L_{\tauvec}$-Lipschitz, {\em i.e.} $\forall \bf{u}$, $\bf{v} \in \R^n$: $
            || \HVAE\left(\bf{u}, \tauvec\right) - \HVAE\left(\bf{v}, \tauvec\right) || \leq L_{\tauvec} ||\bf{u} -\bf{v}||$.
   Then, denoting as $\lambda_{\min}$ the smallest eigenvalue of $A^tA$, we have
    \begin{equation}\label{eq:lips_const}
        ||\T({\bf u}) - \T({\bf v})|| \leq \frac{\sigma^2}{\gamma^2\lambda_{\min} + \sigma^2}L_{\tau}||{\bf u} -{\bf v}||.
    \end{equation}
\end{proposition}


\begin{corollary}\label{cor:convergence}
If $\HVAE\left(\x_k, \tauvec\right)$  is $L_{\tauvec}<1$-Lipschitz, then iterations~\eqref{eq:pnp-vae} converge.
\end{corollary}


\begin{proof}
If  $L_{\tauvec} < 1$, then $\HVAE\left(\x_k, \tauvec\right)$ is a contraction. Hence $T$ is also a contraction form proposition~\ref{eq:prop_lipschitz} and consequently, Banach theorem ensures the convergence of the iteration $\x_{k+1} = T(\x_k)$  to a fixed point of $T$.
 \end{proof}
 

\begin{proposition}[Proof in supplementary]
    \label{prop:fixed_point}
    $\x^{\star}$ is a fixed point of $\T$ if and only if:
    \begin{equation}
        \label{eq:fixed_point}
        \nabla f(\x^{\star}) =  \frac{1}{\gamma^2}\left(\HVAE\left(\x^{\star}, \tauvec\right)-\x^{\star}\right)
    \end{equation}
\end{proposition}



Proposition~\ref{prop:fixed_point} characterizes the solution of the latent-regularization scheme, in the case where the HVAE reconstruction is a contraction.
Under mild assumptions, the fixed point condition can be stated as a critical point condition
$$\nabla f(\x^*)+\nabla g(\x^*)=0,$$
of the objective function $f(\x) + g(\x) = - \log p(\y|\x) - \log \ptt{\x}$, where  the tempered prior is  the marginal
$ \ptt{\x} := \int \ptt{\x,\z} d\z $
of the joint tempered prior defined in~\eqref{eq:hvae_joint_model_temp}.
This result, detailed in the supplementary, follows from an interpretation of $\HVAE(\x,\tauvec)$ as an MMSE denoiser. As a consequence Tweedie's formula provides the link between the right-hand side of equation~\eqref{eq:fixed_point} and $\nabla g$.



\subsection{Numerical convergence with PatchVDVAE }\label{ssec:num_conv}
We illustrate the numerical convergence of Algorithm~\ref{algo:final}. We first analyse the Lipschitz constant of the HVAE reconstruction with the PatchVDVAE architecture proposed  in section~\ref{ssec:patchVDVAE}. Then we study the empirical convergence of the algorithm and show that it outperforms the baseline optimisation of the joint MAP~\eqref{eq:J1} with the Adam optimizer.

\noindent
{\bf   Lipschitz constant of the HVAE reconstruction.}
Corollary~\ref{cor:convergence} establishes the fixed point convergence of our proposed optimization algorithm under the hypothesis that the reconstruction with latent regularization is a contraction, \emph{i.e.} $L_{\tauvec} < 1$. 
We now show thanks to an empirical estimation of the Lipschitz constant $L_{\tauvec}$ that our PatchVDVAE network empirically satisfies such a property when applied to noisy images.  %
We present  in figure~\ref{fig:hist_lipschitz} the histograms of the ratios $r={||\HVAE(\bf{u}, \tauvec)-\HVAE(\bf{v}, \tauvec)||}/{||\bf{u}-\bf{v}||}$, where $\bf{u}$ and $\bf{v}$ are natural images extracted from the BSD dataset and corrupted with  white Gaussian noise. These ratios give a lower bound for the true Lipschitz constant $L_{\tauvec}$.
Although it is possible to set different temperature $\tau_l$ at each level, we fixed a constant temperature amongst all levels to limit the number of hyperparameters.
We realized tests for 3 temperatures $\tau \in \{0.6, 0.8, 0.99\}$, and 3 noise levels $\sigma \in \{0, 25, 50\}$.
On clean images ($\sigma=0$), the distribution of ratios in close to $1$. This suggests that the HVAE is well trained and accurately  models clean images. In some rare case, a ratio $r\geq 1$ is observed for clean images. This indicates that the reconstruction is not a contraction everywhere, in particular on the manifold of clean images. 

On noisy images $\sigma>0$, the reconstruction behaves as a contraction, as the ratio $r<1$ is always  observed. Moreover, reducing the temperature of the latent regularization $\tau$ increases the strength of the contraction. This suggests that with the trained PatchVDVAE architecture, the hypothesis $L_{\tauvec}<1$ in  Corollary~\ref{cor:convergence} holds for noisy images.

\begin{figure}
    \includegraphics[width=\columnwidth,trim={0 0cm 0 .4cm},clip]{data/hist.pdf}\vspace{-0.2cm}
    \caption{\label{fig:hist_lipschitz}Numerical estimation of the Lipschitz constant of PatchVDVAE reconstruction  with different temperatures $\tau$. We present the histogram of ratio values $\frac{||\HVAE(\bf{u}, \tau)-\HVAE(\bf{v}, \tau)||}{||\bf{u}-\bf{v}||}$, where $\bf{u}$ and $\bf{v}$ are  natural images corrupted with white Gaussian noise of different standard deviations $\sigma$. For noisy images ($\sigma>0)$, the observed Lipschitz constant is always less than $1$.\vspace{-0.2cm}}
\end{figure}

\noindent
{\bf   Convergence of Algorithm~\ref{algo:final}}
We now illustrate the effectiveness of PnP-HVAE through comparisons with the optimization of the objective $J_1(\x_k,\z_k)$ in~\eqref{eq:J1} using  the Adam algorithm~\cite{kingma2014adam} for two  learning rates $lr \in \{0.01, 0.001\}$.  The left plot in figure~\ref{fig:conv} shows that Adam is able to estimate a better minimum of $J_1$. However, our alternate algorithm requires a smaller number of iterations to converge. %

On the other hand, as illustrated by the right plot in figure~\ref{fig:conv}, the use of Adam involves numerical instabilities. Oscillations of the ratio $ L_k :=\frac{||\T\left(\x_{k+1}\right) - \T\left(\x_{k}\right)||}{||\x_{k+1} - \x_{k}||} %
$ are even increased  with larger learning rates, whereas our method provides a stable  sequence of iterates.

More important, we finally exhibit the better quality of the restorations obtained with our alternate algorithm  %
on inpaiting, deblurring and super-resolution of face images. In these experiments, we used the hierarchical VDVAE model~\cite{child2021very}  trained on the FFHQ dataset \cite{karras2019style}.
Figure~\ref{fig:face_restoration} (see $2$nd and $4$th columns) and table~\ref{table:comp_ILO} (PSNR, SSIM and LPIPS scores) illustrate that the quality of the images restored with our alternate optimization algorithm is higher than the ones obtained with Adam. This suggests that for image restoration purposes, our optimization method is able to find a more relevant local minimum of $J_1$ than Adam.



\begin{figure}[t]
    \label{fig:convergence}
    \centering
    \begin{tabular}{cc}
    \hspace{-.2cm}\includegraphics[height=.33\linewidth]{{data/adam_vs_jpmap/butterfly_T_0.70_J1_2}.pdf}&\hspace{-.2cm}\includegraphics[height=.33\linewidth]{{data/adam_vs_jpmap/butterfly_T_0.70_Lk_2}.pdf}\vspace{-0.4cm}
    \end{tabular}
    \caption{\label{fig:conv} Comparison of the convergence of PnP-HVAE algorithm~\ref{algo:final} with respect to the baseline Adam optimizer, on a deblurring problem.
     Left (Convergence of the function value): PnP-HVAE converges faster to a minimum of the joint posterior  $J_1(\x_k,\z_k)$  in~\eqref{eq:J1}. 
   Right (Convergence of iterates $\x_k$): PnP-HVAE is more stable than Adam.\vspace{-0.1cm}}
\end{figure}


\begin{figure}[ht]
    \center
    \includegraphics[width=0.95\columnwidth]{figures/demo_face_restoration_wdps.pdf}\vspace{-0.1cm}
    \caption{\label{fig:face_restoration}
        Visual comparaison of image restoration methods based on deep generative models. We studied $3$  tasks on face images:  inpainting (top), deblurring (middle), super-resolution (bottom). 
        Contrary to the optimization of the objective~\eqref{eq:J1} with Adam, our alternate algorithm generates realistic results, on par with ILO~\cite{daras2021intermediate}, while remaining consistant with the observation.\vspace{-0.3cm}
    }
\end{figure}



\newcolumntype{x}[1]{>{\centering\arraybackslash}p{#1}}
\begin{table}[ht]\small
    \centering
        \begin{tabular}{x{1.15cm}x{1.5cm}x{.65cm}x{.65cm}x{.65cm}x{1cm}}
           & &PSNR$\uparrow$ & SSIM$\uparrow$ & LPIPS$\downarrow$ & time (s) \\
        \hline    
        \multirow{2}{*}{SR $\times 4$}&Adam & $28.56$ & $0.75$ & $0.38$ & $\underline{26}$  \\
       \multirow{2}{*}{$\sigma=3$}& ILO & $\underline{28.80}$ & $\underline{0.78}$ & $\mathbf{0.17}$ & $34$\\
       & PnP-HVAE & $\mathbf{29.32}$ & $\mathbf{0.82}$ & $0.28$ & $\mathbf{15}$\\ 
       & DPS & $27.53$ & $0.76$ & $\underline{0.21}$ & $153$\\
        \hline 
        Deblurring &Adam & $26.69$ & $0.75$ & $0.27$ &$ \underline{12}$  \\
        (motion) &ILO & $\underline{29.01}$ & $\underline{0.80}$ & $\underline{0.20}$ & $34$ \\
        $\sigma=8$&PnP-HVAE & $\mathbf{30.40}$ & $\mathbf{0.84}$ & $\mathbf{0.16}$ & $\mathbf{10}$\\
         & DPS & $28.70$ & $\underline{0.80}$ & $0.23$& $142$\\
        \hline 
         Deblurring &Adam & $\underline{30.17}$ & $\underline{0.83}$ & $\underline{0.21}$& $\underline{12}$\\
         (Gaussian) &ILO & $29.12$ & $0.79$ & $\mathbf{0.17}$ & $34$\\
        $\sigma=8$&PnP-HVAE& $\mathbf{30.81}$ & $\mathbf{0.86}$ & $0.24$ & $\mathbf{10}$\\
        & DPS & $29.14$ & $0.81$ & $ $0.24$ $ & $142$\\
        \end{tabular}\vspace*{-0.1cm}
    \caption{\label{table:comp_ILO}Quantitative evaluation on face restoration. Best results in {\bf bold}, second best \underline{underlined}.\vspace{-0.2cm}}
\end{table}



\section{Image restoration results}\label{sec:expe}
We present in section~\ref{ssec:faces} an application of PnP-HVAE on face images, using a pretrained state-of-the-art hierarchical VAE. 
Next, we study the application of our framework to natural images. To that end, we introduce  in section~\ref{ssec:patchVDVAE}  a patch hierachical VAE architecture, that is able to model natural images of different resolutions. In section~\ref{ssec:app_nat}, we provide deblurring, super-resolution and inpainting experiments to demonstrate the relevance of the proposed method.

Additional results are presented in Appendix~\ref{app:add}. All experiments can be reproduced using the code available at \url{https://github.com/jprost76/PnP-HVAE}.



\subsection{Face Image restoration (FFHQ)}\label{ssec:faces}
We first demonstrate the effectiveness of PnP-HVAE on highly structured data, by performing face image restoration.
Latent variable generative models can accurately model structured images such as face images \cite{karras2019style,vahdat2020nvae,child2021very,kingma2018glow}, and then be used to produce high quality restoration of such data. 
In our experiments, we use the VDVAE model of~\cite{child2021very}, pre-trained on the FFHQ dataset~\cite{karras2019style}, as our hierarchical VAE prior.
VDVAE has $L=66$ latent variable groups in its hierarchy and generates images at resolution $256\times256$.

We compare PnP-HVAE with the intermediate layer optimization algorithm (ILO)~\cite{daras2021intermediate} that is based on a different class of generative models than HVAE. ILO is a GAN inversion method which optimizes the image latent code along with the intermediate layer representation of a StyleGAN to generate an image consistent with a degraded observation.
We use the official implementation of ILO, along with a StyleGAN2 model~\cite{karras2020analyzing, stylegan2pytorch}, that was trained for 550k iterations on images of resolution $256\times256$ from FFHQ.  
As VDVAE and StyleGAN models are not trained on the same train-test split of FFHQ, we chose to evaluate the methods on a subset of 100 images from the CelebA dataset~\cite{liu2018large}. 
For super-resolution, the degradation model corresponds to the application of a gaussian low-pass filter followed by a $\times 4$ sub-sampling, and the addition of a gaussian white noise with $\sigma=3$.
For the deblurring, we considered motion blur and  gaussian kernels, both with a noise level $\sigma=8$. %

We provide quantitative comparisons in table~\ref{table:comp_ILO}, along with a visual comparison of the results in figure~\ref{fig:face_restoration}.
PnP-HVAE has the best  PSNR and SSIM results for all the considered restoration tasks, while ILO provides better results  for the perceptual distance.
By jointly optimizing the image and its latent variable, PnP-HVAE provides  results that are both realistic and consistent with the degraded observation.
On the other hand,  ILO  only optimizes on an extended latent space. This method generates  sharp and realistic images with better LPIPS scores,   
but the results lack  of consistency with respect to the observation, which explains the overall lower PSNR performance. 






\subsection{PatchVDVAE: a HVAE for natural images}\label{ssec:patchVDVAE}
Available generative models in the literature operate on images of  fixed resolutions and
are either restrained to datasets of limited diversity, or even to registered face images~\cite{kingma2018glow,child2021very, vahdat2020nvae, karras2019style}, or requiring additional class information~\cite{brock2018large, dhariwal2021diffusion, song2020score, luhman2022optimizing}.
Fitting an unconditional model on natural images appears to be a more difficult task, as their resolution can change, and their content is highly diverse.
The complexity of the problem can be reduced by learning a prior model on patches of reduced dimension. 
For image restoration problems, the patch model can be reused on images of higher dimensions~\cite{zoran2011learning,prost2021learning,altekruger2022patchnr}. When the model is a full CNN, the prior on the set of the  patches can  be computed efficiently by applying the network on the full image~\cite{prost2021learning}.

We thus introduce  patchVDVAE, a fully convolutional hierarchical VAE.
Contrary to existing HVAE models whose resolution is constrained by the constant tensor at the input of the top-down block, patchVDVAE can generate images of different resolutions by controlling the dimension of the input latent. 
This amounts to defining a prior on patches whose dimension corresponds to the receptive field of the VAE. A similar model is used for image denoising in~\cite{prakash2021interpretable}.

 
For PatchVDVAE architecture, we use the same bottom-up and top-down blocks as VDVAE~\cite{child2021very}, and replace the constant trainable input in the first top-down block by a latent variable, to make the model fully convolutional (details on the  architecture are given in Appendix~\ref{app:details}). 
The training dataset is composed of $128\times 128$ patches extracted from a combination of DIV2K~\cite{agustsson2017ntire} and Flickr2K~\cite{Lim_2017_CVPR_workshops} datasets.
We perform data augmentation by extracting  patches at $3$ resolutions: HR-images and $\times 2$ and $\times 4$ downscaled images. 
The model is trained for $7.10^5$ iterations with a batch size of $64$. Following the recommendation of~\cite{hazami2022efficient}, we use Adamax optimizer with an exponential moving average and gradient smoothing of the variance.
We set the decoder model to be a gaussian with diagonal covariance, as in~\cite{luhman2022optimizing}.
PatchVDVAE is fully convolutional and can generate images of dimension that are multiples of $64$ as illustrated by
figure~\ref{fig:vdvae}.

\newlength{\patchwidth}
\setlength{\patchwidth}{0.135\columnwidth}
\begin{figure}[!ht]
    \centering
    \begin{subfigure}[t]{.34\columnwidth}\hspace{0.1cm}
        \setlength{\tabcolsep}{0.02pt}
\renewcommand{\arraystretch}{0}
        \begin{tabular}{*{2}{p{1.03\patchwidth}}}
            \includegraphics[width=\patchwidth]{figures_arxiv/patchVDVAE/samples/generated/64x64/setup-5-image-0018.png} &
            \includegraphics[width=\patchwidth]{figures_arxiv/patchVDVAE/samples/generated/64x64/setup-5-image-0016.png} \\
            \includegraphics[width=\patchwidth]{figures_arxiv/patchVDVAE/samples/generated/64x64/setup-5-image-0008.png} &
            \includegraphics[width=\patchwidth]{figures_arxiv/patchVDVAE/samples/generated/64x64/setup-5-image-0019.png}   
        \end{tabular}
    \end{subfigure}\hspace{-0.15cm}
    \begin{subfigure}[t]{.64\columnwidth}
\begin{tabular}{cc}\vspace{-0.1cm}
\includegraphics[width=2\patchwidth]{figures_arxiv/patchVDVAE/samples/generated/256x256/setup-2-image-0009.png}&
        \includegraphics[width=2\patchwidth]{figures_arxiv/patchVDVAE/samples/generated/256x256/setup-2-image-0002.png}\end{tabular}

    \end{subfigure}
    \caption{\label{fig:vdvae} Left: $64\times64$ patches samples from our patchVDVAE model trained on patches from natural images.
    Right: PatchVDVAE is fully convolutional and it can generate images of higher resolution (here: $128\times128$).\vspace{-0.2cm}}
\end{figure}

\subsection{Natural images restoration}\label{ssec:app_nat}
We  evaluate PnP-HVAE on natural image restoration.
For each task, we report the average value of the PSNR, the SSIM, and the LPIPS metrics on $20$ images from the test set of the BSD dataset~\cite{MartinFTM01}.\\


\noindent
{\bf Image deblurring.}
In the experiments, we consider $2$ gaussian kernels and $2$ motion blur kernels from~\cite{levin2009understanding}, with $3$ different noise levels 
$\sigma \in \{2.55, 7.65, 12.75\}$.
As a baseline we consider  EPLL~\cite{zoran2011learning}, which learns a prior on image patches with a gaussian mixture model.
We also compare PnP-HVAE  with PnP-MMO and GS-PnP, $2$ competing convergent Plug-and-Play methods based on CNN denoisers.
PnP-MMO~\cite{pesquet2021learning} restricts the denoiser to be contraction in order to guarantee the convergence of the PnP forward-backard algorithm. GS-PnP~\cite{hurault2022gradient} considers a gradient step denoiser and reaches state-of-the-art performances of non converging methods~\cite{zhang2021plug}.
We set the temperature $\tau$  in our method as $0.95$, $0.8$ and $0.6$ for noise levels $2.55$, $7.65$ and $12.75$ respectively, and we let it run for a maximum of $50$ iterations. 
For the three compared methods we use the official implementations and pre-trained models provided by the respective authors. 
Details on the choice of hyperparameters for the concurrent methods are provided in the Appendix~\ref{app:details}
Figure~\ref{fig:deblurring_bsd} illustrates that our method provides correct deblurring results. 

According to table~\ref{tab:deb}, the performance of PnP-HVAE is between those of EPLL and GS-PnP and it outperforms PnP-MMO for large noise levels.\\

\begin{table}
\begin{center}\footnotesize
    \begin{tabular}{>{\centering}m{.3cm}*{5}{c}}
    $\sigma$ &Method & PSNR$\uparrow$ & SSIM$\uparrow$ & LPIPS$\downarrow$  \\ 
    \hline
    \multirow{4}{*}{\vcell{$2.55$}}
    & PnP-HVAE & $27.75$ & $0.79$ & $0.31$\\
    & GS-PNP \cite{hurault2022gradient} & $\mathbf{29.59}$ & $\mathbf{0.84}$ & $\mathbf{0.22}$\\
    & EPLL \cite{zoran2011learning} & $26.49$ & $0.71$ & $0.36$\\ 
    & PnP-MMO \cite{pesquet2021learning} & $\underbar{29.50}$ & $\underbar{0.83}$ & $\underbar{0.20}$ \\ \hline
    \multirow{4}{*}{\vcell{$7.65$}}
    & PnP-HVAE & $\underbar{26.36}$ & $\underbar{0.72}$ & $\underbar{0.40}$\\
    & GS-PNP \cite{hurault2022gradient} & $\mathbf{27.33}$ & $\mathbf{0.77}$ & $\mathbf{0.31}$\\
    & EPLL \cite{zoran2011learning} & $24.04$ & $0.66$ & $0.45$ \\ 
    & PnP-MMO \cite{pesquet2021learning} & $25.34$ & $0.69$ & $0.34$\\
    \hline
    \multirow{4}{*}{\vcell{$12.75$}}
    & PnP-HVAE & $\underbar{25.12}$ & $\mathbf{0.73}$ & $\underbar{0.47}$\\
    & GS-PNP \cite{hurault2022gradient} & $\mathbf{26.32}$ & $\mathbf{0.73}$ & $\mathbf{0.37}$\\
    & EPLL \cite{zoran2011learning} & $23.28$ & $0.61$ & $0.51$ \\ 
    & PnP-MMO \cite{pesquet2021learning} & $22.42$ & $0.53$& $0.54$ \\
    \hline
    &\vspace*{-.3cm}\\
            \multicolumn{2}{c}{Blur and motion kernels}& \multicolumn{3}{c}{
        \includegraphics*[scale=1]{figures_arxiv/kernels/4.png}\;\includegraphics*[scale=1]{figures_arxiv/kernels/7.png}\;\includegraphics*[scale=1]{figures_arxiv/kernels/9.png}\;\includegraphics*[scale=1]{figures_arxiv/kernels/11.png}} 
    \end{tabular}
        \caption{\label{tab:deb}Comparison  of PnP-HVAE  and other restoration methods on deblurring. Results are averaged on $4$ kernels.\vspace{-0.2cm}}% on image deblurring.}
    \end{center}
\end{table}

\begin{figure}
    
    \begin{subfigure}[h]{\linewidth}
        \centering
        \includegraphics*[width=\columnwidth]{figures_arxiv/deb_s255_k7.pdf}\vspace{-0.1cm}
        \caption{Gaussian blur, $\sigma=2.55$}
    \end{subfigure}
    \begin{subfigure}[h]{\linewidth}
        \centering
        \includegraphics*[width=\columnwidth]{figures_arxiv/deb_s765_k11.pdf}\vspace{-0.1cm}
        \caption{Motion blur, $\sigma=7.65$}
    \end{subfigure}\vspace*{-0.1cm}
    \caption{\label{fig:deblurring_bsd} Natural image deblurring\vspace{-0.1cm}}
\end{figure}

\noindent {\bf Effect of the temperature.}
PnP-HVAE gives control on the temperature of the prior over the latent space.
In figure~\ref{fig:temp_effect}, we illustrate that reducing the temperature increases the strength of the regularization prior. In this example the tuning $\tau=0.7$ produces the best performance.\\
\begin{figure}[!ht]
   
    \includegraphics[width=\columnwidth]{figures_arxiv/demo_temp.pdf}\vspace{-0.15cm}
    \caption{ \label{fig:temp_effect} Effect of the temperature in PnP-VAE on a deblurring problem, with $\sigma=7.65$.\vspace{-0.15cm}}
\end{figure}


\noindent
{\bf Image inpainting.}
Next we consider the task of noisy image inpainting. 
We compose a test-set of 10 images from the validation set of BSD~\cite{MartinFTM01} and we create masks
  by occluding diverse objects of small size in the images. 
A gaussian white noise with $\sigma=3$ is added to the images.
As a comparaison, we still consider GS-PnP and EPLL.
For PnP-HVAE, the temperature is set to $\tau=0.6$, and the algorithm is run for a maximum of $200$ iterations, unless the residual $||\x_{k+1}-\x_k||$ is on a plateau.
We provide on Table~\ref{tab:inpainting_bsd} the distortion metrics with the ground truth, as well as a visual
\begin{table}



\begin{center}
    \begin{tabular}{cccc}
        & PSNR$\uparrow$ & SSIM$\uparrow$ &LPIPS$\downarrow$ \\\hline
        PnP-HVAE  & $\mathbf{29.54}$ & $\mathbf{0.93}$ & $\mathbf{0.06}$\\
        GS-PNP & $28.52$ & $\mathbf{0.93}$ & $0.09$\\
        EPLL & $\underline{29.16}$ & $\mathbf{0.93}$ & $\mathbf{0.06}$\\
    \end{tabular}
    \caption{\label{tab:inpainting_bsd}Quantitative evaluation for inpainting on BSD.}
    \end{center}
\end{table}
comparison on figure~\ref{fig:inpainting_bsd}. 
With its hierarchical structure,  PnP-HVAE outperforms the compared methods. \vspace{0.05cm}



\begin{figure}[!h]
    \includegraphics[width=\columnwidth]{figures_arxiv/demo_inp_bsd2.pdf}\vspace{-0.1cm}
    \caption{\label{fig:inpainting_bsd}Natural image inpainting\vspace{-0.3cm}}
\end{figure}












\section{Conclusion}
\section{Conclusion}\label{sec:conclusion}
In this work, we focus on addressing the fundamental challenge of OOD detection tasks, which is how to fully understand the semantic discrepancy between the ID/OOD samples. We reveal that the key to success in the realistic SCOOD task is to allocate as many ID samples in the unlabeled set correctly as possible. To this end, we propose a novel uncertainty-aware optimal transport scheme that introduces class-specific energy scores as guidance for effective label assignment. Experimental results show that our method achieves better performance than previous state-of-the-art methods on SCOOD benchmarks.

\textbf{Limitations.} In addition to temperature scaling, other techniques such as feature clipping applied in ReAct~\cite{sun2021react} also enhance the performance of energy score, so how to obtain an OOD score that best fits the SCOOD task can be further explored. Moreover, a setting highly related to SCOOD has been proposed in \cite{katz2022training} and formulated as a constrained optimization problem. We will also theoretically analyze these practical OOD settings in our feature work.

% \section*{Acknowledgments}
\textbf{Acknowledgments.} 
This work is supported by National Key R\&D Program of China under Grant 2020AAA0105701, National Natural Science Foundation of China (NSFC) under Grants 61872327, Major Special Science and Technology Project of Anhui, National Natural Science Foundation of China (62033012) and Ant Group through Ant Research Intern Program.


\ificcvfinal
\section*{Acknowledgements}
This study has  been carried out with financial support from the French Research Agency through the PostProdLEAP project (ANR-19-CE23-0027-01).
Experiments presented in this paper were carried out using the PlaFRIM experimental testbed, supported by Inria, CNRS (LABRI and IMB), Université de Bordeaux, Bordeaux INP and Conseil Régional d’Aquitaine (see https://www.plafrim.fr)
\fi
\clearpage

{\small
\bibliographystyle{ieee_fullname}
\bibliography{biblio}
}

\clearpage

\onecolumn
\appendix
\section*{Summary}

This supplementary material contains:
\begin{itemize}
\item proofs of the theoretical results of the main paper in section~\ref{proofs}
\item additional implementation details in section~\ref{details} 
\item a discussion on the contractivity of the autoencoder and its fixed points in section~\ref{sec:contractivity}
\item additional  comparisons with the competing methods  in section~\ref{add}
\end{itemize}
\section{Proofs of the main results}\label{proofs}
In this section we provide proofs relative to Algorithm~\ref{algo:hierarchical_latent_reg}, Proposition~\ref{eq:prop_lipschitz}, Proposition~\ref{prop:fixed_point} and the characterization of the fixed point given by  Algorithm~\ref{algo:final}.
% To hide proofs : \newcommand{\maybehide}[1]{}
% To show proofs : \newcommand{\maybehide}[1]{#1}
\newcommand{\maybehide}[1]{#1}

\section{Proofs}
\label{sec:proofs}

\subsection{Weak Open CBV}

\subsubsection{General Lemmas}

\begin{proposition}[{\bf Diamond}]
    \label{prop:diamond}
The relation $\redcbv$ enjoys the diamond property: if $t \redcbv t_i\ (i=1,2)$ and $t_1 \neq t_2$, then there exists $t_3$ such that $t_i \redcbv t_3\ i=1,2$.
\end{proposition}

\propcharnfs*

\maybehide{\begin{proof}
    We are going to show this proposition by splitting the original statement into the two following ones:
    \begin{enumerate}
        \item \label{prop:char-nfs:1} $t \not\dred$ and $\neg\isvalue{t}$ iff $t \in \neutral$.
        \item \label{prop:char-nfs:2} $t \not\dred$ iff $t \in \normal$.
    \end{enumerate}
    The proof now follows by simultaneous induction over both these statements:
    \begin{itemize}
        \item[$\Ra$)] By induction over $t$: 
        \begin{enumerate}
            \item Let $t \not\dred$ and $\neg\isvalue{t}$. We want to show that $t \in \neutral$:
            \begin{itemize}
                \item Case $t = x$ or $t = \lam x.u$. Then $\neg\isvalue{t}$ does not hold. Therefore, the statement holds vacuously.
                \item Case $t = u p$. Since $u p \not\dred$, then, in particular, it must be the case that either $\neg\isabs{u}$ or $\neg\isvalue{p}$ must hold, according to rule (\ruleBeta):
                \begin{itemize}
                    \item Assume $\neg\isabs{u}$ holds. It must be the case that $u \not\dred$, according to rule (\ruleAppL). And it also must be the case that $p \not\dred$, according to rule (\ruleAppR). Therefore, $p \in \normal$, by the \ih (\cref{prop:char-nfs}.\ref{prop:char-nfs:2}). Now, we have to consider $u$, which can be a variable, or not:
                    \begin{itemize}
                        \item Case $u = x$. Then $u p \in x \ \normal \in \neutral$.
                        \item Case $u$ is not a variable. Then $\neg\isvalue{u}$ holds. Therefore, we have $u \in \neutral$, by the \ih (\cref{prop:char-nfs}.\ref{prop:char-nfs:1}). Thus, $u p \in \neutral \ \normal \in \neutral$.
                    \end{itemize}
                    \item Assume $\neg\isvalue{p}$ holds. Then it must be the case that $u \not\dred$, according to rule (\ruleAppL). And that $p \not\dred$, according to rule (\ruleAppR). Therefore, $u \in \normal$, by the \ih (\ref{prop:char-nfs}.\ref{prop:char-nfs:2}), and $p \in \neutral$, by the \ih (\cref{prop:char-nfs}.\ref{prop:char-nfs:1}). Thus, $u p \in \normal \ \neutral \in \neutral$.
                \end{itemize}
            \end{itemize}
            \item Let $t \not\dred$. We want to show that $t \in \normal$:
            \begin{itemize}
                \item Case $t \in \val$. Then, clearly $t \in \normal$.
                \item Case $t \not\in \val$. Then, $\neg\isvalue{t}$ holds. Therefore, $t \in \neutral$, by \cref{prop:char-nfs}.\ref{prop:char-nfs:1}. Thus, in particular, $t \in \normal$.
            \end{itemize}
        \end{enumerate}
        \item[$\La$)] By induction over $t \in \normal$:
        \begin{enumerate}
            \item Let $t \in \neutral$. We want to show that $t \not\dred$ and $\neg\isvalue{t}$:
            \begin{itemize}
                \item Case $t = u p \in x \ \normal$. Then $u = x$ and $p \in \normal$. Since $u = x$, then both rules (\ruleBeta) and (\ruleAppL) cannot be applied. Since $p \in \normal$, then $p \not\dred$, by the \ih (\cref{prop:char-nfs}.\ref{prop:char-nfs:2}). Therefore, rule (\ruleAppR) also cannot be applied. Thus, $u p \not\dred$. And we can conclude, since $\neg\isvalue{u p}$ clearly holds.
                \item Case $t = u p \in \normal \ \neutral$. Then $u \in \normal$ and $p \in \neutral$. Since $u \in \normal$, then $u \not\dred$, by the \ih (\cref{prop:char-nfs}.\ref{prop:char-nfs:2}). Since $p \in \neutral$, then $p \not\dred$ and $\neg\isvalue{p}$ holds, by the \ih (\cref{prop:char-nfs}.\ref{prop:char-nfs:1}). Since $\neg\isvalue{p}$, then rule (\ruleBeta) cannot be applied. Since $u \not\dred$ and $p \not\dred$, then rules (\ruleAppL) and (\ruleAppR) cannot be applied. Therefore, $u p \not\dred$. And we can conclude since $\neg\isvalue{u p}$ clearly holds.
                \item Case $t = u p \in \neutral \ \normal$. Then $u \in \neutral$ and $p \in \neutral$. Since $u \in \neutral$, then $u \not\dred$ and $\neg\isvalue{u}$ holds, by the \ih (\cref{prop:char-nfs}.\ref{prop:char-nfs:1}). Since $p \in \normal$, then $p \not\dred$, by the \ih (\cref{prop:char-nfs}.\ref{prop:char-nfs:2}). Since $\neg\isvalue{u}$, then rule (\ruleBeta) cannot be applied. Since $u \not\dred$ and $p \not\dred$, then rules (\ruleAppL) and (\ruleAppR) cannot be applied. Therefore $u p \not\dred$. And we can conclude since $\neg\isvalue{u p}$ clearly holds.
            \end{itemize}
            \item Let $t \in \normal$. We want to show that $t \not\dred$:
            \begin{itemize}
                \item Case $t \in \val$. Then, clearly $t \not\dred$.
                \item Case $t \not\in \val$. Then, $t \in \neutral$, by definition. Thus, $t \not\dred$ holds, by~\cref{prop:char-nfs}.\ref{prop:char-nfs:1}.
            \end{itemize}
        \end{enumerate}
    \end{itemize}
\end{proof}
}
  
\begin{lemma}[Relevance]
    Let $\Phi \tr \seqi{\Gam}{t}{\tau}{(b,s)}$. Then $\dom{\Gam} \subseteq \fv{t}$.
\end{lemma}

\maybehide{\begin{proof}
    The proof following by induction over $\Phi$. Case $\Phi$ ends with rule (\ruleAx) or (\ruleLamP), then $\Phi$ is clearly relevant. The other cases following easily from the \ih.
\end{proof}}

\subsubsection{Soundness (Auxiliary Lemmas)}

\begin{lemma}
    \label{lem:values-not-neutral}
    Let $\Phi \tr \seqi{\Gam}{t}{\tau}{(b,s)}$. If $t \in \val$, then $\tau \not= \tneutral$.
\end{lemma}

\maybehide{\begin{proof}
    By case analysis on the form of $t \in \val$:
    \begin{itemize}
        \item Case $t = x$. Then we have to consider two additional cases according to the last rule used in $\Phi$:
        \begin{itemize}
            \item Case $\Phi$ ends with rule (\ruleAx), then $\tau$ is of the form $\sig \not= \tneutral$.
            \item Case $\Phi$ ends with rule (\ruleMany), then $\tau$ is of the form $\M \not= \tneutral$.
        \end{itemize}
        \item Case $t = \lam x.t$. Then we have to consider three additional cases according to the last rule used in $\Phi$:
        \begin{itemize}
            \item Case $\Phi$ ends with rule (\ruleLam), then $\tau$ is of the form $\M \ta \del \not= \tneutral$.
            \item Case $\Phi$ ends with rule (\ruleMany), then $\tau$ is of the form $\M \not= \tneutral$.
            \item Case $\Phi$ ends with rule (\ruleLamP), then $\tau = \tabs \not= \tneutral$.
        \end{itemize}
    \end{itemize}
\end{proof}}

\begin{lemma}
    \label{lem:notabs-implies-negabs}
    If $\Phi \tr \seqi{\Gam}{t}{\tau}{(b,s)}$, such that $\Gam$ is tight. If $\tau \in \nott{\tabs}$, then $\neg\isabs{t}$.
\end{lemma}

\maybehide{\begin{proof}
    By induction over $\Phi$:
    \begin{itemize}
        \item Case $\Phi$ ends with rule (\ruleAx), (\ruleApp), (\ruleAppPOne), or (\ruleAppPTwo), then $\neg\isabs{t}$ holds by definition.
        \item Case $\Phi$ ends with rule (\ruleLam), (\ruleMany), or (\ruleLamP),  then $\tau \not\in \nott{\tabs}$. Therefore, these cases do not apply.
    \end{itemize}
\end{proof}}

\begin{lemma}[{\bf Zero Steps and Normal Forms}]
    \label{lem:zero-steps-nfs}
    Let $\Phi \tr \seqi{\Gam}{t}{\tau}{(b,s)}$ be tight. $b = 0$ iff $t \in \normal$.
\end{lemma}

\maybehide{\begin{proof} \mbox{}
    \begin{itemize}
        \item[$\Ra$)] We want to show that, if $b = 0$, then $t \in \normal$. For this, we are going to split the original statement into the two following ones:
        \begin{enumerate}
            \item \label{lem:zero-steps-nfs:1} Let $\Phi \tr \seqi{\Gam}{t}{\tau}{(0,s)}$ be tight and $\neg\isvalue{t}$, then $t \in \neutral$.
            \item \label{lem:zero-steps-nfs:2} Let $\Phi \tr \seqi{\Gam}{t}{\tau}{(0,s)}$ be tight, then $t \in \normal$.
        \end{enumerate}
        The proof now follows by simultaneous induction over both these statements:
        \begin{enumerate}
            \item Let $\Phi \tr \seqi{\Gam}{t}{\tau}{(0,s)}$ be tight and $\neg\isvalue{t}$:
            \begin{itemize}
                \item Case $\Phi$ ends with rule (\ruleAx), (\ruleLam), (\ruleMany), or (\ruleLamP), then $\isvalue{t}$ holds. Therefore, these cases do not apply.
                \item Case $\Phi$ ends with rule (\ruleApp), then $b > 0$. Therefore, this case does not apply.
                \item Case $\Phi$ ends with rule (\ruleAppPOne), then $t$ is of the form $up$ and $\Phi$ is of the following form:
                \[ \begin{prooftree}
                    \hypo{\Phi_u \tr \seqi{\Gam_u}{u}{\nott{\tabs}}{(0,s_u)}}
                    \hypo{\Phi_p \tr \seqi{\Gam_p}{p}{\tightt}{(0,s_p)}}
                    \infer2[(\ruleAppPOne)]{\seqi{\Gam_u + \Gam_p}{up}{\tneutral}{(0,1+s_u+s_p)}}
                \end{prooftree} \]
                where $\tau = \tneutral$, $\Gam = \Gam_u + \Gam_p$ is tight, and $s = 1 + s_u + s_p$. Moreover, $\Gam_u$ and $\Gam_p$ are tight. By the \ih (\cref{lem:zero-steps-nfs}.\ref{lem:zero-steps-nfs:2}) over $\Phi_u$ and $\Phi_p$, we have that $u, p \in \normal$. By~\cref{lem:notabs-implies-negabs}, we have that $\neg\isabs{u}$. Therefore, either $u$ is a variable or $u \in \neutral$ by definition. So, in both cases, we can conclude that $u p \in \neutral$.
                \item Case $\Phi$ ends with rule (\ruleAppPTwo), then $t$ is of the form $up$ and $\Phi$ is of the following form:
                \[ \begin{prooftree}
                    \hypo{\Phi_u \tr \seqi{\Gam_u}{u}{\tightt}{(0,s_u)}}
                    \hypo{\Phi_p \tr \seqi{\Gam_p}{p}{\tneutral}{(0,s_p)}}
                    \infer2[(\ruleAppPTwo)]{\seqi{\Gam_u + \Gam_p}{up}{\tneutral}{(0,1+s_u+s_p)}}
                \end{prooftree} \]
                where $\tau = \tneutral$, $\Gam = \Gam_u + \Gam_p$, and $s = 1 + s_u + s_p$. Moreover, $\Gam_u$ and $\Gam_p$ are tight. By the \ih (\cref{lem:zero-steps-nfs}.\ref{lem:zero-steps-nfs:2}) over $\Phi_u$, we have that $u \in \normal$. By applying~\cref{lem:values-not-neutral} to $\Phi_p$, we have that $\neg\isvalue{p}$. By the \ih (\cref{lem:zero-steps-nfs}.\ref{lem:zero-steps-nfs:1}) over $\Phi_p$, we have that $p \in \neutral$. So, in both cases, we can conclude that $up \in \neutral$.
            \end{itemize}
            \item Let $\Phi \tr \seqi{\Gam}{t}{\tau}{(0,s)}$ be tight:
            \begin{itemize}
                \item Case $\Phi$ ends with rule (\ruleAx), (\ruleLam), or (\ruleLamP). Then, clearly $t \in \val$, so we can conclude immediately.
                \item Case $\Phi$ ends with rule (\ruleMany), then $\tau$ is of the form $\M \not\in \tightt$. Therefore, this case does not apply.
                \item In all the remaining cases $\neg\isvalue{t}$ holds. Then $t \in \neutral$, by \cref{lem:zero-steps-nfs}.\ref{lem:zero-steps-nfs:1}, so $t \in \normal$.
            \end{itemize}
        \end{enumerate}
        \item[$\La)$] We want to show that, if $t \in \normal$, then $b = 0$. The proof follows by induction over $t \in \normal$:
        \begin{enumerate}
            \item Case $t \in \neutral$. Then we have to consider the following additional cases:
            \begin{itemize}
                \item Case $t = xp$, such that $p \in \normal$. Then there are three additional cases to consider:
                \begin{itemize}
                    \item Case $\Phi$ ends with (\ruleApp), then it must be of the following form:
                    \[ \begin{prooftree}
                        \hypo{\seqi{x : \mul{\M \ta \tau}}{x}{\M \ta \tau}{(0,0)}}
                        \hypo{\Phi_p \tr \seqi{\Gam_p}{p}{\M}{(b_p,s_p)}}
                        \infer2[(\ruleApp)]{\seqi{(x : \mul{\M \ta \tau}) + \Gam_p}{xp}{\tau}{(1+b_p,s_p)}}
                    \end{prooftree} \]
                    where $\Gam = (x : \mul{\M \ta \tau}) + \Gam_p$ is tight, $b = 1+b_p$, and $s = s_p$. But, $\mul{\M \ta \tau}$ is not tight, since $\M \ta \tau \not\in \tightt$. Therefore, this case does apply.
                    \item Case $\Phi$ ends with (\ruleAppPOne), then $\Phi$ must be of the following form:
                    \[ \begin{prooftree}
                        \hypo{\seqi{(x : \mul{\tvar})}{x}{\tvar}{(0,0)}}
                        \hypo{\Phi_p \tr \seqi{\Gam_p}{p}{\tightt}{(b_p,b_p)}}
                        \infer2[(\ruleAppPOne)]{\seqi{\Gam_u + \Gam_p}{up}{\tneutral}{(b_p,1+s_u+s_p)}}
                    \end{prooftree} \]
                    where $\tau = \tneutral$, $\Gam = (x : \mul{\tvar}) + \Gam_p$ is tight, $b = b_p$, and $s = 1+ s_u + s_p$. Moreover, $\Gam_p$ is tight. By the \ih over $\Phi_p$, we have that $b_p = 0$. So we can conclude with $b = b_u + b_p = 0$.
                    \item Case $\Phi$ ends with (\ruleAppPTwo). This case is very similar to the case where $\Phi$ ends with rule (\ruleAppPOne).
                \end{itemize}
                \item Case $t = up$, such that $u \in \normal$ and $p \in \neutral$. Then there are three additional cases to consider:
                \begin{itemize}
                    \item Case $\Phi$ ends with (\ruleApp), then it must be of the following form:
                    \[ \begin{prooftree}
                        \hypo{\seqi{\Gam_u}{u}{\M \ta \tau}{(b_u,s_u)}}
                        \hypo{\Phi_p \tr \seqi{\Gam_p}{p}{\M}{(b_p,s_p)}}
                        \infer2[(\ruleApp)]{\seqi{\Gam_u + \Gam_p}{up}{\tau}{(1+b_u+b_p,s_u+s_p)}}
                    \end{prooftree} \]
                    where $\tau = \tau$, $\Gam = \Gam_u + \Gam_p$ is tight, $b = 1 + b_u + b_p$, and $s = s_u + s_p$. By~\cref{lem:tight-spreading}.\ref{lem:tight-spreading:2}, we have that $\M \in \tightt$, which is a contradiction. Therefore, this case does not apply.
                    \item Case $\Phi$ ends with (\ruleAppPOne) or (\ruleAppPTwo). These cases are very similar to the corresponding cases when $t = x p$, such that $p \in \normal$.
                \end{itemize}
                \item Case $t = up$, such that $u \in \neutral$ and $p \in \normal$. Then there are three cases to consider:
                \begin{itemize}
                    \item Case $\Phi$ ends with (\ruleApp), then it must be of the following form:
                    \[ \begin{prooftree}
                        \hypo{\seqi{\Gam_u}{u}{\M \ta \tau}{(b_u,s_u)}}
                        \hypo{\Phi_p \tr \seqi{\Gam_p}{p}{\M}{(b_p,s_p)}}
                        \infer2[(\ruleApp)]{\seqi{\Gam_u + \Gam_p}{up}{\tau}{(1+b_u+b_p,s_u+s_p)}}
                    \end{prooftree} \]
                    where $\tau = \tau$, $\Gam = \Gam_u + \Gam_p$ is tight, $b = 1 + b_u + b_p$, and $s = s_u + s_p$. By~\cref{lem:tight-spreading}.\ref{lem:tight-spreading:2} over $u \in \neutral$, we have that $\M \ta \tau \in \tightt$, which is a contradiction. Therefore, this case does not apply.
                    \item Case $\Phi$ ends with (\ruleAppPOne) or (\ruleAppPTwo). These cases are very similar to corresponding cases when $t = x p$, such that $p \in \normal$, or $t = up$, such that $u \in \normal$ and $p \in \neutral$.
                \end{itemize}
            \end{itemize}
            \item Case $t \in \normal$. Then we can consider the two following additional cases:
            \begin{itemize}
                \item Case $t \in \val$. Then $\Phi$ must end with (\ruleAx), (\ruleLam), (\ruleMany), or (\ruleLamP). With the exception of the case where $\Phi$ ends with rule (\ruleMany), we can conclude $b = 0$ immediately for every other case, by definition. Case $\Phi$ ends with rule (\ruleMany), then $\tau$ is of the form $\M \not\in \tightt$. Therefore, this case does not apply.
                \item Case $t \not\in \val$. Then, $t \in \neutral$, by definition. Therefore, $b = 0$, by \cref{lem:zero-steps-nfs}.\ref{lem:zero-steps-nfs:1}.
            \end{itemize}
        \end{enumerate}
    \end{itemize}
\end{proof}}

\begin{lemma}
    \label{lem:corr-size-counter}
    Let $\Phi \tr \seqi{\Gam}{t}{\tau}{(b,s)}$ be tight. If $b = 0$ then $s = \size{t}$.
\end{lemma}

\maybehide{\begin{proof}
    The proof follows by induction over $\Phi$:
    \begin{itemize}
        \item Case $\Phi$ ends with rule (\ruleAx) or (\ruleLamP). Then $t \in \val$ and $s = 0$. So we can conclude with $\size{t} = 0 = s$.
        \item Case $\Phi$ ends with rule (\ruleLam). Then $\tau$ is of the form $\Gam_u(x) \ta \del \not\in \tightt$, so this case does not apply.
        \item Case $\Phi$ ends with rule (\ruleApp). Then $b > 0$, so this case does not apply.
        \item Case $\Phi$ ends with rule (\ruleMany). Then $\tau$ is of the form $\M \not\in \tightt$, so this case does not apply.
        \item Case $\Phi$ ends with rule (\ruleAppPOne). Then $t = up$ and $\Phi$ must be of the following form:
        \[ \begin{prooftree}
            \hypo{\Phi_u \tr \seqi{\Gam_u}{u}{\nott {\tabs}}{(0,s_u)}}
            \hypo{\Phi_p \tr \seqi{\Gam_p}{p}{\tightt}{(0,s_p)}}
            \infer2[(\ruleAppPOne)]{\seqi{\Gam_u + \Gam_p}{up}{\tneutral}{(0,1+s_u+s_p)}}
        \end{prooftree} \]
        where $\tau = \tneutral$, $\Gam = \Gam_u + \Gam_p$, and $s = 1 + s_u + s_p$. Moreover, $\Gam_u$ and $\Gam_p$ are tight. By the \ih over $\Phi_u$ and $\Phi_p$, we have $s_u = \size{u}$ and $s_p = \size{p}$. So we can conclude with $s = 1 + \size{u} + \size{p} = \size{up}$.
        \item Case $\Phi$ ends with rule (\ruleAppPTwo). This case is very similar to the case where $\Phi$ ends with rule (\ruleAppPOne).
    \end{itemize}
\end{proof}}

\begin{lemma}[{\bf Split for Values}]
    \label{lem:split-values}
    Let $\Phi_v \tr \seqi{\Gam}{v}{\M}{(b,s)}$, such that $\M = \sqcup_{\iI} \M_i$. Then, there exist ($\Phi^i_v \tr \seqi{\Gam_i}{v}{\M_i}{(b_i,s_i)})_{\iI}$, such that $\Gam = +_{\iI} \Gam_i$, $b = +_{\iI} b_i$, and $s = +_{\iI} s_i$.
\end{lemma}

\maybehide{\begin{proof}
    We start by noting that $\Phi_v$ must end with the rule ($\ruleMany$). Therefore, we have $\Gam = +_{\jJ} \Gam_j$, $\M = \mul{\sig_j}_{\jJ}$, $b = +_{\jJ} b_j$, $s = +_{\jJ} s_j$, and $(\Phi^j_v \tr \seqi{\Gam_j}{v}{\sig_j}{(b_j,s_j)})_{\jJ}$, for some $J$. Let $\M_i = \mul{\sig_k}_{\kK_i}$, for each $\iI$, such that $J = +_{\iI} K_i$. Then, by using rule ($\ruleMany$), we can build $\Phi^i_v \tr \seqi{\Gam_i}{v}{\M_i}{(b_i, s_i)}$, for each $\iI$, such that $\Gam_i = +_{\kK_i} \Gam_k$, $b_i = +_{\kK_i} b_k$, and $s_i = +_{\kK_i} s_k$. So we can conclude with $\Gam = +_{\jJ} \Gam_j = +_{\iI} (+_{\kK_i} \Gam_k) = +_{\iI} \Gam_i$, $b = +_{\jJ} b_j = +_{\iI} (+_{\kK_i} b_k) = +_{\iI} b_i$, and $s = +_{\jJ} s_j = +_{\iI} (+_{\kK_i} s_k) = +_{\iI} s_i$.
\end{proof}}

\subsubsection{Completeness (Auxiliary Lemmas)}

\begin{lemma}[{\bf Tight Spreading}]
    \label{lem:tight-spreading}
    Let $\Phi \tr \seqi{\Gam}{t}{\tau}{(b,s)}$, such that $\Gam$ is tight:
    \begin{enumerate}
        \item \label{lem:tight-spreading:1} If $b = 0$ and $\tau$ is not an arrow type or a multi-type, then $\tau \in \tightt$.
        \item \label{lem:tight-spreading:2} If $t \in \neutral$, then $\tau \in \tightt$.
    \end{enumerate}
\end{lemma}

\maybehide{\begin{proof} \mbox{}
    \begin{enumerate}
        \item We want to show that, if $b = 0$ and $\tau$ is not an arrow type or a multiset type, then $\tau \in \tightt$. The proof follows by induction over $\Phi$:
        \begin{itemize}
            \item Case $\Phi$ ends with rule ($\ruleAx$), then it is of the following form:
            \[ \begin{prooftree}
                \infer0[(\ruleAx)]{\seqi{x : \mul{\sig}}{x}{\sig}{(0,0)}}
            \end{prooftree} \]
            such that $\tau = \sig$, $\Gam = x : \mul{\sig}$, and $s = 0$. If $x : \mul{\sig}$ is tight, then $\sig \in \{\tabs, \tvar\}$. Therefore, we can conclude with $\sig \in \{\tabs, \tvar\} \subset \tightt$.
            \item Case $\Phi$ ends with rule (\ruleLam), then $\tau$ is an arrow type. Therefore, this case does not apply.
            \item Case $\Phi$ ends with rule (\ruleApp), then $b > 0$. Therefore, this case does not apply.
            \item Case $\Phi$ ends with rule (\ruleMany), then $\tau$ is a multiset type. Therefore, this case does not apply.
            \item Case $\Phi$ ends with rule (\ruleLamP), then $\tau = \tabs \in \tightt$. 
            \item Case $\Phi$ ends with rules (\ruleAppPOne) or (\ruleAppPTwo), then $\tau = \tneutral \in \tightt$.
        \end{itemize}
        \item We want to show that, if $t \in \neutral$, then $\tau \in \tightt$. By induction over $t \in \neutral$:
        \begin{itemize}
            \item Case $t = xp$, such that $p \in \normal$. Then we have to consider the following three cases depending on the last rule in $\Phi$:
            \begin{itemize}
                \item Case $\Phi$ ends with rule (\ruleApp), then it must be of the following form:
                \[ \begin{prooftree}
                    \hypo{\seqi{x : \mul{\M \ta \tau}}{x}{\M \ta \del}{(0,0)}}
                    \hypo{\Phi_p \tr \seqi{\Gam_p}{p}{\M}{(b_p,s_p)}}
                    \infer2[(\ruleApp)]{\seqi{(x : \mul{\M \ta \tau}) + \Gam_p}{xp}{\del}{(1+b_p,s_p)}}
                \end{prooftree} \]
                where $\Gam = (x : \mul{\M \ta \del}) + \Gam_p$ is tight, $b = 1+b_p$, and $s = s_p$. But, $\mul{\M \ta \del}$ is not tight, since $\M \ta \del \not\in \tightt$. Therefore, this case does apply.
                \item Case $\Phi$ ends with rule (\ruleAppPOne) or (\ruleAppPTwo). Then $\tau = \tneutral \in \tightt$, so we can conclude immediately.
            \end{itemize}
            \item Case $t = up$, such that $u \in \normal$ and $p \in \neutral$. Then we have to consider the following three cases depending on the last rule in $\Phi$:
            \begin{itemize}
                \item Case $\Phi$ ends with rule (\ruleApp), then it must be of the following form:
                \[ \begin{prooftree}
                    \hypo{\Phi_u \tr \seqi{\Gam_u}{u}{\M \ta \tau}{(b_u, s_u)}}
                    \hypo{\Phi_p \tr \seqi{\Gam_p}{p}{\M}{(b_p, s_p)}}
                    \infer2[(\ruleApp)]{\seqi{\Gam_u + \Gam_p}{up}{\tau}{(1+b_u+b_p, s_u+s_p)}}
                \end{prooftree} \]
                where $\Gam = \Gam_u + \Gam_p$ is tight, $b = 1 + b_u + b_p$, and $s = s_u + s_p$. Moreover, $\Gam_p$ is tight. By the \ih over $\Phi_p$, we have that $\M \in \tightt$, which is a contradiction. Therefore, this case does not apply.
                \item Case $\Phi$ ends with rule (\ruleAppPOne) or (\ruleAppPTwo). Then $\tau = \tneutral \in \tightt$, so we can conclude immediately.
            \end{itemize}
            \item Case $t = up$, such that $u \in \neutral$ and $p \in \normal$. Then we have to consider the following three cases depending on the last rule in $\Phi$:
            \begin{itemize}
                \item Case $\Phi$ ends with rule (\ruleApp), then it must be of the following form:
                \[ \begin{prooftree}
                    \hypo{\Phi_u \tr \seqi{\Gam_u}{u}{\M \ta \tau}{(b_u, s_u)}}n
                    \hypo{\Phi_p \tr \seqi{\Gam_p}{p}{\M}{(b_p, s_p)}}
                    \infer2[(\ruleApp)]{\seqi{\Gam_u + \Gam_p}{up}{\tau}{(1+b_u+b_p, s_u+s_p)}}
                \end{prooftree} \]
                where $\Gam = \Gam_u + \Gam_p$ is tight, $b = 1 + b_u + b_p$, and $s = s_u + s_p$. Moreover, $\Gam_p$ is tight. By the \ih over $\Phi_p$, we have that $\M \in \tightt$, which is a contradiction. Therefore, this case does not apply.
                \item Case $\Phi$ ends with rule (\ruleAppPOne) or (\ruleAppPTwo). Then $\tau = \tneutral \in \tightt$, so we can conclude immediately.
            \end{itemize}
        \end{itemize}
    \end{enumerate}
\end{proof}}

\begin{lemma}[{\bf Typability of Normal Forms}]
    \label{lem:typ-nfs}
    If $t \in \normal$, then there exists a tight derivation $\Phi \tr \seqi{\Gam}{t}{\tau}{(b,s)}$, such that $s = \size{t}$.
\end{lemma}

\maybehide{To show this proposition we are going to need to split the original statement into the two following ones:
\begin{enumerate}
    \item \label{prop:typ-nfs:1} If $t \in \neutral$, then there exists a tight derivation $\Phi \tr \seqi{\Gam}{t}{\tneutral}{(b,s)}$, such that $s = \size{t}$.
    \item \label{prop:typ-nfs:2} If $t \in \normal$, then there exists a tight derivation $\Phi \tr \seqi{\Gam}{t}{\tightt}{(b,s)}$, such that $s = \size{t}$.
\end{enumerate}
The proof follows by simultaneous induction over both these statements:
\begin{enumerate}
    \item Let $t \in \neutral$. We want to show that there exists a tight derivation $\Phi \tr \seqi{\Gam}{t}{\tneutral}{(b,s)}$:
    \begin{itemize}
        \item Case $t = up \in x \ \normal$. Then $u = x$ and $p \in \normal$. Therefore, there exists a tight derivation $\Phi_p \tr \seqi{\Gam_p}{p}{\tightt}{(b_p,s_p)}$, by the \ih (\cref{lem:typ-nfs}.\ref{prop:typ-nfs:2}), such that $\size{p} = s_p$. Thus, we can build $\Phi$ as follows:
        \[ \begin{prooftree}
            \infer0[(\ruleAx)]{\seqi{x : \mul{\tvar}}{x}{\tvar}{(0,0)}}
            \hypo{\Phi_p \tr \seqi{\Gam_p}{p}{\tightt}{(b_p,s_p)}}
            \infer2[(\ruleAppPOne)]{\seqi{ x : \mul{\tvar} + \Gam_p}{x p}{\tneutral}{(b_p,1+s_p)}}
        \end{prooftree} \]
        And we can conclude with $\Gam = x : \mul{\tvar} + \Gam_p$, $b = b_p$, and $s = 1+s_p = 1 + \size{x} + \size{p} = \size{xp}$.
        \item Case $t = up \in \normal \ \neutral$. Then $u \in \normal$ and $p \in \neutral$. Therefore, there exists a tight derivation $\Phi_u \tr \seqi{\Gam_u}{u}{\tightt}{(b_u,s_u)}$, such that $\size{u} = s_u$, by the \ih (\cref{lem:typ-nfs}.\ref{prop:typ-nfs:2}), and there exists a tight derivation $\Phi_p \tr \seqi{\Gam_p}{p}{\tneutral}{(b_p,s_p)}$, such that $\size{p} = s_p$ by the \ih (\cref{lem:typ-nfs}.\ref{prop:typ-nfs:1}). Thus, we can build $\Phi$ as follows:
        \[ \begin{prooftree}
            \hypo{\Phi_u \tr \seqi{\Gam_u}{u}{\tightt}{(b_u,s_u)}}
            \hypo{\Phi_p \tr \seqi{\Gam_p}{p}{\tneutral}{(b_p,s_p)}}
            \infer2[(\ruleAppPTwo)]{\seqi{\Gam_u + \Gam_p}{up}{\tneutral}{(b_u+b_p,1+s_u+s_p)}}
        \end{prooftree} \]
        And we can conclude with $\Gam = \Gam_u + \Gam_p$, $b = b_u+b_p$, and $s = 1+s_u+s_p = 1 + \size{u} + \size{p} = \size{up}$.
        \item Case $t = up \in \neutral \ \normal$. Then $u \in \neutral$ and $p \in \normal$. Therefore, there exists a tight derivation $\Phi_u \tr \seqi{\Gam_u}{u}{\tneutral}{(b_u,s_u)}$, such that $\size{u} = s_u$, by the \ih (\cref{lem:typ-nfs}.\ref{prop:typ-nfs:1}), and there exists a tight derivation $\Phi_p \tr \seqi{\Gam_p}{p}{\tightt}{(b_p,s_p)}$, such that $\size{p} = s_p$, by the \ih (\cref{lem:typ-nfs}.\ref{prop:typ-nfs:2}). Thus, we can build $\Phi$ as follows:
        \[ \begin{prooftree}
            \hypo{\Phi_u \tr \seqi{\Gam_u}{u}{\tneutral}{(b_u,s_u)}}
            \hypo{\Phi_p \tr \seqi{\Gam_p}{p}{\tightt}{(b_p,s_p)}}
            \infer2[(\ruleAppPOne)]{\seqi{\Gam_u + \Gam_p}{up}{\tneutral}{(b_u+b_p,1+s_u+s_p)}}
        \end{prooftree} \]
        And we can conclude with $\Gam = \Gam_u + \Gam_p$, $b = b_u+b_p$, and $s = 1 + s_u + s_p = 1 + \size{u} + \size{p} = \size{up}$.
    \end{itemize}
    \item Case $t \in \normal$. We want to show that there exists a tight derivation $\Phi \tr \seqi{\Gam}{t}{\tightt}{(b,s)}$:
    \begin{itemize}
        \item Case $t = x$. Then we can build $\Phi$ as follows:
        \[ \begin{prooftree}
            \infer0[(\ruleAx)]{\seqi{x : \mul{\sig}}{x}{\sig}{(0,0)}}
        \end{prooftree} \]
        by picking $\sig \in \{\tabs, \tvar\}$. And we can conclude with $\Gam = \eset$, $b = 0$, and $s = 0 = \size{x}$.
        \item Case $t = \lam x.u$. Then we can build $\Phi$ as follows:
        \[ \begin{prooftree}
            \infer0[(\ruleLamP)]{\seqi{}{\lam x.u}{\tabs}{(0,0)}}
        \end{prooftree} \]
        And we can conclude with $\Gam = \eset$, $b = 0$, and $s = 0 = \size{\lam x.u}$.
        \item The remaining cases are for when $t \in \neutral$, so they are subsumed by previous cases.
    \end{itemize}
\end{enumerate} 
}

\begin{lemma}[{\bf Merge for Values}]
    \label{lem:merge-values}
    Let $(\Phi^i_v \tr \seqi{\Gam_i}{v}{\M_i}{(b_i,s_i)})_{\iI}$. Then, there exists $\Phi_v \tr \seqi{\Gam}{v}{\M}{(b,s)}$, such that $\Gam = +_{\iI} \Gam_i$, $\M = +_{\iI} \M_i$, $b = +_{\iI} b_i$, and $s = +_{\iI}$.
\end{lemma}

\maybehide{\begin{proof}
    We start by noting that each $\Phi^i_v$ must end with the rule ($\ruleMany$). Therefore, for each $\iI$, we have $\Gam_i = +_{\kK_i} \Gam_k$, $\M_i = \mul{\sig_k}_{\kK_i}$, such that $b_i = +_{\kK_i} b_k$ and $s_i = +_{\kK_i} s_k$, and the following derivations $(\Phi^k_v \tr \seqi{\Gam_k}{v}{\sig_k}{(b_k,s_k)})_{\kK_i}$. Let $J = +_{\iI} K_i$ and $\M = \mul{\sig_j}_{\jJ} = \mul{\sig_k}_{\kK_i, \iI}$. We can use rule ($\ruleMany$) to build $\Phi_v \tr \seqi{\Gam}{v}{\M}{(+_{\jJ} b_j, +_{\jJ} s_j)}$. So we can conclude with $\Gam = +_{\jJ} \Gam_j = +_{\iI} (+_{\kK_i} \Gam_k) = +_{\iI} \Gam_i$, $b = +_{\jJ} b_j = +_{\iI} (+_{\kK_i} b_k) = +_{\iI} b_i$, and $s = +_{\jJ} s_j = +_{\iI} (+_{\kK_i} s_k) = +_{\iI} s_i$.
\end{proof}}

\subsubsection{Soundness and Completeness (Main Results)}

\begin{lemma}[{\bf Substitution and Anti-Substitution}]
    \label{lem:subsantisubs}
    \begin{enumerate} \mbox{}
        \item \label{lem:subs} Let $\Phi_t \tr \seqi{\Gam_t; x : \M}{t}{\tau}{(b_t,s_t)}$ and $\Phi_v \tr \seqi{\Gam_v}{v}{\M}{(b_v,s_v)}$, then there exists $\Phi_{t \subs{x}{v}} \tr \seqi{\Gam_t + \Gam_v}{t \subs{x}{v}}{\tau}{(b_t+b_v,s_t+s_v)}$.
        \item \label{lem:antisubs} Let $\Phi_{t \subs{x}{v}} \tr \seqi{\Gam_{t \subs{x}{v}}}{t \subs{x}{v}}{\tau}{(b,s)}$. Then, there exist $\Phi_t \tr \seqi{\Gam_t; x : \M}{t}{\tau}{(b_t,s_t)}$ and $\Phi_v \tr \seqi{\Gam_v}{v}{\M}{(b_v,s_v)}$, such that $\Gam_{t \subs{x}{v}} = \Gam_t + \Gam_v$, $b = b_t + b_v$, and $s = s_t + s_v$.
    \end{enumerate}
\end{lemma}

\maybehide{\begin{proof} \mbox{}
    \begin{enumerate}
        \item %\begin{proof}
    The proof follows by induction over $\Phi_t$:
    \begin{itemize}
        \item Case $\Phi_t$ ends with rule (\ruleAx). Then $t$ must be a variable and we need to consider two cases:
        \begin{itemize}
            \item Assume $t = y = x$. Then $\Gam_t = \eset$, $\tau = \M$, $t \subs{x}{v} = v$, $b_t = 0$, and $s_t = 0$. So we can take $\Phi_{t \subs{x}{v}} = \Phi_v$ and conclude with $\Gam_t + \Gam_v = \Gam_v$, $b_t + b_v = b_v$, and $s_t + s_v = s_v$.
            \item Assume $t = y \not= x$. Then $\M = \emul$, $\Gam_v = \eset$, $t \subs{x}{v} = t$, $b_v = 0$, and $s_v = 0$. So we can take $\Phi_{t \subs{x}{v}} = \Phi_t$ and conclude with $\Gam_t + \Gam_v = \Gam_t$, $b_t + b_v = b_t$, and $s_t + s_v = s_t$.
        \end{itemize}
        \item Case $\Phi_t$ ends with rule (\ruleLam). Then $t$ must be of the form $\lam y.u$ and $\Phi_t$ must be of the following form (by $\alpha$-conversion):
        \[ \begin{prooftree}
            \hypo{\Phi_u \tr \seqi{\Gam; x : \M}{u}{\tau'}{(b_t,s_t)}}
            \infer1[(\ruleLam)]{\seqi{(\Gam \sm y); x : \M}{\lam y.u}{\Gam(y) \ta \tau'}{(b_t, s_t)}}
        \end{prooftree} \]
        where $\tau = \Gam(y) \ta \tau'$ and $\Gam_t = (\Gam \sm y)$. By the \ih, we have the following derivation $\Phi_{u \subs{x}{v}} \tr \seqi{\Gam + \Gam_v}{u \subs{x}{v}}{\tau}{(b_t + b_v, s_t + s_v)}$. Therefore, we can construct $\Phi_{t \subs{x}{v}}$ as follows:
        \[ \begin{prooftree}
            \hypo{\Phi_{u \subs{x}{v}} \tr \seqi{\Gam + \Gam_v}{u \subs{x}{v}}{\tau'}{(b_t + b_v, s_t + s_v)}}
            \infer1[(\ruleLam)]{\seqi{(\Gam + \Gam_v) \sm y}{(\lam y.u) \subs{x}{v}}{\Gam(y) \ta \tau'}{(b_t + b_v, s_t + s_v)}}
        \end{prooftree} \]
        And we can conclude with $(\Gam + \Gam_v) \sm y = (\Gam \sm y) + \Gam_v = \Gam_t + \Gam_v$, by $\alpha$-conversion.
        \item Case $\Phi_t$ ends with rule ($\ruleApp$). Then $t$ must be of the form $up$ and $\Phi_t$ must be of the following form:
        \[ \begin{prooftree}
            \hypo{\Phi_u \tr \seqi{\Gam; x : \M_1}{u}{\M' \ta \tau}{(b_u, s_u)}}
            \hypo{\Phi_p \tr \seqi{\Del; x : \M_2}{p}{\M'}{(b_p,s_p)}}
            \infer2[(\ruleApp)]{\seqi{(\Gam + \Del); x : \M_1 \sqcup \M_2}{up}{\tau}{(1+b_u+b_p, s_u+s_p)}}
        \end{prooftree} \]
        where $\Gam_t = (\Gam + \Del)$, $\M = \M_1 \sqcup \M_2$, $b_t = 1 + b_u + b_p$, and $s_t = s_u + s_p$. By~\cref{lem:split-values}, we know there exist the following derivations $(\Phi^i_v \tr \seqi{\Gam^i_v}{v}{\M_i}{(b_i,s_i)})_{i \in \{1,2\}}$, such that $\Gam_v = \Gam^1_v + \Gam^2_v$, $b_v = b_1 + b_2$, and $s_v = s_1 + s_2$. By the \ih, we know there exist $\Phi_{u \subs{x}{v}} \tr \seqi{\Gam + \Gam^1_v}{u \subs{x}{v}}{\M' \ta \tau}{(b_u+b_1, s_u+s_1)}$ and $\Phi_{p \subs{x}{v}} \tr \seqi{\Del + \Gam^2_v}{p \subs{x}{v}}{\M'}{(b_p + b_2, s_p + s_2)}$. So we can construct $\Phi_{t \subs{x}{v}}$ as follows:
        \[ \begin{prooftree}
            \hypo{\Phi_{u \subs{x}{v}} \tr \seqi{\Gam + \Gam^1_v}{u \subs{x}{v}}{\M' \ta \tau}{(b_u+b_1, s_u+s_1)}}
            \hypo{\Phi_{p \subs{x}{v}} \tr \seqi{\Del + \Gam^2_v}{p \subs{x}{v}}{\M'}{(b_p+b_2,s_p+s_2)}}
            \infer2[(\ruleApp)]{\seqi{(\Gam + \Del) + (\Gam^1_v + \Gam^2_v)}{(u p) \subs{x}{v}}{\tau}{(1+b_u+b_p+b_1+b_2, s_u + s_p + s_1 + s_2)}}
        \end{prooftree} \]
        And we can conclude with $\Gam_t + \Gam_v = (\Gam + \Del) + (\Gam^1_v + \Gam^2_v)$, $b_t + b_v = 1 + b_u + b_p + b_1 + b_2$, and $s_t + s_v = s_u + s_p + s_1 + s_2$.
        \item Case $\Phi_t$ ends with rule ($\ruleMany$). Then $t$ must be of the form $w$ and $\Phi$ must be of the following form:
        \[ \begin{prooftree}
            \hypo{(\Phi^i_w \tr \seqi{\Gam_i; x : \M_i}{w}{\sig_i}{(b_i,s_i)})_{\iI}}
            \infer1[(\ruleMany)]{\seqi{+_{\iI} \Gam_i; x : \sqcup_{\iI} \M_i}{w}{\mul{\sig_i}_{\iI}}{(+_{\iI} b_i, +_{\iI} s_i)}}
        \end{prooftree} \]
        where $\tau = \mul{\sig_i}_{\iI}$, $\Gam_t = +_{\iI} \Gam_i$, $b_t = +_{\iI} b_i$, and $s_t = +_{\iI} s_i$. By~\cref{lem:split-values}, we have the following derivations $(\Phi^i_v \tr \seqi{\Gam^i_v}{v}{\M_i}{(b^i_v, s^i_v)})_{\iI}$, such that $\Gam_v = +_{\iI} \Gam^i_v$, $b_v = +_{\iI} b^i_v$, and $s_v = +_{\iI} s^i_v$. By the \ih over each $\Phi^i_w$, we have $(\Phi^i_{w \subs{x}{v}} \tr \seqi{\Gam_i + \Gam^i_v}{w \subs{x}{v}}{\sig_i}{(b_i + b^i_v, s_i + s^i_v)})_{\iI}$. Therefore, we can construct $\Phi_{t \subs{x}{v}}$ as follows:
        \[ \begin{prooftree}
            \hypo{(\Phi^i_{w \subs{x}{v}} \tr \seqi{\Gam_i + \Gam^i_v}{w \subs{x}{v}}{\sig_i}{(b_i + b^i_v, s_i + s^i_v)})_{\iI}}
            \infer1[(\ruleMany)]{\seqi{+_{\iI} (\Gam_i + \Gam^i_v)}{w \subs{x}{v}}{\mul{\sig_i}_{\iI}}{(+_{\iI} (b_i + b^i_v), +_{\iI} (s_i + s^i_v))}}
        \end{prooftree} \]
        And we can conclude with $\Gam_t + \Gam_v = +_{\iI} \Gam_i +_{\iI} \Gam^i_v = +_{\iI} (\Gam_i + \Gam^i_v)$, $b_t + b_v = +_{\iI} b_i +_{\iI} b^i_v = +_{\iI} (b_i + b^i_v)$, and $s_t + s_v = +_{\iI} s_i +_{\iI} s^i_v = +_{\iI} (s_i + s^i_v)$.
        \item Case $\Phi_t$ ends with rule (\ruleLamP). Then $t$ must be of the form $\lam y.u$, $\Gam_t = \eset$, $\tau = \tabs$, $\M = \emul$, $\Gam_v = \eset$, $t \subs{x}{v} = \lam y.(u \subs{x}{v}) = (\lam y.u) \subs{x}{v}$, $b_t = b_v = 0$, and $s_t = s_v = 0$. So we can construct $\Phi_{t \subs{x}{v}}$ as follows:
        \[ \begin{prooftree}
            \infer0[(\ruleLamP)]{\seqi{}{(\lam y.u) \subs{x}{v}}{\tabs}{(0,0)}}
        \end{prooftree} \]
        And conclude with $\Gam_t + \Gam_v = \eset$, $b_t + b_v = 0$, and $s_t + s_v = 0$.
        \item Case $\Phi_t$ ends with rule (\ruleAppPOne) or (\ruleAppPTwo), the proof is very similar to when $\Phi_t$ ends with rule (\ruleApp).
    \end{itemize}
%\end{proof}

        \item %\begin{proof}
    The proof follows by induction over $t$:
    \begin{itemize}
        \item Case $t = y$. Then we have to consider two cases:
        \begin{itemize}
            \item Case $t = y \not= x$. Then, $t \subs{x}{v} = y$. Let $\Gam_v = \eset$, $\M = \emul$, $b_v = 0$, and $s_v = 0$. Then, $\Phi_v$ is derivable using rule ($\ruleMany$). We also take $\Phi_t = \Phi_{t \subs{x}{v}}$, so that, in particular $\Gam_t = \Gam_{t \subs{x}{v}}$. Then, we conclude with $\Gam_{t \subs{x}{v}} = \Gam_t + \Gam_v = \Gam_t$, $b = b_t + b_v = b_t$, and $s = s_t + s_v = s_t$.
            \item Case $t = y = x$. Then, $t \subs{x}{v} = v$. Let $\Gam_t = \eset$, $b_t = 0$, and $s_t = 0$. Now, we have to consider two cases depending on the last rule used in $\Phi_{t \subs{x}{v}}$: 
            \begin{itemize}
                \item Case $\Phi_{t \subs{x}{v}}$ ends with rule ($\ruleAx$), then $\tau = \sig$. Let $\Gam_v = \Gam_{t \subs{x}{v}}$, $\M = \mul{\sig}$, $b_v = b$, and $s_v = s$. Then, we can build derivation $\Phi_v$ as follows:
                \[ \begin{prooftree}
                    \hypo{\Phi_{t \subs{x}{v}} \tr \seqi{\Gam_{t \subs{x}{v}}}{v}{\sig}{(b,s)}}
                    \infer1[(\ruleMany)]{\seqi{\Gam_{t \subs{x}{v}}}{v}{\mul{\sig}}{(b,s)}}
                \end{prooftree} \]
                Let $\Gam_t = \eset$, $b_t = 0$, and $s_t = 0$. Then, $\Phi_t \tr \seqi{x : \mul{\sig}}{x}{\sig}{(0,0)}$ is given by rule ($\ruleAx$). So we can conclude with $\Gam_{t \subs{x}{v}} = \Gam_v = \Gam_t + \Gam_v$, $b = b_v = b_t + b_v$, and $s = s_v = s_t + s_v$.
                \item Case $\Phi_{t \subs{x}{v}}$ ends with rule ($\ruleMany$), then $\tau = \mul{\sig_i}_{\iI}$, for some $I$. Let $\Gam_t = \eset$, and $\M = \mul{\sig_i}_{\iI}$. Then, we can build $\Phi_t$ as follows:
                \[ \begin{prooftree}
                    \infer0[(\ruleAx)]{(\seqi{x : \mul{\sig_i}}{x}{\sig_i}{(0,0)})_{\iI}}
                    \infer1[(\ruleMany)]{\seqi{x : \mul{\sig_i}_{\iI}}{x}{\mul{\sig_i}_{\iI}}{(0,0)}}
                \end{prooftree} \] 
                Then, we can take $\Phi_v = \Phi_{t \subs{x}{v}}$, so that $\Gam_v = \Gam_{t \subs{x}{v}}$, $b_v = b$, and $s_v = s$. And we can conclude $\Gam_{t \subs{x}{v}} = \Gam_v = \Gam_t + \Gam_v$, $b = b_v = b_t + b_v$, and $s = s_v = s_t + s_v$.
            \end{itemize}
        \end{itemize}
        \item Case $t = \lam y.u$. Then $t \subs{x}{v} = (\lam y.u) \subs{x}{v} = \lam y.(u \subs{x}{v})$ and we have to consider three cases:
        \begin{itemize}
            \item Case $\Phi_{t \subs{x}{v}}$ ends with rule (\ruleLam), then it must be of the following form:
            \[ \begin{prooftree}
                \hypo{\Phi_{u \subs{x}{v}} \tr \seqi{\Gam_{u \subs{x}{v}}; y : \M'}{u \subs{x}{v}}{\tau'}{(b, s)}}
                \infer1[(\ruleLam)]{\seqi{\Gam_{u \subs{x}{v}}}{\lam y.(u \subs{x}{v})}{\M' \ta \tau'}{(b, s)}}
            \end{prooftree} \]
            where $\tau = \M' \ta \tau'$, and $\Gam_{t \subs{x}{v}} = \Gam_{u \subs{x}{v}}$. By the \ih, we have the following derivations $\Phi_u \tr \seqi{\Gam_u; y: \M'; x : \M}{u}{\del}{(b_u, s_u)}$ and $\Phi_v \tr \seqi{\Gam_v}{v}{\M}{(b_v, s_v)}$, such that $\Gam_{u \subs{x}{v}} = \Gam_u + \Gam_v$, $b = b_u + b_v$, and $s = s_u + s_v$. And we can build $\Phi_{\lam y.u}$ as follows:
            \[ \begin{prooftree}
                \hypo{\Phi_u \tr \seqi{\Gam_u; y : \M'; x : \M}{u}{\tau'}{(b_u, s_u)}}
                \infer1[(\ruleLam)]{\seqi{\Gam_u; x : \M}{\lam y.u}{\M' \ta \tau'}{(b_u, s_u)}}
            \end{prooftree} \]
            So we can pick $\Phi_t = \Phi_{\lam y.u}$, and conclude with $\Gam_{t \subs{x}{v}} = \Gam_{u \subs{x}{v}} = \Gam_u + \Gam_v$, $b = b_u + b_v$, and $s = s_u + s_v$.
            \item Case $\Phi_{t \subs{x}{v}}$ ends with rule (\ruleLamP), then is must be of the following form:
            \[ \begin{prooftree}
                \infer0[(\ruleLamP)]{\seqi{}{\lam y.(u \subs{x}{v})}{\tabs}{(0,0)}}
            \end{prooftree} \]
            where $\tau = \tabs$, $\Gam_{t \subs{x}{v}} = \eset$, $b = 0$, and $s = 0$. Let $\Gam_t = \eset$, $\M = \emul$, $b_t = 0$, and $s_t = 0$. Then, we can build $\Phi_t$ as follows:
            \[ \begin{prooftree}
                \infer0[(\ruleLamP)]{\seqi{}{\lam y.u}{\tabs}{(0,0)}}
            \end{prooftree} \]
            Let $\Gam_v = \eset$, $b_v = 0$, and $s_v = 0$. Then $\Phi_v$ can be constructed by using rule (\ruleMany) with no premises. So we can conclude with $\Gam_{t \subs{x}{v}} = \eset = \Gam_t + \Gam_v$, and $b = 0 = b_t + b_v$, and $s = 0 = s_t + s_v$.
            \item Case $\Phi_{t \subs{x}{v}}$ ends with rule ($\ruleMany$). Then $t \subs{x}{v}$ and $t$ are values, and $\Phi_{t \subs{x}{v}}$ must be of the following form:
            \[ \begin{prooftree}
                \hypo{(\Phi_i \tr \seqi{\Gam_i}{t \subs{x}{v}}{\sig_i}{(b_i,s_i)})_{\iI}}
                \infer1[(\ruleMany)]{\seqi{+_{\iI} \Gam_i}{t \subs{x}{v}}{\mul{\sig_i}_{\iI}}{(+_{\iI} b_i, +_{\iI} s_i)}}
            \end{prooftree} \]
            where $\tau = \mul{\sig_i}_{\iI}$, $\Gam_{t \subs{x}{v}} = +_{\iI} \Gam_i$, $b = +_{\iI} b_i$, and $s = +_{\iI} s_i$. By the \ih over each $\Phi_i$, we have the following derivations $\Phi^i_t \tr \seqi{\Gam^i_t; x : \M_i}{t}{\sig_i}{(b^i_t, s^i_t)}$ and $\Phi^i_v \tr \seqi{\Gam^i_v}{v}{\M_i}{(b^i_v, s^i_v)}$, such that $\Gam_i = \Gam^i_t + \Gam^i_v$, $b_i = b^i_t + b^i_v$, and $s_i = s^i_t + s^i_v$,for each $\iI$. So we can build $\Phi_t$ as follows:
            \[ \begin{prooftree}
                \hypo{(\Phi^i_t \tr \seqi{\Gam^i_t; x : \M_i}{t}{\sig_i}{(b^i_t, s^i_t)})_{\iI}}
                \infer1[(\ruleMany)]{\seqi{+_{\iI} \Gam^i_t; x : \sqcup_{\iI} \M_i}{t}{\mul{\sig_i}_{\iI}}{(+_{\iI} b^i_t, +_{\iI} s^i_t)}}
            \end{prooftree} \]
            such that $\Gam_t = +_{\iI} \Gam^i_t$, $\M = \sqcup_{\iI} \M_i$, $b_t = +_{\iI} b^i_t$, and $s_t = +_{\iI} s^i_t$. By~\cref{lem:merge-values}, we can take the following derivation $\Phi_v \tr \seqi{+_{\iI} \Gam^i_v}{v}{\M}{(+_{\iI} b^i_v, +_{\iI} s^i_v)}$. And we can conclude with $\Gam_{t \subs{x}{v}} = +_{\iI} \Gam_i = +_{\iI} (\Gam^i_t + \Gam^i_v) = +_{\iI} \Gam^i_t +_{\iI} \Gam^i_v = \Gam_t + \Gam_v$, $b = +_{\iI} b_i = +_{\iI} (b^i_t + b^i_v) = +_{\iI} b^i_t +_{\iI} b^i_v = b_t + b_v$, and $s = +_{\iI} s_i = +_{\iI} (s^i_t + s^i_v) = +_{\iI} s^i_t +_{\iI} s^i_v = s_t + s_v$.
        \end{itemize}
        \item Case $t = up$. Then $t \subs{x}{v} = (u \subs{x}{v}) (p \subs{x}{v})$ and we have to consider three cases:
        \begin{itemize}
            \item Case $\Phi_{t \subs{x}{v}}$ ends with ($\ruleApp$), then it must be of the following form:
            \[ \begin{prooftree}
                \hypo{\Phi_{u \subs{x}{v}} \tr \seqi{\Gam_{u \subs{x}{v}}}{u \subs{x}{v}}{\M' \ta \tau}{(b', s')}}
                \hypo{\Phi_{p \subs{x}{v}} \tr \seqi{\Gam_{p \subs{x}{v}}}{p \subs{x}{v}}{\M'}{(b'', s'')}}
                \infer2[(\ruleApp)]{\seqi{\Gam_{u \subs{x}{v}} + \Gam_{p \subs{x}{v}}}{(u \subs{x}{v})(p \subs{x}{v})}{\tau}{(1+b'+b'', s'+s'')}}
            \end{prooftree} \]
            where $\Gam_{t \subs{x}{v}} = \Gam_{u \subs{x}{v}} + \Gam_{p \subs{x}{v}}$, $b = 1+b'+b''$, and $s = s' + s''$. By the \ih over $\Phi_{u \subs{x}{v}}$, we have the following derivations $\Phi_u \tr \seqi{\Gam_u; x : \M_1}{u}{\M' \ta \tau}{(b_u,s_u)}$ and $\Phi^1_v \tr \seqi{\Gam^1_v}{v}{\M_1}{(b^1_v,s^1_v)}$, such that $\Gam_{u \subs{x}{v}} = \Gam_u + \Gam^1_v$, $b' = b_u + b^1_v$, and $s' = s_u + s^1_v$. And by the \ih over $\Phi_{p \subs{x}{v}}$, we have the following derivation $\Phi_{p} \tr \seqi{\Gam_p; x : \M_2}{p}{\M'}{(b_p,s_p)}$ and $\Phi^2_v \tr \seqi{\Gam^2_v}{v}{\M_2}{(b^2_v,s^2_v)}$, such that $\Gam_{p \subs{x}{v}} = \Gam_p + \Gam^2_v$, $b'' = b_p + b^2_v$, and $s'' = s_p + s^2_v$. By~\cref{lem:merge-values}, we can take the following derivation $\Phi_v \tr \seqi{\Gam^1_v + \Gam^2_v}{v}{\M_1 \sqcup \M_2}{(b^1_v+b^2_v, s^1_v + s^2_v)}$, such that $\Gam_v = \Gam^1_v + \Gam^2_v$, $b_v = b^1_v + b^2_v$, and $s_v = s^1_v + s^2_v$. And we can build $\Phi_{up}$ as follows:
            \[ \begin{prooftree}
                \hypo{\Phi_u \tr \seqi{\Gam_u; x : \mul{\sig_i}_{\iI_1}}{u}{\M' \ta \tau}{(b_u,s_u)}}
                \hypo{\Phi_{p} \tr \seqi{\Gam_p; x : \mul{\sig_i}_{\iI_2}}{p}{\M'}{(b_p,s_p)}}
                \infer2[(\ruleApp)]{\seqi{(\Gam_u + \Gam_p); x : \mul{\sig_i}_{\iI}}{up}{\tau}{(1+b_u+b_p, s_u+s_p)}}
            \end{prooftree} \]
            such that $\Gam_t = \Gam_u + \Gam_p$, $b_t = 1+b_u + b_p$, and $s_t = s_u + s_p$. So we can pick $\Phi_t = \Phi_{up}$, and conclude with $\Gam_{t \subs{x}{v}} = \Gam_{u \subs{x}{v}} + \Gam_{p \subs{x}{v}} = \Gam_u + \Gam^1_v + \Gam_p + \Gam^2_v = (\Gam_u + \Gam_p) + (\Gam^1_v + \Gam^2_v) = \Gam_t + \Gam_v$, $b = 1+b'+b'' = 1+b_u + b^1_v + b_p + b^2_v = 1 + (b_u + b_p) + (b^1_v + b^2_v) = b_t + b_v$, and $s = s_u + s^1_v + s_p + s^2_v = (s_u + s_p) + (s^1_v + s^2_v) = s_t + s_v$.
            \item Case $\Phi_{t \subs{x}{v}}$ ends with (\ruleAppPOne) and (\ruleAppPTwo). These cases are very similar to the case where $\Phi_{t \subs{x}{v}}$ ends with rule (\ruleApp).
        \end{itemize}
    \end{itemize}
%\end{proof}
    \end{enumerate}
\end{proof}}

\begin{lemma}[{\bf Split Exact Subject Reduction and Expansion}]
    \label{lem:subjred-subjexp} \mbox{}
    \begin{enumerate} 
        \item \label{lem:subj-red} Let $\Phi_t \tr \seqi{\Gam}{t}{\tau}{(b,s)}$ be tight. If $t \dred t'$, then there exists $\Phi_{t'} \tr \seqi{\Gam}{t'}{\tau}{(b-1,s)}$.
        \item \label{lem:subj-exp} Let $\Phi_{t'} \tr \seqi{\Gam}{t'}{\tau}{(b,s)}$ be tight. If $t \dred t'$, then there exists $\Phi_t \tr \seqi{\Gam}{t}{\tau}{(b+1, s)}$.
    \end{enumerate}
\end{lemma}

\maybehide{\begin{proof} \mbox{}
    \begin{enumerate}
        \item %\begin{proof}
    We will actually prove the following stronger version of the statement, which allows us to reason inductively:

    Let $\Phi_t \tr \seqi{\Gam}{t}{\tau}{(b,s)}$, such that $\Gam$ is tight, and either $\tau$ is tight or $\neg\isvalue{t}$. If $t \dred t'$, then there exists $\Phi_{t'} \tr \seqi{\Gam}{t'}{\tau}{(b-1,s)}$.

    The proof now follows by induction over $\dred$:
    \begin{itemize}
        \item Case $t = (\lam x.u) v \dred u \subs{x}{v} = t'$.Assume that $\Phi_t$ ends with rule (\ruleAppPOne). Then $\lam x.u$ must be assigned type $\nott{\tabs}$, which is not possible by~\cref{lem:notabs-implies-negabs}. Now, assume that $\Phi_t$ ends with rule (\ruleAppPTwo). Then $v$ must be assigned typed $\tneutral$, which is not possible by~\cref{lem:values-not-neutral}. Therefore, $\Phi_t$ must be of the following form:
        \[ \begin{prooftree}
            \hypo{\Phi_u \tr \seqi{\Gam_u; x : \M}{u}{\tau}{(b_u,s_u)}}
            \infer1[(\ruleLam)]{\seqi{\Gam_u}{(\lam x.u)}{\M \ta \tau}{(b_u, s_u)}}
            \hypo{\Phi_{v} \tr \seqi{\Gam_v}{v}{\M}{(b_v,s_v)}}
            \infer2[(\ruleApp)]{\seqi{\Gam_u + \Gam_{v}}{(\lam x.u) v}{\tau}{(1+b_u+b_v, s_u+s_v)}}
        \end{prooftree} \]
        where $\tau \in \tightt$, $\Gam = \Gam_u + \Gam_v$ is tight, $b = 1 + b_u + b_v$, and $s = s_u + s_v$. By~\cref{lem:subsantisubs}.\ref{lem:subs}, we know there exists the following derivation $\Phi_{u \subs{x}{v}} \tr \seqi{\Gam_u + \Gam_v}{u \subs{x}{v}}{\tau}{(b_u+b_v,s_u+s_v)}$. So we can take $\Phi_{t'} = \Phi_{u \subs{x}{v}}$ and conclude with $b - 1 = b_u + b_v$.
        \item Case $t = up \dred u'p = t'$, such that $u \dred u'$. Then $\Phi_t$ must either end with (\ruleApp), (\ruleAppPOne), or (\ruleAppPTwo):
        \begin{itemize}
            \item Case $\Phi_t$ ends with rule (\ruleApp), then it must be of the following form:
            \[ \begin{prooftree}
                \hypo{\Phi_u \tr \seqi{\Gam_u}{u}{\M \ta \tau}{(b_u,s_u)}}
                \hypo{\Phi_p \tr \seqi{\Gam_p}{p}{\M}{(b_p,s_p)}}
                \infer2[(\ruleApp)]{\seqi{\Gam_u + \Gam_p}{up}{\tau}{(1 +b_u+b_p,s_u+s_p)}}
            \end{prooftree} \]
            where $\tau = \tau \in \tightt$, $\Gam = \Gam_u + \Gam_p$ is tight, $b = 1+b_u + b_p$, and $s = s_u + s_p$. Since $u \dred u'$, it is clear that $\neg\isvalue{u}$ holds. Moreover, $\Gam_u$ is necessarily tight. Therefore, by the \ih, there exists $\Phi_{u'} \tr \seqi{\Gam_u}{u'}{\M \ta \tau}{(b_u-1, s_u)}$. Thus, we can build $\Phi_{t'}$ as follows:
            \[ \begin{prooftree}
                \hypo{\Phi_{u'} \tr \seqi{\Gam_u}{u'}{\M \ta \tau}{(b_u-1, s_u)}}
                \hypo{\Phi_p \tr \seqi{\Gam_p}{p}{\M}{(b_p,s_p)}}
                \infer2[(\ruleApp)]{\seqi{\Gam_u + \Gam_p}{u'p}{\tau}{(b_u+b_p,s_u+s_p)}}
            \end{prooftree} \]
            And we can conclude with $b - 1= b_u + b_p$.
            \item Case $\Phi_t$ ends with rule (\ruleAppPOne) or (\ruleAppPTwo), the proof are similar to the one where $\Phi_t$ ends with rule (\ruleApp).
        \end{itemize}
        \item Case $t = up \dred up' = t'$, such that $u \not\dred$ and $p \dred p'$. Then $\Phi_t$ must either end with (\ruleApp), (\ruleAppPOne), or (\ruleAppPTwo):
        \begin{itemize}
            \item Case $\Phi_t$ ends with rule (\ruleApp), then it must be of the following form:
            \[ \begin{prooftree}
                \hypo{\Phi_u \tr \seqi{\Gam_u}{u}{\M \ta \tau}{(b_u,s_u)}}
                \hypo{\Phi_p \tr \seqi{\Gam_p}{p}{\M}{(b_p,s_p)}}
                \infer2[(\ruleApp)]{\seqi{\Gam_u + \Gam_p}{up}{\tau}{(1+b_u+b_p,s_u+s_p)}}
            \end{prooftree} \]
            where $\tau \in \tightt$, $\Gam = \Gam_u + \Gam_p$ is tight, $b = 1 + b_u + b_p$, and $s = s_u + s_p$. Since $p \dred p'$, it is clear that $\neg\isvalue{p}$. Moreover, $\Gam_p$ is necessarily tight. Therefore, by the \ih, we know there exists the following derivation $\Phi_{p'} \tr \seqi{\Gam_p}{p'}{\M}{(b_p-1, s_p)}$. Thus, we can build $\Phi_{t'}$ as follows:
            \[ \begin{prooftree}
                \hypo{\Phi_u \tr \seqi{\Gam_u}{u}{\M \ta \tau}{(b_u, s_u)}}
                \hypo{\Phi_{p'} \tr \seqi{\Gam_p}{p'}{\M}{(b_p-1,s_p)}}
                \infer2[(\ruleApp)]{\seqi{\Gam_u + \Gam_p}{up'}{\tau}{(b_u+b_p,s_u+s_p)}}
            \end{prooftree} \]
            And we can conclude with $b - 1 = b_u + b_p$.
            \item Case $\Phi_t$ ends with rule (\ruleAppPOne) or (\ruleAppPTwo), the proofs are similar to the ones where $\Phi_t$ ends with rule (\ruleApp).
        \end{itemize}
    \end{itemize}
%\end{proof}

        \item %\begin{proof}
    Just like for~\cref{lem:subjred-subjexp}.\ref{lem:subj-red}, we will actually prove the following stronger version of the statement, which allows us to reason inductively:

    Let $\Phi_{t'} \tr \seqi{\Gam}{t'}{\tau}{(b,s)}$, such that $\Gam$ is tight, and either ($\tau \in \tightt$ or $\neg\isvalue{t}$). If $t \dred t'$, then there exists $\Phi_t \tr \seqi{\Gam}{t}{\tau}{(b+1,s)}$.
    
    The proof now follows by induction over $\dred$:
    \begin{itemize}
        \item Case $t = (\lam x.u) v \dred u \subs{x}{v} = t'$. Then $\Phi_{t'} \tr \seqi{\Gam}{u \subs{x}{v}}{\tau}{(b,s)}$ and, by~\cref{lem:subsantisubs}.\ref{lem:antisubs}, there exist the following derivations $\Phi_u \tr \seqi{\Gam_u; x : \M}{u}{\tau}{(b_u, s_u)}$ and $\Phi_v \tr \seqi{\Gam_v}{v}{\M}{(b_v,s_v)}$, such that $\tau \in \tightt$, $\Gam = \Gam_u + \Gam_v$ is tight, $b = b_u + b_v$, and $s = s_u + s_v$. So we can build $\Phi_t$ as follows:
        \[ \begin{prooftree}
            \hypo{\Phi_u \tr \seqi{\Gam_u; x : \M}{u}{\tau}{(b_u, s_u)}}
            \infer1[(\ruleLam)]{\seqi{\Gam_u}{\lam x.u}{\M \ta \tau}{(b_u,s_u)}}
            \hypo{\Phi_v \tr \seqi{\Gam_v}{v}{\M}{(b_v,s_v)}}
            \infer2[(\ruleApp)]{\seqi{\Gam_u + \Gam_v}{(\lam x.u)v}{\tau}{(1+b_u+b_v, s_u+s_v)}}
        \end{prooftree} \]
        And we can conclude with $b + 1 = 1 + b_u + b_v$.
        \item Case $t = up \dred u'p = t'$, such that $u \dred u'$. Then $\Phi_{t'}$ must either end with (\ruleApp), (\ruleAppPOne), or (\ruleAppPTwo):
        \begin{itemize}
            \item Case $\Phi_{t'}$ ends with rule (\ruleApp), then it must be of the following form:
            \[ \begin{prooftree}
                \hypo{\Phi_{u'} \tr \seqi{\Gam_u}{u'}{\M' \ta \tau}{(b_u, s_u)}}
                \hypo{\Phi_p \tr \seqi{\Gam_p}{p}{\M'}{(b_p, s_p)}}
                \infer2[(\ruleApp)]{\seqi{\Gam_u + \Gam_p}{u'p}{\tau}{(1 + b_u + b_p, s_u + s_p)}}
            \end{prooftree} \]
            where $\tau \in \tightt$, $\Gam = \Gam_u + \Gam_p$ it tight, $b = 1 + b_u + b_p$, and $s = s_u + s_p$. Since $u \dred u'$, it is clear that $\neg\isvalue{u}$. Moreover, $\Gam_p$ is tight. Therefore, by the \ih, there exists the following derivation $\Phi_u \tr \seqi{\Gam_u}{u}{\M' \ta \tau}{(b_u + 1, s_u)}$. Thus, we can build $\Phi_{t'}$ as follows:
            \[ \begin{prooftree}
                \hypo{\Phi_u \tr \seqi{\Gam_u}{u}{\M' \ta \tau}{(b_u + 1, s_u)}}
                \hypo{\Phi_p \tr \seqi{\Gam_p}{p}{\M'}{(b_p, s_p)}}
                \infer2[(\ruleApp)]{\seqi{\Gam_u + \Gam_p}{up}{\tau}{(1 + b_u + 1 + b_p, s_u + s_p)}}
            \end{prooftree} \]
            And we can conclude with $b + 1 = (1 + b_u + b_p) + 1 = 1 + b_u + 1 + b_p$.
            \item Case $\Phi_{t'}$ ends with rule (\ruleAppPOne) or (\ruleAppPTwo), the proofs are similar to the one where $\Phi_{t'}$ ends with rule (\ruleApp).
        \end{itemize}
        \item Case $t = up \dred up' = t'$, such that $p \dred p'$. Then $\Phi_{t'}$ must either ends with (\ruleApp), (\ruleAppPOne), or (\ruleAppPTwo):
        \begin{itemize}
            \item Case $\Phi_{t'}$ ends with rule ($\ruleApp$), then it must be of the following form:
            \[ \begin{prooftree}
                \hypo{\Phi_u \tr \seqi{\Gam_u}{u}{\M' \ta \tau}{(b_u, s_u)}}
                \hypo{\Phi_{p'} \tr \seqi{\Gam_p}{p'}{\M'}{(b_p, s_p)}}
                \infer2[(\ruleApp)]{\seqi{\Gam_u + \Gam_p}{u p'}{\tau}{(1 + b_u + b_p, s_u + s_p)}}
            \end{prooftree} \]
            where $\tau \in \tightt$, $\Gam = \Gam_u + \Gam_{p'}$ is tight, $b = 1 + b_u + b_p$, $s_t = s_u + s_p$. Since $p \dred p'$, it is clear that $\neg\isvalue{p}$ holds. Moreover, $\Gam_p$ is tight. Therefore, by the \ih, we have the following derivation $\Phi_p \tr \seqi{\Gam_p}{p}{\M' \ta \tau}{(b_p + 1, s_p)}$. Thus, we can build $\Phi_{t'}$ as follows:
            \[ \begin{prooftree}
                \hypo{\Phi_u \tr \seqi{\Gam}{u}{\M' \ta \tau}{(b_u, s_u)}}
                \hypo{\Phi_p \tr \seqi{\Gam_p}{p}{\M'}{(b_p + 1, s_p)}}
                \infer2[(\ruleApp)]{\seqi{\Gam_u + \Gam_p}{up}{\tau}{(1 + b_u + b_p + 1, s_u + s_p)}}
            \end{prooftree} \]
            And we can conclude with $b + 1 = (1 + b_u + b_p) + 1 = 1 + b_u + b_p + 1$.
            \item Case $\Phi_{t'}$ ends with rule (\ruleAppPOne) or (\ruleAppPTwo), the proofs are similar to the one where $\Phi_{t'}$ ends with rule (\ruleApp).
        \end{itemize}   
    \end{itemize}
%\end{proof}
    \end{enumerate}
\end{proof}}

\begin{theorem}[{\bf Quantitative Soundness and Completeness}]
    \label{thm:soundnesscompleteness}
  \item \label{thm:soundness} If $\Phi \tr \seqi{\Gam}{t}{\tau}{(b,s)}$ is tight, then there exists $u \in \normal$ such that  $t \drred^b u$ with $\size{u} = s$.
  \item \label{thm:completeness} If $t \drred^b u$ with  $u \in \normal$, then there exists a tight type derivation $\Phi_t \tr \seqi{\Gam}{t}{\tau}{(b, \size{u})}$.
\end{theorem}

\maybehide{\begin{proof} \mbox{}
    \begin{enumerate} 
        \item %\begin{proof}
    The proof follows by induction over $b$:
    \begin{itemize}
        \item Case $b = 0$. Then $t \in \normal$, by~\cref{lem:zero-steps-nfs}. And $d = \size{t}$, by~\cref{lem:corr-size-counter}. So we can conclude with $u = t$.
        \item Case $b > 0$. Then $t \not\in \normal$, by~\cref{lem:zero-steps-nfs}. Therefore, there exists $t'$ such that $t \dred t'$, by~\cref{prop:char-nfs}. By\cref{lem:subjred-subjexp}.\ref{lem:subj-red}, there exists $\Phi_{t'} \tr \seqi{\Gam}{t'}{\tau}{(b-1, s)}$. By the \ih, there exists $u \in \normal$, such that $t' \drred^{b-1} u$, such that $d = \size{u}$. So we can conclude with $t \dred t' \drred^{b-1} u$, which means that $t \drred^b u$, as expected.
    \end{itemize}
%\end{proof}
        \item %\begin{proof}
    The proof follows by induction over $b$:
    \begin{itemize}
        \item Case $b = 0$. Then $t = u$, which means that $t \in \normal$. Therefore, we can conclude by~\cref{lem:typ-nfs}.
        \item Case $b > 0$. Then there exists $t'$, such that $t \dred t' \drred^{b-1} u$. By the \ih, there exists a tight derivation $\Phi_{t'} \tr \seqi{\Gam}{t'}{\tau}{(b-1, \size{u})}$. By\cref{lem:subjred-subjexp}.\ref{lem:subj-exp}, there exists a tight derivation $\Phi \tr \seqi{\Gam}{t}{\tau}{(b, \size{u})}$. So, we can conclude.
    \end{itemize}
%\end{proof}
    \end{enumerate}
\end{proof}}
  

  




\subsection{A \texorpdfstring{$\lambda$}{Lambda}-Calculus with Global State}

\subsubsection{General Lemmas}

\propnormalifffinal*

\maybehide{\begin{proof}
    \begin{itemize}
    \item[$\Ra$)] Let $(t, s)$ be \final. We consider two cases:
      \begin{itemize}
            \item Case $(t,s)$ is blocked. We reason by induction on blocked configurations. \begin{itemize}
                \item Case $(t,s) = (\get{l}{x}{u}, s)$, such that $l \not\in \dom{s}$. Then $(t, s) \not\ra$ is straightforward.
                \item Case $(t,s) = (v u, s)$ and $(u,s)$ is blocked.
                  Then by the \ih, we have that $(u,s) \not\ra$. Therefore, $(v u, s) \not\ra$ holds.
            \end{itemize}
          \item Case $t \in \normal$. We reason by induction on
            $\normal$. \begin{itemize}
                \item Case $t=v \in \val$. Then $(v,s) \not\ra$  is straightforward.
                \item Case $t \in \neutral$. Then $t = v u$ and we have to consider two different  cases: \begin{itemize}
                  \item Case  $v= x$ and $u \in \normal$. Then by the \ih, we have $(u,s) \not\ra$. Therefore, $(v u, s) \not\ra$ holds.
                    \item Case $v = (\lam x.p)$ and $u \in \neutral$. Then $u \in \normal$, and by the \ih, we have that $(u,s) \not\ra$. Therefore $(v u,s) \not\ra$ holds.
                \end{itemize}
            \end{itemize}
        \end{itemize}
      \item[$\La$)] Let $t \not \ra$. We reason by
        induction on $t$: \begin{itemize}
            \item Case $t = v$. Then $t \in \normal$. Therefore $(t,s)$ is \final.
            \item Case $t = v u$. Since $(v u, s) \not\ra$, then $(u,s) \not\ra$. By the \ih, we have $(u,s)$ \final. Now, we reason
              by cases: \begin{itemize}
                \item Case $(u, s)$ is blocked. Then, $(v u, s)$ is blocked by definition. 
                \item Case $u \in \normal$. Then we have two cases: \begin{itemize}
                    \item Case $u \in \neutral$. Then $vu \in \normal$. Therefore,  $(t,s)$ is \final.
                    \item Case $u \in \val$ and $v = \lam x.p$. Then $((\lam x.p) u, s) \ra (p \subs{x}{u}, s)$, which yields a contradiction with the hypothesis $t=vu\not\ra$. Thus, this case does not apply.
                \end{itemize}
            \end{itemize}
          \item Case $t = \get{l}{x}{u}$. Since $(\get{l}{x}{u},s) \not\ra$, then $l \not\in \dom{s}$. Therefore, $(\get{l}{x}{u},s)$ is blocked, which implies
$(t,s)$ is \final. 
            \item Case $t = \set{l}{v}{u}$. Then $(\set{l}{v}{u}, s) \ra (u, \upd{l}{v}{s})$, which yields to a contraction with the hypothesis  $t\not\ra$. 
              Therefore, this case does not apply.
        \end{itemize}
    \end{itemize}
\end{proof}
} 

\proptypedunblock*

\maybehide{\begin{proof}
    By induction on $t$: \begin{itemize}
        \item Case $t \in \val $ or $t = \set{l}{v}{t}$. Then the conclusion trivially holds, since clearly $(t,s)$ is not a blocked configuration.
        \item Case $t = \get{l}{x}{t}$. We have two  cases: \begin{itemize}
            \item Case $l \in \dom{s}$. Then $(t,s)$ is clearly unblocked.
            \item Case $l \not\in \dom{s}$. Let $\stype_0 = \conj{(l : \Gam(x))} \splus \stype$. Since $t = \get{l}{x}{u}$, then $\Phi$ must be of the following form:
            \[ \begin{prooftree}
                \hypo{\Phi \tr \seqi{\Gam_u \sm x}{\get{l}{x}{u}}{\comptype{
                    \stype_0}{\ctype}}{(b_u,m_u,d_u)}}
                \hypo{\Phi_s \tr \seqi{\Del}{s}{\stype_0}{(b_s,m_s,d_s)}}
                \infer2[(\ruleConf)]{\seqi{(\Gam_u \sm x) + \Del}{(\get{l}{x}{t}, s)}{\ctype}{(b_u+b_s,1+m_u+m_s,d_u+d_s)}}
              \end{prooftree} \] 
              where $\Gam = \Gam_u \sm x$, $b = b_u+b_s$, $m = 1+m_u+m_s$, and $d = d_u + d_s$. Thus, $l \in \dom{\conj{(l : \Gam_u(x))} \splus \stype}$, and so by
            \cref{lem:states-and-state-types} we have 
          $l \in \dom{s}$, which gives a contradiction with the hypothesis $l \not\in \dom{s}$. Therefore, this case does not apply,
        \end{itemize}
        \item Case $t = v u$. Assume $\Phi_v \tr \seqi{\Gam_v}{v}{\M \ta (\comptype{\stype'}{\ctype})}{(b_v,m_v,d_v)}$ and $\Phi_u \tr \seqi{\Gam_u}{u}{\tcomptype{\stype}{\M}{\stype'}}{(b_u,m_u,d_u)}$. Then $\Phi$ must be of the following form:
        \[ \begin{prooftree}
            \hypo{\Phi_v}
            \hypo{\Phi_u}
            \infer2[(\ruleApp)]{\seqi{\Gam_v + \Gam_u}{v u}{\comptype{\stype}{\ctype}}{(1+b_v+b_u,m_v+m_u,d_v+d_u)}}
            \hypo{\Phi_s \tr \seqi{\Del}{s}{\stype}{(b_s,m_s,d_s)}}
            \infer2[(\ruleConf)]{\seqi{(\Gam_v + \Gam_u) + \Del}{(v u, s)}{\ctype}{(1+b_v + b_u + b_s, m_v + m_u + m_s, d_v + d_u + d_s)}}
        \end{prooftree} \]
        where $\Gam = (\Gam_v + \Gam_u) + \Del$, $b = 1+b_v + b_u + b_s$, $m = m_v + m_u + m_s$, and $d = d_v + d_u + d_s$. Thus, we can build the following derivation for $(u,s)$:
        \[ \begin{prooftree}
            \hypo{\Phi_u \tr \seqi{\Gam_u}{u}{\tcomptype{\stype}{\M}{\stype'}}{(b_u,m_u,d_u)}}
            \hypo{\Phi_s \tr \seqi{\Del}{s}{\stype}{(b_s,m_s,d_s)}}
            \infer2[(\ruleConf)]{\seqi{\Gam + \Del}{(u,s)}{\conftype{\M}{\stype'}}{(b_u+b_s,m_u+m_s,d_u+d_s)}}
        \end{prooftree} \]
        By the \ih, we have that $(u,s)$ is unblocked. Therefore, $(v u, s)$ also unblocked.
    \end{itemize}
\end{proof}} 

\begin{lemma}[Relevance]
    Let $\Phi \tr \seqi{\Gam}{t}{\gtype}{(b,m,d)}$ (resp. $\Phi' \tr \seqi{\Gam}{s}{\stype}{(b',m',d')}$). Then $\dom{\Gam} \subseteq \fv{t}$ (resp. $\dom{\Gam} \subseteq \fv{s}$).
\end{lemma}

\maybehide{\begin{proof}
    The proof following by induction over $\Phi$ (resp. $\Phi'$). Case $\Phi$ (resp. $\Phi'$) ends with rule (\ruleAx), (\ruleAxP), or (\ruleLamP) (resp. rule (\ruleEmp)), then $\Phi$ (resp. $\Phi'$) is clearly relevant. The other cases follow easily from the \ih.
\end{proof}}

\subsubsection{Soundness Lemmas (Auxiliary Lemmas)}

\lemzerocounters*

\maybehide{\begin{proof} \mbox{}
    \begin{enumerate}
        \item \input{proofs/lem-zero-counters}
        \item \input{proofs/lem-zero-size-store}
    \end{enumerate}
\end{proof}} 

\begin{lemma}
    \label{lem:zero-counters-normal}
    Let $\Phi \tr \seqi{\Gam}{t}{\del}{(0,0,d)}$ be tight. If $t \in \normal$, then $\del = \stype \ra \tightt \tim \stype'$ and $\stype =\stype'$.
\end{lemma}

\maybehide{\begin{proof} 
  By induction on $t \in \normal$. We consider two cases:
  \begin{itemize}
    \item Case $t \in \val$. Then such a typing derivation can only end with rule (\ruleAx) followed by rule (\ruleLift) or (\ruleLamP)followed by rule (\ruleLift), in which cases the statement is obvious.
    \item Case $t = vu \in \neutral$. Since the first counter of the derivation is $0$, $\Phi$ can only end with a persistent rule (\ruleAppPOne) or (\ruleAppPTwo). In both cases, we can conclude by applying the \ih to $u \in \normal$ or $u \in \neutral$ and their type derivations, which gives  $\stype = \stype'$.
  \end{itemize}
\end{proof}} 

\lemzeronfs*

\maybehide{\begin{proof}\
  \begin{itemize}
    \item[$\Ra$)] By point (1) of~\cref{lem:zero-counters}.
    \item[$\La$)] By induction on $t$: \begin{itemize}
    \item Case $t \in \val$. There are six cases to consider for $\Phi$:
    \begin{itemize}    
      \item $\Phi$ ends with (\ruleAx). This case does not apply since the resulting type is not a monadic type. %Then $\Phi \tr \seqi{x:\mul{\rdel}}{x}{\rdel}{(0,0,0)}$ and the conclusion holds trivially.
      \item $\Phi$ ends with (\ruleLam). This case does not apply since the resulting type is not a monadic type.
      \item $\Phi$ ends with (\ruleMany). This case does not apply since the resulting type is not a monadic type.
      \item $\Phi$ ends with (\ruleLift). This case does not apply, since $\del = \tcomptype{\stype}{\M}{\stype'}$, but $\M \not\in \tightt$.
      \item $\Phi$ ends with (\ruleAxP). Then $\Phi \tr \seqi{x:\mul{\nott{\tneutral}}}{x}{\tcomptype{\stype}{\nott{\tneutral}}{\stype}}{(0,0,0)}$, with $\stype$ tight, and the conclusion holds trivially.
      \item $\Phi$ ends with (\ruleLamP). Then $\Phi \tr \seqi{}{\lambda x.t}{\tcomptype{\stype}{\vl}{\stype}}{(0,0,0)}$, with $\stype$ tight, and the conclusion holds trivially. 
    \end{itemize}
    \item Case $t = xu$. Then $u \in \normal$, by definition and there are two cases to consider for $\Phi$:
    \begin{itemize}
      \item If $\Phi$ ends with (\ruleApp). Then $\Phi_u \tr \seqi{\Gam_u}{u}{\tcomptype{\stype}{\M}{\stype'}}{(b_u,m_u,d_u)}$, $\Phi_x \tr \seqi{x : \M \ta (\comptype{\stype'}{\ctype})}{x}{\M \ta (\comptype{\stype'}{\ctype})}{(b_x,m_x,d_x)}$, such that $\Gam =  (x:\mul{\M \ta (\comptype{\stype'}{\ctype})}) + \Gam_u$ is tight. Absurd, since $\M \ta (\comptype{\stype'}{\ctype})$ is not tight, therefore this case does not apply.
      \item If $\Phi$ ends with (\ruleAppPOne). Then $\Phi_u \tr \seqi{\Gam_u}{u}{\tcomptype{\stype}{\tightt}{\stype}}{(b_u,m_u,d_u)}$, such that $\Gam = (x: \mul{\tvar})+\Gam_u$ is tight, $b = b_u$, $m =m_u$, $d = d_u+ 1$, and $\stype$ is tight. By the \ih\ on $u$, we have $b_u=m_u=0$, therefore $b = m = 0$.
      \end{itemize}
      \item Case $t = (\lam x.p) u$. Then $u \in \neutral$, by definition and there are two cases to consider for $\Phi$:
      \begin{itemize}
        \item If $\Phi$ ends with (\ruleApp). Then $\Phi_u \tr \seqi{\Gam_u}{u}{\tcomptype{\stype}{\M}{\stype'}}{(b_u,m_u,d_u)}$, $\Phi_{\lam x.p} \tr \seqi{\Gam_{\lam x.p}}{\lam x.p}{\M \ta (\comptype{\stype'}{\ctype})}{(b_p,m_p,d_p)}$, such that $\Gam = \Gam_u + \Gam_{\lambda x.p}$ is tight, $b = 1+b_l+b_u$, $m = m_l+m_u$, $d = d_l+ d_m$. Since $\Gam_u$ is tight and $u\in\neutral$, by~\cref{lem:comp-tight-spreading}, $\M \in \tightt$, which is absurd. Therefore, this case does not apply.
        \item If $\Phi$ ends with (\ruleAppPTwo). Then $\Phi_u \tr \seqi{\Gam_u}{u}{\tcomptype{\stype}{\tneutral}{\stype}}{(b_u,m_u,d_u)}$, such that $\Gam = \Gam_u$ is tight, $b = b_u$, $m=m_u$, $d = d_u+ 1$ and $\stype_f$ is tight. By the \ih\ on $u$, we have $b_u=m_u=0$. Therefore $b = m = 0$.
      \end{itemize}
    \end{itemize}
  \end{itemize}
\end{proof}
}

\begin{lemma}
    \label{lem:states-and-state-types}
    Let $\Phi \tr \seqi{\Del}{s}{\stype}{(b,m,d)}$. If $l \in \dom{\stype}$, then $l \in \dom{s}$.
\end{lemma}
  
\maybehide{\begin{proof}
    We proceed by proving the following stronger version of the statement: 
    
    Let $\Phi_s \tr \seqi{\Del_s}{s}{\stype_s}{(b_s,m_s,d_s)}$. If $l \in \dom{\stype_s}$, then $s \equivstate \upd{l}{v}{q}$, for some value $v$ and store $q$.
    
    The proof follows by induction on $\Phi_s$: 
    \begin{itemize}
        \item Case $\Phi_s$ ends with ($\ruleEmp$). Then the conclusion is vacuously true.
        \item Case $\Phi_s$ ends with ($\ruleUpd$). Then $\Phi_s$ is of the following form: 
        \[ \begin{prooftree}
            \hypo{\Phi_v \tr \seqi{\Gam_v}{v}{\M}{(b_v,m_v,d_v)}}
            \hypo{\Phi_q \tr \seqi{\Del_q}{q}{\stype_q}{(b_q,m_q,d_q)}}
            \infer2[(\ruleUpd)]{\seqi{\Gam_v + \Del_q}{\upd{l'}{v}{q}}{\conj{l' : \M}; \stype_q}{(b_v+b_q,m_v+m_q,d_v+v_q)}}
        \end{prooftree} \]
        where $\Del_s = \Gam_v + \Del_q$, $s = \upd{l'}{v}{q}$, $\stype_s = \conj{l' : \M}; \stype_q$, $b_s = b_v + b_q$, $m_s = m_v + m_q$, and $d_s = d_v + d_q$. Now we consider two  cases: 
        \begin{itemize}
            \item Case $l = l'$. Then we are done.
            \item Case $l \not= l'$. Since we are assuming that $l \in \dom{\stype_s}$, then it must be case that $l \in \dom{\stype_q}$. But, then by the \ih, we have $q \equivstate \upd{l}{w}{q'}$, for some value $w$ and store $q'$. Therefore, $s \equivstate \upd{l'}{v}{\upd{l}{w}{q'}} \equivstate \upd{l}{w}{\upd{l'}{v}{q'}}$.
        \end{itemize}
    \end{itemize}
    The correctness of the original statement now follows easily from the fact that, clearly, if $s \equivstate \upd{l}{v}{q}$, then $l \in \dom{s}$, by Definition~\ref{def:domainS}.
\end{proof}
} 

\begin{lemma}[{\bf Split Lemma}] \mbox{} 
    \label{lem:split-values-stores}
    \begin{enumerate}
        \item {(\bf Values)} \label{lem:com-split-values}  Let $\Phi_v \tr \seqi{\Gam}{v}{\M}{(b,m,d)}$, such that $\M = \sqcup_{\iI} \M_i$. Then, there exist ($\Phi^i_v \tr \seqi{\Gam_i}{v}{\M_i}{(b_i,m_i,d_i)})_{\iI}$, such that $\Gam = +_{\iI} \Gam_i$, $b = +_{\iI} b_i$, $m = +_{\iI} m_i$, and $d = +_{\iI} d_i$.
        \item {\bf (States)} \label{lem:split-state} Let $\Phi_s \tr \seqi{\Gam}{s}{\stype}{(b,m,d)}$, such that $l \in \dom{\stype}$. Then, $s \equivstate \upd{l}{v}{q}$, $\Phi_v \tr \seqi{\Gam_v}{v}{\stype(l)}{(b_v,m_v,d_v)}$ and $\Phi_q \tr \seqi{\Gam_q}{q}{\stype'}{(b_q,m_q,d_q)}$, such that $\Gam = \Gam_v + \Gam_q$, $\stype = \conj{(l : \stype(l))}; \stype'$, $b = b_v+b_q$, $m = m_v+m_q$, and $d = d_v + d_q$.
    \end{enumerate}
\end{lemma}

\maybehide{\begin{proof}
    The proof for values is very similar to the corresponding proof for $\lam_s$, so we are only going to show the split lemma for states.
    The proof follows by induction on the structure of $s$: \begin{itemize}
        \item Case $s = \estate$. Then the statement is vacuously true.
        \item Case $s = \upd{l'}{w}{q'}$. Then $\Phi_s$ is of the form: 
        \[ \begin{prooftree}
            \hypo{\Phi_{w} \tr \seqi{\Gam_{w}}{w}{\M}{(b_w,m_w,d_w)}}
            \hypo{\Phi_{q'} \tr \seqi{\Gam_{q'}}{q'}{\stype_{q'}}{(b_{q'}, m_{q'},d_{q'})}}
            \infer2[(\ruleUpd)]{\seqi{\Gam_{w} + \Gam_{q'}}{\upd{l'}{w}{q'}}{\conj{(l' : \M)}; \stype_{q'}}{(b_w+b_{q'},m_w+m_{q'},d_w+d_{q'})}}
        \end{prooftree} \] where $\Gam = \Gam_{w} + \Gam_{q'}$, $\stype = \conj{(l' : \M)}; \stype_{q'}$, $b = b_w + b_{q'}$, $m = m_w + m_{q'}$, and $d = d_w + d_{q'}$. 
        We  consider two cases: \begin{itemize}
            \item Case $l' = l$. Then we simply take $v = w$ and $q = q'$ and we are done.
            \item Case $l' \not= l$.  Since $l \in \dom{\conj{(l' : \M)}; \stype_{q'}}$ and $l' \not= l$, then $l \in \dom{\stype_{q'}}$. By applying the \ih to $q'$, we have that  $q' \equivstate \upd{l}{w'}{q''}$, $\Phi_{w'} \tr \seqi{\Gam_{w'}}{w'}{\stype_{q'}(l)}{(b_{w'},m_{w'},d_{w'})}$ and $\Phi_{q''} \tr \seqi{\Gam_{q''}}{q''}{\stype_{q''}}{(b_{q''},m_{q''},d_{q''})}$, such that $\Gam_{q'} = \Gam_{w'} + \Gam_{q''}$, $\stype_{q'} = \conj{(l : \stype_{q'}(l))}; \stype_{q''}$, $b_{q'} = b_{w'} + b_{q''}$, $m_{q'} = m_{w'} + m_{q''}$, and $d_{q'} = d_{w'} + d_{q''}$. But $s = \upd{l'}{w}{\upd{l}{w'}{q''}} \equivstate \upd{l}{w'}{\upd{l'}{w}{q''}}$, so we can take $v = w'$, $q = \upd{l'}{w}{q''}$, and consider $\Phi_q$ to be the following derivation:
            \[ \begin{prooftree}
                \hypo{\Phi_{w} \tr \seqi{\Gam_{w}}{w}{\M}{(b_w,m_w,d_w)}}
                \hypo{\Phi_{q''} \tr \seqi{\Gam_{q''}}{q''}{\stype_{q''}}{(b_{q''}, m_{q''}, d_{q''})}}
                \infer2[(\ruleUpd)]{\seqi{\Gam_{w} + \Gam_{q''}}{\upd{l'}{w}{q''}}{\conj{(l' : \M)}; \stype_{q''}}{(b_w+b_{q''}, m_w + m_{q''}, d_w+d_{q''})}}
            \end{prooftree} \] where  $\Gam_q = \Gam_{w} + \Gam_{q''}$ and $\stype_q=\conj{(l' : \M)}; \stype_{q''}$. We can then conclude with the following observations:
            \begin{itemize}
            \item $\Gam_v + \Gam_q = \Gam_{w'} +\Gam_{w} + \Gam_{q''} =
              \Gam_{w} + \Gam_{q'} = \Gam$,
                \item Since $\stype = \conj{(l' : \M)}; \stype_{q'}$ and $l' \not= l$, then $\stype(l) = \stype_{q'}(l)$ and
                \begin{align*}
                    \stype = \conj{(l' : \M)}; \stype_{q'} & = \conj{(l': \M)}; \conj{(l : \stype_{q'}(l))}; \stype_{q''} \\
                    & = \conj{(l : \stype_{q'}(l))}; \stype_{q} \\
                    & = \conj{(l : \stype(l))}; \stype_q
                \end{align*}
              \item $b_v + b_q= b_{w'} + b_{w} + b_{q''}=
                 b_w + b_{q'} = b$, $m_v + m_q= m_{w'} + m_{w} + m_{q''}=
                 m_w + m_{q'} = b$ and
                 $d_v + d_q= d_{w'} + d_{w} + d_{q''}=
                 d_w + d_{q'} = d$.
            \end{itemize} 
        \end{itemize}
    \end{itemize}
\end{proof}
} 

\begin{lemma}
    \label{lem:comp-values-not-neutral}
    Let $\Phi \tr \seqi{\Gam}{t}{\tcomptype{\stype}{\tau}{\stype'}}{(b,m,d)}$. If $t \in \val$, then $\tau \neq \tneutral$.
\end{lemma}

\maybehide{\begin{proof}
    By case analysis on the form of $t \in \val$:
    \begin{itemize}
        \item Case $t = x$. Then we have to consider three cases according to the last rule used in $\Phi$:
        \begin{itemize}
            \item Case $\Phi$ ends with rule (\ruleAx), then $t$ can only be assigned $\sig$. Therefore, this case does not apply.
            \item Case $\Phi$ ends with rule (\ruleMany), then $\tau = \M \neq \tneutral$.
        
            \item Case $\Phi$ ends with rule (\ruleLift). Then $\tau \in \{\tvar, \tabs, \M\}$, which means that $\tau \not= \tneutral$.
        \end{itemize}
        \item Case $t = \lam x.t$. Then we have to consider three cases according to the last rule used in $\Phi$:
        \begin{itemize}
            \item Case $\Phi$ ends with rule (\ruleLam), then $t$ can only be assigned $\sig$. Therefore, this case does not apply.
            \item Case $\Phi$ ends with rule (\ruleMany), then $\tau = \M  \neq \tneutral$.
            \item Case $\Phi$ ends with rule (\ruleLamP), then $\tau = \vl$. Therefore, this case does not apply.
            \item Case $\Phi$ ends with rule (\ruleLift). $\tau \in \{\tabs, \M\}$, which means that $\tau \not= \tneutral$.
        \end{itemize}
    \end{itemize}
\end{proof}} 

\begin{lemma}
    \label{lem:comp-notabs-implies-negabs}
    Let $\Phi \tr \seqi{\Gam}{t}{\tcomptype{\stype}{\tau}{\stype'}}{(b,m,d)}$, such that $\Gam$ is tight. If $\tau \in \nott{\vl}$, then $\neg\isabs{t}$.
\end{lemma}

\maybehide{\begin{proof}
    By induction over $\Phi$:
    \begin{itemize}
        \item Case $\Phi$ ends with rule (\ruleAx), (\ruleApp), (\ruleGet), (\ruleSet), (\ruleAxP) (\ruleAppPOne), or (\ruleAppPTwo), then $\neg\isabs{t}$ holds by definition.
        \item Case $\Phi$ ends with rule (\ruleLam), (\ruleMany), or (\ruleLamP), then $\tau \in \nott{\tabs}$ does not hold. Therefore, these cases do not apply.
    \end{itemize}
\end{proof}} 

\subsubsection{Completeness (Auxiliary Lemmas)}

\begin{lemma}[{\bf Merge for Values}]
    \label{lem:comp-merge-values}
    Let $(\Phi^i_v \tr \seqi{\Gam_i}{v}{\M_i}{(b_i,m_i,d_i)})_{\iI}$. Then, there exists $\Phi_v \tr \seqi{\Gam}{v}{\M}{(b,m,d)}$, such that $\Gam = +_{\iI} \Gam_i$, $\M = +_{\iI} \M_i$, $b = +_{\iI} b_i$, $m = +_{\iI} m_i$, and $d = +_{\iI}$.
\end{lemma}
We omit this proof given its similarity with the proof for system $\syscbv$.

\lemcomtightspreading*

\maybehide{\begin{proof}
  We want to show that, if $t \in \neutral$, then $\tau \in \tightt$, for some $\stype'$. We proceed by induction on the predicate  $t \in \neutral$:
    \begin{itemize}
        \item Case $t = xu$, such that $u \in \normal$. Then we have to consider the following two cases depending on the last rule in $\Phi$:
        \begin{itemize}
            \item Case $\Phi$ ends with rule ($\ruleApp$), then it must be of the following form:
            \[ \begin{prooftree}
                \infer0[(\ruleAx)]{\seqi{x : \mul{\M \ta (\comptype{\stype'}{\ctype})}}{x}{\M \ta (\comptype{\stype'}{\ctype})}{(0,0,0)}}
                \hypo{\Phi_u \tr \seqi{\Gam_u}{u}{\tcomptype{\stype}{\M}{\stype'}}{(b_u,m_u,d_u)}}
                \infer2[(\ruleApp)]{\seqi{(x : \mul{\M \ta (\comptype{\stype'}{\ctype})}) + \Gam_u}{xu}{\comptype{\stype}{\ctype}}{(1+b_u,m_u,d_u)}}
            \end{prooftree} \]
            where $\Gam = (x : \mul{\M \ta (\comptype{\stype'}{\ctype})}) + \Gam_p$ is tight, $b = 1+b_u$, $m = m_u$, and $d = d_u$. But $\M \ta (\comptype{\stype'}{\ctype}) \not\in \tightt$, therefore $\Gam$ is not tight and we have a contraction. Thus, this case does not apply.
            \item Case $\Phi$ ends with rule (\ruleAppPOne), then $\tau = \tneutral \in \tightt$, so we can conclude immediately.
        \end{itemize}
        \item Case $t = (\lambda x.p)u$, such that $u \in \neutral$. Then we have to consider the following two cases depending on the last rule in $\Phi$:
        \begin{itemize}
            \item Case $\Phi$ ends with rule ($\ruleApp$), then it must be of the following form:
            \[ \begin{prooftree}
                \hypo{\seqi{\Gam_p}{\lam x.p}{\M \ta (\comptype{\stype'}{\ctype})}{(b_p,m_p,d_p)}}
                \hypo{\Phi_u \tr \seqi{\Gam_u}{u}{\tcomptype{\stype}{\M}{\stype'}}{(b_u,m_u,d_u)}}
                \infer2[(\ruleApp)]{\seqi{\Gam_p + \Gam_u}{(\lam x.p)u}{\comptype{\stype}{\ctype}}{(1+b_p+b_u,m_p+m_u,d_p+d_u)}}
            \end{prooftree} \]
            where $\Gam = \Gam_u + \Gam_p$ is tight, $b = 1 + b_p + b_u$, $m = m_p + m_u$, and $d = d_p + d_u$. By the \ih on $u$, we have that $\M \in \tightt$, which is a contradiction. Therefore, this case does not apply.
            \item Case $\Phi$ ends with rule (\ruleAppPTwo). Then $\tau = \tneutral \in \tightt$, so we can conclude immediately.
        \end{itemize}
    \end{itemize}
\end{proof}
} 

\typstates*

\maybehide{\begin{proof} \mbox{}
    \begin{enumerate}
        \item \input{proofs/lem-typ-states}
        \item \input{proofs/lem-comp-typ-nfs}
    \end{enumerate}
\end{proof}


} 

\subsubsection{Soundness and Completeness (Main Lemmas)}

\lemcompsubsantisubs*

\maybehide{\begin{proof} \mbox{}
    \begin{enumerate}
        \item %\begin{proof}
    We are going to generalize the original statement by replacing $\del$ with $\gtype$.
    \\ \\
    The proof now follows by induction over the structure of $\Phi_t$:
        \begin{itemize}
            \item Case $\Phi_t$ ends with rule ($\ruleAx$). Then $t$ must be a variable and we must consider two cases:
            \begin{itemize}
                \item Assume $t = y = x$. Then $\Gam_t = \eset$, $\gtype = \M$, $t \subs{x}{v} = v$, $b_t = m_t = d_t = 0$. So we can take $\Phi_{t \subs{x}{v}} = \Phi_v$ and conclude with $\Gam_t + \Gam_v = \Gam_v$, $b_t + b_v = b_v$, $m_t + m_v = m_v$, and $d_t + d_v = d_v$.
                \item Assume $t = y \not= x$. Then $\M = \emul$, $\Gam_v = \eset$, $t \subs{x}{v} = t$, $b_v = 0$, $m_v = 0$, and $d_v = 0$. So we can take $\Phi_{t \subs{x}{v}} = \Phi_t$ and conclude with $\Gam_t + \Gam_v = \Gam_t$, $b_t + b_v = b_t$, $m_t + m_v = m_t$, and $d_t + d_v = d_t$.
            \end{itemize}
            \item Case $\Phi_t$ ends with (\ruleLam). Then $t$ must be of the form $\lam y.u$ and $\Phi_t$ must be of the following form (by $\alpha$-conversion):
            \[ \begin{prooftree}
                \hypo{\Phi_u \tr \seqi{\Gam; x : \M}{u}{\comptype{\stype}{\ctype}}{(b_t,m_t,d_t)}}
                \infer1[(\ruleLam)]{\seqi{(\Gam \sm y); x : \M}{\lam y.u}{\Gam(y) \ta (\comptype{\stype}{\ctype})}{(b_t,m_t,d_t)}}
            \end{prooftree} \]
            where $\Gam_t = (\Gam \sm y)$, and $\gtype = \Gam(y) \ta (\comptype{\stype}{\ctype})$. By the \ih, we have the following derivation $\Phi_{u \subs{x}{v}} \tr \seqi{\Gam + \Gam_v}{u \subs{x}{v}}{\comptype{\stype}{\ctype}}{(b_t+b_v,m_t+m_v,d_t+d_v)}$. Therefore, we can build $\Phi_{t \subs{x}{v}}$ as follows:
            \[ \begin{prooftree}
                \hypo{\Phi_{u \subs{x}{v}} \tr \seqi{\Gam + \Gam_v}{u \subs{x}{v}}{\comptype{\stype}{\ctype}}{(b_t+b_v,m_t+m_v,d_t+d_v)}}
                \infer1[(\ruleLam)]{\seqi{(\Gam + \Gam_v) \sm y}{\lambda y.u \subs{x}{v}}{\Gam(y) \ta (\comptype{\stype}{\ctype})}{(b_t+b_v,m_t+m_v,d_t+d_v)}}
            \end{prooftree} \]
            And we conclude with $(\Gam + \Gam_v) \sm y = (\Gam \sm y) + \Gam_v = \Gam_t + \Gam_v$, by $\alpha$-conversion.
            \item Case $\Phi_t$ ends with ($\ruleApp$). Then $t$ must be of the form $wu$ and $\Phi_t$ must be of following form:
            \[ \begin{prooftree}
                \hypo{\Phi_w \tr \seqi{\Gam; x : \M_1}{w}{\M' \ta (\comptype{\stype'}{\ctype})}{(b_w,m_w,d_w)}}
                \hypo{\Phi_u \tr \seqi{\Del; x : \M_2}{u}{\tcomptype{\stype}{\M'}{\stype'}}{(b_u,m_u,d_u)}}
                \infer2[(\ruleApp)]{\seqi{\Gam + \Del; x : \M_1 \sqcup \M_2}{wu}{\comptype{\stype}{\ctype}}{(1+b_w+b_u,m_w+m_u,d_w+d_u)}}
            \end{prooftree} \]
            such that $\Gam_t = \Gam + \Del$, $\M = \M_1 \sqcup \M_2$, $\gtype = \comptype{\stype}{\ctype}$, $b_t = 1+b_w+b_u$, $m_t = m_w+m_u$, and $d_t = d_w + d_u$. By~\cref{lem:split-values-stores}.\ref{lem:com-split-values}, we know there exist the following derivations $(\Phi^i_v \tr \seqi{\Gam^i_v}{v}{\M_i}{(b_i,m_i,d_i)})_{i \in \{1,2\}}$, such that $\Gam_v = \Gam^1_v + \Gam^2_v$, $b_v = b_1 + b_2$, $m_v = m_1 + m_2$, and $d_v = d_1 + d_2$. By the \ih, we know there exist $\Phi_{w \subs{x}{v}} \tr \seqi{\Gam + \Gam^1_v}{w \subs{x}{v}}{\M' \ta (\comptype{\stype'}{\ctype})}{(b_w+b_1,m_w+m_1,d_w+d_1)}$ and $\Phi_{u \subs{x}{v}} \tr \seqi{\Del + \Gam^2_v}{u \subs{x}{v}}{\tcomptype{\stype}{\M'}{\stype'}}{(b_u+b_2,m_u+m_2,d_u+d_2)}$. %Assume $\Phi_{w \subs{x}{v}} \tr \seqi{\Gam + \Gam^1_v}{w \subs{x}{v}}{\tcomptype{\M'}{\stype'}{\ctype}}{(b_w+b_1,m_w+m_1,d_w+d_1)}$ and $\Phi_{u \subs{x}{v}} \tr \seqi{\Del + \Gam^2_v}{u \subs{x}{v}}{\stype \ta (\comptype{\M'}{\stype'})}{(b_u+b_2,m_u+m_2,d_u+d_2)}$. 
            We can build $\Phi_{t \subs{x}{v}}$ as follows:
            \[ \begin{prooftree}
                \hypo{\Phi_{w \subs{x}{v}}}
                \hypo{\Phi_{u \subs{x}{v}}}
                \infer2[(\ruleApp)]{\seqi{(\Gam + \Del) + (\Gam^1_v + \Gam^2_v)}{(wu) \subs{x}{v}}{\comptype{\stype}{\ctype}}{(1+b_w+b_u+b_1+b_2,m_w+m_u+m_1+m_2,d_w+d_u+d_1+d_2)}}
            \end{prooftree} \]
            And we can conclude with $\Gam_t + \Gam_v = (\Gam + \Del) + (\Gam^1_v + \Gam^2_v)$, $b_t + b_v = 1 + b_w+b_u+b_1+b_2$, $m_t + m_v = m_w+m_u+m_1+m_2$, and $d_t + d_v = d_w+d_u+d_1+d_2$.
            \item Case $\Phi_w$ ends with ($\ruleMany$). Then $t$ must be of the form $w$ and $\Phi_t$ must be of the following form:
            \[ \begin{prooftree}
                \hypo{(\Phi^i_w \tr \seqi{\Gam_i; x : \M_i}{w}{\rdel_i}{(b_i,m_i,d_i)})_{\iI}}
                \infer1[(\ruleMany)]{\seqi{+_{\iI} \Gam_i; x : \sqcup_{\iI} \M_i}{w}{\mul{\rdel_i}_{\iI}}{(+_{\iI}b_i, +_{\iI}m_i, +_{\iI}d_i)}}
            \end{prooftree} \]
            such that $\Gam_t = +_{\iI} \Gam_i$, $\gtype = \mul{\rdel_i}_{\iI}$, $b_t = +_{\iI} b_i$, $m_t = +_{\iI} m_i$, and $d_t = +_{\iI} d_i$. By~\cref{lem:split-values-stores}.\ref{lem:com-split-values}, $(\Phi^i_v \tr \seqi{\Gam^i_v}{v}{\M_i}{(b^i_v,m^i_v,d^i_v)})_{\iI}$, such that $\Gam_v = +_{\iI} \Gam^i_v$, $b_v = +_{\iI} b^i_v$, $m_v = +_{\iI} m^i_v$, and $d_v = +_{\iI} d^i_v$. By the \ih over each $\Phi^i_v$, we have $(\Phi^i_{w \subs{x}{v}} \tr \seqi{\Gam_i + \Gam^i_v}{w \subs{x}{v}}{\rdel_i}{(b_i+b^i_v,m_i+m^i_v,d_i+d^i_v)})_{\iI}$. Therefore, we can build $\Phi_{t \subs{x}{v}}$ as follows:
            \[ \begin{prooftree}
                \hypo{(\Phi^i_{w \subs{x}{v}} \tr \seqi{\Gam_i + \Gam^i_v}{w \subs{x}{v}}{\rdel_i}{(b_i+b^i_v,m_i+m^i_v,d_i+d^i_v)})_{\iI}}
                \infer1[(\ruleMany)]{\seqi{+_{\iI} (\Gam^i_v + \Gam^i_w)}{w \subs{x}{v}}{\mul{\tau_i}_{\iI}}{(+_{\iI}(b_i+b^i_v),+_{\iI}(m_i+m^i_v),+_{\iI}(d_i+d^i_v))}}
            \end{prooftree} \]
            And we can conclude with $\Gam_t + \Gam_v = +_{\iI} \Gam_i +_{\iI} \Gam^i_v = +_{\iI} (\Gam_i + \Gam^i_v)$, $b_t + b_v = +_{\iI} b_i +_{\iI} b^i_v = +_{\iI} (b_i + b^i_v)$, $m_t + m_v = +_{\iI} m_i +_{\iI} m^i_v = +_{\iI} (m_i + m^i_v)$, and $d_t + d_v = +_{\iI} d_i +_{\iI} d^i_v = +_{\iI} (d_i + d^i_v)$.
            \item Case $\Phi_t$ ends with (\ruleLift). Then $t$ is a variable and $\Phi_t$ must be of the following form:
            \[ \begin{prooftree}
                \hypo{\Phi_w \tr \seqi{\Gam; x : \M}{w}{\M'}{(b_t,m_t,d_t)}}
                \infer1[(\ruleLift)]{\seqi{\Gam; x : \M}{w}{\tcomptype{\stype}{\M'}{\stype}}{(b_t,m_t,d_t)}}
            \end{prooftree} \]
            where $\gtype = \tcomptype{\stype}{\M'}{\stype}$. By the \ih, we have $\Phi_{w \subs{x}{v}} \tr \seqi{\Gam + \Gam_v}{w \subs{x}{v}}{\M'}{(b_t+b_v, m_t+m_v, d_t +d_v)}$. Therefore, we can build $\Phi_{t \subs{x}{v}}$ as follows:
            \[ \begin{prooftree}
                \hypo{\Phi_{w \subs{x}{v}} \tr \seqi{\Gam + \Gam_v}{w \subs{x}{v}}{\M'}{(b_t+b_v, m_t+m_v, d_t +d_v)}}
                \infer1[(\ruleLift)]{\seqi{\Gam + \Gam_v}{w \subs{x}{v}}{\tcomptype{\stype}{\M'}{\stype}}{(b_t+b_v, m_t+m_v, d_t +d_v)}}
            \end{prooftree} \]
            And we can conclude.
            \item Case $\Phi_t$ ends with ($\ruleGet$). Then $t$ must be of the form $\get{l}{y}{u}$ and $\Phi_t$ must be of the following form:
            \[ \begin{prooftree}
                \hypo{\Phi_u \tr \seqi{\Gam_u; x : \M}{u}{\comptype{\stype}{\ctype}}{(b_u,m_u,d_u)}}
                \infer1[(\ruleGet)]{\seqi{(\Gam_u \sm y); x : \M}{\get{l}{y}{u}}{\comptype{\conj{(l : \Gam_{u}(y))} \splus \stype}{\ctype}}{(b_u,1+m_u,d_u)}}
            \end{prooftree} \]
          where $\gtype = \comptype{\conj{(l : \Gam_{u}(y))} \splus \stype}{\ctype}$, $\Gam_t = \Gam_u \sm y$, $b_t = b_u$, $m_t = 1+m_u$, and $d_t = d_u$. By the \ih, we have $\Phi_{u \subs{x}{v}} \tr \seqi{\Gam_u + \Gam_v}{u \subs{x}{v}}{\stype \ra \ctype}{(b_u+b_v,m_u+m_v,d_u+d_v)}$. Therefore, we can build $\Phi_{t \subs{x}{v}}$ as follows:
            \[ \begin{prooftree}
                \hypo{\Phi_{u \subs{x}{v}} \tr \seqi{\Gam_u + \Gam_v}{u \subs{x}{v}}{\comptype{\stype}{\ctype}}{(b_u+b_v,m_u+m_v,d_u+d_v)}}
                \infer1[(\ruleGet)]{\seqi{(\Gam_u  + \Gam_v) \sm y}{\get{l}{y}{u} \subs{x}{v}}{\comptype{\conj{(l : \Gam_u(y))} \splus \stype}{\ctype}}{(b_u+b_v,1+m_u+m_v,d_u+d_v)}}
            \end{prooftree} \]
            And we can conclude with $(\Gam_u + \Gam_v) \sm y = (\Gam \sm y) + \Gam_v = \Gam_t + \Gam_v$ by $\alpha$-conversion, $b_t + b_v = b_u+b_v$, $m_t + m_v = 1+m_u+m_v$, and $d_t + d_v = d_u +d_v$.
            \item Case $\Phi_t$ ends with ($\ruleSet$). Then $t$ must be of the form $\set{l}{w}{u}$ and $\Phi_t$ must be of the following form:
            \[ \begin{prooftree}
                \hypo{\Phi_w \tr \seqi{\Gam_w; x : \M_1}{w}{\M'}{(b_w,m_w,d_w)}}
                \hypo{\Phi_u \tr \seqi{\Gam_u; x : \M_2}{u}{\comptype{\conj{(l : \M')}; \stype}{\ctype}}{(b_u,m_u,d_u)}}
                \infer2[(\ruleSet)]{\seqi{\Gam_w + \Gam_u; x : \M_1 \sqcup \M_2}{\set{l}{w}{u}}{\comptype{\stype}{\ctype}}{(b_w+b_u,1+m_w+m_u,d_w+d_u)}}
            \end{prooftree} \]
            where $\gtype = \comptype{\stype}{\ctype}$, $\Gam_t = \Gam_w + \Gam_u$, $\del = \comptype{\stype}{\ctype}$, $b_t = b_w+b_u$, $m_t = 1+m_w + m_u$, and $d_t = d_w + d_u$. By~\cref{lem:split-values-stores}.\ref{lem:com-split-values}, we have $\Phi^1_v \tr \seqi{\Gam^1_v}{v}{\M_1}{(b^1_v,m^1_v,d^1_v)}$ and $\Phi^2_v \tr \seqi{\Gam^2_v}{v}{\M_2}{(b^2_v,m^2_v,d^2_v)}$, such that $\Gam_v = \Gam^1_v + \Gam^2_v$, $b_v = b^1_v + b^2_v$, $m_v = m^1_v + m^2_v$, and $d_v = d^1_v + d^2_v$. By the \ih, we have $\Phi_{w \subs{x}{v}} \tr \seqi{\Gam_w + \Gam^1_v}{w \subs{x}{v}}{\M'}{(b_w+b^1_v,m_w+m^1_v,d_w+d^1_v)}$ and $\Phi_{u \subs{x}{v}} \tr \seqi{\Gam_u + \Gam^2_v}{u \subs{x}{v}}{\comptype{\conj{(l : \M')}; \stype}{\ctype}}{(b_u+b^2_v,m_u+m^2_u,d_u+d^2_v)}$. Assume $\Phi_{w \subs{x}{v}} \tr \seqi{\Gam_w + \Gam^1_v}{w \subs{x}{v}}{\M'}{(b_w+b^1_v,m_w+m^1_v,d_w+d^1_v)}$ and $\Phi_{u \subs{x}{v}} \tr \seqi{\Gam_u + \Gam^2_v}{u \subs{x}{v}}{\comptype{\conj{(l : \M')}; \stype}{\ctype}}{(b_u+b^2_v,m_u+m^2_u,d_u+d^2_v)}$. We can build $\Phi_{t \subs{x}{v}}$ as follows:
            \[ \begin{prooftree}
                \hypo{\Phi_{w \subs{x}{v}}}
                \hypo{\Phi_{u \subs{x}{v}}}
                \infer2[(\ruleSet)]{\seqi{(\Gam_w + \Gam_u) + (\Gam^1_v + \Gam^2_v)}{(wu) \subs{x}{v}}{\comptype{\stype}{\ctype}}{(b_w+b_u+b^1_v+b^2_v,1+m_w+m_u+m^1_v+m^2_v,d_w+d_u+d^1_v+d^2_v)}}
            \end{prooftree} \]
            And we can conclude with $\Gam_t + \Gam_v = (\Gam_w + \Gam_u) + (\Gam^1_v + \Gam^2_v)$, $b_t + b_v = b_w+b_u+b^1_v+b^2_v$, $m_t + m_v = 1+m_w+m_u+m^1_v+m^2_v$, $d_t + d_v = d_w+d_u+d^1_v+d^2_v$.
            \item Case $\Phi_t$ ends with (\ruleAxP). Then $t$ must be a variable and we must consider two cases:
            \begin{itemize}
                \item Assume $t = y = x$. Then $\Gam_t = \eset$, $\gtype = \stype \ta (\comptype{\nott{\tneutral}}{\stype})$, $t \subs{x}{v} = v$, $b_t = m_t = d_t = 0$. Moreover, $\M = \mul{\nott{\tneutral}}$. We have to consider two cases:
                \begin{itemize}
                    \item Case $v = z$. Then $\Phi_v \tr \seqi{z : \mul{\nott{\tneutral}}}{z}{\mul{\nott{\tneutral}}}{(0,0,0)}$. So we can take $\Phi_{t \subs{x}{v}}$ as the following derivation:
                    \[ \begin{prooftree}
                        \infer0[(\ruleAxP)]{\seqi{z : \mul{\nott{\tneutral}}}{z}{\tcomptype{\stype}{\nott{\tneutral}}{\stype}}{(0,0)}}
                    \end{prooftree} \]
                    and conclude with $\Gam_t + \Gam_v = \Gam_v = (z : \mul{\nott{\tneutral}})$, $b_t + b_v = b_v = 0$, $m_t + m_v = m_v = 0$, and $d_t + d_v = d_v$.
                    \item Case $v = \lam z.p$. This case does not apply, by~\cref{lem:comp-notabs-implies-negabs}.
                \end{itemize}
                \item Assume $t = y \neq x$. Then $\M = \emul$, $\Gam_v = \eset$, $t \subs{x}{v} = t$, $b_v = 0$, $m_v = 0$, and $d_v = 0$. So we can take $\Phi_{t \subs{x}{v}} = \Phi_t$ and conclude with $\Gam_t + \Gam_v = \Gam_t$, $b_t + b_v = b_t$, $m_t + m_v = m_t$, and $d_t + d_v = d_t$.
            \end{itemize}
            \item Case $\Phi_t$ ends with (\ruleLamP). Then $t$ is of the form $\lam y.u$, $\Gam_t = \eset$, $\gtype = \tcomptype{\stype}{\vl}{\stype}$, $\M = \emul$, $\Gam_v = \eset$, $t \subs{x}{v} = \lam y.(u \subs{x}{v}) = (\lam y.u) \subs{x}{v}$, $b_t = b_v = 0$, $m_t = m_v = 0$, and $d_t = d_v = 0$. So we can build $\Phi_{t \subs{x}{v}}$ as follows:
            \[ \begin{prooftree}
                \infer0[(\ruleLamP)]{\seqi{}{(\lam y.u) \subs{x}{v}}{\tcomptype{\stype}{\vl}{\stype}}{(0,0,0)}}
            \end{prooftree} \]
            And conclude with $\Gam_t + \Gam_v = \eset$, $b_t = b_v = 0$, $m_t = m_v = 0$, and $d_t = d_v = 0$.
            \item Case $\Phi_t$ ends with (\ruleAppPOne). Then $t$ is of the form $yu$ and we have to consider two cases:
            \begin{itemize}
                \item Case $y = x$. Then $\Phi_t$ must be of the following form:
                \[ \begin{prooftree}
                    \hypo{\seqi{\Gam_u}{u}{\tcomptype{\stype}{\tightt}{\stype}}{(b_u,m_u,d_u)}}
                    \infer1[(\ruleAppPOne)]{\seqi{(x : \mul{\tvar} \sqcup \Gam_u(x)); (\Gam_u \sm x)}{x u}{\tcomptype{\stype}{\tneutral}{\stype}}{(b_u,m_u,1+d_u)}}
                \end{prooftree} \]
                such that $\Gam_t = (\Gam_u \sm x)$, $b = b_u$, $m = m_u$, and $d = 1+d_u$. Then $\M = \mul{\tvar} \sqcup \Gam_u(x)$ and, by~\cref{lem:split-values-stores}.\ref{lem:com-split-values}, we have $\Phi^1_v \tr \seqi{\Gam^1_v}{v}{\mul{\tvar}}{(b^1_v,m^1_v,d^1_v)}$ and $\Phi^2_v \tr \seqi{\Gam^2_v}{v}{\Gam_u(x)}{(b^2_v,m^2_v,d^2_v)}$, such that $\Gam_v = \Gam^1_v + \Gam^2_v$, $b_v = b^1_v + b^2_v$, $m_v = m^1_v + m^2_v$, and $d_v = d^1_v + d^2_v$. By the \ih, we know there exists $\Phi_{u \subs{x}{v}} \tr \seqi{(\Gam_u \sm x) + \Gam^2_u}{u \subs{x}{v}}{\tcomptype{\stype}{\tightt}{\stype}}{(b_u+b^2_v, m_u+m^2_v, d_u+d^2_v)}$. Now, we need to consider two cases:
                \begin{itemize}
                    \item Case $v = z$. Then $\Phi^1_v \tr \seqi{z : \mul{\tvar}}{z}{\mul{\tvar}}{(0,0,0)}$ and $\Phi^2_v \tr \seqi{z : \Gam_u(x)}{z}{\Gam_u(x)}{(0,0)}$. Therefore, we can build $\Phi_{t \subs{x}{v}} = \Phi_{v \subs{x}{v}}$ as follows:
                    \[ \begin{prooftree}
                        \hypo{\Phi_{u \subs{x}{v}} \tr \seqi{(\Gam_u \sm x) + \Gam_u(x)}{u \subs{x}{v}}{\tcomptype{\stype}{\tightt}{\stype}}{(b_u+b^2_v, m_u+m^2_v, d_u+d^2_v)}}
                        \infer1[(\ruleAppPOne)]{\seqi{(z : \mul{\tvar}) + (\Gam_u \sm x + (z : \Gam_u(x)))}{z (u \subs{x}{v})}{\tcomptype{\stype}{\tneutral}{\stype}}{(b_u+b^2_v,m_u+m^2_v,1+d_u+d^2_v)}}
                    \end{prooftree} \]
                    where $(z : \mul{\tvar}) + (\Gam_u \sm x + (z : \Gam_u(x))) = (\Gam_u \sm x) + (z : \mul{\tvar} \cup \Gam_u(x)) = \Gam_u + \Gam_v$, $b_u + b^2_v = b + b^1_v + b^2_v = b + b_v$, $m_u + m^2_v = m + m^1_v + m^2_v = m + m_v$, and $d_u + d^2_v = d + d^1_v + d^2_v = d + d_v$.
                    \item Case $v = \lam z.p$. This case does not apply, since it is not possible to assign $\tvar$ to $\lam z.p$, by~\cref{lem:comp-notabs-implies-negabs}.
                \end{itemize}
                \item Case $y \neq x$. Then, the proof is very similar to when $\Phi_t$ ends with rule ($\ruleApp$).
            \end{itemize}
            \item Case $\Phi_t$ ends with (\ruleAppPTwo), the proof is very similar to when $\Phi_t$ ends with rule (\ruleAppPOne).
        \end{itemize}
%\end{proof}

        \item %\begin{proof}
    We are going to generalize the original statement by replacing $\del$ with $\gtype$.
    \\ \\
    The proof follows by induction over $t$:
    \begin{itemize}
        \item Case $t = y$. Then we have to consider two cases:
        \begin{itemize}
            \item Let $t = y \not= x$. Then $t \subs{x}{v} = y$. Let $\Gam_v = \eset$, $\M = \emul$, $b_v = m_v = d_v = 0$. Then, $\Phi_v$ is derivable using rule ($\ruleMany$) with no premise. We also take $\Phi_t = \Phi_{t \subs{x}{v}}$, so that, in particular $\Gam_t = \Gam_{t \subs{x}{v}}$. Then, we can conclude with $\Gam_{t \subs{x}{v}} = \Gam_t + \Gam_v = \Gam_t$, $b = b_t + b_v = b_t$, $m = m_t + m_v = m_t$, and $d = d_t + d_v = d_t$.
            \item Let $t = y = x$. Then $t \subs{x}{v} = v$. Let $\Gam_t = \eset$, and $b_t = m_t = s_t = 0$. Now we will consider two cases depending on the form of $v$:
            \begin{itemize}
                \item Case $v = z$. Then $t \subs{x}{v} = z$ and we can proceed by case analysis of the last rule in $\Phi_{t\subs{x}{v}}$. In all of them, we can build $\Phi_t$ from $\Phi_{t \subs{x}{v}}$, by simply replacing $x$ with $z$, and $\Phi_v$ as follows:
                \[ \begin{prooftree}
                    \infer0[(\ruleAx)]{\seqi{z : \mul{\sig}}{z}{\sig}{(0,0,0)}}
                    \infer1[(\ruleMany)]{\seqi{z : \mul{\sig}}{z}{\mul{\sig}}{(0,0,0)}}
                \end{prooftree} \]
                And we can conclude since all the counters are zero.
                \item Case $v = \lam z.p$. Then $t \subs{x}{v} = \lam z.p$ and we can proceed by case analysis of the last rule in $\Phi_{t \subs{x}{v}}$. In all of them, we can always build $\Phi_t$ using either (\ruleAx) (case (\ruleApp)), (\ruleAxP) (case (\ruleLamP)),  (\ruleAx) plus (\ruleMany) (case (\ruleMany)), or (\ruleAx) plus (\ruleMany) plus (\ruleLift) (case (\ruleLift)). $\Phi_v$  is either $\Phi_{t \subs{x}{v}}$ (case (\ruleMany)), or it can be built from $\Phi_{t \subs{x}{v}}$ plus rule (\ruleMany) (all other cases).
            \end{itemize}
        \end{itemize}
        \item Case $t = \lam y.u$. Then $t \subs{x}{v} = (\lam y.u)\subs{x}{v} = \lam y.(u \subs{x}{v})$ and we must consider three cases:
        \begin{itemize}
            \item Case $\Phi_{t \subs{x}{v}}$ ends with rule (\ruleLam), then it must be of the following form: 
            \[ \begin{prooftree}
                \hypo{\Phi_{u \subs{x}{v}} \tr \seqi{\Gam_{u \subs{x}{v}}; y : \M'}{u \subs{x}{v}}{\comptype{\stype}{\ctype}}{(b,m,d)}}
                \infer1[(\ruleLam)]{\seqi{\Gam_{u \subs{x}{v}}}{\lam y.(u \subs{x}{v})}{\M' \ta (\comptype{\stype}{\ctype})}{(b,m,d)}}
            \end{prooftree} \]
            where $\gtype = \M' \ta (\comptype{\stype}{\ctype})$ and $\Gam_{t \subs{x}{v}} = \Gam_{u \subs{x}{v}}$. By the \ih, we have $\Phi_u \tr \seqi{\Gam_u; y : \M'; x : \M}{u}{\comptype{\stype}{\ctype}}{(b_u,m_u,d_u)}$ and $\Phi_v \tr \seqi{\Gam_v}{v}{\M}{(b_v,m_v,d_v)}$, such that $\Gam_{u \subs{x}{v}} = \Gam_u + \Gam_v$, $b = b_u + b_v$, $m = m_u + m_v$, and $d = d_u + d_v$. So we can build $\Phi_{\lam y.u}$ as follows:
            \[ \begin{prooftree}
                \hypo{\Phi_u \tr \seqi{\Gam_u; y: \M'; x : \M}{u}{\comptype{\stype}{\ctype}}{(b_u,m_u,d_u)}}
                \infer1[(\ruleLam)]{\seqi{\Gam_u; x : \M}{\lam y.u}{\M' \ta (\comptype{\stype}{\ctype})}{(b_u,m_u,d_u)}}
            \end{prooftree} \]
            And we can pick $\Phi_t = \Phi_{\lam y.u}$, and conclude with $\Gam_{t \subs{x}{v}} = \Gam_{u \subs{x}{v}} = \Gam_u + \Gam_v$, $b = b_u + b_v$, $m = m_u + m_v$, and $d = d_u + d_v$.
            \item Case $\Phi_{t \subs{x}{v}}$ ends with rule (\ruleLamP). Then it must be of the following form:
            \[ \begin{prooftree}
                \infer0[(\ruleLamP)]{\seqi{}{\lam y.(u \subs{x}{y})}{\tcomptype{\stype}{\vl}{\stype}}{(0,0,0)}}
            \end{prooftree} \]
            where $\Gam_{t \subs{x}{v}} = \eset$, $\gtype = \tcomptype{\stype}{\vl}{\stype}$, and $b = m = d = 0$. Let $\Gam_t = \eset$, $\M = \emul$, and $b_t = m_t = d_t = 0$. Then, we can construct $\Phi_t$ as follows:
            \[ \begin{prooftree}
                \infer0[(\ruleLamP)]{\seqi{}{\lam y.u}{\tcomptype{\stype}{\vl}{\stype}}{(0,0,0)}}
            \end{prooftree} \]
            Let $\Gam_v = \eset$, and $b_v = m_v = d_v = 0$. Then $\Phi_v$ can be constructed by using rule ($\ruleMany$) with no premises. So we can conclude with $\Gam_{t \subs{x}{v}} = \eset = \Gam_t + \Gam_v$, and $b = 0 = b_t + b_v$, $m = 0 = m_t + m_v$, and $d = 0 = d_t + d_v$.
            \item Case $\Phi_{t \subs{x}{v}}$ ends with rule ($\ruleMany$). Then $t \subs{x}{v}$ is a value, and $\Phi_{t \subs{x}{v}}$ must be of the following form:
            \[ \begin{prooftree}
                \hypo{(\Phi_i \tr \seqi{\Gam_i}{t \subs{x}{v}}{\rdel_i}{(b_i,m_i,d_i)})_{\iI}}
                \infer1[(\ruleMany)]{\seqi{+_{\iI} \Gam_i}{t \subs{x}{v}}{\mul{\rdel_i}_{\iI}}{(+_{\iI} b_i, +_{\iI} m_i, +_{\iI} d_i)}}
            \end{prooftree} \]
            where $\gtype = \mul{\rdel_i}_{\iI}$, $\Gam_{t \subs{x}{v}} = +_{\iI} \Gam_i$, $b = +_{\iI} b_i$, $m = +_{\iI} m_i$, and $d = +_{\iI} d_i$. By the \ih over each $\Phi_i$, we have the following derivations $\Phi^i_t \tr \seqi{\Gam^i_t; x : \M_i}{t}{\rdel_i}{(b^i_t,m^i_t,d^i_t)}$ and $\Phi^i_v \tr \seqi{\Gam^i_v}{v}{\M_i}{(b^i_v, m^i_v, d^i_v)}$, such that $\Gam_i = \Gam^i_t + \Gam^i_v$, $b = b^i_t + b^i_v$, $m = m^i_t + m^i_v$, and $d = d^i_t + d^i_v$, for each $\iI$. So we can construct $\Phi_t$ as follows:
            \[ \begin{prooftree}
                \hypo{(\Phi^i_t \tr \seqi{\Gam^i_t; x : \M_i}{t}{\rdel_i}{(b^i_t,m^i_t,d^i_t)})_{\iI}}
                \infer1[(\ruleMany)]{\seqi{+_{\iI} \Gam^i_t; x : \sqcup_{\iI} \M_i}{t}{\mul{\rdel_i}_{\iI}}{(+_{\iI} b^i_t, +_{\iI} m^i_t, +_{\iI} d^i_t)}}
            \end{prooftree} \]
            such that $\Gam_t = +_{\iI} \Gam^i_t$, $\M = \sqcup_{\iI} \M_i$, $b_t = +_{\iI} b^i_t$, $m_t = +_{\iI} m^i_t$, and $d_t = +_{\iI} d^i_t$. By~\cref{lem:comp-merge-values}, we can take the following derivation $\Phi_v \tr \seqi{+_{\iI} \Gam^i_v}{v}{\M}{(+_{\iI} b^i_v, +_{\iI} m^i_v, +_{\iI} d^i_v)}$. And we can conclude with $\Gam_{t \subs{x}{v}} = +_{\iI} \Gam_i = +_{\iI} (\Gam^i_t + \Gam^i_v) = +_{\iI} \Gam^i_t +_{\iI} \Gam^i_v = \Gam_t + \Gam_v$, $b = +_{\iI} b_i = +_{\iI} (b^i_t + b^i_v) = +_{\iI} b^i_t +_{\iI} b^i_v = b_t + b_v$, $m = +_{\iI} m_i = +_{\iI} (m^i_t + m^i_v) = +_{\iI} m^i_t +_{\iI} m^i_v = m_t + m_v$, and $d = +_{\iI} d_i = +_{\iI} (d^i_t + d^i_v) = +_{\iI} d^i_t +_{\iI} d^i_v = d_t + d_v$.
        \end{itemize}
        \item Let $t = wu$. Then $t \subs{x}{v} = (wu) \subs{x}{v} = (w \subs{x}{v})(u \subs{x}{v})$, and we have to consider three cases:
        \begin{itemize}
            \item Case $\Phi_{t \subs{x}{v}}$ ends with ($\ruleApp$). Assume $\Phi_{w \subs{x}{v}} \tr \seqi{\Gam_{w \subs{x}{v}}}{w \subs{x}{v}}{\M' \ta (\comptype{\stype'}{\ctype})}{(b',m',d')}$ and $\Phi_{u \subs{x}{v}} \tr \seqi{\Gam_{u \subs{x}{v}}}{u \subs{x}{v}}{\tcomptype{\stype}{\M'}{\stype'}}{(b'',m'',d'')}$. $\Phi_{t \subs{x}{v}}$ must be of the following form:
            \[ \begin{prooftree}
                \hypo{\Phi_{w \subs{x}{v}}}
                \hypo{\Phi_{u \subs{x}{v}}}
                \infer2[(\ruleApp)]{\seqi{\Gam_{w \subs{x}{v}} + \Gam_{u \subs{x}{v}}}{(w \subs{x}{v})(u \subs{x}{v})}{\comptype{\stype}{\ctype}}{(1+b'+b'',m'+m'',d'+d'')}}
            \end{prooftree} \]
            where $\gtype = \comptype{\stype}{\ctype}$, $\Gam_{t \subs{x}{v}} = \Gam_{w \subs{x}{v}} + \Gam_{u \subs{x}{v}}$, $b = 1+b'+b''$, $m = m'+m''$, and $d = d'+d''$. By the \ih over $\Phi_{w \subs{x}{v}}$, we have $\Phi_w \tr \seqi{\Gam_w; x : \M_1}{w}{\M' \ta (\comptype{\stype'}{\ctype})}{(b_w,m_w,d_w)}$ and $\Phi^1_v \tr \seqi{\Gam^1_v}{v}{\M_1}{(b^1_v,m^1_v,d^1_v)}$, such that $\Gam_{w \subs{x}{v}} = \Gam_w + \Gam^1_v$, $b' = b_w + b^1_v$, $m' = m_w + m^1_v$, and $d' = d_w + d^1_v$. And by the \ih over $\Phi_{u \subs{x}{v}}$, we have $\Phi_u \tr \seqi{\Gam_u; x : \M_2}{u}{\tcomptype{\stype}{\M'}{\stype'}}{(b_u, m_u,d_u)}$ and $\Phi^2_v \tr \seqi{\Gam^2_v}{v}{\M_2}{(b^2_v,m^2_v,d^2_v)}$, such that $\Gam_{u \subs{x}{v}} = \Gam_u + \Gam^2_v$, $b'' = b_u + b^2_v$, $m'' = m_u + m^2_v$, and $d'' = d_u + d^2_v$. By~\cref{lem:comp-merge-values}, we can take $\Phi_v \tr \seqi{\Gam^1_v + \Gam^2_v}{v}{\M_1 \sqcup \M_2}{(b^1_v+b^2_v, m^1_v+m^2_v, d^1_v+d^2_v)}$, such that $\Gam_v = \Gam^1_v + \Gam^2_v$, $b_v = b^1_v + b^2_v$, $m_v = m^1_v + m^2_v$, and $d_v = d^1_v + d^2_v$. And we can build $\Phi_{wu}$ as follows:
            \[ \begin{prooftree}
                \hypo{\Phi_w}
                \hypo{\Phi_u}
                \infer2[(\ruleApp)]{\seqi{(\Gam_w + \Gam_u); x : \M_1 \sqcup \M_2}{wu}{\comptype{\stype}{\kappa}}{(1+b_w+b_u,m_w+m_u,d_w+d_u)}}
            \end{prooftree} \]
            such that $\Gam_t = \Gam_w + \Gam_u$, $b_t = 1 + b_w + b_u$, $m_t = b_w + b_u$, and $d_t = d_w + d_u$. So we can pick $\Phi_t = \Phi_{wu}$, and conclude with $\Gam_{t \subs{x}{v}} = \Gam_{w \subs{x}{v}} + \Gam_{u \subs{x}{v}} = (\Gam_w + \Gam^1_v) + (\Gam_u + \Gam^2_v) = (\Gam_w + \Gam_u) + (\Gam^1_v + \Gam^2_v) = \Gam_t + \Gam_v$, $b = 1 + b' + b'' = 1 + b_w + b^1_v + b_u + b^2_v = (1 + b_w + b_u) + (b^1_v + b^2_v) = b_t + b_v$, $m = m' + m'' = m_w + m^1_v + m_u + m^2_v = (m_w + m_u) + (m^1_v + m^2_v) = m_t + m_v$, and $d = d' + d'' = d_w + d^1_v + d_u + d^2_v = (d_w + d_u) + (d^1_v + d^2_v) = d_t + d_v$.
            \item Case $\Phi_{t \subs{x}{v}}$ ends with (\ruleAppPOne) or (\ruleAppPTwo). These cases are very similar to the case where $\Phi_{t \subs{x}{v}}$ ends with (\ruleApp).
        \end{itemize}
        \item Let $t = \get{l}{y}{u}$. Then $t \subs{x}{v} = \get{l}{y}{u \subs{x}{v}}$ and $\Phi_{t \subs{x}{v}}$ must be of the following form:
        \[ \begin{prooftree}
            \hypo{\Phi_{u \subs{x}{v}} \tr \seqi{\Gam_{u \subs{x}{v}}; y : \M'}{u \subs{x}{v}}{\comptype{\stype}{\ctype}}{(b,m',d)}}
            \infer1[(\ruleGet)]{\seqi{\Gam_{u \subs{x}{v}}}{\get{l}{y}{u \subs{x}{v}}}{\comptype{\conj{(l : \M')} \splus \stype}{\ctype}}{(b,1+m',d)}}
        \end{prooftree} \]
        where $\Gam_{t \subs{x}{v}} = \Gam_{u \subs{x}{v}}$ and $m = 1+m'$. By the \ih, we have $\Phi_u \tr \seqi{\Gam_u; y : \M'; x : \M}{u}{\comptype{\stype}{\ctype}}{(b_u,m_u,d_u)}$ and $\Phi_v \tr \seqi{\Gam_v}{v}{\M}{(b_v,m_v,d_v)}$, such that $\Gam_{u \subs{x}{v}} = \Gam_u + \Gam_v$, $b = b_u + b_v$, $m' = m_u + m_v$, and $d = d_u + d_v$. So we can build $\Phi_{\get{l}{y}{u}}$ as follows:
        \[ \begin{prooftree}
            \hypo{\Phi_u \tr \seqi{\Gam_u; y : \M'; x : \M}{u}{\comptype{\stype}{\ctype}}{(b_u,m_u,d_u)}}
            \infer1[(\ruleGet)]{\seqi{\Gam_{u}; x : \M}{\get{l}{y}{u}}{\comptype{\conj{(l : \M')} \splus \stype}{\ctype}}{(b_u,1+m_u,d_u)}}
        \end{prooftree} \]
        And we can pick $\Phi_t = \Phi_{\get{l}{y}{u}}$, and conclude with $\Gam_{t \subs{x}{v}} = \Gam_{u \subs{x}{v}} = \Gam_u + \Gam_v$, $b = b_u + b_v$, $m = 1 + m' = 1 + m_u + m_v = (1 + m_u) + m_v$, and $d = d_u + d_v$.
        \item Let $t = \set{l}{w}{u}$. Then $t \subs{x}{v} = (\set{l}{w}{u}) \subs{x}{v} = \set{l}{w \subs{x}{v}}{u \subs{x}{v}}$. Assume $\Phi_{w \subs{x}{v}} \tr \seqi{\Gam_{w \subs{x}{v}}}{w \subs{x}{v}}{\M}{(b',m',d')}$ and $\Phi_{u \subs{x}{v}} \tr \seqi{\Gam_{u \subs{x}{v}}}{u \subs{x}{v}}{\comptype{\conj{(l : \M)}; \stype}{\ctype}}{(b'',m'',d'')}$. $\Phi_{t \subs{x}{v}}$ must be of the following form:
        \[ \begin{prooftree}
            \hypo{\Phi_{w \subs{x}{v}}}
            \hypo{\Phi_{t \subs{x}{v}}}
            \infer2[(\ruleSet)]{\seqi{\Gam_{w \subs{x}{v}} + \Gam_{u \subs{x}{v}}}{\set{l}{w \subs{x}{v}}{u \subs{x}{v}}}{\comptype{\stype}{\ctype}}{(b'+b'',1+m'+m'',d'+d'')}}
        \end{prooftree} \]
        where $\Gam_{t \subs{x}{v}} = \Gam_{w \subs{x}{v}} + \Gam_{u \subs{x}{v}}$, $b = b'+ b''$, $m = 1 + m'+m''$, and $d = d' + d''$. By the \ih over $\Phi_{w \subs{x}{v}}$, we have $\Phi_w \tr \seqi{\Gam_w; x : \M_1}{w}{\M}{(b_w,m_w,d_w)}$ and $\Phi^1_v \tr \seqi{\Gam^1_v}{v}{\M_1}{(b^1_v,m^1_v,d^1_v)}$, such that $\Gam_{w \subs{x}{v}} = \Gam_w + \Gam^1_v$, $b' = b_w + b^1_v$, $m'= m_w + m^1_v$, and $d' = d_w+d^1_v$. And by the \ih over $\Phi_{u \subs{x}{v}}$, we have $\Phi_u \tr \seqi{\Gam_u; x : \M_2}{u}{\comptype{\conj{(l : \M)}; \stype}{\ctype}}{(b_u,m_u,d_u)}$ and $\Phi^2_v \tr \seqi{\Gam^2_v}{v}{\M_2}{(b^2_v,m^2_v,d^2_v)}$, such that $\Gam_{u \subs{x}{v}} = \Gam_u + \Gam^2_v$, $b'' = b_u + b^2_v$, $m'' = m_u + m^2_v$, and $d'' = d_u + d^2_v$. By~\cref{lem:comp-merge-values}, we can take $\Phi_v \tr \seqi{\Gam^1_v + \Gam^2_v}{v}{\M_1 \sqcup \M_2}{(b^1_v + b^2_v, m^1_v+m^2_v, d^1_v + d^2_v)}$, such that $\Gam_v = \Gam^1_v + \Gam^2_v$, $b_v = b^1_v + b^2_v$, $m_v = m^1_v + m^2_v$, and $d_v = d^1_v + d^2_v$. And we can build $\Phi_{\set{l}{w}{u}}$ as follows:
        \[ \begin{prooftree}
            \hypo{\Phi_w \tr \seqi{\Gam_w; x : \M_1}{w}{\M}{(b_w,m_w,d_w)}}
            \hypo{\Phi_u \tr \seqi{\Gam_u; x : \M_2}{u}{\comptype{\conj{(l : \M)}; \stype}{\ctype}}{(b_u,m_u,d_u)}}
            \infer2[(\ruleSet)]{\seqi{(\Gam_w + \Gam_u); x : \M_1 \sqcup \M_2}{\set{l}{w}{u}}{\comptype{\stype}{\ctype}}{(b_w+b_u, 1+m_w+m_u,d_w+d_u)}}
        \end{prooftree} \]
        such that $\Gam_t = \Gam_w + \Gam_u$, $b_t = b_w + b_u$, $m_t = 1 + m_w + m_u$, and $d_t = d_w + d_u$. So we can pick $\Phi_t = \Gam_{\set{l}{w}{u}}$, and conclude with $\Gam_{t \subs{x}{v}} = \Gam_{w \subs{x}{v}} + \Gam_{u \subs{x}{v}} = (\Gam_w + \Gam^1_u) + (\Gam_u + \Gam^2_v) = (\Gam_w + \Gam_u) + (\Gam^1_v + \Gam^2_v) = \Gam_t + \Gam_v$, $b = b' + b'' = (b_w + b^1_v) + (b_u + b^2_v) = (b_w + b_u) + (b^1_v + b^2_v) = b_t + b_v$, $m = 1+ m' + m'' = 1+ (m_w + m^1_v) + (m_u + m^2_v) = (1 + m_w + m_u) + (m^1_v + m^2_v) = m_t + m_v$, and $d = d' + d'' = (d_w + d^1_v) + (d_u + d^2_v) = (d_w + d_u) + (d^1_v + d^2_v) = d_t + d_v$.
    \end{itemize}
%\end{proof}

    \end{enumerate}
\end{proof}}

\lemexactredexp*

\maybehide{\begin{proof} \mbox{}
    \begin{enumerate} 
        \item %\begin{proof}
  We show a stronger statement of the form:

  Let $(t,s) \red[\gname] (u,q)$. If $\Phi \tr \seqi{\Gam}{(t,s)}{\ctype}{(b,m,d)}$, $\Gam$ is tight,  and ($\ctype$ is tight or $\neg \isvalue{t}$), then $\Phi' \tr \seqi{\Gam}{(u,q)}{\ctype}{(b',m',d)}$, where $\gname =\beta$ implies $b' = b - 1$ and $m' = m$, while  $\gname \in \{\getname, \setname\}$ implies $b'=b$ and $m' = m - 1$.

  We proceed by induction on $(t,s) \ra (u,q)$: 
  \begin{itemize}
    \item Case $(t,s) = ((\lam x.p) v,s) \redbeta (p \subs{x}{v}, s) = (u,q)$. Let $\Phi_{(\lam x.p) v}$ be the sub-derivation for $(\lam x.p) v$ in $\Phi$. Assume that $\Phi_{(\lam x.p)v}$ ends with rule (\ruleAppPTwo). Then $v$ must be assigned type $\comptype{\stype}{\tneutral \tim \stype}$, which is not possible by~\cref{lem:comp-values-not-neutral}. Let $\Phi_0$ be the following derivation:
    \[ \begin{prooftree}
      \hypo{\Phi_p \tr \seqi{\Gam_{\lam x.p}; x : \M}{p}{\comptype{\stype}{\ctype}}{(b_p,m_p,d_p)}}
        \infer1[(\ruleLam)]{\seqi{\Gam_{\lam x.p}}{\lam x.p}{\M \ta (\comptype{\stype}{\ctype})}{(b_p,m_p,d_p)}}
        \hypo{\Phi_v \tr \seqi{\Gam_v}{v}{\M}{(b_v,m_v,d_v)}}
        \infer1[(\ruleLift)]{\seqi{\Gam_v}{v}{\tcomptype{\stype}{\M}{\stype}}{(b_v,m_v,d_v)}}
        \infer2[(\ruleApp)]{\seqi{\Gam_{\lam x.p}+\Gam_v}{(\lam x.p)v}{\comptype{\stype}{\ctype}}{(1+b_v+b_p,m_v+m_p,d_v+d_p)}}
    \end{prooftree} \]
    $\Phi_{(\lam x.p) v}$ must end with rule ($\ruleApp$) and $\Phi$ must be of the following form:
    \[ \begin{prooftree}
        \hypo{\Phi_0}
        \hypo{\Phi_s \tr \seqi{\Del}{s}{\stype}{(b_s,m_s,d_s)}}
        \infer2[(\ruleConf)]{\seqi{\Gam_{\lam x.p}+ \Gam_v + \Del}{((\lam x.p)v, s)}{\ctype}{(1+b_v+b_p+b_s,m_v+m_p+m_s,d_v+d_p+d_s)}}
    \end{prooftree} \]
    where  $\Gam = \Gam_{\lam x.p}+ \Gam_v + \Del$,  $b = 1+ b_v + b_p + b_s$, $m = m_v + m_p + m_s$, and $d = d_v + d_p + d_s$. By~\cref{lem:comp-subs-antisubs}.\ref{lem:comp-subs}, there exists $\Phi_{p \subs{x}{v}} \tr \seqi{\Gam_{\lam x.p} +\Gam_v}{p \subs{x}{v}}{\comptype{\stype}{\ctype}}{(b_v+b_p,m_v+m_p,d_v+d_p)}$, therefore we can build $\Phi_{(p\subs{x}{v},s)}$ as follows:
    \[ \begin{prooftree}
        \hypo{\Phi_{p \subs{x}{v}} \tr \seqi{\Gam_{\lam x.p}+\Gam_v}{p \subs{x}{v}}{\comptype{\stype}{\ctype}}{(b_v+b_p,m_v+m_p,d_v+d_p)}}
        \hypo{\Phi_s \tr \seqi{\Del}{s}{\stype}{(b_s,m_s,d_s)}}
        \infer2[(\ruleConf)]{\seqi{\Gam_{\lam x.p} + \Gam_v + \Del}{(p \subs{x}{v}, s)}{\ctype}{(b_v+b_p+b_s,m_v+m_p+m_s,d_v+d_p+d_s)}}
    \end{prooftree} \]
    We can finally conclude since the first counter is equal to $b-1$, while the second and third remain the same.
  \item Case $(t,s) = (vp,s) \ra (vp',q) = (u,q)$, such that $(p,s) \ra (p',q)$. Then we have three cases for the type derivation $\Phi_p$ of $p$ inside $\Phi$: 
    \begin{itemize}
      \item Case $\Phi_p$ ends with ($\ruleApp$). Let $\Phi_0$ be the following derivation:
      \[ \begin{prooftree}
        \hypo{\Phi_v \tr \seqi{\Gam_v}{v}{\M \ta (\comptype{\stype'}{\ctype})}{(b_v,m_v,d_v)}}
            \hypo{\Phi_p \tr \seqi{\Gam_p}{p}{\tcomptype{\stype}{\M}{\stype'}}{(b_p,m_p,d_p)}}
            \infer2[(\ruleApp)]{\seqi{\Gam_v + \Gam_p}{vp}{\comptype{\stype}{\ctype}}{(1+b_v+b_p,m_v+m_p,d_v+d_p)}}
      \end{prooftree} \]
      $\Phi$ must be of the following form:
        \[ \begin{prooftree}
            \hypo{\Phi_0}
            \hypo{\Phi_s \tr \seqi{\Del}{s}{\stype}{(b_s,m_s,d_s)}}
            \infer2[(\ruleConf)]{\seqi{\Gam_v + \Gam_p + \Del}{(vp, s)}{\kappa}{(1+b_v+b_p+b_s,m_v+m_p+m_s,d_v+d_p+d_s)}}
        \end{prooftree} \]
        where $\Gam = \Gam_v + \Gam_p + \Del$ is tight, $b = 1+b_v+b_p+b_s$, $m = m_v+m_p+m_s$, and $s = d_v+d_p+d_s$. Therefore, we can build the following derivation for $(p,s)$:
        \[ \begin{prooftree}
            \hypo{\Phi_p \tr \seqi{\Gam_p}{p}{\tcomptype{\stype}{\M}{\stype'}}{(b_p,m_p,d_p)}}
            \hypo{\Phi_s \tr \seqi{\Del}{s}{\stype}{(b_s,m_s,d_s)}}
            \infer2[(\ruleConf)]{\seqi{\Gam_p + \Del}{(p,s)}{\conftype{\M}{\stype'}}{(b_p+b_s,m_p+m_s,d_p+d_s)}}
        \end{prooftree} \]
        Since $\Gam$ is tight, then $\Gam_p + \Del$ is tight. Moreover, $(p,s) \ra (p',q)$ implies that $\neg \isvalue{p}$. Then we can apply the \ih, and thus there exists a derivation for $(p',q)$ that must be of the following form:
        \[ \begin{prooftree}
            \hypo{\Phi_{p'} \tr \seqi{\Gam_{p'}}{p'}{\tcomptype{\stype''}{\M}{\stype'}}{(b_{p'},m_{p'},d_{p'})}}
            \hypo{\Phi_q \tr \seqi{\Del_q}{q}{\stype''}{(b_q,m_q,d_q)}}
            \infer2[(\ruleConf)]{\seqi{\Gam_{p'} + \Del_q}{(p',q)}{\conftype{\M}{\stype'}}{(b_{p'}+b_q,m_{p'}+m_q,d_{p'}+d_q)}}
        \end{prooftree} \]
        where $\Gam_{p'} + \Del_q = \Gam_p + \Del$ is tight,  and the counters are related properly. Let $\Phi_0$ be the following derivation:
        \[ \begin{prooftree}
          \hypo{\Phi_v \tr \seqi{\Gam_v}{v}{\M \ta \comptype{\stype'}{\ctype'}}{(b_v,m_v,d_v)}}
            \hypo{\Phi_{p'} \tr \seqi{\Gam_{p'}}{p'}{\tcomptype{\stype''}{\M}{\stype'}}{(b_{p'},m_{p'},d_{p'})}}
            \infer2[(\ruleApp)]{\seqi{\Gam_v+\Gam_{p'}}{vp'}{\comptype{\stype''}{\ctype'}}{(1+b_v+b_{p'},m_v+m_{p'},d_v+d_{p'})}}
        \end{prooftree} \]
        We can build $\Phi_{(u,q)}$ as follows:
        \[ \begin{prooftree}
            \hypo{\Phi_0}
            \hypo{\Phi_q \tr \seqi{\Del_q}{q}{\stype''}{(b_q,m_q,d_q)}}
            \infer2[(\ruleConf)]{\seqi{\Gam_v + \Gam_{p'} + \Del_q}{(vp', q)}{\ctype'}{(1+b_v+b_{p'}+b_q,m_v+m_{p'}+m_q,d_v+d_{p'}+d_q)}}
        \end{prooftree} \]
        where $\Gam_{p'} + \Gam_v + \Del_q = \Gam_v + \Gam_p + \Del = \Gam$, $b' = 1+b_v+b_{p'}+b_q$, $m' = m_v+m_{p'}+m_q$, and $d' = d_v+d_{p'}+d_q$. We can conclude since the counters are related properly according to the \ih.  
        \item Case $\Phi_p$ ends with (\ruleAppPOne) or (\ruleAppPTwo). These two cases are very similar to the previous case.
      \end{itemize}
      \item Case $(t,s) = (\get{l}{x}{p},s) \ra (p \subs{x}{v},s) = (u,q)$, where $s \equivstate \upd{l}{v}{s'}$. Let $\Phi_0$ be the following derivation:
      \[ \begin{prooftree}
        \hypo{\Phi_{p} \tr \seqi{\Gam_{p}}{p}{\comptype{\stype}{\ctype}}{(b_p,m_p,d_p)}}
        \infer1[(\ruleGet)]{\seqi{\Gam_{p} \sm x}{\get{l}{x}{p}}{\comptype{\conj{(l : \Gam_{p}(x))} \splus \stype}{\ctype}}{(b_p,1+m_p,d_p)}}
      \end{prooftree} \]
      $\Phi$ must be of the following form:
        \[ \begin{prooftree}
        \hypo{\Phi_0}
          \hypo{\Phi_{s} \tr \seqi{\Del}{s}{\conj{(l : \Gam_{p}(x))} \splus  \stype}{(b_s,m_s,d_s)}}
          \infer2[(\ruleConf)]{\seqi{(\Gam_{p} \sm x) + \Del}{(\get{l}{x}{p}, s)}{\kap}{(b_p+b_s,1+m_p+m_s,d_p+d_s)}}
        \end{prooftree} \] 
        where $\Gam = (\Gam_{p} \sm x) + \Del$ is tight, $b = b_p + b_s$, $m = 1+ m_p + m_s$, and  $d = d_p + d_s$. Since $\Phi_{s} \tr \seqi{\Del}{s}{\conj{(l : \Gam_{p}(x))} \splus \stype}{(b_s,m_s,d_s)}$, then~\cref{lem:split-values-stores}.\ref{lem:split-state} gives $s \equivstate \upd{l}{v_0}{s'_0}$, but we necessarily have $v_0 = v$ and $s'_0 = s'$. Moreover, the lemma also gives $\Phi_v \tr \seqi{\Del_v}{v}{\Gam_{p}(x) \sqcup \stype(l)}{(b_v,m_v,d_v)}$ and $\Phi_{s'} \tr \seqi{\Del_{s'}}{s'}{\stype'}{(b_{s'},m_{s'},d_{s'})}$, where $\conj{(l : \Gam_{p}(x))} \splus \stype = \conj{(l : \Gam_{p}(x) \sqcup \stype(l))};\stype'$, $\Del = \Del_v + \Del_{s'}$, $b_s=b_v + b_{s'}$, $m_s=m_v + m_{s'}$, and $d_s=d_v + d_{s'}$. Thus, by~\cref{lem:split-values-stores}.\ref{lem:com-split-values} there exist $\Phi^1_v \tr \seqi{\Del^1_v}{v}{\Gam_{p}(x)}{(b^1_v,m^1_v,d^1_v)}$ and $\Phi^2_v \tr \seqi{\Del^2_v}{v}{\stype(l)}{(b^2_v,m^2_v,d^2_v)}$, such that $\Del_v = \Del^1_v + \Del^2_v$, $b_v = b^1_v+b^2_v$, $m_v = m^1_v+m^2_v$, and $d_v = d^1_v+d^2_v$. From $\Phi_{p} \tr \seqi{\Gam_{p}}{p}{\comptype{\stype}{\ctype}}{(b_p,m_p,d_p)}$ and $\Phi^1_v \tr \seqi{\Del^1_v}{v}{\Gam_{p}(x)}{(b^1_v,m^1_v,d^1_v)}$, we obtain $\Phi_{p \subs{x}{v}} \tr \seqi{(\Gam_{p} \sm x) +\Del^1_v}{p\subs{x}{v}}{\comptype{\stype}{\ctype}}{(b_p+b^1_v,m_p+m^1_v,d_p+d^1_v)}$, by~\cref{lem:comp-subs-antisubs}.\ref{lem:comp-subs}. We now construct an alternative type derivation for $s$ of the form:
        \[ \begin{prooftree}
            \hypo{\Phi^2_v \tr  \seqi{\Del^2_v}{v}{\stype(l)}{(b^2_v,m^2_v,d^2_v)}}
            \hypo{\Phi_{s'} \tr \seqi{\Del_{s'}}{s'}{\stype'}{(b_{s'},m_{s'},d_{s'})}}
            \infer2[(\ruleUpd)]{\seqi{\Del^2_v+ \Del_{s'}}{\upd{l}{v}{s'}}{\conj{(l:\stype(l))};\stype'}{(b^2_v+b_{s'},m^2_v+m_{s'},d^2_v+d_{s'})}}
        \end{prooftree} \]
        Let $q = s= \upd{l}{v}{s'}$ and let $\Phi_q$ be this new derivation above. Notice also that $\stype = \conj{(l:\stype(l))}; \stype'$. Then we can construct $\Phi'$ as follows:
        \[ \begin{prooftree}
            \hypo{\Phi_{p\subs{x}{v}}}
            \hypo{\Phi_q}
            \infer2[(\ruleConf)]{\seqi{(\Gam_{p} \sm x) + \Del^1_v + \Del^2_v + \Del_{s'}}{(p \subs{x}{v}, s)}{\kap}{(b,m,d)}}
        \end{prooftree} \]
        Notice that the type environment of the conclusion is $(\Gam_{p} \sm x) + \Del^1_v + \Del^2_v + \Del_{s'} = (\Gam_{p} \sm x) + \Del_v + \Del_{s'} = (\Gam_{p} \sm x) + \Del = \Gam $, and the counters are as expected.
        \item Case $(t,s) = (\set{l}{v}{p},s) \ra (p, \upd{l}{v}{s}) = (u,q)$. Let $\Phi_0$ be the following derivation:
        \[ \begin{prooftree}
          \hypo{\Phi_{v} \tr \seqi{\Gam_v}{v}{\M}{(b_v,m_v,d_v)}}
            \hypo{\Phi_{p} \tr \seqi{\Gam_{p}}{p}{\comptype{\conj{(l : \M)}; \stype}{\ctype}}{(b_p,m_p,d_p)}}
            \infer2[(\ruleSet)]{\seqi{\Gam_v + \Gam_{p}}{\set{l}{v}{p}}{\comptype{\stype}{\ctype}}{(b_v+b_p,1+m_v+m_p,s_v+s_p)}}
        \end{prooftree} \]
        $\Phi$ must be of the following form:
        \[ \begin{prooftree}
          \hypo{\Phi_0}    
            \hypo{\Phi_{s} \tr \seqi{\Gam_{s}}{s}{\stype}{(b_s,m_s,d_s)}}
            \infer2[(\ruleConf)]{\seqi{(\Gam_v + \Gam_{p}) + \Gam_{s}}{(\set{l}{v}{p}, s)}{\kap}{(b_v+b_p+b_s,1+m_v+m_p+m_s,d_v+d_p+d_s)}}
        \end{prooftree} \]
        where  $\Gam = (\Gam_v + \Gam_{p}) + \Gam_{s}$ is tight, $b = b_v+b_p+b_s$, $m=1+m_v+m_p+m_s$ and $d=d_v+d_p+d_s$. Therefore, we can build $\Phi_{\upd{l}{v}{s}}$ as follows:
        \[ \begin{prooftree}
          \hypo{\Phi_{v} \tr \seqi{\Gam_v}{v}{\M}{(b_v,m_v,d_v)}}
          \hypo{\Phi_{s} \tr \seqi{\Gam_{s}}{s}{\stype}{(b_s,m_s,d_s)}}
          \infer2[(\ruleUpd)]{\seqi{\Gam_v + \Gam_{s}}{\upd{l}{v}{s}}{\conj{(l : \M)}; \stype}{(b_v+b_s,m_v+m_s,d_v+d_s)}}
        \end{prooftree} \]
        Assume And we can build $\Phi'$ as follows:
        \[ \begin{prooftree}
            \hypo{\Phi_{p} \tr \seqi{\Gam_{p}}{p}{\comptype{\conj{(l : \M)}; \stype}{\ctype}}{(b_p,m_p,d_p)}}
            \hypo{\Phi_{\upd{l}{v}{s}}}
            \infer2[(\ruleConf)]{\seqi{\Gam_{p} + (\Gam_v + \Gam_{s})}{(p, \upd{l}{v}{s})}{\kap}{(b_v+b_v+b_s,m_v+m_v+m_s,d_v+d_v+d_s)}}
        \end{prooftree} \]
        Notice that the type environment of the conclusion is $\Gam_{p} + (\Gam_v + \Gam_{s}) = \Gam$, and the counters are as expected.
    \end{itemize}
%\end{proof}

        \item %\begin{proof}
    We show a stronger statement of the form:

    Let $(t,s) \red[\gname] (u,q)$. If  $\Phi' \tr \seqi{\Gam}{(u,q)}{\ctype}{(b',m',d)}$, $\Gam$ is tight, and ($\ctype$ is tight or $\neg \isvalue{t}$), then $\Phi \tr \seqi{\Gam}{(t,s)}{\ctype}{(b,m,d)}$, where $\gname =\beta$ implies $b' = b - 1$ and $m' = m$, while $\gname \in \{\getname, \setname\}$ implies $b'=b$ and $m' = m - 1$.

    We proceed by induction on $(t, s) \red (u,q)$: 
    \begin{itemize}
        \item Case $(t,s) = ((\lam x.p) v,s) \redbeta (p \subs{x}{v}, s) = (u,q)$. Then $\Phi'$ must be of the following form:
        \[ \begin{prooftree}
            \hypo{\Phi_{p \subs{x}{v}} \tr \seqi{\Gam_{p \subs{x}{v}}}{p \subs{x}{v}}{\comptype{\stype}{\ctype}}{(b'',m'',d'')}}
            \hypo{\Phi_s \tr \seqi{\Gam_s}{s}{\stype}{(b_s,m_s,d_s)}}
            \infer2[(\ruleConf)]{\seqi{\Gam_{p \subs{x}{v}} + \Gam_s}{(p \subs{x}{v}, s)}{\ctype}{(b''+b_s,m''+m_s,d''+d_s)}}
        \end{prooftree} \]
        such that $\Gam = \Gam_{p \subs{x}{v}} + \Gam_s$, $b' = b''+b_s$, $m' = m''+m_s$, and $d' = d''+d_s$. By~\cref{lem:comp-subs-antisubs}.\ref{lem:comp-antisubs}, there exist $\Phi_p \tr \seqi{\Gam_p; x : \M}{p}{\comptype{\stype}{\ctype}}{(b_p,m_p,d_p)}$ and $\Phi_{v} \tr \seqi{\Gam_v}{v}{\M}{(b_v,m_v,d_v)}$, such that $\Gam_{p \subs{x}{v}} = \Gam_p + \Gam_v$, $b'' = b_p+b_v$, $m'' = m_p+m_v$, and $d'' = d_p + d_v$. We can build $\Phi$ as follows:
        \[ \begin{prooftree}
            \hypo{\Phi_p \tr \seqi{\Gam_p; x : \M}{p}{\comptype{\stype}{\ctype}}{(b_p,m_p,d_p)}}
            \infer1[(\ruleLam)]{\seqi{\Gam_p}{\lam x.p}{\M \ta (\comptype{\stype}{\ctype})}{(b_p,m_p,d_p)}}
            \hypo{\Phi_{v} \tr \seqi{\Gam_v}{v}{\M}{(b_v,m_v,d_v)}}
            \infer1[(\ruleLift)]{\seqi{\Gam_v}{v}{\tcomptype{\stype}{\M}{\stype}}{(b_v,m_v,d_v)}}
            \infer2[(\ruleApp)]{\seqi{\Gam_p + \Gam_v}{(\lam x.p)v}{\comptype{\stype}{\ctype}}{(1+b_p+b_v,m_p+m_v,d_p+d_v)}}
            \hypo{\Phi_s}
            \infer2[(\ruleConf)]{\seqi{(\Gam_p + \Gam_v) + \Gam_s}{((\lam x.t')v, s)}{\ctype}{(1+b_p+b_v+b_s,m_p+m_v+m_s,d_p+d_v+d_s)}}
        \end{prooftree} \]
        such that $b = 1+b_p+b_v+b_s$, $m = m_p+m_v+m_s$, and $d = d_p+d_v+d_s$. And we can conclude with $\Gam = \Gam_{p \subs{x}{v}} + \Gam_s = (\Gam_p + \Gam_v) + \Gam_s$, $b' = b'' + b_s = b_p + b_v + b_s = (1 + b_p + b_v + b_s) - 1 = b - 1$, $m' = m'' + m_s = (m_p + m_v) + m_s = m$, and $d' = d'' + d_s = (d_p + d_v) + d_s = d$.
        \item Case $(t,s) = (vp,s) \ra (vp',q) = (u,q)$, such that $(p,s) \ra (p',q)$. Then we have three cases for the type derivation $\Phi_{p'}$ of $p'$ inside $\Phi'$:
        \begin{itemize}
            \item Case $\Phi_{vp'}$ ends with ($\ruleApp$). Let $\Phi_0$ be the following derivation:
            \[ \begin{prooftree}
                \hypo{\Phi_{v} \tr \seqi{\Gam_v}{v}{\M \ta \comptype{\stype'}{\ctype}}{(b_v,m_v,d_v)}}
                    \hypo{\Phi_{p'} \tr \seqi{\Gam_{p'}}{p'}{\tcomptype{\stype}{\M}{\stype'}}{(b'',m'',d'')}}
                    \infer2[(\ruleApp)]{\seqi{\Gam_v + \Gam_{p'}}{v p'}{\comptype{\stype}{\ctype}}{(1+b_v+b'', m_v+m'', d_v+d'')}}
            \end{prooftree} \]
            $\Phi'$ must be of the following form: 
            \[ \begin{prooftree}
                \hypo{\Phi_0}
                \hypo{\Phi_q \tr \seqi{\Gam_q}{q}{\stype}{(b_q,m_q,d_q)}}
                \infer2[(\ruleConf)]{\seqi{(\Gam_v + \Gam_{p'}) + \Gam_q}{(v p', q)}{\ctype}{(1+b_v+b''+b_q,m_v+m''+m_q,d_v+d''+d_q)}}
            \end{prooftree} \]
            such that $\Gam = (\Gam_v + \Gam_{p'}) + \Gam_q$ tight, $b' = 1+b_v+b''+b_q$, $m' = m_v+m''+m_q$, and $d' = d_v+d''+d_q$. So we can build $\Phi_{(p',q)}$ as follows:
            \[ \begin{prooftree}
                \hypo{\Phi_{p'} \tr \seqi{\Gam_{p'}}{p'}{\tcomptype{\stype}{\M}{\stype'}}{(b'',m'',d'')}}
                \hypo{\Phi_q \tr \seqi{\Gam_q}{q}{\stype}{(b_q,m_q,d_q)}}
                \infer2[(\ruleConf)]{\seqi{\Gam_{p'} + \Gam_q}{(p', q)}{\conftype{\M}{\stype'}}{(b''+b_q,m''+m_q,d''+d_q)}}
            \end{prooftree} \]
            Since $\Gam$ is tight, then $\Gam_{p'} + \Gam_q$ is tight. Moreover, $(p, s) \red (p',q)$ implies $\neg\isvalue{p}$. Then we can apply the \ih, and thus there exists a derivation for $(p,s)$ that must be of the following form:
            \[ \begin{prooftree}
                \hypo{\Phi_p \tr \seqi{\Gam_p}{p}{\tcomptype{\stype''}{\M}{\stype'}}{(b_p,m_p,d_p)}}
                \hypo{\Phi_s \tr \seqi{\Gam_s}{s}{\stype''}{(b_s,m_s,d_s)}}
                \infer2[(\ruleConf)]{\seqi{\Gam_p + \Gam_s}{(p, s)}{\conftype{\M}{\stype'}}{(b_p+b_s,m_p+m_s,d_p+d_s)}}
            \end{prooftree} \]
            where $\Gam_p + \Gam_s = \Gam_{p'} + \Gam_q$ is tight, and either (1) $b''+b_q = b_p+b_s-1$, $m''+m_q=m_p+m_s$, and $d''+d_q = d_p+d_s$, or (2) $b''+b_q = b_p+b_s$, $m''+m_q=m_p+m_s-1$, and $d''+d_q = d_p+d_s$. So, we can build $\Phi$ as follows:
            \[ \begin{prooftree}
                \hypo{\Phi_{v} \tr \seqi{\Gam_v}{v}{\M \ta (\comptype{\stype'}{\ctype})}{(b_v,m_v,d_v)}}
                \hypo{\Phi_p \tr \seqi{\Gam_p}{p}{\tcomptype{\stype''}{\M}{\stype'}}{(b_p,m_p,d_p)}}
                \infer2[(\ruleApp)]{\seqi{\Gam_v + \Gam_p}{vp}{\comptype{\stype''}{\ctype}}{(1+b_v+b_p,m_v+m_p,d_v+d_p)}}
                \hypo{\Phi_s}
                \infer2[(\ruleConf)]{\seqi{(\Gam_v + \Gam_p) + \Gam_s}{(vp, s)}{\kappa}{(1 + b_v + b_p+b_s,m_v+m_p+m_s,d_v+d_p+d_s)}}
            \end{prooftree} \]
            where $\Gam_v + \Gam_p + \Gam_s = \Gam_v + \Gam_{p'} + \Gam_q = \Gam$, $b = 1+b_v+b_p+b_s$, $m = m_v+m_p+m_s$, and $d = d_v+d_p+d_s$. We can conclude since:
            \begin{itemize}
                \item Case (1): $b' = 1 + b_v + b'' + b_q = 1 + b_v + b_p + b_s - 1 = b -1$, and the other counters are easy to check;
                \item Case (2): $m' = m_v + m'' + m_q = m_v + m_p + m_s - 1 = m - 1$, and the other counters are easy to check.
            \end{itemize}
            \item Case $\Phi_{vp'}$ ends with (\ruleAppPOne) or (\ruleAppPTwo). These two cases are very similar to the previous case.
        \end{itemize}
        \item Case $(t,s) = (\get{l}{x}{p},s) \ra (p \subs{x}{v},s) = (u,q)$, such that $s \equivstate \upd{l}{v}{s'}$. Let $\Phi_0$ be the following derivation:
        \[ \begin{prooftree}
            \hypo{\Phi^2_v \tr \seqi{\Gam^2_v}{v}{\M_2}{(b^2_v,m^2_v,d^2_v)}}
            \hypo{\Phi_{s'} \tr \seqi{\Gam_{s'}}{s'}{\stype}{(b_{s'},m_{s'},d_{s'})}}
            \infer2[(\ruleUpd)]{\seqi{\Gam^2_v + \Gam_{s'}}{\upd{l}{v}{s'}}{\conj{(l : \M_2)}; \stype}{(b^2_v+b_{s'},m^2_v+m_{s'},d^2_v+d_{s'})}}
        \end{prooftree} \]
        Then $\Phi'$ must be of the following form:
        \[ \begin{prooftree}
            \hypo{\Phi_{p \subs{x}{v}} \tr \seqi{\Gam_{p \subs{x}{v}}}{p \subs{x}{v}}{\comptype{\conj{(l : \M)}; \stype}{\ctype}}{(b'',m'',d'')}}
            \hypo{\Phi_0}
            \infer2[(\ruleConf)]{\seqi{\Gam_{p \subs{x}{v}} + (\Gam^2_v + \Gam_{s'})}{(p \subs{x}{v}, \upd{l}{v}{s'})}{\ctype}{(b''+b^2_v+b_{s'},m''+m^2_v+m_{s'},d''+d^2_v+d_{s'})}}
        \end{prooftree} \]
        such that $\Gam = \Gam_{p \subs{x}{v}} + (\Gam^2_v + \Gam_{s'})$, $b' = b'' + b^2_v + b_{s'}$, $m' = m'' + b^2_v + b_{s'}$, and $d' = d'' +d^2_v+d_{s'}$. By~\cref{lem:comp-subs-antisubs}.\ref{lem:comp-antisubs}, there exist $\Phi_p \tr \seqi{\Gam_p; x : \M_1}{p}{\comptype{\conj{(l : \M_2)}; \stype}{\ctype}}{(b_p,m_p,d_p)}$ and $\Phi^1_v \tr \seqi{\Gam^1_v}{v}{\M_1}{(b^1_v,m^1_v,d^1_v)}$, such that $\Gam_{p \subs{x}{v}} = \Gam_p + \Gam^1_v$, $b'' = b_p + b^1_v$, $m'' = m_p + m^1_v$, and $d'' = d_p + d^1_v$. Therefore, we can build $\Phi_{\get{l}{x}{p}}$ as follows:
        \[ \begin{prooftree}
            \hypo{\Phi_p \tr \seqi{\Gam_p; x : \M_1}{p}{\comptype{\conj{(l : \M_2)}; \stype}{\ctype}}{(b_p,m_p,d_p)}}
            \infer1[(\ruleGet)]{\seqi{\Gam_p}{\get{l}{x}{p}}{\comptype{\conj{(l : \M_1 \sqcup \M_2)}; \stype}{\ctype}}{(b_p,1+m_p,d_p)}}
        \end{prooftree} \]
        By~\cref{lem:comp-merge-values}, we have $\Phi_v \tr \seqi{\Gam^1_v + \Gam^2_v}{v}{\M_1 \sqcup \M_2}{(b^1_v+b^2_v,m^1_v+m^2_v,d^1_v+d^2_v)}$. Thus, we can build $\Phi_{\upd{l}{v}{s'}}$ as follows:
        \[ \begin{prooftree}
            \hypo{\Phi_v \tr \seqi{\Gam^1_v + \Gam^2_v}{v}{\M_1 \sqcup \M_2}{(b^1_v+b^2_v,m^1_v+m^2_v,d^1_v+d^2_v)}}
            \hypo{\Phi_{s'} \tr \seqi{\Gam_{s'}}{s'}{\stype}{(b_{s'},m_{s'},d_{s'})}}
            \infer2[(\ruleUpd)]{\seqi{(\Gam^1_v + \Gam^2_v) + \Gam_{s'}}{\upd{l}{v}{s'}}{\conj{(l : \M_1 \sqcup \M_2)}; \stype}{(b^1_v+b^2_v+b_{s'},m^1_v+m^2_v+m_{s'},d^1_v+d^2_v+d_{s'})}}
        \end{prooftree} \]
        Finally, we can build $\Phi$ as follows:
        \[ \begin{prooftree}
            \hypo{\Phi_{\get{l}{x}{p}}}
            \hypo{\Phi_{\upd{l}{v}{s'}}}
            \infer2[(\ruleConf)]{\seqi{\Gam_p + (\Gam^1_v + \Gam^2_v) + \Gam_{s'}}{(\get{l}{x}{p}, \upd{l}{v}{s'})}{\ctype}{(b_p+b^1_v+b^2_v+b_{s'},1+m_p+m^1_v+m^2_v+m_{s'},d_p+d^1_v+d^2_v+d_{s'})}}
        \end{prooftree} \]
        such that $b = b_p+b^1_v+b^2_v+b_{s'}$, $m = 1+m_p+m^1_v+m^2_v+m_{s'}$, and $d = d_p+d^1_v+d^2_v+d_{s'}$. And we can conclude with $\Gam = \Gam_{p \subs{x}{v}} + (\Gam^2_v + \Gam_{s'}) = \Gam_p + \Gam^1_v + \Gam^2_v + \Gam_{s'}$, $b' = b'' + b^2_v + b_{s'} = b_p + b^1_v + b^2_v + b_{s'} = b$, and $m' = m'' + m^2_v + m_{s'} = m_p + m^1_v + m^2_v + m_{s'} = (1 + m_p + m^1_v + m^2_v + m_{s'}) - 1 = m - 1$, $d' = d'' + d^2_v + d_{s'} = d_p + d^1_v + d^2_v + d_{s'} = d$.
        \item Case $(t,s) = (\set{l}{v}{p},s) \ra (p, \upd{l}{v}{s}) = (u,q)$. Let $\Phi_0$ be the following derivation:
        \[ \begin{prooftree}
            \hypo{\Phi_{v} \tr \seqi{\Gam_v}{v}{\M}{(b_v,m_v,d_v)}}
            \hypo{\Phi_{s} \tr \seqi{\Gam_s}{s}{\stype}{(b_s,m_s,d_s)}}
            \infer2[(\ruleUpd)]{\seqi{\Gam_v + \Gam_s}{\upd{l}{v}{s}}{\conj{(l : \M)}; \stype}{(b_v+b_s,m_v+m_s,d_v+d_s)}}
        \end{prooftree} \]
        $\Phi'$ must be of the following form:
        \[ \begin{prooftree}
            \hypo{\Phi_p \tr \seqi{\Gam_p}{p}{\comptype{\conj{(l : \M)}; \stype}{\ctype}}{(b_p,m_p,d_p)}}
            \hypo{\Phi_0}
            \infer2[(\ruleConf)]{\seqi{\Gam_p + (\Gam_v + \Gam_s)}{(p, \upd{l}{v}{s})}{\kap}{(b_p+b_v+b_s,m_p+m_v+m_s,d_p+d_v+d_s)}}
        \end{prooftree} \]
        such that $\Gam = \Gam_p + (\Gam_v + \Gam_s)$, $b' = b_p + b_v + b_s$, $m' = m_p + m_v + m_s$, and $d' = d_p + d_v + d_s$. Therefore, we can build $\Phi$ as follows:
        \[ \begin{prooftree}
            \hypo{\Phi_{v} \tr \seqi{\Gam_v}{v}{\M}{(b_v,m_v,d_v,)}}
            \hypo{\Phi_p \tr \seqi{\Gam_p}{p}{\comptype{\conj{(l : \M)}; \stype}{\ctype}}{(b_p,m_p,d_p)}}
            \infer2[(\ruleSet)]{\seqi{\Gam_v + \Gam_p}{\set{l}{v}{p}}{\comptype{\stype}{\ctype}}{(b_v+b_p,1+m_v+m_p,d_v+d_p)}}
            \hypo{\Phi_s}
            \infer2[(\ruleConf)]{\seqi{(\Gam_v + \Gam_p) + \Gam_s}{(\set{l}{v}{p}, s)}{\ctype}{(b_v+b_p+b_s,1+m_v+m_p+m_s,d_v+d_p+d_s)}}
        \end{prooftree} \]
        Notice that the type environment of the conclusion is $(\Gam_v + \Gam_p) + \Gam_s = \Gam$, and the counters are as expected.
    \end{itemize}
%\end{proof}
    \end{enumerate}
\end{proof}}

\compsoundness*

\maybehide{\begin{proof} \mbox{}
    \begin{enumerate}
        \item %\begin{proof}
    The proof follows by induction over $b+m$:
    \begin{itemize}
        \item Case $b+m = 0$. Then $b=m=0$, therefore $t \in \normal$, by point (1) of~\cref{lem:zero-counters}, and  $d = \size{t}$,  by point (2) of~\cref{lem:zero-counters}. Let $u = t$ and $q=s$, then  we can conclude since $\size{(u,q)} = \size{u} =\size{t} = d$.
        \item Case $b+m > 0$. Then $b>0$ or $m>0$, and in either case $t \not\in \normal$,  by~\cref{lem:zero-nfs}. Note that $(t,s)$ is not final because $t$ is unblocked by~\cref{prop:typed-unblock}. Therefore, by~\cref{prop:normal-iff-final} there exists $(t',s')$ such that $(t,s) \gsred (t',s')$. By~\cref{lem-exact-red-exp}.\ref{lem:subj-comp-red}, there exists $\Phi' \tr \seqi{\Gam}{(t',s')}{\ctype}{(b',m',d)}$, such that $b'+m'=b+m-1$. By the \ih, there exists $(u,q)$, such that $u\in \normal$, $(t',s') \gsrred^{(b',m')} (u,q)$ and $d = \size{(u,q)}$. So we can conclude with $(t,s) \gsred (t',s') \gsrred^{(b',m')} (u,q)$, which means that $(t,s) \drred^{(b,m)} (u,q)$, as expected.
    \end{itemize}
%\end{proof}

        \item %\begin{proof}
    By induction over $b + m$: \begin{itemize}
        \item Case $b + m = 0$. Then $b = m = 0$ and $(t,s) = (u,q)$. We can conclude by~\cref{lem:typestatesnfs}.\ref{lem:typ-states} and~\cref{lem:typestatesnfs}.\ref{lem:comp-typ-nfs}.
        \item Case $b + m > 0$. Then there exists $(t',s')$, such that $(t,s) \ra^{(1,0)} (t',s') \rra^{(b-1,m)} (u,q)$ or $(t,s) \ra^{(0,1)} (t',s') \rra^{(b,m-1)} (u,q)$. By the \ih, there exists $\Phi' \tr \seqi{\Gam}{(t',s')}{\kap}{(b',m',\size{(u,q)})}$ tight, such that $b' + m' = b + m - 1$. By~\cref{lem-exact-red-exp}.\ref{lem:comp-subj-exp}, we have $\Phi \tr \seqi{\Gam}{(t,s)}{\kap}{(b'',m'',\size{(u,q)})}$ tight, such that $b'' + m'' = 1+ b' + m'$. Therefore, $b'' + m'' = b + m$, since the fact that $b'' = b$, and $m'' = m$ can be easily checked by a simple case analysis.
    \end{itemize}
%\end{proof}
    \end{enumerate}
\end{proof}}




\section{Details on PatchVDVAE architecture}\label{details}
In this section, we provide additional details about the architecture of PatchVDVAE. Then, we present the choice of the hyperparameters used for the concurrent methods (presented  in section~\ref{sec:expe} of the main paper) .
\subsection{PatchVDVAE}
Figure~\ref{fig:patchvdvae_arch} provides a detailed overview of the structure of a PatchVDVAE network.
The architecture follows VDVAE model~\cite{child2021very},
except for the first top-down block, in which we replace the constant input by a latent variable sampled from a gaussian distribution.
 The architecture presented in figure~\ref{fig:patchvdvae_arch}  illustrates the structure of HVAE networks, but the number of blocks is different to the PatchVDVAE network used in our experiments.
Our PatchVDVAE top-down path is composed of $L=30$ top-down blocks of increasing resolution.
The image features are upsampled using an unpooling layer every $5$ blocks.
The first unpooling layer performs a $\times 4$ upsampling, and the following unpooling layers perform $\times 2$ upsampling.
The dimension of the filters is $256$ in all blocks. 
In order to save computations in the residual blocks, the $3\times3$ convolutions are applied on features of reduced channel dimension (divided by $4$). 
$1\times 1$ convolutions are applied before and after the $3\times3$ convolutions to respectively reduce and increase the number of channels.
The latent variables $\z_l$ are tensors of shape $12 \times H_l \times W_l$, where the resolution $H_l$, $W_l$ corresponds to the resolution of the corresponding top-down-block.
The bottom-up network structure is symmetric to the top-down network, with $5$ residual blocks for each scale, and pooling layers between each scale. 


\begin{figure}[!ht]
    \begin{subfigure}[b]{0.4\linewidth}
        \includegraphics[width=\linewidth]{figures_arxiv/top_down_block.pdf}
        \subcaption*{Top-down block}
    \end{subfigure}
    \begin{subfigure}[b]{0.5\linewidth}
        \includegraphics[width=0.8\linewidth]{figures_arxiv/full_network.pdf}
        \subcaption*{Full autoencoder architecture}
    \end{subfigure}
    \begin{subfigure}[b]{0.4\linewidth}
        \centering
        \includegraphics[scale=1]{figures_arxiv/residual_block.pdf}
        \subcaption*{Residual block}
    \end{subfigure}

    \caption{\label{fig:patchvdvae_arch}Structure of the PatchVDVAE architecture. For clarity, we omit the non-linearity after each convolution.}
\end{figure}

\subsection{Hyperparameters of compared methods}
\paragraph{Face image restoration.}
For ILO, we found that optimizing the first 5 layers of the generative network offered the best trade-off between image quality and consistency with the observation.
Hence, we optimize the 5 first layers for 100 iterations each. 
This choice is different from the official implementation, where they only optimize the 4 first layers for a lower number of iterations, trading restoration performance for speed.
For DPS, we set the scale hyper-parameter $\zeta'$ (described in subsection C.2 in\cite{chung2022diffusion})  to $\zeta'=1$ for the deblurring and super-resolution experiments reported in this paper.


\paragraph{Natural images restoration - Deblurring.}
For the three tested methods, we use the official implementation provided by the authors, along with the pretrained models.
For EPLL, we use the default parameters in the official implementation.

For GS-PnP, using the notation of the paper, we use the suggested hyperparameter $\lambda_\nu=0.1$ for the motion blur kernels %
and $\lambda_\nu=0.75$ for the Gaussian kernels.

For PnP-MMO, we use the denoiser trained on $\sigma_{den}=0.007$. 
On deblurring with $\sigma=2.55$ we use the default parameters in the implementation. 
for higher noise levels ($\sigma=7.65$; $\sigma=12.75$), and we  set the strength of the gradient step as $\gamma=\sigma_{den}/(2\sigma||h||)$, where $h$ corresponds to the blur kernel.

\paragraph{Natural images restoration- Inpainting.}
For EPLL, we use the default parameters provided in the authors matlab code.
For GS-PnP, after a grid-search, we chose to set $\lambda_\nu=1$ and $\sigma_{denoiser}=10$.


\section{Discussion on the conctractivity of HVAE}\label{sec:contractivity}

We showed in section~\ref{sec:convergence} that PnP-HVAE converges to a fixed point under the assumption that $\x \rightarrow \HVAE(\x, \tau)$ is contractive. If this condition is met, the sequence of $u_k$ defined by $u_{k+1} = HVAE(u_k, \tau)$ should converge to a fixed point.
Figure ~\ref{fig:fixed} presents the \textbf{evolution of a fixed point iteration} $u_{k+1} = HVAE(u_k, \tau)$. 
 The image is smoothened over the iterations, and finally converges to a piececewise constant image. We used patchVDVAE for this experiment.

\begin{figure}[h]
    \centering
   \includegraphics*[scale=1]{figures/point_fixe/figure_pf.pdf}\vspace{-8pt}
   \caption{
      Fixed-point iterations of  patchVDVAE for $\tau=0.99$.\label{fig:fixed}} 
\end{figure}


\section{Comparisons}\label{add}
In this section, we provide additional visual results on face images and natural images.
\subsection{Additional results on face image restoration}
We provide additional comparisons with the GAN-based ILO method on inpainting (figure~\ref{fig:inp_add}), $\times 4$ super-resolution (figure~\ref{fig:sr_add}) and deblurring (figure~\ref{fig:deb_add}). PnP-HVAE provides equally or more plausible glasses in the first column) inpaiting than ILO. For superresolution, ILO produces sharper but not realistics faces. This is an agreement with the scores presented in table~\ref{table:comp_ILO}). For deblurring, ILO  creates textures on faces that looks realistic (low LPIPS) but are less consistent with the observation (significantly lower PSNR and SSIM).
\begin{figure}
    \includegraphics[width=\linewidth]{figures_arxiv/demo_inpainting.pdf}
    \caption{\label{fig:inp_add}Inpainting}
\end{figure}

\begin{figure}
    \includegraphics[width=\linewidth]{figures_arxiv/demo_sr_sf4_s3.pdf}
    \caption{\label{fig:sr_add}$\times 4$ super-resolution, with kernel (a) from Figure~\ref{fig:kernels} and $\sigma=3$}
\end{figure}

\begin{figure}
    \includegraphics[width=\linewidth]{figures_arxiv/demo_deblurring_kernel_11_s8.pdf}
    \caption{\label{fig:deb_add}Deblurring, with kernel (d) from Figure~\ref{fig:kernels} and $\sigma=8$}
\end{figure}

\subsection{Additional results on natural images restoration}
We finally present additional results on natural images restoration.
All the PnP-HVAE images presented below were produced using our PatchVDVAE model.
We also provide visual comparisons with concurrent PnP methods and EPLL. For deblurring (figures~\ref{fig:deb1} and~\ref{fig:deb2}, PnP methods perform better than EPLL. Following quantitative results of figure~\ref{tab:deb}, for larger noise level, PnP-HVAE outperforms PnP-MMO and provides restoration close to GS-PnP.

For inpainting (figure~\ref{fig:demo_inp}), the hierarchical structure of PatchVDVAE leads to more plausible reconstructions,  and PnP-HVAE outperforms the compared methods. \vspace{0.05cm}

\begin{figure}
    \centering
    \begin{tabular}[]{c c c c}
        \includegraphics[scale=2]{figures_arxiv/kernels/4.png} &
        \includegraphics[scale=2]{figures_arxiv/kernels/7.png} &
        \includegraphics[scale=2]{figures_arxiv/kernels/9.png} &
        \includegraphics[scale=2]{figures_arxiv/kernels/11.png} \\
        (a) & (b) & (c) & (d) 
    \end{tabular}
    \caption{\label{fig:kernels}Kernels used for deblurring experiments, from~\cite{levin2009understanding}}
\end{figure}

\newlength{\fwidth}
\setlength{\fwidth}{0.9\linewidth}
\begin{figure}
    \begin{subfigure}[h]{\linewidth}
        \centering
    
        \includegraphics[width=\fwidth]{figures_arxiv/deb_s255_k4.pdf}
        \caption{kernel (a), $\sigma=2.55$}
    \end{subfigure}
    
    \begin{subfigure}[h]{\linewidth}
    \centering
    \includegraphics[width=\fwidth]{figures_arxiv/deb_s255_k9.pdf}
    \caption{kernel (c), $\sigma=2.55$}
    \end{subfigure}

    \begin{subfigure}[h]{\linewidth}
        \centering
        \includegraphics[width=\fwidth]{figures_arxiv/deb_s765_k4.pdf}
        \caption{kernel (a), $\sigma=7.65$}
    \end{subfigure}
    \caption{\label{fig:deb1}Deblurring results on BSD}
\end{figure}

\begin{figure}
    \begin{subfigure}[h]{\linewidth}
    \centering
    \includegraphics[width=\fwidth]{figures_arxiv/deb_s765_k9.pdf}
    \caption{kernel (d), $\sigma=7.65$}
    \end{subfigure}

    \begin{subfigure}[h]{\linewidth}
    \centering
    \includegraphics[width=\fwidth]{figures_arxiv/deb_s1275_k7.pdf}
    \caption{kernel (b), $\sigma=12.75$}
    \end{subfigure}

    \begin{subfigure}[h]{\linewidth}
    \centering
    \includegraphics[width=\fwidth]{figures_arxiv/deb_s1275_k11.pdf}
    \caption{kernel (d), $\sigma=12.75$}
    \end{subfigure}
    \caption{\label{fig:deb2}Deblurring results on BSD}
\end{figure}




\begin{figure}
    \centering
    
    \includegraphics{figures_arxiv/demo_inp_bsd3.pdf}
    \caption{\label{fig:demo_inp}Natural images inpainting}
\end{figure}


\end{document}