We provide additional comparisons with the GAN-based ILO method on inpainting (figure~\ref{fig:inp_add}), $\times 4$ super-resolution (figure~\ref{fig:sr_add}) and deblurring (figure~\ref{fig:deb_add}). PnP-HVAE provides equally or more plausible glasses in the first column) inpaiting than ILO. For superresolution, ILO produces sharper but not realistics faces. This is an agreement with the scores presented in table~\ref{table:comp_ILO}). For deblurring, ILO  creates textures on faces that looks realistic (low LPIPS) but are less consistent with the observation (significantly lower PSNR and SSIM).
\begin{figure}
    \includegraphics[width=\linewidth]{figures_arxiv/demo_inpainting.pdf}
    \caption{\label{fig:inp_add}Inpainting}
\end{figure}

\begin{figure}
    \includegraphics[width=\linewidth]{figures_arxiv/demo_sr_sf4_s3.pdf}
    \caption{\label{fig:sr_add}$\times 4$ super-resolution, with kernel (a) from Figure~\ref{fig:kernels} and $\sigma=3$}
\end{figure}

\begin{figure}
    \includegraphics[width=\linewidth]{figures_arxiv/demo_deblurring_kernel_11_s8.pdf}
    \caption{\label{fig:deb_add}Deblurring, with kernel (d) from Figure~\ref{fig:kernels} and $\sigma=8$}
\end{figure}