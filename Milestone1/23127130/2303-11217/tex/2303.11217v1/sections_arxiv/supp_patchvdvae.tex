\subsection{PatchVDVAE}
Figure~\ref{fig:patchvdvae_arch} provides a detailed overview of the structure of a PatchVDVAE network.
The architecture follows VDVAE model~\cite{child2021very},
except for the first top-down block, in which we replace the constant input by a latent variable sampled from a gaussian distribution.
 The architecture presented in figure~\ref{fig:patchvdvae_arch}  illustrates the structure of HVAE networks, but the number of blocks is different to the PatchVDVAE network used in our experiments.
Our PatchVDVAE top-down path is composed of $L=30$ top-down blocks of increasing resolution.
The image features are upsampled using an unpooling layer every $5$ blocks.
The first unpooling layer performs a $\times 4$ upsampling, and the following unpooling layers perform $\times 2$ upsampling.
The dimension of the filters is $256$ in all blocks. 
In order to save computations in the residual blocks, the $3\times3$ convolutions are applied on features of reduced channel dimension (divided by $4$). 
$1\times 1$ convolutions are applied before and after the $3\times3$ convolutions to respectively reduce and increase the number of channels.
The latent variables $\z_l$ are tensors of shape $12 \times H_l \times W_l$, where the resolution $H_l$, $W_l$ corresponds to the resolution of the corresponding top-down-block.
The bottom-up network structure is symmetric to the top-down network, with $5$ residual blocks for each scale, and pooling layers between each scale. 


\begin{figure}[!ht]
    \begin{subfigure}[b]{0.4\linewidth}
        \includegraphics[width=\linewidth]{figures_arxiv/top_down_block.pdf}
        \subcaption*{Top-down block}
    \end{subfigure}
    \begin{subfigure}[b]{0.5\linewidth}
        \includegraphics[width=0.8\linewidth]{figures_arxiv/full_network.pdf}
        \subcaption*{Full autoencoder architecture}
    \end{subfigure}
    \begin{subfigure}[b]{0.4\linewidth}
        \centering
        \includegraphics[scale=1]{figures_arxiv/residual_block.pdf}
        \subcaption*{Residual block}
    \end{subfigure}

    \caption{\label{fig:patchvdvae_arch}Structure of the PatchVDVAE architecture. For clarity, we omit the non-linearity after each convolution.}
\end{figure}