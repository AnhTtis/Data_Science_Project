
\subsection{Hyperparameters of compared methods}
\paragraph{Face image restoration.}
We use the default parameters of ILO, except for the number of iterations for each layer.
We observed that increasing the number of iterations from $(50, 50, 25, 25)$ to $(100, 100, 100, 100)$  improves the quality of the generated images, 
so we set the number of iterations per layers to $(100, 100, 100, 100)$.
We also tried to increase the number of optimized layer to $6$ instead of $4$.
We found that this  helps to improve the consistency with the observation, but it  also introduces severe artifacts.
Therefore, we keep the number of optimized layer to $4$.

\paragraph{Natural images restoration - Deblurring.}
For the three tested methods, we use the official implementation provided by the authors, along with the pretrained models.
For EPLL, we use the default parameters in the official implementation.

For GS-PnP, using the notation of the paper, we use the suggested hyperparameter $\lambda_\nu=0.1$ for the motion blur kernels 
and $\lambda_\nu=0.75$ for the gaussian kernels.

For PnP-MMO, we use the denoiser trained on $\sigma_{den}=0.007$. 
On deblurring with $\sigma=2.55$ we use the default parameters in the implementation. 
for higher noise levels ($\sigma=7.65$; $\sigma=12.75$), and we  set the strength of the gradient step as $\gamma=\sigma_{den}/(2\sigma||h||)$, where $h$ corresponds to the blur kernel.

\paragraph{Natural images restoration- Inpainting.}
For EPLL, we use the default parameters provided in the authors matlab code.
For GS-PnP, after a grid-search, we chose to set $\lambda_\nu=1$ and $\sigma_{denoiser}=10$.