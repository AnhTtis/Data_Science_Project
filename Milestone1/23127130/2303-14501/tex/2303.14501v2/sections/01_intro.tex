Networks (or graphs) can serve as efficient representations of real-world, ultra-complex systems and can be further classified into different categories. A prominent category is represented by undirected networks embedded in a Euclidean space constrained by geometry, called spatial networks~\cite{barthelemy2011spatial}. In this work, we are focusing on spatial networks, where a form of physical exchange or \emph{flow} can be used to describe characteristic functional properties of the underlying physical system. Examples include road networks, water bodies, and global exchange networks, but they can also be found in biology (\eg, vascular system, lymphatic system, and connectome). We will refer to such networks as \emph{flow-driven spatial networks} (see Fig.~\ref{fig:intro}).

\begin{figure}[t]
\centerline{\includegraphics[width=\linewidth]{imgs/intro.pdf}}
\caption{
Flow-driven spatial network $\mathcal{G}$ (right), representing vasculature (left). $\mathcal{G}$'s nodes are embedded in a Euclidean space and represent spatial positions specified by $x$-, $y$-, and $z$-coordinates.
}
\vspace{-0.6em}
\label{fig:intro}
\end{figure}

Predominantly, network representations of physical systems originate from imaging methodologies, such as nanometer-scale microscopy in biology or regional to continental scale satellite remote sensing for road networks. The generation of compact network representations from these images is a multi-stage and imperfect process (segmentation, skeletonization, and subsequent graph pruning), which introduces noise and artifacts. However, for many relevant downstream tasks operating on flow-driven spatial networks, such as blood flow modeling~\cite{schmid2021severity} - a crucial component of investigating neurovascular brain disorders - or traffic forecasting~\cite{jiang2022graph}, a flawless network representation is of utmost importance. In this context, the task of link prediction presents itself as a meaningful measure to identify absent connections and reduce local noise arising from the erroneous graph generation process~\cite{paetzold2021whole, hu2020ogb}. Link prediction algorithms tailored to flow-driven spatial networks, however, remain heavily under-explored.

Therefore, we bring a simplistic yet general definition of the principle of physical flow, characterized by a direction and magnitude, to link prediction in graph representation learning. Our \textit{hypothesis} is that for flow-driven spatial networks, link prediction algorithms should heavily benefit from considering known functional properties, such as the aforementioned \emph{physical flow}, which are defined by the structural properties of the network (\eg, bifurcation angles~\cite{schneider2012tissue}). To this end, we propose the \emph{Graph Attentive Vectors} (GAV) link prediction framework. GAV operates on \emph{vector embeddings} representative of the network's structural properties and updates them in a constrained manner, imitating simplified dynamics of physical flow in spatial networks (\eg, blood flow in the vascular system or traffic flow in road networks).
We summarize our core contribution as follows:
\begin{enumerate}
\itemsep0em 
    \item We propose an attentive, neighborhood-aware message passing layer, called GAV layer, which updates vector embeddings, mimicking the (change in) direction and magnitude of physical flow in spatial networks.
     
    \item We introduce a readout module that aggregates vector embeddings in a physically plausible way and thus facilitates the interpretability of results.
     
    \item We prove the above-mentioned hypothesis by demonstrating state-of-the-art performance across all metrics in extensive experiments on eight flow-driven spatial networks, including the Open Graph
    Benchmark's~\cite{hu2020ogb} ogbl-vessel benchmark\footnote{\url{https://ogb.stanford.edu/docs/linkprop}} (98.38 vs. 87.98 AUC).  
\end{enumerate}