\documentclass[10pt,twocolumn,letterpaper]{article}

%\usepackage{wacv}              % To produce the CAMERA-READY version
\usepackage[pagenumbers]{wacv}  % To force page numbers, e.g. for an arXiv version

% Include other packages here, before hyperref.
\usepackage[accsupp]{axessibility}  % Improves PDF readability for those with disabilities.
\usepackage{mathtools}
\usepackage{graphicx}
\usepackage{amsmath}
\usepackage{amssymb}
\usepackage{booktabs}

% TODO add custom/additional packages
\usepackage{multirow}
\usepackage{pifont}
\newcommand{\cmark}{\ding{51}}%
\newcommand{\xmark}{\ding{55}}%
\usepackage[table]{xcolor}
\usepackage{wrapfig}
\usepackage[pagebackref,breaklinks,colorlinks]{hyperref}


% Support for easy cross-referencing
\usepackage[capitalize]{cleveref}
\crefname{section}{Sec.}{Secs.}
\Crefname{section}{Section}{Sections}
\Crefname{table}{Table}{Tables}
\crefname{table}{Tab.}{Tabs.}


%%%%%%%%% PAPER ID  - PLEASE UPDATE
\def\wacvPaperID{1693} % *** Enter the WACV Paper ID here
\def\confName{WACV}
\def\confYear{2024}


\begin{document}

%%%%%%%%% TITLE - PLEASE UPDATE
\title{Link Prediction for Flow-Driven Spatial Networks}

\author{Bastian Wittmann\\
University of Zurich\\
{\tt\small bastian.wittmann@uzh.ch}
\and
Johannes C. Paetzold\\
Technical University of Munich\\
{\tt\small johannes.paetzold@tum.de}
\and
Chinmay Prabhakar\\
University of Zurich\\
{\tt\small chinmay.prabhakar@uzh.ch}
\and
Daniel Rueckert\\
Technical University of Munich\\
{\tt\small daniel.rueckert@tum.de}
\and
Bjoern Menze\\
University of Zurich\\
{\tt\small bjoern.menze@uzh.ch}
}
\maketitle

%%%%%%%%% ABSTRACT
\begin{abstract}


\begin{abstract}
% \vspace{-1em}
The diffusion-based generative models have achieved remarkable success in text-based image generation. However, since it contains enormous randomness in generation progress, it is still challenging to apply such models for real-world visual content editing, especially in videos. 
In this paper, we propose \texttt{FateZero}, a zero-shot text-based editing method on real-world videos without per-prompt training or use-specific mask. 
\RM{Specifically, different from a pipeline of two independent inversion and then generation stages, we find the intermediate attention maps during inversions store better structure and motion information. We thus reform them to temporally casual attention and replace them in the generation progress. To further reduce the unnecessary semantic leakage of source video and enhance the editing quality, we then remix the temporally casual attentions via the cross-attention features of the source prompt as the mask.}
To edit videos consistently, we propose several techniques based on the pre-trained models. Firstly, in contrast to the straightforward DDIM inversion technique, our approach captures intermediate attention maps during inversion, which effectively retain both structural and motion information. These maps are directly fused in the editing process rather than generated during denoising. To further minimize semantic leakage of the source video, we then fuse self-attentions with a blending mask obtained by cross-attention features from the source prompt. Furthermore, we have implemented a reform of the self-attention mechanism in denoising UNet by introducing spatial-temporal attention to ensure frame consistency.
Yet succinct, our method is the first one to show the ability of zero-shot text-driven video style and local attribute editing from the trained text-to-image model. We also have a better zero-shot shape-aware editing ability based on the text-to-video model~\cite{tuneavideo}. \RM{Besides video, our unified method also achieves state-of-the-art performance in zero-shot image editing.\chenyang{Need exp or remove the zero-shot image}} Extensive experiments demonstrate our superior temporal consistency and editing capability than previous works.
% The code will be released.
% \chenyang{emphasize: our observation at inversion time} \xiaodong{replacing the bold part to the actual pipeline: \textbf{Specifically, we work on replacing and mixing the attention maps between the inversion and generation since the self-attention map keeps the structure of the original natural image and the cross-attention is semantic-related, after remixing, we replace them in the corresponding generation steps for denoising.}}
% \footnote{Since there is no general video diffusion model is publicly available, we use one-shot video generation method~(Tune-A-Video~\cite{tuneavideo}) as the pretrained video diffusion model for zero-shot video editing\xiaodong{can be removed if we actually zero-shot on video}.}.
\end{abstract}
\end{abstract}

%%%%%%%%% BODY TEXT
\vspace{-0.4em}
\section{Introduction and Motivation}\label{sec:intro}
\section{Introduction}

\label{sec:intro}

% \textit{"Drawing and colour are not separate at all; in so far as you paint, you draw. The more the colour harmonizes, the more exact the drawing becomes."} - Paul Cezanne.

Art is a reflection of the figments of human imagination. 
While many are limited in their practical creative capabilities, by pushing the boundaries of digital media, new ways can be created for casual artists and experts alike to express their ideas. At the same time, current neural generative art takes away much of the control from humans. In this work, we attempt to take a step towards restoring some of that control, enabling neural networks to complement users and naturally extend their skills rather than taking hold over the generative process.

% \orr{TBD - make the opening colorful : 1. Add quote:  2. Elaborate: art is a rendering of figments of imaginations of humans. Most people are limited in their drawing capabilities, and by pushing the boundaries we allow new ways for casual artists and experts alike in expressing ideas. At the same time, neural generative art takes a lot of the control away. Here, we want to give back some of this control to humans, such that neural networks complement them and compensate their lack of skills, rather than replacing them.}

% The field of image synthesis has been significantly propelled by neural generative models, particularly by the latest text-to-image models that predominantly rely on large language-image models ~\cite{balaji2022eDiff-I, ramesh2022dalle, rombach2021highresolution, imagen2022saharia}. These models have revolutionized the field of computer vision as they can produce astonishing visual outcomes from text prompts only.

The field of image synthesis has been significantly propelled by neural generative models, particularly by the latest text-to-image models that predominantly rely on large language-image models ~\cite{balaji2022eDiff-I, ramesh2022dalle, rombach2021highresolution, imagen2022saharia}. These models have revolutionized the field of computer vision, as they can produce astonishing visual outcomes from text prompts alone.

The ability of text-to-image models has sparked a wave of editing methods that utilize these models. Many of these techniques rely on prompt editing ~\cite{ fu2022shapecrafter, hertz2022prompt, kawar2022imagic,lin2022magic3d,mokady2022null, poole2022dreamfusion}. Nevertheless, simplifying the interface to text alone means users lack the necessary level of granularity to produce their exact desired outcomes.
% which is} insufficient for effectively editing local content. 
% editing and manipulating visual content, as users lack the necessary level of control to achieve their desired outcomes
Sketch-guided editing, on the other hand, provides intuitive control that aligns with user's conventional drawing and painting skills. By incorporating user-guided sketches into text-to-image models, powerful editing systems can be created, offering a high degree of flexibility and fine-grained control for manipulating visual content~\cite{zhang2023controlnet, voynov2022sketch}.

Although sketch-guided and text-driven methods have proven successful in generating and manipulating 2D images \cite{meng2022sdedit, voynov2022sketch, cheng2023wacv}, it immediately raises the intriguing question of whether a similar approach could be developed to edit 3D shapes. 
Since direct text-to-3D models require an abundance of data to scale, state-of-the-art 3D generative models, such as DreamFusion~\cite{poole2022dreamfusion} and Magic3D~\cite{lin2022magic3d}, which build on the capabilities of text-to-image models, may be considered as an alternative.
% Due to the difficulty of scaling general direct text-to-3D models, incorporating conditions into a text-to-3D model is not straightforward. Thus, state of the art 3D generative models, such as DreamFusion~\cite{poole2022dreamfusion} \orrc{and Magic3D~\cite{lin2022magic3d}}, which build on the capabilities of text-to-image models, may be considered as an alternative.
However, maintaining control via conditioning with such models remains a challenging task, as these generative pipelines optimize a Neural Radiance Field (NeRF) \cite{mildenhall2020nerf} by amortizing gradients from a multitude of 2D views. In particular, providing consistent sketches across all possible views presents a hurdle for users. Instead, a plausible user interface should act with guidance from as few views as possible, e.g. up to two or three.


In this paper, we present \textbf{SKED}, a \textbf{SK}etch-guided 3D \textbf{ED}iting technique. Our method acts on reconstructed or generated NeRF models. We assume a text prompt and a minimum of two sketches as input and provide output edits over the neural field faithful to the input conditions.
Meeting all input requirements can be challenging as the text prompt may not match the sketch's semantics, and sketch views may lack coherence.
To undertake this complex task, we conceptually break it down into two subtasks that are easier to handle: one that depends on pure geometric reasoning and the other that exploits the rich semantic knowledge of the generative model. These two subtasks work together, with the former providing a coarse estimate of location and boundary, and the latter adding and refining geometric and texture details through fine-grained operations.


Our experiments highlight the effectiveness of our approach for editing various pretrained NeRF instances. We introduce assorted accessories, objects, and artifacts, which are generated and blended into the original neural field seamlessly.
Finally, we validate our method through quantitative evaluations and ablation studies to assert the contribution of individual components in our method. 
% By presenting examples in the paper, we illustrate that our method can generate realistic 3D artifacts with accurate texture and geometry using only a few basic sketches.



% Due to the absence of a direct text-to-3D model, incorporating conditions into a text-to-3D model is not straightforward. Thus, 3D generative models, such as DreamFusion~\cite{poole2022dreamfusion}, which build on the capabilities of text-to-image models, may be considered as an alternative.
% However, this is a challenging task since DreamFusion generates a NeRF by integrating many different 2D views. It is very hard to provide consistent sketches across all possible views. The challenge is to use sketches as a guide on only a few views (e.g., two or three) and generate 3D edit of the existing NeRF that is subject to being edited. 

% In this paper, we present \textbf{SKED}, a \textbf{SK}etch-guided 3D \textbf{ED}iting technique, that takes as input a text prompt and a few (two or more) sketches and edits a 3D given object represented as a NeRF in a geometrically plausible and controlled way. 
% We acknowledge the difficulty of this task, as there are no existing text-to-3D generative models available for manipulating the geometry of the existing object based on a text prompt. 
% To undertake this complex task, we conceptually break it down into two simpler subtasks that are easier to handle: one that depends on pure geometric reasoning and the other that exploits the rich semantic knowledge of generative model. These two subtasks work together, with the former providing a coarse estimate of location and the latter adding and refining geometric and texture details through fine-grained operations.

% Our experiments showcase the effectiveness of our approach in performing sketch-guided text-based edits on different base nerfs by introducing various accessories, objects, and artifacts. We also conduct ablation studies and experiments to evaluate the performance of individual components in our method. By presenting examples in the paper, we illustrate that our method can generate realistic 3D artifacts with accurate texture and geometry using only a few basic sketches.

%\dcc{Add here the traditional paragraph that tell about what we achieved and evaluated}

\section{Related Works}
This section discusses previous work on link prediction algorithms and message-passing layers.
Particular emphasis is placed on methods featured in our experiments.

\subsection{Link Prediction}
Link prediction algorithms are applied in various fields, such as social network analysis~\cite{murata2007link, liben2003link, daud2020applications}, bioinformatics~\cite{ lei2013novel, stanfield2017drug, kang2022lr}, recommender systems~\cite{ai2019link, talasu2017link, huang2005link, zhang2010solving}, and supply chain improvement~\cite{lu2020discovering}.
Broadly speaking, different link prediction algorithms try to estimate link existence between two nodes either via heuristic or learned methods. 

Heuristic algorithms employ predefined heuristics to encode the similarity between nodes. Some prominent candidates
are represented by common neighbors, resource allocation~\cite{zhou2009predicting}, preferential attachment~\cite{barabasi1999emergence}, Adamic-Adar~\cite{adamic2003friends}, Jaccard~\cite{jaccard1901etude}, Katz~\cite{katz1953new}, and average commute time~\cite{fouss2007random}. However, all heuristic link prediction algorithms suffer from the same underlying issue. They exploit predefined, simple heuristics, which can not be modified to account for different network types. \Eg, common neighbors has been developed for social networks and hence yields underwhelming results when applied to molecular graphs.

On the other hand, learned algorithms do not rely on predesigned heuristics but rather learn a more complex, data-driven heuristic utilizing neural networks. Thus, learned algorithms can easily adapt to different network types while typically outperforming their heuristic counterparts. SEAL~\cite{zhang2018link, zhang2021labeling} represents a prominent, learned link prediction framework, defining link prediction as a subgraph-level classification task by training a binary GNN-based classifier to map from subgraph patterns to link existence. To this end, SEAL first extracts a local subgraph around the link of interest, which is subsequently forwarded to DGCNN~\cite{zhang2018end} for classification. Moreover, SEAL's authors introduce an additional node labeling technique, known as labeling trick~\cite{zhang2021labeling}, to enhance the expressiveness of node features obtained from GNNs. SIEG~\cite{sieg} builds upon SEAL and introduces, inspired by Graphormer~\cite{ying2021transformers}, a pairwise structural attention module between two nodes of interest to capture local structural information more effectively. This results in state-of-the-art performances and allows SIEG to simultaneously overcome Graphormer's issue of exploding computational complexity when applied to ultra-large graphs. SUREL+~\cite{yin2023surel+} introduces the use of node sets to represent subgraphs. To complement the loss of structural information, SUREL+ provides set samplers, structure encoders, and set neural encoders. SUREL+ outperforms the baseline SEAL on various link prediction benchmarks. It should be mentioned that Cai \etal~\cite{cai2021line} investigate the use of line graphs for link prediction. Please note that none of the above-mentioned methods are tailored to flow-driven spatial networks.

\subsection{Message-Passing Layers}
GNNs utilize the concept of message-passing to encode semantically rich features within network-structured data. Over time, multiple variations of message-passing layers have been proposed~\cite{xu2018powerful, defferrard2016convolutional, fey2018splinecnn, he2020lightgcn}. For instance, GCN's message-passing layer~\cite{kipf2016semi} weighs each incoming message with a fixed coefficient, the node degree, before aggregation. In contrast, GAT's message-passing layer~\cite{brody2021attentive} learns aggregation weights dynamically based on attention scores. GraphSAGE's message-passing layer~\cite{hamilton2017inductive} does not directly aggregate central node features with incoming messages. Instead, it distinguishes these two kinds of features and learns two different transformations, one on the central node and another on incoming messages. EdgeConv~\cite{wang2019dynamic} aggregates the feature difference between the central node and its neighbors combined with the central node's features. Thus, EdgeConv draws parallels to aggregating spatial vectors if the nodes embed spatial positions.
However, our proposed GAV layer differs significantly from EdgeConv, as we explicitly constrain the update of vector embeddings to imitate the simplified dynamics of physical flow in flow-driven spatial networks. Importantly, only a few works tried to adapt the message-passing paradigm to spatial networks~\cite{zhang2021representation, danel2020spatial}.

\section{The Graph Attentive Vectors Framework}
\begin{figure*}[t!]
\includegraphics[width=1.0\linewidth, trim={0 0.3cm 0 0.1cm}, clip]{figures/architecture/architecture.pdf}
\vspace{-15pt}
\caption{
\textbf{Point2Vec pre-training.}
Our model divides the input point cloud into %
point patches using farthest point sampling (FPS) and $k$-NN aggregation.
We obtain patch embeddings by applying a mini-PointNet\,\colorsquare{m_pointnet} to each point patch (\emph{right}).
The teacher Transformer encoder\,\colorsquare{m_green} infers a contextualized %
representation for all patch embeddings which, after normalization and averaging over the last $K$ Transformer layers, serve as training targets.
The student's input is a masked view on the input data, \ie we randomly mask out a ratio of patch embeddings and only pass the remaining embeddings into the student Transformer encoder\,\colorsquare{m_blue}.
After applying a shallow decoder\,\colorsquare{m_red} on the outputs of the student, padded with learned mask embeddings\,\protect\maskembedding{}, we train the student and decoder to predict the latent teacher representation of the patch embeddings.
\vspace{-10pt}
}
\label{fig:model}
\end{figure*}
\section{Method}

The aim of this work is to unlock the full potential of data2vec-like\,\cite{baevski2022data2vec} pre-training on point clouds by addressing point cloud specific challenges.
To achieve this, we first summarize the technical concepts of data2vec (\refsec{method_d2v}) and show how to learn rich representations on point clouds using data2vec pre-training (\refsec{method_d2v_pcl}).
Finally, we propose \name{}, which accounts for the point cloud specific limitations of data2vec (\refsec{method_p2v}).

\subsection{Data2vec}\label{sec:method_d2v}
Data2vec\,\cite{baevski2022data2vec} is designed to pre-train Transformer-based models, which involve a feature encoder that maps the input data to a sequence of embeddings.
These embeddings are subsequently passed to a standard Transformer encoder to generate the final latent representations.
During pre-training, two versions of the Transformer encoder are kept: a \emph{student} and a \emph{teacher}.
The teacher is a momentum encoder, \ie its parameters $\Delta$ track the student's parameters $\theta$ by being updated after each training step according to an exponential moving average (EMA) rule\,\cite{caron2021dino, baevski2022data2vec, grill2020BYOL, he2020moco}: $\Delta \leftarrow \tau \Delta + (1-\tau)\theta$,
where $\tau \in [0,1]$ is the EMA decay rate.
The teacher provides the training targets, which the student predicts given a corrupted version of the same input.

In a first step, the teacher encodes the uncorrupted input sequence.
The training targets are then constructed by averaging the outputs of the last $K$ blocks of the teacher, which are normalized beforehand to prevent a single block from dominating the sum.
Due to the self-attention layers, these targets are \emph{contextualized}, \ie they incorporate global information from the whole input sequence.
This is an important difference to other masked-prediction methods such as BERT\,\cite{devlin2018bert} and MAE\,\cite{he2022mae}, where the targets only comprise local information, \eg a word or an image patch. %

The student is given a masked version of the same input, where some of the embeddings in the input sequence are substituted by a special learned \emph{mask embedding}. %
The student's task is to predict the targets corresponding to the masked parts of the input.
The model is trained by optimizing a Smooth L1 loss on the regressed targets. %







\subsection{Data2vec for Point Clouds}\label{sec:method_d2v_pcl}

To apply data2vec to point clouds, we utilize the same underlying model as Point-BERT\,\cite{yu2021pointbert} and Point-MAE\,\cite{pang2022pointmae}.
This model is well suited for data2vec pre-training: it extracts a sequence of patch embeddings from the input point cloud and feeds it to a standard Transformer encoder.
For downstream tasks, we append a task-specific head to the Transformer encoder (\refsec{experiments}).
Next, we describe the point cloud embedding and the Transformer in detail and conclude with a summary of data2vec for point clouds.


\parag{Point Cloud Embedding.}
First, we sample $n$ center points from the input point cloud using farthest point sampling (FPS)\,\cite{qi2017pointnetplusplus}.
Grouping the center points' $k$-nearest neighbors ($k$-NN) in the point cloud yields $n$ contiguous \emph{point patches}, \ie sub-clouds of $k$ elements.
Next, we normalize the point patches by subtracting the corresponding center point from the patch's points.
This untangles the positional and the structural information.
To account for the permutation-invariant property of point clouds, we employ a mini-PointNet\,\cite{qi2016pointnet} (\reffig{model}, \emph{right}) that maps each normalized point patch to a \emph{patch embedding}.

The mini-PointNet involves the following steps:
First, we map each point of a patch to a feature vector using a shared MLP.
Then, we concatenate max-pooled features to each feature vector.
The resulting feature vectors are then passed through a second shared MLP and a final max-pooling layer to obtain the patch embedding.

\paragraph{Transformer Encoder.}
The central component of the model is a standard Transformer encoder.
The patch embeddings form the input sequence to the Transformer encoder.
Since the point patches are normalized, the patch embeddings carry no positional information;
therefore, a two-layer MLP maps each center point to a position embedding, which is then added to the corresponding patch embedding.
Due to the special importance of positional information in point clouds, the position embeddings are added again before each subsequent Transformer block to ensure that the positional information is incorporated at every step of the encoding process.

\paragraph{\emakefirstuc{\datavec{}}.}

To establish a baseline, we apply the unmodified data2vec approach to the previously described underlying model of Point-BERT and Point-MAE.
Going forward, we will refer to this approach as \datavec{}.


\subsection{\emakefirstuc{\name{}}}\label{sec:method_p2v}
In \reffig{model}, we present the complete pipeline of our \name{} model.
Directly applying data2vec to point cloud data without modifications is not optimal, as the position embeddings are also added to the mask embeddings, revealing the overall shape of the point cloud to the student.
As positions are the only features for point clouds, this makes the masking far less effective, as noted by Pang \etal \cite{pang2022pointmae} in the context of masked autoencoders.

To solve this issue, we adopt an approach inspired by MAE\,\cite{he2022mae}, where we only feed the non-masked embeddings to the student\,\colorsquare{m_blue}.
A separate decoder\,\colorsquare{m_red}, implemented as a shallow Transformer encoder, takes the output of the student and the previously held-back masked embeddings\,\maskembedding{} as input and predicts the training targets.
In contrast to \datavec{}, this approach does not suffer from leaking positional information from the masked-out point patches to the student.
Moreover, utilizing an MAE-inspired setup provides additional benefits:
First, the student is more computationally efficient, as it only needs to process the non-masked embeddings.
Second, the model's inputs during fine-tuning are more similar to those during pre-training because the inputs during pre-training are no longer dominated by masked embeddings which are absent during fine-tuning.
This likely makes the learned representations more transferable to downstream tasks.


\section{Experiments and Results}
\begin{table*}[t]
\centering
\scriptsize
\caption{Quantitative results achieved on the test sets. We report mean and standard deviation values on the ogbl-vessel benchmark based on ten different seeds. Please note that the ogbl-vessel benchmark's evaluation metric is AUC. Therefore, Hits@$k$ values of participating algorithms are not available.
GAV outperforms the previous state-of-the-art across all metrics and datasets.}
\label{tab:quantitative_results}
\begin{tabular}{p{50pt}|l|l|c|c|c|c} 
\toprule
Dataset & Model &$\#\ \text{Params}\downarrow$ &$\text{AUC}\uparrow$ (\%)& $\text{Hits@100}\uparrow$  (\%)& $\text{Hits@50}\uparrow$  (\%)& $\text{Hits@20}\uparrow  (\%)$\\ 
\midrule
\multirow{11}{*}{ogbl-vessel}
& GCN~\cite{kipf2016semi} & 396,289 & 43.53 ± 9.61 & - & - & - \\
& MLP & 1,037,577 & 47.94 ± 1.33 & - & - & - \\
& Adamic-Adar~\cite{adamic2003friends} &  \textbf{0} & 48.49 ± 0.00 & - & - & - \\
& GraphSAGE~\cite{hamilton2017inductive} & 396,289 & 49.89 ± 6.78 & - & - & - \\
& SAGE+JKNet~\cite{xu2018representation} & 273 & 50.01 ± 0.07 & - & - & - \\
& SGC~\cite{wu2019simplifying} & 897 & 50.09 ± 0.11 & - & - & - \\
& LRGA~\cite{puny2020global} & 26,577 & 54.15 ± 4.37 & - & - & - \\
& SEAL~\cite{zhang2021labeling} & 172,610 & 80.50 ± 0.21 & - & - & - \\
& SIEG~\cite{hu2020ogb} & 407,338 & 83.07 ± 0.44 & - & - & - \\
\cmidrule{2-7}
& SEAL (EdgeConv) & 49,346 & 97.53 ± 0.32 & 16.09 ± 10.48 & 9.37 ± 6.18 & 4.99 ± 4.24\\
& GAV (ours) & 8,184 & \textbf{98.38 ± 0.02} & \textbf{34.77 ± 0.94} & \textbf{28.02 ± 1.58} & \textbf{19.71 ± 2.31}\\
\midrule
\multirow{3}{*}{c57-tc-vessel}
& SEAL~\cite{zhang2021labeling} & 43,010 & 78.21 & 0.12 & 0.06 & 0.01\\
& SEAL (EdgeConv) & 49,346 & 97.23 & 16.71 & 10.39 & 5.01\\
& GAV (ours) & \textbf{8,184} & \textbf{98.24} & \textbf{33.26} & \textbf{26.89} & \textbf{21.32}\\
\midrule
\multirow{3}{*}{cd1-tc-vessel}
& SEAL~\cite{zhang2021labeling} & 43,010 & 83.60 & 0.27 & 0.16 & 0.06\\
& SEAL (EdgeConv) & 49,346 & 97.91 & 17.05 & 11.57 & 2.98\\
& GAV (ours) & \textbf{8,184} & \textbf{98.72} & \textbf{35.82} & \textbf{27.25} & \textbf{17.23}\\
\midrule
\multirow{3}{*}{c57-cc-vessel}
& SEAL~\cite{zhang2021labeling} & 43,010 & 83.75 & 0.65 & 0.44 & 0.24\\
& SEAL (EdgeConv) & 49,346 & 97.49 & 7.21 & 3.35 & 1.06\\
& GAV (ours) & \textbf{8,184} & \textbf{97.99} & \textbf{18.90} & \textbf{14.58} & \textbf{9.04}\\
\midrule
\multirow{3}{*}{belgium-road}
& SEAL~\cite{zhang2021labeling} & 43,010 & 86.73 & 1.25 & 0.68 & 0.30\\
& SEAL (EdgeConv) & 49,346 & 96.98 & 0.55 & 0.55 & 0.51\\
& GAV (ours) & \textbf{8,184} & \textbf{99.29} & \textbf{47.44} & \textbf{38.60} & \textbf{22.11}\\
\midrule
\multirow{3}{*}{italy-road}
& SEAL~\cite{zhang2021labeling} & 43,010 & 90.07 & 0.32 & 0.16 & 0.08\\
& SEAL (EdgeConv) & 49,346 & 90.24 & 0.26 & 0.17 & 0.07\\
& GAV (ours) & \textbf{8,184} & \textbf{99.41} & \textbf{28.49} & \textbf{20.08} & \textbf{11.99}\\
\midrule
\multirow{3}{*}{netherlands-road}
& SEAL~\cite{zhang2021labeling} & 43,010 & 84.19 & 0.00 & 0.00 & 0.00\\
& SEAL (EdgeConv) & 49,346 & 96.06 & 3.91 & 2.20 & 1.01\\
& GAV (ours) & \textbf{8,184} & \textbf{99.44} & \textbf{37.55} & \textbf{26.97} & \textbf{10.77}\\
\midrule
\multirow{3}{*}{luxembourg-road}
& SEAL~\cite{zhang2021labeling} & 43,010 & 89.79 & 11.39 & 6.15 & 3.12\\
& SEAL (EdgeConv) & 49,346 & 97.53 & 59.79 & 39.15 & 19.42\\
& GAV (ours) & \textbf{8,184} & \textbf{99.31} & \textbf{85.88} & \textbf{76.84} & \textbf{61.95}\\
\bottomrule
\end{tabular}
% \vspace{-0.5em}
\end{table*}

In this section, we demonstrate the performance of our proposed GAV framework on the ogbl-vessel benchmark~\cite{hu2020ogb} and on additional datasets sourced from publicly available flow-driven spatial networks.
We first elaborate on baseline algorithms and the experimental setup, followed by a detailed description of our used datasets. Finally, we introduce the evaluation metrics, report quantitative results, investigate our design choices by conducting detailed ablation studies, and discuss GAV's interpretability.

\subsection{Baselines and Experimental Setup}\label{ref:baselines}
To evaluate GAV properly, we experimented with different baseline algorithms. We ultimately settled for SEAL~\cite{zhang2018link, zhang2021labeling}, which has shown to deliver results on par with or superior to the state-of-the-art on multiple link prediction benchmarks.
% , such as ogbl-ppa~\cite{hu2020ogb}, ogbl-vessel~\cite{hu2020ogb}, ogbl-citation2~\cite{hu2020ogb}, and ogbl-collab~\cite{hu2020ogb}. 
Additionally, we propose a new \emph{secondary baseline} combining SEAL with the EdgeConv message-passing layer~\cite{wang2019dynamic}, following recent trends in graph-based object detection from point clouds~\cite{chai2021point, yang2022graph, wang2021object, yin2020lidar}. Equation~\ref{eq:edgeconv} describes EdgeConv's update function in detail, where $\phi_{\theta}$ represents a two-layer MLP.
\begin{equation} \label{eq:edgeconv}
\hat{n}_{i} = \frac{1}{|\mathcal{N}(n_i)|} \sum_{n_j \in \mathcal{N}(n_i)} \phi_{\theta}(n_{i} \mathbin\Vert n_{j} - n_{i})
\end{equation}
This provides us with an improved, highly competitive secondary baseline for link prediction on spatial networks. An empirical analysis varying SEAL's message-passing layer confirmed this decision. We additionally refined SEAL's parameters via a hyperparameter search.

GAV was trained using the Adam optimizer~\cite{kingma2014adam} with a learning rate of 0.001 and a batch size of 32 on a single Quadro RTX 8000 GPU until convergence.
An ablation study on the number of hops $h$ in the subgraph extraction module and the number of message-passing iterations $k$ indicates that setting both to one is sufficient (see Table \ref{tab:ablations_kh}).
In the GAV layer, the number of heads of the multi-head attention operation is set to 4, while $\phi^{(1)}_{\theta}$ represents a single linear layer with an output dimension of $d_{\text{message}} = 32$, and $\phi^{(2)}_{\theta}$ is given by a two-layer MLP with a hidden dimension of 64.
The GAV layer makes use of leaky ReLU non-linearities~\cite{maas2013rectifier} to increase gradient flow and simplify weight initialization. The readout module's learnable function $\phi^{(3)}_{\theta}$ is represented by a two-layer MLP with a hidden dimension of 128.
% We refrain from utilizing appropriate data augmentation techniques, such as random rotation and scaling, in the subnetwork extraction module since we observed no noticeable performance difference.
All hyperparameters were tuned on the validation set of the ogbl-vessel benchmark.

\subsection{Datasets}
We experiment with multiple 2D and 3D flow-driven spatial networks to demonstrate the generalizability of our approach (see Table~\ref{tab:datasets}). In total, we conduct experiments on eight networks, given by whole-brain vessel graphs of different mouse strains and road networks of various European countries. In this context, link prediction can be interpreted as predicting the probability of the existence of blood vessels and road segments.
%Both network types are characterized by a physical transportation process (physical flow). 

\paragraph{Whole-Brain Vessel Graphs}
Blood vessels represent fascinating structures forming complex networks that transport oxygen and nutrients throughout the human body. The vascular system is, therefore, intuitively represented as a flow-driven spatial network, where branching points of vessels typically represent nodes embedding $x$-, $y$-, and $z$-coordinates, while edges are defined as blood vessels running between branching points~\cite{paetzold2021whole}. We report results on the Open Graph Benchmark's ogbl-vessel benchmark~\cite{hu2020ogb}, which measures the performance of different link prediction algorithms with regard to whole-brain vessel graphs. The ogbl-vessel benchmark consists of millions of nodes and edges (see Table~\ref{tab:datasets}) and describes the murine brain vasculature all the way down to the microcapillary level.
However, we not only experiment with the ogbl-vessel benchmark but also source three additional whole-brain vessel graphs of different mouse strains acquired via different imaging methodologies~\cite{todorov2020machine, walchli2021hierarchical} (see Table~\ref{tab:datasets}, footnote).

% include four different whole-brain vessel graphs of different mouse strains consisting of millions of nodes and edges (see Table~\ref{tab:datasets}) describing the murine brain vasculature all the way down to the microcapillary level.
% In whole brain vessel graphs, branching points of vessels represent nodes embedding x, y, and z coordinates, while edges are defined as blood vessels connecting branching points~\cite{paetzold2021whole}. The whole brain vessel graphs have been acquired via different imaging methodologies~\cite{todorov2020machine, walchli2021hierarchical}.

\paragraph{Road Networks}
Further, we report results on diverse road networks representative of four European countries for a thorough evaluation of GAV's performance. To this end, we adopt publicly available road networks introduced in the DIMACS graph partitioning and clustering challenge~\cite{bader201110th}. These road networks correspond to the largest connected components of OpenStreetMap's~\cite{OpenStreetMap} road networks and are vastly different in size (\eg, luxembourg-road constitutes roughly 100,000 edges, whereas italy-road has more than 7,000,000). 
% Moreover, these road networks are heavily influenced by the countries' topography, e.g., Italy is dominated by many mountain ranges, and the Netherlands is entirely flat, leading to vastly different network shapes. 
In road networks, intersections and locations with strong curvature represent nodes in the form of $x$- and $y$-coordinates, while connecting roads represent edges. 
% Intersections are given by spatial locations in the form of $x$- and $y$-coordinates.

% The data originates from openstreetmap. For reproducibility, we use the archived data version from the DIMACS graph partitioning, and graph clustering challenge \cite{bader201110th}. The road networks are vastly different in size, e.g., Luxembourg constitutes ~100 000 edges, whereas Italy has more than 7 000 000, see Table \ref{tab:datasets}. Moreover, the countries road networks are heavily influenced by the countries' topography, e.g., Italy is dominated by many mountain ranges, and the Netherlands is entirely flat, leading to vastly different network shapes. Similar to the vessel graphs, intersections and locations with strong curvature represent nodes, and the connecting roads represent the edges. The only node features are the x,y coordinates. 


\paragraph{Preprocessing}
Link prediction datasets require positive (label 1) and negative links (label 0). Positive links correspond to existent edges in our datasets, whereas negative links represent artificially created, non-existent edges. As link prediction algorithms are commonly employed to improve the graph representation through the identification of absent connections and the reduction of local noise arising from graph generation, negative links should appear as authentic as possible.
% Therefore, sampling of negative edges requires a sophisticated sampling strategy, which also takes the nodes' spatial locations into account. 
In light of the absence of negative links in our sourced datasets, we prepare our sourced datasets in a manner that aligns with the ogbl-vessel benchmark. Following the ogbl-vessel benchmark, we sample negative links using a spatial sampling strategy. To be precise, we randomly connect nodes in close proximity, taking a maximum distance threshold of $\delta = \overline{e_{ij}} + 2 \sigma $ into account. Here, $\overline{e_{ij}}$ denotes the average edge length estimated over the entire graph $\mathcal{G}$ and $\sigma$ the standard deviation. 
The number of negative, sampled links corresponds to the number of positive, real links across all datasets. 
We finally split positive and negative links into training, validation, and test sets (split 80\%/10\%/10\%).

\subsection{Evaluation Metrics}\label{ref:eval_met}
To compare GAV to existing baseline algorithms, we report quantitative results based on the area under the receiver operating characteristic curve (AUC), following the obgl-vessel benchmark. The AUC metric indicates the performance of a classifier by plotting the true positive rate against the false positive rate at all possible classification thresholds. Therefore, AUC provides an aggregate performance measure indicating the classifier's ability to distinguish between positive and negative links.
%Statistically speaking, ROC-AUC scores should levitate between the values of 50\% and 100\%, where a ROC-AUC score of 50\% indicates a more-or-less random guess, while a perfect classifier achieves a ROC-AUC score of 100\%.

We introduce the evaluation metric Hits@$k$ as an additional, stricter performance measure. Hits@$k$ compares the classifier's prediction of every single positive link to a randomly sampled set of 100,000 negative links, resulting in a ranking among 100,001 links with respect to the probability of link existence. Based on this ranking, Hits@$k$ indicates the ratio of positive links ranked at the $k$-th place and above.
%In the context of this work, we evaluate Hits@$k$ at $k=100$, $k=50$, and $k=20$.


\subsection{Quantitative Results}
GAV demonstrates excellent, superior performances on the task of link prediction across all metrics and datasets, as can be observed in Table~\ref{tab:quantitative_results}.
We outperform the current state-of-the-art algorithm SIEG on the ogbl-vessel benchmark by \textbf{more than 18\%} (98.38 vs. 83.07 AUC) while requiring a significantly smaller amount of trainable parameters (8,184 vs. 407,338). However, GAV not only drastically outperforms the current state-of-the-art but also our introduced strong, secondary baseline, combining the SEAL framework with EdgeConv. The excellent performance and superiority of our GAV framework is even more pronounced when considering the strict evaluation metric of Hits@$k$.

Quantitative results reported in Table~\ref{tab:quantitative_results} additionally indicate the strong performance of our secondary baseline (see Section~\ref{ref:baselines}), surpassing previous state-of-the-art methods in AUC across all but one dataset, namely italy-road. 
It is of note that the luxembourg-road dataset's test set contains only 12,000 negative links. We, therefore, compare predictions of its positive links to 12,000 rather than 100,000 negative links (see Section~\ref{ref:eval_met}). This explains the comparatively strong Hits@$k$ performances on the luxembourg-road dataset.
% In an additional experiment, we train our GAV framework on one mouse brain, we then run inference on an entirely different specimen. We find an equally high performance on this unseen specimen (98.5 vs. 98.10 AUC). We conclude that our GAV framework is capable of learning the properties of specific flow driven spatial networks.


\subsection{Ablation Studies}
To further validate GAV, we conduct detailed ablation studies on the validation set of the ogbl-vessel benchmark. 
Table~\ref{tab:ablations_des} investigates the importance of the readout module, the message-passing module, and the labeling trick.
\begin{table}[h]
\centering
\scriptsize
\caption{Ablations on main design choices.}
\label{tab:ablations_des}
\begin{tabular}{c c c|c c} 
\toprule
Readout Module & Message-Passing & Labeling Trick & $\text{AUC}\uparrow$ &$\Delta$ \\
\midrule
\cmark & \cmark & \cmark & \textbf{98.39} & $-$\\
\xmark & \cmark & \cmark & 98.28 & -0.11\\
\cmark & \xmark & \cmark & 80.56 & -17.83\\
\cmark & \cmark & \xmark & 96.00 & -2.39\\
\bottomrule
\end{tabular}
\end{table}
First, we exchange our readout module with a SortPooling layer followed by two convolutional layers and an MLP, resembling SEAL's readout operation. We note that our readout module is more applicable to flow-driven spatial networks, as it leads to a modest AUC increase of 0.11. Second, we completely deactivate the message-passing module by forwarding $\mathcal{L}(\mathcal{G}^{t}_{h})$ directly to the readout module. We observe a drastic AUC decrease of 17.83, indicating the importance of modifying the vector embeddings via our proposed GAV layer.
Finally, we evaluate the impact of our labeling trick. Excluding the additional labels results in an AUC decrease of 2.39. This proves the significance of link identification via additional, distinct labels.\\

In a second ablation study, we experiment with different message-passing layers, including EdgeConv, in our message-passing module. We report our findings in Table~\ref{tab:ablations_operators}.
\begin{table}[h]
\centering
\scriptsize
\caption{Ablations with different message-passing layers.}
\label{tab:ablations_operators}
\begin{tabular}{c |c c c c} 
\toprule
Message-Passing Layer & $\text{AUC}\uparrow$ & $\text{Hits@100}\uparrow$& $\text{Hits@50}\uparrow$& $\text{Hits@20}\uparrow$ \\
\midrule
GAV layer (ours) &\textbf{98.39} & \textbf{34.46} & \textbf{26.30} & \textbf{19.81}\\
EdgeConv~\cite{wang2019dynamic} &97.43 & 17.30 & 5.97 & 0.78\\
GAT layer~\cite{brody2021attentive} &96.44 & 4.58 & 2.55 & 1.59\\
SAGE layer~\cite{hamilton2017inductive} &93.53 & 0.77 & 0.11 & 0.03\\
GCN layer~\cite{kipf2016semi} &89.31 & 0.39 & 0.22 & 0.16\\
\bottomrule
\end{tabular}
\end{table}
Our proposed GAV layer outperforms the other message-passing layers across all metrics by a considerable amount.\\

Lastly, we vary the number of hops $h$ used to generate $\mathcal{G}^t_h$ and the number of message-passing iterations $k$ (see Table~\ref{tab:ablations_kh}).
\begin{table}[h]
\centering
\scriptsize
\caption{Ablations on $k$ and $h$.}
\label{tab:ablations_kh}
\begin{tabular}{c |c c c c} 
\toprule
$k$ \& $h$ & $\text{AUC}\uparrow$ &$\text{Hits@100}\uparrow$& $\text{Hits@50}\uparrow$& $\text{Hits@20}\uparrow$\\
\midrule
1 & \textbf{98.39} & \textbf{34.46} & 26.30 & 19.81\\
2 & \textbf{98.39} & 34.00 & \textbf{26.93} & \textbf{21.21}\\
3 & \textbf{98.39} & 34.25 & 25.99 & 17.84\\
\bottomrule
\end{tabular}
\end{table}
We observe that simultaneously increasing $k$ and $h$ results in no discernible differences in performance. This finding is in line with the $\gamma$-decaying theory~\cite{zhang2018link}, proving the approximability of high-order heuristics from locally restricted subgraphs.


\begin{figure*}[h]
\centerline{\includegraphics[width=\linewidth]{imgs/inter2.pdf}}
\caption{
Visualization of the effect of our GAV layer on vector embeddings. We visualize subgraph representations $\mathcal{G}^t_h$ ($h$ set to one) of four positive (top row) and four negative target links (bottom row) together with the GAV layer's predicted scalar values $s_{i} \in (-1, 1)$. The scalar values $s_{i}$ used to update vector embeddings in $\mathcal{L}(\mathcal{G}^t_h)$ have been projected to $\mathcal{G}^t_h$ to provide an interpretable visualization. The directionality of edges already incorporates potential shifts in direction enforced by our GAV layer. Please note that following the color coding scheme of Fig.~\ref{fig:method_overview}, the target link $e^{t}_{ij}$ is depicted in orange, whereas the two target nodes $n_{i}^{t}$ and $n_{j}^{t}$ are displayed in red and green. We additionally report the angle $\angle$ between the vector embeddings aggregated around the two target nodes (see Section~\ref{ref:readout}) and the predicted probability of link existence $\hat{y}^t_{ij}$.
}
\label{fig:inter}
\end{figure*}

\subsection{Interpretability and Analysis of Results}
% vector embeddings represent edges in G.
In Fig.~\ref{fig:inter}, we visualize the behavior of our proposed GAV layer to facilitate interpretability. First, we investigate the correlation between the vector embeddings aggregated around the two target nodes represented by $\text{mean}(\mathcal{E}_{\mathcal{N}(n^t_{i})})$ and $\text{mean}(\mathcal{E}_{\mathcal{N}(n^t_{j})})$, which are constructed in the readout module (see Section~\ref{ref:readout}). We find that the angle between these aggregated vector embeddings is highly correlated to the predicted probability of link existence $\hat{y}^t_{ij}$ and hence provides decisive information. We observe a trend of high angles being associated with positive and low angles with negative predictions. 
In the context of flow, this observation draws parallels to the concept of sink and source flow.
% , which present theoretical principles not applicable in real life conditions.
To be precise, GAV may attempt to assign the two target nodes to sink and source nodes for negative predictions (see Fig.~\ref{fig:inter}, second row), which stands in contrast to the behavior of physical flow in spatial networks.

The GAV layer's predicted scalar values $s_{i} \in (-1, 1)$ can not only flip but also scale vector embeddings.
% We tried to interpret the switch in direction above. 
Based on Fig.~\ref{fig:inter}, we identify $|s_{i}|$ as a measure of uncertainty.
This is because with decreasing certainty of $\hat{y}^t_{ij}$, we observe a decrease in $|s_{i}|$ (see Fig.~\ref{fig:inter}, left to right).

% We observe a trend of decreasing $|s_{i}|$ with decreasing certainty of $\hat{y}^t_{ij}$ (see Fig.~\ref{fig:inter}, left to right).

% One property of GAV-layer's is to scale and potentially flip vector embeddings via $s_{i} \in (-1, 1)$. This can flip vector embeddings, resulting in a change of direction, which we interpret above, but it can also scale the vector embeddings. In Fig.~\ref{fig:inter}, we observe a trend, where $|s_{i}|$ decreases with increasing $\hat{y}^t_{ij}$ indicating that $|s_{i}|$ may be interpreted as an indicator of uncertainty (see Fig.~\ref{fig:inter} left to right). This is because $|s_{i}|$ decreases with the certainty of $\hat{y}^t_{ij}$.



% \subsection{Convergence}    % optional
% % show 2 curves; we converge faster

% \subsection{Generalizability}   % optional
% % ckpts at different datasets

\section{Outlook and Conclusion}
In this work, we propose the simple yet effective Graph Attentive Vectors (GAV) link prediction framework. GAV relies on the idea of modeling simplified physical flow by updating vector embeddings in a constrained manner, which intuitively models the underlying physical process and, therefore, presents a strong inductive bias for link prediction in flow-driven spatial networks. This allows GAV to outperform the previous state-of-the-art by an impressive margin on all metrics across multiple whole-brain vessel and road network datasets while requiring a significantly smaller amount of trainable parameters.
GAV's imitation of the dynamics of physical flow, however, represents a simplified concept, which is not entirely representative of physical principles from, \eg, fluid dynamics (see Fig.~\ref{fig:inter}). Future work should, therefore, aim to further investigate GAV's parameters and extend its assumptions by incorporating different physical principles, such as conservation of mass and momentum, resulting in vector embeddings highly representative of physical flow in flow-driven spatial networks.

%%%%%%%%% REFERENCES
{\small
\bibliographystyle{ieee_fullname}
\bibliography{egbib}
}

%%%%%%%%% APPENDIX
\onecolumn
\appendix

\renewcommand\thefigure{\arabic{figure}}
\setcounter{figure}{5}
\setcounter{table}{5}

\section{Notations}\label{sup:notation}
We provide a lookup table for notations used in our work.

\begin{table*}[h]
\centering
% \scriptsize
\caption{Description of notations used in our work.}
\vspace{-1em}
\label{tab:notations}
\renewcommand{\arraystretch}{1.2}
\begin{tabular}{c l} 
\toprule
Notation & Description\\
\midrule
$\mathcal{G}$ & undirected, flow-driven spatial network (or graph)\\
$\mathcal{V}$ & set of nodes in $\mathcal{G}$\\
$n_{i}$ & node $i$ in $\mathcal{G}$\\
$\mathcal{E}$ & set of edges in $\mathcal{G}$\\
$e_{ij}$ & edge (or link) in $\mathcal{G}$ between $n_i$ and $n_j$\\
$\overline{e_{ij}}$ & average edge length estimated over edges in $\mathcal{G}$\\
$\sigma$ & standard deviation of edge length estimated over edges in $\mathcal{G}$\\

$t$ & link prediction target\\
$e^{t}_{ij}$ & target link (or edge) between $n_{i}^{t}$ and $n_{j}^{t}$\\
$n_{i}^{t}$ & target node $i$ affiliated to $e^{t}_{ij}$\\

$h$ & number of hops\\
$\mathcal{G}^{t}_{h}$ & $h$-hop subgraph extracted around $e^{t}_{ij}$\\
%$\mathcal{E}^{t}_{h}$ & set of edges in $\mathcal{G}^{t}_{h}$\\

$\mathcal{L}(\mathcal{G}^{t}_{h})$ & line graph representation of $\mathcal{G}^{t}_{h}$\\
$\mathcal{V'}$ & set of nodes in $\mathcal{L}(\mathcal{G}^{t}_{h})$\\
$n'_{i}$ & node (or vector embedding) $i$ in $\mathcal{L}(\mathcal{G}^{t}_{h})$\\
$\mathcal{E'}$ & set of edges in $\mathcal{L}(\mathcal{G}^{t}_{h})$\\
$e'_{ij}$ & edge (or link) in $\mathcal{L}(\mathcal{G}^{t}_{h})$ between $n'_{i}$ and $n'_{j}$\\

$k$ & number of message-passing iterations\\
$s_{i}$ & scalar value generated in GAV layer\\
$|s_{i}|$ & absolute value of $s_{i}$ \\
$\tilde{n}'_{i}$ & intermediate node representation (or vector embedding) in GAV layer\\
$\hat{n}'_{i}$ & updated, refined node representation (or vector embedding)\\

$Q$ & query sequence in multi-head attention operation\\
$K$ & key sequence in multi-head attention operation\\
$V$ & value sequence in multi-head attention operation\\

$\hat{y}^t_{ij}$ & GAV's predicted probability of existence of $e^{t}_{ij}$\\
$y^t_{ij}$ & ground truth label of existence of $e^{t}_{ij}$\\

$\mathcal{E}_{\mathcal{N}(n^t_{i})}$ & set of refined vector embeddings originally created from edges adjacent to $n^t_{i}$\\
$N_{i}$ & matrix consisting of $n'_i$ and its direct neighbors\\
$\mathcal{N}(n_{i})$ & set of nodes in the direct neighborhood of $n_{i}$\\
$\mathcal{N}(n_{i}) \cup n_i $ & set of nodes in the direct neighborhood of $n_{i}$ including $n_{i}$ itself\\
% $\phi_{\theta}$ & -\\
$\phi^{(1)}_{\theta}$, $\phi^{(2)}_{\theta}$ & learnable functions in GAV layer\\
$\phi^{(3)}_{\theta}$ & learnable function in readout module\\

$d_{\text{spatial}}$ & spatial dimension (2 or 3)\\
$d_{\text{message}}$ & dimension of $\tilde{n}'_{i}$ in GAV layer\\

$\delta$ & maximum distance threshold utilized in spatial sampling during preprocessing\\
$\mathbin\Vert$ & concatenation operation\\
$\mathcal{L}_{\text{BCE}}$ & binary cross-entropy loss function\\
% $\psi_b$ & bifurcation angle\\
\bottomrule
\end{tabular}
\end{table*}
\newpage

\section{More on Interpretability and the Modification of Vector Embeddings}\label{sup:inter}
We provide the interested reader with more visualizations regarding the GAV layer's modification of vector embeddings (see Section~\ref{ref:gav_layer}) on the validation set of the ogbl-vessel benchmark, similar to Fig.~\ref{fig:inter}. These visualizations can also be interpreted as qualitative results. Fig.~\ref{fig:pos_preds} depicts subgraph representations $\mathcal{G}^t_h$ ($h$ set to one) of 12 positive (real, plausible) target links, while Fig.~\ref{fig:neg_preds} depicts subgraph representations of 12 negative (sampled, implausible) target links. Please note that the respective last rows depict challenging cases, as indicated by GAV's predicted probabilities $\hat{y}^t_{ij}$. Additionally, we would like to highlight our hypothesis from Section~\ref{ref:inter} that GAV may attempt to assign the two target nodes (red and green) to sink and source nodes for negative, implausible vessel formations (see Fig.~\ref{fig:neg_preds}), which results in superior representations for link prediction in flow-driven spatial networks that can be effortlessly classified in our physically plausible readout module. Please note that we conduct an additional experiment modifying a toy example in Sec.~\ref{sup:toy} to further facilitate interpretability.\\ \\

\begin{figure*}[h]
\centerline{\includegraphics[width=\linewidth]{imgs_supp/pos3.pdf}}
\caption{
Visualization of the effect of our GAV layer on vector embeddings. We visualize subgraph representations $\mathcal{G}^t_h$ of 12 positive target links ($y_{ij}^t = 1$) together with the GAV layer's predicted scalar values $s_{i} \in (-1, 1)$. The scalar values $s_{i}$ used to update vector embeddings in $\mathcal{L}(\mathcal{G}^t_h)$ have been projected to their corresponding edges in $\mathcal{G}^t_h$ (see Fig.~\ref{fig:method_overview}) to provide an interpretable visualization. The directionality of edges (indicated by arrows) already incorporates potential shifts in the direction of vector embeddings enforced by our GAV layer. We additionally report the angle $\angle$ between the vector embeddings aggregated around the two target nodes (see Section~\ref{ref:readout}) and the predicted probability of link existence $\hat{y}^t_{ij}$.
}
\label{fig:pos_preds}
\end{figure*}
\newpage

\begin{figure*}[h]
\centerline{\includegraphics[width=\linewidth]{imgs_supp/neg2.pdf}}
\caption{
Visualization of the effect of our GAV layer on vector embeddings. We visualize subgraph representations $\mathcal{G}^t_h$ of 12 negative target links ($y_{ij}^t = 0$) together with the GAV layer's predicted scalar values $s_{i} \in (-1, 1)$. The scalar values $s_{i}$ used to update vector embeddings in $\mathcal{L}(\mathcal{G}^t_h)$ have been projected to their corresponding edges in $\mathcal{G}^t_h$ (see Fig.~\ref{fig:method_overview}) to provide an interpretable visualization. The directionality of edges (indicated by arrows) already incorporates potential shifts in the direction of vector embeddings enforced by our GAV layer. We additionally report the angle $\angle$ between the vector embeddings aggregated around the two target nodes (see Section~\ref{ref:readout}) and the predicted probability of link existence $\hat{y}^t_{ij}$.
}
\label{fig:neg_preds}
\end{figure*}

\section{Initialization of Vector Embeddings}\label{sup:imp_det}
The initialization of the direction of vector embeddings represents an important implementation detail and is based on a straightforward intuition. To be precise, we initialize vector embeddings to point away from the target link $e^{t}_{ij}$, \ie, towards nodes with a node degree of one (leaf nodes). The vector embedding representative of the target link $e^{t}_{ij}$ is set to point from $n_{i}^{t}$ to $n_{j}^{t}$. An exemplary initialization of vector embeddings for a 1-hop subgraph can be found in Fig.~\ref{fig:method_overview}.

\section{GAV and Structural Properties}\label{sup:toy} % does gav really learn structural features?
This section elaborates on how GAV's predictions rely heavily on structural properties, such as bifurcation angles, which reflect functional properties of the underlying physical system~\cite{schneider2012tissue}.
% , such as our simplified definition of physical flow.
To this end, we prepare and modify a synthetic mock example in Fig.~\ref{fig:morp}. Specifically, we vary the bifurcation angle $\psi_b$ spanned between two edges connected to $n^t_i$ (red) to generate morphological implausible and plausible blood vessel formations (see Fig.~\ref{fig:morp}).
% , fulfilling morphological and hemodynamic properties.
%formed around the target link (orange).

\begin{figure*}[h]
\centerline{\includegraphics[width=0.7\linewidth]{imgs_supp/morph_exp5.pdf}}
\caption{
Morphological implausible (left) and plausible (middle and right) blood vessel formations formed around the target link (orange) with varying bifurcation angles $\psi_b$. GAV correctly identifies morphological plausible blood vessel formations that fulfill relevant hemodynamic functional properties~\cite{schneider2012tissue}.
%, by assigning a high probability of link existence to plausible formations.
 }
\label{fig:morp}
% \vspace{-1em}
\end{figure*}

\noindent
As expected, GAV differentiates between plausible and implausible blood vessel formations formed around the target link. GAV assigns a high probability of target link existence $\hat{y}^t_{ij}$ to plausible and a low probability of target link existence to implausible formations. Please note that Fig.~\ref{fig:morp} additionally maps the potentially modified directionality of vector embeddings onto their corresponding edges, similar to Fig.~\ref{fig:pos_preds} and Fig.~\ref{fig:neg_preds}. One can observe the shift in the directionality of vector embeddings created from edges adjacent to nodes with high bifurcation angles $\psi_b$, transforming the target nodes (red and green) to sink and source nodes for morphological implausible blood vessel formations (see Fig.~\ref{fig:morp}, left).


\section{Visualization of Datasets}\label{sup:data}
In Fig.~\ref{fig:data}, we graphically visualize two flow-driven spatial networks representative of murine whole-brain vessel graphs and road networks.

\begin{figure*}[h]
\centerline{\includegraphics[width=\linewidth]{imgs_supp/data.pdf}}
\caption{
Visualization of a whole-brain vessel graph and a road network. The depicted flow-driven spatial networks correspond to the raw cd1-tc-vessel and luxembourg-road datasets.
}
\label{fig:data}
\vspace{-1em}
\end{figure*}


\section{More Ablations on the GAV Layer}\label{sup:gav_abl}
Since the GAV layer relies on a set of specific design choices, we conduct additional ablation studies determining their influence on the link prediction performance on the validation set of the ogbl-vessel benchmark. To this end, we experiment with different versions of the GAV layer. First, we deactivate the multi-head attention operation; second, we exclude the residual connection; and third, we exchange the leaky ReLU non-linearity in the GAV layer with the ReLU non-linearity. We report our findings in Table~\ref{tab:ablations_des_layer}.

\begin{table}[h]
\centering
\scriptsize
\caption{Ablations on the GAV layer's main design choices.}
\vspace{-1em}
\label{tab:ablations_des_layer}
\begin{tabular}{c c c|c c c c} 
\toprule
Attention & Residual Connection & Leaky ReLU & $\text{AUC}\uparrow$ &$\text{Hits@100}\uparrow$& $\text{Hits@50}\uparrow$& $\text{Hits@20}\uparrow$ \\
\midrule
\cmark & \cmark & \cmark & \cellcolor{teal!40}98.39 & \cellcolor{teal!20}34.46 & \cellcolor{teal!10}26.30 & \cellcolor{teal!40}19.81\\
\xmark & \cmark & \cmark & \cellcolor{teal!10}97.48 & 15.56 & 9.18 & 5.68\\
\cmark & \xmark & \cmark & \cellcolor{teal!20}98.34 & \cellcolor{teal!40}34.90 & \cellcolor{teal!40}27.61 & \cellcolor{teal!20}19.28\\
\cmark & \cmark & \xmark & \cellcolor{teal!20}98.34 & \cellcolor{teal!10}34.16 & \cellcolor{teal!20}26.47 & \cellcolor{teal!10}17.31\\
\bottomrule
\end{tabular}
\end{table}

\noindent
Deactivating the multi-head attention operation (second row) results in a drastic AUC decrease of 0.91, indicating the importance of neighborhood awareness when modifying vector embeddings via our proposed GAV layer. Excluding the GAV layer's residual connection (third row) and using ReLU instead of Leaky ReLU non-linearities (fourth row) leads to a slight reduction in AUC of 0.05, respectively. Based on our reported standard deviation value of $\pm$ 0.02 (see Table~\ref{tab:quantitative_results}), we argue that this performance decrease is indeed significant.

\section{Evaluation Metrics}\label{sup:metrics}
To compare GAV to existing baseline algorithms, we report quantitative results based on the area under the receiver operating characteristic curve (AUC), following the obgl-vessel benchmark. The AUC metric indicates the performance of a classifier by plotting the true positive rate against the false positive rate at all possible classification thresholds. Therefore, AUC provides an aggregate performance measure indicating the classifier's ability to distinguish between positive and negative links.
% Statistically speaking, AUC scores should levitate between the values of 50\% and 100\%, where a AUC score of 50\% indicates a more-or-less random guess, while a perfect classifier achieves a AUC score of 100\%.

We introduce the evaluation metric Hits@$k$ as an additional, stricter performance measure. Hits@$k$ compares the classifier's prediction of every single positive link to a randomly sampled set of 100,000 negative links, resulting in a ranking among 100,001 links with respect to the probability of link existence. Based on this ranking, Hits@$k$ indicates the ratio of positive links ranked at the $k$-th place and above. In the context of this work, we evaluate Hits@$k$ at $k=100$, $k=50$, and $k=20$, inspired by other Open Graph Benchmark~\cite{hu2020ogb} link prediction benchmarks.


\section{Configuration of Our Secondary Baseline}\label{sup:sec_base}
We incorporate the EdgeConv message-passing layer~\cite{wang2019dynamic} into the SEAL framework~\cite{zhang2018link, zhang2021labeling}, which has been shown to deliver results on par with or superior to the state-of-the-art on multiple link prediction benchmarks, to introduce a strong, \textit{secondary baseline} (SEAL+EdgeConv) for link prediction in spatial networks. To be precise, we incorporate EdgeConv in SEAL's DGCNN~\cite{zhang2018end}. EdgeConv's update function can be observed in Table~\ref{tab:update}. Here, $\phi_{\theta}$ represents a two-layer MLP with an input dimension of 64, a hidden dimension of 32, and an output dimension of 32. Our modified DGCNN employs in total three EdgeConv layers, with the only difference being that the input dimension of the first EdgeConv layer's MLP corresponds to 70 and the output dimension of the third EdgeConv layer's MLP to 1. Our EdgeConv version utilizes a mean feature aggregation scheme. We set the number of in- and output channels of the DGCNN readout operation's two 1D convolutions to 1 \& 16 and 16 \& 32, respectively. The kernel sizes and strides of the two 1D convolutions correspond to 65 \& 65 and 5 \& 1. We set the input, hidden, and output dimensions of the DGCNN readout operation's MLP to 38, 128, and 1. The global sort pooling layer's parameter k is set to 10. All hyperparameters were tuned on the validation set of the ogbl-vessel benchmark.

\section{GAV's Performance on Non-Flow-Based Link Prediction Benchmarks}\label{sup:non_flow}
To additionally confirm that GAV is specialized for link prediction in flow-driven spatial networks, we conduct an experiment on the ogbl-collab benchmark~\cite{hu2020ogb}, which represents a collaboration network given by an undirected graph where nodes are associated with authors while edges indicate collaborations between them. Node features are comprised of 128-dimensional vectors representative of an author's scientific work (averaged word embeddings reflecting the content of scientific papers). Based on the collaboration network, the task is to predict future collaborations between authors. To adjust GAV to the task of the ogbl-collab benchmark, we model vector embeddings representative of edges in the collaboration network as the difference between 128-dimensional feature vectors of two nodes incident to an edge. We report our findings in Table~\ref{tab:quant_res_collab}.

\begin{table}[h]
\centering
\scriptsize
\caption{Comparison between GAV and SEAL on ogbl-collab and ogbl-vessel. Please note that the increase in GAV's trainable parameters in the experiment on ogbl-collab is mostly due to the increased number of node features (128 vs. 3).}
\vspace{-0.5em}
\label{tab:quant_res_collab}
\begin{tabular}{c|c|l|c} 
\toprule
Dataset & Model & $\#\ \text{Params}\downarrow$ & $\text{Eval. Metric}\uparrow$  (\%)\\
\midrule
\multirow{2}{*}{ogbl-collab}
& SEAL~\cite{zhang2021labeling} & \cellcolor{teal!20}501,570 & \cellcolor{teal!40}64.72 $\text{Hits@50}$\\
& GAV (ours) & \cellcolor{teal!40}44,194 & \cellcolor{teal!20}16.72 $\text{Hits@50}$\\
\midrule
\multirow{2}{*}{ogbl-vessel}
& SEAL~\cite{zhang2021labeling} & \cellcolor{teal!20}172,610 & \cellcolor{teal!20}80.50 $\text{AUC}$\\
& GAV (ours) & \cellcolor{teal!40}8,194 & \cellcolor{teal!40}98.38 $\text{AUC}$\\
\bottomrule
\end{tabular}
\end{table}

\noindent
As expected, GAV, relying on the idea of modeling simplified physical flow in flow-driven spatial networks, does not deliver competitive results on the ogbl-collab benchmark. This is because GAV's strong inductive biases are tailored to link prediction in flow-driven spatial networks and are, therefore, too restrictive for non-flow-based networks, such as ogbl-collab. This repeatedly demonstrates GAV's ability to intuitively model the underlying physical process in flow-driven spatial networks.
%This repeatedly demonstrates GAV's ability to intuitively models the underlying physical process in flow-driven spatial networks, representing a strong inductive bias for the link prediction task.

To adapt GAV to non-flow-based networks more appropriately, we encourage future work to explore the use of pseudo-spatial positions embedded in nodes of non-flow-based networks rather than its actual node features for the sake of creating vector embeddings. Pseudo-spatial position could, \eg, be determined based on the Fruchterman-Reingold force-directed algorithm~\cite{kobourov2012spring}.


\section{On Translation and Rotation Invariance}\label{sup:invariance}
\begin{wrapfigure}{r}{0.35\textwidth}
  \vspace{-3em}
  \begin{center}
    \includegraphics[width=0.35\textwidth]{imgs_supp/rot_inv.pdf}
  \end{center}
  \vspace{-1.5em}
  \caption{Exemplary $\mathcal{G}^{t}_{h}$ extracted from the ogbl-vessel benchmark around a negative and positive link used for experiments in Table~\ref{tab:rot_inv}.}
  \vspace{-1em}
  \label{fig:rot_inv}
\end{wrapfigure}
In this section, we briefly investigate GAV's behavior under rotations and translations of the $h$-hop enclosing subgraph $\mathcal{G}^{t}_{h}$. Specifically, we aim to investigate whether rotation and translation of $\mathcal{G}^{t}_{h}$ result in similar predictions. Since GAV encodes edges as vector embeddings spanned between two nodes (see Section~\ref{ref:subnetw_extraction}), translation invariance is \textit{explicitly ensured}.
However, even though rotation preserves the length and relative angles of edges, rotation invariance is \textit{not explicitly ensured}. This is, \eg, because queries and keys forwarded to the GAV layer's attention operation are not explicitly rotation equi- or invariant, which is one of the key requirements for rotation invariant attention weights and hence potential rotation invariant predictions~\cite{fuchs2020se}. An empirical experiment rotating exemplary input graphs around all three axes, however, demonstrates that GAV's predictions are relatively robust to rotation, indicating to some degree implicit, learned rotation invariance (see Table~\ref{tab:rot_inv}). We encourage future work to further explore the necessity of explicitly encoded translation and rotation invariance in the context of link prediction for flow-driven spatial networks.

\begin{table}[h]
\centering
\scriptsize
\caption{Experiment on rotation invariance of predictions. We rotate three exemplary subgraphs $\mathcal{G}^{t}_{h}$ around all three axes with a step size of 1$^{\circ}$ to investigate GAV's behavior under rotation. We report the standard deviation of predicted target link probability $\hat{y}^t_{ij}$ over all 360 predictions.}
\vspace{-0.5em}
\label{tab:rot_inv}
\begin{tabular}{l|c|c|c} 
\toprule
Subgraph $\mathcal{G}^{t}_{h}$ & $\sigma_{\hat{y}^t_{ij}} \text{(x-axis)}$ & $\sigma_{\hat{y}^t_{ij}} \text{(y-axis)}$ & $\sigma_{\hat{y}^t_{ij}} \text{(z-axis)}$\\
\midrule
Fig.~\ref{fig:morp}, right & 5.15$\cdot e^{-\text{5}}$ & 1.86$\cdot e^{-\text{6}}$ & 1.34$\cdot e^{-\text{5}}$\\
Fig.~\ref{fig:rot_inv}, left & 9.63$\cdot e^{-\text{2}}$ & 1.12$\cdot e^{-\text{3}}$ & 7.64$\cdot e^{-\text{3}}$\\
Fig.~\ref{fig:rot_inv}, right & 1.85$\cdot e^{-\text{3}}$ & 1.08$\cdot e^{-\text{3}}$ & 3.37$\cdot e^{-\text{4}}$\\
\bottomrule
\end{tabular}
\end{table}

\section{Commonalities and Differences between SEAL and GAV}\label{sup:commonalities}
Since the influential SEAL link prediction framework~\cite{zhang2018link, zhang2021labeling} represents one of the most prominent works on learned, GNN-based link prediction algorithms, we would like to clearly state the commonalities and differences between SEAL and our proposed link prediction algorithm tailored to flow-driven spatial networks, GAV. Although GAV utilizes some concepts introduced by SEAL (subgraph extraction/classification \& labeling trick), which are provably used in most competitive approaches and can, therefore, be seen as common practices, we, for the first time, introduce the principle of physical flow to link prediction. To this end, we propose not only a novel flow-inspired, parameter-efficient message-passing layer updating vector embeddings but also a physically plausible readout module facilitating interpretability. Our contributions result in an increase of more than 22\% in AUC compared to SEAL on ogbl-vessel.

\section{Message-Passing Update Functions}\label{sup:update}
We provide a concise overview of message-passing layers featured in our work and their respective high-level, final node update functions in Table~\ref{tab:update}. Here, $d_i$ stands for the node degree of $n_i$, $\alpha_{ij}$ for the learned attention coefficient between $n_i$ and $n_j$, and $\phi_{\theta}$ for an arbitrary learnable function. We would like to highlight the simplicity of the GAV layer's final update function.

\begin{table}[h]
\centering
% \scriptsize
\caption{Message-passing update functions.}
\vspace{-0.5em}
\label{tab:update}
\renewcommand{\arraystretch}{2}
\begin{tabular}{c | c} 
\toprule
Message-Passing Layer & Update Function\\
\midrule
GAV layer & $\displaystyle \hat{n}_{i} = s_i \cdot n_i$\\

EdgeConv~\cite{wang2019dynamic} & $\displaystyle \hat{n}_{i} = \frac{1}{|\mathcal{N}(n_i)|} \sum_{n_j \in \mathcal{N}(n_i)} \phi_{\theta}(n_{i} \mathbin\Vert n_{j} - n_{i})$\\
GAT layer~\cite{velivckovic2017graph} & $\displaystyle \hat{n}_{i} = \alpha_{ii} \cdot \phi_{\theta}(n_{i}) + \sum_{n_j \in \mathcal{N}(n_i)} \alpha_{ij} \cdot \phi_{\theta}(n_{j})$\\
 % & $\displaystyle \text{where}\ \alpha_{ij} = \frac{\text{exp}\, (\text{Leaky ReLU}\, ( a^{T} [\phi_{\theta}(n_{i}) \mathbin\Vert  \phi_{\theta}(n_{j})]\,)\,)} {\displaystyle\sum_{{n_k \in \mathcal{N}(n_i)}}  \text{exp}\, (\text{Leaky ReLU}\, ( a^{T} [\phi_{\theta}(n_{i}) \mathbin\Vert  \phi_{\theta}(n_{k})])\,)\,}$\\

SAGE layer~\cite{hamilton2017inductive} & $\displaystyle \hat{n}_{i} =  \phi_{\theta}^{(1)}(n_{i}) +  \phi_{\theta}^{(2)}(\frac{1}{|\mathcal{N}(n_i)|} \sum_{n_j \in \mathcal{N}(n_i)} n_{j})$\\

GCN layer~\cite{kipf2016semi} & $\displaystyle \hat{n}_{i} =  \phi_{\theta}(  \sum_{n_j \in \mathcal{N}(n_i)\, \cup\, n_i} \frac{1}{\sqrt{d_i \cdot d_j}}\, n_j)$\\
\bottomrule
\end{tabular}
\end{table}

\section{Medical Relevance of the Link Prediction Task for Whole-Brain Vessel Graphs}\label{sup:relevance}
Since GAV has been developed around the ogbl-vessel benchmark, we would like to provide more details on the medical relevance and the application of link prediction algorithms for whole-brain vessel graphs. As already mentioned in Section~\ref{sec:intro}, vascular network representations of the brain originate from a multi-stage, imperfect process, typically consisting of a segmentation stage followed by a graph extraction stage (skeletonization and pruning). Detailed pipelines for whole-brain vessel graph generation can be found in the literature~\cite{paetzold2021whole, walchli2021hierarchical, meyer2009voreen, drees2021scalable}. Each stage of the graph generation pipeline introduces noise and artifacts to the extracted whole-brain vessel graphs. The initial segmentation stage~\cite{todorov2020machine}, \eg, often results in under- or over-connected vessel segmentation masks, which in turn result in equally under- or over-connected whole-brain vessel graphs. This is mostly due to the shortage of annotated training data (especially in the 3D domain), which is required for accurate vessel segmentation via supervised state-of-the-art deep-learning-based segmentation techniques.
Under-/over-connectivity, however, limits the application of whole-brain vessel graphs for subsequent medically relevant downstream tasks, such as the diagnosis, treatment, and analysis of neurovascular brain disorders (\eg, aneurysms or strokes). This is because these downstream tasks require flawlessly connected whole-brain vessel graphs free of artifacts to obtain a deeper understanding of neurovascular brain disorders by, \eg, accurately recognizing anomalies in blood flow patterns via blood flow modeling~\cite{schmid2021severity}.
To overcome the obstacle of under-/over-connectivity in whole-brain vessel graphs and, therefore, to enable researchers to obtain a more accurate and advanced understanding of neurovascular brain disorder, one can either optimize whole-brain vessel graph generation pipelines~\cite{shit2022relationformer, shit2021cldice} or utilize the task of link prediction, which we extensively investigate in this work.

\end{document}