% what are flow-driven spatial networks
Networks (or graphs) can serve as efficient representations of real-world, ultra-complex systems and can be further classified into different categories.
% Depending on the characteristics of their underlying systems, networks can be further classified into different categories.
A prominent category is represented by undirected networks embedded in a Euclidean space constrained by geometry, called spatial networks~\cite{barthelemy2011spatial}. In this work, we are focusing on spatial networks, where a form of physical exchange or \emph{flow} can be used to describe characteristic functional properties of the underlying physical system. Examples include road networks, water bodies, and global exchange networks, but they can also be found in biology (\eg, vascular system, lymphatic system, and connectome). We will refer to such networks as \emph{flow-driven spatial networks}.

\begin{figure}[t]
\centerline{\includegraphics[width=\linewidth]{imgs/intro.pdf}}
\caption{
Flow-driven spatial network $\mathcal{G}$, representing vasculature. $\mathcal{G}$'s nodes are embedded in a Euclidean space and represent spatial positions specified by $x$-, $y$-, and $z$-coordinates.
}
\vspace{-0.5em}
\label{fig:intro}
\end{figure}

% how are they generated and why do we need link prediction for this
Predominantly, network representations of physical systems originate from imaging methodologies, such as nanometer-scale microscopy in biology or regional to continental scale satellite remote sensing for road networks. The generation of compact network representations from these images is a multi-stage and imperfect process, which often consists of segmentation, skeletonization, and subsequent graph pruning.
% Since, the erroneous graph generation process typically includes artifacts resulting in a flawed connectivity, 
Since, for flow-driven spatial networks, the correct connectivity is of utmost importance, the erroneous graph generation process clearly motivates the task of link prediction as a meaningful method to optimize the graph representation.

% what are we doing + why are we doing this
Therefore, we bring a simplistic yet general definition of the principle of physical flow, characterized by a direction and magnitude, to link prediction in graph representation learning. Our hypothesis is that for flow-driven spatial networks, link prediction algorithms should heavily benefit from considering known functional properties, such as the aforementioned \emph{physical flow}, which are defined by the structural properties of the network (\eg, bifurcation angles~\cite{schneider2012tissue}).
To this end, we propose the \emph{Graph Attentive Vectors} (GAV) link prediction framework. GAV operates on \emph{vector embeddings} representative of the network's structural properties and updates them in a constrained manner, imitating simplified dynamics of physical flow in spatial networks (\eg, blood flow in the vascular system or traffic flow in road networks).
We summarize our contribution as follows:
\begin{enumerate}
\itemsep0em 
    \item We propose an attentive, neighborhood-aware message-passing layer, called GAV layer, which updates vector embeddings, mimicking the (change in) direction and magnitude of physical flow in spatial networks.
     
    \item We introduce a readout module that aggregates vector embeddings in a physically plausible way and thus facilitates the interpretability of results.
     
    \item We formulate link prediction as a graph-level classification task on a line graph representation and propose a tailored node labeling trick.
    
\end{enumerate}

In extensive validation experiments, we prove our hypothesis by demonstrating superior performance across all metrics on eight flow-driven spatial networks, including the Open Graph
Benchmark's ogbl-vessel benchmark (98.38 vs. 83.07 AUC).

% GAV shows exceptional results on meaningful link prediction tasks, such as . For instance, on missing vessel prediction and missing road connection identification. For example, we outperform the current top-performing algorithm on the public open graph benchmark (ogb-vessel) leaderboard (0.83 AUC) \textbf{by more than 15\% (0.98 AUC)}.
% \footnote{\url{https://ogb.stanford.edu/docs/linkprop/\#ogbl-vessel}}


% \paragraph{Our contribution}
% In this work, we aim to bring a simplistic but general definition of the principle of physical flow, which is only characterized by a direction and magnitude, to link prediction in graph representation learning. Our hypothesis is that: For any flow-driven spatial network that is defined by physical transportation processes (i.e., when a flow exists and no sources and sinks exist), the link prediction task should be solvable by exploiting this simplistic flow information alone. 

% We consider that any flow information can be generalized to a vector of its direction and magnitude. In this paper, we propose a link prediction framework where we embed the given orientation of the local physical graph and learn the flow direction and magnitude to advance link prediction. We achieve this through the following four key contributions in our \underline{GAV} framework:


% \begin{enumerate}
% \itemsep0em 

%     \item We propose a novel neighborhood-aware message passing paradigm for \underline{G}raph learning. It is inspired by our simplistic definition of flow and applies a multi-head \underline{A}ttention mechanism on the local neighborhood graph.
     
%     \item We restrict the update of node embeddings to a flow-inspired function $f$. Inspired by in- and outflow of physical units $f$ restricts the node embedding update to scalar multiplication mimicking the (change of) direction and magnitude of physical flow resulting in a \underline{V}ector representation. 
     
%     \item We develop a dedicated readout function, which aggregates the learned (flow-)vectors of the two subgraph representations surrounding the link prediction target. 
     
%     \item We propose a subnetwork extraction module, which combines a line graph representation with a tailored node labeling trick.
    
% \end{enumerate}

% In extensive validation experiments, GAV shows exceptional results on meaningful link prediction tasks. For instance, on missing vessel prediction and missing road connection identification. For example, we outperform the current top-performing algorithm on the public open graph benchmark (ogb-vessel) leaderboard (0.83 AUC) \textbf{by more than 15\% (0.98 AUC)}.
% \footnote{\url{https://ogb.stanford.edu/docs/linkprop/\#ogbl-vessel}}


% \begin{figure}[t]
% \centerline{\includegraphics[width=0.80\linewidth]{imgs/intro.png}}
% \caption{
% }
% \label{fig:intro}
% \end{figure}


% refine correct graph extraction (very important in vessel for flow simulation), possibly propose meaningful new links to improve eg. traffic flow, missing links which are likely to appear in the future'.
% make it clear hwat link prediction in our case does  road exists, new road exist etc. 
% \Eg, blood vessels, which transport oxygen and nutrients through the entire human body, neurons that pass electrical signals through the connectome, or road networks that define traffic flow.


















%\textbf{TODO:} - characterize and describe our contribution, how do we exactly bring concepts from flow characterization to link prediction - What is our contribution in simple terms - how do we implement it - what do we show in experiments --> substantial improvements on all kinds of link prediction tasks - 


% Network-structured data typically describes complex systems, such as social networks, molecules, or the brain's vasculature, based on a set of nodes and edges. 
% While nodes embed individual entities, edges capture the relations and thus the connectivity among nodes. This results in a abstract, yet powerful data structure, which requires a special set of methods to obtain a deeper understanding of networks and their underlying systems.

% Depending on the characteristics of their underlying systems, networks can be further classified into different categories. In the context of this work, we are focusing on the murine brain vasculature, given by whole-brain blood vessel networks. 


% Since networks usually consist of 
% To get a deeper understanding of networks and their underlying systems, previous work 


% Depending on the characteristics of their underlying systems, networks can be further classified into different categories. In the context of this work, we are focusing on the murine brain vasculature, given by whole-brain blood vessel networks. 

% The edges of a networks can, for example, be either directed or undirected, resulting in so-called directed and undirected networks.



% In the context of this work, we are focusing on pseudo-directed spatial networks given by whole-brain blood vessel networks, describing the murine brain vasculature. A pseudo-directed spatial network 



% Edges, which are also commonly referred to as links, can be either directed or undirected, dependant on the characteristics of the network and its underlying system. This subsequently results in two types of networks, namely directed and undirected networks. A network can also be further categorized based on the  

% % 

% In the context of this work, we are focusing on the task of increasing the connectivity of spatial vessel networks.
% node and hence edges are constrained by geometry

% Recently, machine learning on networs via so-called graph neural networks (GNNs)


% Link prediction poses as a key-problem in the analysis and enhancement of graph connectivity.
% and has wide use cases...


% Graph networks are usually constructed of a set of nodes and edges to model Euclidean relationships among nodes. 

% single node vs multi-node tasks


% link prediction is a key-problem in graph analysis...

% links between two nodes in a graph -> some use cases (water distibution, social networks, recommendation sysyem, etc.) / refine correct graph extraction (very important in vessel for flow simulation), possibly propose meaningful new links to improve eg. traffic flow, missing links which are likely to appear in the future.

% Even though link prediction was in the past dominated by purely heuristic based methods, most current sota architectures utizise GNNs due to their superior expressive modelling power / learned heuristics.

% We motivate our approach by the following intuition: flow direction should allow use to determine links, -> modify existing direction vectors just by scalar to keep important properties of flow vector like angle + dir? (something along this line...)
% why is flow/dir so important for link pred? Pseudo-directed graphs rely on energy optimal... (something like this maybe)

% Therefore, this work aims to pave the path for GNN-based link prediction models tailored to spatial networks.

% We pave the path towards...
% We summarize our contribution as follows:
% \begin{itemize}
%  \item MPN layer and aggreagtion function tailored to pseudo-directed spatial networks\label{item:first}
 
%  \item show that LG of utmost importance\label{item:first}

%  \item reintroduces directionality (data driven / learned), which may be be used for downstream tasks\label{item:first}

%  \item 1\label{item:first}
% \end{itemize}