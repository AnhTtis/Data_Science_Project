%Submitted

%Bezh Ten Cate.
%

\documentclass[a4paper]{article}

 \usepackage{hyperref}
\hypersetup{
    colorlinks=true,
    linkcolor=blue,
    filecolor=blue,
    urlcolor=blue,
    citecolor=blue,
    filecolor=blue
    }

% Here you can include the standard packages you use.
% Try to avoid using non-standard packages.
% If you use a non-standard package you will have
% to submit it when you submit the final version of
% your paper.
\usepackage{graphicx}
\usepackage{amsmath}
\usepackage{amsthm}
\usepackage{amssymb}
\usepackage{xypic}

\usepackage{xcolor}

\usepackage{enumerate}

%%%%%%%%%%%%%%%%%%%%%%%%%%%%%%%%%%%%%%%%%%%%%%%%%%%%%%%%%


%\theoremstyle{thmstyleone}%
\newtheorem{theorem}{Theorem}
\newtheorem{lemma}[theorem]{Lemma}
\newtheorem{proposition}[theorem]{Proposition}%
\newtheorem{corollary}[theorem]{Corollary}%

%\theoremstyle{thmstyletwo}%
\newtheorem{example}[theorem]{Example}%
\newtheorem{problem}[theorem]{Problem}%


%\theoremstyle{thmstylethree}%
\theoremstyle{definition}
\newtheorem{definition}[theorem]{Definition}%
\newtheorem{remark}[theorem]{Remark}

% definitions specific to your article

%\input{_macroAiML}
\newcommand\fkCan[1]{F_L {\langle #1\rangle} }


\newcommand{\ff}[1]{\widehat{#1}}
\newcommand{\fa}{\ff{a}}
\newcommand{\fb}{\smash{\ff{b}}}
\newcommand{\fx}{\ff{x}}
\newcommand{\fy}{\ff{y}}
\newcommand{\fz}{\ff{z}}
\newcommand{\fW}{\smash{\ff{W}}}
\newcommand{\fR}{\smash{\ff{R}}}
\newcommand{\fS}{\smash{\ff{S}}}
\newcommand{\fV}{\smash{\ff{V}}}
\newcommand{\fF}{\smash{\ff{F}}}
\newcommand{\fM}{\smash{\ff{M}}}


\newcommand{\regSymb}{\sharp}%{\bigstar}
\newcommand{\extReg}[1]{{#1}{\lefteqn{}}^\regSymb}
%\newcommand\extRegn[2]{{#1}^{(\regSymb\times #2)}}
\newcommand{\extRegn}[2]{{#1}{}^{(#2)}}
\def\dts{\ldots}
\newcommand{\PDL}{\ensuremath{\mathsf{PDL}}\xspace}
\def\quadr{\mu}
\def\diff{\rightleftharpoons}
\def\zZ{\mathbb{Z}}
\def\rR{\mathbb{R}}



\def\Fm{\operatorname{Fm}}
\def\PV{\operatorname{PV}}
\def\At{\operatorname{A}}
\def\Al{{\At}}
\def\AlB{{\operatorname{B}}}
\def\Prog{\operatorname{Prog}}

\def\PDL{{\bf PDL}}
\def\CPDL{{\bf CPDL}}
\newcommand\temp[2]{{#1}^{\mathrm{C}}_{#2}}
\newcommand\trans[2]{{#1}^{+}_{#2}}

\newcommand{\BP}{{{{\boxplus}}}}
\newcommand{\TR}{{({+})}}

\newcommand{\CT}[1]{[{\color{blue}#1}]}

\newcommand{\BX}[1]{\!\mathop{\texttt{\upshape[}#1\texttt{\upshape]}\!}}

\newcommand{\defeq}{\leftrightharpoons}

%\newcommand{\nTO}{\,\,\,\not\!\!\!\TO}

\newcommand{\mL}{\Models(L)}
\newcommand{\fL}{\Frames(L)}
\def\val{\vartheta}
\newcommand\valext[1]{\overline{\vartheta}(#1)}

\def\bL{\vL}

\newcommand{\Iff}{\Leftrightarrow}
\newcommand{\Fi}{\varphi}
%\newcommand{\Sub}{\ensuremath{\mathop{\mathsf{Sub}}}\xspace}
\newcommand{\Sub}{\mathop{\mathsf{Sub}}}
\newcommand{\se}{\subseteq}
\newcommand{\Ax}[1]{\mathsf{(#1)}}
\def\Frms{\mathsf{Frames}}
\newcommand{\Models}{\mathop{\mathsf{Mod}}}
\def\Frames{\Frms}

\def\cF{\clF}

\newcommand{\LBP}{\vL^+}

\newcommand{\To}{\Rightarrow}
%\newcommand{\TO}{\Longrightarrow}

\def\bs{\mathbf s}
\newcommand\ded[1]{    [ #1 ]   }


\def\Var{\logicts{Var}}
\def\con{\wedge}

\newcommand\comm[1]\empty

\def\cl{\mathop{\mathrm{Cl}}}
\def\Int{\mathop{\mathrm{Int}}}

\def\emp{\varnothing}

\newcommand{\BLUE}[1]{\emph{\color{blue}#1}}
%\renewcommand{\BLUE}[1]{\emph{#1}}
\newcommand{\RED}[1]{\emph{\color{red}#1}}
\newcommand{\blue}[1]{\color{blue}#1}
\newcommand{\red}[1]{\color{red}#1}


\newcommand\cone[2]{{#1}\!\left[#2\right]}
\newcommand\ML{\logicts{ML}}
%\input{macroTACL17}

\newcommand\modelsts[1]{\framets{#1}}
\def\tiff{\quad \quad \textrm{iff} \quad \quad}
\def\C{C}
\def\B{\Box}
%\newcommand\subs[3]{#1^{(#2\lceil #3)}}
\newcommand\subs[3]{\mathbf{#1}}
\newcommand\BL[2]{\subs{B}{#1}{#2}}
\newcommand\CL[2]{\subs{C}{#1}{#2}}

\newcommand{\cM}{\mathcal{M}}

%%%
\newcommand\axiomts[1]{{\mathrm{#1}}}
%\def\Sym{\axiomts{.sym^{\!*}}}
\def\Sym{[1]}
\def\Di{\Diamond}
\def\vf{\varphi}
\def\lra{\leftrightarrow}
\newcommand{\clA}{\mathcal{A}}
\newcommand{\clB}{\mathcal{B}}
%\newcommand\framets[1]{\mathrm{{#1}}}
\newcommand\framets[1]{{#1}}
\newcommand\clP[1]{\mathcal{P}(#1)}
\def\frF{\framets{F}}
\def\frG{\framets{G}}
\def\frA{\framets{A}}
\def\EE{\exists}
\def\AA{\forall}
\def\Imp{\Rightarrow}
\newcommand{\clF}{\mathcal{F}}
\newcommand{\clC}{\mathcal{C}}
\def\clM{\mathcal{M}}
%\newcommand\logicts[1]{{\textsc{#1}}}
\def\clV{\mathcal{V}}
\newcommand{\Log}{\mathop{Log}}
\def\Nat{\mathbb{N}}
\def\h{\mathop{ht}}
%\def\Boxn{\Box_{<n}}
%\def\Din{\Di_{< n}}
%\def\Boxn{\Box}
%\def\Din{\Di}
\newcommand\Rw[2]{#2^{#1}}
\newcommand\Diw[2]{\Di^{#1}#2}
\newcommand\Boxw[2]{\Box^{#1}#2}
\def\mo{\vDash}
\def\imp{\rightarrow}
\def\vd{\vdash}
\newcommand\clust[1]{{\mathop{Clust}}(#1)}
\def\Lind{\mathfrak{A}}
\def\wK4{\logicts{wK4}}
%%%
%\def\vL{\axiomts{L}}
\def\vL{L}
%\newcommand\univ[1]{{#1}^{\mathop{u}}}
\newcommand\univ[1]{{#1}_{\mathop{u}}}
\newcommand\tense[1]{{#1}_{\mathop{t}}}
\def\restr{{\upharpoonright}}
\def\Alg{\mathop{Alg}}

\def\WS5{\logicts{WS5}}
\def\MIPC{\logicts{MIPC}}

\newcommand\logicts[1]{{\textsc{#1}}}
\newcommand{\LK}[1]{\logicts{K#1}}
\newcommand{\LS}[1]{\logicts{S#1}}
\def\LGL{\logicts{GL}}
\def\Grz{\logicts{Grz}}



\def\vf{\varphi}
\def\emp{\varnothing}

\def\clA{\classts{A}}
\def\clB{\classts{B}}
\def\clC{\classts{C}}
\def\clD{\classts{D}}





\newcommand\hide[1]{\empty}

\newcommand\todo[1]{\BLUE{\bf ToDo~  #1}}
\def\ToDo{\todo}

\newcommand\ISH[1]{{~\bf IS}: {\color{teal} #1}}
\renewcommand\ISH[1]\empty
\newcommand\DR[1]{{~\bf DR}: {\color{blue}#1}}
\renewcommand\DR[1]\empty




%%ISH: Something that does not seem to be relevant anymore (say, wrong conjecture)
\newcommand\obsolete[1]{{\color{darkgray}\noindent{\bf Obsolete:} #1}}
\renewcommand\obsolete[1]\empty

%%ISH: Something that I suggest to postpone
\newcommand\later[1]{{\color{lightgray}\noindent{\bf later:} {#1} }}
\renewcommand\later[1]\empty

%%%%%%%%%%%%%%%%%%%%%%%%%%%%%%%%%%%%%%%%%%%%%%%%%%%%%%%%%

%The following line defines the page header consisting of the surnames of the authors.
% Please include only the last names!
% Separate by commas except the last two surnames which are separated by an "and".
%\def\lastname{Rogozin, Shapirovsky}

\author{
  Daniel Rogozin \\
  University College London \\
  \texttt{relationalgebra@gmail.com}
  \and
  Ilya Shapirovsky \\
  New Mexico State University \\
  \texttt{ilshapir@nmsu.edu}
}
\date{}
\title{On decidable extensions of Propositional Dynamic Logic with Converse}

\begin{document}

\maketitle

  %\title[On decidable extensions of PDL]{On decidable extensions of Propositional Dynamic Logic}
%  \author[1]{\fnm{Daniel} \sur{Rogozin}}%\email{relationalgebra@gmail.com}
 % \affil[1]{Institute for Information Transmission Problems RAS}
  %\author[2]{\fnm{Ilya} \sur{Shapirovsky}}%\email{ilshapir@nmsu.edu}
  %\affil[2]{New Mexico State University}
  %\equalcont{Authors contributed equally to this work.}


%\bigskip

\begin{abstract}
We describe a family of decidable propositional dynamic logics,
where atomic modalities satisfy some extra conditions
(for example, given by axioms of the logics $\LK{5}$, $\LS{5}$, or $\LK{45}$
for different atomic modalities).
It follows from recent results \cite{KSZ:AiML:2014}, \cite{KikotShapZolAiml2020} that
if a modal logic $L$ admits a special type of filtration (so-called definable filtration),
then its enrichments with modalities for the transitive closure and converse relations also admit
definable filtration. We
use these results to show that if logics $L_1, \ldots , L_n$ admit definable filtration,
then the propositional dynamic logic with converse extended by the fusion $L_1*\ldots * L_n$
has the finite model property.

\hide{
consider the case when $L$ is the fusion $L_1*\ldots * L_n$ of logics, and show that
if all $L_i$ admit definable filtration, then the Propositional Dynamic
Logic (with converses) extended with the axioms of $L_1,\ldots, L_n $ has the finite model property. In particular,
it follows that if $L_i$ are finitely axiomatizable, then the corresponding extension of Propositional Dynamic
Logic is decidable.

\todo{..}
We apply this result to the case when
$L$ is the fusion $L_1*\ldots L_n$ of different logics (e.g., of ) to show that
the extension of Propositional Dynamic
Logic (with converses) with the axioms of $L$ is complete with respect to its standard finite models.

\todo{..}
Using recent
transfer results for
logics enriched with modalities for the transitive closure and converse relations \cite{KSZ:AiML:2014}, \cite{KikotShapZolAiml2020},
we show that many extensions of Propositional Dynamic
Logic (with converses) is complete with respect to its standard finite models.

\todo{..}We construct a family of decidable propositional dynamic logics, where atomic modalities satisfy some extra conditions (for example, given by axioms of the logics K5, S5, or K45 for different atomic modalities). Using recent transfer results about the finite model property for multimodal logics that admit filtration [Kikot, Shapirovsky, Zolin, AiML2020], we show that many extensions of Propositional Dynamic Logic (with converses) are complete with respect to its standard finite models.
}

\smallskip
\noindent
{\bf Keywords}
Propositional Dynamic Logic with Converse,
definable filtration,
fusion of modal logics,
finite model property,
decidability
\end{abstract}
%end of abstract


%\maketitle


\section{Introduction}
The Propositional Dynamic Logic with Converse is known
to be complete with respect to its standard finite models, and hence is decidable \cite{Parikh1978CPDL}.
We generalize this result for a family of normal extensions of this logic.

Let $\CPDL(\Al)$ be the
propositional dynamic logic with converse modalities, where $\Al$ indicates the set of atomic modalities.
For a set of modal formulas $\Psi$ in the language of $\Al$,
let $\CPDL(\Al)+\Psi$ be the normal extension of $\CPDL(\Al)$ with $\Psi$.


In \cite{KSZ:AiML:2014} and \cite{KikotShapZolAiml2020},
it was shown that
if a modal logic $L$ admits a special type of filtration (so-called definable filtration),
then its enrichments with modalities for the transitive closure and converse relations also admit
definable filtration.  In particular, it follows that if a logic $L$ admits definable filtration,
then $\CPDL(\Al)+L$ has the finite model property.

We will be interested in the case when $\CPDL(\Al)$ is extended by a fusion of logics $L=L_1*\ldots * L_n$.
For example, $\CPDL(\Di_1,\Di_2,\Di_3)+\LK{5}*\LK{45}*\LK{4}$ is the extension
of $\CPDL(\Al)$, where the first and second atomic modalities satisfy the principle $\Di p\imp \Box \Di p$,
the second and third satisfy $\Di\Di p\imp \Di p$.
We show in Theorem \ref{thm:mainTransferNew} that if the logics $L_i$ admit definable filtration, then
its fusion admits definable filtration as well. It follows that in this case
$\CPDL(\Al)+L_1*\ldots * L_n$ has the finite model property, and, if all $L_i$ are finitely axiomatizable,
$\CPDL(\Al)+L_1*\ldots * L_n$ is decidable (Corollary \ref{cor:main}).
Consequently, we have the following decidability result (Corollary \ref{cor:final}):
if each $L_i$
is
\begin{itemize}
\item
one of the logics
$\LK{},~\logicts{T},~\LS{4},~\LK{}+\{p\imp \Box\Di p\},~\LK{}+\{\Di\top\},~
\LK{4}+\{\Di\top\}$,
$\LK{} + \{\Diamond^m p \to \Diamond p\}$ for $m \geq 1$,  or
\item  locally tabular (e.g., $\LK{5},~\LK{45},~\LS{5}$, the difference logic), or
\item  a {\em stable logic} (defined in \cite{bezhanishvili2016stable}), or
\item axiomatizable by canonical MFP-modal formulas (defined in \cite{KikotShapZolAiml2020}),
\end{itemize}
then $\CPDL(\At)+L_1 * \ldots * L_n$ has the finite model property;
if also all $L_i$ are finitely axiomatizable, then $\CPDL(\At)+L_1 * \ldots * L_n$ is decidable.
Some particular instances of this fact (in the language without converse modalities) were known before:
for the case when each $L_i$ is a stable logic, it was announced in \cite{IlinAiML2016};
the case when each $L_i$ is axiomatizable by canonical MFP-modal formulas follows from \cite[Corollary 4.13]{KikotShapZolAiml2020}.

The paper is organized as follows. Section \ref{sec:prel} provides basic syntactic and semantic definitions.
Section \ref{sec:transfer} is an exposition of necessary transfer results from \cite{KSZ:AiML:2014} and \cite{KikotShapZolAiml2020}.
Main results (Theorem \ref{thm:mainTransferNew}, Corollary \ref{cor:main}, and Corollary \ref{cor:final}) are given in Section \ref{sec:main}.

\medskip
 A preliminary report on some results of this paper was given in \cite{RogozinShapAiML2022}.

\section{Syntactic and semantic preliminaries}\label{sec:prel}

\paragraph{Normal logics and Kripke semantics.}
Let $\At$ be a set of {\em atomic modalities},
$\PV = \{ p_i \: \mid \: i < \omega \}$ a set of {\em propositional variables}. The
{\em set of modal $\At$-formulas} $\Fm(\At)$ is generated by the following grammar:

\begin{center}
    $\varphi ::= \bot \: \mid \: p    \: \mid \: (\varphi \to \psi) \: \mid \: \Diamond  \varphi$ \quad $(
    p\in \PV, \; \Di \in \At)$
\end{center}
Other connectives are defined in the standard way.
In particular, $\Box \varphi$ is $\neg \Diamond \neg \varphi$, and if a figure $\langle e\rangle$ is used for an atomic modality, then  $[e]$ abbreviates  $\neg \langle e\rangle\neg$.

A {\em (normal) modal $\At$-logic} is a set of formulas $L\subseteq\Fm(\At)$ such that:
\begin{enumerate}
\item $L$ contains all Boolean tautologies;
\item For all $\Di \in \Al$, $\Diamond \bot \leftrightarrow \bot \in L$ and $\Diamond  (p \lor q) \leftrightarrow \Diamond  p \lor \Diamond  q \in L$;
\item $L$ is closed under the rules of Modus Ponens,  uniform substitution,
and {\em monotonicity}:
$\vf \to \psi\in L$ implies
$\Di \vf \to \Di \psi\in L$
for all $\Di\in \At$.
\end{enumerate}
For an $\At$-logic $L$ and a set $\Psi$ of $\At$-formulas,
$L+\Psi$ is the least normal modal $\At$-logic that contains $L\cup \Psi$. As usual, the smallest normal unimodal logic is denoted by $\LK{}$.


An $\Al$-frame is a structure $F = (W, (R_\Di)_{\Di \in \Al})$,
where each $R_\Di$ is a binary relation on $W$.
A model on an $\Al$-frame
is a structure $M = (F, \vartheta)$,
where $\vartheta : \PV \to \clP{W}$, where $\clP{X}$
is the set of all subsets of $W$. The truth definition is standard:
\begin{itemize}
    \item $M, x \models p_i$ iff $x \in \vartheta(p_i)$;
    \item $M, x \not\models \bot$;
    \item $M, x \models \varphi \to \psi$ iff either $M, x \not\models \varphi$ or $M, x \models \psi$;
    \item $M, x \models \Diamond  \varphi$ iff there exists $y$ such that $xR_\Di y$ and  $M, y \models \varphi$.
\end{itemize}
We set $M \models \varphi$ iff $M, x \models \varphi$ for all $x$  in $M$, and
$F\mo \vf$ iff $M\mo\vf$ for all $M$ based on $F$; $\Log(F)$ is the set
$\{\vf\in \Fm(\At) \mid F\mo \vf\}$. For a class $\clF$ of frames, $\Log(\clF)=\bigcap\{\Log(F)\mid F\in\clF\}$.
A logic $L$ is {\em Kripke complete} iff
$L = \Log(\clF)$ for a class $\clF$ of frames. A logic $L$ has the {\em finite model property} iff
$L = \Log(\clF)$ for a class $\clF$ of finite frames.
\later{Remove unused notation}

For a logic $L$, $\operatorname{Mod}(L)$ is
the class of models such that $M \models L$, i.e., $M \models \varphi$ for all $\varphi \in L$.


\later{
If $L$ is a set of formulas, then $M \models L$ stands for $M \models \varphi$ for all $\varphi \in L$.
}
\later{
The class of $L$-frames, $\operatorname{Frames}(L)$ consists of frames that validate $L$. The logic of a class of frames $\mathcal{F}$, $\operatorname{Log}(\mathcal{F})$,
is the set of formulas valid in each of those frames.
}

\paragraph{Propositional Dynamic Logics.}
Let $\At$ be finite.
The set $\Prog(\At)$ ({\em ``programs''}) is generated by the following grammar:
\begin{center}
    $\langle e\rangle,\langle f \rangle ::= \Di \: \mid \: \langle e \cup f\rangle \: \mid \: \langle e \circ f\rangle \: \mid \: \langle e^+ \rangle$ \quad $(\Di\in \At, \; \langle e \rangle,\langle f\rangle \in\Prog(\At))$
\end{center}
\begin{remark} Our language of programs is test-free.
\end{remark}
\later{More details on (in)finiteness of the alphabet}


\begin{definition}
A {\em normal propositional dynamic $\At$-logic}
is a normal $\Prog(\At)$-logic
that contains the following formulas for all $\langle e \rangle, \langle f \rangle  \in \Prog(\At)$:
\begin{enumerate}%[\normalfont (1)]
\item[\bf A1] $\langle e \cup f \rangle p \leftrightarrow \langle e \rangle p \vee \langle f \rangle p$,
\item[\bf A2] $\langle e \circ f \rangle p \leftrightarrow \langle e \rangle \langle f \rangle p$,
\item[\bf A3] $\langle e \rangle p \to \langle e^{+} \rangle p$,
\item[\bf A4] $\langle e \rangle \langle e^{+} \rangle p \to \langle e^{+} \rangle p$,
\item[\bf A5] $\langle e^{+} \rangle p \to \langle e \rangle p \vee \langle e^{+} \rangle (\neg p \land \langle e \rangle p)$.
\end{enumerate}
The least normal propositional dynamic $\At$-logic
is denoted by $\PDL(\At)$.

\smallskip
We also consider dynamic logics  with converse modalities.
The set $\Prog_t(\At)$ is given by the following grammar:
\begin{center}
    $\langle e\rangle,\langle f \rangle ::= \Di \: \mid \: \langle e \cup f\rangle \: \mid \: \langle e \circ f\rangle \: \mid \: \langle e^+ \rangle \:\mid \langle e^{-1} \rangle$ \quad $(\Di\in \At, \; \langle e \rangle,\langle f\rangle\in\Prog_t(\At))$
\end{center}
\later{\todo{Is everything consistent with angles and boxes?}}
A {\em normal propositional dynamic  $\At$-logic with converse modalities}
is a normal $\Prog_t(\At)$-logic
such that
for all $\langle e \rangle, \langle f \rangle  \in \Prog_t(\At)$
contains the formulas {\bf A1}--{\bf A5}  and the formulas
\begin{enumerate}
\item[\bf A6] $p \to [e]\langle e^{-1} \rangle p$
\item[\bf A7] $p \to [e^{-1}]\langle e \rangle p$
\end{enumerate}
The smallest dynamic $\At$-logic with converses is denoted by $\CPDL(\At)$.
\end{definition}

The validity of formulas
{\bf A1}-{\bf A7} in a frame {$(W,(R_\Di)_{\Di\in \Prog_t(\At)})$}
is equivalent to the following identities:
\begin{eqnarray}
\label{eq:SegConditions}
&&R_{\langle e\circ f\rangle}=R_{\langle e\rangle}\circ R_{\langle f\rangle},~
R_{\langle {e\cup f}\rangle}=R_{\langle e\rangle}\cup R_{\langle f\rangle},~
R_{\langle e^+\rangle}=(R_{\langle e\rangle})^+,~\\
\label{eq:TempConditions}
&&R_{\langle e^{-1}\rangle }=(R_{\langle e\rangle})^{-1},
\end{eqnarray}
\later{More details, since we have a mixture: an $L$-model and frame conditions}
where $R^+$ denotes the transitive closure of $R$, $R^{-1}$ the converse of $R$;
models based of such frames are called {\em standard}; see, e.g., \cite[Chapter 10]{Goldblatt1992LogOfTime}.
It is known that $\CPDL(\At)$ is
complete with respect to its standard finite models \cite{Parikh1978CPDL}.
Our aim is to prove it for a family of extensions of $\CPDL(\At)$.

\section{Filtrations and decidable extensions of dynamic logic}\label{sec:transfer}

\subsection{Logics that admit definable filtration}
For a model $M = (W, (R_\Di)_{\Di \in \Al}, \vartheta)$ and a set of formulas $\Gamma$, put
\begin{center}
$x\sim_{\Gamma} y$ iff $\forall \psi \in\Gamma$ ($M,x\models \psi$ iff $M,y\models \psi$).
\end{center}
The equivalence $\sim_\Gamma$ is said to be {\em induced by $\Gamma$ in $M$}.

For $\varphi \in \operatorname{Fm}(\Al)$, let $\operatorname{Sub}(\varphi)$ be the set of all subformulas of $\varphi$. A set of formulas $\Gamma$ is {\em $\operatorname{Sub}$-closed}, if $\varphi \in \Gamma$ implies $\operatorname{Sub}(\varphi) \subseteq \Gamma$.
\begin{definition}
Let $\Gamma$ be a $\operatorname{Sub}$-closed
set of formulas.
A {\em $\Gamma$-filtration} of a model $M = (W, (R_\Di)_{\Di \in \Al}, \vartheta)$   is a model $\ff{M}=(\ff{W},(\ff{R}_\Di)_{\Di \in \Al},\ff{\theta})$
s.t.
\begin{enumerate}
\item $\ff{W}=W/{\sim}$ for some equivalence relation $\sim$ such that $\sim \;\subseteq \;\sim_\Gamma$, i.e.,
\begin{center}
$x\sim  y$ implies $\forall \psi\in\Gamma\; (M,x\models \psi \Leftrightarrow M,y\models \psi)$.
\end{center}
\item ${\ff{M},[x]\models p}$ iff ${M,x\models p}$ for all $p\in \Gamma$.
Here $[x]$ is the class of $x$ modulo $\sim$.
\item For all $\Di \in \Al$, we have ${(R_\Di)}_{\sim} \subseteq \ff{R}_\Di \subseteq  {(R_\Di)}_{\sim}^\Gamma$, where
$$
\begin{array}{ccl}
~[x]\,{(R_\Di)}_\sim\,[y] & \text{iff} & \exists x'\sim x\ \exists y'\sim
y\;
(x'\,R_\Di\,y'),
\smallskip \\
~[x]\,{(R_\Di)}_{\sim}^\Gamma\,[y] & \text{iff} & \forall \psi\;
   (\Diamond \psi\in \Gamma \: \& \: M,y\models\psi \Rightarrow M,x\models \Di\psi ).
\end{array}
$$
\end{enumerate}
The relations ${(R_\Di)}_\sim$ and ${(R_\Di)}_\sim^\Gamma$ on $\widehat{W}$ are called
the \emph{minimal} and the \emph{maximal filtered relations}, respectively.
\end{definition}

If $\sim\; =\; \sim_\Delta$ for some finite \ISH{``sub-closed'' is redundant} set of formulas ${\Delta\supseteq\Gamma}$,
then $\ff{M}$ is called a \emph{definable $\Gamma$-filtration} of the model~$M$.
If $\sim\; =\; \sim_\Gamma$, the filtration $\ff{M}$ is said to be {\em strict}.

The following fact is well-known, see, e.g., \cite{Ch:Za:ML:1997}:
\begin{lemma}[Filtration lemma]\label{Lemma:Filtration}
Suppose that $\Gamma$ is a finite $\operatorname{Sub}$-closed set of formulas
and $\ff{M}$ is a $\Gamma$-filtration of a model~$M$.
Then, for all points ${x\in W}$ and all formulas ${\varphi \in\Gamma}$,
we have: \ \
\begin{center}
${M,x\models \varphi}$ iff ${\widehat{M},[x]\models\varphi}$.
\end{center}
\end{lemma}

\begin{definition}
We say that a class $\mathcal{M}$  of Kripke models \emph{admits  definable (strict) filtration}
iff for any ${M \in \mathcal{M}}$
and for any finite $\operatorname{Sub}$-closed set of formulas~$\Gamma$,
there exists a finite model in $\mathcal{M}$ that is a definable (strict) $\Gamma$-filtration of~$M$.
A logic {\em admits definable (strict) filtration} iff the class
$\operatorname{Mod}(L)$
of its models does.
\end{definition}

It is immediate from the Filtration lemma
that if a logic admits filtration, then it is complete with respect to the class of its finite models, and consequently, to
the class of its finite frames (see, e.g., \cite[Theorem 3.28]{B:R:V:ML:2002}).

Strict filtrations are the most widespread in the literature;
for example, it is well-known that the logics
$\LK{},~\logicts{T},~\LK{4},~\LS{4},~\LS{5}$ admit strict filtration,
 see e.g., \cite{Ch:Za:ML:1997}.  Constructions where the initial equivalence is refined
 were also used since late 1960s \cite{Segerberg1968}, \cite{Gabbay:1972:JPL}, and
 later, see, e.g., \cite{Shehtman:AiML:2004}.\later{improve}
 Refining the initial equivalence makes the filtration method much more flexible.
 For example, it is not difficult to see that the logic $\LK{5}={\LK{}+\{\Di p\imp \Box \Di p\}}$ does not admit strict filtration.
 However, $\LK{5}$ admits definable filtration, see, e.g., \cite[Theorem 5.35]{Ch:Za:ML:1997}. Another explanation is that
 $\LK{5}$ is locally tabular \cite{nagle_thomason_1985},
 and every locally tabular logic admits definable filtration, see Section \ref{sec:LTimpADF} for details.
\later{\ToDo{K5 is LT: Double Check the ref}; seems to be ok, see Corollary 5}



\subsection{Transfer of admits filtration property}
In \cite{KSZ:AiML:2014} and \cite{KikotShapZolAiml2020}, definable filtrations were used to obtain
transfer results for logics enriched with modalities for the transitive closure
and converse relations. Let $\langle e\rangle\in \Al$.
For an $\Al$-logic $L$, let $\trans{L}{\langle e\rangle}$ be the extension
of the logic
$L$ with axioms {\bf A3}, {\bf A4}, and {\bf A5}, and
let $\temp{L}{\langle e\rangle}$ be the extension of $L$ with the axioms
{\bf A6} and {\bf A7}.

For an $\Al$-model $M=(W,(R_\Di)_{\Di\in\Al},\val)$,
let $\temp{M}{\langle e\rangle}$ be its enrichment  with the converse of $R_a$:
$$\temp{M}{\langle e\rangle}=(W,(R_\Di)_{\Di\in\Al},(R_{\langle e\rangle})^{-1}, \val);$$
similarly,  $\temp{M}{\langle e\rangle}$ denotes the enrichment of $M$ with the transitive closure of $R_{\langle e\rangle}$:
$$\trans{M}{\langle e\rangle}=(W,(R_\Di)_{\Di\in\Al},(R_{\langle e\rangle})^+,\val).$$
\hide{
For a class $\clM$ of $\Al$-models, $\temp{\clM}{\langle e\rangle}=\{\temp{M}{\langle e\rangle}\mid M\in\clM\}$;
likewise for $\trans{\clM}{\langle e\rangle}$.   }
It is straightforward from \eqref{eq:SegConditions}
and
\eqref{eq:TempConditions}
that if $M$ is an $L$-model, then
$\trans{M}{\langle e\rangle}$ is a model of $\trans{L}{\langle e\rangle}$,
and $\temp{M}{\langle e\rangle}$ is a model of $\temp{L}{\langle e\rangle}$.

Assume that a logic $L$ admits definable filtration.
In \cite[Theorem 3.9]{KikotShapZolAiml2020}, it was shown that in this case
the logic $\trans{L}{\langle e\rangle}$ admits definable filtration as well.
This crucial result implied that
$\PDL(\At)+L$ has the finite model property, and if also $L$ is finitely axiomatizable, then
$\PDL(\At)+L$ is decidable \cite[Theorem 4.6]{KikotShapZolAiml2020}.

If follows from
\cite[Theorem 2.4]{KSZ:AiML:2014} that if $\vL$ admits definable filtration,
then so does $\temp{L}{\langle e\rangle}$.
\begin{remark}
Theorem \cite[Theorem 2.4]{KSZ:AiML:2014} was formulated for frames, not for models; however, the definable filtrations given
in the proof of this theorem work for models without any modification.
\end{remark}

\obsolete{
The following theorem is a corollary of
\cite[Theorem 4.6]{KikotShapZolAiml2020} and the proof of \cite[Theorem 2.4]{KSZ:AiML:2014}.
\ToDo{details}
\ISH{This requires some more work, since
 \cite[Theorem 2.4]{KSZ:AiML:2014} was given for different typo of filtrations. It is on me.
}
}
\begin{theorem}[\cite{KikotShapZolAiml2020},\cite{KSZ:AiML:2014}] \label{thm:ADFforCPDL}
Let  $\AlB$  be a subset of a finite\later{``finite'' should be recursive for A, and finite for B}
set $\Al$.
If a $\AlB$-logic $L$ admits definable filtration, then
$\CPDL(\At)+L$ has the finite model property.
If also $L$ is finitely axiomatizable, then $\CPDL(\At)+L$ is decidable.
\end{theorem}
\later{``$A$ is finite'': does it matter?}


\section{Filtrations for fusions}\label{sec:main}

\subsection{Fusions}
Let $L_1, \ldots, L_n$ be logics in languages
%$\Fm_1, \dots, \Fm_n$
that have mutually disjoint sets of modalities. The {\em fusion} $L_1 * \ldots * L_n$ is the smallest logic that contains $L_1, \ldots, L_n$.
We adopt the following convention: for logics $L_1,\ldots, L_n$ in the same language,
we also write $L_1 * \ldots * L_n$  assuming that we ``shift'' modalities; e.g.,
 $\LK{5}*\LK{5}$ denotes the bimodal logic given by the two axioms
 $\Di_i p\imp \Box_i\Di_i p$, $i=1,2$. \later{$i=0,1$}

It is known that the fusion of consistent modal logics is a conservative extension of its components \cite{thomason1980independent}.
Also, the fusion operation preserves
Kripke completeness, decidability, and the finite model property \cite{kracht1991properties,fine1996transfer,wolter1996fusions}.
\later{\ToDo{Double check the second ref}}

In \cite{KikotShapZolAiml2020}, it was noted that if canonical logics $L_1,\ldots, L_n$ admit strict filtration,
then the fusion $L=L_1*\ldots *L_n$ admits strict filtration; it follows from
Theorem \ref{thm:ADFforCPDL} that $\CPDL(\At)+L$ has the finite model property for the case of such $L$.

\begin{example}
The logic $\CPDL(\Di_1,\Di_2)+\LS{4}*\LS{5}$ has the finite model property and decidable.
\end{example}
\hide{
\todo{redo}
Moreover, this observation can be generalized as follows:
if $L$ admits strict filtration and $L'$ admits definable filtration, than
the fusion $L*L'$ admits definable filtration.


\begin{example}
The logic $\CPDL(\Di_1,\Di_2,\Di_3)+\LS{4}*\LS{5}*\LK{5}$ has the finite model property and decidable.
\end{example}
}

This does not cover many important examples where logics $L_i$ do not admit strict filtration
(like in the case of the logic $\LK{5}*\LK{5}$).
Below we show that the admits definable filtration  property
preserves under the operation of fusion, which allows
extending applications of Theorem \ref{thm:ADFforCPDL} significantly.

\subsection{Main result}
Recall that a set of formulas $\Psi$ is {\em valid} in a modal algebra $B$, in symbols $B\mo \Psi$, iff
$\vf=1$ holds in $B$ for every $\vf\in \Psi$.

For a model $M=(W,(R_\Di)_{\Di\in \Al},\val)$ and an $\Al$-formula, put $$\valext{\vf}=\{x\mid M,x\mo\vf\}.$$
Let $D(M)=\{\valext{\vf}\mid \vf\in \Fm(\At)\}$
 be the set of definable subsets of $M$, considered as a Boolean subalgebra of the powerset algebra $\clP{W}$, and
let
$\Alg(M)$ be the modal algebra  $(D(M),(f_\Di)_{\Di\in \Al})$, where
%$f_\Di$ are defined as follows:
$f_\Di(V)=R_\Di^{-1}[V]$ for $V\subseteq W$. \later{\todo{$R^{-1}$}}
The following fact is standard: if $L$ is a logic, then
\begin{equation}\label{eq:model-alg}
M\mo L \text{ iff } \Alg(M)\mo L
\end{equation}
(``if'' is trivial, ``only if'' follows from the fact that logics are closed under substitutions).
If $M'=(W,(R_\Di)_{\Di\in \Al},\val')$ is a model such that
$\val'(p)\in D(M)$ for all variables $p$, then
it follows from \eqref{eq:model-alg} that
\begin{equation}\label{eq:robustmodel}
\text{if $M\mo L$, then $M'\mo L$};
\end{equation}
indeed, $\Alg(M')$ is a subalgebra of $\Alg(M)$.



\begin{proposition}\label{prop:extensible-filtr}
Let $\Gamma$ be a $\operatorname{Sub}$-closed
set of formulas,
$M = (W, (R_\Di)_{\Di \in \Al}, \vartheta)$ a model. If
$\ff{M}=(W/{\sim_\Delta},(\ff{R}_\Di)_{\Di \in \Al},\ff{\theta})$ is
a $\Gamma$-filtration of $M$ for some finite set of formulas $\Delta$, then for every equivalence $\sim$ on $W$ finer than $\sim_\Delta$
there exists a $\Gamma$-filtration $\ff{M}'$ of $M$ such that $W/{\sim}$ is the carrier of $\ff{M}'$, and moreover
\begin{equation}\label{eq:extensible-filtr}
\ff{M}\mo\vf \text{ iff } \ff{M}'\mo \vf.
\end{equation}
for every $\vf\in\Fm(\Al)$.
\end{proposition}
\begin{proof}
%We define the relations $\widehat{R_\Di}'$ and the valuation
%$\hat{\theta}'$
%in the model $\widehat{M}'$ as follows.
Since $\sim \;\subseteq \;\sim_\Delta$, for every $u \in W/{\sim}$ there exists a unique
element of $\widehat{M}$ that contains $u$; we denote it by
$u \Delta$. The binary relations $\ff{R}_\Di'$ in $\ff{M}'$
and the valuation
$\ff{\theta}'$
are defined as follows:
\begin{align*}
% \nonumber % Remove numbering (before each equation)
    &\ff{R}_\Di'=\{(u,v) \mid  (u \Delta,v \Delta)\in  \ff{R}_\Di \};\\
    &\ff{\theta}'(p)=\{u\in W/{\sim}\mid u\subseteq \ff{\theta}(p)\} \text{ for } p\in \PV.
\end{align*}

By a straightforward induction on the structure of $\vf$,
\begin{equation}\label{eq:uDelta}
\ff{M}',u\mo\vf \text{ iff } \ff{M},u\Delta \mo \vf.
\end{equation}
Now \eqref{eq:extensible-filtr} follows.

Trivially, $\sim\;\subseteq \;\sim_\Gamma$. The second filtration condition
follows from the definition of $\ff{\theta}'$.  Let $\Di\in\Al$.
For $x\in W$, let $[x]_\Delta$  and $[x]$ be the $\sim_\Delta$- and $\sim$-classes of $x$, respectively.
If $xR_\Di y$, then $[x]_\Delta \ff{R}_\Di [y]_\Delta$, because $\ff{M}$ is a filtration of $M$;
now $[x] \ff{R}_\Di' [y]$  by the definition of $\ff{R}_\Di'$.
That $\ff{R}'_\Di$ is contained in the maximal filtered relation follows from \eqref{eq:uDelta}.
\end{proof}
\begin{remark}
In the above proof,
the map $u\mapsto u\Delta$ is a {\em p-morphism} of models, and
the algebras $\Alg{\ff{M}}$ and $\Alg{\ff{M}'}$ are isomorphic.
\end{remark}
\later{This proposition is a version of a construction called {\em refinement}, see, e.g., \cite[Chapter 8]{CZ}.  }

The following is a generalization of \cite[Theorem 4.9]{KikotShapZolAiml2020}.
\begin{theorem}\label{thm:mainTransferNew}~ %Let $L_1$ and $L_2$ be logics.
  If logics $L_1$ and $L_2$ admit definable filtration, so does $L_1*L_2$.
\end{theorem}
\begin{proof}
Let $\Al$ and $\AlB$ be   disjoint\later{improve} alphabets of modalities of the logics $L_1$ and $L_2$, respectively.


  Consider an $L_1*L_2$-model $M  = (W, (R_\Di)_{\Di\in \Al}, (R_\Di)_{\Di\in \AlB}, \vartheta)$,
    and a finite $Sub$-closed set of formulas $\Gamma\subset \Fm(\Al\cup\AlB)$.
    Consider a set of fresh variables $V = \{ q_{\varphi} \mid \varphi \in \Gamma \}$, and
    define a valuation $\eta$ in $W$ as follows:
%for $p\in \Gamma$, let $\val'(p)=\val(p)$\todo{?};
for $q_\vf\in V$, let $\eta(q_\vf)=\valext{\vf}$; otherwise, put  $\eta(q)=\emp$.
Let $M_V=(W, (R_\Di)_{\Di\in \Al}, (R_\Di)_{\Di\in \AlB}, \eta)$.
We have:
\begin{equation}\label{eq:red0}
  D(M_V)\subseteq D(M),
\end{equation}
and  by \eqref{eq:robustmodel},
\begin{equation}\label{eq:red1}
M_V\mo L_1*L_2.
\end{equation}
Consider the $\Al$- and $\AlB$-reducts of  $M_V$:
$$M_\Al=(W, (R_\Di)_{\Di\in \Al}, \eta),\quad M_\AlB=(W, (R_\Di)_{\Di\in \AlB}, \eta).$$
It follows from \eqref{eq:red1} that
\begin{equation}\label{eq:red2}
M_\Al\mo L_1, \quad M_\AlB\mo L_2.
\end{equation}
Consider the following sets of formulas:
$$\Gamma_\Al = V \cup \{ \Diamond q_{\varphi} \mid \Diamond \varphi \in\Gamma\,\& \,\Diamond   \in \Al\}, \quad
\Gamma_\AlB = V \cup \{ \Diamond q_{\varphi} \mid \Diamond \varphi \in\Gamma\,\& \,\Diamond   \in \AlB \}.$$
Since logics $L_1$ and $L_2$ admit definable filtration, there are finite sets $\Delta_\Al$ and
$\Delta_\AlB$ of formulas, and models $\ff{M}_\Al$, $\ff{M}_\AlB$ such that
\begin{align}
&\ff{M}_\Al\mo L_1,& &\ff{M}_\AlB\mo L_2,  \label{eq:red3} \\
&\Gamma_\Al\subseteq \Delta_\Al\subset \Fm(\Al),&
&\Gamma_\AlB\subseteq \Delta_\AlB\subset \Fm(\AlB),& \\
&\text{$\ff{M}_\Al$ is a $\Gamma_\Al$-filtration of $M_\Al$},&
&\text{$\ff{M}_\AlB$ is a $\Gamma_\AlB$-filtration of $M_\AlB$},& \\
&\text{the carrier of $\ff{M}_\Al$ is $W/{\sim_\Al}$,}&
&\text{the carrier of $\ff{M}_\AlB$ is $W/{\sim_\AlB}$,}&
\end{align}
where $\sim_\Al$ is the equivalence on $W$ induced by $\Delta_\Al$ in $M_\Al$,
and  $\sim_\AlB$ is the equivalence on $W$ induced by $\Delta_\AlB$ in $M_\AlB$.
Let $\sim$ be the equivalence $\sim_\Al\cap \sim_\AlB$.
By Proposition \ref{prop:extensible-filtr} and \eqref{eq:red3}, there are models
$\ff{M}_\Al'$ and $\ff{M}_\AlB'$ whose carrier is $W{/}{\sim}$ such that
\begin{align}
&\ff{M}_\Al'\mo L_1,& &\ff{M}'_\AlB\mo L_2,  \label{eq:red3-1}  \\
&\text{$\ff{M}'_\Al$ is a $\Gamma_\Al$-filtration of $M_\Al$},& \label{eq:red3-2}
&\text{$\ff{M}'_\AlB$ is a $\Gamma_\AlB$-filtration of $M_\AlB$}.&
\end{align}
Notice that $\Gamma_\Al$ and $\Gamma_\AlB$ contain the same variables, namely $V$.
Hence, the value of any variable in $V$ is the same in $\ff{M}_\Al'$ as in $\ff{M}_\AlB'$.
Also,  we can assume that
the values of variables not in $V$ are empty in these models: making them empty
does not affect \eqref{eq:red3-1} by \eqref{eq:robustmodel}, and \eqref{eq:red3-2}  by the definition of filtration. Consequently, we can assume that $\ff{M}'_\Al$ and $\ff{M}'_\AlB$ have the same valuation:
\begin{equation}
\ff{M}_\Al' = (W/{\sim},(\ff{R}_\Di)_{\Di\in\Al},\ff{\eta}),\quad
\ff{M}_\AlB' = (W/{\sim},(\ff{R}_\Di)_{\Di\in\AlB},\ff{\eta}).
\end{equation}
By \eqref{eq:red3-2},  the model
\begin{equation}\label{eq:ref-Mv}
\ff{M}_V = (W/{\sim},(\ff{R}_\Di)_{\Di\in\Al}, (\ff{R}_\Di)_{\Di\in\AlB},\ff{\eta}) \text{ is a $(\Gamma_\Al\cup \Gamma_\AlB)$-filtration of $M_V$}.
\end{equation}
By \eqref{eq:red3-1},
\begin{equation}\label{eq:red4}
  \ff{M}_V \mo L_1*L_2.
\end{equation}
Finally, let $\ff{M}=(W/{\sim},(\ff{R}_\Di)_{\Di\in\Al}, (\ff{R}_\Di)_{\Di\in\AlB},\ff{\theta})$,
where for a variable $p\in \Gamma$, $\ff{\theta}(p)=\ff{\eta}(q_p)$.
By \eqref{eq:red4} and \eqref{eq:robustmodel},
$$
\ff{M} \mo L_1*L_2.
$$

Let us show that $\ff{M}$ is a definable $\Gamma$-filtration of $M$.

First, observe that $\sim$ is induced in $M_V$ by the set $\Delta_\Al\cup\Delta_\AlB$, and so it is induced in $M$ by a set of formulas according to (\ref{eq:red0}).
Since $V\subseteq \Delta_\Al\cup\Delta_\AlB$, the equivalence $\sim$ refines
the equivalence $\sim_\Gamma$ induced in $M$ by $\Gamma$.

Let $\Di\in \Al\cup\AlB$.
That $\ff{R}_\Di$ contains the corresponding minimal filtered relation follows from   \eqref{eq:ref-Mv}.
Let us  show that $\ff{R}_\Di$ is contained in the maximal filtered relation $(R_\Di)_\sim^\Gamma$.
 Notice that by the definition of $\eta$, for every $\vf\in\Gamma$, $z\in W$,
 \begin{equation}\label{eq:red-eta-di}
 M_V,z\mo q_\vf \text{ iff } M,z\mo \vf, \text{ and hence }
 M_V,z\mo \Di q_\vf \text{ iff } M,z\mo \Di \vf.
 \end{equation}
Consider $\sim$-classes $[x]$, $[y]$ of $x,y\in W$, and assume that $\Di\vf\in \Gamma$ and $M,y\mo \vf$.
By \eqref{eq:red-eta-di},  $M_V,y\mo q_\vf$.
We have $\Di q_\vf\in \Gamma_\Al\cup\Gamma_\AlB$, so by \eqref{eq:ref-Mv}, $M_V,x\mo \Di q_\vf$. By \eqref{eq:red-eta-di} again, $M,x\mo\Di\vf$.
\end{proof}



\begin{example}
By the above theorem, $\LK{5}*\LK{5}$ admits definable filtration.
Consequently, the logic $\CPDL(\Di_1,\Di_2)+\LK{5}*\LK{5}$ has the finite model property and decidable.
\end{example}
\begin{remark}
Dynamic logics based on atomic modalities satisfying $\LK{5}$ are considered in the context of
epistemic logic and logical investigation of game theory, see, e.g., \cite{FittingGames2011}
(in this context, the axiom $\Di p\imp \Box \Di p$ is usually addressed as {\em negative introspection}).
\end{remark}

From Theorems \ref{thm:ADFforCPDL} and \ref{thm:mainTransferNew}, we obtain:


\begin{corollary}\label{cor:main}
Let $\Al$ be a finite set,  $L_1, \ldots, L_n$ be logics
such that $L_1 * \ldots * L_n\subseteq \Fm(\Al)$.
If $L_1, \ldots, L_n$ admit definable filtration, then $\CPDL(\At)+L_1 * \ldots * L_n$ has the finite model property.
If also $L_1, \ldots, L_n$ are finitely axiomatizable,
then $\CPDL(\At)+L_1 * \ldots * L_n$ is decidable.
\end{corollary}

\subsection{Examples}\label{sec:LTimpADF}
Corollary \ref{cor:main} can be applied for a number of modal logics.
As we mentioned, for the logics
$\LK{},~\logicts{T},~\LK{4},~\LS{4},~\LS{5}$ strict filtration are well-known, as well as for logics
$\LK{}+\{p\imp \Box\Di p\},~\LK{}+\{\Di\top\}, ~
\LK{4}+\{\Di\top\}$, and many others,
 see e.g., \cite{Ch:Za:ML:1997}.\later{improve}

 In fact, there is a continuum of modal logics that admit strict filtration.
In \cite{bezhanishvili2016stable}, a family of modal logics called {\em stable} was introduced.
Logics $\logicts{T}$ or $\LK{}+\{\Di\top\}$ are examples of stable logics.
Every stable logic admits strict filtration, which follows
from \cite[Theorem 7.8]{bezhanishvili2016stable}, and  there are
continuum many stable logics \cite[Theorem 6.7]{bezhanishvili2018stable}.

\begin{remark}
Stable logics were also used to construct decidable extensions of $\PDL$.
Namely, in \cite{IlinAiML2016}, it was announced that
extensions of $\PDL$ with axioms of stable logics have the finite model property.\later{DC with Nick and Ilin}
%Namely, in \cite{IlinAiML2016}, it was announced that $\PDL$-extensions of stable logics have the finite model property.\later{DC with Nick and Ilin}
\end{remark}

Another class of logics
 that admit strict filtration
 are logics given by canonical {\em MFP-modal formulas} introduced in \cite[Section 4.2]{KikotShapZolAiml2020}.

\bigskip


There are logics that do not admit strict filtration, but admit definable filtrations.
Consider the family of logics $\LK{} + \{ \Di^m p \to \Di p\}$ for $m \geq 3$.
These logics are Kripke complete, and their frames are characterized by the  conditions
\begin{equation}\label{eq:pretrM}
\forall x \, \AA  y \: (x R^m y \Rightarrow x R y);
\end{equation}
moreover, all these logics admit definable filtration \cite[Theorem 8]{Gabbay:1972:JPL}:
for a given $\Gamma$ and a model, the required filtration is build on the quotient of the model
by $\sim_\Delta$, where $\Delta=\{\Di^m\vf \mid \vf \in\Gamma\}$.\later{DC}
However, these logics do not admit strict filtration.
We will illustrate it with the case when $m = 3$, one can extend it for any greater $m$.
\begin{example}
$L = {\bf K} + \{\Diamond \Diamond \Diamond p \to \Diamond p\}$ does not admit strict filtration.
\end{example}

\begin{proof}
  Consider a five-element model $M = (W, R, \vartheta)$, where the binary relation is defined by the following figure
\medskip

    \xymatrix@R-0.6cm{
  &&&& x \ar[r] & y \\
  &&&&& y' \ar[r] & z \ar[r] & u
  }
\medskip
\noindent
($R$ is assumed to be irreflexive),
and %the valuation $\vartheta$ is defined as
$$
\vartheta(p) = \{ x \}, \quad \vartheta(q) = \{ y, y' \}, \quad \vartheta(r) = \{ u \}.
$$
By \eqref{eq:pretrM}, the frame of $M$ validates  $\Diamond \Diamond \Diamond p \to \Diamond p$,
and so $M$ is a model of the logic $L$.
Let $\Gamma = \{p, q , r, \Di r \}$. Assume
that $\ff{M} = (W{/}{\sim_\Gamma}, \ff{R}, \ff{\vartheta})$ is a $\Gamma$-filtration of $M$ and show that
$\ff{M}$ is not an $L$-model.
Notice  that $y$ and $y'$ are $\sim_\Gamma$-equivalent, and hence the quotient $W{/}{\sim_\Gamma}$
consists of four elements $[x],[y](=[y']),[z],[u]$.
Since $\ff{R}$ contains the minimal filtered relation, we have $[x]\ff{R}[y]\ff{R}[z] \ff{R} [u]$.
We have $\ff{M},[u]\mo r$ by the definition of filtration, and so $\ff{M},[x]\mo \Di\Di\Di r$.
For the sake of contradiction, assume that $\ff{M}\mo L$. In this case, $\ff{M},[x]\mo \Di r$.
Since $\Di r\in\Gamma$ and $\ff{M}$ is a $\Gamma$-filtration of $M$,
we have $M,x\mo \Di r$, which contradicts the definition of $M$. Hence $\ff{M}$ is not an $L$-model.
\end{proof}


\ISH{``Locally tabular logics are a continuum of logics that admit definable filtration as well.'' - I think it is a confusing formulation: ``Locally tabular logics are''}
A continuum of logics that admit definable filtration are locally tabular logics.
Recall that a logic $\vL$ is {\em locally tabular},
if, for every finite $k$, $\vL$ contains only a finite number of pairwise nonequivalent
formulas in a given $k$ variables.
Well-known examples of locally tabular modal logics are $\LK{5}$ \cite{nagle_thomason_1985} and so its extensions (e.g., $\LK{45}$, $\LS{5}$),
or the {\em difference logic} $\LK{}+\{p\imp \Box\Di p,~\Di\Di p\imp \Di p\vee p\}$ \cite{esakia2001weak}.
\later{DC ref for Esakia;
see
Simple weakly transitive modal algebras
https://go.gale.com/ps/i.do?id=GALE%7CA314348065&sid=googleScholar&v=2.1&it=r&linkaccess=abs&issn=00025232&p=AONE&sw=w&userGroupName=nm_p_oweb&isGeoAuthType=true
}


%In \cite{Shehtman:AiML:2014}, a special kind of definable filtrations was described.
%Using a special type of definable filtrations propo


Let $M=(W,(R_\Di)_{\Di\in\Al},\theta)$ be a model of a locally tabular logic $L$,
$\Gamma\subset \Fm(\Al)$ a finite $\Sub$-closed set of formulas. %in the alphabet of $L$.
Let $V$ be the set of all variables occurring in $\Gamma$,
and let $\Delta$ be the set of all $\Al$-formulas with variables in $V$. % (in the alphabet of $L$).
Let $\fkCan{V}$ be the canonical frame of $L$ built from
maximal $\vL$-consistent subsets of $\Delta$; the canonical relations are defined in the standard way.
%Notice that $\fkCan{V}$ is finite due to local tabularity of $L$.
Consider the maximal filtration $\ff{M}$ of $M$ through $\Delta$ with the carrier $W/{\sim_\Delta}$;
in \cite{Shehtman:AiML:2014},
such filtrations are called {\em canonical}.
Since $L$ is locally tabular, $\ff{M}$ is finite. The frame $\ff{F}$ of $\ff{M}$
is isomorphic to a generated subframe
of the canonical frame $\fkCan{V}$ of the logic $L$, see, e.g.,
\cite{Shehtman:AiML:2014} for details.\later{DC}
Since $L$ is locally tabular, $\fkCan{V}$ is finite, and so
$\fkCan{V}\mo L$. It follows that $\ff{M}\mo L$, as required.\later{more details} Hence, we have
\obsolete{
$\ff{M}$ is a p-morphic image of $M$ (for details, see \cite[Proposition 2.32]{Shehtman:AiML:2014}).
\ISH{This ref is not exact. The argument must be that the frame of M is a p-morphic image of a generated subframe of the k-canonical frame}
}
\begin{theorem}[Corollary from \cite{Shehtman:AiML:2014}]\label{thm:LFimpliesADF}
If $L$ is locally tabular, then $L$ admits definable filtration.
\end{theorem}

\hide{
Note that if a logic is locally tabular, it does not have to admit strict filtration, the example is the logic $~\LK{5}$, which is locally finite, but it does not admit strict filtration though.
}
%The proof of this theorem is based %on methods proposed in %\cite{Shehtman:AiML:2014}.



Putting the above examples together, we obtain the following instance of Corollary \ref{cor:main}.
\begin{corollary}\label{cor:final}
Let $\Al$ be a finite set,  $L_1, \ldots, L_n$ be logics
such that $L_1 * \ldots * L_n\subseteq \Fm(\Al)$.
If each $L_i$
is
\begin{itemize}
\item
one of the logics
$\LK{},~\logicts{T},~\LS{4},~\LK{}+\{p\imp \Box\Di p\},~\LK{}+\{\Di\top\},~
\LK{4}+\{\Di\top\}$,
$\LK{} + \{\Diamond^m p \to \Diamond p\}$ for $m \geq 1$,  or
\item  locally tabular (e.g., $\LK{5},~\LK{45},~\LS{5}$, the difference logic), or
\item  a stable logic, or
\item axiomatizable by canonical MFP-modal formulas,
\end{itemize}
then $\CPDL(\At)+L_1 * \ldots * L_n$ has the finite model property.
If also all $L_i$ are finitely axiomatizable, then $\CPDL(\At)+L_1 * \ldots * L_n$ is decidable.
\end{corollary}

\obsolete{
\begin{corollary}
Let $n$ and $\At$ be finite, $L_1, \ldots, L_n, L$ be modal logics
such that $L_1 * \ldots * L_n *L \subseteq \Fm(\At)$.
\begin{enumerate}
\item
If $L_1, \ldots, L_n, L$ admit definable filtration, then $\CPDL(\At)+L_1 * \ldots * L_n*L$ has the finite model property.
\item If each $L_i$
is one of the logics
$\LK{},~\logicts{T},~\LK{4},~\LK{5},~\LK{45},~\LS{4},~\LS{5},~\LK{}+\{p\imp \Box\Di p\},~\LK{}+\{\Di\top\},
\LK{4}+\{\Di\top\}$,
$\LK{} + \{\Diamond^m p \to \Diamond p\}$ for $m \geq 1$, or $L_i$or any stable logic
and $L$ admits definable filtration,
then  $\CPDL(\At)+L_1 * \ldots * L_n*L$ has the finite model property.
If also $L$ and each $L_i$ are finitely axiomatizable, then $\CPDL(\At)+L_1 * \ldots * L_n*L$ is decidable.
\end{enumerate}
\end{corollary}
}

\section{Acknowledgement}

The authors thank Nick Bezhanishvili for valuable discussions.

\bibliographystyle{alpha}
\bibliography{filtration}
\end{document}

\newpage

\section{Old material}


\obsolete{
\begin{proof}
\ISH{Proof of fusions for locally tabular? }
\ISH{Like in Theorem 4.8, with the following key observation:
After we obtain filtrations for all components, take the corresponding refinement;
on the resulting refined partition  take the filtered relations
induced by relations on bigger classes; then these models (in component languages) are p-morphic images
of initial filtrations.
}\ISH{Update: no proof known}
\end{proof}
}


\smallskip


\bigskip
\todo{obsolete; remove after proving the above theorem}
To obtain further results on filtrability of fusions, we introduce the following concept, which allows
to significantly extend applications of Theorem \ref{thm:ADFforCPDL}.



\begin{definition}\label{Def:Admit:Filtration:Models}
We say that a class $\mathcal{M}$ of Kripke models  \emph{admits extensible filtration} %(AEF for brevity)
iff
for any ${M \in \mathcal{M}}$
and for any finite $\operatorname{Sub}$-closed set of formulas~$\Gamma$,
there is a finite $\operatorname{Sub}$-closed set of formulas $\Delta$
such that for every
finite $\operatorname{Sub}$-closed set of formulas $\Lambda\supseteq\Delta$
there exists a $\Gamma$-filtration
$\widehat{M}$ of $M$
which carrier is $W/{\sim_\Lambda}$ and $\widehat{M} \in \mathcal{M}$.
A logic {\em admits extensible filtration} iff the class
$\operatorname{Mod}(L)$
of its models does.
\end{definition}


The following is immediate from definitions.
%facts follow from the definition of AEF directly.

\begin{proposition} Let $\clM$ be a class of Kripke models.
\begin{enumerate}
\item If $\clM$ admits strict filtration, then
 $\clM$ admits extensible filtration.
\item If $\clM$ admits extensible filtration, then $\clM$ admits definable filtration.
\end{enumerate}
\end{proposition}

\todo{Move counterexample here}
\ISH{After I will complete the proof, we can say here that there AEF is stronger that ADF}


The logic $\LK{}+\{\Di \Di \Di p \imp \Di p\}$ is an example of the logic that admits
extensible filtration: this follows from the proof of Theorem  8 in \cite{Gabbay:1972:JPL}.
\ISH{Do we want to include the details in our text? In particular, do we want to illustrate here that strict is weaker then ADF?
}


Another example is the logic
$\LK{5}$:
\begin{proposition}
  The logic $\LK{5}$ admits extensible filtration.
\end{proposition}
\begin{proof}
\todo{Take from previous; repair}
\end{proof}


The following theorem allows to transfer filtrability to fusions for many logics.

\begin{theorem}\label{thm:mainTransfer}~ %Let $L_1$ and $L_2$ be logics.
\begin{enumerate}
  \item \label{thm:mainTransfer1} If $L_1$ and $L_2$ admit extensible filtration, then the logic $L_1*L_2$ admits extensible filtration.
  \item  \label{thm:mainTransfer2}
  If $L_1$ admits extensible filtration
  and $L_2$ admits definable filtration, then the logic $L_1*L_2$ admits definable filtration.
\end{enumerate}
\end{theorem}
\begin{proof}\todo{repair}

(\ref{thm:mainTransfer1}).
Let $M = (W, R_1, R_2, \vartheta)$ be an $L_1*L_2$-model and $\Gamma$ a finite $Sub$-closed set of formulas.

We define the set of fresh variables $V = \{ p_{\varphi} \: \mid \: \varphi \in \Gamma \}$ and the following sets of formulas:

\begin{center}
    $\Gamma_i = V \cup \{ \Diamond p_{\varphi} \: \mid \: \Diamond_i \varphi \in \Gamma \}$
\end{center}

%We also define models $M_1'$ and $M_2'$ such that:
We also define a model $M'$ such that:
\begin{center}
    $M, x \models \varphi$ iff $M', x \models \varphi$ iff $M', x \models p_\varphi$
\end{center}
\ISH{The definition and property are mixed}
Denote the $R_i$-reduct of $M'$ as $M_i'$.

Clearly $M_1' \models L_1$ and $M_2' \models L_2$. $\operatorname{Mod}(L_1)$ and $\operatorname{Mod}(L_2)$ AEF both, so
there are finite $\operatorname{Sub}$-closed $\Delta_i \supseteq \Gamma_i$ such that for every finite $\operatorname{Sub}$-closed $\Lambda_i \supseteq \Delta_i$ ($i = 1, 2$) there exists a $\Gamma_i$-filtration of $M_i$ (namely $\widehat{M_i}$) which carrier is $W / \Lambda_i$.

Let us put $\Delta := \Delta_1 \cup \Delta_2$.
Let $\Lambda$ be a finite $\operatorname{Sub}$-closed extension of $\Delta$. Denote $W / \sim_{\Lambda}$ as $\widehat{W}$. Then $\widehat{M_i} = (\widehat{W}, \widehat{R_i}, \widehat{\vartheta})$ is a filtration of $M_i'$ through $\Gamma_i$ and $\widehat{M_i} \models L_i$, for $i = 1, 2$. \ISH{What are $\widehat{R_i}$? (some filtered relations, I assume).
What is more important -- what is $\widehat{\vartheta}$?
$\widehat{\vartheta}$ is defined only on variables of $\Gamma$.
Let the value of non-essential variables be empty?

Anyway, I believe we need to be very careful here.
}

Consider the model $\widehat{M} = (\widehat{W}, \widehat{R_1}, \widehat{R_2}, \widehat{\vartheta})$.
We have $\widehat{M} \models L_1 * L_2$. Let us show that $\widehat{M}$ is a $\Gamma$-filtration of $M$ through $\Lambda$. Each $\widehat{R_i}$ already contains the minimal filtered relation, so we check the maximal filtration condition.

Let $i = 1, 2$ and $\Diamond_i \psi \in \Gamma$ such that $M, y \models \psi$ and $[x] \widehat{R_i} [y]$. Then $M_i', x \models p_{\varphi}$ by the arrangement of $M'$. We have $\Diamond p_{\varphi} \in \Gamma_i$, then $\widehat{M}_i, [x] \models \Diamond p_{\varphi}$. The latter implies $M_i', x \models \Diamond p_{\varphi}$, so $M_i', x \models \Diamond \varphi$,\ISH{But $\varphi$ is bimodal...}  then $M, x \models \Diamond \varphi$.

\bigskip
(\ref{thm:mainTransfer2}).
Let $M = (W, R_1, R_2, \vartheta) \models L_1*L_2$ and let $\Gamma$ be a finite $\operatorname{Sub}$-closed set of formulas. We define $M'$, $\Gamma_1$, and $\Gamma_2$ as in the previous theorem\todo{wording}. Consider $M_1'$ and $M_2'$, the corresponding reducts of $M'$

$M_1' \models L_1$, so there exists finite $\operatorname{Sub}$-closed $\Delta_1 \supseteq \Gamma_1$ such that for every finite $\operatorname{Sub}$-closed $\Lambda \supseteq \Delta$ such that $\widehat{M_1} = (W / \sim_{\Lambda}, \widehat{R_1}, \widehat{\vartheta})$ is a $\Gamma_1$-filtration through $\Lambda$ and $\widehat{M_1} \models L_1$.

$\Lambda$ is finite and $\operatorname{Sub}$-closed, so $\widehat{M_2} = (W / \sim_{\Lambda'}, \widehat{R_2}, \widehat{\vartheta})$ is a $\Lambda$-filtration of $M_2$ through for some finite and $\operatorname{Sub}$-closed $\Lambda' \supseteq \Lambda$.

Similarly to the previous theorem, we show that $\widehat{M} = (\widehat{W}, \widehat{R_1}, \widehat{R_2}, \widehat{\vartheta})$ is a $\Gamma$-filtration of $\mathcal{M}$ through $\Lambda'$.
\end{proof}

\later{
The theorems above transfers to fusion of an arbitrary length including the countable one.

\ISH{Oh, no... At least I do not see how to do it. What if we need
to extend infinitely many Deltas, for infinitely many logics with ARF?

Have you had any argument for the above statement?}
\DR{Now it might look OK, I mean both theorems}

}


\begin{example}
$\LK{5}*\LK{5}$ admits extensible filtrations.
Consequently, the logic $\CPDL(\Di_1,\Di_2)+\LK{5}*\LK{5}$ has the finite model property and decidable.
\end{example}

From Theorems \ref{thm:ADFforCPDL} and \ref{thm:mainTransfer}, we obtain:


\obsolete{
\begin{proof}
\ISH{Proof of fusions for locally tabular? }
\ISH{Like in Theorem 4.8, with the following key observation:
After we obtain filtrations for all components, take the corresponding refinement;
on the resulting refined partition  take the filtered relations
induced by relations on bigger classes; then these models (in component languages) are p-morphic images
of initial filtrations.
}\ISH{Update: no proof known}
\end{proof}
}


\smallskip

\begin{corollary}
Let $n$ and $At$ be finite, $L_1, \ldots, L_n, L$ be normal logics
such that $L_1 * \ldots * L_n*L\subseteq \Fm(\At)$, and $L$ admits definable filtration.
\begin{enumerate}
\item
If $L_1, \ldots, L_n$ admit extensible filtrations, then $\CPDL(\At)+L_1 * \ldots * L_n*L$ has the finite model property.
\item If each $L_i$
is one of the logics
$\LK{},~\logicts{T},~\LK{4},~\LK{5},~\LK{45},~\LS{4},~\LS{5},~\LK{}+\{p\imp \Box\Di p\},~\LK{}+\{\Di\top\},
\LK{4}+\{\Di\top\}$,
$\LK{} + \{\Diamond^m p \to \Diamond p\}$ for $m \geq 1$, then  $\CPDL(\At)+L_1 * \ldots * L_n*L$ has the finite model property.
If also $L$ is finitely axiomatizable, then $\CPDL(\At)+L_1 * \ldots * L_n*L$ is decidable.
\end{enumerate}
\end{corollary}



\section{Concluding remarks}
\ISH{Dec 2022: I am  completing the proof of counterexample}

We show that the admits extensible filtration property is stronger than the admits definable filtration one.

%\section{Remark on local finiteness}
For $k\leq \omega$, a {\em $k$-formula} is a formula in proposition letters $p_i$, $i<k$.
Recall that a logic is called \emph{locally tabular},
if, for every $k<\omega$, there exist
only finitely many $k$-formulas up to the equivalence in~$L$.




It is known that transitive logics of finite height are locally finite \cite{SegerbergEssay1971}.
Hence, they admit definable filtration according to Theorem \ref{thm:LFimpliesADF}.
Consider the following logics:
$$
L_n=K+\{\Box^n\bot, ~\Di \Di p\to \Di p,~ \Di p\wedge \Di q\imp (\Di p\wedge p) \vee (p\wedge q)\vee (q\wedge \Di p)\}
$$
These logics are locally finite, but do not admit extensible filtration:
\begin{proposition}\label{prop:exampleWO-AEF}
For $n>1$,  $L_n$ does not admit extensible filtration.
\end{proposition}
\begin{proof}
\ISH{The proof below only shows that the class of FRAMES does not admit extensible filtrations.}
\DR{The class of FRAMES doesn't AEF
implies the class of models doesn't AEF assuming Lemma~\ref{modelsimpliesframes}.}
\ISH{Do we have a complete proof? }
\ISH{Update Dec 22: It is not true, I suppose.}

We consider only the case $n=2$.
Let $F=(W,R)$ be a continuum of disjoint two-element chains:
\begin{eqnarray*}
W=\{a_f, b_f:f\in 2^\omega\}, \quad
R=\{(a_f, b_f):f\in 2^\omega\}.
\end{eqnarray*}
Put $\vartheta(p_i)=\{b_f\in W\mid  f(i)=1\}$, $M=(F,\vartheta)$.
Clearly, $F\mo L_n$, so $M\mo L_n$.

Assume that $\Delta$ is a finite $\operatorname{Sub}$-closed set of formulas.
There is an infinite set $S\subseteq 2^\omega$
such that all $a_f$ with $f\in S$ are $\sim_\Delta$-equivalent.
Let $k=\max\{i:p_i\in \Delta\}$.
Since $S$ is infinite, there is $m>k$ such that
$f(m)\neq g(m)$  for some  $f,g\in S$.
Let $\Lambda$ be $\Delta\cup \{p_m\}$.
Then $b_f$ is not $\sim_\Lambda$-equivalent to $b_g$.
On the other hand, $a_f\sim_\Lambda a_g$.
Let us show that for any filtered relation $\hat{R}$,
$\hat{F}=
(W/\Lambda,\hat{R})$
is not an $L_2$-frame.
We have $[a_f]_\Lambda \hat{R} [b_f]_\Lambda$,
and $[a_f]_\Lambda \hat{R} [b_g]_\Lambda$.
Assume that $\hat{F}\mo \Box^2\bot$.
Then $[b_f]_\Lambda$ and  $[b_g]_\Lambda$
are incomparable in $\hat{F}$.
So $\hat{F}$ does not validate
$\Di p\wedge \Di q\imp (\Di p\wedge p) \vee (p\wedge q)\vee (q\wedge \Di p)$.
\ISH{It would complete the proof if AEF on models would imply AEF on frames. But I do not follow the argument (in the current text).}
\end{proof}
\begin{corollary}
There are logics that admit definable filtration,
but do not admit extensible filtration.
\end{corollary}

\obsolete{
\ISH{It seems that to prove what we claimed, we need to consider $\Lambda$
in the same vars as $\Delta$.
Another guess: ADF imply AEF (with the modification above); and it seems that
with the above modification, transfer results hold.

But all these things require some time to think about.
}
\ISH{No proof known; and it seems that we have a counterexample, saying that
ADF DOES NOT imply AEF}
}

\begin{problem}
Let $L_1$ and $L_2$ be locally finite logics.
Does $L_1*L_2$ admit definable filtration?
\end{problem}


%\end{document}

\newpage


\subsection{Main theorem}

In the following subsection, we show several transfer results related to admissibility of extensible filtration for fusions.

The following theorem generalises \cite[Theorem 4.10]{KikotShapZolAiml2020}.

\begin{theorem}
If $\operatorname{Mod}(L_1)$ and $\operatorname{Mod}(L_2)$ AEF, then $\operatorname{Mod}(L_1*L_2)$ AEF.
\end{theorem}
\begin{proof}
Let $\mathcal{M} = (W, R_1, R_2, \vartheta)$ be an $L_1*L_2$-model and $\Gamma$ a finite $Sub$-closed set of formulas.

We define the set of fresh variables $V = \{ p_{\varphi} \: \mid \: \varphi \in \Gamma \}$ and the following sets of formulas:

\begin{center}
    $\Gamma_i = V \cup \{ \Diamond p_{\varphi} \: \mid \: \Diamond_i \varphi \in \Gamma \}$
\end{center}

%We also define models $M_1'$ and $M_2'$ such that:
We also define a model $M'$ such that:
\begin{center}
    $M, x \models \varphi$ iff $M', x \models \varphi$ iff $M', x \models p_\varphi$
\end{center}
\ISH{The definition and property are mixed}
Denote the $R_i$-reduct of $M'$ as $M_i'$.

Clearly $M_1' \models L_1$ and $M_2' \models L_2$. $\operatorname{Mod}(L_1)$ and $\operatorname{Mod}(L_2)$ AEF both, so
there are finite $\operatorname{Sub}$-closed $\Delta_i \supseteq \Gamma_i$ such that for every finite $\operatorname{Sub}$-closed $\Lambda_i \supseteq \Delta_i$ ($i = 1, 2$) there exists a $\Gamma_i$-filtration of $M_i$ (namely $\widehat{M_i}$) which carrier is $W / \Lambda_i$.

Let us put $\Delta := \Delta_1 \cup \Delta_2$.
Let $\Lambda$ be a finite $\operatorname{Sub}$-closed extension of $\Delta$. Denote $W / \sim_{\Lambda}$ as $\widehat{W}$. Then $\widehat{M_i} = (\widehat{W}, \widehat{R_i}, \widehat{\vartheta})$ is a filtration of $M_i'$ through $\Gamma_i$ and $\widehat{M_i} \models L_i$, for $i = 1, 2$. \ISH{What are $\widehat{R_i}$? (some filtered relations, I assume).
What is more important -- what is $\widehat{\vartheta}$?
$\widehat{\vartheta}$ is defined only on variables of $\Gamma$.
I believe we need to be very careful here.
}

Consider the model $\widehat{M} = (\widehat{W}, \widehat{R_1}, \widehat{R_2}, \widehat{\vartheta})$.
We have $\widehat{M} \models L_1 * L_2$. Let us show that $\widehat{M}$ is a $\Gamma$-filtration of $M$ through $\Lambda$. Each $\widehat{R_i}$ already contains the minimal filtered relation, so we check the maximal filtration condition.

Let $i = 1, 2$ and $\Diamond_i \psi \in \Gamma$ such that $M, y \models \psi$ and $[x] \widehat{R_i} [y]$. Then $M_i', x \models p_{\varphi}$ by the arrangement of $M'$. We have $\Diamond p_{\varphi} \in \Gamma_i$, then $\widehat{M}_i, [x] \models \Diamond p_{\varphi}$. The latter implies $M_i', x \models \Diamond p_{\varphi}$, so $M_i', x \models \Diamond \varphi$,\ISH{But $\varphi$ is bimodal...}  then $M, x \models \Diamond \varphi$.
\end{proof}

\begin{theorem}
If $\operatorname{Mod}(L_1)$ AEF and $\operatorname{Mod}(L_2)$ ADF, then $\operatorname{Mod}(L_1*L_2)$ ADF.
\end{theorem}
\begin{proof} Let $M = (W, R_1, R_2, \vartheta) \models L_1*L_2$ and let $\Gamma$ be a finite $\operatorname{Sub}$-closed set of formulas. We define $M'$, $\Gamma_1$, and $\Gamma_2$ as in the previous theorem. Consider $M_1'$ and $M_2'$, the corresponding reducts of $M'$

$M_1' \models L_1$, so there exists finite $\operatorname{Sub}$-closed $\Delta_1 \supseteq \Gamma_1$ such that for every finite $\operatorname{Sub}$-closed $\Lambda \supseteq \Delta$ such that $\widehat{M_1} = (W / \sim_{\Lambda}, \widehat{R_1}, \widehat{\vartheta})$ is a $\Gamma_1$-filtration through $\Lambda$ and $\widehat{M_1} \models L_1$.

$\Lambda$ is finite and $\operatorname{Sub}$-closed, so $\widehat{M_2} = (W / \sim_{\Lambda'}, \widehat{R_2}, \widehat{\vartheta})$ is a $\Lambda$-filtration of $M_2$ through for some finite and $\operatorname{Sub}$-closed $\Lambda' \supseteq \Lambda$.

Similarly to the previous theorem, we show that $\widehat{M} = (\widehat{W}, \widehat{R_1}, \widehat{R_2}, \widehat{\vartheta})$ is a $\Gamma$-filtration of $\mathcal{M}$ through $\Lambda'$.
\end{proof}

\later{
The theorems above transfers to fusion of an arbitrary length including the countable one.
}
\ISH{Oh, no... At least I do not see how to do it. What if we need
to extend infinitely many Deltas, for infinitely many logics with ARF?

Have you had any argument for the above statement?}
\DR{Now it might look OK, I mean both theorems}



\subsection{Proof of the main theorem}

\begin{lemma}
Let $L$ be a logic such that $\operatorname{Mod}(L)$ AEF, then $\operatorname{Mod}(L.t)$ AEF.
\end{lemma}

\begin{lemma}
Let $L$ be a logic such that $\operatorname{Mod}(L)$ AEF, then $\operatorname{Mod}(L^+)$ AEF.
\end{lemma}

\begin{theorem}
PDL theorem
\end{theorem}

%cases
\section{Examples of logics with AEF}

In this section we consider examples of logics models of which do not admit strict filtration, but they ARF. We consider ${\bf K5}$ first. Recall that ${\bf K5} = {\bf K} + \Diamond p \to \Box \Diamond p$. It is known that ${\bf K5}$ is the logic of all Euclidean frames. The latter means:
\begin{center}
    $\forall x, y, z \in W \: (x R y \: \& \: x R z \Rightarrow y R z \vee z R y)$
\end{center}

Note that the class $\operatorname{Mod}({\bf K5})$ admit definable filtration, see \cite[Theorem 5.35]{Ch:Za:ML:1997}. Given a model $M$ and finite $\operatorname{Sub}$-closed $\Gamma$, we define $\Delta$ as the extension of $\Gamma$ with $\{ \Diamond \Box \varphi \: \mid \: \Box \varphi \in \Gamma \}$. A filtration of $M$ is the model $\widehat{M} = (W / \sim_{\Delta}, R_{max}), \widehat{\vartheta})$, which is obviously finite. Note that one can also take the Euclidean closure of the minimal filtered relation modulo $\Delta$ instead of the maximal filtered relation.

But there is no way to avoid extending the original $\Gamma$. Here is a counterexample that has been already appeared in \cite[Section 5.9]{zach2019boxes}.

\begin{proposition}
$\operatorname{Mod}({\bf K5})$ does not admit strict filtration.
\end{proposition}

However, $\operatorname{Mod}({\bf K5})$ admits extensible filtrations. To show that, we introduce the notion of the Euclidean closure and its equivalent version.

\begin{definition}
Let $R$ be a binary relation, then $R^{E}$, the Euclidean closure of $R$, is the smallest Euclidean relation containing $R$.
\end{definition}

There is a more technically convenient way to define $R^E$. Here are two equivalent alternatives, see \cite[Lemma 1.20.1]{ZolinNotes}:

\begin{lemma} \label{equivHorn}
  Let $F = (W, R)$ be a Kripke frame. Then:

  \begin{enumerate}
    \item $R^{E} = S$, where $S = \bigcup \limits_{i < \omega} R_i$ and
    \begin{itemize}
      \item $R_0 = R$
      \item $R_{n + 1} = R_n \cup (R^{-1}_n \circ R_n)$
    \end{itemize}
    \item $x R^{E} y$ iff there exists $n < \omega$ such that
    either $x R y$ or $\exists z_1, \ldots, z_n$ with $z_1 R x$ and $z_{n - 1} R y$ and for each $1 < i \leq n$ one has
    either $z_{i - 1} R z_i$ or $z_i R z_{i - 1}$.
  \end{enumerate}
\end{lemma}

To show the following fact, we modify the argument from \cite[Theorem 5.35]{Ch:Za:ML:1997} and use the lemma above.

In order to show, AEF for ${\bf K5}$-models, we formulate the following fact that has a straightforward proof:

\begin{proposition} \label{k5property}
      Let $\mathcal{M} \models {\bf K5}$ and $\varphi$ a formula, then for all $a, b \in \mathcal{M}$ such that $a R b$ we have $\mathcal{M}, a \models \Box \Diamond \varphi$ iff $\mathcal{M}, b \models \Box \Diamond \varphi$.
\end{proposition}

\begin{proof}
Easy.
\end{proof}

\begin{lemma}
  The class $\operatorname{Mod}({\bf K5})$ admits extensible filtration.
\end{lemma}

\begin{proof}
Let $M$ be a model such that $M \models {\bf K5}$.
and $\Gamma$ a finite $\operatorname{Sub}$-closed set of formulas. Take any finite closed extension of $\operatorname{Sub}\{ \Box \Diamond \varphi \: \mid \: \Diamond \varphi \in \Gamma \}$, say $\Delta$. We show that for every such $\Delta$ there is a $\Gamma$-filtration $\widehat{M} = (W / \sim_{\Delta}, R_{\sim}^E, \widehat{\vartheta})$ of $M$.

For that, we show $R_{\sim} \subseteq R_{\sim}^{E} \subseteq R^{\Gamma}_{max}$. The first is self-evident, so we consider only the second one. Let $\Diamond \varphi \in \Gamma$, $M, y \models \varphi$, and $([x], [y]) \in R_{min}^E$.

Suppose the statement holds for $R_{\sim}^{n}$, $([x],[y]) \in R_{\sim}^{n + 1}$ such that $M,
y \models \psi$ for $\Diamond \psi \in \Gamma$.
\ISH{What is the statement? I believe that something is still missing here}
According to the second item of Lemma~\ref{equivHorn}, there exist $[z_0], [z_1], \ldots, [z_{n-1}], [z_n]$ such that
$[z_1] R_{\sim} [x]$, $[z_n] R_{\sim} [y]$, and for all $0 \leq i \leq n$ we have either $[z_i] R_{\sim} [z_{i + 1}]$ or $[z_{i + 1}] R_{\sim} [z_i]$. We present the latter the following graph:

\vspace{\baselineskip}

  \xymatrix{
  & [x] & \ar[l]_{R_{\sim}} \ar@{<->}[r]^{R'} [z_0] & \ar@{<->}[r]^{R'} [z_1] \ar@{<->}[r]^{R'} & \ldots \ar@{<->}[r]^{R'} & [z_{n - 1}] \ar[r]^{R'} & [z_n] \ar[r]^{R_{\sim}} & [y]
  }

 \vspace{\baselineskip}

  where $R'$ is either $R_{\sim}$ or its converse.

  $M, y \models \varphi$ and $[z_n] R_{\sim} [y]$, so there are $z_n' \sim_{\Delta} z_n$ and $y' \sim_{\Delta} y$ such that $z_n' R y'$. Then $M, z_n' \models \Diamond \varphi$, so $M, z_n' \models \Box \Diamond \varphi$. Note that $\Box \Diamond \varphi \in \Delta$ by the construction.

  $[z_{n - 1}] R' [z_n]$. Consider the case with $[z_{n - 1}] R_{\sim} [z_n]$, the second one is similar. Then we have $z_{n -1}' \sim z_{n - 1}$ and $z_n'' \sim z_n$ such that $z_{n -1}' R z_n''$. Note that $M, z_n'' \models \Box \Diamond \varphi$ and, moreover, $M, z_{n-1}' \models \Box \Diamond \varphi$ by Proposition~\ref{k5property}.

  We iterate the latter until we get $M, z_0 \models \Box \Diamond \varphi$. $[z_0] R_{\sim} [x]$, so there exists $z_0' \sim_{\Delta} z_0$ and $x' \sim_{\Delta} x$ such that $z_0 R x'$, then $M, x' \models \Diamond \varphi$ and  $M, x \models \Diamond \varphi$.
\end{proof}

The second example we consider is the logic $L = {\bf K} + \Diamond \Diamond \Diamond p \to \Diamond p$. Semantically, the latter is equivalent to $R \circ R \circ R \subseteq R$. The class $\operatorname{Mod}(L)$ admit definable filtration, see \cite[Theorem 8]{Gabbay:1972:JPL}. Let $M \models L$
and let $\Gamma$ be a finite $\operatorname{Sub}$-closed set of formulas. Extend $\Gamma$ with $\operatorname{Sub}\{ \Diamond \Diamond \Diamond \varphi \: \mid \: \varphi \in \Gamma \}$ and define the $L$-closure of $R_{\sim}$ as follows:
\begin{center}
$R_{\sim}^{L} = \bigcup \limits_{i < \omega} R_{\sim}^{2i + 1}$
\end{center}
Then a model $\widehat{M} = (W / \sim_{\Delta}, {R_{\sim}}^{L} , \widehat{\vartheta})$ is an $L$-model and this is a filtration of $M$ through $\Delta$. But $L$ does not admit strict filtration.

\begin{proposition}
$L = {\bf K} + \Diamond \Diamond \Diamond p \to \Diamond p$ does not admit strict filtration.
\end{proposition}

\begin{proof}
  Consider the following model $M$:

\vspace{\baselineskip}

  \xymatrix{
  &&&& x \models p \ar@(ur,ul) \ar[r] & y \models q \\
  &&&&& y_1 \models q \ar[r] \ar@/_/[rr] & z_1 \ar[r] & z \models r \ar@(ur,ul)
  }

\vspace{\baselineskip}

  Let us put $\Gamma = \operatorname{Sub} \{ p, q, \Diamond r \}$. We factorise $W$ through $\sim_{\Gamma}$ and consider a model $\widehat{M} = (W / \sim_{\Gamma}, R_{max}, \widehat{\vartheta})$.
  $y$ and $y_1$ are equivalent modulo $\Gamma$. So we have the following model after factorisation:

\vspace{\baselineskip}

  \xymatrix{
  &&&& [x] \models p \ar@(ur,ul) \ar[r] & [y] \models q \ar[r] \ar@/_/[rr] & [z_1] \ar[r] & z \models [r] \ar@(ur,ul) \\
  }

We have $M, z \models r$, but $\Diamond r$ is not true at $x$. Thus, $\widehat{M} \not\models L$.
\end{proof}

However, $\operatorname{Mod}({L})$ for $L = {\bf K} + \Diamond \Diamond \Diamond p \to \Diamond p$ admits extensible filtration since the construction from the proof of \cite[Theorem 8]{Gabbay:1972:JPL} works for the AEF property as well, so we have:

\begin{lemma}\label{extension}
 $L = {\bf K} + \Diamond^n \Diamond p \to \Diamond p$ for $n \geq 1$ admit extensible filtration.
\end{lemma}

\later{wK4(?); S4.2 + fin. height; S4.3+fin. height !!!! - NO(?); S4.1 ??? }

\begin{theorem}\label{thm:MainCases}
The following logics admit extensible filtrations:
\begin{itemize}
\item ${\bf K}$, ${\bf T}$, ${\bf D}$, ${\bf B}$, ${\bf K4}$, ${\bf D4}$, ${\bf S4}$, ${\bf S5}$, ${\bf K5}$, ${\bf K} + \Diamond^n p \to \Diamond p$ for $n \geq 1$.
\end{itemize}
\end{theorem}
\begin{corollary}
Assume that $L_i$ be any logics .... mentioned in Theorem \ref{thm:MainCases}.
The $\PDL(\At)+L_1*\ldots*L_n$ has the fmp. \ToDo{}{What is $\At$?}
\end{corollary}


\later{Let us postpone it}


\section{Concluding remarks}


Let us show that the admits extensible filtration property is stronger than the
admits definable filtration one.


%\section{Remark on local finiteness}
For $k\leq \omega$, a {\em $k$-formula} is a formula in proposition letters $p_i$, $i<k$.
Recall that a logic is called \emph{locally finite} (or \emph{locally tabular}),
if, for every $k<\omega$, there exist
only finitely many $k$-formulas up to the equivalence in~$L$.




%Let $\Fm(\Sigma,k)$ be the set of all $k$-formulas in $\Sigma$.
The following fact follows from \cite{Shehtman:AiML:2014}.
\begin{theorem}\label{thm:LFimpliesADF}
If $L$ is locally finite, then $L$ admits definable filtration.
\end{theorem}
\begin{proof}
Let $M$ be a model, $M\mo L$, $\Gamma$ a finite $\Sub$-closed set of formulas. %in the alphabet of $L$.
For some $k<\omega$, every formula in $\Gamma$ is a $k$-formula.
Let $\Phi$ be the set of all $k$-formulas. % (in the alphabet of $L$).
Consider the maximal filtration $\ff{M}$ of $M$ through $\Phi$ (in \cite{Shehtman:AiML:2014},
such filtrations are called {\em canonical}).
Since $L$ is locally finite, $\ff{M}$ is finite. In this case
$\ff{M}$ is a p-morphic image of $M$ (for details, see \cite[Proposition 2.32]{Shehtman:AiML:2014}).
\ISH{This ref is not exact. The argument must be that the frame of M is a p-morphic image of a generated subrame of the k-canonical frame}
Hence $\ff{M}\mo L$, as required.
\end{proof}


It is known that transitive logics of finite height are locally finite \cite{SegerbergEssay1971}.
Hence, they admit definable filtration according to Theorem \ref{thm:LFimpliesADF}.
Consider the following logics:
$$
L_n=K+\{\Box^n\bot, ~\Di \Di p\to \Di p,~ \Di p\wedge \Di q\imp (\Di p\wedge p) \vee (p\wedge q)\vee (q\wedge \Di p)\}
$$
These logics are locally finite, but do not admit extensible filtration:
\begin{proposition}\label{prop:exampleWO-AEF}
For $n>1$,  $L_n$ does not admit extensible filtration.
\end{proposition}
\begin{proof}
\ISH{The proof below only shows that the class of FRAMES does not admit extensible filtrations.} \DR{The class of FRAMES doesn't AEF implies the class of models doesn't AEF assuming Lemma~\ref{modelsimpliesframes}.}
\ISH{Do we have a complete proof?

The problem is that now I believe that every locally finite logic
AEF... }

We consider only the case $n=2$.
Let $F=(W,R)$ be a continuum of disjoint two-element chains:
\begin{eqnarray*}
W&=&\{a_f, b_f:f\in 2^\omega\} \\
R&=&\{(a_f, b_f):f\in 2^\omega\}
\end{eqnarray*}
Put $\vartheta(p_i)=\{b_f\in W\mid  f(i)=1\}$, $M=(F,\vartheta)$.
Clearly, $F\mo L_n$, so $M\mo L_n$.

Assume that $\Delta$ is a finite $\operatorname{Sub}$-closed set of formulas.
There is an infinite set $S\subseteq 2^\omega$
such that all $a_f$ with $f\in S$ are $\sim_\Delta$-equivalent.
Let $k=\max\{i:p_i\in \Delta\}$.
Since $S$ is infinite, there is $m>k$ such that
$f(i)\neq g(i)$  for some  $f,g\in S$.
Let $\Lambda$ be $\Delta\cup \{p_i\}$.
Then $b_f$ is not $\sim_\Lambda$-equivalent to $b_g$.
On the other hand, $a_f\sim_\Lambda a_g$.
Let us show that for any filtered relation $\hat{R}$,
$\hat{F}=
(W/\Lambda,\hat{R})$
is not an $L_2$-frame.
We have $[a_f]_\Lambda \hat{R} [b_f]_\Lambda$,
and $[a_f]_\Lambda \hat{R} [b_g]_\Lambda$.
Assume that $\hat{F}\mo \Box^2\bot$.
Then $[b_f]_\Lambda$ and  $[b_g]_\Lambda$
are incomparable in $\hat{F}$.
So $\hat{F}$ does not validate
$\Di p\wedge \Di q\imp (\Di p\wedge p) \vee (p\wedge q)\vee (q\wedge \Di p)$.
\ISH{It would complete the proof if AEF on models would imply AEF on frames. But I do not follow the argument (in the current text).}
\end{proof}
\begin{corollary}
There are logics that admit definable filtration,
but do not admit extensible filtration.
\end{corollary}

\ISH{It seems that to prove what we claimed, we need to consider $\Lambda$
in the same vars as $\Delta$.
Another guess: ADF imply AEF (with the modification above); and it seems that
with the above modification, transfer results hold.

But all these things require some time to think about.
}


\begin{problem}
Let $L_1$ and $L_2$ be locally finite logics.
Does $L_1*L_2$ admit definable filtration?
\end{problem}



%\begin{appendices}

\todo{...}
%Here starts the appendix. If you don't wish an %appendix, please remove the \verb\mid\Appendix\mid %command from the \LaTeX\ file.
%\end{appendices}



%\bibliographystyle{sn-apacite}
%\bibliographystyle{sn-mathphys}
%\bibliographystyle{alpha}
%\bibliography{filtration}


\end{document}
