% qmanus/cause/cqc09.tex 3/23/23
% 3/23/23 11:45 after checking spelling: copied cqc09.tex to cqc09s.tex
% 

	% Begin
\documentclass[12pt]{article}
\usepackage{latexsym}
\usepackage{amsmath}
\usepackage{indentfirst}
\usepackage{cite} % Orders and compresses citation strings
%\usepackage{pstricks} 
%\usepackage{pstricks,pst-node}
\usepackage{graphicx}
%\usepackage[notcite]{showkeys}

	% Page dimension
\oddsidemargin 0 cm
\evensidemargin 0 cm
\topmargin -1.5 cm \textheight 23 cm \textwidth 16.5 cm
\raggedbottom

	% RBG style file
%\input /home/rgrif/latex/defs/grifg.sty % For my own use 
%\input /home/rgrif/latex/defs/gril08.sty

%latex/defs/gril08.sty 12/28/11
% Definitions for Latex2e amsmath
% GENERAL, MATH CAL (SCRIPT), BOLD FACE, BLACKBOARD BOLD, GREEK
% 8/19/13 Added \vbB 12/4/15. Corrected \AND, \NOT, \OR using an \mbox{}
% 5/27/18 Added \mats{}

		% GENERAL LIST
\long\def\ca#1\cb{} %Use for commenting out: \ca...\cb

\newcommand{\ad}{^\dagger }
\newcommand{\adp}{^{\phantom{\dagger}}}
\newcommand{\AND}{\mbox{\small AND}}
\newcommand{\Arg}{\mathop{\rm Arg}\nolimits}
\newcommand{\avg}[1]{\langle #1\rangle }
\newcommand{\becs}{\begin{cases}}
\newcommand{\bem}{\begin{matrix}}
\newcommand{\blp}{\bigl(} 
\newcommand{\Blp}{\Bigl(} 
\newcommand{\bpp}[1]{\bigl(#1\bigr)}
\newcommand{\Bpp}[1]{\Bigl(#1\Bigr)}
\newcommand{\bra}[1]{\langle#1|}
\newcommand{\Brp}{\Bigr)}
\newcommand{\brp}{\bigr)}
\newcommand{\bsk}{\bigskip }
\newcommand{\bsl}{$\backslash$}
\newcommand{\ck}{\breve }
\newcommand{\cln}{\colon\!} %Symmetrizes some math expressions
\newcommand{\coln}{\hbox{:}}
\newcommand{\colo}{\,\hbox{:}\,}
\newcommand{\Cov}{{\rm Cov}}
\newcommand{\Det}{{\rm Det}}
\newcommand{\dg}{^\circ } %Degree
\newcommand{\dst}{\displaystyle } %\tst for textstyle
\newcommand{\dya}[1]{|#1\rangle\langle#1|}
\newcommand{\dyad}[2]{|#1\rangle\langle#2|}
\newcommand{\emp}{\emptyset }
\newcommand{\encs}{\end{cases}}
\newcommand{\enm}{\end{matrix}}
\newcommand{\fac}{\mspace{1mu}!}
\newcommand{\hf}{{\textstyle\frac{1}{2} }}
\newcommand{\Hf}{\frac{1}{2} }
\newcommand{\hquad}{\mspace{8 mu}} 
\newcommand{\inp}[1]{\langle#1|#1\rangle }
\newcommand{\inpd}[2]{\langle#1|#2\rangle }
\newcommand{\ket}[1]{|#1\rangle }
\newcommand{\lbrk}[1]{\left\{\vrule height #1cm depth #1cm width 0pt\right.}
\newcommand{\Ln}{\mathop{\rm Ln}\nolimits}
\renewcommand{\Im}{\hbox{Im}} %Imaginary, replaces standard Tex definition
\newcommand{\lgl}{\langle } 
\newcommand{\lp}{\left(} % And see \rp
\newcommand{\Lp}{\Bigl(}  % And see \Rp
\newcommand{\lra}{\leftrightarrow }
\newcommand{\mab}[1]{\begin{matrix}#1\end{matrix}}
\newcommand{\mat}[1]{\left(\begin{matrix}#1\end{matrix}\right)}
\newcommand{\matd}[1]{\left|\begin{matrix}#1\end{matrix}\right|}
\newcommand{\mats}[1]{\left[\begin{matrix}#1\end{matrix}\right]}
\newcommand{\mns}{\negmedspace - \negmedspace }
\renewcommand{\mod}{\mathop{\rm mod}\nolimits} %Replaces Latex form
\newcommand{\msk}{\medskip }
\newcommand{\mt}{\mapsto }
\newcommand{\mte}[2]{\langle#1|#2|#1\rangle }
\newcommand{\mted}[3]{\langle#1|#2|#3\rangle }
\newcommand{\negg}{\sim\!} %Also see \sm
\newcommand{\nid}{\noindent }
\newcommand{\NOT}{\mbox{\small NOT}}
\newcommand{\od}{\odot }
\newcommand{\OR}{\mbox{\small OR}}
\newcommand{\ot}{\otimes }
\newcommand{\phd}[1]{\vphantom{\vrule height 0pt depth #1}} %\phd{4ex}
\newcommand{\phu}[1]{\vphantom{\vrule height #1 depth 0pt}} %\phu{10pt}
\newcommand{\pls}{\negmedspace + \negmedspace }
\newcommand{\pr}{\partial }
\newcommand{\pval}{\rlap{$\mspace{5mu}-$}}%\pval\int ->principal value integral
\newcommand{\ra}{\rightarrow }
\newcommand{\Ra}{\Rightarrow }
\newcommand{\rbrk}[1]{\left.\vrule height #1cm depth #1cm width 0pt\right\}}
\renewcommand{\Re}{\hbox{Re}} %Real, replaces standard Tex definition
\newcommand{\rgl}{\rangle }
\newcommand{\Rn}{{\rm Rn}}
\newcommand{\rp}{\right)}
\newcommand{\Rp}{\Bigr)}
\newcommand{\sep}{\centerline{\vrule height 0.5pt depth 0pt width 8cm}}
\newcommand{\sepb}{%
\vspace{-3ex}\begin{center}\vrule height 0.5pt depth 0pt width 8cm\end{center}}
\newcommand{\sgn}{\mathop{\rm sgn}}
\newcommand{\sm}{\sim\!\!} %Also see \negg
\newcommand{\ssk}{\smallskip }
\newcommand{\st}{\sqrt{2}}
\newcommand{\tm}{\times }
\newcommand{\Tr}{{\rm Tr}}
\newcommand{\trp}{^\textup{T}}
\newcommand{\tst}{\textstyle } %See \dst for displaystyle
\newcommand{\ttt}{\texttt}
\newcommand{\Var}{{\rm Var}}
\newcommand{\vb}{\,|\,}
\newcommand{\vbb}{\mspace{1mu}|\mspace{1mu}}
\newcommand{\vbB}{\boldsymbol{\mid}}
\newcommand{\vbl}{\,\boldsymbol{|}\,}


		% MATH CAL (SCRIPT)

\newcommand{\AC}{{\mathcal A}}
\newcommand{\BC}{{\mathcal B}}
\newcommand{\CC}{{\mathcal C}}
\newcommand{\DC}{{\mathcal D}}
\newcommand{\EC}{{\mathcal E}}
\newcommand{\FC}{{\mathcal F}}
\newcommand{\GC}{{\mathcal G}}
\newcommand{\HC}{{\mathcal H}}
\newcommand{\IC}{{\mathcal I}}
\newcommand{\JC}{{\mathcal J}}
\newcommand{\KC}{{\mathcal K}}
\newcommand{\LC}{{\mathcal L}}
\newcommand{\MC}{{\mathcal M}}
\newcommand{\NC}{{\mathcal N}}
\newcommand{\OC}{{\mathcal O}}
\newcommand{\PC}{{\mathcal P}}
\newcommand{\QC}{{\mathcal Q}}
\newcommand{\RC}{{\mathcal R}}
\newcommand{\SC}{{\mathcal S}}
\newcommand{\TC}{{\mathcal T}}
\newcommand{\UC}{{\mathcal U}}
\newcommand{\VC}{{\mathcal V}}
\newcommand{\WC}{{\mathcal W}}
\newcommand{\XC}{{\mathcal X}}
\newcommand{\YC}{{\mathcal Y}}
\newcommand{\ZC}{{\mathcal Z}}


		% BOLD FACE

\newcommand{\bld}[1]{\boldsymbol{#1}}

	% CAPS BOLD FACE

\newcommand{\AB}{\textbf{A}}
\newcommand{\BB}{\textbf{B}}
\newcommand{\CB}{\textbf{C}}
\newcommand{\DB}{\textbf{D}}
\newcommand{\EB}{\textbf{E}}
\newcommand{\FB}{\textbf{F}}
\newcommand{\GB}{\textbf{G}}
\newcommand{\HB}{\textbf{H}}
\newcommand{\IB}{\textbf{I}}
\newcommand{\JB}{\textbf{J}}
\newcommand{\KB}{\textbf{K}}
\newcommand{\LB}{\textbf{L}}
\newcommand{\MB}{\textbf{M}}
\newcommand{\NB}{\textbf{N}}
\newcommand{\OB}{\textbf{O}}
\newcommand{\PB}{\textbf{P}}
\newcommand{\QB}{\textbf{Q}}
\newcommand{\RB}{\textbf{R}}
\newcommand{\SB}{\textbf{S}}
\newcommand{\TB}{\textbf{T}}
\newcommand{\UB}{\textbf{U}}
\newcommand{\VB}{\textbf{V}}
\newcommand{\WB}{\textbf{W}}
\newcommand{\XB}{\textbf{X}}
\newcommand{\YB}{\textbf{Y}}
\newcommand{\ZB}{\textbf{Z}}

	% LOWER CASE BOLD FACE

\newcommand{\aB}{\textbf{a}}
\newcommand{\bB}{\textbf{b}}
\newcommand{\cB}{\textbf{c}}
\newcommand{\dB}{\textbf{d}}
\newcommand{\eB}{\textbf{e}}
\newcommand{\fB}{\textbf{f}}
\newcommand{\gB}{\textbf{g}}
\newcommand{\hB}{\textbf{h}}
\newcommand{\iB}{\textbf{i}}
\newcommand{\jB}{\textbf{j}}
\newcommand{\kB}{\textbf{k}}
\newcommand{\lB}{\textbf{l}}
\newcommand{\mB}{\textbf{m}}
\newcommand{\nB}{\textbf{n}}
\newcommand{\oB}{\textbf{o}}
\newcommand{\pB}{\textbf{p}}
\newcommand{\qB}{\textbf{q}}
\newcommand{\rB}{\textbf{r}}
\newcommand{\sB}{\textbf{s}}
\newcommand{\tB}{\textbf{t}}
\newcommand{\uB}{\textbf{u}}
\newcommand{\vB}{\textbf{v}}
\newcommand{\wB}{\textbf{w}}
\newcommand{\xB}{\textbf{x}}
\newcommand{\yB}{\textbf{y}}
\newcommand{\zB}{\textbf{z}}

	% SPECIAL BOLD
\newcommand{\zrB}{\textbf{0}}

		% BLACKBOARD BOLD
\newcommand{\Cbb}{\mathbb{C}}
\newcommand{\Rbb}{\mathbb{R}}

                % GREEK
% The following Greek definitions do not replace
% anything in Tex or Latex.  
%The following are NOT abbreviated: eta mu nu xi pi rho tau phi chi psi
% omicron does not occur: use o
% The following caps are NOT abbreviated: Xi, Pi, Phi, Psi
\newcommand{\al}{\alpha }
\newcommand{\bt}{\beta }
\newcommand{\gm}{\gamma }
\newcommand{\Gm}{\Gamma }
\newcommand{\dl}{\delta }
\newcommand{\Dl}{\Delta }
\newcommand{\ep}{\epsilon}
\newcommand{\vep}{\varepsilon}
\newcommand{\zt}{\zeta }
\renewcommand{\th}{\theta } %Latex \th = thor n
\newcommand{\vth}{\vartheta }
\newcommand{\Th}{\Theta }
\newcommand{\io}{\iota }
\newcommand{\kp}{\kappa }
\newcommand{\lm}{\lambda }
\newcommand{\Lm}{\Lambda }
\newcommand{\vpi}{\varpi}
\newcommand{\vrh}{\varrho}
\newcommand{\sg}{\sigma }
\newcommand{\vsg}{\varsigma}
\newcommand{\Sg}{\Sigma }
\newcommand{\up}{\upsilon }
\newcommand{\Up}{\Upsilon }
\newcommand{\vph}{\varphi }
\newcommand{\om}{\omega }
\newcommand{\Om}{\Omega }


          % OUTLINE IN TEXT  [latex/platg 3/27/2017]
%     INSTRUCTIONS. Default is text with outlines, and \ca, \cb are used to 
% comment  out revisions in definitions for other cases. 
%     TEXT, NO OUTLINE:  \ca -> %\ca, \cb -> %\cb below 'TEXT ALONE'. 
%     OUTLINE, NO TEXT:  \ca -> %\ca, \cb -> %\cb below 'OUTLINE ALONE'
% Newpage. \np for text alone, \nq for text + outline, \nr for outline alone
% Material to be EXCLUDED for OUTLINE ALONE is between \xa and \xb. 
% Presence of \xb elsewhere does not matter, as it is ignored.
 
% DEFINITIONS FOR TEXT INCLUDING OUTLINES
\def\outl#1{\par{\medskip\noindent\hspace*{0.1cm}\bf
      \mathversion{bold}#1\mathversion{normal}\smallskip} }
\def\np{} \def\nq{\newpage } \def\nr{} \def\xa{} \def\xb{} \def\xn{} \def\xp{}

% TEXT ALONE USE %\ca %\cb. INCLUDE OUTLINE: \ca + \cb.  
%\ca
 \def\outl#1{}\def\np{\newpage }\def\nq{}\def\xa{}\def\xb{}\def\xn{}\def\xp{}
%\cb

% OUTLINE ALONE USE %\ca %\cb, OTHERWISE \ca \cb
\ca
 \def\outl#1{\par{\medskip\noindent\hspace*{.5cm}\bf
      \mathversion{bold}#1\mathversion{normal}\smallskip} }
 \long\def\xa#1\xb{} %This comments out segments from \xa to the next \xb
 \def\nq{} \def\nr{\newpage }\def\xn{\nopagebreak }\def\xp{\pagebreak }
\cb
          % END OUTLINE IN TEXT

	% Phys Rev A style
%The following will change section, subsection to Phys Rev A style
\renewcommand{\thesection}{\Roman{section}}  %latex
\renewcommand{\thesubsection}{\thesection\ \Alph{subsection}} %latex

	% Numbering equations within section (as 1.1, 1.2, etc.)
%\numberwithin{equation}{section}

%Abbreviations for references, to be replaced later
% BAM Bll901
% Bell inequalities: SEP article: MyGS19
% Bell inequality tests: Gsao15ea
% Bohm 1951 book Cj. 22 Bhm51s
% BVN (Birkhoff & von Neumann): BrvN36
% CHSH CHSH69
% CH logic = Grff14
% CQT Grff02c
% DAG Prl09,SpGS00
% EPR EnPR35
% NONLOC Grff20
% LAM = Lambare preprint Lmbr21
% QCM Daley et al. DlRS22
% SEP = RBG review Grff19b
% Spekkens Spkk07 Spkk19
% VN = Grundlagen vNmn32b
% WQMM Grff17b

\begin{document}

\title{Consistent Quantum Causes}
\author{Robert B. Griffiths\thanks{Electronic address: rgrif@cmu.edu}\\
Department of Physics\\
Carnegie Mellon University\\
Pittsburgh, PA 15213}

\date{Version of 23 March 2023}
\maketitle



\xa
\begin{abstract}
  The fact that projectors on a quantum Hilbert space do not in general commute
  has made it difficult to extend into the quantum domain certain
  well-developed and powerful ideas (think of directed acyclic graphs) for
  studying ordinary classical causes. An approach is proposed which uses the
  single framework rule of consistent histories quantum theory to identify
  intuitively plausible microscopic quantum causes in the case of a simple
  gedanken experiment, and certain extensions thereof, including the
  correlations that violate Bell inequalities. The approach is simpler, both
  formally and intuitively, than that developed under the heading of ``Quantum
  Causal Models,'' and might be extended into a much more satisfactory quantum
  realization of important classical ideas of causation.
\end{abstract}


\xb
\tableofcontents
\xa

\xb
\section{Introduction \label{sct1}}
\xa


\xb
%
\outl{`Causes' important in the laboratory, as in everyday life }
%
\xa

Identifying causes is useful not only in everyday life, but also in the physics
laboratory where the the macroscopic outcomes of measurements are often
interpreted as caused by invisible particles whose properties and motion cannot
be directly detected, and are governed by quantum mechanical rather than
classical laws. Causal thinking is required both to understand what is going
on, and also when designing equipment, and testing it to see if it is working
properly. Consequently any claim to a satisfactory understanding of causality
in the quantum domain must be able to deal in a sensible way with laboratory
phenomena, especially since they provide the empirical support for quantum
theory.

\xb
%
\outl{Sophisticated DAG Cl causal models. Wood \& Spekkens difficulties.
  Quantum Causal Model (QCM) approach cannot deal with simplest lab situations.}
%
\xa

Efforts to understand causes in the everyday classical domain have resulted in
the development of some fairly sophisticated stochastic modeling to represent
causal structures using directed acyclic graphs.%
%
\footnote{%
Standard references are \cite{Prl09,SpGs00}; for a compact introduction
see Sec. 3 of \cite{Htch18}.} %
%END footnote
Attempts to extend these ideas into the quantum domain, starting with the
pioneering work of Wood and Spekkens\cite{WdSp15} in connection with the
correlations associated with Bell inequalities, encountered serious
difficulties. Efforts to deal with them led to the development of what are
called \emph{quantum causal models} (QCMs), which are discussed below in
Sec.~\ref{sct7}. However, the QCM approach appears incapable of dealing with
simple laboratory situations of the kind discussed in Sec.~\ref{sct4}.
This, along with its mathematical and conceptual complexity, suggests the need
for something better.

\xb
%
\outl{Present analysis uses HSQM, ideas of vN, noncommuting projectors in
  Sec.~\ref{sct3}. Causes discussed in Sec.~\ref{sct2}}
%
\xa

The present analysis is based on principles associated with the use of
projectors on the quantum Hilbert space to represent physical properties in the
manner proposed by von Neumann \cite{vNmn32b}. The fact that two projectors do
not in general commute is a central feature of what might be called
\emph{Hilbert space quantum mechanics}, which, along with some extensions used
in the consistent histories (CH) approach, is the subject of Sec.~\ref{sct3}.
This follows some elementary remarks, sufficient for the purposes of this
paper, relating probabilities, correlations, and causes, in Sec.~\ref{sct2}.

\xb
%
\outl{Gedanken experiments, Fig.~\ref{fgr1} \& Sec.~\ref{sct4}. Spin-half
  discussion in Sec.~\ref{sct5} }
%
\xa

The simple gedanken experiments discussed in Sec.~\ref{sct4} using
Fig.~\ref{fgr1} illustrate some of what is needed for a satisfactory quantum
theory of microscopic causes and their effects. An analysis using the
analogous case of a spin-half particle in Sec.~\ref{sct5} shows how paying
attention to alternative incompatible refinements of Hilbert-space projectors,
together with the single-framework rule of CH, provides a consistent and
relatively simple way of understanding the causal relationships in
Sec.~\ref{sct4}. Sections~\ref{sct4} and \ref{sct5} form the heart of the
present paper.

\xb
%
\outl{Extension to general projective measurements, Sec.~\ref{sbct6a}; resolves
  Bell's complaint Against Measurement. Two or more spin-half:
  Sec.~\ref{sbct6b}, resolves Bell inequality paradox}
%
\xa
\xb
%
\outl{Sec.~\ref{sbct6c}: Preliminary ideas for general theory of Qm causes}
%
\xa

The idea of how microscopic quantum properties cause outcomes of spin-half
measurements is easily extended, Sec.~\ref{sbct6a}, to the case of general
projective measurements, thus providing a reply to Bell's tirade \cite{Bll901}
against the way measurements were (and still are) presented in textbooks. The
extension of the discussion in Sec.~\ref{sct5} to two spin-half particles,
Sec.~\ref{sbct6b}, provides a simple solution to the correlation paradox
associated with Bell inequalities.  While a general theory of quantum
causes lies outside the scope of the present paper, Sec.~\ref{sbct6c} indicates
some features of a plausible approach.

\xb
%
\outl{Sec.~\ref{sct7}: QCM. Conclusion in Sec.~\ref{sct8}}
%
\xa

Following the discussion of the QCM approach in  Sec.~\ref{sct7}, a
short summary in Sec.~\ref{sct8} concludes the paper.
  
%END Sec. I Introduction


\section{Causes \label{sct2}}

\xb
%
\outl{Cl or Qm cause: Stochastic process with sample space of histories
  required for std prob theory. Dynamical law assigns probabilities. Events at
  one time: Cl indicator or Hilbert space projector }
%
\xa


A probabilistic model that uses standard (Kolmogorov) probabilities employs a
\emph{sample space} of mutually exclusive possibilities. In the case of
stochastic (random) time development the sample space consists of what we shall
call \emph{histories}, one and only one of which occurs during a particular
experimental run. To these a dynamical law assigns non-negative probabilities
that sum to $1$.
%
In the quantum case a history consists of events, each represented by a
projector on the quantum Hilbert space, at a succession of times. Using
discrete times is adequate, and as in classical stochastic processes, is
mathematically simpler than the continuous case.

\xb
%
\outl{Conditional probabilities: $ \Pr(G\vb F)= \Pr(F\vb G)=1 $ $\lra$ $F$ the
  cause of $G$}
%
\xa

Given this setup, one can use standard probabilistic reasoning to calculate
correlations between events at different times.  Let $F$ be an event at an
earlier time $t_1$ and $G$ one at a later time $t_2$. In the simplest case
one can call  $F$ the \emph{cause} of
$G$, and $G$ an \emph{effect} produced by $F$ provided the probabilities
satisfy two conditions:
\begin{align}
 & \Pr(G\vb F) = \Pr(F,G)/\Pr(F) = 1, 
\notag\\
 & \Pr(F\vb G) = \Pr(F,G)/\Pr(G) = 1.
\label{eqn1}
\end{align}
The first tells us that if the earlier $F$ occurred the later
$G$ was inevitable, and the second that if the later $G$ occurred, one can be
certain that it was preceded by $F$. We will only be concerned with the
simplest situations, and probabilities a bit less than $1$ will not trouble the
working physicist. 

\xb
%
\outl{Issue of CAUSE preceding EFFECT}
%
\xa

What at first seems odd is that the conditions in \eqref{eqn1} remain the
same if $F$ and $G$ are interchanged. Surely causes must \emph{precede} their
effects, or we would not call them causes. One can view this as arising from
the fact that while the fundamental microscopic laws of both classical and
quantum physics do not single out a direction of time, we as macroscopic beings
live in a world of increasing entropy in which we can remember the past but not
the future. Thus if ``cause'' is to be a useful part of reasoning about the
world, the requirement that $F$ precede $G$ in time well help extend our
intuition, based on everyday experience, into the mysterious quantum world.

\xb
%
\outl{Common Cause. Correlations vs Causes}
%
\xa

But there are examples in which two events at different times satisfy
\eqref{eqn1}, and yet one would not want to call $F$ the cause of $G$. In
particular, the correlation between them may result from a yet earlier
\emph{common cause} $C$. For example, Charlie may send identical signals, both
$0$ or both $1$, to two parties Alice and Bob in different directions. That
Alice receives the same signal as Bob, but at an earlier time since she is
closer to Charlie, hardly means that her detecting it causes what Bob receives.
Distinguishing what one might call a ``pseudo-cause'' from a genuine cause is
an important aspect of causal modeling. One way of dealing with it is to
imagine an \emph{intervention} which affects one signal rather than the other.
E.g., Eve intercepts the one sent to Bob and interchanges $0$ and $1$ before
sending it on to its final destination, thus changing the correlation. It helps
to keep in mind the slogan that ``Correlations are Not (necessarily) Causes.''

% END Sec. 2A

\section{ Hilbert Space Quantum Mechanics \label{sct3}}

\xb
\outl{HSQM $\lra$ von Neumann}
\xa

\xb
%
\outl{Finite-dimensional Hilbert space. Projectors $\lra$ Cl indicators}
%
\xa

With apologies to von Neumann, who invented the term to refer to sophisticated
ideas in functional analysis, for our purposes a \emph{Hilbert space} is simply
a finite-dimensional complex linear vector space equipped with the usual inner
product. In Sec.~III.5 of his difficult book \cite{vNmn32b}, von Neumann
identifies a \emph{projector}, an operator $P=P\ad =P^2$, as the proper
mathematical representation of a quantum property. It is the counterpart of an
\emph{indicator} function $J=J^*=J^2$ defined on a classical phase space: a
function which takes the value $1$ at those points that represent some physical
property, and $0$ elsewhere. For example, the phase-space points of a harmonic
oscillator with energy less than some value $E$ lie inside an ellipse centered
at the origin, and the corresponding indicator $J$ takes the value $1$ inside
this ellipse and $0$ outside.

\xb
%
\outl{Logic of indicators. NOT, AND. Qm `AND' fails if PQ!=QP}
%
\xa

Classical indicators can conveniently represent certain logical constructions,
as in Venn diagrams. Thus the complementary property ``NOT $J$'' is represented
by the indicator $I-J$, where $I$, the identity indicator, is equal to $1$
everywhere. The product $JK$ of two indicators represents the property
``$J\ \AND\ K$'', equal to $1$ at all points where both properties hold.
Similarly, the quantum complement of $P$ is $I-P$, with $I$ the identity
operator. However $PQ$ represents the property ``$P\ \AND\ Q$'' if and only if
$P$ and $Q$ \emph{commute}: $PQ=QP$. Otherwise the product, in either order, is
not a projector, and cannot represent a physical property.

\xb
%
\outl{Noncommutation at Cl-Qm boundary. Quantum logic}
%
\xa

\xb
%
\outl{CH: `P AND Q' meaningless in noncommuting case. Framework = set of
commuting projectors. PDI defined; forms probabilistic sample space}
%
\xa

In many ways the noncommutation of projectors marks the boundary between
classical and quantum physics. It underlies various \emph{uncertainty
  relations}, and is at the heart of numerous conceptual headaches and quantum
paradoxes. Von Neumann was aware of this, and along with Birkhoff\cite{BrvN36}
invented a quantum logic to try and make sense of ``$P\ \AND\ Q$'' in the
noncommuting case. It is so complicated that it has been of little help in
understanding quantum mechanics and resolving its conceptual difficulties.
%
The much simpler Consistent Histories%
\footnote{ A compact overview of CH will
  be found in \cite{Grff19b}, while \cite{Grff02c} is a detailed treatment that
  remains largely up-to-date. For CH as a form of quantum logic, see
  \cite{Grff14} and Ch.~16 of \cite{Grff02c}. How CH principles resolve quantum
  paradoxes is discussed in Chs.~20 to 25 of \cite{Grff02c}, and in
  \cite{Grff17b} and \cite{Grff20}.} %END footnote
(CH) approach treats ``$P \AND\ Q$'' as \emph{meaningless} in the noncommuting
case, so one never has to discuss its meaning, and limits quantum reasoning to
collections of commuting projectors called \emph{frameworks}. One starts with
a \emph{projective decomposition of the identity} (PDI), the counterpart of a
\emph{sample space} in standard (Kolmogorov) probability theory, a collection
$\{P_j\}$ of mutually orthogonal projectors that sum to the identity:
%
\begin{equation} I = \sum_j P_j,\quad P_j = (P_j)\ad = (P_j)^2,\quad P_j P_k =
\dl_{jk}P_j.
\label{eqn2}
\end{equation}
%
The corresponding \emph{event algebra} consists of all the projectors which are
members of, or sums of members of the PDI. The CH term ``framework'' can refer
to either the PDI or the corresponding event algebra. Its \emph{single
  framework rule} states that any sort of logical reasoning must be carried out
using a collection of commuting projectors; reasoning which makes simultaneous
use of noncommuting projectors is invalid. Note that any collection $\CC$ of
commuting projectors generates, via complements and products, a PDI, and thus
an associated framework, so the single framework rule is in this sense not very
restrictive. What is ruled out is the application of classical reasoning to
situations in which two or more projectors do not commute, the source of an
endless number of unresolved quantum paradoxes.

\xb
%
\outl{Atom with two lowest energy levels $\ket{0}$, $\ep_0$; $\ket{1}$,
  $\ep_1$; $R=[0]+[1]$. Need PDI $\{ [0], [1], I-R\}$ to conclude: $R\ra$
  $\ep_0$ OR $\ep_1$ }
%
\xa

Here is a simple example that will help illustrate these ideas. Let $\ket{0}$
and $\ket{1}$ be the two lowest energy eigenstates, with energies $\ep_0$ and
$\ep_1$, of an atom. Suppose the energy is not greater than $\ep_1$. This fact
can be represented by the projector
%
\begin{equation} 
R = [0] + [1],
\label{eqn3}
\end{equation}
%
using the notation $[\psi] =\dya{\psi}$ for the projector corresponding to the
normalized ket $\ket{\psi}$. Does the truth of $R$ imply that the energy is
either $\ep_0$ or $\ep_1$? No, for any superposition of  $\ket{0}$ and
$\ket{1}$ also lies in the subspace on which $R$ projects, and so possesses the
property $R$, and most such superpositions
do not have well-defined energies. However, if one uses the framework
corresponding to the PDI
\begin{equation}
 \{ [0],\, [1],\, I-R\},
\label{eqn4}
\end{equation}
then in this framework $R$ implies that the energy is $\ep_0$ or $\ep_1$. There
are other incompatible frameworks, but \eqref{eqn4} is the one needed in order
to discuss the energy.

\xb
%
\outl{History Hilbert space}
%
\xa

The use of PDIs at a single time is easily extended to a family of histories
using the fact that a sequence of quantum properties (projectors) at a
succession of times is an element of a \emph{history Hilbert space}:%
\footnote{As first pointed out by Isham\cite{Ishm94}. } %
%END footnote
a tensor product of one-time Hilbert
spaces. 

% END Sec. III

%Sec. IV
\section{ Beamsplitters and Interference \label{sct4} }


\begin{figure}[h]
$$\includegraphics{fig09.eps}$$
\caption{(a) Photon source $S$ and beamsplitter $BS_1$ followed by two
  detectors at unequal distances. (b) Add mirrors and a second beamsplitter
  $BS_2$ to form a Mach-Zehnder interferometer. $BS_2$ can be removed, as
  sketched to the right of the dashed line.}
\label{fgr1}

\end{figure}

\xb
%
\outl{ In Fig.~\ref{fgr1}(a): Beamsplitter, two detectors. Photon arrival causes
  detection}
%
\xa

The first gedanken experiment is shown in Fig.~\ref{fgr1}(a). A photon is sent
into a beamsplitter and emerges as a coherent superposition of amplitudes in
the two output ports,
\begin{equation}
\ket{\psi_t} = \al \ket{a_t} + \bt \ket{b_t},
\label{eqn5}
\end{equation}
at time $t$. Here $\ket{a_t}$ is a wave packet traveling along the $a$ path;
$\ket{b_t}$ the $b$ path. The two paths lead to detectors $D_a$ and $D_b$
located at unequal distances from the beamsplitter. In a particular run of the
experiment the photon will be detected in either $D_a$ or $D_b$, but never in
both. Given a large number of runs, the fraction in which $D_a$ triggers will be
approximately $|\al|^2$; those in which $D_b$ triggers $|\bt|^2$. Someone with
experience in the laboratory will tend to think that on a particular run in
which $D_a$ is triggered, the photon after emerging from the beamsplitter
followed path $a$ and was absent from path $b$, while in a run in which $D_b$
detects the photon it followed path $b$ and was absent from $a$. But since
since the photon (or other quantum particle, such as a neutron, in an analogous
experiment) is invisible, it would seem useful to do additional tests to
corroborate these ideas. Here are some possibilities.

\xb
%
\outl{Stochastic splitting of paths at $BS_1$}
%
\xa

If during certain runs the $a$ path is blocked by some macroscopic absorbing
object---think of this as an intervention---$D_a$ never triggers, while the
fraction of runs with a $D_b$ detection remains unchanged. Or replace the block
by a mirror which deflects the photon off the $a$ path towards a third detector
$D^*_a$. Its average rate is that of $D_a$, whereas the rate of $D_b$ remains
unchanged. These observations support the idea that some sort of random process
occurs at $BS_1$, where the photon chooses (anthropomorphic language) either
the $a$ or the $b$ path, and then follows it till it reaches the corresponding
detector. In addition, if the photon is emitted at a well-defined time, and the
time of detection measured with sufficient precision, the time to reach
detector $D_a$ is less than that required to reach $D_b$.

\xb
%
\outl{Detection causes wavefn collapse?}
%
\xa

This seems very odd from a textbook perspective in which random processes only
occur when something is measured, resulting in the collapse of a wavefunction.
Let us explore that idea. Since detector $D_a$ is closer to $BS_1$ than $D_b$,
one might suppose that when the $\al\ket{a}$ part of $\ket{\psi_t}$ in
\eqref{eqn5} reaches detector $D_a$ and the photon is detected there, this
instantly removes the amplitude traveling towards $D_b$, which will not be
triggered on this particular run. Such a nonlocal effect seems strange, and may
be hard to reconcile with special relativity. An additional difficulty arises
in runs in which $D_a$ does \emph{not} detect the particle. Does such
nondetection result in the amplitude collapsing onto path $b$ at the instant of
nondetection? Or does the later detection by $D_b$ (retro)cause an earlier
collapse on path $a$ that prevents detection by $D_a$?

\xb
%
\outl{`Decoherence $\ra$ single path' disproved by interference,
  Fig.~\ref{fgr1}(b)}
%
\xa

Might it be that some sort of \emph{decoherence} process, caused by an
interaction of the photon with the environment, either at the beam splitter or
due to gas particles on the way to the detector, is at work, so that the
coherent state \eqref{eqn5} is no longer is the correct quantum description
when the two paths have separated by a macroscopic distance? Decoherence
removes quantum interference, so can be checked with the alternative
experimental arrangement shown in Fig.~\ref{fgr1}(b). Here a second
beamsplitter, $BS_2$, has been added, so that along with the first beamsplitter
$BS_1$ the result is a Mach-Zehnder interferometer. Let us suppose that the
beamsplitters have been chosen in such a way that the photon always emerges on
path $c$, leading to a detection by $D_c$, never $D_d$. Decoherence along the
$a$ or $b$ path inside the interferometer would eliminate the interference, so
can be excluded if the interferometer is functioning properly. 

\xb
%
\outl{Remove 2d beamsplitter (Wheeler Delayed Choice)}
%
\xa

Motivated by Wheeler's delayed choice paradox\cite{Whlr78}, let us suppose that
$BS_2$ can be moved out of the way, the situation sketched at the far right of
Fig.~\ref{fgr1}(b), in which case we are back at something that resembles
Fig.~\ref{fgr1}(a), but now the two paths cross. That nothing special happens
when they cross can be checked, as before, by blocking the paths at various
points before they cross, and by moving $D_c$ further away and checking the
timing. Thus one concludes (as did Wheeler) that the photon detected by $D_c$
was earlier following path $a$ before reaching the crossing point, and likewise
it was following path $b$ in a run in which it was detected by $D_d$. Having
considered the case with $BS_2$ absent, we must now try and understand what
happens when it is present. One can check that blocking either the $a$ or the
$b$ path inside the interferometer not only reduces, by the expected amount,
the number of times either $D_c$ or $D_d$ triggers, but in addition sometimes
$D_d$ triggers instead of $D_c$. What is even more interesting is to insert
variable phase shifters in each of the arms. While these can change the ratio
of the $D_d$ to $D_c$ counts, it remains fixed if \emph{both} phases are
changed by the \emph{same} amount. Thus the photon must in some sense have been
simultaneously present in \emph{both} arms of the interferometer. Very odd
given that with $BS_2$ absent we had good reason to believe the photon was in
either the $a$ arm or the $b$ arm after emerging from $BS_1$. Can removing
$BS_2$, as against leaving it in place, influence what happened \emph{before}
the photon arrived at the crossing point? Providing coherent answers to
questions of this sort is what one should expect from an adequate theory of
quantum causes.

%END Sec. IV

%BEGIN Sec. V
\section{Spin Half \label{sct5}}

\xb
%
\outl{Spin-half analog of Fig.~\ref{fgr1}. Spin-half Ag atom on straight path
to SG measuring $S_z$. Paths $a$, $c$ in figure $\lra$ $S_z=+1/2$; $b$, $d$
to $S_z=-1/2$ }
%
\xa

It is useful to consider a spin-half analog of the situation in Fig.~\ref{fgr1}
of Sec.~\ref{sct4}, since the mathematics of a two-dimensional Hilbert space,
which is all that is needed to discuss spin orientation, is simpler than what
is needed to describe the different spatial positions of the photon in
Fig.~\ref{fgr1}, and simplicity is important for understanding quantum systems
at the most fundamental level. Think of a spin-half silver atom moving along a
straight line until it enters a Stern-Gerlach (SG) measuring device with a
field and field gradient in the $z$ direction, so that it measures $S_z$,
yielding one of the two values $+1/2$ and $-1/2$ in units to $\hbar$. Let
$S_z=+1/2$ correspond to the $a$ path in Fig.~\ref{fgr1}(a), and its extension
into the $c$ path in Fig.~\ref{fgr1}(b), and $S_z=-1/2$ be the counterpart of
the $b$ and $d$ paths. The SG device replaces the pair of detectors $D_a$ and
$D_b$ in Fig.~\ref{fgr1}(a) or $D_c$ and $D_d$ in part (b).

\xb
%
\outl{Regions of uniform field replace Fig.~\ref{fgr1} beamsplitters}
%
\xa
\xb
%
\outl{Can be used to produce particles polarized in any desired direction,
 and to measure $S_w=\pm 1/2$ for any direction $w$ }
%
\xa

The beamsplitters in Fig.~\ref{fgr1} are replaced by limited regions of uniform
magnetic field chosen so that as the atom moves through the region its spin
precesses by the desired amount: a unitary transformation on its $d=2$
dimensional Hilbert space. Given a source of spin-polarized atoms, e.g., the
upper beam emerging from a SG apparatus, an appropriate uniform field allows
the preparation of particles with a spin polarization $S_v=+1/2$ in any spatial
direction $v$. Similarly, if the final SG measurement apparatus is preceded by
a suitable region of uniform field, the combination allows the measurement of
any spin component $S_w$, provided passage through the region of uniform field
maps $S_w=+1/2$ to $S_z=+1/2$, and $S_w=-1/2$ to $S_z=-1/2$. That this analog
of $BS_2$ in Fig.~\ref{fgr1}(b) is functioning properly can be checked, at
least in certain cases, by the less convenient process of rotating the final SG
magnet.

\xb
%
\outl{Alice prepares $S_x=+1/2$; Bob measures $S_z$. Spin at
  intermediate time =?}
%
\xa

Suppose that in this way Alice uses the analog of $BS_1$ to prepare a state
with $S_x=+1/2$ which travels on to Bob, whose SG measures $S_z$, the analog of
Fig.~\ref{fgr1}(b) with $BS_2$ absent. The results are random: in roughly half
the runs the outcome corresponds to $S_z=+1/2$, and the other half to
$S_z=-1/2$. What can one say about the spin state during the the time interval
when the particle is moving through the central field-free region on its way
from Alice's preparation to Bob's measurement? Alice might say $S_x=+1/2$,
since she has carried out a lot of checks to test whether her apparatus is
working properly, perhaps sending the beam into an SG with field gradient in
the $x$ direction. Bob is equally careful in checking his apparatus, and is
sure that $S_z$ was $-1/2$ in those runs where it indicated this
value, and $+1/2$ in the others. But there is no room in the spin-half Hilbert
space, no projector, that can represent the combination
``$S_x=+1/2\ \AND\ S_z=-1/2$'' or ``$S_x=+1/2\ \AND\ S_z=+1/2$''. We seem to
have a dispute between two parties. What shall we to do?

\xb
%
\outl{Bob on vacation. Alice can analyze preparation/measurement results using
$\FC_x$ or $\FC_z$ framework at intermediate time. She knows SFR non-combining
rule, which is useful for Kochen-Specker, Bell paradoxes}
%
\xa


Let us send Bob off on vacation. Alice is perfectly capable of designing and
building both the preparation and the measurement apparatus, and, as with any
competent experimenter, is conversant with the relevant theory. Thus for any
given run she knows both the preparation and the measurement outcome, and faces
the problem of what these data tell her about the spin state at an intermediate
time. She begins with the spin-half identity operator $I$, which is consistent
with any data but totally uninformative. This she can refine using either of
two different incompatible PDIs or frameworks:
\begin{equation}
 \FC_x: \{[x+],[x-]\} \quad \FC_z: \{[z+], [z-]\},
\label{eqn6}
\end{equation}
where $[x+]$ is the projector on the state $S_x=+1/2$; similarly for $[x-]$,
$[z+]$, and $[z-]$. If Alice uses $\FC_x$ she can conclude, on the basis of her
knowledge of the preparation, that at this intermediate time the property was
$[x+]$, $S_x=+1/2$. By using $\FC_z$ she can infer that in a run in which the
measurement outcome corresponds to $S_z=+1/2$---recall that she has carefully
calibrated and checked the measurement device ahead of time---that on this run
$S_z$ was $+1/2$ and not $-1/2$. Similarly a measurement outcome corresponding
to $S_z=-1/2$ leads her to conclude that $S_z$ earlier had this value. Thus the
conditions in \eqref{eqn1} are satisfied, and Alice can properly identify
the intermediate time $S_z$ value as the \emph{cause} of the later measurement
outcome. In addition, having passed the final exam in an up-to-date
introductory quantum mechanics course, Alice knows the rules needed to reach
sensible conclusions in a regime where quantum projectors do not commute,
and that assuming classical unicity will only lead to paradoxes such as
Kochen-Specker and Bell nonlocality. (The latter is discussed in
Sec.~\ref{sbct6b} below.)

\xb
%
\outl{Additional details: histories, consistency conditions: see WQMM}
%
\xa

While the above discussion identifies the central issue, a number of details
needed for a more complete description have been omitted. These would include
the use of quantum histories (Sec.~\ref{sct3}), a proper quantum description,
at least in principle, of the measurement apparatus and its macroscopic
outcomes, and assigning probabilities to histories in a closed quantum system
in a manner satisfying appropriate consistency conditions. For these we refer
the reader to \cite{Grff17b} and references given there. The fundamental point
remains the same: Alice, in order to identify the \emph{cause} of the $S_z$
measurement outcome, must \emph{choose} to use $\FC_z$ rather than $\FC_x$ or
some other framework. There is nothing irrational about this choice, or similar
choices made every day in the laboratory by researchers who regularly interpret
their results in terms of the microscopic causes their apparatus was designed
to detect.



\section{ Generalizations  \label{sct6} }



\subsection{ Projective Measurements \label{sbct6a}}


\xb
\outl{Projective measurement outcomes caused by prior microscopic states;
  Hilbert space projectors $\lra$ Bell's `beables'}
\xa


The ideas just discussed in the case of a spin-half particle can immediately be
generalized to that of \emph{projective measurements}, in which a particular
collection of quantum properties of a microscopic system, represented by a PDI,
is associated in a one-to-one fashion with a set of distinct macroscopic
outcomes (``pointer positions'' in the archaic language of quantum
foundations). That is, there is a one-to-one association between the
microscopic cause represented by the projector $P_j$ and the later pointer at
position $j$. This correspondence can be checked through a set of calibration
runs in which particles with a particular microscopic property are repeatedly
sent into the measurement apparatus, and lead to the expected macroscopic
outcomes. An appropriate quantum analysis using histories then justifies
thinking of the previous microscopic property as the cause of the later. In
this sense a projective measurement actually measures something, a property of
the measured system just before the measurement took place. Bell's tirade in
\cite{Bll901} can be answered by associating what he called ``beables'' with
quantum properties represented by Hilbert space projectors. Textbooks assert
that incompatible properties, thus corresponding to incompatible PDIs, cannot
be simultaneously measured, which is true. What they ought to say is that this
impossibility arises from the fact that such joint properties do not exist, and
even skilled experimenters cannot measure what isn't there.

\xb
%
\outl{Reader referred to WQMM for justification of above + extension
  to POVMs. `Measurement problem' resolved thru use of a framework appropriate
  for discussing lab experiments, rather that Schr\"odinger cat}
%
\xa

The reader is referred to \cite{Grff17b}, along with Chs.~17 and 18 of
\cite{Grff02c}, for a general discussion of projective measurements, as well as
how to think about POVMs (positive-operator-valued measurements), where
identifying causes is a more subtle matter. The infamous \emph{measurement
  problem}, unitary time evolution leading to macroscopic coherent
superpositions (as in Schr\"odinger's cat) which cannot be given a sensible
physical interpretation, dissolves once one adopts Born's proposal (with which
von Neumann concurred, and which is fundamental to the CH approach) that
Schr\"odinger's wave is a means of calculating probabilities, rather than a
unique description of what is going on in a physical system. The point is that
there are various different possible and mutually incompatible quantum
frameworks for describing measurement processes, and if one wants to discuss
what goes on in the laboratory it is reasonable to adopt one that makes
physical sense to the experimenter.

\subsection{Einstein-Podolsky-Rosen-Bohm (EPRB) \label{sbct6b}}

\xb \outl{Bohm version of EPR. (References to Bell inequalities deliberately
  omitted.) Alice \& Bob measurements. Analogous photon measurements agree with
  Qm predictions} \xa

An example that illustrates the general case of a projective measurement as
discussed above arises in the famous Einstein-Podolsky-Rosen paradox
\cite{EnPR35}, here considered in Bohm's spin-half version \cite{Bhm51s}, which
is easily translated into a discussion of polarizations in the case of two
photons produced by down conversion, as measured in tests of Bell inequalities.
Two spin-half particles $S_a$ and $S_b$ are prepared in a singlet state and
sent off in different directions to Alice and Bob who measure particular spin
components. The outcomes can be understood as caused by the corresponding
properties of the individual particles just before the measurements took place.
The experimental results (using photons) have by now amply confirmed the
straightforward quantum mechanical predictions.

\xb
%
\outl{Bell error from combining results from separate runs measuring 
incompatible quantities}
%
\xa

A fundamental mistake behind derivations of Bell inequalities, such as CHSH, is
the idea that one can combine statistics from different runs in which
incompatible properties are measured. Suppose Alice measures $S_x$ on certain
runs, and $S_z$ on others. It is tempting to suppose that on a run in which
$S_x$ was measured, this particle also possessed a value of $S_z$ which would
have been revealed had $S_z$ been measured instead. Such counterfactual
reasoning assumes classical unicity, and fails when dealing with incompatible
quantum properties. When invalid classical reasoning of this sort is abandoned,
derivations of Bell inequalities fall apart, and mysterious nonlocal influences
vanish; see \cite{Grff20} for details.

\xb
%
\outl{Alice measures $S_x$, Bob measures $S_z$, results $\lra$ common cause.
NONLOC has details}
%
\xa

From a causal perspective, if Alice measures $S_x$ and Bob measures $S_z$, to
consider a particular example, the respective measurement outcomes, say $+1/2$
and $-1/2$ can be assigned to the corresponding causes, the properties
$S_{ax}=+1/2$, and $S_{bz}=-1/2$, just before the measurements took place. If
the particles have been traveling through a field-free region, these properties
can be traced back to a common cause right after the entangled state was first
created. One could also introduce interventions by letting either or both of
the particles pass through regions of uniform magnetic field that rotate the
spin polarization, in order to check that the correlations do, indeed, arise
from a common cause.

% END subsection VI B

\subsection{ Further Generalizations \label{sbct6c}}

\xb
%
\outl{CH stochastic histories should be enough to study causes in most cases of
  physical interest}
%
\xa

\xb
%
\outl{CH histories satisfy conditions of Cl stochastic models, so methods of
  assigning probabilities in Cl case can be applied in Qm domain}
%
\xa

\xb
%
\outl{There could be problems dealing with interventions}
%
\xa



There is no reason why the CH approach that works in the case of projective
measurements and POVMs cannot be extended to more complicated situations. The
use of appropriate histories to model quantum stochastic time development would
seem to be sufficient for analyzing causes in most situations of physical
interest, or at least there is no reason to think otherwise until
counterexamples have been identified. For a closed quantum system probabilities
can be assigned to a family of histories using the extended Born rule as long
as certain \emph{consistency conditions} are satisfied.%
\footnote{%
  The earliest formulation of these conditions \cite{Grff84} should be
  replaced by the correct version due to Gell-Mann and Hartle
  \cite{GMHr93}: the vanishing of off-diagonal elements in their decoherence
  functional, or orthogonality of the chain operators in Ch.~10 of
  \cite{Grff02c}. The correct form is given in \cite{Grff19b}.} %
% END footnote
Note that a (quantum!) measuring apparatus can itself be part of such a closed
system. Since such a family satisfies the usual conditions of a classical
stochastic model, the general methods used to analyze causes in the classical
case can at once be applied to the quantum domain. A possible difficulty, worth
exploring through specific models, could arise in the case of interventions
used to distinguish genuine causes from mere correlations, since one would need
to pay some attention to consistency conditions and the single framework rule
in the extended histories framework needed to accommodate the intervention.

\xb
%
\outl{Open Qm systems needs further exploration. CH allows for measuring
  apparatus to be made part of closed system}
%
\xa

The situation with open quantum systems is more complicated, and deserves
further exploration. An approach in which the environment is modeled by a
separate quantum system, which together with the one of interest forms a closed
quantum system to which CH rules can be applied, is an obvious possibility.
Note that there is no difficulty, at least in principle, in making a measuring
apparatus part of a closed system; see \cite{Grff17b} and Chs.~17 and 18 in
\cite{Grff02c} for the basic idea.

% END VI

% BEGIN VII
\section{ \ Quantum Causal Models \label{sct7}}


\xb
%
\outl{Wood \& Spekkens. Quantum Causal Models (QCM)}
%
\xa

The pioneering work of Wood and Spekkens\cite{WdSp15} showed that a
straightforward attempt to apply certain ideas of causal modeling which had
shown considerable success in ordinary (classical) situations, such as
identifying causes of disease, ran into serious difficulties when applied to
the EPRB situation discussed in Sec.~\ref{sbct6b}. Subsequent work attempting
to arrive at more satisfactory results led to the study of \emph{Quantum Causal
  Models} (QCM) in an effort to understand correlations and causes in a quantum
context. Items \cite{CsSh16,Shrp19,DlRS22} are selected from a rather
substantial literature on the topic, and provide some references to other work.
Neither the mathematics nor the associated concepts are easy to understand;
perhaps \cite{Shrp19} is the most accessible for someone unfamiliar with this
approach.

\xb
%
\outl{QCM unable to address simple situations in Secs.~\ref{sct4},\ref{sct5}}
%
\xa
\xb
%
\outl{QCM `sledgehammer' contrasts with EPRB simplicity, including possible
  interventions, in Sec.~\ref{sbct6c}}
%
\xa 

So far as this author can tell, it is unable to address the simple
laboratory situations discussed above in Sections~\ref{sct4} and \ref{sct5}, as
it lacks means to deal with noncommuting Hilbert-space projectors. When it
comes to EPRB correlations, Shrapnel \cite{Shrp19} describes the QCM approach
as a ``sledgehammer'', which is certainly apt when it is compared with the
simplicity of the analysis in Sec.~\ref{sbct6b}, where the measurement outcomes
are caused by microscopic properties of the particles just before the
measurements take place, using devices designed for just that purpose, and
carefully tested to ensure their reliability. No sledgehammer is needed when
microscopic descriptions use quantum Hilbert space projectors.

\xb
%
\outl{Plausible reasons for QCM complexity. Spekkens toy measurement model;
Leibniz principle}
%
\xa

Two reasons among others may have led to the complexity of the QCM approach.
The first is an inadequate understanding of what quantum measurements measure,
as illustrated by Spekkens' toy model \cite{Spkk07} which makes no mention of
Hilbert-space quantum properties. The second is what Spekkens refers to in
\cite{Spkk19} as the \emph{Leibniz principle}, that in a situation with
specific macroscopic inputs and outputs, there is a unique ontological,
thus microscopic,
description applicable at an intermediate time. Clearly this is classical
unicity as opposed to the quantum pluricity used by Alice in the example
discussed in Sec.~\ref{sct5}.

\xb
%
\outl{QCM might be useful, but practitioners need to pay attention to projectors,
PDIs, histories}
%
\xa

None of this is to say that the QCM approach is without value. However, it
would seem worthwhile for its practitioners to pay attention to Hilbert space
projectors, noncommutation, PDIs, and the quantum histories used in CH, in
order to achieve a more useful, and hopefully much simpler, connection between
the mathematical formulation and laboratory physics. There is much that needs
to be understood about the quantum physics of systems at successive times, and
the identification of causes consistent with fundamental quantum principles
could play a very useful role in such investigations.

% END VI

\section{ \ \ Summary and Conclusion \label{sct8}}


\xb
%
\outl{Key ideas: Physical properties $\lra$ HS projectors. They do not always
  commute, unlike Cl indicators. Result is pluricity of Qm HS vs unicity of Cl
  phase space}
%
\xa

The analysis of quantum causes in this paper is based upon certain key ideas,
foremost among them the notion that microscopic causes useful in understanding
the results of laboratory experiments are physical properties represented
mathematically by projectors on the quantum Hilbert space, Sec.~\ref{sct3}. A
key difference between projectors and the corresponding indicator functions on
a classical phase space is that when projectors do not commute their product is
not a projector, and thus cannot represent a physical property. Hence in
Hilbert space quantum mechanics one encounters a \emph{pluricity} of
incompatible descriptions, each based upon a particular collection of commuting
projectors, or equivalently, the corresponding PDI. This contrasts with the
\emph{unicity} of classical phase space, where no such incompatibility arises.

\xb
%
\outl{Ignoring noncommutation $\ra$ paradoxes, e.g., Wheeler delayed choice}
%
\xa
\xb
%
\outl{CH avoids paradoxes via SFR: use single PDI for logical reasoning; allow
choice for analyzing different aspects of an experiment}
%
\xa

Ignoring noncommutation leads to all sorts of insoluble quantum paradoxes,
including Wheeler's delayed choice, Sec.~\ref{sct4}. The consistent histories
approach, illustrated in the spin-half example in Sec.~\ref{sct5}, avoids such
paradoxes through its single-framework rule that specifies that logical
reasoning about a quantum system must be based upon a single PDI, even though
the physicist may very well employ different, incompatible PDI's when analyzing
different aspects of an experiment.

\xb
%
\outl{ Examples in Secs.~\ref{sct4} \& \ref{sct5} are extended to general
  projective measurements in Sec.~\ref{sbct6a}, to EPRB in Sec.~\ref{sbct6b}
  with simpler results than QCM, discussed in Sec.~\ref{sct7}. Possible
  extensions of history approach to including classical causes in QM:
  Sec.~\ref{sbct6c} }
%
\xa

The examples in Sec.~\ref{sct4} and \ref{sct5} involve a two-state projective
measurement, but the idea is easily extended to a general projective
measurement, Sec.~\ref{sbct6a}, resolving Bell's complaint \cite{Bll901} by
identifying ``beables'', his terminology, with properties represented by Hilbert
space projectors. In particular the correlations in the
Einstein-Podolksy-Rosen-Bohm paradox, Sec.~\ref{sbct6b}, can be explained very
simply as resulting from a common cause, with no need for the complicated
``sledgehammer'' developed under the heading of Quantum Causal Models,
Sec.~\ref{sct7}. A much simpler way of consistently embedding ideas from the
classical theory of causes (DAGs and all that) in a quantum context should be
possible using quantum histories, Sec.~\ref{sbct6c}, but this remains to be
developed.

\xb
%
\outl{CH, while counterintuitive, can be taught to undergrads. Is much simpler
  than BvN Qm logic. Need not be final answer to Qm difficulties, but has
  resolved many paradoxes}
%
\xa

In closing it is worth remarking that the CH approach, while it is in certain
ways counterintuitive---perhaps necessary, given that modern quantum mechanics
was developed precisely because classical intuition failed---is much simpler
than the earlier quantum logic of Birkhoff and von Neumann, and can be taught
to undergraduate students in an introductory quantum mechanics course. It may
not resolve all the conceptual difficulties of quantum theory, but its success
in dealing with a number of quantum paradoxes, as in Chs.~20 to 25 of
\cite{Grff02c}, suggests it is worth further exploration.

\xb
\section*{Acknowledgements}
\xa

The author expresses his appreciation to Carnegie-Mellon University and its
Physics Department for continuing support of his activities as an emeritus
faculty member.

\begin{thebibliography}{10}

\bibitem{Prl09}
Judea Pearl.
\newblock {\em Causality}.
\newblock Cambridge University Press, New York, 2009.
\newblock 2d edition.

\bibitem{SpGs00}
Peter Spirtes, Clark Glymour, and Richard Scheines.
\newblock {\em Causation, Prediction, and Search}.
\newblock MIT Press, 2000.
\newblock 2d edition.

\bibitem{Htch18}
Christopher Hitchcock.
\newblock Probabilistic causation.
\newblock {\em Stanford Encyclopedia of Philosophy}, 2018.

\bibitem{WdSp15}
Christopher~J. Wood and Robert~W. Spekkens.
\newblock The lesson of causal discovery algorithms for quantum correlations:
  Causal explanations of bell-inequality violations require fine-tuning.
\newblock {\em New J. Phys.}, 17:033002, 2015.
\newblock arXiv:1208.4119 v2.

\bibitem{vNmn32b}
Johann von Neumann.
\newblock {\em Mathematische Grundlagen der Quantenmechanik}.
\newblock Springer-Verlag, Berlin, 1932.
\newblock English translation by R. T. Beyer: \textit{Mathematical Foundations
  of Quantum Mechanics}, Princeton University Press, Princeton, New Jersey
  (1955 and 2018).

\bibitem{Bll901}
J.~S. Bell.
\newblock Against measurement.
\newblock In Arthur~I. Miller, editor, {\em Sixty-Two Years of Uncertainty},
  pages 17--31. Plenum Press, New York, 1990.
\newblock Reprinted in J. S. Bell, \textit{Speakable and Unspeakable in Quantum
  Mechanics, 2d ed.} (Cambridge University Press, 2004), pp.~213-231.

\bibitem{BrvN36}
G.~Birkhoff and J.~von Neumann.
\newblock The logic of quantum mechanics.
\newblock {\em Ann. Math.}, 37:823--843, 1936.

\bibitem{Grff19b}
Robert~B. Griffiths.
\newblock The {C}onsistent {H}istories {A}pproach to {Q}uantum {M}echanics.
\newblock {\em Stanford Encyclopedia of Philosophy}, 2019.
\newblock https://plato.stanford.edu/entries/qm-consistent-histories/.

\bibitem{Grff02c}
Robert~B. Griffiths.
\newblock {\em Consistent Quantum Theory}.
\newblock Cambridge University Press, Cambridge, U.K., 2002.
\newblock http://quantum.phys.cmu.edu/CQT/.

\bibitem{Grff14}
Robert~B. Griffiths.
\newblock The new quantum logic.
\newblock {\em Found. Phys.}, 44:610--640, 2014.
\newblock arXiv:1311.2619.

\bibitem{Grff17b}
Robert~B. Griffiths.
\newblock What quantum measurements measure.
\newblock {\em Phys. Rev. A}, 96:032110, 2017.
\newblock arXiv:1704.08725.

\bibitem{Grff20}
Robert~B. Griffiths.
\newblock Nonlocality claims are inconsistent with {H}ilbert-space quantum
  mechanics.
\newblock {\em Phys. Rev. A}, 101:022117, 2020.
\newblock arXiv:1901.07050.

\bibitem{Ishm94}
C.~J. Isham.
\newblock Quantum logic and the histories approach to quantum theory.
\newblock {\em J. Math. Phys.}, 35:2157--2185, 1994.

\bibitem{Whlr78}
John~Archibald Wheeler.
\newblock The ``{P}ast'' and the ``{D}elayed-{C}hoice'' {D}ouble-{S}lit
  {E}xperiment.
\newblock In A.~R. Marlow, editor, {\em Mathematical Foundations of Quantum
  Theory}, pages 9--48. Academic Press, New York, 1978.

\bibitem{EnPR35}
A.~Einstein, B.~Podolsky, and N.~Rosen.
\newblock Can quantum-mechanical description of physical reality be considered
  complete?
\newblock {\em Phys. Rev.}, 47:777--780, 1935.

\bibitem{Bhm51s}
David Bohm.
\newblock {\em Quantum Theory}, chapter~22.
\newblock Prentice Hall, Englewood Cliffs, N.J., 1951.

\bibitem{Grff84}
Robert~B. Griffiths.
\newblock Consistent histories and the interpretation of quantum mechanics.
\newblock {\em J. Stat. Phys.}, 36:219--272, 1984.

\bibitem{GMHr93}
Murray Gell-Mann and James~B. Hartle.
\newblock Classical equations for quantum systems.
\newblock {\em Phys. Rev. D}, 47:3345--3382, 1993.

\bibitem{CsSh16}
Fabio Costa and Sally Shrapnel.
\newblock Quantum causal modelling.
\newblock {\em New J. Phys.}, 18:063032, 2016.

\bibitem{Shrp19}
Sally Shrapnel.
\newblock Discovering quantum causal models.
\newblock {\em British J. Phil. Sci.}, 70:1--25, 2019.
\newblock http://philsci-archive.pitt.edu/13098/.

\bibitem{DlRS22}
Patrick~J. Daley, Kevin~J. Resch, and Robert~W. Spekkens.
\newblock Experimentally adjudicating between different causal accounts of
  {B}ell-inequality violations via statistical model selection.
\newblock {\em Phys. Rev. A}, 105:042220, 2022.
\newblock arXiv:2108.00053.

\bibitem{Spkk07}
Robert~W. Spekkens.
\newblock Evidence for the epistemic view of quantum states: {A} toy theory.
\newblock {\em Phys. Rev. A}, 75:032110, 2007.
\newblock arXiv:quant-ph/0401052.

\bibitem{Spkk19}
Robert~W. Spekkens.
\newblock The ontological identity of empirical indiscernibles: Leibniz's
  methodological principle and its significance in the work of {E}instein.
\newblock arXiv:1909.04628, 2019.

\end{thebibliography}

\end{document}

% References. One does NOT need a separate section label. 
% See b/latex/bibtex >info >OVERVIEW for how to run bibtex
\bibliographystyle{unsrt}
\bibliography{/home/rgrif/qms/bibs/main}
% To make self contained: Comment out \bibliography{}, insert cqc09.bbl

\end{document}
