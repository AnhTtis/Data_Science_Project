% qmanus/cause/cqc1101.tex 1/31/24
% 
% Register t: Secs.~\ref{sct} and \ref{sct}
% Remove \nq from main text before final version

	% Begin
\documentclass[12pt]{article}
\usepackage{latexsym}
\usepackage{amsmath}
\usepackage{indentfirst}
\usepackage{cite} % Orders and compresses citation strings
%\usepackage{pstricks} 
%\usepackage{pstricks,pst-node}
\usepackage{graphicx}
%\usepackage[notcite]{showkeys}

	% Page dimension
\oddsidemargin 0 cm
\evensidemargin 0 cm
\topmargin -1  cm \textheight 24  cm \textwidth 16.5 cm
\raggedbottom

	% RBG style file
%\input /home/rgrif/latex/defs/grifg.sty % For my own use 
%\input /home/rgrif/latex/defs/gril08.sty % There may be later versions

%latex/defs/gril08.sty 12/28/11
% Definitions for Latex2e amsmath
% GENERAL, MATH CAL (SCRIPT), BOLD FACE, BLACKBOARD BOLD, GREEK
% 8/19/13 Added \vbB 12/4/15. Corrected \AND, \NOT, \OR using an \mbox{}
% 5/27/18 Added \mats{}

		% GENERAL LIST
\long\def\ca#1\cb{} %Use for commenting out: \ca...\cb

\newcommand{\ad}{^\dagger }
\newcommand{\adp}{^{\phantom{\dagger}}}
\newcommand{\AND}{\mbox{\small AND}}
\newcommand{\Arg}{\mathop{\rm Arg}\nolimits}
\newcommand{\avg}[1]{\langle #1\rangle }
\newcommand{\becs}{\begin{cases}}
\newcommand{\bem}{\begin{matrix}}
\newcommand{\blp}{\bigl(} 
\newcommand{\Blp}{\Bigl(} 
\newcommand{\bpp}[1]{\bigl(#1\bigr)}
\newcommand{\Bpp}[1]{\Bigl(#1\Bigr)}
\newcommand{\bra}[1]{\langle#1|}
\newcommand{\Brp}{\Bigr)}
\newcommand{\brp}{\bigr)}
\newcommand{\bsk}{\bigskip }
\newcommand{\bsl}{$\backslash$}
\newcommand{\ck}{\breve }
\newcommand{\cln}{\colon\!} %Symmetrizes some math expressions
\newcommand{\coln}{\hbox{:}}
\newcommand{\colo}{\,\hbox{:}\,}
\newcommand{\Cov}{{\rm Cov}}
\newcommand{\Det}{{\rm Det}}
\newcommand{\dg}{^\circ } %Degree
\newcommand{\dst}{\displaystyle } %\tst for textstyle
\newcommand{\dya}[1]{|#1\rangle\langle#1|}
\newcommand{\dyad}[2]{|#1\rangle\langle#2|}
\newcommand{\emp}{\emptyset }
\newcommand{\encs}{\end{cases}}
\newcommand{\enm}{\end{matrix}}
\newcommand{\fac}{\mspace{1mu}!}
\newcommand{\hf}{{\textstyle\frac{1}{2} }}
\newcommand{\Hf}{\frac{1}{2} }
\newcommand{\hquad}{\mspace{8 mu}} 
\newcommand{\inp}[1]{\langle#1|#1\rangle }
\newcommand{\inpd}[2]{\langle#1|#2\rangle }
\newcommand{\ket}[1]{|#1\rangle }
\newcommand{\lbrk}[1]{\left\{\vrule height #1cm depth #1cm width 0pt\right.}
\newcommand{\Ln}{\mathop{\rm Ln}\nolimits}
\renewcommand{\Im}{\hbox{Im}} %Imaginary, replaces standard Tex definition
\newcommand{\lgl}{\langle } 
\newcommand{\lp}{\left(} % And see \rp
\newcommand{\Lp}{\Bigl(}  % And see \Rp
\newcommand{\lra}{\leftrightarrow }
\newcommand{\mab}[1]{\begin{matrix}#1\end{matrix}}
\newcommand{\mat}[1]{\left(\begin{matrix}#1\end{matrix}\right)}
\newcommand{\matd}[1]{\left|\begin{matrix}#1\end{matrix}\right|}
\newcommand{\mats}[1]{\left[\begin{matrix}#1\end{matrix}\right]}
\newcommand{\mns}{\negmedspace - \negmedspace }
\renewcommand{\mod}{\mathop{\rm mod}\nolimits} %Replaces Latex form
\newcommand{\msk}{\medskip }
\newcommand{\mt}{\mapsto }
\newcommand{\mte}[2]{\langle#1|#2|#1\rangle }
\newcommand{\mted}[3]{\langle#1|#2|#3\rangle }
\newcommand{\negg}{\sim\!} %Also see \sm
\newcommand{\nid}{\noindent }
\newcommand{\NOT}{\mbox{\small NOT}}
\newcommand{\od}{\odot }
\newcommand{\OR}{\mbox{\small OR}}
\newcommand{\ot}{\otimes }
\newcommand{\phd}[1]{\vphantom{\vrule height 0pt depth #1}} %\phd{4ex}
\newcommand{\phu}[1]{\vphantom{\vrule height #1 depth 0pt}} %\phu{10pt}
\newcommand{\pls}{\negmedspace + \negmedspace }
\newcommand{\pr}{\partial }
\newcommand{\pval}{\rlap{$\mspace{5mu}-$}}%\pval\int ->principal value integral
\newcommand{\ra}{\rightarrow }
\newcommand{\Ra}{\Rightarrow }
\newcommand{\rbrk}[1]{\left.\vrule height #1cm depth #1cm width 0pt\right\}}
\renewcommand{\Re}{\hbox{Re}} %Real, replaces standard Tex definition
\newcommand{\rgl}{\rangle }
\newcommand{\Rn}{{\rm Rn}}
\newcommand{\rp}{\right)}
\newcommand{\Rp}{\Bigr)}
\newcommand{\sep}{\centerline{\vrule height 0.5pt depth 0pt width 8cm}}
\newcommand{\sepb}{%
\vspace{-3ex}\begin{center}\vrule height 0.5pt depth 0pt width 8cm\end{center}}
\newcommand{\sgn}{\mathop{\rm sgn}}
\newcommand{\sm}{\sim\!\!} %Also see \negg
\newcommand{\ssk}{\smallskip }
\newcommand{\st}{\sqrt{2}}
\newcommand{\tm}{\times }
\newcommand{\Tr}{{\rm Tr}}
\newcommand{\trp}{^\textup{T}}
\newcommand{\tst}{\textstyle } %See \dst for displaystyle
\newcommand{\ttt}{\texttt}
\newcommand{\Var}{{\rm Var}}
\newcommand{\vb}{\,|\,}
\newcommand{\vbb}{\mspace{1mu}|\mspace{1mu}}
\newcommand{\vbB}{\boldsymbol{\mid}}
\newcommand{\vbl}{\,\boldsymbol{|}\,}


		% MATH CAL (SCRIPT)

\newcommand{\AC}{{\mathcal A}}
\newcommand{\BC}{{\mathcal B}}
\newcommand{\CC}{{\mathcal C}}
\newcommand{\DC}{{\mathcal D}}
\newcommand{\EC}{{\mathcal E}}
\newcommand{\FC}{{\mathcal F}}
\newcommand{\GC}{{\mathcal G}}
\newcommand{\HC}{{\mathcal H}}
\newcommand{\IC}{{\mathcal I}}
\newcommand{\JC}{{\mathcal J}}
\newcommand{\KC}{{\mathcal K}}
\newcommand{\LC}{{\mathcal L}}
\newcommand{\MC}{{\mathcal M}}
\newcommand{\NC}{{\mathcal N}}
\newcommand{\OC}{{\mathcal O}}
\newcommand{\PC}{{\mathcal P}}
\newcommand{\QC}{{\mathcal Q}}
\newcommand{\RC}{{\mathcal R}}
\newcommand{\SC}{{\mathcal S}}
\newcommand{\TC}{{\mathcal T}}
\newcommand{\UC}{{\mathcal U}}
\newcommand{\VC}{{\mathcal V}}
\newcommand{\WC}{{\mathcal W}}
\newcommand{\XC}{{\mathcal X}}
\newcommand{\YC}{{\mathcal Y}}
\newcommand{\ZC}{{\mathcal Z}}


		% BOLD FACE

\newcommand{\bld}[1]{\boldsymbol{#1}}

	% CAPS BOLD FACE

\newcommand{\AB}{\textbf{A}}
\newcommand{\BB}{\textbf{B}}
\newcommand{\CB}{\textbf{C}}
\newcommand{\DB}{\textbf{D}}
\newcommand{\EB}{\textbf{E}}
\newcommand{\FB}{\textbf{F}}
\newcommand{\GB}{\textbf{G}}
\newcommand{\HB}{\textbf{H}}
\newcommand{\IB}{\textbf{I}}
\newcommand{\JB}{\textbf{J}}
\newcommand{\KB}{\textbf{K}}
\newcommand{\LB}{\textbf{L}}
\newcommand{\MB}{\textbf{M}}
\newcommand{\NB}{\textbf{N}}
\newcommand{\OB}{\textbf{O}}
\newcommand{\PB}{\textbf{P}}
\newcommand{\QB}{\textbf{Q}}
\newcommand{\RB}{\textbf{R}}
\newcommand{\SB}{\textbf{S}}
\newcommand{\TB}{\textbf{T}}
\newcommand{\UB}{\textbf{U}}
\newcommand{\VB}{\textbf{V}}
\newcommand{\WB}{\textbf{W}}
\newcommand{\XB}{\textbf{X}}
\newcommand{\YB}{\textbf{Y}}
\newcommand{\ZB}{\textbf{Z}}

	% LOWER CASE BOLD FACE

\newcommand{\aB}{\textbf{a}}
\newcommand{\bB}{\textbf{b}}
\newcommand{\cB}{\textbf{c}}
\newcommand{\dB}{\textbf{d}}
\newcommand{\eB}{\textbf{e}}
\newcommand{\fB}{\textbf{f}}
\newcommand{\gB}{\textbf{g}}
\newcommand{\hB}{\textbf{h}}
\newcommand{\iB}{\textbf{i}}
\newcommand{\jB}{\textbf{j}}
\newcommand{\kB}{\textbf{k}}
\newcommand{\lB}{\textbf{l}}
\newcommand{\mB}{\textbf{m}}
\newcommand{\nB}{\textbf{n}}
\newcommand{\oB}{\textbf{o}}
\newcommand{\pB}{\textbf{p}}
\newcommand{\qB}{\textbf{q}}
\newcommand{\rB}{\textbf{r}}
\newcommand{\sB}{\textbf{s}}
\newcommand{\tB}{\textbf{t}}
\newcommand{\uB}{\textbf{u}}
\newcommand{\vB}{\textbf{v}}
\newcommand{\wB}{\textbf{w}}
\newcommand{\xB}{\textbf{x}}
\newcommand{\yB}{\textbf{y}}
\newcommand{\zB}{\textbf{z}}

	% SPECIAL BOLD
\newcommand{\zrB}{\textbf{0}}

		% BLACKBOARD BOLD
\newcommand{\Cbb}{\mathbb{C}}
\newcommand{\Rbb}{\mathbb{R}}

                % GREEK
% The following Greek definitions do not replace
% anything in Tex or Latex.  
%The following are NOT abbreviated: eta mu nu xi pi rho tau phi chi psi
% omicron does not occur: use o
% The following caps are NOT abbreviated: Xi, Pi, Phi, Psi
\newcommand{\al}{\alpha }
\newcommand{\bt}{\beta }
\newcommand{\gm}{\gamma }
\newcommand{\Gm}{\Gamma }
\newcommand{\dl}{\delta }
\newcommand{\Dl}{\Delta }
\newcommand{\ep}{\epsilon}
\newcommand{\vep}{\varepsilon}
\newcommand{\zt}{\zeta }
\renewcommand{\th}{\theta } %Latex \th = thor n
\newcommand{\vth}{\vartheta }
\newcommand{\Th}{\Theta }
\newcommand{\io}{\iota }
\newcommand{\kp}{\kappa }
\newcommand{\lm}{\lambda }
\newcommand{\Lm}{\Lambda }
\newcommand{\vpi}{\varpi}
\newcommand{\vrh}{\varrho}
\newcommand{\sg}{\sigma }
\newcommand{\vsg}{\varsigma}
\newcommand{\Sg}{\Sigma }
\newcommand{\up}{\upsilon }
\newcommand{\Up}{\Upsilon }
\newcommand{\vph}{\varphi }
\newcommand{\om}{\omega }
\newcommand{\Om}{\Omega }
        % Additional macros
\def\cites#1{REF #1 FER}

          % OUTLINE IN TEXT  [latex/platg 3/27/2017]
%     INSTRUCTIONS. Default is text with outlines, and \ca, \cb are used to 
% comment  out revisions in definitions for other cases. 
%     TEXT, NO OUTLINE:  \ca -> %\ca, \cb -> %\cb below 'TEXT ALONE'. 
%     OUTLINE, NO TEXT:  \ca -> %\ca, \cb -> %\cb below 'OUTLINE ALONE'
% Newpage. \np for text alone, \nq for text + outline, \nr for outline alone
% Material to be EXCLUDED for OUTLINE ALONE is between \xa and \xb. 
% Presence of \xb elsewhere does not matter, as it is ignored.
 
% DEFINITIONS FOR TEXT INCLUDING OUTLINES
\def\outl#1{\par{\medskip\noindent\hspace*{0.1cm}\bf
      \mathversion{bold}#1\mathversion{normal}\smallskip} }
\def\np{} \def\nq{\newpage } \def\nr{} \def\xa{} \def\xb{} \def\xn{} \def\xp{}

% TEXT ALONE USE: %\ca %\cb. % INCLUDE OUTLINE: \ca + \cb.  
%\ca
 \def\outl#1{}\def\np{\newpage }\def\nq{}\def\xa{}\def\xb{}\def\xn{}\def\xp{}
%\cb

% OUTLINE ALONE USE %\ca %\cb, OTHERWISE \ca \cb
\ca
 \def\outl#1{\par{\medskip\noindent\hspace*{.5cm}\bf
      \mathversion{bold}#1\mathversion{normal}\smallskip} }
 \long\def\xa#1\xb{} %This comments out segments from \xa to the next \xb
 \def\nq{} \def\nr{\newpage }\def\xn{\nopagebreak }\def\xp{\pagebreak }
\cb
          % END OUTLINE IN TEXT

	% Phys Rev A style
%The following will change section, subsection to Phys Rev A style
\renewcommand{\thesection}{\Roman{section}}  %latex
\renewcommand{\thesubsection}{\thesection\ \Alph{subsection}} %latex

\begin{document}

\title{Consistent Quantum Causes}
\author{Robert B. Griffiths\thanks{Electronic address: rgrif@cmu.edu}\\
Department of Physics\\
Carnegie Mellon University\\
Pittsburgh, PA 15213}

\date{Version of 31 January 2024}
\maketitle



\xa
\begin{abstract}

  Developing a quantum analog of the modern classical theory of causation, as
  formulated by Pearl and others using directed acyclic graphs, requires a
  theory of random or stochastic time development at the microscopic level,
  where the noncommutation of Hilbert-space projectors cannot be ignored. The
  Consistent Histories approach provides such a theory. How it works is shown
  by applying it to simple examples involving beam splitters and a Mach-Zehnder
  interferometer. It justifies the usual laboratory intuition that properly
  tested apparatus can reveal the earlier microscopic cause (e.g., radioactive
  decay) of a later macroscopic outcome. The general approach is further
  illustrated by how it resolves the Bell inequalities paradox. The
  difficulties encountered in an approach known as Quantum Causal Models can be
  traced to its lack of a satisfactory theory of quantum random processes.

\end{abstract}


\xb
\tableofcontents
\xa

\xb
\section{Introduction \label{sct1}}
\xa

% Sec. I

\xb%
\outl{`Cause' in laboratory \& everyday life. Cause $\lra$ correlation, but
cause precedes effect, and issue of a common cause }%
\xa%

Understanding causes is significant for interpreting laboratory experiments as
well as in everyday life. The mathematical description of causes uses
probability theory; causes are a type of statistical correlation. They are
special in that a cause precedes the corresponding effect in time, and in the
absence of its cause the effect would not have taken place. In addition it is
important to identify cases in which one event precedes another but is not its
cause, as both are consequences of an earlier common cause.

\xb%
\outl{Classical Theory of causes, Pearl \& DAGs, based on Cl stochastic
  processes}%
\xa%

\xb%
\outl{Present Letter based on consistent approach to Qm stochastic processes}%
\xa%

A very powerful technique for modeling causes and their effects,
and understanding common causes, has been developed based on the use of
directed acyclic graphs.%
%
\footnote{%
Standard references are \cite{Prl09,SpGs00}; for a compact introduction
see Sec. 3 of \cite{Htch18}.} %
%END footnote
It is referred to below as the \emph{Classical Theory} of causes, since it
is widely applied for discussing macroscopic events and processes.
Mathematically it  is based on \emph{classical stochastic processes}, the
theory of random time development. Thus one would expect that a
quantum counterpart of the Classical Theory applicable to microscopic causes
would be built upon a consistent approach to \emph{quantum stochastic
  processes}, and this Letter indicates how to do that.

\xb%
\outl{Problems with previous efforts: Textbooks: Unitary up to measurement}%
\xa%

\xb%
\outl{Physics lab: Microscopic stochasticity illustrated by radioactive decay
occurring randomly and before detection of decay products. Satisfactory theory
of Qm causes must deal with this}%
\xa%

One problem with some previous attempts in this direction is that they start
with textbook quantum theory in which there is unitary, and thus deterministic,
time development starting with some quantum state for a microscopic system, and
randomness only appears when that system is subjected to a macroscopic
measurement.
%
By contrast, in the physics laboratory random time development at the
microscopic level occurs all the time in the absence of any measurement. 
Radioactive decay, for example, occurs in the absence of a detector, and when a
decay product is detected, the detection event is thought to be caused by
the earlier decay. In certain cases the precise time of that earlier decay
can be estimated from the detection time. Any satisfactory theory of quantum
causes needs to be able to deal with situations of this type if it is to be of
interest to other than philosophers.

\xb%
\outl{Outline of rest of paper. Sec.~\ref{sct2}: Cl \& Qm stochastic histories.
  Readers not familiar w Cl Theory or CH may wish to first look at examples in
  Sec.~\ref{sct3} }%
\xa%

Section~\ref{sct2} begins with remarks about the classical theory of stochastic
processes that underlies the Classical Theory of causes, and then introduces
its quantum counterpart in the Consistent Histories (CH) approach to random
quantum time development. Readers not already conversant with the Classical
Theory or CH may find the presentation rather dense, and might wish to first
look at the simple examples in Sec.~\ref{sct3}, which provide a readily
accessible physical picture of the fundamental quantum difficulty in
understanding causes, and are discussed using elementary mathematics.

\xb%
\outl{Qm generalization of Classical Theory: Sec.~\ref{sct4}; applied to Bell
  paradox in Sec.~\ref{sct5}. Shows how Cl reasoning ignoring noncommutation
  $\ra$ results that disagree with experiment }%
\xa%

The scheme for extending these examples to the general quantum situation, at
least in principle, is indicated in Sec.~\ref{sct4}, and in Sec.~\ref{sct5}
applied to Bell inequalities, such as those of Clauser et al. \cite{CHSH69}, to
show how classical reasoning that ignores quantum noncommutation gives rise to
results that, not surprisingly, disagree with experiment.

\xb%
\outl{Wood \& Spekkens, QCM: Sec.~\ref{sct6}; Conclusion in Sec.~\ref{sct7}}%
\xa%

A pioneering attempt by Wood and Spekkens \cite{WdSp15} to embed the Classical
Theory into quantum mechanics ran into serious difficulties (``fine tuning'')
in the case of Bell inequalities, and this motivated the later development of a
mathematically very complicated approach called \emph{Quantum Causal Models}
(QCMs). Some remarks about QCMs and the source of its difficulties are in
Sec.~\ref{sct6}. A brief summary of this Letter together with suggestions for
future research is in the concluding Sec.~\ref{sct7}.

%END Sec. I Introduction


% Sec. II Stochastic Processes
\section{Stochastic Processes \label{sct2}}

\xb%
\outl{CLASSICAL STOCHASTIC PROCESSES}%
\xa%

\xb%
\outl{Cl Theory based on stochastic processes: History $\lra$ sequence of
  events at successive times. Sample space of histories to which probabilities
  (Kolmogorov) can be assigned. Finite sample space is adequate}%
\xa%

The Classical Theory employs the mathematical structure of \emph{random} or
\emph{stochastic processes}---think of random coin flips or a Markov process. A
single run of a stochastic process, e.g., heads on the first two tosses and
tails on the third, gives rise to a specific \emph{history,} a collection of
\emph{events} occurring at a sequence of successive \emph{times}. The
collection of possible histories constitutes a \emph{sample space} of
mutually-exclusive possibilities in the language of standard (Kolmogorov)
probability theory, assumed throughout the discussion that follows.
\emph{Probabilities} are nonnegative numbers summing to $1$ that are assigned
to the individual histories on the basis of a physical theory, mathematical
model or simply guesswork. Choosing a sample space and assigning probabilities
are distinct matters. Collections of elements from the sample space constitute
the \emph{event algebra},and the probability of a particular collection is the
sum of the probabilities of the elements it contains. For the following
discussion, finite discrete sample spaces suffice.


\xb%
\outl{Directed acyclic graphs (DAGs)}%
\xa%

To discuss causes the Classical Theory employs \emph{directed acyclic
  graphs} (DAGs), where nodes denote individual events at specified times, and
the lines with arrows connecting the nodes indicate possible \emph{conditional
probabilities} of later events conditioned on earlier ones. The arrows indicate
a time ordering from past to future. These conditional probabilities together
with those assigned to initial events (nodes with no incoming arrows) generate
the overall probability distribution for the histories that constitute the
corresponding stochastic process. The reader not already familiar with these
graphs and other concepts in the Classical Theory, such as interventions and
the role of common causes, may find it useful to consult \cite{Htch18}.

\xb%
\outl{QUANTUM STOCHASTIC PROCESSES}%
\xa%


\xb%
\outl{Extend Cl to Qm theory requires Hilbert space, Qm stochastic processes}%
\xa%

\xb%
\outl{Textbook QM inadequate. Random events $\lra$ measurements.
  Wavefunction collapse.  $\ra$  measurement problem of Qm founds}%
\xa%

\xb%
\outl{No one considers this satisfactory, including textbook writers}%
\xa%

Extending the Classical Theory of causes to the quantum domain, where a quantum
Hilbert space replaces a classical phase space, thus needs a consistent quantum
theory of stochastic processes with a suitable sample space of histories, and a
rule for assigning them probabilities. 
%
Constructing such a theory, however, runs into serious difficulties if one is
restricted to ideas found in current quantum textbooks, where the approach
to time development for a closed or isolates system involves a deterministic or
unitary time development---a solution to Schr\"odinger's equation---until the
system undergoes a \emph{measurement} by an external apparatus, at which
point a stochastic process occurs with a mysterious ``collapse of the
wavefunction''. The associated obscurities constitute the infamous
``measurement problem'' of quantum foundations.

\xb%
\outl{Physics lab: random processes at microscopic level preceding measurements
(radioactive decay). Micro causes of measurement outcomes}%
\xa%

In physics laboratories, in contrast to textbooks, random time development at
the microscopic level is a common occurrence. Unstable nuclei decay at
relatively precise, but totally random, instants of time, regardless of whether
a measurement apparatus happens to be nearby. If a decay product is detected
the experimenter thinks that the decay occurred at an earlier time, and in
appropriate circumstances can say how much earlier. Another example is
discussed in Sec.~\ref{sbct3a} below. These random microscopic events are then
thought of as \emph{causes} of later macroscopic measurement \emph{outcomes}.

\xb%
\outl{Finite-dimensional Hilbert space suffices.. Projectors $\lra$ Qm
  properties (vN); are Qm counterparts of indicators on a classical phase space.
  Energy of harmonic oscillator an example. Negation, conjunction in terms of
  indicators}%
\xa%

\xb%
\outl{Logic operations using indicators. NOT $P$, $P$ AND $Q$}
%
\xa

As pointed out by von Neumann, Sec.~III.5 of \cite{vNmn32b}, quantum properties
are represented by \emph{projectors} on a quantum \emph{Hilbert} space. (For
present purposes finite-dimensional Hilbert space will suffice.) A projector is
a Hermitian operator equal to its square, $P=P\ad =P^2$, and can be understood
as the quantum counterpart of an \emph{indicator}, a function which on a
classical phase takes the value $1$ where some property---e.g., energy $E$ less
than some $E_0$ for a harmonic oscillator---is true, and $0$ at points where it
is not true. Certain logical operations can be represented by indicators (think
of Venn diagrams). The negation of property $P$, $\NOT\ P$, is the indicator
$I-P$, where $I$ is the identity indicator taking the value $1$ everywhere.
Conjunction of two properties, $P\ \AND\ Q$, is the indicator $PQ$, and so
forth.

\xb%
\outl{Logic of Qm projectors. Negation $I-P$. Conjunction $PQ=QP$, but product
  in either order not a projector when $PQ\neq QP$}%
\xa%

\xb%
\outl{Noncommutation the Qm-Cl boundary. Qm Logic}%
\xa%

Similarly in the quantum case the negation of $P$ is the projector $I-P$, where
$I$ is the identity operator, and the conjunction ``$P$ and $Q$'' is the
projector $PQ$ \emph{provided} $P$ and $Q$ \emph{commute}: $PQ=QP$. If they do
\emph{not} commute the product in either order is not a projector, and we
arrive at the feature that marks the boundary between classical and quantum
physics: \emph{noncommutation,} the essence of quantum uncertainty relations
involving $\hbar$. Von Neumann was aware of this, and along with
Birkhoff\cite{BrvN36}, invented \emph{Quantum Logic} to make sense of
``$P\ \AND\ Q$'' in the noncommuting case. Quantum Logic is so complicated that
it has been of little help in understanding quantum mechanics and resolving its
conceptual difficulties.

\xb%
\outl{CH introduced. Commuting projectors a \emph{framework}. PDI defined;
  $\lra$ sample space. Event algebra. Single framework rule. Cl reasoning in
  noncommuting case $\ra$ paradoxes }%
\xa%

The much simpler Consistent Histories%
\footnote{ A compact overview of CH will be found in \cite{Grff19b}, while
  \cite{Grff02c} is a detailed treatment that remains largely up-to-date. For
  CH as a form of quantum logic, see \cite{Grff14} and Ch.~16 of
  \cite{Grff02c}. How CH principles resolve quantum paradoxes is discussed in
  Chs.~20 to 25 of \cite{Grff02c}, and in \cite{Grff17b} and
  \cite{Grff20}.} %END footnote
(CH) approach treats ``$P \AND\ Q$'' as \emph{meaningless} in the noncommuting
case, so one never has to discuss its meaning, and limits quantum reasoning to
collections of commuting projectors called \emph{frameworks}. One starts with a
\emph{projective decomposition of the identity} (PDI), the counterpart of a
\emph{sample space} in standard (Kolmogorov) probability theory, a collection
of mutually orthogonal projectors that sum to the identity operator. The
corresponding \emph{event algebra} consists of all the projectors which are
members of, or sums of members, of the PDI. The CH term ``framework'' is used
for either the PDI or the corresponding event algebra. Its \emph{single
  framework rule} states that any sort of logical reasoning must be carried out
using a collection of commuting projectors; reasoning which makes simultaneous
use of noncommuting projectors is invalid. Note that any collection of
commuting projectors generates, via complements and products, a PDI, and thus
an associated framework, so the single framework rule is in this sense not very
restrictive. What is ruled out is the application of classical reasoning to
situations in which two or more projectors do \emph{not} commute, the source of
endless numbers of unresolved quantum paradoxes.

\xb%
\outl{HISTORIES}%
\xa%

\xb
%
\outl{Family of histories. Simplest case: single PDI at each time for all of
  the histories; this PDI can depend upon the time. Orthogonality of histories.
  History PDI. Generalization using 'history Hilbert space'}
%
\xa

The use of PDIs at a single time is easily extended to a \emph{family of
  histories} used to discuss quantum stochastic processes in a manner analogous
to the classical case. For many purposes it suffices to consider families in
which the same sequence of times is used for every history in the family, and
in which at a particular time the projectors associated with the different
histories commute with each other, and thus correspond to a single PDI, though
this common PDI may vary from one time to another. Two such histories are
\emph{orthogonal} provided there is at least one time at which the
corresponding events are represented by orthogonal projectors. A collection of
orthogonal history projectors that sum to the \emph{history identity}, a
history in which the event at every time is the identity $I$, constitutes a
\emph{history PDI}, a sample space analogous to that of a classical stochastic
process. This construction suffices for the examples considered in this Letter,
although the generalization to a \emph{history Hilbert space}:%
\footnote{As first pointed out by Isham\cite{Ishm94}. } %
%END footnote
is useful in more complicated situations. For example, two histories in a
single consistent family may be orthogonal in the History Hilbert space even
though at a particular time the respective events are represented by
noncommuting projectors. For a general formulation see \cite{Grff19b} and
Ch.~10 of \cite{Grff02c}.

\xb%
\outl{Classical unicity vs Qm pluricity}%
\xa%

At this point it is worth noting an important difference between a classical
phase space and a quantum Hilbert space. A single \emph{unique} point is a
classical phase space represents the exact state of the system at a particular
time: all indicators that take the value $1$ at this point represent properties
the system possesses at this time, and those with value $0$ indicate properties
that are not true. This is in contrast to the quantum Hilbert space,
where for any nontrivial projector $P$ there will always be other projectors
that do not commute with $P$, and thus cannot be true or false in the classical
sense. Consequently there is never a projector that characterizes the
``actual'' quantum state of affairs in a way that agrees with one's classical
intuition. Thus intrinsic to Hilbert-space quantum theory is the possibility
that in some situations there might not be a unique way to describe a physical
system at a particular time: quantum \emph{pluricity} in contrast to classical
\emph{unicity}. Examples that illustrate this point are found in later sections
of this Letter.

\xb%
\outl{History framework = sample space of histories. Probabilities for closed
  Qm system. Consistency conditions (GMH) illustrated in Secs.~\ref{sbct3a},
  \ref{sbct3b}. References to fuller discussions. CH, unlike textbooks, can under
  appropriate conditions assign probabilities at intermediate times.}%
\xa%

\xb%
\outl{SFR extended to case of combining history families when probabilities
cannot be defined}%
\xa%

Given a sample space of histories, a \emph{history framework}, the next task is
to assign them probabilities. In the case of a \emph{closed} quantum system,
the unitary time development corresponding to Schr\"odinger's equation can be
used to assign probabilities provided certain \emph{consistency conditions}%
\footnote{While first introduced in earlier work, the correct formulation is
  the medium decoherence condition of Gell-Mann and Hartle\cite{GMHr93}} %
%END footnote
are satisfied. This represents a generalization, not restricted to measurement
outcomes, of the usual Born rule. The examples below in Secs.~\ref{sbct3a} and
\ref{sbct3b} will illustrate how this works in some simple cases, as will the
discussion of Bell inequalities in Sec.~\ref{sct5}. The reader is referred to
\cite{Grff19b} and Ch.~10 in \cite{Grff02c} for a comprehensive discussion of
the general case. A crucial difference between CH and textbooks is that the
former allows, under appropriate conditions, assigning probabilities to
microscopic situations in a closed system at \emph{intermediate times} before a
final measurement takes place. A further complication arises in situations in
which two consistent history frameworks can be combined in a single sample
space, a \emph{common refinement}, but for which the consistency conditions no
longer hold. In this case the single framework rule is extended to preclude
such combinations if one considers them part of a closed system and wishes to
assign probabilities using the extended Born rule.

\xb%
\outl{Extension to open systems}%
\xa%

Frameworks and probabilities can be employed for \emph{open} quantum systems by
the usual device of treating the system and a separate environment as a total
closed system. Macroscopic apparatus including measuring devices can be
included as part of the closed system and treated in a fully quantum mechanical
description with the help of \emph{quasiclassical} concepts and frameworks; see
\cite{Grff19b}.

% END Sec. II. Revised 1/19/24

%Sec. III

\section{ Simple Examples \label{sct3}}

\subsection{ Beam Splitters \label{sbct3a} }


\begin{figure}[h]
$$\includegraphics{fig11.eps}$$
\caption{(a) Photon source $S$ and beamsplitter $BS_1$ followed by two
  detectors at unequal distances. (b) Add mirrors and a second beamsplitter
  $BS_2$ to form a Mach-Zehnder interferometer. $BS_2$ can be removed, as
  sketched to the right of the dashed line.}
\label{fgr1}

\end{figure}

\xb
%
\outl{ Fig.~\ref{fgr1}(a): Beamsplitter, two detectors. Photon arrival causes
  detection}
%
\xa

A first gedanken experiment is shown in Fig.~\ref{fgr1}(a). A photon is sent
into a beamsplitter and emerges by unitary time development as a coherent
superposition of amplitudes in the two macroscopically separated paths $a$ and
$b$,
\begin{equation}
\ket{\psi_t} = \al \ket{a_t} + \bt \ket{b_t},
\label{eqn1}
\end{equation}
at time $t$. Here $\ket{a_t}$ is a wave packet traveling along the $a$ path;
$\ket{b_t}$ the $b$ path. The two paths lead to detectors $D_a$ and $D_b$
located at unequal distances from the beamsplitter. In a particular run of the
experiment the photon will be detected in either $D_a$ or $D_b$, but never in
both. Given a large number of runs, the fraction in which $D_a$ triggers will
be approximately $|\al|^2$; those in which $D_b$ triggers $|\bt|^2$. Someone
with experience in the laboratory will be inclined to think that on a
particular run in which $D_a$ is triggered the photon, after emerging from the
beamsplitter followed path $a$ and was absent from path $b$, while in a run in
which $D_b$ detects the photon it followed path $b$ and was absent from $a$.
But since the photon (or other quantum particle, such as a neutron, in an
analogous experiment) is invisible, it would seem useful to do additional tests
to corroborate these ideas. Here are some possibilities.

\xb
%
\outl{Stochastic splitting of paths at $BS_1$}
%
\xa

If during certain runs the $a$ path is blocked by some macroscopic absorbing
object---think of this as an intervention---$D_a$ never triggers, while the
fraction of runs in which $D_b$ triggers remains unchanged. Or replace the
block by a mirror which deflects the photon off the $a$ path towards a third
detector $D'_a$. Its average rate is that of $D_a$, whereas the rate of
$D_b$ remains unchanged. These observations support the idea that some sort of
random process occurs at the first beamsplitter $BS_1$, where (to use
anthropomorphic language) the photon randomly chooses either the $a$ or the $b$
path, and then follows it till it reaches the corresponding detector. In
addition, if the photon is emitted at a well-defined time, and the time of
detection measured with sufficient precision, the time to reach detector $D_a$
is less than that required to reach $D_b$, confirming the foregoing intuition

\xb
%
\outl{Detection causes wavefn collapse?}
%
\xa

This seems very odd from a textbook perspective in which random processes only
occur when something is measured, resulting in the collapse of a wavefunction.
Let us explore that idea. Since detector $D_a$ is closer to $BS_1$ than $D_b$,
one might suppose that when the $\al\ket{a_t}$ part of $\ket{\psi_t}$ in
\eqref{eqn1} reaches detector $D_a$ and the photon is detected there, this
instantly removes the amplitude traveling towards $D_b$, which will not be
triggered on this particular run. Such a nonlocal effect seems strange, and is
hard to reconcile with special relativity. An additional difficulty arises in
runs in which $D_a$ does \emph{not} detect the particle. Does such nondetection
result in the amplitude collapsing onto path $b$ at the instant of
nondetection? Or does the later detection by $D_b$ retroactively ``cause'' an
earlier collapse on path $a$ that prevents detection by $D_a$?

\xb
%
\outl{`Decoherence $\ra$ single path' disproved by interference,
  Fig.~\ref{fgr1}(b)}
%
\xa

Might it be that some sort of \emph{decoherence} process, caused by an
interaction of the photon with the environment, either at the beam splitter or
due to gas particles on the way to the detector, is at work, so that the
coherent state \eqref{eqn1} is no longer the correct quantum description
when the two paths have separated by a macroscopic distance? Decoherence
removes quantum interference, so it can be checked with the alternative
experimental arrangement shown in Fig.~\ref{fgr1}(b). Here a second
beamsplitter, $BS_2$, has been added, so that along with the first beamsplitter
$BS_1$ the result is a Mach-Zehnder interferometer. Let us suppose that the
beamsplitters have been chosen in such a way that the photon always emerges on
path $c$, leading to a detection by $D_c$, never $D_d$. Decoherence along the
$a$ or $b$ path inside the interferometer would eliminate the interference, so
can be excluded if the interferometer is functioning properly. 

\xb
%
\outl{ Compare various situations with $BS_2$ in or out of the way (Wheeler's
  Delayed Choice). Blocking arms. Phase shifters in the two MZI arms }
%
\xa

Motivated by Wheeler's delayed choice paradox\cite{Whlr78}, let us suppose that
$BS_2$ can be moved out of the way, the situation sketched at the far right of
Fig.~\ref{fgr1}(b), in which case we are back at something that resembles
Fig.~\ref{fgr1}(a), but now the two paths cross. That nothing special happens
when they cross can be checked, as before, by blocking the paths at various
points before they cross, and by moving $D_c$ further away and checking the
timing. Thus one concludes (as did Wheeler) that with $BS_2$ absent the photon
detected by $D_c$ was earlier following path $a$ before reaching the crossing
point, and likewise it was following path $b$ in a run in which it was detected
by $D_d$. Having considered the case with $BS_2$ absent, we must now try and
understand what happens when it is present. One approach is to insert variable
phase shifters in each of the two interfereometer arms. While these can change
the ratio of the $D_d$ to $D_c$ counts, it remains fixed if \emph{both} phases
are changed by the \emph{same} amount. Thus the photon must in some sense have
been simultaneously present in \emph{both} arms of the interferometer. This
seems very odd given that with $BS_2$ absent we had good reason to believe the
photon was in either the $a$ arm or the $b$ arm after emerging from $BS_1$. Can
removing $BS_2$, as against leaving it in place, influence what happens
\emph{before} the photon arrives at the crossing point?

\xb%
\outl{Adequate theory of Qm causes should provide explanations. Essentials of
  CH approach in following Sec.~\ref{sbct3b} using spin-half particle}%
\xa%

Providing coherent answers to questions of this sort is what one should expect
from an adequate theory of quantum causes. The essentials of the CH approach
and the key respects in which it differs from textbook thinking can best be
explained by an analogous gedanken experiment involving a spin-half particle,
taken up next.

%END Sec. III A

%BEGIN Sec. III B

\subsection{ Spin Half \label{sbct3b}}

\xb
%
\outl{Spin-half analog of Fig.~\ref{fgr1}. Spin-half Ag atom on straight path
to SG measuring $S_z$. Paths $a$, $c$ in figure $\lra$ $S_z=+1/2$; $b$, $d$
to $S_z=-1/2$. SG detectors }
%
\xa



A spin-half analog of the situation in Fig.~\ref{fgr1} allows a mathematical
discussion using a two-dimensional Hilbert space, all that is needed for such a
particle, and is thus simpler than describing the different spatial positions
of the photon in Fig.~\ref{fgr1}. Think of a spin-half silver atom moving along
a straight line until it enters a Stern-Gerlach (SG) measuring device with its
magnetic field and field gradient in the $z$ direction, so that it measures
$S_z$, yielding one of the two values $+1/2$ and $-1/2$ in units to $\hbar$.
Let $S_z=+1/2$ correspond to the $a$ path in Fig.~\ref{fgr1}(a), and its
extension into the $c$ path in Fig.~\ref{fgr1}(b), and $S_z=-1/2$ be the
counterpart of the $b$ and $d$ paths. A single SG device replaces the pair of
detectors $D_a$ and $D_b$ in Fig.~\ref{fgr1}(a) or $D_c$ and $D_d$ in part (b).

\xb
%
\outl{Regions of uniform field replace Fig.~\ref{fgr1} beamsplitters}
%
\xa
\xb
%
\outl{Uniform H field can be used to produce particles polarized in any desired
  direction, and to measure $S_w=\pm 1/2$ for any direction $w$ }%
%
\xa

The beamsplitters in Fig.~\ref{fgr1} are replaced by limited regions of uniform
magnetic field chosen so that as the atom moves through the region its spin
precesses by the desired amount: a unitary transformation on its $d=2$
dimensional Hilbert space. Given a source of spin-polarized atoms, e.g., the
upper beam emerging from a SG apparatus, an appropriate uniform field allows
the preparation of particles with a spin polarization $S_v=+1/2$ for any
spatial direction $v$. Similarly, if the final SG measurement apparatus is
preceded by a suitable region of uniform field, the combination allows the
measurement of any spin component $S_w$, provided passage through the region of
uniform field maps $S_w=+1/2$ to $S_z=+1/2$, and $S_w=-1/2$ to $S_z=-1/2$. That
this analog of $BS_2$ in Fig.~\ref{fgr1}(b) is functioning properly can in
principle be checked by the less convenient process of rotating the final SG
magnet to alter the direction of its magnetic field.

\xb
%
\outl{Alice prepares $S_x=+1/2$; Bob measures $S_z$. Spin at
  intermediate time =?}
%
\xa

Suppose that in this way Alice uses the analog of $BS_1$ to prepare a state
with $S_x=+1/2$ which travels on to Bob, whose SG measures $S_z$, the analog of
Fig.~\ref{fgr1}(b) with $BS_2$ absent. The results are random: in roughly half
the runs the outcome corresponds to $S_z=+1/2$, and the other half to
$S_z=-1/2$. What can one say about the spin state during the the time interval
when the particle is moving through the central field-free region on its way
from Alice's preparation to Bob's measurement? Alice might say $S_x=+1/2$,
since she has carried out a lot of tests to check that her apparatus is working
properly, perhaps sending the beam into an SG with field gradient in the $x$
direction. Bob has been equally careful in checking his apparatus, and is sure
that $S_z$ was $-1/2$ in those runs where it indicated this value, and $+1/2$
in the others. But there is no room in the spin-half Hilbert space, no
projector, that can represent the combination ``$S_x=+1/2\ \AND\ S_z=-1/2$'' or
``$S_x=+1/2\ \AND\ S_z=+1/2$''. We seem to have a dispute between two parties.
What shall we to do?

\xb
%
\outl{Bob on vacation. Alice can analyze preparation/measurement results using
  alternatives $\PC_x$ or $\PC_z$ PDIs an intermediate time.}
%
\xa

Let us send Bob off on vacation. Alice is perfectly capable of designing and
building both the preparation and the measurement apparatus, and, as with any
competent experimenter, is conversant with the relevant theory. Thus for any
given run she knows both the preparation and the measurement outcome, and faces
the problem of what these data tell her about the spin state at an intermediate
time. She begins with the spin-half history that includes the initial state
$[x+]=\dya{{x+}}$, the projector on the state $S_x=+1/2$, at the
initial time; $[z-]$ at the final time; and the identity operator $I$ at the
intermediate time. This history can be refined in either of two different ways
that will satisfy the consistency conditions, by replacing $I$ at the
intermediate time with one of two PDIs:
\begin{equation}
 \PC_x: \{[x+],[x-]\}; \quad \PC_z: \{[z+], [z-]\},
\label{eqn2}
\end{equation}

\xb%
\outl{Using $\PC_x$ $\ra$ 3-time history family written out explicitly. Implies
$S_x = +1/2$ at intermediate time. Similarly $\PC_z \ra S_z=-1/2$.
SFR means the two cannot be combined }%
\xa%

Using the first results in a family of two histories in what is by now
standard notation, 
\begin{equation}
[x+] \od \{[x+],[x-]\} \od [z-]
\label{eqn3}
\end{equation}
consistent with the initial preparation of $[x^+]$, the final measurement of
yielding $[z-]$ and two possibilities for $S_x$ at the intermediate time.
Using the extended Born rule means that the probability of $S_x=+1/2$ at the
intermediate time, conditional on the initial and final states, is $1$, and
zero for $S_x=-1/2$. Similarly, using the pair of histories in which
the alternatives $\{[z+],[z-]\}$ are employed leads to the conclusion that
$S_z = +1/2$ at the intermediate time. The single framework rule means that
these two families cannot be combined to yield the nonsensical result that
simultaneously $S_x=+1/2$ \emph{and} $S_z = -1/2$, for which there is no
Hilbert-space projector. 

\xb%
\outl{Last minute change in what is measured. Retrocausality mistake. Classical
  analog: color/shape of a piece of paper}%
\xa%

Alice could also decide at the very last instant to measure a different
property, say $S_y$ in place of $S_z$, by altering the detection apparatus.
This ability to change the measurement is sometimes misinterpreted as having
a spurious \emph{retrocausal} effect: If Alice switches from $S_z$ to $S_y$ it
somehow alters the earlier state of the measured particle. A better perspective
is that Alice's choice does not influence the particle, but simply alters what
she can learn about it by carrying out a measurement. A classical analog would
be deciding between measuring the shape versus the color of a slip of paper.
The difference with the quantum case is that shape and color can be
simultaneously ascribed to the slip of paper, whereas $S_x$ and $S_z$ cannot
simultaneously take on values in the case of a spin-half particle.

\xb
%
\outl{Omitted from above: Macroscopic measurement apparatus. See CQT, WQMM}
%
\xa

\xb%
\outl{Alice must \emph{choose} }%
\xa%

While the above discussion identifies the central issue, a number of details
needed for a more complete discussion have been omitted. These would include a
proper quantum description, at least in principle, of the measurement apparatus
and its macroscopic outcomes. For these we refer the reader to Chs.~17 and 18
of \cite{Grff02c}, and to \cite{Grff17b} and references given there. The
fundamental point remains the same: Alice, in order to identify the
\emph{cause} of the $S_z$ measurement outcome, must \emph{choose} to use
$\PC_z$ rather than $\PC_x$. There is nothing irrational about this choice, or
similar choices made every day in the laboratory by researchers who regularly
interpret their results in terms of the microscopic causes their apparatus was
designed to detect in situations analogous to those in Fig.~\ref{fgr1}(a).

% END III B
% END III

% BEGIN IV
\section{Consistent Quantum Causes \label{sct4}}

\xb%
\outl{Qm cause needs stochastic time development, Sec.~\ref{sct2}, not found in textbooks.}%
\xa%

A theory of quantum causes that employs the general approach of the Classical
Theory must, for reasons indicated in Sec.~\ref{sct2}, be based on a
fundamental mathematical structure of stochastic time development. But such
is absent from current quantum textbooks, which do not have an
adequate treatment of noncommuting projectors, and this has led to various
measurement problems in quantum foundations.

\xb%
\outl{CH approach, Sec.~\ref{sct2}, illustrated by simple examples,
Secs.~\ref{sbct3a} + \ref{sbct3b} $\ra$ theory of Qm stochastic time development}%
\xa%

\xb%
\outl{Measuring instrument can be described QM-ly. Use GMH consistency for
  probabilities. See WQMM; includes POVMs}%
\xa%


By contrast, the CH approach as discussed in Sec.~\ref{sct2} and illustrated by
the simple examples in Secs.~\ref{sbct3a} and \ref{sbct3b}, which agree with
laboratory intuition, provides a structure for stochastic quantum time
development that can be used to identify the causes of the outcomes of
projective measurements. The key point is to choose a family of histories that
includes the microscopic properties an instrument was designed to measure, at a
time just before the measurement, or an earlier time as appropriate (as
illustrated in the Bell inequalities example below). The instrument itself can
be given, at least in principle, a fully quantum-mechanical description, and
probabilities assigned to the total closed system using the extended Born rule
employing consistency conditions. One is now dealing with a genuine theory of
quantum stochastic processes, which is what is needed for a quantum counterpart
of the Classical Theory. For more details on how this works see the
discussion of projective measurements and its extension to POVMs in
\cite{Grff17b}. (For POVMs the earlier microscopic properties are randomly,
rather than deterministically, related to later macroscopic measurement
outcomes.)

\xb%
\outl{Generalizations (and see Sec.~\ref{sct2})}%
\xa%

\xb%
\outl{PDI at a given time depends on the history. Classical analogs}%
\xa%

\xb%
\outl{Events ``entangled'' between 2 or more times}%
\xa%

\xb%
\outl{Approximate consistency; altering projectors as per Dowker \& Kent.
  Acceptable to sloppy theoretical physicist}%
\xa%


As noted in Sec.~\ref{sct2}, history families can include cases in which the
PDI of events at a particular time can depend on earlier or later events in the
history in question, and these, which have classical analogs, are easily
incorporated in the general families which can be useful in discussing quantum
causes. Quantum histories in which events are ``entangled'', represented by
quantum superpositions, between two or more times are also possible, but have
not been studied. A feature which has no classical analog is the case in which
consistency conditions are not exactly satisfied. As first pointed out by
Dowker and Kent \cite{DwKn96}, if the violations of consistency conditions are
small there will be a nearby family of histories with slightly altered
projectors which satisfies the consistency conditions, and which might be just
as good from the somewhat sloppy point of view of the theoretical physicist, if
not the quantum philosopher. E.g., the spin of the particle is $-1/2$ for $S_w$,
with the direction $w$ close to, but not exactly along, the $z$ axis.

\xb%
\outl{Open Qm systems using a separate Environment. Need to study simple
  examples to check out ideas currently based on some arm waving}%
\xa%

\xb%
\outl{More on CH in the Conclusion}%
\xa%

The case of open quantum systems can be included, at least in principle, in the
CH approach by combining the system of interest with a second quantum,
system---an ``environment''---and then applying the general principles to the
combination regarded as a single closed quantum system. It would be valuable to
study simple examples using the CH approach to check out various ideas about
open systems which, while not necessarily wrong, are at present based upon
guesses and arm waving. For some further comments on the CH approach, see
Sec.~\ref{sct7}.

The following discussion of Bell inequalities shows how the CH approach can be
applied to more complicated situations than those considered in
Sec.~\ref{sct3}.

% END IV

% BEGIN V 

\section{ Bell Inequalities\label{sct5}}


\xb%
\outl{CHSH version of BIs violated by lab experiments, in agreement with Qm
  predictions. CH analysis of Bell paradox $\lra$ more complicated situation
  than earlier examples in Sec.~\ref{sct3}. Will use spin-half; easily mapped
  to case of photons}%
\xa%


The reader is no doubt aware that certain Bell inequalities, specifically those
of Clauser-Horne-Shimony-Holt (CHSH) \cite{CHSH69}, are violated by experiments
in precisely the way predicted by quantum theory. The use of consistent quantum
causes for understanding what is wrong with derivations of what might be called
the \emph{Bell paradox} will serve to illustrate their use in a situation more
complicated than those addressed earlier in Secs.~\ref{sbct3a} and
\ref{sbct3b}. The discussion here uses spin-half particle language, which is
easily mapped onto the case of entangled photons as measured in the laboratory.

\xb%
\outl{Each round of ``Bell experiment'': Alice and Bob measure one of two
  orthogonal spin components. So 4 different, mutually incompatible, types of
  measurement. Data from different types cannot be combined; SFR of CH. But
  combining is behind CHSH inequalities}%
\xa%

Each round of a Bell experiment consists of measurements by two parties, Alice
and Bob, who are distant from each other, with each making a choice to carry
out one of two possible incompatible measurements, e.g., $S_x$ and $S_z$, that
correspond to mutually perpendicular choices on the Bloch sphere. These
measurements are carried out on an entangled state of the two spin-half
particles, prepared at an earlier time. Hence there are four different, mutually
incompatible, types of measurement carried out jointly by the two parties in
different runs of the experiment, and the statistical data must be accumulated
and analyzed separately for each type of run. The single framework rule of CH
states that these four data sets cannot be combined in the way routinely
used when deriving CHSH inequalities and similar results, as it employs a
process of classical reasoning that ignores the noncommutation of quantum
projectors.

\xb%
\outl{Mistake for Alice to assume  $S_x$ has an unmeasured value in run where
  $S_z$ was measured. But she is correct in employing \emph{local causality},
contrary to Bell }%
\xa%

Consider in particular a run in which Alice measures $S_z$. It would be a
mistake for her to assume that in \emph{this} run the spin-half particle
\emph{also} possessed a value of $S_x$ that was somehow ``there'', even though
it was not measured. See the remarks in Sec.~\ref{sbct3b}. On the other hand,
as she has carried out a projective measurement having properly calibrated her
apparatus, she is correct in concluding that the $S_z$ value revealed by the
(macroscopic) outcome was the value possessed by the spin-half particle just
before the measurement took place: an instance of \emph{local causality} of the
sort that Bell thought inconsistent with quantum theory.

\xb%
\outl{Tracing properties just before measurements back in time $\ra$
  \emph{common cause} of Alice-Bob correlations. Wavefn collapse not needed}%
\xa%

In addition, given that the particle did not pass through a magnetic field on
the way to her apparatus, she can conclude that the spin had the same value at
an earlier time just after the entangled initial of the two particles was
prepared, and the partner particle started on its way to Bob. And for the same
reasons Bob can conclude that his measurement reveals the spin state of his
particle at this earlier time. Thus, by use of an appropriate family of
histories, the observed correlations can be traced backwards in time to a
\emph{common cause}. Alice and Bob can carry out their measurements
at different times, and there is no need to invoke a mysterious and unphysical
``wavefunction collapse'' in order to understand the correlations.

\xb%
\outl{Alice, Bob both free to choose what to measure just before arrival of
  particles, or agree upon a list in advance. Much literature on exptl tests of
  Bell inequalities is based on Cl reasoning, so irrelevant. See details in
  \cites{NONLOC} }

Note that Alice and Bob are free to choose, at any time up to just before their
respective measurements tale place, which spin component to measure, e.g., by
using a random number generator. Or they may employ a shared list agreed upon
in advance as to what choice to make on each run. A large part of the vast
literature on experimental tests of Bell inequalities becomes totally
irrelevant when classical reasoning is abandoned in favor of Hilbert-space
quantum mechanics, and using quantum histories to represent quantum stochastic
processes. For further comments and details see \cite{Grff20}.

\xb%
\outl{CH gets rid of spurious nonlocal influences that violate no-signaling}%
\xa%

Among other things the CH analysis gets rid of nonlocality claims that go back
to Bell: The notion that in the quantum world there are mysterious nonlocal
influences giving rise to statistical correlations seemingly in violations of
special relativity. The simplest explanation of why such influences are never
observed in the laboratory, where experiments have confirmed the
``no-signaling'' principle, is that they do not exist, and are hence incapable
of carrying information from one place to another.

% END Sec. V

% Begin Sec. VI
\section{ Quantum Causal Models \label{sct6}}


\xb%
\outl{Wood-Spekkens difficulties (fine tuning) analyzing Bell
  inequalities. They lacked Qm stochastic processes in contrast to CH, and
  used textbook unitary time development until measurement}%
\xa%

The pioneering attempt of Wood and Spekkens \cite{WdSp15} to produce a quantum
counterpart of the Classical Theory, one useful for discussing the Bell
inequalities problem, uncovered difficulties in the form of an unnatural ``fine
tuning'' condition; see the original paper for details. From the perspective of
this Letter the real problem was the absence of a suitable approach to quantum
stochastic processes, and in its absence starting with the textbook idea of
unitary, and thus deterministic, time development until a later measurement
occurs. This prevented a discussion of causes at an intermediate time,
including those central to the CH analysis of the Bell paradox in
Sec.~\ref{sct5}.

\xb%
\outl{Wood-Spekkens tried to analyze complicated 2-party measurement without
first solving simpler case of single measurement}%
\xa%

\xb%
\outl{Spekkens 2007 toy measurement model, seemingly unaware of earlier
  treatment in CQT}%
\xa%


Another aspect of \cite{WdSp15}, which at least in retrospect seems
problematical, was the attempt to analyze a complicated problem, a joint
measurement by two independent parties, Alice and Bob, without first addressing
what causes the outcome in the simpler case of a \emph{single} measurement. It
seems that Spekkens was aware of this simpler problem, which he addressed using
a toy model \cite{Spkk07}, while seemingly unaware that this had been addressed
in earlier work on measurements that employed the quantum Hilbert space, as
summarized in Chs.~17 and 18 of \cite{Grff02c}.

\xb%
\outl{Development of QCM. Complicated math, a 'Sledgehammer'. Spekkens \&
  associates remain unsatisfied re Bell situation}%
\xa%

Attempts to deal with the issue identified by Wood and Spekkens led to later
work by Allen et al.\ \cite{Alao17} and others resulting in what its advocates
call \emph{Quantum Causal Models} (QCMs). There are by now many publications
devoted to this topic; items \cite{CsSh16,Shrp19,DlRS22} are a small selection
from a substantial literature, and provide some references to other work.
Neither the mathematics nor the associated concepts are easy to understand.
Perhaps \cite{Shrp19} is the most accessible for someone unfamiliar with this
approach, but even its author refers to it as a ``sledgehammer.'' Discussing
the associated concepts lies outside the scope of the present Letter, though it
is perhaps worth noting that some later preprints and publications by Spekkens
and his associates, e.g., \cite{DlRS22} would suggest that they are still not
altogether satisfied by what has been achieved up till now in the case of Bell
inequalities.

\xb%
\outl{Key CH QCM difference: CH has a consistent theory of Qm stochastic
  processes, used to resolve many paradoxes, while QCM does not }%
\xa%

\xb%
\outl{QCM may have useful insights. Its practitioners would benefit community
  by applying CH methods to some of the more complicated situations they have
  studied}%
\xa%

It is worth emphasizing the key difference between the CH and QCM approaches. The
former is based upon a consistent theory of quantum stochastic time
development, one that has been used to resolve numerous quantum paradoxes, as
in Chs.~20 to 25 in \cite{Grff02c}, whereas QCM lacks anything comparable. This
does not mean that the QCM approach is without value, but it suggests that its
practitioners could benefit the community by applying simpler CH methods to
some of the more complicated situations they have studied, to see what useful
insights might emerge.

% END Sec. VI

%Sec. VII
\section{ Conclusion \label{sct7}}

\xb%
\outl{Using histories allows HSQM form of Qm stochastic processes. Evades
  measurement problem of Qm founds}%
\xa%

\xb%
\outl{Allows Classical Cause approach to be mapped in natural way to QM,
  consistent with causes in lab physics.}%
\xa%

\xb%
\outl{Math structure much less complicated than QCM}%
\xa%

\xb%
\outl{Good physics needs conceptual framework along with math}%
\xa%


The CH use of quantum histories based on the mathematical structure of Hilbert
space quantum mechanics provides a framework for discussing random time
development, quantum stochastic processes, without the need to invoke
measurements. It thus evades the infamous measurement problem of quantum
foundations, and in particular allows the modern Classical Theory of causes to
be mapped in a natural way onto the quantum domain, in a manner consistent with
the way microscopic causes are commonly understood in laboratory physics. The
mathematical structure is far less complicated than that of the Quantum Causal
Models approach, as illustrated in the way it resolves the Bell inequalities
paradox. Good physics requires both an appropriate conceptual framework
along with mathematical models.

\xb%
\outl{Central to CH: vN projectors for physical properties \& paying attention
  to noncommutation. There may be other approaches to noncommutation, but
  ignoring it $\ra$ paradoxes. CH has resolved all paradoxes to which it has
  been applied}%
\xa%

\xb%
\outl{CH lacks quantitative notion of approximate consistency}%
\xa%

\xb%
\outl{Further study may $\ra$ serious problems with CH when applied to
  situations more complex than Bell inequalities. But CH worth further
  attention}%
\xa%


Central to the CH approach is von Neumann's use of projectors to represent
quantum physical properties, and then taking seriously the problems that arise
when projectors do not commute. There may be other ways of addressing
noncommutation, which divides quantum from classical physics, but simply
ignoring it is the source of numerous quantum paradoxes. The CH approach has
resolved all to which it has up till now been applied, though this does not
show that it will ultimately lead to a satisfactory understanding of the
quantum mysteries. By its own lights it is incomplete in the sense of lacking a
quantitative notion of approximate consistency, and further studies may well
uncover more serious problems. These could emerge when it is applied to more
complex situations than represented by Bell inequalities. In any case the
general methods outlined in this paper seem worth further attention.

% Acknowledgements
\xb
\section*{Acknowledgements}
\xa

Critical comments in anonymous reviews of an earlier version have led to
substantial improvements in the text. In addition the author is grateful to
Carnegie-Mellon University and its Physics Department for continuing support of
his activities as an emeritus faculty member.

\begin{thebibliography}{10}

\bibitem{Prl09}
Judea Pearl.
\newblock {\em Causality}.
\newblock Cambridge University Press, New York, 2009.
\newblock 2d edition.

\bibitem{SpGs00}
Peter Spirtes, Clark Glymour, and Richard Scheines.
\newblock {\em Causation, Prediction, and Search}.
\newblock MIT Press, 2000.
\newblock 2d edition.

\bibitem{Htch18}
Christopher Hitchcock.
\newblock Probabilistic causation.
\newblock {\em Stanford Encyclopedia of Philosophy}, 2018.

\bibitem{CHSH69}
John~F. Clauser, Michael~A. Horne, Abner Shimony, and Richard~A. Holt.
\newblock Proposed experiment to test local hidden-variable theories.
\newblock {\em Phys. Rev. Lett.}, 23:880--884, 1969.

\bibitem{WdSp15}
Christopher~J. Wood and Robert~W. Spekkens.
\newblock The lesson of causal discovery algorithms for quantum correlations:
  Causal explanations of bell-inequality violations require fine-tuning.
\newblock {\em New J. Phys.}, 17:033002, 2015.
\newblock arXiv:1208.4119 v2.

\bibitem{vNmn32b}
Johann von Neumann.
\newblock {\em Mathematische Grundlagen der Quantenmechanik}.
\newblock Springer-Verlag, Berlin, 1932.
\newblock English translation by R. T. Beyer: \textit{Mathematical Foundations
  of Quantum Mechanics}, Princeton University Press, Princeton, New Jersey
  (1955 and 2018).

\bibitem{BrvN36}
G.~Birkhoff and J.~von Neumann.
\newblock The logic of quantum mechanics.
\newblock {\em Ann. Math.}, 37:823--843, 1936.

\bibitem{Grff19b}
Robert~B. Griffiths.
\newblock The {C}onsistent {H}istories {A}pproach to {Q}uantum {M}echanics.
\newblock {\em Stanford Encyclopedia of Philosophy}, 2019.
\newblock https://plato.stanford.edu/entries/qm-consistent-histories/.

\bibitem{Grff02c}
Robert~B. Griffiths.
\newblock {\em Consistent Quantum Theory}.
\newblock Cambridge University Press, Cambridge, U.K., 2002.
\newblock http://quantum.phys.cmu.edu/CQT/.

\bibitem{Grff14}
Robert~B. Griffiths.
\newblock The new quantum logic.
\newblock {\em Found. Phys.}, 44:610--640, 2014.
\newblock arXiv:1311.2619.

\bibitem{Grff17b}
Robert~B. Griffiths.
\newblock What quantum measurements measure.
\newblock {\em Phys. Rev. A}, 96:032110, 2017.
\newblock arXiv:1704.08725.

\bibitem{Grff20}
Robert~B. Griffiths.
\newblock Nonlocality claims are inconsistent with {H}ilbert-space quantum
  mechanics.
\newblock {\em Phys. Rev. A}, 101:022117, 2020.
\newblock arXiv:1901.07050.

\bibitem{Ishm94}
C.~J. Isham.
\newblock Quantum logic and the histories approach to quantum theory.
\newblock {\em J. Math. Phys.}, 35:2157--2185, 1994.

\bibitem{GMHr93}
Murray Gell-Mann and James~B. Hartle.
\newblock Classical equations for quantum systems.
\newblock {\em Phys. Rev. D}, 47:3345--3382, 1993.

\bibitem{Whlr78}
John~Archibald Wheeler.
\newblock The ``{P}ast'' and the ``{D}elayed-{C}hoice'' {D}ouble-{S}lit
  {E}xperiment.
\newblock In A.~R. Marlow, editor, {\em Mathematical Foundations of Quantum
  Theory}, pages 9--48. Academic Press, New York, 1978.

\bibitem{DwKn96}
Fay Dowker and Adrian Kent.
\newblock On the consistent histories approach to quantum mechanics.
\newblock {\em J. Stat. Phys.}, 82:1575--1646, 1996.

\bibitem{Spkk07}
Robert~W. Spekkens.
\newblock Evidence for the epistemic view of quantum states: {A} toy theory.
\newblock {\em Phys. Rev. A}, 75:032110, 2007.
\newblock arXiv:quant-ph/0401052.

\bibitem{Alao17}
John-Mark~A. Allen, Jonathan Barrett, Dominic~C. Horsman, Ciar\'an~M. Lee, and
  Robert~W. Spekkens.
\newblock Quantum common causes and quantum causal models.
\newblock {\em Phys. Rev. X}, 7:031021, 2017.
\newblock arXiv:1609.09487 v2.

\bibitem{CsSh16}
Fabio Costa and Sally Shrapnel.
\newblock Quantum causal modelling.
\newblock {\em New J. Phys.}, 18:063032, 2016.

\bibitem{Shrp19}
Sally Shrapnel.
\newblock Discovering quantum causal models.
\newblock {\em British J. Phil. Sci.}, 70:1--25, 2019.
\newblock http://philsci-archive.pitt.edu/13098/.

\bibitem{DlRS22}
Patrick~J. Daley, Kevin~J. Resch, and Robert~W. Spekkens.
\newblock Experimentally adjudicating between different causal accounts of
  {B}ell-inequality violations via statistical model selection.
\newblock {\em Phys. Rev. A}, 105:042220, 2022.
\newblock arXiv:2108.00053.

\end{thebibliography}

\end{document}
% References. One does NOT need a separate section label. 
% See b/latex/bibtex >info >OVERVIEW for how to run bibtex
\bibliographystyle{unsrt}
\bibliography{/home/rgrif/qms/bibs/main}
% To make self contained: Put \end{document} ahead of line %Reference ...
% and insert the cqc1101.bbl file just above \end{document}


