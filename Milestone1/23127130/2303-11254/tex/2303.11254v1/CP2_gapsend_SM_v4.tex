% for overleaf -- long
\documentclass[prb,aps,amsfonts,amssymb,floatfix,showpacs,superscriptaddress]{revtex4}
\usepackage{graphicx}
\usepackage{dcolumn}% Align table columns on decimal point
\usepackage{bm}% bold math
\usepackage{color} 
\usepackage{epsfig}
\usepackage{epstopdf}
\usepackage{amsmath}
\newcommand{\beginsupplement}{%
  \setcounter{equation}{0}
  \renewcommand{\theequation}{S\arabic{equation}}%
  \setcounter{table}{0}
  \renewcommand{\thetable}{S\arabic{table}}%
  \setcounter{figure}{0}
  \renewcommand{\thefigure}{S\arabic{figure}}%
  \setcounter{section}{0}
  \renewcommand{\thesection}{SM-\Alph{section}}%\setcounter{subsection}{0}
  \renewcommand{\thesubsection}{SM-\Alph{section}.\arabic{subsection}}%
%To number supplemental material with 'S': %\renewcommand{\thepage}{S\arabic{page}}  
%\renewcommand{\thesection}{S\arabic{section}}   %\renewcommand{\thetable}{S\arabic{table}}   %\renewcommand{\thefigure}{S\arabic{figure}} 
}
\begin{document} 
\title{Mott-Slater Transition in a Textured Cuprate Antiferromagnet\\
%Theory of Cuprate Pseudogap as Antiferromagnetic Order with Domain Walls\\
%II. Mott-Slater Transition\\
Supplementary Material}
%https://www.overleaf.com/project/625a18b00dc2d721b988904c 
\author{R.S. Markiewicz and A. Bansil}
\affiliation{ Physics Department, Northeastern University, Boston MA 02115, USA}

\maketitle
\beginsupplement
\section{Domain wall phase diagram}
\subsection{Data used to construct Fig. 2}
The circular domain walls were calculated from the equation $\theta=[1\pm tanh((r-r_0)/2\xi)]\pi/2$, with $r_0=25a$ and $\xi/a=9$.  The ring is centered in a square $L=2000a$ on a side,


\subsection{Data used to construct Fig. 3}

Figure~3 includes superonducting (green lines) and STM data (blue symbols), where we use the following procedure to convert the STM gaps $\Delta$ to equivalent onset temperatures.  The superconducting gaps have been well studied, and satisfy the relation 
 \begin{equation}
2\Delta=Xk_BT_c.
\label{eq:BCS} 
\end{equation}  
For Bi$_2$Sr$_2$Ca$_0$Cu$_1$O$_6$ (Bi2201), $X$ = 12.4\cite{EHud3}, while for Bi2212 $X=10$,\cite{Vishik} with weak doping dependence.  We assume all STM gap data can be converted to temperatures by the same factor $X$.  


\subsection{Data used to construct Fig. 4}
 
Ref.~\onlinecite{TaillFS} did not provide any estimate of $t'$.  To map their data onto Fig.~4, we assumed that the ARPES estimate of $x_{VHS}$ is essentially $x_{VHS0}$ and plotted the data at the appropriate $t'$.  Then Fig.~\ref{fig:A1} can be used to convert $x_{VHS0}$ to $t'$, both for the data of Refs~\onlinecite{CDMFT2,CDMFT1,Tremblay}, for which $t''=0$ (dark red diamonds), and for the parameters of the Pavarini-Andersen model, with $t''=-t'/2$ (violet diamonds). This translates into experimental\cite{TaillFS} estimates $t'_{PA}$ = -0.11 (LSCO), -0.133 (Nd-substituted LSCO), and -0.23 (Bi2201), in good agreement with other estimates.
\begin{figure}
\rotatebox{0}{\scalebox{0.8}{\includegraphics{lscoBBa_xVHS_G2}}}
\vskip0.5cm
\caption{%(Color online)
VHS doping $x_{VHS}$ in the $t-t'-t''$ model for $t''/t'$ = 0 (red dots) and -0.5 (blue dots).  Dark red and violet diamonds represent $t'/t$ for the data of Ref.~\onlinecite{TaillFS} within the corresponding models. Note that at low doping $t'$ for $t''=-0.5t'$ is approximately half $t'$ for $t''=0$ (blue +-signs).  Golden circles indicate hoVHSs.
}
\label{fig:A1}
\end{figure} 



 


We note one further difference between our results in Paper 1\cite{Paper1} and those of Ref.~\onlinecite{CDMFT2}, Fig.~4: in the pure Hubbard model we find the AFM VHS at the band top, so $x_{VHS}$ should extrapolate to $0^+$ as $x\rightarrow 0$.  However, a small broadening of the VHS would shift the peak to finite doping, and due to the strong divergence of the DOS, a shift to $x\sim$~0.11 would not be surprising.

\section{Hypothesis for confined phase near $x_{VHS}$}
\subsection{SO(8)}

A key question in this model is, what constitutes the second phase confined on the charge stripes?  If it arises near $x_{VHS}$, it should be one of the phases enumerated in the VHS spectrum generating algebra SO(8)\cite{SO8}.  This group is an extension of S.-C. Zhang's SO(5)\cite{SO5} model of AFM-d-wave superconductivity competition, to include additional competing phases.   In SO(8) the competing ordering phases can have ordering vectors $Q$ = $(\pi.\pi)$, $(\pi,0)$, or $\Gamma=(0,0)$.  A competing phase at $\gamma$ arises naturally in a model of inter- vs intra-VHS competition\cite{hoVHS1}. This can include superconductivity, ferromagnetism\cite{Kapit}, or CDW order.  Below we focus on CDW order, since evidence of a CDW has been found experimentally in many cuprates.
The latter is predicted to dominate for $t'/t \le -0.23$,\cite{hoVHS1} but so far there is little evidence for cuprates  with such large $|t'|$. Further, this leads us out of the strict SO(8) model, because the actual Fermi surface nesting is not at $Q=0$, when the VHS crosses the Fermi energy, but at a finite $Q$ which obeys $Q\rightarrow 0$ as $x\rightarrow x_{VHS}$.

This leads to a Hamiltonian matrix
\begin{equation}
\begin{pmatrix}
\epsilon_k & -O_{SDW} & -\Delta_d & \Pi & -O_{CDW} \\
O_{SDW} & \epsilon_{k+Q} & -\Pi & -\Delta_d & -\tau \\
\Delta_d & \Pi & -\epsilon_k & O_{SDW} & 0\\
-\Pi & \Delta_d & -O_{SDW} & -\epsilon_{k+Q} & 0\\
O_{CDW} & \tau & 0& 0& \epsilon_{k+q}\\
\end{pmatrix},
\label{eq:1}
\end{equation}
where we retain the basic notation of Ref.~\onlinecite{SO8} for ease in comparison.
Comparing Eq.~\ref{eq:1} to Eq. 1 of Ref.~\onlinecite{SO8}, we note that the elements of the spectrum-generating algebra are proportional to the various gap functions.  Thus, $\epsilon_k$ is the bare electronic dispersion, $O_{CDW}$ ($O_{SDW})$ is the CDW (SDW) order parameter, $\Delta_d$ is the $d$-wave superconductor order parameter, $\Pi$ is the pair-density wave order parameter, and we focus on the $z$-components of the vector operators $O_{SDW}$ and $\Pi$.


We note the following points.  (1) When CDW order is neglected, Eq.~\ref{eq:1} reduces to a form of Zhang's $SO(5)$, although the form of his Lie algebra looks different.  This is because the elements of the algebra can be interpreted either as order parameters or as symmetry elements, so that, e.g., $S_z$ could be interpreted as the generator of rotations about the $z$-axis, or as a component of the ferromagnetic order parameter.  We recall Zhang's prediction that if two components of the order parameter are present (e.g., $O_{SDW}$ and $\Delta_d$), then the order associated with their commutator ($\Pi$) is also present. (2) For present purposes, we are only interested in pseudogap physics, and so neglect the superconducting orders, thereby reducing Eq.~\ref{eq:1} to
\begin{equation}
\begin{pmatrix}
\epsilon_k & -O_{SDW} &  -O_{CDW} \\
O_{SDW} & \epsilon_{k+Q} &  -\tau \\
O_{CDW} & \tau & \epsilon_{k+q}\\
\end{pmatrix}.
\label{eq:2}
\end{equation}
While the original $SO(8)$ matrix is defined for the VHSs, in which case $Q=(\pi,\pi)$ and $q=0$, we extend the definition to account for the CDW nesting vector $q$, in which case $\tau$ represents a shear distortion connecting $Q$ and $q$.  (3) In this case, if $q$ is rational, $qa/2\pi=n/m$, where $m$ and $n$ are relatively prime integers, then the CDW Hamiltonian by itself becomes an $m\times m$ matrix\cite{smectic}, so self consisting in $q$ is extremely difficult.   
%(4) While the susceptibility usually finds a CDW peak at $(q,q)$, this only defines the initial CDW vector for $O_{CDW}\rightarrow 0$, whereas experiment finds a finite-$O_{CDW}$ CDW at ordering vector $(q,0)$.  

The picture of AFM to CDW crossover, driven by VHS competition between interVHS coupling  at $Q=(\pi,\pi)$ and intraVHS coupling at $q\sim 0$, is very appealing, and consistent with the Mott-Slater crossover.  However, it has serious flaws.  First, it does not explain the presence of the stripes.  Secondly, it does not explain why pseudogap collapse is associated with loss of AFM order\cite{Paper1}. However, it remains possible that the confined phase in the AFM texture evolves into this phase as $x\rightarrow x_{VHS}$.  Moriya\cite{Moriya} expected a similar crossover from an insulating AFM to a $q=0$ metal, but in his case the metal was a ferromagnet -- an alternate choice in $SO(8)$.
\subsection{Beyond the hoVHS}
\begin{figure}
\rotatebox{0}{\scalebox{0.4}{\includegraphics{Qstripe_L1}}}
\vskip0.5cm
\caption{%(Color online)
{\bf Doping beyond the hoVHS.} (a) Dispersions for $t'/t$ = -0.32 (red), -0.40 (blue), and -0.46 (green). (b) Fermi surfaces for $t'/t$ = -0.32 and doping (along the diagonal from top right to bottom left) $x$ = -0.03, 0.04, 0.12, 0.21, 0.31, 0.45, 0.58, 0.67, and 0.70. Blue curves are in the camelsback range where the Fermi surface has three pockets, and green curves show well-nesting regions in the antinodal regimes near $(\pi,0)$ and $(0,\pi)$.  (c)  Similar to (b), except $t'/t$ = -0.40 and doping $x$ = -0.04, 0.02, 0.10, 0.17, 0.27, 0.37, 0.52, 0.71, 0.79, 0.84, and 0.86.
}
\label{fig:A3}
\end{figure} 
Perhaps the best place to look for the competing order is in materials with larger $|t'|$, at doping beyond the hoVHS, where intra-VHS scattering should dominate\cite{hoVHS1}, and both superconductivity and pseudogap are absent.  We have proposed that this situation can arise for bilayer and trilayer cuprates, where the bonding band is heavily doped, has a large $|t'|$ value, and has less to do with superconductivity\cite{hoVHS1}.  Figure~\ref{fig:A3}(a) shows how a camelsback feature develops in the dispersion near $(\pi,0)$ when $t'$ extends beyond the critical value for the hoVHS.  This causes a pocket to split away from the main Fermi surface near $(\pi,0)$ [and $(0,\pi)$], blue Fermi surfaces in frames (b) and (c).  In turn, as doping is reduced away from thr camelsback range the Fermi soufaces become very flat near the antinodal points (green curves), leading to exceptionally strong nesting.  Such a  CDW instability may have already been observed.\cite{reent}

\section{Charge order at large doping}
\subsection{Nesting related origin of $S_2$ curve}
In Fig.~4(b) of the main text, we compare the $Q$ vector of the $S_2$ branch to a variety of models of Fermi surface nesting for Bi2201.  We contrast two tight-binding models, the first an accurate multi-parameter fit to experimental data\cite{JennySci} (the violet dot-dashed and dashed curves), the second to a three-parameter member of the cuprate `reference family'\cite{MBMB} used in Paper 1\cite{Paper1} (blue dashed curve) to simultaneously model all cuprates using a single tuning parameter for each cuprate family.  The NM nesting vector was taken as $Q_x=2k_{Fx}$, for the Fermi surface points $(-k_{Fx},\pi)$ and $(k_{Fx},\pi)$, leading to strong disagreement with the curve $S_2$.  We note however that the nesting $Q_x$ extrapolates to 0 at $x_{VHS}$, close to the linear extrapolation of the experimental $S_2$ curve, as predicted in the NPS model.  These results are consistent with a CDW calculation\cite{Gutzcharge}, where a $6\times 6$ crossed CDW calculation produced an electron Fermi surface analogous to the proposed QO surface, but the $4\times 4$ calculation could not produce nesting for any hole doping.  This also strengthens the case that the second phase in the NPS is a form of CDW.

Recognizing the failure of models nesting the NM Fermi surface, it was proposed that nesting related to the AFM Fermi surface -- e.g., hot-spot\cite{EHud2} or Fermi arc\cite{Comin} nesting -- might prove a better starting point.  Here, we study nesting on the same AFM Fermi surface nesting model used in Paper 1\cite{Paper1}, comparing the two tight-binding models noted above, Fig.~\ref{fig:A2}(a,b).  We note the resemblance to the AFM pockets in electron-doped cuprates. While there are a number of candidates for nesting, we need a $Q$-vector that starts out large at low doping and gets smaller with increased doping.  The main candidates are shown by the green lines in Figs.~\ref{fig:A2}(c) and 3(c) of the Main Text.  The electron-hole nesting in frame (c) would seem to be excluded, as the nesting vector is diagonal, whereas we need horizontal and vertical $Q$s.  However, we allow for the possibility of an exotic two-dimensional CDW, where the $x$ and $y$ nesting vectors are $Q_x$ and $Q_y$, respectively.  The two narrow dashed lines in Fig.4(b) represent the two tight binding models, and it is clear that neither is a good match to the data.  However, Fig. 3(c), the equivalent of hot-spot\cite{EHud2} or Fermi arc\cite{Comin} nesting, provides a good fit, thick violet dashed line in Fig. 4(b).  In Figs.~3(d,e) we illustrate the star=shaped electron pockets expected for double nesting at high magnetic fields.

\begin{figure}
\rotatebox{0}{\scalebox{0.4}{\includegraphics{Qstripe_L0}}}
\vskip0.5cm
\caption{%(Color online)
(a) AFM Fermi surfaces, using tight binding model of Ref.~\onlinecite{JennySci} (red and blue lines) or $3-t$ model with $t'/t=-0.2$ from Ref.~\onlinecite{Paper1} (green line), with AFM gap $\Delta=150$~meV.  Fermi surfaces are electron Fermi surface from upper magnetic band (blue) and hole Fermi surface from lower magnetic band (red or green).  (b) $3-t$ model for $t'/t=-0.26$.  (c) Electron-hole nesting, with arrows denoting nesting vectors $Q'$ (green) and $S'$ (orange).  The corresponding vectors for hot-spot nesting are shown in Fig.~3(c) of the Main Text.  (d,e) Illustrating $Q+S$ for electron-hole nesting (d) and hot-spot nesting (e).
}
\label{fig:A2}
\end{figure} 

In Figs.~\ref{fig:A2}(c) and 3(c), there is a second nesting vector, $S$ (orange arrows).   Putting both arrows on the same line, Figs.~\ref{fig:A2}(d,e), it can be seen that their sum connects a point in one Brillouin zone to the equivalent point in a different Brillouin zone, leading to the relations $S'_x+Q'_x=\pi/a$ for frame (d) and $S_x+Q_x=2\pi/a$ for frame (e).

Finally, we comment on the two tight binding models, the N-component (Fig.~\ref{fig:A2}(a) and thick violet curve in Fig.~3(b)) and the 3-component (green curve in Fig.~\ref{fig:A2}(a) and thick light-blue curve in Fig.~3(b)).  The three component model is a member of the reference family used in Paper 1 to study the evolution of properties with cuprate families.  While it has a very similar shape of the hole Fermi surface, the electron pockets are only present for a small doping range, so the interpocket nesting is absent.  For a slightly larger $t'$ (Fig.~\ref{fig:A2}(b)) the electron pockets reappear, and we used this dispersion to plot interpocket nesting (thin light-blue curve in Fig.~3(b)).  Note that both tight-binding models provide good fits to the hot-spot nesting, Fig.~3b.

%We found there that the Bi2201 $t'$ value seemed too large, and that is the case here as well.  Note that the light-blue curve matches the $S_2$ data well at low doping, but deviates at higher doping, leading to a too large $x_{VHS}$.  In contrast, the N-component model\cite{JennySci} (violet) provides a much improved fit.

\begin{thebibliography}{99}
\bibitem{EHud3}M.C. Boyer, W.D. Wise, K. Chatterjee, M. Yi, T. Kondo, T. Takeuchi, H. Ikuta, and E.W. Hudson,
Imaging the two gaps of the high-temperature superconductor Bi$_2$Sr$_2$CuO$_{6+ x}$,
Nature Physics 3, 802 (2007).
\bibitem{Vishik}I. M. Vishik, M Hashimoto, R.-H. He, W. S. Lee, F. Schmitt, D. H. Lu, R. G. Moore, C. Zhang, W. Meevasana, T. Sasagawa, S. Uchida, K. Fujita, S. Ishida, M. Ishikado, Y. Yoshida, H. Eisaki, Z. Hussain, T. P. Devereaux, and Z.-X. Shen,
Phase competition in trisected superconducting dome,
PNAS 109, 18332 (2012).
\bibitem{TaillFS}N. Doiron-Leyraud, O. Cyr-Choinière, S. Badoux, A. Ataei, C. Collignon, A. Gourgout, S. Dufour-Beauséjour, F.F. Tafti, F. Laliberté, M.-E. Boulanger, M. Matusiak, D. Graf, M. Kim, J.-S. Zhou, N. Momono, T. Kurosawa, H. Takagi, Louis Taillefer,
Pseudogap phase of cuprate superconductors confined by Fermi surface topology,
Nature Communications 8, 2044 (2017) %(arXiv:1712.05113).
\bibitem{CDMFT2}Wei Wu, Mathias S. Scheurer, Shubhayu Chatterjee, Subir Sachdev, Antoine Georges, and Michel Ferrero, Pseudogap and Fermi surface topology in the two-dimensional Hubbard model, Phys. Rev. X 8, 021048 (2018).
\bibitem{CDMFT1}H. Braganca, S. Sakai, M.C.O. Aguiar, and M. Civelli, Correlation-driven Lifshitz transition at the emergence of the pseudogap
phase in the two-dimensional Hubbard model Phys. Rev. Lett. 120, 067002 (2018). 
\bibitem{Tremblay}Wei Wu, Xiang Wang, and Andr\'e-Marie Tremblay,
Non-Fermi liquid phase and linear-in-temperature scattering rate in overdoped two dimensional Hubbard model,
PNAS 113, e2115819119 (2022) (arXiv:2109.02635).
\bibitem{Paper1}R.S. Markiewicz and A. Bansil,
Theory of Cuprate Pseudogap as Antiferromagnetic Order with Domain Walls,
arXiv:2206.00077.
\bibitem{SO8}R.S. Markiewicz and M.T. Vaughn, Classification of the Van Hove Scenario as an SO(8) Spectrum Generating Algebra, 
Phys. Rev. B57, 14052 (1998). 
\bibitem{SO5}S.C. Zhang, 
A Unified Theory Based on SO(5) Symmetry of Superconductivity and Antiferromagnetism,
Science 275, 1089 (1997).
\bibitem{hoVHS1}Robert S. Markiewicz, Bahadur Singh, Christopher Lane, and Arun Bansil,
High-order Van Hove singularities in cuprates and related high-Tc superconductors,
arXiv:2105.04546.
\bibitem{Kapit}Tarapada Sarkar, D. S.Wei, J. Zhang, N.R. Poniatowski, P.R. Mandal, A. Kapitulnik, and Richard L. Greene,
Ferromagnetic order beyond the superconducting dome in a cuprate superconductor,
Science 368, 532 (2020).
\bibitem{Gutzcharge}R.S. Markiewicz, G. Seibold, J. Lorenzana, and A. Bansil, 
Gutzwiller charge phase diagram of cuprates, including electron–phonon coupling effects,
New J. Phys. 17, 023074 (2015) [Supp. Mat.].
\bibitem{smectic}R.S. Markiewicz, J. Lorenzana, G. Seibold, and A. Bansil, 
%Short range smectic order driving long range nematic order: example of cuprates, Scientific reports 6, 19678 (2016).
\bibitem{Moriya}T. Moriya,
{\it Spin Fluctuations in Itinerant Electron Magnetism},
(Springer, Berlin, 1985).
\bibitem{reent}Y.Y. Peng, R. Fumagalli, Y. Ding, M. Minola, S. Caprara, D. Betto, M. Bluschke, G.M. De Luca, K. Kummer, E. Lefrançois, M. Salluzzo, H. Suzuki, M. Le Tacon, X.J. Zhou, N.B. Brookes, B. Keimer, L. Braicovich, M. Grilli, and G. Ghiringhelli,
Re-entrant charge order in overdoped (Bi,Pb)$_{2.12}$Sr$_{1.88}$CuO$_{6+\delta}$ outside the pseudogap regime,
Nat. Mat. 17, 697 (2018).
\bibitem{JennySci}Yang He, Yi Yin, M. Zech, Anjan Soumyanarayanan, Michael M. Yee, Tess Williams, M. C. Boyer, Kamalesh Chatterjee, W. D. Wise, I. Zeljkovic, Takeshi Kondo, T. Takeuchi, H. Ikuta, Peter Mistark, Robert S. Markiewicz, Arun Bansil, Subir Sachdev, E. W. Hudson, and Jennifer E. Hoffman,
Fermi Surface and Pseudogap Evolution in a Cuprate Superconductor,
Science 344, 608 (2014).
\bibitem{MBMB}R.S. Markiewicz, I.G. Buda, P. Mistark, an A. Bansil, 
Entropic origin of pseudogap physics and a Mott-Slater transition in cuprates
{\it Nature Scientific Reports} {\bf 7,} 44008 (2017). 
\bibitem{EHud2}Tatiana A. Webb, Michael C. Boyer, Yi Yin, Debanjan Chowdhury, Yang He, Takeshi Kondo, T. Takeuchi, H. Ikuta, Eric W. Hudson, Jennifer E. Hoffman, Mohammad and H. Hamidian, 
Density wave probes cuprate quantum phase transition,
Phys. Rev. X 9, 021021 (2019).
\bibitem{Comin}R. Comin, A. Frano, M.M. Yee, Y. Yoshida, H. Eisaki, E. Schierle, E. Weschke, R. Sutarto, F. He, A. Soumyanarayanan, Yang He, M. Le Tacon, I.S. Elfimov, J.E. Hoffman, G.A. Sawatzky, B. Keimer, and A. Damascelli,
Charge order driven by Fermi-arc instability in Bi$_2$Sr$_{2-x}$La$_x$CuO$_{6+\delta}$,
Science 343, 390 (2014).
\bibitem{SiLevin}Qimiao Si, Yuyao Zha, K. Levin, and J. P. Lu,
Comparison of spin dynamics in YBa$_2$Cu$_3$O$_{7-\delta}$ and La$_{2-x}$Sr$_x$CuO$4$: Effects of Fermi-surface geometry,
Phys. Rev. B 47, 9055 (1993).



\end{thebibliography}

\end{document}
