\documentclass[11pt]{article}
\setlength{\skip\footins}{0.6cm}

%-------------------PACKAGES-----------------------%
\usepackage[utf8]{inputenc}
\usepackage{lmodern}
\usepackage{subfiles}
\usepackage{enumitem}
	\setenumerate{label={\normalfont(\roman*)}, itemsep=0em} % or \upshape
       \usepackage{pgf,tikz}
\usetikzlibrary{arrows} 
\usepackage{amsfonts}
\usepackage{amsthm}
\usepackage{amsmath}
\usepackage{amssymb}
\usepackage{amscd}
\usepackage{mathrsfs}
\usepackage{mathtools}
\usepackage{bm}
%\usepackage{pxfonts}
\usepackage{esint}
%\usepackage[mathscr]{eucal}
%\usepackage{accents}

\usepackage[margin=3cm]{geometry}
\usepackage{indentfirst}
\usepackage{graphicx}
\usepackage{graphics}
\usepackage{lscape}
\usepackage{tikz-cd}
\usepackage{color}
\usepackage{pict2e}
\usepackage{epic}
\usepackage{epstopdf}
\usepackage{titlesec, titlefoot}
	\titleformat{\section}[block]{\Large\bfseries\filcenter}{\thesection}{1em}{}
\usepackage{commath}
\usepackage{float}
\usepackage{caption}
\usepackage{etoolbox}
\usepackage[affil-it]{authblk}
\usepackage{combelow}

\usepackage[hidelinks]{hyperref}
\hypersetup{bookmarksopen=true} 
\usepackage{hypcap}

\graphicspath{{./Pictures/}}
\allowdisplaybreaks

%----------COMPUTER CODE-----------%

\usepackage{listings}
\lstset{
	language=Mathematica,
	basicstyle=\small\sffamily,
%	numbers=left,
	numberstyle=\tiny,
%	frame=tb,
	columns=fullflexible,
	showstringspaces=false
}

\usepackage{fancybox}
\makeatletter
\newenvironment{CenteredBox}{% 
\begin{Sbox}}{% Save the content in a box
\end{Sbox}\centerline{\parbox{\wd\@Sbox}{\TheSbox}}}% And output it centered
\makeatother


%------------THEOREM-TYPE ENVIRONMENTS-------------%

%\renewcommand*{\thefootnote}{(\arabic{footnote})}

\theoremstyle{plain}
\newtheorem{bigthm}{Theorem}
\renewcommand\thebigthm{\Alph{bigthm}}
\newtheorem{bigprop}[bigthm]{Proposition}
\newtheorem{bigcor}[bigthm]{Corollary}


\renewcommand*\thesection{\arabic{section}}
%\swapnumbers
% \setcounter{section}{-1}
\numberwithin{equation}{section} 

\theoremstyle{plain}
\newtheorem{thm}{Theorem}
\newtheorem{example}[thm]{Example}
\newtheorem{lemma}[thm]{Lemma}
\newtheorem{prop}[thm]{Proposition}
\newtheorem{corollary}[thm]{Corollary}
\newtheorem{theorem}[thm]{Theorem}
\numberwithin{thm}{section}

\expandafter\let\expandafter\oldproof\csname\string\proof\endcsname
\let\oldendproof\endproof
\renewenvironment{proof}[1][\proofname]{%
  \oldproof[\upshape \bfseries #1]%
}{\oldendproof}
%tirar \upshape poe em italico

\makeatletter
\def\@makechapterhead#1{%
  \vspace*{50\p@}%
  {\parindent \z@ \raggedright \normalfont
    \interlinepenalty\@M
    \Huge\bfseries  \thechapter.\quad #1\par\nobreak
    \vskip 40\p@
  }}
\makeatother




%\usepackage{dsfont}

\newcommand{\reqnomode}{\tagsleft@false}
\vfuzz3pt % non annoiare per fuoriuscite verticali di 3 pt
\hfuzz3pt % Don't report over-full h-boxes if over-edge is small
\textwidth = 14.7 cm
\textheight = 22.4 cm 
\oddsidemargin = 0.6cm
\evensidemargin = 1cm 
\topmargin = 1mm
\headheight = 0.2 cm
\headsep = 0.7 cm

%\usepackage[colorlinks,pdfpagelabels,pdfstartview = FitH,bookmarksopen = true,bookmarksnumbered = true,urlcolor=url,linkcolor = formula,plainpages = false,hypertexnames = false,citecolor = citation] {hyperref}
\def\dys{\displaystyle}
\def\vs{\vspace{1mm}}
\def\dxy{\,{\rm d}x{\rm d}y}
\def\dx{\,{\rm d}x}
\def\da{\,{\rm d}a}
\def\dy{\,{\rm d}y}
\def \d{\,{\rm d}}
\def\dist{\,{\rm dist}}
\def\supp{\,{\rm supp }}
\def\diam{\,{\rm diam}}
\newcommand{\const}{\operatorname{const}}
\def\er{\mathbb R}
\allowdisplaybreaks
\makeatletter
\DeclareRobustCommand*{\bfseries}{%
  \not@math@alphabet\bfseries\mathbf
  \fontseries\bfdefault\selectfont
  \boldmath
}

\makeatother

\newlength{\defbaselineskip}
\setlength{\defbaselineskip}{\baselineskip}

%%%%%%%%%%%%%%%%%%%%%%%%%%%%%%%%%%%%%%%%%%%%%%%%

\newcommand{\N}{\mathbb{N}}
\newcommand\eps\varepsilon
\def\mean#1{\mathchoice%
          {\mathop{\kern 0.2em\vrule width 0.6em height 0.69678ex depth -0.58065ex
                  \kern -0.8em \intop}\nolimits_{\kern -0.4em#1}}%
          {\mathop{\kern 0.1em\vrule width 0.5em height 0.69678ex depth -0.60387ex
                  \kern -0.6em \intop}\nolimits_{#1}}%
          {\mathop{\kern 0.1em\vrule width 0.5em height 0.69678ex
              depth -0.60387ex
                  \kern -0.6em \intop}\nolimits_{#1}}%
          {\mathop{\kern 0.1em\vrule width 0.5em height 0.69678ex depth -0.60387ex
                  \kern -0.6em \intop}\nolimits_{#1}}}


\def\loc{\operatorname{loc}}
%modificable commands i.e. if you do not like the notation

\def\eqn#1$$#2$${\begin{equation}\label#1#2\end{equation}}
\delimitershortfall=-0.1pt
\newcommand\R{\mathbb{R}}
\def\ds{\,{\rm d}s}
\newcommand{\Div}{\mathrm{div}\,}
\newcommand{\F}{\mathscr F}
\newcommand{\G}{\mathscr G}

\def \Z {\mathbb{Z}}
\def \red{\textcolor{red}}
\def \tp{\textup}
\def \p{\partial}
\def \e{\varepsilon}
\def \D{\mathrm{D}}
\def \ri{\tp{ri}}
\def \M{\mathbb M}
\def \LL{\mathrm L}

\def \WW{\mathrm{W}}
\def \BV{\mathrm{BV}}
\def \wstar {\overset{\ast}{\rightharpoonup}}
\def \E{\mathscr E}
\def \P{\mathbb P}

%restrictions
\newcommand\restr[2]{{% we make the whole thing an ordinary symbol
  \left.\kern-\nulldelimiterspace % automatically resize the bar with \right
  #1 % the function
  \vphantom{|} % pretend it's a little taller at normal size
  \right|_{#2} % this is the delimiter
  }}
  

\newcommand{\mres}{\mathbin{\vrule height 1.6ex depth 0pt width
0.13ex\vrule height 0.13ex depth 0pt width 1.3ex}}
  

\newcommand*{\ou}[2]{\overset{\text{\large ${#1}$}}{#2}}
\title{Geometric regularisation for optimal transport with strongly $p$-convex cost}
\author{Lukas Koch}

\affil[1]{\small Max Planck Institute for Mathematics in the Sciences, 04103 Leipzig,Germany
\protect \\
  {\tt{lkoch@mis.mpg.de}}
  \vspace{1em} \ }

%\affil[2]{\small University of Oxford, Andrew Wiles Building Woodstock Rd, Oxford OX2 6GG, United Kingdom 
%\protect \\
%  {\tt{jan.kristensen@maths.ox.ac.uk}}
%  \vspace{1em} \ }


%more space in equations
\usepackage{etoolbox}
\makeatletter
\patchcmd{\@adjustvertspacing}
  {\jot\baselineskip \divide\jot 4}
  {\jot=.5\baselineskip}
  {}{}
\makeatother



\makeatother
\begin{document}

\maketitle
\begin{abstract}
We prove a geometric regularisation result for minimisers of optimal transport problems where the cost-function is strongly $p$-convex and of $p$-growth. Initial and target measures are allowed to be rough, but are assumed to be close to Lebesgue measure.
\end{abstract}

\section{Introduction}\label{sec:intro}
The study of the optimal transport problem:
\begin{align}\label{eq:problem}
\min_{\pi\in \Pi(\lambda,\mu)}\int_{\R^d\times \R^d} c(x-y)\d\pi
\end{align}
is well established. We refer the reader to \cite{Villani2009} and \cite{Santambrogio2015} for an introduction and overview of the literature. When solutions take the form of a transport map $\pi = (\tp{Id}\times T)_\#\mu$, under mild assumptions, minimisers are characterised by satisfying the Euler-Lagrange equation
\begin{align}\label{eq:EL}
\lambda(x)\tp{det}\, (\D T(x)) = \mu(T(x))
\end{align}
as well as the additional structure condition
\begin{align}
T(x)=x+ \nabla c^\ast(\D\phi),
\end{align}
where $\phi$ is a $c$-concave function and $c^\ast$ denotes the convex conjugate of $c$. Assuming $\mu\sim\lambda\sim 1$ and linearising the geometric nonlinearity in \eqref{eq:EL}, that is formally expanding $\tp{det}(\tp{Id}+A)=1+\tp{tr}\, A+\ldots$, we find that
\begin{align}\label{eq:linearised}
\tp{div}\, \nabla c^\ast(\D\phi)=\mu-\lambda.
\end{align}
Thus, at least formally, we expect solutions of \eqref{eq:problem} to be well approximated by solutions of \eqref{eq:linearised}. Note that in general \eqref{eq:linearised} is a nonlinear equation. Thus we refer to the process of moving from \eqref{eq:problem} to \eqref{eq:linearised} as geometric linearisation. The aim of this paper is to make this connection rigorous.
We show the following:
\begin{theorem}\label{thm:main}Let $1<p<\infty$.
Suppose $c\colon \R^d\to \R$ is a strongly $p$-convex cost function of controlled-duality $p$-growth. Let $\pi$ be a minimiser of \eqref{eq:problem} for some measures $\lambda$, $\mu$. Denote
\begin{gather*}
E(R) := \frac 1 {\lvert B_{R}\rvert}\int_{(B_{R}\times \R^d)\cup (\R^d\times B_{R})}c(x-y)\d\pi \\
D(R) := \frac 1 {\lvert B_{R}\rvert} W_{p}^p(\lambda\llcorner B_R,\kappa_{\lambda,R}\d x\llcorner B_R)+\frac{R^p}{\kappa_{\lambda,R}^{p-1}}(\kappa_{\lambda,R}-1)^p\\
\qquad\qquad+\frac 1 {\lvert B_{R}\rvert} W_{p}^p(\mu\llcorner B_R,\kappa_{\mu,R}\d x \llcorner B_R)+\frac{R^p}{\kappa_{\mu,R}^{p-1}}(\kappa_{\mu,R}-1)^p.
\end{gather*}
Here $\kappa_{\lambda,R} = \frac{\lambda(B_R)}{\lvert B_R\rvert}$ and $\kappa_{\mu,R} = \frac{\mu(B_R)}{\lvert B_R\rvert}$.
Then, for every $\tau>0$, there exists $\e(\tau)>0$ such that if ${E(4)+D(4)\leq \e}$, then there exists a radius $R\in(2,3)$, $c\in \R$ and $\phi$ satisfying
\begin{align*}
-\Div \nabla c^\ast(\D\phi) = c \text{ in } B_R 
\end{align*}
such that
\begin{align*}
\int_{(B_1\times \R^d)\cup (\R^d\times B_1)}c(x-y-\nabla c^\ast(\D\phi))\d\pi\lesssim \tau E(4)+ D(4).
\end{align*}
Moreover
\begin{align*}
\sup_{B_1}|\D\phi|^{p'}+\int_{B_R}\lvert\D\phi\rvert^{p'}\d x\lesssim E(4)+D(4).
\end{align*}
\end{theorem}

We remark that we explain our assumptions on the cost function in detail in Section \ref{sec:costFunction}.

Traditionally, \eqref{eq:problem} has been approached via the study of \eqref{eq:EL} using the theory of fully nonlinear elliptic equations developed by \textsc{Caffarelli}, see e.g. \cite{Figalli2010,DePhilippis2015} and the references therein. Recently, an alternative approach using variational techniques has been developed by \textsc{Goldman and Otto} in \cite{Goldman2020}. There, partial $C^{1,\alpha}$-regularity for solutions to \eqref{eq:problem} in the case of H\"older-regular densities $\lambda$, $\mu$ and quadratic cost function $c(x-y)=\frac 1 2 |x-y|^2$ was proven. The key tool in the proof was a version of Theorem \ref{thm:main} in this setting. In later papers, continuous densities \cite{Goldman2020a}, rougher measures \cite{Goldman2021}, more general cost functions (albeit still close to the quadratic cost functional) \cite{Otto2021}, as well as almost-minimisers of the quadratic cost functional \cite{Otto2021} were considered. The quadratic version of Theorem \ref{thm:main} was also used to provide a more refined linearisation result of \eqref{eq:EL} in the quadratic set-up in \cite{Goldman2021} and of a similar statement in the context of optimal matching in \cite{Goldman2022}. Finally, quadratic versions of Theorem \ref{thm:main} played a key role in disproving the existence of a stationary cyclically monotone Poisson matching in $2$-d \cite{Huesmann2021}. 

We remark that very little information is available about the regularity of minimiser of \eqref{eq:problem} already in the simplest degenerate/singular case $c(x-y)=\frac{|x-y|^p}{p}$. In order to attempt to extend the techniques of \cite{Goldman2020} to this setting, an essential first step is Theorem \ref{thm:main}. This result will also play a key role in extending the results of \cite{Huesmann2021} to $p$-costs with $p\neq 2$.

The strategy of proof is similar to that used in \cite{Goldman2021}, although with a number of simplifications. Further, we point the reader to \cite{Koch2023} where a detailed account of the proof of Theorem \ref{thm:main} and the motivations behind the strategy are given in the quadratic case.

 Here we only comment on the steps where additional effort is required. An essential part of the proof is to obtain a $L^\infty$-bound for minimisers of \eqref{eq:problem} in the small-energy regime, see Section \ref{sec:Linfty}. In the quadratic case, this relies on the monotonicity (in the classical sense) property of solutions. In the non-quadratic case, $c$-monotonicity needs to be used directly. Our proof relies on the same intuition but highlights how the $L^\infty$-bound is a direct consequence of convexity and growth bounds of the cost function. We remark that $L^\infty$-bounds for $p$-homogeneous cost functions with $p\geq 2$ in all energy regimes were obtained in \cite{Gutierrez2021}.

A further key step is the so-called quasi-orthogonality property, see Section \ref{sec:proof}. In the quadratic case, this relied on expanding squares. Again, if $p\neq 2$, this tool is not available and needs to be replaced by exploiting inequalities expressing the convexity of $c$. Finally, regularity properties of solutions to \eqref{eq:linearised} play a key role in the proof. In the quadratic case, such solutions are harmonic and hence very regular- the proof in \cite{Goldman2021} requires $C^3$-regularity of solutions! Already in the case $c(x-y)=\frac{|x-y|^p}{p}$ with $p\neq 2$, the best regularity that is known for solutions to \eqref{eq:linearised} in general is $C^{1,\beta}$-regularity for some $\beta>0$. Thus, at various places in the proof, more careful estimates are needed.

Finally, we comment why we restrict our attention to cost functions of the form $c(x-y)$. This is due to the fact that our proof relies on the availability of a dynamical formulation. We want to identify points $(x,y)\in \mathrm{spt}\,\pi$ with the trajectory ${X(t)=tx+(1-t)y}$. This is related to the Benamou-Brenier formulation of optimal transport, c.f. \cite{Gangbo1995}, which in our case states that \eqref{eq:problem} can be alternatively characterised as
\begin{align}\label{eq:eulerianFormulation}
\min_{(j,\rho)}\left\{\int c\left(\frac{\d j}{\d \rho}\right)\d\rho\colon \partial_t \rho+\Div j = 0,\, \rho(0)=\lambda,\, \rho(1)=\mu\right\}
\end{align}
Here $\frac{\d j}{\d \rho}$ denotes the Radon-Nikodym derivative. This alternative, dynamical formulation of optimal transport is only available for costs of the form $c(x-y)$ where $c$ is convex.



\subsection{Assumptions on the cost function and its dual}\label{sec:costFunction}
In this section, we explain our assumptions on the cost function $c$. The convex theory we quote can be found in \cite{Rockafellar1970} and \cite{Hiriart-Urruty1993a}.

Let $p\in(1,\infty)$. We consider a $C^1$-cost function $c\colon \R^d\to \R$ satisfying the following properties:
There is $\Lambda\geq 1$ such that
\begin{enumerate} 
\item $c$ is strongly $p$-convex: for any $x,y\in \R^d$ and $\tau\in[0,1]$,
\begin{align}\label{ass:elliptic}
\Lambda^{-1} \tau(1-\tau) V_p(x,y)+c(\tau x+(1-\tau)y)\leq \tau c(x)+(1-\tau)c(y).
\end{align}
where 
$$
V_p(x,y)=\begin{cases}
		(|x|^2+|y|^2)^\frac{p-2} 2|x-y|^2 &\quad \text{ if } p\leq 2\\
		|x-y|^p &\quad \text{ if } p\geq 2.
		\end{cases}
$$
\item $c$ has $p$-growth: for any $x\in \R^d$, 
\begin{align}\label{ass:growth}
\Lambda^{-1} |x|^p\leq c(x)\leq \Lambda |x|^p.
\end{align}
\item for any $x,y\in \R^d$,
\begin{align}\label{ass:Cgrowth}
|c(x)-c(y)|\leq \Lambda U_p(x,y)
\end{align}
where
$$
U_p(x,y)= (|x|+|y|)^{p-1}|x-y|.
$$
\end{enumerate}
If the choice of $p$ is clear from the context, we will write $V=V_p$ and $U=U_p$. We further note the following elementary inequality, valid for any $z_1,z_2\in \R^d$ and with implicit constants depending only on $p$ and $d$,
\begin{align}\label{eq:VDiff}
\lvert V_p(z_1)-V_p(z_2)\rvert \lesssim (\lvert z_1\rvert + \lvert z_2\rvert)^{p-1}\lvert z_1-z_2\rvert.
\end{align}
Note that \eqref{ass:elliptic} and the fact that $c$ is $C^1$ imply that for any $x,y\in \R^d$,
\begin{align}\label{ass:smoothness}
c(x)\geq c(y)+\langle \nabla c(y),x-y\rangle +\lambda V(x,y)\\
\langle \nabla c(x)-\nabla c(y),x-y\rangle \geq \lambda V(x,y).
\end{align}

Introduce the convex conjugate $c^\ast$ defined on $\R^d$ via
$$
c^\ast(y)= \sup_{x} \langle y,x\rangle - c(x).
$$
Note that due to \eqref{ass:growth}, we have
\begin{align}\label{ass:growthDual}
|z|^{p'}\lesssim c^\ast(z)\lesssim |z|^{p'}.
\end{align}
Due to strict $p$-convexity of $c$, $c^\ast$ satisfies
\begin{align}\label{eq:c1growthDual}
|\nabla c^\ast(x)-\nabla c^\ast(y)|\lesssim \begin{cases}
			(|x|+|y|)^{p'-2}|x-y| \quad&\text{ if } p\leq 2 \\
			|x-y|^{p'-1} &\text{ if } p\geq 2.
			\end{cases}						
\end{align}
Finally, it follows from \eqref{ass:growthDual} and \eqref{eq:c1growthDual} that for $x,y\in \R^d$,
\begin{align}\label{eq:CgrowthDual}
|c^\ast(x)-c^\ast(y)|\lesssim U_{p'}(x,y).
\end{align}

In addition to \eqref{ass:elliptic}-\eqref{ass:Cgrowth}, which are standard assumptions quantifying the convexity, smoothness and growth of $c$, we need to ensure that also $c^\ast$ is $p$-convex. This is ensured by requiring a slightly non-standard growth assumption on the Lipschitz constant of $c$. To be precise, we assume controlled duality $p$-growth on $c$, that is
\begin{align}\label{eq:controlledDualityGrowth}
|\nabla c(x)-\nabla c(y)|\leq \begin{cases}
\Lambda (|\nabla c(x)|+|\nabla c(y)|)^\frac{p-2}{p-1}|x-y| \quad&\text{ if } p\geq 2\\
\Lambda |x-y|^{p-1} &\text{ if } p\leq 2.
\end{cases}
\end{align}
Assuming \eqref{eq:controlledDualityGrowth}, $c^\ast$ is $p^\prime$-convex, that is for some $c=c(p,\Lambda)>0$,
\begin{align}\label{eq:p'convex}
c \tau(1-\tau) V_{p'}(x,y)+c^\ast(\tau x+(1-\tau)y)\leq \tau c^\ast(x)+(1-\tau)c^\ast(y).
\end{align}

We remark that \eqref{eq:controlledDualityGrowth} implies the more standard controlled $p$-growth condition
$$
|\nabla c(x)-\nabla c(y)|\lesssim \begin{cases}
\Lambda (|x|+|y|)^{p-2}|x-y| \quad&\text{ if } p\geq 2\\
\Lambda |x-y|^{p-1} &\text{ if } p\leq 2.
\end{cases}
$$
Further, we point out that \eqref{eq:controlledDualityGrowth} is satisfied by polynomial cost functions as well as $p$-cost \cite{DeFilippis2020}.

To close this section, we note that $\nabla c^{\ast} = (\nabla c)^{-1}$ and recall the Fenchel-Young inequality in the form
\begin{align}\label{eq:FenchelYoung}
c(\xi) +c^\ast(x)\geq \langle \xi,x\rangle \quad \forall \xi, x\in \R^d,
\end{align}
with equality if and only if $\xi = \nabla c^\ast(x)$.

\subsection{Regularity assumptions on the dual system}
In this section, we state the regularity assumptions we make on distributional solutions $\phi\in W^{1,p'}(B)$ of the equation
\begin{align}\label{eq:dualEquation}
-\Div \nabla c^\ast(\D\phi) &= c_g \quad\text{ on } B\\
\nabla c^\ast(\D\phi)\cdot\nu &= g \quad\text{ on } \p B,
\end{align}
where $g\in L^{p}(B)$ and $c_g$ satisfies the compatibility condition $\lvert B\rvert c_g = \int_{\p B} g$. $\nu$ denotes the outward pointing normal vector on $\p B$. Note that solutions are only defined up to a constant. Hence, we usually normalise solutions by requiring that $\int_B \phi = 0$.

We assume that solutions to \eqref{eq:dualEquation} exist, are unique up to constant and moreover satisfy the energy estimate
\begin{align}\label{eq:energy}
\int_{B_R}|\D\phi|^{p'}\d x\lesssim \int_{\p B_R} |g|^p.
\end{align}
Using \eqref{eq:FenchelYoung}, \eqref{ass:growthDual} and Young's inequality, \eqref{eq:energy} implies that also
\begin{align}\label{eq:alternativeEnergy}
\int_{B_R} c(\nabla c^\ast(\D\phi))\d x\lesssim \int_{\p B_R} |g|^p.
\end{align}
We further assume interior regularity of solutions: for $r<R$,
\begin{align}\label{eq:interior}
\sup_{x\in B_r} \lvert \D\phi\rvert^{p^\prime} \lesssim_{R-r} \int_{\p B_R} \lvert g\rvert^p.
\end{align}

Suppose $\phi^r$ solves \eqref{eq:dualEquation} with data $g^r$, where $g^r$ denotes convolution with a smooth convolution kernel on $\partial B$ at scale $r$. Then we require
\begin{align}\label{eq:diff}
\int_B |\D\phi-\D\phi^r|^{p'}\lesssim \begin{cases}
		r\int_{\p B} |g|^p \quad&\text{ if } p\leq 2\\
		r^\frac{p'} 2\int_{\p B} |g|^p &\text{ if } p\geq 2
		\end{cases}
\end{align}

Denote by $g^r$ the convolution of $g$ with a smooth convolution kernel at scale $r$ on $\p B$. Denote by $\phi^r$ the solution of \eqref{eq:dualEquation} with data $g^r$.
Then we require that for some $\beta\in(0,1)$,
\begin{align}\label{eq:regularityPhiR}
r^\beta[\D\phi^r]_{C^{0,\beta}(B)}^{p'}+\sup_{B} |\D\phi^r|^{p'}\lesssim \frac 1 {r^{d-1}}\int_{\p B} |g|^{p}.
\end{align}

Moreover, we note by direct calculation using \eqref{ass:Cgrowth}, \eqref{eq:CgrowthDual} and \eqref{eq:c1growthDual} that 
\begin{align*}
[c^\ast(\D\phi)+c(\nabla c^\ast(\D\phi))]_{C^{0,\beta}(B)} \lesssim \|\D\phi\|_{L^\infty(B)}^{p'-1}[D\phi]_{C^{0,\beta}(B)}.
\end{align*}

We note that, if $c$ satisfies controlled-duality $p$-growth \eqref{eq:controlledDualityGrowth}, (or alternatively, making the slightly weaker assumption that $c^\ast$ is strictly $p'$-convex, that is \eqref{eq:p'convex}) our assumptions are satisfied.
\begin{lemma}
If $c^\ast$ satisfies \eqref{eq:p'convex} and \eqref{ass:elliptic}-\eqref{ass:Cgrowth} hold, then solutions to \eqref{eq:dualEquation} exist and moreover \eqref{eq:energy}-\eqref{eq:regularityPhiR} are satisfied.
\end{lemma}
\begin{proof}
As $c^\ast$ is $p'$-convex, by the direct method, solutions in $\phi\in W^{1,p'}(B)$ exist and are unique up to constant. \eqref{eq:energy} follows immediately from testing the equation with $\phi$ and using Young's inequality, the trace estimate and Poincare's inequality. In order to see \eqref{eq:diff}, test the equations for $\phi$ and $\phi^r$ against $\phi-\phi^r$ and use the $p'$-convexity, duality, trace estimate and Poincare's inequality to see
\begin{align*}
\int_B V(\D\phi,\D\phi^r)\leq& \int_B \langle \nabla c^\ast(\D\phi)-\nabla c^\ast(\D\phi^r),\D\phi-\D\phi^r\rangle\\
=& \int_{\p B} (g-g^r)(\phi-\phi^r)\\
\leq& \|g-g^r\|_{W^{-\frac 1 p,p}(\p B)}\|\phi-\phi^r\|_{W^{\frac 1 p,p'}(\p B)}\\
\lesssim& r^\frac 1 {p}\|g\|_{L^p(\p B)}\|\phi-\phi^r\|_{W^{1,p'}(B)}.
\end{align*}
If $p'\geq 2$, this gives the result immediately after re-arranging. If $p'\leq 2$, we note
\begin{align*}
\int_B V(\D\phi,\D\phi')\geq \left(\int_B |\D\phi-\D\phi'|^{p'}\right)^\frac {2} {p'}\left(\int_B |\D\phi|^{p'}+|\D\phi'|^{p'}\right)^\frac{p'-2}{p'},
\end{align*}
so that the claimed inequality follows after re-arranging and using \eqref{eq:energy}.

\eqref{eq:interior} and \eqref{eq:regularityPhiR} follows from \cite{Lieberman1988} and \cite{Ladyzhenskaya1968}. 
\end{proof}

\section{Preliminaries}

\subsection{General notation}
Throughout, we let $1<p<\infty$. $B_r(x)$ will denote a ball of radius $r>0$ centered at $x\in \R^d$. We further write $B_r=B_r(0)$ and $B=B_1(0)$. $c$ denotes a generic constant that may change from line to line. Relevant dependencies on $\Lambda$, say, will be denoted $c(\Lambda)$. We say $a\lesssim b$, if there exists a constant $c>0$ depending only on $d$, $p$ and $\Lambda$ such that $a\leq c b$.

 Given $\Omega\subset\R^d$, we denote by $[\cdot]_{C^{0,\alpha}}$, the $\alpha$-H\"older-seminorm. Given $\alpha\in(0,\infty)$, $L^p(\Omega)$ and $W^{\alpha,p}(\Omega)$ denote the usual Lebesgue and (fractional) Sobolev spaces. If $\mu$ is a measure on $\R^d$, $\mu\llcorner \Omega$ denotes its restriction to $\Omega$. 

Given $R>0$, we let $\Pi_R(x)=R\frac x {|x|}$ be the projection onto $\p B_R$ and define for every measure $\rho$ on $\R^d$ the projected measure on $\p B_R$, $\hat \rho = \Pi_R \# \rho$, i.e.
$$
\int \xi \d\hat\rho = \int \xi\left(R\frac x {|x|}\right)\d\rho(x).
$$

A set $\Omega\subset\R^d\times \R^d$ is said to be $c$-cyclically monotone if for any $N\in \N$ and any points $(x_1,y_1),\ldots,(x_N,y_N)\in\Omega$, there holds
$$
\sum_{i=1}^N c(x_i-y_i)\leq \sum_{i=1}^N c(x_i-y_{i+1}),
$$
where we identify $y_{N+1}=y_1$.

A function $f\colon \R^d\to \R$ is called $c$-concave if there exists a function $g\colon \R^d\to \R$ such that 
$$
f(x)=\inf_{y\in \R^d} c(x-y)-g(y)
$$
for all $x\in \R^d$.

\subsection{Optimal transportation}
We recall some definitions and facts about optimal transportation, see \cite{Villani2009} for more details. Given a measure $\pi$ on $\R^d\times\R^d$ we denote its marginals by $\pi_1$ and $\pi_2$ respectively. The set of measures on $\R^d\times \R^d$ with marginals $\pi_1$ and $\pi_2$ is denoted $\Pi(\pi_1,\pi_2)$. Given two positive measures with compact support and equal mass $\lambda$ and $\mu$ we define
$$
W_c(\lambda,\mu)=\min_{\pi_1 = \lambda,\pi_2=\mu} \int c(x-y)\d\pi.
$$
While our notation is reminiscent of the Wasserstein distance, and in fact gives the Wasserstein $p$-distance in the case  ${c(x-y) = \lvert x-y\rvert^p}$, in general it is not a distance on measures.
Under our hypothesis, an optimal coupling always exists and moreover a coupling $\pi$ is optimal, if and only if its support is $c$-cyclical monotone. 

Moreover, we note the following triangle-type inequality:
\begin{lemma}\label{lem:triangleInequality}
Let $\e\in(0,1)$. There is $c(\e)>0$ such that for any admissible measures $\mu_1, \mu_2, \mu_3$ it holds that
\begin{align*}
W_c(\mu_1,\mu_3)\leq (1+\e) W_c(\mu_1,\mu_2)+c(\e) W_c(\mu_2,\mu_3).
\end{align*}
\end{lemma}
\begin{proof}
Due to the gluing lemma, see e.g. \cite[Lemma 5.5.]{Santambrogio2015}, there exists $\sigma$, a positive measure on $\R^d\times \R^d\times \R^d$ with marginal $\pi_1$ on the first two variables and marginal $\pi_2$ on the last two variables. Here $\pi_1$ and $\pi_2$ are the optimal couplings between $\mu_1$ and $\mu_2$ and $\mu_2$ and $\mu_3$, respectively, with respect to $W_c$. Set $\gamma$ to be the marginal of $\sigma$ with respect to the first and third variable. Then $\gamma\in \Pi(\mu_1,\mu_3)$. It follows using the convexity of $c$ and the triangle inequality in $L^p(\gamma)$ that for any $t\in(0,1)$,
\begin{align*}
W_c(\mu_1,\mu_3)\leq& \left(\int c(x-z)\d\gamma\right)^\frac 1 p\leq \left(\int t c\left(\frac{x-y} t\right)+(1-t)\left(\frac{y-z}{1-t}\right)\d\gamma\right)^\frac 1 p\\
\leq& \left(\int \left( \left(t c\left(\frac{x-y} t\right)\right)^\frac 1 p+\left((1-t)\left(\frac{y-z}{1-t}\right)\right)^\frac 1 p\right)^p\d\gamma\right)^\frac 1 p\\
\leq& \left(\int t c\left(\frac{x-y} t\right)\d\gamma\right)^\frac 1 p+\left(\int (1-t) c\left(\frac{y-z}{1-t}\right)\d\gamma\right)^\frac 1 p.
\end{align*}
Using \eqref{ass:growth} and recalling the definition of $\gamma$, we deduce
\begin{align*}
W_c(\mu_1,\mu_3)\leq \left(\Lambda^2 t^{1-p}\right)^\frac 1 p W_c(\mu_1,\mu_2)+\left(\Lambda^2 (1-t)^{1-p}\right)^\frac 1 pW_c(\mu_2,\mu_3).
\end{align*}
Choosing $t$ sufficiently close to $1$, this gives the desired estimate.
\end{proof}

We require also the following consequence of Lemma \ref{lem:triangleInequality}.
\begin{corollary}\label{cor:addConstant}
Let $\mu_1,\mu_2$ be measures. Then
\begin{align*}
W_c(\mu_1,\mu_2)\lesssim W_c(\mu_1+\mu_2,2\mu_2).
\end{align*}
\end{corollary}
\begin{proof}
Using Lemma \ref{lem:triangleInequality} and sub-additivity of $W_c$, we note for any $\delta>0$,
\begin{align*}
W_c(\mu_1,\mu_2)\leq& (1+\delta) W_c\left(\mu_1,\frac 1 2 (\mu_1+\mu_2)\right)+c(\delta)W_c\left(\frac 1 2 (\mu_1+\mu_2),\mu_2\right)\\
=& (1+\delta) W_c\left(\frac 1 2 \mu_1,\frac 1 2 \mu_2\right)+c(\delta)W_c\left(\frac 1 2 (\mu_1+\mu_2),\mu_2\right)\\
\leq& \frac{1+\delta} 2 W_c\left(\mu_1,\mu_2\right)+c(\delta)W_c\left(\mu_1+\mu_2,2\mu_2\right).
\end{align*}
Re-arranging gives the result.
\end{proof}

Given $O\subset \R^d$ set $\kappa_{\mu,O}$ to be the generic constant such that $W_{c}(\mu\llcorner O,\kappa_{\mu,O}\d x \llcorner O)$ is well-defined, that is $\kappa_{\mu,O} = \frac{\mu(O)}{\lvert O\rvert}$. If $O=B_R$, we write $\kappa_{\mu,R} = \kappa_{\mu,B_R}$.

It will be convenient to denote $\#_R = (B_R\times \R^d)\cup (\R^d\times B_R)$. We recall the definition of the quantities that we use to measure smallness:
\begin{gather*}
E(R):= \frac 1 {\lvert B_R\rvert} \int_{\#_R} c(x-y)\d \pi\\
D(R):= \frac 1 {\lvert B_R\rvert} W_p^p(\lambda\llcorner B_R,\kappa_{\lambda,R}\d x\llcorner B_R)+\frac{R^p}{\kappa_{\lambda,R}^{p-1}}(\kappa_{\lambda,R}-1)^p \\
\qquad\qquad+\frac 1 {\lvert B_R\rvert} W_p^p(\mu\llcorner B_R,\kappa_{\mu,R}\d x\llcorner B_R)+\frac{R^p}{\kappa_{\mu,R}^{p-1}}(\kappa_{\mu,R}-1)^p.
\end{gather*}

We will find it convenient to work with trajectories $X(t) = t x+(1-t)y$. In this context, it is useful to work on the domain
\begin{align*}
\Omega_R = \{(x,y)\in \#_3\colon \exists t\in[0,1] \text{ s.t. } X(t)\in \overline B_R\}.
\end{align*}
To every trajectory $X\in\Omega$, we associate entering and exiting times of $B_R$:
\begin{gather*}
\sigma_R:= \min\{t\in[0,1]\colon X(t)\in \overline B_R\}\\
\tau_R:= \max\{t\in[0,1]\colon X(t)\in \overline B_R\}.
\end{gather*}
Often, we will drop the subscripts and denote $\Omega = \Omega_R$, $\sigma=\sigma_R$ and $\tau = \tau_R$.
Further, we will need to track trajectories entering and leaving $B_R$. This is achieved through the non-negative measures $f_R$ and $g_R$ concentrated on $\p B_R$ and defined by the relations
\begin{gather}\label{eq:2}
\int\zeta \d f_R = \int_{\Omega \cap \{X(\sigma)\in \p B_R\}} \zeta(X(\sigma))\d \pi,\nonumber\\
\int\zeta \d g_R = \int_{\Omega \cap \{X(\tau)\in \p B_R\}} \zeta(X(\sigma))\d \pi.
\end{gather}
Note that the set of trajectories $\Omega \cap \{X(\sigma)\in \p B_R\}$ implicitly defines a Borel measurable subset of $\R^d\times \R^d$, namely the pre-image under the mapping $(x,y)\to X$, which is continuous from $\R^d\times \R^d$ into $C^0([0,1])$. Thus, the integrals in \eqref{eq:2} are well-defined. We will often use similar observations without further justification.

\subsection{Estimating radial projections}
We record a technical estimate concerning radial projections we will require.
\begin{lemma}\label{lem:projection}
For $R>0$, there exists $1\geq \e(d)>0$ such that for every $g\geq 0$ with $\mathrm{Spt}\, g\subset B_{(1+\e)R}\setminus B_{(1-\e)R}$ we have
\begin{align*}
R^{1-d}\left(\int g\right)^p\lesssim \int_{\p B_R} \hat g^p\lesssim \sup g^{p-1} \int |R-|x||^{p-1}g
\end{align*}
\end{lemma}
\begin{proof}
By scaling we may assume $R=\sup g =1$. The first inequality is then a direct consequence of Jensen's inequality.

For the second inequality, note that if $\e\ll 1$, $\sup_{\p B_1} |\hat g|\ll 1$, since we assume ${\mathrm{Spt}\, g\subset B_{1+\e}\setminus B_{1-\e}}$. Fix $\omega\in \p B_1$ and set $\psi(r)=r^{d-1}g(r\omega)$ for $r>0$. Then we have ${0\leq \psi\leq (1+\e)^{d-1}\leq 2}$ and
$$
\int_0^\infty\psi = \hat g(\omega).
$$
We conclude that for $\omega\in \p B_1$,
$$
\int_0^\infty |1-r|^{p-1} r^{d-1}g(r\omega)\geq \min_{0\leq \tilde\psi\leq 2, \int\tilde\psi = \hat g(\omega)} \int_0^\infty |1-r|^{p-1} \tilde \psi(r)\gtrsim \hat g(\omega).
$$
The last inequality holds, since the minimiser of
$$
\min_{0\leq \tilde\psi\leq 2, \int\tilde\psi = \hat g(\omega)} \int_0^\infty |1-r|^{p-1} \tilde \psi(r)
$$
is given by $2I\left(|r-1|\leq \frac 1 4 \hat g(\omega)\right)$.
\end{proof}


\section{A \texorpdfstring{$L^\infty$}{}-bound on the displacement}\label{sec:Linfty}
A key point in our proof will be that trajectories do not move very much. Since we assume $E(4)\ll 1$, this is evidently true on average. However, we will require to control the length of trajectories not just on average, but in a pointwise sense. We establish this result in this section. In the quadratic case, the proof in \cite{Goldman2020a} relies on the fact that $2$-monotonicity is equivalent to standard monotonicity. In our setting this is not available and we hence provide a different proof.

\begin{lemma}\label{lem:linfty}
Let $1<p<\infty$.
Let $\pi$ be a coupling between two measures $\lambda$ and $\mu$. Assume that $\mathrm{Spt}\,\pi $ is cyclically monotone with respect to $c$-cost and that $E(4)+D(4)\ll 1$.
Then for every $(x,y)\in \mathrm{Spt}\,\pi\cap \#_3$,
we have
\begin{align}\label{eq:main}
\lvert x-y\rvert\lesssim \left(E(4)+D(4)\right)^\frac 1 {p+d}.
\end{align}
As a consequence, for $(x,y)\in \mathrm{Spt}\,\pi$ and $t\in[0,1]$,
\begin{align}\label{eq:bound2}
x\in B_{3} \text{ or } y\in B_3 \Rightarrow (1-t)x+ty\in B_{4}.
\end{align}
\end{lemma}

In the proof of Lemma \ref{lem:linfty} we require the following technical result, which we state independently as we will require it again in the future.
\begin{lemma}\label{lem:C2measures}
Let $1<p<\infty$ and $0< \alpha<1$.
For every $R>0$, $\xi\in C^{0,\alpha}(B_R)$ and $\mu$ supported in $B_R$ with $\mu(B_R)\sim \lvert B_R\rvert$,
\begin{align}\label{eq:taylor}
\left\lvert\int_{B_R}\xi(\d\mu-\kappa_{\mu,R}\d x)\right\rvert\leq& [\xi]_{C^{0,\alpha}(B_R)} W_{c}(\mu,\kappa_{\mu,R}\d x \llcorner B_R)^\frac \alpha p R^\frac{2d(p-\alpha)} p.
\end{align}
In case $\alpha =1$, \eqref{eq:taylor} holds with $C^{0,1}$ replaced by $C^1$. Further, if in addition we have ${\xi\in C^{\lfloor{p-1}\rfloor,p-\lfloor{p-1}\rfloor}(B_R)}$, there is $C>0$ such that 
\begin{align*}
\left \lvert \int_{B_R} \xi(\d\mu-\kappa_{\mu,R}\d x)\right\rvert \leq& C\sum_{i=1}^{\lfloor{p-1}\rfloor}\left(\kappa_{\mu,R}\int \lvert \D^i\xi\rvert^\frac{p}{p-i}\d x\right)^\frac{p-i} {p} W_c(\mu,\kappa_{\mu,R}\d x \llcorner B_R)^\frac i p\\
&\quad + [\xi]_{C^{\lfloor{p-1}\rfloor,p-\lfloor{p-1}\rfloor}} W_c(\mu,\kappa_{\mu,R}\ x \llcorner B_R).
\end{align*}
\end{lemma}

\begin{proof}
Integrate the estimate
$$
|\xi(x)-\xi(y)|\leq [\xi]_{C^{0,\alpha}}|x-y|^\alpha
$$
against an optimal transport plan $\pi$ between $\mu$ and $\kappa_{\mu,R}\d x\llcorner B_R$ to find, 
\begin{align*}
\left|\int_{B_R}\xi(\d\mu-\kappa_\mu\d x)\right|\leq& [\xi]_{C^{0,\alpha}(B_R)} \int_{B_R}|x-y|^\alpha \d \pi.
\end{align*}
Applying H\"older and using \eqref{ass:growth} the result follows.

To obtain the second estimate, we proceed similarly, but start with the estimate
\begin{align*}
\lvert\xi(x)-\xi(y)-\sum_{\lvert \alpha\rvert=1}^{\lfloor{p-1}\rfloor}\D^\alpha \xi(y)\frac{(x-y)^\alpha}{\lvert \alpha\rvert!}\rvert \leq [\xi]_{C^{\lfloor{p-1}\rfloor,p-\lfloor{p-1}\rfloor}(B_R)} \lvert x-y\rvert^{p}.
\end{align*}
The result follows using \eqref{ass:growth} and using H\"older to estimate
\begin{align*}
\int \lvert \D^\alpha \xi(y)\rvert \lvert x-y\rvert^{\lvert \alpha\rvert} \d\pi \leq \left(\kappa_{\mu,R}\int \lvert \D^{\lvert \alpha\rvert}\xi\rvert^\frac{p}{p-\lvert \alpha\rvert}\d x\right)^\frac{p-\lvert \alpha\rvert} p W_c(\mu,\kappa_{\mu,R}\d x \llcorner B_R)^\frac {\lvert \alpha\rvert} p.
\end{align*}
\end{proof}

We proceed to prove Lemma \ref{lem:linfty}.
\begin{proof}[Proof of Lemma \ref{lem:linfty}]
Fix $(x,y)\in \mathrm{Spt}\,\pi\cap \#_3$.  Without loss of generality we may assume that $(x,y)\in B_{3}\times \R^d$.

\textbf{Step 1. Barrier points exist in all directions:} In this step we show that in all directions we may find points $(x',y')\in \mathrm{Spt}\,\pi$ with $x'\approx y'$. To be precise, consider an arbitrary unit vector $\bm n\in \R^d$ and let $r>0$. We show that for any $\bm n$, and all $r\ll 1$, there is $M=M(p,d,\Lambda)>0$ and $(x',y')\in \mathrm{Spt}\,\pi\cap (B_r(x+2r \bm n)\times \R^d)$ such that
\begin{align*}
c(x'-y')\leq \frac {ME(4)} {r^d}.
\end{align*}

Assume, for contradiction, that for any $M>0$, there is $\bm n\in \R^d$ and $r>0$ such that for all $(x',y')\in \mathrm{Spt}\,\pi\cap (B_r(x+2r \bm n)\times \R^d)$, $|x'-y'|\geq \frac {M E(4)}{r^d}$. 
 Let $\eta$ be a non-negative, smooth cut-off supported in $B_r(x+2r \bm n)$ satisfying 
 $$\sum_{i=1}^{\lfloor{p-1}\rfloor} r^i\sup|\D^i\eta|+r^{p}[\eta]_{C^{\lfloor{p-1}\rfloor,p-\lfloor{p-1}\rfloor}}\lesssim 1.
 $$
 Then
\begin{align*}
E(4)\gtrsim \int \int \eta(x)c(x-y)\d\pi(x,y)\geq \int \int \frac{M E(4)}{r^d}\eta(x)\d\pi(x,y)= \frac{M E(4)}{r^d}\int \eta(x) \d\mu(x).
\end{align*}
However, due to Lemma \ref{lem:C2measures} and noting $\kappa_{\mu,4}\sim 1$,
\begin{align*}
\left|\int \eta\d\mu(x)-\kappa_{\mu,4}\int \eta\d x\right|\lesssim \sum_{i=1}^{\lfloor{p-1}\rfloor} r^{\frac {d(p-i)}{p}-i}D(4)^\frac i p+r^{-p} D(4).
\end{align*}
Normalising $\eta$ such that $\int_{B_r(x+2r n)}\eta\d x\sim r^d$, we can guarantee $\kappa_{\mu,4}\int \eta\d x\sim \kappa_{\mu,4} r^d\sim r^d$.
 Ensuring $D(4)\ll r^{p+d}$, so that $\sum_{i=1}^{\lfloor{p-1}\rfloor} r^{\frac {d(p-i)}{p}-i}D(4)^\frac i p+r^{-p} D(4) \ll r^d$, we may thus conclude
\begin{align*}
E(4)\gtrsim \frac{M E(4)}{r^d} r^d = M E(4).
\end{align*}
As $M$ was arbitrary, this is a contradiction.

\textbf{Step 2. Building barriers:} In this step, we show that if we are given points $${(x',y')\in \mathrm{Spt}\,\pi\cap (B_r(x+2r \bm n)\times \R^d)}$$ such that $|x'-y'|\leq \frac {ME(4)} {r^d}$ for some $M=M(p,d,\Lambda)>0$, then there is a cone $C_{x,x'}$ with vertex $x'+r\rho(x'-x)$ for some $\rho=\rho(p,d,\Lambda)>0$, aperture $\alpha=\alpha(p,d,\Lambda)$ and axis $x'-x$ such that $y\not\in C_{x,x'}$.

Without loss of generality, we may assume that $x'-x$ points in the $e_n$ direction. Moreover, considering the cost $c(\cdot)-c(x)$, we may assume that $c(x)=0$. 
Suppose for a contradiction that
$$
y\in C_{x,x'}=x'+\{a\in \R^{d-1}\times \R^+\colon d(a,\Gamma)\leq \alpha (|\overline a-x'|-r \rho)\}
$$
for some $\alpha,\rho>0$ to be determined. Here $\Gamma= \{t(x'-x)\colon t\geq 0\}$ and $\overline a$ denotes the orthogonal projection of a point $a\in \R^{d-1}\times \R^+$ onto $\Gamma$. We want to show that then
\begin{align}\label{eq:contraMonoton}
c(x-y)\geq c(x'-y)+c(x-y').
\end{align}
\eqref{eq:contraMonoton} is a contradiction to the $c$-monotonicity of $\pi$ and hence proves the stated claim.

We note that we may assume $x=0$. Indeed, setting $z = y-x$, $z'=y'-x$ and $\tilde z=x'-x$, \eqref{eq:contraMonoton} becomes
\begin{align*}
c(-z)\geq c(\tilde z-z)+c(-z')
\end{align*}
with $|\tilde z|\leq \frac{M E}{r^d}$ and $z\in C_{0,\tilde z}$, which we recognise as precisely the situation we are in if $x=0$.

\definecolor{uququq}{rgb}{0.25,0.25,0.25}
\begin{figure}[ht]
\centering
\resizebox{0.5\textwidth}{!}{
\begin{tikzpicture}[line cap=round,line join=round,>=triangle 45,x=1.0cm,y=1.0cm]
\clip(-1.66,-5.3) rectangle (13.08,7.48);
\draw [domain=4.0:13.080000000000005] plot(\x,{(-4--1*\x)/2});
\draw [domain=4.0:13.080000000000005] plot(\x,{(--4-1*\x)/2});
\draw(1.56,0) circle (0.42cm);
\draw (6.08,-0.68) node[anchor=north west] {$C_{x,x'}$};
\draw [domain=0.0:13.080000000000005] plot(\x,{(-0-0*\x)/8});
\begin{scriptsize}
\fill [color=uququq] (0,0) circle (1.5pt);
\draw[color=uququq] (0.14,0.26) node {$X$};
\fill [color=black] (8.04,1.5) circle (1.5pt);
\draw[color=black] (8.18,1.76) node {$Y$};
\fill [color=black] (8,0) circle (1.5pt);
\draw[color=black] (8.14,0.26) node {$\overline{Y}$};
\fill [color=black] (1.56,0) circle (1.5pt);
\draw[color=black] (1.74,0.26) node {$X'$};
\fill [color=black] (1.48,0.28) circle (1.5pt);
\draw[color=black] (1.66,0.54) node {$Y'$};
\end{scriptsize}
\end{tikzpicture}
}
\caption{Geometric situation in Step 2.}
\end{figure}


Taking $\rho\geq 4$, we then estimate using \eqref{ass:Cgrowth} and \eqref{ass:elliptic}
\begin{align*}
c(-y)\geq& c(-\overline y)+c(x'-y')-\Lambda U(-y,-\overline y)\\
\geq& \frac{|\overline y|}{|-\overline y+x'|}c(-\overline y+x')+\frac{\lambda|x'|}{|\overline y|}V(-\overline y,0)-\Lambda U(-y,-\overline y)\\
\geq& c(-\overline y+x')+c(x')+\frac{\lambda|x'|}{|\overline y|}V(-\overline y,0)+\frac{\lambda|\overline y-2x'|}{|\overline y-x'|}V(-\overline y+x',0)-\Lambda U(-y,-\overline y)\\
\geq& c(-y+x')+c(-x')-\Lambda U(-y+x',-\overline y+x')+\frac{\lambda|x'|}{|\overline y|}V(-\overline y,0)\\
&\quad+\frac{\lambda|\overline y-2x'|}{|\overline y-x'|}V(-\overline y+x',0)-\Lambda U(-y,-\overline y)\\
\geq& c(-y+x')+c(-y')-\Lambda U(-y+x',-\overline y+x')+\frac{\lambda|x'|}{|\overline y|}V(-\overline y,0)\\
&\quad+\frac{\lambda|\overline y-2x'|}{|\overline y-x'|}V(-\overline y+x',0)-\Lambda U(-y,-\overline y)-\Lambda U(-x',-y')\\
\end{align*}

In particular, it suffices to show
\begin{align}\label{eq:sufficient}
&c(x'-y')+ \frac{\lambda|x'|}{|\overline y|}V(-\overline y,0)+\frac{\lambda|\overline y-2x'|}{|\overline y-x'|}V(-\overline y+x',0)\nonumber\\
\geq& \Lambda U(-y,-\overline y)+\Lambda U(-y+x',-\overline y+x').
\end{align}
We note that,  if $\rho\geq 8$, $|\overline y-2x'|\geq \frac 1 2|\overline y|$. Then we can estimate
\begin{align*}
\frac{\lambda|x'|}{|\overline y|}V(-\overline y,0)+\frac{\lambda|\overline y-2x'|}{|\overline y-x'|}V(-\overline y+x',0)
=& \lambda(|x'| |\overline y|^{p-1}+|\overline y-2 x'| |\overline y-x'|^{p-1})\\
\gtrsim &  |\overline y|^{p}.
\end{align*}
Further, if $E(4)\leq \e r^{d+1}$,
\begin{align*}
&\Lambda U(-y,-\overline y)+\Lambda U(-y+x',-\overline y+x')+\Lambda U(-x',-y')\\
\leq& 2\Lambda (2|y|)^{p-1}|y-\overline y|+\Lambda \left(|x'|+|y'|\right)^{p-1} |x'-y'|\\
\lesssim& |y|^{p-1}\alpha |\overline y-x'|+\frac{M E}{r^d} (2r)^{p-1}\\
\lesssim& \alpha|\overline y|^p+\e |\overline y|^p .
\end{align*}
Thus choosing $\alpha$, $\e>0$, sufficiently small, we find \eqref{eq:sufficient} holds, proving our claim.

\textbf{Step 3. Proving the $L^\infty$-bounds:}
Choose $r=c (E(4)+D(4))^{\frac 1 {p+d}}$. For sufficiently large choice of $c>0$ and selecting $c(d)$ directions $n_i$, by Step 1 and Step 2 we obtain points 
$${(x',y')\in \mathrm{Spt}\,\pi\cap (B_r(x+2 r n_i)\times \R^d)}$$
and cones $(C_{x,x_i'})_{i\leq c(d)}$ with vertices $x_i'+r\rho(x_i'-x)$, aperture $\alpha$ and axis $x_i'-x$ such that for some $c(\alpha)>0$,
$${y\not\in \cup C_{y_i}} \text{ and }\R^d\setminus B_{(\rho+c(\alpha))r}(x)\subset\cup C_{y_i}.
$$ In particular, we have
\begin{align*}
|y-x|\leq (\rho+c(\alpha))r\lesssim (E+D)^\frac 1 {p+d},
\end{align*}
that is \eqref{eq:main}.

\eqref{eq:bound2} is a direct consequence of \eqref{eq:main}, concluding the proof.
\end{proof}

We record two consequences of Lemma \ref{lem:linfty} we will use later.
\begin{corollary}\label{cor:linfty}
Under the assumptions of Lemma \ref{lem:linfty}, it holds that
\begin{gather*}
\int_2^3 \int_{\Omega \cap \{\exists t\in[0,1]\colon X(t)\in \p B_R)\}} c(x-y)\d \pi\d R\lesssim (E(4)+D(4))^{1+\frac 1 {p+d}},\\
\int_2^3 \int_{\Omega} I(\{\exists t\in[0,1]\colon X(t)\in \p B_R)\})\d \pi\d R \lesssim (E(4)+D(4))^\frac 1 {p+d}.
\end{gather*}
\end{corollary}
\begin{proof}
We use Lemma \ref{lem:linfty} to deduce there is $C>0$ such that
\begin{align*}
&\int_2^3 \int_{\Omega \cap \{\exists t\in[0,1]\colon X(t)\in \p B_R)} c(x-y)\d \pi\d R\\
\leq& \int_2^3 \int_{(B_{7/2}\setminus B_{3/2})\times (B_{7/2}-B_{3/2})}I(\{\lvert \lvert x\rvert-R\rvert\leq C(E(4)+D(4))^\frac 1 {p+d}\}) c(x-y)\d \pi\d R\\
\lesssim& (E(4)+D(4))^\frac 1 {p+d} \int_{\#_4} c(x-y)\d \pi\\
\lesssim& (E(4)+D(4))^{1+\frac 1 {p+d}}.
\end{align*}

Further, again using Lemma \ref{lem:linfty}, there is $C>0$ such that,
\begin{align*}
&\int_2^3 \int_\Omega I(\{\exists t\in[0,1]\colon X(t)\in \p B_R\})\d \pi\d R\\
\leq& \int_2^3 \pi(\Omega \cap \{\lvert X(0)-R\rvert \leq C(E(4)+D(4))^\frac 1 {p+d}\})\d R\\
=& \int_2^3 \mu(\{\lvert \lvert x\rvert-R\rvert \leq C(E(4)+D(4))^\frac 1 {p+d})\d R\\
\leq& \int_{B_{7/2}\setminus B_{3/2}\times B_{7/2}\setminus B_{3/2}} \int I(\lvert \lvert x\rvert -R\rvert \leq C(E(4)+D(4))^\frac 1 {p+d})\d R\d \mu\\
\lesssim& (E(4)+D(4))^\frac 1 {p+d} \mu(B_4)\\
\lesssim& (E(4)+D(4))^\frac 1 {p+d}.
\end{align*}

\end{proof}


\section{A localisation result}
In order to prove the approximation of the optimal transport plan $\pi$ by a $c^\ast$-harmonic map, we will need to use optimality in a localised way. In general, it is not true that the cost of $\pi$ localised to $\#_R$ is well approximated by the transportation cost between the localised measures $\lambda \llcorner B_R$ and $\mu \llcorner B_R$. However, if we take into account the entering points $f_R$ and exiting points $g_R$, this approximation holds as we show in the next lemma.
\begin{lemma}\label{lem:localisation}
Suppose $\pi\in \Pi(\lambda,\mu)$ minimises \eqref{eq:problem}. Let $R\in[2,3]$. Then for any $\tau,\delta>0$, there is $\e>0$ such that if $E(4)+D(4)\leq \e$, then
\begin{align*}
\int_\Omega c(x-y)\d \pi\leq (1+\delta) W_c(\lambda \llcorner B_R+f_R,\mu\llcorner B_R+ g_R)+\tau \left(E(4) + D(4)\right)
\end{align*}
\end{lemma}
\begin{proof}
Introduce the weakly continuous family of probability measures $\{\lambda_z\}_{z\in \p B_R}$ such that
\begin{align*}
\int_{\Omega \cap \{X(\sigma)\in \p B_R\}} \zeta(x,X(\sigma))\pi(\d x\d y) = \int_{\p B_R}\int \zeta(x,z)\lambda_z(\d x)f_R(\d z)
\end{align*}
for any test function $\zeta$ on $\R^d\times \R^d$. Likewise, introduce $\{\mu_w\}_{w\in \p B_R}$ via
\begin{align*}
\int_{\Omega \cap \{X(\tau)\in \p B_R\}}\zeta(X(\tau),y)\pi(\d x\d y) = \int_{\p B_R}\int \zeta(w,y)\mu_w(\d y)g_R(\d w).
\end{align*}
Let $\overline\pi$ be an optimal plan for $W_c(\lambda\llcorner B_R+f_R,\mu\llcorner B_R+g_R)$. Define a competitor $\tilde \pi$ for $\pi$ by requiring the following formula to hold for any test function $\zeta$ on $\R^d\times \R^d$,
\begin{align}\label{eq:tildepi}
&\int \zeta(x,y)\tilde\pi(\d x \d y) \nonumber\\
=& \int_{\Omega^c}\zeta(x,y)\d \pi(x,y) + \int_{B_R\times B_R} \zeta(x,y)\overline\pi(\d x\d y)+ \int_{\p B_R\times B_R}\int \zeta(x,y)\lambda_z(\d x)\overline\pi(\d z \d y)\nonumber\\
&+ \int_{B_R\times \p B_R}\zeta(x,y)\mu_w(\d y)\overline\pi(\d x \d w) + \int_{\p B_R\times \p B_R} \int\int \zeta(x,y)\mu_w(\d y)\lambda_z(\d x)\overline\pi(\d z \d w)\nonumber\\
=& I + II + III + IV + V.
\end{align}

In order to see that $\tilde \pi\in \Pi(\lambda,\mu)$, by symmetry it suffices to check that the first marginal is $\lambda$. Hence test \eqref{eq:tildepi} against $\zeta(x)$. We begin by noting that due to the definition of $\mu_w$ and using that $\overline\pi$ is supported in $\overline B_R$,
\begin{align*}
II + IV  = \int_{B_R\times \R^d}\zeta(x)\overline\pi(\d x\d y) = \int_{B_R}\zeta(x)\mu(\d x) = \int_{\Omega \cap \{X(\sigma)\in B_R)\}} \zeta(x)\pi(\d x \d y).
\end{align*}
Similarly, using also the definition of $f_R$,
\begin{align*}
III + V =& \int_{\p B_R\times \R^d} \int \zeta(x)\lambda_z(\d x)\overline\pi(\d z \d y) = \int_{\p B_R} \zeta(z)f_R(\d z) \\
=& \int_{\Omega \cap \{X(\sigma)\in \p B_R\}} \zeta(x)\pi(\d x \d y).
\end{align*}
In particular, we have shown
\begin{align*}
\int \zeta(x)\tilde \pi(\d x \d y) = \int \zeta(x)\pi(\d x\d y) = \int \zeta(x)\lambda(\d x)
\end{align*}
as desired.

Using optimality of $\pi$ in the form
\begin{align*}
\int_{\Omega} c(x-y)\d \pi + \int_{\Omega^c} c(x-y)\d \pi \leq \int c(x-y)\d\tilde\pi
\end{align*}
and testing \eqref{eq:tildepi} against $\zeta(x,y)= c(x-y)$, we learn
\begin{align*}
&\int_\Omega c(x-y)\d\pi \\
\leq& \int_{B_R\times B_R} c(x-y)\overline\pi(\d x \d y) + \int_{\p B_R\times B_R} \int c(x-y)\lambda_z(\d x)\overline\pi(\d z \d y) \\
&\quad +\int_{B_R\times \p B_R} c(x-y)\mu_w(\d y)\overline\pi(\d x \d w) \\
&\qquad+ \int_{\p B_R\times \p B_R} \int \int c(x-y)\mu_w(\d y)\lambda_z(\d x) \overline\pi(\d z \d w)\\
=& \int_{B_R\times B_R} f_1 \overline\pi(\d x \dy ) + \int_{\p B_R\times B_R} f_2 \overline\pi(\d z\d y) + \int_{B_R\times \p B_R} f_3 \overline \pi(\d x \d w) \\
&\quad+ \int_{\p B_R\times \p B_R} f_4 \overline \pi(\d z \d w)
\end{align*}
As in the proof of Lemma \ref{lem:triangleInequality}, for any $\delta>0$, there is $C_\delta>0$ such that for any $x,y,z$,
\begin{align*}
c(x-z)\leq (1+\delta) c(x-y) + C_\delta c(y-z).
\end{align*}
Using this in combination with the fact that $\lambda_z$, $\mu_w$ are probability measures we deduce
\begin{align*}
f_2 \leq& (1+\delta)c(z-y) + C(\delta)\tilde f_2, \quad f_3 \leq (1+\delta)c(x-w) + C(\delta)\tilde f_3\\
f_4\leq& (1+\delta)c(z-w)+C(\delta) \tilde f_4,
\end{align*}
where
\begin{gather*}
\tilde f_2(z,y) = \int c(x-z)\lambda_z(\d x),\quad \tilde f_3(x,w) = \int c(w-y)\mu_w(\d y),\\
 \tilde f_4(z,w) = \tilde f_2(z,y) + f_3(x,w).
\end{gather*}
In particular, we deduce
\begin{align*}
&\int_\Omega c(x-y)\d \pi \\
\leq& (1+\delta)\int_{\overline B_R\times \overline B_R}c(x-y)\overline\pi(\d x \d y)+ C(\delta)\int_{\p B_R\times B_R} \tilde f_2 \overline\pi(\d z\d y) \\
&\quad+ C(\delta) \int_{B_R\times \p B_R} \tilde f_3\overline\pi(\d x \d w)+ C(\delta) \int_{\p B_R\times \p B_R} \tilde f_4\overline\pi(\d z \d w)\\
=& (1+\delta)\int_{\overline B_R\times \overline B_R}c(x-y)\overline\pi(\d x \d y) + 2 C(\delta)\int_{\p B_R\times \R^d} \int c(x-z)\lambda_z(\d x)\overline\pi(\d z\d y) \\
&\quad + 2C(\delta)\int_{\R^d\times \p B_R} c(w-y)\mu_y(\d y)\overline\pi(\d x \d w)\\
=& (1+\delta)\int_{\overline B_R\times \overline B_R}c(x-y)\overline\pi(\d x \d y) + 2 C(\delta)\int_{\p B_R\times \R^d} \int c(x-z)\lambda_z(\d x)f_R(\d z) \\
&\quad + 2C(\delta)\int_{\R^d\times \p B_R} c(w-y)\mu_y(\d y)g_R(\d w)\\
=& (1+\delta)\int_{\overline B_R\times \overline B_R}c(x-y)\overline\pi(\d x \d y) + 2 C(\delta)\int_{\Omega \cap \{X(\sigma)\in \p B_R\}} c(x-X(\sigma))\d \pi \\
&\quad + 2C(\delta)\int_{\Omega \cap \{ X(\tau) \in \p B_R\}} c(X(\tau)-y)\d \pi\\
\end{align*}
In order to obtain the second to last line we used the admissibility of $\overline\pi$. Now note on the one hand, that due to optimality of $\overline \pi$,
\begin{align*}
\int_{\overline B_R\times \overline B_R} c(x-y)\d\overline\pi = W_c(\lambda \llcorner B_R + f_R,\mu\llcorner B_R + g_R).
\end{align*}
On the other hand, on $\Omega \cap \{X(\sigma)\in \p B_R)\} \cap \{X(\tau)\in \p B_R\}$, for some $\rho_1, \rho_2\geq 0$ with $\rho_1 + \rho_2\leq 1$, due to convexity of $c$ and $c(0)=0$,
\begin{align*}
c(x-X(\sigma)) + c(X(\tau)-y) = c(\rho_1(x-y)) + c(\rho_2(x-y)) \leq c(x-y).
\end{align*}
Thus, we have shown
\begin{align*}
&\int_\Omega c(x-y)\d\pi\\
\leq& (1+\delta) W_c(\lambda \llcorner B_R + f_R,\mu\llcorner B_R+g_R)+C(\delta) \int_{\Omega \cap (\{X(\sigma\in \p B_R\}\cup \{X(\tau)\in \p B_R\})} c(x-y)\d \pi\\
=& (1+\delta) W_c(\lambda \llcorner B_R + f_R,\mu\llcorner B_R+g_R)+C(\delta) \int_{\Omega \cap \{\exists t\in[0,1]\colon X(t)\in \p B_R\}} c(x-y)\d \pi\\
\leq& (1+\delta) W_c(\lambda \llcorner B_R + f_R,\mu\llcorner B_R+g_R)+C(\delta) (E(4)+D(4))^{1+\frac 1 {p+d}}.
\end{align*}
To obtain the last line, we used Corollary \ref{cor:linfty}. Choosing $\e$ sufficiently small the result follows.
\end{proof}

\section{Approximating the boundary data}
Before we can implement the $c^\ast$-harmonic approximation, we face another problem. Lemma \ref{lem:localisation} suggests that we should impose Neumann boundary data $f_R$ and $g_R$. However $f_R$ and $g_R$ are not sufficiently regular for this to make sense. Hence, we need to approximate them. 

\begin{lemma}\label{lem:approximation}
Let $\tau>0$. There is $\e>0$ such that if $E(4)+D(4)\leq \e$, then for every $R\in[2,3]$ there exist non-negative functions $\overline f_R$, $\overline g_R$ such that
\begin{align*}
W_c(f_R,\overline f_R)+W_c(g_R,\overline g_R)\lesssim \tau E(4)+ D(4)\\
\int_2^3 \int_{\p B_R} \overline g_R^p+\overline f_R^p \d R\lesssim E(4)+D(4)
\end{align*}
\end{lemma}
\begin{proof}
By symmetry it suffices to focus on the terms involving $g$.

Fix $R\in[2,3]$. Let $\overline\pi$ be optimal for $W_c(\mu\llcorner B_4,\kappa_{\mu,4} \d z\llcorner B_4)$. Extend $\overline \pi$, which is supported on $B_4\times B_4$ by the identity to $\R^d\times \R^d$. Note that this extension, still denoted $\overline\pi$ is admissible for $W_c(\mu,\kappa_{\mu,4}\d z \llcorner B_4+\mu \llcorner B_4^c)$. Further due to the definition of $D$, $W_c(\mu\llcorner B_4,\kappa_{\mu,4} \d z \llcorner B_4)\leq D(4)$. Introduce the family of probability measures $\{\overline \pi(\cdot|y)\}_{y\in \R^d}$ via asking that for every test function $\zeta$
\begin{align*}
\int \zeta(y,z)\overline \pi(\d z|y)\mu(\d y) = \int \zeta(y,z)\overline \pi(\d y \d z).
\end{align*}
Then define $\tilde \pi$ on $\R^d\times \R^d\times \R^d$ by the formula
\begin{align*}
\int \zeta(x,y,z)\tilde \pi(\d x \d y \d z) = \int \int \zeta(x,y,z)\overline\pi(\d z|y)\pi(\d x \d y)
\end{align*}
valid for any test function $\zeta$. Note that with respect to the $(x,y)$ variables $\tilde \pi$ has marginal $\pi$, while with respect to the $(y,z)$ variables $\tilde \pi$ has marginal $\overline\pi$. Extend a trajectory $X$ in $\Omega$ in a piecewise affine fashion by setting for $t\in[1,2]$,
\begin{align*}
X(t) = (t-1)z+(2-t)y.
\end{align*}
Note that the distribution $g^\prime$ of the endpoint of those trajectories that exit $\overline B_R$ during the time interval $[0,1]$ is given by
\begin{align}\label{eq:1}
\int\zeta \d g^\prime = \int_{\Omega \cap \{X(\tau)\in \p B_R\}} \zeta(z)\tilde \pi(\d x \d y \d z).
\end{align}
Note that due to Lemma \ref{lem:linfty}, $y=X(1)\in B_4$ for any trajectory $X$ that contributes to \eqref{eq:1}. Since $\overline \pi(B_4,B_4^c) = 0$, we deduce that also $z=X(2)\in B_4$ and hence that $g^\prime$ is supported in $B_4$. In particular, we may estimate for any $\zeta\geq 0$, using the admissibility of $\overline\pi$ for $W_c(\mu,\kappa_{\mu,4}\d z\llcorner B_4+\mu\llcorner B_4^c)$,
\begin{align*}
\int \zeta \d g^\prime \leq \int_{\{z\in B_4\}} \zeta(z)\overline \pi(\d y \d z) = \kappa_{\mu,4}\int_{B_4}\zeta.
\end{align*}
This shows that $g^\prime$ has a density, still denoted $g^\prime$, satisfying $g^\prime \leq \kappa_{\mu,4}$ and allows us to conclude the construction of $\overline g_R$ by defining
\begin{align*}
\int \zeta \d \overline g_R = \int \zeta\left(R \frac z {\lvert z\rvert}\right)g^\prime(\d z).
\end{align*}

We now turn to establishing the claimed estimates for $\overline g_R$. Note that, directly from the definitions of $\tilde \pi$, $g^\prime$ and $\overline g_R$, an admissible plan for $W_c(g_R,\overline g_R)$ is
\begin{align*}
\int_{\Omega \cap \{X(\tau)\in \p B_R\}} \zeta\left(X(\tau),R\frac z {\lvert z\rvert}\right)\d \tilde \pi.
\end{align*}
In particular, using \eqref{ass:growth},
\begin{align*}
W_c(g_R,\overline g_R)\lesssim \int_{\Omega \cap \{X(\tau)\in \p B_R\}} c\left(X(\tau)-R\frac z {\lvert z\rvert}\right)\d\tilde \pi\lesssim \int_{\Omega \cap \{X(\tau)\in \p B_R\}} \left\lvert X(\tau)-R\frac z {\lvert z\rvert}\right\rvert^p\d\tilde \pi.
\end{align*}
Noting that $\lvert X(\tau)-R\frac z {\lvert z\rvert}\rvert \leq 2 \lvert X(\tau)-z\rvert$, we deduce
\begin{align*}
\left\lvert X(\tau)-R\frac z {\lvert z\rvert}\right\rvert^p \lesssim \lvert x-y\rvert^p+\lvert y-z\rvert^p.
\end{align*}
Thus, we deduce using \eqref{ass:growth} and Corollary \ref{cor:linfty},
\begin{align*}
W_c(g_R,\overline g_R) \lesssim \int_{\Omega \cap \{\exists t\in[0,1]\colon X(t)\in \p B_R\}} c(x-y)\d \pi+ D(4)\lesssim E(4)^{1+\frac 1 {p+d}}+D(4).
\end{align*}
Choosing $\e$ sufficiently small, the first estimate holds.

\noindent Noting $\sup g^\prime\leq \kappa_{\mu,4}\lesssim 1$, in order to prove the second inequality, it suffices to prove
\begin{align*}
\int_2^3 \int \lvert R-\lvert x\rvert\rvert^{p-1} \d g^\prime \lesssim E(4)+D(4)
\end{align*}
and to apply Lemma \ref{lem:projection}. The condition on the support of $g$ in Lemma \ref{lem:projection} applies due to Lemma \ref{lem:linfty}.
Note that by definition of $g^\prime$,
\begin{align*}
\int \lvert R-\lvert x\rvert\rvert^{p-1}\d g^\prime =& \int_{\Omega \cap \{X(\tau)\in \p B_R\}} \lvert \lvert z\rvert-R\rvert^{p-1}\tilde\pi(\d x\d y \d z)\\
\lesssim& \int_{\{y\in B_4\}\cap \{\min_{[0,1]}\lvert X\rvert\leq R\leq \max_{[0,1]} \lvert X\rvert\}} \lvert x-y\rvert^{p-1}+\lvert y-z\rvert^{p-1}\tilde \pi(\d x \dy \d z).
\end{align*}
In order to obtain the second line, we observed that since $\lvert X(\tau)\rvert = R$, it holds that ${\lvert \lvert z\rvert-R\rvert \leq \lvert x-y\rvert + \lvert y-z\rvert}$. In addition we noted that, since $X(\tau)\in \p B_R$, we have ${\min_{[0,1]}X\leq R\leq \max_{[0,1]} X}$ and $X(1)\in B_4$ due to Lemma \ref{lem:linfty}. Integrating over $R$, this gives
\begin{align*}
&\int_2^3 \int \lvert \lvert z\rvert -R\rvert^{p-1} \d g^\prime\d R \\
\lesssim& \int_{\{y\in B_4\}} (\max_{[0,1]} X-\min_{[0,1]} X)\left(\lvert x-y\rvert^{p-1}+\lvert y-z\rvert^{p-1}\right)\tilde \pi(\d x \d y \d z)\\
\leq& \int_{\{y\in B_4\}} \lvert x-y\rvert\left(\lvert x-y\rvert^{p-1}+\lvert y-z\rvert^{p-1}\right)\tilde \pi(\d x \d y \d z)\\
\lesssim& \int_{\{y\in B_4\}} \lvert x-y\rvert^p\pi(\d x \d y) + \int \lvert y-z\rvert^p \overline\pi(\d y \d z)\\
\leq & E(4)+D(4).
\end{align*}
The second-to last line was obtained applying Young's inequality. This concludes the proof.
\end{proof}

\section{Restricting the data}

We also need to control $D(R)$, while at the moment we only control $D(4)$. Unfortunately, this does not follow immediately from the definition but requires a technical proof utilising ideas of the previous sections. The outcome of these considerations is the following lemma:
\begin{lemma}\label{lem:dataRestriction}
For any non-negative measure $\mu$ there is $\e>0$ such that if $D(4)\leq \e$, then
\begin{align*}
\int_2^3 \left(W_{c}(\mu\llcorner B_R,\kappa_{\mu,R}\d x \llcorner B_R)+\frac 1 {\kappa_{\mu,R}}(\kappa_{\mu,R}-1)^p\right)\d R\lesssim D(4).
\end{align*}
\end{lemma}
\begin{proof}
In this lemma $\pi$ will denote the optimal transference plan for the problem ${W_{c}(\mu\llcorner B_4,\kappa_{\mu,R}\d x\llcorner B_4)}$. 
Define the measures $0\leq f^\prime\leq \kappa_{\mu,4}$ on $\overline B_R$ and $0\leq g^\prime\leq \kappa_{\mu,4}$ on $\overline B_4\setminus B_R$, which record where exiting and entering trajectories end up:
\begin{align*}
\int \zeta \d f^\prime\coloneqq \int_{\Omega\cap \{X(0)\not\in B_R\}\cap \{X(1)\in B_R\}}\zeta(X(1))\d\pi \\
\int \zeta \d g^\prime\coloneqq \int_{\Omega\cap \{X(0)\in B_R\}\cap \{X(1)\not\in B_R\}}\zeta(X(1))\d\pi.
\end{align*}
Introduce the mass densities
\begin{align*}
\kappa_f = \frac{f^\prime(\R^D)} {\lvert B_R\rvert}\leq \kappa_{\mu,R} \qquad \kappa_g = \frac{g^\prime(\R^d)}{\lvert B_R\rvert}.
\end{align*}
We use Lemma \ref{lem:triangleInequality} to deduce
\begin{align*}
&W_c(\mu \llcorner B_R,\kappa_{\mu,R}\d x \llcorner B_R)\\
\lesssim& W_c(\mu\llcorner B_R,\kappa_{\mu,4}\d x\llcorner B_R- f^\prime+g^\prime) + W_c(\kappa_{\mu,4}\d x\llcorner B_R-f^\prime + g^\prime,(\kappa_{\mu,4}-\kappa_f)\d x \llcorner B_R+g^\prime)\\
&\quad+ W_c((\kappa_{\mu,4}-\kappa_f)\d x \llcorner B_R+g^\prime,\kappa_{\mu,R}\d x \llcorner B_R)\\
=& I + II + III.
\end{align*}
Restricting $\pi$ to trajectories that start in $B_R$ gives an admissible plan for $I$, that is,
\begin{align*}
I=W_c(\mu\llcorner B_R,\kappa_{\mu,4}\d x\llcorner B_R-f^\prime + g^\prime)\leq W_c(\mu\llcorner B_4,\kappa_{\mu,4}\d x \llcorner B_4)\leq D(4).
\end{align*}
Regarding $II$, we begin by noting that using the sub-additivity of $W_c$, we have
\begin{align*}
&W_c(\kappa_{\mu,4}\d x \llcorner B_R-f^\prime + g^\prime,(\kappa_{\mu,4}-\kappa_f)\d x \llcorner B_R+g^\prime)\\
\lesssim& W_c(\kappa_{\mu,4}\d x \llcorner B_R-f^\prime,(\kappa_{\mu,4}-\kappa_f)\d x \llcorner B_R)+W_c(g^\prime,g^\prime)\\
=& W_c(\kappa_{\mu,4}\d x \llcorner B_R-f^\prime,(\kappa_{\mu,4}-\kappa_f)\d x\llcorner B_R ).
\end{align*}
Since this term will be estimated in the same way as $III$ but is slightly more tricky, we first turn to $III$.

In order to estimate $III$, introduce the projection $\overline g$ of $g^\prime$ onto $\p B_R$, that is 
\begin{align*}
\int \zeta \d \overline g = \int \zeta\left(\frac{R x}{\lvert x\rvert}\right)g^\prime(\d x).
\end{align*}
Using Lemma \ref{lem:triangleInequality}, we deduce
\begin{align*}
III\lesssim& W_c((\kappa_{\mu,4}-\kappa_f)\d x\llcorner B_R+\overline g,\kappa_{\mu,R}\d x \llcorner B_R)+W_c(\overline g, g^\prime).
\end{align*}
We claim that
\begin{align*}
W_c((\kappa_{\mu,4}-\kappa_f)\d x \llcorner B_R+\overline g,\kappa_{\mu,R}\d x \llcorner B_R)\lesssim \int_{\p B_R} \overline g^2.
\end{align*}
Indeed, an admissible density-flux pair $(\rho,j)$ for the Benamou-Brenier formulation is given by
\begin{align*}
\begin{cases}
	\rho_t = (\kappa_{\mu,4}\d x\llcorner-\kappa_f+t\kappa_g) +(1-t)\overline g\\
	j_t = \nabla c^\ast(\D\phi) \d x \llcorner B_R,
\end{cases}
\end{align*}
where $\phi$ solves
\begin{align*}
\begin{cases}
-\Div \nabla c^\ast(\D\phi) = \kappa_g \quad&\text{ in } B_R\\
\nu \cdot \nabla c^\ast(\D\phi) = \overline g &\text{ on } \p B_R.
\end{cases}
\end{align*}
We find, writing $s=(\kappa_{\mu,4}-\kappa_f+t\kappa_g)+(1-t)\overline g$, for any $\zeta$ supported in $B_R$,
\begin{align*}
\int \zeta \d j_t -\int c^\ast(\zeta)\d \rho_t =& \int_{B_R} \zeta\cdot \nabla c^\ast(\D\phi)-c^\ast(\zeta)s\d x\\
\leq& \int_{B_R}\int_0^1 s c \left(\frac 1 s\nabla c^\ast(\D\phi)\right)\d t \d x
\end{align*}
To obtain the second line, we used the Fenchel-Young inequality. Assuming $\lvert s-1\rvert \ll 1$ for now, using \eqref{ass:growth} and \eqref{ass:Cgrowth} it is straightforward to see that
\begin{align*}
\int_{B_R}\int_0^1 s c \left(\frac 1 s \nabla c^\ast(\D\phi)\right)\d t \d x \lesssim \int_{B_R} c\left(\nabla c^\ast(\D\phi)\right)\d x \lesssim \int_{\p B_R} \overline g^p.
\end{align*}
To obtain the last inequality, we used \eqref{eq:alternativeEnergy}. Taking into account Lemma \ref{lem:linfty} and Lemma \ref{lem:projection}, we find
\begin{align*}
\int_{\p B_R} \overline g^p\lesssim \int \lvert R-\lvert x\rvert\rvert^{p-1} \d g^\prime.
\end{align*}
Noting $\left\lvert R\frac x {\lvert x\rvert}-x\right\rvert = \lvert \lvert x\rvert-R\rvert$ and that $g^\prime$ is supported in $\overline B_4\setminus B_R$, we obtain using \eqref{ass:growth}
\begin{align*}
W_c(\overline g,g^\prime)\lesssim \int \lvert \lvert x \rvert-R\rvert^{p-1}\d g^\prime.
\end{align*}
Proceeding exactly as in Lemma \ref{lem:dataRestriction} we obtain
\begin{align*}
\int \lvert R-\lvert x\rvert\rvert^{p-1}\d g^\prime\lesssim  D(4).
\end{align*}
Since $\kappa_{\mu,R} = \kappa_{4,R}-\kappa_f+ \kappa_g$ to deduce that $\lvert s-1\rvert \ll 1$ and to conclude the estimate of $III$, it suffices to show $\kappa_f^p+\kappa_g^p\lesssim D(4)$. By symmetry it suffices to consider $\kappa_g$. By definition of $\overline g$ and Young's inequality, we find
\begin{align*}
\kappa_g^p = \frac {\overline g(\R^d)^p}{\lvert B_R\rvert^p}\leq \frac{\lvert \p B_R\rvert^{p-1}}{\lvert B_R\rvert^p}\int_{\p B_R}\overline g^p.
\end{align*}
This concludes the estimate for $III$. It remains to finish the estimate of $II$, by estimating
\begin{align*}
W_c(\kappa_{\mu,4}\d x \llcorner B_R-f^\prime, (\kappa_{\mu,4}-\kappa_f)\d x \llcorner B_R).
\end{align*}
We want to proceed exactly as we did in the estimate for $III$, the only delicate issue being that we do not have $\kappa_{\mu,4}\d x \llcorner B_R-f^\prime \geq c>0$. However, this can be remedied by using Corollary \ref{cor:addConstant} to deduce
\begin{align*}
&W_c(\kappa_{\mu,4}\d x \llcorner B_R-f^\prime, (\kappa_{\mu,4}-\kappa_f)\d x \llcorner B_R)\\
\lesssim& W_c((2\kappa_{\mu,4}-\kappa_f)\d x \llcorner B_R-f^\prime, 2(\kappa_{\mu,4}-\kappa_f)\d x \llcorner B_R).
\end{align*}
We can now proceed using the same argument as for $III$ to conclude
\begin{align*}
W_c(\kappa_{\mu,4}\d x \llcorner B_R-f^\prime, (\kappa_{\mu,4}-\kappa_f)\d x \llcorner B_R)\lesssim D(4).
\end{align*}
This completes the proof.
\end{proof}

\section{The \texorpdfstring{$c^\ast$}{}-harmonic approximation result}\label{sec:proof}
The goal of this section is to prove the $c^\ast$-harmonic approximation result. In order to define the approximation, note that in light of Lemma \ref{lem:approximation}, given $\tau>0$, we fix $R\in(3/2,2)$ such that there exist non-negative $\overline f_R$, $\overline g_R$ such that
\begin{align*}
W_c(f_R,\overline f_R)+W_c(g_R,\overline g_R)\lesssim \tau E(4)+D(4)\\
\int_{\p B_R} \overline f_R^p+\overline g_R^p \lesssim E(4)+D(4).
\end{align*}
Then let $\phi$ be a solution with $\int_{B_R}\phi\d x = 0$ of
\begin{align}\label{eq:phi}
\begin{cases}
-\Div \nabla c^\ast(\D\phi) = c_R \quad&\text{ in } B_R\\
\nabla c^\ast(\D\phi)\cdot\nu = \overline g_R-\overline f_R &\text{ on } \p B_R.
\end{cases}
\end{align}
where $c_R = \lvert B_R\rvert^{-1}\left(\int_{\p B_R} \overline g_R-\overline f_R\right)$ is the constant so that \eqref{eq:phi} is well-posed.
We emphasize that while we do not make the dependence explicit in our notation, $\phi$ depends on the choice of radius $R$.

Finally due to Lemma \ref{lem:dataRestriction}, we will further assume throughout this section that $D(R)\lesssim D(4)$.

With this notation in place, we state a precise version of our main result.
\begin{theorem}\label{thm:mainBody}
For every $0<\tau$, there exist positive constants $\e(\tau),C(\tau)>0$ such that if $E(4)+D(4)\leq \e(\tau)$, then there exists $R\in(2,3)$ and $\phi$ solving \eqref{eq:phi} such that
\begin{align*}
\int_{\#_1} c(x-y-\nabla c^\ast(\D\phi)) \leq \tau E(4) + C(\tau) D(4).
\end{align*}
\end{theorem}
The proof will be a direct consequence of the lemmata we prove in the following subsections.


\subsection{Quasi-orthogonality}
The key observation in order to prove Theorem \ref{thm:mainBody} is contained in the following elementary lemma.
\begin{lemma}\label{lem:orthog}
For any $\pi\in \Pi(\lambda,\mu)$ and $\phi$ continuously differentiable in $\overline B_R$, there is $c(p,\Lambda)$ such that we have
\begin{align*}
&c(p,\Lambda)\int_\Omega \int_\sigma^\tau V\left(\dot X(t)-\nabla c^\ast(\nabla\phi(X(t))\right)\d t\d\pi \\
\leq& \int_\Omega c(x-y)\d\pi - \int_{B_R}c(\nabla ^\ast(\D\phi))\d x - \int_\Omega \int_\sigma^\tau \langle \dot X(t)-\nabla c^\ast(\D \phi(X(t)),\D\phi(X(t))\rangle \d t\d\pi \\
&\quad+ \int_{B_R}c(\nabla c^\ast(\D\phi))\d x-\int_{\Omega}\int_\sigma^\tau c(\nabla c^\ast(\D\phi(X(t)))\d t\d\pi.
\end{align*}
\end{lemma}
\begin{proof}
We apply \eqref{ass:smoothness} with $x = \dot X$ and $y=\nabla c^\ast(\D\phi(X(t)))$. Noting that we have ${\nabla c(\nabla c^\ast(\D\phi)) = \D\phi}$ and $\int_\Omega \int_\sigma^\tau c(\dot X(t))\d t\d\pi \leq \int_\Omega c(x-y)\d\pi$ this gives the desired result.
\end{proof}
\subsection{Error estimates}
We would like to apply Lemma \ref{lem:orthog} with $\phi$ solving \eqref{eq:phi}. However note that $\overline g_R$ and $\overline f_R$ will in general not be sufficiently smooth in order to ensure that $\phi$ is $C^1(\overline B_R)$. Thus, we approximate them using mollification. To be precise, let $0<r\ll 1$ and denote by $\overline f_R^r$ and $\overline g_R^r$, respectively, the convolution with a smooth convolution kernel (on $\p B_R)$ at scale $r$ of $\overline f_R$ and $\overline g_R$. Set $\phi^r$ to be the solution with $\int_{B_R} \phi^r\d x = 0$ of
\begin{align}\label{eq:phiMollif}
\begin{cases}
-\Div \nabla c^\ast(\D\phi^r) = c^r \quad&\text{ in } B_R\\
\nabla c^\ast(\D\phi^r)\cdot\nu = \overline g_R^r-\overline f_R^r &\text{ on } \p B_R.
\end{cases}
\end{align}
Here $c^r= \lvert B_R\rvert^{-1}\left(\int_{\p B_R} \overline g_R^r-\overline f_R^r\right)$ is the constant such that \eqref{eq:phiMollif} is well-posed.

We begin by showing that the left-hand sides of the estimate in Lemma \ref{lem:orthog} are close when evaluated for $\phi$ and $\phi^r$.
\begin{lemma}\label{lem:lefthandSide}
For every $0<\tau$ there exists $\e(\tau)$ and $C(\tau), r_0(\tau)>0$, such that if it holds that ${E(4)+D(4)\leq \e(\tau)}$ and $0<r\leq r_0$, then there exists $R\in[2,3]$ such that if $\phi$ solves \eqref{eq:phi} and $\phi^r$ solves \eqref{eq:phiMollif}, then
\begin{align*}
&\int_{\Omega_1} \int_{\sigma_1}^{\tau_1} V\left(\dot X(t)-\nabla c^\ast(\D\phi(X(t))\right)\d t\d\pi-\int_{\Omega_1} \int_{\sigma_1}^{\tau_1} V\left(\dot X(t)-\nabla c^\ast(\D\phi^r(X(t))\right)\d t\d\pi\\
\lesssim& \tau (E(4)+D(4)).
\end{align*}
\end{lemma}
\begin{proof}
Write $\xi(x) = \nabla c^\ast(\D\phi(x))$, $\xi^r(x) = \nabla c^\ast(\D\phi^r(x))$. We focus on the case $p\leq 2$. The case $p>2$ follows by similar arguments, but is easier.
In light of \eqref{eq:VDiff}, \eqref{eq:c1growthDual} and using H\"older, we find
\begin{align*}
&\left\lvert\int_{\Omega_1}\int_{\sigma_1}^{\tau_1} V\left(\dot X(t)-\xi(X(t))\right)\d t\d\pi-\int_\Omega \int_\sigma^\tau V\left(\dot X(t)-\xi^r(X(t))\right)\d t\d\pi\right\rvert\\
\lesssim& \int_{\Omega_1}\int_{\sigma_1}^{\tau_1} \lvert \xi(X(t))-\xi^r(X(t))\rvert\left(\lvert \dot X(t)+\lvert \xi(X(t))\rvert + \lvert \xi^r(X(t))\rvert\right)^{p-1}\d t\d\pi\\
\lesssim& \int_{\Omega_1}\int_{\sigma_1}^{\tau_1} \lvert \D\phi(X(t))-\D\phi^r(X(t))\rvert \left(\lvert \D\phi(X(t))\rvert + \lvert \D\phi^r( X(t))\rvert\right)^{p^\prime-1}\d t\d\pi\\
\leq& \left(\int_{\Omega_1}\int_{\sigma_1}^{\tau_1}\lvert \D\phi(X(t))-\D\phi^r(X(t))\rvert^{p^\prime}\d t\d\pi\right)^\frac 1 {p^\prime}\\
&\quad \times\left(\int_{\Omega_1}\int_{\sigma_1}^{\tau_1}\lvert \D\phi (X(t))\rvert^{p^\prime}+\lvert \D\phi^r (X(t))\rvert^{p^\prime}\d t\d\pi\right)^\frac 1 p.
\end{align*}
Note that due to Lemma \ref{lem:linfty}, if $X\in\Omega_1$, then $X(0)\in B_{3/2}$. Thus using \eqref{eq:interior} and \eqref{eq:regularityPhiR},
\begin{align*}
&\Big\lvert\int_{\Omega_1}\int_{\sigma_1}^{\tau_1}\lvert \D\phi(X(t)-\D\phi^r(X(t))\rvert^{p^\prime}\d t\d\pi-\int_{\Omega_1}\int_{\sigma_1}^{\tau_1}\lvert \D\phi(X(0)-\D\phi^r(X(0))\rvert^{p^\prime}\d t\d\pi\Big\rvert\\
\lesssim& \left([\D\phi]_{C^{0,\beta}(B_{3/2})}+[\D\phi^r]_{C^{0,\beta}(B_{3/2})}\right)\left(\sup_{B_{3/2}}\lvert \D \phi\rvert+\lvert \D \phi^r\rvert\right)^{p^\prime-1}\int_{\Omega_1}\int_{\sigma_1}^{\tau_1} \lvert X(t)-X(0)\rvert^\beta \d t\d\pi\\
\lesssim& c(r)(E(4)+D(4)) \pi(\Omega_1)(E(4)+D(4))^\frac \beta {p+d}\lesssim c(r)(E(4)+D(4))^{1+\frac \beta {p+d}}.
\end{align*}
In order to obtain the last line, we used Lemma \ref{lem:linfty}. Moreover, using Lemma \ref{lem:C2measures} and \eqref{eq:diff},
\begin{align*}
&\int_{\Omega_1}\int_{\sigma_1}^{\tau_1} \lvert \D \phi(X(0))-\D \phi^r(X(0))\rvert^{p^\prime}\d t\d\pi\leq \int_{B_{3/2}} \lvert \D \phi(x)-\D \phi^r(x)\rvert^{p^\prime}\d\mu\\
\leq& \Big\lvert \int_{B_{3/2}} \lvert \D \phi(x)-\D \phi^r(x)\rvert^{p^\prime}(\d\mu-\kappa_{\mu,R}\d x)\Big\rvert+\kappa_{\mu,R}\int_{B_R}\lvert \D \phi(x)-\D \phi^r(x)\rvert^{p^\prime}\d x\\
\lesssim& \left(\sup_{B_{3/2}} \lvert \D \phi\rvert^{p^\prime-1}+\lvert \D\phi^r\rvert^{p^\prime-1}\right)\left([\D \phi]_{C^{0,\beta}(B_{(1+R)/2})}+[\D^r\phi]_{C^{0,\beta}(B_{(1+R)/2})}\right)\\
&\quad \times W_{c}(\mu\llcorner B_R,\kappa_{\mu,R}\d x\llcorner B_R)^\frac \beta p+ r(E(4)+D(4))\\
\lesssim& c(r)(E(4)+D(4))^{1+\frac \beta p} + r(E(4)+D(4)).
\end{align*}
Arguing similarly, that is first replacing $X(t)$ with $X(0)$, at the cost of making an error of size $c(r)\left(E(4)+D(4)\right)^{1+\frac \beta {p+d}}$, and then replacing $\d \mu$ with $\d x$, making an error $c(r)(E(4)+D(4))^{1+\frac \beta p}+r(E(4)+D(4))$, we find
\begin{align*}
&\int_{\Omega_1}\int_{\sigma_1}^{\tau_1}\lvert \nabla \phi(X(t))\rvert^{p^\prime}+\lvert \nabla \phi^r(X(t))\rvert^{p^\prime}\d t\d\pi\\
\lesssim& r(E(4)+D(4))+ c(r)(E(4)+D(4))^{1+\frac \beta {p+d}}.
\end{align*}
Collecting estimates and choosing $r$ as well as $\e(\tau)$ sufficiently small, the desired result follows.
\end{proof}

We now turn to estimating each of the three terms on the right-hand side of the estimate in Lemma \ref{lem:orthog} in turns. We will see that the second and third term are errors that arise from the approximation of the boundary data and from passing to the perspective of trajectories, respectively. Accordingly, estimating them will be essentially routine. In contrast, estimating the first term requires us to contrast an appropriate competitor to $\pi$.
\begin{lemma}\label{lem:firstTerm}
For every $0<\tau$ there exists $\e(\tau),C(\tau),r_0(\tau)>0$ such that if it holds that ${E(4)+D(4)\leq \e(\tau)}$ and $0<r\leq r_0$, then there exists $R\in[2,3]$ such that if $\phi^r$ solves \eqref{eq:phiMollif}, then
\begin{align*}
&\int_\Omega c(x-y)\d\pi - \int_{B_R} c(\nabla c^\ast(\D\phi^r))\d x \lesssim \tau E(4)+D(4).
\end{align*}
\end{lemma}
\begin{proof}
We note, in the case $p\leq 2$, using \eqref{ass:Cgrowth}, \eqref{eq:c1growthDual} and H\"older,
\begin{align*}
&\int_{B_R} c(\nabla c^\ast(\D\phi^r))-c(\nabla c^\ast(\D\phi))\d x\\
\lesssim& \int_{B_R} \lvert \nabla c^\ast(\D\phi^r)-\nabla c^\ast(\D\phi)\rvert(\lvert \nabla c^\ast(\D\phi^r)\rvert+\lvert \nabla c^\ast(\D\phi)\rvert)^{p-1}\d x\\
\lesssim& \int_{B_R} \lvert \D\phi-\D\phi^r\rvert \left(\lvert D\phi\rvert+\lvert \D\phi^\prime\rvert\right)^{p^\prime-1}\\
\lesssim& \|\D\phi-\D\phi^r\|_{L^{p^\prime}(B_R)}\left(\|\D\phi\|_{L^{p^\prime}(B_R)}+\|\D\phi^r\|_{L^{p^\prime}(B_R)}\right)^{p^\prime-1}\\
\lesssim& r^\frac 1 {p^\prime} (E(4)+D(4)).
\end{align*}
To obtain the last line, we used \eqref{eq:energy} and \eqref{eq:diff}.  In case $p\geq 2$ a similar estimate holds by the same argument. 
Due to Lemma \ref{lem:localisation} and Corollary \ref{cor:linfty},
\begin{align*}
\int_\Omega c(x-y))\d\pi\leq W_c(\lambda\llcorner B_R+f_R,\mu\llcorner B_R+g_R)+2\int_{\Omega \cap \{\exists t\in[0,1]\colon X(t)\in \p B_R \} }c(x-y)\d\pi\\
\leq W_c(\lambda\llcorner B_R+f_R,\mu\llcorner B_R+g_R)+c\left(\tau E(4)+D(4)\right).
\end{align*}
In particular, combining the previous two estimates and choosing $r$ sufficiently small, it suffices to prove
\begin{align*}
&W_c(\lambda \llcorner B_R +f_R,\mu\llcorner B_R+g_R)- \int_{B_R} c(\nabla c^\ast(\D\phi))\d x \lesssim \tau E(4)+D(4).
\end{align*}
Using Lemma \ref{lem:triangleInequality}, we obtain for $\delta\in(0,1)$ to be fixed
\begin{align*}
&W_c(\lambda \llcorner B_R+f_R,\mu\llcorner B_R+g_R)\\
\leq& (1+\delta) W_c(\kappa_{\lambda,R}\d x\llcorner B_R+\overline f_R,\kappa_{\mu,R}\d x\llcorner B_R+\overline g_R)\\
&\quad+ c(\delta) \left(W_c(\lambda\llcorner B_R,\kappa_{\lambda,R} \d x\llcorner B_R)+W_c(\mu\llcorner B_R,\kappa_{\mu,R} \d x\llcorner B_R)\right)\\
&\qquad +c(\delta)\left(W_c(f_R,\overline f_R)+W_c(g_R,\overline g_R)\right).
\end{align*}
Noting that due to the definition of $D$ and our choice of $R$,
\begin{align*}
&W_c(\lambda\llcorner B_R,\kappa_{\lambda,R} \d x\llcorner B_R)+W_c(\mu\llcorner B_R,\kappa_{\mu,R} \d x\llcorner B_R)+W_c(f_R,\overline f_R)+W_c(g_R,\overline g_R)\\
\lesssim& \tau E(4)+D(4),
\end{align*}
we claim that for some $C>0$,
\begin{align}\label{claim:benamou}
W_c(\kappa_{\lambda,R} \d x\llcorner B_R+\overline f_R,\kappa_{\mu,R} \d x\llcorner B_R+\overline g_R)\leq (1+CD(4)^\frac 1 p) \int_{B_R} c(\nabla c^\ast(\D\phi))\d x.
\end{align}
Collecting estimates, choosing first $\delta$ and $r$ small, then $\e$ small, once \eqref{claim:benamou} is established, the proof is complete.

Establishing \eqref{claim:benamou} is easy to do using the Benamou-Brenier formulation. For $t\in[0,1]$ introduce the non-singular, non-negative measure
\begin{align*}
\rho_t = t(\kappa_{\mu,R} \d x \llcorner B_R+\overline f_R)+(1-t)(\kappa_{\lambda,R} \d x\llcorner B_R+\overline g_R),
\end{align*}
and the vector-valued measure
\begin{align*}
j_t = \nabla c^\ast(\D\phi)\d x\llcorner B_R.
\end{align*}
Note that \eqref{eq:phiMollif} can be rewritten as
\begin{align*}
\frac{\d}{\d t} \int \zeta \d\rho_t = \int \nabla \zeta \cdot \d j_t
\end{align*}
for all test functions $\zeta$. Set
\begin{align*}
\int c\left(\frac{\d j_t}{\d\rho_t}\right)\d \rho_t = \sup_{\zeta\in C^0_c(\R^d)} \left\{\int\zeta \d j_t - \int c^\ast(\xi)\d \rho_t\right\}
\end{align*}
The Benamou-Brenier formula gives
\begin{align*}
W_c(\kappa_{\lambda,R} \d x \llcorner B_R+\overline f_R,\kappa_{\mu,R}\d x\llcorner B_R+\overline g_R)\leq \int_0^1 \int c\left(\frac{\d j_t}{\d \rho_t}\right)\d \rho_t \d t.
\end{align*}
Since $j_t$ is supported in $B_R$ it suffices to consider $\zeta$ supported in $B_R$. Then by definition and the Fenchel-Young inequality for any $s>0$,
\begin{align*}
\int \zeta \d j_t-\int c^\ast(\zeta)\d\rho_t =& \int_{B_R}\zeta \cdot \nabla c^\ast(\D\phi)- c^\ast(\zeta) (t\kappa_{\mu,R}+(1-t)\kappa_{\lambda,R})\d x\\
\leq& \int_{B_R}s c^\ast(\zeta)+s c\left(\frac 1 s \nabla c^\ast(\D\phi)\right)-c^\ast(\zeta)(t\kappa_{\mu,R}+(1-t)\kappa_{\lambda,R})\d x.
\end{align*}
Choosing $s = t\kappa_{\mu,R}+(1-t)\kappa_{\lambda,R}$ and integrating in $t$, we deduce
\begin{align*}
W_c(\kappa_{\mu,R}\d x\llcorner B_R+\overline f_R,\kappa_{\lambda,R} \d x\llcorner B_R+\overline g_R)\leq \int_{B_R}\int_0^1 s c\left(\frac{\nabla c^\ast(\D\phi))}{s}\right)\d t\d x.
\end{align*}
Now
\begin{align*}
\int_{B_R}\int_0^1 s c\left(\frac{\nabla c^\ast(\D\phi))}{s}\right)\d t\d x\leq& \int_{B_R} c(\nabla c^\ast(\D\phi))\d x + \int_{B_R}\int_0^1 (s-1) c(\nabla c^\ast(\D\phi))\ d x\d t\\
&\quad + \int_{B_R}\int_0^1 s \left(c\left(\frac{\nabla c^\ast(\D\phi)} s\right)-c(\nabla c^\ast(\D\phi))\right)\d t \d x.
\end{align*}
Note that 
\begin{align*}
\lvert s-1\rvert \lesssim D(4)^\frac 1 p.
\end{align*}
Further using \eqref{ass:Cgrowth} and \eqref{ass:growth},
\begin{align*}
&\int_{B_R}\int_0^1 s \left(c\left(\frac{\nabla c^\ast(\D\phi)} s\right)-c(\nabla c^\ast(\D\phi))\right)\d t \d x\\
\lesssim& \int_{B_R}\int_0^1 \left\lvert 1- \frac 1 s\right\rvert \left(1+\frac 1 s\right)^{p-1} \lvert \nabla c^\ast(\D\phi)\rvert^{p} \d t\d x
\lesssim D(4)^\frac 1 p \int_{B_R} c(\nabla c^\ast(\D\phi))\d x.
\end{align*}
Thus the proof of \eqref{claim:benamou} is complete.
\end{proof}

We turn to the second term on the right-hand side of Lemma \ref{lem:orthog}. This term will be small due to the definition of $\phi^r$.
\begin{lemma}\label{lem:secondTerm}
For every $0<\tau$ there exists $\e(\tau),C(\tau),r_0(\tau)>0$ such that if it holds that ${E(4)+D(4)\leq \e(\tau)}$ and $0<r\leq r_0$, then there exists $R\in[2,3]$ such that if $\phi^r$ solves \eqref{eq:phiMollif}, then
\begin{align*}
&\int_\Omega \int_{\sigma}^\tau \langle \dot X(t)-\nabla c^\ast(\D\phi^r(X(t)),\D\phi^r(X(t))\rangle\d t\d\pi \lesssim \tau E(4)+D(4).
\end{align*}
\end{lemma}
\begin{proof}
Note that $\frac{\d}{\d t} \phi^r(X(t)) = \langle \dot X(t),\nabla \phi^r(X(t))\rangle$. Thus, since $\pi\in \Pi(\lambda,\mu)$ as well as using the definition of $f_R$ and $g_R$,
\begin{align*}
\int_\Omega \int_\sigma^\tau \langle \dot X(t),\D\phi^r(X(t))\rangle\d t\d\pi =& \int_\Omega \phi^r(X(\tau)-\phi^r(X(\sigma))\d\pi\\
=& \int_{\Omega\cap\{X(\tau)\in \p B_R\}} \phi^r(X(\tau))\d \pi+\int_{\{x\in B_R\}} \phi^r(x)\d \pi\\
&\quad-\int_{\Omega \cap\{X(\sigma)\in \p B_R\}}\phi^r(X(\sigma))\d\pi - \int_{\{y\in B_R\}}\phi^r(y)\d\pi\\
=& \int_{B_R}\phi^r \d(\mu-\lambda) + \int_{\p B_R} \phi^r \d(g_R-f_R).
\end{align*}
On the other hand, as in Lemma \ref{lem:lefthandSide}, at cost of an error $c(r)(E(4)+D(4))^{1+\frac \beta {p+d}}$, we may replace $\D\phi^r(X(t))$ with $\D\phi^r(x)$ in the expression
\begin{align*}
\int_\Omega \int_\sigma^\tau \langle \nabla c^\ast(\D\phi^r(X(t)),\D\phi^r(X(t))\rangle \d t\d\pi
\end{align*}
and $\int_\Omega \int_\sigma^\tau\d t\d\pi$ with $\int_{B_R}\d x$ at the cost of a further error $c(r)(E(4)+D(4))^{1+\frac \beta p}$. Thus it suffices to consider
\begin{align*}
\int_{B_R} \langle \nabla c^\ast(\D\phi),\D\phi\rangle \d x = c^r\int_{B_R} \phi^r \d x+\int_{\p B_R} \phi^r\d\left(\overline g^r_R-\overline f^r_R\right).
\end{align*}
Collecting estimates, we have shown
\begin{align*}
&\int_\Omega \int_{\sigma}^\tau \langle \dot X(t)-\nabla c^\ast(\D\phi^r(X(t)),\D\phi^r(X(t))\rangle\d t\d\pi \\
\lesssim& \int_{B_R} \phi^r \d(\mu-\lambda-c^r)+\int_{\p B_R} \phi^r \d(g_R-\overline g_R^r+f_R-\overline f_R^r)+c(r) \left(E(4)+D(4)\right)^{1+\frac \beta {p+d}} \\
=& I + II + III.
\end{align*}
We find using Lemma \ref{lem:C2measures}, Young's inequality, \eqref{eq:regularityPhiR} and \eqref{eq:energy},
\begin{align*}
I\leq& \lvert \int_{B_R} \phi^r \d(\mu-\kappa_{\mu,R}-\lambda+\kappa_{\lambda,R})\rvert + \lvert \int_{B_R} \phi^r \d(\kappa_{\mu,R}-\kappa_{\lambda,R}-c^r)\rvert\\
\lesssim& \sup_{B_R}\lvert \D\phi^r\rvert \left(W_{c}(\lambda\llcorner B_R,\lambda_{\mu,R}\d x\llcorner B_R)^\frac 1 p+W_{c}(\mu\llcorner B_R,\kappa_{\mu,R}\d x\llcorner B_R)^\frac 1 p\right)\\
&\quad+\rvert \kappa_{\mu,R}-\kappa_{\lambda}-c^r\rvert \|\phi^r\|_{\LL^p(B_R)} \\
\lesssim& \tau E(4)+ c(\tau,r) D(4)+c(\tau)\lvert \kappa_{\mu,R}-\kappa_{\lambda,R}-c^r\rvert^{p^\prime}.
\end{align*}
Noting that as $r\to 0$, 
$$c^r=\lvert B_R\rvert^{-1} \left(g^r(\p B_R)-f^r(\p B_R)\right)\to \lvert B_R\rvert^{-1}\left(g(\p B_r)-f(\p B_R)\right) = \kappa_{\mu,R}-\kappa_{\lambda,R},
$$
 we deduce, choosing $r_0$ sufficiently small, that
\begin{align*}
I =\int_{B_R} \phi^r \d(\mu-\lambda-c^r)\lesssim \tau E(4)+D(4).
\end{align*}
Finally consider $II$. By symmetry it suffices to consider the terms involving $g$. We estimate, denoting by $(\phi^r)^r$ convolution of $\phi^r$ with the convolution kernel used to construct $g_R^r$,
\begin{align*}
\int_{\p B_R} \phi^r \d(g_R-\overline g_R^r) = \int_{\p B_R} (\phi^r)^r-\phi^r\d\overline g_R+\int_{\p B_R}\phi^r \d(\overline g_R-g_R).
\end{align*}
Now note that a standard mollification argument shows
\begin{align*}
\int_{\p B_R}\phi^r \d(g_R-\overline g_R^r)\lesssim r^\frac 1 p\|\D\phi^r\|_{\LL^{p^\prime}(B_R)}\|\overline g_R\|_{\LL^p(\p B_R)}\lesssim r^\frac 1 p (E(4)+D(4)),
\end{align*}
and that moreover using Lemma \ref{lem:C2measures} and Young's inequality,
\begin{align*}
\int_{\p B_R} \phi^r(\d\overline g_R-g)\lesssim [\D_{\mathrm{tan}}\phi^r]_{C^{0,\beta}(\p B_R)} W{c}(\overline g_R,\overline g)^\frac 1 p\lesssim c(r)(\tau E(4)+D(4)).
\end{align*}
Thus, collecting estimates and first choosing $r_0$ sufficiently small, then $\e$ small, we conclude the proof.
\end{proof}

We next estimate the third term on the right-hand side of the estimate in Lemma \ref{lem:orthog}.
\begin{lemma}\label{lem:thirdTerm}
For every $0<\tau$ there exists $\e(\tau),C(\tau),r_0(\tau)>0$ such that if it holds that ${E(4)+D(4)\leq \e(\tau)}$ and $0<r\leq r_0$, then there exists $R\in[2,3]$ such that if $\phi^r$ solves \eqref{eq:phiMollif}, then
\begin{align*}
&\int_{B_R}c(\nabla c^\ast(\D\phi^r))\d x-\int_\Omega \int_{\sigma}^\tau c(\nabla c^\ast(\D\phi^r(X(t)))\d t\d\pi \lesssim \tau E(4)+D(4).
\end{align*}
\end{lemma}
\begin{proof}
Set $\xi = c(\nabla c^\ast(\D\phi^r))$. Then
\begin{align*}
&\int_{B_R} c(\nabla c^\ast(\D\phi^r))\d x-\int_\Omega \int_\sigma^\tau c(\nabla c^\ast(\D\phi^r(X(t))\d t\d\pi\\
=&(1-\kappa_{\mu,R})\int_{B_R}\xi\d x+\left(\kappa_{\mu,R} \int_{B_R}\xi\d x-\int_{B_R\times \R^d} \xi \d\pi\right)\\
&\quad+\int_{B_R\times \R^d}\xi\d\pi-\int_\Omega\int_\sigma^\tau \xi(X(t))\d t\d\pi\\
=& I + II + III.
\end{align*}
Using \eqref{eq:alternativeEnergy} and Young's inequality, we find
\begin{align*}
I \leq D(4)^\frac 1 p (E(4)+D(4))\lesssim \tau E(4) + D(4).
\end{align*}
Employing Lemma \ref{lem:C2measures} we deduce
\begin{align*}
II\lesssim \|\xi\|_{C^{0,\beta}(\overline B_R)}W_{c}(\mu\llcorner B_R,\kappa_{\mu,R}\llcorner B_R)^\frac \beta p.
\end{align*}
It is straightforward to check that $\|\xi\|_{C^{0,\beta}(\overline B_R)}\lesssim \|\D\phi^r\|_{C^{0,\beta}(\overline B_R)}\left(\sup_{\overline B_R}\D\phi^r\right)^{p^\prime-1}$, so that using \eqref{eq:regularityPhiR} and Young's inequality,
\begin{align*}
II\lesssim c(r)(E(4)+D(4))D(4)^\frac \beta p\lesssim \tau E(4)+D(4).
\end{align*}
In order to estimate $III$, we first find
\begin{align*}
&I(X(0)\in B_R)\xi(X(0))-I(X(t)\in B_R)\xi(X(t))\\
\leq& I(\exists s\in [0,1]\colon X(s)\in \p B_R, X(0)\in \overline B_R)\xi(X(0))\\
&\quad+ I(\forall s\in[0,1]\; X(s)\in B_R)\left(\xi(X(0))-\xi(X(t)\right).
\end{align*}
Thus, using also \eqref{eq:regularityPhiR} and Jensen's inequality,
\begin{align*}
III\leq&\int_0^1\int I(X(0)\in B_R)\xi(X(0))-I(X(t)\in B_R)\xi(X(t))\d \pi\d t\\
\leq& \sup_{\overline B_R}\lvert \xi\rvert \int_0^1\int I(\exists s\in[0,1]\colon X(s)\in \p B_R, X(0)\in \p B_R)\d\pi\d t\\
&\quad + \|D\xi\|_{C^{0,\beta}(\overline B_R)}\int_0^1 \int I(\forall s\in[0,1]\; X(s)\in B_R)\lvert X(t)-X(0)\rvert^\beta\d t\d\pi\\
\lesssim& c(r)(E(4)+D(4))\pi(\exists t\in[0,1]\colon X(t)\in \p B_R)+c(r)(E+D)\int_\Omega \lvert x-y\rvert^\beta \d\pi\\
\lesssim& c(r) (E(4)+D(4))^{1+\frac 1 {p+d}}+c(r)(E(4)+D(4))^{1+\frac \beta p}.
\end{align*}
In order to obtain the last line we used Corollary \ref{cor:linfty}. Collecting estimates and choosing first $r_0$ small, then $\e$ small, we conclude the proof.
\end{proof}

\subsection{Proof of Theorem \ref{thm:mainBody}}
We are now ready to prove Theorem \ref{thm:mainBody}.
\begin{proof}[Proof of Theorem \ref{thm:mainBody}]
Applying Lemma \ref{lem:orthog} to $\phi^r$ and collecting the output of Lemma \ref{lem:lefthandSide}, Lemma \ref{lem:firstTerm}, Lemma \ref{lem:secondTerm} and Lemma \ref{lem:thirdTerm}, we have shown that for any $0<\tau$, there is $\e,C>0$ such that if $E(4)+D(4)\leq \e$, then
\begin{align*}
\int_{\Omega_1} \int_{\sigma_1}^{\tau_1}V\left(\dot X(t)-\nabla c^\ast(\D \phi(X(t))\right)\d t\d\pi \leq \tau E(4)+C D(4).
\end{align*}
Arguing as in Lemma \ref{lem:lefthandSide}, we may replace $\D(\phi(X(t)))$ by $\D(\phi(X(0)))$ at the cost of an error of size $\left(E(4)+D(4)\right)^{1+\frac \beta {p+d}}$. Noting that $V(z-\nabla c^\ast(\D\phi(x))$ is a convex function of $z$, we employ Jensen's inequality to deduce
\begin{align*}
\int_{\#_1}V\left(x-y-\nabla c^\ast(\D\phi(x))\right)\d \pi\leq \int_{\Omega_1}\int_0^1 V\left(\dot X(t)-\nabla c^\ast(\D\phi(X(0))\right)\d t\d\pi.
\end{align*}
Now using \eqref{eq:CgrowthDual} and \eqref{eq:interior}, as well as Corollary \ref{cor:linfty},
\begin{align*}
&\int_{\Omega_1\cap (\{\sigma_1>0\} \cup \{\tau_1<1\})}\int_{\sigma_1}^{\tau_1} V\left(\dot X(t)-\nabla c^\ast(\D\phi(X(0))\right)\d t\d\pi \\
\lesssim& \int_{\Omega_1\cap (\{\sigma_1>0\} \cup \{\tau_1<1\})} \lvert \dot X(t)\rvert^p+\lvert \D\phi(X(0)))\rvert^{p^\prime}\\
\lesssim& (E(4)+D(4)) \pi(\Omega_1\cap (\{\sigma_1>0\} \cup \{\tau_1<1\}))\\
\lesssim& (E(4)+D(4))^{1+\frac 1 {p+d}}.
\end{align*}
Thus, we conclude
\begin{align*}
\int_{\#_1}V(x-y-\nabla c^\ast(\D\phi(x)))\d \pi\lesssim \tau E(4)+D(4).
\end{align*}
This concludes the proof in the case $p\geq 2$. In the case $p\leq 2$, an application of H\"older's inequality combined with \eqref{ass:growth} concludes the proof.
\end{proof}
\bibliographystyle{plain}
%\bibliography{../Refs/bibtex/pqboundary}
\bibliography{../../bibtex/OptimalTransport}
\end{document}