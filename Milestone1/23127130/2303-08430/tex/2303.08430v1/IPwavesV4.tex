\documentclass{amsart}
\usepackage{amsmath,amssymb}
\usepackage{graphicx,tikz,mathtools}
%\usepackage[pdftex]{hyperref}
%\hypersetup{
%	colorlinks=true,
%	linkcolor=blue,
%	citecolor=blue,
%	filecolor=blue,
%	urlcolor=blue,
%}
\newtheorem{theorem}{Theorem}[section]
\newtheorem{proposition}[theorem]{Proposition}
\newtheorem{corollary}[theorem]{Corollary}
\newtheorem{conjecture}[theorem]{Conjecture}
\newtheorem{lemma}[theorem]{Lemma}



%\usepackage{mathptmx}
\usepackage{mathrsfs}

\newcommand{\sB}{\mathscr{B}}
\newcommand\refl{{\mathrm{r}}}

\newcommand\hX{\widehat{X}}
\newcommand\hGa{\widehat{\Ga}}

\DeclareMathAlphabet{\mathcal}{OMS}{cmsy}{m}{n}
\newcommand\sA{\mathscr{A}}

\theoremstyle{definition}
\newtheorem{definition}[theorem]{Definition}
\newtheorem{assumption}{Assumption}
\newtheorem{hypothesis}[theorem]{Hypothesis}
\newtheorem{example}[theorem]{Example}
\newtheorem*{assumptions}{Assumptions}
%\theoremstyle{problem}
\theoremstyle{remark}
\newtheorem{remark}[theorem]{Remark}


%\numberwithin{equation}{section}

% Greek letters (lowercase)
\newcommand{\al}{\alpha}
\newcommand{\be}{\beta}
\newcommand{\de}{\delta}
\newcommand{\ep}{\varepsilon}
\newcommand{\vep}{\varepsilon}
\newcommand{\ga}{\gamma}
\newcommand{\ka}{\kappa}
\newcommand{\la}{\lambda}
\newcommand{\om}{\omega}
\newcommand{\si}{\sigma}
\newcommand{\te}{\theta}
\newcommand{\vp}{\varphi}
\newcommand{\ze}{\zeta}
%
% Greek letters (uppercase)
\newcommand{\De}{\Delta}
\newcommand{\Ga}{\Gamma}
\newcommand{\La}{\Lambda}
\newcommand{\Si}{\Sigma}
\newcommand{\Om}{\Omega}
\newcommand{\Omc}{{\Omega^c}}


% Tilde
\newcommand{\tA}{\widetilde{A}}
\newcommand{\tC}{\widetilde{C}}
\newcommand{\tPsi}{\widetilde{\Psi}}
\newcommand{\Th}{\tilde{h}}
\newcommand{\tH}{\widetilde{H}}
\newcommand{\tcH}{\widetilde{\cH}}
\newcommand{\tK}{\tilde{K}}
\newcommand{\tS}{\tilde{S}}
\newcommand{\tc}{\tilde{c}}
\newcommand{\tih}{\tilde{h}}
\newcommand{\tl}{\tilde{l}}
\newcommand{\ts}{\tilde{s}}
\newcommand{\tbs}{\tilde{\bs}}
\newcommand{\tPhi}{\widetilde{\Phi}}
\newcommand\tep{\tilde\varepsilon}
%
%
% Hat
\newcommand{\hq}{\hat{q}}
\newcommand{\hG}{\widehat{G}}
\newcommand{\hF}{\widehat{\mathcal F}}
\newcommand{\hPhi}{\widehat{\Phi}}
\newcommand{\hPsi}{\widehat{\Psi}}
\newcommand{\hv}{\widehat{v}}
\newcommand{\hD}{\widehat{D}}
\newcommand{\hJ}{\widehat{J}}
\newcommand{\hcH}{\widehat{\mathcal H}}
\newcommand{\hcT}{\widehat{\mathcal T}}
%
% Blackboard bold
\def\CC{\mathbb{C}}
\def\NN{\mathbb{N}}
\def\RR{\mathbb{R}}
\def\BB{\mathbb{B}}
\def\ZZ{\mathbb{Z}}
\def\RP{\mathbb{RP}}
\def\TT{\mathbb{T}}
\def\HH{\mathbb{H}}
\def\bD{\mathbb{D}}

\newcommand{\bE}{\mathbb{E}}
\newcommand{\bP}{\mathbb{P}}

\renewcommand\SS{\mathbb{S}}
%

\newcommand{\fM}{\mathfrak{M}}
\newcommand{\sD}{\mathscr{D}}
\newcommand{\sS}{\mathscr{S}}
\newcommand{\sX}{\mathscr{X}}
\newcommand{\sF}{\mathscr{F}}
\newcommand{\sG}{\mathscr{G}}
\newcommand{\sH}{\mathscr{H}}
\newcommand{\sR}{\mathscr{R}}
\newcommand{\sY}{\mathscr{Y}}
\newcommand{\sV}{\mathscr{V}}
\newcommand{\sW}{\mathscr{W}}
\newcommand{\sZ}{\mathscr{Z}}
\newcommand{\sJ}{\mathscr{J}}
\newcommand{\sI}{\mathscr{I}}
\newcommand{\sQ}{\mathscr{Q}}
\newcommand{\sE}{\mathscr{E}}
\newcommand{\sL}{\mathscr{L}}
\newcommand{\sC}{\mathscr{C}}


% Calligraphic
\newcommand{\cA}{{\mathscr{A}}}
\newcommand{\cB}{{\mathcal B}}
\newcommand{\cC}{{\mathcal C}}
\newcommand{\cD}{{\mathcal D}}
\newcommand{\cE}{{\mathcal E}}
\newcommand{\cF}{{\mathcal F}}
\newcommand{\cG}{{\mathcal G}}
\newcommand{\cH}{{\mathcal H}}
\newcommand{\cI}{{\mathcal I}}
\newcommand{\cJ}{{\mathcal J}}
\newcommand{\cL}{{\mathcal L}}
\newcommand{\cM}{{\mathcal M}}
\newcommand{\cN}{{\mathcal N}}
\newcommand{\cO}{{\mathcal O}}
\newcommand{\cP}{{\mathcal P}}
\newcommand{\cQ}{{\mathcal Q}}
\newcommand{\cR}{{\mathcal R}}
\newcommand{\cS}{{\mathcal S}}
\newcommand{\cT}{{\mathcal T}}
\newcommand{\cX}{{\mathcal X}}
\newcommand{\cV}{{\mathcal V}}
\newcommand{\cW}{{\mathcal W}}
\newcommand\cZ{\mathcal Z}
%
% Fraktur
\newcommand{\fsl}{\mathfrak{sl}}
\newcommand{\fsu}{\mathfrak{su}}
\newcommand{\fa}{\mathfrak{a}}
\newcommand{\fD}{{\mathfrak D}}
\newcommand{\fB}{{\mathfrak B}}
\newcommand{\fF}{{\mathfrak F}}
\newcommand{\fH}{{\mathfrak H}}
\newcommand{\fK}{{\mathfrak K}}
\newcommand{\fP}{{\mathfrak P}}
\newcommand{\fS}{{\mathfrak S}}
\newcommand{\fW}{{\mathfrak W}}
\newcommand{\fZ}{{\mathfrak Z}}
\newcommand{\su}{{\mathfrak{su}}}
\newcommand{\gl}{{\mathfrak{gl}}}
\newcommand{\fU}{{\mathfrak{U}}}
\newcommand{\fg}{{\mathfrak{g}}}

\newcommand{\ta }{\widetilde{a}}
%
%
\newcommand{\pa}{\partial}
\newcommand{\pd}{\partial}
\newcommand\minus\backslash
\newcommand{\id}{{\rm id}}
\newcommand{\nid}{\noindent}
\newcommand{\abs}[1]{\left|#1\right|}
\def\ket#1{|#1\rangle}
%\def\ket#1#2{|{#1}{#2}\rangle}
%\def\kett#1#2#3{|{#1}{#2}{#3}\rangle}
\def\Strut{\vrule height12pt width0pt}
\def\BStrut{\vrule height12pt depth6pt width0pt}
\let\ds\displaystyle
\newcommand{\ms}{\mspace{1mu}}
\newcommand\lan\langle
\newcommand\ran\rangle
\newcommand\defn[1]{\emph{\textbf{#1}}}
\newcommand\half{\tfrac12}
\DeclareRobustCommand{\Bint}
{\mathop{%
		\text{%
			\settowidth{\intwidth}{$\int$}%
			\makebox[0pt][l]{\makebox[\intwidth]{$-$}}%
			$\int$}}}


%\DeclareRobustCommand{\Bint}
%{\mathop{%
%		\text{%
%			\settowidth{\intwidth}{$\int$}%
%			\makebox[0pt][l]{\makebox[\intwidth]{$-$}}%
%			$\int$}}}
			
\newcommand\Lone{\xrightarrow[\mathrm{a.s.}]{L^1}}

\DeclareMathOperator\singsupp{sing\,supp}
\newcommand\preq\preccurlyeq
\newcommand\pre\prec

\newcommand{\cAirr}{\cA^{\mathrm{irr}}}
\newcommand\gacirc{\ga^{\circ}}

%
% Log-like symbols
\newcommand{\sign}{\operatorname{sign}}
\newcommand{\spec}{\operatorname{spec}}
%\newcommand{\tr}{\operatorname{tr}}
\newcommand{\myspan}{\operatorname{span}}
\newcommand{\card}{\operatorname{card}}
\newcommand{\erf}{\operatorname{erf}}
\newcommand{\supp}{\operatorname{supp}}
\newcommand{\Span}{\operatorname{span}}
\newcommand{\ad}{\operatorname{ad}}
\newcommand{\inte}{\operatorname{int}}
\newcommand{\wlim}{\operatornamewithlimits{w-lim}}
\newcommand{\slim}{\operatornamewithlimits{s-lim}}
%\renewcommand{\limsup}{\operatornamewithlimits{\overline{lim}}}
%\renewcommand{\liminf}{\operatornamewithlimits{\underline{lim}}}
%\newcommand{\ker}{\operatorname{ker}}
\newcommand{\iu}{{\mathrm i}}
\newcommand{\I}{{\mathrm i}}
\newcommand{\e}{{\mathrm e}}
\newcommand{\dd}{{\mathrm d}}
\DeclareMathOperator\Div{div} \DeclareMathOperator\Ric{Ric}
\DeclareMathOperator\Real{Re} \DeclareMathOperator\Met{Met}
\DeclareMathOperator\Imag{Im}

\DeclareMathOperator\Vol{Vol} \DeclareMathOperator\SU{SU}
\DeclareMathOperator\End{End} \DeclareMathOperator\Hom{Hom}
\DeclareMathOperator\Spec{Spec}
\DeclareMathOperator\diag{diag} \DeclareMathOperator\Int{int}
\DeclareMathOperator\dist{dist} \DeclareMathOperator\sdist{sdist} \DeclareMathOperator\diam{diam}
\DeclareMathOperator\Cr{Cr}

\newcommand{\norm}[1]{\left\lVert#1\right\rVert}
\newcommand{\norml}[1]{\|#1\|}
\newcommand{\braket}[2]{\langle #1,#2\rangle}





\newcommand\uno{1}
\renewcommand\leq\leqslant
\renewcommand\geq\geqslant
%
% Barred integrals (perhaps with \usepackage[intlimits]{amsmath}
\newlength{\intwidth}
\DeclareRobustCommand{\bint}[2]
   {\mathop{%
      \text{%
        \settowidth{\intwidth}{$\int$}%
        \makebox[0pt][l]{\makebox[\intwidth]{$-$}}%
        $\int_{#1}^{#2}$}}}
\DeclareRobustCommand{\varbint}
   {\mathop{%
      \text{%
        \settowidth{\intwidth}{$\int$}%
        \makebox[0pt][l]{\makebox[\intwidth]{$-$}}%
        $\int$}}}
%
% Subscripts
\newcommand{\sub}[1]{_{\mathrm{#1}}}
\newcommand{\super}[1]{^{\mathrm{#1}}}
\newcommand\coul{^{\mathrm{C}}}
\newcommand\any{^{\mathrm{A}}}
\newcommand\harm{^{\mathrm{H}}}
\newcommand\Hsc{H\scal}

\newcommand\hphi{\widehat{\phi}}

%
% Roman enumeration
\addtolength{\parskip}{3pt}
\renewcommand{\theenumi}{(\roman{enumi})}
\renewcommand{\labelenumi}{\theenumi}
%
\newcommand\dif[2]{\frac{\pd #1}{\pd #2}}
\newcommand\Dif[2]{\frac{\dd #1}{\dd #2}}
\newcommand\loc{_{\mathrm{loc}}}
\newcommand\BOm{\overline\Om}
\DeclareMathOperator\Fix{Fix}

\newcommand\tV{\widetilde{V}}
\newcommand\tX{\widetilde{X}}


\numberwithin{equation}{section}


%    Blank box placeholder for figures (to avoid requiring any
%    particular graphics capabilities for printing this document).
\newcommand{\blankbox}[2]{%
  \parbox{\columnwidth}{\centering
%    Set fboxsep to 0 so that the actual size of the box will match the
%    given measurements more closely.
    \setlength{\fboxsep}{0pt}%
    \fbox{\raisebox{0pt}[#2]{\hspace{#1}}}%
  }%
} \DeclareMathOperator\curl{curl}

\DeclareMathOperator\hyp{\HH^n(\ka)}

\newcommand\esp{C^\infty_{\mathrm{div}}(\TT^d,\RR^d)}
\newcommand\espR{C^\infty_{\mathrm{div}}(\RR^d,\RR^d)}
\newcommand\espdos{C^\infty_{\mathrm{div}}(\TT^2,\RR^2)}
\newcommand\bv{\bar v}
\newcommand\bp{\bar p}
\newcommand\bu{\bar u}
\newcommand\bF{\bar F}

\DeclareMathOperator\LOS{BLOS}
\newcommand{\bX}{\bar{X}}
\newcommand{\bM}{\bar{M}}
\newcommand\rX{X^{\circ}}
\newcommand\rM{M^{\circ}}
\newcommand\tg{\widetilde{g}}
\newcommand\tga{\widetilde{\ga}}

\begin{document}

\title[Reconstruction of a Lorentzian manifold]{Reconstruction of a Lorentzian manifold\\ from its Dirichlet-to-Neumann map}

 %    Information for first author
 \author{Alberto Enciso}
 %    Address of record for the research reported here
 \address{Instituto de Ciencias Matem\'aticas, Consejo Superior de
   Investigaciones Cient\'\i ficas, 28049 Madrid, Spain}
 \email{aenciso@icmat.es}
 %    \thanks will become a 1st page footnote.


  %    Information for second author
 \author{Gunther Uhlmann}
 \address{Department of Mathematics, University of Washington, Seattle, WA 98195, USA; Institute for Advanced Study, The Hong Kong University of Science and Technology, Kowloon, Hong Kong, China} 
 \email{gunther@math.washington.edu}

 %    Information for first author
 \author{Micha{\l} Wrochna}
 \address{Laboratoire Analyse Géométrie Mod\'elisation, CY Cergy Paris Universit\'e, 95302 Cergy-Pontoise, France}
 \email{michal.wrochna@cyu.fr}
 %    Address of record for the research reported here

 %    \thanks will become a 1st page footnote.




%%    General info
%\subjclass[2010]{35B38, 58J05, 58K45}
%\date{\today}
%
%\keywords{ }
%
\begin{abstract}
We prove that the Dirichlet-to-Neumann map of the linear wave equation determines the topological, differentiable and conformal structure of the underlying Lorentzian manifold, under mild technical assumptions. With more stringent geometric assumptions, the full Lorentzian structure of the manifold can be recovered as well. The key idea of the proof is to show that the singular support of the Schwartz kernel of the Dirichlet-to-Neumann map of a manifold completely determines the so-called boundary light observation set of the manifold together with its natural causal structure.
\end{abstract}


\maketitle


\section{Introduction}

In this paper we are concerned with the problem of determining a Lorentzian metric from the corresponding Dirichlet-to-Neumann map. This is a central question in the study of inverse problems for wave equations, which is a subject that has attracted considerable attention in recent years (see e.g.~\cite{Alex20,Alex21,FO,KLU,L,LU,SY,UZ20,VW,Wang18} and references therein). 

In the case of (genuinely) semilinear wave equations, it has been shown in various settings that, up to natural obstructions, one can indeed recover the metric from the Dirichlet-to-Neumann map~\cite{HUZ,KLU,L,UZ20}. However, the corresponding problems for linear equations are still largely open. The reason for this is twofold. On the one hand, the boundary control method~\cite{B} does not work when the metric is a smooth function of the time variable. This is because it hinges on Tataru's Carleman estimates~\cite{Tataru1,Tataru2} for operators with partially analytic coefficients, which ensures that the unique continuation property holds across any non-characteristic hypersurface. This property can fail without the analyticity hypothesis~\cite{Alinhac}. On the other hand, while the proofs of the existing results consider small-amplitude solutions for which the equation operates in an essentially linear regime, they crucially exploit the known nonlinearity to produce artificial point sources thanks to the nonlinear interaction of waves. 


Our objective in this paper is to fill this gap. Specifically, our main result is that, under mild technical hypotheses, the Dirichlet-to-Neumann map associated with the linear wave operator of a Lorentzian manifold determines the underlying Lorentzian manifold modulo a conformal diffeomorphism. The Riemannian analog of our main therem is a well known result of Lassas and Uhlmann~\cite{LU}. Under more stringent geometric assumptions, the full Lorentzian metric can be recovered as well.


\subsection{Main theorem}


Before getting into specifics, let us start by describing the geometric setting we will work in. Let~$X$ be a smooth $(d+1)$-dimensional manifold with nonempty boundary~$\Ga=\pd X$, and let $\rX$ denote its interior. We endow~$X$ with a Lorentzian metric~$g$, with we will always assume to be $C^\infty$-smooth up to the boundary. Throughout, we assume that this manifold is {\em admissible}\/, which simply means that the Lorentzian manifold $(\rX,g)$ is globally hyperbolic (in the sense of Lorentzian manifolds with a timelike boundary~\cite{Ake}) with a smooth time function~$t$ and a compact Cauchy surface, and that~$\Ga$ is a strictly null-convex timelike surface. This ensures that broken null geodesics are well defined for all time. Details are given in Definition~\ref{D.admissible}.

Let us now consider an admissible manifold $(X,g)$ as above and a function $f\in C^\infty_c(\Ga)$, which one can assume to be supported on the set $t>T$.  Now, let us denote by~$u$ the unique solution to the problem
	\begin{align*}
		\square_g u&=0\qquad \text{in } \rX\,,\\
		u&=f\qquad \text{on } \Gamma\,,\\
		u&=0 \qquad \text{if } t<T\,,
	\end{align*}
where $\square_g$ is the wave operator.
The {\em Dirichlet-to-Neumann map}\/ of the Lorentzian manifold $(X,g)$ is then defined as the linear map $\La_g: C^\infty_c(\Ga)\to C^\infty(\Ga)$ given by
\[
\La_gf:=g(\nu,\nabla u)\,,
\]
where here and in what follows $\nu$ is the outward pointing unit normal at~$\Ga$ and $\nabla$ is the covariant derivative (which also depends on~$g$).

As wave propagation is closely related to null geodesics, it stands to reason that some hypothesis on the structure of the null geodesics of the manifold should be necessary in order to retrieve geometric information about the manifold from the Dirichlet-to-Neumann map. Roughly speaking, we  need to ensure that the intersection of the null cones with~$\Ga$ is rich enough, and  on the boundary there are no conjugate points along  null geodesics. 


To state these hypotheses precisely, given a point~$p\in X$, we shall denote by $\sB^\pm(p)\subset\Ga$ the intersection of the boundary with the future null cone of~$p$, and set $\sB(p):=\sB^+(p)\times\sB^-(p)\subset\Ga\times\Ga$. We emphasize that the null geodesics and null cones we consider here do not include reflections, that is, we are not considering broken null geodesics (which will nevertheless play an important role later~on).



\begin{definition}\label{D.ls}
	Let $(X,g)$ be an admissible Lorentzian manifold. We say that its boundary~$\Ga$ is {\em light-sensitive}\/ if the following conditions hold:
	\begin{enumerate}
		\item If a null geodesic passes through any points $p\in\rX$ and~$q\in\Ga$, then $p$ and~$q$ are not conjugate.\label{I.conj}
	\item For every $p,q\in \rX$, $\sB(p)=\sB(q)$ only if $p=q$.\label{I.sB}
	\item Given any $p\in \Ga $, every null geodesic that starts at~$p$ has another endpoint on~$\Ga$.\label{I.endp}
 \item For every pair of distinct points $p, q\in \rX$, there is a null geodesic passing through~$p$ but not through~$q$ that has an endpoint on~$\Ga$ (and therefore two).\label{I.miss}
	\end{enumerate}
\end{definition}

Let us briefly comment on the meaning of these hypotheses. Condition~\ref{I.conj} ensures that there are no conjugate points on the boundary along null geodesics. Condition~\ref{I.sB} ensures that knowledge of the intersection of the future and past null cones with the boundary is enough to determine any interior point. This is a stronger version of the property that $(X,g)$ is ``distinguishing'' as a Lorentzian manifold with boundary, which is actually a consequence of the global hyperbolicity of~$(X,g)$~\cite[Definition~2.8 and Proposition~2.13]{Ake}.  It is convenient to keep in mind that the key hypotheses of the geometric theorem proven in~\cite{HU}, a variant of which we will need in Section~\ref{S.main}, are essentially~\ref{I.conj} and an analog of~\ref{I.sB} that uses the future null cone with reflections instead of the set~$\sB(p)$. The new conditions~\ref{I.endp} and~\ref{I.miss} respectively guarantee that null geodesics that emanate from the boundary must eventually hit the boundary and that there are enough null geodesics passing through any given point and hitting the boundary that one can avoid any other given point of the manifold.


The main result of this paper can now be stated as follows:

\begin{theorem}\label{T.main}
	The Dirichlet-to-Neumann map encodes the topological, differentiable and conformal structure of an admissible manifold with light-sensitive boundary. Specifically, let $(X_j,g_j)$, with $j=1,2$, be admissible Lorentzian manifolds whose boundaries $\Ga_j$ are light-sensitive. Suppose that their Dirichlet-to-Neumann maps coincide up to a diffeomorphism $\Phi:\Ga_1\to\Ga_2$, that is, $\Phi_* \La_{g_1}= \La_{g_2} \Phi_*$. Then there exists a conformal isometry $\Psi: (X_1,g_1)\to(X_2,g_2)$ and $\Psi|_{\Ga_1}=\Phi$.
\end{theorem}

Once we have identified the conformal class of the metric, it is well known that recovering the conformal factor is equivalent to recovering a potential. Hence, one can invoke the existing results about the determination of a potential for wave equations where the Lorentzian metric is fixed~\cite{Alex20,Alex21} to show that, under further geometric assumptions, one can indeed recover the full Lorentzian structure of the manifold. These assumptions are stated as Hypothesis~\ref{H.Alex} in Section~\ref{S.cf}. For the purposes of this Introduction, it is enough to note that Hypothesis~\ref{H.Alex} is satisfied, for instance, when the metric is a $C^2$-small perturbation of the flat metric.

\begin{theorem}
	\label{T.metric}
	Under the assumptions of Theorem~\ref{T.main}, suppose moreover that the Lorentzian manifold $(X_1,g_1)$ satisfies Hypothesis~\ref{H.Alex}. Then the map $\Psi: (X_1,g_1)\to(X_2,g_2)$ can be taken to be an isometry.
\end{theorem}



\subsection{Strategy of the proof}

The proof of our main theorem can be divided in two steps. The first step consists of a careful analysis of the Schwartz kernel of the Dirichlet-to-Neumann map, which we denote by $\La_g(\cdot,\cdot)\in \sD'(\Ga\times\Ga)$. More precisely, the key property we establish (Theorem~\ref{T.singsupp}) is that a pair of distinct points $(z,z')\in\Ga\times\Ga$ is in the singular support of $\La_g(\cdot,\cdot)$ if and only if 
there is a broken future-oriented null geodesic going from~$z'$ to~$z$. In particular, 
$\La_g(\cdot,\cdot)$ fails to be smooth exactly on the {\em boundary light observation set}\/~\cite{HU} of the manifold (or set of light cone cuts, as it is known in the physics literature~\cite{phys}). 

To prove this fact, the initial observation that, by standard results on propagation of singularities, a pair $(z,z')\in\Ga\times\Ga$ cannot be in the singular support of $\La_g(\cdot,\cdot)$ unless the above condition is satisfied. This is well known, so the crux of the proof is to show the converse implication. It was pointed out in~\cite{HU} that the basic strategy to prove the latter should involve the construction of a rich enough family of singular solutions to the wave equation. The way we articulate this rough idea in this paper is not exactly via singular solutions, but by means of a carefully constructed family of smooth solutions based on Gaussian beams. Gaussian beams have played a role in a number of works on inverse problems, e.g.~\cite{19,FO,HUZ,HUZ2,UZ20}.

Our strategy is to consider a certain broken null geodesic~$\ga$ that connects two distinct points~$(z,z')$ of~$\Ga$, possibly after several reflections. One then considers a family of Gaussian beams, labeled by a parameter $h\ll1$, which concentrates on~$\ga$ and are approximate solutions to the equation for very small~$h$. Then one identifies a family of exact solutions defined by tweaking Gaussian beams using carefully chosen cutoff functions and adding terms to cancel out certain errors that appear in the construction. If one estimates the size of the corresponding Dirichlet data and of the normal derivatives, one concludes that the Schwartz kernel of the Dirichlet-to-Neumann map $\La_g(\cdot,\cdot)$ cannot be a bounded function in any neighborhood of~$(z,z')$. 

The second step of the proof is to show that the geometric information about null geodesics connecting boundary points that we obtain from knowledge of the singular support of $\La_g(\cdot,\cdot)$ suffices to determine the Lorentzian manifold up to a conformal transformation. The key idea is to show that the information about the causal structure of the boundary that we recover from the singular support of~$\La_g(\cdot,\cdot)$ is roughly (yet not exactly) equivalent to the boundary light observation set. Once this connection has been rigorously established, we can use minor modifications of the purely geometric arguments of Hintz and Uhlmann~\cite{HU} to conclude the existence of the desired conformal diffeomorphism.


\subsection{Organization of the paper}

In Section~\ref{S.geom}, we introduce some notation and recall some geometric properties of admissible Lorentzian manifolds that are used in the construction of Gaussian beams, which we carry out in Section~\ref{S.GB}. These Gaussian beams are a key ingredient in the proof of Theorem~\ref{T.singsupp}, which is presented in the next section. Having these results in hand, we complete the proof of Theorem~\ref{T.main}. In the short Section~\ref{S.cf} we show how Theorem~\ref{T.metric} follows readily from our main theorem and existing results in the literature.


 \section{Geometric preliminaries}
 \label{S.geom}
 
 Our purpose in this section is to recall some basic properties about broken null geodesics on globally hyperbolic Lorentzian manifolds with a nice boundary.
 
 
Following~\cite{HU}, in this section we shall assume that the $(d+1)$ Lorentzian manifold with boundary $(X,g)$ is {\em admissible}\/ in the following sense, which basically means that the manifold is globally hyperbolic with a compact Cauchy hypersurface and has a strictly null-convex timelike boundary. We recall that a vector $V\in T_pX$ is called timelike, spacelike or null when $g(V,V)$ is negative, positive or zero, respectively.

\begin{definition}\label{D.admissible}
	A Lorentzian manifold~$(X,g)$ with boundary~$\Ga$, of signature $(-,+,\dots ,+)$, will be called {\em admissible}\/ if:
	\begin{enumerate}
		\item There exists a proper, surjective time function $t:X \to\RR$ whose gradient is timelike. Time slices $X_T:=\{x\in X : t(x)=T\}$ are  compact for all~$T$.
		\item The boundary is timelike, that is, the pullback of~$g$ to~$\Ga$ is Lorentzian.
		\item $\Ga$ is {\em strictly null-convex}, that is, $\mathrm{II}(V,V)>0$ for all nonzero null vectors $V\in T_p\Ga$ and all $p\in\Ga$.	
	\end{enumerate}
\end{definition}

Let us recall that the second fundamental form of the timelike hypersurface~$\Ga$ is defined as $\mathrm{II}(V,V):= g(\nabla_V\nu,V)$, where $\nu$ is the outward pointing unit normal vector field and $V$ is a tangent vector to~$\Ga$. Note that, as is well known (see e.g.~\cite{HU}), the condition~(iii) ensures that all broken null geodesics (whose definition we will recall below) are well defined globally.


To analyze curves that end at a boundary point of~$X$, it is useful to recall that one can regard~$(X,g)$ as a submanifold with boundary of a Lorentzian manifold without boundary of the same dimension. Specifically, it is easy to prove the following:

\begin{proposition}[\cite{HU}, Lemma 2.2]\label{P.isom}
	Let $\tX$ be any $(d+1)$-dimensional manifold where $X$ is embedded as a submanifold (e.g., its double). Then there is a Lorentzian metric~$\widetilde g$ and an associated time function~$\widetilde t$ on~$\tX$ such that $\widetilde g|_X=g$ and $\widetilde t|_X=t$.
\end{proposition}


We also need to know what the assumption that the boundary is null-convex means in terms of the structure of the null geodesics that reach the boundary. For this, let us denote by $\nu$ the outward-pointing unit normal of~$\Ga$. The second fundamental form of~$\Ga$ can then written as $\mathrm{II}(V,V)= g(\nabla_V\nu,V)$. We recall the following fact:

\begin{proposition}[\cite{HU}, Lemma 2.4]\label{L.nullc}
	$\Ga$ is null-convex if and only if any null geodesic segments $\ga_+:(-s_0,0]\to X$ and $\ga_-:[0,s_0)\to X$ with $\ga_\pm(0)\in \Ga$ and $\ga_\pm(s)\in\rX$ for $-s_0<\pm s< 0$ satisfy the sign condition
	\begin{equation}\label{E.signgeod}
		\pm g(\dot\ga_\pm(0),\nu)>0\,.
	\end{equation}
\end{proposition}


Let us start by recalling the precise definition of broken null geodesics. Informally, they are curves which coincide with null geodesics in the interior of~$X$ and that, if they reach the boundary at some point, they get reflected according to Snell's law. Of course, the reason for which these curves are relevant to the study of wave equations is because of the way in which propagation of singularities takes place at the boundary~\cite{Tay}. For concreteness, here we will consider broken null geodesics defined on an open interval, but of course one can similarly consider closed intervals. 

\begin{definition}\label{D.BNG}
	Let $I\subset\RR$ be an open interval. A piecewise smooth curve $\ga: I\to X$ is a {\em future-oriented broken null geodesic} if:
	\begin{enumerate}
		\item If $I'\subset I$ is an interval with $\ga(I')\cap\Ga=\emptyset$, then the restriction $\ga|_{I'}$ is a future-oriented, affinely parametrized null geodesic in $(\rX,g)$.
		\item Suppose that $\ga(s_0)\in\Ga$ for some $s_0\in I$. Then there exists some $\ep_\ga>0$ such that $\ga((s_0-\ep_\ga,s_0))\cap\Ga$ and $\ga((s_0,s_0+\ep_\ga))\cap\Ga $ are empty, and furthermore Snell's law holds:
		\begin{equation}\label{E.reflect}
			\dot\ga(s_0-0)= \dot\ga(s_0+0) - 2 g(\dot\ga(s_0+0),\nu)\, \nu\,.
		\end{equation}
	\end{enumerate}
\end{definition} 

\begin{remark}\label{R.nongrazing}
	The assumption that $\Ga $ is null convex ensures that $\dot\ga(s_0-0)\neq \dot\ga(s_0+0)$.
\end{remark}


It is standard that, given any point $p\in \rX$ and a null vector $V\in T_pX$, there always exists an inextendible broken null geodesic with initial position~$p$ and velocity~$V$. This is also true if $p\in\Ga$ and $g(V,\nu)<0$.

As broken null geodesics are made of null geodesic segments, it will be convenient to introduce suitable coordinates in a neighborhood of such segments. As the endpoints of the segments we will consider are in the boundary~$\Ga$, and these segments can always be extended as strictly larger segments in the larger $(d+1)$-manifold with boundary~$(\tX,\tg)$ introduced in Proposition~\ref{P.isom}, it is technically convenient to consider the problem directly in the larger manifold. 

Therefore, suppose that $\ga:I\to \tX$ be an affinely parametrized null geodesic segment on~$(\tX,\tg)$, where $I\ni 0$ is an interval. Take a smooth Riemannian metric~$g^+$ on~$\tX$ and let $\nabla^+$, $\exp^+_p$ denote the corresponding covariant derivative and exponential map at a point~$p\in\tX$. Using this Riemannian metric, we define the normal bundle of the curve~$\ga$ at~$s$ as
\[
T_s^\perp:= \{V\in T_{\ga(s)}\tX: g^+(\dot\ga(s),V)=0\}\,.
\] 
Given a normal vector $V\in T_0^\perp$, we consider the $g^+$-parallel transport of the vector~$V$ along the curve~$\ga$, which is defined as  $\phi_s (V):=\tV(s)$ with~$\tV$ being the solution to the ODE
\[
\nabla^+_{\dot\ga(s)}\tV(s)=0\,,\qquad \tV(0)=V\,.
\]
It is easy to see that $\tV(s)\in T_s^\perp$. Using the corresponding exponential map defined by the metric~$g^+$, a nice set of coordinates $y=(y^0,\dots, y^d)$ can be obtained as follows:

\begin{proposition}\label{P.Phi}
	Let $\{E_1,\dots,E_d\}$ be a $g^+$-orthonormal basis of $T_0^\perp$ and let $\BB $ be the $d$-dimensional unit ball. For any small enough~$\de>0$, the map $I\times \BB\to\tX$ defined as
	\[
	(y^0,y^1,\dots, y^d)\mapsto \exp^+_{\ga(y^0)}\left( \de  y^1E_1+\cdots+ \de  y^dE_d\right)	
	\]
	is a diffeomorphism onto its image.
\end{proposition}




 
\section{An adapted family of Gaussian beams}
\label{S.GB}



Our goal in this section is to construct a suitable family of Gaussian beams for the equation $\square_g u=0$ on the Lorentzian manifold $(X,g)$, which we assume to be admissible in the sense of Definition~\ref{D.admissible}. To this end, we consider a  future-oriented broken null geodesic $\ga:[s_0,s_K]\to X$ consisting of $K\geq2$ affinely parametrized null geodesic segments $\ga_k:[s_{k-1},s_k]\to X$ (with $1\leq k\leq K$) that are reflected at the boundary following Snell's law. Specifically, this means that $\ga_k(s_{k-1}),\ga_k(s_k)$ are points of~$\Ga$, satisfying $\ga_k(s_k)=\ga_{k+1}(s_k)$, and that, by Equation~\eqref{E.reflect},
\[
\dot\ga_k(s_k)=\dot\ga_{k+1}(s_k)-2 g(\dot\ga_{k+1}(s_k),\nu)\,.
\]
In future sections, we will be particularly interested in the ``initial point'' $z':=\ga_1(s_0)$ and the ``final point'' $z:=\ga_K(s_{K-1})$ (which is, however, {\em not}\/ the endpoint $\ga_K(s_K)$ of the broken geodesic).
The situation is sketched in Figure~\ref{F.BG} in the simplest case $K=2$.

\begin{figure}
	\includegraphics[width=5cm]{bdrysketch.pdf}
	\caption{Sketch of the broken geodesic connecting points $z,z'\in\Ga$ for $K=2$.\label{F.BG}}
\end{figure}


Following the discussion of Section~\ref{S.geom}, we will also consider larger null geodesic segments $\tga_k:[s_k-\ep_\ga,s_{k+1}+\ep_\ga]\to\tX$ which extend~$\ga_k$ in~$\tX$. Let us denote by $d_k(x)$ the distance in~$(\tX,g^+)$ to the segment $\tga_k([s_k-\ep_\ga,s_{k+1}+\ep_\ga ])$. Consider now a nonnegative function $\chi\in C^\infty(\RR)$ such that $\chi(\la)$ equals~1 for $\la<\frac12$ and~0 for $\la>1$. We then set, for each positive $h\ll1$, 
\[
\chi^h_k(x):=\chi(h^{-\si}d_k(x))\,,
\]
where $\si\in (0,\frac12)$ is a fixed constant. 

We claim that, given positive integers~$m,M$, we can construct functions of the form
	\begin{equation}\label{E.uh}
		u^h(x)=\sum_{k=0}^K e^{i\psi_k(x)/h}a^h_k(x)\,,\quad  a^h_k(x):= \sum_{j=0}^J h^j\,a_{k,j}(x)\,,\quad a_{k,j}(x):= \chi^h_k(x)\, \ta_{k,j}(x)
	\end{equation}
	that are approximate solutions to wave equation on~$(X,g)$ in the sense that	
	\begin{equation}\label{E.CmM}
			\|\square_g u^h\|_{C^m(X)}\lesssim h^M\,.
	\end{equation}
	Here and in what follows, we  write $q_1\lesssim q_2$ when there exists an $h$-independent constant such that $q_1\leq Cq_2$. The functions $\ta_{k,j}(x)$ do not depend on~$h$. The functions $a_{k,j}$ do, although this is not reflected notationally because their behavior on the broken geodesic is in fact independent of~$h$. Specifically, we have introduced the $h$-dependent cutoff functions $\chi^h_k$ to ensure that the function $u^h$ is supported in a neighborhood of the broken geodesic~$\ga$ of width $h^\si\ll1$. This will be relevant for the proof of Theorem~\ref{T.singsupp}.
	
As we shall need various details about the structure of the various functions involved, we need to sketch the construction of these approximate solutions, referring to the general reference~\cite{GB} (or to the recent paper~\cite{FO}) for some details. 

We will also need to consider tangent covectors along the curve, so we will denote by 
$\xi_{k}(s):=g(\dot\ga_k(s),\cdot)$  the 1-form dual to the vector $\dot\ga_k(s)$. 
As $\ga_k(s)$ is a null geodesic, it is standard that the curve in the cotangent space $T^*X$ given by $(\ga_k(s),\xi_k(s))$ is a null bicharacteristic of the operator~$\square_g$.

Let us now focus on a fixed geodesic segment $\ga_k([s_{k-1},s_k])$ and drop the script~$k$, when there is no risk of confusion, for ease of notation. The starting point is to choose the function $\psi\equiv\psi_k$ so that, for all $s_{k-1}\leq s\leq s_k$, $\psi(\ga(s))$ is real and 
\begin{equation}\label{E.dpsi}
	d \psi(\ga(s))=\xi(s)\,,
\end{equation}
where $d$ is the exterior differential. As we will need a number of additional conditions and several rather explicit formulas, we will make use of the fact that the support of $\chi_{k,h}$ can be covered by the coordinates $y_{k}=(y^0_{(k)},\dots, y^d_{(k)})$ introduced in Proposition~\ref{P.Phi} and work mostly in these coordinates. We consider the corresponding coordinates in the cotangent space. Dropping the scripts~$k$, we denote them by  $(y,\xi)=(y^0,\dots,y^d,\xi_0,\dots, \xi_d)$. With some abuse of notation, the expression of the bicharacteristic curve in these coordinates will be denoted $(y(s),\xi(s))$. For future reference, let us record here that one can choose an affine parametrization so that the equation for bicharacteristics then reads as
\begin{equation}\label{E.bichar}
	\dot y^\mu(s) = \frac{\pd p}{\pd\xi_\mu}(y(s),\xi(s))\,,\qquad \dot \xi_\mu(s)= -\frac{\pd p}{\pd y^\mu}(y(s),\xi(s))\,,
\end{equation}
with $0\leq\mu\leq d$,
and that the null condition is
\begin{equation}\label{E.null}
	p(y(s),\xi(s))=0\,.
\end{equation}
Here $p(y,\xi)$ is the principal symbol of $\square_g$, which is just $g(\xi,\xi)$. Here here and in what follows, we denote by~$g$ the metric on $T^*X$. In the coordinates $(y,\xi)$, this reads as $ p(y,\xi)=g^{\mu\nu}(y)\xi_\mu\xi_\nu$, where summation over repeated indices is understood.


Let us now consider the identity
\begin{equation}\label{E.squareu}
	\square_g u^h= \sum_{k=1}^K \left\{ -h^{-2} g(d\psi_k,d\psi_k) a^h_k +h^{-1} i \left[g(d\psi_k,d a_k^h)+a_k^h\square_g\psi_k \right]+\square_g a_k^h\right\}=:f^h\,.
\end{equation}
The leading term is controlled by the functions $g(d\psi_k,d\psi_k)$, which in our coordinates can be equivalently written as
\[
H(y):= p(y,\pd_y \psi(y))\,.
\]
It is easy to see that $f^h$ vanishes on the geodesic segment $\ga_k$ to first order. Indeed, for $s_{k-1}\leq s\leq s_k$, one has that
\[
H(y(s))= p(y(s),\xi(s))=0
\]
by Equations~\eqref{E.dpsi} and~\eqref{E.null}. Similarly, for any $0\leq\mu\leq d$,
\[
\frac{\pd H}{\pd y^\mu}(y(s))= \frac{\pd p}{\pd y^\mu}+ \frac{\pd H}{\pd \xi_\nu}\frac{\pd}{\pd y^\nu}\frac{\pd\psi}{\pd y^\mu}\,,
\]
with the various functions being evaluated on $(y(s),\pd_y\psi(y))$. By~\eqref{E.dpsi}-\eqref{E.bichar}, this is just
\[
\frac{\pd H}{\pd y^\mu}(y(s))=\frac{\pd p}{\pd y^\mu}(y(s),\xi(s))+ \dot\xi_\mu(s)=0\,.
\]

To ensure the leading term $H(y)$ vanishes on~$\ga_k$ to second order, one needs to choose the phase function~$\psi$  carefully. Evaluating all the expressions on $(y(s),\xi(s))$, to ensure that $H$ vanishes to second order, we start with the formula
\begin{multline}\label{E.H2}
0=\frac{\pd^2 H}{\pd y^\mu \pd y^\nu}(y(s))= \frac{\pd^2 p}{\pd y^\mu \pd y^\nu} + \frac{\pd^2 p}{\pd y^\mu \pd \xi_\rho }\frac{\pd^2 \psi}{\pd y^\rho \pd y^\nu} + \frac{\pd^2 p}{\pd y^\nu \pd \xi_\rho }\frac{\pd^2 \psi}{\pd y^\rho \pd y^\mu} \\
+ \frac{\pd^2 p}{\pd \xi_\la  \pd \xi_\rho }\frac{\pd^2 \psi}{\pd y^\mu \pd y^\rho }\frac{\pd^2 \psi}{\pd y^\la \pd y^\nu} + \frac{\pd p}{\pd \xi_\rho}\frac{\pd^3 \psi}{\pd y^\mu \pd y^\nu \pd y^\rho }\,.
\end{multline}

Introducing the  $(d+1)\times (d+1)$ matrices
\begin{gather*}
	\Psi_{\mu\nu}(s):= \frac{\pd^2 \psi}{\pd y^\mu \pd y^\nu}(y(s))\,,\qquad A_{\mu\nu}(s):= \frac{\pd^2 p}{\pd y^\mu \pd y^\nu}(y(s),\xi(s))\,,\\
	B^\mu_\nu(s):= \frac{\pd^2 p}{\pd \xi_\mu \pd y^\nu}(y(s),\xi(s))\,, \qquad G^{\mu\nu}(s):= \frac{\pd^2 p}{\pd \xi_\mu \pd \xi_\nu}(y(s),\xi(s))= g^{\mu\nu}(y(s))\,,
\end{gather*}
Equation~\eqref{E.H2} can be rewritten as
\begin{equation}\label{E.H2bis}
	\dot \Psi + \Psi G\Psi + B^\top \Psi + \Psi B+A=0\,.
\end{equation}
As the matrices $A,B$ and~$G$ are known, our goal is to construct~$\psi$ so that the matrix~$\Psi$ satisfies the ODE~\eqref{E.H2bis}. 

It is standard that one can use the fact that~\eqref{E.H2bis} is a Ricatti equation to do this. We will sketch the argument and refer to~\cite{GB} for details. One should keep is mind that the matrix $\Psi(s)$ must be symmetric, as it is given by the Hessian of~$\psi$ on the curve $y(s)$. Furthermore, we want to ensure that the symmetric matrix $\Imag \Psi(s)$ is nonnegative definite, and that its kernel is spanned by $\dot y(s)$. This will ultimately have the effect that, on the support of $\chi_k^h$, 
\begin{equation}\label{E.Gauss}
	e^{-Cd_k(s)^2/h}\leq |e^{i\psi_k(x)/h}|\leq e^{-d_k(s)^2/(Ch)}
\end{equation}
for a positive constant $C$. Of course, this is the reason for which these solutions are known as Gaussian beams.

To construct a solution to the Ricatti-type ODE~\eqref{E.H2bis}, one starts with the auxiliary linear system
\begin{equation}\label{E.aux}
	\dot Y= BY+GN\,,\qquad \dot N=-AY-B^\top N
\end{equation}
for the unknowns $(Y(s),N(s))$. Now pick an arbitrary $(d+1)\times (d+1)$ symmetric matrix $\cM_k$ such that $\Imag \cM_k$ is nonnegative definite and $\ker \cM_k=\CC\, \dot y(s_{k-1})$, so that $\Imag\cM_k$ is positive definite on the orthogonal complement of $\dot y(s_{k-1})$. The basic idea~\cite{GB} is that if $(Y(s),N(s))$ is the solution to the linear system~\eqref{E.aux} with initial data
\[
(Y(s_{k-1}),N(s_{k-1}))=(I, \cM_k)\,,
\]
where $I$ is the identity matrix, then $Y(s)$ is invertible for all $s_{k-1}\leq s\leq s_k$ and 
\[
\Psi(s):= N(s)Y(s)^{-1}
\]
is a symmetric solution to Equation~\eqref{E.H2bis} with the aforementioned positivity condition. Equation~\eqref{E.H2} then holds.

Choosing $\psi$ so that $H$ vanishes on~$y(s)$ to higher order is now considerably simpler. Indeed, suppose that
\[
\pd_y^\be H(y(s))=0
\]
for all multiindices with $|\be|\leq r$ and $r\geq2$. If $|\al|=r+1$, one can write
\begin{align}
	0&=\pd_y^\al H(y(s))= \pd_y^\al p(y,\pd_y\psi(y))= \frac {\pd p}{\pd \xi_\mu}\frac{\pd }{\pd y^\mu} \pd_y^\al \psi+ F_{\mu,\al}(s)\notag \\
	&= \frac{d}{ds}\pd_y^\al\psi(y(s))+ F_{\mu,\al}(s)\,,\label{E.psial}
\end{align}
where $F_{\mu,\al}(s)$ denotes a  quantity that is completely specified in terms of $\pd_y^\be\psi(y(s))$ with $|\be|\leq r$. Therefore, if one chooses the constants $c_{\al,k}:=\pd_y^\al H(y(s_{k-1}))$, $\pd_y^\al H(y(s))$ is completely determined by solving the ODE~\eqref{E.psial}.

It is well known~\cite{GB} that, for $k=1$, the constants $\cM_k$, $c_{\al,k}$ can be chosen freely (up to the obvious constraints coming from the fact that partial derivatives of~$\psi$ must commute), while for $k>1$ the corresponding constants are essentially determined by the reflection conditions, which are related to the zero Dirichlet conditions we want to (approximately) impose on the boundary. But before discussing this point, let us consider the lower order terms that arise in~\eqref{E.squareu} and analyze the constraint they impose on the coefficients $a_{k,j}$.

In Equation~\eqref{E.squareu}, we wrote $\square_g u^h$ as the sum of the functions
\[
-h^{-2} g(d\psi_k,d\psi_k) a^h_k +h^{-1} i \left[g(d\psi_k,d a_k^h)+a_k^h\square_g\psi_k \right]+\square_g a_k^h\,.
\]
We have already seen that by choosing $\psi_k$ as above, we will be able to ensure that the $C^m$ form of the leading term $g(d\psi_k,d\psi_k)$ is of order $h^M$. We now focus on the remaining terms:
\begin{multline*}
	h^{-1} i \big[ g(d\psi_k,d a_k^h)+a_k^h\square_g\psi_k \big ]+\square_g a_k^h= ih^{-1}\big[2g(d\psi_k,da_{k,0})+a_{k,0}\square_g\psi_k\big ]\\
	+\sum_{j=1}^{J} ih^{j-1}\big[2g(d\psi_k,da_{k,j})+a_{k,j}\square_g\psi_k-i\,\square_g a_{k,j-1}\big ]
	 +h^J \square_g a_{k,J}\\
	 =:\sum_{j=-1}^J ih^j H_j\,.
\end{multline*}

Our objective now is to pick~$J$ and $a_{k,j}$ so that the condition~\eqref{E.CmM} holds. Noting that
\[
g(d\psi_k,a_{k,j})(y_k(s))=\frac {\pd p}{\pd \xi_\mu}\frac{\pd }{\pd y^\mu}a_{k,j}(y_k(s))= \frac d{ds}a_{k,j}(y_k(s))\,,
\]
one finds that $a_{k,j}(y_k(s))$ is determined for all $s_{k-1}\leq s\leq s_k$ once one prescribes $b_{k,j,0}:= a_{k,j}(y_k(s_{k-1}))$. Specifically, $a_{k,j} (y_k(s))$ is determined by solving the ODEs
\begin{align*}
	\left(2\frac d{ds}+ \square_g\psi_k(y_k(s))\right) a_{k,j}(y_k(s))=i\,\square_g a_{k,j-1}(y_k(s))\,,\qquad &a_{k,j,0}(y_k(s_{k-1}))=b_{k,j,0}
\end{align*}
for $0\leq j\leq J$, with the proviso that $a_{k,-1}:=0$. In particular, 
\begin{equation}\label{E.ak0}
	a_{k,0}(y_k(s))=b_{k,0,0}\, e^{-\int_{s_{k-1}}^s \square_g\psi_k(y_k(\bar s)) \, d\bar s}
\end{equation}
does not vanish anywhere unless it is identically~0. A similar argument determines $\pd_y^\al a_{k,j}(y_k(s_{k-1}))$ in terms of constants $b_{k,j,\al}:= \pd_y^\al a_{k,j}(y_k(s_{k-1}))$ that can be chosen freely up to the obvious compatibility conditions~\cite{GB}.

Our next objective is to ensure that $u^h$ approximately satisfies zero Dirichlet boundary conditions in a certain spacetime region. As we will see, this will have the effect that for $k\geq 2$ the constants $\cM_k,c_{k,\al},b_{k,j,\al}$ are effectively determined by their $k=1$ counterparts.

To specify the conditions, let use denote by
\[
X_{(T',T)}:= \{x\in X: T'<t(x)<T\}\,.
\]
the open subset of~$X$ delimited by the time slices $X_{T'}$ and $X_{T}$, with $T'<T$, and recall that $X_T:=\{x\in X: t(x)=T\}$. Furthermore, we will use the notation
$\Ga_{(T',T)}:= \Ga\cap X_{(T',T)}$.

Let us now set
\begin{equation}\label{E.defT}
z':= \ga_1(s_0)\,,\quad z:= \ga_{K-1}(s_{K-1})\,,\quad T':= t(z')-\ep \,,\quad T:= t(z)+\ep \,,	
\end{equation}
where $\ep>0$ is smaller than $\frac15[t(\ga_K(s_K)) - t(z)]$. 
We want to ensure that, for some small $\ep>0$ that does not depend on~$h$, 
\begin{equation}\label{E.CmMGa}
	\|u^h\|_{C^m(\Ga_{(T',T)})}\lesssim h^M\,,
\end{equation}
where $M$ is a large positive integer.
This means that $u^h$ approximately satisfies the zero Dirichlet condition for $T'+\ep <t<T-\ep$. This boundary condition can (and will) fail near the points $z'$ and $\ga_K(s_K)$.

To control the boundary conditions, we note that $u^h|_{\Ga_{(T',T )}}$ is essentially supported near the points $\ga_k (s_{k-1})$ for $1\leq k\leq K$, and that near each of these points the only summands in~\eqref{E.uh} that contribute are those of indices $k$ and $k-1$. To take care of the rapidly oscillating exponential factors, one imposes
\begin{equation}\label{E.exp}
	\psi_{k-1}=\psi_k \qquad \text{on }\Ga_{k-1}:=\supp(\chi_{k-1}^{h_0}\chi_k^{h_0}) \cap \Ga 
\end{equation}
for some small but fixed $h_0>0$. It is standard~\cite{GB} that Equation~\eqref{E.exp} is consistent with the assumption that 
\begin{equation}\label{E.dpsi2}
	d\psi_{k-1}(y_{k-1}(s))=\xi_{k-1}(s)\,,\qquad d\psi_k(y_k(s))=\xi_k(s)\,,
\end{equation}
cf.\ Equation~\eqref{E.dpsi},
because the geodesic segment $\ga_k$ is obtained from $\ga_{k-1}$ by means of Snell's refection law. Of course, the identity~\eqref{E.exp} yields the tangential derivatives of $\psi_k$ on~$\Ga_{k-1}$ in terms of those of~$\psi_{k-1}$.  Derivatives involving a normal direction can be computed using~\eqref{E.dpsi2}. In particular, it follows from~\eqref{E.dpsi2} and Snell's law~\eqref{E.reflect} that the normal derivative of the phase function, denoted as
\[
\pd_\nu\psi_k:= g(\nu,\nabla\psi_k)\,,
\]
at $z_{k-1}:=y_{k-1}(s_{k-1})=y_k(s_{k-1})$ satisfies
\begin{equation}\label{E.pdnu}
	\pd_\nu \psi_{k-1}(z_{k-1})=g(\nu,\dot y_{k-1}(s_{k-1}))= - g(\nu,\dot y_k(s_{k-1}))= -\pd_\nu \psi_k(z_{k-1})\,.
\end{equation}
It is also well known that the requirement~\eqref{E.exp} implies that the constants $\cM_k,c_{k,\al}$ are determined by the behavior of $\psi_{k-1}$ on the boundary. To put it differently, the Taylor series of~$\psi_k$ on~$y_k$ is determined, to arbitrarily high orders, by that of~$\psi_{k-1}$ on~$y_{k-1}$.

To determine the constants $b_{k,j,\al}$ appearing in the Taylor series of~$a_k^h$ on $y_k$, we need to use the boundary condition. By~\eqref{E.exp}, one has
\[
u^h|_{\Ga_{k-1}}= e^{i\psi_k/h}\big(a_k^h +a_{k-1}^h\big)= e^{i\psi_k/h}\sum_{j=0}^J h^j \big(a_{k,j}+a_{k-1,j}\big)\,.
\]
It is not hard to see that the assumption~\eqref{E.CmMGa} determines the coefficients $b_{k,j,\al}$ in terms of the Taylor series of~$\psi_{k-1}$ at $z_{k-1}$, to an arbitrarily large order (depending on~$m,M$). In particular, one necessarily has
\begin{equation}\label{E.ba}
	b_{k,j,0}= a_{k,j}(z_{k-1})=-a_{k-1,j}(z_{k-1})\neq0
\end{equation}
as a consequence of the formula~\eqref{E.ak0}.
Explicit expressions for~$b_{k,j,\al}$ can be obtained by differentiating the relevant identities~\cite{GB}.


We are now ready to complete the construction of the family of approximate solutions~$u^h$, following~\cite{GB}. We start by choosing a phase function~$\psi_1$ whose the Taylor series on~$y_1(s)$ is given (to a large order~$J$) by constants $\cM_1, a_{1,j}$ that one can pick freely. We also choose constants $b_{1,j,\al}$, with $b_{1,0,0}\neq0$, and pick functions $\ta_{1,j}$ with that Taylor series (to large order). This determines $\psi_2|_{\Ga_1}$ and the constants $\cM_2, c_{2,\al}$ and $b_{2,j,\al}$ that specify the Taylor series of $\psi_2$ and $\ta_{2,j}$ on $y_2(s)$. One then chooses functions~$\psi_2$ and~$\ta_{2,j}$ with that Taylor series. The argument is repeated~$K$ times, obtaining a well defined Gaussian beam of the form~\eqref{E.uh} which satisfies~\eqref{E.CmM} and~\eqref{E.CmMGa}. 

The only remaining point to explain is why one can introduce the $h$-dependent cutoff functions $\chi_k^h(s):= \chi(h^{-\si} d_k(x))$ in a way that $a_{k,j}:= \ta_{k,j}\chi_k^h$ enjoys the same properties as $\ta_{k,j}$ for the purposes of the construction of Gaussian beams. The idea is that, as $\chi(\la)=1$ for $\la<\frac12$, for any multiindex $\al\neq0$, one has
\[
\big|e^{i\psi_k/h}\pd_y^\al  \chi_k^h\big|\leq e^{-cd_k^2/h} |\pd_y^\al  \chi_k^h|\leq e^{-c'h^{2\si-1}}|\pd_y^\al  \chi_k^h|\leq Ce^{-c'h^{2\si-1}}h^{-|\al|\si}\lesssim h^M
\]
for any~$M$.
Here we have used the Gaussian-type estimate~\eqref{E.Gauss} and the fact that $\si<\frac12$.

Let us now compute the normal derivative of~$u^h$ on~$\Ga_{K-1}$. Since $\psi_{K-1}=\psi_K$ on~$\Ga_{K-1}$ by~\eqref{E.exp}, one has
\begin{align*}
	\pd_\nu u^h|_{\Ga_{K-1}} &= e^{i\psi_{K-1}/h}\big( ih^{-1}\pd_\nu \psi_{K-1}\, a_{K-1}^h + \pd_\nu a_{K-1}^h \big)+ e^{i\psi_{K}/h}\big( ih^{-1}\pd_\nu \psi_{K}\, a_{K}^h + \pd_\nu a_{K}^h \big)\\
	&= e^{i\psi_{K-1}/h}\big[ ih^{-1}\big(\pd_\nu\psi_{K-1}\, a_{K-1,0}+ \pd_\nu\psi_{K}\, a_{K,0}\big) +O(1)\big]\,.
\end{align*}
As $a_{k,0}|_{\Ga_{K-1}}= \ta_{k,j}\chi_k^h$ is supported on a set of diameter $h^\si$, with $\si\in(0,\frac12)$, by Equations~\eqref{E.pdnu}-\eqref{E.ba} one has
\[
\pd_\nu\psi_{K-1}\, a_{K-1,0}+ \pd_\nu\psi_{K}\, a_{K,0}= 2g(\nu,\dot y_{k-1}(s_{k-1}))\, a_{K-1,0}(z_{K-1})+ O(h^{\si})\,.
\]
Therefore, 
\begin{align*}%\label{E.final}
	\pd_\nu u^h|_{\Ga_{K-1}} &= 2ih^{-1}e^{i\psi_{K-1}/h}\big[ g(\nu,\dot y_{k-1}(s_{k-1}))\, a_{K-1,0}(z_{K-1})+ O(h^\si)\big]\,.
\end{align*}	
Furthermore, note that $g(\nu,\dot y_{k-1}(s_{k-1}))\neq0$ because of the null convexity of the boundary, cf.~Equation~\eqref{E.signgeod}, while $a_{K-1,0}(z_{K-1})\neq0$ by~\eqref{E.ba}. This immediately translates into the following estimates, which lie at the heart of the proof of Theorem~\ref{T.main}. To state them, we shall use the notation $q_1\approx q_2$ if there exists an $h$-independent constant such that $C^{-1}q_1\leq q_2\leq Cq_1$.

\begin{lemma}\label{L.estimates}
	Let $z,z',T$ and $T'$ be defined as in~\eqref{E.defT}, and let $U,U'$ respectively be small but fixed open neighborhoods of the points $z,z'$ in~$\Ga$. Then, uniformly for $h\ll1$,
	\begin{align}
		\|u^h\|_{L^2(U')} &\approx h^{\frac d4}\,,\label{E.L2U'}\\
		\|u^h\|_{H^m(\Ga_{(T',T)}\backslash U')} &\lesssim h^M\,,\label{E.HmGa}\\
		\|\pd_\nu u^h\|_{L^2(U)}&\approx h^{-1+\frac d4}\,,\label{E.L2U}\\
		\|\pd_\nu u^h\|_{L^2(U')}&\approx h^{-1+\frac d4}\,,\label{E.pdL2U'}\\
		\|\square_g u^h\|_{H^m(X_{(T',T)})}&\lesssim h^M\,,\label{E.Hmsquare}\,.
	\end{align}
\end{lemma}

\begin{proof}
	Let us cover $U'$ using normal coordinates $Z=(Z^1,\dots, Z^d)$ of the metric $g^+|_{\Ga}$, centered at the point~$z'$. As the neighborhoods are small, one can assume that $\{|Z|<c_1\}\subset U\subset \{|Z|<c'_1\}$ for certain  small positive constants. Likewise, there are constants such that
	\[
	\{|Z|<ch^\si\}\subset U'\cap \supp\chi_k^h\subset \{|Z|<c'h^\si\}\,.
	\]
	As we have shown that
	\[
	a_1^h=[b_{1,0,0}+O(h^\si)]\chi_1^h
	\]
	with $b_{1,0,0}\neq0$ and there is a positive constant such that
	\[
	e^{-C|Z|^2/h}\chi_1^h\leq |e^{i\psi_1/h}|\chi_1^h\leq e^{-|Z|^2/(Ch)}\chi_1^h
	\]
	by~\eqref{E.Gauss}, one readily obtains that
	\begin{align*}
		\|u^h\|_{L^2(U')}^2 \approx \int_{|Z|<ch^\si} e^{-c|Z|^2/h}\, dZ\approx h^{\frac d2}\int_{|Z'|<ch^{\si-\frac12}} e^{-c|Z'|^2}\, dZ'\approx h^{\frac d2}\,.
	\end{align*}
	Here we have used that $\si<\frac12$. The estimate~\eqref{E.L2U'} is then proven.
	
	The proof of~\eqref{E.L2U} is completely analogous, using the formula~\eqref{E.pdnu} for $\pd_\nu u^h|_{\Ga_{K-1}}$ and normal coordinates centered at~$z'$. The estimate~\eqref{E.L2U'} follows from an analogous reasoning as well. To conclude, \eqref{E.HmGa} and~\eqref{E.Hmsquare} are immediate in view of the estimates~\eqref{E.CmM} and~\eqref{E.CmMGa}. 
\end{proof}



 
\section{The singular support of the kernel of the Dirichlet-to-Neumann map}
\label{S.singsupp}

In this section we will use the family of approximate solutions constructed in Section~\ref{S.GB} to show that the Schwartz kernel $\La_g(\cdot,\cdot)\in \sD'(\Ga\times\Ga)$ of the Dirichlet-to-Neumann map fails to be smooth exactly on the boundary light observation set of the manifold. Throughout, we assume that the manifold $(X,g)$ is admissible in the sense of Definition~\ref{D.admissible}.

The key ingredient of the proof is the following lemma. Here the space $L^\infty$ is certainly not optimal, but it suffices for our purpose.


\begin{lemma}\label{L.Linfty}
	Suppose that there is a future-oriented  broken null geodesic that emanates from a point $z'\in \Ga$ and passes through a distinct point $z\in\Ga$, and let $U,U'\subset\Ga$ be arbitrary open neighborhoods of~$z,z'$. Then the restriction of the kernel $\La_g(\cdot,\cdot)$  to the set~${U\times U'}$ (respectively, to $U'\times U'$) is not in $L^\infty(U\times U')$ (respectively, in $L^\infty(U'\times U')$).
\end{lemma}
	
\begin{proof}
	To prove that $\La_g(\cdot,\cdot) \not\in L^\infty(U\times U')$, we will use the notation of Section~\ref{S.GB}. Hence $\ga$ is a broken null geodesic going from $z'$ to~$z$. It consists of $K$ null geodesic segments $\ga_1,\dots\ga_K$, with $K\geq2$; one has $z'=\ga_1(s_0)$ and $z=\ga_{K-1}(s_{K-1})$. The times~$T,T'$ are defined as in~\eqref{E.defT}.

Let us now consider a smooth temporal cutoff of the form
\[
\theta(x):= \Theta(t(x))\,,
\]
where $\Theta(\tau)$ equals~1 for $\tau < T+\frac12\ep$ and~0 for $\tau>T+2\ep$. As $\chi_k^h|_{\Ga}$ is identically zero outside the balls of radius $h^\si$ centered at the points $\ga_k(s_{k-1})$ and $\ga_k(s_k)$,  the Dirichlet trace of the function $\theta u^h$ coincides with that of $u^h$ for $t<T$ and is zero for $t>T$. 

Furthermore, one can write
\[
\square_g(\theta u^h)= \theta \square_g u^h + 2g(d\theta,du^h)+ u^h \square_g \theta=: \theta \square_g u^h + R^h\,.
\]
The first summand is bounded as 
\begin{equation}\label{E.boundtesquare}
	\|\theta \square_g u^h\|_{H^m(X)}\lesssim h^M
\end{equation}
by~\eqref{E.Hmsquare}, so it is a small error term for $h\ll1$. In contrast, the second error term is large: as $\pd^\al u^h$ is of order $h^{-|\al|}$ and $u^h$ is supported on a tube of width $h^\si$ along a null geodesic and has some Gaussian-type decay, one can argue as in the proof of Lemma~\ref{L.estimates} to show that the best estimate one can prove for $R^h\in C^\infty_c(X)$ is
\[
\|R^h\|_{H^m(X)}\lesssim h^{m+\frac d4}\,.
\]
But the redeeming feature of this error is that, by construction, the support of $R^h$ is contained in the set $t>T+\ep$.

For small enough~$h$, it is clear that 
\[
f^h:= u^h|_{U'}\in C^\infty_c(U')\,,
\]
as the cutoff~$\chi_1^h$ identically zero on~$U'$ outside a ball of radius $h^\si$. The rest of the Dirichlet trace of the cut off Gaussian beam,
\[
B^h:= \theta u^h|_{\Ga}-f^h\,,
\]
is in $C^\infty_c(\Ga \backslash U')$ and bounded as
\begin{equation}\label{E.boundB}
	\|B^h\|_{H^m(\Ga)}\lesssim h^M
\end{equation}
by~\eqref{E.HmGa}.

Now consider the unique solutions $\sB^h,\sR^h$ to the Cauchy problems
\begin{gather*}
	\square_g \sB^h=\theta \square_g u^h  \quad\text{in } \rX\,,\qquad 
	\sB^h= B^h  \quad \text{on } \Ga \,, \qquad \sB^h=\pd_t \sB^h=0 \quad \text{on } X_{T'}\\
		\square_g \sR^h=R^h  \quad\text{in } \rX\,,\qquad 
	\sR^h= 0 \quad \text{on } \Ga \,, \qquad \sR^h=\pd_t \sR^h=0 \quad \text{on } X_{T}\,.
\end{gather*}
Here we are using the notation
\[
\pd_t \vp:= g(dt,d\vp)\,.
\]
In view of the bounds~\eqref{E.boundtesquare}-\eqref{E.boundB}, standard energy estimates imply that
\[
\|\sB^h\|_{H^m(X_{(T',T)})}\lesssim h^M\,.
\]
The function~$\sR^h$ is of course not small, but the fact that $R^h$ is supported on $t>2T+\ep$ implies that
\[
\sB^h=0 \quad \text{in } X_{(T',T)}\,.
\]

Let us now define
\[
v^h:=\theta u^h - \sB^h-\sR^h\,.
\]
By construction, $v^h$ is the unique solution to the problem
\[
\square_g v^h=0\quad\text{in } \rX\,,\qquad v^h=f^h\quad \text{on } \Ga \,, \qquad v^h=0\quad \text{for }t<T'\,,
\]
so $\pd_\nu v^h=\La_gf^h$.

If $\La_g(\cdot,\cdot)\in L^\infty(U\times U')$, denoting the area measure on~$\Ga$ by~$dz$, one has
\begin{align}
	\|\pd_\nu v^h\|_{L^2(U)}^2 &= \|\La_g f^h\|_{L^2(U)}^2= \int_{U\times U'\times U'}\overline{ \La_g(z,z')f^h(z')}\,{\La_g(z,z'') f^h(z'')}\, dz\, dz'\, dz''\notag\\
	&\leq\|\La_g\|_{L^\infty(U\times U')}^2  \int_{U\times U'\times U'}\left[|f^h(z')|^2+ |f^h(z'')|^2\right]\, dz\, dz'\, dz''\notag \\
	&\leq 2|U||U'| \|\La_g\|_{L^\infty(U\times U')}^2\|f^h\|_{L^2(U')}^2\,.\label{E.final2}
\end{align}
Since
\begin{gather*}
	\|\pd_\nu v^h\|_{L^2(U)}\gtrsim \|\pd_\nu v^h\|_{L^2(U)} - \|\sB^h\|_{H^m(X_{(T',T)})} \approx  h^{-1+\frac d4}\,,
	\end{gather*}
	and
	\[
		\|f^h\|_{L^2(U')}\lesssim \|u^h\|_{L^2(U')}+ \|B^h\|_{L^2(U')} \lesssim h^{\frac d4}
	\]
by the estimates for $B^h$, $\sB^h$ and $u^h$ (cf.~Lemma~\ref{L.estimates}) and by the trace inequality, the bound~\eqref{E.final2} reads as
\[
\|\La_g\|_{L^\infty(U\times U')}\gtrsim h^{-1}
\]
for any $h\ll1$. This is a contradiction, so the claim follows. The fact that $\La_g(\cdot,\cdot)\not\in L^\infty(U'\times U')$ (which we will not use) also follows from the same reasoning and the estimates in Lemma~\ref{L.estimates}, using now~\eqref{E.L2U'} instead of~\eqref{E.L2U}.
\end{proof}


Armed with the above key lemma, we are ready to present the main technical result we will need later on. To state it in a particularly convenient fashion, we need to introduce some notation. First, given two points $(z,z')\in\Ga\times\Ga$, we will write
\begin{equation}\label{order}
	z' \pre z
\end{equation}
if $z\neq z'$ and there exists a future-oriented broken null geodesic that emanates from~$z'$ and passes through~$z$. As the Lorentzian manifold $(X_j,g_j)$ admits a global time function, it is standard (see e.g.~\cite{Penrose}) that this relation clearly defines a partial order on $\Ga$.
We will also use the notation
\[
z'\preq z
\]
if either $z=z'$ or $z'\pre z$, and consider the set
\begin{equation}\label{cA}
\cA_g:=\{(z,z')\in \Ga\times\Ga : z'\preq z\}\,. 	
\end{equation}


\begin{theorem}\label{T.singsupp}
	The singular support of $\La_g(\cdot,\cdot)$ is~$\cA_g$.
\end{theorem}

\begin{proof}
Standard results about propagation of singularities imply that 
\[
\singsupp \La_g(\cdot,\cdot)\subset \cA_g\,.
\]
Conversely, Lemma~\ref{L.Linfty} shows that $\La_g(\cdot,\cdot)$ is unbounded on any neighborhood of any pair $(z,z')\in \cA_g$, so
\[
\cA_g\subset \singsupp \La_g(\cdot,\cdot)\,.
\]
Thus the theorem follows.
\end{proof}




\section{Proof of Theorem~\ref{T.main}}
\label{S.main}


The starting point of the proof is that, by Theorem~\ref{T.singsupp}, the Dirichlet-to-Neumann map $\La_j$ determines the set~\eqref{cA} defined by the relation $\pre$ in each light-sensitive (in the sense of Definition~\ref{D.ls}) Lorentzian manifold with boundary $(X_j,g_j)$, with $j=1,2$. Let us denote this set by $\cA_j\subset\Ga_j\times\Ga_j$. The assumption that $\La_1$ and~$\La_2$ are conjugated through a diffeomorphism $\Phi:\Ga_1\to\Ga_2$ can be then used to establish the following fact:


\begin{lemma}\label{L.cAj}
	Under the hypotheses of Theorem~\ref{T.main}, for all $z,z'\in\Ga_1$, $z'\pre z$ if and only if $\Phi(z')\pre \Phi(z)$. In particular, $(z,z')\in\cA_1$ if and only if $(\Phi(z),\Phi(z'))\in\cA_2$, and given any subset $A_2\subset \cA_2$ there exists some $A_1\subset\cA_1$ such that $A_2=\Phi(A_1)$.
\end{lemma}

\begin{proof}
	As $\Phi_*\La_1= \La_2\Phi_*$, the Schwartz kernels of the Dirichlet-to-Neumann maps are related by the identity
	\[
	\La_1(z,z')= J_\Phi(z')\, \La_2(\Phi(z),\Phi(z'))\,,
	\]
	where the Jacobian $J_\Phi$ is a $C^\infty$ function. Therefore, $(z,z')\in\singsupp \La_1(\cdot,\cdot)$ if and only if
	\[
	(\Phi(z),\Phi(z'))\in\singsupp \La_2(\cdot,\cdot)\,.
	\]
	The lemma then follows from this relation and Theorem~\ref{T.singsupp}.
\end{proof}

Our objective now is to describe, in very concrete terms, how the structure of the set~$\cA_j$ is related to the causal structure of the manifold $(X_j,g_j)$, as described by the cuts of the null cones with the boundary~$\Ga_j$. To spell out the details we will need to define several geometric objects.

First, we will say $(z,z')\in\cA_j$ is {\em irreducible}\/ if the only points $z''\in\Ga_j$ such that $z'\preq z''$ and $z''\preq z$ are $z''=z$ and $z''=z'$. To put it differently, as there exists a smooth time function on~$(X_j,g_j)$, $(z,z')$ is irreducible if there is a future-oriented null geodesic segment, which we will henceforth denote by $\ga(z,z')$, going from~$z'$ to~$z$. The interior of this geodesic segment will be denoted by
\[
\gacirc(z,z'):=\ga(z,z')\backslash\{z,z'\}\,.
\]




We will denote by $\cAirr_j$ the set of irreducible points of~$\cA_j$.
\begin{definition}\label{D.maximal}
	A subset $A\subset \cAirr_j$ is {\em maximal}\/ if:
	\begin{enumerate}
		\item The subset of $X_j^\circ$ given by
		\begin{equation}\label{E.defsX}
			 Q_j(A):= \bigcap_{(z,z')\in A} \gacirc(z,z')
		\end{equation}
	is not  empty.\vspace{1mm}
	
	\item If $(z,z')\in\cAirr_j\backslash A$, 
	\[
	 Q_j(A)\cap \gacirc(z,z')=\emptyset\,.
	\]
	\end{enumerate}	
\end{definition}



\begin{proposition}\label{P.singleton}
	If $A\subset\cAirr_j$ is maximal, $ Q_j(A)$ is a singleton.
\end{proposition}
\begin{proof}
By~\eqref{E.defsX}, $ Q_j(A)$ is nonempty. Now suppose there are two distinct points $p,q\in  Q_j(A)$. By condition~\ref{I.miss} in Definition~\ref{D.ls}, we can take a null geodesic segment that reaches the boundary and passes through~$p$ but not through~$q$. By condition~\ref{I.endp} in Definition~\ref{D.ls}, let $(z,z')$ be the endpoints of this segment, so we can write it as $\ga(z,z')$. Then $(z,z')\not\in A$ because $q\in  Q_j(A)$ but $q\not\in\gacirc(z,z')$. However, $p\in  Q_j(A)\cap \gacirc(z,z')$. This contradicts the assumption that the set~$A$ is maximal.
\end{proof}



Therefore, by Proposition~\ref{P.singleton}, with some abuse of notation we can assume that $ Q_j$ maps each maximal set $A\subset\cAirr_j$ to a point $ Q_j(A)\in \rX_j$.

\begin{figure}
\begingroup%
  \makeatletter%
  \providecommand\color[2][]{%
    \errmessage{(Inkscape) Color is used for the text in Inkscape, but the package 'color.sty' is not loaded}%
    \renewcommand\color[2][]{}%
  }%
  \providecommand\transparent[1]{%
    \errmessage{(Inkscape) Transparency is used (non-zero) for the text in Inkscape, but the package 'transparent.sty' is not loaded}%
    \renewcommand\transparent[1]{}%
  }%
  \providecommand\rotatebox[2]{#2}%
  \ifx\svgwidth\undefined%
    \setlength{\unitlength}{75.72136671bp}%
    \ifx\svgscale\undefined%
      \relax%
    \else%
      \setlength{\unitlength}{\unitlength * \real{\svgscale}}%
    \fi%
  \else%
    \setlength{\unitlength}{\svgwidth}%
  \fi%
  \global\let\svgwidth\undefined%
  \global\let\svgscale\undefined%
  \makeatother%
  \begin{picture}(1,1.79398271)%
    \put(0,0){\includegraphics[width=\unitlength,page=1]{drawing2.pdf}}%
    \put(0.29860093,0.91080935){\color[rgb]{0,0,0}\makebox(0,0)[lb]{\smash{$p$}}}%
    \put(0.48853231,1.16427806){\color[rgb]{0,0,0}\makebox(0,0)[lb]{\smash{$\sB^+_j(p)$}}}%
    \put(0.57098123,0.37666122){\color[rgb]{0,0,0}\makebox(0,0)[lb]{\smash{$\sB^-_j(p)$}}}%
  \end{picture}%
\endgroup%
	\caption{Null cone cuts $\sB^\pm_j(p)$ of $p\in X^\circ_j$. \label{F.BG2}}
\end{figure}


The next step of the proof is to consider the geometric interpretation of maximal sets. Again, to do so we need to recall some notation we introduced in Section~\ref{S.geom}. Given a point $p\in X_j$, we will denote by $\sB_j^+(p)$ the intersection of the future null cone of~$p$ (without reflections) with~$\Ga_j$. That is, $\sB_j^+(p)$ consists of the future endpoints of all future-oriented inextensible null geodesics emanating from~$p$. Likewise, the intersection of the past null cone of~$p$ (without reflections) with~$\Ga_j$ will be denoted by~$\sB_j^-(p)$; see Figure~\ref{F.BG2}. We also set
\[
\sB_j(p):=\sB_j^+ (p)\times \sB_j^-(p)\subset \Ga_j\times \Ga_j\,.
\]

The key property of maximal subsets of~$\cAirr_j$ is that they coincide with null cone cuts on~$\Ga_j$, and $ Q_j$ maps each maximal element $A$ to the vertex point~$p= Q_j(A)$ of the corresponding null cone cut $\sB_j(p)=A$. Therefore, by item~\ref{I.sB} in Definition~\ref{D.ls}, $ Q_j$ and~$\sB_j$ are in a way inverse to each other. More precisely, one has the following:

\begin{proposition}\label{P.maximal}
		$A\subset\cAirr_j$ is maximal if and only if $A=\sB_j(p)$ for some $p\in\rX_j$. Furthermore, $p= Q_j(A)$.
\end{proposition}

\begin{proof}
	Let us start by showing that, given any point $p\in\rX_j$, $\sB_j(p)$ is contained in~$\cAirr_j$. This is immediate because a point $z'\in\Ga_j$ is in $\sB_j^-(p)$ if and only if there is a future-oriented null geodesic $\ga$ starting at~$z'$ which passes through~$x$. By condition~\ref{I.endp} in Definition~\ref{D.ls}, let $z$ if the future endpoint of~$\ga$, then $(z,z')\in\sB_j(p)$ and $z'\pre z$. The argument for a point $z\in\sB_j^+(p)$ is the same, with ``past'' replacing ``future''.
	
	As $\sB_j(p)$ is given by boundary cuts of the null cone of~$p$ (without reflections),
	\[
	 Q_j[\sB_j(p)]=p\,.
	\]
	Therefore, a null geodesic segment $\gacirc(z,z')$ with $z,z'\in\Ga$ has nontrivial intersection with $ Q_j[\sB_j(p)]$ if and only if it belongs to the null cone of~$p$ (without reflections). Thus $(z,z')\in\sB_j(p)$, which shows that the set $\sB_j(p)\subset\cAirr_j$ is maximal.
	
	To prove the converse implication, assume now that $A\subset\cAirr_j$ is maximal and set $p:= Q_j(A)$. If $\ga$ is a null geodesic passing through~$p$ that has an endpoint on~$\Ga$, by condition~\ref{I.endp} it actually has two endpoints on~$\Ga$, and can therefore be written as $\ga(z,z')$. Then $(z,z')\in A$, so necessarily $\sB_j(p)\subset A$. But we have proved that $\sB_j(p)$ is a maximal subset of~$\cAirr_j$ for any~$p\in\rX_j$, so the inclusion cannot be strict. Therefore $A=\sB_j(p)$ and the claim follows.	
\end{proof}

An important consequence of the results we have established is the following. To state it in a more convenient say, given a diffeomorphism $\Phi:\Ga_1\to\Ga_2$, here and in what follows we will denote by $\tPhi:\Ga_1\times \Ga_1\to\Ga_2\times \Ga_2$ the map
\[
\tPhi(x,y):=(\Phi(x),\Phi(y))\,,
\]
with inverse $\tPhi^{-1}(x,y):=(\Phi^{-1}(x),\Phi^{-1}(y))$.

\begin{proposition}\label{P.needed}
	Under the hypotheses of Theorem~\ref{T.main}, for each $x_2\in\rX_2$ there exists a unique $x_1\in\rX_1$ such that $\sB_2(x_2)=\tPhi[\sB_1(x_1)]$. Furthermore, one can write $x_2=\Psi(x_1)$, where $\Psi:\rX_1\to \rX_2$ is the invertible map
	\begin{equation}\label{E.defPsi}
		\Psi:= Q_2\circ\tPhi\circ \sB_1\,.
	\end{equation}
\end{proposition}

\begin{proof}
	In view of Lemma~\ref{L.cAj} and Proposition~\ref{P.maximal}, we only need to show that if $A_2$ is a maximal subset of $\cAirr_2$ and $A_2=\Phi(A_1)$, then $A_1\subset \cAirr_1$ and $A_1$ is also maximal. But this is in fact a straightforward consequence of the fact that, by Lemma~\ref{L.cAj}, $\Phi$ preserves the partial order. By this lemma and condition~\ref{I.sB}, the maximal subset~$A_1$ and the point $x_1$ are unique. 
	
	The proof remains valid if one replaces the roles of $X_1$ and $X_2$. In this case, $\sB_1(x_1)= \tPhi^{-1}[\sB_2(x_2)]$, with $x_2:=\Psi^{-1}(x_1)$. Here $\Psi^{-1}:= Q_1\circ\tPhi^{-1}\circ \sB_2$ is the inverse of~$\Psi$. 
\end{proof}


Note that to compute~$x_2:=\Psi(x_1)$, as defined in~\eqref{E.defPsi}, one considers the null cone cut associated with this point in the first manifold, $\sB_1(x_1)$, maps it to another null cone cut (now on the second manifold) using the boundary diffeomorphism~$\Phi$, and defines $x_2$ in~$X_2$ as the point to which this second null cone cut corresponds. (Throughout, note we are keeping track of the future and past cones separately by means of the sets $\sB_j(x)$.) This is completely analogous to the conformal diffeomorphism constructed (under slightly different hypotheses and using the boundary cuts of future null cones instead of the sets $\sB_j(p)$) by Hintz and Uhlmann in~\cite[Theorem 3.3]{HU}. Therefore, our goal is to use the geometric results already derived in this paper to show that $\Psi$ is not only an invertible map but a conformal diffeomorphism:


\begin{proposition}\label{P.Psi}
	Under the hypotheses of Theorem~\ref{T.main}, the map~\eqref{E.defPsi} defines a conformal diffeomorphism $\Psi:(\rX_1,g_1)\to(\rX_2,g_1)$, which can be extended to the closure, $(X_1,g_1)\to(X_2,g_2)$, by setting $\Psi|_{\Ga_1}:=\Phi$.
\end{proposition}

\begin{proof}
Let us start by showing that $\Psi$ is a homeomorphism. To this end, let us start by defining a topology on the space
	\[
	\sC_j:=\{\sB_j(p): p\in X_j\}
	\]
by using the collection of sets of the form
	\[
	U_O:=\{C\in\sC_j: C\cap O\neq\emptyset\}\,,\qquad U^K:=\{C\in\sC_j: C\cap K=\emptyset\}\,,
	\]
	where $O\subset\Ga_j$ is open and $K\subset\Ga_j$ is compact, as a subbasis. Thus we are only using~$\sC_j$ as a set, but we are using the a priori known topology on~$\Ga_j$. Replacing future null cone cuts by the sets $\sB_j(x)$, it follows from~\cite[Proposition 3.8]{HU} (which is the heart of the step~3.2 in the proof of their main result, \cite[Theorem~3.3]{HU}) that $Q_j:\sC_j\to\rX_j$ is a homeomorphism with inverse $\sB_j:\rX_j\to\sC_j$. Note that the assumption that on the boundary there are no conjugate points along null geodesics is used crucially here.
	
	We therefore conclude that $\Psi= Q_2\circ\tPhi\circ\sB_1$ and $\Psi^{-1}= Q_1\circ\tPhi^{-1}\circ \sB_2$ are continuous, so $\Psi:\rX_1\to\rX_2$ is a homeomorphism. As a consequence of the explicit expressions $\Psi=Q_2\circ \tPhi\circ\sB_1$ and $\Psi^{-1}=Q_1\circ\tPhi^{-1}\circ\sB_2$, these functions can be extended as continuous maps $X_1\to X_2$ and $X_2\to X_1$, respectively, by setting
	\[
	\Psi|_{\Ga_1}:=\Phi\,,\qquad \Psi^{-1}|_{\Ga_2}:=\Phi^{-1}\,.
	\]
	These extensions still satisfy that $\Psi^{-1}\circ\Psi$ and $\Psi\circ\Psi^{-1}$ are the identity map on~$X_1$ and~$X_2$, respectively, so the extension $\Psi:X_1\to X_2$ is a homeomorphism.
	
The steps 3.3 and 3.4 in the proof of~\cite[Theorem 3.3]{HU} (again using the sets $\sB_j(x)$ instead of the boundary cuts of the future null cone) now show that $\Psi$ is smooth and preserves the conformal structure of the manifold. The proposition then follows.
\end{proof}


Theorem~\ref{T.main} is therefore proven.


\section{Proof of Theorem~\ref{T.metric}}
\label{S.cf}


We start by noting that Theorem~\ref{T.main} ensures the existence of a conformal diffeomorphism $\Psi: (X_1,g_1)\to (X_2,g_1)$ with $\Psi|_{\Ga_1}=\Phi$. As we discussed in the Introduction, it is then standard that recovering the conformal factor is then equivalent to recovering a potential term for the wave equation on~$(X_1,g_1)$.

Let us recall the details here. Writing $(X,\Ga,g)\equiv (X_1,\Ga_1, g_1)$, the fact that the manifolds are conformal means that there exists a positive function $c\in C^\infty(X)$ such that $\Psi ^*g_2= c g$. Furthermore, 
\begin{align*}
		\square_{g_2} u&=0\qquad  \text{in } \rX_2\,,\\
		u=f\circ &\Phi^{-1}\qquad \text{on } \Gamma_2\,,\\
		u&=0 \qquad \text{if } t\ll-1
	\end{align*}
if and only if the function $w:= c^{(n-1)/4} u\circ\Psi^{-1}$ satisfies
\begin{align*}
		\square_{g} w+ Vw&=0\qquad \text{in } \rX\,,\\
		w&=f\qquad \text{on } \Gamma\,,\\
		w&=0 \qquad \text{if } t\ll-1\,,
	\end{align*}
with $V:= -c^{-(n-1)/4}\square_g c^{(n-1)/4}$. Moreover, 
\[
g_2(\nabla u,\nu)\circ \Phi^{-1}= g(\nabla w,\nu)\,,
\]
where the covariant derivative in the leftmost term is computed using the metric~$g_2$.
Setting 
\[
\La_{g,V}f:= g(\nabla w,\nu)\,,
\]
one then concludes that recovering the conformal factor in Theorem~\ref{T.metric} is now equivalent to showing that $\La_{g,V}=\La_{g,0}$ implies that $V=0$.

That this is the case under the assumptions of Theorem~\ref{T.metric} follows from results of Alexakis, Feizmohammadi and Oksanen~\cite{Alex20,Alex21}. To formulate their theorems, let us recall their geometric hypotheses on the manifold $(X,g)$. They actually consider slightly more general manifolds, but this formulation is well suited for the cases we can handle using Theorem~\ref{T.main}. 
One should note~\cite{Alex21} that this Hypothesis is satisfied, in particular, when~$g$ is a $C^2$-small perturbation of the Minkowski metric.


\begin{hypothesis}\label{H.Alex}
	The manifold $(X,g)$ is admissible and satisfies the conditions:
	\begin{enumerate}
		\item For any null geodesic $\ga$ and any two points $p, q$ on $\ga$, the only causal path between $p$ and $q$ is along $\ga$. For all $p \in X$, the exponential map $\exp_p$ is a diffeomorphism from the spacelike vectors (in its maximal domain of definition) onto the complement in~$X$ of the causal cone of~$p$.
		\item 	All null geodesics are non-trapped.
		\item For any spacelike vector $v$ and any null vector $N$ in $T_pX$ with $g(v,N) = 0$, the curvature tensor satisfies
\[
g(R(N,v)v,N) \leq 0\,.
\]
	\end{enumerate}	
\end{hypothesis}
	
		
\begin{theorem}[\cite{Alex20,Alex21}]
\label{T.Alex}
	Assume that either $(X,g)$ satisfies Hypothesis~\ref{H.Alex}. If the potentials $V_j\in C^\infty(X)$ satisfy $\La_{g,V_1}=\La_{g,V_2}$, then $V_1=V_1$.
\end{theorem}	

Theorem~\ref{T.metric} then follows from Theorems~\ref{T.main} and~\ref{T.Alex}.







\section*{Acknowledgements}

This work has received funding from the European Research Council (ERC) under the European Union's Horizon 2020 research and innovation program through the Consolidator Grant agreement~862342 (A.E.). A.E.'s research is also partially supported by the grant CEX2019-000904-S. The research of G.U.\ is partly supported by NSF and a Robert R.~Phelps and Elaine F.~Phelps Professorship at University of Washington. M.W.~gratefully acknowledges support from the grant ANR-20-CE40-0018.


\bibliographystyle{amsplain}
\begin{thebibliography}{99}\frenchspacing

\bibitem{Ake}
L. Ak\'e, J.L. Flores, M. S\'anchez, Structure of globally hyperbolic spacetimes-with-timelike-boundary. Rev. Mat. Iberoam. 37 (2021) 45--94.

\bibitem{Alex20}
S. Alexakis, A. Feizmohammadi, L. Oksanen, Lorentzian Calder\'on problem under curvature bounds, Invent. Math. 229 (2022) 87--138.

\bibitem{Alex21}
S. Alexakis, A. Feizmohammadi, L. Oksanen, Lorentzian Calder\'on problem near the Minkowski geometry, J. Eur. Math. Soc., in press (2112.01663).

\bibitem{Alinhac}
S. Alinhac, Non-unicit\'e du probl\`eme de Cauchy, Ann. of Math. 117 (1983) 77--108.

\bibitem{B}
M. Belishev, Recent progress in the boundary control method, Inverse Problems 23 (2007)
R1--R67.
%
%\bibitem{BK}
%M. Belishev, Y. Kurylev, The reconstruction of the Riemanian manifolds via its spectral data, Comm. in PDE 17 (1992) 767--804.

\bibitem{phys}
N. Engelhardt, G.T. Horowitz, Towards a reconstruction of general bulk metrics, Class. Quant. Grav. 34 (2017) 015004.

\bibitem{19}
A. Feizmohammadi, M. Lassas, L. Oksanen, Inverse problems for non-linear hyperbolic equations with
disjoint sources and receivers, 2006.12158.

\bibitem{FO}
A. Feizmohammadi, L. Oksanen, Recovery of zeroth order coefficients in non-linear wave equations, J. Inst. Math.  Jussieu 21 (2022) 367--393. 

\bibitem{HU}
P. Hintz, G. Uhlmann, Reconstruction of Lorentzian manifolds from boundary light observation sets, Int. Math. Res. Not. 22 (2019) 6949--6987.



\bibitem{HUZ}
P. Hintz, G. Uhlmann, J. Zhai, An inverse boundary value problem for a semilinear wave equation on Lorentzian manifolds, Int. Math. Res. Not. 17 (2022) 13181--13211.


\bibitem{HUZ2}
P. Hintz, G. Uhlmann, J. Zhai, The Dirichlet-to-Neumann map for a semilinear wave equation on Lorentzian manifolds, Comm. PDE 47 (2022) 2363--2400.
%
%\bibitem{equivalence}
%A. Katchalov, Y. Kurylev, M. Lassas, N. Mandache, Equivalence of time-domain inverse problems and boundary spectral problems, Inverse Problems 20 (2004) 419.

\bibitem{KLU}
Y. Kurylev, M. Lassas, G. Uhlmann, Inverse problems for Lorentzian manifolds and non-linear hyperbolic
equations, Invent. Math. 212 (2018) 781--857.

\bibitem{L}
M. Lassas, T. Liimatainen, L. Potenciano-Machado, T. Tyni, Uniqueness and stability of an inverse problem
for a semi-linear wave equation, :2006.13193.
%
%\bibitem{LRadon}
%M. Lassas, L. Oksanen, P. Stefanov, G. Uhlmann, The light ray transform on Lorentzian manifolds, Comm. Math. Phys. 377 (2020) 1349--1379.

\bibitem{LU}
M. Lassas, G. Uhlmann, On determining a Riemannian manifold from the Dirichlet-to-Neumann map, Ann. Sci. \'Ec. Norm. Sup. 34 (2001) 771--787.

\bibitem{Penrose}
R. Penrose, {\em Techniques of differential topology in Relativity}, SIAM, Philadelphia, 1972.

\bibitem{GB}
J. Ralston, Gaussian beams and the propagation of singularities, in: {\em Studies in partial differential equations}\/ (W. Littman, Ed.), Mathematical Association of America, Washington, 1982, pp.\ 206--248.
%
%\bibitem{S}
%P. Stefanov, Support theorems for the light ray transform on analytic Lorentzian manifolds, Proc. Amer. Math. Soc., in press.
%
%\bibitem{SU}
%P. Stefanov, G. Uhlmann, Rigidity for metrics with the same lengths of geodesics, Math. Res. Lett. 5 (1998) 83--96.
%

\bibitem{SY}
P. Stefanov, Y. Yang, The inverse problem for the Dirichlet-to-Neumann map on Lorentzian manifolds,
Anal. PDE 11 (2018) 1381--1414.

\bibitem{UW}
G. Uhlmann, Y. Wang. Determination of space-time structures from gravitational perturbations, Comm. Pure Appl. Math. 73 (2020) 1315--1367.

\bibitem{UZ20}
G. Uhlmann, J. Zhai. Inverse problems for nonlinear hyperbolic equations, Discrete \& Continuous Dynamical Systems A 41 (2020) 455.

\bibitem{Tataru1}
D. Tataru, Unique continuation for solutions to PDE; between H\"ormander's theorem and Holmgren's theorem, Comm. PDE 20 (1995) 855--84.

\bibitem{Tataru2}
D. Tataru, Unique continuation for partial differential operators with partially analytic coefficients, J. Math. Pures Appl. 78 (1999) 505--521. 

\bibitem{Tay}
M. Taylor, Grazing rays and reflection of singularities of solutions to wave equations, Comm. Pure Appl. Math. 29 (1976) 1--38.

\bibitem{VW}
A. Vasy, Y. Wang,  On the light ray transform of wave equation solutions, Comm. Math. Phys. 384 (2021) 503--532.

\bibitem{Wang18}
Y. Wang, Parametrices for the light ray transform on
Minkowski spacetime, Inverse Problems and Imaging 12 (2018) 229--237.




\end{thebibliography}




\end{document}
