\section{Introduction}
\label{sec:introduction}

Reliable estimation of a robot's motion based on its onboard sensors is a fundamental requirement for many downstream tasks including localization and navigation. Devices such as inertial measurement units (IMU) or inertial navigation systems (INS) provide a way to directly measure the robot's motion based on acceleration and GNSS readings. An alternative is to use visual odometry (VO) leveraging image data from monocular or stereo cameras. Such VO methods have been successfully used in UAVs~\cite{fu2015efficient}, mobile applications~\cite{schoeps2014semidense}, and even mars rovers~\cite{maimone2007two}. Similar to other vision tasks, learning-based VO has gained increasing attention as the learnable high-level features can circumvent problems in textureless regions~\cite{valada2018incorporating, valada2018deep} or in the presence of dynamic objects~\cite{bevsic2022dynamic} where classical handcrafted methods suffer. However, learning-based VO lacks the ability to generalize to unseen domains, hindering their open-world deployment. Recently, adaptive VO~\cite{luo2019real} has opened a new avenue of research, \eg, by using continual learning (CL) methodologies to enhance VO during inference time~\cite{voedisch2023continual}.

\begin{figure}[t]
    \centering
    \includegraphics[width=\linewidth]{figures/teaser.pdf}
    % \vspace*{-.5cm}
    \caption{We propose \net for online continual learning of visual-inertial odometry. After pretraining on a source domain that is then discarded, \net further updates the network weights during inference on a target domain. Using experience replay \net successfully mitigates catastrophic forgetting.}
    \label{fig:teaser}
    % \vspace*{-.3cm}
\end{figure}

Most commonly, learning-based VO leverages monocular depth estimation as an auxiliary task~\cite{luo2019real, li2020self, voedisch2023continual} and exploits an unsupervised joint training scheme of a PoseNet, estimating the camera motion between two frames, and a DepthNet, estimating depth from a single image~\cite{godard2019digging}. Due to the unsupervised nature of this approach, learning-based VO can be continuously trained also during inference time.
In addition to classical domain adaptation~\cite{besic2022unsupervised}, where knowledge is transferred from a single source to a single target domain, the recent study on continual SLAM~\cite{voedisch2023continual} also investigates a sequential multi-domain setting as illustrated in \cref{fig:teaser}. The authors introduce CL-SLAM, which fuses adaptive VO with a graph-based SLAM backend.
To avoid catastrophic forgetting, \ie, overfitting to the current domain while losing the ability to perform well on past domains, CL-SLAM employs a dual-network architecture comprising an expert and a generalizer for both efficient domain adaptation and knowledge retention combined with experience replay.
However, the previously proposed CL-SLAM suffers from three main drawbacks:
First, network weights are transferred from the generalizer to the expert upon the start of a new evaluation sequence, \ie, a human supervisor decides when new data should be classified as a domain change.
Second, the utilized replay buffer of the generalizer is of infinite size and, thus, does not consider the limited storage capacity of real-world applications.
Finally, since every received frame triggers an update of the network weights before yielding the VO estimate, real-time usage is difficult to achieve on low-power devices such as embedded hardware in robots.

In this work, we propose a novel adaptive visual-inertial odometry estimation method called \net that explicitly addresses all of the aforementioned drawbacks of \mbox{CL-SLAM}. Similar to Kuznietsov~\etal~\cite{kuznietsov2022towards}, we consider a source-free setting, \ie, experience replay does not include data from the source domain used for pretraining. 
In particular, the contributions of this work can be summarized as follows:
\begin{enumerate}[topsep=0pt, noitemsep]
    \item We replace the dual-network architecture with a single network addressing both domain adaptation and knowledge retention but simplifying the overall architecture and reducing the GPU memory footprint. Additionally, this resolves the issue of transferring network weights without domain classification.
    \item We propose a fixed-size replay buffer that maximizes image diversity and addresses the limited storage capacity of embedded devices.
    \item We present an asynchronous version of \net that separates the core motion estimation from the network update step allowing true continuous inference
    \item We perform extensive evaluations of \net on various datasets, both publicly available and in-house, demonstrating its efficacy compared to other visual odometry methods.
    \item We release the code of our work and trained models at \mbox{\small \url{http://continual-slam.cs.uni-freiburg.de}}.
\end{enumerate}
