\section[Temporal Evolution of the Spectral Data]{Temporal Evolution of the Spectral Data}\label{S:time_evolution}
\separate

The machinery of the IST described in the previous sections suggests a method of solving the initial-value problem formulated by \eqref{Eq:cauchyproblem}
and \eqref{Eq:zeromass}. Considering $u$ to be a potential for the heat operator, if $u$ evolves like \eqref{Eq:cauchyproblem_a}, then $F = F(z, t)$
satisfies a linear evolution equation. Indeed, the \textsc{KP}II equation is the compatibility condition between the perturbed heat operator \eqref{Eq:Lpart}
$\op{L} = -\partial_y + \partial_x^2 + u$  and the evolution operator $\op{B}$ given by
\begin{equation}
\op{B} \= \frac{d}{dt} - \op{M},
\end{equation}
where the operator $\op{M}$ is defined in \eqref{Eq:Mpart}. These conditions are equivalent with the Lax equation
\[
\frac{d}{dt}\op{L} = [ \op{M}, \op{L} ].
\]

To see that the evolution of $u$ via the \textsc{KP}II equation corresponds to linear evolution of the spectral data, consider asymptotically exponential 
solutions $\psi$ to $\op{L}\psi = 0$ as $\abs{y} \to \infty$, with $u$ small. Then, one can write $\psi(x, y, t; z) = \mu(x, y, t; z)\rme^{\rmi z x - z^2 y}$ 
where
$\mu(x, y, t; z)$ is bounded in all variables. Suppose that $\psi$ evolves so as to satisfy $(d/dt)\psi = \op{M}\psi$, i.e., $\op{B}\psi = 0$. From the
asymptotic behaviour of $\mu$, namely $\mu \sim 1$ as $\abs{y} \to \infty$, we can determine that $\emph{\text{\textalpha}}(z) = 4\rmi z^3$.
Thus, $\psi$ satisfies the system
\begin{align}
&-\psi_y + \psi_{xx} + u\psi = 0, \notag\\
&\frac{d}{dt}\psi = 4\psi_{xxx}+ 6u\psi_x + \bigg(3u_x + 3\int_{\!-\ell}^x u_y \,\rmd s + 4\rmi z^3\bigg)\psi.
\end{align}
In this way the evolution of $u$ results in an evolution of $\psi$, hence, of the asymptotic behaviour of $\psi$.
\begin{lemma}\label{L:temporalevol}
Suppose $\psi \sim \rme^{\rmi z x - z^2 y}$ as $\abs{y} \to \infty$ and that $\psi \in \ker \op{B}(z)$. Then,
\begin{equation}\label{Eq:dataevolution}
\frac{d}{dt}F(z, t) = -4\rmi(z^3 + \bar{z}^3)F(z, t).
\end{equation}
\end{lemma}
\begin{proof}
First observe that for $z, k \in \bb{C}$, $\op{B}(z) - \op{B}(k) = \emph{\text{\textalpha}}(z) - \emph{\text{\textalpha}}(k) = 4\rmi(z^3 - k^3)$. Also,
$[ \op{B}, \op{J} ] = 0$ since $\emph{\text{\textalpha}}(z) \in C( \bb{C} )$. Now, equation \eqref{Eq:RiemannHilbert}
yields
\[
\op{J}\psi(x, y, t; z) = F(z, t)\psi^-( x, y, t; -\bar{z} ).
\]
Thus, since $\psi \in \ker \op{B}(z)$ we get
\begin{align*}
0 &= \op{B}(z)\op{J}\psi(x, y, t; z) = \op{B}(z)F(z, t)\psi^-( x, y, t; -\bar{z} )\\
&= \op{B}(k)F(z, t)\psi^-( x, y, t; -\bar{z} ) + 4\rmi(z^3 - k^3)F(z, t)\psi^-( x, y, t; -\bar{z} )\\
&= \psi^-( x, y, t; -\bar{z} )\Big[ \frac{d}{dt}F(z, t) + 4\rmi(z^3 - k^3)F(z, t) \Big] + F(z, t)\op{B}(z)\psi^-( x, y, t; -\bar{z} )
\end{align*}
Utilizing once more the continuity of $\emph{\text{\textalpha}}(z)$, yields $\op{B}(z)\psi^-( x, y, t; -\bar{z} ) = 0$. Setting $k = -\bar{z}$
completes the proof.
\end{proof}
Let us now calculate the evolution of $u$. For convenience, let $\dot{f}$ denote $df/dt$. 
\begin{lemma}\label{L:evolution}
The evolution of $u$ is given by
\begin{equation}
\dot{u}(x, y, t) = \frac{1}{\pi}\partial_x \sideset{}{'}\sum_{n = -\infty}^\infty \int_{L_n} \wt{\mu}(x, y; z) (\dot{ \Scr{S} }\mu)(x, y; z) \,\rmd z,
\end{equation}
where $P(-\partial + w)\wt{\mu} = -u\wt{\mu}$ or $\op{J}\wt{\mu} = \Scr{S}^\ast \wt{\mu}$.
\end{lemma}
\begin{proof}
From \eqref{Eq:solution} one has
\[
\pi\dot{u} = \partial_x \sideset{}{'}\sum_{n = -\infty}^\infty \int_{L_n} (\Scr{S}\mu)\spdot \,\rmd z.
\]
But $(\Scr{S}\mu)\spdot = (\Scr{S}1)\spdot + ( \Scr{S}(\mathcal{C}\Scr{S}\mu) )\spdot =
\dot{ \Scr{S} }1 + \dot{ \Scr{S} }(\mathcal{C}\Scr{S}\mu) + \Scr{S}\mathcal{C}(\Scr{S}\mu)\spdot$. Therefore,\\
$( \op{Id} - \Scr{S}\mathcal{C} )(\Scr{S}\mu)\spdot = \dot{ \Scr{S} }(1 + \mathcal{C}\Scr{S}\mu) = \dot{ \Scr{S} }\mu$. It follows that
\begin{align}
\pi\dot{u} &= \partial_x \sideset{}{'}\sum_{n = -\infty}^\infty \int_{L_n} ( \op{Id} - \Scr{S}\mathcal{C} )^{-1} \dot{ \Scr{S} }\mu \,\rmd z\nonumber\\
&= \partial_x \sideset{}{'}\sum_{n = -\infty}^\infty \int_{L_n} (\op{Id} - \mathcal{C}^t \Scr{S}^t)^{-1} \dot{ \Scr{S} }\mu \,\rmd z\nonumber\\
&= \partial_x \sideset{}{'}\sum_{n = -\infty}^\infty \int_{L_n} \wt{\mu}\dot{ \Scr{S} }\mu \,\rmd z,
\end{align}
where $\wt{\mu}$ satisfies the equation $\wt{\mu} = (\op{Id} - \mathcal{C}^t \Scr{S}^t)^{-1} 1 = (\op{Id} - \mathcal{C}\Scr{S}^\ast)^{-1} 1$
with $\Scr{S}^\ast = -\Scr{S}^t$. The superscript ${}^t$ denotes transposition with respect to the inner product on
$\Omega \times \bb{C} \setminus \wpln$ given by
\begin{equation}
\langle f, g \rangle \= \int_{\!-\infty}^\infty \int_{\!-\ell}^\ell \sideset{}{'}\sum_{n = -\infty}^\infty \int_{L_n} f(x, y; z)g(x, y; z) \,\rmd z\rmd x\rmd y.
\end{equation}
According to this inner product, $\partial^t = -\partial$ and $\mathcal{C}^t = -\mathcal{C}$. The function $\wt{\mu}(x, y; z)$ can be seen to satisfy the
transposed equation $P(-\partial + w)\wt{\mu} = -u\wt{\mu}$.

As a consequence of lemma \ref{L:commutators}, $[P(-\partial + w), \Scr{S}^\ast] = [ P(\partial + w), \Scr{S} ]^t = 0$ and
\[
[ P(-\partial + w), \mathcal{C} ]\Scr{S}^\ast \wt{\mu}(x, y) =
\frac{1}{\pi}\partial_x \sideset{}{'}\sum_{n = -\infty}^\infty \int_{L_n} (\Scr{S}^\ast \wt{\mu})(x, y; \zeta) \,\rmd z \equiv \wt{u}(x, y),
\]
hence, $P(-\partial + w)\wt{\mu} = (\op{Id} - \mathcal{C}\Scr{S}^\ast)^{-1} [ P(-\partial + w), \mathcal{C} ]\Scr{S}^\ast \wt{\mu} = \wt{u}\wt{\mu}$.
Meanwhile, $\wt{u}(x, y) = -u(x, y)$, since $\Scr{S}\mu = ( \op{Id} - \Scr{S}\mathcal{C} )^{-1} \Scr{S}1 = -\Scr{S}^\ast \wt{\mu}$.
\end{proof}

Based on lemma \ref{L:evolution}, we have the following global existence theorem which provides the solution of the
initial-value problem \eqref{Eq:cauchyproblem}\textendash\eqref{Eq:zeromass}.
\begin{theorem}\label{Th:solution}
Suppose the function $u_0(x, y)$ has small derivatives up to order $8$ in $L^1(\Omega) \cap L^2(\Omega)$. Then, the initial-value problem
\eqref{Eq:cauchyproblem}\textendash\eqref{Eq:zeromass} has a solution $u(x, y, t)$ for all $t \geq 0$, uniformly bounded for all $t$ in $L^2(\Omega)$
with bounded Fourier transform.
\end{theorem}
\begin{proof}
Equation \eqref{Eq:dataevolution} is a first order linear ordinary differential equation. Hence
\begin{equation}
F(z, t) = F(z, 0)\rme^{-4\rmi(z^3 + \bar{z}^3)t}.
\end{equation}
For all $z \in \bb{C}$, $\imaginary(z^3 + \bar{z}^3) = 0$. Thus, $F(z, t)$ remains bounded in each $W_\zeta^k$ for all $t$. In particular, setting
$F(z, 0) = F_0(z)$, $\norm{ F(\cdot, t) }_{W_\zeta^k} = \norm{F_0}_{W_\zeta^k}$. The initial value problem for the spectral data $F$
\begin{equation}
\frac{d}{dt}F(z, t) = -4\rmi(z^3 + \bar{z}^3)F(z, t), \qquad F(z, 0) = F_0(z),
\end{equation}
corresponds to the bounded evolution $\dot{ \Scr{S} } = [ \Scr{S}, \emph{\text{\textalpha}} ]$; here $\emph{\text{\textalpha}}$ stands for the operation
of multiplication by $\emph{\text{\textalpha}}(z)$. Thus, the bounded evolutions of $F$ correspond to bounded evolutions of $u$. This evolution is
\begin{align*}
\dot{u} &= \frac{1}{\pi}\partial_x \sideset{}{'}\sum_{n = -\infty}^\infty \int_{L_n} \wt{\mu}[ \Scr{S}, \emph{\text{\textalpha}} ]\mu \,\rmd z
= -\frac{1}{\pi}\partial_x \sideset{}{'}\sum_{n = -\infty}^\infty \int_{L_n}
\emph{\text{\textalpha}}( \wt{\mu}\Scr{S}\mu + \mu\Scr{S}^\ast \wt{\mu} ) \,\rmd z\\
&= -\frac{1}{\pi}\partial_x \sideset{}{'}\sum_{n = -\infty}^\infty \int_{L_n} \emph{\text{\textalpha}}( \wt{\mu}\op{J}\mu + \mu\op{J}\wt{\mu} ) \,\rmd z,
\end{align*}
thus,
\begin{equation}
\dot{u}(x, y, t) = \frac{4}{\pi \rmi}\partial_x \sideset{}{'}\sum_{n = -\infty}^\infty \int_{L_n}
z^3( \wt{\mu}\op{J}\mu + \mu\op{J}\wt{\mu} )(x, y; z) \,\rmd z.
\end{equation}
Observe now that $z^3$ is the coefficient of $1/s^4$ in the geometric series
\[
\sum\limits_{k = 1}^\infty \frac{ z^{k - 1} }{s^k} = \frac{1}{s - z}
\]
for $\abs{z} < \abs{s}$ and for any $s \in \wpln$. This observation suggests the introduction of the linear functional $\phi$ acting on the algebra of
formal power series in $s^{-1}$ with coefficients in some ring $R$ and defined by
\begin{equation}
\phi \bigg( \sum_{k = 0}^\infty \frac{a_k}{s^k} \bigg) \= a_4, \quad a_k \in R.
\end{equation}
Then, $z^3 = \phi(s - z)^{-1}$. Since $\phi$ is linear and commutes with all operations present, we have
\begin{align*}
\dot{u}(x, y, t) &= \phi\frac{4}{\pi \rmi}\partial_x \sideset{}{'}\sum_{n = -\infty}^\infty \int_{L_n}
\frac{ ( \wt{\mu}\op{J}\mu + \mu\op{J}\wt{\mu} )(x, y; z) }{s - z} \,\rmd z\\
&= -8\phi\frac{1}{2\pi \rmi}\partial_x \sideset{}{'}\sum_{n = -\infty}^\infty \int_{L_n}
\frac{ ( \wt{\mu}\op{J}\mu + \mu\op{J}\wt{\mu} )(x, y; z) }{z - s} \,\rmd z.
\end{align*}
Applying the Plemelj--Sokhotski formulae in the above equation yields
\[
\dot{u}(x, y, t) = -8\phi\frac{1}{2\pi \rmi}\partial_x \op{J}( \wt{\mu}\op{J}\mu + \mu\op{J}\wt{\mu} )(x, y; s) =
-16\phi\partial_x \op{J}\mu\op{J}\wt{\mu}(x, y; s),
\]
for $s \in L_{n}$ and $n \in \sZ$. Therefore, $u(x, y, t)$ solves the nonlinear system
\begin{gather}
\dot{u} = -16\phi\partial_x \op{J}\mu\op{J}\wt{\mu},\\
(-\partial_y + \partial_x^2 + 2\rmi s\partial_x)\mu = -u\mu,\\
(\partial_y + \partial_x^2 - 2\rmi s\partial_x)\wt{\mu} = -u\wt{\mu}.
\end{gather}
The asymptotic behaviour as $\abs{\!\imaginary s} \to \infty$ of the functions $\mu$ and $\wt{\mu}$, allows us to express the functions $\op{J}\mu$ and
$\op{J}\wt{\mu}$ as asymptotic series in $s^{-1}$ with coefficients in the ring of smooth functions up to order $3$ if $\Scr{S}$, i.e., $F$ has sufficient
decay. The coefficients can be determined recursively by the following relations
\begin{gather}
m_0(x, y) \equiv 1, \qquad 2i\partial_x m_{k +1}(x, y) = ( \partial_y - \partial_x^2 - u(x, y) )m_k(x, y),\\
\wt{m}_0(x, y) \equiv 1, \qquad 2i\partial_x \wt{m}_{k +1}(x, y) = ( \partial_y + \partial_x^2 + u(x, y) ) \wt{m}_k(x, y),
\end{gather}
where
\begin{equation}
\mu(x, y; s) = \sum_{k = 0}^\infty \frac{ m_k(x, y) }{s^k}, \qquad \wt{\mu}(x, y; s) = \sum_{k = 0}^\infty \frac{ \wt{m}_k(x, y) }{s^k}.
\end{equation}

For $u$ to be a solution of the \textsc{KP}II equation amounts to solving for $m_k$, $\wt{m}_k, \, k \leq 4$, multiply together the two series for $\mu$,
$\wt{\mu}$ and then picking out the coefficient of $s^{-4}$. We conclude the proof by noticing that the order of derivatives of $u$ is high enough to
provide continuity of the forward and inverse spectral transforms.
\end{proof}