\section[The Inverse Problem]{The Inverse Problem}\label{S:inverse}
\smallskip
\subsection[An appropriate Space for the Inverse Problem]{An appropriate Space for the Inverse Problem}\label{s:Hspace}
\separate

Any small spectral operator $\Scr{S}$ of the form in proposition \ref{Pr:important} determines a unique solution $\mu(x, y; z)$ to $\mu = 1 + \mathcal{C}\Scr{S}\mu$ which also solves the 
equation~\eqref{Eq:bvp} with some potential $u(x, y)$. The operator $\Scr{S}$ is defined for $L^\infty(E)$ functions, holomorphic in $\wpln$ and for which the one-sided limits at the
lines $L_n, n \in \sZ$ exist. An appropriate space for the class of functions having the properties as in proposition \ref{Pr:important} is the Hardy-like space defined as follows:
\begin{equation}
\Scr{H}_\omega \= \{f \in H(\wpln) \colon \norm{f}_\omega < \infty\},
\end{equation}
where
\begin{equation}
\norm{f}_\omega \= \sup_{\re{z} \notin \frac{\omega}{2}\sZ} \bigg(\int_{\!-\infty}^\infty \abs{ f( \re{z} + \rmi\im{z} ) }^2 \,\rmd\im{z}\bigg)^\frac{1}{2} =
\sup_{\re{z} \notin \frac{\omega}{2}\sZ} \norm{ f_{ \re{z} } }_2,
\end{equation}
with $f_{ \re{z} }( \im{z} ) = f( \re{z} + \rmi\im{z} )$. The pair $(\Scr{H}_\omega, \,\norm{\cdot}_\omega)$ forms a normed vector space: let $f$ and $g$ in $\Scr{H}_\omega$ and
$\lambda$ a complex number. Evidently, the function $f + \lambda g$ is holomorphic in $\wpln$. Furthermore, if $\re{z} \notin \frac{\omega}{2}\sZ$, by Minkowski's inequality
\begin{align*}
\int_{\!-\infty}^\infty \abs{ (f + \lambda g)( \re{z} + \rmi\im{z} ) }^2 \,\rmd\im{z} &= \int_{\!-\infty}^\infty \abs{ f( \re{z} + \rmi\im{z} ) + \lambda g( \re{z} + \rmi\im{z} ) }^2 \,\rmd\im{z}\\
&= \norm{ f_{ \re{z} } + \lambda g_{ \re{z} } }_2^2\\
&\leq (\norm{ f_{ \re{z} } }_2 + \abs{\lambda}\norm{ g_{ \re{z} } }_2)^2.
\end{align*}
Thus,
\[
\bigg(\int_{\!-\infty}^\infty \abs{ (f + \lambda g)( \re{z} + \rmi\im{z} ) }^2 \,\rmd\im{z}\bigg)^\frac{1}{2} \leq \norm{f}_\omega + \abs{\lambda}\norm{g}_\omega,
\]
which shows that $f + \lambda g$ belongs to $\Scr{H}_\omega$ and $\norm{f + \lambda g}_\omega \leq \norm{f}_\omega + \abs{\lambda}\norm{g}_\omega$. Moreover
$\norm{f}_\omega = 0$ if and only if $f = 0$.
\begin{lemma}
Let $f \in \Scr{H}_\omega$ and $S_{a, b} = \{z \in \wpln \mid a < \re{z} < b, \,\im{z} \in \bb{R}\}$ where either $a = \tfrac{\omega}{2}n, \,b = \tfrac{\omega}{2}(n + 1)$ or
$a = -\tfrac{\omega}{2}(n + 1), \,b = -\tfrac{\omega}{2}n$ or $a = -b = -\tfrac{\omega}{2}$ with $n \in \bb{N}$, i.e., $S_{a, b}$ a strip in $\wpln$. Then, for $z \in S_{a, b}$
\[
\abs{ f( \re{z} + \rmi\im{z} ) } \leq \sqrt{ \frac{2}{\pi} }\norm{f}_\omega ( \min\{ \abs{\re{z} - a}, \,\abs{\re{z} - b} \} )^{ -\frac{1}{2} }.
\]
Furthermore, if $K$ is a compact subset of $S_{a, b}$ then, $f$ is bounded in $K$.
\end{lemma}
\begin{proof}
Let $z \in S_{a, b}$ and consider the closed disc $D(z, r)$ centred at $z$ with radius $r = \min\{ \abs{\re{z} - a}, \,\abs{\re{z} - b} \}$. Then, $D(z, r) \sse
\overline{S}_{\re{z} - r, \re{z} + r} \sse S_{a, b}$: if $\sigma + \rmi\tau \in D(z, r)$, then $\abs{\re{z} - \sigma} \leq r$ and $\abs{\im{z} - \tau} \leq r$, thus
$z \in \overline{S}_{\re{z} - r, \re{z} + r}$ and because $a < \re{z} - r$ and $\re{z} + r < b$, we also have $\overline{S}_{\re{z} - r, \re{z} + r} \sse S_{a, b}$. Since the function
$\abs{ f(z) }^2$ is subharmonic in $S_{a, b}$,
\begin{align*}
\abs{ f( \re{z} + \rmi\im{z} ) }^2 &\leq \frac{1}{\pi r^2}\iint_{ D(z, r) } \abs{ f(z) }^2 \,\rmd z\\
&\leq \frac{1}{\pi r^2}\iint_{ \overline{S}_{\re{z} - r, \re{z} + r} } \abs{ f(z) }^2 \,\rmd z\\
&= \frac{1}{\pi r^2}\int_{\re{z} - r}^{\re{z} + r} \int_{\!-\infty}^\infty \abs{ f( \re{z} + \rmi\im{z} ) }^2 \,\rmd\im{z}\rmd\re{z}\\
&\leq \frac{1}{\pi r^2}\int_{\re{z} - r}^{\re{z} + r} \norm{f}_\omega^2 \,\rmd\re{z} = \frac{2}{\pi}\norm{f}_\omega^2 \frac{1}{r}.
\end{align*}

If $z \in K \sse S_{a, b}$, and $K$ is a compact set, there exist some real numbers $M_1$, $M_2$ such that $a < M_1 \leq \re{z} \leq M_2 < b$, hence
$\re{z} - a \geq M_1 - a$ and $b - \re{z} \geq b - M_2$. Thus,
\[
\abs{ f(z) } \leq \sqrt{ \frac{2}{\pi} }\norm{f}_\omega ( \min\{ \abs{M_1 - a}, \,\abs{M_2 - b} \} )^{ -\frac{1}{2} },
\]
showing that $f$ is bounded in $K$.
\end{proof}
\begin{proposition}
$(\Scr{H}_\omega, \,\norm{\cdot}_\omega)$ is a Banach space.
\end{proposition}
\begin{proof}
Suppose $\{f_m\}_{m = 1}^\infty$ is a Cauchy sequence in $\Scr{H}_\omega$ and $K$ is a compact subset of $\wpln$. If $z \in K$, then $M_1 \leq \re{z} \leq M_2$ for some real
numbers $M_1$ and $M_2$. Since $K \sse \wpln$, there exist nonzero integers $n_1$, $n_2$ such that $\tfrac{\omega}{2}n_1 < M_1$ and $M_2 < \tfrac{\omega}{2}n_2$. Thus,
there exist numbers $a, \,b$ such that $a < M_1 \leq \re{z} \leq M_2 < b$. But then, by the previous lemma
\[
\abs{ f_m(z) - f_l(z) } \leq \sqrt{ \frac{2}{\pi} }\norm{f_m - f_l}_\omega ( \min\{ \abs{M_1 - a}, \,\abs{M_2 - b} \} )^{ -\frac{1}{2} },
\]
which shows that $\{f_m\}$ is uniformly Cauchy in compact subsets of $\wpln$ and thus, converges uniformly on compact subsets of $\wpln$ to some function $f \in H(\wpln)$.
Now, given $\epsilon > 0$, there exists $N \in \bb{N}$ such that $\norm{f_m - f_N}_\omega < \tfrac{\epsilon}{2}$ for all $m \geq N$. Then, by Fatou's lemma
\begin{align*}
\int_{\!-\infty}^\infty \abs{ (f - f_N)( \re{z} + \rmi\im{z} ) }^2 \,\rmd\im{z} &= \int_{\!-\infty}^\infty \lim_{m \to \infty} \abs{ (f_m - f_N)( \re{z} + \rmi\im{z} ) }^2 \,\rmd\im{z}\\
&\leq \lim_{m \to \infty} \int_{\!-\infty}^\infty \abs{ (f_m - f_N)( \re{z} + \rmi\im{z} ) }^2 \,\rmd\im{z}\\
&\leq \lim_{m \to \infty} \norm{f_m - f_N}_\omega^2 < \Big(\frac{\epsilon}{2}\Big)^2.
\end{align*}
Hence, $\norm{f - f_N}_\omega < \tfrac{\epsilon}{2}$ and so it follows that $\norm{f}_\omega < \infty$ and $\norm{f_m - f}_\omega \to 0$ as $m \to \infty$. Thus, the space
$\Scr{H}_\omega$ is complete under the norm $\norm{\cdot}_\omega$.
\end{proof}
Now let $\ell^\infty( L^2( \bb{R} ) ) \equiv \ell^\infty( \sZ, L^2( \bb{R} ) )$, that is
\begin{equation}
\ell^\infty( L^2( \bb{R} ) ) = \{g = \{g_m\} \colon g_m \in L^2( \bb{R} ), \ \forall \:m \in \sZ \text{ and } \norm{g}_{2, \infty} < \infty\},
\end{equation}
where
\begin{equation}
\norm{g}_{2, \infty} \= \sup_{m \in \sZ} \norm{g_m}_2 = \sup_{m \in \sZ} \bigg(\int_{\!-\infty}^\infty \abs{ g_m(x) }^2 \,\rmd x\bigg)^\frac{1}{2}.
\end{equation}
It is easily seen that $\norm{\cdot}_{2, \infty}$ defines a norm on $\ell^\infty( L^2( \bb{R} ) )$ under which it becomes a Banach space.

The following lemma is a Paley--Wiener type theorem for functions holomorphic in strips \cite {PW34} .
\begin{lemma}\label{L:Paley-Wiener}
Let $f \in \Scr{H}_\omega$. Then, there exists a measurable function $G$ such that
\[
\int_{\!-\infty}^\infty \abs{ G(\xi) }^2 \rme^{2\re{z}\xi} \,\rmd\xi < \infty,
\]
and
\begin{equation}
f(z) = \frac{1}{2\pi}\int_{\!-\infty}^\infty G(\xi)\rme^{z\xi} \,\rmd\xi,
\end{equation}
in the sense of $L^2$ convergence, for $z = \re{z} + \rmi\im{z} \in K \sse S_{a, b} \sse \bb{C}_\omega$ where $K$ is a compact set.
\end{lemma}
Let $f \in \Scr{H}_\omega$ and $n \in \sZ$. Suppose $x_k$ is a sequence such that $x_k \to \frac{\omega}{2}n^+$ (without loss of generality, let n be positive). Then, if $\delta > 0$,
there exists a positive integer $k_0$ such that $\frac{\omega}{2}n < x_k \leq \frac{\omega}{2}n + \delta < \frac{\omega}{2}(n + 1)$, for $k \geq k_0$. By lemma \ref{L:Paley-Wiener},
\[
f( x_k + \rmi\im{z} ) = \frac{1}{2\pi}\int_{\!-\infty}^\infty G(\xi)\rme^{ ( x_k + \rmi\im{z} )\xi } \,\rmd\xi,
\]
for some measurable function $G$ such that
\[
\int_{\!-\infty}^\infty \abs{ G(\xi) }^2 \rme^{2x_k \xi} \,\rmd\xi < \infty.
\]
Now, $G(\xi)\rme^{ ( x_k + \rmi\im{z} )\xi } \to G(\xi)\rme^{ ( \frac{\omega}{2}n + \rmi\im{z} )\xi }$ and $\abs{ G(\xi)\rme^{ ( x_k + \rmi\im{z} )\xi } } = \abs{ G(\xi) }\rme^{x_k \xi}\! \in \!L^1$ since
\begin{align*}
\int_{\!-\infty}^\infty \abs{ G(\xi) }\rme^{x_k \xi} \,\rmd\xi &= \int_{\!-\infty}^0 \abs{ G(\xi) }\rme^{ (x_k - \epsilon)\xi }\rme^{\epsilon\xi} \,\rmd\xi
+ \int_0^\infty \abs{ G(\xi) }\rme^{ (x_k + \epsilon)\xi }\rme^{-\epsilon\xi} \,\rmd\xi\\
&\leq \bigg(\int_{\!-\infty}^\infty \abs{ G(\xi) }^2 \rme^{2(x_k - \epsilon)\xi} \,\rmd\xi \int_{\!-\infty}^0 \rme^{2\epsilon\xi} \,\rmd\xi\bigg)^\frac{1}{2}\\
&{}+ \bigg(\int_{\!-\infty}^\infty \abs{ G(\xi) }^2 \rme^{2(x_k + \epsilon)\xi} \,\rmd\xi \int_0^\infty \rme^{-2\epsilon\xi} \,\rmd\xi\bigg)^\frac{1}{2} < \infty,
\end{align*}
for $\epsilon < \min\{x_k - \frac{\omega}{2}n, \,\frac{\omega}{2}n + \delta - x_k\}$. Thus, by dominated convergence, the limit $\lim\limits_{k \to \infty} f( x_k + \rmi\im{z} )$, i.e., the
limit $\lim\limits_{\re{z} \to \frac{\omega}{2}n^+} f( \re{z} + \rmi\im{z} ) \equiv f^+( \frac{\omega}{2}n + \rmi\im{z} )$ exists and also
\begin{equation}
f^+( \tfrac{\omega}{2}n + \rmi\im{z} ) = \frac{1}{2\pi}\int_{\!-\infty}^\infty G(\xi)\rme^{ ( \frac{\omega}{2}n + \rmi\im{z} )\xi } \,\rmd\xi,
\end{equation}
in the sense of $L^2$ convergence and pointwise almost everywhere convergence, since $f( \re{z} + \rmi\im{z} ) = \Scr{F}( G(\xi)\rme^{ \re{z}\xi } )( -\im{z} )$ and by Plancherel's theorem
$\Scr{F}( G(\xi)\rme^{ \re{z}\xi } )$ is square integrable. Similarly, we can establish the existence of the limit
\begin{equation}
f^-( \tfrac{\omega}{2}n + \rmi\im{z} ) \equiv \lim_{\re{z} \to \tfrac{\omega}{2}n^-} f( \re{z} + \rmi\im{z} )
\end{equation}
in the sense of $L^2$ and pointwise almost everywhere convergence.

From the above discussion, it is natural to define the bounded linear operator
$\op{J} \colon (\Scr{H}_\omega, \,\norm{\cdot}_\omega) \to ( \ell^\infty( L^2( \bb{R} ) ), \,\norm{\cdot}_{2, \infty} )$ by
\begin{equation}
(\op{J}f)_n(y) = \op{J}f(n, y) \= f^+(\tfrac{\omega}{2}n + \rmi y) - f^-(\tfrac{\omega}{2}n + \rmi y).
\end{equation}
We also define the linear operator $\op{l} \colon (\Scr{H}_\omega, \,\norm{\cdot}_\omega) \to ( \ell^\infty( L^2( \bb{R} ) ), \,\norm{\cdot}_{2, \infty} )$ with
\begin{equation}
(\op{l}f)_n(y) = \op{l}f(n, y) \= f^-(\!-\tfrac{\omega}{2}n + \rmi y).
\end{equation}
$\op{l}$ is bounded and $\norm{\op{l}f}_{2, \infty} \leq \norm{f}_\omega$.

Let us now state the following lemma, which expresses a simple identity.
\begin{lemma}\label{L:expintegral}
For $\tau \in \bb{R}$
\begin{equation}
\frac{1}{\zeta + \rmi\tau} = \begin{cases}
\displaystyle\int_{\!-\infty}^0 \rme^{ (\zeta + \rmi\tau)\xi } \,\rmd\xi, & \real \zeta > 0\\[15pt]
\displaystyle-\int_0^\infty \rme^{ (\zeta + \rmi\tau)\xi } \,\rmd\xi, & \real \zeta < 0.
\end{cases}
\end{equation}
\end{lemma}

Let $g = g_n(\tau) = g(\frac{\omega}{2}n + \rmi\tau) \in \ell^\infty( L^2( \bb{R} ) )$. Fix a positive integer $n$ (for $n$ negative the analysis is similar) and define the function
\begin{equation}
h_n(z) = \frac{1}{2\pi}\int_{\!-\infty}^\infty \frac{ g(\frac{\omega}{2}n + \rmi\tau) }{\frac{\omega}{2}n + \rmi\tau - z} \,\rmd\tau, \quad z \in S_{n - 1} \cup S_n.
\end{equation}
An application of H\"{o}lder's inequality shows that the function $h_n(z)$ is holomorphic. Assume that $z \in S_n$, i.e., $\frac{\omega}{2}n < \re{z}$.~Using lemma \ref{L:expintegral},
we can rewrite $h_n$ as
\begin{equation}
h_n(z) = \int_{\!-\infty}^\infty G_n(\xi)\Psi_c(\xi)\rme^{-\rmi\im{z}\xi} \,\rmd\xi = \Scr{F}( G_n(\xi)\Psi_c(\xi) )( \im{z} ),
\end{equation}
where $G_n(\xi) = \Scr{F}^{-1}( g(\frac{\omega}{2}n + \rmi\tau) )(\xi), \,c = \re{z} - \frac{\omega}{2}n$ and $\Psi_c(\xi) = -\rme^{-c\xi}$ for $\xi > 0$, $\Psi_c(\xi) = 0$ for $\xi < 0$.
Since $g_n \in L^2( \bb{R} )$, Plancherel's theorem yields
\begin{align*}
\int_{\!-\infty}^\infty \abs{ h_n( \re{z} + \rmi\im{z} ) }^2 \,\rmd\im{z} &= 2\pi\int_{\!-\infty}^\infty \abs{ G_n(\xi)\Psi_c(\xi) }^2 \,\rmd\xi\\
&< 2\pi\int_{\!-\infty}^\infty \abs{ G_n(\xi) }^2 \,\rmd\xi\\
&= \int_{\!-\infty}^\infty \abs{ g(\tfrac{\omega}{2}n + \rmi\tau) }^2 \,\rmd\tau = \norm{g_n}_2^2.
\end{align*}
Similarly, when $z \in S_{n - 1}$, i.e., $\frac{\omega}{2}n > \re{z}$,
\begin{equation}
h_n(z) = \Scr{F}( G_n(\xi)\Psi_c(\xi) )( \im{z} ),
\end{equation}
where this time $\Psi_c(\xi) = \rme^{-c\xi}$ for $\xi < 0$ and $\Psi_c(\xi) = 0$ for $\xi > 0$. Hence, once again by Plancherel's theorem
\[
\int_{\!-\infty}^\infty \abs{ h_n( \re{z} + \rmi\im{z} ) }^2 \,\rmd\im{z} < \norm{g_n}_2^2.
\]
\noi Therefore, since $n$ was arbitrary, we have that $h_n \in H(\wpln)$ for every nonzero integer $n$ and
\begin{equation}
\sup_{\re{z} \notin \frac{\omega}{2}\sZ} \bigg(\int_{\!-\infty}^\infty \abs{ h_n( \re{z} + \rmi\im{z} ) }^2 \,\rmd\im{z}\bigg)^\frac{1}{2} < \norm{g}_{2, \infty},
\end{equation}
in particular $h_n \in \Scr{H}_\omega$ for all $n \in \sZ$. A straightforward change of variables shows that
\begin{equation}
h_n(z) = \frac{1}{2\pi \rmi}\int_{\frac{\omega}{2}n + \rmi\infty}^{\frac{\omega}{2}n - \rmi\infty} \frac{ g(\zeta) }{\zeta - z} \,\rmd\zeta =
\frac{1}{2\pi \rmi}\int_{L_n} \frac{ g(\zeta) }{\zeta - z} \,\rmd\zeta.
\end{equation}
Applying the operator $\op{J}$ to the sequence $h_n$ and using the Plemelj--Sokhotski formulas for $L^2$ potentials, we have
\begin{equation}
(\op{J}h_n)_n(y) = h_n^+(\tfrac{\omega}{2}n + \rmi y) - h_n^-(\tfrac{\omega}{2}n + \rmi y) = g(\tfrac{\omega}{2}n + \rmi y).
\end{equation}
\begin{proposition}\label{Pr:lemma}
Let $a_n(\tau) = a(\frac{\omega}{2}n + \rmi\tau)$ be a sequence of complex functions such that
\begin{equation}
\sup_{ \tau \in \bb{R} } \abs{ a_n(\tau) } \leq \frac{c}{n^2}, \quad n \in \sZ,
\end{equation}
where $c$ is some constant, and let $g = g_n(\tau) = g(\frac{\omega}{2}n + \rmi\tau) \in \ell^\infty( L^2( \bb{R} ) )$. Then, the series
\[
\sideset{}{'}\sum_{n = -\infty}^\infty \frac{1}{2\pi}\int_{\!-\infty}^\infty \frac{ a(\frac{\omega}{2}n + \rmi\tau)g(\frac{\omega}{2}n + \rmi\tau) }{\frac{\omega}{2}n + \rmi\tau - z} \,\rmd\tau
\]
converges uniformly on compact subsets of $\wpln$ to a function $f \in \Scr{H}_\omega$ and
\begin{equation}
\norm{f}_\omega \leq c\frac{\pi^2}{3}\norm{g}_{2, \infty} \equiv \Gamma_c \norm{g}_{2, \infty},
\end{equation}
and
\begin{equation}
(\op{J}f)_n(y) = a(\tfrac{\omega}{2}n + \rmi y)g(\tfrac{\omega}{2}n + \rmi y).
\end{equation}
\end{proposition}
\begin{proof}
For every $n \in \sZ$
\[
\int_{\!-\infty}^\infty \abs{ a(\tfrac{\omega}{2}n + \rmi\tau)g(\tfrac{\omega}{2}n + \rmi\tau) }^2 \,\rmd\tau \leq
\frac{c^2}{n^4}\int_{\!-\infty}^\infty \abs{ g(\tfrac{\omega}{2}n + \rmi\tau) }^2 \,\rmd\tau = \frac{c^2}{n^4}\norm{g_n}_2^2,
\]
hence,
\[
\sup_{n \in \sZ} \norm{a_n g_n}_2 \leq c\norm{g}_{2, \infty} \sup_{n \in \sZ} \frac{1}{n^2} = c\norm{g}_{2, \infty} < \infty.
\]
Therefore,
\[
h_n(z) = \frac{1}{2\pi}\int_{\!-\infty}^\infty \frac{ a(\frac{\omega}{2}n + \rmi\tau)g(\frac{\omega}{2}n + \rmi\tau) }{\frac{\omega}{2}n + \rmi\tau - z} \,\rmd\tau \in \Scr{H}_\omega,
\]
and $\norm{h_n}_\omega \leq \frac{c}{n^2}\norm{g}_{2, \infty}$. Now, for $z \in \wpln$
\begin{align*}
\abs{ h_n(z) } &\leq \frac{1}{2\pi}\bigg(\int_{\!-\infty}^\infty \abs{ a(\tfrac{\omega}{2}n + \rmi\tau)g(\tfrac{\omega}{2}n + \rmi\tau) }^2 \,\rmd\tau\bigg)^\frac{1}{2}
\bigg( \int_{\!-\infty}^\infty \frac{\rmd\tau}{ \abs{\frac{\omega}{2}n + \rmi\tau - z}^2 } \bigg)^\frac{1}{2}\\
&\leq \frac{c}{2\pi n^2}\norm{g}_{2, \infty}\bigg( \int_{\!-\infty}^\infty \frac{\rmd\tau}{ \abs{\frac{\omega}{2}n + \rmi\tau - z}^2 } \bigg)^\frac{1}{2}\\
&= \frac{c}{ 2\sqrt{\pi} }\norm{g}_{2, \infty}\frac{1}{ n^2 \sqrt{ \abs{ \frac{\omega}{2}n - \re{z} } } }.
\end{align*}
An application of Weierstrass M-test shows that the series $\sideset{}{'}\sum\limits_{n = -\infty}^\infty h_n(z)$ converges uniformly on compact subsets of $\wpln$ to a function $f$
which is holomorphic in $\wpln$ since $h_n$ are also holomorphic. Moreover, since $h_n \in \Scr{H}_\omega$ we have
\[
\norm{f}_\omega \leq \sideset{}{'}\sum_{n = -\infty}^\infty \norm{ h_n(z) }_\omega \leq c\norm{g}_{2, \infty}\sideset{}{'}\sum_{n = -\infty}^\infty \frac{1}{n^2} < \infty,
\]
thus $f$ belongs to $\Scr{H}_\omega$.

Finally, fix a non-zero integer $m$. For $z \in S_{m - 1} \cup S_m$, the functions $h_n(z)$ are holomorphic when $n \neq m$. Also,
$(\op{J}h_m)_m(y) = a(\frac{\omega}{2}m + \rmi y)g(\frac{\omega}{2}m + \rmi y)$. Therefore,
\begin{align*}
(\op{J}f)_m(y) &= \bigg( \op{J}\sideset{}{'}\sum_{n = -\infty}^\infty h_n(z) \bigg)(m, y)\\
&= \op{J}\bigg( \sideset{}{'}\sum_{n \neq m} h_n(z) + h_m(z) \bigg)(m, y)\\
&= (\op{J}h_m)(m, y) = a(\tfrac{\omega}{2}m + \rmi y)g(\tfrac{\omega}{2}m + \rmi y).\qedhere
\end{align*}
\end{proof}
Introduce the operator $\op{S} \colon ( \ell^\infty( L^2( \bb{R} ) ), \,\norm{\cdot}_{2, \infty} ) \to (\Scr{H}_\omega, \,\norm{\cdot}_\omega)$ defined by
\begin{equation}
(\op{S}g)(x, y; z) \= \sideset{}{'}\sum_{n = -\infty}^\infty \frac{1}{2\pi}\int_{\!-\infty}^\infty
\frac{F(\frac{\omega}{2}n, \tau)g_n(\tau) }{\frac{\omega}{2}n + \rmi\tau - z}\rme^{ -\rmi\omega n(x - 2\tau y) } \,\rmd\tau,
\end{equation}
for all $(x, y) \in \Omega$ where the function $F$ has the following properties:
\begin{equation}\label{Eq:scatteringbound1}
\abs{ F(\tfrac{\omega}{2}n, \tau) } \leq \frac{c}{n^2}, \quad \forall \:\tau \in \bb{R}, \,n \in \sZ,
\end{equation}
for some constant $c$ and
\begin{equation}\label{Eq:scatteringbound2}
\int_{\!-\infty}^\infty \abs{ F(\tfrac{\omega}{2}n, \tau) }^2 \,\rmd\tau = O\bigg( \frac{1}{n^4} \bigg), \quad \forall \:n \in \sZ.
\end{equation}
By proposition \ref{Pr:lemma}, we see that $\op{S}$ is bounded linear, $\norm{\op{S}g}_\omega \leq \Gamma_c \norm{g}_{2, \infty}$ and that
$(\op{J}\op{S}g)_n(x, y; \tau) = F(\tfrac{\omega}{2}n, \tau)\rme^{ -\rmi\omega n(x - 2\tau y) }g_n(\tau)$.