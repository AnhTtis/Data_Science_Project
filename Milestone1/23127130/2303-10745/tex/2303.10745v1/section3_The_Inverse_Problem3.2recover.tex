\subsection[Recovering the Potential from the Spectral Data]{Recovering the Potential from the Spectral Data}\label{s:recover}
\separate

Returning now to the discussion of the inverse problem, consider the equation
\begin{equation}
\mu(x, y; z) = 1 + \frac{1}{2\pi \rmi}\sideset{}{'}\sum_{n = -\infty}^\infty \int_{L_n} \frac{ (\Scr{S}\mu)(x, y; \zeta) }{\zeta - z} \,\rmd\zeta,
\end{equation}
where $\Scr{S}$ is the spectral operator defined in \eqref{Eq:scatteringoperator} with the function $F$ satisfying \eqref{Eq:scatteringbound1} and 
\eqref{Eq:scatteringbound2}. This equation can be rewritten in the form
\begin{align}\label{Eq:inverse}
(\mu - 1)(x, y; z) &= \sideset{}{'}\sum_{n = -\infty}^\infty \frac{1}{2\pi}\int_{\!-\infty}^\infty
\frac{F(\frac{\omega}{2}n, \tau)\rme^{ -\rmi\omega n(x - 2\tau y) } }{\frac{\omega}{2}n + \rmi\tau - z} \,\rmd\tau\notag\\
&{}+ \op{S}\op{l}(\mu - 1)(x, y; z).
\end{align}
If the constant $c$ is such that $\Gamma_{c} < 1$, then the composition $\op{S}\op{l}$ is a contraction on $\Scr{H}_\omega$ since for
$f \in \Scr{H}_\omega$, $\norm{\op{S}\op{l}f}_\omega \leq \Gamma_{c} \norm{\op{l}f}_{2, \infty} \leq \Gamma_{c} \norm{f}_\omega$. Thus, Banach's 
fixed-point theorem implies that the above equation has a unique solution $\mu$ such that $\mu - 1 \in \Scr{H}_\omega$.

Recall equation~\eqref{Eq:shiftedeq}, $[ (\partial_x + \rmi z)^2 - (\partial_y - z^2) ]\mu = -u\mu$ which is written with the shifted derivatives
$D_1, D_2$, defined in~\eqref{Eq:shifted}, as
\begin{equation}
[D_1^2 - D_2 + u]\mu(x, y; z) = 0.
\end{equation}
The following two lemmas are useful for the proof of the Inverse Spectral theorem that follows.
\begin{lemma}\label{L:dressing}
If $\Scr{S}$ has the form
\begin{equation}
(\Scr{S}f)(x, y; z) = F(z)\rme^{ \rmi r_0(z) \cdot (\omega x, y) } f^-( x, y; -\bar{z} ),
\end{equation}
and the function $f(x, y; z)$ has one continuous derivative (in the $x$, $y$ variables), then $[D_1, \Scr{S}]f = [D_2, \Scr{S}]f = 0$ for $z$ on
$\bb{C} \setminus \wpln$.
\end{lemma}
\begin{proof}
A straightforward calculation.
\end{proof}
It is also useful to calculate the commutators of the translated derivatives $D_1, \,D_2$ with the operator $\mathcal{C}$. Since both $\partial_x$ and
$\partial_y$ commute with $\mathcal{C}$, it suffices to calculate $[z^m, \mathcal{C}]$. The following result holds.
\begin{lemma}\label{L:commutatorss}
Suppose $f(x, y; z)$ is a Schwartz function on $E$. Then,
\begin{gather}
[z^m, \mathcal{C}]f(x, y; z) = \frac{1}{2\pi \rmi}\sum_{k = 0}^{m - 1} z^k \sideset{}{'}\sum_{n = -\infty}^\infty \int_{L_n}
\zeta^{m - k - 1} f(x, y; \zeta) \,\rmd\zeta,\\
[D_1^m, \mathcal{C}]f(x, y; z) = \sum_{k = 0}^m \binom{m}{k}\rmi^k \partial_x^{m - k}[z^k, \mathcal{C}]f(x, y; z),
\end{gather}
for all $m \in \bb{N}$ and $z \in \wpln$.
\end{lemma}
\begin{proof}
The second identity follows immediately from the binomial theorem. The first, is a consequence of the identity
\[
z^m - \zeta^m = (z - \zeta)\sum_{k = 0}^{m - 1} z^k \zeta^{m - k - 1}.\qedhere
\]
\end{proof}
Notice that these lemmas imply lemma \ref{L:commutators}. Also Schwartz regularity is not necessary for the proof but makes the calculations easier. The 
results to follow holds with much weaker regularity conditions cf.~\eqref{Eq:scatteringoperator}.

We can now state and prove the following theorem.
\begin{theorem}[The Inverse Spectral Theorem]\label{Th:inversescattering}
Let $\Scr{S}$ be a spectral operator of the form \eqref{Eq:scatteringoperator} defined by a function $F(z)$ that is small in the sense that
\begin{equation}
\sup_{n \in \sZ} n^2 \sup_{ \tau \in \bb{R} } \abs{ F(\tfrac{\omega}{2}n, \tau) } < 1, \quad F(0, \tau) = 0, \,\forall \:\tau \in \bb{R},
\end{equation}
and
\begin{equation}
\bigg(\int_{\!-\infty}^\infty \abs{ F(\tfrac{\omega}{2}n, \tau) }^2 \,\rmd\tau\bigg)^\frac{1}{2} = O\bigg( \frac{1}{n^2} \bigg), \quad \forall \:n \in \sZ.
\end{equation}
Then, the equation
\begin{equation}\label{Eq:CSm}
\mu = 1 + \mathcal{C}\Scr{S}\mu,
\end{equation}
has a unique solution $\mu$ in $L^\infty(E)$, holomorphic in $\wpln$ with jump
\begin{equation}
\mu^+(x, y; z) - \mu^-(x, y; z) = F(z)\rme^{-\rmi( z + \bar{z} )x + (z^2 - \bar{z}^2)y} \mu^-( x, y; -\bar{z} )
\end{equation}
across the contours $L_n, \,n\in \sZ$ for all $(x, y) \in \Omega$. Moreover this function $\mu$ solves the perturbed heat equation
$P(\partial + w)\mu = -u\mu$ for a potential $u(x, y)$ which may be represented by the formula
\begin{equation}\label{Eq:solution}
u(x, y) = \frac{1}{\pi}\partial_x \sideset{}{'}\sum_{n = -\infty}^\infty \int_{L_n} (\Scr{S}\mu)(x, y; z) \,\rmd z.
\end{equation}
\end{theorem}
\begin{proof}
Proposition \ref{Pr:lemma} and the contraction mapping theorem guarantee a unique solution to~\eqref{Eq:inverse} in $\Scr{H}_\omega$ and hence, to
$\mu = 1 + \mathcal{C}\Scr{S}\mu$. The conditions on $F$ imply that $F$ is small in $L^2( \abs{\!\real z} ) \cap L^\infty(\Lambda)$. Thus, proposition 
\ref{Pr:important} yields that $\mu \in L^\infty(E)$. Also by the smallness assumption, $F \in \ell^\infty( L^2( \bb{R} ) )$. Hence, the term
\[
\sideset{}{'}\sum_{n = -\infty}^\infty \frac{1}{2\pi}\int_{\!-\infty}^\infty
\frac{F(\frac{\omega}{2}n, \tau)\rme^{ -\rmi\omega n(x - 2\tau y) } }{\frac{\omega}{2}n + \rmi\tau - z} \,\rmd\tau
\]
in~\eqref{Eq:inverse} represents a function holomorphic in $\wpln$ with respect to $z$. Thus, again by proposition \ref{Pr:lemma} we see that
\[
\op{J}(\mu - 1)_n(x, y; \tau) = F(\tfrac{\omega}{2}n, \tau)\rme^{ -\rmi\omega n(x - 2\tau y) } \mu^-(x, y; -\tfrac{\omega}{2}n + \rmi\tau),
\]
which implies that across the contour $\real z = \frac{\omega}{2}n, \,\imaginary z = \tau \in \bb{R}$,
\[
\mu^+(x, y; z) - \mu^-(x, y; z) = F(z)\rme^{-\rmi( z + \bar{z} )x + (z^2 - \bar{z}^2)y} \mu^-( x, y; -\bar{z} ).
\]

Now applying $P(\partial + w)$ to both sides of the equation $\mu = 1 + \mathcal{C}\Scr{S}\mu$ and using lemma \ref{L:commutatorss} yields
\[
P(\partial + w)\mu = \mathcal{C}\Scr{S}P(\partial + w)\mu - \frac{1}{\pi}\partial_x \sideset{}{'}\sum_{n = -\infty}^\infty \int_{L_n}
\Scr{S}\mu \,\rmd\zeta.
\]
This last term has no $z$-dependence. Call it $u(x, y)$. Then,
\[
( \op{Id} - \mathcal{C}\Scr{S} )P(\partial + w)\mu = -u(x, y),
\]
hence,
\[
P(\partial + w)\mu = -u(x, y)( \op{Id} - \mathcal{C}\Scr{S} )^{-1} 1 = -u(x, y)\mu.\qedhere
\]
\end{proof}
To successfully complete the inverse spectral transform, we should show that the spectral data associated to the potential $u$ defined in \eqref{Eq:solution} 
coincide with the function $F$ of theorem \ref{Th:inversescattering}. Thus, based on the analysis done in the direct problem, it is of great importance to 
show that $\wh{u}$ is small in $L^2 \cap L^\infty( \Scr{C} )$. A sufficient condition for the right hand side of \eqref{Eq:solution} to make sense is a $F$ to 
be a member of the Schwartz class. However, milder conditions can be found.

Taking the Fourier transform (in $(x, y)$) of equation \eqref{Eq:CSm} we obtain
\[
[\mu - 1]\mh(m, \xi; z) = [\mathcal{C}\Scr{S}1]\mh(m, \xi; z) + [ \mathcal{C}\Scr{S}(\mu - 1) ]\mh(m, \xi; z).
\]
First, we will show that this equation has a unique solution in $L^1 (\Scr{C} )$. 

Now, from theorem \ref{Th:inversescattering}, $\mu$ satisfies equation
$P(\partial + w)\mu = -u\mu$, and taking the Fourier transform once again and splitting the right hand side, we have (recall \eqref{Eq:contractionform})
\begin{equation}
P_z(m, \xi)( \wh{\mu - 1} )(m, \xi; z) = \wh{u\mu}(m, \xi; z) = \wh{u}(m, \xi) + \frac{1}{2\pi}\wh{u} \ast \wh{\mu - 1}(m, \xi; z),
\end{equation}
arriving at the following Fredholm integral equation for $\wh{u}$:
\begin{equation}\label{Eq:fouriertransofu}
\wh{u}(m, \xi) = -\frac{1}{2\pi}( \op{K}_{ [\mu - 1]\mh } \wh{u} )(m, \xi; z) + P_z(m, \xi)( \wh{\mu - 1} )(m, \xi; z),
\end{equation}
where $\op{K}_{ [\mu - 1]\mh }$ denotes the operator ``convolution in $(m, \xi)$ by $\wh{\mu - 1}(m, \xi; z)$'' on $L^p( \Scr{C} )$ for
$1 \leq p \leq \infty$. Since $\wh{\mu - 1}(m, \xi; z)$ is in $L^1( \Scr{C} )$, the first term of equation \eqref{Eq:fouriertransofu} has finite norm on
$L^2 \cap L^\infty( \Scr{C} )$. It will be shown that the other term satisfies the equation
\begin{equation}\label{Eq:RFm}
P_z(m, \xi)( \wh{\mu - 1} )(m, \xi; z) = F( \zeta(m, \xi) ) + P_z(m, \xi)\op{R}_F \wh{\mu - 1}(m, \xi; z),
\end{equation}
for some appropriate operator $\op{R}_F$ depending on the spectral data $F$. This dependence, will indicate appropriate conditions on $F$ for the
smallness of $P_z( \wh{\mu - 1} )$ on $L^2 \cap L^\infty( \Scr{C} )$ and the analysis of the inverse problem will be concluded.
\begin{proposition}
Under the assumptions of theorem \ref{Th:inversescattering}, the equation
\begin{equation}\label{Eq:fouriertransequation}
[\mu - 1]\mh(m, \xi; z) = [\mathcal{C}\Scr{S}1]\mh(m, \xi; z) + [ \mathcal{C}\Scr{S}(\mu - 1) ]\mh(m, \xi; z),
\end{equation}
has a unique solution in $L^1 (\Scr{C} )$, uniformly in $z \in \wpln$. Moreover we have the explicit estimate
\begin{equation}\label{Eq:Loneestimate}
\norm{ \wh{\mu - 1}(\cdot, \cdot; z) }_{L^1 ( \Scr{C} ) } \leq \frac{ \norm{F}_{\Lambda} }{ 1 - \norm{F}_{\Lambda} },
\end{equation}
for all $z \in \wpln$, where
\begin{equation}\label{Eq:scatteringnorm}
\norm{F}_{\Lambda} \= C\max\{ 2\norm{F}_{ L^2( \abs{\!\real z} ) }, \,\norm{F}_{ L^\infty(\Lambda) } \}.
\end{equation}
\end{proposition}
\begin{proof}
First, observe that
\begin{align*}
\mathcal{C}\Scr{S}1(x, y; z) &= \frac{1}{2\pi \rmi}\sideset{}{'}\sum_{n = -\infty}^\infty \int_{L_n} \frac{ \Scr{S}1(x, y; \zeta) }{\zeta - z} \,\rmd\zeta\\
&= \frac{1}{2\pi \rmi}\sideset{}{'}\sum_{n = -\infty}^\infty \int_{L_n} \frac{ F(\zeta)\rme^{ \rmi r_0(\zeta) \cdot (\omega x, y) } }{\zeta - z} \,\rmd\zeta\\
&= \frac{1}{2\pi \rmi}\sideset{}{'}\sum_{n = -\infty}^\infty \int_{\!-\infty}^\infty
\frac{ F( \zeta(n, \tau) )\rme^{ \rmi r_0( \zeta(n, \tau) ) \cdot (\omega x, y) } }{-\frac{\omega}{2}n - \rmi\frac{\tau}{2\omega n}- z}
\Big( \frac{-\rmi}{2\omega n} \Big) \,\rmd\tau\\
&= \frac{1}{2\pi}\sideset{}{'}\sum_{n = -\infty}^\infty \int_{\!-\infty}^\infty
\frac{ F( \zeta(n, \tau) )\rme^{ \rmi(n, \tau) \cdot (\omega x, y) } }{ (\omega n)^2 + 2\omega n z + \rmi\tau } \,\rmd\tau\\
&= \frac{1}{2\pi}\sideset{}{'}\sum_{n = -\infty}^\infty \int_{\!-\infty}^\infty
\frac{ F( \zeta(n, \tau) ) }{ P_z(n, \tau) }\rme^{\rmi\omega n x} \rme^{\rmi\tau y} \,\rmd\tau
= \bigg[ \frac{ F( \zeta(n, \tau) ) }{ P_z(n, \tau) } \bigg]\spcheck\!(x, y; z).
\end{align*}
Thus,
\begin{equation}\label{Eq:fouriertransCS1}
[\mathcal{C}\Scr{S}1]\mh(n, \tau; z) = \frac{ (F \circ \zeta)(n, \tau) }{ P_z(n, \tau) }.
\end{equation}
Furthermore, for a function $f$ in the Schwartz class such that its limits as $z$ approaches $L_n$ from the stripes $S_n$ exist we have
\begin{align}\label{Eq:fouriertransCSf}
[\mathcal{C}\Scr{S}f]\mh(m, \xi; z) &= \bigg[\frac{1}{2\pi \rmi}\sideset{}{'}\sum_{n = -\infty}^\infty \int_{L_n}
\frac{ F(\zeta)\rme^{ \rmi r_0(\zeta) \cdot (\omega x, y) } f^-( x, y; -\bar{\zeta} ) }{\zeta - z} \,\rmd\zeta\bigg]\mh(m, \xi; z) \notag\\
&= \frac{1}{2\ell}\int_{\!-\infty}^\infty \int_{\!-\ell}^\ell \bigg[\frac{1}{2\pi \rmi}\sideset{}{'}\sum_{n = -\infty}^\infty \int_{L_n}
\frac{ F(\zeta)\rme^{ \rmi r_0(\zeta) \cdot (\omega x, y) } f^-( x, y; -\bar{\zeta} ) }{\zeta - z} \,\rmd\zeta\bigg] \notag\\
&\ph{=x} \qquad \times \rme^{-\rmi\omega m x - \rmi\xi y} \,\rmd x\rmd y \notag\\
&= \frac{1}{2\pi \rmi}\sideset{}{'}\sum_{n = -\infty}^\infty \int_{L_n}
\frac{ F(\zeta) }{\zeta - z} \,\wh{f^-}( (m, \xi) - r_0(\zeta); -\bar{\zeta} ) \,\rmd\zeta\notag\\
&= \frac{1}{2\pi}\sideset{}{'}\sum_{n = -\infty}^\infty \int_{\!-\infty}^\infty
\frac{ F( \zeta(n,\tau) ) }{ P_z(n, \tau) } \,\wh{f^-}( m - n, \xi - \tau; -\bar{\zeta}(n, \tau) ) \,\rmd\tau \notag\\
&\equiv \op{R}_F \wh{f}(m, \xi; z).
\end{align}

By ``integrating''~\eqref{Eq:fouriertransCSf} first in $(m, \xi)$ and using Fubini's theorem and the \hyperlink{L:basiclemma}{Basic Lemma} we obtain that 
if the $L^1( \Scr{C} )$ norm of $\wh{f}(m, \xi; z)$ is a bounded function of $z$, then so is the $L^1( \Scr{C} )$ norm of $\op{R}_F \wh{f}(m, \xi; z)$. 
Assuming $F(z)$ to be small in $L^2( \abs{\!\real z} ) \cap L^\infty(\Lambda)$, the map $\wh{f} \mapsto \op{R}_F \wh{f}$ is a contraction of
$L^1( \Scr{C} )$ for every $z \in \wpln$. Likewise, $(F \circ \zeta)(m, \xi)/P_z(m, \xi)$ is in $L^1 (\Scr{C} )$ uniformly in $z$ by the 
\hyperlink{L:basiclemma}{Basic Lemma}. Thus, equation~\eqref{Eq:fouriertransequation} has a unique solution $\wh{\mu - 1}(m, \xi; z)$ in
$L^1 (\Scr{C} )$ uniformly in $z \in \wpln$.

Using equations \eqref{Eq:fouriertransCS1} and \eqref{Eq:fouriertransCSf}, one can see that the function $\wh{\mu - 1}$ satisfies the Fredholm integral
equation
\begin{equation}\label{Eq:newfredholm}
\wh{\mu - 1}(m, \xi; z) = \frac{ (F \circ \zeta)(m, \xi) }{ P_z(m, \xi) } + \op{R}_F \wh{\mu - 1}(m, \xi; z).
\end{equation}
Then, the \hyperlink{L:basiclemma}{Basic Lemma} yields
\begin{equation}
\Norm{ \frac{F \circ \zeta}{P_z} }_{L^1( \Scr{C} ) } \leq \norm{F}_{\Lambda}.
\end{equation}
Also,
\begin{equation}
\norm{ \op{R}_F \wh{\mu - 1} }_{L^1( \Scr{C} ) } \leq \norm{ \wh{\mu - 1} }_{L^1( \Scr{C} ) } \norm{F}_{\Lambda}.
\end{equation}
Thus, we get
\[
\norm{ \wh{\mu - 1}(\cdot, \cdot; z) }_{L^1 ( \Scr{C} ) } \leq \frac{ \norm{F}_{\Lambda} }{ 1 - \norm{F}_{\Lambda} },
\]
which is independent of $z$.
\end{proof}
\begin{remark}
The inverse Fourier transform $\mu(x, y; z)$ of the unique solution of equation \eqref{Eq:fouriertransequation} must be the unique solution to
$\mu = 1 +\mathcal{C}\Scr{S}\mu$. From this we get that for each $z \in \wpln$, $\mu(x, y; z) \to 1$ as
$\abs{y} \to \infty$, which is a direct consequence of the Riemann--Lebesgue lemma. 
\end{remark}

By Young's inequality and the estimate~\eqref{Eq:Loneestimate}, we have the estimate for the norm of the operator $\op{K}_{ [\mu - 1]\mh }$ :
\begin{equation}\label{Eq:young}
\norm{ \op{K}_{ [\mu - 1]\mh } }_\mrm{op} \leq \frac{ \norm{F}_{L_n} }{ 1 - \norm{F}_{L_n} },
\end{equation}
for $1 \leq p \leq \infty$, uniformly in $z \in \wpln$. Therefore, an estimation of the $L^2$ and $L^\infty$ norms of $\op{K}_{ [\mu - 1]\mh } \wh{u}$ is 
immediately provided. Now, multiplying equation \eqref{Eq:newfredholm} by $P_z$ we arrive at equation \eqref{Eq:RFm}. The following lemma which allows 
us to ``commute'' $P_z$ and $\op{R}_F$ shows that \eqref{Eq:RFm} can be written as a Fredholm integral equation:
\begin{equation}\label{Eq:Mfredholm}
P_z \wh{\mu - 1} = F \circ \zeta + \op{A}_F (P_z \wh{\mu - 1} ).
\end{equation}
\begin{lemma}
Suppose $f(m, \xi; z)$ is in the Schwartz class over $\Scr{C} \times \wpln$ and the limits $f^\pm( m, \xi; \zeta )$ as $z$ approaches $L_n$ from the strips 
$S_n$ exist. Then,
\begin{gather}
\op{R}_F f = \op{R}_{ (F/P_z) }(P_z f), \label{Eq:AFidentityone}\\[2pt]
P_z(m, \xi)\op{R}_{ (F/P_z) } f(m, \xi; z) = \op{R}_F f(m, \xi; z) - \op{R}_F f( m, \xi; \zeta(m, \xi) ) \equiv 
\op{A}_F f(m, \xi; z). \label{Eq:AFidentitytwo}
\end{gather}
\end{lemma}
\begin{proof}
For equation \eqref{Eq:AFidentityone}, observe that
\begin{align*}
\op{R}_{ (F/P_z) }(P_z f) &= \frac{1}{2\pi \rmi}\sideset{}{'}\sum_{n = -\infty}^\infty \int_{L_n} \frac{ F(\zeta) }{ P_\zeta(m, \xi)(\zeta - z) }\\
&\ph{=x} \qquad \times P_{ -\bar{\zeta} }( (m, \xi) - r_0(\zeta) )f^-( (m, \xi) - r_0(\zeta); -\bar{\zeta} ) \,\rmd\zeta.
\end{align*}
The result follows from the definition of $P_z$:
\begin{align*}
P_{ -\bar{\zeta} }( (m, \xi) - r_0(\zeta) ) &= P_{ -\bar{\zeta} }( (m, \xi) - [-( \zeta + \bar{\zeta} )/\omega, -\rmi(\zeta^2 - \bar{\zeta}^2) ] )\\
&= P_{ -\bar{\zeta} }( m + ( \zeta +\bar{\zeta} )/\omega, \xi + \rmi(\zeta^2 - \bar{\zeta}^2) )\\
&= -P(\rmi\omega m + \rmi( \zeta + \bar{\zeta} ) - \rmi\bar{\zeta}, \rmi\xi + \bar{\zeta}^2 - \zeta^2 - \bar{\zeta}^2)\\
&= -P(\rmi\omega m + \rmi\zeta, \rmi\xi - \zeta^2) = P_\zeta(m, \xi),
\end{align*}
hence the two polynomials cancel. Likewise, equation~\eqref{Eq:AFidentitytwo} is a consequence of the following observation:
\begin{align*}
\frac{ P_z(m, \xi) }{ (\zeta - z)P_\zeta(m, \xi) } &= \frac{ (\omega m)^2 + 2\omega m z + \rmi\xi }
{ (\zeta - z)( (\omega m)^2 + 2\omega m\zeta + \rmi\xi ) }\\
&= \frac{\frac{\omega m}{2} + \rmi\frac{\xi}{2\omega m} + z}{ (\zeta - z)(\frac{\omega m}{2} + \rmi\frac{\xi}{2\omega m} + \zeta) }
= \frac{-\zeta(m, \xi) + z}{ (\zeta - z)(-\zeta(m, \xi) + \zeta) }\\
&= \frac{1}{\zeta - z} - \frac{1}{ \zeta - \zeta(m, \xi) }.\qedhere
\end{align*}
\end{proof}

In order to ensure continuity of $\op{A}_F$ on $L^2 \cap L^\infty( \Scr{C} )$ we will restrict $F(z)$ to be a member of an appropriate
subspace of $L^2( \abs{\!\real z} ) \cap L^\infty(\Lambda)$, achieving the desired behaviour of $\op{A}_F$.
\begin{definition}
Let $k$ be a nonnegative integer and for $(a, b) \in \bb{C}^2$ set
\[
\langle a, b \rangle^k \equiv ( 1 + \abs{ (a, b) } )^k.
\]
We define the $k$\textsuperscript{th} weighted subspace of $L^2 \cap L^\infty( \Scr{C} )$ as
\[
W^k \equiv W^k ( L^2 \cap L^\infty( \Scr{C} ) ) \=
\{ f(q) \in L^2 \cap L^\infty( \Scr{C} ) \colon \langle q \rangle^k \!f(q) \in L^2 \cap L^\infty( \Scr{C} ) \}.
\]
\end{definition}
\noi This is a Banach space with the norm
\[
\norm{f}_{W^k} \= \sum_{j = 0}^k \binom{k}{j}\norm{ \langle q \rangle^j \!f(q) }_{ L^2 \cap L^\infty( \Scr{C} ) },
\]
where
\[
\norm{f}_{ L^2 \cap L^\infty( \Scr{C} ) } = \norm{f}_{ L^2( \Scr{C} ) } + \norm{f}_{ L^\infty( \Scr{C} ) }.
\]
\begin{definition}
The $k$\textsuperscript{th} weighted subspace of $L^2( \abs{\!\real z} ) \cap L^\infty(\Lambda)$ is denoted by $W_{\zeta}^k \equiv
W_{\zeta}^k ( L^2( \abs{\!\real z} ) \cap L^\infty(\Lambda) )$ and consists of those functions $f(z)$ for which $f \circ \zeta(q) \in W^k$.
This is a Banach space with the norm
\[
\norm{f}_{W_\zeta^k} \= \norm{f \circ \zeta}_{W^k}.
\]
Finally, if $f(q; z) \in L^\infty(\Scr{C} \times \wpln)$, introduce the function
\[
f^\star(q) \= \essup_{z \in \wpln} \abs{ f(q; z) },
\]
and define the $k$\textsuperscript{th} weighted max subspace
\begin{align*}
W_\infty^k = W_\infty^k ( L^\infty(\Scr{C} \times \wpln) ) &\= \{f \in L^\infty(\Scr{C} \times \wpln) \colon \langle q \rangle^j \!f^\star(q) \\
&\ph{\=} \quad \text{ is essentially bounded for all } 0 \leq j \leq k\}.
\end{align*}
\end{definition}
\noi Again, this is a Banach space with the norm
\[
\norm{f}_{W_\infty^k} \= \max_{0 \leq j \leq k} \norm{ \langle q \rangle^j \!f^\star(q) }_{ L^\infty( \Scr{C} ) }.
\]

The spaces $W^k$ and $W_\infty^k$ satisfy the following embedding properties.
\begin{proposition}\label{Pr:embeddingthm}
For every nonnegative integer $k$,
\begin{equation}
W^k \subset W_\infty^k \;\text{ and }\; \norm{f}_{W_\infty^k} \leq \norm{f}_{W^k}, \ \text{ for } f \in W^k.
\end{equation}
Moreover, $W_\infty^{k + 2} \subset W^k$ in the sense that if $f \in W_\infty^{k + 2}$, then $f^\star \in W^k$ and the embedding inequality
\begin{equation}\label{Eq:embedding}
\norm{f^\star}_{W^k} < 3 \cdot 2^k \norm{f}_{ W_\infty^{k + 2} }
\end{equation}
holds.
\end{proposition}
\begin{proof}
The first embedding follows immediately from the definitions of these spaces and $f^\star$ (notice that if $f \in W^k$, then $f^\star = \abs{f}$).
Likewise, since $1/\langle q \rangle^2$ belongs to $L^2(\Scr{C}^*)$ with norm less than 2 and $f \in W_\infty^{k + 2}$, then
$\langle q \rangle^j \!f^\star(q)$ is essentially bounded for all $0 \leq j \leq k + 2$, in particular $f^\star(q) \in L^\infty( \Scr{C} )$
and $\langle q \rangle^k \!f^\star(q) \in L^\infty( \Scr{C} )$. Also, for all $0 \leq r \leq k$
\begin{align*}
\sideset{}{'}\sum_{m = -\infty}^\infty \int_{\!\infty}^\infty \abs{ \langle m, \xi \rangle^r \!f^\star(m, \xi) }^2 \,\rmd\xi &=
\sideset{}{'}\sum_{m = -\infty}^\infty \int_{\!\infty}^\infty
\frac{\abs{ \langle m, \xi \rangle^{r + 2} \!f^\star(m, \xi) }^2}{\abs{ \langle m, \xi \rangle^2 }^2} \,\rmd\xi\\
&\leq \norm{ \langle q \rangle^{r + 2} \!f^\star(q) }_{ L^\infty( \Scr{C} ) }^2 \Norm{ \frac{1}{\langle q \rangle^2} }_{ L^2(\Scr{C}^*) }^2,
\end{align*}
hence, $\langle q \rangle^r \!f^\star(q) \in L^2( \Scr{C} )$ and so $f^\star(q) \in L^2( \Scr{C} )$ and
$\langle q \rangle^k \!f^\star(q) \in L^2( \Scr{C} )$ as well. Therefore, $f^\star \in W^k$. Comparing norms,
\begin{align*}
\norm{f^\star}_{W^k} &= \sum_{j = 0}^k \binom{k}{j}\norm{ \langle m, \xi \rangle^j \!f^\star(m, \xi) }_{ L^2 \cap L^\infty( \Scr{C} ) }\\
&= \sum_{j = 0}^k \binom{k}{j}
\big[ \norm{ \langle m, \xi \rangle^j \!f^\star(m, \xi) }_{L^2( \Scr{C} ) } + \norm{ \langle m, \xi \rangle^j \!f^\star(m, \xi) }_{L^\infty( \Scr{C} ) } \big]\\
&\leq \sum_{j = 0}^k \binom{k}{j} \bigg[ \norm{ \langle m, \xi \rangle^{j + 2} \!f^\star(m, \xi) }_{ L^\infty( \Scr{C} ) }
\Norm{ \frac{1}{\langle m, \xi \rangle^2} }_{ L^2( \Scr{C} ) } + \norm{f}_{ W_\infty^{k + 2} } \bigg]\\
&< \sum_{j = 0}^k \binom{k}{j} \big[ \,2\norm{f}_{ W_\infty^{k + 2} } + \norm{f}_{ W_\infty^{k + 2} } \big].
\end{align*}
Thus, we get the inequality \eqref{Eq:embedding}.
\end{proof}

We can now obtain a bound for $\op{A}_F$ on these subspaces.
\begin{lemma}
Let $F \in W_\zeta^k$ and $f \in W_\infty^k$. Then, for $k \in \bb{N}_0$
\[
\op{A}_F f \in W_\infty^k,
\]
and
\[
\norm{\op{A}_F}_\mrm{op} \leq \frac{C}{\pi} \norm{F}_{W_\zeta^k}.
\]
\end{lemma}
\begin{proof}
By the definition of $\op{A}_F$ and the triangle inequality
\[
\norm{\op{A}_F f}_{W_\infty^k} \leq 2\norm{\op{R}_F f}_{W_\infty^k}.
\]
Now
\begin{align*}
\langle m, \xi \rangle = 1 + \abs{ (m, \xi) } &= 1 + \abs{ (m - m', \xi - \xi') + (m', \xi') }\\
&\leq 1 + \abs{ (m - m', \xi - \xi') } + \abs{ (m', \xi') }\\
&< \langle m - m', \xi - \xi' \rangle + \langle m', \xi' \rangle.
\end{align*}
Thus, by the binomial theorem
\[
\langle m, \xi \rangle^k \leq \sum_{j = 0}^k \binom{k}{j} \langle m - m', \xi - \xi' \rangle^{k - j} \langle m', \xi' \rangle^j.
\]
Therefore,
\begin{align*}
\abs{ \langle m, \xi \rangle^k \op{R}_F f(m, \xi; z) }&\leq \frac{1}{2\pi}\sideset{}{'}\sum_{n = -\infty}^\infty \int_{\!-\infty}^\infty
\Abs{ \frac{ \langle m, \xi \rangle^k F( \zeta(n,\tau) ) }{ P_z(n, \tau) } }\\
&\ph{\leq x} \quad \times \abs{ f^-( m - n, \xi - \tau; -\bar{\zeta}(n, \tau) ) } \,\rmd\tau\\
&\leq \frac{1}{2\pi}\sum_{j = 0}^k \binom{k}{j} \sideset{}{'}\sum_{n = -\infty}^\infty \int_{\!-\infty}^\infty
\Abs{ \frac{ \langle n, \tau \rangle^j F( \zeta(n,\tau) ) }{ P_z(n, \tau) } }\\
&\ph{\leq x} \quad \times \abs{ \langle m - n, \xi - \tau \rangle^{k - j} f^-( m - n, \xi - \tau; -\bar{\zeta}(n, \tau) ) } \,\rmd\tau\\
&\leq \frac{1}{2\pi}\norm{f}_{W_\infty^k} \sum_{j = 0}^k \binom{k}{j}\sideset{}{'}\sum_{n = -\infty}^\infty \int_{\!-\infty}^\infty
\Abs{ \frac{ \langle n, \tau \rangle^j F \circ \zeta(n,\tau) }{ P_z(n, \tau) } } \,\rmd\tau\\
&\leq \frac{C}{2\pi}\norm{f}_{W_\infty^k}
\sum_{j = 0}^k \binom{k}{j}\norm{ \langle n, \tau \rangle^j F \circ \zeta(n,\tau) }_{ L^2 \cap L^\infty( \Scr{C} ) }\\
&= \frac{C}{2\pi}\norm{f}_{W_\infty^k} \norm{F}_{W_\zeta^k}.\qedhere
\end{align*}
\end{proof}

Finally we have the following theorem.
\begin{theorem}\label{Th:inversedecay}
Suppose that $(1 + (\real z)^2 + (\real z \imaginary z)^2)F(z)$ is sufficiently small in $L^2( \abs{\!\real z} ) \cap L^\infty(\Lambda)$. Then, there exists a 
function $u(x, y) \in L^2(\Omega)$ with bounded Fourier transform, such that $F(z)$ is the spectral data associated to $u$.
\end{theorem}
\begin{proof}
Using equation \eqref{Eq:Mfredholm} and the contraction mapping principle in $W_\infty^k$, we obtain the estimate
\[
\norm{P_z \wh{\mu - 1} }_{W_\infty^k} \leq \frac{ \norm{F \circ \zeta}_{W_\infty^k} }{ 1 - \frac{C}{\pi}\norm{F}_{W_\zeta^k} }.
\]
On the other hand, comparing norms with the aid of inequality \eqref{Eq:embedding}, and observing that $W^0 = L^2 \cap L^\infty( \Scr{C} )$, one obtains 
(for $k = 2$)
\[
\norm{P_z \wh{\mu - 1} }_{ L^2 \cap L^\infty( \Scr{C} ) } < \frac{ 3\norm{F \circ \zeta}_{W_\infty^2} }{ 1 - \frac{C}{\pi}\norm{F}_{W_\zeta^2} }
\leq \frac{ 3\norm{F}_{W_\zeta^2} }{ 1 - \frac{C}{\pi}\norm{F}_{W_\zeta^2} },
\]
this estimate being uniform in $z$. Combining it with \eqref{Eq:young} and \eqref{Eq:fouriertransofu} yields
\begin{align}\label{Eq:2inftyestimate}
\norm{ \wh{u} }_{ L^2 \cap L^\infty( \Scr{C} ) } &\leq \frac{ \norm{P_z \wh{\mu - 1} }_{ L^2 \cap L^\infty( \Scr{C} ) } }
{ 1 - \frac{1}{2\pi}\norm{ \op{K}_{ [\mu - 1]\mh } }_\mrm{op} } \notag\\
&\leq \bigg( \frac{ 1 - \norm{F}_{\Lambda} }{ 2\pi - (2\pi + 1)\norm{F}_{\Lambda} } \bigg)
\bigg( \frac{ 6\pi\norm{F}_{W_\zeta^2} }{ 1 - \frac{C}{\pi}\norm{F}_{W_\zeta^2} } \bigg).
\end{align}

$F$ is a member of $W_\zeta^2$ as well as $L^2( \abs{\!\real z} ) \cap L^\infty(\Lambda)$ because of the decay hypothesis on $F$. Let $\Scr{S}$ be
the spectral operator associated to $F$
and $\mu(x, y; z)$ the unique solution to $\mu = 1 + \mathcal{C}\Scr{S}\mu$ provided by the Inverse Spectral theorem. This $\mu$ is a solution to 
equation $P(\partial + w)\mu = -u\mu$ with $u(x, y)$ given by
\[
u(x, y) = \frac{1}{\pi}\partial_x \sideset{}{'}\sum_{n = -\infty}^\infty \int_{L_n} (\Scr{S}\mu)(x, y; z) \,\rmd z.
\]
By \eqref{Eq:2inftyestimate} and Plancherel's theorem, $u$ is small in $L^2 (\Omega)$ and has a small bounded Fourier transform. By the proof of theorem 
\ref{Th:exist_unique}, this guarantees that there is exactly one solution $\mu'$ to equation $P(\partial + w)\mu = -u\mu$ that has the property
$\wh{\mu' - 1} \in L^1( \Scr{C} )$. This fact allows the extension of the forward spectral transform to the ball
$\{ u \in L^2(\Omega) \colon \max\{\sqrt{\omega}\norm{u}_2, \,\norm{ \wh{u} }_\infty\} < \frac{2\pi}{C} \}$ by using
$(\Scr{S}'\mu')(x, y; z) = \mu'^+(x, y; z) - \mu'^-(x, y; z)$. This extension is well defined and agrees with the definition of the spectral data 
\eqref{Eq:scatteringoperator} on the
ball $\{ u \in L^1(\Omega) \cap L^2(\Omega) \colon \max\{\omega\norm{u}_1, \,\sqrt{\omega}\norm{u}_2\} < \frac{2\pi}{C} \}$.

Now $\mu$ solves $P(\partial + w)\mu = -u\mu$ and it also has the property $\wh{\mu - 1} \in L^1( \Scr{C} )$ by~\eqref{Eq:Loneestimate}. By 
uniqueness, $\mu = \mu'$, therefore, the spectral operator $\Scr{S}'$ associated to $u$ is the same as $\Scr{S}$.
\end{proof}