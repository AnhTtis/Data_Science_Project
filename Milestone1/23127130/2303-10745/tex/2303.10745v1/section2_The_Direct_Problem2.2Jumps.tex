\subsection[Sectionally Holomorphicity of the Eigenfunctions and formulation of a Riemann--Hilbert problem]{Sectionally Holomorphicity of the Eigenfunctions and formulation of a
Riemann--Hilbert problem}\label{s:jumps}
\separate

The function $\wh{u\mu}/P_z$ is discontinuous whenever the real part of $z$ belongs to $\frac{\omega}{2}\sZ$. Consider the vertical lines
$\real z = \frac{\omega}{2}n$, $n \in \sZ$. When $z$ lies on such a line, the integral corresponding to $m = -n$ in equation \eqref{Eq:eigenfunction} is 
singular. As it turns out, this singularity is integrable if we assume some smoothness for the potential $u$. If $u$ belongs in some suitable Sobolev space, the 
function $\mu$ will have a limit from the left and from the right of each one of these lines. Call these limits $\mu^-$ and $\mu^+$ respectively, and let
$\op{J}\mu$ be the jump of $\mu$ across such a line, i.e., $\op{J}\mu = \mu^+ - \mu^-$. Now, $\mu$ satisfies the analytic family of differential
equations $P( \partial + w(z) )\mu = -u\mu$ in the parameter $z \in \bb{C}$. Hence,
\begin{equation}
P( \partial + w(z) )\op{J}\mu = -u\op{J}\mu.
\end{equation}
From the existence and uniqueness theorem for this equation, we have $\op{J}\mu = \Scr{S}\mu$ for some linear operator $\Scr{S}$. Call the map
$u \mapsto \Scr{S}$ the forward spectral transform. Knowledge of this map amounts to knowing $\mu$ and consequently $u$. It remains to calculate these 
jumps. We begin by establishing an important lemma.
\begin{lemma}\label{L:rapiddeacy}
Suppose $\maxnorm{u} < 2\pi/C$ and that for some multi-index $\alpha = (\alpha_1, \alpha_2)$, $u$ is smooth to order $\abs{\alpha}$ and
$\abs{\partial^{\alpha'} u} \in L^1(\Omega) \cap L^2(\Omega)$, for all multi-indices $\alpha'$ such that $\abs{\alpha'} \leq \abs{\alpha}$. Then, the 
function $\mu$ is smooth to order $\abs{\alpha}$ and
\begin{equation}\label{Eq:bigo}
\abs{ \wh{u\mu}(m, \xi; z) } = O\bigg( \frac{1}{ 1 + \abs{\omega m}^{\alpha_1} + \abs{\xi}^{\alpha_2} } \bigg),
\end{equation}
for $(m, \xi; z) \in \Scr{C} \times \wpln$.
\end{lemma}
\begin{proof}
Let $z \in \wpln$. To establish the smoothness of $\mu$ we use induction. Suppose $\partial^{\alpha'} \mu \in L^\infty(E)$ for all $\alpha' < \alpha$. Then,
\[
\partial^\alpha \mu = \partial^\alpha(\op{Id} - \Scr{N}_ u)^{-1} 1 = (\op{Id} - \Scr{N}_u)^{-1} [\partial^\alpha, \Scr{N}_u]\mu,
\]
since
\begin{align*}
(\op{Id} - \Scr{N}_u)\partial^\alpha \mu &= \partial^\alpha \mu - \Scr{N}_u \partial^\alpha \mu\\
&= \partial^\alpha(1 +  \Scr{N}_u \mu) - \Scr{N}_u \partial^\alpha \mu\\
&= \partial^\alpha \Scr{N}_u \mu - \Scr{N}_u \partial^\alpha \mu.
\end{align*}
By Leibniz' rule, 
\begin{align*}
[\partial^\alpha, \Scr{N}_u]\mu &= \partial^\alpha \Scr{N}_u \mu - \Scr{N}_u \partial^\alpha \mu\\
&= \sum_{\alpha' \leq \alpha} \binom{\alpha}{\alpha'}(\partial^{\alpha - \alpha'} \Scr{N}_u)(\partial^{\alpha'} \mu) - \Scr{N}_u \partial^\alpha \mu\\
&= \sum_{\alpha' < \alpha} \binom{\alpha}{\alpha'}(\partial^{\alpha - \alpha'} \Scr{N}_u)(\partial^{\alpha'} \mu).
\end{align*}
 
Each operator $\partial^{\alpha - \alpha'} \Scr{N}_u$ is bounded from $L^\infty(\Omega)$ to $L^\infty(\Omega)$ for every $z \in \wpln$:
if $h \in L^\infty(\Omega)$, then
\begin{align*}
\partial^{\alpha - \alpha'} \bigg( \frac{ \wh{u h}(m, \xi) }{ P_z(m, \xi) }\rme^{\rmi\omega m x + \rmi\xi y} \bigg) &=
\frac{ \wh{u h}(m, \xi) }{ P_z(m, \xi) }(\rmi\omega m, \rmi\xi)^{\alpha - \alpha'} \rme^{\rmi\omega m x + \rmi\xi y}\\
&= \frac{ [\partial^{\alpha - \alpha'} u h]\mh(m, \xi) }{ P_z(m, \xi) }\rme^{\rmi\omega m x + \rmi\xi y}.
\end{align*}
But because of the smoothness of $u$, $[\partial^{\alpha - \alpha'} u h]\mh \in L^2( \Scr{C} ) \cap L^\infty( \Scr{C} )$. The rest follows by an application 
of dominated convergence and the \hyperlink{L:basiclemma}{Basic Lemma}.

Meanwhile, each term $\partial^{\alpha'} \mu$ belongs to $L^\infty(E)$ by the induction hypothesis, and the operator $(\op{Id} - \Scr{N}_u)^{-1}$ is 
bounded on $L^\infty(\Omega)$ for every $z \in \wpln$ because $\maxnorm{u} < 2\pi/C$. Thus, $\partial^\alpha \mu$ is bounded.

Now,
\begin{align*}
( \abs{\omega m}^{\alpha_1} + \abs{\xi}^{\alpha_2} )\abs{ \wh{u\mu}(m, \xi; z) } &= \abs{ (\rmi\omega m)^{\alpha_1}\wh{u\mu}(m, \xi; z) }
+ \abs{ (\rmi\xi)^{\alpha_2}\wh{u\mu}(m, \xi; z) }\\
&= \abs{ [\partial_x^{\alpha_1} u\mu]\mh(m, \xi; z) } + \abs{ [\partial_y^{\alpha_2} u\mu]\mh(m, \xi; z) }\\
&\leq \norm{\partial_x^{\alpha_1} u\mu}_1 + \norm{\partial_y^{\alpha_2} u\mu}_1.
\end{align*}
Since we also have that $\abs{ \wh{u\mu}(m, \xi; z) } \leq \norm{u\mu}_1$, equation \eqref{Eq:bigo} follows, hence the lemma is proved.
\end{proof}
\begin{theorem}\label{Th:zasympotics}
Suppose $\max\{\omega\norm{u}_1, \sqrt{\omega}\norm{u}_2\}$ is small, so that $\mu(x, y; z)$ is the unique solution to
\[
P( \partial + w(z) )\mu + u\mu = 0, \quad \mu \in L^\infty(E), \ \lim_{\abs{y} \to \infty} \mu(x, y; z) = 1.
\]
If in addition $y u(x, y) \in L^1(\Omega)$ and
\begin{equation}\label{Eq:weakderivatives}
u_x, u_y \in L^1(\Omega) \cap L^2(\Omega),
\end{equation}
then $\mu$ has pointwise one-sided limits at the lines $\real z= \frac{\omega}{2}n$, $n \in \sZ$, and
\begin{equation}
\lim_{ \substack{\abs{ \im{z} } \to \infty \\ \re{z} = const.} } \mu(x, y; z) = 1.
\end{equation}
\end{theorem}
\begin{proof}
Letting
\begin{equation}\label{Eq:Fouriercoefficients}
\mu_m(y; z) = \frac{1}{2\pi}\int_{\!-\infty}^\infty \frac{ \wh{u\mu}(m, \xi; z) }{ P_z(m, \xi) }\rme^{\rmi\xi y} \,\rmd\xi,
\end{equation}
we can convert $\mu(x, y; z)$ in the form of a Fourier series
\begin{equation}\label{Eq:Fourierseries}
\mu(x, y; z) = 1 + \sideset{}{'}\sum_{m = -\infty}^\infty \mu_m(y; z)\rme^{\rmi\omega m x}.
\end{equation}
Since $u_x$, $u_y \in L^1(\Omega) \cap L^2(\Omega)$, lemma \ref{L:rapiddeacy} shows that
\begin{equation}\label{Eq:firstorderbound}
\abs{ \wh{u\mu}(m, \xi; z) } = O\bigg( \frac{1}{ 1 + \abs{\omega m} + \abs{\xi} } \bigg).
\end{equation}
Hence,
\[
\abs{ \mu_m(y; z) } \leq \frac{1}{2\pi}\int_{\!-\infty}^\infty \frac{ \abs{ \wh{u\mu}(m, \xi; z) } }{ \abs{ P_z(m, \xi) } } \,\rmd\xi
\leq \frac{c}{2\pi}\int_{\!-\infty}^\infty \frac{1}{ 1 + \abs{\omega m} + \abs{\xi} }\frac{1}{ \abs{ P_z(m, \xi) } } \,\rmd\xi,
\]
for some positive, real constant $c$. Thus,
{\allowdisplaybreaks
\begin{align*}
\abs{ \mu_m(y; z) } &< \frac{c}{2\pi}\int_{\!-\infty}^\infty \frac{1}{ ( 1 + \abs{\xi} )\abs{ P_z(m, \xi) } } \,\rmd\xi\\
&\leq \frac{c}{2\pi}\bigg(\int_{\!-\infty}^\infty \frac{1}{ ( 1 + \abs{\xi} )^\frac{3}{2} } \,\rmd\xi\bigg)^\frac{2}{3}
\bigg(\int_{\!-\infty}^\infty \frac{1}{\abs{ P_z(m, \xi) }^3} \,\rmd\xi\bigg)^\frac{1}{3}\\
&= \frac{c}{2\pi}\bigg(2\int_0^\infty \frac{1}{ (1 + \xi)^\frac{3}{2} } \,\rmd\xi\bigg)^\frac{2}{3} \bigg(\int_{\!-\infty}^\infty
\frac{1}{\abs{ P_z(m, \xi) }^3} \,\rmd\xi\bigg)^\frac{1}{3}\\
&= \frac{c}{2\pi}4^\frac{2}{3}
\bigg(\int_{\!-\infty}^\infty
\frac{1}{ [ ( ( \omega m + \re{z} )^2 - \re{z}^2 )^2 + ( \xi + 2\omega m \im{z} )^2 ]^\frac{3}{2} } \,\rmd\xi\bigg)^\frac{1}{3}\\
&= \frac{c}{2\pi}4^\frac{2}{3} \bigg(\int_{\!-\infty}^\infty
\frac{1}{ [ ( ( \omega m + \re{z} )^2 - \re{z}^2 )^2 + \xi^2 ]^\frac{3}{2} } \,\rmd\xi\bigg)^\frac{1}{3}\\
&= \frac{c}{2\pi}4^\frac{2}{3} \abs{ ( \omega m + \re{z} )^2 - \re{z}^2 }^{ -\frac{2}{3} } \bigg(\int_{\!-\infty}^\infty
\frac{1}{ (1 + v^2)^\frac{3}{2} } \,\rmd v\bigg)^\frac{1}{3}\\
&= \frac{c}{2\pi}4^\frac{2}{3} 2^\frac{1}{3} \frac{1}{ \abs{ ( \omega m + \re{z} )^2 - \re{z}^2 }^\frac{2}{3} }.
\end{align*}}
Now
\[
\sideset{}{'}\sum_{m = -\infty}^\infty \frac{1}{ \abs{ ( \omega m + \re{z} )^2 - \re{z}^2 }^\frac{2}{3} } =
\sum_{m \in \mZ} \frac{1}{ \abs{m^2 - \re{z}^2}^\frac{2}{3} } = 2\sum_{m \in \mZ^+} \frac{1}{ \abs{m^2 - \re{z}^2}^\frac{2}{3} }.
\]
But inequality \eqref{Eq:basicinequality} yields
\[
\frac{1}{ \abs{m^2 - \re{z}^2}^\frac{2}{3} } < \frac{1}{ ( m - \abs{ \re{z} } )^\frac{4}{3} }.
\]
Using the same arguments as we did in the corresponding part of the proof of the \hyperlink{L:basiclemma}{Basic Lemma},
\[
\sum_{m \in \mZ^+} \frac{1}{ ( m - \abs{ \re{z} } )^\frac{4}{3} } < \frac{2}{ \omega^\frac{4}{3} }\sum_{m = 1}^\infty \frac{1}{ m^\frac{4}{3} },
\]
for every $z \in \wpln$. Therefore, the series
\[
\sideset{}{'}\sum_{m = -\infty}^\infty \mu_m(y; z)\rme^{\rmi\omega m x}
\]
converges uniformly in $\wpln$.

Let $m$ be a nonzero integer. Denote by $\mu^+$ and $\mu^-$ the non-tangential limits of $\mu$ from the right and from the left of the line
\begin{equation}
L_m \= \{ \zeta \in \bb{C} \colon \re{\zeta} = -\tfrac{\omega}{2}m, \ \im{\zeta} \in \bb{R} \}
\end{equation}
respectively. By the uniform convergence of the series in \eqref{Eq:Fourierseries}, in order to establish the existence of the limits $\mu^\pm$, it is enough
to show that these limits exists for the function $\mu_m(y; z)$ for all $y \in \bb{R}$. Write $\mu_m(y; z)$ in the form
\[
\mu_m(y; z) = \frac{1}{2\pi \rmi}\int_{\!-\infty}^\infty \frac{ \wh{u\mu}(m, \xi; z)\rme^{\rmi\xi y} }{ \xi - \rmi\omega m(\omega m + 2z) } \,\rmd\xi =
\frac{1}{2\pi \rmi}\int_{\!-\infty}^\infty \frac{ \wh{u\mu}(m, \xi; z)\rme^{\rmi\xi y} }{ \xi - p_0(z) } \,\rmd\xi,
\]
where $p_0(z) =  i\omega m(\omega m + 2z)$. This is a Cauchy type integral. Hence, to show the existence of the limit of $\mu_m(y; z)$ as $z$
approaches the line $L_m$ from the sides along any non-tangential path, or equivalently as $p_0$ approaches the real axes from the upper and from the 
lower half-planes, it suffices to show that for every $y \in \bb{R}$, the function $\wh{u\mu}(m, \xi; z)\rme^{\rmi\xi y}$ is H\"{o}lder continuous for all finite
$\xi$, tends to a definite limit $\wh{u\mu}(m, \infty; z)\rme^{\rmi\infty y}$ as $\abs{\xi} \to \infty$, and that for large $\xi$, the inequality
\begin{equation}\label{Eq:Holderinfinity}
\wh{u\mu}(m, \xi; z)\rme^{\rmi\xi y} - \wh{u\mu}(m, \infty; z)\rme^{\rmi\infty y} \leq \frac{M}{\abs{\xi}^\kappa},
\end{equation}
holds for some positive, real constants $M$ and $\kappa$.
For $\xi_1$, $\xi_2 \in \bb{R}$ we have the following:
\begin{align*}
&\abs{ \wh{u\mu}(m, \xi_1; z)\rme^{\rmi\xi_1 y} - \wh{u\mu}(m, \xi_2; z)\rme^{\rmi\xi_2 y} }\\
&\leq \frac{1}{2\ell}\int_{\!-\infty}^\infty \int_{\!-\ell}^\ell
\abs{ u(x, y')\mu(x, y'; z) }\abs{ \rme^{ \rmi\xi_1(y - y') } - \rme^{ \rmi\xi_2(y - y') } } \,\rmd x\rmd y'\\
&\leq \frac{1}{2\ell}\int_{\!-\infty}^\infty \int_{\!-\ell}^\ell \abs{ u(x, y') } \,\norm{\mu}_\infty \abs{y - y'}\abs{\xi_1 - \xi_2} \,\rmd x\rmd y'\\
&\leq \frac{1}{2\ell}\norm{\mu}_\infty \bigg(\abs{y}\norm{u}_1 +  \int_{-\infty}^\infty \int_{\!-\ell}^\ell
\abs{ u(x, y') }\abs{y'} \,\rmd x\rmd y'\bigg)\abs{\xi_1 - \xi_2},
\end{align*}
hence $\wh{u\mu}(m, \xi; z)\rme^{\rmi\xi y}$ is indeed H\"{o}lder continuous for all finite $\xi$. Furthermore, from \eqref{Eq:firstorderbound} there exists 
a real number $c > 0$ such that $\abs{ \wh{u\mu}(m, \xi; z) } \leq c/\abs{\xi}$. Hence, the limit $\wh{u\mu}(m, \infty; z)\rme^{\rmi\infty y}$ is definite
(in fact $\wh{u\mu}(m, \infty; z)\rme^{\rmi\infty y} = 0$) and the inequality \eqref{Eq:Holderinfinity} is satisfied (with $M = c$ and $\kappa = 1$).

Finally, fix $\lambda$ and suppose $z = \lambda + \rmi\im{z}$ is a complex number with $\lambda \notin \tfrac{\omega}{2}\sZ$. By symmetry, we can 
assume that $\lambda > 0$. Using once again the uniform convergence of the Fourier series \eqref{Eq:Fourierseries} on $\wpln$,
\[
\lim_{\abs{ \im{z} } \to \infty} (\mu( x, y; \lambda + \rmi\im{z} ) - 1) = \sideset{}{'}\sum_{m = -\infty}^\infty
\lim_{\abs{ \im{z} } \to \infty} \mu_m( y; \lambda + \rmi\im{z} )\rme^{\rmi\omega m x}.
\]
Split the sum as follows:
\begin{align*}
\sideset{}{'}\sum_{m = -\infty}^\infty \lim_{\abs{ \im{z} } \to \infty} \mu_m( y; \lambda + \rmi\im{z} )\rme^{\rmi\omega m x} &=
\sum_{ \substack{ m > 0 \\ \substack{\text{or} \\ \omega m + 2\lambda < 0} } } \lim_{\abs{ \im{z} } \to \infty}
\mu_m( y; \lambda + \rmi\im{z} )\rme^{\rmi\omega m x}\\
&{}+ \sum_{-2\lambda < \omega m < 0} \lim_{\abs{ \im{z} } \to \infty} \mu_m( y; \lambda + \rmi\im{z} )\rme^{\rmi\omega m x}.
\end{align*}
For $m > 0$ or $\omega m + 2\lambda < 0$, we have $(\omega m)^2 + 2\omega m\lambda > 0$. Hence,
\begin{align*}
\mu_m(y; z) &= \frac{1}{2\pi}\int_{\!-\infty}^\infty
\frac{ \wh{u\mu}(m, \xi; z) }{ (\omega m)^2 + 2\omega m\lambda + \rmi( \xi + 2\omega m\im{z} ) }\rme^{\rmi\xi y} \,\rmd\xi\\
&= \frac{1}{2\pi}\int_{\!-\infty}^\infty \wh{u\mu}(m, \xi; z)\rme^{\rmi\xi y}
\bigg(\int_{\!-\infty}^0 \rme^{ ( (\omega m)^2 + 2\omega m\lambda + \rmi( \xi + 2\omega m\im{z} ) )\tau } \,\rmd\tau\bigg)\rmd\xi\\
&= \frac{1}{2\pi}\int_{\!-\infty}^0 \rme^{ ( (\omega m)^2 + 2\omega m\lambda )\tau }
\bigg(\int_{\!-\infty}^\infty \wh{u\mu}(m, \xi; z)\rme^{ \rmi\xi(y + \tau) } \,\rmd\xi\bigg)\rme^{\rmi2\omega m\im{z}\tau} \,\rmd\tau\\
&= \int_{\!-\infty}^0 f_{m, y, \lambda}(\tau)\rme^{\rmi2\omega m\im{z}\tau} \,\rmd\tau,
\end{align*}
where
\begin{align*}
f_{m, y, \lambda}(\tau) &= \rme^{ ( (\omega m)^2 + 2\omega m\lambda )\tau }
\frac{1}{2\pi}\int_{\!-\infty}^\infty \wh{u\mu}(m, \xi; z)\rme^{ \rmi\xi(y + \tau) } \,\rmd\xi\\
&= \rme^{ ( (\omega m)^2 + 2\omega m\lambda )\tau }\frac{1}{2\ell}\int_{\!-\ell}^\ell u(x, y + \tau)\mu(x, y + \tau; z)\rme^{-\rmi\omega m x} \,\rmd x.
\end{align*}
But since $\rme^{ ( (\omega m)^2 + 2\omega m\lambda )\tau } < 1$ for $\tau < 0$,
\begin{align*}
\int_{\!-\infty}^0 \abs{ f_{m, y, \lambda}(\tau) } \,\rmd\tau &< \frac{1}{2\ell}\int_{\!-\infty}^0 \int_{\!-\ell}^\ell
\abs{ u(x, y + \tau) }\abs{ \mu(x, y + \tau; z) } \,\rmd x\rmd\tau\\
&\leq \frac{1}{2\ell}\norm{\mu}_\infty \int_{\!-\infty}^0 \int_{\!-\ell}^\ell \abs{ u(x, y + \tau) } \,\rmd x\rmd\tau\\
&\leq \frac{1}{2\ell}\norm{\mu}_\infty \norm{u}_1.
\end{align*}
Thus,
\[
\mu_m(y; z) = \int_{\!-\infty}^0 f_{m, y, \lambda}(\tau)\rme^{\rmi2\omega m\im{z}\tau} \,\rmd\tau,
\]
with $f_{m, y, \lambda}(\tau) \in L^1(-\infty, 0)$. Hence, from the Riemann--Lebesgue lemma,
\[
\lim_{\abs{ \im{z} } \to \infty} \mu_m(y; z) = 0.
\]

Now, for $-2\lambda < \omega m < 0$, $(\omega m)^2 + 2\omega m\lambda$ is negative, and so
\[
\mu_m(y; z) = \int_0^\infty f_{m, y, \lambda}(\tau)\rme^{\rmi2\omega m\im{z}\tau} \,\rmd\tau,
\]
where this time
\[
f_{m, y, \lambda}(\tau) = -\rme^{ ( (\omega m)^2 + 2\omega m\lambda )\tau } \frac{1}{2\ell}\int_{\!-\ell}^\ell
u(x, y + \tau)\mu(x, y + \tau; z)\rme^{-\rmi\omega m x} \,\rmd x,
\]
and $f_{m, y, \lambda}(\tau) \in L^1(0, \infty)$. Hence, $\lim\limits_{\abs{ \im{z} } \to \infty} \mu_m(y; z) = 0$ in this case as well.
Therefore, $\lim\limits_{\abs{ \im{z} } \to \infty} \mu_m( y; \lambda + i\im{z} ) = 0$ for every $m \in \sZ$ and consequently
\[
\lim_{\abs{ \im{z} } \to \infty} (\mu( x, y; \lambda + i\im{z} ) - 1) = 0,
\]
thus, concluding the proof of the theorem.
\end{proof}
Theorems \ref{Th:holomorphicity} and \ref{Th:zasympotics} show that $\mu(x, y; z)$ is a sectionally holomorphic function (with respect to $z$) in the strips
\begin{gather}
S_n \= \{ z \in \bb{C} \colon \tfrac{\omega}{2}n < \re{z} < \tfrac{\omega}{2}(n + 1), \ \im{z} \in \bb{R} \}, \label{Eq:stripsplus}\\
S_{-n} \= \{ z \in \bb{C} \colon -\tfrac{\omega}{2}(n + 1) < \re{z} < -\tfrac{\omega}{2}n, \ \im{z} \in \bb{R} \}, \label{Eq:stripsminus}\\
\intertext{for $n \in \bb{N}$, and}
S_0 \= \{ z \in \bb{C} \colon \abs{ \re{z} } < \tfrac{\omega}{2}, \ \im{z} \in \bb{R} \}. \label{Eq:zerostrip}
\end{gather}
\begin{figure}[H]
\centering
\begin{tikzpicture}
\draw[->](-5,0) -- (5,0) node[below]{$\real z$};
\draw[->](0,-2) -- (0,2) node[right]{$\imaginary z$};
\node[below] at (-0.2, 0){$0$};
\node[fill=white] at (0, 1){$S_0$};
\draw[dashed,red] (1,2)--(1,-2);
\node at (1.5, 1){$S_1$};
\node[below] at (0.8,0){$\frac{\omega}{2}$};
\draw[fill,red](1,0) circle[radius=1pt];
\draw[dashed,red] (2,2)--(2,-2);
\node at (2.5, 1){$S_2$};
\node[below] at (1.8,0) {$\omega$};
\draw[fill,red](2,0) circle[radius=1pt];
\draw[dashed,red] (3,2)--(3,-2);
\node[below] at (2.8,0) {$\frac{3\omega}{2}$};
\draw[fill,red](3,0) circle[radius=1pt];
\draw[dashed,red] (-1,2)--(-1,-2);
\node at (-1.5, 1){$S_{-1}$};
\node[below] at (-1.3,0) {$-\frac{\omega}{2}$};
\draw[fill,red](-1,0) circle[radius=1pt];
\draw[dashed,red] (-2,2)--(-2,-2);
\node at (-2.5, 1){$S_{-2}$};
\node[below] at (-2.3,0) {$-\omega$};
\draw[fill,red](-2,0) circle[radius=1pt];
\draw[dashed,red] (-3,2)--(-3,-2);
\node[below] at (-3.4,0) {$-\frac{3\omega}{2}$};
\draw[fill,red](-3,0) circle[radius=1pt];
\draw[dotted,very thick] (3.5,1)--(3.8,1);
\draw[dotted,very thick] (-3.8,1)--(-3.5,1);
\end{tikzpicture}
\caption{Strips of holomorphicity\label{fig:strips}}
\end{figure}
To calculate the jump of $\mu$ across the lines
\begin{equation}\label{Eq:jumpcontours}
L_n \= \{ z \in \bb{C} \colon \re{z} = -\tfrac{\omega}{2}n, \ \im{z} \in \bb{R} \}, \quad n \in \sZ,
\end{equation}
it is convenient to calculate the integrals in the equation \eqref{Eq:eigenfunction} for each integer $m \in \sZ$.
\begin{proposition}
Suppose $\maxnorm{u} < 2\pi/C$. Then, for $z \in \wpln$, the function $\mu(x, y; z)$ satisfies the integral equation
\begin{equation}\label{Eq:classicrepresentation}
\mu(x, y; z) = 
\begin{cases}
1 + [\op{g}_{ u, \mrm{r} } \mu](x, y; z), & \re{z} > 0\\[2pt]
1 + [\op{g}_{ u, \mrm{l} } \mu](x, y; z), & \re{z} < 0,
\end{cases}
\end{equation}
where
\begin{equation}\label{Eq:groperator}
\begin{split}
[\op{g}_{ u, \mrm{r} } h](x, y; z) &\= \frac{1}{2\ell}\Bigg(\sum_{ \substack{ m > 0 \\ \substack{ \tn{or} \\ \omega m < -2\re{z} } } }
\int_{\!-\infty}^y \int_{\!-\ell}^\ell - \sum_{-2\re{z} < \omega m < 0} \int_y^\infty \int_{\!-\ell}^\ell\Bigg)\\
&\quad\ph{1 + \frac{1}{2\ell}} u(x', y')h(x', y')\rme^{ \rmi\omega m(x - x') - \omega m(\omega m + 2z)(y - y') } \,\rmd x'\rmd y',
\end{split}
\end{equation}
and
\begin{equation}\label{Eq:gloperator}
\begin{split}
[\op{g}_{ u, \mrm{l} } h](x, y; z) &\= \frac{1}{2\ell}\Bigg(\sum_{ \substack{ m < 0 \\ \substack{ \tn{or} \\ \omega m > -2\re{z} } } }
\int_{\!-\infty}^y \int_{\!-\ell}^\ell - \sum_{ 0 < \omega m < -2\re{z} } \int_y^\infty \int_{\!-\ell}^\ell\Bigg)\\
&\quad\ph{1 + \frac{1}{2\ell}} u(x', y')h(x', y')\rme^{ \rmi\omega m(x - x') - \omega m(\omega m + 2z)(y - y') } \,\rmd x'\rmd y',
\end{split}
\end{equation}
for every function $h \in L^\infty(\Omega)$.
\end{proposition}
\begin{proof}
Fix a number $z$ in $\wpln$ and suppose $\re{z} > 0$. We have
\begin{align*}
\mu(x, y; z) - 1 &= \frac{1}{2\pi}\sideset{}{'}\sum_{m = -\infty}^\infty \int_{\!-\infty}^\infty \frac{1}{ P_z(m, \xi) }\\
\begin{split}
&\qquad \bigg(\frac{1}{2\ell}\int_{\!-\infty}^\infty \int_{\!-\ell}^\ell
u(x', y')\mu(x', y'; z)\rme^{ \rmi\omega m(x - x') + \rmi\xi(y - y') } \,\rmd x'\rmd y'\bigg)\rmd\xi
\end{split}\\ 
&= \frac{1}{2\ell}\sideset{}{'}\sum_{m = -\infty}^\infty \frac{1}{2\pi i}\int_{\!-\infty}^\infty \frac{1}{ \xi - \rmi\omega m(\omega m + 2z) }\\
\begin{split}
&\qquad \bigg(\int_{\!-\infty}^y \int_{\!-\ell}^\ell u(x', y')\mu(x', y'; z)\rme^{ \rmi\omega m(x - x') + \rmi\xi(y - y') } \,\rmd x'\rmd y'\bigg)\rmd\xi
\end{split}\\
&\qquad + \frac{1}{2\ell}\sideset{}{'}\sum_{m = -\infty}^\infty \frac{1}{2\pi i}\int_{\!-\infty}^\infty \frac{1}{\xi - \rmi\omega m(\omega m + 2z) }\\
\begin{split}
&\qquad \bigg(\int_y^\infty \int_{\!-\ell}^\ell u(x', y')\mu(x', y'; z)\rme^{ \rmi\omega m(x - x') + \rmi\xi(y - y') } \,\rmd x'\rmd y'\bigg)\rmd\xi.
\end{split}
\end{align*}

Let $p_0 = \rmi\omega m(\omega m + 2z)$ and $s = \xi + \rmi\tau$ be a complex number in the upper-half plane. The integral
\[
\int_{\!-\infty}^y \int_{\!-\ell}^\ell u(x', y')\mu(x', y'; z)\rme^{ \rmi\omega m(x - x') + \rmi s(y - y') } \,\rmd x'\rmd y'
\]
converges absolutely since
\begin{align*}
\int_{\!-\infty}^y \int_{\!-\ell}^\ell \abs{ u(x', y') }\abs{ \mu(x', y'; z) }\rme^{ -\tau(y - y') } \,\rmd x'\rmd y' &<
\norm{\mu}_\infty \int_{\!-\infty}^y \int_{\!-\ell}^\ell \abs{ u(x', y') } \,\rmd x'\rmd y'\\
&\leq \norm{\mu}_\infty \norm{u}_1.
\end{align*}
Thus, it defines a holomorphic function with respect to $s$; apply Fubini's and Morera's theorems. Hence, the function
\begin{align*}
f(s; m, x, y) &= \frac{1}{s - p_0}g(s; m, x, y)\\ 
&\equiv \frac{1}{s - p_0}\int_{\!-\infty}^y \int_{\!-\ell}^\ell u(x', y')\mu(x', y'; z)\rme^{ \rmi\omega m(x - x') + \rmi s(y - y') } \,\rmd x'\rmd y'
\end{align*}
is holomorphic, if $\imaginary p_0 < 0$, and meromorphic with a simple pole at the point $p_0$, if $\imaginary p_0 > 0$. Therefore, by the residue theorem
\[
\frac{1}{2\pi i}\int_\gamma f(s; m, x, y) \,ds = 
\begin{cases}
0, & \imaginary p_0 < 0\\
\res(f, s = p_0), & \imaginary p_0 > 0,
\end{cases}
\]
where the curve $\gamma = [-R, R] + C_R$, and $C_R$ is the semi-circle in the upper-half plane, centred at the origin with radius $R$, such that
$R > \abs{p_0}$. Now,
\begin{align*}
\Abs{\int_{C_R} f(s; m, x, y) \,\rmd s} &\leq \frac{\pi R}{ R - \abs{p_0} }\max_{s \in C_R} \abs{ g(s; m, x, y) }\\
&= \frac{\pi R}{ R - \abs{p_0} }\max_{ \theta \in [0, \pi] } \abs{ g(R\rme^{\rmi\theta}; m, x, y) }.
\end{align*}
But
\[
\abs{ g(R\rme^{\rmi\theta}; m, x, y) } \leq \norm{\mu}_\infty \int_{\!-\infty}^y \int_{\!-\ell}^\ell
\abs{ u(x', y') }\rme^{ -R\sin\theta(y - y') } \,\rmd x'\rmd y',
\]
and since $\sin\theta > 0$, absolute and dominated convergence implies
\[
\lim_{R \to \infty} \int_{\!-\infty}^y \int_{\!-\ell}^\ell \abs{ u(x', y') }\rme^{ -R\sin\theta(y - y') } \,\rmd x'\rmd y' = 0,
\]
hence $\max_{ \theta \in [0, \pi] } \abs{ g(R\rme^{\rmi\theta}; m, x, y) } \to 0$ as $R$ tends to $\infty$. Thus, writing
\[
\frac{1}{2\pi \rmi}\int_\gamma f(s; m, x, y) \,\rmd s = \frac{1}{2\pi \rmi}\int_{\!-R}^R f(\xi; m, x, y) \,\rmd\xi +
\frac{1}{2\pi \rmi}\int_{C_R} f(s; m, x, y) \,\rmd s
\]
and taking the limit as $R \to \infty$, yields
\[
\frac{1}{2\pi \rmi}\int_{\!-\infty}^\infty f(\xi; m, x, y) \,\rmd\xi = 
\begin{cases}
0, & \imaginary p_0 < 0 \\
g(p_0; m, x, y), & \imaginary p_0 > 0.
\end{cases}
\]
Similarly, choosing $\tilde{\gamma} = \wt{C}_R + [-R, R]$ with positive orientation, where $\wt{C}_R$ is the semi-circle, situated in the lower-half plane,
centred at the origin with radius $R$ such that $R > \abs{p_0}$, we find
\[
\frac{1}{2\pi i}\int_{\!-\infty}^\infty \tilde{f}(\xi; m, x, y) \,\rmd\xi = 
\begin{cases}
0, & \imaginary p_0 > 0 \\
-\tilde{g}(p_0; m, x, y), & \imaginary p_0 < 0,
\end{cases}
\]
where
\begin{align*}
\tilde{f}(s; m, x, y) &= \frac{1}{s - p_0}\tilde{g}(s; m, x, y) \\ 
&\equiv \frac{1}{s - p_0}\int_y^\infty \int_{\!-\ell}^\ell u(x', y')\mu(x', y'; z)\rme^{ \rmi\omega m(x - x') + \rmi s(y - y') } \,\rmd x'\rmd y'.
\end{align*}
Now, $\imaginary p_0 = \imaginary( \rmi\omega m(\omega m + 2z) ) = \real( \omega m(\omega m + 2z) )$. Thus,
\[
\imaginary p_0 > 0 \Equiv m > 0 \text{ or } m < -\tfrac{ 2\re{z} }{\omega},
\]
and
\[
\imaginary p_0 < 0 \Equiv -\tfrac{ 2\re{z} }{\omega} < m < 0.
\]
Putting all this together,
\begin{multline*}
\mu(x, y; z) = 1 + \frac{1}{2\ell}\Bigg(\sum_{ \substack{ m > 0 \\ \substack{\text{or} \\ m < -2\re{z}/\omega} } }
\int_{\!-\infty}^y \int_{\!-\ell}^\ell - \sum_{-2\re{z}/\omega < m < 0} \int_y^\infty \int_{\!-\ell}^\ell\Bigg)\\
u(x', y')\mu(x', y'; z)\rme^{ \rmi\omega m(x - x') - \omega m(\omega m + 2z)(y - y') } \,\rmd x'\rmd y'.
\end{multline*}

In the case where $\re{z} < 0$, we have
\[
\imaginary p_0 > 0 \Equiv m < 0 \text{ or } m > -\tfrac{ 2\re{z} }{\omega},
\]
and
\[
\imaginary p_0 < 0 \Equiv 0 < m < -\tfrac{ 2\re{z} }{\omega}.
\]
Thus, repeating the previous arguments,
\begin{equation*}
\begin{split}
\mu(x, y; z) &= 1 + \frac{1}{2\ell}\Bigg(\sum_{ \substack{ m < 0 \\ \substack{\text{or} \\ m > -2\re{z}/\omega} } }
\int_{\!-\infty}^y \int_{\!-\ell}^\ell - \sum_{0 < m < -2\re{z}/\omega} \int_y^\infty \int_{\!-\ell}^\ell\Bigg)\\
&\qquad\ph{1 + \frac{1}{2\ell}(} u(x', y')\mu(x', y'; z)\rme^{ \rmi\omega m(x - x') - \omega m(\omega m + 2z)(y - y') } \,\rmd x'\rmd y'.\qedhere
\end{split}
\end{equation*}
\end{proof}

Since $\mu = 1 + \Scr{N}_u\mu$, equation \eqref{Eq:classicrepresentation} implies that, for all $z \in \wpln$ and $h \in L^\infty(\Omega)$,
\[
(\Scr{N}_u h)(x, y; z) = 
\begin{cases}
[\op{g}_{ u, \mrm{r} } h](x, y; z), & \re{z} > 0\\[2pt]
[\op{g}_{ u, \mrm{l} } h](x, y; z), & \re{z} < 0.
\end{cases}
\]
Furthermore, condition \eqref{Eq:zeromass} aloows us to rewrite equations \eqref{Eq:groperator} and \eqref{Eq:gloperator} as
\begin{equation}
\begin{split}
[\op{g}_{ u, \mrm{r} } h](x, y; z) &\= \frac{1}{2\ell}\Bigg(\sum_{ \substack{ m > 0 \\ \substack{ \tn{or} \\ \omega m < -2\re{z} } } }
\int_{\!-\infty}^y \int_{\!-\ell}^\ell - \sum_{-2\re{z} < \omega m \leq 0} \int_y^\infty \int_{\!-\ell}^\ell\Bigg)\\
&\quad\ph{1 + \frac{1}{2\ell}} u(x', y')h(x', y')\rme^{ \rmi\omega m(x - x') - \omega m(\omega m + 2z)(y - y') } \,\rmd x'\rmd y',
\end{split}
\end{equation}
and
\begin{equation}
\begin{split}
[\op{g}_{ u, \mrm{l} } h](x, y; z) &\= \frac{1}{2\ell}\Bigg( \sum_{ \substack{ m \leq 0 \\ \substack{ \tn{or} \\ \omega m > -2\re{z} } } }
\int_{\!-\infty}^y \int_{\!-\ell}^\ell - \sum_{ 0 < \omega m < -2\re{z} } \int_y^\infty \int_{\!-\ell}^\ell \Bigg)\\
&\quad\ph{1 + \frac{1}{2\ell}}u(x', y')h(x', y')\rme^{ \rmi\omega m(x - x') - \omega m(\omega m + 2z)(y - y') } \,\rmd x'\rmd y'.
\end{split}
\end{equation}
We are now ready to calculate $\op{J}\mu$ across the contours $L_n$ and derive the spectral data.
\begin{theorem}\label{Th:departurefromholomorphicity}
Suppose the potential $u(x, y)$ is small and regular, such that $\maxnorm{u} < 2\pi/C, \,y u(x, y) \in L^1(\Omega)$ and $u_x$, $u_y \in L^1(\Omega) \cap L^2(\Omega)$. Then,
\begin{equation}\label{Eq:RiemannHilbert}
\op{J}\mu(x, y; z) \equiv \mu^+(x, y; z) - \mu^-(x, y; z) = F(z)\rme^{-\rmi( z + \bar{z} )x + (z^2 - \bar{z}^2)y} \mu^-( x, y; -\bar{z} ),
\end{equation}
for $z$ on the contour $L_n$, $n \in \sZ$, where
\begin{equation}\label{Eq:scatteringdata}
F(z) \= -\frac{ \sgn( \re{z} ) }{2\ell}\int_{\!-\infty}^\infty \int_{\!-\ell}^\ell
u(x, y)\mu^+(x, y; z)\rme^{ \rmi( z + \bar{z} )x - (z^2 - \bar{z}^2)y } \,\rmd x\rmd y,
\end{equation}
defines the \emph{spectral data}.
\end{theorem}
\begin{proof}
Fix a positive integer $n$ and let $z = \tfrac{\omega}{2}n + \rmi\im{z}$ (for $n$ negative integer the analysis is similar and thus, omitted). Then,
{\allowdisplaybreaks
\begin{align}\label{Eq:rightlimit}
\mu^+(x, y; z) &= 1 +  [\op{g}_{ u, \mrm{r} } \mu]^+(x, y; z)\nonumber\\ 
&= 1 + \frac{1}{2\ell}\Bigg(\sum_{ \substack{ m > 0 \\ \substack{\tn{or} \\ \omega m < -\omega n} } } \int_{\!-\infty}^y \int_{\!-\ell}^\ell -
\sum_{-\omega n \leq \omega m \leq 0} \int_y^\infty \int_{\!-\ell}^\ell\Bigg)\nonumber\\
&\qquad\ph{1 + \frac{1}{2\ell}} u(x', y')\mu^+(x', y'; z)\rme^{ \rmi\omega m(x - x') - ( (\omega m + z)^2 - z^2 )(y - y') } \,\rmd x'\rmd y'\nonumber\\
&= 1 + \frac{1}{2\ell}\Bigg(\sum_{ \substack{ k > z \\ \substack{ \tn{or} \\ k < -\bar{z} } } } \int_{\!-\infty}^y \int_{\!-\ell}^\ell -
\sum_{-\bar{z} \leq k \leq z} \int_y^\infty \int_{\!-\ell}^\ell\Bigg)\nonumber\\
&\qquad\ph{1 + \frac{1}{2\ell}} u(x', y')\mu^+(x', y'; z)\rme^{ \rmi(k - z)(x - x') - (k^{2} - z^2)(y - y') } \,\rmd x'\rmd y'\nonumber\\
&= 1 + \frac{1}{2\ell}\Bigg(\sum_{k \in U_{ z, \mrm{r} }^+}
\int_{\!-\infty}^y \int_{\!-\ell}^\ell - \sum_{k \in V_{ z, \mrm{r} }^+} \int_y^\infty \int_{\!-\ell}^\ell\Bigg)\nonumber\\
&\qquad\ph{1 + \frac{1}{2\ell}} u(x', y')\mu^+(x', y'; z)\rme^{ \rmi(k - z)(x - x') - (k^{2} - z^2)(y - y') } \,\rmd x'\rmd y'\nonumber\\
&\equiv 1 +  [\op{g}_{ u, \mrm{r} }^+ \mu^+](x, y; z),
\end{align}}
where
\begin{subequations}\label{Eq:+sets}\begin{align}
U_{ z, \mrm{r} }^+ &= (\omega\bb{N} + z) \cup ( -\omega\bb{N} - \bar{z} ) \label{Eq:U+},\\
V_{ z, \mrm{r} }^+ &= \{-\bar{z}, -\bar{z} + \omega, \dotsc, z - \omega, z\} \label{Eq:V+},
\end{align}\end{subequations}
and
\begin{align}\label{Eq:leftlimit}
\mu_-(x, y; z) &= 1 + [\op{g}_{ u, \mrm{r} } \mu]^-(x, y; z)\nonumber\\
&= 1 + \frac{1}{2\ell}\Bigg(\sum_{ \substack{ m > 0 \\ \substack{\tn{or} \\ \omega m \leq -\omega n} } } \int_{\!-\infty}^y \int_{\!-\ell}^\ell -
\sum_{-\omega n < \omega m \leq 0} \int_y^\infty \int_{\!-\ell}^\ell\Bigg)\nonumber\\
&\qquad\ph{1 + \frac{1}{2\ell}} u(x', y')\mu^-(x', y'; z)\rme^{ \rmi\omega m(x - x') - ( (\omega m + z)^2 - z^2 )(y - y') } \,\rmd x'\rmd y'\nonumber\\
&= 1 + \frac{1}{2\ell}\Bigg(\sum_{ \substack{ k > z \\ \substack{ \tn{or} \\ k \leq -\bar{z} } } } \int_{\!-\infty}^y \int_{\!-\ell}^\ell -
\sum_{-\bar{z} < k \leq z} \int_y^\infty \int_{\!-\ell}^\ell\Bigg)\nonumber\\
&\qquad\ph{1 + \frac{1}{2\ell}} u(x', y')\mu^-(x', y'; z)\rme^{ \rmi(k - z)(x - x') - (k^{2} - z^2)(y - y') } \,\rmd x'\rmd y'\nonumber\\
&= 1 + \frac{1}{2\ell}\Bigg(\sum_{k \in U_{ z, \mrm{r} }^-}
\int_{\!-\infty}^y \int_{\!-\ell}^\ell - \sum_{k \in V_{ z, \mrm{r} }^-} \int_y^\infty \int_{\!-\ell}^\ell\Bigg)\nonumber\\
&\qquad\ph{1 + \frac{1}{2\ell}} u(x', y')\mu^-(x', y'; z)\rme^{ \rmi(k - z)(x - x') - (k^{2} - z^2)(y - y') } \,\rmd x'\rmd y'\nonumber\\
&\equiv 1 +  [\op{g}_{ u, \mrm{r} }^- \mu^-](x, y; z),
\end{align}
with
\begin{subequations}\label{Eq:-sets}\begin{align}
U_{ z, \mrm{r} }^- &= (\omega\bb{N} + z) \cup ( -\omega\bb{N}_0 - \bar{z} ), \label{Eq:U-}\\
V_{ z, \mrm{r} }^- &= \{-\bar{z} + \omega, \dotsc, z - \omega, z\} \label{Eq:V-}.
\end{align}\end{subequations}
Thus,
\begin{align*}
\op{J}\mu &= \op{g}_{ u, \mrm{r} }^+ \mu^+ - \op{g}_{ u, \mrm{r} }^- \mu^-\\
&= \op{g}_{ u, \mrm{r} }^+ \mu^+ - \op{g}_{ u, \mrm{r} }^- \mu^+ + \op{g}_{ u, \mrm{r} }^- \mu^+ - \op{g}_{ u, \mrm{r} }^- \mu^-\\
&= (\op{g}_{ u, \mrm{r} }^+ - \op{g}_{ u, \mrm{r} }^-)\mu^+ + \op{g}_{ u, \mrm{r} }^- \op{J}\mu.
\end{align*}
But from equations \eqref{Eq:rightlimit}--\eqref{Eq:-sets},
\[
\begin{split}
(\op{g}_{ u, \mrm{r} }^+ - \op{g}_{ u, \mrm{r} }^-)\mu^+ = -\frac{1}{2\ell}\int_{\!-\infty}^\infty \int_{\!-\ell}^\ell &u(x', y')\mu^+(x', y'; z)\\
&\, \rme^{ \rmi(-\bar{z} - z)(x - x') - (\bar{z}^2 - z^2)(y - y') } \,\rmd x'\rmd y'.
\end{split}
\]
Recognizing $F(z)$ from its definition and setting
\[
d(x, y; z) = -\rmi( z + \bar{z} )x + (z^2 - \bar{z}^2)y,
\]
we obtain
\[
\op{J}\mu(x, y; z) = F(z)\rme^{ d(x, y; z) } + \op{g}_{ u, \mrm{r} }^- \op{J}\mu(x, y; z),
\]
or equivalently,
\[
(\op{Id} - \op{g}_{ u, \mrm{r} }^-)\op{J}\mu(x, y; z) = F(z)\rme^{ d(x, y; z) }.
\]
Now, $\op{g}_{ u, \mrm{r} }^-$ is the limit of the operator $\op{g}_{ u, \mrm{r} }$ as $\zeta$ approaches $z$ from the left of the line $L_n$ where
$\zeta$ is situated in the strip $S_{n - 1}$. But $\op{g}_{ u, \mrm{r} } = \Scr{N}_u$. Since $\Scr{N}_u$ is bounded with norm less than one, so is its limit. 
Thus, $\op{g}_{ u, \mrm{r} }^-$ is bounded and has norm less than one. Hence,
\[
\op{J}\mu(x, y; z) = (\op{Id} - \op{g}_{ u, \mrm{r} }^-)^{-1} F(z)\rme^{ d(x, y; z) } = F(z)(\op{Id} - \op{g}_{ u, \mrm{r} }^-)^{-1} \rme^{ d(x, y; z) },
\]
since $\op{g}_{ u, \mrm{r} }^-$ commutes with multiplication by functions of $z$ alone. It remains to calculate
$(\op{Id} - \op{g}_{ u, \mrm{r} }^-)^{-1} \rme^{ d(x, y; z) }$. Call this function $\nu(x, y; z)$. It is bounded because the exponential is bounded and
$\op{Id} - \op{g}_{ u, \mrm{r} }^-$ is invertible on $L^\infty(\Omega)$. It satisfies the equation
\[
\nu(x, y; z) = \rme^{ d(x, y; z) } + \op{g}_{ u, \mrm{r} }^- \nu(x, y; z).
\]
But then, $\wt{\nu}(x, y; z) = \nu(x, y; z)\rme^{ -d(x, y; z) }$ satisfies
\begin{align*}
\wt{\nu}(x, y; z) &= 1 + \rme^{ -d(x, y; z) } \op{g}_{ u, \mrm{r} }^- \nu(x, y; z)\\
&= 1 + \frac{1}{2\ell}\Bigg(\sum_{k \in U_{ z, \mrm{r} }^-}
\int_{\!-\infty}^y \int_{\!-\ell}^\ell - \sum_{k \in V_{ z, \mrm{r} }^-} \int_y^\infty \int_{\!-\ell}^\ell\Bigg)\\
&\qquad\ph{1 + \frac{1}{2\ell}} u(x', y')\wt{\nu}(x', y'; z)\rme^{ \rmi( k + \bar{z} )(x - x') - (k^{2} - \bar{z}^2)(y - y') } \,\rmd x'\rmd y'.
\end{align*}
This equation has a unique solution, which has already been named $\mu^-( x, y; -\bar{z} )$. Indeed, the number $-\bar{z}$ is located on the line
$L_{-n}$ since
\[
\real( -\bar{z} ) = \real(-z) = -\re{z} = -\frac{\omega}{2}n.
\]
Therefore,
\begin{align}
\mu_-( x, y; -\bar{z} ) &= 1 + [\op{g}_{ u, \mrm{l} } \mu]^-( x, y; -\bar{z} )\nonumber\\
&= 1 + \frac{1}{2\ell}\Bigg(\sum_{ \substack{ m \leq 0 \\ \substack{\tn{or} \\ \omega m > \omega n} } } \int_{\!-\infty}^y \int_{\!-\ell}^\ell -
\sum_{0 < \omega m \leq \omega n} \int_y^\infty \int_{\!-\ell}^\ell\Bigg)\nonumber\\
&\quad\ph{1 + \frac{1}{2\ell}} u(x', y')\mu^-( x', y'; -\bar{z} )
\rme^{ \rmi\omega m(x - x') - ( ( \omega m  -\bar{z} )^2 - \bar{z}^2 )(y - y') } \,\rmd x'\rmd y'\nonumber\\
&= 1 + \frac{1}{2\ell}\Bigg(\sum_{ \substack{ k \leq -\bar{z} \\ \substack{\tn{or} \\ k > z} } } \int_{\!-\infty}^y \int_{\!-\ell}^\ell -
\sum_{-\bar{z} < k \leq z} \int_y^\infty \int_{\!-\ell}^\ell\Bigg)\nonumber\\
&\quad\ph{1 + \frac{1}{2\ell}} u(x', y')\mu^-( x', y'; -\bar{z} )\rme^{ \rmi( k + \bar{z} )(x - x') - (k^{2} - \bar{z}^2)(y - y') } \,\rmd x'\rmd y'\nonumber\\
&= 1 + \frac{1}{2\ell}\Bigg(\sum_{k \in U_{ z, \mrm{r} }^-}
\int_{\!-\infty}^y \int_{\!-\ell}^\ell - \sum_{k \in V_{ z, \mrm{r} }^-} \int_y^\infty \int_{\!-\ell}^\ell\Bigg)\nonumber\\
&\quad\ph{1 + \frac{1}{2\ell}} u(x', y')\mu^-( x, y; -\bar{z} )\rme^{ \rmi( k + \bar{z} )(x - x') - (k^{2} - \bar{z}^2)(y - y') } \,\rmd x'\rmd y'.
\end{align}
Thus, $\wt{\nu}(x, y; z) = \mu^-( x, y; -\bar{z} )$ and so $\nu(x, y; z) = \mu^-( x, y; -\bar{z} )\rme^{ d(x, y; z) }$, which yields
\[
\op{J}\mu(x, y; z) = F(z)\mu^-( x, y; -\bar{z} )\rme^{ d(x, y; z) }.\qedhere
\]
\end{proof}
One easy consequence of the definition of the spectral data is the following estimate on $F(z)$.
\begin{proposition}
The function $F(z)$ is bounded. More precisely,
\begin{equation}
\abs{ F(z) } < \frac{\omega\norm{u}_1}{ 1 - \tfrac{C}{2\pi}\maxnorm{u} }, \quad \forall z \in \bb{C} \setminus \wpln.
\end{equation}
\end{proposition}
\begin{proof}
Let $z \in \bb{C} \setminus \wpln$. Then,
\begin{align*}
F(z) &= -\frac{ \sgn( \re{z} ) }{2\ell}\int_{\!-\infty}^\infty \int_{\!-\ell}^\ell
u(x, y)\mu^+(x, y; z)\rme^{ \rmi( z + \bar{z} )x - (z^2 - \bar{z}^2)y } \,\rmd x\rmd y\\
&= -\frac{ \sgn( \re{z} ) }{2\ell}\int_{\!-\infty}^\infty \int_{\!-\ell}^\ell u(x, y)\mu^+(x, y; z)\rme^{\rmi2\re{z}x - \rmi4\re{z}\im{z}y} \,\rmd x\rmd y \\
&= -\sgn( \re{z} )\wh{u\mu^+}(-\tfrac{ 2\re{z} }{\omega}, 4\re{z}\im{z}; z)\\
&= -\sgn( \re{z} )\wh{u\mu^+}(r_0(z); z).
\end{align*}
Consequently,
\[
\abs{ F(z) } = \abs{ \wh{u\mu^+}(r_0(z); z) } \leq \norm{ \wh{u\mu^+} }_\infty < \omega\norm{u\mu^+}_1 \leq
\omega\norm{u}_1 \norm{\mu^+}_\infty \leq \omega\norm{u}_1 \norm{\mu}_\infty.
\]
But $\mu = (\op{Id} - \Scr{N}_u)^{-1} 1$ and $\norm{\Scr{N}_u 1}_\infty \leq \tfrac{C}{2\pi}\maxnorm{u}$. Thus,
$\norm{\mu}_\infty \leq ( 1 - \tfrac{C}{2\pi}\maxnorm{u} )^{-1}$ and the proposition is proved.
\end{proof}
\begin{definition}
The \emph{spectral data} associated to a small potential $u(x, y)$ of the perturbed heat operator is the function defined by
\begin{equation}
F(z) = -\sgn( \re{z} )\wh{u\mu^+}(r_0(z); z), \quad z \in \bb{C} \setminus \wpln.
\end{equation}
Abusing notation, the bounded linear map $\Scr{S}$ determined by $F(z)$ shall also be called spectral data. Here
\begin{equation}\label{Eq:scatteringoperator}
(\Scr{S}\mu)(x, y; z) = F(z)\rme^{ \rmi r_0(z) \cdot (\omega x, y) } \mu^-( x, y; -\bar{z} ).
\end{equation}
\end{definition}
The function $F(z)$ behaves much like the Fourier transform of $u$. If $u$ is smooth, then $F$ has rapid decay in some directions. In particular, if $u$
is small, $y u \in L^1(\Omega)$ and
\[
\abs{\partial^{\alpha'} u} \in L^1(\Omega) \cap L^2(\Omega), \;\forall \:\abs{\alpha'} \leq \abs{\alpha},
\]
for some multi-index $\alpha = (\alpha_1, \alpha_2)$, then
\begin{equation}\label{Eq:datadecay}
\abs{ F(\tfrac{\omega}{2}n, \tau) } = O\bigg( \frac{1}{ 1 + \abs{\omega n}^{\alpha_1} + \abs{2\omega n\tau}^{\alpha_2} } \bigg),
\end{equation}
for $n \in \sZ$ and $\tau \in \bb{R}$. This follows immediately from the definition of $F$ and lemma~\ref{L:rapiddeacy}.
\begin{corollary}\label{C:squareintegrability}
Suppose that $\maxnorm{u}$ is small, $y u \in L^1(\Omega)$ and that $u(x, y)$ has two continuous derivatives in $L^1(\Omega) \cap L^2(\Omega)$. 
Then,
\begin{equation}\label{Eq:infinitybound}
\sup_{n \in \sZ} \abs{n}\sup_{ \tau \in \bb{R} } \abs{ F(\tfrac{\omega}{2}n, \tau) }< \infty,
\end{equation}
and
\begin{equation}\label{Eq:infinitytwobound}
\int_{\!-\infty}^\infty \abs{ F(\tfrac{\omega}{2}n, \tau) }^2 \,\rmd\tau = O\bigg( \frac{1}{n^4} \bigg),
\end{equation}
for all $n \in \sZ$.
\end{corollary}
\begin{proof}
Let $n \in \sZ$. From equation \eqref{Eq:datadecay},
\[
\abs{ F(\tfrac{\omega}{2}n, \tau) } \leq \frac{c}{1 + \abs{\omega n}^2 + \abs{2\omega n\tau}^2} < \frac{c}{\omega^2 n^2},
\]
for some positive, real constant $c$. Hence, $\abs{n}\sup_{ \tau \in \bb{R} } \abs{ F(\tfrac{\omega}{2}n, \tau) } < c \,\omega^{-2} \abs{n}^{-1}$
which yields
$\sup_{n \in \sZ} \abs{n}\sup_{ \tau \in \bb{R} } \abs{ F(\tfrac{\omega}{2}n, \tau) } < c \,\omega^{-2}$. Furthermore,
\[
\abs{ F(\tfrac{\omega}{2}n, \tau) } < \frac{c}{\abs{\omega n}^2 + \abs{2\omega n\tau}^2} = \frac{c}{ (\omega n)^2 (1 + (2\tau)^2) },
\]
thus,
\begin{align*}
\int_{\!-\infty}^\infty \abs{ F(\tfrac{\omega}{2}n, \tau) }^2 \,\rmd\tau &< \int_{\!-\infty}^\infty \frac{c^2}{ (\omega n)^4 (1 + (2\tau)^2)^2 } \,\rmd\tau\\
&= \frac{1}{n^4}\frac{c^2}{2\omega^4}\int_{\!-\infty}^\infty \frac{1}{ (1 + v^2)^2 } \,\rmd v < \infty.\qedhere
\end{align*}
\end{proof}
Suppose that $u(x, y)$ is a small potential in the sense of the preceding corollary. Then, the heat operator perturbed by $u$ has the associated spectral 
operator $\Scr{S}$, and its unique solution $\mu$ satisfies $\mu^+ - \mu^- = \Scr{S}\mu$. This is a Riemann--Hilbert problem on the infinite contours 
$L_n$, $n \in \sZ$. Since $\mu$ has the asymptotic behaviour $\mu \to 1$ as $\abs{ \im{z} } \to \infty$, the general solution to this problem is given by the 
Fredholm integral equation
\begin{equation}\label{Eq:Cauchyintegral}
\mu(x, y; z) = 1 + \frac{1}{2\pi \rmi}\sideset{}{'}\sum_{n = -\infty}^\infty \int_{L_n} \frac{ (\Scr{S}\mu)(x, y; \zeta) }{\zeta - z} \,\rmd\zeta.
\end{equation}
This equation is understood as the limit of the solution of the same Riemann--Hilbert problem on the contour $L = \sum_{\abs{n} \leq k} L_n$, when
$k \to \infty$ (see \cite{G66}). It can be shown that equation \eqref{Eq:Cauchyintegral} has a unique solution. Let $L^2( \abs{\!\real z} )$ denote the space 
of all measurable functions $f(z)$ on the jump contours such that
\begin{equation}
\norm{f}_{ L^2( \abs{\!\real z} ) }^2 \= \sideset{}{'}\sum_{n = -\infty}^\infty \int_{L_n} \,\abs{ f(z) }^2 \,\abs{\!\real z} \,\rmd z < \infty,
\end{equation}
and $\Lambda = \bigcup\limits_{n \in \sZ} L_n$. We have the following important result.
\begin{proposition}\label{Pr:important}
Suppose $\Scr{S} \colon L^\infty(E) \to L^2(\Scr{C}^*) \cap L^\infty(\Scr{C}^*)$ with
\[
(\Scr{S}f)(x, y; z) = F(z)\rme^{ \rmi r_0(z) \cdot (\omega x, y) } f^-( x, y; -\bar{z} ).
\]
If the function $F(z)$ is sufficiently small in $L^2( \abs{\!\real z} ) \cap L^\infty(\Lambda)$, then $\mathcal{C}\Scr{S}$ is a contraction of $L^\infty(E)$. 
Especially equation \eqref{Eq:Cauchyintegral} has a unique solution.
\end{proposition}
\begin{proof}Let $\mathcal{C}\Scr{S}\mu(x, y; z)$ denote the second term in the right hand side of the equation~\eqref{Eq:Cauchyintegral} where
$\mathcal{C}$ is the operator
\begin{equation}
\mathcal{C}f(z) = \frac{1}{2\pi \rmi}\sideset{}{'}\sum_{n = -\infty}^\infty \int_{L_n} \frac{ f(\zeta) }{\zeta - z} \,\rmd\zeta,
\end{equation}
which is defined at least for Schwartz functions on $\Scr{C}^*$. Choosing the param\-etrization
$\zeta = -\tfrac{\omega}{2}n - \rmi\tfrac{\tau}{2\omega n} \equiv \zeta(n, \tau)$ for the contour $L_n$, equation \eqref{Eq:Cauchyintegral} is
transformed into
\begin{equation}
\mu(x, y; z) = 1 + \frac{1}{2\pi}\sideset{}{'}\sum_{n = -\infty}^\infty \int_{\!-\infty}^\infty
\frac{ F \circ \zeta(n, \tau)\mu^-( x, y; -\bar{\zeta}(n, \tau) ) }{ P_z(n, \tau) }\rme^{\rmi\omega n x + \rmi\tau y} \,\rmd\tau.
\end{equation}
Taking absolute values, immediately shows
\[
\abs{ \mathcal{C}\Scr{S}\mu(x, y; z) } \leq \norm{\mu}_\infty \frac{1}{2\pi}\sideset{}{'}\sum_{n = -\infty}^\infty \int_{\!-\infty}^\infty
\Abs{ \frac{ F \circ \zeta(n, \tau) }{ P_z(n, \tau) } } \,\rmd\tau.
\]
Now, observe that the function $z \mapsto r_0(z)$ (on $\bb{C} \setminus \wpln$), defined in equation \eqref{Eq:nontrivialzero}, and
$(n, \tau) \mapsto \zeta(n, \tau)$ are inverses of each other. Thus,
\[
\abs{ F \circ \zeta(n, \tau) } = \abs{ \wh{u\mu_+}( n, \tau; \zeta(n, \tau) ) } < \omega\norm{u}_1 \norm{\mu}_\infty,
\]
which shows that $F \circ \zeta \in L^\infty(\Scr{C}^*)$. Also, from \eqref{Eq:datadecay}
\[
\abs{ F \circ \zeta(n, \tau) }^2 = \abs{ F( -\tfrac{\omega}{2}n, -\tfrac{\tau}{2\omega n} ) }^2 = O\bigg( \frac{1}{ (\omega^2 n^2 + \tau^2)^2 } \bigg).
\]
A simple calculation yields $1/(\omega^2 n^2 + \tau^2)^2 \in L^1(\Scr{C}^*)$, thus, $F \circ \zeta \in L^2(\Scr{C}^*)$. Hence, as in 
the \hyperlink{L:basiclemma}{Basic Lemma}
\[
\sideset{}{'}\sum_{n = -\infty}^\infty \int_{\!-\infty}^\infty \Abs{ \frac{ F \circ \zeta(n, \tau) }{ P_z(n, \tau) } } \,\rmd\tau \leq C\max
\{ \norm{F \circ \zeta}_{ L^2(\Scr{C}^*) }, \,\norm{F \circ \zeta}_{ L^\infty(\Scr{C}^*) } \}.
\]
It is easy to see that $\norm{F \circ \zeta}_{ L^\infty(\Scr{C}^*) } = \norm{F}_{ L^\infty(\Lambda) }$ and
$\norm{F \circ \zeta}_{ L^2(\Scr{C}^*) }^2 = 4\norm{F}_{ L^2( \abs{\!\real z} ) }^2$. Therefore, if we have
$C\max\{ \norm{F \circ \zeta}_{ L^2(\Scr{C}^*) }, \,\norm{F \circ \zeta}_{ L^\infty(\Scr{C}^*) } \} < 2\pi$ or equivalently
$C\max\{ 2\norm{F}_{ L^2( \abs{\!\real z} ) }, \,\norm{F}_{ L^\infty(\Lambda) } \} < 2\pi$, then $\mathcal{C}\Scr{S}$ is a contraction of $L^\infty(E)$, 
thus, equation \eqref{Eq:Cauchyintegral} has a unique bounded solution.
\end{proof}

We consider again equation \eqref{Eq:Cauchyintegral}, i.e., $\mu = 1 + \mathcal{C}\Scr{S}\mu$, and apply the operator $P(\partial + w)$. Then, 
\[
-u\mu = P(\partial + w)\mu = P(\partial + w)1 + P(\partial + w)\mathcal{C}\Scr{S}\mu = P(\partial + w)\mathcal{C}\Scr{S}\mu.
\]
Since the expression in the definition of $\mathcal{C}\Scr{S}\mu$ converges absolutely, $\mathcal{C}\Scr{S}\mu$ may be differentiated
in the parameters $x$ and $y$ and the following formulas are easily derived.
\begin{lemma}\label{L:commutators}\hfill
\begin{enumerate}
\item $[ P(\partial + w), \Scr{S} ] = 0$,
\item $\displaystyle[ P(\partial + w), \mathcal{C} ]f(x, y; z) = -\frac{1}{\pi}\partial_x \sideset{}{'}\sum_{n = -\infty}^\infty \int_{L_n}
f(x, y; \zeta) \,\rmd\zeta$. This is independent of $z$.
\end{enumerate}
\end{lemma}
As a consequence of this lemma,
\begin{align*}
-u\mu &= \mathcal{C}\Scr{S}(P(\partial + w)\mu) + [ P(\partial + w), \mathcal{C} ]\Scr{S}\mu\\
&= -u\mathcal{C}\Scr{S}\mu - \frac{1}{\pi}\partial_x \sideset{}{'}\sum_{n = -\infty}^\infty \int_{L_n} \Scr{S}\mu \,\rmd\zeta\\
&= -u(\mu - 1) - \frac{1}{\pi}\partial_x \sideset{}{'}\sum_{n = -\infty}^\infty \int_{L_n} \Scr{S}\mu \,\rmd\zeta,
\end{align*}
hence
\begin{equation}\label{Eq:potenial}
u(x, y) = \frac{1}{\pi}\partial_x \sideset{}{'}\sum_{n = -\infty}^\infty \int_{L_n} (\Scr{S}\mu)(x, y; \zeta) \,\rmd\zeta.
\end{equation}
The proof of the main result of this section is obtained by combining the results of theorems~\ref{Th:exist_unique}, \ref{Th:holomorphicity}, 
\ref{Th:zasympotics}, \ref{Th:departurefromholomorphicity} and proposition \ref{Pr:important}.
\begin{theorem}[The Forward Spectral Theorem]\label{Th:forwardscattering}
Suppose that $u(x, y)$ is small in $L^1(\Omega) \cap L^2(\Omega)$ with $y u(x, y) \in L^1(\Omega)$ and that
$\partial^\alpha u \in L^1(\Omega) \cap L^2(\Omega)$ for all $\abs{\alpha} \leq 2$. Then, the unique solution $\mu(x, y; z)$ to the equation 
\[
(-\partial_y + \partial_x^2 + 2i z\partial_x + u)\mu(x, y; z) = 0, \qquad \lim_{\abs{y} \to \infty} \mu(x, y; z) = 1,
\]
is also the unique solution to the equation
\[
\mu(x, y; z) = 1 + \frac{1}{2\pi \rmi}\sideset{}{'}\sum_{n = -\infty}^\infty \int_{L_n} \frac{ (\Scr{S}\mu)(x, y; \zeta) }{\zeta - z} \,\rmd\zeta,
\]
where $\Scr{S}$ is the spectral data associated to u by the heat operator, defined in equation \eqref{Eq:scatteringoperator} and $u$ can be found via 
equation \eqref{Eq:potenial}.
\end{theorem}
We finish this section proving a property of the Jost function $\mu$, crucial to the inverse problem.
\begin{theorem}\label{Th:squareintegrability}
Suppose $\maxnorm{u} < 2\pi/C$ and that $\abs{\partial^\alpha u} \in L^1(\Omega) \cap L^2(\Omega)$ for all $\abs{\alpha} \leq 3$. Then, the function
$\mu(x, y; z) - 1$ is holomorphic with respect to $z \in \wpln$ and
\begin{equation}\label{Eq:meanintegrability}
\sup_{\re{z} \notin \frac{\omega}{2}\sZ} \bigg(\int_{\!-\infty}^\infty \abs{\mu( x, y; \re{z} + \rmi\im{z} ) - 1}^2 \,\rmd\im{z}\bigg)^\frac{1}{2} < \infty,
\end{equation}
for all $(x, y) \in \Omega$.
\end{theorem}
\begin{proof}
Fix a $z$ in $\wpln$. A straightforward algebraic manipulation yields
\begin{equation*}
\frac{1}{ P_z(m, \xi) } = \frac{1}{2\omega m z}\bigg( 1 - \frac{ (\omega m)^2 + \rmi\xi }{ P_z(m, \xi) } \bigg) =
\frac{1}{2z}\bigg( \frac{1}{\omega m} - \frac{\omega m}{ P_z(m, \xi) } - \frac{\rmi\xi}{ \omega m P_z(m, \xi) } \bigg).
\end{equation*}
Hence,
\begin{align*}
\frac{ \wh{u\mu}(m, \xi; z) }{ P_z(m, \xi) } &=
\frac{1}{2z}\bigg( \frac{ \wh{u\mu}(m, \xi; z) }{\omega m} - \frac{ \omega m\wh{u\mu}(m, \xi; z) }{ P_z(m, \xi) }
- \frac{ \rmi\xi\wh{u\mu}(m, \xi; z) }{ \omega m P_z(m, \xi) } \bigg)\\
&= \frac{1}{2z}\bigg( \frac{ \wh{u\mu}(m, \xi; z) }{\omega m} + \rmi\frac{ \wh{\partial_x u\mu}(m, \xi; z) }{ P_z(m, \xi) }
- \frac{ \wh{\partial_y u\mu}(m, \xi; z) }{ \omega m P_z(m, \xi) } \bigg).
\end{align*}
From lemma \ref{L:rapiddeacy} we have
\[
\abs{ \wh{u\mu}(m, \xi; z) } \leq \frac{c_1}{1 + \abs{\omega m}^2 + \abs{\xi}^2}
< \frac{c_1}{ (\omega m)^2 \big(1 + \big( \frac{\xi}{ \omega\abs{m} } \big)^2\big) },
\]
for some positive, real constant $c_1$. Thus,
\[
\frac{ \abs{ \wh{u\mu}(m, \xi; z) } }{ \omega\abs{m} } < c_1\frac{1}{ (\omega m)^2 }
\frac{1}{ \omega\abs{m}\big(1 + \big( \frac{\xi}{ \omega\abs{m} } \big)^2\big) } \in L^1(\Scr{C}^*).
\]
Another application of lemma \ref{L:rapiddeacy} shows that
\[
\abs{ \wh{\partial_x u\mu}(m, \xi; z) } = O\bigg( \frac{1}{1 + \abs{\omega m}^2 + \abs{\xi}^2} \bigg),
\]
from which it follows that $\wh{\partial_x u\mu}(m, \xi; z) \in L^2(\Scr{C}^*) \cap L^\infty(\Scr{C}^*)$. Thus, as a consequence of the 
\hyperlink{L:basiclemma}{Basic Lemma}, $\wh{\partial_x u\mu}(m, \xi; z)/P_z(m, \xi) \in L^1(\Scr{C}^*)$. Finally, by lemma \ref{L:rapiddeacy} again, 
there exists a positive, real constant $c_2$ such that
\[
\abs{ \wh{\partial_y u\mu}(m, \xi; z) } \leq \frac{c_2}{1 + \abs{\omega m}^2 + \abs{\xi}^2}.
\]
But then, an application of H\"{o}lder's inequality gives
\begin{align*}
\Norm{ \frac{\wh{\partial_y u\mu}(m, \xi; z) }{ \omega m P_z(m, \xi) } }_{ L^1(\Scr{C}^*) } &\leq
\Norm{ \frac{c_2}{1 + \abs{\omega m}^2 + \abs{\xi}^2}\frac{1}{ \omega m P_z(m, \xi) } }_{ L^1(\Scr{C}^*) }\\
&\leq \Norm{ \frac{c_2}{ \omega m(1 + \abs{\omega m}^2 + \abs{\xi}^2) } }_{ L^2(\Scr{C}^*) } \Norm{ \frac{1}{ P_z(m, \xi) } }_{ L^2(\Scr{C}^*) }\\
&< \infty.
\end{align*}
Therefore,
\[
I \equiv \sideset{}{'}\sum_{m = -\infty}^\infty \int_{\!-\infty}^\infty
\,\Abs{ \frac{ \wh{u\mu}(m, \xi; z) }{\omega m} + \rmi\frac{ \wh{\partial_x u\mu}(m, \xi; z) }{ P_z(m, \xi) }
- \frac{ \wh{\partial_y u\mu}(m, \xi; z) }{ \omega m P_z(m, \xi) } }\,\rmd\xi < \infty.
\]

Hence,
\[
\abs{\mu(x, y ; z) - 1}^2 \leq \frac{1}{ (2\pi)^2 }\Bigg(\sideset{}{'}\sum_{m = -\infty}^\infty \int_{\!-\infty}^\infty
\,\Abs{ \frac{ \wh{u\mu}(m, \xi; z) }{ P_z(m, \xi) } } \,\rmd\xi\Bigg)^2 = \frac{1}{ (2\pi)^2 }\frac{1}{4\abs{z}^2}I^2 = \frac{c_3}{\abs{z}^2},
\]
with $c_3$ a real constant. For $\re{z} \neq 0$, $1/\abs{z}^2$ is integrable over the real line (with respect to $\imaginary z$) and
\[
\int_{\!-\infty}^\infty \frac{1}{\abs{z}^2} \,\rmd\imaginary z = \int_{\!-\infty}^\infty \frac{1}{\abs{ \re{z} + \rmi\im{z} }^2} \,\rmd\im{z} =
\int_{\!-\infty}^\infty \frac{1}{\re{z}^2 + \im{z}^2} \,\rmd\im{z} = \frac{\pi}{ \abs{ \re{z} } }.
\]
Thus,
\[
\int_{\!-\infty}^\infty \abs{\mu( x, y; \re{z} + \rmi\im{z} ) - 1}^2 \,\rmd\im{z} < \frac{c_3 \pi}{ \abs{ \re{z} } } < \infty.
\]
When $\re{z} = 0$ the conclusion follows from Fatou's lemma: let $\{ { \re{z} }_n \}$ be a sequence of real numbers not in $\frac{\omega}{2}\bb{Z}$ such 
that ${ \re{z} }_n \to 0$. Then,
\begin{align*}
\int_{\!-\infty}^\infty \abs{\mu( x, y; \rmi\im{z} ) - 1}^2 \,\rmd\im{z} &\leq \liminf_{n \to \infty} \int_{\!-\infty}^\infty
\abs{\mu( x, y; { \re{z} }_n + \rmi\im{z} ) - 1}^2 \,\rmd\im{z}\\
&< \liminf_{n \to \infty} \frac{c_3 \pi}{ \abs{ { \re{z} }_n } } = \infty.\qedhere
\end{align*}
\end{proof}