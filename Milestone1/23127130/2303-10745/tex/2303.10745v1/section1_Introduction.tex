\section[Introduction]{Introduction}\label{S:intro}
\medskip

The Kadomtsev--Petviashvili (\textsc{KP}) equation
\begin{equation}\label{Eq:kp}
( u_t + 6u u_x + u_{xxx} )_x = -3\sigma^2 u_{yy},
\end{equation}
where $\sigma^2 = \pm1$, was first introduced in 1970 by B.~Kadomtsev and V.~Petviashvili~\cite{KP70} as a model to study the evolution of long
ion-acoustic waves of small amplitude, propagating in plasma under the effect of long transverse perturbations and later derived as a model for surface
water waves \cite{FD75} and \cite{AS79} (see also \cite{SF85}). It may be thought of as a two spatial dimension analog of the Korteweg--de Vries
(KdV) equation
\begin{equation}\label{Eq:kdv}
u_t + 6u u_x + u_{xxx} = 0.
\end{equation}
The choice of the sign of $\sigma^2$ is critical with respect to the stability of solitons of the KdV equation subject to transverse perturbations
(in the $y$ direction). For $\sigma^2 = -1$, they are unstable, while for $\sigma^2 = 1$ they are stable.
The case $\sigma^2 = 1$ is known as the \textsc{KP}II equation and, in the context of fluid mechanics, appears in the study of long waves in shallow water 
under weak surface tension whereas the case $\sigma^2 = -1$ is called the \textsc{KP}I equation and can be employed to model water waves in thin films, 
where the very high surface tension dominates the gravitational force. The \textsc{KP} equation is one of the most notable integrable nonlinear evolution
PDEs in $2 + 1$ dimensions (i.e., two spatial and one temporal) and is solvable by use of the so called inverse scattering transform.

The Inverse scattering transform method can be viewed, as explained in \cite{AKNS74}, as a nonlinear analog of the Fourier transform and reduces the 
solution of the Cauchy problem to the solution of an inverse scattering problem for an associated linear eigenvalue equation. This method was discovered in 
1967 in the famous article~\cite{GGKM67}, as a way to solve the initial-value problem for the KdV equation with decaying initial data on the real line. The 
possibility of using the inverse scattering transform method for the \textsc{KP} equation follows from the existence of a Lax pair. Such a pair was discovered 
by Dryuma \cite{D74} and Zakharov and Shabat \cite{ZS74} independently. For the \textsc{KP}I equation, the possibility of implementing the inverse 
scattering transform method was suggested by Manakov \cite{M81} and Segur \cite {S82} and was implemented by Fokas and Ablowitz \cite{FA83}. This 
formulation was improved and corrected by several authors, as reviewed in \cite{F09}. The analysis of \textsc{KP}II was implemented by Ablowitz, Bar 
Yaacov and Fokas \cite{ABF83} using the so-called $\conj{\partial}$ formalism. This formalism was introduced earlier by Beals and Coifman 
in \cites{BC81,BC82} for the analysis of evolution PDEs in one space variable where the Riemann--Hilbert problem approach is not only adequate but also 
preferable. Rigorous aspects of the new methodology, often referred to as the inverse spectral transform (\textsc{IST}), were developed by several authors 
including Beals and Coifman \cites{BC84,BC85}. In particular, rigorous treatment of the \textsc{KP}II equation for the decaying in the plane problem was 
given by Wickerhauser in \cite{MV87} and by Fokas and Sung in \cite{FS92}.

The aim of this paper is to establish that another class of initial-value problems in $2 + 1$ dimensions can be incorporated in the above techniques of the 
\textsc{IST} scheme: those with initial data periodic in one spatial direction and decaying in the other. Associated with the \textsc{KP} equation there exist 
four such problems: \textsc{KP}I periodic in $x$, \textsc{KP}I periodic in $y$, \textsc{KP}II periodic in $x$, \textsc{KP}II periodic in $y$. In this work we 
consider the initial-value problem for the \textsc{KP}II equation, assuming that $u$ is a periodic function in the $x$ spatial variable, with period $2\ell > 0$, 
and decaying in the $y$ direction, i.e., we study the following Cauchy problem:
\begin{subequations}\label{Eq:cauchyproblem}\begin{alignat}{2}
( u_t + 6u u_x + u_{xxx} )_x &= -3u_{yy}, \label{Eq:cauchyproblem_a}\\[2pt]
u(x + 2\ell, y, t) &= u(x, y, t), \quad& &(x, y) \in \bb{R}^2, \ t \geq 0, \label{Eq:cauchyproblem_b}\\
u(x, y, 0) &= u_0(x, y), & &x \in [-\ell, \ell\,], \ y \in \bb{R}, \label{Eq:cauchyproblem_c}
\end{alignat}\end{subequations}
where $u(x, y, t) \to 0$ sufficiently rapidly as $\abs{y} \to \infty$ and $u_0(x, y)$ is a known function which belongs to some appropriate functional space, 
satisfying the \emph{zero mass} constraint, i.e.,
\begin{equation}\label{Eq:zeromass}
\int_{\!-\ell}^\ell u_0(x, y) \,\rmd x = 0, \quad \forall \:y \in \bb{R}.
\end{equation}
Modified accordingly, we believe that the method presented here can be used for solving the other three semi-periodic problems mentioned earlier.\newline
The zero mass constraint for the corresponding problem for $(x, y) \in \bb{R}^2$, arises when equation \eqref{Eq:kp} is put in evolution form, namely,
\begin{equation}\label{Eq:evolkp}
u_t + 6u u_x + u_{xxx} = -3\sigma^2 \partial_x^{-1} u_{yy},
\end{equation}
and a meaning of $\partial_x^{-1}$, corresponding to the initial data, has to be properly defined. The implications of this constraint was studied 
in \cite{AV91}. It was realized that if one chooses $\partial_x^{-1} = \int_{-\infty}^x$ or $\partial_x^{-1} = \int_x^\infty$, then the constraint
\begin{equation}\label{Eq:const}
\frac{\partial^2}{\partial y^2}\biggl(\int_{-\infty}^\infty u(x, y, t) \,\rmd x\biggr) = 0
\end{equation}
is required. However, given sufficiently decaying and smooth initial data, then
\begin{equation}\label{Eq:primit}
\partial_x^{-1} = \frac{1}{2}\biggl(\int_{-\infty}^x - \int_x^\infty\biggr),
\end{equation}
and no further constraints on the initial data appear. Interestingly, the authors found that even with the choice~\eqref{Eq:primit}, the condition 
\eqref{Eq:const} is eventually achieved. For more about the zero mass constraint see also \cite{MST07} and \cites{FS99,S99}.

Caudrey in \cite{C86} considered the Lax pair of the \textsc{KP} equation as a certain limit of a suitable $N \times N$ problem in $1 + 1$ dimensions after 
``discretising'' one of the spatial variables. Then, letting $N \to \infty$, Caudrey obtained formal results with initial data periodic in one direction and
decaying in the other. Here we obtain rigorous results by considering the semi-periodic problem ($x$-periodic) for \textsc{KP}II directly using the \textsc{IST}
method. We also mention \cite{GPDS09} for results on the Cauchy problem for a class of \textsc{KP}II equations with a general $x$-dispersion of order
$\geq 2$ including the classical \textsc{KP}II equation on the spatial domain $\bb{T}_x \times \bb{R}_y$ (periodicity in $x$) via PDE techniques. For a 
detailed review of results of the \textsc{KP} equations with \textsc{IST} and PDE methods we refer to the recent monograph \cite{KS22}.

The rigorous analysis carried throughout this paper follows the work of \cite{MV87}. In the rest of this section we give a brief description of our results.
The Lax pair associated with the \textsc{KP}II equation in our case is given by
\begin{subequations}\label{Eq:laxpair}\begin{align}
\op{L} &= -\partial_y + \partial_{xx} + u, \label{Eq:Lpart}\\
\op{M} &= 4\partial_{xxx} + 6u\partial_x + 3u_x + 3\int_{\!-\ell}^x u_y \,\rmd s + \emph{\text{\textalpha}}, \label{Eq:Mpart}
\end{align}\end{subequations}
i.e., the compatibility of the linear problems $\op{L}\psi = \lambda\psi$ and $\psi_t = \op{M}\psi$ with $\lambda_t = 0$, yields \textsc{KP}II. Here
$\emph{\text{\textalpha}}$ is an arbitrary constant, independent of $x$ and $y$. Later on it will acquire dependence on $z \in \bb{C}$. For small smooth 
functions $u \in L^1 \cap L^2( [-\ell, \ell\,] \times \bb{R} )$, the operator $\op{L}$ can be determined by the leading coefficients of asymptotically 
exponential functions in its kernel. More precisely, let $z \in \bb{C}$ such that $2\real z \neq \frac{\pi}{\ell}n$, for every nonzero integer $n$, and let
$\mu(x, y; z)$ be a bounded function such that the function $\psi(x, y; z) = \mu(x, y; z)\rme^{\rmi z x - z^2 y}$ is in the kernel of \eqref{Eq:Lpart}.
If $u(x, y)$ is small in $L^1 \cap L^2( [-\ell, \ell\,] \times \bb{R} )$, there exists a unique such $\mu$ with the asymptotic behaviour $\mu(x, y; z) \to 1$ as
$\abs{y} \to \infty$~(theorem \ref{Th:exist_unique}). Hence, $\psi$ is asymptotic to $\rme^{\rmi z x - z^2 y}$. This function $\mu$ is holomorphic for such 
complex numbers $z$~(theorem \ref{Th:holomorphicity}). If $u$ has first order partial derivatives in $L^1 \cap L^2( [-\ell, \ell\,] \times \bb{R} )$,
then $\mu$ satisfies a Riemann--Hilbert problem with a shift:
\begin{equation}\label{Eq:RHproblem}
\mu^+(x, y; z) - \mu^-(x, y; z) = F(z)\rme^{-\rmi( z + \conj{z} )x + (z^2 - \conj{z}^2)y} \mu^-( x, y; -\conj{z} ) \equiv \Scr{S}\mu,
\end{equation}
with
\begin{equation}
F(z) = \frac{ \sgn(-\real z) }{2\ell}\int_{\!-\infty}^\infty \int_{\!-\ell}^\ell
u(x, y)\mu^+(x, y; z)\rme^{\rmi( z + \conj{z} )x - (z^2 - \conj{z}^2)y} \,\rmd x\rmd y,
\end{equation}
where $\mu^\pm$ are the pointwise limits of $\mu$ from the left and from the right side of the lines $\real z = \frac{\pi}{\ell}n/2$
(theorem \ref{Th:zasympotics} and theorem \ref{Th:departurefromholomorphicity}). The function $F(z)$ determines the departure from holomorphicity of
$\mu$ across these lines. If $u$ has partial derivatives up to second order in $L^1 \cap L^2( [-\ell, \ell\,] \times \bb{R} )$, the spectral data $F(z)$ has 
enough decay to ensure the existence of a unique bounded solution $\mu$ to the Riemann--Hilbert problem~\eqref{Eq:RHproblem} with $\mu(x, y; z) \to 1$ 
as $\abs{\!\imaginary z} \to \infty$ (theorem \ref{Th:forwardscattering}). Hence, both $\mu$ and $\Scr{S}\mu$ are determined by $F(z)$.
Then, $u$ is determined by
\begin{equation}\label{Eq:eigenfunctions}
u(x, y) = \frac{1}{\pi}\partial_x \sum_{ \substack{n \in \bb{Z} \\ n \neq 0} } \int_{\real z = \frac{\pi}{\ell}n/2}
F(z)\rme^{-\rmi( z + \conj{z} )x + (z^2 - \conj{z}^2)y} \mu^-( x, y; -\conj{z} ) \,\rmd \imaginary z
\end{equation}
(theorem \ref{Th:inversescattering}). Therefore, knowledge of the spectral data $F(z)$ suffices to determine $u$.

The maps $u \mapsto F$ and $F \mapsto u$ might be called the forward and inverse spectral transforms, respectively. They behave like the Fourier 
transform and its inverse. If $u(x, y)$ has derivatives up to second order in $L^1 \cap L^2( [-\ell, \ell\,] \times \bb{R} )$, then $F(z)$ decays like
$( 1 + \abs{ (\real z, \real z \imaginary z) } )^{-2}$ (lemma \ref{L:rapiddeacy} and equation \eqref{Eq:datadecay}). On the other hand, if $F(z)$ decays like
$( 1 + \abs{ (\real z, \real z \imaginary z) } )^{-4}$, then $u(x, y)$ has derivatives up to second order in $L^2( [-\ell, \ell\,] \times \bb{R} )$ and a bounded 
Fourier transform (theorem \ref{Th:inversedecay}).

In order to complete the procedure of the \textsc{IST} and establish a solution to our problem, the time evolution of the spectral data $F(z, t)$ needs to be 
determined. Let $t > 0$ be thought of as time. If $u(x, y, t)$ evolves according to the \textsc{KP}II equation, then
$\frac{\rmd}{\rmd t}F(z, t) = -4\rmi(z^3 + \conj{z}^3)F(z, t)$ or $F(z, t) = F(z, 0)\rme^{-4\rmi(z^3 + \conj{z}^3)t}$ (lemma \ref{L:temporalevol}).
Since $z^3 + \conj{z}^3$ is real, one has $\abs{ F(z, t) } = \abs{ F(z, 0) }$ for all $z$ on the lines $\real z = \frac{\pi}{\ell}n/2$ and $t \geq 0$. By the 
forward and inverse spectral transform, there is a solution $u(x, y, t)$ for all time to the initial-value problem \eqref{Eq:cauchyproblem}:
if the initial value $u_0(x, y)$ is sufficiently small and has derivatives up to order eight in $L^1 \cap L^2( [-\ell, \ell\,] \times \bb{R} )$, then the initial
spectral data $F(z, 0)$ is known, hence the spectral data is known for all time. Thus, $\mu(x, y, t; z)$ is known for all time and finally $u(x, y, t)$ can be 
recovered via \eqref{Eq:eigenfunctions} (theorem \ref{Th:solution}).\newline
The following figure depicts the inverse spectral transform method.
\begin{figure}[H]
\[
\begin{tikzcd}[row sep=large, column sep=large]
u(x, y, 0) \arrow{r}[above]{\substack{\text{forward} \\ \text{transform}}} \arrow[d, dashrightarrow, ""{name=D, right}]
 & F(z, 0) \arrow{d}[right, ""{name=U, left}]{\substack{\text{spectral data} \\ \text{evolution}}}\\
u(x, y, t)
 & \arrow{l}[above]{\substack{\text{inverse} \\ \text{transform}}} F(z, t)
\arrow[Rightarrow, from=U, to=D]
\end{tikzcd}
\]
\caption{Inverse spectral transform scheme\label{fig:istscheme}}
\end{figure}

For clarity of the exposition, let us collect here the notation that will frequently be used throughout this paper.
For a complex number $z$, $\re{z}$ and $\im{z}$ denote its real and imaginary part respectively, thus, $z = \re{z} + \rmi\im{z}$. $\bb{N}$ denotes the
set of natural numbers, i.e., $\bb{N} = \{1, 2, \dotsc\}$, and $\bb{Z}$ the set of integers while $\bb{N}_0 = \bb{N} \cup \{0\}$ and
$\sZ = \bb{Z} \setminus \!\{0\}$. $\Omega = \{ (x, y) \colon \!-\ell \leq x \leq \ell, \ y \in \bb{R} \}$, $\Scr{C} = \bb{Z} \times \bb{R}$ and
$\wpln = \{z \in \bb{C} \colon 2\re{z} \notin \omega\sZ\}$, where $\omega = \frac{\pi}{\ell}$. We write $E$ for the set $\Omega \times \wpln$ and
$\Scr{C}^*$ for the set $\sZ \times \bb{R}$ . For $1 \leq p \leq \infty$, we define the sets $L^p( \Scr{C} )$ by
\begin{equation}
L^p( \Scr{C} ) \= \{f \colon \Scr{C} \to \bb{C} \mid f \text{ is measurable and } \norm{f}_{ L^p( \Scr{C} ) } < \infty\},
\end{equation}
where
\begin{equation}
\norm{f}_{ L^p( \Scr{C} ) } \= \Bigg(\sum_{m = -\infty}^\infty \int_{\!-\infty}^\infty \,\abs{ f(m, \xi) }^p \,\rmd\xi\Bigg)^\frac{1}{p},
\quad 1 \leq p < \infty,
\end{equation}
and
\begin{equation}
\norm{f}_{ L^\infty( \Scr{C} ) } \= \essup_{ (m, \xi) \in \Scr{C} } \abs{ f(m, \xi) }.
\end{equation}
It is easy to establish that $( L^p( \Scr{C} ), \,\norm{\cdot}_{  L^p( \Scr{C} ) } )$ are Banach spaces. For a set $A$, other than
$\Scr{C}$ (or $\Scr{C}^*$), the standard notation for the $p$-norm of the Lebesgue spaces $L^p(A)$, will be used, i.e., $\norm{\cdot}_p$. $C(\Omega)$
is the set of continuous (complex) functions on $\Omega$ and $H(\wpln)$ is the set of holomorphic functions on $\wpln$.

The identity operator on a normed space is denoted by $\op{Id}$. If $\op{A}$ and $\op{B}$ are two operators, we denote with $[ \op{A}, \op{B} ]$ their
commutator, namely
\begin{equation}
[ \op{A}, \op{B} ] = \op{A}\op{B} - \op{B}\op{A},
\end{equation}
whenever this make sense. For a (linear) bounded operator $\op{T}$ between two normed spaces, we write $\norm{ \op{T} }_\mrm{op}$ for its norm.

If $f(x, y)$ is a function which is $2\ell$-periodic in $x$ and decaying in $y$ (i.e., $f$ and $\partial_y^n f$ tend to zero as $\abs{y} \to \infty$, for all 
$n \in \bb{N}$), define its Fourier transform by the integral
\begin{equation}
\wh{f}(m, \xi) \= \frac{1}{2\ell}\int_{\!-\infty}^\infty \int_{\!-\ell}^\ell f(x, y)\rme^{-\rmi\omega m x - \rmi\xi y} \,\rmd x\rmd y, \quad (m, \xi) \in \Scr{C},
\end{equation}
and the inverse transform by
\begin{equation}
f(x, y) = \big[ \wh{f}(m, \xi) \big]\spcheck\!(x, y) \= \frac{1}{2\pi}\sum_{m = -\infty}^\infty \int_{\!-\infty}^\infty
\wh{f}(m, \xi)\rme^{\rmi\omega m x + \rmi\xi y} \,\rmd\xi.
\end{equation}
This definition implies that $f \mapsto \wh{f}$ has norm less than $\omega$ as a map from $L^1(\Omega)$ to $L^\infty( \Scr{C} )$ and that when
$f \in L^1(\Omega) \cap L^2(\Omega)$, $\norm{\wh{f}\,}_{ L^2( \Scr{C} ) } = \sqrt{\omega}\norm{f}_2$. We will also use the notation
$\Scr{F}( f(x) )(k)$ for the Fourier transform of some function $f(x)$ in the real line namely
\[
\Scr{F}( f(x) )(k) \= \frac{1}{2\pi}\int_{\!-\infty}^\infty f(x)\rme^{-\rmi k x} \,\rmd x.
\]

The notation $\partial_v^n$ will be used to denote the $n$-th order partial derivative operator with respect to the variable $v$, thus,
$\partial_v^n = \partial^n\!/\partial v^n$, where $n$ is a \text{non-neg}\-ative integer. If $\alpha = (\alpha_1, \alpha_2)$ is a multi-index, then
\begin{equation}
\partial^\alpha = \partial_x^{\alpha_1} \partial_y^{\alpha_2}
\end{equation}
denotes a differential operator of order $\abs{\alpha} = \alpha_1 + \alpha_2$, with $x \in [-\ell, \ell\,]$ and $y \in \bb{R}$. If $\abs{\alpha} = 0$, then
$\partial^\alpha f(x, y) = f(x, y)$. For a point $s = (s_1, s_2)$ in $\bb{R}^2$ its norm is given by
\begin{equation}
\abs{s} = \sqrt{s_1^2 + s_2^2}.
\end{equation}
For a multi-index $\alpha = (\alpha_1, \alpha_2)$, we denote by $s^\alpha$ the monomial $s_1^{\alpha_1} s_2^{\alpha_2}$, which has degree
$\abs{\alpha}$ .

A sum with a prime next to it will mean summation over all integers except zero, i.e.,
\begin{equation}
\sideset{}{'}\sum_{m = -\infty}^\infty = \sideset{}{'}\sum_{ m \in \bb{Z} } = \sum_{m \in \sZ}.
\end{equation}

The Bachmann--Landau notation will be used in the normal way: given two functions $f$, $g$ defined on a set of real numbers, we write
\begin{equation}
f(x) = O( g(x) ),
\end{equation}
if $\abs{ f(x) } \leq M\abs{ g(x) }$, for some positive number $M$ and
\begin{equation}
f(x) = o( g(x) ),
\end{equation}
if $f(x)/g(x) \to 0$ as $x \to \infty$.

For brevity we will suppress the $t$-dependence for the functions until section \ref{S:time_evolution}.