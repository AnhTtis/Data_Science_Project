\section[The Direct Problem]{The Direct Problem}\label{S:direct}
\smallskip
\subsection[Bounded Eigenfunctions of the Perturbed Heat Operator]{Bounded Eigenfunctions of the Perturbed Heat Operator}\label{s:bdd_eigen}
\separate

Consider the spectral problem for the operator $\op{L}$, i.e., the spectral equation
\begin{equation}\label{Eq:spectral}
\op{L}\psi = -\psi_y + \psi_{xx} + u\psi = \lambda\psi.
\end{equation}
We wish to reconstruct the potential $u$ through the spectral data of $\op{L}$. Using the transformation $\psi \to \psi \rme^{-\lambda y}$, the spectral
variable $\lambda$ is eliminated from equation \eqref{Eq:spectral}, hence we arrive at the equation
\begin{equation}\label{Eq:heat}
-\psi_y + \psi_{xx} + u\psi = 0,
\end{equation}
which is the well known one-dimensional heat equation, perturbed by $u$. If $u$ decays, then a class of solutions to this equation may be
specified---functions that are asymptotic to an exponential solution to the unperturbed equation. More precisely, introduce the \emph{Jost}
function $\mu$ defined by
\begin{equation}\label{Eq:Jost}
\psi(x, y; z) = \mu(x, y; z)\rme^{\rmi z x - z^2 y}; \qquad \lim_{\abs{y} \to \infty} \mu(x, y; z) = 1, \text{ for each } z,
\end{equation}
where $z$ is a complex variable and $\psi$ is a solution to equation \eqref{Eq:heat}. This function satisfies the boundary-value problem
\begin{subequations}\label{Eq:bvp}\begin{align}
&-\mu_y + \mu_{xx} + 2\rmi z\mu_x + u\mu = 0, \label{Eq:bvp_a}\\
&\text{ for each } z, \ \lim_{\abs{y} \to \infty} \mu(x, y; z) = 1.
\end{align}\end{subequations}
If we know $\mu$, we can determine $u$. Hence, we must show that for a given $u$ there is a unique function $\mu$ which solves \eqref{Eq:bvp}. 
Introduce the shifted derivatives
\begin{equation}\label{Eq:shifted}
D_1 = \partial_x + \rmi z, \ D_2 = \partial_y - z^2, \quad z \in \bb{C}.
\end{equation}
By ``completing the square'', rewrite equation \eqref{Eq:bvp_a} in operator form as a polynomial in $D_1, D_2$, i.e.,
\begin{equation}\label{Eq:shiftedeq}
[ (\partial_x + \rmi z)^2 - (\partial_y - z^2) ]\mu \equiv P( \partial + w(z) )\mu = -u\mu,
\end{equation}
where $\partial = (\partial_x, \partial_y)$, $w(z) = (\rmi z, -z^2)$ and $P(a, b) = a^2 - b$. Since $P( \partial + w(z) )$ annihilates the constant function 1, 
this equation can be written as
\begin{equation}
P( \partial + w(z) )(\mu - 1) = -u\mu.
\end{equation}
Applying the Fourier transform (with $t$ and $z$ considered parameters) yields
\begin{equation}\label{Eq:Transformed}
[ (\rmi\omega m + \rmi z)^2 - (\rmi\xi - z^2) ]( \wh{\mu - 1} )(m, \xi; z) = -\wh{u\mu}(m, \xi; z).
\end{equation}
Let us introduce some convenient notation:
\begin{equation}
q = (m, \xi) \in \Scr{C}, \quad P_z(m, \xi) = -P( \rmi(\omega m, \xi) + w(z) ).
\end{equation}
Then, $P_z(m, \xi) = (\omega m)^2 + 2\omega m z + \rmi\xi$, hence \eqref{Eq:Transformed} reads
\begin{equation}\label{Eq:bvp_atransformed}
P_z(m, \xi)( \wh{\mu - 1} )(m, \xi; z) = \wh{u\mu}(m, \xi; z).
\end{equation}
For each $z \in \bb{C}$, there are two distinct roots $(m, \xi)$ of $P_z(m, \xi)$, namely $(0, 0)$ and
\begin{equation}\label{Eq:nontrivialzero}
r_0(z) \equiv \Big( \frac{ -2\re{z} }{\omega}, 4\re{z}\im{z} \Big) = -\Big( \frac{ z + \bar{z} }{\omega}, \,\rmi(z^2 - \bar{z}^2) \Big).
\end{equation}
Using the linearity of the Fourier transform, we can write equation \eqref{Eq:bvp_atransformed} in the form
\begin{equation}\label{Eq:contractionform}
( \wh{\mu - 1} )(m, \xi; z) = \frac{ \wh{u}(m, \xi) }{ P_z(m, \xi) } + \frac{ ( \wh{u} \ast \wh{\mu - 1} )(m, \xi; z) }{ 2\pi P_z(m, \xi) },
\end{equation}
where $\ast$ denotes convolution in the $m$ and $\xi$ variables.

The study of this equation (equivalently of equation~\eqref{Eq:bvp_a}) is based on the following \tbf{basic} lemma which will be the main tool of our analysis 
and will be extensively used in the proofs of the theorems to follow.
\begin{basiclemma}\hypertarget{L:basiclemma}
Let $f \in L^2( \Scr{C} ) \cap L^\infty( \Scr{C} )$ such that $f(0, \xi) = 0$ for all $\xi \in \bb{R}$. Then,
\begin{equation}
(m, \xi) \mapsto \frac{ f(m, \xi) }{ P_z(m, \xi) } \in L^1 ( \Scr{C} ),
\end{equation}
and
\begin{equation}\label{Eq:basiclone}
\Norm{ \frac{f}{P_z} }_{ L^1 ( \Scr{C} ) } \equiv \sum_{m = -\infty}^\infty \int_{\!-\infty}^\infty \,\Abs{ \frac{ f(m, \xi) }{ P_z(m, \xi) } } \,\rmd\xi
\leq C\max\{ \norm{f}_{ L^2 ( \Scr{C} ) }, \norm{f}_{ L^\infty ( \Scr{C} ) } \},
\end{equation}
uniformly in $z \in \wpln$, where
\begin{equation}\label{Eq:constant}
C = \frac{4\pi^2}{3\omega^2} + \frac{\pi}{\omega}\sqrt{ \frac{\pi}{3} }.
\end{equation}
\end{basiclemma}
\begin{proof}
Since $f(0, \xi) = 0$ for every real number $\xi$, it is sufficient to show that
\[
\sideset{}{'}\sum_{m = -\infty}^\infty \int_{\!-\infty}^\infty \,\Abs{ \frac{ f(m, \xi) }{ P_z(m, \xi) } } \,\rmd\xi \leq
C\max\{ \norm{f}_{ L^2 ( \Scr{C} ) }, \norm{f}_{ L^\infty ( \Scr{C} ) } \}.
\]
Let $z \in \wpln$. Then, for $(m, \xi) \in \Scr{C}^*$,
\begin{align*}
P_z(m, \xi) &= (\omega m)^2 + 2\omega m\re{z} + \rmi( \xi + 2\omega m\im{z} )\\
&= ( \omega m + \re{z} )^2 - \re{z}^2 +\rmi( \xi + 2\omega m\im{z} ).
\end{align*}
Thus,
\begin{equation}
\sideset{}{'}\sum_{m = -\infty}^\infty \int_{\!-\infty}^\infty \,\Abs{ \frac{ f(m, \xi) }{ P_z(m, \xi) } } \,\rmd\xi = \sum_{m \in \mZ} \int_{\!-\infty}^\infty
\frac{ \abs{ f( \frac{ m - \re{z} }{\omega}, \xi - 2( m - \re{z} )\im{z} ) } }{ \abs{m^2 - \re{z}^2 + \rmi\xi} } \,\rmd\xi,
\end{equation}
where $\mZ = \omega\sZ + \re{z}$. Split the above integral into one with $\abs{\xi} < 1$ and its complement with $\abs{\xi} \geq 1$. Now,
\begin{align*}
\sum_{m \in \mZ} \int_{\!-1}^1 \frac{1}{ \abs{m^2 - \re{z}^2 + \rmi\xi} } \,\rmd\xi &= \sum_{m \in \mZ} \int_{\!-1}^1
\frac{1}{ \big[ (m^2 - \re{z}^2)^2 + \xi^2 \big]^\frac{1}{2} } \,\rmd\xi\\
&\leq \sum_{m \in \mZ} \int_{\!-1}^1 \frac{1}{ \abs{m^2 - \re{z}^2} } \,\rmd\xi\\
&= 2\sum_{m \in \mZ} \frac{1}{ \abs{m^2 - \re{z}^2} },
\end{align*}
and
\begin{align*}
\sum_{m \in \mZ} \int_{\abs{\xi} \geq 1} \frac{1}{\abs{m^2 - \re{z}^2 + \rmi\xi}^2} \,\rmd\xi
&= \sum_{m \in \mZ} \int_{\abs{\xi} \geq 1} \frac{1}{ (m^2 - \re{z}^2)^2 + \xi^2 } \,\rmd\xi\\
&= 2\sum_{m \in \mZ} \int_1^\infty \frac{1}{ (m^2 - \re{z}^2)^2 + \xi^2} \,\rmd\xi\\
&= 2\sum_{m \in \mZ} \int_1^\infty \frac{ \abs{m^2 - \re{z}^2}^{-1} }{1 + ( \xi/\abs{m^2 - \re{z}^2} )^2} \,\rmd( \xi/\abs{m^2 - \re{z}^2} )\\
&= \frac{\pi}{2}\sum_{m \in \mZ} \frac{1}{ \abs{m^2 - \re{z}^2} }.
\end{align*}
Therefore, the estimation of the sum
\[
\sum_{m \in \mZ} \frac{1}{ \abs{m^2 - \re{z}^2} },
\]
will lead to the desired result.

Since $m = \omega k + \re{z}$ and $\re{z} \neq \frac{\omega}{2}2k = \omega k$ for every nonzero integer $k$, we see that $m$ is not zero.
Thus, we can write
\[
\sum_{m \in \mZ} \frac{1}{ \abs{m^2 - \re{z}^2} } = \sum_{m \in \mZ^+} \frac{1}{ \abs{m^2 - \re{z}^2} } +
\sum_{m \in \mZ^-} \frac{1}{ \abs{m^2 - \re{z}^2} } = 2\sum_{m \in \mZ^+} \frac{1}{ \abs{m^2 - \re{z}^2} }.
\]
Now, for every pair of distinct, positive real numbers $a$ and $b$, the following inequality holds true:
\begin{equation}\label{Eq:basicinequality}
(a - b)^4 < (a^2 - b^2)^2,
\end{equation}
for $0 < 4a b$, so we can conclude that $(a - b)^2 < (a + b)^2$ and the inequality follows.
Suppose $\re{z} \neq 0$. Replacing $a$ and $b$ with $m$ and $\abs{ \re{z} }$ and taking square roots we arrive at
\[
( m - \abs{ \re{z} } )^2 < \abs{m^2 - \re{z}^2}.
\]
Thus,
\[
\sum_{m \in \mZ^+} \frac{1}{ \abs{m^2 - \re{z}^2} } < \sum_{m \in \mZ^+} \frac{1}{ ( m - \abs{ \re{z} } )^2 }.
\]
When $\re{z} > 0$,
\[
\sum_{m \in \mZ^+} \frac{1}{ ( m - \abs{ \re{z} } )^2 } = \frac{1}{\omega^2}\sum_{ \substack{m \in \sZ \\ \omega m + \re{z} > 0} } \frac{1}{m^2}
= \frac{1}{\omega^2}\sum_{ \substack{m > -\frac{ \re{z} }{\omega} \\ m \neq 0} } \frac{1}{m^2} < \frac{2}{\omega^2}\sum_{m = 1}^\infty \frac{1}{m^2}.
\]
Assume now that $\re{z} < 0$. Then, again
\begin{align*}
\sum_{m \in \mZ^+} \frac{1}{ ( m - \abs{ \re{z} } )^2 } &= \sum_{m \in \mZ^+} \frac{1}{ ( m + \re{z} )^2 } =
\sum_{m \in \mZ^-} \frac{1}{ ( m - \re{z} )^2 } \\
&= \frac{1}{\omega^2}\sum_{ \substack{m \in \sZ \\ \omega m + \re{z} < 0} } \frac{1}{m^2} =
\frac{1}{\omega^2}\sum_{ \substack{m < -\frac{ \re{z} }{\omega} \\ m \neq 0} } \frac{1}{m^2} \\
&< \frac{2}{\omega^2}\sum_{m = 1}^\infty \frac{1}{m^2}.
\end{align*}
Putting the parts together yields the estimate
\begin{equation}
\sum_{m \in \mZ} \frac{1}{ \abs{m^2 - \re{z}^2} } < \frac{4}{\omega^2}\sum_{m = 1}^\infty \frac{1}{m^2}.
\end{equation}
Observe that this inequality also holds when $\re{z} = 0$ for
\begin{equation*}
\sum_{m \in \mZ} \frac{1}{ \abs{m^2 - \re{z}^2} }  = \sum_{m \in \omega\sZ} \frac{1}{m^2} =
\frac{1}{\omega^2}\sideset{}{'}\sum_{m = -\infty}^\infty \frac{1}{m^2}
= \frac{2}{\omega^2}\sum_{m = 1}^\infty \frac{1}{m^2} < \frac{4}{\omega^2}\sum_{m = 1}^\infty \frac{1}{m^2}.
\end{equation*}

Returning to the original problem, we find that
\begin{align}
\sum_{m \in \mZ} \int_{\!-1}^1 \frac{ \abs{ f( \frac{ m - \re{z} }{\omega}, \xi - 2( m - \re{z} )\im{z} ) } }{ \abs{m^2 - \re{z}^2 + \rmi\xi} } \,\rmd\xi
&\leq \sum_{m \in \mZ} \int_{\!-1}^1 \frac{ \norm{f}_{ L^\infty( \Scr{C} ) } }{ \abs{m^2 - \re{z}^2 + \rmi\xi} } \,\rmd\xi \nonumber\\
&< \norm{f}_{ L^\infty( \Scr{C} ) } \frac{8}{\omega^2}\sum_{m = 1}^\infty \frac{1}{m^2},
\end{align}
and
\begin{align}
\sum_{m \in \mZ} \int_{\abs{\xi} \geq 1}
\frac{ \abs{ f( \frac{ m - \re{z} }{\omega}, \xi - 2( m - \re{z} )\im{z} ) } }{ \abs{m^2 - \re{z}^2 + \rmi\xi} } \,\rmd\xi
&\leq \Bigg(\sum_{m \in \mZ} \int_{\abs{\xi} \geq 1}
\frac{\norm{f}_{ L^2( \Scr{C} ) }^2}{\abs{m^2 - \re{z}^2 + \rmi\xi}^2} \,\rmd\xi\Bigg)^\frac{1}{2} \nonumber\\
&< \norm{f}_{ L^2( \Scr{C} ) } \Bigg( \frac{2\pi}{\omega^2}\sum_{m = 1}^\infty \frac{1}{m^2} \Bigg)^\frac{1}{2}.
\end{align}
The combination of these two inequalities yields inequality \eqref{Eq:basiclone} which concludes the proof.
\end{proof}
It is now possible to prove the basic theorem concerning the existence and uniqueness of the solution of equation \eqref{Eq:bvp_a}.
\begin{theorem}\label{Th:exist_unique}
Suppose the function $u(x, y)$  belongs to both $L^1(\Omega)$ and $L^2(\Omega)$ and is small in the sense that
\begin{equation}\label{Eq:smallnorm}
\max\{\omega\norm{u}_1, \sqrt{\omega}\norm{u}_2\} < \frac{2\pi}{C},
\end{equation}
where the constant $C$ is defined by~\eqref{Eq:constant}. Then, there is a unique, bounded solution $\mu(x, y; z)$ to the boundary-value problem
\begin{subequations}\label{Eq:BVP}\begin{align}
&(-\partial_y + \partial_x^2 +2\rmi z\partial_x)\mu + u\mu = 0, \label{Eq:BVP_a}\\
&\text{ for each } z \in \wpln, \ \lim_{\abs{y} \to \infty} \mu(x, y; z) = 1,
\end{align}\end{subequations}
such that $\wh{\mu - 1} \in L^1( \Scr{C} )$, for every $z \in \wpln$.
\end{theorem}
\begin{proof}
Consider the map $f \mapsto (\wh{u} \ast f)/2\pi P_z$. This map is bounded from $L^1( \Scr{C} )$ to $L^1( \Scr{C} )$, uniformly in $z \in \wpln$, and has 
norm less than one. Indeed, since $u \in L^1(\Omega) \cap L^2(\Omega)$, it follows that $\wh{u} \in L^2( \Scr{C} ) \cap L^\infty( \Scr{C} )$. Thus,
if $f \in L^1( \Scr{C} )$, then
\[
\norm{\wh{u} \ast f}_{ L^2( \Scr{C} ) } \leq \norm{ \wh{u} }_{ L^2( \Scr{C} ) } \norm{f}_{ L^1( \Scr{C} ) } =
\sqrt{\omega}\norm{u}_2 \norm{f}_{ L^1 ( \Scr{C} ) },
\]
and
\[
\norm{\wh{u} \ast f}_{ L^\infty ( \Scr{C} ) } \leq \norm{ \wh{u} }_{ L^\infty ( \Scr{C} ) } \norm{f}_{ L^1 ( \Scr{C} ) } <
\omega\norm{u}_1 \norm{f}_{ L^1 ( \Scr{C} ) }.
\]
Furthermore, $(\wh{u} \ast f)(0, \xi) = 0$ for every $\xi \in \bb{R}$ because $\wh{u}(0, \xi) = 0$, as a consequence of the zero mass constraint.
Hence, the function $\wh{u} \ast f$ satisfies the assumptions of the \hyperlink{L:basiclemma}{Basic Lemma} thus,
\begin{align*}
\Norm{ \frac{\wh{u} \ast f}{2\pi P_z} }_{ L^1( \Scr{C} ) } &\leq
\frac{C}{2\pi}\max\{ \norm{\wh{u} \ast f}_{ L^2( \Scr{C} ) }, \norm{\wh{u} \ast f}_{ L^\infty( \Scr{C} ) } \}\\
&< \frac{C}{2\pi}\max\{\sqrt{\omega}\norm{u}_2, \,\omega\norm{u}_1\}\norm{f}_{ L^1( \Scr{C} ) },
\end{align*}
uniformly in $z \in \wpln$. By assumption, $\frac{C}{2\pi}\max \{\omega\norm{u}_1, \sqrt{\omega}\norm{u}_2\} < 1$.

Applying Banach's fixed-point theorem in $L^1( \Scr{C} )$, equation \eqref{Eq:contractionform} has a unique solution $(\wh{\mu - 1})(m, \xi; z)$ for each 
$z \in \wpln$. Its inverse Fourier transform $\mu(x, y; z)$ solves equation \eqref{Eq:BVP_a}. Furthermore, $\mu \in L^\infty(E)$: for every $z \in \wpln$
\[
\abs{ (\mu - 1)(x, y; z) } = \big\lvert [ \wh{\mu - 1} ]\spcheck\!(x, y; z) \big\rvert \leq
\frac{1}{2\pi}\big\lVert (\wh{\mu - 1})(\cdot, \cdot; z) \big\rVert_{ L^1( \Scr{C} ) }.
\]
But
\begin{align*}
\big\lVert (\wh{\mu - 1})(\cdot, \cdot; z) \big\rVert_{ L^1( \Scr{C} ) } &\leq \Norm{ \frac{ \wh{u} }{P_z} }_{ L^1( \Scr{C} ) }
+ \Norm{ \frac{ \wh{u} \ast ( \wh{\mu - 1} )(\cdot, \cdot; z) }{2\pi P_z} }_{ L^1( \Scr{C} ) }\\
&< C\maxnorm{u} + \frac{C}{2\pi}\maxnorm{u} \big\lVert ( \wh{\mu - 1} )(\cdot, \cdot; z) \big\rVert_{ L^1( \Scr{C} ) },
\end{align*}
and so
\begin{equation}\label{Eq:lonenorm}
\big\lVert ( \wh{\mu - 1} )(\cdot, \cdot; z) \big\rVert_{ L^1( \Scr{C} ) } < \frac{ C\maxnorm{u} }{ 1 - \frac{1}{2\pi}C\maxnorm{u} },
\end{equation}
where we set $\maxnorm{\cdot} \equiv \max\{\omega\norm{\cdot}_1, \sqrt{\omega}\norm{\cdot}_2\}$.

Equation \eqref{Eq:contractionform} can be written in the form
\begin{equation}\label{Eq:eigenfunction}
\mu(x, y; z) = 1 + \frac{1}{2\pi}\sideset{}{'}\sum_{m = -\infty}^\infty \int_{-\infty}^\infty
\frac{ \wh{u\mu}(m, \xi; z) }{ P_z(m, \xi) }\rme^{\rmi\omega m x + \rmi\xi y} \,\rmd\xi,
\end{equation}
for $(x, y) \in \Omega$, $z \in \wpln$. Since
\begin{equation}
\frac{ \wh{u\mu}(m, \xi; z) }{ P_z(m, \xi) } \in L^1( \Scr{C} ),
\end{equation}
the Riemann--Lebesgue lemma implies that $\mu(x, y; z) \to 1$ as $\abs{y} \to \infty$ for each $z \in \wpln$.
\end{proof}
More precise asymptotics of $\mu(x, y; z)$ as $\abs{y} \to \infty$ is given by the following proposition.
\begin{proposition}\label{Pr:asymptotics}
Suppose $\mu(x, y; z)$ is a solution to equation \eqref{Eq:eigenfunction} in $L^\infty(E)$ and that
\begin{equation}\label{Eq:decaycondition}
( 1 + \abs{y} )\abs{ u(x, y) } \in L^1(\Omega) \cap L^2(\Omega).
\end{equation}
Then,
\begin{equation}
\mu(x, y; z) = 1 + o\bigg( \frac{1}{ \abs{y} } \bigg), \text { as } \abs{y} \to \infty.
\end{equation}
\end{proposition}
\begin{proof}
To see why this is true, let us calculate $\partial_\xi( \wh{u\mu}(m, \xi; z)/P_z(m, \xi) )$:
\begin{align*}
\partial_\xi\bigg( \frac{ \wh{u\mu}(m, \xi; z) }{ P_z(m, \xi) } \bigg) &=
\frac{ \partial_\xi( \wh{u\mu}(m, \xi; z) )P_z(m, \xi) - \rmi\wh{u\mu}(m, \xi; z) }{ {P_z}^2(m, \xi) }\\
&= \frac{ \partial_\xi\wh{u\mu}(m, \xi; z) }{ P_z(m, \xi) } - \rmi\frac{ \wh{u\mu}(m, \xi; z) }{ {P_z}^2(m, \xi) }\\
&= \frac{ [-\rmi y u\mu]\mh(m, \xi; z) }{ P_z(m, \xi) } - \rmi\frac{ \wh{u\mu}(m, \xi; z) }{ {P_z}^2(m, \xi) }.
\end{align*}
Since $\mu$ is bounded, \eqref{Eq:decaycondition} guarantees that $[-\rmi y u\mu]\mh$ belongs to $L^2( \Scr{C} )$ and $L^\infty( \Scr{C} )$. Hence,
by the \hyperlink{L:basiclemma}{Basic Lemma}, $[-\rmi y u\mu]\mh/P_z \in L^1( \Scr{C} )$. Now a simple calculation shows that
\[
\sideset{}{'}\sum_{m = -\infty}^\infty \int_{\!-\infty}^\infty \frac{1}{ {P_z}^2(m, \xi) } \,\rmd\xi =
\frac{\pi}{2}\sum_{m \in \mZ} \frac{1}{ \abs{m^2 - \re{z}^2} },
\]
and the sum in the right hand side converges. By the boundedness of $\mu$, it follows that $\wh{u\mu} \in L^2( \Scr{C} ) \cap L^\infty( \Scr{C} )$.
Thus, $\wh{u\mu}/{P_z}^2$ is also a member of $L^1( \Scr{C} )$. Therefore, by the Riemann--Lebesgue lemma
\[
\bigg[ \partial_\xi\bigg( \frac{ \wh{u\mu}(m, \xi; z) }{ P_z(m, \xi) } \bigg) \bigg]\spcheck\!(x, y; z) \to 0, \text{ as } \abs{y} \to \infty,
\]
and the result follows from the identity
\[
\bigg[ \partial_\xi\bigg( \frac{ \wh{u\mu}(m, \xi; z) }{ P_z(m, \xi) } \bigg) \bigg]\spcheck =
-\rmi y\bigg[ \frac{ \wh{u\mu}(m, \xi; z) }{ P_z(m, \xi) } \bigg]\spcheck.\qedhere
\]
\end{proof}
The eigenfunction $\mu$ has several regularity properties. To derive them we will use equation \eqref{Eq:eigenfunction}. An immediate result is that
$\mu(x, y; z)$ belongs to $C(\Omega)$ for every $z \in \wpln$.
\begin{proposition}\label{Pr:continuity}
Suppose $\mu(x, y; z)$ is a solution to equation \eqref{Eq:eigenfunction}. Then, for every $z \in \wpln$, $\mu$ is continuous in $y$ for all
$x \in [-\ell, \ell\,]$ and continuous in $x$ for all $y \in \bb{R}$.
\end{proposition}
\begin{proof}
Let $x \in [-\ell, \ell\,]$ and $ \{y_n\}_{n = 1}^\infty$ be a sequence of real numbers, converging to $y_0$. Define the sequence of functions
$\{f_n\}_{n = 1}^\infty$,
\[
f_n(x, m, \xi; z) = \frac{ \wh{u\mu}(m, \xi; z) }{ P_z(m, \xi) }\rme^{\rmi\omega m x + \rmi\xi y_n}.
\]
For every $n \in \bb{N}$ the function $\rme^{\rmi\omega m x + \rmi\xi y_n}$ is continuous, hence measurable. Thus, the functions $f_n(x, m, \xi; z)$, 
being the product of measurable functions, are measurable for all $n$. From the continuity of the exponential function, it follows that the sequence
$\{f_n\}$ converges pointwise to the function
\[
\frac{ \wh{u\mu}(m, \xi; z) }{ P_z(m, \xi) }\rme^{\rmi\omega m x + \rmi\xi y_0}.
\]
Moreover,
\[
\abs{ f_n(x, m, \xi; z) } = \Abs{ \frac{ \wh{u\mu}(m, \xi; z) }{ P_z(m, \xi) } }.
\]
But since the function $\abs{\wh{u\mu}/P_z}$ belongs to $L^1( \Scr{C} )$, an application of Lebesgue's dominated convergence theorem yields the 
following:
\begin{align*}
\lim_{n \to \infty} \mu(x, y_n; z) &= 1 + \frac{1}{2\pi}\lim_{n \to \infty} \sideset{}{'}\sum_{m = -\infty}^\infty \int_{-\infty}^\infty
f_n(x, m, \xi; z) \,\rmd\xi\\
&= 1 + \frac{1}{2\pi}\sideset{}{'}\sum_{m = -\infty}^\infty \int_{-\infty}^\infty \lim_{n \to \infty} f_n(x, m, \xi; z) \,\rmd\xi\\
&= 1 + \frac{1}{2\pi}\sideset{}{'}\sum_{m = -\infty}^\infty \int_{-\infty}^\infty
\frac{ \wh{u\mu}(m, \xi; z) }{ P_z(m, \xi) }\rme^{\rmi\omega m x + \rmi\xi y_0} \,\rmd\xi\\
&= \mu(x, y_0; z).
\end{align*}

Using similar arguments, we can easily see that $\mu$ is also continuous with respect to $x$ for every $y \in \bb{R}$, hence the result follows.
\end{proof}
The following proposition provides us with an alternative way of writing equation \eqref{Eq:eigenfunction}, which will proven to be quite useful, in particular 
proving analytic properties of $\mu$ with respect to $z$.
\begin{proposition}\label{Pr:Neumannseries}
Suppose $u(x, y)$ satisfies condition \eqref{Eq:smallnorm}. Then, $\mu$ admits the representation in Neumann series
\begin{equation}\label{Eq:Neumannseries}
\mu(x, y; z) = \sum_{n = 0}^\infty (\Scr{N}_u^n 1)(x, y; z),
\end{equation}
where the operator $\Scr{N}_u$ is defined for every function $h \in L^\infty(\Omega)$ by
\begin{equation}\label{Eq:Neumann}
(\Scr{N}_u h)(x, y; z) \= \frac{1}{2\pi}\sideset{}{'}\sum_{m = -\infty}^\infty \int_{-\infty}^\infty
\frac{ \wh{u h}(m, \xi) }{ P_z(m, \xi) }\rme^{\rmi\omega m x + \rmi\xi y} \,\rmd\xi.
\end{equation}
\end{proposition}
\begin{proof}
Let $h \in  L^\infty(\Omega)$. Then, $u h \in L^1(\Omega) \cap L^2(\Omega)$, since $u \in L^1(\Omega) \cap L^2(\Omega)$, hence
$\wh{u h} \in L^2( \Scr{C} ) \cap L^\infty( \Scr{C} )$. Thus, from the \hyperlink{L:basiclemma}{Basic Lemma},
\[
\sideset{}{'}\sum_{m = -\infty}^\infty \int_{-\infty}^\infty \frac{ \wh{u h}(m, \xi) }{ P_z(m, \xi) } \,\rmd\xi \leq
C\max\{ \norm{ \wh{u h} }_{ L^2( \Scr{C} ) }, \norm{ \wh{u h} }_{ L^\infty( \Scr{C} ) } \},
\]
uniformly in $z \in \wpln$. This yields,
\begin{align*}
\abs{ (\Scr{N}_u h)(x, y; z) } &\leq \frac{1}{2\pi}C\max\{ \norm{ \wh{u h} }_{ L^2( \Scr{C} ) }, \norm{ \wh{u h} }_{ L^\infty( \Scr{C} ) } \}\\
&\leq \frac{1}{2\pi}C\max\{\omega\norm{u}_1, \sqrt{\omega}\norm{u}_2\}\norm{h}_\infty.
\end{align*}
Therefore, for all $z \in \wpln$, the operator $\Scr{N}_u \colon L^\infty(\Omega) \to L^\infty(\Omega)$ is bounded with norm less than one. Write 
equation \eqref{Eq:eigenfunction} as $\mu = 1 + \Scr{N}_u \mu$. Then, $(\op{Id} - \Scr{N}_u)\mu = 1$, and since $\norm{\Scr{N}_u}_\mrm{op} < 1$
we are allowed to write
\begin{equation}
\mu = (\op{Id} - \Scr{N}_u)^{-1} 1 = \sum_{n = 0}^\infty \Scr{N}_u^n 1.\qedhere
\end{equation}
\end{proof}
\begin{remark}
The Neumann series \eqref{Eq:Neumannseries} converges uniformly for $z \in \wpln$. This can be deduced by an application of the Weierstrass M-test:
for $n \in \bb{N}$,
\begin{equation}\label{Eq:Mtest}
\abs{ (\Scr{N}_u^n 1)(x, y; z) } \leq \bigg( \frac{1}{2\pi}C\maxnorm{u} \bigg)^n,
\end{equation}
while the geometric series
\begin{equation}
\sum_{n = 0}^\infty \bigg( \frac{1}{2\pi}C\maxnorm{u} \bigg)^n,
\end{equation}
converges since $C\maxnorm{u}/2\pi < 1$.
\end{remark}
A consequence of the representation \eqref{Eq:Neumannseries}, is that $\mu(x, y; z)$ is a holomorphic function with respect to $z \in \wpln$.
\begin{theorem}\label{Th:holomorphicity}
Suppose $u(x, y)$ belongs to $L^1(\Omega) \cap L^2(\Omega)$ such that $\maxnorm{u}$ is small. Then, for every $(x, y) \in \Omega$,
$\mu(x, y; \cdot) \in H(\wpln)$.
\end{theorem}
\begin{proof}
We will show that each function $(\Scr{N}_u^n 1)(x, y; z)$ in the Neumann series \eqref{Eq:Neumannseries} of $\mu(x, y; z)$ is a holomorphic function 
with respect to $z \in \wpln$. Since the series converges uniformly in $\wpln$, it defines a holomorphic function there.

We will use induction. Let $z_0 \in \wpln$. For $(m, \xi) \in \Scr{C}^*$, the function $1/P_z(m, \xi)$ is holomorphic in $z \in \wpln$. By
the \hyperlink{L:basiclemma}{Basic Lemma},
\[
\sideset{}{'}\sum_{ m \in \bb{Z} } \int_{\!-\infty}^\infty \frac{ \wh{u}(m, \xi) }{ P_z(m, \xi) }\rme^{\rmi\omega m x + \rmi\xi y} \,\rmd\xi
\]
converges absolutely and uniformly in $\wpln$. Hence, dominated convergence yields
\begin{align*}
\lim_{z \to z_0} (\Scr{N}_u 1)(x, y; z) &= \frac{1}{2\pi}\sideset{}{'}\sum_{ m \in \bb{Z} } \int_{\!-\infty}^\infty \lim_{z \to z_0}
\frac{ \wh{u}(m, \xi) }{ P_z(m, \xi) }\rme^{\rmi\omega m x + \rmi\xi y} \,\rmd\xi\\
&= \frac{1}{2\pi}\sideset{}{'}\sum_{ m \in \bb{Z} } \int_{\!-\infty}^\infty
\frac{ \wh{u}(m, \xi) }{ P_{z_0}(m, \xi) }\rme^{\rmi\omega m x + \rmi\xi y} \,\rmd\xi,
\end{align*}
which shows that $\Scr{N}_u 1$ is continuous at $z_0$ and therefore, continuous in $\wpln$. Now let $T$ be a triangle in $\wpln$. By Fubini's theorem,
\[
\int_T (\Scr{N}_u 1)(x, y; z) \,\rmd z = \frac{1}{2\pi}\sideset{}{'}\sum_{ m \in \bb{Z} } \int_{\!-\infty}^\infty \wh{u}(m, \xi)
\bigg(\int_T \frac{1}{ P_z(m, \xi) } \,\rmd z\bigg)\rme^{\rmi\omega m x + \rmi\xi y} \,\rmd\xi = 0.
\]
Applying Morera's theorem, we conclude that $(\Scr{N}_u 1)(x, y; \cdot) \in H(\wpln)$.

Suppose now that $(\Scr{N}_u^k 1)(x, y; \cdot) \in H(\wpln)$ for all $k < n$. Since $\Scr{N}_u^{n -1} 1$ is bounded for every $z \in \wpln$, and $u$ 
belongs to $L^1(\Omega) \cap L^2(\Omega)$, it follows that for every $z \in \wpln$,
$[u\Scr{N}_u^{n - 1} 1]\mh \in L^2( \Scr{C} ) \cap L^\infty( \Scr{C} )$. Therefore, \hyperlink{L:basiclemma}{Basic Lemma} implies that
\[
\sideset{}{'}\sum_{ m \in \bb{Z} } \int_{\!-\infty}^\infty
\frac{ [u\Scr{N}_u^{n - 1} 1]\mh(m, \xi; z) }{ P_z(m, \xi) }\rme^{\rmi\omega m x + \rmi\xi y} \,\rmd\xi
\]
converges absolutely and uniformly for all $z \in \wpln$. For each $(m, \xi) \in \Scr{C}^*$, the function $[u\Scr{N}_u^{n - 1} 1]\mh(m, \xi; z)$ is
holomorphic in $z \in \wpln$: by definition
\[
[u\Scr{N}_u^{n - 1} 1]\mh(m, \xi; z) = \frac{1}{2\ell}\int_{\!-\infty}^\infty \int_{\!-\ell}^\ell
u(x, y)(\Scr{N}_u^{n - 1} 1)(x, y; z)\rme^{-\rmi\omega m x - \rmi\xi y} \,\rmd x\rmd y.
\]
By the induction hypothesis, $\Scr{N}_u^{n - 1} 1$ is holomorphic. Also, as a consequence of inequality \eqref{Eq:Mtest}, for every $z \in \wpln$
\[
\int_{\!-\infty}^\infty \int_{\!-\ell}^\ell \abs{ u(x, y)(\Scr{N}_u^{n - 1} 1)(x, y; z) } \,\rmd x\rmd y \leq
\Big( \frac{ C\maxnorm{u} }{2\pi} \Big)^{n - 1} \norm{u}_1 < \infty.
\]
Thus, from the continuity of $\Scr{N}_u^{n - 1} 1$ and dominated convergence, we get that\\$[u\Scr{N}_u^{n - 1} 1]\mh$ is continuous and, changing the 
order of integration via Fubini's theorem,
\[
\int_T [u\Scr{N}_u^{n - 1} 1]\mh(m, \xi; z) \,\rmd z = 0.
\]

But now, continuity of the function $[u\Scr{N}_u^{n - 1} 1]\mh(m, \xi; z)/P_z(m, \xi)$ and dominated convergence implies the continuity of
$(\Scr{N}_u^n 1)(x, y; z)$ in $\wpln$ and as the function\\
$[u\Scr{N}_u^{n - 1} 1]\mh(m, \xi; z)/P_z(m, \xi)$ is holomorphic in $z \in \wpln$,
yet another application of Fubini's theorem gives
\begin{align*}
\int_T (\Scr{N}^n_u 1)(x, y; z) \,\rmd z &= \int_T \Bigg(\frac{1}{2\pi}\sideset{}{'}\sum_{ m \in \bb{Z} } \int_{\!-\infty}^\infty
\frac{ [u\Scr{N}_u^{n - 1} 1]\mh(m, \xi; z) }{ P_z(m, \xi) }\rme^{\rmi\omega m x + \rmi\xi y} \,\rmd\xi\Bigg) \rmd z\\
&= \frac{1}{2\pi}\sideset{}{'}\sum_{ m \in \bb{Z} } \int_{\!-\infty}^\infty
\bigg(\int_T \frac{ [u\Scr{N}_u^{n - 1} 1]\mh(m, \xi; z) }{ P_z(m, \xi) } \,\rmd z\bigg)\rme^{\rmi\omega m x + \rmi\xi y} \,\rmd\xi\\
&= 0,
\end{align*}
for every triangle $T$ in $\wpln$. A final application of Morera's theorem, shows that $(\Scr{N}_u^n 1)(x, y; \cdot) \in H(\wpln)$, thus,
the theorem is proved.
\end{proof}