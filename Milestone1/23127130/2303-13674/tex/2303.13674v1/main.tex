%\documentclass[11pt]{article}
%\documentclass[reprint, amsmath,amssymb,longbibliography]{revtex4-1}
%\usepackage[dvips,letterpaper,margin=0.8in,bottom=1in]{geometry}
\documentclass[twocolumn]{revtex4-1}
\usepackage{amsmath}
\usepackage{amssymb}
\usepackage{graphicx}
\usepackage[normalem]{ulem}
\usepackage{float}
\usepackage{comment}
\usepackage{xcolor}
\usepackage{tikz}
\usepackage{lipsum}
\usepackage{hyperref}
\usepackage[utf8]{inputenc}
\usepackage[english]{babel}
%\usepackage{cuted}
\usepackage{cancel}

\setcounter{section}{0}
\pagenumbering{gobble}

\newcommand{\citeasnoun}[1]{Ref.~\cite{#1}}
\newcommand{\figref}[1]{Fig.~\ref{fig:#1}}
\newcommand{\figsref}[2]{Figs.~\ref{fig:#1}~and~\ref{fig:#2}}
\newcommand{\figrefbegin}[1]{Figure~\ref{fig:#1}}
\newcommand{\secref}[1]{Sec.~\ref{#1}}
\renewcommand{\eqref}[1]{Eq.~(\ref{eq:#1})}
%\renewcommand{\eqref}[1]{(\ref{eq:#1})}
%\renewcommand{\figurename}{Fig.}
\addto\captionsenglish{\renewcommand{\figurename}{Fig.}}
\newcommand{\eqrefbegin}[1]{Equation~(\ref{eq:#1})}
\newcommand{\eqsrefbegin}[2]{Equations~(\ref{eq:#1}) and~(\ref{eq:#2})}
\newcommand{\eqssrefbegin}[2]{Equations~(\ref{eq:#1}--\ref{eq:#2})}
\newcommand{\eqsref}[2]{Eqs.~(\ref{eq:#1},\ref{eq:#2})}
\newcommand{\eqssref}[2]{Eqs.~(\ref{eq:#1}--\ref{eq:#2})}
\renewcommand{\vec}[1]{\mathbf{#1}}
\newcommand{\vecg}[1]{\boldsymbol{#1}}
\newcommand{\mat}[1]{\mathbb{#1}}
\renewcommand{\appendixname}{ }
\newcommand{\appendixsec}[2]{\section{\uppercase{#1}}\label{#2}}
\newcommand{\appendixsubsec}[2]{\subsection{\uppercase{#1}}\label{#2}}
\newcommand{\BRA}[1]{\left<{#1}\right.\!|}
\newcommand{\KET}[1]{|\!\left.{#1}\right>}
\newcommand{\BRACKET}[2]{\left<{#1}|{#2}\right>}
\newcommand{\Tr}[0]{\mbox{Tr}\hspace{2pt}}
\newcommand{\RE}[1]{\mbox{Re}\left[{#1}\right]}
\newcommand{\IM}[1]{\mbox{Im}\left[{#1}\right]}
\renewcommand{\abstractname}{}    % clear the title
\newcommand{\absnamepos}{empty} % originally center
\newcommand{\Scale}[2][4]{\scalebox{#1}{$#2$}}


\newcommand{\tb}{\textcolor{blue}}
\newcommand{\trr}{\textcolor{red}}
\newcommand{\tg}{\textcolor{green}}
\newcommand{\tcyan}{\textcolor{cyan}}



\begin{document}


\title{Inertial geometric quantum logic gates}
\author{D. Turyansky$^1$}
\author{O. Ovdat$^2$}
\author{R. Dann$^3$}
\author{Z. Aqua$^2$}
\author{R. Kosloff$^3$}
\author{B. Dayan$^2$}
\author{A. Pick$^1$}


\affiliation{\vspace*{0.1in}$^1$Department of Applied Physics, The Hebrew University of Jerusalem, Jerusalem 9190401, Israel}
\affiliation{$^2$Department of Chemical Physics, Weizmann Institute of Science, Rehovot
76100, Israel}
\affiliation{$^3$The Institute of Chemistry, The Hebrew University of Jerusalem, Jerusalem 9190401, Israel}






%----------------------------------------------------------------------
\begin{abstract}
We present rapid and robust protocols for single- and two-qubit quantum logic gates.  Our gates are based on geometric phases acquired by instantaneous eigenstates of a \emph{slowly accelerating} ``inertial'' Hamiltonian. We begin by defining conditions for  an  inertial population transfer protocol and, then, find pulse shapes that meet those  conditions. We use those pulses  to perform inertial quantum logic gates and  optimize their performance using quantum optimal control. By including  adiabaticity and  inertiality conditions in the optimization process, we show that inertial protocols can have   reduced pulse energy for given performance merits.   Finally, we analyze  an implementation of our protocol  with  $^{87}$Rb atoms including  polarization and leakage  errors. Our approach  extends beyond geometric gates and is useful for speeding up   adiabatic  quantum computation protocols. 
\end{abstract}
%----------------------------------------------------------------------


\maketitle
%----------------------------------------------------------------------


Rapid high-fidelity quantum logic gates  are essential   for   large-scale atom-based   quantum computers~\cite{ladd2010quantum,saffman2016quantum,henriet2020quantum}. State-of-the-art experiments   can reach    fidelities of  99$\%$  and 95$\%$ for single- and two-qubit gates respectively~\cite{henriet2020quantum,morgado2021quantum,zhang2020submicrosecond}, getting closer but still not reaching  the required  threshold values for fault-tolerant quantum  computation~\cite{aharonov1997fault}. Pushing these abilities further is a key goal in quantum computation research.   Atomic qubits are typically encoded  in  long-lived hyperfine states and  transitions between them are driven indirectly  via    excited  states. Physical processes that limit gate fidelity   include depolarization and dephasing,  induced by  spontaneous emission from  the excited   states,  and  imperfections in the control fields that implement the gate. An elegant approach to     combat these issues  involves  adiabatic following of optically dark states. By slowly varying the control fields, the system adheres to a ``dark combination'' of the long lived states, which  is decoupled from radiation and is, therefore, protected from incoherent  emission. 
This is the key principle behind  stimulated Raman adiabatic passage (STIRAP)~\cite{vitanov2017stimulated} and geometric quantum logic gates~\cite{moller2007geometric}.
The outcome of such protocols depends only on  the geometric phase acquired by the dark eigenstate and  on the final state of the system. Hence, it is robust to  certain perturbations (i.e., those  that do not affect the geometric phase or  final state).  For the system to adhere to the dark state, the protocol must be sufficiently slow, limiting the computational speed.




% ------------
% FIGURE 0
% ------------
\begin{figure}[b]
  \includegraphics[width=0.5\textwidth]{animation.pdf}
  \caption{ \textbf{Inertial motion versus transitionless driving.} (a--c) Transitionless driving (borrowed  from~\cite{sels2017minimizing}):  A moving waiter carrying a cup  may tilt the tray to create a counter-diabatic force and  prevent spilling. (d) Inertial motion: In the frame of the cup, the water remains still as long as the acceleration is kept small.}
   \label{fig:Fig0}
\end{figure}
% ----------------

Many efforts  have focused on speeding up adiabatic schemes and increasing their fidelity  while maintaining their robustness~\cite{vitanov2017stimulated}.   A  seminal paper by Berry~\cite{berry2009transitionless} showed  that adiabatic approximations can be made exact and rapid  by  introducing appropriate fields  that counteract diabatic transitions, resulting in ``transitionless driving.'' This idea is  illustrated in \figref{Fig0}(a--c) (borrowed from~\cite{sels2017minimizing}), where a  cup of a running waiter is being tilted to avoid spilling.  An application of this approach to STIRAP successfully produced a high-fidelity rapid and robust population transfer protocol~\cite{unanyan1997laser,demirplak2003adiabatic,chen2010shortcut,torrontegui2013shortcuts,del2013shortcuts,guery2019shortcuts}.   Recent work shows how to achieve transitionless driving with weaker counter-diabatic terms that easier to construct~\cite{benseny2021adiabatic}. Although significantly improving  conventional STIRAP~\cite{du2016experimental}, 
errors in the  implementation of the counter-diabatic Hamiltoinan terms translate into  errors in the   eigenstate and, therefore,  limit the gate  fidelity and  speed.
Alternatively, one can  try to improve adiabatic protocols by  optimizing the parameters of analytic   adiabatic pulses~\cite{vasilev2009optimum,chen2012engineering}.  For example, in Ref.~\cite{vasilev2009optimum},   optimized    pulses were used to  reduce diabatic errors below the threshold required for   fault tolerant quantum computation (using an idealized theoretical model system).  Despite its simplicity, this approach is limited in its ability to improve  performance merits in a  general setting.  To push these results further, one needs to resort to brute-force optimization methods, such as  quantum optimal control (QOCT)~\cite{malinovsky1997simple,werschnik2007quantum,koch2022quantum}. It remains  a challenge to find optimal solutions that are  practical and globally optimal.  In this work, we  use the recent ``inertial theorem''~\cite{dann2021inertial,hu2021experimental} to find optimal STIRAP pulses with simple  pulse shapes.  By transforming the Hamiltonian into an appropriate ``inertial frame''  and rescaling the time coordinate, we find solutions that are adiabatic in the new frame  although   may vary rapidly in the original frame. This concept is  illustrated in \figref{Fig0}(d), where   the cup  does not spill as long as the waiter's acceleration is small.  We use   inertial solutions as a  guess solution for QOCT and impose   adiabaticity and inertiality constraints in the optimization algorithm. 



We use the  optimized  protocols  to improve   quantum logic gates that  utilize geometric phases acquired during cyclic evolution~\cite{unanyan1999laser,duan2001geometric}. Specifically,  these protocols use tripod-level atoms, as  proposed in~\cite{moller2007geometric} and demonstrated   experimentally in~\cite{toyoda2013realization}. Our approach is generic and can be used to improve additional adiabatic  protocols.  For example,  a recent proposal associates  SU(2) rotations with  fixed  ratios between the pump and Stokes STIRAP fields~\cite{lacour2006arbitrary}; The   gate fidelity is limited by diabatic transitions and, hence, can be  improved by our approach.  Another example is a  STIRAP motivated approach  for   Grover’s search algorithm, which finds a marked item in an unsorted list~\cite{daems2008analog}. Finally, adiabatic protocols are used in quantum simulated annealing, where a system is adiabatically brought into  a final state that solves a complex  computational problem. Efforts have concentrated on speeding up  quantum annealing,  for example, by using a variational transitionless-driving approach~\cite{sels2017minimizing}. Based on our present work, inertial solutions are expected to be useful for improving such protocols. %Before   introducing the inertial STIRAP solution, let us  first review the inertial theorem.




\emph{Inertial evolution:} 
The inertial theorem utilizes a timescale separation between variables to obtain approximate  solutions for the system's dynamics under rapid external driving~\cite{dann2021inertial}. The theorem is derived  in   Lioville space -- the vector space of operators on the Hilbert space. Starting with the Heisenberg equations of motion for the system operators, one identifies a suitable time-dependent operator basis  for which the equations of motion vary on two timescales.  By rescaling the generalized time coordinates,  one can eliminate the fast dynamics and, then, identify new  dynamical symmetries -- operators whose Heisenberg representation is independent of time. These symmetries enable the construction of  approximate analytic solutions. For the system considered in this paper, the construction in Liouville space inspires an equivalent formulation in Hilbert space.






Let $\KET{\psi}$ satisfy the Schr\"{o}dinger equation, $i\hbar\partial\psi/\partial t = H(t)\KET{\psi}$, and introduce the eigenvalue decomposition  of the Hamiltonian, 
\begin{equation}
H(t) =P\Lambda P^{-1},
\end{equation}
where $P$ and $\Lambda$ are  instantaneous matrices of eigenvectors and eigenvalues respectively. Let us use the eigenvector matrix  to define the state  $|\tilde{\psi}\rangle=P\KET{\psi}$, which satisfies
\begin{equation}
i\hbar\frac{\partial \,|\tilde{\psi}\rangle}{\partial t} = 
\left(
P^\dagger H P - i\hbar P^\dagger \frac{\partial P}{\partial t}
\right)|\tilde{\psi}\rangle\equiv \tilde{H}|\tilde{\psi}\rangle.
\label{eq:inertial-H0}
\end{equation}
(Note that \eqref{inertial-H0} is valid for  any matrix $P$, not necessarily the eigenvector matrix.) Inertial frames are those in which $\tilde{H}$ can be written in the form 
\begin{equation}
\tilde{H} = \Omega(t) \mat{M}(\chi),
\label{eq:factor-H}
\end{equation}
where \emph{$\Omega(t)$ is a possibly rapidly varying scalar while $\chi$ is nearly stationary}. Introducing a rescaled time coordinate, $\tau = \int_0^t \Omega(t')dt'$,  we obtain the inertial frame Schr\"{o}dinger equation
\begin{equation}
i\hbar\frac{\partial |\tilde{\psi}\rangle}{\partial\tau} = \mat{M}(\chi) |\tilde{\psi}\rangle.
\end{equation}
The solutions, $|\tilde{\psi}(\tau)\rangle$, 
can be approximated by adiabatically following the  eigenstates of $\mat{M}$. The validity condition is that $\mat{M}$ is slowly varying compared to its energy gap. We will see that for    STIRAP,  this condition  amounts to constraining the second-order temporal derivative of the control parameters (i.e.,  acceleration), versus the  adiabatic condition constraining the velocity. 


% ------------
% FIGURE 1 
% ------------
\begin{figure}
  \includegraphics[width=0.5\textwidth]{Fig1}
  \caption{ \textbf{Inertial and conventional STIRAP.} (a-b) Population is transferred  from  $\KET{1}$ to $\KET{2}$ by applying the pulse  $\Omega_{2}$ and then $\Omega_{1}$. (c) Three pulse shapes are analyzed: Gaussian  (solid green), sine squared (dashed blue) and inertial (dash-dot red). Pulses are parameterized by the plotted angle, $\theta(t)\equiv\tan^{-1}(\Omega_1/\Omega_2)$.  (d) Infidelity as a function of protocol duration for the three pulse shapes.   Parameters: $\Delta = 50 \mbox{MHz},$ maximal  intensity $\Omega = 50\mbox{MHz}$ and $\gamma/2\pi = 6\, \mbox{MHz}$. Optimized results shown by triangles. }
   \label{fig:Fig1}
\end{figure}
% ----------------


\emph{STIRAP: Basic theory and the inertial 
protocol.} 
 Consider a three-level system  driven by two fields as shown in \figref{Fig1}(a). The goal of the protocol is to transfer population between the ground states, from $\KET{1}$ to $\KET{2}$, without populating the third state, $\KET{3}$. Assuming that the fields are both detuned from the atomic transition  by $\Delta$,  the Hamiltonian of this system is
\begin{gather}
H = \Omega_1\KET{1}\!\BRA{3} + \Omega_2\KET{2}\!\BRA{3} + \mbox{h.c.} +\Delta\KET{3}\!\BRA{3}\nonumber\\ =\Omega\KET{B}\!\BRA{3} + \Omega^*\KET{3}\!\BRA{B} +\Delta\KET{3}\!\BRA{3},
\label{eq:STIRAP-Hamiltonian}
\end{gather}
where  the ''bright'' superposition state is defined as:  $\KET{B} \equiv (\Omega_1\KET{1}+ \Omega_2\KET{2})/\Omega$, with $\Omega \equiv \sqrt{\Omega_1^2 + \Omega_2^2}$. The orthogonal ``dark'' state,  $\KET{D} \equiv (\Omega_2\KET{1} - \Omega_1\KET{2})/\Omega,$ is an eigenvector of $H$ with eigenvalue 0. In the adiabatic limit, a system initialized in $\KET{D}$ remains in $\KET{D}$.  Therefore, to transfer population from $\KET{1}$ to $\KET{2}$ by following the dark state, one needs to first turn on only $\Omega_2$, then dim  it slowly and turn on $\Omega_1$. This ``counter-intuitive'' pulse sequence  is shown in~\figref{Fig1}(b).



 
Next, we present conditions for finding   inertial STIRAP solutions. We introduce the parameterization
\begin{gather}
\Omega_1 = \Omega(t) \sin\theta(t)  \hspace{0.5in}
\Omega_2 = \Omega(t)\cos\theta(t).
\label{eq:sinusoidal}
\end{gather}
The inertial conditions are 
\begin{align}
&(i)\hspace{0.2in}\theta(0) = 0 \hspace{0.25in} ,\hspace{0.25in} \theta(t_f)= \pi/2 \nonumber\\
&(ii)\hspace{0.2in}\dot{\theta}(0)=
\dot{\theta}(t_f) = 0,
\nonumber\\
&(iii)\hspace{0.2in}
\ddot{\theta}\ll \Omega^4/\Delta^2.
\label{eq:inertial-conditions}
\end{align}
where $t_f$ is  the protocol duration. Condition (\emph{i}) guarantees    that the dark state is $\KET{1}$ at the beginning and $\KET{2}$ at the end of the protocol. Condition (\emph{ii}) implies that the eigenstate of the inertial Hamiltonian coincides with the dark state at $t = 0$ and $t = t_f$. Finally, condition  (\emph{iii}) constrains the magnitude of the acceleration, minimizing non-adiabatic transitions in the inertial frame. In contrast to adiabatic conditions, the slope of $\theta$ during the protocol does not need to be small. This property enables, in principle, finding faster-than adiabatic solutions. The derivation is given below.

To simplify the equations, we present the derivation in the limit of large detuning, although the result holds in the  general case~\cite{dann2021inertial}.  
In this limit, the excited-state dynamics can be eliminated and the evolution in the ground-state manifold is  governed by  the Hamiltonian
\begin{equation}
H = \left( \begin{array}{cc}
\frac{\Omega_2^2 - \Omega_1^2}{2\Delta} & \frac{\Omega_2 \Omega_1}{\Delta}  \\
\frac{\Omega_2 \Omega_1}{\Delta} & \frac{\Omega_1^2 - \Omega_2^2}{2\Delta} \end{array} \right).
\label{eq:STIRAP-H-2level}
\end{equation}
(see Appendix A.)
%To satisfy the ``counter-intuitive'' STIRAP pulse order, we introduce  the parameterization \\ XXXX\\ with $\theta$ varying from 0  to $\frac{\pi}{2}$ during the protocol. To find inertial solutions, we  formulate  conditions that must be satisfied by such solutions [\eqref{inertial-conditions}]. 
Substituting~\eqref{sinusoidal} into~\eqref{STIRAP-H-2level}, one can  rewrite the Hamiltonian in the form
\begin{equation}
H = \tfrac{\Omega^2}{2\Delta}
\left(\cos2\theta\sigma_z + \sin2\theta\sigma_x\right).
\end{equation}
Seeking an inertial solution, we transform the Hamiltonian to the eigenvector basis and find 
\begin{gather}
\tilde{H} = \tfrac{\Omega^2}{2\Delta}
\left(\sigma_z - \tfrac{\chi}{2}\sigma_y\right) = f(t)\mat{M}(\chi)\quad,\quad
\chi \equiv  \tfrac{\Delta \dot{\theta}}{\Omega^2}
\label{eq:inertial-H}
\end{gather}
with dot denoting a time derivative (see  appendix B). 
The inertial protocol succeeds provided that   $\mat{M}(\chi)$ varies  slowly     [condition ($iii$)] and that  the eigenstates of $\mat{M}(\chi)$ and $H$ coincide  at the beginning and end of the protocol [condition (\emph{ii}), because given this condition, both matrices are diagonal]. 
We considered Hamiltonian dynamics, but the same conditions apply to   open systems (proven using   Liouville space formalism presented in~\cite{dann2021inertial}). 

 



%-------
% FIGURE 2
% -----
\begin{figure}
  \includegraphics[width=0.5\textwidth]{Fig2}
                  \caption{\textbf{Robust and efficient STIRAP solutions. }(a) Robustness of the four protocols versus single-photon  detuning $\Delta$. (b) 
                  We perform QOCT and update the control fields using  \eqref{updates}. 
                  Increasing the weights of  adiabatic and inertial constraints, $\lambda_{2}$ and $\lambda_3$, our algorithm finds solutions that reach fidelity of  $F = 0.99$ during $1\mu$s with reduced pulse energy.}
                  \label{fig:Fig2}
  \end{figure}
% ------------


%-------
% FIGURE 3
% -----
\begin{figure*}
  \includegraphics[width=\textwidth]{Fig3}
                  \caption{\textbf{Inertial geometric  gates} 
                  (a) Single-qubit phase gate,  implemented using tripod-level atoms and two fields performing two STIRAP steps (top). Gate infidelity versus protocol duration for the three protocols (bottom). (b) Hadamard gate, implemented with three fields using two STIRAPs (top) and  gate infidelity (bottom). (c) Controlled phase gate, based on two STIRAP steps and Rydberg-induced phase shifts (top) and gate infidelity versus maximal Rabi frequency  (bottom).}
                  \label{fig:Fig3}
  \end{figure*}
% ------------







In \figref{Fig1}(c),  we compare three analytic pulse shapes:  Gaussian  (solid green), squared sinusoidal  (dashed blue), and  ``inertial'' (dash-dot red),   with $\theta(t)$ satisfying  conditions (\emph{i})--(\emph{iii}) above. The pulses are defined as
\begin{subequations}
\begin{gather}
\mathrm{Gaussian}:\hspace{0.1in}
\Omega_{1,2}(t) = A(t) e^{-(t\pm t_0)^2/\sigma^2}\\
\mathrm{SinSQ}:
\hspace{0.05in}
\mathrm{Use}\hspace{0.05in}(6)
\hspace{0.05in}
\mathrm{with}\hspace{0.05in}
\theta(t) = \sin^2(t/\tau)\\
\mathrm{Inertial}:
\hspace{0.05in}
\mathrm{Use}\hspace{0.05in}(6)\hspace{0.05in}
\mathrm{and}\hspace{0.05in}(7)
\hspace{0.05in}
\mathrm{with}\hspace{0.05in}
\theta(t) = \sum_{i = 0}^3C_it^i
\label{eq:pulse-shapes}
\end{gather}
\end{subequations}
In \figref{Fig1}(d), we compare their infidelity as a function of protocol duration. 
The infidelity, $1 - \mbox{Tr}{\sqrt{\sqrt{\rho_f}\rho_t\sqrt{\rho_f}}}$, is defined in terms of the overlap between the desired target, $\rho_t$, and  final state, $\rho_f$. The infidelity of the protocols  decreases upon increasing protocol duration. Small oscillations in the infidelity are observed in all three plots. These arise from non-adiabatic transitions that can be controlled by using  mask functions that smooth the derivatives of the pulses, as explained in~\cite{laine1996adiabatic}. We find that the inertial protocol outperforms the other protocols.  We improve these results further with optimization (red triangles).  To this end, we perform an  iterative  algorithm, where we update the control fields in each step so that the target state is approached  while minimizing non-adiabatic and non-inertial transitions. 



 %Panel (c) shows the robustness of the protocol when deviating the time delay between the pulses from its optimal value and when introducing a non-zero single-photon detuning (see insets). The relative robustness of the cubic polynomial   emanates from the fact that it is the most inertial, in the sense that it has the smallest maximal curvature [$\max_t \ddot{\theta}(t)$] during the protocol [condition (\emph{iii})]. 




\emph{QOCT with adiabatic and inertial constraints.}
We construct a functional $\mathcal{J}$, whose minimal solution is an optimal inertial STIRAP solution.   %In QOCT,  a cost function is constructed, $\mathcal{J}$, whose minimal solution performs a desired unitary or drives the system towards  a desired target, while satisfying the appropriate dynamics  with a bounded amount of  power in the control pulses. 
Consider  the  functional
\begin{gather}
\mathcal{J}(\rho,\Omega) = \mathrm{Tr}\{\rho_t \rho_f\} - 
\int_0^{t_f}\!\! dt \mathrm{Tr}\{
\xi(t)\left[\tfrac{d}{dt}-\hat{\mathcal{L}}
\right]\rho(t)\}\nonumber\\
+\lambda_1 \int_0^{t_f}\!\! dt|\Omega(t)|^2+
\lambda_2 \int_0^{t_f}\!\! dt|\dot{\Omega}(t)|^2+
\lambda_3 \int_0^{t_f}\!\! dt|\ddot{\Omega}(t)|^2
\end{gather}
The quantum state is represented by a density matrix, $\rho$, and we use the symbol $\Omega$ to represent control fields (e.g., $\Omega_{1,2}$ in STIRAP).
The first term  is maximized when the system reaches the target state; that is, when $\rho_f=\rho_t $. The second  enforces  solutions to satisfy the Lindblad equation, $\dot{\rho}=\hat{\mathcal{L}}\rho$, where  $\hat{\mathcal{L}}$ is the Lindbladian operator,  $\xi$  is the associated  Lagrange multiplier. The third  restricts the amount of power in the control pulses. We include the  last two terms to restrict the time-averaged velocity and acceleration (in STIRAP, $\dot{\theta}$ and $\ddot{\theta}$ respectively). 

We use  Krotov’s method to optimize $\mathcal{J}$ \cite{krotov1983iterative,bartana1997laser,tannor1992control,krotov1988technique,krotov1995global,konnov1999global}, modified   to include adiabatic and  inertiality constraints. The algorithm starts from an initial guess for the controls, $\Omega$, solves the  equations of motion for $\rho$ and $\xi$, and   updates the guess, $\Omega\rightarrow\Omega+\Delta\Omega$, to decrease  $\Delta\mathcal{J}$ in each iteration. 
The equations of motion for $\rho$ and $\xi$  are the Lindblad equation and its conjugate. The update equation for $\Omega$ (required to  decrease $\mathcal{J}$)  is
\begin{gather}
2\lambda_1\Delta\Omega_R - 2\lambda_2\Delta\ddot{\Omega}_R + 2\lambda_3\Delta\ddddot{\Omega}_R = \mathrm{Tr}\{\xi^k \tfrac{\partial\Delta\hat{\mathcal{L}}}{\partial\Delta\Omega_R}\rho^{k+1}\}
\nonumber\\
2\lambda_1\Delta\Omega_I - 2\lambda_2\Delta\ddot{\Omega}_I + 2\lambda_3\Delta\ddddot{\Omega}_I = \mathrm{Tr}\{\xi^k \tfrac{\partial\Delta\hat{\mathcal{L}}}{\partial\Delta\Omega_I}\rho^{k+1}\}.
\label{eq:updates}
\end{gather}
Here $\Delta\Omega_R$ and $\Delta\Omega_I$  are the update rules of the real and imaginary parts of each control field in the $k+1$ iteration. The second and third terms on the left-hand side of~\eqref{updates} arise due to inertial and adiabatic constraints. These correspond to second- and fourth-order temporal derivatives.  
A complete derivation of the update equations is given in Appendix C, along with an alternative approach to impose the constraints (by maximizing the state projection onto the adiabatic or inertial subspaces). 

We use~\eqref{updates} to find an optimal  solution. Its fidelity is compared  with unoptimized protocols in~\figref{Fig1}(d) and their robustness is shown in \figref{Fig2}(a), demonstrating that the optimal protocol is  highly robust with respect to variations in the single-photon detuning, $\Delta$. \figrefbegin{Fig2}(b)   shows that when increasing the weight of the inertial constraint relative to the adiabatic constraint (e.g., taking $\lambda_3/\lambda_2 = 10$),  one obtains a 2dB reduction in pulse energy for the same protocol duration.  Having found optimal inertial and adiabatic solutions, we use those solutions to perform high-fidelity  quantum logic gates. 

% ----




\noindent\emph{Geometric quantum logic  gates.} 
Adiabatic protocols have many applications in quantum information science. An important class of  protocols involves cyclic evolution (which starts and ends with the same Hamiltonian) under which a qubit’s state acquires a geometric phase that depends only on its path in the Hilbert space~\cite{kato1950adiabatic}. While non-degenerate Hamiltonians are associated with scalar (abelian) phases, degeneracy gives rise to matrix (non-abelian) phases~\cite{zanardi1999holonomic}. When including both abelian and non-abelian ``holonomic'' operations, it is possible to realize a universal set of  gates that are robust against certain types of errors~\cite{saffman2009efficient,rao2013dark,rao2014deterministic,petrosyan2017high,khazali2020fast}. A protocol for holonomic quantum logic gates was proposed in~\cite{moller2007geometric}, where the qubit state acquires geometric phases by driving transitions out and back into the qubit subspace. Such transitions  can be driven  using STIRAP. In this section, we apply our optimized  pulses to implement these gates. 

Let us  review the principle of operation of the   gates.  Consider the tripod-level scheme shown in  \figref{Fig3}(a).  The qubit state is encoded in the levels $\KET{0}$ and $\KET{1}$. To implement a phase gate, one uses two fields, $\Omega_1$ and $\Omega_2$,  to transfer population from $\KET{1}$ to the auxiliary level $\KET{2}$ and back into $\KET{1}$,  using two   STIRAP steps (right inset).  A phase gate is implemented by shifting the phase of $\Omega_2$ in the second STIRAP step, which leads to an accumulated phase of the  dark state. % -- here, the $\KET{1}$ component of the wavefunction at the beginning and end of the protocol. 
To implement a Hadamard gate [\figref{Fig3}(b)], one needs a third field, $\Omega_0$. The three fields, $\Omega_{0,1,2}$, are applied following the protocol in the right inset. 
%One can understand the gate operation qualitatively in the following way. 
In this case, the dark  state is a  combination of  $\KET{0}$ and $\KET{1}$, whose weights are determined by the  ratio of $\Omega_0$ and $\Omega_1$. During the protocol, a phase is acquired by the dark state, which   translates  into a rotation in the qubit basis \cite{moller2007geometric,toyoda2013realization}. Finally, a  controlled-phase gate is achieved using Rydberg interaction between  excited states [\figref{Fig3}(c)]. This gate operates in a regime of a weak Rydberg interaction compared to the Rabi frequencies, $V_R\ll\Omega_{1},\Omega_{2}$. In this limit, when two atoms occupy the excited  state, their interaction produces an energy shift and  the wavefunction acquires a phase proportional to  $V_R$.  A controlled phase gate is implemented by applying two STIRAP steps (right inset) and tuning the time delay between  the pulses so that the  phase  acquired  by the $\KET{11}$ component of the wavefunction is  $\pi$. 



We simulated  these protocols  using the three pulse shapes from~\figref{Fig1}.  For each protocol, we reconstructed the simulated gate, $\rho\rightarrow \hat{G}(\rho)$, with process tomography (appendix E), which  provides the Kraus operators, $G_k$, that can be used to write   $\hat{\mathcal{G}}$ as:~\cite{nielsen2002quantum}
\begin{equation}
\hat{\mathcal{G}}(\rho) = \sum_k G_k\rho G_k^\dagger.
\end{equation}
Using this expansion, we computed the average gate fidelity, $\mathcal{F}$, defined as the overlap between the simulated and target gates ($\hat{\mathcal{G}}$ and $U_0$ respectively) averaged   over all possible initial states. For an initial pure state, $\KET{\psi}$, the average gate fidelity $\mathcal{F}$ is given by~\cite{nielsen2002simple,beterov2016simulated,goerz2014robustness,pedersen2007fidelity}:
\begin{gather}
\mathcal{F} \equiv 
\int_{S^{2n-1}} \hspace{-0.2in}dV\,
\BRA{\psi}
U_0^\dagger
\hat{G}(\KET{\psi}\!\BRA{\psi})
U_0\KET{\psi}
=
\nonumber\\
\frac{1}{n(n+1)}\sum_k\left(
\mbox{Tr}(M_kM_k^\dagger)+\left|\mbox{Tr}(M_k)\right|^2
\right),
\end{gather}
where $n$ is the dimension of the Hilbert space (2 for  single-qubit gates  and 4 for 2-qubit gates) and   $M_k \equiv U_0^\dagger G_k$. 

The results are shown in the bottom plots.  For  single-qubit gates, we plot the infidelity as a function of protocol duration, where the limit of large $t_f$ corresponds to the adiabatic limit. For the controlled-phase gate, we plot the infidelity versus Rabi frequency, since $t_f$ is set by requiring that the conditional phase shift  is $\pi$. 
The inertial  protocol approaches $\mathcal{F}=1$ at the fastest rate, as expected based on \figref{Fig1}. Unfortunately, the gate fidelity is much smaller than the  STIRAP fidelity.
%(e.g., we get $10^{-4}$ for the phase gate compared to $10^{-8}$ for STIRAP  for $t_f = 1\mu$s). 
The reason is that  gate fidelities are limited by phase errors, which are  larger than population errors that limit STIRAP \footnote{It  is a consequence of first-order perturbation theory that population errors scale quadratically with  the small parameter while phase errors scale linearly.}.  
  


%----------------------------
% FIGURE 4
% ------------------------------
\begin{figure}[t] \includegraphics[width=0.45\textwidth]{Fig4}
 \caption{\textbf{Inertial STIRAP protocol.} (a,b) Two possible realizations of a tripod level system using  the D2 line of  $^{87}$Rb.  Desired transitions (solid lines) are shown  with the corresponding   polarizations: circular ($\sigma_\pm$) and $\pi$. Additional transitions are shown by grey arrows. 
   (c) Infidelity of the phase gate [\figref{Fig2}(a)] versus polarization error, $\varepsilon$, defined as the fraction of  undesired  polarization in each STIRAP pulse.}           \label{fig:Fig4}
  \end{figure}
% ------------



\noindent\emph{Experimental realization.} 
We analyze a realization  of our protocol using $^{87}$Rb atoms. 
Two possible choices  of energy levels and transitions from the D2 line are shown  in \figref{Fig4}(a-b).   
In particular, we are  interested in analyzing realistic imperfections, including polarization  and  leakage errors (following the analysis from~\cite{shomroni2014all,rosenblum2016extraction,rosenblum2017analysis,bechler2018passive}). 
For example, consider a single-qubit phase gate implemented using the levels shown in  \figref{Fig4}(a). Ideally, to drive STIRAP transitions from the state $\KET{0_L}$ to $\KET{A}$ and back, we want to apply only two beams: The pulse $\Omega_1$ with  $\sigma_+$ polarization, which drives $\KET{F = 1,m_F = -1}\leftrightarrow\KET{F'=0,m_F = 0}$, and the pulse $\Omega_2$ with  $\sigma_-$ polarization,  which drives  $\KET{F = 1,m_F = 1}\leftrightarrow\KET{F'=0,m_F = 0}$.
However, polarization errors in the second pulse, $\Omega_2$, produce a $\sigma_+$ component that drives the first transition, and  errors in the first pulse drive the second transition. This process reduces the success probability of the protocol. To combat this issue, one can apply  a DC magnetic field to shift the energy levels and tune the frequencies of the control fields to be in resonance with the shifted levels. By off-setting the frequencies of the $\Omega_1$ and $\Omega_2$ pulses, the error fields become off resonant and their effect is suppressed. The magnetic field should not exceed 72MHz to avoid mixing of the excited states  ($F'=0$ and $F'=1$).  Our simulations demonstrate that the fidelity approaches unity when  increasing the magnetic field [\figref{Fig4}(c)]. 




 





%Most generally, taking into account all possible transitions and imperfections is a complex computation. It involves  10 energy levels  and  three pulses  with imperfect polarization.   When including polarization errors, the Hamiltonian contains atomic transitions that are driven by multiple fields. This gives rise to oscillating terms in the Hamiltonian that cannot be eliminated by a   rotating frame approximation.\footnote{For example, when implementing the Hadamard gate from Fig. 2(b), the   $\KET{-1,1}\leftrightarrow\KET{1,0}$ transitions is driven by the main component of $\Omega_1$ and by the polarization error of $\Omega_0$.}   The full dynamics including radiative decay of the excited levels via the Lindblad equations becomes pretty involved.  To analyze this effect, we develop an approximate time-independent Hamiltonian and test the approximation against brute-force temporal simulations.   The details of the derivation appear in Appendix F.   The results of our numerical simulations are shown in Fig.~3(c). Additionally, we use the fact that we can tune the cavity resonance to enhance transitions of our choice to further suppress spurious transitions. We model cavity enhancement by  multiplying the transition dipole moment and spontaneous emission rate of the relevant transition by the cavity enhancement factor. We find that  6-fold enhancement of the cooperativity increases the gate fidelity to above $96\%$, using the parameters of our simulation from Fig.~2(a) but considering all possible transitions. 



%% POINTS FOR DISCUSSION: 
% In this work, we presented  inertial protocols for population transfer and geometric single- and two-qubit  gates. We estimated the fidelity and robustness of our protocols using realistic experimental parameters, taking full account of practical  practical errors including polarization impurity, spontaneous emission, and populating levels outside of the tripod level system. We have shown how Zeeman shifts can be used to suppress polarization errors. We have not presented that by using an optical  cavity to enhance  certain transitions, we can suppress errors due to population leakage. This approach was used in [Dayan] so we know it is feasible.  While reaching extremely high fidelity for STIRAP, it is much harder to achieve high fidelity  geometric gates since phase errors are much larger than population errors. However, the inertial protocol is still superior to traditional pulse shapes. To put this work in context, we must address the large body of previous work on improving adiabatic protocols and STIRAP in particular. The novelty of this work is that we find simple analytic pulse shapes that yield nearly optimal solutions by satisfying a new condition of small acceleration, which a new concept. The application we chose - geometric gates - has a downside that the protocol is not robust to fluctuations in the geometric phase, for example the single-qubit phase gate relies on fixing the phase on of the pulses but this is sensitive to perturbations. It would be intriguing to optimizae topological adiabatic protocols in future work.  Another strong point of our approach is that we account for decay directly in the optimization process, so finite lifetimes, decay channels and branching ratios affect the solution. Indeed, we find that optimal pulse shapes depend on the dissipation, yielding in some cases double-humped pulses instead of pulses with a single local maximum as in traditional STIRAP. For honesty, we must note that the inertial solutions we fund are also adiabatic in the sense that we were not able to find an inertial solution with a large velocity that achieves high fidelity at a short time. The reason is that strong violations of adiabaticity are associated with violations of inertiality for the inertial conditions of our protocols [Eq.~(X)].   Due to the impressive  recent  progress in atomic quantum copmuters, we believe that our protocols could be directly applicable and useful to improve further gate fidelities. 
 
 \emph{Conclusion:} In this work, we presented  inertial protocols for STIRAP and geometric single- and two-qubit  gates. We estimated the fidelity and robustness of our protocols using realistic experimental parameters, taking  account of practical   considerations including polarization impurity, spontaneous emission, and populating leakage (i.e., transitions out of the tripod  system). We used a DC magnetic field to suppress polarization errors. The fidelity could be improved further by introducing  a resonant  optical  cavity to enhance  desired transitions, e.g., using the setup of~\cite{bechler2018passive}. 
 %While reaching extremely high STIRAP   fidelities, it is  much harder to achieve similar values for   geometric gates. This problem is fundamental,  since phase errors are much larger than population errors. However, the inertial protocol is still superior to traditional pulse shapes. 
In light of the long list of existing ``improved adiabatic protocols,''  our work adds  simple analytic pulse shapes that yield nearly optimal solutions by satisfying the   condition of small acceleration~\cite{dann2021inertial}. Another strength  of our approach is that we consider Lindbladian dynamics  in the optimization process, so the various decay channels with their respective branching ratios affect the optimal solution. Indeed, we find in some cases optimal  pulses with two humps, as opposed to standard quasi-Gaussian pulses. A downside of our approach is that our protocols are not robust to perturbations in the   geometric phase acquired during the protocol, but are rather determined by it.  It would be intriguing to optimize  adiabatic protocols 
that  are topologically robust to errors (such as~\cite{kitaev2010topological,lahtinen2017short}) in future work.   Finally, we note that  the presented inertial solutions are  also adiabatic in the sense that the control fields do not vary rapidly in comparison to the energy gap. The reason is that
when considering protocols that satisfy the   conditions of \eqref{inertial-conditions},  strong violations of adiabaticity imply  violations of inertiality.   We hypothesize that  inertial non-adiabatic solutions exist in other cases.  To conclude, due to immense   recent  progress in  quantum computation research  and the ability to prepare and manipulate large  registers of  atomic qubits~\cite{barredo2016atom,bernien2017probing,henriet2020quantum},  our protocols are directly  applicable and could be used to improve  gate fidelities in these platforms. 



 
 %In this paper we developed a rapid and robust  protocol for STIRAP and STIRAP-based quantum logic gates, using inertial solutions and quantum optimal control. We have analyzed the theoretical performance of our proposed protocol and performed simulations for a realistic model system. Our calculations demonstrate the superiority of the inertial protocol over traditional pulse shapes in the long-time limit in all studied examples. Although we reach population transfer infidelities of $10^{-8}$ under realistic time and energy constraints, geometric quantum logic gates suffer from phase errors and reach high (although not as high) fidelities of $10^{-4}$. We overcome  practical challenges of polarization impurity and spurious transitions with standard tools of Zeeman shifts and tunable  cavity enhancement, as proposed and implemented in [x]. 
 


 \section*{Acknowledgement}
 The authors would like to thank Aviv Aroch for helping us set up and run Krotov's algorithm. 
 A.P. acknowledges support from the Alon Fellowship of the Israeli Council of Higher Education. 
 B.D. acknowledges support from the Israeli Science Foundation, the Binational Science Foundation, H2020
Excellent Science (DAALI, 899275), the Minerva Foundation.
 R.K. acknowledges support from the Israeli Science Foundation, grant  number 526/21.
  R.D. acknowledges support from the
 Adams Fellowship Program of the Israel Academy of Science and Humanities. 
 %Aviv - Should he be an author?
%David Tannor. Christianne Koch. 
%Alon. Golda Meir Fellowship. 
 % Ask others about their funding.
 % BARAK!!!!

\bibliography{STIRAPbibli}
\bibliographystyle{ieeetr}

\newpage


\section*{Appendix}

\section*{A. Effective 2-level Hamiltonian for STIRAP}
In this appendix, we present a Derivation  of Eq.~(6)
from the main text. 
Consider the  Hamiltonian of a 3-level system from driven by two optical  fields (as shown in Fig.~2):
\begin{equation}
H = \Omega_1\KET{1}\!\BRA{3} + \Omega_2\KET{1}\!\BRA{2} + h.c. +
\Delta\KET{3}\!\BRA{3}.
\end{equation}
Let us express the wavefunction as
\begin{equation}
\KET{\psi} = c_1\KET{1}+c_2\KET{2}+c_3\KET{3}.
\end{equation}
We  substitute this expansion  into the  Schr\"{o}dinger equation $i\hbar\tfrac{d\KET{\psi}}{dt} = H\KET{\psi}$, to obtain the  equations of motion for the coefficients:
\begin{gather}
i\hbar\dot{c}_1 = \Omega_1c_3\nonumber\\
i\hbar\dot{c}_2 = \Omega_2c_3\nonumber\\
i\hbar\dot{c}_3 = \Omega_1^*c_1+\Omega_2^*c_2+\Delta c_3.
\label{eq:emo}
\end{gather}
In the limit of large detuning, $\Delta\gg\Omega_1,\Omega_2$, on can adiabatically eliminate the dynamics of the third level, setting $\dot{c}_3 = 0$. This assumption yields the relation
\begin{equation}
c_3 = -\tfrac{\Omega_1^*}{\Delta}c_1 - \tfrac{\Omega_2^*}{\Delta}c_2.
\label{eq:elimination}
\end{equation}
Substituting \eqref{elimination} into \eqref{emo}, we obtain
\begin{gather}
i\hbar\dot{c}_1 = -\tfrac{|\Omega_1|^2}{\Delta}c_1 -
\tfrac{\Omega_1\Omega_2^*}{\Delta}c_2\nonumber\\
i\hbar\dot{c}_2 = -\tfrac{|\Omega_2|^2}{\Delta}c_2 -
\tfrac{\Omega_2\Omega_1^*}{\Delta}c_1
\end{gather}
The dynamics is described by the effective Hamiltonian
\begin{equation}
H = \left( \begin{array}{cc}
\frac{|\Omega_1|^2}{2\Delta} & \frac{\Omega_2^*\Omega_1}{\Delta}  \\
\frac{\Omega_2\Omega_1^*}{\Delta} & \frac{|\Omega_2|^2}{2\Delta} \end{array} \right).
\label{eq:H-before-subtract}
\end{equation}
By subtracting a  constant-diagonal matrix, $\tfrac{|\Omega_1|^1-|\Omega_2|^2}{2\Delta}\cdot\mathcal{I}$, from \eqref{H-before-subtract} one obtains Eq.~(6). Subtracting a constant  diagonal matrix only shifts the energy spectrum by a constant, symmetrizing it around zero, without affecting the dynamics otherwise.


\section*{B. Inertial STIRAP Hamiltonian}
In this section, we  derive   Eq.~(9) from the main text. 
We choose the parameterization
\begin{equation}
\Omega_1 = \Omega\sin\theta \quad,\quad
\Omega_2 = \Omega\cos\theta.
\end{equation}
The 2-level Hamiltonian [Eq.~(6) from the main text] becomes
\begin{equation}
H = \tfrac{\Omega^2}{2\Delta}
\left(
\cos2\theta\sigma_z + \sin2\theta\sigma_x\right)
\end{equation}
where $\sigma_x$ and $\sigma_z$ are Pauli matrices. 
We obtain an explicit expression for the inertial-frame Hamiltonian defined as 
\begin{equation}
\tilde{H}\equiv
P^\dagger H P - i\hbar P^\dagger \frac{\partial P}{\partial t}
\end{equation}
The eigenvalues are $\pm\Omega$. The eigenvector matrix is
\begin{equation}
P = \left( \begin{array}{cc}
\frac{\cot2\theta+\csc2\theta}{\sqrt{\cot{\theta}^2+1}} & \frac{\cot2\theta-\csc2\theta}{\sqrt{\tan{\theta}^2+1}}  \\
\frac{1}{\sqrt{\cot{\theta}^2+1}} & \frac{1}{\sqrt{\tan{\theta}^2+1}} \end{array} \right).
\end{equation}
For $0<\theta<\pi$, we find
\[
P^\dagger\frac{dP}{d\theta} = 
\tfrac{\dot{\theta}}{2}
\left( \begin{array}{cc}
0 & -1  \\
1 & 0 \end{array} \right).
\]
This completes the proof of Eq.~(9) in the main text.


One finds a striking similarity between the inertial-frame Hamiltonian, Eq.~(9), and the Hamiltonian one obtains when adding counter-diabatic terms to the lab-frame Hamiltonian realizing the shortcut-to-adiabaticity protocol (see Eqs.~(4-5) in the Methods section of~\cite{du2016experimental}). 
The similarity is  expected, since the inner product of the instantaneous eigenvectors and their derivatives appear in both formulations. 
However, there is a conceptual difference which is important to point out. 
In shortcut-to-adiabaticity, one constructs a Hamiltonian, which may be challenging to realize, but whose eigenstates are exact solutions to the dynamics. In contrast, in our approach, the lab-frame Hamiltonian is straightforward, and the instantaneous eigenstates of the inertial-frame solutions are only approximations to the true dynamics.




\section*{C. Adiabatic and inertial constraints in QOCT}
In this section, we generalize Krotov's QOCT  algorithm  to include inertial and adiabatic constrains. We present to approaches: 
in the first, we add a penalty for the time averaged velocity and acceleration, while in the second, we penalize instantaneous population in non-adiabatic and non-inertial states. We follow the derivation of~\cite{bartana1997laser} with appropriate modifications to our problem. 

\subsection*{C1. Introducing penalty terms for the  time-averaged velocity and acceleration}

Let us consider the following functional to be minimized:
\begin{gather}
\mathcal{J} = \mathrm{Tr}\{\rho_t \rho_f\} - 
\int_0^{t_f}\!\! dt \mathrm{Tr}\{
\xi(t)\left[\tfrac{d}{dt}-\hat{\mathcal{L}}
\right]\rho(t)\}\nonumber\\
+\lambda_1 \int_0^{t_f}\!\! dt|\Omega(t)|^2+
\lambda_2 \int_0^{t_f}\!\! dt|\dot{\Omega}(t)|^2+
\lambda_3 \int_0^{t_f}\!\! dt|\ddot{\Omega}(t)|^2
\label{eq:J-appendix}
\end{gather}
where the two last terms correspond to the integrated velocity and acceleration. Using integration by parts, one can rewrite  the first line of \eqref{J-appendix} as:
\begin{gather}
\mathrm{Tr}
\left\{\rho_t \rho_f - 
(\xi\,\rho)|_0^{t_f}+
\int_0^{t_f}\!\! dt \left[
\rho\tfrac{d\xi}{dt}+
\xi\hat{\mathcal{L}}\rho\right]\right\}.
\label{eq:integration-by-parts}
\end{gather}
Krotov's algorithm starts with an initial guess from the control fields, evolves $\rho$ and $\xi$ and finds an updated control pulse that decreases $\mathcal{J}$. By iteratively updating the control pulse, convergence is reached.   Let us  compute the change in $\mathcal{J}$  between   iteration $k$ and $k+1$ in terms of the change in $\rho$. Using the definitions
\begin{gather}
\Delta\mathcal{J}=\mathcal{J}^{(k+1)}-\mathcal{J}^{(k)}\quad,\quad
\Delta\rho=\rho^{(k+1)}-\rho^{(k)},
\end{gather}
 we obtain:
 \begin{gather}
\Delta\mathcal{J} = \mathrm{Tr}
\left\{\rho_t \Delta\rho - 
\xi^k\,\Delta\rho]|_0^{t_f}\right\}+\nonumber\\
\mathrm{Tr}
\left\{\int_0^{t_f}\!\! dt \left[
\Delta\rho\tfrac{d\xi}{dt}+
\xi^k\hat{\mathcal{L}}\Delta\rho+
\xi^k\Delta\hat{\mathcal{L}}\rho^{(k+1)}\right]\right\}+\nonumber\\
-\int_0^{t_f}
\left\{2\lambda_1\mathrm{Re}[\varepsilon^k\Delta\varepsilon^*]+\lambda_1[\Delta\varepsilon_R^2+\Delta\varepsilon_I^2]\right\}+\nonumber\\
-\int_0^{t_f}
\left\{2\lambda_2\mathrm{Re}[\dot{\varepsilon}^k\Delta\dot{\varepsilon}^*]+\lambda_2[\Delta\dot{\varepsilon}_R^2+\Delta\dot{\varepsilon}_I^2]\right\}+\nonumber\\
-\int_0^{t_f}
\left\{2\lambda_3\mathrm{Re}[\ddot{\varepsilon}^k\Delta\ddot{\varepsilon}^*]+\lambda_3[\Delta\ddot{\varepsilon}_R^2+\Delta\ddot{\varepsilon}_I^2]
\right\}.
\end{gather}
The first 4 terms do not depend on the field	increment and produce the dynamical equations for $\rho$ and $\xi$. The additional terms produce the differential: $\Delta\mathcal{J}_1+
\Delta\mathcal{J}_2+\Delta\mathcal{J}_3+\Delta\mathcal{J}_4$ where
\begin{gather}
\Delta\mathcal{J}_1 = 
\mathrm{Tr}
\left\{\int_0^{t_f}\!\! dt 
\xi^k\Delta\hat{\mathcal{L}}\rho^{(k+1)}\right\}
\nonumber\\
\Delta\mathcal{J}_2 =  
-2\mathrm{Re}\left[\int_0^{t_f}
\lambda_1\varepsilon^k\Delta\varepsilon^*
+\lambda_2\dot{\varepsilon}^k\Delta\dot{\varepsilon}^*
+\lambda_3\ddot{\varepsilon}^k\Delta\ddot{\varepsilon}^*\right]\nonumber\\
\Delta\mathcal{J}_3 =  
-\int_0^{t_f}\!
\lambda_1\Delta{\varepsilon}_R^2+\!
\lambda_2\Delta\dot{\varepsilon}_R^2+\!
\lambda_3\Delta\ddot{\varepsilon}_R^2\nonumber\\
\Delta\mathcal{J}_4 =  
-\int_0^{t_f}\!
\lambda_1\Delta{\varepsilon}_I^2+\!
\lambda_2\Delta\dot{\varepsilon}_I^2+\!
\lambda_3\Delta\ddot{\varepsilon}_I^2.
\end{gather}
The term $\Delta\mathcal{J}_2$ vanishes on average since the integrand is rapidly oscillating. The complete differential of the third term is
\begin{gather}
d(\Delta\mathcal{J}_3) = \int_0^{t_f} df(\Delta\varepsilon_R,\Delta\dot{\varepsilon}_R,\Delta\ddot{\varepsilon}_R)dt = \nonumber\\
\int_0^{t_f}
\left[\tfrac{df}{d\Delta\varepsilon_R}\Delta\varepsilon_R+
\tfrac{df}{d\Delta\dot{\varepsilon}_R}\Delta\dot{\varepsilon}_R+
\tfrac{df}{d\Delta\ddot{\varepsilon}_R}\Delta\ddot{\varepsilon}_R\right]dt \nonumber\\
\int_0^{t_f}
\left[
\tfrac{\partial f}{\partial\Delta\varepsilon_R} -
\tfrac{d}{dt}\tfrac{\partial f}{\partial\Delta\dot{\varepsilon}_R} + 
\tfrac{d^2}{dt^2}\tfrac{\partial f}{\partial\Delta\ddot{\varepsilon}_R}
\right]\Delta\varepsilon_R\,dt +\mbox{surface terms}\nonumber\\
2\int_0^{t_f}(-\lambda_1\Delta\varepsilon_R+\lambda_2\Delta\ddot{\varepsilon}_R-\lambda_3\Delta\!\ddddot{\varepsilon}\!\!\!_R)+\mbox{surface terms}
\end{gather}
The surface terms vanish since the field update at the start and end point is zero, $\Delta\varepsilon_R(0)= \Delta\varepsilon_R(t_f) = 0$. Using this result (and repeating the same procedure for $\Delta\mathcal{J}_4$), we find update equations:
\begin{gather}
-\lambda_1\Delta\varepsilon_R+\lambda_2\Delta\ddot{\varepsilon}_R-\lambda_3\Delta\!\ddddot{\varepsilon}\!\!\!_R = \tfrac{1}{2}\mbox{Tr}
[\xi^{k}\tfrac{\partial\Delta\hat{\mathcal{L}}}{\partial\Delta\varepsilon_R}
\rho^{k+1}]\equiv F(t)\nonumber\\
-\lambda_1\Delta\varepsilon_I+\lambda_2\Delta\ddot{\varepsilon}_I-\lambda_3\Delta\!\ddddot{\varepsilon}\!\!\!_I = \tfrac{1}{2}\mbox{Tr}[
\xi^{k}\tfrac{\partial\Delta\hat{\mathcal{L}}}{\partial\Delta\varepsilon_I}
\rho^{k+1}]\equiv G(t)
\end{gather}
These equations are easily solved by discretizing  time and rewriting the equation in matrix form (the second-order time derivatives becomes the tridiagonal Hessian matrix and the fourth-order time derivatives is a 5-diagonal matrix).
We obtain equations of the form
\begin{gather}    \mat{M}_1\Delta\varepsilon_R = \mat{F}\nonumber\\
    \mat{M}_2\Delta\varepsilon_I = \mat{G}
\end{gather}
The update rules for the control fields are found by inverting $\mat{M}_1$ and $\mat{M}_2$.




%------------------------
% FIGURE 5
% ------------
\begin{figure*}[t]
\includegraphics[width=\textwidth]{tomography}
  \caption{\textbf{Quantum gate tomography:}
  Visualization of  target (top) and simulated (bottom)  single- and  two-qubit  gates. 
The plots show the matrix $\chi$, which   expresses the   expansion coefficients of the  gate transformations in terms of Pauli basis operators,
$\tilde{E}_i$ [\eqref{Pauli-basis}] for single-qubit gates  and Kronecker products of $\tilde{E}_i$ for two-qubit gates respectively.
  (a--d) The matrix  $\chi$, 
  calculated using  \eqref{chi-single-qubit},  for Hadamard and Pauli-Z single-qubit gates. (e,f) The matrix  $\chi$, calculated via \eqref{chi-two-qubit} for a two-qubit controlled-phase gate. }
   \label{fig:tomography}
\end{figure*}
% ----------------


\subsection*{C2. Maximize projection onto desired subspace}


In this section, we present an alternative approach for optimization. We used both protocols in our simulations, choosing the protocol with the fastest convergence by trial and error.  In this case, we impose adiabatic and inertial constraints by maximizing  the projections onto adiabatic or inertial eigenstates. Denoting the projection operator by $P$ (e.g., $P = \KET{D}\BRA{D}$ for adiabatic protocols), we penalize undesired transitions by introducing the last term in  the functional
\begin{gather}
\mathcal{J} = \mathrm{Tr}\,\{\rho_t \rho_f\} - 
\int_0^{t_f}\!\! dt \mathrm{Tr}\{
\xi(t)\left[\tfrac{d}{dt}-\hat{\mathcal{L}}
\right]\rho(t)\}\nonumber\\
+\lambda_1 \int_0^{t_f}\!\! dt|\Omega(t)|^2-
\lambda_2 \int_0^{t_f}\!\! dt\, \mbox{Tr}\,(P\rho)(P\rho)^\dagger
\label{eq:project-constraint}
\end{gather}
For convenience,  we rewrite the integrand of the last term in the form
\begin{gather}
\mbox{Tr}\,(P\rho)(P\rho)^\dagger = 
\mbox{Tr}\,(P\rho)(\rho P) = 
\mbox{Tr}\,\rho P \rho.
\label{eq:J-projection-constraint}
\end{gather} 
To derive optimization equations, we compute the full  differential of $\mathcal{J}$ with respect to $\delta\rho, \delta\xi$ and $\Omega$. The new constraint modifies only  terms related to  $\delta\rho$. Specifically, taking the differential of \eqref{J-projection-constraint} with respect to $\delta\rho$, we obtain
$\mbox{Tr}\,(2P\rho)\delta\rho$.
Revisiting all terms in \eqref{project-constraint}, one can see that the second term also contributed to $\delta\rho$. After integration by parts, from the integral term in \eqref{integration-by-parts}, we have 
\begin{gather}
\frac{\partial\mathcal{J}}{\partial\rho}\delta\rho = 
\mathrm{Tr}
\int_0^{t_f}\!\! dt \left[
\tfrac{d\xi}{dt}+
\xi\hat{\mathcal{L}} +  2 \lambda_2P \rho \right]\delta \rho
\end{gather}
We obtain a modified equation of motion  for the Lagrange multiplier
\begin{gather}
\tfrac{d\xi}{dt}+
\hat{\mathcal{L}}^\dagger\xi +  2 \lambda_2P \rho  = 0
\end{gather}
The equation of motion of $\rho$ and the update equation for the field are unaltered by the projection constraint. We obtain an equation of the Lindblad form with an inhomogeneous source term, which we can solve by standard numerical methods. 





 

\section*{D. Quantum gate tomography} 
Quantum process tomography (QPT) is a method by which a quantum gate is reconstructed from simulated data (experimental or numerical).  In this work, we use QPT to quantify the overlap between our simulated and target gates. In this appendix, we review the method presented in \cite{nielsen2002quantum}.


Suppose that  a density matrix $\rho$ evolves under our simulation  into $\mathcal{G}(\rho)$. The goal of QPT is to determine  $\mathcal{G}$.  
Given a  complete  basis of matrices, $\tilde{E}_m$,  that spans the operator  space (i.e., the Liouville space),  any unitary $\mathcal{G}$ can be written in the form
\begin{equation}
 \mathcal{G}(\rho)=\sum_{m,n} \chi_{mn}\tilde{E}_m\rho \tilde{E}_n^\dagger,
\end{equation}
where $\chi_{mn}$ are coefficients that we wish to find.  For single-qubit gates,  the set $\tilde{E_m}$ contains four  $2\times2$ matrices ($m=1,...,4$) and $\chi_{mn}$ is a $4\times4$ matrix. Let us choose the Pauli and identity matrices as basis operators:
\begin{gather}
\tilde{E}_1 = I\quad,\quad
\tilde{E}_2 = \sigma_x\quad,\quad
\tilde{E}_3 = -i\sigma_y\quad,\quad
\tilde{E}_4 = \sigma_z.
\label{eq:Pauli-basis}
\end{gather}
To determine $\chi$, we run our simulations on 4 initial states 
\begin{gather}
\rho_1 = \KET{0}\!\BRA{0}\hspace{0.05in},\hspace{0.05in}
\rho_2 = \rho_1\sigma_x \hspace{0.05in},\hspace{0.05in}
\rho_3 = \sigma_x\rho_1\hspace{0.05in},\hspace{0.05in}
\rho_4 = \sigma_x\rho_1\sigma_x.
\end{gather}
We compute the final state for every  input: $\rho_i'\equiv\mathcal{E}(\rho_i)$. Then, 
following \cite{nielsen2002quantum}, $\chi$ is given by   
\begin{equation}
\chi = 
\frac{1}{4} \begin{pmatrix}
I & \sigma_x\\
\sigma_x & I
 \end{pmatrix}
 \begin{pmatrix}
\rho_1' & \rho_2' \\
\rho_3' & \rho_4'
 \end{pmatrix}
 \begin{pmatrix}
I & \sigma_x\\
\sigma_x & I
 \end{pmatrix}.
 \label{eq:chi-single-qubit}
\end{equation}
where $I, \sigma_x, \rho'_{1},...,\rho'_{4} \in \mathbb{R}^{2\times2}$  and $\chi \in \mathbb{R}^{4\times4}$.

 





For two-qubit gates, 
we write
\begin{equation}
 \mathcal{G}(\rho)=\sum_{m,n} \chi_{mn}\tilde{E}_m^{(2)}\rho (\tilde{E}_n^{(2)})^\dagger
\end{equation}
in terms of two-qubit operators, $\tilde{E}_m^{(2)}\in\mathbb{R}^{4\times4}$, defined as 
Kronecker products of the single-qubit  basis operators, $\tilde{E}_m$, from \eqref{Pauli-basis}.
 The matrix $\chi\in\mathbb{R}^{16\times16}$ is given by
\begin{equation}
\chi=\Lambda\,\Bar{\rho}\,
\Lambda, 
 \label{eq:chi-two-qubit}
\end{equation}
where $\Lambda\in\mathbb{R}^{16\times16}$ is a rotation matrix
\begin{align}
\Lambda\equiv
\frac{1}{4}\left[
\begin{pmatrix}
I & \sigma_x\\
\sigma_x & I
\end{pmatrix}\otimes\begin{pmatrix}
I & \sigma_x\\
\sigma_x & I
 \end{pmatrix}\right]
\end{align}
and $\Bar{\rho}\in\mathbb{R}^{16\times16}$ is defined as
\begin{align}
\Bar{\rho}=P^T \rho' P. 
\end{align}
Here, $\rho'\in\mathbb{R}^{16\times16}$
the matrix of final states 
\begin{align}
\rho' = \mathcal{G}(\rho),
\end{align}
found by propagating  16 input  states 
\begin{align}
\rho_{mn} = T_n\KET{00}\BRA{00}T_m\quad
\forall n,m =1,...,4
\end{align}
with 
\begin{gather}
T_1=I\otimes I \quad,\quad
T_2=I\otimes\sigma_x\nonumber\\
T_3=\sigma_x\otimes I  \quad,\quad
T_4=\sigma_x\otimes \sigma_x.
\end{gather}
The permutation matrix $P$ is given by 
\begin{align}
P = I\otimes\ 
 \left[  \begin{pmatrix}
    1 & 0& 0 & 0 \\
   0 & 0&1 &0 \\
    0& 1 &0 &0 \\
 0 & 0 & 0 & 1
\end{pmatrix}
 \otimes I  \right].
 \end{align}
 Note that $I\in\mathbb{R}^{2\times2}$ and, hence, $P\in\mathbb{R}^{16\times16}$. 

 





\end{document}
