% \vspace{-0.05in}
\begin{figure*}
    \vspace{-.2in}
    \centering
    \begin{subfigure}{0.325\textwidth}
    \includegraphics[width=1.0\linewidth]{figure/2DReach_loss.pdf}
    \caption{2DReach}
    \label{fig:bc_2dreach}
    \end{subfigure}
    \begin{subfigure}{0.325\textwidth}
    \includegraphics[width=1.0\linewidth]{figure/Pusher_loss.pdf}
    \caption{Pusher}
    \label{fig:bc_pusher}
    \end{subfigure}
    \begin{subfigure}{0.325\textwidth}
    \includegraphics[width=1.0\linewidth]{figure/BC_ablation_AntMazeL.pdf}
    \caption{L-shaped AntMaze}
    \label{fig:bc_antmazel}
    \end{subfigure}
    % \begin{subfigure}{0.325\textwidth}
    % \includegraphics[width=1.0\linewidth]{figure/lambda ablation.pdf}
    % \caption{Balancing coefficient $\lambda$}
    % \label{fig:lambda}
    % \end{subfigure}
    \caption{Ablation studies about self-imitation learning for training on (a) 2DReach, (b) Pusher, and (c) L-shaped Ant Maze with four runs.
    MSS + $\mathcal{L}_{\mathtt{\ALGname}}$ and MSS + $\mathcal{L}_{\mathtt{GCSL}}$ refer to an algorithm that applies loss term $\mathcal{L}_{\mathtt{\ALGname}}$ and $\mathcal{L}_{\mathtt{GCSL}}$ on top of MSS method, respectively; subgoal skipping is not applied. We find that our loss term $\mathcal{L}_{\mathtt{\ALGname}}$ is more effective than $\mathcal{L}_{\mathtt{GCSL}}$ as an auxiliary term. 
    % For (b), we investigate the effectiveness of balancing coefficient $\lambda$ that determines influence of our loss term $\mathcal{L}_{\mathtt{\ALGname}}$ relative to actor loss term $\mathcal{L}_{\mathtt{actor}}$. 
    }
    \label{fig:ablation}
\end{figure*}