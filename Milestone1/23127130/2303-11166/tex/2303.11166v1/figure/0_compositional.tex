\begin{figure}[t!]
\vspace{-.2in}
% \begin{wrapfigure}{r}{0.5\textwidth}
\centering
\begin{subfigure}{0.325\textwidth}
\centering
\includegraphics[width=1.0\textwidth]{figure/opt.png}
\caption{
If $(l^{1}, l^{2}, l^{3}, l^{4}, l^{5})$ is an optimal $l^{5}$-reaching path, all the sub-paths are optimal for reaching $l^{5}$.
% its endpoint.
% , i.e., $l^{5}$. 
}\label{fig:comp1}
\end{subfigure}
\vspace{10mm}
\hfill
\begin{subfigure}{0.66\textwidth}
\centering
\includegraphics[width=1.0\textwidth]{figure/intuition.png}
\caption{
Previous works guide the agent using a $l^{2}$-reaching sub-path. Our work uses all the possible sub-paths that reach $l^{2}, l^{3}, l^{4}, l^{5}$.
}\label{fig:comp2}
\end{subfigure}
\vspace{-10mm}
\caption{Illustration of (a) optimal substructure property
and (b) sub-paths considered in previous works and our approach for guiding the training of a goal-reaching agent.
}
\label{fig:comp}
% \end{wrapfigure}
\end{figure}