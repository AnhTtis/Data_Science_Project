% \vspace{-0.05in}
\begin{figure*}[t]
\vspace{-.2in}
\centering
\begin{subfigure}[t]{0.99\textwidth}
\centering
  \includegraphics[width=0.55\linewidth]{figure/legend.pdf}
  \vspace{0.05in}
  \label{fig:Legend1}
\end{subfigure}
% \vspace{0.5in}
\hspace*{\fill}
\begin{subfigure}{0.325\textwidth}
\includegraphics[width=1.0\linewidth]{figure/2dreach_8seeds.pdf}
\caption{2DReach}
\end{subfigure}
\begin{subfigure}{0.325\textwidth}
\includegraphics[width=1.0\linewidth]{figure/Reacher_8seeds.pdf}
\caption{Reacher}
\end{subfigure}
\begin{subfigure}{0.325\textwidth}
\includegraphics[width=1.0\linewidth]{figure/Pusher_8seeds.pdf}
\caption{Pusher}
\end{subfigure}

\begin{subfigure}{0.325\textwidth}
\includegraphics[width=1.0\linewidth]{figure/L-shaped_AntMaze_8seeds.pdf}
\caption{L-shaped AntMaze}
\end{subfigure}
\begin{subfigure}{0.325\textwidth}
\includegraphics[width=1.0\linewidth]{figure/U-shaped_AntMaze_8seeds.pdf}
\caption{U-shaped AntMaze}
\end{subfigure}
\begin{subfigure}{0.325\textwidth}
\includegraphics[width=1.0\linewidth]{figure/Large_U-shaped_AntMaze_8seeds.pdf}
\caption{Large U-shaped AntMaze}
\end{subfigure}

\begin{subfigure}{0.325\textwidth}
\includegraphics[width=1.0\linewidth]{figure/S-shaped_AntMaze_8seeds.pdf}
\caption{S-shaped AntMaze}
\end{subfigure}
\begin{subfigure}{0.325\textwidth}
\includegraphics[width=1.0\linewidth]{figure/w-shaped_AntMaze_8seeds.pdf}
\caption{$\omega$-shaped AntMaze}
\end{subfigure}
\begin{subfigure}{0.325\textwidth}
\includegraphics[width=1.0\linewidth]{figure/Pi-shaped_Ant_Maze_8seeds.pdf}
\caption{$\Pi$-shaped AntMaze}
\end{subfigure}

\caption{
Learning curves on various continuous control tasks as measured on the success rate. We report mean and standard deviation, which are represented as solid lines and shaded regions, respectively, over eight runs.
We observe that \ALGname significantly improves the sample-efficiency of MSS on most tasks.
Note that HER and HIGL perform success rate of 0 for Large U-, S-, $\omega$-, and $\Pi$- shaped AntMaze, and \LthreeP performs success rate of 0 for Large U-shaped AntMaze.
GCSL performs success rate of 0 because it does not use planner in execution, which boosts performance in complex long-horizon tasks, so we compare ours with GCSL-variant that uses a planner 
% in execution 
in Figure~\ref{fig:ablation}. 
% Median success rate with 25 \% - 75 \% percentile over four random seeds in various continuous control tasks. We observe that applying \ALGname on top of an existing GCRL method, (i.e., MSS) improves sample-efficiency in a significant margin over many environments including navigation and fetch tasks.
}
\label{fig:main}
\end{figure*}