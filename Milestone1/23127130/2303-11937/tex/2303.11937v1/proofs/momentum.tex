\subsection{Proof of \Cref{lem:mom_bound}}\label{apx:mom_error}
Our first goal is to bound the term $\norm{\nabla F(\x_t) -\gb_t}$. Substituting $\gb_t = (1-\rho_t) \gb_{t-1} + \rho_t \nabla \tilde{F}(\x_t,\z_t)$ we have
\begin{align}
    \norm{\nabla F(\x_t) -\gb_t} = \norm{\nabla F(\x_t) - (1-\rho_t) \gb_{t-1} - \rho_t \nabla \tilde{F}(\x_t,\z_t)}
\end{align}
Adding and subtracting $(1-\rho_t)\nabla F(\x_{t-1})$ on the right hand side  and applying the triangle inequality gives us
\begin{align}
    \norm{\nabla F(\x_t) -\gb_t} = \norm{\rho_t(\nabla F(\x_t) - \nabla\tilde{F}(\x_t,\z_t)) - (1-\rho_t) (\nabla F(\x_t) - \nabla F(\x_{t-1})) + (1-\rho_t) (\nabla F(\x_{t-1}) -\gb_{t-1})} \nonumber \\
    \leq \rho_t \norm{\nabla F(\x_t) - \nabla\tilde{F}(\x_t,\z_t)} + (1-\rho_t)\norm{\nabla F(\x_t) - \nabla F(\x_{t-1})} + (1-\rho_t)\norm{\nabla F(\x_{t-1})-\gb_{t-1}}
\end{align}
Now using our assumptions on the smoothness of F and boundedness of our domain, we know that $\norm{\nabla F(\x_t) - \nabla F(\x_{t-1})} \leq L\norm{\x_t-\x_{t-1}} = L\norm{\frac{\vb_t}{T}} \leq \frac{LD}{T}$ and hence 
\begin{align}\label{eq:mom_recursive}
    \norm{\nabla F(\x_t) -\gb_t} & \leq
    \rho_t \norm{\nabla F(\x_t) - \nabla\tilde{F}(\x_t,\z_t)} + (1-\rho_t)\frac{LD}{T} + (1-\rho_t)\norm{\nabla F(\x_{t-1})-\gb_{t-1}}
\end{align}
Define $a_i := \norm{\nabla F(\x_i) - \nabla\tilde{F}(\x_i,\z_i)}$ as a random variable and recursively apply the inequality in \cref{eq:mom_recursive} to get
\begin{align}
    \norm{\nabla F(\x_t) -\gb_t} \leq  \sum_{i=1}^{t} \rho_i \left(\prod_{j=i+1}^{t}(1-\rho_j) \right) a_{i} + \frac{LD}{T} \sum_{i=1}^{t} \prod_{j=i}^{t}(1-\rho_j) 
\end{align}
Given any sequence of momentum terms $\rho_j$ that is monotonic non-increasing  we have the upper bound $\rho_j \leq \rho_1$ and $(1-\rho_j) \leq (1-\rho_t)$ and therefore
\begin{align}
    \norm{\nabla F(\x_t) -\gb_t} \leq \rho_1 \sum_{i=1}^{t}(1-\rho_t)^{t-i} a_{i} + \sum_{i=1}^{t}(1-\rho_t)^{t-i+1} \frac{LD}{T}
\end{align}
We now bound $S_T := \sum_{t=1}^T \norm{\nabla F(\x_t) -\gb_t}$ as
\begin{align} \label{eq:init_st_bound}
    S_T \leq \sum_{t=1}^T\rho_1 \sum_{i=1}^{t}(1-\rho_t)^{t-i} a_{i} + \frac{LD}{T} \sum_{t=1}^T \sum_{i=1}^t(1-\rho_t)^{t+1-i}
\end{align}
Using the following lemma from \cite{li2020high}, we can bound the first term on the right hand side
\begin{lemma}{(Li and Orabona 2020 \cite{li2020high}, Lemma 4)}\label{eq:switch}
$\forall T \geq 1$, it holds that
\begin{equation}
    \sum_{t=1}^{T}a_t \sum_{i=1}^t b_i = \sum_{t=1}^T b_t \sum_{i=t}^T a_i
\end{equation}
\end{lemma}
Applying  \cref{eq:switch} (step a) to the first term on the r.h.s. we have
\begin{align}\label{eq:series_A}
    \sum_{t=1}^T\rho_1 \sum_{i=1}^{t}(1-\rho_t)^{t-i} a_{i} &\leq \rho_1 \sum_{t=1}^T (1-\rho_t)^{t}\sum_{i=1}^{t}(1-\rho_t)^{-i} a_{i} \nonumber \\
    &\stackrel{(a)}{=} \rho_1 \sum_{t=1}^T (1-\rho_t)^{-t} a_{t} \sum_{i=t}^T (1-\rho_t)^{i} \nonumber \\
    &= \rho_1 \sum_{t=1}^T a_{t} \sum_{i=t}^T (1-\rho_t)^{i-t}
\end{align}

To get a simple upper bound on this quantity, one could again use the fact that $\rho_t$ is monotonic non-increasing to bound $1-\rho_t \leq  1-\rho_T$. Using the properties of geometric series this would lead to an upper bound of 
\begin{align}
    \rho_1 \sum_{t=1}^T a_{t} \sum_{i=t}^T (1-\rho_t)^{i-t} &\leq \rho_1 \sum_{t=1}^T a_{t} \sum_{i=t}^T (1-\rho_T)^{i-t} \leq \frac{\rho_1}{\rho_T}\sum_{t=1}^T a_{t}
\end{align}
If $\rho_1=\rho_T = \rho$ is a constant, then $\frac{\rho_1}{\rho_T}=1$ and we are just left with a sum of error terms. However, if our momentum term is of the form $\rho_t = \frac{1}{t^\alpha}$, then we have a coefficient $\frac{\rho_1}{\rho_T}=\mathcal{O}(t^\alpha)$ in this bound. In order to achieve a tighter constant bound in the general case of $\rho_t$ we need the following technical lemma:
\begin{lemma}\label{lem:series}
For $\alpha \in (0,1)$ non-inclusive, we have
\begin{equation}
    \sum_{t=1}^\infty (1-\frac{1}{t^\alpha})^t \leq \frac{1}{1-\alpha}\Gamma\left(\frac{1}{1-\alpha}\right)
\end{equation}
Where $\Gamma(z):=\int_0^\infty t^{z-1} e^{-t}dt$ is the Gamma function
\end{lemma}
\begin{proof}(\Cref{lem:series}) Using the well known inequality $(1-\frac{1}{n})^n \leq \frac{1}{e}$ and substituting $n=t^\alpha$ we get
\begin{equation}
    (1-\frac{1}{t^\alpha})^{t^\alpha} \leq e^{-1} \implies (1-\frac{1}{t^\alpha})^{t} \leq e^{-t^{1-\alpha}}
\end{equation}
Where the second inequality can be found by raising both sides to the $t^{1-\alpha}$ power. Since each term in our sequence is positive and decreasing we know that we can bound our series by
\begin{align}
    \sum_{t=1}^\infty (1-\frac{1}{t^\alpha})^t \leq \sum_{t=1}^\infty e^{-t^{1-\alpha}} < \int_0^\infty e^{-t^{1-\alpha}} dt
\end{align}
Lastly we use a change of variables $u = t^{1-\alpha}$, $du = (1-\alpha)t^{-\alpha}dt$ to get the integral into the following form
\begin{align}
    \sum_{t=1}^\infty (1-\frac{1}{t^\alpha})^t &< \frac{1}{1-\alpha}\int_0^\infty e^{-u}u^{\frac{\alpha}{1-\alpha}}du = \frac{1}{1-\alpha}\Gamma(\frac{1}{1-\alpha})
\end{align}
\end{proof}

Using \cref{lem:series} and denoting $K:=\frac{1}{1-\alpha}\Gamma(\frac{1}{1-\alpha})$ as a constant, we can bound the summation in \cref{eq:series_A} by 
\begin{equation}
    \rho_1 \sum_{t=1}^T a_{t} \sum_{i=t}^T (1-\rho_t)^{i-t} \leq \rho_1 K \sum_{t=1}^T a_{t}
\end{equation}

Again using \cref{eq:switch} and \cref{lem:series} we also have an upper bound on the second term in \cref{eq:init_st_bound}: 
\begin{align}
    \frac{LD}{T} \sum_{t=1}^T \sum_{i=1}^t(1-\rho_t)^{t+1-i} \leq LDK
\end{align}
Therefore our bound on $S_T$ simplifies to
\begin{equation}
    S_T \leq \rho_1K\sum_{t=1}^T a_{t} + LDK
\end{equation}
We know by our original assumption that $a_i$ is a zero-mean sub-gaussian random variable, therefore the weighted sum will also be sub-gaussian with variance proxy
\begin{equation}
    \bar{\sigma} ^2 = \rho_1^2K^2\sum_{i=1}^T \sigma_i^2 \leq  K^2\sigma^2T
 \end{equation}
 This means by Hoeffding's inequality
 \begin{equation}
     \mathbb{P}\left( \rho_1 K\sum_{t=1}^T a_{t} \geq \lambda\right) \leq \exp\left\{\frac{-\lambda^2}{2K^2\sigma^2T}\right\}
 \end{equation}
 Rearranging we also find the equivalent statement is true
 \begin{equation}
    \mathbb{P}\left(\sum_{t=1}^T a_{t} \leq  \sqrt{2K^2\sigma^2T\ln(1/\delta)}  \right) \geq 1-\delta
 \end{equation}
 
 Therefore we know that with probability greater than $1-\delta$
 \begin{equation}
     S_T \leq  \sqrt{2K^2\sigma^2T\log(1/\delta)} + LDK
 \end{equation}
 \qedsymbol


