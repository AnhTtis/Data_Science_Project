\subsection{Background on Martingale Sequences}
We first provide a brief review on Martingale sequences, which can be found in any standard text on stochastic processes such as \cite{chung2006concentration}. Given probability space $\Omega$ and probability distribution p, we first denote $\mathcal{F}$ as a $\sigma$-field on $\Omega$. We also denote the filtration $\mathbb{F}$ as a nested sequence of $\sigma$-subfields:
\begin{equation}
    \mathbb{F} := \{\mathcal{F}_t\}_{t\leq n} \quad s.t. \quad
    \mathcal{F}_0\subset\mathcal{F}_1\subset\dots \subset \mathcal{F}_n = \mathcal{F}
\end{equation}
We say that a sequence of random variables $\{X_t\}_{t=1}^n:=\{X_1, \dots, X_n\}$ is Martingale with respect to filtration $\mathbb{F}$ if the following properties hold:
\begin{align}
    \mathbb{E}[|X_t|]\leq \infty \nonumber\\
    \mathbb{E}[X_{t+1}|\mathcal{F}_t] = X_t
\end{align}
In other words, given past observations, the next random variable in the sequence is expected to take on the value of the previous one. 

A Martingale \emph{difference} sequence $\{Y_t\}_{t\leq n}$ with respect to filtration $\mathbb{F}$ is defined as having a conditional expectation of zero
\begin{equation}
    \mathbb{E}[Y_{t+1}|\mathcal{F}_t] = 0
\end{equation}
It is easy to see that given a Martingale sequence $\{Y_t\}_{t=1}^n$, one can construct a difference sequence by setting $Y_t=X_t-X_{t-1}$.

Lastly, we say a Martingale is $c$-lipschitz if $\forall t$
\begin{equation}
    |X_t - X_{t-1}| \leq c_t
\end{equation}
This is equivalent to saying all random variables $Y_t$ in the corresponding Martingale difference sequence are bounded.