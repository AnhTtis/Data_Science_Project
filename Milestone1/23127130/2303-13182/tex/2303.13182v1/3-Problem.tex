\newcommand{\Rmatrix}{\mbox{${\bf{R}}$}\xspace}
\newcommand{\PC}{\mbox{${\bf{PC}}$}\xspace}

\section{Overview}\label{sec:overview}
\subsection{Problem Statement}
We address the problem of grasp generation for unknown objects in a cluttered environment with a multi-finger robotic hand. We take a single-shot of point cloud $\bf{PC}$ from a given viewpoint as input and predict a set of high-quality grasps $\bf{G}=\{\bf{p}, \bf{q}\}$ where $\bf{p}$ is hand pose and $\textbf{q}$ is hand joint configuration.


\begin{figure}[H]
\centering
\includegraphics[width=0.4\textwidth]{./fig/threefinger.pdf}
\caption{A three-finger robotic hand.}
\label{fig:BarrettHand}
\end{figure}


In this paper, we use a \textit{BarrettHand} as our multi-finger dexterous hand, as shown in Fig.~\ref{fig:BarrettHand}.  The hand's pose consists of translation and orientation, denoted by
$
{\bf{p}}=\left\{x, y, z, \phi, \gamma, \varphi\right\}.
$
A hand joint configuration $\bf{q}$ consists of four inner joint angles, denoted by
$
{\bf{q}}=\left\{\theta_{0}, \theta_{1}, \theta_{2}, \theta_{3}\right\}.
$
Here, $\theta_{0}$ is the spread joint angle. $\theta_{1}$, $\theta_{2}$ and $\theta_{3}$ are the inner joint angles. The four outer joint angles $\theta_{11}$, $\theta_{21}$ and $\theta_{31}$ are dependent on their individual inner joints.


\subsection{Our Approach: Overview}

Fig.~\ref{fig:overview} illustrates the overview of our approach. For a cluttered environment, we aim to grasp an unknown object using a multi-finger robotic hand. First, we obtain a point cloud captured from any viewpoint. Second, an end-to-end network, CMG-Net, is trained using our synthetic grasp dataset for the three-finger robotic hand based on contact representation. Third, we use the trained CMG-Net to obtain the final hand pose and grasp configuration. Finally, we execute the grasping task in real-world scenarios.


\begin{figure*}[thbp]
\centering
\includegraphics[width=0.9\textwidth]{./fig/overview.pdf}
\caption{The overview of our approach.}
\label{fig:overview}
\end{figure*}

