\section{Grasp Representation}\label{sec:representation}
Due to the ambiguous and discontinuous distribution of multi-finger grasps, it is difficult to directly regress a 6-DoF hand pose and a 4-DoF hand configuration in such a high dimensional space\cite{DBLP:conf/iccv/MousavianEF19,DBLP:conf/icra/LundellCLVWRMK21}.  Therefore, we define a novel grasp representation that can generate a constrained grasp space for our learning-based data-driven method.

\textbf{Multi-finger grasp representation}: For human hand grasp, Cutkosky~\cite{DBLP:journals/trob/Cutkosky89} suggested a taxonomy of two main categories: power and precision.
We observe that power grasp is not friendly to grasp objects on the desktop. In many scenarios, it requires the palm and fingers to entirely enclose an object, but this can easily cause collisions with the table during the hand closing action. In a real-world scenario, this may cause damage to expensive robotic hands.
%
Therefore, we consider the precision grasp when executing a grasping task.
This requires each finger to have contact with the object and but the contact between the palm, and the object is unnecessary.
For a dexterous hand like BarrettHand, whose two finger joints are coupled by a single motor, the contact points with the object for the precision grasp are usually at the end of fingertips.
%

To describe the contact between the fingertips and the object, we fit a fingertip end with a circle (Fig.~\ref{fig:representation}-(a)). The circle center at each fingertip (i.e., fingertip center) and its radius can be computed by the given gripper model. We denote this circle as the \textit{fingertip circle}. The local vector from the fingertip center to the origin of the outer joint is denoted as the fingertip vector ${\mathbf{v}}_{finger}$.
Based on this denotation, we generate a novel multi-finger grasping representation from the contact point to the corresponding finger joint pose to the hand pose and finally to the hand configuration.
We assume that the fingertip model of the dexterous hand in the side view can be matched with a particular circle.

\begin{figure}[h]
\centering
\includegraphics[width=0.45\textwidth]{./fig/contact_1.pdf}
\caption{Grasp representation based on contacts. (a) A contact between a fingertip and an object to be grasped. The black dot represents the contact point. The white arrow indicates the surface normal vector at the contact. The red circle highlights fingertip circle, the red point is the fingertip center and a coordinate frame ${\mathbf{T}}_{ft}$ is attached to the fingertip. The yellow arrow indicates the fingertip vector ${\mathbf{v}}_{finger}$. The yellow point is outer-joint position and a coordinate frame ${\mathbf{T}}_{oj}$ is attached. (b) A grasp example for a three-finger robotic hand.}
\label{fig:representation}
\end{figure}

Given a suitable contact point predicted on the object surface, the fingertip center ${\mathbf{t}}_{ft}$ in the world coordinate frame is computed as
\begin{align}
{\mathbf{t}}_{ft} = {\mathbf{t}}_{contact} + r\mathbf{n},
\end{align}
where ${\mathbf{t}}_{contact}$ is the position of the contact point, $\mathbf{n}$ is its surface normal vector and ${r}$ is the circle radius.
To compute the hand pose from the fingertip center, we attach a coordinate frame at the fingertip center.
At the contact point, The ${z}$-axis is the same as the normal vector.
To avoid confusion, we denote this vector as $\mathbf{v}=\left[v_1, v_2, v_3\right]^{T}$. Then the rotation matrix $\mathbf{R}_{ft}$ can be calculated by
\begin{align}
\mathbf{R}_{ft}=\left[\mathbf{R}^{1}, \left[0,-v_3, v_2\right]^{T},  \mathbf{v}\right],
\end{align}
where $\mathbf{R}^{1}=\left[0,-v_3, v_2\right]^T \times \mathbf{v}$. Then the hand pose $\mathbf{T}_{ft}$ is represented by the rotation matrix $\mathbf{R}_{ft}$ and the position of fingertip center $\mathbf{t}_{ft}$.

Then we compute a fixed-length vector from the fingertip center to the outer-joint position ($\mathbf{v}_{finger}$), by finger projections $\emph{x} \in \left[-1, 1\right]$ and $\emph{y} \in \left[-1, 1\right]$. $\emph{x}$ and $\emph{y}$ are the projections of the fingertip vector onto the $\emph{x}$ axis and $\emph{y}$ axis respectively in the fingertip center coordinate frame. The outer-joint position $\mathbf{t}_{oj}$ can be computed by
\begin{align}
\mathbf{t}_{oj}=\|\mathbf{v}_{finger}\| \left({\mathbf{R}}_{ft} \cdot \left[x, y, z\right]^T\right) + {\mathbf{t}}_{ft},
\end{align}
where $z=\sqrt{1-x^{2}-y^{2}}$.
Please refer to Fig.~\ref{fig:representation}-(a) for illustration. Here, $\emph{z}$ axis of the outer-joint coordinate frame is orthogonal to the fingertip vector $\mathbf{v}_{finger}$ and the $\emph{z}$ axis $\mathbf{v}_{z}$ of the fingertip center coordinate frame. Then we have the rotation matrix $\mathbf{R}_{oj}$ of the outer-joint coordinate frame
\begin{align}
\mathbf{R}_{oj} = \left[\mathbf{v}_{finger},
\mathbf{R}^2
, \mathbf{v}_{finger} \times \mathbf{v}_{z} \right] \cdot \mathbf{R}_{0},
\end{align}
where $\mathbf{R}^{2}=\mathbf{v}_{finger} \times \left[ \mathbf{v}_{finger} \times \mathbf{v}_{z} \right]$. $\mathbf{R}_{0}$ is a transformation matrix with a fixed angle of rotation around the $\emph{z}$ axis.

To further obtain the gripper pose from the outer-joint coordinate frame, we divided the four DoFs of the Barrett hand into two main joints($\theta_{ms}, \theta_{m}$) and two supporting joints($\theta_{s1}, \theta_{s2}$) according to whether they are involved in the calculation of the grasp pose or not. For example, if the contact as shown in Fig.~\ref{fig:representation}-(b) corresponds to finger $3$, the main joint $\theta_{m}$ is the joint $\theta_{3}$ and $\theta_{ms}$ is the spread joint $\theta_{0}$. Then the supporting joints $\theta_{s1}$ and $\theta_{s2}$ are $\theta_{1}$ for finger 1 and $\theta_{2}$ for finger 2, respectively. Therefore, the grasp pose is computed as
\begin{align} \label{xy_to_pose}
\mathbf{T}_{pose} = \mathbf{T}_{oj} \cdot \left(\mathbf{T}\right)^{-1},
\end{align}
where $\mathbf{T}_{oj}$ is composed by $\mathbf{R}_{oj}$ and $\mathbf{t}_{oj}$.
$\mathbf{T}$ is the transformation matrix from the base coordinate frame of the gripper to the out-joint coordinate frame. $\mathbf{T}$ can be computed from the main joints($\theta_{ms}, \theta_{m}$) using forward kinematics. Finally, we can combine the supporting joints to obtain the final grasp configuration. As a result, 10 dimensional grasp space is reduced to 6 dimensions, represented by
\begin{equation}
\mathbf{G}=\{\emph{x}, \emph{y}, \theta_{ms}, \theta_{m}, \theta_{s1}, \theta_{s2}\}.
\end{equation}

Using such a grasp representation, we can significantly reduce the grasp space to be searched, which greatly facilitates the learning process by analyzing the characteristics of the multi-finger hand structure. Object shape and hand structure are connected through contacts. This allow more reasonable predictions.
