\section{Toward Consistent FID Evaluation}
\label{sec:consistent_eval}


\begin{table}[t]
    \caption{Consistent \textbf{FID and KID scores.} Lower is better.}
    \label{tab:app_results_hq}
    \resizebox{\linewidth}{!}{
    \centering
        \begin{tabular}{l|rr|rr}
            \toprule
            {} & \multicolumn{2}{r|}{Female to Male}
               & \multicolumn{2}{r}{Male to Female} \\
            {} & FID & KID ($\times100$) & FID & KID ($\times100$) \\ 
            \midrule
            \thename   & $29.7$ & $0.41 \pm 0.18$ & $24.2$ & $0.20 \pm 0.15$ \\
            \midrule
            \midrule
            {} & \multicolumn{2}{r|}{Dog to Cat}
               & \multicolumn{2}{r}{Cat to Dog} \\
            {} & FID & KID ($\times100$) & FID & KID ($\times100$) \\
            \midrule
            \thename   & $24.8$ & $0.23 \pm 0.13$ & $44.2$ & $0.76 \pm 0.23$ \\
            \midrule
            \midrule
            {} & \multicolumn{2}{r|}{Dog to Wild}
               & \multicolumn{2}{r}{Wild to Dog} \\
            {} & FID & KID ($\times100$) & FID & KID ($\times100$) \\
            \midrule
            \thename   & $18.7$ & $0.15 \pm 0.14$ & $44.7$ & $0.68 \pm 0.23$ \\
            \midrule
            \midrule
            {} & \multicolumn{2}{r|}{Cat to Wild}
               & \multicolumn{2}{r}{Wild to Cat} \\
            {} & FID & KID ($\times100$) & FID & KID ($\times100$) \\
            \midrule
            \thename   & $12.1$ & $0.01 \pm 0.09$ & $21.2$ & $0.20 \pm 0.13$ \\
            \bottomrule
        \end{tabular}
    }
\end{table}


The evaluation protocols used in the paper for \celebahq and \afhq are provided by \egsde~\cite{zhao2022egsde}.
Being ad-hoc, these protocols lack consistency and differ significantly
depending on the dataset.
A variety of different evaluation protocols makes the evaluation of the unpaired I2I methods rather quirky and error-prone.

As a step toward consistent FID evaluation, we provide results of an alternative,
but consistent evaluation protocol for \thename in \autoref{tab:app_results_hq}.
The consistent evaluation protocol uses only test splits (or validation splits if the test ones are not available) of each dataset to assess the quality of image translation.

The evaluation protocol begins with pre-processing all the 
 %test
 datasets in a consistent manner. The pre-processing step resizes images from their
original size down to $256 \times 256$ pixels (the same image size as is used for model training and inference). To avoid FID score inconsistencies created by aliasing artifacts~\cite{parmar2022aliased} we rely on the \texttt{Pillow} library~\cite{pillow} and Lanczos interpolation method.

Once the data pre-processing and image translation are done, the actual evaluation can begin. To perform the FID/KID score computation we use
a \texttt{torch-fidelity} package~\cite{obukhov2020torchfidelity}, which provides
a validated implementation of these metrics. The KID evaluation procedure depends on a free parameter -- the KID subset size. In this section, we choose the KID subset size of 100 for all the datasets.

The suggested evaluation protocol differs in a number of ways from the evaluation
protocols of the \afhq and \celebahq datasets of \egsde. It differs from the
ad-hoc \celebahq evaluation protocol~\cite{zhao2022egsde} because the latter compares FID scores between samples of validation and train splits, while the consistent version only uses validation split. The consistent evaluation protocol is also different from  the ad-hoc version of the \afhq one, which performs FID evaluation between translated images of size $256 \times 256$ and target images of size $512 \times 512$. The consistent protocol always uses pre-processed images of size $256 \times 256$.
