
\documentclass[10pt,twocolumn,letterpaper]{article}

\usepackage[review,algorithms]{wacv}      % To produce the REVIEW version for the algorithms track

\usepackage{graphicx}
\usepackage{amsmath}
\usepackage{amssymb}
\usepackage{booktabs}

\usepackage{times}
\usepackage{epsfig}
\usepackage{graphicx}
\usepackage[usenames,dvipsnames]{xcolor}
\usepackage{booktabs}
\usepackage{soul}
\usepackage{enumitem}

\usepackage[pagebackref,breaklinks,colorlinks]{hyperref}


\usepackage[capitalize]{cleveref}
\crefname{section}{Sec.}{Secs.}
\Crefname{section}{Section}{Sections}
\Crefname{table}{Table}{Tables}
\crefname{table}{Tab.}{Tabs.}


\def\wacvPaperID{*****} % *** Enter the WACV Paper ID here
\def\confName{WACV}
\def\confYear{2024}

\usepackage{tikz}
\usepackage{pgfplots}
\usepackage{ifthen}
\usetikzlibrary{
    fit,
    math,
    calc,
    arrows,
    shapes,
    shadows,
    fadings,
    external,
    plotmarks,
    arrows.meta,
    backgrounds,
    positioning,
    shadows.blur,
    intersections,
    pgfplots.groupplots,
    pgfplots.statistics,
    pgfplots.fillbetween,
}

\definecolor{b1}{HTML}{099CF0}
\definecolor{b2}{HTML}{0FA5C7}
\definecolor{yg}{HTML}{A1CB2E}
\definecolor{pg}{HTML}{5AAA4F}
\definecolor{or}{HTML}{F8A300}
\definecolor{pr}{HTML}{7000AD}
\definecolor{gr}{HTML}{EAEAF2}

\pgfdeclarelayer{back}
\pgfsetlayers{back,main}

\makeatletter
\pgfkeys{%
  /tikz/on layer/.code={
    \def\tikz@path@do@at@end{\endpgfonlayer\endgroup\tikz@path@do@at@end}%
    \pgfonlayer{#1}\begingroup%
  }%
}
\makeatother

\newcommand{\round}[2][2]{
    \pgfmathparse{#2}
    \pgfmathprintnumber[precision=#1, zerofill]{\pgfmathresult}
}

\makeatletter
\pgfplotsset{
    boxplot/hide outliers/.code={
        \def\pgfplotsplothandlerboxplot@outlier{}%
    }
}
\makeatother
\renewcommand{\nu}{\ensuremath{\mathbf{n}(\mathbf{u})}\xspace}  % the normal vector at pixel location \V{u}
\newcommand{\pu}{\ensuremath{\mathbf{p}(\mathbf{u})}\xspace}   % the 3d point correspoinding the pixel \V{u}
\newcommand{\du}{\ensuremath{d(\mathbf{u})}\xspace}  
\newcommand{\zu}{\ensuremath{z(\mathbf{u})}\xspace}
\newcommand{\eu}{\ensuremath{\mathbf{e}(\mathbf{u})}\xspace}
\newcommand{\up}{\ensuremath{\V{u}_{\V{p}}}\xspace}
\newcommand{\tup}{\ensuremath{\tilde{\V{u}}_{\V{p}}}\xspace}

\newcommand{\oz}{\ensuremath{\Omega_z}\xspace}  
\newcommand{\on}{\ensuremath{\Omega_n}\xspace}
\newcommand{\Nu}{\ensuremath{\mathcal{N}(\V{u})}\xspace}

\renewcommand{\ni}{normal integration\xspace}
\newcommand{\NI}{Normal Integration\xspace}
\newcommand{\dpe}{discrete Poisson's equation\xspace}
\newcommand{\Dpe}{Discrete Poisson's equation\xspace}


\newcommand{\z}{\ensuremath{\V{z}}\xspace}
\newcommand{\zs}{\ensuremath{\V{z}^*}\xspace}
\newcommand{\rz}{\ensuremath{\red{\V{z}}}\xspace}
\newcommand{\zt}{\ensuremath{\V{z}_{t}}\xspace}
\newcommand{\zto}{\ensuremath{\V{z}_{t+1}}\xspace}
\newcommand{\R}{\ensuremath{\mathbb{R}}\xspace}
\newcommand{\fz}{\ensuremath{f(\V{z})}\xspace}

\newcommand{\rt}{\ensuremath{\V{r}_{t}}\xspace}
\newcommand{\rto}{\ensuremath{\V{r}_{t+1}}\xspace}


\newcommand{\dup}{\ensuremath{\V{D}_u^{+}}\xspace}
\newcommand{\dun}{\ensuremath{\V{D}_u^{-}}\xspace}
\newcommand{\dvp}{\ensuremath{\V{D}_v^{+}}\xspace}
\newcommand{\dvn}{\ensuremath{\V{D}_v^{-}}\xspace}
\newcommand{\nx}{\ensuremath{\V{n}_x}\xspace}
\newcommand{\ny}{\ensuremath{\V{n}_y}\xspace}
\newcommand{\nz}{\ensuremath{\V{n}_z}\xspace}
\newcommand{\Nz}{\ensuremath{\V{N}_z}\xspace}

\newcommand{\ft}{\ensuremath{F(\red{\V{z}};\V{z}_t)}\xspace}
\newcommand{\ftt}{\ensuremath{F(\V{z}_t;\V{z}_t)}\xspace}
\newcommand{\fto}{\ensuremath{F(\V{z}_{t+1};\V{z}_t)}\xspace}

\newcommand{\dpu}{\ensuremath{\partial_u \V{p}}\xspace}
\newcommand{\dpv}{\ensuremath{\partial_v \V{p}}\xspace}

\renewcommand{\u}{\ensuremath{\V{u}}\xspace}
\newcommand{\dzdu}{\ensuremath{\partial_u z}\xspace}
\newcommand{\dzdv}{\ensuremath{\partial_v z}\xspace}
\newcommand{\dztdu}{\ensuremath{\partial_u \tilde{z}}\xspace}
\newcommand{\dztdv}{\ensuremath{\partial_v \tilde{z}}\xspace}
\newcommand{\dzpdu}{\ensuremath{\partial_{u}^{+} z}\xspace}
\newcommand{\dzpdv}{\ensuremath{\partial_{v}^{+} z}\xspace}
\newcommand{\dzndu}{\ensuremath{\partial_{u}^{-} z}\xspace}
\newcommand{\dzndv}{\ensuremath{\partial_{v}^{-} z}\xspace}

\newcommand{\dzpduv}{\ensuremath{\partial_{\{u,v\}}^{+} z}\xspace}
\newcommand{\dznduv}{\ensuremath{\partial_{\{u,v\}}^{-} z}\xspace}
\newcommand{\dzduv}{\ensuremath{\partial_{\{u,v\}} z}\xspace}

\newcommand{\dupz}{\ensuremath{\Delta_{u}^{+} z}\xspace}
\newcommand{\dunz}{\ensuremath{\Delta_{u}^{-} z}\xspace}
\newcommand{\dvpz}{\ensuremath{\Delta_{v}^{+} z}\xspace}
\newcommand{\dvnz}{\ensuremath{\Delta_{v}^{-} z}\xspace}

\newcommand{\nuv}{\ensuremath{\V{n}(u,v)}\xspace}
\newcommand{\zuv}{\ensuremath{z(u,v)}\xspace}
\newcommand{\puv}{\ensuremath{\V{p}(u,v)}\xspace}

\newcommand{\halfpi}{\ensuremath{\pm {\pi \over 2}}\xspace}


\newcommand{\curve}{\ensuremath{\mathbb{S}}\xspace}
\newcommand{\zenith}{zenith\xspace}
\newcommand{\surface}{\ensuremath{\mathcal{M}}\xspace}
\newcommand{\visibility}{\ensuremath{\Phi_{i}}\xspace}
\newcommand{\point}{\ensuremath{\V{x}}\xspace}
\newcommand{\normal}{\ensuremath{\V{n}}\xspace}
\newcommand{\tangent}{\ensuremath{\V{t}}\xspace}
\newcommand{\cameraNum}{\ensuremath{C}\xspace}
\newcommand{\cameraCenter}{\ensuremath{\V{o}_{i}}\xspace}
\newcommand{\viewDirection}{\ensuremath{\V{v}}\xspace}
\newcommand{\batchsize}{\ensuremath{P}\xspace}
\newcommand{\mask}{\ensuremath{O}\xspace}
\newcommand{\projectedTangentVector}{projected tangent vector\xspace}
\newcommand{\projectedTangentVectors}{projected tangent vectors\xspace}
\newcommand{\stackedTangentVectors}{\ensuremath{\V{T}(\point)}\xspace}
\newcommand{\diligentmv}{\mbox{DiLiGenT-MV}\xspace}
\newcommand{\diligent}{DiLiGenT}
\newcommand{\loss}{\mathcal{L}\xspace}
\newcommand{\opticalAxis}{\ensuremath{\V{e}_{z}\xspace}}
\newcommand{\opticalAxisViewI}{\ensuremath{\V{e}_{z_{i}}}\xspace}
\newcommand{\opticalAxisMatrix}{\ensuremath{\V{C}}\xspace}
\newcommand{\ms}{Mumford-Shah integrator\xspace}
\newcommand{\made}{MADE\xspace}

\newcommand{\pandora}{\mbox{PANDORA}\xspace}
\newcommand{\psnerf}{\mbox{PS-NeRF}\xspace}
\newcommand{\sdps}{\mbox{SDPS}\xspace}
\newcommand{\uanet}{\mbox{UA-MVPS}\xspace}
\newcommand{\rmvps}{\mbox{R-MVPS}\xspace}
\newcommand{\bmvps}{\mbox{B-MVPS}\xspace}
\newcommand{\volsdf}{\mbox{VolSDF}\xspace}
\newcommand{\unisurf}{\mbox{UNISURF}\xspace}


\newcommand{\mvas}{MVAS\xspace}

\newcommand{\tsc}{\mbox{TSC}\xspace}

\newcommand{\pointOne}{\ensuremath{\point_1}\xspace}
\newcommand{\pointTwo}{\ensuremath{\point_2}\xspace}
\newcommand{\pointsetOne}{\ensuremath{\chi_{1}}\xspace}
\newcommand{\pointsetTwo}{\ensuremath{\chi_{2}}\xspace}
\newcommand{\fscoreThreshold}{\ensuremath{\tau}\xspace}
\newcommand{\chamferDist}{\ensuremath{d(\pointsetOne, \pointsetTwo)}\xspace}
\newcommand{\precision}{\ensuremath{\mathcal{P}}\xspace}
\newcommand{\recall}{\ensuremath{\mathcal{R}}\xspace}
\newcommand{\fscore}{\ensuremath{\mathcal{F}}\xspace}

\newcommand{\phaseangle}{\ensuremath{\hat{\phi}}\xspace}
\newcommand{\azimuthangle}{\ensuremath{\phi}\xspace}

\newcommand{\colorbar}[3]{
\begin{tabular}[t]{@{}l@{}l@{}}
	\includegraphics[height=#1\linewidth,width=0.5em]{colorbar.pdf} & 
	\begin{tabular}[b]{@{}l}
		#2 \\ [#3pt]
		$0$
	\end{tabular}
\end{tabular}
}


\tikzexternalize[prefix=fig/external/]

\begin{document}


\title{Rethinking CycleGAN: Improving Quality of GANs for Unpaired Image-to-Image Translation}

\author{First Author\\
Institution1\\
Institution1 address\\
{\tt\small firstauthor@i1.org}
\and
Second Author\\
Institution2\\
First line of institution2 address\\
{\tt\small secondauthor@i2.org}
}
\maketitle


One way we used to measure the consistency of translation to the input
is by calculating the distance between face landmarks of a translated 
figure to the source figure. Face landmarks are a set of key points on 
a human face from which much information could be extracted. 
On the macroscopic level, the landmarks contain information on the face 
position, orientation, and structure. On the microscopic level, they also
capture subtler cues such as age, gender, and emotional expression. Using 
face landmarks as a measurement of consistency also eliminates the influence
of the addition of hairs -- a modification that is logical and even 
desirable in male-to-female translation but typically leads to large 
differences in pixel-wise consistency assessments. 

For this work, we use \textsf{mediapipe (v0.10.1)} for generating 
face landmarks. The landmarks form an ordered set of $468$ points. 
On the male-to-female translation with \celebahq, we calculate the landmarks
for the source image, the translation produced by \egsde and \egsdeDG,
and those produced by \thename and \thename without identity loss 
(denoted by \thename$^\dag$). Then, we calculated the distance between
the landmarks of the translation to the source. A smaller distance
indicates a more consistent translation. Among the $1000$ test examples 
in the \celebahq dataset, \textsf{mediapipe} successfully generates
landmarks for the source and all the translations on $991$ images. 
For the remaining $9$ images, it fails to generate landmarks on either 
the input or at least one of the translations. We show three examples 
and the box plots of the $\ell_1$ distances between
the translations and the source in Figure~\ref{fig:grid_face_landmarks}.
We can see that, \thename exhibits more consistency 
according to this measurement. Considering that both \thename 
also produce translations with lower FID and KID scores 
than the benchmarking algorithms, we hypothesize that the \thename
manages to capture the nuances of the two domains and translate
in a more subtle but efficient manner.


\draw[step=1cm, grid-grey,very thick](1,1)grid(8,6); 

\foreach \x in {1,...,8}
	\foreach \y in {1,...,6}
		\draw[grid-grey, fill=white, very thick](\x,\y) circle(0.07cm);
\draw[-stealth, line width=0.85mm] (1.5, 5.9) -- (1.5, 5.1);
\begin{figure}[ht]
    \centering
    \begin{tikzpicture}
        \def\folder{fig/face_landmarks}
        \def\index{1}
        
        \def\width{.12\textwidth}
        \def\expand{1.02}

        \node[inner sep=0, fill=white] at (0, 0) {
            \begin{tikzpicture}
                \foreach \name/\label [count=\col] in {input/input, 
                                                       egsde1/\egsde, 
                                                       egsde2/\egsdeDG, 
                                                       uvcgan/\thename} {
                    % image
                    \def\fname{\folder/\name_\index.jpg}
                    \coordinate (C) at (\width * \expand * \col, 0);
        		  \node[inner sep=0] (image_\col) at (C) {\includegraphics[width=\width]{\fname}};
                    % label
                    \tikzmath{coordinate \C;\C = (image_\col.north east) - (image_\col.south west);}
                    \node[header, minimum width=\Cx, minimum height=.2 * \width, anchor=south west] at (image_\col.north west) {\label};
                }
                \foreach \name/\label [count=\col] in {input/input, 
                                                       egsde1/\egsde, 
                                                       egsde2/\egsdeDG, 
                                                       uvcgan/\thename} {
                    % image
                    \def\fname{\folder/\name_\index_lm.jpg}
                    \coordinate (C) at (\width * \expand * \col, -\width * \expand);
        		  \node[inner sep=0] (image_\col) at (C) {\includegraphics[width=\width]{\fname}};
                }
            \end{tikzpicture}
        };
    \end{tikzpicture}
    \caption{Caption}
    \label{fig:face_landmark}
\end{figure}


\begin{table*}[ht]
    \centering
    \caption{\textbf{Face landmarks}}
    \vskip-3mm
    \tikzexternaldisable
    \tikzsetnextfilename{grid_face_landmarks}
    \resizebox{\textwidth}{!}{
    \begin{tikzpicture}[fill=white]
        \node[inner sep=0] (G1) at (0, 0) {
            \def\folder{fig/face_landmarks}
            \def\width{.104\textwidth}
            \def\expand{1.03}
            \gridlm{0, 1, 2, 3}
                   {input/input, 
                    egsde1/\egsde, 
                    egsde2/\egsdeDG, 
                    uvcgan/\thename,
                    uvcgan_const/\thename-C}
        };
        \node[inner sep=0, anchor=north] (G2) at ([yshift=-5pt]G1.south) {
            \def\folder{fig/face_landmarks}
            \def\width{.104\textwidth}
            \def\expand{1.03}
            \gridlm[4]{4, 5, 6, 7}
                   {input/input, 
                    egsde1/\egsde, 
                    egsde2/\egsdeDG, 
                    uvcgan/\thename,
                    uvcgan_const/\thename-C}
        };
    \end{tikzpicture}
    }
    
    \label{fig:grid_face_landmarks}

\end{table*}


\end{document}