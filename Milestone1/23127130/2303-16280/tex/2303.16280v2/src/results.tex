\section{Results}

\begin{figure*}[t]
    \centering
    \tikzexternaldisable
    \tikzsetnextfilename{grid_LQ}
    \draw[step=1cm, grid-grey,very thick](1,1)grid(8,6); 

\foreach \x in {1,...,8}
	\foreach \y in {1,...,6}
		\draw[grid-grey, fill=white, very thick](\x,\y) circle(0.07cm);
\draw[-stealth, line width=0.85mm] (1.5, 5.9) -- (1.5, 5.1);
    \begin{tikzpicture}
        
        \def\width{.1\textwidth}
        \def\expand{1.03}
        \def\folder{fig/grid_LQ}
        
        \node[inner sep=0, fill=white] at (0, 0) {
            \begin{tikzpicture}
                \node[inner sep=0] (G1) at (0, 0) {\gridWithHeader[1]{selfie2anime}{0, 3, 6}{input/input, UVCGAN2/\thename}{\selfieanime}};
            	\node[inner sep=0, anchor=west] (G2) at ([xshift=.03 * \width]G1.east) {\gridWithHeader{male2female}{4, 1, 9}{input/input, UVCGAN2/\thename}{\malefemale}};
            	\node[inner sep=0, anchor=west] (G3) at ([xshift=.03 * \width]G2.east) {\gridWithHeader{rmvGlasses}{4, 7, 8}{input/input, UVCGAN2/\thename}{\rmvGlasses}}; 
            \end{tikzpicture}
        };
    \end{tikzpicture}
    \caption{\textbf{Sample translations for \celeba and \anime. } Translations produced by \ugatit, \uvcgan, and \thename for three tasks: \selfieanime, \malefemale, and \rmvGlasses. The full grid with all benchmarking results and those for the three opposite translations can be found in \autoref{sec:additionalTransSamples}.}
    \label{fig:grid_LQ}
\end{figure*}

\begin{figure*}[t]
    \centering
    \tikzexternaldisable
    \tikzsetnextfilename{grid_HQ}
    \draw[step=1cm, grid-grey,very thick](1,1)grid(8,6); 

\foreach \x in {1,...,8}
	\foreach \y in {1,...,6}
		\draw[grid-grey, fill=white, very thick](\x,\y) circle(0.07cm);
\draw[-stealth, line width=0.85mm] (1.5, 5.9) -- (1.5, 5.1);
    \begin{tikzpicture}
    
        \def\width{.1\textwidth}
        \def\expand{1.03}
        \def\folder{fig/grid_HQ}
    
        \node[inner sep=0, fill=white] at (0, 0) {
            \begin{tikzpicture}
                \node[inner sep=0] (G1) at (0, 0) {\gridWithHeader[1]{male2female}{1, 5, 8}{input/input, EGSDE1/\egsde, EGSDE2/\egsdeDG, UVCGAN2wo/\thename}{\malefemale}};
            	\node[inner sep=0, anchor=west] (G2) at ([xshift=.03 * \width]G1.east) {\gridWithHeader{cat2dog}{4, 7, 8}{input/input, EGSDE1/\egsde, EGSDE2/\egsdeDG, UVCGAN2wo/\thename}{\catdog}};
            	\node[inner sep=0, anchor=west] (G3) at ([xshift=.03 * \width]G2.east) {\gridWithHeader{wild2dog}{1, 4, 5}{input/input, EGSDE1/\egsde, EGSDE2/\egsdeDG, UVCGAN2wo/\thename}{\wilddog}}; 
            \end{tikzpicture}
        };
    \end{tikzpicture}
    \caption{\textbf{Sample translations for \celebahq and \afhq. } Translations for three tasks: \malefemale, \catdog, and \wilddog. More translations for these three tasks and those for \wildcat can be found in \autoref{sec:additionalTransSamples}.}
    \label{fig:grid_HQ}
\end{figure*}

\subsection{Metrics of Realism and Faithfulness}

There are two dimensions along which the unpaired I2I style transfer models can be evaluated:
\textit{Faithfulness} and \textit{Realism}. Faithfulness captures the degree of similarity between the
source and its translated image at an individual level. 
Realism attempts to estimate the overlap of the distributions
of the translated images and the ones in the target.

In terms of realism, image translation quality is commonly judged according to the
FID~\cite{heusel2017gans} and kernel inception distance
(KID)~\cite{binkowski2018demystifying} metrics. 
Both metrics measure the
distance between the distributions of the latent Inception-v3~\cite{szegedy2016rethinking}
features extracted from samples of the translated and target images. Smaller FID and KID values
indicate more realistic images.

Early GAN-based works (e.g.,~\cite{zhu2017unpaired,nizan2020breaking,zhao2020unpaired,kim2019u}) do not
explicitly evaluate the faithfulness of the translation. 
To the best of our knowledge, there is no widely accepted faithfulness metric available. 
Some works~\cite{zhao2022egsde} try to employ simple pixel-wise $L_2$, peak-signal-to-noise ratio (PSNR), or structural similarity index measure (SSIM)~\cite{wang2004image} scores to capture the agreement between the source and translation. 
Yet, it is unclear how well these pixel-wise metrics relate to the perceived image faithfulness,
and we explore more advanced alternatives in \autoref{sec:better_faithfulness}.




\subsection{Evaluation Protocol}

Evaluation protocols differ drastically between different papers (see \autoref{sec:RemarksMetricEvalConsistency}). %(see SM-Section 3).
This makes the direct comparison of the translation quality metrics extremely challenging.
For the fairness of comparisons with older works, we follow different evaluation protocols,
depending on the dataset.

\paragraph{\celeba and \anime.}
When evaluating the quality of translation on the \celeba \malefemale, \celeba \glasses Removal, and \anime datasets,
we use the evaluation protocol of \uvcgan~\cite{torbunov2023uvcgan}, which uniformized FID/KID evaluation across multiple datasets and models, allowing for a simple FID/KID comparison.
For the actual FID/KID evaluation, we rely on \texttt{torch-fidelity}~\cite{obukhov2020torchfidelity}, which provides
a validated implementation of these metrics.

The actual evaluation protocol for \celeba and \anime relies only on test splits to perform the FID/KID
evaluation. For the \celeba dataset, we use KID subset size of $1000$. 
For the \anime dataset, we use the KID subset size of $50$. 
We use unprocessed images of size $256 \times 256$ when evaluating on the \anime dataset. 
For the \celeba dataset, we apply a simple pre-processing to both domains: resizing
the smaller side to $256$ pixels then taking a center crop of size $256 \times 256$.

\paragraph{\celebahq and \afhq.} 
\egsde~\cite{zhao2022egsde} has evaluated multiple models on the \celebahq and \afhq datasets under
similar conditions. To compare our results to \egsde, we replicate its evaluation protocol for \celebahq 
and \afhq.
For the \afhq dataset, we evaluate FID and KID scores between the translated images of size $256 \times 256$ and
the target images of size $512 \times 512$ from the validation split.
For the \celebahq dataset, we evaluate the FID/KID scores between the translated images of size $256 \times 256$
and the downsized target images of size $256 \times 256$ from the train split. We perform the same channel
standardization as \egsde with $\mu = (0.485, 0.456, 0.406)$ and $\sigma = (0.229, 0.224, 0.225)$.
To ensure full consistency, we use the reference evaluation code provided by \egsde~\cite{egsderef}. 
\autoref{sec:consistent_eval} %SM-Section 5 
provides results of an alternative evaluation protocol that is uniform across all the datasets.


\subsection{Quantitative Results}

\paragraph{\celeba and \anime.}
\autoref{tab:results_lq} shows a comparison of the \thename (trained without piwel-wise consistency loss) performance against
\aclgan~\cite{zhao2020unpaired}, \council~\cite{nizan2020breaking},
\cyclegan~\cite{zhu2017unpaired}, \ugatit~\cite{kim2019u}, and
\uvcgan~\cite{torbunov2023uvcgan}.
The competitor models' performance is obtained from the \uvcgan paper~\cite{torbunov2023uvcgan}.
According to \autoref{tab:results_lq}, \thename outperforms all competitor models in all translation directions, except \animeselfie. The degree of improvement ranges from about $5\%$
in terms of FID on the \selfieanime translation to around $51\%$ on the \malefemale
translation. Likewise, there is a significant improvement in the KID scores from about
$13\%$ on \animeselfie to $79\%$ on \malefemale. This improvement demonstrates
the effectiveness of modern additions to the traditional \cyclegan architecture. \autoref{fig:grid_LQ} provides some translation samples, while more can be found in \autoref{sec:additionalTransSamples}. %SM-Section 6.


\begin{table}[!ht]
    \caption{\textbf{FID and KID scores.} Lower is better.}
    \vskip-3mm
    \label{tab:results_lq}
    \resizebox{\linewidth}{!}{
    \centering
        \begin{tabular}{l|rr|rr}
            \toprule
            {} & \multicolumn{2}{r|}{Selfie-to-Anime} & \multicolumn{2}{r}{Anime-to-Selfie} \\
            {} & FID & KID ($\times100$) & FID & KID ($\times100$) \\ 
            \midrule
            \aclgan   & $99.3$ & $3.22\pm0.26$ & $128.6$ & $3.49\pm0.33$ \\
            \council  & $91.9$ & $2.74\pm0.26$ & $126.0$ & $2.57\pm0.32$ \\
            \cyclegan & $92.1$ & $2.72\pm0.29$ & $127.5$ & $2.52\pm0.34$ \\
            \ugatit   & $95.8$ & $2.74\pm0.31$ & $\mathbf{108.8}$ & \underline{$1.48\pm0.34$} \\
            \uvcgan   & \underline{$79.0$} & \underline{$1.35 \pm 0.20$}
                      & $122.8$ & $2.33\pm0.38$ \\
            \midrule
            \thename  & $\mathbf{75.8}$ & $\mathbf{1.18 \pm 0.28}$
                      & $\underline{113.8}$ & $\mathbf{1.26 \pm 0.23}$ \\
            \midrule
            \midrule
            {} & \multicolumn{2}{r|}{Male-to-Female} & \multicolumn{2}{r}{Female-to-Male} \\
            {} & FID & KID ($\times100$) & FID & KID ($\times100$) \\
            \midrule
            \aclgan   & \underline{$9.4$} & \underline{$0.58\pm0.06$} & $19.1$ & $1.38\pm0.09$ \\
            \council  & $10.4$ & $0.74\pm0.08$ & $24.1$ & $1.79\pm0.10$ \\
            \cyclegan & $15.2$ & $1.29\pm0.11$ & $22.2$ & $1.74\pm0.11$ \\
            \ugatit   & $24.1$ & $2.20\pm0.12$ & $15.5$ & $0.94\pm0.07$ \\
            \uvcgan   & $9.6$ & $0.68\pm0.07$ & \underline{$13.9$} & \underline{$0.91\pm0.08$} \\
            \midrule
            \thename  & $\mathbf{4.7}$ & $\mathbf{0.14 \pm 0.02}$ & $\mathbf{7.6}$ & $\mathbf{0.24 \pm 0.02}$ \\
            \midrule
            \midrule
            {} & \multicolumn{2}{r|}{Remove Glasses} & \multicolumn{2}{r}{Add Glasses} \\
            {} & FID & KID ($\times100$) & FID & KID ($\times100$) \\
            \midrule
            \aclgan   & $16.7$ & $0.70\pm0.06$ & $20.1$ & $1.35\pm0.14$ \\
            \council  & $37.2$ & $3.67\pm0.22$ &    $19.5$ & $1.33\pm0.13$ \\
            \cyclegan & $24.2$ & $1.87\pm0.17$ & $19.8$ & $1.36\pm0.12$ \\
            \ugatit   & $23.3$ & $1.69\pm0.14$ & $19.0$ & $1.08\pm0.10$ \\
            \uvcgan   & \underline{$14.4$} & \underline{$0.68\pm0.10$}
                      & \underline{$13.6$} & \underline{$0.60\pm0.08$} \\
            \midrule
            \thename  & $\mathbf{10.6}$ & $\mathbf{0.27 \pm 0.06}$ & $\mathbf{11.3}$ & $\mathbf{0.34 \pm 0.07}$ \\
            \midrule
            \bottomrule
        \end{tabular}
    }
    \vskip-3mm
\end{table}


\paragraph{\celebahq and \afhq.}
\autoref{tab:results_hq} compares the results of the \thename evaluation against \cut~\cite{park2020contrastive}, \ilvr~\cite{choi2021ilvr}, \sdedit~\cite{meng2021sdedit},
and two versions of the \egsde~\cite{zhao2022egsde}.
In particular, this table compares two versions of the \thename: \thename and \mbox{\thename-C}. \mbox{\thename-C} is a version trained with a pixel-wise consistency loss and $\lambda_\text{consist} = 0.2$.
The competitor models' performance is extracted from \egsde~\cite{zhao2022egsde}. 


\begin{table}[t]
    \caption{\textbf{FID, PSNR, and SSIM scores.}}
    \label{tab:results_hq}
    \centering
    \resizebox{.72\linewidth}{!}{
    \begin{tabular}{l|rrr}
        \toprule
        {} & \multicolumn{3}{c}{Male to Female}  \\
        {} & FID$\downarrow$ & PSNR$\uparrow$ & SSIM$\uparrow$ \\ 
        \midrule
        \cut             & $46.61$ & $19.87$ & $\mathbf{0.74}$ \\
        \ilvr            & $46.12$ & $18.59$ & $0.510$ \\
        \sdedit          & $49.43$ & $20.03$ & $0.572$ \\
        \egsde  & $41.93$ & \underline{$20.35$} & $0.574$ \\
        \egsdeDG & $30.61$ & $18.32$ & $0.510$ \\
        \midrule
        \thename  & \underline{$17.65$} & $19.44$ & \underline{$0.681$} \\
        \thename-C & $\mathbf{17.34}$ & $\mathbf{21.18}$ & $\mathbf{0.738}$ \\
        \midrule
        \midrule
        {} & \multicolumn{3}{c}{Cat to Dog} \\
        {} & FID$\downarrow$ & PSNR$\uparrow$ & SSIM$\uparrow$ \\
        \midrule
        \cut             & $76.21$ & $17.48$ & \underline{$0.601$} \\
        \ilvr            & $74.37$ & $17.77$ & $0.363$ \\
        \sdedit          & $74.17$ & \underline{$19.19$} & $0.423$ \\
        \egsde  & $65.82$ & $\mathbf{19.31}$ & $0.415$ \\
        \egsdeDG & \underline{$51.04$}  & $17.17$ & $0.361$ \\
        \midrule
        \thename & $\mathbf{44.76}$ & $15.55$ & $0.562$ \\
        \thename-C & $52.48$ & $18.30$ & $\mathbf{0.638}$ \\
        \midrule
        \midrule
        {} & \multicolumn{3}{c}{Wild to Dog} \\
        {} & FID$\downarrow$ & PSNR$\uparrow$ & SSIM$\uparrow$ \\
        \midrule
        \cut             & $92.94$ & $17.2$ & \underline{$0.592$} \\
        \ilvr            & $75.33$ & $16.85$ & $0.287$ \\
        \sdedit          & $68.51$ & $17.98$ & $0.343$ \\
        \egsde   & $59.75$ & \underline{$18.14$} & $0.343$ \\
        \egsdeDG & \underline{$50.43$} & $16.40$ & $0.300$ \\
        \midrule
        \thename & $\mathbf{45.56}$ & $15.59$ & $0.551$ \\
        \thename-C & $55.61$ & $\mathbf{18.65}$ & $\textbf{0.631}$ \\
        \midrule
        \bottomrule
    \end{tabular}
    }
\end{table}

\autoref{tab:results_hq} shows that \thename achieves the best translation quality, according
to the FID scores, with improvements ranging from $10\%$ on \wilddog translation to
$43\%$ on \malefemale translation. The addition of the consistency loss allows the \thename-C
model to improve its pixel-wise PSNR and SSIM metrics---but at the expense of the FID score on \afhq translation.
\thename and \thename-C achieve competitive SSIM scores but lose in terms of the PSNR ratio
to the other models. However, as previously noted~\cite{zhang2018unreasonable}, pixel-wise measures PSNR and SSIM are not good metrics to judge perceptual image
faithfulness.
Overall, the gains in SSIM and PNSR scores provided
by the consistency loss to \thename-C do not seem to outweigh the associated FID losses.

Finally, it may be instructive to examine the diversity of the translated images.
We compare the diversity of the generated images according to a pairwise LPIPS distance~\cite{zhang2018unreasonable}, following an approach similar to~\cite{liu2021divco}.
To calculate the LPIPS distance, we use a VGG-based implementation (v0.0) in a consecutive pair mode.
\autoref{tab:results_hq} shows that the \thename produces a much larger diversity of the generated images compared to the \egsde variants.
It also indicates that the presence of the pixel-wise consistency loss
leads to a small but repeatable increase in image diversity.

\subsection{Ablation Study}

\section{Hyperparameter Sensitivity}
\label{appendix:hyper}
\noindent\textbf{Experiment Settings.} To explore the impact of the hyperparameters and the structure of LOP, we implement the LOP with different values of $B$ and different architectures. 
Table \ref{tab:ablation_normal} report the AP and precision of PointRCNN when our defense is deployed with different settings under the normal circumstance. Fig.\ref{fig:ablation_atk} report the defense effectiveness and the precision of PointRCNN when our defense is deployed with different settings under attacks.

%%%%%%% BEGIN Ablation
\begin{figure*}[t]
    \centering
    \includegraphics[width=0.9\textwidth]{figs/PointRCNN_ablation.png}
    \caption{The ASR of different attacks and the precision of PointRCNN when deploying the LOP with different values of $B$ and different model structures on PointRCNN.
    % \lyf{The precision and ASR metrics are represented in triangles and circles respectively, while the structures of LOP are distinguished in colors.}
    % The blue horizontal dotted lines in (a), (b) and (c) all represent the precision of PointRCNN on cars in the normal circumstances.
    }
    \label{fig:ablation_atk}
\end{figure*}


\begin{table}[ht]
  \centering
  \vspace{-0.2in}
  \caption{The AP and precision of PointRCNN equipped with the LOP with different $B$ and different model structures.}
\scalebox{0.7}{
    \begin{tabular}{lcccc}
    \toprule
          % & \multicolumn{4}{c}{PointRCNN} \\
\cmidrule{2-5}          & AP    & \multicolumn{3}{c}{Precision} \\
\cmidrule(lr){2-2} \cmidrule(lr){3-5}          & Car   & Car   & Pedestrian & Cyclist \\
    \midrule
    w/o. defense  & 75.13\% & 75.04\% & 47.08\% & 56.87\% \\
    \midrule
    Ours(PointNet, B=0.2) & 75.96\% & 76.49\% & 50.70\% & 58.66\% \\
    Ours(PointNet, B=0.3) & 76.21\% & 77.15\% & \textbf{50.94\%} & 58.93\% \\
    Ours(PointNet, B=0.4) & \textbf{76.50\%} & 78.05\% & 50.49\% & 61.59\% \\
    Ours(PointNet, B=0.5) & 76.49\% & 79.29\% & 49.63\% & 61.92\% \\
    Ours(PointNet, B=0.6) & 76.37\% & \textbf{80.03\%} & 49.43\% & \textbf{63.87\%} \\
    \midrule
    Ours(DGCNN, B=0.2) & 76.44\% & 77.56\% & 51.04\% & 61.03\% \\
    Ours(DGCNN, B=0.3) & 76.66\% & 78.36\% & \textbf{51.31\%} & 61.72\% \\
    Ours(DGCNN, B=0.4) & 76.58\% & 79.44\% & 51.21\% & 63.74\% \\
    Ours(DGCNN, B=0.5) & 76.77\% & 80.75\% & 49.59\% & 65.14\% \\
    Ours(DGCNN, B=0.6) & \textbf{76.84\%} & \textbf{81.52\%} & 49.42\% & \textbf{67.24\%} \\
    \bottomrule
    \end{tabular}%
}
%   \vspace{-0.1in}
  \label{tab:ablation_normal}%
\end{table}%
%%%%%%%% END Ablation

\noindent\textbf{Results \& Analysis.}  
As we can see from Table \ref{tab:ablation_normal}, the choice of LOP's structure has limited influence on the performance of PointRCNN under the normal circumstance. The differences between their precision are at most $1.49\%$, $0.72\%$ and $3.37\%$ on cars, pedestrians and cyclists, and the differences between their AP are at most $0.48\%$ on cars.
In contrast, the value of $B$ greatly affects the performance of PointRCNN.
Normally, the higher value of $B$ is realted with better performance of PointRCNN with the LOP: the precision of PointRCNN on cars and on cyclists increase with a larger $B$, while the change of AP on cars is always less than $2\%$. 
% However, the LOP with a lower $B$ may bring PointRCNN better performance on detecting pedestrians, which is mainly because the larger precision of PointRCNN on pedestrians. As we can see from Fig.\ref{fig:ablation_atk}, although the LOP with the structure of DGCNN performs slightly better than the LOP with the structure of PointNet, the differences of the precision are always less than $3.48\%$, and the differences of the ASR are less than $4.00\%$ in most cases, a relatively small gap. Similarly, a higher $B$ always brings the better performance and the better defense effectiveness. For example, when $B=0.6$, the precision of PointRCNN with the LOP on cars increase \lyf{$10.79\%$ on} average, and the ASR of these appearing attacks decreases by $26.56\%$ on average compared with the results of $B=0.2$.

In fact, the point-wise PC model also performs well in other downstream tasks such as classification and semantic segmentation, which means the key features extracted by them is general enough to handle different CV tasks \cite{charles2017PN,charles2017PN++,yue2019DGCNN}. Thus, the LOP with different structures can both perform well in recognizing the components of real objects. However, in the pipeline of our proposed defense the value of $B$ directly determines whether a predicted object is preserved or eliminated. Therefore, the value of $B$ affects the performance of 3D object detectors equipped with the LOP.

% Besides, as a trade-off, the increase in $B$ always causes the degradation in recall in the normal circumstances and under attacks. However, the degradation is limited by $10\%$ on average. Based on our discussion in Section \ref{sec:Limitation}, we consider it as a reasonable trade-off because the slight decrease in recall has limited influence on the normal function of ADS in the real word. In summary, the effectiveness of our proposed defense is insensitive to different choices of the model structure of LOP, while the value of $B$ does play an important role on the contrary. Regardless of the performance of the 3D object detectors in the normal circumstances and the acceptable trading of recall, the LOP with a higher $B$ can always bring the 3D object detector larger improvement.

\autoref{tab:app_ablation} summarizes \thename ablation results on the \malefemale
translation of \celeba. To produce this table, we start with the final \thename configuration and make one of following modifications separately:
(a) disable style modulation in the generator;
(b) disable batch head of the discriminator;
(c) revert to \uvcgan training setup: linear scheduler, full GP, and disable generator averaging.

According to \autoref{tab:app_ablation}, the generator modifications (a) account for the majority of the performance improvement. Removing these modifications degrades the FID score of
the \malefemale translation from $4.7$ to $8.1$.
The discriminator modifications (b) provide a significant but relatively smaller improvement in the I2I performance. The new training setup (c) produces an improvement somewhere in between the generator and discriminator modifications.


\subsection{Toward Better Faithfulness Measures}
\label{sec:better_faithfulness}


\begin{table}[t]
    \caption{\textbf{Faithfulness scores.}}
    \vskip-3mm
    \label{tab:results_consist}
    \centering
    \resizebox{.90\linewidth}{!}{
    \begin{tabular}{l|rrrr}
        \toprule
        {} & \multicolumn{4}{c}{Male-to-Female (CelebA-HQ)}  \\
        {} & CLIP$\uparrow$ & LPIPS$\downarrow$ & I-$L_2$$\downarrow$ & $L_2$$\downarrow$ \\ 
        \midrule
        \egsde   & $0.637$ & $0.116$ & $13.93$ & $\mathbf{43.41}$ \\
        \egsdeDG & $0.600$ & $0.131$ & $15.22$ & $54.50$ \\
        \uvcgan & $\mathbf{0.734}$ & $\mathbf{0.089}$ & $\mathbf{13.12}$ & $54.08$ \\
        \midrule
        \thename   & \underline{$0.723$} & $0.101$ & \underline{$13.36$} & $64.23$ \\
        \thename-C & $0.720$ & \underline{$0.093$} & $13.47$ & \underline{$47.94$} \\
        \midrule
        \midrule
        {} & \multicolumn{4}{c}{Cat-to-Dog} \\
        {} & CLIP$\uparrow$ & LPIPS$\downarrow$ & I-$L_2$$\downarrow$ & $L_2$$\downarrow$ \\ 
        \midrule
        \egsde   & $0.762$ & $0.165$ & $16.65$ & $\mathbf{49.20}$ \\
        \egsdeDG & $0.749$ & $0.181$ & $16.74$ & $62.44$ \\
        \uvcgan
            & $\mathbf{0.822}$ & $\mathbf{0.126}$ & $16.44$ & \underline{$54.45$} \\
        \midrule
        \thename   & $0.802$ & $0.150$ & $\mathbf{16.31}$ & $77.94$ \\
        \thename-C
            & \underline{$0.809$} & \underline{$0.140$}
            & \underline{$16.43$} & $56.83$ \\
        \midrule
        \midrule
        {} & \multicolumn{4}{c}{Wild-to-Dog} \\
        {} & CLIP$\uparrow$ & LPIPS$\downarrow$ & I-$L_2$$\downarrow$ & $L_2$$\downarrow$ \\ 
        \midrule
        \egsde   & $0.672$ & $0.176$ & $15.18$ & $\mathbf{56.65}$ \\
        \egsdeDG & $0.649$ & $0.192$ & $15.41$ & $68.46$ \\
        \uvcgan  & $\mathbf{0.765}$ & $\mathbf{0.124}$ & $15.14$ & $61.05$ \\
        \midrule
        \thename   & $0.720$ & $0.150$ & $\mathbf{14.75}$ & $81.75$ \\
        \thename-C & \underline{$0.740$} & \underline{$0.136$} & \underline{$14.81$} & \underline{$58.77$} \\
        \midrule
        \bottomrule
    \end{tabular}
    }
    \vskip-3mm
\end{table}


Pixel-wise image similarity measures (such as $L_2$, PSNR, and SSIM) have been shown~\cite{zhang2018unreasonable} to be weakly correlated with the human perception of similarity. However, they currently are being used~\cite{zhao2022egsde} as a faithfulness metric in the area of unpaired I2I translation.
In this section, we explore using CLIP~\cite{radford2021learning}
and LPIPS~\cite{zhang2018unreasonable} scores as similarity measures because they have been shown to match human perception.
In addition, it may be more natural to build a similarity measure reusing Inception-v3 features that are already employed in the FID calculation.
This will avoid introducing new dependencies, parameter choices, and other sources of inconsistency.
Thus, as another faithfulness metric, we propose an \textit{Inception-v3 $L_2$ distance} (I-$L_2$) formed between the Inception-v3 features.

\autoref{tab:results_consist} compares the faithfulness scores according to CLIP, LPIPS, I-$L_2$, and pixel-wise $L_2$.
It indicates that \thename variants outperform \egsde variants on the perceptual metrics.
On the other hand, \egsde shows superiority in the pixel-wise scores.
At the same time, \uvcgan demonstrates better perceptual faithfulness compared to \thename.
Given that \thename has better realism than \uvcgan,
this likely demonstrates an example of a realism-faithfulness trade-off.
The trends of the proposed I-$L_2$ roughly match those of the more complex CLIP and LPIPS measures, illustrating the feasibility of an I-$L_2$ as a simpler similarity measure.



Fundamentally, one still may wonder if any of these metrics are ``proper'' measurements of faithfulness?
When measuring faithfulness, one should differentiate between the \textit{domain-contrastive}
features (changing between domains) and \textit{domain-consistent} features (expected to be preserved).
For example, in the \celebahq dataset, males tend to have shorter hair compared to females.
Thus, we expect a Male-to-Female I2I algorithm will increase hair length on average
(domain-contrastive feature).
On the other hand, image background, skin color, facial expression, and facial orientation
are approximately the same in both domains.
Therefore, an I2I algorithm is expected to preserve these features (domain-consistent features).
We argue a proper faithfulness metric should pay the most attention
to domain-consistent features
and be indifferent to the domain-contrastive features.


\begin{figure}[ht]
    \centering
    \tikzsetnextfilename{face_landmark}
    \resizebox{\linewidth}{!}{
    \begin{tikzpicture}
        \draw[step=1cm, grid-grey,very thick](1,1)grid(8,6); 

\foreach \x in {1,...,8}
	\foreach \y in {1,...,6}
		\draw[grid-grey, fill=white, very thick](\x,\y) circle(0.07cm);
\draw[-stealth, line width=0.85mm] (1.5, 5.9) -- (1.5, 5.1);
        \def\folder{fig/face_landmarks}
        \def\index{1}
        
        \def\width{.12\textwidth}
        \def\expand{1.02}

        \node[inner sep=0, fill=white] at (0, 0) {
            \begin{tikzpicture}
                \foreach \name/\label [count=\col] in {input/input, 
                                                       egsde1/\egsde, 
                                                       egsde2/\egsdeDG,
                                                       uvcgan1/\uvcgan,
                                                       uvcgan/\thename} {
                    % image
                    \def\fname{\folder/\name_\index.jpg}
                    \coordinate (C) at (\width * \expand * \col, 0);
        		  \node[inner sep=0] (image_\col) at (C) {\includegraphics[width=\width]{\fname}};
                    % label
                    \tikzmath{coordinate \C;\C = (image_\col.north east) - (image_\col.south west);}
                    \node[header, minimum width=\Cx, minimum height=.2 * \width, anchor=south west] at (image_\col.north west) {\label};
                }
                \foreach \name/\label/\ll [count=\col] in {input/input/0, 
                                                           egsde1/\egsde/0.0150, 
                                                           egsde2/\egsdeDG/0.0250, 
                                                           uvcgan1/\uvcgan/0.0063,
                                                           uvcgan/\thename/0.0055} {
                    % image
                    \def\fname{\folder/\name_\index_lm.jpg}
                    \coordinate (C) at (\width * \expand * \col, -\width * \expand);
        		  \node[inner sep=0] (image_\col) at (C) {\includegraphics[width=\width]{\fname}};
                    \ifthenelse{\col > 1}{
                        % \pgfmathtruncatemacro\exampleNo{\col + #1}
                        % \tikzmath{coordinate \C;\C = (input_\col_lm.north east) - (input_\col.south west);}
                        \node[fill=white, fill opacity=.7, text opacity=1, inner sep=2pt, anchor=north east] at (image_\col.north east) {\scriptsize Lm-$L_2 =\ll$};
                    }{}
                }
            \end{tikzpicture}
        };
    \end{tikzpicture}
    }
    \caption{Examples of face landmark distance (Lm-$L_2$) between an input and a translated image. 
             % On average: EGSDE~(0.0188), EGSDE$^\dag$~(0.0227), and UVCGAN~(0.0152). 
             (\autoref{sec:Faithfulness})}
    \label{fig:face_landmark}
\end{figure}


As an initial attempt, we propose
an approximation to such a metric: similarity of face landmarks~\cite{facelm_creusot20103d,facelm_Burgos-Artizzu_2013_ICCV,facelm_mediapipe}.
Face landmarks are key points on a human face, capturing information about its
position, orientation, structure, expression, etc. That is, capturing those features expected to be largely preserved during the Male-to-Female translation.
Therefore, one may consider forming an Euclidean distance (Lm-$L_2$) between the landmark
locations of the source and translated images (\autoref{fig:face_landmark}). %illustrates how Lm-$L_2$ corresponds to a facial expression change.
Better faithfulness metrics also will consider other domain-consistent features. This development is left for future work.





    
        










