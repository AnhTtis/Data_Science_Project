\begin{abstract}
An unpaired image-to-image (I2I) translation technique seeks to find a mapping between
two domains of data in a fully unsupervised manner.
While the initial solutions to the I2I problem were provided by the generative adversarial neural networks (GANs), currently, diffusion models (DM) hold the state-of-the-art status on the I2I translation benchmarks in terms of FID.
Yet, they suffer from some limitations, such as not using data from the source domain during the training, or maintaining consistency of the source and translated images
only via simple pixel-wise errors.
This work revisits the classic CycleGAN model and equips it with recent advancements in model architectures and model training procedures.
The revised model is shown to significantly outperform other advanced GAN- and DM-based competitors on a variety of benchmarks.
In the case of \malefemale translation of \celeba, the model achieves over $40\%$ improvement in FID score compared to the \sota results.
This work also demonstrates the ineffectiveness of the pixel-wise I2I translation faithfulness metrics and suggests their revision.
The code and trained models are available at \url{https://github.com/LS4GAN/uvcgan2}.

\end{abstract}