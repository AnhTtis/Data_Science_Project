
\section{Additional Quantitative Results (\vocs and \cocos)}
\label{supp:sec:quanti}


In this section, we provide additional quantitative results from our experiments on the \vocs and \cocos datasets. Specifically, in \cref{supp:sec:quantitative:classvsloc}, we show additional results comparing classification and localization performance. In \cref{supp:sec:quantitative:gradcam} we present results for guiding models via \gradcam attributions. In \cref{supp:sec:quantitative:intermediate}, we show that training at intermediate layers can be a cost-effective way approach to performing model guidance.
% , leading to gains in \epg even at the input layer (\cref{supp:sec:quantitative:tdes}).
In \cref{supp:sec:quantitative:segmentepg}, we evaluate how well the attributions localize to on-object features (as opposed to background features) within the bounding boxes, and find that the \energyloss outperforms other localization losses in this regard. In \cref{supp:sec:quantitative:limited}, we provide additional analyses regarding training with a limited number of annotated images. Finally, in \cref{supp:sec:quantitative:dilation}, we provide additional analyses regarding the usage of coarse, dilated bounding boxes during training.

\subsection{Comparing Classification and Localization Performance}
\label{supp:sec:quantitative:classvsloc}
In this section, we discuss additional quantitative findings with respect to localization and classification performance metrics (\iou, \map) for a selected subset of the experiments; for a full comparison of all layers and metrics, please see \cref{fig:supp:voc:f1_results,,fig:supp:coco:f1_results,,fig:supp:voc:map_results:2,,fig:supp:coco:map_results}. 

\myparagraph{Additional \iou results.} In \cref{fig:sub:iou:voc,fig:sub:iou:coco}, we show the remaining results comparing \iou vs.~F1 scores that were not shown in the main paper for \vocs and \cocos respectively.
Similar to the results in the main paper for the \epg metric (\cref{fig:epg_results}), we find that the results between datasets are highly consistent for the \iou metric.

In particular, as discussed in \cref{sec:results:epg+iou}, we find that the \lone loss yields the largest improvements in \iou when optimized at the final layer, see bottom rows of \cref{fig:sub:iou:voc,,fig:sub:iou:coco}. At the input layer, we find that \vanilla and \xdnn \resnet models are not improving their \iou scores noticeably, whereas the \bcos models show significant improvements. We attribute this to the noisy patterns in the attribution maps of \vanilla and \xdnn models, which might be difficult to optimize.

\begin{figure}[h]
    \centering
    \textbf{\iou results} on {\vocs}.\vspace{.25cm}\\
    \begin{subfigure}[c]{\textwidth}
    \includegraphics[width=\textwidth]{results/VOC/figures/iou/all_results_f1.pdf}
    \end{subfigure}
    \caption{\textbf{\iou results on \voc.} We show \iou vs.~\fone for all localization loss functions, attribution methods, and layers. In contrast to the consistent improvements observed at the final layer with the \lone loss, the \iou metric only noticeably improves for the \bcos models after model guidance. We attribute this to the high amount of noise present in the attribution maps of \vanilla and \xdnn models, see \eg \cref{fig:supp:quali_voc_1,,fig:supp:quali_coco_1}. For results on the \cocos dataset, please see \cref{fig:sub:iou:coco}.}
    \label{fig:sub:iou:voc}
\end{figure}

\begin{figure}[h]
    \centering
    {\textbf{\iou results} on \cocos}.\vspace{.25cm}\\
    \begin{subfigure}[c]{\textwidth}
    \includegraphics[width=\textwidth]{results/COCO/figures/iou/all_results_f1.pdf}
    \end{subfigure}
    \caption{\textbf{\iou results on \coco.} We show \iou vs.~\fone for all localization loss functions, attribution methods, and layers. In contrast to the consistent improvements observed at the final layer with the \lone loss, the \iou metric only noticeably improves for the \bcos models after model guidance. We attribute this to the high amount of noise present in the attribution maps of \vanilla and \xdnn models, see \eg \cref{fig:supp:quali_voc_1,,fig:supp:quali_coco_1}. For results on the \vocs dataset, please see \cref{fig:sub:iou:voc}.}
    \label{fig:sub:iou:coco}
\end{figure}

\myparagraph{Using \map to evaluate classification performance.} In all results so far, we plotted the localization metrics (\epg, \iou) versus the \fone score as a measure of classification performance. In order to highlight that the observed trends are independent of this particular choice of metric, in \cref{fig:supp:voc:map_results:1}, we show both \epg as well as \iou results plotted against the \map score. 

In general, we find the results obtained for the \map metric to be highly consistent with the previously shown results for the \fone metric. \Eg, across all configurations, we find the \epgloss to yield the highest gains in \epg score, whereas the \lone loss provides the best trade-offs with respect to the \iou metric. In order to easily compare between all results for all datasets and metrics, please see \cref{fig:supp:voc:f1_results,,fig:supp:coco:f1_results,,fig:supp:voc:map_results:2,,fig:supp:coco:map_results}.

\begin{figure}
    \centering
    \vspace{.5cm}
    \textbf{Mean Average Precision (\map) results} on \vocs.\vspace{.25cm}\\
    \begin{subfigure}[c]{.9\textwidth}
    \includegraphics[width=\textwidth]{results/VOC/figures/loc/all_results_map.pdf}
    \caption{\textbf{\epg vs.~\map.}}
    \end{subfigure}
    \begin{subfigure}[c]{.9\textwidth}
    \includegraphics[width=\textwidth]{results/VOC/figures/iou/all_results_map.pdf}
    \caption{\textbf{\iou vs.~\map.}}
    \end{subfigure}
    \caption{\textbf{Quantitative comparison of \epg and \iou  vs.~\map scores for \vocs.} To ensure that the trends observed and described in the main paper generalize beyond the \fone metric, in this figure we show the \epg and \iou scores plotted against the \map metric. In general, we find the results obtained for the \map metric to be highly consistent with the previously shown results for the \fone metric, see \eg \cref{fig:epg_results,,fig:iou_results}. \Eg, across all configurations, we find the \epgloss to yield the highest gains in \epg score, whereas the \lone loss provides the best trade-offs with respect to the \iou metric. To compare between all results for all datasets and metrics, please see \cref{fig:supp:voc:f1_results,,fig:supp:coco:f1_results,,fig:supp:voc:map_results:1,,fig:supp:coco:map_results}.}
    \label{fig:supp:voc:map_results:1}
\end{figure}



\subsection{Model Guidance via \gradcam}
\label{supp:sec:quantitative:gradcam}
\begin{figure}[h]
    \centering
    {\textbf{Comparison to \gradcam} on {\vocs}.}\vspace{.25cm}\\
    \begin{subfigure}[c]{\textwidth}
    \includegraphics[width=\textwidth]{results/VOC/figures/loc/gradcam_f1.pdf}
    \end{subfigure}
    \caption{\textbf{Quantitative results using \gradcam.} We show \epg scores vs.~F1 scores for all localization losses and models using \gradcam at the final layer (\textbf{bottom row}) and compare it to the results shown in the main paper (\textbf{top row}). 
    As expected, \gradcam performs very similarly to \ixg (\vanilla) and \intgrad (\xdnn) used at the final layer---in particular, note that for \resnet architectures, \ixg and \intgrad are very similar to \gradcam for \vanilla and \xdnn models respectively (see \cref{supp:sec:quantitative:gradcam}). Similarly, we find \gradcam to also perform comparably to the \bcos explanations when used at the final layer; for \iou results and results on \cocos, see \cref{fig:supp:gradcam:voc,,fig:supp:gradcam:coco}.
    }
    \label{fig:supp:gradcam_epg_voc}
\end{figure}

In \cref{fig:supp:gradcam_epg_voc}, we show the \epg vs.~\fone results of training models with \gradcam applied at the final layer on the \vocs dataset; for \iou results and results on \cocos, please see \cref{fig:supp:gradcam:voc,fig:supp:gradcam:coco}. When comparing between rows (\textbf{top:} main paper results; \textbf{bottom:} \gradcam), it becomes clear that \gradcam performs very similarly to \ixg\ / \intgrad\  / \bcos attributions on \vanilla\ / \xdnn\ / \bcos models. In fact, note that \gradcam is very similar to \ixg and \intgrad (equivalent up to an additional zero-clamping) for the respective models and any differences in the results can be attributed to the non-deterministic training pipeline and the similarity between the results should thus be expected. 


\subsection{Model Guidance at Intermediate Layers}
\label{supp:sec:quantitative:intermediate}


In \cref{sec:results}, we show results for guidance on two `model depths', \ie at the input and the final layer. This corresponds to the two depths at which attributions are typically computed, \eg \ixg and \intgrad are typically computed at the input, while \gradcam is typically computed using final spatial layer activations. Following \citeApp{rao2022towards}, for a fair comparison we optimize using each attribution methods at identical depths. For the final and intermediate layers in the network, this is done by treating the output activations at that layer as effective inputs over which attributions are to be computed. As done with \gradcam \citeApp{selvaraju2017grad}, we then upscale the attribution maps to image dimensions using bilinear interpolation and then use them for model guidance.

In \cref{fig:sub:intermediate:epg}, we show results for performing model guidance at additional intermediate layers: Mid1, Mid2, and Mid3. Specifically, for the \resnet models we use, these layers correspond to the outputs of \verb|conv2_x|, \verb|conv3_x|, and \verb|conv4_x| respectively in the ResNet nomenclature (\citeApp{he2016deep}), while the final layer corresponds to the output of \verb|conv5_x|. We find that the \epg performance at these intermediate layers through the network follows the trends when moving from the input to the final layer. Similar results for \iou can be found in \cref{fig:intermediate:iou}.

\begin{figure}[h]
    \centering    
    {\epg results for \textbf{intermediate layers} on \vocs.}\vspace{.25cm}\\
    \begin{subfigure}[c]{\textwidth}
    \includegraphics[width=\textwidth]{results/VOC/figures/loc/intermediate_layer_results_f1.pdf}
    \end{subfigure}
    \caption{\textbf{Intermediate layer results comparing \epg vs.~F1.} We compare the effectiveness of model guidance at varying network depths (\textbf{rows}) for each attribution method and model (\textbf{columns}) across localization loss functions. For the \bcos model, we find similar trends at all network depths, with the \energyloss localization loss outperforming all other losses. For the \vanilla and \xdnn models, the \energyloss loss similarly performs the best, but we also observe improved performance across losses when optimizing at deeper layers of the network. Full results can be found in \cref{fig:intermediate:epg,fig:intermediate:iou}.}
    \label{fig:sub:intermediate:epg}
\end{figure}

\subsection{Evaluating On-Object Localization}
\label{supp:sec:quantitative:segmentepg}



\begin{figure}
    \centering    
    {Evaluating \textbf{on-object localization} within bounding boxes.}\vspace{.25cm}\\
    \begin{subfigure}[c]{\textwidth}
    \centering
    \includegraphics[width=.55\textwidth]{results/figures/qualitative/segmentation_schema.pdf}
    \caption{\textbf{Evaluating \emph{on-object} localization within the bounding boxes: On-object \epg.} In the standard \epg metric (\textbf{middle} column), we compute the fraction of positive attributions within the bounding boxes. In other words, attributions within the bounding box (\textbf{green} region) positively impact the metric, while attributions outside (\textbf{blue} region) negatively impact it. Since bounding boxes are coarse annotations and often include background regions, the standard \epg does not evaluate how well attributions localize \textit{on-object} features, \eg the person in the figure. To measure this, we evaluate with an additional Segmentation \epg metric (\textbf{right} column), where we compute the fraction of positive attributions in the bounding box that lie within the segmentation mask of the object. Here, attributions within the segmentation mask (\textbf{green} region) positively impact the metric, and attributions outside the segmentation mask and inside the bounding box (\textbf{blue} region) negatively impact it. Note that attributions outside the bounding box have no effect on Segmentation \epg. As an example and to visualize qualitative differences between losses, in the bottom rows (\lone, \epgloss), we show attributions for a \bcos model guided at the input layer. As becomes clear, by employing a uniform prior on attributions within the bounding box, the \lone loss is effectively  optimized to fill the entire bounding box and thus to not only highlight \emph{on-object} features. This can also be observed quantitatively, see \eg \cref{fig:seg_epg:epg_vs_f1}, right column.}
    \label{fig:seg_epg:schema}
    \end{subfigure}
    \begin{subfigure}[c]{\textwidth}
    \includegraphics[width=\textwidth]{results/VOCSegment/figures/loc/all_results_f1.pdf}
    \caption{\textbf{On-object \epg results.} We evaluate across models (\textbf{columns}) and layers (\textbf{rows}) for the \energyloss and \loneloss localization losses. As seen qualitatively (\eg \cref{fig:loss_comp}), we find that the \energyloss loss is more effective than the \loneloss loss in localizing attributions to the object as opposed to background regions within the bounding boxes. This is explained by the fact that the \loneloss loss promotes uniformity in attributions within the bounding box, and treats both on-object and background features inside the box with equal importance, while the \energyloss loss only optimizes for attributions to lie within the bounding box without placing any constraints on where they may lie, leaving the model free to decide which regions within the box are important for its decision.}
    \label{fig:seg_epg:epg_vs_f1}
    \end{subfigure}
    \caption{\textbf{Evaluating \emph{on-object} localization via \epg.} We show \textbf{(a)} the schema for the on-object \epg metric and how it differs the standard bounding box \epg metric, and \textbf{(b)} quantitative results on evaluating with on-object \epg.}
    \label{fig:seg_epg}
\end{figure}


The standard \epg metric (\cref{eq:epg}) evaluates the extent to which attributions localize to the bounding boxes. However, since such boxes often include background regions, the \epg score does not distinguish between attributions that focus on the object and attributions that focus on such background regions within the bounding boxes. 

To additionally evaluate for on-object localization, we use a variant of \epg that we call On-object \epg. In contrast to standard \epg, we compute the fraction of positive attributions in pixels contained within the segmentation mask of the object out of positive attributions within the bounding box. This measures how well attributions \textit{within the bounding boxes} localize to the object, and is not influenced by attributions outside the bounding boxes. A visual comparison of the two metrics is shown in \cref{fig:seg_epg}.

We find that the \energyloss localization loss outperforms the \loneloss localization loss both qualitatively (\cref{fig:seg_epg:schema}) and quantitatively (\cref{fig:seg_epg:epg_vs_f1}) on this metric. This is explained by the fact that the \loneloss promotes uniformity in attributions across the bounding box, giving equal importance to on-object and background features within the box. In contrast, the \energyloss loss only optimizes for attributions to lie within the box, without any constraint on \textit{where} in the box they lie. This also corroborates our previous qualitative observations (\eg \cref{fig:loss_comp}).


\subsection{Model Guidance with Limited Annotations}
\label{supp:sec:quantitative:limited}
\begin{figure}
    \centering
    {Additional results for training with \textbf{limited annotations}}\vspace{.5cm}\\
    \begin{subfigure}[c]{\textwidth}
    \begin{subfigure}[c]{.49\textwidth}
    \centering
    \textbf{\epg score}\\\vspace{.2cm}
    \includegraphics[width=\textwidth]{results/VOC/figures/loc/input/limited_annotation_results_bcos.pdf}
    \end{subfigure}\hfill
    \begin{subfigure}[c]{.49\textwidth}
    \centering
    \textbf{\iou score}\\\vspace{.2cm}
    \includegraphics[width=\textwidth]{results/VOC/figures/iou/input/limited_annotation_results_bcos.pdf}
    \end{subfigure}
    \end{subfigure}
    \caption{\textbf{\epg and \iou scores for limited annotations.} We show \epg vs.~F1 (\textbf{left}) and \iou vs.~F1 (\textbf{right}) for \bcos attributions at the input when optimizing with the \energyloss and \loneloss localization losses, when using $\{1\%,10\%,100\%\}$ training annotations. We find that model guidance is generally effective even when training with annotations for a limited number of images. While the performance slightly worsens when using 1\% annotations, using just 10\% annotated images yields similar gains to using a fully annotated training set. Full results can be found in \cref{fig:supp:limited:full:input,fig:supp:limited:full:final}.}
    \label{fig:supp:limited:sub}
\end{figure}

In \cref{fig:supp:limited:sub}, we show the impact of using limited annotations when training (\cref{sec:results:ablations}) when optimizing with the \energyloss and \loneloss localization losses for \bcos attributions at the input. We find that in addition to \epg, trends in \iou scores also remain consistent even when using bounding boxes for just 1\% of the the training images.

    
\subsection{Model Guidance with Noisy Annotations} 
\label{supp:sec:quantitative:dilation}
\begin{figure}[h]
    \centering
    {Additional results for training with \textbf{coarse bounding boxes}}\vspace{.5cm}\\
    \begin{subfigure}[c]{.495\textwidth}
    \centering
    \includegraphics[width=\textwidth]{results/VOC/figures/final/coarse_bbox_results_normal.pdf}
    \caption{\textbf{\vanilla \resnet  @ Final.}}
    \end{subfigure}\hfill
    \begin{subfigure}[c]{.495\textwidth}
    \centering
    \includegraphics[width=\textwidth]{results/VOC/figures/final/coarse_bbox_results_xdnn.pdf}
    \caption{\textbf{\xdnn \resnet  @ Final.}}
    \end{subfigure}
    \caption{\textbf{Coarse bounding box results.} We show the impact of dilating bounding boxes during training for the \textbf{(a)} \vanilla and \textbf{(b)} \xdnn models. Similar to the results seen with \bcos models (\cref{fig:coarse_annotations}), we find that the \energyloss localization loss is generally robust to coarse annotations, while the effectiveness of guidance with the \loneloss localization loss worsens as the extent of coarseness (dilations) increases. Full results in \cref{fig:supp:dilation_quanti:full}.}
    \label{fig:supp:dilation_quanti}
\end{figure}


In \cref{fig:supp:dilation_quanti}, we additionally show the impact of training with coarse, dilated bounding boxes for \ixg attributions on the \vanilla model, and \intgrad attributions on the \xdnn model. Similar to the results seen with \bcos attributions (\cref{fig:coarse_annotations}), we find that the \energyloss localization loss is robust to coarse annotations, while the performance with \loneloss localization loss worsens as the dilations increase.







