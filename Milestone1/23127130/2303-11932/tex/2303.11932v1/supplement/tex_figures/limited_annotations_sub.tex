\begin{figure}
    \centering
    {Additional results for training with \textbf{limited annotations}}\vspace{.5cm}\\
    \begin{subfigure}[c]{\textwidth}
    \begin{subfigure}[c]{.49\textwidth}
    \centering
    \textbf{\epg score}\\\vspace{.2cm}
    \includegraphics[width=\textwidth]{results/VOC/figures/loc/input/limited_annotation_results_bcos.pdf}
    \end{subfigure}\hfill
    \begin{subfigure}[c]{.49\textwidth}
    \centering
    \textbf{\iou score}\\\vspace{.2cm}
    \includegraphics[width=\textwidth]{results/VOC/figures/iou/input/limited_annotation_results_bcos.pdf}
    \end{subfigure}
    \end{subfigure}
    \caption{\textbf{\epg and \iou scores for limited annotations.} We show \epg vs.~F1 (\textbf{left}) and \iou vs.~F1 (\textbf{right}) for \bcos attributions at the input when optimizing with the \energyloss and \loneloss localization losses, when using $\{1\%,10\%,100\%\}$ training annotations. We find that model guidance is generally effective even when training with annotations for a limited number of images. While the performance slightly worsens when using 1\% annotations, using just 10\% annotated images yields similar gains to using a fully annotated training set. Full results can be found in \cref{fig:supp:limited:full:input,fig:supp:limited:full:final}.}
    \label{fig:supp:limited:sub}
\end{figure}