\begin{figure}
    \centering
    {Additional results for training with \textbf{coarse bounding boxes}}\vspace{.5cm}\\
    \begin{subfigure}[c]{.495\textwidth}
    \centering
    \includegraphics[width=\textwidth]{results/VOC/figures/final/coarse_bbox_results_normal.pdf}
    \caption{\textbf{\vanilla \resnet  @ Final.}}
    \end{subfigure}\hfill
    \begin{subfigure}[c]{.495\textwidth}
    \centering
    \includegraphics[width=\textwidth]{results/VOC/figures/final/coarse_bbox_results_xdnn.pdf}
    \caption{\textbf{\xdnn \resnet  @ Final.}}
    \end{subfigure}
    \begin{subfigure}[c]{.495\textwidth}
    \centering
    \includegraphics[width=\textwidth]{results/VOC/figures/input/coarse_bbox_results_bcos.pdf}
    \caption{\textbf{\bcos \resnet @ Input.}}
    \end{subfigure}
    \caption{\textbf{Coarse bounding box results}. We show the impact of dilating bounding boxes during training for the \textbf{(a)} \vanilla and \textbf{(b)} \xdnn, and \textbf{(c)} \bcos models. Similar to the results seen with \bcos models (c), we find that the \energyloss localization loss is generally robust to coarse annotations, while the effectiveness of guidance with the \loneloss localization loss worsens as the extent of coarseness increases.   
    }
    \label{fig:supp:dilation_quanti:full}
\end{figure}