\begin{figure}[h]
    \centering
    {\textbf{Additional qualitative results} on the \waterbirds dataset.}\vspace{.25cm}\\
    \begin{subfigure}[c]{.46\textwidth}
    \includegraphics[width=\textwidth]{results/figures/WB/bcos_Input_1.pdf}
    \caption{\textbf{Landbirds on Water.}}
    \label{fig:supp:waterbirds_quali:a}
    \end{subfigure}\hfill
    \begin{subfigure}[c]{.46\textwidth}
    \includegraphics[width=\textwidth]{results/figures/WB/bcos_Input_2.pdf}
    \caption{\textbf{Waterbirds on Land.}}
    \label{fig:supp:waterbirds_quali:b}
    \end{subfigure}
    \caption{\textbf{Qualitative results for the Waterbirds dataset.} Specifically, we show input layer attributions for \bcos models trained without guidance (`Baseline') as well as guided via the \epgloss or \lone loss. We find that model guidance can be effective both for focusing on the bird and the background. For example, in the top row of (a), the model originally focuses on the background (col.~2) and classifies the image (col.~1) as Water/Waterbird. In the conventional setting, both the \energyloss and \loneloss localization losses are effective in redirecting the focus to the bird (cols.~3-4), changing the model's prediction to Landbird with high confidence. Similarly, in the reversed setting, both localization losses direct the focus to the background (cols.~5-6), which increases the model's confidence in classifying the image as Water.}
    \label{fig:supp:waterbirds_quali}
\end{figure}
