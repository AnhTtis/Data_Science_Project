\documentclass[10pt,twocolumn,letterpaper]{article}

\usepackage{iccv}
\usepackage{times}
\usepackage{epsfig}
\usepackage{graphicx}
\usepackage{amsmath}
\usepackage{amssymb}



% MULTIBIB ADJUSTMENTS
\usepackage{xparse}
\usepackage[numbers]{natbib}
\usepackage[resetlabels,labeled]{multibib}
\newcites{S}{References for Supplement}
\newcites{M}{References}
\newcommand{\citeApp}[1]{\citeS{#1_2}}
\newcommand{\citeMain}[1]{\cite{#1}}

% REDEFINE TO ADD LETTER "A"
\let\oriCiteS\citeS
\RenewDocumentCommand{\citeS}{O{} O{} m}{%
  \renewcommand{\citenumfont}[1]{S##1}%
  \oriCiteS[#1][#2]{#3}%
  \renewcommand{\citenumfont}[1]{##1}%
}


\usepackage{subcaption}
% \usepackage{subfig}
% \usepackage{auto-pst-pdf}

% Include other packages here, before hyperref.

% If you comment hyperref and then uncomment it, you should delete
% egpaper.aux before re-running latex.  (Or just hit 'q' on the first latex
% run, let it finish, and you should be clear).
\usepackage[pagebackref=true,breaklinks=true,letterpaper=true,colorlinks,bookmarks=false,pdftex]{hyperref}
\usepackage{cleveref}
\crefname{section}{Sec.}{Secs.}
\crefname{figure}{Fig.}{Figs.}
\crefname{table}{Tab.}{Tabs.}
\crefname{equation}{Eq.}{Eqs.}
\Crefname{section}{Section}{Sections}

\iccvfinalcopy % *** Uncomment this line for the final submission

\def\iccvPaperID{10062} % *** Enter the ICCV Paper ID here
\def\httilde{\mbox{\tt\raisebox{-.5ex}{\symbol{126}}}}

% Pages are numbered in submission mode, and unnumbered in camera-ready
\ificcvfinal\pagestyle{empty}\fi

\DeclareMathOperator*{\veccat}{%
    \mathchoice%
        {\Bigg\Vert}%
        {\Big\Vert}%
        {\Vert}%
        {\Vert}%
}%



\begin{document}

%%%%%%%%% TITLE
\title{Using Explanations to Guide Models}

% \renewcommand*{\thefootnote}{\fnsymbol{footnote}}
\author{Sukrut Rao$^*$, Moritz Böhle$^*$, Amin Parchami-Araghi, Bernt Schiele\\
Max Planck Institute for Informatics, Saarland Informatics Campus, Saarbrücken, Germany\\
{\tt\small \{sukrut.rao,mboehle,mparcham,schiele\}@mpi-inf.mpg.de}
% For a paper whose authors are all at the same institution,
% omit the following lines up until the closing ``}''.
% Additional authors and addresses can be added with ``\and'',
% just like the second author.
% To save space, use either the email address or home page, not both
% \and
% Second Author\\
% Institution2\\
% First line of institution2 address\\
% {\tt\small secondauthor@i2.org}
}
% \renewcommand*{\thefootnote}{\arabic{footnote}}

\maketitle
\def\thefootnote{*}\footnotetext{Equal contribution.}
\def\thefootnote{\arabic{footnote}}
% Remove page # from the first page of camera-ready.
\ificcvfinal\thispagestyle{empty}\fi


%%%%%%%%% ABSTRACT
\begin{abstract}
Deep neural networks are highly performant, 
but might base their decision on spurious or background features that co-occur with certain classes, which can hurt generalization.
To mitigate this issue, the usage of `model guidance' has gained popularity recently: for this, models are guided to be ``right for the right reasons''
\citeMain{ross2017right}
by regularizing the models' explanations to highlight the right features.
Experimental validation of these approaches has thus far however been limited to relatively simple and / or synthetic datasets.
To gain a better understanding of which model-guiding approaches actually transfer to more challenging real-world datasets, in this work we conduct an in-depth evaluation across various loss functions, attribution methods, models, and `guidance depths' on the \voc and \coco datasets, and show that model guidance can sometimes even improve model performance. In this context, we further propose a novel energy loss, show its effectiveness in directing the model to focus on object features.
We also show that these gains
can be achieved even with a small fraction (\eg $1\%$)
of bounding box annotations, highlighting the cost effectiveness of this approach. Lastly, we show that this approach can also improve generalization under distribution shifts. Code will be made available.
\end{abstract}

%%%%%%%%% BODY TEXT
% \begin{figure}[t]
%     % \begin{subfigure}{1\linewidth}
%     %   \centering
%     % %   \includegraphics[width=1\linewidth]{figs/fig_1_moti_textattn.pdf}  
%     % %   \includegraphics[width=1\linewidth]{figs/fig_1_moti_textattn_v2.pdf}  
%     %   \includegraphics[width=1\linewidth]{figs/fig_1_moti_textattn_v5.pdf}  
%     %   \vspace{-0.5cm}
%     %     \caption{Amount of attention added to each video clip from the source video and query text in the self-attention layers of Moment-DETR encoder.}
%     %     % \caption{Distribution of attention for source and query in Moment-DETR encoder}
%     %     % Visualization of video clip's self-attention score in Moment-DETR encoder.
%     %   \label{fig:fig1_text_attn_ex}
%     % \end{subfigure}%\hfill% or  or \hspace{0.3\textwidth}
%     \vspace{0.2cm}
%     % \begin{subfigure}{1\linewidth}
%       \centering
%     %   \includegraphics[width=1\linewidth]{figs/fig1_moti_negattn.pdf}  
%       \includegraphics[width=1\linewidth]{figs/fig1_moti_negattn_v3.pdf}  
%       \vspace{-0.4cm}
%     %   \caption{Correspondence of saliency scores on the relevance between video clips and the text query.}
%     % \caption{Predicted saliency scores against the video relevant positive query and video irrelevant negative query}
%       \label{fig:fig1_neg_attn_ex}
%     % \end{subfigure}%\hfill% or  or \hspace{0.3\textwidth}
%     \caption{
%     % 원준 원본
%     % (a) Comparison between attention scores of source and query for each video clip~(We sum the attention scores from video and text). 
%     % We observe that the attention scores are dominated by other clips in the source video. 
%     % Text queries do not account for much attention regardless of the relevance to the video clips.
%     % \textbf{(a)} Inspection of the query dependency in Moment-DETR encoder.
%     % % We visualize the attention score of video tokens in the transformer encoder and observe that text query accounts for only a low portion of attention.
%     % % This tendency occurs regardless of the relevance between the text query and video clips. 
%     % We visualize the attention score of video tokens in the transformer encoder and observe 1) text query only accounts for a low portion of attention, and 2) relevance between video-query pair does not affect the attention scores ratio of text.
%     \textbf{(b)} Comparison of highlight-ness when relevant and non-relevant queries are input.
%     As observed in , existing work only uses queries to play an insignificant role, thereby may not be capable of detecting false queries and considering the video-query relevance even when the problem in (a) is resolved. 
%     % \SE{} % 이 부분이 "not capable of" 란 용어가 세다는 피드백이 있는 듯 합니다. 이러한 능력이 없다는 것은 굉장히 강한 어조인거 같기는 하고, 이러한 경우들이 종종 있다거나 좀 약화시킬 필요가 있어보이긴 하네요.
%     On the other hand, our QD-DETR yields a query-dependent representation that the relevance between the source video and query text is updated in the saliency scores.
%     There is a large gap between positive and negative saliency scores, and scores are consistent since the clips are all highly correlated to others.
%     }
%     \label{fig:motivation_ex}
%     % \captionsetup{belowskip=13pt}
%     % \setlength{\belowcaptionskip}{-10pt}
% \end{figure}
\begin{figure}
    \centering
    \includegraphics[width=1\linewidth]{figs/fig1_moti_negattn_1111.pdf}
    % \includegraphics[width=1\linewidth]{figs/fig1_moti_negattn_1109.pdf}
    % \includegraphics[width=1\linewidth]{figs/fig1_moti_negattn_stat.pdf}
    \vspace{-0.6cm}
    \caption{
        % \SE{} % 수정 필요
        Comparison of highlight-ness~(saliency score) when relevant and non-relevant queries are given.
        We found that the existing work only uses queries to play an insignificant role, thereby may not be capable of detecting negative queries and video-query relevance; saliency scores for clips in ground-truth~(GT) moments are low and equivalent for positive and negative queries.
        % This also results in mispredicted moments when ground-truth~(GT) moment is dominated by clips unrelated to GT since their prediction is highly focused on the video.
        % \SE{} % 여기 한번 더 보면 좋을 듯 합니다. GT moment에 unrelated한 clip이 많으면? label이 틀렷을 경우를 말씀하시는건지?
        % As observed in saliency graph, existing work only uses queries to play an insignificant role, thereby may not be capable of detecting false queries and considering the video-query relevance.
        On the other hand, query-dependent representations of QD-DETR result in corresponding saliency scores to the video-query relevance and precisely localized moments.
        % On the other hand, our QD-DETR yields a query-dependent representation that the
        % saliency scores are in accordance with the relevance between the video and query.
        % text is in accordance with the saliency scores.
        % There is a large gap between positive and negative saliency scores, and scores are consistent since the clips are all highly correlated to others.
}
    \label{fig:motivation_ex}
\end{figure}


\section{Introduction}
% 원준 원본
% Along with the advance of digital devices and platforms, video is now one of the most desired data type for consumers. However, although the large information capacity of videos may be beneficial in many aspects, e.g., informative and entertaining, on the contrary perspective, videos are time-consuming, and hard to search for desirable moments. 
% This has led many creators to use extra manpower to crop and edit the video to generate highlight clips to gain the consumer’s attention.
Along with the advance of digital devices and platforms, video is now one of the most desired data types for consumers~\cite{apostolidis2021video,wu2017deep}.
% SE: Video aware deep learning application & survey papers?
Although the large information capacity of videos might be beneficial in many aspects, e.g., informative and entertaining, inspecting the videos is time-consuming, so that it is hard to capture the desired moments~\cite{anne2017localizing,apostolidis2021video}. 
% This has led many creators to use extra manpower to crop and edit the video to generate highlight clips to gain the consumer’s attention.


% On the other side, 
Indeed, the need to retrieve user-requested or highlight moments within videos is greatly raised.
Numerous research efforts were put into the search for the requested moments in the video~\cite{anne2017localizing, gao2017tall, liu2015multi, escorcia2019temporal} and summarizing the video highlights~\cite{zhang2016video, mahasseni2017unsupervised, badamdorj2022contrastive, wei2022learning}.
% Numerous research efforts were put into the search for the requested moments in the video~\cite{anne2017localizing, gao2017tall, liu2015multi, escorcia2019temporal}, summarizing the video to generate highlights was another popular topic~\cite{zhang2016video, mahasseni2017unsupervised, badamdorj2022contrastive, wei2022learning}.
Recently, Moment-DETR~\cite{momentdetr} further spotlighted the topic by proposing a QVHighlights dataset that enables the model to perform both tasks, retrieving the moments with their highlight-ness, simultaneously.

% 원준 원본
% To detect the desired moments, previous works employed transformer encoder-decoder architectural designs to fuse the text query into the video representations. Moment-DETR~\cite{mDETR} modified detection transformer to process capture the moment as a set, and UMT~\cite{umt} implemented transformer decoder as to output clip-wise saliency. 
% Yet to their outstanding breakthroughs in the literature of moment retrieval with the seminal architectures, their limitation is that the role of the given text query is insignificant in representing the query-conditioned video representation; the attention mechanism of moment DETR is not explicitly conditioned on the text query, and the text query is conditioned on multi-modal clips where the differences between the clips are smoothed after encoding process in UMT.



% \begin{figure}[t]
% \centering
%     \begin{subfigure}[l]{0.37\linewidth}
%       \centering
%       \vspace{0.20cm}
%     %   \includegraphics[width=1\linewidth]{figs/fig_1_moti_textattn.pdf}  
%     %   \includegraphics[width=1\linewidth]{figs/fig_1_moti_textattn_v2.pdf}  
%       \includegraphics[width=1\linewidth]{figs/fig1_moti_violin_a.pdf}  
%       \vspace{-0.60cm}
%     %   \caption{text attention}
%         \caption{Importance of queries in video representation}
%       \label{fig:fig1_text_attn}
%     \end{subfigure}%\hfill% or  or \hspace{0.3\textwidth}
%     \vspace{0.2cm}
%     \begin{subfigure}[r]{0.61\linewidth}
%       \centering
%     %   \includegraphics[width=1\linewidth]{figs/fig1_moti_negattn.pdf}  
%       \includegraphics[width=1\linewidth]{figs/fig1_moti_violin_b.pdf}  
%     %   \caption{neg attention}
%         % \caption{Relation between the highlight-ness and the relevance between videos and query texts.}
%         \caption{Highlight-ness~(saliency) histogram of positive and negative video-query pairs\SE{}}
%       \label{fig:fig1_neg_attn}
%     \end{subfigure}%\hfill% or  or \hspace{0.3\textwidth}
%     % \vspace{-0.2cm}
%     \caption{Overall statistics for attention scores in Fig.~\ref{fig:motivation_ex} in QVHighlights dataset. 
%     (a) For the attention scores that measure how much the text query is generally involved in video representation, we use violin plots to show the probability density. We plot the score for each layer in the encoder.
%     % (b) Using the histogram, we compare how the baseline and QD-DETR yield different salient scores given the positive and negative video-text pairs.
%     (b) Saliency histogram shows the distributional gap between positive and negative video-text query pairs of baseline~(Moment-DETR) and proposed QD-DETR.\SE{}
%     }
%     \label{fig:motivation}
%     % \captionsetup{belowskip=13pt}
%     % \setlength{\belowcaptionskip}{-10pt}
% \end{figure}

% \begin{figure}[t]
% \centering

%     \begin{subfigure}[r]{1\linewidth}
%       \centering
%       \hspace{-0.2cm}
%     %   \includegraphics[width=1\linewidth]{figs/fig1_moti_negattn.pdf}  
%       \includegraphics[width=1.1\linewidth]{figs/fig1_moti_violin_a_v2.pdf}  
%     %   \caption{neg attention}
%         % \caption{Relation between the highlight-ness and the relevance between videos and query texts.}
%         \vspace{-0.5cm}
%         % \caption{Saliency histogram of positive and negative video-query pairs}
%         \caption{We plot the histograms and its average value~(dotted line) to compare saliency scores when true and false text queries are given for each method. (left) Since the video representations do not include much textual information, both the true and false queries yield similar saliency scores. (Middle) Even when the video representation is enforced to be updated with the textual information, the issue is not much resolved. (Right) By extracting discriminative features in the text query, distributions are differentiated.
%         % \SE{} % R1@0.5 설명
%         Also, R1@0.5 indicates evaluation metric, Recall at 1 with IoU 0.5 threshold on QVhighlight \textit{val} set.
%         }
%       \label{fig:fig1_neg_attn}
%     \end{subfigure}%\hfill% or  or \hspace{0.3\textwidth}
%     \\
%     \begin{tabular}{cc}
%     \hspace{-0.2cm}
%         \begin{minipage}{.4\linewidth}
%             \begin{subfigure}[l]{1\linewidth}
%               \centering
%             %   \vspace{0.20cm}
%             %   \includegraphics[width=1\linewidth]{figs/fig_1_moti_textattn.pdf}  
%             %   \includegraphics[width=1\linewidth]{figs/fig_1_moti_textattn_v2.pdf}  
%               \includegraphics[width=1\linewidth]{figs/fig1_moti_violin_a.pdf}  
%               \vspace{-0.60cm}
%             %   \caption{text attention}
%                 \caption{Importance of queries in video representation}
%               \label{fig:fig1_text_attn}
%             \end{subfigure}%\hfill% or  or \hspace{0.3\textwidth}
%         \end{minipage}
        
%         \begin{minipage}{.6\linewidth}
%             \vspace{-0.2cm}
%             \caption{Overall statistics of Fig.~\ref{fig:motivation_ex} in QVHighlights dataset. 
%             (a) Saliency histogram shows the distributional gap between positive and negative video-text query pairs.
%             % (a) For the attention scores that measure how much the text query is generally involved in video representation, we use violin plots to show the probability density. We plot the score for each layer in the encoder.
%             % (b) Using the histogram, we compare how the baseline and QD-DETR yield different salient scores given the positive and negative video-text pairs.
%             % (b) Text ratio in self-attention layer to  of Moment-DETR
%             % (b) Ratio of text when representing video tokens in self-attention of Moment-DETR.
%             % (b) Magnitude of attention text query involved.
%             % (b) Attention score of video tokens
%             % (b) Magnitude of text query to refine the video tokens in self-attention layer of Moment-DETR.
%             (b) Probability density depicting the weight of the text query in attention score for video clips. Scores are from the self-attention layers in Moment-DETR encoder.
%             % (b) The text query ratio in attention score of video clips (Self-attention layer in Moment-DETR encoder). We use violin plots to show probability density.
%             % 텍스트 쿼리가, 비디오 피쳐에 얼만큼 attend 하는지
%             }
%         \end{minipage}
    
%     \end{tabular}
%     \vspace{-0.5cm}
%     \label{fig:moti}
%     % \captionsetup{belowskip=13pt}
%     % \setlength{\belowcaptionskip}{-10pt}
% \end{figure}


% \begin{figure}
%     \centering
%     % \includegraphics[width=1\linewidth]{figs/fig1_moti_negattn_1109.pdf}
%     \includegraphics[width=1\linewidth]{figs/fig1_moti_negattn_stat_v2.pdf}
%     \vspace{-0.8cm}
%     \caption{
%         Histogram of saliency when the positive and negative queries are given. We plot the histograms and its average value~(dotted line) to compare saliency scores when relevant~(positive) and irrelevant~(negative) text queries are given for each method. (Left) Since the video representations do not properly reflect textual information, both the positive and negative queries yield similar saliency scores. 
%         % (Middle) Even when the video representation is enforced to be updated with the textual information, the issue is not much resolved. 
%         (Right) By representing video clips in query-dependent manner, distributions are differentiated.
%     }
%     \vspace{-0.6cm}
%     \label{fig:motivation}
% \end{figure}


% One of the demanding task is moment retrieval task, which is detecting the desired moments from the given query, typically the text query.
When describing the moment, one of the most favored types of query is the natural language sentence~(text)\cite{anne2017localizing}. 
While early methods utilized convolution networks~\cite{zhang2020learning, gao2021fast, wang2020temporally}, recent approaches have shown that deploying the attention mechanism of transformer architecture is more effective to fuse the text query into the video representation.
% To handle these modalities, previous works simply employed the attention mechanism of transformer architecture to fuse the text query into the video representation.
For example, Moment-DETR~\cite{momentdetr} introduced the transformer architecture which processes both text and video tokens as input by modifying the detection transformer~(DETR), and UMT~\cite{umt} proposed transformer architectures to take multi-modal sources, e.g., video and audio. 
Also, they utilized the text queries in the transformer decoder.
Although they brought breakthroughs in the field of MR/HD with seminal architectures, they overlooked the role of the text query.
To validate our claim, we investigate the Moment-DETR~\cite{momentdetr} in terms of the impact of text query in MR/HD~(Fig.\ref{fig:motivation_ex}).
Given the video clips with a relevant positive query and an irrelevant negative query, we observe that the baseline often neglects the given text query when estimating the query-relevance scores, i.e., saliency scores, for each video clip.
% the output saliency score, i.e. query-relevance scores.
% Based on the observation, we traced the actual saliency prediction of the model against both the video-relevant query and the irrelevant dummy one where we find that the baseline often neglects the given text query when estimating the query-relevance scores of video clips.
% For example, in Fig.~\ref{fig:motivation_ex}, saliency scores are not affected even when the query is substituted with the dummy.
% % General statistics for Fig.~\ref{fig:motivation_ex} is shown in Fig.~\ref{fig:motivation}. 
% General statistics corresponding to Fig.~\ref{fig:motivation_ex} are also shown in Fig.~\ref{fig:motivation}.



% The limitation of the concrete baseline~\cite{momentdetr} is inspected in two different aspects; 1) Utilization of text-query in the encoding process and 2) the output saliency score, i.e. query-relevance scores.
% Firstly, we visualize the attention score when video clips are given as a query in self-attention. 
% We observe that the text queries have relatively small impacts compared to other video features, as shown in Fig.~\ref{fig:fig1_text_attn_ex}.
% That is, the text does not account for much in representing every video clip, although the goal of MR/HD is to detect query-relevant moments.
% Based on the observation, we traced the actual saliency prediction of the model against both the video-relevant query and the irrelevant dummy one where we find that the baseline often neglects the given text query when estimating the query-relevance scores of video clips.
% For example, in Fig.~\ref{fig:motivation_ex}, saliency scores are not affected even when the query is substituted with the dummy.
% % General statistics for Fig.~\ref{fig:motivation_ex} is shown in Fig.~\ref{fig:motivation}. 
% General statistics are also shown in Fig.~\ref{fig:motivation}.

% Consequently, in Fig.~\ref{fig:fig1_neg_attn_ex}~(b), we found that the baseline often neglects the given text query when estimating the query-relevance scores of video clips; 
% For example, 


% We validate the previous work sometimes neglects the given query when estimating the saliency of video clips.
% For example, there is an example that the saliency scores from positive and negative queries cannot be distinguishable, as shown in Fig.~\ref{fig:fig1_neg_attn_ex}.
% % 우리는 추가로 text attention을 추가도 해봤지만, 효과가 있긴 했으나, still 이슈가 있는 것을 확인하였다?
% % Still, we observe that assuring the high attendance of text queries does not resolve the overlap which motivates us to question the quality of the naive use of task-agnostic text representation~\cite{momentdetr, umt}.
% We found that introducing the text-attention for ensuring the high attendance of text queries relieve the overlap, but there still be a severe overlap.


% To validate their limitations, we inspect the impacts of text queries in the concrete baseline~\cite{momentdetr} with the two different aspects, 1) tendency of attention in self-attention layer and 2) saliency score, i.e. query-relevance scores. \SE{} % attention 이 갑자기 등장하는가?
% Firstly, we visualize the attention score when video clips are given as a query in self-attention. We observe the text queries have relatively low attention scores compared to the video features, as shown in Fig.~\ref{fig:fig1_text_attn_ex}.
% That is, the text does not account for much in representing every video clip, although the goal of MR/HD is to detect query-relevant moments.
% Based on this observation, we trace the actual saliency prediction of the model against both positive and negative text queries.
% We validate the previous work sometimes neglects the given query when estimating the saliency of video clips.
% For example, there is an example that the saliency scores from positive and negative queries cannot be distinguishable, as shown in Fig.~\ref{fig:fig1_neg_attn_ex}.
% % 우리는 추가로 text attention을 추가도 해봤지만, 효과가 있긴 했으나, still 이슈가 있는 것을 확인하였다?
% % Still, we observe that assuring the high attendance of text queries does not resolve the overlap which motivates us to question the quality of the naive use of task-agnostic text representation~\cite{momentdetr, umt}.
% We found that introducing the text-attention for ensuring the high attendance of text queries relieve the overlap, but there still be a severe overlap.



% Thus, we 
% query dependency를 높이기 위해 
% Cross-attention? text-attention? detailed explanation on text-attention should be needed?
% By handling these two issues, we find that more precise retrieval can be achieved.
% 
% 
%
% By projecting video-discriminative text features with high text attendance to source video, we f 
% We also find the need to improve the quality of query features since assuring high text attendance also results in...
% pairs are not finetuned to be discriminative that even the similarity within the pairs does not reflect the relevance between the query and the video clips.
% General statistics for Fig.~\ref{fig:motivation_ex} is shown in Fig.~\ref{fig:motivation}. 
% \SE{} % 이거 ??로 뜨는데, 위처럼 figure 그리면 label이 안되는걸까요
% \SE{}
% 형님 아래 사항 생각 좀 해보는게 좋을 거 같아요.
% fig 1. (a) 그림만 봤을 때 모든 clip에 대해 text attention이 일정이상 존재하긴 하니까, 뭔가 not assured to be conditioned가 와닿지 않는거 같아요.
% + 왜 text가 항상 attend 해야하나?
% not assured to be conditioned --> text shows relatively low affects compared to video 같이 실제 나타난 현상까지 같이 적으면 어떨까 싶어요.
% fig 1. (b) 덜 반영한다?

% \SU{}
% 일단 text가 attend 잘 되어야 한다는 것에 좀 궁금점이 생깁니다. 결국에는 text와 관련있는 frame들을 attend해서 higlight를 찾아야 하는게 아닐까요? 그리고, 현제 저희의 모델 구조상 text query가 Key와 Value로 거의 활용되고 있는데 그렇다면 결국에는 해당 모델은 text에 대한 attention이 전혀 없다고 봐도 무방하지 않을까요? 그런 면에서 text attention을 강조하는게 좀 걸리긴 합니다.

% Specifically, the text query is not assured to be explicitly conditioned on every clip of the video, and as the query texts are evenly treated, discriminative keywords may not be spotlighted.
% attention mechanism of Moment-DETR is not explicitly conditioned on the text query as shown in Fig~\ref{}(d), and in UMT, the text are only used for conditioning the queries while the video representation are refined itself by self-attention.

% \begin{figure}[t]
%     \begin{subfigure}{1\linewidth}
%       \centering
%     %   \includegraphics[width=1\linewidth]{figs/fig_1_moti_textattn.pdf}  
%     %   \includegraphics[width=1\linewidth]{figs/fig_1_moti_textattn_v2.pdf}  
%       \includegraphics[width=1\linewidth]{figs/fig_1_moti_textattn_v4.pdf}  
%       \vspace{-0.5cm}
%     %   \caption{text attention}
%         \caption{Distribution of attention scores in Moment-DETR encoder}
%       \label{fig:fig1_text_attn}
%     \end{subfigure}%\hfill% or  or \hspace{0.3\textwidth}
%     \vspace{0.2cm}
%     \begin{subfigure}{1\linewidth}
%       \centering
%     %   \includegraphics[width=1\linewidth]{figs/fig1_moti_negattn.pdf}  
%       \includegraphics[width=1\linewidth]{figs/fig1_moti_negattn_v2.pdf}  
%       \vspace{-0.5cm}
%     %   \caption{neg attention}
%         \caption{Saliency score against positive and negative text queries}
%       \label{fig:fig1_neg_attn}
%     \end{subfigure}%\hfill% or  or \hspace{0.3\textwidth}
%     \vspace{0.2cm}
%     \begin{subfigure}{1\linewidth}
%       \centering
%     %   \includegraphics[width=1\linewidth]{figs/fig1_moti_violin.pdf}  
%       \includegraphics[width=1\linewidth]{figs/fig1_moti_violin_v2.pdf}  
%       \vspace{-0.5cm}
%       \caption{violin}
%       \label{fig:fig1_violin}
%     \end{subfigure}%\hfill% or  or \hspace{0.3\textwidth}
%     \vspace{-0.2cm}
%     \caption{(a) 1. portion of text attention vs. video attention 2. relation with text query and content (e.g. fg, bg) of clip seems not to affect the attention score
%     (b) 1. high variability even though entire clips are highly correlated with the given text query 2. positive and negative query makes overlaps on saliency score distribution
%     (3) actual distribution on validation dataset.}
%     \label{fig:motivation}
%     % \captionsetup{belowskip=13pt}
%     % \setlength{\belowcaptionskip}{-10pt}
% \end{figure}

To this end, we propose Query-Dependent DETR~(QD-DETR) that produces query-dependent video representation.
% Our key focus is to ensure each clip in predicted moments is explicitly conditioned by the query, particularly on the video-descriptive portion of the text query.
% Our key focus is to ensure that query-relevant clips are predicted by enforcing each clip to be explicitly conditioned by the query.
%Our key focus is to ensure that the model prediction for each clip is highly relevant to the query.
Our key focus is to ensure that the model's prediction for each clip is highly dependent on the query.
% by enforcing each clip to be explicitly conditioned by the query. :)
% hmm...
% \SE {} % "query-relevant clips are predicted" 이 문장이 좀 애매한거 같습니다. relevant 클립을 놓지지 않고 찾는 것을 보장한다? 이런 느낌인지 아니면 높은 saliency 를 주는게 목적이다? model prediction이 query-relevance를 반영하는 것을 보장한다?
% Our key focus is to ensure that the model prediction reflects query-relevance of clips by enforcing each clip to be explicitly conditioned by the query.
First, to fully utilize the contextual information in the query, we revise the transformer encoder to be equipped with cross-attention layers at the very first layers.
% 상익's thought :  single video - query간의 관계만 고려 - 같은 word가 더 많이 쓰이는 것을 보고 
% 교수님's thought : neg pair 를 쓰면 쿼리를 보지 않고서는 video clip간만 고려하는 것이 사라짐. 왜냐면 0으로 내보내야 하기 때문. --> SE: relative difference 만 고려하다가, 
By inserting a video as the query and a text as the key and value of the cross-attention layers, our encoder enforces the engagement of the text query in extracting video representation.
% 원준 교수님 코멘트 반영해서 다시
Then, in order to not only inject a lot of textual information into the video feature but also make it fully exploited, we leverage the negative video-query pairs generated by mixing the original pairs.
Specifically, the model is learned to suppress the saliency scores of such  negative~(irrelevant) pairs.
Our expectation is the increased contribution of the text query in prediction since the videos will be sometimes required to yield high saliency scores and sometimes low ones depending on whether the text query is relevant or not.
% \SE{}
% learns to?
% By suppressing the saliency scores of the irrelevant video-query pairs, the model learns to spotlight only the video-specific discriminative words in the query.
% % \SE{} % ====================== 상익 수정 ========================
% However, this architectural design still lacks the capability of identifying the video-descriptive keywords in the query.
% % However, this architectural design still lacks in identifying proper query relevance.
% This is because the current training scheme only focuses on the interactions of video and clips within a single video while neglecting information shared throughout the entire video.
% % We argue the problem of the current training scheme that only focuses on distinguishing the clips in a single video while neglecting information shared throughout the entire video.
% Therefore, we leverage the negative video-query relationships to enhance the capability of identifying the contextual similarity of query and video clips.
% 
% 원준 원본 
% However, this architectural design heavily relies on the quality of the text query.
% Therefore, we leverage the negative video-query relationships to enable the model to emphasize key corresponding query features.
% By suppressing the saliency scores of the irrelevant video-query pairs, the model learns to spotlight only the video-specific discriminative words in the query.
% =========================================================
Lastly, to apply the dynamic criterion to mark highlights for each instance, we deploy a saliency token to represent the entire video and utilize it as an input-adaptive saliency criterion. 
With all components combined, our QD-DETR produces query-dependent video representation by integrating source and query modalities.
This further allows the use of positional queries~\cite{dabdetr} in the transformer decoder.
% Furthermore, we can exploit the advanced DETR decoder architectures using the positional information, e.g., DAB-DETR, since our encoded tokens consist of identical position representations from a single modality.
% \SE{} % ====================== 상익 수정 ========================
% Furthermore, we can exploit the advanced DETR decoder architectures using the positional information, e.g., DAB-DETR, since our video clip tokens consist of identical position representations from a single modality.
% 원준 원본
% It also enables the use of advanced DETR decoder architectures, e.g., DAB-DETR, for the first time, as these works exploit the position information within a single modality.
% =========================================================
Overall, our superior performances over the existing approaches validate the significance of the role of text query for MR/HD.
% Our extensive experiments on QVHighlights, TVSum, and Charades-STA datasets validate the significance of considering the role and the quality of text query.

% All components combined with dynamic anchor moments for the query of decoder, our FOQUE fosters the query-dependent video representation, thereby making the 
% All components combined, our modified transformer encoding process fosters the query-dependent video representation thereby achieving the state-of-the-art results on various benchmarks of moment-retrieval and highlight detection.
	
% -	Video Platform & Streamer & Consumer의 증가. 
% Video는 다른 데이터 타입보다 정보가 많아 유용하지만, 이는 다른 말로 해석하면 video를 보는 것은 time-consuming 하고, 원하는 것을 찾아보기에는 힘들 수 있음.
% 따라서, 많은 매체에서는 사람들의 더 많은 이목을 끌기 위해 highlight 비디오라는 것을 편집하여 공유도 함.
% 하지만, highlight video를 만들기 위해 사람의 노력이 필요한 현 시점에서, This spotlights the need to retrieve the user-requested / Highlight moments in the video.

% -	이전에도 이러한 문제를 해결하기 위해 (asdfasdf) for moment retrieval, (asdfasdf) for highlight detection 등이 제안 되었지만, 이들은 비디오의 특정 영역을 찾는다는 공통된 목적을 가지고 있으면서도, 데이터 셋의 한계로 인해 따로 연구되었음. 이를 문제 삼으며, 최근에는 두 task를 동시에 학습할 수 있는 dataset이 소개 되었는데, 컴퓨터비전에서 최근 각광을 받고 있는 Transformer 모델 도입과 함께 큰 발전을 거듭하고 있음.

% -	구체적으로, 이 두가지 task를 수행하기 위해서는 transformer를 두가지 방법으로 이용할 수 있는데, moment-DETR 처럼 moment 를 clip의 set 단위로 예측할 수 있고, UMT 처럼 clip-wise prediction을 할 수 있음. 하지만, 이들은 query를 condition이 아닌 video와 동등한 레벨로 취급하거나 [mDETR], 매 클립이 self-attention으로 mixing 된 후에 condition을 걸어주어 clip간의 차이를 확실하지 이용하지 못하였고, 또한, 확실하게 condition으로 주지 못하였고, video와 query 사이의 관계를 한정적으로만 이용하였다.

% -	따라서, we explore three different ways to fully exploit query information. First, we design one-way cross-attention layer to condition every clip with the query features. Then, we utilized the negative video-text pairs to better model the relationships between the video and the text embeddings. Lastly, we define the saliency token to be the video-query dependent saliency estimator.


















% ===================== neg pair 부분 ===========================
% Nevertheless, the current training scheme, only considering the given video-query pair, still disturbs the model from identifying proper query-relevance prediction.
% In detail, the model focus on learning the fine-grained discrepancy between video clips, while neglecting the information they share, which contains significant clues to understand the context of video.
% Therefore, we leverage the negative video-query relationships to enhance the capability of identifying the contextual similarity of query and video clips.
% Therefore, we leverage the negative video-query relationships by suppressing those pairs, so that enhance the capability of identifying the contextual similarity of query and video clips.
% We hypothsize the diversity in query-video pairs are insufficient to learn the general relationship between text query and video.
% Therefore, we leverage the negative video-query relationships by suppressing the saliency scores of the irrelevant video-query pairs.
% However, this architectural design still lacks in identifying proper query relevance.
% We argue that the current training scheme only focuses on learning the fine-grained discrepancy between clips in a single video, while neglecting the information they share, which contains significant clues to understand the context of the video.
% Therefore, we leverage the negative video-query relationships to enhance the capability of identifying the contextual similarity of query and video clips.
% However, this architectural design still lacks in identifying proper query relevance.
% We argue the problem of the current training scheme that only focuses on learning the fine-grained discrepancy between clips in a single video.
% That is, the current design neglects the information shared throughout the video, although it contains significant clues to understand the context of the video.
\section{Related work}

In recent years, large language models have improved significantly in various NLP areas, especially in generative tasks.
A lot of new concepts were introduced, starting from attention mechanism~\cite{bahdanau2014neural}, transformers~\cite{vaswani2017attention} to multitask, learning from instructions~\cite{wang2022super} and human feedback~\cite{wang2021putting}.
The last becomes extremely popular in the generative context including machine translation. 
% new architectures were proposed~\cite{radford2019language,brown2020language}, and, 
Consequently, the usage of machine translation tools has become a necessary compound for understanding a foreign language. 
Unfortunately, like other neural network-based algorithms, these tools are vulnerable to adversarial examples~\cite{DBLP:journals/corr/GoodfellowSS14}. 
Starting from text classification \cite{li-etal-2020-bert-attack,DBLP:conf/acl/EbrahimiRLD18,Li2018TextBuggerGA}, vulnerability and robustness received a lot of attention in the NLP community. 
For MT systems one of the pioneering works was~\cite{ebrahimi2018adversarial}, where authors proposed a character-level approach to generate adversarial examples.
% that neural MT systems are vulnerable to character-level perturbations, where only a few symbols in an input query are subject to change. 
Inheriting HotFlip~\cite{ebrahimi-etal-2018-hotflip} there were considered white-box and black-box settings, where only a few symbols in an input query are subject to change imitating typos.

While white-box optimization may yield stronger adversarial perturbations it implies access to the model's architecture and weights which is impractical in the case of online MT tools. 
In~\cite{wallace} there was considered a white-box universal approach to a targeted attack on conditional text generation. 
The authors modeled perturbation as an insertion of a trigger, a token sequence of small length, that results in a generated sequence similar to the target set of sentences. 
While during experiments certain triggers cause a model to produce sensitive racist output, they are generally meaningless and similarly to character-level attacks are easy to detect. 
Authors of~\cite{guo-etal-2021-gradient,9747475} reported high attack transferability making this approach promising for black-box setup, however,  the research is limited only to the GPT-2 model for generation task. 
The above papers use greedy techniques to walk through the searching space during the optimization, on the other hand, attacks on NLP models could be found via projection onto embeddings~\cite{wallace}, and for MT task this was discovered in~\cite{Seq2Sick,Sadrizadeh2023TargetedAA,sadrizadeh2023transfool}. 
In~\cite{zhang2021crafting}, it was shown that black-box optimization may yield transferable word-level attack that fools online translation tools, for example Baidu and Bing translators. 
This work proposed to use the word saliency as the measure of uncertainty. 
Masking candidates the saliency was estimated via additional BERT model~\cite{devlin2018bert}  which lead to strong readable and imperceptible adversaries, however, neither human evaluation was performed nor quantities results for online tools were given. In~\cite{wan2022paeg}, a gradient-based approach to generate phrase-level adversarial examples for neural MT systems was proposed. Similarly to~\cite{zhang2021crafting}, it is proposed to estimate the vulnerable word positions are estimated in an input phrase with the use of gradient information and replace corresponding words by the candidates computed with an auxiliary model.

% \mynote{actually we may underline that we do not generate adversarial examples per se (we arent aimed at misclassification), but rather generate inputs that are been translated though they should not}

% \mynote{TODO: Maybe add more criticism of zhang2021crafting and point out the differences in our approach.}

% \todopa{}{}{
% https://www.semanticscholar.org/paper/AdvAug\%3A-Robust-Adversarial-Augmentation-for-Neural-Cheng-Jiang/1e7d3a9846da556bc7b84ae1410d257b89448c30
% }

%\todopa{}{}{
%https://www.semanticscholar.org/paper/A-Targeted-Attack-on-Black-Box-Neural-Machine-with-Xu-Wang/2a46eb47e8742be29b16a5b83dc1a38616b24ce6
%}

%\todopa{}{}{https://www.semanticscholar.org/paper/PAEG\%3A-Phrase-level-Adversarial-Example-Generation-Wan-Yang/a6dd2a8debb5d5324c4f2be7fb7bb52ce109cbaf}

% \todopa{}{}{
% https://download.huan-zhang.com/events/srml2022/accepted/bhandari22lost.pdf
% }

%\todopa{}{}{http://fan-yao.com/paper/2021_SEED_nmtstroke.pdf}

% \todopa{}{}{https://arxiv.org/pdf/2303.01068v1.pdf}

%\todopa{kosinski2023theory}
%    {Theory of mind may have spontaneously emerged in large language models}
%    {https://arxiv.org/pdf/2302.02083.pdf}
%    {We can say that large language models are very clever now, etc...}

% \todopa{ebrahimi2018adversarial}
%     {On adversarial examples for character-level neural machine translation}
%     {https://arxiv.org/pdf/1806.09030.pdf}
%     {Very related work (see beamsearch in the text also)...}

% \todopa{zhang2021crafting}
%     {Crafting adversarial examples for neural machine translation}
%     {https://github.com/JHL-HUST/AdvNMT-WSLS}
%     {Very related work. See: ``Besides, WSLS exhibits strong transferability on attacking Baidu and Bing online translators.''}

% \todopa{sadrizadeh2023transfool}
%     {TransFool: An Adversarial Attack against Neural Machine Translation Models}
%     {https://arxiv.org/pdf/2302.00944.pdf}
%     {Very related work!}
\section{Method}
\label{sec: method}
% This section introduces the rendering pipeline of our proposed hierarchical compositional scene. 
% our pipeline consists of three processes, including decomposing the text into editable 3D layout, rendering the compositional views with local (object) NeRFs and global (scene) NeRF and the joint optimization on these hierarchical 3D representations.

% Note that the transformation between the object and the scene frame is defined by ${p}_o$ and ${D}_o$. 
%
% Next, we build a residual connection to add ${\sigma}_o$ and the referenced global color, and the rendering result will be used to calculate the SDS loss based on the global text.  
% Fig.~\ref{fig:framework} illustrates our pipeline, which consists of three main components, including the editable 3D scene layout based on multi-object text (Sec.~\ref{ssec:layout}), the scene rendering pipeline that composites the predictions from all local NeRFs (Sec.~\ref{ssec:render}), and the joint optimization on both local and global representation models (Sec.~\ref{sec:optimization}).
% To elaborate, our editable 3D scene layout represents a global frame of the scene by decomposing it into a set of local frames, where each is parameterized by a local NeRF, a 3D bounding box, and a corresponding local text prompt.
% For instance, the text prompt `A teddy bear and a stuffed monkey sit side by side' is interpreted as a 3D scene layout, as shown in Fig.~\ref{fig:framework}.  
% The whole 3D layout, \ie, scene frame, consists of two 3D bounding boxes, \ie local frames \#1 and \#2, with specific local text prompts, \ie, `a teddy bear' and `a stuffed monkey'. 
% %
% To render the scene view, we first calculate the ray-box intersections between the boxes and rays $({\boldsymbol{r}}_o, \boldsymbol{\phi}_d, {\boldsymbol{\theta}}_d)$, where the ${\boldsymbol{r}}_o$ is the ray origin and the $({\boldsymbol{r}}_o, \boldsymbol{\phi}_d)$ is its direction.
% Then, to infer each object's properties in local NeRFs, we sample the global points $({\boldsymbol{x}}_g, {\boldsymbol{y}}_g, {\boldsymbol{z}}_g)$ in the global frame within the ray-box intersection intervals and project them into the normalized local location $({\boldsymbol{x}}_l, {\boldsymbol{y}}_l, {\boldsymbol{z}}_l)$ in the local frame.
% %
% Given the local sampling points $({\boldsymbol{x}}_l, {\boldsymbol{y}}_l, {\boldsymbol{z}}_l)$, the implicit local NeRF ${\boldsymbol{\theta}}_l$ outputs four pseudo-color channels ${\boldsymbol{C}}_l$ and density $\boldsymbol{\sigma}$, which can be used to render a local view of the local frame to match its local text prompt.
% %
% We further calibrate the predicted pseudo-color $\boldsymbol{C}_l$ from local frames by adding the global embeddings ${\boldsymbol{emb}}_g$ to improve the global view consistency.
% Then, the calibrated predictions after composition are used to reconstruct the scene view by volumetric rendering along the rays.
% %
% Lastly, the rendered views based on local and global frames are guided by score distillation sampling loss $\nabla \mathcal{L}_{\text{SDS}}$~\cite{poole2022dreamfusion} to optimize all the learnable parameters. 
To resolve the issue of guidance collapse, our principal strategy is to \textit{decompose the scene into reusable components and compose/recompose them into a unified and consistent one}.
This enables flexible control over the generated content with direct use of prompts and box layouts, as illustrated in \cref{fig:teaser}.
%
Our proposed CompoNeRF confers several key benefits:
1) \textbf{Semantic Coherence}: It reliably creates 3D objects with detailed textures and global consistency, exemplified by authentic light interactions, such as reflections on the bed surface.
2) \textbf{Modularity and Reusability}: CompoNeRF functions as an ensemble of independently trained NeRF models. These can be efficiently stored and later retrieved from a cached dataset, enabling their reuse in various cases.
3) \textbf{Editability}: Our approach allows for flexible scene modification, such as interchanging the lamp for a vase filled with sunflowers or altering its scale, by simply adjusting the box dimensions for later finetuning. This feature enhances flexibility and creative possibilities. 


% Furthermore, the usage of layout boxes enables more flexible control over the generated content compared with the intricate sketch shape in Latent-NeRF\cite{metzer2022latent}. 
\begin{figure*}[t]
    \centering
    \includegraphics[width=0.9\linewidth]{figures/method.pdf}
    % \vspace{-12pt}
    \caption{\textbf{Framework Overview}.
The CompoNeRF model unfolds in three stages: 1) Editing 3D scene, which initiates the process by structuring the scene with 3D boxes and textual prompts; 2) Scene rendering, which encapsulates the composition/recomposition process, facilitating the transformation of NeRFs to a global frame, ensuring cohesive scene construction. Here, we specify design choices between density-based or color-based(without refining density) composition; 3) Joint Optimization, which leverages textual directives to amplify the rendering quality of both global and local views, while also integrating revised text prompts and NeRFs for refined scene depiction.
  % The model is structured into three components: Composition, Decomposition, and Recomposition. Composition deals with the foundational setup, detailed with choices for density-based and color-based composition. Decomposition utilizes the modularity of the CompoNeRF feature, caching each NeRF module offline for efficient recalibration. Recomposition reuses these cached NeRFs and adjusts the semantic context, providing a revised output with the inclusion of the offline NeRF enhancements.
    % Our model consists of two branches where the upper part is individual NeRFs, and the lower part denotes global calibration with our tailored composition model. The specific designs for density-based and color-based composition modules are highlighted. 
    % CompoNeRF consists of three parts: 1). The editable 3D scene layout configures the scene representations with 3D boxes and text prompts; 2).  The scene rendering includes the global calibration and the compositional process; 3). The joint optimization applies global and local text guidance on global and local render views.
    % The global frame (scene space) contains a set of local frames. Each is  represented by a local NeRF associated with a 3D box and text prompt defined by the editable 3D layout.
    % The scene view is volumetric rendered by sampling the points $({\boldsymbol{x}}_g, \boldsymbol{y}_g, \boldsymbol{z}_g)$ intersected with any local frame along the ray $(\boldsymbol{r}_o, {\boldsymbol{\phi}}_d, \boldsymbol{\theta}_d)$.
    % The sampling points are first inferred through the local NeRF with the local frame locations $({\boldsymbol{x}}_l, \boldsymbol{y}_l, \boldsymbol{z}_l)$ projected from the global location $({\boldsymbol{x}}_g, \boldsymbol{y}_g, \boldsymbol{z}_g)$.
    % And then, all the local predictions are calibrated by a global MLP with conditional input to render the scene view.
    % During the optimization, the text guidance is applied to both local views predicted by local frames only and global views predicted by the composition of all local frame predictions.
    }
    \label{fig:framework}
    % \vspace{-8pt}
\end{figure*}

\subsection{Preliminaries}
Defining individual object bounding boxes as \textit{local frames} and the overall scene coordinate system as the \textit{global frame}, we build the foundation of NeRF and diffusion processes.

\label{sec:background}
\noindent \textbf{3D Representation in Latent Space.}
Our methodology capitalizes on the state-of-the-art text-to-image generative model—Stable Diffusion as described by Rombach et al\cite{rombach2022high}.
We build upon the Latent-NeRF framework~\cite{metzer2022latent}, which computes latent colors for individual objects by considering their sample positions within a localized frame. Specifically, it maps a three-dimensional point in local coordinates \(\boldsymbol{x}_l = (x_l, y_l, z_l)\) to a volumetric density \(\boldsymbol{\sigma}_l\) and an associated color \(\boldsymbol{C}_l\), expressed as \((\boldsymbol{C}_l, \boldsymbol{\sigma}_l) = f_{\boldsymbol{\theta}_l}(x_l, y_l, z_l)\). Here, \(f\) represents a Multi-Layer Perceptron (MLP) characterized by parameters \(\boldsymbol{\theta}_l\).
 This NeRF-generated color is then assessed in the context of the Stable Diffusion model, using text prompts to guide NeRF toward spatially coherent inference with intricate context.
% to infer pseudo-color for each object using local NeRF.
% Specifically, the representation maps a point $\boldsymbol{x}_l = \left({x}_l, {y}_l, {z}_l\right)\in [-1, 1]$ in the local frame to its corresponding volumetric density $\boldsymbol{\sigma}_l$ and emitted color $\boldsymbol{C}_l$, \ie,  $\left(\boldsymbol{C}_l, {\boldsymbol{\sigma}_l}\right)=\boldsymbol{\theta}_{_l}\left({x_l}, {y}_l, {z}_l\right)$.
% The predicted pseudo-color is fed forward into the decoder of the Stable Diffusion model to obtain the final rendering result.

\noindent \textbf{Volume Rendering with Multiple Objects.}
% For each local frame $j$ with NeRF parameterized as $\theta_j$, we follow original NeRF design\cite{nerf} to integrate $(\boldsymbol{C}_l, \boldsymbol{\sigma}_l)$ of   sampled points from any hit ray $r_l=(\boldsymbol{o}_l, \boldsymbol{d}_l)$ by,
% For consistent scene rendering, object transmittance $T_k$ must be recalculated in the global frame based on independent properties inferred from local NeRFs. Hence, we sort predictions according to their distance to $\boldsymbol{o}_g$. 
% Similar to \cref{eq:volrend}, global color $\hat{\boldsymbol{C}}_g$ of ray $\boldsymbol{r}_g=(\boldsymbol{o}_g, \boldsymbol{d}_g)$ is predicted by the volumetric rendering integrating over $m$ objects,
We extend the volume rendering process to accommodate multiple objects by assigning each a local frame, denoted as $j$, with NeRF parameters $\boldsymbol{\theta}_{l, j}$. Drawing from the foundational NeRF approach \cite{nerf}, in each local frame, we integrate the color $\boldsymbol{C}_l$ and density $\boldsymbol{\sigma}_l$ for points $\boldsymbol{x}_l$ sampled along a ray $\boldsymbol{r}_l$, emanates from the camera origin $\boldsymbol{o}_l$ in direction $\boldsymbol{d}_l$. This is formalized in the predicted color integration for $\hat{\boldsymbol{C}}_l$ as:
{\setlength\abovedisplayskip{2pt}
\setlength\belowdisplayskip{2pt}
\begin{equation}
\label{eq:volrend}
{\hat{\boldsymbol{C}}_l}({\boldsymbol{r}_l})=\sum_{k=1}^{N} T_{l, k} \left(1-\exp \left(-\sigma_{l, k} \delta_k\right) \right) {\boldsymbol{C}}_{l,k},
\end{equation}}where $T_{l, k}=\exp \left(-\sum_{j=1}^{k-1} \sigma_{l,j} \delta_j\right)$ represents the transmittance to the $k$-th of total $N$ sample, calculated exponentially over the cumulative density along $\boldsymbol{r}_l$, and $\delta_k$ is the interval between adjacent samples.
%
To synthesize a coherent scene, we transition from processing individual local frames to a collective global frame. Within this global context, we reconcile object attributes inferred from their individual local NeRFs for refined $\boldsymbol{\sigma}_g, \boldsymbol{C}_g$ along with $T_{g, k}$. The samples $\boldsymbol{x}_g$ are ordered based on their spatial distances from the origin $\boldsymbol{o}_g$ following the coordinate transformation. We then express the volumetric rendering of a ray $\boldsymbol{r}_g$ integrating $m$ objects within the global frame as follows:
{
\setlength\abovedisplayskip{2pt}
\setlength\belowdisplayskip{2pt}
\begin{equation}
\label{eq:multi_volrend}
{\hat{\boldsymbol{C}}_g}({\boldsymbol{r}_g})=\sum_{k=1}^{m*N} T_{g, k} \left(1-\exp \left(-\sigma_{g, k} \delta_k\right) \right) {\boldsymbol{C}}_{g,k}. 
\end{equation}}

\noindent \textbf{Score Distillation Sampling.}
% During the SDS process, a noise image $\boldsymbol{X}_t$ is first generated by adding a sampled noise $\epsilon \sim \mathcal{N}(0, I)$ in noise level $t$ into a rendered view $\boldsymbol{X}$ from a NeRF.
To facilitate the conversion from text descriptions to 3D models, DreamFusion~\cite{poole2022dreamfusion} utilizes Score Distillation Sampling (SDS), leveraging the generative capabilities of a diffusion model, denoted as $\phi$, to guide the optimization of NeRF parameters, symbolized as $\boldsymbol{\theta}$.
%
Initially, SDS creates a noisy image $\boldsymbol{X}_t$ by infusing a randomly sampled noise $\epsilon$, which follows a normal distribution $\mathcal{N}(0, I)$, into a NeRF-rendered image $\boldsymbol{X}$ at a given noise level $t$.
The diffusion model $\phi$ then estimates the noise $\epsilon_\phi\left(\boldsymbol{X}_t, t, T\right)$ from this noisy image, conditioned by the noise level $t$ and an optional text prompt $T$. 
The key step in SDS involves calculating the gradient of the loss function, which measures the discrepancy between the estimated noise and the originally added noise:
{\setlength\abovedisplayskip{2pt}
\setlength\belowdisplayskip{2pt}
\begin{equation}
\label{eq:sds_loss}
\nabla_\theta \mathcal{L}_{\text{SDS}}(\boldsymbol{X}_t, T)=  w(t)\left(\epsilon_\phi\left(\boldsymbol{X}_t, t, T\right)-\epsilon\right),
\end{equation}}where $w(t)$ is a weighting function that adjusts the influence of the gradient based on the noise level. 
The gradients across all rendered views direct the update of $\boldsymbol{\theta}$, ensuring that the NeRF-generated images align with the text descriptions. Additionally, we incorporate the 'perturb and average' technique from SJC for more robust $\mathcal{L}_{\text{SDS}}$. For a comprehensive understanding of these methods, the reader is directed to the detailed explanations provided in \cite{poole2022dreamfusion,wang2022score}.

%
%
% \subsection{Editable 3D Scene Layout}
% \label{ssec:layout}
% The 3D scene layout explicitly combines language structures with 3D layouts in an editable way.
% Given the input text prompt $T$, the attribute-object pairs can be easily obtained based on user control.
% Note that the text prompt indicates the multi-object text prompt by default.
% % available for free in many structured representations, such as the constituency tree.
% As shown in Fig.~\ref{fig:framework}, we can extract multiple noun phrases with their binding attributes and map these local text prompts into corresponding regions.
% Specifically, we define the scene structure with $m$ local frames, each employs a local NeRF $\boldsymbol{\theta}_l$ as representation, the local text prompt $T_{l} \subseteq{T}$ and its spatial layout with 3D boxes $\mathbf{b} = \{\mathbf{p}, \mathbf{s}\} \in  \mathbb{R}^6$ of each object entity, where $\mathbf{p}=\{p_x, p_y, p_z\}$ refers to the center point and $\mathbf{s}=\{s_x, s_y, s_z\}$ denotes the box scale. 
% \textit{Our editable 3D layout is easy to be collected and edited with its simplicity, allowing for versatile and interactive user control by modifying the box's or text's properties to define a new scene}.
% Moreover, as depicted in Fig.~\ref{fig:teaser}, each component in a 3D scene layout can be replaced or re-composited with other trained local NeRFs, which is more friendly for flexible user editions compared with using only text prompts.
% We fine-tuned the new layout by global rendering, which enables scalable re-editing.
% Each relationship $r_k \in R$ is a triplet in a <subject-predictive object> format, where a subject node is. After we generate the scene graph from the complex prompts, we can sample the closest relationship with the 2d spatial layout as the initial 3D position. fine-tuned the new layout by global rendering, which enables scalable re-editing
%
% \subsection{Scene Rendering Pipeline}
% \label{ssec:render}
% In CompoNeRF, the scene images are rendered by a ray-casting approach following the design of NeRF.
% % Each ray to be cast is generated based on the camera pose, intrinsic, and transformation.
% The camera is defined by a pinhole camera model, casting a set of rays $(\boldsymbol{r}_o, \boldsymbol{\phi}_d, {\boldsymbol{\theta}}_d)=\boldsymbol{o}+t\boldsymbol{d}$ through each pixel on the frame of size $H \times W$, where the $\boldsymbol{r}_o \in  \mathbb{R}^3$ is the origin and the $(\boldsymbol{\phi}_d, \boldsymbol{\theta}_d)$ is the viewing direction.
% Along this ray, we sample all the points intersected with any layout box of local frames.
% For each hit sampled point, the color and volumetric density are computed through the local NeRF of the hit local frame.
% The ray color perdition is calculated by the differentiable integration applied on all the point-predicted colors and volumetric density along the ray.
%
% \noindent \textbf{Ray-box Intersection with Local Frames.}
% Given a ray $\boldsymbol{r}_i$, each box $\boldsymbol{b}_j$ of the local frame is applied with the AABB ray intersection test algorithm to check the intersections.
% When the ray $r_i$ is hit with a box $\boldsymbol{b}_j$ of the local frame, we use the entrance and exit points as near $\boldsymbol{t}_{in}$ and far $\boldsymbol{t}_{out}$ bounds to sample $N$ equidistant quadrature points, $
% \boldsymbol{t}_{i,j,n}=\frac{n-1}{N-1}\left(\boldsymbol{t}_{out}-\boldsymbol{t}_{in}\right)+\boldsymbol{t}_{in} , n \in \left[1, N\right]$
% % Despite each local frame only having a small number of hit rays compared to the scene, we observe that it is enough to represent each object accurately while maintaining short rendering times.
% Note that the coordinates of sampled points are first projected into normalized coordinates using the box scale of local frames to enable each local NeRF to learn the scale-independent representation.
% The bounding box $\mathbf{b}$ of the local frame in global coordinate can be transformed into a canonical bounding box by ${(\mathbf{b}} - \boldsymbol{p}) / \mathbf{s}$.
% Considering the rendering efficiency, we only calculate the valid points, interacted with the boxes, and set all the empty points with a constant background color.
%
% The appearance of a set object representations depends on its interaction with the scene and illumination which should be decided by the local frame location.
% To ensure the volumetric consistency, we only calibrate the emitted color with scene location, while the gradient still can be propagated.
% Since the overall color depends on both the global  positions $({x}_w, {y}_w, {z}_w)$ and ray directions $({\phi}_d, {\theta}_d)$, the global color embedding is learned based on both the positions and ray directions.
% Since the overall color depends on both the global  positions $({x}_w, {y}_w, {z}_w)$ and ray directions $({\phi}_d, {\theta}_d)$, the global color embedding is learned based on both the positions and ray directions.
% \subsection{The Proposed CompoNeRF}
% \subsubsection{Composition Module}
% CompoNeRF aims to composite multiple NeRFs to reconstruct multi-object scenes with both box and prompt guidance.
% %
% Our framework, as shown in \cref{fig:framework}, applies the AABB ray intersection test algorithm to check for intersections on each box in the global frame. We then samples $\boldsymbol{x}_g$ within the ray box intervals, and project them to $\boldsymbol{x}_l$ to infer  $\left(\boldsymbol{C}_l, {\boldsymbol{\sigma}_l}\right)$ in separate NeRF models. 
% %
% We then utilize volume rendering to obtain rendered views for each local frame respectively. 
% %
% After that, they would be passed on to our tailored composition Module to infer 
% $\left(\boldsymbol{C}_g, {\boldsymbol{\sigma}_g}\right)$
% for global rendering. 
% Next, we match local and global texts with their corresponding image outputs by SDS losses. 
% We also support recomposition by passing samples from cached models into $\boldsymbol{x}_l$ to continue the above process.
\begin{figure}[t!]
    \centering
    \includegraphics[width=\linewidth]{figures/abls.pdf}
    % \vspace{-22pt}
    % \caption{Ablation study on text guidance. (a) without local SDS losses. (b) without global SDS losses. (c) vanilla SDS losses without perturb and average scoring~\cite{wang2022score}. (d) full model.}
    \caption{\textbf{Design Impact Comparison: Density vs. Color-based Methods.} The top row illustrates the density-based approach's detailed rendering and quick convergence in the 'table wine' scene. The bottom row highlights the color-based method's enhancements and its drawbacks, such as geometric and shadow inaccuracies, particularly in close-up views and slow convergence.
    % \textbf{(a)} global text guidance(integrating local frames by \cref{eq:multi_volrend}) and global calibration(integrating local frames, then aligning the rendering result directly with the full text). 
    }
    \label{fig:abls}
    % \vspace{-20pt}
\end{figure}
\subsection{The Proposed CompoNeRF}
\subsubsection{Composition Module}
CompoNeRF is designed to composite multiple NeRFs to reconstruct scenes featuring multiple objects, utilizing guidance from both bounding boxes and textual prompts. Within our framework, depicted in \cref{fig:framework}, the Axis-Aligned Bounding Box (AABB) ray intersection test algorithm is applied to ascertain intersections across each box in the global frame. Subsequently, we sample points \(\boldsymbol{x}_g\) within the intervals of the ray-box and project them to \(\boldsymbol{x}_l\) to deduce the corresponding color \(\boldsymbol{C}_l\) and density \(\boldsymbol{\sigma}_l\) within individual NeRF models.
%
These properties are processed through our composition module to infer the global color \(\boldsymbol{C}_g\) and density \(\boldsymbol{\sigma}_g\), crucial for the global rendering.
%
Volume rendering techniques~\cite{kajiya1984ray} are then employed to procure the rendered views for both local and global frames. We propose dual SDS losses to ensure coherence between the image outputs and their corresponding textual descriptions. Additionally, our approach facilitates recomposition by channeling samples from cached models back into local frames along with the text revision, thereby streamlining the integration.

% As shown in \cref{fig:abls}(a), we verify its necessity by dropping $\nabla \mathcal{L}_{\text{SDS}_g}$. 
% %
% Compared with our full model, its layout does not fit our shared sense of a room, \ie, \emph{nightstand} is usually lower than \emph{bed}; \emph{lamp} needs a base to support it. Additionally,  it lacks global consistency, such as light reflection, to make it more realistic. 
% %
% Therefore, we leverage the full text semantics to ensure consistent global rendering across local frames. 
% %
% Instead of conditioning the global rendering view with the full prompt directly, we note that global calibration is necessary for geometry and color to be learned sufficiently.
% For example, we observe that geometric completeness and texture of \emph{nightstand} are not ideal. Although reflection appears around \emph{nightstand}, \emph{bed} is stripped of the light. 
% %
% Therefore, we opt to leverage the correlation between the rendering output of the combined NeRFs and the overall semantics to perform multi-object scene reconstruction.  
%

\noindent\textbf{Global Composition.}
The independent optimization of each local frame may inadvertently result in a lack of global coherence within the scene. To address this, our scene composition process is designed to integrate these frames, thereby achieving a more consistent result.
%
Before exploring the specifics of the module, it is imperative to discuss two critical design decisions within the composition module, as depicted in \cref{fig:framework}.
%
Upon integrating the properties inferred from \(\boldsymbol{x}_g\) into the composition module, they are fine-tuned through gradients derived from the global SDS loss.  This process leads to a critical consideration: the necessity and implications of refining the global density \(\boldsymbol{\sigma}_g\). This can be divided into two approaches: \textbf{1) Density-based:} The advantage of adjusting \(\boldsymbol{\sigma}_g\) is that it can adjust geometry, thus yielding a scene more congruent with the global text prompt. 
However, this comes at the cost of potentially compromising the optimal color \(\boldsymbol{C}_g\), as calibrating \(\boldsymbol{\sigma}_g\) introduces more uncertainty for subsequent color refinement as it requires prior density features $\boldsymbol{h}$ as shown at \cref{fig:compo}. 
\textbf{2) Color-based:} Conversely, directly employing \(\boldsymbol{\sigma}_l\) mitigates this uncertainty but at the expense of reduced geometric control, presenting a challenging balance to strike in the pursuit of precise scene composition.
% , which may lead to suboptimal outcomes.
%
After thorough experiments, exemplified in \cref{fig:abls}, we have opted for the density-based approach to refine \(\boldsymbol{\sigma}_g\)  prioritizing both \textbf{accuracy and efficiency}. The test revealed that it excels in rendering intricate details, such as enhanced wood grain textures and more naturally contoured 'salad', as accentuated by boxes. This method also demonstrated a swifter convergence rate. Conversely, while the color-based improved reflections and reduced flickering on the 'wine cup', it was plagued by issues such as sparse density, which adversely brings holes at the base of the 'cup' and the corner of the 'table'.
Furthermore, upon close examination, it becomes evident that shadow artifacts of 'wine' on the 'table' are pronounced, suggesting that its disadvantages outweigh its advantages.
%  in this context
% \textbf{Global Composition.}
% Each local frame is optimized independently, causing a lack of global connections for scene composition.
% Before delving into module details, there are two choices (see \cref{fig:framework}) on the composition module design we need to elaborate on first. 
% %
% In \cref{fig:framework}, by taking $\boldsymbol{x}_g$ into the composition module, their inferred properties are calibrated with gradients propagated from the global SDS loss. 
% However, it remains unclear whether $\boldsymbol{\sigma}_g$ should be refined or not. 
% %
% The trade-off on its usage is the density adjustment bringing a more reasonable layout and more geometric details that fit the global text prompt. While its potential downside is that $\boldsymbol{C}_g$ may not be optimal as $\boldsymbol{\sigma}_g$ has more uncertainty compared to $\boldsymbol{\sigma}_l$, bringing sub-optimal rendering results. 

% We choose the density-based method after comparing them with the experiment shown in \cref{fig:abls}. 
% %
% Specifically, we test both designs on the scene \emph{table wine} and discover that the density-based design provides more intrinsic details(as indicated by green boxes), \eg, enriched wood grains, and a more natural shape for \emph{salad} and has much faster convergence speed. In contrast, the color-based method enhances the reflection and smooths flickering on \emph{wine cup}, (as indicated by red boxes), but it suffers from 1) sparse density, resulting in poorly generated geometry at the base of  \emph{cup} and the wood \emph{table} corner. Additionally, shadow artifacts appeared on \emph{table} when viewed up close, outweighing benefits of the color-based method.

\begin{figure}[t!]
    \centering
    \includegraphics[width=\linewidth]{figures/compo_module.pdf}
    % \vspace{-24pt}
    % \caption{Ablation study on text guidance. (a) without local SDS losses. (b) without global SDS losses. (c) vanilla SDS losses without perturb and average scoring~\cite{wang2022score}. (d) full model.}
    \caption{\textbf{Detail of Composition module}: density-based design. 
    }
    \label{fig:compo}
    % \vspace{-18pt}
\end{figure}
\noindent\textbf{Network Design.}
The compositional framework of our network, as delineated in \cref{fig:compo}, is predicated on an architecture that employs a suite of MLPs, represented as \(\{\boldsymbol{\theta}_l\}_{l=1}^{m}\),  each dedicated to a distinct local frame. To harmonize \(\boldsymbol{\sigma}_l\) and \(\boldsymbol{C}_l\), we incorporate global MLPs, including density calibrator $f_{\boldsymbol{\theta}_{g_d}}$ and color calibrator $f_{\boldsymbol{\theta}_{g_c}}$.
%
A transformation module complements this system, tasked with maintaining the spatial coherence between the global and local frames. It governs the transformation of sampling points $\boldsymbol{x}$, ray directions $\boldsymbol{d}$, and adjacent sampling distances $\delta$. This module also orders the points $\{\boldsymbol{x}_{g,j}\}_j$ by their distance to the global camera origin $\boldsymbol{o}_g$, ensuring that each local point $\boldsymbol{x}_l$ is accurately matched with its corresponding global point $\boldsymbol{x}_g$ for subsequent volume rendering. 
%
The network design is:
{
\setlength\abovedisplayskip{4.5pt}
\setlength\belowdisplayskip{4.5pt}
\begin{align}
\label{eq:g_c_d}
{\boldsymbol{\sigma}_g}  &= \alpha_d f_{\boldsymbol{\theta}_{g_d}}({\boldsymbol{x}_g}) + \boldsymbol{\sigma}_l, \\
{\boldsymbol{C}_g}  &= \alpha_c f_{\boldsymbol{\theta}_{g_c}}(\boldsymbol{h}, {\boldsymbol{d}_g}) + \boldsymbol{C}_l. 
\end{align}}In contrast to the local frames, the global frame's color output $\boldsymbol{C}_g$ is inferred based on $\boldsymbol{h}$ and conditional on $\boldsymbol{d}_g$ to enable a view-dependent lighting effect.
% Denote the density features as $\boldsymbol{h}$. 
%
%
Residual learning is leveraged here, where \(\boldsymbol{\sigma}_l, \boldsymbol{C}_l\) serve as foundational elements that support the learning of global density \(\boldsymbol{\sigma}_g\) and color \(\boldsymbol{C}_g\). The parameters \(\alpha_d, \alpha_c\) are adjustable, allowing fine-tuning of the influence that local components exert on the global outputs.
%
It is imperative to acknowledge that in our color-based method, density calibration is intentionally excluded to concentrate solely on the refinement of color dynamics as shown at \cref{fig:framework}. This is achieved by conditioning the process on both spatial and directional global inputs \((\boldsymbol{x}_g, \boldsymbol{d}_g)\), as demonstrated in the following equations:
\begin{align}
\setlength\abovedisplayskip{4.5pt}
\setlength\belowdisplayskip{4.5pt}
\label{eq:g_c_c}
\boldsymbol{\sigma}_g = \boldsymbol{\sigma}_l, \quad
{\boldsymbol{C}_g} = \alpha_c f_{\boldsymbol{\theta}_{g_c}}({\boldsymbol{x}_g}, {\boldsymbol{d}_g}) + \boldsymbol{C}_l.
\end{align}
The integration of extra $\boldsymbol{x}_g$ aims to facilitate a fair comparison under same inputs with the density-based. It enhances the visual appeal of effects like the wine cup's reflection, as demonstrated in \cref{fig:abls}. However, this method is not without its compromises. It tends to produce artifacts and is characterized by a slower convergence rate. Additionally, this approach limits the ability to precisely control density, subsequently impacting the intricate geometric details.


\begin{figure*}[t!]
    \centering
    \includegraphics[width=\linewidth]{figures/sota.pdf}
    % \vspace{-24pt}
    \caption{\textbf{Qualitative comparison with other text-to-3D methods using multi-object text prompts}. Cases 1-3 demonstrate simpler settings characterized by compositions involving two objects. In contrast, Cases 4-8 delve into more intricate scenarios featuring compositions with more than two objects. Smaller images are presented to illustrate the generated local NeRFs(partially shown in Cases 4-8).}
    \label{fig:sota}
    % \vspace{-5pt}
\end{figure*}
%
% \begin{table*}[t!]
% \centering
% \resizebox{\textwidth}{!}
% {
% \begin{tabular}{cccccccc}
% \toprule
% Method            & \rotatebox{60}{table wine}  & \rotatebox{60}{teddy monkey} & \rotatebox{60}{computer mouse} & \rotatebox{60}{bed room}  & \rotatebox{60}{chess} & \rotatebox{60}{pisa tower} & \rotatebox{60}{astronaut} & \rotatebox{60}{tesla}  \\ \midrule
% LatentNeRF  & 21.55 & 27.38 & 17.13 & 21.86 & 31.19 & 24.31 & 27.07 & 25.16 \\
% SJC & 23.33 & 27.37 & 18.00 & 22.54 & 30.53 & \textbf{26.18 }& 27.84 & 23.55 \\
% CompoNeRF & \textbf{32.68} & \textbf{28.57}	 &\textbf{ 22.34} &\textbf{ 28.65} & \textbf{31.45} & \textbf{28.96} & 25.82 & 25.95 & 24.42 & \textbf{32.71} & \textbf{26.13 }& \textbf{26.38} & \textbf{30.98} & \textbf{33.37} \\
% \bottomrule
% \end{tabular}
% }
% \vspace{-10pt}
% \caption{Performance of our CompoNeRF in different 3D scenes. We use CLIP score \cite{parmar2023zero,zhang2023sine,wang2023imagen} as our evaluation metric, which is a common evaluation metric in text-to-image generation tasks to evaluate the similarity of the generated image to the text prompt. }
% \label{perclass}
% \end{table*}
%
\begin{table*}[t!]
% \scalebox{0.8}
\renewcommand{\arraystretch}{1.2}
\fontsize{4pt}{4pt}
\selectfont 
\centering
% \vspace{-8pt}
\resizebox{\textwidth}{!}
{
% \begin{tabular}{lcccccccc}
% \hline
% Method     & table\_wine    & tesla          & pyramid        & chess          & apple and banana      & astronaut      & glass\_balls   & Eiffel\_tower    \\ \hline
% LatentNeRF & 21.55          & 25.16          & 27.43          & 31.19          & 27.69          & 27.07          & 29.51          & 26.32          \\
% SJC        & 23.33          & 23.55          & 25.62          & 30.53          & 28.21          & 27.84          & 28.76          &27.41 \\
% \textbf{CompoNeRF(Ours)}     & \textbf{32.68} & \textbf{26.13} & \textbf{28.96} & \textbf{31.45} & \textbf{33.37} & \textbf{32.71} & \textbf{30.98} & \textbf{28.44}          \\ \hline
% \end{tabular}
\begin{tabular}{lcccccccc}
\hline
Method                   & Case 1         & Case 2         & Case 3         & Case 4         & Case 5         & Case 6         & Case 7         & Case 8         \\ 
\hlineB{1.1}
LatentNeRF               & 25.16          & 27.07          & 27.69          & 31.19          & 21.55          & 26.32          & 27.43          & 29.51          \\
SJC                      & 23.55          & 27.84          & 28.21          & 30.53          & 23.33          & 27.41          & 25.62          & 28.76          \\
\textbf{CompoNeRF (Ours)} & \textbf{26.13} & \textbf{32.71} & \textbf{33.37} & \textbf{31.45} & \textbf{36.06} & \textbf{28.44} & \textbf{28.96} & \textbf{30.98} \\ \hlineB{1.1}
\end{tabular}
}

% \vspace{-6pt}
\caption{\textbf{Performance comparison of our CompoNeRF in different 3D scenes}. For our evaluation metric, we utilize the average of CLIP scores~\cite{parmar2023zero,zhang2023sine,wang2023imagen} across different views, which serve to assess the similarity between the generated images and the global text prompt. }
\label{tb:perclass}
\end{table*}
% \cref{fig:framework} depicts the network architecture of the composition module. Denote $m$ as local MLP $\{\boldsymbol{\theta}_l\}_{l=1}^{m}$ for each local frame. Then, we introduce the global MLPs including density $\boldsymbol{\theta}_{g_d}$ and $\boldsymbol{\theta}_{g_c}$ calibrators to refine $\boldsymbol{\sigma}_l$ and $\boldsymbol{C}_l$. 
% %
% In detail, the network design is, 
% {
% % \setlength\abovedisplayskip{4.5pt}
% % \setlength\belowdisplayskip{4.5pt}
% \begin{align}
% \label{eq:g_c_d}
% {\boldsymbol{\sigma}_g}  &= \alpha_d \boldsymbol{\theta}_{g_d}({\boldsymbol{\sigma}_l}) + \boldsymbol{\sigma}_l, \\  
% {\boldsymbol{C}_g}  &= \alpha_c \boldsymbol{\theta}_{g_c}({\boldsymbol{C}_l},  {\boldsymbol{d}_g}) + \boldsymbol{C}_l, 
% \end{align}}
% %
% where residual $\boldsymbol{\sigma}_l, \boldsymbol{C}_l$ assist in learning $\boldsymbol{\sigma}_g$ and $\boldsymbol{C}_g$, while $\alpha_d, \alpha_c$ balance their contribution as learnable parameters.
% %
% Note that the color-based omits density calibration, and simply uses the shared color refinement.



% The 3D boxes are only used for the spatial configuration of local NeRFs, while the implicit representation of local NeRFs is inferred by the canonical samples inside the local frame without considering the global relationship across different objects.
% To relieve such location-dependent effects, we further calibrate the output color and density from the local NeRF with global coordinates $({\boldsymbol{x}}_g, {\boldsymbol{y}}_g, {\boldsymbol{z}}_g)$ and ray directions $\left({\boldsymbol{\phi}}_{d}, {\boldsymbol{\theta}}_{d}\right)$ as the conditional input.
% % to inject the global visual clues.
% %
% %
% Specifically, we adopt a shared MLP $\boldsymbol{\theta}_{g}$ to calibrate all the predicted object colors, that is,
% {\setlength\abovedisplayskip{4.5pt}
% \setlength\belowdisplayskip{4.5pt}
% \begin{align}
% \label{eq:MLP_dyn_2}
% {\boldsymbol{C}_g} = {\boldsymbol{C}_l} + \boldsymbol{emb}_{g} &= {\boldsymbol{C}_l} + \boldsymbol{\theta}_{g}({\boldsymbol{x}}_g, {\boldsymbol{y}}_g, {\boldsymbol{z}}_g, {\boldsymbol{\phi}}_{d}, {\boldsymbol{\theta}}_{d}),
% \end{align}}
% where ${\boldsymbol{C}_l}$ is the color predicted by the local NeRF.
% Therefore, the scene color can preserve the view-consistent behavior from the original architecture and add consistency across poses for the volumetric density.
% Since the color and density values share the same latent expression in $({\boldsymbol{x}}_l, {\boldsymbol{y}}_l, {\boldsymbol{z}}_l)$, we only calibrate the emitted scene color explicitly with the scene location, as the densities of local NeRFs also are implicitly adjusted during optimization.

% \noindent \textbf{Global and Local Volumetric Rendering.}
% After compositing all the interacted points, each ray $\boldsymbol{r}_i$ collects a set sampling points by $\{\boldsymbol{t}_{i,j,n} \}_{j=1, n=1}^{m_j, N}$, where $m_j$ is the number of the hit object.
% For each sampling point, the inference results with the respective 3D representations are the local color $\boldsymbol{c}_{l}$, global color $\boldsymbol{c}_{g}$, and density $\sigma$.

% In fact, the local view $\hat{C}_{l,j}$ of single object $j$ also can be rendered by the sampled points  belongs to the same local frames as shown at Fig.~\ref{fig:framework}.

\subsubsection{Recomposition}
Our architecture advances scene reconstruction by providing an intuitive interface for layout manipulation.  This capability is crucial for the reconfiguration of scene elements into novel scenes, as depicted in \cref{fig:framework}. Here, the input panel allows for adjustments in the attributes of bounding boxes, such as modifying the position and scale of the 'apple' bounding box prior to composition. The refinement process further involves sampling ray-box intervals from the global frame, leading to transformed coordinates with the corresponding ray samples that are then incorporated into the pipeline, as demonstrated in \cref{fig:compo}.
%
Each bounding box represents an individual NeRF, providing the flexibility to move, scale, or remove elements as needed. CompoNeRF's capabilities also extend to textual edits, exemplified by the transformation of 'wine' into 'juice'.
%
Since NeRFs have been well trained, we only finetune \(\theta_g, \theta_l\) to align text prompts to promote consistency of both local and global views.
%
Moreover, the NeRFs once retrained within the edited scene, are also structured to be decomposable and cacheable in future scene compositions.
% Our CompoNeRF architecture facilitates the seamless reconstruction of scenes leveraging existing models. It enables precise editing of bounding boxes parameterized by \(\{\boldsymbol{\theta}_l\}_{l=1}^{m}\), allowing for their reconfiguration into new layouts. Refer to \cref{fig:framework}, the input panel permits the modification of attributes such as the position and scale of the 'apple' node's bounding box prior to composition. The process is further refined by sampling from the updated ray-box intervals within the global frame, which are then projected onto \(\boldsymbol{x}_l\), ensuring a streamlined reconstruction that integrates the 'apple' effectively. This addition is executed with careful attention to color consistency, positioning the 'apple' adjacent to the 'French bread' to complement the scene's overall palette. Each bounding box represents an individual NeRF, which means they can be manipulated through moving, scaling, and removal operations. CompoNeRF also extends its editing prowess to textual modifications, as evidenced by the 'wine cup' now appearing filled with juice—a change propagated through both subtexts and the global test. 
% %
% Since NeRFs have been well trained, we only finetune $\theta_g, \theta_l$ to align text prompts to promote consistency of both local and global views . 
% %
% Moreover, the NeRFs, once retrained within the reimagined scene, are also structured to be decomposable and cacheable for subsequent scene compositions.

% , as shown in Fig.~\ref{fig:framework}.
% For each scene described by the multi-object text prompt $T$, we
% To enhance the guidance of local representations, we use the local text prompt $T_l \subseteq T$ of a single object to optimize the local NeRFs in local views.
% The scene views $\hat{\boldsymbol{X}}_g=\{\hat{\boldsymbol{C}}_{g,i}\}_{i=1}^{H\times W}$ is obtained from the predicted pixel values of $H \times W$ rays by compositing all the ray-box interaction values.
% Similarly, the rendered view $\hat{\boldsymbol{X}}_{l,j}$ of the local frame $\boldsymbol{\theta}_j$ without compositing other objects can be calculated by $\hat{\boldsymbol{C}}_{l,j}$, as depicted in Sec.~\ref{ssec:render}.
% We use the local color instead of the globally calibrated color to obtain a local view because the local NeRF should learn the object identity unrelated to its placed position, as the position can be different during user edition.
% % Compared to cropping the local region from a global view for training, separate rendering can avoid the undesired information from other objects brought by the occlusion and resolution adjustments.
% Formally, we employ the following loss as the learning objective,
\begin{figure*}[t!]
    \centering
    \includegraphics[width=\linewidth]{figures/editing.pdf}
    % \vspace{-23pt}
    \caption{\textbf{Scene Editing Outcome:} Demonstrated here are the stages of our recomposition, utilizing cached source scenes. Each NeRF is individually identified by colorful labels. These decomposed nodes are then positioned in the initial layout and subsequently calibrated to form the final composition. The detailed description of the ambient environment is underscored, enhancing the scene's realism.}
    \label{fig:app}
    % \vspace{-12pt}
\end{figure*} 
\subsubsection{Optimization}
\label{sec:optimization}
During optimization, our method employs dual text guidance to align rendering results with both global and local textual descriptions. The optimization objective is:
{
\small
\setlength\abovedisplayskip{2pt}
\setlength\belowdisplayskip{2pt}
\begin{equation}
\label{eqn:loss_f}
\mathcal{L}= {\alpha_g}\nabla\mathcal{L}_{\text{SDS}}(\hat{\boldsymbol{X}}_{g}, T) + {\alpha_l}\sum_{j=1}^{m} \nabla\mathcal{L}_{\text{SDS}}(\hat{\boldsymbol{X}}_{l,j}, T_{l,j}) + \beta\mathcal{L}_{\text{sparse}},\nonumber
\end{equation}
}where $T$ signifies the global text prompt, while $T_{l}$ pertains to a specific object within the global context. The hyperparameters $\alpha_{g}, \alpha_{l}$, and $\beta$ modulate the respective loss weights. 
% $\nabla \mathcal{L}_{\text{SDS}}$ is the score distillation sampling loss, as described in Sec.~\ref{sec:background}.
As suggested in~\cite{metzer2022latent}, we use $L_{\text{sparse}}$ included to penalize the binary entropy of local NeRFs' densities, thereby mitigating the issue of extraneous floating radiance.
Additionally, incorporating directional cues such as "front view" or "side view" into the input text, as suggested by \cite{poole2022dreamfusion,metzer2022latent} proves beneficial in specifying camera poses during the training phase, further enhancing the alignment of our generated scenes with the intended perspectives.
% Note that the global calibration in the scene frame can adaptively revise both $({C}_l, {\sigma})$ in local NeRF with $\nabla \mathcal{L}_{SDS}$ along with the back-propagating gradient.

\section{Experimental Setup}
\label{sec:experiments}
\begin{figure}[t]
    \centering 
    \hspace{-.04\columnwidth}
    \includegraphics[width=1.025\columnwidth]{results/VOC/figures/pareto_example.pdf}
    \caption{\textbf{Selecting models for evaluation.} For each configuration, we evaluate every model at every checkpoint and measure its performance across various metrics (\fone, \epg, \iou) on the validation set; \ie every point in the left graph corresponds to one model (for \bcos models optimized via the \epgloss loss at the input layer). Instead of evaluating a single model on the test set, we evaluate \emph{all Pareto-dominant} models, as indicated in the center and right plot.
    % \moritz{Did we not update the results to be consistent with this? I distinctly remember creating the plots for this. (The Pareto front here as a lot more points than those in the result figures...)}
    }
    \label{fig:pareto_example}
\end{figure}

In this section, we describe our experimental setup
and how we select the best models across metrics. {Full training details can be found in the supplement.} We evaluate across the full sweep of combinations of choices for each category, and discuss our results in \cref{sec:results}. 

\myparagraph{Datasets:} We evaluate on \voc \citeMain{everingham2009pascal} and \coco \citeMain{lin2014microsoft} for multi-label image classification. {In \cref{sec:results:waterbirds}, to understand the effectiveness of model guidance in mitigating spurious correlations, we also evaluate on the synthetically constructed Waterbirds-100 dataset \citeMain{sagawa2019distributionally,petryk2022guiding}, where landbirds are perfectly correlated with land backgrounds on the training and validation sets, but are equally likely to occur on land or water in the test set (similar for waterbirds and water). With this dataset, we evaluate model guidance for suppressing undesired features.}

\myparagraph{Attribution Methods and Architectures:} As described in \cref{sec:method:attributions}, we evaluate with \ixg \citeMain{shrikumar2017learning}, \intgrad \citeMain{sundararajan2017axiomatic}, \bcos \citeMain{bohle2022b}, and \gradcam \citeMain{selvaraju2017grad} using models with a \resnet \citeMain{he2016deep} backbone. For \intgrad, we use an \xdnn \resnet \citeMain{hesse2021fast} to reduce the computational cost, and a \bcos \resnet for the \bcos attributions. We optimize the attributions at the input and final layer\footnote{As typically used in \ixg (input) and \gradcam (final) respectively.}; for intermediate layer results, see supplement. Given the similarity of the results between \gradcam and \ixg, and since \bcos attributions performed better than \gradcam for \bcos models, we show \gradcam results in the supplement. 
All models were pretrained on \imagenet \citeMain{imagenet}, and model guidance was performed starting from a baseline model fine-tuned on the target dataset.

\myparagraph{Localization Losses:} As described in \cref{sec:method:losses}, we compare four localization losses in our evaluation: (i) \energyloss, (ii) \loneloss \citeMain{gao2022aligning,gao2022res}, (iii) \ppceloss \citeMain{shen2021human}, and (iv) \rrrloss (cf.~\cref{sec:method:losses}, \citeMain{ross2017right}).

\myparagraph{Evaluation Metrics:} As discussed in \cref{sec:method:metrics}, we evaluate both for classification and localization performance of the models. For classification, we report the F1 scores, similar results with \map scores can be found in the supplement. For localization, we evaluate using the \epg and \iou scores.

\myparagraph{Selecting the best models:} As we evaluate for two distinct objectives (classification and localization), it is non-trivial to decide which models to select during training. \Eg, a model that provides the best classification performance might provide significantly worse localization performance than a model that provides slightly lower classification performance but much better localization. Finding the right balance and deciding which of those models in fact constitutes the `better' model depends on the preference of the end user. 
Hence, instead of selecting models based on a single metric, we select the set of Pareto-dominant models \citeMain{pareto1894massimo,pareto2008maximum,backhaus1980pareto} across three metrics---F1, \epg, and \iou---for each training configuration, as defined by a combination of attribution method, layer, and loss. Specifically, as shown in \cref{fig:pareto_example}, we train for each configuration using three different choices of $\lambda_\text{loc}$, and select the set of Pareto-dominant models among all checkpoints (epochs and $\lambda_\text{loc}$). This provides a more holistic view of the general trends on the effectiveness of model guidance for each configuration.
% !TEX root = root.tex

\section{Conclusions and Future Work}
\label{sec:5_discussion}
Our work unifies the SF-GPI and value composition to the continuous concurrent composition framework and allows reconstructing task policy from a set of primitives. The proposed method was extended to composition at the action component level. We demonstrate in the Pointmass environment that our multi-task agents can reconstruct the task policy from a set of primitives in real time and transfer the skills to solve unseen tasks while the single-task performance is competitive with SAC.
This flexible framework incorporates well with the reward-shaping techniques, such as entropy regularization, curiosity\cite{pmlr-v70-pathak17a}, etc. In addition, the task-agnostic property should benefit the autotelic framework \cite{colas2022autotelic} where agents can set goals and curriculum for themselves \cite{narvekar2020curriculum}. 

However, the primary concern at this stage is whether the proposed approach can scale to higher dimensional problems. Additionally, two important topics are left as future works. First, look for the corresponding value composition for DAC. A good starting point might be thinking of the MSF composition with weights evaluated by GPE. 
Second, the optimality of each composition method. One might start with bounding the loss incurred by the policy and value composition. 
 \section{Conclusion}
 In this paper, we have presented a tactile manipulation system that is able to rotate different objects without vision. We showed an end-to-end reinforcement learning framework to learn tactile dexterity over the proposed system. We carried out experiments both in simulation and real to demonstrate its effectiveness. Our work demonstrated that we are able to achieve tactile dexterity as humans in real for the first time. In the future, there are many promising future directions to investigate, such as exploring the use of a more dense contact sensor array and scaling up the system to solve more diverse tasks. We hope that our work can pave the way for more intelligent robot hands.


{\small
\bibliographystyle{ieee_fullname}
\bibliography{references}
}

\clearpage

\appendix

\renewcommand\thesection{\Alph{section}}
% \setcounter{section}{0}
\numberwithin{equation}{section}
\numberwithin{figure}{section}
\numberwithin{table}{section}
\renewcommand{\thefigure}{\thesection\arabic{figure}}
\renewcommand{\thetable}{\thesection\arabic{table}}
% \renewcommand{\thetable}{\arabic{table}}
% \renewcommand\appendixtocname{section}
\crefname{appendix}{Sec.}{Secs.}

{\onecolumn 
%\maketitle
{\begin{center}
\Large\bf
\phantom{skip}\\[.25em]
% \bigskip
{Studying How to Efficiently and Effectively Guide Models with Explanations}\\[1em]
\large
Appendix
\end{center}
}
\newcommand{\additem}[2]{%
\item[\textbf{(\ref{#1})}] 
    \textbf{#2} \dotfill\makebox{\textbf{\pageref{#1}}}
}

% \newcommand{\addsubitem}[2]{%
%     \\[.5em]\indent\hspace{1em}
%     \textbf{(\ref{#1})}
%     #2 %\dotfill\makebox{\textbf{\pageref{#1}}}
% }

\newcommand{\addsubitem}[2]{%
% \vspace{.2em}
    \textbf{(\ref{#1})}\hspace{1em}
    #2\\[.1em] %\dotfill\makebox{\textbf{\pageref{#1}}}
}

\newcommand{\adddescription}[1]{\newline
\begin{adjustwidth}{.25cm}{.25cm}
#1
\end{adjustwidth}
}
% \begin{multicols}{2}

% \begin{minipage}[c]{\textwidth}


{\vspace{2em}\bf\large Table of Contents\\[1em]}

In this supplement to our work on using explanations to guide models, we provide:
\\[1em]

\begin{adjustwidth}{1cm}{1cm}
\begin{enumerate}[label={({\arabic*})}, topsep=1em, itemsep=.25em]
    \additem{supp:sec:quali}{
    Additional qualitative results (VOC and COCO)
    }
    \adddescription{ 
    In this section, we present additional \emph{qualitative} results. In particular, we provide:
    
    \addsubitem{supp:sec:main:qualitative:examples}{Detailed comparisons between \textbf{models, layers, and losses}. (\vocs + \cocos).}
    \addsubitem{supp:sec:main:qualitative:dilation}{Additional visualizations for training with \textbf{dilated bounding boxes} (\vocs + \cocos).}
    }    
    \additem{supp:sec:quanti}{
    Additional quantitative results (VOC and COCO)
    }
    \adddescription{ 
    In this section, we present additional \emph{quantitative} results. In particular, we show:
    
    \addsubitem{supp:sec:quantitative:classvsloc}{The remaining \textbf{localization vs.\ accuracy comparisons} (\vocs + \cocos).}
    \addsubitem{supp:sec:quantitative:gradcam}{The results of guiding models via \textbf{\gradcam}. (\vocs + \cocos).}
    \addsubitem{supp:sec:quantitative:intermediate}{Results for optimizing at \textbf{intermediate layers} (\vocs).}
    \addsubitem{supp:sec:quantitative:segmentepg}{Results for measuring \textbf{on-object \epg scores} (\vocs).}
    \addsubitem{supp:sec:quantitative:limited}{Additional analyses regarding training with \textbf{few annotated images} (\vocs).}
    \addsubitem{supp:sec:quantitative:dilation}{Additional analyses regarding the usage of \textbf{coarse bounding boxes} (\vocs).}
    \addsubitem{supp:sec:quantitative:morearchs}{Additional results using other \textbf{model backbones}.}
    }
    \additem{supp:sec:waterbirds}{
    Additional results on the Waterbirds dataset
    }
    \adddescription{ 
    In this section, we provide additional results for the \waterbirds dataset. In particular, we provide full results regarding \textbf{classification performance} with and without model guidance as well as \textbf{additional qualitative visualizations} of the attribution maps.\\
    
    }
    \additem{supp:sec:implementation}{
    Implementation details
    }
    \adddescription{ In this section, we provide relevant implementation details; note that all code will be made available upon publication. In particular, we provide:
    
    \addsubitem{supp:sec:implementation:training}{Training details across datasets (\vocs + \cocos + Waterbirds).}
    \addsubitem{supp:sec:implementation:bcos}{Implementation details for twice-differentiable \bcos models.}
    }
    \additem{supp:sec:full}{
    Full results across all experiments.
    }
    \adddescription{ 
    Given the large amount of experimental results, in each of the preceding sections we show only a sub-selection of those results for improved readability. In section \ref{supp:sec:full}, we provide the \emph{full} results across datasets, models, layers, experiments, and metrics, to peruse at the reader's convenience.    
    ~\\
    }
\end{enumerate}
\end{adjustwidth}
}

\setlength{\parskip}{.5em}
\clearpage
\section{Additional Qualitative Results (\vocs and \cocos)}
\label{supp:sec:quali}

\subsection{Qualitative Examples Across Losses, Attribution Methods, and Layers}
\label{supp:sec:main:qualitative:examples}

\begin{figure}
    \centering
    \begin{subfigure}[c]{\textwidth}
    \centering
    \textbf{\large \voc.}\\\vspace{.25cm}
    \begin{subfigure}[c]{.475\columnwidth}
    \centering
    \textbf{Input}\\\vspace{.25cm}
    \includegraphics[width=\textwidth]{results/VOC/figures/qualitative/loss_comp_bcos_Input.pdf}
    \end{subfigure}
    \begin{subfigure}[c]{.475\columnwidth}
    \centering
    \textbf{Final}\\\vspace{.25cm}
    \includegraphics[width=\textwidth]{results/VOC/figures/qualitative/loss_comp_bcos_Final.pdf}
    \end{subfigure}
    \caption{\textbf{\bcos \resnet}.}
    \label{fig:supp:quali_voc_1:bcos}
    \end{subfigure}
    \begin{subfigure}[c]{\textwidth}
    \centering
    \begin{subfigure}[c]{.475\columnwidth}
    \includegraphics[width=\textwidth]{results/VOC/figures/qualitative/loss_comp_normal_Input.pdf}
    \end{subfigure}
    \begin{subfigure}[c]{.475\columnwidth}
    \includegraphics[width=\textwidth]{results/VOC/figures/qualitative/loss_comp_normal_Final.pdf}
    \end{subfigure}
    \caption{\textbf{\vanilla \resnet}.}
    \label{fig:supp:quali_voc_1:vanilla}
    \end{subfigure}
    \begin{subfigure}[c]{\textwidth}
    \centering
    \begin{subfigure}[c]{.475\columnwidth}
    \includegraphics[width=\textwidth]{results/VOC/figures/qualitative/loss_comp_xdnn_Input.pdf}
    \end{subfigure}
    \begin{subfigure}[c]{.475\columnwidth}
    \includegraphics[width=\textwidth]{results/VOC/figures/qualitative/loss_comp_xdnn_Final.pdf}
    \end{subfigure}
    \caption{\textbf{\xdnn \resnet}.}
    \label{fig:supp:quali_voc_1:xdnn}
    \end{subfigure}
    \caption{Qualitative examples from the \textbf{\vocs dataset}. In particular, this figure allows to compare between models (\textbf{major rows}, \ie (a), (b), and (c)) losses (\textbf{major columns}) and layers (\textbf{left+right}) for multiple images (\textbf{minor rows}).}
    \label{fig:supp:quali_voc_1}
\end{figure}
\begin{figure}
    \centering
    \begin{subfigure}[c]{\textwidth}
    \centering
    \textbf{\large \coco.}\\\vspace{.25cm}
    \begin{subfigure}[c]{.475\columnwidth}
    \centering
    \textbf{Input}\\\vspace{.25cm}
    \includegraphics[width=\textwidth]{results/COCO/figures/qualitative/loss_comp_bcos_Input.pdf}
    \end{subfigure}\hfill
    \begin{subfigure}[c]{.475\columnwidth}
    \centering
    \textbf{Final}\\\vspace{.25cm}
    \includegraphics[width=\textwidth]{results/COCO/figures/qualitative/loss_comp_bcos_Final.pdf}
    \end{subfigure}
    \caption{\textbf{\bcos \resnet}.}
    \label{fig:supp:quali_coco_1:bcos}
    \end{subfigure}
    \begin{subfigure}[c]{\textwidth}
    \centering
    % \vspace{.5cm}\textbf{\large \vanilla \resnet.}\\\vspace{.25cm}
    \begin{subfigure}[c]{.475\columnwidth}
    \includegraphics[width=\textwidth]{results/COCO/figures/qualitative/loss_comp_normal_Input.pdf}
    \end{subfigure}\hfill
    \begin{subfigure}[c]{.475\columnwidth}
    \includegraphics[width=\textwidth]{results/COCO/figures/qualitative/loss_comp_normal_Final.pdf}
    \end{subfigure}
    \caption{\textbf{\vanilla \resnet}.}
    \label{fig:supp:quali_coco_1:vanilla}
    \end{subfigure}
    \begin{subfigure}[c]{\textwidth}
    \centering
    % \vspace{.5cm}\textbf{\large \xdnn \resnet.}\\\vspace{.25cm}
    \begin{subfigure}[c]{.475\columnwidth}
    \includegraphics[width=\textwidth]{results/COCO/figures/qualitative/loss_comp_xdnn_Input.pdf}
    \end{subfigure}\hfill
    \begin{subfigure}[c]{.475\columnwidth}
    \includegraphics[width=\textwidth]{results/COCO/figures/qualitative/loss_comp_xdnn_Final.pdf}
    \end{subfigure}
    \caption{\textbf{\xdnn \resnet}.}
    \label{fig:supp:quali_coco_1:xdnn}
    \end{subfigure}
    \caption{Qualitative examples from the \textbf{\cocos dataset}. In particular, this figure allows to compare between models (\textbf{major rows}, \ie (a), (b), and (c)) losses (\textbf{major columns}) and layers (\textbf{left+right}) for multiple images (\textbf{minor rows}).}
    \label{fig:supp:quali_coco_1}
\end{figure}


\begin{figure}
    \centering
    \begin{subfigure}[c]{\textwidth}
    \centering
    \textbf{\large Additional qualitative examples.}\\\vspace{.25cm}
    \begin{subfigure}[c]{.485\columnwidth}
    \centering
    \textbf{\voc}\\\vspace{.25cm}
    \includegraphics[width=\textwidth]{results/VOC/figures/qualitative/loss_comp_bcos_Input_full_page.pdf}
    \end{subfigure}\hfill
    \begin{subfigure}[c]{.485\columnwidth}
    \centering
    \textbf{\coco}\\\vspace{.25cm}
    \includegraphics[width=\textwidth]{results/COCO/figures/qualitative/loss_comp_bcos_Input_full_page.pdf}
    \end{subfigure}
    \end{subfigure}
    \caption{Qualitative examples from the \textbf{\vocs (left) and \cocos (right)} datasets. In particular, here we just show additional examples for the \bcos models with input attributions, as this configuration exhibits the most detail. We show results for such models trained with different losses (\textbf{columns}) for multiple images (\textbf{rows}).}
    \label{fig:supp:quali_combined}
\end{figure}



In \cref{fig:supp:quali_voc_1,fig:supp:quali_coco_1}, we visualize attributions across losses, attribution methods, and layers for the same set of examples from the \vocs and \cocos datasets respectively. As discussed in the main paper, we make the following observations.

First, when guiding models at the \emph{final layer}, we observe a marked improvement in the granularity of the attribution maps for all losses (\finding5), except for \ppceloss, for which we do not observe notable differences. The improvements are particularly noticeable on the \cocos dataset (\cref{fig:supp:quali_coco_1}, ``Final'' column), in which the objects tend to be smaller. \Eg, when looking at the airplane image (last row per model), we observe much fewer attributions in the background after applying model guidance.

Second, as the \lone loss optimizes for uniform coverage \emph{within} the bounding boxes, it provides coarser attributions that tend to fill the entire bounding box (cf.~\finding3). This can be observed particularly well for the large objects from the \vocs dataset: \eg, whereas models trained with the \epgloss and the \rrr loss highlight just a relatively small area within the bounding box of the cat (\cref{fig:supp:quali_voc_1}, "Final" column, third row), the \lone loss yields  much more distributed attributions for all models.

Third, at the input layer, the \bcos models show the most notable qualitative improvements (cf.~\finding4). In particular, although the \xdnn models show some reduction in noisy background attributions (\eg last rows in \cref{fig:supp:quali_voc_1:xdnn} and \cref{fig:supp:quali_coco_1:xdnn}), the attributions remain rather noisy for many of the images; for the \vanilla models, the improvements are even less pronounced (\cref{fig:supp:quali_voc_1:vanilla}, \cref{fig:supp:quali_coco_1:vanilla}). The \bcos models, on the other hand, seem to lend themselves better to such guidance being applied to the attributions at the input layer (\cref{fig:supp:quali_voc_1:bcos}, \cref{fig:supp:quali_coco_1:bcos}) and the resulting attributions show much more detail (\epgloss + \rrr) or an increased focus on the entire bounding box (\lone). Especially with the \epgloss, the \bcos models are able to clearly focus on even small objects, see \cref{fig:supp:quali_coco_1:bcos}.

For additional results from both the \vocs as well as the \cocos dataset, please see \cref{fig:supp:quali_combined}.


\clearpage
\subsection{Additional visualizations for training with coarse bounding boxes}
\label{supp:sec:main:qualitative:dilation}
In this section, we show more detailed and additional examples of models trained with coarser bounding boxes, \ie with bounding boxes that are purposefully dilated during training by various amounts (10\%, 25\%, or 50\%), see \cref{fig:supp:dilation_quali}. In accordance with our findings in the main paper (cf.~\finding8), we observe that the \epgloss loss is highly robust to such `annotation errors': the attribution maps improve noticeably in all cases (compare the \epgloss row with the respective baseline result). In contrast, the \lone loss seems more dependent on high-quality annotations, which we also observe quantitatively, see \cref{fig:supp:dilation_quanti}.
\begin{figure}[h]
    \centering
    \begin{subfigure}[c]{.475\textwidth}
    \includegraphics[width=\textwidth]{results/VOC/figures/qualitative/dilation_comp_17_0.pdf}
    \end{subfigure}\hfill
    \begin{subfigure}[c]{.475\textwidth}
    \includegraphics[width=\textwidth]{results/VOC/figures/qualitative/dilation_comp_18_0.pdf}
    \end{subfigure}\hfill
    \begin{subfigure}[c]{.475\textwidth}
    \includegraphics[width=\textwidth]{results/VOC/figures/qualitative/dilation_comp_69_0.pdf}
    \end{subfigure}\hfill
    \begin{subfigure}[c]{.475\textwidth}
    \includegraphics[width=\textwidth]{results/VOC/figures/qualitative/dilation_comp_128_0.pdf}
    \end{subfigure}\hfill
    \begin{subfigure}[c]{.475\textwidth}
    \includegraphics[width=\textwidth]{results/VOC/figures/qualitative/dilation_comp_194_0.pdf}
    \end{subfigure}\hfill
    \begin{subfigure}[c]{.475\textwidth}
    \includegraphics[width=\textwidth]{results/VOC/figures/qualitative/dilation_comp_95_0.pdf}
    \end{subfigure}\hfill
    \caption{\textbf{Qualitative examples of the impact of using coarse bounding boxes for guidance.} We show examples of \bcos attributions from the input layer on the baseline model and on models guided with the \energyloss and \loneloss localization losses with varying degrees of dilation $\{10\%,25\%,50\%\}$ in bounding boxes during training. For each example (\textbf{block} in the figure), we show the image and bounding boxes with varying degrees of dilation (\textbf{top} row), attributions with the \loneloss localization loss (\textbf{middle} row), and attributions with the \energyloss localization loss (\textbf{bottom} row). We find that in contrast to using the \loneloss localization  loss, guidance with \energyloss localization loss maintains localization of attributions to on-object features even with dilated bounding boxes. Note that bounding boxes are dilated only during training, not during evaluation. Bounding boxes in \textbf{light blue} show the extent of dilation that \textit{would have been used} had the image been from the training set, while those in \textbf{dark blue} show undilated bounding boxes that are used during evaluation.}
    \label{fig:supp:dilation_quali}
\end{figure}
\clearpage

\section{Additional Quantitative Results (\vocs and \cocos)}
\label{supp:sec:quanti}


In this section, we provide additional quantitative results from our experiments on the \vocs and \cocos datasets. Specifically, in \cref{supp:sec:quantitative:classvsloc}, we show additional results comparing classification and localization performance. In \cref{supp:sec:quantitative:gradcam} we present results for guiding models via \gradcam attributions. In \cref{supp:sec:quantitative:intermediate}, we show that training at intermediate layers can be a cost-effective way approach to performing model guidance.
% , leading to gains in \epg even at the input layer (\cref{supp:sec:quantitative:tdes}).
In \cref{supp:sec:quantitative:segmentepg}, we evaluate how well the attributions localize to on-object features (as opposed to background features) within the bounding boxes, and find that the \energyloss outperforms other localization losses in this regard. In \cref{supp:sec:quantitative:limited}, we provide additional analyses regarding training with a limited number of annotated images. Finally, in \cref{supp:sec:quantitative:dilation}, we provide additional analyses regarding the usage of coarse, dilated bounding boxes during training.

\subsection{Comparing Classification and Localization Performance}
\label{supp:sec:quantitative:classvsloc}
In this section, we discuss additional quantitative findings with respect to localization and classification performance metrics (\iou, \map) for a selected subset of the experiments; for a full comparison of all layers and metrics, please see \cref{fig:supp:voc:f1_results,,fig:supp:coco:f1_results,,fig:supp:voc:map_results:2,,fig:supp:coco:map_results}. 

\myparagraph{Additional \iou results.} In \cref{fig:sub:iou:voc,fig:sub:iou:coco}, we show the remaining results comparing \iou vs.~F1 scores that were not shown in the main paper for \vocs and \cocos respectively.
Similar to the results in the main paper for the \epg metric (\cref{fig:epg_results}), we find that the results between datasets are highly consistent for the \iou metric.

In particular, as discussed in \cref{sec:results:epg+iou}, we find that the \lone loss yields the largest improvements in \iou when optimized at the final layer, see bottom rows of \cref{fig:sub:iou:voc,,fig:sub:iou:coco}. At the input layer, we find that \vanilla and \xdnn \resnet models are not improving their \iou scores noticeably, whereas the \bcos models show significant improvements. We attribute this to the noisy patterns in the attribution maps of \vanilla and \xdnn models, which might be difficult to optimize.

\begin{figure}[h]
    \centering
    \textbf{\iou results} on {\vocs}.\vspace{.25cm}\\
    \begin{subfigure}[c]{\textwidth}
    \includegraphics[width=\textwidth]{results/VOC/figures/iou/all_results_f1.pdf}
    \end{subfigure}
    \caption{\textbf{\iou results on \voc.} We show \iou vs.~\fone for all localization loss functions, attribution methods, and layers. In contrast to the consistent improvements observed at the final layer with the \lone loss, the \iou metric only noticeably improves for the \bcos models after model guidance. We attribute this to the high amount of noise present in the attribution maps of \vanilla and \xdnn models, see \eg \cref{fig:supp:quali_voc_1,,fig:supp:quali_coco_1}. For results on the \cocos dataset, please see \cref{fig:sub:iou:coco}.}
    \label{fig:sub:iou:voc}
\end{figure}

\begin{figure}[h]
    \centering
    {\textbf{\iou results} on \cocos}.\vspace{.25cm}\\
    \begin{subfigure}[c]{\textwidth}
    \includegraphics[width=\textwidth]{results/COCO/figures/iou/all_results_f1.pdf}
    \end{subfigure}
    \caption{\textbf{\iou results on \coco.} We show \iou vs.~\fone for all localization loss functions, attribution methods, and layers. In contrast to the consistent improvements observed at the final layer with the \lone loss, the \iou metric only noticeably improves for the \bcos models after model guidance. We attribute this to the high amount of noise present in the attribution maps of \vanilla and \xdnn models, see \eg \cref{fig:supp:quali_voc_1,,fig:supp:quali_coco_1}. For results on the \vocs dataset, please see \cref{fig:sub:iou:voc}.}
    \label{fig:sub:iou:coco}
\end{figure}

\myparagraph{Using \map to evaluate classification performance.} In all results so far, we plotted the localization metrics (\epg, \iou) versus the \fone score as a measure of classification performance. In order to highlight that the observed trends are independent of this particular choice of metric, in \cref{fig:supp:voc:map_results:1}, we show both \epg as well as \iou results plotted against the \map score. 

In general, we find the results obtained for the \map metric to be highly consistent with the previously shown results for the \fone metric. \Eg, across all configurations, we find the \epgloss to yield the highest gains in \epg score, whereas the \lone loss provides the best trade-offs with respect to the \iou metric. In order to easily compare between all results for all datasets and metrics, please see \cref{fig:supp:voc:f1_results,,fig:supp:coco:f1_results,,fig:supp:voc:map_results:2,,fig:supp:coco:map_results}.

\begin{figure}
    \centering
    \vspace{.5cm}
    \textbf{Mean Average Precision (\map) results} on \vocs.\vspace{.25cm}\\
    \begin{subfigure}[c]{.9\textwidth}
    \includegraphics[width=\textwidth]{results/VOC/figures/loc/all_results_map.pdf}
    \caption{\textbf{\epg vs.~\map.}}
    \end{subfigure}
    \begin{subfigure}[c]{.9\textwidth}
    \includegraphics[width=\textwidth]{results/VOC/figures/iou/all_results_map.pdf}
    \caption{\textbf{\iou vs.~\map.}}
    \end{subfigure}
    \caption{\textbf{Quantitative comparison of \epg and \iou  vs.~\map scores for \vocs.} To ensure that the trends observed and described in the main paper generalize beyond the \fone metric, in this figure we show the \epg and \iou scores plotted against the \map metric. In general, we find the results obtained for the \map metric to be highly consistent with the previously shown results for the \fone metric, see \eg \cref{fig:epg_results,,fig:iou_results}. \Eg, across all configurations, we find the \epgloss to yield the highest gains in \epg score, whereas the \lone loss provides the best trade-offs with respect to the \iou metric. To compare between all results for all datasets and metrics, please see \cref{fig:supp:voc:f1_results,,fig:supp:coco:f1_results,,fig:supp:voc:map_results:1,,fig:supp:coco:map_results}.}
    \label{fig:supp:voc:map_results:1}
\end{figure}



\subsection{Model Guidance via \gradcam}
\label{supp:sec:quantitative:gradcam}
\begin{figure}[h]
    \centering
    {\textbf{Comparison to \gradcam} on {\vocs}.}\vspace{.25cm}\\
    \begin{subfigure}[c]{\textwidth}
    \includegraphics[width=\textwidth]{results/VOC/figures/loc/gradcam_f1.pdf}
    \end{subfigure}
    \caption{\textbf{Quantitative results using \gradcam.} We show \epg scores vs.~F1 scores for all localization losses and models using \gradcam at the final layer (\textbf{bottom row}) and compare it to the results shown in the main paper (\textbf{top row}). 
    As expected, \gradcam performs very similarly to \ixg (\vanilla) and \intgrad (\xdnn) used at the final layer---in particular, note that for \resnet architectures, \ixg and \intgrad are very similar to \gradcam for \vanilla and \xdnn models respectively (see \cref{supp:sec:quantitative:gradcam}). Similarly, we find \gradcam to also perform comparably to the \bcos explanations when used at the final layer; for \iou results and results on \cocos, see \cref{fig:supp:gradcam:voc,,fig:supp:gradcam:coco}.
    }
    \label{fig:supp:gradcam_epg_voc}
\end{figure}

In \cref{fig:supp:gradcam_epg_voc}, we show the \epg vs.~\fone results of training models with \gradcam applied at the final layer on the \vocs dataset; for \iou results and results on \cocos, please see \cref{fig:supp:gradcam:voc,fig:supp:gradcam:coco}. When comparing between rows (\textbf{top:} main paper results; \textbf{bottom:} \gradcam), it becomes clear that \gradcam performs very similarly to \ixg\ / \intgrad\  / \bcos attributions on \vanilla\ / \xdnn\ / \bcos models. In fact, note that \gradcam is very similar to \ixg and \intgrad (equivalent up to an additional zero-clamping) for the respective models and any differences in the results can be attributed to the non-deterministic training pipeline and the similarity between the results should thus be expected. 


\subsection{Model Guidance at Intermediate Layers}
\label{supp:sec:quantitative:intermediate}


In \cref{sec:results}, we show results for guidance on two `model depths', \ie at the input and the final layer. This corresponds to the two depths at which attributions are typically computed, \eg \ixg and \intgrad are typically computed at the input, while \gradcam is typically computed using final spatial layer activations. Following \citeApp{rao2022towards}, for a fair comparison we optimize using each attribution methods at identical depths. For the final and intermediate layers in the network, this is done by treating the output activations at that layer as effective inputs over which attributions are to be computed. As done with \gradcam \citeApp{selvaraju2017grad}, we then upscale the attribution maps to image dimensions using bilinear interpolation and then use them for model guidance.

In \cref{fig:sub:intermediate:epg}, we show results for performing model guidance at additional intermediate layers: Mid1, Mid2, and Mid3. Specifically, for the \resnet models we use, these layers correspond to the outputs of \verb|conv2_x|, \verb|conv3_x|, and \verb|conv4_x| respectively in the ResNet nomenclature (\citeApp{he2016deep}), while the final layer corresponds to the output of \verb|conv5_x|. We find that the \epg performance at these intermediate layers through the network follows the trends when moving from the input to the final layer. Similar results for \iou can be found in \cref{fig:intermediate:iou}.

\begin{figure}[h]
    \centering    
    {\epg results for \textbf{intermediate layers} on \vocs.}\vspace{.25cm}\\
    \begin{subfigure}[c]{\textwidth}
    \includegraphics[width=\textwidth]{results/VOC/figures/loc/intermediate_layer_results_f1.pdf}
    \end{subfigure}
    \caption{\textbf{Intermediate layer results comparing \epg vs.~F1.} We compare the effectiveness of model guidance at varying network depths (\textbf{rows}) for each attribution method and model (\textbf{columns}) across localization loss functions. For the \bcos model, we find similar trends at all network depths, with the \energyloss localization loss outperforming all other losses. For the \vanilla and \xdnn models, the \energyloss loss similarly performs the best, but we also observe improved performance across losses when optimizing at deeper layers of the network. Full results can be found in \cref{fig:intermediate:epg,fig:intermediate:iou}.}
    \label{fig:sub:intermediate:epg}
\end{figure}

\subsection{Evaluating On-Object Localization}
\label{supp:sec:quantitative:segmentepg}



\begin{figure}
    \centering    
    {Evaluating \textbf{on-object localization} within bounding boxes.}\vspace{.25cm}\\
    \begin{subfigure}[c]{\textwidth}
    \centering
    \includegraphics[width=.55\textwidth]{results/figures/qualitative/segmentation_schema.pdf}
    \caption{\textbf{Evaluating \emph{on-object} localization within the bounding boxes: On-object \epg.} In the standard \epg metric (\textbf{middle} column), we compute the fraction of positive attributions within the bounding boxes. In other words, attributions within the bounding box (\textbf{green} region) positively impact the metric, while attributions outside (\textbf{blue} region) negatively impact it. Since bounding boxes are coarse annotations and often include background regions, the standard \epg does not evaluate how well attributions localize \textit{on-object} features, \eg the person in the figure. To measure this, we evaluate with an additional Segmentation \epg metric (\textbf{right} column), where we compute the fraction of positive attributions in the bounding box that lie within the segmentation mask of the object. Here, attributions within the segmentation mask (\textbf{green} region) positively impact the metric, and attributions outside the segmentation mask and inside the bounding box (\textbf{blue} region) negatively impact it. Note that attributions outside the bounding box have no effect on Segmentation \epg. As an example and to visualize qualitative differences between losses, in the bottom rows (\lone, \epgloss), we show attributions for a \bcos model guided at the input layer. As becomes clear, by employing a uniform prior on attributions within the bounding box, the \lone loss is effectively  optimized to fill the entire bounding box and thus to not only highlight \emph{on-object} features. This can also be observed quantitatively, see \eg \cref{fig:seg_epg:epg_vs_f1}, right column.}
    \label{fig:seg_epg:schema}
    \end{subfigure}
    \begin{subfigure}[c]{\textwidth}
    \includegraphics[width=\textwidth]{results/VOCSegment/figures/loc/all_results_f1.pdf}
    \caption{\textbf{On-object \epg results.} We evaluate across models (\textbf{columns}) and layers (\textbf{rows}) for the \energyloss and \loneloss localization losses. As seen qualitatively (\eg \cref{fig:loss_comp}), we find that the \energyloss loss is more effective than the \loneloss loss in localizing attributions to the object as opposed to background regions within the bounding boxes. This is explained by the fact that the \loneloss loss promotes uniformity in attributions within the bounding box, and treats both on-object and background features inside the box with equal importance, while the \energyloss loss only optimizes for attributions to lie within the bounding box without placing any constraints on where they may lie, leaving the model free to decide which regions within the box are important for its decision.}
    \label{fig:seg_epg:epg_vs_f1}
    \end{subfigure}
    \caption{\textbf{Evaluating \emph{on-object} localization via \epg.} We show \textbf{(a)} the schema for the on-object \epg metric and how it differs the standard bounding box \epg metric, and \textbf{(b)} quantitative results on evaluating with on-object \epg.}
    \label{fig:seg_epg}
\end{figure}


The standard \epg metric (\cref{eq:epg}) evaluates the extent to which attributions localize to the bounding boxes. However, since such boxes often include background regions, the \epg score does not distinguish between attributions that focus on the object and attributions that focus on such background regions within the bounding boxes. 

To additionally evaluate for on-object localization, we use a variant of \epg that we call On-object \epg. In contrast to standard \epg, we compute the fraction of positive attributions in pixels contained within the segmentation mask of the object out of positive attributions within the bounding box. This measures how well attributions \textit{within the bounding boxes} localize to the object, and is not influenced by attributions outside the bounding boxes. A visual comparison of the two metrics is shown in \cref{fig:seg_epg}.

We find that the \energyloss localization loss outperforms the \loneloss localization loss both qualitatively (\cref{fig:seg_epg:schema}) and quantitatively (\cref{fig:seg_epg:epg_vs_f1}) on this metric. This is explained by the fact that the \loneloss promotes uniformity in attributions across the bounding box, giving equal importance to on-object and background features within the box. In contrast, the \energyloss loss only optimizes for attributions to lie within the box, without any constraint on \textit{where} in the box they lie. This also corroborates our previous qualitative observations (\eg \cref{fig:loss_comp}).


\subsection{Model Guidance with Limited Annotations}
\label{supp:sec:quantitative:limited}
\begin{figure}
    \centering
    {Additional results for training with \textbf{limited annotations}}\vspace{.5cm}\\
    \begin{subfigure}[c]{\textwidth}
    \begin{subfigure}[c]{.49\textwidth}
    \centering
    \textbf{\epg score}\\\vspace{.2cm}
    \includegraphics[width=\textwidth]{results/VOC/figures/loc/input/limited_annotation_results_bcos.pdf}
    \end{subfigure}\hfill
    \begin{subfigure}[c]{.49\textwidth}
    \centering
    \textbf{\iou score}\\\vspace{.2cm}
    \includegraphics[width=\textwidth]{results/VOC/figures/iou/input/limited_annotation_results_bcos.pdf}
    \end{subfigure}
    \end{subfigure}
    \caption{\textbf{\epg and \iou scores for limited annotations.} We show \epg vs.~F1 (\textbf{left}) and \iou vs.~F1 (\textbf{right}) for \bcos attributions at the input when optimizing with the \energyloss and \loneloss localization losses, when using $\{1\%,10\%,100\%\}$ training annotations. We find that model guidance is generally effective even when training with annotations for a limited number of images. While the performance slightly worsens when using 1\% annotations, using just 10\% annotated images yields similar gains to using a fully annotated training set. Full results can be found in \cref{fig:supp:limited:full:input,fig:supp:limited:full:final}.}
    \label{fig:supp:limited:sub}
\end{figure}

In \cref{fig:supp:limited:sub}, we show the impact of using limited annotations when training (\cref{sec:results:ablations}) when optimizing with the \energyloss and \loneloss localization losses for \bcos attributions at the input. We find that in addition to \epg, trends in \iou scores also remain consistent even when using bounding boxes for just 1\% of the the training images.

    
\subsection{Model Guidance with Noisy Annotations} 
\label{supp:sec:quantitative:dilation}
\begin{figure}[h]
    \centering
    {Additional results for training with \textbf{coarse bounding boxes}}\vspace{.5cm}\\
    \begin{subfigure}[c]{.495\textwidth}
    \centering
    \includegraphics[width=\textwidth]{results/VOC/figures/final/coarse_bbox_results_normal.pdf}
    \caption{\textbf{\vanilla \resnet  @ Final.}}
    \end{subfigure}\hfill
    \begin{subfigure}[c]{.495\textwidth}
    \centering
    \includegraphics[width=\textwidth]{results/VOC/figures/final/coarse_bbox_results_xdnn.pdf}
    \caption{\textbf{\xdnn \resnet  @ Final.}}
    \end{subfigure}
    \caption{\textbf{Coarse bounding box results.} We show the impact of dilating bounding boxes during training for the \textbf{(a)} \vanilla and \textbf{(b)} \xdnn models. Similar to the results seen with \bcos models (\cref{fig:coarse_annotations}), we find that the \energyloss localization loss is generally robust to coarse annotations, while the effectiveness of guidance with the \loneloss localization loss worsens as the extent of coarseness (dilations) increases. Full results in \cref{fig:supp:dilation_quanti:full}.}
    \label{fig:supp:dilation_quanti}
\end{figure}


In \cref{fig:supp:dilation_quanti}, we additionally show the impact of training with coarse, dilated bounding boxes for \ixg attributions on the \vanilla model, and \intgrad attributions on the \xdnn model. Similar to the results seen with \bcos attributions (\cref{fig:coarse_annotations}), we find that the \energyloss localization loss is robust to coarse annotations, while the performance with \loneloss localization loss worsens as the dilations increase.








\clearpage
\clearpage
\section{Waterbirds Results}
\label{supp:sec:waterbirds}

\definecolor{lightgrey}{rgb}{.925, .925, .925}
\definecolor{grey}{rgb}{.4, .4, .4}

\setlength\tabcolsep{.125em}
\begin{table}[h!]
\centering
\begin{tabular}{=c+c+c@{\hskip3pt}@{\hskip3pt}+c+c+c+c+c@{\hskip3pt}|@{\hskip3pt}+c+c+c+c+c}
% \toprule
&&&
&\multicolumn{3}{c}{\footnotesize \textbf{Conventional Setting}}&&
&\multicolumn{3}{c}{\footnotesize \textbf{Reversed Setting}}&\\[.25em]
& \footnotesize \bf Layer & \footnotesize \bf Loss      &                             \footnotesize{\bf G1 Acc} &                             \footnotesize{\bf G2 Acc} &                             \footnotesize{\bf G3 Acc} &                             \footnotesize{\bf G4 Acc} &                            \footnotesize{\bf Overall} &                             \footnotesize{\bf G1 Acc} &                             \footnotesize{\bf G2 Acc} &                             \footnotesize{\bf G3 Acc} &                             \footnotesize{\bf G4 Acc} &                            \footnotesize{\bf Overall} \\[.5em]\hline
% \midrule



\multirow{4}{*}{\footnotesize \phantom{\scriptsize0}\rotatebox[origin=c]{90}{\textbf{\bcos}}}\phantom{\scriptsize0} & \phantom{\scriptsize 0}\multirow{2}{*}{\footnotesize Input}\phantom{\scriptsize 0} &\footnotesize Energy &            \footnotesize{ 99.2 \scriptsize($\pm$0.1) } &           \footnotesize{ 40.4 \scriptsize($\pm$1.0) } &  \textbf{\footnotesize{ 56.1 \scriptsize($\pm$4.0) }} &           \footnotesize{ 96.6 \scriptsize($\pm$0.4) } &  \textbf{\footnotesize{ 71.2 \scriptsize($\pm$0.1) }} &  \textbf{\footnotesize{ 99.4 \scriptsize($\pm$0.1) }} &  \textbf{\footnotesize{ 70.2 \scriptsize($\pm$2.1) }} &  \textbf{\footnotesize{ 62.8 \scriptsize($\pm$2.1) }} &           \footnotesize{ 96.5 \scriptsize($\pm$0.6) } &  \textbf{\footnotesize{ 83.6 \scriptsize($\pm$1.1) }} \\
&       & \footnotesize{L1} &           \footnotesize{ 99.3 \scriptsize($\pm$0.1) } &           \footnotesize{ 37.0 \scriptsize($\pm$0.8) } &           \footnotesize{ 51.1 \scriptsize($\pm$1.9) } &  \textbf{\footnotesize{ 97.2 \scriptsize($\pm$0.3) }} &           \footnotesize{ 69.5 \scriptsize($\pm$0.2) } &           \footnotesize{ 99.3 \scriptsize($\pm$0.3) } &           \footnotesize{ 67.7 \scriptsize($\pm$3.3) } &           \footnotesize{ 58.8 \scriptsize($\pm$5.0) } &  \textbf{\footnotesize{ 96.7 \scriptsize($\pm$0.7) }} &           \footnotesize{ 82.2 \scriptsize($\pm$0.9) } \\
% \cline{2-13}
       & \phantom{\scriptsize 0}\multirow{2}{*}{\footnotesize Final}\phantom{\scriptsize 0} &\footnotesize Energy &            \footnotesize{ 99.3 \scriptsize($\pm$0.1) } &  \textbf{\footnotesize{ 41.0 \scriptsize($\pm$2.1) }} &           \footnotesize{ 53.1 \scriptsize($\pm$0.8) } &           \footnotesize{ 96.3 \scriptsize($\pm$0.5) } &           \footnotesize{ 71.1 \scriptsize($\pm$0.9) } &  \textbf{\footnotesize{ 99.4 \scriptsize($\pm$0.2) }} &           \footnotesize{ 70.1 \scriptsize($\pm$3.1) } &           \footnotesize{ 60.2 \scriptsize($\pm$3.9) } &           \footnotesize{ 95.8 \scriptsize($\pm$1.1) } &           \footnotesize{ 83.2 \scriptsize($\pm$1.1) } \\
       &       & \footnotesize{L1} &           \footnotesize{ 99.3 \scriptsize($\pm$0.1) } &           \footnotesize{ 34.3 \scriptsize($\pm$3.2) } &           \footnotesize{ 49.4 \scriptsize($\pm$2.6) } &           \footnotesize{ 96.6 \scriptsize($\pm$0.6) } &           \footnotesize{ 68.2 \scriptsize($\pm$1.1) } &  \textbf{\footnotesize{ 99.4 \scriptsize($\pm$0.1) }} &           \footnotesize{ 69.8 \scriptsize($\pm$2.1) } &           \footnotesize{ 56.3 \scriptsize($\pm$1.8) } &           \footnotesize{ 96.1 \scriptsize($\pm$0.7) } &           \footnotesize{ 82.8 \scriptsize($\pm$0.8) } \\
% \cline{2-13}
       \rowcolor{lightgrey}\rowstyle{\color{gray}}&\multicolumn{2}{c}{\footnotesize \color{grey} Baseline} &  \textbf{\footnotesize{ 99.4 \scriptsize($\pm$0.1) }} &           \footnotesize{ 37.2 \scriptsize($\pm$0.2) } &           \footnotesize{ 43.4 \scriptsize($\pm$2.4) } &           \footnotesize{ 96.5 \scriptsize($\pm$0.1) } &           \footnotesize{ 68.7 \scriptsize($\pm$0.2) } &  \textbf{\footnotesize{ 99.4 \scriptsize($\pm$0.1) }} &           \footnotesize{ 62.8 \scriptsize($\pm$0.2) } &           \footnotesize{ 56.6 \scriptsize($\pm$2.4) } &           \footnotesize{ 96.5 \scriptsize($\pm$0.1) } &           \footnotesize{ 80.1 \scriptsize($\pm$0.2) } \\
%\cline{1-13}
\hline



\multirow{4}{*}{\footnotesize \phantom{\scriptsize0}\rotatebox[origin=c]{90}{\textbf{\xdnn}}}\phantom{\scriptsize0} & \phantom{\scriptsize 0}\multirow{2}{*}{\footnotesize Input}\phantom{\scriptsize 0} &\footnotesize Energy &           \footnotesize{ 99.3 \scriptsize($\pm$0.2) } &  \textbf{\footnotesize{ 47.0 \scriptsize($\pm$9.1) }} &           \footnotesize{ 49.2 \scriptsize($\pm$4.8) } &  \textbf{\footnotesize{ 96.8 \scriptsize($\pm$0.7) }} &  \textbf{\footnotesize{ 73.1 \scriptsize($\pm$3.4) }} &           \footnotesize{ 99.0 \scriptsize($\pm$0.3) } &  \textbf{\footnotesize{ 67.6 \scriptsize($\pm$4.8) }} &  \textbf{\footnotesize{ 63.9 \scriptsize($\pm$3.6) }} &           \footnotesize{ 96.1 \scriptsize($\pm$0.7) } &  \textbf{\footnotesize{ 82.6 \scriptsize($\pm$2.0) }} \\
&       & \footnotesize{L1} &           \footnotesize{ 99.1 \scriptsize($\pm$0.6) } &           \footnotesize{ 40.4 \scriptsize($\pm$7.3) } &           \footnotesize{ 41.8 \scriptsize($\pm$3.8) } &           \footnotesize{ 96.5 \scriptsize($\pm$0.6) } &           \footnotesize{ 69.6 \scriptsize($\pm$3.2) } &  \textbf{\footnotesize{ 99.3 \scriptsize($\pm$0.2) }} &           \footnotesize{ 59.1 \scriptsize($\pm$4.7) } &           \footnotesize{ 63.6 \scriptsize($\pm$6.1) } &           \footnotesize{ 96.0 \scriptsize($\pm$0.9) } &           \footnotesize{ 79.3 \scriptsize($\pm$1.3) } \\
% \cline{2-13}
       & \phantom{\scriptsize 0}\multirow{2}{*}{\footnotesize Final}\phantom{\scriptsize 0} &\footnotesize Energy &           \footnotesize{ 99.2 \scriptsize($\pm$0.4) } &          \footnotesize{ 42.5 \scriptsize($\pm$10.4) } &  \textbf{\footnotesize{ 54.2 \scriptsize($\pm$3.2) }} &           \footnotesize{ 96.6 \scriptsize($\pm$0.9) } &           \footnotesize{ 71.9 \scriptsize($\pm$4.2) } &           \footnotesize{ 99.2 \scriptsize($\pm$0.2) } &           \footnotesize{ 65.3 \scriptsize($\pm$2.0) } &           \footnotesize{ 62.3 \scriptsize($\pm$3.3) } &           \footnotesize{ 96.0 \scriptsize($\pm$0.5) } &           \footnotesize{ 81.5 \scriptsize($\pm$0.9) } \\
       &       & \footnotesize{L1} &  \textbf{\footnotesize{ 99.4 \scriptsize($\pm$0.1) }} &           \footnotesize{ 45.1 \scriptsize($\pm$4.0) } &           \footnotesize{ 42.8 \scriptsize($\pm$2.8) } &           \footnotesize{ 96.5 \scriptsize($\pm$0.5) } &           \footnotesize{ 71.7 \scriptsize($\pm$1.4) } &  \textbf{\footnotesize{ 99.3 \scriptsize($\pm$0.2) }} &           \footnotesize{ 62.9 \scriptsize($\pm$4.8) } &           \footnotesize{ 59.8 \scriptsize($\pm$4.8) } &           \footnotesize{ 95.8 \scriptsize($\pm$0.7) } &           \footnotesize{ 80.4 \scriptsize($\pm$1.8) } \\
% \cline{2-13}
       \rowcolor{lightgrey}\rowstyle{\color{gray}}&\multicolumn{2}{c}{\footnotesize \color{grey} Baseline} &           \footnotesize{ 99.3 \scriptsize($\pm$0.1) } &           \footnotesize{ 39.8 \scriptsize($\pm$0.7) } &           \footnotesize{ 38.6 \scriptsize($\pm$2.5) } &           \footnotesize{ 96.3 \scriptsize($\pm$0.7) } &           \footnotesize{ 69.1 \scriptsize($\pm$0.6) } &           \footnotesize{ \textbf{99.3 \scriptsize($\pm$0.1)} } &           \footnotesize{ 60.2 \scriptsize($\pm$0.7) } &           \footnotesize{ 61.4 \scriptsize($\pm$2.5) } &           \footnotesize{ \textbf{96.3 \scriptsize($\pm$0.7)} } &           \footnotesize{ 79.6 \scriptsize($\pm$0.5) } \\
%\cline{1-13}
\hline



\multirow{4}{*}{\footnotesize \phantom{\scriptsize0}\rotatebox[origin=c]{90}{\textbf{\vanilla}}}\phantom{\scriptsize0} & \phantom{\scriptsize 0}\multirow{2}{*}{\footnotesize Input}\phantom{\scriptsize 0} &\footnotesize Energy &          \footnotesize{ 99.4 \scriptsize($\pm$0.2) } &           \footnotesize{ 42.4 \scriptsize($\pm$2.6) } &           \footnotesize{ 47.9 \scriptsize($\pm$3.5) } &           \footnotesize{ 97.1 \scriptsize($\pm$0.4) } &           \footnotesize{ 71.2 \scriptsize($\pm$1.0) } &  \textbf{\footnotesize{ 99.6 \scriptsize($\pm$0.2) }} &           \footnotesize{ 50.7 \scriptsize($\pm$7.3) } &           \footnotesize{ 52.4 \scriptsize($\pm$1.7) } &           \footnotesize{ 97.2 \scriptsize($\pm$0.5) } &           \footnotesize{ 75.1 \scriptsize($\pm$2.9) } \\
&       & \footnotesize{L1} &  \textbf{\footnotesize{ 99.5 \scriptsize($\pm$0.1) }} &           \footnotesize{ 46.1 \scriptsize($\pm$4.4) } &           \footnotesize{ 51.1 \scriptsize($\pm$4.0) } &           \footnotesize{ 97.5 \scriptsize($\pm$0.1) } &           \footnotesize{ 73.1 \scriptsize($\pm$1.6) } &  \textbf{\footnotesize{ 99.6 \scriptsize($\pm$0.1) }} &           \footnotesize{ 48.0 \scriptsize($\pm$7.8) } &           \footnotesize{ 49.7 \scriptsize($\pm$3.7) } &           \footnotesize{ 96.8 \scriptsize($\pm$0.6) } &           \footnotesize{ 73.7 \scriptsize($\pm$2.7) } \\
% \cline{2-13}
       & \phantom{\scriptsize 0}\multirow{2}{*}{\footnotesize Final}\phantom{\scriptsize 0} &\footnotesize Energy & \textbf{\footnotesize{ 99.5 \scriptsize($\pm$0.0) }} &           \footnotesize{ 56.1 \scriptsize($\pm$7.0) } &  \textbf{\footnotesize{ 60.7 \scriptsize($\pm$5.5) }} &           \footnotesize{ 97.0 \scriptsize($\pm$0.5) } &  \textbf{\footnotesize{ 78.1 \scriptsize($\pm$2.6) }} &           \footnotesize{ 99.5 \scriptsize($\pm$0.1) } &           \footnotesize{ 59.4 \scriptsize($\pm$5.9) } &  \textbf{\footnotesize{ 56.5 \scriptsize($\pm$3.7) }} &           \footnotesize{ 97.2 \scriptsize($\pm$0.5) } &  \textbf{\footnotesize{ 78.9 \scriptsize($\pm$1.9) }} \\
       &       & \footnotesize{L1} &  \textbf{\footnotesize{ 99.5 \scriptsize($\pm$0.1) }} &  \textbf{\footnotesize{ 57.1 \scriptsize($\pm$2.9) }} &           \footnotesize{ 55.4 \scriptsize($\pm$2.5) } &           \footnotesize{ 96.7 \scriptsize($\pm$0.6) } &           \footnotesize{ 77.8 \scriptsize($\pm$1.0) } &           \footnotesize{ 99.5 \scriptsize($\pm$0.1) } &           \footnotesize{ 56.3 \scriptsize($\pm$6.7) } &           \footnotesize{ 51.6 \scriptsize($\pm$3.1) } &           \footnotesize{ 97.3 \scriptsize($\pm$0.6) } &           \footnotesize{ 77.1 \scriptsize($\pm$2.5) } \\
% \cline{2-13}
       \rowcolor{lightgrey}\rowstyle{\color{gray}}&\multicolumn{2}{c}{\footnotesize \color{grey} Baseline} &           \footnotesize{ 99.4 \scriptsize($\pm$0.0) } &           \footnotesize{ 39.6 \scriptsize($\pm$0.7) } &           \footnotesize{ 53.7 \scriptsize($\pm$2.1) } &  \textbf{\footnotesize{ 97.7 \scriptsize($\pm$0.0) }} &           \footnotesize{ 70.8 \scriptsize($\pm$0.0) } &           \footnotesize{ 99.4 \scriptsize($\pm$0.0) } &  \textbf{\footnotesize{ 60.4 \scriptsize($\pm$0.7) }} &           \footnotesize{ 46.3 \scriptsize($\pm$2.1) } &  \textbf{\footnotesize{ 97.7 \scriptsize($\pm$0.0) }} &           \footnotesize{ 78.1 \scriptsize($\pm$0.1) } \\
       \hline
% \bottomrule
\end{tabular}
\caption{\textbf{Classification performance on Waterbirds} after model guidance with the \lone and the \epgloss loss. We find that both losses consistently improve the models' classification performance over the baseline model (\ie a model without guidance). These improvements are particularly pronounced in the groups \emph{not seen during training}, \ie landbirds on water (``G2'') and waterbirds on land (``G3''). For qualitative visualizations of the effect of model guidance on the waterbirds dataset, see \cref{fig:supp:waterbirds_quali}.}
\label{tab:supp:waterbirds_table}
\end{table}

As discussed in section \cref{sec:results:waterbirds}, we use the \waterbirds dataset \citeApp{sagawa2019distributionally,petryk2022guiding} to evaluate the effectiveness of model guidance in a setting where strong spurious correlations are present in the training data. This dataset consists of four groups---\textit{Landbird} on \textit{Land} (\textbf{G1}), \textit{Landbird} on \textit{Water} (\textbf{G2}), \textit{Waterbird} on \textit{Land} (\textbf{G3}), and \textit{Waterbird} on \textit{Water} (\textbf{G4})---of which only groups \textbf{G1} and \textbf{G4} appear during training and the background is thus perfectly correlated with the type of bird (\eg Landbird on land).

To evaluate the effectiveness of model guidance, we train the models on two binary classification tasks: to classify the type of birds (the \emph{conventional setting}) or the background (the \emph{reversed setting}, as described in \citeApp{petryk2022guiding}) and evaluate models without guidance (baselines), as well as with guidance: specifically, for guiding the models, we evaluate different models (\vanilla, \xdnn, \bcos) with different guidance losses (\epgloss, \lone) applied at different layers (Input and Final), see \cref{tab:supp:waterbirds_table}. {For each model, we use its corresponding attribution method, \ie \ixg for \vanilla, \intgrad for \xdnn, and \bcos for \bcos.}

In \cref{tab:supp:waterbirds_table} we present the classification performance for the individual groups (\textbf{G1-G4}) as well as the average over all samples  (`Overall') across all configurations; note that the group sizes differ in the test set and the average over the individual group acccuracies thus differs from the overall accuracy. For each row, the results are averaged over 4 runs (2 random training seeds and 2 different sets of 1\% annotated samples) with the exception of the baseline results being an average over 2 runs.

In almost all cases, we find that both of the evaluated losses (\epgloss, \lone) improve the models' classification performance over the baseline. As expected, these improvements are particularly pronounced in the groups not seen during training, \ie landbirds on water (\textbf{G2}) and waterbirds on land (\textbf{G3}). 

Further, in \cref{fig:supp:waterbirds_quali}, we show attribution maps of the baseline models, as well as the guided models. As can be seen, model guidance not only improves the accuracy, but is also reflected in the attribution maps: \eg, in row 1 of \cref{fig:supp:waterbirds_quali:a}, we see that while the baseline model originally focused on the background (water) to classify the image, it is possible to guide the model to use the desired features (\ie the bird in conventional setting and the background in the reversed setting) and consequently arrive at the desired classification decision. As this guidance is `soft', we also observe cases in which the model still focused on the wrong feature and thus arrived at the wrong prediction: \eg in \cref{fig:supp:waterbirds_quali:b} row 1 (reversed setting), the \epgloss-guided model still focuses on the bird and thus incorrectly predicts `Water', similar to the \lone-guided model in row 4. 

\begin{figure}[h]
    \centering
    {\textbf{Additional qualitative results} on the \waterbirds dataset.}\vspace{.25cm}\\
    \begin{subfigure}[c]{.46\textwidth}
    \includegraphics[width=\textwidth]{results/figures/WB/bcos_Input_1.pdf}
    \caption{\textbf{Landbirds on Water.}}
    \label{fig:supp:waterbirds_quali:a}
    \end{subfigure}\hfill
    \begin{subfigure}[c]{.46\textwidth}
    \includegraphics[width=\textwidth]{results/figures/WB/bcos_Input_2.pdf}
    \caption{\textbf{Waterbirds on Land.}}
    \label{fig:supp:waterbirds_quali:b}
    \end{subfigure}
    \caption{\textbf{Qualitative results for the Waterbirds dataset.} Specifically, we show input layer attributions for \bcos models trained without guidance (`Baseline') as well as guided via the \epgloss or \lone loss. We find that model guidance can be effective both for focusing on the bird and the background. For example, in the top row of (a), the model originally focuses on the background (col.~2) and classifies the image (col.~1) as Water/Waterbird. In the conventional setting, both the \energyloss and \loneloss localization losses are effective in redirecting the focus to the bird (cols.~3-4), changing the model's prediction to Landbird with high confidence. Similarly, in the reversed setting, both localization losses direct the focus to the background (cols.~5-6), which increases the model's confidence in classifying the image as Water.}
    \label{fig:supp:waterbirds_quali}
\end{figure}

\clearpage
\section{Implementation Details}
\label{supp:sec:implementation}

\subsection{Training and Evaluation Details}
\label{supp:sec:implementation:training}

\myparagraph{Implementations:} We implement our code using PyTorch\footnote{\url{https://github.com/pytorch/pytorch}} \citeApp{paszke2019pytorch}. The \voc \citeApp{everingham2009pascal} and \coco \citeApp{lin2014microsoft} datasets and the \vanilla \resnet model were obtained from the Torchvision library\footnote{\url{https://github.com/pytorch/vision}} \citeS{paszke2019pytorch_2,torchvision2016_2}. Official implementations were used for the \bcos\footnote{\label{footnote:bcoscode}\url{https://github.com/B-cos/B-cos-v2}} \citeApp{bohle2023b} and \xdnn\footnote{\url{https://github.com/visinf/fast-axiomatic-attribution}} \citeApp{hesse2021fast} networks. Some of the utilities for data loading and evaluation were derived from NN-Explainer\footnote{\url{https://github.com/stevenstalder/NN-Explainer}} \citeApp{stalder2022wyswyc}, and for visualization from the Captum library\footnote{\url{https://github.com/pytorch/captum}} \citeApp{kokhlikyan2020captum}.

\subsubsection{Experiments with \vocs and \cocos}

\myparagraph{Training baseline models:} We train starting from models pre-trained on \imagenet \citeApp{imagenet}. We fine-tune with fixed learning rates in $\{10^{-3},10^{-4},10^{-5}\}$ using an Adam optimizer \citeApp{kingma2014adam} and select the checkpoint with the best validation F1-score. For \vocs, we train for 300 epochs, and for \cocos, we train for 60 epochs.

\myparagraph{Training guided models:} We train the models jointly optimized for classification and localization (\cref{eq:overall}) by fine-tuning the baseline models. We use a fixed learning rate of $10^{-4}$ and a batch size of $64$. For each configuration (given by a combination of attribution method, localization loss, and layer), we train using three different values of $\lambda_{\text{loc}}$, as detailed in \cref{tab:supp:lambdas}. For \vocs, we train for 50 epochs, and for \cocos, we train for 10 epochs.

\myparagraph{Selecting models to visualize:} As described in \cref{sec:experiments}, we select and evaluate on the set of Pareto-dominant models for each configuration after training. Each model on the Pareto front represents the extent of trade-off made between classification (F1) and localization (\epg) performance. In practice, the `best' model to choose would depend on the requirements of the end user. However, to evaluate the effectiveness of model guidance (\eg \cref{fig:teaser,fig:teaser2,fig:loss_comp}), we select a representative model on the front whose attributions we visualize. This is done by selecting the model with the highest \epg score with an at most 5 p.p. drop in F1-score.

\myparagraph{Efficient Optimization:} As described in \cref{sec:method:efficient}, for each image in a batch, we optimize for localization of a single class selected at random. This approximation allows us to perform model guidance efficiently and keeps the training cost tractable. However, to accurately evaluate the impact of this optimization, we evaluate the localization of all classes in the image at test time.

\myparagraph{Training with Limited Annotations:} As described in \cref{sec:results:ablations}, we show that training with a limited number of annotations can be a cost effective way of performing model guidance. In order to maintain the relative magnitude of $\mathcal{L}_{\text{loc}}$ as compared to $\mathcal{L}_{\text{class}}$ in this setting, we scale up the values of $\lambda_{\text{loc}}$ when training. The values of $\lambda_{\text{loc}}$ we use are shown in \cref{tab:supp:dilation:lambdas}.

\subsubsection{Experiments with \waterbirds}

\myparagraph{Data distributions:} The conventional binary classification task includes classifying \textit{Landbird} from \textit{Waterbird}, irrespective of their backgrounds.  We use the same splits generated and published by \citeApp{petryk2022guiding}. As discussed in \cref{supp:sec:waterbirds}, at training time there are no samples from \textbf{G2} or \textbf{G3}, making the bird type and the background perfectly correlated.  In contrast, both the validation and test sets are balanced across foregrounds and backgrounds, \ie a landbird is equally likely to occur on land or water, and vice-versa. However, as noted by \citeApp{sagawa2019distributionally}, using a validation set with the same distribution as the test set leaks information on the test distribution in the process of hyperparameter and checkpoint selection during training. Therefore, we modify the validation split to avoid such information leakage; in particular, we use a validation set with the same distribution as the training set, and only use examples of groups \textbf{G1} and \textbf{G4}. Note that \cref{tab:waterbirds} refers to \textbf{G3} as the ``Worst Group''. \\

\myparagraph{Training details:} We train starting from models pre-trained on \imagenet \citeApp{imagenet}. We fine-tune with fixed learning rate of $10^{-5}$ with $\lambda_{\text{loc}}$ of $5\times10^{-2}$ ($5\times10^{-4} \times 100$ for using 1\% of annotations) using an Adam optimizer \citeApp{kingma2014adam} . We train for 350 epochs with random cropping and horizontal flipping and select the checkpoint with the highest accuracy on the modified validation set.

\begin{table}[t]
    \def\arraystretch{1.2}
    \centering
    \begin{tabular}{c@{\hskip10pt}|@{\hskip10pt}c}
	    % \hline
	    \bf Localization Loss & \bf Values of $\lambda_{\text{loc}}$ \\
	    % \hhline{|=||=|}
     \hline
        \energyloss & \footnotesize$5\mytimes10^{-4}$, \phantom{0} $1\mytimes10^{-3}$, \phantom{0} $5\mytimes10^{-3}$ \\
        % \hline 
        \loneloss &\footnotesize $1\mytimes10^{-3}$, \phantom{0} $5\mytimes10^{-3}$, \phantom{0} $1\mytimes10^{-2}$ \\
        % \hline 
        \ppceloss &\footnotesize $1\mytimes10^{-4}$, \phantom{0} $5\mytimes10^{-4}$, \phantom{0} $1\mytimes10^{-3}$ \\
        % \hline 
        \rrrloss &\footnotesize $5\mytimes10^{-6}$, \phantom{0} $1\mytimes10^{-5}$, \phantom{0} $5\mytimes10^{-5}$  \\
        % \hline
	\end{tabular}
	\caption{\textbf{Hyperparameter $\lambda_{\text{loc}}$: Default training.} used for when training on \vocs and \cocos with each localization loss. Different values are used for different loss functions since the magnitudes of each loss varies.}
    \label{tab:supp:lambdas}
\end{table}

\begin{table}[t]
    \def\arraystretch{1.2}
    \centering
    \begin{tabular}{c@{\hskip10pt}|@{\hskip10pt}c}
	    % \hline
	    \bf Localization Loss & \bf Values of $\lambda_{\text{loc}}$ \\
	    % \hhline{|=||=|}
     \hline
        \energyloss & \footnotesize$0.05$, \phantom{0} $0.100$, \phantom{0} $0.50$ \\
        % \hline 
        \loneloss &\footnotesize $0.01$, \phantom{0} $0.100$, \phantom{0} $1.00$ \\
	\end{tabular}
	\caption{\textbf{Hyperparameter $\lambda_{\text{loc}}$: Limited annotations.} used for when training on \vocs and \cocos with \textbf{limited data} for each localization loss. Different values are used for different loss functions since the magnitudes of each loss varies. We use larger values of $\lambda_{\text{loc}}$ when training with limited annotations to maintain the relative magnitudes of the classification and localization losses during training.}
    \label{tab:supp:dilation:lambdas}
\end{table}

\subsection{Optimizing \bcos Attributions}
\label{supp:sec:implementation:bcos}

Training for optimizing the localization of attributions (\cref{eq:overall}) requires backpropagating through the attribution maps, which implies that they need to be differentiable. While \bcos attributions \citeApp{bohle2022b} as formulated are mathematically differentiable, the original implementation\footnoteref{footnote:bcoscode} \citeApp{bohle2023b} for computing them involves detaching the dynamic weights from the computational graph, which prevents them from being used for optimization. In this work, to use them for model guidance, we develop a twice-differentiable implementation of \bcos attributions.


\clearpage

\setcounter{section}{23}
\section{Full Results}
\label{supp:sec:full}
\begin{figure}[h]
    \centering
    {Full results on \textbf{\voc} (\fone score).}\vspace{.25cm}\\
    \begin{subfigure}[c]{.9\textwidth}
    \includegraphics[width=\textwidth]{results/VOC/figures/loc/all_results_f1.pdf}
    \caption{\textbf{\epg vs.~\fone.}}
    \end{subfigure}
    \begin{subfigure}[c]{.9\textwidth}
    \includegraphics[width=\textwidth]{results/VOC/figures/iou/all_results_f1.pdf}
    \caption{\textbf{\iou vs.~\fone.}}
    \end{subfigure}
    \caption{\textbf{EPG (a) and \iou (b) vs.~\fone on \vocs,} for different losses (\textbf{markers}) and models (\textbf{columns}), optimized at different layers (\textbf{rows}); additionally, we show the performance of the baseline model before fine-tuning and demarcate regions that strictly dominate (are strictly dominated by) the baseline performance in green (grey). 
    For each configuration, we show the Pareto fronts (cf.\ \cref{fig:pareto_example}) across regularization strengths $\lambda_\text{loc}$ and epochs (cf.\ \cref{sec:results} and \cref{fig:pareto_example}). 
    We find the \epgloss loss to give the best trade-off between \epg and \fone, whereas the \lone loss (especially at the final layer) provides the best trade-off between \iou and \fone. We further find these results to be consistent across datasets, see \cref{fig:supp:coco:f1_results}.}
    \label{fig:supp:voc:f1_results}
\end{figure}

\input{supplement/tex_figures/coco_f1}
\begin{figure}
    \centering
    \textbf{Mean Average Precision (\map) results} on \vocs.\vspace{.25cm}\\
    \begin{subfigure}[c]{.9\textwidth}
    \includegraphics[width=\textwidth]{results/VOC/figures/loc/all_results_map.pdf}
    \caption{\textbf{\epg vs.~\map.}}
    \end{subfigure}
    \begin{subfigure}[c]{.9\textwidth}
    \includegraphics[width=\textwidth]{results/VOC/figures/iou/all_results_map.pdf}
    \caption{\textbf{\iou vs.~\map.}}
    \end{subfigure}
    \caption{\textbf{Quantitative comparison of \epg and \iou  vs.~\map scores for \vocs.} To ensure that the trends observed and described in the main paper generalize beyond the \fone metric, in this figure we show the \epg and \iou scores plotted against the \map metric. In general, we find the results obtained for the \map metric to be highly consistent with the previously shown results for the \fone metric, see \cref{fig:supp:voc:f1_results}. \Eg, across all configurations, we find the \epgloss to yield the highest gains in \epg score, whereas the \lone loss provides the best trade-offs with respect to the \iou metric. These results are further also consistent with those observed on \cocos, see \cref{fig:supp:coco:map_results}.}
    \label{fig:supp:voc:map_results:2}
\end{figure}
\begin{figure}[h]
    \centering
    {Full results on \textbf{\coco} (\map).}\vspace{.25cm}\\
    \begin{subfigure}[c]{.9\textwidth}
    \includegraphics[width=\textwidth]{results/COCO/figures/loc/all_results_map.pdf}
    \caption{\textbf{\epg vs.~\map.}}
    \end{subfigure}
    \begin{subfigure}[c]{.9\textwidth}
    \includegraphics[width=\textwidth]{results/COCO/figures/iou/all_results_map.pdf}
    \caption{\textbf{\iou vs.~\map.}}
    \end{subfigure}
    \caption{\textbf{Quantitative comparison of \epg and \iou  vs.~\map scores for \cocos.} To ensure that the trends observed and described in the main paper generalize beyond the \fone metric, in this figure we show the \epg and \iou scores plotted against the \map metric. In general, we find the results obtained for the \map metric to be highly consistent with the previously shown results for the \fone metric, see \cref{fig:supp:coco:f1_results}. \Eg, across all configurations, we find the \epgloss to yield the highest gains in \epg score, whereas the \lone loss provides the best trade-offs with respect to the \iou metric. These results are further also consistent with those observed on \vocs, see \cref{fig:supp:voc:map_results:2}.}
    \label{fig:supp:coco:map_results}
\end{figure}
\input{supplement/tex_figures/gradcam_voc}
\begin{figure}[h]
    \centering
    {\textbf{Comparison to \gradcam} on {\cocos}.}\vspace{.25cm}\\
    \begin{subfigure}[c]{\textwidth}
    \includegraphics[width=\textwidth]{results/COCO/figures/loc/gradcam_f1.pdf}
    \caption{\textbf{\epg vs.~F1.}}
    \end{subfigure}
    \begin{subfigure}[c]{\textwidth}
    \includegraphics[width=\textwidth]{results/COCO/figures/iou/gradcam_f1.pdf}
    \caption{\textbf{\iou vs.~F1.}}
    \end{subfigure}
    \caption{\textbf{Quantitative results using \gradcam on \cocos.} We show \epg \textbf{(a)} and  \iou \textbf{(b)} scores vs.~F1 scores  for all localization losses and models using \gradcam at the final layer (\textbf{bottom rows} in (a)+(b) and compare it to the results shown in the main paper (\textbf{top rows}). 
    As expected, \gradcam performs very similarly to \ixg (\vanilla) and \intgrad (\xdnn) used at the final layer---in particular, note that for \resnet architectures, \ixg and \intgrad are very similar to \gradcam for \vanilla and \xdnn models respectively (see \cref{supp:sec:quantitative:gradcam}). Similarly, we find \gradcam to also perform comparably to the \bcos explanations when used at the final layer; for results on \vocs, see \cref{fig:supp:gradcam:voc}.}
    \label{fig:supp:gradcam:coco}
\end{figure}
\input{supplement/tex_figures/intermediate_epg}
\begin{figure}[h]
    \centering    
    {\iou results for \textbf{intermediate layers} on \vocs.}\vspace{.25cm}\\
    \begin{subfigure}[c]{\textwidth}
    \includegraphics[width=\textwidth]{results/VOC/figures/iou/intermediate_layer_results_f1.pdf}
    \end{subfigure}
    \caption{\textbf{Intermediate layer results comparing \iou vs.~F1.} We compare the effectiveness of model guidance at varying network depths (\textbf{rows}) for each attribution method and model (\textbf{columns}) across localization loss functions. We find similar trends across all configurations, with the \lone  loss outperforming all other losses. For the \vanilla and \xdnn models,  we observe improved performance across losses when optimizing at deeper layers of the network, whereas the results seem very stable for the \bcos models. For \epg results, see \cref{fig:intermediate:epg}.}
    \label{fig:intermediate:iou}
\end{figure}
\begin{figure}
    \centering
    {\large \textbf{Limited annotations --- Input layer}}\vspace{.5cm}\\
    \begin{subfigure}[c]{\textwidth}
    \begin{subfigure}[c]{.485\textwidth}
    \centering
    \textbf{\epg score}\\\vspace{.2cm}
    \includegraphics[width=\textwidth]{results/VOC/figures/loc/input/limited_annotation_results_normal.pdf}
    \end{subfigure}\hfill
    \begin{subfigure}[c]{.485\textwidth}
    \centering
    \textbf{\iou score}\\\vspace{.2cm}
    \includegraphics[width=\textwidth]{results/VOC/figures/iou/input/limited_annotation_results_normal.pdf}
    \end{subfigure}
    \caption{\textbf{\vanilla \resnet}}\vspace{.5cm}
    \end{subfigure}
    %
    %
    %
    \begin{subfigure}[c]{\textwidth}
    \begin{subfigure}[c]{.485\textwidth}
    \centering
    \textbf{\epg score}\\\vspace{.2cm}
    \includegraphics[width=\textwidth]{results/VOC/figures/loc/input/limited_annotation_results_xdnn.pdf}
    \end{subfigure}\hfill
    \begin{subfigure}[c]{.485\textwidth}
    \centering
    \textbf{\iou score}\\\vspace{.2cm}
    \includegraphics[width=\textwidth]{results/VOC/figures/iou/input/limited_annotation_results_xdnn.pdf}
    \end{subfigure}
    \caption{\textbf{\xdnn \resnet}}\vspace{.5cm}
    \end{subfigure}
    %
    %
    %
    \begin{subfigure}[c]{\textwidth}
    \begin{subfigure}[c]{.485\textwidth}
    \centering
    \textbf{\epg score}\\\vspace{.2cm}
    \includegraphics[width=\textwidth]{results/VOC/figures/loc/input/limited_annotation_results_bcos.pdf}
    \end{subfigure}\hfill
    \begin{subfigure}[c]{.485\textwidth}
    \centering
    \textbf{\iou score}\\\vspace{.2cm}
    \includegraphics[width=\textwidth]{results/VOC/figures/iou/input/limited_annotation_results_bcos.pdf}
    \end{subfigure}
    \caption{\textbf{\bcos \resnet}}
    \end{subfigure}
    \caption{\textbf{\epg and \iou scores for model guidance at the input layer with a limited number of annotations.} We show \epg vs.~F1 (\textbf{left}) and \iou vs.~F1 (\textbf{right}) for all models, optimized with the \energyloss and \loneloss localization losses, when using $\{1\%,10\%,100\%\}$ training annotations. We find that model guidance is generally effective even when training with annotations for a limited number of images. While the performance slightly worsens when using 1\% annotations, using just 10\% annotated images yields similar gains to using a fully annotated training set. Results at the final layer can be found in \cref{fig:supp:limited:full:final}.}
    \label{fig:supp:limited:full:input}
\end{figure}
\input{supplement/tex_figures/limited_annotations_final_full}
\begin{figure}
    \centering
    {Additional results for training with \textbf{coarse bounding boxes}}\vspace{.5cm}\\
    \begin{subfigure}[c]{.495\textwidth}
    \centering
    \includegraphics[width=\textwidth]{results/VOC/figures/final/coarse_bbox_results_normal.pdf}
    \caption{\textbf{\vanilla \resnet  @ Final.}}
    \end{subfigure}\hfill
    \begin{subfigure}[c]{.495\textwidth}
    \centering
    \includegraphics[width=\textwidth]{results/VOC/figures/final/coarse_bbox_results_xdnn.pdf}
    \caption{\textbf{\xdnn \resnet  @ Final.}}
    \end{subfigure}
    \begin{subfigure}[c]{.495\textwidth}
    \centering
    \includegraphics[width=\textwidth]{results/VOC/figures/input/coarse_bbox_results_bcos.pdf}
    \caption{\textbf{\bcos \resnet @ Input.}}
    \end{subfigure}
    \caption{\textbf{Coarse bounding box results}. We show the impact of dilating bounding boxes during training for the \textbf{(a)} \vanilla and \textbf{(b)} \xdnn, and \textbf{(c)} \bcos models. Similar to the results seen with \bcos models (c), we find that the \energyloss localization loss is generally robust to coarse annotations, while the effectiveness of guidance with the \loneloss localization loss worsens as the extent of coarseness increases.   
    }
    \label{fig:supp:dilation_quanti:full}
\end{figure}

\clearpage


{\small
\bibliographystyleS{ieee_fullname}
\bibliographyS{references_supp}
}

\end{document}