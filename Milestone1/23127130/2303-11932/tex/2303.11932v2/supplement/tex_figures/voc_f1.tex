\begin{figure}[h]
    \centering
    {Full results on \textbf{\voc} (\fone score).}\vspace{.25cm}\\
    \begin{subfigure}[c]{.9\textwidth}
    \includegraphics[width=\textwidth]{results/VOC/figures/loc/all_results_f1.pdf}
    \caption{\textbf{\epg vs.~\fone.}}
    \end{subfigure}
    \begin{subfigure}[c]{.9\textwidth}
    \includegraphics[width=\textwidth]{results/VOC/figures/iou/all_results_f1.pdf}
    \caption{\textbf{\iou vs.~\fone.}}
    \end{subfigure}
    \caption{\textbf{EPG (a) and \iou (b) vs.~\fone on \vocs,} for different losses (\textbf{markers}) and models (\textbf{columns}), optimized at different layers (\textbf{rows}); additionally, we show the performance of the baseline model before fine-tuning and demarcate regions that strictly dominate (are strictly dominated by) the baseline performance in green (grey). 
    For each configuration, we show the Pareto fronts (cf.\ \cref{fig:pareto_example}) across regularization strengths $\lambda_\text{loc}$ and epochs (cf.\ \cref{sec:results} and \cref{fig:pareto_example}). 
    We find the \epgloss loss to give the best trade-off between \epg and \fone, whereas the \lone loss (especially at the final layer) provides the best trade-off between \iou and \fone. We further find these results to be consistent across datasets, see \cref{fig:supp:coco:f1_results}.}
    \label{fig:supp:voc:f1_results}
\end{figure}
