\begin{figure}
    \centering
    {\large \textbf{Limited annotations --- Input layer}}\vspace{.5cm}\\
    \begin{subfigure}[c]{\textwidth}
    \begin{subfigure}[c]{.485\textwidth}
    \centering
    \textbf{\epg score}\\\vspace{.2cm}
    \includegraphics[width=\textwidth]{results/VOC/figures/loc/input/limited_annotation_results_normal.pdf}
    \end{subfigure}\hfill
    \begin{subfigure}[c]{.485\textwidth}
    \centering
    \textbf{\iou score}\\\vspace{.2cm}
    \includegraphics[width=\textwidth]{results/VOC/figures/iou/input/limited_annotation_results_normal.pdf}
    \end{subfigure}
    \caption{\textbf{\vanilla \resnet}}\vspace{.5cm}
    \end{subfigure}
    %
    %
    %
    \begin{subfigure}[c]{\textwidth}
    \begin{subfigure}[c]{.485\textwidth}
    \centering
    \textbf{\epg score}\\\vspace{.2cm}
    \includegraphics[width=\textwidth]{results/VOC/figures/loc/input/limited_annotation_results_xdnn.pdf}
    \end{subfigure}\hfill
    \begin{subfigure}[c]{.485\textwidth}
    \centering
    \textbf{\iou score}\\\vspace{.2cm}
    \includegraphics[width=\textwidth]{results/VOC/figures/iou/input/limited_annotation_results_xdnn.pdf}
    \end{subfigure}
    \caption{\textbf{\xdnn \resnet}}\vspace{.5cm}
    \end{subfigure}
    %
    %
    %
    \begin{subfigure}[c]{\textwidth}
    \begin{subfigure}[c]{.485\textwidth}
    \centering
    \textbf{\epg score}\\\vspace{.2cm}
    \includegraphics[width=\textwidth]{results/VOC/figures/loc/input/limited_annotation_results_bcos.pdf}
    \end{subfigure}\hfill
    \begin{subfigure}[c]{.485\textwidth}
    \centering
    \textbf{\iou score}\\\vspace{.2cm}
    \includegraphics[width=\textwidth]{results/VOC/figures/iou/input/limited_annotation_results_bcos.pdf}
    \end{subfigure}
    \caption{\textbf{\bcos \resnet}}
    \end{subfigure}
    \caption{\textbf{\epg and \iou scores for model guidance at the input layer with a limited number of annotations.} We show \epg vs.~F1 (\textbf{left}) and \iou vs.~F1 (\textbf{right}) for all models, optimized with the \energyloss and \loneloss localization losses, when using $\{1\%,10\%,100\%\}$ training annotations. We find that model guidance is generally effective even when training with annotations for a limited number of images. While the performance slightly worsens when using 1\% annotations, using just 10\% annotated images yields similar gains to using a fully annotated training set. Results at the final layer can be found in \cref{fig:supp:limited:full:final}.}
    \label{fig:supp:limited:full:input}
\end{figure}