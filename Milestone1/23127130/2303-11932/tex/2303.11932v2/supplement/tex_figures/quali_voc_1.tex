\begin{figure}
    \centering
    \begin{subfigure}[c]{\textwidth}
    \centering
    \textbf{\large \voc.}\\\vspace{.25cm}
    \begin{subfigure}[c]{.475\columnwidth}
    \centering
    \textbf{Input}\\\vspace{.25cm}
    \includegraphics[width=\textwidth]{results/VOC/figures/qualitative/loss_comp_bcos_Input.pdf}
    \end{subfigure}
    \begin{subfigure}[c]{.475\columnwidth}
    \centering
    \textbf{Final}\\\vspace{.25cm}
    \includegraphics[width=\textwidth]{results/VOC/figures/qualitative/loss_comp_bcos_Final.pdf}
    \end{subfigure}
    \caption{\textbf{\bcos \resnet}.}
    \label{fig:supp:quali_voc_1:bcos}
    \end{subfigure}
    \begin{subfigure}[c]{\textwidth}
    \centering
    \begin{subfigure}[c]{.475\columnwidth}
    \includegraphics[width=\textwidth]{results/VOC/figures/qualitative/loss_comp_normal_Input.pdf}
    \end{subfigure}
    \begin{subfigure}[c]{.475\columnwidth}
    \includegraphics[width=\textwidth]{results/VOC/figures/qualitative/loss_comp_normal_Final.pdf}
    \end{subfigure}
    \caption{\textbf{\vanilla \resnet}.}
    \label{fig:supp:quali_voc_1:vanilla}
    \end{subfigure}
    \begin{subfigure}[c]{\textwidth}
    \centering
    \begin{subfigure}[c]{.475\columnwidth}
    \includegraphics[width=\textwidth]{results/VOC/figures/qualitative/loss_comp_xdnn_Input.pdf}
    \end{subfigure}
    \begin{subfigure}[c]{.475\columnwidth}
    \includegraphics[width=\textwidth]{results/VOC/figures/qualitative/loss_comp_xdnn_Final.pdf}
    \end{subfigure}
    \caption{\textbf{\xdnn \resnet}.}
    \label{fig:supp:quali_voc_1:xdnn}
    \end{subfigure}
    \caption{Qualitative examples from the \textbf{\vocs dataset}. In particular, this figure allows to compare between models (\textbf{major rows}, \ie (a), (b), and (c)) losses (\textbf{major columns}) and layers (\textbf{left+right}) for multiple images (\textbf{minor rows}).}
    \label{fig:supp:quali_voc_1}
\end{figure}