\begin{figure}[h]
    \centering
    \begin{subfigure}[c]{.475\textwidth}
    \includegraphics[width=\textwidth]{results/VOC/figures/qualitative/dilation_comp_17_0.pdf}
    \end{subfigure}\hfill
    \begin{subfigure}[c]{.475\textwidth}
    \includegraphics[width=\textwidth]{results/VOC/figures/qualitative/dilation_comp_18_0.pdf}
    \end{subfigure}\hfill
    \begin{subfigure}[c]{.475\textwidth}
    \includegraphics[width=\textwidth]{results/VOC/figures/qualitative/dilation_comp_69_0.pdf}
    \end{subfigure}\hfill
    \begin{subfigure}[c]{.475\textwidth}
    \includegraphics[width=\textwidth]{results/VOC/figures/qualitative/dilation_comp_128_0.pdf}
    \end{subfigure}\hfill
    \begin{subfigure}[c]{.475\textwidth}
    \includegraphics[width=\textwidth]{results/VOC/figures/qualitative/dilation_comp_194_0.pdf}
    \end{subfigure}\hfill
    \begin{subfigure}[c]{.475\textwidth}
    \includegraphics[width=\textwidth]{results/VOC/figures/qualitative/dilation_comp_95_0.pdf}
    \end{subfigure}\hfill
    \caption{\textbf{Qualitative examples of the impact of using coarse bounding boxes for guidance.} We show examples of \bcos attributions from the input layer on the baseline model and on models guided with the \energyloss and \loneloss localization losses with varying degrees of dilation $\{10\%,25\%,50\%\}$ in bounding boxes during training. For each example (\textbf{block} in the figure), we show the image and bounding boxes with varying degrees of dilation (\textbf{top} row), attributions with the \loneloss localization loss (\textbf{middle} row), and attributions with the \energyloss localization loss (\textbf{bottom} row). We find that in contrast to using the \loneloss localization  loss, guidance with \energyloss localization loss maintains localization of attributions to on-object features even with dilated bounding boxes. Note that bounding boxes are dilated only during training, not during evaluation. Bounding boxes in \textbf{light blue} show the extent of dilation that \textit{would have been used} had the image been from the training set, while those in \textbf{dark blue} show undilated bounding boxes that are used during evaluation.}
    \label{fig:supp:dilation_quali}
\end{figure}