\begin{figure}[h]
    \centering
    {\textbf{Comparison to \gradcam} on {\cocos}.}\vspace{.25cm}\\
    \begin{subfigure}[c]{\textwidth}
    \includegraphics[width=\textwidth]{results/COCO/figures/loc/gradcam_f1.pdf}
    \caption{\textbf{\epg vs.~F1.}}
    \end{subfigure}
    \begin{subfigure}[c]{\textwidth}
    \includegraphics[width=\textwidth]{results/COCO/figures/iou/gradcam_f1.pdf}
    \caption{\textbf{\iou vs.~F1.}}
    \end{subfigure}
    \caption{\textbf{Quantitative results using \gradcam on \cocos.} We show \epg \textbf{(a)} and  \iou \textbf{(b)} scores vs.~F1 scores  for all localization losses and models using \gradcam at the final layer (\textbf{bottom rows} in (a)+(b) and compare it to the results shown in the main paper (\textbf{top rows}). 
    As expected, \gradcam performs very similarly to \ixg (\vanilla) and \intgrad (\xdnn) used at the final layer---in particular, note that for \resnet architectures, \ixg and \intgrad are very similar to \gradcam for \vanilla and \xdnn models respectively (see \cref{supp:sec:quantitative:gradcam}). Similarly, we find \gradcam to also perform comparably to the \bcos explanations when used at the final layer; for results on \vocs, see \cref{fig:supp:gradcam:voc}.}
    \label{fig:supp:gradcam:coco}
\end{figure}