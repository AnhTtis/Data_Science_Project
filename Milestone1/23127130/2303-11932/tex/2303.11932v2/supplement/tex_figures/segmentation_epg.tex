
\begin{figure}
    \centering    
    {Evaluating \textbf{on-object localization} within bounding boxes.}\vspace{.25cm}\\
    \begin{subfigure}[c]{\textwidth}
    \centering
    \includegraphics[width=.55\textwidth]{results/figures/qualitative/segmentation_schema.pdf}
    \caption{\textbf{Evaluating \emph{on-object} localization within the bounding boxes: On-object \epg.} In the standard \epg metric (\textbf{middle} column), we compute the fraction of positive attributions within the bounding boxes. In other words, attributions within the bounding box (\textbf{green} region) positively impact the metric, while attributions outside (\textbf{blue} region) negatively impact it. Since bounding boxes are coarse annotations and often include background regions, the standard \epg does not evaluate how well attributions localize \textit{on-object} features, \eg the person in the figure. To measure this, we evaluate with an additional Segmentation \epg metric (\textbf{right} column), where we compute the fraction of positive attributions in the bounding box that lie within the segmentation mask of the object. Here, attributions within the segmentation mask (\textbf{green} region) positively impact the metric, and attributions outside the segmentation mask and inside the bounding box (\textbf{blue} region) negatively impact it. Note that attributions outside the bounding box have no effect on Segmentation \epg. As an example and to visualize qualitative differences between losses, in the bottom rows (\lone, \epgloss), we show attributions for a \bcos model guided at the input layer. As becomes clear, by employing a uniform prior on attributions within the bounding box, the \lone loss is effectively  optimized to fill the entire bounding box and thus to not only highlight \emph{on-object} features. This can also be observed quantitatively, see \eg \cref{fig:seg_epg:epg_vs_f1}, right column.}
    \label{fig:seg_epg:schema}
    \end{subfigure}
    \begin{subfigure}[c]{\textwidth}
    \includegraphics[width=\textwidth]{results/VOCSegment/figures/loc/all_results_f1.pdf}
    \caption{\textbf{On-object \epg results.} We evaluate across models (\textbf{columns}) and layers (\textbf{rows}) for the \energyloss and \loneloss localization losses. As seen qualitatively (\eg \cref{fig:loss_comp}), we find that the \energyloss loss is more effective than the \loneloss loss in localizing attributions to the object as opposed to background regions within the bounding boxes. This is explained by the fact that the \loneloss loss promotes uniformity in attributions within the bounding box, and treats both on-object and background features inside the box with equal importance, while the \energyloss loss only optimizes for attributions to lie within the bounding box without placing any constraints on where they may lie, leaving the model free to decide which regions within the box are important for its decision.}
    \label{fig:seg_epg:epg_vs_f1}
    \end{subfigure}
    \caption{\textbf{Evaluating \emph{on-object} localization via \epg.} We show \textbf{(a)} the schema for the on-object \epg metric and how it differs the standard bounding box \epg metric, and \textbf{(b)} quantitative results on evaluating with on-object \epg.}
    \label{fig:seg_epg}
\end{figure}
