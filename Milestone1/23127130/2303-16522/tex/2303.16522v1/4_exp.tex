\section{Experiments}

\subsection{Performance of the proposed model}
The accuracy, sensitivity, specificity, and AUC of our model using the test dataset are presented in Table~\ref{table:metrics}. In the deep wound task, our model achieved an accuracy of 0.739, sensitivity of 0.693, specificity of 0.806, and AUC of 0.804. In terms of infected wound, the corresponding values for our model are 0.685, 0.653, 0.737, and 0.751, respectively. For arterial wound, the corresponding values for our model are 0.864, 0.688, 0.915, and 0.897, respectively. For venous wound, the corresponding values for our model are 0.960, 0.632, 0.975, and 0.924, respectively. In terms of pressure wound, the corresponding values for our model are 0.906, 0.750, 0.932, and 0.940, respectively.
\begin{table}[t]
\caption{Accuracy, sensitivity, specificity, and AUC of our model for the test dataset.}
\label{table:metrics}
\centering
\vspace{0.2mm}
\resizebox{\columnwidth}{!}{
\begin{tabular}{@{}ccccc@{}}
\toprule
% \multicolumn{2}{l}{}
% & \multicolumn{3}{c}{Dataset}
% & \\
% \cmidrule(lr){3-5}
% \multicolumn{2}{c}{}
 & Accuracy & Sensitivity & Specificity & AUC \\
\midrule
% \multirow{2}{*}{{\makecell{Number \\in dataset}}}
Deep wound & 0.739 & 0.693 & 0.806 & 0.804 \\
Infected wound & 0.685 & 0.653 & 0.737 & 0.751 \\
Arterial wound & 0.864 & 0.688 & 0.915 & 0.897 \\
Venous wound & 0.960 & 0.632 & 0.975 & 0.924 \\
Pressure wound & 0.906 & 0.750 & 0.932 & 0.940 \\
\bottomrule
\end{tabular}
}
\vspace{-2mm}
\end{table}



\subsection{Performance comparison with the humans}
The accuracy, sensitivity, specificity, and Cohen’s kappa with 95\% CI for the proposed model and seven medical personnel are showed in Figure~\ref{fig:Figure3}. For the deep wound task, resident A, resident B, and nurse C exhibited low sensitivity values of 0.525, 0.333, and 0.293, respectively. Nurse A exhibited a low specificity of 0.487. For infected wound, our model and all medical personnel yielded the worst average performance. Attending B and resident B exhibited a low sensitivity of 0.511, and 0.233, respectively. Resident A, nurse A, nurse B, and nurse C exhibited low specificity values of 0.517, 0.495, 0.533, and 0.472, respectively.  For arterial wound, attending B, resident B, nurse B, and nurse C exhibited low sensitivity values of 0.526, 0.526, 0.591, and 0.591, respectively. For venous wound, our model yielded a low sensitivity of 0.504. Resident B exhibited a low specificity of 0.330. In terms of pressure wound, resident B exhibited a low sensitivity of 0.370.

It can be observed that the proposed model performed significantly better than resident B, nurse A, and nurse C in the task of deep wound and is non-inferior to the other four medical personnel based on the Cohen’s kappa difference listed in Table~\ref{table:kappa}. In the task of infected wound, our model performed significantly better than attending B, resident B, nurse B, and nurse C and is non-inferior to the others. In the task of arterial wound, our model performed significantly better than attending B, and resident B and is non-inferior to the others. For venous wound, our model performed significantly better than resident B and nurse C and is non-inferior to the others. For pressure wound, our model performed significantly better than resident B and is non-inferior to the others.  

Figure~\ref{fig:Figure4} shows the ROC curves with their 95\% confidence band and the characteristic points of the medical personnel. In the task of deep wound, all the points fell inside the confidence band, implying comparable performance. In the task of infected wound, the characteristic points of attending A, attending B, resident A, and nurse A fell inside the confidence band (implying comparable performance), whereas the rest of the points are on the lower right-hand side of the ROC curve (implying inferior performance to the model). In the tasks of arterial wound and pressure wound, the characteristic points of attending B and resident B fell on the lower right-hand side of the ROC confidence band, implying inferior performance to the model, whereas the characteristic points of the rest of the medical personnel fell inside the ROC confidence band. In the task of venous wound, the characteristic point of the resident B is on the lower right-hand side of the ROC curve (implying inferior performance to the model). From the relative positions of the characteristic points and ROC confidence band, our proposed model is either superior or has a comparable performance as human medical personnel.
\begin{figure*}[t!]
\centerline{
	\hspace{3mm}\includegraphics[width=1.0\textwidth]{figures/Figure3_reform.pdf}
 	% \vspace{-2mm}
}  
    \caption{The accuracy, sensitivity, specificity, and Cohen’s kappa with 95\% CI for the proposed model and seven medical personnel.}
    \vspace{-2mm}
	\label{fig:Figure3}
\end{figure*}
\begin{table}[t]
\caption{Difference in Cohen’s kappa with 95\% confidence intervals for performance comparison of our model and medical personnel. * indicates significant difference.}
\label{table:kappa}
\centering
\resizebox{\columnwidth}{!}{
\begin{tabular}{@{}ccccccc@{}}
\toprule
% \cmidrule(lr){3-5}

& & Deep wound & Infected wound & Arterial wound & Venous wound & Pressure wound \\
\midrule
\multirow{2}{*}{Attending A} & Difference & -0.007 & 0.057 & 0.062 & -0.162 & -0.037 \\
&95\% CI & [-0.117,0.107] & [-0.045,0.156] & [-0.075,0.194] & [-0.459,0.130] & [-0.192,0.125]\\
\multirow{2}{*}{Attending B} & Difference & 0.035 & \textbf{0.142*} & \textbf{0.177*} & 0.032 & 0.148 \\ 
&95\% CI & [-0.070,0.141] & [0.033,0.253] & [0.045,0.313] & [-0.209,0.266] & [-0.008,0.307] \\
\multirow{2}{*}{Resident A} & Difference & 0.053 & 0.101 & 0.051 & 0.186 & -0.073 \\
&95\% CI & [-0.053,0.159] & [-0.026,0.225] & [-0.069,0.167] & [-0.044,0.403] & [-0.191,0.052]\\
\multirow{2}{*}{Resident B} & Difference & \textbf{0.189*} & \textbf{0.347*} & \textbf{0.328*} & \textbf{0.408*} & \textbf{0.223*} \\
&95\% CI & [0.085,0.291] & [0.242,0.446] & [0.199,0.456] & [0.162,0.632] & [0.036,0.406]\\
\multirow{2}{*}{Nurse A} & Difference & \textbf{0.126*} & 0.101 & 0.093 & 0.037 & 0.101 \\
&95\% CI & [0.012,0.247] & [-0.018,0.222] & [-0.008,0.189] & [-0.216,0.296] & [-0.018,0.219] \\
\multirow{2}{*}{Nurse B} & Difference & 0.07 & \textbf{0.204*} & 0.024 & 0.200 & -0.048 \\ 
&95\% CI & [-0.047,0.186] & [0.080,0.332] & [-0.083,0.132] & [-0.037,0.431] & [-0.190,0.087] \\
\multirow{2}{*}{Nurse C} & Difference & \textbf{0.232*} & \textbf{0.249*} & 0.060 & \textbf{0.269*} & 0.06 \\
&95\% CI & [0.139,0.322] & [0.120,0.376] & [-0.045,0.165] & [0.073,0.456] & [-0.096,0.221]\\

\bottomrule
\end{tabular}
}
\end{table}



% \paragraph{}


