\section{Introduction}\label{sec:intro}

Wound healing is one of the oldest and yet the most important challenges in the medical field. Normal wound healing is a dynamic, interactive process involving soluble mediators, blood cells, extracellular matrix, and parenchymal cells~\cite{ref1_singer1999cutaneous}. Factors that deviate from the normal process will result in abnormal healing and chronic wounds. Chronic wounds affect the quality of life of nearly 2.5\% of the total population in the United States~\cite{ref2_sen2021human}. The management of wounds has a significant economic impact on health care; this becomes even more challenging with aging of the population. The underlying mechanism of chronic wounds varies significantly, but includes factors that influence blood supply (peripheral vascular disease), immune function (such as immunosuppression or acquired immunodeficiency), metabolic diseases (such as diabetes), medications, or previous local tissue injury (such as radiation therapy)~\cite{ref3_han2017chronic}. Well-trained medical wound specialists are crucial for correct diagnosis and proper wound treatment, but are usually not readily available in primary healthcare facilities.

The increasing use of artificial intelligence (AI) technologies and portable devices such as smartphones has led to timely development of remote and intelligent diagnosis and prognosis systems for wound care~\cite{ref4_anisuzzaman2022image}. Deep learning, a subdomain of machine learning and AI, was inspired by the human brain and requires a large amount of data for automatic mapping between the input and output. Certain pre-set rules are not required. Image-based wound classification using deep learning is a new field of interest. Convolutional neural network (CNN) -based methods have been proposed to detect infected wounds~\cite{ref5_wang2015unified,ref6_shenoy2018deepwound}, identify the feature differences between a healthy skin and diabetic foot wounds~\cite{ref7_goyal2018dfunet}, and perform optimized segmentation of different tissue types present in pressure injuries~\cite{ref8_zahia2018tissue}.

Well-trained medical wound specialists are crucial for correct diagnosis and proper wound treatment. However, they are usually not readily available in primary healthcare facilities. An intelligent system can provide information about patients’ wounds to general practitioners, nurses, and even the patients themselves before seeing a wound specialist such that timely and efficient referrals can be achieved. We chose five wound assessment tasks to be classified using our deep learning model, namely deep wound, infected wound, arterial wound, venous wound, and pressure wound. Deep wounds are closely related to wound severity. Infected wounds are the most common cause of poor wound healing. Arterial, venous, and pressure wounds are among the most important pathogenesis of chronic non-healing wounds.