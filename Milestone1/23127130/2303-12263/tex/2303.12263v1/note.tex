\documentclass[a4paper,12pt]{article}
\usepackage{enumerate}

\usepackage[svgnames]{xcolor}% tikzより前に読み込む必要あり
%\usepackage{tikz}


\usepackage{geometry}
\usepackage{tikz}
\usepackage{amsmath,amsfonts,amsthm,amssymb,graphicx,natbib,geometry,arydshln,umoline,subfig,txfonts,setspace,enumerate}
\usepackage[title]{appendix}

%\usepackage{pxfonts}
%\usepackage{txfonts}
%\usepackage{ascmac}

\renewcommand{\baselinestretch}{1.46}

\geometry{a4paper, left=1.15in,right=1.15in,top=1.15in,bottom=1.15in}
%\geometry{a4paper, left=1.0in,right=1.0in,top=1.0in,bottom=1.0in}


\newtheorem{theorem}{Theorem}
\newtheorem{acknowledgement}{Acknowledgement}
\newtheorem{algorithm}{Algorithm}
\newtheorem{axiom}{Axiom}
\newtheorem{case}{Case}
\newtheorem{conclusion}{Conclusion}
\newtheorem{condition}{Condition}
\newtheorem{conjecture}{Conjecture}
\newtheorem{criterion}{Criterion}
\newtheorem{exercise}{Exercise}
\newtheorem{lemma}{Lemma}
\newtheorem{notation}{Notation}
\newtheorem{problem}{Problem}
\newtheorem{proposition}{Proposition}
\newtheorem{corollary}[proposition]{Corollary}
\newtheorem{claim}{Claim}

\theoremstyle{definition}
%\newtheorem*{definition}{Definition}
\newtheorem{definition}{Definition}
\newtheorem{example}{Example}

\theoremstyle{remark}
\newtheorem{remark}{Remark}
\begin{document}
\[
e^{-\frac{\eta ^2 \xi  (y-\kappa )^2}{2 (\eta +\xi ) (\eta +2 \xi )}}
\leq 
\frac{\sqrt{2 \pi } \sqrt{\xi } \sqrt{\eta ^2+3 \eta  \xi +2 \xi ^2}}{\eta  (\eta +\xi )}
\]

\begin{align}
4-\frac{\sqrt{\frac{2}{\pi }} \left(\eta ^2-2 \xi ^2\right) \sqrt{\eta ^2+3 \eta  \xi +2 \xi ^2} e^{-\frac{\eta ^2 \xi  (y-\kappa )^2}{2 (\eta +\xi ) (\eta +2 \xi )}}}{\sqrt{\xi } (\eta +2 \xi )^2}
\geq 
4-\frac{\sqrt{\frac{2}{\pi }} \left(\eta ^2-2 \xi ^2\right) \sqrt{\eta ^2+3 \eta  \xi +2 \xi ^2} 
\frac{\sqrt{2 \pi } \sqrt{\xi } \sqrt{\eta ^2+3 \eta  \xi +2 \xi ^2}}{\eta  (\eta +\xi )}
}{\sqrt{\xi } (\eta +2 \xi )^2}\\
a
\end{align}




\end{document}
