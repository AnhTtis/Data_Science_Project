\begin{table*}[h]
\footnotesize
\begin{subtable}[t]{0.33\linewidth}
\centering
\vspace{-17.5pt}
\resizebox{\linewidth}{!}{
\begin{tabular}{cc c c}
\Xhline{1.2pt}
\textbf{Classification}       & \textbf{Localization}   & mAP & GFLOPs \\ \Xhline{1.2pt}
\rowcolor[HTML]{EFEFEF} 
\multicolumn{2}{c}{4D Feature Space} & 23.1 & 44.4       \\
Key-Frame Features   & Res5 Features  & 22.8 & 53.7       \\ \Xhline{1.2pt}
\end{tabular}

}
%\vspace{6pt}
\caption{\textbf{Feature space.} Sampling features from 4D feature space is more effective and efficient than sampling from key-frame features for classification and res5 features for localization.}
\label{tab:featspace}
\end{subtable}
\hfill
\begin{subtable}[t]{0.25\linewidth}
\centering
\vspace{-17.5pt}
\resizebox{0.98\linewidth}{!}{
\begin{tabular}{l c c}
\Xhline{1.2pt}
                           & mAP & GFLOPs \\ \Xhline{1.2pt}
\rowcolor[HTML]{EFEFEF} 
Simple Lateral Conv. & 23.1 & 44.4  \\
Full FPN                   & 22.9 & 45.8  \\ \Xhline{1.2pt}
\end{tabular}
}
%\vspace{6pt}
\caption{\textbf{4D feature space construction.} Simple lateral convolution achieves comparable performance while being more efficient than using full FPN.}
\label{tab:featspacecons}
\end{subtable}
\hfill
\begin{subtable}[t]{0.38\linewidth}
\centering
\resizebox{\linewidth}{!}{
\begin{tabular}{l c c c}
\Xhline{1.2pt}
\textbf{Sampling Strategy}                    & $P_{in}$  & mAP & GFLOPs \\ \Xhline{1.2pt}
Fixed Grid Sampling                  & 49 & 21.9 & 
45.6 \\
\rowcolor[HTML]{EFEFEF} 
Adaptive Sampling + Temporal Copying & 32 & 23.1 & 44.4 \\
Adaptive Sampling + Temporal Moving  & 32 & 23.3 &   44.6 \\ \Xhline{1.2pt}
\end{tabular}
}
% \vspace{1pt}
\caption{\textbf{Sampling strategy.} Sampling fewer feature points, our adaptive sampling strategy achieves better performance than fixed grid sampling. Temporal moving brings slight improvement in mAP.}
\label{tab:samplestrategy}
\end{subtable}

\begin{subtable}[t]{0.33\linewidth}
\centering
\vspace{-4pt}
\resizebox{0.89\linewidth}{!}{
\begin{tabular}{c c c >{\columncolor[HTML]{EFEFEF}}c c c}
\Xhline{1.2pt}
$P_{in}$ & 8 & 16 & 32 & 48 & 64 \\ \Xhline{1.2pt}
mAP      & 22.4 & 22.5 & 23.1 & 23.0 & 22.5 \\ 
GFLOPs   & 42.4 & 43.1 & 44.4 & 45.6 & 46.9 \\ \Xhline{1.2pt}
\end{tabular}
}
\caption{\textbf{Number of sampling points.} Sampling 32 points per frame achieves the best performance.}
\label{tab:samplenum}
\end{subtable}
\hfill
\begin{subtable}[t]{0.25\linewidth}
\centering
\vspace{-4pt}
\resizebox{\linewidth}{!}{
\begin{tabular}{c c c >{\columncolor[HTML]{EFEFEF}}c c}
\Xhline{1.2pt}
$N$     & 15    & 50    & 100    & 150    \\ \Xhline{1.2pt}
mAP     & 22.1  & 22.9  & 23.1   & 22.5   \\
GFLOPs  & 31.7  & 36.9  & 44.4   & 51.8   \\ \Xhline{1.2pt}
\end{tabular}
}
\caption{\textbf{Number of queries.} Using 100 queries works the best.}
\label{tab:numquery}
\end{subtable}
\hfill
\begin{subtable}[t]{0.38\linewidth}
\centering
\vspace{5pt}
%\vspace{22pt}
\resizebox{0.9\linewidth}{!}{

\begin{tabular}{l c c}
\Xhline{1.2pt}
\textbf{Mixing Strategy }                  & mAP  & GFLOPs \\ \Xhline{1.2pt}
fixed parameter dual-branch mixing            & 22.5 & 36.8   \\
\rowcolor[HTML]{EFEFEF} 
dual-branch spatiotemporal mixing & 23.1 & 44.4   \\
spatial mixing only               & 22.4 & 40.4   \\
temporal mixing only              & 22.1 & 33.5   \\
sequential spatiotemporal mixing  & 22.6 & 43.6   \\
coupled spatiotemperal  mixing    & 22.8 & 93.2   \\
  \Xhline{1.2pt}
\end{tabular}
}
% \vspace{3pt}
\caption{\textbf{Mixing strategy.} Query-guided adaptive feature mixing outperforms fixed parameter mixing and our dual-branch spatiotemporal feature mixing strategy works the best. }
\label{tab:mixingstrategy}
\end{subtable}

\begin{subtable}[t]{0.33\linewidth}
\centering
% \vspace{-40pt}
\vspace{-48pt}
\resizebox{1.0\linewidth}{!}{

\begin{tabular}{c c c >{\columncolor[HTML]{EFEFEF}}c c c}
\Xhline{1.2pt}
$P_{out}$/$T_{out}$  & 64/8   & 96/12   & 128/16  & 160/20  & 192/24  \\\Xhline{1.2pt}

mAP        & 22.5 & 22.8 & 23.1 & 22.9 & 22.6 \\
GFLOPs     & 40.2 & 42.3 & 44.4 & 46.4 & 48.4 \\ \Xhline{1.2pt}
\end{tabular}
}
\caption{\textbf{Number of mixing out patterns.} A moderate number of mixing out patterns works the best. }
\label{tab:outpatterns}
\end{subtable}
\quad
\begin{subtable}[t]{0.25\linewidth}
\centering
% \vspace{-40pt}
\vspace{-48.5pt}
\resizebox{0.95\linewidth}{!}{
\begin{tabular}{c c c >{\columncolor[HTML]{EFEFEF}}c c}
\Xhline{1.2pt}
$M$     & 1    & 3    & 6    & 9    \\ \Xhline{1.2pt}
mAP     & 18.4 & 22.5 & 23.1 & 22.6 \\
GFLOPs  & 32.0 & 36.9 & 44.4 & 51.8 \\ \Xhline{1.2pt}
\end{tabular}
}
% \vspace{7.5pt}
\caption{\textbf{Number of ASAM modules.} Using 6 ASAM modules works the best.}
\label{tab:nummodules}
\end{subtable}

\vspace{-3pt}
\caption{\textbf{Ablations Experiments.} We use a SlowOnly ResNet-50 backbone to perform our ablation studies. Models are trained on the training set of AVA v2.2 and evaluated on the validation set. Default choices for our model are colored in \colorbox[HTML]{EFEFEF}{gray}. }
\label{tab:ablations}
\vspace{-2mm}
\end{table*}
















