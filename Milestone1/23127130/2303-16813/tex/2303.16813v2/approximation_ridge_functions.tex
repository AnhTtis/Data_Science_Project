
\subsection{Approximation using Ridge Functions}
\label{sec: ridge_reordered}
In this section we prove for $s \in \NN_{\geq 2}$ that every function in $C^k([-1,1]^s; \RR)$ can be uniformly approximated with an error of the order $m^{-k/(s-1)}$ using a linear combination of $m$ so-called \emph{ridge functions}. In fact, we only consider ridge \emph{polynomials}, meaning functions of the form
\begin{equation*}
\RR^s \to \RR, \quad x \mapsto p(a^T x)
\end{equation*}
for a fixed vector $a \in \RR^s$ and a polynomial $p: \RR \to \RR$. Note that this result has already been obtained in a slightly different form in \cite[Theorem 1]{maiorov_best_1999}; namely, it is shown there that the rate of approximation $m^{-k/(s-1)}$ can be achieved by functions of the form $\sum_{j=1}^m f_j(a_j^T x)$ with $a_j \in \RR^s$ and $f_j \in L^1_\text{loc} (\RR^s)$. We will need the fact that the $f_j$ can actually be chosen as polynomials and that the vectors $a_1, ..., a_m$ can be chosen independently from the particular function $f$. This is shown in the proof of \cite{maiorov_best_1999}, but not stated explicitly. For this reason, and in order to clarify the proof itself and to make the paper more self-contained, we decided to present the proof in this appendix.

\begin{lemma} \label{lem: hom_dim}
Let $m,s \in \NN$. Then we denote by
\begin{equation*}
P_m^s \defeq \left\{ \RR^s \to \RR, \quad x \mapsto \underset{\vert \kk \vert \leq m}{\sum_{\kk \in \NN_0^s}} a_\kk x^\kk : \ a_\kk \in \RR\right\}
\end{equation*}
the set of real polynomials of degree at most $m$. The subset of \emph{homogeneous} polynomials of degree $m$ is defined as
\begin{equation*}
H_m^s \defeq \left\{ \RR^s \to \RR, \quad x \mapsto \underset{\vert \kk \vert = m}{\sum_{\kk \in \NN_0^s}} a_\kk x^\kk : \ a_\kk \in \RR\right\}.
\end{equation*}
Then there exists a constant $c = c(s) > 0$ satisfying
\begin{equation*}
\dim (H_m^s) \leq c \cdot m^{s-1} \quad \forall m \in \NN.
\end{equation*}
\end{lemma}
\begin{proof}
It is immediate that the set
\begin{equation*}
\left\{ \RR^s \to \RR, \ x \mapsto x^\kk: \  \kk \in \NN_0^s, \ \vert \kk \vert =m\right\}
\end{equation*}
forms a basis of $H_m^s$, hence 
\begin{equation*}
\dim (H_m^s) = \# \left\{ \kk \in \NN_0^s: \ \vert \kk \vert = m\right\}.
\end{equation*}
This quantity clearly equals the number of possibilities for drawing $m$ times from a set with $s$ elements with replacement. Hence, we see
\begin{equation*}
\dim (H_m^s) = \binom{s+m-1}{m},
\end{equation*}
see for instance \cite[Identity 143]{benjamin_proofs_2003}. A further estimation shows
\begin{align*}
 \binom{s+m-1}{m} = \prod_{j=1}^{s-1} \frac{m+j}{j} = \prod_{j=1}^{s-1} \left(1 + \frac{m}{j}\right) \leq (1+m)^{s-1} \leq 2^{s-1} \cdot m^{s-1}.
\end{align*}
Hence, the claim follows with $c(s) = 2^{s-1}$.
\end{proof}

A combination of results from \cite{pinkus_ridge_2016} together with the fact that it is possible to approximate $C^k$-functions using polynomials of degree at most $m$ with an error of the order $m^{-k}$, as shown in \Cref{app: fourier_approx}, yields the desired result.

\begin{theorem} \label{app: ridge}
Let $s,k \in \NN$ with $s \geq 2$ and $r>0$. Then there exists a constant $c = c(s,k) > 0$ with the following property: For every $m \in \NN$ there exist $a_1, ..., a_m \in \RR^s \setminus \{0\}$ with $\Vert a_j \Vert_2 = r$, such that for every function $f \in C^k ([-1,1]^s ; \RR)$ there exist polynomials $p_1, ..., p_m \in P_m^1$ satisfying
\begin{equation*}
\left\Vert f(x) - \sum_{j=1}^m p_j (a_j^T x) \right\Vert_{L^\infty ([-1,1]^s ; \RR)} \leq c \cdot m^{-k/(s-1)} \cdot \Vert f \Vert_{C^k([-1,1]^s ; \RR)}.
\end{equation*}  
\end{theorem}
\begin{proof}
We first pick the constant $c_1 = c_1(s) $ according to \Cref{lem: hom_dim}. Then we define the constant $c_2 = c_2(s) \defeq (2s)^{s-1} \cdot c_1(s)$ and let $M \in \NN$ be the largest integer satisfying 
\begin{equation*}
c_2 \cdot M^{s-1} \leq m.
\end{equation*}
Here, we assume without loss of generality that $m \geq c_2$, which can be justified by choosing $p_j = 0$ for every $j \in \{1,...,m\}$ if $m < c_2$, at the cost of possibly enlarging $c$. Note that the choice of $M$ implies $c_2 \cdot (2M)^{s-1} \geq c_2 \cdot (M+1)^{s-1}> m$, and thus
\begin{equation} \label{eq: bound_lower}
M \geq \frac{1}{2} \cdot c_2^{-1/(s-1)} \cdot m^{1/(s-1)} = c_3 \cdot m^{1/(s-1)}
\end{equation}
with $c_3 = c_3(s) \defeq 1/2 \cdot c_2^{-1/(s-1)}$. 

Using \cite[Proposition 5.9]{pinkus_ridge_2016} and \Cref{lem: hom_dim} we can pick $a_1, ..., a_m \in \RR^s \setminus \{0\}$ satisfying
\begin{equation} \label{span: homogeneous}
H_{s(2M-1)}^s = \spann\left\{x \mapsto (a_j^T x )^{s(2M-1)}: \ j \in \{1,...,m\}\right\},
\end{equation}
where we used that
\begin{equation*}
c_1 \cdot (s(2M-1))^{s-1} \leq c_1 \cdot (2s)^{s-1} \cdot M^{s-1} = c_2 \cdot M^{s-1} \leq m. 
\end{equation*}
Here we can assume $\Vert a_j \Vert_2 = r$ for every $j \in \{1,...,m\}$ since multiplying each $a_j$ with a positive constant does not change the span in \eqref{span: homogeneous}.
From \cite[Corollary 5.12]{pinkus_ridge_2016} we infer that
\begin{equation} \label{eq: p_span}
P_{s(2M-1)}^s = \spann\left\{x \mapsto (a_j^T x )^{r}: \ j \in \{1,...,m\}, \ 0 \leq r \leq s(2M-1)\right\}.
\end{equation}

Let $f \in C^k([-1,1]^s ; \RR)$. Then, according to \Cref{app: fourier_approx}, there exists a polynomial $P : \RR^s \to \RR$ of \emph{coordinatewise} degree at most $2M -1$ satisfying
\begin{equation*}
\Vert f - P \Vert_{L^\infty ([-1,1]^s ; \RR)} \leq c_4 \cdot M^{-k} \cdot \Vert f \Vert_{C^k([-1,1]^s ; \RR)},
\end{equation*}
where $c_4 = c_4(s,k) > 0$. Note that by construction it holds $P \in P_{s(2M-1)}^s$. Using \eqref{eq: p_span} we deduce the existence of polynomials $p_1, ..., p_m : \RR \to \RR$ such that
\begin{equation*}
P(x) = \sum_{j= 1}^m p_j(a_j^T x) \quad \text{for all }x \in \RR^s. 
\end{equation*}
Combining the previously shown bounds, we get
\begin{align*}
\left\Vert f(x) - \sum_{j= 1}^m p_j(a_j^T x) \right\Vert_{L^\infty ([-1,1]^s ; \RR)} &= \Vert f(x) - P(x)\Vert_{L^\infty ([-1,1]^s ; \RR)} \leq  c_4 \cdot M^{-k} \cdot \Vert f \Vert_{C^k([-1,1]^s ; \RR)} \\ \overset{\eqref{eq: bound_lower}}&{\leq} c \cdot m^{-k/(s-1)} \cdot \Vert f \Vert_{C^k([-1,1]^s ; \RR)},
\end{align*}
as desired. Here, we defined $c = c(s,k) \defeq c_4 \cdot c_3^{-k}$.
\end{proof}