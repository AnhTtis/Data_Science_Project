
\subsection{Prerequisites from Fourier Analysis} \label{sec:fourier_reordered}
This section is dedicated to reviewing some notations and results from Fourier Analysis. In the end, a quantitative result for the approximation of $C^k \left( [-1,1]^s; \RR\right)$-functions using linear combinations of multivariate Chebyshev polynomials is derived; see \Cref{app: fourier_approx}.

We start by recalling several notations and concepts from Fourier Analysis. 
\begin{definition}
    Let $s \in \NN$ and $k \in \NN_0$. We define
    \begin{equation*}
        C_{2\pi}^k\left(\RR^s; \CC\right) \defeq \left\{ f \in C^k \left(\RR^s; \CC\right) : \ \forall \pp \in \ZZ^s \ \forall x \in \RR^s : \ f(x + 2\pi \pp ) = f(x)\right\}.
    \end{equation*}
    and $C_{2\pi}\left(\RR^s; \CC\right) \defeq C_{2\pi}^0\left(\RR^s; \CC\right)$. For a function $f \in C_{2\pi}^k \left(\RR^s; \CC\right)$ we write 
\begin{align*}
 \left\Vert f \right\Vert_{C^k \left([-\pi, \pi]^s ; \CC\right)} &\defeq \underset{\vert \kk \vert \leq k}{\underset{\kk \in \NN_0^s}{\max}} \left\Vert \partial^\kk f\right\Vert_{L^\infty \left([-\pi, \pi]^s; \CC\right)} \text{ and } \\
\left\Vert f \right\Vert_{L^p \left([-\pi, \pi]^s ; \CC \right)} &\defeq \left(\frac{1}{(2\pi)^s} \cdot \int_{[-\pi, \pi]^s} \left\vert f(x) \right\vert^p dx \right)^{1/p} \text{ for } p \in [1, \infty).
\end{align*}
Moreover, we set $\Vert f \Vert_{L^\infty([-\pi, \pi]^s ; \RR)} \defeq \left\Vert f \right\Vert_{C^0 \left([-\pi, \pi]^s ; \CC\right)}$.
\end{definition}

\begin{definition}
    For any $s \in \NN$ and $\kk \in \Z^s$, we write
    \begin{equation*}
        e_\kk : \quad \RR^s \to \CC, \quad e_\kk (x) = e^{i \langle \kk, x\rangle}
    \end{equation*}
    where $\langle \cdot, \cdot \rangle$ denotes the usual inner product of two vectors. 
    For any $f \in C_{2\pi} \left(\RR^s; \CC\right)$ we define the $\kk$-th Fourier coefficient of $f$ to be
    \begin{equation*}
        \hat{f}(\kk) \defeq \frac{1}{(2\pi)^s} \int_{[-\pi, \pi]^s} f(x) \overline{e_{\kk}(x)}dx.
    \end{equation*}
\end{definition}
\begin{definition}
    For two functions $f,g \in C_{2\pi}\left(\RR^s;\CC\right)$, we define their convolution as
    \begin{equation*}
        f * g : \quad \RR^s \to \CC, \quad (f*g)(x) \defeq \frac{1}{(2\pi)^s} \int_{[-\pi, \pi]^s} f(t)g(x-t) dt.
    \end{equation*}
\end{definition}
In the following we define several so-called kernels.
\begin{definition} 
     Let $m \in \NN_0$ be arbitrary.
    \begin{enumerate}
        \item The $m$-th one-dimensional \emph{Dirichlet-kernel} is defined as
        \begin{equation*}
            D_m \defeq \sum_{h= - m}^{m} e_h.
        \end{equation*}
        \item The $m$-th one-dimensional \emph{Fejèr-kernel} is defined as
        \begin{equation*}
            F_m \defeq \frac{1}{m}\sum_{h=0}^{m-1} D_h.
        \end{equation*}
        \item The $m$-th one-dimensional \emph{de-la-Vallée-Poussin-kernel} is defined as
        \begin{equation*}
            V_m \defeq \left( 1 + e_m + e_{-m}\right) \cdot F_m.
        \end{equation*}
        \item Let $s \in \NN$. We extend the above definitions to dimension $s$ by letting
        \begin{align*}
            D_m^s \left(x_1, ..., x_s\right) &\defeq \prod_{p=1}^s D_m \left(x_p\right), \\
            F_m^s \left(x_1, ..., x_s\right) &\defeq \prod_{p=1}^s F_m \left(x_p\right), \\
            V_m^s \left(x_1, ..., x_s\right) &\defeq \prod_{p=1}^s V_m \left(x_p\right). 
        \end{align*}
    \end{enumerate}
\end{definition}
We need the following property of the multivariate extension of the de-la-Vallée-Poussin-kernel.
\begin{proposition} \label{prop: dlvp-bound}
    Let $m,s \in \NN$. Then one has $\left\Vert V_m^s \right\Vert_{L^1 \left([- \pi, \pi]^s; \CC\right)} \leq 3^s$.
\end{proposition}
\begin{proof}
   From \cite[Exercise 1.3 and Lemma 1.4]{muscalu_classical_2013} it follows $\Vert F_m\Vert_{L^1 ([-\pi, \pi] ; \CC)} = 1$ and hence using the triangle inequality $\Vert V_m \Vert_{L^1 ([-\pi,\pi] ; \CC)} \leq 3$. The claim then follows using Tonelli's theorem.
\end{proof}
The following definition introduces the term of trigonometric polynomial. 
\begin{definition}
    For any $s \in \NN$ and $m \in \NN_0$ we call a function of the form
    \begin{equation*}
    \RR^s \to \CC, \quad x \mapsto \underset{-m \leq \kk \leq m}{\sum_{\kk \in \ZZ_0^s}} a_\kk e^{i \langle \kk , x\rangle}
\end{equation*}
with coefficients $a_\kk \in \CC$ a \emph{trigonometric polynomial of coordinatewise degree at most $m$} and denote the space of all those functions with $\mathbb{H}_m^s$. Here, we consider the sum over all $\kk \in \ZZ^s$ with $-m \leq \kk_j \leq m$ for all $j \in \{1,...,s\}$. We then write
\begin{equation} \label{eq:mintrigo}
    E^s_m(f) \defeq \underset{T \in \mathbb{H}_{m}^s}{\mathrm{min}} \left\Vert f - T\right\Vert_{L^\infty \left(\RR^s;\CC\right)}
\end{equation}
for any function $f \in C_{2\pi} \left(\RR^s ; \CC\right)$.
\end{definition}
The following proposition shows that convolving with the Fejèr kernel produces a trigonometric polynomial of degree at most $2m-1$, while reproducing trigonometric polynomials of degree $m$. Furthermore, the norm of the convolution operator is bounded uniformly in $m$. These properties will be useful for our proof of \Cref{app: fourier_approx}.
\begin{proposition}
\label{vm}
    Let $s,m \in \NN$ and $k \in \NN_0$. The map
    \begin{equation*}
        v_m : \quad C_{2\pi} \left(\RR^s; \CC\right) \to \mathbb{H}_{2m-1}^s, \quad f \mapsto f * V_m^s 
    \end{equation*}
    is well-defined and satisfies
    \begin{equation}
    \label{ident}
        v_m(T) = T \quad \text{for all} \quad T \in \mathbb{H}_m^s.
    \end{equation}
    Furthermore, there exists a constant $c = c(s) > 0$ (independent of $m$), such that
    \begin{align}
        \left\Vert v_m(f)\right\Vert_{C^k\left([-\pi,\pi]^s; \CC\right)} \leq c \cdot \left\Vert f\right\Vert_{C^k\left([-\pi,\pi]^s; \CC\right)} \ \forall f \in C^k_{2\pi} \left(\RR^s; \CC\right), \nonumber \\
        \label{constant}
        \left\Vert v_m(f)\right\Vert_{L^\infty \left([\pi,\pi]^s; \CC\right)} \leq c \cdot \left\Vert f\right\Vert_{L^\infty \left([-\pi, \pi]^s; \CC\right)} \ \forall f\in C_{2\pi} \left(\RR^s; \CC\right).
    \end{align}
    In fact, it holds $c(s) \leq \exp(C \cdot s)$ with an absolute constant $C>0$.
\end{proposition}
\begin{proof}
    A direct computation shows that $f \ast e_{\kk} = \hat{f}(\kk) \cdot e_{\kk}$. This implies that $v_m$ is well-defined since $V_m^s$ is a trigonometric polynomial of coordinatewise degree at most $2m-1$. 

    The operator is bounded on $C^k_{2\pi}(\RR^s;\CC)$ and $C_{2\pi}(\RR^s;\CC)$ with norm at most $c = 3^s$, as follows from Young's inequality \cite[Lemma 1.1 (ii)]{muscalu_classical_2013}, \Cref{prop: dlvp-bound}, and the fact that one has for all $\kk \in \NN_0^s$ with $\vert \kk \vert \leq k$ the identity
    \begin{equation*}
         \partial^\kk \left( f * V_m^s\right) = \left(\partial^\kk f\right) * V_m^s \quad \text{for } f \in C^k_{2\pi}(\RR^s ; \CC).
    \end{equation*}
    It remains to show that $v_m$ is the identity on $\mathbb{H}_m^s$. We first prove that 
    \begin{equation}
    \label{firstident}
        e_k * V_m = e_k
    \end{equation}
    holds for all $k \in \Z$ with $\vert k \vert \leq m$. First note that
    \begin{equation*}
        e_k * V_m = e_k * F_m + e_k * \left(e_m \cdot F_m\right) + e_k * \left(e_{-m} \cdot F_m\right).
    \end{equation*}
    We then compute
    \begin{align*}
        e_k * F_m = \frac{1}{m} \sum_{\ell = 0}^{m-1} D_\ell * e_k = \frac{1}{m}\sum_{\ell = 0}^{m-1} \sum_{h = - \ell}^{\ell} \underbrace{e_h * e_k}_{= \delta_{k,h} \cdot e_k} = \frac{1}{m} \sum_{\ell = \vert k \vert}^{m-1} e_k = \frac{m - \vert k \vert}{m} \cdot e_k.
    \end{align*}
    Similarly, we get
    \begin{align*}
        e_k *\left( e_m \cdot F_m \right) &= \frac{1}{m} \sum_{\ell = 0}^{m-1} \left( e_m D_\ell \right) * e_k = \frac{1}{m}\sum_{\ell = 0}^{m-1} \sum_{h = - \ell}^{\ell} \underbrace{e_{h+m} * e_k}_{= \delta_{k,h+m} \cdot e_k}\\
        & = \frac{1}{m} \underset{\ell \geq m-k}{\sum_{0 \leq \ell \leq m-1 }} e_k = \delta_{k \geq 1} \cdot \frac{k}{m} \cdot e_k
    \end{align*}
    and 
    \begin{align*}
        e_k *\left( e_{-m} \cdot F_m \right) &= \frac{1}{m} \sum_{\ell = 0}^{m-1} \left( e_{-m} D_\ell \right) * e_k = \frac{1}{m}\sum_{\ell = 0}^{m-1} \sum_{h = - \ell}^{\ell} \underbrace{e_{h-m} * e_k}_{= \delta_{k,h-m} \cdot e_k} \\
        &= \frac{1}{m} \underset{\ell \geq k+m}{\sum_{0 \leq \ell \leq m-1}} e_k = \delta_{k \leq -1} \cdot \frac{-k}{m} \cdot e_k.
    \end{align*}
    Adding up those three identities yields (\ref{firstident}). 

    To finally prove \eqref{ident}, it clearly suffices to show that 
    \begin{equation*}
        e_\kk * V_m^s = e_\kk
    \end{equation*}
    for all $\kk \in \Z^s$ with $-m \leq \kk \leq m$. But for such $\kk$, using $e_\kk(x) = \prod_{j=1}^{s} e_{\kk_j}\left( x_j\right)$, one obtains
    \begin{align*}
        \left(e_\kk * V_m^s\right)(x) &= \frac{1}{(2\pi)^s} \int_{[-\pi, \pi]^s} \prod_{j=1}^{s} e_{\kk_j} \left(t_j\right) \cdot V_m \left(x_j - t_j\right) dt \\
        \overset{\text{Fubini}}&{=} \prod_{j=1}^s \left(e_{\kk_j} * V_m\right)\left(x_j \right) 
        \overset{\eqref{firstident}}{=} \prod_{j=1}^s e_{\kk_j} \left(x_j\right) 
        = e_\kk(x)
    \end{align*}
    for any $ x \in \RR^s$, as was to be shown.
\end{proof}
The following result follows from a theorem in \cite{lorentz_approximation_2005}.
\begin{proposition}
\label{lorentz_aussage}
    Let $s,k \in \NN$. Then there exists a constant $c = c(s,k) > 0$, such that, for $E_m^s$ as defined in \eqref{eq:mintrigo},
    \begin{equation*}
        E_m^s(f) \leq \frac{c}{m^k} \cdot \left\Vert f\right\Vert_{C^k\left([-\pi, \pi]^s; \RR\right)}
    \end{equation*}
    for all $m\in \NN$ and $f \in C^k_{2\pi} \left(\RR^s; \RR\right)$. \newline
    In fact, it holds $c(s,k) \leq \exp(C \cdot ks) \cdot k^k $ with an absolute constant $C>0$.
\end{proposition}
\begin{proof}
    We apply \cite[Theorem 6.6]{lorentz_approximation_2005} with $n_i = m$ and $p_i = k$, which yields the existence of a constant $c_1 = c_1(s,k)>0$, such that
    \begin{equation*}
        E_m^s(f) \leq c_1 \cdot \sum_{\ell=1}^s \frac{1}{m^k} \cdot \omega_{\ell} \left(f,\frac{1}{m}\right)
    \end{equation*}
    for all $m \in \NN$ and $f \in C^k_{2\pi}(\RR^s ; \RR)$, where $\omega_\ell(f, \bullet)$ denotes the modulus of continuity of $\frac{\partial^k f }{\partial x_\ell ^k}$ with respect to $x_\ell$, where we have the trivial bound
    \begin{equation*}
         \omega_\ell \left(f,\frac{1}{m}\right)  \leq 2 \cdot \left\Vert f \right\Vert_{C^k\left([-\pi, \pi]^s; \RR\right)}.
    \end{equation*}
    Hence, we get
    \begin{equation*}
        E_m^s(f) \leq c_1 \cdot s \cdot 2 \cdot \left\Vert f \right\Vert_{C^k\left([-\pi, \pi]^s; \RR\right)} \frac{1}{m^k},
    \end{equation*}
    so the claim follows by choosing $c := 2s \cdot c_1$.
    
    We refer to \Cref{sec:const_bound_reordered} (see \Cref{thm:const_lorentz_bound}) for a proof of the claimed bound on the constant $c(s,k)$.
\end{proof}
The above proposition bounds the best possible error of approximating $f$ by trigonometric polynomials of coordinatewise degree at most $m$, but this is in general non-constructive. Our next result shows that a similar bound holds for approximating $f$ by $v_m(f)$.
\begin{theorem}
\label{vm_approx}
    Let $s \in \NN$. Then there exists a constant $ c = c(s) > 0 $, such that the operator $v_m$ from \Cref{vm} satisfies
    \begin{equation*}
        \left\Vert f - v_m(f) \right\Vert_{L^\infty \left( \RR^s\right)} \leq c \cdot E^s_m(f)
    \end{equation*}
    for any $m \in \NN$ and $f \in C_{2\pi} \left(\RR^s; \CC\right)$. \newline
    In fact, it holds $c(s) \leq \exp(C \cdot s)$ with an absolute constant $C>0$.
\end{theorem}
\begin{proof}
    For any $T \in \mathbb{H}_m^s$ one has
    \begin{equation*}
        \left\Vert f - v_m(f) \right\Vert_{L^\infty \left( \RR^s\right)} \overset{\eqref{ident}}{\leq} \left\Vert f - T \right\Vert_{L^\infty \left( \RR^s\right)} + \left\Vert v_m(T) - v_m(f) \right\Vert_{L^\infty \left( \RR^s\right)} \overset{\eqref{constant}}{\leq} (c+1) \left\Vert f - T \right\Vert_{L^\infty \left( \RR^s\right)}. 
    \end{equation*}
    Taking the infimum over all $T \in \mathbb{H}_m^s$ yields the claim. 
\end{proof}
By combining \Cref{lorentz_aussage} and \Cref{vm_approx}, we immediately get the following bound.
\begin{corollary}
\label{dlvp}
    Let $s,k \in \NN_0$. Then there exists a constant $c = c(s,k)>0$, such that
    \begin{equation*}
        \left\Vert f - v_m(f) \right\Vert_{L^\infty \left(\RR^s\right)} \leq \frac{c}{m^k} \cdot \left\Vert f \right\Vert_{C^k \left([-\pi,\pi]^s; \RR\right)}
    \end{equation*}
    for every $m \in \NN$ and $f \in C^k_{2\pi} \left(\RR^s ; \RR\right)$. \newline
    In fact, we have $c(s,k) \leq \exp(C \cdot ks) \cdot k^k$ with an absolute constant $C>0$.
\end{corollary}
Up to now, we have studied the approximation of periodic functions by trigonometric polynomials, but our actual goal is to approximate non-periodic functions by algebraic polynomials. The next lemma establishes a connection between the two settings. 
\begin{lemma}
\label{star_operator}
    Let $k \in \NN_0$ and $ s \in \NN$. For any function $f \in C^k \left([-1,1]^s ; \CC\right)$, we define the corresponding periodic function via
    \begin{equation*}
        f^* : \quad \RR^s \to \CC, \quad  f^* \left(x_1, ..., x_s\right) = f(\mathrm{cos} \left(x_1\right), ..., \mathrm{cos} \left( x_s\right))
    \end{equation*}
    and note $f^* \in C_{2\pi}^k \left(\RR^s; \CC\right)$. The map
\begin{equation*}
	C^k \left([-1,1]^s ; \CC\right) \to C^k _ {2\pi}\left(\RR^s; \CC\right), \quad f \mapsto f^*
\end{equation*}
is a continuous linear operator with respect to the $C^k$-norms on $C^k \left([-1,1]^s ; \CC\right)$ and $C^k _ {2\pi}\left(\RR^s; \CC\right)$. \newline
The operator norm can be bounded from above by $k^k$.
\end{lemma}
\begin{proof}
	The map is well-defined since $\cos$ is a smooth function and $2\pi$-periodic. The linearity of the operator is obvious, so it remains to show its continuity.

	The goal is to apply the closed graph theorem \cite[Theorem 5.12]{folland_real_1999}. By definition of $f^*$, and since $\cos:[-\pi,\pi] \to [-1,1]$ is surjective, we have the equality $\left\Vert f\right\Vert_{L^\infty \left([-1, 1]^s; \CC\right)} = \left\Vert f^* \right\Vert_{L^\infty \left([-\pi, \pi]^s; \CC\right)} $. Let then $\left(f_n\right)_{n \in \NN}$ be a sequence of functions $f_n \in C^k \left([-1,1]^s ; \CC\right)$ and $g^* \in C^k_{2\pi}\left(\RR^s ; \CC\right)$ such that $f_n \to f$ in $C^k \left([-1,1]^s; \CC\right)$ and $f_n^* \to g^*$ in $C^k_{2\pi} \left(\RR^s; \CC\right)$. We then have
\begin{align*}
	\left\Vert f^* - g^* \right\Vert_{L^\infty \left([-\pi, \pi]^s\right)} &\leq \left\Vert f^* - f_n^* \right\Vert_{L^\infty \left([-\pi, \pi]^s\right)} + \left\Vert f_n^* - g^* \right\Vert_{L^\infty \left([-\pi, \pi]^s\right)} \\
&= \left\Vert f - f_n \right\Vert_{L^\infty  \left([-1, 1]^s ; \CC\right)} + \left\Vert f_n^* - g^* \right\Vert_{L^\infty \left([-\pi, \pi]^s\right)} \\
&\leq \left\Vert f - f_n \right\Vert_{C^k  \left([-1, 1]^s ; \CC\right)} + \left\Vert f_n^* - g^* \right\Vert_{C^k([-\pi, \pi]^s ; \CC)} \to 0 \ (n \to \infty).
\end{align*}
It follows $f^* = g^*$ and the closed graph theorem yields the desired continuity. 

We refer to \Cref{sec:const_bound_reordered} (see \Cref{thm:faa,rem:multiindex}) for a proof of the claimed bound on the operator norm.
\end{proof}

For a function $f \in C^k([-1,1]^s;\CC)$ we want to express $v_m(f^*)$ in a convenient way, involving a product of cosines. To this end, we make use of the following identity, which is a generalization of the well-known product-to-sum formula for $\cos$.
\begin{lemma}
\label{prod_sum}
Let $s \in \NN$. Then it holds for any $x \in \RR^s$ that
\begin{equation*}
\prod_{j=1}^s \cos (x_j) = \frac{1}{2^s} \sum_{\sigma \in \{-1,1\}^s} \cos (\langle\sigma,x \rangle).
\end{equation*} 
\end{lemma}
\begin{proof}
This is an inductive generalization of the product-to-sum formula 
\begin{equation}
\label{product_sum_form}
2\cos(x) \cos(y) = \cos(x-y) + \cos (x+y)
\end{equation}
for $x,y \in \RR$, which can be found for instance in \cite[Eq.~4.3.32]{abramowitz_handbook_2013}. The case $s= 1$ holds since $\cos$ is an even function. Assume that the claim holds for a fixed $s \in \NN$ and take $x \in \RR^{s+1}$. Writing $x' = (x_1, ..., x_s)$, we derive
\begin{align*}
\prod_{j=1}^{s+1} \cos(x_j) &= \left(\frac{1}{2^s} \sum_{\sigma \in \{-1,1\}^s} \cos (\langle \sigma , x' \rangle) \right) \cdot \cos(x_{s+1}) \\
&= \frac{1}{2^s} \sum_{\sigma \in \{-1,1\}^s}\cos (\langle \sigma, x' \rangle) \cos(x_{s+1})\\
\overset{\eqref{product_sum_form}}&{=} \frac{1}{2^{s+1}} \sum_{\sigma \in \{-1,1\}^s} \left[\cos (\langle \sigma, x' \rangle + x_{s+1}) + \cos (\langle \sigma, x' \rangle - x_{s+1}) \right]  \\
&= \frac{1}{2^{s+1}} \sum_{\sigma \in \{-1,1\}^{s+1}} \cos (\langle \sigma, x\rangle), 
\end{align*}
as was to be shown. 
\end{proof}
The following proposition states that $v_m(f^*)$ can be expressed as a linear combination of products of cosines. This representation is useful since these cosines can be interpolated by Chebyshev polynomials which in the end leads to the desired approximation result.
\let \hat \widehat
\begin{proposition}
\label{representation}
    Let $s\in \NN$ and $k \in \NN_0$. For any $f \in C^k \left([-1,1]^s ; \CC\right)$ and $m \in \NN$ the de-la-Vallée-Poussin operator given as $f \mapsto v_m \left(f^*\right)$ with $v_m$ as in \Cref{vm} and $f \mapsto f^*$ as in \Cref{star_operator} has a representation
    \begin{equation*}
        v_m \left(f^*\right) \left(x_1, ..., x_s\right) = \underset{\kk \leq 2m -1}{\sum_{\kk \in \NN_0^s}} \mathcal{V}_\kk^m(f) \prod_{j=1}^{s} \mathrm{cos} \left(\kk_j x_j\right)
    \end{equation*}
    for continuous linear functionals
    \begin{equation*}
        \mathcal{V}_{\kk}^m : \ C^k \left([-1,1]^s; \CC\right) \to \CC, \quad f \mapsto 2^{\Vert \kk \Vert_0}\cdot  a_{\kk}^m \cdot \widehat{f^*}(\kk),
    \end{equation*}
where $\Vert \kk \Vert_0 = \# \{j \in \{1,...,s\}: \ \kk_j \neq 0\}$ and $a_{\kk}^m = \widehat{V_m^s}(\kk)$. Furthermore, if $f \in C^k ([-1,1]^s ; \RR)$, then $\mathcal{V}_\kk^m (f) \in \RR$ for every $\kk \in \NN_0^s$ with $\kk \leq 2m-1$.
\end{proposition}
\begin{proof}
    First of all, it is easy to see that $v_m \left( f^* \right)$ is even in each variable, which follows directly from the fact that $ f^*$ and $V_m^s$ are both even in each variable. Furthermore, if we write 
    \begin{equation*}
        V_m^s = \underset{-(2m-1) \leq \kk \leq 2m-1}{\sum_{\kk \in \ZZ^s}} a_\kk^m e_\kk
    \end{equation*}
    with appropriately chosen coefficients $a_\kk^m \in \RR$, we easily see
    \begin{equation*}
        v_m \left(f^*\right) = \underset{-(2m-1) \leq \kk \leq 2m-1}{\sum_{\kk \in \ZZ^s}} a_\kk^m \hat{f^*}(\kk) e_\kk.
    \end{equation*}
    Using Euler's identity and the fact that $v_m\left(f^*\right)$ is an even function, we get the representation 
    \begin{equation*}
        v_m \left(f^*\right)(x) = \underset{-(2m-1)\leq \kk \leq 2m-1}{\sum_{\kk \in \ZZ^s}} a_\kk^m \hat{f^*}(\kk) \text{cos}(\langle \kk, x\rangle)
    \end{equation*}
    for all $x \in \RR^s$. Using $\odot$ to denote the componentwise product of two vectors of the same size, i.e., $x \odot y = (x_i \cdot y_i)_i$, and using the identity $\langle \kk, \sigma \odot x\rangle = \langle \sigma, \kk \odot x \rangle$, we see since $v_m \left(f^*\right)$ is even in every variable that
    \begin{align*}
        v_m \left(f^*\right) (x) &= \frac{1}{2^s} \cdot \sum_{\sigma \in \{-1,1\}^s} v_m \left( f^* \right) (\sigma \odot x) \\
        &=\frac{1}{2^s} \cdot \sum_{\sigma \in \{-1,1\}^s} \underset{-(2m-1) \leq \kk \leq 2m-1}{\sum_{\kk \in \ZZ^s}} a_\kk^m \hat{f^*}(\kk) \text{cos}(\langle \kk, \sigma \odot x\rangle) \\
        &= \underset{-(2m-1) \leq \kk \leq 2m-1}{\sum_{\kk \in \ZZ^s}} \left(a_\kk^m \hat{f^*}(\kk) \frac{1}{2^s} \sum_{\sigma \in \{-1,1\}^s}\text{cos}(\langle \sigma, \kk \odot x\rangle)\right) \\
        \overset{\text{\Cref{prod_sum}}}&{=} \underset{-(2m-1) \leq \kk \leq 2m-1}{\sum_{\kk \in \ZZ^s}} a_\kk^m \hat{f^*}(\kk) \prod_{j=1}^{s} \cos \left(\kk_j x_j\right) \\
        &= \underset{\kk \leq 2m-1}{\sum_{\kk \in \NN_0^s}} 2^{\Vert \kk \Vert_0} a_\kk^m \hat{f^*}(\kk) \prod_{j=1}^{s} \cos \left(\kk_j x_j\right)
    \end{align*}
    with
    \begin{equation*}
        \left\Vert \kk \right\Vert_0 \defeq \#\big\{ j \in \{1,...,s\} : \ \kk_j \neq 0\big\}.
    \end{equation*}
    In the last step we again used that cos is an even function and that
    \begin{equation*}
        \hat{f^*}(\kk) = \hat{f^*}(\sigma \odot \kk)
    \end{equation*}
    for all $\sigma \in \{-1,1\}^s$, which also follows easily since $f^*$ and $V_m^s$ are even in every component. Letting \begin{equation*}
        \mathcal{V}_\kk^m(f) \defeq 2^{\Vert \kk \Vert_0} a_\kk^m \hat{f^*}(\kk),
    \end{equation*} 
    we have the desired form. The fact that $\mathcal{V}_{\kk}^m$ is a continuous linear functional on $C^k_{2\pi}\left([-1,1]^s;\CC\right)$ follows directly since $f \mapsto \widehat{f^*}(\kk)$ is a continuous linear functional for every $\kk$. If $f$ is real-valued, so is $\hat{f^*}(\kk)$ for every $\kk \in \NN_0^s$ with $\kk \leq 2m-1$, since $f^*$ is real-valued and even in every component.
\end{proof}
\let \hat \hat

Our main approximation result involves linear combinations of Chebyshev polynomials where the coefficients in this linear combination are given as $\mathcal{V}_{\kk}^m(f)$. It is therefore important to be able to bound the sum of the absolute values $\vert \mathcal{V}_{\kk}^m(f) \vert$.
\begin{lemma}
\label{sum_bound}
    Let $s \in \NN$. There exists a constant $c = c(s)> 0$, such that the inequality
    \begin{equation*}
        \underset{\kk \leq 2m-1}{\sum_{\kk \in \NN_0^s}} \left\vert \mathcal{V}_\kk^m(f) \right\vert \leq c \cdot m^{s/2} \cdot \left\Vert f \right\Vert_{L^\infty \left([-1, 1]^s ; \CC\right)}
    \end{equation*}
    holds for all $m \in \NN$ and $f \in C \left([-1, 1]^s ; \CC\right)$, where $\mathcal{V}_\kk^m$ is as in \Cref{representation}. \newline
    In fact, we have $c(s) \leq \exp(C \cdot s)$ with an absolute constant $C>0$.
\end{lemma}
\begin{proof}
    Let $f \in C \left([-1, 1]^s ; \CC\right)$ and $m \in \NN$. For any multi-index $\elll \in \NN_0^s$, it follows from \Cref{representation} that 
    \begin{equation*}
        \widehat{v_m\left(f^*\right)} (\elll) = \underset{\kk \leq 2m-1}{\sum_{\kk \in \NN_0^s}} \mathcal{V}_\kk^m(f) \widehat{g_\kk}(\elll),
    \end{equation*}
    with
    \begin{equation*}
        g_\kk : \quad \RR^s \to \RR, \quad \left(x_1, ..., x_s\right) \mapsto \prod_{j=1}^s \cos \left(\kk_j x_j\right).
    \end{equation*}
    Now, a calculation using Fubini's theorem and using 
    \begin{equation*}
    g_{\kk} = \prod_{j=1}^s\frac{1}{2} \left( e_{\kk_j} + e_{-\kk_j}\right) = \underset{\kk_j \neq 0}{\prod_{1 \leq j \leq s}} \frac{1}{2} \left(e_{\kk_j} + e_{-\kk_j}\right)
    \end{equation*} for any number $k \in \NN_0$ shows 
    \begin{equation*}
        \widehat{g_\kk}(\elll) =  \begin{cases} \frac{1}{2^{\Vert \kk \Vert_0}},&\kk=\elll, \\ 0, & \text{otherwise}\end{cases} \quad\text{for } \kk, \elll \in \NN_0^s.
    \end{equation*}
    Therefore, we have the bound $\left\vert\mathcal{V}^m_{\elll} (f)\right\vert \leq 2^s \cdot \left\vert \widehat{v_m\left(f^*\right)}(\elll)\right\vert$ for $\elll \in \NN_0^s $ with $\vert \elll \vert \leq 2m-1$. Using the Cauchy-Schwarz and the Parseval inequality, we therefore see
    \begin{align*}
        \underset{\kk \leq 2m-1}{\sum_{\kk \in \NN_0^s}} \left\vert \mathcal{V}_\kk^m(f)\right\vert &\leq 2^s \cdot \underset{\kk \leq 2m-1}{\sum_{\kk \in \NN_0^s}} \left\vert \widehat{v_m\left(f^*\right)}(\kk)\right\vert \overset{\text{CS}}{\leq} 2^s \cdot (2m)^{s/2} \cdot \left(\underset{\kk \leq 2m-1}{\sum_{\kk \in \NN_0^s}} \left\vert \widehat{v_m\left(f^*\right)}(\kk)\right\vert^2 \right)^{1/2} \\
        \overset{\text{Parseval}}&{\leq} 2^s \cdot 2^{s/2} \cdot m^{s/2} \cdot \left\Vert v_m \left(f^*\right)\right\Vert_{L^2 \left([-\pi, \pi]^s; \CC\right)} \\
        &\leq \underbrace{2^s \cdot 2^{s/2} }_{=: c_1(s)}\cdot m^{s/2} \cdot \left\Vert v_m \left(f^*\right)\right\Vert_{L^\infty \left([-\pi, \pi]^s; \CC\right)}.
    \end{align*}
    Using \Cref{vm}, we get a constant $c_2(s) \leq \exp(C_0 \cdot s)$ such that
    \begin{align*}
        \underset{\kk \leq 2m-1}{\sum_{\kk \in \NN_0^s}} \left\vert \mathcal{V}_\kk^m(f)\right\vert &\leq c_1(s) \cdot c_2(s) \cdot m^{s/2} \cdot \left\Vert f^*\right\Vert_{L^\infty \left([-\pi, \pi]^s;\CC\right)} = c(s) \cdot m^{s/2} \cdot \left\Vert f\right\Vert_{L^\infty \left([-1, 1]^s; \CC\right)},
    \end{align*}
    as claimed.
\end{proof}

For any natural number $\ell \in \NN_0$, we denote by $T_\ell$ the $\ell$-th \emph{Chebyshev polynomial}, satisfying
\begin{equation*}
    T_\ell\left(\cos(x)\right) = \cos(\ell x), \quad x \in \RR.
\end{equation*}
One can show that $T_\ell$ is in fact a polynomial of degree $\ell$. For a multi-index $\kk \in \NN_0^s$, we define
\begin{equation*}
    T_\kk (x) \defeq \prod_{j=1}^s T_{\kk_j}\left(x_j\right), \quad x \in \RR^s.
\end{equation*}
We then get the following approximation result about approximating (non-periodic) $C^k$-functions by linear combinations of Chebyshev polynomials. 
\begin{theorem} \label{app: fourier_approx}
    Let $k,s,m \in \NN$. Then there exists a constant $c=c(s,k)>0$ with the following property: For any $f \in C^k \left([-1,1]^s; \RR\right)$ the polynomial $P$ defined as
    \begin{equation*}
        P(x) \defeq \underset{\kk \leq 2m-1}{\sum_{\kk \in \NN_0^s}}\mathcal{V}_\kk^m(f) \cdot T_\kk(x),
    \end{equation*}
    with $\mathcal{V}^m_\kk$ as in \Cref{representation}, satisfies
    \begin{equation*}
        \left\Vert f - P \right\Vert_{L^\infty \left([-1,1]^s ;\RR\right)} \leq \frac{c}{m^k} \cdot \left\Vert f \right\Vert_{C^k\left([-1,1]^s;\RR\right)}.
    \end{equation*}
    Here, the maps
\begin{equation*}
C\left([-1,1]^s ; \RR\right) \to \RR, \quad f \mapsto \mathcal{V}_{\kk}^m(f)
\end{equation*}
are continuous and linear functionals with respect to the $L^\infty$-norm. Furthermore, there exists a constant $\tilde{c} = \tilde{c}(s)> 0$, such that the inequality
    \begin{equation*}
        \underset{\kk \leq 2m-1}{\sum_{\kk \in \NN_0^s}} \left\vert \mathcal{V}_\kk^m(f) \right\vert \leq \tilde{c} \cdot m^{s/2} \cdot \left\Vert f \right\Vert_{L^\infty \left([-1, 1]^s;\RR\right)}
    \end{equation*}
    holds for all $f \in C \left([-1, 1]^s ; \RR\right)$. \newline
    Moreover, we have $c(s,k) \leq \exp(C \cdot ks) \cdot k^{2k}$ and $\tilde{c}(s) \leq \exp(C \cdot s)$ with an absolute constant $C>0$.
    
\end{theorem}
\begin{proof}
    We choose the constant $c_0 = c_0(s,k)$ according to \Cref{dlvp}. Let $f \in C^k\left([-1,1]^s;\RR\right)$ be arbitrary. Then we define the corresponding function $f^* \in C^k_{2\pi}\left(\RR^s; \RR\right)$ as above. Let $P$ be defined as in the statement of the theorem. Then it follows from the definition of the Chebyshev polynomials $T_\kk$, the definition of $P$, and the formula for $v_m(f^*)$ from \Cref{representation} that 
    \begin{equation*}
        P^* (x) = v_m\left(f^*\right)(x)
    \end{equation*}
    is satisfied, where $P^*$ is the corresponding function to $P$ defined similarly to $f^*$. Overall, we get the bound
    \begin{align*}
        \left\Vert f - P \right\Vert_{L^\infty \left([-1,1]^s; \RR\right)} = \left\Vert f^* - P^* \right\Vert_{L^\infty \left([-\pi,\pi]^s; \RR\right)} \overset{\text{Cor. \ref{dlvp}}}{\leq} \frac{c_0}{m^k} \cdot \left\Vert f^* \right\Vert_{C^k\left([-\pi, \pi]^s; \RR\right)}.
    \end{align*}
   The first claim then follows using the continuity of the map $f \mapsto f^*$ as proven in \Cref{star_operator}. The second part of the theorem has already been proven in \Cref{sum_bound}.
\end{proof}