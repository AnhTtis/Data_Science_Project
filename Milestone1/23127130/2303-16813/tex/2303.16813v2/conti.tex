
\section{Postponed proofs for the optimality results in the case of continuous weight selection}
In this section we provide the proofs for the optimality results derived if a continuous weight selection is assumed.
Specifically, we prove \Cref{thm:intrac,thm:opti_conti}. 
The proofs of both results rely on a very general result about the approximation in normed spaces by subsets that are parametrizable with $m$ parameters \cite{devore_optimal_1989}. 
We decided to include a detailed proof for this approximation result in this paper since the nature of the continuity assumption in \cite{devore_optimal_1989} is not completely clear. 

\begin{proposition}[{\cite[Theorem 3.1]{devore_optimal_1989}}] \label{thm: devore}
  Let $(X, \Vert \cdot \Vert_X)$ be a normed space,
  $\emptyset \neq K \subseteq X$ a subset and $V \subseteq X$ a linear,
  not necessarily closed subspace of $X$ containing $K$.
  Let $m\in \NN$, let $\overline{a} : K \to \RR^m$ be a map which is continuous
  with respect to some norm $\Vert \cdot \Vert_V$ on $V$ and $M : \RR^m \to X$ some arbitrary map.
  Let
  \begin{equation} \label{eq:bmkx}
    b_m(K)_X
    \defeq \underset{X_{m+1}}{\sup}
             \sup
               \left\{\varrho \geq 0: \  U_\varrho(X_{m+1}) \subseteq K\right\},
  \end{equation} 
  where the first supremum is taken over all $(m+1)$-dimensional linear subspaces $X_{m+1}$ of $X$ and
  \begin{equation*}
    U_\varrho(X_{m+1})
    \defeq \{y \in X_{m+1} : \ \Vert y \Vert_X \leq \varrho\}.
  \end{equation*}
  Further, we set $b_m(K)_X \defeq 0$ if the supremum in \eqref{eq:bmkx}
  is not well-defined as a quantity in $[0, \infty]$.
  Then it holds
  \begin{equation*}
    \underset{x \in K}{\sup} \Vert x - M (\overline{a}(x)) \Vert_X \geq b_m(K)_X.
  \end{equation*}
\end{proposition}

\begin{proof}
The claim is trivial if $b_m(K)_X = 0$.
Thus, assume $b_m(K)_X > 0$.
Let $0 < \varrho \leq b_m(K)_X$ be any number such that there exists
an $(m+1)$-dimensional subspace $X_{m+1}$ of $X$ with $ U_\varrho(X_{m+1}) \subseteq K$.
It follows $U_\varrho(X_{m+1}) \subseteq V$, hence $X_{m+1} \subseteq V$,
so $\Vert \cdot \Vert_V$ defines a norm on $X_{m+1}$.
Thus, the restriction of $\overline{a}$ to $\partial  U_\varrho(X_{m+1})$ is a continuous mapping
to $\RR^m$ with respect to $\Vert \cdot \Vert_V$.
Since all norms are equivalent on the finite-dimensional space $X_{m+1}$,
the Borsuk-Ulam-Theorem \cite[Corollary 4.2]{deimling2013nonlinear}
yields the existence of a point $x_0 \in \partial U_\varrho(X_{m+1})$
with $\overline{a}(x_0) = \overline{a}(-x_0)$.
We then see
\begin{align*}
  2\varrho
  &=    2 \Vert x_0 \Vert_X
   \leq \Vert x_0 - M(\overline{a}(x_0)) \Vert_X
        + \Vert x_0 + M(\overline{a}(-x_0)) \Vert_X \\
  &=    \Vert x_0 - M (\overline{a}(x_0)) \Vert_X
        + \Vert - x_0 - M(\overline{a}(-x_0)) \Vert_X,
\end{align*}
and hence at least one of the two summands on the right has to be larger than or equal to $\varrho$.
\end{proof}
\subsection{Proof of Theorem \ref{thm:opti_conti}}
\label{optimality_section_reordered}

Using \Cref{thm: devore}, we can deduce our lower bound in the context of $C^k$-spaces.
The proof is in fact almost identical to what is done in \cite[Theorem~4.2]{devore_optimal_1989}.
However, we decided to include a detailed proof in this paper,
since \cite{devore_optimal_1989} considers Sobolev functions and not $C^k$-functions.


\begin{theorem} \label{app: devore_real}
Let $s,k \in \NN$. Then there exists a constant $c = c(s,k)>0$ with the following property:  For any $m \in \NN$ and any map $\overline{a} : C^k ([-1,1]^s; \RR) \to \RR^m$ that is continuous with respect to some norm on $C^k([-1,1]^s; \RR)$ and any (possibly discontinuous) map $M  : \RR^m \to C([-1,1]^s ; \RR)$, we have
\begin{equation*}
\underset{\Vert f \Vert _{C^k ([-1,1]^s ; \RR)} \leq 1}{\underset{f \in C^k([-1,1]^s ; \RR)}{\sup}} \Vert f - M (\overline{a}(f)) \Vert_{L^\infty ([-1,1]^s ; \RR)} \geq c \cdot m^{-k/s}.
\end{equation*}
\end{theorem}
\begin{proof}
The idea is to apply \Cref{thm: devore} to $X \defeq C([-1,1]^s ; \RR)$, $V \defeq C^k ([-1,1]^s ; \RR)$ and the set $K \defeq \{f \in C^k ([-1,1]^s ; \RR): \  \Vert f \Vert_{C^k ([-1,1]^s ; \RR)} \leq 1\}$. 

Assume in the beginning that $m = n^s$ with an integer $n >1$. Pick $\phi \in C^\infty(\RR^s)$ with $\phi \equiv 1$ on $[-3/4, 3/4]^s$ and $\phi \equiv 0$ outside of $[-1,1]^s$. Fix $c_0 = c_0(s,k) > 0$ with 
\begin{equation*}
1 \leq \Vert \phi \Vert_{C^k([-1,1]^s ; \RR) } \leq c_0.
\end{equation*} 
Let $Q_1, ..., Q_m$ be the partition (disjoint up to null-sets) of $[-1,1]^s$ into closed cubes of sidelength $2/n$. For every $j \in \{1,...,m\}$ we write $Q_j = \bigtimes_{\ell = 1}^{s} [a_\ell^{(j)} - 1/n, a_\ell^{(j)} + 1/n]$ with an appropriately chosen vector $a = (a_1^{(j)}, ..., a_s^{(j)}) \in [-1,1]^s$ and let 
\begin{equation*}
\phi_j (x) \defeq \phi (n(x - a^{(j)})) \text{ for } x \in \RR^s.
\end{equation*}
By choice of $\phi$, the maps $\phi_j$ are supported on a proper subset of $Q_j$ for every $j \in \{1,...,m\}$ and an inductive argument shows
\begin{equation*} 
\partial^\kk \phi_j (x) = n^{\vert \kk \vert} \cdot (\partial^\kk \phi) (n(x - a^{(j)})) \quad\text{for every } \kk \in \NN_0^s \text{ and }x \in \RR^s
\end{equation*}
and hence in particular
\begin{equation} \label{eq: derivative}
\Vert \phi_j \Vert_{C^k([-1,1]^s ; \RR)} \leq n^{\vert \kk \vert} \cdot c_0.
\end{equation}
Let $X_m \defeq \spann \{\phi_1, ..., \phi_m\}$ and $S \in U_1(X_m) = \{f \in X_m: \ \Vert f \Vert_{L^\infty([-1,1]^s;\RR)}\leq 1\} $. Then we can write $S$ in the form $S= \sum_{j = 1}^m c_j \phi_j$ with real numbers $c_1, ..., c_m \in \RR$. Suppose there exists $j^* \in \{1,...,m\}$ with $\vert c_{j^*}\vert  > 1$. Then we have
\begin{equation*}
\Vert S \Vert_{L^\infty([-1,1]^s ; \RR)} \geq \vert S(a^{(j^*)})\vert \geq \vert c_{j^*} \vert> 1,
\end{equation*}
since the functions $\phi_j$ have disjoint support and $\phi_j(a^{(j)}) = 1$. This is a contradiction to $S \in U_1(X_m)$ and we can thus infer that $\underset{j}{\max} \ \vert c_j \vert \leq 1$. Furthermore, we see again because the functions $\phi_j$ have disjoint support that
\begin{align*}
\Vert \partial^\kk S \Vert_{L^\infty ([-1,1]^s ; \RR)} \leq \underset{j}{\max} \ \vert c_j \vert \cdot \Vert \partial^\kk \phi_j \Vert_{L^\infty([-1,1]^s ; \RR)} \overset{\eqref{eq: derivative}}{\leq} n^{\vert \kk \vert} \cdot c_0 \leq c_0 \cdot n^k = c_0 \cdot m^{k/s}
\end{align*}
for every $\kk \in \NN_0^s$ with $\vert \kk \vert \leq k$ and hence
\begin{equation*}
\Vert S \Vert_{C^k ([-1,1]^s ; \RR)} \leq c_0 \cdot m^{k/s}.
\end{equation*}
Thus, letting $\varrho \defeq c_0^{-1} \cdot m^{-k/s}$ yields $ U_\varrho(X_m) \subseteq K$, so we see by \Cref{thm: devore} that
\begin{equation*}
\underset{f \in K}{\sup} \Vert f - M_{m-1} (\overline{a}(f)) \Vert_{L^\infty([-1,1]^s ; \RR)} \geq \varrho = c_1 \cdot m^{-k/s}
\end{equation*} 
with $c_1 = c_0^{-1}$ for every map $\overline{a} : X \to \RR^{m-1}$ which is continuous with respect to some norm on $V$ and any map $M_{m-1}: \RR^{m-1} \to X$. Using the inequality $m \leq 2(m-1)$ (note $m>1$) we get
\begin{align*}
\underset{f \in K}{\sup} \Vert f - M_{m-1} (\overline{a}(f)) \Vert_{L^\infty([-1,1]^s ; \RR)} \geq c_1 \cdot m^{-k/s} \geq c_1 \cdot (2(m-1))^{-k/s} \geq c_2 \cdot (m-1)^{-k/s}
\end{align*}
with $c_2 = c_1 \cdot 2^{-k/s}$. Hence, the claim has been shown for all numbers $m$ of the form $n^s - 1$ with an integer $n >1$.

In the end, let $m \in \NN$ be arbitrary and pick $n \in \NN$ with $n^s \leq m < (n+1)^s$. For given maps $\overline{a} : V \to \RR^m$ and $M: \RR^m \to X$ with $\overline{a}$ continuous with respect to some norm on $V$, let
\begin{align*}
\tilde{a}:& \quad V \to \RR^{(n+1)^s-1}, \quad f \mapsto (\overline{a}(f), 0) \quad \text{and}\\ 
 M_{(n+1)^s-1}: &\quad \RR^{(n+1)^s-1}\to X, \quad (x,y) \mapsto M(x),
\end{align*}
where $x \in \RR^m, \  y \in \RR^{(n+1)^s -1 -m}$. Then we get
\begin{align*}
\underset{f \in K}{\sup} \Vert f - M (\overline{a}(f)) \Vert_{L^\infty([-1,1]^s ; \RR)} &= \underset{f \in K}{\sup} \Vert f - M_{(n+1)^s - 1} (\tilde{a}(f)) \Vert_{L^\infty([-1,1]^s ; \RR)} 
\\
&\geq c_2 \cdot ((n+1)^s - 1)^{-k/s} 
\geq c_2 \cdot (2^s n^s)^{-k/s} 
\geq c_3 \cdot m^{-k/s}
\end{align*}
with $c_3 = c_2 \cdot 2^{-k}$. Here we used the bound $(n+1)^s - 1 \leq (2n)^s$. This proves the full claim. \qedhere
\end{proof}
Using this theorem, we can now prove \Cref{thm:opti_conti}.

\begin{proof}[Proof of \Cref{thm:opti_conti}]
Let $\overline{a}: C^k(\Omega_n ; \CC) \to \CC^m$ be any map that is continuous with respect to some norm $\Vert \cdot \Vert_V$ on $C^k(\Omega_n; \CC)$, and let $M: \CC^m \to C(\Omega_n;\CC)$ be arbitrary. With $\varphi_n, \varphi_m$ defined as in Equation \eqref{isomorphism_intro}, let
\begin{equation*}
\tilde{a}: \quad C^k ([-1,1]^{2n}; \RR) \to \RR^{2m}, \quad \tilde{f} \mapsto \varphi_m^{-1} \left( \overline{a} \left( \tilde{f} \circ \fres{\varphi_n^{-1}}{\Omega_n}\right)\right).
\end{equation*}
Clearly, $\tilde{a}$ is continuous on $C^k([-1,1]^{2n}; \RR)$ with respect to the norm $\Vert \cdot \Vert_{\tilde{V}}$ on $C^k([-1,1]^{2n}; \RR)$ defined as
\begin{equation*}
\Vert \tilde{f} \Vert_{\tilde{V}} \defeq \left\Vert \tilde{f} \circ \fres{\varphi_n^{-1}}{\Omega_n}\right\Vert_V \quad \text{for } \tilde{f} \in C^k([-1,1]^{2n}; \RR).
\end{equation*}
Let
\begin{equation*}
\widetilde{M}: \quad \RR^{2m} \to C([-1,1]^{2n}; \RR), \quad \widetilde{M}(x) \defeq \RE(M(\varphi_m(x))) \circ \fres{\varphi_n}{[-1,1]^{2n}}.
\end{equation*}
Then it holds
\begin{align*}
&\norel \underset{\Vert f \Vert _{C^k (\Omega_n ; \CC)} \leq 1}{\underset{f \in C^k(\Omega_n ; \CC)}{\sup}} \Vert f - M (\overline{a}(f)) \Vert_{L^\infty (\Omega_n; \CC)} \\
&\geq \underset{\Vert f \Vert _{C^k (\Omega_n ; \RR)} \leq 1}{\underset{f \in C^k(\Omega_n ; \RR)}{\sup}} \Vert f - \RE(M (\overline{a}(f))) \Vert_{L^\infty (\Omega_n; \RR)} \\
&= \underset{\Vert \tilde{f} \Vert _{C^k ([-1,1]^{2n} ; \RR)} \leq 1}{\underset{\tilde{f} \in C^k([-1,1]^{2n} ; \RR)}{\sup}} \left\Vert \tilde{f} \circ \varphi_n^{-1} - \RE\left(M \left(\overline{a}\left(\tilde{f} \circ \fres{\varphi_n^{-1}}{\Omega_n}\right)\right)\right) \right\Vert_{L^\infty (\Omega_n; \RR)} \\
&= \underset{\Vert \tilde{f} \Vert _{C^k ([-1,1]^{2n} ; \RR)} \leq 1}{\underset{\tilde{f} \in C^k([-1,1]^{2n} ; \RR)}{\sup}} \left\Vert \tilde{f} - \RE\left(M \left(\varphi_m\left(\varphi_m^{-1}\left(\overline{a}\left(\tilde{f} \circ \fres{\varphi_n^{-1}}{\Omega_n}\right)\right)\right)\right)\right) \circ \varphi_n \right\Vert_{L^\infty ([-1,1]^{2n}; \RR)} \\
&= \underset{\Vert \tilde{f} \Vert _{C^k ([-1,1]^{2n} ; \RR)} \leq 1}{\underset{\tilde{f} \in C^k([-1,1]^{2n} ; \RR)}{\sup}} \left\Vert \tilde{f} - \widetilde{M}(\tilde{a}(\tilde{f})) \right\Vert_{L^\infty ([-1,1]^{2n}; \RR)} \geq \tilde{c} \cdot (2m)^{-k/(2n)},\\
\end{align*}
with a constant $\tilde{c} = \tilde{c}(n,k)$ provided by \Cref{app: devore_real}. Hence, the claim follows by choosing $c = c(n,k) \defeq 2^{-k/(2n)} \cdot \tilde{c}$.
\end{proof}

As a corollary, we formulate a special case of \Cref{thm:opti_conti} for the case of shallow complex-valued neural networks.

\begin{corollary}
\label{main_optimality}
    Let $n,k \in \NN$. Then there exists a constant $c=c(n,k) > 0$ with the following property: For any $m \in \NN$, $\phi \in C(\CC; \CC)$ and any map
    \begin{equation*}
        \eta : \quad C^k \left( \Omega_n ; \CC\right) \to \left(\CC^n\right)^m \times \CC^m \times \CC^m, \quad g \mapsto \left(\eta_1(g), \eta_2(g), \eta_3(g)\right)
    \end{equation*}
which is continuous with respect to some norm on $C^k (\Omega_n ; \CC)$, there exists $f \in C^k\left(\Omega_n; \CC\right)$ satisfying $\Vert f \Vert_{C^k (\Omega_n ; \CC)} \leq 1$ and
    \begin{equation*}
        \left\Vert f - \Psi(f)\right\Vert_{L^\infty \left(\Omega_n ; \CC\right)} \geq c \cdot m^{-k/(2n)},
    \end{equation*}
    where $\Psi(f) \in C(\Omega_n; \CC)$ is given by
\begin{equation*}
\Psi(f)(z) \defeq \sum_{j=1}^m \left(\eta_3(f)\right)_j \phi \left(\left[\eta_1 (f)\right]_j^T z + \left(\eta_2(f)\right)_j\right).
\end{equation*}
\end{corollary}

\begin{proof}
Using \Cref{thm:opti_conti}, we deduce that there exists $f \in C^k(\Omega_n;\CC)$ satisfying $\Vert f \Vert_{C^k(\Omega_n;\CC)} \leq 1$ and
\begin{equation*}
\Vert f - \Psi(f) \Vert_{L^\infty(\Omega_n; \CC)} \geq c' \cdot (m(n+2))^{-k/(2n)}
\end{equation*}
for a constant $c' = c'(n,k)>0$. Hence, the claim follows by letting $c \defeq c' \cdot (n+2)^{-k/(2n)}$.
\end{proof}

\subsection{Proof of Theorem \ref{thm:intrac}} \label{sec:intrac_reordered}


We can use \Cref{thm: devore} not only to show that the rate of convergence established in this paper is optimal (which is done in \Cref{optimality_section_reordered}) but also to show that the problem of approximating $C^k$-functions using a set of functions that can be parametrized with finitely many parameters is intractable in the sense that it suffers from the curse of dimensionality, provided that the map which assigns to each $C^k$-function the parameters of the approximating function is continuous. This is the subject of this section. 

In \cite{NOVAK2009398} a certain space of polynomials was used to show the intractability in the case of \emph{linear} approximation methods. We are also going to use this class of polynomials, but combine it with \Cref{thm: devore} to infer intractability in the case of \emph{continuous} approximation methods. We start with a lemma discussing an important property of this space of polynomials.
This property is stated as part of a proof in \cite{NOVAK2009398}, but no complete proof is provided.
\begin{lemma}\label{lem:poly_class}
Let $s \in \NN$ and consider a function $f \in C^\infty([-1,1]^s; \RR)$ which is given via
\begin{equation}\label{eq:form_poly_0_1}
f(x) = \sum_{\kk \in \{0,1\}^s} a_\kk x^\kk
\end{equation}
with coefficients $a_\kk \in \RR$ for every $\kk \in \{0,1\}^s$. Then it holds
\begin{equation*}
\Vert f \Vert_{C^k ([-1,1]^s; \RR)} = \Vert f \Vert_{L^\infty([-1,1]^s; \RR)}
\end{equation*}
for every $k \in \NN$.
\end{lemma}
\begin{proof}
The proof is by induction over $s$. We start with the case $s=1$ and note that we can write $f(x) = ax + b$ with $a,b \in \RR$ in that case. Switching to $-f$ if necessary, we can assume $a \geq 0$. Clearly, $\Vert f \Vert_{L^\infty([-1,1]; \RR)} \leq \vert a \vert + \vert b \vert$. Conversely, if $b \geq 0$ then $\vert f(1) \vert = \vert a + b \vert = \vert a \vert + \vert b \vert$. If otherwise $b < 0$ then $\vert f(-1) \vert = \vert b  - a \vert = \vert a - b \vert = \vert a \vert + \vert b \vert$. Thus, $\Vert f \Vert_{L^\infty([-1,1]; \RR)} = \vert a \vert + \vert b \vert$. For the derivatives, we have $\Vert f' \Vert_{L^\infty([-1,1]; \RR)} = \vert a \vert$ and $\Vert f^{(k)} \Vert_{L^\infty([-1,1]; \RR)} = 0$ for $k\geq 2$. This proves the claim in the case $s=1$. 

We now assume that the claim holds for some arbitrary but fixed $s \in \NN$. We further let $\alpha \in \NN_0^{s+1}$ and \emph{fix} a point $(x_1, ..., x_{s+1}) \in [-1,1]^{s+1}$. We decompose $\alpha = (\alpha', \alpha_{s+1})$ with $\alpha' \in \NN_0^s$. Let
\begin{equation*}
\widetilde{f}: \quad [-1,1] \to \RR, \quad y_{s+1} \mapsto \partial^{(\alpha',0)}f(x_1, ..., x_s, y_{s+1})
\end{equation*}
and note 
\begin{equation*}
\partial^\alpha f(x_1, ..., x_{s+1}) =  \widetilde{f}^{(\alpha_{s+1})} (x_{s+1}).
\end{equation*}
Note that $f$ is affine-linear with respect to each variable (with all other variables hold fixed). Hence, $\widetilde{f}$ is an affine function and we can thus apply the case $s=1$ to $\widetilde{f}$ and get
\begin{equation*}
\Vert \widetilde{f}^{(\alpha_{s+1})} \Vert_{L^\infty([-1,1]; \RR)} \leq \Vert \widetilde{f} \Vert_{L^\infty([-1,1]; \RR)}.
\end{equation*}
Putting this together, we infer
\begin{equation}\label{eq:jj1}
\vert  \partial^\alpha f(x_1, ..., x_{s+1}) \vert \leq \Vert \widetilde{f} \Vert_{L^\infty([-1,1]; \RR)} = \underset{y_{s+1} \in [-1,1]}{\sup} \vert \partial^{(\alpha',0)}f(x_1, ..., x_s, y_{s+1})\vert.
\end{equation}
We now \emph{fix} an arbitrary point $y_{s+1} \in [-1,1]$ and consider
\begin{equation*}
\widehat{f}: \quad [-1,1]^s \to \RR, \quad (y_1, ..., y_{s}) \mapsto f(y_1, ..., y_s, y_{s+1}).
\end{equation*}
Then it holds
\begin{equation*}
\partial^{(\alpha',0)}f(x_1, ..., x_s, y_{s+1}) = \partial^{\alpha'} \widehat{f}(x_1, ..., x_s).
\end{equation*}
Applying the induction hypothesis to $\widehat{f}$ (which is easily seen to be of the form \eqref{eq:form_poly_0_1}) we get
\begin{equation}\label{eq:jj2}
\vert \partial^{\alpha'} \widehat{f}(x_1, ..., x_s)\vert \leq \Vert \widehat{f} \Vert_{L^\infty([-1,1]^s ; \RR)} \leq \Vert f \Vert_{L^\infty([-1,1]^{s+1}; \RR)}.
\end{equation}
Combining \eqref{eq:jj1} and \eqref{eq:jj2} yields
\begin{equation*}
\vert  \partial^\alpha f(x_1, ..., x_{s+1}) \vert \leq \Vert f \Vert_{L^\infty([-1,1]^{s+1}; \RR)}.
\end{equation*}
Since $\alpha \in \NN_0^{s+1}$ was arbitrary, we get the claim by noting that 
\begin{equation*}
\Vert f \Vert_{C^k ([-1,1]^s; \RR)} \geq \Vert f \Vert_{L^\infty([-1,1]^s; \RR)}
\end{equation*}
holds trivially for every $k \in \NN$.
\end{proof}
Using the above lemma, we can now deduce that the approximation of smooth functions
using continuous approximation methods is intractable in terms of the input dimension.

\begin{proof}[Proof of \Cref{thm:intrac}]
We apply \Cref{thm: devore} to $X \defeq C([-1,1]^s ; \RR)$, $V \defeq C^{\infty, \ast,s}$ and to the set $K \defeq \{f \in C^{\infty, \ast, s}: \  \Vert f \Vert_{C^\infty ([-1,1]^s ; \RR)} \leq 1\}$ and $m \defeq 2^s - 1$. The space 
\begin{equation*}
X_{m+1} \defeq \left\{ [-1,1]^s \ni x \mapsto \sum_{\kk \in \{0,1\}^s} a_\kk x^\kk: \ a_\kk \in \RR\right\}
\end{equation*}
consisting of all functions considered in the previous \Cref{lem:poly_class} is an $(m+1)$-dimensional subspace of $C([-1,1]^s ; \RR)$. For every $f \in X_{m+1}$ with $\Vert f \Vert_{L^\infty([-1,1]^s ; \RR)} \leq 1$, \Cref{lem:poly_class} tells us $\Vert f \Vert_{C^\infty([-1,1]^s ; \RR)} \leq 1$. Hence, $U_1(X_{m+1}) \subseteq K$ and \Cref{thm: devore} then yields the claim.
\end{proof}

\begin{remark}
The statement of \Cref{thm:intrac} also holds if the functions satisfy $\overline{a}: C^{\infty, *, s} \to \RR^m$ and $M: \RR^m \to C([-1,1]^s;\RR)$ with $m \leq 2^s - 1$. This can be seen by defining 
\begin{equation*}
\tilde{a} : \quad C^{\infty, *, s} \to \RR^{2^s -1}, \quad f \mapsto (\overline{a}(f), 0, ..., 0)
\end{equation*}
and
\begin{equation*}
\widetilde{M}: \quad \RR^{2^s -1} \to C([-1,1]^s;\RR), \quad (a,b) \mapsto M(a)
\end{equation*}
with $a \in \RR^m$ and $b \in \RR^{2^s -1 -m}$.
\end{remark}
\renewcommand*{\proofname}{Proof}
The following \Cref{corr:intrac_complex} transfers \Cref{thm:intrac} to the complex-valued setting.
\begin{corollary}\label{corr:intrac_complex}
Let $n \in \NN$. For any function $f \in C^\infty(\Omega_n ; \CC)$ we write
\begin{equation*}
\Vert f \Vert_{C^\infty(\Omega_n ; \CC)} \defeq \underset{k \in \NN}{\sup} \ \Vert f \Vert_{C^k(\Omega_n; \CC)}
\end{equation*}
and let $C^{\infty, *, n}_{\CC}$ denote the space consisting of all functions for which this expression is finite. Let $\overline{a}:  C^{\infty, *, n}_{\CC} \to \CC^{2^{2n-1} - 1}$ be continuous with respect to some norm on $C^{\infty, *, n}_{\CC}$ and moreover, let $M: \CC^{2^{2n-1}-1} \to C(\Omega_n; \CC)$ be an arbitrary map. Then it holds
\begin{equation*}
\underset{\Vert f \Vert_{C^\infty(\Omega_n ; \CC)} \leq 1}{\underset{f \in C^{\infty, *, n}_{\CC}}{\sup}} \Vert f - M(\overline{a}(f))\Vert_{L^\infty(\Omega_n ; \CC)} \geq 1.
\end{equation*}
\end{corollary}
\begin{proof}
The transfer to the complex-valued setting works in the same manner as the proof of \Cref{thm:opti_conti} (see \Cref{optimality_section_reordered}). We write $m \defeq 2^{2n-1}-1$ and note $2m = 2^{2n} - 2 \leq 2^{2n}-1$. We define $\tilde{a} : C^{\infty, \ast, 2n}\to \RR^{2m}$ and $\widetilde{M}: \RR^{2m} \to C([-1,1]^{2n}; \RR)$ in the same way as in the proof of \Cref{thm:opti_conti}. Using again the same technique as in the proof of \Cref{thm:opti_conti}, we get
\begin{equation*}
\underset{\Vert f \Vert _{C^\infty (\Omega_n ; \CC)} \leq 1}{\underset{f \in C^{\infty, *, n}_{\CC}}{\sup}} \Vert f - M (\overline{a}(f)) \Vert_{L^\infty (\Omega_n; \CC)} \geq  \underset{\Vert \tilde{f} \Vert _{C^\infty ([-1,1]^{2n} ; \RR)} \leq 1}{\underset{\tilde{f} \in C^{\infty, \ast, 2n}}{\sup}} \left\Vert \tilde{f} - \widetilde{M}(\tilde{a}(\tilde{f})) \right\Vert_{L^\infty ([-1,1]^{2n}; \RR)} \geq 1,
\end{equation*}
applying \Cref{thm:intrac} in the last inequality, using $2m \leq 2^{2n}-1$.
\end{proof}
We conclude this appendix by adding a note on the constant appearing in our main approximation bound.
\begin{corollary} \label{corr:const_intrac}
Let $n \in \NN$ with $n \geq 2$ and $\alpha > 0$ and let $\phi \in C(\CC;\CC)$. Let $\tilde{c} = \tilde{c}(n,\alpha)>0$ be such that for every $m \in \NN$ there exists a mapping
\begin{equation*}
        \eta : \quad  C^{\infty, *, n}_{\CC} \to \left(\CC^n\right)^m \times \CC^m \times \CC^m, \quad g \mapsto \left(\eta_1(g), \eta_2(g), \eta_3(g)\right)
\end{equation*}
that is continuous with respect to any norm on $C^{\infty, *, n}_{\CC}$ and such that
\begin{equation*}
        \left\Vert f - \Psi(f)\right\Vert_{L^\infty \left(\Omega_n ; \CC\right)} \leq \left( \tilde{c} \cdot m\right)^{-\alpha} \cdot \Vert f \Vert_{C^\infty(\Omega_n; \CC)},
    \end{equation*}
    for every $f \in C^{\infty, *, n}_{\CC}$. Here, $\Psi(f) \in C(\Omega_n; \CC)$ is given by
\begin{equation*}
\Psi(f)(z) \defeq \sum_{j=1}^m \left(\eta_3(f)\right)_j \phi \left(\left[\eta_1 (f)\right]_j^T z + \left(\eta_2(f)\right)_j\right).
\end{equation*}
 Then it necessarily holds $\tilde{c}\leq 16 \cdot 2^{-n}$.
\end{corollary}
\begin{proof}
We first assume $n \geq 4$. We take $m= \left\lfloor \frac{2^{2n - 1} - 1}{n+2} \right\rfloor$ and note that then $m(n+2) \leq 2^{2n-1}-1$. Therefore, \Cref{corr:intrac_complex} applies and we infer that for each $\eps \in (0,1)$, there exists $f = f_\eps \in C^{\infty, *, n}_{\CC}$ with $\Vert f \Vert_{C^\infty(\Omega_n;\CC)} \leq 1$ and such that
\begin{equation*}
1 - \eps \leq \left\Vert f - \Psi(f)\right\Vert_{L^\infty \left(\Omega_n ; \CC\right)} \leq \left( \tilde{c} \cdot m\right)^{-\alpha} \cdot \Vert f \Vert_{C^\infty(\Omega_n; \CC)} \leq \left( \tilde{c} \cdot m\right)^{-\alpha}.
\end{equation*}
This then necessarily implies $\tilde{c} \cdot m \leq 1$ or equivalently $\tilde{c}\leq 1/m$. It therefore suffices to derive a lower bound for $m$. Firstly, we note
\begin{equation*}
2^{2n-1} = 2^{n-3} \cdot 2^{n+2} = 2^{n-3} \cdot (1+1)^{n+2} \geq 2^{n-3}(n+3),
\end{equation*}
where we applied Bernoulli's inequality. Because of $n \geq 4 \geq 3$, this yields
\begin{equation*}
2^{2n-1} - 1 \geq 2^{n-3}(n+3) - 2^{n-3} =  2^{n-3} (n+2).
\end{equation*}
Hence, we get
\begin{equation*}
m \geq \frac{2^{2n - 1} - 1}{n+2} -1  \geq 2^{n-3} -1 = 2^{n-3}(1- 2^{3-n}) \geq 2^{n - 4} = \frac{2^n}{16}.
\end{equation*}
Here, we used $n \geq 4$ in the last inequality. An explicit computation shows that the same bounds also holds in the cases $n=2$ and $n=3$. This proves the claim. 
\end{proof}
