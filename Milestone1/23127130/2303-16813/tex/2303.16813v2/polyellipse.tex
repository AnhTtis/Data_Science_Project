\section{Approximation of holomorphically extendable functions} \label{sec:polyellipse}
In this appendix we provide the proofs for the statements contained in \Cref{remark:holo}. The proofs mainly rely on results about sparse polynomial approximation \cite{adcock_sparse_2022}.
\begin{definition}[{cf. \cite[Assumption~2.3]{adcock_sparse_2022}}]
Let $s \in \NN$ and $\nuu \in (1, \infty)^s$. For every $j \in \{1,...,s\}$ let
\begin{equation*}
\mathcal{E}_{\nuu_j} \defeq \left\{ \frac{z + z^{-1}}{2}: \ z \in \CC, 1 \leq \vert z \vert \leq \nuu_j\right\}.
\end{equation*}
We then define the \emph{(filled-in) Bernstein polyellipse} of parameter $\nuu$ as
\begin{equation*}
\mathcal{E}_{\nuu} \defeq \mathcal{E}_{\nuu_1} \times \cdots \times \mathcal{E}_{\nuu_s} \subseteq \CC^s
\end{equation*}
and observe that $[-1,1]^s \subseteq \mathcal{E}_{\nuu}$ when we interpret $[-1,1]^s$ as a subset of $\CC^s$.
We further define 
\begin{equation*}
\mathcal{V}_{s}(\nuu) \defeq \left\{ f: [-1,1]^s \to \CC: \  \exists \text{ open } U \supseteq \mathcal{E}_{\nuu} \text{ and }\tilde{f} : U \to \CC \text{ holomorphic with } \fres{\tilde{f}}{[-1,1]^s} = f \right\}.
\end{equation*}
Here, $[-1,1]^s$ is again interpreted as a subset of $\CC^s$. Moreover, note that such an extension $\tilde{f}$ is, if existent, unique, as follows from the identity theorem for holomorphic functions. Hence, the expression
\begin{equation*}
\Vert f \Vert_{\mathcal{V}_s(\nuu)} \defeq \Vert \tilde{f} \Vert_{L^\infty (\mathcal{E}_{\nuu}; \CC)}
\end{equation*}
is well-defined. 
\end{definition}
\begin{definition} [{cf. \cite[pp.~25,28~ff.]{adcock_sparse_2022}}]
Let $s \in \NN$. We define a probability measure on $[-1,1]^s$ via
\begin{equation*}
d\mu_s \defeq  \prod_{j=1}^s \frac{1}{\pi\sqrt{1-x_j^2}} \ dx.
\end{equation*}
We define the \emph{normalized} Chebyshev polynomials for $\kk \in \NN_0^s$ as
\begin{equation*}
\widetilde{T_\kk} (x) \defeq 2^{\Vert \kk \Vert_0 / 2} \prod_{j=1}^s \ \cos(\kk_j \arccos(x_j)),
\end{equation*}
where $\Vert \kk \Vert_0 \defeq \#\{1 \leq j \leq s: \ \kk_j \neq 0\}$.
Note that this definition differs slightly from the notion used in \Cref{sec:fourier_reordered,ck_functions_reordered}.
\end{definition}

The following lemma is crucial for deriving the approximation rate in \Cref{thm:polyellipse}. The proofs can be found in \cite{adcock_sparse_2022}.
\begin{lemma}[{cf. \cite[Remark~2.15~and~Theorem~3.2]{adcock_sparse_2022}}] \label{lem:decay_poly}
Let $s \in \NN$, $\kk \in \NN_0^s$, $\nuu \in (1,\infty)^s$ and $f \in \mathcal{V}_s(\nuu)$. Then it holds
\begin{enumerate}
\item $\Vert \widetilde{T_{\kk}}\Vert_{L^\infty([-1,1]^s;\RR)} = 2^{\Vert \kk \Vert_0 / 2}$, where
      $\| \kk \|_0 = \# \{ 1 \leq j \leq s : \kk_j \neq 0  \}$;
\item $|\langle \widetilde{T_{\kk}}, f \rangle_{\mu_s} | \leq \nuu^{-\kk} \cdot 2^{\Vert \kk \Vert_0 / 2}\cdot \Vert f \Vert_{\mathcal{V}_s(\nuu)}$.
\end{enumerate}
\end{lemma}

It is a well-known fact that the Chebyshev polynomials $\widetilde{T_\kk}$
form an \emph{orthonormal basis} of $L^2_{\mu_s}([-1,1]^s ; \CC)$.
The following proposition states that functions from $\mathcal{V}_s(\nuu)$
can even be approximated uniformly by linear combinations of the $\widetilde{T_\kk}$ at a certain rate.
The proof follows essentially by applying \Cref{lem:decay_poly}.

\begin{proposition}\label{prop:polyellipse_approx}
Let $s,m \in \NN$ and $\nuu \in (1, \infty)^s$. Let $\nu \defeq \underset{j=1,...,s}{\min} \nuu_j$. Then there exists a constant $c = c(s, \nuu) > 0$ with the following property: For every $f \in \mathcal{V}_s(\nuu)$, defining
\begin{equation*}
P_m \defeq \underset{\kk \leq m}{\sum_{\kk \in \NN_0^s}} \langle f, \widetilde{T_\kk} \rangle_{\mu_s} \cdot \widetilde{T_\kk},
\end{equation*}
it holds
\begin{equation*}
\Vert f- P_m\Vert_{L^\infty([-1,1]^s; \CC)} \leq c \cdot \nu^{-m} \cdot \Vert f \Vert_{\mathcal{V}_s(\nuu)}.
\end{equation*}
\end{proposition}

\begin{proof}
Let $f \in \mathcal{V}_s(\nuu)$. Since the $\widetilde{T_\kk}$ form an orthonormal basis of $L^2_{\mu_s}([-1,1]^s; \CC)$ and since $\mu_s$ is a probability measure, so that $f \in C([-1,1]^s ; \CC) \subseteq L^2_{\mu_2}([-1,1]^s ; \CC)$, it follows that
\begin{equation*}
f = \sum_{\kk \in \NN_0^s}  \langle f, \widetilde{T_\kk} \rangle_{\mu_s} \cdot \widetilde{T_\kk}
\end{equation*}
with unconditional convergence in $L^2_{\mu_s}$. For $x \in [-1,1]^s$, note that
\begin{align*}
\sum_{\kk \in \NN_0^s}  \vert \langle f, \widetilde{T_\kk} \rangle_{\mu_s}\vert \cdot \vert \widetilde{T_\kk} (x) \vert \leq \sum_{\kk \in \NN_0^s}  \vert \langle f, \widetilde{T_\kk} \rangle_{\mu_s}\vert \cdot \Vert \widetilde{T_\kk} \Vert_{L^\infty ([-1,1]^s; \CC)} \leq 2^{s} \Vert f \Vert_{\mathcal{V}_s(\nuu)} \cdot \sum_{\kk \in \NN_0^s} \nuu^{-\kk}.
\end{align*}
Here, we employed \Cref{lem:decay_poly} at the last inequality. From $\nuu > 1$ it follows that
\begin{equation*}
\sum_{\kk \in \NN_0^s} \langle f, \widetilde{T_\kk} \rangle_{\mu_s} \cdot \widetilde{T_\kk}
\end{equation*}
converges \emph{pointwise} (even uniformly) and, since all the involved functions are continuous, this pointwise limit then has to coincide with the $L^2_{\mu_s}$-limit $f$. Hence, it holds
\begin{align*}
\Vert f- P_m\Vert_{L^\infty([-1,1]^s; \CC)}  &\leq \underset{\kk \nleq m}{\sum_{\kk \in \NN_0^s}} \vert \langle f, \widetilde{T_\kk} \rangle_{\mu_s}\vert \cdot \Vert \widetilde{T_\kk} \Vert_{L^\infty ([-1,1]^s; \CC)} \\
&\leq 2^{s} \Vert f \Vert_{\mathcal{V}_s(\nuu)} \cdot \underset{\vert \kk \vert \nleq m}{\sum_{\kk \in \NN_0^s}} \nuu^{-\kk},
\end{align*}
where we again used \Cref{lem:decay_poly}.
To complete the proof we compute
\begin{align*}
\underset{\vert \kk \vert \nleq m}{\sum_{\kk \in \NN_0^s}} \nuu^{-\kk} &\leq \sum_{j=1}^s \underset{\kk_j > m}{\sum_{\kk \in \NN_0^s}} \nuu^{-\kk } = \sum_{j=1}^s \left(\underset{l \neq j}{\prod_{1 \leq \ell \leq s}}\left(\sum_{k=0}^\infty \nuu_\ell ^{-k}\right) \cdot \sum_{k= m+1}^\infty \nuu_j^{-k}\right)\\
&\leq \sum_{j=1}^s \left(\nuu_j^{-(m+1)}\prod_{1 \leq \ell \leq s}\left(\sum_{k=0}^\infty \nuu_\ell ^{-k}\right)\right)\\
&= \prod_{1 \leq \ell \leq s}\left(\sum_{k=0}^\infty \nuu_\ell ^{-k}\right) \cdot \sum_{j=1}^s \nuu_j^{-(m+1)} \\
&= \prod_{1 \leq \ell \leq s} \left( \frac{\nuu_\ell}{\nuu_\ell - 1}\right) \cdot s \cdot \nu^{-(m+1)}
\end{align*}
and define $c(s, \nuu) \defeq 2^s \cdot s \cdot \prod_{\ell=1}^s \frac{\nuu_j}{\nuu_j- 1} \cdot \nu^{-1}$.
\end{proof}
To formulate the result for the approximation of holomorphically extendable functions using CVNNs we need to transfer the definition of $\mathcal{V}_s(\nuu)$ to the complex setting. For $\nuu \in (1,\infty)^{2n}$ and with $\varphi_n$ as in Equation \eqref{isomorphism_intro}, we hence write
\begin{equation*}
\mathcal{W}_n(\nuu) \defeq \left\{ f: \Omega_n \to \CC: \ f \circ \fres{\varphi_n}{[-1,1]^{2n}} \in \mathcal{V}_{2n}(\nuu)\right\}
\end{equation*}
 For $f \in \mathcal{W}_n(\nuu)$ we define
\begin{equation*}
\Vert f \Vert_{\mathcal{W}_n(\nuu)} \defeq \Vert f \circ \varphi_n \Vert_{\mathcal{V}_{2n}(\nuu)}.
\end{equation*}
Thus, $\mathcal{W}_n(\nuu)$ consists of all complex-valued functions defined on $\Omega_n$ that can be holomorphically extended onto some polyellipse in $\CC^{2n}$, where $\Omega_n \subseteq \CC^n$ is interpreted as a subset of $\RR^{2n}$ and then as a subset of $\CC^{2n}$. The final approximation result then reads as follows.
\begin{theorem}\label{thm:polyellipse}
Let $n \in \NN$ and $\nuu \in (1, \infty)^{2n}$. Set 
\begin{equation*}
\nu \defeq \underset{1 \leq j \leq 2n}{\min} \nuu_j.
\end{equation*}
 Then there exists a constant $c = c(n, \nuu)> 0$ with the following property: For every function $\phi: \CC \to \CC$ that is smooth and non-polyharmonic on some open set $\emptyset \neq U \subset \CC$ and for every $m \in \NN$ there exists a first layer $\Phi \in \mathcal{F}^\phi_{n,m}$ with the property that for every $f \in \mathcal{W}_n(\nuu)$ there exist coefficients $\sigma = \sigma(f) \in \CC^m$ such that
\begin{equation*}
\Vert f - \sigma^T \Phi \Vert_{L^\infty(\Omega_n ; \CC)} \leq c \cdot \nu^{-m^{1/(2n)} / 17}\cdot \Vert f \Vert_{\mathcal{W}_n(\nuu)}.
\end{equation*}
Moreover, the map $f \mapsto \sigma(f)$ is a continuous linear functional with respect to the $L^\infty$-norm.
\end{theorem}
\begin{proof}
Choose $M \in \NN$ as the largest integer satisfying $(8M+1)^{2n} \leq m$, where we assume without loss of generality that $9^{2n} \leq m$. This can be done by choosing $\sigma = 0$ for $m < 9^{2n}$ at the cost of possibly enlarging $c$. Note that the maximality of $M$ implies $(8M + 9)^{2n}> m$. By using $8M +9 \leq 17M$, this gives us
\begin{equation}\label{eq:M_lower_bound}
M > \frac{1}{17} \cdot m^{1/(2n)}.
\end{equation}
Choose the constant $c_1 = c_1(2n, \nuu)$ according to \Cref{prop:polyellipse_approx}. 

Fix $\kk \in \NN_0^{2n}$ with $\kk \leq M$. Since $\widetilde{T_\kk}$ is a polynomial of componentwise degree at most $M$, we have a representation 
    \begin{equation*}
        \left(\widetilde{T_\kk} \circ \varphi_n^{-1} \right)(z) = \underset{\elll^1, \elll^2 \leq M}{\sum_{\elll^1, \elll^2 \in \NN_0^n}} a_{\elll^1, \elll^2}^\kk \prod_{t=1}^n \RE \left(z_t\right)^{\elll^1_t} \IM \left(z_t\right)^{\elll^2_t}
    \end{equation*}
    with suitably chosen coefficients $a_{\elll^1, \elll^2}^\kk \in \CC$. 
    By using the identities $\RE\left(z_t\right) = \frac{1}{2}\left(z_t + \overline{z_t}\right)$ and also $\IM\left(z_t\right) = \frac{1}{2i}\left( z_t - \overline{z_t}\right)$, we can rewrite $\widetilde{T_\kk} \circ \varphi_n^{-1}$ into a complex polynomial in $z$ and $\overline{z}$, i.e.,
    \begin{equation*}
        \left(\widetilde{T_\kk} \circ \varphi_n^{-1}\right)\left(z\right) = \underset{\elll^1, \elll^2 \leq 2M}{\sum_{\elll^1, \elll^2 \in \NN_0^{n}}} b_{\elll^1, \elll^2}^\kk z^{\elll^1} \overline{z}^{\elll^2}
    \end{equation*}
    with complex coefficients $b_{\elll^1, \elll^2}^\kk \in \CC$.
    Using \Cref{main_1}, we choose $\rho_1, ..., \rho_m \in \CC^n$ and $b \in \CC$, such that for any polynomial $P \in \left\{ \widetilde{T_\kk} \circ \varphi_n^{-1} : \ \kk \leq M\right\} \subseteq \mathcal{P}_{2M}^n$ there exist coefficients $\sigma_1(P), ..., \sigma_m(P) \in \CC$, such that
    \begin{equation} 
    \label{eq:gp}
        \left\Vert g_P - P \right\Vert_{L^\infty \left(\Omega_n; \CC\right)} \leq \left(\sum_{\kk \in \NN_0^{2n}} \nuu^{-\kk} \right)^{-1} \cdot \nu^{-(M+1)}, 
    \end{equation}
    where 
    \begin{equation*}
        g_P \defeq \sum_{t=1}^m \sigma_t(P) \phi \left(\rho_t^T z + b\right).
    \end{equation*}
    Note that here we implicitly use the bound $(4 \cdot (2M) + 1)^{2n} \leq m$. We are now going to show that the first layer $\Phi \in \mathcal{F}^\phi_{n,m}$ defined using the $\rho_t$ and $b$ (i.e., $\Phi(z) = (\phi (\rho_t^T z  + b))_{t=1}^m$) has the desired property.
    
    To this end, take an arbitrary function $f \in \mathcal{W}_n(\nuu)$. \Cref{prop:polyellipse_approx} tells us that
    \begin{equation}\label{eq:prop_from_above}
    \Vert f- P_M \circ \varphi_n^{-1}\Vert_{L^\infty(\Omega_n; \CC)} = \Vert f \circ \varphi_n - P_M\Vert_{L^\infty([-1,1]^{2n}; \CC)} \leq c_1 \cdot \nu^{-(M+1)} \cdot \Vert f \circ \varphi_n \Vert_{\mathcal{V}_{2n}(\nuu)},
    \end{equation} 
    where
    \begin{equation*}
    P_M \circ \varphi_n^{-1} = \underset{\kk \leq M}{\sum_{\kk \in \NN_0^{2n}}} \langle f \circ \varphi_n, \widetilde{T_\kk} \rangle_{\mu_{2n}} \cdot (\widetilde{T_\kk} \circ \varphi_n^{-1}).
    \end{equation*}
    We then define the approximating network 
    \begin{equation*}
    g\defeq  \underset{\kk \leq M}{\sum_{\kk \in \NN_0^{2n}}} \langle f \circ \varphi_n, \widetilde{T_\kk} \rangle_{\mu_{2n}} \cdot g_{\widetilde{T_\kk} \circ \varphi_n^{-1}}.
    \end{equation*}
    From the definition of the functions $g_{\widetilde{T_\kk} \circ \varphi_n^{-1}}$ it follows directly that $g$ is a shallow CVNN with first layer $\Phi \in \mathcal{F}^\phi_{n,m}$. Furthermore, it holds
    \begin{align*}
    \Vert P_M \circ \varphi_n^{-1} - g\Vert_{L^\infty(\Omega_n; \CC)} &\leq \underset{\kk \leq M}{\sum_{\kk \in \NN_0^{2n}}} \vert\langle f \circ \varphi_n, \widetilde{T_\kk} \rangle_{\mu_{2n}} \vert \cdot \Vert\widetilde{T_\kk} \circ \varphi_n^{-1} - g_{\widetilde{T_\kk} \circ \varphi_n^{-1}}\Vert_{L^\infty(\Omega_n; \CC)} \\
    \overset{\eqref{eq:gp}, \text{Lem. }\mathrm{\ref{lem:decay_poly}}}&{\leq} 2^{n} \cdot \Vert f \circ \varphi_n\Vert_{\mathcal{V}_{2n}(\nuu)} \cdot \underset{\kk \leq M}{\sum_{\kk \in \NN_0^{2n}}} \nuu^{-\kk} \cdot \left(\sum_{\kk \in \NN_0^{n}} \nuu^{-\kk} \right)^{-1} \cdot \nu^{-(M+1)} \\
    &\leq 2^{n} \cdot \Vert f \Vert_{\mathcal{W}_n(\nuu)} \cdot \nu^{-(M+1)}.
    \end{align*}
    Combining this estimate with \Cref{eq:prop_from_above} and applying the triangle inequality gives us
    \begin{equation*}
    \Vert f-g\Vert_{L^\infty(\Omega_n; \CC)} \leq (c_1 + 2^n) \cdot \nu^{-(M+1)} \cdot \Vert f \Vert_{L^\infty(\mathcal{E}_{\nuu};\CC)}.
    \end{equation*}
    Hence, the claim follows taking $c \defeq c_1 + 2^n$ and using \eqref{eq:M_lower_bound}.
    
    The continuity of the linear map $f \mapsto \sigma(f)$ with respect to the $L^\infty$-norm follows directly from the fact that $f \mapsto \langle f \circ \varphi_n, \widetilde{T_\kk} \rangle_{\mu_{2n}}$ is trivially continuous with respect to the $L^2_{\mu_{2n}}$-norm and hence also continuous with respect to the $L^\infty$-norm since $\mu_{2n}$ is a probability measure.
\end{proof}

