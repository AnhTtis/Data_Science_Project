
\section{Proof of Theorem \ref{thm:opti_conti}}
\label{optimality_section}

We first state a theorem which establishes a lower bound for the approximation of $C^k$-functions by continuous functions that can be parametrized by a certain number of parameters. The proof, which is already contained in \cite{devore_optimal_1989} in a slightly different setting, is deferred to \Cref{prelim_opti} (\Cref{app: devore_real}).
\begin{theorem} \label{thm: devore_real}
Let $s,k \in \NN$. Then there exists a constant $c = c(s,k)>0$ with the following property:  For any $m \in \NN$ and any map $\overline{a} : C^k ([-1,1]^s; \RR) \to \RR^m$ that is continuous with respect to some norm on $C^k([-1,1]^s; \RR)$ and any (possibly discontinuous) map $M  : \RR^m \to C([-1,1]^s ; \RR)$, we have
\begin{equation*}
  \underset{\Vert f \Vert _{C^k ([-1,1]^s ; \RR)} \leq 1}{\underset{f \in C^k([-1,1]^s ; \RR)}{\sup}}
    \Vert f - M (\overline{a}(f)) \Vert_{L^\infty ([-1,1]^s ; \RR)}
  \geq c \cdot m^{-k/s}.
\end{equation*}
\end{theorem}
Using this theorem, we can now prove \Cref{thm:opti_conti}.
\medskip 
\renewcommand*{\proofname}{Proof of \Cref{thm:opti_conti}}
\begin{proof}
Let $\overline{a}: C^k(\Omega_n ; \CC) \to \CC^m$ be any map that is continuous with respect to some norm $\Vert \cdot \Vert_V$ on $C^k(\Omega_n; \CC)$, and let $M: \CC^m \to C(\Omega_n;\CC)$ be arbitrary. With $\varphi_n, \varphi_m$ defined as in Equation \eqref{isomorphism_intro}, let
\begin{equation*}
\tilde{a}: \quad C^k ([-1,1]^{2n}; \RR) \to \RR^{2m}, \quad \tilde{f} \mapsto \varphi_m^{-1} \left( \overline{a} \left( \tilde{f} \circ \fres{\varphi_n^{-1}}{\Omega_n}\right)\right).
\end{equation*}
Clearly, $\tilde{a}$ is continuous on $C^k([-1,1]^{2n}; \RR)$ with respect to the norm $\Vert \cdot \Vert_{\tilde{V}}$ on $C^k([-1,1]^{2n}; \RR)$ defined as
\begin{equation*}
\Vert \tilde{f} \Vert_{\tilde{V}} \defeq \left\Vert \tilde{f} \circ \fres{\varphi_n^{-1}}{\Omega_n}\right\Vert_V \quad \text{for } \tilde{f} \in C^k([-1,1]^{2n}; \RR).
\end{equation*}
Let
\begin{equation*}
\widetilde{M}: \quad \RR^{2m} \to C([-1,1]^{2n}; \RR), \quad \widetilde{M}(x) \defeq \RE(M(\varphi_m(x))) \circ \fres{\varphi_n}{[-1,1]^{2n}}.
\end{equation*}
Then it holds
\begin{align*}
&\norel \underset{\Vert f \Vert _{C^k (\Omega_n ; \CC)} \leq 1}{\underset{f \in C^k(\Omega_n ; \CC)}{\sup}} \Vert f - M (\overline{a}(f)) \Vert_{L^\infty (\Omega_n; \CC)} \\
&\geq \underset{\Vert f \Vert _{C^k (\Omega_n ; \RR)} \leq 1}{\underset{f \in C^k(\Omega_n ; \RR)}{\sup}} \Vert f - \RE(M (\overline{a}(f))) \Vert_{L^\infty (\Omega_n; \RR)} \\
&= \underset{\Vert \tilde{f} \Vert _{C^k ([-1,1]^{2n} ; \RR)} \leq 1}{\underset{\tilde{f} \in C^k([-1,1]^{2n} ; \RR)}{\sup}} \left\Vert \tilde{f} \circ \varphi_n^{-1} - \RE\left(M \left(\overline{a}\left(\tilde{f} \circ \fres{\varphi_n^{-1}}{\Omega_n}\right)\right)\right) \right\Vert_{L^\infty (\Omega_n; \RR)} \\
&= \underset{\Vert \tilde{f} \Vert _{C^k ([-1,1]^{2n} ; \RR)} \leq 1}{\underset{\tilde{f} \in C^k([-1,1]^{2n} ; \RR)}{\sup}} \left\Vert \tilde{f} - \RE\left(M \left(\varphi_m\left(\varphi_m^{-1}\left(\overline{a}\left(\tilde{f} \circ \fres{\varphi_n^{-1}}{\Omega_n}\right)\right)\right)\right)\right) \circ \varphi_n \right\Vert_{L^\infty ([-1,1]^{2n}; \RR)} \\
&= \underset{\Vert \tilde{f} \Vert _{C^k ([-1,1]^{2n} ; \RR)} \leq 1}{\underset{\tilde{f} \in C^k([-1,1]^{2n} ; \RR)}{\sup}} \left\Vert \tilde{f} - \widetilde{M}(\tilde{a}(\tilde{f})) \right\Vert_{L^\infty ([-1,1]^{2n}; \RR)} \geq \tilde{c} \cdot (2m)^{-k/(2n)},\\
\end{align*}
with a constant $\tilde{c} = \tilde{c}(n,k)$ provided by \Cref{thm: devore_real}. Hence, the claim follows by choosing $c = c(n,k) \defeq 2^{-k/(2n)} \cdot \tilde{c}$.
\end{proof}
\renewcommand*{\proofname}{Proof}
As a corollary, we formulate a special case of \Cref{thm:opti_conti} for the case of shallow complex-valued neural networks.
\begin{corollary}
\label{main_optimality}
    Let $n,k \in \NN$. Then there exists a constant $c=c(n,k) > 0$ with the following property: For any $m \in \NN$, $\phi \in C(\CC; \CC)$ and any map
    \begin{equation*}
        \eta : \quad C^k \left( \Omega_n ; \CC\right) \to \left(\CC^n\right)^m \times \CC^m \times \CC^m, \quad g \mapsto \left(\eta_1(g), \eta_2(g), \eta_3(g)\right)
    \end{equation*}
which is continuous with respect to some norm on $C^k (\Omega_n ; \CC)$, there exists $f \in C^k\left(\Omega_n; \CC\right)$ satisfying $\Vert f \Vert_{C^k (\Omega_n ; \CC)} \leq 1$ and
    \begin{equation*}
        \left\Vert f - \Psi(f)\right\Vert_{L^\infty \left(\Omega_n ; \CC\right)} \geq c \cdot m^{-k/(2n)},
    \end{equation*}
    where $\Psi(f) \in C(\Omega_n; \CC)$ is given by
\begin{equation*}
\Psi(f)(z) \defeq \sum_{j=1}^m \left(\eta_3(f)\right)_j \phi \left(\left[\eta_1 (f)\right]_j^T z + \left(\eta_2(f)\right)_j\right).
\end{equation*}
\end{corollary}

\begin{proof}
Using \Cref{thm:opti_conti}, we deduce that there exists $f \in C^k(\Omega_n;\CC)$ satisfying $\Vert f \Vert_{C^k(\Omega_n;\CC)} \leq 1$ and
\begin{equation*}
\Vert f - \Psi(f) \Vert_{L^\infty(\Omega_n; \CC)} \geq c' \cdot (m(n+2))^{-k/(2n)}
\end{equation*}
for a constant $c' = c'(n,k)>0$. Hence, the claim follows by letting $c \defeq c' \cdot (n+2)^{-k/(2n)}$.
\end{proof}






\section{Proof of Theorem \ref{main_4}} \label{sec:main_4}
Our proof of \Cref{main_4} is based on the following result (proven in \Cref{app: ridge}) which is based on the theory of ridge functions \cite{maiorov_best_1999,pinkus_ridge_2016}.
\begin{theorem} \label{thm: ridge}
Let $s,k \in \NN$ with $s \geq 2$ and $r>0$. Then there exists a constant $c = c(s,k) > 0$ with the following property: For every $m \in \NN$ there exist $a_1, ..., a_m \in \RR^s \setminus \{0\}$ with $\Vert a_j \Vert_2 = r$, such that for every function $f \in C^k ([-1,1]^s ; \RR)$ there exist polynomials $p_1, ..., p_m : \RR \to \RR$ satisfying
\begin{equation*}
\left\Vert f(x) - \sum_{j=1}^m p_j (a_j^T x) \right\Vert_{L^\infty ([-1,1]^s ; \RR)} \leq c \cdot m^{-k/(s-1)}\cdot \Vert f \Vert_{C^k([-1,1]^s; \RR)}.
\end{equation*}  
\end{theorem}
\begin{proof}
See \Cref{app: ridge}.
\end{proof}
Using the previous theorem, we can prove the following statement for complex-valued $C^k$-functions.
\begin{proposition}
\label{ridge_approx}
    Let $n,k \in \NN$. Then there exists a constant $c=c(n,k)>0$ with the following property: For any $m \in \NN$ there exist complex vectors $b_1, ..., b_m \in \CC^n$ with $\left\Vert b_j \right\Vert_2 = 1/ \sqrt{2n}$ for $j = 1,...,m$ and with the property that for any function $f \in C^k \left(\Omega_n ; \CC\right)$ there exist functions $g_1, ..., g_m \in C(\Omega_1; \CC)$ such that
    \begin{equation*}
        \left\Vert f(z) - \sum_{j=1}^m g_j \left( b_j^T \cdot z \right)\right\Vert_{L^\infty \left(\Omega_n; \CC\right)} \leq c \cdot m^{-k /(2n-1)} \cdot \left\Vert f \right\Vert_{C^k \left( \Omega_n; \CC\right)}.
    \end{equation*}
    Note that the vectors $b_1, ... b_m$ can be chosen independently from the considered function $f$, whereas $g_1, ..., g_m$ do depend on $f$.
\end{proposition}
\begin{proof}
     \Cref{thm: ridge} yields the existence of a constant $c_1 = c_1(n,k)>0$ with the property that for any $m \in \NN$ there exist real vectors $a_1, ..., a_m \in \RR^{2n}$ with $\left\Vert a_j \right\Vert_2 = 1 / \sqrt{2n}$ such that for any function $\tilde{f} \in C^k \left([-1,1]^{2n}; \RR\right)$ there exist functions $\tilde{g}_1, ..., \tilde{g}_m \in C([-1,1]; \RR)$ satisfying
    \begin{equation*}
        \left\Vert \tilde{f}(x) - \sum_{j=1}^m \tilde{g}_j \left( a_j^T  x \right)\right\Vert_{L^\infty \left([-1,1]^{2n}; \RR\right)} \leq c_1 \cdot m^{-k /(2n-1)} \cdot \Vert \tilde{f} \Vert_{C^k \left( [-1,1]^{2n}; \RR\right)}.
    \end{equation*}
    We then define the vectors $b_1, ..., b_m \in \CC^n$ componentwise via
    \begin{equation*}
        \left(b_j\right)_\ell \defeq \left(a_j\right)_\ell - i \cdot \left(a_j\right)_{n+\ell}, \quad \ell \in \{1,...,n\}, \ j \in \{1,...,m\}.
    \end{equation*}
    First we see $\left\Vert b_j \right\Vert_2 = \left\Vert a_j \right\Vert_2 = 1/\sqrt{2n}$. We first consider real-valued functions, i.e., $f \in C^k \left(\Omega_n; \RR\right)$. Let $\varphi_n$ be defined as in \eqref{isomorphism_intro}. By the choice of the constant $c_1$ we can find continuous functions $\tilde{g}_1,..., \tilde{g}_m \in C \left([-1,1]; \RR\right)$ such that
    \begin{equation*}
        \left\Vert (f \circ \varphi_n) (x)- \sum_{j=1}^m \tilde{g}_j \left( a_j^T x \right)\right\Vert_{L^\infty \left([-1,1]^{2n}\right)} \leq c_1 \cdot m^{-k /(2n-1)} \cdot \left\Vert f \circ \varphi_n \right\Vert_{C^k \left( [-1,1]^{2n}; \RR\right)}.
    \end{equation*}
    We then define $g_j \in C(\Omega_1; \RR)$ by $g_j(z) \defeq \tilde{g}_j \left(\RE(z)\right)$ for any $j \in \{1, ..., m\}$. For $z \in \Omega_n$ we then have
    \begin{align}
        g_j \left(\left( b_j \right)^T z \right) &= \tilde{g}_j \left( \RE \left(\sum_{\ell = 1}^n \left( b_j\right)_\ell \cdot z_\ell\right)\right) \nonumber \\
        &= \tilde{g}_j \left( \RE \left( \sum_{\ell = 1}^n \left(\left(a_j\right)_\ell - i\cdot \left(a_j\right)_{n+\ell}\right)\left(\varphi_n^{-1}(z)_\ell + i\cdot \varphi_n^{-1}(z)_{n+\ell}\right)\right)\right) \nonumber \\
        &= \tilde{g}_j \left(\sum_{\ell = 1}^n\left[\left(a_j\right)_\ell \varphi_n^{-1}(z)_\ell + \left(a_j\right)_{n+\ell} \varphi_n^{-1}(z)_{n+\ell}\right]\right) \nonumber \\
        \label{trafo_ident}
        &= \tilde{g}_j \left( \left( a_j \right)^T \cdot \varphi_n^{-1}(z)\right).
    \end{align}
    Therefore, 
    \begin{align*}
        \left\Vert f(z) - \sum_{j=1}^m g_j \left( b_j^T z\right)\right\Vert_{L^\infty \left(\Omega_n; \RR\right)} &= \left\Vert (f \circ \varphi_n)(x) - \sum_{j=1}^m g_j \left( b_j^T \cdot \varphi_n(x) \right) \right\Vert_{L^\infty \left([-1,1]^{2n}; \RR\right)} \\
        \overset{\text{(\ref{trafo_ident})}}&{=} \left\Vert (f \circ \varphi_n)(x) - \sum_{j=1}^m \tilde{g}_j \left( a_j^T \cdot x \right)\right\Vert_{L^\infty \left([-1,1]^{2n} ; \RR\right)} \\
        &\leq c_1 \cdot m^{-k /(2n-1)} \cdot \left\Vert f \circ \varphi_n\right\Vert_{C^k \left( [-1,1]^{2n}; \RR\right)}.
    \end{align*}
    By the above, for a function $f \in C^k \left(\Omega_n ; \CC\right)$ we can pick functions $g^{\RE}_1, ..., g^{\RE}_m, g^{\IM}_1, ..., g^{\IM}_m \in C \left(\Omega_1 ; \RR \right)$ satisfying
    \begin{align*}
        \left\Vert \RE(f(z)) - \sum_{j=1}^m g^{\RE}_j \left( b_j^T z \right)\right\Vert_{L^\infty \left(\Omega_n; \RR\right)} &\leq c_1 \cdot m^{-k /(2n-1)} \cdot \left\Vert {\RE} \left(f \circ \varphi_n \right)\right\Vert_{C^k \left( [-1,1]^{2n}; \RR\right)}, \\
        \left\Vert \IM(f(z)) - \sum_{j=1}^m g^{\IM}_j \left( b_j^T z \right)\right\Vert_{L^\infty \left(\Omega_n; \RR\right)} &\leq c_1 \cdot m^{-k /(2n-1)} \cdot \left\Vert \IM \left(f \circ \varphi_n \right)\right\Vert_{C^k \left( [-1,1]^{2n}; \RR\right)}.
    \end{align*}
    Defining $g_j := g^{\RE}_j + i \cdot g^{\IM}_j$ yields
    \begin{equation*}
        \left\Vert f(z)- \sum_{j=1}^m g_j \left(b_j^T z\right)\right\Vert_{L^\infty \left( \Omega_n; \CC\right)} \leq c_1 \cdot \sqrt{2}\cdot m^{-k /(2n-1)} \cdot \left\Vert f \right\Vert_{C^k \left( \Omega_n; \CC\right)},
    \end{equation*}
    completing the proof.
\end{proof}
The special activation function that yields the improved approximation rate of $m^{-k/(2n-1)}$ (see \Cref{main_4}) is constructed in the following lemma. 
\begin{lemma}
\label{special_acti_func}
    Let $\left\{ u_\ell\right\}_{\ell = 1}^\infty$ be an enumeration of the set of complex polynomials in $z$ and $\overline{z}$ with coefficients in $\QQ + i\QQ$.
    Then there exists a function $\phi \in C^\infty \left( \CC; \CC\right)$ with the following properties:
    \begin{enumerate}
        \item For every $\ell \in \NN$ and $z \in \Omega_1$ one has
        \begin{equation*}
            \phi(z+3\ell) = u_\ell(z).
        \end{equation*}
        \item $\phi$ is non-polyharmonic.
    \end{enumerate}
\end{lemma}
\begin{proof}
    Let $\psi \in C^\infty\left(\CC; \RR\right)$ with $0 \leq \psi \leq 1$ and
    \begin{equation*}
        \fres{\psi}{\Omega_1} \equiv 1, \qquad \supp(\psi) \subseteq \widetilde{\Omega},
    \end{equation*}
    where $\widetilde{\Omega} \defeq \left\{ z \in \CC : \ \left\vert \RE \left(z\right)\right\vert, \left\vert \IM \left(z\right) \right\vert < \frac{3}{2} \right\}$. We then define
    \begin{equation*}
        \phi \defeq f \cdot \psi + \sum_{\ell = 1}^\infty u_\ell(\bullet - 3\ell) \cdot \psi(\bullet - 3\ell),
    \end{equation*}
    where $f(z) = e^{\RE(z)}$. Note that $\phi$ is smooth since it is a locally finite sum of smooth functions. Furthermore, $\phi$ is non-polyharmonic on the interior of $\Omega_1$, since the calculation in the proof of \Cref{admissible} shows for $z$ in the interior of $\Omega_1$ and $\rho: \RR \to \RR, \ t \mapsto e^t$ that 
\begin{equation*}
\left\vert \wirt^m \wirtq^\ell \phi (z)\right\vert = \left\vert \wirt^m \wirtq^\ell f (z)\right\vert = \frac{1}{2^{m+\ell}} \left\vert \rho^{(m+ \ell)}(\RE(z))\right\vert>0
\end{equation*}
for arbitrary $m, \ell \in \NN_0$. Finally, property (1) follows directly by construction of $\phi$ because 
    \begin{equation*}
        (\widetilde{\Omega} + 3\ell) \cap (\widetilde{\Omega} + 3\ell') = \emptyset
    \end{equation*}
    for $\ell \neq \ell'$.
\end{proof}

Using the properties of the special activation function constructed in \Cref{special_acti_func} and applying the approximation result from \Cref{ridge_approx} we can now prove \Cref{main_4}.
\medskip
\renewcommand*{\proofname}{Proof of \Cref{main_4}}
\begin{proof}
    Let $\phi$ be the activation function constructed in \Cref{special_acti_func}. We choose the constant $c$ according to Proposition \ref{ridge_approx}. Let $m \in \NN$ and $f \in C^k \left(\Omega_n; \CC\right)$. We can without loss of generality assume that $f \not\equiv 0$. Again, according to \Cref{ridge_approx}, we can choose $\rho_1, ..., \rho_m \in \CC^n$ with $\left\Vert_2 \rho_j \right\Vert = 1 / \sqrt{2n}$ and $g_1, ..., g_m \in C\left(\Omega; \CC\right)$ with the property
    \begin{equation*}
        \left\Vert f(z) - \sum_{j=1}^m g_j \left( \rho_j^T z \right)\right\Vert_{L^\infty \left(\Omega_n\right)} \leq c \cdot m^{-k /(2n-1)} \cdot \left\Vert f \right\Vert_{C^k \left( \Omega_n; \CC\right)}.
    \end{equation*}
    Recall from \Cref{special_acti_func} that $\{u_\ell\}_{\ell = 1}^\infty$ is an enumeration of the set of complex polynomials in $z$ and $\overline{z}$. Hence, using the complex version of the Stone-Weierstraß-Theorem (see for instance \cite[Theorem 4.51]{folland_real_1999}), we can pick $\ell_1, ..., \ell_m \in \NN$ such that
    \begin{equation}
    \label{approxx}
        \left\Vert g_j - u_{\ell_j} \right\Vert_{L^\infty \left(\Omega_1 ; \CC\right)} \leq  m^{-1-k /(2n-1)} \cdot \left\Vert f \right\Vert_{C^k \left( \Omega_n; \CC\right)}
    \end{equation}
    for every $j \in \{1,...,m\}$. Since $\phi\left(\bullet + 3\ell\right) = u_\ell$ on $\Omega_1$ for each $\ell \in \NN$, and since $\rho_j^T z \in \Omega_1$ for $j \in \{1,...,m\}$ and $z \in \Omega_n$, we estimate
    \begin{align*}
        & \norel
\left\Vert f(z) - \sum_{j=1}^m \phi \left( \rho_j^T z + 3\ell_j\right)\right\Vert_{L^\infty \left(\Omega_n; \CC\right)} \\
        &\leq \left\Vert f(z) - \sum_{j=1}^m g_j \left( \rho_j^T z \right)\right\Vert_{L^\infty \left(\Omega_n; \CC\right)} + \sum_{j=1}^m \left\Vert g_j \left( \rho_j^T z\right) - \phi \left( \rho_j^T \cdot z + 3\ell_j\right)\right\Vert_{L^\infty \left(\Omega_n; \CC\right)} \\
        &\leq c \cdot m^{-k /(2n-1)} \cdot \left\Vert f \right\Vert_{C^k \left( \Omega_n; \CC\right)} + \sum_{j=1}^m \left\Vert g_j \left( z\right) - u_{\ell_j} \left( z\right)\right\Vert_{L^\infty \left(\Omega_1; \CC\right)} \\
        \overset{\eqref{approxx}}&{\leq}  c \cdot m^{-k /(2n-1)} \cdot \left\Vert f \right\Vert_{C^k \left( \Omega_n; \CC\right)} + m^{-k /(2n-1)} \cdot \left\Vert f \right\Vert_{C^k \left( \Omega_n; \CC\right)} \\
        &= (c+1) \cdot m^{-k/(2n-1)} \cdot \Vert f \Vert_{C^k \left(\Omega_n; \CC\right)}.
\qedhere
    \end{align*}
\end{proof}

\renewcommand*{\proofname}{Proof}



\section{Proof of Theorem \ref{main_5}}
As a preparation for the proof of \Cref{main_5}, we first prove a similar result in the real-valued setting. We remark that the proof idea is inspired by the proof of \cite[Theorem 4]{yarotsky_error_2017}.
\label{sec:main_5}

\begin{theorem}
\label{sigmoidoptimality}
Let $n, k \in \NN$ and 
\begin{equation*}
    \phi: \quad \RR \to \RR, \quad \phi(x) \defeq \frac{1}{1+e^{-x}}
\end{equation*}
be the sigmoid function. Then there exists a constant $c = c(n,k) > 0$ with the following property: If the numbers $\varepsilon \in (0,\frac{1}{2})$ and $m \in \NN$ are such that for every function $f \in C^k \left([-1,1]^n ; \RR\right)$ with $\left\Vert f \right\Vert_{C^k\left([-1,1]^n; \RR\right)} \leq 1$ there exist coefficients $\rho_1, ..., \rho_m \in \RR^n$, $\eta_1, ..., \eta_m \in \RR$ and $\sigma_1, ..., \sigma_m \in \RR$ satisfying
\begin{equation*}
    \left\Vert f(x) - \sum_{j=1}^m \sigma_j \cdot \phi \left( \rho_j^T x + \eta_j\right)\right\Vert_{L^\infty \left([-1,1]^n ; \RR\right)} \leq \varepsilon,
\end{equation*}
then necessarily
\begin{equation*}
    m \geq c \cdot \frac{\varepsilon^{-n/k}}{  \mathrm{ln}\left( 1/\varepsilon\right)}.
\end{equation*}
\end{theorem}
\begin{proof}
    We first pick a function $\psi \in C^\infty \left(\RR^n; \RR\right)$ with the property that $\psi(0) = 1$ and $\psi(x) = 0$ for every $x \in \RR^n$ with $\Vert x \Vert_2 > \frac{1}{4}$. We then choose
    \begin{equation*}
        c_1 = c_1(n,k) \defeq \left(\left\Vert \psi\right\Vert_{C^k\left([-1,1]^n;\RR\right)}\right)^{-1}.
    \end{equation*}
    Now, let $\varepsilon\in (0, \frac{1}{2})$ and $m \in \NN$ be arbitrary with the property stated in the formulation of the theorem. If $\varepsilon > \frac{c_1}{2}\cdot \frac{1}{6^k}$, then $m \geq c \cdot \frac{\varepsilon ^{-n/k}}{\ln (1 / \varepsilon)}$ trivially holds (as long as $c = c(n,k)> 0$ is sufficiently small). Hence, we can assume that $\varepsilon \leq \frac{c_1}{2} \cdot \frac{1}{6^k}$. Now, let $N$ be the smallest integer with $N \geq 2$, for which
    \begin{equation*}
        \frac{c_1}{2^{k+1}} \cdot N^{-k} \leq \varepsilon.
    \end{equation*}
    Note that this implies 
\begin{equation*}
N^k \geq \frac{c_1}{\varepsilon}\cdot \frac{1}{2^{k+1}} \geq \frac{c_1}{2^{k+1}} \cdot \frac{2}{c_1} \cdot 6^k = 3^k
\end{equation*}
 and hence $N \geq 3$, whence $N-1 \geq 2$. Therefore, by minimality of $N$, and since $\frac{N}{2} \leq N-1 $ because of $N \geq 2$, it follows that
    \begin{equation}
    \label{epsilonbound}
        \varepsilon < \frac{c_1}{2^{k+1}} \cdot (N-1)^{-k} \leq \frac{c_1}{2^{k+1}} 2^k \cdot N^{-k} = \frac{c_1}{2} \cdot N^{-k}.
    \end{equation}
    Now, for every $\alpha \in \{-N, ..., N\}^n$ pick $z_\alpha \in \{0,1\}$ arbitrary and let $y_\alpha \defeq z_\alpha c_1 N^{-k}$. Define the function
    \begin{equation*}
        f(x) \defeq \sum_{\alpha \in \{-N, ..., N\}^n} y_\alpha \cdot \psi \left( N x - \alpha\right), \quad x \in \RR^n.
    \end{equation*}
    Clearly, $f \in C^\infty (\RR^n; \RR)$. Furthermore, since the supports of the functions $\psi(\bullet - \alpha), \ \alpha \in \ZZ^n$ are pairwise disjoint, we see for any multi-index $\kk \in \NN_0^n$ with $\vert \kk \vert \leq k$ that
    \begin{align*}
        \left\Vert \partial^\kk f\right\Vert_{L^\infty \left([-1,1]^n; \RR\right)} &\leq N^{\vert \kk \vert} \cdot \underset{\alpha}{\max} \left\vert y_\alpha \right\vert \cdot \left\Vert \partial ^{\kk}\psi\right\Vert_{L^\infty \left([-1,1]^n; \RR\right)} \\
        &\leq N^{ k} \cdot \underset{\alpha}{\max} \left\vert y_\alpha \right\vert \cdot \left\Vert \psi\right\Vert_{C^k \left([-1,1]^n; \RR\right)} \leq 1,
    \end{align*}
    so we conclude that $\left\Vert f \right\Vert_{C^k \left([-1,1]^n; \RR\right)} \leq 1$. Additionally, for any fixed $\beta \in \{-N, ..., N\}^n$ we see 
    \begin{equation*}
        f\left(\frac{\beta}{N}\right) = y_\beta,
    \end{equation*}
    by choice of $\psi$. 

    By assumption, we can choose suitable coefficients $\rho_1, ..., \rho_m \in \RR^n$, $\eta_1, ..., \eta_m \in \RR$ and furthermore $\sigma_1, ..., \sigma_m \in \RR$ such that
    \begin{equation*}
        \left\Vert f - g \right\Vert_{L^\infty \left([-1,1]^n ; \RR\right)} \leq \varepsilon
    \end{equation*}
    for
    \begin{equation*}
   	g \defeq \sum_{j=1}^m \sigma_j \cdot \phi \left( \rho_j^T \cdot \bullet + \eta_j\right).
    \end{equation*}
Letting 
\begin{equation}  \label{gform}
\tilde{g} \defeq  g(\bullet / N) = \sum_{j=1}^m \sigma_j \cdot \phi \left( \frac{\rho_j^T}{N} \cdot\bullet + \eta_j\right),
\end{equation}
    we see for every $\alpha \in \{-N, ..., N\}^n$ that
    \begin{align*}
        \tilde{g}(\alpha) = g\left( \frac{\alpha}{N}\right) \begin{cases} \geq y_\alpha - \varepsilon = c_1N^{-k} - \varepsilon \overset{\eqref{epsilonbound}}{>} \left(c_1/2\right) N^{-k},& \text{if }z_\alpha = 1, \\ \leq y_\alpha + \varepsilon 
 \overset{\eqref{epsilonbound}}{<} \left ( c_1/2\right)N^{-k},& \text{if }z_\alpha = 0.\end{cases}
    \end{align*}
    Therefore, we get $\mathbbm{1}\left(\tilde{g} > (c_1/2)N^{-k}\right)\left(\alpha\right) = z_\alpha$ for any $\alpha \in \{-N, ..., N\}^n$. Since the choice of $z_\alpha$ has been arbitrary, it follows that the set
    \begin{equation*}
        H \defeq \left\{ \fres{\mathbbm{1}\left(\tilde{g} > (c_1/2)N^{-k}\right)}{\{-N,...,N\}^n} : \ \tilde{g} \text{ of form } \eqref{gform} \right\}
    \end{equation*}
    shatters the whole set $\{-N, ..., N\}^n$. Therefore, we conclude that
    \begin{equation*}
        \text{VC}(H) \geq (2N+1)^n \geq N^n.
    \end{equation*}
    On the other hand,  \cite[Theorem 8.11]{anthony_neural_1999} shows that
    \begin{equation*}
        \text{VC}(H) \leq 2m(n+2)\log_2 (60n\cdot N) \leq c_3 \cdot m \cdot \ln(N)
    \end{equation*}
    with a suitably chosen constant $c_3 = c_3(n)$. Here we used that $N \geq 3$ so that $\ln(N) \geq \ln(3) > 0$. Combining those two inequalities yields 
    \begin{equation*}
        m \geq \frac{N^n}{c_3 \cdot \ln(N)}.
    \end{equation*}
    Using that $N \geq c_4 \cdot \varepsilon^{-1/k}$ with $c_4 \defeq c_4(n,k) = \left(\frac{c_1}{2^{k+1}}\right)^{1/k}$ and $N \leq c_5 \cdot \varepsilon^{-1/k}$ with the definition $c_5 \defeq c_5(n,k) = \left(\frac{c_1}{2}\right)^{1/k}$, we see that
    \begin{equation*}
        m \geq \frac{c_4^n \cdot \varepsilon^{-n/k}}{c_3 \cdot \ln \left(c_5 \cdot \varepsilon^{-1/k}\right)} \geq c_6 \cdot \frac{\varepsilon^{-n/k}}{ \ln \left(1/\varepsilon\right)}
    \end{equation*}
    with $c_6=c_6(n,k)>0$ chosen appropriately.
\end{proof}
\begin{figure}[t]
\centering
\includegraphics[trim=2.7cm 2cm 0 6cm, width=1.2\textwidth, clip]{proof_illus.pdf}
\caption{Illustration of the function $f$ considered in the proof of \Cref{sigmoidoptimality}}
\end{figure}
As a corollary, we get a similar result for complex-valued neural networks.
\begin{corollary}
\label{sigmoidcomplex}
    Let $n,k \in \NN$ and 
\begin{equation*}
    \phi: \quad \CC \to \CC, \quad \phi(z) \defeq \frac{1}{1+e^{-\RE(z)}}.
\end{equation*}
Then there exists a constant $c = c(n,k) > 0$ with the following property: If $\varepsilon \in (0, \frac{1}{2})$ and $m \in \NN$ are such that for every function $f \in C^k \left(\Omega_n ; \CC\right)$ with $\left\Vert f \right\Vert_{C^k\left(\Omega_n; \CC\right)} \leq 1$ there exist coefficients $\rho_1, ..., \rho_m \in \CC^n$, $\eta_1, ..., \eta_m \in \CC$ and $\sigma_1, ..., \sigma_m \in \CC$ satisfying
\begin{equation*}
    \left\Vert f (z)- \sum_{j=1}^m \sigma_j \cdot \phi \left( \rho_j^T z + \eta_j\right)\right\Vert_{L^\infty \left(\Omega_n ; \CC\right)} \leq \varepsilon,
\end{equation*}
then necessarily 
\begin{equation*}
    m \geq c \cdot \frac{\varepsilon^{-2n/k}}{  \mathrm{ln}\left( 1/\varepsilon\right)}.
\end{equation*}
\end{corollary}
\begin{proof}
    We choose the constant $c=c(2n,k)$ according to the previous \Cref{sigmoidoptimality} and let $\varphi_n$ as in \eqref{isomorphism_intro}. Then, let $\varepsilon \in (0, \frac{1}{2})$ and $m \in \NN$ with the properties assumed in the statement of the corollary. If we then take an arbitrary function $f \in C^k \left([-1,1]^{2n}; \RR\right)$ with $\left\Vert f \right\Vert_{C^k \left([-1,1]^{2n}; \RR\right)} \leq 1$, we deduce the existence of $\rho_1,..., \rho_m \in \CC^n$, $\eta_1, ..., \eta_m \in \CC$ and $\sigma_1, ..., \sigma_m \in \CC$, such that
    \begin{align*}
        &\left\Vert f(x)  - \RE \left( \sum_{j=1}^m \sigma_j \cdot \phi \left( \rho_j^T \cdot \varphi_n(x) + \eta_j\right)\right)  \right\Vert_{L^\infty \left([-1,1]^{2n}; \RR\right)} \\
        \leq & \left\Vert (f \circ \varphi_n^{-1})(z) - \sum_{j=1}^m \sigma_j \cdot \phi \left( \rho_j^T z + \eta_j\right)\right\Vert_{L^\infty \left(\Omega_n ; \CC\right)} \leq \varepsilon.
    \end{align*}
    In the next step, we show that 
    \begin{equation*}
        \RR^{2n} \ni x \mapsto \RE \left( \sum_{j=1}^m \sigma_j \cdot \phi \left( \rho_j^T \cdot \varphi_n(x) + \eta_j\right)\right)  
    \end{equation*}
    is a real-valued shallow neural network with $m$ neurons in the hidden layer and the real sigmoid function as activation function. Then the claim follows using \Cref{sigmoidoptimality}. 

    For every $j \in \{1,...,m\}$ we pick a matrix $\tilde{\rho_j} \in \RR^{2n \times 2}$ with the property that one has
    \begin{equation*}
        \tilde{\rho_j} ^T \cdot \varphi_n^{-1}(z) = \varphi_1^{-1} \left(\rho_j^T \cdot z\right) 
    \end{equation*}
    for every $z \in \CC^n$. This is possible, since this is equivalent to 
\begin{equation*}
\tilde{\rho_j}^T v = \varphi_1^{-1} (\rho_j^T \varphi_n(v))
\end{equation*}
for all $v \in \RR^{2n}$, where the right-hand side is an $\RR$-linear map $\RR^{2n} \to \RR^2$. Denoting the first column of $\tilde{\rho_j}$ by $\hat{\rho_j}$, we get
    \begin{equation*}
        \hat{\rho_j} ^T \cdot \varphi_n^{-1}(z) = \RE \left(\rho_j^T \cdot z\right) \quad \text{for all } z \in\CC^n.
    \end{equation*}
    Writing $\gamma$ for the classical real-valued sigmoid function (i.e. $\gamma(x) = \frac{1}{1 + e^{-x}}$), we deduce for arbitrary $x \in \RR^{2n}$ that
    \begin{align*}
        \RE \left( \sum_{j=1}^m \sigma_j \cdot \phi \left( \rho_j^T \cdot \varphi_n(x) + \eta_j\right)\right) &= \RE \left( \sum_{j=1}^m \sigma_j \cdot \gamma\left( \RE \left( \rho_j^T \cdot \varphi_n(x) + \eta_j\right)\right)\right)  \\
        &= \RE \left( \sum_{j=1}^m \sigma_j \cdot \gamma\left(  \hat{\rho_j}^T x + \RE\left(\eta_j\right)\right)\right)  \\
        &= \sum_{j=1}^m \RE\left(\sigma_j\right) \cdot \gamma\left(  \hat{\rho_j}^T x + \RE\left(\eta_j\right)\right),
    \end{align*}
    where in the last step we used that $\gamma$ is real-valued. As noted above, this completes the proof.
\end{proof}

Now, we can finally prove \Cref{main_5}.

\renewcommand*{\proofname}{Proof of \Cref{main_5}}
\medskip
\begin{proof}
Let $\alpha = \frac{2n}{k}$ and choose $c_2 = c_2(\alpha) = c_2(n,k) > 0$ such that the inequality $\ln(x) \leq c_2 \cdot x^{\alpha /2}$ holds for all $x \geq 1$. Furthermore, let $c_1 = c_1(n,k) > 0$ as in \Cref{sigmoidcomplex}. By choosing $c = c(n,k) > 0$ sufficiently small , we can ensure that $\eps_m \defeq c \cdot (m \cdot \ln(m))^{-k/(2n)}$ satisfies 
\begin{equation*}
\ln \left(\frac{c_1}{c_2}\right) + \frac{\alpha}{2} \ln (1/\eps_m) \geq \frac{\alpha}{4} \cdot \ln(1/\eps_m) \quad \text{for all } m \in \NN_{\geq 2}.
\end{equation*}
    By further shrinking $c = c(n,k)>0$ if necessary, we may assume 
\begin{equation*}
c \cdot (2 \cdot \ln (2))^{-k/(2n)} < \frac{1}{2}
\end{equation*}
 and hence $c \cdot (m \cdot \ln (m))^{-k/(2n)} < \frac{1}{2}$ for all $m \in \NN_{\geq 2}$. Finally, setting $c_3 \defeq \frac{\alpha}{4}$ and shrinking $c$ even further, we can arrange that $c^\alpha < c_1 \cdot c_3$. Now, assume towards a contradiction that for every $f \in C^k \left(\Omega_n ; \CC\right)$ with $\Vert f \Vert_{C^k \left(\Omega_n; \CC\right)} \leq 1$ there are coefficients $\rho_1, ..., \rho_m \in \CC^n, \sigma_1, ..., \sigma_m \in \CC$ and $\eta_1, ..., \eta_m \in \CC$ such that
    \begin{equation*}
        \left\Vert f(z) - \sum_{j=1}^m \sigma_j \cdot \phi \left( \rho_j^T z + \eta_j\right)\right\Vert_{L^\infty \left(\Omega_n ; \CC\right)} < c \cdot \left(m \cdot \ln (m)\right)^{-k/(2n)}.
    \end{equation*}
    Applying \Cref{sigmoidcomplex}, we then get
    \begin{equation*}
        m \geq c_1 \cdot \frac{\varepsilon^{-2n/k}}{  \mathrm{ln}\left( 1/\varepsilon\right)}
    \end{equation*}
    with $\varepsilon = c \cdot \left(m \cdot \ln{m}\right)^{-k/(2n)} \in (0, \frac{1}{2})$ and $c_1 = c_1(n,k) > 0$. Recall from the beginning of the proof that $\alpha := 2n/k$ and that $\ln(x) \leq c_2 x^{\alpha / 2}$ for every $x \geq 1$. We observe
    \begin{equation*}
        m \geq c_1 \cdot \frac{\varepsilon^{-\alpha}}{  \mathrm{ln}\left( 1/\varepsilon\right)} \geq \frac{c_1}{c_2} \varepsilon^{- \alpha /2},
    \end{equation*}
    which implies
    \begin{equation*}
        \ln(m) \geq \ln \left(\frac{c_1}{c_2}\right) + \frac{\alpha}{2} \cdot \ln(1/\varepsilon) \geq \frac{\alpha}{4} \cdot \ln(1/\varepsilon) = c_3 \cdot \ln(1/\varepsilon) .
    \end{equation*}
    Overall we then get
    \begin{equation*}
        m \geq c_1 \cdot \frac{\varepsilon^{-\alpha}}{\ln(1/\varepsilon)} = c_1 \cdot c^{-\alpha} \cdot \frac{m \cdot \ln(m)}{\ln(1/\varepsilon)} \geq c_1 \cdot c_3 \cdot c^{-\alpha} \cdot \frac{m \cdot \ln(m)}{\ln(m)} = c_1 \cdot c_3 \cdot c^{-\alpha} \cdot m.
    \end{equation*}
    Since we chose $c$ such that $c^\alpha < c_1 \cdot c_3$ we get the desired contradiction.
\end{proof}
\renewcommand*{\proofname}{Proof}

\section{Proof of Theorem \ref{thm:intrac}} \label{sec:intrac}

The following result is the key to proving \Cref{thm:opti_conti} and can be found in \cite{devore_optimal_1989}.
For completion, its (short) proof is presented in \Cref{prelim_opti}. 

\begin{proposition}[{\cite[Theorem 3.1]{devore_optimal_1989}}] \label{thm: devore_pre}
Let $(X, \Vert \cdot \Vert_X)$ be a normed space,
$\emptyset \neq K \subseteq X$ a subset and $V \subseteq X$ a linear,
not necessarily closed subspace of $X$ containing $K$.
Let $m\in \NN$, let $\overline{a} : K \to \RR^m$ be a map which is continuous
with respect to some norm $\Vert \cdot \Vert_V$ on $V$ and $M : \RR^m \to X$ some arbitrary map.
Let
\begin{equation} \label{eq:bmkx2}
  b_m(K)_X
  \defeq \underset{X_{m+1}}{\sup}
           \sup
             \left\{\varrho \geq 0: \  U_\varrho(X_{m+1}) \subseteq K\right\},
\end{equation} 
where the first supremum is taken over all $(m+1)$-dimensional linear subspaces $X_{m+1}$ of $X$ and
\begin{equation*}
  U_\varrho(X_{m+1})
  \defeq \{y \in X_{m+1} : \ \Vert y \Vert_X \leq \varrho\}.
\end{equation*}
Further, we set $b_m(K)_X \defeq 0$ if the supremum in \eqref{eq:bmkx2}
is not well-defined as a quantity in $[0, \infty]$.
Then it holds
\begin{equation*}
  \underset{x \in K}{\sup} \Vert x - M (\overline{a}(x)) \Vert_X \geq b_m(K)_X.
\end{equation*}
\end{proposition}

\begin{proof}
See \Cref{thm: devore}.
\end{proof}

We can use \Cref{thm: devore_pre} not only to show that the rate of convergence established in this paper is optimal (which is done in \Cref{optimality_section,prelim_opti}) but also to show that the problem of approximating $C^k$-functions using a set of functions that can be parametrized with finitely many parameters is intractable in the sense that it suffers from the curse of dimensionality, provided that the map which assigns to each $C^k$-function the parameters of the approximating function is continuous. This is the subject of this section. 

In \cite{NOVAK2009398} a certain space of polynomials was used to show the intractability in the case of \emph{linear} approximation methods. We are also going to use this class of polynomials, but combine it with \Cref{thm: devore_pre} to infer intractability in the case of \emph{continuous} approximation methods. We start with a lemma discussing an important property of this space of polynomials.
This property is stated as part of a proof in \cite{NOVAK2009398}, but no complete proof is provided.
\begin{lemma}\label{lem:poly_class}
Let $s \in \NN$ and consider a function $f \in C^\infty([-1,1]^s; \RR)$ which is given via
\begin{equation}\label{eq:form_poly_0_1}
f(x) = \sum_{\kk \in \{0,1\}^s} a_\kk x^\kk
\end{equation}
with coefficients $a_\kk \in \RR$ for every $\kk \in \{0,1\}^s$. Then it holds
\begin{equation*}
\Vert f \Vert_{C^k ([-1,1]^s; \RR)} = \Vert f \Vert_{L^\infty([-1,1]^s; \RR)}
\end{equation*}
for every $k \in \NN$.
\end{lemma}
\begin{proof}
The proof is by induction over $s$. We start with the case $s=1$ and note that we can write $f(x) = ax + b$ with $a,b \in \RR$ in that case. Switching to $-f$ if necessary, we can assume $a \geq 0$. Clearly, $\Vert f \Vert_{L^\infty([-1,1]; \RR)} \leq \vert a \vert + \vert b \vert$. Conversely, if $b \geq 0$ then $\vert f(1) \vert = \vert a + b \vert = \vert a \vert + \vert b \vert$. If otherwise $b < 0$ then $\vert f(-1) \vert = \vert b  - a \vert = \vert a - b \vert = \vert a \vert + \vert b \vert$. Thus, $\Vert f \Vert_{L^\infty([-1,1]; \RR)} = \vert a \vert + \vert b \vert$. For the derivatives, we have $\Vert f' \Vert_{L^\infty([-1,1]; \RR)} = \vert a \vert$ and $\Vert f^{(k)} \Vert_{L^\infty([-1,1]; \RR)} = 0$ for $k\geq 2$. This proves the claim in the case $s=1$. 

We now assume that the claim holds for some arbitrary but fixed $s \in \NN$. We further let $\alpha \in \NN_0^{s+1}$ and \emph{fix} a point $(x_1, ..., x_{s+1}) \in [-1,1]^{s+1}$. We decompose $\alpha = (\alpha', \alpha_{s+1})$ with $\alpha' \in \NN_0^s$. Let
\begin{equation*}
\widetilde{f}: \quad [-1,1] \to \RR, \quad y_{s+1} \mapsto \partial^{(\alpha',0)}f(x_1, ..., x_s, y_{s+1})
\end{equation*}
and note 
\begin{equation*}
\partial^\alpha f(x_1, ..., x_{s+1}) =  \widetilde{f}^{(\alpha_{s+1})} (x_{s+1}).
\end{equation*}
Note that $f$ is affine-linear with respect to each variable (with all other variables hold fixed). Hence, $\widetilde{f}$ is an affine function and we can thus apply the case $s=1$ to $\widetilde{f}$ and get
\begin{equation*}
\Vert \widetilde{f}^{(\alpha_{s+1})} \Vert_{L^\infty([-1,1]; \RR)} \leq \Vert \widetilde{f} \Vert_{L^\infty([-1,1]; \RR)}.
\end{equation*}
Putting this together, we infer
\begin{equation}\label{eq:jj1}
\vert  \partial^\alpha f(x_1, ..., x_{s+1}) \vert \leq \Vert \widetilde{f} \Vert_{L^\infty([-1,1]; \RR)} = \underset{y_{s+1} \in [-1,1]}{\sup} \vert \partial^{(\alpha',0)}f(x_1, ..., x_s, y_{s+1})\vert.
\end{equation}
We now \emph{fix} an arbitrary point $y_{s+1} \in [-1,1]$ and consider
\begin{equation*}
\widehat{f}: \quad [-1,1]^s \to \RR, \quad (y_1, ..., y_{s}) \mapsto f(y_1, ..., y_s, y_{s+1}).
\end{equation*}
Then it holds
\begin{equation*}
\partial^{(\alpha',0)}f(x_1, ..., x_s, y_{s+1}) = \partial^{\alpha'} \widehat{f}(x_1, ..., x_s).
\end{equation*}
Applying the induction hypothesis to $\widehat{f}$ (which is easily seen to be of the form \eqref{eq:form_poly_0_1}) we get
\begin{equation}\label{eq:jj2}
\vert \partial^{\alpha'} \widehat{f}(x_1, ..., x_s)\vert \leq \Vert \widehat{f} \Vert_{L^\infty([-1,1]^s ; \RR)} \leq \Vert f \Vert_{L^\infty([-1,1]^{s+1}; \RR)}.
\end{equation}
Combining \eqref{eq:jj1} and \eqref{eq:jj2} yields
\begin{equation*}
\vert  \partial^\alpha f(x_1, ..., x_{s+1}) \vert \leq \Vert f \Vert_{L^\infty([-1,1]^{s+1}; \RR)}.
\end{equation*}
Since $\alpha \in \NN_0^{s+1}$ was arbitrary, we get the claim by noting that 
\begin{equation*}
\Vert f \Vert_{C^k ([-1,1]^s; \RR)} \geq \Vert f \Vert_{L^\infty([-1,1]^s; \RR)}
\end{equation*}
holds trivially for every $k \in \NN$.
\end{proof}
Using the above lemma, we can now deduce that the approximation of smooth functions using continuous approximation methods is intractable in terms of the input dimension.
\medskip
\renewcommand*{\proofname}{Proof of \Cref{thm:intrac}}
\begin{proof}
We apply \Cref{thm: devore_pre} to $X \defeq C([-1,1]^s ; \RR)$, $V \defeq C^{\infty, \ast,s}$ and to the set $K \defeq \{f \in C^{\infty, \ast, s}: \  \Vert f \Vert_{C^\infty ([-1,1]^s ; \RR)} \leq 1\}$ and $m \defeq 2^s - 1$. The space 
\begin{equation*}
X_{m+1} \defeq \left\{ [-1,1]^s \ni x \mapsto \sum_{\kk \in \{0,1\}^s} a_\kk x^\kk: \ a_\kk \in \RR\right\}
\end{equation*}
consisting of all functions considered in the previous \Cref{lem:poly_class} is an $(m+1)$-dimensional subspace of $C([-1,1]^s ; \RR)$. For every $f \in X_{m+1}$ with $\Vert f \Vert_{L^\infty([-1,1]^s ; \RR)} \leq 1$, \Cref{lem:poly_class} tells us $\Vert f \Vert_{C^\infty([-1,1]^s ; \RR)} \leq 1$. Hence, $U_1(X_{m+1}) \subseteq K$ and \Cref{thm: devore_pre} then yields the claim.
\end{proof}
\begin{remark}
The statement of \Cref{thm:intrac} also holds if the functions satisfy $\overline{a}: C^{\infty, *, s} \to \RR^m$ and $M: \RR^m \to C([-1,1]^s;\RR)$ with $m \leq 2^s - 1$. This can be seen by defining 
\begin{equation*}
\tilde{a} : \quad C^{\infty, *, s} \to \RR^{2^s -1}, \quad f \mapsto (\overline{a}(f), 0, ..., 0)
\end{equation*}
and
\begin{equation*}
\widetilde{M}: \quad \RR^{2^s -1} \to C([-1,1]^s;\RR), \quad (a,b) \mapsto M(a)
\end{equation*}
with $a \in \RR^m$ and $b \in \RR^{2^s -1 -m}$.
\end{remark}
\renewcommand*{\proofname}{Proof}
The following \Cref{corr:intrac_complex} transfers \Cref{thm:intrac} to the complex-valued setting.
\begin{corollary}\label{corr:intrac_complex}
Let $n \in \NN$. For any function $f \in C^\infty(\Omega_n ; \CC)$ we write
\begin{equation*}
\Vert f \Vert_{C^\infty(\Omega_n ; \CC)} \defeq \underset{k \in \NN}{\sup} \ \Vert f \Vert_{C^k(\Omega_n; \CC)}
\end{equation*}
and let $C^{\infty, *, n}_{\CC}$ denote the space consisting of all functions for which this expression is finite. Let $\overline{a}:  C^{\infty, *, n}_{\CC} \to \CC^{2^{2n-1} - 1}$ be continuous with respect to some norm on $C^{\infty, *, n}_{\CC}$ and moreover, let $M: \CC^{2^{2n-1}-1} \to C(\Omega_n; \CC)$ be an arbitrary map. Then it holds
\begin{equation*}
\underset{\Vert f \Vert_{C^\infty(\Omega_n ; \CC)} \leq 1}{\underset{f \in C^{\infty, *, n}_{\CC}}{\sup}} \Vert f - M(\overline{a}(f))\Vert_{L^\infty(\Omega_n ; \CC)} \geq 1.
\end{equation*}
\end{corollary}
\begin{proof}
The transfer to the complex-valued setting works in the same manner as the proof of \Cref{thm:opti_conti} (see \Cref{optimality_section}). We write $m \defeq 2^{2n-1}-1$ and note $2m = 2^{2n} - 2 \leq 2^{2n}-1$. We define $\tilde{a} : C^{\infty, \ast, 2n}\to \RR^{2m}$ and $\widetilde{M}: \RR^{2m} \to C([-1,1]^{2n}; \RR)$ in the same way as in the proof of \Cref{thm:opti_conti}. Using again the same technique as in the proof of \Cref{thm:opti_conti}, we get
\begin{equation*}
\underset{\Vert f \Vert _{C^\infty (\Omega_n ; \CC)} \leq 1}{\underset{f \in C^{\infty, *, n}_{\CC}}{\sup}} \Vert f - M (\overline{a}(f)) \Vert_{L^\infty (\Omega_n; \CC)} \geq  \underset{\Vert \tilde{f} \Vert _{C^\infty ([-1,1]^{2n} ; \RR)} \leq 1}{\underset{\tilde{f} \in C^{\infty, \ast, 2n}}{\sup}} \left\Vert \tilde{f} - \widetilde{M}(\tilde{a}(\tilde{f})) \right\Vert_{L^\infty ([-1,1]^{2n}; \RR)} \geq 1,
\end{equation*}
applying \Cref{thm:intrac} in the last inequality, using $2m \leq 2^{2n}-1$.
\end{proof}
We conclude this appendix by adding a note on the constant appearing in our main approximation bound.
\begin{corollary} \label{corr:const_intrac}
Let $n \in \NN$ with $n \geq 2$ and $\alpha > 0$ and let $\phi \in C(\CC;\CC)$. Let $\tilde{c} = \tilde{c}(n,\alpha)>0$ be such that for every $m \in \NN$ there exists a mapping
\begin{equation*}
        \eta : \quad  C^{\infty, *, n}_{\CC} \to \left(\CC^n\right)^m \times \CC^m \times \CC^m, \quad g \mapsto \left(\eta_1(g), \eta_2(g), \eta_3(g)\right)
\end{equation*}
that is continuous with respect to any norm on $C^{\infty, *, n}_{\CC}$ and such that
\begin{equation*}
        \left\Vert f - \Psi(f)\right\Vert_{L^\infty \left(\Omega_n ; \CC\right)} \leq \left( \tilde{c} \cdot m\right)^{-\alpha} \cdot \Vert f \Vert_{C^\infty(\Omega_n; \CC)},
    \end{equation*}
    for every $f \in C^{\infty, *, n}_{\CC}$. Here, $\Psi(f) \in C(\Omega_n; \CC)$ is given by
\begin{equation*}
\Psi(f)(z) \defeq \sum_{j=1}^m \left(\eta_3(f)\right)_j \phi \left(\left[\eta_1 (f)\right]_j^T z + \left(\eta_2(f)\right)_j\right).
\end{equation*}
 Then it necessarily holds $\tilde{c}\leq 16 \cdot 2^{-n}$.
\end{corollary}
\begin{proof}
We first assume $n \geq 4$. We take $m= \left\lfloor \frac{2^{2n - 1} - 1}{n+2} \right\rfloor$ and note that then $m(n+2) \leq 2^{2n-1}-1$. Therefore, \Cref{corr:intrac_complex} applies and we infer that for each $\eps \in (0,1)$, there exists $f = f_\eps \in C^{\infty, *, n}_{\CC}$ with $\Vert f \Vert_{C^\infty(\Omega_n;\CC)} \leq 1$ and such that
\begin{equation*}
1 - \eps \leq \left\Vert f - \Psi(f)\right\Vert_{L^\infty \left(\Omega_n ; \CC\right)} \leq \left( \tilde{c} \cdot m\right)^{-\alpha} \cdot \Vert f \Vert_{C^\infty(\Omega_n; \CC)} \leq \left( \tilde{c} \cdot m\right)^{-\alpha}.
\end{equation*}
This then necessarily implies $\tilde{c} \cdot m \leq 1$ or equivalently $\tilde{c}\leq 1/m$. It therefore suffices to derive a lower bound for $m$. Firstly, we note
\begin{equation*}
2^{2n-1} = 2^{n-3} \cdot 2^{n+2} = 2^{n-3} \cdot (1+1)^{n+2} \geq 2^{n-3}(n+3),
\end{equation*}
where we applied Bernoulli's inequality. Because of $n \geq 4 \geq 3$, this yields
\begin{equation*}
2^{2n-1} - 1 \geq 2^{n-3}(n+3) - 2^{n-3} =  2^{n-3} (n+2).
\end{equation*}
Hence, we get
\begin{equation*}
m \geq \frac{2^{2n - 1} - 1}{n+2} -1  \geq 2^{n-3} -1 = 2^{n-3}(1- 2^{3-n}) \geq 2^{n - 4} = \frac{2^n}{16}.
\end{equation*}
Here, we used $n \geq 4$ in the last inequality. An explicit computation shows that the same bounds also holds in the cases $n=2$ and $n=3$. This proves the claim. 
\end{proof}

