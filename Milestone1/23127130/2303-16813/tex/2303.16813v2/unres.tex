
\section{Postponed proofs for the optimality results in the case of unrestricted weight selection}


\subsection{Approximation using Ridge Functions}
\label{sec: ridge_reordered}
In this section we prove for $s \in \NN_{\geq 2}$ that every function in $C^k([-1,1]^s; \RR)$ can be uniformly approximated with an error of the order $m^{-k/(s-1)}$ using a linear combination of $m$ so-called \emph{ridge functions}. In fact, we only consider ridge \emph{polynomials}, meaning functions of the form
\begin{equation*}
\RR^s \to \RR, \quad x \mapsto p(a^T x)
\end{equation*}
for a fixed vector $a \in \RR^s$ and a polynomial $p: \RR \to \RR$. Note that this result has already been obtained in a slightly different form in \cite[Theorem 1]{maiorov_best_1999}; namely, it is shown there that the rate of approximation $m^{-k/(s-1)}$ can be achieved by functions of the form $\sum_{j=1}^m f_j(a_j^T x)$ with $a_j \in \RR^s$ and $f_j \in L^1_\text{loc} (\RR^s)$. We will need the fact that the $f_j$ can actually be chosen as polynomials and that the vectors $a_1, ..., a_m$ can be chosen independently from the particular function $f$. This is shown in the proof of \cite{maiorov_best_1999}, but not stated explicitly. For this reason, and in order to clarify the proof itself and to make the paper more self-contained, we decided to present the proof in this appendix.

\begin{lemma} \label{lem: hom_dim}
Let $m,s \in \NN$. Then we denote by
\begin{equation*}
P_m^s \defeq \left\{ \RR^s \to \RR, \quad x \mapsto \underset{\vert \kk \vert \leq m}{\sum_{\kk \in \NN_0^s}} a_\kk x^\kk : \ a_\kk \in \RR\right\}
\end{equation*}
the set of real polynomials of degree at most $m$. The subset of \emph{homogeneous} polynomials of degree $m$ is defined as
\begin{equation*}
H_m^s \defeq \left\{ \RR^s \to \RR, \quad x \mapsto \underset{\vert \kk \vert = m}{\sum_{\kk \in \NN_0^s}} a_\kk x^\kk : \ a_\kk \in \RR\right\}.
\end{equation*}
Then there exists a constant $c = c(s) > 0$ satisfying
\begin{equation*}
\dim (H_m^s) \leq c \cdot m^{s-1} \quad \forall m \in \NN.
\end{equation*}
\end{lemma}
\begin{proof}
It is immediate that the set
\begin{equation*}
\left\{ \RR^s \to \RR, \ x \mapsto x^\kk: \  \kk \in \NN_0^s, \ \vert \kk \vert =m\right\}
\end{equation*}
forms a basis of $H_m^s$, hence 
\begin{equation*}
\dim (H_m^s) = \# \left\{ \kk \in \NN_0^s: \ \vert \kk \vert = m\right\}.
\end{equation*}
This quantity clearly equals the number of possibilities for drawing $m$ times from a set with $s$ elements with replacement. Hence, we see
\begin{equation*}
\dim (H_m^s) = \binom{s+m-1}{m},
\end{equation*}
see for instance \cite[Identity 143]{benjamin_proofs_2003}. A further estimation shows
\begin{align*}
 \binom{s+m-1}{m} = \prod_{j=1}^{s-1} \frac{m+j}{j} = \prod_{j=1}^{s-1} \left(1 + \frac{m}{j}\right) \leq (1+m)^{s-1} \leq 2^{s-1} \cdot m^{s-1}.
\end{align*}
Hence, the claim follows with $c(s) = 2^{s-1}$.
\end{proof}

A combination of results from \cite{pinkus_ridge_2016} together with the fact that it is possible to approximate $C^k$-functions using polynomials of degree at most $m$ with an error of the order $m^{-k}$, as shown in \Cref{app: fourier_approx}, yields the desired result.

\begin{theorem} \label{app: ridge}
Let $s,k \in \NN$ with $s \geq 2$ and $r>0$. Then there exists a constant $c = c(s,k) > 0$ with the following property: For every $m \in \NN$ there exist $a_1, ..., a_m \in \RR^s \setminus \{0\}$ with $\Vert a_j \Vert_2 = r$, such that for every function $f \in C^k ([-1,1]^s ; \RR)$ there exist polynomials $p_1, ..., p_m \in P_m^1$ satisfying
\begin{equation*}
\left\Vert f(x) - \sum_{j=1}^m p_j (a_j^T x) \right\Vert_{L^\infty ([-1,1]^s ; \RR)} \leq c \cdot m^{-k/(s-1)} \cdot \Vert f \Vert_{C^k([-1,1]^s ; \RR)}.
\end{equation*}  
\end{theorem}
\begin{proof}
We first pick the constant $c_1 = c_1(s) $ according to \Cref{lem: hom_dim}. Then we define the constant $c_2 = c_2(s) \defeq (2s)^{s-1} \cdot c_1(s)$ and let $M \in \NN$ be the largest integer satisfying 
\begin{equation*}
c_2 \cdot M^{s-1} \leq m.
\end{equation*}
Here, we assume without loss of generality that $m \geq c_2$, which can be justified by choosing $p_j = 0$ for every $j \in \{1,...,m\}$ if $m < c_2$, at the cost of possibly enlarging $c$. Note that the choice of $M$ implies $c_2 \cdot (2M)^{s-1} \geq c_2 \cdot (M+1)^{s-1}> m$, and thus
\begin{equation} \label{eq: bound_lower}
M \geq \frac{1}{2} \cdot c_2^{-1/(s-1)} \cdot m^{1/(s-1)} = c_3 \cdot m^{1/(s-1)}
\end{equation}
with $c_3 = c_3(s) \defeq 1/2 \cdot c_2^{-1/(s-1)}$. 

Using \cite[Proposition 5.9]{pinkus_ridge_2016} and \Cref{lem: hom_dim} we can pick $a_1, ..., a_m \in \RR^s \setminus \{0\}$ satisfying
\begin{equation} \label{span: homogeneous}
H_{s(2M-1)}^s = \spann\left\{x \mapsto (a_j^T x )^{s(2M-1)}: \ j \in \{1,...,m\}\right\},
\end{equation}
where we used that
\begin{equation*}
c_1 \cdot (s(2M-1))^{s-1} \leq c_1 \cdot (2s)^{s-1} \cdot M^{s-1} = c_2 \cdot M^{s-1} \leq m. 
\end{equation*}
Here we can assume $\Vert a_j \Vert_2 = r$ for every $j \in \{1,...,m\}$ since multiplying each $a_j$ with a positive constant does not change the span in \eqref{span: homogeneous}.
From \cite[Corollary 5.12]{pinkus_ridge_2016} we infer that
\begin{equation} \label{eq: p_span}
P_{s(2M-1)}^s = \spann\left\{x \mapsto (a_j^T x )^{r}: \ j \in \{1,...,m\}, \ 0 \leq r \leq s(2M-1)\right\}.
\end{equation}

Let $f \in C^k([-1,1]^s ; \RR)$. Then, according to \Cref{app: fourier_approx}, there exists a polynomial $P : \RR^s \to \RR$ of \emph{coordinatewise} degree at most $2M -1$ satisfying
\begin{equation*}
\Vert f - P \Vert_{L^\infty ([-1,1]^s ; \RR)} \leq c_4 \cdot M^{-k} \cdot \Vert f \Vert_{C^k([-1,1]^s ; \RR)},
\end{equation*}
where $c_4 = c_4(s,k) > 0$. Note that by construction it holds $P \in P_{s(2M-1)}^s$. Using \eqref{eq: p_span} we deduce the existence of polynomials $p_1, ..., p_m : \RR \to \RR$ such that
\begin{equation*}
P(x) = \sum_{j= 1}^m p_j(a_j^T x) \quad \text{for all }x \in \RR^s. 
\end{equation*}
Combining the previously shown bounds, we get
\begin{align*}
\left\Vert f(x) - \sum_{j= 1}^m p_j(a_j^T x) \right\Vert_{L^\infty ([-1,1]^s ; \RR)} &= \Vert f(x) - P(x)\Vert_{L^\infty ([-1,1]^s ; \RR)} \leq  c_4 \cdot M^{-k} \cdot \Vert f \Vert_{C^k([-1,1]^s ; \RR)} \\ \overset{\eqref{eq: bound_lower}}&{\leq} c \cdot m^{-k/(s-1)} \cdot \Vert f \Vert_{C^k([-1,1]^s ; \RR)},
\end{align*}
as desired. Here, we defined $c = c(s,k) \defeq c_4 \cdot c_3^{-k}$.
\end{proof}

\subsection{Proof of Theorem \ref{main_4}}
\label{sec:main_4_reordered}

Using \Cref{app: ridge}, we can prove the following statement for complex-valued $C^k$-functions,
which will play an important role in the proof of \Cref{main_4}.

\begin{proposition}\label{ridge_approx}
    Let $n,k \in \NN$. Then there exists a constant $c=c(n,k)>0$ with the following property: For any $m \in \NN$ there exist complex vectors $b_1, ..., b_m \in \CC^n$ with $\left\Vert b_j \right\Vert_2 = 1/ \sqrt{2n}$ for $j = 1,...,m$ and with the property that for any function $f \in C^k \left(\Omega_n ; \CC\right)$ there exist functions $g_1, ..., g_m \in C(\Omega_1; \CC)$ such that
    \begin{equation*}
        \left\Vert f(z) - \sum_{j=1}^m g_j \left( b_j^T \cdot z \right)\right\Vert_{L^\infty \left(\Omega_n; \CC\right)} \leq c \cdot m^{-k /(2n-1)} \cdot \left\Vert f \right\Vert_{C^k \left( \Omega_n; \CC\right)}.
    \end{equation*}
    Note that the vectors $b_1, ... b_m$ can be chosen independently from the considered function $f$, whereas $g_1, ..., g_m$ do depend on $f$.
\end{proposition}

\begin{proof}
     \Cref{app: ridge} yields the existence of a constant $c_1 = c_1(n,k)>0$ with the property that for any $m \in \NN$ there exist real vectors $a_1, ..., a_m \in \RR^{2n}$ with $\left\Vert a_j \right\Vert_2 = 1 / \sqrt{2n}$ such that for any function $\tilde{f} \in C^k \left([-1,1]^{2n}; \RR\right)$ there exist functions $\tilde{g}_1, ..., \tilde{g}_m \in C([-1,1]; \RR)$ satisfying
    \begin{equation*}
        \left\Vert \tilde{f}(x) - \sum_{j=1}^m \tilde{g}_j \left( a_j^T  x \right)\right\Vert_{L^\infty \left([-1,1]^{2n}; \RR\right)} \leq c_1 \cdot m^{-k /(2n-1)} \cdot \Vert \tilde{f} \Vert_{C^k \left( [-1,1]^{2n}; \RR\right)}.
    \end{equation*}
    We then define the vectors $b_1, ..., b_m \in \CC^n$ componentwise via
    \begin{equation*}
        \left(b_j\right)_\ell \defeq \left(a_j\right)_\ell - i \cdot \left(a_j\right)_{n+\ell}, \quad \ell \in \{1,...,n\}, \ j \in \{1,...,m\}.
    \end{equation*}
    First we see $\left\Vert b_j \right\Vert_2 = \left\Vert a_j \right\Vert_2 = 1/\sqrt{2n}$. We first consider real-valued functions, i.e., $f \in C^k \left(\Omega_n; \RR\right)$. Let $\varphi_n$ be defined as in \eqref{isomorphism_intro}. By the choice of the constant $c_1$ we can find continuous functions $\tilde{g}_1,..., \tilde{g}_m \in C \left([-1,1]; \RR\right)$ such that
    \begin{equation*}
        \left\Vert (f \circ \varphi_n) (x)- \sum_{j=1}^m \tilde{g}_j \left( a_j^T x \right)\right\Vert_{L^\infty \left([-1,1]^{2n}\right)} \leq c_1 \cdot m^{-k /(2n-1)} \cdot \left\Vert f \circ \varphi_n \right\Vert_{C^k \left( [-1,1]^{2n}; \RR\right)}.
    \end{equation*}
    We then define $g_j \in C(\Omega_1; \RR)$ by $g_j(z) \defeq \tilde{g}_j \left(\RE(z)\right)$ for any $j \in \{1, ..., m\}$. For $z \in \Omega_n$ we then have
    \begin{align}
        g_j \left(\left( b_j \right)^T z \right) &= \tilde{g}_j \left( \RE \left(\sum_{\ell = 1}^n \left( b_j\right)_\ell \cdot z_\ell\right)\right) \nonumber \\
        &= \tilde{g}_j \left( \RE \left( \sum_{\ell = 1}^n \left(\left(a_j\right)_\ell - i\cdot \left(a_j\right)_{n+\ell}\right)\left(\varphi_n^{-1}(z)_\ell + i\cdot \varphi_n^{-1}(z)_{n+\ell}\right)\right)\right) \nonumber \\
        &= \tilde{g}_j \left(\sum_{\ell = 1}^n\left[\left(a_j\right)_\ell \varphi_n^{-1}(z)_\ell + \left(a_j\right)_{n+\ell} \varphi_n^{-1}(z)_{n+\ell}\right]\right) \nonumber \\
        \label{trafo_ident}
        &= \tilde{g}_j \left( \left( a_j \right)^T \cdot \varphi_n^{-1}(z)\right).
    \end{align}
    Therefore, 
    \begin{align*}
        \left\Vert f(z) - \sum_{j=1}^m g_j \left( b_j^T z\right)\right\Vert_{L^\infty \left(\Omega_n; \RR\right)} &= \left\Vert (f \circ \varphi_n)(x) - \sum_{j=1}^m g_j \left( b_j^T \cdot \varphi_n(x) \right) \right\Vert_{L^\infty \left([-1,1]^{2n}; \RR\right)} \\
        \overset{\text{(\ref{trafo_ident})}}&{=} \left\Vert (f \circ \varphi_n)(x) - \sum_{j=1}^m \tilde{g}_j \left( a_j^T \cdot x \right)\right\Vert_{L^\infty \left([-1,1]^{2n} ; \RR\right)} \\
        &\leq c_1 \cdot m^{-k /(2n-1)} \cdot \left\Vert f \circ \varphi_n\right\Vert_{C^k \left( [-1,1]^{2n}; \RR\right)}.
    \end{align*}
    By the above, for $f \in C^k \left(\Omega_n ; \CC\right)$ we can pick functions $g^{\RE}_1, ..., g^{\RE}_m, g^{\IM}_1, ..., g^{\IM}_m \in C \left(\Omega_1 ; \RR \right)$ satisfying
    \begin{align*}
        \left\Vert \RE(f(z)) - \sum_{j=1}^m g^{\RE}_j \left( b_j^T z \right)\right\Vert_{L^\infty \left(\Omega_n; \RR\right)} &\leq c_1 \cdot m^{-k /(2n-1)} \cdot \left\Vert {\RE} \left(f \circ \varphi_n \right)\right\Vert_{C^k \left( [-1,1]^{2n}; \RR\right)}, \\
        \left\Vert \IM(f(z)) - \sum_{j=1}^m g^{\IM}_j \left( b_j^T z \right)\right\Vert_{L^\infty \left(\Omega_n; \RR\right)} &\leq c_1 \cdot m^{-k /(2n-1)} \cdot \left\Vert \IM \left(f \circ \varphi_n \right)\right\Vert_{C^k \left( [-1,1]^{2n}; \RR\right)}.
    \end{align*}
    Defining $g_j := g^{\RE}_j + i \cdot g^{\IM}_j$ yields
    \begin{equation*}
        \left\Vert f(z)- \sum_{j=1}^m g_j \left(b_j^T z\right)\right\Vert_{L^\infty \left( \Omega_n; \CC\right)} \leq c_1 \cdot \sqrt{2}\cdot m^{-k /(2n-1)} \cdot \left\Vert f \right\Vert_{C^k \left( \Omega_n; \CC\right)},
    \end{equation*}
    completing the proof.
\end{proof}
The special activation function that yields the improved approximation rate of $m^{-k/(2n-1)}$ (see \Cref{main_4}) is constructed in the following lemma. 
\begin{lemma}
\label{special_acti_func}
    Let $\left\{ u_\ell\right\}_{\ell = 1}^\infty$ be an enumeration of the set of complex polynomials in $z$ and $\overline{z}$ with coefficients in $\QQ + i\QQ$.
    Then there exists a function $\phi \in C^\infty \left( \CC; \CC\right)$ with the following properties:
    \begin{enumerate}
        \item For every $\ell \in \NN$ and $z \in \Omega_1$ one has
        \begin{equation*}
            \phi(z+3\ell) = u_\ell(z).
        \end{equation*}
        \item $\phi$ is non-polyharmonic.
    \end{enumerate}
\end{lemma}
\begin{proof}
    Let $\psi \in C^\infty\left(\CC; \RR\right)$ with $0 \leq \psi \leq 1$ and
    \begin{equation*}
        \fres{\psi}{\Omega_1} \equiv 1, \qquad \supp(\psi) \subseteq \widetilde{\Omega},
    \end{equation*}
    where $\widetilde{\Omega} \defeq \left\{ z \in \CC : \ \left\vert \RE \left(z\right)\right\vert, \left\vert \IM \left(z\right) \right\vert < \frac{3}{2} \right\}$. We then define
    \begin{equation*}
        \phi \defeq f \cdot \psi + \sum_{\ell = 1}^\infty u_\ell(\bullet - 3\ell) \cdot \psi(\bullet - 3\ell),
    \end{equation*}
    where $f(z) = e^{\RE(z)}$. Note that $\phi$ is smooth since it is a locally finite sum of smooth functions. Furthermore, $\phi$ is non-polyharmonic on the interior of $\Omega_1$, since the calculation in the proof of \Cref{admissible} shows for $z$ in the interior of $\Omega_1$ and $\rho: \RR \to \RR, \ t \mapsto e^t$ that 
\begin{equation*}
\left\vert \wirt^m \wirtq^\ell \phi (z)\right\vert = \left\vert \wirt^m \wirtq^\ell f (z)\right\vert = \frac{1}{2^{m+\ell}} \left\vert \rho^{(m+ \ell)}(\RE(z))\right\vert>0
\end{equation*}
for arbitrary $m, \ell \in \NN_0$. Finally, property (1) follows directly by construction of $\phi$ because 
    \begin{equation*}
        (\widetilde{\Omega} + 3\ell) \cap (\widetilde{\Omega} + 3\ell') = \emptyset
    \end{equation*}
    for $\ell \neq \ell'$.
\end{proof}

Using the properties of the special activation function constructed in \Cref{special_acti_func}
and applying the approximation result from \Cref{ridge_approx} we can now prove \Cref{main_4}.

\begin{proof}[Proof of \Cref{main_4}]
    Let $\phi$ be the activation function constructed in \Cref{special_acti_func}. We choose the constant $c$ according to Proposition \ref{ridge_approx}. Let $m \in \NN$ and $f \in C^k \left(\Omega_n; \CC\right)$. We can without loss of generality assume that $f \not\equiv 0$. Again, according to \Cref{ridge_approx}, we can choose $\rho_1, ..., \rho_m \in \CC^n$ with $\left\Vert_2 \rho_j \right\Vert = 1 / \sqrt{2n}$ and $g_1, ..., g_m \in C\left(\Omega; \CC\right)$ with the property
    \begin{equation*}
        \left\Vert f(z) - \sum_{j=1}^m g_j \left( \rho_j^T z \right)\right\Vert_{L^\infty \left(\Omega_n\right)} \leq c \cdot m^{-k /(2n-1)} \cdot \left\Vert f \right\Vert_{C^k \left( \Omega_n; \CC\right)}.
    \end{equation*}
    Recall from \Cref{special_acti_func} that $\{u_\ell\}_{\ell = 1}^\infty$ is an enumeration of the set of complex polynomials in $z$ and $\overline{z}$. Hence, using the complex version of the Stone-Weierstraß-Theorem (see for instance \cite[Theorem 4.51]{folland_real_1999}), we can pick $\ell_1, ..., \ell_m \in \NN$ such that
    \begin{equation}
    \label{approxx}
        \left\Vert g_j - u_{\ell_j} \right\Vert_{L^\infty \left(\Omega_1 ; \CC\right)} \leq  m^{-1-k /(2n-1)} \cdot \left\Vert f \right\Vert_{C^k \left( \Omega_n; \CC\right)}
    \end{equation}
    for every $j \in \{1,...,m\}$. Since $\phi\left(\bullet + 3\ell\right) = u_\ell$ on $\Omega_1$ for each $\ell \in \NN$, and since $\rho_j^T z \in \Omega_1$ for $j \in \{1,...,m\}$ and $z \in \Omega_n$, we estimate
    \begin{align*}
        & \norel
\left\Vert f(z) - \sum_{j=1}^m \phi \left( \rho_j^T z + 3\ell_j\right)\right\Vert_{L^\infty \left(\Omega_n; \CC\right)} \\
        &\leq \left\Vert f(z) - \sum_{j=1}^m g_j \left( \rho_j^T z \right)\right\Vert_{L^\infty \left(\Omega_n; \CC\right)} + \sum_{j=1}^m \left\Vert g_j \left( \rho_j^T z\right) - \phi \left( \rho_j^T \cdot z + 3\ell_j\right)\right\Vert_{L^\infty \left(\Omega_n; \CC\right)} \\
        &\leq c \cdot m^{-k /(2n-1)} \cdot \left\Vert f \right\Vert_{C^k \left( \Omega_n; \CC\right)} + \sum_{j=1}^m \left\Vert g_j \left( z\right) - u_{\ell_j} \left( z\right)\right\Vert_{L^\infty \left(\Omega_1; \CC\right)} \\
        \overset{\eqref{approxx}}&{\leq}  c \cdot m^{-k /(2n-1)} \cdot \left\Vert f \right\Vert_{C^k \left( \Omega_n; \CC\right)} + m^{-k /(2n-1)} \cdot \left\Vert f \right\Vert_{C^k \left( \Omega_n; \CC\right)} \\
        &= (c+1) \cdot m^{-k/(2n-1)} \cdot \Vert f \Vert_{C^k \left(\Omega_n; \CC\right)}.
\qedhere
    \end{align*}
\end{proof}



\subsection{Proof of Theorem \ref{main_5}}
As a preparation for the proof of \Cref{main_5}, we first prove a similar result in the real-valued setting. We remark that the proof idea is inspired by the proof of \cite[Theorem 4]{yarotsky_error_2017}.
\label{sec:main_5_reordered}

\begin{theorem}
\label{sigmoidoptimality}
Let $n, k \in \NN$ and 
\begin{equation*}
    \phi: \quad \RR \to \RR, \quad \phi(x) \defeq \frac{1}{1+e^{-x}}
\end{equation*}
be the sigmoid function. Then there exists a constant $c = c(n,k) > 0$ with the following property: If the numbers $\varepsilon \in (0,\frac{1}{2})$ and $m \in \NN$ are such that for every function $f \in C^k \left([-1,1]^n ; \RR\right)$ with $\left\Vert f \right\Vert_{C^k\left([-1,1]^n; \RR\right)} \leq 1$ there exist coefficients $\rho_1, ..., \rho_m \in \RR^n$, $\eta_1, ..., \eta_m \in \RR$ and $\sigma_1, ..., \sigma_m \in \RR$ satisfying
\begin{equation*}
    \left\Vert f(x) - \sum_{j=1}^m \sigma_j \cdot \phi \left( \rho_j^T x + \eta_j\right)\right\Vert_{L^\infty \left([-1,1]^n ; \RR\right)} \leq \varepsilon,
\end{equation*}
then necessarily
\begin{equation*}
    m \geq c \cdot \frac{\varepsilon^{-n/k}}{  \mathrm{ln}\left( 1/\varepsilon\right)}.
\end{equation*}
\end{theorem}

\begin{proof}
    We first pick a function $\psi \in C^\infty \left(\RR^n; \RR\right)$ with the property that $\psi(0) = 1$ and $\psi(x) = 0$ for every $x \in \RR^n$ with $\Vert x \Vert_2 > \frac{1}{4}$. We then choose
    \begin{equation*}
        c_1 = c_1(n,k) \defeq \left(\left\Vert \psi\right\Vert_{C^k\left([-1,1]^n;\RR\right)}\right)^{-1}.
    \end{equation*}
    Now, let $\varepsilon\in (0, \frac{1}{2})$ and $m \in \NN$ be arbitrary with the property stated in the formulation of the theorem. If $\varepsilon > \frac{c_1}{2}\cdot \frac{1}{6^k}$, then $m \geq c \cdot \frac{\varepsilon ^{-n/k}}{\ln (1 / \varepsilon)}$ trivially holds (as long as $c = c(n,k)> 0$ is sufficiently small). Hence, we can assume that $\varepsilon \leq \frac{c_1}{2} \cdot \frac{1}{6^k}$. Now, let $N$ be the smallest integer with $N \geq 2$, for which
    \begin{equation*}
        \frac{c_1}{2^{k+1}} \cdot N^{-k} \leq \varepsilon.
    \end{equation*}
    Note that this implies 
\begin{equation*}
N^k \geq \frac{c_1}{\varepsilon}\cdot \frac{1}{2^{k+1}} \geq \frac{c_1}{2^{k+1}} \cdot \frac{2}{c_1} \cdot 6^k = 3^k
\end{equation*}
 and hence $N \geq 3$, whence $N-1 \geq 2$. Therefore, by minimality of $N$, and since $\frac{N}{2} \leq N-1 $ because of $N \geq 2$, it follows that
    \begin{equation}
    \label{epsilonbound}
        \varepsilon < \frac{c_1}{2^{k+1}} \cdot (N-1)^{-k} \leq \frac{c_1}{2^{k+1}} 2^k \cdot N^{-k} = \frac{c_1}{2} \cdot N^{-k}.
    \end{equation}
    Now, for every $\alpha \in \{-N, ..., N\}^n$ pick $z_\alpha \in \{0,1\}$ arbitrary and let $y_\alpha \defeq z_\alpha c_1 N^{-k}$. Define the function
    \begin{equation*}
        f(x) \defeq \sum_{\alpha \in \{-N, ..., N\}^n} y_\alpha \cdot \psi \left( N x - \alpha\right), \quad x \in \RR^n.
    \end{equation*}
    Clearly, $f \in C^\infty (\RR^n; \RR)$. Furthermore, since the supports of the functions $\psi(\bullet - \alpha), \ \alpha \in \ZZ^n$ are pairwise disjoint, we see for any multi-index $\kk \in \NN_0^n$ with $\vert \kk \vert \leq k$ that
    \begin{align*}
        \left\Vert \partial^\kk f\right\Vert_{L^\infty \left([-1,1]^n; \RR\right)} &\leq N^{\vert \kk \vert} \cdot \underset{\alpha}{\max} \left\vert y_\alpha \right\vert \cdot \left\Vert \partial ^{\kk}\psi\right\Vert_{L^\infty \left([-1,1]^n; \RR\right)} \\
        &\leq N^{ k} \cdot \underset{\alpha}{\max} \left\vert y_\alpha \right\vert \cdot \left\Vert \psi\right\Vert_{C^k \left([-1,1]^n; \RR\right)} \leq 1,
    \end{align*}
    so we conclude that $\left\Vert f \right\Vert_{C^k \left([-1,1]^n; \RR\right)} \leq 1$. Additionally, for any fixed $\beta \in \{-N, ..., N\}^n$ we see 
    \begin{equation*}
        f\left(\frac{\beta}{N}\right) = y_\beta,
    \end{equation*}
    by choice of $\psi$. 
    See also \Cref{fig:VCDimensionFunction} for an illustration of the function $f$.

    By assumption, we can choose suitable coefficients $\rho_1, ..., \rho_m \in \RR^n$, $\eta_1, ..., \eta_m \in \RR$ and furthermore $\sigma_1, ..., \sigma_m \in \RR$ such that
    \begin{equation*}
        \left\Vert f - g \right\Vert_{L^\infty \left([-1,1]^n ; \RR\right)} \leq \varepsilon
    \end{equation*}
    for
    \begin{equation*}
   	g \defeq \sum_{j=1}^m \sigma_j \cdot \phi \left( \rho_j^T \cdot \bullet + \eta_j\right).
    \end{equation*}
Letting 
\begin{equation}  \label{gform}
\tilde{g} \defeq  g(\bullet / N) = \sum_{j=1}^m \sigma_j \cdot \phi \left( \frac{\rho_j^T}{N} \cdot\bullet + \eta_j\right),
\end{equation}
    we see for every $\alpha \in \{-N, ..., N\}^n$ that
    \begin{align*}
        \tilde{g}(\alpha) = g\left( \frac{\alpha}{N}\right) \begin{cases} \geq y_\alpha - \varepsilon = c_1N^{-k} - \varepsilon \overset{\eqref{epsilonbound}}{>} \left(c_1/2\right) N^{-k},& \text{if }z_\alpha = 1, \\ \leq y_\alpha + \varepsilon 
 \overset{\eqref{epsilonbound}}{<} \left ( c_1/2\right)N^{-k},& \text{if }z_\alpha = 0.\end{cases}
    \end{align*}
    Therefore, we get $\mathbbm{1}\left(\tilde{g} > (c_1/2)N^{-k}\right)\left(\alpha\right) = z_\alpha$ for any $\alpha \in \{-N, ..., N\}^n$. Since the choice of $z_\alpha$ has been arbitrary, it follows that the set
    \begin{equation*}
        H \defeq \left\{ \fres{\mathbbm{1}\left(\tilde{g} > (c_1/2)N^{-k}\right)}{\{-N,...,N\}^n} : \ \tilde{g} \text{ of form } \eqref{gform} \right\}
    \end{equation*}
    shatters the whole set $\{-N, ..., N\}^n$. Therefore, we conclude that
    \begin{equation*}
        \text{VC}(H) \geq (2N+1)^n \geq N^n.
    \end{equation*}
    On the other hand,  \cite[Theorem 8.11]{anthony_neural_1999} shows that
    \begin{equation*}
        \text{VC}(H) \leq 2m(n+2)\log_2 (60n\cdot N) \leq c_3 \cdot m \cdot \ln(N)
    \end{equation*}
    with a suitably chosen constant $c_3 = c_3(n)$. Here we used that $N \geq 3$ so that $\ln(N) \geq \ln(3) > 0$. Combining those two inequalities yields 
    \begin{equation*}
        m \geq \frac{N^n}{c_3 \cdot \ln(N)}.
    \end{equation*}
    Using that $N \geq c_4 \cdot \varepsilon^{-1/k}$ with $c_4 \defeq c_4(n,k) = \left(\frac{c_1}{2^{k+1}}\right)^{1/k}$ and $N \leq c_5 \cdot \varepsilon^{-1/k}$ with the definition $c_5 \defeq c_5(n,k) = \left(\frac{c_1}{2}\right)^{1/k}$, we see that
    \begin{equation*}
        m \geq \frac{c_4^n \cdot \varepsilon^{-n/k}}{c_3 \cdot \ln \left(c_5 \cdot \varepsilon^{-1/k}\right)} \geq c_6 \cdot \frac{\varepsilon^{-n/k}}{ \ln \left(1/\varepsilon\right)}
    \end{equation*}
    with $c_6=c_6(n,k)>0$ chosen appropriately.
\end{proof}

\begin{figure}[t]
\centering
\includegraphics[trim=1cm 2cm 0 6cm, width=0.95\textwidth, clip]{proof_illus.pdf}
\caption{Illustration of the function $f$ considered in the proof of \Cref{sigmoidoptimality}.}
\label{fig:VCDimensionFunction}
\end{figure}

As a corollary, we get a similar result for complex-valued neural networks.
\begin{corollary}
\label{sigmoidcomplex}
    Let $n,k \in \NN$ and 
\begin{equation*}
    \phi: \quad \CC \to \CC, \quad \phi(z) \defeq \frac{1}{1+e^{-\RE(z)}}.
\end{equation*}
Then there exists a constant $c = c(n,k) > 0$ with the following property: If $\varepsilon \in (0, \frac{1}{2})$ and $m \in \NN$ are such that for every function $f \in C^k \left(\Omega_n ; \CC\right)$ with $\left\Vert f \right\Vert_{C^k\left(\Omega_n; \CC\right)} \leq 1$ there exist coefficients $\rho_1, ..., \rho_m \in \CC^n$, $\eta_1, ..., \eta_m \in \CC$ and $\sigma_1, ..., \sigma_m \in \CC$ satisfying
\begin{equation*}
    \left\Vert f (z)- \sum_{j=1}^m \sigma_j \cdot \phi \left( \rho_j^T z + \eta_j\right)\right\Vert_{L^\infty \left(\Omega_n ; \CC\right)} \leq \varepsilon,
\end{equation*}
then necessarily 
\begin{equation*}
    m \geq c \cdot \frac{\varepsilon^{-2n/k}}{  \mathrm{ln}\left( 1/\varepsilon\right)}.
\end{equation*}
\end{corollary}
\begin{proof}
    We choose the constant $c=c(2n,k)$ according to the previous \Cref{sigmoidoptimality} and let $\varphi_n$ as in \eqref{isomorphism_intro}. Then, let $\varepsilon \in (0, \frac{1}{2})$ and $m \in \NN$ with the properties assumed in the statement of the corollary. If we then take an arbitrary function $f \in C^k \left([-1,1]^{2n}; \RR\right)$ with $\left\Vert f \right\Vert_{C^k \left([-1,1]^{2n}; \RR\right)} \leq 1$, we deduce the existence of $\rho_1,..., \rho_m \in \CC^n$, $\eta_1, ..., \eta_m \in \CC$ and $\sigma_1, ..., \sigma_m \in \CC$, such that
    \begin{align*}
        &\left\Vert f(x)  - \RE \left( \sum_{j=1}^m \sigma_j \cdot \phi \left( \rho_j^T \cdot \varphi_n(x) + \eta_j\right)\right)  \right\Vert_{L^\infty \left([-1,1]^{2n}; \RR\right)} \\
        \leq & \left\Vert (f \circ \varphi_n^{-1})(z) - \sum_{j=1}^m \sigma_j \cdot \phi \left( \rho_j^T z + \eta_j\right)\right\Vert_{L^\infty \left(\Omega_n ; \CC\right)} \leq \varepsilon.
    \end{align*}
    In the next step, we show that 
    \begin{equation*}
        \RR^{2n} \ni x \mapsto \RE \left( \sum_{j=1}^m \sigma_j \cdot \phi \left( \rho_j^T \cdot \varphi_n(x) + \eta_j\right)\right)  
    \end{equation*}
    is a real-valued shallow neural network with $m$ neurons in the hidden layer and the real sigmoid function as activation function. Then the claim follows using \Cref{sigmoidoptimality}. 

    For every $j \in \{1,...,m\}$ we pick a matrix $\tilde{\rho_j} \in \RR^{2n \times 2}$ with the property that one has
    \begin{equation*}
        \tilde{\rho_j} ^T \cdot \varphi_n^{-1}(z) = \varphi_1^{-1} \left(\rho_j^T \cdot z\right) 
    \end{equation*}
    for every $z \in \CC^n$. This is possible, since this is equivalent to 
\begin{equation*}
\tilde{\rho_j}^T v = \varphi_1^{-1} (\rho_j^T \varphi_n(v))
\end{equation*}
for all $v \in \RR^{2n}$, where the right-hand side is an $\RR$-linear map $\RR^{2n} \to \RR^2$. Denoting the first column of $\tilde{\rho_j}$ by $\hat{\rho_j}$, we get
    \begin{equation*}
        \hat{\rho_j} ^T \cdot \varphi_n^{-1}(z) = \RE \left(\rho_j^T \cdot z\right) \quad \text{for all } z \in\CC^n.
    \end{equation*}
    Writing $\gamma$ for the classical real-valued sigmoid function (i.e. $\gamma(x) = \frac{1}{1 + e^{-x}}$), we deduce for arbitrary $x \in \RR^{2n}$ that
    \begin{align*}
        \RE \left( \sum_{j=1}^m \sigma_j \cdot \phi \left( \rho_j^T \cdot \varphi_n(x) + \eta_j\right)\right) &= \RE \left( \sum_{j=1}^m \sigma_j \cdot \gamma\left( \RE \left( \rho_j^T \cdot \varphi_n(x) + \eta_j\right)\right)\right)  \\
        &= \RE \left( \sum_{j=1}^m \sigma_j \cdot \gamma\left(  \hat{\rho_j}^T x + \RE\left(\eta_j\right)\right)\right)  \\
        &= \sum_{j=1}^m \RE\left(\sigma_j\right) \cdot \gamma\left(  \hat{\rho_j}^T x + \RE\left(\eta_j\right)\right),
    \end{align*}
    where in the last step we used that $\gamma$ is real-valued. As noted above, this completes the proof.
\end{proof}

Now, we can finally prove \Cref{main_5}.

\begin{proof}[Proof of \Cref{main_5}]
Let $\alpha = \frac{2n}{k}$ and choose $c_2 = c_2(\alpha) = c_2(n,k) > 0$ such that the inequality $\ln(x) \leq c_2 \cdot x^{\alpha /2}$ holds for all $x \geq 1$. Furthermore, let $c_1 = c_1(n,k) > 0$ as in \Cref{sigmoidcomplex}. By choosing $c = c(n,k) > 0$ sufficiently small , we can ensure that $\eps_m \defeq c \cdot (m \cdot \ln(m))^{-k/(2n)}$ satisfies 
\begin{equation*}
\ln \left(\frac{c_1}{c_2}\right) + \frac{\alpha}{2} \ln (1/\eps_m) \geq \frac{\alpha}{4} \cdot \ln(1/\eps_m) \quad \text{for all } m \in \NN_{\geq 2}.
\end{equation*}
    By further shrinking $c = c(n,k)>0$ if necessary, we may assume 
\begin{equation*}
c \cdot (2 \cdot \ln (2))^{-k/(2n)} < \frac{1}{2}
\end{equation*}
 and hence $c \cdot (m \cdot \ln (m))^{-k/(2n)} < \frac{1}{2}$ for all $m \in \NN_{\geq 2}$. Finally, setting $c_3 \defeq \frac{\alpha}{4}$ and shrinking $c$ even further, we can arrange that $c^\alpha < c_1 \cdot c_3$. Now, assume towards a contradiction that for every $f \in C^k \left(\Omega_n ; \CC\right)$ with $\Vert f \Vert_{C^k \left(\Omega_n; \CC\right)} \leq 1$ there are coefficients $\rho_1, ..., \rho_m \in \CC^n, \sigma_1, ..., \sigma_m \in \CC$ and $\eta_1, ..., \eta_m \in \CC$ such that
    \begin{equation*}
        \left\Vert f(z) - \sum_{j=1}^m \sigma_j \cdot \phi \left( \rho_j^T z + \eta_j\right)\right\Vert_{L^\infty \left(\Omega_n ; \CC\right)} < c \cdot \left(m \cdot \ln (m)\right)^{-k/(2n)}.
    \end{equation*}
    Applying \Cref{sigmoidcomplex}, we then get
    \begin{equation*}
        m \geq c_1 \cdot \frac{\varepsilon^{-2n/k}}{  \mathrm{ln}\left( 1/\varepsilon\right)}
    \end{equation*}
    with $\varepsilon = c \cdot \left(m \cdot \ln{m}\right)^{-k/(2n)} \in (0, \frac{1}{2})$ and $c_1 = c_1(n,k) > 0$. Recall from the beginning of the proof that $\alpha := 2n/k$ and that $\ln(x) \leq c_2 x^{\alpha / 2}$ for every $x \geq 1$. We observe
    \begin{equation*}
        m \geq c_1 \cdot \frac{\varepsilon^{-\alpha}}{  \mathrm{ln}\left( 1/\varepsilon\right)} \geq \frac{c_1}{c_2} \varepsilon^{- \alpha /2},
    \end{equation*}
    which implies
    \begin{equation*}
        \ln(m) \geq \ln \left(\frac{c_1}{c_2}\right) + \frac{\alpha}{2} \cdot \ln(1/\varepsilon) \geq \frac{\alpha}{4} \cdot \ln(1/\varepsilon) = c_3 \cdot \ln(1/\varepsilon) .
    \end{equation*}
    Overall we then get
    \begin{equation*}
        m \geq c_1 \cdot \frac{\varepsilon^{-\alpha}}{\ln(1/\varepsilon)} = c_1 \cdot c^{-\alpha} \cdot \frac{m \cdot \ln(m)}{\ln(1/\varepsilon)} \geq c_1 \cdot c_3 \cdot c^{-\alpha} \cdot \frac{m \cdot \ln(m)}{\ln(m)} = c_1 \cdot c_3 \cdot c^{-\alpha} \cdot m.
    \end{equation*}
    Since we chose $c$ such that $c^\alpha < c_1 \cdot c_3$ we get the desired contradiction.
\end{proof}

