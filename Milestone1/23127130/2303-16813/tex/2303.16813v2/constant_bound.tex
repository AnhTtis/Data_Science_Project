

\subsection{Details on bounding the constant \texorpdfstring{$c$}{c} in Theorem \ref{main_2}}
\label{sec:const_bound_reordered}

In this appendix we provide details on the bound of the constant $c$ that appears in the formulation of \Cref{main_2}. Specifically, we perform a careful investigation of several results from \cite{lorentz_approximation_2005} to get an upper bound for the constant appearing in \Cref{lorentz_aussage}. Moreover, we analyze the operator norm of the operator
\begin{equation*}
C^k([-1,1]^s; \CC) \to C_{2\pi}^k(\RR^s; \CC), \quad f \mapsto f^* \quad \text{with}\quad f^*(x) \defeq f(\cos(x_1),..., \cos(x_s))
\end{equation*}
appearing in \Cref{star_operator} and show that it is bounded from above by $k^k$.

We start with the analysis of some bounds in \cite[Chapter 4.3]{lorentz_approximation_2005}. Here, a generalization of \emph{Jackson's kernel} is defined for any $m,r \in \NN$ as
\begin{equation*}
L_{m,r}(t) \defeq \lambda_{m,r}^{-1} \cdot \left(\frac{\sin(mt/2)}{\sin(t/2)}\right)^{2r}, \quad t \in \RR,
\end{equation*}
where $\lambda_{m,r}$ is chosen such that
\begin{equation*}
\int_{[-\pi, \pi]} L_{m,r}(t) \ dt = 1.
\end{equation*}
The first two important bounds are provided in the following proposition.
\begin{proposition}\label{prop:lorentz_bound_1}
Let $m,r \in \NN$. Then it holds 
\begin{equation*}
\lambda_{m,r}^{-1} \leq \exp(C \cdot r) \cdot m^{1-2r} \quad \text{and} \quad \int_{[0, \pi]} t^k L_{m,r}(t) \ dt \leq \exp(C \cdot r) \cdot m^{-k}
\end{equation*}
for any $k \leq 2r-2$, with an absolute constant $C>0$.
\end{proposition}
\begin{proof}
Since $L_{m,r} \geq 0$ and since $ \sin(t/2) \leq t/2$ for $t \in [0,\pi]$, we get
\begin{align*}
\lambda_{m,r} &\geq \int_{[0, \pi]} \left(\frac{\sin(mt/2)}{t/2}\right)^{2r} \ dt = 2^{2r} \cdot \int_{[0,\pi]} \left(\frac{\sin(mt/2)}{t}\right)^{2r} \ dt \\
&= 2^{2r} \cdot \int_{[0,\pi m/2]} \left(\frac{\sin(u)}{(2u)/m}\right)^{2r} \ du \cdot \frac{2}{m} \geq m^{2r-1} \cdot \int_{[0, \pi m/2]} \left(\frac{\sin(u)}{u}\right)^{2r} \ du \\
&\geq m^{2r-1} \cdot \int_{[0, \pi/2]} \left(\frac{\sin(u)}{u}\right)^{2r} \ du \geq m^{2r-1} \cdot \int_{[0, \pi/2]} \left(\frac{2u}{\pi \cdot u}\right)^{2r} \ du \geq \left(\frac{2}{\pi}\right)^{2r} \cdot m^{2r-1}.
\end{align*}
Here, we employed the inequality $\sin(u) \geq \frac{2}{\pi} u$ for $u \in [0, \pi/2]$ in the penultimate step.\footnote{To see that this inequality holds, note that $\sin''(u) = -\sin(u) \leq 0$ for $u \in [0,\pi/2]$, so that $\sin$ is concave on that interval, and hence $\sin(u) = \sin((1- \frac{2u}{\pi})\cdot 0 + \frac{2u}{\pi} \cdot \frac{\pi}{2}) \geq \frac{2u}{\pi}\sin(\frac{\pi}{2})= \frac{2u}{\pi}$.} This shows the first part of the claim. 

For the second part, we again use the estimate $\sin(u) \geq \frac{2}{\pi}u$ for $u \in [0, \pi/2]$ to compute
\begin{align*}
\int_{[0, \pi]} t^k L_{m,r}(t) \ dt &= \lambda_{m,r}^{-1} \cdot \int_{[0,\pi]} t^k \left(\frac{\sin(mt/2)}{\sin(t/2)}\right)^{2r} \ dt \leq \lambda_{m,r}^{-1} \cdot \int_{[0,\pi]} t^k \left(\frac{\sin(mt/2)}{t/\pi}\right)^{2r} \ dt \\
&= \lambda_{m,r}^{-1} \cdot \pi^{2r} \cdot \int_{[0,\pi]} t^{k-2r} \sin(mt/2)^{2r} \ dt \\
&= \lambda_{m,r}^{-1} \cdot \pi^{2r} \cdot \int_{[0,\pi m /2]} \left(\frac{2u}{m}\right)^{k-2r} \sin(u)^{2r} \ du \cdot \frac{2}{m} \\
&\leq \lambda_{m,r}^{-1} \cdot \pi^{2r} \cdot m^{2r -1 - k}\int_{[0,\pi m /2]} u^{k-2r} \sin(u)^{2r} \ du \\
&\leq \exp(C_1 \cdot r) \cdot \int_{[0, \infty)} u^{k-2r} \cdot \sin(u)^{2r} \ du \cdot m^{-k}
\end{align*}
with an absolute constant $C_1 > 0$. Here, we employed the first part of this proposition. It remains to bound the integral. This is done via
\begin{align*}
\int_{[0, \infty)} u^{k-2r} \cdot \sin(u)^{2r} \ du &= \int_{[0, 1]} u^{k-2r} \cdot \sin(u)^{2r} \ du + \int_{[1, \infty)} u^{k-2r} \cdot \sin(u)^{2r} \ du \\
&\leq \int_{[0, 1]} \underbrace{u^k \cdot \left(\frac{\sin(u)}{u}\right)^{2r}}_{\leq 1} \ du + \int_{[1, \infty)} u^{-2}  \ du  \leq C_2
\end{align*}
with an absolute constant $C_2> 0$. This proves the claim. 
\end{proof}
The proof in \cite{lorentz_approximation_2005} proceeds by defining
\begin{equation*}
K_{m,r}(t) \defeq L_{m',r}(t), \quad m' = \left\lfloor\frac{m}{r} \right\rfloor + 1.
\end{equation*}
\Cref{prop:lorentz_bound_1} shows for $k \leq 2r-2$ that 
\begin{equation*}
\int_{[0,\pi]} t^k K_{m,r} (t) \ dt \leq \exp( C\cdot r) \cdot (m')^{-k}.
\end{equation*}
Since $m' \geq \frac{m}{r}$ we infer
\begin{equation}\label{eq:rauteraute}
\int_{[0,\pi]} t^k K_{m,r} (t) \ dt \leq \exp( C\cdot r) \cdot \left(\frac{r}{m}\right)^{k} \leq \exp(C \cdot r) \cdot r^k \cdot m^{-k}
\end{equation}
with an absolute constant $C > 0$.

We can now quantify the constant appearing in \cite[Theorem 4.3]{lorentz_approximation_2005}.
\begin{theorem}[{cf. \cite[Theorem 4.3]{lorentz_approximation_2005}}]\label{thm:lorentz_1d}
Let $k,m \in \NN$ and $f \in C^k_{2\pi}(\RR; \RR)$. Let 
\begin{equation*}
\omega(f^{(k)}, 1/m) \defeq \underset{x \in \RR, \vert t \vert \leq 1/m}{\max} \vert f^{(k)}(x+t)- f^{(k)}(x)\vert.
\end{equation*}
Then it holds 
\begin{equation*}
E_m^1 (f) \leq (\exp(C \cdot k) \cdot k^k) \cdot m^{-k} \cdot \omega(f^{(k)}, 1/m). 
\end{equation*}
Here, we recall that $E_m^1 (f)$ denotes the best possible approximation error when approximating $f$ using trigonometric polynomials of degree $m$; see \Cref{eq:mintrigo}. 
\end{theorem}
\begin{proof}
We follow the proof of \cite[Theorem 4.3]{lorentz_approximation_2005}. Take $r = k+1$ and define 
\begin{equation*}
I_m(x) \defeq - \int_{[-\pi, \pi]} K_{m,r}(t) \sum_{\ell =1}^{k+1} (-1)^\ell \binom{k+1}{\ell} f(x + \ell t) \ dt.
\end{equation*}
Then it is shown in the proof of \cite[Theorem 4.3]{lorentz_approximation_2005} that $I_m$ is a trigonometric polynomial of degree at most $m$ and that 
\begin{equation*}
\vert f(x) - I_m(x) \vert \leq 2 \cdot \omega_{k+1}(f, 1/m) \cdot \int_{[0,\pi]} (mt+1)^{k+1} K_{m,r}(t) \ dt.
\end{equation*}
Here, $\omega_{k+1}(f, 1/m)$ denotes the \emph{modulus of smoothness} of $f$ as defined on \cite[p.~47]{lorentz_approximation_2005}. The integral can be bounded via
\begin{align*}
&\norel\int_{[0,\pi]} (mt+1)^{k+1} K_{m,r}(t) \ dt \\
&= \int_{[0, 1/m]} (\underbrace{mt+1}_{\leq 2})^{k+1} K_{m,r}(t) \ dt + \int_{[1/m, \pi]}(\underbrace{mt+1}_{\leq 2mt})^{k+1} K_{m,r}(t) \ dt \\
&\leq 2^{k+1} \cdot \underbrace{\int_{[-\pi, \pi]}K_{m,r}(t) \ dt}_{= 1} + 2^{k+1}m^{k+1} \cdot \int_{[0,\pi]} \ t^{k+1}K_{m,r}(t) \ dt \\
\overset{\eqref{eq:rauteraute}}&{\leq} 2^{k+1} + 2^{k+1} m ^{k+1}\exp(C_1 \cdot r) \cdot (k+1)^{k+1} \cdot m^{-(k+1)} \overset{r\leq 2k}{\leq} \exp(C \cdot k ) \cdot k^{k} 
\end{align*}
with absolute constants $C,C_1 > 0$. Since $\omega_{k+1}(f, 1/m) \leq m^{-k} \cdot \omega(f^{(k)}, 1/m)$ follows from \cite[Equation~3.6(5)]{lorentz_approximation_2005}, the claim is shown.
\end{proof}
Therefore, we can bound the constant appearing in \cite[Theorem 4.3]{lorentz_approximation_2005} by $\exp(C \cdot k) \cdot k^k$. It remains to deal with the approximation of \emph{multivariate} periodic functions by \emph{multivariate} trigonometric polynomials which is contained in \cite[Theorem 6.6]{lorentz_approximation_2005}.

\begin{theorem}[{cf. \cite[Theorem 6.6]{lorentz_approximation_2005}}] \label{thm:const_lorentz_bound}
Let $s,k \in \NN$ and $f \in C^k_{2\pi}(\RR^s; \RR)$. Let $\omega_j$ denote the modulus of continuity of $\frac{\partial^k f}{ \partial x_j^k}$ for $j=1,...,s$. Then, with $E_m^s$ as introduced in \Cref{eq:mintrigo}, it holds
\begin{equation*}
E^s_m(f) \leq \exp(C \cdot ks) \cdot k^k \cdot m^{-k} \sum_{j=1}^s \omega_j(1/m),
\end{equation*}
with an absolute constant $C>0$.
\end{theorem}
\begin{proof}
We follow the proof of \cite[Theorem 6.6]{lorentz_approximation_2005} with $p_j = k$ and $n_j = m$ for every index $j = 1,...,s$. For $j = 1,..., s+1$ define the set $\mathcal{T}_j$ consisting of all functions $g \in C_{2\pi}(\RR^s; \RR)$ that are a trigonometric polynomial in $x_\ell$ of degree at most $m$ for $\ell < j$; in $x_\ell$ for $\ell \geq j$ they should have continuous partial derivatives $\frac{\partial^p g}{ \partial x_\ell^p}$ for $0\leq p \leq k$; the modulus of continuity of $\frac{\partial^{k} g}{\partial x_\ell^k}$ should not exceed $2^{K_j}\omega_j$, where
\begin{equation*}
K_j = (j-1)(k+1) \leq 2ks \quad \text{for } j>1 \text{ and } K_1=1\leq 2ks.
\end{equation*} 
Then it is shown that if $j \in \{1,...,s\}$ and $f_j \in \mathcal{T}_j$ there exists a function $f_{j+1} \in \mathcal{T}_{j+1}$ for which
\begin{equation*}
\Vert f_j - f_{j+1} \Vert_{L^\infty(\RR^s; \RR)} \leq \exp(C_1 \cdot k) \cdot k^k \cdot m^{-k} \cdot 2^{2ks} \cdot \omega_j(1/m) \leq \exp(C_2 \cdot ks) \cdot k^k \cdot m^{-k} \cdot \omega_j(1/m)
\end{equation*}
for absolute constants $C_1, C_2 > 0$. This is an application of \Cref{thm:lorentz_1d}. Hence, defining $f_1 \defeq f$, we see
\begin{equation*}
\Vert f - f_{s+1} \Vert_{L^\infty(\RR^s; \RR)} \leq \sum_{j=1}^{s} \Vert f_j - f_{j+1} \Vert_{L^\infty(\RR^s; \RR)} \leq \exp(C_2 \cdot ks) \cdot k^k \cdot m^{-k} \cdot \sum_{j=1}^s\omega_j(1/m). \qedhere
\end{equation*}
\end{proof}
Therefore, we have shown that the constant appearing in \cite[Theorem 6.6]{lorentz_approximation_2005} can be bounded from above by $\exp(C \cdot ks) \cdot k^k$. 

In the rest of this section, we discuss the operator norm of the operator defined in \Cref{star_operator}. In \Cref{star_operator} the closed graph theorem is used to show that the operator is bounded. However, the closed graph theorem does not provide any bound on the norm of the operator. Therefore, in order to quantify the operator norm, we need to apply a different technique, which is \emph{Faa di Bruno's formula}. This formula is a generalization of the chain rule to higher order derivatives.

\begin{theorem} \label{thm:faa}
Let $s,k \in \NN$. We define the operator
\begin{equation*}
T: \quad C^k([-1,1]^s ; \CC) \to C^k_{2\pi}([-\pi, \pi]^s;\CC), \quad (Tf)(x_1, ..., x_s) \defeq f(\cos(x_1), ..., \cos(x_s)).
\end{equation*}
Let $\aalpha \in \NN_0^s$ with $\vert \aalpha \vert \leq k$. Then, for any $f \in C^k([-1,1]^s;\CC)$, we have
\begin{equation*}
\Vert \partial^{\aalpha} (Tf) \Vert_{L^\infty([\pi, \pi]^s;\CC)} \leq \prod_{j=1}^s {\aalpha}_j^{{\aalpha}_j} \cdot \Vert f \Vert_{C^{\vert \aalpha \vert}([-1,1]^s)}.
\end{equation*}
\end{theorem}
\begin{proof}
The proof is by induction over $s$. The case $s=1$ is an application of Faa di Bruno's formula: We can write $Tf = f \circ g$ with $g(x) = \cos(x)$. We then take $\ell \in \NN_0$ with $\ell \leq k$ and some $x \in [-\pi, \pi]$. The set partition version of Faa di Bruno's formula (see for instance \cite[p.~219]{johnson_curious_2002}) then yields
\begin{equation*}
\left\vert (f \circ g)^{(\ell)}(x)\right\vert \leq \sum_{\pi \in \Pi_\ell} \left(\left\vert f^{(\left\vert\pi\right\vert)}(g(x))\right\vert \cdot \prod_{B \in \pi} \left\vert g^{(\vert B \vert)}(x)\right\vert\right).
\end{equation*}
Here, $\Pi_\ell$ denotes the set of all partitions of the set $\{1, ..., \ell\}$. Since all derivatives of $g$ are bounded by $1$ in absolute value and $\vert \pi \vert \leq \ell$ for every partition $\pi\in \Pi_\ell$ we get
\begin{equation*}
\Vert (f \circ g)^{(\ell)} \Vert_{L^\infty([-\pi, \pi];\CC)} \leq \vert \Pi_\ell \vert \cdot \Vert f \Vert_{C^\ell ([-1,1]; \CC)}.
\end{equation*}
The number $\vert \Pi_\ell \vert$ is the number of possible partitions of the set $\{1,..., \ell\}$ and is the so-called $\ell$-th \emph{Bell number}. It can be bounded from above by $\ell^\ell$ (see \cite[Theorem 2.1]{berend2010improved}). This proves the case $s=1$.

We now assume that the claim holds for an arbitrary but fixed $s \in \NN$. Take $\aalpha \in \NN_0^{s+1}$ with $\vert \aalpha \vert \leq k$. We decompose $\aalpha = (\aalpha', \aalpha_{s+1})$ with $\aalpha' \in \NN_0^s$. For a fixed variable $y_{s+1} \in [-1,1]$, we define 
\begin{equation*}
f_{y_{s+1}}(y_1,..., y_s) \defeq f(y_1, ..., y_s, y_{s+1}) \quad \text{for} \quad (y_1, ..., y_s) \in [-1,1]^s.
\end{equation*}
We denote $g(x_1,...,x_{s+1}) \defeq (\cos(x_1),..., \cos(x_{s+1}))$, $g_s(x_1,..., x_s) \defeq (\cos(x_1),..., \cos(x_s))$ and $\theta(x_{s+1}) \defeq \cos(x_{s+1})$. For every $(x_1,..., x_{s+1}) \in [-\pi,\pi]^{s+1}$ it then holds
\begin{equation*}
(f \circ g)(x_1, ..., x_{s+1}) = \left(f_{\theta(x_{s+1})} \circ g_s\right)(x_1,..., x_s).
\end{equation*}
We now differentiate $f \circ g$ with respect to the multiindex $\aalpha$ and get
\begin{align*}
\left[\partial^{\aalpha}(f \circ g) \right] (x_1,...,x_{s+1}) &= \frac{\partial^{\aalpha_{s+1}}}{\partial x_{s+1}^{\aalpha_{s+1}}}\left[\partial^{\aalpha'} \left(f_{\theta(x_{s+1}) } \circ g_s\right)(x_1,..., x_s)\right] \\
&= (h_{x_1,...,x_s} \circ \theta)^{(\aalpha_{s+1})}(x_{s+1})
\end{align*}
where we define
\begin{equation*}
h_{x_1,...,x_s}(y_{s+1})\defeq \partial^{\aalpha'} \left(f_{y_{s+1}} \circ g_s\right)(x_1,...,x_s) \quad \text{for} \quad (x_1,...,x_s) \in [-\pi, \pi]^s  \text{ and }  y_{s+1} \in [-1,1].
\end{equation*}
Using the case $s=1$, we get
\begin{equation*}
\left\vert\left[\partial^{\aalpha}(f \circ g) \right] (x_1,...,x_{s+1}) \right\vert = \left\vert (h_{x_1,...,x_s} \circ \theta)^{(\aalpha_{s+1})}(x_{s+1})\right\vert \leq \aalpha_{s+1}^{\aalpha_{s+1}} \cdot \left\Vert h_{x_1,...,x_s}\right\Vert_{C^{\aalpha_{s+1}}([-1,1]; \CC)}
\end{equation*}
for any fixed $(x_1,...,x_s) \in [-\pi,\pi]^s$. 

It remains to bound $\left\Vert h_{x_1,...,x_s}\right\Vert_{C^{\aalpha_{s+1}}([-1,1]; \CC)}$. To this end, we fix $\ell \in \NN_0$ with $\ell \leq \aalpha_{s+1}$. We further denote
\begin{equation*}
F_{y_1,...,y_s}(y_{s+1}) \defeq f(y_1,...,y_s,y_{s+1}) \quad \text{for} \quad (y_1, ..., y_{s+1}) \in [-1,1]^{s+1}.
\end{equation*} 
For arbitrary $(x_1,..., x_s) \in [-\pi,\pi]^{s}$ and $y_{s+1} \in [-1, 1]$ we then see
\begin{align*}
h_{x_1,...,x_s}^{(\ell)}(y_{s+1}) = \partial^{\aalpha'} \left[(x_1,...,x_s) \mapsto F^{(\ell)} _{g_s(x_1,...,x_s)}(y_{s+1})\right]  =  \partial^{\aalpha'}\left[H_{y_{s+1}} \circ g_s \right](x_1,...,x_s)
\end{align*} 
where 
\begin{equation*}
H_{y_{s+1}}(y_1,...,y_s) \defeq F^{(\ell)}_{y_1,...,y_s}(y_{s+1}) \quad \text{for } (y_1,...,y_s) \in [-1,1]^s.
\end{equation*}
Hence, we see by induction that
\begin{align*}
\left\vert h_{x_1,...,x_s}^{(\ell)}(y_{s+1}) \right\vert &= \left\vert \partial^{\aalpha'}\left[H_{y_{s+1}} \circ g_s \right](x_1,...,x_s) \right\vert \overset{\text{IH}}{\leq} \prod_{j=1}^{s} \aalpha_j^{\aalpha_j} \cdot \left\Vert H_{y_{s+1}}\right\Vert_{C^{\vert \aalpha'\vert}([-1,1]^s; \CC)} \\
&\leq \prod_{j=1}^{s} \aalpha_j^{\aalpha_j} \cdot \Vert f \Vert_{C^{\vert \aalpha \vert}([-1,1]^{s+1} ; \CC)}
\end{align*}
as was to be shown.
\end{proof}

\begin{remark}\label{rem:multiindex}
For a multiindex $\aalpha \in \NN_0^s$ with $\vert \aalpha \vert \leq k$ we see
\begin{equation*}
\prod_{j=1}^{s} \aalpha_j^{\aalpha_j} \leq k^{\sum_{j=1}^s \aalpha_j} \leq k^k.
\end{equation*}
Hence, the norm of the operator introduced in \Cref{star_operator} can be bounded from above by $k^k$.
\end{remark}

