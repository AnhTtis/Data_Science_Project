\section{Postponed proofs concerning the approximation of polynomials}

% !TeX encoding = UTF-8
% !TeX spellcheck = en_US
% !TeX root = main_paper.tex

\section{Divided Differences}
Divided differences are well-known in numerical mathematics as they are for example used to calculate the coefficients of an interpolation polynomial in its Newton representation. In our case, we are interested in divided differences, since they can be used to obtain a generalization of the classical mean-value theorem for differentiable functions: Given an interval $I \subseteq \RR$ and $n+1$ pairwise distinct data points $x_0, ..., x_n \in I$ as well as an $n$-times differentiable real-valued function $f : I \to \RR$, there exists $\xi \in \left( \text{min}\left\{x_0 ,..., x_n\right\}, \text{max}\left\{x_0 ,..., x_n\right\}\right)$, such that
\begin{equation*}
    f\left[x_0, ..., x_n\right] = \frac{f^{(n)}(\xi)}{n!},
\end{equation*}
where the left-hand side is a divided difference of $f$ (defined below). The classical mean-value theorem is obtained in the case $n=1$. Our goal in this section is to generalize this result to a multivariate setting by considering divided differences in multiple variables. Such a generalization is probably well-known, but since we could not locate a convenient reference and to make the paper more self-contained, we provide a proof.

Let us first define divided differences formally. Given $n+1$ data points $\left(x_0, y_0\right), ..., \left(x_n, y_n\right) \in \RR \times \RR$ with pairwise distinct $x_k$, we define the associated divided differences recursively via
\begin{align*}
    \left[ y_k\right] &\defeq y_k, \ k \in \{0,...,n\}, \\
    \left[y_k ,..., y_{k+j}\right] &\defeq \frac{\left[y_{k+1},..., y_{k+j}\right] - \left[y_k,..., y_{k+j-1}\right]}{x_{k+j}-x_k}, \ j \in \{1,...,n\}, \ k \in \{0, ..., n-j\}.
\end{align*}
If the data points are defined by a function $f$ (i.e. $y_k = f\left(x_k\right)$ for all $k \in \{0,...,n\}$), we write
\begin{equation*}
    \left[x_k,...,x_{k+j}\right] f \defeq \left[y_k, ..., y_{k+j}\right].
\end{equation*}
We first consider an alternative representation of divided differences, the so-called \emph{Hermite-Genocchi-Formula}. To state it, we introduce the notation $\Sigma^k$ for a certain $k$-dimensional simplex.
\begin{definition}
    Let $s \in \NN$. Then we define
    \begin{equation*}
        \Sigma^s \defeq \left\{ x \in \RR^s : \ x_\ell \geq 0 \ \mathrm{ for} \ \mathrm{all}  \ \ell \ \mathrm{ and  } \sum_{\ell = 1}^s x_\ell \leq 1 \right\}.
    \end{equation*}
    The identity $\lambda^s\left(\Sigma^s\right)= \frac{1}{s!}$ holds true.
\end{definition}
A proof for the fact that the volume of $\Sigma^s$ is indeed $\frac{1}{s!}$ can be found for example in \cite{stein_note_1966}.
We can now consider the alternative representation of divided differences.
\begin{lemma}[Hermite-Genocchi-Formula]
    For real numbers $a,b \in \RR$, a function $f \in C^k([a,b]; \RR)$ and pairwise distinct $x_0, ..., x_k \in [a,b]$, the divided difference of $f$ at the data points $x_0, ..., x_k$ is given as
\begin{equation}
\label{hg}
    \left[x_0, ..., x_k\right]f = \int_{\Sigma^k} f^{(k)}\left(x_0 + \sum_{\ell=1}^{k}s_\ell\left(x_\ell-x_0\right)\right) ds.
\end{equation}
\end{lemma}
\begin{proof}
    See \cite[Theorem 3.3]{atkinson_introduction_1989}.
\end{proof}
We need the following easy generalization of the mean-value theorem for integrals.
\begin{lemma}
Let $D \subseteq \RR^s$ be a connected and compact set with positive Lesbesgue measure and furthermore $f : D \to \RR$ a continuous function. Then there exists $\xi \in D$ such that
\begin{equation*}
    f(\xi) = \frac{1}{\lambda(D)} \cdot \int_D f(x) dx.
\end{equation*}
\end{lemma}
\begin{proof} 
Since $D$ is compact and $f$ continuous, there exist $x_{\text{min}} \in D$ and $x_{\text{max}} \in D$ satisfying
\begin{equation*}
    f\left(x_{\text{min}}\right) \leq f(x) \leq f\left(x_{\text{max}} \right)
\end{equation*}
for all $x \in D$. Thus, one gets
\begin{equation*}
    f\left( x_{\text{min}}\right) \leq \frac{1}{\lambda(D)} \int_D f(x) dx \leq  f\left( x_{\text{max}}\right)
\end{equation*}
so the claim follows using the fact that $f(D) \subseteq \RR$ is connected, i.e., an interval.
\end{proof}
We also get a convenient representation of divided differences if the data points are equidistant.
\begin{lemma}
Let $f: \RR \to \RR$, $x_0 \in \RR$ and $h > 0$. We consider the case of equidistant data points, meaning $x_{j} \defeq x_0 + jh$ for all $j = 1,...,k$ for a fixed $h > 0$. In this case, we have the formula
\begin{equation}
\label{alternativdarstellung}
    \left[x_0, ..., x_k\right]f = \frac{1}{k!h^k} \cdot \sum_{r=0}^k (-1)^{k-r}\binom{k}{r} f\left(x_r\right). 
\end{equation}
\end{lemma}
\begin{proof}
We prove the result via induction over the number $j$ of considered data points, meaning the following: For all $j \in \{0,...,k\}$ we have
\begin{equation*}
    \left[x_\ell, ..., x_{\ell+j}\right]f = \frac{1}{j!h^j} \cdot \sum_{r=0}^j (-1)^{j-r}\binom{j}{r} f\left(x_{\ell+r}\right)
\end{equation*}
for all $\ell \in \{0, ..., k\}$ such that $\ell + j \leq k$. The case $j = 0$ is trivial. Therefore, we assume the claim to be true for a fixed $j \in \{0,...,k-1\}$ and choose an arbitrary $\ell \in \{0,...,k\}$ such that $\ell+j+1 \leq k$. We then get
\begin{align*}
    \left[x_\ell, ..., x_{\ell+j+1}\right]f &= \frac{\left[x_{\ell+1}, ..., x_{\ell+j+1}\right]f - \left[x_\ell, ..., x_{\ell+j}\right]f}{x_{\ell+j+1}-x_\ell} \\
    &= \frac{1}{j!h^j}\cdot \frac{\sum_{r=0}^j (-1)^{j-r}\binom{j}{r} \left(f\left(x_{\ell+r+1}\right) - f\left(x_{\ell+r}\right)\right)}{(j+1)h} \\
    &= \frac{1}{(j+1)!h^{j+1}}\sum_{r=0}^j (-1)^{j-r}\binom{j}{r} \left(f\left(x_{\ell+r+1}\right) - f\left(x_{\ell+r}\right)\right).
\end{align*}
Using an index shift we deduce
\begin{align*}
    & \norel \sum_{r=0}^j (-1)^{j-r}\binom{j}{r}f\left(x_{\ell+r+1}\right) - \sum_{r=0}^j (-1)^{j-r}\binom{j}{r}f\left(x_{\ell+r}\right) \\
    &= \sum_{r=1}^{j+1} (-1)^{j+1-r}\binom{j}{r-1}f\left(x_{\ell+r}\right) + \sum_{r=0}^j (-1)^{j+1-r}\binom{j}{r}f\left(x_{\ell+r}\right) \\
    &= (-1)^{j+1} f\left(x_\ell\right) + \sum_{r=1}^{j} \left((-1)^{j+1-r}f\left(x_{\ell+r}\right) \left[\binom{j}{r-1} + \binom{j}{r}\right]\right) + f\left(x_{\ell+j+1}\right) \\
    &= \sum_{r=0}^{j+1} (-1)^{j+1-r}\binom{j+1}{r} f\left(x_{\ell+r}\right)
\end{align*}
which yields the claim.
\end{proof}
The final result for divided differences is stated as follows:
\begin{theorem}
\label{div_differences_mainresult}
Let $f: \RR^s \to \RR$ and $k \in \NN_0, r>0$, such that $\fres{f}{(-r,r)^s} \in C^k \left((-r,r)^s; \RR\right)$. For $\textbf{p} \in \NN_0^s$ with $\vert \pp \vert \leq k$ and $h>0$ let
\begin{equation*}
    f_{\pp,h} \defeq (2h)^{-\vert \pp \vert} \sum_{0 \leq \textbf{r} \leq \textbf{p}} (-1)^{\vert \pp \vert -\vert \rr \vert} \binom{\textbf{p}}{\textbf{r}} f \left( h(2\rr-\pp)\right).
\end{equation*}
Let $m \defeq \underset{j}{\mathrm{max}} \  \pp_j$. Then, for $0 <h < \frac{r}{\max\{1,m\}}$ there exists $\xi \in h[-m,m]^s$ satisfying
\begin{equation*}
    f_{\pp,h} = \partial^\pp f(\xi).
\end{equation*}
\end{theorem}
\begin{proof}
We may assume $m > 0$, since $\pp=0$ implies $f_{\pp,h} = f(0)$ and hence, the claim follows with $\xi = 0$.

We prove via induction over $s \in \NN$ that the formula
\begin{equation}
\label{to_prove}
    f_{\pp,h}= \pp! \int_{\Sigma^{\pp_s}}\int_{\Sigma^{\pp_{s-1}}} \cdot\cdot\cdot \int_{\Sigma^{\pp_1}} \partial^\pp f \left( -h\pp_1 + 2h\sum_{\ell=1}^{\pp_1}\ell\sigma_\ell^{(1)}, ..., -h\pp_s + 2h\sum_{i=1}^{\pp_s}\ell\sigma_\ell^{(s)}\right)d\sigma^{(1)} \cdot \cdot \cdot d\sigma^{(s)}
\end{equation}
holds for all $\pp \in \NN_0^s$ with $1 \leq \vert \pp \vert \leq k$ and all $0 < h < \frac{r}{m}$. The case $s=1$ is exactly the Hermite-Genocchi-Formula (\ref{hg}) combined with (\ref{alternativdarstellung}) applied to the data points $-hp, -hp + 2h, ..., hp-2h, hp$. 

By induction, assume that the claim holds for some $s \in \NN$.
For a fixed point $y \in (-r,r)$, let 
\begin{equation*}
    f_y: \quad (-r,r)^s \to \RR, \quad x \mapsto f(x,y).
\end{equation*}
For $\pp \in \NN_0^{s+1}$ with $\vert \pp \vert \leq k$ and $\pp' := \left(p_1,...,p_s\right)$ we define
\begin{equation*}
    \Gamma: \quad (-r,r) \to \RR, \quad y \mapsto \left( f_y\right)_{\pp',h} = (2h)^{- \vert \pp' \vert} \sum_{0 \leq \rr' \leq \pp'} (-1)^{\vert  \pp' \vert - \vert \rr' \vert} \binom{\pp'}{\rr'} f\left( h(2\rr' - \pp'),y\right).
\end{equation*}
Using the induction hypothesis, we get
\begin{align*}
\label{IV}
    &\Gamma(y) \\
    =&\pp'! \int\limits_{\Sigma^{\pp_s}}\int\limits_{\Sigma^{\pp_{s-1}}} \cdot\cdot\cdot \int\limits_{\Sigma^{\pp_1}} \partial^{\left(\pp',0 \right)} f \left( -h\pp_1 + 2h\sum_{i=1}^{\pp_1}i\sigma_i^{(1)}, ..., -h\pp_s + 2h\sum_{i=1}^{\pp_s}i\sigma_i^{(s)},y\right)d\sigma^{(1)} \cdot \cdot \cdot d\sigma^{(s)}
\end{align*}
for all $y \in (-r,r)$. Furthermore, we calculate
\begin{align*}
    &\norel \pp_{s+1}! \cdot [-h \cdot \pp_{s+1}, -h \cdot \pp_{s+1} + 2h, ..., h \cdot \pp_{s+1}]\Gamma  \\
    \overset{\eqref{alternativdarstellung}}&{=} (2h) ^{- \pp_{s+1}} \sum_{r' = 0}^{\pp_{s+1}} (-1)^{\pp_{s+1}-r'}\binom{\pp_{s+1}}{r'} \Gamma\left(h\left(2r'-\pp_{s+1}\right)\right) \\
    &= (2h) ^{- \pp_{s+1}} \sum_{r' = 0 }^{\pp_{s+1}} (-1)^{\pp_{s+1}-r'}\binom{\pp_{s+1}}{r'} (2h)^{- \vert \pp' \vert} \sum_{0 \leq \rr' \leq \pp'}(-1)^{\vert \pp' \vert -\vert \rr' \vert} \binom{\pp'}{\rr'} f\left( h(2\rr' - \pp'), h(2r' - \pp_{s+1})\right) \\
    &= (2h)^{-\vert \pp \vert} \sum_{0 \leq \textbf{r} \leq \textbf{p}} (-1)^{\vert \pp \vert -\vert \rr \vert} \binom{\textbf{p}}{\textbf{r}} f \left( h(2\rr-\pp)\right) \\
    &= f_{\pp, h}.
\end{align*}
On the other hand, we get
\begin{align*}
    &\norel [-h \cdot \pp_{s+1}, -h \cdot \pp_{s+1} + 2h, ..., h \cdot \pp_{s+1}]\Gamma \\
    \overset{\eqref{hg}}&{=} \int_{\Sigma^{\pp_{s+1}}}\Gamma^{(\pp_{s+1})}\left(-h\pp_{s+1} + 2h\sum_{\ell=1}^{\pp_{s+1}}\ell t_\ell\right)dt \\
    \overset{}&{=}\pp'! \int\limits_{\Sigma^{\pp_{s+1}}} \cdot\cdot\cdot \int\limits_{\Sigma^{\pp_1}} \partial^{\pp} f \left( -h\pp_1 + 2h\sum_{\ell=1}^{\pp_1}\ell\sigma_\ell^{(1)}, ..., -h\pp_{s+1}+2h\sum_{\ell=1}^{\pp_{s+1}}\ell\sigma^{(s+1)}_\ell\right)d\sigma^{(1)} \cdot \cdot \cdot d\sigma^{(s+1)}.
\end{align*}
Changing the order of integration and derivative is possible, since we integrate on compact sets and only consider continuously differentiable functions.

We have thus proven (\ref{to_prove}) using the principle of induction. The claim of the theorem then follows directly using the mean-value theorem for integrals and by the fact that all the simplices $\Sigma^{\pp_\ell}$ are compact and connected (in fact convex).
\end{proof}

% !TeX encoding = UTF-8
% !TeX spellcheck = en_US
% !TeX root = main_paper.tex

\section{Admissibility of Activation Functions}
\label{admissibility}

In this section we discuss the notion of admissibility of activation functions $\phi: \CC \to \CC$, i.e., the sufficient conditions that a function has to fulfill in order to be suitable for the approximation results established in Sections \ref{approx_polynomials} and \ref{ck_functions}. We are going to define a slightly weaker notion of admissibility than that the function is smooth and non-polyharmonic on a non-empty open subset of $\CC$.

Let us recall that in \cite{mhaskar_neural_1996} the sufficient condition on an activation function $\phi: \RR \to \RR$  is that it is smooth on an open interval with one point in the interval where no derivative vanishes. In fact, the proofs in \cite{mhaskar_neural_1996} show that it is sufficient to assume that for every $M \in \NN$ there is an open set $\emptyset \neq U \subseteq \RR$ such that $\fres{\phi}{U} \in C^M \left( U ; \RR\right)$ with a point $b \in U$ satisfying
\begin{equation*}
f^{(n)} (b) \neq 0 \text{ for all } n \leq M.
\end{equation*}  
In the case of complex-valued neural networks it turns out that the usual real derivative has to be replaced by mixed Wirtinger derivatives of the form $\wirt^m \wirtq^\ell$. This leads to the following definition.
\begin{definition}
Let $\phi: \CC \to \CC$ and $M \in \NN_0$. We call $\phi$ $M$\emph{-admissible} (in $b \in \CC$) if there is an open set $U \subseteq \CC$ with $b \in U$ and $\fres{\phi}{U} \in C^{2M} (U ; \CC)$ such that
\begin{equation*}
\wirt^m \wirtq^\ell \phi (b) \neq 0 \text{ for all } m,\ell \leq M.
\end{equation*}
\end{definition}
In order to approximate complex polynomials in $z$ and $\overline{z}$ with bounded degree it will be enough to assume $M$-admissibility of an activation function for a certain $M \in \NN$. However, if one wants to approximate arbitrary $C^k$-functions one needs the following definition.
\begin{definition}
A function $\phi : \CC \to \CC$ is called \emph{admissible} if it is $M$-admissible for every $M \in \NN_0$.
\end{definition}
Note that for an admissible function there does not necessarily have to be an open set $\emptyset \neq U \subseteq \CC$ such that the function is smooth on $U$. However, if we assume smoothness we can derive the following elegant result.
\begin{theorem}
\label{charac}
    Let $\phi: \CC \to \CC$, $\emptyset \neq U \subseteq \CC$ an open set and $\fres{\phi}{U} \in C^\infty(U ; \CC)$. Then the following are equivalent:
    \begin{enumerate}
        \item $\fres{\phi}{U}$ is not polyharmonic.
        \item For every $M \in \NN_0$ there exists $z_M \in U$ such that $\phi$ is $M$-admissible at $z_M$. 
    \end{enumerate}
In particular, in both cases $\phi$ is admissible.
\end{theorem}
\begin{proof}
    (1) $\Rightarrow$ (2): 
    Let $M \in \NN_0$. Since $\fres{\phi}{U}$ is not polyharmonic we can pick $z \in U$ with $\Delta^M \phi (z) \neq 0$. By continuity we can choose $\delta > 0$ with $B_\delta(z) \subseteq U$ and $\Delta^M \phi(w) \neq 0$ for all $w \in B_\delta(z)$. For all $m, \ell \in \NN_0$ let
    \begin{equation*}
        A_{m,\ell} \defeq \left\{ w \in B_\delta(z) : \ \wirt^m \wirtq^\ell \phi (w) = 0\right\}
    \end{equation*}
    and assume towards a contradiction that
    \begin{equation*}
        \bigcup_{m,\ell \leq M} A_{m,\ell} = B_\delta(z).
    \end{equation*}
    By \cite[Corollary 3.35]{aliprantis_infinite_2006}, $B_\delta(z)$ with its usual topology is completely metrizable. By continuity of $\wirt^m \wirtq^\ell \phi$ the sets $A_{m,\ell}$ are closed in $B_\delta(z)$. Hence, using the Baire category theorem \cite[Theorems 3.46 and 3.47]{aliprantis_infinite_2006}, there are $m,\ell \in \NN_0$ with $m,\ell \leq M$, $z' \in A_{m,\ell}$ and $\varepsilon > 0$ such that
    \begin{equation*}
        B_\varepsilon\left(z'\right) \subseteq A_{m,\ell} \subseteq B_\delta(z).
    \end{equation*}
    But thanks to $\Delta^M \phi = 4^M \wirt^M \wirtq^M \phi = 4^M \wirt^{M - \ell} \wirtq^{M-m}\wirt^\ell \wirtq^m \phi$ (see (\ref{laplace_ident}) on p. 6), this directly implies $\Delta^M \phi \equiv 0$ on $B_{\varepsilon} \left(z'\right)$ in contradiction to the choice of $B_\delta(z)$. 
\medskip

(2) $\Rightarrow$ (1): If $\fres{\phi}{U}$ is polyharmonic there is $M \in \NN_0$ with
    \begin{equation*}
        \wirt^M \wirtq^M \phi(z) = 0
    \end{equation*}
    for every $z \in U$, which contradicts (2).
\end{proof}
\Cref{charac} justifies the formulation of \Cref{main_1,,main_2}, since it shows that every activation function $\phi$ satisfying the assumptions of the theorem is $M$-admissible for all $M \in \NN_0$.

\begin{remark}
In the case of a real function it follows that if the function is smooth and non-polynomial on an open interval, there has to be a point at which no derivative vanishes, see \cite[p. 53]{donoghue_distributions_1969}. It is an interesting and currently unsolved question if a similar statement also holds true in the case of complex functions and Wirtinger derivatives, i.e., if the following holds: Let $z \in \CC, \ r > 0$ and $\phi \in C^\infty (B_r(z); \CC)$ be non-polyharmonic. Then there exists $b \in B_r(z)$ such that
\begin{equation*}
\wirt^m \wirtq^\ell \phi (b) \neq 0 \text{ for all } m,\ell \in \NN_0.
\end{equation*}
\end{remark}





\subsection{Proof of Theorem \ref{main_1}}
\label{approx_polynomials_reordered}

The following section is dedicated to proving \Cref{main_1}.
We are going to show that it is possible to approximate complex polynomials in $z$ and $\overline{z}$
arbitrarily well on $\Omega_n$ using shallow complex-valued neural networks. 
To do so, we follow the proof sketch given after the statement of \Cref{main_1},
starting with the following lemma.


\begin{lemma}
\label{extraction}
    Let $\phi:\CC \to \CC$ and $\delta>0, \ b \in \CC, \ k \in \NN_0$, such that $\fres{\phi}{B_\delta(b)} \in C^k      \left(B_\delta(b); \CC\right)$.
    For fixed $z \in \Omega_n$, where we recall that $\Omega_n = [-1,1]^n + i [-1,1]^n$, we consider the map
    \begin{equation*}
        \phi_z: \quad B_{\frac{\delta}{\sqrt{2n}}}(0) \to \CC, \quad w \mapsto \phi\left(w^T z + b\right),
    \end{equation*}
    which is in $C^k$ since for $w \in B_{\frac{\delta}{\sqrt{2n}}}(0) \subseteq \CC^n$ we have
    \begin{equation*}
        \left\vert w^T z\right\vert \leq \Vert w \Vert_2 \cdot \Vert z \Vert_2 < \frac{\delta}{\sqrt{2n}} \cdot \sqrt{2n} = \delta.
    \end{equation*}
    For all multi-indices $\m, \elll \in \NN_0^n$ with $\vert \m + \elll \vert \leq k$ we have
    \begin{equation*}
        \wirt^\m \wirtq^{\elll} \phi_z(w) = z^\m \overline{z}^{\elll} \cdot \left(\wirt^{\vert \m \vert}\wirtq^{\vert \elll \vert}\phi\right) \left(w^T z + b \right)
    \end{equation*}
    for all $w \in B_{\frac{\delta}{\sqrt{2n}}}(0)$.
\end{lemma}

\begin{proof}
    First we prove the statement
    \begin{equation}
    \label{conj}
        \wirtq^{\elll} \phi_z(w) = \overline{z}^{\elll} \cdot (\wirtq^{\vert {\elll} \vert}\phi)\left( w^Tz +b\right) \quad \text{for all } w \in B_{\frac{\delta}{\sqrt{2n}}}(0)
    \end{equation}
    by induction over $0 \leq \vert {\elll} \vert \leq k$. The case ${\elll} = 0$ is trivial. Assuming that (\ref{conj}) holds for fixed ${\elll} \in \NN_0^n$ with $\vert \elll\vert < k $, we want to show
    \begin{equation}
    \label{conj_wirt}
        \wirtq^{{\elll} + e_j} \phi_z(w) = \overline{z}^{{\elll}+e_j} \cdot \left( \wirtq^{|{\elll}| + 1}\phi\right)\left(w^Tz +b\right)
    \end{equation}
    for all $w \in B_{\frac{\delta}{\sqrt{2n}}}(0)$, where $j \in \{1,...,n\}$ is chosen arbitrarily. To this end, first note
    \begin{align*}
        \wirtq^{{\elll} + e_j} \phi_z(w) &= \wirtq^{e_j}\wirtq^{\elll} \phi_z(w)  \overset{\text{induction}}{=} \wirtq^{e_j}\left[w \mapsto \overline{z}^{\elll} \cdot \left(\wirtq^{\vert {\elll} \vert}\phi\right)\left(w^Tz + b \right)\right] \\
        &= \overline{z}^{\elll} \wirtq^{e_j}\left[w \mapsto \left(\wirtq^{\vert {\elll} \vert}\phi\right)\left(w^Tz + b\right)\right].
    \end{align*}
    Applying the chain rule for Wirtinger derivatives and using that
    \begin{equation*}
        \wirtq^{e_j} \left[ w \mapsto w^T z +b\right]= 0
    \end{equation*} since $w \mapsto w^T z  + b$ is holomorphic in every variable, we see
    \begin{align*}
        \wirtq^{e_j}\left[w \mapsto \left(\wirtq^{\vert {\elll} \vert}\phi\right)\left(w^Tz + b\right)\right] &= \left(\wirt\wirtq^{\vert {\elll}\vert}\phi\right) \left(w^Tz +b\right) \cdot \wirtq^{e_j}\left[w \mapsto w^Tz + b\right] \\
        & \hspace{0.4cm}+ \left(\wirtq^{\vert {\elll}\vert + 1}\phi\right) \left(w^Tz +b\right) \cdot \wirtq^{e_j}\left[w \mapsto \overline{w^Tz + b}\right] \\
        &= \left(\wirtq^{\vert {\elll}\vert + 1}\phi\right) \left( w^Tz +b\right) \cdot \overline{\wirt^{e_j}\left[w \mapsto w^Tz + b\right]} \\
        &= \overline{z}^{e_j} \cdot \left(\wirtq^{\vert {\elll}\vert + 1}\phi\right) \left(w^Tz +b\right),
    \end{align*}
    using the fact that $w_j \mapsto w^Tz + b$ is holomorphic and hence
    \begin{equation*}
        \wirtq^{e_j}\left[ w \mapsto w^T z + b\right] = 0 \quad \text{and} \quad \wirt^{e_j}\left[ w \mapsto w^T z + b\right] = z_j.
    \end{equation*}
    Thus, we have proven (\ref{conj_wirt}) and induction yields (\ref{conj}). 

    It remains to show the full claim. We use induction over $\vert \m \vert$ and note that the case $\m=0$ has just been shown. We assume that the claim holds true for fixed $\m \in \NN_0^n$ with $\vert \m + \elll \vert < k$ and choose $j \in \{1,...,n\}$. Thus, we get
    \begin{align*}
        \wirt^{\m + e_j} \wirtq^{{\elll}}\phi_z(w) &= \wirt^{e_j}\wirt^\m \wirtq^{\elll} \phi_z(w) \overset{\text{IH}}{=} \wirt^{e_j}\left( w \mapsto z^\m \overline{z}^{\elll} \cdot\left(\wirt^{\vert \m \vert} \wirtq^{\vert {\elll} \vert} \phi \right)\left(w^T z + b\right)\right) \\
        &= z^\m \overline{z}^{\elll} \cdot \wirt^{e_j}\left[ w \mapsto \left(\wirt^{\vert \m \vert} \wirtq^{\vert {\elll} \vert} \phi \right)\left(w^T z + b\right)\right].
    \end{align*}
    Using the chain rule again, we calculate
    \begin{align*}
        &\norel \wirt^{e_j}\left[w \mapsto \left(\wirt^{\vert \m \vert} \wirtq^{\vert {\elll} \vert} \phi \right)\left(w^T z + b\right)\right] \\
        &= \left(\wirt^{\vert \m \vert +1}\wirtq^{\vert {\elll} \vert}\phi\right)\left(w^Tz + b\right) \cdot \wirt^{e_j}\left[ w \mapsto w^Tz + b\right] \\
        &\norel + \left(\wirt^{\vert \m \vert }\wirtq^{\vert {\elll} \vert + 1}\phi\right)\left(w^Tz + b\right) \cdot \wirt^{e_j}\left[w \mapsto \overline{w^Tz + b}\right] \\
        &= z^{e_j}\cdot \left(\wirt^{\vert \m \vert +1}\wirtq^{\vert {\elll} \vert}\phi\right)\left(w^Tz + b\right) + \left(\wirt^{\vert \m \vert }\wirtq^{\vert {\elll} \vert + 1}\phi\right)\left(w^Tz + b\right) \cdot \overline{\wirtq^{e_j}\left[w \mapsto w^Tz + b\right]} \\
        &=z^{e_j}\cdot \left(\wirt^{\vert \m \vert +1}\wirtq^{\vert {\elll} \vert}\phi\right)\left( w^Tz + b\right).
    \end{align*}
    By induction, this proves the claim.
\end{proof}

As the last preparation for the proof of \Cref{main_1}, we need the following lemma.

\begin{lemma}
\label{previous}
    Let $\phi:\CC \to \CC$ and $\delta>0, \ b \in \CC, \ k \in \NN_0$, such that $\fres{\phi}{B_\delta(b)} \in C^k      \left(B_\delta(b); \CC\right)$. Let $m,n \in \NN$ and $\varepsilon>0$. For $\pp \in \NN_0^{2n}, h>0$ and $z \in \Omega_n$ we write
    \begin{align*}
        \Phi_{\pp,h} (z)  &\defeq (2h)^{-\vert \pp \vert} \sum_{0 \leq \textbf{r} \leq \textbf{p}} (-1)^{\vert \pp \vert -\vert \rr \vert} \binom{\textbf{p}}{\textbf{r}} \left(\phi_z \circ \varphi_n\right)\left(h(2\rr - \pp)\right) \\ 
        &=(2h)^{-\vert \pp \vert} \sum_{0 \leq \textbf{r} \leq \textbf{p}} (-1)^{\vert \pp \vert -\vert \rr \vert} \binom{\textbf{p}}{\textbf{r}} \phi \left( \left[\varphi_n\left(h(2\rr-\pp)\right)\right]^T \cdot z + b\right),
    \end{align*}
where $\phi_z$ is as introduced in \Cref{extraction} and $\varphi_n$ is as in \eqref{isomorphism_intro}.
    Furthermore, let
    \begin{equation*}
        \phi_\pp : \quad \Omega_n \times B_{\frac{\delta}{\sqrt{2n}}}(0)\to \CC, \quad (z,w) \mapsto \partial^\pp \phi_z(w).
    \end{equation*}
    Then there exists $h^* > 0$ such that
    \begin{equation*}
        \Vert \Phi_{\pp,h} - \phi_\pp(\cdot, 0) \Vert_{L^\infty\left(\Omega_n; \CC\right)} \leq \varepsilon
    \end{equation*}
    for all $\pp \in \NN_0^{2n}$ with $\vert \pp \vert \leq k$ and $ \pp \leq m$ and $h \in (0, h^*)$.
\end{lemma}

\begin{proof}
    Fix $\pp \in \NN_0^{2n}$ with $\vert \pp \vert \leq k$ and $  \pp \leq m$. The map
    \begin{equation*}
        B_{\sqrt{2n} + 1}(0) \times B_{\delta / (\sqrt{2n} + 1)} (0) \to \CC, \quad (z,w) \mapsto \phi \left(w^T z + b\right)
    \end{equation*}
    is in $C^k$ since
    \begin{equation*}
        \left\vert w^T z \right\vert \leq \Vert w \Vert \cdot \Vert z \Vert < \frac{\delta}{\sqrt{2n} + 1} \cdot (\sqrt{2n} + 1) = \delta.
    \end{equation*}
    Therefore, the map 
    \begin{equation*}
        B_{\sqrt{2n} + 1}(0) \times B_{\delta / (\sqrt{2n} + 1)} (0) \to \CC, \quad (z,w) \mapsto \partial^\pp \phi_z(w)
    \end{equation*}
    is continuous and in particular uniformly continuous on the compact set
    \begin{equation*}
        \Omega_n \times \overline{B_{\delta/(3n)}}(0) \subseteq B_{\sqrt{2n} + 1}(0) \times B_{\delta / (\sqrt{2n} + 1)}(0).
    \end{equation*}
    Here, we employed $\sqrt{2n}+1 < 3n$ for every $n \in \NN$. Hence, there exists $ h_\pp \in (0,\frac{\delta}{3n \cdot \sqrt{2n} \cdot m})$, such that
    \begin{equation*}
        \left\vert \phi_\pp(z, \xi) - \phi_\pp (z,0)\right\vert \leq \frac{\varepsilon}{\sqrt{2}}
    \end{equation*}
    for all $\xi \in \varphi_{n} \left(h \cdot [-m,m]^{2n}\right), \ h \in (0, h_\pp)$ and $z \in \Omega_n$.
    Now fix such an $h \in (0, h_\pp)$ and $z \in \Omega_n$.
    Applying \Cref{div_differences_mainresult} to both components of
    $\left(\varphi_1^{-1} \circ \Phi_{\pp, h}\right)(z)$
    and $\varphi_1^{-1} \circ \phi_z \circ \fres{\varphi_n}{\left(-\frac{\delta}{3n},\frac{\delta}{3n}\right)^{2n}}$
    separately yields the existence of two real vectors $\xi_{1}, \ \xi_{2} \in h \cdot [-m,m]^{2n}$, such that
    \begin{align*}
        \left(\varphi_1^{-1} \circ \Phi_{\pp,h}(z)\right)_1
        & = \left[\partial^\pp \left(\varphi_1^{-1} \circ \phi_z \circ \varphi_n\right)\left(\xi_1\right)\right]_1 \\
        \text{and} \quad
        \left(\varphi_1^{-1} \circ \Phi_{\pp,h}(z)\right)_2
        & = \left[\partial^\pp \left(\varphi_1^{-1} \circ \phi_z \circ \varphi_n\right)\left(\xi_2\right)\right]_2.
    \end{align*}
    Rewriting this yields
    \begin{equation*}
        \RE\left(\Phi_{\pp,h}(z)\right) = \RE\left(\phi_\pp (z, \varphi_n\left(\xi_1\right))\right) \quad \text{and}\quad  \IM\left(\Phi_{\pp,h}(z)\right) = \IM\left(\phi_\pp (z, \varphi_n\left(\xi_2\right))\right).
    \end{equation*}
    Using this property, we deduce
    \begin{equation*}
        \left\vert \text{Re}\left( \Phi_{\pp,h}(z) - \phi_\pp(z,0)\right)\right\vert = \left\vert \text{Re}\left( \phi_\pp(z, \varphi_n\left(\xi_{1}\right)) - \phi_\pp(z,0)\right)\right\vert \leq \left\vert \phi_\pp(z, \varphi_n\left(\xi_{1}\right)) - \phi_\pp (z,0)\right\vert \leq \frac{\varepsilon}{\sqrt{2}}
    \end{equation*}
    and analogously for the imaginary part. Since $z \in \Omega_n$ and $h \in \left(0, h_\pp \right)$ have been chosen arbitrarily we get the claim by choosing
    \begin{equation*}
        h^* \defeq \text{min} \left\{ h_\pp  : \  \pp \in \NN_0^{2n} \text{ with } \vert \pp \vert \leq k \text{ and }\pp \leq m\right\}. \qedhere
    \end{equation*}
\end{proof}

Using the previous two lemmas and the results from \Cref{sec:div_diff_reordered}
and \Cref{admissibility_reordered}, we can now prove \Cref{main_1}.

\begin{proof}[Proof of \Cref{main_1}]
Let $b \in U$ satisfy 
\begin{equation*}
\wirt ^{\ell_1} \wirtq^{\ell_2} \phi(b) \neq 0 \quad \text{for all } \ell_1, \ell_2 \in \NN_0 \text{ with } \ell_1, \ell_2\leq mn.
\end{equation*}
Such a point $b$ exists according to \Cref{prop:nonpoly}.
Let $p \in \PP'$ and fix $\m, \elll \in \NN_0^{n}$ with $ \m, \elll \leq m$. For each $z \in \Omega_n$, using \Cref{extraction}, we then have
\begin{align}
    z^\m \overline{z}^{\elll} &= \left[\left( \wirt^{\vert\m\vert} \wirtq^{\vert\elll\vert} \phi\right)(b)\right]^{-1}\wirt^\m \wirtq^{\elll} \phi_z (0) \nonumber\\
    \label{eq:monomial_equal}
    \overset{\text{Prop. }\mathrm{\ref{wirtreal}}}&{=} \left[\left( \wirt^{\vert\m\vert} \wirtq^{\vert\elll\vert} \phi\right)(b)\right]^{-1}  \cdot \underset{\pp' + \pp'' = \m + \elll}{\underset{ \pp = (\pp', \pp'') \in \NN_0^{2n}}{\sum}} b_{\pp', \pp''}\partial^{(\pp', \pp'')} \phi_z(0)
\end{align}
with suitably chosen complex coefficients $b_{\pp', \pp''} \in \CC$ depending only on $\pp', \ \pp'', \ \m$ and $\elll$. Here we used that $\vert \m \vert, \ \vert \elll \vert \leq mn$.
Since $\mathcal{P}' \subseteq \mathcal{P}_m^n$ is bounded and $p \in \PP'$, we can write
\begin{equation*}
p(z) = \underset{\m, \elll \leq m}{\sum_{\m, \elll \in \NN_0^n}} a_{\m, \elll} z^\m \overline{z}^{\elll}
\end{equation*}
with $\vert a_{\m, \elll}\vert \leq c$ for some constant $c = c(\PP') > 0$. In combination with \eqref{eq:monomial_equal}, this easily implies that we can rewrite $p$ as
\begin{equation} \label{eq:p_form}
    p(z) = \underset{\pp  \leq 2m}{\sum_{\pp \in \NN_0^{2n}}} c_\pp(p) \partial^\pp \phi_z(0)
\end{equation}
with coefficients $c_\pp(p) \in \CC$ satisfying $\vert c_\pp (p)\vert \leq c'$ for some constant $c' = c'(\phi, b, \PP', m, n)$. By \Cref{previous}, we choose $h^*>0$, such that
\begin{equation*}
    \left\vert \Phi_{\pp,h^*}(z) - \partial^\pp\phi_z(0)\right\vert \leq \frac{\varepsilon}{\sum_{ \qq \in \NN_0^{2n},\qq \leq 2m}c'}
\end{equation*}
for all $z \in \Omega_n$ and $\pp \in \NN_0^{2n}$ with $ \pp \leq 2m$. Furthermore, we can rewrite each function $\Phi_{\pp, h^*}$ as
\begin{equation*}
    \Phi_{\pp, h^*}(z) = \sum_{\underset{\vert \aalpha_j \vert \leq 2m  \ \forall j}{\aalpha \in \Z^{2n}}} \lambda_{\aalpha, \pp} \phi(\left[\varphi_n\left(h^* \aalpha\right)\right]^T \cdot z + b)
\end{equation*}
with suitable coefficients $\lambda_{\aalpha, \pp} \in \CC$. Since the cardinality of the set 
\begin{equation*}
    \left\{ \aalpha \in \Z^{2n} : \ \left\vert \aalpha_j \right\vert \leq 2m \ \forall j\right\}
\end{equation*}
is $(4m+1)^{2n}$, this can be converted to 
\begin{equation*}
   \Phi_{\pp, h^*}(z) =  \sum_{j=1}^N \lambda_{j,\pp} \phi \left( \rho_j^T \cdot z + b\right).
\end{equation*}
For $p$ as in \eqref{eq:p_form}, we then define
\begin{equation*}
    \theta(z) \defeq \underset{\pp \leq 2m}{\sum_{\pp \in \NN_0^{2n}}} c_\pp(p) \cdot \Phi_{\pp, h^*}(z) = \sum_{j=1}^{N} \left[ \left(\underset{\pp \leq 2m}{\sum_{\pp \in \NN_0^{2n}}}c_{\pp}(p) \lambda_{j, \pp} \right)\phi(\rho_j^T \cdot z + b)\right]
\end{equation*}
and note
\begin{equation*}
    \left\vert \theta(z) - p(z)\right\vert \leq \sum_{\pp \leq 2m} \left\vert c_\pp(p) \right\vert \cdot \left\vert \Phi_{\pp, h^*}(z) - \partial^\pp \phi_z(0)\right\vert \leq \varepsilon.
\end{equation*}
Since the coefficients $\rho_j$ have been chosen independently of the polynomial $p$, we can rewrite $\theta$ in the desired form.
\end{proof}
