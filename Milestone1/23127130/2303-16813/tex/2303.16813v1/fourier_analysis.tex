% !TeX encoding = UTF-8
% !TeX spellcheck = en_US
% !TeX root = main_paper.tex

\section{Prerequisites from Fourier Analysis}
This section is dedicated to reviewing some notations and results from Fourier Analysis. In the end, a quantitative result for the approximation of $C^k \left( [-1,1]^s; \RR\right)$-functions using linear combinations of multivariate Chebyshev polynomials is derived. 
\begin{definition}
    Let $s \in \NN$ and $k \in \NN_0$. We define
    \begin{equation*}
        C_{2\pi}^k\left(\RR^s; \CC\right) \defeq \left\{ f \in C^k \left(\RR^s; \CC\right) : \ \forall \pp \in \ZZ^s \ \forall x \in \RR^s : \ f(x + 2\pi \pp ) = f(x)\right\}.
    \end{equation*}
    For a function $f \in C_{2\pi}^k \left(\RR^s; \CC\right)$ we write 
\begin{align*}
 \left\Vert f \right\Vert_{C^k \left([-\pi, \pi]^s ; \CC\right)} &\defeq \underset{\vert \kk \vert \leq k}{\max} \left\Vert \partial^\kk f\right\Vert_{L^\infty \left([-\pi, \pi]; \CC\right)} \text{ and } \\
\left\Vert f \right\Vert_{L^p \left([-\pi, \pi]^s ; \CC \right)} &\defeq \left(\frac{1}{(2\pi)^s} \cdot \int_{[-\pi, \pi]^s} \left\vert f(x) \right\vert^p dx \right)^{1/p} \text{ for } p \in [1, \infty).
\end{align*}
\end{definition}

\begin{definition}
    For any $s \in \NN$ and $\kk \in \Z^s$ we write
    \begin{equation*}
        e_\kk : \quad \RR^s \to \CC, \quad e_\kk (x) = e^{i \langle \kk, x\rangle}.
    \end{equation*}
    For any $f \in C_{2\pi} \left(\RR^s; \CC\right)$ we define the $\kk$-th Fourier coefficient of $f$ to be
    \begin{equation*}
        \hat{f}(\kk) \defeq \frac{1}{(2\pi)^s} \int_{[-\pi, \pi]^s} f(x) \overline{e_{\kk}(x)}dx.
    \end{equation*}
\end{definition}
\begin{definition}
    For two functions $f,g \in C_{2\pi}\left(\RR^s;\CC\right)$ we define their convolution as
    \begin{equation*}
        f * g : \quad \RR^s \to \CC, \quad (f*g)(x) \defeq \frac{1}{(2\pi)^s} \int_{[-\pi, \pi]^s} f(t)g(x-t) dt.
    \end{equation*}
\end{definition}

\begin{definition} 
    In the following we define several so-called kernels. Let $m \in \NN_0$ be arbitrary.
    \begin{enumerate}
        \item The $m$-th one-dimensional \emph{Dirichlet-kernel} is defined as
        \begin{equation*}
            D_m \defeq \sum_{h= - m}^{m} e_h.
        \end{equation*}
        \item The $m$-th one-dimensional \emph{Fejèr-kernel} is defined as
        \begin{equation*}
            F_m \defeq \frac{1}{m}\sum_{h=0}^{m-1} D_h.
        \end{equation*}
        \item The $m$-th one-dimensional \emph{de-la-Vallée-Poussin-kernel} is defined as
        \begin{equation*}
            V_m \defeq \left( 1 + e_m + e_{-m}\right) \cdot F_m.
        \end{equation*}
        \item Let $s \in \NN$. We extend the above definitions to dimension $s$ by letting
        \begin{align*}
            D_m^s \left(x_1, ..., x_s\right) &\defeq \prod_{p=1}^s D_m \left(x_p\right), \\
            F_m^s \left(x_1, ..., x_s\right) &\defeq \prod_{p=1}^s F_m \left(x_p\right), \\
            V_m^s \left(x_1, ..., x_s\right) &\defeq \prod_{p=1}^s V_m \left(x_p\right). 
        \end{align*}
    \end{enumerate}
\end{definition}
\begin{proposition} \label{prop: dlvp-bound}
    Let $m,s \in \NN$. Then one has $\left\Vert V_m^s \right\Vert_{L^1 \left([- \pi, \pi]^s; \CC\right)} \leq 3^s$.
\end{proposition}
\begin{proof}
   From \cite[Exercise 1.3 and Lemma 1.4]{muscalu_classical_2013} it follows $\Vert F_m\Vert_{L^1 ([-\pi, \pi] ; \CC)} = 1$ and hence using the triangle inequality $\Vert V_m \Vert_{L^1 ([-\pi,\pi] ; \CC)} \leq 3$. The claim then follows using Tonelli's theorem.
\end{proof}
\begin{definition}
    For any $s \in \NN, m \in \NN_0$ we call a function of the form
    \begin{equation*}
    \RR^s \to \CC, \quad x \mapsto \sum_{-m \leq \kk \leq m} a_\kk e^{i \langle \kk , x\rangle}
\end{equation*}
with coefficients $a_\kk \in \CC$ a \emph{trigonometric polynomial of coordinatewise degree at most $m$} and denote the space of all those functions with $\mathbb{H}_m^s$. Here, we again consider the sum over all $\kk \in \ZZ^s$ with $-m \leq \kk_j \leq m$ for all $j \in \{1,...,s\}$. We then write
\begin{equation*}
    E^s_m(f) \defeq \underset{T \in \mathbb{H}_{m}^s}{\mathrm{min}} \left\Vert f - T\right\Vert_{L^\infty \left(\RR^s\right)}
\end{equation*}
for any function $f \in C_{2\pi} \left(\RR^s ; \CC\right)$.
\end{definition}

\begin{proposition}
\label{vm}
    Let $s,m \in \NN$ and $k \in \NN_0$. The map
    \begin{equation*}
        v_m : \quad C_{2\pi} \left(\RR^s; \CC\right) \to \mathbb{H}_{2m-1}^s, \quad f \mapsto f * V_m^s
    \end{equation*}
    satisfies
    \begin{equation}
    \label{ident}
        v_m(T) = T
    \end{equation}
    for all $T \in \mathbb{H}_m^s$. Furthermore, there is a constant $c = c(s) > 0$ (independent of $m$), such that
    \begin{align}
        \left\Vert v_m(f)\right\Vert_{C^k\left(\RR^s; \CC\right)} \leq c \cdot \left\Vert f\right\Vert_{C^k\left(\RR^s; \CC\right)} \ \forall f \in C^k_{2\pi} \left(\RR^s; \CC\right), \nonumber \\
        \label{constant}
        \left\Vert v_m(f)\right\Vert_{L^\infty \left(\RR^s; \CC\right)} \leq c \cdot \left\Vert f\right\Vert_{L^\infty \left(\RR^s; \CC\right)} \ \forall f\in C_{2\pi} \left(\RR^s; \CC\right).
    \end{align}
\end{proposition}
\begin{proof}
    The map is well-defined since $V_m$ is a trigonometric polynomial of coordinatewise degree at most $2m-1$. 

    The operator is bounded with norm at most $c = 3^s$ as follows from Young's inequality \cite[Lemma 1.1 (ii)]{muscalu_classical_2013}, \Cref{prop: dlvp-bound} and the fact that one has for all $\kk \in \NN_0^s$ with $\vert \kk \vert \leq k$ the identity
    \begin{equation*}
         \partial^\kk \left( f * V_m^s\right) = \left(\partial^\kk f\right) * V_m^s \quad \text{for } f \in C^k_{2\pi}(\RR^s ; \CC).
    \end{equation*}
    It remains to show that $v_m$ is the identity on $\mathbb{H}_m^s$. We first prove that 
    \begin{equation}
    \label{firstident}
        e_k * V_m = e_k
    \end{equation}
    holds for all $k \in \Z$ with $\vert k \vert \leq m$. First note that
    \begin{equation*}
        e_k * V_m = e_k * F_m + e_k * \left(e_m \cdot F_m\right) + e_k * \left(e_{-m} \cdot F_m\right).
    \end{equation*}
    We then compute
    \begin{align*}
        e_k * F_m = \frac{1}{m} \sum_{\ell = 0}^{m-1} D_\ell * e_k = \frac{1}{m}\sum_{\ell = 0}^{m-1} \sum_{h = - \ell}^{\ell} \underbrace{e_h * e_k}_{= \delta_{k,h} \cdot e_k} = \frac{1}{m} \sum_{\ell = \vert k \vert}^{m-1} e_k = \frac{m - \vert k \vert}{m} \cdot e_k.
    \end{align*}
    Similarly, we get
    \begin{align*}
        e_k *\left( e_m \cdot F_m \right) = \frac{1}{m} \sum_{\ell = 0}^{m-1} \left( e_m D_\ell \right) * e_k = \frac{1}{m}\sum_{\ell = 0}^{m-1} \sum_{h = - \ell}^{\ell} \underbrace{e_{h+m} * e_k}_{= \delta_{k,h+m} \cdot e_k} = \frac{1}{m} \underset{\ell \geq m-k}{\sum_{0 \leq \ell \leq m-1 }} e_k = \delta_{k \geq 1} \cdot \frac{k}{m} \cdot e_k
    \end{align*}
    and 
    \begin{align*}
        e_k *\left( e_{-m} \cdot F_m \right) = \frac{1}{m} \sum_{\ell = 0}^{m-1} \left( e_{-m} D_\ell \right) * e_k = \frac{1}{m}\sum_{\ell = 0}^{m-1} \sum_{h = - \ell}^{\ell} \underbrace{e_{h-m} * e_k}_{= \delta_{k,h-m} \cdot e_k} = \frac{1}{m} \underset{\ell \geq k+m}{\sum_{0 \leq \ell \leq m-1}} e_k = \delta_{k \leq -1} \cdot \frac{-k}{m} \cdot e_k.
    \end{align*}
    Adding up those three identities yields (\ref{firstident}). 

    Clearly, it suffices to show that 
    \begin{equation*}
        e_\kk * V_m^s = e_\kk
    \end{equation*}
    for all $\kk \in \Z^s$ with $-m \leq \kk \leq m$. But for such $\kk$, using $e_\kk(x) = \prod_{j=1}^{s} e_{\kk_j}\left( x_j\right)$, one obtains
    \begin{align*}
        \left(e_\kk * V_m^s\right)(x) &= \frac{1}{(2\pi)^s} \int_{[-\pi, \pi]^s} \prod_{j=1}^{s} e_{\kk_j} \left(t_j\right) \cdot V_m \left(x_j - t_j\right) dt \\
        \overset{\text{Fubini}}&{=} \prod_{j=1}^s \left(e_{\kk_j} * V_m\right)\left(x_j \right) 
        \overset{\eqref{firstident}}{=} \prod_{j=1}^s e_{\kk_j} \left(x_j\right) 
        = e_\kk(x)
    \end{align*}
    for any $ x \in \RR^s$, as was to be shown.
\end{proof}
The following result follows from a theorem in \cite{lorentz_approximation_2005}.
\begin{proposition}
\label{lorentz_aussage}
    Let $s \in \NN$, $k \in \NN_0$. Then there is a constant $c = c(s,k) > 0$, such that
    \begin{equation*}
        E_m^s(f) \leq \frac{c}{m^k} \cdot \left\Vert f\right\Vert_{C^k\left([-\pi, \pi]^s; \RR\right)}
    \end{equation*}
    for all $m\in \NN$ and $f \in C^k_{2\pi} \left(\RR^s; \RR\right)$.
\end{proposition}
\begin{proof}
    We apply \cite[p. 87, Theorem 6]{lorentz_approximation_2005} with $n_i = m$ and $p_i = k$ which yields the existence of a constant $c_1 = c_1(s,k)>0$, such that
    \begin{equation*}
        E_m^s(f) \leq c_1 \cdot \sum_{\ell=1}^s \frac{1}{m^k} \cdot \omega_\ell \left(\frac{1}{m}\right)
    \end{equation*}
    for all $m \in \NN$ and $f \in C^k_{2\pi}(\RR^s ; \RR)$, where $\omega_\ell$ denotes the modulus of continuity of $\frac{\partial^k f }{\partial x_\ell ^k}$, where we have the trivial bound
    \begin{equation*}
         \omega_\ell \left(\frac{1}{m}\right)  \leq 2 \cdot \left\Vert f \right\Vert_{C^k\left([-\pi, \pi]^s; \RR\right)}.
    \end{equation*}
    Hence, we get
    \begin{equation*}
        E_m^s(f) \leq c_1 \cdot s \cdot 2 \cdot \left\Vert f \right\Vert_{C^k\left([-\pi, \pi]^s; \RR\right)} \frac{1}{m^k}.
    \end{equation*}
    so the claim follows by choosing $c := 2s \cdot c_1$.
\end{proof}
\begin{theorem}
\label{vm_approx}
    Let $s \in \NN$. Then there is a constant $ c = c(s) > 0 $, such that the operator $v_m$ from \Cref{vm} satisfies
    \begin{equation*}
        \left\Vert f - v_m(f) \right\Vert_{L^\infty \left( \RR^s\right)} \leq c \cdot E^s_m(f)
    \end{equation*}
    for any $m \in \NN$ and $f \in C_{2\pi} \left(\RR^s; \CC\right)$.
\end{theorem}
\begin{proof}
    For any $T \in \mathbb{H}_m^s$ one has
    \begin{equation*}
        \left\Vert f - v_m(f) \right\Vert_{L^\infty \left( \RR^s\right)} \overset{\eqref{ident}}{\leq} \left\Vert f - T \right\Vert_{L^\infty \left( \RR^s\right)} + \left\Vert v_m(T) - v_m(f) \right\Vert_{L^\infty \left( \RR^s\right)} \overset{\eqref{constant}}{\leq} (c+1) \left\Vert f - T \right\Vert_{L^\infty \left( \RR^s\right)}. \qedhere
    \end{equation*}
\end{proof}
By combining \Cref{lorentz_aussage} and \Cref{vm_approx}, we get the following bound.
\begin{corollary}
\label{dlvp}
    Let $s \in \NN, k \in \NN_0$. Then there is a constant $c = c(s,k)>0$, such that
    \begin{equation*}
        \left\Vert f - v_m(f) \right\Vert_{L^\infty \left(\RR^s\right)} \leq \frac{c}{m^k} \cdot \left\Vert f \right\Vert_{C^k \left(\RR^s; \RR\right)}
    \end{equation*}
    for every $m \in \NN$ and $f \in C^k_{2\pi} \left(\RR^s ; \RR\right)$.
\end{corollary}
\begin{lemma}
\label{star_operator}
    Let $k \in \NN_0$ and $ s \in \NN$. For any function $f \in C^k \left([-1,1]^s ; \CC\right)$ we define the corresponding periodic function via
    \begin{equation*}
        f^* : \quad \RR^s \to \CC, \quad  f^* \left(x_1, ..., x_s\right) = f(\mathrm{cos} \left(x_1\right), ..., \mathrm{cos} \left( x_s\right))
    \end{equation*}
    and note $f^* \in C_{2\pi}^k \left(\RR^s; \CC\right)$. The map
\begin{equation*}
	C^k \left([-1,1]^s ; \CC\right) \to C^k _ {2\pi}\left(\RR^s; \CC\right), \quad f \mapsto f^*
\end{equation*}
is a continuous linear operator with respect to the $C^k$-norms on $C^k \left([-1,1]^s ; \CC\right)$ and $C^k _ {2\pi}\left(\RR^s; \CC\right)$.
\end{lemma}
\begin{proof}
	The map is well-defined since $\cos$ is a smooth function and $2\pi$-periodic. The linearity of the operator follows directly so it remains to show its continuity.

	The goal is to apply the closed graph theorem \cite[Theorem 5.12]{folland_real_1999}. By definition, we have the equality $\left\Vert f\right\Vert_{L^\infty \left([-1, 1]^s; \CC\right)} = \left\Vert f^* \right\Vert_{L^\infty \left([-\pi, \pi]^s; \CC\right)} $. Let then $\left(f_n\right)_{n \in \NN}$ be a sequence of functions $f_n \in C^k \left([-1,1]^s ; \CC\right)$ and $g^* \in C^k_{2\pi}\left(\RR^s ; \CC\right)$ such that $f_n \to f$ in $C^k \left([-1,1]^s; \CC\right)$ and $f_n^* \to g^*$ in $C^k_{2\pi} \left(\RR^s; \CC\right)$. We then have
\begin{align*}
	\left\Vert f^* - g^* \right\Vert_{L^\infty \left([-\pi, \pi]^s\right)} &\leq \left\Vert f^* - f_n^* \right\Vert_{L^\infty \left([-\pi, \pi]^s\right)} + \left\Vert f_n^* - g^* \right\Vert_{L^\infty \left([-\pi, \pi]^s\right)} \\
&\leq \left\Vert f - f_n \right\Vert_{L^\infty  \left([-1, 1]^s ; \CC\right)} + \left\Vert f_n^* - g^* \right\Vert_{L^\infty \left([-\pi, \pi]^s\right)} \\
&\leq \left\Vert f - f_n \right\Vert_{C^k  \left([-1, 1]^s ; \CC\right)} + \left\Vert f_n^* - g^* \right\Vert_{C^k(\RR^s ; \CC)} \to 0 \ (n \to \infty).
\end{align*}
It follows $f^* = g^*$ and the closed graph theorem yields the desired continuity.
\end{proof}

The following identity is a generalization of the well-known product-to-sum formula for $\cos$.
\begin{lemma}
\label{prod_sum}
Let $s \in \NN$. Then it holds for any $x \in \RR^s$ that
\begin{equation*}
\prod_{j=1}^s \cos (x_j) = \frac{1}{2^s} \sum_{\sigma \in \{-1,1\}^s} \cos (\langle\sigma,x \rangle).
\end{equation*} 
\end{lemma}
\begin{proof}
This is an inductive generalization of the product-to-sum formula 
\begin{equation}
\label{product_sum_form}
2\cos(x) \cos(y) = \cos(x-y) + \cos (x+y)
\end{equation}
for $x,y \in \RR$, which can be found for instance in \cite[4.3.32]{abramowitz_handbook_2013}. The case $s= 1$ follows since $\cos$ is an even function. Assume that the claim holds for a fixed $s \in \NN$ and take $x \in \RR^{s+1}$. Writing $x' = (x_1, ..., x_s)$ we derive
\begin{align*}
\prod_{j=1}^{s+1} \cos(x_j) &= \left(\frac{1}{2^s} \sum_{\sigma \in \{-1,1\}^s} \cos (\langle \sigma , x' \rangle) \right) \cdot \cos(x_{s+1}) \\
&= \frac{1}{2^s} \sum_{\sigma \in \{-1,1\}^s}\cos (\langle \sigma, x' \rangle) \cos(x_{s+1})\\
\overset{\eqref{product_sum_form}}&{=} \frac{1}{2^{s+1}} \sum_{\sigma \in \{-1,1\}^s} \left[\cos (\langle \sigma, x' \rangle + x_{s+1}) + \cos (\langle \sigma, x' \rangle - x_{s+1}) \right]  \\
&= \frac{1}{2^{s+1}} \sum_{\sigma \in \{-1,1\}^{s+1}} \cos (\langle \sigma, x\rangle), 
\end{align*}
as was to be shown. 
\end{proof}
\begin{proposition}
\label{representation}
    For any $f \in C^k \left([-1,1]^s ; \CC\right)$ and $m \in \NN$ the de-la-Vallée-Poussin operator given as $f \mapsto v_m \left(f^*\right)$ has a representation
    \begin{equation*}
        v_m \left(f^*\right) \left(x_1, ..., x_s\right) = \underset{\kk \leq 2m -1}{\sum_{\kk \in \NN_0^s}} \mathcal{V}_\kk^m(f) \prod_{j=1}^{s} \mathrm{cos} \left(\kk_j x_j\right)
    \end{equation*}
    for continuous linear functionals
    \begin{equation*}
        \mathcal{V}_{\kk}^m : \ C^k \left([-1,1]^s; \CC\right) \to \CC.
    \end{equation*}
Furthermore, if $f \in C^k ([-1,1]^s ; \RR)$ we have $\mathcal{V}_\kk^m (f) \in \RR$ for every $\kk \in \NN_0^s$ with $\kk \leq 2m-1$.
\end{proposition}
\begin{proof}
    First of all, it is easy to see that $v_m \left( f^* \right)$ is even in every variable, which follows directly from the fact that $ f^*$ and $V_m^s$ are both even in each variable. Furthermore, if we write 
    \begin{equation*}
        V_m^s = \underset{-(2m-1) \leq \kk \leq 2m-1}{\sum_{\kk \in \ZZ^s}} a_\kk^m e_\kk
    \end{equation*}
    with appropriately chosen coefficients $a_\kk^m \in \RR$, we easily see
    \begin{equation*}
        v_m \left(f^*\right) = \underset{-(2m-1) \leq \kk \leq 2m-1}{\sum_{\kk \in \ZZ^s}} a_\kk^m \hat{f^*}(\kk) e_\kk.
    \end{equation*}
    Using Euler's identity and the fact that $v_m\left(f^*\right)$ is an even function, we get the representation 
    \begin{equation*}
        v_m \left(f^*\right)(x) = \sum_{\vert \kk \vert \leq 2m-1} a_\kk^m \hat{f^*}(\kk) \text{cos}(\langle \kk, x\rangle)
    \end{equation*}
    for any $x \in \RR^s$. Using $\cdot$ to denote the componentwise product of two vectors of the same size, we see since $v_m \left(f^*\right)$ is even in every variable that
    \begin{align*}
        v_m \left(f^*\right) (x) &= \frac{1}{2^s} \cdot \sum_{\sigma \in \{-1,1\}^s} v_m \left( f^* \right) (\sigma \cdot x) \\
        &=\frac{1}{2^s} \cdot \sum_{\sigma \in \{-1,1\}^s} \underset{-(2m-1) \leq \kk \leq 2m-1}{\sum_{\kk \in \ZZ^s}} a_\kk^m \hat{f^*}(\kk) \text{cos}(\langle \kk, \sigma \cdot x\rangle) \\
        &= \underset{-(2m-1) \leq \kk \leq 2m-1}{\sum_{\kk \in \ZZ^s}} \left(a_\kk^m \hat{f^*}(\kk) \frac{1}{2^s} \sum_{\sigma \in \{-1,1\}^s}\text{cos}(\langle \kk, \sigma \cdot x\rangle)\right) \\
        \overset{\text{\Cref{prod_sum}}}&{=} \underset{-(2m-1) \leq \kk \leq 2m-1}{\sum_{\kk \in \ZZ^s}} a_\kk^m \hat{f^*}(\kk) \prod_{j=1}^{s} \cos \left(\kk_j x_j\right) \\
        &= \underset{\kk \leq 2m-1}{\sum_{\kk \in \NN_0^s}} 2^{\Vert \kk \Vert_0} a_\kk^m \hat{f^*}(\kk) \prod_{j=1}^{s} \cos \left(\kk_j x_j\right)
    \end{align*}
    with
    \begin{equation*}
        \left\Vert \kk \right\Vert_0 \defeq \#\big\{ j \in \{1,...,s\} : \ \kk_j \neq 0\big\}.
    \end{equation*}
    In the last step we again used that cos is an even function and that
    \begin{equation*}
        \hat{f^*}(\kk) = \hat{f^*}(\sigma \cdot \kk)
    \end{equation*}
    for all $\sigma \in \{-1,1\}^s$, which also follows easily since $f^*$ is even in every component. Letting \begin{equation*}
        \mathcal{V}_\kk^m(f) \defeq 2^{\Vert \kk \Vert_0} a_\kk^m \hat{f^*}(\kk)
    \end{equation*} 
    we have the desired form. The fact that $\mathcal{V}_{\kk}^m$ is a continuous linear functional on $C^k_{2\pi}\left([-1,1]^s;\CC\right)$ follows directly since $f \mapsto \widehat{f^*}(\kk)$ is a continuous linear functional for every $\kk$. If $f$ is real-valued, so is $\hat{f^*}(\kk)$ for every $\kk \in \NN_0^s$ with $\kk \leq 2m-1$, since $f^*$ is real-valued and even in every component.
\end{proof}
\begin{lemma}
\label{sum_bound}
    Let $s \in \NN$. There is a constant $c = c(s)> 0$, such that the inequality
    \begin{equation*}
        \underset{\kk \leq 2m-1}{\sum_{\kk \in \NN_0^s}} \left\vert \mathcal{V}_\kk^m(f) \right\vert \leq c \cdot m^{s/2} \cdot \left\Vert f \right\Vert_{L^\infty \left([-1, 1]^s ; \CC\right)}
    \end{equation*}
    holds for any $m \in \NN$ and $f \in C \left([-1, 1]^s ; \CC\right)$, where $\mathcal{V}_\kk^m$ is as in \Cref{representation}.
\end{lemma}
\begin{proof}
    Let $f \in C \left([-1, 1]^s ; \CC\right)$ and $m \in \NN$. For any multi-index $\elll \in \NN_0^s$ one has
    \begin{equation*}
        \hat{v_m\left(f^*\right)} (\elll) = \underset{\kk \leq 2m-1}{\sum_{\kk \in \NN_0^s}} \mathcal{V}_\kk^m(f) \hat{g_\kk}(\elll),
    \end{equation*}
    with
    \begin{equation*}
        g_\kk : \quad \RR^s \to \RR, \quad \left(x_1, ..., x_s\right) \mapsto \prod_{j=1}^s \cos \left(\kk_j x_j\right).
    \end{equation*}
    Now, a calculation using Fubini's theorem and using $g_k = \frac{1}{2} \left( e_k + e_{-k}\right)$ for any number $k \in \NN_0$ shows 
    \begin{equation*}
        \hat{g_\kk}(\elll) =  \begin{cases} \frac{1}{2^{\Vert \kk \Vert_0}},&\kk=\elll, \\ 0, & \text{otherwise}\end{cases} \quad\text{for } \kk, \elll \in \NN_0^s.
    \end{equation*}
    Therefore, we have the bound $\left\vert\mathcal{V}^m_{\elll} (f)\right\vert \leq 2^s \cdot \left\vert \hat{v_m\left(f^*\right)}(\elll)\right\vert$ for $\elll \in \NN_0^s $ with $\vert \elll \vert \leq 2m-1$. Using Cauchy-Schwarz' and Parseval's inequality one sees
    \begin{align*}
        \underset{\kk \leq 2m-1}{\sum_{\kk \in \NN_0^s}} \left\vert \mathcal{V}_\kk^m(f)\right\vert &\leq 2^s \cdot \underset{\kk \leq 2m-1}{\sum_{\kk \in \NN_0^s}} \left\vert \hat{v_m\left(f^*\right)}(\kk)\right\vert \overset{\text{CS}}{\leq} 2^s \cdot (2m)^{s/2} \cdot \left(\underset{\kk \leq 2m-1}{\sum_{\kk \in \NN_0^s}} \left\vert \hat{v_m\left(f^*\right)}(\kk)\right\vert^2 \right)^{1/2} \\
        \overset{\text{Parseval}}&{\leq} 2^s \cdot 2^{s/2} \cdot m^{s/2} \cdot \left\Vert v_m \left(f^*\right)\right\Vert_{L^2 \left([-\pi, \pi]^s; \CC\right)} \\
        &\leq \underbrace{2^s \cdot 2^{s/2} }_{=: c_1(s)}\cdot m^{s/2} \cdot \left\Vert v_m \left(f^*\right)\right\Vert_{L^\infty \left([-\pi, \pi]^s; \CC\right)}.
    \end{align*}
    Using \Cref{vm} we get
    \begin{align*}
        \underset{\kk \leq 2m-1}{\sum_{\kk \in \NN_0^s}} \left\vert \mathcal{V}_\kk^m(f)\right\vert &\leq c_1(s) \cdot c_2(s) \cdot m^{s/2} \cdot \left\Vert f^*\right\Vert_{L^\infty \left([-\pi, \pi]^s\right)} = c(s) \cdot m^{s/2} \cdot \left\Vert f\right\Vert_{L^\infty \left([-1, 1]^s; \CC\right)},
    \end{align*}
    as claimed.
\end{proof}

For any natural number $\ell \in \NN_0$, we denote by $T_\ell$ the $\ell$-th \emph{Chebyshev polynomial}, satisfying
\begin{equation*}
    T_\ell\left(\cos(x)\right) = \cos(\ell x), \quad x \in \RR.
\end{equation*}
For a multi-index $\kk \in \NN_0^s$, we define
\begin{equation*}
    T_\kk (x) \defeq \prod_{j=1}^s T_{\kk_j}\left(x_j\right), \quad x \in [-1,1]^s.
\end{equation*}
We then get the following approximation result.
\begin{theorem} \label{app: fourier_approx}
    Let $s, m \in \NN$, $k \in \NN_0$. Then there is a constant $c=c(s,k)>0$ with the following property: For any $f \in C^k \left([-1,1]^s; \RR\right)$ the polynomial $P$ defined as
    \begin{equation*}
        P(x) \defeq \underset{\kk \leq 2m-1}{\sum_{\kk \in \NN_0^s}}\mathcal{V}_\kk^m(f) \cdot T_\kk(x)
    \end{equation*}
    with $\mathcal{V}^m_\kk$ as in \Cref{representation} satisfies
    \begin{equation*}
        \left\Vert f - P \right\Vert_{L^\infty \left([-1,1]^s ;\RR\right)} \leq \frac{c}{m^k} \cdot \left\Vert f \right\Vert_{C^k\left([-1,1]^s;\RR\right)}.
    \end{equation*}
    Here, the maps
\begin{equation*}
C\left([-1,1]^s ; \RR\right) \to \RR, \quad f \mapsto \mathcal{V}_{\kk}^m(f)
\end{equation*}
are continuous and linear functionals with respect to the $L^\infty$-norm. Furthermore, there is a constant $\tilde{c} = \tilde{c}(s)> 0$, such that the inequality
    \begin{equation*}
        \underset{\kk \leq 2m-1}{\sum_{\kk \in \NN_0^s}} \left\vert \mathcal{V}_\kk^m(f) \right\vert \leq \tilde{c} \cdot m^{s/2} \cdot \left\Vert f \right\Vert_{L^\infty \left([-1, 1]^s\right)}
    \end{equation*}
    holds for any $f \in C \left([-1, 1]^s ; \RR\right)$.
    
\end{theorem}
\begin{proof}
    We choose the constant $c_0 = c_0(s,k)$ according to \Cref{dlvp}. Let $f \in C^k\left([-1,1]^s;\RR\right)$ be arbitrary. Then we define the corresponding function $f^* \in C^k_{2\pi}\left(\RR^s; \RR\right)$ as above. Let $P$ be defined as in the statement of the theorem. Then the property
    \begin{equation*}
        P^* (x) = v_m\left(f^*\right)(x)
    \end{equation*}
    is satisfied, where $P^*$ is the corresponding function to $P$ defined similarly to $f^*$. Overall we get the bound
    \begin{align*}
        \left\Vert f - P \right\Vert_{L^\infty \left([-1,1]^s; \RR\right)} = \left\Vert f^* - P^* \right\Vert_{L^\infty \left([-\pi,\pi]^s; \RR\right)} \overset{\text{Cor. \ref{dlvp}}}{\leq} \frac{c_0}{m^k} \cdot \left\Vert f^* \right\Vert_{C^k\left(\RR^s; \RR\right)}.
    \end{align*}
   The first claim then follows using the continuity of the map $f \mapsto f^*$ as proven in \Cref{star_operator}. The second part of the Theorem has already been proven in \Cref{sum_bound}.
\end{proof}