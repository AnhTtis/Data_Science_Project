% !TeX encoding = UTF-8
% !TeX spellcheck = en_US
% !TeX root = main_paper.tex
\pdfoutput=1 
\documentclass[10pt, reqno]{amsart}

\usepackage[T1]{fontenc}
\usepackage[utf8]{inputenc}
\usepackage{lmodern}
\usepackage[USenglish]{babel}
\usepackage{amsmath,amssymb,amsthm,upgreek}
\usepackage{aligned-overset}
\usepackage{mathtools}
\usepackage{colortbl,color} % colors
\usepackage[dvipsnames]{xcolor}
\usepackage{graphicx,tikz,pgfplots,pgfplotstable} % graphics
\usepackage{enumitem}
\pgfplotsset{compat=1.6}
\usepackage{subcaption,float,longtable}
\usepackage[bookmarks=true,bookmarksnumbered,plainpages=false,linktocpage,colorlinks=true,citecolor=green!80!black,linkcolor=red!70!black,filecolor=magenta,urlcolor=magenta,hidelinks,breaklinks,pdfauthor={Paul Geuchen, Felix Voigtlaender},pdftitle={Optimal_Approximation_with_shallow_CVNNs},unicode=true]{hyperref}
\usepackage[nameinlink,capitalize,noabbrev]{cleveref}
%\usepackage{cite}

%\usepackage{setspace}
%\setstretch{1.1}

\usepackage{adforn} %symbols for iteration comments
\usepackage{bbm}
\usepackage{csquotes}
\usepackage{comment}
\usepackage[margin=3cm]{geometry}
%\usepackage{biblatex-readbbl}
\usepackage[style = numeric, url = false, giveninits=true, eprint = false, sortcites=true,backend = biber, maxbibnames=99]{biblatex}
\addbibresource{references.bib}
\AtBeginBibliography{\small}
\AtEveryBibitem{\clearlist{language}}
\AtEveryBibitem{\clearfield{issn}}
\AtEveryCitekey{\clearfield{issn}}


\usepackage{ifthen}

\usetikzlibrary{decorations.fractals,calc,intersections,through,backgrounds,arrows,patterns,shapes.geometric,shapes.misc,spy,fadings,decorations.pathreplacing}
\newcommand{\rus}[1]{\selectlanguage{russian}{\fontfamily{cmr}\fontsize{10pt}{12pt}\selectfont #1}\selectlanguage{USenglish}}
\newcommand{\bibrus}[1]{\selectlanguage{russian}{\fontfamily{cmr}\fontsize{8pt}{10pt}\selectfont #1}\selectlanguage{USenglish}}


%abbreviations

%\newcommand{\enquote}[1]{``#1''}    

\newcommand{\hint}[1]{\textcolor{red}{#1}}
\let\emptyset\varnothing
\newcommand{\ii}{\mathrm{i}}
\newcommand{\ee}{\mathrm{e}}
\newcommand{\dd}{\mathrm{d}}
\newcommand{\NN}{\mathbb{N}}                                     % natural numbers
\newcommand{\RR}{\mathbb{R}}                                     % real numbers
\newcommand{\ZZ}{\mathbb{Z}}                                     % integer numbers
\newcommand{\QQ}{\mathbb{Q}}                                     % rational numbers
\newcommand{\CC}{\mathbb{C}}
\newcommand{\PP}{\mathcal{P}}
\newcommand{\pp}{\textbf{p}}
%\newcommand{\C}{\mathbb{C}}\newcommand{\R}{\mathbb{R}} \newcommand{\Q}{\mathbb{Q}} \newcommand{\Z}{\mathbb{Z}} \newcommand{\N}{\mathbb{N}} 
\newcommand{\fall}{\:\forall\:}                                  % for all
\newcommand{\ex}{\:\exists\:}                                    % exists
\newcommand{\abs}[1]{\left\lvert#1\right\rvert}                  % absolute value, variable sized delimiters
\newcommand{\abss}[1]{\lvert#1\rvert}                            % absolute value, small delimiters
\newcommand{\absb}[1]{\big\lvert#1\big\rvert}						 % absolute value, big delimiters
\newcommand{\mnorm}[1]{\left\lVert#1\right\rVert}                % another norm, variable sized delimiters
\newcommand{\mnorms}[1]{\lVert#1\rVert}                          % another norm, small delimiters
\newcommand{\setn}[1]{\left\{#1\right\}}                         % set, variable sized delimiters
\newcommand{\setns}[1]{\{#1\}}                                   % set, small delimiters
\newcommand{\setcond}[2]{\left\{#1 \::\: #2\right\}}  % set with condition, variable sized delimiters
%\newcommand{\setcond}[2]{\left\{#1 \:\middle\vert\: #2\right\}}  % set with condition, variable sized delimiters
%\newcommand{\setconds}[2]{\{#1 \:\vert\: #2\}}                   % set with condition, small delimiters
\newcommand{\defeq}{\mathrel{\mathop:}=}                               % defining equality
\newcommand{\defequiv}{\mathrel{\mathop:}\Leftrightarrow}
\newcommand{\lr}[1]{\!\left(#1\right)}                           % variable sized parentheses
\newcommand{\p}{\partial}                                        % subdifferential and partial derivatives
\newcommand{\skpr}[2]{\left\langle#1 \,\middle\vert\, #2\right\rangle}               % inner product
\newcommand{\norel}{\mathrel{\phantom{=}}}                                           % phantom equal sign
\newcommand{\noequiv}{\mathrel{\phantom{\Longleftrightarrow}}}                       % phantom equivalence
\newcommand{\boxx}{\ensuremath{\mbox{\small$\,\square\,$}}}                          % infimal convolution
\newcommand{\eqdef}{=\mathrel{\mathop:}}                                             % reversed defining equation
\newcommand{\floor}[1]{\left\lfloor #1\right\rfloor}
\newcommand{\ceil}[1]{\left\lceil #1\right\rceil}

\newcommand{\smpl}{\Xi}
\newcommand{\sph}{\mathbb{S}}
\newcommand{\prt}{\mathcal{Z}}
\newcommand{\Z}{\mathbb{Z}}
\newcommand{\TT}{\mathbb{T}}

\newcommand{\rr}{\textbf{r}}
\newcommand{\pM}[1]{ \begin{pmatrix} #1  \end{pmatrix} }
\newcommand{\psM}[1]{ \begin{psmallmatrix} #1  \end{psmallmatrix} }
\renewcommand{\hat}{\widehat}
\renewcommand{\tilde}{\widetilde}

\DeclareMathOperator{\proj}{Pr}
\DeclareMathOperator{\ext}{ext}
\DeclareMathOperator{\prox}{prox}
\DeclareMathOperator*{\esssup}{ess\,sup}
\DeclareMathOperator{\id}{id}
\DeclareMathOperator{\diam}{diam}
\DeclareMathOperator{\inte}{int}
\DeclareMathOperator{\card}{card}
\DeclareMathOperator{\RE}{Re}
\DeclareMathOperator{\IM}{Im}
\DeclareMathOperator{\spann}{span}
\DeclareMathOperator{\supp}{supp}
\let \eps \varepsilon
\let \piup \uppi

\theoremstyle{plain} 
\newtheorem{theorem}{Theorem}[section]
\newtheorem{corollary}[theorem]{Corollary}
\newtheorem{lemma}[theorem]{Lemma}
\newtheorem{proposition}[theorem]{Proposition}
\newtheorem{Folg}[theorem]{Consequence}
\newtheorem{Verm}[theorem]{Conjecture}
\newtheorem{Probl}[theorem]{Problem}
\newtheorem*{Fakt}{Fact}

\theoremstyle{definition} % definition, condition, problem, example
\newtheorem{definition}[theorem]{Definition}
\newtheorem{Bsp}[theorem]{Example}
\newtheorem{Bem}[theorem]{Remark}
\newtheorem{Alg}[theorem]{Algorithm}

\theoremstyle{remark} % definition, condition, problem, example
\newtheorem{remark}[theorem]{Remark}
\newtheorem{myproof}[theorem]{Proof}

%For usage with cleveref and amsthm, put triples of environment names, singular forms, and plural forms here.
\crefname{theorem}{theorem}{theorems}
\crefname{Prop}{Proposition}{Propositions}
\crefname{Lem}{Lemma}{Lemmas}
\crefname{Kor}{Corollary}{Corollaries}
\crefname{Bem}{Remark}{Remarks}
\crefname{Bsp}{Example}{Examples}
\crefname{Def}{Definition}{Definitions}
\crefname{Alg}{Algorithm}{Algorithms}

\numberwithin{equation}{section}

%Redefines amsmath's \eqref command in order to have not only the equation numbers hyperlinked but also the parentheses
\makeatletter
\renewcommand*{\eqref}[1]{%
  \hyperref[{#1}]{\textup{\tagform@{\ref*{#1}}}}%
}
\makeatother

%When using cleveref package and putting three or more labels into one \cref, the last item is preceded by "and". The comma which is missing in the default version is added here.
\newcommand{\creflastconjunction}{, and~}
\newcommand{\creflastgroupconjunction}{, and~}

%Define \textcommabelow for Romanian alphabet
\makeatletter
\@ifundefined{textcommabelow}{%
  \DeclareTextCommandDefault\textcommabelow[1]
    {\hmode@bgroup\ooalign{\null#1\crcr\hidewidth\raise-.31ex
     \hbox{\check@mathfonts\fontsize\ssf@size\z@
     \math@fontsfalse\selectfont,}\hidewidth}\egroup}%
}{}
\makeatother

\definecolor{pgcol}{rgb}{0,0,1.} % Blue
\definecolor{tjcol}{cmyk}{.74, 0, 1, .41} % Green
\definecolor{hmcol}{rgb}{0.9,.5,0} % Orange


\newcommand{\pg}[1]{{\color{pgcol}{[P: #1]}}}
\newcommand{\tj}[1]{{\color{tjcol}{[T: #1]}}}
\newcommand{\hm}[1]{{\color{hmcol}{{[H: #1]}}}}

%%%%%%%%%%%%%%%% macros for iterations %%%%%%%%%%%%%%%%%
\newcommand{\xx}{\mbox{\textcolor{red}{$\clubsuit$}}}
\newcommand{\xxx}[1]{\xx{\textcolor{red}{#1}}\xx}
\newcommand{\colory}{\color[rgb]{0,.6,.4}}
\newcommand{\discuss}{{\color[rgb]{0,.96,0}{discuss\,\,}}}
\newcommand{\yy}{\mbox{\textcolor{blue}{$\spadesuit$}}}
\newcommand{\yyy}[1]{{\yy\color{blue}#1\yy}}
\newcommand{\omitted}[1]{}
\newcommand{\DONE}{\mbox{\textcolor{green}{$\surd $}}}
%%%%%%%%%%%%%%%%%%%%%%%%%%%%%%%%%%
\definecolor{islamicgreen}{rgb}{0.0, 0.56, 0.0}
\definecolor{darkpastelgreen}{rgb}{0.01, 0.75, 0.24}
\newcommand{\zzzcolorone}{islamicgreen} %WildStrawberry
\newcommand{\zzzcolortwo}{darkpastelgreen}
\newcommand{\zzzleft}{\mbox{\textcolor{\zzzcolortwo}{\adfflatleafleft}}}%\adfflowerleft
\newcommand{\zzzright}{\mbox{\textcolor{\zzzcolortwo}{\adfflatleafright}}}%\adfflowerright
\newcommand{\zzz}[1]{\zzzleft\,{\leavevmode\color{\zzzcolorone}{#1}}\,\zzzright}
\newcommand{\zz}{\zzzleft}
\newcommand{\elll}{\boldsymbol{\ell}}
\newcommand{\aalpha}{\boldsymbol{\alpha}}
\newcommand{\m}{\textbf{m}}
\newcommand{\wirt}{\partial_{\mathrm{wirt}}}
\newcommand{\wirtq}{\overline{\partial}_{\mathrm{wirt}}}
\newcommand*{\fres}[2]{ {\left.\kern-\nulldelimiterspace #1 \vphantom{\big|} \right|_{\kern-1pt #2} }}
\newcommand{\modrelu}{\sigma_{\mathrm{modReLU}, b}}
%\newcommand{\card}{\mathrm{card}}
\newcommand{\kk}{\textbf{k}}
%\newcommand{\RE}{\operatorname{Re}}
%\newcommand{\IM}{\operatorname{Im}}
\usepackage{url}
\def\UrlBreaks{\do\/\do-}

\begin{document}
\allowdisplaybreaks
%\parindent 0pt

\title[Optimal approximation with shallow complex-valued neural networks]{Optimal approximation of $C^k$-functions using shallow complex-valued neural networks}



%    Remove any unused author tags.


\author{Paul Geuchen}
\address[P. Geuchen]{Mathematical Institute for Machine Learning and Data Science (MIDS), Catholic University of Eichstätt--Ingolstadt (KU), Auf der Schanz 49, 85049 Ingolstadt, Germany}
\email{paul.geuchen@ku.de}
\thanks{}

\author{Felix Voigtlaender }\thanks{Both authors contributed equally to this work.}
\address[F. Voigtlaender]{Mathematical Institute for Machine Learning and Data Science (MIDS), Catholic University of Eichstätt--Ingolstadt (KU), Auf der Schanz 49, 85049 Ingolstadt, Germany}
\email{felix.voigtlaender@ku.de}




\subjclass[2020]{68T07, 41A25, 41A30, 41A63, 31A30, 30E10}

\keywords{Complex-valued neural networks, Shallow neural networks, Continuously differentiable functions, approximation rates}

\date{\today}

\begin{abstract}    We prove a quantitative result for the approximation of functions of regularity $C^k$ (in the sense of real variables) defined on the complex cube $\Omega_n \defeq [-1,1]^n +i[-1,1]^n\subseteq \CC^n$ using shallow complex-valued neural networks. Precisely, we consider neural networks with a single hidden layer and $m$ neurons, i.e., networks of the form $z \mapsto \sum_{j=1}^m \sigma_j \cdot \phi\big(\rho_j^T z + b_j\big)$ and show that one can approximate every function in  $C^k \left( \Omega_n; \CC\right)$ using a function of that form with error of the order $m^{-k/(2n)}$ as $m \to \infty$, provided that the activation function $\phi: \CC \to \CC$ is smooth but not polyharmonic on some non-empty open set. Furthermore, we show that the selection of the weights $\sigma_j, b_j \in \CC$ and $\rho_j \in \CC^n$ is continuous with respect to $f$ and prove that the derived rate of approximation is optimal under this continuity assumption. We also discuss the optimality of the result for a possibly \emph{discontinuous} choice of the weights.
\end{abstract}



\maketitle


%%%%%%%%%%%%%%%%%%%%%%%%%%%%%%%%%%%%%%%%%%%%%%%%%%%%%%%%
\section{Introduction}
\label{sec:introduction}
% \begin{itemize}
%     % Diffusion of FL
%     \item {\st{Diffusion of FL}}
%     % Security threats to FL
%     \item {\st{Security threats to FL with particular focus on model poisoning}}
%     % Limitations of existing countermeasures
%     \item {\st{Current countermeasures (e.g., KRUM) and their limitations}}
%     % Proposed method and its advantages
%     \item {\st{Intuitive description of the proposed method and its difference (i.e., advantages) w.r.t. state of the art}}
%     % Main contributions
%     \item {\st{Summary of the main contributions of this work}}
%     % Paper's structure and organization
%     \item {\st{Paper's structure and organization}}
% \end{itemize}

% Diffusion of FL
Recently, {\em federated learning} (FL) has emerged as the leading paradigm for training distributed, large-scale, and privacy-preserving machine learning (ML) systems~\cite{mcmahan2017googleai,mcmahan2017aistats}. 
The core idea of FL is to allow multiple edge clients to collaboratively train a shared, global model without disclosing their local private training data.
%Specifically, an FL system consists of a central server and many edge clients; 
A typical FL round involves the following steps: {\em(i)} the server randomly picks some clients and sends them the current, global model; {\em(ii)} each selected client locally trains its model with its own private data; then, it sends the resulting local model to the server;\footnote{Whenever we refer to global/local model, we mean global/local model {\em parameters}.} {\em(iii)} the server updates the global model by computing an \emph{aggregation function}, usually the average (FedAvg), on the local models received from clients.
% \begin{enumerate}
%     \item[{\em(i)}] the server sends the current, global model to the clients and appoints some of them for training;
%     \item[{\em(ii)}] each selected client locally trains its copy of the global model with its own private data; then, it sends the resulting local model back to the server;\footnote{Whenever we refer to global/local model, we mean global/local model {\em parameters}.}
%     \item[{\em(iii)}] the server updates the global model by computing an \emph{aggregation function} on the local models received from clients (by default, the average, also referred to as FedAvg~\cite{mcmahan2017aistats}).
% \end{enumerate}
This process goes on until the global model converges. %(e.g., after a certain number of rounds or other similar stopping criteria).
%\\
% The advantages of FL over the traditional, centralized learning paradigm are undoubtedly clear in terms of flexibility/scalability (clients can join/disconnect from the FL network dynamically), network communications (only model weights\footnote{We will use \textit{parameters} and \textit{weights} interchangeably.} are exchanged between clients and server), and privacy (each client's private training data is kept local at the client's end and not uploaded to the server).
\\
% Security threats to FL
%However, the growing adoption of FL also raises security concerns~\cite{costa2022covert}, particularly about its confidentiality, integrity, and availability.
Although its advantages over standard ML, FL also raises security concerns~\cite{costa2022covert}. %, particularly about its confidentiality, integrity, and availability~\cite{costa2022covert}.
% OLD, LONG VERSION
% Indeed, some work deals with privacy leakage that may expose the local data of some clients~\cite{melis2019sp}. 
% A large body of work, instead, investigates attacks that usually aim to detriment the predictive accuracy of the learned global model. For instance, \emph{data poisoning} attacks achieve this goal by letting an adversary pollute the training set of some corrupt FL clients with maliciously crafted examples~\cite{jagielski2018sp}.
% Similarly, in \emph{model poisoning} the attacker attempts to tweak the global model weights~\cite{bhagoji2019pmlr} by directly perturbing the local model's weights of some infected FL clients before these are sent to the central server for aggregation, usually via so-called Byzantine attacks. 
% It turns out that Byzantine model poisoning attacks severely impact standard FedAvg; therefore, more robust aggregation functions must be designed to make FL systems secure.
Here, we focus on \emph{untargeted model poisoning} attacks~\cite{bhagoji2019pmlr}, where an adversary attempts to tweak the global model weights %\footnote{We will use the terms \textit{parameters} and \textit{weights} interchangeably.} 
by directly perturbing the local model's parameters of some infected clients before these are sent to the central server for aggregation.
In doing so, the adversary aims to jeopardize the global model \textit{indiscriminately} at inference time.
Such model poisoning attacks severely impact standard FedAvg; therefore, more robust aggregation functions must be designed to secure FL systems.
\\
% In this paper, we focus on designing a novel robust aggregation scheme at the server's end to contrast the effect of Byzantine model poisoning attacks.
%
% Current countermeasures and their limitations
%Several countermeasures have been proposed in the literature to combat model poisoning attacks on FL systems.
% Some methods use simple statistics more robust than plain average to smooth the impact of malicious updates (e.g., Trimmed Mean and FedMedian~\cite{yin2018icml}). 
% Other defenses implement outlier detection techniques to discard malicious updates from the aggregation performed at the server's end. Those are either based on heuristics (e.g., Krum/Multi-Krum~\cite{blanchard2017nips} and Bulyan~\cite{mhamdi2018pmlr}) or data-driven approaches (e.g., K-means clustering~\cite{shen2016acm} or DnC via spectral analysis~\cite{shejwalkar2021ndss}). 
% Finally, some strategies rely on a centralized ``source of trust'' to spot potential malicious updates (e.g., FLTrust~\cite{cao2020fltrust}).
% Several countermeasures have been proposed in the literature to combat model poisoning attacks on FL systems, i.e., to discard possible malicious local updates from the aggregation performed at the server's end. 
% These techniques range from simple statistics more robust than plain average (e.g., Trimmed Mean and FedMedian~\cite{yin2018icml}) to outlier detection heuristics (e.g., Krum/Multi-Krum~\cite{blanchard2017nips} and Bulyan~\cite{mhamdi2018pmlr}) or data-driven approaches (e.g., spectral analysis via K-means clustering~\cite{shen2016acm} or spectral analysis), or methods based on ``source of trust'' (e.g., FLTrust~\cite{cao2020fltrust}).
% OLD, LONG VERSION
%Several countermeasures have been proposed in the literature to combat Byzantine model poisoning attacks on FL systems.
% Descriptive statistics
% For example, Trimmed Mean and FedMedian aggregate local model updates using more robust statistics than standard average~\cite{yin2018icml}.
%
% % Heuristics for outlier detection
% Many existing Byzantine-resilient strategies implement some outlier detection heuristics to discard the model updates sent by potentially malicious clients from the input of the aggregation function.
% One of the most popular heuristics is Krum~\cite{blanchard2017nips}.
% This strategy tries to mitigate the impact of Byzantine attacks by selecting as a global model the local model with the smallest sum of Euclidean distances to {\em all} the other local models.
% Although powerful, Krum requires the server to know (or, at least, estimate) the number of malicious FL clients upfront, which is generally impossible in a realistic attack scenario. %
% Moreover, Krum may become ineffective for complex, high-dimensional model parameter spaces due to the curse of dimensionality.
% Bulyan~\cite{mhamdi2018pmlr} tries to overcome this issue by combining Krum with a variant of Trimmed Mean.
% % Data-driven outlier detection
% Other strategies use data-driven outlier detection techniques -- e.g., via K-means clustering~\cite{shen2016acm} -- to spot potential malicious local model updates. 
% %For instance, Shen et al. propose to cluster local model updates with K-means and thus identify outliers.
%
% % Other techniques
% As far as the server is concerned, any local model received can be from a potential malicious client. 
% FLTrust~\cite{cao2020fltrust} assumes the server acts as a client, i.e., trains a local model on an additional {\em trustworthy} dataset at the server's end and compares it against all the local models from other clients. 
% This way, the server can rely on some ``source of trust'' when discarding potentially malicious clients.
%\\
% Limitations of existing Byzantine-resilient strategies
Unfortunately, existing defense mechanisms either rely on simple heuristics (e.g., Trimmed Mean and FedMedian by~\cite{yin2018icml}) or need strong and unrealistic assumptions to work effectively (e.g., foreknowledge or estimation of the number of malicious clients in the FL system, as for Krum/Multi-Krum~\cite{blanchard2017nips} and Bulyan~\cite{mhamdi2018pmlr}, which, however, cannot exceed a fixed threshold).
Furthermore, outlier detection methods using K-means clustering~\cite{shen2016acm} or spectral analysis like DnC~\cite{shejwalkar2021ndss} do not directly consider the temporal evolution of local model updates received.
Finally, strategies like FLTrust~\cite{cao2020fltrust} require the server to collect its own dataset and act as a proper client, thereby altering the standard FL protocol.
\\
% OLD, LONG VERSION
% Overall, existing Byzantine-resilient strategies are either simple heuristics (e.g., FedMedian) or, if they are more complex, they rely on strong and unrealistic assumptions to work effectively (e.g., knowing the number of malicious clients in the FL system in advance, as for Krum and alike).
% Furthermore, data-driven outlier detection methods do not consider the temporary evolution of local model updates received (e.g., K-means clustering). 
% Finally, strategies like FLTrust requires the server to collect its own dataset and act as a proper client, thereby altering the standard FL protocol.
%
% Description of the proposed method
This work introduces a novel pre-aggregation \textit{filter} robust to untargeted model poisoning attacks. Notably, this filter $(i)$ operates without requiring prior knowledge or constraints on the number of malicious clients and $(ii)$ inherently integrates temporal dependencies. 
The FL server can employ this filter as a preprocessing step before applying \textit{any} aggregation function, be it standard like FedAvg or robust like Krum or Bulyan.
Specifically, we formulate the problem of identifying corrupted updates as a multidimensional (i.e., matrix-valued) time series anomaly detection task. 
The key idea is that legitimate local updates, resulting from well-calibrated iterative procedures like stochastic gradient descent (SGD) with an appropriate learning rate, show \textit{higher predictability} compared to malicious updates. This hypothesis stems from the fact that the sequence of gradients (thus, model parameters) observed during legitimate training exhibit regular patterns, as validated in Section~\ref{subsec:intuition}. %until convergence. 
%This regularity may be more pronounced for smooth convex loss functions, but it can still be captured within an appropriate time window, even for more complex and convoluted loss surfaces. 
%We provide evidence of this claim in Appendix~B, where we show that the average mutual information (i.e., ``predictability''), calculated over pairs of legitimate model updates sent at different FL rounds, is significantly higher than the corresponding computation for a malicious client.
\\
Inspired by the matrix autoregressive (MAR) framework for multidimensional time series forecasting~\cite{chen2021je}, we propose the FLANDERS ({\em \textbf{F}ederated \textbf{L}earning meets \textbf{AN}omaly \textbf{DE}tection for a \textbf{R}obust and \textbf{S}ecure}) filter.
The main advantages of FLANDERS over existing strategies like FLDetector~\cite{zhao2020multivariate} are its resilience to large-scale attacks, where $50\%$ or more FL participants are hostile, and the capability of working under realistic non-iid scenarios.
We attribute such a capability to two key factors: $(i)$ FLANDERS works without knowing a priori the ratio of corrupted clients, and $(ii)$ it embodies temporal dependencies between intra- and inter-client updates, quickly recognizing local model drifts caused by evil players. Below, we summarize our main contributions:

\begin{itemize}
\item[{\em(i)}]
We provide empirical evidence that the sequence of models sent by legitimate clients is more predictable than those of malicious participants performing untargeted model poisoning attacks.
\\
\item[{\em(ii)}] 
We introduce FLANDERS, the first pre-aggregation filter for FL robust to untargeted model poisoning based on multidimensional time series anomaly detection.
\\
\item[{\em(iii)}] 
We integrate FLANDERS into Flower,\footnote{\scriptsize{\url{https://flower.dev/}}} a popular FL simulation framework for reproducibility.
\\
\item[{\em(iv)}] 
We show that FLANDERS improves the robustness of the existing aggregation methods under multiple settings: different datasets, client's data distribution (non-iid), models, and attack scenarios.
\\
\item[{\em(v)}] 
We publicly release all the implementation code of FLANDERS along with our experiments.\footnote{\scriptsize{\url{https://anonymous.4open.science/r/flanders_exp-7EEB}}}
\end{itemize}

% Paper's structure and organization
The remainder of the paper is structured as follows. %some related work and the current state-of-the-art solutions to security issues that FL entails. 
Section~\ref{sec:background} covers background and preliminaries. 
In Section~\ref{sec:related}, we discuss related work.
Section~\ref{sec:problem} and Section~\ref{sec:method} describe the problem formulation and the method proposed. % to tackle it. 
Section~\ref{sec:experiments} gathers experimental results. %, and Section~\ref{sec:limitations} discusses some limitations of this work.
Finally, we conclude in Section~\ref{sec:conclusion}.
 %discusses the limitations of this work and draws future research directions.
%reports conclusions and draws perspectives for future research directions.

%%%%%%% OLD %%%%%%%
%to overcome the resilience of Byzantine failures in distributed Stochastic Gradient Descent computations. 
% The strength of Krum is its time complexity, which is linear in the gradient dimension. 
% However, the robustness of the approach is guaranteed for gradient-based learning applications only when the majority of the clients are not compromised. 
% Besides, the aggregation mechanism of Krum, as well as that of similar methods, is robust from a coarse-grained perspective and does not provide solutions to errors and perturbations that may occur at inference time.
%A related approach to~\cite{blanchard2017nips} is the work of Su et al.~\cite{su2016dc}. Here, the authors propose an iterated approximate agreement to tackle a multi-layer scenario attacked by Byzantine agents. 
%However, the method works efficiently on the sole discrete context and it is inapplicable to continuous state environments.
%\gabri{Maybe, we should just talk about the main limitations of existing countermeasures without digging into their details (or, we can just mention Krum as this is the most popular one). I will move the description of all these methods to the Related Work section.}
% !TeX encoding = UTF-8
% !TeX spellcheck = en_US
% !TeX root = main_paper.tex

\section{Admissibility of Activation Functions}
\label{admissibility}

In this section we discuss the notion of admissibility of activation functions $\phi: \CC \to \CC$, i.e., the sufficient conditions that a function has to fulfill in order to be suitable for the approximation results established in Sections \ref{approx_polynomials} and \ref{ck_functions}. We are going to define a slightly weaker notion of admissibility than that the function is smooth and non-polyharmonic on a non-empty open subset of $\CC$.

Let us recall that in \cite{mhaskar_neural_1996} the sufficient condition on an activation function $\phi: \RR \to \RR$  is that it is smooth on an open interval with one point in the interval where no derivative vanishes. In fact, the proofs in \cite{mhaskar_neural_1996} show that it is sufficient to assume that for every $M \in \NN$ there is an open set $\emptyset \neq U \subseteq \RR$ such that $\fres{\phi}{U} \in C^M \left( U ; \RR\right)$ with a point $b \in U$ satisfying
\begin{equation*}
f^{(n)} (b) \neq 0 \text{ for all } n \leq M.
\end{equation*}  
In the case of complex-valued neural networks it turns out that the usual real derivative has to be replaced by mixed Wirtinger derivatives of the form $\wirt^m \wirtq^\ell$. This leads to the following definition.
\begin{definition}
Let $\phi: \CC \to \CC$ and $M \in \NN_0$. We call $\phi$ $M$\emph{-admissible} (in $b \in \CC$) if there is an open set $U \subseteq \CC$ with $b \in U$ and $\fres{\phi}{U} \in C^{2M} (U ; \CC)$ such that
\begin{equation*}
\wirt^m \wirtq^\ell \phi (b) \neq 0 \text{ for all } m,\ell \leq M.
\end{equation*}
\end{definition}
In order to approximate complex polynomials in $z$ and $\overline{z}$ with bounded degree it will be enough to assume $M$-admissibility of an activation function for a certain $M \in \NN$. However, if one wants to approximate arbitrary $C^k$-functions one needs the following definition.
\begin{definition}
A function $\phi : \CC \to \CC$ is called \emph{admissible} if it is $M$-admissible for every $M \in \NN_0$.
\end{definition}
Note that for an admissible function there does not necessarily have to be an open set $\emptyset \neq U \subseteq \CC$ such that the function is smooth on $U$. However, if we assume smoothness we can derive the following elegant result.
\begin{theorem}
\label{charac}
    Let $\phi: \CC \to \CC$, $\emptyset \neq U \subseteq \CC$ an open set and $\fres{\phi}{U} \in C^\infty(U ; \CC)$. Then the following are equivalent:
    \begin{enumerate}
        \item $\fres{\phi}{U}$ is not polyharmonic.
        \item For every $M \in \NN_0$ there exists $z_M \in U$ such that $\phi$ is $M$-admissible at $z_M$. 
    \end{enumerate}
In particular, in both cases $\phi$ is admissible.
\end{theorem}
\begin{proof}
    (1) $\Rightarrow$ (2): 
    Let $M \in \NN_0$. Since $\fres{\phi}{U}$ is not polyharmonic we can pick $z \in U$ with $\Delta^M \phi (z) \neq 0$. By continuity we can choose $\delta > 0$ with $B_\delta(z) \subseteq U$ and $\Delta^M \phi(w) \neq 0$ for all $w \in B_\delta(z)$. For all $m, \ell \in \NN_0$ let
    \begin{equation*}
        A_{m,\ell} \defeq \left\{ w \in B_\delta(z) : \ \wirt^m \wirtq^\ell \phi (w) = 0\right\}
    \end{equation*}
    and assume towards a contradiction that
    \begin{equation*}
        \bigcup_{m,\ell \leq M} A_{m,\ell} = B_\delta(z).
    \end{equation*}
    By \cite[Corollary 3.35]{aliprantis_infinite_2006}, $B_\delta(z)$ with its usual topology is completely metrizable. By continuity of $\wirt^m \wirtq^\ell \phi$ the sets $A_{m,\ell}$ are closed in $B_\delta(z)$. Hence, using the Baire category theorem \cite[Theorems 3.46 and 3.47]{aliprantis_infinite_2006}, there are $m,\ell \in \NN_0$ with $m,\ell \leq M$, $z' \in A_{m,\ell}$ and $\varepsilon > 0$ such that
    \begin{equation*}
        B_\varepsilon\left(z'\right) \subseteq A_{m,\ell} \subseteq B_\delta(z).
    \end{equation*}
    But thanks to $\Delta^M \phi = 4^M \wirt^M \wirtq^M \phi = 4^M \wirt^{M - \ell} \wirtq^{M-m}\wirt^\ell \wirtq^m \phi$ (see (\ref{laplace_ident}) on p. 6), this directly implies $\Delta^M \phi \equiv 0$ on $B_{\varepsilon} \left(z'\right)$ in contradiction to the choice of $B_\delta(z)$. 
\medskip

(2) $\Rightarrow$ (1): If $\fres{\phi}{U}$ is polyharmonic there is $M \in \NN_0$ with
    \begin{equation*}
        \wirt^M \wirtq^M \phi(z) = 0
    \end{equation*}
    for every $z \in U$, which contradicts (2).
\end{proof}
\Cref{charac} justifies the formulation of \Cref{main_1,,main_2}, since it shows that every activation function $\phi$ satisfying the assumptions of the theorem is $M$-admissible for all $M \in \NN_0$.

\begin{remark}
In the case of a real function it follows that if the function is smooth and non-polynomial on an open interval, there has to be a point at which no derivative vanishes, see \cite[p. 53]{donoghue_distributions_1969}. It is an interesting and currently unsolved question if a similar statement also holds true in the case of complex functions and Wirtinger derivatives, i.e., if the following holds: Let $z \in \CC, \ r > 0$ and $\phi \in C^\infty (B_r(z); \CC)$ be non-polyharmonic. Then there exists $b \in B_r(z)$ such that
\begin{equation*}
\wirt^m \wirtq^\ell \phi (b) \neq 0 \text{ for all } m,\ell \in \NN_0.
\end{equation*}
\end{remark}




\section{Postponed proofs concerning the approximation of polynomials}

% !TeX encoding = UTF-8
% !TeX spellcheck = en_US
% !TeX root = main_paper.tex

\section{Divided Differences}
Divided differences are well-known in numerical mathematics as they are for example used to calculate the coefficients of an interpolation polynomial in its Newton representation. In our case, we are interested in divided differences, since they can be used to obtain a generalization of the classical mean-value theorem for differentiable functions: Given an interval $I \subseteq \RR$ and $n+1$ pairwise distinct data points $x_0, ..., x_n \in I$ as well as an $n$-times differentiable real-valued function $f : I \to \RR$, there exists $\xi \in \left( \text{min}\left\{x_0 ,..., x_n\right\}, \text{max}\left\{x_0 ,..., x_n\right\}\right)$, such that
\begin{equation*}
    f\left[x_0, ..., x_n\right] = \frac{f^{(n)}(\xi)}{n!},
\end{equation*}
where the left-hand side is a divided difference of $f$ (defined below). The classical mean-value theorem is obtained in the case $n=1$. Our goal in this section is to generalize this result to a multivariate setting by considering divided differences in multiple variables. Such a generalization is probably well-known, but since we could not locate a convenient reference and to make the paper more self-contained, we provide a proof.

Let us first define divided differences formally. Given $n+1$ data points $\left(x_0, y_0\right), ..., \left(x_n, y_n\right) \in \RR \times \RR$ with pairwise distinct $x_k$, we define the associated divided differences recursively via
\begin{align*}
    \left[ y_k\right] &\defeq y_k, \ k \in \{0,...,n\}, \\
    \left[y_k ,..., y_{k+j}\right] &\defeq \frac{\left[y_{k+1},..., y_{k+j}\right] - \left[y_k,..., y_{k+j-1}\right]}{x_{k+j}-x_k}, \ j \in \{1,...,n\}, \ k \in \{0, ..., n-j\}.
\end{align*}
If the data points are defined by a function $f$ (i.e. $y_k = f\left(x_k\right)$ for all $k \in \{0,...,n\}$), we write
\begin{equation*}
    \left[x_k,...,x_{k+j}\right] f \defeq \left[y_k, ..., y_{k+j}\right].
\end{equation*}
We first consider an alternative representation of divided differences, the so-called \emph{Hermite-Genocchi-Formula}. To state it, we introduce the notation $\Sigma^k$ for a certain $k$-dimensional simplex.
\begin{definition}
    Let $s \in \NN$. Then we define
    \begin{equation*}
        \Sigma^s \defeq \left\{ x \in \RR^s : \ x_\ell \geq 0 \ \mathrm{ for} \ \mathrm{all}  \ \ell \ \mathrm{ and  } \sum_{\ell = 1}^s x_\ell \leq 1 \right\}.
    \end{equation*}
    The identity $\lambda^s\left(\Sigma^s\right)= \frac{1}{s!}$ holds true.
\end{definition}
A proof for the fact that the volume of $\Sigma^s$ is indeed $\frac{1}{s!}$ can be found for example in \cite{stein_note_1966}.
We can now consider the alternative representation of divided differences.
\begin{lemma}[Hermite-Genocchi-Formula]
    For real numbers $a,b \in \RR$, a function $f \in C^k([a,b]; \RR)$ and pairwise distinct $x_0, ..., x_k \in [a,b]$, the divided difference of $f$ at the data points $x_0, ..., x_k$ is given as
\begin{equation}
\label{hg}
    \left[x_0, ..., x_k\right]f = \int_{\Sigma^k} f^{(k)}\left(x_0 + \sum_{\ell=1}^{k}s_\ell\left(x_\ell-x_0\right)\right) ds.
\end{equation}
\end{lemma}
\begin{proof}
    See \cite[Theorem 3.3]{atkinson_introduction_1989}.
\end{proof}
We need the following easy generalization of the mean-value theorem for integrals.
\begin{lemma}
Let $D \subseteq \RR^s$ be a connected and compact set with positive Lesbesgue measure and furthermore $f : D \to \RR$ a continuous function. Then there exists $\xi \in D$ such that
\begin{equation*}
    f(\xi) = \frac{1}{\lambda(D)} \cdot \int_D f(x) dx.
\end{equation*}
\end{lemma}
\begin{proof} 
Since $D$ is compact and $f$ continuous, there exist $x_{\text{min}} \in D$ and $x_{\text{max}} \in D$ satisfying
\begin{equation*}
    f\left(x_{\text{min}}\right) \leq f(x) \leq f\left(x_{\text{max}} \right)
\end{equation*}
for all $x \in D$. Thus, one gets
\begin{equation*}
    f\left( x_{\text{min}}\right) \leq \frac{1}{\lambda(D)} \int_D f(x) dx \leq  f\left( x_{\text{max}}\right)
\end{equation*}
so the claim follows using the fact that $f(D) \subseteq \RR$ is connected, i.e., an interval.
\end{proof}
We also get a convenient representation of divided differences if the data points are equidistant.
\begin{lemma}
Let $f: \RR \to \RR$, $x_0 \in \RR$ and $h > 0$. We consider the case of equidistant data points, meaning $x_{j} \defeq x_0 + jh$ for all $j = 1,...,k$ for a fixed $h > 0$. In this case, we have the formula
\begin{equation}
\label{alternativdarstellung}
    \left[x_0, ..., x_k\right]f = \frac{1}{k!h^k} \cdot \sum_{r=0}^k (-1)^{k-r}\binom{k}{r} f\left(x_r\right). 
\end{equation}
\end{lemma}
\begin{proof}
We prove the result via induction over the number $j$ of considered data points, meaning the following: For all $j \in \{0,...,k\}$ we have
\begin{equation*}
    \left[x_\ell, ..., x_{\ell+j}\right]f = \frac{1}{j!h^j} \cdot \sum_{r=0}^j (-1)^{j-r}\binom{j}{r} f\left(x_{\ell+r}\right)
\end{equation*}
for all $\ell \in \{0, ..., k\}$ such that $\ell + j \leq k$. The case $j = 0$ is trivial. Therefore, we assume the claim to be true for a fixed $j \in \{0,...,k-1\}$ and choose an arbitrary $\ell \in \{0,...,k\}$ such that $\ell+j+1 \leq k$. We then get
\begin{align*}
    \left[x_\ell, ..., x_{\ell+j+1}\right]f &= \frac{\left[x_{\ell+1}, ..., x_{\ell+j+1}\right]f - \left[x_\ell, ..., x_{\ell+j}\right]f}{x_{\ell+j+1}-x_\ell} \\
    &= \frac{1}{j!h^j}\cdot \frac{\sum_{r=0}^j (-1)^{j-r}\binom{j}{r} \left(f\left(x_{\ell+r+1}\right) - f\left(x_{\ell+r}\right)\right)}{(j+1)h} \\
    &= \frac{1}{(j+1)!h^{j+1}}\sum_{r=0}^j (-1)^{j-r}\binom{j}{r} \left(f\left(x_{\ell+r+1}\right) - f\left(x_{\ell+r}\right)\right).
\end{align*}
Using an index shift we deduce
\begin{align*}
    & \norel \sum_{r=0}^j (-1)^{j-r}\binom{j}{r}f\left(x_{\ell+r+1}\right) - \sum_{r=0}^j (-1)^{j-r}\binom{j}{r}f\left(x_{\ell+r}\right) \\
    &= \sum_{r=1}^{j+1} (-1)^{j+1-r}\binom{j}{r-1}f\left(x_{\ell+r}\right) + \sum_{r=0}^j (-1)^{j+1-r}\binom{j}{r}f\left(x_{\ell+r}\right) \\
    &= (-1)^{j+1} f\left(x_\ell\right) + \sum_{r=1}^{j} \left((-1)^{j+1-r}f\left(x_{\ell+r}\right) \left[\binom{j}{r-1} + \binom{j}{r}\right]\right) + f\left(x_{\ell+j+1}\right) \\
    &= \sum_{r=0}^{j+1} (-1)^{j+1-r}\binom{j+1}{r} f\left(x_{\ell+r}\right)
\end{align*}
which yields the claim.
\end{proof}
The final result for divided differences is stated as follows:
\begin{theorem}
\label{div_differences_mainresult}
Let $f: \RR^s \to \RR$ and $k \in \NN_0, r>0$, such that $\fres{f}{(-r,r)^s} \in C^k \left((-r,r)^s; \RR\right)$. For $\textbf{p} \in \NN_0^s$ with $\vert \pp \vert \leq k$ and $h>0$ let
\begin{equation*}
    f_{\pp,h} \defeq (2h)^{-\vert \pp \vert} \sum_{0 \leq \textbf{r} \leq \textbf{p}} (-1)^{\vert \pp \vert -\vert \rr \vert} \binom{\textbf{p}}{\textbf{r}} f \left( h(2\rr-\pp)\right).
\end{equation*}
Let $m \defeq \underset{j}{\mathrm{max}} \  \pp_j$. Then, for $0 <h < \frac{r}{\max\{1,m\}}$ there exists $\xi \in h[-m,m]^s$ satisfying
\begin{equation*}
    f_{\pp,h} = \partial^\pp f(\xi).
\end{equation*}
\end{theorem}
\begin{proof}
We may assume $m > 0$, since $\pp=0$ implies $f_{\pp,h} = f(0)$ and hence, the claim follows with $\xi = 0$.

We prove via induction over $s \in \NN$ that the formula
\begin{equation}
\label{to_prove}
    f_{\pp,h}= \pp! \int_{\Sigma^{\pp_s}}\int_{\Sigma^{\pp_{s-1}}} \cdot\cdot\cdot \int_{\Sigma^{\pp_1}} \partial^\pp f \left( -h\pp_1 + 2h\sum_{\ell=1}^{\pp_1}\ell\sigma_\ell^{(1)}, ..., -h\pp_s + 2h\sum_{i=1}^{\pp_s}\ell\sigma_\ell^{(s)}\right)d\sigma^{(1)} \cdot \cdot \cdot d\sigma^{(s)}
\end{equation}
holds for all $\pp \in \NN_0^s$ with $1 \leq \vert \pp \vert \leq k$ and all $0 < h < \frac{r}{m}$. The case $s=1$ is exactly the Hermite-Genocchi-Formula (\ref{hg}) combined with (\ref{alternativdarstellung}) applied to the data points $-hp, -hp + 2h, ..., hp-2h, hp$. 

By induction, assume that the claim holds for some $s \in \NN$.
For a fixed point $y \in (-r,r)$, let 
\begin{equation*}
    f_y: \quad (-r,r)^s \to \RR, \quad x \mapsto f(x,y).
\end{equation*}
For $\pp \in \NN_0^{s+1}$ with $\vert \pp \vert \leq k$ and $\pp' := \left(p_1,...,p_s\right)$ we define
\begin{equation*}
    \Gamma: \quad (-r,r) \to \RR, \quad y \mapsto \left( f_y\right)_{\pp',h} = (2h)^{- \vert \pp' \vert} \sum_{0 \leq \rr' \leq \pp'} (-1)^{\vert  \pp' \vert - \vert \rr' \vert} \binom{\pp'}{\rr'} f\left( h(2\rr' - \pp'),y\right).
\end{equation*}
Using the induction hypothesis, we get
\begin{align*}
\label{IV}
    &\Gamma(y) \\
    =&\pp'! \int\limits_{\Sigma^{\pp_s}}\int\limits_{\Sigma^{\pp_{s-1}}} \cdot\cdot\cdot \int\limits_{\Sigma^{\pp_1}} \partial^{\left(\pp',0 \right)} f \left( -h\pp_1 + 2h\sum_{i=1}^{\pp_1}i\sigma_i^{(1)}, ..., -h\pp_s + 2h\sum_{i=1}^{\pp_s}i\sigma_i^{(s)},y\right)d\sigma^{(1)} \cdot \cdot \cdot d\sigma^{(s)}
\end{align*}
for all $y \in (-r,r)$. Furthermore, we calculate
\begin{align*}
    &\norel \pp_{s+1}! \cdot [-h \cdot \pp_{s+1}, -h \cdot \pp_{s+1} + 2h, ..., h \cdot \pp_{s+1}]\Gamma  \\
    \overset{\eqref{alternativdarstellung}}&{=} (2h) ^{- \pp_{s+1}} \sum_{r' = 0}^{\pp_{s+1}} (-1)^{\pp_{s+1}-r'}\binom{\pp_{s+1}}{r'} \Gamma\left(h\left(2r'-\pp_{s+1}\right)\right) \\
    &= (2h) ^{- \pp_{s+1}} \sum_{r' = 0 }^{\pp_{s+1}} (-1)^{\pp_{s+1}-r'}\binom{\pp_{s+1}}{r'} (2h)^{- \vert \pp' \vert} \sum_{0 \leq \rr' \leq \pp'}(-1)^{\vert \pp' \vert -\vert \rr' \vert} \binom{\pp'}{\rr'} f\left( h(2\rr' - \pp'), h(2r' - \pp_{s+1})\right) \\
    &= (2h)^{-\vert \pp \vert} \sum_{0 \leq \textbf{r} \leq \textbf{p}} (-1)^{\vert \pp \vert -\vert \rr \vert} \binom{\textbf{p}}{\textbf{r}} f \left( h(2\rr-\pp)\right) \\
    &= f_{\pp, h}.
\end{align*}
On the other hand, we get
\begin{align*}
    &\norel [-h \cdot \pp_{s+1}, -h \cdot \pp_{s+1} + 2h, ..., h \cdot \pp_{s+1}]\Gamma \\
    \overset{\eqref{hg}}&{=} \int_{\Sigma^{\pp_{s+1}}}\Gamma^{(\pp_{s+1})}\left(-h\pp_{s+1} + 2h\sum_{\ell=1}^{\pp_{s+1}}\ell t_\ell\right)dt \\
    \overset{}&{=}\pp'! \int\limits_{\Sigma^{\pp_{s+1}}} \cdot\cdot\cdot \int\limits_{\Sigma^{\pp_1}} \partial^{\pp} f \left( -h\pp_1 + 2h\sum_{\ell=1}^{\pp_1}\ell\sigma_\ell^{(1)}, ..., -h\pp_{s+1}+2h\sum_{\ell=1}^{\pp_{s+1}}\ell\sigma^{(s+1)}_\ell\right)d\sigma^{(1)} \cdot \cdot \cdot d\sigma^{(s+1)}.
\end{align*}
Changing the order of integration and derivative is possible, since we integrate on compact sets and only consider continuously differentiable functions.

We have thus proven (\ref{to_prove}) using the principle of induction. The claim of the theorem then follows directly using the mean-value theorem for integrals and by the fact that all the simplices $\Sigma^{\pp_\ell}$ are compact and connected (in fact convex).
\end{proof}

% !TeX encoding = UTF-8
% !TeX spellcheck = en_US
% !TeX root = main_paper.tex

\section{Admissibility of Activation Functions}
\label{admissibility}

In this section we discuss the notion of admissibility of activation functions $\phi: \CC \to \CC$, i.e., the sufficient conditions that a function has to fulfill in order to be suitable for the approximation results established in Sections \ref{approx_polynomials} and \ref{ck_functions}. We are going to define a slightly weaker notion of admissibility than that the function is smooth and non-polyharmonic on a non-empty open subset of $\CC$.

Let us recall that in \cite{mhaskar_neural_1996} the sufficient condition on an activation function $\phi: \RR \to \RR$  is that it is smooth on an open interval with one point in the interval where no derivative vanishes. In fact, the proofs in \cite{mhaskar_neural_1996} show that it is sufficient to assume that for every $M \in \NN$ there is an open set $\emptyset \neq U \subseteq \RR$ such that $\fres{\phi}{U} \in C^M \left( U ; \RR\right)$ with a point $b \in U$ satisfying
\begin{equation*}
f^{(n)} (b) \neq 0 \text{ for all } n \leq M.
\end{equation*}  
In the case of complex-valued neural networks it turns out that the usual real derivative has to be replaced by mixed Wirtinger derivatives of the form $\wirt^m \wirtq^\ell$. This leads to the following definition.
\begin{definition}
Let $\phi: \CC \to \CC$ and $M \in \NN_0$. We call $\phi$ $M$\emph{-admissible} (in $b \in \CC$) if there is an open set $U \subseteq \CC$ with $b \in U$ and $\fres{\phi}{U} \in C^{2M} (U ; \CC)$ such that
\begin{equation*}
\wirt^m \wirtq^\ell \phi (b) \neq 0 \text{ for all } m,\ell \leq M.
\end{equation*}
\end{definition}
In order to approximate complex polynomials in $z$ and $\overline{z}$ with bounded degree it will be enough to assume $M$-admissibility of an activation function for a certain $M \in \NN$. However, if one wants to approximate arbitrary $C^k$-functions one needs the following definition.
\begin{definition}
A function $\phi : \CC \to \CC$ is called \emph{admissible} if it is $M$-admissible for every $M \in \NN_0$.
\end{definition}
Note that for an admissible function there does not necessarily have to be an open set $\emptyset \neq U \subseteq \CC$ such that the function is smooth on $U$. However, if we assume smoothness we can derive the following elegant result.
\begin{theorem}
\label{charac}
    Let $\phi: \CC \to \CC$, $\emptyset \neq U \subseteq \CC$ an open set and $\fres{\phi}{U} \in C^\infty(U ; \CC)$. Then the following are equivalent:
    \begin{enumerate}
        \item $\fres{\phi}{U}$ is not polyharmonic.
        \item For every $M \in \NN_0$ there exists $z_M \in U$ such that $\phi$ is $M$-admissible at $z_M$. 
    \end{enumerate}
In particular, in both cases $\phi$ is admissible.
\end{theorem}
\begin{proof}
    (1) $\Rightarrow$ (2): 
    Let $M \in \NN_0$. Since $\fres{\phi}{U}$ is not polyharmonic we can pick $z \in U$ with $\Delta^M \phi (z) \neq 0$. By continuity we can choose $\delta > 0$ with $B_\delta(z) \subseteq U$ and $\Delta^M \phi(w) \neq 0$ for all $w \in B_\delta(z)$. For all $m, \ell \in \NN_0$ let
    \begin{equation*}
        A_{m,\ell} \defeq \left\{ w \in B_\delta(z) : \ \wirt^m \wirtq^\ell \phi (w) = 0\right\}
    \end{equation*}
    and assume towards a contradiction that
    \begin{equation*}
        \bigcup_{m,\ell \leq M} A_{m,\ell} = B_\delta(z).
    \end{equation*}
    By \cite[Corollary 3.35]{aliprantis_infinite_2006}, $B_\delta(z)$ with its usual topology is completely metrizable. By continuity of $\wirt^m \wirtq^\ell \phi$ the sets $A_{m,\ell}$ are closed in $B_\delta(z)$. Hence, using the Baire category theorem \cite[Theorems 3.46 and 3.47]{aliprantis_infinite_2006}, there are $m,\ell \in \NN_0$ with $m,\ell \leq M$, $z' \in A_{m,\ell}$ and $\varepsilon > 0$ such that
    \begin{equation*}
        B_\varepsilon\left(z'\right) \subseteq A_{m,\ell} \subseteq B_\delta(z).
    \end{equation*}
    But thanks to $\Delta^M \phi = 4^M \wirt^M \wirtq^M \phi = 4^M \wirt^{M - \ell} \wirtq^{M-m}\wirt^\ell \wirtq^m \phi$ (see (\ref{laplace_ident}) on p. 6), this directly implies $\Delta^M \phi \equiv 0$ on $B_{\varepsilon} \left(z'\right)$ in contradiction to the choice of $B_\delta(z)$. 
\medskip

(2) $\Rightarrow$ (1): If $\fres{\phi}{U}$ is polyharmonic there is $M \in \NN_0$ with
    \begin{equation*}
        \wirt^M \wirtq^M \phi(z) = 0
    \end{equation*}
    for every $z \in U$, which contradicts (2).
\end{proof}
\Cref{charac} justifies the formulation of \Cref{main_1,,main_2}, since it shows that every activation function $\phi$ satisfying the assumptions of the theorem is $M$-admissible for all $M \in \NN_0$.

\begin{remark}
In the case of a real function it follows that if the function is smooth and non-polynomial on an open interval, there has to be a point at which no derivative vanishes, see \cite[p. 53]{donoghue_distributions_1969}. It is an interesting and currently unsolved question if a similar statement also holds true in the case of complex functions and Wirtinger derivatives, i.e., if the following holds: Let $z \in \CC, \ r > 0$ and $\phi \in C^\infty (B_r(z); \CC)$ be non-polyharmonic. Then there exists $b \in B_r(z)$ such that
\begin{equation*}
\wirt^m \wirtq^\ell \phi (b) \neq 0 \text{ for all } m,\ell \in \NN_0.
\end{equation*}
\end{remark}





\subsection{Proof of Theorem \ref{main_1}}
\label{approx_polynomials_reordered}

The following section is dedicated to proving \Cref{main_1}.
We are going to show that it is possible to approximate complex polynomials in $z$ and $\overline{z}$
arbitrarily well on $\Omega_n$ using shallow complex-valued neural networks. 
To do so, we follow the proof sketch given after the statement of \Cref{main_1},
starting with the following lemma.


\begin{lemma}
\label{extraction}
    Let $\phi:\CC \to \CC$ and $\delta>0, \ b \in \CC, \ k \in \NN_0$, such that $\fres{\phi}{B_\delta(b)} \in C^k      \left(B_\delta(b); \CC\right)$.
    For fixed $z \in \Omega_n$, where we recall that $\Omega_n = [-1,1]^n + i [-1,1]^n$, we consider the map
    \begin{equation*}
        \phi_z: \quad B_{\frac{\delta}{\sqrt{2n}}}(0) \to \CC, \quad w \mapsto \phi\left(w^T z + b\right),
    \end{equation*}
    which is in $C^k$ since for $w \in B_{\frac{\delta}{\sqrt{2n}}}(0) \subseteq \CC^n$ we have
    \begin{equation*}
        \left\vert w^T z\right\vert \leq \Vert w \Vert_2 \cdot \Vert z \Vert_2 < \frac{\delta}{\sqrt{2n}} \cdot \sqrt{2n} = \delta.
    \end{equation*}
    For all multi-indices $\m, \elll \in \NN_0^n$ with $\vert \m + \elll \vert \leq k$ we have
    \begin{equation*}
        \wirt^\m \wirtq^{\elll} \phi_z(w) = z^\m \overline{z}^{\elll} \cdot \left(\wirt^{\vert \m \vert}\wirtq^{\vert \elll \vert}\phi\right) \left(w^T z + b \right)
    \end{equation*}
    for all $w \in B_{\frac{\delta}{\sqrt{2n}}}(0)$.
\end{lemma}

\begin{proof}
    First we prove the statement
    \begin{equation}
    \label{conj}
        \wirtq^{\elll} \phi_z(w) = \overline{z}^{\elll} \cdot (\wirtq^{\vert {\elll} \vert}\phi)\left( w^Tz +b\right) \quad \text{for all } w \in B_{\frac{\delta}{\sqrt{2n}}}(0)
    \end{equation}
    by induction over $0 \leq \vert {\elll} \vert \leq k$. The case ${\elll} = 0$ is trivial. Assuming that (\ref{conj}) holds for fixed ${\elll} \in \NN_0^n$ with $\vert \elll\vert < k $, we want to show
    \begin{equation}
    \label{conj_wirt}
        \wirtq^{{\elll} + e_j} \phi_z(w) = \overline{z}^{{\elll}+e_j} \cdot \left( \wirtq^{|{\elll}| + 1}\phi\right)\left(w^Tz +b\right)
    \end{equation}
    for all $w \in B_{\frac{\delta}{\sqrt{2n}}}(0)$, where $j \in \{1,...,n\}$ is chosen arbitrarily. To this end, first note
    \begin{align*}
        \wirtq^{{\elll} + e_j} \phi_z(w) &= \wirtq^{e_j}\wirtq^{\elll} \phi_z(w)  \overset{\text{induction}}{=} \wirtq^{e_j}\left[w \mapsto \overline{z}^{\elll} \cdot \left(\wirtq^{\vert {\elll} \vert}\phi\right)\left(w^Tz + b \right)\right] \\
        &= \overline{z}^{\elll} \wirtq^{e_j}\left[w \mapsto \left(\wirtq^{\vert {\elll} \vert}\phi\right)\left(w^Tz + b\right)\right].
    \end{align*}
    Applying the chain rule for Wirtinger derivatives and using that
    \begin{equation*}
        \wirtq^{e_j} \left[ w \mapsto w^T z +b\right]= 0
    \end{equation*} since $w \mapsto w^T z  + b$ is holomorphic in every variable, we see
    \begin{align*}
        \wirtq^{e_j}\left[w \mapsto \left(\wirtq^{\vert {\elll} \vert}\phi\right)\left(w^Tz + b\right)\right] &= \left(\wirt\wirtq^{\vert {\elll}\vert}\phi\right) \left(w^Tz +b\right) \cdot \wirtq^{e_j}\left[w \mapsto w^Tz + b\right] \\
        & \hspace{0.4cm}+ \left(\wirtq^{\vert {\elll}\vert + 1}\phi\right) \left(w^Tz +b\right) \cdot \wirtq^{e_j}\left[w \mapsto \overline{w^Tz + b}\right] \\
        &= \left(\wirtq^{\vert {\elll}\vert + 1}\phi\right) \left( w^Tz +b\right) \cdot \overline{\wirt^{e_j}\left[w \mapsto w^Tz + b\right]} \\
        &= \overline{z}^{e_j} \cdot \left(\wirtq^{\vert {\elll}\vert + 1}\phi\right) \left(w^Tz +b\right),
    \end{align*}
    using the fact that $w_j \mapsto w^Tz + b$ is holomorphic and hence
    \begin{equation*}
        \wirtq^{e_j}\left[ w \mapsto w^T z + b\right] = 0 \quad \text{and} \quad \wirt^{e_j}\left[ w \mapsto w^T z + b\right] = z_j.
    \end{equation*}
    Thus, we have proven (\ref{conj_wirt}) and induction yields (\ref{conj}). 

    It remains to show the full claim. We use induction over $\vert \m \vert$ and note that the case $\m=0$ has just been shown. We assume that the claim holds true for fixed $\m \in \NN_0^n$ with $\vert \m + \elll \vert < k$ and choose $j \in \{1,...,n\}$. Thus, we get
    \begin{align*}
        \wirt^{\m + e_j} \wirtq^{{\elll}}\phi_z(w) &= \wirt^{e_j}\wirt^\m \wirtq^{\elll} \phi_z(w) \overset{\text{IH}}{=} \wirt^{e_j}\left( w \mapsto z^\m \overline{z}^{\elll} \cdot\left(\wirt^{\vert \m \vert} \wirtq^{\vert {\elll} \vert} \phi \right)\left(w^T z + b\right)\right) \\
        &= z^\m \overline{z}^{\elll} \cdot \wirt^{e_j}\left[ w \mapsto \left(\wirt^{\vert \m \vert} \wirtq^{\vert {\elll} \vert} \phi \right)\left(w^T z + b\right)\right].
    \end{align*}
    Using the chain rule again, we calculate
    \begin{align*}
        &\norel \wirt^{e_j}\left[w \mapsto \left(\wirt^{\vert \m \vert} \wirtq^{\vert {\elll} \vert} \phi \right)\left(w^T z + b\right)\right] \\
        &= \left(\wirt^{\vert \m \vert +1}\wirtq^{\vert {\elll} \vert}\phi\right)\left(w^Tz + b\right) \cdot \wirt^{e_j}\left[ w \mapsto w^Tz + b\right] \\
        &\norel + \left(\wirt^{\vert \m \vert }\wirtq^{\vert {\elll} \vert + 1}\phi\right)\left(w^Tz + b\right) \cdot \wirt^{e_j}\left[w \mapsto \overline{w^Tz + b}\right] \\
        &= z^{e_j}\cdot \left(\wirt^{\vert \m \vert +1}\wirtq^{\vert {\elll} \vert}\phi\right)\left(w^Tz + b\right) + \left(\wirt^{\vert \m \vert }\wirtq^{\vert {\elll} \vert + 1}\phi\right)\left(w^Tz + b\right) \cdot \overline{\wirtq^{e_j}\left[w \mapsto w^Tz + b\right]} \\
        &=z^{e_j}\cdot \left(\wirt^{\vert \m \vert +1}\wirtq^{\vert {\elll} \vert}\phi\right)\left( w^Tz + b\right).
    \end{align*}
    By induction, this proves the claim.
\end{proof}

As the last preparation for the proof of \Cref{main_1}, we need the following lemma.

\begin{lemma}
\label{previous}
    Let $\phi:\CC \to \CC$ and $\delta>0, \ b \in \CC, \ k \in \NN_0$, such that $\fres{\phi}{B_\delta(b)} \in C^k      \left(B_\delta(b); \CC\right)$. Let $m,n \in \NN$ and $\varepsilon>0$. For $\pp \in \NN_0^{2n}, h>0$ and $z \in \Omega_n$ we write
    \begin{align*}
        \Phi_{\pp,h} (z)  &\defeq (2h)^{-\vert \pp \vert} \sum_{0 \leq \textbf{r} \leq \textbf{p}} (-1)^{\vert \pp \vert -\vert \rr \vert} \binom{\textbf{p}}{\textbf{r}} \left(\phi_z \circ \varphi_n\right)\left(h(2\rr - \pp)\right) \\ 
        &=(2h)^{-\vert \pp \vert} \sum_{0 \leq \textbf{r} \leq \textbf{p}} (-1)^{\vert \pp \vert -\vert \rr \vert} \binom{\textbf{p}}{\textbf{r}} \phi \left( \left[\varphi_n\left(h(2\rr-\pp)\right)\right]^T \cdot z + b\right),
    \end{align*}
where $\phi_z$ is as introduced in \Cref{extraction} and $\varphi_n$ is as in \eqref{isomorphism_intro}.
    Furthermore, let
    \begin{equation*}
        \phi_\pp : \quad \Omega_n \times B_{\frac{\delta}{\sqrt{2n}}}(0)\to \CC, \quad (z,w) \mapsto \partial^\pp \phi_z(w).
    \end{equation*}
    Then there exists $h^* > 0$ such that
    \begin{equation*}
        \Vert \Phi_{\pp,h} - \phi_\pp(\cdot, 0) \Vert_{L^\infty\left(\Omega_n; \CC\right)} \leq \varepsilon
    \end{equation*}
    for all $\pp \in \NN_0^{2n}$ with $\vert \pp \vert \leq k$ and $ \pp \leq m$ and $h \in (0, h^*)$.
\end{lemma}

\begin{proof}
    Fix $\pp \in \NN_0^{2n}$ with $\vert \pp \vert \leq k$ and $  \pp \leq m$. The map
    \begin{equation*}
        B_{\sqrt{2n} + 1}(0) \times B_{\delta / (\sqrt{2n} + 1)} (0) \to \CC, \quad (z,w) \mapsto \phi \left(w^T z + b\right)
    \end{equation*}
    is in $C^k$ since
    \begin{equation*}
        \left\vert w^T z \right\vert \leq \Vert w \Vert \cdot \Vert z \Vert < \frac{\delta}{\sqrt{2n} + 1} \cdot (\sqrt{2n} + 1) = \delta.
    \end{equation*}
    Therefore, the map 
    \begin{equation*}
        B_{\sqrt{2n} + 1}(0) \times B_{\delta / (\sqrt{2n} + 1)} (0) \to \CC, \quad (z,w) \mapsto \partial^\pp \phi_z(w)
    \end{equation*}
    is continuous and in particular uniformly continuous on the compact set
    \begin{equation*}
        \Omega_n \times \overline{B_{\delta/(3n)}}(0) \subseteq B_{\sqrt{2n} + 1}(0) \times B_{\delta / (\sqrt{2n} + 1)}(0).
    \end{equation*}
    Here, we employed $\sqrt{2n}+1 < 3n$ for every $n \in \NN$. Hence, there exists $ h_\pp \in (0,\frac{\delta}{3n \cdot \sqrt{2n} \cdot m})$, such that
    \begin{equation*}
        \left\vert \phi_\pp(z, \xi) - \phi_\pp (z,0)\right\vert \leq \frac{\varepsilon}{\sqrt{2}}
    \end{equation*}
    for all $\xi \in \varphi_{n} \left(h \cdot [-m,m]^{2n}\right), \ h \in (0, h_\pp)$ and $z \in \Omega_n$.
    Now fix such an $h \in (0, h_\pp)$ and $z \in \Omega_n$.
    Applying \Cref{div_differences_mainresult} to both components of
    $\left(\varphi_1^{-1} \circ \Phi_{\pp, h}\right)(z)$
    and $\varphi_1^{-1} \circ \phi_z \circ \fres{\varphi_n}{\left(-\frac{\delta}{3n},\frac{\delta}{3n}\right)^{2n}}$
    separately yields the existence of two real vectors $\xi_{1}, \ \xi_{2} \in h \cdot [-m,m]^{2n}$, such that
    \begin{align*}
        \left(\varphi_1^{-1} \circ \Phi_{\pp,h}(z)\right)_1
        & = \left[\partial^\pp \left(\varphi_1^{-1} \circ \phi_z \circ \varphi_n\right)\left(\xi_1\right)\right]_1 \\
        \text{and} \quad
        \left(\varphi_1^{-1} \circ \Phi_{\pp,h}(z)\right)_2
        & = \left[\partial^\pp \left(\varphi_1^{-1} \circ \phi_z \circ \varphi_n\right)\left(\xi_2\right)\right]_2.
    \end{align*}
    Rewriting this yields
    \begin{equation*}
        \RE\left(\Phi_{\pp,h}(z)\right) = \RE\left(\phi_\pp (z, \varphi_n\left(\xi_1\right))\right) \quad \text{and}\quad  \IM\left(\Phi_{\pp,h}(z)\right) = \IM\left(\phi_\pp (z, \varphi_n\left(\xi_2\right))\right).
    \end{equation*}
    Using this property, we deduce
    \begin{equation*}
        \left\vert \text{Re}\left( \Phi_{\pp,h}(z) - \phi_\pp(z,0)\right)\right\vert = \left\vert \text{Re}\left( \phi_\pp(z, \varphi_n\left(\xi_{1}\right)) - \phi_\pp(z,0)\right)\right\vert \leq \left\vert \phi_\pp(z, \varphi_n\left(\xi_{1}\right)) - \phi_\pp (z,0)\right\vert \leq \frac{\varepsilon}{\sqrt{2}}
    \end{equation*}
    and analogously for the imaginary part. Since $z \in \Omega_n$ and $h \in \left(0, h_\pp \right)$ have been chosen arbitrarily we get the claim by choosing
    \begin{equation*}
        h^* \defeq \text{min} \left\{ h_\pp  : \  \pp \in \NN_0^{2n} \text{ with } \vert \pp \vert \leq k \text{ and }\pp \leq m\right\}. \qedhere
    \end{equation*}
\end{proof}

Using the previous two lemmas and the results from \Cref{sec:div_diff_reordered}
and \Cref{admissibility_reordered}, we can now prove \Cref{main_1}.

\begin{proof}[Proof of \Cref{main_1}]
Let $b \in U$ satisfy 
\begin{equation*}
\wirt ^{\ell_1} \wirtq^{\ell_2} \phi(b) \neq 0 \quad \text{for all } \ell_1, \ell_2 \in \NN_0 \text{ with } \ell_1, \ell_2\leq mn.
\end{equation*}
Such a point $b$ exists according to \Cref{prop:nonpoly}.
Let $p \in \PP'$ and fix $\m, \elll \in \NN_0^{n}$ with $ \m, \elll \leq m$. For each $z \in \Omega_n$, using \Cref{extraction}, we then have
\begin{align}
    z^\m \overline{z}^{\elll} &= \left[\left( \wirt^{\vert\m\vert} \wirtq^{\vert\elll\vert} \phi\right)(b)\right]^{-1}\wirt^\m \wirtq^{\elll} \phi_z (0) \nonumber\\
    \label{eq:monomial_equal}
    \overset{\text{Prop. }\mathrm{\ref{wirtreal}}}&{=} \left[\left( \wirt^{\vert\m\vert} \wirtq^{\vert\elll\vert} \phi\right)(b)\right]^{-1}  \cdot \underset{\pp' + \pp'' = \m + \elll}{\underset{ \pp = (\pp', \pp'') \in \NN_0^{2n}}{\sum}} b_{\pp', \pp''}\partial^{(\pp', \pp'')} \phi_z(0)
\end{align}
with suitably chosen complex coefficients $b_{\pp', \pp''} \in \CC$ depending only on $\pp', \ \pp'', \ \m$ and $\elll$. Here we used that $\vert \m \vert, \ \vert \elll \vert \leq mn$.
Since $\mathcal{P}' \subseteq \mathcal{P}_m^n$ is bounded and $p \in \PP'$, we can write
\begin{equation*}
p(z) = \underset{\m, \elll \leq m}{\sum_{\m, \elll \in \NN_0^n}} a_{\m, \elll} z^\m \overline{z}^{\elll}
\end{equation*}
with $\vert a_{\m, \elll}\vert \leq c$ for some constant $c = c(\PP') > 0$. In combination with \eqref{eq:monomial_equal}, this easily implies that we can rewrite $p$ as
\begin{equation} \label{eq:p_form}
    p(z) = \underset{\pp  \leq 2m}{\sum_{\pp \in \NN_0^{2n}}} c_\pp(p) \partial^\pp \phi_z(0)
\end{equation}
with coefficients $c_\pp(p) \in \CC$ satisfying $\vert c_\pp (p)\vert \leq c'$ for some constant $c' = c'(\phi, b, \PP', m, n)$. By \Cref{previous}, we choose $h^*>0$, such that
\begin{equation*}
    \left\vert \Phi_{\pp,h^*}(z) - \partial^\pp\phi_z(0)\right\vert \leq \frac{\varepsilon}{\sum_{ \qq \in \NN_0^{2n},\qq \leq 2m}c'}
\end{equation*}
for all $z \in \Omega_n$ and $\pp \in \NN_0^{2n}$ with $ \pp \leq 2m$. Furthermore, we can rewrite each function $\Phi_{\pp, h^*}$ as
\begin{equation*}
    \Phi_{\pp, h^*}(z) = \sum_{\underset{\vert \aalpha_j \vert \leq 2m  \ \forall j}{\aalpha \in \Z^{2n}}} \lambda_{\aalpha, \pp} \phi(\left[\varphi_n\left(h^* \aalpha\right)\right]^T \cdot z + b)
\end{equation*}
with suitable coefficients $\lambda_{\aalpha, \pp} \in \CC$. Since the cardinality of the set 
\begin{equation*}
    \left\{ \aalpha \in \Z^{2n} : \ \left\vert \aalpha_j \right\vert \leq 2m \ \forall j\right\}
\end{equation*}
is $(4m+1)^{2n}$, this can be converted to 
\begin{equation*}
   \Phi_{\pp, h^*}(z) =  \sum_{j=1}^N \lambda_{j,\pp} \phi \left( \rho_j^T \cdot z + b\right).
\end{equation*}
For $p$ as in \eqref{eq:p_form}, we then define
\begin{equation*}
    \theta(z) \defeq \underset{\pp \leq 2m}{\sum_{\pp \in \NN_0^{2n}}} c_\pp(p) \cdot \Phi_{\pp, h^*}(z) = \sum_{j=1}^{N} \left[ \left(\underset{\pp \leq 2m}{\sum_{\pp \in \NN_0^{2n}}}c_{\pp}(p) \lambda_{j, \pp} \right)\phi(\rho_j^T \cdot z + b)\right]
\end{equation*}
and note
\begin{equation*}
    \left\vert \theta(z) - p(z)\right\vert \leq \sum_{\pp \leq 2m} \left\vert c_\pp(p) \right\vert \cdot \left\vert \Phi_{\pp, h^*}(z) - \partial^\pp \phi_z(0)\right\vert \leq \varepsilon.
\end{equation*}
Since the coefficients $\rho_j$ have been chosen independently of the polynomial $p$, we can rewrite $\theta$ in the desired form.
\end{proof}


\section{Postponed proofs for the approximation of \texorpdfstring{$C^k$}{Cᵏ}-functions}


\subsection{Prerequisites from Fourier Analysis} \label{sec:fourier_reordered}
This section is dedicated to reviewing some notations and results from Fourier Analysis. In the end, a quantitative result for the approximation of $C^k \left( [-1,1]^s; \RR\right)$-functions using linear combinations of multivariate Chebyshev polynomials is derived; see \Cref{app: fourier_approx}.

We start by recalling several notations and concepts from Fourier Analysis. 
\begin{definition}
    Let $s \in \NN$ and $k \in \NN_0$. We define
    \begin{equation*}
        C_{2\pi}^k\left(\RR^s; \CC\right) \defeq \left\{ f \in C^k \left(\RR^s; \CC\right) : \ \forall \pp \in \ZZ^s \ \forall x \in \RR^s : \ f(x + 2\pi \pp ) = f(x)\right\}.
    \end{equation*}
    and $C_{2\pi}\left(\RR^s; \CC\right) \defeq C_{2\pi}^0\left(\RR^s; \CC\right)$. For a function $f \in C_{2\pi}^k \left(\RR^s; \CC\right)$ we write 
\begin{align*}
 \left\Vert f \right\Vert_{C^k \left([-\pi, \pi]^s ; \CC\right)} &\defeq \underset{\vert \kk \vert \leq k}{\underset{\kk \in \NN_0^s}{\max}} \left\Vert \partial^\kk f\right\Vert_{L^\infty \left([-\pi, \pi]^s; \CC\right)} \text{ and } \\
\left\Vert f \right\Vert_{L^p \left([-\pi, \pi]^s ; \CC \right)} &\defeq \left(\frac{1}{(2\pi)^s} \cdot \int_{[-\pi, \pi]^s} \left\vert f(x) \right\vert^p dx \right)^{1/p} \text{ for } p \in [1, \infty).
\end{align*}
Moreover, we set $\Vert f \Vert_{L^\infty([-\pi, \pi]^s ; \RR)} \defeq \left\Vert f \right\Vert_{C^0 \left([-\pi, \pi]^s ; \CC\right)}$.
\end{definition}

\begin{definition}
    For any $s \in \NN$ and $\kk \in \Z^s$, we write
    \begin{equation*}
        e_\kk : \quad \RR^s \to \CC, \quad e_\kk (x) = e^{i \langle \kk, x\rangle}
    \end{equation*}
    where $\langle \cdot, \cdot \rangle$ denotes the usual inner product of two vectors. 
    For any $f \in C_{2\pi} \left(\RR^s; \CC\right)$ we define the $\kk$-th Fourier coefficient of $f$ to be
    \begin{equation*}
        \hat{f}(\kk) \defeq \frac{1}{(2\pi)^s} \int_{[-\pi, \pi]^s} f(x) \overline{e_{\kk}(x)}dx.
    \end{equation*}
\end{definition}
\begin{definition}
    For two functions $f,g \in C_{2\pi}\left(\RR^s;\CC\right)$, we define their convolution as
    \begin{equation*}
        f * g : \quad \RR^s \to \CC, \quad (f*g)(x) \defeq \frac{1}{(2\pi)^s} \int_{[-\pi, \pi]^s} f(t)g(x-t) dt.
    \end{equation*}
\end{definition}
In the following we define several so-called kernels.
\begin{definition} 
     Let $m \in \NN_0$ be arbitrary.
    \begin{enumerate}
        \item The $m$-th one-dimensional \emph{Dirichlet-kernel} is defined as
        \begin{equation*}
            D_m \defeq \sum_{h= - m}^{m} e_h.
        \end{equation*}
        \item The $m$-th one-dimensional \emph{Fejèr-kernel} is defined as
        \begin{equation*}
            F_m \defeq \frac{1}{m}\sum_{h=0}^{m-1} D_h.
        \end{equation*}
        \item The $m$-th one-dimensional \emph{de-la-Vallée-Poussin-kernel} is defined as
        \begin{equation*}
            V_m \defeq \left( 1 + e_m + e_{-m}\right) \cdot F_m.
        \end{equation*}
        \item Let $s \in \NN$. We extend the above definitions to dimension $s$ by letting
        \begin{align*}
            D_m^s \left(x_1, ..., x_s\right) &\defeq \prod_{p=1}^s D_m \left(x_p\right), \\
            F_m^s \left(x_1, ..., x_s\right) &\defeq \prod_{p=1}^s F_m \left(x_p\right), \\
            V_m^s \left(x_1, ..., x_s\right) &\defeq \prod_{p=1}^s V_m \left(x_p\right). 
        \end{align*}
    \end{enumerate}
\end{definition}
We need the following property of the multivariate extension of the de-la-Vallée-Poussin-kernel.
\begin{proposition} \label{prop: dlvp-bound}
    Let $m,s \in \NN$. Then one has $\left\Vert V_m^s \right\Vert_{L^1 \left([- \pi, \pi]^s; \CC\right)} \leq 3^s$.
\end{proposition}
\begin{proof}
   From \cite[Exercise 1.3 and Lemma 1.4]{muscalu_classical_2013} it follows $\Vert F_m\Vert_{L^1 ([-\pi, \pi] ; \CC)} = 1$ and hence using the triangle inequality $\Vert V_m \Vert_{L^1 ([-\pi,\pi] ; \CC)} \leq 3$. The claim then follows using Tonelli's theorem.
\end{proof}
The following definition introduces the term of trigonometric polynomial. 
\begin{definition}
    For any $s \in \NN$ and $m \in \NN_0$ we call a function of the form
    \begin{equation*}
    \RR^s \to \CC, \quad x \mapsto \underset{-m \leq \kk \leq m}{\sum_{\kk \in \ZZ_0^s}} a_\kk e^{i \langle \kk , x\rangle}
\end{equation*}
with coefficients $a_\kk \in \CC$ a \emph{trigonometric polynomial of coordinatewise degree at most $m$} and denote the space of all those functions with $\mathbb{H}_m^s$. Here, we consider the sum over all $\kk \in \ZZ^s$ with $-m \leq \kk_j \leq m$ for all $j \in \{1,...,s\}$. We then write
\begin{equation} \label{eq:mintrigo}
    E^s_m(f) \defeq \underset{T \in \mathbb{H}_{m}^s}{\mathrm{min}} \left\Vert f - T\right\Vert_{L^\infty \left(\RR^s;\CC\right)}
\end{equation}
for any function $f \in C_{2\pi} \left(\RR^s ; \CC\right)$.
\end{definition}
The following proposition shows that convolving with the Fejèr kernel produces a trigonometric polynomial of degree at most $2m-1$, while reproducing trigonometric polynomials of degree $m$. Furthermore, the norm of the convolution operator is bounded uniformly in $m$. These properties will be useful for our proof of \Cref{app: fourier_approx}.
\begin{proposition}
\label{vm}
    Let $s,m \in \NN$ and $k \in \NN_0$. The map
    \begin{equation*}
        v_m : \quad C_{2\pi} \left(\RR^s; \CC\right) \to \mathbb{H}_{2m-1}^s, \quad f \mapsto f * V_m^s 
    \end{equation*}
    is well-defined and satisfies
    \begin{equation}
    \label{ident}
        v_m(T) = T \quad \text{for all} \quad T \in \mathbb{H}_m^s.
    \end{equation}
    Furthermore, there exists a constant $c = c(s) > 0$ (independent of $m$), such that
    \begin{align}
        \left\Vert v_m(f)\right\Vert_{C^k\left([-\pi,\pi]^s; \CC\right)} \leq c \cdot \left\Vert f\right\Vert_{C^k\left([-\pi,\pi]^s; \CC\right)} \ \forall f \in C^k_{2\pi} \left(\RR^s; \CC\right), \nonumber \\
        \label{constant}
        \left\Vert v_m(f)\right\Vert_{L^\infty \left([\pi,\pi]^s; \CC\right)} \leq c \cdot \left\Vert f\right\Vert_{L^\infty \left([-\pi, \pi]^s; \CC\right)} \ \forall f\in C_{2\pi} \left(\RR^s; \CC\right).
    \end{align}
    In fact, it holds $c(s) \leq \exp(C \cdot s)$ with an absolute constant $C>0$.
\end{proposition}
\begin{proof}
    A direct computation shows that $f \ast e_{\kk} = \hat{f}(\kk) \cdot e_{\kk}$. This implies that $v_m$ is well-defined since $V_m^s$ is a trigonometric polynomial of coordinatewise degree at most $2m-1$. 

    The operator is bounded on $C^k_{2\pi}(\RR^s;\CC)$ and $C_{2\pi}(\RR^s;\CC)$ with norm at most $c = 3^s$, as follows from Young's inequality \cite[Lemma 1.1 (ii)]{muscalu_classical_2013}, \Cref{prop: dlvp-bound}, and the fact that one has for all $\kk \in \NN_0^s$ with $\vert \kk \vert \leq k$ the identity
    \begin{equation*}
         \partial^\kk \left( f * V_m^s\right) = \left(\partial^\kk f\right) * V_m^s \quad \text{for } f \in C^k_{2\pi}(\RR^s ; \CC).
    \end{equation*}
    It remains to show that $v_m$ is the identity on $\mathbb{H}_m^s$. We first prove that 
    \begin{equation}
    \label{firstident}
        e_k * V_m = e_k
    \end{equation}
    holds for all $k \in \Z$ with $\vert k \vert \leq m$. First note that
    \begin{equation*}
        e_k * V_m = e_k * F_m + e_k * \left(e_m \cdot F_m\right) + e_k * \left(e_{-m} \cdot F_m\right).
    \end{equation*}
    We then compute
    \begin{align*}
        e_k * F_m = \frac{1}{m} \sum_{\ell = 0}^{m-1} D_\ell * e_k = \frac{1}{m}\sum_{\ell = 0}^{m-1} \sum_{h = - \ell}^{\ell} \underbrace{e_h * e_k}_{= \delta_{k,h} \cdot e_k} = \frac{1}{m} \sum_{\ell = \vert k \vert}^{m-1} e_k = \frac{m - \vert k \vert}{m} \cdot e_k.
    \end{align*}
    Similarly, we get
    \begin{align*}
        e_k *\left( e_m \cdot F_m \right) &= \frac{1}{m} \sum_{\ell = 0}^{m-1} \left( e_m D_\ell \right) * e_k = \frac{1}{m}\sum_{\ell = 0}^{m-1} \sum_{h = - \ell}^{\ell} \underbrace{e_{h+m} * e_k}_{= \delta_{k,h+m} \cdot e_k}\\
        & = \frac{1}{m} \underset{\ell \geq m-k}{\sum_{0 \leq \ell \leq m-1 }} e_k = \delta_{k \geq 1} \cdot \frac{k}{m} \cdot e_k
    \end{align*}
    and 
    \begin{align*}
        e_k *\left( e_{-m} \cdot F_m \right) &= \frac{1}{m} \sum_{\ell = 0}^{m-1} \left( e_{-m} D_\ell \right) * e_k = \frac{1}{m}\sum_{\ell = 0}^{m-1} \sum_{h = - \ell}^{\ell} \underbrace{e_{h-m} * e_k}_{= \delta_{k,h-m} \cdot e_k} \\
        &= \frac{1}{m} \underset{\ell \geq k+m}{\sum_{0 \leq \ell \leq m-1}} e_k = \delta_{k \leq -1} \cdot \frac{-k}{m} \cdot e_k.
    \end{align*}
    Adding up those three identities yields (\ref{firstident}). 

    To finally prove \eqref{ident}, it clearly suffices to show that 
    \begin{equation*}
        e_\kk * V_m^s = e_\kk
    \end{equation*}
    for all $\kk \in \Z^s$ with $-m \leq \kk \leq m$. But for such $\kk$, using $e_\kk(x) = \prod_{j=1}^{s} e_{\kk_j}\left( x_j\right)$, one obtains
    \begin{align*}
        \left(e_\kk * V_m^s\right)(x) &= \frac{1}{(2\pi)^s} \int_{[-\pi, \pi]^s} \prod_{j=1}^{s} e_{\kk_j} \left(t_j\right) \cdot V_m \left(x_j - t_j\right) dt \\
        \overset{\text{Fubini}}&{=} \prod_{j=1}^s \left(e_{\kk_j} * V_m\right)\left(x_j \right) 
        \overset{\eqref{firstident}}{=} \prod_{j=1}^s e_{\kk_j} \left(x_j\right) 
        = e_\kk(x)
    \end{align*}
    for any $ x \in \RR^s$, as was to be shown.
\end{proof}
The following result follows from a theorem in \cite{lorentz_approximation_2005}.
\begin{proposition}
\label{lorentz_aussage}
    Let $s,k \in \NN$. Then there exists a constant $c = c(s,k) > 0$, such that, for $E_m^s$ as defined in \eqref{eq:mintrigo},
    \begin{equation*}
        E_m^s(f) \leq \frac{c}{m^k} \cdot \left\Vert f\right\Vert_{C^k\left([-\pi, \pi]^s; \RR\right)}
    \end{equation*}
    for all $m\in \NN$ and $f \in C^k_{2\pi} \left(\RR^s; \RR\right)$. \newline
    In fact, it holds $c(s,k) \leq \exp(C \cdot ks) \cdot k^k $ with an absolute constant $C>0$.
\end{proposition}
\begin{proof}
    We apply \cite[Theorem 6.6]{lorentz_approximation_2005} with $n_i = m$ and $p_i = k$, which yields the existence of a constant $c_1 = c_1(s,k)>0$, such that
    \begin{equation*}
        E_m^s(f) \leq c_1 \cdot \sum_{\ell=1}^s \frac{1}{m^k} \cdot \omega_{\ell} \left(f,\frac{1}{m}\right)
    \end{equation*}
    for all $m \in \NN$ and $f \in C^k_{2\pi}(\RR^s ; \RR)$, where $\omega_\ell(f, \bullet)$ denotes the modulus of continuity of $\frac{\partial^k f }{\partial x_\ell ^k}$ with respect to $x_\ell$, where we have the trivial bound
    \begin{equation*}
         \omega_\ell \left(f,\frac{1}{m}\right)  \leq 2 \cdot \left\Vert f \right\Vert_{C^k\left([-\pi, \pi]^s; \RR\right)}.
    \end{equation*}
    Hence, we get
    \begin{equation*}
        E_m^s(f) \leq c_1 \cdot s \cdot 2 \cdot \left\Vert f \right\Vert_{C^k\left([-\pi, \pi]^s; \RR\right)} \frac{1}{m^k},
    \end{equation*}
    so the claim follows by choosing $c := 2s \cdot c_1$.
    
    We refer to \Cref{sec:const_bound_reordered} (see \Cref{thm:const_lorentz_bound}) for a proof of the claimed bound on the constant $c(s,k)$.
\end{proof}
The above proposition bounds the best possible error of approximating $f$ by trigonometric polynomials of coordinatewise degree at most $m$, but this is in general non-constructive. Our next result shows that a similar bound holds for approximating $f$ by $v_m(f)$.
\begin{theorem}
\label{vm_approx}
    Let $s \in \NN$. Then there exists a constant $ c = c(s) > 0 $, such that the operator $v_m$ from \Cref{vm} satisfies
    \begin{equation*}
        \left\Vert f - v_m(f) \right\Vert_{L^\infty \left( \RR^s\right)} \leq c \cdot E^s_m(f)
    \end{equation*}
    for any $m \in \NN$ and $f \in C_{2\pi} \left(\RR^s; \CC\right)$. \newline
    In fact, it holds $c(s) \leq \exp(C \cdot s)$ with an absolute constant $C>0$.
\end{theorem}
\begin{proof}
    For any $T \in \mathbb{H}_m^s$ one has
    \begin{equation*}
        \left\Vert f - v_m(f) \right\Vert_{L^\infty \left( \RR^s\right)} \overset{\eqref{ident}}{\leq} \left\Vert f - T \right\Vert_{L^\infty \left( \RR^s\right)} + \left\Vert v_m(T) - v_m(f) \right\Vert_{L^\infty \left( \RR^s\right)} \overset{\eqref{constant}}{\leq} (c+1) \left\Vert f - T \right\Vert_{L^\infty \left( \RR^s\right)}. 
    \end{equation*}
    Taking the infimum over all $T \in \mathbb{H}_m^s$ yields the claim. 
\end{proof}
By combining \Cref{lorentz_aussage} and \Cref{vm_approx}, we immediately get the following bound.
\begin{corollary}
\label{dlvp}
    Let $s,k \in \NN_0$. Then there exists a constant $c = c(s,k)>0$, such that
    \begin{equation*}
        \left\Vert f - v_m(f) \right\Vert_{L^\infty \left(\RR^s\right)} \leq \frac{c}{m^k} \cdot \left\Vert f \right\Vert_{C^k \left([-\pi,\pi]^s; \RR\right)}
    \end{equation*}
    for every $m \in \NN$ and $f \in C^k_{2\pi} \left(\RR^s ; \RR\right)$. \newline
    In fact, we have $c(s,k) \leq \exp(C \cdot ks) \cdot k^k$ with an absolute constant $C>0$.
\end{corollary}
Up to now, we have studied the approximation of periodic functions by trigonometric polynomials, but our actual goal is to approximate non-periodic functions by algebraic polynomials. The next lemma establishes a connection between the two settings. 
\begin{lemma}
\label{star_operator}
    Let $k \in \NN_0$ and $ s \in \NN$. For any function $f \in C^k \left([-1,1]^s ; \CC\right)$, we define the corresponding periodic function via
    \begin{equation*}
        f^* : \quad \RR^s \to \CC, \quad  f^* \left(x_1, ..., x_s\right) = f(\mathrm{cos} \left(x_1\right), ..., \mathrm{cos} \left( x_s\right))
    \end{equation*}
    and note $f^* \in C_{2\pi}^k \left(\RR^s; \CC\right)$. The map
\begin{equation*}
	C^k \left([-1,1]^s ; \CC\right) \to C^k _ {2\pi}\left(\RR^s; \CC\right), \quad f \mapsto f^*
\end{equation*}
is a continuous linear operator with respect to the $C^k$-norms on $C^k \left([-1,1]^s ; \CC\right)$ and $C^k _ {2\pi}\left(\RR^s; \CC\right)$. \newline
The operator norm can be bounded from above by $k^k$.
\end{lemma}
\begin{proof}
	The map is well-defined since $\cos$ is a smooth function and $2\pi$-periodic. The linearity of the operator is obvious, so it remains to show its continuity.

	The goal is to apply the closed graph theorem \cite[Theorem 5.12]{folland_real_1999}. By definition of $f^*$, and since $\cos:[-\pi,\pi] \to [-1,1]$ is surjective, we have the equality $\left\Vert f\right\Vert_{L^\infty \left([-1, 1]^s; \CC\right)} = \left\Vert f^* \right\Vert_{L^\infty \left([-\pi, \pi]^s; \CC\right)} $. Let then $\left(f_n\right)_{n \in \NN}$ be a sequence of functions $f_n \in C^k \left([-1,1]^s ; \CC\right)$ and $g^* \in C^k_{2\pi}\left(\RR^s ; \CC\right)$ such that $f_n \to f$ in $C^k \left([-1,1]^s; \CC\right)$ and $f_n^* \to g^*$ in $C^k_{2\pi} \left(\RR^s; \CC\right)$. We then have
\begin{align*}
	\left\Vert f^* - g^* \right\Vert_{L^\infty \left([-\pi, \pi]^s\right)} &\leq \left\Vert f^* - f_n^* \right\Vert_{L^\infty \left([-\pi, \pi]^s\right)} + \left\Vert f_n^* - g^* \right\Vert_{L^\infty \left([-\pi, \pi]^s\right)} \\
&= \left\Vert f - f_n \right\Vert_{L^\infty  \left([-1, 1]^s ; \CC\right)} + \left\Vert f_n^* - g^* \right\Vert_{L^\infty \left([-\pi, \pi]^s\right)} \\
&\leq \left\Vert f - f_n \right\Vert_{C^k  \left([-1, 1]^s ; \CC\right)} + \left\Vert f_n^* - g^* \right\Vert_{C^k([-\pi, \pi]^s ; \CC)} \to 0 \ (n \to \infty).
\end{align*}
It follows $f^* = g^*$ and the closed graph theorem yields the desired continuity. 

We refer to \Cref{sec:const_bound_reordered} (see \Cref{thm:faa,rem:multiindex}) for a proof of the claimed bound on the operator norm.
\end{proof}

For a function $f \in C^k([-1,1]^s;\CC)$ we want to express $v_m(f^*)$ in a convenient way, involving a product of cosines. To this end, we make use of the following identity, which is a generalization of the well-known product-to-sum formula for $\cos$.
\begin{lemma}
\label{prod_sum}
Let $s \in \NN$. Then it holds for any $x \in \RR^s$ that
\begin{equation*}
\prod_{j=1}^s \cos (x_j) = \frac{1}{2^s} \sum_{\sigma \in \{-1,1\}^s} \cos (\langle\sigma,x \rangle).
\end{equation*} 
\end{lemma}
\begin{proof}
This is an inductive generalization of the product-to-sum formula 
\begin{equation}
\label{product_sum_form}
2\cos(x) \cos(y) = \cos(x-y) + \cos (x+y)
\end{equation}
for $x,y \in \RR$, which can be found for instance in \cite[Eq.~4.3.32]{abramowitz_handbook_2013}. The case $s= 1$ holds since $\cos$ is an even function. Assume that the claim holds for a fixed $s \in \NN$ and take $x \in \RR^{s+1}$. Writing $x' = (x_1, ..., x_s)$, we derive
\begin{align*}
\prod_{j=1}^{s+1} \cos(x_j) &= \left(\frac{1}{2^s} \sum_{\sigma \in \{-1,1\}^s} \cos (\langle \sigma , x' \rangle) \right) \cdot \cos(x_{s+1}) \\
&= \frac{1}{2^s} \sum_{\sigma \in \{-1,1\}^s}\cos (\langle \sigma, x' \rangle) \cos(x_{s+1})\\
\overset{\eqref{product_sum_form}}&{=} \frac{1}{2^{s+1}} \sum_{\sigma \in \{-1,1\}^s} \left[\cos (\langle \sigma, x' \rangle + x_{s+1}) + \cos (\langle \sigma, x' \rangle - x_{s+1}) \right]  \\
&= \frac{1}{2^{s+1}} \sum_{\sigma \in \{-1,1\}^{s+1}} \cos (\langle \sigma, x\rangle), 
\end{align*}
as was to be shown. 
\end{proof}
The following proposition states that $v_m(f^*)$ can be expressed as a linear combination of products of cosines. This representation is useful since these cosines can be interpolated by Chebyshev polynomials which in the end leads to the desired approximation result.
\let \hat \widehat
\begin{proposition}
\label{representation}
    Let $s\in \NN$ and $k \in \NN_0$. For any $f \in C^k \left([-1,1]^s ; \CC\right)$ and $m \in \NN$ the de-la-Vallée-Poussin operator given as $f \mapsto v_m \left(f^*\right)$ with $v_m$ as in \Cref{vm} and $f \mapsto f^*$ as in \Cref{star_operator} has a representation
    \begin{equation*}
        v_m \left(f^*\right) \left(x_1, ..., x_s\right) = \underset{\kk \leq 2m -1}{\sum_{\kk \in \NN_0^s}} \mathcal{V}_\kk^m(f) \prod_{j=1}^{s} \mathrm{cos} \left(\kk_j x_j\right)
    \end{equation*}
    for continuous linear functionals
    \begin{equation*}
        \mathcal{V}_{\kk}^m : \ C^k \left([-1,1]^s; \CC\right) \to \CC, \quad f \mapsto 2^{\Vert \kk \Vert_0}\cdot  a_{\kk}^m \cdot \widehat{f^*}(\kk),
    \end{equation*}
where $\Vert \kk \Vert_0 = \# \{j \in \{1,...,s\}: \ \kk_j \neq 0\}$ and $a_{\kk}^m = \widehat{V_m^s}(\kk)$. Furthermore, if $f \in C^k ([-1,1]^s ; \RR)$, then $\mathcal{V}_\kk^m (f) \in \RR$ for every $\kk \in \NN_0^s$ with $\kk \leq 2m-1$.
\end{proposition}
\begin{proof}
    First of all, it is easy to see that $v_m \left( f^* \right)$ is even in each variable, which follows directly from the fact that $ f^*$ and $V_m^s$ are both even in each variable. Furthermore, if we write 
    \begin{equation*}
        V_m^s = \underset{-(2m-1) \leq \kk \leq 2m-1}{\sum_{\kk \in \ZZ^s}} a_\kk^m e_\kk
    \end{equation*}
    with appropriately chosen coefficients $a_\kk^m \in \RR$, we easily see
    \begin{equation*}
        v_m \left(f^*\right) = \underset{-(2m-1) \leq \kk \leq 2m-1}{\sum_{\kk \in \ZZ^s}} a_\kk^m \hat{f^*}(\kk) e_\kk.
    \end{equation*}
    Using Euler's identity and the fact that $v_m\left(f^*\right)$ is an even function, we get the representation 
    \begin{equation*}
        v_m \left(f^*\right)(x) = \underset{-(2m-1)\leq \kk \leq 2m-1}{\sum_{\kk \in \ZZ^s}} a_\kk^m \hat{f^*}(\kk) \text{cos}(\langle \kk, x\rangle)
    \end{equation*}
    for all $x \in \RR^s$. Using $\odot$ to denote the componentwise product of two vectors of the same size, i.e., $x \odot y = (x_i \cdot y_i)_i$, and using the identity $\langle \kk, \sigma \odot x\rangle = \langle \sigma, \kk \odot x \rangle$, we see since $v_m \left(f^*\right)$ is even in every variable that
    \begin{align*}
        v_m \left(f^*\right) (x) &= \frac{1}{2^s} \cdot \sum_{\sigma \in \{-1,1\}^s} v_m \left( f^* \right) (\sigma \odot x) \\
        &=\frac{1}{2^s} \cdot \sum_{\sigma \in \{-1,1\}^s} \underset{-(2m-1) \leq \kk \leq 2m-1}{\sum_{\kk \in \ZZ^s}} a_\kk^m \hat{f^*}(\kk) \text{cos}(\langle \kk, \sigma \odot x\rangle) \\
        &= \underset{-(2m-1) \leq \kk \leq 2m-1}{\sum_{\kk \in \ZZ^s}} \left(a_\kk^m \hat{f^*}(\kk) \frac{1}{2^s} \sum_{\sigma \in \{-1,1\}^s}\text{cos}(\langle \sigma, \kk \odot x\rangle)\right) \\
        \overset{\text{\Cref{prod_sum}}}&{=} \underset{-(2m-1) \leq \kk \leq 2m-1}{\sum_{\kk \in \ZZ^s}} a_\kk^m \hat{f^*}(\kk) \prod_{j=1}^{s} \cos \left(\kk_j x_j\right) \\
        &= \underset{\kk \leq 2m-1}{\sum_{\kk \in \NN_0^s}} 2^{\Vert \kk \Vert_0} a_\kk^m \hat{f^*}(\kk) \prod_{j=1}^{s} \cos \left(\kk_j x_j\right)
    \end{align*}
    with
    \begin{equation*}
        \left\Vert \kk \right\Vert_0 \defeq \#\big\{ j \in \{1,...,s\} : \ \kk_j \neq 0\big\}.
    \end{equation*}
    In the last step we again used that cos is an even function and that
    \begin{equation*}
        \hat{f^*}(\kk) = \hat{f^*}(\sigma \odot \kk)
    \end{equation*}
    for all $\sigma \in \{-1,1\}^s$, which also follows easily since $f^*$ and $V_m^s$ are even in every component. Letting \begin{equation*}
        \mathcal{V}_\kk^m(f) \defeq 2^{\Vert \kk \Vert_0} a_\kk^m \hat{f^*}(\kk),
    \end{equation*} 
    we have the desired form. The fact that $\mathcal{V}_{\kk}^m$ is a continuous linear functional on $C^k_{2\pi}\left([-1,1]^s;\CC\right)$ follows directly since $f \mapsto \widehat{f^*}(\kk)$ is a continuous linear functional for every $\kk$. If $f$ is real-valued, so is $\hat{f^*}(\kk)$ for every $\kk \in \NN_0^s$ with $\kk \leq 2m-1$, since $f^*$ is real-valued and even in every component.
\end{proof}
\let \hat \hat

Our main approximation result involves linear combinations of Chebyshev polynomials where the coefficients in this linear combination are given as $\mathcal{V}_{\kk}^m(f)$. It is therefore important to be able to bound the sum of the absolute values $\vert \mathcal{V}_{\kk}^m(f) \vert$.
\begin{lemma}
\label{sum_bound}
    Let $s \in \NN$. There exists a constant $c = c(s)> 0$, such that the inequality
    \begin{equation*}
        \underset{\kk \leq 2m-1}{\sum_{\kk \in \NN_0^s}} \left\vert \mathcal{V}_\kk^m(f) \right\vert \leq c \cdot m^{s/2} \cdot \left\Vert f \right\Vert_{L^\infty \left([-1, 1]^s ; \CC\right)}
    \end{equation*}
    holds for all $m \in \NN$ and $f \in C \left([-1, 1]^s ; \CC\right)$, where $\mathcal{V}_\kk^m$ is as in \Cref{representation}. \newline
    In fact, we have $c(s) \leq \exp(C \cdot s)$ with an absolute constant $C>0$.
\end{lemma}
\begin{proof}
    Let $f \in C \left([-1, 1]^s ; \CC\right)$ and $m \in \NN$. For any multi-index $\elll \in \NN_0^s$, it follows from \Cref{representation} that 
    \begin{equation*}
        \widehat{v_m\left(f^*\right)} (\elll) = \underset{\kk \leq 2m-1}{\sum_{\kk \in \NN_0^s}} \mathcal{V}_\kk^m(f) \widehat{g_\kk}(\elll),
    \end{equation*}
    with
    \begin{equation*}
        g_\kk : \quad \RR^s \to \RR, \quad \left(x_1, ..., x_s\right) \mapsto \prod_{j=1}^s \cos \left(\kk_j x_j\right).
    \end{equation*}
    Now, a calculation using Fubini's theorem and using 
    \begin{equation*}
    g_{\kk} = \prod_{j=1}^s\frac{1}{2} \left( e_{\kk_j} + e_{-\kk_j}\right) = \underset{\kk_j \neq 0}{\prod_{1 \leq j \leq s}} \frac{1}{2} \left(e_{\kk_j} + e_{-\kk_j}\right)
    \end{equation*} for any number $k \in \NN_0$ shows 
    \begin{equation*}
        \widehat{g_\kk}(\elll) =  \begin{cases} \frac{1}{2^{\Vert \kk \Vert_0}},&\kk=\elll, \\ 0, & \text{otherwise}\end{cases} \quad\text{for } \kk, \elll \in \NN_0^s.
    \end{equation*}
    Therefore, we have the bound $\left\vert\mathcal{V}^m_{\elll} (f)\right\vert \leq 2^s \cdot \left\vert \widehat{v_m\left(f^*\right)}(\elll)\right\vert$ for $\elll \in \NN_0^s $ with $\vert \elll \vert \leq 2m-1$. Using the Cauchy-Schwarz and the Parseval inequality, we therefore see
    \begin{align*}
        \underset{\kk \leq 2m-1}{\sum_{\kk \in \NN_0^s}} \left\vert \mathcal{V}_\kk^m(f)\right\vert &\leq 2^s \cdot \underset{\kk \leq 2m-1}{\sum_{\kk \in \NN_0^s}} \left\vert \widehat{v_m\left(f^*\right)}(\kk)\right\vert \overset{\text{CS}}{\leq} 2^s \cdot (2m)^{s/2} \cdot \left(\underset{\kk \leq 2m-1}{\sum_{\kk \in \NN_0^s}} \left\vert \widehat{v_m\left(f^*\right)}(\kk)\right\vert^2 \right)^{1/2} \\
        \overset{\text{Parseval}}&{\leq} 2^s \cdot 2^{s/2} \cdot m^{s/2} \cdot \left\Vert v_m \left(f^*\right)\right\Vert_{L^2 \left([-\pi, \pi]^s; \CC\right)} \\
        &\leq \underbrace{2^s \cdot 2^{s/2} }_{=: c_1(s)}\cdot m^{s/2} \cdot \left\Vert v_m \left(f^*\right)\right\Vert_{L^\infty \left([-\pi, \pi]^s; \CC\right)}.
    \end{align*}
    Using \Cref{vm}, we get a constant $c_2(s) \leq \exp(C_0 \cdot s)$ such that
    \begin{align*}
        \underset{\kk \leq 2m-1}{\sum_{\kk \in \NN_0^s}} \left\vert \mathcal{V}_\kk^m(f)\right\vert &\leq c_1(s) \cdot c_2(s) \cdot m^{s/2} \cdot \left\Vert f^*\right\Vert_{L^\infty \left([-\pi, \pi]^s;\CC\right)} = c(s) \cdot m^{s/2} \cdot \left\Vert f\right\Vert_{L^\infty \left([-1, 1]^s; \CC\right)},
    \end{align*}
    as claimed.
\end{proof}

For any natural number $\ell \in \NN_0$, we denote by $T_\ell$ the $\ell$-th \emph{Chebyshev polynomial}, satisfying
\begin{equation*}
    T_\ell\left(\cos(x)\right) = \cos(\ell x), \quad x \in \RR.
\end{equation*}
One can show that $T_\ell$ is in fact a polynomial of degree $\ell$. For a multi-index $\kk \in \NN_0^s$, we define
\begin{equation*}
    T_\kk (x) \defeq \prod_{j=1}^s T_{\kk_j}\left(x_j\right), \quad x \in \RR^s.
\end{equation*}
We then get the following approximation result about approximating (non-periodic) $C^k$-functions by linear combinations of Chebyshev polynomials. 
\begin{theorem} \label{app: fourier_approx}
    Let $k,s,m \in \NN$. Then there exists a constant $c=c(s,k)>0$ with the following property: For any $f \in C^k \left([-1,1]^s; \RR\right)$ the polynomial $P$ defined as
    \begin{equation*}
        P(x) \defeq \underset{\kk \leq 2m-1}{\sum_{\kk \in \NN_0^s}}\mathcal{V}_\kk^m(f) \cdot T_\kk(x),
    \end{equation*}
    with $\mathcal{V}^m_\kk$ as in \Cref{representation}, satisfies
    \begin{equation*}
        \left\Vert f - P \right\Vert_{L^\infty \left([-1,1]^s ;\RR\right)} \leq \frac{c}{m^k} \cdot \left\Vert f \right\Vert_{C^k\left([-1,1]^s;\RR\right)}.
    \end{equation*}
    Here, the maps
\begin{equation*}
C\left([-1,1]^s ; \RR\right) \to \RR, \quad f \mapsto \mathcal{V}_{\kk}^m(f)
\end{equation*}
are continuous and linear functionals with respect to the $L^\infty$-norm. Furthermore, there exists a constant $\tilde{c} = \tilde{c}(s)> 0$, such that the inequality
    \begin{equation*}
        \underset{\kk \leq 2m-1}{\sum_{\kk \in \NN_0^s}} \left\vert \mathcal{V}_\kk^m(f) \right\vert \leq \tilde{c} \cdot m^{s/2} \cdot \left\Vert f \right\Vert_{L^\infty \left([-1, 1]^s;\RR\right)}
    \end{equation*}
    holds for all $f \in C \left([-1, 1]^s ; \RR\right)$. \newline
    Moreover, we have $c(s,k) \leq \exp(C \cdot ks) \cdot k^{2k}$ and $\tilde{c}(s) \leq \exp(C \cdot s)$ with an absolute constant $C>0$.
    
\end{theorem}
\begin{proof}
    We choose the constant $c_0 = c_0(s,k)$ according to \Cref{dlvp}. Let $f \in C^k\left([-1,1]^s;\RR\right)$ be arbitrary. Then we define the corresponding function $f^* \in C^k_{2\pi}\left(\RR^s; \RR\right)$ as above. Let $P$ be defined as in the statement of the theorem. Then it follows from the definition of the Chebyshev polynomials $T_\kk$, the definition of $P$, and the formula for $v_m(f^*)$ from \Cref{representation} that 
    \begin{equation*}
        P^* (x) = v_m\left(f^*\right)(x)
    \end{equation*}
    is satisfied, where $P^*$ is the corresponding function to $P$ defined similarly to $f^*$. Overall, we get the bound
    \begin{align*}
        \left\Vert f - P \right\Vert_{L^\infty \left([-1,1]^s; \RR\right)} = \left\Vert f^* - P^* \right\Vert_{L^\infty \left([-\pi,\pi]^s; \RR\right)} \overset{\text{Cor. \ref{dlvp}}}{\leq} \frac{c_0}{m^k} \cdot \left\Vert f^* \right\Vert_{C^k\left([-\pi, \pi]^s; \RR\right)}.
    \end{align*}
   The first claim then follows using the continuity of the map $f \mapsto f^*$ as proven in \Cref{star_operator}. The second part of the theorem has already been proven in \Cref{sum_bound}.
\end{proof}



\subsection{Details on bounding the constant \texorpdfstring{$c$}{c} in Theorem \ref{main_2}}
\label{sec:const_bound_reordered}

In this appendix we provide details on the bound of the constant $c$ that appears in the formulation of \Cref{main_2}. Specifically, we perform a careful investigation of several results from \cite{lorentz_approximation_2005} to get an upper bound for the constant appearing in \Cref{lorentz_aussage}. Moreover, we analyze the operator norm of the operator
\begin{equation*}
C^k([-1,1]^s; \CC) \to C_{2\pi}^k(\RR^s; \CC), \quad f \mapsto f^* \quad \text{with}\quad f^*(x) \defeq f(\cos(x_1),..., \cos(x_s))
\end{equation*}
appearing in \Cref{star_operator} and show that it is bounded from above by $k^k$.

We start with the analysis of some bounds in \cite[Chapter 4.3]{lorentz_approximation_2005}. Here, a generalization of \emph{Jackson's kernel} is defined for any $m,r \in \NN$ as
\begin{equation*}
L_{m,r}(t) \defeq \lambda_{m,r}^{-1} \cdot \left(\frac{\sin(mt/2)}{\sin(t/2)}\right)^{2r}, \quad t \in \RR,
\end{equation*}
where $\lambda_{m,r}$ is chosen such that
\begin{equation*}
\int_{[-\pi, \pi]} L_{m,r}(t) \ dt = 1.
\end{equation*}
The first two important bounds are provided in the following proposition.
\begin{proposition}\label{prop:lorentz_bound_1}
Let $m,r \in \NN$. Then it holds 
\begin{equation*}
\lambda_{m,r}^{-1} \leq \exp(C \cdot r) \cdot m^{1-2r} \quad \text{and} \quad \int_{[0, \pi]} t^k L_{m,r}(t) \ dt \leq \exp(C \cdot r) \cdot m^{-k}
\end{equation*}
for any $k \leq 2r-2$, with an absolute constant $C>0$.
\end{proposition}
\begin{proof}
Since $L_{m,r} \geq 0$ and since $ \sin(t/2) \leq t/2$ for $t \in [0,\pi]$, we get
\begin{align*}
\lambda_{m,r} &\geq \int_{[0, \pi]} \left(\frac{\sin(mt/2)}{t/2}\right)^{2r} \ dt = 2^{2r} \cdot \int_{[0,\pi]} \left(\frac{\sin(mt/2)}{t}\right)^{2r} \ dt \\
&= 2^{2r} \cdot \int_{[0,\pi m/2]} \left(\frac{\sin(u)}{(2u)/m}\right)^{2r} \ du \cdot \frac{2}{m} \geq m^{2r-1} \cdot \int_{[0, \pi m/2]} \left(\frac{\sin(u)}{u}\right)^{2r} \ du \\
&\geq m^{2r-1} \cdot \int_{[0, \pi/2]} \left(\frac{\sin(u)}{u}\right)^{2r} \ du \geq m^{2r-1} \cdot \int_{[0, \pi/2]} \left(\frac{2u}{\pi \cdot u}\right)^{2r} \ du \geq \left(\frac{2}{\pi}\right)^{2r} \cdot m^{2r-1}.
\end{align*}
Here, we employed the inequality $\sin(u) \geq \frac{2}{\pi} u$ for $u \in [0, \pi/2]$ in the penultimate step.\footnote{To see that this inequality holds, note that $\sin''(u) = -\sin(u) \leq 0$ for $u \in [0,\pi/2]$, so that $\sin$ is concave on that interval, and hence $\sin(u) = \sin((1- \frac{2u}{\pi})\cdot 0 + \frac{2u}{\pi} \cdot \frac{\pi}{2}) \geq \frac{2u}{\pi}\sin(\frac{\pi}{2})= \frac{2u}{\pi}$.} This shows the first part of the claim. 

For the second part, we again use the estimate $\sin(u) \geq \frac{2}{\pi}u$ for $u \in [0, \pi/2]$ to compute
\begin{align*}
\int_{[0, \pi]} t^k L_{m,r}(t) \ dt &= \lambda_{m,r}^{-1} \cdot \int_{[0,\pi]} t^k \left(\frac{\sin(mt/2)}{\sin(t/2)}\right)^{2r} \ dt \leq \lambda_{m,r}^{-1} \cdot \int_{[0,\pi]} t^k \left(\frac{\sin(mt/2)}{t/\pi}\right)^{2r} \ dt \\
&= \lambda_{m,r}^{-1} \cdot \pi^{2r} \cdot \int_{[0,\pi]} t^{k-2r} \sin(mt/2)^{2r} \ dt \\
&= \lambda_{m,r}^{-1} \cdot \pi^{2r} \cdot \int_{[0,\pi m /2]} \left(\frac{2u}{m}\right)^{k-2r} \sin(u)^{2r} \ du \cdot \frac{2}{m} \\
&\leq \lambda_{m,r}^{-1} \cdot \pi^{2r} \cdot m^{2r -1 - k}\int_{[0,\pi m /2]} u^{k-2r} \sin(u)^{2r} \ du \\
&\leq \exp(C_1 \cdot r) \cdot \int_{[0, \infty)} u^{k-2r} \cdot \sin(u)^{2r} \ du \cdot m^{-k}
\end{align*}
with an absolute constant $C_1 > 0$. Here, we employed the first part of this proposition. It remains to bound the integral. This is done via
\begin{align*}
\int_{[0, \infty)} u^{k-2r} \cdot \sin(u)^{2r} \ du &= \int_{[0, 1]} u^{k-2r} \cdot \sin(u)^{2r} \ du + \int_{[1, \infty)} u^{k-2r} \cdot \sin(u)^{2r} \ du \\
&\leq \int_{[0, 1]} \underbrace{u^k \cdot \left(\frac{\sin(u)}{u}\right)^{2r}}_{\leq 1} \ du + \int_{[1, \infty)} u^{-2}  \ du  \leq C_2
\end{align*}
with an absolute constant $C_2> 0$. This proves the claim. 
\end{proof}
The proof in \cite{lorentz_approximation_2005} proceeds by defining
\begin{equation*}
K_{m,r}(t) \defeq L_{m',r}(t), \quad m' = \left\lfloor\frac{m}{r} \right\rfloor + 1.
\end{equation*}
\Cref{prop:lorentz_bound_1} shows for $k \leq 2r-2$ that 
\begin{equation*}
\int_{[0,\pi]} t^k K_{m,r} (t) \ dt \leq \exp( C\cdot r) \cdot (m')^{-k}.
\end{equation*}
Since $m' \geq \frac{m}{r}$ we infer
\begin{equation}\label{eq:rauteraute}
\int_{[0,\pi]} t^k K_{m,r} (t) \ dt \leq \exp( C\cdot r) \cdot \left(\frac{r}{m}\right)^{k} \leq \exp(C \cdot r) \cdot r^k \cdot m^{-k}
\end{equation}
with an absolute constant $C > 0$.

We can now quantify the constant appearing in \cite[Theorem 4.3]{lorentz_approximation_2005}.
\begin{theorem}[{cf. \cite[Theorem 4.3]{lorentz_approximation_2005}}]\label{thm:lorentz_1d}
Let $k,m \in \NN$ and $f \in C^k_{2\pi}(\RR; \RR)$. Let 
\begin{equation*}
\omega(f^{(k)}, 1/m) \defeq \underset{x \in \RR, \vert t \vert \leq 1/m}{\max} \vert f^{(k)}(x+t)- f^{(k)}(x)\vert.
\end{equation*}
Then it holds 
\begin{equation*}
E_m^1 (f) \leq (\exp(C \cdot k) \cdot k^k) \cdot m^{-k} \cdot \omega(f^{(k)}, 1/m). 
\end{equation*}
Here, we recall that $E_m^1 (f)$ denotes the best possible approximation error when approximating $f$ using trigonometric polynomials of degree $m$; see \Cref{eq:mintrigo}. 
\end{theorem}
\begin{proof}
We follow the proof of \cite[Theorem 4.3]{lorentz_approximation_2005}. Take $r = k+1$ and define 
\begin{equation*}
I_m(x) \defeq - \int_{[-\pi, \pi]} K_{m,r}(t) \sum_{\ell =1}^{k+1} (-1)^\ell \binom{k+1}{\ell} f(x + \ell t) \ dt.
\end{equation*}
Then it is shown in the proof of \cite[Theorem 4.3]{lorentz_approximation_2005} that $I_m$ is a trigonometric polynomial of degree at most $m$ and that 
\begin{equation*}
\vert f(x) - I_m(x) \vert \leq 2 \cdot \omega_{k+1}(f, 1/m) \cdot \int_{[0,\pi]} (mt+1)^{k+1} K_{m,r}(t) \ dt.
\end{equation*}
Here, $\omega_{k+1}(f, 1/m)$ denotes the \emph{modulus of smoothness} of $f$ as defined on \cite[p.~47]{lorentz_approximation_2005}. The integral can be bounded via
\begin{align*}
&\norel\int_{[0,\pi]} (mt+1)^{k+1} K_{m,r}(t) \ dt \\
&= \int_{[0, 1/m]} (\underbrace{mt+1}_{\leq 2})^{k+1} K_{m,r}(t) \ dt + \int_{[1/m, \pi]}(\underbrace{mt+1}_{\leq 2mt})^{k+1} K_{m,r}(t) \ dt \\
&\leq 2^{k+1} \cdot \underbrace{\int_{[-\pi, \pi]}K_{m,r}(t) \ dt}_{= 1} + 2^{k+1}m^{k+1} \cdot \int_{[0,\pi]} \ t^{k+1}K_{m,r}(t) \ dt \\
\overset{\eqref{eq:rauteraute}}&{\leq} 2^{k+1} + 2^{k+1} m ^{k+1}\exp(C_1 \cdot r) \cdot (k+1)^{k+1} \cdot m^{-(k+1)} \overset{r\leq 2k}{\leq} \exp(C \cdot k ) \cdot k^{k} 
\end{align*}
with absolute constants $C,C_1 > 0$. Since $\omega_{k+1}(f, 1/m) \leq m^{-k} \cdot \omega(f^{(k)}, 1/m)$ follows from \cite[Equation~3.6(5)]{lorentz_approximation_2005}, the claim is shown.
\end{proof}
Therefore, we can bound the constant appearing in \cite[Theorem 4.3]{lorentz_approximation_2005} by $\exp(C \cdot k) \cdot k^k$. It remains to deal with the approximation of \emph{multivariate} periodic functions by \emph{multivariate} trigonometric polynomials which is contained in \cite[Theorem 6.6]{lorentz_approximation_2005}.

\begin{theorem}[{cf. \cite[Theorem 6.6]{lorentz_approximation_2005}}] \label{thm:const_lorentz_bound}
Let $s,k \in \NN$ and $f \in C^k_{2\pi}(\RR^s; \RR)$. Let $\omega_j$ denote the modulus of continuity of $\frac{\partial^k f}{ \partial x_j^k}$ for $j=1,...,s$. Then, with $E_m^s$ as introduced in \Cref{eq:mintrigo}, it holds
\begin{equation*}
E^s_m(f) \leq \exp(C \cdot ks) \cdot k^k \cdot m^{-k} \sum_{j=1}^s \omega_j(1/m),
\end{equation*}
with an absolute constant $C>0$.
\end{theorem}
\begin{proof}
We follow the proof of \cite[Theorem 6.6]{lorentz_approximation_2005} with $p_j = k$ and $n_j = m$ for every index $j = 1,...,s$. For $j = 1,..., s+1$ define the set $\mathcal{T}_j$ consisting of all functions $g \in C_{2\pi}(\RR^s; \RR)$ that are a trigonometric polynomial in $x_\ell$ of degree at most $m$ for $\ell < j$; in $x_\ell$ for $\ell \geq j$ they should have continuous partial derivatives $\frac{\partial^p g}{ \partial x_\ell^p}$ for $0\leq p \leq k$; the modulus of continuity of $\frac{\partial^{k} g}{\partial x_\ell^k}$ should not exceed $2^{K_j}\omega_j$, where
\begin{equation*}
K_j = (j-1)(k+1) \leq 2ks \quad \text{for } j>1 \text{ and } K_1=1\leq 2ks.
\end{equation*} 
Then it is shown that if $j \in \{1,...,s\}$ and $f_j \in \mathcal{T}_j$ there exists a function $f_{j+1} \in \mathcal{T}_{j+1}$ for which
\begin{equation*}
\Vert f_j - f_{j+1} \Vert_{L^\infty(\RR^s; \RR)} \leq \exp(C_1 \cdot k) \cdot k^k \cdot m^{-k} \cdot 2^{2ks} \cdot \omega_j(1/m) \leq \exp(C_2 \cdot ks) \cdot k^k \cdot m^{-k} \cdot \omega_j(1/m)
\end{equation*}
for absolute constants $C_1, C_2 > 0$. This is an application of \Cref{thm:lorentz_1d}. Hence, defining $f_1 \defeq f$, we see
\begin{equation*}
\Vert f - f_{s+1} \Vert_{L^\infty(\RR^s; \RR)} \leq \sum_{j=1}^{s} \Vert f_j - f_{j+1} \Vert_{L^\infty(\RR^s; \RR)} \leq \exp(C_2 \cdot ks) \cdot k^k \cdot m^{-k} \cdot \sum_{j=1}^s\omega_j(1/m). \qedhere
\end{equation*}
\end{proof}
Therefore, we have shown that the constant appearing in \cite[Theorem 6.6]{lorentz_approximation_2005} can be bounded from above by $\exp(C \cdot ks) \cdot k^k$. 

In the rest of this section, we discuss the operator norm of the operator defined in \Cref{star_operator}. In \Cref{star_operator} the closed graph theorem is used to show that the operator is bounded. However, the closed graph theorem does not provide any bound on the norm of the operator. Therefore, in order to quantify the operator norm, we need to apply a different technique, which is \emph{Faa di Bruno's formula}. This formula is a generalization of the chain rule to higher order derivatives.

\begin{theorem} \label{thm:faa}
Let $s,k \in \NN$. We define the operator
\begin{equation*}
T: \quad C^k([-1,1]^s ; \CC) \to C^k_{2\pi}([-\pi, \pi]^s;\CC), \quad (Tf)(x_1, ..., x_s) \defeq f(\cos(x_1), ..., \cos(x_s)).
\end{equation*}
Let $\aalpha \in \NN_0^s$ with $\vert \aalpha \vert \leq k$. Then, for any $f \in C^k([-1,1]^s;\CC)$, we have
\begin{equation*}
\Vert \partial^{\aalpha} (Tf) \Vert_{L^\infty([\pi, \pi]^s;\CC)} \leq \prod_{j=1}^s {\aalpha}_j^{{\aalpha}_j} \cdot \Vert f \Vert_{C^{\vert \aalpha \vert}([-1,1]^s)}.
\end{equation*}
\end{theorem}
\begin{proof}
The proof is by induction over $s$. The case $s=1$ is an application of Faa di Bruno's formula: We can write $Tf = f \circ g$ with $g(x) = \cos(x)$. We then take $\ell \in \NN_0$ with $\ell \leq k$ and some $x \in [-\pi, \pi]$. The set partition version of Faa di Bruno's formula (see for instance \cite[p.~219]{johnson_curious_2002}) then yields
\begin{equation*}
\left\vert (f \circ g)^{(\ell)}(x)\right\vert \leq \sum_{\pi \in \Pi_\ell} \left(\left\vert f^{(\left\vert\pi\right\vert)}(g(x))\right\vert \cdot \prod_{B \in \pi} \left\vert g^{(\vert B \vert)}(x)\right\vert\right).
\end{equation*}
Here, $\Pi_\ell$ denotes the set of all partitions of the set $\{1, ..., \ell\}$. Since all derivatives of $g$ are bounded by $1$ in absolute value and $\vert \pi \vert \leq \ell$ for every partition $\pi\in \Pi_\ell$ we get
\begin{equation*}
\Vert (f \circ g)^{(\ell)} \Vert_{L^\infty([-\pi, \pi];\CC)} \leq \vert \Pi_\ell \vert \cdot \Vert f \Vert_{C^\ell ([-1,1]; \CC)}.
\end{equation*}
The number $\vert \Pi_\ell \vert$ is the number of possible partitions of the set $\{1,..., \ell\}$ and is the so-called $\ell$-th \emph{Bell number}. It can be bounded from above by $\ell^\ell$ (see \cite[Theorem 2.1]{berend2010improved}). This proves the case $s=1$.

We now assume that the claim holds for an arbitrary but fixed $s \in \NN$. Take $\aalpha \in \NN_0^{s+1}$ with $\vert \aalpha \vert \leq k$. We decompose $\aalpha = (\aalpha', \aalpha_{s+1})$ with $\aalpha' \in \NN_0^s$. For a fixed variable $y_{s+1} \in [-1,1]$, we define 
\begin{equation*}
f_{y_{s+1}}(y_1,..., y_s) \defeq f(y_1, ..., y_s, y_{s+1}) \quad \text{for} \quad (y_1, ..., y_s) \in [-1,1]^s.
\end{equation*}
We denote $g(x_1,...,x_{s+1}) \defeq (\cos(x_1),..., \cos(x_{s+1}))$, $g_s(x_1,..., x_s) \defeq (\cos(x_1),..., \cos(x_s))$ and $\theta(x_{s+1}) \defeq \cos(x_{s+1})$. For every $(x_1,..., x_{s+1}) \in [-\pi,\pi]^{s+1}$ it then holds
\begin{equation*}
(f \circ g)(x_1, ..., x_{s+1}) = \left(f_{\theta(x_{s+1})} \circ g_s\right)(x_1,..., x_s).
\end{equation*}
We now differentiate $f \circ g$ with respect to the multiindex $\aalpha$ and get
\begin{align*}
\left[\partial^{\aalpha}(f \circ g) \right] (x_1,...,x_{s+1}) &= \frac{\partial^{\aalpha_{s+1}}}{\partial x_{s+1}^{\aalpha_{s+1}}}\left[\partial^{\aalpha'} \left(f_{\theta(x_{s+1}) } \circ g_s\right)(x_1,..., x_s)\right] \\
&= (h_{x_1,...,x_s} \circ \theta)^{(\aalpha_{s+1})}(x_{s+1})
\end{align*}
where we define
\begin{equation*}
h_{x_1,...,x_s}(y_{s+1})\defeq \partial^{\aalpha'} \left(f_{y_{s+1}} \circ g_s\right)(x_1,...,x_s) \quad \text{for} \quad (x_1,...,x_s) \in [-\pi, \pi]^s  \text{ and }  y_{s+1} \in [-1,1].
\end{equation*}
Using the case $s=1$, we get
\begin{equation*}
\left\vert\left[\partial^{\aalpha}(f \circ g) \right] (x_1,...,x_{s+1}) \right\vert = \left\vert (h_{x_1,...,x_s} \circ \theta)^{(\aalpha_{s+1})}(x_{s+1})\right\vert \leq \aalpha_{s+1}^{\aalpha_{s+1}} \cdot \left\Vert h_{x_1,...,x_s}\right\Vert_{C^{\aalpha_{s+1}}([-1,1]; \CC)}
\end{equation*}
for any fixed $(x_1,...,x_s) \in [-\pi,\pi]^s$. 

It remains to bound $\left\Vert h_{x_1,...,x_s}\right\Vert_{C^{\aalpha_{s+1}}([-1,1]; \CC)}$. To this end, we fix $\ell \in \NN_0$ with $\ell \leq \aalpha_{s+1}$. We further denote
\begin{equation*}
F_{y_1,...,y_s}(y_{s+1}) \defeq f(y_1,...,y_s,y_{s+1}) \quad \text{for} \quad (y_1, ..., y_{s+1}) \in [-1,1]^{s+1}.
\end{equation*} 
For arbitrary $(x_1,..., x_s) \in [-\pi,\pi]^{s}$ and $y_{s+1} \in [-1, 1]$ we then see
\begin{align*}
h_{x_1,...,x_s}^{(\ell)}(y_{s+1}) = \partial^{\aalpha'} \left[(x_1,...,x_s) \mapsto F^{(\ell)} _{g_s(x_1,...,x_s)}(y_{s+1})\right]  =  \partial^{\aalpha'}\left[H_{y_{s+1}} \circ g_s \right](x_1,...,x_s)
\end{align*} 
where 
\begin{equation*}
H_{y_{s+1}}(y_1,...,y_s) \defeq F^{(\ell)}_{y_1,...,y_s}(y_{s+1}) \quad \text{for } (y_1,...,y_s) \in [-1,1]^s.
\end{equation*}
Hence, we see by induction that
\begin{align*}
\left\vert h_{x_1,...,x_s}^{(\ell)}(y_{s+1}) \right\vert &= \left\vert \partial^{\aalpha'}\left[H_{y_{s+1}} \circ g_s \right](x_1,...,x_s) \right\vert \overset{\text{IH}}{\leq} \prod_{j=1}^{s} \aalpha_j^{\aalpha_j} \cdot \left\Vert H_{y_{s+1}}\right\Vert_{C^{\vert \aalpha'\vert}([-1,1]^s; \CC)} \\
&\leq \prod_{j=1}^{s} \aalpha_j^{\aalpha_j} \cdot \Vert f \Vert_{C^{\vert \aalpha \vert}([-1,1]^{s+1} ; \CC)}
\end{align*}
as was to be shown.
\end{proof}

\begin{remark}\label{rem:multiindex}
For a multiindex $\aalpha \in \NN_0^s$ with $\vert \aalpha \vert \leq k$ we see
\begin{equation*}
\prod_{j=1}^{s} \aalpha_j^{\aalpha_j} \leq k^{\sum_{j=1}^s \aalpha_j} \leq k^k.
\end{equation*}
Hence, the norm of the operator introduced in \Cref{star_operator} can be bounded from above by $k^k$.
\end{remark}



\subsection{Proof of Theorem \ref{main_2}}
\label{ck_functions_reordered}
For any natural number $\ell \in \NN_0$, we denote by $T_\ell$ the $\ell$-th Chebyshev polynomial, satisfying
\begin{equation*}
    T_\ell\left(\cos(x)\right) = \cos(\ell x), \quad x \in \RR.
\end{equation*}
For a multi-index $\kk \in \NN_0^s$ we define
\begin{equation*}
    T_\kk (x) \defeq \prod_{j=1}^s T_{\kk_j}\left(x_j\right), \quad x \in [-1,1]^s.
\end{equation*}
The proof of \Cref{main_2} relies on the fact that $C^k$-functions can by approximated
at a certain rate using linear combinations of the $T_\kk$ (see \Cref{app: fourier_approx}).
We also refer to \Cref{fig:proof} for an illustration of the overall proof strategy
of \Cref{main_2}.

\medskip

\begin{proof}[Proof of \Cref{main_2}]
    Choose $M \in \NN$ as the largest integer for which $(16M-7)^{2n} \leq m$, where we assume without loss of generality that $9^{2n} \leq m$, which can be done by choosing $\sigma_j = 0$ for all $j \in \{1,...,m\}$ for $m < 9^{2n}$, at the cost of possibly enlarging $c$. First we note that by the choice of $M$ the inequality
    \begin{equation*}
        m \leq (16M + 9)^{2n}
    \end{equation*}
    holds true. Since $16M +9 \leq 25M$, we get $m \leq 25^{2n} \cdot M^{2n}$ or equivalently
    \begin{equation}
    \label{M_bound}
        \frac{m^{1/2n}}{25} \leq M.
    \end{equation}
    According to \Cref{app: fourier_approx} we choose a constant $c_1 = c_1(n,k)$ with the property that for any function $f \in C^k \left( [-1,1]^{2n}; \RR\right)$ there exists a polynomial 
\begin{equation*}
	P = \sum_{0 \leq \kk \leq 2M-1} \mathcal{V} _\kk^M (f) \cdot T_\kk
\end{equation*}
 of coordinatewise degree at most $2M-1$ satisfying
    \begin{equation*}
        \left\Vert f - P \right\Vert_{L^\infty \left([-1,1]^{2n}; \RR\right)} \leq \frac{c_1}{M^k} \cdot \left\Vert f\right\Vert_{C^k \left([-1,1]^{2n}; \RR\right)}.
    \end{equation*}
    Furthermore, according to \Cref{app: fourier_approx}, we choose a constant $c_2 = c_2(n)$, such that the inequality
    \begin{equation*}
        \sum_{0 \leq \kk \leq 2M-1} \left\vert \mathcal{V}_\kk^M (f) \right\vert \leq c_2 \cdot M^n \cdot \Vert f \Vert_{L^\infty([-1,1]^{2n}; \RR)}\leq c_2 \cdot M^{n} \cdot \left\Vert f\right\Vert_{C^k \left([-1,1]^{2n}; \RR\right)}    
    \end{equation*}
    holds for all $f \in C^k \left( [-1,1]^{2n} ; \RR\right)$. The final constant is then defined to be
    \begin{equation*}
        c= c(n,k) \defeq \sqrt{2} \cdot 25^k \cdot \left(c_1 + c_2\right).
    \end{equation*}
    Fix $\kk \leq 2M-1$. Since $T_\kk$ is a polynomial of componentwise degree less or equal to $2M-1$ with $\varphi_n$ as in \eqref{isomorphism_intro}, we have a representation 
    \begin{equation*}
        \left(T_\kk \circ \varphi_n^{-1} \right)(z) = \underset{\elll^1, \elll^2 \leq 2M-1}{\sum_{\elll^1, \elll^2 \in \NN_0^n}} a_{\elll^1, \elll^2}^\kk \prod_{t=1}^n \RE \left(z_t\right)^{\elll^1_t} \IM \left(z_t\right)^{\elll^2_t}
    \end{equation*}
    with suitably chosen coefficients $a_{\elll^1, \elll^2}^\kk \in \CC$. 
    By using the identities $\RE\left(z_t\right) = \frac{1}{2}\left(z_t + \overline{z_t}\right)$ and also $\IM\left(z_t\right) = \frac{1}{2i}\left( z_t - \overline{z_t}\right)$ we can rewrite $T_\kk \circ \varphi_n^{-1}$ into a complex polynomial in $z$ and $\overline{z}$, i.e.,
    \begin{equation*}
        \left(T_\kk \circ \varphi_n^{-1}\right)\left(z\right) = \underset{\elll^1, \elll^2 \leq 4M - 2}{\sum_{\elll^1, \elll^2 \in \NN_0^{n}}} b_{\elll^1, \elll^2}^\kk z^{\elll^1} \overline{z}^{\elll^2}
    \end{equation*}
    with complex coefficients $b_{\elll^1, \elll^2}^\kk \in \CC$.
    Using \Cref{main_1}, we choose $\rho_1, ..., \rho_m \in \CC^n$ and $b \in \CC$, such that for any polynomial $P \in \left\{ T_\kk \circ \varphi_n^{-1} : \ \kk \leq 2M-1\right\} \subseteq \mathcal{P}_{4M-2}^n$ there are coefficients $\sigma_1(P), ..., \sigma_m(P) \in \CC$, such that
    \begin{equation}
    \label{gp}
        \left\Vert g_P - P \right\Vert_{L^\infty \left(\Omega_n; \CC\right)} \leq M^{-k-n}, 
    \end{equation}
    where 
    \begin{equation*}
        g_P \defeq \sum_{t=1}^m \sigma_t(P) \phi \left(\rho_t^T z + b\right).
    \end{equation*}
    Note that here we implicitly use the bound $(4\cdot (4M-2) + 1)^{2n} \leq m$. We are now going to show that the chosen constant and the chosen vectors $\rho_t$ have the desired property.

    Let $f \in C^k\left(\Omega_n; \CC\right)$. By splitting $f$ into real and imaginary part, we write $f = f_1 + i \cdot f_2$ with $f_1 , f_2 \in C^k \left(\Omega_n ; \RR\right)$. For the following, fix $ j \in \{1,2\}$ and note that $f_j \circ \varphi_n \in C^k\left([-1,1]^{2n}; \RR\right)$. By choice of $c_1$, there exists a polynomial $P$ with the property
    \begin{equation*}
        \left\Vert f_j \circ \varphi_n - P \right\Vert_{L^\infty \left([-1,1]^{2n}; \RR \right)} \leq \frac{c_1}{M^k} \cdot \left\Vert f_j \circ \varphi_n\right\Vert_{C^k \left([-1,1]^{2n}; \RR\right)}
    \end{equation*}
    or equivalently
    \begin{equation}
    \label{bound_1}
        \left\Vert f_j  - P \circ \varphi_n^{-1}\right\Vert_{L^\infty \left(\Omega_n; \RR\right)} \leq \frac{c_1}{M^k} \cdot \left\Vert f_j \right\Vert_{C^k \left(\Omega_n; \RR\right)},
    \end{equation}
    where $P \circ \varphi_n^{-1}$ can be written in the form 
    \begin{equation*}
        \left(P \circ \varphi_n^{-1}\right)\left(z\right) = \sum_{0 \leq \kk \leq 2M-1} \mathcal{V}_\kk^M\left(f_j \circ \varphi_n\right) \cdot \left(T_\kk \circ \varphi_n^{-1}\right)(z).
    \end{equation*}
    We choose the function $g_{T_\kk \circ \varphi_n^{-1}}$ according to (\ref{gp}). Thus, writing
    \begin{equation*}
        g_j \defeq \sum_{0 \leq \kk \leq 2M-1} \mathcal{V}_\kk^M\left(f_j \circ \varphi_n\right) \cdot g_{T_\kk \circ \varphi_n^{-1}},
    \end{equation*}
    we obtain
    \begin{align}
    \label{bound_2}
        \left\Vert P \circ \varphi_n^{-1} - g_j \right\Vert_{L^\infty \left(\Omega_n ; \RR \right)} &\leq \sum_{0 \leq \kk \leq 2M-1} \left\vert \mathcal{V}_\kk^M \left(f_j \circ \varphi_n\right)\right\vert \cdot \underbrace{\left\Vert T_\kk \circ \varphi_n^{-1} - g_{T_{\kk} \circ \varphi_n^{-1}}\right\Vert_{L^\infty \left(\Omega_n ; \RR\right)}}_{\leq M^{-k-n}} \nonumber\\
        &\leq M^{-k-n} \cdot \sum_{0 \leq \kk \leq 2M-1} \left\vert \mathcal{V}_\kk^M \left(f_j \circ \varphi_n\right)\right\vert \nonumber \\
        &\leq \frac{c_2}{M^{k}} \left\Vert f_j \circ \varphi_n \right\Vert_{C^k \left([-1,1]^{2n}; \RR\right)} = \frac{c_2}{M^{k}} \left\Vert f_j  \right\Vert_{C^k \left(\Omega_n; \RR\right)}.
    \end{align}
    Combining (\ref{bound_1}) and (\ref{bound_2}), we see 
    \begin{equation*}
        \left\Vert f_j - g_j \right\Vert_{L^\infty \left(\Omega_n; \RR\right)} \leq \frac{c_1 + c_2}{M^k} \cdot \left\Vert f_j \right\Vert_{C^k \left(\Omega_n; \RR\right)} \leq \frac{c_1 + c_2}{M^k} \cdot \left\Vert f \right\Vert_{C^k \left(\Omega_n; \CC\right)}.
    \end{equation*}
    In the end, define 
    \begin{equation*}
        g \defeq g_1 + i \cdot g_2.
    \end{equation*}
    Since the vectors $\rho_t$ have been chosen fixed, it is clear that, after rearranging, $g$ has the desired form, i.e., $g = \sigma^T \Phi$ where $\Phi(z) = \left(\phi(\rho_t z + b)\right)_{t=1}^m$. Furthermore, one obtains the bound
    \begin{align*}
        \left\Vert f-g\right\Vert_{L^\infty \left(\Omega_n; \CC\right)} &\leq \sqrt{\left\Vert f_1 - g_1 \right\Vert^2_{L^\infty \left(\Omega_n; \RR\right)} + \left\Vert f_2 - g_2 \right\Vert^2_{L^\infty \left(\Omega_n; \RR\right)}} \\
        &\leq \frac{c_1 + c_2}{M^k} \cdot \sqrt{\left\Vert f \right\Vert^2_{C^k \left(\Omega_n; \CC\right)} + \left\Vert f \right\Vert^2_{C^k \left(\Omega_n; \CC\right)}} \\
        &\leq \frac{\sqrt{2} \cdot \left(c_1 + c_2\right)}{M^k} \cdot  \ \left\Vert f \right\Vert_{C^k \left(\Omega_n; \CC\right)}.
    \end{align*}
    Using (\ref{M_bound}), we see
    \begin{equation*}
        \left\Vert f-g\right\Vert_{L^\infty \left(\Omega_n; \CC\right)} \leq \frac{\sqrt{2} \cdot 25^k \cdot \left(c_1 + c_2\right)}{m^{k/2n}} \cdot  \ \left\Vert f \right\Vert_{C^k \left(\Omega_n; \CC\right)},
    \end{equation*}
    as desired. 

    The linearity and continuity of the maps $f \mapsto \sigma_j(f)$ (with respect to the $\Vert \cdot \Vert_{L^\infty}$-norm) follow easily from the fact that the map $f \mapsto \mathcal{V}_\kk^M(f)$ is a continuous linear functional for every multiindex $0 \leq \kk \leq 2M-1$.
\end{proof}

\begin{figure}[t]
\centering
\begin{tikzpicture}[
roundnode/.style={circle, draw=green!60, fill=green!5, very thick, minimum size=7mm},
squarednode/.style={rectangle, draw=black!60,  very thick, minimum size=5mm, text width=3cm,align=center},
sidenode/.style={rectangle, draw=red!60,  very thick, minimum size=5mm, text width=3cm,align=center}
]
\node[squarednode]      (partialdev)                              {Approximation of partial derivatives};
\node[squarednode]        (polyn)       [right=of partialdev] {Approximation of polynomials};
\node[squarednode]      (ckfunctions)       [right=of polyn] {Approximation of $C^k$-functions};
\node[sidenode]            (divided_differences)  [below=of partialdev] {Divided Differences};
\node[sidenode]            (phi)  [below=of polyn] {Wirtinger derivatives of $w \mapsto \phi(w^T z + b)$};
\node[sidenode]            (cheby)  [below=of ckfunctions] {Chebyshev polynomials};

\draw[->, very thick] (partialdev.east) -- (polyn.west);
\draw[->, very thick] (polyn.east) -- (ckfunctions.west);
\draw[->, thick] (divided_differences.north) -- (partialdev.south);
\draw[->, thick] (phi.north) -- (polyn.south);
\draw[->, thick] (cheby.north) -- (ckfunctions.south);
\end{tikzpicture}
\caption{Schematic for the proof of the main result (\Cref{main_2}). The first row shows the different steps of the proof and the second row indicates the main tools used.}
\label{fig:proof}
\end{figure}


% !TeX encoding = UTF-8
% !TeX spellcheck = en_US
% !TeX root = main_paper.tex

\section{Concrete examples of admissible functions}
\label{concrete_activation_functions}

In this section we want to show the admissibility of concrete activation functions that are commonly used when applying complex-valued neural networks to machine learning. 
\Cref{admissible} introduces a large class of admissible functions.
\begin{proposition}
\label{admissible}
    Let $\rho \in C^\infty(\RR; \RR)$ be non-polynomial and let $\psi : \CC \to \CC$ be defined as
    \begin{equation*}
        \psi(z) \defeq \rho(\RE(z)) \quad\text{resp.}\quad \psi(z) \defeq \rho(\IM(z)).
    \end{equation*}
    Then $\psi$ is admissible.
\end{proposition}
\begin{proof}
    Since $\psi$ depends only on the real resp. imaginary part of the input, we see directly from the definition of the Wirtinger derivatives that
    \begin{equation*}
        \wirt \psi(z) = \wirtq \psi(z)= \frac{1}{2} \rho' (\RE(z)) \quad \text{resp.} \quad \wirt \psi(z) = - \wirtq \psi(z)= -\frac{i}{2} \rho' \IM(z).
    \end{equation*}
    Hence we see for arbitrary $m, \ell \in \NN_0$ that
    \begin{equation*}
        \left\vert\wirt^m \wirtq^\ell \psi (z)\right\vert = \frac{1}{2^{m+\ell}} \left\vert\rho^{(m + \ell)} (\RE(z))\right\vert \quad\text{resp.} \quad  \left\vert\wirt^m \wirtq^\ell \psi (z)\right\vert = \frac{1}{2^{m+\ell}} \left\vert\rho^{(m + \ell)} (\IM(z))\right\vert.
    \end{equation*}
    Since $\rho$ is non-polynomial we can choose a real number $x$, such that all derivatives of $\rho$ in $x$ do not vanish (cf. for instance \cite[p. 53]{donoghue_distributions_1969}). Thus, all the Wirtinger derivatives $\wirt^m \wirtq^\ell \psi$ do not vanish for all complex numbers with real, resp. imaginary part $x$.
\end{proof}
In the following, we consider a special activation function, which has been proposed in \cite{arjovsky_unitary_2016}. Its representative power has already been discussed to some extent in \cite{caragea_quantitative_2022}.
\begin{definition}
    For $b \in (-\infty, 0)$ we define 
    \begin{equation*}
        \modrelu: \quad \CC \to \CC, \quad \modrelu(z) \defeq \left\{ \begin{matrix}(\vert z \vert + b)\frac{z}{\vert z \vert},& \vert z \vert + b \geq 0, \\ 0,& \mathrm{otherwise.}\end{matrix}\right.
    \end{equation*}
\end{definition}

\begin{theorem} \label{mod_relu_dev}
    Let $b \in (-\infty, 0)$. Writing $\sigma = \modrelu$ one has for every $z \in \CC$ with $\vert z \vert + b > 0$ the identity
    \begin{equation*}
        \left( \wirt^m \wirtq^\ell \sigma\right)(z) = \begin{cases}z + b\frac{z}{\vert z \vert},& m=\ell=0,\\1 + \frac{b}{2} \cdot \frac{1}{\vert z \vert},&m=1,\ell=0,\\ b \cdot q_{m,\ell} \cdot \frac{z^{\ell-m +1}}{\vert z \vert^{2\ell+1}},&m \leq \ell+1, \ell \geq 1, \\ b \cdot q_{m,\ell} \cdot \frac{\overline{z}^{m-\ell-1}}{\vert z \vert ^{2m-1}},& m \geq  \ell+1, m \geq 2\end{cases}
    \end{equation*}
    for every $m \in \NN_0$ and $\ell \in \NN_0$. Here, the numbers $q_{m,\ell}$ are non-zero and rational. Furthermore, note that all cases for choices of $m$ and $\ell$ are covered by observing that we can either have the case $m \geq \ell +1$ (where either $m \geq 2$ or $m=1, \ell = 0$) or $m \leq \ell + 1$ where the latter is again split into $\ell = 0$ and $\ell \geq 1$.
\end{theorem}
\begin{proof}
   We first calculate certain Wirtinger derivatives for $z \neq 0$. First note
    \begin{align*}
        \wirtq \left(\frac{1}{\vert z \vert^m}\right) &= \frac{1}{2}\left( \partial^{(1,0)}\left(\frac{1}{\vert z \vert^m}\right) + i \cdot \partial^{(0,1)}\left(\frac{1}{\vert z \vert^m}\right)\right) \\
        &= \frac{1}{2} \left(\left(- \frac{m}{2}\right)\frac{2\RE(z) + i\cdot 2\IM(z)}{\vert z \vert ^{m+2}}\right) \\
        &= - \frac{m}{2} \cdot \frac{z}{\vert z \vert^{m+2}}
    \end{align*}
    and similarly
    \begin{equation*}
        \wirt \left(\frac{1}{\vert z \vert^m}\right) = - \frac{m}{2} \cdot \frac{\overline{z}}{\vert z \vert^{m+2}}
    \end{equation*}
    for any $m \in \NN$. Using the product rule for Wirtinger derivatives, we see
    \begin{align}
    \label{wirt1}
         \wirtq \left(\frac{z^\ell}{\vert z \vert^m}\right) = \underbrace{\wirtq \left(z^\ell\right)}_{=0} \cdot \frac{1}{\vert z \vert^m} + z^\ell \cdot  \wirtq\left(\frac{1}{\vert z \vert^m}\right) = -\frac{m}{2} \cdot \frac{z^{\ell + 1}}{\vert z \vert^{m+2}}
    \end{align}
    for any $m \in \NN$ and $\ell \in \NN_0$ and furthermore
    \begin{align}
        \wirt \left(\frac{z^\ell}{\vert z \vert^m}\right) &= \wirt \left(z^\ell\right) \cdot \frac{1}{\vert z \vert^m} + z^\ell \cdot  \wirt\left(\frac{1}{\vert z \vert^m}\right) \nonumber\\
        &= \ell \cdot z^{\ell-1} \cdot \frac{1}{\vert z \vert ^m} - z^\ell \cdot  \frac{m}{2} \cdot \frac{\overline{z}}{\vert z \vert^{m+2}} \nonumber\\
        \label{wirt2}
        &=  \left(\ell - \frac{m}{2}\right) \cdot \frac{z^{\ell -1}}{\vert z \vert^m}
    \end{align}
    for $m, \ell \in \NN$, and finally
    \begin{align}
    \label{wirt3}
        \wirt \left(\frac{\overline{z}^\ell}{\vert z \vert^m}\right) &= \underbrace{\wirt \left(\overline{z}^\ell\right)}_{=0} \cdot \frac{1}{\vert z \vert^m} + \overline{z}^\ell \cdot  \wirt\left(\frac{1}{\vert z \vert^m}\right) = - \frac{m}{2} \cdot \frac{\overline{z}^{\ell + 1}}{\vert z \vert^{m+2}}
    \end{align}
    for $m \in \NN$ and $\ell \in \NN_0$. 

    Having proven those three identities, we can start with the actual computation. We first fix $m = 0$ and prove the claimed identity by induction over $\ell$. The case $\ell = 0$ is just the definition of the function and furthermore, we calculate
    \begin{align*}
        \wirtq \sigma (z) = \underbrace{\wirtq (z)}_{=0} + b \cdot \wirtq\left(\frac{z}{\vert z \vert}\right)  \overset{(\ref{wirt1})}{=} b \cdot \left(- \frac{1}{2}\right) \frac{z^2}{\vert z \vert^3},
    \end{align*}
    which is the claimed form. Then, using induction, we compute
    \begin{align*}
       \wirtq^{\ell + 1}\sigma(z) =  \wirtq \left(b \cdot q_{0,\ell} \cdot \frac{z^{\ell+1}}{\vert z \vert^{2\ell+1}}\right) \overset{(\ref{wirt1})}{=} b \cdot \underbrace{q_{0,\ell} \cdot \left(- \frac{2\ell+1}{2}\right)}_{=: q_{0, \ell+1}} \cdot \frac{z^{\ell+2}}{\vert z \vert^{2\ell+3}},
    \end{align*}
    so that the case $m = 0$ is complete. 

    Now we deal with the case $m \leq \ell+1$. The case $\ell = 0$ is proven by computing
    \begin{align*}
        \wirt \sigma (z) = \wirt (z) + b \cdot \wirt \left(\frac{z}{\vert z \vert}\right) \overset{\eqref{wirt2}}{=} 1 + b \cdot \frac{1}{2} \cdot \frac{1}{\vert z \vert}
    \end{align*}
    so we can assume $\ell > 0$. Since we already dealt with the case $m = 0$ we can inductively assume the claim to be true for a fixed $m \leq \ell$. Then we compute
    \begin{align*}
        \left( \wirt^{m+1} \wirtq^\ell \sigma\right)(z) &= \wirt \left( b \cdot q_{m,\ell} \cdot \frac{z^{\ell-m+1}}{\vert z \vert^{2\ell+1}}\right) \overset{(\ref{wirt2})}{=} b \cdot \underbrace{q_{m,\ell} \cdot \left(-m + \frac{1}{2}\right)}_{=: q_{m+1,\ell}} \cdot \frac{z^{\ell-m}}{\vert z \vert^{2\ell+1}},
    \end{align*}
    which is the desired form. Note that (\ref{wirt2}) is indeed applicable because $ \ell - m +1 \geq 1$. 

    Finally, we consider the case where $m \geq \ell+1$ and $m \geq 2$. The case $m = \ell+1$ has already been shown. Using induction, we see
    \begin{align*}
        \left( \wirt^{m+1} \wirtq^\ell\sigma\right)(z) &= \wirt \left( \delta_{(m, \ell) = (1,0)} + b \cdot q_{m,\ell} \cdot \frac{\overline{z}^{m-\ell-1}}{\vert z \vert ^{2m-1}}\right) \overset{(\ref{wirt3})}{=} b \cdot \underbrace{q_{m,\ell} \cdot \left(- m + \frac{1}{2}\right)}_{=: q_{m + 1,\ell}}\cdot \frac{\overline{z}^{m - \ell}}{\vert z \vert ^{2m + 1}}, 
    \end{align*}
    so the proof is complete.
\end{proof}
From \Cref{mod_relu_dev} we can now deduce the admissibility of the modReLU.
\begin{corollary}
Let $b \in (-\infty, 0)$ and $z \in \CC$ with $\vert z \vert >  -b $. Then we have 
\begin{equation*}
    \wirt^m \wirtq^\ell \modrelu (z) \neq 0
\end{equation*}
for every $m, \ell \in \NN_0$. In particular, $\modrelu$ is admissible.
\end{corollary}
\begin{proof}
This follows from \Cref{mod_relu_dev} by noting that if $\vert z \vert > -b$ we have in particular $\vert z \vert > -b/2$ and $\vert z \vert > 0$.
\end{proof}


The second activation function that we want to analyze is proposed in \cite{virtue_better_2017} where it is used for MRI fingerprinting to achieve significant results with complex-valued neural networks, outperforming their real-valued counterparts. Therefore it shall be studied, if this activation function is also admissible.
\begin{definition}
    The function
    \begin{equation*}
        \card:\ \CC \to \CC, \ f(z) \defeq  \frac{1}{2}(1+\mathrm{cos}(\sphericalangle z ))z
    \end{equation*}
    is called \emph{complex cardioid function}. Here, $\sphericalangle z = \theta \in \RR$ denotes the polar angle of a complex number $z = re^{i\theta}$, where we define $\sphericalangle 0 := 0$. Even though $\theta$ is only well-defined modulo $2\pi$, this is not an issue here, since $\cos(\theta)$ is $2\pi$-periodic. Furthermore, $\cos(\theta)= \frac{\RE(z)}{\vert z \vert}$ for $z \neq 0$.
\end{definition}
\begin{theorem} \label{thm: cardioid}
    For any $z \in \CC$ with $z \neq 0$ and any $m, \ell \in \NN_0$ we have
    \begin{equation*}
        \wirt^m \wirtq^\ell \card (z) = \begin{cases}
        \frac{1}{2}z +\frac{1}{4} \frac{z^2}{\vert z \vert} + \frac{\vert z \vert}{4},& m=\ell=0,\\
            a_{m,\ell} \frac{z^{\ell - m}}{\vert z \vert ^{2\ell -1}} + b_{m, \ell} \frac{z^{\ell + 2 -m}}{\vert z \vert ^{2\ell + 1}}, & m \leq \ell \neq 0, \\
            \frac{1}{2} + \frac{1}{8} \cdot \frac{\overline{z}}{\vert z \vert} + \frac{3}{8} \cdot \frac{z}{\vert z \vert},& m = \ell + 1 = 1,\\
            a_{m,\ell} \frac{\overline{z}}{\vert z \vert ^{2\ell +1}} + b_{m, \ell} \frac{z}{\vert z \vert ^{2\ell + 1}}, & m = \ell + 1 > 1,\\
            a_{m,\ell} \frac{\overline{z}^{m - \ell}}{\vert z \vert ^{2m -1}} + b_{m, \ell} \frac{\overline{z}^{m - \ell - 2 }}{\vert z \vert ^{2m-3}}, & m \geq \ell + 2.
        \end{cases} 
    \end{equation*}
    Here, the numbers $a_{m, \ell}$ and $b_{m,\ell}$ are again non-zero and rational. Furthermore, note that all cases for possible choices of $m$ and $\ell$ are covered: The case $m \leq \ell$ is split into $\ell = 0$ and $ \ell \neq 0$. The case $m = \ell + 1$ is split into $m = 1$ and $m > 1$. Then, the case $m \geq \ell +2$ remains. 
\end{theorem}
\begin{proof}
    For the following we always assume $z \in \CC$ with $z \neq 0$. Then we can simplify $\cos(\sphericalangle z) = \frac{\RE(z)}{\vert z \vert}$, so we can rewrite
    \begin{equation*}
        \card(z) = \frac{1}{2}\left(1 + \frac{\RE(z)}{\vert z \vert}\right)z = \frac{1}{2} z + \frac{1}{4}\frac{\left(z+\overline{z} \right)z}{\vert z \vert} = \frac{1}{2}z +\frac{1}{4} \frac{z^2}{\vert z \vert} + \frac{\vert z \vert}{4}.
    \end{equation*}
    First, we compute
    \begin{equation}
\label{wirtquerbetrag}
        \wirtq \left(\vert z \vert\right) = \frac{1}{2} \left( \frac{1}{2} \frac{2\RE(z)}{\vert z \vert} + \frac{i}{2}\frac{2\IM(z)}{\vert z \vert}\right) = \frac{1}{2} \frac{z}{\vert z \vert}
    \end{equation}
    and similarly
    \begin{equation}
\label{wirtbetrag}
        \wirt\left( \vert z \vert\right) = \frac{1}{2} \frac{\overline{z}}{\vert z \vert}.
    \end{equation}
    We deduce
    \begin{align*}
        \wirtq \card (z) \overset{(\ref{wirt1}),(\ref{wirtquerbetrag})}{=} \underbrace{\frac{1}{4} \cdot \left(- \frac{1}{2} \right)}_{=: b_{0, 1}} \cdot  \frac{z^3}{\vert z \vert^3} + \underbrace{\frac{1}{8}}_{=: a_{0,1}} \cdot \frac{z}{\vert z \vert}.
    \end{align*}
    Using induction, we derive
    \begin{align*}
        \wirtq^{\ell + 1} \card(z) &= a_{0, \ell }  \wirtq \left(\frac{z^{\ell }}{\vert z \vert ^{2\ell -1}}\right) + b_{0, \ell} \wirtq\left(\frac{z^{\ell + 2 }}{\vert z \vert^{2\ell + 1}}\right) \\
        \overset{(\ref{wirt1})}&{=} \underbrace{a_{0, \ell} \cdot \left( - \frac{2\ell - 1}{2}\right)}_{=: a_{0, \ell + 1}} \cdot \frac{z^{\ell + 1}}{\vert z \vert^{2\ell + 1}} + \underbrace{b_{0, \ell} \cdot \left(-\frac{2\ell + 1}{2}\right)}_{=: b_{0, \ell + 1}} \cdot \frac{z^{\ell + 3}}{\vert z \vert^{2\ell + 3}},
    \end{align*}
    so the claim has been shown if $m = 0$. If we now fix any $\ell \in \NN$ and assume that the claim holds true for some $m \in \NN_0$ with $m < \ell$, we get
    \begin{align*}
        \wirt^{m+1}\wirtq^\ell \card(z) &= a_{m, \ell} \wirt \left(\frac{z^{\ell - m}}{\vert z \vert^{2\ell - 1}}\right) + b_{m, \ell} \wirt \left( \frac{z^{\ell + 2 - m}}{\vert z \vert^{2\ell + 1}}\right) \\
        \overset{(\ref{wirt2})}&{=} \underbrace{a_{m,\ell} \cdot \left( \frac{1}{2} - m\right)}_{=: a_{m+1, \ell}} \cdot \frac{z^{\ell - m -1}}{\vert z \vert^{2\ell - 1}} + \underbrace{b_{m, \ell} \cdot \left( \frac{3}{2} - m\right)}_{=: b_{m+1, \ell}}\cdot \frac{z^{\ell + 2 - m - 1}}{\vert z \vert^{2\ell + 1}},
    \end{align*}
    so the claim holds true if $m \leq \ell$. 

    The case $m = \ell + 1$ is split into the case $m= 1$ and $m > 1$. If $m = 1$, then $\ell = 0$ and we compute
    \begin{equation*}
        \wirt \card (z) \overset{(\ref{wirt2}), (\ref{wirtbetrag})}{=} \frac{1}{2} + \frac{1}{4}\left(2 - \frac{1}{2}\right) \cdot \frac{z}{\vert z \vert } + \frac{1}{8} \cdot \frac{\overline{z}}{\vert z \vert}.
    \end{equation*}
    If $m > 1$ we get
    \begin{align*}
        \wirt^{\ell + 1}\wirtq^{\ell} \card (z) &= a_{\ell, \ell} \wirt \left(\frac{1}{\vert z \vert^{2\ell - 1}}\right) + b_{\ell, \ell} \cdot \wirt \left(\frac{z^{ 2 }}{\vert z \vert^{2\ell + 1}}\right) \\
        \overset{\eqref{wirt2},(\ref{wirt3})}&{=} \underbrace{a_{\ell, \ell} \cdot \left(-\frac{2\ell - 1}{2} \right)}_{=: a_{\ell + 1, \ell}} \cdot \frac{\overline{z}}{\vert z \vert^{2\ell + 1}} +\underbrace{ b_{\ell, \ell} \cdot \left(2 - \frac{2\ell + 1}{2}\right)}_{=: b_{\ell + 1, \ell}} \cdot \frac{z}{\vert z \vert^{2\ell + 1}}.
    \end{align*}
    Next, we deal with the case $m = \ell + 2$: Here we see
\begin{align*}
	\wirt^{\ell + 2} \wirtq^{\ell} \card (z) &= \wirt \left(\frac{1}{2} \delta_{(m,\ell) = (1, 0)}  + a_{\ell +1, \ell} \frac{ \overline{z}}{\vert z \vert^{2\ell + 1}} + b_{\ell + 1,\ell} \frac{z}{\vert z \vert^{2\ell + 1}}\right) \\
\overset{(\ref{wirt2}),(\ref{wirt3})}&{=} \underbrace{a_{\ell + 1, \ell} \cdot \left(-\frac{2\ell + 1}{2}\right)}_{=: a_{\ell +2, \ell}} \cdot \frac{\overline{z}^2}{\vert z \vert^{2\ell + 3}} + \underbrace{b_{\ell + 1, \ell}\cdot \left(1 - \frac{2\ell + 1}{2}\right)}_{=: b_{\ell + 2, \ell}}\cdot \frac{1}{\vert z \vert^{2\ell + 1}}.
\end{align*}
If we assume the claim to be true for a fixed $m \geq \ell + 2$ we get 
\begin{align*}
\wirt^{m + 1} \wirtq^\ell \card(z) &= a_{m, \ell} \cdot \wirt \left(\frac{\overline{z}^{m - \ell}}{\vert z \vert^{2m - 1}}\right) + b_{m, \ell} \cdot \wirt\left( \frac{\overline{z}^{m - \ell - 2}}{\vert z \vert^{2m - 3}}\right) \\
\overset{(\ref{wirt3})}&{=} \underbrace{a_{m,\ell} \cdot \left(- \frac{2m - 1}{2}\right)}_{=: a_{m + 1, \ell}} \cdot \frac{\overline{z}^{m + 1 - \ell}}{\vert z \vert^{2m + 1}} + \underbrace{b_{m,\ell} \cdot \left( - \frac{2m - 3}{2}\right)}_{ =: b_{m + 1, \ell}} \cdot \frac{\overline{z}^{m - \ell - 1}}{\vert z \vert^{2m - 1}}.
\end{align*}
Hence, using induction, we have proven the claimed identity.
\end{proof}
The statement regarding the admissibility of the complex cardioid is formulated in the following corollary.
\begin{corollary}
    For every $z \in \CC$ with $z \notin \RR \cup i \RR$ and every $m, \ell \in \NN_0$ we have 
    \begin{equation*}
        \wirt^m \wirtq^\ell \card (z)\neq 0.
    \end{equation*}
    In particular, $\card$ is admissible.
\end{corollary}
\begin{proof}
     For the following, let $z  \in \CC$ with $\RE(z), \IM(z) > 0$.
    Using the definition of $\card$, we see 
    \begin{equation*}
        \card(z) = 0 \Leftrightarrow z = 0 \text{ or } \cos(\sphericalangle z ) = - 1 \Leftrightarrow z \in \RR^{\leq 0},
    \end{equation*}
which is never fulfilled. 
    In the case $m \leq \ell \neq 0$ we see that the relation
    \begin{align*}
        \wirt^m \wirtq^\ell \card (z) = 0 \Leftrightarrow a_{m, \ell} + b_{m, \ell}\frac{z^2}{\vert z \vert ^2} = 0 \Leftrightarrow z^2 = - \vert z \vert^2 \cdot \frac{a_{m, \ell}}{b_{m, \ell}} \Rightarrow z^2 \in \RR \Leftrightarrow z \in \RR \cup i\RR
    \end{align*}
    holds, which is impossible by assumption.  

    For the case $m = \ell + 1 = 1$, consider
    \begin{equation*}
        \wirt \card(z) = 0 \Leftrightarrow \frac{1}{8} \cdot \overline{z} + \frac{3}{8} \cdot z = -\frac{\vert z \vert}{2} \Rightarrow \frac{3}{8}\IM(z) - \frac{1}{8} \IM(z) = 0 \Leftrightarrow \IM(z) = 0,
    \end{equation*}
which does not hold since $z \notin \RR$.

    For $m = \ell + 1 > 1$ we see by considering the real- and imaginary parts that
    \begin{align*}
        \wirt^{\ell + 1}\wirtq^\ell \card (z) = 0 &\Leftrightarrow a_{m,\ell}\overline{z} + b_{m, \ell} z = 0 \overset{\RE(z), \IM(z) \neq 0}{\Leftrightarrow} a_{m,\ell} + b_{m, \ell}=0 \text{ and } -a_{m, \ell} + b_{m, \ell}=0 \\ &\Leftrightarrow a_{m,\ell} = b_{m,\ell}=0,
    \end{align*}
which is not fulfilled, since $a_{m,\ell} \neq 0 \neq b_{m,\ell}$ by \Cref{thm: cardioid}. 

    Therefore it remains the case $m \geq \ell + 2$. But here we easily see
    \begin{align*}
        \wirt^m \wirtq^{\ell}\card(z) = 0 \Leftrightarrow a_{m,\ell} \frac{\overline{z}^2}{\vert z \vert^2} + b_{m,\ell} = 0 \Rightarrow \overline{z}^2 \in \RR \Leftrightarrow z \in \RR \cup i\RR,
    \end{align*}
which is not fulfilled by assumption.

    Since all cases have been considered, the claim follows.
\end{proof}

\section{Proof of Theorem \ref{thm:opti_conti}}
\label{optimality_section}

We first state a theorem which establishes a lower bound for the approximation of $C^k$-functions by continuous functions that can be parametrized by a certain number of parameters. The proof, which is already contained in \cite{devore_optimal_1989} in a slightly different setting, is deferred to \Cref{prelim_opti} (\Cref{app: devore_real}).
\begin{theorem} \label{thm: devore_real}
Let $s,k \in \NN$. Then there exists a constant $c = c(s,k)>0$ with the following property:  For any $m \in \NN$ and any map $\overline{a} : C^k ([-1,1]^s; \RR) \to \RR^m$ that is continuous with respect to some norm on $C^k([-1,1]^s; \RR)$ and any (possibly discontinuous) map $M  : \RR^m \to C([-1,1]^s ; \RR)$, we have
\begin{equation*}
  \underset{\Vert f \Vert _{C^k ([-1,1]^s ; \RR)} \leq 1}{\underset{f \in C^k([-1,1]^s ; \RR)}{\sup}}
    \Vert f - M (\overline{a}(f)) \Vert_{L^\infty ([-1,1]^s ; \RR)}
  \geq c \cdot m^{-k/s}.
\end{equation*}
\end{theorem}
Using this theorem, we can now prove \Cref{thm:opti_conti}.
\medskip 
\renewcommand*{\proofname}{Proof of \Cref{thm:opti_conti}}
\begin{proof}
Let $\overline{a}: C^k(\Omega_n ; \CC) \to \CC^m$ be any map that is continuous with respect to some norm $\Vert \cdot \Vert_V$ on $C^k(\Omega_n; \CC)$, and let $M: \CC^m \to C(\Omega_n;\CC)$ be arbitrary. With $\varphi_n, \varphi_m$ defined as in Equation \eqref{isomorphism_intro}, let
\begin{equation*}
\tilde{a}: \quad C^k ([-1,1]^{2n}; \RR) \to \RR^{2m}, \quad \tilde{f} \mapsto \varphi_m^{-1} \left( \overline{a} \left( \tilde{f} \circ \fres{\varphi_n^{-1}}{\Omega_n}\right)\right).
\end{equation*}
Clearly, $\tilde{a}$ is continuous on $C^k([-1,1]^{2n}; \RR)$ with respect to the norm $\Vert \cdot \Vert_{\tilde{V}}$ on $C^k([-1,1]^{2n}; \RR)$ defined as
\begin{equation*}
\Vert \tilde{f} \Vert_{\tilde{V}} \defeq \left\Vert \tilde{f} \circ \fres{\varphi_n^{-1}}{\Omega_n}\right\Vert_V \quad \text{for } \tilde{f} \in C^k([-1,1]^{2n}; \RR).
\end{equation*}
Let
\begin{equation*}
\widetilde{M}: \quad \RR^{2m} \to C([-1,1]^{2n}; \RR), \quad \widetilde{M}(x) \defeq \RE(M(\varphi_m(x))) \circ \fres{\varphi_n}{[-1,1]^{2n}}.
\end{equation*}
Then it holds
\begin{align*}
&\norel \underset{\Vert f \Vert _{C^k (\Omega_n ; \CC)} \leq 1}{\underset{f \in C^k(\Omega_n ; \CC)}{\sup}} \Vert f - M (\overline{a}(f)) \Vert_{L^\infty (\Omega_n; \CC)} \\
&\geq \underset{\Vert f \Vert _{C^k (\Omega_n ; \RR)} \leq 1}{\underset{f \in C^k(\Omega_n ; \RR)}{\sup}} \Vert f - \RE(M (\overline{a}(f))) \Vert_{L^\infty (\Omega_n; \RR)} \\
&= \underset{\Vert \tilde{f} \Vert _{C^k ([-1,1]^{2n} ; \RR)} \leq 1}{\underset{\tilde{f} \in C^k([-1,1]^{2n} ; \RR)}{\sup}} \left\Vert \tilde{f} \circ \varphi_n^{-1} - \RE\left(M \left(\overline{a}\left(\tilde{f} \circ \fres{\varphi_n^{-1}}{\Omega_n}\right)\right)\right) \right\Vert_{L^\infty (\Omega_n; \RR)} \\
&= \underset{\Vert \tilde{f} \Vert _{C^k ([-1,1]^{2n} ; \RR)} \leq 1}{\underset{\tilde{f} \in C^k([-1,1]^{2n} ; \RR)}{\sup}} \left\Vert \tilde{f} - \RE\left(M \left(\varphi_m\left(\varphi_m^{-1}\left(\overline{a}\left(\tilde{f} \circ \fres{\varphi_n^{-1}}{\Omega_n}\right)\right)\right)\right)\right) \circ \varphi_n \right\Vert_{L^\infty ([-1,1]^{2n}; \RR)} \\
&= \underset{\Vert \tilde{f} \Vert _{C^k ([-1,1]^{2n} ; \RR)} \leq 1}{\underset{\tilde{f} \in C^k([-1,1]^{2n} ; \RR)}{\sup}} \left\Vert \tilde{f} - \widetilde{M}(\tilde{a}(\tilde{f})) \right\Vert_{L^\infty ([-1,1]^{2n}; \RR)} \geq \tilde{c} \cdot (2m)^{-k/(2n)},\\
\end{align*}
with a constant $\tilde{c} = \tilde{c}(n,k)$ provided by \Cref{thm: devore_real}. Hence, the claim follows by choosing $c = c(n,k) \defeq 2^{-k/(2n)} \cdot \tilde{c}$.
\end{proof}
\renewcommand*{\proofname}{Proof}
As a corollary, we formulate a special case of \Cref{thm:opti_conti} for the case of shallow complex-valued neural networks.
\begin{corollary}
\label{main_optimality}
    Let $n,k \in \NN$. Then there exists a constant $c=c(n,k) > 0$ with the following property: For any $m \in \NN$, $\phi \in C(\CC; \CC)$ and any map
    \begin{equation*}
        \eta : \quad C^k \left( \Omega_n ; \CC\right) \to \left(\CC^n\right)^m \times \CC^m \times \CC^m, \quad g \mapsto \left(\eta_1(g), \eta_2(g), \eta_3(g)\right)
    \end{equation*}
which is continuous with respect to some norm on $C^k (\Omega_n ; \CC)$, there exists $f \in C^k\left(\Omega_n; \CC\right)$ satisfying $\Vert f \Vert_{C^k (\Omega_n ; \CC)} \leq 1$ and
    \begin{equation*}
        \left\Vert f - \Psi(f)\right\Vert_{L^\infty \left(\Omega_n ; \CC\right)} \geq c \cdot m^{-k/(2n)},
    \end{equation*}
    where $\Psi(f) \in C(\Omega_n; \CC)$ is given by
\begin{equation*}
\Psi(f)(z) \defeq \sum_{j=1}^m \left(\eta_3(f)\right)_j \phi \left(\left[\eta_1 (f)\right]_j^T z + \left(\eta_2(f)\right)_j\right).
\end{equation*}
\end{corollary}

\begin{proof}
Using \Cref{thm:opti_conti}, we deduce that there exists $f \in C^k(\Omega_n;\CC)$ satisfying $\Vert f \Vert_{C^k(\Omega_n;\CC)} \leq 1$ and
\begin{equation*}
\Vert f - \Psi(f) \Vert_{L^\infty(\Omega_n; \CC)} \geq c' \cdot (m(n+2))^{-k/(2n)}
\end{equation*}
for a constant $c' = c'(n,k)>0$. Hence, the claim follows by letting $c \defeq c' \cdot (n+2)^{-k/(2n)}$.
\end{proof}






\section{Proof of Theorem \ref{main_4}} \label{sec:main_4}
Our proof of \Cref{main_4} is based on the following result (proven in \Cref{app: ridge}) which is based on the theory of ridge functions \cite{maiorov_best_1999,pinkus_ridge_2016}.
\begin{theorem} \label{thm: ridge}
Let $s,k \in \NN$ with $s \geq 2$ and $r>0$. Then there exists a constant $c = c(s,k) > 0$ with the following property: For every $m \in \NN$ there exist $a_1, ..., a_m \in \RR^s \setminus \{0\}$ with $\Vert a_j \Vert_2 = r$, such that for every function $f \in C^k ([-1,1]^s ; \RR)$ there exist polynomials $p_1, ..., p_m : \RR \to \RR$ satisfying
\begin{equation*}
\left\Vert f(x) - \sum_{j=1}^m p_j (a_j^T x) \right\Vert_{L^\infty ([-1,1]^s ; \RR)} \leq c \cdot m^{-k/(s-1)}\cdot \Vert f \Vert_{C^k([-1,1]^s; \RR)}.
\end{equation*}  
\end{theorem}
\begin{proof}
See \Cref{app: ridge}.
\end{proof}
Using the previous theorem, we can prove the following statement for complex-valued $C^k$-functions.
\begin{proposition}
\label{ridge_approx}
    Let $n,k \in \NN$. Then there exists a constant $c=c(n,k)>0$ with the following property: For any $m \in \NN$ there exist complex vectors $b_1, ..., b_m \in \CC^n$ with $\left\Vert b_j \right\Vert_2 = 1/ \sqrt{2n}$ for $j = 1,...,m$ and with the property that for any function $f \in C^k \left(\Omega_n ; \CC\right)$ there exist functions $g_1, ..., g_m \in C(\Omega_1; \CC)$ such that
    \begin{equation*}
        \left\Vert f(z) - \sum_{j=1}^m g_j \left( b_j^T \cdot z \right)\right\Vert_{L^\infty \left(\Omega_n; \CC\right)} \leq c \cdot m^{-k /(2n-1)} \cdot \left\Vert f \right\Vert_{C^k \left( \Omega_n; \CC\right)}.
    \end{equation*}
    Note that the vectors $b_1, ... b_m$ can be chosen independently from the considered function $f$, whereas $g_1, ..., g_m$ do depend on $f$.
\end{proposition}
\begin{proof}
     \Cref{thm: ridge} yields the existence of a constant $c_1 = c_1(n,k)>0$ with the property that for any $m \in \NN$ there exist real vectors $a_1, ..., a_m \in \RR^{2n}$ with $\left\Vert a_j \right\Vert_2 = 1 / \sqrt{2n}$ such that for any function $\tilde{f} \in C^k \left([-1,1]^{2n}; \RR\right)$ there exist functions $\tilde{g}_1, ..., \tilde{g}_m \in C([-1,1]; \RR)$ satisfying
    \begin{equation*}
        \left\Vert \tilde{f}(x) - \sum_{j=1}^m \tilde{g}_j \left( a_j^T  x \right)\right\Vert_{L^\infty \left([-1,1]^{2n}; \RR\right)} \leq c_1 \cdot m^{-k /(2n-1)} \cdot \Vert \tilde{f} \Vert_{C^k \left( [-1,1]^{2n}; \RR\right)}.
    \end{equation*}
    We then define the vectors $b_1, ..., b_m \in \CC^n$ componentwise via
    \begin{equation*}
        \left(b_j\right)_\ell \defeq \left(a_j\right)_\ell - i \cdot \left(a_j\right)_{n+\ell}, \quad \ell \in \{1,...,n\}, \ j \in \{1,...,m\}.
    \end{equation*}
    First we see $\left\Vert b_j \right\Vert_2 = \left\Vert a_j \right\Vert_2 = 1/\sqrt{2n}$. We first consider real-valued functions, i.e., $f \in C^k \left(\Omega_n; \RR\right)$. Let $\varphi_n$ be defined as in \eqref{isomorphism_intro}. By the choice of the constant $c_1$ we can find continuous functions $\tilde{g}_1,..., \tilde{g}_m \in C \left([-1,1]; \RR\right)$ such that
    \begin{equation*}
        \left\Vert (f \circ \varphi_n) (x)- \sum_{j=1}^m \tilde{g}_j \left( a_j^T x \right)\right\Vert_{L^\infty \left([-1,1]^{2n}\right)} \leq c_1 \cdot m^{-k /(2n-1)} \cdot \left\Vert f \circ \varphi_n \right\Vert_{C^k \left( [-1,1]^{2n}; \RR\right)}.
    \end{equation*}
    We then define $g_j \in C(\Omega_1; \RR)$ by $g_j(z) \defeq \tilde{g}_j \left(\RE(z)\right)$ for any $j \in \{1, ..., m\}$. For $z \in \Omega_n$ we then have
    \begin{align}
        g_j \left(\left( b_j \right)^T z \right) &= \tilde{g}_j \left( \RE \left(\sum_{\ell = 1}^n \left( b_j\right)_\ell \cdot z_\ell\right)\right) \nonumber \\
        &= \tilde{g}_j \left( \RE \left( \sum_{\ell = 1}^n \left(\left(a_j\right)_\ell - i\cdot \left(a_j\right)_{n+\ell}\right)\left(\varphi_n^{-1}(z)_\ell + i\cdot \varphi_n^{-1}(z)_{n+\ell}\right)\right)\right) \nonumber \\
        &= \tilde{g}_j \left(\sum_{\ell = 1}^n\left[\left(a_j\right)_\ell \varphi_n^{-1}(z)_\ell + \left(a_j\right)_{n+\ell} \varphi_n^{-1}(z)_{n+\ell}\right]\right) \nonumber \\
        \label{trafo_ident}
        &= \tilde{g}_j \left( \left( a_j \right)^T \cdot \varphi_n^{-1}(z)\right).
    \end{align}
    Therefore, 
    \begin{align*}
        \left\Vert f(z) - \sum_{j=1}^m g_j \left( b_j^T z\right)\right\Vert_{L^\infty \left(\Omega_n; \RR\right)} &= \left\Vert (f \circ \varphi_n)(x) - \sum_{j=1}^m g_j \left( b_j^T \cdot \varphi_n(x) \right) \right\Vert_{L^\infty \left([-1,1]^{2n}; \RR\right)} \\
        \overset{\text{(\ref{trafo_ident})}}&{=} \left\Vert (f \circ \varphi_n)(x) - \sum_{j=1}^m \tilde{g}_j \left( a_j^T \cdot x \right)\right\Vert_{L^\infty \left([-1,1]^{2n} ; \RR\right)} \\
        &\leq c_1 \cdot m^{-k /(2n-1)} \cdot \left\Vert f \circ \varphi_n\right\Vert_{C^k \left( [-1,1]^{2n}; \RR\right)}.
    \end{align*}
    By the above, for a function $f \in C^k \left(\Omega_n ; \CC\right)$ we can pick functions $g^{\RE}_1, ..., g^{\RE}_m, g^{\IM}_1, ..., g^{\IM}_m \in C \left(\Omega_1 ; \RR \right)$ satisfying
    \begin{align*}
        \left\Vert \RE(f(z)) - \sum_{j=1}^m g^{\RE}_j \left( b_j^T z \right)\right\Vert_{L^\infty \left(\Omega_n; \RR\right)} &\leq c_1 \cdot m^{-k /(2n-1)} \cdot \left\Vert {\RE} \left(f \circ \varphi_n \right)\right\Vert_{C^k \left( [-1,1]^{2n}; \RR\right)}, \\
        \left\Vert \IM(f(z)) - \sum_{j=1}^m g^{\IM}_j \left( b_j^T z \right)\right\Vert_{L^\infty \left(\Omega_n; \RR\right)} &\leq c_1 \cdot m^{-k /(2n-1)} \cdot \left\Vert \IM \left(f \circ \varphi_n \right)\right\Vert_{C^k \left( [-1,1]^{2n}; \RR\right)}.
    \end{align*}
    Defining $g_j := g^{\RE}_j + i \cdot g^{\IM}_j$ yields
    \begin{equation*}
        \left\Vert f(z)- \sum_{j=1}^m g_j \left(b_j^T z\right)\right\Vert_{L^\infty \left( \Omega_n; \CC\right)} \leq c_1 \cdot \sqrt{2}\cdot m^{-k /(2n-1)} \cdot \left\Vert f \right\Vert_{C^k \left( \Omega_n; \CC\right)},
    \end{equation*}
    completing the proof.
\end{proof}
The special activation function that yields the improved approximation rate of $m^{-k/(2n-1)}$ (see \Cref{main_4}) is constructed in the following lemma. 
\begin{lemma}
\label{special_acti_func}
    Let $\left\{ u_\ell\right\}_{\ell = 1}^\infty$ be an enumeration of the set of complex polynomials in $z$ and $\overline{z}$ with coefficients in $\QQ + i\QQ$.
    Then there exists a function $\phi \in C^\infty \left( \CC; \CC\right)$ with the following properties:
    \begin{enumerate}
        \item For every $\ell \in \NN$ and $z \in \Omega_1$ one has
        \begin{equation*}
            \phi(z+3\ell) = u_\ell(z).
        \end{equation*}
        \item $\phi$ is non-polyharmonic.
    \end{enumerate}
\end{lemma}
\begin{proof}
    Let $\psi \in C^\infty\left(\CC; \RR\right)$ with $0 \leq \psi \leq 1$ and
    \begin{equation*}
        \fres{\psi}{\Omega_1} \equiv 1, \qquad \supp(\psi) \subseteq \widetilde{\Omega},
    \end{equation*}
    where $\widetilde{\Omega} \defeq \left\{ z \in \CC : \ \left\vert \RE \left(z\right)\right\vert, \left\vert \IM \left(z\right) \right\vert < \frac{3}{2} \right\}$. We then define
    \begin{equation*}
        \phi \defeq f \cdot \psi + \sum_{\ell = 1}^\infty u_\ell(\bullet - 3\ell) \cdot \psi(\bullet - 3\ell),
    \end{equation*}
    where $f(z) = e^{\RE(z)}$. Note that $\phi$ is smooth since it is a locally finite sum of smooth functions. Furthermore, $\phi$ is non-polyharmonic on the interior of $\Omega_1$, since the calculation in the proof of \Cref{admissible} shows for $z$ in the interior of $\Omega_1$ and $\rho: \RR \to \RR, \ t \mapsto e^t$ that 
\begin{equation*}
\left\vert \wirt^m \wirtq^\ell \phi (z)\right\vert = \left\vert \wirt^m \wirtq^\ell f (z)\right\vert = \frac{1}{2^{m+\ell}} \left\vert \rho^{(m+ \ell)}(\RE(z))\right\vert>0
\end{equation*}
for arbitrary $m, \ell \in \NN_0$. Finally, property (1) follows directly by construction of $\phi$ because 
    \begin{equation*}
        (\widetilde{\Omega} + 3\ell) \cap (\widetilde{\Omega} + 3\ell') = \emptyset
    \end{equation*}
    for $\ell \neq \ell'$.
\end{proof}

Using the properties of the special activation function constructed in \Cref{special_acti_func} and applying the approximation result from \Cref{ridge_approx} we can now prove \Cref{main_4}.
\medskip
\renewcommand*{\proofname}{Proof of \Cref{main_4}}
\begin{proof}
    Let $\phi$ be the activation function constructed in \Cref{special_acti_func}. We choose the constant $c$ according to Proposition \ref{ridge_approx}. Let $m \in \NN$ and $f \in C^k \left(\Omega_n; \CC\right)$. We can without loss of generality assume that $f \not\equiv 0$. Again, according to \Cref{ridge_approx}, we can choose $\rho_1, ..., \rho_m \in \CC^n$ with $\left\Vert_2 \rho_j \right\Vert = 1 / \sqrt{2n}$ and $g_1, ..., g_m \in C\left(\Omega; \CC\right)$ with the property
    \begin{equation*}
        \left\Vert f(z) - \sum_{j=1}^m g_j \left( \rho_j^T z \right)\right\Vert_{L^\infty \left(\Omega_n\right)} \leq c \cdot m^{-k /(2n-1)} \cdot \left\Vert f \right\Vert_{C^k \left( \Omega_n; \CC\right)}.
    \end{equation*}
    Recall from \Cref{special_acti_func} that $\{u_\ell\}_{\ell = 1}^\infty$ is an enumeration of the set of complex polynomials in $z$ and $\overline{z}$. Hence, using the complex version of the Stone-Weierstraß-Theorem (see for instance \cite[Theorem 4.51]{folland_real_1999}), we can pick $\ell_1, ..., \ell_m \in \NN$ such that
    \begin{equation}
    \label{approxx}
        \left\Vert g_j - u_{\ell_j} \right\Vert_{L^\infty \left(\Omega_1 ; \CC\right)} \leq  m^{-1-k /(2n-1)} \cdot \left\Vert f \right\Vert_{C^k \left( \Omega_n; \CC\right)}
    \end{equation}
    for every $j \in \{1,...,m\}$. Since $\phi\left(\bullet + 3\ell\right) = u_\ell$ on $\Omega_1$ for each $\ell \in \NN$, and since $\rho_j^T z \in \Omega_1$ for $j \in \{1,...,m\}$ and $z \in \Omega_n$, we estimate
    \begin{align*}
        & \norel
\left\Vert f(z) - \sum_{j=1}^m \phi \left( \rho_j^T z + 3\ell_j\right)\right\Vert_{L^\infty \left(\Omega_n; \CC\right)} \\
        &\leq \left\Vert f(z) - \sum_{j=1}^m g_j \left( \rho_j^T z \right)\right\Vert_{L^\infty \left(\Omega_n; \CC\right)} + \sum_{j=1}^m \left\Vert g_j \left( \rho_j^T z\right) - \phi \left( \rho_j^T \cdot z + 3\ell_j\right)\right\Vert_{L^\infty \left(\Omega_n; \CC\right)} \\
        &\leq c \cdot m^{-k /(2n-1)} \cdot \left\Vert f \right\Vert_{C^k \left( \Omega_n; \CC\right)} + \sum_{j=1}^m \left\Vert g_j \left( z\right) - u_{\ell_j} \left( z\right)\right\Vert_{L^\infty \left(\Omega_1; \CC\right)} \\
        \overset{\eqref{approxx}}&{\leq}  c \cdot m^{-k /(2n-1)} \cdot \left\Vert f \right\Vert_{C^k \left( \Omega_n; \CC\right)} + m^{-k /(2n-1)} \cdot \left\Vert f \right\Vert_{C^k \left( \Omega_n; \CC\right)} \\
        &= (c+1) \cdot m^{-k/(2n-1)} \cdot \Vert f \Vert_{C^k \left(\Omega_n; \CC\right)}.
\qedhere
    \end{align*}
\end{proof}

\renewcommand*{\proofname}{Proof}



\section{Proof of Theorem \ref{main_5}}
As a preparation for the proof of \Cref{main_5}, we first prove a similar result in the real-valued setting. We remark that the proof idea is inspired by the proof of \cite[Theorem 4]{yarotsky_error_2017}.
\label{sec:main_5}

\begin{theorem}
\label{sigmoidoptimality}
Let $n, k \in \NN$ and 
\begin{equation*}
    \phi: \quad \RR \to \RR, \quad \phi(x) \defeq \frac{1}{1+e^{-x}}
\end{equation*}
be the sigmoid function. Then there exists a constant $c = c(n,k) > 0$ with the following property: If the numbers $\varepsilon \in (0,\frac{1}{2})$ and $m \in \NN$ are such that for every function $f \in C^k \left([-1,1]^n ; \RR\right)$ with $\left\Vert f \right\Vert_{C^k\left([-1,1]^n; \RR\right)} \leq 1$ there exist coefficients $\rho_1, ..., \rho_m \in \RR^n$, $\eta_1, ..., \eta_m \in \RR$ and $\sigma_1, ..., \sigma_m \in \RR$ satisfying
\begin{equation*}
    \left\Vert f(x) - \sum_{j=1}^m \sigma_j \cdot \phi \left( \rho_j^T x + \eta_j\right)\right\Vert_{L^\infty \left([-1,1]^n ; \RR\right)} \leq \varepsilon,
\end{equation*}
then necessarily
\begin{equation*}
    m \geq c \cdot \frac{\varepsilon^{-n/k}}{  \mathrm{ln}\left( 1/\varepsilon\right)}.
\end{equation*}
\end{theorem}
\begin{proof}
    We first pick a function $\psi \in C^\infty \left(\RR^n; \RR\right)$ with the property that $\psi(0) = 1$ and $\psi(x) = 0$ for every $x \in \RR^n$ with $\Vert x \Vert_2 > \frac{1}{4}$. We then choose
    \begin{equation*}
        c_1 = c_1(n,k) \defeq \left(\left\Vert \psi\right\Vert_{C^k\left([-1,1]^n;\RR\right)}\right)^{-1}.
    \end{equation*}
    Now, let $\varepsilon\in (0, \frac{1}{2})$ and $m \in \NN$ be arbitrary with the property stated in the formulation of the theorem. If $\varepsilon > \frac{c_1}{2}\cdot \frac{1}{6^k}$, then $m \geq c \cdot \frac{\varepsilon ^{-n/k}}{\ln (1 / \varepsilon)}$ trivially holds (as long as $c = c(n,k)> 0$ is sufficiently small). Hence, we can assume that $\varepsilon \leq \frac{c_1}{2} \cdot \frac{1}{6^k}$. Now, let $N$ be the smallest integer with $N \geq 2$, for which
    \begin{equation*}
        \frac{c_1}{2^{k+1}} \cdot N^{-k} \leq \varepsilon.
    \end{equation*}
    Note that this implies 
\begin{equation*}
N^k \geq \frac{c_1}{\varepsilon}\cdot \frac{1}{2^{k+1}} \geq \frac{c_1}{2^{k+1}} \cdot \frac{2}{c_1} \cdot 6^k = 3^k
\end{equation*}
 and hence $N \geq 3$, whence $N-1 \geq 2$. Therefore, by minimality of $N$, and since $\frac{N}{2} \leq N-1 $ because of $N \geq 2$, it follows that
    \begin{equation}
    \label{epsilonbound}
        \varepsilon < \frac{c_1}{2^{k+1}} \cdot (N-1)^{-k} \leq \frac{c_1}{2^{k+1}} 2^k \cdot N^{-k} = \frac{c_1}{2} \cdot N^{-k}.
    \end{equation}
    Now, for every $\alpha \in \{-N, ..., N\}^n$ pick $z_\alpha \in \{0,1\}$ arbitrary and let $y_\alpha \defeq z_\alpha c_1 N^{-k}$. Define the function
    \begin{equation*}
        f(x) \defeq \sum_{\alpha \in \{-N, ..., N\}^n} y_\alpha \cdot \psi \left( N x - \alpha\right), \quad x \in \RR^n.
    \end{equation*}
    Clearly, $f \in C^\infty (\RR^n; \RR)$. Furthermore, since the supports of the functions $\psi(\bullet - \alpha), \ \alpha \in \ZZ^n$ are pairwise disjoint, we see for any multi-index $\kk \in \NN_0^n$ with $\vert \kk \vert \leq k$ that
    \begin{align*}
        \left\Vert \partial^\kk f\right\Vert_{L^\infty \left([-1,1]^n; \RR\right)} &\leq N^{\vert \kk \vert} \cdot \underset{\alpha}{\max} \left\vert y_\alpha \right\vert \cdot \left\Vert \partial ^{\kk}\psi\right\Vert_{L^\infty \left([-1,1]^n; \RR\right)} \\
        &\leq N^{ k} \cdot \underset{\alpha}{\max} \left\vert y_\alpha \right\vert \cdot \left\Vert \psi\right\Vert_{C^k \left([-1,1]^n; \RR\right)} \leq 1,
    \end{align*}
    so we conclude that $\left\Vert f \right\Vert_{C^k \left([-1,1]^n; \RR\right)} \leq 1$. Additionally, for any fixed $\beta \in \{-N, ..., N\}^n$ we see 
    \begin{equation*}
        f\left(\frac{\beta}{N}\right) = y_\beta,
    \end{equation*}
    by choice of $\psi$. 

    By assumption, we can choose suitable coefficients $\rho_1, ..., \rho_m \in \RR^n$, $\eta_1, ..., \eta_m \in \RR$ and furthermore $\sigma_1, ..., \sigma_m \in \RR$ such that
    \begin{equation*}
        \left\Vert f - g \right\Vert_{L^\infty \left([-1,1]^n ; \RR\right)} \leq \varepsilon
    \end{equation*}
    for
    \begin{equation*}
   	g \defeq \sum_{j=1}^m \sigma_j \cdot \phi \left( \rho_j^T \cdot \bullet + \eta_j\right).
    \end{equation*}
Letting 
\begin{equation}  \label{gform}
\tilde{g} \defeq  g(\bullet / N) = \sum_{j=1}^m \sigma_j \cdot \phi \left( \frac{\rho_j^T}{N} \cdot\bullet + \eta_j\right),
\end{equation}
    we see for every $\alpha \in \{-N, ..., N\}^n$ that
    \begin{align*}
        \tilde{g}(\alpha) = g\left( \frac{\alpha}{N}\right) \begin{cases} \geq y_\alpha - \varepsilon = c_1N^{-k} - \varepsilon \overset{\eqref{epsilonbound}}{>} \left(c_1/2\right) N^{-k},& \text{if }z_\alpha = 1, \\ \leq y_\alpha + \varepsilon 
 \overset{\eqref{epsilonbound}}{<} \left ( c_1/2\right)N^{-k},& \text{if }z_\alpha = 0.\end{cases}
    \end{align*}
    Therefore, we get $\mathbbm{1}\left(\tilde{g} > (c_1/2)N^{-k}\right)\left(\alpha\right) = z_\alpha$ for any $\alpha \in \{-N, ..., N\}^n$. Since the choice of $z_\alpha$ has been arbitrary, it follows that the set
    \begin{equation*}
        H \defeq \left\{ \fres{\mathbbm{1}\left(\tilde{g} > (c_1/2)N^{-k}\right)}{\{-N,...,N\}^n} : \ \tilde{g} \text{ of form } \eqref{gform} \right\}
    \end{equation*}
    shatters the whole set $\{-N, ..., N\}^n$. Therefore, we conclude that
    \begin{equation*}
        \text{VC}(H) \geq (2N+1)^n \geq N^n.
    \end{equation*}
    On the other hand,  \cite[Theorem 8.11]{anthony_neural_1999} shows that
    \begin{equation*}
        \text{VC}(H) \leq 2m(n+2)\log_2 (60n\cdot N) \leq c_3 \cdot m \cdot \ln(N)
    \end{equation*}
    with a suitably chosen constant $c_3 = c_3(n)$. Here we used that $N \geq 3$ so that $\ln(N) \geq \ln(3) > 0$. Combining those two inequalities yields 
    \begin{equation*}
        m \geq \frac{N^n}{c_3 \cdot \ln(N)}.
    \end{equation*}
    Using that $N \geq c_4 \cdot \varepsilon^{-1/k}$ with $c_4 \defeq c_4(n,k) = \left(\frac{c_1}{2^{k+1}}\right)^{1/k}$ and $N \leq c_5 \cdot \varepsilon^{-1/k}$ with the definition $c_5 \defeq c_5(n,k) = \left(\frac{c_1}{2}\right)^{1/k}$, we see that
    \begin{equation*}
        m \geq \frac{c_4^n \cdot \varepsilon^{-n/k}}{c_3 \cdot \ln \left(c_5 \cdot \varepsilon^{-1/k}\right)} \geq c_6 \cdot \frac{\varepsilon^{-n/k}}{ \ln \left(1/\varepsilon\right)}
    \end{equation*}
    with $c_6=c_6(n,k)>0$ chosen appropriately.
\end{proof}
\begin{figure}[t]
\centering
\includegraphics[trim=2.7cm 2cm 0 6cm, width=1.2\textwidth, clip]{proof_illus.pdf}
\caption{Illustration of the function $f$ considered in the proof of \Cref{sigmoidoptimality}}
\end{figure}
As a corollary, we get a similar result for complex-valued neural networks.
\begin{corollary}
\label{sigmoidcomplex}
    Let $n,k \in \NN$ and 
\begin{equation*}
    \phi: \quad \CC \to \CC, \quad \phi(z) \defeq \frac{1}{1+e^{-\RE(z)}}.
\end{equation*}
Then there exists a constant $c = c(n,k) > 0$ with the following property: If $\varepsilon \in (0, \frac{1}{2})$ and $m \in \NN$ are such that for every function $f \in C^k \left(\Omega_n ; \CC\right)$ with $\left\Vert f \right\Vert_{C^k\left(\Omega_n; \CC\right)} \leq 1$ there exist coefficients $\rho_1, ..., \rho_m \in \CC^n$, $\eta_1, ..., \eta_m \in \CC$ and $\sigma_1, ..., \sigma_m \in \CC$ satisfying
\begin{equation*}
    \left\Vert f (z)- \sum_{j=1}^m \sigma_j \cdot \phi \left( \rho_j^T z + \eta_j\right)\right\Vert_{L^\infty \left(\Omega_n ; \CC\right)} \leq \varepsilon,
\end{equation*}
then necessarily 
\begin{equation*}
    m \geq c \cdot \frac{\varepsilon^{-2n/k}}{  \mathrm{ln}\left( 1/\varepsilon\right)}.
\end{equation*}
\end{corollary}
\begin{proof}
    We choose the constant $c=c(2n,k)$ according to the previous \Cref{sigmoidoptimality} and let $\varphi_n$ as in \eqref{isomorphism_intro}. Then, let $\varepsilon \in (0, \frac{1}{2})$ and $m \in \NN$ with the properties assumed in the statement of the corollary. If we then take an arbitrary function $f \in C^k \left([-1,1]^{2n}; \RR\right)$ with $\left\Vert f \right\Vert_{C^k \left([-1,1]^{2n}; \RR\right)} \leq 1$, we deduce the existence of $\rho_1,..., \rho_m \in \CC^n$, $\eta_1, ..., \eta_m \in \CC$ and $\sigma_1, ..., \sigma_m \in \CC$, such that
    \begin{align*}
        &\left\Vert f(x)  - \RE \left( \sum_{j=1}^m \sigma_j \cdot \phi \left( \rho_j^T \cdot \varphi_n(x) + \eta_j\right)\right)  \right\Vert_{L^\infty \left([-1,1]^{2n}; \RR\right)} \\
        \leq & \left\Vert (f \circ \varphi_n^{-1})(z) - \sum_{j=1}^m \sigma_j \cdot \phi \left( \rho_j^T z + \eta_j\right)\right\Vert_{L^\infty \left(\Omega_n ; \CC\right)} \leq \varepsilon.
    \end{align*}
    In the next step, we show that 
    \begin{equation*}
        \RR^{2n} \ni x \mapsto \RE \left( \sum_{j=1}^m \sigma_j \cdot \phi \left( \rho_j^T \cdot \varphi_n(x) + \eta_j\right)\right)  
    \end{equation*}
    is a real-valued shallow neural network with $m$ neurons in the hidden layer and the real sigmoid function as activation function. Then the claim follows using \Cref{sigmoidoptimality}. 

    For every $j \in \{1,...,m\}$ we pick a matrix $\tilde{\rho_j} \in \RR^{2n \times 2}$ with the property that one has
    \begin{equation*}
        \tilde{\rho_j} ^T \cdot \varphi_n^{-1}(z) = \varphi_1^{-1} \left(\rho_j^T \cdot z\right) 
    \end{equation*}
    for every $z \in \CC^n$. This is possible, since this is equivalent to 
\begin{equation*}
\tilde{\rho_j}^T v = \varphi_1^{-1} (\rho_j^T \varphi_n(v))
\end{equation*}
for all $v \in \RR^{2n}$, where the right-hand side is an $\RR$-linear map $\RR^{2n} \to \RR^2$. Denoting the first column of $\tilde{\rho_j}$ by $\hat{\rho_j}$, we get
    \begin{equation*}
        \hat{\rho_j} ^T \cdot \varphi_n^{-1}(z) = \RE \left(\rho_j^T \cdot z\right) \quad \text{for all } z \in\CC^n.
    \end{equation*}
    Writing $\gamma$ for the classical real-valued sigmoid function (i.e. $\gamma(x) = \frac{1}{1 + e^{-x}}$), we deduce for arbitrary $x \in \RR^{2n}$ that
    \begin{align*}
        \RE \left( \sum_{j=1}^m \sigma_j \cdot \phi \left( \rho_j^T \cdot \varphi_n(x) + \eta_j\right)\right) &= \RE \left( \sum_{j=1}^m \sigma_j \cdot \gamma\left( \RE \left( \rho_j^T \cdot \varphi_n(x) + \eta_j\right)\right)\right)  \\
        &= \RE \left( \sum_{j=1}^m \sigma_j \cdot \gamma\left(  \hat{\rho_j}^T x + \RE\left(\eta_j\right)\right)\right)  \\
        &= \sum_{j=1}^m \RE\left(\sigma_j\right) \cdot \gamma\left(  \hat{\rho_j}^T x + \RE\left(\eta_j\right)\right),
    \end{align*}
    where in the last step we used that $\gamma$ is real-valued. As noted above, this completes the proof.
\end{proof}

Now, we can finally prove \Cref{main_5}.

\renewcommand*{\proofname}{Proof of \Cref{main_5}}
\medskip
\begin{proof}
Let $\alpha = \frac{2n}{k}$ and choose $c_2 = c_2(\alpha) = c_2(n,k) > 0$ such that the inequality $\ln(x) \leq c_2 \cdot x^{\alpha /2}$ holds for all $x \geq 1$. Furthermore, let $c_1 = c_1(n,k) > 0$ as in \Cref{sigmoidcomplex}. By choosing $c = c(n,k) > 0$ sufficiently small , we can ensure that $\eps_m \defeq c \cdot (m \cdot \ln(m))^{-k/(2n)}$ satisfies 
\begin{equation*}
\ln \left(\frac{c_1}{c_2}\right) + \frac{\alpha}{2} \ln (1/\eps_m) \geq \frac{\alpha}{4} \cdot \ln(1/\eps_m) \quad \text{for all } m \in \NN_{\geq 2}.
\end{equation*}
    By further shrinking $c = c(n,k)>0$ if necessary, we may assume 
\begin{equation*}
c \cdot (2 \cdot \ln (2))^{-k/(2n)} < \frac{1}{2}
\end{equation*}
 and hence $c \cdot (m \cdot \ln (m))^{-k/(2n)} < \frac{1}{2}$ for all $m \in \NN_{\geq 2}$. Finally, setting $c_3 \defeq \frac{\alpha}{4}$ and shrinking $c$ even further, we can arrange that $c^\alpha < c_1 \cdot c_3$. Now, assume towards a contradiction that for every $f \in C^k \left(\Omega_n ; \CC\right)$ with $\Vert f \Vert_{C^k \left(\Omega_n; \CC\right)} \leq 1$ there are coefficients $\rho_1, ..., \rho_m \in \CC^n, \sigma_1, ..., \sigma_m \in \CC$ and $\eta_1, ..., \eta_m \in \CC$ such that
    \begin{equation*}
        \left\Vert f(z) - \sum_{j=1}^m \sigma_j \cdot \phi \left( \rho_j^T z + \eta_j\right)\right\Vert_{L^\infty \left(\Omega_n ; \CC\right)} < c \cdot \left(m \cdot \ln (m)\right)^{-k/(2n)}.
    \end{equation*}
    Applying \Cref{sigmoidcomplex}, we then get
    \begin{equation*}
        m \geq c_1 \cdot \frac{\varepsilon^{-2n/k}}{  \mathrm{ln}\left( 1/\varepsilon\right)}
    \end{equation*}
    with $\varepsilon = c \cdot \left(m \cdot \ln{m}\right)^{-k/(2n)} \in (0, \frac{1}{2})$ and $c_1 = c_1(n,k) > 0$. Recall from the beginning of the proof that $\alpha := 2n/k$ and that $\ln(x) \leq c_2 x^{\alpha / 2}$ for every $x \geq 1$. We observe
    \begin{equation*}
        m \geq c_1 \cdot \frac{\varepsilon^{-\alpha}}{  \mathrm{ln}\left( 1/\varepsilon\right)} \geq \frac{c_1}{c_2} \varepsilon^{- \alpha /2},
    \end{equation*}
    which implies
    \begin{equation*}
        \ln(m) \geq \ln \left(\frac{c_1}{c_2}\right) + \frac{\alpha}{2} \cdot \ln(1/\varepsilon) \geq \frac{\alpha}{4} \cdot \ln(1/\varepsilon) = c_3 \cdot \ln(1/\varepsilon) .
    \end{equation*}
    Overall we then get
    \begin{equation*}
        m \geq c_1 \cdot \frac{\varepsilon^{-\alpha}}{\ln(1/\varepsilon)} = c_1 \cdot c^{-\alpha} \cdot \frac{m \cdot \ln(m)}{\ln(1/\varepsilon)} \geq c_1 \cdot c_3 \cdot c^{-\alpha} \cdot \frac{m \cdot \ln(m)}{\ln(m)} = c_1 \cdot c_3 \cdot c^{-\alpha} \cdot m.
    \end{equation*}
    Since we chose $c$ such that $c^\alpha < c_1 \cdot c_3$ we get the desired contradiction.
\end{proof}
\renewcommand*{\proofname}{Proof}

\section{Proof of Theorem \ref{thm:intrac}} \label{sec:intrac}

The following result is the key to proving \Cref{thm:opti_conti} and can be found in \cite{devore_optimal_1989}.
For completion, its (short) proof is presented in \Cref{prelim_opti}. 

\begin{proposition}[{\cite[Theorem 3.1]{devore_optimal_1989}}] \label{thm: devore_pre}
Let $(X, \Vert \cdot \Vert_X)$ be a normed space,
$\emptyset \neq K \subseteq X$ a subset and $V \subseteq X$ a linear,
not necessarily closed subspace of $X$ containing $K$.
Let $m\in \NN$, let $\overline{a} : K \to \RR^m$ be a map which is continuous
with respect to some norm $\Vert \cdot \Vert_V$ on $V$ and $M : \RR^m \to X$ some arbitrary map.
Let
\begin{equation} \label{eq:bmkx2}
  b_m(K)_X
  \defeq \underset{X_{m+1}}{\sup}
           \sup
             \left\{\varrho \geq 0: \  U_\varrho(X_{m+1}) \subseteq K\right\},
\end{equation} 
where the first supremum is taken over all $(m+1)$-dimensional linear subspaces $X_{m+1}$ of $X$ and
\begin{equation*}
  U_\varrho(X_{m+1})
  \defeq \{y \in X_{m+1} : \ \Vert y \Vert_X \leq \varrho\}.
\end{equation*}
Further, we set $b_m(K)_X \defeq 0$ if the supremum in \eqref{eq:bmkx2}
is not well-defined as a quantity in $[0, \infty]$.
Then it holds
\begin{equation*}
  \underset{x \in K}{\sup} \Vert x - M (\overline{a}(x)) \Vert_X \geq b_m(K)_X.
\end{equation*}
\end{proposition}

\begin{proof}
See \Cref{thm: devore}.
\end{proof}

We can use \Cref{thm: devore_pre} not only to show that the rate of convergence established in this paper is optimal (which is done in \Cref{optimality_section,prelim_opti}) but also to show that the problem of approximating $C^k$-functions using a set of functions that can be parametrized with finitely many parameters is intractable in the sense that it suffers from the curse of dimensionality, provided that the map which assigns to each $C^k$-function the parameters of the approximating function is continuous. This is the subject of this section. 

In \cite{NOVAK2009398} a certain space of polynomials was used to show the intractability in the case of \emph{linear} approximation methods. We are also going to use this class of polynomials, but combine it with \Cref{thm: devore_pre} to infer intractability in the case of \emph{continuous} approximation methods. We start with a lemma discussing an important property of this space of polynomials.
This property is stated as part of a proof in \cite{NOVAK2009398}, but no complete proof is provided.
\begin{lemma}\label{lem:poly_class}
Let $s \in \NN$ and consider a function $f \in C^\infty([-1,1]^s; \RR)$ which is given via
\begin{equation}\label{eq:form_poly_0_1}
f(x) = \sum_{\kk \in \{0,1\}^s} a_\kk x^\kk
\end{equation}
with coefficients $a_\kk \in \RR$ for every $\kk \in \{0,1\}^s$. Then it holds
\begin{equation*}
\Vert f \Vert_{C^k ([-1,1]^s; \RR)} = \Vert f \Vert_{L^\infty([-1,1]^s; \RR)}
\end{equation*}
for every $k \in \NN$.
\end{lemma}
\begin{proof}
The proof is by induction over $s$. We start with the case $s=1$ and note that we can write $f(x) = ax + b$ with $a,b \in \RR$ in that case. Switching to $-f$ if necessary, we can assume $a \geq 0$. Clearly, $\Vert f \Vert_{L^\infty([-1,1]; \RR)} \leq \vert a \vert + \vert b \vert$. Conversely, if $b \geq 0$ then $\vert f(1) \vert = \vert a + b \vert = \vert a \vert + \vert b \vert$. If otherwise $b < 0$ then $\vert f(-1) \vert = \vert b  - a \vert = \vert a - b \vert = \vert a \vert + \vert b \vert$. Thus, $\Vert f \Vert_{L^\infty([-1,1]; \RR)} = \vert a \vert + \vert b \vert$. For the derivatives, we have $\Vert f' \Vert_{L^\infty([-1,1]; \RR)} = \vert a \vert$ and $\Vert f^{(k)} \Vert_{L^\infty([-1,1]; \RR)} = 0$ for $k\geq 2$. This proves the claim in the case $s=1$. 

We now assume that the claim holds for some arbitrary but fixed $s \in \NN$. We further let $\alpha \in \NN_0^{s+1}$ and \emph{fix} a point $(x_1, ..., x_{s+1}) \in [-1,1]^{s+1}$. We decompose $\alpha = (\alpha', \alpha_{s+1})$ with $\alpha' \in \NN_0^s$. Let
\begin{equation*}
\widetilde{f}: \quad [-1,1] \to \RR, \quad y_{s+1} \mapsto \partial^{(\alpha',0)}f(x_1, ..., x_s, y_{s+1})
\end{equation*}
and note 
\begin{equation*}
\partial^\alpha f(x_1, ..., x_{s+1}) =  \widetilde{f}^{(\alpha_{s+1})} (x_{s+1}).
\end{equation*}
Note that $f$ is affine-linear with respect to each variable (with all other variables hold fixed). Hence, $\widetilde{f}$ is an affine function and we can thus apply the case $s=1$ to $\widetilde{f}$ and get
\begin{equation*}
\Vert \widetilde{f}^{(\alpha_{s+1})} \Vert_{L^\infty([-1,1]; \RR)} \leq \Vert \widetilde{f} \Vert_{L^\infty([-1,1]; \RR)}.
\end{equation*}
Putting this together, we infer
\begin{equation}\label{eq:jj1}
\vert  \partial^\alpha f(x_1, ..., x_{s+1}) \vert \leq \Vert \widetilde{f} \Vert_{L^\infty([-1,1]; \RR)} = \underset{y_{s+1} \in [-1,1]}{\sup} \vert \partial^{(\alpha',0)}f(x_1, ..., x_s, y_{s+1})\vert.
\end{equation}
We now \emph{fix} an arbitrary point $y_{s+1} \in [-1,1]$ and consider
\begin{equation*}
\widehat{f}: \quad [-1,1]^s \to \RR, \quad (y_1, ..., y_{s}) \mapsto f(y_1, ..., y_s, y_{s+1}).
\end{equation*}
Then it holds
\begin{equation*}
\partial^{(\alpha',0)}f(x_1, ..., x_s, y_{s+1}) = \partial^{\alpha'} \widehat{f}(x_1, ..., x_s).
\end{equation*}
Applying the induction hypothesis to $\widehat{f}$ (which is easily seen to be of the form \eqref{eq:form_poly_0_1}) we get
\begin{equation}\label{eq:jj2}
\vert \partial^{\alpha'} \widehat{f}(x_1, ..., x_s)\vert \leq \Vert \widehat{f} \Vert_{L^\infty([-1,1]^s ; \RR)} \leq \Vert f \Vert_{L^\infty([-1,1]^{s+1}; \RR)}.
\end{equation}
Combining \eqref{eq:jj1} and \eqref{eq:jj2} yields
\begin{equation*}
\vert  \partial^\alpha f(x_1, ..., x_{s+1}) \vert \leq \Vert f \Vert_{L^\infty([-1,1]^{s+1}; \RR)}.
\end{equation*}
Since $\alpha \in \NN_0^{s+1}$ was arbitrary, we get the claim by noting that 
\begin{equation*}
\Vert f \Vert_{C^k ([-1,1]^s; \RR)} \geq \Vert f \Vert_{L^\infty([-1,1]^s; \RR)}
\end{equation*}
holds trivially for every $k \in \NN$.
\end{proof}
Using the above lemma, we can now deduce that the approximation of smooth functions using continuous approximation methods is intractable in terms of the input dimension.
\medskip
\renewcommand*{\proofname}{Proof of \Cref{thm:intrac}}
\begin{proof}
We apply \Cref{thm: devore_pre} to $X \defeq C([-1,1]^s ; \RR)$, $V \defeq C^{\infty, \ast,s}$ and to the set $K \defeq \{f \in C^{\infty, \ast, s}: \  \Vert f \Vert_{C^\infty ([-1,1]^s ; \RR)} \leq 1\}$ and $m \defeq 2^s - 1$. The space 
\begin{equation*}
X_{m+1} \defeq \left\{ [-1,1]^s \ni x \mapsto \sum_{\kk \in \{0,1\}^s} a_\kk x^\kk: \ a_\kk \in \RR\right\}
\end{equation*}
consisting of all functions considered in the previous \Cref{lem:poly_class} is an $(m+1)$-dimensional subspace of $C([-1,1]^s ; \RR)$. For every $f \in X_{m+1}$ with $\Vert f \Vert_{L^\infty([-1,1]^s ; \RR)} \leq 1$, \Cref{lem:poly_class} tells us $\Vert f \Vert_{C^\infty([-1,1]^s ; \RR)} \leq 1$. Hence, $U_1(X_{m+1}) \subseteq K$ and \Cref{thm: devore_pre} then yields the claim.
\end{proof}
\begin{remark}
The statement of \Cref{thm:intrac} also holds if the functions satisfy $\overline{a}: C^{\infty, *, s} \to \RR^m$ and $M: \RR^m \to C([-1,1]^s;\RR)$ with $m \leq 2^s - 1$. This can be seen by defining 
\begin{equation*}
\tilde{a} : \quad C^{\infty, *, s} \to \RR^{2^s -1}, \quad f \mapsto (\overline{a}(f), 0, ..., 0)
\end{equation*}
and
\begin{equation*}
\widetilde{M}: \quad \RR^{2^s -1} \to C([-1,1]^s;\RR), \quad (a,b) \mapsto M(a)
\end{equation*}
with $a \in \RR^m$ and $b \in \RR^{2^s -1 -m}$.
\end{remark}
\renewcommand*{\proofname}{Proof}
The following \Cref{corr:intrac_complex} transfers \Cref{thm:intrac} to the complex-valued setting.
\begin{corollary}\label{corr:intrac_complex}
Let $n \in \NN$. For any function $f \in C^\infty(\Omega_n ; \CC)$ we write
\begin{equation*}
\Vert f \Vert_{C^\infty(\Omega_n ; \CC)} \defeq \underset{k \in \NN}{\sup} \ \Vert f \Vert_{C^k(\Omega_n; \CC)}
\end{equation*}
and let $C^{\infty, *, n}_{\CC}$ denote the space consisting of all functions for which this expression is finite. Let $\overline{a}:  C^{\infty, *, n}_{\CC} \to \CC^{2^{2n-1} - 1}$ be continuous with respect to some norm on $C^{\infty, *, n}_{\CC}$ and moreover, let $M: \CC^{2^{2n-1}-1} \to C(\Omega_n; \CC)$ be an arbitrary map. Then it holds
\begin{equation*}
\underset{\Vert f \Vert_{C^\infty(\Omega_n ; \CC)} \leq 1}{\underset{f \in C^{\infty, *, n}_{\CC}}{\sup}} \Vert f - M(\overline{a}(f))\Vert_{L^\infty(\Omega_n ; \CC)} \geq 1.
\end{equation*}
\end{corollary}
\begin{proof}
The transfer to the complex-valued setting works in the same manner as the proof of \Cref{thm:opti_conti} (see \Cref{optimality_section}). We write $m \defeq 2^{2n-1}-1$ and note $2m = 2^{2n} - 2 \leq 2^{2n}-1$. We define $\tilde{a} : C^{\infty, \ast, 2n}\to \RR^{2m}$ and $\widetilde{M}: \RR^{2m} \to C([-1,1]^{2n}; \RR)$ in the same way as in the proof of \Cref{thm:opti_conti}. Using again the same technique as in the proof of \Cref{thm:opti_conti}, we get
\begin{equation*}
\underset{\Vert f \Vert _{C^\infty (\Omega_n ; \CC)} \leq 1}{\underset{f \in C^{\infty, *, n}_{\CC}}{\sup}} \Vert f - M (\overline{a}(f)) \Vert_{L^\infty (\Omega_n; \CC)} \geq  \underset{\Vert \tilde{f} \Vert _{C^\infty ([-1,1]^{2n} ; \RR)} \leq 1}{\underset{\tilde{f} \in C^{\infty, \ast, 2n}}{\sup}} \left\Vert \tilde{f} - \widetilde{M}(\tilde{a}(\tilde{f})) \right\Vert_{L^\infty ([-1,1]^{2n}; \RR)} \geq 1,
\end{equation*}
applying \Cref{thm:intrac} in the last inequality, using $2m \leq 2^{2n}-1$.
\end{proof}
We conclude this appendix by adding a note on the constant appearing in our main approximation bound.
\begin{corollary} \label{corr:const_intrac}
Let $n \in \NN$ with $n \geq 2$ and $\alpha > 0$ and let $\phi \in C(\CC;\CC)$. Let $\tilde{c} = \tilde{c}(n,\alpha)>0$ be such that for every $m \in \NN$ there exists a mapping
\begin{equation*}
        \eta : \quad  C^{\infty, *, n}_{\CC} \to \left(\CC^n\right)^m \times \CC^m \times \CC^m, \quad g \mapsto \left(\eta_1(g), \eta_2(g), \eta_3(g)\right)
\end{equation*}
that is continuous with respect to any norm on $C^{\infty, *, n}_{\CC}$ and such that
\begin{equation*}
        \left\Vert f - \Psi(f)\right\Vert_{L^\infty \left(\Omega_n ; \CC\right)} \leq \left( \tilde{c} \cdot m\right)^{-\alpha} \cdot \Vert f \Vert_{C^\infty(\Omega_n; \CC)},
    \end{equation*}
    for every $f \in C^{\infty, *, n}_{\CC}$. Here, $\Psi(f) \in C(\Omega_n; \CC)$ is given by
\begin{equation*}
\Psi(f)(z) \defeq \sum_{j=1}^m \left(\eta_3(f)\right)_j \phi \left(\left[\eta_1 (f)\right]_j^T z + \left(\eta_2(f)\right)_j\right).
\end{equation*}
 Then it necessarily holds $\tilde{c}\leq 16 \cdot 2^{-n}$.
\end{corollary}
\begin{proof}
We first assume $n \geq 4$. We take $m= \left\lfloor \frac{2^{2n - 1} - 1}{n+2} \right\rfloor$ and note that then $m(n+2) \leq 2^{2n-1}-1$. Therefore, \Cref{corr:intrac_complex} applies and we infer that for each $\eps \in (0,1)$, there exists $f = f_\eps \in C^{\infty, *, n}_{\CC}$ with $\Vert f \Vert_{C^\infty(\Omega_n;\CC)} \leq 1$ and such that
\begin{equation*}
1 - \eps \leq \left\Vert f - \Psi(f)\right\Vert_{L^\infty \left(\Omega_n ; \CC\right)} \leq \left( \tilde{c} \cdot m\right)^{-\alpha} \cdot \Vert f \Vert_{C^\infty(\Omega_n; \CC)} \leq \left( \tilde{c} \cdot m\right)^{-\alpha}.
\end{equation*}
This then necessarily implies $\tilde{c} \cdot m \leq 1$ or equivalently $\tilde{c}\leq 1/m$. It therefore suffices to derive a lower bound for $m$. Firstly, we note
\begin{equation*}
2^{2n-1} = 2^{n-3} \cdot 2^{n+2} = 2^{n-3} \cdot (1+1)^{n+2} \geq 2^{n-3}(n+3),
\end{equation*}
where we applied Bernoulli's inequality. Because of $n \geq 4 \geq 3$, this yields
\begin{equation*}
2^{2n-1} - 1 \geq 2^{n-3}(n+3) - 2^{n-3} =  2^{n-3} (n+2).
\end{equation*}
Hence, we get
\begin{equation*}
m \geq \frac{2^{2n - 1} - 1}{n+2} -1  \geq 2^{n-3} -1 = 2^{n-3}(1- 2^{3-n}) \geq 2^{n - 4} = \frac{2^n}{16}.
\end{equation*}
Here, we used $n \geq 4$ in the last inequality. An explicit computation shows that the same bounds also holds in the cases $n=2$ and $n=3$. This proves the claim. 
\end{proof}


\appendix
% !TeX encoding = UTF-8
% !TeX spellcheck = en_US
% !TeX root = main_paper.tex

\section{Divided Differences}
Divided differences are well-known in numerical mathematics as they are for example used to calculate the coefficients of an interpolation polynomial in its Newton representation. In our case, we are interested in divided differences, since they can be used to obtain a generalization of the classical mean-value theorem for differentiable functions: Given an interval $I \subseteq \RR$ and $n+1$ pairwise distinct data points $x_0, ..., x_n \in I$ as well as an $n$-times differentiable real-valued function $f : I \to \RR$, there exists $\xi \in \left( \text{min}\left\{x_0 ,..., x_n\right\}, \text{max}\left\{x_0 ,..., x_n\right\}\right)$, such that
\begin{equation*}
    f\left[x_0, ..., x_n\right] = \frac{f^{(n)}(\xi)}{n!},
\end{equation*}
where the left-hand side is a divided difference of $f$ (defined below). The classical mean-value theorem is obtained in the case $n=1$. Our goal in this section is to generalize this result to a multivariate setting by considering divided differences in multiple variables. Such a generalization is probably well-known, but since we could not locate a convenient reference and to make the paper more self-contained, we provide a proof.

Let us first define divided differences formally. Given $n+1$ data points $\left(x_0, y_0\right), ..., \left(x_n, y_n\right) \in \RR \times \RR$ with pairwise distinct $x_k$, we define the associated divided differences recursively via
\begin{align*}
    \left[ y_k\right] &\defeq y_k, \ k \in \{0,...,n\}, \\
    \left[y_k ,..., y_{k+j}\right] &\defeq \frac{\left[y_{k+1},..., y_{k+j}\right] - \left[y_k,..., y_{k+j-1}\right]}{x_{k+j}-x_k}, \ j \in \{1,...,n\}, \ k \in \{0, ..., n-j\}.
\end{align*}
If the data points are defined by a function $f$ (i.e. $y_k = f\left(x_k\right)$ for all $k \in \{0,...,n\}$), we write
\begin{equation*}
    \left[x_k,...,x_{k+j}\right] f \defeq \left[y_k, ..., y_{k+j}\right].
\end{equation*}
We first consider an alternative representation of divided differences, the so-called \emph{Hermite-Genocchi-Formula}. To state it, we introduce the notation $\Sigma^k$ for a certain $k$-dimensional simplex.
\begin{definition}
    Let $s \in \NN$. Then we define
    \begin{equation*}
        \Sigma^s \defeq \left\{ x \in \RR^s : \ x_\ell \geq 0 \ \mathrm{ for} \ \mathrm{all}  \ \ell \ \mathrm{ and  } \sum_{\ell = 1}^s x_\ell \leq 1 \right\}.
    \end{equation*}
    The identity $\lambda^s\left(\Sigma^s\right)= \frac{1}{s!}$ holds true.
\end{definition}
A proof for the fact that the volume of $\Sigma^s$ is indeed $\frac{1}{s!}$ can be found for example in \cite{stein_note_1966}.
We can now consider the alternative representation of divided differences.
\begin{lemma}[Hermite-Genocchi-Formula]
    For real numbers $a,b \in \RR$, a function $f \in C^k([a,b]; \RR)$ and pairwise distinct $x_0, ..., x_k \in [a,b]$, the divided difference of $f$ at the data points $x_0, ..., x_k$ is given as
\begin{equation}
\label{hg}
    \left[x_0, ..., x_k\right]f = \int_{\Sigma^k} f^{(k)}\left(x_0 + \sum_{\ell=1}^{k}s_\ell\left(x_\ell-x_0\right)\right) ds.
\end{equation}
\end{lemma}
\begin{proof}
    See \cite[Theorem 3.3]{atkinson_introduction_1989}.
\end{proof}
We need the following easy generalization of the mean-value theorem for integrals.
\begin{lemma}
Let $D \subseteq \RR^s$ be a connected and compact set with positive Lesbesgue measure and furthermore $f : D \to \RR$ a continuous function. Then there exists $\xi \in D$ such that
\begin{equation*}
    f(\xi) = \frac{1}{\lambda(D)} \cdot \int_D f(x) dx.
\end{equation*}
\end{lemma}
\begin{proof} 
Since $D$ is compact and $f$ continuous, there exist $x_{\text{min}} \in D$ and $x_{\text{max}} \in D$ satisfying
\begin{equation*}
    f\left(x_{\text{min}}\right) \leq f(x) \leq f\left(x_{\text{max}} \right)
\end{equation*}
for all $x \in D$. Thus, one gets
\begin{equation*}
    f\left( x_{\text{min}}\right) \leq \frac{1}{\lambda(D)} \int_D f(x) dx \leq  f\left( x_{\text{max}}\right)
\end{equation*}
so the claim follows using the fact that $f(D) \subseteq \RR$ is connected, i.e., an interval.
\end{proof}
We also get a convenient representation of divided differences if the data points are equidistant.
\begin{lemma}
Let $f: \RR \to \RR$, $x_0 \in \RR$ and $h > 0$. We consider the case of equidistant data points, meaning $x_{j} \defeq x_0 + jh$ for all $j = 1,...,k$ for a fixed $h > 0$. In this case, we have the formula
\begin{equation}
\label{alternativdarstellung}
    \left[x_0, ..., x_k\right]f = \frac{1}{k!h^k} \cdot \sum_{r=0}^k (-1)^{k-r}\binom{k}{r} f\left(x_r\right). 
\end{equation}
\end{lemma}
\begin{proof}
We prove the result via induction over the number $j$ of considered data points, meaning the following: For all $j \in \{0,...,k\}$ we have
\begin{equation*}
    \left[x_\ell, ..., x_{\ell+j}\right]f = \frac{1}{j!h^j} \cdot \sum_{r=0}^j (-1)^{j-r}\binom{j}{r} f\left(x_{\ell+r}\right)
\end{equation*}
for all $\ell \in \{0, ..., k\}$ such that $\ell + j \leq k$. The case $j = 0$ is trivial. Therefore, we assume the claim to be true for a fixed $j \in \{0,...,k-1\}$ and choose an arbitrary $\ell \in \{0,...,k\}$ such that $\ell+j+1 \leq k$. We then get
\begin{align*}
    \left[x_\ell, ..., x_{\ell+j+1}\right]f &= \frac{\left[x_{\ell+1}, ..., x_{\ell+j+1}\right]f - \left[x_\ell, ..., x_{\ell+j}\right]f}{x_{\ell+j+1}-x_\ell} \\
    &= \frac{1}{j!h^j}\cdot \frac{\sum_{r=0}^j (-1)^{j-r}\binom{j}{r} \left(f\left(x_{\ell+r+1}\right) - f\left(x_{\ell+r}\right)\right)}{(j+1)h} \\
    &= \frac{1}{(j+1)!h^{j+1}}\sum_{r=0}^j (-1)^{j-r}\binom{j}{r} \left(f\left(x_{\ell+r+1}\right) - f\left(x_{\ell+r}\right)\right).
\end{align*}
Using an index shift we deduce
\begin{align*}
    & \norel \sum_{r=0}^j (-1)^{j-r}\binom{j}{r}f\left(x_{\ell+r+1}\right) - \sum_{r=0}^j (-1)^{j-r}\binom{j}{r}f\left(x_{\ell+r}\right) \\
    &= \sum_{r=1}^{j+1} (-1)^{j+1-r}\binom{j}{r-1}f\left(x_{\ell+r}\right) + \sum_{r=0}^j (-1)^{j+1-r}\binom{j}{r}f\left(x_{\ell+r}\right) \\
    &= (-1)^{j+1} f\left(x_\ell\right) + \sum_{r=1}^{j} \left((-1)^{j+1-r}f\left(x_{\ell+r}\right) \left[\binom{j}{r-1} + \binom{j}{r}\right]\right) + f\left(x_{\ell+j+1}\right) \\
    &= \sum_{r=0}^{j+1} (-1)^{j+1-r}\binom{j+1}{r} f\left(x_{\ell+r}\right)
\end{align*}
which yields the claim.
\end{proof}
The final result for divided differences is stated as follows:
\begin{theorem}
\label{div_differences_mainresult}
Let $f: \RR^s \to \RR$ and $k \in \NN_0, r>0$, such that $\fres{f}{(-r,r)^s} \in C^k \left((-r,r)^s; \RR\right)$. For $\textbf{p} \in \NN_0^s$ with $\vert \pp \vert \leq k$ and $h>0$ let
\begin{equation*}
    f_{\pp,h} \defeq (2h)^{-\vert \pp \vert} \sum_{0 \leq \textbf{r} \leq \textbf{p}} (-1)^{\vert \pp \vert -\vert \rr \vert} \binom{\textbf{p}}{\textbf{r}} f \left( h(2\rr-\pp)\right).
\end{equation*}
Let $m \defeq \underset{j}{\mathrm{max}} \  \pp_j$. Then, for $0 <h < \frac{r}{\max\{1,m\}}$ there exists $\xi \in h[-m,m]^s$ satisfying
\begin{equation*}
    f_{\pp,h} = \partial^\pp f(\xi).
\end{equation*}
\end{theorem}
\begin{proof}
We may assume $m > 0$, since $\pp=0$ implies $f_{\pp,h} = f(0)$ and hence, the claim follows with $\xi = 0$.

We prove via induction over $s \in \NN$ that the formula
\begin{equation}
\label{to_prove}
    f_{\pp,h}= \pp! \int_{\Sigma^{\pp_s}}\int_{\Sigma^{\pp_{s-1}}} \cdot\cdot\cdot \int_{\Sigma^{\pp_1}} \partial^\pp f \left( -h\pp_1 + 2h\sum_{\ell=1}^{\pp_1}\ell\sigma_\ell^{(1)}, ..., -h\pp_s + 2h\sum_{i=1}^{\pp_s}\ell\sigma_\ell^{(s)}\right)d\sigma^{(1)} \cdot \cdot \cdot d\sigma^{(s)}
\end{equation}
holds for all $\pp \in \NN_0^s$ with $1 \leq \vert \pp \vert \leq k$ and all $0 < h < \frac{r}{m}$. The case $s=1$ is exactly the Hermite-Genocchi-Formula (\ref{hg}) combined with (\ref{alternativdarstellung}) applied to the data points $-hp, -hp + 2h, ..., hp-2h, hp$. 

By induction, assume that the claim holds for some $s \in \NN$.
For a fixed point $y \in (-r,r)$, let 
\begin{equation*}
    f_y: \quad (-r,r)^s \to \RR, \quad x \mapsto f(x,y).
\end{equation*}
For $\pp \in \NN_0^{s+1}$ with $\vert \pp \vert \leq k$ and $\pp' := \left(p_1,...,p_s\right)$ we define
\begin{equation*}
    \Gamma: \quad (-r,r) \to \RR, \quad y \mapsto \left( f_y\right)_{\pp',h} = (2h)^{- \vert \pp' \vert} \sum_{0 \leq \rr' \leq \pp'} (-1)^{\vert  \pp' \vert - \vert \rr' \vert} \binom{\pp'}{\rr'} f\left( h(2\rr' - \pp'),y\right).
\end{equation*}
Using the induction hypothesis, we get
\begin{align*}
\label{IV}
    &\Gamma(y) \\
    =&\pp'! \int\limits_{\Sigma^{\pp_s}}\int\limits_{\Sigma^{\pp_{s-1}}} \cdot\cdot\cdot \int\limits_{\Sigma^{\pp_1}} \partial^{\left(\pp',0 \right)} f \left( -h\pp_1 + 2h\sum_{i=1}^{\pp_1}i\sigma_i^{(1)}, ..., -h\pp_s + 2h\sum_{i=1}^{\pp_s}i\sigma_i^{(s)},y\right)d\sigma^{(1)} \cdot \cdot \cdot d\sigma^{(s)}
\end{align*}
for all $y \in (-r,r)$. Furthermore, we calculate
\begin{align*}
    &\norel \pp_{s+1}! \cdot [-h \cdot \pp_{s+1}, -h \cdot \pp_{s+1} + 2h, ..., h \cdot \pp_{s+1}]\Gamma  \\
    \overset{\eqref{alternativdarstellung}}&{=} (2h) ^{- \pp_{s+1}} \sum_{r' = 0}^{\pp_{s+1}} (-1)^{\pp_{s+1}-r'}\binom{\pp_{s+1}}{r'} \Gamma\left(h\left(2r'-\pp_{s+1}\right)\right) \\
    &= (2h) ^{- \pp_{s+1}} \sum_{r' = 0 }^{\pp_{s+1}} (-1)^{\pp_{s+1}-r'}\binom{\pp_{s+1}}{r'} (2h)^{- \vert \pp' \vert} \sum_{0 \leq \rr' \leq \pp'}(-1)^{\vert \pp' \vert -\vert \rr' \vert} \binom{\pp'}{\rr'} f\left( h(2\rr' - \pp'), h(2r' - \pp_{s+1})\right) \\
    &= (2h)^{-\vert \pp \vert} \sum_{0 \leq \textbf{r} \leq \textbf{p}} (-1)^{\vert \pp \vert -\vert \rr \vert} \binom{\textbf{p}}{\textbf{r}} f \left( h(2\rr-\pp)\right) \\
    &= f_{\pp, h}.
\end{align*}
On the other hand, we get
\begin{align*}
    &\norel [-h \cdot \pp_{s+1}, -h \cdot \pp_{s+1} + 2h, ..., h \cdot \pp_{s+1}]\Gamma \\
    \overset{\eqref{hg}}&{=} \int_{\Sigma^{\pp_{s+1}}}\Gamma^{(\pp_{s+1})}\left(-h\pp_{s+1} + 2h\sum_{\ell=1}^{\pp_{s+1}}\ell t_\ell\right)dt \\
    \overset{}&{=}\pp'! \int\limits_{\Sigma^{\pp_{s+1}}} \cdot\cdot\cdot \int\limits_{\Sigma^{\pp_1}} \partial^{\pp} f \left( -h\pp_1 + 2h\sum_{\ell=1}^{\pp_1}\ell\sigma_\ell^{(1)}, ..., -h\pp_{s+1}+2h\sum_{\ell=1}^{\pp_{s+1}}\ell\sigma^{(s+1)}_\ell\right)d\sigma^{(1)} \cdot \cdot \cdot d\sigma^{(s+1)}.
\end{align*}
Changing the order of integration and derivative is possible, since we integrate on compact sets and only consider continuously differentiable functions.

We have thus proven (\ref{to_prove}) using the principle of induction. The claim of the theorem then follows directly using the mean-value theorem for integrals and by the fact that all the simplices $\Sigma^{\pp_\ell}$ are compact and connected (in fact convex).
\end{proof}

\subsection{Prerequisites from Fourier Analysis} \label{sec:fourier_reordered}
This section is dedicated to reviewing some notations and results from Fourier Analysis. In the end, a quantitative result for the approximation of $C^k \left( [-1,1]^s; \RR\right)$-functions using linear combinations of multivariate Chebyshev polynomials is derived; see \Cref{app: fourier_approx}.

We start by recalling several notations and concepts from Fourier Analysis. 
\begin{definition}
    Let $s \in \NN$ and $k \in \NN_0$. We define
    \begin{equation*}
        C_{2\pi}^k\left(\RR^s; \CC\right) \defeq \left\{ f \in C^k \left(\RR^s; \CC\right) : \ \forall \pp \in \ZZ^s \ \forall x \in \RR^s : \ f(x + 2\pi \pp ) = f(x)\right\}.
    \end{equation*}
    and $C_{2\pi}\left(\RR^s; \CC\right) \defeq C_{2\pi}^0\left(\RR^s; \CC\right)$. For a function $f \in C_{2\pi}^k \left(\RR^s; \CC\right)$ we write 
\begin{align*}
 \left\Vert f \right\Vert_{C^k \left([-\pi, \pi]^s ; \CC\right)} &\defeq \underset{\vert \kk \vert \leq k}{\underset{\kk \in \NN_0^s}{\max}} \left\Vert \partial^\kk f\right\Vert_{L^\infty \left([-\pi, \pi]^s; \CC\right)} \text{ and } \\
\left\Vert f \right\Vert_{L^p \left([-\pi, \pi]^s ; \CC \right)} &\defeq \left(\frac{1}{(2\pi)^s} \cdot \int_{[-\pi, \pi]^s} \left\vert f(x) \right\vert^p dx \right)^{1/p} \text{ for } p \in [1, \infty).
\end{align*}
Moreover, we set $\Vert f \Vert_{L^\infty([-\pi, \pi]^s ; \RR)} \defeq \left\Vert f \right\Vert_{C^0 \left([-\pi, \pi]^s ; \CC\right)}$.
\end{definition}

\begin{definition}
    For any $s \in \NN$ and $\kk \in \Z^s$, we write
    \begin{equation*}
        e_\kk : \quad \RR^s \to \CC, \quad e_\kk (x) = e^{i \langle \kk, x\rangle}
    \end{equation*}
    where $\langle \cdot, \cdot \rangle$ denotes the usual inner product of two vectors. 
    For any $f \in C_{2\pi} \left(\RR^s; \CC\right)$ we define the $\kk$-th Fourier coefficient of $f$ to be
    \begin{equation*}
        \hat{f}(\kk) \defeq \frac{1}{(2\pi)^s} \int_{[-\pi, \pi]^s} f(x) \overline{e_{\kk}(x)}dx.
    \end{equation*}
\end{definition}
\begin{definition}
    For two functions $f,g \in C_{2\pi}\left(\RR^s;\CC\right)$, we define their convolution as
    \begin{equation*}
        f * g : \quad \RR^s \to \CC, \quad (f*g)(x) \defeq \frac{1}{(2\pi)^s} \int_{[-\pi, \pi]^s} f(t)g(x-t) dt.
    \end{equation*}
\end{definition}
In the following we define several so-called kernels.
\begin{definition} 
     Let $m \in \NN_0$ be arbitrary.
    \begin{enumerate}
        \item The $m$-th one-dimensional \emph{Dirichlet-kernel} is defined as
        \begin{equation*}
            D_m \defeq \sum_{h= - m}^{m} e_h.
        \end{equation*}
        \item The $m$-th one-dimensional \emph{Fejèr-kernel} is defined as
        \begin{equation*}
            F_m \defeq \frac{1}{m}\sum_{h=0}^{m-1} D_h.
        \end{equation*}
        \item The $m$-th one-dimensional \emph{de-la-Vallée-Poussin-kernel} is defined as
        \begin{equation*}
            V_m \defeq \left( 1 + e_m + e_{-m}\right) \cdot F_m.
        \end{equation*}
        \item Let $s \in \NN$. We extend the above definitions to dimension $s$ by letting
        \begin{align*}
            D_m^s \left(x_1, ..., x_s\right) &\defeq \prod_{p=1}^s D_m \left(x_p\right), \\
            F_m^s \left(x_1, ..., x_s\right) &\defeq \prod_{p=1}^s F_m \left(x_p\right), \\
            V_m^s \left(x_1, ..., x_s\right) &\defeq \prod_{p=1}^s V_m \left(x_p\right). 
        \end{align*}
    \end{enumerate}
\end{definition}
We need the following property of the multivariate extension of the de-la-Vallée-Poussin-kernel.
\begin{proposition} \label{prop: dlvp-bound}
    Let $m,s \in \NN$. Then one has $\left\Vert V_m^s \right\Vert_{L^1 \left([- \pi, \pi]^s; \CC\right)} \leq 3^s$.
\end{proposition}
\begin{proof}
   From \cite[Exercise 1.3 and Lemma 1.4]{muscalu_classical_2013} it follows $\Vert F_m\Vert_{L^1 ([-\pi, \pi] ; \CC)} = 1$ and hence using the triangle inequality $\Vert V_m \Vert_{L^1 ([-\pi,\pi] ; \CC)} \leq 3$. The claim then follows using Tonelli's theorem.
\end{proof}
The following definition introduces the term of trigonometric polynomial. 
\begin{definition}
    For any $s \in \NN$ and $m \in \NN_0$ we call a function of the form
    \begin{equation*}
    \RR^s \to \CC, \quad x \mapsto \underset{-m \leq \kk \leq m}{\sum_{\kk \in \ZZ_0^s}} a_\kk e^{i \langle \kk , x\rangle}
\end{equation*}
with coefficients $a_\kk \in \CC$ a \emph{trigonometric polynomial of coordinatewise degree at most $m$} and denote the space of all those functions with $\mathbb{H}_m^s$. Here, we consider the sum over all $\kk \in \ZZ^s$ with $-m \leq \kk_j \leq m$ for all $j \in \{1,...,s\}$. We then write
\begin{equation} \label{eq:mintrigo}
    E^s_m(f) \defeq \underset{T \in \mathbb{H}_{m}^s}{\mathrm{min}} \left\Vert f - T\right\Vert_{L^\infty \left(\RR^s;\CC\right)}
\end{equation}
for any function $f \in C_{2\pi} \left(\RR^s ; \CC\right)$.
\end{definition}
The following proposition shows that convolving with the Fejèr kernel produces a trigonometric polynomial of degree at most $2m-1$, while reproducing trigonometric polynomials of degree $m$. Furthermore, the norm of the convolution operator is bounded uniformly in $m$. These properties will be useful for our proof of \Cref{app: fourier_approx}.
\begin{proposition}
\label{vm}
    Let $s,m \in \NN$ and $k \in \NN_0$. The map
    \begin{equation*}
        v_m : \quad C_{2\pi} \left(\RR^s; \CC\right) \to \mathbb{H}_{2m-1}^s, \quad f \mapsto f * V_m^s 
    \end{equation*}
    is well-defined and satisfies
    \begin{equation}
    \label{ident}
        v_m(T) = T \quad \text{for all} \quad T \in \mathbb{H}_m^s.
    \end{equation}
    Furthermore, there exists a constant $c = c(s) > 0$ (independent of $m$), such that
    \begin{align}
        \left\Vert v_m(f)\right\Vert_{C^k\left([-\pi,\pi]^s; \CC\right)} \leq c \cdot \left\Vert f\right\Vert_{C^k\left([-\pi,\pi]^s; \CC\right)} \ \forall f \in C^k_{2\pi} \left(\RR^s; \CC\right), \nonumber \\
        \label{constant}
        \left\Vert v_m(f)\right\Vert_{L^\infty \left([\pi,\pi]^s; \CC\right)} \leq c \cdot \left\Vert f\right\Vert_{L^\infty \left([-\pi, \pi]^s; \CC\right)} \ \forall f\in C_{2\pi} \left(\RR^s; \CC\right).
    \end{align}
    In fact, it holds $c(s) \leq \exp(C \cdot s)$ with an absolute constant $C>0$.
\end{proposition}
\begin{proof}
    A direct computation shows that $f \ast e_{\kk} = \hat{f}(\kk) \cdot e_{\kk}$. This implies that $v_m$ is well-defined since $V_m^s$ is a trigonometric polynomial of coordinatewise degree at most $2m-1$. 

    The operator is bounded on $C^k_{2\pi}(\RR^s;\CC)$ and $C_{2\pi}(\RR^s;\CC)$ with norm at most $c = 3^s$, as follows from Young's inequality \cite[Lemma 1.1 (ii)]{muscalu_classical_2013}, \Cref{prop: dlvp-bound}, and the fact that one has for all $\kk \in \NN_0^s$ with $\vert \kk \vert \leq k$ the identity
    \begin{equation*}
         \partial^\kk \left( f * V_m^s\right) = \left(\partial^\kk f\right) * V_m^s \quad \text{for } f \in C^k_{2\pi}(\RR^s ; \CC).
    \end{equation*}
    It remains to show that $v_m$ is the identity on $\mathbb{H}_m^s$. We first prove that 
    \begin{equation}
    \label{firstident}
        e_k * V_m = e_k
    \end{equation}
    holds for all $k \in \Z$ with $\vert k \vert \leq m$. First note that
    \begin{equation*}
        e_k * V_m = e_k * F_m + e_k * \left(e_m \cdot F_m\right) + e_k * \left(e_{-m} \cdot F_m\right).
    \end{equation*}
    We then compute
    \begin{align*}
        e_k * F_m = \frac{1}{m} \sum_{\ell = 0}^{m-1} D_\ell * e_k = \frac{1}{m}\sum_{\ell = 0}^{m-1} \sum_{h = - \ell}^{\ell} \underbrace{e_h * e_k}_{= \delta_{k,h} \cdot e_k} = \frac{1}{m} \sum_{\ell = \vert k \vert}^{m-1} e_k = \frac{m - \vert k \vert}{m} \cdot e_k.
    \end{align*}
    Similarly, we get
    \begin{align*}
        e_k *\left( e_m \cdot F_m \right) &= \frac{1}{m} \sum_{\ell = 0}^{m-1} \left( e_m D_\ell \right) * e_k = \frac{1}{m}\sum_{\ell = 0}^{m-1} \sum_{h = - \ell}^{\ell} \underbrace{e_{h+m} * e_k}_{= \delta_{k,h+m} \cdot e_k}\\
        & = \frac{1}{m} \underset{\ell \geq m-k}{\sum_{0 \leq \ell \leq m-1 }} e_k = \delta_{k \geq 1} \cdot \frac{k}{m} \cdot e_k
    \end{align*}
    and 
    \begin{align*}
        e_k *\left( e_{-m} \cdot F_m \right) &= \frac{1}{m} \sum_{\ell = 0}^{m-1} \left( e_{-m} D_\ell \right) * e_k = \frac{1}{m}\sum_{\ell = 0}^{m-1} \sum_{h = - \ell}^{\ell} \underbrace{e_{h-m} * e_k}_{= \delta_{k,h-m} \cdot e_k} \\
        &= \frac{1}{m} \underset{\ell \geq k+m}{\sum_{0 \leq \ell \leq m-1}} e_k = \delta_{k \leq -1} \cdot \frac{-k}{m} \cdot e_k.
    \end{align*}
    Adding up those three identities yields (\ref{firstident}). 

    To finally prove \eqref{ident}, it clearly suffices to show that 
    \begin{equation*}
        e_\kk * V_m^s = e_\kk
    \end{equation*}
    for all $\kk \in \Z^s$ with $-m \leq \kk \leq m$. But for such $\kk$, using $e_\kk(x) = \prod_{j=1}^{s} e_{\kk_j}\left( x_j\right)$, one obtains
    \begin{align*}
        \left(e_\kk * V_m^s\right)(x) &= \frac{1}{(2\pi)^s} \int_{[-\pi, \pi]^s} \prod_{j=1}^{s} e_{\kk_j} \left(t_j\right) \cdot V_m \left(x_j - t_j\right) dt \\
        \overset{\text{Fubini}}&{=} \prod_{j=1}^s \left(e_{\kk_j} * V_m\right)\left(x_j \right) 
        \overset{\eqref{firstident}}{=} \prod_{j=1}^s e_{\kk_j} \left(x_j\right) 
        = e_\kk(x)
    \end{align*}
    for any $ x \in \RR^s$, as was to be shown.
\end{proof}
The following result follows from a theorem in \cite{lorentz_approximation_2005}.
\begin{proposition}
\label{lorentz_aussage}
    Let $s,k \in \NN$. Then there exists a constant $c = c(s,k) > 0$, such that, for $E_m^s$ as defined in \eqref{eq:mintrigo},
    \begin{equation*}
        E_m^s(f) \leq \frac{c}{m^k} \cdot \left\Vert f\right\Vert_{C^k\left([-\pi, \pi]^s; \RR\right)}
    \end{equation*}
    for all $m\in \NN$ and $f \in C^k_{2\pi} \left(\RR^s; \RR\right)$. \newline
    In fact, it holds $c(s,k) \leq \exp(C \cdot ks) \cdot k^k $ with an absolute constant $C>0$.
\end{proposition}
\begin{proof}
    We apply \cite[Theorem 6.6]{lorentz_approximation_2005} with $n_i = m$ and $p_i = k$, which yields the existence of a constant $c_1 = c_1(s,k)>0$, such that
    \begin{equation*}
        E_m^s(f) \leq c_1 \cdot \sum_{\ell=1}^s \frac{1}{m^k} \cdot \omega_{\ell} \left(f,\frac{1}{m}\right)
    \end{equation*}
    for all $m \in \NN$ and $f \in C^k_{2\pi}(\RR^s ; \RR)$, where $\omega_\ell(f, \bullet)$ denotes the modulus of continuity of $\frac{\partial^k f }{\partial x_\ell ^k}$ with respect to $x_\ell$, where we have the trivial bound
    \begin{equation*}
         \omega_\ell \left(f,\frac{1}{m}\right)  \leq 2 \cdot \left\Vert f \right\Vert_{C^k\left([-\pi, \pi]^s; \RR\right)}.
    \end{equation*}
    Hence, we get
    \begin{equation*}
        E_m^s(f) \leq c_1 \cdot s \cdot 2 \cdot \left\Vert f \right\Vert_{C^k\left([-\pi, \pi]^s; \RR\right)} \frac{1}{m^k},
    \end{equation*}
    so the claim follows by choosing $c := 2s \cdot c_1$.
    
    We refer to \Cref{sec:const_bound_reordered} (see \Cref{thm:const_lorentz_bound}) for a proof of the claimed bound on the constant $c(s,k)$.
\end{proof}
The above proposition bounds the best possible error of approximating $f$ by trigonometric polynomials of coordinatewise degree at most $m$, but this is in general non-constructive. Our next result shows that a similar bound holds for approximating $f$ by $v_m(f)$.
\begin{theorem}
\label{vm_approx}
    Let $s \in \NN$. Then there exists a constant $ c = c(s) > 0 $, such that the operator $v_m$ from \Cref{vm} satisfies
    \begin{equation*}
        \left\Vert f - v_m(f) \right\Vert_{L^\infty \left( \RR^s\right)} \leq c \cdot E^s_m(f)
    \end{equation*}
    for any $m \in \NN$ and $f \in C_{2\pi} \left(\RR^s; \CC\right)$. \newline
    In fact, it holds $c(s) \leq \exp(C \cdot s)$ with an absolute constant $C>0$.
\end{theorem}
\begin{proof}
    For any $T \in \mathbb{H}_m^s$ one has
    \begin{equation*}
        \left\Vert f - v_m(f) \right\Vert_{L^\infty \left( \RR^s\right)} \overset{\eqref{ident}}{\leq} \left\Vert f - T \right\Vert_{L^\infty \left( \RR^s\right)} + \left\Vert v_m(T) - v_m(f) \right\Vert_{L^\infty \left( \RR^s\right)} \overset{\eqref{constant}}{\leq} (c+1) \left\Vert f - T \right\Vert_{L^\infty \left( \RR^s\right)}. 
    \end{equation*}
    Taking the infimum over all $T \in \mathbb{H}_m^s$ yields the claim. 
\end{proof}
By combining \Cref{lorentz_aussage} and \Cref{vm_approx}, we immediately get the following bound.
\begin{corollary}
\label{dlvp}
    Let $s,k \in \NN_0$. Then there exists a constant $c = c(s,k)>0$, such that
    \begin{equation*}
        \left\Vert f - v_m(f) \right\Vert_{L^\infty \left(\RR^s\right)} \leq \frac{c}{m^k} \cdot \left\Vert f \right\Vert_{C^k \left([-\pi,\pi]^s; \RR\right)}
    \end{equation*}
    for every $m \in \NN$ and $f \in C^k_{2\pi} \left(\RR^s ; \RR\right)$. \newline
    In fact, we have $c(s,k) \leq \exp(C \cdot ks) \cdot k^k$ with an absolute constant $C>0$.
\end{corollary}
Up to now, we have studied the approximation of periodic functions by trigonometric polynomials, but our actual goal is to approximate non-periodic functions by algebraic polynomials. The next lemma establishes a connection between the two settings. 
\begin{lemma}
\label{star_operator}
    Let $k \in \NN_0$ and $ s \in \NN$. For any function $f \in C^k \left([-1,1]^s ; \CC\right)$, we define the corresponding periodic function via
    \begin{equation*}
        f^* : \quad \RR^s \to \CC, \quad  f^* \left(x_1, ..., x_s\right) = f(\mathrm{cos} \left(x_1\right), ..., \mathrm{cos} \left( x_s\right))
    \end{equation*}
    and note $f^* \in C_{2\pi}^k \left(\RR^s; \CC\right)$. The map
\begin{equation*}
	C^k \left([-1,1]^s ; \CC\right) \to C^k _ {2\pi}\left(\RR^s; \CC\right), \quad f \mapsto f^*
\end{equation*}
is a continuous linear operator with respect to the $C^k$-norms on $C^k \left([-1,1]^s ; \CC\right)$ and $C^k _ {2\pi}\left(\RR^s; \CC\right)$. \newline
The operator norm can be bounded from above by $k^k$.
\end{lemma}
\begin{proof}
	The map is well-defined since $\cos$ is a smooth function and $2\pi$-periodic. The linearity of the operator is obvious, so it remains to show its continuity.

	The goal is to apply the closed graph theorem \cite[Theorem 5.12]{folland_real_1999}. By definition of $f^*$, and since $\cos:[-\pi,\pi] \to [-1,1]$ is surjective, we have the equality $\left\Vert f\right\Vert_{L^\infty \left([-1, 1]^s; \CC\right)} = \left\Vert f^* \right\Vert_{L^\infty \left([-\pi, \pi]^s; \CC\right)} $. Let then $\left(f_n\right)_{n \in \NN}$ be a sequence of functions $f_n \in C^k \left([-1,1]^s ; \CC\right)$ and $g^* \in C^k_{2\pi}\left(\RR^s ; \CC\right)$ such that $f_n \to f$ in $C^k \left([-1,1]^s; \CC\right)$ and $f_n^* \to g^*$ in $C^k_{2\pi} \left(\RR^s; \CC\right)$. We then have
\begin{align*}
	\left\Vert f^* - g^* \right\Vert_{L^\infty \left([-\pi, \pi]^s\right)} &\leq \left\Vert f^* - f_n^* \right\Vert_{L^\infty \left([-\pi, \pi]^s\right)} + \left\Vert f_n^* - g^* \right\Vert_{L^\infty \left([-\pi, \pi]^s\right)} \\
&= \left\Vert f - f_n \right\Vert_{L^\infty  \left([-1, 1]^s ; \CC\right)} + \left\Vert f_n^* - g^* \right\Vert_{L^\infty \left([-\pi, \pi]^s\right)} \\
&\leq \left\Vert f - f_n \right\Vert_{C^k  \left([-1, 1]^s ; \CC\right)} + \left\Vert f_n^* - g^* \right\Vert_{C^k([-\pi, \pi]^s ; \CC)} \to 0 \ (n \to \infty).
\end{align*}
It follows $f^* = g^*$ and the closed graph theorem yields the desired continuity. 

We refer to \Cref{sec:const_bound_reordered} (see \Cref{thm:faa,rem:multiindex}) for a proof of the claimed bound on the operator norm.
\end{proof}

For a function $f \in C^k([-1,1]^s;\CC)$ we want to express $v_m(f^*)$ in a convenient way, involving a product of cosines. To this end, we make use of the following identity, which is a generalization of the well-known product-to-sum formula for $\cos$.
\begin{lemma}
\label{prod_sum}
Let $s \in \NN$. Then it holds for any $x \in \RR^s$ that
\begin{equation*}
\prod_{j=1}^s \cos (x_j) = \frac{1}{2^s} \sum_{\sigma \in \{-1,1\}^s} \cos (\langle\sigma,x \rangle).
\end{equation*} 
\end{lemma}
\begin{proof}
This is an inductive generalization of the product-to-sum formula 
\begin{equation}
\label{product_sum_form}
2\cos(x) \cos(y) = \cos(x-y) + \cos (x+y)
\end{equation}
for $x,y \in \RR$, which can be found for instance in \cite[Eq.~4.3.32]{abramowitz_handbook_2013}. The case $s= 1$ holds since $\cos$ is an even function. Assume that the claim holds for a fixed $s \in \NN$ and take $x \in \RR^{s+1}$. Writing $x' = (x_1, ..., x_s)$, we derive
\begin{align*}
\prod_{j=1}^{s+1} \cos(x_j) &= \left(\frac{1}{2^s} \sum_{\sigma \in \{-1,1\}^s} \cos (\langle \sigma , x' \rangle) \right) \cdot \cos(x_{s+1}) \\
&= \frac{1}{2^s} \sum_{\sigma \in \{-1,1\}^s}\cos (\langle \sigma, x' \rangle) \cos(x_{s+1})\\
\overset{\eqref{product_sum_form}}&{=} \frac{1}{2^{s+1}} \sum_{\sigma \in \{-1,1\}^s} \left[\cos (\langle \sigma, x' \rangle + x_{s+1}) + \cos (\langle \sigma, x' \rangle - x_{s+1}) \right]  \\
&= \frac{1}{2^{s+1}} \sum_{\sigma \in \{-1,1\}^{s+1}} \cos (\langle \sigma, x\rangle), 
\end{align*}
as was to be shown. 
\end{proof}
The following proposition states that $v_m(f^*)$ can be expressed as a linear combination of products of cosines. This representation is useful since these cosines can be interpolated by Chebyshev polynomials which in the end leads to the desired approximation result.
\let \hat \widehat
\begin{proposition}
\label{representation}
    Let $s\in \NN$ and $k \in \NN_0$. For any $f \in C^k \left([-1,1]^s ; \CC\right)$ and $m \in \NN$ the de-la-Vallée-Poussin operator given as $f \mapsto v_m \left(f^*\right)$ with $v_m$ as in \Cref{vm} and $f \mapsto f^*$ as in \Cref{star_operator} has a representation
    \begin{equation*}
        v_m \left(f^*\right) \left(x_1, ..., x_s\right) = \underset{\kk \leq 2m -1}{\sum_{\kk \in \NN_0^s}} \mathcal{V}_\kk^m(f) \prod_{j=1}^{s} \mathrm{cos} \left(\kk_j x_j\right)
    \end{equation*}
    for continuous linear functionals
    \begin{equation*}
        \mathcal{V}_{\kk}^m : \ C^k \left([-1,1]^s; \CC\right) \to \CC, \quad f \mapsto 2^{\Vert \kk \Vert_0}\cdot  a_{\kk}^m \cdot \widehat{f^*}(\kk),
    \end{equation*}
where $\Vert \kk \Vert_0 = \# \{j \in \{1,...,s\}: \ \kk_j \neq 0\}$ and $a_{\kk}^m = \widehat{V_m^s}(\kk)$. Furthermore, if $f \in C^k ([-1,1]^s ; \RR)$, then $\mathcal{V}_\kk^m (f) \in \RR$ for every $\kk \in \NN_0^s$ with $\kk \leq 2m-1$.
\end{proposition}
\begin{proof}
    First of all, it is easy to see that $v_m \left( f^* \right)$ is even in each variable, which follows directly from the fact that $ f^*$ and $V_m^s$ are both even in each variable. Furthermore, if we write 
    \begin{equation*}
        V_m^s = \underset{-(2m-1) \leq \kk \leq 2m-1}{\sum_{\kk \in \ZZ^s}} a_\kk^m e_\kk
    \end{equation*}
    with appropriately chosen coefficients $a_\kk^m \in \RR$, we easily see
    \begin{equation*}
        v_m \left(f^*\right) = \underset{-(2m-1) \leq \kk \leq 2m-1}{\sum_{\kk \in \ZZ^s}} a_\kk^m \hat{f^*}(\kk) e_\kk.
    \end{equation*}
    Using Euler's identity and the fact that $v_m\left(f^*\right)$ is an even function, we get the representation 
    \begin{equation*}
        v_m \left(f^*\right)(x) = \underset{-(2m-1)\leq \kk \leq 2m-1}{\sum_{\kk \in \ZZ^s}} a_\kk^m \hat{f^*}(\kk) \text{cos}(\langle \kk, x\rangle)
    \end{equation*}
    for all $x \in \RR^s$. Using $\odot$ to denote the componentwise product of two vectors of the same size, i.e., $x \odot y = (x_i \cdot y_i)_i$, and using the identity $\langle \kk, \sigma \odot x\rangle = \langle \sigma, \kk \odot x \rangle$, we see since $v_m \left(f^*\right)$ is even in every variable that
    \begin{align*}
        v_m \left(f^*\right) (x) &= \frac{1}{2^s} \cdot \sum_{\sigma \in \{-1,1\}^s} v_m \left( f^* \right) (\sigma \odot x) \\
        &=\frac{1}{2^s} \cdot \sum_{\sigma \in \{-1,1\}^s} \underset{-(2m-1) \leq \kk \leq 2m-1}{\sum_{\kk \in \ZZ^s}} a_\kk^m \hat{f^*}(\kk) \text{cos}(\langle \kk, \sigma \odot x\rangle) \\
        &= \underset{-(2m-1) \leq \kk \leq 2m-1}{\sum_{\kk \in \ZZ^s}} \left(a_\kk^m \hat{f^*}(\kk) \frac{1}{2^s} \sum_{\sigma \in \{-1,1\}^s}\text{cos}(\langle \sigma, \kk \odot x\rangle)\right) \\
        \overset{\text{\Cref{prod_sum}}}&{=} \underset{-(2m-1) \leq \kk \leq 2m-1}{\sum_{\kk \in \ZZ^s}} a_\kk^m \hat{f^*}(\kk) \prod_{j=1}^{s} \cos \left(\kk_j x_j\right) \\
        &= \underset{\kk \leq 2m-1}{\sum_{\kk \in \NN_0^s}} 2^{\Vert \kk \Vert_0} a_\kk^m \hat{f^*}(\kk) \prod_{j=1}^{s} \cos \left(\kk_j x_j\right)
    \end{align*}
    with
    \begin{equation*}
        \left\Vert \kk \right\Vert_0 \defeq \#\big\{ j \in \{1,...,s\} : \ \kk_j \neq 0\big\}.
    \end{equation*}
    In the last step we again used that cos is an even function and that
    \begin{equation*}
        \hat{f^*}(\kk) = \hat{f^*}(\sigma \odot \kk)
    \end{equation*}
    for all $\sigma \in \{-1,1\}^s$, which also follows easily since $f^*$ and $V_m^s$ are even in every component. Letting \begin{equation*}
        \mathcal{V}_\kk^m(f) \defeq 2^{\Vert \kk \Vert_0} a_\kk^m \hat{f^*}(\kk),
    \end{equation*} 
    we have the desired form. The fact that $\mathcal{V}_{\kk}^m$ is a continuous linear functional on $C^k_{2\pi}\left([-1,1]^s;\CC\right)$ follows directly since $f \mapsto \widehat{f^*}(\kk)$ is a continuous linear functional for every $\kk$. If $f$ is real-valued, so is $\hat{f^*}(\kk)$ for every $\kk \in \NN_0^s$ with $\kk \leq 2m-1$, since $f^*$ is real-valued and even in every component.
\end{proof}
\let \hat \hat

Our main approximation result involves linear combinations of Chebyshev polynomials where the coefficients in this linear combination are given as $\mathcal{V}_{\kk}^m(f)$. It is therefore important to be able to bound the sum of the absolute values $\vert \mathcal{V}_{\kk}^m(f) \vert$.
\begin{lemma}
\label{sum_bound}
    Let $s \in \NN$. There exists a constant $c = c(s)> 0$, such that the inequality
    \begin{equation*}
        \underset{\kk \leq 2m-1}{\sum_{\kk \in \NN_0^s}} \left\vert \mathcal{V}_\kk^m(f) \right\vert \leq c \cdot m^{s/2} \cdot \left\Vert f \right\Vert_{L^\infty \left([-1, 1]^s ; \CC\right)}
    \end{equation*}
    holds for all $m \in \NN$ and $f \in C \left([-1, 1]^s ; \CC\right)$, where $\mathcal{V}_\kk^m$ is as in \Cref{representation}. \newline
    In fact, we have $c(s) \leq \exp(C \cdot s)$ with an absolute constant $C>0$.
\end{lemma}
\begin{proof}
    Let $f \in C \left([-1, 1]^s ; \CC\right)$ and $m \in \NN$. For any multi-index $\elll \in \NN_0^s$, it follows from \Cref{representation} that 
    \begin{equation*}
        \widehat{v_m\left(f^*\right)} (\elll) = \underset{\kk \leq 2m-1}{\sum_{\kk \in \NN_0^s}} \mathcal{V}_\kk^m(f) \widehat{g_\kk}(\elll),
    \end{equation*}
    with
    \begin{equation*}
        g_\kk : \quad \RR^s \to \RR, \quad \left(x_1, ..., x_s\right) \mapsto \prod_{j=1}^s \cos \left(\kk_j x_j\right).
    \end{equation*}
    Now, a calculation using Fubini's theorem and using 
    \begin{equation*}
    g_{\kk} = \prod_{j=1}^s\frac{1}{2} \left( e_{\kk_j} + e_{-\kk_j}\right) = \underset{\kk_j \neq 0}{\prod_{1 \leq j \leq s}} \frac{1}{2} \left(e_{\kk_j} + e_{-\kk_j}\right)
    \end{equation*} for any number $k \in \NN_0$ shows 
    \begin{equation*}
        \widehat{g_\kk}(\elll) =  \begin{cases} \frac{1}{2^{\Vert \kk \Vert_0}},&\kk=\elll, \\ 0, & \text{otherwise}\end{cases} \quad\text{for } \kk, \elll \in \NN_0^s.
    \end{equation*}
    Therefore, we have the bound $\left\vert\mathcal{V}^m_{\elll} (f)\right\vert \leq 2^s \cdot \left\vert \widehat{v_m\left(f^*\right)}(\elll)\right\vert$ for $\elll \in \NN_0^s $ with $\vert \elll \vert \leq 2m-1$. Using the Cauchy-Schwarz and the Parseval inequality, we therefore see
    \begin{align*}
        \underset{\kk \leq 2m-1}{\sum_{\kk \in \NN_0^s}} \left\vert \mathcal{V}_\kk^m(f)\right\vert &\leq 2^s \cdot \underset{\kk \leq 2m-1}{\sum_{\kk \in \NN_0^s}} \left\vert \widehat{v_m\left(f^*\right)}(\kk)\right\vert \overset{\text{CS}}{\leq} 2^s \cdot (2m)^{s/2} \cdot \left(\underset{\kk \leq 2m-1}{\sum_{\kk \in \NN_0^s}} \left\vert \widehat{v_m\left(f^*\right)}(\kk)\right\vert^2 \right)^{1/2} \\
        \overset{\text{Parseval}}&{\leq} 2^s \cdot 2^{s/2} \cdot m^{s/2} \cdot \left\Vert v_m \left(f^*\right)\right\Vert_{L^2 \left([-\pi, \pi]^s; \CC\right)} \\
        &\leq \underbrace{2^s \cdot 2^{s/2} }_{=: c_1(s)}\cdot m^{s/2} \cdot \left\Vert v_m \left(f^*\right)\right\Vert_{L^\infty \left([-\pi, \pi]^s; \CC\right)}.
    \end{align*}
    Using \Cref{vm}, we get a constant $c_2(s) \leq \exp(C_0 \cdot s)$ such that
    \begin{align*}
        \underset{\kk \leq 2m-1}{\sum_{\kk \in \NN_0^s}} \left\vert \mathcal{V}_\kk^m(f)\right\vert &\leq c_1(s) \cdot c_2(s) \cdot m^{s/2} \cdot \left\Vert f^*\right\Vert_{L^\infty \left([-\pi, \pi]^s;\CC\right)} = c(s) \cdot m^{s/2} \cdot \left\Vert f\right\Vert_{L^\infty \left([-1, 1]^s; \CC\right)},
    \end{align*}
    as claimed.
\end{proof}

For any natural number $\ell \in \NN_0$, we denote by $T_\ell$ the $\ell$-th \emph{Chebyshev polynomial}, satisfying
\begin{equation*}
    T_\ell\left(\cos(x)\right) = \cos(\ell x), \quad x \in \RR.
\end{equation*}
One can show that $T_\ell$ is in fact a polynomial of degree $\ell$. For a multi-index $\kk \in \NN_0^s$, we define
\begin{equation*}
    T_\kk (x) \defeq \prod_{j=1}^s T_{\kk_j}\left(x_j\right), \quad x \in \RR^s.
\end{equation*}
We then get the following approximation result about approximating (non-periodic) $C^k$-functions by linear combinations of Chebyshev polynomials. 
\begin{theorem} \label{app: fourier_approx}
    Let $k,s,m \in \NN$. Then there exists a constant $c=c(s,k)>0$ with the following property: For any $f \in C^k \left([-1,1]^s; \RR\right)$ the polynomial $P$ defined as
    \begin{equation*}
        P(x) \defeq \underset{\kk \leq 2m-1}{\sum_{\kk \in \NN_0^s}}\mathcal{V}_\kk^m(f) \cdot T_\kk(x),
    \end{equation*}
    with $\mathcal{V}^m_\kk$ as in \Cref{representation}, satisfies
    \begin{equation*}
        \left\Vert f - P \right\Vert_{L^\infty \left([-1,1]^s ;\RR\right)} \leq \frac{c}{m^k} \cdot \left\Vert f \right\Vert_{C^k\left([-1,1]^s;\RR\right)}.
    \end{equation*}
    Here, the maps
\begin{equation*}
C\left([-1,1]^s ; \RR\right) \to \RR, \quad f \mapsto \mathcal{V}_{\kk}^m(f)
\end{equation*}
are continuous and linear functionals with respect to the $L^\infty$-norm. Furthermore, there exists a constant $\tilde{c} = \tilde{c}(s)> 0$, such that the inequality
    \begin{equation*}
        \underset{\kk \leq 2m-1}{\sum_{\kk \in \NN_0^s}} \left\vert \mathcal{V}_\kk^m(f) \right\vert \leq \tilde{c} \cdot m^{s/2} \cdot \left\Vert f \right\Vert_{L^\infty \left([-1, 1]^s;\RR\right)}
    \end{equation*}
    holds for all $f \in C \left([-1, 1]^s ; \RR\right)$. \newline
    Moreover, we have $c(s,k) \leq \exp(C \cdot ks) \cdot k^{2k}$ and $\tilde{c}(s) \leq \exp(C \cdot s)$ with an absolute constant $C>0$.
    
\end{theorem}
\begin{proof}
    We choose the constant $c_0 = c_0(s,k)$ according to \Cref{dlvp}. Let $f \in C^k\left([-1,1]^s;\RR\right)$ be arbitrary. Then we define the corresponding function $f^* \in C^k_{2\pi}\left(\RR^s; \RR\right)$ as above. Let $P$ be defined as in the statement of the theorem. Then it follows from the definition of the Chebyshev polynomials $T_\kk$, the definition of $P$, and the formula for $v_m(f^*)$ from \Cref{representation} that 
    \begin{equation*}
        P^* (x) = v_m\left(f^*\right)(x)
    \end{equation*}
    is satisfied, where $P^*$ is the corresponding function to $P$ defined similarly to $f^*$. Overall, we get the bound
    \begin{align*}
        \left\Vert f - P \right\Vert_{L^\infty \left([-1,1]^s; \RR\right)} = \left\Vert f^* - P^* \right\Vert_{L^\infty \left([-\pi,\pi]^s; \RR\right)} \overset{\text{Cor. \ref{dlvp}}}{\leq} \frac{c_0}{m^k} \cdot \left\Vert f^* \right\Vert_{C^k\left([-\pi, \pi]^s; \RR\right)}.
    \end{align*}
   The first claim then follows using the continuity of the map $f \mapsto f^*$ as proven in \Cref{star_operator}. The second part of the theorem has already been proven in \Cref{sum_bound}.
\end{proof}
% !TeX encoding = UTF-8
% !TeX spellcheck = en_US
% !TeX root = main_paper.tex

\section{Prerequisites for the optimality result}
In this subsection we prove a very general lower bound for the approximation of functions in $C^k ([-1,1]^s ; \RR)$ using a subset of $C([-1,1]^s; \RR)$ that can be parametrized using a certain amount of parameters. Precisely, we prove a lower bound of $m^{-k/s}$ where $m$ is the number of parameters that are used, provided that the selection of the parameters, i.e., the map that maps a function in $C^k([-1,1]^s; \RR)$ to the parameters of the approximating function, is continuous with respect to some norm on $C^k ([-1,1]^s ; \RR)$. The proofs are in fact almost identical to what is done in \cite{devore_optimal_1989}. However, we decided to include a detailed proof in this paper, since \cite{devore_optimal_1989} considers Sobolev functions and not $C^k$-functions and since the continuity assumption in \cite{devore_optimal_1989} is not completely clear. 

\begin{proposition}[{\cite[Theorem 3.1]{devore_optimal_1989}}] \label{thm: devore}
Let $(X, \Vert \cdot \Vert_X)$ be a normed space, $\emptyset \neq K \subseteq X$ a subset and $V \subseteq X$ a linear, not necessarily closed subspace of $X$ containing K. Let $m\in \NN$, $\overline{a} : K \to \RR^m$ be a map which is continuous with respect to some norm $\Vert \cdot \Vert_V$ on $V$ and $M_m : \RR^m \to X$ some arbitrary map. Let
\begin{equation*}
b_m(K)_X \defeq \underset{X_{m+1}}{\sup} \sup \left\{\varrho \geq 0: \  U_\varrho(X_{m+1}) \subseteq K\right\},
\end{equation*} 
where the first supremum is taken over all $(m+1)$-dimensional linear subspaces $X_{m+1}$ of $X$ and
\begin{equation*}
U_\varrho(X_{m+1}) \defeq \{y \in X_{m+1} : \ \Vert y \Vert_X \leq \varrho\}.
\end{equation*}
Then it holds
\begin{equation*}
\underset{x \in K}{\sup} \Vert x - M_m (\overline{a}(x)) \Vert_X \geq b_m(K)_X.
\end{equation*}
\end{proposition}

\begin{proof}
The claim is trivial if $b_m(K)_X = 0$. Thus, assume $b_m(K)_X > 0$. Let $0 < \varrho \leq b_m(K)_X$ be any number such that there exists an $(m+1)$-dimensional subspace $X_{m+1}$ of $X$ with $ U_\varrho(X_{m+1}) \subseteq K$. It follows $U_\varrho(X_{m+1}) \subseteq V$, hence $X_{m+1} \subseteq V$, so $\Vert \cdot \Vert_V$ defines a norm on $X_{m+1}$. Thus, the restriction of $\overline{a}$ to $\partial  U_\varrho(X_{m+1})$ is a continuous mapping to $\RR^m$ with respect to $\Vert \cdot \Vert_V$. Since all norms are equivalent on the finite-dimensional space $X_{m+1}$, the Borsuk-Ulam-Theorem \cite[Corollary 4.2]{deimling2013nonlinear} yields the existence of a point $x_0 \in \partial U_\varrho(X_{m+1})$ with $\overline{a}(x_0) = \overline{a}(-x_0)$. We then see
\begin{align*}
2\varrho &= 2\Vert x_0\Vert_X =\Vert x_0 - M_m(\overline{a}(x_0))\Vert_X  + \Vert x_0 +M_m(\overline{a}(-x_0)) \Vert_X\\ 
&\leq \Vert x_0 - M_m (\overline{a}(x_0)) \Vert_X + \Vert - x_0 - M_m(\overline{a}(-x_0))\Vert_X,
\end{align*}
and hence, at least one of the two summands on the right has to be larger than or equal to $\varrho$.
\end{proof}

Using this very general result we can deduce our lower bound in the context of $C^k$-spaces. 
\begin{theorem} \label{app: devore_real}
Let $s,k \in \NN$. Then there exists a constant $c = c(s,k)>0$ with the following property:  For any $m \in \NN$ and any map $\overline{a} : C^k ([-1,1]^s; \RR) \to \RR^m$ that is continuous with respect to some norm on $C^k([-1,1]^s; \RR)$ and any map $M_m  : \RR^m \to C([-1,1]^s ; \RR)$ we have
\begin{equation*}
\underset{\Vert f \Vert _{C^k ([-1,1]^s ; \RR)} \leq 1}{\underset{f \in C^k([-1,1]^s ; \RR)}{\sup}} \Vert f - M_m (\overline{a}(f)) \Vert_{L^\infty ([-1,1]^s ; \RR)} \geq c \cdot m^{-k/s}.
\end{equation*}
\end{theorem}
\begin{proof}
The idea is to apply \Cref{thm: devore} to the spaces $X \defeq C([-1,1]^s ; \RR)$, $V \defeq C^k ([-1,1]^s ; \RR)$ and the set $K \defeq \{f \in C^k ([-1,1]^s ; \RR): \  \Vert f \Vert_{C^k ([-1,1]^s ; \RR)} \leq 1\}$. 

Assume in the beginning that $m = n^s$ with an integer $n >1$. Pick $\phi \in C^\infty(\RR^s)$ with $\phi \equiv 1$ on $[-3/4, 3/4]^s$ and $\phi \equiv 0$ outside of $[-1,1]^s$. Fix $c_0 = c_0(s,k) > 0$ with 
\begin{equation*}
1 \leq \Vert \phi \Vert_{C^k([-1,1]^s ; \RR) } \leq c_0.
\end{equation*} 
Let $Q_1, ..., Q_m$ be the partition (disjoint up to null-sets) of $[-1,1]^s$ into closed cubes of sidelength $2/n$. For every $j \in \{1,...,m\}$ write $Q_j = \bigtimes_{\ell = 1}^{s} [a_\ell - 1/n, a_\ell + 1/n]$ with some vector $a = (a_1, ..., a_s) \in [-1,1]^s$ and let 
\begin{equation*}
\phi_j (x) \defeq \phi (nx - na) \text{ for } x \in \RR^s.
\end{equation*}
By choice of $\phi$, the maps $\phi_j$ are supported on a proper subset of $Q_j$ for every $j \in \{1,...,m\}$ and an inductive argument shows
\begin{equation*} 
\partial^\kk \phi_j (x) = n^{\vert \kk \vert} \cdot \partial^\kk \phi (nx - na) \text{ for every } \kk \in \NN_0^s \text{ and }x \in \RR^s
\end{equation*}
and hence in particular
\begin{equation} \label{eq: derivative}
\Vert \phi_j \Vert_{C^k([-1,1]^s ; \RR)} \leq n^{\vert \kk \vert} \cdot c_0.
\end{equation}
Let $X_m \defeq \spann \{\phi_1, ..., \phi_m\}$ and $S \in U(X_m)$. Then we can write $S$ in the form $S= \sum_{j = 1}^m c_j \phi_j$ with real numbers $c_1, ..., c_m \in \RR$. Suppose there is $j^* \in \{1,...,m\}$ with $\vert c_{j^*}\vert  > 1$. Then we have
\begin{equation*}
\Vert S \Vert_{L^\infty([-1,1]^s ; \RR)} \geq \underbrace{\vert c_{j^*} \vert}_{> 1} \cdot \underbrace{\Vert \phi_{j^*} \Vert_{L^\infty([-1,1]^s ; \RR)}}_{ = 1} > 1,
\end{equation*}
since the functions $\phi_j$ have disjoint support. This is a contradiction to $S \in U(X_m)$ and we can thus infer that $\underset{j}{\max} \ \vert c_j \vert \leq 1$. Furthermore, we see again because the functions $\phi_j$ have disjoint support that
\begin{align*}
\Vert \partial^\kk S \Vert_{L^\infty ([-1,1]^s ; \RR)} = \underset{j}{\max} \ \vert c_j \vert \cdot \Vert \partial^\kk \phi_j \Vert_{L^\infty([-1,1]^s ; \RR)} \overset{\eqref{eq: derivative}}{\leq} n^{\vert \kk \vert} \cdot c_0 \leq c_0 \cdot n^k = c_0 \cdot m^{k/s}
\end{align*}
for every $\kk \in \NN_0^s$ with $\vert \kk \vert \leq k$ and hence
\begin{equation*}
\Vert S \Vert_{C^k ([-1,1]^s ; \RR)} \leq c_0 \cdot m^{k/s}.
\end{equation*}
Thus, letting $\varrho \defeq c_0^{-1} \cdot m^{-k/s}$ yields $ U_\varrho(X_m) \subseteq K$, so we see by \Cref{thm: devore} that
\begin{equation*}
\underset{f \in K}{\sup} \Vert f - M_{m-1} (\overline{a}(f)) \Vert_{L^\infty([-1,1]^s ; \RR)} \geq \varrho = c_1 \cdot m^{-k/s}
\end{equation*} 
with $c_1 = c_0^{-1}$ for every map $\overline{a} : X \to \RR^{m-1}$ which is continuous with respect to some norm on $V$ and any map $M_{m-1}: \RR^{m-1} \to X$. Using the inequality $m \leq 2(m-1)$ (note $m>1$) we get
\begin{align*}
\underset{f \in K}{\sup} \Vert f - M_{m-1} (\overline{a}(f)) \Vert_{L^\infty([-1,1]^s ; \RR)} \geq c_1 \cdot m^{-k/s} \geq c_1 \cdot (2(m-1))^{-k/s} \geq c_2 \cdot (m-1)^{-k/s}
\end{align*}
with $c_2 = c_1 \cdot 2^{-k/s}$. Hence, the claim has been shown for all numbers $m$ of the form $n^s - 1$ with an integer $n >1$.

In the end, let $m \in \NN$ be arbitrary and pick $n \in \NN$ with $n^s \leq m < (n+1)^s$. For given maps $\overline{a} : V \to \RR^m$ and $M_m: \RR^m \to X$ with $\overline{a}$ continuous with respect to some norm on $V$, let
\begin{equation*}
\tilde{a}: \quad V \to \RR^{(n+1)^s-1}, \quad f \mapsto (\overline{a}(f), 0) \quad\text{and}\quad M_{(n+1)^s-1}: \quad \RR^{(n+1)^s-1}\to X, \quad (x,y) \mapsto M_m(x),
\end{equation*}
where $x \in \RR^m, \  y \in \RR^{(n+1)^s -1 -m}$. Then we get
\begin{align*}
\underset{f \in K}{\sup} \Vert f - M_{m} (\overline{a}(f)) \Vert_{L^\infty([-1,1]^s ; \RR)} &= \underset{f \in K}{\sup} \Vert f - M_{(n+1)^s - 1} (\tilde{a}(f)) \Vert_{L^\infty([-1,1]^s ; \RR)} 
\geq c_2 \cdot ((n+1)^s - 1)^{-k/s} \\ 
&\geq c_2 \cdot (2^s n^s)^{-k/s} 
\geq c_3 \cdot m^{-k/s}
\end{align*}
with $c_3 = c_2 \cdot 2^{-k}$. Here we used the bound $(n+1)^s - 1 \leq (2n)^s$. This proves the full claim. \qedhere

\end{proof}



\subsection{Approximation using Ridge Functions}
\label{sec: ridge_reordered}
In this section we prove for $s \in \NN_{\geq 2}$ that every function in $C^k([-1,1]^s; \RR)$ can be uniformly approximated with an error of the order $m^{-k/(s-1)}$ using a linear combination of $m$ so-called \emph{ridge functions}. In fact, we only consider ridge \emph{polynomials}, meaning functions of the form
\begin{equation*}
\RR^s \to \RR, \quad x \mapsto p(a^T x)
\end{equation*}
for a fixed vector $a \in \RR^s$ and a polynomial $p: \RR \to \RR$. Note that this result has already been obtained in a slightly different form in \cite[Theorem 1]{maiorov_best_1999}; namely, it is shown there that the rate of approximation $m^{-k/(s-1)}$ can be achieved by functions of the form $\sum_{j=1}^m f_j(a_j^T x)$ with $a_j \in \RR^s$ and $f_j \in L^1_\text{loc} (\RR^s)$. We will need the fact that the $f_j$ can actually be chosen as polynomials and that the vectors $a_1, ..., a_m$ can be chosen independently from the particular function $f$. This is shown in the proof of \cite{maiorov_best_1999}, but not stated explicitly. For this reason, and in order to clarify the proof itself and to make the paper more self-contained, we decided to present the proof in this appendix.

\begin{lemma} \label{lem: hom_dim}
Let $m,s \in \NN$. Then we denote by
\begin{equation*}
P_m^s \defeq \left\{ \RR^s \to \RR, \quad x \mapsto \underset{\vert \kk \vert \leq m}{\sum_{\kk \in \NN_0^s}} a_\kk x^\kk : \ a_\kk \in \RR\right\}
\end{equation*}
the set of real polynomials of degree at most $m$. The subset of \emph{homogeneous} polynomials of degree $m$ is defined as
\begin{equation*}
H_m^s \defeq \left\{ \RR^s \to \RR, \quad x \mapsto \underset{\vert \kk \vert = m}{\sum_{\kk \in \NN_0^s}} a_\kk x^\kk : \ a_\kk \in \RR\right\}.
\end{equation*}
Then there exists a constant $c = c(s) > 0$ satisfying
\begin{equation*}
\dim (H_m^s) \leq c \cdot m^{s-1} \quad \forall m \in \NN.
\end{equation*}
\end{lemma}
\begin{proof}
It is immediate that the set
\begin{equation*}
\left\{ \RR^s \to \RR, \ x \mapsto x^\kk: \  \kk \in \NN_0^s, \ \vert \kk \vert =m\right\}
\end{equation*}
forms a basis of $H_m^s$, hence 
\begin{equation*}
\dim (H_m^s) = \# \left\{ \kk \in \NN_0^s: \ \vert \kk \vert = m\right\}.
\end{equation*}
This quantity clearly equals the number of possibilities for drawing $m$ times from a set with $s$ elements with replacement. Hence, we see
\begin{equation*}
\dim (H_m^s) = \binom{s+m-1}{m},
\end{equation*}
see for instance \cite[Identity 143]{benjamin_proofs_2003}. A further estimation shows
\begin{align*}
 \binom{s+m-1}{m} = \prod_{j=1}^{s-1} \frac{m+j}{j} = \prod_{j=1}^{s-1} \left(1 + \frac{m}{j}\right) \leq (1+m)^{s-1} \leq 2^{s-1} \cdot m^{s-1}.
\end{align*}
Hence, the claim follows with $c(s) = 2^{s-1}$.
\end{proof}

A combination of results from \cite{pinkus_ridge_2016} together with the fact that it is possible to approximate $C^k$-functions using polynomials of degree at most $m$ with an error of the order $m^{-k}$, as shown in \Cref{app: fourier_approx}, yields the desired result.

\begin{theorem} \label{app: ridge}
Let $s,k \in \NN$ with $s \geq 2$ and $r>0$. Then there exists a constant $c = c(s,k) > 0$ with the following property: For every $m \in \NN$ there exist $a_1, ..., a_m \in \RR^s \setminus \{0\}$ with $\Vert a_j \Vert_2 = r$, such that for every function $f \in C^k ([-1,1]^s ; \RR)$ there exist polynomials $p_1, ..., p_m \in P_m^1$ satisfying
\begin{equation*}
\left\Vert f(x) - \sum_{j=1}^m p_j (a_j^T x) \right\Vert_{L^\infty ([-1,1]^s ; \RR)} \leq c \cdot m^{-k/(s-1)} \cdot \Vert f \Vert_{C^k([-1,1]^s ; \RR)}.
\end{equation*}  
\end{theorem}
\begin{proof}
We first pick the constant $c_1 = c_1(s) $ according to \Cref{lem: hom_dim}. Then we define the constant $c_2 = c_2(s) \defeq (2s)^{s-1} \cdot c_1(s)$ and let $M \in \NN$ be the largest integer satisfying 
\begin{equation*}
c_2 \cdot M^{s-1} \leq m.
\end{equation*}
Here, we assume without loss of generality that $m \geq c_2$, which can be justified by choosing $p_j = 0$ for every $j \in \{1,...,m\}$ if $m < c_2$, at the cost of possibly enlarging $c$. Note that the choice of $M$ implies $c_2 \cdot (2M)^{s-1} \geq c_2 \cdot (M+1)^{s-1}> m$, and thus
\begin{equation} \label{eq: bound_lower}
M \geq \frac{1}{2} \cdot c_2^{-1/(s-1)} \cdot m^{1/(s-1)} = c_3 \cdot m^{1/(s-1)}
\end{equation}
with $c_3 = c_3(s) \defeq 1/2 \cdot c_2^{-1/(s-1)}$. 

Using \cite[Proposition 5.9]{pinkus_ridge_2016} and \Cref{lem: hom_dim} we can pick $a_1, ..., a_m \in \RR^s \setminus \{0\}$ satisfying
\begin{equation} \label{span: homogeneous}
H_{s(2M-1)}^s = \spann\left\{x \mapsto (a_j^T x )^{s(2M-1)}: \ j \in \{1,...,m\}\right\},
\end{equation}
where we used that
\begin{equation*}
c_1 \cdot (s(2M-1))^{s-1} \leq c_1 \cdot (2s)^{s-1} \cdot M^{s-1} = c_2 \cdot M^{s-1} \leq m. 
\end{equation*}
Here we can assume $\Vert a_j \Vert_2 = r$ for every $j \in \{1,...,m\}$ since multiplying each $a_j$ with a positive constant does not change the span in \eqref{span: homogeneous}.
From \cite[Corollary 5.12]{pinkus_ridge_2016} we infer that
\begin{equation} \label{eq: p_span}
P_{s(2M-1)}^s = \spann\left\{x \mapsto (a_j^T x )^{r}: \ j \in \{1,...,m\}, \ 0 \leq r \leq s(2M-1)\right\}.
\end{equation}

Let $f \in C^k([-1,1]^s ; \RR)$. Then, according to \Cref{app: fourier_approx}, there exists a polynomial $P : \RR^s \to \RR$ of \emph{coordinatewise} degree at most $2M -1$ satisfying
\begin{equation*}
\Vert f - P \Vert_{L^\infty ([-1,1]^s ; \RR)} \leq c_4 \cdot M^{-k} \cdot \Vert f \Vert_{C^k([-1,1]^s ; \RR)},
\end{equation*}
where $c_4 = c_4(s,k) > 0$. Note that by construction it holds $P \in P_{s(2M-1)}^s$. Using \eqref{eq: p_span} we deduce the existence of polynomials $p_1, ..., p_m : \RR \to \RR$ such that
\begin{equation*}
P(x) = \sum_{j= 1}^m p_j(a_j^T x) \quad \text{for all }x \in \RR^s. 
\end{equation*}
Combining the previously shown bounds, we get
\begin{align*}
\left\Vert f(x) - \sum_{j= 1}^m p_j(a_j^T x) \right\Vert_{L^\infty ([-1,1]^s ; \RR)} &= \Vert f(x) - P(x)\Vert_{L^\infty ([-1,1]^s ; \RR)} \leq  c_4 \cdot M^{-k} \cdot \Vert f \Vert_{C^k([-1,1]^s ; \RR)} \\ \overset{\eqref{eq: bound_lower}}&{\leq} c \cdot m^{-k/(s-1)} \cdot \Vert f \Vert_{C^k([-1,1]^s ; \RR)},
\end{align*}
as desired. Here, we defined $c = c(s,k) \defeq c_4 \cdot c_3^{-k}$.
\end{proof}




\textbf{Acknowledgements.} PG acknowledges support by
the German Science Foundation (DFG) in the context of the Emmy Noether junior research
group VO 2594/1-1. FV is grateful to Hrushikesh Mhaskar for several helpful and interesting discussions.

\printbibliography
\end{document}

