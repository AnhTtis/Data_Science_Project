% !TeX encoding = UTF-8
% !TeX spellcheck = en_US
% !TeX root = main_paper.tex

\section{Concrete examples of admissible functions}
\label{concrete_activation_functions}

In this section we want to show the admissibility of concrete activation functions that are commonly used when applying complex-valued neural networks to machine learning. 
\Cref{admissible} introduces a large class of admissible functions.
\begin{proposition}
\label{admissible}
    Let $\rho \in C^\infty(\RR; \RR)$ be non-polynomial and let $\psi : \CC \to \CC$ be defined as
    \begin{equation*}
        \psi(z) \defeq \rho(\RE(z)) \quad\text{resp.}\quad \psi(z) \defeq \rho(\IM(z)).
    \end{equation*}
    Then $\psi$ is admissible.
\end{proposition}
\begin{proof}
    Since $\psi$ depends only on the real resp. imaginary part of the input, we see directly from the definition of the Wirtinger derivatives that
    \begin{equation*}
        \wirt \psi(z) = \wirtq \psi(z)= \frac{1}{2} \rho' (\RE(z)) \quad \text{resp.} \quad \wirt \psi(z) = - \wirtq \psi(z)= -\frac{i}{2} \rho' \IM(z).
    \end{equation*}
    Hence we see for arbitrary $m, \ell \in \NN_0$ that
    \begin{equation*}
        \left\vert\wirt^m \wirtq^\ell \psi (z)\right\vert = \frac{1}{2^{m+\ell}} \left\vert\rho^{(m + \ell)} (\RE(z))\right\vert \quad\text{resp.} \quad  \left\vert\wirt^m \wirtq^\ell \psi (z)\right\vert = \frac{1}{2^{m+\ell}} \left\vert\rho^{(m + \ell)} (\IM(z))\right\vert.
    \end{equation*}
    Since $\rho$ is non-polynomial we can choose a real number $x$, such that all derivatives of $\rho$ in $x$ do not vanish (cf. for instance \cite[p. 53]{donoghue_distributions_1969}). Thus, all the Wirtinger derivatives $\wirt^m \wirtq^\ell \psi$ do not vanish for all complex numbers with real, resp. imaginary part $x$.
\end{proof}
In the following, we consider a special activation function, which has been proposed in \cite{arjovsky_unitary_2016}. Its representative power has already been discussed to some extent in \cite{caragea_quantitative_2022}.
\begin{definition}
    For $b \in (-\infty, 0)$ we define 
    \begin{equation*}
        \modrelu: \quad \CC \to \CC, \quad \modrelu(z) \defeq \left\{ \begin{matrix}(\vert z \vert + b)\frac{z}{\vert z \vert},& \vert z \vert + b \geq 0, \\ 0,& \mathrm{otherwise.}\end{matrix}\right.
    \end{equation*}
\end{definition}

\begin{theorem} \label{mod_relu_dev}
    Let $b \in (-\infty, 0)$. Writing $\sigma = \modrelu$ one has for every $z \in \CC$ with $\vert z \vert + b > 0$ the identity
    \begin{equation*}
        \left( \wirt^m \wirtq^\ell \sigma\right)(z) = \begin{cases}z + b\frac{z}{\vert z \vert},& m=\ell=0,\\1 + \frac{b}{2} \cdot \frac{1}{\vert z \vert},&m=1,\ell=0,\\ b \cdot q_{m,\ell} \cdot \frac{z^{\ell-m +1}}{\vert z \vert^{2\ell+1}},&m \leq \ell+1, \ell \geq 1, \\ b \cdot q_{m,\ell} \cdot \frac{\overline{z}^{m-\ell-1}}{\vert z \vert ^{2m-1}},& m \geq  \ell+1, m \geq 2\end{cases}
    \end{equation*}
    for every $m \in \NN_0$ and $\ell \in \NN_0$. Here, the numbers $q_{m,\ell}$ are non-zero and rational. Furthermore, note that all cases for choices of $m$ and $\ell$ are covered by observing that we can either have the case $m \geq \ell +1$ (where either $m \geq 2$ or $m=1, \ell = 0$) or $m \leq \ell + 1$ where the latter is again split into $\ell = 0$ and $\ell \geq 1$.
\end{theorem}
\begin{proof}
   We first calculate certain Wirtinger derivatives for $z \neq 0$. First note
    \begin{align*}
        \wirtq \left(\frac{1}{\vert z \vert^m}\right) &= \frac{1}{2}\left( \partial^{(1,0)}\left(\frac{1}{\vert z \vert^m}\right) + i \cdot \partial^{(0,1)}\left(\frac{1}{\vert z \vert^m}\right)\right) \\
        &= \frac{1}{2} \left(\left(- \frac{m}{2}\right)\frac{2\RE(z) + i\cdot 2\IM(z)}{\vert z \vert ^{m+2}}\right) \\
        &= - \frac{m}{2} \cdot \frac{z}{\vert z \vert^{m+2}}
    \end{align*}
    and similarly
    \begin{equation*}
        \wirt \left(\frac{1}{\vert z \vert^m}\right) = - \frac{m}{2} \cdot \frac{\overline{z}}{\vert z \vert^{m+2}}
    \end{equation*}
    for any $m \in \NN$. Using the product rule for Wirtinger derivatives, we see
    \begin{align}
    \label{wirt1}
         \wirtq \left(\frac{z^\ell}{\vert z \vert^m}\right) = \underbrace{\wirtq \left(z^\ell\right)}_{=0} \cdot \frac{1}{\vert z \vert^m} + z^\ell \cdot  \wirtq\left(\frac{1}{\vert z \vert^m}\right) = -\frac{m}{2} \cdot \frac{z^{\ell + 1}}{\vert z \vert^{m+2}}
    \end{align}
    for any $m \in \NN$ and $\ell \in \NN_0$ and furthermore
    \begin{align}
        \wirt \left(\frac{z^\ell}{\vert z \vert^m}\right) &= \wirt \left(z^\ell\right) \cdot \frac{1}{\vert z \vert^m} + z^\ell \cdot  \wirt\left(\frac{1}{\vert z \vert^m}\right) \nonumber\\
        &= \ell \cdot z^{\ell-1} \cdot \frac{1}{\vert z \vert ^m} - z^\ell \cdot  \frac{m}{2} \cdot \frac{\overline{z}}{\vert z \vert^{m+2}} \nonumber\\
        \label{wirt2}
        &=  \left(\ell - \frac{m}{2}\right) \cdot \frac{z^{\ell -1}}{\vert z \vert^m}
    \end{align}
    for $m, \ell \in \NN$, and finally
    \begin{align}
    \label{wirt3}
        \wirt \left(\frac{\overline{z}^\ell}{\vert z \vert^m}\right) &= \underbrace{\wirt \left(\overline{z}^\ell\right)}_{=0} \cdot \frac{1}{\vert z \vert^m} + \overline{z}^\ell \cdot  \wirt\left(\frac{1}{\vert z \vert^m}\right) = - \frac{m}{2} \cdot \frac{\overline{z}^{\ell + 1}}{\vert z \vert^{m+2}}
    \end{align}
    for $m \in \NN$ and $\ell \in \NN_0$. 

    Having proven those three identities, we can start with the actual computation. We first fix $m = 0$ and prove the claimed identity by induction over $\ell$. The case $\ell = 0$ is just the definition of the function and furthermore, we calculate
    \begin{align*}
        \wirtq \sigma (z) = \underbrace{\wirtq (z)}_{=0} + b \cdot \wirtq\left(\frac{z}{\vert z \vert}\right)  \overset{(\ref{wirt1})}{=} b \cdot \left(- \frac{1}{2}\right) \frac{z^2}{\vert z \vert^3},
    \end{align*}
    which is the claimed form. Then, using induction, we compute
    \begin{align*}
       \wirtq^{\ell + 1}\sigma(z) =  \wirtq \left(b \cdot q_{0,\ell} \cdot \frac{z^{\ell+1}}{\vert z \vert^{2\ell+1}}\right) \overset{(\ref{wirt1})}{=} b \cdot \underbrace{q_{0,\ell} \cdot \left(- \frac{2\ell+1}{2}\right)}_{=: q_{0, \ell+1}} \cdot \frac{z^{\ell+2}}{\vert z \vert^{2\ell+3}},
    \end{align*}
    so that the case $m = 0$ is complete. 

    Now we deal with the case $m \leq \ell+1$. The case $\ell = 0$ is proven by computing
    \begin{align*}
        \wirt \sigma (z) = \wirt (z) + b \cdot \wirt \left(\frac{z}{\vert z \vert}\right) \overset{\eqref{wirt2}}{=} 1 + b \cdot \frac{1}{2} \cdot \frac{1}{\vert z \vert}
    \end{align*}
    so we can assume $\ell > 0$. Since we already dealt with the case $m = 0$ we can inductively assume the claim to be true for a fixed $m \leq \ell$. Then we compute
    \begin{align*}
        \left( \wirt^{m+1} \wirtq^\ell \sigma\right)(z) &= \wirt \left( b \cdot q_{m,\ell} \cdot \frac{z^{\ell-m+1}}{\vert z \vert^{2\ell+1}}\right) \overset{(\ref{wirt2})}{=} b \cdot \underbrace{q_{m,\ell} \cdot \left(-m + \frac{1}{2}\right)}_{=: q_{m+1,\ell}} \cdot \frac{z^{\ell-m}}{\vert z \vert^{2\ell+1}},
    \end{align*}
    which is the desired form. Note that (\ref{wirt2}) is indeed applicable because $ \ell - m +1 \geq 1$. 

    Finally, we consider the case where $m \geq \ell+1$ and $m \geq 2$. The case $m = \ell+1$ has already been shown. Using induction, we see
    \begin{align*}
        \left( \wirt^{m+1} \wirtq^\ell\sigma\right)(z) &= \wirt \left( \delta_{(m, \ell) = (1,0)} + b \cdot q_{m,\ell} \cdot \frac{\overline{z}^{m-\ell-1}}{\vert z \vert ^{2m-1}}\right) \overset{(\ref{wirt3})}{=} b \cdot \underbrace{q_{m,\ell} \cdot \left(- m + \frac{1}{2}\right)}_{=: q_{m + 1,\ell}}\cdot \frac{\overline{z}^{m - \ell}}{\vert z \vert ^{2m + 1}}, 
    \end{align*}
    so the proof is complete.
\end{proof}
From \Cref{mod_relu_dev} we can now deduce the admissibility of the modReLU.
\begin{corollary}
Let $b \in (-\infty, 0)$ and $z \in \CC$ with $\vert z \vert >  -b $. Then we have 
\begin{equation*}
    \wirt^m \wirtq^\ell \modrelu (z) \neq 0
\end{equation*}
for every $m, \ell \in \NN_0$. In particular, $\modrelu$ is admissible.
\end{corollary}
\begin{proof}
This follows from \Cref{mod_relu_dev} by noting that if $\vert z \vert > -b$ we have in particular $\vert z \vert > -b/2$ and $\vert z \vert > 0$.
\end{proof}


The second activation function that we want to analyze is proposed in \cite{virtue_better_2017} where it is used for MRI fingerprinting to achieve significant results with complex-valued neural networks, outperforming their real-valued counterparts. Therefore it shall be studied, if this activation function is also admissible.
\begin{definition}
    The function
    \begin{equation*}
        \card:\ \CC \to \CC, \ f(z) \defeq  \frac{1}{2}(1+\mathrm{cos}(\sphericalangle z ))z
    \end{equation*}
    is called \emph{complex cardioid function}. Here, $\sphericalangle z = \theta \in \RR$ denotes the polar angle of a complex number $z = re^{i\theta}$, where we define $\sphericalangle 0 := 0$. Even though $\theta$ is only well-defined modulo $2\pi$, this is not an issue here, since $\cos(\theta)$ is $2\pi$-periodic. Furthermore, $\cos(\theta)= \frac{\RE(z)}{\vert z \vert}$ for $z \neq 0$.
\end{definition}
\begin{theorem} \label{thm: cardioid}
    For any $z \in \CC$ with $z \neq 0$ and any $m, \ell \in \NN_0$ we have
    \begin{equation*}
        \wirt^m \wirtq^\ell \card (z) = \begin{cases}
        \frac{1}{2}z +\frac{1}{4} \frac{z^2}{\vert z \vert} + \frac{\vert z \vert}{4},& m=\ell=0,\\
            a_{m,\ell} \frac{z^{\ell - m}}{\vert z \vert ^{2\ell -1}} + b_{m, \ell} \frac{z^{\ell + 2 -m}}{\vert z \vert ^{2\ell + 1}}, & m \leq \ell \neq 0, \\
            \frac{1}{2} + \frac{1}{8} \cdot \frac{\overline{z}}{\vert z \vert} + \frac{3}{8} \cdot \frac{z}{\vert z \vert},& m = \ell + 1 = 1,\\
            a_{m,\ell} \frac{\overline{z}}{\vert z \vert ^{2\ell +1}} + b_{m, \ell} \frac{z}{\vert z \vert ^{2\ell + 1}}, & m = \ell + 1 > 1,\\
            a_{m,\ell} \frac{\overline{z}^{m - \ell}}{\vert z \vert ^{2m -1}} + b_{m, \ell} \frac{\overline{z}^{m - \ell - 2 }}{\vert z \vert ^{2m-3}}, & m \geq \ell + 2.
        \end{cases} 
    \end{equation*}
    Here, the numbers $a_{m, \ell}$ and $b_{m,\ell}$ are again non-zero and rational. Furthermore, note that all cases for possible choices of $m$ and $\ell$ are covered: The case $m \leq \ell$ is split into $\ell = 0$ and $ \ell \neq 0$. The case $m = \ell + 1$ is split into $m = 1$ and $m > 1$. Then, the case $m \geq \ell +2$ remains. 
\end{theorem}
\begin{proof}
    For the following we always assume $z \in \CC$ with $z \neq 0$. Then we can simplify $\cos(\sphericalangle z) = \frac{\RE(z)}{\vert z \vert}$, so we can rewrite
    \begin{equation*}
        \card(z) = \frac{1}{2}\left(1 + \frac{\RE(z)}{\vert z \vert}\right)z = \frac{1}{2} z + \frac{1}{4}\frac{\left(z+\overline{z} \right)z}{\vert z \vert} = \frac{1}{2}z +\frac{1}{4} \frac{z^2}{\vert z \vert} + \frac{\vert z \vert}{4}.
    \end{equation*}
    First, we compute
    \begin{equation}
\label{wirtquerbetrag}
        \wirtq \left(\vert z \vert\right) = \frac{1}{2} \left( \frac{1}{2} \frac{2\RE(z)}{\vert z \vert} + \frac{i}{2}\frac{2\IM(z)}{\vert z \vert}\right) = \frac{1}{2} \frac{z}{\vert z \vert}
    \end{equation}
    and similarly
    \begin{equation}
\label{wirtbetrag}
        \wirt\left( \vert z \vert\right) = \frac{1}{2} \frac{\overline{z}}{\vert z \vert}.
    \end{equation}
    We deduce
    \begin{align*}
        \wirtq \card (z) \overset{(\ref{wirt1}),(\ref{wirtquerbetrag})}{=} \underbrace{\frac{1}{4} \cdot \left(- \frac{1}{2} \right)}_{=: b_{0, 1}} \cdot  \frac{z^3}{\vert z \vert^3} + \underbrace{\frac{1}{8}}_{=: a_{0,1}} \cdot \frac{z}{\vert z \vert}.
    \end{align*}
    Using induction, we derive
    \begin{align*}
        \wirtq^{\ell + 1} \card(z) &= a_{0, \ell }  \wirtq \left(\frac{z^{\ell }}{\vert z \vert ^{2\ell -1}}\right) + b_{0, \ell} \wirtq\left(\frac{z^{\ell + 2 }}{\vert z \vert^{2\ell + 1}}\right) \\
        \overset{(\ref{wirt1})}&{=} \underbrace{a_{0, \ell} \cdot \left( - \frac{2\ell - 1}{2}\right)}_{=: a_{0, \ell + 1}} \cdot \frac{z^{\ell + 1}}{\vert z \vert^{2\ell + 1}} + \underbrace{b_{0, \ell} \cdot \left(-\frac{2\ell + 1}{2}\right)}_{=: b_{0, \ell + 1}} \cdot \frac{z^{\ell + 3}}{\vert z \vert^{2\ell + 3}},
    \end{align*}
    so the claim has been shown if $m = 0$. If we now fix any $\ell \in \NN$ and assume that the claim holds true for some $m \in \NN_0$ with $m < \ell$, we get
    \begin{align*}
        \wirt^{m+1}\wirtq^\ell \card(z) &= a_{m, \ell} \wirt \left(\frac{z^{\ell - m}}{\vert z \vert^{2\ell - 1}}\right) + b_{m, \ell} \wirt \left( \frac{z^{\ell + 2 - m}}{\vert z \vert^{2\ell + 1}}\right) \\
        \overset{(\ref{wirt2})}&{=} \underbrace{a_{m,\ell} \cdot \left( \frac{1}{2} - m\right)}_{=: a_{m+1, \ell}} \cdot \frac{z^{\ell - m -1}}{\vert z \vert^{2\ell - 1}} + \underbrace{b_{m, \ell} \cdot \left( \frac{3}{2} - m\right)}_{=: b_{m+1, \ell}}\cdot \frac{z^{\ell + 2 - m - 1}}{\vert z \vert^{2\ell + 1}},
    \end{align*}
    so the claim holds true if $m \leq \ell$. 

    The case $m = \ell + 1$ is split into the case $m= 1$ and $m > 1$. If $m = 1$, then $\ell = 0$ and we compute
    \begin{equation*}
        \wirt \card (z) \overset{(\ref{wirt2}), (\ref{wirtbetrag})}{=} \frac{1}{2} + \frac{1}{4}\left(2 - \frac{1}{2}\right) \cdot \frac{z}{\vert z \vert } + \frac{1}{8} \cdot \frac{\overline{z}}{\vert z \vert}.
    \end{equation*}
    If $m > 1$ we get
    \begin{align*}
        \wirt^{\ell + 1}\wirtq^{\ell} \card (z) &= a_{\ell, \ell} \wirt \left(\frac{1}{\vert z \vert^{2\ell - 1}}\right) + b_{\ell, \ell} \cdot \wirt \left(\frac{z^{ 2 }}{\vert z \vert^{2\ell + 1}}\right) \\
        \overset{\eqref{wirt2},(\ref{wirt3})}&{=} \underbrace{a_{\ell, \ell} \cdot \left(-\frac{2\ell - 1}{2} \right)}_{=: a_{\ell + 1, \ell}} \cdot \frac{\overline{z}}{\vert z \vert^{2\ell + 1}} +\underbrace{ b_{\ell, \ell} \cdot \left(2 - \frac{2\ell + 1}{2}\right)}_{=: b_{\ell + 1, \ell}} \cdot \frac{z}{\vert z \vert^{2\ell + 1}}.
    \end{align*}
    Next, we deal with the case $m = \ell + 2$: Here we see
\begin{align*}
	\wirt^{\ell + 2} \wirtq^{\ell} \card (z) &= \wirt \left(\frac{1}{2} \delta_{(m,\ell) = (1, 0)}  + a_{\ell +1, \ell} \frac{ \overline{z}}{\vert z \vert^{2\ell + 1}} + b_{\ell + 1,\ell} \frac{z}{\vert z \vert^{2\ell + 1}}\right) \\
\overset{(\ref{wirt2}),(\ref{wirt3})}&{=} \underbrace{a_{\ell + 1, \ell} \cdot \left(-\frac{2\ell + 1}{2}\right)}_{=: a_{\ell +2, \ell}} \cdot \frac{\overline{z}^2}{\vert z \vert^{2\ell + 3}} + \underbrace{b_{\ell + 1, \ell}\cdot \left(1 - \frac{2\ell + 1}{2}\right)}_{=: b_{\ell + 2, \ell}}\cdot \frac{1}{\vert z \vert^{2\ell + 1}}.
\end{align*}
If we assume the claim to be true for a fixed $m \geq \ell + 2$ we get 
\begin{align*}
\wirt^{m + 1} \wirtq^\ell \card(z) &= a_{m, \ell} \cdot \wirt \left(\frac{\overline{z}^{m - \ell}}{\vert z \vert^{2m - 1}}\right) + b_{m, \ell} \cdot \wirt\left( \frac{\overline{z}^{m - \ell - 2}}{\vert z \vert^{2m - 3}}\right) \\
\overset{(\ref{wirt3})}&{=} \underbrace{a_{m,\ell} \cdot \left(- \frac{2m - 1}{2}\right)}_{=: a_{m + 1, \ell}} \cdot \frac{\overline{z}^{m + 1 - \ell}}{\vert z \vert^{2m + 1}} + \underbrace{b_{m,\ell} \cdot \left( - \frac{2m - 3}{2}\right)}_{ =: b_{m + 1, \ell}} \cdot \frac{\overline{z}^{m - \ell - 1}}{\vert z \vert^{2m - 1}}.
\end{align*}
Hence, using induction, we have proven the claimed identity.
\end{proof}
The statement regarding the admissibility of the complex cardioid is formulated in the following corollary.
\begin{corollary}
    For every $z \in \CC$ with $z \notin \RR \cup i \RR$ and every $m, \ell \in \NN_0$ we have 
    \begin{equation*}
        \wirt^m \wirtq^\ell \card (z)\neq 0.
    \end{equation*}
    In particular, $\card$ is admissible.
\end{corollary}
\begin{proof}
     For the following, let $z  \in \CC$ with $\RE(z), \IM(z) > 0$.
    Using the definition of $\card$, we see 
    \begin{equation*}
        \card(z) = 0 \Leftrightarrow z = 0 \text{ or } \cos(\sphericalangle z ) = - 1 \Leftrightarrow z \in \RR^{\leq 0},
    \end{equation*}
which is never fulfilled. 
    In the case $m \leq \ell \neq 0$ we see that the relation
    \begin{align*}
        \wirt^m \wirtq^\ell \card (z) = 0 \Leftrightarrow a_{m, \ell} + b_{m, \ell}\frac{z^2}{\vert z \vert ^2} = 0 \Leftrightarrow z^2 = - \vert z \vert^2 \cdot \frac{a_{m, \ell}}{b_{m, \ell}} \Rightarrow z^2 \in \RR \Leftrightarrow z \in \RR \cup i\RR
    \end{align*}
    holds, which is impossible by assumption.  

    For the case $m = \ell + 1 = 1$, consider
    \begin{equation*}
        \wirt \card(z) = 0 \Leftrightarrow \frac{1}{8} \cdot \overline{z} + \frac{3}{8} \cdot z = -\frac{\vert z \vert}{2} \Rightarrow \frac{3}{8}\IM(z) - \frac{1}{8} \IM(z) = 0 \Leftrightarrow \IM(z) = 0,
    \end{equation*}
which does not hold since $z \notin \RR$.

    For $m = \ell + 1 > 1$ we see by considering the real- and imaginary parts that
    \begin{align*}
        \wirt^{\ell + 1}\wirtq^\ell \card (z) = 0 &\Leftrightarrow a_{m,\ell}\overline{z} + b_{m, \ell} z = 0 \overset{\RE(z), \IM(z) \neq 0}{\Leftrightarrow} a_{m,\ell} + b_{m, \ell}=0 \text{ and } -a_{m, \ell} + b_{m, \ell}=0 \\ &\Leftrightarrow a_{m,\ell} = b_{m,\ell}=0,
    \end{align*}
which is not fulfilled, since $a_{m,\ell} \neq 0 \neq b_{m,\ell}$ by \Cref{thm: cardioid}. 

    Therefore it remains the case $m \geq \ell + 2$. But here we easily see
    \begin{align*}
        \wirt^m \wirtq^{\ell}\card(z) = 0 \Leftrightarrow a_{m,\ell} \frac{\overline{z}^2}{\vert z \vert^2} + b_{m,\ell} = 0 \Rightarrow \overline{z}^2 \in \RR \Leftrightarrow z \in \RR \cup i\RR,
    \end{align*}
which is not fulfilled by assumption.

    Since all cases have been considered, the claim follows.
\end{proof}