% !TeX encoding = UTF-8
% !TeX spellcheck = en_US
% !TeX root = main_paper.tex

\section{Prerequisites for the optimality result}
In this subsection we prove a very general lower bound for the approximation of functions in $C^k ([-1,1]^s ; \RR)$ using a subset of $C([-1,1]^s; \RR)$ that can be parametrized using a certain amount of parameters. Precisely, we prove a lower bound of $m^{-k/s}$ where $m$ is the number of parameters that are used, provided that the selection of the parameters, i.e., the map that maps a function in $C^k([-1,1]^s; \RR)$ to the parameters of the approximating function, is continuous with respect to some norm on $C^k ([-1,1]^s ; \RR)$. The proofs are in fact almost identical to what is done in \cite{devore_optimal_1989}. However, we decided to include a detailed proof in this paper, since \cite{devore_optimal_1989} considers Sobolev functions and not $C^k$-functions and since the continuity assumption in \cite{devore_optimal_1989} is not completely clear. 

\begin{proposition}[{\cite[Theorem 3.1]{devore_optimal_1989}}] \label{thm: devore}
Let $(X, \Vert \cdot \Vert_X)$ be a normed space, $\emptyset \neq K \subseteq X$ a subset and $V \subseteq X$ a linear, not necessarily closed subspace of $X$ containing K. Let $m\in \NN$, $\overline{a} : K \to \RR^m$ be a map which is continuous with respect to some norm $\Vert \cdot \Vert_V$ on $V$ and $M_m : \RR^m \to X$ some arbitrary map. Let
\begin{equation*}
b_m(K)_X \defeq \underset{X_{m+1}}{\sup} \sup \left\{\varrho \geq 0: \  U_\varrho(X_{m+1}) \subseteq K\right\},
\end{equation*} 
where the first supremum is taken over all $(m+1)$-dimensional linear subspaces $X_{m+1}$ of $X$ and
\begin{equation*}
U_\varrho(X_{m+1}) \defeq \{y \in X_{m+1} : \ \Vert y \Vert_X \leq \varrho\}.
\end{equation*}
Then it holds
\begin{equation*}
\underset{x \in K}{\sup} \Vert x - M_m (\overline{a}(x)) \Vert_X \geq b_m(K)_X.
\end{equation*}
\end{proposition}

\begin{proof}
The claim is trivial if $b_m(K)_X = 0$. Thus, assume $b_m(K)_X > 0$. Let $0 < \varrho \leq b_m(K)_X$ be any number such that there exists an $(m+1)$-dimensional subspace $X_{m+1}$ of $X$ with $ U_\varrho(X_{m+1}) \subseteq K$. It follows $U_\varrho(X_{m+1}) \subseteq V$, hence $X_{m+1} \subseteq V$, so $\Vert \cdot \Vert_V$ defines a norm on $X_{m+1}$. Thus, the restriction of $\overline{a}$ to $\partial  U_\varrho(X_{m+1})$ is a continuous mapping to $\RR^m$ with respect to $\Vert \cdot \Vert_V$. Since all norms are equivalent on the finite-dimensional space $X_{m+1}$, the Borsuk-Ulam-Theorem \cite[Corollary 4.2]{deimling2013nonlinear} yields the existence of a point $x_0 \in \partial U_\varrho(X_{m+1})$ with $\overline{a}(x_0) = \overline{a}(-x_0)$. We then see
\begin{align*}
2\varrho &= 2\Vert x_0\Vert_X =\Vert x_0 - M_m(\overline{a}(x_0))\Vert_X  + \Vert x_0 +M_m(\overline{a}(-x_0)) \Vert_X\\ 
&\leq \Vert x_0 - M_m (\overline{a}(x_0)) \Vert_X + \Vert - x_0 - M_m(\overline{a}(-x_0))\Vert_X,
\end{align*}
and hence, at least one of the two summands on the right has to be larger than or equal to $\varrho$.
\end{proof}

Using this very general result we can deduce our lower bound in the context of $C^k$-spaces. 
\begin{theorem} \label{app: devore_real}
Let $s,k \in \NN$. Then there exists a constant $c = c(s,k)>0$ with the following property:  For any $m \in \NN$ and any map $\overline{a} : C^k ([-1,1]^s; \RR) \to \RR^m$ that is continuous with respect to some norm on $C^k([-1,1]^s; \RR)$ and any map $M_m  : \RR^m \to C([-1,1]^s ; \RR)$ we have
\begin{equation*}
\underset{\Vert f \Vert _{C^k ([-1,1]^s ; \RR)} \leq 1}{\underset{f \in C^k([-1,1]^s ; \RR)}{\sup}} \Vert f - M_m (\overline{a}(f)) \Vert_{L^\infty ([-1,1]^s ; \RR)} \geq c \cdot m^{-k/s}.
\end{equation*}
\end{theorem}
\begin{proof}
The idea is to apply \Cref{thm: devore} to the spaces $X \defeq C([-1,1]^s ; \RR)$, $V \defeq C^k ([-1,1]^s ; \RR)$ and the set $K \defeq \{f \in C^k ([-1,1]^s ; \RR): \  \Vert f \Vert_{C^k ([-1,1]^s ; \RR)} \leq 1\}$. 

Assume in the beginning that $m = n^s$ with an integer $n >1$. Pick $\phi \in C^\infty(\RR^s)$ with $\phi \equiv 1$ on $[-3/4, 3/4]^s$ and $\phi \equiv 0$ outside of $[-1,1]^s$. Fix $c_0 = c_0(s,k) > 0$ with 
\begin{equation*}
1 \leq \Vert \phi \Vert_{C^k([-1,1]^s ; \RR) } \leq c_0.
\end{equation*} 
Let $Q_1, ..., Q_m$ be the partition (disjoint up to null-sets) of $[-1,1]^s$ into closed cubes of sidelength $2/n$. For every $j \in \{1,...,m\}$ write $Q_j = \bigtimes_{\ell = 1}^{s} [a_\ell - 1/n, a_\ell + 1/n]$ with some vector $a = (a_1, ..., a_s) \in [-1,1]^s$ and let 
\begin{equation*}
\phi_j (x) \defeq \phi (nx - na) \text{ for } x \in \RR^s.
\end{equation*}
By choice of $\phi$, the maps $\phi_j$ are supported on a proper subset of $Q_j$ for every $j \in \{1,...,m\}$ and an inductive argument shows
\begin{equation*} 
\partial^\kk \phi_j (x) = n^{\vert \kk \vert} \cdot \partial^\kk \phi (nx - na) \text{ for every } \kk \in \NN_0^s \text{ and }x \in \RR^s
\end{equation*}
and hence in particular
\begin{equation} \label{eq: derivative}
\Vert \phi_j \Vert_{C^k([-1,1]^s ; \RR)} \leq n^{\vert \kk \vert} \cdot c_0.
\end{equation}
Let $X_m \defeq \spann \{\phi_1, ..., \phi_m\}$ and $S \in U(X_m)$. Then we can write $S$ in the form $S= \sum_{j = 1}^m c_j \phi_j$ with real numbers $c_1, ..., c_m \in \RR$. Suppose there is $j^* \in \{1,...,m\}$ with $\vert c_{j^*}\vert  > 1$. Then we have
\begin{equation*}
\Vert S \Vert_{L^\infty([-1,1]^s ; \RR)} \geq \underbrace{\vert c_{j^*} \vert}_{> 1} \cdot \underbrace{\Vert \phi_{j^*} \Vert_{L^\infty([-1,1]^s ; \RR)}}_{ = 1} > 1,
\end{equation*}
since the functions $\phi_j$ have disjoint support. This is a contradiction to $S \in U(X_m)$ and we can thus infer that $\underset{j}{\max} \ \vert c_j \vert \leq 1$. Furthermore, we see again because the functions $\phi_j$ have disjoint support that
\begin{align*}
\Vert \partial^\kk S \Vert_{L^\infty ([-1,1]^s ; \RR)} = \underset{j}{\max} \ \vert c_j \vert \cdot \Vert \partial^\kk \phi_j \Vert_{L^\infty([-1,1]^s ; \RR)} \overset{\eqref{eq: derivative}}{\leq} n^{\vert \kk \vert} \cdot c_0 \leq c_0 \cdot n^k = c_0 \cdot m^{k/s}
\end{align*}
for every $\kk \in \NN_0^s$ with $\vert \kk \vert \leq k$ and hence
\begin{equation*}
\Vert S \Vert_{C^k ([-1,1]^s ; \RR)} \leq c_0 \cdot m^{k/s}.
\end{equation*}
Thus, letting $\varrho \defeq c_0^{-1} \cdot m^{-k/s}$ yields $ U_\varrho(X_m) \subseteq K$, so we see by \Cref{thm: devore} that
\begin{equation*}
\underset{f \in K}{\sup} \Vert f - M_{m-1} (\overline{a}(f)) \Vert_{L^\infty([-1,1]^s ; \RR)} \geq \varrho = c_1 \cdot m^{-k/s}
\end{equation*} 
with $c_1 = c_0^{-1}$ for every map $\overline{a} : X \to \RR^{m-1}$ which is continuous with respect to some norm on $V$ and any map $M_{m-1}: \RR^{m-1} \to X$. Using the inequality $m \leq 2(m-1)$ (note $m>1$) we get
\begin{align*}
\underset{f \in K}{\sup} \Vert f - M_{m-1} (\overline{a}(f)) \Vert_{L^\infty([-1,1]^s ; \RR)} \geq c_1 \cdot m^{-k/s} \geq c_1 \cdot (2(m-1))^{-k/s} \geq c_2 \cdot (m-1)^{-k/s}
\end{align*}
with $c_2 = c_1 \cdot 2^{-k/s}$. Hence, the claim has been shown for all numbers $m$ of the form $n^s - 1$ with an integer $n >1$.

In the end, let $m \in \NN$ be arbitrary and pick $n \in \NN$ with $n^s \leq m < (n+1)^s$. For given maps $\overline{a} : V \to \RR^m$ and $M_m: \RR^m \to X$ with $\overline{a}$ continuous with respect to some norm on $V$, let
\begin{equation*}
\tilde{a}: \quad V \to \RR^{(n+1)^s-1}, \quad f \mapsto (\overline{a}(f), 0) \quad\text{and}\quad M_{(n+1)^s-1}: \quad \RR^{(n+1)^s-1}\to X, \quad (x,y) \mapsto M_m(x),
\end{equation*}
where $x \in \RR^m, \  y \in \RR^{(n+1)^s -1 -m}$. Then we get
\begin{align*}
\underset{f \in K}{\sup} \Vert f - M_{m} (\overline{a}(f)) \Vert_{L^\infty([-1,1]^s ; \RR)} &= \underset{f \in K}{\sup} \Vert f - M_{(n+1)^s - 1} (\tilde{a}(f)) \Vert_{L^\infty([-1,1]^s ; \RR)} 
\geq c_2 \cdot ((n+1)^s - 1)^{-k/s} \\ 
&\geq c_2 \cdot (2^s n^s)^{-k/s} 
\geq c_3 \cdot m^{-k/s}
\end{align*}
with $c_3 = c_2 \cdot 2^{-k}$. Here we used the bound $(n+1)^s - 1 \leq (2n)^s$. This proves the full claim. \qedhere

\end{proof}

