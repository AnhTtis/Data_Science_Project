% !TeX encoding = UTF-8
% !TeX spellcheck = en_US
% !TeX root = main_paper.tex

\section{Approximation of Polynomials}
\label{approx_polynomials}
The following section is dedicated to proving \Cref{main_1}. We are going to show that it is possible to approximate complex polynomials in $z$ and $\overline{z}$ arbitrarily well on $\Omega_n$ using shallow complex-valued neural networks. 

The following theorem can be seen as a generalization of the classical mean-value theorem. Its proof is deferred to \Cref{div_differences_mainresult} in the appendix.
\begin{theorem}
\label{divdiff}
Let $f: \RR^s \to \RR$ and $k \in \NN_0, r>0$, such that $\fres{f}{(-r,r)^s} \in C^k \left((-r,r)^s; \RR\right)$. For $\textbf{p} \in \NN_0^s$ with $\vert \pp \vert \leq k$ and $h>0$ let
\begin{equation*}
    f_{\pp,h} \defeq (2h)^{-\vert \pp \vert} \sum_{0 \leq \textbf{r} \leq \textbf{p}} (-1)^{\vert \pp \vert -\vert \rr \vert} \binom{\textbf{p}}{\textbf{r}} f \left( h(2\rr-\pp)\right).
\end{equation*}
Let $m \defeq \underset{j}{\max} \  \pp_j$. Then for $0 <h < \frac{r}{\max\{1,m\}}$ there exists $\xi \in h[-m,m]^s$ satisfying
\begin{equation*}
    f_{\pp,h} = \partial^\pp f(\xi).
\end{equation*}
\end{theorem}
The following theorem describes how one can extract monomials of the form $z^\m \overline{z}^{\elll}$ by calculating 
\begin{equation*}
    \wirt^\m \wirtq^{\elll} \left(w \mapsto \phi\left(w^T z + b\right)\right)
\end{equation*}
for a $C^k$-function $\phi:\CC \to \CC$.
\begin{theorem}
\label{extraction}
    Let $\phi:\CC \to \CC$ and $\delta>0, \ b \in \CC, \ k \in \NN_0$, such that $\fres{\phi}{B_\delta(b)} \in C^k      \left(B_\delta(b); \CC\right)$.
    For fixed $z \in \Omega_n$, where we recall that $\Omega_n = [-1,1]^n + i [-1,1]^n$, we consider the map
    \begin{equation*}
        \phi_z: \quad B_{\frac{\delta}{\sqrt{2n}}}(0) \to \CC, \quad w \mapsto \phi\left(w^T z + b\right),
    \end{equation*}
    which is in $C^k$ since for $w \in B_{\frac{\delta}{\sqrt{2n}}}(0) \subseteq \CC$ we have
    \begin{equation*}
        \left\vert w^T z\right\vert \leq \Vert w \Vert_2 \cdot \Vert z \Vert_2 < \frac{\delta}{\sqrt{2n}} \cdot \sqrt{2n} = \delta.
    \end{equation*}
    For all multi-indices $\m, \elll \in \NN_0^n$ with $\vert \m + \elll \vert \leq k$ we have
    \begin{equation*}
        \wirt^\m \wirtq^{\elll} \phi_z(w) = z^\m \overline{z}^{\elll} \cdot \left(\wirt^{\vert \m \vert}\wirtq^{\vert \elll \vert}\phi\right) \left(w^T z + b \right)
    \end{equation*}
    for all $w \in B_{\frac{\delta}{\sqrt{2n}}}(0)$.
\end{theorem}
\begin{proof}
    First we prove the statement
    \begin{equation}
    \label{conj}
        \wirtq^{\elll} \phi_z(w) = \overline{z}^{\elll} \cdot (\wirtq^{\vert {\elll} \vert}\phi)\left( w^Tz +b\right) \quad \text{for all } w \in B_{\frac{\delta}{\sqrt{2n}}}(0)
    \end{equation}
    by induction over $0 \leq \vert {\elll} \vert \leq k$. The case ${\elll} = 0$ is trivial. Assuming that (\ref{conj}) holds for fixed ${\elll} \in \NN_0^n$ with $\vert \elll\vert < k $, we want to show
    \begin{equation}
    \label{conj_wirt}
        \wirtq^{{\elll} + e_j} \phi_z(w) = \overline{z}^{{\elll}+e_j} \cdot \left( \wirt^{|{\elll}| + 1}\phi\right)\left(w^Tz +b\right)
    \end{equation}
    for all $w \in B_{\frac{\delta}{\sqrt{2n}}}(0)$, where $j \in \{1,...,n\}$ is chosen arbitrarily. Firstly we get
    \begin{align*}
        \wirtq^{{\elll} + e_j} \phi_z(w) &= \wirtq^{e_j}\wirtq^{\elll} \phi_z(w)  \overset{\text{induction}}{=} \wirtq^{e_j}\left[w \mapsto \overline{z}^{\elll} \cdot \left(\wirt^{\vert {\elll} \vert}\phi\right)\left(w^Tz + b \right)\right] \\
        &= \overline{z}^{\elll} \wirtq^{e_j}\left[w \mapsto \left(\wirtq^{\vert {\elll} \vert}\phi\right)\left(w^Tz + b\right)\right].
    \end{align*}
    Applying the chain rule for Wirtinger derivatives and using  
    \begin{equation*}
        \wirtq^{e_j} \left[ w \mapsto w^T z +b\right]= 0
    \end{equation*} we see
    \begin{align*}
        \wirtq^{e_j}\left[w \mapsto \left(\wirtq^{\vert {\elll} \vert}\phi\right)\left(w^Tz + b\right)\right] &= \left(\wirt\wirtq^{\vert {\elll}\vert}\phi\right) \left(w^Tz +b\right) \cdot \wirtq^{e_j}\left[w \mapsto w^Tz + b\right] \\
        & \hspace{0.4cm}+ \left(\wirtq^{\vert {\elll}\vert + 1}\phi\right) \left(w^Tz +b\right) \cdot \wirtq^{e_j}\left[w \mapsto \overline{w^Tz + b}\right] \\
        &= \left(\wirtq^{\vert {\elll}\vert + 1}\phi\right) \left( w^Tz +b\right) \cdot \overline{\wirt^{e_j}\left[w \mapsto w^Tz + b\right]} \\
        &= \overline{z}^{e_j} \cdot \left(\wirtq^{\vert {\elll}\vert + 1}\phi\right) \left(w^Tz +b\right),
    \end{align*}
    using the fact that $w_j \mapsto w^Tz + b$ is holomorphic and hence
    \begin{equation*}
        \wirtq^{e_j}\left[ w \mapsto w^T z + b\right] = 0 \quad \text{and} \quad \wirt^{e_j}\left[ w \mapsto w^T z + b\right] = z_j.
    \end{equation*}
    Thus, we have proven (\ref{conj_wirt}) and induction yields (\ref{conj}). 

    It remains to show the full claim. We use induction over $\vert \m \vert$ and note that the case $\m=0$ has just been shown. We assume that the claim holds true for fixed $\m \in \NN_0^n$ with $\vert \m + \elll \vert < k$ and choose $j \in \{1,...,n\}$. Thus, we get
    \begin{align*}
        \wirt^{\m + e_j} \wirtq^{{\elll}}\phi_z(w) &= \wirt^{e_j}\wirt^\m \wirtq^{\elll} \phi_z(w) \overset{\text{IH}}{=} \wirt^{e_j}\left( w \mapsto z^\m \overline{z}^{\elll} \cdot\left(\wirt^{\vert \m \vert} \wirtq^{\vert {\elll} \vert} \phi \right)\left(w^T z + b\right)\right) \\
        &= z^\m \overline{z}^{\elll} \cdot \wirt^{e_j}\left[ w \mapsto \left(\wirt^{\vert \m \vert} \wirtq^{\vert {\elll} \vert} \phi \right)\left(w^T z + b\right)\right].
    \end{align*}
    Using the chain rule again, we calculate
    \begin{align*}
        &\norel \wirt^{e_j}\left[w \mapsto \left(\wirt^{\vert \m \vert} \wirtq^{\vert {\elll} \vert} \phi \right)\left(w^T z + b\right)\right] \\
        &= \left(\wirt^{\vert \m \vert +1}\wirtq^{\vert {\elll} \vert}\phi\right)\left(w^Tz + b\right) \cdot \wirt^{e_j}\left[ w \mapsto w^Tz + b\right] \\
        &\norel + \left(\wirt^{\vert \m \vert }\wirtq^{\vert {\elll} \vert + 1}\phi\right)\left(w^Tz + b\right) \cdot \wirt^{e_j}\left[w \mapsto \overline{w^Tz + b}\right] \\
        &= z^{e_j}\cdot \left(\wirt^{\vert \m \vert +1}\wirtq^{\vert {\elll} \vert}\phi\right)\left(w^Tz + b\right) + \left(\wirt^{\vert \m \vert }\wirtq^{\vert {\elll} \vert + 1}\phi\right)\left(w^Tz + b\right) \cdot \overline{\wirtq^{e_j}\left[w \mapsto w^Tz + b\right]} \\
        &=z^{e_j}\cdot \left(\wirt^{\vert \m \vert +1}\wirtq^{\vert {\elll} \vert}\phi\right)\left( w^Tz + b\right).
    \end{align*}
    By induction, this proves the claim.
\end{proof}
The preceding result shows that monomials can be represented by Wirtinger derivatives, which in turn are linear combinations of "usual" partial derivatives, see \Cref{wirtreal}. We now show that these partial derivatives can be approximated by shallow neural networks.
\begin{lemma}
\label{previous}
    Let $\phi:\CC \to \CC$ and $\delta>0, \ b \in \CC, \ k \in \NN_0$, such that $\fres{\phi}{B_\delta(b)} \in C^k      \left(B_\delta(b); \CC\right)$. Let $m,n \in \NN$ and $\varepsilon>0$. For $\pp \in \NN_0^{2n}, h>0$ and $z \in \Omega_n$ we write
    \begin{align*}
        \Phi_{\pp,h} (z)  &\defeq (2h)^{-\vert \pp \vert} \sum_{0 \leq \textbf{r} \leq \textbf{p}} (-1)^{\vert \pp \vert -\vert \rr \vert} \binom{\textbf{p}}{\textbf{r}} \left(\phi_z \circ \varphi_n\right)\left(h(2\rr - \pp)\right) \\ 
        &=(2h)^{-\vert \pp \vert} \sum_{0 \leq \textbf{r} \leq \textbf{p}} (-1)^{\vert \pp \vert -\vert \rr \vert} \binom{\textbf{p}}{\textbf{r}} \phi \left( \left[\varphi_n\left(h(2\rr-\pp)\right)\right]^T \cdot z + b\right),
    \end{align*}
where $\phi_z$ is as introduced in \Cref{extraction} and $\varphi_n$ is as in \eqref{isomorphism_intro} on page 5.
    Furthermore, let
    \begin{equation*}
        \phi_\pp : \quad \Omega_n \times B_{\frac{\delta}{\sqrt{2n}}}(0)\to \CC, \quad (z,w) \mapsto \partial^\pp \phi_z(w).
    \end{equation*}
    Then there exists $h^* > 0$ such that
    \begin{equation*}
        \Vert \Phi_{\pp,h} - \phi_\pp(\cdot, 0) \Vert_{L^\infty\left(\Omega_n; \CC\right)} \leq \varepsilon
    \end{equation*}
    for all $\pp \in \NN_0^{2n}$ with $\vert \pp \vert \leq k$ and $ \pp_j \leq m$ for all $j \in \{1,...,2n\}$ and $h \in (0, h^*)$.
\end{lemma}
\begin{proof}
    Fix $\pp \in \NN_0^{2n}$ with $\vert \pp \vert \leq k$ and $  \pp_j  \leq m$ for all $j \in \left\{ 1,...,2n\right\}$. The map
    \begin{equation*}
        B_{\sqrt{2n}}(0) \times B_{\delta / \sqrt{2n}} (0) \to \CC, \quad (z,w) \mapsto \phi \left(w^T z + b\right)
    \end{equation*}
    is in $C^k$ since
    \begin{equation*}
        \left\vert w^T z \right\vert \leq \Vert w \Vert \cdot \Vert z \Vert < \frac{\delta}{\sqrt{2n}} \cdot \sqrt{2n} = \delta.
    \end{equation*}
    Therefore, the map 
    \begin{equation*}
        B_{\sqrt{2n}}(0) \times B_{\delta / \sqrt{2n}} (0) \to \CC, \quad (z,w) \mapsto \partial^\pp \phi_z(w)
    \end{equation*}
    is continuous and in particular uniformly continuous on the compact set
    \begin{equation*}
        \Omega_n \times \overline{B_{\delta/2n}}(0) \subseteq B_{\sqrt{2n}}(0) \times B_{\delta / \sqrt{2n}}(0).
    \end{equation*}
    Hence, there exists $ h_\pp \in (0,\frac{\delta}{2mn})$, such that
    \begin{equation*}
        \left\vert \phi_\pp(z, \xi) - \phi_\pp (z,0)\right\vert \leq \frac{\varepsilon}{\sqrt{2}}
    \end{equation*}
    for all $\xi \in \varphi_{n} \left(h \cdot [-m,m]^{2n}\right), \ h \in (0, h_\pp)$ and $z \in \Omega_n$. Now fix such an $h \in (0, h_\pp)$ and $z \in \Omega_n$. Applying \Cref{divdiff} to both components of $\left(\varphi_1^{-1} \circ \Phi_{\pp, h}\right)(z)$ and $\varphi_1^{-1} \circ \phi_z \circ \fres{\varphi_n}{\left(-\frac{\delta}{2n},\frac{\delta}{2n}\right)^{2n}}$ separately yields the existence of two real vectors $\xi_{1}, \ \xi_{2} \in h \cdot [-m,m]^{2n}$, such that
    \begin{equation*}
        \left(\varphi_1^{-1} \circ \Phi_{\pp,h}(z)\right)_1= \left[\partial^\pp \left(\varphi_1^{-1} \circ \phi_z \circ \varphi_n\right)\left(\xi_1\right)\right]_1, \ \left(\varphi_1^{-1} \circ \Phi_{\pp,h}(z)\right)_2= \left[\partial^\pp \left(\varphi_1^{-1} \circ \phi_z \circ \varphi_n\right)\left(\xi_2\right)\right]_2.
    \end{equation*}
    Rewriting this yields
    \begin{equation*}
        \RE\left(\Phi_{\pp,h}(z)\right) = \RE\left(\phi_\pp (z, \varphi_n\left(\xi_1\right))\right) \quad \text{and}\quad  \IM\left(\Phi_{\pp,h}(z)\right) = \IM\left(\phi_\pp (z, \varphi_n\left(\xi_2\right))\right).
    \end{equation*}
    Using this property we deduce
    \begin{equation*}
        \left\vert \text{Re}\left( \Phi_{\pp,h}(z) - \phi_\pp(z,0)\right)\right\vert = \left\vert \text{Re}\left( \phi_\pp(z, \varphi_n\left(\xi_{1}\right)) - \phi_\pp(z,0)\right)\right\vert \leq \left\vert \phi_\pp(z, \varphi_n\left(\xi_{1}\right)) - \phi_\pp (z,0)\right\vert \leq \frac{\varepsilon}{\sqrt{2}}
    \end{equation*}
    and analogously for the imaginary part. Since $z \in \Omega_n$ and $h \in \left(0, h_\pp \right)$ have been chosen arbitrarily we get the claim by choosing
    \begin{equation*}
        h^* \defeq \text{min} \left\{ h_\pp  : \  \pp \in \NN_0^{2n} \text{ with } \vert \pp \vert \leq k \text{ and }\underset{j}{\text{max}} \ \pp_j \leq m\right\}. \qedhere
    \end{equation*}
\end{proof}

\begin{definition}
    For $m \in \NN_0$ we define
    \begin{equation*}
        \mathcal{P}_m := \left\{ \CC^n \to \CC, \ z \mapsto \sum_{ \m  \leq m} \sum_{ \elll  \leq m} a_{\m,\elll}  z^\m  \overline{z}^{\elll} : \ a_{\m, \elll} \in \CC\right\}
    \end{equation*}
    as the space of functions from $\CC^n$ to $\CC$ that are represented by a complex polynomial in $z$ and $\overline{z}$ of coordinatewise degree at most $m$.
\end{definition}

We can now state and prove the first major result of this paper, which already appeared in a slightly simplified form in the introduction as \Cref{main_1}. Specifically, we are going to show that it is possible to approximate complex polynomials in $z$ and $\overline{z}$ of coordinatewise degree of at most $m$ using shallow complex-valued neural networks with $(4m+1)^{2n}$ neurons in the hidden layer of the network, provided that the activation function is $mn$-admissible.
\begin{theorem}
\label{polyapprox}
    Let $m,n \in \NN$, $\varepsilon > 0$ and let $\phi: \CC \to \CC$ be $mn$-admissible in $b \in \CC$.
     Let $\PP' \subseteq \PP_m$ be bounded and set $N := (4m+1)^{2n}$. Then there exist certain coefficients $\rho_1, ..., \rho_N \in \CC^n$ with the following property: For each polynomial $p \in \mathcal{P}'$ there are coefficients $\sigma_1, ..., \sigma_N \in \CC$, such that
    \begin{equation*}
        \left\Vert p - f\right\Vert_{L^\infty \left(\Omega_n; \CC\right)}  \leq \varepsilon,
    \end{equation*}
    where 
    \begin{equation}
    \label{polynomform}
        f: \quad \Omega_n \to \CC, \quad z \mapsto \sum_{j=1}^N \sigma_j \phi\left(\rho_j^T z + b\right). 
    \end{equation}
\end{theorem}
\begin{proof}
Let $p \in \PP'$. Fix $\m, \elll \in \NN_0^{n}$ with $ \m, \elll \leq m$. For each $z \in \Omega_n$, using \Cref{extraction}, we then have
\begin{equation*}
    z^\m \overline{z}^{\elll} = \left[\left( \wirt^{\vert\m\vert} \wirtq^{\vert\elll\vert} \phi\right)(b)\right]^{-1}\wirt^\m \wirtq^{\elll} \phi_z (0) \overset{\ref{wirtreal}}{=} \left[\left( \wirt^{\vert\m\vert} \wirtq^{\vert\elll\vert} \phi\right)(b)\right]^{-1}  \cdot \underset{\pp' + \pp'' = \m + \elll}{\underset{ \pp', \pp'' \in \NN_0^n}{\sum}} b_{\pp', \pp''}\partial^{(\pp', \pp'')} \phi_z(0)
\end{equation*}
with suitably chosen complex coefficients $b_{\pp', \pp''} \in \CC$ depending only on $\pp', \ \pp'', \ \m$ and $\elll$. Here we used that $\vert \m \vert, \ \vert \elll \vert \leq mn$ and that $\phi$ is $mn$-admissible in $b$. 
Since $\mathcal{P}'$ is bounded and $p \in \PP'$, we can write
\begin{equation*}
p(z) = \underset{\m, \elll \leq m}{\sum_{\m, \elll \in \NN_0^n}} a_{\m, \elll} z^\m \overline{z}^{\elll}
\end{equation*}
with $\vert a_{\m, \elll}\vert \leq c$ for some constant $c = c(\PP') > 0$. This easily implies that we can rewrite $p$ as
\begin{equation*}
    p(z) = \sum_{ \pp  \leq 2m} c_\pp(p) \partial^\pp \phi_z(0)
\end{equation*}
with coefficients $c_\pp(p) \in \CC$ satisfying $\vert c_\pp (p)\vert \leq c'$ for some constant $c' = c'(\phi, b, \PP', m, n)$. By \Cref{previous}, we choose $h^*>0$, such that
\begin{equation*}
    \left\vert \Phi_{\pp,h^*}(z) - \partial^\pp\phi_z(0)\right\vert \leq \frac{\varepsilon}{\sum_{ \pp \leq 2m}c'}
\end{equation*}
for all $z \in \Omega_n$ and $\pp \in \NN_0^{2n}$ with $ \pp \leq 2m$ (note that we then automatically have $\vert \pp \vert \leq 2mn$ and that $\fres{\phi}{B_\delta(b)} \in C^{2mn}(B_\delta(b); \CC)$ for some $\delta > 0$). Furthermore, we can rewrite each function $\Phi_{\pp, h^*}$ as
\begin{equation*}
    \Phi_{\pp, h^*}(z) = \sum_{\underset{\vert \aalpha_j \vert \leq 2m  \ \forall j}{\aalpha \in \Z^{2n}}} \lambda_{\aalpha, \pp} \phi(\left[\varphi_n\left(h^* \aalpha\right)\right]^T \cdot z + b)
\end{equation*}
with suitable coefficients $\lambda_{\aalpha, \pp} \in \CC$. Since the cardinality of the set 
\begin{equation*}
    \left\{ \aalpha \in \Z^{2n} : \ \left\vert \aalpha_j \right\vert \leq 2m \ \forall j\right\}
\end{equation*}
is $(4m+1)^{2n}$, this can be converted to 
\begin{equation*}
   \Phi_{\pp, h^*}(z) =  \sum_{j=1}^N \lambda_{j,\pp} \phi \left( \rho_j^T \cdot z + b\right).
\end{equation*}
If we then again take any arbitrary polynomial $p \in \PP'$, we define
\begin{equation*}
    \theta(z) \defeq \sum_{\pp \leq 2m} c_\pp(p) \cdot \Phi_{\pp, h^*}(z)
\end{equation*}
and note
\begin{equation*}
    \left\vert \theta(z) - p(z)\right\vert \leq \sum_{\pp \leq 2m} \left\vert c_\pp(p) \right\vert \cdot \left\vert \Phi_{\pp, h^*}(z) - \partial^\pp \phi_z(0)\right\vert \leq \varepsilon.
\end{equation*}
Since the coefficients $\rho_j$ have been chosen independently from the polynomial $p$, we can rewrite $\theta$ so it has the form (\ref{polynomform}).
\end{proof}