% !TeX encoding = UTF-8
% !TeX spellcheck = en_US
% !TeX root = main_paper.tex

\section{Admissibility of Activation Functions}
\label{admissibility}

In this section we discuss the notion of admissibility of activation functions $\phi: \CC \to \CC$, i.e., the sufficient conditions that a function has to fulfill in order to be suitable for the approximation results established in Sections \ref{approx_polynomials} and \ref{ck_functions}. We are going to define a slightly weaker notion of admissibility than that the function is smooth and non-polyharmonic on a non-empty open subset of $\CC$.

Let us recall that in \cite{mhaskar_neural_1996} the sufficient condition on an activation function $\phi: \RR \to \RR$  is that it is smooth on an open interval with one point in the interval where no derivative vanishes. In fact, the proofs in \cite{mhaskar_neural_1996} show that it is sufficient to assume that for every $M \in \NN$ there is an open set $\emptyset \neq U \subseteq \RR$ such that $\fres{\phi}{U} \in C^M \left( U ; \RR\right)$ with a point $b \in U$ satisfying
\begin{equation*}
f^{(n)} (b) \neq 0 \text{ for all } n \leq M.
\end{equation*}  
In the case of complex-valued neural networks it turns out that the usual real derivative has to be replaced by mixed Wirtinger derivatives of the form $\wirt^m \wirtq^\ell$. This leads to the following definition.
\begin{definition}
Let $\phi: \CC \to \CC$ and $M \in \NN_0$. We call $\phi$ $M$\emph{-admissible} (in $b \in \CC$) if there is an open set $U \subseteq \CC$ with $b \in U$ and $\fres{\phi}{U} \in C^{2M} (U ; \CC)$ such that
\begin{equation*}
\wirt^m \wirtq^\ell \phi (b) \neq 0 \text{ for all } m,\ell \leq M.
\end{equation*}
\end{definition}
In order to approximate complex polynomials in $z$ and $\overline{z}$ with bounded degree it will be enough to assume $M$-admissibility of an activation function for a certain $M \in \NN$. However, if one wants to approximate arbitrary $C^k$-functions one needs the following definition.
\begin{definition}
A function $\phi : \CC \to \CC$ is called \emph{admissible} if it is $M$-admissible for every $M \in \NN_0$.
\end{definition}
Note that for an admissible function there does not necessarily have to be an open set $\emptyset \neq U \subseteq \CC$ such that the function is smooth on $U$. However, if we assume smoothness we can derive the following elegant result.
\begin{theorem}
\label{charac}
    Let $\phi: \CC \to \CC$, $\emptyset \neq U \subseteq \CC$ an open set and $\fres{\phi}{U} \in C^\infty(U ; \CC)$. Then the following are equivalent:
    \begin{enumerate}
        \item $\fres{\phi}{U}$ is not polyharmonic.
        \item For every $M \in \NN_0$ there exists $z_M \in U$ such that $\phi$ is $M$-admissible at $z_M$. 
    \end{enumerate}
In particular, in both cases $\phi$ is admissible.
\end{theorem}
\begin{proof}
    (1) $\Rightarrow$ (2): 
    Let $M \in \NN_0$. Since $\fres{\phi}{U}$ is not polyharmonic we can pick $z \in U$ with $\Delta^M \phi (z) \neq 0$. By continuity we can choose $\delta > 0$ with $B_\delta(z) \subseteq U$ and $\Delta^M \phi(w) \neq 0$ for all $w \in B_\delta(z)$. For all $m, \ell \in \NN_0$ let
    \begin{equation*}
        A_{m,\ell} \defeq \left\{ w \in B_\delta(z) : \ \wirt^m \wirtq^\ell \phi (w) = 0\right\}
    \end{equation*}
    and assume towards a contradiction that
    \begin{equation*}
        \bigcup_{m,\ell \leq M} A_{m,\ell} = B_\delta(z).
    \end{equation*}
    By \cite[Corollary 3.35]{aliprantis_infinite_2006}, $B_\delta(z)$ with its usual topology is completely metrizable. By continuity of $\wirt^m \wirtq^\ell \phi$ the sets $A_{m,\ell}$ are closed in $B_\delta(z)$. Hence, using the Baire category theorem \cite[Theorems 3.46 and 3.47]{aliprantis_infinite_2006}, there are $m,\ell \in \NN_0$ with $m,\ell \leq M$, $z' \in A_{m,\ell}$ and $\varepsilon > 0$ such that
    \begin{equation*}
        B_\varepsilon\left(z'\right) \subseteq A_{m,\ell} \subseteq B_\delta(z).
    \end{equation*}
    But thanks to $\Delta^M \phi = 4^M \wirt^M \wirtq^M \phi = 4^M \wirt^{M - \ell} \wirtq^{M-m}\wirt^\ell \wirtq^m \phi$ (see (\ref{laplace_ident}) on p. 6), this directly implies $\Delta^M \phi \equiv 0$ on $B_{\varepsilon} \left(z'\right)$ in contradiction to the choice of $B_\delta(z)$. 
\medskip

(2) $\Rightarrow$ (1): If $\fres{\phi}{U}$ is polyharmonic there is $M \in \NN_0$ with
    \begin{equation*}
        \wirt^M \wirtq^M \phi(z) = 0
    \end{equation*}
    for every $z \in U$, which contradicts (2).
\end{proof}
\Cref{charac} justifies the formulation of \Cref{main_1,,main_2}, since it shows that every activation function $\phi$ satisfying the assumptions of the theorem is $M$-admissible for all $M \in \NN_0$.

\begin{remark}
In the case of a real function it follows that if the function is smooth and non-polynomial on an open interval, there has to be a point at which no derivative vanishes, see \cite[p. 53]{donoghue_distributions_1969}. It is an interesting and currently unsolved question if a similar statement also holds true in the case of complex functions and Wirtinger derivatives, i.e., if the following holds: Let $z \in \CC, \ r > 0$ and $\phi \in C^\infty (B_r(z); \CC)$ be non-polyharmonic. Then there exists $b \in B_r(z)$ such that
\begin{equation*}
\wirt^m \wirtq^\ell \phi (b) \neq 0 \text{ for all } m,\ell \in \NN_0.
\end{equation*}
\end{remark}



