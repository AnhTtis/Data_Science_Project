\section{Introduction}
\label{sec:intro}
Song translation is a meaningful human endeavor to climb high Tower of Babel for inter-culture exchange.
% Song translation, a meaningful human endeavor for inter-culture exchange, 
Yet it has not received much attention in the natural language processing~(NLP) community despite the advancement of machine translation technologies, especially Neural Machine Translation (NMT)~\citep{nmt, vaswani2017attention, hassan2018achieving}, and the expanding interests of solving real-world problems using artificial intelligence techniques. Challenges include the lack of efficient means to collect parallel lyrics and alignment data, the difficulty of modeling the complex interaction between texts and melody and imperceptive evaluation of scores.
While closely related to text translation, song translation is a more involved task. In addition to the general considerations of word choice and word order in translation, human translators of songs need to have a mastery of cultural traditions and the poetic usage of both source and target languages. Furthermore, the translated lyrics need to be properly aligned with the melody, as shown in Figure \ref{fig:task_exp}, to maintain the intact beauty of the song, a factor that is indispensable in song translation~\citep{three_d_of_singability}.


% Song translation, a meaningful endeavor especially for the exchange of cultures, has not received enough attention in the natural language processing community partly because it seems to require interdisciplinary expertise in at least two areas, that of NLP and that of music, and partly because its sister problem, that of lyrics translation, seems to be readily subsumed in the general field of machine translation. 
% It is the task of translating the lyrics of a song from one language to another while maintaining the integrity between words and music. 
% While Neural Machine Translation (NMT)~\citep{nmt, vaswani2017attention, hassan2018achieving} has seen much success in domains such as news, translation technology for other forms of text, particularly those of art, is lacking, possibly with good reasons. 
% Song translation by humans is a more involved task than other text translations. 
% In addition to general translation considerations such as diction and word order, human translators of songs need to have a mastery of poetic usage of language and cultural traditions of both the source and the target. 
% Furthermore, the translated lyrics need to be re-aligned with the melody so that they can be sung appropriately. 

\begin{figure}
    \centering
    \includegraphics[width=0.5\textwidth]{figures/exp.pdf}
    \caption{An example of the comprehensive translation for ``But you play it to the beat'' in \textit{Rolling In the Deep}. }
    % \modelname~aims to translate the lyrics and predict the number of aligned notes for each target token.
    \label{fig:task_exp}
\end{figure}

Researchers have explored Singing Voice Synthesis (SVS)~\citep{diffsinger, pndm, nsvb} to automate the vocal singing of songs
% , and proposed approaches to produce
% natural and accurate singing voice with realistic vocal timbre 
given the input lyrics and scores, which laid the foundation of convenient and perceptive evaluation and a prospective empirical usage of automatically generated songs. However, there is very few previous studies in the direction of Automatic Song Translation (AST). The sole work~\citep{gagast} we are aware of focuses on matching tones and rhythms for the translated target words for tonal languages, by imposing constraints during NMT inference. Their direct use of text translation models and strict mapping between notes and tokens, however, is unable to capture the more involved nature of song translation. While the number of notes provides an easy upper bound on the length of translation, the delicate alignment between lyrics and melody, as observed in \citet{interplay_lyrics_melody}, should not be dictated solely by simple rigid rules. 

% Various stages in this process have been the goals of automation. 
% For example, the vocal singing of songs has been extensively explored as a Singing Voice Synthesis (SVS) problem~\citep{diffsinger, pndm, nsvb}. 
% It is capable of producing natural and accurate singing voice with realistic vocal timbre given the input lyrics and scores. 
% Recent Automatic Song Translation (AST)~\citep{gagast} research focuses on the translation quality of the lyrics by imposing special constraints on the decoder, for example, on the length or on the tones of words if the target language is tonal. 
% Although these special considerations ensure the length of the lyrics does not exceed that of the music, the entire process is still more text translation than song translation. 
% For song translation, \citet{three_d_of_singability} has shown that singability is an indispensable factor a crucial component of which is the alignment between the lyrics and the notes. 
% In this area, both rigid and flexible alignments have been proposed. 
% \citet{gagast} ) is a one-to-one mapping between notes and words, while \citet{interplay_lyrics_melody} is a flexible alignment which works more reasonably in a real situation. 
% Although the number of notes for one verse of lyrics is an easy upper bound of the length of the translation, more can be utilized for better results. 
% In the auto-regression translation model, the correct number of aligned notes can be dynamically determined and assigned to the corresponding target words in the lyrics.

In this paper, we propose Lyrics-Melody Translation with Adaptive Grouping (\modelname), the first comprehensive solution to the AST problem, by jointly modeling of both lyrics translation and lyrics-melody alignment within the transformer-based encoder-decoder framework. \modelname~incorporates both lyrics and melody in an end-to-end manner and employs an adaptive grouping module to explicitly model the alignment between lyrics and melody. To facilitate training, we produce the first (Chinese-English) bilingual lyrics-melody alignment data set. To address the data scarcity problem, we also generate a large amount of bilingual lyrics-melody data through back-translation of monolingual lyrics-melody alignment data, which is used together with the high quality manual annotations through a curriculum training strategy. Our experiments show that songs translated by \modelname~are both faithful to the original lyrics and singable to the melody, as measured by both automatic metrics and human judges majoring in music. Main contributions of this work are as follows:

% To tackle these challenges, on top of the transformer encoder-decoder framework, we design a novel adaptive grouping module for implicit alignment inference. 
% Our model of Lyrics-Melody Translation with Adaptive Grouping (\modelname) is the first co-translation framework to approach the AST problem in a more comprehensive manner as shown in Figure~\ref{fig:task_exp}. 
% For this purpose, we produce the first bilingual lyrics-melody alignment data set. 
% In addition to this supervised set, we also use the popular technique of back translation to add more monolingual lyrics-melody alignment data.
% A curriculum learning way is used to balance the usage between back-translated data and human-annotated data. 
% Songs translated by \modelname~are both faithful to the original lyrics and adapted to the melody. 
% In addition to automatic metrics, we also conduct human evaluations on the results by annotators majoring in music. The main contributions of this work are:

%\begin{enumerate}

\noindent
\textbf{(1)}
%\item 
We propose the first joint lyrics translation and lyrics-melody alignment framework \modelname~to solve the AST task in a comprehensive manner. 

\noindent
\textbf{(2)}
%\item 
We design an adaptive grouping method for monotonic lyrics-melody alignment prediction that helps achieve high-quality lyrics translation and to provide flexible and reasonable lyrics-to-melody alignments in the same auto-regressive process.

\noindent
\textbf{(3)}
%\item 
We produce the first bilingual lyrics-melody alignment data set that will be released publicly to facilitate further research in this field. 
We also leverage the back-translation and the curriculum learning strategy to boost performance.

\noindent
\textbf{(4)}
%\item 
Our experiments show that \modelname~outperforms baselines by a notable margin. 
Human evaluations indicate that our proposed flexible alignments together with lyrics translation achieves satisfying song translation results.

%\end{enumerate}
