
\documentclass{article}
\usepackage{PRIMEarxiv}

\usepackage[utf8]{inputenc} % allow utf-8 input
%\usepackage[a4paper, margin=2cm]{geometry}
%\usepackage[T1]{fontenc}    % use 8-bit T1 fonts
\usepackage{hyperref}       % hyperlinks
\usepackage{url}            % simple URL typesetting
\usepackage{booktabs}       % professional-quality tables
\usepackage{amsfonts}       % blackboard math symbols
\usepackage{nicefrac}       % compact symbols for 1/2, etc.
\usepackage{microtype}      % microtypography
\usepackage{lipsum}
\usepackage{fancyhdr}       % header
\usepackage{graphicx}       % graphics
\graphicspath{{media/}}     % organize your images and other figures under media/ folder


% Figures and Tables
\usepackage{graphicx}
\usepackage[most]{tcolorbox}
\usepackage{caption}
\usepackage{subcaption}
\usepackage{longtable}


% Mathematics 
\usepackage{amsmath,amssymb,amsfonts}

\usepackage{textcomp}

\usepackage{verbatim}

\usepackage{xcolor}
\usepackage{url}
\usepackage{xr-hyper}
\usepackage{hyperref}
\usepackage{pdfpages} 
\usepackage{bm}
\usepackage{multirow}
\usepackage{multicol}
\usepackage{array}
\newcolumntype{L}{>{\arraybackslash}m{3cm}}

\usepackage{fullpage}
\usepackage{rotating}
\usepackage{stmaryrd}
\usepackage{proof}


%% Tikz 
\usepackage{tikz}
\usetikzlibrary{bayesnet}
\usetikzlibrary{decorations.pathmorphing} % noisy shapes
\usetikzlibrary{fit}% fitting shapes to coordinates
\usetikzlibrary{backgrounds}	

% Theorem 
\usepackage{amsthm}
\usepackage{amsmath}
\theoremstyle{definition}
\newtheorem{theorem}{Theorem}[section]
\newtheorem{definition}{Definition}
\newtheorem{lemma}[theorem]{Lemma}
\newtheorem{proposition}[theorem]{Proposition}


% Algorithms 
\usepackage[ruled]{algorithm2e}
\usepackage{algorithmic} %% Used in Chapter 1 


% References 
\usepackage[style=numeric,sorting=none,firstinits=true]{biblatex}
\addbibresource{references.bib}



%Header
\pagestyle{fancy}
\thispagestyle{empty}
\rhead{ \textit{ }} 

% Update your Headers here
\fancyhead[LO]{THOR Paper Series 2}
% \fancyhead[RE]{Firstauthor and Secondauthor} % Firstauthor et al. if more than 2 - must use \documentclass[twoside]{article}

  
%% Title
\title{Deep incremental learning models for temporal tabular datasets with distribution shifts
%%%% Cite as
%%%% Update your official citation here when published 
\thanks{\textit{\underline{Citation}}: 
\textbf{Authors. Title. Pages.... DOI:000000/11111.}} 
}

\author{
  Thomas Wong \\
  Imperial College London \\
  London\\
  \texttt{mw4315@ic.ac.uk} \\
  %% examples of more authors
   \And
  Mauricio Barahona \\
  Imperial College London \\
  London\\
  \texttt{m.barahona@imperial.ac.uk} \\
}


\begin{document}
\maketitle


\begin{abstract}
In this paper, we present a robust incremental learning model for regression tasks on temporal tabular datasets. Using commonly available tabular and time-series prediction models as building blocks, a machine-learning model is built incrementally to adapt to distributional shifts in data.  Using the concept of self-similarity, the model uses only two basic building blocks of machine learning models, gradient boosting decision trees and neural networks to build models for any required complexity. The model is efficient as no specialised neural architectures are used and each model building block can be independently trained in parallel. The model is demonstrated to have robust performances under adverse situations such as regime changes, fat-tailed distributions and low signal-to-noise ratios. Model robustness are studied under different hyper-parameters and complexities.  
%The model is universal to all standardised datasets as no data-dependent feature engineering methods is required.
%Difference use cases of the model are studied, including replicating black-box predictions from other machine-learning methods and understanding the uniqueness of newly proposed machine-learning methods. The model is part of the auto-ML tool, THOR developed by the authors and is available on GitHub. 

\end{abstract}

\keywords{Machine Learning, Time-Series Prediction, Deep Learning, }



\section{Introduction} 
\label{section:overview}

%% At the request of reviwers, need to add more references for background
\paragraph{Background of incremental learning}

Incremental learning is a new research area as machine learning models are increasingly used in real-life applications. A key example would be class incremental learning \cite{zhu2021class,NEURIPS2022_c8ac22c0,NEURIPS2022_ae817e85,} for classification problems. Many important applications, such as Internet of Things (IoT) \cite{song2018situ} and cyber-security \cite{buczak2015survey}, would involve stream data, where data(features) are continuously updated and predictions are made \textbf{point-in-time}. A key challenge in incremental learning is that when there are distributional shifts in data that results in \textbf{model drifts} during inference,  which is the drop of out-of-sample performances of prediction models when the models learnt relationships from the training set that significantly differ from those in the test set. 

Reinforcement learning \cite{arulkumaran2017deep,NEURIPS2022_d112fdd3,NEURIPS2022_eb4898d6,} is a related modelling approach where a model(agent) learns a policy to optimise rewards by interacting with the environment. Reinforcement learning is particular useful when the actions of the model can influence the environment and there are multiple agents interacting with each other \cite{NEURIPS2020_77441296,}. Under the assumption that the actions of models have no influence to the data stream (environment), reinforcement learning reduces to incremental learning when the reward of agents is not used as input features. In other words, the input features (environment) are \textbf{independent} to the actions of the agent(predictions of the model) as there is no feedback loop between the agent and environment. Applying trained reinforcement learning agents into unknown situations, such as trading \cite{deng2016deep}, perception in self-driving cars \cite{NEURIPS2021_0d5bd023} and robotics \cite{kober2013reinforcement} remains a great challenge. Domain adaption for reinforcement learning algorithms \cite{arndt2020meta,higgins2017darla,} bridges the gap between controlled environments and real-life situations using different advanced and complex algorithms. Key challenges such as robustness of agent behaviour \cite{ma2018improved} and interpretability \cite{mott2019towards,} remains unsolved. 

Deep incremental learning, introduced in this paper, is a hybrid approach which allows predictions from (base) models to be reused in future predictions for prediction tasks performed data streams. Unlike reinforcement learning which allows flow of information between agent and environment in both directions, deep incremental learning only allows information to flow from one layer to the next one as a waterfall. The \textbf{point-in-time} nature of predictions are preserved so that no look-ahead bias is introduced. As demonstrated below, this \textbf{single} direction of information flow is enough to create predictions that are better than simple incremental learning models. Deep incremental learning can be considered as an extension of model stacking \cite{naimi2018stacked,} taken the stream nature of data into account. 

The incremental ranking task studied here is based on data streams of tabular features and targets, defined as temporal tabular datasets \cite{wong2023dynamic} as follows. 


%% The beginning of mathematics 

\paragraph{Definitions}

\begin{definition}[Temporal Tabular Datasets]
\label{def:temptable}
A temporal tabular dataset is a collection of matrices $\{ X_i, y_i \}_{1 \leq i \leq T}$ collected over time eras 1 to $T$. Each matrix $X_i$ represents data available at era $i$ with shape $N_i \times M$, where $N_i$ is the number of data samples in era $i$ and $M$ is the number of features describing the samples. $y_i$ are the targets corresponding to the features $X_i$, which can be single-dimensional or multi-dimensional. Note that the definition of the features are fixed throughout the eras, in the sense that the same computational formula is used to compute the features at each era. On the other hand, the number of data samples $N_i$ does not have to be constant across time. In practical applications, the number of data samples in each era is assumed to be globally bounded.
%% Data Lag
Unlike standard online learning problems, where the data arrived can be used to update machine learning models \text{right after} prediction is made, there is a \textbf{fixed} and known time lag for the targets from an era to be known, defined as \textbf{data embargo}. For example, for a dataset with a data embargo of $5$, at era $X$ the features of Era $X$ would be known and then a prediction of the ranking of items at that time could be made. The targets of era $X$ are only known at era $X+5$, which can then be used to calculate the quality of predictions according to a suitably chosen metric. 
%For many applications, the temporal tabular dataset can grow to \textbf{infinite} size. 
\end{definition}


For general tabular datasets, the features can be in different formats, such as numerical, ordinal or categorical. With suitable pre-processing techniques, features can be transformed into equal-sized or Gaussian-binned numerical(ordinal) values. The datasets used in this paper are already standardised into discrete bins. 

Researchers might be tempted to use one-hot encoding to create "categorical" features out of these discrete bins, as they might want to capture the non-linearity effects of features. However, this procedure violates the assumptions made in the data creation process, where the raw features before standardisation are continuous measures \footnote{The datasets considered in this paper consists of factors which capture known economic effects in predicting stock prices, such as momentum, these factors are continuous measures with natural ordering.}. 



There are two approaches to modelling the temporal tabular dataset. The first is to apply standard tabular machine learning models as usual, taking into account the temporal order of data during cross-validation. The second is to model the \textbf{derived} time-series defined in \ref{def:derivedts}.

\begin{definition}[Derived Time Series]
\label{def:derivedts}
Time series can be derived from the temporal tabular data with different transformations. Transformations considered here are restricted to those that can be applied to each data era \textbf{independently}. 
%A major limitation of this assumption is that it precludes the use of feature engineering methods such as Auto-Encoders which are applied across eras. However this assumption can make sure there is no look-ahead bias in the model as these derived time series would be used to build time-series models. 
The transformation can be defined as dimension reduction transformation applied on the matrix of a single slice of tabular features at era $i$ with its targets to a one-dimensional tensor, which is formally defined as $f(X,y): (\mathbb{R}^{N_i \times M}, \mathbb{R}^{N_i \times 1}) \mapsto \mathbb{R}^{M})$ where $X$ and $y$ are the features and targets, $N_i$ is the number of observations at era $i$, $M$ is the number of features. This procedure is defined as deriving the \textbf{feature performances} of the temporal tabular data.
\end{definition}


Any machine learning model can be considered as a sequence of transformations between different tensors. Restricting to transformations between tabular and time-series data only, 4 different basic transformations are obtained as follows:


\begin{enumerate}
    \item Transformation from tabular data to tabular 
    \begin{itemize}
        \item Standard tabular machine learning models which transform the given feature target pair $(X,y)$ into $y'$ where the first(data) dimension of $X,y,y'$ are equal dimensions and $y$ and $y'$ matches all dimensions. The transformations is performed \textbf{point-wise}, where at inference each item can be predicted independently. There are many examples for this class of machine learning models, including (and not limited to) gradient-boosting decision trees, and multi-layer perceptron networks. 
        \item List-wise tabular machine learning models which transform the \textbf{whole} list of items at a time. For a temporal tabular dataset, this means the item rankings are predicted \textbf{all at once}. Examples include various list-wise models for learn-to-rank problems \cite{li2020learning}. 
    \end{itemize}
    \item Transformation from tabular data to time-series
    \begin{itemize}
        \item Deriving time series using the procedure defined in \ref{def:derivedts} or other suitable transformations. 
    \end{itemize}   
    \item Transformation from time-series to tabular data 
    \begin{itemize}
        \item Feature Engineering methods for time-series data such as Signature transforms \cite{Terry22} and Random Fourier transforms \cite{Sutherland15} which transforms a slice of time-series into a single dimensional tensor which captures the characteristics of the time-series. 
    \end{itemize}       
    \item Transformation from time-series to time-series 
    \begin{itemize}
        \item Sequence models which are common in deep learning, such as LSTM \cite{HochSchm97} and Transformers \cite{Bryan19}. 
    \end{itemize}
\end{enumerate}


The above transformations forms the basis of the deep incremental learning model. For the tabular prediction tasks, any chain of transformations starting with tabular features and ending with tabular targets can be used. Apart from standard tabular models, \textbf{factor-timing} models can be used. The features are first transformed into a time series, capturing the history of feature performances and then using the ranking of predicted feature performances to formulate a factor-timing portfolio. The time-series model for predicting feature performances could be any of the above-mentioned methods. 


\paragraph{How incremental learning is different from traditional machine learning} 


The incremental nature of prediction tasks on data streams challenges the traditional assumptions of machine learning problems. In particular, a (single-pass) cross-validation which splits the data into \textbf{fixed} training, validation and test periods is not the most suitable framework. Instead, an incremental learning framework should be used to retrain and/or update model parameters. Under this framework, a model is represented by a continuous stream of parameters instead of a single set of parameters. The procedure also adds new hyper-parameters, training size and retrain period to the machine learning model. In practice, these hyper-parameters could be selected based on computational resources available rather than optimised. The size of memory will limit the maximum training size and the amount of GPU or other processing units available will limit how often the models are retrained/updated. 

When retrain period is greater than 1, there is an extra degree of freedom in when to start the incremental learning procedure. For example, if models are retrained every 10th era, then starting the incremental learning procedure at Era 1, Era 2, ... to Era 10 would give 10 different training procedures using different training sets. This choice could have a non-negligible impact on prediction performances \cite{hoffstein2020rebalance} for non-stationary datasets. 


The clear distinction between features, targets and predictions is also blurred in the incremental learning setting, as predictions from each model can be used as additional features in building other models. New targets can also be created by subtracting against the predictions made. This suggests model training should be considered as a \textbf{multi-step} problem instead of a \text{single-step} problem. 

As an example, consider an incremental learning problem with a data lag of 1, then in figure \ref{fig:gantt},  three models are trained under the incremental learning framework. The training period of each model is fixed at 4. At Era 6, information up to Week 4 (where both the features and targets are known) is used to train Model 1 and then obtain predictions from Era 6 on-wards. At Era 11, information up to Week 9 is used for model training. Features between Week 6 and Week 9 are combined with predictions from Model 1 to  train a new model (Model 2). Similarly, at Era 16,  Model 3 is trained using data from Era 11 to Era 14, which consists of the original given features, and predictions from Model 1 and Model 2. 

\begin{figure}
    \centering
    \includegraphics[width=\textwidth]{figure/chapter3/gantt.png}
    \caption{Example of reusing model predictions in an incremental learning model}
    \label{fig:gantt}
\end{figure}

There are various benefits of reusing model predictions within an incremental learning model. It provides a hierarchical structure of models, in which each model can be interpreted as an improvement/adaption of the previous ones to distributional shifts in data and the prediction quality of each model can be inspected independently. Moreover, reusing model predictions introduced a feedback learning loop where model predictions correct themselves in an incremental manner.  

%Finally, it allows machine learning models to process an infinite stream of data with a finite memory size. By limiting to training each model using the original features and predictions from (a fixed number of) previous models \textbf{only} for a training set with fixed window size, the size of the training set is capped as both the number of features and observations are bounded. Information from previous eras can be passed indirectly to the latest machine learning models. 


The above incremental learning model can be applied to not just a \textbf{single} machine learning model, but a \textbf{collection} of machine learning models in parallel, which is defined as \textbf{layers} of models. The notion is borrowed from MLP, where each node within a layer is now an independently trained model. 

A major difference between our incremental learning model from MLP is that training is done in a single forward pass rather than by back-propagation. This allows us to train very complicated models even with little resources, as there is no need to put the whole model in (distributed) memory to pass gradients between layers. As each model within a layer can be trained independently, training models \textbf{within} a layer becomes an embarrassingly parallel task when multiple GPUs can be used. The program code of the model is also easier to maintain as there is no need to use specialised software packages to distribute data between GPUs.

Recent work in deep learning research suggests back-propagation is not strictly necessary for model training \cite{Hinton22}, Deep Regression Ensemble (DRE) \cite{Kelly22deep} is an example where back-propagation is not used in training. The model is built by training multiple layers of ridge regression models with random feature projections. The model can be considered as a multi-layer perceptron network where some layers have frozen weights. The deep incremental model presented here is significantly \textbf{different} from DRE as no restrictions are made on the machine learning models used. 


\subsection{Practical considerations} 
Unlike standard machine learning problems where model training is done offline, where memory and computational time is usually not a major consideration in model design, incremental learning models deployed to live data streams would be limited by both the memory and computational time during inference and online retrain/update of model parameters. The exact requirements varied between problems and therefore it is not possible to offer an one-size-fits-all solution.

All models in this paper are trained on a single CUDA-enabled GPU of 10GB memory for model training. While this assumption precludes the use of very advanced deep learning methods, incremental learning models(models) can be built for any model complexity/expressiveness using commonly used tabular and time-series models as building blocks. The model can be designed to process an infinite data stream of temporal tabular data effectively without ever-growing memory consumption, by caching only the latest values in the incremental learning process. %In other words, an arbitrarily complex model using a \textbf{fixed} amount of computational resources can be built given a sufficient amount of computational time. 



\section{Machine Learning Models for Time Series Datasets} 
\label{section:ML-TS}

% Extract features 
\paragraph{Multi-variate time-series}
A multi-variate time-series $X$ of $T$ steps and $N$ channels can be represented as $X = (\bm{x}_1, \bm{x}_2, \dots, \bm{x}_i, \dots, \bm{x}_T) $, with $1 \leq i \leq T$ and each vector $\bm{x}_i \in \mathbb{R}^N$ represents the values of the $N$ channels at time $i$. The number of channels of the time series is assumed to be fixed throughout time with regular and synchronous sampling, i.e. the values in each vector from multiple channels arrive at the same time at a fixed frequency. 

\begin{definition}[Time Series Model] 
Given a time series $X_T \in \mathbb{R^{T\times M}} $ where $T$ is size of time dimension and $M$ is the number of features. A (one-step) ahead time-series model is a function $f: \mathbb{R^{T\times M}} \mapsto \mathbb{R}^M$. 
\end{definition}

In practice, the function $f$ is often learned by training statistical/machine learning models using future values of $X_T$, which is obtained by shifting the values of $X_T$ across the time dimension. 


\subsection{Factor Timing Models}

Factor-timing models are created based on predicted rankings of features from the time series models as shown in Algorithm \ref{alg:factor-timing}. The raw predicted values from the time-series models are converted into normalised rankings which can be used as weights of a linear factor-timing model. Within an incremental learning model, a new factor-timing model is trained at each time step using the latest historical values of the derived feature importance time series. 
 
\begin{algorithm}[hbt!]
\caption{Factor Timing Model}
\label{alg:factor-timing}
\KwIn{At prediction era $t$, predicted values from time series model $\hat{y}_t \in \mathbb{R}^d$, temporal tabular dataset $X_t \in \mathbb{R}^{N_t \times d}$ where $d$ is the number of features}
\KwOut{factor timing model $\hat{z}_t \in \mathbb{R}^{N_t}$ }
Calculate the normalised ranking of features $\hat{r}_t$ from predictions of the time series model 
\begin{equation*}
    \hat{r}_t = \text{rank}(\hat{y}_t) - 0.5
\end{equation*}
where rank is the function which calculates the percentile rank of a value within a vector.  $\hat{r}_t$ are ranged between $-0.5$ and $0.5$ \\
Apply truncation to the normalised ranking of features if needed, given upper bound $u$ and lower bound $l$ on the normalised rankings, $-0.5 < l < u <0.5$
\begin{equation*}
    r_t = \max(\min(\hat{r}_t,u),l)
\end{equation*} 
Calculate linear factor-timing predictions $\hat{z}_t = X_t r_t $ 
\end{algorithm}


\subsection{Statistical Rules}
\label{section:stats}

Simple statistical rules can be applied on each feature time-series \textbf{independently} to summarise the history of time-series. Moving averages is a very common statistical method to capture trends in time-series. In this paper, rather than fixing different look-back sizes to compute the moving averages of time-series, exponential moving averages (EMA) with different weight decays are used. This approach avoids the need of warmup period in computing traditional window-based moving averages. The formula to compute EMA is given in the following. 


\begin{definition}[Exponential Moving Average]
\label{equation:EMA}
Given an univariate time series $X = (X_1,X_2,\dots,X_t,\dots)$, the exponential moving average $y_t$ of the time-series at time $t$ with weight decay $\alpha$ is defined as 
\begin{align*}
    y_1 &= X_1 \\
    y_t &= (1-\alpha) y_{t-1} + \alpha X_t
\end{align*}
\end{definition}


\subsection{Feature Engineering Models}
\label{section:feature-eng-multi}

Feature engineering methods can be applied to multi-variate time series so that tabular features are obtained.  

\paragraph{Feature extraction}
Feature extraction methods are defined as functions that map the two-dimensional time-series $X \in \mathbb{R}^{T \times N}$ to a one-dimensional feature space $f(X) \in \mathbb{R}^K$ where $K$ is the number of features. Feature extraction methods reduce the dimension and noise in time-series data. With feature extraction methods, traditional machine learning models such as ridge regression or gradient boosting decision trees can be used without relying on advanced neural network architectures such as Recurrent Neural Networks (RNN), Long-Short-Term-Memory (LSTM) Networks or Transformers \cite{Bryan19}. 

%% Look-back windows
\paragraph{Look-back windows}
For time series which can potentially grow in infinity, a look-back window is needed to restrict the data size when calculating features from time series. To avoid look-ahead bias, features that represent the state of time-series at time $i$ can only be calculated using values obtained up to time $i$, which is $(\bm{x}_1, \bm{x}_2, \dots, \bm{x}_i)$. In most use cases, data collected more recently often have more importance than data collected from a more distant past. Therefore, analysis is often restricted to use the most recent $k$ data points only, which are $(\bm{x}_{i-k}, \bm{x}_{i+1-k}, \dots, \bm{x}_i)$. This represents the state of the time series at time $i$ with a look-back window of size $k$. Feature extraction methods are applied on data within the look-back window only. Multiple look-back windows can be used to extract features corresponding to short-term and long-term price trends. At each time $i$, features extracted with different look-back windows are concatenated to represent the state of the time series.  

To model multivariate time series effectively, better methods which can be applied on multiple time-series in parallel are needed. In this study, both deterministic and random transformations are considered to create tabular features for the prediction task. 

% Signatures
\subsubsection{Signature Transforms} 
Signature transforms \cite{Lyons07, Chevyrev16, Terry22} are \textbf{deterministic} transformations based on rough path theory which can be used to extract features from multi-variate time series. Signature transforms are applied on continuous paths. A path $X$ is defined as a continuous function from a finite interval $[a,b]$ to $\mathbb{R}^d$ with $d$ the dimension of the path. $X$ can be parameterised in coordinate form as $X_t = (X_t^1,X_t^2,\dots,X_t^d)$ with each $X_t^i$ being a single dimensional path. 

% Iterated Integrals 
For each index $ 1 \leq i \leq d$, the increment of $i$-th coordinate of path at time $t \in [a,b]$, $S(X)_{a,t}^i$, is defined as 
\begin{equation*}
    S(X)_{a,t}^i = \int_{a<s<t} \mathrm{d}X_s^i = X_t^i - X_a^i
\end{equation*}
As $S(X)_{a,\cdot}^i$ is also a real-valued path, the integrals can be calculated iteratively. A $k$-fold iterated integral of $X$ along the indices $i_1,\dots,i_k$ is defined as 
\begin{equation*}
    S(X)_{a,t}^{i_1,\dots,i_k} = \int_{a<t_k<t} \dots \int_{a<t_1<t_2}   \mathrm{d}X_{t_1}^{i_1}  \dots \mathrm{d}X_{t_k}^{i_k} 
\end{equation*}

% Definition of Signature 
The Signature of a path $X: [a,b] \mapsto \mathbb{R}^d$, denoted by $S(X)_{a,b}$, is defined as the infinite series of all iterated integrals of $X$, which can be represented as follows 
\begin{align*}
    S(X)_{a,b} &= (1, S(X)_{a,b}^1, \dots, S(X)_{a,b}^d,  S(X)_{a,b}^{1,1}, \dots ) \\
                &=  \bigoplus_{n=1}^{\infty} S(X)_{a,b}^n
\end{align*}

An alternative definition of signature as the response of an exponential nonlinear system is given in \cite{Terry22}. 

% Log Signature 
Log Signature can be computed by taking the logarithm on the formal power series of Signature. No information is lost as it is possible to recover the (original) Signature from Log Signature by taking the exponential \cite{Chevyrev16,Terry22}. Log Signature provides a more compact representation of the time series than Signature. 
\begin{equation*}
    log S(X)_{a,b} =  \bigoplus_{n=1}^{\infty}  \frac{(-1)^{(n-1)}}{n} S(X)_{a,b}^{\bigotimes n} 
\end{equation*}


%%% Theoretical properties of signatures 
%%% Multiplicative Functional 
%%% Universal Property of Signature in predictions (flexible) 
Signatures can be computed efficiently using the Python package signatory \cite{kidger2021signatory}. The signature is a multiplicative functional in which Chen's identity holds. This allows quick computation of signatures on overlapping slices in a path. Signatures provide a unique representation of a path which is invariant under reparameterisation \cite{Chevyrev16, Terry22}. Rough Path Theory suggests the signature of a path is a good candidate set of linear functionals which captures the aspects of the data necessary for forecasting. In particular, continuous functions of paths are approximately linear on signatures \cite{pmlr-v130-lemercier21a}. This can be considered as a version of universal approximation theorem \cite{cybenko1989approximation} for signature transforms. 


%% Interpretation of Signatures
% Level 1 Signature corresponds to the difference of two series (tail-head). When log price series are given as input, it corresponds to log return 
% Basic Statistical features can be recovered from signatures 

\paragraph{Limitations for signature transforms in high dimensional datasets}
The number of signatures and log-signatures increases exponentially with the number of channels. For time series with a large number of channels, random sampling can be applied to select a small number ($5 < N < 20$) of time-series with replacement from the original time series which signature transforms are applied on. The random sampling can be repeated for a given number of times to generate representative features of the whole multivariate time series. Similar ideas are considered in \cite{James20}, in which random projections on the high dimensional time series is used to reduce dimensionality before applying signature transforms.  

%% Lookback window
Let $\tilde{X}$ be a multivariate time series with $T$ time-steps and $d$ dimensional features, denote $\tilde{X}_s \in \mathbb{R}^d$ be the observation of the time series at timestep $s$. Procedure \ref{alg:lookback} can be used to obtain paths, which are slices of time-series with different lookback windows. Random Signature transforms \ref{alg:randomsig} can then be used to compute the signature of the path, which summarises the information of the time series. 

\begin{algorithm}[hbt!]
\caption{Lookback Window Slicing}\label{alg:lookback}
\KwIn{time series $\tilde{X} \in \mathbb{R}^{T \times d}$, lookback $\delta$}
\KwOut{paths $X_t \in \mathbb{R}^{t \times d}$}
\For{$1 \leq t \leq T$}{
    Set start of slice $s_1 = \max(1, t - \delta)$ \;
    Set end of slice $s_2 = t$ \;
    $X_t = (\tilde{X}_{s_1},\tilde{X}_{s_1+1}, \dots, \tilde{X}_{s_2}) $ \;
}
\end{algorithm}


%% Random Signature Transform Algorithms 
\begin{algorithm}[hbt!]
\caption{Random Signature Transform}\label{alg:randomsig}
\KwIn{path $X_t \in \mathbb{R}^{t \times d}$, level of signature $L$, number of channels $C$, number of feature sets $p$,} 
where $d > C$ \;
\KwOut{log signatures $s_t \in \mathbb{R}^{pN}$ }
Define $N= \text{Number of Log Signatures of a path with } C \text{ channels up to level }  L $ \;
\For{$1 \leq i \leq p$}{
    Sample with replacement $C$ Columns from $X_t$, defined as $\tilde{X}^i_t$ \;
    Compute the Log Signatures $s_t^i \in \mathbb{R}^N$ of $\tilde{X}^i_t$ \;
}
Combine all log signatures $s_t = (s_t^1, \dots, s_t^p)$ 
\end{algorithm}


\subsubsection{Random Fourier Transforms} 
Random Fourier Transforms are used in \cite{kelly2022virtue} to model the return of financial price time series. 

\begin{comment}
A price time series $P_t \in \mathbb{R}^T$ is first transformed into a return series $X_t \in \mathbb{R}^{T \times d} $ by taking the percentage change of price at different lookback intervals. Let $\delta_1, \dots, \delta_d \in \mathbb{N}$ be a given a list of lookback intervals, 
\begin{equation*}
    X_{t,\delta_i} = \frac{P_t - P_{t-\delta_i}}{P_{t-\delta_i}}
\end{equation*}
where $1 \leq t \leq T$ and $1 \leq i \leq d$     
\end{comment}


Random Fourier Transforms are then applied on the feature performance time-series at each time step as in Algorithm \ref{alg:rft}. The key idea is to approximate a mixture model of Gaussian kernels with trigonometric functions \cite{Sutherland15}.  

%% Random Fourier Transform Algorithms 
\begin{algorithm}[hbt!]
\caption{Random Fourier Transform \cite{kelly2022virtue}}\label{alg:rft}
\KwIn{signal vector $x_t \in \mathbb{R}^d$, number of features sets $p$, }
\KwOut{transformed vector $s_t \in \mathbb{R}^{14p}$ }
\For{$1 \leq i \leq p$}{
    Sample $w_i \sim \mathcal{N}(0, I_{d\times d})$ \;
    Set grid $(\gamma_i)_{i=1}^14 = (0.1, 0.5, 1, 2, 4, 8, 16, 0.1, 0.5, 1, 2, 4, 8, 16)$ \;
    \For{$1 \leq j \leq 7$}{
        Set $ s_{t,14i+j} = \frac{1}{\sqrt{7p}} \sin(\gamma_j w_i^T x_t)$
    }
    \For{$8 \leq j \leq 14$}{
        Set $ s_{t,14i+j} = \frac{1}{\sqrt{7p}} \cos(\gamma_j w_i^T x_t)$
    }
}
\end{algorithm}


\subsubsection{Ridge Regression}
Applying the above feature engineering methods, tabular features that have sizes greater than the number of observations is created. This results in an over-parameterised model where regularisation is required as there are multiple models that can perfectly fit the data. In this paper, ridge regression, which is a standard choice of linear regression models with L2-regularisation is used. Ridge regression has closed-form solutions and there is an efficient implementation which can calculate ridge regression models with different L2-regularisation on the same dataset. 


\subsubsection{Selection of Look-back Window} 
For all the feature engineering methods mentioned above, a look-back window needs to be selected. Choosing different sizes of look-back windows corresponds to extracting the dynamics of feature performances at different time scales in the data stream. While in theory the lookback can be set to "infinite", which means using all the available history of the time series at the time of prediction, this is rarely done in practical implementation as it would require an arbitrage large amount of memory to store the data. By using a fixed-size lookback window, the amount of memory required can be limited for the incremental learning prediction model during deployment. 




\section{Machine Learning Models for Tabular Datasets} 
\label{section:tabular-inc}

\subsection{Gradient Boosting Models} 

Gradient Boosting Decision Trees (GBDT) are a standard tool for tabular data modelling. There are many efficient implementations, for example, XGBoost \cite{XGBoost}, LightGBM \cite{LightGBM} and CatBoost \cite{CatBoost}. In this paper, XGBoost is used as it offers good GPU acceleration and flexible feature sub-sampling at the node level. 


\paragraph{Gradient Boosting Algorithm} 

Gradient Boosting is a generic algorithm for combining base learners in a sequential manner to obtain better predictions. A common choice of base learners would be decision trees. The aim of (gradient) boosting is to reduce \textbf{bias} of the ensemble learner. The aim is different to that of bagging, which fits multiple independent base learners at the same time, and the ensemble learner has a lower \textbf{variance} than each base learner. For practical implementations, bagging and boosting can often be used together. For example, in XGBoost, multiple trees (forests) can be fit in each boosting round. However empirical experience (and many textbooks \cite{pml1Book}) suggests boosting works better than bagging. Therefore, only a single tree is trained in each boosting round. Algorithm \ref{alg:gradient-boosting} shows the pseudo-code for the general gradient boosting for a regression problem. 

\begin{algorithm}[htb!]
Given $N$ data samples $(\mathbf{x_i}, y_i), 1 
\leq i \leq N$ with the aim to find an increasing better estimate $\hat{f}(\mathbf{x})$ of the minimising function $f(x)$ which minimise the loss $\mathcal{L}(f)$ between targets and predicted values. $\mathcal{L}(f) = \sum_i l(y_i,f(\mathbf{x_i})) $ where $l$ is a given loss function such as mean square losses for regression problems. Function $f$ is restricted to the class of additive models $f(\mathbf{x}) = \sum_{k=1}^K w_k h(\mathbf{x},	\bm{\alpha_k})$ where $h(\cdot,\bm{\alpha})$ is a weak learner with parameters $\bm{\alpha}$ and $w_k$ are the weights. \\

Initialise  $f_0(\mathbf{x}) = \arg \min_{\bm{\alpha_0}} \sum_{i=1}^N l(y_i, h(\mathbf{x_i}, 	\bm{\alpha_0}))$  \\

\For{k = 1 : K}{
    Compute the gradient residual using $g_{ik} = - \left [ \frac{\partial l(y_i, f_{k-1}(\mathbf{x_i})) }{\partial f_{k-1}(\mathbf{x_i}) } \right ]  $ \\
    Use the weak learner to compute $\bm{\alpha_k}$ which minimises $ \sum_{i=1}^N (g_{ik} - h(\mathbf{x_i},	\bm{\alpha_k}))^2 $  \\
    Update with learning rate $\lambda$ $f_k(\mathbf{x}) = f_{k-1}(\mathbf{x}) + \lambda h(\mathbf{x}, \bm{\alpha_k}) $ \\ 
}
\textbf{Return} $f(\mathbf{x}) = f_K(\mathbf{x})$ \\
\caption{Gradient boosting algorithm \cite{FriedmanJeromeH.2001GfaA, B_hlmann_2007} }
\label{alg:gradient-boosting}
\end{algorithm} 


\paragraph{XGBoost Implementation}

XGBoost \cite{XGBoost} modifies the above "standard" gradient boosting algorithms with approximation algorithms in split finding. Instead of finding the best(exact) split by searching over all possible split points on all the features, a histogram is constructed where splitting is based on percentiles of features. XGBoost supports two different growth policies for the leaf nodes, where nodes closest to the root are split (depth-wise) or the nodes with the highest change of loss function are split (loss-guide). The default tree-growing policy is depth-wise and performs better in most benchmark studies. XGBoost also supports L1 and L2 regularisation of model weights. Other standard model regularisation techniques such as limiting the maximum depth of trees and the minimum number of data samples in a leaf node are also supported. 


\paragraph{Model Snapshots}

For GBDT models, it is easy to extract model snapshots, defined as the model parameters captured at the different parts of the training process. This can be done without any additional memory costs. 

Model snapshots of a GBDT model can be obtained as follows. The snapshots start with the first tree and the number of trees to be used (the ending tree) are set to be $10\%,20\%,\dots,100\%$ of the number of boosting rounds. This trivially gives 10 different GBDT models representing different model complexities from a \textbf{single} model. To avoid early convergence of the learning process, the learning rate can be set to a small value. 


\subsection{Deep Learning Models} 

In this paper, deep learning is defined as any machine learning model that makes use of \textbf{multiple} layers of artificial neural networks in training. 

\paragraph{Feature Neutralisation and Loss Function} 

Pearson correlation calculated on the whole \textbf{era} of target and predictions is used as the loss function at each training epoch. Feature neutralisation, if needed, can be applied from the outputs of network architecture. The neutralised predictions are further standardised to zero mean and unit norm. The negative Pearson correlation of the standardised predictions and targets is then used as the loss function to train the network parameters. 

\paragraph{Training process}

PyTorch Lightning \cite{Falcon_PyTorch_Lightning_2019} is used to build neural network models as it supports modular design and allows rapid prototyping. The learning rate of neural networks is found by the Learning Rate Finder over a parameter grid of $(1e-3,0.1)$. Early Stopping is applied based on the validation set based on a given number of rounds (patience). The batch size of the neural network is set to be the size of each era. The Adam optimiser in PyTorch with the default settings for the learning rate schedule is used. L2-regularisation on the model weights is also applied. Gradient clipping is also be applied to prevent the gradient explosion problem for correlation-based loss functions. 

\paragraph{Architecture}

The network architecture is a sequential neural network with two parts, firstly an "Feature Engineering" part which consists of multiple feature engineering blocks and then the "funnel" part which is a standard MLP with decreasing layer sizes. 

Each feature engineering block has an Auto-Encoder like structure, where the number of features are unchanged after passing each block. Setting a neuron scale ratio less than 1 corresponds to the case of introducing bottleneck to the network architecture so to learn a latent representation of data in a lower dimensional space. Setting a neuron scale ratio greater than 1 corresponds to the case of introducing random combinations of features which are then refined during the model training process. Algorithm \ref{alg:encoding} shows how to create the feature engineering part of the network.

Funnel architecture, as used in \cite{Zimmer_Auto-PyTorch_Tabular_Multi-Fidelity_2021} is an effective way to define the neuron sizes in a network for different input feature sizes. Algorithm \ref{alg:funnel} shows how to create the funnel  part of the network. 

Each Linear layer is followed by a ReLU activation layer and dropout layer where $10\%$ of weights are randomly zeroed. %Batch Normalisation is not used. 

\begin{definition}[Linear Layer] ~\\
    A Linear Layer $X_2 = f(X_1)$ within a sequential neural network is a transformation with input tensor $X_1 \in \mathbb{R}^{N \times M_1}$ and output tensor $X_2 \in \mathbb{R}^{N \times M_2}$ where $N$ is the batch size of data. For a given non-linear activation function $\sigma(\cdot)$ such as ReLU, let $W \in \mathbb{R}^{M_2 \times M_1}$ be the weight tensor and $b \in mathbb{R}^{M_2}$ be the bias tensor to be learnt in the training process, the Linear layer is defined as 
    \begin{equation*}
        f(X_1) = \sigma(X_1 W^T + b)
    \end{equation*}
    % where the addition of bias tensor is performed by broadcast multiplication 
\end{definition}


\begin{algorithm}[hbt!]
\caption{Feature Engineering network architecture}
\label{alg:encoding}
\KwIn{Input feature size $M$, Number of encoding layers $L$, neuron scale ratio $r$}
\KwOut{Sequential Feature Engineering Network Architecture}
\For{$1 \leq l \leq L$}{
    Encoding Layer $l$: Linear layer $(M,M*r)$ \\
    Decoding Layer $l$: Linear layer $(M*r,M)$ 
}
\end{algorithm}

\begin{algorithm}[hbt!]
\caption{Funnel network architecture}
\label{alg:funnel}
\KwIn{Input feature size $M$, Output feature size $K$, Number of intermediate layers $L$, neuron scale ratio $r$}
\KwOut{Sequential Funnel Network Architecture}
Input Layer: Linear layer (M, $M*r$) \\
\For{$1 \leq l \leq L$}{
    Intermediate Layer $l$: Linear layer $(M*r^{l}, M*r^{l+1})$ 
}
Output Layer: Linear layer $(M*r^{L+1},K)$ \\
\end{algorithm}



\section{Deep incremental learning model}

A deep incremental learning model can be built using the tabular and factor-timing models as components. Algorithms \ref{alg:incremental-ml} and  \ref{alg:incremental-stack} outline the overall structure of the model for two different modes, gradient boosting and model stacking. Each layer in the deep incremental learning model is trained sequentially. At the start of training in a layer, features and targets that are shared by each component model are prepared. Targets can be adjusted by gradient boosting formula if needed.

Each factor-timing and tabular model within the layer is trained in an incremental manner as described in Section \ref{section:overview}, in which model parameters are updated as new data arrives. Each component model within a layer can be trained in parallel. 

\begin{algorithm}[hbt!]
\caption{Deep Incremental learning model with gradient boosting}
\label{alg:incremental-ml}

\KwIn{Temporal Tabular Dataset $\{ X_i, y_i \}_{1 \leq i \leq T}$, number of layers $L$, the number of models within each layer $(K_1, \dots, K_L)$, training size $a$, data embargo $b$, learning rate $\eta$}

%Build the feature importance time series using the given features and targets $\{ X_i, y_i \}$ for each era \\
\For{$1 \leq l \leq L$}{
    Calculate the start of model predictions for layer $l$ as $1+l(a+b)$ \\ 
    Prepare Features $\{ X_j^l \}$ where $ (l-1)(a+b) \le j \leq (l-1)(a+b) + a $ where $X_j^l$ consists of the given temporal tabular datasets $X_j$.\\
    Calculate the average predictions $\hat{y}_i$ from the previous layer (if there is one), otherwise set $\hat{y}_i=0$ \\
    Update target with gradient boosting formula if needed: $y_i = y_i - \eta \hat{y}_i$ \\ 
    \For{$1 \leq k \leq K_L$}{
    Perform Data and Feature Sub-sampling for each component model \\
    Train component model using features $\{ X_j^l, y_j \}$ \\
    Obtain predictions of the model from era $1+l(a+b)$ onwards \\
    }
    %Build the feature importance time series for the next layer using predictions from the models trained in the current layer starting at era $1+l(a+b)$ 
}
\end{algorithm}


\begin{algorithm}[hbt!]
\caption{Deep Incremental learning model with model stacking}
\label{alg:incremental-stack}

\KwIn{Temporal Tabular Dataset $\{ X_i, y_i \}_{1 \leq i \leq T}$, number of layers $L$, the number of models within each layer $(K_1, \dots, K_L)$, training size $a$, data embargo $b$, learning rate $\eta$}

%Build the feature importance time series using the given features and targets $\{ X_i, y_i \}$ for each era \\
\For{$1 \leq l \leq L$}{
    Calculate the start of model predictions for layer $l$ as $1+l(a+b)$ \\ 
    Prepare Features $\{ X_j^l \}$ where $ (l-1)(a+b) \le j \leq (l-1)(a+b) + a $ where $X_j^l$ consists of predictions from previous layers. \\
    \For{$1 \leq k \leq K_L$}{
    Perform Data and Feature Sub-sampling for each component model \\
    Train component model using features $\{ X_j^l, y_j \}$ \\
    Obtain predictions of the model from era $1+l(a+b)$ onwards \\
    }
    %Build the feature importance time series for the next layer using predictions from the models trained in the current layer starting at era $1+l(a+b)$ 
}
\end{algorithm}




\paragraph{Self-Similarity nature of model}

The incremental learning model demonstrates self-similarity at various \textbf{scales}. In particular, the overall model shared a similar structure with each of its components. 

The incremental learning model can be interpreted as a gradient-boosted model with $\sum_{i=1}^L K_i$ base learners trained with $L$ boosting rounds, where $K_i$ base learners are trained in the $i$-th boosting round. Ideas from bagging and boosting are integrated within the model. Each layer consists of multiple models trained in parallel as in bagging so that variance is reduced by combining predictions from different models within a layer. Boosting is achieved by adjusting the target based on predictions from previous layers for training in the following layers if needed. Therefore, if each component model is set to be a gradient-boosted model, the overall model would then apply gradient boosting at \textbf{both} the model and component levels.

The incremental learning model can also be interpreted as a neural network model where each node is now a machine learning model instead of a parameter. As predictions from previous layers can be used as features for the following layers, this simulates the residual connections in some neural network architecture. Each layer in the model will refine the prediction as neural networks. Therefore, if each building is set to be a neural network, the overall model would then be a neural network of neural networks. 

The self-similarity structure can be extended repeatedly by interpreting the incremental learning model as a base learner for another incremental learning model. 


\paragraph{Trees and Neural Networks are alike}

Gradient Boosting Decision Trees (GBDT) and neural networks are usually considered as two distinct classes of machine learning models. However recent researches such as soft decision trees \cite{NODE} and transformers \cite{Transformer17} suggests it is possible to make neural network models more tree-like. Similarly, replacing the base learner in GBDT models with weakly trained shallow nets can make GBDT models more neural-like. 

A better way to understand machine learning models is to place each model in a \textbf{continuous} spectrum of model density, from sparse to dense representations/structures. Trees are sparse models. Neural networks are dense models. The choice of model depends on the nature of the input features. In general, sparse models are good for unstructured data or categorical features. Dense models are good for structured data (For example images, text) or numerical features. The binned ordinal features for the temporal ranking task are somewhere in between, making both trees and networks good choices of models. 

The incremental learning model can combine any machine learning models within each layer, and no assumptions are made on the nature of the model. Each layer will be non-binary in nature. As discussed above, the incremental learning model itself also shares properties of gradient-boosting models and neural networks. 


\paragraph{Adaptive nature of model} 

The incremental learning model supports \textbf{dynamic} model training, as parameters of each component model are updated regularly to adapt to distributional shifts in data. Under the traditional machine learning framework, hyper-parameters of machine learning models are selected by cross-validations. A big limitation of using cross-validations for incremental learning problems is that the optimal hyper-parameters based on a \textbf{single} test period might not work in future. This is replaced by \textbf{dynamic} hyper-parameter optimisation in the incremental learning model. Predictions from previous layers based on different model hyper-parameters are combined in the next layer. The model parameters in the next layer can then be interpreted as the dynamic soft selection of hyper-parameters. Soft hyper-parameter selection is also related to Bayesian methods of learning the regularisation hyper-parameter of regression models. Instead of attempting to derive the posterior distributions of the model hyper-parameters, which is difficult when there are no closed-form solutions, the weight parameters can be considered as something equivalent to the posterior distribution. 

Model stacking over deep learning models with different random seeds \cite{lakshminarayanan2017simple}, different hyper-parameters \cite{Florian20} and \cite{zaidi2021neural} are shown to demonstrate robust performances for \textbf{static} datasets. The deep incremental learning model presented here can be considered as an extension of these techniques to \textbf{stream} datasets and other machine learning models in general. 


%Residual connections between layers allow passing information across time (without look-ahead bias). This further allow the incremental learning model to iteratively improve its predictions. 



\paragraph{Universal Approximation} 

It is well-known that Multi-layer perceptron (MLP) and gradient boosting decision trees (GBDT) models have the universal function approximation property \cite{cybenko1989approximation} for \textbf{fixed} tabular datasets. Deep Learning models for sequences, such as LSTM \cite{schafer2006recurrent} are also shown to have the universal function approximation property for any dynamical systems. The above incremental learning model as a composition of individual models that each has universal function approximation property also has the universal approximation property for the underlying stochastic processes that drive the infinite data generation of the temporal tabular datasets. 

The universal approximation provides a theoretical guarantee that if there is an infinite amount of computational resources then the above model can be used for any modelling tasks of temporal tabular datasets. In reality, a wide range of heuristics is applied to simplify the model design with our finite amount of computational resources so that the approximated model can be as close to the theoretical optimal as possible. 


\subsection{Connections with common machine learning tasks}

The incremental learning model is a complete end-to-end autoML tool which transforms the given features into predictions. Different machine learning tasks, such as feature engineering and model stacking are integrated within the model as follows. A key feature is that tabular models such as GBDT or MLP can be used repeatedly in different layers of the model to achieve the purpose of different machine learning tasks, which were previously considered separate tasks. In particular, even with just two building blocks, GBDT and MLP models, very complicated and effective incremental learning models can be built. 


\paragraph{Feature Engineering/Selection} 

Feature Engineering is an important part of modelling tabular data. A wide range of standard feature engineering methods, including random non-linear transforms \cite{Horn19} and data-based methods such as feature tools \cite{James15} are proposed to learn higher-order feature relationships. Standard tabular models are then trained on the original features with the newly created features. More recent methods would use deep learning methods, in particular sequential attention for feature selection and engineering, examples include TabNet\cite{Arik_Pfister_2021} and TabTransformer \cite{TabTransformer}. For these models, feature engineering and tabular data modelling are performed in a single model. A major limitation of these models is high (active) memory consumption during model training. Another limitation is that many feature engineering methods are developed for \textbf{static} datasets whereas a set of relationships learn on \textbf{fixed} training set. Unless these feature engineering components are retrained regularly as the downstream tabular models, the learnt relationships might not hold over distribution shifts of data. 

Most feature engineering methods applied to tabular data are a transformation from tabular features to another set of tabular features and are not much different than a collection of weakly trained tabular models. Under the incremental learning framework, predictions from shallow trees and networks in the early layers can be considered as new features generated for downstream layers. It is not strictly necessary to make a distinction between feature engineering and tabular modelling. Training models in a multi-step layered structure avoids the look-ahead bias issue naturally. 

Feature Selection can be considered a special case of feature engineering where the feature engineering transformation is a Boolean mask.


\paragraph{Feature Neutralisation} 

Feature Neutralisation and its dynamic variant (Dynamic Feature Neutralisation) introduced in the previous Numerai study \cite{wong2023dynamic} can also be reformulated under the incremental learning framework as follows. 

The usefulness of feature neutralisation is based on the assumption that unconstrained model training methods will result in model predictions that can be explained by a small subset of features, which is defined as feature concentration risks. While these models work well under data regimes that are similar to the training set, these models will suffer when there are distributional shifts in data. 

Feature neutralisation, which can be used to limit both the maximal feature concentration risks and (linear)-dependencies of the model on the selected features can mitigate the impact of data distributional shifts by both spreading the model risk across a wider range of features and focusing more on the non-linear relationships between features. Dynamic feature neutralisation suggests for data streams with distributional shifts, the optimal subset of features for neutralisation is \textbf{dynamic} rather than \textbf{fixed}.  

A key issue that remained unresolved in the previous study \cite{ThomasW23} is that the degree of feature neutralisation and the size of the subset of features to be neutralised is given by useful heuristics. Clearly, these parameters should be learned from the dataset and adjusted according to data distribution shifts. Under the incremental learning model, model predictions undergoing different feature neutralisation processes can be combined as a soft selection of model hyper-parameters. 

The optimal degree of feature neutralisation depends on both the feature correlation structure and the target construction process. 



\paragraph{Advanced Architectures neural networks} 

There is no consensus on whether the No Free Lunch Theorem \cite{Wolpert1997} is relevant in machine learning research. In particular, for \textbf{static} machine learning problems, many advanced architectures of neural networks did outperform basic multi-layer perceptron networks \cite{Arik_Pfister_2021,NODE,TabTransformer,}. This observation is also validated in many bench-marking studies on neural network architectures. But at the same time, there are researchers that suggest the contrary when running the models using different hyper-parameters or on \textbf{different} datasets \cite{Arlind21,Leo22,Shwartz21}. As a result, the decision of whether to use more advanced architectures in place of a standard MLP remains an open problem. 

% No man ever steps in the same river twice, for it's not the same river and he's not the same man. 
For the temporal tabular ranking problem, the only purpose is to generate high-quality and robust forecasts for \textbf{future} using models trained on \textbf{past} data only. A good performance based on \textbf{historical} data is not relevant if the performances cannot carry on into the future. An inherent limitation in research design for incremental learning tasks is that all the results reported are simply a \textbf{snapshot} of the infinite model evaluation process. When there are significant distribution shifts in data, conclusions from the previous researchers can become invalid. 

%% Interpretability 
Interpretability is also an important consideration when choosing a machine learning method. Deep learning methods with advanced architectures had an inherent black-box problem as it is difficult to interpret the weights of neural networks. In the incremental learning model, the output of each component can be evaluated independently and compared to inspect the changes in prediction quality in different layers. This provides a transparent view of the usefulness of each component and layer. 


\paragraph{Model Stacking/Selection}

Stacking is a simple but highly effective technique to combine different machine learning model predictions. A closely related problem in finance would be portfolio optimisation, where a convex optimisation is solved at each time step to find the linear combination of assets or strategies that maximise risk-adjusted return.

Under the incremental learning framework, model stacking can be performed at \textbf{prediction} level and not just at \textbf{model} level, which means the predicted rankings from each model are combined and not just assign a global weight for each model at each era. Instead of considering model stacking as a \textbf{separate} step to model training, model stacking can be considered as an extra layer in the incremental learning model as in soft hyper-parameter selection. 




\section{Applying the incremental learning model to Numerai competition}

\subsection{Numerai Dataset} 
\label{section:numerai-sunshine-dataset} 


\paragraph{Features and Targets}
The v4.1 of the Numerai dataset \cite{numerai-datav4.1}, which consists of a total of 1586 features is used for analysis in this paper. The dataset is in the format of a Temporal Tabular dataset. The dataset contains multiple targets, which represent normalised stock returns by different statistical methods. 'target-cyrus-v4-20' is used for scoring the trained models. 

To reduce computational time for grid searches, two different methods are used for feature selection. The first one is by random sampling and the second one is by taking the median over correlated features. For the Numerai dataset, a random sample of features can be drawn from the 1586 features. This is a generic method that can be applied to any tabular dataset. 

A different method is applied to reduce the dimensionality of data when calculating the feature importance's time series. This method makes use of the observed feature correlation structure in the Numerai dataset. The features that demonstrate high multicollinearity are grouped and replaced with their Median (called` Group X Median'). A reason to use Median instead of Mean is to preserve the binned data structures (as each feature is binned values between $-2$ to $2$ after making the mean zero). 

The features from Numerai can be separated into two different sets. One of the set consists of 1445 features which can be grouped in 289 groups by the multicollinearity criteria as these features have a high correlation (>0.9) consistently in different eras. Each group consists of an equal number of 5 features. These features demonstrate a consistent relationship over different data eras. This set of 1445 features is called \textbf{Set A}. The other set consists of 141 features that cannot be grouped by the multicollinearity criteria. This set of 141 features is called  \textbf{Set B}. For features in \textbf{Set A}, the median of each group is computed. This group of 289 Median features and the \textbf{Set B} features, in total 430 features are then used to calculate the \textbf{feature performances} time-series, which are then used to build a factor timing portfolio described above. The feature performance time series is calculated based on the Pearson correlation of each feature with the target in each era. 


\paragraph{Data Lag}
Setting the data lag for predictions would depend on the robustness of the data pipeline from Numerai. The theoretical minimum for the scoring target to resolve is 5 weeks (4 weeks of market data and 1 week for data processing) but some participants has pointed out it can take up to 6 weeks to resolve the target. To take account into both the data lag for the data generation process from Numerai and the time to train models, a conservative data lag of 15 weeks is used for running the models below. 


\paragraph{Scoring Function} 

Numerai calculates a variant of Pearson correlation score with the following formula \cite{numerai-corr}.

Let $y_p$ be the predictions ranked between 0 and 1, $y_t$ be the targets centred between -0.5 and 0.5, $\Phi(\cdot)$ be the Cumulative Distribution function of standard Gaussian, $\textbf{sgn}(\cdot)$ and $\textbf{abs}(\cdot)$ be the element-wise sign and absolute value function, then Numerai Corr $\rho_n$ is given by 
\begin{align*}
    y_g &= \Phi^{-1}(y_p) \\
    y_{g15} &= \textbf{sgn}(y_g) \cdot \textbf{abs}(y_g)^{1.5} \\
    y_{t15} &= \textbf{sgn}(y_t) \cdot \textbf{abs}(y_t)^{1.5} \\
    \rho_n &= \textbf{Corr}(y_{g15},y_{t15})
\end{align*}

Corr is the Pearson correlation function. The purpose of applying a power transformation is to emphasise the contribution from the highest and lowest predictions. 




\section{Training incremental learning models with Numerai data}

A two-step process is used to train incremental learning models with Numerai dataset. The first step is to use data before Era 800 for the hyper-parameter optimisation of tabular and factor-timing models. After that, the incremental learning model is run to get online predictions from Era 800 onwards. 


\subsection{Hyper-parameter optimisation for tabular models} 

For different tabular models introduced in section \ref{section:tabular-inc}, hyper-parameter optimisation is performed using data before Era 800. The training and validation set is data between Era 1 and Era 600, with  $25\%$ of data as the validation set. Due to memory constraints, data sub-sampling is applied during model training. $25\%$ of the eras in the training period is used with sampling performed at regular intervals. The performance of the models between Era 601 and Era 800 (evaluation period) is then used to select hyper-parameters for the tabular models. The Mean Corr and Sharpe Ratio of the prediction ranking correlation in the evaluation period is reported. Details of computing these performance metrics can be found in \cite{ThomasW23}. Due to memory issues for training neural network models, a global feature selection process is used to select $50\%$ of the 1586 features at the start of each model process by random. 


\paragraph{Factor Timing Models}

Different factor timing models are trained as described in Section \ref{section:ML-TS}. 

Signature transforms and random Fourier transforms models with different model complexities, defined as the ratio of the number of time-series features computed to the number of data points in the time-series are trained. In order to study models from different complexity regimes, the model complexity is varied between 0.1 and 10 to obtain models from under-parameterised regime to over-parameterised regime. 

Table \ref{table:stats} reported the model performances in the evaluation period for factor timing models based on exponential moving averages. Tables \ref{table:signatures}, \ref{table:fourier} reported the model performances in the evaluation period for signatures transforms, random Fourier transforms models, averaged over different ridge regularisation parameters (0.01,0.1,1.0,10.0,100.0) and 5 different random seeds. The lookback period of calculating the signatures and random Fourier transforms is set to using all the available data %or the most recent 400,200,100 eras. 

Exponential Moving Averages models with weight decay between 0.0025 and 0.002 performed better than signatures and random Fourier transforms models in general. Weight decay of 0.005 is the best weight decay parameter between era 601 and 800. For signature transforms models, higher complexity gives slightly better performances but not significant. For random Fourier transforms models, higher complexity gives poorer performances when model complexity ratio is higher than 5. 

\begin{table}[!ht]
    \centering
    \begin{tabular}{|l|l|l|l|}
    \hline
        Weight Decay & Mean Corr & Sharpe & Calmar \\ \hline
        0.000625 & 0.0047  & 0.3181 & 0.0273 \\ \hline
        0.00125 & 0.0057  & 0.3590 & 0.0378 \\ \hline
        0.0025 & 0.0082  & 0.4791 & 0.0598 \\ \hline
        0.005 & 0.0091  & 0.5488 & 0.0850 \\ \hline
        0.01  & 0.0088  & 0.5396 & 0.0757 \\ \hline
        0.02  & 0.0085  & 0.5161 & 0.0741 \\ \hline
        0.04   & 0.0068  & 0.4157 & 0.0360 \\ \hline
        0.08   & 0.0058  & 0.3614 & 0.0260 \\ \hline
        0.16   & 0.0046  & 0.2930 & 0.0203 \\ \hline
    \end{tabular}
    \caption{Exponential Moving Averages models}
    \label{table:stats}
\end{table}

\begin{table}[!ht]
    \centering
    \begin{tabular}{|l|l|l|l|l|}
    \hline
        Lookback & Model Complexity & Mean Corr & Sharpe & Calmar\\ \hline
        All & 0.1 & $0.0074 \pm 0.0008$   & $0.4392 \pm 0.0503$  & $0.0582 \pm 0.0071$ \\ \hline
        All & 0.25 & $0.0076 \pm 0.0009$   & $0.4517 \pm 0.0571$  & $0.0605 \pm 0.0102$ \\ \hline
        All & 0.5 & $0.0077 \pm 0.0009$   & $0.4637 \pm 0.0630$  & $0.0635 \pm 0.0129$ \\ \hline
        All & 0.75 & $0.0079 \pm 0.0010$   & $0.4721 \pm 0.0653$  & $0.0658 \pm 0.0140$ \\ \hline
        All & 1 & $0.0080 \pm 0.0010$   & $0.4800 \pm 0.0702$  & $0.0688 \pm 0.0171$ \\ \hline
        All & 2.5 & $0.0082 \pm 0.0010$   & $0.4923 \pm 0.0671$  & $0.0685 \pm 0.0123$ \\ \hline
        All & 5 & $0.0083 \pm 0.0009$   & $0.5005 \pm 0.0660$  & $0.0701 \pm 0.0142$ \\ \hline
        All & 7.5 & $0.0083 \pm 0.0008$   & $0.5038 \pm 0.0628$  & $0.0720 \pm 0.0162$ \\ \hline
        All & 10 & $0.0084 \pm 0.0008$   & $0.5056 \pm 0.0607$  & $0.0727 \pm 0.0184$ \\ \hline
    \end{tabular}
    \caption{Signature transform models}
    \label{table:signatures}
\end{table}
	
\begin{table}[!ht]
    \centering
    \begin{tabular}{|l|l|l|l|l|}
    \hline
        Lookback & Model Complexity & Mean Corr & Sharpe & Calmar\\ \hline
        All & 0.1 & $0.0077 \pm 0.0008$   & $0.4684 \pm 0.0513$  & $0.0585 \pm 0.0116$ \\ \hline
        All & 0.25 & $0.0082 \pm 0.0008$   & $0.4954 \pm 0.0496$  & $0.0632 \pm 0.0120$ \\ \hline
        All & 0.5 & $0.0075 \pm 0.0008$   & $0.4551 \pm 0.0441$  & $0.0570 \pm 0.0117$ \\ \hline
        All & 0.75 & $0.0075 \pm 0.0010$   & $0.4499 \pm 0.0540$  & $0.0525 \pm 0.0063$ \\ \hline
        All & 1 & $0.0067 \pm 0.0005$   & $0.3975 \pm 0.0298$  & $0.0502 \pm 0.0075$ \\ \hline
        All & 2.5 & $0.0065 \pm 0.0011$   & $0.3900 \pm 0.0651$  & $0.0530 \pm 0.0107$ \\ \hline
        All & 5 & $0.0029 \pm 0.0025$   & $0.1443 \pm 0.1316$  & $0.0207 \pm 0.0194$ \\ \hline
        All & 7.5 & $0.0019 \pm 0.0020$   & $0.1284 \pm 0.1304$  & $0.0174 \pm 0.0182$ \\ \hline
        All & 10 & $0.0024 \pm 0.0021$   & $0.1523 \pm 0.1309$  & $0.0182 \pm 0.0191$ \\ \hline
    \end{tabular}
    \caption{Fourier transform models between Eras 601 and 800}
    \label{table:fourier}
\end{table}


Repeating the above analysis between Eras 601 and 1050, tables \ref{table:statsall}, \ref{table:signaturesall} and \ref{table:fourierall} show performances of statistical rule based models, signature and random Fourier transforms models. Exponential Moving Averages models with weight decay 0.005 has the highest Sharpe ratio and the second highest Mean Corr. However, models with weight decay 0.01 and 0.02 have a higher Calmar ratio. Signature transforms models have similar performances over different model complexities. Signature transforms models performed better than random Fourier transforms models over different model complexities. Exponential Moving Average models with suitable weight decay parameters performed the best. 

Comparing the partial results between Eras 601 and 800 and the "full" results between Eras 601 and 1050, it demonstrates why selecting model hyper-parameters for incremental learning tasks on non-stationary data streams based on \textbf{finite} snapshots of data might not be robust over longer horizons. Some hyper-parameters might perform better than others over a short amount of time but not over longer observations. For models depend on randomness, calculating model performances over different random seeds and then performing hypothesis tests can identify situations where the improvement of performances is not significant. Hypothesis tests can only be used to check if we can/cannot \textbf{reject} the null hypothesis that performances between different hyper-parameters are equal on average (equal with respect different measures such as mean, distribution depending on what hypothesis tests are used). 




\begin{table}[!ht]
    \centering
    \begin{tabular}{|l|l|l|l|}
    \hline
        Weight Decay & Mean Corr & Sharpe & Calmar \\ \hline
        0.000625 & 0.0044  & 0.2540 & 0.0090 \\ \hline
        0.00125 & 0.0058  & 0.3099 & 0.0111 \\ \hline
        0.0025 & 0.0087  & 0.4220 & 0.0152 \\ \hline
        0.005 & 0.0086  & 0.4992 & 0.0248 \\ \hline
        0.01  & 0.0077  & 0.4977 & 0.0454 \\ \hline
        0.02  & 0.0075  & 0.4759 & 0.0436 \\ \hline
        0.04   & 0.0059  & 0.3594 & 0.0211 \\ \hline
        0.08   & 0.0045  & 0.2494 & 0.0108 \\ \hline
        0.16   & 0.0029  & 0.1598 & 0.0049 \\ \hline
    \end{tabular}
    \caption{Exponential Moving Averages models between Era 601 and 1050}
    \label{table:statsall}
\end{table}

\begin{table}[!ht]
    \centering
    \begin{tabular}{|l|l|l|l|l|}
    \hline
        Lookback & Model Complexity & Mean Corr & Sharpe & Calmar\\ \hline
        All & 0.1 & $0.0079 \pm 0.0006$   & $0.4269 \pm 0.0370$  & $0.0196 \pm 0.0030$ \\ \hline
        All & 0.25 & $0.0079 \pm 0.0005$   & $0.4334 \pm 0.0373$  & $0.0216 \pm 0.0062$ \\ \hline
        All & 0.5 & $0.0080 \pm 0.0005$   & $0.4420 \pm 0.0433$  & $0.0236 \pm 0.0093$ \\ \hline
        All & 0.75 & $0.0080 \pm 0.0005$   & $0.4468 \pm 0.0451$  & $0.0243 \pm 0.0097$ \\ \hline
        All & 1 & $0.0081 \pm 0.0005$   & $0.4516 \pm 0.0468$  & $0.0250 \pm 0.0096$ \\ \hline
        All & 2.5 & $0.0081 \pm 0.0005$   & $0.4557 \pm 0.0428$  & $0.0268 \pm 0.0089$ \\ \hline
        All & 5 & $0.0080 \pm 0.0007$   & $0.4486 \pm 0.0498$  & $0.0267 \pm 0.0095$ \\ \hline
        All & 7.5 & $0.0079 \pm 0.0009$   & $0.4430 \pm 0.0622$  & $0.0265 \pm 0.0121$ \\ \hline
        All & 10 & $0.0078 \pm 0.0011$   & $0.4384 \pm 0.0737$  & $0.0260 \pm 0.0129$ \\ \hline
    \end{tabular}
    \caption{Signature transform models between Era 601 and 1050}
    \label{table:signaturesall}
\end{table}
 	
\begin{table}[!ht]
    \centering
    \begin{tabular}{|l|l|l|l|l|}
    \hline
        Lookback & Model Complexity & Mean Corr & Sharpe & Calmar\\ \hline
        All & 0.1 & $0.0071 \pm 0.0005$   & $0.3834 \pm 0.0235$  & $0.0163 \pm 0.0028$ \\ \hline
        All & 0.25 & $0.0070 \pm 0.0010$   & $0.3737 \pm 0.0452$  & $0.0152 \pm 0.0025$ \\ \hline
        All & 0.5 & $0.0064 \pm 0.0008$   & $0.3525 \pm 0.0364$  & $0.0151 \pm 0.0038$ \\ \hline
        All & 0.75 & $0.0064 \pm 0.0006$   & $0.3416 \pm 0.0270$  & $0.0123 \pm 0.0018$ \\ \hline
        All & 1 & $0.0063 \pm 0.0006$   & $0.3364 \pm 0.0246$  & $0.0136 \pm 0.0026$ \\ \hline
        All & 2.5 & $0.0057 \pm 0.0012$   & $0.3141 \pm 0.0561$  & $0.0146 \pm 0.0031$ \\ \hline
        All & 5 & $0.0019 \pm 0.0018$   & $0.0937 \pm 0.0940$  & $0.0070 \pm 0.0069$ \\ \hline
        All & 7.5 & $0.0016 \pm 0.0010$   & $0.0997 \pm 0.0632$  & $0.0051 \pm 0.0037$ \\ \hline
        All & 10 & $0.0014 \pm 0.0014$   & $0.0893 \pm 0.0870$  & $0.0059 \pm 0.0060$ \\ \hline
    \end{tabular}
    \caption{Fourier transform models between Eras 601 and 1050}
    \label{table:fourierall}
\end{table}


For models hyper-parameters that do not depend on randomness, such as exponential moving averages or other rule-based trend indicators, hypothesis testing cannot be applied directly to select hyper-parameters directly. Instead, soft hyper-parameter selection within an incremental learning framework are used to find an optimal \textbf{dynamic} combination of predictions. 

A deep XGBoost model is created by combining the 9 exponential moving averages (EWMA) models incrementally. For every era, a shallow XGBoost model with the following hyper-parameters is trained using model predictions from the most recent 185 eras (with suitable data lag). Monotonic constraints are imposed so that we do not assign negative weights to the model predictions. 

%% Layer 2 XGBoost 
\begin{itemize}
    \item Grow policy: Depth-wise
    \item Number of boosting rounds: 20
    \item Early Stopping 20
    \item Learning rate: 0.1
    \item Max Depth: 8 
    \item Max Leaves: 128 
    \item Min Samples per node: 10 
    \item Data subsample: 0.75
    \item Feature subsample by tree: 0.75
    \item Feature subsample by level/node: 1
    \item L1 regularisation: 0.001
    \item L2 regularisation: 0.001 
\end{itemize}


In table \ref{table:EWMADeep}, the performance from the deep EWMA model (Deep) is compared with the \textbf{average} EWMA model, formed by taking a simple average of the 9 predictions (Average) in Era 601 to Era 1050. Deep incremental learning can improve different model metrics. 

\begin{table}[!ht]
    \centering
    \caption{Deep EWMA models}
    \begin{tabular}{|l|l|l|l|}
    \hline
        Method  & Mean Corr  & Sharpe  & Calmar  \\ \hline
        Average & 0.0084  & 0.4642 	& 0.0267 \\ \hline
        Deep  & 0.0087  & 0.4792  & 0.0285 \\ \hline
    \end{tabular}
    \label{table:EWMADeep}
\end{table}






\paragraph{XGBoost}

RMSE, the standard loss function for regression problems is used to train the XGBoost models. Early stopping based on \textbf{Pearson} correlation in the validation set is applied to control the model complexity. A grid search is performed to select the data sub-sample and feature sub-sample ratios of the XGBoost models. 

\begin{itemize}
    \item Data subsample: 0.25,0.5,0.75
    \item Feature subsample by tree: 0.25,0.5,0.75
\end{itemize}


Other hyper-parameters of the XGBoost models are fixed as follows. 
%% Sensible Defaults for XGBoost 
\begin{itemize}
    \item Grow policy: Depth-wise
    \item Number of boosting rounds: 5000
    \item Learning rate: 0.01
    \item Early stopping: 250
    \item Max Depth: 8 
    \item Max Leaves: 128 
    \item Min Samples per node: 10 
    \item Feature subsample by level/node: 1
    \item L1 regularisation: 0.001
    \item L2 regularisation: 0.001 
\end{itemize}


%% Comparison of XGBoost performances over different parameters 
Table \ref{table:XGB1} compares performances of XGBoost models by different data subsample ratios and feature subsample ratios \textbf{without} feature neutralisation, mean and standard deviation over 5 different random seeds are reported.

Calmar ratio is the performance metric with the most variance between different random seeds, suggesting selecting models based on Calmar ratio is not robust. Mean Corr is the least varied metric between random seeds and therefore should be used for hyper-parameter selection. 

Models with data sub-sampling of $75\%$ performed better than models with data sub-sampling of $50\%$

For the study of incremental learning in the next section, the data sub-sampling is set to $75\%$ and the feature sub-sampling is set to $75\%$. Selection is based on having the highest Mean Corr. 



\begin{table}[!ht]
    \centering
    \caption{XGBoost models without feature neutralisation}
    \begin{tabular}{|l|l|l|l|l|}
    \hline
        Data subsample  & Feature subsample  & Mean Corr & Sharpe  & Calmar \\ \hline
        0.75  & 0.75  & $0.0262 \pm 0.0007$   & $1.3450 \pm 0.0643$  & $0.6316 \pm 0.1958$ \\ \hline
        0.75  & 0.50  & $0.0259 \pm 0.0006$   & $1.3372 \pm 0.0440$  & $0.6682 \pm 0.1698$ \\ \hline
        0.75  & 0.25  & $0.0257 \pm 0.0004$   & $1.3029 \pm 0.0264$  & $0.5339 \pm 0.1332$ \\ \hline
        0.50  & 0.75  & $0.0240 \pm 0.0007$   & $1.2146 \pm 0.0485$  & $0.3911 \pm 0.0448$ \\ \hline
        0.50  & 0.50  & $0.0245 \pm 0.0013$   & $1.2786 \pm 0.1002$  & $0.5224 \pm 0.2025$ \\ \hline
        0.50  & 0.25  & $0.0242 \pm 0.0014$   & $1.2139 \pm 0.0602$  & $0.3413 \pm 0.0409$ \\ \hline
        0.25  & 0.75  & $0.0211 \pm 0.0011$   & $1.0994 \pm 0.0592$  & $0.2488 \pm 0.0785$ \\ \hline
        0.25  & 0.50  & $0.0213 \pm 0.0012$   & $1.0982 \pm 0.0621$  & $0.2405 \pm 0.0786$ \\ \hline
        0.25  & 0.25  & $0.0211 \pm 0.0012$   & $1.1156 \pm 0.1068$  & $0.3492 \pm 0.1011$ \\ \hline
    \end{tabular}
    \label{table:XGB1}
\end{table}


\paragraph{Neural Networks}

Neural Networks models \textbf{without} feature neutralisation are trained with different number of encoding and funnel layers using the architecture described in Section \ref{section:tabular-inc}. 

\begin{itemize}
    \item Degree of Feature Neutralisation: 0.0
    \item Number of Feature Eng Layers: 0,1,2,3,4
    \item Number of Funnel Layers: 1,2,3 
\end{itemize}

Other hyper-parameters of the neural network models are fixed in the grid search as follows. 
%% Sensible Defaults 
\begin{itemize}
    \item Loss Function: Pearson Corr
    \item Number of epochs: 100 
    \item Early Stopping: 10
    \item Dropout: 0.1
    \item Encoding Neuron Scale: 0.8
    \item Funnel Neuron Scale: 0.8 
    \item Gradient Clip: 0.5 
\end{itemize}


In Table \ref{table:MLP1} shows the performances of neural network models with different network architectures over 5 different random seeds, where mean and standard deviation of Mean Corr, Sharpe and Calmar ratios are reported. 


The architecture with the highest Mean Corr is the model without feature engineering layers and a standard MLP model with 2 linear layers. When the number of funnel layers equals to 1, the MLP model is equivalent to a (regularised) linear model and has the worst performance. Increasing the number of feature engineering layers does not significantly improve Mean Corr. As model complexity increases, model performances are more varied over different random seeds, suggesting the lack of robustness of deep neural network models. 


\begin{table}[!ht]
    \centering
    \caption{Neural Network models without feature neutralisation}
    \begin{tabular}{|l|l|l|l|l|l|}
    \hline
        Feature Eng Layers  & Funnel Layers  & Mean Corr & Sharpe  & Calmar \\ \hline
        0  & 1  & $0.0159 \pm 0.0016$   & $0.8042 \pm 0.0631$  & $0.0826 \pm 0.0087$ \\ \hline
        0  & 2  & $0.0235 \pm 0.0001$   & $1.1344 \pm 0.0119$  & $0.2692 \pm 0.0201$ \\ \hline
        0  & 3  & $0.0223 \pm 0.0005$   & $1.0478 \pm 0.0372$  & $0.2117 \pm 0.0072$ \\ \hline
        1  & 1  & $0.0222 \pm 0.0003$   & $1.0509 \pm 0.0112$  & $0.2118 \pm 0.0068$ \\ \hline
        1  & 2  & $0.0216 \pm 0.0003$   & $1.0061 \pm 0.0377$  & $0.2021 \pm 0.0128$ \\ \hline
        1  & 3  & $0.0224 \pm 0.0003$   & $1.0575 \pm 0.0121$  & $0.2212 \pm 0.0269$ \\ \hline
        2  & 1  & $0.0217 \pm 0.0004$   & $1.0176 \pm 0.0357$  & $0.2104 \pm 0.0178$ \\ \hline
        2  & 2  & $0.0218 \pm 0.0009$   & $1.0346 \pm 0.0571$  & $0.2005 \pm 0.0076$ \\ \hline
        2  & 3  & $0.0226 \pm 0.0006$   & $1.0754 \pm 0.0352$  & $0.2348 \pm 0.0242$ \\ \hline
        3  & 1  & $0.0224 \pm 0.0006$   & $1.0467 \pm 0.0402$  & $0.2226 \pm 0.0281$ \\ \hline
        3  & 2  & $0.0221 \pm 0.0009$   & $1.0564 \pm 0.0441$  & $0.2332 \pm 0.0291$ \\ \hline
        3  & 3  & $0.0217 \pm 0.0007$   & $1.0245 \pm 0.0414$  & $0.2049 \pm 0.0156$ \\ \hline
        4  & 1  & $0.0215 \pm 0.0006$   & $1.0131 \pm 0.0192$  & $0.1980 \pm 0.0146$ \\ \hline
        4  & 2  & $0.0219 \pm 0.0010$   & $1.0490 \pm 0.0673$  & $0.2229 \pm 0.0309$ \\ \hline
        4  & 3  & $0.0218 \pm 0.0017$   & $1.0459 \pm 0.0880$  & $0.2513 \pm 0.0229$ \\ \hline
    \end{tabular}
    \label{table:MLP1}
\end{table}


\paragraph{Model Snapshots of XGBoost models}

Traditional machine learning suggests there exists an optimal model complexity where the trade-off of bias and variance is optimal (for a loss function that behaves like the Mean-Squared error). However, modern machine learning research suggests using an over-parameterised model might improve performance in a test set even when model training loss cannot be further improved. The improvement is significant in cases where the model specification is incomplete and low signal-to-noise ratio in the given features. This counter-intuitive phenomenon is explored in different research papers \cite{Hastie19,NakkiranPreetum2021Dddw,Teresa22} from both theoretical and empirical perspectives. 

The two viewpoints are summarised as follows. 
\begin{itemize}
    \item Viewpoint 1: (Classical Approach): There exists an optimal model complexity which can be found by performing bootstrapping-like procedures, such as cross-validation. 
    \item Viewpoint 2: (Modern Approach): Over-parameterised model will outperform the optimal model under the classical approach. 
\end{itemize}


XGBoost models are trained \textbf{without} early stopping for 5000 rounds and learning rate of 0.01 for 5 different random seeds. Detailed hyper-parameters are listed below. 

\begin{itemize}
    \item Grow policy: Depth-wise
    \item Number of boosting rounds: 5000
    \item Learning rate: 0.01
    \item Max Depth: 8 
    \item Max Leaves: 128 
    \item Min Samples per node: 10 
    \item Data subsample: 0.75
    \item Feature subsample by tree: 0.75
    \item Feature subsample by level/node: 1
    \item L1 regularisation: 0.001
    \item L2 regularisation: 0.001 
\end{itemize}

Model snapshots are created by running the first $10\%,20\%,\dots,100\%$ of boosting rounds during inference. Performances of model snapshots from Era 601 to Era 800 are reported in table \ref{table:XGBSnapshot}. Increasing the number of boosting rounds does not significantly improve Mean Corr. Models with around 1000-1500 boosting rounds have the highest Mean Corr. Sharpe ratios do not change significantly for models with more than 1500 boosting rounds. Models with 4000 boosting rounds have the highest Calmar ratio averaged over the random seeds but are also the most varied. It suggests that over-parameterised models might be able to reduce the Max Drawdown and other downside risks in predictions. However, it cannot be ruled out whether over-parameterised models performed better due to chance. 


\begin{table}[!ht]
    \centering
    \caption{Model Snapshots of XGBoost Models (Max Boosting rounds is 5000)}
    \begin{tabular}{|l|l|l|l|}
    \hline
        Snapshot  & Mean Corr  & Sharpe  & Calmar  \\ \hline
        0.1  & $0.0248 \pm 0.0005$   & $1.2390 \pm 0.0419$  & $0.4566 \pm 0.0799$ \\ \hline
        0.2  & $0.0264 \pm 0.0006$   & $1.3673 \pm 0.0426$  & $0.6460 \pm 0.1041$ \\ \hline
        0.3  & $0.0262 \pm 0.0006$   & $1.4057 \pm 0.0637$  & $0.7951 \pm 0.1211$ \\ \hline
        0.4  & $0.0255 \pm 0.0007$   & $1.4040 \pm 0.0552$  & $0.8435 \pm 0.1663$ \\ \hline
        0.5  & $0.0247 \pm 0.0005$   & $1.3882 \pm 0.0387$  & $0.9256 \pm 0.2884$ \\ \hline
        0.6  & $0.0242 \pm 0.0004$   & $1.3935 \pm 0.0406$  & $1.1110 \pm 0.3865$ \\ \hline
        0.7  & $0.0238 \pm 0.0003$   & $1.4009 \pm 0.0506$  & $1.1242 \pm 0.3917$ \\ \hline
        0.8  & $0.0234 \pm 0.0005$   & $1.3973 \pm 0.0421$  & $1.3520 \pm 0.5284$ \\ \hline
        0.9  & $0.0228 \pm 0.0005$   & $1.3610 \pm 0.0357$  & $1.1738 \pm 0.4327$ \\ \hline
        1.0  & $0.0223 \pm 0.0005$   & $1.3560 \pm 0.0320$  & $1.0753 \pm 0.3592$ \\ \hline
    \end{tabular}
    \label{table:XGBSnapshot}
\end{table}


\paragraph{Conclusion} 

%% Factor Timing models under-performed. 
Factor timing models, which are based on learning the dynamics of weights from the feature correlations on each era under-performed other tabular models such as XGBoost and MLP. In fact, feature engineering models do not perform better than simple rule-based models when evaluated over a long enough timeframe. All the factor-timing models considered above are highly correlated in both performances and predictions. 

The Numerai dataset demonstrates a very strong non-stationary nature such that the linear weights of the regression model trained on each era cannot be effectively used to predict future weights. In other words, \textbf{linear} relationships between features and target is not stable such that forecast cannot be made accurately using historical values. It shows the dataset should be modelled with tabular machine learning models, with regular updates on model weights, rather than using time-series methods to predict the regression parameters for each era. 

%% EWMA models 
Exponential Moving Averages models have a strong and robust performance compared to other factor timing models despite their simplicity. The key hyper-parameter, weight decay can be selected using a deep incremental learning framework which adjust dynamically over time. 

%% XGBoost models performed better than MLP 
XGBoost models performed better than MLP models over a wide range of hyper-parameters.  The binned nature of features favours the use of decision trees over neural networks. 

%% Early Stopping for XGBoost 
A key design choice for XGBoost models is whether to apply early stopping, as it reflects two differnet viewpoints (Classical vs Modern) towards optimal model complexity. For the above analysis, early stopping is useful as Mean Corr drops when the number of boosting rounds increases beyond a threshold around the early stopping criteria. Large XGBoost models can reduce the downside risks of models within the period shown but it is not certain whether the effect is significant.

%% Complex MLP models are not helpful 
Increasing complexity of MLP models cannot improve model performances as expected by the Modern ML viewpoint. The best performance is achieved by a standard MLP with two layers, which provides the minimal amount of non-linearity required so that the model does not degenerate to a ridge regression model. As suggested in \cite{Teresa22}, the performance of over-parameterised models are affected by a myriad of factors including model architecture and training process. It cannot be ruled out that there are other model architectures that can make deep learning models performing better than XGBoost. Only the most basic neural network architectures are considered here due to computational resources constraints. However, as suggested from research on bench-marking of tabular ML models \cite{Leo22}, recent deep learning models for tabular data such as TabNet might not always perform better than XGBoost. If computational costs are also taken into account, then applying deep learning models directly might not be the best choice. 


%% 





\subsection{Incremental Learning model for XGBoost models} 

The XGBoost models with the optimised design choices (based on data up to Era 800) are used in incremental learning models with different training sizes and retrain periods. Train size is how many eras of data are used to create the training and validation set. retrain period represents how often the model parameters are updated with the latest data. A retrain period of 100 means the model is retrained every 100th era. 

%% When Train size is 585 
\paragraph{Adjusting the retrain period} 

The train size of models is fixed to be 585 as above. The retrain period is set to be 400,200,100,50,25, respectively. Table \ref{table:XGBInc585} lists the performances of the XGBoost models trained with the same optimised hyper-parameters between Era 801 and Era 1050 over 5 different random seeds. Retraining models more frequently improves model performances up to a retraining period of 50. 

\begin{table}[!ht]
    \centering
    \caption{XGBoost Models with different retrain periods, train size is fixed to 585}
    \begin{tabular}{|l|l|l|l|}
    \hline
        retrain period  & Mean Corr  & Sharpe  & Calmar  \\ \hline
        25   & $0.0223 \pm 0.0010$   & $1.0854 \pm 0.0419$  & $0.2573 \pm 0.0232$ \\ \hline
        50   & $0.0224 \pm 0.0007$   & $1.1096 \pm 0.0385$  & $0.2975 \pm 0.0305$ \\ \hline
        100  & $0.0206 \pm 0.0004$   & $1.0134 \pm 0.0352$  & $0.2739 \pm 0.0250$ \\ \hline
        200  & $0.0210 \pm 0.0004$   & $1.0166 \pm 0.0295$  & $0.3010 \pm 0.0605$ \\ \hline
        400  & $0.0192 \pm 0.0007$   & $0.9293 \pm 0.0632$  & $0.1445 \pm 0.0382$ \\ \hline
    \end{tabular}
    \label{table:XGBInc585}
\end{table}






\paragraph{When train size is small} 

Tables \ref{table:XGBInc485}, \ref{table:XGBInc385}, \ref{table:XGBInc285}, \ref{table:XGBInc185}, \ref{table:XGBInc85} list the model performances when train size is set to 485,385,285,185,85 with different retrain periods. Model performances drop as train size is decreased. Increasing the number of models retrains have limited effects on model performances. All models with train sizes less than or equal to 385 have smaller Sharpe ratios than models with train sizes 585 or 785. 

\begin{table}[!ht]
    \centering
    \caption{XGBoost Models with different retrain periods, train size is fixed to 485}
    \begin{tabular}{|l|l|l|l|}
    \hline
        retrain period  & Mean Corr  & Sharpe  & Calmar  \\ \hline
        25   & $0.0170 \pm 0.0010$   & $0.8406 \pm 0.0611$  & $0.1230 \pm 0.0279$ \\ \hline
        50   & $0.0188 \pm 0.0006$   & $0.8885 \pm 0.0486$  & $0.1281 \pm 0.0288$ \\ \hline
        100  & $0.0176 \pm 0.0007$   & $0.8394 \pm 0.0233$  & $0.1191 \pm 0.0226$ \\ \hline
    \end{tabular}
    \label{table:XGBInc485}
\end{table}



\begin{table}[!ht]
    \centering
    \caption{XGBoost Models with different retrain periods, train size is fixed to 385}
    \begin{tabular}{|l|l|l|l|}
    \hline
        retrain period  & Mean Corr  & Sharpe  & Calmar  \\ \hline
        25   & $0.0164 \pm 0.0007$   & $0.7451 \pm 0.0360$  & $0.0915 \pm 0.0210$ \\ \hline
        50   & $0.0177 \pm 0.0004$   & $0.8109 \pm 0.0338$  & $0.0999 \pm 0.0182$ \\ \hline
        100  & $0.0160 \pm 0.0006$   & $0.7438 \pm 0.0287$  & $0.0902 \pm 0.0178$ \\ \hline
    \end{tabular}
    \label{table:XGBInc385}
\end{table}




\begin{table}[!ht]
    \centering
    \caption{XGBoost Models with different retrain periods, train size is fixed to 285}
    \begin{tabular}{|l|l|l|l|}
    \hline
        retrain period  & Mean Corr  & Sharpe  & Calmar  \\ \hline
        25   & $0.0125 \pm 0.0014$   & $0.6511 \pm 0.0655$  & $0.0776 \pm 0.0117$ \\ \hline
        50   & $0.0140 \pm 0.0017$   & $0.6733 \pm 0.0776$  & $0.0779 \pm 0.0287$ \\ \hline
        100  & $0.0130 \pm 0.0008$   & $0.6209 \pm 0.0381$  & $0.0716 \pm 0.0237$ \\ \hline
    \end{tabular}
    \label{table:XGBInc285}
\end{table}





\begin{table}[!ht]
    \centering
    \caption{XGBoost Models with different retrain periods, train size is fixed to 185}
    \begin{tabular}{|l|l|l|l|}
    \hline
        retrain period  & Mean Corr  & Sharpe  & Calmar  \\ \hline
        25   & $0.0081 \pm 0.0017$   & $0.4278 \pm 0.0993$  & $0.0306 \pm 0.0165$ \\ \hline
        50   & $0.0073 \pm 0.0015$   & $0.3779 \pm 0.0823$  & $0.0297 \pm 0.0139$ \\ \hline
    \end{tabular}
    \label{table:XGBInc185}
\end{table}



%% Needs update 
\begin{table}[!ht]
    \centering
    \caption{XGBoost Models with different retrain periods, train size is fixed to 85}
    \begin{tabular}{|l|l|l|l|}
    \hline
        retrain period  & Mean Corr  & Sharpe  & Calmar  \\ \hline
        25   & $0.0059 \pm 0.0012$   & $0.3318 \pm 0.0787$  & $0.0294 \pm 0.0143$ \\ \hline
        50   & $0.0051 \pm 0.0015$   & $0.2708 \pm 0.0813$  & $0.0230 \pm 0.0137$ \\ \hline
    \end{tabular}
    \label{table:XGBInc85}
\end{table}





\paragraph{When training size is larger} 

Tables \ref{table:XGBInc685} and  \ref{table:XGBInc785} list the performances of the XGBoost models trained with the same optimised hyper-parameters between Era 801 and Era 1050, with the train size increased to 685 and 785.

Model performances do not always increase with models trained on more data. This is the key difference between \textbf{stationary} and \textbf{non-stationary} datasets. For learning problems where data distribution does not change much between training and test, adding more (noiseless) data in general will improve model performances. The Numerai dataset used here is non-stationary in nature and increasing the training set might hurt model performance by including less relevant history. 

When the train size is 685, decreasing the retrain period can improve Mean Corr and Sharpe ratio but the effect is not significant when retrain period is less than 100.


\begin{table}[!ht]
    \centering
    \caption{XGBoost Models with different retrain periods, train size is fixed to 685}
    \begin{tabular}{|l|l|l|l|}
    \hline
        retrain period  & Mean Corr  & Sharpe  & Calmar  \\ \hline
        25   & $0.0208 \pm 0.0005$   & $1.0250 \pm 0.0324$  & $0.2054 \pm 0.0164$ \\ \hline
        50   & $0.0202 \pm 0.0005$   & $0.9875 \pm 0.0460$  & $0.1906 \pm 0.0280$ \\ \hline
        100  & $0.0197 \pm 0.0009$   & $0.9291 \pm 0.0577$  & $0.1855 \pm 0.0263$ \\ \hline
        200  & $0.0189 \pm 0.0014$   & $0.8907 \pm 0.0869$  & $0.1214 \pm 0.0290$ \\ \hline
        400  & $0.0182 \pm 0.0014$   & $0.9013 \pm 0.1028$  & $0.1188 \pm 0.0340$ \\ \hline
    \end{tabular}
    \label{table:XGB685}
\end{table}



\begin{table}[!ht]
    \centering
    \caption{XGBoost Models with different retrain periods, train size is fixed to 785}
    \begin{tabular}{|l|l|l|l|}
    \hline
        retrain period  & Mean Corr  & Sharpe  & Calmar  \\ \hline
        25   & $0.0207 \pm 0.0005$   & $1.0116 \pm 0.0423$  & $0.2048 \pm 0.0509$ \\ \hline
        50   & $0.0211 \pm 0.0006$   & $1.0463 \pm 0.0187$  & $0.2246 \pm 0.0259$ \\ \hline
        100  & $0.0202 \pm 0.0006$   & $0.9853 \pm 0.0213$  & $0.2281 \pm 0.0278$ \\ \hline
        200  & $0.0216 \pm 0.0004$   & $1.0968 \pm 0.0345$  & $0.2444 \pm 0.0355$ \\ \hline
        400  & $0.0209 \pm 0.0004$   & $1.0561 \pm 0.0285$  & $0.1957 \pm 0.0125$ \\ \hline
    \end{tabular}
    \label{table:XGB785}
\end{table}



\paragraph{Discussion} 

The non-stationary nature of data streams is demonstrated in the lack of relationship between model performances and retrain periods for models with different train sizes. Each XGBoost model has the same hyper-parameters and the training process only differs in the training data used to train XGBoost models regularly. There are no significant differences between model performances once retrain periods are shorter than 100 eras. 

A useful empirical rule to select training size is to ensure the training set is large enough to cover different data regimes that are possible to come up in future. Once the training set includes enough data samples from different regimes, further increase in train size will have limited improvement to model performance. 




\section{Deep Incremental Learning Models}
\label{section:paradigm-ML} 

\paragraph{Deep XGBoost ensembles with different retrain periods}

20 XGBoost models are trained with the following hyperparameters. The train size is fixed to 585 with models retrained every 200,100,50,25 eras, where 5 models with different random seeds are trained for different retrain periods. Other hyper-parameters of the XGBoost models are set as follows. 

\begin{itemize}
    \item Grow policy: Depth-wise
    \item Number of boosting rounds: 5000
    \item Early Stopping 250
    \item Learning rate: 0.01
    \item Max Depth: 8 
    \item Max Leaves: 128 
    \item Min Samples per node: 10 
    \item Data subsample: 0.75
    \item Feature subsample by tree: 0.75
    \item Feature subsample by level/node: 1
    \item L1 regularisation: 0.001
    \item L2 regularisation: 0.001 
\end{itemize}

A deep XGBoost model is created by combining the 20 models incrementally. For every era, a shallow XGBoost model with the following hyper-parameters is trained using model predictions from the most recent 185 eras (with suitable data lag). Monotonic constraints are imposed so that we do not assign negative weights to the model predictions. 

%% Layer 2 XGBoost 
\begin{itemize}
    \item Grow policy: Depth-wise
    \item Number of boosting rounds: 100
    \item Early Stopping 100
    \item Learning rate: 0.1
    \item Max Depth: 8 
    \item Max Leaves: 128 
    \item Min Samples per node: 10 
    \item Data subsample: 0.75
    \item Feature subsample by tree: 0.75
    \item Feature subsample by level/node: 1
    \item L1 regularisation: 0.001
    \item L2 regularisation: 0.001 
\end{itemize}


In table \ref{table:XGBDeep5000}, the performance from the deep XGBoost model (Deep) is compared with the \textbf{average} performances of the 20 XGBoost model (Average) in Era 801 to Era 1050. The performance of each individual XGBoost model (Base) is also reported. 

Both ensemble methods (Deep, Average) can improve Mean Corr and Sharpe ratio. However, machine learning-based ensemble method (Deep) does not perform better than a simple average (Average). The optimisation task in the second layer is to allocate \textbf{dynamic} weights towards XGBoost models with different random seeds and retrain periods. As there is no retrain period which is a clear winner due to the non-stationary nature of data, any optimisation techniques based on data mining of historical predictions is unlikely to provide any edge over the simple approach of taking an equal-weighted average. 


\begin{table}[!ht]
    \centering
    \caption{Deep XGBoost models with different retrain periods (5000 boosting rounds) }
    \begin{tabular}{|l|l|l|l|}
    \hline
        Method  & Mean Corr  & Sharpe  & Calmar  \\ \hline
        Base  & $0.0211 \pm 0.0010$   & $1.0374 \pm 0.0544$  & $0.2399 \pm 0.0536$ \\ \hline
        Average & 0.0241  & 1.1455 & 0.2624 \\ \hline
        Deep  & 0.0239 & 1.1379 & 0.2713 \\ \hline
    \end{tabular}
    \label{table:XGBDeep5000}
\end{table}


Repeating the above analysis with the number of boosting rounds changed from 5000 (standard) to 500(small)/50000(large), tables \ref{table:XGBDeep500} and \ref{table:XGBDeep50000} compare the performance of the deep XGBoost models with the average models. 

\begin{itemize}
    \item Number of boosting rounds: 500
    \item Early Stopping 25
    \item Learning rate: 0.1
\end{itemize}

\begin{itemize}
    \item Number of boosting rounds: 50000
    \item Early Stopping 2500
    \item Learning rate: 0.001
\end{itemize}


For both small XGBoost models (500 boosting rounds) and large XGBoost models (50000 boosting rounds), machine learning-based ensemble method (Deep) does not perform better than a simple average. 

For small XGBoost models (500 boosting rounds), model ensemble significantly improves Mean Corr with some trade-offs in Sharpe and Calmar ratios. Model performances are poorer than models trained with 5000 boosting rounds as expected.  Even if the base learners (small XGBoost models with 500 trees) are very weak predictors on their own (and are only $10\%$ size of the original trees), a simple ensemble of models trained with different random seeds and retrain periods can recover $90\%$ performance of the average XGBoost model with 5000 trees. 

For large XGBoost models (50000 boosting rounds), the gap between machine learning-based method (Deep) and simple average (Average) is even larger than the case for standard XGBoost models (5000 boosting rounds) and small XGBoost models. It suggests the better the quality of base learners, the lesser the need to apply further optimisation on those. Attempting to learn the differences between large XGBoost models actually resulted in over-fitted dynamic model weights, which then under-perform the baseline choice of ensemble, simple averaging. 


\begin{table}[!ht]
    \centering
    \caption{Deep XGBoost models with different retrain periods (500 boosting rounds) }
    \begin{tabular}{|l|l|l|l|}
    \hline
        Method  & Mean Corr  & Sharpe  & Calmar  \\ \hline
        Base  & $0.0122 \pm 0.0005$   & $0.6987 \pm 0.0959$  & $0.1161 \pm 0.0455$ \\ \hline
        Average & 0.0213  & 1.0349 & 0.2109 \\ \hline
        Deep  & 0.0209   & 1.0623 &  0.1933  \\ \hline
    \end{tabular}
    \label{table:XGBDeep500}
\end{table}

\begin{table}[!ht]
    \centering
    \caption{Deep XGBoost models with different retrain periods (50000 boosting rounds) }
    \begin{tabular}{|l|l|l|l|}
    \hline
        Method  & Mean Corr  & Sharpe  & Calmar  \\ \hline
        Base  & $0.0231 \pm 0.0009$   & $1.1354 \pm 0.0696$  & $0.3506 \pm 0.1182$ \\ \hline
        Average & 0.0245  & 1.1811 & 0.4115 \\ \hline
        Deep  & 0.0235  & 1.1628  & 0.3343 \\ \hline
    \end{tabular}
    \label{table:XGBDeep50000}
\end{table}



\paragraph{Deep XGBoost ensembles with different model complexity} 

XGBoost models with 5000,50000 boosting rounds are trained for 5 different random seeds. For each XGBoost model, 10 model snapshots are collected. The 50 predictions are then combined in the second layer of the deep incremental model. For every era, a shallow XGBoost model with the following hyper-parameters is trained using model predictions from the most recent 185 eras (with suitable data lag). Monotonic constraints are imposed so that we do not assign negative weights to the model predictions. The hyper-parameters of the XGBoost models in the first and second layer are as above. 

In tables \ref{table:XGBComplexity5000} and \ref{table:XGBComplexity50000}, the performance from the deep XGBoost model (Deep) is compared with the \textbf{average} performances of the 50 XGBoost model snapshots (Average) in Era 801 to Era 1050. The performance of each individual XGBoost snapshot (Base) is also reported. 

Both ensemble methods (Deep, Average) can improve Mean Corr and Sharpe ratio. However, machine learning-based method (Deep) does not perform better than a simple average (Average). For a non-stationary data stream, it is not surprising none of the retrain periods demonstrated significant performance better than others, if there is no periodicity in the data stream. The example given here shows even model complexity, an inherent property of machine learning model rather than data dependent hyper-parameters, cannot be optimised through incremental learning. In other words, both viewpoints favouring either regularised or over-parameterised models are wrong for this dataset. There is no range of model complexity that is consistently better than others. 


\begin{table}[!ht]
    \centering
    \caption{Deep XGBoost models with different model complexity (5000 boosting rounds)}
    \begin{tabular}{|l|l|l|l|}
    \hline
        Method  & Mean Corr  & Sharpe  & Calmar  \\ \hline
        Base  & $0.0215 \pm 0.0010$   & $1.1354 \pm 0.0696$  & $0.3506 \pm 0.1182$ \\ \hline
        Average & 0.0239  & 1.2233 & 0.3853 \\ \hline
        Deep  & 0.0235 & 1.1628 & 0.3343 \\ \hline
    \end{tabular}
    \label{table:XGBComplexity5000}
\end{table}


\begin{table}[!ht]
    \centering
    \caption{Deep XGBoost models with different model complexity (50000 boosting rounds)}
    \begin{tabular}{|l|l|l|l|}
    \hline
        Method  & Mean Corr  & Sharpe  & Calmar  \\ \hline
        Base  & $0.0231 \pm 0.0009$   & $1.1896 \pm 0.0614$  & $0.4256 \pm 0.1435$ \\ \hline
        Average & 0.0240  & 1.2197 & 0.4306 \\ \hline
        Deep  & 0.0239 & 1.1675 & 0.3395 \\ \hline
    \end{tabular}
    \label{table:XGBComplexity50000}
\end{table}




\paragraph{Discussion} 

Model ensemble is an effective way to improve robustness of model predictions, in particular when the base learners are not strong predictors on their own, such as the case with small XGBoost models (500 boosting rounds). As the quality of base learners gets better, such as the case with large XGBoost models (50000 boosting rounds), the improvement becomes less significant. A possible reason is because the base learner considered here, GBDT models already made use of gradient-boosting to refine its predictions, therefore applying the same technique again will give limited gains in performances. 

The above analysis on XGBoost models with different retrain periods and model complexity demonstrated why optimisation under non-stationarity and limited data is not always desirable. Comparing the two model ensemble methods, simple average and deep incremental learning, a simple average on XGBoost models from the first layer can improve performances at zero additional training cost. Adding extra machine learning layers, such as non-negative GBDT models cannot improve model performances. 

While the above analysis demonstrated that deep incremental learning is not necessary for the incremental learning pipeline with a data split of 585/185 eras with data embargo of 15, it does not rule out the possibility that deep incremental learning pipeline can outperform simple averaging when there are more data for training. For example, under a data split of 585/585 eras with data embargo 15, where both layers can have enough history to learn the characteristics over different regimes, might perform better than models combined by simple averaging. The availability of data and diversity of base learners (predictions from previous layers) are the key drivers on whether deep incremental learning can improve performances. 


%% How to compare GBDT models and deep learning models 
In most bench-marking studies of tabular machine learning models, performances of individual models trained with different random seeds, for different model configurations are reported, and the ensemble performances is not reported. As model complexity is not standardised across different machine learning methods under comparison, it is unfair to compare the performances of an over-parameterised model, typically deep-learning based with standard XGBoost models. A single over-parameterised deep learning model might outperform a single XGBoost model, but it is unlikely that the same deep learning model can outperform an ensemble of XGBoost models over different complexities and incremental learning settings.

The typical workflow of running hyper-parameter optimisation of different machine learning methods with the same computational budget might not be the best approach since it does not separate the efficient implementation of algorithms, which is an issue on software engineering/high performance computing from model complexity, which is purely related to model design and unrelated to the exact implementations of algorithms. In order to make sure results related to computational resources are replicable, detailed documentation of hardware and OS environment used to train the models are required. 

%% Diversified incremental learning models 
For incremental learning tasks, diversified models can be created using degree of freedoms that are unrelated to a particular machine learning methods(random seeds, retrain period, global feature and data sub-sampling) and to some extent, model designs that are applicable to a wide range of models (model snapshots over different complexities). For fair comparison of machine learning methods, it is not enough to simply compare the methods with the same set of data cross-validations but also under different data sampling schemes and model complexity regimes. Reporting the aggregated performances over different training settings would provide more useful insights for real-life performances of prediction systems. 







\subsection{Comparison with Numerai Meta Model}

Model performances obtained from the optimised XGBoost Incremental Learning model are compared with the Meta Model, which is an aggregation of the predictions from participants. There are two different methods to measure model performances. The first method is to \textbf{directly} compare the performances of predictions obtained from the incremental learning model to the Meta Model. The second method is to \textbf{indirectly} compare the performance of an \textbf{ensemble} consisting of $50\%$ incremental learning model and $50\%$ Meta Model to the Meta Model. The first method can identify models that are good as standalone models. The second method can identify models that are good within \textbf{ensemble}. These models might not have the best performances on a standalone basis but are still important for the Meta Model as they give orthogonal(uncorrelated) predictive signals to the known ones, which is demonstrated by the improvement of different portfolio metrics within the ensemble. 

Tables \ref{table:XGBEnsemble500}, \ref{table:XGBEnsemble5000}, \ref{table:XGBEnsemble50000}, \ref{table:XGBEnsembleComplexity5000}, \ref{table:XGBEnsembleComplexity5000}, compare the performance of the above XGBoost models trained with different number of boosting rounds with the Meta Model from Era 888 to Era 1050 \footnote{The Numerai Meta Model data starts at Era 888}. All XGBoost models have a higher Mean Corr than the Meta Model. The equal-weighted ensemble of Meta Model and XGBoost Model have a better performance in all metrics (Mean Corr, Sharpe and Calmar), even though the improvement is limited for XGBoost models with more than 5000 boosting rounds. The XGBoost models are both good models on a \textbf{standalone} and \textbf{ensemble} basis. 

%% Do not always need larget models 
XGBoost models with 5000 and 50000 boosting rounds have similar Mean Corr. The largest XGBoost models have a slightly higher Calmar ratio. Whether spending 10 times more computational resources to train larger XGBoost models is economical depends on the utility function of the computational costs of researchers. 

%% Complex Models are better in complmentary in nature 
XGBoost models without early stopping slightly under-performed XGBoost models with early stopping on a standalone basis, but these models can improve the Ensemble model performances more.  A possible reason is due to most participants do not train large XGBoost models without early stopping. Other than constraints in computational resources, the unconscious bias stemming from classical viewpoint on machine learning leads to most participants applied different regularisation techniques in model training. 

%% Alpha Decay 
The Meta Model is determined by the collective behaviour of participants. The improvement of using XGBoost models without early stopping might decay over time as participants started to shift towards over-parameterised models after the publication of this paper. 


\begin{table}[!ht]
    \centering
    \caption{XGBoost models with different retrain periods (500 boosting rounds)}
    \begin{tabular}{|l|l|l|l|}
    \hline
        Method  & Mean Corr  & Sharpe  & Calmar  \\ \hline
        XGBoost Model  & 0.0227  & 1.0274  & 0.2239 \\ \hline
        Meta Model  & 0.0198  & 0.9563  & 0.3345 \\ \hline
        Ensemble  & 0.0235  & 1.0536  & 0.3250 \\ \hline
    \end{tabular}
    \label{table:XGBEnsemble500}
\end{table}


\begin{table}[!ht]
    \centering
    \caption{XGBoost models with different retrain periods (5000 boosting rounds)}
    \begin{tabular}{|l|l|l|l|}
    \hline
        Method  & Mean Corr  & Sharpe  & Calmar  \\ \hline
        XGBoost Model  & 0.0253  & 1.1430  & 0.3795 \\ \hline
        Meta Model  & 0.0198  & 0.9563  & 0.3345 \\ \hline
        Ensemble  & 0.0253  & 1.1500  & 0.3997 \\ \hline
    \end{tabular}
    \label{table:XGBEnsemble5000}
\end{table}



\begin{table}[!ht]
    \centering
    \caption{XGBoost models with different retrain periods (50000 boosting rounds)}
    \begin{tabular}{|l|l|l|l|}
    \hline
        Method  & Mean Corr  & Sharpe  & Calmar  \\ \hline
        XGBoost Model  & 0.0254  & 1.1581  & 0.4264 \\ \hline
        Meta Model  & 0.0198  & 0.9563  & 0.3345 \\ \hline
        Ensemble  & 0.0255  & 1.1741  & 0.4100 \\ \hline
    \end{tabular}
    \label{table:XGBEnsemble50000}
\end{table}



\begin{table}[!ht]
    \centering
    \caption{XGBoost models with different model complexities (5000 boosting rounds)}
    \begin{tabular}{|l|l|l|l|}
    \hline
        Method  & Mean Corr  & Sharpe  & Calmar  \\ \hline
        XGBoost Model  & 0.0248  & 1.1757  & 0.4001 \\ \hline
        Meta Model  & 0.0198  & 0.9563  & 0.3345 \\ \hline
        Ensemble  & 0.0263  & 1.2162  & 0.4994 \\ \hline
    \end{tabular}
    \label{table:XGBEnsembleComplexity5000}
\end{table}


\begin{table}[!ht]
    \centering
    \caption{XGBoost models with different model complexities (50000 boosting rounds)}
    \begin{tabular}{|l|l|l|l|}
    \hline
        Method  & Mean Corr  & Sharpe  & Calmar  \\ \hline
        XGBoost Model  & 0.0247  & 1.1556  & 0.4428 \\ \hline
        Meta Model  & 0.0198  & 0.9563  & 0.3345 \\ \hline
        Ensemble  & 0.0261  & 1.2211  & 0.5068 \\ \hline
    \end{tabular}
    \label{table:XGBEnsemble50000}
\end{table}



\section{Discussion} 

%% Summary
In this paper, both traditional tabular and factor-timing models are studied for the incremental learning problem on the temporal tabular dataset from Numerai. Traditional tabular models, if retrained regularly can adapt to distribution shifts in data which cannot be modelled by factor-timing models effectively. 

Not all methods of increasing model complexities can improve model performances. For factor-timing models, increasing the size of random feature sets or increasing the number of layers in transformers layers are inefficient ways of data learning. For transformer models, there is no clear relationship between model complexities and performances, results from hyper-parameter optimisation are not robust and might not be useful for future predictions. For MLP models, increasing the number of layers does not always improve model performance. The optimal model complexity depends on both the architecture and training process. For XGBoost models, increasing the complexity of a \textbf{single} model by disabling early stopping will increase the risk of over-fitting. However, increasing model complexity horizontally through different model ensemble methods such as bagging is more robust than training a single large model as there is fewer risks in over-fitting the data over different models. Not all approaches to building complex models are equal. Some approaches are more robust than others based on both theoretical and empirical findings. 

There is no simple rule to select whether to use classic or over-parameterised machine learning models. The optimal model complexity depends on the training process and datasets for a \textbf{static} machine learning problem. Within an incremental learning framework, training multiple models with different complexities and then applying soft hyper-parameter selection within a deep model framework is better than using cross-validation to select a \textbf{single} set of 'optimal' hyper-parameters. 

Incremental learning is an effective way to adapt to distribution shifts in data. Both the training size and model retrain/update period should be selected based on walk-forward prediction performances of the early parts of the data stream. The training size needs to be large enough to cover different data regimes in history and not too large to include old data that are no longer relevant. In general, increasing the model retrain period can improve model performances but the requirements on computational resources also increase. Therefore, trade-offs between computational costs and the marginal gain in model performances are made for practical incremental learning systems. 


%% Is optimisation always useful? 
Unlike the previous work \cite{wong2023dynamic} which demonstrates \textbf{dynamic} learning can improve prediction models based on known statistical rules, applying \textbf{dynamic} learning iteratively on models with variability due to randomness on feature and data sampling cannot generate predictions better than simple averages. The findings are in direct contrary to empirical evidences of model stacking on \textbf{stationary} data, where model stacking are usually considered to have low risk of over-fitting. 

Some design choices, such as to use shallow trees instead of deep trees in a GBDT model will be generally true for different datasets. These design choices can be set with human expertise and do not require hyper-parameter optimisation. For design choices which are reasonably known to hold for models \textbf{globally} regardless of distribution shifts in data, such as the data sampling ratio of decision trees in XGBoost models, hyper-parameter optimisation can be performed using early observations of data and then used in the rest of the incremental learning pipeline. For design choices that are not certain to have \textbf{global} optimum, such as the number of boosting rounds of XGBoost models (model complexity) or the choice of machine learning models, applying \textbf{dynamic} optimisation techniques such as deep incremental learning can learn model parameters/weights that adapt to changes in the data stream. For design choices due to randomness or other hyper-parameters that do not demonstrate any pattern over observed history, a simple average over possible choice is preferred to other statistical/machine-learning-based optimisation methods. 


%% Both viewpoints about the model complexity is wrong 
The Numerai datasets used in this paper showed that \textbf{both} viewpoints on model complexity are wrong. 

Classical viewpoint expects over-fitting of model after certain number of boosting rounds. Mean Corr drops slightly as number of boosting rounds increases but Sharpe ratio is not changed significantly. Robustness of models improves as Calmar ratio increases when models become larger. This suggests rather than being an over-fitted model, larger models are more robust towards downside risks.  

The model ML viewpoint is also not correct as both deep factor-timing models based on Signature and random Fourier transforms under-performed simple trend-based models. Complex MLP architectures also under-performed a simple MLP model with 2 layers. This observation is unexpected since these methods are used as examples in research papers advocating the modern ML viewpoint \cite{kelly2022virtue,belkin2019reconciling,}. 

For XGBoost models of different number of boosting rounds, snapshots taken after $20\%$ of boosting rounds are all useful candidates to be used in an ensemble. This is further supported by the fact that an equal weighted ensemble of model snapshots performed better than deep incremental learning models which attempts to overweight snapshots of particular complexities. A key assumption made is the product of learning rate and number of boosting rounds are kept constant in the trained XGBoost models. The speed of convergence of the training process so that it will not converge too quickly. 

Selection of optimal model complexity and other model hyper-parameters would remain problem-dependent for different incremental learning tasks. The baseline of a simple averaging over possible hyper-parameters would gives a good estimate when there is not enough data for optimising the small differences between different hyper-parameter choices. 

%% Recommendations for non-stationary datasets 
For modelling non-stationary data streams, the philosophy of "less is more" can be used. Bootstrap techniques like data/feature sub-sampling are often as effective as feature Engineering and synthetic data generation methods. Simple rule-based methods for model selection/stacking are more robust compared to ML-based methods under distribution changes. If non-linearity is required, simpler models such as GBDT or random forests are better than deep learning methods. The simple trick of applying equal-weighted ensembles over different model designs will often work as good as sophisticated hyper-parameter optimisation methods. %This viewpoint is supported by recent findings that design spaces for hyper-parameter optimisation are often over-engineered such that random searches can perform well \cite{zimmer-tpami21a}. 


%% Further Work 
In most practical applications, \textbf{multiple} machine learning methods are used together to create an ensemble prediction. The incremental learning model presented in this papers provides a comprehensive way to integrate different machine learning models in a consistent and systematic way to create point-in-time predictions. With a multi-layer structure and modularised design within each layer, the deep incremental learning model can flexibly model datasets with different complexities and structures. Further work can be done by integrating different deep tabular models into the model and benchmark different machine learning method under the incremental learning framework. 






\section{Acknowledgements}
This was was supported in part by the Wellcome Trust under Grant 108908/B/15/Z and by the EPSRC under grant EP/N014529/1 funding the EPSRC Centre for Mathematics of Precision Healthcare at Imperial. We sincerely thank Numerai GP, LLC for providing the datasets used in the study.  


%Bibliography
\newpage
\printbibliography







\end{document}



