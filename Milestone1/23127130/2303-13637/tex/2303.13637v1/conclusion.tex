\section{Conclusion}
\label{sec:Conclusion}

% The results from our preliminary study based on the limited PPG and ECG data validate the feasibility of the proposed system, which can accurately obtain HRV directly based on PPG sensor data in real-time powered by various ML models and algorithms. 
% In the experiment, we compared the estimation accuracy, model size, and inference time for different ML algorithms.
% We find DT, RF, and MLP are more suitable than SVM and KNN, and we can achieve MAPE smaller than 17\%.

% In the future, we plan to collect more data from different subjects to further validate the results.
% To further improve the monitor system accuracy, we plan to explore the accelerometer data to improve the estimation accuracy.
% Furthermore, to further validate and investigate the efficiency of the monitoring system, we will deploy the system on resource-limited microcontroller-based hardware.

Photoplethysmography (PPG) sensors have been shown to be a good alternative for electrocardiographic (ECG) in Heart Rate Variability (HRV) monitoring. However, to be applied to practical and medical use, PPG HRV inference methods must be carefully designed. Prior work typically employed signal-processing-only or machine-learning-only methods to indirectly infer HRV from PPG signals, leading to low accuracy and large models. In this paper, we presented a compound and direct HRV inference method, which combines signal processing and machine learning to directly infer HRV. Evaluation results show that our method has errors as low as 3.5\% with model sizes of a few hundred KBs, suggesting that our method can be applied in small embedded devices and potentially for medical uses. 
%We plan to expand this work to more human subjects to evaluate its effectiveness at a much larger scale in the future.
