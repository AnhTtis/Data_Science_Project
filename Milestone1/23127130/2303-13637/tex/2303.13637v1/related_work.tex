\section{Related Work}
\label{sec:related_work}

%In this section, we present the related works on HRV estimation. 
%Clinically, ECG is the device that measures HRV.
%Since ECG needs to be connected by wires to electrodes that are attached to the human body, it is inconvenient to install and operate in daily life. 
%Photoplethysmography (PPG) sensors have become a popular replacement for ECG in HR/HRV monitoring due to their small sizes and ease of deployment.
Although there are many existing works on HR monitoring \cite{yuntong2022hr,bashar2019machine,chang2021deepheart,biswas2019cornet,zhang2014troika,panwar2020pp}, only a few studied HRV monitoring.
%Therefore, we want to explore PPG for accurate HRV monitoring in this paper.
%The studies on HRV monitoring can be partitioned into two categories based on their methodologies -- one category primarily employs signal processing, whereas the other employs machine learning. 
%The rest of this section discusses prior HRV monitoring studies.

% PPG uses light signals to monitor blood flow.
% Heartbeats cause periodical changes in the blood flow, which cause periodical changes in the reflected light received by the PPG sensor.
% Hence, periodical PPG light wave changes can be exploited to show the heartbeat and the RR intervals, which can be used to generate HRV using Equations~\ref{eq:sdnn} and~\ref{eq:rmssd}.
% The main issue with PPG sensors is their noisy signal, which is usually the result of motion artifacts (MA).
% The common method to remove MA is signal processing, which is employed by many prior works to estimate RR intervals. 

% On the contrary, PPG is easy to install, and that is one of the reasons why PPG is a great substitute for ECG.
% Another reason that PPG is more suitable than ECG for the HR and HRV monitor system is that PPG is already widely integrated into many wearable devices such as chest straps, smart rings, and smartwatches.

\subsection{PPG HRV Monitoring with Signal Processing}
Ghamari et al. proposed a signal processing algorithm including High-Pass and low-Pass Filters to detect R peaks~\cite{ghamari2016design}.
Srinivas et al. proposed a signal processing method including moving average and FFT to measure HRV based on the PPG wave~\cite{srinivas2007estimation}. 
However, both of them did not evaluate the accuracy of HRV estimations.
%. nor compare PPG HRV with ECG HRV.
Wang et al. proposed an algorithm to estimate RR intervals from a smartwatch PPG sensor and accelerometer~\cite{wang2019s}. 
They calculated HRV measurements, such as SDNN and RMSSD. 
%based on the RR interval obtained, and used them for health condition monitoring like drowsiness detection.
%Although this work compared PPG HRV with ECG HRVm  
However, no quantitative accuracy was reported.
%The PPG is installed on the ear-lobe.% and they tried to optimize the system power consumption. 
%But in our work, we install PPG on the fingertip which produces higher MA in the PPG signal, and we want to improve the estimated HRV accuracy using ML, not just using signal processing methods.
Blake et al. devised an HRV estimation hardware that contains a PPG sensor, an accelerometer, a Bluetooth module, and a battery \cite{blake2015development}. 
%They focused on hardware design and estimated SDNN with both PPG signals and accelerometer data. 
This study only compared the HRV readings from a chest strap instead of ECG with no quantitative report on accuracy.
Jankovi{\'c} and Stojanovi{\'c} designed a signal processing algorithm including a low-pass filter, Sum Slope Function, and a peak extraction function to find R peaks~\cite{jankovic2017flexible}.
They collected data for 8 minutes in the experiment and obtained one HRV for each trace, and compared them with ECG HRV. 
However, it was unclear which HRV metric they used.
%and their collected data is not public so it is hard to compare with it.
Bhowmik et al. proposed an algorithm including wavelet denoising, trend removal, and peak extraction to detect R peaks in PPG signals~\cite{bhowmik2017novel}. 
They found that a 100Hz PPG sampling rate is not suitable for a smartwatch  %considering the trade-off between accuracy and 
due to high power consumption and chose 25Hz. 
% Inspired by this study, we also adopted a 25Hz sampling rate. However, we relied on ML to handle the noises due to the low sample rate. Moreover, this paper is only about peak detection instead of HRV monitoring.
%they did not calculate HR or HRV based on peak intervals, thus it is not possible to evaluate their HRV monitoring performance. 
Inspired by this work, we also choose 25Hz as the PPG sampling rate to save energy. 
%However, the low sampling rate also makes it necessary to combine signal processing with ML to improve the estimation accuracy.
Saadeh et al. 
%focused on the hardware design of a PPG-based HRV detection system 
%They detected HRV with an 
employed earlobe-attached PPG and wavelet decomposition and moving average filters to estimate RR intervals~\cite{saadeh20180}.

Note that, due to the nature of PPG light signals, the above prior studies all estimated
%R peaks and 
RR intervals first, and then converted into HRV. %\cite{blake2015development,jankovic2017flexible,bhowmik2017novel,ghamari2016design,srinivas2007estimation}. 
As discussed previously, inferring HRV through RR intervals can lead to low accuracy due to error amplification. Moreover, as we show in this paper, pure signal processing may not be able to handle all types of PPG noises.


\subsection{PPG HRV Monitoring with Machine Learning}
%Recently, ML has been found to be a promising method to remove MA from PPG signals and estimate HRV.

While there are prior studies that employed ML models to estimate HRV, most of these studies focused on R peak/RR interval estimation, instead of predicting HRV directly.
Everson et al. proposed a CNN-based encoder-decoder network to construct ECG waves from PPG waves and evaluated HRV based on the predicted wave \cite{everson2019biotranslator}.
They evaluated the model with the small ISPC dataset, which led to only one HRV estimation per recording. 
%and since ISPC is too short for HRV, they only have one HRV reading per recording.
%Besides, as shown in section \ref{sec:motivation}, their HRV accuracy is not gratifying. 
Similarly, Chiu et al. designed a CNN-based encoder-decoder with a sequence transformer network to generate ECG waves from PPG waves \cite{chiu2020reconstructing}.
They evaluated the model with the UQVSD dataset and the BIDMC dataset - both are datasets from barely moving patients with low motion artifacts. %, whereas we consider healthy people with higher activity intensity.
Xu et al. classified PPG signals to systolic or diastolic phase using an RNN model with the assistance of an accelerometer \cite{xu2019deep}. 
They obtained RR intervals based on the classification results and evaluated the RR interval estimations.
%However, since they did not evaluate HRV, it is hard to gain insights into the accuracy of their HRV.
Wittenberg et al. compared a few CNN and GRU classification models for PPG R peaks detection \cite{wittenberg2020evaluation}. 
They classified short PPG waves based on whether the first sample in it is an R peak. %, while in our paper, we focus on using a regression algorithm to estimate HRV directly.
% they used ISPC for HRV estimation and the MAPE is low.
Maritsch et al. proposed a CNN model to predict the error of the RMSSD estimations from a smartwatch~\cite{maritsch2019improving}. This work did not involve PPG signals. 
Alqaraawi et al. explored Bayesian learning to detect the PPG peaks with their collected data \cite{alqaraawi2016heart}.
However, their monitoring length was only 5 or 8 minutes, and hence, only one HRV was provided, while our data lasted 2 hours. % and the statistical error is more realistic.
%However, they did not mention the detailed CNN architecture nor public their collected data, hence it is difficult to compare with it.
Choudhury et al. used a phone camera to capture signals of a human fingertip and extract RR intervals from the collected data with adaptive neural network (ANN) and SVM \cite{choudhury2013heartsense}.
%instead of the RMSSD itself% and found that the error is closely related to the accelerometer data
%
% However, as we showed earlier in section \ref{sec:motivation}, it is usually more accurate to directly estimate HRV from PPG signals than using RR intervals as a proxy because the errors of RR intervals tend to amplify when converted to HRV.
Instead of predicting HRV, most of the above studies mainly predicted/detected R peaks or RR intervals.
However, as we show in Section \ref{sec:motivation}, small errors in RR peak intervals can be magnified into large errors in HRV estimates.
Therefore, in this work, we estimated HRV directly (represented by SDNN and RMSSD) from PPG signals.

Commercial wearable devices may also provide HRV estimations, such as Garmin smartwatches~\cite{GraminHRV2}. However, prior work has shown that smartwatch estimations may have high errors~\cite{maritsch2019improving}. Due to their proprietary nature, we were not able to rigorously evaluate commercial wearable devices. Therefore, we focused on comparing and analyzing existing research studies in our motivation and evaluation sections.




% There are two types of PPG, reflection, and transmission, the difference is whether the light source and the detector are on the same side or on different sides.
%, an accelerometer is usually used with PPG at the same time to collect motion signals, which can reduce the motion artifact.
%based on that, and obtain HRV eventually.
% For example, Lu et al. showed how an ear-clip PPG sensor and the same processing method as the ECG can effectively estimate HRV \cite{lu2009comparison}.
% Pinheiro et al. investigated whether PPG can be used to analyze HRV in healthy subjects at rest, healthy subjects after physical exercise, and subjects with cardiovascular diseases (CVD) \cite{pinheiro2016can}. They concluded PPG can be used to estimate HRV for healthy subjects at rest while for post-exercise and CVD subjects, PPG should be used with caution.
% Jeyhani et al. analyzed beat-to-beat intervals obtained from ECG and PPG in 19 healthy male subjects to evaluate whether we can use PPG to estimate HRV \cite{jeyhani2015comparison}. 
% They compared 5 different time-domain HRV parameters and found SDNN and SD2 have the smallest error while pNN50 has the largest error.
% These works concluded PPG is useful in certain cases, but they did not apply extra techniques to improve PPG's HRV accuracy. 
% However, in this work, we study how to process PPG data with different ML algorithms and signal processing algorithms to obtain more accurate HRV.

% The traditional way to obtain HR and HRV in a clinic is using an electrocardiogram (ECG), hence we first illustrate how to estimate HR and HRV from ECG.
% Then, since ECG is difficult to implement in daily life, and a great alternative for ECG is PPG, we also explore the conventional signal processing method to calculate HR and HRV from the PPG signal.
% Lastly, we present some popular ML regression algorithms that can be utilized for HR and HRV estimation and introduce some existing works that employ ML to estimate HR and HRV.
% \subsubsection{HR from PPG - signal processing}

% The signal processing method typically tracks the peak of the PPG signal. Zhang et al. proposed an algorithm that consists of signal decomposition, sparse signal reconstruction, and spectrum peak tracking to extract HR from PPG signal in intensive physical activities environment~\cite{zhang2014troika}.
% Their dataset, the IEEE Signal Processing Cup (ISPC) 2015 dataset, is popular for evaluating HR monitoring solutions. 
% It includes PPG sensor data, accelerometer sensor data, and ECG data. 
% %Different from our work, 
% In this ISPC dataset, the PPG sensor is installed on the wrist and the ground truth is a one-channel ECG.
% While in our work, we installed PPG on fingertips and utilized a three-lead ECG.
% Additionally, each ISPC data recording lasts for only a few minutes, whereas each trace in our data lasts for more than one hour.
