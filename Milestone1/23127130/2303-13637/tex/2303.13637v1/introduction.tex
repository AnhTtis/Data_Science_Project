\section{Introduction}
Heart Rate Variability (HRV) measures the variation of the time intervals of consecutive heartbeats and is a major indicator for health conditions, such as coronary artery disease, heart failure, hyperlipidemia, and hypertension~\cite{2012-Xhyheri-HRVToday}. HRV is traditionally measured using electrocardiographic (ECG) devices, %or Holter monitors, 
which record the heart's rhythm. However, ECGs 
%or Holter monitors 
can be expensive and require attaching several electrodes to human bodies, which can be inconvenient to use.

As an alternative to ECG, Photoplethysmography (PPG) sensors, which monitor light signal changes in blood flows, can also be utilized to measure heart rate (HR) and HRV~\cite{2007-Wang-TBCS-PPGEar,zhang2014troika}. A PPG sensor can be placed on the human skin to provide HR/HRV readings and is more convenient to use. Many studies demonstrated the potential of PPG sensors in HRV monitoring~\cite{everson2019biotranslator,wittenberg2020evaluation}. However, these studies usually face the following four limitations, restraining their application to small embedded devices and healthcare in the future.

First, some of these studies relied on only signal processing for HRV estimation~\cite{zhang2014troika,bhowmik2017novel}, which may negatively affect the inference accuracy. PPG sensor signals typically suffer from large noises, particularly, the noises from motion artifacts (MA). To infer HRV with good accuracy, these noises must be removed or reduced. Although signal processing techniques can remove many noises (including MA), these relatively static techniques may not be able to handle all types of noises, leading to low HRV accuracy.

% These noises can be handled by either signal processing or machine learning. %For example, \TODO{please add some examples}. 
% However, relying on purely signal processing may lead to high HRV estimation errors, as the motion-induced noises can be too high for a fixed signal processing algorithm.

Second, many studies also only employed machine learning (ML) techniques to infer HRV~\cite{everson2019biotranslator,chiu2020reconstructing}.
Although many ML models by themselves are sophisticated enough to handle all types of noises, the resulting models can be too large and/or too slow for small embedded devices. Moreover, training and tuning ML models with raw PPG data can be quite challenging. ML models without enough tuning may also have low accuracy.

Third, some prior studies also focused on the inference of the interval lengths between consecutive heart beats~\cite{everson2019biotranslator,xu2019deep}, which are known as RR intervals~\cite{dohare2014efficient,GOLDBERGER201811}. That is, these studies do not direct infer HRV metrics, such as SDNN or RMSSD (more in Section~\ref{sec:background}). This indirect inference strictly follows HRV's definition. However, due to error amplification, this indirection can significantly increase the errors of the final HRV estimations.

Fourth, many prior studies also relied on small datasets. For example, a popular PPG dataset, the IEEE Signal Processing Cup (ISPC) dataset, has PPG signals recording lasted for only five minutes~\cite{zhang2014troika}. However, a typical HRV inference requires an ECG monitoring length of half to five minutes~\cite{acharya2006heart}. Hence, it is difficult to conduct HRV inference studies with short recordings, especially for studies with neural networks.
%The small dataset also makes it different to develop neural network-based HRV estimations. 

We have studied ML-based HR estimation using PPG in our prior work \cite{yuntong2022hr}. In this paper, our goal is to design an HRV inference methodology that can provide high accuracy for resource-constrained embedded devices. To achieve this goal, we designed a \textit{compound and direct} method that combines signal processing and ML to directly infer HRV (i.e., RMSSD/SDNN). More specifically, we first employed signal processing to remove outliers and noises from raw PPG signals and convert the PPG signals into rough HR readings and HRV readings. These rough HR/HRV readings are then fed into an ML model to infer RMSSD/SDNN. Applying ML after signal processing provides additional/better noise removal, and hence, more accurate HRV estimations. Applying signal processing before ML avoids the need for large and slow ML models. 
%Apply ML after signal processing provides better HRV accuracy, whereas using signal processing before ML helps to avoid employing complex and large machine learning models. 
Moreover, the direct inference of RMSSD/SDNN, instead of inferring RR intervals as a proxy, %our methodology directly produces HRV estimations as SDNN or RMSSD, which 
further improves accuracy.
%Another benefit of this compound method is that it allows us to sample the PPG sensors at only 25Hz, which saves energy for embedded devices~\cite{bhowmik2017novel}.

To explore the impact of various ML algorithms, 
%on the accuracy, model size, and inference time, 
%and feature sizes (i.e., the size of early HR/HRV readings as input features), 
we also evaluated 
%the accuracy, model size, and inference time for 
different ML models, % with different feature sizes (i.e., number of early HR/HRV readings), 
including Decision Tree (DT), Random Forest (RF), K-nearest neighbor (KNN), Support vector machines (SVM), and Multi-layer perceptron (MLP). To provide more reliable results, we also collected a new dataset of PPG signals, with ECG readings as ground truth. This dataset contains three 2-hour-long PPG/ECG traces for one human subject performing different activities, including office work, sleeping, and sitting.
% and each recording lasted for more than 2 hours.

% for RMSSD, it is sit-DT-300 case, MAPE 5%, its sig-proc only MAPE is 21%. 
% for SDNN, it is sit-mlp-300 case, MAPE 5.25%, its sig-proc only MAPE is 5.73%. 
% the highest Sig-proc-only MAPE is 51.02%
% raw2hrv MAPE is 5.2% for RMSSD and 8.4% for sdnn when the model is large although the predicted HRV is a straight line. Cannot train an ML-only model with a similar model size
Evaluation results show that our compound and direct method has 3.5\% to 25.7\% errors for various activities and monitoring lengths. Both the lowest 3.5\% error for RMSSD and the lowest 5.1\% error for SDNN were obtained with a monitoring length of 300 seconds per HRV estimation. Note that, 300-second HRV monitoring can be used for caring for chronic renal failure and diabetes~\cite{acharya2006heart}, showing the healthcare potential of PPG-based HRV monitoring. The evaluation results also show that our method is significantly more accurate than the signal-processing-only and ML-only methods.
%, whose errors were as high as 51.02\%.

Moreover, the model exploration showed that 
%all ML algorithms have average errors of around 12\%. However, 
DT and MLP models are usually smaller with good accuracy, making them more suitable for small embedded devices. DT models have an average error of 13.0\% and are usually less than 10KB with inference time less than 10$\mu$s. MLP models have an average error of 9.1\% and are less than 469KB with inference time less than 1ms.  
These results corroborate the "Rashomon" theory~\cite{Rashomon-ML} that, for some problems, there exist simple models with good accuracy and meet special requirements, such as the limited memory size of embedded devices in this case.

%less than \TODO{20\%} error for HRV estimation with different types of machine learning models, with the lowest error being \TODO{10\%}. It usually requires 300 HR readings over 5 minutes of continuous monitoring to achieve the lowest accuracy for most types of models. Moreover, our exploration showed that Decision Tree models usually could provide accurate estimation with low model sizes of at most 10 KB. It is also worth noting

%Note that, when the model size is unrestricted, it is possible to reach an error of only \TODO{5\%} with an MLP model with \TODO{xx} parameters. This result shows that PPG-based HRV monitoring has great potential, although more research effort is required to achieve such high accuracy that is proper for deployment in small embedded devices.

The contributions of this paper include,
\vspace{-2mm}
\begin{itemize}
    \item The compound and direct HRV inference methodology combines signal processing and ML to directly infer RMSSD/SDNN to achieve high accuracy with small and fast ML models.

    \item A systematic exploration of different ML algorithms to study their impact on the accuracy, model size, and time on HRV inference. This exploration showed that Decision Trees and MLP can achieve high accuracy with small/fast models suitable for embedded devices.

    \item A comprehensive PPG/ECG dataset to study HR/HRV inference, which contains traces of different activity intensities lasting for 2 hours. 
\end{itemize}

The rest of the paper is structured as follows.
Section~\ref{sec:background} discusses the background on HRV and the motivation of our work. 
Section~\ref{sec:method} presents our compound and direct method.
Section~\ref{sec:Evaluation} presents evaluation results. 
Section \ref{sec:related_work} discusses related work, and section~\ref{sec:Conclusion} concludes the paper.


% HR and HRV are critical health vital signs and it is necessary to continuously monitor them for different purposes; 
% traditional approach to obtain HR and HRV is to adopt ECG devices, which are hard to operate and inconvenient to utilize; 
% Several wearable devices (such as chest straps, and smartwatches) have been designed to monitor HRV using PPG sensors. However, most such devices do not provide an interface to access data for further analysis. 
% In this work, with the objective of developing a low-power and sustainable wearable device to monitor HR and HRV using PPG sensors, we study various ML models and data processing methods to obtain accurate HR and HRV from PPG sensor data. Explain data processing methodology and ML models, and summary of results; 
% list of contributions; paper organization
% Heart rate (HR) is an important vital sign for the cardiovascular system and has been widely used as a biomarker for diagnostic and early prognostic of several diseases (such as heart failure)~\cite{fox2007resting}. In addition, heart rate variability (HRV), which reflects the variation in the time intervals between successive heartbeats, is also a key health marker that helps measure the autonomic nervous system state~\cite{acharya2006heart}. Besides the critical health condition monitoring in the hospital setting, many applications also depend on continuously measured HR and/or HRV, such as arrhythmia detection~\cite{rajpurkar2017cardiologist} and fatigue driving detection~\cite{patel2011applying}. Therefore, it is desirable to have a real-time HR/HRV monitoring system, which can conveniently provide accurate data in an effective manner to support such applications. 

% The traditional and most reliable approach to continuously monitoring HR/HRV is to utilize electrocardiogram (ECG) devices, which are generally expensive and inconvenient to deploy for outpatients or other users to operate in a continuous manner. Photoplethysmography (PPG), a handy sensor, when applied to the surface of the skin, like fingers, wrist, or earlobe, can utilize light signal to monitor changes in blood flow, which can be exploited to derive HR/HRV~\cite{castaneda2018review}. Given their low cost and convenience, PPG sensors have been widely utilized as an inexpensive alternative to monitor HR in wearable embedded devices (such as smartwatches)~\cite{castaneda2018review}, which have limited energy and resources. However, despite the wide interest, existing work using PPG to detect HR/HRV typically has three limitations.

%To make sure the monitoring system is easy to use in everyday life and to avoid transmitting sensitive health data over the Internet, there is a need to implement it in an edge device rather than utilize detection hardware that is portable and easy to install.
% To make sure the monitoring systems are easy to use in everyday life and protect user data privacy, there is a need to implement it in an edge device with detection hardware that is portable and easy to install.
% where a certain number of electrodes are placed at proper spots on the human torso to produce an electrocardiogram that shows the heart's electrical activity over time~\cite{wittenberg2020evaluation}. 
% For patients who need intensive care in hospitals, such ECG devices can provide accurate and real-time HR/HRV monitoring. 
% However, they are generally expensive and the cable-connected electrodes make their deployment inconvenient for outpatients or other users to operate in a continuous manner.

%First, the classical method for extracting HR/HRV from PPG data is based on signal processing, which usually requires a relatively high sampling frequency (hundreds of Hz) and a complex algorithm to achieve high accuracy. This high sampling frequency may incur high energy consumption, preventing these signal-processing techniques from being applied to energy and resource-constraint devices. 
% to obtain satisfactory estimation accuracy, and therefore not suitable for edge devices.
%For example, a widely utilized dataset for HR estimation is IEEE Signal Processing Cup (ISPC) dataset, which was proposed by Zhang et al., along with a signal processing-based algorithm that includes signal decomposition, sparse signal reconstruction, and spectrum peak tracking, to extract HR from PPG data~\cite{zhang2014troika}.
%The ISPC dataset contains PPG signals sampled at a high frequency of 125 Hz, which was widely adopted by other studies~\cite{zhang2015photoplethysmography,bashar2019machine,puranik2019heart,chang2021deepheart}.
%Bhowmik et al. also reported that sampling at 100hz consumed significantly more energy than 25 Hz~\cite{bhowmik2017novel}.
%Bhowmik et al. came up with an algorithm to detect peaks in smartwatch PPG signals and found that keeping the PPG sampling rate at 100 Hz consumed significantly more energy than 25 Hz, and hence chose 25 Hz for their experiments\cite{bhowmik2017novel}. 
%Furthermore, signal processing usually requires an intricate algorithm to achieve high accuracy, which further worsens energy consumption. 
%and a relatively higher PPG sampling rate, which is energy-consuming, thus it is difficult for resource-constrained edge devices to support continuous HR/HRV monitoring for a long time.
% Therefore, the signal processing method is not suitable for edge devices.
% Based on the PPG sensing data, signal processing has been the classical approach to deriving various health-related information (including respiratory rate~\cite{johansson2003neural}, HR~\cite{zhang2014troika}, or HRV~\cite{morresi2020analysing}). 

%Second, with the advancement of machine learning (ML), several studies exploited various machine learning models to estimate HR or HRV.
%However, ML models, especially neural networks, usually are computationally intensive, limiting their application to embedded devices with limited resources. 
%In most existing works that use ML to predict HR/HRV from PPG signals, they used PPG signals as learning features, which result in large feature dimensions and possibly large model sizes
%For example, Wittenberg et al. used deep learning models including convolutional neural network (CNN) and Gated recurrent unit (GRU) for PPG peaks detection \cite{wittenberg2020evaluation}. 
%There were also studies that employed ML models such as K-means, Random Forest (RF), and Bayesian learning algorithm%, but they did not evaluate the model size 
%~\cite{bashar2019machine,alqaraawi2016heart}. 
%Moreover, existing works employ ML models~\cite{everson2019biotranslator,biswas2019cornet,xu2019deep} also employed a large number of features (i.e., PPG signal) to achieve high accuracy, which led to large models that may not fit in small embedded devices.
%To enable ML-based HR/HRV monitoring in a resource-constrained embedded device, there needs research to investigate feature dimension reduction to allow smaller ML models. There also needs exploration to determine the type of ML models that can provide accurate HR/HRV readings with low resource usage. 
%of an ML model is needed to ensure that it can run smoothly on resource-constrained devices such as embedded devices or edge devices.
%It also is necessary to evaluate the ML model efficiency on these resource-limited devices to find the recommended ML models for the monitoring system.
%However, for both HR and HRV monitoring, previous works have not compared the accuracy and model size between different ML models, which cannot provide sufficient insights on which ML model is suitable for edge devices. 
%Some prior works used deep learning models, which usually are more resource-intensive than simple ML models. 

%Therefore, a systematic evaluation of different ML models is in demand to find the best ML model that achieves a good balance between accuracy and model size.

% Recently, with the advancement of machine learning (ML), there have been several studies that exploited various machine learning models (such as Bayesian learning~\cite{alqaraawi2016heart}, RF~\cite{bashar2019machine}, and neural network~\cite{wittenberg2020evaluation,everson2019biotranslator}) to estimate HR or HRV.
% For example, Bashar et al. proposed a K-means clustering and RF-based HR estimation algorithm \cite{bashar2019machine}.
% Alqaraawi et al. used a Bayesian learning algorithm to detect the PPG peaks and concluded that Bayesian learning is helpful and improved HRV peak detection performance \cite{alqaraawi2016heart}.
% However, ML algorithms usually are computationally intensive, while PPG is usually used in a wearable device, which is resource-limited.
% Therefore, it is necessary to evaluate the ML model efficiency on these resource-limited devices to ensure that they can run smoothly on devices with limited resources.
% For both HR and HRV monitoring, previous work has not compared the accuracy and model size between different ML models, which cannot provide sufficient insights on which ML model is suitable for edge devices. 
% Some of them used demeaning simple ML models. Some of them used ML models, but they did not evaluate the model size. 

%Third, HRV estimation from PPG is relatively less studied.
%To the best of our knowledge, there is no prior work that use PPG signals to directly estimate HRV - prior studies usually used PPG signal to estimate R peaks or RR intervals, which may not provide accurate readings when converting to HRV.
%Furthermore, there also lacks a dataset for HRV estimation study.
%As HRV is measured within a certain time window (usually more than 30 seconds \cite{acharya2006heart}), a long experiment duration is essential to acquire enough data for developing and evaluating HRV estimation techniques. 
%Giving consideration to that, the ISPC dataset is too short for HRV estimation since each recording is only about 5 minutes long.
% As an alternative, Photoplethysmography (PPG) sensor is an inexpensive optical measurement device that has been widely used in commercial wearable devices such as smartwatches to monitor HR \cite{castaneda2018review}, and can also be exploited to monitor HRV. 
% But PPG is suffering from motion artifacts (MA) and many other factors which degrade the signal quality and hinder the HR and HRV accuracy \cite{fine2021sources}. 
% Besides, most commercial HR and HRV monitoring devices do not provide an interface to access data for further analysis.
% Moreover, the frequency of HR and HRV results provided by most commercial devices is relatively low - it may take a few seconds or even minutes to update a data reading.
% But many applications require more timely data updates.
% For example, for driving fatigue detection, since a lot of things can happen in a minute when driving, the data may need to be updated every second or faster.
% In this work, to provide an accurate, real-time, low-cost, and sustainable system to monitor HR and HRV, we design an intelligent real-time HR and HRV monitoring system that use a PPG sensor powered with machine learning (ML). 


%Therefore, in this paper, to design an ML-based real-time HR and HRV monitoring system which utilizes a PPG sensor on a resource-constrained device, we devise a monitoring device with an off-the-shelf PPG sensor for customized data collection and investigate its performance in accuracy and efficiency of various ML models for HR/HRV prediction. 
% To the best of our knowledge, all existing works apply PPG light signals as ML features in HR and HRV prediction, which may result in large feature dimensions and thus large models.
% In this paper, we report the design of our HR/HRV monitoring solution using an off-the-shelf PPG sensor. Our system is specially designed for resource-constrained devices and addresses the above three limitations. To address the first two limitations, we combined signal processing and ML methods. That is, we first applied signal processing to generate rough HR/HRV estimations. Then these rough HRs/HRVs are passed through an ML model to generate more accurate HR/HRV estimations.
%to exploit the strengths of both, improve HR/HRV prediction accuracy and mitigate their weaknesses.
% On one hand, applying the ML model on signal-processing-generated HRs allowed $\mu$s to sample PPG signals at only 25 Hz while achieving higher accuracy because ML models can be trained specifically to improve accuracy with low-frequency samples. On the other hand, applying signal processing before ML eliminates the need for the ML model to directly take large numbers of PPG signals as inputs, reducing feature sizes and model sizes. 
%We use the signal-processed PPG HR (and PPG HRV) as ML features for HR (HRV) prediction to relax the high-frequency requirement - with a PPG sampling rate at 25 Hz, reduce the complexity of the signal processing algorithm, and in the meantime, greatly reduce the ML feature dimension to facilitate smaller ML model size.

% To address the ML model type issue of the second limitation, we compared the accuracy and model size of 5 different ML models, including Decision Tree (DT), Random Forest (RF), K-nearest neighbor (KNN), Support vector machines (SVM) and Multi-layer Perceptron (MLP), and provide insights on which are more suitable for the resource-constrained device.
% To address the third limitation, we estimated HRV directly from PPG signals and demonstrated that directly predicting HRV can provide higher accuracy than estimating RR interval or R peak.
% We also collected a new dataset for the HRV study which is long enough for HRV ML model training and testing.

% Experimental evaluations showed that our system achieved less than 5\% error for HR estimation and 10\% error for HRV estimation with a PPG sampling rate of 25 Hz. Moreover, our exploration showed that Decision Tree models usually could provide accurate estimation with low model sizes of at most 10 KB.

% The contributions of this paper include:
% In this work, we combine signal processing and ML - using signal-processed PPG HR (and PPG HRV) as ML features for HR (HRV) estimation, to reduce the feature dimension and contribute to smaller model size.
% Through combining signal processing with ML, we explore the advantage of both of them to provide high estimation accuracy and relieve their disadvantage of them, the high energy consumption for signal processing and the high memory requirement of ML.

% \eat{
%The main challenge of the system is to design a suitable ML model that can predict HR and HRV accurately from the noisy raw PPG data in a real-time manner.
%Therefore, there are two major components in the system, one is a PPG sensor component that collects raw PPG data continuously, and another is an edge device that focuses on processing the data and provides accurate HR and HRV in real-time.
% The PPG sensor is mounted on a fingertip and connected to the edge device. 
% We also deployed an ECG device to obtain HR and HRV as the ground truth.
%On the edge device, 4 modules are cooperating to provide accurate and real-time readings: firstly, a data collection module to collect the PPG reading, then a data preprocessing module to reduce noises, after that, an ML module to estimate HR and HRV, and finally, a report module to report HR and HRV to users, or save them to cloud for further analysis. 

%The key modules in terms of HR and HRV accuracy are the data preprocessing module and ML module.
%The less noise contained in the data entering the ML model, the better the model estimation result is.
%Therefore, for the data preprocessing module, we need to design an efficient preprocessing algorithm to reduce noise and generate better input for the ML model.
%As for the ML module, we need to find an efficient ML algorithm to accurately estimate HR and HRV.
%There are existing work that utilizes ML in HR or HRV monitoring, but this work has a few differences from them.
% Firstly, to the best of our knowledge, all existing works apply PPG light signals as ML features in HR and HRV estimation, while in this work, we use processed PPG HR (and PPG HRV) as ML features for HR (HRV) estimation.
% The advantage of doing this is that we can apply our system to further improve commercial device results. 
% Since most commercial devices do not provide PPG light signals to users, the existing methods are not able to be applied to improve them.
% Secondly, for HR, in addition to our collected data, we also used the ISPC dataset, and for the ISPC dataset, we combine both signal processing and ML in HR estimation, while other work either use signal processing method only or use a simple preprocess before sending the PPG data to the learning model.
% Thirdly, for HRV, to the best of our knowledge, all existing works utilize ML to estimate RR and then calculate HRV based on that, while in our work the ML model output HRV directly instead of RR. 
% As proved in section \ref{sec:method_hrv_vs_rr}, although predicted RR may have a small error, if we calculate HRV based on them, the error could grow and becomes large. 

% Another challenge of the system is to keep the system small and reliable to make sure that it can run smoothly with few resources.
% The edge device could be any smart device such as a smartphone or smart home gateway. 
% Since the resources on the edge device are limited, the ML model should not consume too many resources. 

% \begin{itemize}
% \item Novel HR and HRV estimation methodologies that combined both signal processing and ML models, allowing high-accuracy HR/HRV estimations with low-frequency PPG signals, fewer ML features, and smaller ML models. 
% %We are the first that combines signal processing with ML, using PPG HR and PPG HRV as ML features instead of PPG light signal, which achieves higher accuracy for both HR and HRV estimation. This combination also allowed
% %, MAPE less than 5\% for HR, and 10\% for HRV, and contributes to a
% %smaller feature dimension, which leads to smaller ML models. 

% \item %For HRV, we are the first to
% The first methodology to directly estimate HRV from PPG signals, which achieves higher accuracy than just estimating RR interval or R peaks. 
% %We demonstrate that it is more beneficial to estimate HRV directly because it is more accurate than predicting RR. 

% \item A systematic analysis of different ML model types and feature sizes to study their impact on the accuracy and model size in HR and HRV estimation. This analysis showed that DT can provide a good trade-off between accuracy and model size. %when the dimension of the feature is around 20 for HR and 300 for HRV.
% % We compared several machine learning models, including DT, RF, KNN, SVM, and MLP for accurate real-time HR and HRV monitoring. We evaluate their estimation accuracy and memory consumption in both HR and HRV estimation.

% \item A comprehensive dataset to study HR/HRV estimation with PPG signals, which contain different levels of activity intensity with each last for more than 2 hours. 
% %We devise a real-time HR and HRV monitoring system powered by PPG sensors and ML, and collect data in different settings.
% \end{itemize}


