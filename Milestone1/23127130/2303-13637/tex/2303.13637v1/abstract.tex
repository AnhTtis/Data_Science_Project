\begin{abstract}

Heart Rate Variability (HRV) measures the variation of the time between consecutive heartbeats %(known as RR intervals) 
and is a major indicator of physical and mental health. Recent research has demonstrated that photoplethysmography (PPG) sensors can be used to infer HRV. However, many prior studies had high errors because they only employed signal processing or machine learning (ML), or because they indirectly
inferred HRV, 
%through heartbeat intervals, 
or because there lacks large training datasets. 
Many prior studies may also require large ML models. 
The low accuracy and large model sizes limit their applications to small embedded devices and potential future use in healthcare.

%, limiting their applications to small embedded devices. Moreover, there is also a lack of large datasets for more comprehensive HRV studies.

To address the above issues, we first collected a large dataset of PPG signals and HRV ground truth. %Each trace of this dataset contains monitoring data for more than 2 hours. 
With this dataset, we developed HRV models that combine signal processing and ML to directly infer HRV. Evaluation results show that our method had errors between $3.5\%$ to $25.7\%$ and outperformed signal-processing-only and ML-only methods. We also explored different ML models, which showed that Decision Trees and Multi-level Perceptrons have $13.0\%$ and $9.1\%$ errors on average with models at most hundreds of KB and inference time less than 1ms. Hence, they are more suitable for small embedded devices and potentially enable the future use of PPG-based HRV monitoring in healthcare. 


%various machine learning algorithms, including Decision Tree (DT), Random Forest (RF), K-nearest neighbor (KNN), Support vector machines (SVM), and Multi-layer perceptron (MLP). For deployment in resource-constrained devices, we also explored the model design space to study the trade-off between estimation accuracy, model sizes, and inference time. Evaluation results show that the errors of different HRV models are between \TODO{$5\%$ to $25\%$}. It usually requires 5 minutes of PPG data as input data to achieve high accuracy. Moreover, DT models usually have high accuracy with the smallest model size, which can be $2$ to $4$ magnitude smaller than other models.  


%Many recent studies have exploited Photoplethysmography (PPG) sensors embedded in wearable devices to estimate heart rate (HR) and/or heart rate variability (HRV) with various techniques, where the main challenge comes from the varying accuracy due to noisy PPG data. In this work, focusing on different machine learning models, including Decision Tree (DT), Random Forest (RF), K-nearest neighbor (KNN), Support vector machines (SVM), and Multi-layer perceptron (MLP), we investigate their effectiveness in predicting HR and HRV using PPG sensing data. To deploy the models in resource-constrained wearable devices, we also study the trade-off between prediction accuracy and the size of resulting models\eat{ as well as the processing delay}. In addition to a widely-utilized ISPC dataset, we devised a device with an off-the-shelf PPG sensor to obtain a high-fidelity PPG dataset with a tunable setting. The evaluation results show that, for HR predictions, based on both ISPC and our customized datasets, all the considered learning models can accurately predict with  {\em mean average prediction errors (MAPE)} being around $5\%$, which is comparable with the results reported in the prior work. For HRV, the MAPEs of different learning models are in the range of $5\%$ to $11\%$, where DT and MLP perform relatively better. Moreover, the DT models have the smallest size for different settings, which can be $2$ to $4$ magnitude smaller than other models. 


% Photoplethysmography (PPG) is a widely deployed sensor in wearable devices which used to observe HR and can also be exploited to monitor HRV. 
% However, the accuracy of PPG is suffering from motion artifacts. 
% Besides, a system that simultaneously monitors HR and HRV in real-time is a challenge. 
% In this paper, we design a real-time HR and HRV monitor framework that uses PPG and machine learning to monitor HR and HRV simultaneously in real-time. 
% We install the PPG on the fingertips and apply machine learning to improve the PPG HR/HRV accuracy.
% We also installed a three-lead ECG as ground truth.
% Compared with the ground truth, the system achieves xxx.
% Therefore, we proved the effectiveness of the proposed system and the feasibility of applying machine learning in real-time HR and HRV monitoring. 


\end{abstract}
