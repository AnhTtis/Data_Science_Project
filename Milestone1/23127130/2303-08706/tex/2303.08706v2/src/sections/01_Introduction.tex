\section{Introduction}\label{section:Introduction}
\glspl{cps} integrate real-time computing and communication capabilities with monitoring and control actions over components in the physical world~\cite{shi_survey_2011}. To face the harshness of the space environment, modern space systems such as satellites and spacecraft require tight coupling between onboard processing, communication (cyber), sensing, and actuation (physical) elements~\cite{klesh_cyber-physical_2012}. The orbit determination and control subsystems on a small spacecraft or in satellites' constellations provide a clear link between onboard processing and sensing elements of the spacecraft's physical environment~\cite{di_mascio_-board_2021}, becoming increasingly critical as small spacecraft become ever more capable~\cite{klesh_cyber-physical_2012}.
Radiation-induced soft errors such as \glspl{set} and \glspl{seu} can occur more frequently in space than at ground level, creating the need for additional hardware to mitigate detrimental effects on the system~\cite{wachter_survey_2019}.
In this scenario, digital computing systems representing the decisional part of a \gls{scps} must be designed to be reliable and tolerate faults induced by cosmic radiation.

Various solutions exist to protect electronics from the adverse effects of radiation~\cite{wachter_survey_2019}. \Glspl{scps} typically rely on \glsreset{rhbd}\gls{rhbd} techniques to add reliability at the technology level, yet scaling of these approaches lags behind the scaling of their commercial counterparts, significantly impacting \gls{ppa} of these designs. Insulating techniques~\cite{alles_radiation_2011}, and polymer shielding~\cite{shahzad_views_2022} can also help mitigate soft errors. Furthermore, it is also possible to enhance the fault tolerance capabilities of digital systems by introducing redundancy at different levels in their design flow. Temporal redundancy techniques rely on repeated executions of the same work to determine the correct result~\cite{feng_shoestring_2010}. Spatial or modular redundancy techniques rely on multiple hardware blocks executing the same task and comparing the results~\cite{ginosar_survey_2012}. These approaches rely on rigid schemes for repetition in space and time of redundant blocks or tasks. Hence, they can severely impact the \gls{ppa} of computing platforms. This leads to a significant gap between \glspl{soc} for space and their commercial counterparts, which make use of technological advances unencumbered by radiation-induced faults.

Therefore, the increasing demand for strong processing capabilities in space~\cite{xie_resource-cost-aware_2018} is pushing researchers toward lower-overhead solutions and use of more advanced technologies to close this gap in \gls{ppa}. In recent years, the advent of RISC-V and open-source hardware has encouraged the development of high-performance \glspl{soc} for various domains without licensing or other restrictions. This includes the space domain, where custom modifications to improve properties such as reliability~\cite{di_mascio_open-source_2021} and fault tolerance are often required. Among proposed architectures, heterogeneous systems with multi-core computing clusters have gained traction in the space industry~\cite{ginosar_rc64_2016} due to increased performance and flexibility for computation and \gls{dsp} workloads~\cite{kurth_hero_2018}.
While multiple processors offer increased performance for parallelizable tasks, they also provide a unique opportunity for reliability enhancements: multiple cores can execute identical tasks, comparing their results to detect and react to faults.

% \todo{Add Trikarenos-style info on technology scaling \& rad-hard cells. From Trikarenos: To achieve the increasing computational requirements within the reduced power budgets in CubeSats and nano satellites, denser technologies for on-board processors are required. While commercial technologies have been agressively scaled to improve performance and power efficiency, \glspl{soc} for space lag behind with their scaling, relying on older, larger technology nodes~\cite{hillman_space_2003} and custom \glsreset{rhbd}\gls{rhbd} techniques. Scaling \gls{rhbd} standard cells and tuning \glspl{pdk} of smaller technology nodes is often too costly and requires too much time for the limited quantity of these designs. However, some of these smaller technologies, such as TSMC 28nm, have shown tolerance to the destructive effects of radiation ~\cite{borghello_total_2023}, still requiring tolerance to \glspl{seu} both within sequential cells and due to transient effects in combinatorial logic~\cite{di_mascio_open-source_2021}. One of the most effective and affordable methods to implement \gls{seu} tolerance in modern designs is through architectural modifications, which allows for integration of redundancy in modern commercial technology nodes, providing high-performance and efficient \glspl{soc}.}
% \todo{TODO: Expand more on the gaps in research that we are targeting with this paper. I think we can focus on the performance gap between reliable SoCs to their commercial counterparts, using sota commercial technologies. With this approach, the commercial technologies can be used as-is without requiring significant additional work to harden their technology to ratiation.}

In this paper, we introduce a multi-core RISC-V-based computing system for space featuring an innovative \gls{hmr} approach. We leverage the independent cores available in a multi-core RISC-V cluster built on a commercial technology for redundant execution in a dynamically runtime configurable manner, supporting on-demand reconfiguration in \gls{dcls} and \gls{tcls} modes. This extends the \gls{odrg} presented in~\cite{rogenmoser_-demand_2022}, which presents a \gls{tcls} configurable cluster with software-based recovery and configuration prior to startup.
Our design extends this by allowing each application to configure its reliability setting according to its requirements, possibly decided at runtime.
% Without sacrificing performance in the general case, this architecture allows for the safe execution of a safety-critical section at a 2.3\% area overhead.
Furthermore, we implement two recovery alternatives, software and hardware-assisted, comparing their impact on the hardware resources and performance in case of a fault. The checking, voting, and switching hardware in the implemented design does not affect the internals of the processor core, allowing for the use of verified RISC-V processor cores without requiring any internal (potentially erroneous) modifications to rapidly build a reliable system. What we propose is, to the best of our knowledge, the most compact, flexible, and configurable computing cluster offering the best trade-off between performance and reliability.

% The key contributions of this paper are:
% \begin{itemize}
%     \item Combined Dual Core Lockstep and Triple Core Lockstep extensions within a RISC-V-based multi-core cluster.
%     \item Design of a split-lock mechanism to enable runtime-selectable redundant configurations switching in just 550 clock cycles between the available redundant modes. 
%     % \item Exploring the trade-offs for \gls{dmr} and \gls{tmr} on a core-level perspective of the implemented cluster and relating this to classical redundancy mechanisms.
%     \item Design of hardware extensions for fast fault recovery in just 24 clock cycles with only $\sim9\%$ area impact and no timing and performance impacts over the original architecture.
% \end{itemize}
% We propose the first system integrating these functionalities on an open-source RISC-V-based multi-core cluster for fine-tunable reliability vs. performance trade-offs. The key contributions of this paper are:
Besides extending our previous work~\cite{rogenmoser_-demand_2022}, we introduce the following novel contributions:
\begin{itemize}
    \item A re-configurable computing cluster for Dual-Core Lockstep and Triple-Core Lockstep execution capable of tackling compute-intensive and safety-critical applications at the same time. The proposed cluster can be configured so that the computing cores can operate independently  during high-performance execution or in Dual/Triple-Core Lockstep mode in case of safety-critical tasks.
    \item Robust hardware extension for fast fault recovery. We introduced dedicated Error-Correcting Codes-protected status registers to restore the state of the computing cores to the closest reliable state in time. This feature allows the cores to perform cycle-by-cycle backups of their internal state in the protected registers, reducing by $15\times$ the required time to recover from a fault  over the software-based approach.
    \item A novel runtime-programmable split-lock mechanism, allowing for fast switching and re-configuration between the available redundant modes. With these features, it is possible to explicitly define portions of code within a \textit{mission-critical section}, configuring the cores for safe lockstep execution, or within a \textit{performance section}, disabling the lockstep execution temporarily, with minimum configuration switching overhead.
\end{itemize}

To validate our proposed approach, we implemented the RISC-V cluster in Global Foundries 22~\si{\nano\meter} technology, achieving up to \SI{430}{\mega\hertz} operating frequency and \SI{1160}{\mega OPS} when configured in independent mode and 617 and 414 MOPS in \gls{dmr} and \gls{tmr} mode, respectively. With only software-based recovery features, the proposed cluster occupies \SI{0.612}{\milli\meter\squared} with just 1.3\% area overhead over the non-redundant configuration, featuring 363 clock cycles time-to-recovery in triple mode. When enhanced with hardware-based recovery features, it provides rapid fault recovery in just 24 clock cycles occupying \SI{0.660}{\milli\meter\squared}, $\sim$9.4\% area overhead over the baseline RISC-V cluster. The proposed split-lock mechanism allows for entering and exiting a redundancy mode in <400 clock cycles for mission-critical code execution, or <200 exiting and re-entering the redundancy mode for temporarily more performance.
To foster future research in computer architecture for space, we release the proposed architecture as the first fully open-source~\footnote{https://github.com/pulp-platform/redundancy\_cells} RISC-V-based multi-core cluster with a finely tunable trade-off between reliability and performance.
