% Michael Rogenmoser
\par\noindent
\parbox[t]{\linewidth}{
\noindent{\includegraphics[height=1.5in,width=1in,clip,keepaspectratio]{src/bios/michael_rog.png}
}
\noindent {Michael Rogenmoser}
Michael's fanciest bio.}
\vspace{4\baselineskip}

% Yvan Tortorella
\par\noindent 
\parbox[t]{\linewidth}{
\noindent{\includegraphics[height=1.5in,width=1in,clip,keepaspectratio]{src/bios/yvan.jpg}}
\noindent { Yvan Tortorella}
received his Master's Degree in Electronic Engineering in October 2021 from the University of Bologna. He is currently pursuing a Ph. D. in Digital Systems Design in the group of Professor Luca Benini at the Department of Electrical and Information Engineering (DEI) of the University of Bologna. His research interests include the design of PULP (Parallel Ultra-Low Power)-based hardware accelerators for ultra-low power Machine Learning and the design of RISC-V-based computer architectures for satellite applications.}
\vspace{4\baselineskip}

% Davide Rossi
\par\noindent 
\parbox[t]{\linewidth}{
\noindent{\includegraphics[height=1.5in,width=1in,clip,keepaspectratio]{src/bios/rossi.png}}
\noindent {Davide Rossi}
received the Ph.D. degree from the University of Bologna, Bologna, Italy, in 2012. He has been a Post-Doctoral Researcher with the Department of Electrical, Electronic and Information Engineering “Guglielmo Marconi,” University of Bologna, since 2015, where he is currently an Associate Professor. His research interests focus on energy-efficient digital architectures. In this field, he has published more than 100 papers in international peer-reviewed conferences and journals. He is recipient of Donald O. Pederson Best Paper Award 2018, 2020 IEEE TCAS Darlington Best Paper Award, 2020 IEEE TVLSI Prize Paper Award.}
\vspace{4\baselineskip}

% Francesco Conti
\par\noindent 
\parbox[t]{\linewidth}{
\noindent{\includegraphics[height=1.5in,width=1in,clip,keepaspectratio]{src/bios/fconti_new.jpg}}
\noindent {Francesco Conti}
received the Ph.D. degree in electronic engineering from the University of Bologna, Italy, in 2016. He is currently a Tenure-Track Assistant Professor in the DEI Department of the University of Bologna. From 2016 to 2020, he held a research grant in the DEI department of University of Bologna and a position as postdoctoral researcher at the Integrated Systems Laboratory of ETH Zurich in the Digital Systems group. His research focuses on the development of deep learning based intelligence on top of ultra-low power, ultra-energy efficient programmable Systems-on-Chip. His research work has resulted in more than 40 publications in international conferences and journals and has been awarded several times, including the 2020 IEEE TCAS-I Darlington Best Paper Award.}
\vspace{4\baselineskip}

% Luca Benini
\par\noindent 
\parbox[t]{\linewidth}{
\noindent{\includegraphics[height=1.5in,width=1in,clip,keepaspectratio]{src/bios/benini.jpg}}
\noindent {Luca Benini}
holds the chair of digital Circuits and systems at ETHZ and is Full Professor at the Università di Bologna. He received a PhD from Stanford University. Dr. Benini's research interests are in energy-efficient parallel computing systems, smart sensing micro-systems and machine learning hardware. He has published more than 1000 peer-reviewed papers and five books. He is a Fellow of the IEEE, of the ACM and a member of the Academia Europaea. He received the IEEE Mac Van Valkenburg award in 2016 and the ACM/IEEE A. Richard Newton Award in 2020.}
\vspace{4\baselineskip}