\section{Introduction}\label{section:Introduction}
\glspl{cps} integrate real-time computing and communication capabilities with monitoring and control actions over components in the physical world~\cite{shi_survey_2011}. To face the harshness of the space environment, modern space systems such as satellites and spacecraft require tight coupling between onboard processing, communication (cyber), sensing, and actuation (physical) elements~\cite{klesh_cyber-physical_2012}. The orbit determination and control subsystems on a small spacecraft or in satellites' constellations provide a clear link between onboard processing and sensing elements of the spacecraft's physical environment~\cite{di_mascio_-board_2021}, becoming increasingly critical as small spacecrafts become ever more capable~\cite{klesh_cyber-physical_2012}. In this scenario, digital computing systems representing the decisional part of a \gls{scps} must be designed to be reliable and tolerate faults induced by cosmic radiation. Radiation-induced soft errors such as \glspl{set} and \glspl{seu} can occur more frequently in space than at ground level, creating the need for additional hardware to mitigate detrimental effects on the system~\cite{wachter_survey_2019}.

Various solutions exist to protect electronics from the adverse effects of radiation~\cite{wachter_survey_2019}. Costly radiation-hardened technologies, insulating techniques~\cite{alles_radiation_2011}, and polymer shielding~\cite{shahzad_views_2022} help mitigate soft errors. It is also possible to enhance the fault tolerance capabilities of digital systems by introducing redundancy at different levels in their design flow. Temporal redundancy techniques rely on repeated executions of the same work to determine the correct result~\cite{feng_shoestring_2010}. Spatial or modular redundancy techniques rely on multiple hardware blocks executing the same task and comparing the results~\cite{ginosar_survey_2012}. These approaches rely on rigid schemes for repetition in space and time of redundant blocks or tasks, hence they can severely impact the \gls{ppa} of computing platforms.

The increasing demand for strong processing capabilities in space~\cite{xie_resource-cost-aware_2018} is pushing researchers toward lower-overhead solutions. In recent years, the advent of RISC-V and open-source hardware has encouraged the development of high-performance \glspl{soc} for various domains without licensing or other restrictions. This includes the space domain, where custom modifications to improve properties such as reliability~\cite{di_mascio_open-source_2021} and fault tolerance are often required. Among proposed architectures, heterogeneous systems with multi-core computing clusters have gained traction in the space industry~\cite{ginosar_rc64_2016} due to increased performance and flexibility for computation and \gls{dsp} workloads~\cite{kurth_hero_2018}.
While multiple processors offer increased performance for parallelizable tasks, they also provide a unique opportunity for reliability enhancements: multiple cores can execute identical tasks, comparing their results to detect and react to faults.

In this paper, we introduce a space-ready multi-core RISC-V-based computing system featuring a \gls{hmr} approach. We leverage the independent cores available in a multi-core RISC-V cluster for redundant execution in a dynamically runtime configurable manner and introduce \gls{dcls} and \gls{tcls} modes, extending the On-Demand Redundancy Grouping with \gls{tcls} configurable under reset presented in~\cite{rogenmoser_-demand_2022}.
Our design allows each application to configure its reliability setting according to its requirements, possibly decided at runtime.
% Without sacrificing performance in the general case, this architecture allows for the safe execution of a safety-critical section at a 2.3\% area overhead.
Furthermore, we implement two recovery alternatives, software and hardware-assisted, comparing their impact on the hardware resources and performance in case of a fault. The checking, voting, and switching hardware in the implemented design does not affect the internals of the processor core, allowing for the use of verified RISC-V processor cores without requiring any internal (potentially erroneous) modifications to rapidly build a reliable system.

% The key contributions of this paper are:
% \begin{itemize}
%     \item Combined Dual Core Lockstep and Triple Core Lockstep extensions within a RISC-V-based multi-core cluster.
%     \item Design of a split-lock mechanism to enable runtime-selectable redundant configurations switching in just 550 clock cycles between the available redundant modes. 
%     % \item Exploring the trade-offs for \gls{dmr} and \gls{tmr} on a core-level perspective of the implemented cluster and relating this to classical redundancy mechanisms.
%     \item Design of hardware extensions for fast fault recovery in just 24 clock cycles with only $\sim9\%$ area impact and no timing and performance impacts over the original architecture.
% \end{itemize}
% We propose the first system integrating these functionalities on an open-source RISC-V-based multi-core cluster for fine-tunable reliability vs. performance trade-offs. The key contributions of this paper are:
In summary, we introduce the following key contributions:
\begin{itemize}
    \item A re-configurable computing cluster for Dual-Core Lockstep and Triple-Core Lockstep execution capable of tackling compute-intensive and safety-critical applications. The proposed cluster can be configured so that the computing cores can operate independently if the application requires high-performance capabilities or in Dual/Triple-Core Lockstep mode, depending on the criticality of the executed task.
    \item Robust hardware support for fast fault recovery execution, featuring dedicated Error-Correcting Codes-protected registers to restore the state of the computing cores to the closest reliable state in time. This feature allows the cores to perform cycle-by-cycle backups of their internal state in the protected registers, reducing by $15\times$ the required time to recover from a fault  over the software-based approach.
    \item A runtime-programmable split-lock mechanism allowing for fast switching and re-configuration between the available redundant modes. With these features, it is possible to explicitly define portions of code within a \textit{safety-critical section}, configuring the cores for safe lockstep execution with minimum configuration switching overhead.
\end{itemize}

To validate our proposed approach, we implemented the RISC-V cluster in Global Foundries 22~\si{\nano\meter} technology, achieving up to \SI{430}{\mega\hertz} operating frequency and \SI{1160}{\mega OPS} when configured in independent mode and 617 and 414 MOPS in \gls{dmr} and \gls{tmr} mode, respectively. With only software-based recovery features, the proposed cluster occupies \SI{0.612}{\milli\meter\squared} with just 1.3\% area overhead over the non-redundant configuration, featuring 363 clock cycles time-to-recovery in triple mode. When enhanced with hardware-based recovery features, it provides rapid fault recovery in just 24 clock cycles occupying \SI{0.660}{\milli\meter\squared}, $\sim$9.4\% area overhead over the baseline RISC-V cluster. The proposed split-lock mechanism allows for entering and exiting a redundant mode in just 390 clock cycles for safety-critical code execution.
To foster future research in space-ready computer architecture, we release the proposed architecture as the first fully open-source RISC-V-based multi-core cluster with a finely tunable trade-off between reliability and performance.