\documentclass[10pt,conference,hidelinks]{IEEEtran}

\usepackage{url}
\usepackage{subfig}
\usepackage{graphicx}
\usepackage{amsmath,amssymb}
\usepackage{hyperref}
%\usepackage{subcaption}
\usepackage{multirow}
\usepackage{listings}
\usepackage{tabularx}
\usepackage{xcolor}
\usepackage{comment}
\usepackage{listings,newtxtt}
\lstset{basicstyle=\ttfamily, keywordstyle=\bfseries}

\usepackage{tikz}
\usepackage{xcolor}
\newcommand*\circled[1]{\tikz[baseline=(char.base)]{
            \node[shape=circle,fill,inner sep=1pt] (char) {\textcolor{white}{#1}};}}


\usepackage{pbox}
\usepackage{comment}
\usepackage{csvsimple}
\usepackage{fancyhdr}
\usepackage{mathtools}
\usepackage{nicefrac}
\usepackage{caption}
\usepackage{amsmath}
\usepackage{makecell}
\usepackage{pifont}
\usepackage{tikz}
\def\checkmark{\tikz\fill[scale=0.4](0,.35) -- (.25,0) -- (1,.7) -- (.25,.15) -- cycle;}
%\usepackage[top=1in, bottom=1in, left=0.75in, right=0.75in]{geometry}
\usepackage{siunitx}
\sisetup{
    table-figures-decimal = 3,
    table-number-alignment=center-decimal-marker
}

\newcommand\Ayaz[1]{\textcolor{blue}{#1}}
\newcommand\note[1]{\textcolor{blue}{#1}}
\newcommand\Maryam[1]{\textcolor{red}{#1}}

%\pagenumbering{ roman }

\begin{document}

% Use these macros to cite our tools, etc.
% This ensures consistency.
\newcommand{\gem}{gem{5}}
\newcommand{\bear}{BEAR-Wr-Opt}
\newcommand{\oracle}{Oracle}

\title{Enabling Design Space Exploration of DRAM Caches in Emerging Memory Systems}

\author{
%\normalsize{Note: This work is submitted to ISPASS 2023 and is under review  -- This is a \emph{confidential draft} -- Do not distribute.}\\
\\\IEEEauthorblockN{Maryam Babaie, Ayaz Akram, Jason Lowe-Power}
\IEEEauthorblockA{Department of Computer Science,
University of California, Davis\\
Email: \{mbabaie, yazakram, jlowepower\}@ucdavis.edu}}

\maketitle
\thispagestyle{plain}
\pagestyle{plain}

\begin{abstract}
%%%% OLD VERSION
    % The increasing growth of applications' memory demands has led the CPU vendors to deploy large DRAM caches,
    % backed by large non-volatile memories like Intel Optane (e.g., Intel's Cascade Lake and Sapphire Rapids).
    % The existing memory simulators and computer architecture simulators do not provide support to
    % model and evaluate systems which use DRAM devices as a cache to the main memory.
    % In this work, we present a cycle-level DRAM cache model which is integrated with \gem{}.
    % % a general DRAM cache protocol for the aforementioned systems.
    % % We extend gem5 (widely used full-system simulator) with a unified DRAM cache and main memory controller
    % % (\textit{UDCC}) to implement the DRAM cache protocol. \textit{UDCC} controls two different memory devices at the same time,
    % % in which one of them caches data for the other one and data is kept consistent between the two.
    % This model leverages the flexibility of \gem{}'s memory devices models and full system support to enable  exploration of many different DRAM cache designs.
    % We demonstrate the usefulness of this new model by exploring the design space of a DRAM cache controller through several case studies including the impact of
    % scheduling policies, required buffering, combining different memory technologies (e.g., HBM, DDR3/4/5, 3DXPoint, High latency)
    % as the cache and main memory, and the effect of wear-leveling when DRAM cache is backed by NVM main memory.
    % We also perform experiments with real workloads in full-system simulations to validate the proposed model
    % and show the sensitivity of these workloads to the DRAM cache sizes.
    % % We address the complexities involved in timing and micro-architecture details of \textit{UDCC} that are vital to support the data
    % % consistency across DRAM cache and main memory and reason about the performance degradation caused by these complexities.
The increasing growth of applications' memory capacity and performance demands has led the CPU vendors to deploy heterogeneous memory systems either within a single system or via disaggregation.
For instance, systems like Intel's Knights Landing and Sapphire Rapids can be configured to use high bandwidth memory as a cache to main memory.
While there is significant research investigating the designs of DRAM caches, there has been little research investigating DRAM caches from a full system point of view, because
% As these new technologies emerge, further studies are required to support DRAM caches in these systems.
there is not a suitable model available to the community to accurately study large-scale systems with DRAM caches at a cycle-level.
In this work we describe a new cycle-level DRAM cache model in the \gem{} simulator which can be used for heterogeneous and disaggregated systems.
We believe this model enables the community to perform a design space exploration for future generation of memory systems supporting DRAM caches.


\end{abstract}

% \begin{IEEEkeywords}
% Heterogeneous Memories, Disaggregated Memories, DRAM Cache, \gem{}.
% \end{IEEEkeywords}


\section{Introduction}

The last decade has seen significant academic research on DRAM caches, and today
these ideas are becoming a reality with CPU vendors implementing DRAM cache-based computer systems, e.g., Intel's Cascade Lake and Sapphire Rapids.
Hardware-managed DRAM caches are seen as one way to enable heterogeneous memory systems (e.g., systems with DRAM and non-volatile memory) to be more easily programmable.
DRAM caches are transparent to the programmer and easier to use than manual data movement.

However, recent work has shown that these transparent hardware-based data movement designs are much less efficient than manual data movement~\cite{hildebrand2021case}.
While the work by Hildebrand et al.~\cite{hildebrand2021case} and other recent work investigating Intel's Cascade Lake systems provides some insight into real implementations on DRAM caches~\cite{izraelevitz2019basic,wang2020characterizing}, there is a gap in the community's access to cycle-level simulation models for DRAM caches.
This paper describes a new \gem{}-based model of a unified DRAM cache controller inspired by the Cascade Lake hardware to fill this gap.

Previous work has explored many aspects of DRAM cache design in simulation such as the replacement policy, caching granularity~\cite{qureshi2012fundamental,jevdjic2013stacked}, dram cache tag placement~\cite{huang2014atcache,loh2012supporting,loh2011efficiently}, associativity~\cite{qureshi2012fundamental,kotra2018chameleon,young2018accord}, and other metadata to improve performance~\cite{loh2011efficiently,jevdjic2013stacked,young2018accord}.
These mostly high-level memory system design investigations can appropriately be evaluated with trace-based or non-cycle-level simulation.
However, as shown in recent work, the micro-architecture of the unified DRAM and non-volatile main memory (NVRAM) controller can lead to unexpected performance pathologies not captured in these prior works (e.g., Hildebrand et al. showed that a dirty miss to the DRAM cache requires up to \emph{five accesses} to memory~\cite{hildebrand2021case}).

Thus, to better understand these realistic DRAM cache systems, it is imperative to build a detailed DRAM cache simulation model which can be used to perform a design space exploration around the DRAM cache idea.
The previous research works on DRAM cache design improvements do not provide any (open-source) DRAM cache modeling platform for a detailed micro-architectural and timing analysis.
To the best of our knowledge, most research works do not consider systems where the hardware-managed DRAM cache and NVRAM are sharing the same physical interface and are controlled by a unified memory controller (as is the case in real platforms like Intel Cascade Lake).

In this work, we describe our unified DRAM cache and main memory controller (\textit{UDCC}) cycle-level DRAM cache model for \gem{}~\cite{lowepower2020gem5}.
The protocol takes inspiration from the actual hardware providing DRAM cache, such as Intel's Cascade Lake, in which an NVRAM accompanies a DRAM cache as the off-chip main memory sharing the same bus.
To model such hardware, we leverage the cycle-level DRAM~\cite{hansson2014simulating} and NVRAM~\cite{gem5-workshop-presentation} models in \gem{}.
Our model implements the timing and micro-architectural details enforced by the memory interfaces including the DRAM timing constraints, scheduling policies, buffer sizes, and internal queues.
We propose a DRAM cache model that is direct-mapped, insert-on-miss, and write-back to model Intel's Cascade Lake design.
% FUTURE WORK: though in our model these policy decisions are parameterized \note{IS THIS TRUE???}.

Using this model, we present validation data and investigate five case studies.

\emph{What is the impact of memory scheduling policies in a unified DRAM cache and memory controller?}
We find that using FR-FCFS is highly impactful when the cache hit ratio is high, but less so when the hit ratio is low and the NVRAM's bandwidth limits performance.

\emph{What is the impact of DRAM technology on performance and memory controller architecture?}
We find that higher performing memory technologies require more buffering to achieve peak performance. 
Moreover, we find that the composition of the memory access patterns and their hit/miss ratio on DRAM cache, 
can also affect the amount of buffering to achieve the peak bandwidth.

\emph{What is the impact of backing ``main memory'' performance?}
We find that while slower backing memory hurts performance, the performance of the backing memory does not have a significant affect on the micro-architecture of the cache controller.

\emph{What is the impact of the UDCC model for full-system applications?}
We find that our model shows similar performance characteristics on real applications as previously shown on real hardware providing further evidence for the importance of cycle-level simulation.

\emph{What is the impact of NVRAM wear leveling on memory system performance with a DRAM cache?}
We find that while wear leveling has a very small direct impact, the impact when using NVRAM as backing memory with a DRAM cache can be much higher. Although only 1 in 14,000 requests experience a wear-leveling event, the performance impact is up to an 8\% slowdown.

Our model is open-source and publicly available for the use of research community~\cite{dcacheGem5Code} and will be integrated into \gem{} mainstream. Using this new model which implements the micro-architectural details of realistic DRAM caches on a simulator can help find any potential improvement for the next generation of memory systems.

%The rest of the paper is organized as follows...

%\section{Background on Network Calculus}
\label{sec: background}


\begin{figure*}[tbh]
\centering
\begin{subfigure}[b]{0.3\textwidth}
    \centering
    \includegraphics[width=\linewidth]{images/in-out.png}
    \caption{Arrival and departure data and their relation with delay $d(t)$ and backlog $b(t)$. For a FIFO system, the delay is the horizontal distance between $R(t)$ and $R^*(t)$ but some other multiplexing techniques may shift the data to a later priority, causing a longer delay.}
    \label{fig: data in-out}
\end{subfigure}
\hfill
\begin{subfigure}[b]{0.35\textwidth}
    \centering
    \includegraphics[width=\linewidth]{images/arrival-service.png}
    \caption{Characteristics of an arrival curve and a service curve. From any point of observation, the arriving data never exceeds its arrival curve; the departure data is also never less than the service curve with respect to the data arrival.}
    \label{fig: arrival-service curves}
\end{subfigure}
\hfill
\begin{subfigure}[b]{0.33\textwidth}
    \centering
    \includegraphics[width=\linewidth]{images/bound.png}
    \caption{Delay and backlog bounds of a system. Backlog is the maximum vertical distance between $\alpha(t)$ and $\beta(t)$; FIFO delay is their maximum horizontal distance; but for arbitrary multiplexing, the delay guarantee is when the system clears its buffer, thus it's the intersection of $\alpha(t)$ and $\beta(t)$.}
    \label{fig: system bounds}
\end{subfigure}
\caption{Network calculus framework. We let $R(t)$ and $R^*(t)$ be the arrival and departure data flow of a system; $\alpha(t)$ be the piecewise linear concave arrival curve and $\beta(t)$ be the piecewise linear convex service curve of a system.}
% \hossein{Better to show piece-wise linear concave arrival curve and piece-wise linear convex service curve instead of token-bucket and rate-latency.}}
\end{figure*}

We recall some of the network calculus essentials for a better understanding of the framework used in Saihu. In the following context, we use the following notation: $\mbb{R}^+$ is the set of non-negative real numbers; $[x]_+$ denotes $\max(0, x)$

The data flow is by convention modeled as a left-continuous wide-sense increasing function $R(t): \mbb{R}^+ \mapsto \mbb{R}^+$ with respect to time $t$~\cite{ncbook2001leboudec}. 

A system $\mcal{S}$ receives arrival data described as a cumulative function $R(t)$ and delivers departure data as another cumulative function $R^*(t)$. Figure~\ref{fig: data in-out} illustrates such a system $\mcal{S}$. The benefit of representing a system like this is that we can observe system backlog and delay with such a model. 

\begin{definition}[Backlog and Delay~\cite{ncbook2001leboudec}]
    The backlog of a system at time~$t$ is
    \begin{equation}
        b(t) = R(t) - R^*(t)
    \end{equation}
    
    The virtual delay of a FIFO system at time $t$ is
    \begin{equation}
        d_{FIFO}(t) = \inf \lbp \tau \geq 0 : R(t) \leq R^*(t+\tau) \rbp
    \end{equation}
\end{definition}



The backlog of a system can be viewed as the vertical distance between $R$ and $R^*$. The FIFO (\textit{First-in First-out}) delay is the horizontal distance between $R$ and $R^*$. One may obtain other delay values if the multiplexing technique is not FIFO.

% \begin{figure}
%     \centering
%     \includegraphics[width=0.9\linewidth]{images/in-out.png}
%     \caption{In/out data flow; delay and backlog}
%     \label{fig: data in-out}
% \end{figure}

Since we are interested in the system guarantee instead of a single instance of data flow, we would like to have general bounds to the arrival and departure data flows. Therefore, we define \textit{arrival curve} and \textit{service curve} as the bounds of arrival and departure data flows.

\begin{definition}[Arrival Curve~\cite{ncbook2001leboudec}]
    Given a wide-sense increasing function $\alpha: \mbb{R}^+ \mapsto \mbb{R}^+$, we say that a flow $R(t)$ is $\alpha$-constrained if and only if for all $s \leq t$:
    \begin{equation}
        R(t) - R(s) \leq \alpha(t-s)
    \end{equation}
    We say $R(t)$ has $\alpha$ as an arrival curve.
\end{definition}

\begin{definition}[Service Curve~\cite{ncbook2001leboudec}]
    Given a wide-sense increasing function $\beta: \mbb{R}^+ \mapsto \mbb{R}^+$ and $\beta(0) = 0$. A system $\mcal{S}$ having $R(t)$ and $R^*(t)$ as its arrival and departure flows. We say $\mcal{S}$ offers a service curve $\beta$ if and only if
    \begin{equation}
        R^*(t) \geq (R \otimes \beta)(t) =: \inf_{s \leq t} \lbp R(s) + \beta(t-s) \rbp
    \end{equation}
    where $\otimes$ denotes the min-plus convolution
\end{definition}

Figure~\ref{fig: arrival-service curves} illustrates the arrival and service curves. Any segment of arrival flow $R(t)$ is constrained by arrival curve $\alpha$ and the output curve $R^*(t)$ is always no less than the curve $R\otimes\beta$. As a result, an arrival curve upper bounds the incoming traffic, and a service curve lower bounds the outgoing traffic.

% \begin{figure}
%     \centering
%     \includegraphics[width=\linewidth]{images/arrival-service.png}
%     \caption{Arrival/Service curve}
%     \label{fig: arrival-service curves}
% \end{figure}

We consider 2 special types of curves throughout this paper, \textit{token-bucket} (or sometimes called \textit{leaky-bucket}) curve and \textit{rate-Latency} curve.

\begin{definition}[Token-bucket and Rate-latency~\cite{ncbook2001leboudec}]
    A token-bucket curve $\gamma_{r,b}$ with arrival rate $r$ and burst $b$ is defined as
    \begin{equation}
        \gamma_{r,b}(t) = b + rt
    \end{equation}

    A rate-latency curve $\beta_{R,T}$ with service rate $R$ and latency $T$ is defined as
    \begin{equation}
        \beta_{R,T}(t) = R \lb t - T \rb_+
    \end{equation}
\end{definition}

A token-bucket curve is determined by a burst $b$ and an arrival rate~$r$. Burst represents the maximum possible data volume that can arrive simultaneously, and arrival rate represents the maximum long-term data rate~\cite{bouillard2022tradeoff}.
A rate-latency curve is determined by a latency~$T$ and a service rate~$R$. Latency represents the time a server needs before starting to process the incoming data, and service rate represents the minimum rate to process data after the initial latency.

With the help of arrival and service curves, we can derive delay and backlog bounds for a system $\mcal{S}$ illustrated in Figure~\ref{fig: system bounds}. Suppose a system $\mcal{S}$ has arrival curve $\alpha$ and service curve~$\beta$, its worst-case backlog $b^*$ is the maximum vertical distance between~$\alpha$ and~$\beta$. Similarly, depending on the multiplexing technique applied to the system, its worst-case delay bound $d^*$ is the maximum horizontal distance between $\alpha$ and $\beta$ if $\mcal{S}$ is a FIFO system. If we don't have any information about its multiplexing technique, referred to as arbitrary multiplexing, the best we can say is that when $\alpha$ and $\beta$ intersect each other, where all data has been delivered out of the system. Consequently, the worst-case delay bound for arbitrary multiplexing is the time required for $\mcal{S}$ to clear its buffer.

% \begin{figure}
%     \centering
%     \includegraphics[width=\linewidth]{images/bound.png}
%     \caption{System delay/backlog bounds}
%     \label{fig: system bounds}
% \end{figure}

While a service curve captures the slowest possible output speed of a system, a link's transmission capacity limits the speed as well. Hence, we model this phenomenon using a \textit{greedy shaper} with a sub-additive function $\sigma: \mbb{R}^+ \mapsto \mbb{R}^+$ concatenated with a server. We consider a concatenation as shown in Figure \ref{fig: system}. By convention we assume $\sigma(0) = 0$ and $\beta(t) \leq \sigma(t), \forall t \in \mbb{R}^+$, meaning that the buffer is cleared at the beginning and the service never exceed its physical limitation. With the above definition, such greedy shaper conserves the service provided by the system due to theorem \ref{thm: shaping}.

\begin{figure}[thb]
    \centering
    \includegraphics[width=0.7\linewidth]{images/system.png}
    \caption{Shaping of departure data. A flow that has an arrival curve $\alpha$ feeds into a server with an arrival data flow $R(t)$. The server having service curve $\beta$ takes $R(t)$ and gives a departure data flow $R^*(t)$ to a shaper with shaping function $\sigma$. The shaper takes $R^*(t)$ and shape the data flow as another departure $D(t)$.}
    \label{fig: system}
\end{figure}


\begin{theorem}[Shaping conserves service \cite{ncbook2001leboudec}]
\label{thm: shaping}
Following the system shown in Figure \ref{fig: system}, we have
\begin{equation}
     D = R^* \otimes \sigma \geq \lp R \otimes \beta \rp \otimes \sigma = R \otimes \lp \beta \otimes \sigma \rp = R \otimes \beta
\end{equation}
\end{theorem}

In the following context, we model the shaping function $\sigma$ as a token-bucket curve $\gamma_{C,L}$ with transmission capacity $C$ and the packet size $L$ to capture the link capacity and packetization~\cite{bouillard2022tradeoff}.

\section{Design of DRAM cache model} \label{design}
% We strive to support a generic DRAM cache model. For this purpose, our design models two main scenarios which are flexible enough to cover a set of different DRAM cache designs:

We strive to support a generic DRAM cache model. For this purpose, we implement a discrete DRAM cache model which is flexible enough to cover a set of different DRAM cache designs.
This model relies on a separate DRAM cache manager which interacts with a near/fast/local memory controller (for a DRAM cache) and a far/slow/remote memory controller (for the backing store). This model does not require DRAM cache and the backing store to share a data bus.

A prior work proposed a unified DRAM cache controller to model the hardware of Intel's Cascade Lake~\cite{babaie2022cycle}.
It implements only one DRAM cache architecture where a DDR4 and NVM memory devices as the cache and the main memory are controlled by the same controller and share the data bus. This model is not flexible and generic to simulate different DRAM cache designs.
Our model in this work is able to cover this case. Since the discrete DRAM cache model is more flexible and can model future systems with DRAM cache designs, we will focus on that model in this paper.

% \begin{itemize}
% \item Discrete DRAM cache model: This model relies on a separate DRAM cache manager which interacts with a near/fast/local memory controller (for a DRAM cache) and a far/slow/remote memory controller (for the backing store). This model does not require DRAM cache and the backing store to share a data bus.
% \item Unified DRAM cache model: This is a model proposed by a prior work which uses a unified DRAM cache controller. It implements only one caching policy and also controls a DDR4 and NVM memory devices as the cache and the main memory while sharing the data bus~\cite{babaie2022cycle}.
% \end{itemize}

\subsection{Background on \gem{}'s memory subsystem}

The \gem{} simulator is based on an event-driven full-system simulation engine.
It supports models of many system components, including memory controllers, memory device models, CPUs, and others.
The original memory controller module added to \gem{} by Hansson et al.~\cite{hansson2014simulating} is a \emph{cycle level} memory controller model designed to enable fast and accurate memory system exploration. The memory controller in \gem{} was refactored in~\cite{gem5-workshop-presentation} where two components were defined to simulate any controller.
(1) The \textit{memory controller} receives commands from the CPU and enqueues them into appropriate queues and manages their scheduling to the memory device.
(2) The \textit{memory interface} deals with device-specific timings and operations and communicates with the memory controller.
Moreover, the most recent release of \gem{} further improved the flexibility and modularity of memory controller and interfaces~\cite{akram2022modeling} and added HBM2 controller and interfaces.
Most importantly, it provides support for HBM2 interface and memory controller, where each physical channel is consisted of two pseudo-channels with peak theoretical bandwidth of 32 GB/s per channel.
The DRAM cache model presented in this work can employ any of the memory controller and interface model of \gem{} without any modifications.
Like \gem{}'s current memory controller, our DRAM cache model's goal is for cycle-level simulation to enable micro-architectural exploration and flexibility, not cycle-by-cycle accuracy to a single design.

\subsection{DRAM Cache Model}

Figure~\ref{fig:dcache} shows an overview of the DRAM cache model we implement in this work. DRAM cache manager receives all the incoming memory traffic (coming from the CPU, LLC, DMA).
This cache manager is not in the on-chip coherence domain and can be a drop-in replacement for the memory controller.

The DRAM cache manager is responsible for implementing different DRAM cache policies and interacts with two controllers.
In this work, near memory refers to the memory device which acts as a cache of the far memory (or the backing store).
The DRAM cache manager sends requests to and receives responses from the near and far memories. These requests include reads and writes
to the near and far memories and receiving a response for read requests and an acknowledgement for the write requests.
We allow the use of any memory controller in \gem{} as local and far memory controllers.
The \gem{} memory controllers take care of all the device-specific timing control and the DRAM cache manager allow us to isolate all DRAM cache specific controls.
This isolation makes the model modular and generic to implement different DRAM cache policies and architectures, as shown in the three case studies.

Our goal is to keep our DRAM cache model flexible enough to be able to simulate different DRAM cache designs.
Therefore, instead of modeling one specific micro-architecture with buffers for every device, we abstract away the hardware resources required to implement a DRAM cache controller by the use of two simple buffers: \textit{Outstanding Requests Buffer (ORB)} and \textit{Conflicting Requests Buffer (CRB)} shown in Figure~\ref{fig:dcache}.
All incoming memory requests reside in the \textit{ORB} unless a request conflicts with an already existing request in the \textit{ORB}.
Two requests are conflicting if they both map to the same location in the DRAM cache.
The conflicting request goes to the \textit{CRB} until the request it was conflicting with has been serviced and is taken out of the \textit{ORB}.
Each entry in these buffers contains other metadata in addition to the address of the request, as shown in Figure~\ref{fig:dcache}.
This metadata provides helpful information about the request, e.g., current state, and relative arrival time.
We also model a \textit{Write Back (WB) Buffer} for the DRAM cache dirty lines that are to be written back to the backing store.
What each entry in these buffers holds depend on the DRAM cache architecture and our model is flexible.


  % \begin{figure*}
  % \centering
  % \scriptsize{Unified cache controller layout}{
  %   \fbox{\includegraphics[scale=0.6]{figures/simulatedHW.jpg}}
  %   \label{fig:buffers}
  % }
  %  \scriptsize{State machine of unified cache controller}{
  %   \fbox{\includegraphics[scale=0.5]{figures/protocol.jpg}}
  %   \label{fig:StateMachine}
  % }
  % \caption{Unified DRAM cache controller design}
  % \label{fig:DController}
  % \end{figure*}


% \begin{figure}
%   \centering
%   \includegraphics[scale=0.45]{figures/simulatedHW.jpg}
%   \vspace{-1ex}
%   \caption{Hardware abstraction of \textit{UDCC}. (\Ayaz{Need to update the figure}) (\Maryam{Do we need to show the hardware and buffers?}). These buffers are implemented in the DRAM cache manager or the unified DRAM cache controller.}
%   \label{fig:buffers}
% \end{figure}


% \begin{figure}
%   \centering
%   \includegraphics[scale=0.4]{figures/design.jpg}
%   \vspace{-1ex}
%   \caption{Overview of the discrete DRAM cache model. DRAM cache manager is responsible for implementing different DRAM caching policies and interacting with memory controllers of near and far memory.}
%   \label{fig:dcache}
% \end{figure}

% \begin{figure}
%   \centering
%   \includegraphics[scale=0.4]{figures/state_machine.jpg}
%   \vspace{-1ex}
%   \caption{Example of a state machine diagram implemented by the DRAM cache manager.}
%   \label{fig:StateMachine}
% \end{figure}

\begin{figure}
  \centering
  \includegraphics[width=0.7\linewidth]{figures/hardware.pdf}
  \vspace{-1ex}
  \caption{Overview of DRAM cache manager hardware.
  DRAM cache manager implements different DRAM caching policies and interacts with memory controllers of DRAM cache and its backing store.
  It includes three main buffers: (i) Outstanding Requests Buffer (ORB), (ii) Conflicting Requests Buffer (CRB), (iii) Write Back (WB) Buffer.
  }
  \label{fig:dcache}
  \vspace{-1.5em}
\end{figure}

Figure~\ref{fig:dcache} also shows a state machine and a memory request goes through different steps while in the DRAM cache manager.
These steps depend on the DRAM cache design and the state machine contains the logic for implementing them.
Our model uses a tag and metadata storage in the DRAM cache manager to track the status of the DRAM cache locations.
This separate storage allows flexible implementation of different DRAM cache designs.
The model assures the timing implementation of the architecture as if this storage does not exist. Thus, the performance of each design is captured.

The steps in Figure~\ref{fig:dcache} represents an example of a DRAM cache design which is inspired by the real DRAM
cache implemented by Intel's Cascade Lake.
%For a different design, the state machine should be modified and our model comes with APIs to handle this.
%\Maryam{come back here if you have time}
For the baseline architecture, we implement a DRAM cache which is direct-mapped (with caching granularity of 64 bytes), inserts on misses, and writes back the dirty cache lines upon evictions.
The tag and metadata are stored in ECC bits alongside the data.% in the same cache line.

Since \gem{} is an event driven simulator, the simulation model relies on scheduled events to transition between different states.
Below is an example of the state machine shown in Figure~\ref{fig:dcache}:

\paragraph*{Initializing a Request}
\textcircled{1} The CPU package (cores or LLC) sends a request to the main memory.
\textcircled{2} The DRAM cache manager (as the replacement for the memory controller) receives this request and places it in the ORB (if not conflicting with an existing request in the ORB). In case of writes, it sends an acknowledgement to the CPU/LLC.
\textcircled{3} If a conflict exists, the DRAM cache manager holds it in the CRB until the conflicting request leaves ORB.

\paragraph*{Request to the Local Memory}
\textcircled{4} Based on a scheduling policy (such as First-Come First-Serve) the DRAM cache manager sends a read request to the DRAM cache controller for tag check, and it receives
the response whenever it is ready. This response will provide the whole cache line that the demand access maps to in the DRAM cache, including data, tag and metadata.
If the tag matches, it is a hit, otherwise it is a miss on DRAM cache.
In case of read demand hit, the DRAM cache manager sends the response to the CPU/LLC.
In case of write demand hit or miss, the DRAM cache manager sends a write request with the data from demand access to the DRAM cache controller.
\textcircled{5} If any of the misses had a dirty flag set in the metadata, the DRAM cache manager inserts that cache line into the WB buffer.
Whenever there is bandwidth available or if the WB buffer becomes full, the DRAM cache manager sends these write backs as a write request to the main memory controller.

\paragraph*{Request to the Remote Memory}
\textcircled{6} In case of read demand miss, the DRAM cache manager sends a read request to the main memory controller to fetch the missing cache line from backing store.
Once the DRAM cache manager receives the response from main memory controller, it finishes the miss handling process.
First, \textcircled{7} it sends a write request to the DRAM cache controller to fill the missing cache line.
Second, \textcircled{8} it sends the response for the demand to CPU/LLC.

% \noindent
% \textbf{Receive Request}: All memory requests (from the CPU package) are received by the DRAM cache manager and placed in the \textit{ORB} (if not conflicting with an existing request in the \textit{ORB}).
%
% \noindent
% \textbf{Local DRAM Read:} Since every request must first check the tag in the DRAM cache, a memory packet moves to the \textit{DRAM Read} state and is sent to the local memory controller. Local memory controller will perform the DRAM read operation and send the response back to the DRAM cache manager. This operation is not needed if the designed DRAM cache model does not have the tag-store in the DRAM itself.
%
% \noindent
% \textbf{DRAM Read Response:} Once the response of a DRAM (cache) read is received (this operation models a tag read), the DRAM cache manager checks the DRAM cache tags to ascertain the hit/miss and clean/dirty status of the request and perform needed actions accordingly.
% For example, in case of read request if the tag check shows a hit there is no need to perform any other operation as the DRAM read would have brought the correct data along-with the tags.
%
% \noindent
% \textbf{Far Mem Read:} If the tag check by the DRAM cache manager indicates a DRAM cache miss for a read or a write request (if allocate on a write-miss policy is used), the DRAM cache manager sends a read request to the far memory controller. If the original request was a read request, on getting the response back from the far memory, the response is also sent to the requestor.
%
% \noindent
% \textbf{Local DRAM Write:} DRAM cache manager sends a local DRAM write request in multiple cases. For example, if the tag read for a DRAM cache write request indicates a hit, the cache manager sends the write request to local DRAM controller to perform actual data write. Similarly, in case of a DRAM cache miss and then a read from the far memory, DRAM cache manager sends a write request to local memory controller to perform cache fill operation.
%
% \noindent
% \textbf{Far Mem Write:} DRAM cache manager sends write requests to far memory controller only if there is a dirty miss in the DRAM cache. This operation leads to writing of dirty dirty line is written back to the far memory.
%
% \noindent
% \textbf{Done:}  When a memory request is fully serviced, it moves to the \textit{Done} state and the request is eventually removed from \textit{ORB}.

% \subsection{Unified DRAM cache model}
% Unified DRAM cache is largely based on Intel Cascade Lake's DRAM cache management strategy.
% This model uses a unified DRAM cache controller which manages both local and far memory devices/interfaces.
% This model does not require separate memory controllers but is less flexible.

% \subsection{Evaluation procedure}
\label{sec_validation}
\newcommand{\iterDayCmp}{j}
\newcommand{\iterDayExp}{i}
\newcommand{\iterDay}{i}
The objective of the evaluation procedure is to answer whether the new price- and forecast-aware controller saves money when compared to the existing benchmark controller described in Section \ref{sec_test_house}. The key performance indicator is daily cost given the weather conditions. It is inherently difficult to benchmark and validate the performance of a controller operating in a complex environment with many uncontrollable external factors such as weather and occupant activities. Further, the long time-constants play a significant role by demanding long test periods. Ideally, the benchmark  and \MPC-controller should be run in parallel on exact copies of the same building placed at the same location, with occupants doing the same activities. Although, some buildings support such circumstances, this can obviously not be asked of the occupants. Instead, a benchmark \dataSet\ from the same house is used for the evaluation. The benchmark \dataSet\ is based on data collected from the former heating period (2021-2022) where the original benchmark controller was operating. The data is sorted into full days creating a collection of comparison days $\daySetCmp$, seen in \eqref{eq_days}, from which appropriate subsets can be selected. The daily generated data on set form is:
% \begin{align}
%     \label{eq_days}
%     \daily^i = \{\Eg^\iterDay, \Ta^\iterDay, \priceElecBuySeti{\iterDay}, \Epv^\iterDay \in \realv{\Nday}\}, \hspace{0.1cm} \daily^i \in \mathcal{D}
% \end{align}
\newcommand{\iterDayOne}{n}
\begin{align}
    \label{eq_days}
    &\mathcal{D}_\compare = \left\{day^\iterDayOne = \left(\Eg^\iterDayOne, \Ta^\iterDayOne, \priceElecBuySeti{\iterDayOne},\Epv^\iterDayOne  \right)\right\}\\
    &\iterDayOne = 1,\dots,\Ncmp, \hspace{4mm} \Eg^\iterDayOne, \Ta^\iterDayOne, \priceElecBuySeti{\iterDayOne},\Epv^\iterDayOne \in \realv{\Nday}
    %\daily^i = \{\Eg^\iterDay, \Ta^\iterDay, \priceElecBuySeti{\iterDay}, \Epv^\iterDay \in \realv{\Nday}\}, \hspace{0.1cm} \daily^i \in \mathcal{D}
\end{align}
with $\Eg^\iterDay$ and $\Epv^\iterDay$ being the electricity consumption from grid and production from \pv\ in \si{\kWh} during day $i$, respectively, $\Ta^\iterDay$ the ambient temperature, $\priceElecBuySeti{\iterDay}$ the hourly electricity price for day $i$, and $\Nday = 24$. Note that benchmark days where the system has been manipulated or a significant amount of data is missing are dropped to minimise pollution of the results. A similar data collection, $\daySetExp$, is generated from the experiment period. %To evaluate daily controller performance a subset of days, $\daySetCmp^\iterDayExp \subset \daySetCmp$, with similar average ambient temperature and sun irradiation are drawn from the comparison collection $\daySetCmp$ for each experiment day $\iterDayExp$:
The \MPC-controller is evaluated daily by comparing the operation cost of day $\iterDayExp$ to a subset of benchmark days, $\daySetCmp^\iterDayExp \subset \daySetCmp$, drawn from the full benchmark \dataSet. The subset, $\daySetCmp^\iterDayExp$, is drawn according to the following rule:
% \begin{align}
%     \label{eq_subset_cmp_days}
%     \mathcal{D}_{\compare}^\iterDayExp &= \{day^\iterDayCmp \hspace{1mm}\vert\hspace{1mm}  \nonumber\\ &-\Delta\avgTai{\dn} \leq \avgTaCmp^{\iterDayCmp} - \avgTaExp^\iterDayExp \leq \Delta\avgTai{\up}, \nonumber\\
%     &-\Delta\Epvi{\dn}  \leq \Sigma \EpvCmp^{\iterDayCmp}-\Sigma\EpvExp^\iterDayExp \leq \Delta\Epvi{\up} \nonumber \\
%     &day^\iterDayCmp \in \daySetCmp \}
% \end{align}
\begin{align}
    \label{eq_subset_cmp_days}
    \mathcal{D}_{\compare}^\iterDayExp &= \{day \hspace{1mm}\vert\hspace{1mm}  \nonumber\\ &-\Delta\avgTai{\dn} \leq \avgTaCmp - \avgTaExp^\iterDayExp \leq \Delta\avgTai{\up}, \nonumber\\
    &-\Delta\Epvi{\dn}  \leq \Sigma \EpvCmp-\Sigma\EpvExp^\iterDayExp \leq \Delta\Epvi{\up}, \nonumber \\
    & \avgTaCmp,\Sigma \EpvCmp \in day \in \daySetCmp \}
\end{align}
with $\avgTa$, $\Sigma \Epv$ being average ambient temperature and accumulated electricity production from \pv, respectively. The constants $\Delta\avgTai{\dn}$ and $\Delta\avgTai{\up}$ are the down- and up-search range for ambient temperature, respectively. Similar, $\Delta\Epvi{\dn}$, $\Delta\Epvi{\up}$ makes out the search-range for accumulated electricity produced by the \pv. Here the \pv\ is used as an indicator for sun radiation. This is not a perfect indicator, since the sun altitude and intensity vary with the seasons, thereby creating a bias. However, it is found to be a good indicator for dealing with cloud conditions on-site, since it directly measures the level of shadow on the building. With ambient temperature and sun irradiation accounted for, factors such as occupant behavior and previous day heating patterns are left out. This undeniably causes noise, making the electricity consumption of the \hp\ distribute randomly for any given day. To decrease the influence of the noise, the controller is run over a long period to obtain more consistent results.

We calculate a virtual cost for benchmark day $\iterDayCmp$, with respect to experiment day $\iterDayExp$,
\begin{align}
    \label{eq_cost_comp}
    &\costElecCmp^\iterDayCmp = \sum_{k = 0}^{\Nday} \priceElecBuySeti{\iterDay}(k)\Eg^\iterDayCmp(k)\nonumber \\  &\priceElecBuySeti{\iterDay} \in \dailyExp^\iterDayExp, \hspace{0.4cm} \Eg^\iterDayCmp \in \dailyCmp^\iterDayCmp \in \daySetCmp^\iterDayExp
\end{align}
% \begin{align}
%     \label{eq_cost_comp}
%     \costElecCmp^\iterDayCmp = &\sum_{k = 0}^{\Nday-1} \priceElecBuySeti{\iterDay}(k)\Eg(k) \nonumber \\ &  \priceElecBuySeti{\iterDay} \in \dailyExp^\iterDayExp,\hspace{0.1cm} \Eg \in \dailyCmp \in \daySetCmp^\iterDayExp
% \end{align}
It simply means that electricity consumption from similar benchmark days are imposed onto the price of the experiment day to calculate the virtual cost. This provides a plausible alternate outcome for the case where the benchmark controller had been running instead. This is done since the benchmark controller is price ignorant and thereby acts independently of the price. This manoeuvre would not be possible if the comparison was between two price-aware controllers. In that case price curves would have to be accounted for as well. The cost of the experiment day $i$, $\costElecExp^\iterDayExp$, is of course calculated using the actual electricity consumption for the day.
\section{Methodology}
\label{method}

% We explain our simulation methodology for evaluating DRAM cache designs using our DRAM cache model.

\begin{figure}
  \centering
  \subfloat[Target AMD EPYC-like system.]{
    \centering
  \includegraphics[width=0.44\linewidth]{figures/whole-system}
  \label{fig:whole-system}
  }
  \hfill
  \subfloat[A $\nicefrac{1}{8}$ system used in simulation. The system used in each case study is shown.]{
    \centering
  \includegraphics[width=0.48\linewidth]{figures/case-studies}
  \label{fig:case-studies}
  }
  \caption{(a) Target system: AMD EPYC-like system, which contains eight cores per CCD. CCDs are connected to the eight main memory channels via an I/O die.
  The memory system contains an HBM stack which can act as a cache to the 8-channel DDR4-based main memory.
  We model $\nicefrac{1}{8}$ as shown on the right.}
  \label{fig:system}
  \vspace{-1.5em}
\end{figure}


\subsection{Modeled System}
Figure~\ref{fig:whole-system} shows the target system we simulated for our experiments.
We consider $\nicefrac{1}{8}$ of a single socket of an AMD EPYC-like system~\cite{naffziger2021pioneering} which consists of a single
core complex die (CCD) and a single channel of main memory, as shown in Figure~\ref{fig:case-studies} on the top. The modeled CCD contains eight cores that have private L1 caches and share the last level cache among the cores.
The main reason for our decision to model $\nicefrac{1}{8}$ of the system is that with the current core and on-chip cache models of \gem{}, it will take a long simulation time to model the entire system in a simulated environment.

To incorporate our DRAM cache model with this system, we replace the memory controller with a DRAM cache manager,
as shown in Figure~\ref{fig:case-studies} on the bottom. For the baseline, the cache manager uses a single channel of HBM2
(which consists of two pseudo-channels and acts as a DRAM cache), and a single channel of DDR4 (which serves as the main memory in the system).
Note that in these figures the memory interfaces and their controllers are shown in the same box. Table \ref{tab:baseConfig} summarizes the baseline system configuration.

% The type of memory technologies for DRAM cache and its backing store can change based on the experiment. Moreover, the connection between
% the cache manager and DRAM cache and the backing store has a configurable latency. In case studies 1 and 2, we assumed no extra latency for this connection.
% In case study 3, we varied the latency of the backing store's connection as the remote memory.

\begin{table}[!h]
  \begin{center}
  \caption{\label{tab:baseConfig}Baseline System Configuration}
  \begin{tabular}{ |p{3.5cm}||p{2.5cm}|}
    \hline
    \multicolumn{2}{|c|}{Processors} \\
    \hline
    Number of cores & 8\\
    Frequency & 5 GHz\\
    \hline
    \multicolumn{2}{|c|}{On-chip Caches} \\
    \hline
    Private L1 Instruction & 32 KB\\
    Private L1 Data & 512 KB\\
    Shared L2 & 8 MB\\
    \hline
    \multicolumn{2}{|c|}{DRAM Cache Manager} \\
    \hline
    ORB & 128 entries\\
    CRB & 32 entries\\
    WB Buffer & 64 entries\\
    Frontend/Backend Latencies& 20 ns round-trip\\
    \hline
    \multicolumn{2}{|c|}{DRAM Cache (HBM2)} \\
    \hline
    Capacity & 128 MB\\
    Theoretical Peak Bandwidth & 32 GB/s\\
    Read/Write Buffer & 64 entries each\\
    \hline
    \multicolumn{2}{|c|}{Main Memory (DDR4/NVM)} \\
    \hline
    Capacity & 3 GB\\
    Theoretical Peak Bandwidth & 19.2 GB/s\\
    Read/Write Buffer & 64 entries each\\
    \hline
  \end{tabular}
  \end{center}
  \end{table}

\subsection{Workloads used}
We evaluated DRAM caches using a subset of multithreaded workloads in NPB~\cite{bailey1991parallel} and GAPBS~\cite{beamer2015gap} that assess high-performance systems.
We use the C class of the NPB workloads and a synthetic graph as an input (size of $2^{22}$ vertices) for the GAPBS workloads.
These workloads' working-set sizes varies between few hundreds MB to 1 GB. Thus, we set the size of the DRAM cache and the main memory to 128 MB and 3 GB, respectively, so the DRAM cache is smaller than the workloads' memory footprints.

\subsection{Simulation methodology for benchmarks}
Figure~\ref{fig:method} shows our methodology for the experiments we ran in this paper.
First, Linux kernel boots on the target system in \gem{} and the execution of the program starts and continues until the start of the region of interest (ROI) of the workload. The benchmarks are marked with ROI begin markers using \gem{} pseudo instruction support.
Beginning at the ROI, we simulate the workload for 100ms to warm up the system including the DRAM cache. At the end of 100ms, we take a checkpoint. This process is done once per workload. Later, we restore from the checkpoint to run all our simulations with different DRAM cache configurations.
Using a checkpoint ensures that all of our runs have the same starting point with the same system state for a fair comparison across different tested configurations.
% Without the checkpoints, mainly because we simulate multithreaded workloads on a multicore system, the starting system state for the detailed simulation can be different across different runs.
We simulate the restored checkpoint for either one second of simulation time or reaching the end of ROI, whichever comes first.
Our results show an average cold-miss ratio of 3.5\% for DRAM cache, during the restore simulation.

Figure~\ref{fig:trafGen} demonstrates a validation of our DRAM cache model. We used a traffic generator to replace LLC/CPU side in Figure~\ref{fig:dcache}. This traffic generator creates synthetic memory traffic configurable for read/write percentage, random/linear pattern, etc. We used a single physical channel of HBM2 (32 GB/s peak bandwidth) as the DRAM cache and a single channel of DDR4 (19.2 GB/s peak bandwidth) as the main memory. We controlled the miss ratio,
percentage of read requests, and the ratio of dirty or clean cache lines in case of misses. We ran the tests for 10 ms, enough to reach the saturated bandwidth in all cases.
%was enough for all cases to reach the saturated bandwidth and
%keeping the pipeline of the controllers full.
For read-only (RO) traffic with a 100\% hit ratio, the DRAM cache should perform the same as the main memory, which is observed in Figure~\ref{fig:trafGen}. DRAM cache shows effective traffic of 29.94 GB/s (similar to \gem{} HBM2 memory controller effective bandwidth).
%We observe in Figure~\ref{fig:trafGen} that for this case, the effective traffic is
%29.94 GB/s. The HBM2 memory controller of \gem{} also provides the same utilized bandwidth
%for a similar pattern.
As we add more writes to the accesses and test with 67\% read (33\% write) and 100\% write patterns, the effective bandwidth drops as write hits require two accesses to the DRAM cache.
The effective bandwidth drops further for the 100\% miss ratio due to more extra accesses each demand request needs to make to the DRAM cache and the backing store.
%As we incorporate more write accesses to the 100\% hit ratio,
%the effective bandwidth drops for 67\% read (the remaining 33\% are writes) and write-only (WO)
%since write hits require two accesses from the DRAM cache. For 100\% miss ratios in all six cases,
%the effective traffic drops due to the extra accesses each request needs to do in the DRAM cache and
%the backing store.
For the Miss-Dirty case, there is a write-back to the main memory. However,
the main memory controller has enough bandwidth to handle these writes while the DRAM cache
pipeline is saturated.
% Thus, these writes are not on the critical path for this test.
%\Maryam{this needs be read by another reader too}

\begin{figure}
    \centering
    \frame{\includegraphics[width=0.7\linewidth]{figures/methodology.pdf}}
    \caption{Summary of the simulation methodology. Ckpt: checkpoint, ROI: region of interest. We use a warm-up period of 100ms and detailed simulation time of 1s.}
    \label{fig:method}
    \vspace{-1.5em}
\end{figure}

\begin{figure}
  \centering
  \includegraphics[width=0.8\linewidth]{figures/traf_gen.pdf}

  \caption{Effective traffic at the LLC. Synthetic traffic was injected to the DRAM cache manager.
  The traffic was controlled for percentage of read/writes, miss ratio, and clean or dirty line eviction ratio.}
  \label{fig:trafGen}
  \vspace{-1.5em}
\end{figure}

% \subsection{Traffic generator based studies}
% In addition to the use of benchmarks or real workloads, we rely on \gem{}'s traffic generators to create synthetic traffic patterns.
% Using traffic generators allows us to explore the behavior of the DRAM cache design more directly and more clearly understand the fundamental trade-offs in its design.
% We use \gem{}'s traffic generator to generate two patterns of addresses: linear or random.





\section{Case Study 1: Performance of Baseline Cache}
\label{cs1}

% In this section, we use our simulation model to evaluate the performance of DRAM cache systems.
% We simulate the target system from Figure~\ref{fig:case-studies} for this study, where a single channel of HBM2 DRAM cache
% is backed up by a single channel of DDR4 main memory.

In this case study, we will investigate the performance and different pathologies memory requests go through, in the baseline DRAM cache design.
Specifically, we ask \emph{how a DRAM cache system's performance compares to those without a DRAM cache.} For this purpose, we consider the simulated system described in Section~\ref{method} in two different configurations: 1) HBM2 as a DRAM cache with DDR4 as main memory, and 2) no DRAM cache and DDR4 as main memory.
We run all workloads in these two configurations.

Typically, we would expect an HBM2-based DRAM cache to perform better, specifically in comparison to DDR4 main memory, because of its higher bandwidth. This case study considers the scenario where there is no additional latency, and the main memory is close to the local memory (e.g., connected to same package).
The similar latency of HBM2 and DDR4 might result in low-performance improvements with the HBM2 DRAM cache. In Section~\ref{sec:cs3}, we will evaluate a more realistic scenario of a high latency to main memory (e.g., disaggregated systems).
We expect that the DRAM caches perform similarly to a DRAM main memory if the working set of the workload fits in the cache (or has a high DRAM cache hit rate). Therefore, rather than focus on the obvious case, we evaluate the performance of the DRAM cache-based system when the workload does not fit in the DRAM cache. We assess the performance of selected workloads on the target system discussed in Section~\ref{method} to accomplish the previously mentioned goal.
We compare the amount of work each configuration has done during its execution to evaluate its performance.

\begin{figure}

    \subfloat[Performance]{%
      \includegraphics[clip,width=\linewidth]{figures/cs1_all_bips.pdf}%
      \vspace{-3ex}
      \label{fig:cs1Bips}
    } \\
    \subfloat[Speedup]{%
      \includegraphics[clip,width=\linewidth]{figures/cs1_all_spu.pdf}%
      \label{fig:cs1Sup}
    }
    \caption{Performance comparison of the baseline DRAM cache to the same system without the DRAM cache, based on billion instructions per second (BIPS). DRAM cache based system always performs less than the system without DRAM cache.
    \string*$bt$ is excluded from this analysis, as it started a different phase of program execution.}
    \label{fig:cs1BipsSu}
    \vspace{-1.5em}

\end{figure}

% \begin{figure}
%     \centering
%     \includegraphics[width=\linewidth]{figures/cs1_all_bips.pdf}
%     \vspace{-1ex}
%     \caption{Performance comparison of the baseline DRAM cache to two systems without the DRAM cache, based on billion instructions per second (BIPS). DRAM cache based system always perform worse than systems without DRAM cache.}
%     \label{fig:cs1Bips}
% \end{figure}
%
% \begin{figure}
%     \centering
%     \includegraphics[width=\linewidth]{figures/cs1_all_spu.pdf}
%     \vspace{-1ex}
%     \caption{Speedup the baseline DRAM cache compared to the system with DDR4 main memory only.
%     The DRAM cache system shows performance degradation compared to the system without the DRAM cache.}
%     \label{fig:cs1Sup}
% \end{figure}

\begin{figure}
    \centering
    \includegraphics[width=\linewidth]{figures/cs1_all_miss_ratio.pdf}
    \vspace{-1ex}
    \caption{Miss ratio of different workloads on DRAM cache. Miss ratios range
    from 20\% to 100\% approximately.}
    \label{fig:cs1MissRatio}
    \vspace{-1.5em}
\end{figure}

\subsection*{Results and Discussions}
Figure~\ref{fig:cs1Bips} shows the performance for NPB and GAPBS workloads in the two configurations we described above. We use a metric of billion instructions per second (BIPS) to represent performance.
Figure~\ref{fig:cs1Sup} shows the speed-up achieved with a DRAM cache-based system compared to a system without a DRAM cache (and DDR4 main memory only).
We expect DRAM caches to perform better than a far memory alone.
However, as Figure~\ref{fig:cs1Sup} shows, DRAM cache configuration performs worse than just a DDR4 main memory for all these tests (speed-up is consistently below 1).
We observe that the DRAM caches start to perform poorly as the DRAM cache miss rate is above 20\% (in the workloads we ran). Figure~\ref{fig:cs1MissRatio} shows the miss ratio of DRAM cache per workload.
We define the miss ratio as the total number of miss accesses to the DRAM cache divided by the sum of total miss and hit accesses.
As shown in Figure~\ref{fig:cs1MissRatio}, GAPBS workloads have miss ratios ranging from 20\% to 50\%, and NPB has miss ratios mostly around 50\%. These miss ratios are high but not unreasonable, and we can expect real-life workloads on high-performance systems to show such miss ratios~\cite{hildebrand2021case}.
As described in Section~\ref{design}, miss handling in current hardware-managed DRAM caches can lead to multiple accesses to the cache (local) and backing store (far) interfaces.

This mostly serialized process cannot fully utilize the available bandwidth at the local and far interfaces and
affects to the overall system performance by increasing the latency.
~The interference between these extra accesses and the actual demand access leads to performance degradation, and the previous
DRAM cache studies were not able to capture it. Note that we made sure that the DRAM cache had been warmed-up, so cold misses
are not contributing to the performance observed from the system.
Note: $bt$ in NPB is an outlier in this evaluation because our analysis shows it started another phase of execution. Thus, we exclude it from this analysis.

In addition to the total number of DRAM cache misses, we must look at the types of misses to fully explain the DRAM cache slowdown.
Figure~\ref{fig:cs1Mpki} shows the misses per thousand instruction (MPKI), and the breakdown of the misses into different types in terms of
read or write request and whether the cache location that the request mapped to was clean or dirty.
The type of DRAM cache miss determines how many extra accesses or operations need to be performed for the demand access under consideration.
For example, $tc$ and $bc$ in Figure~\ref{fig:cs1Sup} show the highest slowdown among the GAPBS workloads as they have the highest fraction of dirty misses (Figure~\ref{fig:cs1Mpki}). Similar behavior is true for $sp$ from NPB benchmarks.
$pr$ from GAPBS and $cg$ from NPB do not have a high fraction of dirty cache misses but still show significant DRAM cache slowdown as the total number of misses demonstrated by these benchmarks is relatively high.
Figure~\ref{fig:cs1Hpki} shows hits per thousand instructions (HPKI) and the read/write distribution of DRAM cache hits.
Since write hits have one more access to the DRAM than read hits, even write hits can cause performance degradation. For example, $tc$ and $bc$ from GAPBS have a high fraction of write hits contributing to their large slowdowns.

Figure~\ref{fig:cs1BwUtil} shows the bandwidth utilization per workload for each memory interface (local and far memory). The HBM2 DRAM cache
interface and the DDR4 backing store interface of \gem{} used in this study have a theoretical peak bandwidth of 32 GB/s and 19.2 GB/s, respectively.
Figure~\ref{fig:cs1BwUtil} shows that for all the cases, a significant amount of bandwidth remains unused for both memory interfaces. The reason for low utilization is the latency cost because of extra memory accesses needed in case of a DRAM cache miss (for both read and write), or a DRAM cache write hit.

\begin{figure}
    \centering
    \includegraphics[width=\linewidth]{figures/cs1_all_mpki.pdf}
    \vspace{-1ex}
    \caption{DRAM cache misses per thousand instructions for GAPBS and NPB.
    The figure also shows the distribution of different miss cases out of the total number of miss accesses.}
    \label{fig:cs1Mpki}
    \vspace{-1.5em}
\end{figure}

\begin{figure}
    \centering
    \includegraphics[width=\linewidth]{figures/cs1_all_hpki.pdf}
    \vspace{-1ex}
    \caption{DRAM cache hits per thousand instructions for GAPBS and NPB.
    The figure also shows the distribution of read and write hits out of the total number of hit accesses.}
    \label{fig:cs1Hpki}
    \vspace{-1.5em}
\end{figure}

\begin{figure}
    \centering
    \includegraphics[width=\linewidth]{figures/cs1_all_bw_util.pdf}
    \vspace{-1ex}
    \caption{Bandwidth utilization of different workloads on DRAM cache (local) and backing store (far) interfaces. Peak bandwidth is 32 GB/s and 19.2 GB/s for local and far memory. Figure shows that the workloads significantly under-utilize the available bandwidth due to access amplification.}
    \label{fig:cs1BwUtil}
    \vspace{-1.5em}
\end{figure}

\textbf{Takeaway:} the current DRAM cache designs perform poorly even when the DRAM cache miss ratio is as low as 20\%. In case of DRAM cache miss, the extra memory accesses required for single demand access lead to increased access latency and under-utilization of available bandwidth to the memory devices.
This behavior shows that we need to optimize the DRAM cache design for both miss and write-hit handling to improve the performance.

\section{Case Study 2: Cache Architecture Exploration}
\label{sec:cs2}

In previous section we showed the performance degradation of the baseline system with DRAM cache (Figure \ref{fig:case-studies} bottom left) compared to the same system without the DRAM cache (Figure \ref{fig:case-studies} top). In this study,
we investigate optimized DRAM cache designs to mitigate the overheads shown in Section \ref{cs1}.
The baseline DRAM cache protocol of our model which is used in Section \ref{cs1} is inspired by the real hardware
of Intel's Cascade Lake. However, our model is capable of implementing different DRAM cache
designs. To show this capability, in this section we show two different cache architecture optimizations
on top of the baseline model.

% For this purpose we focus on optimizing the initial read access to the DRAM cache for tag check,
% since in the baseline model the tag and metadata are part of ECC, stored along with the data within the same cache line.

%We get the motivation for the two optimizations from the results shown in Figures \ref{fig:cs1Mpki} and \ref{fig:cs1Hpki}.
As explained in the Section \ref{design}, the baseline DRAM cache architecture stores the tag and metadata in ECC bits in the cache line, along with the data.
Thus, to check whether a memory request is hit or miss on the DRAM cache, it reads the entire cache line from DRAM, while
only the tag and metadata is needed for tag comparison. If the memory request is miss clean (for both read and write requests) or write hit on DRAM cache,
reading the data becomes a waste of bandwidth. We elaborate the details of these cases as follows:

\paragraph {Write Hits}
Figure \ref{fig:cs1Hpki} shows write hit requests consist a noticeable
portion of total hit accesses. For these accesses in the baseline design, cache manager reads the cache line from DRAM cache for tag comparison.
Once the cache manager receives the cache line and the tag in the request address and the tag in the cache line match, it sends a write request with the data from demand access to the cache line address.
In this process, the data in the cache line fetched for tag check is never used and is overwritten, eventually. Figure \ref{fig:cs1Hpki} shows 
write hit accesses consist a large part of all hit accesses.

\paragraph {Read or Write Miss-Cleans}
%This is more noticeable for read requests, but the write requests are also contributing to this portion.
As we described above for write hits, the DRAM cache manager reads the entire cache line that the request maps to for tag check.
Once the tag comparison in the DRAM cache manager fails to match, if the request is a write, the cache manager sends a write request with the data from the demand access to the DRAM cache line address.
If the request is read, the DRAM cache manager sends a read request to the backing store to get the missing cache line from the main memory.
In these processes, the data in the cache line which was fetched for tag check is never used. 
Figure \ref{fig:cs1Mpki} shows that significant portions of the total memory requests that miss on DRAM cache consist of miss accesses to clean cache lines.


Given these two scenarios, where there is a bandwidth waste by the baseline DRAM cache, we investigate two different designs for the DRAM cache state machine.
First, we implement the optimization introduced by the BEAR cache for some write accesses~\cite{chou2015bear}.
BEAR cache tries to avoid reading the line from DRAM cache for tag check, for the write hit accesses to the DRAM cache.
BEAR determines the write hit accesses using the metadata stored in the last level cache.
We apply this optimization to the baseline architecture of our model and name this case \emph{\bear{}}.

Second, we change the baseline DRAM cache design in a way that avoids the read access to DRAM cache for tag
check, not only for write hit demands, but also if the demand access (either read or write) will miss on DRAM cache and the cache line is
clean. We call this case \emph{\oracle{}}. We assume \oracle{} has a zero latency SRAM storage in the cache manager to hold all the tag and metadata
it needs to determine if the demand access will hit or miss to a clean or dirty cache line. Table \ref{tab:accAmp} summarizes the optimizations of \bear{} and \oracle{} compared to the baseline in terms of
reducing extra accesses from pathology of DRAM cache design in all possible memory request cases.

To compare the performance of these three different designs (baseline, \bear{}, and \oracle{}), we run the GAPBS and NPB through full-system simulation
as explained in Section \ref{method}. In all three cases, the cache manager uses HBM2 DRAM cache and DDR4 main memory.

\begin{table}[!h]
    \begin{center}
    \caption{\label{tab:accAmp} Comparison of number of extra accesses needed for a given memory request in
    Baseline, \bear{}, and \oracle{} designs, written in black, blue, and red, respectively.}
    \resizebox{\linewidth}{!}{
    \begin{tabular}[width=\linewidth]{| c | c | c | c | c | c | c | c | c |}
      \hline
      Access & \multicolumn{4}{c|}{Read} & \multicolumn{4}{c|}{Write}\\
      \hline
      Hit/Miss & \multicolumn{2}{c|}{Hit} & \multicolumn{2}{c|}{Miss} & \multicolumn{2}{c|}{Hit} & \multicolumn{2}{c|}{Miss}\\
      \hline
      Dirty/Clean & \multicolumn{1}{c|}{Dirty} & \multicolumn{1}{c|}{Clean} & \multicolumn{1}{c|}{Dirty} & \multicolumn{1}{c|}{Clean} & \multicolumn{1}{c|}{Dirty} & \multicolumn{1}{c|}{Clean} & \multicolumn{1}{c|}{Dirty} & \multicolumn{1}{c|}{Clean}\\
      \hline
      Local Read &
       \ding{51} \textcolor{blue}{\ding{51}} \textcolor{red}{\ding{51}} &
       \ding{51} \textcolor{blue}{\ding{51}} \textcolor{red}{\ding{51}} &
       \ding{51} \textcolor{blue}{\ding{51}} \textcolor{red}{\ding{51}} &
       \ding{51} \textcolor{blue}{\ding{51}} \textcolor{red}{\ding{55}} &
       \ding{51} \textcolor{blue}{\ding{55}} \textcolor{red}{\ding{55}} &
       \ding{51} \textcolor{blue}{\ding{55}} \textcolor{red}{\ding{55}} &
       \ding{51} \textcolor{blue}{\ding{51}} \textcolor{red}{\ding{51}} &
       \ding{51} \textcolor{blue}{\ding{51}} \textcolor{red}{\ding{55}} \\
      \hline
      Far Read &
      \ding{55} \textcolor{blue}{\ding{55}} \textcolor{red}{\ding{55}} &
      \ding{55} \textcolor{blue}{\ding{55}} \textcolor{red}{\ding{55}} &
       \ding{51} \textcolor{blue}{\ding{51}} \textcolor{red}{\ding{51}} &
       \ding{51} \textcolor{blue}{\ding{51}} \textcolor{red}{\ding{51}} &
       \ding{55} \textcolor{blue}{\ding{55}} \textcolor{red}{\ding{55}} &
       \ding{55} \textcolor{blue}{\ding{55}} \textcolor{red}{\ding{55}} &
       \ding{55} \textcolor{blue}{\ding{55}} \textcolor{red}{\ding{55}} &
       \ding{55} \textcolor{blue}{\ding{55}} \textcolor{red}{\ding{55}} \\
      \hline
      Local Write &
      \ding{55} \textcolor{blue}{\ding{55}} \textcolor{red}{\ding{55}} &
      \ding{55} \textcolor{blue}{\ding{55}} \textcolor{red}{\ding{55}} &
       \ding{51} \textcolor{blue}{\ding{51}} \textcolor{red}{\ding{51}} &
       \ding{51} \textcolor{blue}{\ding{51}} \textcolor{red}{\ding{51}} &
       \ding{51} \textcolor{blue}{\ding{51}} \textcolor{red}{\ding{51}} &
       \ding{51} \textcolor{blue}{\ding{51}} \textcolor{red}{\ding{51}} &
       \ding{51} \textcolor{blue}{\ding{51}} \textcolor{red}{\ding{51}} &
       \ding{51} \textcolor{blue}{\ding{51}} \textcolor{red}{\ding{51}} \\
      \hline
      Far Write &
      \ding{55} \textcolor{blue}{\ding{55}} \textcolor{red}{\ding{55}} &
      \ding{55} \textcolor{blue}{\ding{55}} \textcolor{red}{\ding{55}} &
       \ding{51} \textcolor{blue}{\ding{51}} \textcolor{red}{\ding{51}} &
       \ding{55} \textcolor{blue}{\ding{55}} \textcolor{red}{\ding{55}} &
       \ding{55} \textcolor{blue}{\ding{55}} \textcolor{red}{\ding{55}} &
       \ding{55} \textcolor{blue}{\ding{55}} \textcolor{red}{\ding{55}} &
       \ding{51} \textcolor{blue}{\ding{51}} \textcolor{red}{\ding{51}} &
       \ding{55} \textcolor{blue}{\ding{55}} \textcolor{red}{\ding{55}} \\
      \hline
      \hline
      Tot. Baseline & 1 & 1 & 4 & 3 & 2 & 2 & 3 & 2 \\
      \hline
      Tot. \bear{} & \textcolor{blue}{1} & \textcolor{blue}{1} & \textcolor{blue}{4} & \textcolor{blue}{3} & \textcolor{blue}{1} & \textcolor{blue}{1} & \textcolor{blue}{3} & \textcolor{blue}{2} \\
      \hline
      Tot. \oracle{} & \textcolor{red}{1} & \textcolor{red}{1} & \textcolor{red}{4} & \textcolor{red}{2} & \textcolor{red}{1} & \textcolor{red}{1} & \textcolor{red}{3} & \textcolor{red}{1} \\
      \hline
    \end{tabular}
    }
    \end{center}
\end{table}

\subsection*{Results and Discussions}

Figure \ref{fig:cs2SuAll} shows the performance speedup of \oracle{} and \bear{} compared to the baseline.
\oracle{} and \bear{} outperform the baseline architecture. The speedup gain for
\oracle{} is higher compared to \bear{}, because the optimization of \oracle{} is more extensive than \bear{}, and
it optimizes more demand access pathologies in DRAM cache protocol, compared to \bear{}.

\begin{figure}
    \centering
    \includegraphics[width=\linewidth]{figures/cs2_all_sup.pdf}
    \caption{Speedup of \bear{} and \oracle{} compared to the baseline DRAM cache design in GAPBS and NPB.
    \string*$bt$ is excluded from \oracle{} analysis, as it started a different phase of program execution.}
    \label{fig:cs2SuAll}
    \vspace{-1.5em}
\end{figure}

\subsection{GAPBS in \bear{}}

$bc$ has a significant number of write hits in its hits per thousand instructions or HPKI (Figure \ref{fig:cs1Hpki}).
$bc$ also has the lowest miss ratio of all the workloads in GAPBS (Figure \ref{fig:cs1MissRatio}).
Removing the DRAM read for tag check from write hit handling path (what \bear{} does) reduces the latency of servicing misses
and increases system throughput.
Figure \ref{fig:cs2SuAll} shows $bc$ has the highest speedup amongst all workloads in GAPBS for \bear{}.

$pr$ has the lowest amount of speedup for \bear{} because it has the least amount of write hits in its HPKI
(Figure \ref{fig:cs1Hpki}) and it has the highest miss ratio (Figure \ref{fig:cs1MissRatio}).
Optimizing a non-significant number of write hits in this workload does not reduce the latency of misses.
As a result, the speedup is hardly above one.

Another interesting case is $tc$ which has a significant number of write hits in its HPKI (Figure \ref{fig:cs1Hpki});
however, it does not benefit from \bear{}.
The reason is $tc$ has significantly higher ratio of miss dirty lines, than the rest of workloads in GAPBS.
All the dirty evicted lines must be written back to the main memory. DRAM cache manager holds them in Write Back buffer
(WB buffer in Figure \ref{fig:dcache}). Once the WB buffer is full, the cache manager prioritizes the writes in
the WB buffer over the requests in outstanding requests buffer (ORB in Figure \ref{fig:dcache}).
In this way, the cache manager makes entry available in the WB buffer for incoming DRAM cache line
evictions that are dirty. This prioritization can increase the latency of the requests in ORB.
Even though the write hit optimization in \bear{} can reduce the latency of misses, for a write intensive application
with relatively high miss ratio like $tc$ is not beneficial.
The rest of the workloads’ speedups are proportional to their write hits and miss ratios.

\subsection{NPB in \bear{}}

Figure \ref{fig:cs2SuAll} shows $lu$ has the highest speedup in NPB for \bear{}.
$lu$ has a large number of write hits in its HPKIs (Figure \ref{fig:cs1Hpki})
and the lowest miss ratios compared to the rest of workloads in NPB (Figure \ref{fig:cs1MissRatio}).
As we explained above, removing the DRAM read for tag check from write hit handling path in \bear{},
reduces the latency of servicing misses and increases system throughput.

$cg$ dose not benefit from \bear{} since it has close to 100\% miss ratio (Figure \ref{fig:cs1MissRatio})
and very limited number of write hits (Figure \ref{fig:cs1Hpki}).

\subsection{GAPBS in \oracle{}}

Figure \ref{fig:cs2SuAll} shows $tc$ and $bc$ have the highest speedup amongst all workloads for \oracle{}.
$tc$ and $bc$ have highest portion of write hits in their HPKIs (Figure \ref{fig:cs1Hpki}), also,
lower miss ratios compared to the other workloads in the suite (Figure \ref{fig:cs1MissRatio}).
Moreover, they have the highest portion of write miss cleans in their Misses per Thousand Instructions or MPKI
(Figure \ref{fig:cs1Mpki}) which can have an out of size impact on the performance improvement.
The speedups of the other workloads in GAPBS is proportional to the number of read miss cleans accesses (Figure \ref{fig:cs1Mpki})
that \oracle{} optimizes.

\subsection{NPB in \oracle{}}

The MPKIs of $cg$, $is$, and $sp$ workloads are dominated by read miss cleans (Figure \ref{fig:cs1Mpki})
and they have the highest miss ratios (Figure \ref{fig:cs1MissRatio}).
Removing the DRAM read for tag check from read miss cleans handling path (what \oracle{} does)
reduces their latencies and increases system throughput. Thus, $cg$, $is$, and $sp$ have the highest
speedups amongst all workloads as Figure \ref{fig:cs2SuAll}. $lu$ has the lowest speedup because its MPKI is dominated by read miss dirties, not optimized by \oracle{}.

$bt$ in NPB is an outlier in this evaluation because our analysis shows it started another phase of execution. Thus, we exclude it from analysis of \bear{} and \oracle{}.

\begin{figure}
  \centering
  \includegraphics[width=\linewidth]{figures/cs2_all_sup_ddr4mm.pdf}
  \caption{Speedup of \bear{} and \oracle{} compared to the system without the DRAM cache in GAPBS and NPB.
  \string*$bt$ is excluded from \oracle{} analysis, as it started a different phase of program execution.}
  \label{fig:cs2SuAllMm}
  \vspace{-1.5em}
\end{figure}

Finally, we show the speedup of \bear{} and \oracle{} compared to the same system
without the DRAM cache (with a DDR4 main memory only) in Figure~\ref{fig:cs2SuAllMm}. We observe that both optimized DRAM cache designs we investigate
in this case study still perform less than a system without the DRAM cache. \oracle{} optimizes more pathologies
in the baseline DRAM cache design compared to \bear{}; thus, it has higher speedup than \bear{}. Even though
\oracle{} uses a zero-latency SRAM tag and metadata storage in its implementation, it still needs further optimization
to outperform a system without the DRAM cache.

\textbf{Takeaway:} 
% In Section \ref{cs1} we showed the baseline DRAM cache design which is inspired by real hardware of Intel's
% Cascade Lake has performance degradation compared to the same system without the DRAM cache. The baseline DRAM cache design stores
% tag and metadata along with the data in the same DRAM cache location. Thus, for tag comparison the cache manager reads the entire cache line from DRAM.
% If the demand access is write and it hits on DRAM cache, or if the demand access misses on DRAM cache to a clean cache location, the cache manager will not use the data in the read line for tag check.
% Essentially, for such memory requests the latency will be increased, and it reduces the system throughput.
In this case study we showed if a cache design can provide a mechanism to determine the hit/miss and clean/dirty for a given request without accessing the DRAM,
it can provide performance improvement compared to the baseline DRAM cache design. However, these optimizations 
are not enough for the DRAM cache system to outperform the same system without the DRAM cache.

% This data shows that even though storing tag and metadata along with the data in the same cache line simplifies
% the DRAM cache architecture, it contributes to performance degradation (baseline design).
% By fetching the entire cache line for
% demand accesses that will eventually be discarded (miss-clean for both read and write demands) or overwritten
% (write demands hits), essentially two downsides will happen. First, a waste of bandwidth will be enforced
% to the system, since the data part (significant portion of the fetched line) is not going to be used.
% Second, the miss handling process will be delayed until after the entire cache line
% (containing tag and metadata) is received from the DRAM. This will increase the latency.
% In \oracle{} and \bear{} we explored potential performance benefits of overcoming these downsides existing in the real hardware of DRAM caches
% as well as the prior research work in this area. Given the miss ratio and high portion of aforementioned demand accesses all the
% workloads had, they all benefited from such optimizations.



\section{Case Study 3: Impact of Link Latency}
\label{sec:cs3}

So far, we have assumed there is no extra latency between the local (near) memory as DRAM cache and the far (remote)
memory as its backing store. In this case study, we will analyze the impact of latency of the link between these two nodes
on the system's performance. For this purpose, we consider the system shown in Figure \ref{fig:case-studies} on the bottom right.
\begin{figure}

    \subfloat[DDR4]{%
      \includegraphics[clip,width=\linewidth]{figures/cs3_bips_ddr4.pdf}%
      \vspace{-1em}
      \label{fig:cs3BipsDdr4}
    }
    \vspace{-1em} \\
    \subfloat[NVM]{%
      \includegraphics[clip,width=\linewidth]{figures/cs3_bips_nvm.pdf}%
      \vspace{-1em}
      \label{fig:cs3BipsNvm}
    }
    \caption{Performance of GAPBS and NPB on DRAM cache for 100 ns, 500 ns, and 100 ns round-trip
    link latency of the far memory.
    BIPS refers to a billion instructions per second. DDR4 and NVM technologies are used for remote main memory.}
    \label{fig:cs3BipsDDR4NVM}
    \vspace{-2em}
\end{figure}

% \Maryam{add CXL here} One way to use DRAM cache is to have it as a cache for memories that are lower bandwidth and larger capacity compared to DRAMs.
% This is highly desirable for disaggregated systems to attach remote large capacity memories through a link to the local
% DRAMs. Another use case is to have DRAMs as a cache for NVM memories. NVM memories are known to provide large capacity;
% however, they have limited bandwidth and higher latency compared to DRAM. Thus, DRAM caches, if designed properly,
% can help hide the higher latency and limited bandwidth of far and larger capacity memories.

%  We ask
% \emph{what is the impact of latency of the link between local DRAM caches and far backing stores on the
% performance of systems and does DRAM cache system perform better than the same system without DRAM cache?}

The DRAM cache model we described in this paper is capable of employing any memory technologies that are modeled in \gem{},
including DDR3, DDR4, HBM1, HBM2, and NVM. In this study, we use the baseline DRAM cache design (inspired by Intel's Cascade Lake). We use a single channel of HBM2 (2 pseudo-channels) DRAM cache
with a single channel of DDR4 remote backing store. In a second case, we change the remote main memory to a single channel of NVM. The DDR4 and NVM models of
\gem{} provide 19.2 GB/s theoretical peak bandwidth per single channel, though NVM has higher read and write latencies than DDR4.
% Both systems will have the same configuration except the technology used for the
% backing store as the far memory.
We also add a link between the DRAM cache manager and the remote backing store
with a configurable latency, as shown in Figure \ref{fig:case-studies} on the bottom right. We conduct a full-system simulation to run GAPBS and NPB on the two systems in three
different cases where the round-trip latency of the link will be set to 100 ns, 500 ns, and 1000 ns.
Similar upper bound latency numbers are reported for direct attached, serially attached, and network attached memory devices in the related industrial~\cite{gupta2020genz} and academic~\cite{maruf2022tpp} work.
%Directly attached memory refers to the traditional memory devices connected to the CPU using a parallel bus and might be shared among CPU sockets.
%Serially attached memory uses serial interfaces like CXL and will be inside the same chassis as the CPU.
%Network-attached memory devices do not have to be in the same chassis as the CPU and today can rely on interfaces like RDMA. However, we expect that CXL will also enable network-attached memory in the near future.
The rest of the methodology for the experiments remains the same as described in Section~\ref{method}.

\subsection*{Results and Discussions}

Figures \ref{fig:cs3BipsDdr4} and \ref{fig:cs3BipsNvm} show a billion instructions per second (BIPS) of DRAM cache systems while running GAPBS and NPB
workloads for DDR4 and NVM remote main memories, respectively.
For 100 ns link latency, the system with DDR4 performs better than the system with NVM.
This is expected given the higher read and write latencies of NVM devices compared to DDR4.

Once we increase the link latency to 500 ns, as shown in the figures, the DDR4 system still has higher performance compared to the NVM system.
However, the performance gap between the two systems in 500 ns is smaller than the gap in 100 ns link latency. 
Once the link latency increases to 1000 ns, for most of the workloads
the performance gap between DDR4 and NVM decreases compared to the 500 ns case.
As the link latency increases, the overall system performance drops in each system (DDR4 and NVM);
however, it drops less for the NVM system. 
With increased link latency both DDR4 and NVM remote memories have enough bandwidth to respond to the requests and the
latency of the NVM device gets amortized over the requests. 
Thus, the performance of the system with NVM drops less.

We also see in the figures that once the latency increase from 500 ns to 1000 ns the performances of $bt$ and $sp$
drop significantly in NVM system compared to DDR4 system.
These two workloads have the highest ratio of dirty line misses (Figure \ref{fig:cs1Mpki}). The DRAM
cache manager require to write the evicted dirty lines to the backing store. The limited write buffer of NVM
devices~\cite{wang2020characterizing} and the high link latency to access the remote NVM, create back pressure on the
NVM backing store. Thus, the performance drops significantly if the application is write intensive with a high miss
ratio for the NVM system.

Finally, we compare the performance of the DRAM cache systems to the same systems without the DRAM cache. Figures
\ref{fig:cs3SuDdr4} and \ref{fig:cs3SuNvm} show the speedup of the DRAM cache system compared to the same system
without the DRAM cache for DDR4 and NVM remote main memories, respectively. 
For GAPBS, Figure \ref{fig:cs3SuDdr4} shows once the link latency increases toward 500 ns and 1000 ns, the DRAM cache with DDR4 far main memory outperforms
the same system without the DRAM cache. However, as we observe in Figure \ref{fig:cs3SuNvm}, the DRAM cache with remote NVM main memory outperforms the same system without the DRAM cache for all the link latencies. The
differences between the read and write latencies of the DDR4 and NVM devices can explain the case for 100 ns link latency.
For NPB, the DRAM cache systems do not outperform the systems without the DRAM cache, for both DDR4 and NVM cases.
In Figures \ref{fig:cs1MissRatio} and \ref{fig:cs1Mpki} we observe that NPB is more write-intensive and has higher miss ratio compared to the GAPBS.
Thus, workloads in NPB generate more write backs to the main memory compared to GAPBS. Thus,
as the link latency grows, the pressure on the remote main memory increases. This becomes more critical for
NVM main memory systems due to their limited write buffer. This pressure does not exist in the system without DRAM cache.
Thus, NPB's performance drops as link latency increases in Figure \ref{fig:cs3SuNvm}.


% To better demonstrate how the NVM system's performance compares to the DDR4 system in these experiments, Figure \ref{fig:cs3suNvmDdr4}
% shows the speedup of NVM system compared to DDR4 system per benchmark suite for three link latencies. For speedup calculation,
% we used the geometric mean of all workloads' BIPS within each suite.
% For GAPBS, we see that as the link latency increases the speedup of the NVM system also increases. In other words, the NVM system's
% performance gets closer to the DDR4 system. For NPB, the figure shows as the link latency increases from 100 ns to 500 ns, the
% speedup also increases. However, once the latency increased from 500 ns to 1000 ns the speedup drops.
% The reason for this case is that the performance of two specific workloads in the suite, $bt$ and $sp$, drop significantly in NVM system with 1000 ns link latency, compared to DDR4 system.
% Our data shows these two workloads
% have the highest ratios of write-backs to the backing store. Given the limited write buffer NVM devices
% have (64 entries in our experiment) and adding the high link latency to access the NVM, has put back pressure on the backing store.

% As the link latency increases both DDR4 and NVM remote memories have enough bandwidth to respond to the requests and the
% latency of the NVM device gets amortized over the requests. Thus, the performance of the system with NVM grows
% towards the performance of the DDR4 system. This is shown in Figure \ref{fig:suNVM1000Cs3} for link latency of 1000 ns.
% As explained before, $bt$ and $sp$ have the highest write-back ratios, thus, they have less speedup than other workloads.

% Note that we used the baseline DRAM cache architecture for this experiments which can be optimized further in many ways, as
% discussed in the previous section, for future works.

\textbf{Takeaway:} This case study shows for systems with long latencies (e.g., 1000 ns) between local DRAM caches and remote main memories, NVM can perform close to
DDR4 devices. This can turn into an interesting use case to utilize large capacity of NVM devices. The study
also shows that for the workloads that are not write-intensive with high miss ratio, DRAM cache systems can outperform
the same system without the DRAM cache, if the remote main memory's link latency is higher than 100 ns.

% \begin{figure}
%     \centering
%     \includegraphics[scale=0.48]{figures/cs3_su_geomean.pdf}
%     \vspace{-1ex}
%     \caption{Speedup of NVM compared to DDR4 for GAPBS and NPB.
%              Geometric mean of workload's BIPS has been calculated for overall performance for each benchmark suite. }
%     \label{fig:cs3suNvmDdr4}
%   \end{figure}


% \begin{figure}
%     \center
%     \subfloat[\scriptsize{GAPBS}]{
%     \includegraphics[width=.48\linewidth, scale=0.5]{figures/cs3_gap_suNVM_1000ns.pdf}
%     \label{fig:gapSuNVM1000Cs3}
%     }
%     \subfloat[\scriptsize{NPB}]{
%     \includegraphics[width=.48\linewidth, scale=0.5]{figures/cs3_npb_suNVM_1000ns.pdf}
%     \label{fig:npbSuNVM1000Cs3}
%     }
%     \caption{Speedup of NVM system compared to DDR4 system for link latency of 1000 ns.}
%     \label{fig:suNVM1000Cs3}
% \end{figure}

\begin{figure}
    \center
    \subfloat[\scriptsize{DDR4}]{
    \includegraphics[width=.49\linewidth]{figures/cs3_su_geomean_mm_ddr4.pdf}
    \label{fig:cs3SuDdr4}
    }
    \subfloat[\scriptsize{NVM}]{
    \includegraphics[width=.49\linewidth]{figures/cs3_su_geomean_mm_nvm.pdf}
    \label{fig:cs3SuNvm}
    }
    \caption{Speedup of DRAM cache system backed up by (a) DDR4 remote memory and (b) NVM remote memory compared to the same system without the DRAM cache.
    Geometric mean of workloads' throughput is used for speedup.}
    \label{fig:cs3SuDcacheMm}
    \vspace{-1.5em}
\end{figure}

\section{Related work}
\noindent \textbf{Video foundation models.}
With sufficient computational power and an abundant source of data, there have been attempts to build a single large-scale foundation model that can be adapted to diverse downstream tasks.
Along with the success of foundations models in the natural language processing domain~\cite{brown2020language,chen2021evaluating,devlin2019bert} and in computer vision~\cite{bertasius2021space,jia2021scaling,radford2021learning}, video data has become another data type of interest, as it has grown in scale due to numerous internet video-sharing platforms.
Accordingly, several methods to train a video foundation model have been proposed.
Due to the innate multi-modality of video data, \textit{i.e.}, a combination of visual $\cdot$ vocal $\cdot$ textual context, most works have centered around the variations of the cross-modal attention mechanism \cite{akbari2021vatt,bertasius2021space,gabeur2020multi,luo2020univl,neimark2021video,tan2021look,wei2020multi,yang2021taco}.
In addition, as most video data lack proper labels or descriptions, contrastive learning methods were studied to learn meaningful feature representations or enhance video-text alignment in a self-supervised manner \cite{akbari2021vatt,kuang2021video,luo2020univl,yang2021taco}.

More specifically, MERLOT \cite{zellers2021merlot} proposed a multi-modal representation learning method for visual commonsense reasoning, which also performed well in twelve video reasoning tasks.
VATT \cite{akbari2021vatt} introduced a multi-modal learning method via contrastive learning. 
The pre-trained model performed well in a variety of vision tasks from image classification to video action recognition and zero-shot video retrieval.
Another representative work, UniVL \cite{luo2020univl} proposed a straightforward pre-training method with auxiliary loss functions. 
After fine-tuning on a specific task, the pre-trained model performed outstandingly in a wide range of tasks of text-to-video retrieval, action segmentation, action step localization, video sentiment analysis, and video captioning.
Other foundation models for multiple video tasks include \cite{li2020hero,sun2019learning,sun2019videobert,zhu2020actbert,fu2021violet,wang2022all}. 

\noindent \textbf{Auxiliary learning.}
In order to enhance the performance of one or a multitude of primary tasks, auxiliary learning methods can be incorporated.
\cite{ruder2017overview} introduced Multi-task learning (MTL) to the deep neural networks by training a single model with multiple task losses to assist learning on the main task.
Such a method is generally adapted to pre-train the foundation models in the self-supervised manner~\cite{li2020hero,sun2019learning,sun2019videobert,zhu2020actbert,fu2021violet,wang2022all}.
However, these various pretext task losses used in the pre-training phase are ignored in the fine-tuning phase, and only the primary task loss is minimized.

Recently, meta-learning methods have been introduced for auxiliary learning.
\cite{liu2019self,navon2020auxiliary,shu2019meta} proposed a meta-learning method in which the model learns auxiliary tasks to generalize well to unseen data. 
In these settings, a separate subset of data is held out as the primary task, while the others are used as auxiliary tasks that aid the primary task's performance.
Similar methods were adopted for computer vision tasks such as semantic segmentation \cite{xu2021leveraging}.
Other domain applications include navigation tasks with reinforcement learning \cite{ye2021auxiliary}, or self-supervised learning methods on graph data \cite{hwang2020self}.

\section{Conclusion}

In this work, we described our detailed cycle-level DRAM cache
model implemented in \gem{}, which enables design space exploration for DRAM caches
in emerging memory systems.
The model presented in this work
can enable many interesting research works in the domain of heterogeneous and disaggregated memory systems.
For instance, using our DRAM cache model we can address questions such as: what is
the efficient data placement and data movement policy and mechanism in systems composed of fast and slow memories.
Since our model is implemented in a full system simulation platform, it can also enable the hardware and software co-design
research in such systems.

% Moreover, \gem{} is highly modular and allows composing a simulated system based on a variety of components.
% \textit{UDCC} can enable experimenting with different and new memory device models which might have features to be a better fit to be used as a cache to a backing memory.

%\section*{ACKNOWLEDGMENT}

\bibliographystyle{IEEEtran}
\bibliography{IEEEabrv,references}

\end{document}
