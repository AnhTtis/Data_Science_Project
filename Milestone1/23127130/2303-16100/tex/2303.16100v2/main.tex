\documentclass[conference]{IEEEtran}
\IEEEoverridecommandlockouts
% The preceding line is only needed to identify funding in the first footnote. If that is unneeded, please comment it out.

\usepackage{cite}
\renewcommand\citepunct{, }
\usepackage{amsmath,amssymb,amsfonts}
\usepackage{algorithmic}
\usepackage{algorithm}
\usepackage{graphicx}
\usepackage{textcomp}
\usepackage{xcolor}
\usepackage{microtype}
\usepackage{subfigure}
\usepackage{booktabs}
\usepackage{hyperref}
\usepackage{multirow}

\newcommand{\fixme}[1]{\textcolor{red}{#1}}

\begin{document}

\title{Energy-efficient Task Adaptation for NLP Edge Inference Leveraging Heterogeneous Memory Architectures}

\author{
\IEEEauthorblockN{Zirui Fu, Aleksandre Avaliani, Marco Donato}
\IEEEauthorblockA{
Tufts University, Medford, MA, USA\\}
}

\maketitle

\begin{abstract}
    Executing machine learning inference tasks on resource-constrained edge devices requires careful hardware-software co-design optimizations. Recent examples have shown how transformer-based deep neural network models such as ALBERT can be used to enable the execution of natural language processing (NLP) inference on mobile systems-on-chip housing custom hardware accelerators. 
    However, while these existing solutions are effective in alleviating the latency, energy, and area costs of running single NLP tasks, achieving multi-task inference requires running computations over multiple variants of the model parameters, which are tailored to each of the targeted tasks. This approach leads to either prohibitive on-chip memory requirements or paying the cost of off-chip memory access. 
    This paper proposes adapter-ALBERT, an efficient model optimization for maximal data reuse across different tasks. 
    The proposed model's performance and robustness to data compression methods are evaluated across several language tasks from the GLUE benchmark. 
    Additionally, we demonstrate the advantage of mapping the model to a heterogeneous on-chip memory architecture by performing simulations on a validated NLP edge accelerator to extrapolate performance, power, and area improvements over the execution of a traditional ALBERT model on the same hardware platform.
\end{abstract}

\begin{IEEEkeywords}
energy-efficiency hardware software co-design, deep neural network, heterogeneous memory architecture, edge computing
\end{IEEEkeywords}

\section{Introduction}\label{sec:intro}

Making co-speech gestures is an innate human behavior in daily conversations, which helps the speakers to express their thoughts and the listeners to comprehend the meanings~\cite{cassell1999speech, mcneill2011hand, 2014Gesture}. 
%
Previous linguistic studies verify that such non-verbal behaviors could liven up the atmosphere and improve mutual intimacy~\cite{burgoon1990nonverbal, 1989Gesture, huang2012robot}.
%
Therefore, animating virtual avatars to gesticulate co-speech movements is crucial in embodied AI. 
To this end, recent researches focus on the problem of audio-driven co-speech gesture generation~\cite{ginosar2019learning, yoon2020speech, liu2022learning, li2021audio2gestures}, which synthesizes human upper body gesture sequences that are aligned to the speech audio.

Early attempts downgrade this task as a searching-and-connecting problem, where they predefine the corresponding gestures of each speech unit and stitch them together by optimizing the transitions between consecutive motions for coherent results~\cite{cassell1994animated, huang2012robot, marsella2013virtual}. 
%
In recent years, the compelling performance of deep neural networks has prompted data-driven approaches. 
%
Previous studies establish large-scale speech-gesture corpus to learn the mapping from speech audio to human skeletons in an end-to-end manner~\cite{alexanderson2020style, liu2022beat, xu2022freeform, qian2021speech, liu2022learning, li2021audio2gestures, ao2022rhythmic}. 
%
To attain more expressive results, Ginosar \textit{et al.}~\cite{ginosar2019learning} and Yoon \textit{et al.}~\cite{yoon2020speech} propose GAN-based methods to guarantee realism by adversarial mechanism, where the discriminator is trained to distinguish real gestures from the synthetic ones while the generator's objective is to fool the discriminator. 
%
However, such pipelines suffer from the inherent mode collapse and unstable training, making them difficult to capture the \textit{high-fidelity audio-conditioned} gesture distribution, resulting in dull or unreasonable poses.

\begin{figure}[t]
\centering
\includegraphics[width=1.00\columnwidth]{figure/diffusion_process.pdf}
\caption{\textbf{Illustration of Conditional Generation Process in Co-Speech Gesture Generation.} The diffusion process $q$ gradually adds Gaussian noise to the gesture sequence (\textit{i.e.}, $\bm{x}_0$ sampled from the real data distribution). The generation process $p_{\theta}$ learns to denoise the white noise (\textit{i.e.}, $\bm{x}_T$ sampled from the normal distribution) conditioned on context information $\bm{c}$. Note that $\bm{x}_t$ denotes the corrupted gesture sequence at the $t$-th diffusion step.}
\label{overview}
\vspace{-0.2cm}
\end{figure}

The recent paradigm of diffusion probabilistic models provides a new perspective for realistic generation~\cite{ho2020denoising, song2021scorebased}, facilitating high-fidelity synthesis with desirable properties such as good distribution coverage and stable training compared to GANs.
%
However, it is non-trivial to adapt existing diffusion models for co-speech gesture generation. 
%
Most existing conditional diffusion models deal with \textit{static} data and conditions~\cite{Saharia2022Photorealistic, ramesh2022hierarchical} (\textit{e.g.}, the image-text pairs without temporal dimension), while co-speech gesture generation requires generating \textit{temporally coherent} gesture sequences conditioned on continual audio clips.
%
Further, the commonly used denoising strategy in existing diffusion models samples independently and identically distributed (\textit{i.i.d.}) noises in latent space to increase diversity. However, this strategy tends to introduce variation for each gesture frame and lead to temporal inconsistency in skeleton sequences. 
%
Therefore, how to generate high-fidelity co-speech gestures with strong audio correlations and temporal consistency is quite challenging within the diffusion paradigm.

To address the above challenges, we propose a tailored Diffusion Co-Speech Gesture framework to \textit{capture the cross-modal audio-gesture associations while maintaining temporal coherence} for high-fidelity audio-driven co-speech gesture generation, named \textbf{DiffGesture}.
%
As shown in Figure~\ref{overview}, we formulate our task as a diffusion-conditional generation process on clips of skeleton and audio, where the diffusion phase is defined by gradually adding noise to gesture sequence, and the generation phase is referred as a parameterized Markov chain with conditional context features of audio clips to denoise the corrupted gestures. 
%
As we treat the multi-frame gesture clip as the diffusion latent space, the skeletons can be efficiently synthesized in a non-autoregressive manner to bypass error accumulation.
%
To better attend to the sequential conditions from multiple modalities and enhance the temporal coherence, we then devise a novel \textit{Diffusion Audio-Gesture Transformer} architecture to model audio-gesture long-term temporal dependency.
%
Particularly, the per-frame skeleton and contextual features are concatenated in the aligned temporal dimension and embedded as individual input tokens to a Transformer block.
%
Further, to eliminate the temporal inconsistency caused by the naive denoising strategy in the inference stage, we thus propose a new \textit{Diffusion Gesture Stabilizer} module to gradually anneal down the noise discrepancy in the temporal dimension.
%
Finally, we incorporate implicit classifier-free guidance by jointly training the conditional and unconditional models, which allows us to trade off between the diversity and sample quality during inference.

Extensive experiments on two benchmark datasets show that our synthesized results are coherent with stronger audio correlations and outperform the state-of-the-arts with superior performance on co-speech gesture generation.
%
To summarize, our main contributions are three-fold: \textbf{1)} As an early attempt at taming diffusion models for co-speech gesture generation, we formally define the diffusion and denoising process in gesture space, which synthesizes audio-aligned gestures of high-fidelity. \textbf{2)} We devise the \textit{Diffusion Audio-Gesture Transformer} with implicit classifier-free diffusion guidance to better deal with the input conditional information from multiple sequential modalities. \textbf{3)} We propose the \textit{Diffusion Gesture Stabilizer} to eliminate temporal inconsistency with an annealed noise sampling strategy. 
\section{Model and Memory Architectures}

\begin{figure*}[ht]
    \centering
    \includegraphics[width=\textwidth]{fig/adapter-albert.pdf}
    \caption{The overview of the adapter-ALBERT model (a) and the HMA (b) architectures. The colors of the adapter-ALBERT model indicate the backbone layers (red) and non-fixed layers (blue). The colors of the HMA architecture indicate different roles of components: red and blue are HMA memory blocks and their colors match the parts of the adapter-ALBERT model; Purple DRAM block is off-chip memory; Yellow blocks belong to the EdgeBERT accelerator.}
    \label{fig:adapter-albert}
\end{figure*}

\subsection{The Adapter-ALBERT Model Architecture}
The block diagram of Figure \ref{fig:adapter-albert}(a) shows the structure of the adapter-ALBERT model. At a high level, the model is based on the same Embedding-Transformer-Classifier skeleton as the vanilla ALBERT model. The blue and red colors are used to differentiate between task-specific and backbone layers, respectively. The embedding layer, highlighted in red, can be fully shared across a variety of NLP tasks, while the classifier, highlighted in blue, is specifically trained on the task at hand by design. For traditional BERT-based language models, the transformer layers would need to be entirely fine-tuned to learn new tasks. However, the transformer layers in adapter-ALBERT are defined by a mixture of trainable and fixed layers. 
The transformer layer partition is shown in greater detail in the inset flow diagram in the middle of Figure \ref{fig:adapter-albert}(a). Going through the diagram from bottom to top, the multi-head attention layer and all feed-forward layers are fixed and use parameters inherited from a pre-trained model, while layer normalization parameters and the newly introduced adapters are considered task-specific layers. In particular, layer normalization parameters are made re-trainable to ensure correct normalization of current data, preventing unmatched data alignments to damage the model performance.
The introduction of adapter blocks, allows the model to learn new features at the cost of small parameter overhead.  
The two residual adapter modules have an identical structure as shown on the right end of the flow diagram.
An adapter module consists of an adapter-down layer that reduces the input feature size down to the adapter bottleneck size, an adapter-activation layer, and an adapter-up layer that increases the feature size back up to the output layer size before being added to the input feature values through a shortcut connection.
The input size of the adapter-down layer and output size of the adapter-up layer match the hidden feature size of 768 in the ALBERT model. The bottleneck size is defined in our model as a hyperparameter, and it plays an important role in determining the model inference accuracy. 
In particular, we will show that distinct adapter blocks can use different bottleneck sizes to prevent accuracy loss.

\subsection{The Heterogeneous Memory Architecture}
The block diagram of Figure \ref{fig:adapter-albert}(b) shows the architecture of the targeted edge system, with emphasis on the heterogeneous scratchpad memory. 
We use the same color coding to describe how task-specific and backbone parameters for the adapter-ALBERT model are stored on chip. 
The SRAM memory is used to store both activations and task-specific parameters. In addition, two separate RRAM memory blocks are used for storing the backbone parameters in sparse format using bitmask encoding. We perform bitmask encoding after pruning both the embedding and backbone transformer layers. The resulting sparse matrices are represented by a vector of non-zero values and a corresponding bitmask array to encode their location. 
Many non-volatile memories, including RRAM, are capable of storing multiple bits in a single memory cell. The multi-level cell (MLC) feature in RRAM is often used to increase storage density. However, MLC RRAM suffers from reduced read margins between adjacent resistance levels and therefore is subject to higher bit error rates. While DNN applications have a demonstrated resilience to fluctuations in the value of the model weights, even single bit-flips in the bitmask could have catastrophic effects on the model accuracy. Error correcting codes (ECCs) are a common approach to make fault-prone memories more robust, but they introduce additional complexity in terms of encoding and decoding data stored in memory, which would quash the benefits of implementing a simple sparse encoding scheme. Therefore, we limit the storage of non-zero values to MLC RRAM, while we adopt a more conservative single-level cell (SLC) RRAM array for storing the bitmask arrays. 

The memory architecture described above is designed to support the operation of a NLP hardware accelerator modeled after EdgeBERT~\cite{tambeEdgeBERT2021a}. The main computational units in the accelerator are depicted in the block diagram with the yellow blocks.
The processing unit contains bitmask encoders and decoders, multiply-and-accumulate (MAC) units, and activation units.
Following the original design for the accelerator, we consider a datapath with 256 MAC units operating at a clock frequency of 1 GHz.
The special function units, integrated within the heterogeneous scratchpad memory, are responsible for near-memory data computation, such as element-wise add and layer normalization. 

\begin{table*}
    \caption{Models Performance Comparison on GLUE Datasets}
    \label{tab:model-acc}
    \vspace{-1em}
    \begin{center}
    %\begin{small}
    %\begin{sc}
        \begin{tabular}{cc|cccccccccc}
            \toprule
            Model Name & Adapter Size & COLA & MNLI-MM & MRPC & SST-2 & STS-B & QQP & QNLI & RTE & WNLI \\
            \midrule
            BERT            & N/A   & 52.1   & 84.6  & 88.9   & 93.5   & 85.8   & 71.2   & 90.5   & 66.4   & 56.3   \\
            Adapter-BERT    & 64    & 56.9   & 85.3   & 89.6   & 94.2   & 87.3   & 71.8   & 91.4   & 68.8   & 54.9   \\
            ALBERT          & N/A   & 55.6   & 85.4   & 91.6   & 92.8   & 91.1   & 89.4   & 91.3   & 67.9   & 57.8   \\
            Adapter-ALBERT  & 64    & 50.1   & 84.2   & 91.03   & 90.9   & 87.6   & 85.3   & 91.2   & 69.7   & 50.7   \\
            Adapter-ALBERT  & 32 $\sim$ 128 & 51.7   & 84.7   & 90.8   & 91.39   & 87.5   & 85.9   & 91.5   & 71.1   & 54.9   \\
            \bottomrule
        \end{tabular}
    %\end{sc}
    %\end{small}
    \end{center}
    \vspace{-1em}
\end{table*}

\section{Model optimizations for efficient adaptation}
\subsection{Model Performance Evaluation}
We evaluate the accuracy performance of our adapter-ALBERT model against three alternatives from the same family, namely BERT \cite{devlinBERT2019}, adapter-BERT \cite{houlsbyParameterEfficient2019}, and ALBERT \cite{lanALBERT2020}, on the GLUE benchmark~\cite{wangGLUE2019}. As a first step, we perform a hyperparameter search across the different GLUE datasets. We begin by sweeping the learning rate selecting between 5E-4 and 5E-5, and consider training epochs from 5 to 10. Training and evaluation batch size are set both to 64. This initial set of experiments is performed in two phases: In the first phase, BERT and ALBERT are fine-tuned for each of the GLUE datasets; 
% In the second phase, we transfer the backbone layers from BERT and ALBERT to the corresponding adapter model and train only the task-specific layers.
In the second phase, we transfer the backbone layers from BERT and ALBERT to their corresponding adapter-enhanced model and train only the task-specific layers. As mentioned in the previous section, the adapter modules are designed to have their size independently adjusted as an hyperparameter. We show results for both the case in which the adapter size is fixed to 64 across all tasks, and the case in which the value of the adapter size is selected between 32 and 128. The results from this training experiments are summarized in Table~\ref{tab:model-acc}. 
and show that our adapter-ALBERT model is capable of maintaining competitive results to the other three models, and that varying the adapter size can compensate for the lost accuracy. In particular, the adapter-ALBERT model with an adapter size of 64 matches, or in some cases bests, the other three models on MNLI, MRPC, STS-B, QQP, QNLI, and RTE datasets. 
For the rest three datasets, adjusting the adapter size ranging from 32 to 128 helps boosting the performance close to the other models. 

In addition to evaluating accuracy, we also compare the size of the models in terms of number of parameters as it gives a indication of the relative memory footprint for each of these variants (Table \ref{tab:model-size}). The comparison of trainable parameters reflects the cost of re-training the model for new downstream tasks and the storage and movement costs for task-specific parameters under the MTI scenario. 
Besides showing a small parameter overhead compared to their traditional counterpart, the adapter models also have a much lower number of trainable parameters due to the partition between backbone and task-specific layers This distinction is the key feature that allows the model to perform multi-task inference more efficiently: In the event of the edge system changing the performed inference task, the traditional models would require an update on the entire set of trainable parameters while, for adapters, that number of parameters that need to be refreshed is kept to a few percent of the entire model. Note that, even in the case of adopting the largest adapters size, the parameter overhead would not exceed 4$\%$, or 400K parameters.






% \subsection{Critical Pruning Threshold Exploration}
\subsection{Model Compression}
After identifying the best set of hyperparameters to train the backbone and adapter layers, we focus on evaluating different compression techniques to make the model more suitable for running on edge devices. We evaluate pruning compression techniques that are ultimately combined to minimize the memory footprint for the adapter-ALBERT model.

\subsubsection{\textbf{Cumulative Sparsity Evaluation}}
Pruning is an effective method for compressing a DNN model and remove redundant parameters to reduce computational complexity and improve generalization capability \cite{liangPruning2021}. However, pruning usually causes accuracy degradation that can be recovered when combined with re-training. Previous work has shown that BERT can endure sparsity between 30\% and 40\% with no detrimental effects on the accuracy of either the pre-trained model or the transferred downstream tasks~\cite{gordonCompressing2020}. However, the new learning approach introduced by the adapter modules requires to re-assess the potential impact of pruning on the model's multi-task learning capabilities. Therefore, we conduct a sensitivity study to ascertain the highest sparsity level that can be achieved by adapter-ALBERT. In particular, we designed a series of experiments to identify the critical sparsity point (CSP) as the limit at which pruning causes the accuracy of the model to drop below the un-pruned baseline accuracy. Note that we apply pruning only to the backbone layers,~\textit{i.e.,} embeddings and fixed transformer layers. The task-specific layers represent only a small portion of the entire model and while pruning them would not affect the model size considerably, 
it would trigger dramatic accuracy degradation. 

%The pruning steps that lead to identifying the CSP are described in Algorithm \ref{alg:CSP}.

%\begin{algorithm}[b]
%    \caption{CSP Identification}
%    \label{alg:CSP}
%    \begin{algorithmic}
%        \STATE {set Diff(i) = $Accuracy_i$ - $Accuracy_{i-1}$}
%        \FOR{$i=0.9$ {\bfseries to} $0.2$}
%            \IF{Diff(i) $>$ Diff(i+1) and Acc(1.0) - Acc(i) $\geq$ 1.0}
%                \STATE {CSP $\Leftarrow$ i}
%            \ENDIF
%        \ENDFOR
%    \end{algorithmic}
%\end{algorithm}

We conduct several initial experiments that prune and re-train the embedding and transformer layers down to 90\% sparisty. We have summarized two observations from the model during these experiments as a guideline for cumulative sparsity evaluation.

\textit{Observation 1} - Task bias induced in the pre-trained backbone can significantly impact the overall performance. % This is the de-bias experiment process
% However, adapter-ALBERT maintains a fixed copy of the 
It is important to highlight that in our initial evaluation of the adapter models (Table~\ref{tab:model-acc}) we transferred the backbone parameters form the pre-trained vanilla ALBERT model and re-trained only the non-fixed layers. Pruning the model however, requires a different approach since at each incremental pruning step, the model needs to be retrained to recover from the loss of information. Retraining on BookCorpus would be extremely expensive. Therefore, we tested the effect of re-training the model during pruning using one of largest datasets with abundant information among the GLUE datasets, namely MNLI. While this version of the model could noticeably improve the accuracy performance on the MNLI task after pruning, it would also suffer a noticeable accuracy loss of 3\% on QNLI, which can be interpreted as having biased the backbone parameters towards MNLI. 
To minimize this biasing effect on the backbone parameters, we explored an iterative training approach by further fine-tuning the backbone parameters on QNLI and identifying the next dataset with the highest accuracy drop. Proceeding with this approach quickly leads to a loss of generalization capabilities in the backbone layers, especially when required to re-train on the smaller datasets, which exposes the model to overfitting. Therefore, we decided to explore an alternative route to recover the accuracy loss resulting from pruning.
%We later modified the experiment to employ a fine-tuned ALBERT model from MNLN dataset of GLUE tasks as the backbone of a second adapter-ALBERT, with the same re-training process on its non-fixed layers as the first model.
% First model -> original vanilla albert
% Second model -> pretrained on MNLI
%For un-pruned accuracy comparison, the second model yields superior result on MNLI than the first model, showing the first sign of pre-train bias.
%WE proceeded to prune the second model and identified the CSP at sparsity level 40\%. At this level, the adapter-ALBERT model shows the capability to delay rapid accuracy degradation similar to a vanilla ALBERT model, showing that the residual adapters have no negative impact to the model's performance from pruning.
%However, the QNLI result of the second model showed an unusual drop of 3\% accuracy. 

\textit{Observation 2} - Embedding layer and transformer layers show different pruning sensitivities.
In \textit{observation 1}, we analyzed the adapter-ALBERT model's response to a common pruning threshold applied to both embedding and transformer layers. However, a more accurate picture of how adapter-ALBERT reacts to pruning can be drawn by considering independent pruning thresholds for embedding and transformers.
The embedding layer accounts for approximately 39.9\% of the total model parameters, while the transformer layers contain about 60\%. We use cumulative sparsity as a way of normalizing the effect of pruning the individual blocks on the overall sparsity of the entire model. 
The cumulative sparsity is calculated as shown in Equation \ref{eq:1}, with $S_{c}$ representing cumulative sparsity, $S_{embd}$ and $S_{tf}$ denoting the sparsity for embedding and transformer layers, respectively, and $P_{embd}$ and $P_{tf}$ indicating the ratio of the embedding layer and transformer layers parameters to the entire model size.

\begin{equation}
    \label{eq:1}
    S_{c} = S_{embd} \times P_{embd} + S_{tf} \times P_{tf}
\end{equation}
\begin{figure}[t]
    \centering
    \includegraphics[width=0.48\textwidth]{fig/Sparsity.png}
    \caption{Accuracy comparison for different pruning configurations. The blue and orange lines show the performance trends of the embedded layer and the transformer encoder layers with their isolated pruning cumulative sparsity, respectively. The green star represents the optimal result when combining pruning of both embedding and transformer layers. The gray dots show other pruning combinations with worse accuracy.}
    \label{fig:sparsity}
\end{figure}
\begin{figure*}[ht!]
    \centering
    
\vspace{-1em}
    % \hspace{-1.5em}
    \includegraphics[width=\textwidth]{fig/VASE_scatter_inset.pdf}
    % \vspace{-2em}
    \caption{Performance comparison of unpruned model, CSP-embd, CSP-tf, and CSP-all models using fine-tuning and VASE. VASE ($\bigcirc$) outperforms fine-tuning ($\times$) for MRPC, QQP, and RTE, while the two approaches yield similar results for MNLI, QNLI, and SST-2.}
    \label{fig:vase}
\end{figure*}
Based on cumulative sparsity calculated from Eq.~\ref{eq:1}, we show results on accuracy and sparsity level for embedding, transformers, and embedding and transformers together, as summarized in Figure \ref{fig:sparsity}. 
All pruning experiments are executed using the MNLI fine-tuned backbone and adapters.
Pruning only a backbone's embedding layer (blue curve) results in a significant accuracy degradation while providing only moderate sparsity levels around 20\%. These results indicate the high sensitivity of the embedding layer to pruning. 
The transformer layers show a slightly higher resilience to pruning (orange line), with the accuracy dropping below our baseline around 40\% sparsity. This results are in line with findings in literature~\cite{gordonCompressing2020}.
The gray dots show different combinations of embedding-transformer pruning with individual sparsity levels set between 0\% and 90\%. The green star represent the best combination that allows to achieve a cumulative CPS of 50\%, a clear improvement over applying pruning only to embeddings or transformers, which would lead to a CPS of 18.8\% and 41.2\%.
These results indicate that the adapter-ALBERT model is more resilient to embedding layer's sparsity, as the non-fixed layers in the transformer layers are able to learn and compensate these information loss.

\begin{table*}[ht!]
    \caption{Model parameter comparisons in terms of total parameters, trainable parameters, and parameter overhead percentage.}
    \label{tab:model-size}
    %\vskip 0.15in
    \begin{center}
    \begin{small}
    \begin{sc}
        \begin{tabular}{cc|ccc}
            \toprule
            Model Name & Adapter Size & Total Params & Trainable Params & Overhead \\
            \midrule
            BERT            & N/A           & 110M      & 110M      & 0\% \\
            Adapter-BERT    & 64            & 111.2M    & 2.4M      & 1.1\%   \\
            ALBERT          & N/A           & 11.6M     & 11.6M     & 0\% \\
            Adapter-ALBERT  & 64            & 11.88M   & 200.4K     & 2.4\%    \\
            Adapter-ALBERT  & 32 $\sim$ 128 & 11.77M $\sim$ 12.06M   & 100K $\sim$ 400K   & 1.4\%$\sim$3.9\%   \\
            \bottomrule
        \end{tabular}
    \end{sc}
    \end{small}
    \end{center}
\end{table*}

\subsubsection{\textbf{Variable Adapter Size Evaluation}}
While our experiments have shown that the combined pruning of embedding and transformer layers is more conducive to higher CPS values, we still need to verify the multi-task learning capabilities of the pruned backbone. In the next set of experiments, we use four different backbones which we define as follows:
\begin{itemize}
 \item \textbf{Unpruned}: The original un-pruned pre-trained model.
 \item \textbf{CSP-embd}: The model with only embedding layer pruned to the corresponding CSP.
 \item \textbf{CSP-tf}: The model with only transformer layers pruned to the corresponding CSP.
 \item \textbf{CSP-all}:The model with both embedding layer and transformer layers pruned to the corresponding CSPs.
\end{itemize}

We evaluate these backbone variants over six datasets from the GLUE benchmark to highlight three distinct NLP tasks: MRPC and QQP for paraphrasing, SST-2 for sentiment analysis, and MNLI, QNLI, and RTE for natural language inference. To better assist the adapter-ALBERT model in learning this diverse set of tasks we employ a variable adapter size which can be set for each module individually, choosing a value between 32, 64, or 128. This strategy is also compared against the vanilla ALBERT pruned model in which the transformer layers are fine-tuned for each of the sample datasets. The impact of the variable adapter size evaluation (VASE) strategy, can be observed on each model in terms of accuracy performance is presented in Figure \ref{fig:vase}. 

As a general trend, VASE provides significant accuracy improvements on most datasets and the ability to compensate for accuracy loss in pruned models compared to fine-tuning. 

% For the unpruned model, the majority of the datasets considered in the evaluation require a VASE configuration larger than the default size (64+64), which requires to introduce an additional parameter overhead to preserve inference accuracy. 
% For the CSP-embd and CSP-tf models, smaller configuration can be achieved.
% For the CSP-all model, almost all optimal VASE combinations are larger than the default 64+64 configuration, which can be imputed to the need for compensating increased sparsity in both embedding layer and transformer layers.

For the unpruned model, almost all of the datasets considered in the evaluation require an adapter size larger than the default setting (64+64), which requires to introduce an additional parameter overhead to preserve inference accuracy.
For the three pruned backbones (CSP-embd, CSP-tf, and CSP-all) the behavior is influenced by the specific task. Nonetheless, we can observe more tasks requiring larger adapter sizes going from CSP-tf to CSP-embd to CSP-all. Although not conclusive, this trend is in line with the growing sparsity level and the need to implement larger adapter sizes to compensate for increasing information loss.

%Comparing optimal VASE scores with the two baselines of directly fine-tuning pruned and un-pruned vanilla ALBERT models, it can be observed that fine-tuning pruned backbone models could receive unacceptable penalties on certain datasets.
%For example, the baseline model, the CSP-tf model, and the CSP-both model on QQP dataset is reporting zero or extremely low scores.

%Overall, VASE has shown positive results in compensate pruned models.
%As VASE has shown obvious task-specific characteristics, a selection strategy is needed to provide optimal VASE settings for best results and parameter overhead reduction merits.

Figure~\ref{fig:vase} shows the comparison of the unpruned and the three pruned backbones trained using VASE against the finetuned version of the model. The baseline thresholds are derived from the results in Table~\ref{tab:model-acc}. The added flexibility in setting different adapter sizes provides a way to combat the backbone bias introduced by fine-tuning the model using the MNLI dataset during the pruning stage. Notably, VASE can recover accuracy loss even in extreme cases such as the finetuned version of QQP, where the accuracy drops below 15\%. Comparing the different backbones in terms of accuracy, CPS-all shows comparable or better results against CPS-embd. CPS-tf has the best accuracy performance across all pruned backbones. As we will show in more detail in the next section, the choice of backbone does not affect the SRAM memory requirements, however selecting CSP-tf over CSP-all would increase the on-chip RRAM capacity by 29.7\%. 

%Considering the outstanding MTI capability, both adapter-BERT and adapter-ALBERT can keep each and every dataset at competitive accuracy with trivial overhead costs in stead of fine-tuning vanilla models every time for new tasks or store multiple copies of them.
%However, our adapter-ALBERT model has only 11.9M parameters v.s. the adapter-BERT model's 111.2M and only 0.2M parameters more than the smallest vanilla ALBERT model in comparison, under the adapter size setting 64+64.

%The four selected CSP models (baseline, CSP-embd, CSP-tf, and CSP-all) are re-trained on the six GLUE datasets with their adapter sizes sweeped through VASE for each dataset.
%The results of optimal VASE accuracy and baseline comparisons are shown in Figure \ref{fig:vase}.


% As a general picture of all six VASE and GLUE sweep results, we can observe that for the MNLI dataset, re-training the adapters on it always shows worse accuracy than directly fine-tuning on the MNLI dataset. Since the backbone model is already fine-tuned based on the MNLI dataset, it is more straightforward to infer it directly using the backbone model without adding extra adapter modules. This can lead to a more accurate and overhead-free result. For all other datasets, the intervention of the adapters module leads to good results. In many cases, the best combination of adapter sizes can obtain accuracy very close to or beyond those obtained by directly fine-tuning.


% We first use NVMExplorer to estimate the on-chip area requirements of both SRAM and RRAM arrays based on the total number of fixed parameters after pruning with cumulative sparsity evaluation.
% We also consider the worst-case scenario in terms of classifier labels and adapter sizes across all inference tasks to ensure enough storage for any possible cases.
% As a result, the area requirement does not change with the numbers of tasks, and new task-specific parameters are loaded from DRAM.
% We then consider the total number of compute cycles based on the 256 MAC units provided by the accelerator, and NVMEXplorer's memory read and write energy and latency estimates to produce energy and latency results.
% The total energy consumption is the summation of all memory components' own consumption, and the total latency is their long pole.
% Following this rule, multiple possible design choices are generated.
% We perform a memory design exploration study and pick an optimal solution balancing the trade-off between energy and latency while pursuing the smallest area, presented in Figure \ref{fig:PPA}.
% The results are comparison between the adapter-ALBERT and vanilla ALBERT models running 3-task MTI.
% The vanilla ALBERT model is evaluated following the same settings: its embedding layers are identically pruned as the adapter-ALBERT model, and is bit-mask encoded and stored in SLC and MLC RRAM.
% Similarly, we considered its worst-case with the largest number of parameters and non-linearity's across all layers.


% The memory architecture (Fig. 2) consists of a heterogeneous dedicated global buffer for the accelerator and off-chip DRAM. The accelerator (Tambe et al., 2020) can access these memories using DMA through a common memory interface. The sparse shared parameters are stored using bit-masks and non-zero values in single-level and multi-level RRAM arrays, respectively. This is done to prevent accuracy degradation due to faults in multi-level RRAM.  Task-specific parameters and activations are stored in SRAM. For multi-task inference, the accelerator reuses the parameters stored in RRAM and updates the task-specific parameters reloading the values from DRAM. 

% Results in figure 6 compare running 3-task MTI with our proposed solution and an implementation running a vanilla ALBERT model. For the latter, we still store embeddings in RRAM and consider an on-chip SRAM for the largest requirement in terms of model parameters and activations across all layers.  
% As explained in Section IIB, the adapter-ALBERT model is mapped onto three memory devices: non-fixed parts on SRAM, backbone non-zero values on MLC RRAM, and backbone bitmasks on SLC RRAM. 
% Similar to the MEMTI architecture \cite{Donato19} that integrates with NVDLA, NVSim \cite{Dong12}, and DRAM estimates, our HMA collaborates with EdgeBERT \cite{Tambe20} and NVM Explorer \cite{Pentecost21} simulations for power, performance, and area results.
% We develop layer-by-layer dataflow models for both adapter-ALBERT and vanilla ALBERT to generate RRAM, SRAM, and DRAM traffic analysis, memory access cycles, and multiple-and-accumulate operation cycles based on EdgeBERT specifications and NVM Explorer simulations.
% The integration with NVM Explorer also allows us to generate multiple design choices including different combinations of RRAM and SRAM memories that optimized for area, energy, and bandwidths.
% \fixme{The hardware models for adapter-ALBERT and vanilla ALBERT are decoupled with tasks to ensure scalability. 
% The downstream tasks have no influence towards dataflow and traffic between each layer of the model and memories. 
% MTI capability is also considered in the models with two options for SRAM to either expand to store multiple tasks or maintain storage while constantly refreshing for different tasks.
% The former option changes the SRAM area estimates and the latter option changes the energy requirements for data reads and writes.}

% The EdgeBERT framework \cite{Tambe20} proposes a primitive heterogeneous memory architecture that stores the ALBERT model's embedding layer leveraging RRAM device with quantization and bitmask encoding techniques, while the transformer layers and classifier are stored in a normal SRAM device. We expand the architecture in accordance with the MEMTI framework \cite{Donato19} with the open-source NVDLA accelerator design, to develop the novel heterogeneous memory architecture model. By identifying the data flow of the vanilla ALBERT and adapter-ALBERT models, the memory architecture is capable to generate precise layer-by-layer RRAM, SRAM, and DRAM traffic reports and corresponding memory access cycles and multiple-and-accumulate operation cycles from the EdgeBERT framework. By flexible integration with NVM Explorer \cite{Pentecost21}, a simulation tool evolved from NVSim \cite{Dong12}, we could further obtain optimal RRAM and SRAM models with targeted memory capacity, design optimization targets, and traffic types. Utilizing RRAM and SRAM models read and write bandwidth, energy consumption, and latency, we are able to generate energy reports and specific memory designs for vanilla ALBERT and adapter-ALBERT models.

% To keep consistent with the EdgeBERT framework, we implement quantization of 8-bit and 16-bit data representation models and bitmask encoding techniques to our heterogeneous memory architecture. Similar to EdgeBERT, we utilize 2 bit-per-cell RRAM devices to store the non-zero values and 1 bit-per-cell RRAM for bitmask matrix. With the comprehensive hardware models, we are able to identify the energy and latency bottleneck of the adapter-ALBERT model and apply different memory designs accordingly. MIT capability is also integrated with three tasks chosen from the three different categories of the GLUE benchmark, to demonstrate and evaluate the adapter-ALBERT model's MTI characteristics with performance, power, and area reports.

\section{Hardware optimizations for edge deployment}
%\subsection{Model Performance Evaluation}

% \subsection{Critical Pruning Threshold Exploration}
%\subsection{Cumulative Sparsity Evaluation}

%\subsection{Variable Adapter Size Evaluation}

\subsection{Quantization}

While floating point values are still the most common data representation for DNN training, reduced-precision numerical formats are a good compromise for targeting efficient inference~\cite{hubaraBinarized2016,zhuTrained2017,reagenMinerva2016}, since quantized operands can significantly decrease data storage and movement costs, as well as reduce the complexity of the hardware used for implementing MAC processing units~\cite{carmichaelPerformanceEfficiency2019}. 
%By carefully balancing this trade-off, DNN models can be deployed more efficiently on energy- and storage-limited hardware without losing data accuracy. 
The vanilla ALBERT model has been demonstrated to benefit from quantization down to 16-bit floating-point without negatively impacting its inference performance~\cite{tambeEdgeBERT2021a}. 
For our adapter-ALBERT model, we hypothesize that, given the adapter module's proven ability to compensate for accuracy loss caused by pruning, it may also exhibit similar trends for quantization.

We have designed experiments with quantization configurations in 16-bit and 8-bit fixed-point representations using the CSP-both backbone. Using the conventional notation, $Q_{i,f}$, to represent the quantization scheme using $i$ bits for integer and sign, and $f$ fractional bits, we focus on $Q_{3, 13}$ and $Q_{3, 5}$.  
%Introducing a similar notation as the one used to denote the quantization scheme, we use an adapter size of $A_{64,64}$, where the two numbers in the suffix represent the size of the first and second adapter module in the transformer layer. 
To ensure data consistency, the adapter modules are quantized using the same settings as the backbone. The goal for these experiments is to show to what extent the ability of adapter modules in improving or maintaining the inference accuracy is limited by using lower precision quantization. Moreover, we want to verify if adapter modules can learn reduced data information and recover the loss via retraining of the task-specific parameters. 


\begin{figure}[t]
    \centering
    \includegraphics[width=8.5cm]{fig/qt.pdf}
    \caption{Quantization results on the CSP-all backbone. Although the $Q_{3,13}$ quantization example provides competitive results across all the considered datasets, reducing the number of bits per operand to $Q_{3,5}$ shows a drastic accuracy reduction for some of the QNLI and SST-2 tasks.}
    \label{fig:qt}
\end{figure}

%During the experiments, we have observed unacceptable accuracy degradation when applying quantization methods directly onto the model before inference. 
%As both the model's backbone layers and non-fixed layers are quantified, we speculate that as the information stored in the adapter modules' limited parameter matrices are full-precision data, the adapter modules are very sensitive to any data accuracy deduction caused by quantization if no extra actions are taken.
%We hence re-train the adapter modules while quantifying the entire model to ensure the adapter modules could receive and parameterize information from quantified matrices from the backbone layers.
As shown in Figure~\ref{fig:qt}, the $Q_{3,13}$ configuration provides competitive accuracy results when compared with the single-precision floating point (FP32) baseline. 
On the other hand, the results for $Q_{3, 5}$ quantization are much less consistent, suggesting that the optimal quantization scheme will dependent on the subset of tasks used by an application. 

\begin{table*}[ht]
    \caption{Memory Footprint Breakdown for Vanilla ALBERT and Adapter-ALBERT}
    \label{tab:mem-req}
    %\vskip 0.15in
    \begin{center}
    \begin{small}
    \begin{sc}
        \begin{tabular}{c|c|ccc}
            \toprule
            Model & Quant Config & MLC RRAM & SLC RRAM & SRAM  \\
            \midrule
            \multirow{3}{*}{Vanilla ALBERT}  & FP32 & 3.73MB   & 0.47MB & 9.03MB    \\
                                                & $Q_{3, 13}$ & 1.87MB   & 0.47MB & 4.52MB    \\
                                                & $Q_{3, 5}$ & 0.94MB    & 0.47MB & 2.26MB  \\
            \midrule
            \multirow{3}{*}{Adapter-ALBERT}  & FP32 & 11.13MB   & 1.4MB & 0.43MB    \\
                                                & $Q_{3, 13}$ & 5.57MB   & 1.4MB & 0.22MB    \\
                                                & $Q_{3, 5}$ & 2.79MB    & 1.4MB & 0.11MB  \\
            \bottomrule
        \end{tabular}
    \end{sc}
    \end{small}
    \end{center}
\end{table*}

\subsection{Bitmask Encoding}
Leveraging pruning to improve storage density requires an additional step in how the sparse matrices are mapped into the storage system. Several sparse encoding techniques have been proposed in the past and previous work has highlighted how critical it is to guarantee the robustness of these data structures against faults~\cite{pentecostMaxNVM2019}.  
Among the existing techniques, bitmask encoding is a lightweight approach that can be implemented with minimal encoding and decoding hardware overhead. The non-zero values from the sparse matrix are saved in an ordered array and their location is mapped to a binary matrix. At this point that we apply pruning exclusively to the backbone layers. For this reason, it would be advantageous to store the sparse layers in RRAM. While easy to implement, this solution is susceptible to large errors if any of the bits in the bitmask is flipped. To address this issue and preserve the advantage of the density of RRAMs, we split the bitmask and non-zero value data structures to SLC and MLC RRAM arrays respectively. 

\subsection{Accelerator architecture modeling}
To verify the expected improvements introduced by the adapter-ALBERT model optimizations and the associated on-chip memory architecture, we extrapolate the overall system area, energy, and latency by combining results from NVMExplorer~\cite{pentecostNVMExplorer2021} with a performance model tailored around the EdgeBERT accelerator specifications. In order to minimize the on-chip area overhead and be able to deploy our solution onto a mobile SoC, we select a combination of memory macros with different capacities so that the overall memory size is at the closest value exceeding the application requirements. 
When multiple combinations of memory macros are possible, we evaluate each proposal according to their latency and energy consumption. Since macros of different sizes will display different bandwidths, we compute the overall bandwidth as the weighted average of the bandwidth based on the macro's individual capacity. Off-chip memory access to DRAM are modeled using a similar approach to the one introduced in TETRIS~\cite{gaoTETRIS2017}. In our model, we ignore the energy contribution for the computational units, noting that, with the exception of the adapter computations, the two models will perform the same operations.


The baseline for our evaluations is based on the same accelerator and memory architecture, but for the latter, the capacity requirements consider a different model partition in which only the embedding parameters are stored in RRAM, while the rest of the data uses SRAM. As with adapter-ALBERT, the vanilla ALBERT case uses bitmask encoding for the sparse embeddings. The corresponding memory footprint for these two design options under different quantization configurations are shown in Table \ref{tab:mem-req}.
The model parameters partition of adapter-ALBERT introduces higher storage requirements for SLC and MLC RRAM compared to vanilla ALBERT. This is due to the fact that a larger number of parameters are shared across tasks and therefore are kept in the non-volatile portion of the memory. As a consequence, even though we introduce a parameter overhead associated with adapters, the SRAM storage is reduced by a larger factor. The overall effect is that we can take advantage of the denser RRAM storage for a larger portion of the model, leading to a smaller memory footprint.


The workload scenario we consider is that of a MTI application. The on-chip memory stores only the parameters required to process the current task, and the system performs a parameter update from DRAM whenever a new task needs to be executed. Therefore, we provision the on-chip memory footprint based on the worst-case scenario, \textit{i.e.}, the largest set of parameters for any of the given tasks. Note that this corresponds to the dataset with the largest number of classification labels for the vanilla ALBERT case, and in addition,  
%must consider the largest adapter size from applying VASE to the adapter-ALBERT case.
must consider the largest adapter size adopted by any of the target tasks when using VASE in the adapter-ALBERT case.

%By calculating the model's total number of parameters, removing parameters that are repeatedly reused, and considering sparsity and data representation formats, we obtain the actual memory capacity requirements for RRAMs and SRAM. We then use NVMExplorer to estimate their area, read and write bandwidth, energy, and latency. 

Under this strategy, we compare adapter-ALBERT and vanilla ALBERT's area, energy per inference, and latency per inference under three quantization configurations~\ref{fig:PPA}.
The results are normalized to the FP32 vanilla ALBERT design.
We can observe that for all configurations, the adapter-ALBERT provides all-round advantages against vanilla ALBERT.
% For 8-bit quantization configuration where the two models have the least gaps, the adapter-ALBERT requires 64.9\% area, 1.52\% energy per inference, and 6.37\% latency compared to the vanilla ALBERT model.
% For 32-bit and 16-bit configurations, the advantage of adapter-ALBERT with the memory architecture is more significant.
Compared to the FP32 implementation under a 3-task MTI scenario, adapter-ALBERT enjoys 2.04$\times$, 146.78$\times$, and 2.46$\times$ reductions in area, energy per inference, and latency per inference. The advantage is even more significant when it comes to 16-bit and 8-bit quantified comparisons. For instance, using the $Q_{3, 5}$ configuration, leads to 5.9$\times$, 682$\times$, and 62$\times$ improvements in area, energy and latency.


\begin{figure}[ht]
    \centering
    \includegraphics[width=0.48\textwidth]{fig/ppa.pdf}
    \caption{Area, energy/inference, and latency comparison between adapter-ALBERT and vanilla ALBERT using FP32, $Q_{3,13}$, and $Q_{3, 5}$ data types. The results are normalized to the FP32 vanilla ALBERT design.}
    \label{fig:PPA}
\end{figure}

\section{Related Work}
% \fixme{More papers to be added later}
% The emergence of BERT model \cite{Devlin18} based on Transformer \cite{Vaswani17} architecture has become a milestone in the NLP domain. Many variants of BERT model have explored and expanded the performance envelope of NLP models in multiple dimensions. ALBERT \cite{Lan19}, as a slim BERT variant, could achieve comparable inference accuracy as its predecessor with only 1/10 numbers of parameters. The concept of transfer learning proposed by \cite{Weiss16} has significantly reduced the training cost of large-scale CV models. Among different transfer learning techniques, the residual adapter modules proposed by \cite{Rebuffi17, Rebuffi18} have drawn researchers' attention by showing the potential of enabling heterogeneous hardware storage platforms of traditional SRAM and eNVM devices for multi-task learning. The adapter-BERT model by \cite{Houlsby19} has shown the feasibility that BERT-variant models for NLP tasks could also enjoy the benefits of the adapter modules. Other works, including \cite{bertnpals} have shown other possible paths of enabling efficient MTL capability in the NLP models.

% Data compression methods, including pruning, clustering, and quantization, have been adequate to reduce the storage requirements of DNN models while maintaining their accuracy. We have utilized magnitude pruning in our model, and \cite{Gordon20} has categorized the effects of magnitude pruning on the NLP models as a cross-check to our pruned model behaviors.
% % Here needs Quantization-related works.
% Many have proposed other DNN model compression methods. \cite{Liu17}, \cite{Wen16} have explored channel-level sparsity on CNN models.


% \textbf{Transfer Learning and Multi-Task Learning in Computer Vision -}
\textbf{Multi-Task Learning in Computer Vision -}
Transfer learning DNN models pre-trained on ImageNet~\cite{ImageNetDeng09} has been a common approach for efficiently training new tasks in computer vision. However, preserving inference accuracy requires to fine-tune a large fraction of the model. Residual adapter modules~\cite{ResidualRebuffi17, ResidualRebuffi18} have been introduced as alternative and more efficient way of achieving multi-task learning. The Squeeze-and-Excite blocks proposed by Hu et al.~\cite{SEHu17} presents a similar bottle-neck module as residual adapters for channel-wise feature extraction and adaptation. 

Alternative multi-task inference approaches generate multiple copies of the same model which are aggressively compressed to limit the overall model size. Examples of this approach include network slimming~\cite{NetworkSlimmingLiu17} and structured sparsity learning~\cite{SSLWen16}. The latter approach learns the sparsity from a complex DNN model to accelerate DNN inference. TinyML-inspired projects such as SqueezeNet~\cite{SqueezeNetIandola16}, TinyTL~\cite{TinyTLCai20}, and MCUNet~\cite{MCUNetLin20}, have sought ways to compress the model size to fit in microcontroller-based platforms.

\textbf{Natural Language Processing -}
The field of natural language processing has been rapidly advancing since the proposal of attention mechanism along with the Transformer model~\cite{TransformerVaswani17, TransformerLiu18}. Edge-cutting attention-based NLP models, including BERT~\cite{BERTDevlin18} and GPT~\cite{GPTRadford18, GPTBrown20}, have brought significant performance improvement over traditional CNN, RNN, and LSTM-based language models and challenges of local deployments due to massive computational and data movement costs from their uncontrollable sizes. Slimmer BERT variants, like ALBERT~\cite{ALBERTLan19}, TinyBERT~\cite{TinyBERTJiao19}, and RoBERTa~\cite{RoBERTaLiu19}, have introduced multiple structural and data compression optimizations to reduce their sizes while keeping acceptable performance. Additionally, Gordon et al.~\cite{PruningBERTGordon20} have categorized the weight pruning effects on the BERT model, indicating the potential boundary of pruning our adapter-ALBERT model.

Transfer learning is also closely studied in the NLP domain. The adapter-BERT model proposed by Houlsby et al.~\cite{AdapterBERTHoulsby19} shows feasibility of transplanting adapter modules from CV domain to NLP domain for efficient training compared to traditional fine-tuning methods. Stickland et al.~\cite{BERTPALsStickland19} utilize project attention layers across their BERT-PALs model as an alternative task-adaption approach for a lower parameter overhead compared to adapter-BERT model (1.13$\times$ vs 1.3$\times$). However, this model, like many of the alternatives discussed in this section, requires to update all parameters which imposes a heavy toll on the memory bus.


\textbf{Embedded Non-Volatile Memories -}
To match the needs of data-intensive applications, new memory technologies have been proposed since the memristors was first hypotesized in 1971~\cite{MemristorChua71}. After the first demonstrated physical implementation in 2008~\cite{MemristorBreakthroughStrukov08}, the resistive random access memory (RRAM) gradually became one of the most promising eNVM technologies due to its scalability~\cite{RRAMDesignDeng13}, storage density~\cite{RRAMDensityDeng13}, competitive read performance~\cite{RRAMWriteSheu11}, and CMOS compatibility~\cite{RRAMCMOSTanachutiwat11}. Collective studies and reviews of RRAM and other eNVM technologies are performed by Panda et al.~\cite{CollectiveRRAMPanda18} and Chen et al.~\cite{ENVMReviewChen16}.

DNN storage architecture realizations utilizing RRAM have been closely studied. Donato et al. have presented an on-chip memory optimization utilizing SRAM and RRAM for CV multi-task inference \cite{MEMTIDonato19}, which covers the shortcoming of RRAM's write costs by assigning frequently-updated parameters of an adapter-equipped ResNet model onto the SRAM portion of the design. 

% no need to refer to NVDLA/EdgeBERT papers?

% The EdgeBERT accelerator proposed by~\cite{EdgeBERTTambe20}.

%NVDLA
%DRAM model

\section{Conclusion and Limitations}
\label{sec:conclusion}
In conclusion, we have shown that our method of pre-training local features on rigid 3D scenes can generalize well to new and unseen classes of deformable organic shapes, enabling effective performance in various shape analysis tasks. Our study has highlighted the importance of selecting the right receptive field size to ensure feature transferability, which has led to the \textit{first general-purpose local feature pre-training}  for deformable shape analysis tasks. This research also sheds light on the relationship between rigid and non-rigid processing tasks, providing a link between two fields that have traditionally used different tools.

One limitation of our method is its reliance on differentiable voxelization, which can be memory and time-consuming, particularly during pre-training. Nonetheless, our results outperform PointContrast \cite{xie2020pointcontrast}, a point-based method that requires \textit{more training data} and has limited generalizability. Another limitation is that our features rely on LRF estimation, which might lack robustness to thin structures or boundaries of partial shapes. Exploring alternative scalable and robust local feature pre-training strategies is an fascinating direction for future work.

\mypara{Acknowledgements}
The authors would like to thank the anonymous reviewers for their valuable suggestions. 
Parts of this work were supported by the ERC Starting Grant No. 758800 (EXPROTEA) and the ANR AI Chair AIGRETTE.









% In this work, we demonstrated that local features trained for rigid alignment of 3D scenes can generalize remarkably well to new unseen classes and especially deformable organic shapes in a wide range of shape analysis tasks. For this, we first showed the critical role that the receptive field size plays in the transferability of local features and proposed an optimization strategy to enable feature transfer across significantly different shape classes. 

% Our approach leads to the \textit{first general-purpose local feature pre-training} method that is applicable in deformable shape analysis tasks. Remarkably, our learned features enable tasks such as generalizable unsupervised shape matching without relying on shape pre-alignment. Our work also sheds light on the utility of low-level features in 3D transfer learning and creates an interesting link between rigid (3D scene or man-made object) shape analysis and non-rigid processing tasks -- two fields that have traditionally been considering very different tools.

% Perhaps the biggest limitation of our work is that it relies on differentiable voxelization and is thus relatively memory and time-consuming, especially during pre-training. Nevertheless, we obtain better results than PointContrast \cite{xie2020pointcontrast} that, despite being point-based, requires more training data and
% has limited generalizability across domains.

% \mypara{Acknowledgements}
% The authors would like to acknowledge the anonymous reviewers for their valuable suggestions. 
% Parts of this work were supported by the ERC Starting Grant No. 758800 (EXPROTEA) and the ANR AI Chair AIGRETTE.

% \paragraph{Societal impact}
% Efficient methods for non-rigid shape analysis have immediate impact in many
% areas of science and engineering from medical imaging
% (for instance for detecting anomalies, and performing follow-up analysis) to shape recognition and classification in areas such as computational biology, archaeology and paleontology to name a few. Our approach can immediately be adapted and tested in such diverse scenarios, due to the strong generalization power of the proposed descriptors. Our work also opens major avenues for future research as it can facilitate geometric deep learning methods without training, thus potentially enabling small labs to do research in this field without having big clusters or access to large-scale datasets. Finally we note that avoiding extensive training for each application also reduces the environmental impact of geometric deep learning, by significantly reducing the computation requirements for each independent application.



\bibliography{main}
\bibliographystyle{hieeetr}

\end{document}
