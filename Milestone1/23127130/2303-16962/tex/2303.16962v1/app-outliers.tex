%\subsection{Misidentified Gaussian Peaks}
\subsection{Misplaced Gaussian Peaks}

\begin{figure}
    \centering
    \includegraphics[width=0.48\linewidth]{fit-hist-frequencies-outlier-u25.8_BHBlp_histogram.pdf}
    \includegraphics[width=0.48\linewidth]{fit-hist-frequencies-outlier-u25.8_BHBlp_evolution.pdf} \\
    \includegraphics[width=0.48\linewidth]{fit-hist-frequencies-outlier-s18.4_BHBlp_histogram.pdf}
    \includegraphics[width=0.48\linewidth]{fit-hist-frequencies-outlier-s18.4_BHBlp_evolution.pdf} \\
    \includegraphics[width=0.48\linewidth]{fit-hist-frequencies-outlier-u24.0_DD2_histogram.pdf}
    \includegraphics[width=0.48\linewidth]{fit-hist-frequencies-outlier-u24.0_DD2_evolution.pdf}
    \caption{
    Left column: Histogram of the $f$-mode frequencies and the corresponding Gaussian mixture fit for models \code{u25.8_BHBlp} (top), \code{s18.4_BHBlp} (middle), and \code{u24.0_DD2} (bottom).
    Right column: Time evolution of the $f$-mode frequencies between 0.2 and 0.7 s post-bounce for models
    The discontinuity in $f$-mode frequencies is seen at ${\sim} 0.36$ seconds post-bounce (top), at ${\sim} 0.6$ seconds post-bounce (middle), and at ${\sim} 0.6$ seconds post-bounce (bottom). 
    Horizontal lines identify the $\mu_1$ and $\mu_2$ frequencies from the Gaussian mixture fit for each model.
    }
    \label{fig:poorly-fit}
\end{figure}


% \begin{figure}
%     \centering
%     \includegraphics[width=\linewidth]{figures/app-outliers/fit-hist-frequencies-outlier-u25.8_BHBlp.pdf}
%     \caption{
%     Top: Histogram of $f$-mode frequencies and the Gaussian mixture fit to this histogram for our \code{u25.8_BHBlp} model.
%     Bottom: Evolution of the $f$-mode from 0.2 to 0.7 seconds post-bounce for the same model, with the location of the low- and high-frequency Gaussian peaks identified by the Gaussian mixture fit noted with solid horizontal lines.
%     Here, there is a discontinuity in the time evolution of the frequencies at ${\sim} 0.36$ seconds post-bounce, which corresponds to the peak in the frequency histogram at ${\sim} 825$ Hz.
%     However, this peak is less significant than the remaining features in the data at higher frequencies, and only reduces the confidence in the location of $\mu_1$.}
%     \label{fig:poorly-fit-ok-id}
% \end{figure}

% \begin{figure}
%     \centering
%     \includegraphics[width=\linewidth]{figures/app-outliers/fit-hist-frequencies-outlier-s18.4_BHBlp.pdf}
%     \caption{
%     Top: Histogram of $f$-mode frequencies and the Gaussian mixture fit to this histogram for our \code{s18.4_BHBlp} model.
%     Bottom: Evolution of the $f$-mode from 0.2 to 0.7 seconds post-bounce for the same model, with the location of the low- and high-frequency Gaussian peaks identified by the Gaussian mixture fit noted with solid horizontal lines.
%     Here, there is a discontinuity in the time evolution of the frequencies at ${\sim} 0.6$ seconds post-bounce, which corresponds to the peak in the frequency histogram at ${\sim} 850$ Hz.
%     This peak appears potentially significant, and so here the fitting algorithm places $\mu_1$ at the location of a discontinuity in the time-frequency evolution.}
%     \label{fig:poorly-fit-bad-id}
% \end{figure}

% \begin{figure}
%     \centering
%     \includegraphics[width=\linewidth]{figures/app-outliers/fit-hist-frequencies-outlier-u24.0_DD2.pdf}
%     \caption{Top: Histogram of $f$-mode frequencies and the Gaussian mixture fit to this histogram for our \code{u24.0_DD2} model.
%     Bottom: Evolution of the $f$-mode from 0.2 to 0.7 seconds post-bounce for the same model, with the location of the low- and high-frequency Gaussian peaks identified by the Gaussian mixture fit noted with solid horizontal lines.
%     Here, there is a discontinuity in the time evolution of the frequencies at ${\sim} 0.2$ seconds post-bounce, which corresponds to the peak in the frequency histogram at ${\sim} 915$ Hz.
%     This peak is large compared to the other features of the data, and so the fitting algorithm places $\mu_1$ to fit this feature.
%     While discontinuous, it lies near the distribution of the rest of the frequencies.
%     Appearing significant and less extreme, this peak is confidently fit, and so the error in $\mu_1$ is low.}
%     \label{fig:poorly-fit-dd2}
% \end{figure}

Despite the ability of a two-component Gaussian Mixture to identify characteristic $f$-mode frequencies for the majority of our models, for some models this functional form is a poor fit to the histogram of frequencies and so does not confidently identify two characteristic frequencies.
These poorly-fit models all contain some type of low-frequency discontinuity in their $f$-modes which impacts the confident identification of the Gaussian peak locations $\mu_1$ and $\mu_2$.
These discontinuities fall $\gtrsim 20$ Hz or more below the rest of the time-frequency evolution of the $f$-mode.
Since all of our models are one-dimensional, there is no physical explanation for why the resonant fundamental frequency of the proto-neutron star would change so severely and suddenly, as e.g. phenomena like downflows are not possible.
For a single model (\code{u24.0_BHBlp}) with a clear discontinuity in the $f$-mode, we have double-checked that this feature does not coincide with any abrupt changes in the evolution of the radius, density, enthalpy, and internal energy of the outermost radial zone within the PNS (defined by the $> 10^{11} g/{\rm cm}^{-3}$), nor in the central density of the PNS.
We also did not observe any coincident features in the radial profile of the rest mass, gravitational mass, fluid velocity, density, temperature, lapse, internal energy, potential energy, pressure, adiabatic index, entropy, and hyperon fraction at times near such a non-physical frequency feature. Thus, we conclude that these discontinuities are non-physical, even if we cannot precisely determine their origin.

Among the poorly-fit models, we find three distinct groups.
(i) In some models, any low-frequency discontinuities constitute only a small portion of the total histogram of frequencies, such that they do not change $\mu_1$ and $\mu_2$. However, the discontinuities noticeably increase the uncertainty in the location of $\mu_1$ (some realizations of the fit in the bootstrap procedure place the first Gaussian peak at a lower frequency to capture this discontinuity in the $f$-modes). An example of this behavior is shown in the top row of Figure~\ref{fig:poorly-fit}, which shows a sudden decrease in the $f$-mode frequencies from ${\sim} 940$ Hz down to ${\sim} 825$ Hz, at ${\sim}0.36$ seconds post-bounce. These unphysically low frequencies appear as a small peak at ${\sim} 825$ Hz in the histogram. Note that the $\mu_1$ and $\mu_2$ frequencies are close to the values one would identify by eye, ignoring the discontinuity in the $f$-mode frequencies.

% mis-fit of type 1:
% In Figure~\ref{fig:poorly-fit-ok-id} we show an example of this behavior. In the bottom panel of this Figure, we see a sudden jump from ${\sim} 940$ Hz down to ${\sim} 825$ Hz, at ${\sim}0.36$ seconds post-bounce. In the top panel, this manifests in the frequency histogram as a small peak at ${\sim} 825$ Hz. This feature noticeably increases the uncertainty in the location of $\mu_1$, with some realizations of the fit in the bootstrap procedure placing the first Gaussian peak to capture this discontinuity in the $f$-modes. However, $\mu_1$ is still identified at roughly the same location as it would be if this discontinuity did not exist in the data.

(ii) The low-frequency discontinuities do constitute a large enough fraction of the total histogram of frequencies to change the identification of $\mu_1$ and $\mu_2$. For example, model \code{s18.4_BHBlp} (middle row in Figure~\ref{fig:poorly-fit}) shows a serious discontinuity in the $f$-modes at ${\sim} 0.6$ seconds post-bounce, which manifests as a peak at ${\sim} 850$ Hz in the frequency histogram. Note that this (unphysical) peak is of similar height and width as the (physical) peak at ${\sim} 985$ Hz. In these cases, the Gaussian mixture prefers to place $\mu_1$ at the lowest-frequency peak originating from the discontinuity in many realizations of the bootstrap procedure. This, in turn, lowers the position of the $\mu_2$ frequency to the peak at ${\sim}925$ Hz. These values for $\mu_1$ and $\mu_2$ are lower than what we expect if the discontinuity were not present (we would expect $\mu_1 \sim 925$ Hz and $\mu_2 \sim 985$ Hz).

% poor fit of type 2:
% In the bottom panel of Figure~\ref{fig:poorly-fit-bad-id}, we see a serious discontinuity in the $f$-modes, at ${\sim} 0.6$ seconds post-bounce, which manifests in the top panel as a peak in the frequency histogram at ${\sim} 850$ Hz. Unlike in the example of Figure~\ref{fig:poorly-fit-ok-id}, however, this low-frequency peak is large enough that the Gaussian mixture prefers to place $\mu_1$ to fit this non-physical feature. In particular, there appears to be a peak of similar height and width at ${\sim} 975$ Hz, and so both features appear equally significant, allowing the fitting algorithm to place $\mu_1$ at the frequency of a discontinuity in the time-frequency evolution. This, in turn, lowers $\mu_2$, which is roughly at the location where $\mu_1$ would be placed if not for the large, low-frequency discontinuity, consistent with the 2$\sigma$ confidence region of the fit\footnote{i.e. in some realizations of the data when bootstrapping this confidence interval, the low-frequency discontinuity may be sampled much less than the rest of the data, and so the fit places $\mu_1$ where we might do so by-eye.}.

(iii) The low-frequency discontinuities are a large enough fraction of the overall data to lower the value of $\mu_1$ but are still closer in value to the rest of the $f$-mode frequencies than in the other two cases. 
This increases the statistical confidence in the location of $\mu_1$, as this feature is less extreme in an absolute sense, even if still likely non-physical.
In the bottom row of Figure~\ref{fig:poorly-fit}, we can see an example of this pattern.
Here, the discontinuity at ${\sim} 0.2$ seconds post-bounce yields a low-frequency peak at ${\sim} 915$ Hz in the histogram.
Since this is within ${\sim} 20$ Hz of the next peak in the frequency histogram, the confidence interval around the location of $\mu_1$ is narrow. The resulting value for $\mu_1$ is similar (albeit slightly lower) than what we would expect without the discontinuity. As a consequence, $\mu_2$ is visibly shifted to a lower value than expected without discontinuity. 
% \cf{Wouldn't we expect $\mu_2$ to be higher? Ie only $\mu_1$ is approximately where we would expect it?}
% \noah{Yes, you're right-- mu1 is pretty close to where we might expect it, but still lower than where it would be w/o the discontinuity, so mu2 is also lower than we'd expect. This is better wording for that sentence:}
% \noah{...the confidence interval around the location of $\mu_1$ is much more narrow, even while $\mu_1$ and especially $\mu_2$ are lower than we would expect if the discontinuity were not present.}


%\subsection{Impact on Surface Gravity Correlation}
\subsection{Impact on the Correlation between NS Surface Gravity and GW Frequencies}

\begin{figure}
    \centering
    \includegraphics[width=0.5\linewidth]{mu1err-mu2.pdf}
    \caption{
    Error (1$\sigma$ confidence interval) in the location of the low-frequency Gaussian peak, $\mu_1$, versus the location of the high-frequency Gaussian peak, $\mu_2$, from the Gaussian mixture fit procedure for all models. 
    Points that lie outside the 2$\sigma$ confidence region of the linear fit in Figure~\ref{fig:mu_results} are shown with fully opaque symbols, and all other models are shown with semi-transparent symbols. All models with large $1\sigma$ confidence intervals are associated with one of the three cases of misidentified Gaussian peaks (see text for details).
    }
    \label{fig:mu1err-mu2}
\end{figure}

In the right panel of Figure~\ref{fig:mu_results}, we show a linear correlation (independent of the nuclear EOS) between the surface gravity of the cold, remnant neutron star and $\mu_2$. 
There is a clear subset of models which lie outside the 2$\sigma$ confidence region for this linear fit, consisting mostly of models with the BHB$\lambda \varphi$ EOS and three models with the DD2 EOS. 
We can fully explain these outliers with the three cases of poorly-fit Gaussian mixtures described above. 
In Figure~\ref{fig:mu1err-mu2}, we plot the 1$\sigma$ confidence interval at the location of the low-frequency Gaussian peak, $\mu_1$, against the location of the high-frequency peak, $\mu_2$, from the Gaussian mixture fit procedure for all models. 
The models that lie outside of the $2\sigma$ confidence interval in the right panel of Figure~\ref{fig:mu_results} are shown with opaque symbols. 
% BHBlp models:
Of these, the BHB$\lambda\phi$ models correspond to models with a non-physical discontinuity in the frequency evolution, which ultimately shifts both $\mu_1$ and $\mu_2$ to much lower values than expected without such a discontinuity (see case (ii) above). Identifying $\mu_2$ at a lower frequency than expected shifts the models to the left in Figure~\ref{fig:mu_results}. 
% DD2 models:
The three DD2 models also shown with opaque symbols are examples of case (iii) discussed above, where $\mu_1$ is lower than would be expected without discontinuities, however, the fit is more confident in the location of $\mu_1$. 
% Transparent points with large mu1 error
Finally, there are models (semi-transparent points with large $\mu_1$ error) with a broad $\mu_1$ confidence interval that still lie within $2\sigma$ of the fit in Figure~\ref{fig:mu_results}. These are examples of case (i) discussed above, where the location of $\mu_1$ is not shifted much however the confidence in the location is low, resulting in a broader confidence interval.
%\cf{we need to clean up the use of `error', `interval', `confidence region', etc.}

% Here, we see three distinct groups separate from the rest of our model suite.
% The outliers identified in Figure~\ref{fig:mu_results} are shown fully opaque, and except for three DD2 models, have $\mu_1$ errors that are an order-of-magnitude larger than for the rest of our models.
% This is consistent with the second case of poorly-fit Gaussian mixtures shown in Figure~\ref{fig:poorly-fit-bad-id}.
% For these outlying models, there is some non-physical discontinuity in the frequencies which creates a spurious low-frequency peak, shifting $\mu_1$ and then $\mu_2$ to much lower values.
% By shifting $\mu_2$ down to the approximate value of $\mu_1$ without these discontinuities, these models are moved to the left of the trend line in the right panel of Figure~\ref{fig:mu2_results}.
% The three DD2 models, whose fits appear to be far more confident in the location of $\mu_1$, all follow the third case of poor fitting, where there is a non-physical though less extreme discontinuity in the $f$-modes.
% Finally, we see a collection of transparent points, corresponding to models that lie within the linear fit of the surface gravity versus $\mu_2$.
% The fits for these models are consistent with the first case of poorly-fit Gaussian mixtures, where there is a low-frequency discontinuity in the data that broadens the confidence region for $\mu_1$, but does not consistent a significant-enough fraction of the frequencies to move the Gaussian peaks to much lower values.
