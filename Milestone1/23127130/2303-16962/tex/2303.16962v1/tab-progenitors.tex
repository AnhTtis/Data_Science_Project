\begin{table}    % Table: progenitor models
\begin{center}
	\caption{Progenitor models used in this study.
	} \label{tab:progenitors}
	\begin{tabular}{llllllc}
	\tableline \tableline 
Series & Metallicity    & $M_{\mathrm{min}}$  & $M_{\mathrm{max}}$      & $\Delta m$    & EOS \\
       &  ($Z_{\odot}$) & ($M_{\odot}$)  & ($M_{\odot}$) & ($M_{\odot}$) &   &  \\
	\tableline
   s   & 1 & $10.8$ & $28.2$ & $0.2$ & all \\
       &   & $29.0$ & $40.0$ & $1.0$ & all \\
%
   w   & 1 & $12.0$ & $33.0$ & $1.0$ & all \\
       &   & $35.0$ & $40.0$ & $5.0$ & all \\
	%
   u   & $10^{-4}$ & $11.0$ & $40.0$ & $1.0$ & DD2 \\
   u   & $10^{-4}$ & $11.0$ & $40.0$ & $0.2$ & not DD2 \\
%
   z   & 0         & $11.0$ & $40.0$ & $1.0$ & all \\
%%
	\tableline
	\end{tabular}
\end{center}
\tablecomments{All simulations were performed using six nuclear EOS (SFHo, SFHx, DD2, BHB$\lambda\varphi$, TM1, NL3), for a total of \nsims~ supernova models, of which \nns{} are exploding models and are analyzed in this work. 
Note that the simulations with the DD2 EOS are taken from \citet{push2} and \citet{push4}. In the case of the u-series, the simulations with the DD2 EOS have a larger mass spacing $\Delta m$ than for the other five EOSs. 
The progenitors of the s-, u-, and z-series are from \citet{Woosley.Heger:2002}; those of the w-series are from \citet{Woosley.Heger:2007}. When referencing specific models studied in this work, we will label them as \code{{metallicity series}{ZAMS mass}_{EOS}. }
}
\end{table}