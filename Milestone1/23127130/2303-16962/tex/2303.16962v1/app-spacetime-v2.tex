The Einstein equations of general relativity $G_{\mu\nu}= 8\pi G T_{\mu\nu}$ do not uniquely constrain the metric tensor $g_{\mu\nu}$. One must also specify a \textit{coordinate system} or \textit{gauge} to fully constrain the system. In the 3+1 formulation, the gauge can be expressed as equations for the lapse $\alpha$ and shift vector $\beta$ \citep{alcubierre}. The eigenfrequency analysis of TF19 assumes a \textit{conformally-flat} metric:
%
\begin{equation}
    ds^2 = \left( -\alpha^2 + \beta_i \beta^i \right) dt^2 + 2 \beta_i dt dx^i  + \psi^4\left( \eta_{ij} dx^i dx^j \right)
\end{equation}
%
where latin indices $i,j$ denote spatial coordinates, $t$ denotes a timelike coordinate and $x^i$ denotes spacelike coordinates, $\eta_{ij}$ is the spatial part of the Minkowski metric, and $\psi$ is known as the ``conformal factor''.
The factor $\psi$ may be a constant or a function of spacetime variables.
When $\psi$ is constant in space, the metric is spatially flat. The background metric for the linearized system from TF19 that we solved to compute the gravitational-wave eigenfrequencies of the proto-neutron star is expressed in spherical coordinates that are also isotropic, i.e. shiftless, as,
%
\begin{equation} \label{eq:tf-metric}
    ds^2 = -\alpha^2 dt^2 + \psi^4\left[ dr^2 + r^2 \left( d\theta^2 + sin^2(\theta) d\varphi^2 \right) \right]
\end{equation}
%
where $(r, \theta, \varphi)$ are the usual spherical coordinates of radius, polar angle, and azimuthal angle. In spherical symmetry, this is entirely a gauge choice. The assumption of conformal flatness constrains the lapse and the vanishing shift constrains the shift.

The metric assumed by \code{Agile} is expressed under a different gauge choice, based upon a Lagrangian formulation of general-relativistic hydrodynamics, where
%The metric assumed by \code{Agile} is not expressed in the full ADM formalism, which was developed for application to relativistic hydrodynamics after the creation of \code{Agile} \TODO{is that true?} \jmm{This is definitely not true. The first paper expressing the ADM formalism was published in 1959 when work was focused on understanding whether or not it was possible to write down a well-posed initial-boundary value problem in PDE theory for GR. What \textit{IS} true is that Agile is a \textit{Lagrangian} formulation of the GR hydro eqns, which is why the metric is tied to the fluid four-velocity. This picture is perfectly reasonable as a place to start. It's just different than the ADM picture.} \pablo{[Pablo: I would call it 3+1 formalism instead of ADM, because the ADM formalism refers to the hamiltonian formulation of the Einstein equations, not to the 3+1 split, which was actually introduced before ADM's paper.]} \jmm{Jonah: That's a good point---I agree.}\citeme{}; the \code{Agile} metric is
%
\begin{equation} \label{eq:agile-metric}
    ds^2 = -\alpha^2 dt^2 + \left( \frac{\partial r}{\partial a} \frac{1}{\Gamma} \right)^2 da^2 + r^2 \left( d\theta^2 + \sin^2(\theta) d\varphi^2 \right).
\end{equation}
%
Here, $r$ is the areal radius,\footnote{Such that spheres have surface area $4\pi r^2$.} $a$ is a spatial coordinate tracking the mass of the fluid, and $\Gamma$ is the Lorentz factor, defined as 
%
\begin{equation}
    \Gamma = \sqrt{1 + u^2 - 2m / r}
\end{equation}
%
with a fluid velocity $u = \left( \partial r / \partial t \right) / \alpha$ and enclosed rest mass $m$.
Additional details on the metric used by \code{Agile} can be found in Chapter 7 of \cite{liebendoerfer_thesis_2000}.

The areal radius $r = r(a,t)$ prevents by-eye identification of a conformally-flat decomposition of this metric, as $da$ will decompose into cross terms of $dr$ and $dt$.
To express the \code{Agile} metric in a conformally-flat manner, we introduce a coordinate transformation of the areal radius into a ``conformal radius" $\tilde{r}$, where
%
\begin{equation} \label{eqn:conf_radius_defn}
    \tilde{r} = \Gamma r \frac{\partial \tilde{r}}{\partial r}.
\end{equation}
%
Now, we show that this coordinate transformation yields a metric that is isotropic like Equation~\ref{eq:tf-metric}.
By definition, $\tilde{r} = \tilde{r}(r, t)$, and so
%
\begin{equation} \label{eq:drtilde_initial}
    d\tilde{r} = \frac{\partial \tilde{r}}{\partial r} dr + \frac{\partial \tilde{r}}{dt} dt.
\end{equation}
%
Since $r = r(a,t)$,
%
\begin{equation}
    dr = \frac{\partial r}{\partial a} da + \frac{\partial r}{\partial t} dt,
\end{equation}
%
which we substitute into Equation~\ref{eq:drtilde_initial} to find
%
\begin{equation} \label{eq:drtilde_clean}
    d\tilde{r} = \frac{\partial \tilde{r}}{\partial r} \frac{\partial r}{\partial a} da + \left( \frac{\partial \tilde{r}}{\partial r}\frac{\partial r}{\partial t} + \frac{\partial \tilde{r}}{\partial t} \right) dt.
\end{equation}
%
To rewrite our metric in terms of the new coordinate $\tilde{r}$, we will need to manipulate this expression for $da$ in terms of $d\tilde{r}$.
However, as it currently stands, the additional $dt$ term would induce a cross-term $d\tilde{r} dt$, and thus a shift-full metric, when substituted in for $da$ in Equation~\ref{eq:agile-metric}.
From the definition of our proposed coordinate transformation alone, there is no immediately obvious reason that the $dt$ term should vanish.
Luckily, it turns out to be \textit{approximately} zero for our models.

For the linear perturbation analysis conducted in this work, our integration region is restricted to the proto-neutron star, whose surface is defined by a density of $10^{11}$ g/cm$^3$.
In this region of our models, the fluid velocity in geometric units $u \ll 1$, as our models are spherically-symmetric.\footnote{In higher dimensions, this may not be true due to convection within the proto-neutron star.}
We have inspected a few of our models carefully to confirm this property and found that $u$ is of
$\mathcal{O}(10^{-5})$ within the proto-neutron star, rising to at most values of $\mathcal{O}(10^{-3})$ at the outermost edge of the proto-neutron star.
Since $\partial r/\partial t = u \alpha$, and $\alpha$ is typically order 1 or smaller, we have that $\partial r/\partial t \approx 0$.
In a similar manner, we can argue that $\partial \tilde{r} / \partial t$ is small.
Given the definition of $\tilde{r}$, we can compute via the chain rule that
%
\begin{equation}
    \frac{\partial \tilde{r}}{\partial t} = \frac{\partial \Gamma}{\partial t} r \frac{\partial \tilde{r}}{\partial t} + \Gamma \frac{\partial r}{\partial t} \frac{\partial \tilde{r}}{\partial r} + \Gamma r \frac{\partial}{\partial t}\frac{\partial \tilde{r}}{\partial r}.
\end{equation}
%
The derivative of $\Gamma$ is
%
\begin{align}
    \frac{\partial \Gamma}{\partial t} &= \frac{1}{\Gamma}\left( 2u \frac{\partial u}{\partial t} + \frac{2 m}{r^2} \frac{\partial r}{\partial t} - \frac{2}{r} \frac{\partial m}{\partial t} \right) \approx -\frac{2}{r \Gamma} \frac{\partial m}{\partial t}
\end{align}
%
where we drop the first term as $u$ as small, and the second term as $\partial r / \partial t$ is small.
Generally, if the fluid velocity within and near the proto-neutron star is small, we would expect the mass flux $\partial m / \partial t$ to be small.
So, $\partial \Gamma / \partial t$ is small as well.
Therefore,
%
\begin{equation}
    \frac{\partial \tilde{r}}{\partial t} \approx \Gamma r \frac{\partial}{\partial t}\frac{\partial \tilde{r}}{\partial r}.
\end{equation}
%
By equality of mixed partial derivatives, we can rearrange the second derivatives on the right-hand side, to find
%
\begin{equation}
    \frac{\partial \tilde{r}}{\partial t} \approx \Gamma r \frac{\partial}{\partial r}\frac{\partial \tilde{r}}{\partial t}.
\end{equation}
%
One solution to this equation is that $\partial \tilde{r} / \partial t = 0$.
While there may be other solutions, we have directly computed $\partial \tilde{r} / \partial t$ (as described at the end of this appendix) and confirmed that it is approximately zero within the proto-neutron star (at most $\mathcal{O}(10^{-8})$, again in geometric units).

With $\partial r / \partial t$ and $\partial \tilde{r} / \partial t$ both small, we return to Equation~\ref{eq:drtilde_clean} to conclude that for $\tilde{r}$,
%
\begin{equation}
    d\tilde{r} \approx \frac{\partial \tilde{r}}{\partial r}\frac{\partial r}{\partial a} da,
\end{equation}
%
or alternatively,
%
\begin{equation}
    da \approx \left( \frac{\partial \tilde{r}}{\partial r} \frac{\partial r}{\partial a} \right)^{-1} d\tilde{r},
\end{equation}
%
which we substitute into the \code{Agile} metric, Equation~\ref{eq:agile-metric}, to find
%
\begin{align}
    ds^2 &= -\alpha^2 dt^2 + \left( \frac{\partial r}{\partial a} \frac{1}{\Gamma} \right)^2 \left( \frac{\partial \tilde{r}}{\partial r} \frac{\partial r}{\partial a} \right)^{-2} d\tilde{r}^2 + r^2 \left( d\theta^2 + \sin^2(\theta) d\varphi^2 \right) \\
    &= -\alpha^2 dt^2 + \left( \Gamma \frac{\partial \tilde{r}}{\partial r}  \right)^{-2} d\tilde{r}^2 + r^2 \left( d\theta^2 + \sin^2(\theta) d\varphi^2 \right).
\end{align}
%
We manipulate our definition of $\tilde{r}$ for $\tilde{r} / r$, as well as insert identity $(\tilde{r} / \tilde{r})^2$ into the last term of the metric, so
%
\begin{align}
    ds^2 &= -\alpha^2 dt^2 + \left( \frac{\tilde{r}}{r} \right)^{-2} d\tilde{r}^2 + r^2 \left(\frac{\tilde{r}}{\tilde{r}}\right)^2 \left( d\theta^2 + \sin^2(\theta) d\varphi^2 \right) \\
    &= -\alpha^2 dt^2 + \left( \frac{r}{\tilde{r}} \right)^{2} \left[ d\tilde{r}^2 + \tilde{r}^2 \left( d\theta^2 + \sin^2(\theta) d\varphi^2 \right) \right].
\end{align}
%
Comparing this to the conformally flat, isotropic metric in Equation~\ref{eq:tf-metric}, we can identify the conformal factor as
%
\begin{equation}
    \psi^4 = \left( \frac{r}{\tilde{r}} \right)^2.
\end{equation}
%

At each timestep of the eigenfrequency analysis, we calculate the conformal radius $\tilde{r}$ and conformal factor $\psi$ via the Backwards Eulerian method, and provide both to \code{GREAT} (using $\tilde{r}$ as part of the hydrodynamic background instead of the areal radius $r$ from \code{Agile}).
We manipulate our definition of $\tilde{r}$, at a particular timestep, as
%
\begin{equation}
    \frac{\partial \tilde{r}}{\partial r} = \frac{\tilde{r}}{r} \frac{1}{\Gamma(r)} \label{eqn:conf_diffeq}.
\end{equation}
%
Our outer boundary condition is that, as $r \rightarrow \infty$, we expect $\psi \rightarrow 1$, as we expect space to be flat far from the star.
In theory, we would implement this boundary condition by enforcing that $\tilde{r} = r$, $\Gamma = 1$, and thus $\partial \tilde{r} / \partial r = 1$ in the outermost zone of the star, number 180.
Then, for the remaining zones, we discretize Equation~\ref{eqn:conf_diffeq} as
%
\begin{equation} \label{eqn:discrete_conf_diffeq}
    \tilde{r}_i = \tilde{r}_{i + 1} - \left(r_{i+1} - r_i\right) \left( \frac{\partial \tilde{r}}{\partial r} \right)_{i + 1}
\end{equation}
%
which we evaluate for each zone $i = 1, ..., 179$, iterating backward from zone 179, and where after evaluating each $\tilde{r}_i$ we also calculate
%
\begin{equation}
    \left( \frac{\partial \tilde{r}}{\partial r} \right)_{i} = \frac{\tilde{r}_i}{r_i \Gamma(r_i)}
\end{equation}
%
as input to the iteration for the next zone.

In practice, due to the irregular spacing of radial zones resulting from the adaptive mesh in \code{Agile}, the numerical integration of Equation~\ref{eqn:conf_diffeq} will yield solutions where $\partial \psi / \partial r$ changes sign and falls below 1.
Specifically, we found that when a few radial zones are much closer together relative to the spacing of other zones, e.g. near the reverse shock, $\psi$ would begin decreasing towards the center of the star.
However, we expect $\psi$ to continually increase towards the center of the star as the compactness of the material increases.
So, we dynamically enforce the outer boundary condition in such a way as to always find physical solutions for $\psi$.
In particular, we attempt the integration of Equation~\ref{eqn:conf_diffeq} from zone 179 to 1, in the discrete form of Equation~\ref{eqn:discrete_conf_diffeq} with the outer boundary condition enforced in zone 180.
If $\psi < 1$ in any radial zone, we re-attempt the integration, but instead enforce the outer boundary condition in zone 180 \textit{and} 179, and then integrate from zone 178 to 1.
Again, if $\psi < 1$ in any radial zone, we re-attempt the integration once more with the outer boundary condition enforced in zones 180, 179, and 178.
We continue in this manner until $\psi \geq 1$ in every zone.
We have found that, in the most extreme cases, this procedure stops when the outer boundary has been enforced down to roughly zone number 100; however, this is still well outside the proto-neutron star at any iteration of our models, and so does not change the eigenfrequencies.