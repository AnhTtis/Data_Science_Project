% show that, picking characteristic frequencies from characteristic times, we get basically the same thing as our fancy histogram method

%\begin{figure*}
%    \centering
%
%    \begin{tabular}{cc}
%       \includegraphics[width=0.49\linewidth]{single-times_eos-mu1-hists.pdf} & %\includegraphics[width=0.49\linewidth]{single-times_mrem-mu2-fits.pdf}   
%    \end{tabular}
%    
%    \includegraphics[width=\linewidth]{figures/app-single-times/single-times_eos-mu1-hists.pdf}
%    \caption{
%        Left: Histograms of the characteristic frequency $f_1$ for each equation of state, %identified at the characteristic time $t_1$ as noted in each panel.
%        Each histogram is plotted with a bin width of 10 Hz and normalized to have an area of one. %\TODO{check!}
%        While $t_1 \lesssim 0.4$ seconds post-bounce, prior to the avoided crossing between the %$f$- and $g_1$-modes, these results reflect those of \TODO{FIG}, showing a dependence of $f_1$ on %the choice of nuclear equation of state.
%
%        Right: We plot remnant surface gravity of each model versus characteristic frequency $f_2$ %as identified at the time $t_2$ noted in each subplot. 
%        For $t_2 \gtrsim 0.4$ seconds post-bounce, these results reflect those of \TODO{fig}, %showing a linear correlation between the surface gravity of the cold neutron star and $f_2$ %independent of the equation of state.
%    }
%    \label{fig:single-times}
%\end{figure*}

\begin{figure*}
    \centering
    \includegraphics[width=\linewidth]{single-times.pdf}
    \caption{
    Top: Histograms of the characteristic frequency $f_1$ for each equation of state, identified at the characteristic time $t_1$ as noted in each panel.
    Each histogram is plotted with a bin width of 10 Hz and normalized to have an area of one.
    While $t_1 \lesssim 0.4$ seconds post-bounce, prior to the avoided crossing between the $f$- and $g_1$-modes, these results reflect those of the left panel of Figure~\ref{fig:mu_results}, showing a dependence of $f_1$ on the choice of nuclear equation of state.
    Bottom: We plot remnant surface gravity of each model versus characteristic frequency $f_2$ as identified at the time $t_2$ noted in each subplot. 
    For $t_2 \gtrsim 0.4$ seconds post-bounce, these results reflect those of the right panel of Figure~\ref{fig:mu_results}, showing a linear correlation between the surface gravity of the cold neutron star and $f_2$ independent of the equation of state.
    }
    \label{fig:single-times}
\end{figure*}

In Section~\ref{sec:results-gmm-fits}, we identified characteristic frequencies of the gravitational-wave signal from our models by fitting a the histogram of $f$-mode frequencies for each model to a two-component Gaussian mixture.
Then, in Section~\ref{sec:sim-relations}, we explored the correlations between these characteristic frequencies and the nuclear equation of state and surface gravity of the proto-neutron star. 
In this appendix, we check whether relying on characteristic frequencies identified by the Gaussian mixture fitting procedure have introduced spurious correlations in our data. For this, we repeat our analysis using the frequencies at a specific post-bounce time ($t_1$) instead of $\mu_1$ (cf.\ left panel of Figure~\ref{fig:mu_results}) and using the frequencies at a specific post-bounce time ($t_2$) instead of $\mu_2$ (cf.\ right panel of Figure~\ref{fig:mu_results}). We select both $t_1$ and $t_2$ to be at 0.3, 0.4, 0.5, 0.6, and 0.7 seconds post-bounce (in the language of this paper, the first two would be considered `early time' and the last three would be considered `late time'). 
% figure (top row)
The top row of Figure~\ref{fig:single-times} shows histograms for the $f$-mode frequencies at the five specific post-bounce times and is the analogous figure to the left panel of Figure~\ref{fig:mu_results}. For `early times' (i.e. for $t_1 \lesssim 0.4$~s; first two panels) these results are qualitatively similar to those obtained using the characteristic frequency $\mu_1$. At later times ($t_1 > 0.4$~s), the histograms overlap more and the distinction between different nuclear equations of state disappears. 
% figure (bottom row)
In the bottom row of Figure~\ref{fig:single-times}, we plot the surface gravity of the cold remnant NS star against the $f$-mode frequencies identified at the same five specific post-bounce times. Here, we find qualitatively similar results for `late times' (i.e.\ $t_2 > 0.4$~s; last three panels) as in Figure~\ref{fig:mu_results}. However, at `early times' (first two panels), the surface gravity is degenerate with the $f$-mode frequency. 
From this, we conclude that at `early times' the $f$-mode frequency can distinguish between nuclear equations of state, and at `late times' the $f$-mode frequency correlates with the surface gravity of the remnant NS star. This replicates our findings using $\mu_1$ and $\mu_2$. Hence, using time-dependent frequencies replicates the results from the time-agnostic characteristic frequency method.

% Here, we attempt to identify characteristic frequencies by characteristic early- and late-times, post-bounce, and repeat our analysis to check whether our fit to Gaussian mixtures introduces spurious correlations in our data.
% Instead of characteristic frequencies $\mu_1$ and $\mu_2$ identified as the component means in a fit of the density of frequencies to 0.7 seconds post-bounce to a two-component Gaussian mixture, we identify characteristic frequencies $f_1$ and $f_2$ at characteristic early- and late-times $t_1$ and $t_2$, respectively.
% In the top panels of Figure~\ref{fig:single-times}, we show the characteristic frequency $f_1$ identified at five different choices of the characteristic time $t_1$, measured in seconds post-bounce; this figure is analogous to the result presented in the left panel of Figure~\ref{fig:mu_results}.
% We observe that when $t_1 \lesssim 0.4$, we achieve results that are qualitatively similar to those obtained with a characteristic frequency $\mu_1$.
% In particular, we observe a distinct peak for TM1 at relatively low $f_1$ overlapping the NL3 distribution, a sharp shared peak for SFHo and SFHx at high $f_1$, and a shared peak for DD2 \& BHB$\lambda \varphi$ in between those for TM1 and SFHo.
% At values of $t_1 > 0.4$ seconds, we see the value of $f_1$ become increasingly degenerate with equation of state as the late-time frequency evolution becomes more strongly correlated with surface gravity.
% In the bottom panels of Figure~\ref{fig:single-times}, we plot the surface gravity of the cold remnant against the characteristic frequency $f_2$ identified at five different choices of the characteristic time $t_2$, analogous to the result presented in the right panel of Figure~\ref{fig:mu_results}.
% For $t_2 \lesssim 0.4$ seconds, we observe that the remnant mass is nearly degenerate with $f_2$ as the early-time evolution is correlated with the equation of state.
% For $t_2 > 0.4$ seconds, we observe a flattening linear relationship between remnant surface gravity and $f_2$ per equation of state, strongly reminiscent of the relationship between remnant surface gravity and characteristic frequency $\mu_2$.
% We also note that, compared to the right panel of Figure~\ref{fig:mu_results}, there is noticeably more scatter among a subset of the DD2 and BHB$\lambda\varphi$ models at intermediate frequencies, as well as SFHo/SFHx models at the highest frequencies, which is not captured by $\mu_1$.
% In total, we observe that choosing characteristic early- and late-time frequencies $f_1$ and $f_2$ in a time-dependent manner replicates the time-agnostic procedure to calculate similar characteristic frequencies via fits to a Gaussian mixture model.