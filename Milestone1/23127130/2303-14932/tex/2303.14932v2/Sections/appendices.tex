% The \appendix command is used to start a single appendix.
% An optional argument can be used to specify a title:
% \appendix[Proof of the Zonklar Equations]
% After issuing \appendix, the \section command will be
% disabled and any attempt to use \section will be ignored
% and will cause a warning message to be generated. (The
% single appendix marks the end of the enumerated sections
% and the section counter is fixed at zero—one does not state
% “see Appendix A” when there is only one appendix, instead
% “see the Appendix” is used.) However, all lower \subsecti
% on commands and the \section* form will work as normal
% as these may still be needed for things like acknowledgments.
\appendices
% is used when there is more than one appendix
% section. \section is then used to declare each appendix:
% \section{Proof of the First Zonklar Equation}

% The mandatory argument to section can be left blank (\sect
% ion{}) if no title is desired. It is important to remember to
% declare a section before any additional subsections or labels
% that refer to section (or subsection, etc.) numbers. As with \appendix, the \section* command and the lower \subsection commands will still work as usual.

% Some authors prefer to have the appendix number to be part
% of equation numbers for equations that appear in an appendix.
% This can be accomplished by redefining the equation numbers
% as
\renewcommand{\theequation}{\thesection.\arabic{equation}}
% before the first appendix equation. For a single appendix, the
% constant “A” should be used in place of \thesection.


%---------------------------------------------%
%---- INTERMEDIARY RESULTS FOR THEOREM 1 -----%
%---------------------------------------------%
\begin{comment}
\section{Intermediary Results for Theorem \ref{thm:strongduality}}\label{seC:appendix:intermediary_results}

\begin{lem}[Equivalence between Behavioral Policy-Profiles and their (decentralized) Mixtures]\label{lem:dominance}
Fix a (factorized) measure $\mu \in \uspacemix$. Then there exists a behavioral policy-profile $\udl{u}=\udl{u}(\mu) \in \uspace$, such that for any $t \in \mbb{N}$, $\hst{h}{t} \in \hstspace{t}$, and $\at{t} \in \aspace$,
\begin{align*}
p\l( \mu, t, \hst{h}{t}, \at{t} \r) = p\l( u, t, \hst{h}{t}, \at{t} \r),
\end{align*}
where, for brevity and with slight abuse of notation,
\begin{align*}
p\l( \cdot, t, \hst{h}{t}, \at{t} \r) &= \prup{\cdot}{P_1}\l(\Hst{t} = \hst{h}{t}, \At{t} = \at{t} \r), \text{ and}\\
p\l( \cdot, t, \hst{h}{t} \r) &= \prup{\cdot}{P_1}\l(\Hst{t} = \hst{h}{t} \r).
\end{align*}
\end{lem}
\begin{proof}
Define $\udl{u}=\udl{u}(\mu) \in \uspace$ such that
\begin{align*}
&\ut{\udl{u}}{t}\l( \at{t} | \hst{h}{t}\ \r) = \prod_{n=1}^{N} 
\utn{\udl{u}}{t}{n} 
\l( \atn{t}{n} | \hstn{h}{t}{0}, \hstn{h}{t}{n} \r)  \\
&= \begin{cases}
\frac{ p\l( \mu, t, \hst{h}{t}, \at{t} \r)
}{p\l( \mu, t, \hst{h}{t}\r)}, &\text{if } p\l(\mu, t, \hst{h}{t} \r) \ne 0,\\
\prod_{n=1}^{N} \frac{1}{|\anspace{n}|}, &\text{otherwise}.
\end{cases}\numberthis\label{eq:dominance:u}
\end{align*}
The above assignment is correct because the right-hand-side of \eqref{eq:dominance:u} is a fully-factorized function of $\atn{t}{n}$'s.
\begin{align*}
&p\l( \mu, t, \hst{h}{t}, \at{t} \r) = \int_{U} \mu\l( du \r) \prup{u}{P_1} \l( \Hst{t} = \hst{h}{t}, \At{t} = \at{t} \r)\\
&= \int_{\uspace} P_1\l( \sspace, \hst{h}{1} \r) \prod_{t'=2}^{t} \pr_{P_1} \l( \ot{t'} | \hst{h}{t'-1}, \at{t'-1} \r)\\
&\hspace{10pt} \times \prod_{n=1}^{N} \prod_{t'=1}^{t} \utn{u}{t'}{n}\l( \atn{t'}{n} |  \hstn{h}{t'}{0}, \hstn{h}{t'}{n} \r) 
\mu\l( du \r)
% \l( \l( \mymathop{\times}_{n=1}^{N} \mun{n}\r)(du) \r)
\\
&=P_1\l( \sspace, \hst{h}{1} \r) \prod_{t'=1}^{t} \pr_{P_1} \l( \ot{t'} | \hst{h}{t-1}, \at{t'-1} \r)\\
&\hspace{10pt} \times \prod_{n=1}^{N} \int_{\uspacen{n}}  \prod_{t'=1}^{t} \utn{u}{t'}{n}\l( \atn{t'}{n} |  \hstn{h}{t'}{0}, \hstn{h}{t'}{n} \r) \mun{n}\l( d\un{u}{n}\r),
\end{align*}
where the last equality follows from Tonneli's Theorem (see Proposition \ref{prop:tonneli}). We will now prove, by forward induction, that for all $t\in\mbb{N}$, $\udl{u}$ and $\mu$ induce the same distribution on the pair $\l( \Hst{t}, \At{t} \r)$.
\begin{enumerate}
% [leftmargin=0pt, itemindent=20pt,
% labelwidth=15pt, labelsep=5pt, 
% % listparindent=0.7cm,
% align=left]
[leftmargin=0pt, 
itemindent=10pt,
labelwidth=0pt, 
labelsep=5pt, 
% listparindent=0.7cm,
% align=left
]
\item \textbf{Base Case}: For time $t=1$, let $\ot{1} \in \hstspace{1} = \ospace$ and $\at{1} \in \aspace$. We have
\begin{align*}
p\l( \mu, 1, \ot{1}, \at{1}\r) &= P_1\l( \sspace, \ot{1} \r) \int_{\uspace} \mu\l( du \r) \ut{u}{1}\l( \at{1} | \ot{1} \r),
\end{align*}
and
\begin{align*}
p\l( \udl{u}, 1, \ot{1}, \at{1}\r) &= P_1\l( \sspace, \ot{1} \r) \ut{\udl{u}}{1}\l( \at{1} | \ot{1} \r) \\
&\hspace{-30pt}= P_1\l( \sspace, \ot{1} \r) \frac{p\l( \mu, 1, \ot{1}, \at{1}\r)}{p\l( \mu, 1, \ot{1}\r)}\\ 
&%\hspace{-30pt}= P_1\l( \sspace, \ot{1} \r) \frac{p\l( \mu, 1, \ot{1}, \at{1}\r)}{P_1\l( \sspace, \ot{1} \r) } 
\hspace{-30pt}=p\l( \mu, 1, \ot{1}, \at{1}\r),
\end{align*}
where the last equality follows from $p\l( \mu, 1, \ot{1}\r) = P_1\l( \sspace, \ot{1} \r) $.

\item \textbf{Induction Step}. Assuming that the statement is true for time $t$, we show that it is true for time $t+1$. Let $\hst{h}{t+1} = \l( \ot{1:t+1}, \at{1:t} \r) = \l( \hst{h}{t}, \at{t}, \ot{t+1} \r) \in \hstspace{t+1}$ and $\at{t+1} \in \aspace$. We have
\begin{align*}
p\l(\mu, t+1, \hst{h}{t+1} \r) &= \int_{\uspace} \mu\l( du \r) \prup{u}{P_1} \l( \Hst{t+1} = \hst{h}{t+1} \r)\\
&\hspace{-60pt} = \int_{\uspace} \mu\l( du \r) \prup{u}{P_1} \l( \Hst{t} = \hst{h}{t}, \At{t} = \at{t}, \Ot{t+1} = \ot{t+1} \r)\\
% &\hspace{-60pt}= \int_{\uspace} \mu\l( du \r) \prup{u}{P_1} \l( \Hst{t} = \hst{h}{t}, \At{t} = \at{t}, \r) \\
% &\hspace{-40pt} \times \prup{u}{P_1} \l( \Ot{t+1} = \ot{t+1} | \Hst{t} = \hst{h}{t}, \At{t} = \at{t} \r)\\
&\hspace{-60pt}= \int_{\uspace} \mu\l( du \r) \prup{u}{P_1} \l( \hst{h}{t}, \at{t} \r)  \pr_{P_1} \l( \ot{t+1} | \hst{h}{t}, \at{t} \r)\\
&\hspace{-60pt}= p\l(\mu, t, \hst{h}{t}, \at{t}\r) \pr_{P_1} \l( \ot{t+1} |  \hst{h}{t},  \at{t} \r)\\
&\hspace{-60pt}\labelrel{=}{eqr:dominance:ind} p\l(\udl{u}, t, \hst{h}{t}, \at{t}\r) \pr_{P_1} \l(  \ot{t+1} |  \hst{h}{t}, \at{t} \r)\\
&\hspace{-60pt}=p\l(\udl{u}, t+1, \hst{h}{t+1} \r),
\end{align*}
where \eqref{eqr:dominance:ind} uses the inductive hypothesis. The above work implies 
\begin{align*}
&p\l(\udl{u}, t+1, \hst{h}{t+1}, \at{t+1} \r) \\
&\hspace{0pt} = p\l(\udl{u}, t+1, \hst{h}{t+1} \r) \cdot\ \ut{\udl{u}}{t+1}\l(\at{t+1} | \hst{h}{t+1} \r) \\
% &\hspace{0pt} = p\l(\mu, t+1, \hst{h}{t+1} \r) \frac{p\l(\mu, t+1, \hst{h}{t+1}, \at{t+1} \r)}{p\l(\mu, t+1, \hst{h}{t+1} \r)}\\
&\hspace{0pt} = p\l(\mu, t+1, \hst{h}{t+1}, \at{t+1} \r).
\end{align*}
\end{enumerate}
This completes the proof.
\end{proof}

\begin{cor}\label{cor:lbar_and_l}
Fix $\lambda \in \mcl{Y}$. For any $\mu \in \uspacemix$, there exists $u = u(\mu) \in \uspace$ such that $\lags{u}{\lambda} = \lagsmix{\mu}{\lambda}$.
\end{cor}
\begin{proof}
One notes that $\wh{C}(\mu)$ and $\wh{D}(\mu)$ can be written as:
\begin{align*}
\wh{C}(\mu) &= \sum_{t=1}^{\infty} \alpha^{t-1} \E{\mu}{P_1} \l[ \mbb{E}_{P_1} \l[ c\l( \Stt{t}, \At{t} \r)  \r] | \Hst{t}, \At{t} \r],\\
\wh{D}(\mu) &= \sum_{t=1}^{\infty} \alpha^{t-1} \E{\mu}{P_1} \l[ \mbb{E}_{P_1} \l[ d\l( \Stt{t}, \At{t} \r)  \r] | \Hst{t}, \At{t} \r],
\end{align*}
and the result follows.
\end{proof}

\begin{lem}\label{lem:puth}[Limit Probabilities for a converging sequence of policy-profiles]
Let $\l\{ \useq{i}{u} \r\}_{i=1}^{\infty}$ be a sequence in $\uspace$ that converges to $u$. Then, for any $t \in \mbb{N}$, $ \hst{h}{t} \in \hstspace{t} $, and $\at{t} \in \mcl{A}$,
\begin{align*}
\lim_{i\ra \infty}  \pruphsts{\useq{i}{u}}{t}{\hst{h}{t}, \at{t}} = \pruphsts{u}{t}{\hst{h}{t}, \at{t}},
\end{align*}
where $\pruphsts{\cdot}{t}{\hst{h}{t}, \at{t}} = \prup{\cdot}{P_1} \l( \Hst{t} = \hst{h}{t}, \At{t} = \at{t} \r)$. In other words, for every $t \in \mbb{N}$, the sequence of measures $\l\{ \pruphsts{ \useq{i}{u}}{t}{\cdot, \cdot} \r\}_{i=1}^{\infty}$ converges weakly to $\pruphsts{u}{t}{\cdot, \cdot}$.
\end{lem}
\begin{proof}
Given that $\useq{i}{u}$ converges to $u$, by the definition of convergence in product topology, for every $n \in [N]$, $\useq{i}{\utn{u}{t}{n}} (\hstn{h}{t}{0}, \hstn{h}{t}{n} )$ converges weakly to $ \utn{u}{t}{n} ( \hstn{h}{t}{0}, \hstn{h}{t}{n} ) $. Since $\mcl{A}^n$ is finite, this means that for each $\an{n}\in \anspace{n}$, $\useq{i}{\utn{u}{t}{n}} ( \an{n} | \hstn{h}{t}{0}, \hstn{h}{t}{n} )$ converges to $\utn{u}{t}{n} ( \an{n} | \hstn{h}{t}{0}, \hstn{h}{t}{n} )$, which further implies that for all $a \in \aspace$, $ \useq{i}{\ut{u}{t}} ( a | \hst{h}{t}) $ converges to $ \ut{u}{t} ( a | \hst{h}{t}) $. Now, we use forward induction to prove the statement. 
\begin{enumerate}
[leftmargin=0pt, 
itemindent=10pt,
labelwidth=0pt, 
labelsep=5pt, 
% listparindent=0.7cm,
% align=left
]
\item \textbf{Base Case}:  For time $t=1$, let $\ot{1} \in \hstspace{1} = \ospace$ and $\at{1} \in \mcl{A}$. We have
\begin{align*}
\pruphsts{\useq{i}{u}}{1}{\ot{1}, \at{1}}
=P_1\l( \sspace, o \r) \useq{i}{\ut{u}{1}} \l( \at{1} | \ot{1} \r) 
\ra \pruphsts{u}{1}{\ot{1}, \at{1}}.
\end{align*}

\item \textbf{Induction Step}: Assuming that the statement is true for time $t$, we show that it is true for time $t+1$. Let $\hst{h}{t+1} = \l( \ot{1:t+1}, \at{1:t} \r) = \l( \hst{h}{t}, \at{t}, \ot{t+1} \r) \in \hstspace{t+1}$ and $\at{t+1} \in \aspace$. We have
\begin{align*}
&\pruphsts{\useq{i}{u}}{t+1}{\hst{h}{t+1}, \at{t+1}} =  
\pruphsts{\useq{i}{u}}{t}{\hst{h}{t}, \at{t}} \\
&\hspace{25pt} \times \useq{i}{\ut{u}{t+1}} \l( \at{t+1} | \hst{h}{t+1} \r) \pr_{P_1} \l( \ot{t+1} | \hst{h}{t}, \at{t} \r).
\end{align*}
By inductive hypothesis, $\pruphsts{\useq{i}{u}}{t}{\hst{h}{t}, \at{t}} $ converges to $\pruphsts{u}{t}{\hst{h}{t}, \at{t}}$, and $ \useq{i}{\ut{u}{t}}\l( \at{t+1} | \hst{h}{t+1}\r) $ converges to $ \ut{u}{t} \l( \at{t+1} | \hst{h}{t+1}\r) $ by assumption. We conclude that $\pruphsts{\useq{i}{u}}{t+1}{\hst{h}{t+1}, \at{t+1}}$ converges to $\pruphsts{u}{t+1}{\hst{h}{t+1}, \at{t+1}}$.
\end{enumerate}
This completes the proof.
\end{proof}
\end{comment}


%---------------------------------------------%
%-------------- HELPFUL FACTS ----------------%
%---------------------------------------------%
\section{Helpful Facts and Results}\label{sec:appendix:helpful_facts}
\begin{dfn}[Semi-continuity]\label{dfn:lsc}
A function $f : \mcl{X} \mapsto [-\infty, \infty]$ on a topological space $\mcl{X}$ is called \emph{lower semi-continuous} if for every point $x_0 \in \mcl{X}$, 
% \begin{align*}
$\liminf\limits_{x\ra x_0} f(x) \ge f(x_0)$. 
% \end{align*}
We call $f$ \emph{upper semi-continuous} function if $-f$ is lower semi-continuous.
\end{dfn}

\begin{prop}[Monotone Convergence Theorem]\label{prop:mct}
Let $\l(X, \mcl{M}, \mu \r)$ be a measure-space. Let $\l\{ f_i \r\}_{i=1}^{\infty}$ be an increasing sequence of measurable functions which are uniformly bounded-from-below. Then, 
\begin{align*}
&\int_{X} \lim_{i\ra\infty} f_i(x) \mu(dx) = \lim_{i\ra\infty} \int_{X} f_i(x) \mu(dx). 
\end{align*}
% $\int_{X} \lim_{i\ra\infty} f_i(x) \mu(dx) = \lim_{i\ra\infty} \int_{X} f_i(x) \mu(dx)$.
\end{prop}

\begin{prop}[Convergence in Product Topology]\label{prop:conv_in_prod_topology}
Let $ \{^ix\}_{i=1}^{\infty}$ be a sequence of points of the product space $\prod_{j} X_j$. Then $\{^ix\}_{i=1}^{\infty}$ converges to a point $x \in \prod_{j} X_j$ if and only if the sequence $\{ \pi_j(^ix) \}_{i=1}^{\infty}$ converges to $\pi_j (x) $ for each $j$.
\end{prop}


\begin{comment}
\begin{prop}[Tonneli's Theorem]\label{prop:tonneli}
Let $f$ be a measurable function on the cartesian product of two $\sigma$-finite measure spaces $(X, \mcl{M}, \mu)$ and $(Y, \mcl{N}, \nu)$ which is bounded from below. Then, 
\begin{align*}
&\int_{X\times Y} f(x,y) (\mu \times \nu) (d(x,y))\\ 
&\hspace{10pt} =\int_X \l( \int_Y f(x,y) \nu(dy)\r) \mu(dx)\\
&\hspace{10pt} =\int_Y \l(\int_X f(x,y) \mu(dx)\r) \nu(dy).
\end{align*}
\end{prop}
\end{comment}

\begin{prop}[Fatou's Lemma]\label{prop:fatou}
Let  $(X, \mcl{M}, \mu)$ be a measure-space and let $\{ f_i \}_{i=1}^{\infty}$ be a sequence of measurable functions which are uniformly bounded from below. Then,
\begin{align*}
& \liminf_{i\ra\infty} \int f_i(x) \mu (dx)\ge \int \liminf_{i\ra\infty} f_i(x) \mu(dx).
\end{align*}
\end{prop}


\begin{prop}[Tychonoff's Theorem]\label{prop:tychonoff}
Product of a collection of compact spaces is compact under the product topology.
\end{prop}

\begin{prop}[Metrizability of Product Topology on Countable Product of Metric Spaces]\label{prop:metrizability}
Product of countable number of metric spaces, when endowed with the product topology, is metrizable.
\end{prop}

\begin{prop}[Prokhorov's Theorem]\label{prop:prokhorov}
Let $\l( \mcl{X}, \metric{\mcl{X}} \r)$ be a complete separable metric space with distance metric $\metric{\mcl{X}}$ and let $\borel{\mcl{X}}$ denote the Borel $\sigma$-algebra generated by $\metric{\mcl{X}}$. Let $\m{\mcl{X}}$ be the set of all probability measures on $\borel{\mcl{X}}$ and let $\tau$ denote the topology of weak-convergence on $\m{\mcl{X}}$. Then,  
\begin{enumerate}
\item The topological space $ \l(\m{\mcl{X}} , \tau\r)$ is completely-metrizable, i.e., there exists a complete metric $\metric{\m{\mcl{X}}}$ on $ \m{\mcl{X}}$ that induces the same topology as $ \tau $.
\item An arbitrary collection $W \subseteq \m{\mcl{X}}$ of probability measures in $ \m{\mcl{X}}$ is tight iff its closure in $\tau $ is compact (i.e., $W$ is precompact in $\tau$).
\end{enumerate}
\end{prop}

% \begin{comment}
\begin{prop}[Hyperplane Separation Theorem]\label{prop:separation_theorem}
Let $M$ be a non-empty convex subset of 
$\mbb{R}^n$. If $x_0 \in \mbb{R}^n$ does not belong to $M$, there exists $\rho \in \mbb{R}^n$ such that
\begin{align*}
\rho \neq 0 \text { and } \inf_{x \in M} \dotp{p}{x} \geq \dotp{p}{x_0}.
\end{align*}
\end{prop}
% \end{comment}


\begin{prop}[Integral of Bounded-from-Below function with respect to Convex Combination of Non-negative Measures]\label{prop:integral_linearity}
Let $\l(X, \mcl{M}\r)$ be a measure-space. Let $f : X \ra \mbb{R} \cup \{ \infty \}$ be a measurable function that is bounded from below, and let $\mu, \nu$ be two non-negative measures on $\mcl{M}$. Then, for any $\theta \in [0,1]$,
\begin{align*}
&\int f(x) \l(\theta \mu + (1-\theta) \nu \r)(dx) \\
&\hspace{10pt} = \theta \int f(x) \mu(dx) + (1-\theta) \int f(x)\nu(dx). 
\end{align*}
\end{prop}

\begin{prop}[Behavior of Integrals of a Bounded-from-Below and Lower Semi-Continuous Function]\label{prop:lsc}
Let $(\mcl{X}, \metric{\mcl{X}})$ be a 
%compact (\textcolor{red}{do I need it to be compact})
complete separable metric space with distance metric $\metric{\mcl{X}}$ and let $\borel{\mcl{X}}$ denote the Borel $\sigma$-algebra generated by $\metric{\mcl{X}}$. Let $\l( \m{\mcl{X}} , \metric{\m{\mcl{X}}} \r)$ be the complete metric space of all probability measures on $\borel{\mcl{X}}$ with the topology of weak-convergence.\footnote{Prokhorov's theorem (see Proposition \ref{prop:prokhorov}) ensures completeness and metrizability of $\m{\mcl{X}}$.} Let $\mu \in \m{\mcl{X}}$ and let $f : \mcl{X} \ra \mbb{R} \cup \l\{ \infty\r\} $ be a function that is lower semi-continuous $\mu$-amost-everywhere\footnote{Lower semi-continuity of $f$ ensures that it is measurable.} and is bounded from below. Then, the function
\begin{align*}
H : \m{\mcl{X}} \mapsto \mbb{R} \cup \l\{ \infty \r\},\  %\\
H(\mu') \defeq \int f(x) \mu'(dx)
\end{align*}
is lower semi-continuous at $\mu$. In particular, if $f$ is point-wise lower semi-continuous, then $H$ is also point-wise lower semi-continuous (on $\m{\mcl{X}}$).
\end{prop}
\begin{proof}
% The proof is omitted due to space reasons.
% See \cite{khan23}[Appendix B]. 
% \begin{comment}
Define $f' : \mcl{X} \ra \mbb{R} \cup \{\infty \}$ as $f'(x) \defeq f(x) \wedge \liminf_{y\ra x} f(y)$. Then, $f'$ minorizes $f$\footnote{That is, $f'(x) \le f(x)$.}, is lower semi-continuous, and coincides with $f$ at $x$ if and only if $f$ is lower semi-continuous at $x$. Also, $f'$ is bounded from below (since $f$ is). By Proposition \ref{prop:lsc3}, $f'$ can be written as the point-wise limit of increasing sequence of uniformly bounded-from-below continuous functions from $\mcl{X}$ into $\mbb{R} \cup \{ \infty \}$, say $\l\{ g_i %: \mcl{X} \ra \mbb{R} \cup \{\infty\}
\r\}_{i=1}^{\infty} $, i.e., $f'(x) = \lim_{i\ra \infty} g_i(x)$. Then, for every $\mu' \in \m{\mcl{X}}$,
\begin{align*}
\int f'(x)\mu'(dx) = \int \lim_{i\ra \infty} g_i(x) \mu'(dx) = \lim_{i\ra \infty} \int g_i(x)\mu'(dx),
\end{align*}
where the last equality follows from the Monotone Convergence Theorem (see Proposition \ref{prop:mct}). The above equality shows that the function $H' : \m{\mcl{X}} \ra \mbb{R} \cup \{ \infty\}$ such that $H'(\mu') = \int f'(x) \mu'(dx)$, is the point-wise limit of an increasing sequence of uniformly bounded-from-below continuous functions. Therefore, by Proposition \ref{prop:lsc3}, $H'$ is lower semi-continuous. Now, if $f$ is lower semi-continuous $\mu$-almost-everywhere, then $f = f'$ $\mu-$almost-everywhere. This gives,
\begin{align*}
H(\mu) &= \int f(x) \mu(dx) \\
&= \int f'(x) \mu(dx) \\
&\labelrel{=}{eqr:lsc:H2islsc} \liminf_{\mu'\ra\mu} H'(\mu') \\
&\labelrel{\le}{eqr:lsc:H2minorizesH} \liminf_{\mu'\ra\mu} H(\mu'),
\end{align*}
Here, \eqref{eqr:lsc:H2islsc} uses lower semi-continuity of $H'$ and \eqref{eqr:lsc:H2minorizesH} follows from the fact that $H'$ minorizes $H$ (since $f'$ minorizes $f$). The inequality $H(\mu) \le \liminf_{\mu'\ra\mu} H(\mu')$ is the definition of lower semi-continuity at $\mu$. 
% \end{comment}
\end{proof}

\begin{prop}[Equivalent Characterization of a Bounded-from-Below Lower Semi-Continuous Function]\label{prop:lsc3}
Let $\l( \mcl{X}, \metric{\mcl{X}} \r)$ be a metric space. Then, a function $f : \mcl{X} \ra \mbb{R} \cup \{ \infty \}$ is a bounded-from-below lower semi-continuous function if and only if it can be written as the point-wise limit of an increasing sequence of uniformly bounded-from-below continuous functions from $\mcl{X}$ into $\mbb{R} \cup \{ \infty \}$. 
\end{prop}
\begin{proof}
% The proof is omitted due to space reasons.
% See \cite{khan23}[Appendix B]. 
% \begin{comment}
\textbf{Necessity}: Define $f_n : \mcl{X} \ra \mbb{R} \cup \{ \infty \}$ as follows:
\begin{align*}
f_n\l( x \r) &\defeq \inf_{y\in\mcl{X}} \l\{ f(y) + n \metric{\mcl{X}} \l(x, y\r) \r\}.
\end{align*}
\begin{enumerate}
\item \textit{Increasing}: 
\begin{align*}
f_{n+1}\l( x \r) = \inf_{y\in\mcl{X}} \l\{ f(y) + (n+1)\metric{\mcl{X}}\l( x,y\r) \r\} \ge f_n(x).
\end{align*}
\item \textit{Uniformly Bounded-from-Below}: Since $f_n\l(x \r) \ge \inf_{y\in\mcl{X}} \l\{ f(y) \r\}$ and $f$ is bounded-from-below, the functions $\l\{ f_n\r\}_{n=1}^{\infty}$ are uniformly bounded-from-below.
\item \textit{Continuity}: By triangle-inequality,
\begin{align*}
f(y) + n\metric{\mcl{X}}\l(y, z\r) \le
f(y) + n\metric{\mcl{X}}\l(y, w\r) +  n\metric{\mcl{X}}\l(w, z\r),
\end{align*}
and therefore, taking the infimum over $y$ on both sides gives $ f_n\l( z \r) - f_n\l( w \r) \le n\metric{\mcl{X}}\l( w, z\r) $. Similarly, we can get $ f_n\l( w \r) - f_n\l( z \r) \le n\metric{\mcl{X}}\l( w, z\r) $, and so
\begin{align*}
|f_n\l( z \r) - f_n\l( w \r)| \le n \metric{\mcl{X}} \l(w, z\r).
\end{align*}
The above relation shows that $f_n$ is Lipschitz and thus continuous.
\item \textit{Point-wise Convergence to $f$}: Fix $x_0 \in \mcl{X}$ and $\eps>0$. We would like to show that there exists a positive integer $n' = n'(x_0, \eps)$ such that, for all $ n \ge n'$, $| f_n\l(x_0\r) - f\l(x_0\r) | < \eps$. Since $f$ is lower semi-continuous at $x_0$, there exists $\delta = \delta(x_0, \eps) > 0$ such that
\begin{align*}
\metric{\mcl{X}}\l( x_0, y\r) < \delta \implies f(y) >  f(x_0) -\eps.\numberthis\label{eq:lsc2:implication}
\end{align*}
Since $f$ is bounded-from-below (and $\delta>0$), there exists a positive integer $n'=n'(\delta(x_0,\eps))$ such that
\begin{align*}
&\metric{\mcl{X}}\l( x_0, y\r) \ge \delta\\
&\hspace{0pt} \implies \forall\  n\ge n', f(y) + n\metric{\mcl{X}}\l( x_0, y\r) > f(x_0)\\
&\hspace{0pt} \implies \forall\  n\ge n', \\
&\hspace{40pt} \inf_{\metric{\mcl{X}}\l( x_0, y\r) \ge \delta } \l\{ f(y) + n\metric{\mcl{X}}(x_0, y) \r\}\ge f\l(x_0\r).
\end{align*}
So, for all $n\ge n'$, we have
\begin{align*}
f(x_0) \ge f_n\l( x_0 \r) &= \inf_{\metric{\mcl{X}}\l( x_0, y\r) \le \delta } \l\{ f(y) + n\metric{\mcl{X}}(x_0, y) \r\}\\
&\ge\inf_{\metric{\mcl{X}}\l( x_0, y\r) \le \delta } \l\{ f(y) \r\}\\
&\labelrel{>}{eqr:lsc2:1}\inf_{\metric{\mcl{X}}\l( x_0, y\r) \le \delta } \l\{ f(x_0) - \eps \r\}\\
&=f(x_0) - \eps.
\end{align*}
where \eqref{eqr:lsc2:1} uses \eqref{eq:lsc2:implication}.
\end{enumerate}
\textbf{Sufficiency}: Let $\l\{ f_n \r\}_{n=1}^{\infty} $ be an increasing sequence of uniformly bounded-from-below continuous functions from $\mcl{X}$ into $\mbb{R} \cup \l\{ \infty \r\}$. Since the sequence is monotonic, it has a point-wise-limit $f : \mcl{X} \ra \mbb{R} \cup \l\{ \infty \r\}$ which is bounded-from-below because all the functions in the sequence are uniformly bounded-from-below. We need to show that $f$ is lower semi-continuous. 

Fix $x_0 \in \mcl{X}$ and $\eps>0$. We would like to show that there exists $\delta = \delta(x_0,\eps)>0$ such that $\metric{\mcl{X}}\l( x_0, y\r) < \delta \implies f(y) >  f(x_0) -\eps $. Since  $\l\{ f_n \r\}_{n=1}^{\infty} $ is increasing (and converges point-wise to $f$), there exists a positive integer $n'=n'(x_0, \eps)$ such that, for all $n\ge n'$, $f(x_0) \ge f_n(x_0) \ge f(x_0) - \frac{\eps}{2}$. Since $f_{n'}$ is lower semi-continuous, there exists $\delta=\delta(n'(x_0, \eps)) > 0$ such that $\metric{\mcl{X}}\l( x_0, y\r)<\delta \implies f(y) \ge f_{n'}(y) > f_{n'}(x_0) - \frac{\eps}{2} \ge f(x_0) - \eps$. 
% \end{comment}
\end{proof}



\begin{comment}
\begin{prop}[A Minimax Theorem For Functions with Positive Infinity]\label{prop:sionminimax}
Let $\mcl{X}$ and $\mcl{Y}$ be convex topological spaces where $\mcl{X}$ is also compact. Consider a function $f : \mcl{X} \times \mcl{Y} \ra \mbb{R} \cup \{ \infty \} $ such that
\begin{enumerate}
\item $\forall\ y \in \mcl{Y}$, $f\l(\cdot, y \r)$ is convex and lower semi-continuous.
\item $\forall\ x \in \mcl{X}$, $f\l(x, \cdot \r)$ is concave.
\item If $f (x, y) = \infty$, then $f(x, y') = \infty$ for all $y'\in\mcl{Y}$.
\end{enumerate}
Then, there exists $x^\star \in \mcl{X}$ such that
\begin{align*}
\sup_{y\in \mcl{Y}} f\l( x^\star, y \r) &=
\inf_{x \in \mcl{X}} \sup_{y \in \mcl{Y}} f\l( x, y \r) =\sup_{y \in \mcl{Y}} \inf_{x \in \mcl{X}} f(x, y).
% &\hspace{0pt}
\end{align*}
\end{prop}
\begin{proof}
    % The proposition is a mild adaptation of the Minimax theorem presented in \cite{aubin_book_2002}[Theorem 8.1] where a real-valued function is considered. For space reasons, the proof is omitted.
    See \cite{khan23}[Appendix C]. 
\end{proof}
\end{comment}

%---------------------------------------------%
%------------- MINIMAX THEOREM ---------------%
%---------------------------------------------%
\input{Sections/minimax_theorem}


