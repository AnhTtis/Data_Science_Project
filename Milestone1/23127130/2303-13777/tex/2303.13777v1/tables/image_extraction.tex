\begin{table}[ht]
	\centering
	\scalebox{0.75}{
	%\resizebox{\linewidth}{!}{
		\begin{tabular}{l l l}
			\toprule
			
			Layer &\multicolumn{1}{l}{Input (\textbf{ID} :  HWC )} 
			&\multicolumn{1}{c}{Output (\textbf{ID} :  HWC )}
			\\
			\midrule
			
			Conv, 64, k=7, s=2 & \textbf{0} : $512\times 512\times 3$   & \textbf{1} : $256\times 256\times 64$ \\
			Residual layer 1 & \textbf{1} : $256\times 256\times 64$ & \textbf{2} : $256\times 256\times 64$  \\
			
			Residual layer 2 & \textbf{2} : $256\times 256\times 64$ & \textbf{3} : $128\times 128\times 128$  \\

                Residual layer 3 & \textbf{3} : $128\times 128\times 128$ & \textbf{4} : $64\times 64\times 256$  \\
			
			 Upconv, 128, k=3, f=2 & \textbf{4} : $64\times 64\times 256$ & \textbf{5} : $128\times 128\times 128$  \\
			
			 iConv, 128, k=3, s=1 & \textbf{3 © 5}: $128\times 128\times 256$  & \textbf{6} : $128\times 128\times 128$  \\

                Upconv, 64, k=3, f=2 & \textbf{6} : $128\times 128\times 128$ & \textbf{7} : $256\times 256\times 64$  \\

                iConv, 64, k=3, s=1 & \textbf{2 © 7}: $256\times 256\times 128$  & \textbf{8} : $256\times 256\times 64$  \\

                iConv, 64, k=3, s=1 & \textbf{1 © 8}: $256\times 256\times 128$  & \textbf{9} : $256\times 256\times 64$  \\

                Conv, 64, k=1, s=1 & \textbf{9} : $256\times 256\times 64$  & \textbf{10} : $256\times 256\times 64$  \\

			\bottomrule
	\end{tabular}}
	% \vspace{-0.5em}
	\caption{{\bf Image feature extraction network}.  'Conv' stands
for a sequence of operations: convolution (k is kernal size and s is stride), rectified linear units
(ReLU) and Batch Normalization\cite{BN}. 'iConv' replace the Batch Normalization with Instance Normalization\cite{IN} compare with 'Conv'. 'Upconv' stands for a bilinear upsampling with specific factor (f), followed by a 'iConv' operation with stride=1. © represents channel-wise concatenation. 'Residual layer' is the residual blocks of the original ResNet34\cite{ResNet} design,
of two feature maps}
    \vspace{-1em}
	\label{tab:image_extraction}
\end{table}