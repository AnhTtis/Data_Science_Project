%!TEX root = BF.tex


\section{Introduction}\label{sec:intro}
		
The relation between birational geometry and $\C^*$-actions has been deeply studied over the years (see for instance \cite{ReidFlip}, \cite{Thaddeus} and in recent years \cite{WORS1}).
One famous example for instance is the notion of \emph{Morelli-W\l odarczyk cobordism} (cfr. \cite[Definition 2]{Wlodarczyk}, see Definition \ref{def: cobordism}), introduced to study birational maps among normal projective varieties which are realized as geometric quotients by a $\C^*$-action. This algebraic version of cobordism has been used to prove the \emph{Weak factorization conjecture} (see \cite{Wlodarczyk}). 

On the other hand, the relation between birational maps and $\C^*$-actions becomes evident in the example of the \emph{Atiyah flop} (see \cite[\S 1.3]{ReidFlip}). 
Consider the $\C^*$-action on $\C^4$ given by 
\[
t \cdot \left(x_0,x_1,y_0,y_1\right)=\left( tx_0,tx_1,t^{-1}y_0,t^{-1}y_1\right),
\]
where $t\in\C^*$ and $(x_0,x_1,y_0,y_1)\in \C^4$. The GIT quotient $\C^4\to \C^4\git \C^*$ is the affine cone over the Segre embedding of $\P^1 \times \P^1$.
The variety $\C^4 \git \C^*$  has a cone singularity at the origin, which can be resolved by blowing-up the vertex of the cone.
The exceptional divisor $\P^1\times \P^1$ of the blow-up can be contracted in two different ways, producing two smooth varieties $X_1$ and $X_2$, which are isomorphic on the complement of a closed subset of codimension greater or equal than $2$. The resulting birational map $X_1 \dashrightarrow X_2$ is called the Atiyah flop and $\C^4$ is an example of cobordism associated to it. The following picture summarizes the example:
\begin{figure}[h!]
\centering
\tikzset{every picture/.style={line width=0.75pt}} %set default line width to 0.75pt        

\begin{tikzpicture}[x=0.4pt,y=0.4pt,yscale=-1,xscale=1]
%uncomment if require: \path (0,789); %set diagram left start at 0, and has height of 789

%Shape: Polygon [id:ds4850704739937841] 
\draw  [draw opacity=0][fill={rgb, 255:red, 144; green, 19; blue, 254 }  ,fill opacity=0.2 ] (660,150) -- (640,250) -- (490,310) -- (490,310) -- (520,230) -- cycle ;
%Straight Lines [id:da41606237713260785] 
\draw    (30,250) -- (180,310) ;
%Straight Lines [id:da7936370010344177] 
\draw    (180,250) -- (30,310) ;
%Straight Lines [id:da15086398653346844] 
\draw    (200,260) -- (60,340) ;
%Straight Lines [id:da6385894053551876] 
\draw [line width=1.5]    (30,140) -- (180,200) ;
%Straight Lines [id:da25609817756390163] 
\draw [line width=1.5]    (660,150) -- (520,230) ;
%Shape: Polygon [id:ds30843765904369747] 
\draw  [draw opacity=0][fill={rgb, 255:red, 144; green, 19; blue, 254 }  ,fill opacity=0.2 ] (420,40) -- (420,150) -- (280,230) -- (280,230) -- (280,120) -- cycle ;
%Shape: Polygon [id:ds7839677942934724] 
\draw  [draw opacity=0][fill={rgb, 255:red, 74; green, 144; blue, 226 }  ,fill opacity=0.2 ] (400,90) -- (400,200) -- (250,140) -- (250,140) -- (250,30) -- cycle ;
%Shape: Polygon [id:ds5220547568697286] 
\draw  [draw opacity=0][fill={rgb, 255:red, 144; green, 19; blue, 254 }  ,fill opacity=0.2 ] (660,150) -- (660,260) -- (520,340) -- (520,340) -- (520,230) -- cycle ;
%Shape: Polygon [id:ds377636801452376] 
\draw  [draw opacity=0][fill={rgb, 255:red, 74; green, 144; blue, 226 }  ,fill opacity=0.2 ] (180,200) -- (200,280) -- (50,240) -- (50,240) -- (30,140) -- cycle ;
%Shape: Polygon [id:ds253289493622184] 
\draw  [draw opacity=0][fill={rgb, 255:red, 74; green, 144; blue, 226 }  ,fill opacity=0.2 ] (420,60) -- (420,170) -- (270,130) -- (270,130) -- (270,20) -- cycle ;
%Shape: Polygon [id:ds2831812809771401] 
\draw  [draw opacity=0][fill={rgb, 255:red, 144; green, 19; blue, 254 }  ,fill opacity=0.2 ] (400,30) -- (400,140) -- (250,200) -- (250,200) -- (250,90) -- cycle ;
%Shape: Polygon [id:ds881112108538183] 
\draw  [draw opacity=0][fill={rgb, 255:red, 74; green, 144; blue, 226 }  ,fill opacity=0.2 ] (335,325) -- (410,465) -- (260,405) -- (260,405) -- (335,325) -- cycle ;
%Shape: Polygon [id:ds6130139984748016] 
\draw  [draw opacity=0][fill={rgb, 255:red, 74; green, 144; blue, 226 }  ,fill opacity=0.2 ] (335,325) -- (430,435) -- (280,395) -- (280,395) -- (335,325) -- cycle ;
%Shape: Polygon [id:ds39194821844628935] 
\draw  [draw opacity=0][fill={rgb, 255:red, 144; green, 19; blue, 254 }  ,fill opacity=0.2 ] (335,325) -- (430,415) -- (290,495) -- (290,495) -- (335,325) -- cycle ;
%Shape: Polygon [id:ds8888399328308494] 
\draw  [draw opacity=0][fill={rgb, 255:red, 144; green, 19; blue, 254 }  ,fill opacity=0.2 ] (335,325) -- (410,405) -- (260,465) -- (260,465) -- (335,325) -- cycle ;
%Straight Lines [id:da2141826957544144] 
\draw    (170,330) -- (248.3,378.94) ;
\draw [shift={(250,380)}, rotate = 212.01] [color={rgb, 255:red, 0; green, 0; blue, 0 }  ][line width=0.75]    (10.93,-3.29) .. controls (6.95,-1.4) and (3.31,-0.3) .. (0,0) .. controls (3.31,0.3) and (6.95,1.4) .. (10.93,3.29)   ;
%Straight Lines [id:da059430083941788725] 
\draw    (431.63,378.84) -- (500,330) ;
\draw [shift={(430,380)}, rotate = 324.46] [color={rgb, 255:red, 0; green, 0; blue, 0 }  ][line width=0.75]    (10.93,-3.29) .. controls (6.95,-1.4) and (3.31,-0.3) .. (0,0) .. controls (3.31,0.3) and (6.95,1.4) .. (10.93,3.29)   ;
%Straight Lines [id:da09909378224217413] 
\draw  [dash pattern={on 4.5pt off 4.5pt}]  (448,260) -- (220,260) ;
\draw [shift={(450,260)}, rotate = 180] [color={rgb, 255:red, 0; green, 0; blue, 0 }  ][line width=0.75]    (10.93,-3.29) .. controls (6.95,-1.4) and (3.31,-0.3) .. (0,0) .. controls (3.31,0.3) and (6.95,1.4) .. (10.93,3.29)   ;
%Shape: Polygon [id:ds4175242386704846] 
\draw  [draw opacity=0][fill={rgb, 255:red, 74; green, 144; blue, 226 }  ,fill opacity=0.2 ] (650,155) -- (660,280) -- (660,280) -- (510,240) -- (510,240) -- cycle ;
%Shape: Polygon [id:ds7329252346902249] 
\draw  [draw opacity=0][fill={rgb, 255:red, 74; green, 144; blue, 226 }  ,fill opacity=0.2 ] (580,195) -- (640,310) -- (640,310) -- (490,250) -- (490,250) -- cycle ;
%Straight Lines [id:da8359005539718207] 
\draw    (151.7,168.94) -- (230,120) ;
\draw [shift={(150,170)}, rotate = 327.99] [color={rgb, 255:red, 0; green, 0; blue, 0 }  ][line width=0.75]    (10.93,-3.29) .. controls (6.95,-1.4) and (3.31,-0.3) .. (0,0) .. controls (3.31,0.3) and (6.95,1.4) .. (10.93,3.29)   ;
%Straight Lines [id:da7055480908011266] 
\draw    (440,120) -- (518.3,168.94) ;
\draw [shift={(520,170)}, rotate = 212.01] [color={rgb, 255:red, 0; green, 0; blue, 0 }  ][line width=0.75]    (10.93,-3.29) .. controls (6.95,-1.4) and (3.31,-0.3) .. (0,0) .. controls (3.31,0.3) and (6.95,1.4) .. (10.93,3.29)   ;
%Shape: Circle [id:dp16639528362810663] 
\draw  [fill={rgb, 255:red, 0; green, 0; blue, 0 }  ,fill opacity=1 ] (330,325) .. controls (330,322.24) and (332.24,320) .. (335,320) .. controls (337.76,320) and (340,322.24) .. (340,325) .. controls (340,327.76) and (337.76,330) .. (335,330) .. controls (332.24,330) and (330,327.76) .. (330,325) -- cycle ;
%Straight Lines [id:da6020295014555623] 
\draw    (240,415) -- (380,495) ;
%Straight Lines [id:da6951138713448372] 
\draw    (430,415) -- (290,495) ;
%Straight Lines [id:da8097512426943787] 
\draw    (410,405) -- (260,465) ;
%Straight Lines [id:da13247989979656716] 
\draw    (390,395) -- (240,435) ;
%Straight Lines [id:da48040891417002174] 
\draw    (260,405) -- (410,465) ;
%Straight Lines [id:da34274536279481416] 
\draw    (280,395) -- (430,435) ;
%Straight Lines [id:da4308686990205012] 
\draw    (50,240) -- (200,280) ;
%Shape: Polygon [id:ds7461534503115913] 
\draw  [draw opacity=0][fill={rgb, 255:red, 74; green, 144; blue, 226 }  ,fill opacity=0.2 ] (180,200) -- (180,310) -- (30,250) -- (30,250) -- (30,140) -- cycle ;
%Straight Lines [id:da17577793998936353] 
\draw    (10,260) -- (150,340) ;
%Shape: Polygon [id:ds6151751696139521] 
\draw  [draw opacity=0][fill={rgb, 255:red, 144; green, 19; blue, 254 }  ,fill opacity=0.2 ] (200,260) -- (200,260) -- (60,340) -- (60,340) -- (140,185) -- cycle ;
%Shape: Polygon [id:ds3190444980145123] 
\draw  [draw opacity=0][fill={rgb, 255:red, 144; green, 19; blue, 254 }  ,fill opacity=0.2 ] (180,250) -- (180,250) -- (30,310) -- (30,310) -- (90,165) -- cycle ;
%Straight Lines [id:da9454912707573251] 
\draw    (160,240) -- (10,280) ;
%Straight Lines [id:da8627817917798487] 
\draw    (620,240) -- (470,280) ;
%Straight Lines [id:da7849167182699056] 
\draw    (640,250) -- (490,310) ;
%Straight Lines [id:da6426667173315391] 
\draw    (660,260) -- (520,340) ;
%Straight Lines [id:da036192532873840944] 
\draw    (470,260) -- (610,340) ;
%Straight Lines [id:da5436269492804261] 
\draw    (490,250) -- (640,310) ;
%Straight Lines [id:da3328774675830918] 
\draw    (510,240) -- (660,280) ;
%Straight Lines [id:da15128144566750634] 
\draw [line width=1.5]    (270,20) -- (420,60) ;
%Straight Lines [id:da000043131873048385394] 
\draw [line width=1.5]    (250,30) -- (400,90) ;
%Straight Lines [id:da6139619875956873] 
\draw [line width=1.5]    (230,40) -- (370,120) ;
%Straight Lines [id:da299862555826932] 
\draw [line width=1.5]    (380,20) -- (230,60) ;
%Straight Lines [id:da44278809508998673] 
\draw [line width=1.5]    (400,30) -- (250,90) ;
%Straight Lines [id:da45668783873732766] 
\draw [line width=1.5]    (420,40) -- (280,120) ;
%Straight Lines [id:da9216283818418308] 
\draw    (230,150) -- (370,230) ;
%Straight Lines [id:da8236025108977942] 
\draw    (250,140) -- (400,200) ;
%Straight Lines [id:da3807742075199273] 
\draw    (270,130) -- (420,170) ;
%Straight Lines [id:da6293654960419728] 
\draw    (420,150) -- (280,230) ;
%Straight Lines [id:da2652516871601991] 
\draw    (400,140) -- (250,200) ;
%Straight Lines [id:da13928252133219965] 
\draw    (380,130) -- (230,170) ;

% Text Node
\draw (410,0) node [anchor=north west][inner sep=0.75pt]    {$\P^1 \times \P^1$};
% Text Node
\draw (620,130) node [anchor=north west][inner sep=0.75pt]    {$\mathbb{P}^{1}$};
% Text Node
\draw (70,120) node [anchor=north west][inner sep=0.75pt]    {$\mathbb{P}^{1}$};
% Text Node
\draw (300,500) node [anchor=north west][inner sep=0.75pt]    {$\C^4 \git \C^*$};
% Text Node
\draw (90,340) node [anchor=north west][inner sep=0.75pt]    {$X_1$};
% Text Node
\draw (550,340) node [anchor=north west][inner sep=0.75pt]    {$X_2$};
% Text Node
%\draw (315,220) node [anchor=north west][inner sep=0.75pt]    {$\Bl_0(\C^4 \git \C^*)$};


\end{tikzpicture}
\end{figure}

\noindent
One may generalize this example by considering a similar $\C^*$-action on $\C^{n+1}$ (see for instance \cite[Example 1]{Wlodarczyk}), or allowing also weights different from $\pm 1$ (cf. \cite[\S 3]{BR}): the resulting birational map is called the \emph{Atiyah flip}.
In the above construction of the Atiyah flop, the varieties $X_1$ and $X_2$ are defined using the two projective bundle structures on $\P^1 \times \P^1$: 
\begin{equation*}\label{eq:rooftopP1xP1}
\xymatrix{&\P^1 \times \P^1 \ar[ld]_{} \ar[rd]^{} & \\ \P^1 && \P^1}
\end{equation*}
Smooth projective varieties with two different projective bundle structures have been already studied in the literature (see for instance \cite{ORS}): in particular, in \cite[Lemma 4.4]{WORS1} the authors have constructed a correspondence between them and smooth projective varieties $X$ of Picard number $1$ admitting a $\C^*$-action having only two fixed point components. 

Motivated by the example of Atiyah flop and its connection with a variety admitting two projective bundle structures, we introduce the notion of \emph{rooftop flip}.
Given a smooth projective variety $\Lambda$ of Picard number $2$ admitting two projective bundle structures 
\begin{equation*}\label{eq:rooftopP1xP1}
	\xymatrix{&\Lambda \ar[ld]_{} \ar[rd]^{} & \\ \Lambda_- && \Lambda_+,}
\end{equation*}
a \emph{rooftop flip modeled by $\Lambda$} (see Definition \ref{def:Dflips}) is a small modification $\psi: W_-\dashrightarrow W_+$ among normal projective varieties which can be resolved by a common divisorial extraction 

\[
\xymatrix{ &W \ar[ld]_{b_-} \ar[rd]^{b_+} & \\ W_- \ar@{-->}[rr]_\psi && W_+}
\]
such that the restriction of $b_{\pm}: W\to W_{\pm}$ to the exceptional loci are modeled by the projective bundle maps $p_{\pm}: \Lambda\to \Lambda_{\pm}$.


In this setting, the Atiyah flop is a rooftop flip modeled by $\P^1\times \P^1$.  As we will see, the class of rooftop flips includes classical birational transformations: the Atiyah flip is a rooftop flip modeled by $\P^m\times \P^l$, and the Mukai flop (see \cite{HZ04}, \cite{WW}) is a rooftop flip modeled by $\P\left(T_{\P^2}\right)$. Moreover, rooftop flips appear as local models of the birational transformations of the geometric quotients of a smooth polarized pair $(X,L)$ under a $\C^*$-actions (see \cite[Theorem 1.1 (1)]{WORS3})
In the paper we discuss the definition of  rooftop flip and its connection with smooth projective varieties having a $\C^*$-action with only two fixed point components.
The main result of the paper is the following:
		
\begin{theorem}\label{theorem:naive}
	Given a smooth projective variety $\Lambda$ of Picard number $2$ with two projective bundle structures, there exist two quasi-projective varieties and a rooftop flip modeled by $\Lambda$ among them.
\end{theorem}

\subsection*{Outline}

In Section \ref{sec:preliminaries} we first recall some notions regarding $\C^*$-actions on smooth polarized pairs. We then focus on the case of $\C^*$-action whose associated fixed point locus consists only of $2$ connected components; such varieties are called \emph{drums} (see Definition \ref{definition:drum}) and they are constructed upon the choice of a triple $(Y,\cL_-,\cL_+)$, where $Y$ is a smooth projective variety of Picard number $2$ admitting two projective bundle structure, and $\cL_{\pm}$ are semiample line bundles on $Y$. We conclude by recalling the relation between smooth drums and the classification of horospherical varieties done by \cite{Pas} (see Remark \ref{remark:pasquier}).

In Section \ref{sec:atiyah}, after recalling the Morelli-W\l odarczyk cobordism (see Definition \ref{def: cobordism}), we introduce and discuss the notion of rooftop flip (see Definition \ref{def:Dflips}). We then revisit the example of Atiyah flip and prove that it is indeed a rooftop flip (see \S \ref{ssec:atiyah}).

In Section \ref{sec:mainresult} we prove Theorem \ref{theorem:naive}. The idea of the proof is to view our case as a restriction of the Morelli-W\l odarczyk cobordism associated to the Atiyah flip. We conclude by deducing that the drum structure on the smooth quadric hypersurface induces a rooftop flip modeled by $\P\left(T_{\P^n}\right)$, see Corollary \ref{corollary:rooftopflipquadric}. 

\subsection*{Acknowledgements} We would like to thank Luis E. Sol\'a Conde for having suggested this problem, for all the stimulating conversations and the important suggestions. %We would like also to thank Gianluca Occhetta, Eleonora A. Romano and Jaroslaw A. Wi\'sniewski for the careful regarding of this work. 