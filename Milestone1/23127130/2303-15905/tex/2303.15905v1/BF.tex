\documentclass%[12pt]
{amsproc}
%\usepackage{showkeys}

\usepackage{amsmath}
\usepackage{amsthm}
\usepackage{amssymb}
\usepackage{graphicx}
\usepackage{multirow}
%\usepackage{mathtools}
\usepackage{tikz-cd}


\DeclareGraphicsExtensions{.png,.pdf,.eps}
%\usepackage{amsthm}
\usepackage[all,2cell]{xy}
\usepackage{tikz}
\usepackage{pgf}
\usepackage{enumerate}
\usepackage{mathdots}
\usetikzlibrary{cd}
\usepackage{caption}
\captionsetup[table]{skip=20pt}
\usepackage{adjustbox}
\usetikzlibrary{matrix,arrows,backgrounds}

\usepackage{enumitem}
\setlist[itemize,enumerate,description]{leftmargin=*}

\usepackage{array}
\newcolumntype{H}{>{\setbox0=\hbox\bgroup}c<{\egroup}@{}}
\usepackage{hyperref}

\newenvironment{pf}{\begin{proof}}{\end{proof}}
\newtheorem{theorem}{Theorem}[section]
\newtheorem{lemma}[theorem]{Lemma}
\newtheorem{corollary}[theorem]{Corollary}
\newtheorem{proposition}[theorem]{Proposition}
\newtheorem{conjecture}[theorem]{Conjecture}
\newtheorem{claim}[theorem]{Claim}
\newtheorem{step}{Step}
\newtheorem{step2}{Step} % ADD ONE FOR EVERY THEOREM THAT WE WANT TO DIVIDE INTO STEPS 

\theoremstyle{definition}

\newtheorem{defs}[theorem]{Definitions}
\newtheorem{definition}[theorem]{Definition}
\newtheorem{notation}[theorem]{Notation}
\newtheorem{setup}[theorem]{Set-up}
\newtheorem{problem}{Problem}
\newtheorem{remark}[theorem]{Remark}
\newtheorem{assumption}[theorem]{Assumptions}
\newtheorem{set}[theorem]{Setup}
\newtheorem{example}[theorem]{Example}
\newtheorem{question}{Question}
\newtheorem{construction}{Construction}

%\numberwithin{equation}{section}

\setcounter{tocdepth}{1}
\setcounter{secnumdepth}{4}

\title[Morelli-W\l odarczyk cobordism and examples of rooftop flips]{Morelli-W\l odarczyk cobordism and examples of rooftop flips}

\author[Barban]{Lorenzo Barban}
\address{Dipartimento di Matematica, Universit\`a degli Studi di Trento, via Sommarive 14, I-38123 Povo di Trento (TN), Italy}
\email{lorenzo.barban@unitn.it}

\author[Franceschini]{Alberto Franceschini}
\address{Dipartimento di Matematica, Universit\`a degli Studi di Trento, via Sommarive 14, I-38123 Povo di Trento (TN), Italy}
\email{alberto.bobech@gmail.com}

\subjclass[2010]{Primary 14L30; Secondary 14E30, 14L24, 14M17}

\thanks{}



%%%%%%%%%%%%%%%%%% macros
% variables
%\vec{\mathcal{M}}
\renewcommand\vec[1]{\ensuremath\boldsymbol{#1}}
\renewcommand\cdots{...}
\newcommand{\tB}{\vec{\mathcal{B}}}
\newcommand{\tY}{\vec{\mathcal{Y}}}
\newcommand{\tF}{\vec{\mathcal{F}}}
\newcommand{\cB}{\mathcal{B}}
\newcommand{\mB}{\mathbf{B}}
\newcommand{\mY}{\mathbf{Y}}
\newcommand{\mZ}{\mathbf{Z}}
\newcommand{\vb}{\mathbf{b}}
\newcommand{\vy}{\mathbf{y}}
\newcommand{\valpha}{\boldsymbol{\alpha}}
\newcommand{\tA}{\vec{\mathcal{A}}}
\newcommand{\tD}{\vec{\mathcal{D}}}
\newcommand{\tX}{\vec{\mathcal{X}}}
\newcommand{\tM}{\vec{\mathcal{M}}}
\newcommand{\cX}{\mathcal{X}}
\newcommand{\mX}{\mathbf{X}}
\newcommand{\mA}{\mathbf{A}}
\newcommand{\vx}{\mathbf{x}}
\newcommand{\vq}{\mathbf{q}}
\newcommand{\mbrp}[1]{\mathbb{R}_{+}^{#1}}
\newcommand{\mbr}[1]{\mathbb{R}^{#1}}
\newcommand{\mbn}[1]{\mathbb{N}^{#1}}
\newcommand{\mbnz}[1]{\mathbb{N}_{0^+}^{#1}}
\newcommand{\mbnp}[1]{\mathbb{N}_{+}^{#1}}
\newcommand{\stackThree}{{;}_{3}}
\newcommand{\vbeta}{\vec{\beta}}
%\newcommand{\rank}[1]{\text{Rank}({#1})}
\newcommand{\tI}{\vec{\mathcal{I}}}

\newcommand{\tAnb}{\mathcal{A}}
\newcommand{\tMnb}{\mathcal{M}}
\newcommand{\tXnb}{\mathcal{X}}
\newcommand{\tYnb}{\mathcal{Y}}
\newcommand{\tInb}{\mathcal{I}}

\newcommand{\vectorise}{\text{Vec}}


%\newcommand{\tXS}{\vec{\mathcal{X}}^{*}}
\newcommand{\vv}{\mathbf{v}}
\newcommand{\tV}{\vec{\mathcal{V}}}
\newcommand{\tE}{\vec{\mathcal{E}}}
\newcommand{\tEH}{\vec{\mathcal{\hat{E}}}}
\newcommand{\tVH}{\vec{\mathcal{\bar{V}}}}
\newcommand{\tVT}{\vec{\mathcal{\hat{V}}}}
\newcommand{\idx}[1]{\mathcal{I}_{#1}}
\newcommand{\semipd}[1]{\mathcal{S}_{+}^{#1}}
\newcommand{\spd}[1]{\mathcal{S}_{++}^{#1}}

\newcommand{\tR}{\vec{\mathcal{R}}}
\newcommand{\vu}{\mathbf{u}}
\newcommand{\vup}{\mathbf{u^{'}}}
\newcommand{\vz}{\mathbf{z}}
\newcommand{\vzeta}{\boldsymbol{\zeta}}
\newcommand{\vc}{\mathbf{c}}

\newcommand{\vphi}{\boldsymbol{\phi}}
\newcommand{\vpsi}{\boldsymbol{\psi}}
\newcommand{\tPsi}{\vec{\mathcal{V}}}
\newcommand{\bigoh}{\mathcal{O}}
\newcommand{\mPsi}{\vec{\Psi}}
\newcommand{\vj}{\vec{j}}

% operators
\newcommand{\enorm}[1]{\left\|{#1}\right\|_2}
\newcommand{\fnorm}[1]{\left\|{#1}\right\|_F}
\newcommand{\lnorm}[1]{\left\|{#1}\right\|_1}
\newcommand{\riem}{\mathbf{d}_{\mathcal{R}}}
\newcommand{\spdp}[1]{\mathbb{S}_{++}^{#1}}
\newcommand{\simplex}[1]{\Delta^{#1}}
\newcommand{\set}[1]{\left\{#1\right\}}

\DeclareMathOperator*{\argmin}{arg\,min}
\DeclareMathOperator*{\argmax}{arg\,max}
\DeclareMathOperator*{\supp}{Supp}
\DeclareMathOperator*{\unique}{Unique}
\DeclareMathOperator*{\TRank}{TRank}
\DeclareMathOperator*{\rank}{Rank}
\DeclareMathOperator*{\spann}{Span}
\DeclareMathOperator*{\sym}{Sym}
% \DeclareMathOperator*{\softmingg}{SoftMin_{\bar{\gamma}}}
% \DeclareMathOperator*{\softming}{SoftMin_\gamma}
% \DeclareMathOperator*{\topminb}{TopMin_\beta}
% \DeclareMathOperator*{\topmaxbb}{TopMax_{NZ\beta}}

% \DeclareMathOperator*{\softminsel}{SoftMinSel_\gamma}

\newcommand{\flatt}[1]{\text{Flatten}\!\left({#1}\right)}
\newcommand{\unflatt}[1]{\text{Flatten}^{-1}\!\left({#1}\right)}
\newcommand{\myspan}[1]{\spann\left(#1\right)}

\DeclareMathOperator*{\trace}{Tr}
%\DeclareMathOperator*{\rank}{Rank}
\DeclareMathOperator*{\kronstack}{\uparrow\!\otimes}

\DeclareMathOperator*{\diag}{Diag}
\DeclareMathOperator*{\avg}{avg}
\DeclareMathOperator*{\sgn}{Sgn}
\DeclareMathOperator*{\hosvd}{HOSVD}
\DeclareMathOperator*{\logm}{Log}
\DeclareMathOperator*{\expm}{{Exp}}
\DeclareMathOperator*{\detm}{Det}
\DeclareMathOperator*{\fg}{g}
\newcommand{\expl}[1]{\text{e}^{#1}}
\DeclareMathOperator*{\res}{Res}
\DeclareMathOperator*{\asinh}{Asinh}
\DeclareMathOperator*{\vect}{vec}
\DeclareMathOperator*{\detach}{Detach}
%\newcommand{\exp}[1]{e^{#1}}




\newcommand{\mI}{\mathbf{I}}
\newcommand{\normvec}[1]{\frac{#1}{\|{#1}\|_2}}
\newcommand{\suptensor}[1]{\mathfrak{S}^{#1}}
\newcommand{\suptensorr}[2]{\mathfrak{S}^{#1}_{\times^{#2}}}
\newcommand{\region}{\mathcal{R}}

%\newtheorem{theorem}{Theorem}
%\newtheorem{definition}{Definition}
%\newtheorem{lemma}{Lemma}
%\newtheorem{proposition}{Proposition}
%\newtheorem{remark}{Remark}
%\newtheorem*{Proof}{Proof}

%\newcommand{\todo}[1]{{\bf \textcolor{red}{[TODO: #1]}}}


\newcommand{\mLa}{\boldsymbol{\lambda}^{*}}
\newcommand{\mLambda}{\boldsymbol{\lambda}}
\newcommand{\mU}{\mathbf{U}}
\newcommand{\mV}{\mathbf{V}}
\newcommand{\timetplone}{{(t+1)}}
\newcommand{\timet}{{(t)}}

\newcommand{\mBOvl}{{\mB^{*}}}

\newcommand{\mPi}{{\boldsymbol\Pi}}

\newcommand{\piA}{{\Pi_A}}
\newcommand{\piB}{{\Pi_B}}

\newcommand{\sigmav}{{^v\!\!\,{\sigma}}}
%\newcommand{\thickhat}[1]{\mathbf{\ddot{\text{$#1$}}}}
\newcommand{\sigmas}{{^s\!\!\,{\sigma}}}

\newcommand{\fvx}{{\boldsymbol{f}(\vx)}}
\newcommand{\fvy}{{\boldsymbol{f}(\vy)}}

\newcommand{\vsss}{\boldsymbol{s}}
\newcommand{\vw}{\boldsymbol{w}}

\newcommand{\vphibar}{\boldsymbol{\bar{\phi}}}
\newcommand{\vsigma}{\boldsymbol{\sigma}}

\def\eg{\emph{e.g.}}

\newcommand{\myg}[1]{\boldsymbol{G}\left(#1\right)}
\newcommand{\mygtwo}[1]{\boldsymbol{G}\Big(#1\Big)}
%\newcommand{\mygthree}[1]{\boldsymbol{G}\left(#1\right)}
\newcommand{\mygthree}[1]{\boldsymbol{\mathcal{G}}\!\left(\!#1\!\right)}
\newcommand{\mygthrees}[1]{\boldsymbol{\mathcal{G}^*\!}\!\left(\!#1\!\right)}
\newcommand{\mygfour}[1]{\boldsymbol{\mathcal{G}}\!\Bigg(\!#1\!\Bigg)}
\newcommand{\tG}{\boldsymbol{\mathcal{G}}}
\newcommand{\tGhat}{\widehat{\boldsymbol{\mathcal{G}}}}

\newcommand{\mygthreep}[1]{\boldsymbol{\mathcal{G}'}\!\left(\!#1\!\right)}
\newcommand{\mygthreee}[2]{\boldsymbol{\mathcal{G}}_{{\text{#1}}}\!\left(\!#2\!\right)}

\newcommand{\mygthreehat}[1]{\boldsymbol{\widehat{\mathcal{G}}}\!\left(\!#1\!\right)}
\newcommand{\mygthreephat}[1]{\boldsymbol{\widehat{\mathcal{G}}'}\!\left(\!#1\!\right)}

\newcommand{\mygthreeehat}[2]{\boldsymbol{\widehat{\mathcal{G}}_{{\text{#1}}}}\!\left(\!#2\!\right)}
\newcommand{\mygthreeep}[2]{\boldsymbol{\mathcal{G}'_{{\text{#1}}}}\!\left(\!#2\!\right)}
\newcommand{\mygthreeephat}[2]{\boldsymbol{\widehat{\mathcal{G}}'_{{\text{#1}}}}\!\left(\!#2\!\right)}

\newcommand{\vPhi}{\boldsymbol{\Phi}}
\newcommand{\invbeta}{{(1\!-\!\beta)}}
\newcommand{\invsqrtbeta}{\sqrt{1\!-\!\beta}}
\newcommand{\sqrtbeta}{\sqrt{\beta}}


\newcommand{\mIdent}{\boldsymbol{\mathds{I}}}
\newcommand{\sIdent}{\mathds{I}}
\newcommand{\vOnes}{\boldsymbol{1}}

\newcommand{\mJ}{\mathbf{J}}
\newcommand{\sXkl}{{X_{kl}}}

\newcommand{\mQ}{\mathbf{Q}}
\newcommand{\mK}{\mathbf{K}}
\newcommand{\mKb}{\bar{\mK}}
\newcommand{\mKbb}{\bar{\mKb}}
\newcommand{\Kb}{\bar{K}}
\newcommand{\Kbb}{\bar{\Kb}}
\newcommand{\mC}{\mathbf{C}}

\newcommand{\mKro}{{\mK^{q}}}
\newcommand{\mKbro}{{\mKb{\,\!}^{q}}}
\newcommand{\mKbbro}{{\mKbb^{q}}}
\newcommand{\Kbro}{{\Kb^{q}}}
\newcommand{\Kbbro}{{\Kbb^{q}}}



\newcommand{\fvxt}{{\boldsymbol{f}^{(t)}(\vx)}}
\newcommand{\Fvxt}{{\boldsymbol{F}^{(t)}(\vx)}}
\newcommand{\fvxtplusone}{{\boldsymbol{f}^{(t+1)}(\vx)}}
\newcommand{\Fvxtplusone}{{\boldsymbol{F}^{(t+1)}(\vx)}}
\newcommand{\vxzero}{\mathbf{x}_0}
\newcommand{\fvxzerotplusone}{{\boldsymbol{f}^{(t+1)}(\vxzero)}}
\newcommand{\Fvxzerotplusone}{{\boldsymbol{F}^{(t+1)}(\vxzero)}}
\newcommand{\fvxzerotplusonei}{{\boldsymbol{f}_i^{(t+1)}(\vxzero)}}

\newcommand{\tFvxt}{{{\vec{\mathcal{F}}}^{(t)}(\vx)}}
\newcommand{\tFvxtplusone}{{{\vec{\mathcal{F}}}^{(t+1)}(\vx)}}
\newcommand{\tFvxzerotplusone}{{{\vec{\mathcal{F}}}^{(t+1)}(\vxzero)}}




\newcommand{\swbar}{\bar{w}}
\newcommand{\vwbar}{\bar{\boldsymbol{w}}}

\newcommand{\vvartheta}{\boldsymbol{\vartheta}}
\newcommand{\tprim}[1]{{\uparrow T_{#1}}}


\newcommand{\vS}{\boldsymbol{S}}

\newcommand{\barM}{{\bar{M}}}
\newcommand{\barmM}{{\bar{\vec{M}}}}

\newcommand{\sN}{\vec{N}}
\newcommand{\tN}{\vec{\mathcal{N}}}
\newcommand{\tP}{\vec{\mathcal{P}}}
\newcommand{\tS}{\vec{\mathcal{S}}}
\newcommand{\tSnb}{\mathcal{S}}
\newcommand{\mS}{\vec{S}}
\newcommand{\tNnb}{\mathcal{N}}

\newcommand{\cov}{\boldsymbol{\Sigma}}
\newcommand{\covb}{\boldsymbol{\Sigma}^{(\!\diamond\!)}}
\newcommand{\covw}{\boldsymbol{\Sigma}^{(\!*\!)}}
\newcommand{\vphix}[1]{{\boldsymbol{\phi}\left({#1}\right)}}
\newcommand{\covbb}[1]{{\boldsymbol{\Sigma}^{(\diamond,{#1})}}}
\newcommand{\covww}[1]{{\boldsymbol{\Sigma}_c^{(*,{#1})}}}

\newcommand{\muw}{\boldsymbol{\mu}^{(*)}}
\newcommand{\mub}{\boldsymbol{\mu}^{(\diamond)}}

\newcommand{\mubb}[1]{{\boldsymbol{\mu}^{(\diamond,{#1})}}}
\newcommand{\muww}[1]{{\boldsymbol{\mu}_c^{(*,{#1})}}}
\newcommand{\muwww}[2]{{\boldsymbol{\mu}_{#1}^{(*,{#2})}}}

\newcommand{\mPhi}{\boldsymbol{\Phi}}
\newcommand{\mPhibar}{\boldsymbol{\bar{\Phi}}}

\newcommand{\parsmP}{\!\left(\mPhi\right)}
\newcommand{\parsmPc}{\!\left(\mPhi_c\right)}
\newcommand{\parsmPA}{\!\left(\mPhi^A\right)}
\newcommand{\parsmPB}{\!\left(\mPhi^B\right)}
\newcommand{\parsmPAB}{\!\left(\mPhi^A\!,\mPhi^B\right)}
\newcommand{\parsmPcA}{\!\left(\mPhi_c^A\right)}
\newcommand{\parsmPcB}{\!\left(\mPhi_c^B\right)}
\newcommand{\parsmPxY}[2]{{\!\left(\mPhi_{#1}^{#2}\right)}}

\newcommand{\mOmega}{\boldsymbol{\Omega}}
\newcommand{\mMu}{\boldsymbol{M}}
\newcommand{\mM}{\boldsymbol{M}}
\newcommand{\mF}{\boldsymbol{F}}
\newcommand{\parsmMu}{\!\left(\mMu\right)}
\newcommand{\mW}{\boldsymbol{W}}
\newcommand{\mD}{\boldsymbol{D}}
\newcommand{\vd}{\boldsymbol{d}}
\newcommand{\mT}{\boldsymbol{T}}
\newcommand{\mG}{\boldsymbol{G}}

\newcommand{\vp}{\boldsymbol{p}}

\newcommand{\vm}{\boldsymbol{m}}
\newcommand{\vmu}{\boldsymbol{\mu}}
\newcommand{\bvmu}{\boldsymbol{\overline{\mu}}}
\newcommand{\mP}{\boldsymbol{\Theta}}
\newcommand{\vmubar}{\boldsymbol{\bar{\mu}}}
\newcommand{\vvarphi}{\boldsymbol{\varphi}}


\newcommand{\stkout}[1]{{\ifmmode\text{\sout{\ensuremath{#1}}}\else\sout{#1}\fi}}

\newcommand{\mL}{\mathbf{L}}

\DeclareMathOperator*{\arcsinh}{arcsinh}
%%%%%%%%%%%%%%%%%%


\begin{document}
\begin{abstract}
	We introduce the notion of rooftop flip, namely a small modification among normal projective varieties which is modeled by a smooth projective variety of Picard number 2 admitting two projective bundle structures. Examples include the Atiyah flop and the Mukai flop, modeled respectively by $\P^1\times \P^1$ and by $\P\left(T_{\P^2}\right)$. Using the Morelli-W\l odarczyk cobordism, we prove that any smooth projective variety of Picard number 1, endowed with a $\C^*$-action with only two fixed point components, induces a rooftop flip.
\end{abstract}
\maketitle
\tableofcontents

%%%%%%%%%%%%%%%%% inputs

%!TEX root = BF.tex


\section{Introduction}\label{sec:intro}
		
The relation between birational geometry and $\C^*$-actions has been deeply studied over the years (see for instance \cite{ReidFlip}, \cite{Thaddeus} and in recent years \cite{WORS1}).
One famous example for instance is the notion of \emph{Morelli-W\l odarczyk cobordism} (cfr. \cite[Definition 2]{Wlodarczyk}, see Definition \ref{def: cobordism}), introduced to study birational maps among normal projective varieties which are realized as geometric quotients by a $\C^*$-action. This algebraic version of cobordism has been used to prove the \emph{Weak factorization conjecture} (see \cite{Wlodarczyk}). 

On the other hand, the relation between birational maps and $\C^*$-actions becomes evident in the example of the \emph{Atiyah flop} (see \cite[\S 1.3]{ReidFlip}). 
Consider the $\C^*$-action on $\C^4$ given by 
\[
t \cdot \left(x_0,x_1,y_0,y_1\right)=\left( tx_0,tx_1,t^{-1}y_0,t^{-1}y_1\right),
\]
where $t\in\C^*$ and $(x_0,x_1,y_0,y_1)\in \C^4$. The GIT quotient $\C^4\to \C^4\git \C^*$ is the affine cone over the Segre embedding of $\P^1 \times \P^1$.
The variety $\C^4 \git \C^*$  has a cone singularity at the origin, which can be resolved by blowing-up the vertex of the cone.
The exceptional divisor $\P^1\times \P^1$ of the blow-up can be contracted in two different ways, producing two smooth varieties $X_1$ and $X_2$, which are isomorphic on the complement of a closed subset of codimension greater or equal than $2$. The resulting birational map $X_1 \dashrightarrow X_2$ is called the Atiyah flop and $\C^4$ is an example of cobordism associated to it. The following picture summarizes the example:
\begin{figure}[h!]
\centering
\tikzset{every picture/.style={line width=0.75pt}} %set default line width to 0.75pt        

\begin{tikzpicture}[x=0.4pt,y=0.4pt,yscale=-1,xscale=1]
%uncomment if require: \path (0,789); %set diagram left start at 0, and has height of 789

%Shape: Polygon [id:ds4850704739937841] 
\draw  [draw opacity=0][fill={rgb, 255:red, 144; green, 19; blue, 254 }  ,fill opacity=0.2 ] (660,150) -- (640,250) -- (490,310) -- (490,310) -- (520,230) -- cycle ;
%Straight Lines [id:da41606237713260785] 
\draw    (30,250) -- (180,310) ;
%Straight Lines [id:da7936370010344177] 
\draw    (180,250) -- (30,310) ;
%Straight Lines [id:da15086398653346844] 
\draw    (200,260) -- (60,340) ;
%Straight Lines [id:da6385894053551876] 
\draw [line width=1.5]    (30,140) -- (180,200) ;
%Straight Lines [id:da25609817756390163] 
\draw [line width=1.5]    (660,150) -- (520,230) ;
%Shape: Polygon [id:ds30843765904369747] 
\draw  [draw opacity=0][fill={rgb, 255:red, 144; green, 19; blue, 254 }  ,fill opacity=0.2 ] (420,40) -- (420,150) -- (280,230) -- (280,230) -- (280,120) -- cycle ;
%Shape: Polygon [id:ds7839677942934724] 
\draw  [draw opacity=0][fill={rgb, 255:red, 74; green, 144; blue, 226 }  ,fill opacity=0.2 ] (400,90) -- (400,200) -- (250,140) -- (250,140) -- (250,30) -- cycle ;
%Shape: Polygon [id:ds5220547568697286] 
\draw  [draw opacity=0][fill={rgb, 255:red, 144; green, 19; blue, 254 }  ,fill opacity=0.2 ] (660,150) -- (660,260) -- (520,340) -- (520,340) -- (520,230) -- cycle ;
%Shape: Polygon [id:ds377636801452376] 
\draw  [draw opacity=0][fill={rgb, 255:red, 74; green, 144; blue, 226 }  ,fill opacity=0.2 ] (180,200) -- (200,280) -- (50,240) -- (50,240) -- (30,140) -- cycle ;
%Shape: Polygon [id:ds253289493622184] 
\draw  [draw opacity=0][fill={rgb, 255:red, 74; green, 144; blue, 226 }  ,fill opacity=0.2 ] (420,60) -- (420,170) -- (270,130) -- (270,130) -- (270,20) -- cycle ;
%Shape: Polygon [id:ds2831812809771401] 
\draw  [draw opacity=0][fill={rgb, 255:red, 144; green, 19; blue, 254 }  ,fill opacity=0.2 ] (400,30) -- (400,140) -- (250,200) -- (250,200) -- (250,90) -- cycle ;
%Shape: Polygon [id:ds881112108538183] 
\draw  [draw opacity=0][fill={rgb, 255:red, 74; green, 144; blue, 226 }  ,fill opacity=0.2 ] (335,325) -- (410,465) -- (260,405) -- (260,405) -- (335,325) -- cycle ;
%Shape: Polygon [id:ds6130139984748016] 
\draw  [draw opacity=0][fill={rgb, 255:red, 74; green, 144; blue, 226 }  ,fill opacity=0.2 ] (335,325) -- (430,435) -- (280,395) -- (280,395) -- (335,325) -- cycle ;
%Shape: Polygon [id:ds39194821844628935] 
\draw  [draw opacity=0][fill={rgb, 255:red, 144; green, 19; blue, 254 }  ,fill opacity=0.2 ] (335,325) -- (430,415) -- (290,495) -- (290,495) -- (335,325) -- cycle ;
%Shape: Polygon [id:ds8888399328308494] 
\draw  [draw opacity=0][fill={rgb, 255:red, 144; green, 19; blue, 254 }  ,fill opacity=0.2 ] (335,325) -- (410,405) -- (260,465) -- (260,465) -- (335,325) -- cycle ;
%Straight Lines [id:da2141826957544144] 
\draw    (170,330) -- (248.3,378.94) ;
\draw [shift={(250,380)}, rotate = 212.01] [color={rgb, 255:red, 0; green, 0; blue, 0 }  ][line width=0.75]    (10.93,-3.29) .. controls (6.95,-1.4) and (3.31,-0.3) .. (0,0) .. controls (3.31,0.3) and (6.95,1.4) .. (10.93,3.29)   ;
%Straight Lines [id:da059430083941788725] 
\draw    (431.63,378.84) -- (500,330) ;
\draw [shift={(430,380)}, rotate = 324.46] [color={rgb, 255:red, 0; green, 0; blue, 0 }  ][line width=0.75]    (10.93,-3.29) .. controls (6.95,-1.4) and (3.31,-0.3) .. (0,0) .. controls (3.31,0.3) and (6.95,1.4) .. (10.93,3.29)   ;
%Straight Lines [id:da09909378224217413] 
\draw  [dash pattern={on 4.5pt off 4.5pt}]  (448,260) -- (220,260) ;
\draw [shift={(450,260)}, rotate = 180] [color={rgb, 255:red, 0; green, 0; blue, 0 }  ][line width=0.75]    (10.93,-3.29) .. controls (6.95,-1.4) and (3.31,-0.3) .. (0,0) .. controls (3.31,0.3) and (6.95,1.4) .. (10.93,3.29)   ;
%Shape: Polygon [id:ds4175242386704846] 
\draw  [draw opacity=0][fill={rgb, 255:red, 74; green, 144; blue, 226 }  ,fill opacity=0.2 ] (650,155) -- (660,280) -- (660,280) -- (510,240) -- (510,240) -- cycle ;
%Shape: Polygon [id:ds7329252346902249] 
\draw  [draw opacity=0][fill={rgb, 255:red, 74; green, 144; blue, 226 }  ,fill opacity=0.2 ] (580,195) -- (640,310) -- (640,310) -- (490,250) -- (490,250) -- cycle ;
%Straight Lines [id:da8359005539718207] 
\draw    (151.7,168.94) -- (230,120) ;
\draw [shift={(150,170)}, rotate = 327.99] [color={rgb, 255:red, 0; green, 0; blue, 0 }  ][line width=0.75]    (10.93,-3.29) .. controls (6.95,-1.4) and (3.31,-0.3) .. (0,0) .. controls (3.31,0.3) and (6.95,1.4) .. (10.93,3.29)   ;
%Straight Lines [id:da7055480908011266] 
\draw    (440,120) -- (518.3,168.94) ;
\draw [shift={(520,170)}, rotate = 212.01] [color={rgb, 255:red, 0; green, 0; blue, 0 }  ][line width=0.75]    (10.93,-3.29) .. controls (6.95,-1.4) and (3.31,-0.3) .. (0,0) .. controls (3.31,0.3) and (6.95,1.4) .. (10.93,3.29)   ;
%Shape: Circle [id:dp16639528362810663] 
\draw  [fill={rgb, 255:red, 0; green, 0; blue, 0 }  ,fill opacity=1 ] (330,325) .. controls (330,322.24) and (332.24,320) .. (335,320) .. controls (337.76,320) and (340,322.24) .. (340,325) .. controls (340,327.76) and (337.76,330) .. (335,330) .. controls (332.24,330) and (330,327.76) .. (330,325) -- cycle ;
%Straight Lines [id:da6020295014555623] 
\draw    (240,415) -- (380,495) ;
%Straight Lines [id:da6951138713448372] 
\draw    (430,415) -- (290,495) ;
%Straight Lines [id:da8097512426943787] 
\draw    (410,405) -- (260,465) ;
%Straight Lines [id:da13247989979656716] 
\draw    (390,395) -- (240,435) ;
%Straight Lines [id:da48040891417002174] 
\draw    (260,405) -- (410,465) ;
%Straight Lines [id:da34274536279481416] 
\draw    (280,395) -- (430,435) ;
%Straight Lines [id:da4308686990205012] 
\draw    (50,240) -- (200,280) ;
%Shape: Polygon [id:ds7461534503115913] 
\draw  [draw opacity=0][fill={rgb, 255:red, 74; green, 144; blue, 226 }  ,fill opacity=0.2 ] (180,200) -- (180,310) -- (30,250) -- (30,250) -- (30,140) -- cycle ;
%Straight Lines [id:da17577793998936353] 
\draw    (10,260) -- (150,340) ;
%Shape: Polygon [id:ds6151751696139521] 
\draw  [draw opacity=0][fill={rgb, 255:red, 144; green, 19; blue, 254 }  ,fill opacity=0.2 ] (200,260) -- (200,260) -- (60,340) -- (60,340) -- (140,185) -- cycle ;
%Shape: Polygon [id:ds3190444980145123] 
\draw  [draw opacity=0][fill={rgb, 255:red, 144; green, 19; blue, 254 }  ,fill opacity=0.2 ] (180,250) -- (180,250) -- (30,310) -- (30,310) -- (90,165) -- cycle ;
%Straight Lines [id:da9454912707573251] 
\draw    (160,240) -- (10,280) ;
%Straight Lines [id:da8627817917798487] 
\draw    (620,240) -- (470,280) ;
%Straight Lines [id:da7849167182699056] 
\draw    (640,250) -- (490,310) ;
%Straight Lines [id:da6426667173315391] 
\draw    (660,260) -- (520,340) ;
%Straight Lines [id:da036192532873840944] 
\draw    (470,260) -- (610,340) ;
%Straight Lines [id:da5436269492804261] 
\draw    (490,250) -- (640,310) ;
%Straight Lines [id:da3328774675830918] 
\draw    (510,240) -- (660,280) ;
%Straight Lines [id:da15128144566750634] 
\draw [line width=1.5]    (270,20) -- (420,60) ;
%Straight Lines [id:da000043131873048385394] 
\draw [line width=1.5]    (250,30) -- (400,90) ;
%Straight Lines [id:da6139619875956873] 
\draw [line width=1.5]    (230,40) -- (370,120) ;
%Straight Lines [id:da299862555826932] 
\draw [line width=1.5]    (380,20) -- (230,60) ;
%Straight Lines [id:da44278809508998673] 
\draw [line width=1.5]    (400,30) -- (250,90) ;
%Straight Lines [id:da45668783873732766] 
\draw [line width=1.5]    (420,40) -- (280,120) ;
%Straight Lines [id:da9216283818418308] 
\draw    (230,150) -- (370,230) ;
%Straight Lines [id:da8236025108977942] 
\draw    (250,140) -- (400,200) ;
%Straight Lines [id:da3807742075199273] 
\draw    (270,130) -- (420,170) ;
%Straight Lines [id:da6293654960419728] 
\draw    (420,150) -- (280,230) ;
%Straight Lines [id:da2652516871601991] 
\draw    (400,140) -- (250,200) ;
%Straight Lines [id:da13928252133219965] 
\draw    (380,130) -- (230,170) ;

% Text Node
\draw (410,0) node [anchor=north west][inner sep=0.75pt]    {$\P^1 \times \P^1$};
% Text Node
\draw (620,130) node [anchor=north west][inner sep=0.75pt]    {$\mathbb{P}^{1}$};
% Text Node
\draw (70,120) node [anchor=north west][inner sep=0.75pt]    {$\mathbb{P}^{1}$};
% Text Node
\draw (300,500) node [anchor=north west][inner sep=0.75pt]    {$\C^4 \git \C^*$};
% Text Node
\draw (90,340) node [anchor=north west][inner sep=0.75pt]    {$X_1$};
% Text Node
\draw (550,340) node [anchor=north west][inner sep=0.75pt]    {$X_2$};
% Text Node
%\draw (315,220) node [anchor=north west][inner sep=0.75pt]    {$\Bl_0(\C^4 \git \C^*)$};


\end{tikzpicture}
\end{figure}

\noindent
One may generalize this example by considering a similar $\C^*$-action on $\C^{n+1}$ (see for instance \cite[Example 1]{Wlodarczyk}), or allowing also weights different from $\pm 1$ (cf. \cite[\S 3]{BR}): the resulting birational map is called the \emph{Atiyah flip}.
In the above construction of the Atiyah flop, the varieties $X_1$ and $X_2$ are defined using the two projective bundle structures on $\P^1 \times \P^1$: 
\begin{equation*}\label{eq:rooftopP1xP1}
\xymatrix{&\P^1 \times \P^1 \ar[ld]_{} \ar[rd]^{} & \\ \P^1 && \P^1}
\end{equation*}
Smooth projective varieties with two different projective bundle structures have been already studied in the literature (see for instance \cite{ORS}): in particular, in \cite[Lemma 4.4]{WORS1} the authors have constructed a correspondence between them and smooth projective varieties $X$ of Picard number $1$ admitting a $\C^*$-action having only two fixed point components. 

Motivated by the example of Atiyah flop and its connection with a variety admitting two projective bundle structures, we introduce the notion of \emph{rooftop flip}.
Given a smooth projective variety $\Lambda$ of Picard number $2$ admitting two projective bundle structures 
\begin{equation*}\label{eq:rooftopP1xP1}
	\xymatrix{&\Lambda \ar[ld]_{} \ar[rd]^{} & \\ \Lambda_- && \Lambda_+,}
\end{equation*}
a \emph{rooftop flip modeled by $\Lambda$} (see Definition \ref{def:Dflips}) is a small modification $\psi: W_-\dashrightarrow W_+$ among normal projective varieties which can be resolved by a common divisorial extraction 

\[
\xymatrix{ &W \ar[ld]_{b_-} \ar[rd]^{b_+} & \\ W_- \ar@{-->}[rr]_\psi && W_+}
\]
such that the restriction of $b_{\pm}: W\to W_{\pm}$ to the exceptional loci are modeled by the projective bundle maps $p_{\pm}: \Lambda\to \Lambda_{\pm}$.


In this setting, the Atiyah flop is a rooftop flip modeled by $\P^1\times \P^1$.  As we will see, the class of rooftop flips includes classical birational transformations: the Atiyah flip is a rooftop flip modeled by $\P^m\times \P^l$, and the Mukai flop (see \cite{HZ04}, \cite{WW}) is a rooftop flip modeled by $\P\left(T_{\P^2}\right)$. Moreover, rooftop flips appear as local models of the birational transformations of the geometric quotients of a smooth polarized pair $(X,L)$ under a $\C^*$-actions (see \cite[Theorem 1.1 (1)]{WORS3})
In the paper we discuss the definition of  rooftop flip and its connection with smooth projective varieties having a $\C^*$-action with only two fixed point components.
The main result of the paper is the following:
		
\begin{theorem}\label{theorem:naive}
	Given a smooth projective variety $\Lambda$ of Picard number $2$ with two projective bundle structures, there exist two quasi-projective varieties and a rooftop flip modeled by $\Lambda$ among them.
\end{theorem}

\subsection*{Outline}

In Section \ref{sec:preliminaries} we first recall some notions regarding $\C^*$-actions on smooth polarized pairs. We then focus on the case of $\C^*$-action whose associated fixed point locus consists only of $2$ connected components; such varieties are called \emph{drums} (see Definition \ref{definition:drum}) and they are constructed upon the choice of a triple $(Y,\cL_-,\cL_+)$, where $Y$ is a smooth projective variety of Picard number $2$ admitting two projective bundle structure, and $\cL_{\pm}$ are semiample line bundles on $Y$. We conclude by recalling the relation between smooth drums and the classification of horospherical varieties done by \cite{Pas} (see Remark \ref{remark:pasquier}).

In Section \ref{sec:atiyah}, after recalling the Morelli-W\l odarczyk cobordism (see Definition \ref{def: cobordism}), we introduce and discuss the notion of rooftop flip (see Definition \ref{def:Dflips}). We then revisit the example of Atiyah flip and prove that it is indeed a rooftop flip (see \S \ref{ssec:atiyah}).

In Section \ref{sec:mainresult} we prove Theorem \ref{theorem:naive}. The idea of the proof is to view our case as a restriction of the Morelli-W\l odarczyk cobordism associated to the Atiyah flip. We conclude by deducing that the drum structure on the smooth quadric hypersurface induces a rooftop flip modeled by $\P\left(T_{\P^n}\right)$, see Corollary \ref{corollary:rooftopflipquadric}. 

\subsection*{Acknowledgements} We would like to thank Luis E. Sol\'a Conde for having suggested this problem, for all the stimulating conversations and the important suggestions. %We would like also to thank Gianluca Occhetta, Eleonora A. Romano and Jaroslaw A. Wi\'sniewski for the careful regarding of this work. 
\section{Notation and Preliminaries}\label{sec_prel}
Let $\mathbb{Z}_{>0}$ denote the set of positive integers and let $\mathbb{Z}_{[a,b]}$ denote the set of integers in the interval $[a,b]$. The $m\times m$ identity matrix is denoted by $I_m$ and its columns by $e_i$ for $i\in\mathbb{Z}_{[1,m]}$. We use $\mathbf{0}$ to denote a vector or a matrix of zeros of appropriate dimensions. For a sequence $\{z_k\}_{k=0}^{N-1}$ with $z_k\in\mathbb{R}^\eta$, we denote its stacked vector as $z = \begin{bmatrix}z_0^\top &z_1^\top & \dots & z_{N-1}^\top\end{bmatrix}^\top$ and a stacked window of it as $z_{[l,j]} = \begin{bmatrix}z_l^\top &z_{l+1}^\top & \dots & z_{j}^\top\end{bmatrix}^\top$ with $0\leq l<j$.\par
Persistence of excitation of a sequence and its extension to multiple sequences \cite{vanWaarde20} are defined as follows.
\begin{definition} The sequence \(\{z_k\}_{k=0}^{N-1}\), $z_k\in\mathbb{R}^{\eta}$, is said to be persistently exciting of order \(L\) if \(\textup{rank}(\mathscr{H}_{L}(z))=\eta L\), where $\mathscr{H}_L(z) = \begin{bmatrix}
		z_{[0,L-1]} & z_{[1,L]} & \cdots & z_{[N-L,N-1]}
	\end{bmatrix}$.
	\label{def_PE}
\end{definition}
\begin{definition}[\cite{vanWaarde20}]\label{def_cPE}
	The sequences $\{z_k^{(j)}\}_{k=0}^{N_j-1}$, with $z_k^{(j)}\in\mathbb{R}^\eta$ and $j\in\mathbb{Z}_{[1,r]}$, are said to be \textit{collectively persistently exciting} of order $L$ if rank$(\mathcal{H}_L(\mathscr{Z}))=\eta L$, where $\mathscr{Z} = \begin{bmatrix}
		(z^{(1)})^\top & \cdots & (z^{(r)})^\top
	\end{bmatrix}^\top,$ and
	\begin{equation*}
		\mathcal{H}_L(\mathscr{Z}) = \begin{bmatrix}
			\mathscr{H}_L(z^{(1)}) & \cdots & \mathscr{H}_L(z^{(r)})
		\end{bmatrix}.
	\end{equation*}
\end{definition}
%!TEX root = BF.tex


\section{Morelli-W\l odarczyk cobordism and rooftop flips}\label{sec:atiyah}

In this section we first recall the definition of \emph{Morelli-W\l odarczyk cobordism} and we introduce the notion of \emph{rooftop flip}. We will then concentrate on studying the Atiyah flip; the reason is two-fold: on one hand this birational transformation, as we will see, can be constructed in terms of the Morelli-W\l odarczyk cobordism; on the other hand, it is the motivating example of the notion of rooftop flip.

\begin{definition}\label{def: cobordism}
	Let $X_1,X_2$ be birationally equivalent normal varieties. The \emph{Morelli--W\l odarczyk cobordism} between $X_1$ and $X_2$ is a normal variety $B$, endowed with a $\C^*$-action such that 
	\begin{equation*}
		\begin{split}
			B_+&=\{p\in B\mid \lim_{t\to 0} tp \text{ does not exists}\},\\
			B_-&=\{p\in B\mid \lim_{t\to \infty} tp \text{ does not exists}\}
		\end{split}
	\end{equation*}
	are non-empty open subsets of $B$, such that there exist geometric quotients $B_-/\C^*$ and $B_+/\C^*$ satisfying
	$$B_-/\C^*\simeq X_1 \dashrightarrow X_2\simeq B_+/\C^*,$$
	where the birational equivalence is realized by the open subset $(B_-\cap B_+)/\C^*$ contained in $B_{\pm}/\C^*$.
\end{definition}

We stress that the notation in the above Definition is slightly different from the original one (cf. \cite[Definition 2]{Wlodarczyk}), in particular the role on $B_-$ and $B_+$ are switched. The reason behind this apparent misleading decision is that in this setting it will be less confusing in the next section to keep track of the $\pm$-signs.

The following definition is the core of the section.
Note that a similar definition appears in the contest of \emph{$K$-equivalence}, see for instance \cite{Kan2}.

\begin{definition}\label{def:Dflips}
	Consider a normal projective variety $\Lambda$ with $\rho_{\Lambda}=2$ admitting two projective bundle structures:
	\[
	\xymatrix{ & \Lambda \ar[ld]_{p_-} \ar[rd]^{p_+} & \\ \Lambda_- && \Lambda_+}
	\]
	A small modification $\psi: W_- \dashrightarrow W_+$ between normal projective varieties is called a \emph{rooftop flip modeled by $\Lambda$} if the following holds: 
	\begin{enumerate}
		\item There are small contractions $s_\pm: W_{\pm}\to W_0$, with $W_0$ a normal projective variety,
			\[
			\xymatrix{W_- \ar@{-->}[rr]^\psi \ar[rd]_{s_-} && W_+, \ar[ld]^{s_+} \\ & W_0 & }
			\]  such that, denoting by $Z_\pm \subset W_\pm$ their exceptional loci, the restrictions $s_\pm|_{Z_\pm}: Z_\pm \to Z_0 \subset W_0$ are smooth and the fibers are $\Lambda_\pm$-bundles.
		\item There is a resolution
			\[
			\xymatrix{ &W \ar[ld]_{b_-} \ar[rd]^{b_+} & \\ W_- \ar@{-->}[rr]_\psi && W_+}
			\]
			such that $Z:=b_\pm^{-1}(Z_\pm) \subset W$ is a divisor, and $b_\pm|_Z: Z \to Z_\pm$ defines projective bundle structures on $Z$.
	\item For any $z_0 \in Z_0$ we have that $b^{-1}_\pm|_{s^{-1}_\pm(z_0)}=p_\pm^{-1}$:
	\[
	\xymatrix@C0.5pt{ &(b_-^{-1}\circ s_-^{-1})(z_0)\simeq \Lambda \simeq (b_+^{-1}\circ s_+^{-1})(z_0) \ar[ld]_{p_-} \ar[rd]^{p_+} & \\ s_-^{-1}(z_0) \simeq \Lambda_- \ar[rd]_{s_-} %\ar@{-->}[rr] 
		&& \Lambda_+ \simeq s_+^{-1}(z_0) \ar[ld]^{s_+} \\ & z_0& }
	\]
	\end{enumerate}	
\end{definition}

\begin{remark}\label{remark:why}
	As we will see, the definition of rooftop flip includes some classical birational transformations: the Atiyah flip is a rooftop flip modeled by $\P^m\times \P^l$ (see \S\ref{ssec:atiyah}), and the classic Mukai flop (see \cite{HZ04}, \cite{WW}) is a rooftop flip modeled by $\P\left(T_{\P^2}\right)$ (see Corollary \ref{corollary:rooftopflipquadric}).
\end{remark}


\subsection{Atiyah rooftop flips}\label{ssec:atiyah}

In this section we show that the Atiyah flip is in particular a rooftop flip modeled by $\P^{m}\times \P^{l}$.


\begin{setup}\label{setup:atiyah}
	Let $V_-$ and $V_+$ denote the complex vector spaces of dimension respectively $m+1$ and $l+1$, with $l,m\ge 1$, and set $V:=V_-\oplus V_+$.
	Consider the $\C^*$-action on $V$ having weight $-1$ on $V_-$ and weight $1$ on $V_+$. 
	There is an induced $\C^*$-action on $V^\vee$ given by $t\cdot v= \left(t v_-,t^{-1}v_+\right)$, where $v=\left(v_-,v_+\right)\in V^\vee$.
	We will frequently abuse notation by writing $V_{-}$ (resp. $V_+$) for $V_-\times \{0\}$ (resp. $\{0\}\times V_+$).
\begin{figure}[h!]
	\centering
	

\tikzset{every picture/.style={line width=0.75pt}} %set default line width to 0.75pt        

\begin{tikzpicture}[x=0.6pt,y=0.6pt,yscale=-1,xscale=1]
%uncomment if require: \path (0,300); %set diagram left start at 0, and has height of 300

%Shape: Rectangle [id:dp18937098784415363] 
\draw   (20,20) -- (340,20) -- (340,180) -- (20,180) -- cycle ;
%Straight Lines [id:da1318788053019485] 
\draw    (20,20) -- (340,180) ;
%Straight Lines [id:da9054914373607481] 
\draw    (20,180) -- (340,20) ;
%Straight Lines [id:da9510825035310609] 
\draw    (30,210) -- (330,210) ;
\draw [shift={(186,210)}, rotate = 180] [color={rgb, 255:red, 0; green, 0; blue, 0 }  ][line width=0.75]    (10.93,-3.29) .. controls (6.95,-1.4) and (3.31,-0.3) .. (0,0) .. controls (3.31,0.3) and (6.95,1.4) .. (10.93,3.29)   ;
%Shape: Circle [id:dp05608194391391974] 
\draw  [fill={rgb, 255:red, 0; green, 0; blue, 0 }  ,fill opacity=1 ] (175,100) .. controls (175,97.24) and (177.24,95) .. (180,95) .. controls (182.76,95) and (185,97.24) .. (185,100) .. controls (185,102.76) and (182.76,105) .. (180,105) .. controls (177.24,105) and (175,102.76) .. (175,100) -- cycle ;
%Shape: Triangle [id:dp24989332432828237] 
\draw  [draw opacity=0][fill={rgb, 255:red, 144; green, 19; blue, 254 }  ,fill opacity=0.2 ] (180,100) -- (20,180) -- (20,20) -- cycle ;
%Shape: Triangle [id:dp5377581483415403] 
\draw  [draw opacity=0][fill={rgb, 255:red, 208; green, 2; blue, 27 }  ,fill opacity=0.2 ] (180,100) -- (340,20) -- (340,180) -- cycle ;
%Straight Lines [id:da7000627750204172] 
\draw    (30,40) -- (150,90) ;
%Straight Lines [id:da5322630758087111] 
\draw    (30,70) -- (150,95) ;
%Straight Lines [id:da8248391946037721] 
\draw    (30,160) -- (150,110) ;
%Straight Lines [id:da2719071105268458] 
\draw    (30,130) -- (150,105) ;
%Straight Lines [id:da7756346880199279] 
\draw    (210,95) -- (330,70) ;
%Straight Lines [id:da21911398751314515] 
\draw    (210,90) -- (330,40) ;
%Straight Lines [id:da2981416153893458] 
\draw    (210,110) -- (330,160) ;
%Straight Lines [id:da27585222842907664] 
\draw    (210,105) -- (330,130) ;
%Curve Lines [id:da11209114331809367] 
\draw    (40,25) .. controls (180.33,90.17) and (180.33,90.17) .. (320,25) ;
%Curve Lines [id:da3693485194056053] 
\draw    (55,25) .. controls (175,70.17) and (185.33,70.17) .. (305,25) ;
%Curve Lines [id:da664510041065356] 
\draw    (40,175) .. controls (180,110.75) and (180,110.5) .. (320,175) ;
%Curve Lines [id:da46252259164860765] 
\draw    (55,175) .. controls (180,130.17) and (180.33,130.17) .. (305,175) ;

% Text Node
\draw (180,215) node [anchor=north west][inner sep=0.75pt]    {$t$};
% Text Node
\draw (20,215) node [anchor=north west][inner sep=0.75pt]    {$\infty $};
% Text Node
\draw (325,215) node [anchor=north west][inner sep=0.75pt]    {$0$};
% Text Node
\draw (50,95) node [anchor=north west][inner sep=0.75pt]    {$V_{-}^{\lor }$};
% Text Node
\draw (285,95) node [anchor=north west][inner sep=0.75pt]    {$V_{+}^{\lor }$};
% Text Node
\draw (350,15) node [anchor=north west][inner sep=0.75pt]    {$V^{\lor }$};


\end{tikzpicture}
	\end{figure}
\end{setup}


\begin{remark}\label{remark:homoteties}
	The restriction of the $\C^*$-action on $V^\vee$ defined in Set-up \ref{setup:atiyah} to $V_-^\vee$ (resp. $V_+^\vee$) coincides with $\C^*_h$-action on $V^\vee_-$ (resp. $V^\vee_+$) by homoteties.
\end{remark}

 The fixed point locus of this action on $V^\vee$ coincides with the origin. 
 We can consider the induced $\C^*$-action of the coordinate ring of $V^\vee$, namely $\C[V^\vee]=\C\left[y_0,\ldots,y_m,x_0,\ldots,x_l\right]$.
 The GIT-quotient $V^\vee\to V^\vee\git \C^*:=\Spec \C[V^\vee]^{\C^*}$ is the affine cone over the Segre embedding of $\P\left(V_-\right)\times \P\left(V_+\right) \simeq \P^m \times \P^l$, therefore singular at the origin of $V_-^\vee \otimes V_+^\vee$.

\begin{remark}\label{remark:cobordism}
	Under the notation of Definition \ref{def: cobordism}, we have that
	\begin{align*}
	&B_- \simeq V^\vee\setminus \left\{y_0=\ldots=y_m=0\right\},\\
	&B_+\simeq V^\vee\setminus \left\{x_0=\ldots=x_l=0\right\}.
	\end{align*}
	In particular $B_{\pm}$ are non-empty open subsets of stable points under the $\C^*$-action, therefore we have two geometric quotients $B_{\pm}\to B_{\pm}/\C^*$.
\end{remark}


\begin{lemma}\label{lemma:atiyahexceptionallocus}
	The geometric quotients $B_-/\C^*$ and $B_+/\C^*$ are birational, and the exceptional locus of the birational map $\psi: B_-/\C^*\dashrightarrow B_+/\C^*$ is $\P\left(V_-\right)$. In particular, $\psi$ is a small modification.
\end{lemma}

\begin{proof}
	Notice that $B_-/\C^*$ and $B_+/\C^*$ are birational since they contain the open subset $(B_-\cap B_+)/\C^*\neq \emptyset$ %= (V^\vee\setminus \left(\left\{y_0=\ldots=y_m=0\right\} \cup \left\{x_0=\ldots=x_l=0\right\} \right))/\C^*\neq \emptyset$.
	Let us study the exceptional locus of $\psi$. 
	It suffices to show that $V_-^\vee\setminus 0\subset B_-$ (then $\P\left(V_-\right)\subset B_-/\C^*$) and that $V_-^\vee\setminus 0\not\subset B_+$. 
	Let $p\in \P\left(V_-\right)$, and consider the associated line $\hat{p}$ in $V_-^\vee$ passing through the origin. Then $\hat{p}=\left(hv_-,0\right)$, for $h\in \C^*_h$ and $v_-\in V_-^\vee$. 
	For any point $q\in \hat{p} \setminus 0$ we have that that $\lim_{t\to \infty} t\cdot q$ does not exist, hence every point $q$ of $\hat{p}\setminus 0$ belongs to $B_-$. 
	By Remark \ref{remark:homoteties}, the restriction of the $\C^*$ on $V_-^\vee \setminus 0$ coincides with the action by homoteties, so we have that $p\in B_-/\C^*$. 
	On the other hand it is easy to see that $\lim_{t\to 0} t\cdot q$ exists, hence $q\notin B_+$.
	 Finally $\psi$ is a small modification since $\codim_{B_-/\C^*} \text{Exc}(\psi)\geq 2$. 
\end{proof}

\begin{remark}
	Let us recall that the example of Atiyah flip can be also easily described using toric geometry. In particular the birational map is induced by two different subdivisions of a cone. We refer to \cite[\S 3]{BR} for a detailed discussion from the toric point of view.
\end{remark}

\begin{corollary}
	The exceptional locus of $\psi^{-1}: B_+/\C^*\dashrightarrow B_-/\C^*$ is $\P(V_+)$.
\end{corollary}

\begin{remark}\label{remark:atiyahblowup}
	Let $\beta: W\to V^\vee\git \C^*$ be the blow-up of $V^\vee \git \C^*$ along the origin. Then the exceptional divisor is $\P(V_-)\times \P(V_+)$. 
\end{remark}

\begin{lemma}\label{lemma:atiyahfactorization}
	The blow-up $\beta: W\to V^\vee\git \C^*$ can be factorized through $b_{\pm}: W\to B_{\pm}/\C^*$ and the small contractions $s_{\pm}: B_{\pm}/\C^*\to V^\vee\git \C^*$ with exceptional loci $\P(V_{\pm})$. 
\end{lemma}

\begin{proof}
	There exist two natural birational morphisms $s_{\pm}: B_{\pm}/\C^*\to V^\vee\git \C^*$, isomorphisms over the set of points which are semistable but not stable. From this it follows that $\Exc(s_{\pm})=\P(V_{\pm})$. Since they have codimension greater than two, we conclude $s_{\pm}$ are small contractions. 
	Moreover, since $W$ is birational to $V^\vee\git \C^*$, which is birational to $B_{\pm}/\C^*$, we conclude there exist birational maps $b_{\pm}: W\dashrightarrow  B_{\pm}/\C^*$, defined over $W\setminus (\P(V_-)\times \P(V_+))$. Since the exceptional locus has codimension one, $b_{\pm}$ are morphisms.	
	Let us prove that $b_{\pm}\circ s_{\pm}=\beta$. For sake of simplicity, let us consider $b_-\circ s_-$. It is immediate to notice that $b_-^{-1}(s_-^{-1}(0))=b_-^{-1}(\P(V_-))=\P(V_-)\times \P(V_+) =\beta^{-1}(0)$. 
\end{proof}



We summarize the above construction by means of the diagram

%\begin{center}
\[
\xymatrix{ & W \ar[ld]_{b_-} \ar[rd]^{b_+} \ar[dd]^<<<<<<\beta & \\ B_-/\C^* \ar@{-->}[rr]^<<<<<<\psi \ar[rd]_{s_-} && B_+/\C^* \ar[ld]^{s_+} \\ & V^\vee \git \C^* & }
\]


and its restriction to the exceptional loci:
\[
\xymatrix{ & \P\left(V_-\right) \times \P\left(V_+\right) \ar[ld]_{b_-} \ar[rd]^{b_+}  & \\ \P\left(V_-\right) \ar[rd]_{s_-} && \P\left(V_+\right) \ar[ld]^{s_+} \\ & 0 &}
\]


\begin{theorem}
	The birational map $\psi:B_-/\C^*\dashrightarrow B_+/\C^*$ is a rooftop flip modeled by $\P(V_-)\times \P(V_+)$.
\end{theorem}

\begin{proof}
	We verify that each condition of Definition \ref{def:Dflips} is satisfied.
	For $(1)$, Using Lemma \ref{lemma:atiyahexceptionallocus}, the birational map $\psi: B_-/\C^*\dashrightarrow B_+/\C^*$ is a small modification, the exceptional loci of $s_{\pm}: B_{\pm}/\C^*\to \hat{X}\git \C^*$ are $\P(V_{\pm})$ and the restriction $s_{\pm}|_{\P(V_{\pm})}: \P(V_{\pm}) \to 0$ are obviously $\P(V_{\pm})$-bundles. 
	Let us prove verify $(2)$: if we consider the resolution $b_{\pm}: W\to B_{\pm}/\C^*$ we have that $\P(V_-)\times \P(V_+)=b^{-1}_{\pm}(\P(V_{\pm}))$ is a divisor, and $\P(V_-)\times \P(V_+)\to \P(V_{\pm})$ defines two projective bundle structures.
	Finally, we consider $(3)$: In this case $Z_0$ is the origin, and we know that $s^{-1}_{\pm}(0)\simeq \P(V_{\pm})$. Moreover $(b_{\pm}^{-1}\circ s_{\pm}^{-1})(0)\simeq \P(V_-)\times \P(V_+)$, hence we conclude.
\end{proof}

\begin{corollary}
	The geometric quotients $B_{\pm} /\C^*$ are smooth, and the rooftop flip $\psi: B_-/\C^*\dashrightarrow B_+/\C^*$ is in particular a small $\Q$-factorial modification.
\end{corollary}

\begin{proof}
	Since the non-empty open subset $B_{\pm}$ are smooth, and $\C^*$ acts freely on them, using \cite[Corollary p.199]{MFK} we obtain that $B_{\pm}$ are $\C^*$-principal bundles over $B_{\pm}/\C^*$, hence they are smooth. 
\end{proof}







%!TEX root = BF.tex


\section{Main result}\label{sec:mainresult}
	
This section is devoted to prove Theorem \ref{theorem:naive}; we will adapt the example of Atiyah flip explained in Section \ref{sec:atiyah} in greater generality.

\begin{setup}\label{setup:maintheorem}
	Let $X$ be a smooth drum constructed upon a triple $(Y,\cL_-,\cL_+)$ (cfr. \S \ref{ssec:drums}).
	Let $\hat{X}$ be the affine cone over $X$, contained in the affine space $V^\vee:=V_-^\vee\oplus V_+^\vee$ (cfr. Remark \ref{remark:drumembedding}), where 
	\[
	V_-:=\HH^0(Y_-,L_-), \qquad V_+:=\HH^0(Y_+,L_+).
	\]
	The $\C^*$-action on $V^\vee$ is given by $t\cdot v=\left (t v_-,t^{-1}v_+\right)$, where $v=\left(v_-,v_+\right)\in V^\vee$.
\end{setup}
		
Notice that the $\C^*$-action on $V^\vee$ is induced by a $\C^*$-action on $V$ as described in Set-up \ref{setup:atiyah}. 

Consider the restriction of the $\C^*$-action on $V^\vee$ to $\hat{X}$; there exists a GIT quotient $\hat{X}\to \hat{X}\git\C^*$, singular at the origin. Moreover we can also consider the intersection $\hat{X}\cap B_{\pm}$, which are non-empty open subsets of stable points giving  geometric quotients $\pi_{\pm}:\hat{X}\cap B_{\pm}\to \hat{X}\cap B_{\pm}/\C^*$. 
	
\begin{proposition}\label{proposition:mainexceptionallocus}
	There exist a small modification \[\psi: \hat{X}\cap B_-/\C^*\dashrightarrow \hat{X}\cap B_+/\C^*,\] whose exceptional locus is $Y_-$. 
\end{proposition}

\begin{proof}
	The existence of such a birational map is immediate after noticing that $\hat{X}\cap B_-\cap B_+$ is open and non-empty.
	Let us prove that the exceptional locus of $\psi$ is $Y_-$. It suffices to show that 
	$$(\hat{X}\cap B_-/\C^*) \cap \P(V_-) =Y_-.$$
	Notice that the $\supset$ inclusion is trivial, so let us focus on the $\subset$ inclusion. Let $p\in (\hat{X}\cap B_-/\C^*) \cap \P(V_-)$, and let $\hat{p}=(hv_-,0)$ the corresponding line in $V_-^\vee$ through the origin, with $h\in \C^*_h$. The preimage $\pi^{-1}_-(p)$ is a closed orbit $\C^*\cdot q$ such that $\lim_{t\to \infty} t\cdot q$ does not exist, hence $\C^*\cdot q=\{(tv_-,v_+)\mid t\in \C^*\}$. Since $p$ belongs to the intersection, $\C^*\cdot q=\{(tv_-,0)\mid t\in \C^*\}$. Therefore since the restriction to $V^\vee_-$ of the $\C^*$action on $V^\vee$ coincides with the $\C^*_h$-action on $V^\vee_-$ by homoteties, we have that $\hat{p}=\C^*\cdot q$. We conclude since $\hat{X}\cap V_-^\vee=\hat{Y_-}$, therefore $p\in Y_-$. Since $\codim_{\hat{X}\cap B_{\pm}/\C^*} (Y) \geq 2$, we conclude.
\end{proof}

\begin{corollary}	
The exceptional locus of the birational map $\psi^{-1}$ is $Y_+$.
\end{corollary}
Consider the blow-up $\beta: W\to V^\vee\git \C^*$ along the vertex of the affine cone $V^\vee \git \C^*$, as in Proposition \ref{remark:atiyahblowup}, with exceptional divisor $\P(V_-)\times \P(V_+)$. 

\begin{definition}\label{definition:mainresolution}
	Let $R:=\overline{\beta^{-1}((\hat{X}\git \C^*)\setminus 0)}$ be the strict transform of $\hat{X}\git \C^*$ under $\beta: W\to V^\vee\git \C^*$. 
\end{definition}

We abuse notation by denoting with $b_{\pm}: R\to \hat{X}\cap B_{\pm}/\C^*$ the restriction of the blow-up $b_{\pm}: W\to B_{\pm}/\C^*$.	Notice that $R\simeq \overline{b_{\pm}^{-1}((\hat{X}\cap B_{\pm}/\C^*) \setminus \hat{Y_{\pm}})}$, where again we abuse notation by denoting with $s_{\pm}: \hat{X}\cap B_{\pm} \to \hat{X}\git \C^*$ the restriction of $s_{\pm}: B_{\pm}/\C^*\to V^\vee\git \C^*$.  We obtain a diagram:
\[
\xymatrix{ & R \ar[ld]_{b_-} \ar[rd]^{b_+} \ar[dd]^<<<<<<\beta & \\ \hat X \cap B_-/\C^* \ar@{-->}[rr]^<<<<<<\psi \ar[rd]_{s_-} && \hat X \cap B_+/\C^* \ar[ld]^{s_+} \\ & \hat X \git \C^* & }
\]

\begin{proposition}\label{proposition:mainexceptionaldivisor}
	It holds that $b_{\pm}^{-1}(Y_{\pm})\simeq Y$. 
\end{proposition}

\begin{proof}
	We proceed by steps. First, let us denote by $X/\C^*$ the geometric quotient of $\left(X,L\right)$ under the $\C^*$-action, defined over the set of stable points $X\setminus (Y_-\cup Y_+)$ (see \cite[Proposition 2.9]{WORS3})
	\begin{description}
	\item [Step 1] We want to prove that $Y\simeq X/\C^*$.
	 Thanks to \cite[Remark 4.2]{WORS1}, the contraction $f:\P(\cE)\to X$ is $\C^*$-equivariant, in particular the geometric quotients of $(\P(\cE),\cO_{\P(\cE)}(1))$ and $(X,L)$ with respect to the $\C^*$-action are isomorphic. Since the former is a $\P^1$-bundle on $Y$, and therefore its geometric quotient is isomorphic to $Y$, we conclude.
	 \item [Step 2] We show that the GIT quotient $\hat{X}\git \C^*$ is the affine cone over over $Y$.
	 Let us recall that by $\C^*_h$ we denote the natural $\C^*$-action on the affine space $V^\vee$ given by the homoteties. We claim that 
		\[
		\left(\hat{X}\git \C^* \setminus 0\right)/\C^*_h\simeq Y.
		\]
		To this end, let us note that the two $\C^*$-actions commute over the open subset of the points stable under both the $\C^*$ and the $\C^*_h$ actions.
		Therefore we have that
		\begin{equation}\label{eq:quotients}
		\left(\hat{X}\git \C^* \setminus 0\right)/\C^*_h \simeq \left(\hat X \setminus\left(\hat Y_- \cup \hat Y_+\right)\right)/\left(\C^*_h \times \C^*\right).
		\end{equation}
		Notice that 
		\[
		\frac{\hat X \setminus \left(\hat Y_- \cup \hat Y_+\right)}{\C^*_h}=\frac{\left( \hat X \setminus 0\right) \setminus \left( \left(\hat Y_- \setminus 0\right) \cup \left(\hat Y_+ \setminus 0\right) \right)}{\C^*_h}\simeq X \setminus \left(Y_- \cup Y_+\right)
		\]
		and that
		\[
		\left(X \setminus \left(Y_- \cup Y_+\right)\right)/\C^*=X/\C^* \simeq Y.
		\]
		Then the right-hand side of \eqref{eq:quotients} is isomorphic to $Y$ and we conclude.
	\item [Step 3] We want to prove that $\beta^{-1}(0)=Y$.
	It follows immediately after recalling that we are considering the restriction of the blow-up map to $\hat{X}\git \C^*$, whose base of the cone is precisely $Y$.
	\item [Step 4] We show that  $s_{\pm} \circ b_{\pm}=\beta$ and that $s_{\pm}^{-1}(0)\simeq Y$.
	 The first claim follows directly from Lemma \ref{lemma:atiyahfactorization}. Since $s_{\pm}: B_{\pm}/\C^*\to \hat{X}\git \C^*$ are small contractions whose exceptional locus is $\P(V_{\pm})\cap \hat{X}=Y_{\pm}$ by Proposition \ref{proposition:mainexceptionallocus}, we conclude. \qedhere
	\end{description}
\end{proof}


We can now rephrase Theorem \ref{theorem:naive} as follows:
	
\begin{theorem}\label{theorem:maintheorem}
	With the notation of Set-up \ref{setup:maintheorem}, for any smooth drum $X$ constructed upon $(Y,\cL_-,\cL_+)$ there exist a rooftop flip $\psi: \hat{X}\cap B_-/\C^*\dashrightarrow \hat{X}\cap B_+/\C^*$ modeled by $Y$. 
\end{theorem}
	
\begin{proof}
	We verify each condition of Definition \ref{def:Dflips} is satisfied.
	\begin{enumerate}
	\item Using Proposition \ref{proposition:mainexceptionallocus}, the birational map $\psi: \hat{X}\cap B_-/\C^*\dashrightarrow \hat{X}\cap B_+/\C^*$ is a small modification with exceptional locus $Y_-$. The exceptional loci of $s_{\pm}: \hat{X}\cap B_{\pm}/\C^*\to \hat{X}\git \C^*$ are $Y_{\pm}$ and the restriction $s_{\pm}|_{Y_{\pm}}: Y_{\pm} \to 0$ are precisely $Y_{\pm}$-bundles. 
	\item If we consider the resolution $b_{\pm}: R\to \hat{X}\cap B_{\pm}/\C^*$ we have that $Y=b^{-1}_{\pm}(Y_{\pm})$ is a divisor in $R$, and $Y\to Y_{\pm}$ defines two projective bundle structures by definition of smooth drum.
	\item  In this case $Z_0=0$, and we know that $s^{-1}_{\pm}(0)\simeq Y_{\pm}$. Moreover $(b_{\pm}^{-1}\circ s_{\pm}^{-1})(0)\simeq Y$ by Proposition \ref{proposition:mainexceptionaldivisor}, hence we conclude. \qedhere
	\end{enumerate}
\end{proof}

\begin{corollary}
	The geometric quotients $\hat{X}\cap B_{\pm} /\C^*$ are smooth and in particular the rooftop flip $\psi: \hat{X}\cap B_-/\C^*\dashrightarrow \hat{X}\cap B_+/\C^*$ is a small $\Q$-factorial modification.
\end{corollary}

\begin{proof}
	Since the affine variety $\hat{X}$ has only a singularity at the origin, $\hat{X}\cap B_{\pm}$ is smooth. Moreover, the $\C^*$-action is free on $\hat{X}\cap B_{\pm}$, therefore using \cite[Corollary p.199]{MFK} $\hat{X}\cap B_{\pm}$ is a $\C^*$-principal bundle over $\hat{X}\cap B_{\pm}/\C^*$, hence they are also smooth. By definition $\psi$ is in particular a small $\Q$-factorial modification.
\end{proof}

We conclude by using Theorem \ref{theorem:maintheorem} to show that the smooth drum $Q^{2n}\subset \P^{2n+1}$ induces a rooftop flip modeled by $\P\left(T_{\P^n}\right)$. Notice that for $n=2$ this is the famous \emph{Mukai flop} (see \cite{HZ04}, \cite{WW}):

\begin{corollary}\label{corollary:rooftopflipquadric}
	Let $Q^{2n}\subset \P^{2n+1}$ be the smooth quadric hypersurface, viewed as the smooth drum constructed upon $\left(\P\left(T_{\P^n}\right), p^{*}_-\cO_{\P^n}(1), p^{*}_+\cO_{(\P^n)^\vee}(1)\right)$ (cf. Example \ref{example:quadric}). Then the small modification
	$$\psi: \hat{Q}^{2n}\cap B_-/\C^* \dashrightarrow \hat{Q}^{2n}\cap B_+/\C^*$$
	is a rooftop flip modeled by $\P\left(T_{\P^n}\right)$.
\end{corollary}

% \subsection{Rooftop flips modeled by $\P\left(T_{\P^n}\right)$}\label{ssec:mukai}

% We conclude this section by showing that if $X$ is the even quadric hypersurface $Q^{2n} \subset \P^{2n+1}$, the associated rooftop flip is modeled by $\P\left(T_{\P^n}\right)$. Note that the case $n=2$ is the famous Mukai flop (see \cite{HZ04}, \cite{WW}).

% Let $V$ be a complex vector space of dimension $2n+2$, and consider the following $\C^*$-action on $\P(V)\simeq \P^{2n+1}$:
% \[
% t \cdot \left[x_0:\ldots:x_{2n+1}\right]=\left[tx_0:\ldots:tx_n: x_{n+1}:\ldots:x_{2n+1}\right].
% \]
% Let $Q^{2n} \subset \P^{2n+1}$ be the smooth quadric hypersurface defined by $x_0x_{n+1}+x_1x_{n+2}+\ldots+x_nx_{2n+1}=0$. By construction $Q^{2n}$ is $\C^*$-invariant and its fixed point locus consists only of two connected components: the sink and the source of the $\C^*$-action on the quadric are respectively
% \[
% Y_-=\left\{x_{n+1}=\ldots=x_{2n+1}=0\right\} \simeq \P^n, \qquad Y_+=\left\{x_0=\ldots=x_n=0\right\} \simeq (\P^n)^\vee,
% \]
% where the duality between $Y_-$ and $Y_+$ is provided by the non-degenerate quadratic form defining $Q^{2n}$.
% Since the $\C^*$-action on $Q^{2n}$ has bandwidth $1$, using Theorem \ref{theorem:smoothdrum} we argue that $Q^{2n}$ is a smooth drum.

% \begin{lemma}
	% The drum structure on the smooth quadric hypersurface $Q^{2n} \subset \P^{2n+1}$ in induced by the smooth projective variety $\P\left(T_{\P^n}\right)$, admitting a two projective bundle structure:
	% \[
	% \xymatrix{&\P\left(T_{\P^n}\right) \ar[ld]_{p_-} \ar[rd]^{p_+}& \\ \P^n && (\P^n)^\vee.}
	% \]
	% \end{lemma}

% \begin{proof}
	% Set $y_i:=x_{n+1+i}$ to stress the duality between $\P^n$ and $(\P^n)^\vee$, so that the equation defining $Q^{2n}$ becomes $\sum_{i=0}^n x_iy_i=0$.
	
	% Let us consider the affine cone $\hat Q^{2n} \subset V^\vee$ over the quadric, that is
	% \[
	% \hat Q^{2n} = \Spec \left(\frac{\C\left[x_0,\ldots,x_n,y_0,\ldots,y_n\right]}{\left(x_0y_0+x_1y_1+\ldots+x_ny_n\right)}\right) \subset V^\vee.
	% \]
	% We know by Section \ref{ssec:atiyah} that the GIT quotient $V^\vee \to V^\vee \git \C^*$ is the affine cone over the Segre embedding of $\P^n \times (\P^n)^\vee$.
	% The quotient map is explicitly given by
	% \[
	% \left(x_0,\ldots,x_n,y_0,\ldots,x_n\right) \longmapsto \left(x_0y_0,\ldots, x_iy_j,\ldots,x_ny_n\right) \in \C^{n^2}.
	% \]
	% Let us denote by $a_{ij}:=x_iy_j$, for $0\leq i,j\leq n$, the coordinates on $\C^{n^2}$ obtained using the Segre embedding. Then
	% \[
	% V^\vee \git \C^*=\Spec \left(\frac{\C\left[a_{00},a_{01},\ldots,a_{nn}\right]}{\left(a_{ij}a_{kl}-a_{il}a_{jk}\right)}\right).
	% \] 
	% The image of the restriction of the quotient map $\hat Q^{2n} \to \hat Q^{2n} \git \C^*$ is then
	% \[
	% \hat Q^{2n} \git \C^*=V^\vee \git \C^* \cap \left\{ a_{00}+a_{11}+\ldots+a_{nn}=0\right\},
	% \]
	% which is the affine cone over the hyperplane section of $\P^n \times (\P^n)^\vee$.
	% Then by (2) of the proof of Proposition \ref{proposition:mainexceptionaldivisor}, we have that the unique geometric quotient of $Q^{2n}$ is isomorphic to the smooth hyperplane section of $\P^n \times (\P^n)^\vee$, that is $\P\left(T_{\P^n}\right)$.
	% \end{proof}

% \begin{corollary}
	% There exists a smooth drum structure on the smooth quadric hypersurface $Q^{2n} \subset \P^{2n+1}$ constructed upon the triple 
	% \[
	% \left(\P\left(T_{\P^n}\right), p_-^*\cO_{\P^n}(1), p_+^*\cO_{(\P^n)^\vee}(1)\right).
	% \]
	% \end{corollary}

% As a consequence of Theorem \ref{theorem:maintheorem} we have that:

% \begin{corollary}
	% 	The birational transformation 
	% 	$$\psi: \hat Q^{2n}\cap B_-/\C^*\dashrightarrow \hat Q^{2n}\cap B_+/\C^*$$
	% 	is a rooftop flip modeled by $\P\left(T_{\P^n}\right)$.
	% \end{corollary}

\bibliographystyle{plain}
\bibliography{bibliomin}
\end{document}
