%!TEX root = BF.tex


\section{Preliminaries}\label{sec:preliminaries}

\subsection{Notation}

We work over $\C$. In this paper $X$ will be a smooth projective variety. We denote by $\rho_X:= \dim \NU(X)$ the Picard number of $X$, where $\NU(X)$ is the finite-dimensional real vector space of Cartier divisors modulo numerical equivalence. 
Given $V$ a finite-dimensional vector space, by $\P\left(V\right)$ we mean the space of one-dimensional quotients of $V$.
The standard $\C^*$-action by homoteties is denoted by $\C^*_h$, with coordinate $h$. For sake of notation, by $t$ we will denote the coordinate of a $\C^*$-action which does not act by homoteties.  
By a \emph{contraction} we mean a proper surjective morphism with connected fibers.
A birational contraction $f$ is \emph{small} if $\codim \Exc(f)\geq 2$. 
By a \emph{flip} we mean a $D$-flip as in \cite[Definition 5.1.4]{hacon}. 


\subsection{$\C^*$-actions on smooth projective varieties}\label{ssec:actions}

In this section, we collect some preliminaries about $\C^*$-actions on smooth projective varieties. We refer to \cite[Section 2]{WORS1} for a detailed discussion.

Let $X$ be a smooth projective variety endowed with a $\C^*$-action. By $X^{\C^*}$ we denote the fixed point locus of the action, and by $\cY$ we denote its set of connected components, namely:
$$X^{\C^*}=\bigsqcup_{Y \in \cY} Y.$$
By \cite[Main Theorem]{IVERSEN}, the subvarieties $Y\in \cY$ are smooth and irreducible.

For any $Y\in \cY$, we define the \emph{Bia{\l}ynicki-Birula cells} as
\[
X^{\pm}(Y):=\left\{x\in X:\lim_{t\to 0}t^{\pm 1}\cdot x\in Y\right\}.
\]
The \emph{sink} (resp. \emph{source}) of the $\C^*$-action is the unique component $Y_-\in \cY$ (resp. $Y_+\in\cY$) such that $X^-(Y_-)$ (resp. $X^+(Y_+)$) are dense subsets of $X$.
As noticed in \cite[Remark 2.5]{WORS1}, the uniqueness of the sink and the source follows from the well-known Bia\l ynicki-Birula Theorem (see \cite{BB} for the original exposition).
 
\begin{definition}
	A $\C^*$-action is \emph{equalized} at $Y\in \cY$ if for every point $p\in \left(X^-(Y)\cup X^+(Y)\right)\setminus Y$ the isotropy group of the $\C^*$-action at $p$ is trivial. Moreover, a $\C^*$-action is equalized if it is equalized at every fixed point component $Y\in \cY$. 
\end{definition}

Let $L$ be an ample line bundle on $X$.
Consider the induced $\C^*$-action on $L$; by \cite[Proposition 2.4 and the subsequent Remark]{KKLV}, a linearization of $L$ exists. 
For any $Y\in \cY$, the induced action of $\C^*$ on the fibers of $L|_{Y}$ is by multiplication with a character, which we denote by $\mu_L(Y) \in \Z$. 
Since $L$ is ample, one may show that
\[
\mu_L(Y_-)=\min_{Y\in\cY}\mu_L(Y), \qquad \mu_L(Y_+)=\max_{Y\in\cY}\mu_L(Y).
\]
By a \emph{$\C^*$-action on the smooth polarized pair $(X,L)$} we mean a non-trivial $\C^*$-action on a smooth projective variety $X$ and a linearization of the ample line bundle $L$.

\begin{remark}
	A $\C^*$-action on $(X,L)$ induces a $\C^*$-action on $(X,mL)$ for every $m\geq 0$. 
	In particular, for $m\gg 0$, the line bundle $mL$ is very ample, and we have an embedding $\phi_{|mL|}: X\hookrightarrow \P\left(\HH^0(X,mL)\right)$. 
	Following \cite[Section 2]{BWW}, a linearization of $mL$ provides a $\C^*$-action on $\HH^0(X,mL)$, and so on $\P \left(\HH^0(X,mL)\right)$.
	Moreover, the $\C^*$-action on $X$ can be seen as the restriction of the $\C^*$-action on $\P\left(\HH^0(X,mL)\right)$, that is the embedding is $\C^*$-equivariant.
\end{remark}

\begin{definition}
	Let $X$ be a smooth projective variety and let $L\in \Pic(X)$ be ample. 
The \emph{bandwidth} of the $\C^*$-action on $(X,L)$ is defined as 
\[
|\mu_L|:=\mu_L(Y_+)-\mu_L(Y_-).
\]
\end{definition}

\subsection{Smooth drums}\label{ssec:drums}

In this section we recall the characterization of smooth drums, that is smooth polarized pairs admitting a $\C^*$-action of bandwidth $1$. We refer to \cite[Section 4]{WORS1}. 

\begin{lemma}\label{lemma:gg}
	Let $Y$ be a smooth projective variety such that $\rho_Y=2$, admitting two elementary contractions $p_{\pm}:Y\to Y_{\pm}$, with $Y_{\pm}$ smooth. Let $L_{\pm}$ be very ample line bundle on $Y_{\pm}$, and denote $\cL_{\pm}:=p_{\pm}^*L_{\pm}$. Consider the projective bundle $\pi:\P(\cL_-\oplus \cL_+)\to Y$. Then $\cO_{\P(\cL_-\oplus \cL_+)}(1)$ is globally generated, and there exists a contraction, birational onto the image,
	\[
\phi=\phi_{\cO_{\P(\cL_-\oplus \cL_+)}(1)}:\P\left(\cL_-\oplus \cL_+\right)\longrightarrow \P\left(\HH^0\left(\P(\cL_-\oplus \cL_+), \cO_{\P(\cL_-\oplus \cL_+)}(1)\right)\right).
\]
\end{lemma}  

\begin{proof}
	The global generation of $\cO_{\P(\cL_-\oplus \cL_+)}(1)$ is immediate. Let us just notice that, by the projection formula, we have an isomorphism
	\begin{equation}\label{equation:drumembedding}
		\HH^0\left(\P(\cL_-\oplus \cL_+),\cO_{\P(\cL_-\oplus \cL_+)}(1)\right)= \HH^0(Y_-,L_-)\oplus\HH^0(Y_+,L_+).
	\end{equation}
	Let us prove that the morphism 
	\[
	\phi: \P(\cL_-\oplus \cL_+) \to \P(\HH^0(Y_-,L_-)\oplus \HH^0(Y_+,L_+)),
	\]
	associated to evaluation of sections is a contraction, birational onto the image.
	 Consider the sections $\sigma_{\pm}: Y\to \P\left(\cL_-\oplus \cL_+\right)$ associated to the quotients $\cL_-\oplus \cL_+\to \cL_{\pm}$. The compositions $\phi\circ \sigma_{\pm}$ coincide with the bundle maps $p_{\pm}$, in particular they have connected fibers. 
	 On the other hand the restriction of $\phi$ to $\P\left(\cL_-\oplus \cL_+\right)\setminus \left(\sigma_-(Y_-)\cup \sigma_+(Y)\right)$ is an isomorphism onto the image.	
\end{proof}

% \begin{proof}
	% 	We want to prove that the natural map given by the evaluation of sections 
	% 	\[
	% \HH^0\left(\P(\cL_-\oplus \cL_+),\cO_{\P(\cL_-\oplus \cL_+)}(1)\right)\otimes \cO_{\P(\cL_-\oplus \cL_+)}\longrightarrow \cO_{\P(\cL_-\oplus \cL_+)}(1)\]
	% 	is surjective. 
	% Firstly, notice that using the projection formula we have that
	% 	\begin{equation}\label{equation:drumembedding}
		% 		\HH^0\left(\P(\cL_-\oplus \cL_+),\cO_{\P(\cL_-\oplus \cL_+)}(1)\right)= \HH^0(Y_-,L_-)\oplus\HH^0(Y_+,L_+).
		% 	\end{equation}
	% 	Since a very ample line bundle is globally generated, we have a surjective morphism
	% 	\[
	% \HH^0(Y_{\pm},L_{\pm})\otimes \cO_{Y_{\pm}}\longrightarrow L_{\pm}.
	% \]
	% 	Since $p^*_{\pm}$ is left exact and the direct sum of globally generated line bundles is globally generated, we still have a surjective morphism
	% 	$$\HH^0(Y,\cL_-\oplus\cL_+)\otimes \cO_{Y}\longrightarrow \cL_-\oplus \cL_+.$$
	% 	Again, using the left exactness of $\pi^*$, we have a surjective morphism 
	% \[
	% \HH^0\left(\P(\cL_-\oplus \cL_+), \cO_{\P(\cL_-\oplus \cL_+)}(1)\right)\otimes \cO_{\P(\cL_-\oplus \cL_+)}\longrightarrow \pi^*(\cL_-\oplus \cL_+).
	% \]
	% 	Moreover there is a surjective morphism $\pi^*(\cL_-\oplus \cL_+)\to \cO_{\P(\cL_-\oplus \cL_+)}(1)$ given by the Euler sequence
	% 	\[
	% 0 \to \Omega_{\P(\cL_-\oplus \cL_+)\mid Y}(1)\longrightarrow \pi^*(\cL_-\oplus\cL_+)\longrightarrow \cO_{\P(\cL_-\oplus \cL_+)}(1)\to 0
	% \]
	% 	We can now compose the two surjective maps
	% 	\[
	% \HH^0\left(\P(\cL_-\oplus \cL_+),\cO_{\P(\cL_-\oplus \cL_+)}(1)\right)\oplus \cO_{\P(\cL_-\oplus \cL_+)}\longrightarrow  \pi^*(\cL_-\oplus \cL_+)\longrightarrow \cO_{\P(\cL_-\oplus \cL_+)}(1)
	% \]
	% 	to obtain that $\cO_{\P(\cL_-\oplus \cL_+)}(1)$ is globally generated.
	% \end{proof}

\begin{definition}\label{definition:drum}
	The image $X:=\phi \left(\P(\cL_-\oplus \cL_+)\right)$ is called the \emph{drum constructed upon the triple $(Y,\cL_-,\cL_+)$}.
\end{definition}

\begin{remark}\label{remark:drumembedding}
	Using Equation \ref{equation:drumembedding}, we obtain that there is an embedding 
\[
X \subset \P\left(\HH^0(Y_-,L_-)\oplus \HH^0(Y_-,L_+)\right).
\]
\end{remark}

\begin{remark}\label{remark:drumdimension}
	By construction we have that $\dim X=\dim Y +1.$
\end{remark}

\begin{theorem}\label{theorem:smoothdrum}\cite[Lemma 4.4]{WORS1}
	A drum $X$ is smooth if and only if the following conditions are satisfied:
	\begin{itemize}
		\item $\Nef(Y)=\langle \cL_-,\cL_+ \rangle$;
		\item $p_\pm: Y\to  Y_\pm$ has a projective bundle structure;
		\item $\deg(\cL_{\mp}|_{F_{\pm}})=1$,  where $F_\pm$ denotes a fiber of $p_\pm$.
	\end{itemize}
\end{theorem}

\begin{remark}
Since $\rho_{Y_\pm}=1$, the third condition of the above theorem implies that $L_\pm$ must be the respective generators of the Picard groups of $Y_\pm$.
\end{remark}

Since we will work only in the context of smooth drums, we fix the notation for the rest of the section in the following:

\begin{setup}\label{setup:smoothdrum}
	Let $Y$ be a smooth projective variety such that $\rho_Y=2$, admitting two projective bundle structure $p_{\pm}:Y\to Y_{\pm}$, with $Y_{\pm}$ smooth. Let $L_{\pm}$ be very ample line bundles on $Y_{\pm}$, and denote $\cL_{\pm}:=p_{\pm}^*L_{\pm}$. Suppose that $\deg(\cL_{\mp}|_{F_{\pm}})=1$,  where $F_\pm$ denotes a fiber of $p_\pm$. Consider the projective bundle $\pi:\P(\cL_-\oplus \cL_+)\to Y$ and the morphism $\phi$ given by $\cO_{\P(\cL_-\oplus \cL_+)}(1)$. Then let $X$ be the smooth drum constructed upon $(Y,\cL_-,\cL_+)$.
\end{setup}

\begin{remark}
	Notice that $X$ comes with a natural ample line bundle $L$, which is the restriction of the hyperplane class in $\P(\cL_-\oplus\cL_+)$, such that $\phi^*L=\cO_{\P(\cL_-\oplus\cL_+)}(1)$.
\end{remark}

For sake of clarity, Set-up \ref{setup:smoothdrum} can be summarized by means of the following diagram:
\[
\xymatrix{ & \P\left(\cL_-\oplus \cL_+\right) \ar[r]^-\phi \ar[d]_\pi & X \\  L_- \ar[d]& Y \ar[ld]_{p_-}  \ar[rd]^{p_+} &L_+ \ar[d] \\ Y_- && Y_+ }
\]

\begin{lemma}
Let $\pi: \P\left(\cL_-\oplus \cL_+\right)\to Y$ be a projective bundle as in Set-up \ref{setup:smoothdrum}. Then $\P\left(\cL_-\oplus \cL_+\right)$ admits a $\C^*$-action of bandwidth $1$, whose sink and source are both isomorphic to $Y$.
\end{lemma}

\begin{proof}
Consider the \emph{standard $\C^*$-action} on $\left(\P^1, \cO_{\P^1}(1)\right)$: 
\[
\C^* \times \P^1 \to \P^1, \quad \left(t, \left[x_0:x_1\right]\right) \longmapsto \left[tx_0:x_1\right] .
\]
This action has bandwidth $1$. Let $\sigma_\pm:Y \to \P\left(\cL_-\oplus\cL_+\right)$ be the sections corresponding to the quotients $\cL_-\oplus\cL_+ \to \cL_\pm$.
Then the $\P^1$-bundle structure on $\P\left(\cL_-\oplus\cL_+\right)$ allows us to define a $\C^*$-action given fiberwise by the standard $\C^*$-action on $\P^1$ and such that the sink is $\sigma_-(Y)$ and the source is $\sigma_+(Y)$.
\end{proof}

\begin{remark}\label{remark:sections}\cite[Remark 4.2]{WORS1}
Let $X$ be a smooth drum constructed upon $\left(Y,\cL_-,\cL_+\right)$ as in Set-up \ref{setup:smoothdrum}. We recall that, by Remark \ref{remark:drumembedding}, $\phi$ provides an embedding $X \subset \P\left(\HH^0(Y_-,L_-)\oplus \HH^0(Y_-,L_+)\right)$.
As we note in the previous proof, $\sigma_\pm(Y)$ contain the limit points for $t^{\pm 1} \to 0$. Thus we have that:
\begin{itemize}
\item $(\phi \circ \sigma_\pm)(Y) \simeq Y_\pm$, embedded in $\P\left(\HH^0(Y_\pm,L_\pm)\right)$ via $\cO_{\P(\cL_-\oplus \cL_+)}(1)$;
\item For all $y \in Y$, $\pi^{-1}(y)$ is mapped by $\phi$ into a $\C^*$-orbit.
\end{itemize}
Then $X$ admits a $\C^*$-action of bandwidth $1$, whose sink and source are $Y_-$ and $Y_+$, respectively.
\end{remark}


\begin{example}\label{example:drumPn}
Consider $Y_-=\P^m, Y_+=\P^l$, let $Y=\P^m \times \P^l$, and let $p_\pm$ be the projections onto the two factors of $\P^m \times \P^l$. Denote by $L_-=\cO_{\P^m}(1)$ and $L_+=\cO_{\P^l}(1)$ the very ample line bundles on $\P^m$ and $\P^l$, respectively.
Then $\cL_-=\cO_{\P^m \times \P^l}(1,0)$, $\cL_+=\cO_{\P^m \times \P^l}(0,1)$, and we have a diagram of the form
\[
\xymatrix{ &\P\left(\cO_{\P^m\times\P^l}(1,0) \oplus \cO_{\P^m\times\P^l}(0,1)\right) \ar[r]^-\phi \ar[d]_\pi & X \\  \cO_{\P^m}(1) \ar[d]& \P^m\times\P^l \ar[ld]_{p_-}  \ar[rd]^{p_+} &\cO_{\P^l}(1) \ar[d] \\ \P^m && \P^l }
\]
As shown in Remark \ref{remark:drumembedding}, $X \subset \P\left( \HH^0(\P^m,\cO_{\P^m}(1))\oplus \HH^0(\P^l,\cO_{\P^l}(1))\right) \simeq \P^{m+l+1}$.
By Remark \ref{remark:drumdimension}, $\dim X= \dim \P^m\times \P^l +1$, hence $\phi$ is surjective and $X \simeq \P^{m+l+1}$. 
In particular, $\P^{m+l+1}$ is the drum constructed upon 
\[
\left(\P^m\times \P^l, \cO_{\P^m\times\P^l}(1,0),\cO_{\P^m\times \P^l}(0,1)\right).
\]
On the other hand, consider the $\C^*$-action on $\P^{m+l+1} $ given by
	\begin{equation*}\label{eq:actionP}
	t \cdot \left[x_0:\ldots:x_{m+l+1}\right]=\left[tx_0:\ldots:t x_{l}:x_{l+1}:\ldots:x_{m+l+1}\right].
	\end{equation*}
	The sink and the source are, respectively, 
\[
\P^m=\left\{x_{l+1}=\ldots=x_{m+l+1}=0\right\},\qquad \P^l=\left\{x_0=\ldots=x_l=0\right\}.
\]
Then the $\C^*$-action of bandwidth $1$ on $X$ induced by the drum structure is precisely the action defined above.
\end{example}

\begin{theorem}\label{thm:bw1}\cite[Theorem 4.8]{WORS1}
Let $X$ be a smooth projective variety with $\rho_X=1$ different from the projective space and let $L$ be an ample line bundle on $X$. Then $(X,L)$ admits a $\C^*$-action of bandwidth $1$ if and only if $X$ is a smooth drum.
\end{theorem}



\begin{example}\label{example:quadric}
	Consider the $\C^*$-action on $\P^{2n+1}$ defined as follows:
	\[
	t \cdot \left[x_0:\ldots:x_{2n+1}\right]=\left[tx_0:\ldots:tx_n: x_{n+1}:\ldots:x_{2n+1}\right].
	\]
	Let $Q^{2n} \subset \P^{2n+1}$ be the smooth quadric hypersurface defined by the equation $x_0x_{n+1}+x_1x_{n+2}+\ldots+x_nx_{2n+1}=0$. By construction $Q^{2n}$ is $\C^*$-invariant and its fixed point locus consists only of the sink and the source, which are respectively:
	\[
	Y_-=\left\{x_{n+1}=\ldots=x_{2n+1}=0\right\} \simeq \P^n, \qquad Y_+=\left\{x_0=\ldots=x_n=0\right\} \simeq (\P^n)^\vee,
	\]
	where the duality between $Y_-$ and $Y_+$ is provided by the non-degenerate quadratic form defining $Q^{2n}$.
	Since the $\C^*$-action on $Q^{2n}$ has bandwidth $1$, using Theorem \ref{theorem:smoothdrum} we argue that $Q^{2n}$ is a smooth drum. By \cite[Proposition 1.9]{Pas}, the smooth projective variety of Picard number $2$ inducing this drum is 
	\begin{equation*}
		\xymatrix{&\P\left(T_{\P^n}\right) \ar[ld]_{p_-} \ar[rd]^{p_+} & \\
			\P^n && (\P^n)^\vee}
	\end{equation*}
	where the two projective bundle structures are induced by the well known description $\P\left(T_{\P^n}\right)=\left\{(p,H)\in \P^n\times (\P^n)^\vee\mid p\in H\right\}$. 
	Let us briefly explain this fact; let $X$ be the smooth drum associated to $\P\left(T_{\P^n}\right)$. By Remark \ref{remark:drumembedding}, $X \subset \P^{2n+1}$ and, by Remark \ref{remark:sections}, the sink and the source of the $\C^*$-action on $X$ are respectively $\P^n$, $(\P^n)^\vee$. Write $\P^n=\P(V)$, $(\P^n)^\vee=\P(V^\vee)$, with $V$ a complex vector space of dimension $n+1$. By construction the drum $X$ of $\P(T_{\P^n})$ is an hypersurface in $\P(V\oplus V^\vee)$. A point in $\P(V\oplus V^\vee)$ belongs to $X$ if it is the class of a vector $p+ h\in V\oplus V^\vee$ such that $h(p)=0$. Algebraically this says that $X$ is given by a nondegenerate quadratic equation, hence we conclude.
	%Given two points $p\in \P^n$ and $q \in (\P^n)^\vee$, the line joining $p$ and $q$ in $\P^{2n+1}$ has to be orthogonal to both $\P^n$ and $ (\P^n)^\vee$, hence we obtain a quadratic equation which is precisely the defining equation of $Q^{2n}$.		
	
	%Let us briefly explain why $Q^{2n}$ has a drum structure associated to $\P\left(T_{\P^n}\right)$: let $X$ be the smooth drum associated to $\P\left(T_{\P^n}\right)$.
	%By Remark \ref{remark:drumembedding}, $X \subset \P^{2n+1}$ and, by Remark \ref{remark:sections}, the sink and the source of the $\C^*$-action on $X$ are respectively $\P^n$, $(\P^n)^\vee$.
	 %Given two points $p\in \P^n$ and $q \in (\P^n)^\vee$, the line joining $p$ and $q$ in $\P^{2n+1}$ has to be orthogonal to both $\P^n$ and $ (\P^n)^\vee$, hence we obtain a quadratic equation which is precisely the defining equation of $Q^{2n}$.		
\end{example}




\begin{remark}\label{remark:pasquier}
To our understanding, the only known examples of smooth drums are constructed upon a smooth projective variety $Y$ admitting two projective bundle structures such that $Y_\pm$ are rational homogeneous varieties, i.e. quotient of semisimple linear algebraic groups by parabolic subgroups.
Except for a non-homogeneous sporadic example appearing in \cite{Ott2} (see also \cite[\S 2]{Kan}), in all the known examples $Y$ is also a rational homogeneous variety.
In \cite[Remark 3.3]{ORS} there is a complete classification of rational homogeneous varieties with Picard number $2$ admitting two projective bundle structures.
The corresponding smooth drums $X$ constructed from a rational homogeneous variety $Y$ are well-known in literature: they are one of the horospherical varieties classified by Pasquier, see  \cite[Theorem 0.1]{Pas}.
%Notice that some horospherical varieties are homogeneous; in particular $X$ can be:
%\begin{itemize}
%	\item a smooth quadric hypersurface of even dimension;
%	\item a Grassmannian (including the case of $X$ being the projective space);
%	\item a spinor variety.
%\end{itemize}
%By Theorem \ref{thm:bw1}, they correspond to the unique rational homogeneous varieties admitting a $\C^*$-action of bandwidth $1$, cf. \cite[Theorem 1.1]{Fra1}. 
\end{remark}




