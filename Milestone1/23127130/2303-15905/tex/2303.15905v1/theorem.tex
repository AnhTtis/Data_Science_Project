%!TEX root = BF.tex


\section{Main result}\label{sec:mainresult}
	
This section is devoted to prove Theorem \ref{theorem:naive}; we will adapt the example of Atiyah flip explained in Section \ref{sec:atiyah} in greater generality.

\begin{setup}\label{setup:maintheorem}
	Let $X$ be a smooth drum constructed upon a triple $(Y,\cL_-,\cL_+)$ (cfr. \S \ref{ssec:drums}).
	Let $\hat{X}$ be the affine cone over $X$, contained in the affine space $V^\vee:=V_-^\vee\oplus V_+^\vee$ (cfr. Remark \ref{remark:drumembedding}), where 
	\[
	V_-:=\HH^0(Y_-,L_-), \qquad V_+:=\HH^0(Y_+,L_+).
	\]
	The $\C^*$-action on $V^\vee$ is given by $t\cdot v=\left (t v_-,t^{-1}v_+\right)$, where $v=\left(v_-,v_+\right)\in V^\vee$.
\end{setup}
		
Notice that the $\C^*$-action on $V^\vee$ is induced by a $\C^*$-action on $V$ as described in Set-up \ref{setup:atiyah}. 

Consider the restriction of the $\C^*$-action on $V^\vee$ to $\hat{X}$; there exists a GIT quotient $\hat{X}\to \hat{X}\git\C^*$, singular at the origin. Moreover we can also consider the intersection $\hat{X}\cap B_{\pm}$, which are non-empty open subsets of stable points giving  geometric quotients $\pi_{\pm}:\hat{X}\cap B_{\pm}\to \hat{X}\cap B_{\pm}/\C^*$. 
	
\begin{proposition}\label{proposition:mainexceptionallocus}
	There exist a small modification \[\psi: \hat{X}\cap B_-/\C^*\dashrightarrow \hat{X}\cap B_+/\C^*,\] whose exceptional locus is $Y_-$. 
\end{proposition}

\begin{proof}
	The existence of such a birational map is immediate after noticing that $\hat{X}\cap B_-\cap B_+$ is open and non-empty.
	Let us prove that the exceptional locus of $\psi$ is $Y_-$. It suffices to show that 
	$$(\hat{X}\cap B_-/\C^*) \cap \P(V_-) =Y_-.$$
	Notice that the $\supset$ inclusion is trivial, so let us focus on the $\subset$ inclusion. Let $p\in (\hat{X}\cap B_-/\C^*) \cap \P(V_-)$, and let $\hat{p}=(hv_-,0)$ the corresponding line in $V_-^\vee$ through the origin, with $h\in \C^*_h$. The preimage $\pi^{-1}_-(p)$ is a closed orbit $\C^*\cdot q$ such that $\lim_{t\to \infty} t\cdot q$ does not exist, hence $\C^*\cdot q=\{(tv_-,v_+)\mid t\in \C^*\}$. Since $p$ belongs to the intersection, $\C^*\cdot q=\{(tv_-,0)\mid t\in \C^*\}$. Therefore since the restriction to $V^\vee_-$ of the $\C^*$action on $V^\vee$ coincides with the $\C^*_h$-action on $V^\vee_-$ by homoteties, we have that $\hat{p}=\C^*\cdot q$. We conclude since $\hat{X}\cap V_-^\vee=\hat{Y_-}$, therefore $p\in Y_-$. Since $\codim_{\hat{X}\cap B_{\pm}/\C^*} (Y) \geq 2$, we conclude.
\end{proof}

\begin{corollary}	
The exceptional locus of the birational map $\psi^{-1}$ is $Y_+$.
\end{corollary}
Consider the blow-up $\beta: W\to V^\vee\git \C^*$ along the vertex of the affine cone $V^\vee \git \C^*$, as in Proposition \ref{remark:atiyahblowup}, with exceptional divisor $\P(V_-)\times \P(V_+)$. 

\begin{definition}\label{definition:mainresolution}
	Let $R:=\overline{\beta^{-1}((\hat{X}\git \C^*)\setminus 0)}$ be the strict transform of $\hat{X}\git \C^*$ under $\beta: W\to V^\vee\git \C^*$. 
\end{definition}

We abuse notation by denoting with $b_{\pm}: R\to \hat{X}\cap B_{\pm}/\C^*$ the restriction of the blow-up $b_{\pm}: W\to B_{\pm}/\C^*$.	Notice that $R\simeq \overline{b_{\pm}^{-1}((\hat{X}\cap B_{\pm}/\C^*) \setminus \hat{Y_{\pm}})}$, where again we abuse notation by denoting with $s_{\pm}: \hat{X}\cap B_{\pm} \to \hat{X}\git \C^*$ the restriction of $s_{\pm}: B_{\pm}/\C^*\to V^\vee\git \C^*$.  We obtain a diagram:
\[
\xymatrix{ & R \ar[ld]_{b_-} \ar[rd]^{b_+} \ar[dd]^<<<<<<\beta & \\ \hat X \cap B_-/\C^* \ar@{-->}[rr]^<<<<<<\psi \ar[rd]_{s_-} && \hat X \cap B_+/\C^* \ar[ld]^{s_+} \\ & \hat X \git \C^* & }
\]

\begin{proposition}\label{proposition:mainexceptionaldivisor}
	It holds that $b_{\pm}^{-1}(Y_{\pm})\simeq Y$. 
\end{proposition}

\begin{proof}
	We proceed by steps. First, let us denote by $X/\C^*$ the geometric quotient of $\left(X,L\right)$ under the $\C^*$-action, defined over the set of stable points $X\setminus (Y_-\cup Y_+)$ (see \cite[Proposition 2.9]{WORS3})
	\begin{description}
	\item [Step 1] We want to prove that $Y\simeq X/\C^*$.
	 Thanks to \cite[Remark 4.2]{WORS1}, the contraction $f:\P(\cE)\to X$ is $\C^*$-equivariant, in particular the geometric quotients of $(\P(\cE),\cO_{\P(\cE)}(1))$ and $(X,L)$ with respect to the $\C^*$-action are isomorphic. Since the former is a $\P^1$-bundle on $Y$, and therefore its geometric quotient is isomorphic to $Y$, we conclude.
	 \item [Step 2] We show that the GIT quotient $\hat{X}\git \C^*$ is the affine cone over over $Y$.
	 Let us recall that by $\C^*_h$ we denote the natural $\C^*$-action on the affine space $V^\vee$ given by the homoteties. We claim that 
		\[
		\left(\hat{X}\git \C^* \setminus 0\right)/\C^*_h\simeq Y.
		\]
		To this end, let us note that the two $\C^*$-actions commute over the open subset of the points stable under both the $\C^*$ and the $\C^*_h$ actions.
		Therefore we have that
		\begin{equation}\label{eq:quotients}
		\left(\hat{X}\git \C^* \setminus 0\right)/\C^*_h \simeq \left(\hat X \setminus\left(\hat Y_- \cup \hat Y_+\right)\right)/\left(\C^*_h \times \C^*\right).
		\end{equation}
		Notice that 
		\[
		\frac{\hat X \setminus \left(\hat Y_- \cup \hat Y_+\right)}{\C^*_h}=\frac{\left( \hat X \setminus 0\right) \setminus \left( \left(\hat Y_- \setminus 0\right) \cup \left(\hat Y_+ \setminus 0\right) \right)}{\C^*_h}\simeq X \setminus \left(Y_- \cup Y_+\right)
		\]
		and that
		\[
		\left(X \setminus \left(Y_- \cup Y_+\right)\right)/\C^*=X/\C^* \simeq Y.
		\]
		Then the right-hand side of \eqref{eq:quotients} is isomorphic to $Y$ and we conclude.
	\item [Step 3] We want to prove that $\beta^{-1}(0)=Y$.
	It follows immediately after recalling that we are considering the restriction of the blow-up map to $\hat{X}\git \C^*$, whose base of the cone is precisely $Y$.
	\item [Step 4] We show that  $s_{\pm} \circ b_{\pm}=\beta$ and that $s_{\pm}^{-1}(0)\simeq Y$.
	 The first claim follows directly from Lemma \ref{lemma:atiyahfactorization}. Since $s_{\pm}: B_{\pm}/\C^*\to \hat{X}\git \C^*$ are small contractions whose exceptional locus is $\P(V_{\pm})\cap \hat{X}=Y_{\pm}$ by Proposition \ref{proposition:mainexceptionallocus}, we conclude. \qedhere
	\end{description}
\end{proof}


We can now rephrase Theorem \ref{theorem:naive} as follows:
	
\begin{theorem}\label{theorem:maintheorem}
	With the notation of Set-up \ref{setup:maintheorem}, for any smooth drum $X$ constructed upon $(Y,\cL_-,\cL_+)$ there exist a rooftop flip $\psi: \hat{X}\cap B_-/\C^*\dashrightarrow \hat{X}\cap B_+/\C^*$ modeled by $Y$. 
\end{theorem}
	
\begin{proof}
	We verify each condition of Definition \ref{def:Dflips} is satisfied.
	\begin{enumerate}
	\item Using Proposition \ref{proposition:mainexceptionallocus}, the birational map $\psi: \hat{X}\cap B_-/\C^*\dashrightarrow \hat{X}\cap B_+/\C^*$ is a small modification with exceptional locus $Y_-$. The exceptional loci of $s_{\pm}: \hat{X}\cap B_{\pm}/\C^*\to \hat{X}\git \C^*$ are $Y_{\pm}$ and the restriction $s_{\pm}|_{Y_{\pm}}: Y_{\pm} \to 0$ are precisely $Y_{\pm}$-bundles. 
	\item If we consider the resolution $b_{\pm}: R\to \hat{X}\cap B_{\pm}/\C^*$ we have that $Y=b^{-1}_{\pm}(Y_{\pm})$ is a divisor in $R$, and $Y\to Y_{\pm}$ defines two projective bundle structures by definition of smooth drum.
	\item  In this case $Z_0=0$, and we know that $s^{-1}_{\pm}(0)\simeq Y_{\pm}$. Moreover $(b_{\pm}^{-1}\circ s_{\pm}^{-1})(0)\simeq Y$ by Proposition \ref{proposition:mainexceptionaldivisor}, hence we conclude. \qedhere
	\end{enumerate}
\end{proof}

\begin{corollary}
	The geometric quotients $\hat{X}\cap B_{\pm} /\C^*$ are smooth and in particular the rooftop flip $\psi: \hat{X}\cap B_-/\C^*\dashrightarrow \hat{X}\cap B_+/\C^*$ is a small $\Q$-factorial modification.
\end{corollary}

\begin{proof}
	Since the affine variety $\hat{X}$ has only a singularity at the origin, $\hat{X}\cap B_{\pm}$ is smooth. Moreover, the $\C^*$-action is free on $\hat{X}\cap B_{\pm}$, therefore using \cite[Corollary p.199]{MFK} $\hat{X}\cap B_{\pm}$ is a $\C^*$-principal bundle over $\hat{X}\cap B_{\pm}/\C^*$, hence they are also smooth. By definition $\psi$ is in particular a small $\Q$-factorial modification.
\end{proof}

We conclude by using Theorem \ref{theorem:maintheorem} to show that the smooth drum $Q^{2n}\subset \P^{2n+1}$ induces a rooftop flip modeled by $\P\left(T_{\P^n}\right)$. Notice that for $n=2$ this is the famous \emph{Mukai flop} (see \cite{HZ04}, \cite{WW}):

\begin{corollary}\label{corollary:rooftopflipquadric}
	Let $Q^{2n}\subset \P^{2n+1}$ be the smooth quadric hypersurface, viewed as the smooth drum constructed upon $\left(\P\left(T_{\P^n}\right), p^{*}_-\cO_{\P^n}(1), p^{*}_+\cO_{(\P^n)^\vee}(1)\right)$ (cf. Example \ref{example:quadric}). Then the small modification
	$$\psi: \hat{Q}^{2n}\cap B_-/\C^* \dashrightarrow \hat{Q}^{2n}\cap B_+/\C^*$$
	is a rooftop flip modeled by $\P\left(T_{\P^n}\right)$.
\end{corollary}

% \subsection{Rooftop flips modeled by $\P\left(T_{\P^n}\right)$}\label{ssec:mukai}

% We conclude this section by showing that if $X$ is the even quadric hypersurface $Q^{2n} \subset \P^{2n+1}$, the associated rooftop flip is modeled by $\P\left(T_{\P^n}\right)$. Note that the case $n=2$ is the famous Mukai flop (see \cite{HZ04}, \cite{WW}).

% Let $V$ be a complex vector space of dimension $2n+2$, and consider the following $\C^*$-action on $\P(V)\simeq \P^{2n+1}$:
% \[
% t \cdot \left[x_0:\ldots:x_{2n+1}\right]=\left[tx_0:\ldots:tx_n: x_{n+1}:\ldots:x_{2n+1}\right].
% \]
% Let $Q^{2n} \subset \P^{2n+1}$ be the smooth quadric hypersurface defined by $x_0x_{n+1}+x_1x_{n+2}+\ldots+x_nx_{2n+1}=0$. By construction $Q^{2n}$ is $\C^*$-invariant and its fixed point locus consists only of two connected components: the sink and the source of the $\C^*$-action on the quadric are respectively
% \[
% Y_-=\left\{x_{n+1}=\ldots=x_{2n+1}=0\right\} \simeq \P^n, \qquad Y_+=\left\{x_0=\ldots=x_n=0\right\} \simeq (\P^n)^\vee,
% \]
% where the duality between $Y_-$ and $Y_+$ is provided by the non-degenerate quadratic form defining $Q^{2n}$.
% Since the $\C^*$-action on $Q^{2n}$ has bandwidth $1$, using Theorem \ref{theorem:smoothdrum} we argue that $Q^{2n}$ is a smooth drum.

% \begin{lemma}
	% The drum structure on the smooth quadric hypersurface $Q^{2n} \subset \P^{2n+1}$ in induced by the smooth projective variety $\P\left(T_{\P^n}\right)$, admitting a two projective bundle structure:
	% \[
	% \xymatrix{&\P\left(T_{\P^n}\right) \ar[ld]_{p_-} \ar[rd]^{p_+}& \\ \P^n && (\P^n)^\vee.}
	% \]
	% \end{lemma}

% \begin{proof}
	% Set $y_i:=x_{n+1+i}$ to stress the duality between $\P^n$ and $(\P^n)^\vee$, so that the equation defining $Q^{2n}$ becomes $\sum_{i=0}^n x_iy_i=0$.
	
	% Let us consider the affine cone $\hat Q^{2n} \subset V^\vee$ over the quadric, that is
	% \[
	% \hat Q^{2n} = \Spec \left(\frac{\C\left[x_0,\ldots,x_n,y_0,\ldots,y_n\right]}{\left(x_0y_0+x_1y_1+\ldots+x_ny_n\right)}\right) \subset V^\vee.
	% \]
	% We know by Section \ref{ssec:atiyah} that the GIT quotient $V^\vee \to V^\vee \git \C^*$ is the affine cone over the Segre embedding of $\P^n \times (\P^n)^\vee$.
	% The quotient map is explicitly given by
	% \[
	% \left(x_0,\ldots,x_n,y_0,\ldots,x_n\right) \longmapsto \left(x_0y_0,\ldots, x_iy_j,\ldots,x_ny_n\right) \in \C^{n^2}.
	% \]
	% Let us denote by $a_{ij}:=x_iy_j$, for $0\leq i,j\leq n$, the coordinates on $\C^{n^2}$ obtained using the Segre embedding. Then
	% \[
	% V^\vee \git \C^*=\Spec \left(\frac{\C\left[a_{00},a_{01},\ldots,a_{nn}\right]}{\left(a_{ij}a_{kl}-a_{il}a_{jk}\right)}\right).
	% \] 
	% The image of the restriction of the quotient map $\hat Q^{2n} \to \hat Q^{2n} \git \C^*$ is then
	% \[
	% \hat Q^{2n} \git \C^*=V^\vee \git \C^* \cap \left\{ a_{00}+a_{11}+\ldots+a_{nn}=0\right\},
	% \]
	% which is the affine cone over the hyperplane section of $\P^n \times (\P^n)^\vee$.
	% Then by (2) of the proof of Proposition \ref{proposition:mainexceptionaldivisor}, we have that the unique geometric quotient of $Q^{2n}$ is isomorphic to the smooth hyperplane section of $\P^n \times (\P^n)^\vee$, that is $\P\left(T_{\P^n}\right)$.
	% \end{proof}

% \begin{corollary}
	% There exists a smooth drum structure on the smooth quadric hypersurface $Q^{2n} \subset \P^{2n+1}$ constructed upon the triple 
	% \[
	% \left(\P\left(T_{\P^n}\right), p_-^*\cO_{\P^n}(1), p_+^*\cO_{(\P^n)^\vee}(1)\right).
	% \]
	% \end{corollary}

% As a consequence of Theorem \ref{theorem:maintheorem} we have that:

% \begin{corollary}
	% 	The birational transformation 
	% 	$$\psi: \hat Q^{2n}\cap B_-/\C^*\dashrightarrow \hat Q^{2n}\cap B_+/\C^*$$
	% 	is a rooftop flip modeled by $\P\left(T_{\P^n}\right)$.
	% \end{corollary}