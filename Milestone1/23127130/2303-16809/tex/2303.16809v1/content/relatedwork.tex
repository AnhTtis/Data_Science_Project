To the best of our knowledge, \srep{} is a unique distributed
algorithm that explicitly tackles the problem of network-wide
synchronization of unconfirmed transactions --- \emph{transaction
  pools}~\cite{blockchain_consensus_survey}. To achieve its goals,
\srep{} relies on \emph{communication-efficient} solutions to the
\emph{set reconciliation} problem~\cite{minsky2002practical}, which is
defined as follows. Given two remote parties with their corresponding
data sets $S_A$ and $S_B$, each party needs to discover the elements
local to the other. Communication-efficient solutions to this problem
exchange only messages of size proportional to the number of
\emph{mutual differences} defined as
$(S_A \setminus S_B) \cup (S_B \setminus S_A)$ and often denoted as
$S_A \oplus S_B$.

In fact, there has been several communication-efficient set
reconciliation algorithms proposed in the literature including
Characteristic Polynomial Interpolation~\cite{minsky2003set} (CPI),
BCH codes~\cite{dodis2004fuzzy}, and Invertible Bloom Lookup Tables
(IBLT)~\cite{goodrich2011invertible,eppstein2011s,iblt_new}. For
instance, CPI incurs a communication cost \emph{equal} to the number
of mutual differences plus a small constant, which makes it nearly
\emph{optimal} in communication~\cite{minsky2003set}. On the other
hand, IBLT-based solutions typically offer better \emph{computational}
complexity at the cost of increasing their communication cost by a
constant factor. To further reduce this communication overhead, Lázaro
and Matuz~\cite{iblt_new} have recently proposed an IBLT-based
solution that brings the communication cost closer to that of CPI
while keeping the computational complexity low.

% Novak: we already mention Graphene in Introduction.

% Similarly, some methods use set reconciliation for block relay. Ozisik
% ~\ea{}~\cite{ozisik2019graphene} provide an efficient solution to the
% set reconciliation problem in the P2P network, by combining a small
% Bloom filter and a very small IBLT to substitute the explicit list of
% transaction IDs, which is considerably large. Graphene is more
% efficient than the state of the art Compact Blocks
% \cite{compact_block} and Xtreme Thin Blocks \cite{xthin} unless the
% number of unconfirmed transactions held by peers are extremely large
% (1.2M) and reduces traffic overhead by 60\%.

On the other hand, when it comes to our analytical model and
simulations, we make use of the findings from the blockchain
topology-discovering literature. In particular,
Wang~\ea{}~\cite{ethna} and Gao~\ea{}~\cite{mes_ether_topo}
independently verified that the Ethereum network exhibits
``small-world'' property. Recently,
Shahsavari~\ea{}~\cite{theory_model} used a random graph model to
simulate Bitcoin network and Ma~\ea{}~\cite{cblocksim} proposed a
topology generation based on Watts-Strogatz~\cite{Watts1998} random
graph model to capture the Bitcoin network in their \emph{CBlockSim}
simulator.

%For block relay, there has been many protocols introduced over the years. The classical Bitcoin P2P (peer-to-peer) protocol has been proven inefficient in terms of bandwidth since when relaying a block, it included all the transactions in it, which are already present in other blocks. This caused many problems including buffer bloat, unstable connections, and large delays in relay of blocks. As an effort to solve these problems,  BIP-152 Compact Block Relay [1] was introduced. In Compact Block Relay, the two modes high-bandwidth and low-bandwidth are selected depending on the peers and bandwidth available. The compact block only consists of the IDs of transactions and the block header, therefore achieving transmission of 1 MB block using only 15KB and using only 10\% of the bandwidth other methods require. The biggest contribution of this method is that in the worst-case scenario, the block has to be sent only once, requiring 2.5 RTT.

%Erlay propose to improve the bandwidth efficiency of relaying unconfirmed transactions between Bitcoin full nodes. Erlay is a method that reduces the number of advertised transaction ids. First, instead of receiving the new transaction ids from every one of its peers, Erlay limits this to only the necessary ones. Second, after this transaction id is received and added to the mempool, the block avoids sending the id of transaction that it already is aware of by using set reconciliation based on libminisketch with rest of its peers. Erlay reduces the bandwidth consumption by 40\% and improves the security of the Bitcoin network by keeping the bandwidth (almost) constant as the connectivity increases.

%\subsection{Synchronizing Pairs of Nodes}
  %  As a different approach, some methods use set reconciliation for block relay. Ozisik et. al [3] provide an efficient solution to the set reconciliation problem in the P2P network, by combining Bloom filters with IBLTs, called Graphene. Graphene is proposed to solve the inefficiency of blockchain systems in disseminating block data. Graphene uses a small Bloom filter and a very small IBLT to substitute the explicit list of transaction IDs, which is considerably large. Graphene is more efficient than Compact Blocks and Xtreme Thin Blocks unless the number of unconfirmed transactions held by peers are extremely large (~1.2M) and reduces traffic overhead by 60\%.  Also, there exist some protocols that have runtime proportional to the number of differences O(d). For instance, Lázaro and Matuz [4] focus on IBLT-based, context-free set reconciliation problem which does not require an upper bound to the set difference. Estimating the cardinality of the set difference requires an additional communication in most cases, therefore removing this requirement saves this substantial overhead. The proposed scheme, relying on multi-edge-type (MET) IBLTs, achieves performance close to CPI with lower complexity. Unlike other methods, this approach only requires the transmission of the necessary cells and does not require oversizing the IBLT to correct errors. However, this method does not yet support multi-party set reconciliation, which makes it inefficient for topologies like the Bitcoin network.

%\subsection{Existing Set Reconciliation Algorithms}
%    There exist various set reconciliation methods developed for content synchronization in distributed topologies. Several of these methods were compared and analyzed in our (?) previous work, GenSync [ref], which is a unified set reconciliation framework.

%Throughout this work we use \emph{primal} sync to denote any of the protocols described in this section. We say that a primal sync is \emph{efficient} when its communication cost is proportional to the number of differences (\eg CPI and IBLT are efficient).

%%% Local Variables:
%%% mode: latex
%%% TeX-master: "../main.tex"
%%% End:
