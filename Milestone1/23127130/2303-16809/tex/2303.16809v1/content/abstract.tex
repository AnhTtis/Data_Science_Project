% Block propagation (relaying) is the process of disseminating blocks of
% transactions over the blockchain's underlying peer-to-peer
% network. Due to the bandwidth inefficiency of legacy protocols,
% several proposals have emerged to improve it using the techniques of
% set reconciliation. However, it has recently been shown through
% measurement studies that the performance of these protocols
% deteriorates in the conditions of churn (intermittent network
% connectivity). To address this deficiency, we propose a generic block
% propagation-enhancing protocol called \srep{} that increases
% blockchain's resilience to churn by continuously maintaining the
% similarity between the transaction pools across the network...x
Synchronization of transaction pools (\emph{mempools})  has shown
potential for improving the performance and block propagation delay
of state-of-the-art blockchains.
%Such synchronization typically requires neighboring nodes in a blockchain network
%to periodically exchange their unconfirmed transactions.
Indeed, various heuristics have been proposed in the literature to this end, all of
which incorporate exchanges of unconfirmed transactions into their block
propagation protocol. In this work, we take a different approach, maintaining
transaction synchronization outside (and independently) of the block propagation
channel.  In the process, we formalize the synchronization problem within a graph
theoretic framework and
introduce a novel algorithm (\srep{} - \emph{Set Reconciliation-Enhanced Propagation})
with quantifiable guarantees. We analyze the algorithm's performance for various realistic network topologies,
and show that it converges on any connected graph in a number of steps that is bounded by the
diameter of the graph. We confirm our analytical findings through extensive simulations that include
comparison with \mempoolsync{}, a recent approach from the literature. Our simulations show that
\srep{} incurs reasonable overall bandwidth overhead and, unlike \mempoolsync{}, scales gracefully with the
size of the network.
%However,
%transaction pool synchronization approaches from the literature may
%incur a prohibitive communication overhead. In this work, we introduce
%a communication-efficient out-of-band transaction pool synchronization
%algorithm based on set reconciliation called \srep{}. We analyze the
%performance of our algorithm as a function of the network topology and
%the degree of similarity among transaction pools. \srep{} achieves
%full network synchronization in time bounded by the network diameter
%while incurring only tens of gigabytes of overall bandwidth
%overhead for realistic-size networks. We confirm our analytical
%findings through simulations and compare \srep{}'s bandwidth
%overhead with another approach from the literature.

% TODO: SREP is efficient in communication (uses Cuckoo when Graphene
% would fall back to normal block, for the nodes with no differences,
% communication is negligible as it's linear in $d$)

% TODO: Erlay ('believe that churn is negligible in practice', thanks
% to our network model, we can evaluate SREP's performance in such
% conditions and show...)

% First abstract draft:
% In blockchains, block propagation is the process of disseminating
% newly created blocks over the underlying peer-to-peer network. As
% such, block propagation is responsible for synchronizing the global
% state of the distributed ledger and affects the scalability of the
% blockchain system. In recent years, there have been several
% improvements to legacy block propagation protocols, and some of the
% most successful ones utilize set reconciliation to achieve a superior
% performance. In this work, we first propose a generic block
% propagation model that encompasses the system aspects that are
% critical to the performance of set reconciliation. In this model, we
% quantify the speedup that block propagation protocols may achieve via
% set reconciliation. Finally, we evaluate the relative performance of
% several existing set reconciliation algorithms when utilized for block
% propagation. Our results show that ...

% The measurements papers:
% - Discovering Bitcoin’s Public Topology and Influential Nodes
% (average/mean degree of Bitcoin is 8)
% - CBlockSim

% Papers to consider citing:
% - Behind Block Explorers: Public Blockchain Measurement and Security
% Implication (ICDCS 2021)
% - Root Cause Analyses for the Deteriorating Bitcoin Network
% Synchronization (ICDCS 2021)... It has been shown that block
% propagation can cause significant deterioration in network
% synchronization. As the network becomes less synchronized, it
% not only looses its performance, but also becomes more prone
% to attacks.
% - [Partially related:] Occam: A Secure and Adaptive Scaling Scheme
% for Permissionless Blockchain (a difficulty and mining power
% distribution adjustment scheme to achieve blockchain's adaptation to dynamic
% transaction throughput needs) (ICDCS 2021)
% - [Different smart contracts may require different 'level of
% synchronization', which may not have to be perfect:] On the
% Synchronization Power of Token Smart Contracts (ICDCS 2021)

% The papers that tackled block propagation delay theory:
% - A Theoretical Model for Block Propagation Analysis in Bitcoin
% Network

%%% Local Variables:
%%% mode: latex
%%% TeX-master: "../main"
%%% End:
