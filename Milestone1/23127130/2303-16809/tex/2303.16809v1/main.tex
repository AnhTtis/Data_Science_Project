\pdfoutput=1
\documentclass[10pt,conference]{IEEEtran}
\IEEEoverridecommandlockouts
% The preceding line is only needed to identify funding in the first footnote. If that is unneeded, please comment it out.
\usepackage{cite}
\usepackage{amsmath,amssymb,amsfonts}
% \usepackage{algorithmic}
\usepackage{graphicx}
% clashes with stix
% \usepackage{textcomp}
\usepackage{xcolor}
\def\BibTeX{{\rm B\kern-.05em{\sc i\kern-.025em b}\kern-.08em
    T\kern-.1667em\lower.7ex\hbox{E}\kern-.125emX}}

% Novak's code highlighting
\def\HiLi{\leavevmode\rlap{\hbox to \hsize{\color{yellow!50}\leaders\hrule height .8\baselineskip depth .5ex\hfill}}}

% Novak's packages
\usepackage{pifont}
\usepackage{mathtools}
\usepackage[inline]{enumitem}
\usepackage{caption}
\captionsetup[table]{skip=5pt}
\usepackage{booktabs}
\usepackage{array}
\usepackage{multirow}
\usepackage{xspace}
\usepackage{letltxmacro}
% \usepackage{algorithm}
\usepackage[linesnumbered,ruled,vlined,procnumbered]{algorithm2e}
\usepackage{etoolbox}
\newcounter{procedure}
\makeatletter
\AtBeginEnvironment{procedure}{\let\c@algocf\c@procedure}
\makeatother

% Novak's functions
\newcommand{\subscript}[2]{$#1 _ #2$}

% Novak's math operators
\DeclareMathOperator{\E}{\mathbb{E}}

\definecolor{purple}{rgb}{1, 0, 1}

\newcommand{\ie}{\emph{i.e.,}\xspace}
\newcommand{\eg}{\emph{e.g.,}\xspace}
\newcommand{\abr}{\emph{abbr.}\xspace}
\newcommand{\ea}{\emph{et al.}\xspace}
\newcommand{\gensync}{\emph{GenSync}\xspace}
\newcommand{\colosseum}{\emph{Colosseum}\xspace}
\newcommand{\srep}{\emph{SREP}\xspace} % Set Reconciliation Enhances
\newcommand{\srepsim}{\emph{SREPSim}\xspace}
% Propagation
\newcommand{\esrep}{\emph{E-SREP}\xspace}
\newcommand{\epsrep}{\emph{EP-SREP}\xspace}
\newcommand{\mesrep}{\emph{ME-SREP}\xspace}
\newcommand{\mempoolsync}{\emph{MempoolSync}}

\newcommand{\fref}[1]{Fig.~\ref{#1}}
\newcommand{\tref}[1]{Table~\ref{#1}}
\newcommand{\aref}[1]{Algorithm~\ref{#1}}
\newcommand{\procref}[1]{Procedure~\ref{#1}}
\newcommand{\sref}[1]{Section~\ref{#1}}
\newcommand{\lineref}[1]{line~\ref{#1}}
\newcommand{\appref}[1]{Appendix~\ref{#1}}

% Change \eqref
\LetLtxMacro{\originaleqref}{\eqref}
\renewcommand{\eqref}{Eq.~\originaleqref}

% Theorems and corollaries
\newcounter{theoremcount}
\setcounter{theoremcount}{0}
\DeclareRobustCommand{\theorem}[1]{%
  \refstepcounter{theoremcount}%
  \noindent\textit{\textbf{Theorem \thetheoremcount\label{theorem:#1}: }}%
}
\DeclareRobustCommand{\theoremref}[1]{Theorem~\ref{theorem:#1}}

\DeclareRobustCommand{\proof}{\emph{Proof:}\xspace}
\DeclareRobustCommand{\qqed}{\hfill$\blacksquare$}

\newcounter{corollcount}
\setcounter{corollcount}{0}
\DeclareRobustCommand{\coroll}[1]{%
  \refstepcounter{corollcount}%
  \noindent\textit{\textbf{Corollary \thecorollcount\label{coroll:#1}: }}%
}
\DeclareRobustCommand{\corollref}[1]{Corollary~\ref{coroll:#1}}

\newcounter{lemmacount}
\setcounter{lemmacount}{0}
\DeclareRobustCommand{\lemma}[1]{%
  \refstepcounter{lemmacount}%
  \noindent\textit{\textbf{Lemma \thelemmacount\label{lemma:#1}: }}%
}
\DeclareRobustCommand{\lemmaref}[1]{Lemma~\ref{lemma:#1}}

\newcounter{definitioncount}
\setcounter{definitioncount}{0}
\DeclareRobustCommand{\definition}[1]{%
  \refstepcounter{definitioncount}%
  \noindent\textit{\textbf{Definition \thedefinitioncount\label{definition:#1}: }}%
}
\DeclareRobustCommand{\defref}[1]{Definition~\ref{definition:#1}}

%notes of different authors
\newif\ifnotes
\notestrue
\notesfalse

\newif\ifdiff
\difftrue
\difffalse

\newcommand{\anote}[1]{\ifnotes $\ll$\textsf{\textcolor{purple}{Ari: {#1}}}$\gg$ \fi}
\newcommand{\nnote}[1]{\ifnotes $\ll$\textsf{\textcolor{orange}{Novak: {#1}}}$\gg$ \fi}
\newcommand{\diff}[1]{\ifdiff\textcolor{orange}{#1}\else#1\fi}

%%% Local Variables:
%%% mode: latex
%%% TeX-master: "main"
%%% End:


% Novak: to enable IEEEpubidadjcol
\IEEEoverridecommandlockouts

\begin{document}

% Comparison of Set Reconciliation Protocols for Block Propagation in
% Blockchains

% The 'semicolon' title is a ICDCS folklore (not ICBC)
% Old titles:
% - SREP: Improving Synchronization Strategies of Permissionless
% Blockchains
% - SREP: Enhanced Block Propagation via Continuous Set Reconciliation
\title{\srep{}: Out-Of-Band Sync of Transaction Pools for
  Large-Scale Blockchains}

\author{\IEEEauthorblockN{Novak Boškov, Sevval Simsek, Ari
    Trachtenberg, and David Starobinski}
  \IEEEauthorblockA{\textit{Department of Electrical and Computer
      Engineering}\\
    \textit{Boston University, Boston, Massachusetts, USA}\\
    \{boskov,sevvals,trachten,staro\}@bu.edu}
}

% \author{\IEEEauthorblockN{1\textsuperscript{st} Given Name Surname}
% \IEEEauthorblockA{\textit{dept. name of organization (of Aff.)} \\
% \textit{name of organization (of Aff.)}\\
% City, Country \\
% email address or ORCID}
% \and
% \IEEEauthorblockN{2\textsuperscript{nd} Given Name Surname}
% \IEEEauthorblockA{\textit{dept. name of organization (of Aff.)} \\
% \textit{name of organization (of Aff.)}\\
% City, Country \\
% email address or ORCID}
% \and
% \IEEEauthorblockN{3\textsuperscript{rd} Given Name Surname}
% \IEEEauthorblockA{\textit{dept. name of organization (of Aff.)} \\
% \textit{name of organization (of Aff.)}\\
% City, Country \\
% email address or ORCID}
% \and
% \IEEEauthorblockN{4\textsuperscript{th} Given Name Surname}
% \IEEEauthorblockA{\textit{dept. name of organization (of Aff.)} \\
% \textit{name of organization (of Aff.)}\\
% City, Country \\
% email address or ORCID}
% \and
% \IEEEauthorblockN{5\textsuperscript{th} Given Name Surname}
% \IEEEauthorblockA{\textit{dept. name of organization (of Aff.)} \\
% \textit{name of organization (of Aff.)}\\
% City, Country \\
% email address or ORCID}
% \and
% \IEEEauthorblockN{6\textsuperscript{th} Given Name Surname}
% \IEEEauthorblockA{\textit{dept. name of organization (of Aff.)} \\
% \textit{name of organization (of Aff.)}\\
% City, Country \\
% email address or ORCID}
% }

\IEEEoverridecommandlockouts

\IEEEpubid{\makebox[\columnwidth]{979-8-3503-1019-1/23/\$31.00~\copyright2023 IEEE \hfill} \hspace{\columnsep}\makebox[\columnwidth]{ }}

\maketitle

\IEEEpubidadjcol

\begin{abstract}
  

Over the past few years, there has been a significant amount of research focused on studying the ReLU activation function, with the aim of achieving neural network convergence through over-parametrization. However, recent developments in the field of Large Language Models (LLMs) have sparked interest in the use of exponential activation functions, specifically in the attention mechanism.

Mathematically, we define the neural function $F: \R^{d \times m} \times  \mathbb{R}^d \rightarrow \mathbb{R}$ using an exponential activation function. Given a set of data points with labels $\{(x_1, y_1), (x_2, y_2), \dots, (x_n, y_n)\} \subset \mathbb{R}^d \times \mathbb{R}$ where $n$ denotes the number of the data. Here $F(W(t),x)$ can be expressed as $F(W(t),x) := \sum_{r=1}^m a_r \exp(\langle w_r, x \rangle)$, where $m$ represents the number of neurons, and $w_r(t)$ are weights at time $t$. It's standard in literature that $a_r$ are the fixed weights and it's never changed during the training. We initialize the weights $W(0) \in \mathbb{R}^{d \times m}$ with random Gaussian distributions, such that $w_r(0) \sim \mathcal{N}(0, I_d)$ and initialize $a_r$ from random sign distribution for each $r \in [m]$.

Using the gradient descent algorithm, we can find a weight $W(T)$ such that $\| F(W(T), X) - y \|_2 \leq \epsilon$ holds with probability $1-\delta$, where $\epsilon \in (0,0.1)$ and $m = \Omega(n^{2+o(1)}\log(n/\delta))$. To optimize the over-parametrization bound $m$, we employ several tight analysis techniques from previous studies [Song and Yang arXiv 2019, Munteanu, Omlor, Song and Woodruff ICML 2022]. 

 

\end{abstract}

\begin{IEEEkeywords}
  Blockchains, Overlay networks, Peer-to-peer computing
\end{IEEEkeywords}

% Recent work with similar idea:
% - Speeding up block propagation in Bitcoin network: Uncoded and coded
% designs (2022, Computer Networks journal): use rateless codding to
% improve compact block. Instead of waiting to receive the entire block,
% forward the pieces of the block while receiving the rest. Table 1
% says that Graphene 'introduce large propagation delay when relaying
% block of a large size'. That is not true for set reconciliation in
% general, as the duration of the sync is proportional to the number
% of differences (for IBLT and CPI, at least).

\section{Introduction and Related Work}
Block propagation represents a fundamental aspect of many blockchain
networks in which blockchain nodes forward newly created blocks to
their neighbors. Historically, block propagation has been performed by
sending all the transactions belonging to the block alongside the
block's metadata. Often, a substantial number of the block's
transactions are present on the receiving end, resulting in
unnecessarily high \emph{bandwidth overhead}. To cope with such
overhead, more advanced block propagation protocols such as
\emph{CompactBlock}~\cite{compact_block}, \emph{Xtreme Thin
  Blocks}~\cite{xthin}, \emph{Graphene}~\cite{ozisik2019graphene}, and
\emph{Gauze}~\cite{ding2022} have been introduced. % These protocols
% build on efficient data structures approaches, such as hashes and
% Bloom filters, to compress data.

Yet, it has recently been demonstrated through \emph{in-situ}
measurements in live blockchains, including Bitcoin, that the
performance of these advanced block propagation protocols can
significantly degrade when transaction pools go out of
sync~\cite{anas_empir,anas_churn_tnsm,anas_churn_icbc,churn_misic}.
One approach to prevent such performance degradation is to have
neighboring nodes regularly synchronize their pools of unconfirmed
% (unspent)
transactions. Toward this end, the recent work
in~\cite{anas_churn_tnsm} proposes a heuristic, called \mempoolsync{},
that is shown to reduce the average block propagation delay by~50\% in
the Bitcoin network. Yet, \mempoolsync{} does not provide any
quantifiable \emph{guarantees} on overall communication or delay
performance.

In this work, we study the problem of transaction pool synchronization
(\emph{sync}) from a fundamental, graph-theoretic perspective, which
allows us to analyze synchronization performance metrics in various
network topologies. Our main contributions are as follows:
\begin{itemize}
\item We introduce a novel transaction pool sync algorithm, called
  \srep{}, which functions in an assistive capacity
  %, which abstracts transaction pools as sets of globally unique
  %elements (transactions) and works towards achieving a consensus on
  %the set of (unconfirmed) transactions across all nodes in the network. \srep{} is
  outside of the existing block propagation protocols.
\item We analyze the performance of \srep{} in general network
  topologies, including a more specialized model that captures
  topological properties of actual blockchains (\eg{} the
  ``small-world'' property) as well as the statistics of transaction
  pools.
  %Under this model, \srep{} syncs the entire network in time
  %bounded by the diameter of the network while incurring only tens of
  %gigabytes of overall communication cost in Ethereum and Bitcoin-size
  %networks.
\item We develop a simulation approach based on realistic transaction
  pool data from measurement campaigns, and confirm our analytical
  findings through simulations.
\item We show that \srep{} has significantly lower bandwidth
  overhead than \mempoolsync{}.
\end{itemize}

The rest of this paper is organized as follows. In \sref{sec:rw}, we
overview the related work. In \sref{sec:srep_algo}, we introduce
\srep{}. In \sref{sec:analysis}, we analyze the properties of \srep{}
and validate our findings through simulations in \sref{sec:sim}. We
compare \srep{} with a transaction pool synchronization approach from
the literature in \sref{sec:mempoolsync}. Finally, we give a
conclusion and propose future work in \sref{sec:conclusion}.

% Old contributions
% - We propose a novel blockchain synchronization performance model
% based on set reconciliation,
% - Within our model, we capture the system performance of the
% underlying network and quantify the effects of the system parameters
% to the performance of the blockchain overlay.
% - We construct a generic set reconciliation-based block
% propagation protocol, and characterize the trade-offs of existing
% set reconciliation protocols in this new context,
% - Finally, we demonstrate the superiority of our set
% reconciliation-based block propagation over traditional techniques
% through simulation.

\section{Background}\label{sec:rw}

\section{Related Work}
\label{sec:relatedwork}

%%%%%%%%%%%%%%%%%%%%%%%%%% Outline %%%%%%%%%%%%%%%%%%%%%%%%%%%%%%%%%%%%%
%(1) Evasion Attacks
%(1.1) Surveys on evasion attacks and their relation to data properties - Michael
%(1.2) Individual papers that study non-data related reasons behind evasion attacks - Michael
%(1.3) Techniques related to evasion attacks and defenses (new) - Gabby
%(2) Non-Evasion Attacks (new), and - ???
%(3) Effects of training data on standard generalization - done 
%
%
%
%(1) Evasion Attacks
%(1.1) A number of surveys review literature on evasion attacks. - Michael
%Most of them do not focus specifically on properties of data but also discuss attack and defense mechanisms, non-data-related reasons for adversarial vulnarability, and  more. ~\jr{cite 4}.
%Yet, they these surveys mention data and its relation to evasion attacks. Specifically \jr{what they say about data.}
%The most close to ours is concurrent work by XXX + concrete facts that we have and they don't.
%
%(1.2) individual papers that study non-data related reasons behind evasion attacks, - Michael
%Literature identifies multiple reasons for adversarial vulnerability, in particular, for evasion attacks. 
%These include data-related properties extensively discussed in this survey, as well as reasons related to the models 		   themselves, computations resources, and feature representations. We discuss these below. 
%
%\jr{the rest is from the paper (non-data related reasons for adversarial vulnerability), with sections potentially renamed.}
%
%{\bf Model.}
%
%{\bf Computational Resources.}
%
%{\bf Robustness of Features.}
%
%(1.3) Techniques Related to Evasion Attacks and Defenses (new) - Gabby
%A number of works focus on techniques for generating evasion attacks, countermeasures against these attacks, 
%and defining the notion of the attack itself.   
%
%{\bf Attacks and Defense.}
%Here are the 5 remaining surveys + 1 additional paper for the reviewer.
%
%{\bf Adversarial Examples.}
%2 surveys lines 13 and 14 + 1 additional paper for the reviewer.
%
%(2) Non-Evasion Attacks (new) 
%Need to say that there are other type of attacks, define them, cite surveys (Bo's survey, maybe something else). 
%Only one work explicitly focus on effects of data. 
%
%
%(3) Effects of training data on standard generalization (done)

%%%%%%%%%%%%%%%%%%%%%%%%% Outline %%%%%%%%%%%%%%%%%%%%%%%%%%%%%%%%%%%%%


\revreplace{
We divide related work into three categories:
(1) surveys on adversarial robustness and its relation to data properties,
(2) surveys that discuss the influence of data properties on standard generalization, and
(3) individual papers that study non-data-related reasons for adversarial vulnerability.\\
}
{
This survey investigates properties of training data in the context of model robustness under evasion attacks. 
We start the discussion of related work by reviewing other surveys that focus on evasion attacks and 
include some discussion about data (Section~\ref{sec:relatedwork-surveys-data}).  
We then discuss non-data related reasons behind evasion attacks (Section~\ref{sec:relatedwork-not-data}),
as well as techniques related to evasion attacks and defenses (Section~\ref{sec:relatedwork-attacks}). 
Finally, we discuss data-related concerns for non-evasion attacks (Section~\ref{sec:relatedwork-poisoning}) and
the effects of training data on standard generalization (Section~\ref{sec:relatedwork-standard}).
}

%\vspace{-0.1in}
\subsection{Surveys on Evasion Attacks that Discuss Data}
\label{sec:relatedwork-surveys-data}
Numerous existing surveys 
\revreplace{focus on attack and defense techniques for adversarial robustness. 
%~\cite{Biggio:Roli:PR:2018,
%Rosenberg:Shabtai:Elovici:Rokach:CSUR:2021,
%Li:Li:Ye:Xu:CSUR:2021,
%Maiorca:Biggio:Giorgio:CSUR:2019,
%Demetrio:Coull:Biggio:Lagorio:Armando:Roli:ACMTPS:2021,
%Liu:Tantithamthavorn:Li:Liu:CSUR:2022,
%Liu:Nogueria:Fernandes:Kantarci:IEEECST:2022,
%Akhtar:Mian:IEEEAccess:2018,
%Akhtar:Mian:Kardan:Shah:IEEEAccess:2021,
%Serban:Poll:Visser:CSUR:2020,
%Machado:Silva:Goldschmidt:CSUR:2021,
%Zhang:Sheng:Alhazmi:Li:ACMTIST:2020}.
Only a few of these works mention the relationship between adversarial robustness and properties of the underlying data.} 
{review the literature on evasion attacks.
Most of these works do not focus specifically on properties of data but discuss attack and defense mechanisms, non-data-related reasons for adversarial vulnerability, 
and the different threat models. 
Only a few of these works mention data-related reasons for the existence of adversarial examples~\cite{Serban:Poll:Visser:CSUR:2020, Machado:Silva:Goldschmidt:CSUR:2021, Akhtar:Mian:Kardan:Shah:IEEEAccess:2021, Akhtar:Mian:IEEEAccess:2018}.
}
Specifically, Serban et al.~\cite{Serban:Poll:Visser:CSUR:2020} observe that adversarial vulnerability can be caused by an insufficient training sample size %~\cite{Schmidt:Santurkar:Tsipras:Talwar:Madry:NeurIPS:2018}
and high data dimensionality. %~\cite{Gilmer:Metz:Faghri:Schoenholz:Raghu:Wattenberg:Goodfellow:ICLR:2018}.
Similarly, Machado et al.~\cite{Machado:Silva:Goldschmidt:CSUR:2021} mention that the lack of sufficient training data, high dimensionality, 
and high concentration contribute to adversarial vulnerability.
\revadd{
Akhtar et al.~\cite{Akhtar:Mian:IEEEAccess:2018, Akhtar:Mian:Kardan:Shah:IEEEAccess:2021} also mention high dimensionality, along with other non-data-related reasons, 
as a source of adversarial examples.}

\revadd{A concurrent work by Han et al.~\cite{Han:Lin:Shen:Wang:Guan:CSUR:2023} (published at the end of April 2023) 
studies the origins of adversarial vulnerability in deep learning w.r.t. the model, data, and other perspectives.
The authors mention high dimensionality, distributions with high concentration, a small number of output classes, data imbalance, and the perceptual difference in image frequencies as potential sources of adversarial examples.
However, as (a) the focus of that survey is not on data-related properties in particular, 
(b) its paper search was conducted in 2021, and 
(c) it focuses on deep learning models only, 
our work was able to identify more than 50 additional relevant papers which focus on other types of models, 
e.g., non-parametric and linear classifiers, 
and/or discuss additional types of data-related properties, 
such as, types of distribution, class density, separation, and label quality.}
\revreplace{Yet, none of these surveys explicitly collect and analyze work that focuses on the effects of data properties
on adversarial robustness.}
{In summary, by explicitly focusing on the effects of data properties on evasion attacks in our survey, 
we are able to provide a more complete and detailed discussion on this topic, not covered in prior surveys.}

\vspace{-0.05in}
\subsection{Non-data-related Reasons Behind Evasion Attacks}
\label{sec:relatedwork-not-data}

%\vspace{-0.1in}
%\subsection{Non-data Related Reasons for Adversarial Vulnerability}

There has been a variety of hypotheses regarding the reasons behind adversarial vulnerability of ML systems, particularly for evasion attacks.
%\revreplace{
%In addition to the data used for training,  adversarial robustness could also depend on the choice of the model architecture,
%the training procedure, and the interplay between data and the learning algorithm, i.e., correspondence between the complexity of a model to that of the data.
%This section summarizes the key hypotheses regarding these aspects.
%%The hypotheses reviewed in this section are complementary to the potential influence from the data.
%}
These include data-related properties extensively discussed in this survey, as well as reasons related to the models themselves, 
computational resources, and feature learning procedures. We discuss these below.

%\jr{there is a lot of undefined terminology and jargon in this section.}

\vspace{0.02in}
\noindent
\textbf{Model.}
When Szegedy et al.~\cite{Szegedy:Zaremba:Sutskever:Bruna:Erhan:Goodfellow:Fergus:ICLR:2014} first discovered adversarial examples for visual models, they suspected that the high non-linearity of DNNs resulted in low probability `pockets' of adversarial examples in the learned representation manifold.
They hypothesize that while these pockets can be found through attack algorithms, the samples residing in these pockets have different distributions compared to normal samples and are thus subsequently harder to find when randomly sampling from the input space.
Instead, Goodfellow et al.~\cite{Goodfellow:Shlens:Szegedy:ICLR:2015} hypothesize that
the linearity from activation functions, like ReLU and sigmoid found in high-dimensional neural networks, induce vulnerability towards adversarial perturbations.
To support their claim, they present the attack method FGSM that exploits the linearity of the target classifier.
Fawzi et al.~\cite{Fawzi:Fawzi:Frossard:ICMLWorkshop:2015} also argue against the hypothesis of high non-linearity as the cause for adversarial examples.
They show that all classifiers are susceptible to adversarial attacks and claim that it is the low flexibility of the classifier compared to the complexity of the classification task that results in vulnerability.
The lack of consensus on the primary causes of model vulnerability invites more studies on this topic.

Singla et al.~\cite{Singla:Ge:Basri:Jacobs:NeurIPS:2021} show that enforcing invariance to circular shifts (e.g., rotation) in neural networks induces decision boundaries with a smaller margin than normal, fully connected networks,
which, in turn, reduces the adversarial robustness of the model.
Moosavi{-}Dezfooli et al.~\cite{Moosavi-Dezfooli:Fawzi:Fawzi:Frossard:Soatto:ICLR:2018} introduce universal,
input-agnostic perturbations to mislead the classifier and hypothesize that the vulnerability of a multi-class classifier to such perturbations is related to the shape of its decision boundaries, e.g.,
linear classifiers with decision boundaries that are parallel to each other and
nonlinear classifier with decision boundaries that are curved in a similar way
tend to be less robust as
perturbations in one direction can change the prediction label for a different class.

Tanay and Griffin~\cite{Tanay:Griffin:ArXiv:2016} conjecture that the decision boundary learned by the classifier being too close to (or `tilted towards') the data manifold instead of being perpendicular to it,
results in small perturbations being sufficient to move samples across the decision boundary for misclassification.
%data manifold refers to the underlying structure that the data exhibit

\vspace{0.02in}
\noindent
\textbf{Computational Resources.}
Bubeck et al.~\cite{Bubeck:Lee:Price:Razenshteyn:ICML:2019} use computational hardness theory to show that the time complexity for learning a robust model is exponential to the size of input data and thus is computationally intractable.
Hence, they attribute adversarial vulnerability to computational limitations of current learning algorithms.
Degwekar et al.~\cite{Degwekar:Nakkiran:Vaikuntanathan:COLT:2019} further extend this work and also show the impossibility of efficiently training robust classifiers.

%\subsubsection{Ineffective Learning Perspective}
\vspace{0.02in}
\noindent
\textbf{Feature Learning.}
Ilyas et al.~\cite{Ilyas:Santurkar:Tsipras:Engstrom:Tran:Madry:NeurIPS:2019} show that adversarial vulnerability can be a consequence of a model exploiting well-generalizing but non-robust features,
i.e., features that are spurious and sometimes incomprehensible to humans;
when constraining the model to use robust features, the adversarial robustness increases together with the
interpretability of the learned features.
However, Tsipras et al.~\cite{Tsipras:Santurkar:Engstrom:Turner:Madry:ICLR:2019} note that, as the features for achieving high accuracy may be different from the ones for achieving high robustness, robustness may be at odds with standard accuracy.
%
%\jr{why is it called Ineffective learning when it is about features.}\gx{I put it under ineffective learning as in this case, the model learns/decides the features for generalization, and when given the correct objective, the model in fact, can learn more robust features, so I think the underlying reason is objective we gave for the model didn't guide the model to learn the right features}
%
Instead of seeing adversarial vulnerability as a product of classifiers being overly sensitive to changes in spurious features, Jacobsen et al.~\cite{Jacobsen:Behrmann:Zemel:Bethge:ICLR:2019} hypothesize that classifiers can rather be
overly insensitive to relevant semantic information, e.g., images with drastically different content can share similar latent representations.
The authors introduce a new type of adversarial examples that exploit such insensitivity, where the content of images is altered without changing the resulting prediction label.
%As both insensitivity to semantic content and sensitivity to spurious changes can simultaneously exist in models,
%more investigation into how to define proper objectives for models to effectively distinguish the relevant information is needed.

While all these works propose possible reasons for adversarial vulnerabilities, they are orthogonal to our survey, which focuses particularly on the influence of training data.

\vspace{-0.05in}
\revadd{
\subsection{Evasion Attacks and Defenses}
\label{sec:relatedwork-attacks}
A number of works focus on techniques for generating evasion attacks, countermeasures against these attacks, 
and defining the notion of the attack itself.

%\jr{need to include~\cite{Biggio:Roli:PR:2018,
%Rosenberg:Shabtai:Elovici:Rokach:CSUR:2021,
%Li:Li:Ye:Xu:CSUR:2021,
%Maiorca:Biggio:Giorgio:CSUR:2019,
%Demetrio:Coull:Biggio:Lagorio:Armando:Roli:ACMTPS:2021,
%Liu:Tantithamthavorn:Li:Liu:CSUR:2022,
%Liu:Nogueria:Fernandes:Kantarci:IEEECST:2022,
%Zhang:Sheng:Alhazmi:Li:ACMTIST:2020} x and one more survey.}
%\js{\cite{Biggio:Roli:PR:2018, Rosenberg:Shabtai:Elovici:Rokach:CSUR:2021} moved to Adversarial Examples.
%\cite{Rosenberg:Shabtai:Elovici:Rokach:CSUR:2021,
%Li:Li:Ye:Xu:CSUR:2021,
%Maiorca:Biggio:Giorgio:CSUR:2019, Liu:Tantithamthavorn:Li:Liu:CSUR:2022,
%Liu:Nogueria:Fernandes:Kantarci:IEEECST:2022,
%Zhang:Sheng:Alhazmi:Li:ACMTIST:2020, Demetrio:Coull:Biggio:Lagorio:Armando:Roli:ACMTPS:2021} in Attacks and Defense. \cite{Sun:Dou:Yang:Zhang:Wang:Philip:He:Li:TKDE:2022} was the "one more survey" and is also in Attacks and Defenses.}

\vspace{0.02in}
\noindent
{\bf Attacks and Defense.}
Several works~\cite{Liu:Tantithamthavorn:Li:Liu:CSUR:2022,Liu:Nogueria:Fernandes:Kantarci:IEEECST:2022,Sun:Dou:Yang:Zhang:Wang:Philip:He:Li:TKDE:2022, Demetrio:Coull:Biggio:Lagorio:Armando:Roli:ACMTPS:2021} survey adversarial attacks and defenses, observing that most work focuses on computer vision and NLP domains. 
Zhang et al.~\cite{Zhang:Sheng:Alhazmi:Li:ACMTIST:2020}, 
Rosenberg et al.~\cite{Rosenberg:Shabtai:Elovici:Rokach:CSUR:2021},
Li et al.~\cite{Li:Li:Ye:Xu:CSUR:2021}, and 
Maiorca et al.~\cite{Maiorca:Biggio:Giorgio:CSUR:2019}, 
survey attacks and defenses in the NLP domain, cybersecurity domain for networks, Android malware, and PDF malware, respectively. 
These works identify a similar trend of new attacks constantly bypassing defenses, which gives rise to new defenses being proposed, only to be broken again (a.k.a. the `cat and mouse race' or the `arms race'). 
They also observe that research in this field studies attacks / defenses at a feature-level, which restricts 
the practicality of the developed techniques by the feasibility of perturbing the corresponding features in real life. 

%practical attacks are quite difficult and require some basic knowledge about the model or training data such as the feature set or model architecture. 
%Zhang et al.~\cite{Zhang:Sheng:Alhazmi:Li:ACMTIST:2020}, who study adversarial attacks and defenses in the NLP domain,  
%also find that there are obstacles to generating attacks in real-time. 
%For instance, methods that iteratively use gradients to create adversarial examples can be time-consuming, while one-time approaches may fail to produce potent adversarial examples.
%Several works~\cite{Liu:Tantithamthavorn:Li:Liu:CSUR:2022,Liu:Nogueria:Fernandes:Kantarci:IEEECST:2022,Sun:Dou:Yang:Zhang:Wang:Philip:He:Li:TKDE:2022, Demetrio:Coull:Biggio:Lagorio:Armando:Roli:ACMTPS:2021} 
%discuss how most new attacks and defenses are explored in computer vision and NLP, prior to other fields.


%our survey finds the state of the art w.r.t. data properties
%our survey finds that dimensionality is bad ...
%
%%%Here are the 5 remaining surveys + 1 additional paper for the reviewer.
%Numerous surveys have explored the landscape of adversarial evasion attacks and defenses. 
%For instance, Akhtar et al.~\cite{Akhtar:Mian:IEEEAccess:2018, Akhtar:Mian:Kardan:Shah:IEEEAccess:2021} survey the literature on adversarial robustness of deep learning models from Computer Vision field.
%They review popular attacks on visual models, and provided a categorization of existing defense techniques based on the components it modify in the visual model system \gx{Check}.
%
%Rosenberg et al.~\cite{Rosenberg:Shabtai:Elovici:Rokach:ACMComputingSurvey:2021}, Li et al. ~\cite{Li:Li:Ye:Xu:ACMComputingSurvey:2021} and Demetrio et al.~\cite{Demetrio:Coull:Biggio:Lagorio:Armando:Roli:ACMTPS:2021} review the literature on evasion attacks for cyber-security fields. 
%Li et al. proposed a partial order scheme to compare key attacks and defenses techniques for malware detection in Windows, Android, and PDF domains. 
%
%Zhang et al.~\cite{Zhang:Sheng:Alhazmi:Li:ACMTIST:2020} review the literature on adversarial attacks on deep-learning models for textual classification.
%They pointed out the intrinsic differences between Computer Vision and Natural Language Processing fields that pose challenges to directly apply attacks proposed for Visual models to NLP models and identified the strategies proposed that overcomes the barriers.
%The challenges they identified for creating realistic attacks in NLP fields are from a domain characteristics perspective (e.g., definition of imperceptible perturbations, measurement of the semantic changes),  we differ from them by trying to understand the adversarial robustness of machine learning from the characteristics of underlying data. 
%
%Attack and Defenses for wireless and Mobile systems~\cite{Liu:Nogueria:Fernandes:Kantarci:IEEECST:2022}
%
%

More recent research, not included in the surveys above, has also started investigating the 
susceptibility of newer models to adversarial evasion attacks. 
For example, several studies~\cite{Wang:Pan:Hu:Duan:Pan:IJSWIS:2022,Yin:Lin:Sun:Wei:Chen:TIFS:2023, 
Shi:Han:Tan:Kuang:NeurIPS:2022, Wang:Xie:Microsoft:ChatGPT:ArXiv:2023} proposed attack techniques against contemporary models, 
such as Graph Neural Networks, Generative Pre-training Transformers (GPT), and Vision Transformers. 
These studies showed that adversarial examples persist even for the newer models, some of which are 
trained with large volumes of data. 
As all these works focus on attack and defense mechanisms rather than 
the effects of data on adversarial robustness, our work extends and complements this research.
}

\revadd{
\vspace{0.02in}
\noindent
{\bf Adversarial Examples.}
%2 surveys lines 13 and 14 + 1 additional paper for the reviewer.
Adversarial examples are inputs constructed by perturbing a correctly classified sample in a way that makes the change imperceptible to a human. % but causes the model to misclassify the sample.
However, as `imperceptible to a human' is hard to define, existing research on adversarial examples approximates imperceptibility with a small perturbation measured through $L_p$ norms.
A line of research~\cite{Gilmer:Adams:Goodfellow:Anderson:Dahl:ArXiv:2018,Sharif:Bauer:Reiter:CVPRW:2018,Fezza:Bakhti:Hamidouche:Deforges:QoMEX:2019, Mezher:Deng:Karam:EUVIP:2022} 
investigates the validity of this assumption. 
This work shows that perturbations generated by $L_p$ norms do not entirely align with human perceptions, 
i.e., some changes with a small $L_p$ norm can be apparent to humans. 
In addition, adversarial examples with the minimum $L_p$ perturbation may be less effective and transferable than 
higher perturbation~\cite{Biggio:Roli:PR:2018,Rosenberg:Shabtai:Elovici:Rokach:CSUR:2021}. 
Hence, a number of approaches explore metrics for imperceptibility 
in computer vision and NLP domains~\cite{Fezza:Bakhti:Hamidouche:Deforges:QoMEX:2019,Mezher:Deng:Karam:EUVIP:2022, Zhang:Sheng:Alhazmi:Li:ACMTIST:2020}. 
Yet another issue with $L_p$ norms is that they cannot be used reliably in domains other than images. 
For example, in the case of software/malware, simply generating adversarial examples with $L_p$ norms 
may result in feature representations that are not possible in 
the problem space~\cite{Rosenberg:Shabtai:Elovici:Rokach:CSUR:2021,Pierazzi:Pendlebury:Cortellazz:Cavallaro:2020}. 

While all these works focus on the properties of adversarial examples, 
they are orthogonal to the topic of our survey, as we rather focus on how properties of the training data 
affect the success of adversarial examples.
}

%Gilmer et al.~\cite{Gilmer:Adams:Goodfellow:Anderson:Dahl:ArXiv:2018} argue that, while constraining the perturbations by sufficiently small $L_p$ norms can generate indistinguishable samples for most inputs, the actual imperceptibility of the changes depends on the input sample. 
%Several individual studies~\cite{Sharif:Bauer:Reiter:CVPRW:2018,Fezza:Bakhti:Hamidouche:Deforges:QoMEX:2019, Mezher:Deng:Karam:EUVIP:2022} find faults with using $L_p$ norms to generate adversarial examples. They show that the changes measured by $L_p$ norm, does not entirely align with human perceptions, i.e., some changes with a small $L_p$ norm appear apparent to humans. 
%In some domains adversarial examples do not need to be imperceptible but rather semantically preserving. 
%For example, in the case of Android malware~\cite{Rosenberg:Shabtai:Elovici:Rokach:CSUR:2021}, adversarial examples are small perturbations which fool a model while preserving the semantics of the sample, 
%i.e., a malware stays malicious even after the perturbation. 
%This highlights another problem with $L_p$ norm based adversarial examples as Dong et al.~\cite{Dong:Liu:Shang:NeurIPS:2022} show that the semantics of a sample change during adversarial training. 
%Hence, there is a need for metrics to measure the size of perturbations that is imperceptible or semantically preserving.
%Fezza et al.~\cite{Fezza:Bakhti:Hamidouche:Deforges:QoMEX:2019} and Mezher et al.~\cite{Mezher:Deng:Karam:EUVIP:2022} propose to use objective metrics for image quality to approximate the imperceptibility in the computer vision domain.
%Zhang et al.~\cite{Zhang:Sheng:Alhazmi:Li:ACMTIST:2020}, focusing on providing such a metric for Natural Language Processing.
%Vadillo et al.~\cite{Vadillo:Santana:CS:2022} also highlight conducted subject studies to evaluate the noticeability of audio adversarial examples.

%Even in computer vision, adversarial examples are not always imperceptible. For example, Machado et al.~\cite{Machado:Silva:Goldschmidt:CSUR:2021} find that visible perturbations such as adversarial patch~\cite{Brown:Mane:Roy:Abadi:Gilmer:ArXiv:2017}, and graffiti on stop signs~\cite{Eykholt:Evtimov:Fernandes:Li:Rahmati:Xiao:Prakash:Kohno:Song:CVPR:2018} are also considered adversarial examples in research.

%The aforementioned research examines the work on defining and creating adversarial examples, demonstrating the insufficiency of using conventional $L_p$ norms to evaluate the imperceptibility and semantics between clean and adversarial examples. 

\vspace{-0.1in}
\revadd{
\subsection{Non-Evasion Attacks}
\label{sec:relatedwork-poisoning}
Similar to evasion attacks, data poisoning and backdoor attacks aim to compromise model accuracy. 
However, they achieve it by tampering the training data to create deceptive model decision boundaries. 
%Data poisoning attacks involve modifying the training data to create deceptive decision boundaries, either to manipulate the prediction outcomes of a specific input or the entire model.
%Meanwhile, Backdoor attacks are a form of poisoning attacks where the attacker inject tempered training data with triggers 
% and then activates the attack by showing the trigger pattern at inference time.
In addition, backdoor attacks also require perturbing the test instance to result in a misclassification. 
This is achieved by introducing manipulated training data with triggers that can be activated during the testing phase.

Goldblum et al.~\cite{Goldblum:Tsipras:Xie:Chen:Schwarzchild:song:Madry:Li:Goldstein:TPAMI:2022} and Cinà et al.~\cite{Cina:Grosse:Demontis:Sebastiano:Zellinger:Moser:Oprea:Biggio:Pelillo:Roli:CSUR:2023} 
review recent literature on attack methodologies and countermeasures for both poisoning and backdoor attacks.
Both of these surveys found that existing research made overly-optimistic assumptions when designing / validating attack techniques, e.g., assuming the knowledge of a large portion of training data. 
They advocate for researchers to test proposed methods in more realistic situations to better assess the potential threats. 
Furthermore, they encourage exploration of the relationship between poisoning attacks and evasion attacks. 
This could lead to the creation of attacks that produce less noticeable poisoning examples, 
or defensive strategies that can safeguard models against both backdoor and evasion attacks.
%Their survey catalogs and systematizes the threats in the dataset creation process, and discuss the open problems that benefits the understanding of dataset security. 

In addition to undermining model accuracy, 
adversarial attacks also aim at breaching the privacy and confidentiality of training data. 
In particular, membership inference attacks~\cite{Shokri:Stronati:Song:Shmatikov:SP:2017} attempt to determine whether a specific data point was part of the training set used to train the model.
Hu et al.~\cite{Hu:Salcic:Sun:Dobbie:Yu:Zhang:CSUR:2022} present a comprehensive survey of existing research efforts on membership inference attacks. 
They find that, similar to evasion attacks, the membership inference attack success rate decreases as 
%the training data better represents the whole data distribution, i.e., 
the number of training samples increases.
%and model stealing attacks~\cite{Oliynyk:Mayer:Rauber:CSUR:2023} are designed to breach the privacy of training data and machine learning models. 
However, all these attacks are orthogonal to our survey, as we focus on adversarial evasion attacks.

%Li et al. ~\cite{Li:Jiang:Li:Xia:TNNLS:2022} 
%provide the first survey that focuses on backdoor attacks and identified common scenarios in which backdoor attack happen in real life. 
%Furthermore, they proposed a systematic taxonomy for backdoor attacks and defenses for researchers and practitioners to identify the characteristics and limitations of each method. 

%Wang et al.~\cite{Wang:Ma:Wang:Hu:Qin:Ren:CSUR:2022} and Tian et al.~\cite{Tian:Cui:Liang:Yu:CSUR:2022} argue federated learning~\cite{McMahan:Moore:Ramage:Hampson:Arcas:AISTATS:2017} 
%creates new venue for poisoning attack, and survey recent literature on poisoning attacks for both standard and federated learning scenarios. 
%They present a unified framework to categorize both data poisoning and model poisoning attacks, and compared the defense techniques proposed for each of the learning framework, analyzed their advantages and disadvantages.
}

\vspace{-0.1in}
\subsection{Effects of Training Data on Standard Generalization}
\label{sec:relatedwork-standard}
A number of surveys investigate the influence of data properties on standard
rather than robust generalization.
One of the earliest is probably the work of Raudys and Jain~\cite{Raudys:Jain:TPAMI:1991},
who review studies related to the influence of sample size on binary classifiers, showing that
a limited sample size usually leads to sub-optimal generalization.
%With the development of deep learning and the ever-increasing need for larger training datasets,
%a variety of data augmentation techniques have been proposed.
Bansal et al.~\cite{Bansal:Sharma:Kathuria:CSUR:2021} and
Bayer et al.~\cite{Bayer:Kaufhold:Reuter:CSUR:2022} also survey papers addressing the data scarcity problem,
focusing in particular on the recent advancements in data augmentation techniques in the fields of computer vision, security, and text classification.
Their results show that augmentation techniques %exist for various application domain and
can help improve a model's generalization by reducing the problem of model overfitting.
%They evaluate the effectiveness of such techniques in improving the accuracy of machine learning models.

%Limited sample size is also one of the culprit behind poor robust generalization~\cite{Schmidt:Santurkar:Tsipras:Talwar:Madry:NeurIPS:2018}, we collected a number of researches characterize the sample complexity for robust generalization or propose data augmentation techniques to fill in the sample complexity gap.

Label noise is another aspect of data that influences both standard and robust generalization.
Most works on this topic find that the presence of noisy labels increases the need for a greater number of training samples and may result in unnecessarily complex decision boundaries~\cite{Frenay:Verleysen:TNNLS:2014,Song:Kim:Park:Shin:Lee:TNNLS:2022}.
For example, Fr\'{e}nay and Verleysen~\cite{Frenay:Verleysen:TNNLS:2014} show
that overfitting to label noise greatly degrades a model's standard generalization;
the same effect has been observed in the case of robust generalization~\cite{Sanyal:Dokania:Kanade:Torr:ICLR:2021}.
Song et al.~\cite{Song:Kim:Park:Shin:Lee:TNNLS:2022} survey the impact of label noise in deep learning, arguing
that the presence of noisy labels is a more serious concern for deep models as they contain a larger number of parameters which makes them prone to overfitting to the noise in training data.
%They also point out the connection between adversarial poisoning attacks and noisy labels as
%the countermeasures for both share the goal of learning noise-resilient representations.
They mention that adversarial defense techniques, e.g., adversarial training, are effective against label noise~\cite{Zhu:Zhang:Han:Liu:Niu:Yang:Kankanhalli:Sugiyama:ArXiv:2021, Fatras:Damodaran:Lobry:Flamary:Tuia:Courty:TPAMI:2022}
but do not discuss how label noise influences a deep learning model's robustness under attacks.

Lorena et al.~\cite{Lorena:Garcia:Lehmann:Souto:Ho:CSUR:2020} identify a collection of 26 quantitative metrics that measure data complexity with respect to
(1) ambiguity of classes, i.e., whether the classes can be clearly distinguished with the given features,
(2) sparsity and dimensionality of data, 
%i.e., whether enough information are provided to learn confident decision boundaries, and
(3) complexity of boundary separating the classes, i.e., whether more intricate functions are required to describe the decision boundaries.
The authors also discuss how these metrics help estimate the difficulty of performing classification on a given dataset.
Similar to our survey, the authors show that high dimensionality and small separation between classes hinder standard generalization.
However, the relationship of some of the metrics reviewed by these authors, e.g.,
%faction of borderline points (i.e., a measure for the complexity of the required decision boundary) and
%the fraction of hyperspheres covering data (i.e.,
the number of non-intersecting spheres needed to enclose all data points of a class,
to robust generalization is not studied, according to our survey.

%Moreover, the effect of XXX on standard generalization needs future investigation as well (that is if we found something they do not have).

%Knowing the characteristics of a dataset according to these perspectives can assist researchers and practitioners to select optimal learning algorithms~\cite{Ho:Basu:TPAMI:2002}.

He and Garcia~\cite{He:Garcia:TKDE:2009} focus on the imbalance learning problem. %~--
%the disproportion in the number of samples belonging to each class in a given dataset.
The authors found that most standard algorithms %are designed with the assumption of a balanced class distribution.
%These algorithms
fail to reliably represent the characteristics of the imbalanced data and result in unfavorable performance across classes.
Furthermore, L\'{o}pez et al.~\cite{Lopez:Fernandez:Garcia:Palade:Herrera:InfSci:2013} discuss six intrinsic data characteristics that potentially complicate learning from imbalanced data:
low density, sample overlap between classes, noisy data, borderline instances,
dataset shift between training and testing distributions, and
small disjuncts, i.e., disperse small clusters of samples from a single class.
Their analysis concludes that while all these ``unfavorable'' data characteristics further complicate the data imbalance
issues, data overlap between classes is probably one of the most harmful.
To follow up on this point, Santos et al.~\cite{Santos:Henriques:Pedro:Japkowicz:Fernandez:Soares:Wilk:Santos:AIR:2022}
focus on the joint effect of data imbalance and class overlap on model generalization.
The negative impact of data imbalance, low separation, and noisy data on robust generalization was also discussed in our survey.
Yet, the compounding effect of these factors, as well as the effect of other properties,
on robust generalization needs future investigation.

Recently, Yang et al.~\cite{Yang:Jiang:Song:Guo:IJCV:2022} summarized relevant studies focusing on
long-tailed distributions in the field of Computer Vision.
% and categorize the main methods for alleviating the issues caused by long-tailed distribution.
%They present quantitative metrics for measuring data imbalance and .
This survey also includes work on the influence of long-tail distributions on a model's adversarial robustness~\cite{Wu:Liu:Huang:Wang:Lin:CVPR:2021}, which is covered in our survey.
%which is included in our survey,
The authors advocate for more research on adapting long-tailed-based approaches for standard generalization to improve robust generalization.

Finally, Moreno-Torres et al.~\cite{MorenoTorres:Raeder:Rodrigues:Chawla:Herrera:PR:2012} present a unifying framework to categorize existing definitions of dataset shift~-- the case where the joint distribution of inputs and outputs differs between training and testing data.
While ML models are normally trained under the premise that testing data has a similar distribution to the training data,
in reality, the observed data distribution may be different from the historical data that the model is trained on.
Such difference can substantially compromise the quality of model predictions.
The authors analyze the possible causes for dataset shift, e.g., malicious software that evolves over time, and
review the techniques dealing with dataset shift.
They characterize adversarial attacks as one form of dataset shift, where adversaries adaptively
change test instances to create a distribution that differs from training data.
%All works discussed in our survey assumed similar distribution on training and testing data, treating adversarial attacks as the only dataset shift in the problem setup.
%However, in real applications, the underlying data distribution itself can be non-stationary, and the characterize the influence of the dataset shift between training and testing data on the adversarial robustness is yet to be investigated.

\revadd{Overall, despite the similarities with our work, literature discussed in this section focuses on standard generalization while our survey discusses 
the effect of data on robust generalization.}

%More works use the connection between adversarial attacks and distributional shift to analyze the effect of adversaries on generalization performance~\cite{Tu:Zhang:Tao:NeurIPS:2019}.
%However, we do not discuss them in detail, as they focus more on models instead of data.
%\jr{How is that relevant to data properties section?} \gx{This can be removed, as it an individual work we filtered}

\vspace{-0.1in}
\subsection{Summary}
\revadd{
Our survey is the first to explicitly focus on properties of training data in the context of model robustness under evasion attacks.
Numerous other surveys on evasion attacks discuss attack and defense mechanisms, non-data-related reasons for adversarial vulnerability, and the different threat models. 
We identified only five surveys that considered data-related reasons for evasion attacks. 
However, as these surveys are older and do not focus on data in particular, our work provides a more extensive
and comprehensive view on this topic. 
By including more than 50 papers not covered in prior work, we were able to 
identify additional relevant properties, practical suggestions, and future research directions in this area. 

Additional work studies non-data-related reasons for evasion attacks, as well as non-evasion attacks, 
such as poisoning and backdoor. 
Yet another body of literature examines how data properties affect standard generalization. These works show that 
some of the properties discussed in our survey, such as 
the number of samples, dimensionality, and label quality, also affect clean accuracy. 
There are also additional data properties that are covered exclusively by these or by our work. 
Studying the interplay between data properties for clean and robust accuracy is an interesting research direction, 
which could be facilitated by our work. 
However, all these current works are orthogonal and complementary to ours.
}

%\ad{
%The related work of our survey can be categorized into four key topics: 
%The first topic examines data for other adversarial attacks, this include the research that investigates the link between the data characteristics and model's resilience against poisoning attacks as well as the studies that explore data poisoning and backdoor attacks and their countermeasures. \jr{same issues as before: this is meta-summary, we need a concrete summary.}
%These studies complement our survey as they highlight the threats directly aimed at data, thus emphasizing the importance of secure data collection. 
%The second topic focuses on the relationship between various properties of training data and model's standard generalization ability. 
%This body of work suggests that data traits such as number of samples, dimensionality, label quality also influence model's ability to generalize in standard classification. \jr{this looks more concrete!}
%
%The third strand of research concerns adversarial evasion attacks. 
%The work in this area encompasses the research frontier in evasion attacks and the countermeasures. 
%Due to the large volume of work in this area, there are numerous surveys that gives more detail on the advancement. 
%\jr{meta-summary again}
%In addition to attacks and defenses, one relevant line of work investigates the alignment of the conventional similarity metrics used for adversarial examples and human perception, showing the need for supplementary metrics. \jr{why important?}
%These studies \jr{which "these studies"?} collectively present an extensive overview of other types of work conducted on adversarial robustness.
%The last category of work proposes alternative explanations for model vulnerability to adversarial examples.
%These studies presented hypothesis showing the characteristics of machine learning models, e.g., nonlinearity, invariance to rotational shift etc, induces susceptibility to attacks, as well as limited computational resources and non-robust feature representations. \jr{all text based on previous related work looks somewhat concrete; the new additions should be at least at the same level, or better.}
%These studies supplement our work, offering a broader perspective of potential factors affecting model's robust generalization ability. }
%



%Set reconciliation is the problem of syncing multiple data sets without prior context~\cite{minsky2003set,orlitsky1993interactive,karpovsky2003data,dodis2004fuzzy,eppstein2011s,luo2021capacity,gensync,reconsimilar,iblt_new}. In the classical setup, the problem involves two remote parties Alice and Bob with their data sets $S_A$ and $S_B$. The goal is for Alice to learn the elements local to Bob (\ie{} $S_B \setminus S_A$) without requesting Bob's entire data set, which can be huge. When Bob also seeks to learn the elements local to Alice (\ie{} $S_A \setminus S_B$), we use the term \emph{two-way} set reconciliation, and denote the set of \emph{symmetric differences} as:
%\[
%  S_A \oplus S_B = (S_B \setminus S_A) \cup (S_A \setminus S_B).
%\]

%In the context of \srep{}, we are particularly interested in \emph{communication-efficient} solutions to set reconciliation such as Characteristic Polynomial Interpolation~\cite{minsky2003set} (CPI), BCH codes~\cite{dodis2004fuzzy}, and Invertible Bloom Lookup Tables~\cite{goodrich2011invertible,eppstein2011s}. For instance, CPI incurs a communication cost \emph{equal} to the number of symmetric differences plus a small constant, which makes it almost optimal in that regard.

%On the other hand, when it comes to our analytical model and simulations, we make use of the findings from the blockchain topology-discovering literature. In particular, Wang~\ea{}~\cite{ethna} and Gao~\ea{}~\cite{mes_ether_topo} independently verified that the Ethereum network exhibits ``small-world'' property. Recently, Shahsavari~\ea{}~\cite{theory_model} used a random graph model to simulate Bitcoin network and Ma~\ea{}~\cite{cblocksim} proposed a topology generation based on Watts-Strogatz~\cite{Watts1998} random graph model to capture the Bitcoin network in their \emph{CBlockSim} simulator.

% Why Watts-Strogatz? Network measurement research that finds that
% blockchains can be reasonably modeled using Watts-Strogatz.

% \subsection{The Problem of Block Propagation}
% The traditional three protocols, Graphene, (maybe) Erlay.

% Graphene's inefficiency in cases (3), (4),
% (5)~\cite{anas_empir}... Churn~\cite{anas_churn_tnsm}.

% \subsection{Synchronizing Pairs of Nodes}
% Formulate the basic set reconciliation problem with two nodes,
% introduce block propagation as (sub)set reconciliation (see Graphene)...

% \subsection{Existing Set Reconciliation Algorithms}\label{sec:existinc_syncs}
% Existing algorithms; theoretical guarantees \& experimental
% properties...

% Throughout this work we use \emph{primal} sync to denote any of the
% protocols described in this section. We say that a primal sync is
% \emph{efficient} when its communication cost is proportional to the
% number of differences (\eg CPI and IBLT are efficient).

\section{SREP Algorithm}\label{sec:srep_algo}
We propose a novel distributed algorithm for network-wide transaction
pool synchronization called \srep{} (\emph{Set Reconciliation-Enhanced
  Propagation}). The core building block of \srep{} is a concept that
we denote as \emph{primal} sync --- a set reconciliation protocol with
communication complexity linear in the number of symmetric differences
(\eg{} CPI~\cite{minsky2003set}). Given the local transaction pool as
a set of globally unique identifiers~\cite{utxo}, \srep{} invokes one
primal sync per each neighbor in parallel.

% Specifically, the main idea behind \srep{} is to invoke many
% \emph{primal} syncs in parallel at each node in the network. Primal
% syncs operate on the local transaction pool viewed as a set of
% globally unique transaction hashes~\cite{utxo}.

One way to support many parallel invocations of primal syncs is to
create one transaction pool \emph{replica} per each neighbor. Then run
primal syncs in parallel using the corresponding replicas to avoid
write collisions. Upon the completion of all parallel tasks, we can
reuse the primal sync to incorporate new elements into the local
transaction pool. We describe \srep{} in \aref{algo:parallel_srep}
using $S_{n}$ to denote the transaction pool at node $n$, $d_{in}$ to
denote the differences between $S_i$ and $S_n$ that reside in $S_i$,
and \textbf{Sync} to denote a primal sync. As an illustration, in
\fref{fig:parallel_srep}, we depict one iteration of \srep{}'s main
loop (\lineref{line:srep_main_loop}), assuming that each node $n$
holds only one transaction whose hash is also $n$.

%\begin{algorithm}
%  \caption{\srep{} Algorithm.}\label{algo:parallel_srep}
%  \SetKwBlock{Ateach}{At each node $n \in \{0, |V| - 1\}$}{end}
%  \SetKwBlock{InThread}{Do in thread $T_i$}{end}
%  \SetKwBlock{Loop}{Loop}{EndLoop}
%  \SetKw{kwSpawn}{spawn thread}
%  \SetKw{kwSync}{Sync}
%  \SetKw{kwForIn}{in}
%  \KwIn{Network $G = (V, E)$ as adjacency list.}
%  \Ateach{
%    \Loop { \label{algo:parallel_srep:loop}\label{line:s%rep_main_loop}
%      T $\gets$ [ ] \tcp*[l]{List of threads}
%      \For(\tcp*[h]{Neighbors of $n$}){$i$ \kwForIn $G[n]$} {
%        $S_n^i \gets S_n$ \tcp*[l]{Replicate data set}
 %       \kwSpawn $T_i$ \;
  %      T.append ( $T_i$ ) \;
   %     \InThread{
    %      \tcp{Network sync}
     %     $d_{in} \gets$ \kwSync( $S_n^i$, $S_i$ ) \;
      %    $S_n^i \gets S_n^i \cup d_{in}$ \;
       % }
      %}
%      \For{$i \gets 0$ \KwTo $T.size - 1$} {
 %       T[i].join( ) \;
  %    }
   %   \For{$i$ \kwForIn $G[n]$} {
    %    \tcp{Local sync}
     %   $S_n^i \setminus S_n \gets$ \kwSync( $S_n$, %$S_n^i$ ) \;
     %   $S_n \gets S_n \cup (S_n^i \setminus S_n)$ \;
      %}
    %}
  %}
  %\end{algorithm}

\begin{algorithm}
  \small
  \caption{\srep{} Algorithm.}\label{algo:parallel_srep}
  \SetKwBlock{Ateach}{At each node $n \in \{0, |V| - 1\}$}{end}
  \SetKwBlock{InParallel}{Do in parallel}{end}
  \SetKwBlock{Loop}{Loop}{EndLoop}
  \SetKw{kwSpawn}{spawn thread}
  \SetKw{kwSync}{Sync}
  \SetKw{kwForIn}{in}
  \KwIn{Network $G = (V, E)$ as adjacency list.}
  \Ateach{
    \Loop { \label{algo:parallel_srep:loop}\label{line:srep_main_loop}
      \For(\tcp*[h]{Neighbors of $n$}){$i$ \kwForIn $G[n]$} {
        $S_n^i \gets S_n$ \tcp*[l]{Replicate data set}
        \InParallel{
            \tcp{Network sync}
            $d_{in} \gets$ \kwSync( $S_n^i$, $S_i$ ) \;
            $S_n^i \gets S_n^i \cup d_{in}$ \;
        }
      }
      \For{$i$ \kwForIn $G[n]$} {
        \tcp{Local sync}
        $S_n^i \setminus S_n \gets$ \kwSync( $S_n$, $S_n^i$ ) \;
        $S_n \gets S_n \cup (S_n^i \setminus S_n)$ \;
      }
    }
  }
\end{algorithm}

\begin{figure*}
  \centering
  \includegraphics[width=\linewidth]{figures/parallel_srep_3.pdf}
  \caption{One iteration of \srep{} on a tractably small
    network.}
  \label{fig:parallel_srep}
\end{figure*}

\subsection*{Avoiding Full Replication}
% In typical Watts-Strogatz networks, the number of these replicas is
% determined by the average node degree ($\overline{deg}$), while the
% worst case replication factor is as high as $|V|$ for complete
% graphs.

\srep{} from \aref{algo:parallel_srep} has a significant memory
overhead caused by transaction pool replication for each
neighbor. However, certain primal syncs allow us to implement \srep{}
without replication, thus mitigating this memory overhead. In
particular, multiple set reconciliation algorithms mentioned in
\sref{sec:rw} use data set \textit{sketches} to perform
synchronization and modify the underlying data sets only at the end of
the protocol.

For instance, CPI reads from the set only once, at the beginning of
the protocol, and writes to it only once at the end of the protocol.
Suppose that we choose CPI as the primal sync in \srep{}. Then we can
construct the characteristic polynomial~\cite{minsky2002practical} of
$S_n$ as the very first step in each iteration (after
\lineref{algo:parallel_srep:loop} in
\aref{algo:parallel_srep}). Instead of using the neighbor replicas, we
can now use the same characteristic polynomial in all neighbor
threads.
% (see \aref{algo:alternative_parallel_srep} in
% \appref{appendix:alternative_parallel_srep})
As no thread will modify the polynomial, the procedure is thread-safe
and the threads can now write directly to the underlying set.
% (\lineref{algo:alternative_parallel_srep:line_update} in
% \aref{algo:alternative_parallel_srep})
Although the write operation will need to acquire the corresponding
lock, since set union is commutative and associative, the order in
which the threads update the set does not matter. % In addition, the
% problem of duplicate elements arriving from different neighbors can be
% mitigated through the standard procedures such as hashing.
As we now avoid replication, the local synchronization step can be
safely eliminated altogether.

Note that this implementation improvement does not change the
functional properties of \srep{}. That is, each thread still operates
on its own version of the sketch and will update its sketch only at
the beginning of the subsequent iteration. Hence, a difference that
arrives in iteration $i$ via some neighbor thread will only get
acknowledged by other threads in iteration $i + 1$. For that reason,
we use the notion of ``replicas'' in the subsequent analysis.

\begin{table}
  \centering
  \begin{tabular}{>{\centering\arraybackslash}m{2cm} m{5cm}}
    \toprule[1pt]
    $G = (V, E)$ & Network of $|E|$ edges and $|V|$ nodes \\ \addlinespace[.3em]
    $S_n$ & Transaction pool at node $n \in \{0..|V| - 1\}$ \\ \addlinespace[.3em]
    $d_{ij} = S_i \setminus S_j$ & Differences between $i$ and $j$ that
                                   reside in $i$\\ \addlinespace[.3em]
    $\overline{deg}$ & Average node degree \\ \addlinespace[.3em]
    $t_n$ & Time node $n$ spends to synchronize with all its neighbors once \\ \addlinespace[.3em]
    $T_{x\%}$ & Time until $x$\% of $G$ is synchronized \\ \addlinespace[.3em]
    $\Sigma_{x\%}$ & Number of primal sync invocations \\ \addlinespace[.3em]
    $C_{x\%}$ & Overall communication cost \\
    \bottomrule[1pt]
  \end{tabular}
  \caption{Summary of notation.}
  \label{tab:not}
\end{table}

\section{\srep{} Performance Analysis}\label{sec:analysis}
Several aspects affect the performance of \srep{}, including the
network topology and the statistics of transaction pools. To aid our
analysis, we first define an explicit network model, and then analyze
\srep{} in a step-by-step fashion. In each stage of our analysis, we
describe a \srep{} variant with the corresponding set of
\emph{simplifying assumptions} and analyze its performance. By
successively relaxing these assumptions, we arrive at the final
version of \srep{}. \tref{tab:not} summarizes notation used throughout
this work.

\definition{percentage} We use $T_{x\%}$, $\Sigma_{x\%}$, and
$C_{x\%}$ to denote time, total number of primal sync invocations, and
total communication cost until $x\%$ of transaction pools in the
network are equal. When $x = 100$, we say that \emph{full network}
synchronization is achieved --- the ultimate goal of \srep{}.

% Note that we use the notion of \emph{full sync} ($100\%$
% sync) to denote the state of the network when all nodes keep
% equal sets of transactions.

\subsection{Network Model}\label{sec:net_model}
Watts-Strogatz~\cite{Watts1998} random graphs allow us to describe a
wide range of realistic blockchain network topologies reasonably
well~\cite{mes_ether_topo, ethna, under_the_hood, cblocksim}. A
typical set of parameters to Watts-Strogats model are the number of
nodes in the network $|V|$, average node degree $\overline{deg}$, and
rewire probability $p$~\cite{Watts1998}.

For instance, each Bitcoin node selects 8 random neighbors upon
joining the
network~\cite{bitcoin,txprobe,degwithunreachable},
which has been shown to yield an unstructured random
graph~\cite{theory_model}. We can capture this in the Watts-Strogatz
model by setting $\overline{deg} = 8$ and $p = 1$.  Ethereum's
neighbor selection mechanism, on the other hand, relies on a Kademlia
distributed hash table (DHT)~\cite{kademlia}, and yields a network
with more structure~\cite{mes_ether_topo}. Notwithstanding this,
multiple recent measurement results have independently confirmed that
the generated network exhibits the ``small world'' property and fits
the Watts-Strogatz model~\cite{mes_ether_topo,
  ethna,under_the_hood}. That is, the average shortest path between
any two nodes can be reasonably approximated by
$O\left( log_{\overline{deg}}|V| \right)$, and the diameter of the
network is small~\cite{chung2001diameter}.

% as illustrated in \fref{fig:high_level}.

% \begin{figure}[h]
%   \centering
%   \includegraphics[width=\columnwidth]{figures/small_world.pdf}
%   \caption{The small world property in random graphs generated through
%   Watts-Strogatz model with $k = \overline{deg} = 19$ and rewire
%   probability $p = 0.24$.}
%   \label{fig:high_level}
% \end{figure}

% Topology (Erdos-Reny \& Watts-Strogatz)... Results from blockchain
% measurements literature... Early results on Bitcoin network suggest
% that average degree is between 8 and
% 12~\cite{Miller2015DiscoveringB,txprobe}. However, a more recent work
% that accounts for the unreachable nodes~\cite{degwithunreachable}
% finds that about a half of all peers have the degree close to 125,
% which is the default maximum of the Bitcoin Core
% implementation~\cite{bitcoin_core_defcount}.

% TODO: for this reason, we keep our network model general and analyze
% \srep{}'s performance as a function of topology.

Besides the graph topology, our network model also captures the states
of transaction pools across the network. In particular, we define the
\emph{pool assignment} $A$ as a collection of sets
$S_{0}..S_{|V| - 1}$ where set $S_{i}$ represents the transaction pool
at node $i$. We model the statistical properties of $A$ through the
following \emph{pool parameters}:
\begin{enumerate}
\item[$\mathcal{S}$:] \emph{sizes distribution}. A discrete random
  variable describing the sizes of transaction pools $S_{i}$ for
  $i \in \{0...|V| - 1\}$,
\item[$s$:] \emph{sizes vector}. A $|V|$-size vector where
  elements are drawn from $\mathcal{S}$,
\item[$\mathcal{P}$:] \emph{differences distribution}. A discrete
  random variable describing the sizes of mutual differences between
  the pairs of transaction pools (\ie{} $|S_{i} \oplus S_{j}|$),
\item[$M$:] \emph{mutual differences matrix}. A $|V| \times |V|$ upper
  triangular matrix of mutual differences. For the given topology
  $G = (V, E)$, the elements of the matrix are defined as:
  \[
    m_{ij} =
    \begin{dcases}
      |S_{i} \oplus S_{j}| & \text{when } (i, j) \in E \text{ and } i < j,\\
      0 & \text{otherwise}.
    \end{dcases}
  \]
  Non-zero elements are drawn from $\mathcal{P}$.
% \item[$D$:] \emph{differences partition matrix}. A $|V| \times |V|$
%   square matrix with zero diagonal where
%   $d_{ij} = |S_{i} \setminus S_{j}|$. For the given mutual differences
%   matrix $M$, there are $\prod_{i < j} (m_{ij} + 1)$ difference partition
%   matrices (see \appref{appendix:triangle}).
\item[$\mathcal{U}$:] \emph{universe}. A discrete random variable
  from which we draw transaction IDs. We choose $\mathcal{U}\{0, u\}$
  to be a uniform random variable for some $u \geq |V|$.
\end{enumerate}

% Comments: there is a measurement work which claims that Bitcoin is
% not a random graph in practice (Discovering Bitcoin’s Public
% Topology and Influential Nodes). There is also a work that claims
% how discovering an actual Bitcoin topology is an open problem since
% 2014 (Evolutionary Random Graph for Bitcoin Overlay and Blockchain
% Mining Networks).

\subsection{Elementary \srep{} (\esrep{})}
The starting point for our build up of \srep{} is called \emph{elementary} \srep{}
(\aref{algo:elem_srep}). We summarize its simplifying assumptions as
follows:
\begin{enumerate}[label=(\subscript{A}{{\arabic*}})]
\item All nodes have global view of the network.\label{a:global_view}
% \item Network topology is generated as in \sref{sec:net_model}.\label{a:topology}
\item Initially, the transaction pools at each node contain only one
  element (transaction) that is unique across all network nodes (\eg{}
  index of the node). Strictly speaking, we set the pool parameters
  as: $\mathcal{S} = 1$, $\mathcal{P} = 2$ % , $D = J - I$, where $J$ is all-ones matrix and $I$ is the identity matrix
  , and
  $u \gg |V|$.\label{a:single_elem}
\item No new transactions arrive to the network after the
  initialization.\label{a:data_gen}
\item In one iteration of elementary \srep{}
  (\lineref{line:elem_srep_iter}), nodes take turns to perform their
  synchronization duties such that no two nodes invoke primal sync at
  the same time. For instance, nodes with smaller indices go first. An
  iteration ends when all nodes have invoked synchronization once for
  all their neighbors.\label{a:precedence}
\item Nodes synchronize with their neighbors sequentially. For
  instance, the neighbors with smaller indices get synchronized first
  (\lineref{line:elem_srep_sort}).\label{a:local_order}
\item All synchronizations are two-way (lines~\ref{line:sync_one_way}
  and~\ref{line:sync_other_way}), meaning that the differences are
  exchanged in both directions.\label{a:two_way}
\item All synchronizations take equally long.\label{a:equal_duration}
\end{enumerate}

\begin{algorithm}
  \small
  \caption{Elementary \srep{}.}\label{algo:elem_srep}
  \SetKwBlock{Ateach}{At each node $n \in \{0, |V| - 1\}$}{end}
  \SetKw{kwSync}{Sync}
  \SetKw{kwForIn}{in}
  \SetKw{kwNot}{not}
  \SetKw{kwSort}{sort}
  \KwIn{Network $G = (V, E)$ as adjacency list.}
  \While {network is \kwNot fully synchronized} { \label{line:elem_srep_iter}
    \For{$n \gets 0$ \KwTo $\{0..|V| - 1\}$} {
      $neighbors \gets$ \kwSort( $G[n]$ ) \; \label{line:elem_srep_sort}
      \For{$i$ \kwForIn $neighbors$} {
        $d_{in} \gets$ \kwSync( $S_n$, $S_i$ ) \;
        $d_{ni} \gets$ \kwSync( $S_i$, $S_n$ ) \;
        $S_{n} \gets S_{n} \cup d_{in}$ \; \label{line:sync_one_way}
        $S_{i} \gets S_{i} \cup d_{ni}$ \; \label{line:sync_other_way}
      }
    }
  }
\end{algorithm}

In the context of \esrep{}, the following special case is particularly
significant for the analysis.

\lemma{elemsrepcomplete} For \esrep{} over a complete graph
$G = (V, E)$, the communication cost to sync the entire network is
\[
  C_{100\%}(G) = |V| \cdot (|V| - 1).
\]

% and we visualize its execution on a small network in
% \fref{fig:elem_model}.

% \begin{figure*}
%   \centering
%   \includegraphics[width=\linewidth]{figures/simple_model_sync.pdf}
%   \caption{\esrep on Watts-Strogatz graph with $N = 5$ and
%   $\overline{deg} = 2$ until $T_{100}$. The leftmost is the state
%   before the first iteration. The rightmost is the state after the
%   second iteration. All 5 nodes have been synced by then.}
%   \label{fig:elem_model}
% \end{figure*}


% \coroll{elemsrepsingledur} In \esrep{}, the longest single-node
% synchronization is proportional to the maximum degree of the network
% \[
%   \max_{i \in V} t_i = \mathcal{O}(\Delta(G)).
% \]
% As a direct corollary of \ref{a:precedence} and
% \ref{a:equal_duration}.

% \subsubsection{\esrep{} Over Complete Graphs}
% It has been shown in measurement research that popular permissionless
% blockchains do not rely on complete graph
% topologies~\cite{degwithunreachable,txprobe,mes_ether_topo,Miller2015DiscoveringB}. Notwithstanding,
% \srep{} performance analysis over complete graphs is useful as a
% performance benchmark. Given that our topology model relies on Watts
% Strogatz network generation, we can easily obtain complete graphs by
% setting $k = \overline{\deg} = |V|$. Furthermore, we will see that
% complete graphs give some useful bounds on communication cost to
% achieve full network sync ($C_{100}$), time %AT - I prefer C_{100%%}
% until the entire network is in sync ($T_{100}$), and number of primal
% sync invocations to achieve that ($\Sigma_{100}$).

% \theorem{elemsrepcomplete}For \esrep{} over a complete graph
% $G = (V, E)$, the following holds:
% \begin{align*}
    %     & \Sigma_{100}(G) = |E| = {|V| \choose 2},\\
%   &T_{100}(G) = \mathcal{O}(\Sigma_{100}),\\
    %     &C_{100}(G) = |V| \cdot (|V| - 1),\\
%   &\max_{i \in V} t_i = |V| - 1 = t_i, \text{ for } i \in V.
% \end{align*}

% \proof{} As per \ref{a:two_way}, each edge $(u, v)$ transfers $d_{uv}$ to $v$
% and $d_{vu}$ to $u$. After edge $(u, v)$ is consumed, nodes $u$ and
% $v$ will hold $S_u \cup S_v$. Since a complete graph has
% $|V| \choose 2$ such edges, each node will be fully synchronized
% (\ie{} $\cup_{i \in V} S_i$) when all edges are consumed. As one edge
% corresponds to one sync and all syncs take equally long
% \ref{a:equal_duration}, we get the expressions for $\Sigma_{100}$ and
% $T_{100}$. Since each node has only one element \ref{a:single_elem}
% and can send it directly to every other node (complete graph), we get
% the expression for $C_{100}$. The longest single node sync expression
% follows from the fact that each node has \emph{exactly} $|V| - 1$
% incident edges and \ref{a:equal_duration}. Finally, note that for
% \esrep{} over complete graphs, \ref{a:local_order} is not
% necessary. The order in which nodes consume their incident edges does
% not matter.\qqed{}

\subsection{Elementary Parallel \srep{} (\epsrep{})}\label{sec:p_srep}
The main aim of the \textit{elementary parallel} \srep{} is to relax
\ref{a:global_view}, \ref{a:precedence} and
\ref{a:local_order}. Instead of invoking synchronization in order,
\epsrep{} invokes synchronization for all neighbors at once (\ie{}
\aref{algo:parallel_srep}). In addition to that, we also relax
\ref{a:equal_duration}. The synchronization between nodes $u$ and $v$
now takes time \emph{equal} to the number of their mutual differences
(\ie $|d_{uv} \cup d_{vu}|$). As discussed earlier in \sref{sec:rw},
this is a reasonable assumption to make (\eg{} CPI has such a
property).

% Next, we derive the communication cost, number of iterations, and time
% needed for \srep{} to achieve the full network synchronization.

\theorem{comm_bounds} In \epsrep{} and for any connected network
$G = (V, E)$, we have the following bounds on the overall
communication cost until the network is fully synchronized:
\[
  |V| \cdot (|V| - 1) \leq C_{100\%} < |V| \cdot (|V|^2 - 1).
\]

\proof{} The lower bound is obtained similarly as in
\lemmaref{elemsrepcomplete}. The least amount of communication to achieve full synchronization is equivalent to each node sending its element to all the other nodes directly. On the other hand, we get the upper bound by observing that there cannot be more than
$|V|^{2} \cdot (|V| - 1)$ \emph{redundant} element transmissions on top of the lower bound. Redundant transmissions happen when a node receives an element via multiple replicas in the same iteration. % (see
% \appref{appendix:redundant})
To count all redundant transmissions, we observe that, in each iteration, each node either receives some new elements or does not receive any. In the latter case, obviously, no redundant transmissions happen. Otherwise, if there are some new elements received, the
following holds:
\begin{enumerate*}[label=(\arabic*)]
\item there will be no more than $|V|$ new elements arriving at the
  node across all iterations, as there is only that much elements in the
  network, and
\item for each element, there cannot be more than $|V| - 1$ redundant
  transmissions, as there cannot be more than that much replicas at
  any node.
\end{enumerate*}
Thus, there cannot be more than $|V|^{2} \cdot (|V| - 1)$ redundant
transmissions at all nodes in all iterations.\qqed{}

As in Watts-Strogatz networks we have $\overline{deg}$ replicas at
each node on average, the same counting argument from above applies in
the following form.

\coroll{comm_ws} For \epsrep{} in Watts-Strogatz networks:
\[
  C_{100\%} < |V| \cdot (|V| \cdot \overline{deg} + |V| -
  1).
\]

On the other hand, to infer the upper bound on the time that \epsrep{}
needs to complete a full sync ($T_{100\%}$), we rely on following
definition.

\definition{iteration} $I_{x\%}(G)$ is the maximal number of \epsrep{}
iterations (\lineref{line:srep_main_loop} in
\aref{algo:parallel_srep}) at any node to achieve x\% network
synchronization.

\theorem{iter_bound} In \epsrep{} and for any connected network
$G = (V, E)$, with the shortest path between nodes $u$ and $v$ denoted
as $dist(u, v)$, the maximum number of iterations required for a full
network synchronization is equal to the diameter of the network:
\[
  I_{100\%}(G) = \max_{u, v \in V} dist(u, v).
\]

\proof{} By the definition of full synchronization, all elements need
to reach every other node. Without a loss of generality, suppose that
we follow the propagation of some element $i \in V$ during the
execution of \epsrep{}. Since the graph is connected, in each
iteration of \epsrep{}, $i$ will progress exactly one step further
through the network. The number of iterations required to synchronize
the entire network is then equivalent to the maximum distance between
any two nodes in the network (\ie diameter).\qqed{}

\lemma{par_srep_compl} In \epsrep{} over complete graphs $G = (E, V)$:
\[
  I_{100\%}(G) = 1 \text{ and } C_{100\%} = |V| \cdot (|V| - 1).
\]
The former holds as the diameter of complete graphs is 1. The latter
is a consequence of the former; as no element traverses more than one
edge, there cannot be any redundant transmissions.

\coroll{iter_ws} For \epsrep{} and Watts-Strogatz networks, the
maximal number of iterations at any node to synchronize the entire
network ($I_{100\%}$) is logarithmic in the size of the network.

Counting the number of nodes that have heard about an element
$n \in V$ in iteration $i$ of \epsrep{} over a Watts-Strogatz network,
we get the following sum:
\[
  1 + \overline{deg} + \overline{deg}^2 + \ldots + \overline{deg}^{i}.
\]
By equating it to $|V|$, we can express $i$, the number of iterations
until all nodes have heard of $n$, as a logarithmic function of
$|V|$~\cite{chung2001diameter}. Practically speaking, \epsrep{} will
complete in logarithmically small number of iterations
($\approx 4\log_{\overline{deg}}(10))$) for the blockchain networks of
realistic sizes (\eg{} Blockchain and
Ethereum~\cite{txprobe,degwithunreachable}).

\theorem{par_srep_time} In general graphs $G = (V, E)$, the following
holds for \epsrep{}:
\begin{align*}
  & T_{100\%} \leq I_{100\%}(G) \cdot \max_{i \in V} t_{i} < I_{100\%}(G) \cdot
    |V|,\\
  & \Sigma_{100\%} \leq I_{100\%}(G) \cdot |E|.
\end{align*}

\proof{} Since synchronizations happen in parallel, the overall
elapsed time is proportional to the number of iterations. Any sync
invocation at any node will take strictly less than $|V|$, as no two
data sets can differ in more than $|V| - 1$ elements (each data set
keeps exactly one element at the beginning). Since in each iteration
nodes sync with all their neighbors and each sync is two-way by
\ref{a:two_way}, there will be no more than $|E|$ syncs in each
iteration.\qqed{}

% Recall that we relaxed \ref{a:equal_duration} for the purposes of
% \epsrep{}, meaning that \corollref{elemsrepsingledur} does not hold
% anymore.

\begin{figure}
  \centering
  \includegraphics[width=.9\columnwidth]{figures/redundant_transmission_of_deg.pdf}
  \caption{Amount of redundant transmissions in \epsrep{} over a
    network of 100 nodes ($p = 0.24$).} \label{fig:redundant}
\end{figure}

\subsubsection*{The $\overline{deg}$ Dilemma}
Due to the counting argument from \theoremref{comm_bounds}, the upper
bound on overall communication cost is \emph{not} tight; there must be
at least some elements that will \emph{not} generate redundant
transmissions in any connected network. % (see
% \appref{appendix:counting})
On top of that, the topology of the network plays a complex role in
generating redundant transmissions. Intuitively speaking, the impact
of $\overline{deg}$ in Watts-Strogatz networks is twofold, and
conflicting:
\begin{enumerate*}[label=(\arabic*)]
\item the larger $\overline{deg}$, the larger the average number of
  replicas per node, which may cause redundant transmissions, and
\item the larger $\overline{deg}$, the shorter the average pair-wise
  shortest path among the nodes in the network, which makes each
  element traverse less intermediate nodes to reach the entire network,
  % (see \fref{fig:small_world_vary_k} in
  % \appref{appendix:deg_dilemma})
  thus reducing the probability
  of redundant transmissions.
\end{enumerate*}
We plot this non-monotonic effect that $\overline{deg}$ has on the
amount of redundant transmissions in \fref{fig:redundant} for a
tractably small network. Up to a point, the first effect (replicas
count) prevails and drives the overall communication cost up. After
that point, the second effect (path shortening) prevails and drives
the overall communication cost down all the way to the point when the
network becomes a complete graph and there is no redundant
transmissions at all.

\subsection{Multi-element \srep{}}\label{sec:me_srep}
% TODO:
% - As we mentioned in the introduction, modern block propagation
% protocols resemble reconciliation of transaction pools. Since these
% pools represent sets of many transactions, we develop our SREP model
% further to capture this property. Specifically, we relax (A_2) and
% let each node keep multiple set elements of which some are common
% among some peers.
%
% To better describe the properties of transaction pools and the
% degree of their mutual similarity, we conducted a measurement
% campaign using the logging subsystem of Imtiaz et al.

The final stage in building \srep{} is \emph{multi-element
  \srep{}}. We build it by relaxing \ref{a:single_elem} ---
transaction pools can now initially contain multiple elements. In
terms of our network model, this means that our $\mathcal{S}$ (sizes
distribution) and $\mathcal{P}$ (differences distribution) are no more
constant.
% On the other hand, note that \aref{algo:parallel_srep} can be reused
% for \mesrep{}. Indeed, \epsrep{} operates with multi-element
% transaction pools in all but the first iteration.
Thus, \srep{} is a \emph{generalization} of \epsrep{}.

\definition{f_func} Function $f: (G, A) \mapsto \mathbb{Z} $ maps a
pair of a topology $G$ and a pool assignment $A$ to a non-negative
integer via first constructing the corresponding mutual differences
matrix $M$, then computing $\sum m_{ij}$.

\definition{g_func} Function $g: (G, A) \mapsto (G, A_{(next)})$ maps a pair
of a topology $G$ and a pool assignment $A$ to the same topology $G$
and a transformed pool assignment $A_{(next)}$. We define the transaction
pools in the transformed pools assignment $A_{(next)}$ as:
\[
  S_{(next)i} = S_{i} \cup (\bigcup_{j \in G[i]} S_{j}).
\]

We use $\bigcup_{j \in G[i]} S_{j}$ to denote the union of all
transaction pools $S_j$ corresponding to the neighbors of node $i$ in
the previous iteration.

\definition{comp_with_itself} For some function $h$, we write
$h^{(n)}(x)$ to denote the composition of function $h$ with itself
$n$ times, starting with argument $x$:
\[
  h^{(n)}(x) = \underbrace{h \circ h \cdots h}_{n} (x).
\]

\definition{asgn_g_foo} $A_{(n)}$ is the assignment resulting from $n$
compositions of $g$ with itself starting with the initial pool
assignment that we denote as $A = A_{(0)}$.

\lemma{me_srep} For a network model $(G, A)$ where $G$ is a
connected graph and $A$ the initial pool assignment,
the number of \srep{} iterations to achieve the full network
synchronization $I_{100\%}(G, A)$ is given as a solution to the
following equation:
\[
  f(g^{(I_{100\%}(G, A))}(G, A)) = 0.
\]

% TODO: function is not monotonically non-increasing (Ari's
% example). Base the proof on the fact that a zero exists. There are
% no new arrivals, once the function reaches zero, it will 'stay in
% zero forever'.

Note that by \defref{g_func}, $g$ exactly corresponds to one iteration
of \srep{}. That is, the transformed pool assignment $A_{(next)}$
reflects the state of the transaction pools after an iteration of
\srep{} at all nodes in the network. Composing $g$ with itself $n$
times corresponds to repeating an iteration of \srep{} at all nodes
$n$ times. By a similar argument as in \theoremref{iter_bound}, all
elements will reach all nodes after some number of iterations. Since
this implies that no two sets have any differences, $M$ will be an
all-zeros matrix. That is, $(f \circ g^{(n)}) (G, A)$ has at least one
zero. Thus, the number of times we need to compose $g$ with itself
until $f(G, A_{(n)}) = 0$ gives us the maximal number of \srep{}
iterations to achieve full network synchronization.

\theorem{me_srep_iter} For a connected graph $G = (V, E)$ and an
initial pool assignment $A$, the number of \srep{} iterations to
achieve the full network synchronization is bounded by the diameter of
the network:
\[
  I_{100\%}(G, A) \leq \max_{u, v \in V} dist(u, v).
\]

\proof{} As \srep{} is a generalization of \epsrep{}, the argument
here is similar to that of \theoremref{iter_bound}. To achieve the
full network synchronization, elements need to traverse at most the
diameter of $G$. As opposed to \epsrep{}, in \srep{} each element may
initially appear at more than one node, dictated by the differences
distribution $\mathcal{P}$. Thus the diameter is an upper bound on
\srep{} iterations.\qqed{}

\lemma{me_srep_comm} For a connected graph $G = (V, E)$ and initial
pool assignment $A$ with the corresponding mutual differences matrix
$M$, the communication cost of \srep{} is:
\begin{align*}
  &C_{100}(G, A) = \sum_{i = 0}^{I_{100\%}(G, A)} f(G, A_{(i)}) \\
                & < I_{100\%}(G, A) \cdot \max\{f(G, A),\dotsc,f(G,
                  A_{(I_{100\%(G, A)})})\}.
\end{align*}

In $i$th iteration of \srep{}, we transmit exactly as much elements
as there are in the differences matrix that corresponds to
$A_{(i)}$. Given $I_{100\%}(G, A)$ from \lemmaref{me_srep}, we get the
overall communication cost of \srep{}.

% We get the upper bound by observing that $\sum_{i, j \in V} m_{ij}$
% is the maximum of function $f$ for the initial pool assignment $A$.

\lemma{me_srep_inv} In \srep{} over a connected network
$G = (V, E)$ with the given initial pool assignment $A$ and the
largest order statistics of differences distribution $\mathcal{P}$
denoted as $\mathcal{P}_{(n)}$:
\begin{align*}
  & T_{100\%} \leq I_{100\%}(G, A) \cdot \max_{i \in V} t_{i} =
    I_{100\%}(G, A) \cdot \mathcal{P}_{(n)}, \\
  & \Sigma_{100\%} \leq I_{100\%}(G, A) \cdot |E|.
\end{align*}

The argument is similar to that of \theoremref{par_srep_time}.

% \coroll{me_srep_expected_time} In \srep{} over a connected network
% $G = (V, E)$ with the given initial pool assignment $A$ and
% differences distribution $\mathcal{P}$, expected time to full network
% synchronization is:
% \[
%   \E[T_{100}] = I_{100}(G, A) \cdot \E[\mathcal{P}].
% \]

%%%% Old TODOs

% TODO: continue here.
% However, as we saw in ME-SREP, there exist primal syncs whose
% duration is proportional to d, thus, if we need to choose between
% the two, we opt for modeling X accurately.

% The amount of mutual differences among the neighbors is drawn from a
% random distribution\cite{citation}.

% TODO: Explain here how the measurement results can be used here to
% model these distributions.

% TODO: here I stopped. What can we say about the number of iterations
% to T_100, communication cost, and number of sync invocations
% compared to elementary SREP and elementary parallel SREP?

% The ``differences generating algorithm'' (which is done through
% transaction dissemination, in practice)

% \subsection{Full \srep{}}
% Relax \ref{a:data_gen}. New differences arrive according to some
% random distribution.

% \subsection{Multi-party \srep{}}
% Do \srep{} using multi-party set reconciliation.

% \subsection{Discussion}
% % TODO: 1) no new transaction arrivals, 2) multiparty sync
% Note that assumption~\ref{a:data_gen} pertaining to the arrivals of
% new transactions serves to simplify the analysis and does not
% constrain \srep{} in practice.

% On the other hand, multi-party to further reduce communication...

%\subsection{Discussion}
Finally, note that the assumptions in our analysis such
as~\ref{a:data_gen} --- no new transactions arrive after \srep{}
starts, are artificial in that they simplify our analysis, but they do
not constrain \srep{} in practice. The properties such as the overall
communication cost ($C_{100\%}$) and time ($T_{100\%}$) to sync the
entire network relate to the transactions that have arrived before
\srep{} begins.

\section{Simulations}\label{sec:sim}
To validate our analytical findings about \srep{}, we construct an
event-based simulator called \srepsim{}~\cite{srepsim_code} that
shares the topology generation procedure with \emph{CBlockSim} of Ma
\ea{}~\cite{cblocksim} and adds the other parameters of our network
model described in \sref{sec:net_model}.

% This approach allows us to model realistic network topologies such
% as Bitcoin and Ethereum, while controlling the statistical
% properties of transaction pools as further described in
% \sref{sec:config_net}.

In the rest of this section, we first describe a method to parameterize our network model. Then, we use such parameterized model to validate the main analytical properties of \srep{}. We then compare
the overall communication cost of \srep{} with a similar approach from
the literature. At the end, we present a \srepsim{} optimization that
allows for easy \srep{} communication cost calculation over
large-scale networks.

\subsection{Configuring Network Model
  Parameters}\label{sec:config_net}
Unlike the simulation approaches from the literature (\eg{}
% \emph{BlockSim:Faria}~\cite{blocksim_faria},
% \emph{BlockSim:Alharby}~\cite{blocksim_alharby}, and
\emph{SimBlock}~\cite{simblock}), our network model can seamlessly
integrate real-world transaction pool data. For instance, the
empirical
%distributions of $\mathcal{S}$ and $\mathcal{P}$ can be generated using the measurement software\footnote{For some small subset of all nodes in the network.} such as \emph{log-to-file} of Imtiaz \ea{}~\cite{anas_orphan_icbc,anas_orphan_tnsm}. This software
distributions of $\mathcal{S}$ and $\mathcal{P}$ can be generated for
some small subset of all nodes in the network using the measurement
software such as \emph{log-to-file} of Imtiaz
\ea{}~\cite{anas_orphan_icbc,anas_orphan_tnsm}. This software
instruments adjacent Bitcoin nodes and periodically serializes the
snapshots of their transaction pools. From these transaction pool
snapshots, we can measure transaction pool sizes and their mutual
differences to construct the empirical distributions for $\mathcal{S}$
and $\mathcal{P}$.

For the purpose of this work, we have conducted a 3-day long
measurement campaign on two time-synchronized Bitcoin nodes and
requested the transaction pool snapshots each
minute. \fref{fig:diffs_dist} depicts the results that we
obtained. Roughly speaking, the set sizes fit the Maxwell distribution
reasonably well, while the number of mutual differences fits the
Hyperbolic distribution. Next, given the empirical distribution of
$\mathcal{S}$, we need to configure the rest of our network model's
pool parameters\footnote{Direct usage of $\mathcal{P}$ is also
  possible but perhaps harder. % (see
  % \appref{appendix:triangle})
  }. Ultimately, we need to construct a pool
  assignment $A$ that conforms to the differences distribution
$\mathcal{P}$.

In \srepsim{}, we construct such assignments through
\procref{proc:a_gen}. For the given network topology $G = (V, E)$ and
the sizes distribution $\mathcal{S}$, we need to configure the
parameter $\psi$ such that the resulting assignment $A$ produces a
differences distribution that resembles $\mathcal{P}$. As shown in
\fref{fig:psi}, $\psi = 0.35$ works reasonably well with our empirical
sizes distribution. Note that by increasing $\psi$, we can decrease
the average similarity among the transaction pools (\ie{} increase the
number of their mutual
differences).% \footnote{See \appref{appendix:proc_disc} for
  % a discussion about \procref{proc:a_gen}.}.

\begin{procedure}
  \small
  \caption{Network parameterization in SREPSim.()}\label{proc:a_gen}
  \KwIn{Network $G = (V, E)$.}
  \KwIn{Sizes distribution $\mathcal{S}$.}
  \KwIn{Parameter $\psi$.}
  \KwOut{Pool assignment $A$.}
  \SetKw{kwSample}{sample}
  \SetKw{kwElemsFrom}{elements from}
  \SetKw{kwForIn}{in}
  $u \gets \lceil\ {\psi \E[\mathcal{S}]}\ \rceil$ \;
  $\mathcal{U}\{0, u - 1\}$ \tcp*{Uniform distribution}
  sizes $\gets$ \kwSample $|V|$ \kwElemsFrom $\mathcal{S}$ \;
  A $\gets$ [ ] \;
  \For{$i \gets 0$ \KwTo $|V| - 1$} {
    $S_{i} \gets$ \kwSample sizes[i] \kwElemsFrom $\mathcal{U}$ \;\label{proc:a_gen:line_col}
    A.append ( $S_{i}$ ) \;
  }
\end{procedure}

\begin{figure}
  \centering
  \includegraphics[width=.8\columnwidth]{figures/set_sizes_dist.pdf}
  \includegraphics[width=.8\columnwidth]{figures/diffs_dist.pdf}
  \caption{Empirical distributions of transaction pool sizes
    $\mathcal{S}$ for two adjacent Bitcoin nodes (up) and their mutual
    differences $\mathcal{P}$ (down). Best distribution fits in red
    (using Error Sum of Squares).}
  \label{fig:diffs_dist}
\end{figure}

\begin{figure}
  \centering
  \includegraphics[width=.9\columnwidth]{figures/psi_variation.pdf}
  \caption{Empirical differences distribution for two adjacent Bitcoin
    nodes versus the differences distribution generated by
    \procref{proc:a_gen} for various $\psi$. Watts-Strogatz network
    with 100 nodes ($\overline{deg} = 19$ and $p = 0.24$).}
  \label{fig:psi}
\end{figure}

\subsection{\srep{} Properties Validation}
The main analytical properties that we want to validate through
simulations are \srep{}'s communication cost to achieve full network
sync ($C_{100\%}$) and the time required to achieve this state
($T_{100\%}$). In particular, we want to show how these two quantities
change as a function of the network topology and the measure of
difference among the transaction pools. % As discussed earlier in
% \sref{sec:net_model}, one of the main parameters that affects the
% topology of the network is the average node degree ($\overline{deg}$).

In \fref{fig:i_vs_diam}, we plot the maximal number of \srep{}
iterations $I_{100\%}$ and the network diameter as functions of the
average network degree $\overline{deg}$. In \fref{fig:comm_and_t}, we
plot the communication cost and time to full network sync as a
function of $\overline{deg}$. The main observation is that the overall
communication increases with the average node degree as a consequence
of using more replicas per node, which increases the number of
redundant transmissions (see \fref{fig:redundant}). On the other hand,
the time to achieve full network sync does not exhibit such a
trend. Since primal syncs run in parallel, it is the maximal number of
differences among any two nodes in the network that dominates the
total time to sync the network (see \lemmaref{me_srep_inv}).

\begin{figure}
  \centering
  \includegraphics[width=.8\columnwidth]{figures/sim_I_vs_diam.pdf}
  \caption{Maximal number of \srep{} iterations at any node
    ($I_{100\%}$) bounded by the network diameter for Watts-Strogatz
    graphs with 1000 nodes ($p = 0.24$). 95\% confidence intervals.}
  \label{fig:i_vs_diam}
\end{figure}

\begin{figure}
  \centering
  \includegraphics[width=.8\columnwidth]{figures/comm_and_t.pdf}
  \caption{Relative communication cost ($C_{100\%}$) and time to fully
    synchronize the network ($T_{100\%}$). Network with 1000 nodes
    ($p = 0.24$).}
  \label{fig:comm_and_t}
\end{figure}

\subsection{Comparison with \mempoolsync{}}\label{sec:mempoolsync}

\mempoolsync{} of Imtiaz \ea{} is a transaction pool synchronization
protocol that can improve the average transaction propagation delay by
50\% in the event of churn in the Bitcoin
network~\cite{anas_churn_tnsm}. Here we describe this protocol and
compare its communication efficiency with our newly proposed \srep{}
through simulations.

As pointed out in~\cite{anas_churn_tnsm}, the main reason
for slow block propagation times is a large number of missing
transactions in the transaction pools of the block-receiving
nodes. This effect occurs in the legacy block propagation protocols
such as \emph{CompactBlock}~\cite{compact_block} and the more recent
improvements such as
\emph{Graphene}~\cite{ozisik2019graphene,anas_empir}. Thus, the goal
of \mempoolsync{} is to supply the nodes with potentially missing
transactions, and it does so through an \emph{ancestor score}-based
heuristics~\cite{bitcoin_ancestor_score}. The protocol uses a small
constant \texttt{DefTXtoSync} as the default number of transaction
hashes that the transmitting node will select from its transaction
pool in descending order of ancestor score. The transmitting node will
send exactly \texttt{DefTXtoSync} selected transaction hashes
\emph{unless} one of the following holds:
\begin{enumerate}[label=\arabic*)]
\item Transmitting node's transaction pool is much larger than
  \texttt{DefTXtoSync} (\eg{} 10 times). In this case, the node
  will send $Y \times \text{\texttt{DefTXtoSync}}$ top rated
  transactions, where $Y$ is a constant between 0 and 1, or
\item Transmitting node's transaction pool is smaller than
  \texttt{DefTXtoSync}. In this case, the node will send its entire
  transaction pool. Because \texttt{DefTXtoSync} is a small constant,
  this is a quite rare event. It occurs only when the node has just
  joined the Bitcoin network or has just propagated a large block that
  triggered a massive transaction pool cleanup~\cite{anas_churn_tnsm}.
\end{enumerate}

\begin{figure}
  \centering
  \includegraphics[width=.8\columnwidth]{figures/srep_vs_mempoolsync_netsize.pdf}
  \caption{Normalized overall communication cost of \srep{}
    ($C_{100\%}$) and \mempoolsync{} as a function of network
    size. Data from~\sref{sec:config_net}. $DefTXtoSync = 1000$. $Y$
    is the \mempoolsync{} heuristic constant.}
  \label{fig:mempool}
\end{figure}

In \fref{fig:mempool}, we compare the overall communication costs of
\mempoolsync{} and \srep{}. For \srep{}, we plot the communication
cost to sync the entire network $(C_{100\%})$. For \mempoolsync{}, we
plot the communication cost that \mempoolsync{} incurs until \srep{}
would achieve a full sync.

Note that this kind of comparison gives an advantage to
\mempoolsync{}. While \srep{}'s $C_{100\%}$ implies that the network
is fully synced, \mempoolsync{}'s communication cost does not. In
fact, \mempoolsync{} has no guarantees about the communication (or
time) needed to sync the entire network. Note also that \mempoolsync{}
uses Bitcoin internals to calculate the ancestor score of the
transactions and later uses this score to determine which transactions
to transmit. As opposed to \mempoolsync{}, \srep{} is a general
approach that does not rely on any Bitcoin internals and can be
seamlessly integrated into other blockchains that keep transaction
pools.

% One of the MempoolSync drawbacks are non-efficient communication...
% Note that the transmitting node in \mempoolsync{} may still send
% certain amount of transactions that are present at the receiving end.
% TODO: here
% Mempool is not a sync. It transmits what receiver may already
% have. Even after this much communication has already been spent,
% there is not guarantee that Mempool will have the network synced.

\begin{table}
  \centering
  \begin{tabular}{>{\centering\arraybackslash}p{.5cm}>{\centering\arraybackslash}p{1cm}>{\centering\bfseries\arraybackslash}p{1cm}>{\centering\bfseries\arraybackslash}p{1cm}>{\centering\bfseries\arraybackslash}p{2cm}}
    \toprule
    $\overline{deg}$ & $\psi$ & Diameter \textit{average} & $I_{100\%}$ \textit{average} & $C_{100\%}$ (GB) \textit{average} \\
    \midrule
    \multirow{3}{*}{4} & 0.355 & \multirow{3}{*}{16} & 2.5 & 1.214397\\
                     & 0.5 & & 3.0 & 3.165879\\
                     & 0.6 & & 3.1 & 4.801665\\
    \midrule
    \multirow{3}{*}{8} & 0.355 & \multirow{3}{*}{9} & 1.7 & 2.428649\\
                     & 0.5 & & 2.0 & 6.317304\\
                     & 0.6 & & 2.0 & 9.569259\\
    \midrule
    \multirow{3}{*}{12} & 0.355 & \multirow{3}{*}{7} & 1.0 & 3.642738\\
                     & 0.5 & & 1.5 & 9.485572\\
                     & 0.6 & & 2.0 & 14.347242\\
    \midrule
    \multirow{3}{*}{16} & 0.355 & \multirow{3}{*}{6} & 1.0 & 4.876714\\
                     & 0.5 & & 1.0 & 12.649385\\
                     & 0.6 & & 1.0 & 19.135943\\
    \midrule
    \multirow{3}{*}{20} & 0.355 & \multirow{3}{*}{5} & 1.0 & 6.065679\\
                     & 0.5 & & 1.0 & 15.804836\\
                     & 0.6 & & 1.0 & 23.886079\\
    \midrule
    \multirow{3}{*}{24} & 0.355 & \multirow{3}{*}{5} & 1.0 & 7.294909\\
                     & 0.5 & & 1.0 & 18.966694\\
                     & 0.6 & & 1.0 & 28.672272\\
    \midrule
    \multirow{3}{*}{28} & 0.355 & \multirow{3}{*}{5} & 1.0 & 8.465624\\
                     & 0.5 & & 1.0 & 22.156316\\
                     & 0.6 & & 1.0 & 33.446278\\
    \bottomrule
  \end{tabular}
  \caption{\srep{} over a 10,000 nodes network. $p = 0.24$.}
  \label{tab:srep_large}
\end{table}

\subsection{Communication Cost in Large-Scale
  Networks}\label{sec:large_net}

Event-based simulators such as \srepsim{} may consume prohibitive
amounts of memory and take a long time to complete simulations when
the simulated network is large~\cite{cblocksim}. To address this
issue, we designed a \srepsim{} module that computes \srep{}'s
performance metrics analytically. In particular, we implement the
functions from Definitions~\ref{definition:f_func}
and~\ref{definition:g_func}, and rely on the results from
\lemmaref{me_srep_comm} to compute $C_{100\%}$ and $I_{100\%}$. We
describe the \srepsim{}'s analytical module in
\procref{proc:comm_analytic}. Using this module, we can easily compute
the desired performance metrics for the networks of realistic sizes
(\eg{} Bitcoin and
Ethereum)~\cite{txprobe,degwithunreachable}.

In~\tref{tab:srep_large}, we summarize the results for a 10,000 nodes
network with various average node degrees ($\overline{deg}$) and the
measure of similarity among transaction pools ($\psi$). As we report
the communication cost, we assume that the transaction pools represent
each transaction as a 32-byte long globally unique
hash~\cite{utxo}. All simulations complete in tens of minutes.

\begin{procedure}
  \small
  \caption{SREPSim's analytical module.()}\label{proc:comm_analytic}
  \KwIn{Network $G = (V, E)$.}
  \KwIn{Initial pool assignment $A$ as $S_0..S_{|V| - 1}$.}
  \KwOut{Overall network communication cost $C_{100\%}$.}
  \KwOut{Maximal number of iterations $I_{100\%}$.}
  \SetKwFunction{FCalcM}{CalculateM}
  \SetKwProg{Fn}{function}{:}{}

  \Fn{\FCalcM{$A$}}{
    $M \gets$ zeros($|V| \times |V|$) \tcp*{Zero matrix}
    \For{$i \gets 0$ \KwTo $|V| - 1$} {
      \For{$j \gets i + 1$ \KwTo $|V| - 1$} {
        \uIf(\tcp*[h]{i neighbor of j}){$i \in G[j]$} {
          $M[i][j] \gets |S_i \oplus S_j|$ \;
        }
      }
    }
    \KwRet $M$\;
  }

  $C_{100\%} \gets 0$ \;
  $I_{100\%} \gets 0$ \;
  $M \gets$ \FCalcM($A$) \;
  \While{$\sum m_{ij} > 0$} {
    \For{$i \gets 0$ \KwTo $|V| - 1$} {
      $S'_{i} \gets S_{i}$ \tcp*{New assignment}
      \For{$j \in G[i]$} {
        $S'_{i} \gets S'_{i} \cup S_{j}$ \;
      }
    }
    $C_{100\%} = C_{100\%} +  \sum m_{ij}$ \;
    $I_{100\%} \gets I_{100\%} + 1$ \;
    $A \gets A'$ \;
    $M \gets$ \FCalcM($A$) \;
  }
\end{procedure}

% \vspace{-14pt}
\section{Conclusion}\label{sec:conclusion}
In this work, we have developed and analyzed \srep{}, an independent
protocol that assists block propagation in large-scale blockchains.
This new protocol synchronizes transaction pools of nodes in the
blockchain network using communication-efficient set reconciliation
approaches from the literature.  However, rather than inserting itself
directly into the block propagation process, as previous works have
done, \srep{} operates in a distributed manner \emph{outside} the
block propagation channels of the network.  As a result, it is easier
to formally analyze its performance, and, indeed, we have shown that
it completes in time bounded by the network diameter (or logarithmic
in network size for the ``small-world'' networks that reasonably model
blockchain networks).

We have also validated our analytical findings against a novel
event-based simulator that we have developed.  We run the simulator on
real-world transaction pool statistics drawn from our own measurement
campaign.  In our simulations, \srep{} incurs only tens of gigabytes
of overall bandwidth overhead to synchronize networks with ten
thousand nodes, which is several times better than the current
approach in the literature.

% The bandwidth overhead of \srep{} is a function of the transaction pool statistics. By incorporating the transaction pool statistics from our measurement campaign into our simulation model, we show that \srep{} incurs only tens of gigabytes of overall bandwidth overhead to synchronize networks with ten thousand nodes. This is several times better than the heuristic approach from the literature.

For future work, we propose to consider \emph{multi-party} set
reconciliation~\cite{multiparty,multiparty_cpi} in the context of
transaction pool sync. Though the main benefit may be further
reduction in overall communication cost, it is not clear whether an
advantage over pairwise approaches can be achieved when an average
pairwise intersection is large compared to the total intersection
($\cap_{i}S_{i}$)~\cite{multiparty}.
% Novak: Discussion on the no arrivals in between the start and end of
% SREP moved to the separate subsection in the Analysis section.

% The limitations of our analysis (such as the assumption that no new
% transactions arrive to the pools between syncrhonization start and
% completion) are artificial, in that they
% simplify our analysis, but they do not constrain our approach
% in practice.
% % The limitations of our analysis that we aim at overcoming in the
% % future work are
% % \begin{enumerate*}[label=(\arabic*)]
% % \item the assumption that no new transactions arrive to the pools
% %   between \srep{}'s start and completion, and
% % \item no nodes are churning (appear as intermittently connected).
% % \end{enumerate*}
% Finally, \emph{multi-party} set
% reconciliation~\cite{multiparty,multiparty_cpi} should also be
% considered in the context of transaction pool sync. While the main
% benefit may be further reduction in overall communication cost, it is
% not clear whether an advantage over pairwise approaches can be
% achieved when an average pairwise intersection is large compared to
% the total intersection ($\cap_{i}S_{i}$)~\cite{multiparty}.

\vspace{-5pt}
\section*{Acknowledgments}
The authors would like to thank Red Hat, the Boston University Red Hat
Collaboratory (award \# 2022-01-RH03), and the US National Science
Foundation (award \# CNS-2210029) for their support.

\bibliographystyle{IEEEtran}
\bibliography{bibliography}

% \appendices
% \section{Appendix for Proofs}

\paragraph{Proof of Theorem \ref{thm:main}.}

\begin{proof}
\label{proof:main}
Our proof has two steps. In Step 1, we will show that SimCLR is equivalent to minimizing the cross entropy loss defined in Eqn.~(\ref{eqn:cross-entropy}). 
In Step 2, we will show  that minimizing the cross-entropy loss 
is equivalent to spectral clustering on $\bfpi$. 
Combining the two steps together, we have proved our theorem. 

\textbf{Step 1: } SimCLR is equivalent to minimizing the cross entropy loss.

The cross-entropy loss takes expectation over 
$\bfW_\bfX\sim \mathbb{P}(\cdot ; \bfpi)$, 
which means $\bfW_\bfX$ has exactly one non-zero entry in each row $i$. By Lemma~\ref{lem:multinomial}, we know every row $i$ of $\bfW_\bfX$ is independent of other rows. Moreover, 
$\bfW_{\bfX,i}\sim \mathcal{M}(1, \bfpi_i/\sum_j \bfpi_{i,j})=\mathcal{M}(1, \bfpi_i)$, because $\bfpi_i$ itself is a probability distribution.
Similarly, we know $\bfW_\bfZ$ also has the row-independent property by sampling over $\mathbb{P}(\cdot;\bfK_\bfZ)$.
Therefore, by Lemma~\ref{lem:cross_split}, we know Eqn.~(\ref{eqn:cross-entropy}) is equivalent to:
\[
 -\sum_{i=1}^n \mathbb{E}_{\bfW_{\bfX,i}}[\log \mathbb{P}(\bfW_{\bfZ,i}=\bfW_{\bfX,i};\bfK_\bfZ)],
\]

This expression takes expectation over $\bfW_{\bfX,i}$ for the given row $i$. Notice that 
$\bfW_{\bfX,i}$ has exactly one non-zero entry, which equals $1$ (same for $\bfW_{\bfZ,i}$). 
As a result
we expand the above expression to be:
\begin{equation}
 -\sum_{i=1}^n \sum_{j\neq i} \Pr(\bfW_{\bfX,i,j}=1)\log \Pr(\bfW_{\bfZ,i,j}=1).
\label{eqn:detailed-expansion}    
\end{equation}


By Lemma~\ref{lem:multinomial}, $\Pr(\bfW_{\bfZ,i,j}=1)=\bfK_{\bfZ,i,j}/\|\bfK_{\bfZ,i}\|_1$ for $j\neq i$. Recall that $\bfK_\bfZ=(k(\bfZ_i-\bfZ_j))_{(i,j)\in[n]^2}$, which means 
$\bfK_{\bfZ,i,j}/\|\bfK_{\bfZ,i}\|_1=\frac{\exp(-\|\bfZ_i-\bfZ_j\|^2/{2\tau})}{\sum_{k\neq i}
\exp(-\|\bfZ_i-\bfZ_k\|^2/{2\tau})
}$ for $j\neq i$, when $k$ is the Gaussian kernel with variance $\tau$. 

Notice that $\bfZ_i=f(\bfX_i)$, so we know
\begin{equation}
-\log \Pr(\bfW_{\bfZ,i,j}=1)=
-\log \frac{\exp(-\|f(\bfX_i)-f(\bfX_j)\|^2/{2\tau})}{\sum_{k\neq i}
\exp(-\|f(\bfX_i)-f(\bfX_k)\|^2/{2\tau}),
}
\label{eqn:infonce-equivalence}    
\end{equation}


The right hand side is exactly the InfoNCE loss defined in Eqn.~(\ref{eqn:infonce}).
Inserting Eqn.~(\ref{eqn:infonce-equivalence}) into Eqn.~(\ref{eqn:detailed-expansion}), we get the SimCLR algorithm, which first samples augmentation pairs $(i,j)$ with $\Pr(\bfW_{\bfX,i,j}=1)$ for each row $i$, and then optimize the InfoNCE loss. 

\textbf{Step 2: } minimizing the cross entropy loss 
is equivalent to spectral clustering on $\bfpi$.


By Lemma~\ref{lem:convert_to_spectral}, we may further convert the loss to 
\begin{equation}
\label{eqn:main-theorem-repul-attr}
\min_{\bfZ}
-\sum_{(i,j)\in [n]^2} \mathbf{P}_{i,j}
\log k (\bfZ_i-\bfZ_j)+\log \mathbf{R}(\bfZ).
\end{equation}
Since $k$ is the Gaussian kernel, this reduces to \[
\min_\bfZ \mathrm{tr}(\bfZ^\top \mathbf{L}(\bfpi) \bfZ)
+\log \mathbf{R}(\bfZ),
\]

where we use the fact that $\mathbb{E}_{\bfW_\bfX\sim \mathbb{P}(\cdot; \bfpi)}[\mathbf{L}(\bfW_\bfX)]
=\mathbf{L}(\bfpi)
$, because the Laplacian operator is linear and $
\mathbb{E}_{\bfW_\bfX\sim \mathbb{P}(\cdot; \bfpi)}(\bfW_\bfX)=\bfpi
$.
\end{proof}

\paragraph{Proof of Theorem \ref{thm:clip}.}
\begin{proof}
Since $\bfW_\bfX\sim \mathbb{P}(\cdot;\bfpi_{\mathbf{A}, \mathbf{B}})$, we know 
$\bfW_\bfX$ has exactly one non-zero entry in each row, denoting the pair that got sampled. 
A notable difference compared to the previous proof is we now have $n_\mathcal{A}+n_\mathcal{B}$ objects in our graph. CLIP deals with this by taking a mini-batch of size $2N$, 
such that $n_\mathcal{A}=n_\mathcal{B}=N$, and adding the $2N$ InfoNCE losses together. We label the objects in $\mathcal{A}$ as $[n_\mathcal{A}]$, and the objects in $\mathcal{B}$ as $\{n_\mathcal{A}+1, \cdots, n_\mathcal{A}+n_\mathcal{B}\}$. 

Notice that $\bfpi_{\mathbf{A}, \mathbf{B}}$ is a bipartite graph, so the edges of objects in $\mathcal{A}$ will only connect to object in $\mathcal{B}$ and vice versa. We can define the similarity matrix in $\cZ$ as $\bfK_\bfZ$, 
where $\bfK_\bfZ(i, j+n_\mathcal{A})=\bfK_\bfZ(j+n_\mathcal{A},i)= k(\bfZ_i-\bfZ_j)$ for $i\in [n_\mathcal{A}], j\in [n_\mathcal{B}]$, and otherwise we set $\bfK_\bfZ(i,j)=0$. 
The rest is same as the previous proof. 
\end{proof}

\paragraph{Proof of Theorem \ref{thm:exponential}.}

\begin{proof}
\label{proof:exponential}
Since the objective function consists of a linear term combined with an entropy regularization, which is a strongly concave function, the maximization problem is a convex optimization problem. Owing to the implicit constraints provided by the entropy function, the problem is equivalent to having only the equality constraint. We then introduce the Lagrangian multiplier $\lambda$ and obtain the following relaxed problem:

$$
\widetilde{E}(\boldsymbol{\alpha})=\psi_{1}-\sum_{i=1}^n \alpha_{i} \psi_{i}+\tau \sum_{i=1}^n \alpha_{i}\log \alpha_{i}+\lambda\left(\boldsymbol{\alpha}^{\top} \mathbf{1}_n-1\right).
$$

As the relaxed problem is unconstrained, taking the derivative with respect to $\alpha_{i}$ yields

$$
\frac{\partial \widetilde{E}(\boldsymbol{\alpha})}{\partial \alpha_{i}}=-\psi_{i}+\tau\left(\log \alpha_{i}+\alpha_{i} \frac{1}{\alpha_{i}}\right)+\lambda=0.
$$

Solving the above equation implies that $\alpha_{i}$ takes the form
$
\alpha_{i}=\exp \left(\frac{1}{\tau} \psi_{i}\right) \exp \left(\frac{-\lambda}{\tau}-1\right).
$ Since $\alpha_{i}$ lies on the probability simplex, the optimal $\alpha_{i}$ is explicitly given by
$
\alpha^{*}_{i}=\frac{\exp \left(\frac{1}{\tau} \psi_{i}\right)}{\sum_{i^{\prime}=1}^n \exp \left(\frac{1}{\tau} \psi_{i^{\prime}}\right)} .
$ Substituting the optimal point into the objective function, we obtain
$$
\begin{aligned}
E\left(\boldsymbol{\alpha}^*\right)  &=\psi_1-\sum_{i=1}^n \frac{\exp \left(\frac{1}{\tau} \psi_{i}\right)}{\sum_{i^{\prime}=1}^n \exp \left(\frac{1}{\tau} \psi_{i^{\prime}}\right)} \psi_{i}+\tau \sum_{i=1}^n \frac{\exp \left(\frac{1}{\tau} \psi_{i}\right)}{\sum_{i^{\prime}=1}^n \exp \left(\frac{1}{\tau} \psi_{i^{\prime}}\right)}\log \frac{\exp \left(\frac{1}{\tau} \psi_{i}\right)}{\sum_{i^{\prime}=1}^n \exp \left(\frac{1}{\tau} \psi_{i^{\prime}}\right)} \\
& =\psi_1 - \tau \log \left(\sum_{i=1}^n \exp \left(\frac{1}{\tau} \psi_{i}\right)\right).
\end{aligned}
$$
Thus, the Lagrangian dual function is given by
\begin{equation*}
-E\left(\boldsymbol{\alpha}^*\right)= -\tau \log \frac{\exp \left(\frac{1}{\tau} \psi_{1}\right)}{\sum_{i=1}^n \exp \left(\frac{1}{\tau} \psi_{i}\right)}.\qedhere
\end{equation*}
\end{proof}



\section{More on Experiments} \label{section: experiment_details}

\paragraph{CIFAR-10 and CIFAR-100} CIFAR-10 ~\citep{krizhevsky2009learning} and CIFAR-100 ~\citep{krizhevsky2009learning} are well-known classic image classification datasets. Both CIFAR-10 and CIFAR-100 contain a total of 60k $32 \times 32$ labeled images of different classes, with 50k for training and 10k for testing. CIFAR-10 is similar to CIFAR-100, except there are 10 different classes in CIFAR-10 and 100 classes in CIFAR-100.

\paragraph{TinyImageNet} TinyImageNet ~\citep{le2015tiny} is a subset of ImageNet ~\citep{deng2009imagenet}. There are 200 different object classes in TinyImageNet, with 500 training images, 50 validation images, and 50 test images for each class. All the images in TinyImageNet are colored and labeled with a size of $64 \times 64$.

\textbf{Pseudo-code.} Algorithm \ref{alg:Training Procedure} presents the pseudo-code for our empirical training procedure.

\begin{algorithm}[!htbp]
\caption{Training Procedure}
\label{alg:Training Procedure}
\begin{algorithmic}[1]
\REQUIRE trainable encoder network $f$, batch size $N$, augmentation strategy \textit{aug}, loss function $L$ with hyperparameters \textit{args}
\FOR {sampled minibatch ${x_i}_{i=1}^N$}
\FORALL{$i \in { 1, ..., N }$}
\STATE draw two augmentations $t_i = \textit{aug}\left(x_i\right) $, $t_i' = \textit{aug}\left(x_i\right) $
\STATE $z_i = f\left(t_i\right)$, $z_i' = f\left(t_i'\right)$
\ENDFOR
\STATE compute loss $\mathcal{L} = L(N, z, z', \textit{args})$
\STATE update encoder network $f$ to minimize $\mathcal{L}$
\ENDFOR
\STATE \textbf{Return} encoder network $f$
\end{algorithmic}
\end{algorithm}

We also provide the pseudo-code for our core loss function used in the training procedure in Algorithm \ref{alg:Core loss}. The pseudo-code is almost identical to SimCLR's loss function, with the exception of an extra parameter $\gamma$.

\begin{algorithm}[!htbp]
\caption{Core loss function $\mathcal{C}$}
\label{alg:Core loss}
\begin{algorithmic}[1]
\REQUIRE batch size $N$, two encoded minibatches $z_1, z_2$, $\gamma$, temperature $\tau$
\STATE $z = \textit{concat}\left(z_1, z_2\right)$
\FOR {$i \in {1, ..., 2N }, j \in {1, ..., 2N}$ }
\STATE $s_{i,j} = \Vert z_i - z_j \Vert_2^{\gamma}$
\ENDFOR
\STATE \textbf{define} $l(i, j)$ \textbf{as} $l(i, j) = - \log \frac{exp\left(s_{i,j}/\tau \right)}{\sum_{k=1}^{2N} \mathbf{1}{[k \ne i]} exp\left(s{i, j} / \tau \right)} $
\STATE \textbf{Return} $\frac{1}{2N} \sum_{k=1}^N\left[l(i, i+N) + l(i+N, i)\right]$
\end{algorithmic}
\end{algorithm}

Utilizing the core loss function $\mathcal{C}$, we can define all kernel loss functions used in our experiments in Table \ref{table: loss definition}. For all $z_i \in z$ with even dimensions $n$, we define $z_{L_i} = z_i\left[0:n/2\right]$ and $z_{R_i} = z_i\left[n/2:n\right]$.

\begin{table}[ht]
\centering
\begin{tabular}{{@{}l|l@{}}}
Kernel  &  Loss function \\ \midrule
Laplacian & $\mathcal{C}\left(N, z, z', \gamma=1, \tau\right)$\\ \midrule
Sum       & $\lambda * \mathcal{C}\left(N, z, z', \gamma=1, \tau_1\right) + (1-\lambda) * \mathcal{C}\left(N, z, z', \gamma=2, \tau_2\right)$  \\ \midrule
Concatenation Sum&$\lambda * \mathcal{C}\left(N, z_L, z'_L, \gamma=1, \tau_1\right) + (1-\lambda) * \mathcal{C}\left(N, z_R, z'_R, \gamma=2, \tau_2\right)$\\ \midrule
$\gamma = 0.5$ & $\mathcal{C}\left(N, z, z', \gamma=0.5, \tau\right)$          \\ 

\end{tabular}

\caption{Definition of kernel loss functions in our experiments}
\label {table: loss definition}
\end{table}

\textbf{Baselines.} We reproduce the SimCLR algorithm using PyTorch Lightning~\citep{PytorchLightning}.

\textbf{Encoder details.}
The encoder $f$ consists of a backbone network and a projection network. We employ ResNet50~\citep{ResNet} as the backbone and a 2-layer MLP (connected by a batch normalization~\citep{ioffe2015batch} layer and a ReLU \cite{nair2010rectified} layer) with hidden dimensions 2048 and output dimensions 128 (or 256 in the concatenation kernel case).

\textbf{Encoder hyperparameter tuning.}
For each encoder training case, we randomly sample 500 hyperparameter groups (sample details are shown in Table \ref{table: Hyperparameter sample}) and train these samples simultaneously using Ray Tune ~\citep{RayTune}, with the ASHA scheduler~\citep{li2018massively}. Ultimately, the hyperparameter group that maximizes the online validation accuracy (integrated in PyTorch Lightning) within 5000 validation steps is chosen for the given encoder training case.

\begin{table}[ht]
\centering

\begin{tabular}{@{}l|l|l@{}}
\midrule
Hyperparameter  & Sample Range & Sample Strategy \\ \midrule
start learning rate & $\left[10^{-2}, 10\right]$ & log uniform \\ \midrule
$\lambda$       & $\left[0, 1\right]$ & uniform \\ \midrule
$\tau$, $\tau_1$, $\tau_2$ & $\left[0, 1\right]$ & log uniform \\ \midrule
\end{tabular}

\caption{Hyperparameters sample strategy}
\label {table: Hyperparameter sample}
\end{table}

\textbf{Encoder training.} 
We train each encoder using the LARS optimizer~\citep{LARSOptimizer}, LambdaLR Scheduler in PyTorch, momentum 0.9, weight decay $10^{-6}$, batch size 256, and the aforementioned hyperparameters for 400 epochs on a single A-100 GPU.

\textbf{Image transformation.} The image transformation strategy, including augmentation, is identical to the default transformation strategy provided by PyTorch Lightning.

\textbf{Linear evaluation.}
The linear head is trained using the SGD optimizer with a cosine learning rate scheduler, batch size 64, and weight decay $10^{-6}$ for 100 epochs. The learning rate starts at $0.3$ and ends at $0$.

\textbf{Moco Experiments.} We also tested our method based on MoCo~\citep{he2019moco}. The results are summarized in Table \ref{tab:results-moco}. Here we choose ResNet18~\citep{ResNet} as the backbone and set a temperature of $0.1$ as default. For our simple sum kernel, we set $\lambda=0.8$. The results show that our method outperforms the original MoCo method.

\begin{table}[thb]
\centering
\caption{MoCo Experiment Results on CIFAR-10 and CIFAR-100.}
\label{tab:results-moco}
\resizebox{\textwidth}{!}{%
\begin{tabular}{@{}c|ccc|ccc@{}}
\toprule
\multirow{3}{*}{Method} & \multicolumn{3}{c|}{CIFAR-10} & \multicolumn{3}{c}{CIFAR-100} \\ \cmidrule(lr){2-4} \cmidrule(lr){5-7} 
                        & 200 epochs & 400 epochs    & 1000 epochs   & 200 epochs & 400 epochs & 1000 epochs         \\ \midrule
MoCo (repro.)         & $76.41 \pm 0.12$    & $80.01 \pm 0.15$          & $84.45 \pm 0.08$    & $\mathbf{47.02 \pm 0.11}$ & $52.50 \pm 0.07$ & $57.62 \pm 0.15$            \\
\midrule
Laplacian Kernel        & ${78.09 \pm 0.10}$    & $\mathbf{83.85 \pm 0.09}$          & $\mathbf{88.34 \pm 0.16}$    & $46.12 \pm 0.22$   & $53.44 \pm 0.17$ & $59.10 \pm 0.14$        \\
Simple Sum Kernel & $\mathbf{78.12 \pm 0.15}$   & $83.23 \pm 0.18$ & $87.50 \pm 0.20$ & $46.65 \pm 0.06$ & $\mathbf{53.62 \pm 0.19}$ & $\mathbf{59.83 \pm 0.12}$\\
\bottomrule
\end{tabular}
}
\end{table}



\section{More Experiments on Synthetic Data}


Consider a scenario with $n$ clusters, each containing $k$ vertices. Let the probability of vertices $u$ and $v$ from the same cluster belonging to $\bfpi$ be $p$. Conversely, for vertices $u$ and $v$ from different clusters, let the probability of belonging to $\pi$ be $q$. We generate the graph $\bfpi$ randomly, based on $p$ and $q$. We experiment with values of $k=100$ and $n=6$ for ease of visualization, embedding all points in a two-dimensional space. Each vertex's initial position originates from a normal distribution. In each iteration, we sample a subgraph of $\bfpi$ uniformly, ensuring each vertex has an out-degree of $1$. We then optimize the corresponding vectors using InfoNCE loss with an SGD optimizer and iterate until convergence. Our experimental setup consists of an SGD learning rate of $1$, an InfoNCE loss temperature of $0.5$, and a batch size of $50$. We evaluate two scenarios with different $p$ and $q$ values: $p=1$, $q=0$, and $p=0.75$, $q=0.2$. The results of these experiments are visualized in Figure \ref{fig:vis-spectral-cluster}. The obtained embeddings exhibit the hallmark pattern of spectral clustering of graph $\bfpi$.

\begin{figure}[!tb]
\centering
\subfigure{
\includegraphics[width=1\textwidth]{Figures/cluster_pi.png}
\label{fig:vis-cluster}
}
\subfigure{
\includegraphics[width=1\textwidth]{Figures/noised_cluster_pi.png}
\label{fig:vis-noised-cluster}
}
\caption{Visualizations of the optimization process using InfoNCE Loss on the vectors corresponding to $\bfpi$. Points of identical color belong to the same cluster within $\bfpi$. To showcase the internal structure of $\bfpi$, we randomly select 10 vertices from each cluster to display the edge distribution of $\bfpi$.}
\label{fig:vis-spectral-cluster}
\end{figure}



\end{document}

%%% Local Variables:
%%% mode: latex
%%% TeX-master: t
%%% End:
