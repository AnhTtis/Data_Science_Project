    %%%%%%%%%%%%%%%%%%%%%%%%%%%%%%%%%%%%%%%%%%%%%%%%%%%%%%%%%%%%%%%%%%%%%
%%                                                                 %%
%% Please do not use \input{...} to include other tex files.       %%
%% Submit your LaTeX manuscript as one .tex document.              %%
%%                                                                 %%
%% All additional figures and files should be attached             %%
%% separately and not embedded in the \TeX\ document itself.       %%
%%                                                                 %%
%%%%%%%%%%%%%%%%%%%%%%%%%%%%%%%%%%%%%%%%%%%%%%%%%%%%%%%%%%%%%%%%%%%%%

%%\documentclass[referee,sn-basic]{sn-jnl}% referee option is meant for double line spacing

%%=======================================================%%
%% to print line numbers in the margin use lineno option %%
%%=======================================================%%

%%\documentclass[lineno,sn-basic]{sn-jnl}% Basic Springer Nature Reference Style/Chemistry Reference Style

%%======================================================%%
%% to compile with pdflatex/xelatex use pdflatex option %%
%%======================================================%%

%%\documentclass[pdflatex,sn-basic]{sn-jnl}% Basic Springer Nature Reference Style/Chemistry Reference Style

%%\documentclass[sn-basic]{sn-jnl}% Basic Springer Nature Reference Style/Chemistry Reference Style
\documentclass[pdflatex,sn-mathphys]{sn-jnl}% Math and Physical Sciences Reference Style
%%\documentclass[sn-aps]{sn-jnl}% American Physical Society (APS) Reference Style
%%\documentclass[sn-vancouver]{sn-jnl}% Vancouver Reference Style
%%\documentclass[sn-apa]{sn-jnl}% APA Reference Style
%%\documentclass[sn-chicago]{sn-jnl}% Chicago-based Humanities Reference Style
%%\documentclass[sn-standardnature]{sn-jnl}% Standard Nature Portfolio Reference Style
%%\documentclass[default]{sn-jnl}% Default
%%\documentclass[default,iicol]{sn-jnl}% Default with double column layout

%%%% Standard Packages
%%<additional latex packages if required can be included here>
%%%%

%%%%%=============================================================================%%%%
%%%%  Remarks: This template is provided to aid authors with the preparation
%%%%  of original research articles intended for submission to journals published 
%%%%  by Springer Nature. The guidance has been prepared in partnership with 
%%%%  production teams to conform to Springer Nature technical requirements. 
%%%%  Editorial and presentation requirements differ among journal portfolios and 
%%%%  research disciplines. You may find sections in this template are irrelevant 
%%%%  to your work and are empowered to omit any such section if allowed by the 
%%%%  journal you intend to submit to. The submission guidelines and policies 
%%%%  of the journal take precedence. A detailed User Manual is available in the 
%%%%  template package for technical guidance.
%%%%%=============================================================================%%%%

\jyear{2021}%
\usepackage{float}
\usepackage{graphicx,caption}
\usepackage{multirow}
\usepackage{comment}
%% as per the requirement new theorem styles can be included as shown below
\theoremstyle{thmstyleone}%
\newtheorem{theorem}{Theorem}%  meant for continuous numbers
%%\newtheorem{theorem}{Theorem}[section]% meant for sectionwise numbers
%% optional argument [theorem] produces theorem numbering sequence instead of independent numbers for Proposition
\newtheorem{proposition}[theorem]{Proposition}% 
%%\newtheorem{proposition}{Proposition}% to get separate numbers for theorem and proposition etc.
\usepackage[]{xcolor}
%\theoremstyle{thmstyletwo}%
\newtheorem{example}{Example}%
\newtheorem{remark}{Remark}%
\theoremstyle{thmstylethree}%
\newtheorem{definition}{Definition}%


\usepackage{colortbl} %for table row color

\newcommand\wordcount{
    \immediate\write18{texcount -sub=section \jobname.tex  | grep "Section" | sed -e 's/+.*//' | sed -n \thesection p > 'count.txt'}
(\input{count.txt}words)}

\raggedbottom
%%\unnumbered% uncomment this for unnumbered level heads

\usepackage{soul}
\sethlcolor{green}

\begin{document}

%\title[SmartBook]{SmartBook: AI-Assisted Intelligence Analysis Report Generation for Emerging Situations}

\title[SmartBook]{SmartBook: AI-Assisted Situation Report Generation}

%\title[SmartBook]{SmartBook: Situation Report Generation for Emergent Situations with a Case Study on the Ukraine War}
% \heng{add Paul (before Clare?)} 
% \revanth{Do we need to mention the case study part? Also, "Intelligence Report" instead of "Intelligent Report"?}

%\title[SmartBook]{SmartBook: Automatic Intelligence Analyst Report Generation}

%%=============================================================%%
%% Prefix	-> \pfx{Dr}
%% GivenName	-> \fnm{Joergen W.}
%% Particle	-> \spfx{van der} -> surname prefix
%% FamilyName	-> \sur{Ploeg}
%% Suffix	-> \sfx{IV}
%% NatureName	-> \tanm{Poet Laureate} -> Title after name
%% Degrees	-> \dgr{MSc, PhD}
%% \author*[1,2]{\pfx{Dr} \fnm{Joergen W.} \spfx{van der} \sur{Ploeg} \sfx{IV} \tanm{Poet Laureate} 
%%                 \dgr{MSc, PhD}}\email{iauthor@gmail.com}
%%=============================================================%%

\author[1]{\fnm{Revanth} \sur{Gangi Reddy}}\email{revanth3@illinois.com}

\author[1]{\fnm{Yi R.} \sur{Fung}}\email{yifung2@illinois.edu}

\author[1]{\fnm{Qi} \sur{Zeng}}\email{qizeng2@illinois.edu}

\author[1]{\fnm{Manling} \sur{Li}}\email{manling2@illinois.edu}


\author[1]{\fnm{Ziqi} \sur{Wang}}\email{ziqiw9@illinois.edu}


%\author[2]{\fnm{Brad} \sur{Goodman}}\email{bgoodman@mitre.org}
%\author[2]{\fnm{Lisa} \sur{Ferro}}\email{lferro@mitre.org}

\author{\fnm{Paul} \sur{Sullivan}}%\email{} 
%\author[2]{\fnm{Paul} \sur{Sullivan}}%\email{} %paul.sullivan@intpt.net}
%\author[2]{\fnm{Clare} \sur{Voss}}\email{clare.r.voss.civ@army.mil}

\author[1]{\fnm{Heng} \sur{Ji}}\email{hengji@illinois.edu}

\affil[1]{\orgdiv{Department of Computer Science}, \orgname{University of Illinois Urbana-Champaign}, \city{Urbana}, \postcode{61820}, \state{IL}, \country{USA}}

%\affil[2]{\orgname{DARPA}, \orgaddress{\street{Street}, \city{Washington D.C.}, \postcode{20001}, \country{USA}}}

%\affil[2]{\orgname{Army Research Laboratory}, \orgaddress{\street{Powder Mill Road},
% \city{Adelphi, MD},  \postcode{20783}, \country{USA}}}

%\affil[2]{\orgname{MITRE}, \orgaddress{\street{Street}, \city{Washington D.C.},  \postcode{10587}, \country{USA}}}

 % \affil[2]{\orgname{IntelPoint}, \city{Washington D.C.},  \postcode{20006}, \country{USA}}}
 
%\iftrue  % show comments or not
%\NewDocumentCommand{\heng}
%{ mO{} }{\textcolor{red}{\textsuperscript{\textit{Heng}}\textsf{\textbf{\small[#1]}}}}
%\NewDocumentCommand{\revanth}
%{ mO{} }{\textcolor{brown}{\textsuperscript{\textit{Revanth}}\textsf{\textbf{\small[#1]}}}}
%\newcommand{\Yi}[1]{{\color{blue}{\textsuperscript{\textit{Yi}}[\textsf{#1}]}}}
%%{ mO{} }{\textcolor{blue}{\textsuperscript{\textit{Manling}}\textsf{\textbf{\small[#1]}}}}
%\NewDocumentCommand{\ken}
%{ mO{} }{\textcolor{purple}{\textsuperscript{\textit{Ken}}\textsf{\textbf{\small[#1]}}}}
%\NewDocumentCommand{\vicki}
%{ mO{} }{\textcolor{purple}%{\textsuperscript{\textit{Vicki}}\textsf{\textbf{\small[#1]}}}}
%\newcommand{\ziqi}[1]{{\color{orange}{\textsuperscript{\textit{Ziqi}}[\textsf{#1}]}}}
%\else
%\newcommand{\heng}[1]{}
%\newcommand{\Yi}[1]{}

%\fi 

%%==================================%%
%% sample for unstructured abstract %%
%%==================================%%
\abstract{Emerging events, such as the COVID pandemic and the Ukraine Crisis, require a time-sensitive comprehensive understanding of the situation to allow for appropriate decision-making and effective action response. Automated generation of situation reports can significantly reduce the time, effort, and cost for domain experts when preparing their official human-curated reports. However, AI research toward this goal has been very limited, and no successful trials have yet been conducted to automate such report generation. %Since
Pre-existing natural language processing methods, large language model based text generation, and information retrieval techniques are insufficient to identify, locate, and summarize important information,
% for generating good-quality intelligence reports, 
and lack detailed, structured, and strategic awareness. 
We propose \textbf{SmartBook}, a novel task formulation targeting situation report generation, which consumes large volumes of news data to produce a structured situation report with multiple hypotheses (claims) summarized and grounded with rich links to factual evidence. %\heng{elaborate a little more on why our framework is innovative and better, there are other reasons: our data sources are in multilingual multimedia, we construct claim-claim networks, each claim is associated with knowledge elements.}
We realize SmartBook for the Ukraine-Russia crisis by automatically generating intelligence analysis reports to assist expert analysts. The machine-generated reports are structured in the form of timelines, with each timeline organized by major events (or chapters), corresponding strategic questions (or sections) and their grounded summaries (or section content).
%\heng{I remember Brad only deleted 2\% tokens, which means the information in our output is highly accurate. Maybe we should mention that in the result summary.}
% performs claim summarization and event prediction,  based on expert-provided or automatically generated topic queries. 
%We further develop evaluation metrics to benchmark the situation report generation task, including human expert edit distance and time efficiency comparisons. 
 Our proposed framework automatically detects real-time event-related strategic questions, which are more directed than manually-crafted analyst questions, which tend to be too complex, hard to parse, vague and high-level. 
Results from thorough qualitative evaluations show that %intelligence analysts find 
roughly 82\% of the questions in Smartbook have strategic importance, with at least 93\% of the sections in the report being tactically useful.
%Further, SmartBook outperforms a web search $+$ large language model based generation approach by up to 9\% in terms of relevance and strategicness of information presented. 
Further, experiments show that expert analysts tend to add more information into the SmartBook reports, with only 2.3\% of the existing tokens being deleted, meaning SmartBook can serve as a useful foundation for analysts to build upon when creating intelligence reports.}\footnote{We deeply regret that one of our project contributors, Paul, passed away before the publication of this paper. His contributions to this research were invaluable, and the loss is felt deeply by all of us who had the privilege of working with him. We honor the memory and express our gratitude for his enduring commitment to the pursuit of knowledge.}
%Results show that the token overlap between generated and post-edited summaries is high, with a BLEU score of 59.0\% and a Rouge-2 score of 74.1\%.
%\heng{TODO: update results summary}

%%================================%%
%% Sample for structured abstract %%
%%================================%%

% \abstract{\textbf{Purpose:} The abstract serves both as a general introduction to the topic and as a brief, non-technical summary of the main results and their implications. The abstract must not include subheadings (unless expressly permitted in the journal's Instructions to Authors), equations or citations. As a guide the abstract should not exceed 200 words. Most journals do not set a hard limit however authors are advised to check the author instructions for the journal they are submitting to.
% 


\keywords{Complex Event Understanding, Knowledge Acquisition, Natural Language Generation, Intelligence Analysis Research \& Development Toolkit}

%%\pacs[JEL Classification]{D8, H51}

%%\pacs[MSC Classification]{35A01, 65L10, 65L12, 65L20, 65L70}

\maketitle

\section{Introduction} \label{sec:intro}

Large amounts of time and effort are devoted to
verification and validation of every microprocessor design project.
Broadly, design verification can be broken into two large categories:
(1) functional and (2) performance verification, which is to identify design bugs that degrade performance without affecting functionality. Performance bugs are different from performance bottleneck as the former is due to design mistakes while the later is caused by tight resource constraints. Performance loss due to performance bugs  can 
be very significant, with recent reported cases shown to be
$>10\%$~\cite{mccalpin2018hpl}. This demonstrates a critical 
need for automated mechanisms for performance debugging.  As 
recent designs from Intel~\cite{corei7-11}, AMD~\cite{ryzen-9},
ARM~\cite{cortex-a}, and others place an even greater emphasis
on core performance, design complexity has scaled
dramatically, likewise scaling the difficulty in all forms of
verification.


%Functional verification has received extensive attention from researchers and, although complex, it benefits from the availability of known correct outputs that can be used to compare against.

Performance verification at microarchitecture level ensures that a
design correctly achieves expected performance in terms of execution
time or cycle count.  The main challenge in this task is that, unlike
functional verification, there is no exact golden reference to compare
against.  This is because of the high difficulty of modeling all the
interactions between the different units in complex microprocessor
designs, and accurately represent how they affect the overall system
performance.  %This task also suffers from 
%the lack of a good debugging infrastructure, as well as from 
%limited visibility into intermediate points in the design, which are mostly exposed through performance counters. Although useful for estimating the performance of the system, these counters are very difficult to use for manual debugging because of their complex relationship with processor performance and due to the large amounts of data they generate.  
Traditionally, performance
verification is conducted mostly through manual techniques which rely
on rough estimations of performance gain expected by
microarchitectural changes~\cite{Singhal2004}. Such manual processes
are not only very lengthy but also error-prone.



The process of performance verification and debugging roughly consists of two steps: (1)~detection, which determines whether a
design achieves expected performance or not, and (2)~localization,
which identifies the microarchitectural units causing the performance
issues and is the focus of this work.

There are few previous studies on automating detection of
microprocessor performance bugs~\cite{Bose1994,
  surya1994architectural,carvajal2021detection}. 
The majority of those~\cite{Bose1994, surya1994architectural} relies on capturing
design intentions using a bespoke performance model as a golden
reference, this  entails long development time and may contain
errors by itself. Recently, a data driven and machine learning
(ML)-based approach~\cite{carvajal2021detection} was developed for
automatic performance bug detection with high accuracy. Although
significant, these works do not solve the 
problem of performance bug localization.
%pressing problem of identifying where the performance bug is.

Works in automating microprocessor performance bug localization
are even scarcer.  Adir \emph{et al.}~\cite{adir2005generic}
propose perhaps the only partially related work of which we are
aware.  Their work focuses on formal planning of test program
generation for individual units, such as issue queues. This strategy
follows conventional functional verification, involving heavy
manual effort, costing significant engineer-time to develop a test
plan, and as much as ten days of computer runtime per functional
unit. To the best of our knowledge, there has been no systematic study
on automatic performance bug localization for microarchitecture
designs.

Performance bug localization is a complicated task, which is currently
solved using mostly manual techniques.
Even
in the more widely studied area of functional validation, the industry
lacks a standardized mechanism to automate bug localization, it has
been only recently that academic efforts have attempted to automate
this task~\cite{BugMD}. Considering this, it is important to note that
any type of design automation which successfully reduces the 
time and effort required by engineers to debug their designs is highly
valuable. Since automatic performance debug for microprocessors is
a huge yet under-studied challenge, it is very difficult, if not impossible, to find a perfect solution in a single work. Although our work is not perfect, it serves a key stepping stone 
 toward solving the problem.

This work tackles the performance bug localization problem by
using ML to generate a ranked list of most likely mi\-cro\-ar\-chi\-tec\-tur\-al units that 
might contain the bug.  This list may be used
to prioritize the debugging order, as well as to identify
teams with the right expertise to perform further debug. Two different methodologies are
proposed, evaluated, and contrasted. These data-driven
techniques achieve high
accuracy, while being fully automated. Further, they
consider intra- and inter-unit interactions, as opposed to other
techniques proposed in the partially related previous work~\cite{adir2005generic} which
considered only intra-unit behavior.

%Our methods are based on ML, wherein our models are
%trained using data from legacy designs.
%To take the full advantage of
%these approaches, we assume that architectural changes in a new design
%are incremental when compared to its previous
%generations. Examining recent processors from major vendors including
%Intel, ARM, and AMD, we find this assumption holds true, since the generational change in microarchitectures
%is largely incremental. Thus, the methodologies proposed here provide
%alue for a multitude of upcoming designs.  However, even when
%disruptive changes occur, the methodologies can still be beneficial for bug localization on structures that conform to previous microarchitectures, using workloads that
%do not exercise new functionalities. Further, as general purpose microarchitectures become ever more mature, and the inter-generational performance gains decrease, 
%it is even more important to retain as much performance as possible, making performance debugging ever more important.

The major contributions of this work include the following:
\begin{compactitem}
\itemsep0em 
\item This is the first systematic study on fully automatic
  performance bug localization for microarchitecture designs, to the
  best of our knowledge.

\item Two ML-based approaches to tackle performance bug
  localization, as well as a hybrid of both, are evaluated
  and contrasted.

\item For bugs with an average IPC impact greater than 1\%, our best
  performing methodology identifies the correct bug location as the
  most likely unit in $\sim77\%$ of the cases, and achieves over 90\%
  accuracy when the three most likely options (out of 11 possible) are
  considered.  

\item One of the proposed methodologies is not only very accurate localizing
performance bugs, but it can also be applied to confirm the results
of performance bug detection with high accuracy.

\item Although the focus of this work is on microprocessor core,
we evaluated our methodologies on the processor memory hierarchy. This evaluation
uses a different experimental setup, showing the robustness of the proposed techniques.

\end{compactitem}

As an early work on performance bug localization, the design of this study is subject to potential limitations, however, we feel it still represents a good first step towards solving the problem. The scope of our work and its limitations are as follows:
\begin{compactitem}

\item Legacy
designs with identified performance bugs are required, so that the ML
models can be trained. Bug-free legacy designs are required only 
in one of the methodologies, yet, if available, the other can take advantage of the additional data.
However, thanks to the thorough pre- and post-silicon
debug to which the designs are submitted, these legacy designs are
generally available.

\item We assume that only one bug is present at a time, 
in parallel to the single-fault model which is common practice
in VLSI testing works. As explained in Section~\ref{subsec:impl_bugs}, we still expect 
our methodologies to work well in the presence of multiple bugs in a single design.

\item Our methodologies do not provide a quantitative coverage guarantee.  
In general, performance bug
coverage is extremely difficult to define and is a potential
research topic on its own. We know of no prior work which presents a
definition of such coverage. Nonetheless, the evaluated bugs are based on published errata, cover a large amount of microarchitectural units and affect the system in a variety of ways. Thus, we feel these bugs represent a reasonable start for early work in this area.

\item We assume that there are no dramatic structural
microarchitectural changes between the legacy designs and the
designs under debug. Examining recent processors from major vendors, including Intel, ARM, and AMD, we find this assumption holds true, since the generational change in microarchitectures
is largely incremental. That said, even when larger shifts occur, the
methodologies can be partially reused. For example, consider the
introduction of the AVX instructions with Intel's Sandy Bridge
architecture in 2011.  Initially there would be no available data to
test these instructions using our methodologies, however the rest of
the Sandy Bridge design could be debugged with our methodology,
leveraging workloads that do not exercise the new instructions.  In
future implementations, data from Sandy Bridge can be used to build
the models required to use our methods for debugging AVX. 
%Further, as general purpose microarchitectures become even more mature, and the inter-generational performance gains decrease, it is even more important to retain as much performance as possible, making performance debugging even more important.}

\item We limit our evaluation to a pre-silicon setup, because
it is infeasible for us to inject known design bugs in silicon to
evaluate the methodologies.  Further, should our methodologies be
applied to a commercially available design, and an actual bug be
found and localized, we would not be able to verify that such
localization is correct without prior knowledge of its existence so
as to verify our findings. However, our methodologies can be applied in both pre- and post-silicon scenarios. During pre-silicon stages fixing performance 
bugs is easier and cheaper, 
the availability of performance counters is greater (due to the usage of a
simulator) and the counters can be sampled at a much faster rate. 
By using only counters available in-silicon, and adjusting the sampling frequency, we could use the proposed
methodologies during post-silicon stages. In post-silicon analysis the methodology
could be applied to longer workloads, providing a way to exercise complicated bugs that
are not possible to trigger with short pre-silicon traces.
%Further, we can follow hybrid approaches where the ML model training is performed using simulations, and the techniques are applied to data obtained from microprocessors during post-silicon debug. }

\end{compactitem}

Despite the aforementioned, we present a first, useful, yet attainable,
step towards the goal of performance bug localization, and we hope this work can draw the attention of the research community
to the broader performance validation domain.

\iffalse{
In Section~\ref{sec:scope} we describe the problem
formulation and outline the scope of this work.  We note that, to
date, very little work exists in automating performance bug
localization. 

Section~\ref{sec:methodology} 
describes the approaches developed to tackle the performance bug
localization task.  Section~\ref{sec:experimental_setup} provides
details of the architectures, and performance bugs used for
evaluation. Section~\ref{sec:evaluation} presents results obtained in
several experiments developed to evaluate the methodologies. A brief
review of previous work related to performance debugging is presented
in Section~\ref{sec:related_work}.  And finally,
Section~\ref{sec:conclusion} concludes the paper.
}
\fi

\section{Related Work}
\label{sec:related}

\noindent \textbf{Action Recognition.}
%
Most recent approaches for action recognition are to exploit appearance and motion cues jointly and achieve remarkable success~\cite{feichtenhofer2019slowfast,i3d,lin2019tsm,huang2021tada,qing2022learning,wang2021oadtr,pei2022learning}.
%
Typically, two-stream networks~\cite{two-stream,two-stream-2,TSN} consist of two branches that explore spatial information and temporal dynamics, respectively.
%
% To overcome the limitation of 2D CNN in modeling long-range dependencies, some attempts~\cite{lin2019tsm,r2+1d,TDN} begin to introduce additional temporal mining operations.
Some attempts~\cite{lin2019tsm,r2+1d,TDN} introduce additional temporal mining operations to overcome the limited temporal information extraction ability of 2D CNN.
%
3D CNN-based methods~\cite{feichtenhofer2019slowfast, i3d,C3D} inflated 2D kernels for joint spatio-temporal modeling.
%
\cite{bai2020prototype} proposes the prototype similarity learning which pushes the learned representation to the corresponding prototype as close as possible, while our PSL keeps the differences among the same class.

\noindent \textbf{Open-set Action Recognition.} The related work of OSAR is limited~\cite{krishnan2018bar,shu2018odn,yang2019open,bao2021evidential}. Recently, \cite{bao2021evidential} systematically studies the OSAR problem and transfers several open-set image recognition methods to the video domain, including SoftMax~\cite{hendrycks2016baseline}, MC Dropout~\cite{gal2016dropout}, OpenMax~\cite{bendale2016towards}, and RPL~\cite{chen2020learning}. In the benchmark of~\cite{bao2021evidential}, the only two methods designed specifically for the video domain are BNN SVI~\cite{krishnan2018bar} and their proposed DEAR. BNN SVI is a Bayesian NN application in the OSAR, while DEAR adopts the deep evidential learning~\cite{amini2020deep} to calculate the uncertainty, and utilizes two modules to alleviate the over-confidence prediction and appearance bias problem, respectively. Existing methods pursue better uncertainty scores, while the objective of our PSL is to learn more diverse feature representations for better open-set distinguishability.

\noindent \textbf{Information Bottleneck Theory.} Based on the IB theory~\cite{tishby2000information,tishby2015deep}, the NN intends to extract minimum sufficient information of the inputs for the current task. More recent~\cite{tian2020makes,federici2020learning,wang2022rethinking} adopt the IB theory on unsupervised contrastive learning to analyze the representation learning behavior under the corresponding tasks. In this work, we provide a new view to analyze the OSAR problem based on the IB theory.

\section{Method}
Our method, {\moniker}, extends the volume rendering equation to accurately reconstruct the geometry and appearance robust to hazy conditions.
Our key idea is to introduce a series of important biases in the network architecture along with regularizers in the loss function that together underpin physically based scattering phenomena.

\subsection{Preliminary on Neural Radiance Fields}\label{sec:nerf}
Neural Radiance Fields (NeRFs)~\cite{mildenhall2020nerf} map a 3D sample point \(\p\) into a color $\mathbf{c}$ and volume density $\sigma$.
Considering only emission from classic volume rendering~\cite{kajiya1984ray,tagliasacchi2022volume}, the expected color ${C}(\r)$ of a camera ray $\r(t)=\mathbf{o} + t\mathbf{d}$ with the near and far boundary $t_n$ and $t_f$ can be written as
\begin{gather}
	{C}(\r, \mathbf{d})=\int_{t_n}^{t_f}T(t)\sigma(\r(t))c(\r(t), \mathbf{d}) \ dt \;\textrm{with} \label{eq:nerf}\\
    T(t)=\mathrm{exp}\left( - \int_{t_n}^{t}\sigma(\r(t')) \ dt'\right),
	\label{eq:occlusion}
\end{gather}
where \(T(t)\) is the accumulated transmittance between the ray section \(t_{n}\) to \(t \).
The predicted pixel value is then compared to the ground truth $\widehat{C}(\r,\d)$ for optimization.

\subsection{3D Haze Formation}\label{sec:rte_haze}
To address the 3D dehazing problem, we propose an alternative rendering equation to the image formation model.
We start from the radiative transfer equation (RTE)~\cite{chandrasekhar2013radiative,van1999multiple}, which describes the behavior of light in a medium that absorbs, scatters and emits radiation.
Assuming, a ray \(\r\left( t \right) = \mathbf{o} + t\d\) hits a surface point at \(\r\left( t_{0} \right)\), the incident radiance at the near image plane \(t_{n}\) can be divided into three parts~\cite{pharr2016physically}:
{\small
\begin{align}
C(\r, \d) &= C_{\textrm{emission}}(\r) + C_{\textrm{surface}}(\r) + C_{\textrm{in-scattering}}(\r)\nonumber\\
C_{\textrm{emission}}(\r, \d) &=
\int_{t_{n}}^{t_{0}}\epsilon\left(\r\left( t\right),\d\right)T_{\sigma_{t}}\left( t\right)dt\nonumber\\
C_{\textrm{surface}}(\r, \d) & =C_e\left(\r\left( t_{0} \right),\d\right)T_{\sigma_{t}}\left( t_{0}\right)\nonumber\\
C_{\textrm{in-scattering}}(\r, \d) &=
\int_{t_{n}}^{t_{0}}c_{\textrm{s}}\left( \r\left( t \right), \d \right)\sigma_{s}\left(\r\left( t \right)\right)T_{\sigma_{t}}\left( t \right)dt,\nonumber
\end{align}
}
where \(\epsilon\) is the emission, \(C_{e}\) is the outgoing radiance at the surface intersection, \(c_{\textrm{s}}\left(\r\left( t \right), \d \right)\) is the in-scattered light and \(\sigma_{s}\) is the scattering coefficient.
In particular, the transmittance here is computed from the attenuation coefficient \(\sigma_{t}\), \ie,
\(T_{\sigma_{t}}\left( t\right)=\exp\left( -\int_{t_{n}}^{t}\sigma_{t}(t')dt' \right)\),
where \(\sigma_{t}=\sigma_{a} + \sigma_{s}\) including the absorption and out-scattering effect.
For common haze formation, the participating particles are considered non-luminous~\cite{narasimhan2003contrast}, therefore we can drop the emission part, which leads to
{
\small
\begin{align}
\begin{split}
C(\r,\d)= {} & C_e(\r\left( t_{0} \right),\d)T_{\sigma_{t}}\left( t_{0} \right)+\\
&\int_{t_{n}}^{t_{0}}c_{\textrm{s}}\left( \r\left( t \right), \d \right)\sigma_{s}\left(\r\left( t \right)\right)T_{\sigma_{t}}\left( t \right)dt.
\end{split}
\label{eq:RTE_Haze}
\end{align}
}

Following NeRF~\cite{mildenhall2020nerf}, we represent the surface as a continuous density field with emission \(\epsilon\left(\r\left(t\right), \d\right)\coloneqq c\left(\r\left( t \right),\d\right)\sigma\left(\r\left( t \right)\right)\).
Meanwhile, the absorption part in the attenuation \(\sigma_{t}\) can be interpreted as the surface density \(\sigma\), since the volume density $\sigma$ is equal to absorption coefficient $\sigma_{a}$ in that they both determine the probability of a photon or a ray terminating at a given location.
As a result, we can write the rendering equation as
{\small
\begin{align}\begin{split}
    C(\r,\d)=&
    \underbrace{\int_{t_{n}}^{t_{0}}c(\r(t),\d)\sigma(t)T_{\sigma+\sigma_{s}}\left( t \right)dt}_{C_{\textrm{Surface}}} +\\
    &\underbrace{\int_{t_{n}}^{t_{0}}c_{s}(\r(t))\sigma_{s}(t)T_{\sigma+\sigma_{s}}\left( t \right)dt}_{C_{\textrm{Haze}}}.
    \label{eq:3D_haze_formation}
\end{split}
\end{align}
}
\cref{eq:3D_haze_formation} formally disentangles the surface and haze, represented by \(\left\{ c, \sigma \right\}\) and \(\left\{  c_{s}, \sigma_{s} \right\}\) respectively, in a principled manner.
Once successfully optimized (see the next Section), the clear-view surfaces can be recovered using \(\left\{ c, \sigma \right\}\):
\begin{equation}
C(\r,\d)=
\int_{t_{n}}^{t_{0}}c(\r(t),\d)\sigma(t)T_{\sigma}\left( t \right)dt\label{eq:clear_view}.
\end{equation}

\begin{figure}[t!]
\centering \includegraphics[width=\linewidth]{images/architecture.pdf}
\makeatother
\caption{\textbf{\moniker{} architecture.} Given a set of hazy images, our method augments the existing NeRF pipeline (gray) with a haze module (yellow), which explicitly models the scattering phenomenon using atmospheric light and scattering coefficient. During training, we render the hazy reconstruction as a composition of surface and haze, which is compared to the input hazy images to optimize the learnable parameters (in green) jointly. During inference, we use the surface module (gray) to render clear views.}
\vspace{-0.5cm}\label{fig:architecture}
\end{figure}

\subsection{Haze-aware Neural Radiance Field}\label{sec:dehaze_nerf}
Given multiple images of a hazy scene, we aim to jointly optimize for the surface appearance and geometry, \(\left\{ c, \sigma \right\}\) as well as the haze's scattering coefficient and in-scattered light (atmospheric light),  \(\left\{c_{s}, \sigma_{s} \right\}\) based on the enhanced scattering-aware rendering equation~\cref{eq:3D_haze_formation}.
However, the effects of these variables are interdependent. In order to correctly disentangle them, our model adopts suitable architecture designs and training regularizers to capture the distinct physical properties of haze and surface.
An overview of \moniker{} is illustrated in \cref{fig:architecture}.

\paragraph{Architecture.} Now we introduce inductive biases to match the physical properties of haze and surface.
For clarity, we highlight the quantities directly modeled by neural networks in \nn{green}.

\emph{Modeling a Surface.} Recall our goal is to learn the surface appearance and geometry, \(\left\{ c, \sigma \right\}\).
Similar to previous works~\cite{mildenhall2020nerf}, we model the appearance \(\cnet\left( \p, \d \right)\) with an MLP, which takes the sample location \(\p\) and viewing direction \(\d\) as inputs.
However, in order to encourage volume density \(\sigma\) to form a well-defined solid surface, instead of directly learning the volume density, we adopt the reparameterization of the volume density using signed distance function (SDF), \(\sdf\left( \r\left( t \right) \right)\in \R\), as proposed in NeuS~\cite{wang2021neus,wang2022hfs}.
The modified surface volume density \(\sigma\left( \r\left( t \right) \right) \), referred to as opaque density, can be parameterized as \(\sdf\left( \r\left( t \right) \right)\):
\begin{equation}
\sigma\left( \r\left( t \right) \right) = s\left( \Phi_{s}\left( \sdf\left( \r\left( t \right) \right)\right) -1 \right)\nabla \sdf\left( \r\left( t \right) \right)\mathbf{d},\label{eq:hfneus-sigma}
\end{equation}
where $\Phi_{s}(x)$ is the sigmoid function $\Phi_s(x) = (1 + e^{-sx})^{-1}$, whose derivative is a bell-shaped density function centered at 0 and has a learnable standard deviation of \(\nicefrac{1}{s}\).
We derive the discrete approximate following~\cite{mildenhall2020nerf,tagliasacchi2022volume}.
It samples $n$ points $\left\{ \p_{i}=\mathbf{o}+t_n\mathbf{d}|n=1,...,N,t_n<t_{n+1} \right\}$ along the ray.
The approximate pixel color of the ray is computed based on quadrature rule~\cite{max1995optical}, yielding
\begin{align}\begin{gathered}
C_{\textrm{surface}}(\r,\d) = \sum_{n=1}^{N}\frac{\sigma^{n}}{\sigma_{t}^{n}} T_{t}^{n}\alpha_{t}^{n}\nn{c}^{n} \textrm{ with } T_{t}^{n}=\prod_{m=1}^{n-1}\left(1 - \alpha_{t}^{m}\right) \label{eq:C_surface},
\end{gathered}\end{align}
where \(\alpha_{t}\) denotes the discrete \(\alpha\)-compositional weight defined as~\cite{wang2021neus,wang2022hfs}
\begin{equation}
 \resizebox{1\hsize}{!}{
 $
    \alpha_{t}^{n}=\textsc{clamp}\left( 1-\exp\left( -\sigma_{t}^{n}\delta^{n} \right),0, 1 \right) \textrm{ with } \delta^{n}=t^{n+1}-t^{n}\label{eq:alpha}\nonumber,$}
\end{equation}
where \(\sigma_{t}^{n}=\sigma^{n}+\nn{\sigma_{s}}^{n}\) denotes the total attenuation at sample \(n\), including the attenuation due to surface occlusion and the out-scattering.

\emph{Modeling Haze.} We use a low-frequency prior to compute the scattering coefficient and atmospheric light, \(\left\{c_{s}, \sigma_{s} \right\}\), since these components usually vary slowly in a common hazy scenes~\cite{li2015simultaneous}.
In practice, we use a small band-limited \textsc{MLP}~\cite{lindell2022bacon} for the scattering coefficient \(\sigma_{s}\) to capture inhomogenous haze.
Analogous to \cref{eq:C_surface}, the haze color can be approximated as
% \begin{equation}
% \begin{gathered}
% C_{\textrm{haze}}(\r) = \sum_{i=1}^{n}\frac{\nn{\sigma_{s}}^{n}}{\sigma_{t}^{n}} T_{t}^{n}\alpha_{t}^{n}\nn{c_{s}}^{n}.\label{eq:C_haze}
% \end{gathered}
% \end{equation}
\begin{equation}
\begin{gathered}
C_{\textrm{haze}}(\r) = \sum_{n=1}^{N}\frac{\nn{\sigma_{s}}^{n}}{\sigma_{t}^{n}} T_{t}^{n}\alpha_{t}^{n}\nn{c_{s}}^{n}.\label{eq:C_haze}
\end{gathered}
\end{equation}
During optimization, the color for an arbitrary input hazy image can be written as $C = C_{\textrm{surface}} + C_{\textrm{haze}}$.
At test time, we can reconstruct the clear-view color by discretizing \cref{eq:clear_view}, namely:
\begin{gather}
 C_{\textrm{clear}}\left( \r,\d \right) = \sum_{n=1}^{N}T_{\sigma}^{n}\alpha^{n} \nn{c}^{n}, \label{eq:clear_view_discrete}\\
 \resizebox{1\hsize}{!}{
 $T_{\sigma}^{n} = \prod_{j=1}^{n-1}\left(1 - \alpha^{j}\right)\, \textrm{and }\, \alpha^{n} = \textsc{clamp}\left( 1 - \exp\left( -\sigma^{n}\delta^{n} \right),0, 1 \right).\nonumber$}
\end{gather}
\paragraph{Optimization.} While the inductive biases separate the high-frequency surface appearance and geometry from the low-frequency color and density of the scattering medium, we introduce further regularizers to guide the optimization process to converge to more plausible clear-view geometry and color.

\emph{Koschmieder Consistency.}
Given an accurate depth map \(D\), assuming globally constant scattering coefficient \(\bar{\sigma}_{s}\) and airlight \(\bar{c}_{s}\), the relation between a clear-view image \(C_{\textrm{clear}}\) and the hazy image \(C\) can be described by the Koschmieder law~\cite{israel1959koschmieders} as
\begin{equation}
\resizebox{0.88\hsize}{!}{
\(C(\r)=C_{\textrm{clear}}(\r)\exp(-\bar{\sigma}_{s} D(\r))+\bar{c}_{s}(1-\exp(-\bar{\sigma}_{s} D(\r)))\).
}\label{eq:koschmieder}
\end{equation}
This model is widely adopted as the basis for image-based single and multiview dehazing.
The Koschmider model is an approximation of our rendering equation~\cref{eq:3D_haze_formation} under the assumption of
spatially-invariant (i.e., homogeneous) scattering coefficient and an ideal surface
\begin{align}
C_{\textrm{surface}}\left( \r \right) & \approx C_{\textrm{clear}}(\r)\exp(-\bar{\sigma}_{s} D(\r)) = \tilde{C}_{\textrm{surface}}\left( \r \right)\\
C_{\textrm{haze}}\left( \r \right) & \approx \bar{c}_{s}(1-\exp(-\bar{\sigma}_{s} D(\r)) = \tilde{C}_{\textrm{haze}}\left( \r \right),
\end{align}

We promote this relation with
%
\begin{align}
&\loss_{\textrm{2D}} = \left\|C_{\textrm{surface}}\left( \r \right) -  \tilde{C}_{\textrm{surface}}\left( \r \right)\right\|_{1} \\+
&\left\| C_{\textrm{haze}}\left( \r \right)\! - \!\tilde{C}_{\textrm{haze}}\left( \r \right)\right\|_{1} \!\!+\!
 \left\| C\! -\! \tilde{C}_{\textrm{surface}}\left( \r \right)\! -\! \tilde{C}_{\textrm{haze}}\left( \r \right)\right\|_{1}\!, \nonumber
\end{align}
%
where \(\bar{\sigma}_{s}\) and \(\bar{c}_{s}\) are the average over the samples on the ray, while
the depth value \(D\left( \r \right)\) is computed via the learned surface geometry~\cite{mildenhall2020nerf,yu2022monosdf} by accumulating over ray-length over all the samples on a ray:
\begin{equation}
    D\left( \r \right) = \sum_{n=1}^{N} T_{\sigma}^{n}\alpha^{n}t^{n}.
\end{equation}

\emph{Color Prior.}
Without knowing the original image, the heavily attenuated color in the hazy image can be explained by the haze but also by a dull surface color.
In order to reconstruct plausible clear-view colors, we adopt the popular 2D prior widely used in image-based dehazing methods, Dark Channel Prior (DCP)~\cite{he2010single}, which arises from the observation, that for most pixels in a natural haze-free image, the minimum of three color channels is close to zero.
We apply this prior to the estimated clear image \(C_{\textrm{clear}}\)
\begin{align}
DC(C_{\textrm{clear}})\left(\x\right)&=\underset{\y\in\Omega\left(\x\right)}{\min}\left(\underset{c\in\left\{r,g,b\right\}}{\min}C_\textrm{clear}^{c}\left(\y\right)\right),
\label{eq:DCP_definition}\\
\loss_{\textrm{dcp}}&=\frac{1}{K}\sum\limits_{k=1}^{K}\Vert DC\left(C_{\textrm{clear}}\right)\Vert_{1}.
\label{eq:loss_dcp}
\end{align}


\subsection{Implementation Details}
We adopt the same setting as that in HF-NeuS~\cite{wang2021neus} wherever possible.
This includes the MLPs for the surface SDF, \(\sdf\) and the view-dependent surface color, \(\cnet\), as well as the sampling strategy, the background composition, and learning rate schedule.

\paragraph{Loss.}
Our loss is composed of several terms:
\begin{equation}
    \loss = \loss_{\textrm{color}} + \lambda\loss_{\textrm{eikonal}} + \alpha\loss_{\textrm{dcp}} + \beta\loss_{\textrm{2D}},\label{eq:total_loss}
\end{equation}
where \(\loss_{\textrm{dcp}}\) and \(\loss_{\textrm{2D}}\) are the regularizations introduced in \cref{sec:dehaze_nerf},
while the photo-consistency loss, $\loss_{\textrm{color}}$, is the standard NeRF loss, and the eikonal loss, \(\loss_{\textrm{eikonal}}\), is commonly used to regularize SDF~\cite{gropp2020implicit},
\begin{align}
    \loss_{\textrm{color}}& = \frac{1}{K}\sum_{k=1}^{K}\left\|\widehat{C}_{k}(\r,\d) - C_{k}(\r,\d)\right\|_{1},\\
    \loss_{\textrm{eikonal}} &= \frac{1}{KN}\sum_{k}^{K}\sum_{n}^{N}(\|\nabla f({\mathbf{r}}_{k}(t_n))\|_2 - 1)^2,
\label{eq:loss_color}
\end{align}
where $\widehat{C}_{k}(\r,\d)$ is the pixel color. $N$ and $K$ denote the total sampling points on a ray and the total number of rays sampled per training batch.

Finally, because of the surface representation using SDF, we can optionally adopt the object masks for supervision~\cite{yariv2021volume,wang2021neus,wang2022hfs}.
Specifically, given the object mask, \(M\), the mask loss $\loss_{\textrm{mask}}$ for a sampled ray $k$ is defined as
\begin{equation}
    \loss_{\textrm{mask}} = \text{BCE}(M_k, \hat{O}_k),\label{eq:mask_loss}
\end{equation}
where $\hat{O}_k = \sum_{i=1}^{N}T_{\sigma}^{i}\alpha^{i}$ is the total weight for the clear-view surface color along the camera ray, and $\text{BCE}$ is the binary cross entropy loss.



\begin{table*}
\begin{center}
\caption{Comparison with \sota\ methods on the public crowd analysis benchmarks: \jhu, ShanghaiTech, UCF, and \nwpu. 
The best results are shown in \first{red}. The second-best results are shown in \second{blue}. 
}
\vspace{\tablegap}
\resizebox{0.95\textwidth}{!}{
\begin{tabular}{l c c c c c c c c c c c c c}
\toprule
 \multirow{2}{*}{Method} & \multirow{2}{*}{Venue} &\multicolumn{2}{c}{\jhu} &\multicolumn{2}{c}{\shha} &\multicolumn{2}{c}{\shhb} &\multicolumn{2}{c}{\ucf} &\multicolumn{2}{c}{\qnrf} &\multicolumn{2}{c}{\nwpu}\\[0.2ex]
 \cmidrule(lr){3-4}\cmidrule(lr){5-6}\cmidrule(lr){7-8}\cmidrule(lr){9-10}\cmidrule(lr){11-12}\cmidrule(lr){13-14}
& & MAE$\downarrow$ & MSE$\downarrow$ & MAE$\downarrow$ & MSE$\downarrow$ & MAE$\downarrow$ & MSE$\downarrow$ & MAE$\downarrow$ & MSE$\downarrow$ & MAE$\downarrow$ & MSE$\downarrow$ & MAE$\downarrow$ & MSE$\downarrow$\\[0.2ex]
\midrule\midrule
TopoCount \cite{abousamra2021localization}	& AAAI'21	& {60.9}	& {267.4}	& {61.2}	& {104.6}	& {7.8}	& {13.7}	& {184.1}	& {258.3}	& {89.0}	& {159.0}	& {107.8}	& {438.5}	\\[0.2ex]
SUA \cite{meng2021spatial}	& ICCV'21	& {80.7}	& {290.8}	& {68.5}	& {121.9}	& {14.1}	& {20.6}	& {-}	& {-}	& {130.3}	& {226.3}	& {111.7}	& {443.2}	\\[0.2ex]
ChfL \cite{shu2022crowd}	& CVPR'22	& {57.0}	& {235.7}	& {57.5}	& {94.3}	& {6.9}	& {11.0}	& {-}	& {-}	& {80.3}	& {137.6}	& {76.8}	& {343.0}	\\[0.2ex]
MAN \cite{lin2022boosting}	& CVPR'22	& {53.4}	& \second{209.9}	& {56.8}	& {90.3}	& {-}	& {-}	& {-}	& {-}	& {77.3}	& {131.5}	& {76.5}	& {323.0}	\\[0.2ex]
GauNet \cite{cheng2022rethinking}	& CVPR'22	& {58.2}	& {245.1}	& {54.8}	& {89.1}	& {6.2}	& {9.9}	& {186.3}	& {256.5}	& {81.6}	& {153.7}	& {-}	& {-}	\\[0.2ex]
CLTR \cite{liang2022end}	& ECCV'22	& {59.5}	& {240.6}	& {56.9}	& {95.2}	& {6.5}	& {10.6}	& {-}	& {-}	& {85.8}	& {141.3}	& {74.3}	& {333.8}	\\[0.2ex]
CrwodHat \cite{wu2023boosting}	& CVPR'23	& \second{52.3}	& {211.8}	& {51.2}	& {81.9}	& \first{5.7}	& {9.4}	& {-}	& {-}	& {75.1}	& \second{126.7}	& {68.7}	& \second{296.9}	\\[0.2ex]
STEERER \cite{han2023steerer}	& ICCV'23	& {54.3}	& {238.3}	& {54.5}	& {86.9}	& {5.8}	& \second{8.5}	& {-}	& {-}	& {74.3}	& {128.3}	& \second{63.7}	& {309.8}	\\[0.2ex]
PET \cite{liu2023point}	& ICCV'23	& {58.5}	& {238.0}	& \second{49.3}	& \second{78.8}	& {6.2}	& {9.7}	& {-}	& {-}	& {79.5}	& {144.3}	& {74.4}	& {328.5}	\\[0.2ex]
\rowcolor{black!10}\method\	& 	& \first{47.3}	& \first{198.9}	& \first{47.4}	& \first{75.0}	& \first{5.7}	& \first{8.2}	& \first{160.8}	& \first{225.0}	& \first{68.9}	& \first{125.6}	& \first{57.8}	& \first{221.2}	\\[0.2ex]
\bottomrule
\end{tabular}
}
\vspace{\tablegap}
\label{table: crowd counting performance}
\end{center}
\end{table*}
\begin{table*}[!t]
    \begin{center}
    
    \resizebox{\textwidth}{!}{
    \begin{tabular}{l|cc|ccccc|c}
\toprule

Model & VLM & Additional Backbone & General & Earth Monit. & Medical Sciences & Engineering & Agri. and Biology & Mean \\
\midrule\midrule
\textit{Random (LB)} & - & - & \phantom{0}\textit{1.17} & \phantom{0}\textit{7.11} & \textit{29.51} & \textit{11.71} & \phantom{0}\textit{6.14} & \textit{10.27} \\
\textit{Best supervised (UB)} & - & - & \textit{48.62} & \textit{79.12} & \textit{89.49} & \textit{67.66} & \textit{81.94} & \textit{70.99} \\
\midrule
ZSSeg~\citep{xu2022simple} & CLIP ViT-B/16 & ResNet-101 & 19.98 & 17.98 & \underline{41.82} & 14.0\phantom{0} & 22.32 & 22.73 \\
ZegFormer~\citep{ding2022decoupling} & CLIP ViT-B/16 & ResNet-101 & 13.57 & 17.25 & 17.47 & 17.92 & \underline{25.78} & 17.57 \\
X-Decoder~\citep{zou2023generalized} & UniCL-T & Focal-T & 22.01 & 18.92 & 23.28 & 15.31 & 18.17 & 19.8\phantom{0} \\
OpenSeeD~\citep{zhang2023simple} & UniCL-B & Swin-T & 22.49 & 25.11 & \textbf{44.44} & 16.5\phantom{0} & 10.35 & 24.33 \\
SAN~\citep{xu2023side} & CLIP ViT-B/16 & - & \underline{29.35} & \underline{30.64} & 29.85 & \textbf{23.58} & 15.07 & \underline{26.74} \\

\hlrow & & & \textbf{38.69} & \textbf{35.91} & 28.09 & \underline{20.34} & \textbf{32.57} & \textbf{31.96} \\
\hlrow\multirow{-2}{*}{\ours (ours)} & \multirow{-2}{*}{CLIP ViT-B/16} & \multirow{-2}{*}{-} & \textcolor{ForestGreen}{(+9.34)} & \textcolor{ForestGreen}{(+5.27)} & \color{gray}{(-16.35)} & \color{gray}{(-3.24)} & \textcolor{ForestGreen}{(+6.79)} & \textcolor{ForestGreen}{(+5.22)} \\
\midrule
OVSeg~\citep{liang2022open} & CLIP ViT-L/14 & Swin-B & 29.54 & 29.04 & \textbf{31.9\phantom{0}} & 14.16 & \underline{28.64} & 26.94 \\
SAN~\citep{xu2023side} & CLIP ViT-L/14 & - & \underline{36.18} & \underline{38.83} & \underline{30.27} & \underline{16.95} & 20.41 & \underline{30.06} \\
\hlrow & & & \textbf{44.69} & \textbf{39.99} & 24.70 & \textbf{20.20} & \textbf{38.61} & \textbf{34.70} \\
\hlrow\multirow{-2}{*}{\ours (ours)} & \multirow{-2}{*}{CLIP ViT-L/14} & \multirow{-2}{*}{-} & \textcolor{ForestGreen}{(+8.51)} & \textcolor{ForestGreen}{(+1.16)} & \color{gray}{(-7.2)} & \textcolor{ForestGreen}{(+3.25)} & \textcolor{ForestGreen}{(+9.97)} & \textcolor{ForestGreen}{(+4.64)} \\
        \bottomrule
    \end{tabular}
    }

    \vspace{-5pt}        
    \caption{\textbf{Quantitative evaluation on MESS~\citep{blumenstiel2023mess}.} MESS includes a wide range of domain-specific datasets, which pose significant challenges due to their substantial domain differences from the training dataset. We report the average score for each domain. Please refer to the supplementary material for the results of all 22 datasets. \textit{Random} is the result of uniform distributed prediction which represents the lower-bound, while \textit{Best supervised} represents the upper-bound performance for the datasets.}
    \label{tab:mess}
    \vspace{-20pt}
    \end{center}
\end{table*}


\section{Experiments}
\subsection{Datasets and Evaluation}
We train our model on the COCO-Stuff~\cite{caesar2018coco}, which has 118k densely annotated training images with 171 categories, following \cite{liang2022open}. We employ the mean Intersection-over-Union (mIoU) as the evaluation metric for all experiments. For the evaluation, we conducted experiments on two different sets of datasets~\cite{zhou2019semantic,everingham2009pascal,mottaghi2014role}: a commonly used in-domain datasets~\cite{ghiasi2022scaling}, and a multi-domain evaluation set~\cite{blumenstiel2023mess} containing domain-specific images and class labels. 

\vspace{-10pt}
\paragraph{Datasets for standard benchmarks.} 
For in-domain evaluation, we evaluate our model on ADE20K~\cite{zhou2019semantic}, PASCAL VOC~\cite{everingham2009pascal}, and PASCAL-Context~\cite{mottaghi2014role} datasets. ADE20K has 20k training and 2k validation images, with two sets of categories: A-150 with 150 frequent classes and A-847 with 847 classes~\cite{ding2022decoupling}. PASCAL-Context contains 5k training and validation images, with 459 classes in the full version (PC-459) and the most frequent 59 classes in the PC-59 version. PASCAL VOC has 20 object classes and a background class, with 1.5k training and validation images. We report PAS-20 using 20 object classes. We also report the score for PAS-$20^b$, which defines the ``background" as classes present in PC-59 but not in PAS-20, as in \citet{ghiasi2022scaling}.

\vspace{-10pt}
\paragraph{Datasets for multi-domain evaluation.}
We conducted a multi-domain evaluation on the MESS benchmark~\cite{blumenstiel2023mess}, specifically designed to stress-test the real-world applicability of open-vocabulary models with 22 datasets. The benchmark includes a wide range of domain-specific datasets from fields such as earth monitoring, medical sciences, engineering, agriculture, and biology. Additionally, the benchmark contains a diverse set of general domains, encompassing driving scenes, maritime scenes, paintings, and body parts. We report the average scores for each domain in the main text for brevity. For the complete results and details of the 22 datasets, please refer to the supplementary material.

\subsection{Implementation Details}
We train the CLIP image encoder and the cost aggregation module with per-pixel binary cross-entropy loss. We set $d_F=128$, $N_B=2$, $N_U=2$ for all of our models. We implement our work using PyTorch~\cite{paszke2019pytorch} and Detectron2~\cite{wu2019detectron2}. AdamW~\cite{loshchilov2017decoupled} 
optimizer is used with a learning rate of $2\cdot10^{-4}$ for our model  and $2\cdot10^{-6}$ for the CLIP, with weight decay set to $10^{-4}$. The batch size is set to 4. We use 4 NVIDIA RTX 3090 GPUs for training. All of the models are trained for 80k iterations. 

\subsection{Main Results}
\paragraph{Results of standard benchmarks.}
The evaluation of standard open-vocabulary semantic segmentation benchmarks is shown in Table~\ref{tab:main_table}. Overall, our method significantly outperforms all competing methods, including those~\cite{ghiasi2022scaling,liang2022open} that leverage additional datasets~\cite{chen2015microsoft,pont2020connecting} for further performance improvements. To ensure a fair comparison, we categorize the models based on the scale of the vision-language models (VLMs) they employ. First, we present results for models that use VLMs of comparable scale to ViT-B/16~\cite{dosovitskiy2020image}, and our model surpasses all previous methods, even achieving performance that matches or surpasses those using the ViT-L/14 model as their VLM~\cite{xu2023side}.
For models employing the ViT-L/14 model as their VLM, our model demonstrates remarkable results, achieving a 16.0 mIoU in the challenging A-847 dataset and a 23.8 mIoU in PC-459. These results represent a 29\% and 52\% increase, respectively, compared to the previous state-of-the-art.
We also present qualitative results of PASCAL-Context with 459 categories in Fig.~\ref{fig:qualitative}, demonstrating the efficacy of our proposed approach in comparison to the current state-of-the-art methods~\cite{ding2022decoupling, xu2022simple,liang2022open}. 


\begin{figure*}[t]
  \centering
    \subfloat[SAN]
{\includegraphics[width=0.1595\linewidth]{figures/fig4/pc_san.pdf}}\hfill
    \subfloat[\textbf{Ours}]
{\includegraphics[width=0.1595\linewidth]{figures/fig4/pc_ours.pdf}}\hfill
     \subfloat[GT]
 {\includegraphics[width=0.1595\linewidth]{figures/fig4/pc_gt.pdf}}\hfill
    \subfloat[SAN]
{\includegraphics[width=0.166\linewidth]{figures/fig4/mess_san.pdf}}\hfill
    \subfloat[\textbf{Ours}]
{\includegraphics[width=0.166\linewidth]{figures/fig4/mess_ours.pdf}}\hfill
    \subfloat[GT]
{\includegraphics[width=0.166\linewidth]{figures/fig4/mess_gt.pdf}}\hfill
\\
\vspace{-10pt}
\caption{\textbf{Qualitative comparison to SAN~\citep{xu2023side}.} We visualize the results of PC-459 dataset in (a-c). For (d-f), we visualize the results from the MESS benchmark~\citep{blumenstiel2023mess} across three domains: underwater (top), human parts (middle), and agriculture (bottom).} 
\label{fig:qualitative}
\vspace{-10pt}
\end{figure*}

\begin{table}[t]
    \centering
    \resizebox{0.48\textwidth}{!}{%
    \begin{tabular}{cl|cccccc}
    \toprule
        & Methods & A-847 & PC-459 & A-150 & PC-59 & PAS-20 & $\textnormal{PAS-20}^b$
        \\
        \midrule\midrule
        \textbf{(I)} & Feature agg. + Freeze & 3.1 & 8.7 & 16.6 & 46.8 & 92.3 & 69.7\\
        \textbf{(II)} & Feature agg. + F.T. & 5.6 & {12.8} & {23.6} & \underline{58.1} & \underline{96.3} & \underline{77.7}\\
        \midrule
        \textbf{(III)} & Cost agg. + Freeze& \underline{10.0} & \underline{14.5} & \underline{26.0} & 46.9 & 94.2 & 65.1\\
        \hlrow\textbf{(IV)} & Cost agg. + F.T. & \textbf{14.7} & \textbf{23.2} & \textbf{35.3} & \textbf{60.3} & \textbf{96.7} & \textbf{78.9}\\
        \bottomrule
    \end{tabular}%
    }
    \vspace{-5pt}
    \caption{\textbf{Quantitative comparison between feature and cost aggregation.} Cost aggregation acts as an effective alternative to direct fine-tuning of CLIP image encoder. \textit{F.T.: Fine-Tuning.}
    }
    \label{tab:feature-vs-cost}
    \vspace{-15pt}

\end{table}



\vspace{-10pt}
\paragraph{Results of multi-domain evaluation.}
In Table~\ref{tab:mess}, we present the qualitative results obtained from the MESS benchmark~\cite{blumenstiel2023mess}. This benchmark assesses the real-world performance of a model across a wide range of domains. Notably, our model demonstrates a significant performance boost over other models, achieving the highest mean score. It particularly excels in the general domain as well as in agriculture and biology, showing its strong generalization ability. However, in the domains of medical sciences and engineering, the results exhibit inconsistencies with respect to the size of the VLM. Additionally, the scores for medical sciences are comparable to random predictions. We speculate that CLIP may have limited knowledge in these particular domains~\cite{radford2021learning}.


\subsection{Analysis and Ablation Study}\label{sec:ablation}

\paragraph{Comparison between feature and cost aggregation.} We provide quantitative and qualitative comparison of two aggregation baselines, feature aggregation, and cost aggregation, in Table~\ref{tab:feature-vs-cost}. For both of baseline architectures, we simply apply the upsampling decoder and note that both methods share most of the architecture, but differ in whether they aggregate the concatenated features or aggregate the cosine similarity between image and text embeddings of CLIP.
\begin{table}[t]
    \centering
    \resizebox{0.48\textwidth}{!}{%
    \begin{tabular}{cl|cccccc}
    \toprule
        & Methods & A-847 & PC-459 & A-150 & PC-59 & PAS-20 & $\textnormal{PAS-20}^b$
        \\
        \midrule\midrule
        \textbf{(I)} & Feature agg. + Freeze & 3.1 & 8.7 & 16.6 & 46.8 & 92.3 & 69.7\\
        \textbf{(II)} & Feature agg. + F.T. & 5.6 & {12.8} & {23.6} & \underline{58.1} & \underline{96.3} & \underline{77.7}\\
        \midrule
        \textbf{(III)} & Cost agg. + Freeze& \underline{10.0} & \underline{14.5} & \underline{26.0} & 46.9 & 94.2 & 65.1\\
        \hlrow\textbf{(IV)} & Cost agg. + F.T. & \textbf{14.7} & \textbf{23.2} & \textbf{35.3} & \textbf{60.3} & \textbf{96.7} & \textbf{78.9}\\
        \bottomrule
    \end{tabular}%
    }
    \vspace{-5pt}
    \caption{\textbf{Quantitative comparison between feature and cost aggregation.} Cost aggregation acts as an effective alternative to direct fine-tuning of CLIP image encoder. \textit{F.T.: Fine-Tuning.}
    }
    \label{tab:feature-vs-cost}
    \vspace{-15pt}

\end{table}


For \textbf{(I)} and \textbf{(III)}, we freeze the encoders of CLIP and only optimize the upsampling decoder. Subsequently, in \textbf{(II)} and \textbf{(IV)}, we fine-tune the encoders of CLIP on top of \textbf{(I)} and \textbf{(III)}. Our results show that feature aggregation can benefit from fine-tuning, but the gain is only marginal. On the other hand, cost aggregation benefits significantly from fine-tuning, highlighting the effectiveness of cost aggregation for adapting CLIP to the task of segmentation. 

For the qualitative results in Fig.~\ref{fig:feature_cost}, we show the prediction results from \textbf{(II)} and \textbf{(IV)}. As seen in Fig.~\ref{fig:feature_cost}(c-d), we observe that feature aggregation shows overfitting to the seen class of ``bucket," while cost aggregation successfully identifies the unseen class ``birdcage." 

\begin{table}[t!]
\centering
\resizebox{\linewidth}{!}{
\begin{tabular}{ll|cccccc}
        \toprule
        &Components & A-847 & PC-459 & A-150 & PC-59 & PAS-20 & $\textnormal{PAS-20}^b$
         \\
        \midrule\midrule
        \textbf{(I)} & Feature Agg. & 5.6 & 12.8 & 23.6 & 58.1 & 96.3 & 77.7\\
        \midrule
        \textbf{(II)} & Cost Agg.  & 14.7& \underline{23.2}& 35.3& 60.3& \underline{96.7}&78.9\\
        \textbf{(III)} &\textbf{(II)} + Spatial agg.  & 14.9& 23.1& 35.9& 60.3& \underline{96.7}&79.5\\
        \textbf{(IV)} &\textbf{(II)} + Class agg.  & 14.7& 21.5& 36.6& 60.6& 95.5&80.5\\
        \textbf{(V)} &\textbf{(II)} + Spatial and Class agg. & \underline{15.5}& \underline{23.2}& \underline{37.0}& \underline{62.3}& \underline{96.7}&\underline{81.3}\\
        \hlrow\textbf{(VI)} &\textbf{(V)} + Embedding guidance  & \textbf{16.0} & \textbf{23.8}& \textbf{37.9}& \textbf{63.3}& \textbf{97.0}&\textbf{82.5}\\
        \bottomrule
\end{tabular}}
\vspace{-5pt}
\caption{\textbf{Ablation study for \ours.} We conduct ablation study by gradually adding components to the cost aggregation baseline.}
    \vspace{-10pt}
    \label{tab:ablation}
\end{table}


\begin{table}[h!]\scriptsize	
    \centering
    \begin{tabular}{c|c|c|c|c|c}
        \multicolumn{2}{c|}{} & Baseline & w/o normals & w/o viscosity & w/o coarea \\ \hline
        \multirow{4}{*}{Anchor}
            & $d_C$ & \textbf{0.21} & 0.61 & 0.55 & 0.72 \\
            & $d_H$ & \textbf{3.00} & 7.82 & 10.83 & 10.24 \\
            & $d_C^\too$ & 0.15 & 0.37 & 0.27 & 0.36 \\
            & $d_H^\too$ & 1.07 & 7.84 & 1.44 & 9.68 \\ \hline
        \multirow{4}{*}{Daratech}
            & $d_C$ & 0.26 & 0.24 & 0.24 & \textbf{0.23} \\
            & $d_H$ & 4.06 & 4.2 & 4.3 & \textbf{2.19} \\
            & $d_C^\too$ & 0.14 & 0.13 & 0.12 & 0.13 \\
            & $d_H^\too$ & 1.76 & 2.69 & 1.77 & 1.77 \\ \hline
        \multirow{4}{*}{DC}
            & $d_C$ & \textbf{0.15} & \textbf{0.15} & \textbf{0.15} & 0.34 \\
            & $d_H$ & \textbf{2.22} & 2.24 & 2.24 & 6.58 \\
            & $d_C^\too$ & 0.09 & 0.08 & 0.08 & 0.16 \\
            & $d_H^\too$ & 2.76 & 2.76 & 2.79 & 2.82 \\ \hline
        \multirow{4}{*}{Gargoyle}
            & $d_C$ & \textbf{0.17} & 0.58 & 0.47 & 0.59 \\
            & $d_H$ & \textbf{4.40} & 6.32 & 10.38 & 6.35 \\
            & $d_C^\too$ & 0.11 & 0.07 & 0.26 & 0.38 \\
            & $d_H^\too$ & 0.96 & 2.39 & 1.34 & 1.25 \\ \hline
        \multirow{4}{*}{Lord Quas}
            & $d_C$ & \textbf{0.12} & 0.12 & 0.12 & 0.58 \\
            & $d_H$ & 1.06 & 1.38 & \textbf{1.04} & 6.05 \\
            & $d_C^\too$ & 0.07 & 0.37 & 0.06 & 0.32 \\
            & $d_H^\too$ & 0.64 & 0.69 & 0.64 & 3.73 \\ \hline %
            
    \end{tabular} \vspace{5pt}
    \caption{Ablations study. We show the contribution of each component of VisCo Grids. Baseline is the full method. The remaining columns correspond to optimizing without normal loss, viscosity loss and coarea loss, respectively. We show results for each mesh of the benchmark \cite{williams2019deep}. The results justify the use of the different components in VisCo Grids.}
    \label{tab:ablations}
\end{table}
\vspace{-10pt}
\paragraph{Component analysis.}
Table~\ref{tab:ablation} shows the effectiveness of the main components within our architecture through quantitative results. 
First, we introduce the baseline models in \textbf{(I)} and \textbf{(II)}, identical to the fine-tuned baseline models from Table~\ref{tab:feature-vs-cost}.
We first add the proposed spatial and class aggregations to the cost aggregation baseline in \textbf{(III)} and \textbf{(IV)}, respectively. In \textbf{(V)}, we interleave the spatial and class aggregations. Lastly, we add the proposed embedding guidance to \textbf{(V)}, which becomes our final model.

As shown, we stress the gap between \textbf{(I)} and  \textbf{(II)}, which supports the findings presented in Fig.~\ref{fig:feature_cost}. Given that PAS-20 shares most of its classes with the training datasets\cite{xu2022simple}, the performance gap between \textbf{(I)} and \textbf{(II)} is minor. However, for challenging datasets such as A-847 or PC-459, the difference is notably significant, validating our cost aggregation framework for its generalizability.
We also highlight that as we incorporate the proposed spatial and class aggregation techniques, our approach \textbf{(V)} outperforms \textbf{(II)}, demonstrating the effectiveness of our design.
Finally, \textbf{(VI)} shows that our embedding guidance further improves performance across all the benchmarks.
Furthermore, we provide quantitative results of adopting the upsampling decoder in Table ~\ref{tab:conv-decoder}. The results show consistent improvements across all the benchmarks.


\begin{table}[!t]
\centering
\resizebox{\linewidth}{!}{
   \begin{tabular}{ll|cccccc|cc}
        \toprule
        &\multirow{2}{*}{Methods} & \multirow{2}{*}{A-847} & \multirow{2}{*}{PC-459} & \multirow{2}{*}{A-150} & \multirow{2}{*}{PC-59}& \multirow{2}{*}{PAS-20} & \multirow{2}{*}{$\textnormal{PAS-20}^b$} &\#param.  & Memory
         \\
         &&&&&&&&(M)&(GiB)
         \\
        \midrule\midrule
        \textbf{(I)} &Freeze & 10.4& 15.0& 31.8& 52.5& 92.2& 71.3& 5.8 & 20.0\\
        \textbf{(II)} &Prompt  & 8.8& 14.3 & 30.5& 55.8 & 93.2 & 74.7 & 7.0 & 20.9\\
        \textbf{(III)} &Full F.T.  & 13.6& 22.2& 34.0& 61.1& \textbf{97.3}& 79.7 & 393.2 & 26.8\\
        \textbf{(IV)} &Attn. F.T. & 15.7& \underline{23.7}& 37.1& \underline{63.1}& \underline{97.1}& 81.5 & 134.9 & 20.9\\
        \textbf{(V)} &QK F.T. & 15.3& 23.0& 36.3& 62.0& 95.9& 81.9 & 70.3 & 20.9\\
        \textbf{(VI)} &KV F.T. & \textbf{16.1}& \textbf{23.8}& \underline{37.6}& 62.4& 96.7& \underline{82.0} & 70.3 & 20.9\\
        \midrule
        \textbf{(VII)} & QV F.T. (Img.)  & 13.9& 22.8& 35.1& 62.0& 96.3& \underline{82.0} & 56.7 & 20.9\\
        \textbf{(VIII)} & QV F.T. (Txt.)  & 14.7& 22.2& 35.1& 60.0& 95.8& 80.3 & 19.9 & 20.0\\
        \hlrow \textbf{(IX)} & QV F.T. (Both) & \underline{16.0}& \textbf{23.8}& \textbf{37.9}& \textbf{63.3}& 97.0& \textbf{82.5} & 70.3 & 20.9\\
        \bottomrule       
\end{tabular}
}
    \vspace{-5pt}
\caption{\textbf{Analysis of fine-tuning methods for CLIP.} We additionally note the number of learnable parameters of CLIP and memory consumption during training. Our method not only outperforms full fine-tuning, but also requires smaller computation.}
\label{tab:finetuning-ablation}
\vspace{-10pt}
\end{table}

\begin{figure}[t]
  \centering
    \subfloat[CLIP]
{\includegraphics[width=0.4999\linewidth]{figures/fig_embedding/tsne_clip_final.pdf}}\hfill
     \subfloat[Fine-tuned CLIP]
 {\includegraphics[width=0.4999\linewidth]{figures/fig_embedding/tsne_clip_2_v3.pdf}}\hfill\\
        
\vspace{-5pt}
\caption{\textbf{Effects of fine-tuning CLIP.} We show the t-SNE~\cite{van2008visualizing} 
visualization of CLIP image embeddings based on its predictions. In contrast to (a), we observe well-grouped clusters in (b), showing the adaptation of CLIP to segmentation for both seen and unseen classes.} 
\label{fig:embedding_space}
\vspace{-10pt}
\end{figure}


\vspace{-10pt}
\label{finetune}
\paragraph{Analysis on fine-tuning of CLIP.}
In this section, we analyze the effects and methods of fine-tuning of the encoders of CLIP. In Table~\ref{tab:finetuning-ablation}, we report the results of different approaches, which include the variant \textbf{(I)}:~without fine-tuning, \textbf{(II)}:~adopting Prompt Tuning~\cite{zhou2022learning, jia2022visual}, \textbf{(III)}:~fine-tuning the entire CLIP, \textbf{(IV)}:~fine-tuning the attention layer only~\cite{touvron2022three}, \textbf{(V)}:~fine-tuning query and key projections only, \textbf{(VI)}:~fine-tuning key and value projections only, \textbf{(VII)}:~our approach for CLIP image encoder only, \textbf{(VIII)}:~our approach for text encoder only, and  \textbf{(IX)}:~our approach for both encoders. Note that both image and text encoders are fine-tuned in \textbf{(I-VI)}. Overall, we observed that fine-tuning enhances the performance of our framework. Among the various fine-tuning methods, fine-tuning only the query and value projection yields the best performance improvement while also demonstrating high efficiency. Additionally, as can be seen in \textbf{(VII-IX)}, fine-tuning both encoders leads to better performance compared to fine-tuning only one of them in our framework.

In Fig.~\ref{fig:embedding_space}, we show the t-SNE~\cite{van2008visualizing} visualization of the dense image embeddings of CLIP within the A-150~\cite{zhou2019semantic} dataset. We color the embeddings based on the prediction with text classes. From (a), we can observe that the clusters are not well-formed for each classes, due to the image-level training of CLIP. In contrast, we observe well-formed clusters in (b) for both seen and unseen classes, showing the adaptation of CLIP for the downstream task.

\vspace{-10pt}
\paragraph{Training with various datasets.}
In this experiment, we further examine the generalization power of our method in comparison to other methods~\cite{ding2022decoupling, xu2022simple} by training our model on smaller-scale datasets, which include A-150 and PC-59, that poses additional challenges to achieve good performance.  The results are shown in Table~\ref{tab:cross-dataset-ablation}. As shown, we find that although we observe some performance drops, which seem quite natural when a smaller dataset is used, our work significantly outperforms other competitors. These results highlight the strong generalization power of our framework, a favorable characteristic that suggests the practicality of our approach.

\begin{table}[!t]
    \centering
    
    \resizebox{\linewidth}{!}{
    \begin{tabular}{l|c|ccccccc}
    \toprule
        Methods & Training dataset & A-847 & PC-459 & A-150 & PC-59 & PAS-20 & $\textnormal{PAS-20}^b$
        \\
        \midrule\midrule
        ZegFormer & COCO-Stuff & 5.6 & \underline{10.4} & 18.0 & 45.5 & \underline{89.5} & 65.5\\
        ZSseg & COCO-Stuff & \underline{7.0} & 9.0 & \underline{20.5} & \underline{47.7} & 88.4 & \underline{67.9}\\
        \hlrow \ours (ours) & COCO-Stuff & \textbf{12.0} & \textbf{19.0} & \textbf{31.8} & \textbf{57.5} & \textbf{94.6} & \textbf{77.3}\\
        \midrule
        ZegFormer & A-150 & 6.8 & \underline{7.1} & \color{gray}{33.1} & 34.7 & 77.2 & 53.6 \\
        ZSseg & A-150 & \underline{7.6} & \underline{7.1} & \color{gray}{40.3} & \underline{39.7} & \underline{80.9} & \underline{61.1}\\
        \hlrow \ours (ours) & A-150 & \textbf{14.4} & \textbf{16.2} & \color{gray}{47.7} & \textbf{49.9} & \textbf{91.1} & \textbf{73.4} \\
        \midrule
        ZegFormer & PC-59 & \underline{3.8} & \underline{8.2} & \underline{13.1} & \color{gray}{48.7} & 86.5 & 66.8 \\
        ZSseg & PC-59 & 3.0 & 7.6 & 11.9 & \color{gray}{54.7} & \underline{87.7} & \underline{71.7}\\
        \hlrow \ours (ours) & PC-59 & \textbf{9.6} & \textbf{16.7} & \textbf{27.4} & \color{gray}{63.7} & \textbf{93.5} & \textbf{79.9} \\
        \bottomrule
    \end{tabular}
    }

    
    \vspace{-5pt}
    \caption{\textbf{Training on various datasets.} CLIP with ViT-B is used for all methods. Our model demonstrates remarkable generalization capabilities even on relatively smaller datasets. The scores evaluated on the same dataset used for training are colored in \textcolor{gray}{gray}.}
    \vspace{-10pt}
    \label{tab:cross-dataset-ablation}
\end{table}

\begin{table}[t!]
    \centering
    \caption{
    \textbf{Efficiency comparison of optimization algorithms.}
    R@1 scores evaluated on MSRVTT-7k for video retrieval are recorded.
    Multi-task learning simultaneously trains all tasks with even loss weights. 
    CG and FP are abbreviations of conjugate gradient and fixed-point optimization. 
    In terms of time costs, average training time per epoch is reported. 
    $^\dagger$ refers to our optimization algorithm which approximates $\nabla^2_w \aux$ as the identity matrix $\mathrm{I}$.}
    \begin{adjustbox}{width=\linewidth}
    \begin{tabular}{l |c| c  c}
        \toprule
        \textbf{Method}  & \textbf{Opt. Scheme}  & \textbf{R@1} &  \textbf{Time} \\
        \midrule
        \midrule
        Multi-task Learning   & 
        - &  
        26.1 \scriptsize(+0.0)    & 
        547 \scriptsize(+0.0\%) \\
        
        \textbf{MELTR} + Meta-Weight Net~\cite{shu2019meta}  & 
        ITD &  
        27.3 \scriptsize(\textcolor{red}{+1.2})  & 
        1,296 \scriptsize(\textcolor{red}{+136.9\%}) \\ 
        
        \textbf{MELTR} + StocBIO~\cite{ji2021bilevel} & 
        N/A  &  
        26.8 \scriptsize(\textcolor{red}{+0.7})   &   
        686 \scriptsize(\textcolor{red}{+25.4\%})\\
        
        \textbf{MELTR} + CG & 
        AID-CG &  
        28.0 \scriptsize(\textcolor{red}{+1.9})   &   
        624 \scriptsize(\textcolor{red}{+14.1\%})\\
        
        \textbf{MELTR} + AuxiLearn~\cite{navon2020auxiliary} &  
        AID-FP    &  
        27.9 \scriptsize(\textcolor{red}{+1.8})    &
        638 \scriptsize(\textcolor{red}{+16.6\%})      \\
        
        \textbf{MELTR} + \textbf{AID-FP-Lite}$^\dagger$ & 
        AID-FP &  
        28.5 \scriptsize(\textcolor{red}{+2.4})   &   
        574 \scriptsize(\textcolor{red}{+4.9\%})\\
        \bottomrule
    \end{tabular}
    \end{adjustbox}
    \label{tab:efficiency}
    \vspace{-3mm}
\end{table}

\vspace{-10pt}
\paragraph{Efficiency comparison.}
In Table~\ref{tab:efficiency}, we thoroughly compare the efficiency of our method to recent methods~\cite{ding2022decoupling,xu2022simple,liang2022open}. We measure the number of learnable parameters, the total number of parameters, training time, inference time, and inference GFLOPs. Our model demonstrates strong efficiency in terms of both training and inference. This efficiency is achieved because our framework does not require an additional mask generator~\cite{ding2022decoupling}.


\section{Conclusion}
In conclusion, we introduce a cost aggregation framework for open-vocabulary semantic segmentation, aggregating the cosine-similarity scores between image and text embeddings of CLIP. Through our \ours framework, we fine-tune the encoders of CLIP for its adaptation for the downstream task of segmentation. Our method surpasses the previous state-of-the-art in standard benchmarks and also in scenarios with a vast domain difference. The success in diverse domains underscores the promise and potential of our cost aggregation framework in advancing the field of open-vocabulary semantic segmentation.\\
\vspace{-10pt}\paragraph{Acknowledgement.} This research was supported by the MSIT, Korea (IITP-2023-2020-0-01819, RS-2023-00266509).




\section*{Acknowledgments}
We are grateful to Dr. Clare Voss from the Army Research Laboratory for her helpful insights and feedback. We would also like to acknowledge Lisa Ferro and Brad Goodman from MITRE for their valuable comments and help with expert evaluation. This research is based upon work supported by U.S. DARPA AIDA Program No. FA8750-18-2-0014, DARPA KAIROS Program No. FA8750-19-2-1004, DARPA SemaFor Program No. HR001120C0123, DARPA INCAS Program No. HR001121C0165 and DARPA MIPS Program No. HR00112290105. 
The views and conclusions contained herein are those of the authors and should not be interpreted as necessarily representing the official policies, either expressed or implied, of DARPA, or the U.S. Government. The U.S. Government is authorized to reproduce and distribute reprints for governmental purposes notwithstanding any copyright annotation therein.

%\section*{Declarations}

%Some journals require declarations to be submitted in a standardised format. Please check the Instructions for Authors of the journal to which you are submitting to see if you need to complete this section. If yes, your manuscript must contain the following sections under the heading `Declarations':

%\begin{itemize}
%\item Funding
%\item Conflict of interest/Competing interests (check journal-specific guidelines for which heading to use)
%\item Ethics approval 
%\item Consent to participate
%\item Consent for publication
%\item Availability of data and materials
%\item Code availability 
%\item Authors' contributions
%\end{itemize}

%\noindent
%If any of the sections are not relevant to your manuscript, please include the heading and write `Not applicable' for that section. 

%%===================================================%%
%% For presentation purpose, we have included        %%
%% \bigskip command. please ignore this.             %%
%%===================================================%%
%\bigskip
%\begin{flushleft}%
%Editorial Policies for:

%\bigskip\noindent
%Springer journals and proceedings: \url{https://www.springer.com/gp/editorial-policies}

%\bigskip\noindent
%Nature Portfolio journals: \url{https://www.nature.com/nature-research/editorial-policies}

%\bigskip\noindent
%\textit{Scientific Reports}: \url{https://www.nature.com/srep/journal-policies/editorial-policies}

%\bigskip\noindent
%BMC journals: \url{https://www.biomedcentral.com/getpublished/editorial-policies}
%\end{flushleft}

%\section{Comment Tracker (completed?)}

\heng{add a special footnote for Paul}


\heng{change intel report to situation report} \Yi{done}

\heng{the intro needs to be stronger: 1. emphasize any sort of application of mature AI techniques for intel analysis is near to zero; } \Yi{1-done}

\heng{don't over claim we are automating the whole intel report generation pipeline; machine needs to help human on generating report} \Yi{done}

\manling{I have addressed the comments via tracking changes.} \revanth{@heng: I have added my responses in brown to your comments. Feel free toƒ remove the comments for which you feel has been handled according to the response.}
\heng{A complete story: we need intel report for emergent situations, human analysts need to finish report within a few hours, so the reports tend to be incomplete and biased, hard to keep up to date, and cannot understand information in foreign languages, etc.. we developed a novel platform, leveraging SOTA NLP, and proposed new techniques, to automate intel report generation. the system knows how to ask good questions, extract and summarize related claims with evidence etc.}

This grounding of information element forms an interactive, multimedia knowledge graph enriched with relevant images. Additionally, our \textbf{SmartBook} framework extends the concept of schema induction and event prediction \cite{li-etal-2020-connecting} in the traditional ML/NLP research communities, to generate a natural language ``perspective" section, examining the probable future events, towards the end of the situation report. \revanth{Should we drip the preceding two sentences given we don't have multimedia/event prediction for now?} 

%\begin{appendices}

%\section{Section title of first appendix}\label{secA1}

%%=============================================%%
%% For submissions to Nature Portfolio Journals %%
%% please use the heading ``Extended Data''.   %%
%%=============================================%%

%%=============================================================%%
%% Sample for another appendix section			       %%
%%=============================================================%%

%% \section{Example of another appendix section}\label{secA2}%
%% Appendices may be used for helpful, supporting or essential material that would otherwise 
%% clutter, break up or be distracting to the text. Appendices can consist of sections, figures, 
%% tables and equations etc.

%\end{appendices}
%
%%===========================================================================================%%
%% If you are submitting to one of the Nature Portfolio journals, using the eJP submission   %%
%% system, please include the references within the manuscript file itself. You may do this  %%
%% by copying the reference list from your .bbl file, paste it into the main manuscript .tex %%
%% file, and delete the associated \verb+\bibliography+ commands.                            %%
%%===========================================================================================%%

\bibliography{sn-bibliography}% common bib file
%% if required, the content of .bbl file can be included here once bbl is generated
%%\input sn-article.bbl

%% Default %%
%%\input sn-sample-bib.tex%

\end{document}
