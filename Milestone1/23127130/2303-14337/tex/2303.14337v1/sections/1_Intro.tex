\section{Introduction}\label{sec1}
%\heng{Introduction - highlight our novelty in automatic chapter construction (human analyst questions very vague and hard for machines to parse)} \Yi{thought: claim in parenthesis may be too strong?}

%begin{itemize}
%    \item Intelligence analysts have the task of gathering information from a large variety of sources, such as news articles, visual sources (image or video), social media, as well from human sources.
%\end{itemize}

%It endows machines with the capability to \textit{read}, \textit{think}, and \textit{write}.

% \vicki{TODO: add GPT results (as one of the motivations)?}
% \heng{before "current multi-document news summarization..." add something like "In particular, current large language models based text generation approaches such as Chat-GPT either return no answers or false answers to important questions, because they cannot process information in real-time (e.g., Chat-GPT is limited to the data before 2021) or lack fact-checking mechanisms to filter out their hallucinated false information." and you can put in a table to show the comparison between smartbook and chatgpt using the examples I put in the slides}

In today's fast-paced world, staying informed is more important than ever. With the increasing availability of information sources, it can be challenging to separate fact from fiction. This is particularly true during times of crisis, when rumors and false information can spread quickly, causing unnecessary panic or confusion. In these situations, having access to reliable and up-to-date information can make all the difference. It can help individuals and organizations make informed and safe decisions, take appropriate actions, and even save lives. As such, the ability to understand the situation amidst the overwhelming amount of news and rumors becomes paramount. 

Important events and emerging crises often come with a vast amount of news and rumors, which vary in credibility but offer diverse story coverage. Situation understanding amidst this information overload is extremely valuable for decision-makers, stakeholders, and the general public interest, alike. For example, timely updates about the COVID pandemic can help people better prepare for social distancing and prevention of spread. Staying informed about the Russia-Ukraine crisis can help organizations determine where humanitarian efforts are most needed, as well as how energy and food security are impacted across the world. 

The key to situation understanding for such events is to extract the common truths across heterogeneous data that meaningfully impact action planning and strategic development. Along this direction, there are many expert analysts today who manually prepare \textit{situation reports} about the state of affairs, potential risks or threats, and perspective, with recommended actions. While playing a foundational role in government policies and organization response, situation reports are time-consuming and resource-intensive to manually develop. They are also often limited to a few topics, scenarios, or regions, and thus not scalable. Further, manual reports can be biased, incomplete, difficult to keep up-to-date, and incapable of consolidating information across multiple data sources and languages. Therefore, it is crucial to explore the automatic generation of situation reports to facilitate the timely, comprehensive, and organized presentation of information for prompt policy actions and tactical operations. %, reducing the time and cost of traditional human curation efforts. \manling{I suggest not to emphasize time and cost but emphasize the comprehensiveness and unbiased merits, according to our eval metrics }

\begin{table}[t]
\centering
\renewcommand{\arraystretch}{1.5}
\small
\begin{tabular}{p{52em}}
\hline
\textbf{Question}: 
Is Russian positioning forces %inside of Russia 
for possible follow-on actions into Ukraine after the initial invasion? \\
\hline
\textbf{GPT-4}: As an AI language model, I cannot provide real-time information on current events. However, it is essential for intelligence analysts to monitor the activities of various countries, including troop movements and military exercises,to assess potential threats and anticipate future actions. To determine if Russia is positioning forces for possible actions into Ukraine, one must analyze the latest intelligence reports, satellite imagery, and open-source information. Additionally, understanding the political and historical context of the region can provide valuable insight into the intentions of the countries involved.\\
\hline
\textbf{SmartBook}: The arrival of Russian forces in Slavutych came after several days of shelling against the city, which is strategically located close to the Dnieper River, about 95 miles (150 kilometers) north of Kyiv and close to the border with Belarus.
Russia has been using Belarus as a springboard for many of its air operations in Ukraine, according to intelligence collected by NATO surveillance planes flying over the Polish-Ukrainian border and radar seen by CNN.
So far, however, US officials have not seen Belarusian troops ``being readied to move into Ukraine" or ``that they are moving or are in Ukraine", a senior US defense official said Monday, adding that the forces inside Ukraine are Russian.\\
% \textbf{Answer}: I don't have the current information regarding any Russian military \\\cellcolor{lightgray}actions or deployments. It's important to rely on credible new sources for the \\\cellcolor{lightgray}most up-to-date and accurate information on political and military developments. \\
% \textbf{Question}: What is the status of efforts to secure the release of Paul Whelan and \\ Brittney Griner?\\
% \cellcolor{lightgray}\textbf{Answer}: I'm sorry, but as a language model AI, I do not have the most \\\cellcolor{lightgray}up-to-date information on the current status of individual cases like Paul \\\cellcolor{lightgray}Whelan and Brittney  Griner. However. as of my knowledge cutoff in 2021, Paul \\\cellcolor{lightgray}Whelan, a former US Marine, was being held in Russia on espionage charges \\\cellcolor{lightgray}and has been advocating for his release through his legal team and government \\\cellcolor{lightgray}representatives. Similarly, Brittney Griner is a basketball player who was \\\cellcolor{lightgray}playing in Russia and has been involved in legal issues there, but I don't have \\\cellcolor{lightgray}any more specific details on her case. It's best to check reliable news sources \\\cellcolor{lightgray}for the most recent updates on their cases.\\
\hline
\end{tabular}
\caption{Example of situation understanding for the Ukraine-Russia crisis. GPT-4 is incapable of generating situation reports because it has limited knowledge of up-to-date events, while Smartbook is able to generate timely, multi-source, and trustworthy answers.%\heng{maybe change the example to "Why is the Zaporizhzhia nuclear power plant a target for Russia?" because for that question GPT returns factual error "The Zaporizhzhia Nuclear Power Plant is not a specific target for Russia..." that seems to be more serious error than providing generic information}
}
\label{tab:chatgpt}
\vspace{-1em}
\end{table}


Automatic situation report generation poses an exciting new problem domain that has not been addressed by previous research work. Existing approaches to news report generation typically lack details, structures, and a high-level strategic awareness for downstream end-user actionability. 
In particular, current large language model (LLM) based approaches such as ChatGPT~\cite{openai2021gpt35} 
% \heng{add citations for chatgpt and gpt-4} 
may return no answers or  incorrect answers to important questions because they cannot process information in real-time or lack fact-checking mechanisms.
Current multi-document news summarization tasks aim to condense the information overlap across a news article cluster into a succinct ($\sim$5 sentence) paragraph \cite{fabbri-etal-2019-multi}. However, real-world situation reports that policymakers and military experts utilize as actual references for their decision-making require a lot more details. For example, recent human-written situation reports \cite{bowen2022russia} on the Russia-Ukraine crisis, publicized online by the United Nations and the U.S. Congressional Research, aggregate critical information with regards to the preluding societal background; onset of the military invasion; development of subsequent military operation events; economic, political, and humanitarian ramifications; as well as outlook forecast – in over twenty dense pages. Moreover, generating a situation report is \textit{not only} about increasing the sample size of relevant news source retrieval \cite{brody-kantor-2011-automatic} and the length of subsequent summary generation, \textit{but also} about maintaining a structured organization ({e.g.,} chapters and sections) for complex event understanding, with an agenda for strategic planning in mind. As such, the current application of AI techniques to situation analysis and report generation is far from mature. %, and cannot achieve the level of structured organization and strategic planning necessary required for complex event understanding in situation analysis. 

%related events addressing a similar subtopic/query as individual chapters.

In this work, we present \textbf{SmartBook}, %a novel formulation  
a generalizable framework designed to assist human analysts in situation report generation. SmartBook ingests data from multiple sources to generate a comprehensive report with real-time information being updated in a timely fashion. The framework automatically discovers questions important for situation analysis and gathers salient information for generating the report. For each question, the report comprises summaries with tactical information coming from relevant claims within news articles, and grounded to the corresponding trustworthy factual evidence.

%SmartBook automatically comes up with sets of questions (i.e.,  chapter headers and topic queries) important to situation analysis, and for each question, summarizes the relevant claims from news articles spanning multiple sources. %and languages
%Two key merits of SmartBook are that it leverages state-of-the-art information extraction techniques to help identify information of interest across the documents, and also grounds the corresponding information elements to the summary in the situation report.%, forming an interactive citation network of complex news events. 
%\heng{we need to emphasize more about the advantages of Smartbook: 1. comprehensive: multi-source; 2. timely: keep up to date; 3. trusthworthy: show claim evidence, summary has citations; 4. salient: automatically discover important questions to ask. And these points need to be repeated briefly in the contribution bullets.}

As a case study, we consider the case of automatic situation report generation for the Ukraine-Russia crisis. 
This scenario is particularly challenging, given the complexity of the conflict, the politicization of the issue, and the need to constantly update the reports as the situation evolves. Overall, manually creating situation reports for the Ukraine-Russia crisis requires analysts to have a combination of technical skills, analytical expertise, and strategic thinking. Our proposed automatic situation report is created from a corpus of news articles and is organized in the form of timelines, with new timelines being added in real-time as the conflict progresses. Each timeline covers a 2-week duration and major events within that duration serve as the corresponding chapters. The report incorporates a logical structure for each chapter by creating section headings in the form of automatically identified strategic questions, and providing grounded query-focused summaries within the individual sections. Figure \ref{fig:1} shows an example from SmartBook for the Ukraine-Russia crisis, with the structured hierarchy in the form of timelines, chapters and corresponding sections.
%We further devise a suite of evaluation metrics for assessing the quality of situation report generation, including comparing the edit distance by human experts.

Our contributions can be summarized as follows:
\begin{itemize}
    \item We identify an exciting new problem domain and task formulation of \textit{automated situation report generation}, which is not meant to replace human analysts but rather to assist them.
    %\item We propose \textsc{\textbf{SmartBook}}, for identifying strategically important topic queries and performing claim summarization, utilizing information extraction and directed question answering techniques.
    \item We propose \textsc{\textbf{SmartBook}}, an automated framework that generates comprehensive, real-time situation reports from various sources, by discovering critical strategic questions and grounding the report with trustworthy factual evidence.
    \item We quantitatively and qualitatively evaluate our automatic situation report generation approach for the Ukraine-Russia crisis and demonstrate that SmartBook is well-received by expert analysts in terms of informativeness and effectiveness for strategic planning.
    \item We show that SmartBook outperforms a web search $+$ large language model based generation approach (akin to Bing + ChatGPT) by up to 9\% in terms of relevance and strategicness of information presented. 
    \item Finally, we contribute a publicly accessible UI\footnote{\url{https://blenderdemo.com/smartbook}} and code\footnote{\url{https://github.com/blender-nlp/SmartBook}} for SmartBook so that others can explore and improve upon.
\end{itemize}

%\begin{figure*}
%\includegraphics[width=\textwidth]{figures/SmartBook_sugar_candy.pdf}
%\caption{An example from SmartBook for the Ukraine-Russia Crisis, corresponding to the section ``\textit{How might the delays in grain shipment impact global food supply}'' regarding a chapter on the ``\textit{Black Sea Grain Deal}'' from the \textit{Oct 16 - Oct 30} timeline.} 
%\label{fig:1}
%\end{figure*}

\begin{figure*}
\includegraphics[width=\textwidth]{figures/SmartBook_example.png}
\caption{Figure showing an example from SmartBoook for the Ukraine-Russia Crisis. SmartBook is organized by timelines, with each timeline containing chapters and corresponding sections. The section headings are strategic questions with the section content being grounded summaries that are linked to relevant claims, each highlighed with factual evidence and knowledge elements. The entities and events (along with the corresponding types), which comprise the associated knowledge elements, are highlighted in red and blue respectively.} 
\label{fig:1}
\end{figure*}

%\wordcount