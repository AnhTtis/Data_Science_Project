\begin{table}[t]
    \centering
    \footnotesize
    \renewcommand{\arraystretch}{1.5}
    \begin{tabular}{p{40em}}
      % \textbf{SmartBook Summary}   & \textbf{Human-edited Summary} \\
      \hline
      \textbf{SmartBook Summary}: Russia has reportedly stepped up its use of kamikaze drones in its assault against Ukraine. The increased reliance on kamikaze drones over artillery fire likely signals a paradigm shift in Russian tactics a shift introduced to counter high mobility offensive probing by Ukrainian forces.  \\
      \hline
      \textbf{Analyst-edited Summary}: Russia has reportedly stepped up its use of kamikaze drones in its assault against Ukraine. \textcolor{blue}{The aircraft are called kamikaze drones because they attack once and don’t come back.} The increased reliance on kamikaze drones over artillery fire likely signals a paradigm shift in Russian tactics - a shift introduced to counter high mobility offensive probing by Ukrainian forces. \textcolor{blue}{Their low price means the drones can be deployed in large numbers and they hover before they strike, so they have a psychological effect on civilians as they watch and wait for them to strike. These drones allow Russia to target Ukrainians far away from the front line, away from the primary battle space.} \textcolor{red}{The emergence of swarms of drones in Ukraine is part of a shift in the nature of the Russian offensive, which some speculate indicates that Moscow may be running low on long-range missiles.} \\
      \hline
    \end{tabular}
    \caption{Example showing intelligence analyst edits for an automatic machine generated SmartBook summary on the use of ``kamikaze'' drones in the Ukraine-Russia crisis. Text that has been added by the expert is colored, with \textcolor{blue}{blue} corresponding to additional tactical information, whereas \textcolor{red}{red} corresponds to insights/conclusions added by the analyst.}
    \label{tab:human_edits}
    %\vspace{-1em}
\end{table}

\begin{comment}
\begin{table}[htbp] %H]
\begin{tabular}{ |p{1.3cm}|p{0.9 cm}|p{8.5 cm}| } 
 \hline
\textbf{Time} & \textbf{Topic} & \textbf{Summaries} \\ 
 \hline %Why has Russia step up its use of "kamikaze drones" in its assault against Ukraine?
 Oct 15-30 & Drones & {\textbf{\underline{M}}: {\smallssp \textcolor{red}{Russia has reportedly stepped up its use of kamikaze drones in its assault against Ukraine}. The increased reliance on kamikaze drones over artillery fire likely signals a paradigm shift in Russian tactics a shift introduced to counter high mobility offensive probing by Ukrainian forces.} \newline 
 \textbf{\underline{M} (edited)}: {\smallssp \textcolor{red}{Russia has reportedly stepped up its use of kamikaze drones in its assault against Ukraine}. The aircraft are called kamikaze drones because they attack once and don’t come back. The increased reliance on kamikaze drones over artillery fire likely signals a paradigm shift in Russian tactics - a shift introduced to counter high mobility offensive probing by Ukrainian forces. Their low price means the drones can be deployed in large numbers and they hover before they strike, so they have a psychological effect on civilians as they watch and wait for them to strike. These drones allow Russia to target Ukrainians far away from the front line, away from the primary battle space. The emergence of swarms of drones in Ukraine is part of a shift in the nature of the Russian offensive, which some speculate indicates that \textcolor{blue}{Moscow may be running low on long-range missiles}.} \newline \textbf{\underline{H}}}: {\smallssp \textcolor{red}{The increased use of Iranian Shahed-136 kamikaze drones} that can be launched from a distance and explode on impact could indicate strength or desperation \textcolor{red}{from Russian forces}. Experts take there use as a sign that \textcolor{blue}{Russia is lacking precision-guided missiles} that can target the howitzers that Ukraine has deployed to great effect. But it also shows that Russian ties with Iran remain strong. The Shahed-136 is an inexpensive autonomous weapon that affords Russia the ability to inflict damage to Ukrainian infrastructure and civilian targets without direct risk to military personnel since they can shoot and scoot. Additionally, Russia may be resorting to these new weapons if it has used up the majority of its guided missiles. The drones are both military weapons and psychological weapons. Attacks on major cities that are supposed to be well-defended against aerial threats demonstrate that Russia still has the capacity to inflict damage, whether military or civilian targets are struck. They can also slip past Ukraine’s air defenses — or force the military to use its limited air defense resources to neutralize them before they can strike their target. The drones offer Russia a way to cause more civilian and military casualties in Ukraine while keeping Russia in the fight while saving their own reserves.} \\ 
 \hline
\end{tabular}
\vspace{0.1cm}
\caption{A comparison of our machine generated (\textbf{M}) chapter summaries and human analyst (\textbf{H}) written chapter summaries. We highlight the similar content in color.}
\label{tab:my_label}
\end{table}
\end{comment}