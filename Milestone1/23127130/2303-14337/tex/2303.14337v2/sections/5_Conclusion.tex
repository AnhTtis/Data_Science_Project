\section{Conclusion and Future Extensions}\label{sec13}

In this paper, we present the novel task of automated situation report generation, which is pertinent for timely and scalable complex event understanding of emergent crises. This problem formulation differs from the traditional multi-document news summarization domain in that it requires comprehensive, broad coverage of important information, which is to be presented in a structured format that is spread across a timeline.
%This problem formulation differs from the traditional multi-document news summarization domain in that it requires comprehensive, broad coverage of important information from a large corpus of documents, spanning multiple sources and languages. 
In addition, the vast amount of information needs to be organized into chapters and sections with a strategic structure to facilitate analysis and planning. To address these challenges, we propose \textit{SmartBook}, a generalizable framework for clustering news topics and summarizing news claims automatically. It is important to note that while SmartBook offers automated assistance in generating situation reports, it is not meant to replace human analysts, but rather to assist them in the process. 


For future work, we aim to explore the following avenues to improve the reliability and expand the scope of SmartBook:

\begin{itemize}
    %\item \textbf{Incorporating multimodal information}: \\
    \item \textbf{Controlling the bias of sources:} Different news sources present information from varying angles, with different levels of detail and interpretation. Thereby, we plan to control for the bias of news sources used in the creation of SmartBook to help cross-check information and avoid the potential pitfalls of relying on a single news source or a single interpretation of events. Further, alternate perspectives/hypotheses can be identified by independently considering news sources on the left, in the center, and on the right. \\
    \item \textbf{Adding more languages:} The world is diverse, and news articles from different languages allow capturing different perspectives, insights, and information relating to global events, that might not be available in a single language. For this reason, we aim to add news articles from more languages into SmartBook to present a more comprehensive and accurate picture. Further, the ability to ingest information from different languages also enables analysts to understand local customs, cultures, and nuances that may influence the interpretation of events.\\    
    \item \textbf{Verifying the extracted claims:} Given that situation reports contribute towards strategic planning, it is important to present accurate information to prevent misguided decisions and actions. Hence, we intend to provide a verification score, for each claim presented within SmartBook, which can be interpreted as a measure of trustworthiness. Claims with lower scores signal the need for additional oversight by the intelligence analysts before incorporating into them the final report. 
\end{itemize}



%For future work, we aim to incorporate news sources from additional languages as input data to expand on the coverage and usefulness of SmartBook formulation. \\

%\noindent \textbf{Controlling the bias of sources:} 
%\textbf{TODO: Revanth}
%\noindent \textbf{Verification of Claims:} 
%\textbf{TODO: Revanth}
%\noindent \textbf{Adding More Languages:}
%\textbf{TODO: Yi}
%\noindent \textbf{Incorporating Multimodality:}
%\textbf{TODO: Manling}