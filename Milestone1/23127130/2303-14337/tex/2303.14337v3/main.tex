%%
%% This is file `sample-manuscript.tex',
%% generated with the docstrip utility.
%%
%% The original source files were:
%%
%% samples.dtx  (with options: `manuscript')
%% 
%% IMPORTANT NOTICE:
%% 
%% For the copyright see the source file.
%% 
%% Any modified versions of this file must be renamed
%% with new filenames distinct from sample-manuscript.tex.
%% 
%% For distribution of the original source see the terms
%% for copying and modification in the file samples.dtx.
%% 
%% This generated file may be distributed as long as the
%% original source files, as listed above, are part of the
%% same distribution. (The sources need not necessarily be
%% in the same archive or directory.)
%%
%% Commands for TeXCount
%TC:macro \cite [option:text,text]
%TC:macro \citep [option:text,text]
%TC:macro \citet [option:text,text]
%TC:envir table 0 1
%TC:envir table* 0 1
%TC:envir tabular [ignore] word
%TC:envir displaymath 0 word
%TC:envir math 0 word
%TC:envir comment 0 0
%%
%%
%% The first command in your LaTeX source must be the \documentclass command.
%%%% Small single column format, used for CIE, CSUR, DTRAP, JACM, JDIQ, JEA, JERIC, JETC, PACMCGIT, TAAS, TACCESS, TACO, TALG, TALLIP (formerly TALIP), TCPS, TDSCI, TEAC, TECS, TELO, THRI, TIIS, TIOT, TISSEC, TIST, TKDD, TMIS, TOCE, TOCHI, TOCL, TOCS, TOCT, TODAES, TODS, TOIS, TOIT, TOMACS, TOMM (formerly TOMCCAP), TOMPECS, TOMS, TOPC, TOPLAS, TOPS, TOS, TOSEM, TOSN, TQC, TRETS, TSAS, TSC, TSLP, TWEB.
% \documentclass[acmsmall]{acmart}

%%%% Large single column format, used for IMWUT, JOCCH, PACMPL, POMACS, TAP, PACMHCI
% \documentclass[acmlarge,screen]{acmart}

%%%% Large double column format, used for TOG
% \documentclass[acmtog, authorversion]{acmart}

%%%% Generic manuscript mode, required for submission
%%%% and peer review
\documentclass[manuscript]{acmart}
%% Fonts used in the template cannot be substituted; margin 
%% adjustments are not allowed.
%%
%% \BibTeX command to typeset BibTeX logo in the docs
\AtBeginDocument{%
  \providecommand\BibTeX{{%
    \normalfont B\kern-0.5em{\scshape i\kern-0.25em b}\kern-0.8em\TeX}}}

%% Rights management information.  This information is sent to you
%% when you complete the rights form.  These commands have SAMPLE
%% values in them; it is your responsibility as an author to replace
%% the commands and values with those provided to you when you
%% complete the rights form.
\setcopyright{acmcopyright}
\copyrightyear{2018}
\acmYear{2018}
\acmDOI{XXXXXXX.XXXXXXX}

%% These commands are for a PROCEEDINGS abstract or paper.
\acmConference[Conference acronym 'XX]{Make sure to enter the correct
  conference title from your rights confirmation emai}{June 03--05,
  2018}{Woodstock, NY}
%
%  Uncomment \acmBooktitle if th title of the proceedings is different
%  from ``Proceedings of ...''!
%
\acmBooktitle{Woodstock '18: ACM Symposium on Neural Gaze Detection,
 June 03--05, 2018, Woodstock, NY} 
\acmPrice{15.00}
\acmISBN{978-1-4503-XXXX-X/18/06}


%%
%% Submission ID.
%% Use this when submitting an article to a sponsored event. You'll
%% receive a unique submission ID from the organizers
%% of the event, and this ID should be used as the parameter to this command.
%%\acmSubmissionID{123-A56-BU3}

%%
%% For managing citations, it is recommended to use bibliography
%% files in BibTeX format.
%%
%% You can then either use BibTeX with the ACM-Reference-Format style,
%% or BibLaTeX with the acmnumeric or acmauthoryear sytles, that include
%% support for advanced citation of software artefact from the
%% biblatex-software package, also separately available on CTAN.
%%
%% Look at the sample-*-biblatex.tex files for templates showcasing
%% the biblatex styles.
%%

%%
%% The majority of ACM publications use numbered citations and
%% references.  The command \citestyle{authoryear} switches to the
%% "author year" style.
%%
%% If you are preparing content for an event
%% sponsored by ACM SIGGRAPH, you must use the "author year" style of
%% citations and references.
%% Uncommenting
%% the next command will enable that style.
%%\citestyle{acmauthoryear}

\usepackage{graphicx}
\usepackage{float}
\usepackage{graphicx,caption}
\usepackage{multirow}
\newcommand{\quickwordcount}[1]{%
  \immediate\write18{texcount -1 -sum -merge -q #1.tex output.bbl > #1-words.sum }%
  \input{#1-words.sum}%
}
\usepackage{colortbl} %for table row color
\usepackage{soul}
\sethlcolor{green}
\usepackage{wrapfig}

\newcommand\wordcount{
    \immediate\write18{texcount -sub=section \jobname.tex  | grep "Section" | sed -e 's/+.*//' | sed -n \thesection p > 'count.txt'}
(\input{count.txt}words)}

\NewDocumentCommand{\revanth}
{ mO{} }{\textcolor{brown}{\textsuperscript{\textit{Revanth}}\textsf{\textbf{\small[#1]}}}}

\NewDocumentCommand{\daniel}
{ mO{} }{\textcolor{green}{\textsuperscript{\textit{Daniel}}\textsf{\textbf{\small[#1]}}}}

\newcommand{\name}{\textsc{SmartBook}}

\NewDocumentCommand{\heng}
{ mO{} }{\textcolor{red}{\textsuperscript{\textit{Heng}}\textsf{\textbf{\small[#1]}}}}

\settopmatter{printacmref=false} % Removes citation information below abstract
\renewcommand\footnotetextcopyrightpermission[1]{} % removes footnote with conference information in first column

%%
%% end of the preamble, start of the body of the document source.
\begin{document}
%%
%% The "title" command has an optional parameter,
%% allowing the author to define a "short title" to be used in page headers.
\title{\name{}: AI-Assisted Situation Report Generation for Intelligence Analysts} %\heng{Nature papers usually use present tense, may change the title to "AI Generated Situation Report Assists Intelligence Analysis"}}
\pagestyle{plain}
%%
%% The "author" command and its associated commands are used to define
%% the authors and their affiliations.
%% Of note is the shared affiliation of the first two authors, and the
%% "authornote" and "authornotemark" commands
%% used to denote shared contribution to the research.
\author{Revanth Gangi Reddy}
%\authornote{Both authors contributed equally to this research.}
\email{revanth3@illinois.edu}
%\orcid{1234-5678-9012}
%\author{G.K.M. Tobin}
%\authornotemark[1]
%\email{webmaster@marysville-ohio.com}
\affiliation{%
  \institution{University of Illinois at Urbana-Champaign}
  \city{Champaign}
  \state{Illinois}
  \country{USA}
}

\author{Daniel Lee}
\affiliation{%
  \institution{University of Calgary}
  %\streetaddress{1 Th{\o}rv{\"a}ld Circle}
  \city{Calgary}
  \city{Alberta}
  \country{Canada}}
\email{daniel.lee1@ucalgary.ca}

\author{Yi Fung}
\email{yifung2@illinois.edu}
\affiliation{%
  \institution{University of Illinois at Urbana-Champaign}
  \city{Champaign}
  \state{Illinois}
  \country{USA}
}

\author{Khanh Duy Nguyen
 }
\email{knguye71@illinois.edu}
\affiliation{%
  \institution{University of Illinois at Urbana-Champaign}
  \city{Champaign}
  \state{Illinois}
  \country{USA}
}

\author{Qi Zeng}
\email{qizeng2@illinois.edu}
\affiliation{%
  \institution{University of Illinois at Urbana-Champaign}
  \city{Champaign}
  \state{Illinois}
  \country{USA}
}

\author{Manling Li}
\email{manling.li@northwestern.edu}
\affiliation{%
  \institution{Northwestern University}
  \city{Chicago}
  \state{Illinois}
  \country{USA}}

\author{Ziqi Wang}
\email{ziqiw9@illinois.edu}
\affiliation{%
  \institution{University of Illinois at Urbana-Champaign}
  \city{Champaign}
  \state{Illinois}
  \country{USA}
}

%author{Paul Sullivan}
%\authornote{We deeply regret that one of the project contributors, Paul, passed away before the publication of this paper. His invaluable contributions and enduring commitment to knowledge are greatly missed and deeply appreciated.}
%\affiliation{%
%  \institution{INTELPOINT}
%  \city{Washington}
%  \state{District of Columbia}
%  \country{USA}}
%\email{paul.sullivan@intpt.net}

\author{Clare Voss}
\affiliation{%
  \institution{Army Research Laboratory}
  \city{Adelphi}
  \state{Maryland}
  \country{USA}}
\email{clare.r.voss.civ@army.mil}

\author{Heng Ji}
\affiliation{%
  \institution{University of Illinois at Urbana-Champaign}
  \city{Champaign}
  \state{Illinois}
  \country{USA}
}

\email{hengji@illinois.edu}
%%
%% By default, the full list of authors will be used in the page
%% headers. Often, this list is too long, and will overlap
%% other information printed in the page headers. This command allows
%% the author to define a more concise list
%% of authors' names for this purpose.
%\renewcommand{\shortauthors}{Trovato and Tobin, et al.}

%%
%% The abstract is a short summary of the work to be presented in the
%% article.


Over the past few years, there has been a significant amount of research focused on studying the ReLU activation function, with the aim of achieving neural network convergence through over-parametrization. However, recent developments in the field of Large Language Models (LLMs) have sparked interest in the use of exponential activation functions, specifically in the attention mechanism.

Mathematically, we define the neural function $F: \R^{d \times m} \times  \mathbb{R}^d \rightarrow \mathbb{R}$ using an exponential activation function. Given a set of data points with labels $\{(x_1, y_1), (x_2, y_2), \dots, (x_n, y_n)\} \subset \mathbb{R}^d \times \mathbb{R}$ where $n$ denotes the number of the data. Here $F(W(t),x)$ can be expressed as $F(W(t),x) := \sum_{r=1}^m a_r \exp(\langle w_r, x \rangle)$, where $m$ represents the number of neurons, and $w_r(t)$ are weights at time $t$. It's standard in literature that $a_r$ are the fixed weights and it's never changed during the training. We initialize the weights $W(0) \in \mathbb{R}^{d \times m}$ with random Gaussian distributions, such that $w_r(0) \sim \mathcal{N}(0, I_d)$ and initialize $a_r$ from random sign distribution for each $r \in [m]$.

Using the gradient descent algorithm, we can find a weight $W(T)$ such that $\| F(W(T), X) - y \|_2 \leq \epsilon$ holds with probability $1-\delta$, where $\epsilon \in (0,0.1)$ and $m = \Omega(n^{2+o(1)}\log(n/\delta))$. To optimize the over-parametrization bound $m$, we employ several tight analysis techniques from previous studies [Song and Yang arXiv 2019, Munteanu, Omlor, Song and Woodruff ICML 2022]. 

 


%%
%% The code below is generated by the tool at http://dl.acm.org/ccs.cfm.
%% Please copy and paste the code instead of the example below.
%%
\begin{comment}
\begin{CCSXML}
<ccs2012>
   <concept>
       <concept_id>10003120.10003121</concept_id>
       <concept_desc>Human-centered computing~Human computer interaction (HCI)</concept_desc>
       <concept_significance>500</concept_significance>
       </concept>
   <concept>
       <concept_id>10010147.10010178.10010179.10010182</concept_id>
       <concept_desc>Computing methodologies~Natural language generation</concept_desc>
       <concept_significance>500</concept_significance>
       </concept>
 </ccs2012>
\end{CCSXML}

\ccsdesc[500]{Human-centered computing~Human computer interaction (HCI)}
\ccsdesc[500]{Computing methodologies~Natural language generation}
\end{comment}


%%
%% Keywords. The author(s) should pick words that accurately describe
%% the work being presented. Separate the keywords with commas.
%\keywords{Human-AI collaboration, Intelligence Analysis, News Summarization}

%% A "teaser" image appears between the author and affiliation
%% information and the body of the document, and typically spans the
%% page.
\begin{comment}
    \begin{teaserfigure}
  \includegraphics[width=\textwidth]{sampleteaser}
  \caption{Seattle Mariners at Spring Training, 2010.}
  \Description{Enjoying the baseball game from the third-base
  seats. Ichiro Suzuki preparing to bat.}
  \label{fig:teaser}
\end{teaserfigure}
\end{comment}


%\received{20 February 2007}
%\received[revised]{12 March 2009}
%\received[accepted]{5 June 2009}

%%
%% This command processes the author and affiliation and title
%% information and builds the first part of the formatted document.
\maketitle

\begin{comment}
\textbf{Nature Content}\\
Abstract: \quickwordcount{nature_sections/abstract} out of 200 words\\
Introduction: \quickwordcount{nature_sections/introduction} words\\
Results: \quickwordcount{nature_sections/results} words\\
Discussion: \quickwordcount{nature_sections/discussion} words\\
Total MainText: \quickwordcount{nature_sections/introduction} + \quickwordcount{nature_sections/results} + \quickwordcount{nature_sections/discussion} out of 5000 words\\
Methods: \quickwordcount{nature_sections/methodology} out of 3000 words\\
\end{comment}

\begin{comment}
\textbf{CHI Content}\\
Abstract: \quickwordcount{sections/0_Abstract} words\\
Introduction: \quickwordcount{sections/1_Intro} words\\
Related Work: \quickwordcount{sections/2_Related_Work} words\\
Formative Study: \quickwordcount{sections/3_Formative_Study} words \\
Collaborative Design: \quickwordcount{sections/4_Collaborative_Design} words \\
Method: \quickwordcount{sections/5_SmartBook} words\\
Eval Overview: \quickwordcount{sections/6_System_Evaluation_Overview} words \\
User Study: \quickwordcount{sections/7_User_Study} words \\
Content Review: \quickwordcount{sections/8_Content_Review} words \\
Discussion: \quickwordcount{sections/9_Discussion} words \\
Conclusion: \quickwordcount{sections/10_Conclusion} words \\
Appendix: \quickwordcount{sections/appendix} words \\
Total: \quickwordcount{sections/0_Abstract} + \quickwordcount{sections/1_Intro} + \quickwordcount{sections/2_Related_Work} + \quickwordcount{sections/3_Formative_Study} + 
\quickwordcount{sections/4_Collaborative_Design} +
\quickwordcount{sections/5_SmartBook} + \quickwordcount{sections/6_System_Evaluation_Overview} + 
\quickwordcount{sections/7_User_Study} +
\quickwordcount{sections/8_Content_Review} +
\quickwordcount{sections/9_Discussion} + 
\quickwordcount{sections/10_Conclusion} words
\end{comment}

\section{Introduction}
\label{sec:introduction}
% \begin{itemize}
%     % Diffusion of FL
%     \item {\st{Diffusion of FL}}
%     % Security threats to FL
%     \item {\st{Security threats to FL with particular focus on model poisoning}}
%     % Limitations of existing countermeasures
%     \item {\st{Current countermeasures (e.g., KRUM) and their limitations}}
%     % Proposed method and its advantages
%     \item {\st{Intuitive description of the proposed method and its difference (i.e., advantages) w.r.t. state of the art}}
%     % Main contributions
%     \item {\st{Summary of the main contributions of this work}}
%     % Paper's structure and organization
%     \item {\st{Paper's structure and organization}}
% \end{itemize}

% Diffusion of FL
Recently, {\em federated learning} (FL) has emerged as the leading paradigm for training distributed, large-scale, and privacy-preserving machine learning (ML) systems~\cite{mcmahan2017googleai,mcmahan2017aistats}. 
The core idea of FL is to allow multiple edge clients to collaboratively train a shared, global model without disclosing their local private training data.
%Specifically, an FL system consists of a central server and many edge clients; 
A typical FL round involves the following steps: {\em(i)} the server randomly picks some clients and sends them the current, global model; {\em(ii)} each selected client locally trains its model with its own private data; then, it sends the resulting local model to the server;\footnote{Whenever we refer to global/local model, we mean global/local model {\em parameters}.} {\em(iii)} the server updates the global model by computing an \emph{aggregation function}, usually the average (FedAvg), on the local models received from clients.
% \begin{enumerate}
%     \item[{\em(i)}] the server sends the current, global model to the clients and appoints some of them for training;
%     \item[{\em(ii)}] each selected client locally trains its copy of the global model with its own private data; then, it sends the resulting local model back to the server;\footnote{Whenever we refer to global/local model, we mean global/local model {\em parameters}.}
%     \item[{\em(iii)}] the server updates the global model by computing an \emph{aggregation function} on the local models received from clients (by default, the average, also referred to as FedAvg~\cite{mcmahan2017aistats}).
% \end{enumerate}
This process goes on until the global model converges. %(e.g., after a certain number of rounds or other similar stopping criteria).
%\\
% The advantages of FL over the traditional, centralized learning paradigm are undoubtedly clear in terms of flexibility/scalability (clients can join/disconnect from the FL network dynamically), network communications (only model weights\footnote{We will use \textit{parameters} and \textit{weights} interchangeably.} are exchanged between clients and server), and privacy (each client's private training data is kept local at the client's end and not uploaded to the server).
\\
% Security threats to FL
%However, the growing adoption of FL also raises security concerns~\cite{costa2022covert}, particularly about its confidentiality, integrity, and availability.
Although its advantages over standard ML, FL also raises security concerns~\cite{costa2022covert}. %, particularly about its confidentiality, integrity, and availability~\cite{costa2022covert}.
% OLD, LONG VERSION
% Indeed, some work deals with privacy leakage that may expose the local data of some clients~\cite{melis2019sp}. 
% A large body of work, instead, investigates attacks that usually aim to detriment the predictive accuracy of the learned global model. For instance, \emph{data poisoning} attacks achieve this goal by letting an adversary pollute the training set of some corrupt FL clients with maliciously crafted examples~\cite{jagielski2018sp}.
% Similarly, in \emph{model poisoning} the attacker attempts to tweak the global model weights~\cite{bhagoji2019pmlr} by directly perturbing the local model's weights of some infected FL clients before these are sent to the central server for aggregation, usually via so-called Byzantine attacks. 
% It turns out that Byzantine model poisoning attacks severely impact standard FedAvg; therefore, more robust aggregation functions must be designed to make FL systems secure.
Here, we focus on \emph{untargeted model poisoning} attacks~\cite{bhagoji2019pmlr}, where an adversary attempts to tweak the global model weights %\footnote{We will use the terms \textit{parameters} and \textit{weights} interchangeably.} 
by directly perturbing the local model's parameters of some infected clients before these are sent to the central server for aggregation.
In doing so, the adversary aims to jeopardize the global model \textit{indiscriminately} at inference time.
Such model poisoning attacks severely impact standard FedAvg; therefore, more robust aggregation functions must be designed to secure FL systems.
\\
% In this paper, we focus on designing a novel robust aggregation scheme at the server's end to contrast the effect of Byzantine model poisoning attacks.
%
% Current countermeasures and their limitations
%Several countermeasures have been proposed in the literature to combat model poisoning attacks on FL systems.
% Some methods use simple statistics more robust than plain average to smooth the impact of malicious updates (e.g., Trimmed Mean and FedMedian~\cite{yin2018icml}). 
% Other defenses implement outlier detection techniques to discard malicious updates from the aggregation performed at the server's end. Those are either based on heuristics (e.g., Krum/Multi-Krum~\cite{blanchard2017nips} and Bulyan~\cite{mhamdi2018pmlr}) or data-driven approaches (e.g., K-means clustering~\cite{shen2016acm} or DnC via spectral analysis~\cite{shejwalkar2021ndss}). 
% Finally, some strategies rely on a centralized ``source of trust'' to spot potential malicious updates (e.g., FLTrust~\cite{cao2020fltrust}).
% Several countermeasures have been proposed in the literature to combat model poisoning attacks on FL systems, i.e., to discard possible malicious local updates from the aggregation performed at the server's end. 
% These techniques range from simple statistics more robust than plain average (e.g., Trimmed Mean and FedMedian~\cite{yin2018icml}) to outlier detection heuristics (e.g., Krum/Multi-Krum~\cite{blanchard2017nips} and Bulyan~\cite{mhamdi2018pmlr}) or data-driven approaches (e.g., spectral analysis via K-means clustering~\cite{shen2016acm} or spectral analysis), or methods based on ``source of trust'' (e.g., FLTrust~\cite{cao2020fltrust}).
% OLD, LONG VERSION
%Several countermeasures have been proposed in the literature to combat Byzantine model poisoning attacks on FL systems.
% Descriptive statistics
% For example, Trimmed Mean and FedMedian aggregate local model updates using more robust statistics than standard average~\cite{yin2018icml}.
%
% % Heuristics for outlier detection
% Many existing Byzantine-resilient strategies implement some outlier detection heuristics to discard the model updates sent by potentially malicious clients from the input of the aggregation function.
% One of the most popular heuristics is Krum~\cite{blanchard2017nips}.
% This strategy tries to mitigate the impact of Byzantine attacks by selecting as a global model the local model with the smallest sum of Euclidean distances to {\em all} the other local models.
% Although powerful, Krum requires the server to know (or, at least, estimate) the number of malicious FL clients upfront, which is generally impossible in a realistic attack scenario. %
% Moreover, Krum may become ineffective for complex, high-dimensional model parameter spaces due to the curse of dimensionality.
% Bulyan~\cite{mhamdi2018pmlr} tries to overcome this issue by combining Krum with a variant of Trimmed Mean.
% % Data-driven outlier detection
% Other strategies use data-driven outlier detection techniques -- e.g., via K-means clustering~\cite{shen2016acm} -- to spot potential malicious local model updates. 
% %For instance, Shen et al. propose to cluster local model updates with K-means and thus identify outliers.
%
% % Other techniques
% As far as the server is concerned, any local model received can be from a potential malicious client. 
% FLTrust~\cite{cao2020fltrust} assumes the server acts as a client, i.e., trains a local model on an additional {\em trustworthy} dataset at the server's end and compares it against all the local models from other clients. 
% This way, the server can rely on some ``source of trust'' when discarding potentially malicious clients.
%\\
% Limitations of existing Byzantine-resilient strategies
Unfortunately, existing defense mechanisms either rely on simple heuristics (e.g., Trimmed Mean and FedMedian by~\cite{yin2018icml}) or need strong and unrealistic assumptions to work effectively (e.g., foreknowledge or estimation of the number of malicious clients in the FL system, as for Krum/Multi-Krum~\cite{blanchard2017nips} and Bulyan~\cite{mhamdi2018pmlr}, which, however, cannot exceed a fixed threshold).
Furthermore, outlier detection methods using K-means clustering~\cite{shen2016acm} or spectral analysis like DnC~\cite{shejwalkar2021ndss} do not directly consider the temporal evolution of local model updates received.
Finally, strategies like FLTrust~\cite{cao2020fltrust} require the server to collect its own dataset and act as a proper client, thereby altering the standard FL protocol.
\\
% OLD, LONG VERSION
% Overall, existing Byzantine-resilient strategies are either simple heuristics (e.g., FedMedian) or, if they are more complex, they rely on strong and unrealistic assumptions to work effectively (e.g., knowing the number of malicious clients in the FL system in advance, as for Krum and alike).
% Furthermore, data-driven outlier detection methods do not consider the temporary evolution of local model updates received (e.g., K-means clustering). 
% Finally, strategies like FLTrust requires the server to collect its own dataset and act as a proper client, thereby altering the standard FL protocol.
%
% Description of the proposed method
This work introduces a novel pre-aggregation \textit{filter} robust to untargeted model poisoning attacks. Notably, this filter $(i)$ operates without requiring prior knowledge or constraints on the number of malicious clients and $(ii)$ inherently integrates temporal dependencies. 
The FL server can employ this filter as a preprocessing step before applying \textit{any} aggregation function, be it standard like FedAvg or robust like Krum or Bulyan.
Specifically, we formulate the problem of identifying corrupted updates as a multidimensional (i.e., matrix-valued) time series anomaly detection task. 
The key idea is that legitimate local updates, resulting from well-calibrated iterative procedures like stochastic gradient descent (SGD) with an appropriate learning rate, show \textit{higher predictability} compared to malicious updates. This hypothesis stems from the fact that the sequence of gradients (thus, model parameters) observed during legitimate training exhibit regular patterns, as validated in Section~\ref{subsec:intuition}. %until convergence. 
%This regularity may be more pronounced for smooth convex loss functions, but it can still be captured within an appropriate time window, even for more complex and convoluted loss surfaces. 
%We provide evidence of this claim in Appendix~B, where we show that the average mutual information (i.e., ``predictability''), calculated over pairs of legitimate model updates sent at different FL rounds, is significantly higher than the corresponding computation for a malicious client.
\\
Inspired by the matrix autoregressive (MAR) framework for multidimensional time series forecasting~\cite{chen2021je}, we propose the FLANDERS ({\em \textbf{F}ederated \textbf{L}earning meets \textbf{AN}omaly \textbf{DE}tection for a \textbf{R}obust and \textbf{S}ecure}) filter.
The main advantages of FLANDERS over existing strategies like FLDetector~\cite{zhao2020multivariate} are its resilience to large-scale attacks, where $50\%$ or more FL participants are hostile, and the capability of working under realistic non-iid scenarios.
We attribute such a capability to two key factors: $(i)$ FLANDERS works without knowing a priori the ratio of corrupted clients, and $(ii)$ it embodies temporal dependencies between intra- and inter-client updates, quickly recognizing local model drifts caused by evil players. Below, we summarize our main contributions:

\begin{itemize}
\item[{\em(i)}]
We provide empirical evidence that the sequence of models sent by legitimate clients is more predictable than those of malicious participants performing untargeted model poisoning attacks.
\\
\item[{\em(ii)}] 
We introduce FLANDERS, the first pre-aggregation filter for FL robust to untargeted model poisoning based on multidimensional time series anomaly detection.
\\
\item[{\em(iii)}] 
We integrate FLANDERS into Flower,\footnote{\scriptsize{\url{https://flower.dev/}}} a popular FL simulation framework for reproducibility.
\\
\item[{\em(iv)}] 
We show that FLANDERS improves the robustness of the existing aggregation methods under multiple settings: different datasets, client's data distribution (non-iid), models, and attack scenarios.
\\
\item[{\em(v)}] 
We publicly release all the implementation code of FLANDERS along with our experiments.\footnote{\scriptsize{\url{https://anonymous.4open.science/r/flanders_exp-7EEB}}}
\end{itemize}

% Paper's structure and organization
The remainder of the paper is structured as follows. %some related work and the current state-of-the-art solutions to security issues that FL entails. 
Section~\ref{sec:background} covers background and preliminaries. 
In Section~\ref{sec:related}, we discuss related work.
Section~\ref{sec:problem} and Section~\ref{sec:method} describe the problem formulation and the method proposed. % to tackle it. 
Section~\ref{sec:experiments} gathers experimental results. %, and Section~\ref{sec:limitations} discusses some limitations of this work.
Finally, we conclude in Section~\ref{sec:conclusion}.
 %discusses the limitations of this work and draws future research directions.
%reports conclusions and draws perspectives for future research directions.

%%%%%%% OLD %%%%%%%
%to overcome the resilience of Byzantine failures in distributed Stochastic Gradient Descent computations. 
% The strength of Krum is its time complexity, which is linear in the gradient dimension. 
% However, the robustness of the approach is guaranteed for gradient-based learning applications only when the majority of the clients are not compromised. 
% Besides, the aggregation mechanism of Krum, as well as that of similar methods, is robust from a coarse-grained perspective and does not provide solutions to errors and perturbations that may occur at inference time.
%A related approach to~\cite{blanchard2017nips} is the work of Su et al.~\cite{su2016dc}. Here, the authors propose an iterated approximate agreement to tackle a multi-layer scenario attacked by Byzantine agents. 
%However, the method works efficiently on the sole discrete context and it is inapplicable to continuous state environments.
%\gabri{Maybe, we should just talk about the main limitations of existing countermeasures without digging into their details (or, we can just mention Krum as this is the most popular one). I will move the description of all these methods to the Related Work section.}

\section{Method}
\label{s:method}

We consider the 3D euclidean space $\Real^3$ with points $p=(x,y,z)\in\Real^3$. We discretize the unit cube $\gC=[0,1]^3$ with a 3D voxel grid $\gG=\set{p_I}$, with nodes $p_I$ indexed by $I=(i,j,k)$, $i,j,k\in [n]=\set{1,\ldots,n}$, \ie, $p_I=(x_{ijk},y_{ijk},z_{ijk})$. We denote by $h=n^{-1}$, and by $N=n^3$ the total number of nodes.   
We represent our reconstructed surface as a zero level of a scalar function $f$ defined over the cube $\gC$. $f$ is defined by prescribing its values at the grid's nodes $f_I\in\Real$ and trilinear interpolating in each voxel. We will denote by $f(p)$ the interpolated value at point $p$. 

Given an input point cloud consisting of $m$ points $q_k\in\Real^3$ with or without (unit norm) normals $n_k\in \Real^3$, $k\in [m]$, our goal is to compute $f$ so that its zero level set approximates the unknown surface, \ie, 
\begin{equation}
    \gS = \set{p\in\gC \ \vert \ f(p)=0}.
\end{equation}
Our approach to compute $f$ is to minimize a loss function of the form
\begin{equation}
    \gL = \gL_{\text{data}} + \gL_{\text{prior}}
\end{equation}
where 
\begin{equation}\label{e:loss_data}
    \gL_{\text{data}} = \frac{\lambda_{\text{p}}}{m}\sum_{k=1}^m \abs{f(q_k)}^2 + \frac{\lambda_{\text{n}}}{m}\sum_{k=1}^m \norm{\nabla f(q_k) - n_k}^2
\end{equation}
where $\norm{\cdot}$ is the standard euclidean norm in $\Real^3$, $\nabla f(p) \in \Real^3$ is the gradient of $f$ sampled at point $p$. Note that $\nabla f$ is defined in interior of voxels, which is generically where the input points $q_k$ resides. $\gL_{\text{data}}$ is the standard data loss encouraging the zero level to pass through the input points $q_k$, and its normals (defined by gradients of $f$) to coincide with input normals $n_k$. 

The prior, $\gL_{\text{prior}}$, is the main contribution of this work, where we combine two novel losses,
\begin{equation}
    \gL_{\text{prior}} = \lambda_{\text{v}} \gL_{\text{viscosity}} + \lambda_{\text{c}} \gL_{\text{coarea}}
\end{equation}
Intuitively, the viscosity loss optimizes for a smooth Signed Distance Function (SDF) solutions, avoiding auxiliary bad minima of the Eikonal equation, while the coarea loss strives to minimize the area of the zero level surface. Our loss has $4$ hyper-parameters $\lambda_{\text{p}},\lambda_{\text{n}},\lambda_{\text{v}},\lambda_{\text{c}}$. We provide more details on these priors next. 


\subsection{Viscosity Loss}\label{ss:viscosity_loss}
The goal of the viscosity loss is to make $f$ approximate an SDF over $\gC$. Given boundary conditions asking $f$ to vanish on some closed compact surface $\gS$, the SDF solves the Eikonal equation PDE, \ie, $\norm{\nabla f(p)}=1$, in a certain well defined sense (viscosity). This motivated some previous work to directly optimize the Eikonal loss \citep{gropp2020implicit,sitzmann2020implicit}
\begin{equation}\label{e:loss_eikonal}
    \gL_{\text{eikonal}} = \int_\gC \Big (\norm{\nabla f(p)}-1\Big )^2 dp
\end{equation}
\begin{wrapfigure}[14]{r}{0.28\textwidth}\vspace{-15pt}
  \begin{center}
    \includegraphics[width=0.25\textwidth]{figs/illustrations/eikonl_1d.png}
  \end{center}
  \caption{Two global minimizers of the Eikonal loss over a grid in 1D. Top solution is not an SDF. }\label{fig:eikonal_1d}
\end{wrapfigure}
Unfortunately, the Eikonal loss has many undesirable minima which are not SDFs. Figure \ref{fig:eikonal_1d} shows a 1D example: both depicted solutions (denoted $f$) vanish at the input points $q_1,q_2$ (black points) and globally minimize the Eikonal loss over the grid (grid points are shown in blue). The INR works mentioned above use neural networks for representing $f$ which injects an inductive bias avoiding these bad minima, however on grids, minimizing \eqref{e:loss_eikonal} cannot avoid these solutions. See, \eg, middle column in Figure \ref{fig:teaser}. 

More classical Eikonal solvers do work with grids however use mostly fast marching or sweeping methods \citep{osher1988fronts,sethian1996fast,zhao2005fast,chacon2012fast}. Namely, use a special discretization of the Eikonal equation favoring the viscosity  solution of the Eikonal \cite{rouy1992viscosity}, and update node values according to a moving front \cite{sethian1996fast}. Since this discretization is up-wind (will only propagate values in one direction) and requires choosing the maximal among its solution, its success in adaptation to a loss is not clear. 

We use a different approach to build a loss favoring SDF solutions over grids motivated by the vanishing viscosity method \cite{crandall1983viscosity}. Namely, adding to the Eikonal PDE a small perturbation of the Laplacian of $f$ (denoted by $\Delta f$), \ie, $\norm{\nabla f(p)}-1 - \eps\Delta f(p)=0$, makes the PDE semi-linear elliptic \citep{calder2018lecture}, and hence with suitable boundary conditions it is uniquely solvable inside $\gS$ with a smooth solution, approaching the viscosity positive distance function to the boundary as $\eps\too 0$. Similarly, for $1-\norm{\nabla f(p)} - \eps \Delta f(p)=0$ the solution approaches the negative distance function inside the domain. 
Motivated by the vanishing viscosity principle we suggest the following viscosity loss:
\begin{equation}\label{e:loss_viscosity_eikonal}
\gL_{\text{viscosity}} = \int_\gC \Big((\norm{\nabla f (p)}-1)\mathrm{sign}(f(p)) - \eps \Delta f(p)\Big)^2 dp
\end{equation}
We discretize this loss over the grid $\gG$ by replacing the first order derivatives and second order derivatives with symmetric finite  differences, \ie,
\begin{align*}
    D_x f_I=D_x f_{i,j,k} = \frac{f_{i+1,j,k}-f_{i-1,j,k}}{2h}, \quad D^2_x f_I = D^2_x f_{i,j,k}=\frac{f_{i+1,j,k}-2f_{i,j,k}+f_{i-1,j,k}}{h^2}
\end{align*}
and similarly for $D_y$ and $D_z$. We use these discrete operators to approximate the gradient $\widehat{\nabla} f(p_I) = (D_x f_I, D_y f_I, D_z f_I)$ and Laplacian $\widehat{\Delta}f(p_I) = D_x^2f_I + D_y^2 f_I + D_z^2 f_I$. The discretized viscosity loss now takes the form
\begin{equation}
    \widehat{\gL}_{\text{viscosity}} = \frac{1}{N}\sum_{I} \Big((\|\widehat{\nabla} f (p_I)\|-1)\mathrm{sign}(f(p_I)) - \eps \widehat{\Delta} f(p_I)\Big)^2
\end{equation}



\subsection{Coarea loss}\label{ss:coarea_loss}
The coarea loss is approximating the area of the zero level set, and therefore incorporating it in the optimization pushes the reconstructed surface to be economic in area. 

First, similarly to  \citep{yariv2021volume} we use the centered Laplace CDF
\begin{equation}
   \Psi\beta(s)= \begin{cases}
   \frac{1}{2}\exp\parr{\frac{s}{\beta}} & s\leq 0 \\ 1-\frac{1}{2}\exp\parr{-\frac{s}{\beta}} & s\geq  0
   \end{cases}
\end{equation} to transform the SDF $f$ to a smooth approximation of the indicator function:
\begin{equation}
    \chi_\beta(p)=\Psi\beta (-f(p))
\end{equation}
As $\beta\too 0$, $\chi_\beta$ converges to an indicator function leading to $1$ inside $\gS$ and $0$ outside. The coarea loss is defined as 
\begin{equation}
    \gL_{\text{coarea}} = \int_\gC \norm{\nabla \chi_\beta (p)} dp
\end{equation}
To understand why this loss approximates the area of $\gS$ we can use the coarea formula \citep{rindler2018calculus}:
\begin{equation}\label{e:coarea}
    \int_\gC \norm{\nabla \chi_\beta(p)}dp = \int_{-\infty}^{\infty} \mathrm{area}(\chi_\beta^{-1}(s))ds,
\end{equation}
where $\chi_\beta^{-1}(s)=\set{p\ \vert \ \chi_\beta(p)=s}$ is the preimage of the value $s$. Since $\chi_x(p)\in [0,1]$ the r.h.s.~integral can be restricted to the interval $[0,1]$, and therefore the coarea loss averages the area of the level sets of $\chi_\beta$. Next,  $$\chi_\beta^{-1}(s)= \set{p\ \vert \ \Psi\beta (-f(p)) = s } = \{p\ \vert \ f(p) = -\Psi\beta^{-1} (s) \} = f^{-1}(-\Psi\beta^{-1} (s)),$$
\begin{wrapfigure}[11]{r}{0.28\textwidth}\vspace{-20pt}
  \begin{center}
  \includegraphics[width=0.25\textwidth]{figs/semi.png}
  \end{center}
  \caption{Reconstruction of a semisphere point cloud (white dots) without (left) and with (right) coarea loss. }\label{fig:coarea_semisphere}
\end{wrapfigure}

which shows that the level set $s\in (0,1)$ of $\chi_\beta$ is the level set $-\Psi\beta^{-1}(s)$ of the SDF $f$. As $\beta\too 0$, $-\Psi\beta^{-1}(s)\too 0$ for all $s\in (0,1)$ (and uniformly in $(\eps,1-\eps)$ for fixed $\eps>0$). Therefore the average of the level set area (\ie, the r.h.s.~of \eqref{e:coarea}) converges to the area of $f^{-1}(0)=\gS$. Figure \ref{fig:teaser} (right) shows how removing the coarea loss introduces an extraneous zero level set, and hence results in an undesired surface part. Figure \ref{fig:coarea_semisphere} shows a comparison of a reconstruction of semisphere with and without coarea. In the experiments section we provide more ablation tests with the coarea and viscosity losses.

To discretize the coarea loss we let $w_I$ denote the centers of grid's voxels, and note that $\nabla \chi_\beta(w_I) = \Phi_\beta(-f(w_I))\nabla f(w_I)$, where 
\begin{equation*}
    \Phi_\beta(s) = \frac{1}{2\beta}\exp\parr{\frac{\abs{s}}{\beta}}
\end{equation*}
is the PDF of the Laplace distribution, and $\nabla f(w_I)$ is computed as a linear combination of the voxel's corner values $f_{I_1},\ldots,f_{I_8}$, see more details in the Appendix. We end up with the discretized loss:
\begin{equation}
    \widehat{\gL}_{\text{coarea}} = \frac{1}{N}\sum_{I}\Phi_\beta(-f(w_I))\norm{\nabla f(w_I)}
\end{equation}
This loss is usually incorporated with a small hyper-parameter $\lambda_{\text{c}}$ with the purpose of eliminating redundant surface parts.



\section{Results}
\label{results}

\begin{figure*}[ht]
    \centering
    \includegraphics[scale=0.15,trim={0 2.5cm 0 5cm},clip]{images/aoi-single_burst}
    \caption{The time average peak Age of Information with burst and \gls{soa} loss values against the dynamic reliability logic for different network topologies.}
    \label{fig:aoi_burst}\vspace{-0.4cm}
\end{figure*}


This paper focuses on both transport layer and application layer metrics to determine the feasibility of dynamic reliability. For this, we have selected the session packet volume, as transmitted, retransmitted, lost and backlogged packets as \glspl{kpi} for the transport layer; while focusing on the \gls{aoi} for the application layer. The \gls{aoi} was chosen as a crucial indicator for the freshness of packets in real-time applications. More specifically, this work adopts the time average peak \gls{aoi} equation \cite{aoi_equation} depicted in Eq. \ref{aoi}, where $\Delta(r_{i+1})$ is the $i$th update at the time it was received at the server, for a session time period of $\tau$.

\begin{equation}
    \label{aoi}
    \gls{aoi}_\tau = \frac{1}{n-1}\sum_{i=1}^{n-1} \Delta(r_{i+1})
\end{equation}

We include a comparison between the vanilla QUIC implementation which does not enjoy the dynamic reliability extension, with a number of dynamic reliability policies. The tests were run a number of times for statistical significance, with the mean value of vanilla implementation used as a baseline for comparison. The topology utilised both random loss and bursty loss to explore the bounds of dynamic reliability. The \gls{soa} loss in the figures correspond to the loss values presented in Table. \ref{tab:path_char}, for ease of comparison between bursty and random loss scenarios.

\subsection{Transport-Layer KPIs}

To analyse the performance gain at the transport layer due to dynamic reliability, the volume of transmitted and backlogged packets is examined. The figures are in the form of boxplots, which take the vanilla implementation as a benchmark, depicted as the red dashed line.

As seen in Fig. \ref{fig:sent_burst}, the loss plays a crucial role in the performance of the reliability policies. The policies under random loss did incredibly well for the networks with a larger capacity, namely \gls{mmwave} and Sub-6~GHz, whereas for burst loss, the lower network capacities had a larger packet reduction. With the increase in burst loss, the behaviour of the set split reliable policies became unpredictable, if a reliable assignment happened to coincide with a burst loss, the number of transmitted packets increases, and vice versa. On the other hand, in smarter policies, such as Loss-Aware, the performance lightly matched the vanilla baseline, as the reliable assignment dominated the session to compensate for a higher burst loss. Not only that but, the burst loss also impacted the variance of the transmitted packets for the policies.

Unsurprisingly, the unreliable focused policy, 80-20 split, outperformed other policies for all topologies in random and bursty loss scenarios, with an approximate reduction of 80\%. That being said, the majority of the policies reduced the transmitted packets on the link by approximately 70\% for random loss, while the reduction started at $\approx 15\%$ and decreased as the loss increased for the burst loss scenario.

The retransmitted and lost packets, not shown due to space limitations, followed the same trend as the transmitted packets for the random loss scenarios. However, for the burst loss scenarios, the larger capacity networks had a lower reduction in the retransmitted and lost packets. This can be seen as a favorable outcome since the lower capacity networks are scarce on resources. It is important to note that the Loss-Aware policy mimicked the vanilla approach as the burst loss increased, signifying the overwhelming appointment of reliable packets in adapting to the harsh burst loss conditions.
 
Alternatively, Fig. \ref{fig:backlog_burst} clearly shows a stark comparison between the policies and loss scenario in the reduction of the backlogged packets. The Loss-Aware policy for random loss scenario reduced the backlogged packets by up to 50\%, beating all other policies by approximately 30\%. Furthermore, it is clear that the unreliability focused policies resulted in the lowest backlog for the session. In comparison, we notice that the burst loss and the backlogged frequency have a positive correlation, where the maximum reduction of the backlogged packets for the policies is at most 20\%. Much like the transmitted packets, the probability of a burst loss occurrence plays a vital role in the number of retransmissions sent and by extension the number of backlogged packets. Thus, we can conclude that the stress placed on the buffer is a result of the reliable packets which is tightly coupled with the congestion on the session. Whereas, unreliable focused policies did not encounter such a phenomenon regardless if it was experiencing a burst loss.


\subsection{Application-Layer KPIs}

The feasibility of dynamic reliability for real-time applications can be determined by the \gls{aoi}, with comparison across different topologies and policies. If we take a strict approach and consider anything below $10$~ms is real-time \cite{real-time}, then all the reliability policies passed that requirement, which is attractive for real-time applications, as shown in Fig. \ref{fig:aoi_burst}. Utilising the median as an estimate of the runs, the policies in the WLAN and Sub-6~GHz topology with random loss floated around $4-5$~ms with negligible difference, while the \gls{aoi} for \gls{mmwave} was $\approx 2-3$~ms. It is clear that the \gls{aoi} and the network capacity have a negative correlation, as the network capacity decreases, the \gls{aoi} increases. The same correlation is extended to the bursty loss scenarios, where \gls{mmwave} dominated the other topologies. That being said, it is crucial to note that the \gls{aoi} for the reliability policies is often slightly better than or equal to the \gls{aoi} of the vanilla implementation, proving that dynamic reliability reduces the congestion of the session at no cost to the \gls{aoi}.


We provide some comments on the growth conditions which constituted the majority of our analysis in sections \ref{sec:Hmixing} and \ref{sec:Hsigma}. In the simplest cases of Lemma \ref{lemma:unstableGrowth}, growth was established in an analogous fashion to the old one-step expansion condition (\ref{eq:oldOneStepExpansion}), finding the relevant Jacobians $M_j$ and checking that their expansion factors $K(M_j)$ satisfy
\begin{equation}
    \label{eq:discussionOneStep}
    \sum_j \frac{1}{K(M_j)} <1.
\end{equation}
For the more complicated cases, the inductive method used to establish growth near the accumulation points in Lemma \ref{lemma:unstableGrowth} and the weakened one-step expansion condition (\ref{eq:oneStep}) both address the same fundamental issue: the splitting of unstable curves by singularities into an unbounded number of small components. They circumvent this obstacle in rather different ways, however. While (\ref{eq:oneStep}) generalises (\ref{eq:discussionOneStep}) to ensure an growth of unstable curves `on average' (see \cite{chernov_statistical_2009} for a precise statement), our inductive method is a more direct adaptation of (\ref{eq:discussionOneStep}), using it to generate contradictory geometric conditions which a hypothetical non-growing unstable curve must satisfy. It may be possible to prove Theorem \ref{sec:Hmixing} using (\ref{eq:oneStep}) as the basis for growth. Since we required (\ref{eq:oneStep}) anyway for proving Theorem \ref{thm:HsigmaExp}, this could potentially condense our analysis, but only to a minor extent. A convenience of the method used in section \ref{sec:Hmixing} is that, by way of the `simple intersection' property, it naturally gives geometric information on the images of manifolds, useful for proving the property \textbf{(M)} of Theorem \ref{thm:katok-strelcyn}.

We expect that essentially analogous analysis can be applied to establish mixing properties in a wide class of piecewise linear non-uniformly hyperbolic maps, including those (like the OTM) which sit on the boundary of ergodicity and beyond. While we have relied on the precise partition structure of $H_\sigma$, its fundamental feature (self-similar sequences of elements $A^k$, sharing boundaries with its neighbours $A^{k-1},A^{k+1}$ and accumulating onto some point $p$) is quite typical to return map systems. See, for example, those of various stadium billiards \cite{chernov_chaotic_2006,chernov_improved_2008,chernov_statistical_2009} and LTMs \cite{springham_polynomial_2014}. Indeed, the same method can be used to prove the Bernoulli property for non-monotonic LTMs \cite{myers_hill_mixing_2022}, where monotonicity of the manifold images cannot be assumed and the classical argument \cite{sturman_mathematical_2006} fails. The OTM is the pointwise limit of these maps as the boundary shrinks to null measure. It further has utility in proving growth conditions for maps which are uniformly hyperbolic but possess regions $A_j$ where the hyperbolicity is very weak, signified by $K(M_j) \approx 1$, so that (\ref{eq:discussionOneStep}) fails. Typically this leads to suboptimal bounds on mixing windows, see e.g. \cite{wojtkowski_model_1981,przytycki_ergodicity_1983,myers_hill_family_2022}. The map $H_{(\eta,\eta)}$ for $\eta \approx 1/2$ is another example, possessing weak hyperbolicity over $A_2, A_3$. Letting $\varepsilon = |\eta-1/2|>0$, there is an upper bound $N = N(\varepsilon)$ on escape times from the intersections $A_2\cap \sigma, A_3 \cap \sigma$. The growth lemma then follows by applying the inductive step roughly $N$ times and can be established for arbitrarily small $\varepsilon$, opening the door to establishing optimal mixing windows.

The above gives two examples of piecewise linear perturbations to $H$ where mixing with respect to Lebesgue is preserved and our methods can be applied. Nonlinear perturbations to the shear profiles complicate the analysis in several ways. Firstly as the map's Jacobians takes on a broader range of values, cone invariance becomes an increasingly harder condition to establish. Cones must be widened, giving looser bounds on expansion factors, which may already be weak due to new regions of weaker stretching. This, together with the change from polygonal to curvilinear return time partition elements and nonlinear local manifolds, adds some complexity to showing growth conditions. This does not rule out certain (small) nonlinear perturbations however. There is some leeway in the inequalities which govern cone invariance and growth of local manifolds, the latter of which is not too dissimilar from the piecewise linear setting (see Lemmas \ref{lemma:piecewiseApprox}, \ref{lemma:componentLength}). Certain small perturbations would not alter the \emph{topological} structure of the return time partition, i.e. which elements share boundaries, the key information needed for setting up the induction. Finally while the partition elements would no longer be polygonal, only coarse geometric information is required for verifying each inductive step. Following the above, a potential perturbation could be to replace the linear portions of each shear by a cubic, perturbing the tent profile
\[  f(t) = \begin{cases} 2t & 0 \leq t \leq 1/2, \\ 2(1-t) & 1/2 \leq t \leq 1 ,\end{cases} \]
of the OTM shears to
\[  f_a(t) = \begin{cases} \frac{1}{8} t \left(16 - a + 6at - 8at^{2} \right) & 0 \leq t \leq 1/2, \\ \frac{1}{8}\left(1-t\right)\left( 16 - a + 6a\left(1-t\right) - 8a\left(1-t\right)^{2}\right)  & 1/2 \leq t \leq 1, \end{cases}   \]
for $a>0$. For small enough $a$ the gradient range $f'(t)$ is restricted to small neighbourhoods of $\{ 2, -2\}$ and the escape time partition retains a similar structure. We illustrate this in Figure \ref{fig:perturbations}, showing escapes from the square $S_3$ under the map $G \circ F$, equivalent to escapes from the perturbed $A_3$ under the $G \circ F$, but with a cleaner geometry for comparison. When $a$ is too large the analogy to the OTM breaks down. At $a=16$ the map is twice differentiable everywhere and features a new source of slowed mixing, the Jacobian is the identity at the corner points $x,y \in \{  0, 1/2 \}$ giving locally parabolic behaviour (visible in the escape time partition). 

\begin{figure}
    \centering
    \includegraphics[width=0.24 \linewidth]{0.png}
    \includegraphics[width=0.24 \linewidth]{4.png}
    \includegraphics[width=0.24 \linewidth]{8.png}
    \includegraphics[width=0.24 \linewidth]{16.png}
    \caption{Partition of escape times from $S_3$ under the mapping $F \circ G$ for $a= 0,4,8,16$. }
    \label{fig:perturbations}
\end{figure}

\documentclass[./main.tex]{subfiles}
\begin{document}

\title{Supplemental Material\\From Clean Room to Machine Room: Commissioning of the First-Generation BrainScaleS Wafer-Scale Neuromorphic System}

\DeclareRobustCommand{\enumauthorrefmark}[1]{\smash{\textsuperscript{\footnotesize #1}}}

\newcommand{\contributedSymbol}{\IEEEauthorrefmark{1}}
\newcommand{\uheiSymbol}{\enumauthorrefmark{1}}
\newcommand{\ugoeSymbol}{\enumauthorrefmark{2}}


\author{
	\IEEEauthorblockN{%
		Hartmut Schmidt\contributedSymbol,
		José Montes\contributedSymbol,
		Andreas Grübl,
		Maurice Güttler,
		Dan Husmann,
		Joscha Ilmberger,\\
		Jakob Kaiser,
		Christian Mauch,
		Eric Müller,
		Lars Sterzenbach,
		Johannes Schemmel,
		Sebastian Schmitt\\
	}

	\thanks{
		\IEEEauthorblockA{%
		\contributedSymbol%
		Contributed equally\\
		}
	}
}

\maketitle
Next, we present the Supplementary Materials for the paper ``Re-ReND: Real-time Rendering of NeRFs across Devices''.
Specifically, in addition to the results reported in the paper, we report results of \methodname w.r.t. Image Quality~(Section~\ref{sec:im_qual}) and (Section~\ref{sec:quali}), Rendering Speed~(Section~\ref{sec:fps}), Mesh Size~(Section~\ref{sec:mesh_size} and Section~\ref{sec:meshi}), Disk Space~(Section~\ref{sec:disk_space}), validation of view-dependent effects (Section~\ref{sec:val}),  sensitivity to geometry variations (Section~\ref{sec:geo}) and Photo-metric quality w.r.t. embedding dimensionality $D$ (Section~\ref{sec:dim}).
Furthermore, we encourage the reviewers to watch the \textbf{associated video}, \texttt{Re-ReND.mp4}, demonstrating \methodname's capabilities of real-time rendering across devices.
% In particular, please refer to .
This video demonstrates how \methodname can render, in real time, a scene composed of tens (\Figure{composit}) or even thousands (\Figure{many_objects}) of objects. % , respectively. %  , or even with thousands of . %  in an AR headset.
\Figure{composit} illustrates such a scene, composed of moving chairs, hotdogs, the drumset, and a microphone.


% Finally, we also provide the PyTorch~\cite{NEURIPS2019_9015} and GLSL implementations of our method inside the folders called \texttt{Re-ReND\_Pytorch\_code} and \texttt{Re-ReND\_GLSL\_code}.

% \thispagestyle{empty}
% \appendix

%%%%%%%%% BODY TEXT - ENTER YOUR RESPONSE BELOW
% \section{The PyTorch code and GLSL code}

%  \begin{itemize}
%     \item Clean and README.md
%     \item Should I upload only pur method or MipNeRF and NeRF++?
%     \item Should I upload the generated data and the meshes in a google drive? What happens with anonymity?
% \end{itemize}

% \section{A video showing how we were measuring the FPS}
% \section{A video showing real scenes in comparison with MobileNeRF and SNeRG}
% \section{Qualitative Results}

%  \begin{itemize}
%     \item all objects visualizations 
% \end{itemize}

%-------------------------------------------------------------------------


\begin{figure}
    \centering
    \includegraphics[width=\linewidth]{pics/quantitative.pdf}
    \caption{Box plots of quantitative benchmarks MIG, FactorVAE, Disentanglement, and reconstruction error on dSprites and Shapes3D.}\label{fig:quantitative}
\end{figure}


\bibliographystyle{style/IEEEtran}
\bibliography{bib/vision}

\end{document}




%\section*{EVERYTHING BELOW IS OLD CHI CONTENT}

%\section{Introduction} \label{sec:intro}

Large amounts of time and effort are devoted to
verification and validation of every microprocessor design project.
Broadly, design verification can be broken into two large categories:
(1) functional and (2) performance verification, which is to identify design bugs that degrade performance without affecting functionality. Performance bugs are different from performance bottleneck as the former is due to design mistakes while the later is caused by tight resource constraints. Performance loss due to performance bugs  can 
be very significant, with recent reported cases shown to be
$>10\%$~\cite{mccalpin2018hpl}. This demonstrates a critical 
need for automated mechanisms for performance debugging.  As 
recent designs from Intel~\cite{corei7-11}, AMD~\cite{ryzen-9},
ARM~\cite{cortex-a}, and others place an even greater emphasis
on core performance, design complexity has scaled
dramatically, likewise scaling the difficulty in all forms of
verification.


%Functional verification has received extensive attention from researchers and, although complex, it benefits from the availability of known correct outputs that can be used to compare against.

Performance verification at microarchitecture level ensures that a
design correctly achieves expected performance in terms of execution
time or cycle count.  The main challenge in this task is that, unlike
functional verification, there is no exact golden reference to compare
against.  This is because of the high difficulty of modeling all the
interactions between the different units in complex microprocessor
designs, and accurately represent how they affect the overall system
performance.  %This task also suffers from 
%the lack of a good debugging infrastructure, as well as from 
%limited visibility into intermediate points in the design, which are mostly exposed through performance counters. Although useful for estimating the performance of the system, these counters are very difficult to use for manual debugging because of their complex relationship with processor performance and due to the large amounts of data they generate.  
Traditionally, performance
verification is conducted mostly through manual techniques which rely
on rough estimations of performance gain expected by
microarchitectural changes~\cite{Singhal2004}. Such manual processes
are not only very lengthy but also error-prone.



The process of performance verification and debugging roughly consists of two steps: (1)~detection, which determines whether a
design achieves expected performance or not, and (2)~localization,
which identifies the microarchitectural units causing the performance
issues and is the focus of this work.

There are few previous studies on automating detection of
microprocessor performance bugs~\cite{Bose1994,
  surya1994architectural,carvajal2021detection}. 
The majority of those~\cite{Bose1994, surya1994architectural} relies on capturing
design intentions using a bespoke performance model as a golden
reference, this  entails long development time and may contain
errors by itself. Recently, a data driven and machine learning
(ML)-based approach~\cite{carvajal2021detection} was developed for
automatic performance bug detection with high accuracy. Although
significant, these works do not solve the 
problem of performance bug localization.
%pressing problem of identifying where the performance bug is.

Works in automating microprocessor performance bug localization
are even scarcer.  Adir \emph{et al.}~\cite{adir2005generic}
propose perhaps the only partially related work of which we are
aware.  Their work focuses on formal planning of test program
generation for individual units, such as issue queues. This strategy
follows conventional functional verification, involving heavy
manual effort, costing significant engineer-time to develop a test
plan, and as much as ten days of computer runtime per functional
unit. To the best of our knowledge, there has been no systematic study
on automatic performance bug localization for microarchitecture
designs.

Performance bug localization is a complicated task, which is currently
solved using mostly manual techniques.
Even
in the more widely studied area of functional validation, the industry
lacks a standardized mechanism to automate bug localization, it has
been only recently that academic efforts have attempted to automate
this task~\cite{BugMD}. Considering this, it is important to note that
any type of design automation which successfully reduces the 
time and effort required by engineers to debug their designs is highly
valuable. Since automatic performance debug for microprocessors is
a huge yet under-studied challenge, it is very difficult, if not impossible, to find a perfect solution in a single work. Although our work is not perfect, it serves a key stepping stone 
 toward solving the problem.

This work tackles the performance bug localization problem by
using ML to generate a ranked list of most likely mi\-cro\-ar\-chi\-tec\-tur\-al units that 
might contain the bug.  This list may be used
to prioritize the debugging order, as well as to identify
teams with the right expertise to perform further debug. Two different methodologies are
proposed, evaluated, and contrasted. These data-driven
techniques achieve high
accuracy, while being fully automated. Further, they
consider intra- and inter-unit interactions, as opposed to other
techniques proposed in the partially related previous work~\cite{adir2005generic} which
considered only intra-unit behavior.

%Our methods are based on ML, wherein our models are
%trained using data from legacy designs.
%To take the full advantage of
%these approaches, we assume that architectural changes in a new design
%are incremental when compared to its previous
%generations. Examining recent processors from major vendors including
%Intel, ARM, and AMD, we find this assumption holds true, since the generational change in microarchitectures
%is largely incremental. Thus, the methodologies proposed here provide
%alue for a multitude of upcoming designs.  However, even when
%disruptive changes occur, the methodologies can still be beneficial for bug localization on structures that conform to previous microarchitectures, using workloads that
%do not exercise new functionalities. Further, as general purpose microarchitectures become ever more mature, and the inter-generational performance gains decrease, 
%it is even more important to retain as much performance as possible, making performance debugging ever more important.

The major contributions of this work include the following:
\begin{compactitem}
\itemsep0em 
\item This is the first systematic study on fully automatic
  performance bug localization for microarchitecture designs, to the
  best of our knowledge.

\item Two ML-based approaches to tackle performance bug
  localization, as well as a hybrid of both, are evaluated
  and contrasted.

\item For bugs with an average IPC impact greater than 1\%, our best
  performing methodology identifies the correct bug location as the
  most likely unit in $\sim77\%$ of the cases, and achieves over 90\%
  accuracy when the three most likely options (out of 11 possible) are
  considered.  

\item One of the proposed methodologies is not only very accurate localizing
performance bugs, but it can also be applied to confirm the results
of performance bug detection with high accuracy.

\item Although the focus of this work is on microprocessor core,
we evaluated our methodologies on the processor memory hierarchy. This evaluation
uses a different experimental setup, showing the robustness of the proposed techniques.

\end{compactitem}

As an early work on performance bug localization, the design of this study is subject to potential limitations, however, we feel it still represents a good first step towards solving the problem. The scope of our work and its limitations are as follows:
\begin{compactitem}

\item Legacy
designs with identified performance bugs are required, so that the ML
models can be trained. Bug-free legacy designs are required only 
in one of the methodologies, yet, if available, the other can take advantage of the additional data.
However, thanks to the thorough pre- and post-silicon
debug to which the designs are submitted, these legacy designs are
generally available.

\item We assume that only one bug is present at a time, 
in parallel to the single-fault model which is common practice
in VLSI testing works. As explained in Section~\ref{subsec:impl_bugs}, we still expect 
our methodologies to work well in the presence of multiple bugs in a single design.

\item Our methodologies do not provide a quantitative coverage guarantee.  
In general, performance bug
coverage is extremely difficult to define and is a potential
research topic on its own. We know of no prior work which presents a
definition of such coverage. Nonetheless, the evaluated bugs are based on published errata, cover a large amount of microarchitectural units and affect the system in a variety of ways. Thus, we feel these bugs represent a reasonable start for early work in this area.

\item We assume that there are no dramatic structural
microarchitectural changes between the legacy designs and the
designs under debug. Examining recent processors from major vendors, including Intel, ARM, and AMD, we find this assumption holds true, since the generational change in microarchitectures
is largely incremental. That said, even when larger shifts occur, the
methodologies can be partially reused. For example, consider the
introduction of the AVX instructions with Intel's Sandy Bridge
architecture in 2011.  Initially there would be no available data to
test these instructions using our methodologies, however the rest of
the Sandy Bridge design could be debugged with our methodology,
leveraging workloads that do not exercise the new instructions.  In
future implementations, data from Sandy Bridge can be used to build
the models required to use our methods for debugging AVX. 
%Further, as general purpose microarchitectures become even more mature, and the inter-generational performance gains decrease, it is even more important to retain as much performance as possible, making performance debugging even more important.}

\item We limit our evaluation to a pre-silicon setup, because
it is infeasible for us to inject known design bugs in silicon to
evaluate the methodologies.  Further, should our methodologies be
applied to a commercially available design, and an actual bug be
found and localized, we would not be able to verify that such
localization is correct without prior knowledge of its existence so
as to verify our findings. However, our methodologies can be applied in both pre- and post-silicon scenarios. During pre-silicon stages fixing performance 
bugs is easier and cheaper, 
the availability of performance counters is greater (due to the usage of a
simulator) and the counters can be sampled at a much faster rate. 
By using only counters available in-silicon, and adjusting the sampling frequency, we could use the proposed
methodologies during post-silicon stages. In post-silicon analysis the methodology
could be applied to longer workloads, providing a way to exercise complicated bugs that
are not possible to trigger with short pre-silicon traces.
%Further, we can follow hybrid approaches where the ML model training is performed using simulations, and the techniques are applied to data obtained from microprocessors during post-silicon debug. }

\end{compactitem}

Despite the aforementioned, we present a first, useful, yet attainable,
step towards the goal of performance bug localization, and we hope this work can draw the attention of the research community
to the broader performance validation domain.

\iffalse{
In Section~\ref{sec:scope} we describe the problem
formulation and outline the scope of this work.  We note that, to
date, very little work exists in automating performance bug
localization. 

Section~\ref{sec:methodology} 
describes the approaches developed to tackle the performance bug
localization task.  Section~\ref{sec:experimental_setup} provides
details of the architectures, and performance bugs used for
evaluation. Section~\ref{sec:evaluation} presents results obtained in
several experiments developed to evaluate the methodologies. A brief
review of previous work related to performance debugging is presented
in Section~\ref{sec:related_work}.  And finally,
Section~\ref{sec:conclusion} concludes the paper.
}
\fi

%\section{Related Work}
% To-do : need to rewrite the related work text since it's currently copied from GENEA paper

%We first review traditional gesture generation methods and later introduce some work related to VQ-VAE. 
\subsection{Co-Speech Gesture Generation}
The early works in co-speech gesture synthesis utilized pre-defined sets of manually created gesture units to build a gesture database. The new gestures were generated via keyword matching or prosody analysis to find the best corresponding gesture units from the database \cite{kopp2003max, kopp2009, marsella2013}. The gesture unit database can also be created automatically from speech-gesture data by segmenting and clustering gesture motions based on the similarity of motions and speech contents. During synthesis, the desired gesture property and speech attributes are used to search for the best gesture segment in the database that matches the input speech content. Our method is motivated by these ideas from early works, but instead of manually building a gesture database, we used residual quantization to implicitly learn the discrete codebook of gesture tokens. 

Recent learning-based methods train an end-to-end model from speech-gesture datasets to predict gesture motions from speech. The methods based on direct regressions find a deterministic mapping from speech to gestures \cite{kucherenko2020gesticulator, ginosar2019gestures, Yoon2020Speech, bhattacharya2021speech2affectivegestures}. Since these methods do not handle the issue of one-to-many mapping, the adversarial scheme is sometimes utilized by training the model with an additional discriminator to improve the resulting motion qualities. Recent methods further improve synthesized results by using hierarchical architecture to model multi-level skeletal poses \cite{liu2022learning} or adding semantic prompter to force semantic alignment in output gestures \cite{liang2022seeg}. 

Probabilistic frameworks were also used in the recent gesture generation works to handle the gesture ambiguities \cite{ahuja2019language2pose, Alexanderson2020, Qian2021, Li2021, bhattacharya2021text2gestures}. This type of methods learns a latent space generative model and could generate multiple gesture motions from the same speech input using conditional sampling from the latent space during inference. Similar to the previous methods, our method utilizes a latent space model and conditional sampling to handle the one-to-many problems for gesture synthesis. However, instead of learning a continuous latent space, we applied residual quantization to learn \textit{discrete} latent codes. The discrete latent codebook provides a compact representation for gesture units and the inference process is reduced to selecting the most probable latent code from the codebook. Thus by learning the conditional prior distribution over the discrete latent codes, the method naturally handles the mapping from one speech input to multiple different gestures with varying probabilities.  

\subsection{Discrete Latent Space Learning}

Vector-quantized variational autoencoder (VQ-VAE) \cite{van2017neural} learns discrete representations as a codebook from images. In the two-staged generative architecture like in Video GPT \cite{yan2021videogpt}, an autoregressive prior can then be trained to model the categorical distributions for these discrete latent codes. It was first introduced for image synthesizing or compression tasks and is able to produce sharper and higher quality image synthesis results. It was further improved in \cite{razavi2019generating} using a multi-scale hierarchical architecture to model higher-resolution images. 

%One issue that often affects the reconstruction quality when training VQ-VAE is codebook collapse. It happens when the model only learns to use a small subset of the codes in the codebook, leaving a majority of the codes unused. This limits the expressiveness of the model and results in lower-quality results. Several methods and techniques have been proposed to prevent codebook collapse. Jukebox \cite{dhariwal2020jukebox} introduced re-initializing unused codes to a random vector to prevent dead codes during each training iteration. Video GPT \cite{yan2021videogpt} finds normalizing MSE for the reconstruction loss also mitigates codebook collapse. Hierarchical models were proposed for better codebook utilization in VQ-VAE2 \cite{razavi2019generating} by first extracting bottom and top features unconditionally to mitigate the codebook collapse. RQ-VAE~\cite{lee2022autoregressive} uses a fixed size of codebook to recursively quantize the feature map represented as a stacked map of discrete codes, which reduces the codebook size and stabilizes the codebook training. 

One issue that often affects the reconstruction quality when training VQ-VAE is codebook collapse, which leaves a majority of the codes unused and limits the expressiveness of the model. Several methods and techniques have been proposed to prevent codebook collapse. Jukebox \cite{dhariwal2020jukebox} introduced re-initializing unused codes to a random vector to prevent dead codes during each training iteration. Video GPT \cite{yan2021videogpt} finds normalizing MSE for the reconstruction loss also mitigates codebook collapse. Hierarchical models were proposed for better codebook utilization in VQ-VAE2 \cite{razavi2019generating} by first extracting bottom and top features unconditionally to mitigate the codebook collapse. RQ-VAE~\cite{lee2022autoregressive} uses a fixed size of codebook to recursively quantize the feature map represented as a stacked map of discrete codes, which reduces the codebook size and stabilizes the codebook training. 

The discrete latent space has also been applied for text-to-image synthesis, which generates new images based on input textual description using a two-stage architecture. It first learns a discrete representation for image patches and modeling the auto-regressive priors using transformers. Learning with discrete codes is more efficient over raw pixels since the transformer may not learn the fully dependencies between pixels. The work by Esser \etal \cite{esser2021} further applied adversarial training to learn VQ-GAN that produces a perceptually rich codebook. 

In addition to image synthesis, discrete latent space model is also known to be one of the state-to-the-art methods for modeling time-series data such as audio. Jukebox \cite{dhariwal2020jukebox} utilized VQ-VAE to generate singing music. It trained multi-level networks to compress audio in different resolutions into discrete space and then used autoregressive transformers to learn the latent codes for music generation. The same idea was also adapted to generate repetitive rhythms of music by learning from extracted music loops. Multi-Instrumentalist Net \cite{su2020multi} was proposed to generate multi-instrumental music from videos, which trained VQ-VAE along with an autoregressive prior conditioned on the musician’s body key points movements. 

%Our method is motivated by the recent success of applying a discrete latent space model in cross-modal synthesis tasks. 
Compared to previous works, the proposed co-speech gesture synthesis methods utilize RQ-VAE for modeling discrete gesture tokens and a two-level RQ-Transformer architecture to model conditional priors for gesture-generating tasks. The evaluation results show its potential for retaining motion quality while allowing non-deterministic motion synthesis from the same speech input.


% In our work, we do suffers from the codebook collapse problem in the beginning, output gestures are limited. We then applied used random restart of the codebook, exponential moving average updates for the codebook and adjusting MSE weight, and the results are much better. We may consider hierarchical architecture for VQ-VAE in the future as well.

%\subsection{Multi-modal Text-to-Image Synthesis}




%\section{Formative Study}
\label{sec:formative_study}
To gain an operational understanding of the range of processes that take place during intelligence analysis and generating situation reports, we conducted a formative study to gather information on: (1) regular practices in authoring situation reports and (2) general needs and expectations from intelligence analysts for AI-driven systems. Below, we start with a brief description of situation reports to provide background and then proceed with specifics on the formative study.

%\heng{I think the study description section can benefit from more concrete writing. maybe even attach the assessment guidelines etc. in appendix}

\subsection{Background}
\label{sec:situation_report_background}
%\textcolor{blue}{\textbf{TODO: Daniel - Needs some explanation (maybe at the beginning of Section 3) on why this situation report overview section is here.}} \revanth{Is this `our understanding' of what situation reports are and how they are generated, based on the formative study?}
%\daniel{I've added a short sentence below; however, the study task also describes that a component of the formative study is to understand what situation reports are and how they're created.}\item \textcolor{blue}{\textbf{I'll edit above at start of section.}}
A situation report is a document produced by an intelligence analyst to disseminate critical information. The report typically focuses on a specific issue or topic and provides a comprehensive overview of the information available on that topic. Intelligence analysts draft situation reports through a meticulous and systematic approach that begins with the collection of information from diverse sources, such as news articles, broadcasts, and potentially classified intelligence feeds. Once gathered, they prioritize the verification of this data, procuring reliable sources to ascertain accuracy. The subsequent analysis phase involves delving deep into the raw data to discern patterns, motivations, and implications, often requiring specialized knowledge or technical expertise. Post-analysis, the synthesized information is integrated into a cohesive narrative, emphasizing relevance and urgency, resulting in a clear, concise, and actionable report that communicates the prevailing situation to stakeholders. %Generating a situation report can take from 30 minutes to 3 hours, depending on the urgency, scale and complexity of the given situation.
The ultimate goal of such reports from intelligence analysts is to support downstream readers (i.e., consumers of the reports such policy-makers, government officials, and military personnel) in keeping track of developments by providing them with the time-critical, necessary information to make informed decisions and take appropriate actions.

\subsection{Method}
\label{sec:formative_method}
\subsubsection{Recruitment}
\begin{table}[!htb]
\centering
\label{tab:participants}
%\resizebox{\textwidth}{!}{
\begin{tabular}{llllll}
\toprule
\textbf{PID} &
  \textbf{Age} &
  \textbf{Education} &
  \textbf{Intelligence Exp.} &
  \textbf{AI Knowledge} &
  \textbf{LLM Usage} \\\midrule
IA1 & 25 - 34 & Bachelor's & 5 - 10 years & 2 - Intermediate & Rarely \\
IA2 & 25 - 34 & High School Diploma & 2 - 5 years & 3 - Proficient & Rarely  \\ 
IA3 & 18 - 24 & High School Diploma & 1 - 2 years & 2 - Intermediate & Daily  \\
IA4 & 25 - 34 & Bachelor's & 2 - 5 years & 2 - Intermediate & Rarely  \\
IA5 & 45 - 54 & Bachelor's & 2 - 5 years & 1 - Basic & Rarely  \\
IA6 & 45 - 54 & Master's & 1 - 2 years & 1  - Basic & Rarely  \\
IA7 & 35 - 44 & High School Diploma & 5 - 10 years & 1  - Basic & Weekly  \\
IA8 & 45 - 54 & Master's & 1 - 2 years & 2 - Intermediate & Rarely  \\
IA9 & 35 - 44 & Bachelor's & 2 - 5 years & 2 - Intermediate & Rarely  \\
IA10 & 25 - 34 & Bachelor & 2 - 5 years & 4 - Advanced & Weekly   \\ 
\bottomrule      
\end{tabular}
\vspace{0.3em}
\caption{Demographic information of study participants along with their experience in intelligence report generation.}
\label{tab:analyst_demographic}
\vspace{-1.5em}
\end{table}
For the formative study, we targeted individuals with experience in government and military roles. We distributed a pre-screening survey to ascertain their background in creating situation reports. Participants qualifying for the study were either experienced intelligence analysts or had a minimum of one year of equivalent experience. The final group comprised 10 military personnel from different branches, as shown in Table \ref{tab:analyst_demographic}. Their experience in intelligence report writing varied between 1 and 10 years.

%\textcolor{blue}{In our study\footnote{The study protocol was approved by our institution's IRB.}, we engaged with individuals who have served, or are currently serving, in various capacities within government, including the military. To identify potential participants, we disseminated a pre-screening survey focused on eliciting information about their prior experience in generating situation reports. Those who qualified for our study self-identified as experienced intelligence analysts (or with 2 years of equivalent experience). Our final participant pool consisted of 10 individuals: 10 armed forces personnel from various branches of the military. As detailed in Table \ref{tab:analyst_demographic}, the range of their experience in intelligence report authoring spans from 2 - 10 years.}

%{\textcolor{red}{\textbf{need to spell out what``ML Prof.'' and the values 1,2,4 refer to in caption of Table 2  -- Clare}}

 %\heng{do we need to provide more details who these analysts are? which agencies are they from?}
%\daniel{In the interest of privacy, especially for military personnel, we don't particularly require additional information. From a reviewer's perspective + paper contribution, the granularity in military demographic details don't add more value per say. }
%\heng{maybe we should also add more details on why we choose them as our users - first, we choose ukraine crisis as a case study because it's a complex situation and we need AI to identify important information and generate strategically important questions and answers; so we choose users from the military for this study? also we need to emphasize our techniques are not limited to military domain, we also have done other case studies on earthquake, dipomacy, etc.}
%\daniel{For an HCI study, we come from an inductive approach. Meaning, we don't begin with a tool or a hypothesis, rather a user group. Therefore, a deductive approach common in NLP, where we have the tool, and identify the correct user group to evaluate against is not used. It is true, that our techniques are not limited to the military domain; therefore, this is within the discussions + future work section, where we can discuss the technology is generalizable.}

\subsubsection{Study Task} Over a two-week period, semi-structured interviews were conducted to examine two key areas: Authoring Experience and AI-Assisted Tools in situation report generation. The Authoring Experience section focused on participants' personal experiences in creating intelligence reports, including their methodologies, challenges, and key aspects of their authoring process. The AI-Assisted Tools section explored their understanding, attitudes, and recommendations regarding AI use in professional settings.

%The interview sessions, conducted over 2 weeks, adhered to a semi-structured format, systematically exploring two aspects of generating situation reports: Authoring Experience and AI-Assisted Tools. In the Authoring Experience segment, we delved into the participants' own experiences with creating intelligence reports, exploring their processes, notable challenges, and focal points in their authoring journey. The AI-Assisted Tools component concentrated on participants' understanding, perspectives, and suggestions of using AI, including their general attitudes towards AI, and its integration into their professional practices.}

The interviews began with open-ended questions to encourage unrestricted responses, followed by targeted inquiries for deeper insight into their report authoring perspectives. Each 60-minute session, conducted via video conferencing, allowed participants to extensively share their experiences and views. Compensation ranged from \$25 to \$35 per hour, reflecting participants' levels of experience.
%Each interview commenced with broad, open-ended questions, designed to minimize constraints on participant responses, while probing the topics that arose naturally in follow-up inquiries to deepen our understanding of their perspectives on authoring reports. This approach allowed participants to elaborate at length on their experiences and views. Each session spanned 60 minutes and was conducted via video conferencing, ensuring a thorough and consistent exploration of the topics raised by participants. Participants were compensated \$25 - \$35 per hour, commensurate with their experience.}
%\item \textcolor{red}{\textbf{I switched 'themes' to 'topics' since the term 'themes' is used below for content that is discovered and coded.  -- Clare}}

\subsubsection{Analysis Procedure}%\revanth{I agree with Agathe here. We might need to show the themes/codes. Should we add in that flowchart that you had in the slides?}
\label{sec:formative_analysis}
The interview data analysis employed an inductive method, abstaining from any pre-existing theories or hypotheses. Two members of the research team concurrently examined the same data segments and noted their findings independently. This stage highlighted emerging \textit{themes} related to analysts' perceptions and expectations of AI-assisted authoring tools, such as trust, efficiency, and interaction. A thorough coding of the interview transcripts followed, aiming to systematically pinpoint these themes. This entailed a repetitive process of code development and adjustment as the team navigated the transcript collection. To ensure accuracy and thoroughness, periodic cross-checks were conducted by two coders.

%The analysis of the interview data followed an inductive approach, in which our observations and data were collected without preconceived theory or hypothesis. Two research team members collaboratively simultaneously reviewed identical data segments and independently documented their observations. Throughout this phase, we curated recurring {\textit{themes}} that arose in the analysts' perceptions and expectations of authoring experiences with AI-assisted tools (e.g. trust, efficiency, and interaction). Subsequently, we conducted a comprehensive coding of the interview transcripts, to systematically identify these themes. This process involved an iterative development and modification of codes as research team members worked their way through the collection of transcripts. To maintain the rigor and precision of our coding process, two coders periodically conducted cross-checks.}

\subsection{Key Findings}
\label{sec:formative_findings}
\subsubsection{\textbf{KF1.} Viewing technology as a means to enhance human capability.}
While examining intelligence analysts' views on advanced technology, an overwhelming majority (9 out of 10) emphasized the crucial role of digital technologies and artificial intelligence in enhancing their capabilities, specifically in generating situation reports. These tools are regarded not as mere process accelerators, but as essential elements that enrich their work by improving research efficiency, idea generation, and clarity of information. This perspective contrasts with the simplistic media depiction of these technologies as mere replacements for human effort. Instead, analysts view them as valuable enhancements to their workflow, particularly in high-pressure or resource-limited scenarios. For instance, one analyst commented:

%When exploring the intelligence analysts' perspectives on advanced technology, 9 out of 10 participants (IA1-IA7 \& IA9-10), described the vital role of digital technologies and AI in augmenting their capabilities, particularly in the context of situation report generation. These technologies were not seen merely as separate tools for speeding up processes, but as key contributors integral to enriching the analysts' work, by improving aspects like research efficiency, brainstorming, and clarity of information. Contrary to the often simplistic portrayal of these technologies in popular media as mere substitutes for human effort, analysts perceived them as providing valuable enhancements to their workflow, especially under high-pressure situations or resource constraints. For instance, one analyst commented:} %\heng{what was the negative comment from the person who disagree?}

\begin{quote}
``There's so many parts of authoring a situation report, each requiring strict and robust procedures. I wish I could have an AI partner to help me. It could make me better at my job.'' (IA6)
\end{quote}

This highlights a paradigm shift in the perception of technology within intelligence work, where it is seen not as a substitute, but as a vital complement that amplifies human skills.

%This insight underscores a paradigm difference in how technology is viewed in intelligence work – not as a replacement for human skills, but as a powerful ally that complements and strengthens human capabilities.}

\subsubsection{\textbf{KF2.} Trusting and relying on machines, as with humans.} The majority of participants (8 out of 10) exhibited a tendency to attribute human-like qualities of trust and reliability to AI systems, akin to their assessment of human colleagues. Analysts recognized AI as a dependable information source, comparable to a human co-worker, with a focus on its consistent delivery of high-quality outputs. The criteria for trusting AI closely resembled those for human interactions: the ability to provide dependable information, transparency in reasoning, and a foundation in verifiable facts. One participant articulated this equivalence in trust establishment:

%Most participants (8 out of 10) conveyed a tendency to attribute human-like qualities of trust and reliability to AI systems, paralleling the way they evaluate their human counterparts. Analysts designated AI as a reliable source of information, similar to a human colleague, with an emphasis on the AI's ability to consistently deliver quality outputs. The criteria for trust in AI, therefore, mirrored those applied to human interactions: the ability to provide dependable information, transparency in the process of deriving conclusions, and a foundation in verifiable facts. When inquired about how and why they would trust AI and human and the similarity in response, an intelligence analyst pointed out:} 

\begin{quote}
``After comparing my expectations between a human and an AI colleague, I can't say there's much of a difference. Trust is built the same way regardless of who it's towards. Just as I would trust a colleague that can explain their work, I could trust an AI that could show its workings.'' (IA1) 
\end{quote}

Interestingly, analysts did not set higher standards for AI than for human colleagues. This parity in trust and reliability criteria suggests that participants viewed AI as an equal collaborative partner, assessing its competence and trustworthiness on the same grounds as a human team member.

%Curiously, the analysts did not impose additional requirements or expectations on AI systems beyond what they would expect from a human colleague. The equivalence in the standards of trust and reliability, as expressed by participants, indicated that they viewed AI as a collaborative partner and judged its competency and trustworthiness by the same standards and merits as those of a human team member.}

\subsubsection{\textbf{KF3.} Training and guiding AI} 
Our research identified a split in intelligence analysts' perspectives on their role in training and guiding AI systems. Four out of ten participants advocated for substantial control over AI, emphasizing the need for an interactive system that allows them to influence everything from information source selection to narrative shaping in reports. They prioritized the ability to refine AI outputs based on their expert judgment. Two participants stated:

%We found a split in intelligence analysts' attitude to their involvement in training and guiding AI systems. 4 out of 10 participants expressed a strong desire for high-level control over AI systems. This group emphasized the need for a highly interactive system where they could exert significant influence, from selecting information sources to tailoring the narrative in reports. They believed in defining the quality of AI outputs based on their expert judgment, desiring systems that allowed for extensive fine-tuning. Two analysts stated:} 

\begin{quote}
``We're trained to be great at what we do. Isn't AI, like picking a random university student, and asking them to do our job?'' (IA2)
\end{quote}

\begin{quote}
''It's not about control. If someone is going to help me, I'm going to make sure they do it right.'' (IA7)
\end{quote}

In contrast, the majority (six out of ten) favored a more hands-off approach, highlighting that situation report creation follows well-established, standardized procedures suitable for AI implementation. They perceived AI involvement as an extension of routine oversight, akin to reviewing a junior colleague's work:

%Conversely, the remaining six participants viewed the role of AI differently. They pointed out that the process of creating situation reports is well-documented and follows strict, standardized procedures. According to them, these procedures are areas where AI could excel, as these standards and protocols could be effectively distilled into AI systems to produce concise and accurate reports. While acknowledging the need for oversight and occasional fine-tuning, they saw this as a routine aspect of their work, not unique to AI.} 
\begin{quote}
''There's a specific way we have to do each action, and making reports is no different. I just need to tweak small details as I would when reviewing other junior ranking officer's work.'' (IA10)
\end{quote}

This variance in attitudes seems to highlight a broader debate within intelligence work: the balance between human expertise and automated efficiency.

%This split in attitude and perceptions among intelligence analysts regarding the degree of human involvement and control desirable in AI-assisted processes also reflects a broader debate about the balance between human expertise and automated efficiency in intelligence work.}

%\section{Collaborative Design}
\label{sec:collaborative_design}

%\heng{Echo Revanth's comments - it's important to align section 5's components back with these design principles, e.g., how do the innovative components such as question generation and claim detection align with these principles? You can also point to section 5's subsections here in section 4.}

To gain a better understanding of the composition process of situation reports, we expanded the design opportunities identified in the formative study with subsequent collaborative design (CD) sessions with 10 participants (IA1-IA10). The goal was to capture tangible design strategies and recommendations from users about how they, as intelligence analysts, navigate, research, and author their situation reports.

\subsection{Process}
\subsubsection{Participants}
To enable us to build on the insights and findings of the formative study, the original participants (IA1-IA10) were sustained in the collaborative design phase.

\subsubsection{Procedure}
Each study session consisted of three distinct components: a Workflow Review using Storyboards, a hands-on Training Task, and a Simulation Report exercise. These sessions, conducted virtually, spanned approximately one hour each, with participants receiving compensation consistent with the rates in our formative study. 

%\heng{Did you also ask them to compare with gpt+bing? maybe good to add for camera-ready}

\begin{enumerate}
    \item \textbf{Workflow Review with Storyboards}
    \begin{enumerate}
        \item \textbf{Introduction to Storyboards:} In our study, participants engaged with a low-fidelity storyboard (shown in Figure \ref{fig:storyboard}), where each panel depicted a distinct phase in situation report creation. The storyboard, designed based on insights from our formative study, aimed to establish a comprehensive understanding of the workflow for generating situation reports. The construction of the storyboard was guided by detailed information on situation reports (in \S{\ref{sec:situation_report_background}}), whereas the key findings (in \S{\ref{sec:formative_findings}}) shaped the formulation of probing questions presented during participant interaction with the storyboard.
        \item \textbf{Annotation and Brainstorming:} Participants were tasked with providing detailed descriptions of each storyboard panel to ensure comprehension of the depicted scenario and workflow. Additionally, they annotated the storyboards for subsequent analysis, establishing a basis for identifying user needs and interaction strategies. This phase also included a semi-structured interview, with questions derived from the key findings of the formative study. (e.g. KF1: In which of these steps, would you be willing to use technology? How and why would you use it?) 
        \end{enumerate}
    \item \textbf{Training Task}
    \begin{enumerate}
        \item \textbf{Familiarization Exercise:} Participants simulated each storyboard step using sample situations to gain practical workflow experience. Two distinct question sets, varying in scope and complexity, were assigned to IA1-IA5 and IA6-IA10.          
        \item \textbf{Tool Utilization:} Candidates were advised to utilize diverse resources, including web searches like Google and Bing, and Large Language Models such as ChatGPT, for task completion.     
        \end{enumerate}
    \item \textbf{Simulation Report}
    \begin{enumerate}
        \item  \textbf{Situation Handling:} Participants were each presented with a unique situation and tasked (as above, in the training task) with facilitating the end-to-end process of creating a complete situation report.
        \item \textbf{Process Documentation and Ideation:} Throughout this simulation, participants documented their rationale for tool selection, challenges encountered, and potential system improvements. This feedback was critical to defining the Design Strategies (in \S{\ref{sec:design_strategies}}).
    \end{enumerate}
\end{enumerate}

\begin{figure}[t]
    \centering
    \includegraphics[width=1.0\textwidth]{tables/storyboard.jpg}
    %\vspace{-1em}
    \caption{Storyboard used in the collaborative design sessions with intelligence analysts.}
    \label{fig:storyboard}
\end{figure}

\subsection{Results}
\label{sec:collaborative_results}
In analyzing the data collected during the collaborative design sessions, we found three clear themes emerged.

\label{sec:formative_results}
\subsubsection{Enhancing analytical efficiency and reducing cognitive load} 
\label{sec:4_2_1}
Participants expressed a need for user interfaces that closely mirror their mental models of data analysis and report generation. Specifically, one analyst described their ideal tool as, ``A system that reflects our thought processes, almost as if the tool 'thinks' like an analyst.'' (IA1). Such feedback indicates a preference for interfaces that are instinctive and reduce the need for deliberate navigation, thereby enabling analysts to concentrate on the strategic aspects of their work. Moreover, analysts (IA1-IA5, IA7) underscored the significance of customizing data presentation in the interface to efficiently distill valuable insights from raw data, thereby decreasing the time and effort needed for data analysis.

\subsubsection{Trust and reliability in AI-systems} 
\label{sec:4_2_2}
Here the focus was on building trust in how the system operates, through its interpretability and transparency. Participants stressed the importance of understanding the automated system's underlying logic and methodology. Analysts called for explicit clarity in the system's data processing and analysis methods. Specifically, one analyst stated, ``For me to trust the system, I need to understand how and what it does. I want to know the brains behind it.'' (IA3). 

Transparency in data sources and information flow was another key aspect highlighted by the participants, emphasizing the importance of traceability from report contents to original sources. This traceability enhances the credibility and trustworthiness of the automated system. As IA9 noted, "Understanding where the information is coming from is so so important. It's knowing that the information didn't just come from nowhere." (IA9).

The sessions indicated a marked preference for outputs that enhance user comprehension and verification of the system's conclusions. Analysts favored reports offering summaries with annotations or data source references, allowing them to corroborate the system's findings against their own expertise. This method promotes an interplay between building trust with personal verification. One participant emphasized the need for transparency, requesting, ``Explain to me how you came up with the answer.'' (IA2). 

\subsubsection{Customization and flexibility with automated tools.}
\label{sec:4_2_3}
Participants desired a tool that not only accommodates different analytical styles but also various levels of detail and complexity in reporting. One analyst (IA1) highlighted the necessity for an interface that can dynamically alternate between in-depth data analysis and high-level overviews, depending on the intended reason. Additionally, there was a notable interest in a platform that allows users to specify the sources of data, such as news reports, social media feeds, and official records. This would enable the aggregation and analysis of information from these diverse origins in a unified manner, providing a more comprehensive perspective on the subject matter.

\subsection{Design Strategies}
\label{sec:design_strategies}
From the findings of the formative study (in \S{\ref{sec:formative_study}}) and the results of the collaborative design above, we identified the following design strategies:

\begin{itemize}
    \item \textbf{DS1:} Given the emphasis on reducing cognitive load and enhancing analytical efficiency (KF1 and \S{\ref{sec:4_2_1}}), the system will be designed with an interface, that mirrors intelligence analysts' natural processes of data analysis and report generation.
    \item \textbf{DS2:} To increase efficiency (\S{\ref{sec:4_2_1}}), the system will integrate features to automate time-intensive tasks such as question curation and preliminary research, thereby reducing analysts' manual workload and enabling greater focus on strategic analysis and decision-making.
    \item \textbf{DS3:} The design, addressing the need for trust and reliability (KF2 and \S{\ref{sec:4_2_2}}), will convey clear explanations of the system’s data processing algorithms and criteria. This includes transparent data sourcing, providing references within reports, and tools for users to easily understand and verify the system's conclusions. The design will also facilitate incremental trust-building through consistent and validated performance over time.
    \item \textbf{DS4:} Addressing the themes of customization and flexibility (KF3 and \S{\ref{sec:4_2_3}}), the system will offer a high degree of adaptability to accommodate various analytical styles and levels of detail in reporting. It will include features for adjusting the depth of analysis, focusing on specific data sets, and seamlessly integrating various data sources.
    %\item \textbf{DS5:} \textcolor{blue}{The system will be equipped with transparent mechanisms that elucidate the processes of data handling and summary creation (\S{\ref{sec:4_2_1}}). By making these processes clear and understandable, the design aims to deepen user comprehension of the automated functions, thus enhancing confidence and trust in the system's outputs.}
\end{itemize}










%% \section{Problem formulation}
\section{SmartBook System}\label{sec:smartbook}

\begin{figure}[t]
% \def\w{0.55\linewidth}
\centering
% \begin{tabular}{*3c}
% \includegraphics[height=\w, trim=35 30 35 10]{Web.pdf}
\includegraphics[width=1\linewidth]{tables/smartbook.jpg}
% \end{tabular}
\caption{A screengrab of SmartBook's front-end interface. Within the given situation, the user can navigate timelines (F1), explore strategic questions related to an event (F2), read the overarching summary on a given strategic question (F3), control the depth and length of information (F4), investigate all the claims in the summary (F5), correlate each claim to corresponding summary fragment (F6), investigate the source metadata (F7) and read the context in which the claims were extracted (F8).}
\label{fig:system1}
\end{figure}
%\subsection{Framework}
%\revanth{Notes: This section has been considerably compressed, with details moved into the appendix as per reviewer suggestions.}
\begin{figure}
    \centering
    \includegraphics[width=0.95\linewidth]{figures/SmartBook_arch2.png}
    \vspace{-0.5em}
    \caption{\small Overall workflow for constructing \textit{SmartBook}. Given the articles corresponding to a specific timeline, the figure shows the process for obtaining the chapters, their section headings, and the corresponding section content.}
    \label{fig:overall_workflow}
    \vspace{-1em}
\end{figure}

Using the four design strategies, we developed \textit{SmartBook}, an AI-assisted system for situation report generation that provides analysts with a first-draft report to work from as they respond to time-critical information requirements on emerging events.

SmartBook consists of: 1) An intuitive user interface (shown in Figure \ref{fig:system1}) built using the design strategies from \S{\ref{sec:design_strategies}}, and 2) a back-end framework (shown in Figure \ref{fig:overall_workflow}) that, when given a collection of documents from a variety of news sources, automatically generates a situation report. The situation reports, as per analysts' expectations and recommendations, are presented with a logical structure and organized chronologically as timelines, while being grounded to factual content.

%Given a collection of documents from a variety of sources, we aim to generate a situation report that embodies characteristics such as containing salient information that is presented with a logical structure and organized chronologically as timelines, while being grounded to factual content.} 

In this section, we describe the various components within SmartBook, along with emphasizing the advantages of each aspect of SmartBook's design for users (i.e., intelligence analysts) and for recipients of the final SmartBook report (i.e., decision-makers) who both initiate information requirements and are downstream readers. For in-depth technical details on SmartBook's back-end functionality, we point the reader to Section \ref{sec:appendix} of the Appendix. 
 
%\textcolor{blue}{Figure \ref{fig:overall_workflow} gives an overall workflow diagram for our approach, with each of the steps briefly described below. 
%Further,  

\begin{enumerate}
    \item \textbf{Major Events within Timelines as Chapters}: Situation reports cover event progressions over considerably long periods. Hence, it is beneficial to organize such reports in the form of timelines (F1 in Fig. \ref{fig:system1}) informed by DS1, which enables seamless report updates~\cite{ma2023structured} with new events and helps facilitate~\cite{singh2016expedition} users tracking and understanding of situation context. Timelines aid intelligence analysts in understanding event progression and predicting future trends by organizing events chronologically and highlighting cause-and-effect relationships. For readers, especially those less familiar with the subject, timelines provide a visual guide to easily grasp the sequence and significance of events in a scenario.    
    Our automatic situation report has timelines to provide a coherent, chronological representation of event developments (DS1, DS2). Each timeline spans a duration of N weeks\footnote{While more fine-grained timelines can be considered for relatively short-term events, we consider $N=2$ for a more spread-out scenario such as the Ukraine-Russia crisis. This reflects similar biweekly reporting timelines considered by organizations such as \href{https://datafriendlyspace.org/reports/}{Data Friendly Space}.}. %, allowing for manageable capturing and focused analysis of significant newsworthy occurrences. 
    The newsworthy major events within a timeline are identified via clustering~\cite{jain1988algorithms} news articles, to serve as the foundation for corresponding chapters within SmartBook. A key to further improving interpretability is deriving a short chapter name for the event, which we achieve by generating a short headline using multi-document summarization~\cite{headline2020}. %to facilitate information readability and retrieval for the timeline within the situation report. 
    Moreover, with the generated chapter name, we query Google News to obtain an expanded set of news articles relevant to the event.\\    
    
    \item \textbf{Strategic Questions as Section Headings}: Guided by DS2, a situation report should be logically structured with descriptive chapter headings and section titles for ease of understanding and information access.
    (F2 in Fig. \ref{fig:system1}). Structured reports benefit intelligence analysts by simplifying complex situation analysis, offering easy navigation, clear information hierarchy, and improved context understanding. For less experienced readers or those seeking specific information, the logical structure enhances information retrieval, and comprehension, while providing an intuitive mental map, making reports more reader-friendly.     
    Beyond simply describing event details in each major event chapter, SmartBook aims to provide information from a strategic perspective that can help aid decision-making and policy planning. To guide such detailed chapter analysis, we incorporate a logical structure through automatically generating section headings in the form of strategic questions relating to each major event. These question cover detail such as the possible motivations of the actors in an event and the future implications of the event. The questions are generated by prompting LLMs~\cite{brown2020language} with a grounded context in the form of news articles from the event cluster. Further, we promote diversity in the questions by using sampling strategies~\cite{holtzman2019curious} during generation, and then automatically de-duplicate~\cite{Chen2017QuoraQP} to obtain a set of unique strategic questions about the event. 
    
    \item \textbf{Extraction of Claims and Hypotheses:} 
    Automated situation report generation should be able to identify and extract the most scenario-relevant and crucial information across multiple documents (F5 in Fig. \ref{fig:system1}). Intelligence analysts, given high stakes nature of their work, but their limited time, need systems that quickly identify key information in documents (DS2). This enables them to focus on urgent matters without sorting through irrelevant data. Readers of the situation report benefit from information salience because they are presented with a concise, relevant overview of a situation. Essential points are highlighted, enhancing accessibility and clarity of the report's implications. Moreover, we present the bias of each news source\footnote{We collect information from \href{https://www.allsides.com/media-bias/ratings}{AllSides Media Bias} ratings.}, to help analysts consider information presented from different perspectives.
    Providing readers with a comprehensive understanding of event context requires foraging for different claims and hypotheses from the source documents (i.e., news articles) that help explain a situation \cite{toniolo2023human}. We adopt a Question Answering (QA) formulation to identify claims relevant to a given strategic question, motivated by recent studies \cite{reddy2022newsclaims, reddy2022zero} which have shown that directed queries can automatically extract claims from news articles relevant to a specific topic.\\
        
    \item \textbf{Grounded Summaries as Section Content}: 
    To be reliable, a situation report must be grounded in verifiable sources, as grounded factual content helps to build credibility (DS3). Reliable information is crucial for analysts, enabling them to verify facts from sources and draw solid, evidence-based conclusions. This reduces the need for additional fact-checking, as evidence is directly linked. Moreover, for those readers wanting to delve deeper or explore related topics, the embedded links act as a springboard to more extensive research (F6, F7, F8 in Fig. \ref{fig:system1}). Further, we present summaries with three levels of detail (less detailed, normal and more detailed corresponding to 2-3 sentences, 4-6 sentences and 2 paragraphs respectively) (F4) to allow readers to customize the level of detail at which they prefer to consume content, building from DS4.
    The individual sections comprise the core content of our situation report, and contain grounded, query-focused summaries which address the strategic questions. The extracted claims are combined into a concise summary by using LLMs with a novel prompting mechanism, which also generates citations linking each summary fragment to its factual input claim. This allows analysts to verify the summary information against input sources as needed.
\end{enumerate}
%
Ultimately, the structured incorporation of timeline chunking, major event chapter clustering, strategic question headings, claim extraction and query-focused section summaries within chapters enables our SmartBook system to generate reliable, insightful reports to assist analysts in responding to information requirements about time-sensitive, emerging situations. 





% \section{System Evaluation Overview}
To comprehensively evaluate SmartBook and its efficacy in aiding intelligence analysis, our approach encompassed two distinct but interrelated studies: a user study and a content review. The user study (in (\S{\ref{sec:usability_study}}) aimed to answer four research questions concerning SmartBook's usability, focusing on its effectiveness for intelligence analysts and decision-makers in generating and interacting with situation reports. Complementing this, the content review (in \S{\ref{sec:content_review_study}})  assessed the quality of SmartBook's summaries, focusing on readability, coherence, and relevance, alongside an editing study where an expert analyst edited the generated reports to understand their utility as preliminary utility. These combined efforts provided a holistic understanding of SmartBook's practicality and effectiveness in the field of intelligence analysis, offering insights into both its user experience and content quality.

%{\textcolor{red}{\textbf{is this Section 6 going to remain this way, as a standalone section to overview the next 2 sections? -- Clare}}}
%\revanth{Yep, that's the plan. Does this seem like an outlier?}

%\section{User Study: Evaluating the System}
\label{sec:usability_study}
During the user study, we conducted an interview with semi-structured questions and a post-study questionnaire on the usability of the SmartBook. The study investigated the following research questions:
%\revanth{RQ3 seems abrupt. Maybe move it to last RQ and also brief mention earlier that this was done from the perspective of both intelligence analyst (user) and decision-makers (downstream reader)}
\begin{itemize}
    \item \textbf{RQ1:} Can intelligence analysts successfully use SmartBook to generate situation reports?
    \item \textbf{RQ2:} How do intelligence analysts interact and leverage the features within SmartBook?
    \item \textbf{RQ3:} Do intelligence analysts find SmartBook intuitive, usable, trustable and useful?
    \item \textbf{RQ4:} How do decision-makers interact, perceive and use SmartBook?
\end{itemize}

\subsection{Method}

%\heng{a lot of these content overlaps with those in section 3, is this intentional?}
%\daniel{Thematically, there will be content overlaps with the previous sections. However, this is a reaffirmation that we developed SmartBook in line with user expectations and needs. I've added more details, to be granular about the findings in the user study.}

\subsubsection{Recruitment}
For this user study, we recruited a total of 12 participants, comprising of 10 intelligence analysts (IA1-IA10) and 2 decision-makers (DM1 and DM2). The recruitment process for the intelligence analysts involved maintaining the existing cohort from the formative study and collaborative design. Each analyst was compensated at the same rate as previous studies. 

We implemented a targeted outreach strategy to recruit decision-makers for our study. These participants, identified through email addresses sourced from government boards and social media profiles (e.g., LinkedIn, Twitter), were self-identified professionals who make decisions based on prepared reports in their official capacity. Our group comprised of individuals with current or past experience on Canadian government boards. These decision-makers engaged in the initial qualitative study but did not partake in the subsequent post-study questionnaire, due to time constraints. Unlike intelligence analysts, these participants did not receive compensation due to their government affiliation.

%\textcolor{blue}{For the decision makers, we employed a targeted outreach strategy. We reached out to potential participants utilizing emails obtained from government boards or social media platforms (e.g. LinkedIn, Twitter). The candidates self-identified as decision makers, defined as individuals who have made decisions based on reports which had been prepared for them in their professional capacity. The decision makers recruited for our study included individuals who are currently or have previously served on government boards in Canada. The decision makers participated in the qualitative study described first below, but not the post-study interview due to their limited availability. Unlike the intelligence analysts, the decision makers were not compensated due to their government affiliation.} 

\subsubsection{Study Task}
Each user study session was structured into four segments: (i)~an introductory overview, (ii)~a free-form investigation, (iii)~a guided exploration, and (iv)~a concluding reflective discussion. These sessions, each lasting about an hour, were remotely conducted, with participants engaging through their personal computers and browsers, and screen sharing their activity.

Upon commencement of the study~(i), participants received a concise introduction to SmartBook, outlining its core premise. This orientation was designed to acquaint them with the system without biasing their exploration. Subsequently~(ii), participants were invited to freely investigate SmartBook, with the specific task of exploring a minimum of three questions across five chapters of their choosing. This approach afforded substantial freedom, enabling interactions with the user interface that reflected their natural inclinations and interests.

Following the free-form investigation, participants were systematically introduced to the SmartBook framework. This process entailed tracing the journey from a chosen question related to a specific event within a designated timeframe, to the corresponding claims, contexts, and sources, culminating in a summarized answer. This tracing aligned with the F-numbered features illustrated in Figure~\ref{fig:system1}. The selection of questions for this exercise were informed by criteria such as question complexity and the variety of sources referenced. During this guided exploration phase~(iii), we ensured comprehensive exposure to each SmartBook feature, aiming for participants to fully grasp the system's capabilities.
Subsequently, a semi-structured interview~(iv) was conducted to gather reflective feedback on the participants' experience with SmartBook, focusing on its efficacy and potential areas of enhancement.

Finally, to conclude the session, participants were asked to complete a post-study questionnaire. This questionnaire was focused on assessing the usability of SmartBook, gathering quantitative data to complement the qualitative insights gained from the semi-structured interviews. Decision-makers were exempted from the post-study questionnaire.

\subsection{Results}
The data collection and analysis followed a similar method as in \S{\ref{sec:formative_analysis}}, with the addition of a post-study questionnaire. Upon collecting, discussing and iterating on the data, behaviors and insights were merged into the following themes:

\begin{figure}[t]
    \centering
    \includegraphics[width=1.0\textwidth]{tables/userstudy.jpg}
    \vspace{-3em}
    \caption{Quantitative results from the post-study questionnaire in the user study.}
    \label{fig:userstudy}
\end{figure}

%\revanth{Write a brief outline here on how you came up with or grouped these themes together.}
%\revanth{Add in a reference to and discuss the quantitative results shown in Figure 5.}
%\revanth{For subsection, we should add in a relevant quote or two from what the analysts/decision makers said in the studies/interviews.}

\subsubsection{UI Understandability and Interaction}

%The user interface (UI) of SmartBook was highly effective, as evidenced by the ease of navigation and feature discovery demonstrated in our data analysis. All participants (10 out of 10) successfully discovered each feature in SmartBook, as seen in (ii), demonstrating the UI's ease-of-understanding and low learning curve. Participants (IA2, IA3, and IA6) notably appreciated the logical organization into chapters and questions, clear navigation, and easy access to various system components. The high feature discovery rate indicates the UI's alignment with users' cognitive processes during information gathering, allowing intuitive interaction and access to system capabilities with minimal training. This positively addresses {\textbf{RQ1}}. IA2 particularly noted the advantages of SmartBook's similarity to their manual process of creating situation reports.:

The structured layout of the user interface (UI) was notably effective, as seen in the free-form investigation~(ii) with a 100\% feature discovery rate (10 out of 10 participants). All participants successfully engaged with key features, underscoring the UI's design intuitiveness. Participants IA2, IA3, and IA6 commended the logical organization into chapters and questions, as well as the straightforward navigation and component access. This high feature discovery rate indicates that the UI aligns with users' cognitive patterns, facilitating intuitive interaction without extensive training, providing us with a positive response to {\textbf{RQ1}}. Participant IA2 appreciated the UI's alignment with their existing workflow:

%The structured layout of the user interface (UI) stood out in the data analyses for the ease with which participants freely navigated and discovered the features in SmartBook, as seen in the free-form investigation~(ii) with a 100\% task completion success rate (10 out of 10 participants).Each participant successfully discovered and utilized key features of the system, a testament to the UI's design effectiveness. Participants (IA2, IA3, and IA6) highlighted the logical breakout of information into chapters and questions, the readily accessible and clear navigation from the summary to each relevant claims, and the ease with which they could access different system components. The high feature discovery rate also suggests that the UI aligns well with users' cognitive organization of content collected during their information foraging. This enables them to intuitively understand how to interact with the system and access its capabilities without extensive training or guidance, providing us with a positive answer to {\textbf{RQ1}}. IA2 mentioned the benefits of the SmartBook mirroring their manual process of generating situation reports: 

\begin{quote}
``I appreciate how I don't have to learn a new workflow just for a tool. It works how I would work.'' (IA2)
\end{quote}

This sentiment was further emphasized in the post-study questionnaire. As seen in the results in Figure~\ref{fig:userstudy}, 70\% of participants strongly agreed that most intelligence analysts would learn to use the system very quickly, while 80\% strongly disagreed that the system was difficult to navigate and use.
%This sentiment was further emphasized in the post-study questionnaire (\ref{sec:usability_study}, where 7 and 3 participants, respectively, strongly agreed and agreed that they believed most intelligence analysts would learn to use the system very quickly. This was also consistent with the responses from 8 and 3 participants, respectively, who strongly disagreed and disagreed that they found the system difficult to navigate and use. 
However, responses to issues raised by \textbf{KF3} and \textbf{DS4} revealed mixed views on UI flexibility, specifically in personalization and customization. While most found the tool easily integrateable into their workflow, a minority (30\%) remained neutral. When further prompted, participants IA8 and IA9 described the UI as a ``one size fits all'' solution, lacking personalization. As per \textbf{RQ2}, IA9 further highlighted that although it is a streamlined experience, it is inflexible to differing priorities:

%However, the participants' responses provided us with mixed answers to issues raised by {\textbf{KF3}} and {\textbf{DS4}} concerning the flexibility of the design (personalization and customization).  While most participants agreed that the tool could easily be integrated into a situation report generation workflow, some participants (3 out of 10) were neutral on its integration. When further prompted, they (IA8 and IA9) stated while the tools was generalizable and adaptable, it was nonetheless a "one size fits all" solution, that they couldn't personalize to their own experiences. An analyst further highlighted that although it is a streamlined experience, it is inflexible to differing priorities, per {\textbf{RQ2}}: 

\begin{quote}
``I don't deny that this tool is easy to use and quite understandable. But, for me to use some parts like finding sources that back a question, I have to interact with every part of the tool before I get there.''
\end{quote}

\subsubsection{Trust and Severity of Impact}
In the study, trust in the SmartBook system developed progressively, resembling the formation of trust in a human analyst. Initially, users exhibited skepticism, thoroughly scrutinizing the system's sources and assertions. This included validating the authenticity of sources, the accuracy and pertinence of ratings, and the correct representation of context. As users became more acquainted with SmartBook and evaluated its reliability, their dependence on extensive source verification lessened. As can be seen in Figure \ref{fig:userstudy}, three out of ten participants did not feel the need to conduct additional research beyond the presented information to trust the tool, while another three felt additional research was necessary. The necessity of further research was linked to the context of use; as one analyst stated:
%Trust formation with SmartBook emerged gradually during the study, akin to building a relationship with a human analyst. Initially, users approached the system with a degree of skepticism, meticulously verifying the sources and claims presented. This rigorous validation process involved checking the authenticity of each source, assessing the accuracy and relevance of the ratings, and ensuring the context was correctly captured. Over time, as users familiarized themselves with SmartBook's structures and assessed its content for reliability, their reliance on intensive source verification decreased. Three out of ten (\ref{sec:usability_study} participants disagreed that they found it necessary to research the information beyond what was presented in order to trust the tool, while 3 out of 10 participants either agreed or strongly agreed that they did find it necessary to do further research. However, when probed specifically on why they found this necessary, an analyst explained: 

\begin{quote}
``It's more about what situation we're in. If we provide incorrect information, people can get hurt.'' (IA3)
\end{quote}

This trust was found to be context-dependent, aligning with the concerns in \textbf{KF2} and \textbf{DS3}. Participants IA1 and IA2 noted that their trust varied based on the ``impact severity'' or the potential negative consequences of disseminating incorrect information. Nonetheless, a majority (eight out of ten) concurred that the information provided was accurate and reliable, a positive answer to {\textbf{RQ3}}. Upon probing, IA3 contrasted this with traditional military intelligence, which is often unquestioned.

%Put another way, the users' trust in the system was deemed context-sensitive, harkening back to issues raised in {\textbf{KF2}} and {\textbf{DS3}}. Participants (IA1 and IA2) suggested that their willingness to trust the system's output, depended on "impact severity", i.e., the negative consequences to providing incorrect information in their situation report to the downstream reader. However, in contrast to the previous statement, 8 out of 10 participants agreed that the information provided by the report was accurate and reliable, a positive answer to {\textbf{RQ3}}. Due to IA3's feedback on the prior question, we requested further elaboration: 
\begin{quote}
``Traditional intel within the military is a black box. You're given information, and you're told to blindly trust, rely and not question it. If we're talking about the accuracy and reliability of information, SmartBook at least reassures me of where it all came from, and shows me I could grow to trust it.'' (IA3)
\end{quote}

\subsubsection{Perceived Benefits for Intelligence Analysts}
The primary benefit identified was the substantial reduction in time and effort required for compiling and analyzing complex data. SmartBook's automation of initial report generation processes, such as data collection and summarization, was highly valued by analysts as it allowed more focus on thorough analysis and strategic planning. As seen in Figure \ref{fig:userstudy}, all participants agreed on the tool's utility in assisting intelligence analysts in creating situation reports, expressing satisfaction with the system-generated reports, supporting a positive response to \textbf{RQ1}.

%The most significant perceived benefit was the expected, dramatic reduction in time and effort otherwise required to compile and analyze complex data sets. Analysts appreciated how SmartBook automated the initial stages of report generation, such as data collection and summarization, allowing them to dedicate more time to in-depth analysis and strategic thinking. All 10 participants agreed to a degree (\ref{sec:usability_study}, that the tool could be useful for intelligence analysts to create situation reports and were satisfied with the report generated by the system, providing us with another positive response to {\textbf{RQ1}}. 

Regarding the necessity of significant edits to the system-generated reports, participants IA5-IA9 suggested that this need was not inherently due to report deficiencies but varied based on the report's intended purpose or audience. We later conduct a study, described later in \S{\ref{sec:editing_study}}), to understand the extent of edits needed.

%When prompted with their varying perspectives on whether it was necessary to make major edits to the system-generated situation reports, participants (IA5-IA9) noted this wasn't due to deficiencies within the reports per se, rather this would be dependent on the "intended purpose or audience". To investigate the objective basis for these varying claims about the need to edit, we also completed an editing study (Section 8.3).

%\revanth{Needs a bit of expansion and tt reads too generic currently. IMO we can't claim we did study with decision makers without giving thorough/specific details. Might need to add one more point here.}
\subsubsection{SmartBook as a Learning Tool for Decision-Makers}
Decision-makers highly valued SmartBook's ability to rapidly deliver accurate and easily digestible information. Their primary engagement with the interface centered around utilizing the summaries, to learn about different topics in various degrees. The system's effectiveness in simplifying complex data into structured, clear formats was particularly appreciated, to help aid in swift understanding and decision-making processes.
%For decision makers, the capability of SmartBook to provide rapid, accurate, and easily digestible information was considered invaluable. Rather than exploring different components of the SmartBook interface, they focused in on the summaries, learning about different topics in various degrees. They particularly valued the system's capacity to distill complex data into concise, structured formats that would facilitate their own quick understanding and decision-making. One of the decision makers remarked:

\begin{quote}
''When something happens, I'm expected to know about it. I want to learn as fast as I can, and get the information I only need.'' (DM2)
\end{quote}

Although data lineage and source transparency were recognized as important, these were considered secondary to the primary need for timely and format-specific information delivery. While SmartBook's target users are intelligence analysts, the decision-makers' view highlights SmartBook's dual functionality as both an analytical tool and a decision-support system, providing a key capability for the high-paced and information-intensive needs of decision-makers, addressing \textbf{RQ4}.

%While aspects like data lineage and source transparency were important, these were considered secondary to the primary need for timely and format-specific information delivery. While SmartBook's target users are intelligence analysts, the decision makers' view highlights SmartBook's dual functionality as both an analytical tool and a decision-support system, providing a key capability for the high-paced and information-intensive needs of decision makers, addressing {\textbf{RQ4}}.



%

\section{Content Review: Evaluating the Generated Content}
\label{sec:content_review_study}

%\heng{It will be more exciting to add the examples comparing with gpt+bing, including pages 9-14, 81-82 and in this set of slides: https://blender.cs.illinois.edu/ALTA\_NLP\_SmartBook\_HengJi.pdf}


In this section, we seek to evaluate the quality of the text summaries generated by \textsc{SmartBook}. Our approach is complementary to the usability study in \S{\ref{sec:usability_study}}, which assessed whether \textsc{SmartBook} is useful to intelligence analysts and decision-makers. We first evaluate the system's summaries for readability, based on coherence, relevance, and tactical usefulness of the information (in \S{\ref{sec:readability_study}}). Next, we evaluate whether the automatically generated situation reports can serve as a useful preliminary draft with an editing study to ascertain how extensively an expert analyst edits the summaries to deem them acceptable (in \S{\ref{sec:editing_study}}).

%To do so, an evaluation schema was developed by the expert intelligence analyst, based on coherence, relevance and tactical usefulness of the information. 
\subsection{Participant Recruitment}
%\revanth{Should we be calling the students as text evaluators or text annotators or text analysts?}
The studies involved two participant groups: one senior expert analyst (SA1) and six text evaluators (TE1-TE10). 
The expert, affiliated with a US national defense and security firm, possessed a decade of experience in intelligence analysis, training, and AI-based tool assessment. This included researching intelligence tutoring systems and testing AI-based search tools for analysts. The text evaluators, all fluent in English and US-based, held Computer Science degrees ranging from Bachelor's to PhDs, and had prior text annotation experience. There were three natural language processing graduate students and three undergraduates in computer science. The participants reflected the composition of a training unit, with the senior expert providing supervision and training in system evaluation. Compensation varied between groups, with the expert working under a research contract and evaluators receiving \$20 per hour.
%Two types of participants were recruited: (1) an expert analyst and (2) 6 text evaluators. The expert analyst, sourced through a partnership with an independent US-based national defense and security firm, had 10 years of experience collaborating with intelligence analysts, particularly in the realms of training and tool evaluation. This included researching intelligence tutoring systems and testing AI-based search tools with for analysts. The six text evaluators had advanced degrees in Computer Science (ranging from Bachelors to PhDs) and prior experience in text annotation. They included three natural language processing graduate students and three computer science undergraduates, all fluent in English and based in the US. The recruitment of a single expert analyst and six text evaluators was curated to reflect the dynamics of a training unit, where the text evaluators underwent supervision and training for system evaluation by the expert analyst trainer. Compensation was structured differently for each group: the expert analyst operated under an independent research contract, while each text evaluator received 20 USD per hour. 

%The intelligence analyst trainer had 10 years of experience in intelligence analyst training and intelligence tool evaluation, and was sourced through a partnership with an independent US-based national defense and security firm. 
%Next, six text analysts with advanced degrees in Computer Science (ranging from Bachelors to PhDs) and at least a year of text annotation experience were recruited. These included three natural language processing graduate students and three computer science undergraduate students, all fluent in English and based in the US. The recruitment of the single intelligence analyst trainer and six text analysts was curated to reflect a trained unit dynamic, where the text analysts underwent supervision and training by the intelligence analyst trainer for the system evaluation. Compensation was structured differently for each group: the intelligence analyst operated under an independent research contract, while each text analyst received 20 USD / hour. 

\subsection{Readability Study}
\label{sec:readability_study}

\subsubsection{Baselines}

\textsc{SmartBook} leverages question-driven claim extraction (as described in \S{\ref{sec:smartbook}}) to pinpoint relevant information, from which it then generates the final summary. For baseline comparisons, we consider two alternatives. One is a \textit{query-focused summarization} baseline that takes the news articles as input directly (no extraction) and uses a large language model (LLM) to generate a query-focused summary\footnote{We use the same model to generate SmartBook and query-focused summaries. The prompt used is the same as in the summary generation example in Figure \ref{fig:overall_workflow}, except the entire news article texts are passed as context.}. This baseline's query here corresponds to the \textsc{SmartBook}'s strategic question. 
Further, since \textsc{SmartBook} uses news articles that are relevant to the event as the background corpus, we also compare it against a second web search + LLM baseline, using the web as the background corpora, first obtaining relevant web pages from internet search and then summarizing them using a large language model. This second baseline simulates a competitive strategy of directly obtaining information from an LLM-enabled web search engine\footnote{Specifically, we leverage \href{https://www.perplexity.ai}{perplexity.ai}, which combines web-search with LLM summarization. The query used is the strategic question, along with a phrase ``\textit{with respect to the Ukraine-Russia crisis between $<$timeline$>$}'' to ground the query.} (the \textit{web search + LLM} baseline is similar to a Bing search + a ChatGPT query for summary). We randomly selected summaries for 50 of \textsc{SmartBook}'s strategic questions to perform the readability evaluation.

\subsubsection{Study Setup}

For each generated summary, three text evaluators are asked to assess in terms of coherence, relevance, and usefulness with the scores ranging from 1 (worst) to 5 (best). For coherence and relevance, we follow the guidelines as in \cite{fabbri2021summeval}. Simply, \textit{coherence} measures the quality of all summary sentences collectively regardless of the question. \textit{Relevance} quantifies whether the summary contains key points that answer the given question. We define \textit{usefulness} as an indication of whether the summary provides non-trivial and insightful information for analysts, and suggests the breadth and depth of the provided key points. 

\begin{table}
\centering
\begin{tabular}{lcccc}
\hline
\textit{Summary-Type} & \textit{Background} &  \textit{Coherence} & \textit{Relevance}  &\textit{Usefulness} \\
\hline
% (1) Query-Focused & News & 87.90& 76.93& 75.47\\
% (2) Web Search + LLM & Web &88.80 & 81.33& 76.53\\
% (3) SmartBook & News & \textbf{91.20} & \textbf{84.67} & \textbf{83.47}\\

(1) Query-Focused & News & 4.39& 3.85& 3.77\\
(2) Web Search + LLM & Web &4.44 & 4.07& 3.83\\
(3) SmartBook & News & \textbf{4.56} & \textbf{4.23} & \textbf{4.17}\\


\hline
\end{tabular}
\caption{Results of human evaluation comparing the summaries within SmartBook (3) against those either generated directly from news articles via query-focused summarization (1) or obtained from the internet by web search combined with generation from a large language model (2). Results are on a 1-5 scale (worst to best). 50 summaries were evaluated with a text evaluator coverage of 3.}
\label{tab:sum_human}
\vspace{-2em}
\end{table}


\begin{table}[t]
    \centering
    \small
    \renewcommand{\arraystretch}{1.5}
    \begin{tabular}{p{17.9em}|p{14.3em}|p{17.4em}}
    \multicolumn{3}{c}{\textbf{Question:} How have kamikaze drones changed the nature of the Russia-Ukraine conflict?} \\
    \hline
     \multicolumn{1}{c|}{\centering \textbf{Query-Focused}}  & \multicolumn{1}{c|}{\centering \textbf{Web Search + LLM}}  &\multicolumn{1}{c}{\centering \textbf{SmartBook}}\\
     \hline
    Kamikaze drones, or suicide drones, have been used by Russia in its attacks on Ukrainian cities such as Kyiv, Vinnytsia and Odesa. They are known as a ``loitering munition'' because they are \textcolor{blue}{capable of waiting in an area identified as a potential target and only strike once an enemy asset is identified}. Russia is using Iranian-made Shahed-136 drones, which have a payload of approximately 50 kilograms and are \textcolor{blue}{hard to detect due to their small size}. Ukraine has responded by \textcolor{blue}{developing an app that civilians can use to report sightings of incoming drones} and missiles, and increasing the proportion shot down before they hit the ground. &  Russia \textcolor{red}{and Ukraine} have been using Iranian-made Shahed-136 drones in the conflict. These drones are known as `kamikaze' drones due to their \textcolor{blue}{ability to self-destruct on impact}. The use of these drones has been highlighted by \textcolor{blue}{Russia's growing reliance on them and their use as both military weapons and psychological weapons}. On October 13, 2022, kamikaze drones and missiles hit north and south Ukraine, though the Kremlin denied any involvement from President Putin in talks between Russia and the West. & The Russia-Ukraine conflict has seen a \textcolor{blue}{paradigm shift in the nature of Russian tactics, with the increased reliance of kamikaze drones over artillery fire}. These drones, which are \textcolor{blue}{designed to crash into a target and explode, can spray potentially deadly debris} and are \textcolor{blue}{being used to plug gaps in the deployment of Russian forces along the front line}. Russia is reportedly using Iranian-made Shahed series drones, which are showcased at Kashan Airfield south of Tehran. This shift in tactics has seen an \textcolor{blue}{escalation in the aerial bombardment of densely populated urban areas and an attack on critical infrastructure and power sources}.\\
       \hline
    \end{tabular}
    \caption{An example comparing the summaries produced by different approaches evaluated in Table \ref{tab:sum_human}. It can be seen that the SmartBook summary contains considerably more question-relevant information (highlighted in \textcolor{blue}{blue}). The \textit{Web Search + LLM} baseline output contains hallucinated content (highlighted in \textcolor{red}{red}) that suggests \textit{both} Russia and Ukraine have been using the drones.}
    \label{tab:summaries_comparison}
    \vspace{-2.5em}
\end{table}

\subsubsection{Results}
%\begin{table}
\centering
\begin{tabular}{lcccc}
\hline
\textit{Summary-Type} & \textit{Background} &  \textit{Coherence} & \textit{Relevance}  &\textit{Usefulness} \\
\hline
% (1) Query-Focused & News & 87.90& 76.93& 75.47\\
% (2) Web Search + LLM & Web &88.80 & 81.33& 76.53\\
% (3) SmartBook & News & \textbf{91.20} & \textbf{84.67} & \textbf{83.47}\\

(1) Query-Focused & News & 4.39& 3.85& 3.77\\
(2) Web Search + LLM & Web &4.44 & 4.07& 3.83\\
(3) SmartBook & News & \textbf{4.56} & \textbf{4.23} & \textbf{4.17}\\


\hline
\end{tabular}
\caption{Results of human evaluation comparing the summaries within SmartBook (3) against those either generated directly from news articles via query-focused summarization (1) or obtained from the internet by web search combined with generation from a large language model (2). Results are on a 1-5 scale (worst to best). 50 summaries were evaluated with a text evaluator coverage of 3.}
\label{tab:sum_human}
\vspace{-2em}
\end{table}



Table~\ref{tab:sum_human} shows the results from the evaluation study of the summaries. We observe that SmartBook outperforms alternative competitive strategies on coherence, relevance, and usefulness. The benefit of our question-driven claim extraction step can be seen from the considerably more relevant summaries within SmartBook, compared to the direct query-focused summarization of the news articles without an explicit extraction step (row (3) vs (1)). Further, we see that directly obtaining information from the internet can give less useful content compared to the focused news-driven summarization within \textsc{SmartBook} (row (2) vs (3)). A set of summary examples is shown in Table \ref{tab:summaries_comparison}.
This evaluation demonstrates across the metrics that \textsc{SmartBook} provides valuable content in its summaries and outperforms the two baselines lacking an extraction step. 

\subsection{Editing Study}
\label{sec:editing_study}

\subsubsection{Study Setup}
% Revise to make it read like more a basic evaluation setup.

 The study was conducted with the expert analyst. As an initial exercise, the expert analyst was provided with 5 strategic questions from \textsc{SmartBook} and wrote summaries without the assistance of AI-assisted tools in the report generation. Following this, the analyst was encouraged to actively explore the features of \textsc{SmartBook}. The analyst was provided with 94 randomly sampled summaries from SmartBook-generated situation reports, and was asked to edit them till their content was acceptable as a professional report from an intelligence analyst. The \textsc{SmartBook} interface to the summaries were provided, and the analyst completed edits in a text editing software of their choice.

\subsubsection{Results}
\label{sec:first_draft}
%\begin{table}[t]
    \centering
    \small
    \renewcommand{\arraystretch}{1.5}
    \begin{tabular}{p{17.9em}|p{14.3em}|p{17.4em}}
    \multicolumn{3}{c}{\textbf{Question:} How have kamikaze drones changed the nature of the Russia-Ukraine conflict?} \\
    \hline
     \multicolumn{1}{c|}{\centering \textbf{Query-Focused}}  & \multicolumn{1}{c|}{\centering \textbf{Web Search + LLM}}  &\multicolumn{1}{c}{\centering \textbf{SmartBook}}\\
     \hline
    Kamikaze drones, or suicide drones, have been used by Russia in its attacks on Ukrainian cities such as Kyiv, Vinnytsia and Odesa. They are known as a ``loitering munition'' because they are \textcolor{blue}{capable of waiting in an area identified as a potential target and only strike once an enemy asset is identified}. Russia is using Iranian-made Shahed-136 drones, which have a payload of approximately 50 kilograms and are \textcolor{blue}{hard to detect due to their small size}. Ukraine has responded by \textcolor{blue}{developing an app that civilians can use to report sightings of incoming drones} and missiles, and increasing the proportion shot down before they hit the ground. &  Russia \textcolor{red}{and Ukraine} have been using Iranian-made Shahed-136 drones in the conflict. These drones are known as `kamikaze' drones due to their \textcolor{blue}{ability to self-destruct on impact}. The use of these drones has been highlighted by \textcolor{blue}{Russia's growing reliance on them and their use as both military weapons and psychological weapons}. On October 13, 2022, kamikaze drones and missiles hit north and south Ukraine, though the Kremlin denied any involvement from President Putin in talks between Russia and the West. & The Russia-Ukraine conflict has seen a \textcolor{blue}{paradigm shift in the nature of Russian tactics, with the increased reliance of kamikaze drones over artillery fire}. These drones, which are \textcolor{blue}{designed to crash into a target and explode, can spray potentially deadly debris} and are \textcolor{blue}{being used to plug gaps in the deployment of Russian forces along the front line}. Russia is reportedly using Iranian-made Shahed series drones, which are showcased at Kashan Airfield south of Tehran. This shift in tactics has seen an \textcolor{blue}{escalation in the aerial bombardment of densely populated urban areas and an attack on critical infrastructure and power sources}.\\
       \hline
    \end{tabular}
    \caption{An example comparing the summaries produced by different approaches evaluated in Table \ref{tab:sum_human}. It can be seen that the SmartBook summary contains considerably more question-relevant information (highlighted in \textcolor{blue}{blue}). The \textit{Web Search + LLM} baseline output contains hallucinated content (highlighted in \textcolor{red}{red}) that suggests \textit{both} Russia and Ukraine have been using the drones.}
    \label{tab:summaries_comparison}
    \vspace{-2.5em}
\end{table}

 Given the revised summaries from the expert analyst, we aimed to measure the extent to which they made changes to the original summary content. Specifically, these edits were quantified in the form of (a) commonly used token-overlap-based metrics such as BLEU [51] and ROUGE [38] scores, and (b) Levenshtein edit distance [47], which measures the character-level changes required to transform the original summary to the revised summary. %These analyses were completed across all 94 randomly sampled summaries from SmartBook. 
Empirical results indicate that token overlap between generated and post-edited texts were high, with BLEU and ROUGE-L scores respectively at 59.0\% and 74.1\%. These metrics suggest that the generated reports from SmartBook are of sufficiently good-quality, such that extensive human expert revision may not be necessary. However, we acknowledge that a gap still exists between the automatically generated and human expert summary curation, as the Levenshtein edit distance computed at the character level is 34.4\%. Interestingly, 15\% of generated summaries were determined to be perfect, with no edits made by the expert analyst. 

Next, we examined how the edits were made. The general observation was that more content was inserted than was deleted by the expert analyst. The percentage of tokens inserted was 49.6\%, whereas the percentage of tokens deleted was only 2.3\%, suggesting that automated summary generation may generally need to be more detailed. 
While investigating the efficacy of SmartBook as an AI-assisted tool, the expert analyst reported: 

\begin{quote}
"Pointing out to individual claims, while possible, can be hard sometimes since you will be abstracting information across multiple claims." and that "SmartBook saves time." (SA1)
\end{quote}


%By quantitatively and qualitatively measuring the expert analyst's interaction with SmartBook, we understand that it acts a collaborative partner for situation report generation.
WASP-80 	 & 	20.1141	 & 	49.71	 & 	XMM	 & 	0744940101	 & 	PN	 & 	-11.65	 & 	29.83	 & 	This work  \\
HAT-P-18	 & 	6.1863	 & 	161.6	 & 	XMM	 & 	Slew      	 & 	PN	 & 	$\leq-11$	 & 	$\leq31.49$	 & 	Undetected  \\
55 Cnc   	 & 	79.4482	 & 	12.59	 & 	XMM	 & 	0551020801	 & 	PN	 & 	-13.22	 & 	27.05	 & 	This work  \\
HD63433 	 & 	44.6848	 & 	22.38	 & 	XMM	 & 	0882870101	 & 	EPIC	 & 	-11.9	 & 	28.88	 & 	Zhang+2022, 0.124-2.48 keV band  \\
WASP-12 	 & 	2.4213	 & 	413	 & 	XMM	 & 	0853380101	 & 	M2	 & 	$\leq-13.91$	 & 	$\leq29.4$	 & 	Undetected  \\
WASP-76 	 & 	5.2899	 & 	189	 & 	XMM	 & 	0853380501	 & 	M1	 & 	$\leq-13.7$	 & 	$\leq28.93$	 & 	Undetected  \\
HAT-P-32	 & 	3.4938	 & 	286.2	 & 	XMM	 & 	0853381001	 & 	PN	 & 	-13.24	 & 	28.75	 & 	 This work  \\
Trappist-1	 & 	80.2123	 & 	12.47	 & 	XMM	 & 	ALL-XMM     	 & 	EPIC	 & 	-14.04	 & 	26.23	 & 	This work  \\
WASP-52 	 & 	5.7262	 & 	174.6	 & 	Chandra	 & 	     15728	 & 	ACIS	 & 	-13.11	 & 	29.45	 & 	This work$^(a)$  \\
WASP-177	 & 	5.8129	 & 	172	 & 	XMM	 & 	Slew	 & 	PN	 & 	$\leq-12.1$	 & 	$\leq30.45$	 & 	Undetected  \\
HAT-P-26	 & 	6.9995	 & 	142	 & 	XMM	 & 	0804790101	 & 	PN	 & 	$\leq-14.63$	 & 	$\leq27.76$	 & 	Undetected  \\
NGTS-5  	 & 	3.2114	 & 	311.4	 & 	XMM	 & 	Slew	 & 	PN	 & 	$\leq-12.36$	 & 	$\leq30.7$	 & 	Undetected  \\

Table \ref{tab:human_edits} shows an example of edits (in color) made by an expert analyst for a machine-generated summary in SmartBook. We can see here that the human analyst added additional tactical information (in blue) to elaborate on certain aspects (e.g. what is special about the ``kamikaze'' type of drone). Further, the analyst can also added some interesting insights (in green) based on the information in the summary. Overall, this shows that \textsc{SmartBook} provides a good starting point for analysts to expand upon for the generation of situation reports. 

%{\textcolor{red}{\textbf{I am NOT seeing any green in Table 5. Was this lost somehow in creating it? if it cannot be found, then the caption should be adjusted and one sentence in the last paragraph, starting with "Further,..." should be dropped  -- Clare}}}

\subsubsection{Error Analysis}
\label{sec:error_analysis}
As noted in \S{\ref{sec:first_draft}}, it is noteworthy that 15\% of summaries produced by \textsc{SmartBook} needed no modifications, highlighting its proficiency in creating acceptable reports in some scenarios without human intervention. This suggests that as technology advances and iterative refinements are applied, this rate will likely improve, reducing the workload for analysts in the future. 
To gain a better understanding of the different types of errors in the remaining summaries, we asked the expert analyst to categorize the errors within them. The analyst was also shown the strategic question and the corresponding extracted contexts that were used to automatically generate the summary. The summary errors were categorized as follows:

\begin{itemize}
    \item \textit{No relevant contexts:} None of the extracted contexts are relevant to the question (and thereby the summary is expected to be fully irrelevant too).
    \item \textit{Inaccurate information in summary:} Summary has incorrect information, that is not reflective of the underlying input contexts.
    \item \textit{Incoherent summary:} Summary is incomprehensible and unclear.
    \item \textit{Incomplete summary:} Important information in the input contexts is missing in the summary.
    \item \textit{Irrelevant information in summary:} Summary has material that is not relevant to the question, despite some extracted contexts being relevant.
\end{itemize}

\begin{wrapfigure}{r}{0.4\linewidth}
\vspace{-2em} 
        \centering
        \includegraphics[width=0.85\linewidth]{figures/error_categories.png}
        \caption{Distribution of different error categories for the summaries within SmartBook.}
        \label{fig:error_cat} 
        \vspace{-1em} 
\end{wrapfigure}

Figure \ref{fig:error_cat} shows the distribution of error categories for the summaries. It can be seen that incompleteness of summaries is a predominant error, with more than 50\% of the summaries missing important information or not being sufficiently complete.  While the predominance of incomplete summaries could be a concern, this can also be framed positively: it represents a conservative approach ensuring that \textsc{SmartBook} does not over-extend based on limited data, and instead offers a foundational understanding.  Other errors corresponded to summaries with inaccurate (18.6\%) or irrelevant (14.6\%) information. Very few summaries were judged incoherent, as expected given that large language models such as GPT-3 have been shown \cite{goyal2022news, zhang2023benchmarking} to generate fluent and easy-to-read output. 




%\section{Discussion}
%Intelligence is more than data aggregation and hypothesis formation.


% In this direction, we study whether SmartBook-generated situation reports can be a useful preliminary draft, in comparison to intelligence analysts' prior experience in solely authoring situation reports.


 %This leaves room for human analysts to augment and fine-tune, merging AI's broad processing capacity with human expertise.

%\revanth{As per Agathe's feedback, this section needs to have a broad high-ranging discussion of the results and potential impact of this work.}
%\textcolor{blue}{SmartBook provides a reliable, accurate, and efficient method for digesting a complex and large corpus of news and thus also supports a wide-open vision for a new approach to the authoring of situation reports, as well as the vast range of summary reports in the business world (\ref{sec:outlook}). Our approach to the development of SmartBook represents an application of the ``research-through-design'' methodology within AI design, a strategy not widely explored in this domain. This approach offers a pragmatic, real-world setting for human-AI collaboration, by emphasizing the tangible needs of practitioners, in this case intelligence analysts and decision makers. Consequently, SmartBook stands as a significant exemplar of how AI can be effectively rooted in addressing specific, practical challenges, thus making a remarkable contribution to both Human-Computer Interaction (HCI) and AI disciplines.}

SmartBook leverages the zero-shot capabilities~\cite{brown2020language} of LLMs for tasks with limited training data. Complex tasks like grounded summary generation with citations and strategic question identification are tackled by LLMs, while smaller models address well-defined tasks with available data, such as event headline (chapter name) generation~\cite{headline2020}, duplicate question detection~\cite{cer2017semeval}, and claim extraction through question answering~\cite{kwiatkowski2019natural}. Our experiments revealed a tendency of the LLMs to generate similar strategic questions, necessitating sampling methods~\cite{holtzman2019curious} for variety. Directing the LLMs to reference input documents while generating summaries enhanced factuality, aligning with recent findings~\cite{gao-etal-2023-enabling}. Nevertheless, LLMs' tendency to hallucinate~\cite{ji2022survey, tam2022evaluating}, evident from inaccuracy errors in \S{\ref{sec:error_analysis}}, necessitates future enhancements  in validation and cross-referencing techniques (discussed in \S{\ref{sec:future_verification}}) to mitigate this issue.


In light of the ongoing developments in AI-assisted writing~\cite{WangACL2019,Wang2020,selfcollaboration2023,cardon2023challenges}, where tools have been developed for a variety of end-users like academics~\cite{gero2022sparks}, screenwriters~\cite{mirowski2023co} and developers~\cite{chen2021evaluating}, SmartBook introduces a unique automated framework designed specifically for generating situation reports for intelligence analysts. Prior AI-assisted writing systems, such as CoAuthor~\cite{lee2022coauthor}, Creative Help~\cite{roemmele2015creative}, Writing Buddy~\cite{samuel2016design} and WordCraft~\cite{yuan2022wordcraft}, are designed to be collaborative and aid in creative tasks~\cite{klein1973automatic}. SmartBook highlights a different aspect of human-AI collaboration, where AI's role is more about efficient information processing with analytical rigor and less about creative or iterative human engagement. This distinction underscores SmartBook's design with a focus on factuality, analytical depth, and efficiency in more structured, data-driven environments, such as intelligence analysis. Clearly the design also stands to benefit from iterative experimentation and refinement, to explore where deeper integration of SmartBook into analysts' workflows improve their task performance. We discuss this as future extensions in \S{\ref{sec:iterative_refinement}}.


\subsection{Outlook}
\label{sec:outlook}
%\textcolor{blue}{\textbf{TODO: Daniel - Connect brief overview of what were our takeaways / understandings (based on formative study) of how analysts originally perceived AI generated outputs, and how they were using SmartBook reports in the studies. Discussion should be high-level and ideally not repeat anything that was mentioned in results of the formative study/user study.}}

The central challenge faced by analysts is massive data overload that they must filter in addressing information requirements to produce situation reports.  The key findings of the formative study show that analysts, though appropriately cautious, are open to incorporating AI assistance into their workflow.  The desiderata in the design strategies that they articulated during the collaborative design also make clear that they are eager to participate in crafting the new vision that will include AI assistance that not only adjusts to the data content of shifting scenarios but also to their individual analytic methods.

SmartBook, characterized by its generalizable and modular design, can efficiently automate the generation of situation reports across various scenarios, including geopolitical, environmental, or humanitarian contexts. Unlike human analysts who face considerable challenges in transferring their expertise across domains, SmartBook requires minimal configuration for such shifts. This transition in traditional settings involves extensive background research and the recruitment of domain experts, demanding substantial time and effort. Furthermore, each new scenario introduces complex variables and details that must be meticulously integrated. SmartBook addresses these challenges with a streamlined and consistent approach, thereby offering accurate reports in diverse and rapidly changing global contexts.

The effectiveness of SmartBook was primarily evaluated on news articles from the Ukraine-Russia military conflict. Here, we briefly explore SmartBook's applicability in a humanitarian crisis, specifically the Turkey-Syria earthquake. By changing the source articles\footnote{Here is a URL corresponding to news for February 6: \href{https://edition.cnn.com/middleeast/live-news/turkey-earthquake-latest-020623/index.html}{link}} for major event identification, SmartBook successfully generated pertinent situation reports for the specified time window (February 6-13). These reports accurately captured critical aspects of the earthquake disaster, such as aftershock impacts (\href{https://www.cnn.com/middleeast/live-news/turkey-syria-earthquake-updates-2-7-23-intl/h_a9e1d082dd54c67187fd51322f95eaf2}{link}), international humanitarian assistance (\href{https://www.incirlik.af.mil/News/Article-Display/Article/3292036/urban-search-and-rescue-teams-arrive-at-incirlik-air-base/}{link}), rising death tolls (\href{https://www.reuters.com/world/middle-east/death-toll-syria-turkey-quake-rises-more-than-8700-2023-02-08/}{link}), and specific events like the disappearance of a Ghanaian soccer star (\href{https://www.marketwatch.com/story/soccer-star-christian-atsu-still-missing-after-turkey-earthquakes-i-still-pray-and-believe-that-hes-alive-says-partner-cfd47272}{link}). 

This adaptability of SmartBook, demonstrated in the domain shift from military to humanitarian crisis contexts, underscores its robustness and versatility. It also emphasizes the importance of the source articles and suggests that, with mission-relevant data input, SmartBook has the potential to be an invaluable tool for analysts across a multitude of scenarios. Overall, SmartBook presents an exciting step forward in AI-assisted situation report generation and signifies a momentous stride in the domain of intelligence analysis.



%\textcolor{blue}{In this section, we describe some key learnings and implications of this project (in \S{\ref{sec:learnings}}) both from the perspective of HCI+AI and intelligence analysis communities. Next, we discuss explorations (in \S{\ref{sec:transferability}}) for applying SmartBook to scenarios in other intelligence domains . Finally, we highlight some limitations of SmartBook (in \S{\ref{sec:limitations}}) and present directions for future work (in \S{\ref{sec:extensions}}).}

%Given the surprisingly high-performance zero-shot capabilities of LLMs, SmartBook leverages LLMs for tasks without considerable training data, such as grounded summary generation with citations, and identifying strategic questions. On the other hand, smaller models were leveraged for focused tasks that had training data available, such as event headline (chapter name) generation~\cite{headline2020}, duplicate question detection~\cite{cer2017semeval}, and claim extraction using question answering~\cite{kwiatkowski2019natural}. From our experiments, we observed that even when prompted to generate diverse strategic questions, the model was prone to generating very similar ones, thereby requiring sampling tricks and repeated generation to promote variety. In addition, initial explorations suggested that specifically instructing the language model to cite input documents improved the factuality of the generated summaries, as evidenced in recent works~\cite{gao-etal-2023-enabling}.However, the hallucination problem \cite{ji2022survey, tam2022evaluating} of such LLMs is evident from the considerable number of summaries (in \S{\ref{sec:error_analysis}}) with inaccurate information. This could be mitigated in future iterations (as discussed in \S{\ref{sec:future_verification}}) by incorporating stronger validation mechanisms and cross-referencing techniques.} 


%AI-assisted writing~\cite{WangACL2019,Wang2020,selfcollaboration2023,cardon2023challenges} is a rapidly growing field in human-computer interaction research. In recent years, researchers have explored various approaches to develop AI systems that can assist users in writing tasks. AI-assisted writing tools have demonstrated benefits to a variety of end-users, like academics~\cite{gero2022sparks}, screenwriters~\cite{mirowski2023co} and developers~\cite{chen2021evaluating}. CoAuthor~\cite{lee2022coauthor} crafts a dataset that encapsulates the rich interactions observed between multiple writers and instances of GPT-3 over numerous writing sessions, thereby highlighting the efficacy of AI in fostering collaborative writing endeavors. In the area of automated story-writing~\cite{klein1973automatic}, attempts such as Creative Help~\cite{roemmele2015creative}, Writing Buddy~\cite{samuel2016design} and WordCraft~\cite{yuan2022wordcraft} aid the user in producing full stories with creative control throughout the process. More recently, with humans struggling to identify text as being machine generated ~\cite{wahle2022large}, concerns have been raised about academic integrity~\cite{perkins2023academic} and plagiarism\cite{wahle2022identifying} when using AI-assisted writing tools.In light of the ongoing developments in AI-assisted writing, SmartBook introduces a unique automated framework designed specifically for generating situation reports for intelligence analysts. While the existing body of work predominantly focuses on augmenting creative and collaborative writing, SmartBook aims to assist intelligence analysts' workflows with a focus on factuality and analytical depth.

%Qualitative evaluations underscore the potent functionality of SmartBook, with a significant majority of its automatically discovered questions directing focus on strategic information. Additionally, in addressing the current geopolitical tensions in the Ukraine-Russia context, it has demonstrated its indispensability in strategizing and planning, showcasing not only its adaptability to real-world scenarios but also its capacity to provide actionable intelligence. }


%\subsection{Implications}

%\textcolor{blue}{ Firstly, it is noteworthy that 15\% of summaries produced by SmartBook needed no modifications, highlighting its proficiency in creating acceptable reports in some scenarios without human intervention. This suggests that as technology advances and iterative refinements are applied, this rate could potentially increase, reducing the workload for analysts even further.  Lastly, the hallucination problem, although present, could be mitigated in future iterations (which we discuss in Section \ref{sec:future_verification}) by incorporating stronger validation mechanisms and cross-referencing techniques.}


%\subsection{Implications}


%SmartBook emphasizes generalizability in automating the generation of situation reports. Its modular design ensures that when the need arises to transition to a new scenario - be it geopolitical, environmental, or humanitarian - SmartBook can adapt with minimal configuration. Meanwhile, traditional intelligence analysts often grapple with a multifaceted challenge when pivoting between scenarios. The transition demands a significant time investment for deep dive into background research, recruitment of domain experts, etc. Additionally, each new scenario introduces a vast array of variables and particulars that need to be meticulously accounted for and incorporated. The inherent cognitive constraints on an individual analyst's capacity to process and synthesize vast amounts of information mean that the risk of oversight or error is always present. In stark contrast, SmartBook offers a streamlined, consistent, and efficient mechanism to adapt and provide accurate situation reports across diverse scenarios in rapidly changing global contexts. %This makes it an invaluable asset in rapidly changing global contexts, ensuring timely and reliable generation of situation reports.

 %We evaluated SmartBook primarily in the context of a military conflict by choosing Ukraine-Russia crisis as the evaluation scenario. In this section, we explore the possibility of how well SmartBook can generalize to intelligence analysis for a humanitarian crisis. Specifically, we consider situation report generation for the Turkey-Syria earthquake scenario by considering a single timeline for the time window of February 6 - 13. This aims to demonstrate portability to new scenarios, by simply switching the source articles\footnote{Here is a URL corresponding to news for February 6: \href{https://edition.cnn.com/middleeast/live-news/turkey-earthquake-latest-020623/index.html}{link}} used to identify major events (described on Section \ref{subsec:major_event}). We observe that SmartBook is able to identify and present important aspects of the Turkey-Syria earthquake disaster, such as the continued impact of aftershocks (\href{https://www.cnn.com/middleeast/live-news/turkey-syria-earthquake-updates-2-7-23-intl/h_a9e1d082dd54c67187fd51322f95eaf2}{link}), international humanitarian assistance in the form of search and rescue teams (\href{https://www.incirlik.af.mil/News/Article-Display/Article/3292036/urban-search-and-rescue-teams-arrive-at-incirlik-air-base/}{link}), the rising death toll in the aftermath of the earthquake (\href{https://www.reuters.com/world/middle-east/death-toll-syria-turkey-quake-rises-more-than-8700-2023-02-08/}{link}) and missing of a soccer star from Ghana (\href{https://www.marketwatch.com/story/soccer-star-christian-atsu-still-missing-after-turkey-earthquakes-i-still-pray-and-believe-that-hes-alive-says-partner-cfd47272}{link}). The results from the Turkey-Syria earthquake scenario indicate that SmartBook's underlying framework is adaptable and effective across diverse situation report contexts. The ability to transition from a military conflict to a humanitarian crisis, while maintaining accuracy in capturing key events, highlights robustness and versatility. It also emphasizes the importance of the source articles and suggests that with the right data input, SmartBook has the potential to be an invaluable tool for analysts across a multitude of scenarios.


\subsection{Limitations}
\label{sec:limitations}
While SmartBook represents a significant advancement in the automated generation of situation reports, it is essential to acknowledge certain limitations that stem from both the technical aspects of the system and the scope of its application. These include unverified news source credibility, potential inaccuracies in reflecting source material despite citations, and user studies focused mainly on military intelligence analysts, which may not represent the needs of a wider analytical audience. Recognizing these limitations is crucial for guiding future improvements and ensuring the framework's applicability across across diverse intelligence and analysis sectors.

\begin{itemize}
    %\item \textcolor{blue}{In addressing the capabilities of SmartBook, it is important to clarify that its current functionalities are directed towards the goal of providing a comprehensive initial draft of situation reports. SmartBook currently does not support multi-turn editing or the iterative refinement of reports. In its role as an initial draft provider to streamline the early stages of situation report generation, SmartBook is intended to be a valuable aid in the intelligence analysis workflow, though not as a standalone solution for the entire report development cycle.}
    \item The generation of situation reports within SmartBook leverages news articles aggregated from Google News. Nonetheless, the process does not involve a rigorous assessment of the news sources' credibility, nor does it incorporate an explicit verification of the factual accuracy of the claims used in the summary generation model. Considering the extensive exploration of these aspects within computational social science~\cite{lee2023pandemic} and natural language processing~\cite{gong2023fake}, we deemed them beyond the scope of SmartBook's framework in this work.
    \item SmartBook enhances the reliability and credibility of the generated situation reports by incorporating citations. However, it does not rigorously ensure the accuracy of these reports in reflecting the content of the source documents.
     In contrast, LLMs have been shown~\cite{mallen2022not, baek2023knowledge} to tend to produce content that may not directly correlate to the source materials. While this is still an active area of research, recent studies~\cite{tian2023fine} have shown that LLMs can be effectively optimized to enhance factual accuracy and attribution capabilities~\cite{gao2023enabling}. Consequently, we propose that future advancements could involve substituting the current summary-generating language model with one that is specifically refined for improved attribution~\cite{gao2023enabling}.%{While SmartBook helps promote trust and credibility of the generated situation reports through the addition of citations, it does not explicitly verify the faithfulness of the generated reports to the input documents. On the other hand, LLMs can generate content that is not attributable~\cite{} to input documents. While this is still an active area of research, recent work~\cite{} has demonstrated that large language models can be specifically tuned to improve factuality and the ability to attribute. Therefore, we assert that this can be tackled in future work by replacing the language model generating the summaries with a model fine-tuned for attribution.}
    \item The selection of analysts for the user studies in SmartBook was primarily comprised of individuals from military intelligence. This limitation arose due to the challenges encountered in recruiting analysts who could participate without breaching confidentiality agreements. Consequently, our recruitment efforts were confined to analysts within our existing networks. It is important to recognize that our studies may not fully represent the needs and perspectives of analysts in other fields, such as political, corporate, or criminal analysis. Acknowledging this gap, future work can aim to extend and adapt SmartBook's capabilities for generating situation reports, tailoring them to meet the distinct requirements of each of these domains.%{The analysts recruited for the user studies in SmartBook are predominantly military analysts. This stems from the difficulty in recruiting analysts given confidentiality agreements, hence we were restricted to recruiting those based on prior connections. However, we acknowledge that our studies do not account for other domains such as political analysis, corporate analysis, or criminal analysis. We leave this for future work, to expand and customize SmartBook's formulation of situation report generation to the specific requirements for each of these domains.}
\end{itemize}

\subsection{Future Extensions}
\label{sec:extensions}
This section outlines key extensions planned for SmartBook, each aiming to enhance its functionality and utility. SmartBook's future enhancements include integrating multimodal and multilingual information to enrich intelligence analysis (in \S{\ref{sec:multimodal_info}}) and employing a balanced approach to news sources to mitigate biases (in \S{\ref{sec:source_bias}}). Additionally, the system will evolve with a co-authoring feature for dynamic report refinement (in \S{\ref{sec:iterative_refinement}}) and introduce a `verification score' to assess the reliability of claims, ensuring precise and dependable decision-making support (in \S{\ref{sec:future_verification}}). Overall, these extensions are designed to elevate SmartBook's utility as a comprehensive, bias-balanced, and reliable tool for situation report generation in intelligence analysis.


\subsubsection{Incorporating Multimodal, Multilingual Information:}
\label{sec:multimodal_info} %In intelligence analysis, situation reports often integrate diverse data types, including textual accounts, news articles, photographs, videos, and audio recordings, each offering unique insights. Figure \ref{fig:multimodal_example} illustrates how images can corroborate or challenge textual evidence in SmartBook, highlighting the value of multimodal data. We aim to enhance SmartBook's comprehensiveness by integrating various data modalities, correlating textual claims with pertinent multimedia elements using advanced multimedia knowledge extraction systems such as GAIA~\cite{li2020gaia}. This integration envisions an ecosystem where AI-assisted report generation merges multiple data types, providing analysts a comprehensive, nuanced, and corroborated narrative.\\
 Intelligence analysis increasingly requires the integration of diverse data types, including text, images, videos, and audio recordings, to provide a comprehensive understanding of global events. We plan to enrich SmartBook's situation reports by correlating textual claims with relevant multimedia elements, leveraging systems like GAIA~\cite{li2020gaia} for advanced multimedia knowledge extraction. Figure \ref{fig:multimodal_example} exemplifies how images can support or contradict the text, underscoring the value of incorporating various data modalities. In addition to multimodal data, the globalization of events necessitates a multilingual approach to intelligence analysis, as reliance on English alone can overlook critical local nuances. SmartBook aims to overcome this limitation by incorporating multiple languages, enabling a more profound understanding of local dynamics that influence global situations, capturing cultural subtleties, regional politics, and localized social phenomena in native languages. Moreover, integrating multilingual sources democratizes intelligence analysis, moving beyond a predominantly English-speaking worldview. Ultimately, SmartBook seeks to redefine intelligence reporting standards, promoting a more inclusive and globally aware approach.
\begin{figure}[!htb]
    \centering
    \includegraphics[width=1.0\linewidth]{figures/multimodal_example.png}
    \caption{Figure showing an example of how multimodal information (in the form of images) supports and provides additional context to the claims presented in SmartBook. In this example, the presence of anti-aircraft weapons (as seen in the image) in Ukraine provides background for the discussion in NATO on whether to impose a no-fly zone.}
    \label{fig:multimodal_example}
\end{figure}

    \subsubsection{Controlling the Bias of News Sources:} \label{sec:source_bias} Different news sources inevitably bring with them biases that stem from factors such as editorial policies, audience demographics, and geographical influences. When a single news source or perspective dominates the narrative, key details or alternative viewpoints can be easily missed. In our vision for the next iteration of SmartBook, we aim to address these inherent biases. To overcome such challenges, it is essential to source information from a diverse array of news outlets, encompassing left-leaning, centrist, and right-leaning perspectives. %This strategy enables SmartBook to facilitate the recognition of various hypotheses and interpretations of events. 
    The goal is to enhance SmartBook's utility for intelligence analysts by reducing informational blind spots and broadening the range of considered scenarios.
%Different news sources present information from varying angles, with different levels of detail and interpretation. Thereby, we plan to control for the bias of news sources used in the creation of SmartBook to help cross-check information and avoid the potential pitfalls of relying on a single news source or a single interpretation of events. Further, alternate perspectives/hypotheses can be identified by independently considering news sources on the left, in the center, and on the right. \\
    %\item \textbf{Controlling the bias of sources}: To control bias in SmartBook, we will cross-check information from various news sources with differing angles, levels of detail, and interpretations, and also consider independent sources from the left, center, and right to identify alternate perspectives and hypotheses. \\

    \subsubsection{Co-Authoring with Iterative Refinement:} \label{sec:iterative_refinement} Authoring situation reports is an iterative process where analysts continuously refine and update the reports. In the next iteration, we aim to incorporate advanced co-authoring capabilities that will enable intelligence analysts to engage with SmartBook in a dynamic, multi-turn editing process. By integrating machine learning algorithms for continual learning~\cite{lopez2017gradient, zenke2017continual} and personalization~\cite{wu2019npa, monzer2020user}, SmartBook can learn from each interaction to progressively adapt to the analysts' preferences and decision-making styles. Further, recent techniques like Reinforcement Learning with Human Feedback (RLHF)~\cite{stiennon2020learning, bai2022training} can enable the system to assimilate human feedback, thereby refining its understanding of analysts' editing and reasoning patterns.
%can enable the system to analyze changes and feedback provided by human analysts, allowing it to understand and replicate the nuances of their editing and reasoning processes. 
This approach aims to improve SmartBook's alignment with human decisions over time and help generate reports that are tailored to the specific needs of intelligence analysis workflows. %This learning mechanism will not only improve SmartBook's alignment with human decisions over time, but also ensure that the generated reports are increasingly are relevant, and tailored to the specific needs of intelligence analysis workflows. 
     As a result, SmartBook will evolve from a preliminary drafting tool to a comprehensive AI co-author for generating situation reports.

    %\subsubsection{Adding More Languages} In an era of globalization, events that transpire in one part of the world often ripple out and influence other regions. While English remains a dominant language for international news, relying solely on it can lead to a narrowed viewpoint, potentially missing local nuances and sentiments. By integrating more languages, SmartBook seeks to transcend these linguistic barriers, enhancing the depth and breadth of the intelligence analysis. Furthermore, the integration of multilingual sources will be a critical step towards facilitating a deeper comprehension of intricate local dynamics that influence global situations. Cultural nuances, regional politics, and localized social phenomena are often best captured in native languages, offering unfiltered and nuanced insights that might be lost or distorted in translation. Additionally, this expansion promises to foster a more inclusive approach where information is not monopolized by predominantly English-speaking perspectives, thereby democratizing the intelligence analysis process. Through this endeavor, SmartBook aspires to set a new benchmark in intelligence reporting, championing a more inclusive, empathetic, and globally conscious approach.
    
    %The world is diverse, and news articles from different languages allow capturing different perspectives, insights, and information relating to global events, that might not be available in a single language. For this reason, we aim to add news articles from more languages into SmartBook to present a more comprehensive and accurate picture. Further, the ability to ingest information from different languages also enables analysts to understand local customs, cultures, and nuances that may influence the interpretation of events.\\   
    \subsubsection{Verifying the extracted claims:}\label{sec:future_verification} The integrity of data and claims in situation reports is crucial for strategic decision-making. A key challenge in automating these reports is guaranteeing the accuracy and reliability of extracted data from extensive sources. To address this, we plan to introduce a `verification score' for each claim. This score assesses the reliability of information, considering factors like source credibility, corroborative data, and historical accuracy of similar claims. It provides a confidence metric for intelligence analysts, aiding in swift evaluation of data reliability. High scores enhance confidence and facilitate quick integration into reports, while low scores signal the need for thorough review and potential further verification. This blend of automation and human oversight is essential to ensure the effectiveness of SmartBook's situation reports in strategic planning.
    %Given that situation reports contribute towards strategic planning, it is important to present accurate information to prevent misguided decisions and actions. Hence, we intend to provide a verification score, for each claim presented within SmartBook, which can be interpreted as a measure of trustworthiness. Claims with lower scores signal the need for additional oversight by the intelligence analysts before incorporating into them the final report.
    %\item \textbf{Verifying the extracted claims}: Since situation reports inform strategic planning, we will provide a trustworthiness verification score for each SmartBook claim to ensure accuracy and prevent misguided actions, with lower scores indicating the need for further analyst oversight before final report inclusion.
%\end{itemize}

%\section{Conclusion}\label{sec13}

In this work, we dive into the novel challenges and nuances of automated situation report generation, emphasizing the significant role of these reports in contributing to decision-makers' timely, in-depth understanding of emergent crises. Our approach, informed by a formative study and collaborative design, builds design strategies to address real-world intelligence analysts' needs. We have introduced \textbf{SmartBook}, an innovative framework crafted to generate situation reports with comprehensive, current, and factually-grounded information extracted from diverse sources. SmartBook goes beyond mere information aggregation; it presents chronologically-ordered sequences of topic-summarized events extracted from news sources and placed into a UI layout structure that aligns to the workflow of intelligence analysis. By involving analysts at each stage of SmartBook's conceptualization from design through evaluation, we pave the way for a more cohesive and productive analytical workflow. Our user study, with ten intelligence analysts and two decision-makers, validated SmartBook's effectiveness in providing grounded and easy-to-digest information with its intuitive interface. Both intelligence analysts and decision-makers found SmartBook to be usable, feasible to integrate into operational environments, and trustworthy.



%In this paper, we present the novel task of automated situation report generation, which is pertinent for timely and scalable complex event understanding of emergent crises. This problem formulation differs from the traditional multi-document news summarization domain in that it requires comprehensive, broad coverage of important information, which is to be presented in a structured format that is spread across a timeline.
%In addition, the vast amount of information needs to be organized into chapters and sections with a strategic structure to facilitate analysis and planning. To address these challenges, we propose \textit{SmartBook}, a generalizable framework for clustering news topics and summarizing news claims automatically. SmartBook offers automated generation of situation reports as an initial draft consolidating information to assist analysts, not to replace them. 

%For future work, we aim to explore the following avenues to improve the reliability and expand the scope of SmartBook:

%\begin{itemize}
  %  \item \textbf{Controlling the bias of sources:} To control bias in SmartBook, we will cross-check information from various news sources with differing angles, levels of detail, and interpretations, and also consider independent sources from the left, center, and right to identify alternate perspectives and hypotheses. \\
%    \item \textbf{Adding more languages:} By incorporating news articles from various languages, we plan to capture diverse perspectives, insights, and information on global events, while also enabling analysts to understand local customs, cultures, and nuances that influence event interpretation, ultimately presenting a more comprehensive and accurate picture. \\
 %   \item \textbf{Verifying the extracted claims:} Since situation reports inform strategic planning, we will provide a trustworthiness verification score for each SmartBook claim to ensure accuracy and prevent misguided actions, with lower scores indicating the need for further analyst oversight before final report inclusion.
%\end{itemize}

%To summarize, this work presents the novel task of automated situation report generation, which is pertinent for timely and scalable complex event understanding of emergent crises. This problem formulation differs from the traditional multi-document news summarization domain in that it requires comprehensive, broad coverage of important information, which is to be presented in a structured format that is spread across a timeline. In addition, the vast amount of information needs to be organized into chapters and sections with a strategic structure to facilitate analysis and planning. To address these challenges, we propose \textit{SmartBook}, a generalizable framework for clustering news topics and summarizing news claims automatically. It is important to note that while SmartBook offers automated assistance in generating situation reports, it is not meant to replace human analysts, but rather to assist them in the process. 
 

% 
% move Table 1 w ChatGPT here, per Revanth

%
% move Table 2 w question & summary from SmartBook here, per Revanth



%%
%% The acknowledgments section is defined using the "acks" environment
%% (and NOT an unnumbered section). This ensures the proper
%% identification of the section in the article metadata, and the
%% consistent spelling of the heading.
%\begin{acks}
%To Robert, for the bagels and explaining CMYK and color spaces.
%\end{acks}

%%
%% The next two lines define the bibliography style to be used, and
%% the bibliography file.
\bibliographystyle{ACM-Reference-Format}
\bibliography{sample-base}

%%
%% If your work has an appendix, this is the place to put it.
%\appendix

%\section{Appendix for Proofs}

\paragraph{Proof of Theorem \ref{thm:main}.}

\begin{proof}
\label{proof:main}
Our proof has two steps. In Step 1, we will show that SimCLR is equivalent to minimizing the cross entropy loss defined in Eqn.~(\ref{eqn:cross-entropy}). 
In Step 2, we will show  that minimizing the cross-entropy loss 
is equivalent to spectral clustering on $\bfpi$. 
Combining the two steps together, we have proved our theorem. 

\textbf{Step 1: } SimCLR is equivalent to minimizing the cross entropy loss.

The cross-entropy loss takes expectation over 
$\bfW_\bfX\sim \mathbb{P}(\cdot ; \bfpi)$, 
which means $\bfW_\bfX$ has exactly one non-zero entry in each row $i$. By Lemma~\ref{lem:multinomial}, we know every row $i$ of $\bfW_\bfX$ is independent of other rows. Moreover, 
$\bfW_{\bfX,i}\sim \mathcal{M}(1, \bfpi_i/\sum_j \bfpi_{i,j})=\mathcal{M}(1, \bfpi_i)$, because $\bfpi_i$ itself is a probability distribution.
Similarly, we know $\bfW_\bfZ$ also has the row-independent property by sampling over $\mathbb{P}(\cdot;\bfK_\bfZ)$.
Therefore, by Lemma~\ref{lem:cross_split}, we know Eqn.~(\ref{eqn:cross-entropy}) is equivalent to:
\[
 -\sum_{i=1}^n \mathbb{E}_{\bfW_{\bfX,i}}[\log \mathbb{P}(\bfW_{\bfZ,i}=\bfW_{\bfX,i};\bfK_\bfZ)],
\]

This expression takes expectation over $\bfW_{\bfX,i}$ for the given row $i$. Notice that 
$\bfW_{\bfX,i}$ has exactly one non-zero entry, which equals $1$ (same for $\bfW_{\bfZ,i}$). 
As a result
we expand the above expression to be:
\begin{equation}
 -\sum_{i=1}^n \sum_{j\neq i} \Pr(\bfW_{\bfX,i,j}=1)\log \Pr(\bfW_{\bfZ,i,j}=1).
\label{eqn:detailed-expansion}    
\end{equation}


By Lemma~\ref{lem:multinomial}, $\Pr(\bfW_{\bfZ,i,j}=1)=\bfK_{\bfZ,i,j}/\|\bfK_{\bfZ,i}\|_1$ for $j\neq i$. Recall that $\bfK_\bfZ=(k(\bfZ_i-\bfZ_j))_{(i,j)\in[n]^2}$, which means 
$\bfK_{\bfZ,i,j}/\|\bfK_{\bfZ,i}\|_1=\frac{\exp(-\|\bfZ_i-\bfZ_j\|^2/{2\tau})}{\sum_{k\neq i}
\exp(-\|\bfZ_i-\bfZ_k\|^2/{2\tau})
}$ for $j\neq i$, when $k$ is the Gaussian kernel with variance $\tau$. 

Notice that $\bfZ_i=f(\bfX_i)$, so we know
\begin{equation}
-\log \Pr(\bfW_{\bfZ,i,j}=1)=
-\log \frac{\exp(-\|f(\bfX_i)-f(\bfX_j)\|^2/{2\tau})}{\sum_{k\neq i}
\exp(-\|f(\bfX_i)-f(\bfX_k)\|^2/{2\tau}),
}
\label{eqn:infonce-equivalence}    
\end{equation}


The right hand side is exactly the InfoNCE loss defined in Eqn.~(\ref{eqn:infonce}).
Inserting Eqn.~(\ref{eqn:infonce-equivalence}) into Eqn.~(\ref{eqn:detailed-expansion}), we get the SimCLR algorithm, which first samples augmentation pairs $(i,j)$ with $\Pr(\bfW_{\bfX,i,j}=1)$ for each row $i$, and then optimize the InfoNCE loss. 

\textbf{Step 2: } minimizing the cross entropy loss 
is equivalent to spectral clustering on $\bfpi$.


By Lemma~\ref{lem:convert_to_spectral}, we may further convert the loss to 
\begin{equation}
\label{eqn:main-theorem-repul-attr}
\min_{\bfZ}
-\sum_{(i,j)\in [n]^2} \mathbf{P}_{i,j}
\log k (\bfZ_i-\bfZ_j)+\log \mathbf{R}(\bfZ).
\end{equation}
Since $k$ is the Gaussian kernel, this reduces to \[
\min_\bfZ \mathrm{tr}(\bfZ^\top \mathbf{L}(\bfpi) \bfZ)
+\log \mathbf{R}(\bfZ),
\]

where we use the fact that $\mathbb{E}_{\bfW_\bfX\sim \mathbb{P}(\cdot; \bfpi)}[\mathbf{L}(\bfW_\bfX)]
=\mathbf{L}(\bfpi)
$, because the Laplacian operator is linear and $
\mathbb{E}_{\bfW_\bfX\sim \mathbb{P}(\cdot; \bfpi)}(\bfW_\bfX)=\bfpi
$.
\end{proof}

\paragraph{Proof of Theorem \ref{thm:clip}.}
\begin{proof}
Since $\bfW_\bfX\sim \mathbb{P}(\cdot;\bfpi_{\mathbf{A}, \mathbf{B}})$, we know 
$\bfW_\bfX$ has exactly one non-zero entry in each row, denoting the pair that got sampled. 
A notable difference compared to the previous proof is we now have $n_\mathcal{A}+n_\mathcal{B}$ objects in our graph. CLIP deals with this by taking a mini-batch of size $2N$, 
such that $n_\mathcal{A}=n_\mathcal{B}=N$, and adding the $2N$ InfoNCE losses together. We label the objects in $\mathcal{A}$ as $[n_\mathcal{A}]$, and the objects in $\mathcal{B}$ as $\{n_\mathcal{A}+1, \cdots, n_\mathcal{A}+n_\mathcal{B}\}$. 

Notice that $\bfpi_{\mathbf{A}, \mathbf{B}}$ is a bipartite graph, so the edges of objects in $\mathcal{A}$ will only connect to object in $\mathcal{B}$ and vice versa. We can define the similarity matrix in $\cZ$ as $\bfK_\bfZ$, 
where $\bfK_\bfZ(i, j+n_\mathcal{A})=\bfK_\bfZ(j+n_\mathcal{A},i)= k(\bfZ_i-\bfZ_j)$ for $i\in [n_\mathcal{A}], j\in [n_\mathcal{B}]$, and otherwise we set $\bfK_\bfZ(i,j)=0$. 
The rest is same as the previous proof. 
\end{proof}

\paragraph{Proof of Theorem \ref{thm:exponential}.}

\begin{proof}
\label{proof:exponential}
Since the objective function consists of a linear term combined with an entropy regularization, which is a strongly concave function, the maximization problem is a convex optimization problem. Owing to the implicit constraints provided by the entropy function, the problem is equivalent to having only the equality constraint. We then introduce the Lagrangian multiplier $\lambda$ and obtain the following relaxed problem:

$$
\widetilde{E}(\boldsymbol{\alpha})=\psi_{1}-\sum_{i=1}^n \alpha_{i} \psi_{i}+\tau \sum_{i=1}^n \alpha_{i}\log \alpha_{i}+\lambda\left(\boldsymbol{\alpha}^{\top} \mathbf{1}_n-1\right).
$$

As the relaxed problem is unconstrained, taking the derivative with respect to $\alpha_{i}$ yields

$$
\frac{\partial \widetilde{E}(\boldsymbol{\alpha})}{\partial \alpha_{i}}=-\psi_{i}+\tau\left(\log \alpha_{i}+\alpha_{i} \frac{1}{\alpha_{i}}\right)+\lambda=0.
$$

Solving the above equation implies that $\alpha_{i}$ takes the form
$
\alpha_{i}=\exp \left(\frac{1}{\tau} \psi_{i}\right) \exp \left(\frac{-\lambda}{\tau}-1\right).
$ Since $\alpha_{i}$ lies on the probability simplex, the optimal $\alpha_{i}$ is explicitly given by
$
\alpha^{*}_{i}=\frac{\exp \left(\frac{1}{\tau} \psi_{i}\right)}{\sum_{i^{\prime}=1}^n \exp \left(\frac{1}{\tau} \psi_{i^{\prime}}\right)} .
$ Substituting the optimal point into the objective function, we obtain
$$
\begin{aligned}
E\left(\boldsymbol{\alpha}^*\right)  &=\psi_1-\sum_{i=1}^n \frac{\exp \left(\frac{1}{\tau} \psi_{i}\right)}{\sum_{i^{\prime}=1}^n \exp \left(\frac{1}{\tau} \psi_{i^{\prime}}\right)} \psi_{i}+\tau \sum_{i=1}^n \frac{\exp \left(\frac{1}{\tau} \psi_{i}\right)}{\sum_{i^{\prime}=1}^n \exp \left(\frac{1}{\tau} \psi_{i^{\prime}}\right)}\log \frac{\exp \left(\frac{1}{\tau} \psi_{i}\right)}{\sum_{i^{\prime}=1}^n \exp \left(\frac{1}{\tau} \psi_{i^{\prime}}\right)} \\
& =\psi_1 - \tau \log \left(\sum_{i=1}^n \exp \left(\frac{1}{\tau} \psi_{i}\right)\right).
\end{aligned}
$$
Thus, the Lagrangian dual function is given by
\begin{equation*}
-E\left(\boldsymbol{\alpha}^*\right)= -\tau \log \frac{\exp \left(\frac{1}{\tau} \psi_{1}\right)}{\sum_{i=1}^n \exp \left(\frac{1}{\tau} \psi_{i}\right)}.\qedhere
\end{equation*}
\end{proof}



\section{More on Experiments} \label{section: experiment_details}

\paragraph{CIFAR-10 and CIFAR-100} CIFAR-10 ~\citep{krizhevsky2009learning} and CIFAR-100 ~\citep{krizhevsky2009learning} are well-known classic image classification datasets. Both CIFAR-10 and CIFAR-100 contain a total of 60k $32 \times 32$ labeled images of different classes, with 50k for training and 10k for testing. CIFAR-10 is similar to CIFAR-100, except there are 10 different classes in CIFAR-10 and 100 classes in CIFAR-100.

\paragraph{TinyImageNet} TinyImageNet ~\citep{le2015tiny} is a subset of ImageNet ~\citep{deng2009imagenet}. There are 200 different object classes in TinyImageNet, with 500 training images, 50 validation images, and 50 test images for each class. All the images in TinyImageNet are colored and labeled with a size of $64 \times 64$.

\textbf{Pseudo-code.} Algorithm \ref{alg:Training Procedure} presents the pseudo-code for our empirical training procedure.

\begin{algorithm}[!htbp]
\caption{Training Procedure}
\label{alg:Training Procedure}
\begin{algorithmic}[1]
\REQUIRE trainable encoder network $f$, batch size $N$, augmentation strategy \textit{aug}, loss function $L$ with hyperparameters \textit{args}
\FOR {sampled minibatch ${x_i}_{i=1}^N$}
\FORALL{$i \in { 1, ..., N }$}
\STATE draw two augmentations $t_i = \textit{aug}\left(x_i\right) $, $t_i' = \textit{aug}\left(x_i\right) $
\STATE $z_i = f\left(t_i\right)$, $z_i' = f\left(t_i'\right)$
\ENDFOR
\STATE compute loss $\mathcal{L} = L(N, z, z', \textit{args})$
\STATE update encoder network $f$ to minimize $\mathcal{L}$
\ENDFOR
\STATE \textbf{Return} encoder network $f$
\end{algorithmic}
\end{algorithm}

We also provide the pseudo-code for our core loss function used in the training procedure in Algorithm \ref{alg:Core loss}. The pseudo-code is almost identical to SimCLR's loss function, with the exception of an extra parameter $\gamma$.

\begin{algorithm}[!htbp]
\caption{Core loss function $\mathcal{C}$}
\label{alg:Core loss}
\begin{algorithmic}[1]
\REQUIRE batch size $N$, two encoded minibatches $z_1, z_2$, $\gamma$, temperature $\tau$
\STATE $z = \textit{concat}\left(z_1, z_2\right)$
\FOR {$i \in {1, ..., 2N }, j \in {1, ..., 2N}$ }
\STATE $s_{i,j} = \Vert z_i - z_j \Vert_2^{\gamma}$
\ENDFOR
\STATE \textbf{define} $l(i, j)$ \textbf{as} $l(i, j) = - \log \frac{exp\left(s_{i,j}/\tau \right)}{\sum_{k=1}^{2N} \mathbf{1}{[k \ne i]} exp\left(s{i, j} / \tau \right)} $
\STATE \textbf{Return} $\frac{1}{2N} \sum_{k=1}^N\left[l(i, i+N) + l(i+N, i)\right]$
\end{algorithmic}
\end{algorithm}

Utilizing the core loss function $\mathcal{C}$, we can define all kernel loss functions used in our experiments in Table \ref{table: loss definition}. For all $z_i \in z$ with even dimensions $n$, we define $z_{L_i} = z_i\left[0:n/2\right]$ and $z_{R_i} = z_i\left[n/2:n\right]$.

\begin{table}[ht]
\centering
\begin{tabular}{{@{}l|l@{}}}
Kernel  &  Loss function \\ \midrule
Laplacian & $\mathcal{C}\left(N, z, z', \gamma=1, \tau\right)$\\ \midrule
Sum       & $\lambda * \mathcal{C}\left(N, z, z', \gamma=1, \tau_1\right) + (1-\lambda) * \mathcal{C}\left(N, z, z', \gamma=2, \tau_2\right)$  \\ \midrule
Concatenation Sum&$\lambda * \mathcal{C}\left(N, z_L, z'_L, \gamma=1, \tau_1\right) + (1-\lambda) * \mathcal{C}\left(N, z_R, z'_R, \gamma=2, \tau_2\right)$\\ \midrule
$\gamma = 0.5$ & $\mathcal{C}\left(N, z, z', \gamma=0.5, \tau\right)$          \\ 

\end{tabular}

\caption{Definition of kernel loss functions in our experiments}
\label {table: loss definition}
\end{table}

\textbf{Baselines.} We reproduce the SimCLR algorithm using PyTorch Lightning~\citep{PytorchLightning}.

\textbf{Encoder details.}
The encoder $f$ consists of a backbone network and a projection network. We employ ResNet50~\citep{ResNet} as the backbone and a 2-layer MLP (connected by a batch normalization~\citep{ioffe2015batch} layer and a ReLU \cite{nair2010rectified} layer) with hidden dimensions 2048 and output dimensions 128 (or 256 in the concatenation kernel case).

\textbf{Encoder hyperparameter tuning.}
For each encoder training case, we randomly sample 500 hyperparameter groups (sample details are shown in Table \ref{table: Hyperparameter sample}) and train these samples simultaneously using Ray Tune ~\citep{RayTune}, with the ASHA scheduler~\citep{li2018massively}. Ultimately, the hyperparameter group that maximizes the online validation accuracy (integrated in PyTorch Lightning) within 5000 validation steps is chosen for the given encoder training case.

\begin{table}[ht]
\centering

\begin{tabular}{@{}l|l|l@{}}
\midrule
Hyperparameter  & Sample Range & Sample Strategy \\ \midrule
start learning rate & $\left[10^{-2}, 10\right]$ & log uniform \\ \midrule
$\lambda$       & $\left[0, 1\right]$ & uniform \\ \midrule
$\tau$, $\tau_1$, $\tau_2$ & $\left[0, 1\right]$ & log uniform \\ \midrule
\end{tabular}

\caption{Hyperparameters sample strategy}
\label {table: Hyperparameter sample}
\end{table}

\textbf{Encoder training.} 
We train each encoder using the LARS optimizer~\citep{LARSOptimizer}, LambdaLR Scheduler in PyTorch, momentum 0.9, weight decay $10^{-6}$, batch size 256, and the aforementioned hyperparameters for 400 epochs on a single A-100 GPU.

\textbf{Image transformation.} The image transformation strategy, including augmentation, is identical to the default transformation strategy provided by PyTorch Lightning.

\textbf{Linear evaluation.}
The linear head is trained using the SGD optimizer with a cosine learning rate scheduler, batch size 64, and weight decay $10^{-6}$ for 100 epochs. The learning rate starts at $0.3$ and ends at $0$.

\textbf{Moco Experiments.} We also tested our method based on MoCo~\citep{he2019moco}. The results are summarized in Table \ref{tab:results-moco}. Here we choose ResNet18~\citep{ResNet} as the backbone and set a temperature of $0.1$ as default. For our simple sum kernel, we set $\lambda=0.8$. The results show that our method outperforms the original MoCo method.

\begin{table}[thb]
\centering
\caption{MoCo Experiment Results on CIFAR-10 and CIFAR-100.}
\label{tab:results-moco}
\resizebox{\textwidth}{!}{%
\begin{tabular}{@{}c|ccc|ccc@{}}
\toprule
\multirow{3}{*}{Method} & \multicolumn{3}{c|}{CIFAR-10} & \multicolumn{3}{c}{CIFAR-100} \\ \cmidrule(lr){2-4} \cmidrule(lr){5-7} 
                        & 200 epochs & 400 epochs    & 1000 epochs   & 200 epochs & 400 epochs & 1000 epochs         \\ \midrule
MoCo (repro.)         & $76.41 \pm 0.12$    & $80.01 \pm 0.15$          & $84.45 \pm 0.08$    & $\mathbf{47.02 \pm 0.11}$ & $52.50 \pm 0.07$ & $57.62 \pm 0.15$            \\
\midrule
Laplacian Kernel        & ${78.09 \pm 0.10}$    & $\mathbf{83.85 \pm 0.09}$          & $\mathbf{88.34 \pm 0.16}$    & $46.12 \pm 0.22$   & $53.44 \pm 0.17$ & $59.10 \pm 0.14$        \\
Simple Sum Kernel & $\mathbf{78.12 \pm 0.15}$   & $83.23 \pm 0.18$ & $87.50 \pm 0.20$ & $46.65 \pm 0.06$ & $\mathbf{53.62 \pm 0.19}$ & $\mathbf{59.83 \pm 0.12}$\\
\bottomrule
\end{tabular}
}
\end{table}



\section{More Experiments on Synthetic Data}


Consider a scenario with $n$ clusters, each containing $k$ vertices. Let the probability of vertices $u$ and $v$ from the same cluster belonging to $\bfpi$ be $p$. Conversely, for vertices $u$ and $v$ from different clusters, let the probability of belonging to $\pi$ be $q$. We generate the graph $\bfpi$ randomly, based on $p$ and $q$. We experiment with values of $k=100$ and $n=6$ for ease of visualization, embedding all points in a two-dimensional space. Each vertex's initial position originates from a normal distribution. In each iteration, we sample a subgraph of $\bfpi$ uniformly, ensuring each vertex has an out-degree of $1$. We then optimize the corresponding vectors using InfoNCE loss with an SGD optimizer and iterate until convergence. Our experimental setup consists of an SGD learning rate of $1$, an InfoNCE loss temperature of $0.5$, and a batch size of $50$. We evaluate two scenarios with different $p$ and $q$ values: $p=1$, $q=0$, and $p=0.75$, $q=0.2$. The results of these experiments are visualized in Figure \ref{fig:vis-spectral-cluster}. The obtained embeddings exhibit the hallmark pattern of spectral clustering of graph $\bfpi$.

\begin{figure}[!tb]
\centering
\subfigure{
\includegraphics[width=1\textwidth]{Figures/cluster_pi.png}
\label{fig:vis-cluster}
}
\subfigure{
\includegraphics[width=1\textwidth]{Figures/noised_cluster_pi.png}
\label{fig:vis-noised-cluster}
}
\caption{Visualizations of the optimization process using InfoNCE Loss on the vectors corresponding to $\bfpi$. Points of identical color belong to the same cluster within $\bfpi$. To showcase the internal structure of $\bfpi$, we randomly select 10 vertices from each cluster to display the edge distribution of $\bfpi$.}
\label{fig:vis-spectral-cluster}
\end{figure}



\end{document}
\endinput
%%
%% End of file `sample-authordraft.tex'.
