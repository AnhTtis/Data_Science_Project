

\subsection{Discussion}

%\name{} leverages the zero-shot capabilities~\cite{brown2020language} of LLMs for tasks with limited training data. Complex tasks like grounded summary generation with citations and strategic question identification are tackled by LLMs, while smaller models address simpler data-rich tasks like event headline (chapter name) generation~\cite{headline2020}, duplicate question detection~\cite{cer2017semeval}, and claim extraction through question answering~\cite{kwiatkowski2019natural}.

Amid  advancements in AI-assisted writing tools~\cite{WangACL2019,Wang2020,selfcollaboration2023,cardon2023challenges} tailored for diverse end-users, including academics~\cite{gero2022sparks}, screenwriters~\cite{mirowski2023co} and developers~\cite{chen2021evaluating}, \name{} introduces a specialized automated framework for generating situation reports for intelligence analysts. Unlike previous systems such as CoAuthor~\cite{lee2022coauthor}, Creative Help~\cite{roemmele2015creative}, Writing Buddy~\cite{samuel2016design} and WordCraft~\cite{yuan2022wordcraft}, which support collaborative and creative tasks, \name{} focuses on factual accuracy, analytical depth, and efficiency in structured, data-driven tasks. This distinct approach emphasizes less creative interaction and more rigorous information processing, aligning \name{} closely with the needs of intelligence analysis. However, the design also stands to benefit from iterative refinement (discussed in \S{\ref{sec:future_extensions}}), exploring deeper integration into analysts' workflows.

%In light of the ongoing developments in AI-assisted writing~\cite{WangACL2019,Wang2020,selfcollaboration2023,cardon2023challenges}, where tools have been developed for a variety of end-users like academics~\cite{gero2022sparks}, screenwriters~\cite{mirowski2023co} and developers~\cite{chen2021evaluating}, SmartBook introduces a unique automated framework designed specifically for generating situation reports for intelligence analysts. Prior AI-assisted writing systems, such as CoAuthor~\cite{lee2022coauthor}, Creative Help~\cite{roemmele2015creative}, Writing Buddy~\cite{samuel2016design} and WordCraft~\cite{yuan2022wordcraft}, are designed to be collaborative and aid in creative tasks~\cite{klein1973automatic}. SmartBook highlights a different aspect of human-AI collaboration, where AI's role is more about efficient information processing with analytical rigor and less about creative or iterative human engagement. This distinction underscores SmartBook's design with a focus on factuality, analytical depth, and efficiency in more structured, data-driven environments, such as intelligence analysis. Clearly the design also stands to benefit from iterative experimentation and refinement (discussed in \S{\ref{sec:future_extensions}}), to explore deeper integration of SmartBook into analysts' workflows. 

\name{}, with its modular design, automates the generation of situation reports efficiently across various contexts, including geopolitical, environmental, or humanitarian situations. %Unlike human analysts who face considerable challenges in transferring their expertise across domains, SmartBook requires minimal configuration for such shifts. % \revanth{mentioned in next paragraph} \heng{mention we also extended it to other domains such as natural disaster and diplomacy} 
%This transition of domains in traditional settings with human analysts involves extensive background research and the recruitment of domain experts, demanding substantial time and effort. SmartBook, with a streamlined approach, requires minimal configuration for such shifts, thereby offering accurate reports in diverse and rapidly changing global contexts. 
Unlike traditional approaches that require substantial research and domain-specific expertise, \name{} can adapt to new domains with minimal configuration, thereby delivering accurate reports in diverse and rapidly changing global contexts. \name{}'s efficacy was primarily evaluated on news from the Ukraine-Russia military conflict. Here, we briefly explore \name{}'s applicability in a humanitarian scenario--the Turkey-Syria earthquake. %By changing the source articles for major event identification, SmartBook successfully generated relevant situation reports for the specified time window (February 6-13, 2023). 
By modifying the input data to identify key events in the specified time window (February 6-13, 2023), \name{} produced relevant reports detailing the earthquake’s aftermath (\href{https://www.cnn.com/middleeast/live-news/turkey-syria-earthquake-updates-2-7-23-intl/h_a9e1d082dd54c67187fd51322f95eaf2}{link}), international aid (\href{https://www.incirlik.af.mil/News/Article-Display/Article/3292036/urban-search-and-rescue-teams-arrive-at-incirlik-air-base/}{link}), rising casualties (\href{https://www.reuters.com/world/middle-east/death-toll-syria-turkey-quake-rises-more-than-8700-2023-02-08/}{link}), and notable incidents like the disappearance of a Ghanaian soccer star (\href{https://www.marketwatch.com/story/soccer-star-christian-atsu-still-missing-after-turkey-earthquakes-i-still-pray-and-believe-that-hes-alive-says-partner-cfd47272}{link}).
%critical aspects of the earthquake disaster, such as aftershock impacts (\href{https://www.cnn.com/middleeast/live-news/turkey-syria-earthquake-updates-2-7-23-intl/h_a9e1d082dd54c67187fd51322f95eaf2}{link}), international humanitarian assistance (\href{https://www.incirlik.af.mil/News/Article-Display/Article/3292036/urban-search-and-rescue-teams-arrive-at-incirlik-air-base/}{link}), rising death tolls (\href{https://www.reuters.com/world/middle-east/death-toll-syria-turkey-quake-rises-more-than-8700-2023-02-08/}{link}), and specific events like the disappearance of a Ghanaian soccer star (\href{https://www.marketwatch.com/story/soccer-star-christian-atsu-still-missing-after-turkey-earthquakes-i-still-pray-and-believe-that-hes-alive-says-partner-cfd47272}{link}). 
This adaptability from military to humanitarian crises highlights \name{}’s robustness and versatility. It also emphasizes the importance of the source articles and suggests that, with mission-relevant data input, \name{} has the potential to be an invaluable tool for analysts across diverse scenarios.

\subsubsection{Outlook}
\label{sec:outlook}

%\heng{maybe add a paragraph about possibly use event schema induction for event prediction, mention news simulator for disaster forecasting, cite: https://blender.cs.illinois.edu/paper/hierarchicalschema2023.pdf https://blender.cs.illinois.edu/paper/schemamatching2022.pdf and feel free to use words from my talk abstract: History repeats itself, sometimes in a bad way. Preventing natural or man-made disasters requires being aware of these patterns and taking pre-emptive action to address and reduce them, or ideally, eliminate them. Emerging events, such as the COVID pandemic and the Ukraine Crisis, require a time-sensitive comprehensive understanding of the situation to allow for appropriate decision-making and effective action response. Automated generation of situation reports can significantly reduce the time, effort, and cost for domain experts when preparing their official human-curated reports. However, AI research toward this goal has been very limited, and no successful trials have yet been conducted to automate such report generation and “what-if” disaster forecasting. Pre-existing natural language processing and information retrieval techniques are insufficient to identify, locate, and summarize important information, and lack detailed, structured, and strategic awareness. In this talk I will present SmartBook, a novel framework that cannot be solved by large language models alone, to consume large volumes of multimodal multilingual news data and produce a structured situation report with multiple hypotheses (claims) summarized and grounded with rich links to factual evidence through multimodal knowledge extraction, claim detection, fact checking, misinformation detection and factual error correction. Furthermore, SmartBook can also serve as a novel news event simulator, or an intelligent prophetess.  Given “What-if” conditions and dimensions elicited from a domain expert user concerning a disaster scenario, SmartBook will induce schemas from historical events, and automatically generate a complex event graph along with a timeline of news articles  By effectively simulating disaster scenarios in both event graph and natural language format, we expect SmartBook will greatly assist humanitarian workers and policymakers to exercise reality checks, and thus better prevent and respond to future disasters. }

We have developed \textsc{SmartBook}, an innovative framework that generates situation reports with comprehensive, current, and factually-grounded information extracted from diverse sources. \name{} goes beyond mere information aggregation; it presents chronologically ordered sequences of topically summarized events extracted from news sources and placed into a UI layout structure that aligns with the workflow of intelligence analysis. Throughout the development of \textsc{SmartBook}, from its conceptual design to its evaluation, we engaged analysts to ensure the tool meets their practical needs and enhances their workflow. Our formative study and collaborative design efforts have been guided by the needs of intelligence analysts, particularly in addressing the challenge of data overload. The findings from these studies indicate that analysts are cautiously optimistic about integrating AI assistance into their processes, expressing a clear desire to shape a future in which AI tools can both adapt to the evolving data landscape and personalize to their individual analytical techniques.

%By involving analysts at each stage of SmartBook's conceptualization from design through evaluation, we pave the way for a more cohesive and productive analytical workflow. Our approach, informed by a formative study and collaborative design, builds design strategies to address real-world intelligence analysts' needs. The central challenge faced by analysts is the massive data overload that they must filter in addressing information requirements to produce situation reports.  The key findings of the formative study show that analysts, though appropriately cautious, are open to incorporating AI assistance into their workflow.  The desiderata in the design strategies that they articulated during the collaborative design also makes clear that they are eager to participate in crafting the new vision that will include AI assistance that not only adjusts to the data content of shifting scenarios but also to their individual analytic methods.
 %Overall, SmartBook presents an exciting step forward in AI-assisted situation report generation and signifies a momentous stride in the domain of intelligence analysis.

\name{} establishes a foundation for situation understanding in a variety of scenarios. However, history repeats itself, sometimes in a bad way. Historical patterns, particularly those leading to natural or man-made disasters, can be a useful signal to trigger proactive measures for crisis mitigation.  %Preventing natural or man-made disasters requires being aware of these patterns and taking pre-emptive action to address and reduce them, or ideally, eliminate them. 
%In this regard, we expect SmartBook to act as a foundation for disaster forecasting. Leveraging extensive work in event prediction~\cite{li2021future, wang2022schema, li2023open}, SmartBook's situation reports can guide the development of a news simulator than can forecast how events play out in crisis scenarios. 
Building on prior research in event prediction~\cite{li2021future, wang2022schema, li2023open}, \name{} can guide the development of a news simulator that forecasts event outcomes in crisis situations.
This tool would be invaluable for humanitarian workers and policymakers to exercise reality checks, enhancing their capacity to prevent and respond to disasters effectively.


\subsubsection{Limitations and Future Extensions}
\label{sec:future_extensions}

While \name{} represents a significant advancement in the automated generation of situation reports, it is essential to acknowledge certain limitations that stem from both the technical aspects of the system and the scope of its application. These include (i) unverified news source credibility, (ii) potential inaccuracies in reflecting source material despite citations, and (iii) user studies focused mainly on military intelligence analysts, which may not represent the needs of a wider analytical audience. Recognizing these limitations is crucial for guiding future improvements and ensuring the framework's applicability across diverse intelligence analysis sectors. We elaborate on the limitations in more detail in the supplementary material. Here, we provide key future extensions designed to elevate \name{}'s utility as a comprehensive, unbiased, and reliable tool for situation report generation in intelligence analysis:
%We further detail , each enhancing SmartBook's functionality and utility. SmartBook's future enhancements include integrating multimodal and multilingual information to enrich intelligence analysis and employing a balanced approach to news sources to mitigate biases. Additionally, the system will evolve with a co-authoring feature for dynamic report refinement and introduce a `verification score' to assess the reliability of claims, ensuring precise and dependable decision-making support. Overall, these extensions are designed to elevate SmartBook's utility as a comprehensive, bias-balanced, and reliable tool for situation report generation in intelligence analysis. %\revanth{Added it in future work for verification}\heng{A frequent question I often got from audience when I presented SmartBook is how we choose sources. You can say that due to resource constraints we choose English as example, but our backend information extraction and claim detection systems have been applied to process many foreign languages such as Russian, Ukrainian, Chinese, Spanish; also in the future, ideally we should have source, claim and evidence to mutually update each other's credibility so our system can automatically rank the reliability/credibility of sources}

\begin{itemize}
    \item \textit{Incorporating Multimodal, Multilingual Information}: %Intelligence analysis increasingly requires the integration of diverse data types, including text, images, videos, and audio recordings, to provide a comprehensive understanding of global events. We plan to enrich SmartBook's situation reports by correlating textual claims with relevant multimedia elements, leveraging systems like GAIA~\cite{li2020gaia} for advanced multimedia knowledge extraction. Figure \ref{fig:multimodal_example} in supplementary material exemplifies how images can support or contradict the text, underscoring the value of incorporating various data modalities. In addition to multimodal data, the globalization of events necessitates a multilingual approach to intelligence analysis, as reliance on English alone can overlook critical local nuances. Leveraging cross-lingual processing techniques~\cite{du2022resin} enables us to overcome this limitation by incorporating multiple languages, enabling a more profound understanding of local dynamics that influence global situations, capturing cultural subtleties, regional politics, and localized social phenomena in native languages. Moreover, integrating multilingual sources democratizes intelligence analysis, moving beyond a predominantly English-speaking worldview. Ultimately, SmartBook seeks to redefine intelligence reporting standards, promoting a more inclusive and globally aware approach. \\
    Intelligence analysis increasingly relies on integrating diverse data types such as text, images, videos, and audio to understand global events comprehensively. We aim to enhance \name{}'s situation reports by correlating textual claims with corresponding multimedia, utilizing advanced systems like GAIA~\cite{li2020gaia} for multimedia knowledge extraction. %Supplementary Figure 1 illustrates the utility of multimodal data in supporting or refuting textual information, highlighting the importance of incorporating various data modalities. 
    Furthermore, global events' reach necessitates multilingual intelligence analysis to capture local nuances missed by solely using English. Employing cross-lingual techniques~\cite{du2022resin} allows for the inclusion of multiple languages, thereby enhancing understanding of local dynamics and cultural subtleties that influence global situations. Thereby, integrating multilingual sources democratizes intelligence analysis, shifting away from a predominantly English-centric perspective. Ultimately, \name{} aims to set new standards in intelligence reporting, fostering a more inclusive and globally informed approach. \\
    \item \textit{Controlling the Bias of News Sources}: %Different news sources inevitably bring with them biases that stem from factors such as editorial policies, audience demographics, and geographical influences. When a single news source or perspective dominates the narrative, key details or alternative viewpoints can be easily missed. In our vision for the next iteration of SmartBook, we aim to address these inherent biases. To overcome such challenges, it is essential to source information from a diverse array of news outlets, encompassing broad political stances. The goal is to enhance SmartBook's utility for intelligence analysts by reducing informational blind spots and broadening the range of considered scenarios.
News sources carry inherent biases influenced by editorial policies, audience demographics, and geographical locations. Dominance of a single perspective may obscure crucial details or alternative viewpoints. In developing the next version of \name{}, we seek to mitigate these biases by incorporating a diverse spectrum of news outlets representing various political stances. This approach aims to diminish informational blind spots and expand the scope of scenarios available to intelligence analysts, thereby enhancing \name{}'s utility.\\
    \item \textit{Co-Authoring with Iterative Refinement}: %Authoring situation reports is an iterative process where analysts continuously refine and update the reports. In the next iteration, we aim to incorporate advanced co-authoring capabilities that will enable intelligence analysts to engage with SmartBook in a dynamic, multi-turn editing process. As a first step, SmartBook can use feedback from human analysts to self-improve the generated reports. By integrating machine learning algorithms for continual learning~\cite{lopez2017gradient, zenke2017continual} and personalization~\cite{wu2019npa, monzer2020user}, SmartBook can learn from each interaction to progressively adapt to the analysts' preferences and decision-making styles. Further, recent techniques like Reinforcement Learning with Human Feedback (RLHF)~\cite{stiennon2020learning, bai2022training} can enable the system to assimilate human feedback, thereby refining its understanding of analysts' editing and reasoning patterns. This approach aims to improve SmartBook's alignment with human decisions over time and help generate reports that are tailored to the specific needs of intelligence analysis workflows. As a result, SmartBook will evolve from a preliminary drafting tool to a comprehensive AI co-author for generating situation reports. 
    Authoring situation reports is an iterative process, where analysts continuously refine and update the reports. In the next iteration, we aim to provide a dynamic multi-turn editing process with \name{}, by leveraging analyst feedback for self-improvement. Techniques like Reinforcement Learning with Human Feedback~\cite{stiennon2020learning, bai2022training} and other personalization algorithms~\cite{wu2019npa, monzer2020user} will further enhance the system's capability to integrate human feedback, and progressively adapt to the preferences and decision-making styles of analysts. This strategy will align \name{} more closely with human analysts, leading to tailored situation reports that meet the specific requirements of intelligence analysis workflows. Ultimately, \name{} will evolve from a preliminary drafting tool to a comprehensive AI co-author for generating situation reports.\\
    \item \textit{Improving Reliability of the Generated Reports}: The integrity of data and claims in situation reports is crucial for strategic decision-making. A key challenge in automating these reports is guaranteeing the accuracy and reliability of data from diverse sources. To address this, we propose introducing a `verification score' for each claim, which evaluates the reliability of information based on source credibility, corroborative data, and historical accuracy of similar claims. % Futher, having the news source, extracted claim and corresponding evidence mutually update each other's credibility enables our system to automatically rank the reliability of sources. 
    This mechanism provides intelligence analysts with a confidence metric to quickly assess data reliability. Higher scores indicate greater confidence and facilitate rapid integration into reports, whereas lower scores necessitate comprehensive review and possibly further verification.
\end{itemize}