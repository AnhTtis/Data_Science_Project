\section{Supplementary Material}

\subsection{Recruitment Details}
\label{sec:analyst_recruitment}

For our studies, we targeted individuals with experience in government and military roles. We distributed a pre-screening survey to ascertain their background in creating situation reports. Participants qualifying for the study were either experienced intelligence analysts or had a minimum of one year of equivalent experience. The final group comprised 10 military personnel from different branches, as shown in Table \ref{tab:analyst_demographic}. Their experience in intelligence report writing varied between 1 and 10 years. Compensation ranged from \$25 to \$35 per hour,
reflecting participants’ levels of experience.

We implemented a targeted outreach strategy to recruit decision-makers for our study. These participants, identified through email addresses sourced from government boards and social media profiles (e.g., LinkedIn, Twitter), were self-identified professionals who make decisions based on prepared reports in their official capacity. Our group comprised of individuals with current or past experience on Canadian government boards. %These decision-makers engaged in the initial qualitative study but did not partake in the subsequent post-study questionnaire, due to time constraints. 
Unlike intelligence analysts, these participants did not receive compensation due to their government affiliation.

\begin{table}[!htb]
\centering
\label{tab:participants}
%\resizebox{\textwidth}{!}{
\begin{tabular}{llllll}
\toprule
\textbf{PID} &
  \textbf{Age} &
  \textbf{Education} &
  \textbf{Intelligence Exp.} &
  \textbf{AI Knowledge} &
  \textbf{LLM Usage} \\\midrule
IA1 & 25 - 34 & Bachelor's & 5 - 10 years & 2 - Intermediate & Rarely \\
IA2 & 25 - 34 & High School Diploma & 2 - 5 years & 3 - Proficient & Rarely  \\ 
IA3 & 18 - 24 & High School Diploma & 1 - 2 years & 2 - Intermediate & Daily  \\
IA4 & 25 - 34 & Bachelor's & 2 - 5 years & 2 - Intermediate & Rarely  \\
IA5 & 45 - 54 & Bachelor's & 2 - 5 years & 1 - Basic & Rarely  \\
IA6 & 45 - 54 & Master's & 1 - 2 years & 1  - Basic & Rarely  \\
IA7 & 35 - 44 & High School Diploma & 5 - 10 years & 1  - Basic & Weekly  \\
IA8 & 45 - 54 & Master's & 1 - 2 years & 2 - Intermediate & Rarely  \\
IA9 & 35 - 44 & Bachelor's & 2 - 5 years & 2 - Intermediate & Rarely  \\
IA10 & 25 - 34 & Bachelor & 2 - 5 years & 4 - Advanced & Weekly   \\ 
\bottomrule      
\end{tabular}
\vspace{0.3em}
\caption{Demographic information of study participants along with their experience in intelligence report generation.}
\label{tab:analyst_demographic}
\vspace{-1.5em}
\end{table}

\begin{figure}[b]
    \centering
    \includegraphics[width=0.8\linewidth]{figures/SmartBook_question_student_eval.png}
    \caption{Results from the human evaluation of the strategic questions (section headings) in \name{}. Judgments are based on the strategic importance of the question and tactical information that can be gathered using the question.}
    \label{fig:question_eval}
\end{figure}

\begin{table}[!htb]
    \centering
    \renewcommand{\arraystretch}{1.5}
    \small
    \begin{tabular}{p{24.5em}|p{26.5em}}
      \textbf{Question:}  What were the motives behind the attack on Russian missile in Dnipro?
  &  \textbf{Question:} What is the capability of Ukraine's air defense systems in repelling the missile attacks from Russian missiles\\
  \hline
    \textbf{Summary:} The Russian missile attack on the city of Dnipro on Saturday was \textcolor{blue}{part of the Kremlin's efforts to limit Ukraine's ability to heat and power itself in the middle of winter}. The attack was reportedly \textcolor{red}{carried out with a Kh-22 missile}, an older and less accurate weapon than most modern missiles. It is believed that the attack was \textcolor{blue}{part of the Kremlin's attempts to retake a town, Soledar, and the nearby city of Bakhmut, as a way to grind down the best Ukrainian forces and prevent them from launching counterattacks} elsewhere. Nine people were killed, including a 15-year old girl, Kyrylo Tymoshenko, and \textcolor{red}{Russia denied responsibility for the attack}, claiming it was caused by Ukrainian air defences.
     & 
     \textbf{Summary:} Ukraine's air defense systems have demonstrated their capability in repelling the missile attacks from Russian forces in Kyiv. [1] reported that \textcolor{blue}{Ukrainian air defense forces shot down 37 of "about 40" rockets that entered the city's airspace}, with one person injured. [2] revealed that a commander, who asked to be identified only by the call sign "Hera" for security reasons, stated that \textcolor{blue}{a cruise missile was shot down by a machine gun} from the city's territorial defense mobile group. [3] mentioned that U.S. President Joe Biden had \textcolor{blue}{pledged to deliver one Patriot surface-to-air missile battery system} to Ukraine, which is one of the most advanced U.S. air defense systems and is \textcolor{blue}{capable of intercepting threats such as aircraft and ballistic missiles.}\\
     \hline
    \end{tabular}
    \caption{Table showing summaries for two strategic questions corresponding to a \name{} chapter on Russian missile attacks. The tactically useful and relevant information has been highlighted in \textcolor{blue}{blue}. Tactically useful but irrelevant information has been highlighted in \textcolor{red}{red}.}
    \label{tab:tactical_info_examples}
    \vspace{-2em}
\end{table}

\subsection{\name{} Limitations}
\begin{itemize}
    \item The generation of situation reports within \name{} leverages news articles aggregated from Google News. Nonetheless, the process does not involve a rigorous assessment of the news sources' credibility, nor does it incorporate an explicit verification of the factual accuracy of the claims used in the summary generation model. Considering the extensive exploration of these aspects within computational social science~\cite{lee2023pandemic} and natural language processing~\cite{gong2023fake}, we deemed them beyond the scope of \name{}'s framework in this work.
    \item \name{} enhances the reliability and credibility of the generated situation reports by incorporating citations. However, it does not rigorously ensure the accuracy of these reports in reflecting the content of the source documents.
     In contrast, LLMs have been shown~\cite{mallen2022not, baek2023knowledge} to tend to produce content that may not directly correlate to the source materials. While this is still an active area of research, recent studies~\cite{tian2023fine} have shown that LLMs can be effectively optimized to enhance factual accuracy and attribution capabilities~\cite{gao2023enabling}. Consequently, we propose that future advancements could involve substituting the current summary-generating language model with one that is specifically refined for improved attribution~\cite{gao2023enabling}.
     \item The selection of analysts for the user studies in \name{} was primarily comprised of individuals from military intelligence. This limitation arose due to the challenges encountered in recruiting analysts who could participate without breaching confidentiality agreements. Consequently, our recruitment efforts were confined to analysts within our existing networks. It is important to recognize that our studies may not fully represent the needs and perspectives of analysts in other fields, such as political, corporate, or criminal analysis. Acknowledging this gap, future work can aim to extend and adapt \name{}'s capabilities for generating situation reports, tailoring them to meet the distinct requirements of each of these domains.
\end{itemize}

\begin{comment}
\begin{figure}[b]
    \centering
    \includegraphics[width=1.0\linewidth]{figures/multimodal_example.png}
    \caption{Figure showing an example of how multimodal information (in the form of images) supports and provides additional context to the claims presented in \name{}. In this example, the presence of anti-aircraft weapons (as seen in the image) in Ukraine provides background for the discussion in NATO on whether to impose a no-fly zone.}%\heng{fix the citation format in the summary}}
    \label{fig:multimodal_example}
\end{figure}
\end{comment}



\section*{Acknowledgement}
We would like to acknowledge the invaluable contributions of Paul Sullivan, who passed away before the publication of this paper. His enduring commitment to knowledge is greatly missed and deeply appreciated. We are grateful to Lisa Ferro and Brad Goodman from MITRE for their valuable comments and help with expert evaluation. This research is based upon work supported by U.S. DARPA AIDA Program No. FA8750-18-2-0014, DARPA KAIROS Program No. FA8750-19-2-1004, DARPA SemaFor Program No. HR001120C0123, DARPA INCAS Program No. HR001121C0165 and DARPA MIPS Program No. HR00112290105. The views and conclusions contained herein are those of the authors and should not be interpreted as necessarily representing the official policies, either expressed or implied, of DARPA, or the U.S. Government. The U.S. Government is authorized to reproduce and distribute reprints for governmental purposes notwithstanding any copyright annotation therein.

