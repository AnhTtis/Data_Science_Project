\section{Introduction}\label{sec1}

%\heng{We should also talk about criteria for a good situation report - use the bullets in page 6 of these slides: https://blender.cs.illinois.edu/ALTA\_NLP\_SmartBook\_HengJi.pdf}

In today's rapidly changing world, intelligence analysts face the constant challenge of staying informed amidst an overwhelming influx of news, rumors, and evolving narratives. Currently, intelligence analysts prepare \textit{situation reports}, about the state of affairs, potential risks or threats, and perspective, with recommended actions \cite{world2020coronavirus,world2022emergency}. The downstream consumers of these reports (general public, decision-makers, and stakeholders) are not just looking for raw information; they are seeking clarity, context, and actionable insights. However, these reports, being manually crafted, come with limitations: they are time-consuming~\cite{national2019decadal}, might exhibit biases~\cite{biases1, belton2020cognitive}, and may be restricted due to the large volume of information they must process~\cite{doggette2020information, overload1}. %Most importantly, the process of generating and consuming these reports is far from optimized for the user (analysts), impacting its downstream recipients. 
Intelligence analysts spend extensive time sifting through vast and complex data sources~\cite{golson2018resist, national2011intelligence} ({\em e.g.,} social media rumor propagation, news outbreak, background statistics, domain expert opinions, etc.), and readers often grapple with information that's either too dense or not sufficiently comprehensive. Given the sheer volume of information, it is challenging for analysts to do deep analysis and critical thinking, to be able to formulate strategic questions and generate insights.

We hypothesize that automatic situation report generation can bridge the gap in reading and writing intelligence information. Existing summarization approaches~\cite{fabbri-etal-2019-multi, abdi2017query, li2014improving} generate plain-form text and do not yield the details, structures, and high-level strategic information to support decision-making for the downstream consumer. In particular, current Large Language Model (LLM) based approaches such as GPT-4/ChatGPT~\cite{openai2021gpt35} may return no answers or incorrect answers to user questions as they do not update information in real-time and lack fact-checking mechanisms ({e.g.,} Table~\ref{tab:chatgpt}). 

To establish a basis for the automated generation of situation reports, we initially carried out a formative study\footnote{All study protocols were approved by our institution's IRB.} (see \S{\ref{sec:formative_study}}) to grasp the standard practices in situation report creation and intelligence analysts' expectations from AI-driven systems. Our findings reveal that analysts are open to AI assistance, seek clarity in AI operations, and vary in their desire to control the AI system. To further comprehend the composition process of situation reports, we augmented the initial study's findings with collaborative design sessions, aiming to define practical design strategies and recommendations, particularly on how intelligence analysts navigate, research, and write their reports. On a high level, we infer (in \S{\ref{sec:collaborative_results}}) that the system interface must visually align with analysts' sequential thought processes, provide transparency of information flow to build trust, and enable analysts to have some level of control over the AI tool's processes. These conclusions consequently guided the design strategies (in \S{\ref{sec:design_strategies}}) for automating the generation of situation reports.

%\textcolor{blue}{\textbf{TODO: Revanth - Summarize design recommendations brought forward in collaborative study.}}

\begin{table}[t]
\centering
\renewcommand{\arraystretch}{1.5}
\small
\begin{tabular}{p{52em}}
\hline
\textbf{Question}: 
Is Russian positioning forces %inside of Russia 
for possible follow-on actions into Ukraine after the initial invasion? \\
\hline
\textbf{GPT-4}: As an AI language model, I cannot provide real-time information on current events. However, it is essential for intelligence analysts to monitor the activities of various countries, including troop movements and military exercises,to assess potential threats and anticipate future actions. To determine if Russia is positioning forces for possible actions into Ukraine, one must analyze the latest intelligence reports, satellite imagery, and open-source information. Additionally, understanding the political and historical context of the region can provide valuable insight into the intentions of the countries involved.\\
\hline
\textbf{SmartBook}: The arrival of Russian forces in Slavutych came after several days of shelling against the city, which is strategically located close to the Dnieper River, about 95 miles (150 kilometers) north of Kyiv and close to the border with Belarus.
Russia has been using Belarus as a springboard for many of its air operations in Ukraine, according to intelligence collected by NATO surveillance planes flying over the Polish-Ukrainian border and radar seen by CNN.
So far, however, US officials have not seen Belarusian troops ``being readied to move into Ukraine" or ``that they are moving or are in Ukraine", a senior US defense official said Monday, adding that the forces inside Ukraine are Russian.\\
% \textbf{Answer}: I don't have the current information regarding any Russian military \\\cellcolor{lightgray}actions or deployments. It's important to rely on credible new sources for the \\\cellcolor{lightgray}most up-to-date and accurate information on political and military developments. \\
% \textbf{Question}: What is the status of efforts to secure the release of Paul Whelan and \\ Brittney Griner?\\
% \cellcolor{lightgray}\textbf{Answer}: I'm sorry, but as a language model AI, I do not have the most \\\cellcolor{lightgray}up-to-date information on the current status of individual cases like Paul \\\cellcolor{lightgray}Whelan and Brittney  Griner. However. as of my knowledge cutoff in 2021, Paul \\\cellcolor{lightgray}Whelan, a former US Marine, was being held in Russia on espionage charges \\\cellcolor{lightgray}and has been advocating for his release through his legal team and government \\\cellcolor{lightgray}representatives. Similarly, Brittney Griner is a basketball player who was \\\cellcolor{lightgray}playing in Russia and has been involved in legal issues there, but I don't have \\\cellcolor{lightgray}any more specific details on her case. It's best to check reliable news sources \\\cellcolor{lightgray}for the most recent updates on their cases.\\
\hline
\end{tabular}
\caption{Example of situation understanding for the Ukraine-Russia crisis. GPT-4 is incapable of generating situation reports because it has limited knowledge of up-to-date events, while Smartbook is able to generate timely, multi-source, and trustworthy answers.%\heng{maybe change the example to "Why is the Zaporizhzhia nuclear power plant a target for Russia?" because for that question GPT returns factual error "The Zaporizhzhia Nuclear Power Plant is not a specific target for Russia..." that seems to be more serious error than providing generic information}
}
\label{tab:chatgpt}
\vspace{-1em}
\end{table}




% However, real-world situation intellgence reports that policymakers and subject matter experts rely on for decision-making require detailed, real-time information. For example, recent expert-written situation reports \cite{bowen2022russia} on the Russia-Ukraine crisis, published by the United Nations and the U.S. Congressional Research Service, aggregate critical information on social conditions and history as background; onset of the military invasion; development of subsequent military operations; economic, political, and humanitarian ramifications; and present an outlook forecast – in over twenty dense pages. Automating the generation of situation reports \textit{goes beyond} increasing the volume of retrieved, relevant news sources \cite{brody-kantor-2011-automatic} and the length of generated summaries. It \textit{extends to} maintaining a structured organization ({e.g.,} chapters and sections) for ready access and understanding of complex events, in support of strategic planning. As such, the current application of AI techniques to situation analysis and report generation is far from mature.  

\begin{figure}[t]
\includegraphics[width=\textwidth]{figures/SmartBook_example.png}
\caption{Figure showing an example from SmartBoook for the Ukraine-Russia Crisis. SmartBook is organized on a timeline, with 2-week time spans containing chapters and corresponding sections. The section headings are strategic questions and each section content consists of a grounded summary with links to relevant claims, each highlighted for factual evidence and knowledge elements. The entities and events (along with corresponding types), which comprise the associated knowledge elements, are highlighted in red and blue respectively.} 
\label{fig:1}
\end{figure}

Building on the foundation set by the formative study and collaborative design, we present \textbf{SmartBook}, a framework designed to assist human analysts who author situation reports. SmartBook ingests data from multiple sources to generate a comprehensive report with information updated on a regular basis. The framework automatically discovers questions important for situation analysis and gathers salient information for generating the report. For all questions about a major event, the report contains summaries with tactical information coming from relevant claims, presented with local context and links to source news articles. SmartBook structures data in a manner that mirrors current intelligence analyst workflows--breaking down events into timelines, chapters, and question-based sections. Each section contains a grounded query-focused summary with its relevant claims. This intuitive structure facilitates easier assimilation of information for both reading and writing. Designed with a human-centered approach, our goal is to augment the capabilities of human analysts rather than replace them. Figure \ref{fig:1} shows an example from SmartBook for the Ukraine-Russia crisis, with the structured hierarchy of timespans, chapters, and corresponding sections.


In our comprehensive evaluation of SmartBook, we conducted two complementary studies: a user study (in \S{\ref{sec:usability_study}}) for assessing usability and interaction, and a content review (in \S{\ref{sec:content_review_study}}) for examining the quality of text summaries. The user study, involving intelligence analysts and decision-makers, focused on key research questions to explore SmartBook's usability, intuitiveness, and effectiveness in situation report generation. Participants engaged in semi-structured interviews and post-study questionnaires, reflecting on their experience with the system. The content review complemented this by assessing the readability, coherence, and relevance of SmartBook-generated reports, including an editing study where an expert analyst revised the summaries until they met their standards of acceptability. This process aimed to determine the viability of using SmartBook as a tool for creating preliminary drafts in analytical settings. %Results indicated that SmartBook seamlessly integrates into analysts' workflows, mirroring their conventional report generation processes with an intuitive and user-friendly interface. 
The system's generated reports were noted for their accuracy and relevance, enhancing efficiency in report generation. Furthermore, the editing study with an expert analyst suggested that while SmartBook lays a solid foundation, it benefits from human refinement. Overall, SmartBook emerged as a robust tool for intelligence analysis, balancing automated efficiency with the need for human expertise.

%Cut down and add in the content review.

%\textcolor{blue}{\textbf{TODO: Daniel - Add a brief paragraph showing the overall flow of the paper, i.e. formative study guides collaborative design guides design strategies guides Smartbook, and then system + content evaluation studies}}

 %\textcolor{blue}{\textbf{TODO Revanth: add in sentence, per Agathe, that our approach is presented for how to proceed in integrating an AI into the task of situation report generation, the approach and insights can be applied more broadly to other tasks. In other words, this is not a one-off exemplar.}}

% As a case study, we consider the case of automatic situation report generation for the Ukraine-Russia crisis. This scenario is particularly challenging, given the complexity of the conflict, the politicization of the issue, and the need to regularly update the reports as the situation evolves. 

%Our proposed automatic situation report is created from a corpus of news articles and is organized along a timeline, with new timespans added as the conflict progresses. Each timespan covers a 2-week period and each major event identified within that period is the basis for a separate chapter. Each chapter is further subdivided into sections corresponding to the automatically identified strategic questions that are relevant to the chapter event.  Each section contains a grounded query-focused summary with its relevant claims. Figure \ref{fig:1} shows an example from SmartBook for the Ukraine-Russia crisis, with the structured hierarchy of timespans, chapters and corresponding sections.


%Our contributions, centered around enhancing user experience and promoting effective human-computer collaboration in the intelligence analysis space, can be summarized as follows:
The contributions of this work are as follows:
\begin{itemize}
    \item A comprehensive formative study and collaborative design process for identifying the design strategies to guide the automated generation of situation reports.
    \item \textbf{SmartBook}, an automated framework that generates comprehensive, up-to-date situation reports from various sources and presents them in an intuitive and user-friendly manner. It identifies critical strategic questions, ensuring that downstream readers receive targeted, relevant, and evidence-grounded information to aid their decision-making processes.
    \item A user study with 10 intelligence analysts and 2 decision-makers investigating the usability of the system.
    \item A content review to grade the quality of the information generated, along with an editing study to understand how viable SmartBook is for producing preliminary first drafts.    
    %\item Through both quantitative and qualitative evaluation focusing on situation report generation for the Ukraine-Russia crisis, we demonstrate that SmartBook is not just technologically robust but also meets the usability and informativeness expectations of expert analysts, proving vital for strategic planning.
    %\item In terms of providing actionable intelligence to users, we show that SmartBook outperforms a web search $+$ large language model approach (akin to Bing + ChatGPT) by up to 9\% in terms of relevance and strategic value of the information presented. 
    %\item \textcolor{blue}{Finally, we show results from a study on a senior expert intelligence analyst performing the task of situation report generation, by using SmartBook's output as a preliminary first draft.}
    \item A publicly accessible interface for SmartBook and underlying code to foster collaboration, feedback, and further innovation in the intelligence analysis domain.
\end{itemize}

%\item We identify an exciting new problem domain and task formulation of \textit{automated situation report generation}, which aims to assist human analysts, not replace them.
    %\item We propose \textsc{\textbf{SmartBook}}, an automated framework that generates comprehensive, up-to-date situation reports from various sources, by discovering critical strategic questions and grounding the report with trustworthy factual evidence.
    %\item We quantitatively and qualitatively evaluate our automatic situation report generation approach for the Ukraine-Russia crisis and demonstrate that SmartBook is well-received by expert analysts in terms of informativeness and effectiveness for strategic planning.
    %\item We show that SmartBook outperforms a web search $+$ large language model based generation approach (akin to Bing + ChatGPT) by up to 9\% in terms of relevance and strategicness of information presented. 
    %\item Finally, we contribute a publicly accessible UI\footnote{\url{https://blenderdemo.com/smartbook}} and code\footnote{\url{https://github.com/blender-nlp/SmartBook}} for SmartBook so that others can explore and improve upon these.
    
%\wordcount