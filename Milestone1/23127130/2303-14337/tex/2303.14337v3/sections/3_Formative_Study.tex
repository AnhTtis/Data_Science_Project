\section{Formative Study}
\label{sec:formative_study}
To gain an operational understanding of the range of processes that take place during intelligence analysis and generating situation reports, we conducted a formative study to gather information on: (1) regular practices in authoring situation reports and (2) general needs and expectations from intelligence analysts for AI-driven systems. Below, we start with a brief description of situation reports to provide background and then proceed with specifics on the formative study.

%\heng{I think the study description section can benefit from more concrete writing. maybe even attach the assessment guidelines etc. in appendix}

\subsection{Background}
\label{sec:situation_report_background}
%\textcolor{blue}{\textbf{TODO: Daniel - Needs some explanation (maybe at the beginning of Section 3) on why this situation report overview section is here.}} \revanth{Is this `our understanding' of what situation reports are and how they are generated, based on the formative study?}
%\daniel{I've added a short sentence below; however, the study task also describes that a component of the formative study is to understand what situation reports are and how they're created.}\item \textcolor{blue}{\textbf{I'll edit above at start of section.}}
A situation report is a document produced by an intelligence analyst to disseminate critical information. The report typically focuses on a specific issue or topic and provides a comprehensive overview of the information available on that topic. Intelligence analysts draft situation reports through a meticulous and systematic approach that begins with the collection of information from diverse sources, such as news articles, broadcasts, and potentially classified intelligence feeds. Once gathered, they prioritize the verification of this data, procuring reliable sources to ascertain accuracy. The subsequent analysis phase involves delving deep into the raw data to discern patterns, motivations, and implications, often requiring specialized knowledge or technical expertise. Post-analysis, the synthesized information is integrated into a cohesive narrative, emphasizing relevance and urgency, resulting in a clear, concise, and actionable report that communicates the prevailing situation to stakeholders. %Generating a situation report can take from 30 minutes to 3 hours, depending on the urgency, scale and complexity of the given situation.
The ultimate goal of such reports from intelligence analysts is to support downstream readers (i.e., consumers of the reports such policy-makers, government officials, and military personnel) in keeping track of developments by providing them with the time-critical, necessary information to make informed decisions and take appropriate actions.

\subsection{Method}
\label{sec:formative_method}
\subsubsection{Recruitment}
\begin{table}[!htb]
\centering
\label{tab:participants}
%\resizebox{\textwidth}{!}{
\begin{tabular}{llllll}
\toprule
\textbf{PID} &
  \textbf{Age} &
  \textbf{Education} &
  \textbf{Intelligence Exp.} &
  \textbf{AI Knowledge} &
  \textbf{LLM Usage} \\\midrule
IA1 & 25 - 34 & Bachelor's & 5 - 10 years & 2 - Intermediate & Rarely \\
IA2 & 25 - 34 & High School Diploma & 2 - 5 years & 3 - Proficient & Rarely  \\ 
IA3 & 18 - 24 & High School Diploma & 1 - 2 years & 2 - Intermediate & Daily  \\
IA4 & 25 - 34 & Bachelor's & 2 - 5 years & 2 - Intermediate & Rarely  \\
IA5 & 45 - 54 & Bachelor's & 2 - 5 years & 1 - Basic & Rarely  \\
IA6 & 45 - 54 & Master's & 1 - 2 years & 1  - Basic & Rarely  \\
IA7 & 35 - 44 & High School Diploma & 5 - 10 years & 1  - Basic & Weekly  \\
IA8 & 45 - 54 & Master's & 1 - 2 years & 2 - Intermediate & Rarely  \\
IA9 & 35 - 44 & Bachelor's & 2 - 5 years & 2 - Intermediate & Rarely  \\
IA10 & 25 - 34 & Bachelor & 2 - 5 years & 4 - Advanced & Weekly   \\ 
\bottomrule      
\end{tabular}
\vspace{0.3em}
\caption{Demographic information of study participants along with their experience in intelligence report generation.}
\label{tab:analyst_demographic}
\vspace{-1.5em}
\end{table}
For the formative study, we targeted individuals with experience in government and military roles. We distributed a pre-screening survey to ascertain their background in creating situation reports. Participants qualifying for the study were either experienced intelligence analysts or had a minimum of one year of equivalent experience. The final group comprised 10 military personnel from different branches, as shown in Table \ref{tab:analyst_demographic}. Their experience in intelligence report writing varied between 1 and 10 years.

%\textcolor{blue}{In our study\footnote{The study protocol was approved by our institution's IRB.}, we engaged with individuals who have served, or are currently serving, in various capacities within government, including the military. To identify potential participants, we disseminated a pre-screening survey focused on eliciting information about their prior experience in generating situation reports. Those who qualified for our study self-identified as experienced intelligence analysts (or with 2 years of equivalent experience). Our final participant pool consisted of 10 individuals: 10 armed forces personnel from various branches of the military. As detailed in Table \ref{tab:analyst_demographic}, the range of their experience in intelligence report authoring spans from 2 - 10 years.}

%{\textcolor{red}{\textbf{need to spell out what``ML Prof.'' and the values 1,2,4 refer to in caption of Table 2  -- Clare}}

 %\heng{do we need to provide more details who these analysts are? which agencies are they from?}
%\daniel{In the interest of privacy, especially for military personnel, we don't particularly require additional information. From a reviewer's perspective + paper contribution, the granularity in military demographic details don't add more value per say. }
%\heng{maybe we should also add more details on why we choose them as our users - first, we choose ukraine crisis as a case study because it's a complex situation and we need AI to identify important information and generate strategically important questions and answers; so we choose users from the military for this study? also we need to emphasize our techniques are not limited to military domain, we also have done other case studies on earthquake, dipomacy, etc.}
%\daniel{For an HCI study, we come from an inductive approach. Meaning, we don't begin with a tool or a hypothesis, rather a user group. Therefore, a deductive approach common in NLP, where we have the tool, and identify the correct user group to evaluate against is not used. It is true, that our techniques are not limited to the military domain; therefore, this is within the discussions + future work section, where we can discuss the technology is generalizable.}

\subsubsection{Study Task} Over a two-week period, semi-structured interviews were conducted to examine two key areas: Authoring Experience and AI-Assisted Tools in situation report generation. The Authoring Experience section focused on participants' personal experiences in creating intelligence reports, including their methodologies, challenges, and key aspects of their authoring process. The AI-Assisted Tools section explored their understanding, attitudes, and recommendations regarding AI use in professional settings.

%The interview sessions, conducted over 2 weeks, adhered to a semi-structured format, systematically exploring two aspects of generating situation reports: Authoring Experience and AI-Assisted Tools. In the Authoring Experience segment, we delved into the participants' own experiences with creating intelligence reports, exploring their processes, notable challenges, and focal points in their authoring journey. The AI-Assisted Tools component concentrated on participants' understanding, perspectives, and suggestions of using AI, including their general attitudes towards AI, and its integration into their professional practices.}

The interviews began with open-ended questions to encourage unrestricted responses, followed by targeted inquiries for deeper insight into their report authoring perspectives. Each 60-minute session, conducted via video conferencing, allowed participants to extensively share their experiences and views. Compensation ranged from \$25 to \$35 per hour, reflecting participants' levels of experience.
%Each interview commenced with broad, open-ended questions, designed to minimize constraints on participant responses, while probing the topics that arose naturally in follow-up inquiries to deepen our understanding of their perspectives on authoring reports. This approach allowed participants to elaborate at length on their experiences and views. Each session spanned 60 minutes and was conducted via video conferencing, ensuring a thorough and consistent exploration of the topics raised by participants. Participants were compensated \$25 - \$35 per hour, commensurate with their experience.}
%\item \textcolor{red}{\textbf{I switched 'themes' to 'topics' since the term 'themes' is used below for content that is discovered and coded.  -- Clare}}

\subsubsection{Analysis Procedure}%\revanth{I agree with Agathe here. We might need to show the themes/codes. Should we add in that flowchart that you had in the slides?}
\label{sec:formative_analysis}
The interview data analysis employed an inductive method, abstaining from any pre-existing theories or hypotheses. Two members of the research team concurrently examined the same data segments and noted their findings independently. This stage highlighted emerging \textit{themes} related to analysts' perceptions and expectations of AI-assisted authoring tools, such as trust, efficiency, and interaction. A thorough coding of the interview transcripts followed, aiming to systematically pinpoint these themes. This entailed a repetitive process of code development and adjustment as the team navigated the transcript collection. To ensure accuracy and thoroughness, periodic cross-checks were conducted by two coders.

%The analysis of the interview data followed an inductive approach, in which our observations and data were collected without preconceived theory or hypothesis. Two research team members collaboratively simultaneously reviewed identical data segments and independently documented their observations. Throughout this phase, we curated recurring {\textit{themes}} that arose in the analysts' perceptions and expectations of authoring experiences with AI-assisted tools (e.g. trust, efficiency, and interaction). Subsequently, we conducted a comprehensive coding of the interview transcripts, to systematically identify these themes. This process involved an iterative development and modification of codes as research team members worked their way through the collection of transcripts. To maintain the rigor and precision of our coding process, two coders periodically conducted cross-checks.}

\subsection{Key Findings}
\label{sec:formative_findings}
\subsubsection{\textbf{KF1.} Viewing technology as a means to enhance human capability.}
While examining intelligence analysts' views on advanced technology, an overwhelming majority (9 out of 10) emphasized the crucial role of digital technologies and artificial intelligence in enhancing their capabilities, specifically in generating situation reports. These tools are regarded not as mere process accelerators, but as essential elements that enrich their work by improving research efficiency, idea generation, and clarity of information. This perspective contrasts with the simplistic media depiction of these technologies as mere replacements for human effort. Instead, analysts view them as valuable enhancements to their workflow, particularly in high-pressure or resource-limited scenarios. For instance, one analyst commented:

%When exploring the intelligence analysts' perspectives on advanced technology, 9 out of 10 participants (IA1-IA7 \& IA9-10), described the vital role of digital technologies and AI in augmenting their capabilities, particularly in the context of situation report generation. These technologies were not seen merely as separate tools for speeding up processes, but as key contributors integral to enriching the analysts' work, by improving aspects like research efficiency, brainstorming, and clarity of information. Contrary to the often simplistic portrayal of these technologies in popular media as mere substitutes for human effort, analysts perceived them as providing valuable enhancements to their workflow, especially under high-pressure situations or resource constraints. For instance, one analyst commented:} %\heng{what was the negative comment from the person who disagree?}

\begin{quote}
``There's so many parts of authoring a situation report, each requiring strict and robust procedures. I wish I could have an AI partner to help me. It could make me better at my job.'' (IA6)
\end{quote}

This highlights a paradigm shift in the perception of technology within intelligence work, where it is seen not as a substitute, but as a vital complement that amplifies human skills.

%This insight underscores a paradigm difference in how technology is viewed in intelligence work – not as a replacement for human skills, but as a powerful ally that complements and strengthens human capabilities.}

\subsubsection{\textbf{KF2.} Trusting and relying on machines, as with humans.} The majority of participants (8 out of 10) exhibited a tendency to attribute human-like qualities of trust and reliability to AI systems, akin to their assessment of human colleagues. Analysts recognized AI as a dependable information source, comparable to a human co-worker, with a focus on its consistent delivery of high-quality outputs. The criteria for trusting AI closely resembled those for human interactions: the ability to provide dependable information, transparency in reasoning, and a foundation in verifiable facts. One participant articulated this equivalence in trust establishment:

%Most participants (8 out of 10) conveyed a tendency to attribute human-like qualities of trust and reliability to AI systems, paralleling the way they evaluate their human counterparts. Analysts designated AI as a reliable source of information, similar to a human colleague, with an emphasis on the AI's ability to consistently deliver quality outputs. The criteria for trust in AI, therefore, mirrored those applied to human interactions: the ability to provide dependable information, transparency in the process of deriving conclusions, and a foundation in verifiable facts. When inquired about how and why they would trust AI and human and the similarity in response, an intelligence analyst pointed out:} 

\begin{quote}
``After comparing my expectations between a human and an AI colleague, I can't say there's much of a difference. Trust is built the same way regardless of who it's towards. Just as I would trust a colleague that can explain their work, I could trust an AI that could show its workings.'' (IA1) 
\end{quote}

Interestingly, analysts did not set higher standards for AI than for human colleagues. This parity in trust and reliability criteria suggests that participants viewed AI as an equal collaborative partner, assessing its competence and trustworthiness on the same grounds as a human team member.

%Curiously, the analysts did not impose additional requirements or expectations on AI systems beyond what they would expect from a human colleague. The equivalence in the standards of trust and reliability, as expressed by participants, indicated that they viewed AI as a collaborative partner and judged its competency and trustworthiness by the same standards and merits as those of a human team member.}

\subsubsection{\textbf{KF3.} Training and guiding AI} 
Our research identified a split in intelligence analysts' perspectives on their role in training and guiding AI systems. Four out of ten participants advocated for substantial control over AI, emphasizing the need for an interactive system that allows them to influence everything from information source selection to narrative shaping in reports. They prioritized the ability to refine AI outputs based on their expert judgment. Two participants stated:

%We found a split in intelligence analysts' attitude to their involvement in training and guiding AI systems. 4 out of 10 participants expressed a strong desire for high-level control over AI systems. This group emphasized the need for a highly interactive system where they could exert significant influence, from selecting information sources to tailoring the narrative in reports. They believed in defining the quality of AI outputs based on their expert judgment, desiring systems that allowed for extensive fine-tuning. Two analysts stated:} 

\begin{quote}
``We're trained to be great at what we do. Isn't AI, like picking a random university student, and asking them to do our job?'' (IA2)
\end{quote}

\begin{quote}
''It's not about control. If someone is going to help me, I'm going to make sure they do it right.'' (IA7)
\end{quote}

In contrast, the majority (six out of ten) favored a more hands-off approach, highlighting that situation report creation follows well-established, standardized procedures suitable for AI implementation. They perceived AI involvement as an extension of routine oversight, akin to reviewing a junior colleague's work:

%Conversely, the remaining six participants viewed the role of AI differently. They pointed out that the process of creating situation reports is well-documented and follows strict, standardized procedures. According to them, these procedures are areas where AI could excel, as these standards and protocols could be effectively distilled into AI systems to produce concise and accurate reports. While acknowledging the need for oversight and occasional fine-tuning, they saw this as a routine aspect of their work, not unique to AI.} 
\begin{quote}
''There's a specific way we have to do each action, and making reports is no different. I just need to tweak small details as I would when reviewing other junior ranking officer's work.'' (IA10)
\end{quote}

This variance in attitudes seems to highlight a broader debate within intelligence work: the balance between human expertise and automated efficiency.

%This split in attitude and perceptions among intelligence analysts regarding the degree of human involvement and control desirable in AI-assisted processes also reflects a broader debate about the balance between human expertise and automated efficiency in intelligence work.}