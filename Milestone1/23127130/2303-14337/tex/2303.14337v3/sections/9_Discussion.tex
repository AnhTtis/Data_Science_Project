\section{Discussion}
%Intelligence is more than data aggregation and hypothesis formation.


% In this direction, we study whether SmartBook-generated situation reports can be a useful preliminary draft, in comparison to intelligence analysts' prior experience in solely authoring situation reports.


 %This leaves room for human analysts to augment and fine-tune, merging AI's broad processing capacity with human expertise.

%\revanth{As per Agathe's feedback, this section needs to have a broad high-ranging discussion of the results and potential impact of this work.}
%\textcolor{blue}{SmartBook provides a reliable, accurate, and efficient method for digesting a complex and large corpus of news and thus also supports a wide-open vision for a new approach to the authoring of situation reports, as well as the vast range of summary reports in the business world (\ref{sec:outlook}). Our approach to the development of SmartBook represents an application of the ``research-through-design'' methodology within AI design, a strategy not widely explored in this domain. This approach offers a pragmatic, real-world setting for human-AI collaboration, by emphasizing the tangible needs of practitioners, in this case intelligence analysts and decision makers. Consequently, SmartBook stands as a significant exemplar of how AI can be effectively rooted in addressing specific, practical challenges, thus making a remarkable contribution to both Human-Computer Interaction (HCI) and AI disciplines.}

SmartBook leverages the zero-shot capabilities~\cite{brown2020language} of LLMs for tasks with limited training data. Complex tasks like grounded summary generation with citations and strategic question identification are tackled by LLMs, while smaller models address well-defined tasks with available data, such as event headline (chapter name) generation~\cite{headline2020}, duplicate question detection~\cite{cer2017semeval}, and claim extraction through question answering~\cite{kwiatkowski2019natural}. Our experiments revealed a tendency of the LLMs to generate similar strategic questions, necessitating sampling methods~\cite{holtzman2019curious} for variety. Directing the LLMs to reference input documents while generating summaries enhanced factuality, aligning with recent findings~\cite{gao-etal-2023-enabling}. Nevertheless, LLMs' tendency to hallucinate~\cite{ji2022survey, tam2022evaluating}, evident from inaccuracy errors in \S{\ref{sec:error_analysis}}, necessitates future enhancements  in validation and cross-referencing techniques (discussed in \S{\ref{sec:future_verification}}) to mitigate this issue.


In light of the ongoing developments in AI-assisted writing~\cite{WangACL2019,Wang2020,selfcollaboration2023,cardon2023challenges}, where tools have been developed for a variety of end-users like academics~\cite{gero2022sparks}, screenwriters~\cite{mirowski2023co} and developers~\cite{chen2021evaluating}, SmartBook introduces a unique automated framework designed specifically for generating situation reports for intelligence analysts. Prior AI-assisted writing systems, such as CoAuthor~\cite{lee2022coauthor}, Creative Help~\cite{roemmele2015creative}, Writing Buddy~\cite{samuel2016design} and WordCraft~\cite{yuan2022wordcraft}, are designed to be collaborative and aid in creative tasks~\cite{klein1973automatic}. SmartBook highlights a different aspect of human-AI collaboration, where AI's role is more about efficient information processing with analytical rigor and less about creative or iterative human engagement. This distinction underscores SmartBook's design with a focus on factuality, analytical depth, and efficiency in more structured, data-driven environments, such as intelligence analysis. Clearly the design also stands to benefit from iterative experimentation and refinement, to explore where deeper integration of SmartBook into analysts' workflows improve their task performance. We discuss this as future extensions in \S{\ref{sec:iterative_refinement}}.


\subsection{Outlook}
\label{sec:outlook}
%\textcolor{blue}{\textbf{TODO: Daniel - Connect brief overview of what were our takeaways / understandings (based on formative study) of how analysts originally perceived AI generated outputs, and how they were using SmartBook reports in the studies. Discussion should be high-level and ideally not repeat anything that was mentioned in results of the formative study/user study.}}

The central challenge faced by analysts is massive data overload that they must filter in addressing information requirements to produce situation reports.  The key findings of the formative study show that analysts, though appropriately cautious, are open to incorporating AI assistance into their workflow.  The desiderata in the design strategies that they articulated during the collaborative design also make clear that they are eager to participate in crafting the new vision that will include AI assistance that not only adjusts to the data content of shifting scenarios but also to their individual analytic methods.

SmartBook, characterized by its generalizable and modular design, can efficiently automate the generation of situation reports across various scenarios, including geopolitical, environmental, or humanitarian contexts. Unlike human analysts who face considerable challenges in transferring their expertise across domains, SmartBook requires minimal configuration for such shifts. This transition in traditional settings involves extensive background research and the recruitment of domain experts, demanding substantial time and effort. Furthermore, each new scenario introduces complex variables and details that must be meticulously integrated. SmartBook addresses these challenges with a streamlined and consistent approach, thereby offering accurate reports in diverse and rapidly changing global contexts.

The effectiveness of SmartBook was primarily evaluated on news articles from the Ukraine-Russia military conflict. Here, we briefly explore SmartBook's applicability in a humanitarian crisis, specifically the Turkey-Syria earthquake. By changing the source articles\footnote{Here is a URL corresponding to news for February 6: \href{https://edition.cnn.com/middleeast/live-news/turkey-earthquake-latest-020623/index.html}{link}} for major event identification, SmartBook successfully generated pertinent situation reports for the specified time window (February 6-13). These reports accurately captured critical aspects of the earthquake disaster, such as aftershock impacts (\href{https://www.cnn.com/middleeast/live-news/turkey-syria-earthquake-updates-2-7-23-intl/h_a9e1d082dd54c67187fd51322f95eaf2}{link}), international humanitarian assistance (\href{https://www.incirlik.af.mil/News/Article-Display/Article/3292036/urban-search-and-rescue-teams-arrive-at-incirlik-air-base/}{link}), rising death tolls (\href{https://www.reuters.com/world/middle-east/death-toll-syria-turkey-quake-rises-more-than-8700-2023-02-08/}{link}), and specific events like the disappearance of a Ghanaian soccer star (\href{https://www.marketwatch.com/story/soccer-star-christian-atsu-still-missing-after-turkey-earthquakes-i-still-pray-and-believe-that-hes-alive-says-partner-cfd47272}{link}). 

This adaptability of SmartBook, demonstrated in the domain shift from military to humanitarian crisis contexts, underscores its robustness and versatility. It also emphasizes the importance of the source articles and suggests that, with mission-relevant data input, SmartBook has the potential to be an invaluable tool for analysts across a multitude of scenarios. Overall, SmartBook presents an exciting step forward in AI-assisted situation report generation and signifies a momentous stride in the domain of intelligence analysis.



%\textcolor{blue}{In this section, we describe some key learnings and implications of this project (in \S{\ref{sec:learnings}}) both from the perspective of HCI+AI and intelligence analysis communities. Next, we discuss explorations (in \S{\ref{sec:transferability}}) for applying SmartBook to scenarios in other intelligence domains . Finally, we highlight some limitations of SmartBook (in \S{\ref{sec:limitations}}) and present directions for future work (in \S{\ref{sec:extensions}}).}

%Given the surprisingly high-performance zero-shot capabilities of LLMs, SmartBook leverages LLMs for tasks without considerable training data, such as grounded summary generation with citations, and identifying strategic questions. On the other hand, smaller models were leveraged for focused tasks that had training data available, such as event headline (chapter name) generation~\cite{headline2020}, duplicate question detection~\cite{cer2017semeval}, and claim extraction using question answering~\cite{kwiatkowski2019natural}. From our experiments, we observed that even when prompted to generate diverse strategic questions, the model was prone to generating very similar ones, thereby requiring sampling tricks and repeated generation to promote variety. In addition, initial explorations suggested that specifically instructing the language model to cite input documents improved the factuality of the generated summaries, as evidenced in recent works~\cite{gao-etal-2023-enabling}.However, the hallucination problem \cite{ji2022survey, tam2022evaluating} of such LLMs is evident from the considerable number of summaries (in \S{\ref{sec:error_analysis}}) with inaccurate information. This could be mitigated in future iterations (as discussed in \S{\ref{sec:future_verification}}) by incorporating stronger validation mechanisms and cross-referencing techniques.} 


%AI-assisted writing~\cite{WangACL2019,Wang2020,selfcollaboration2023,cardon2023challenges} is a rapidly growing field in human-computer interaction research. In recent years, researchers have explored various approaches to develop AI systems that can assist users in writing tasks. AI-assisted writing tools have demonstrated benefits to a variety of end-users, like academics~\cite{gero2022sparks}, screenwriters~\cite{mirowski2023co} and developers~\cite{chen2021evaluating}. CoAuthor~\cite{lee2022coauthor} crafts a dataset that encapsulates the rich interactions observed between multiple writers and instances of GPT-3 over numerous writing sessions, thereby highlighting the efficacy of AI in fostering collaborative writing endeavors. In the area of automated story-writing~\cite{klein1973automatic}, attempts such as Creative Help~\cite{roemmele2015creative}, Writing Buddy~\cite{samuel2016design} and WordCraft~\cite{yuan2022wordcraft} aid the user in producing full stories with creative control throughout the process. More recently, with humans struggling to identify text as being machine generated ~\cite{wahle2022large}, concerns have been raised about academic integrity~\cite{perkins2023academic} and plagiarism\cite{wahle2022identifying} when using AI-assisted writing tools.In light of the ongoing developments in AI-assisted writing, SmartBook introduces a unique automated framework designed specifically for generating situation reports for intelligence analysts. While the existing body of work predominantly focuses on augmenting creative and collaborative writing, SmartBook aims to assist intelligence analysts' workflows with a focus on factuality and analytical depth.

%Qualitative evaluations underscore the potent functionality of SmartBook, with a significant majority of its automatically discovered questions directing focus on strategic information. Additionally, in addressing the current geopolitical tensions in the Ukraine-Russia context, it has demonstrated its indispensability in strategizing and planning, showcasing not only its adaptability to real-world scenarios but also its capacity to provide actionable intelligence. }


%\subsection{Implications}

%\textcolor{blue}{ Firstly, it is noteworthy that 15\% of summaries produced by SmartBook needed no modifications, highlighting its proficiency in creating acceptable reports in some scenarios without human intervention. This suggests that as technology advances and iterative refinements are applied, this rate could potentially increase, reducing the workload for analysts even further.  Lastly, the hallucination problem, although present, could be mitigated in future iterations (which we discuss in Section \ref{sec:future_verification}) by incorporating stronger validation mechanisms and cross-referencing techniques.}


%\subsection{Implications}


%SmartBook emphasizes generalizability in automating the generation of situation reports. Its modular design ensures that when the need arises to transition to a new scenario - be it geopolitical, environmental, or humanitarian - SmartBook can adapt with minimal configuration. Meanwhile, traditional intelligence analysts often grapple with a multifaceted challenge when pivoting between scenarios. The transition demands a significant time investment for deep dive into background research, recruitment of domain experts, etc. Additionally, each new scenario introduces a vast array of variables and particulars that need to be meticulously accounted for and incorporated. The inherent cognitive constraints on an individual analyst's capacity to process and synthesize vast amounts of information mean that the risk of oversight or error is always present. In stark contrast, SmartBook offers a streamlined, consistent, and efficient mechanism to adapt and provide accurate situation reports across diverse scenarios in rapidly changing global contexts. %This makes it an invaluable asset in rapidly changing global contexts, ensuring timely and reliable generation of situation reports.

 %We evaluated SmartBook primarily in the context of a military conflict by choosing Ukraine-Russia crisis as the evaluation scenario. In this section, we explore the possibility of how well SmartBook can generalize to intelligence analysis for a humanitarian crisis. Specifically, we consider situation report generation for the Turkey-Syria earthquake scenario by considering a single timeline for the time window of February 6 - 13. This aims to demonstrate portability to new scenarios, by simply switching the source articles\footnote{Here is a URL corresponding to news for February 6: \href{https://edition.cnn.com/middleeast/live-news/turkey-earthquake-latest-020623/index.html}{link}} used to identify major events (described on Section \ref{subsec:major_event}). We observe that SmartBook is able to identify and present important aspects of the Turkey-Syria earthquake disaster, such as the continued impact of aftershocks (\href{https://www.cnn.com/middleeast/live-news/turkey-syria-earthquake-updates-2-7-23-intl/h_a9e1d082dd54c67187fd51322f95eaf2}{link}), international humanitarian assistance in the form of search and rescue teams (\href{https://www.incirlik.af.mil/News/Article-Display/Article/3292036/urban-search-and-rescue-teams-arrive-at-incirlik-air-base/}{link}), the rising death toll in the aftermath of the earthquake (\href{https://www.reuters.com/world/middle-east/death-toll-syria-turkey-quake-rises-more-than-8700-2023-02-08/}{link}) and missing of a soccer star from Ghana (\href{https://www.marketwatch.com/story/soccer-star-christian-atsu-still-missing-after-turkey-earthquakes-i-still-pray-and-believe-that-hes-alive-says-partner-cfd47272}{link}). The results from the Turkey-Syria earthquake scenario indicate that SmartBook's underlying framework is adaptable and effective across diverse situation report contexts. The ability to transition from a military conflict to a humanitarian crisis, while maintaining accuracy in capturing key events, highlights robustness and versatility. It also emphasizes the importance of the source articles and suggests that with the right data input, SmartBook has the potential to be an invaluable tool for analysts across a multitude of scenarios.


\subsection{Limitations}
\label{sec:limitations}
While SmartBook represents a significant advancement in the automated generation of situation reports, it is essential to acknowledge certain limitations that stem from both the technical aspects of the system and the scope of its application. These include unverified news source credibility, potential inaccuracies in reflecting source material despite citations, and user studies focused mainly on military intelligence analysts, which may not represent the needs of a wider analytical audience. Recognizing these limitations is crucial for guiding future improvements and ensuring the framework's applicability across across diverse intelligence and analysis sectors.

\begin{itemize}
    %\item \textcolor{blue}{In addressing the capabilities of SmartBook, it is important to clarify that its current functionalities are directed towards the goal of providing a comprehensive initial draft of situation reports. SmartBook currently does not support multi-turn editing or the iterative refinement of reports. In its role as an initial draft provider to streamline the early stages of situation report generation, SmartBook is intended to be a valuable aid in the intelligence analysis workflow, though not as a standalone solution for the entire report development cycle.}
    \item The generation of situation reports within SmartBook leverages news articles aggregated from Google News. Nonetheless, the process does not involve a rigorous assessment of the news sources' credibility, nor does it incorporate an explicit verification of the factual accuracy of the claims used in the summary generation model. Considering the extensive exploration of these aspects within computational social science~\cite{lee2023pandemic} and natural language processing~\cite{gong2023fake}, we deemed them beyond the scope of SmartBook's framework in this work.
    \item SmartBook enhances the reliability and credibility of the generated situation reports by incorporating citations. However, it does not rigorously ensure the accuracy of these reports in reflecting the content of the source documents.
     In contrast, LLMs have been shown~\cite{mallen2022not, baek2023knowledge} to tend to produce content that may not directly correlate to the source materials. While this is still an active area of research, recent studies~\cite{tian2023fine} have shown that LLMs can be effectively optimized to enhance factual accuracy and attribution capabilities~\cite{gao2023enabling}. Consequently, we propose that future advancements could involve substituting the current summary-generating language model with one that is specifically refined for improved attribution~\cite{gao2023enabling}.%{While SmartBook helps promote trust and credibility of the generated situation reports through the addition of citations, it does not explicitly verify the faithfulness of the generated reports to the input documents. On the other hand, LLMs can generate content that is not attributable~\cite{} to input documents. While this is still an active area of research, recent work~\cite{} has demonstrated that large language models can be specifically tuned to improve factuality and the ability to attribute. Therefore, we assert that this can be tackled in future work by replacing the language model generating the summaries with a model fine-tuned for attribution.}
    \item The selection of analysts for the user studies in SmartBook was primarily comprised of individuals from military intelligence. This limitation arose due to the challenges encountered in recruiting analysts who could participate without breaching confidentiality agreements. Consequently, our recruitment efforts were confined to analysts within our existing networks. It is important to recognize that our studies may not fully represent the needs and perspectives of analysts in other fields, such as political, corporate, or criminal analysis. Acknowledging this gap, future work can aim to extend and adapt SmartBook's capabilities for generating situation reports, tailoring them to meet the distinct requirements of each of these domains.%{The analysts recruited for the user studies in SmartBook are predominantly military analysts. This stems from the difficulty in recruiting analysts given confidentiality agreements, hence we were restricted to recruiting those based on prior connections. However, we acknowledge that our studies do not account for other domains such as political analysis, corporate analysis, or criminal analysis. We leave this for future work, to expand and customize SmartBook's formulation of situation report generation to the specific requirements for each of these domains.}
\end{itemize}

\subsection{Future Extensions}
\label{sec:extensions}
This section outlines key extensions planned for SmartBook, each aiming to enhance its functionality and utility. SmartBook's future enhancements include integrating multimodal and multilingual information to enrich intelligence analysis (in \S{\ref{sec:multimodal_info}}) and employing a balanced approach to news sources to mitigate biases (in \S{\ref{sec:source_bias}}). Additionally, the system will evolve with a co-authoring feature for dynamic report refinement (in \S{\ref{sec:iterative_refinement}}) and introduce a `verification score' to assess the reliability of claims, ensuring precise and dependable decision-making support (in \S{\ref{sec:future_verification}}). Overall, these extensions are designed to elevate SmartBook's utility as a comprehensive, bias-balanced, and reliable tool for situation report generation in intelligence analysis.


\subsubsection{Incorporating Multimodal, Multilingual Information:}
\label{sec:multimodal_info} %In intelligence analysis, situation reports often integrate diverse data types, including textual accounts, news articles, photographs, videos, and audio recordings, each offering unique insights. Figure \ref{fig:multimodal_example} illustrates how images can corroborate or challenge textual evidence in SmartBook, highlighting the value of multimodal data. We aim to enhance SmartBook's comprehensiveness by integrating various data modalities, correlating textual claims with pertinent multimedia elements using advanced multimedia knowledge extraction systems such as GAIA~\cite{li2020gaia}. This integration envisions an ecosystem where AI-assisted report generation merges multiple data types, providing analysts a comprehensive, nuanced, and corroborated narrative.\\
 Intelligence analysis increasingly requires the integration of diverse data types, including text, images, videos, and audio recordings, to provide a comprehensive understanding of global events. We plan to enrich SmartBook's situation reports by correlating textual claims with relevant multimedia elements, leveraging systems like GAIA~\cite{li2020gaia} for advanced multimedia knowledge extraction. Figure \ref{fig:multimodal_example} exemplifies how images can support or contradict the text, underscoring the value of incorporating various data modalities. In addition to multimodal data, the globalization of events necessitates a multilingual approach to intelligence analysis, as reliance on English alone can overlook critical local nuances. SmartBook aims to overcome this limitation by incorporating multiple languages, enabling a more profound understanding of local dynamics that influence global situations, capturing cultural subtleties, regional politics, and localized social phenomena in native languages. Moreover, integrating multilingual sources democratizes intelligence analysis, moving beyond a predominantly English-speaking worldview. Ultimately, SmartBook seeks to redefine intelligence reporting standards, promoting a more inclusive and globally aware approach.
\begin{figure}[!htb]
    \centering
    \includegraphics[width=1.0\linewidth]{figures/multimodal_example.png}
    \caption{Figure showing an example of how multimodal information (in the form of images) supports and provides additional context to the claims presented in SmartBook. In this example, the presence of anti-aircraft weapons (as seen in the image) in Ukraine provides background for the discussion in NATO on whether to impose a no-fly zone.}
    \label{fig:multimodal_example}
\end{figure}

    \subsubsection{Controlling the Bias of News Sources:} \label{sec:source_bias} Different news sources inevitably bring with them biases that stem from factors such as editorial policies, audience demographics, and geographical influences. When a single news source or perspective dominates the narrative, key details or alternative viewpoints can be easily missed. In our vision for the next iteration of SmartBook, we aim to address these inherent biases. To overcome such challenges, it is essential to source information from a diverse array of news outlets, encompassing left-leaning, centrist, and right-leaning perspectives. %This strategy enables SmartBook to facilitate the recognition of various hypotheses and interpretations of events. 
    The goal is to enhance SmartBook's utility for intelligence analysts by reducing informational blind spots and broadening the range of considered scenarios.
%Different news sources present information from varying angles, with different levels of detail and interpretation. Thereby, we plan to control for the bias of news sources used in the creation of SmartBook to help cross-check information and avoid the potential pitfalls of relying on a single news source or a single interpretation of events. Further, alternate perspectives/hypotheses can be identified by independently considering news sources on the left, in the center, and on the right. \\
    %\item \textbf{Controlling the bias of sources}: To control bias in SmartBook, we will cross-check information from various news sources with differing angles, levels of detail, and interpretations, and also consider independent sources from the left, center, and right to identify alternate perspectives and hypotheses. \\

    \subsubsection{Co-Authoring with Iterative Refinement:} \label{sec:iterative_refinement} Authoring situation reports is an iterative process where analysts continuously refine and update the reports. In the next iteration, we aim to incorporate advanced co-authoring capabilities that will enable intelligence analysts to engage with SmartBook in a dynamic, multi-turn editing process. By integrating machine learning algorithms for continual learning~\cite{lopez2017gradient, zenke2017continual} and personalization~\cite{wu2019npa, monzer2020user}, SmartBook can learn from each interaction to progressively adapt to the analysts' preferences and decision-making styles. Further, recent techniques like Reinforcement Learning with Human Feedback (RLHF)~\cite{stiennon2020learning, bai2022training} can enable the system to assimilate human feedback, thereby refining its understanding of analysts' editing and reasoning patterns.
%can enable the system to analyze changes and feedback provided by human analysts, allowing it to understand and replicate the nuances of their editing and reasoning processes. 
This approach aims to improve SmartBook's alignment with human decisions over time and help generate reports that are tailored to the specific needs of intelligence analysis workflows. %This learning mechanism will not only improve SmartBook's alignment with human decisions over time, but also ensure that the generated reports are increasingly are relevant, and tailored to the specific needs of intelligence analysis workflows. 
     As a result, SmartBook will evolve from a preliminary drafting tool to a comprehensive AI co-author for generating situation reports.

    %\subsubsection{Adding More Languages} In an era of globalization, events that transpire in one part of the world often ripple out and influence other regions. While English remains a dominant language for international news, relying solely on it can lead to a narrowed viewpoint, potentially missing local nuances and sentiments. By integrating more languages, SmartBook seeks to transcend these linguistic barriers, enhancing the depth and breadth of the intelligence analysis. Furthermore, the integration of multilingual sources will be a critical step towards facilitating a deeper comprehension of intricate local dynamics that influence global situations. Cultural nuances, regional politics, and localized social phenomena are often best captured in native languages, offering unfiltered and nuanced insights that might be lost or distorted in translation. Additionally, this expansion promises to foster a more inclusive approach where information is not monopolized by predominantly English-speaking perspectives, thereby democratizing the intelligence analysis process. Through this endeavor, SmartBook aspires to set a new benchmark in intelligence reporting, championing a more inclusive, empathetic, and globally conscious approach.
    
    %The world is diverse, and news articles from different languages allow capturing different perspectives, insights, and information relating to global events, that might not be available in a single language. For this reason, we aim to add news articles from more languages into SmartBook to present a more comprehensive and accurate picture. Further, the ability to ingest information from different languages also enables analysts to understand local customs, cultures, and nuances that may influence the interpretation of events.\\   
    \subsubsection{Verifying the extracted claims:}\label{sec:future_verification} The integrity of data and claims in situation reports is crucial for strategic decision-making. A key challenge in automating these reports is guaranteeing the accuracy and reliability of extracted data from extensive sources. To address this, we plan to introduce a `verification score' for each claim. This score assesses the reliability of information, considering factors like source credibility, corroborative data, and historical accuracy of similar claims. It provides a confidence metric for intelligence analysts, aiding in swift evaluation of data reliability. High scores enhance confidence and facilitate quick integration into reports, while low scores signal the need for thorough review and potential further verification. This blend of automation and human oversight is essential to ensure the effectiveness of SmartBook's situation reports in strategic planning.
    %Given that situation reports contribute towards strategic planning, it is important to present accurate information to prevent misguided decisions and actions. Hence, we intend to provide a verification score, for each claim presented within SmartBook, which can be interpreted as a measure of trustworthiness. Claims with lower scores signal the need for additional oversight by the intelligence analysts before incorporating into them the final report.
    %\item \textbf{Verifying the extracted claims}: Since situation reports inform strategic planning, we will provide a trustworthiness verification score for each SmartBook claim to ensure accuracy and prevent misguided actions, with lower scores indicating the need for further analyst oversight before final report inclusion.
%\end{itemize}