\section{User Study: Evaluating the System}
\label{sec:usability_study}
During the user study, we conducted an interview with semi-structured questions and a post-study questionnaire on the usability of the SmartBook. The study investigated the following research questions:
%\revanth{RQ3 seems abrupt. Maybe move it to last RQ and also brief mention earlier that this was done from the perspective of both intelligence analyst (user) and decision-makers (downstream reader)}
\begin{itemize}
    \item \textbf{RQ1:} Can intelligence analysts successfully use SmartBook to generate situation reports?
    \item \textbf{RQ2:} How do intelligence analysts interact and leverage the features within SmartBook?
    \item \textbf{RQ3:} Do intelligence analysts find SmartBook intuitive, usable, trustable and useful?
    \item \textbf{RQ4:} How do decision-makers interact, perceive and use SmartBook?
\end{itemize}

\subsection{Method}

%\heng{a lot of these content overlaps with those in section 3, is this intentional?}
%\daniel{Thematically, there will be content overlaps with the previous sections. However, this is a reaffirmation that we developed SmartBook in line with user expectations and needs. I've added more details, to be granular about the findings in the user study.}

\subsubsection{Recruitment}
For this user study, we recruited a total of 12 participants, comprising of 10 intelligence analysts (IA1-IA10) and 2 decision-makers (DM1 and DM2). The recruitment process for the intelligence analysts involved maintaining the existing cohort from the formative study and collaborative design. Each analyst was compensated at the same rate as previous studies. 

We implemented a targeted outreach strategy to recruit decision-makers for our study. These participants, identified through email addresses sourced from government boards and social media profiles (e.g., LinkedIn, Twitter), were self-identified professionals who make decisions based on prepared reports in their official capacity. Our group comprised of individuals with current or past experience on Canadian government boards. These decision-makers engaged in the initial qualitative study but did not partake in the subsequent post-study questionnaire, due to time constraints. Unlike intelligence analysts, these participants did not receive compensation due to their government affiliation.

%\textcolor{blue}{For the decision makers, we employed a targeted outreach strategy. We reached out to potential participants utilizing emails obtained from government boards or social media platforms (e.g. LinkedIn, Twitter). The candidates self-identified as decision makers, defined as individuals who have made decisions based on reports which had been prepared for them in their professional capacity. The decision makers recruited for our study included individuals who are currently or have previously served on government boards in Canada. The decision makers participated in the qualitative study described first below, but not the post-study interview due to their limited availability. Unlike the intelligence analysts, the decision makers were not compensated due to their government affiliation.} 

\subsubsection{Study Task}
Each user study session was structured into four segments: (i)~an introductory overview, (ii)~a free-form investigation, (iii)~a guided exploration, and (iv)~a concluding reflective discussion. These sessions, each lasting about an hour, were remotely conducted, with participants engaging through their personal computers and browsers, and screen sharing their activity.

Upon commencement of the study~(i), participants received a concise introduction to SmartBook, outlining its core premise. This orientation was designed to acquaint them with the system without biasing their exploration. Subsequently~(ii), participants were invited to freely investigate SmartBook, with the specific task of exploring a minimum of three questions across five chapters of their choosing. This approach afforded substantial freedom, enabling interactions with the user interface that reflected their natural inclinations and interests.

Following the free-form investigation, participants were systematically introduced to the SmartBook framework. This process entailed tracing the journey from a chosen question related to a specific event within a designated timeframe, to the corresponding claims, contexts, and sources, culminating in a summarized answer. This tracing aligned with the F-numbered features illustrated in Figure~\ref{fig:system1}. The selection of questions for this exercise were informed by criteria such as question complexity and the variety of sources referenced. During this guided exploration phase~(iii), we ensured comprehensive exposure to each SmartBook feature, aiming for participants to fully grasp the system's capabilities.
Subsequently, a semi-structured interview~(iv) was conducted to gather reflective feedback on the participants' experience with SmartBook, focusing on its efficacy and potential areas of enhancement.

Finally, to conclude the session, participants were asked to complete a post-study questionnaire. This questionnaire was focused on assessing the usability of SmartBook, gathering quantitative data to complement the qualitative insights gained from the semi-structured interviews. Decision-makers were exempted from the post-study questionnaire.

\subsection{Results}
The data collection and analysis followed a similar method as in \S{\ref{sec:formative_analysis}}, with the addition of a post-study questionnaire. Upon collecting, discussing and iterating on the data, behaviors and insights were merged into the following themes:

\begin{figure}[t]
    \centering
    \includegraphics[width=1.0\textwidth]{tables/userstudy.jpg}
    \vspace{-3em}
    \caption{Quantitative results from the post-study questionnaire in the user study.}
    \label{fig:userstudy}
\end{figure}

%\revanth{Write a brief outline here on how you came up with or grouped these themes together.}
%\revanth{Add in a reference to and discuss the quantitative results shown in Figure 5.}
%\revanth{For subsection, we should add in a relevant quote or two from what the analysts/decision makers said in the studies/interviews.}

\subsubsection{UI Understandability and Interaction}

%The user interface (UI) of SmartBook was highly effective, as evidenced by the ease of navigation and feature discovery demonstrated in our data analysis. All participants (10 out of 10) successfully discovered each feature in SmartBook, as seen in (ii), demonstrating the UI's ease-of-understanding and low learning curve. Participants (IA2, IA3, and IA6) notably appreciated the logical organization into chapters and questions, clear navigation, and easy access to various system components. The high feature discovery rate indicates the UI's alignment with users' cognitive processes during information gathering, allowing intuitive interaction and access to system capabilities with minimal training. This positively addresses {\textbf{RQ1}}. IA2 particularly noted the advantages of SmartBook's similarity to their manual process of creating situation reports.:

The structured layout of the user interface (UI) was notably effective, as seen in the free-form investigation~(ii) with a 100\% feature discovery rate (10 out of 10 participants). All participants successfully engaged with key features, underscoring the UI's design intuitiveness. Participants IA2, IA3, and IA6 commended the logical organization into chapters and questions, as well as the straightforward navigation and component access. This high feature discovery rate indicates that the UI aligns with users' cognitive patterns, facilitating intuitive interaction without extensive training, providing us with a positive response to {\textbf{RQ1}}. Participant IA2 appreciated the UI's alignment with their existing workflow:

%The structured layout of the user interface (UI) stood out in the data analyses for the ease with which participants freely navigated and discovered the features in SmartBook, as seen in the free-form investigation~(ii) with a 100\% task completion success rate (10 out of 10 participants).Each participant successfully discovered and utilized key features of the system, a testament to the UI's design effectiveness. Participants (IA2, IA3, and IA6) highlighted the logical breakout of information into chapters and questions, the readily accessible and clear navigation from the summary to each relevant claims, and the ease with which they could access different system components. The high feature discovery rate also suggests that the UI aligns well with users' cognitive organization of content collected during their information foraging. This enables them to intuitively understand how to interact with the system and access its capabilities without extensive training or guidance, providing us with a positive answer to {\textbf{RQ1}}. IA2 mentioned the benefits of the SmartBook mirroring their manual process of generating situation reports: 

\begin{quote}
``I appreciate how I don't have to learn a new workflow just for a tool. It works how I would work.'' (IA2)
\end{quote}

This sentiment was further emphasized in the post-study questionnaire. As seen in the results in Figure~\ref{fig:userstudy}, 70\% of participants strongly agreed that most intelligence analysts would learn to use the system very quickly, while 80\% strongly disagreed that the system was difficult to navigate and use.
%This sentiment was further emphasized in the post-study questionnaire (\ref{sec:usability_study}, where 7 and 3 participants, respectively, strongly agreed and agreed that they believed most intelligence analysts would learn to use the system very quickly. This was also consistent with the responses from 8 and 3 participants, respectively, who strongly disagreed and disagreed that they found the system difficult to navigate and use. 
However, responses to issues raised by \textbf{KF3} and \textbf{DS4} revealed mixed views on UI flexibility, specifically in personalization and customization. While most found the tool easily integrateable into their workflow, a minority (30\%) remained neutral. When further prompted, participants IA8 and IA9 described the UI as a ``one size fits all'' solution, lacking personalization. As per \textbf{RQ2}, IA9 further highlighted that although it is a streamlined experience, it is inflexible to differing priorities:

%However, the participants' responses provided us with mixed answers to issues raised by {\textbf{KF3}} and {\textbf{DS4}} concerning the flexibility of the design (personalization and customization).  While most participants agreed that the tool could easily be integrated into a situation report generation workflow, some participants (3 out of 10) were neutral on its integration. When further prompted, they (IA8 and IA9) stated while the tools was generalizable and adaptable, it was nonetheless a "one size fits all" solution, that they couldn't personalize to their own experiences. An analyst further highlighted that although it is a streamlined experience, it is inflexible to differing priorities, per {\textbf{RQ2}}: 

\begin{quote}
``I don't deny that this tool is easy to use and quite understandable. But, for me to use some parts like finding sources that back a question, I have to interact with every part of the tool before I get there.''
\end{quote}

\subsubsection{Trust and Severity of Impact}
In the study, trust in the SmartBook system developed progressively, resembling the formation of trust in a human analyst. Initially, users exhibited skepticism, thoroughly scrutinizing the system's sources and assertions. This included validating the authenticity of sources, the accuracy and pertinence of ratings, and the correct representation of context. As users became more acquainted with SmartBook and evaluated its reliability, their dependence on extensive source verification lessened. As can be seen in Figure \ref{fig:userstudy}, three out of ten participants did not feel the need to conduct additional research beyond the presented information to trust the tool, while another three felt additional research was necessary. The necessity of further research was linked to the context of use; as one analyst stated:
%Trust formation with SmartBook emerged gradually during the study, akin to building a relationship with a human analyst. Initially, users approached the system with a degree of skepticism, meticulously verifying the sources and claims presented. This rigorous validation process involved checking the authenticity of each source, assessing the accuracy and relevance of the ratings, and ensuring the context was correctly captured. Over time, as users familiarized themselves with SmartBook's structures and assessed its content for reliability, their reliance on intensive source verification decreased. Three out of ten (\ref{sec:usability_study} participants disagreed that they found it necessary to research the information beyond what was presented in order to trust the tool, while 3 out of 10 participants either agreed or strongly agreed that they did find it necessary to do further research. However, when probed specifically on why they found this necessary, an analyst explained: 

\begin{quote}
``It's more about what situation we're in. If we provide incorrect information, people can get hurt.'' (IA3)
\end{quote}

This trust was found to be context-dependent, aligning with the concerns in \textbf{KF2} and \textbf{DS3}. Participants IA1 and IA2 noted that their trust varied based on the ``impact severity'' or the potential negative consequences of disseminating incorrect information. Nonetheless, a majority (eight out of ten) concurred that the information provided was accurate and reliable, a positive answer to {\textbf{RQ3}}. Upon probing, IA3 contrasted this with traditional military intelligence, which is often unquestioned.

%Put another way, the users' trust in the system was deemed context-sensitive, harkening back to issues raised in {\textbf{KF2}} and {\textbf{DS3}}. Participants (IA1 and IA2) suggested that their willingness to trust the system's output, depended on "impact severity", i.e., the negative consequences to providing incorrect information in their situation report to the downstream reader. However, in contrast to the previous statement, 8 out of 10 participants agreed that the information provided by the report was accurate and reliable, a positive answer to {\textbf{RQ3}}. Due to IA3's feedback on the prior question, we requested further elaboration: 
\begin{quote}
``Traditional intel within the military is a black box. You're given information, and you're told to blindly trust, rely and not question it. If we're talking about the accuracy and reliability of information, SmartBook at least reassures me of where it all came from, and shows me I could grow to trust it.'' (IA3)
\end{quote}

\subsubsection{Perceived Benefits for Intelligence Analysts}
The primary benefit identified was the substantial reduction in time and effort required for compiling and analyzing complex data. SmartBook's automation of initial report generation processes, such as data collection and summarization, was highly valued by analysts as it allowed more focus on thorough analysis and strategic planning. As seen in Figure \ref{fig:userstudy}, all participants agreed on the tool's utility in assisting intelligence analysts in creating situation reports, expressing satisfaction with the system-generated reports, supporting a positive response to \textbf{RQ1}.

%The most significant perceived benefit was the expected, dramatic reduction in time and effort otherwise required to compile and analyze complex data sets. Analysts appreciated how SmartBook automated the initial stages of report generation, such as data collection and summarization, allowing them to dedicate more time to in-depth analysis and strategic thinking. All 10 participants agreed to a degree (\ref{sec:usability_study}, that the tool could be useful for intelligence analysts to create situation reports and were satisfied with the report generated by the system, providing us with another positive response to {\textbf{RQ1}}. 

Regarding the necessity of significant edits to the system-generated reports, participants IA5-IA9 suggested that this need was not inherently due to report deficiencies but varied based on the report's intended purpose or audience. We later conduct a study, described later in \S{\ref{sec:editing_study}}), to understand the extent of edits needed.

%When prompted with their varying perspectives on whether it was necessary to make major edits to the system-generated situation reports, participants (IA5-IA9) noted this wasn't due to deficiencies within the reports per se, rather this would be dependent on the "intended purpose or audience". To investigate the objective basis for these varying claims about the need to edit, we also completed an editing study (Section 8.3).

%\revanth{Needs a bit of expansion and tt reads too generic currently. IMO we can't claim we did study with decision makers without giving thorough/specific details. Might need to add one more point here.}
\subsubsection{SmartBook as a Learning Tool for Decision-Makers}
Decision-makers highly valued SmartBook's ability to rapidly deliver accurate and easily digestible information. Their primary engagement with the interface centered around utilizing the summaries, to learn about different topics in various degrees. The system's effectiveness in simplifying complex data into structured, clear formats was particularly appreciated, to help aid in swift understanding and decision-making processes.
%For decision makers, the capability of SmartBook to provide rapid, accurate, and easily digestible information was considered invaluable. Rather than exploring different components of the SmartBook interface, they focused in on the summaries, learning about different topics in various degrees. They particularly valued the system's capacity to distill complex data into concise, structured formats that would facilitate their own quick understanding and decision-making. One of the decision makers remarked:

\begin{quote}
''When something happens, I'm expected to know about it. I want to learn as fast as I can, and get the information I only need.'' (DM2)
\end{quote}

Although data lineage and source transparency were recognized as important, these were considered secondary to the primary need for timely and format-specific information delivery. While SmartBook's target users are intelligence analysts, the decision-makers' view highlights SmartBook's dual functionality as both an analytical tool and a decision-support system, providing a key capability for the high-paced and information-intensive needs of decision-makers, addressing \textbf{RQ4}.

%While aspects like data lineage and source transparency were important, these were considered secondary to the primary need for timely and format-specific information delivery. While SmartBook's target users are intelligence analysts, the decision makers' view highlights SmartBook's dual functionality as both an analytical tool and a decision-support system, providing a key capability for the high-paced and information-intensive needs of decision makers, addressing {\textbf{RQ4}}.

