% \section{Problem formulation}
\section{SmartBook System}\label{sec:smartbook}

\begin{figure}[t]
% \def\w{0.55\linewidth}
\centering
% \begin{tabular}{*3c}
% \includegraphics[height=\w, trim=35 30 35 10]{Web.pdf}
\includegraphics[width=1\linewidth]{tables/smartbook.jpg}
% \end{tabular}
\caption{A screengrab of SmartBook's front-end interface. Within the given situation, the user can navigate timelines (F1), explore strategic questions related to an event (F2), read the overarching summary on a given strategic question (F3), control the depth and length of information (F4), investigate all the claims in the summary (F5), correlate each claim to corresponding summary fragment (F6), investigate the source metadata (F7) and read the context in which the claims were extracted (F8).}
\label{fig:system1}
\end{figure}
%\subsection{Framework}
%\revanth{Notes: This section has been considerably compressed, with details moved into the appendix as per reviewer suggestions.}
\begin{figure}
    \centering
    \includegraphics[width=0.95\linewidth]{figures/SmartBook_arch2.png}
    \vspace{-0.5em}
    \caption{\small Overall workflow for constructing \textit{SmartBook}. Given the articles corresponding to a specific timeline, the figure shows the process for obtaining the chapters, their section headings, and the corresponding section content.}
    \label{fig:overall_workflow}
    \vspace{-1em}
\end{figure}

Using the four design strategies, we developed \textit{SmartBook}, an AI-assisted system for situation report generation that provides analysts with a first-draft report to work from as they respond to time-critical information requirements on emerging events.

SmartBook consists of: 1) An intuitive user interface (shown in Figure \ref{fig:system1}) built using the design strategies from \S{\ref{sec:design_strategies}}, and 2) a back-end framework (shown in Figure \ref{fig:overall_workflow}) that, when given a collection of documents from a variety of news sources, automatically generates a situation report. The situation reports, as per analysts' expectations and recommendations, are presented with a logical structure and organized chronologically as timelines, while being grounded to factual content.

%Given a collection of documents from a variety of sources, we aim to generate a situation report that embodies characteristics such as containing salient information that is presented with a logical structure and organized chronologically as timelines, while being grounded to factual content.} 

In this section, we describe the various components within SmartBook, along with emphasizing the advantages of each aspect of SmartBook's design for users (i.e., intelligence analysts) and for recipients of the final SmartBook report (i.e., decision-makers) who both initiate information requirements and are downstream readers. For in-depth technical details on SmartBook's back-end functionality, we point the reader to Section \ref{sec:appendix} of the Appendix. 
 
%\textcolor{blue}{Figure \ref{fig:overall_workflow} gives an overall workflow diagram for our approach, with each of the steps briefly described below. 
%Further,  

\begin{enumerate}
    \item \textbf{Major Events within Timelines as Chapters}: Situation reports cover event progressions over considerably long periods. Hence, it is beneficial to organize such reports in the form of timelines (F1 in Fig. \ref{fig:system1}) informed by DS1, which enables seamless report updates~\cite{ma2023structured} with new events and helps facilitate~\cite{singh2016expedition} users tracking and understanding of situation context. Timelines aid intelligence analysts in understanding event progression and predicting future trends by organizing events chronologically and highlighting cause-and-effect relationships. For readers, especially those less familiar with the subject, timelines provide a visual guide to easily grasp the sequence and significance of events in a scenario.    
    Our automatic situation report has timelines to provide a coherent, chronological representation of event developments (DS1, DS2). Each timeline spans a duration of N weeks\footnote{While more fine-grained timelines can be considered for relatively short-term events, we consider $N=2$ for a more spread-out scenario such as the Ukraine-Russia crisis. This reflects similar biweekly reporting timelines considered by organizations such as \href{https://datafriendlyspace.org/reports/}{Data Friendly Space}.}. %, allowing for manageable capturing and focused analysis of significant newsworthy occurrences. 
    The newsworthy major events within a timeline are identified via clustering~\cite{jain1988algorithms} news articles, to serve as the foundation for corresponding chapters within SmartBook. A key to further improving interpretability is deriving a short chapter name for the event, which we achieve by generating a short headline using multi-document summarization~\cite{headline2020}. %to facilitate information readability and retrieval for the timeline within the situation report. 
    Moreover, with the generated chapter name, we query Google News to obtain an expanded set of news articles relevant to the event.\\    
    
    \item \textbf{Strategic Questions as Section Headings}: Guided by DS2, a situation report should be logically structured with descriptive chapter headings and section titles for ease of understanding and information access.
    (F2 in Fig. \ref{fig:system1}). Structured reports benefit intelligence analysts by simplifying complex situation analysis, offering easy navigation, clear information hierarchy, and improved context understanding. For less experienced readers or those seeking specific information, the logical structure enhances information retrieval, and comprehension, while providing an intuitive mental map, making reports more reader-friendly.     
    Beyond simply describing event details in each major event chapter, SmartBook aims to provide information from a strategic perspective that can help aid decision-making and policy planning. To guide such detailed chapter analysis, we incorporate a logical structure through automatically generating section headings in the form of strategic questions relating to each major event. These question cover detail such as the possible motivations of the actors in an event and the future implications of the event. The questions are generated by prompting LLMs~\cite{brown2020language} with a grounded context in the form of news articles from the event cluster. Further, we promote diversity in the questions by using sampling strategies~\cite{holtzman2019curious} during generation, and then automatically de-duplicate~\cite{Chen2017QuoraQP} to obtain a set of unique strategic questions about the event. 
    
    \item \textbf{Extraction of Claims and Hypotheses:} 
    Automated situation report generation should be able to identify and extract the most scenario-relevant and crucial information across multiple documents (F5 in Fig. \ref{fig:system1}). Intelligence analysts, given high stakes nature of their work, but their limited time, need systems that quickly identify key information in documents (DS2). This enables them to focus on urgent matters without sorting through irrelevant data. Readers of the situation report benefit from information salience because they are presented with a concise, relevant overview of a situation. Essential points are highlighted, enhancing accessibility and clarity of the report's implications. Moreover, we present the bias of each news source\footnote{We collect information from \href{https://www.allsides.com/media-bias/ratings}{AllSides Media Bias} ratings.}, to help analysts consider information presented from different perspectives.
    Providing readers with a comprehensive understanding of event context requires foraging for different claims and hypotheses from the source documents (i.e., news articles) that help explain a situation \cite{toniolo2023human}. We adopt a Question Answering (QA) formulation to identify claims relevant to a given strategic question, motivated by recent studies \cite{reddy2022newsclaims, reddy2022zero} which have shown that directed queries can automatically extract claims from news articles relevant to a specific topic.\\
        
    \item \textbf{Grounded Summaries as Section Content}: 
    To be reliable, a situation report must be grounded in verifiable sources, as grounded factual content helps to build credibility (DS3). Reliable information is crucial for analysts, enabling them to verify facts from sources and draw solid, evidence-based conclusions. This reduces the need for additional fact-checking, as evidence is directly linked. Moreover, for those readers wanting to delve deeper or explore related topics, the embedded links act as a springboard to more extensive research (F6, F7, F8 in Fig. \ref{fig:system1}). Further, we present summaries with three levels of detail (less detailed, normal and more detailed corresponding to 2-3 sentences, 4-6 sentences and 2 paragraphs respectively) (F4) to allow readers to customize the level of detail at which they prefer to consume content, building from DS4.
    The individual sections comprise the core content of our situation report, and contain grounded, query-focused summaries which address the strategic questions. The extracted claims are combined into a concise summary by using LLMs with a novel prompting mechanism, which also generates citations linking each summary fragment to its factual input claim. This allows analysts to verify the summary information against input sources as needed.
\end{enumerate}
%
Ultimately, the structured incorporation of timeline chunking, major event chapter clustering, strategic question headings, claim extraction and query-focused section summaries within chapters enables our SmartBook system to generate reliable, insightful reports to assist analysts in responding to information requirements about time-sensitive, emerging situations. 



