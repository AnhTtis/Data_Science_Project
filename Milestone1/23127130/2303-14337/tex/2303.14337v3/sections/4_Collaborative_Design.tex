\section{Collaborative Design}
\label{sec:collaborative_design}

%\heng{Echo Revanth's comments - it's important to align section 5's components back with these design principles, e.g., how do the innovative components such as question generation and claim detection align with these principles? You can also point to section 5's subsections here in section 4.}

To gain a better understanding of the composition process of situation reports, we expanded the design opportunities identified in the formative study with subsequent collaborative design (CD) sessions with 10 participants (IA1-IA10). The goal was to capture tangible design strategies and recommendations from users about how they, as intelligence analysts, navigate, research, and author their situation reports.

\subsection{Process}
\subsubsection{Participants}
To enable us to build on the insights and findings of the formative study, the original participants (IA1-IA10) were sustained in the collaborative design phase.

\subsubsection{Procedure}
Each study session consisted of three distinct components: a Workflow Review using Storyboards, a hands-on Training Task, and a Simulation Report exercise. These sessions, conducted virtually, spanned approximately one hour each, with participants receiving compensation consistent with the rates in our formative study. 

%\heng{Did you also ask them to compare with gpt+bing? maybe good to add for camera-ready}

\begin{enumerate}
    \item \textbf{Workflow Review with Storyboards}
    \begin{enumerate}
        \item \textbf{Introduction to Storyboards:} In our study, participants engaged with a low-fidelity storyboard (shown in Figure \ref{fig:storyboard}), where each panel depicted a distinct phase in situation report creation. The storyboard, designed based on insights from our formative study, aimed to establish a comprehensive understanding of the workflow for generating situation reports. The construction of the storyboard was guided by detailed information on situation reports (in \S{\ref{sec:situation_report_background}}), whereas the key findings (in \S{\ref{sec:formative_findings}}) shaped the formulation of probing questions presented during participant interaction with the storyboard.
        \item \textbf{Annotation and Brainstorming:} Participants were tasked with providing detailed descriptions of each storyboard panel to ensure comprehension of the depicted scenario and workflow. Additionally, they annotated the storyboards for subsequent analysis, establishing a basis for identifying user needs and interaction strategies. This phase also included a semi-structured interview, with questions derived from the key findings of the formative study. (e.g. KF1: In which of these steps, would you be willing to use technology? How and why would you use it?) 
        \end{enumerate}
    \item \textbf{Training Task}
    \begin{enumerate}
        \item \textbf{Familiarization Exercise:} Participants simulated each storyboard step using sample situations to gain practical workflow experience. Two distinct question sets, varying in scope and complexity, were assigned to IA1-IA5 and IA6-IA10.          
        \item \textbf{Tool Utilization:} Candidates were advised to utilize diverse resources, including web searches like Google and Bing, and Large Language Models such as ChatGPT, for task completion.     
        \end{enumerate}
    \item \textbf{Simulation Report}
    \begin{enumerate}
        \item  \textbf{Situation Handling:} Participants were each presented with a unique situation and tasked (as above, in the training task) with facilitating the end-to-end process of creating a complete situation report.
        \item \textbf{Process Documentation and Ideation:} Throughout this simulation, participants documented their rationale for tool selection, challenges encountered, and potential system improvements. This feedback was critical to defining the Design Strategies (in \S{\ref{sec:design_strategies}}).
    \end{enumerate}
\end{enumerate}

\begin{figure}[t]
    \centering
    \includegraphics[width=1.0\textwidth]{tables/storyboard.jpg}
    %\vspace{-1em}
    \caption{Storyboard used in the collaborative design sessions with intelligence analysts.}
    \label{fig:storyboard}
\end{figure}

\subsection{Results}
\label{sec:collaborative_results}
In analyzing the data collected during the collaborative design sessions, we found three clear themes emerged.

\label{sec:formative_results}
\subsubsection{Enhancing analytical efficiency and reducing cognitive load} 
\label{sec:4_2_1}
Participants expressed a need for user interfaces that closely mirror their mental models of data analysis and report generation. Specifically, one analyst described their ideal tool as, ``A system that reflects our thought processes, almost as if the tool 'thinks' like an analyst.'' (IA1). Such feedback indicates a preference for interfaces that are instinctive and reduce the need for deliberate navigation, thereby enabling analysts to concentrate on the strategic aspects of their work. Moreover, analysts (IA1-IA5, IA7) underscored the significance of customizing data presentation in the interface to efficiently distill valuable insights from raw data, thereby decreasing the time and effort needed for data analysis.

\subsubsection{Trust and reliability in AI-systems} 
\label{sec:4_2_2}
Here the focus was on building trust in how the system operates, through its interpretability and transparency. Participants stressed the importance of understanding the automated system's underlying logic and methodology. Analysts called for explicit clarity in the system's data processing and analysis methods. Specifically, one analyst stated, ``For me to trust the system, I need to understand how and what it does. I want to know the brains behind it.'' (IA3). 

Transparency in data sources and information flow was another key aspect highlighted by the participants, emphasizing the importance of traceability from report contents to original sources. This traceability enhances the credibility and trustworthiness of the automated system. As IA9 noted, "Understanding where the information is coming from is so so important. It's knowing that the information didn't just come from nowhere." (IA9).

The sessions indicated a marked preference for outputs that enhance user comprehension and verification of the system's conclusions. Analysts favored reports offering summaries with annotations or data source references, allowing them to corroborate the system's findings against their own expertise. This method promotes an interplay between building trust with personal verification. One participant emphasized the need for transparency, requesting, ``Explain to me how you came up with the answer.'' (IA2). 

\subsubsection{Customization and flexibility with automated tools.}
\label{sec:4_2_3}
Participants desired a tool that not only accommodates different analytical styles but also various levels of detail and complexity in reporting. One analyst (IA1) highlighted the necessity for an interface that can dynamically alternate between in-depth data analysis and high-level overviews, depending on the intended reason. Additionally, there was a notable interest in a platform that allows users to specify the sources of data, such as news reports, social media feeds, and official records. This would enable the aggregation and analysis of information from these diverse origins in a unified manner, providing a more comprehensive perspective on the subject matter.

\subsection{Design Strategies}
\label{sec:design_strategies}
From the findings of the formative study (in \S{\ref{sec:formative_study}}) and the results of the collaborative design above, we identified the following design strategies:

\begin{itemize}
    \item \textbf{DS1:} Given the emphasis on reducing cognitive load and enhancing analytical efficiency (KF1 and \S{\ref{sec:4_2_1}}), the system will be designed with an interface, that mirrors intelligence analysts' natural processes of data analysis and report generation.
    \item \textbf{DS2:} To increase efficiency (\S{\ref{sec:4_2_1}}), the system will integrate features to automate time-intensive tasks such as question curation and preliminary research, thereby reducing analysts' manual workload and enabling greater focus on strategic analysis and decision-making.
    \item \textbf{DS3:} The design, addressing the need for trust and reliability (KF2 and \S{\ref{sec:4_2_2}}), will convey clear explanations of the system’s data processing algorithms and criteria. This includes transparent data sourcing, providing references within reports, and tools for users to easily understand and verify the system's conclusions. The design will also facilitate incremental trust-building through consistent and validated performance over time.
    \item \textbf{DS4:} Addressing the themes of customization and flexibility (KF3 and \S{\ref{sec:4_2_3}}), the system will offer a high degree of adaptability to accommodate various analytical styles and levels of detail in reporting. It will include features for adjusting the depth of analysis, focusing on specific data sets, and seamlessly integrating various data sources.
    %\item \textbf{DS5:} \textcolor{blue}{The system will be equipped with transparent mechanisms that elucidate the processes of data handling and summary creation (\S{\ref{sec:4_2_1}}). By making these processes clear and understandable, the design aims to deepen user comprehension of the automated functions, thus enhancing confidence and trust in the system's outputs.}
\end{itemize}








