\section{Introduction}

Streaming speech translation (ST) systems generate real-time translations by incrementally processing audio frames, unlike their offline counterparts that have access to complete utterances before translating. 
Typically, streaming ST models use uni-directional encoders \citep{zhang2019lattice,ren-etal-2020-simulspeech,ma-etal-2020-simulmt,zeng-etal-2021-realtrans,zhang2023simple} and are trained with a read/write policy that determines whether to wait for more speech frames or emit target tokens. 
However, it can be expensive to maintain multiple models to satisfy different latency requirements~\citep{zhang-feng-2021-universal,liu2021cross} in real-world applications. 
Recently, some works \citep{papi-etal-2022-simultaneous,dong-etal-2022-learning} have shown that a single offline model with bidirectional encoders (such as Wav2Vec2.0 \cite{NEURIPS2020_92d1e1eb}) can be adapted to streaming scenarios with a \textit{wait-$k$} policy \citep{ma-etal-2019-stacl} to meet different latency requirements and achieve comparable or better performance. 
However, there is an inherent mismatch in using a model bidirectionally trained with complete utterances on incomplete streaming speech during online inference. 

\section{Threat Model and Advantages of Our Hardware-based Adversarial Detector} \label{sec: motivation}
\ry{In this part, I want to highlight the comparison between hardware and software attacks}
%Normally, software-based adversarial detectors are easier to implement, cheaper to develop and more well-studied than those based on hardware computational signals.
% We would like to stress that our goal for investigating hardware-based adversarial detectors is not to achieve better performance in detection than the conventional white-box software based methods.  
\subsection{Threat Model} \label{sec: threat model}
\ry{This section is threat model: attack is `white-box', detector is `black-box'}
The victim is a DNN classifier, which is pre-trained with a public dataset. The testing dataset may be kept private.
We assume the strongest `white-box' attack model, where the attacker has full knowledge of the victim model and training dataset in order to generate adversarial samples with minimum perturbations. 
On the contrary, the detection system assumes the most limited scenario, under a `black-box' view of the victim, without access to the victim's inputs, parameters, and intermediate outputs or execution details. 
The only information available to the detector to distinguish adversarial samples is the EM side-channel measurement and the victim model's prediction class.
For training the adversarial detector with EM traces, a public benign dataset is used. 

\if false 
\ry{In this part, we discuss more settings of the detector especially the data used in two phases.}
In general, the detecting process can be summed up into two phases, training phase and detecting phase.
To begin with, we train an Out-of-Distribution(OOD) detector on a public benign dataset of the same classification task, which should be distinct from the victim's training dataset.
For each query, the detector will obtain the classification result and an EM trace along with the model execution to fit its EM classifiers and anomaly detectors.  
During the detection phase, the victim model is in operation and under attack when the pre-trained detector decides whether the current input is adversarial or not, only based on the victim model output and its EM trace.
\fi 

\subsection{Advantages}
Compared to software-based adversarial detection methods, our hardware-based detector, EMShepherd, has three distinct advantages: privacy-preserving, portability, and robustness.

\begin{itemize}[leftmargin=*]
    \item \ry{Add a new motivation here. The motivation is that using \name can help the user protect their privacy.} 
    \name protects the DNN model user's data privacy as it is agnostic to the model's inputs, which instead are always required by prior reconstruction-based detection methods~\cite{meng2017magnet, yang2022you}. 
    %Most model users are benign whose inputs may be sensitive and should not be shared with \textit{third-party detectors}. 
    The sensitive inputs should not be shared with \textit{third-party detectors}. 
    Our design only requires the output class labels and the EM signals, which are passively leaked to common acquisition equipment. 
    %    Our design is suitable for such cases as it only requires the EM signals and the inference outputs during the model execution. Generally speaking, EM signals and labels have less private information leakage.
    \item \ry{The second motivation is still related to privacy. This time we consider model privacy when the model structure or parameters should be kept private.}
   \name also protects the model confidentiality.  No model information, including %Using hardware-based detectors can prevent the third-party defender from accessing some confidential model information such as  
   hyper-parameters, parameters, and logits, is needed, in stark contrast to the previous software-based detection methods~\cite{ma2019nic,feinman2017detecting}.
    %Our \name only acquires the EM traces during model inference in a passive and noninvasive manner, 
    The EM data processing and the adversarial detector training process are both victim model-agnostic. 
    Therefore, our method has more general usage, applicable to closed-source DNN applications, which are pervasive in edge devices where the user only queries the models for the final prediction output. 
    \item \ry{The third motivation is portability.}  
    Owing to the model-agnostic feature, EMShepherd can be easily ported for wide-range hardware devices with different DNN implementations for diverse applications. It can be used as a `plug and play' (PnP) device, aside from the target system, to work automatically without user intervention or contact with the victim system. 
    \item \ry{The last motivation is about adaptive attacks, we should propose that EM signal is hard to imitate, so it is hard for adaptive attacks to generate sample fraud both detector and victim.} 
    Adaptive attack~\cite{adaptive} is a threat to most software defense methods where the attacker adjusts the adversarial perturbations to mislead both the victim models and defense systems.
   %  The hardware-based detection method can provide a double protection on top of most software defense methods such as adversarial training.
   %  Although the adptive adversarial example fools the robust model, its computation patterns during the DNN model execution are still well kept in the EM traces and our EMShepherd framework still works well for detecting the new type of adversarial examples.  
   %  Meanwhile, due to the high complexity of EM signals and non-explicit dependency of the EM signals on computations, it is extremely hard to have an adaptive attack on our detection method, i.e., adversarial examples whose EM signals are deliberately controlled to evade the EM-based detector.
   However, due to the high complexity and non-explicit dependency of the EM signals on computations and data, 
   it is extremely hard to have an adaptive attack on our detection method, 
   i.e., adversarial examples whose EM signals are deliberately controlled to evade the EM-based detector. 
\end{itemize}







Intuitively, speech representations extracted from streaming inputs (Figure~\ref{fig:streaming}) are less informative than those from full speech encoding (Figure~\ref{fig:full}) due to limited future context, especially toward the end of the streaming inputs, which can be exacerbated by the aforementioned mismatch problem. This raises a natural question: how much do the speech representations differ between the two inference modes? We analyze the gap in speech representations, measured by cosine similarity, at different positions in the streaming input compared to using the full speech (Section \ref{sec:analysis}). We observe a significantly greater gap for representations closer to the end of a streaming segment, with an average similarity score as low as 0.2 for the last frame, and the gap quickly narrows for earlier frames (Figure~\ref{fig:analysis_on_lastpos}). Additionally, we observe more degradation in translation quality for utterances with the greatest gap in speech representations between online and offline inference (see Appendix~\ref{apd:degree}).



We conjecture that the lack of future contexts at the end of streaming inputs can be detrimental to streaming speech translation when using an offline model. 
To this end, we propose a novel Future-Aware Inference (FAI) strategy. %, illustrated in Figure~\ref{fig:method}(b). 
This approach is inspired by masked language models' ability~\cite{NEURIPS2020_92d1e1eb} to construct representations for masked tokens from their context. 
Specifically, we append a few mask embeddings to the end of the streaming input and leverage the acoustic encoder (Wav2Vec2.0)'s ability to implicitly construct representations for future contexts, which can lead to more accurate representations for the other frames in the streaming input.


Furthermore, we propose a Future-Aware Distillation (FAD) framework that adapts the offline model to extract representations from streaming inputs that more closely resemble those from full speech encoding. 
We expand the original streaming input with two types of future contexts: one with $m$ oracle speech tokens for the teacher model, and another with $m$ mask tokens for the student model, which is initialized from the teacher model. 
We minimize several distillation losses between the output of the teacher and student models. 
By incorporating additional oracle future contexts, the speech representations for the frames in the original streaming input extracted by the teacher model resemble those when the full speech is available. 
FAD aims to adjust the offline model to extract similar representations for streaming input as it would for full speech. 
In combination with FAI, we improve the model's ability to extract quality representations during both training and inference, alleviating the aforementioned mismatch problem. 
We refer to our approach as FAST, which stands for Future-Aware Streaming Translation.



We conducted experiments on the MuST-C EnDe, EnEs, and EnFr benchmarks. 
The results show that our methods outperform several strong baselines in terms of the trade-off between translation quality and latency. 
Particularly, in the lower latency range (when AL is less than 1000\emph{ms}), our approach achieved BLEU improvements of 12 in EnDE, 16 in EnEs, and 14 in EnFr over baseline. 
Extensive analyses demonstrate that our future-aware approach significantly reduces the representation gap between partial streaming encoding and full speech encoding.

