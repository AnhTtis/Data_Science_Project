

\documentclass[twocolumn]{raa}
 
\usepackage{graphicx,times}             
\usepackage{natbib}
\usepackage{amssymb,amsmath}
\usepackage{mathrsfs}
%\usepackage{ulem}
\bibpunct{(}{)}{;}{a}{}{,}
 

\usepackage[pagebackref=true]{hyperref}






\newcommand{\mpc}{{\, {\rm Mpc}}}
\newcommand{\hmpc}{{\, h^{-1}\, {\rm Mpc}}}


\def\aj{AJ}
\def\apj{ApJ}
\def\apjs{ApJS}
\def\jcap{JCAP}
\def\mnras{MNRAS}
\def\aap{A\&A}
\def\prd{Physical Review D}
\def\nat{Nature}      
\def\apjs{ApJS}
\def\apjl{ApJ Letters}
\def\physrep{Physics Reports}


\begin{document}

\title{Galaxy interactions in filaments and sheets: insights from EAGLE simulations}  
 

   \volnopage{Vol.0 (20xx) No.0, 000--000}     
   \setcounter{page}{1}           

    \author{Apashanka Das
      \inst{1}
   \and Biswajit Pandey
      \inst{2}
   \and Suman Sarkar
      \inst{3}
   }
 \institute{Department of Physics, Visva-Bharati University, Santiniketan, 
              Birbhum, 731235, India
             {\it a.das.cosmo@gmail.com}\\
        \and
	     {Department of Physics, Visva-Bharati University, Santiniketan, 
	     Birbhum, 731235, India {\it biswap@visva-bharati.ac.in}}\\
        \and
            {Department of Physics, Indian Institute of Technology, Kharagpur, 721302, India
              {\it suman2reach@gmail.com}}\\
   }
       
\vs\no
   {\small Received 20xx month day; accepted 20xx month day}


\abstract{We study the colour and star formation rates of paired
  galaxies in filaments and sheets using the EAGLE simulations. We
  find that the major pairs with pair separation $<50$ kpc are bluer
  and more star forming in filamentary environments compared to those
  hosted in sheet-like environments. This trend reverses beyond a pair
  separation of $\sim 50$ kpc. The interacting pairs with larger
  separations ($>50$ kpc) in filaments are on average redder and
  low-star forming when compared to those embedded in sheets. The
  galaxies in filaments and sheets may have different stellar mass and
  cold gas mass distributions. Using a KS test, we find that for
  paired galaxies with pair separation $<50$ kpc, there are no
  significant differences in these properties in sheets and
  filaments. The filaments transport gas towards the cluster of
  galaxies. Some earlier studies find preferential alignment of galaxy
  pairs with filament axis. Such alignment of galaxy pairs may lead to
  different gas accretion efficiency in galaxies residing in filaments
  and sheets.  We propose that the enhancement of star formation rate
  at smaller pair separation in filaments is caused by the alignment
  of galaxy pairs. A recent study with the SDSS data \citep{das23}
  reports the same findings. The confirmation of these results by the
  EAGLE simulations suggests that the hydrodynamical simulations are
  powerful theoretical tools for studying the galaxy formation and
  evolution in the cosmic web. \keywords{methods: data analysis ---
    statistical --- galaxies: interactions --- evolution ---
    cosmology: large scale structure of the universe} }

\authorrunning{A. Das, B. Pandey \& S. Sarkar}
%author_head in even pages
\titlerunning{Galaxy interactions in filaments and sheets}% title_head in odd pages
 
\maketitle

\section{Introduction}           %% first-level sections will be auto-capitalized
\label{sect:intro}

Understanding the formation and evolution of galaxies is one of the
most challenging problems in cosmology. The formation and evolution of
galaxies are expected to be influenced by both the initial conditions
at the location of their formation and their interactions with the
surrounding environment. The primordial density perturbations in the
dark matter density field grow due to gravitational instability and
eventually collapse into dark matter halos. The dark matter halos
represent the peaks in the density field. The halos are surrounded by
diffuse neutral Hydrogen distribution after the recombination. They
accrete the gas which radiate away their kinetic energy and settle
down at their centers. The cooling and condensation of the gas at the
centre of the dark matter halos are believed to be the primary
mechanism for the formation of galaxies \citep{reesostriker77, silk77,
  white78, fall80}.

The galaxies interact with other galaxies and their environments. The
galaxy-galaxy interactions are known to enhance the star formation
activity in galaxies \citep{barton00, lambas03, alonso04, nikolic04,
  alonso06, woods06, woods07, barton07, ellison08, heiderman09,
  knapen09, robaina09, ellison10, woods10, patton11}. The environment
of a galaxy plays a decisive role in its evolution. The galaxies in
high density regions are redder and have lower star formation rates
\citep{lewis02,gomez03,kauffmann04}. The suppression of star formation
in high density regions may occur due to different physical
mechanisms. Some of the known physical mechanisms are ram pressure
stripping \citep{gunn72}, galaxy harassment \citep{moore96, moore98},
strangulation \citep{gunn72, balogh00}, starvation \citep{larson80,
  somerville99, kawata08} and gas loss through starburst, AGN or
shock-driven winds \citep{cox04, murray05, springel05}. A number of
other physical processes such as mass quenching \citep{birnboim03,
  dekel06, keres05, gabor10}, morphological quenching
\citep{martig09}, bar quenching \citep{masters10} and angular momentum
quenching \citep{peng20} can also halt star formation in galaxies.


The environmental dependence of galaxy properties remains an active
area of research for the last few decades. The galaxy properties are
know to strongly depend on environment \citep{oemler74, davis76,
  dress80, guzo97, zevi02, hog03, blan03, einas03, gotto03,
  kauffmann04, pandey06, park07, mocine07, pandey08, porter08,
  bamford09, cooper10, koyama13, pandey17, sarkar20, bhattacharjee20,
  pandey20}. Most of the studies on the environmental dependence of
galaxy properties treat the local density as a proxy for the
environment. The local density undoubtedly plays a decisive role in
galaxy evolution. However, the local density alone can not
characterize the environment of a galaxy. The observations from
different redshift surveys
\citep{gregory78,joeveer78,einasto80,zeldovich82,einasto84,
  bharadwaj00, pandey05} indicate that galaxies are distributed in an
interconnected network-like pattern of clusters, filaments and sheets
surrounded by vast empty regions. This network is often referred to as
the ``cosmic web'' \citep{bond96}. The galaxies and their host halos
are embedded in different geometric environments of the cosmic web.
\citet{pandey08} find that the star forming blue galaxies have a more
filamentary distribution compared to the red galaxies. A number of
studies with hydrodynamical simulations suggest that $80\%$ of the
baryonic budget in the Universe is accounted by low density gas (WHIM)
in filaments \citep{tuominen21, galarraga21}. Consequently, the gas
accretion efficiency in the dark matter halos in filaments and sheets
may differ in a significant manner. Thus, cosmic web can have
significant impact on the properties and evolution of galaxies. The
galaxies that are located in different parts of the cosmic web can
experience different physical conditions, such as different densities
of gas, different levels of tidal forces, different frequency of
interactions and mergers. All these factors may play important roles
in shaping the properties and evolution of galaxies.

The interactions between galaxies with comparable masses are known as
the major interactions. Such interactions are known to trigger new
star formation in galaxies. The interacting pairs are expected to be
present in different geometric environments of the cosmic web. They
are more frequently observed in the denser regions of the cosmic
web. The filaments and sheets, being the denser parts of the cosmic
web, host a significant number of major galaxy pairs. In a recent
work, \citet{das23} analyze the SDSS data to compare the star
formation rate and colour of major pairs hosted in filaments and
sheets. They find that the major galaxy pairs with separation $<50$
kpc are relatively high star forming and bluer when hosted in
filaments. On the other hand, the major pairs with separation $>50$
kpc show a significantly higher SFR and colour in sheet-like
environments.  This behaviour may be related to the preferential
alignment of galaxy pairs with the filament axis \citep{tempel15,
  mesa18}. The available gas mass is a primary regulator of the star
formation in a galaxy. The gas mass is modulated by the inflows and
outflows \citep{dekel09, dave11, dave12, lilly13}. The transient
events like interactions and mergers can drive the galaxies out of
equilibrium. The alignment of galaxy pairs with filament spines may
lead to anisotropic accretion and higher gas accretion efficiency in
these galaxies. The primary aim of this work is to verify these
findings with hydrodynamical simulation.

The EAGLE simulation \citep{eagle16} is a hydrodynamical simulation
that studies the galaxy formation and evolution in a cosmological
volume. It describes the formation of large-scale structures as well
as the formation of galaxies by gas falling into the dark matter halos
and their subsequent cooling and condensation. It would be interesting
to study the SFR and colour of the major pairs in filaments and sheets
using EAGLE simulations. In observations, the galaxy pairs are usually
identified by employing simultaneous cuts on the projected separation
and the rest frame velocity difference of the galaxies. However, all
these pairs may not be undergoing interactions. Some of the pairs
identified in observations may not be close in three dimensions due to
the chance superposition in the high-density regions like groups and
clusters \citep{alonso04}. Also, we can not construct a mock catalogue
for the observational sample of galaxy pairs used in \citet{das23} due
to the smaller volume of the EAGLE simulations. So we decided to use
the real-space positions of galaxies available in simulation to
identify the major pairs. This would avoid any errors in
identification of galaxy pairs due to the projection effects. We
identify the geometric environments of galaxy pairs using the local
dimension \citep{sarkar09}. Our primary aim of this work is to study
the interaction induced star formation in filaments and sheets using
EAGLE simulations. This would help us to asses the roles of the
large-scale structures like filaments and sheets in galaxy evolution.

We organize the paper as follows: we describe the data and the method
of analysis in Section 2 and present the results and conclusions in
Section 3.

\section{Data and Method of Analysis}           

\subsection{EAGLE simulation data}
\label{sec:data}

The EAGLE simulation \citep{eagle16} is a set of cosmological
hydrodynamical simulation in periodic, cubic comoving volumes ranging
from side of length 25 to 100 megaparsec that track the evolution of
both baryonic (gas, stars and massive black holes) and non-baryonic
(dark matter) elements from a starting redshift of $z=127$ to the
present day. This simulation adopts a flat $\Lambda CDM$ cosmology
with parameters taken from the $Planck$ mission results
\citep{planck14}. The values of cosmological parameters used in this
simulation are $\Omega_{\Lambda}=0.693$, $\Omega_m=0.307$,
$\Omega_b=0.04825$ and $H_0=67.77$ km/s/Mpc, where symbols have
their usual meanings.

We download the various properties of galaxies from the publicly
available EAGLE run simulation using $wget$ command. We extract the
information of position of the centre of mass of galaxies in three
dimension within a comoving cubic volume of size $100$ $Mpc^3$ from
$Ref-L0100N1504\_Subhalo$ table. We consider the last snapshot of the
simulation having $Snapnum=28$ which corresponds to redshift $z=0$. We
select only those galaxies which are flagged as $Spurious=0$. This
ensures that we select only the genuine simulated galaxies by
discarding all the unusual objects with very small stellar mass and
anomalously high metallicity or black hole mass. We also download the
star formation rate, stellar mass and cold gas mass of simulated
galaxies using $Ref-L0100N1504\_Aperture$ table. These are estimated
within a spherical $3D$ aperture of radius 30 physical kpc centered at
the location of minimum gravitational potential of a galaxy. Use of
this criteria gives well suited stellar mass and star formation
estimates as compared to observational results and is also recommended
for use by the EAGLE simulation team \citep{eagle16}.  We also
consider only those galaxies with their stellar mass $>0$. Combining
the two tables with $GalaxyID$ , we obtain all of the above mentioned
information for 325358 galaxies.  We also extract the non-dust
attenuated rest frame broad band magnitudes of galaxies estimated in
$u$ and $r$ band filters \citep{doi10} from
$Ref-L0100N1504\_Magnitude$ table, where $u$ and $r$ respectively
denote Ultraviolet and Red filter bands of Sloan Digital Sky Survey
(SDSS). We combine this table with $Ref-L0100N1504\_Subhalo$ and
$Ref-L0100N1504\_Aperture$ table using $GalaxyID$ to get all the
required information. The magnitude of galaxies in different SDSS
filters have been computed in 30 physical kpc spherical apertures
following the procedure described in \cite{trayford15}. Finally, we
have all the information for 29754 galaxies. For the rest of the
analysis, we refer to $u-r$ colour of galaxies as the difference of
its rest frame non-dust attenuated absolute magnitudes in $u$ and $r$
band respectively. Only the magnitudes of the galaxies with stellar
mass $log(M_{stellar}/M_{sun}) > 10^{8.5}$ are provided in
$Ref-L0100N1504\_Subhalo$ table.  However, here we use the stellar
mass estimates of galaxies from $Ref-L0100N1504\_Aperture$ table where
the minimum stellar mass of a galaxy is $log(M_{stellar}/M_{sun}) \sim
10^{8.2}$.  Observations show that the galaxies with stellar mass
$M_{stellar}<3 \times 10^{10}\,M_{sun}$ are actively star forming and
the galaxies having stellar masses above this critical value are
generally quiescent systems \citep{kauffmann03}. For the present
analysis, we consider only those galaxies which have their stellar
mass in between $8.5 \leq log(M_{stellar}/M_{sun}) \leq 10.5$. Our
mass limited sample contains a total 21305 galaxies.

We identify the nearest neighbour in three dimensions for each galaxy
in our mass limited sample. The distance to the nearest neighbour for
each galaxy is denoted by $r$, where $r$ represents the three
dimensional distance between the centre of mass of the
galaxies. Initially, we label each galaxy and its nearest neighbour in
our sample as a possible pair. We then select only those pairs for
which $r \leq 200$ kpc. We also apply a cut on their stellar mass
ratio $1 \leq \frac{M_{1}}{M_{2}} < 3$ to include only the major pairs
in our analysis. This provides us with a total 2264 major pairs.  The
smallest pair separation in our sample is $\sim 6$ kpc.

 We quantify the geometric environment of the galaxies in the EAGLE
 simulation by estimating their local dimension on a length scale of
 $10$ Mpc (\autoref{subsec:ldim}). We use $GalaxyID$ to cross match
 these galaxies with our pair sample. The cross-maching yields a total
 2537 galaxies in major pairs. We find that 373 and 276 out of these
 galaxies are residing in filaments and sheets respectively. It may be
 noted that the local dimension can not be determined for all the
 galaxies in the simulation.


\begin{table}
\centering
\begin{tabular}{|c|c|}
\hline
Local dimension & Geometric environment\\
\hline
$0.75 \leq D < 1.25$ & $D1$\\
$1.25 \leq D < 1.75$ & $D1.5$\\
$1.75 \leq D < 2.25$ & $D2$\\
$2.25 \leq D < 2.75$ & $D2.5$\\
$D \geq 2.75$ & $D3$\\
\hline 

\end{tabular}
\caption{This table shows range of local dimension values $D$ and the
  associated geometric environment of galaxies.}
\label{tabld}
\end{table}

\subsection{Morphology of the local environment}
\label{subsec:ldim}
We characterize the different geometric environments of the cosmic web
using the local dimension \citep{sarkar09}. The local dimension is a
simple measure based on the number counts of galaxies within spheres
of different radii centered on a galaxy. The number counts of
galaxies within a sphere of radius $R$ centered on a galaxy can be
written as,
\begin{equation}
N(< R) = A\,R^D
\label{ld}
\end{equation}
where $A$ is a constant and $D$ is the local dimension. The number
counts $N(< R)$ would scale differently with the radius $R$ depending
on the local geometry of the embedding environment. We vary the radius
of the measuring sphere between $R_1 \mpc \leq R \leq R_2 \mpc$ and
consider only those galaxies for which at least 10 galaxies are
available within this range. We fit the observed number counts $N(<R)$
to \autoref{ld} and determine the best fit values of $A$ and $D$ using
a least-square fitting. The goodness of each fit is determined by
estimating the $\chi^2$ per degree of freedom. Only the fits with
chi-square per degree of freedom $\frac{\chi^2}{\nu} \leq 0.5$ are
retained for our analysis \citep{sarkar19}. We choose $R_1=2 \mpc$ and
$R_2=10 \mpc$ for the analysis presented here. 

The local dimension $D$ describes the morphology of the embedding
environment. Ideally, a filamentary environment should have $D=1$ and
sheetlike environment should have $D=2$. A homogeneous distribution in
three-dimension is represented by $D=3$. However, the filaments,
sheets, clusters and voids are not idealized structures and they can
have a wide variety of shapes and sizes. We assign a finite range of
local dimension to each type of morphological environment as shown in
\autoref{tabld}. We determine the geometry of the local environment of
a galaxy based on these definitions. Accordingly, the $D1$-type
galaxies reside in one dimensional nearly straight filament. A
$D2$-type galaxy is residing in a two-dimensional sheet-like
structure. A $D3$-type galaxy is embedded in a three dimensional
distribution of homogeneous nature. The galaxies can also reside near
the junction of different types of morphological environments. $D1.5$
and $D2.5$ can be treated as intermediate environments.


\begin{figure*}
\resizebox{14.4cm}{6cm}{\rotatebox{0}{\includegraphics{ldim.ps}}}
\caption{The plot in the left and right panel respectively shows
  cumulative mean $u-r$ colour and SFR of major pairs as a function of
  pair separation $r$. The $1\sigma$ error bars at each data point
  shown are obtained using 10 Jackknife samples.}
\label{Fig1}
\vspace{0.2cm}
\end{figure*}


\begin{figure*}
\resizebox{14.4cm}{6cm}{\rotatebox{0}{\includegraphics{pdf_all.ps}}}
\caption{The plot in the left and right panel respectively shows
  probability distribution function of $log(M_{stellar}/M_{sun})$ and
  $Mass_{cold gas}$ of major paired galaxies residing in $D1$ and $D2$
  type environments.}
\label{Fig2}
\end{figure*}

%\begin{figure*}
%\resizebox{14.4cm}{6cm}{\rotatebox{0}{\includegraphics{pdf_low.ps}}}
%\caption{Same as \autoref{Fig2} , but for major pairs having separation $ r < 50$ kpc.}
%\vspace{0.2cm}
%\label{Fig3}
%\end{figure*}


%\begin{figure*}
%\resizebox{14.4cm}{6cm}{\rotatebox{0}{\includegraphics{pdf_high.ps}}}
%\caption{Same as \autoref{Fig2} , but for major pairs having separation $ r \ge 50$ kpc.}
%\label{Fig4}
%\end{figure*}


%\begin{figure*}
%\resizebox{14.4cm}{6cm}{\rotatebox{0}{\includegraphics{clr_min.ps}}}
%\caption{Same as \autoref{Fig1} , but for minor pairs ($3 \leq M_{1}/%M_{2} \leq 10$).}
%\label{Fig4}
%\end{figure*}


\section{Results and Conclusions}

We show the cumulative mean of the $u-r$ colour for the major pairs
residing in sheets and filaments as a function of three-dimensional
separation in the left panel of \autoref{Fig1}. This shows that the
major pairs with pair separation $r<50$ kpc are on average bluer in
filamentary environment compared to those residing in sheet-like
environment. However, this trend only persists up to a pair separation
of $\sim 50$ kpc. A cross over of the two curves corresponding to $D1$
and $D2$ type environments is observed at $r \sim 50$ kpc. The major
pairs with pair separation $r>50$ kpc are significantly redder in
filaments compared to those located in sheets. We also analyze the SFR
in major pairs residing in filaments and sheets and show the results
in the right panel of \autoref{Fig1}. We find that the major pairs at
closer pair separation ($<50$ kpc) are more star forming in filaments
compared to those in sheets. We see an exactly opposite trend for the
major pairs with larger pair separation ($>50$ kpc). The colour and
SFR are correlated properties \cite{strateva01, baldry04, pandey20a}
and the results shown in the two panels of \autoref{Fig1} are
consistent with each other. It is also interesting to note that the
crossover is observed at nearly the same pair separation ($\sim 50$
kpc) for both colour and SFR.


\begin{table*}
\centering
\begin{tabular}{|c|c|c|c|c|c|c|c|}
\hline 
& \multicolumn{2}{c|}{$D_{KS}$} & \multicolumn{5}{c|}{$D_{KS}(\alpha)$}\\
\cline{2-8}
Major pairs & $log(M_{stellar}/M_{sun})$ & $Mass_{cold gas}$ & 99\% & 90\% & 80\% & 70\% & 60\%\\
\hline
All & 0.1037 & 0.5105 & 0.1292 & 0.0972 & 0.0852 & 0.0773 & 0.0712 \\
$ r < 50$ kpc & 0.1032 & 0.3512 & 0.3754 & 0.2826 & 0.2478 & 0.2249 & 0.2072\\
$ r \ge 50$ kpc & 0.1055 & 0.5104 & 0.1388 & 0.1043 & 0.0915 & 0.0830 & 0.0765\\
\hline
\end{tabular}
\caption{This table shows Kolmogorov-Smirnov statistic $D_{KS}$ for
  comparison of $log(M_{stellar}/M_{sun})$ and $Mass_{cold gas}$ of
  major pairs residing in $D1$ and $D2$ type environments. The
  comparison is also done for major pairs having $r < 50$ kpc and $r
  \ge 50$ kpc. The table also shows the critical values
  $D_{KS}(\alpha)$ above which null hypothesis can be rejected at
  different confidence levels.}
\label{tab:ks}
\end{table*}


It is well known that the stellar mass \citep{kauffmann03, birnboim03,
  dekel06, keres05, gabor10} and the available cold gas mass content
\citep{dave12, lilly13, saintonge12, violino18, thorp22} also play a
very important role in deciding the star formation in a galaxy.  We
test if the differences occurring in $u-r$ colour and SFR of galaxies
in major pairs residing in $D1$ and $D2$ type environments arise due
to the differences in their stellar mass and cold gas content. We use
a Kolmogorov-Smirnov (KS) test to compare the distributions of stellar
mass and cold gas mass of major paired galaxies in $D1$ and $D2$ type
environments. The probability distribution functions of the two
properties in $D1$ and $D2$ type environments are shown in the two
panels of \autoref{Fig2}. We first carry out the test for the major
pairs with all possible pair separations. We then conduct separate
tests for the major pairs with pair separation $>50$ kpc and $<50$
kpc. The results for the KS test are tabulated in \autoref{tab:ks}.
The results show that null hypothesis for stellar mass and cold gas
mass for all the major pairs can be rejected at $90\%$ and $99\%$
confidence levels respectively. This implies that the stellar mass
distribution of galaxies in major pairs residing in $D1$ and $D2$ type
environment are likely to be drawn from the same parent population.
However, the galaxies in major pairs residing in filaments and sheets
have a significantly different cold gas mass distribution. We also
arrive at the same conclusions for the major pairs with $r>50$ kpc
Interestingly, the results for the major pairs with $r< 50$ kpc
suggest that the null hypothesis for stellar mass can be rejected at a
very low confidence level ($< 60\%$), whereas for cold gas mass, it
can be rejected at $\leq 90\%$ confidence level. Thus, stellar mass of
major pair galaxies with $r<50$ kpc in $D1$ and $D2$ type environment
are highly likely to be drawn from the same parent population. This
clearly shows that stellar mass and available cold gas mass of the
paired galaxies are not responsible for the differences observed in
their $u-r$ colour and SFR in $D1$ and $D2$ type environments at
smaller pair separations ($r< 50$ kpc).

 The filaments in the cosmic web are generally located at the
 intersection of sheets. Analysis with N-body simulations suggest that
 matter successively flows from voids to sheets, sheets to filaments
 and finally from filaments to the clusters \citep{arag10, cautun14,
   ramachandra, galarraga22}. A number of earlier studies find that
 the galaxy pairs are preferentially aligned with the filament axis
 \citep{tempel15, mesa18}. The alignment signal is reported to be
 stronger for closer pairs residing near the filament spine. The
 anisotropic accretion along the filaments may significantly influence
 the gas accretion efficiency in these aligned galaxy pairs and
 trigger interaction induced star formation in them. On the other
 hand, the major pairs with $r>50$ kpc show a lower star formation in
 filaments than in sheets. The filaments are generally denser than the
 sheets. The $D1$-type galaxies are embedded in high density
 environment as compared to the $D2$-type galaxies
 \citep{pandey20}. The galaxies in high density environments are known
 to be less star forming and redder compared to those residing in low
 density environments \citep{lewis02,gomez03,kauffmann04}. So naively
 one would expect the galaxies in filamentary environment to be less
 star forming and redder compared to the galaxies in sheet-like
 environment. We find that this is true for the galaxies in major
 pairs with separation larger than $50$ kpc. However, the galaxies in
 major pairs at closer pair separation show a strikingly opposite
 behaviour.

The results reported in this work are very similar to the results
obtained in a recent study \citep{das23} of the colour and SFR of
major pairs in filaments and sheets using the SDSS data. \citet{das23}
use volume limited sample of galaxies ($M_r \leq -19$) for their
analysis and find a crossover in these properties at nearly the same
length scale ($\sim 50$ kpc). It is interesting to note that we
observe exactly the same trend in the EAGLE simulation data. This
provides a strong theoretical support to the observational findings
that large-scale structures like sheets and filaments affect galaxy
interactions. This also indicates that the galaxy properties are
modulated by their large-scale environment.

 Finally, we conclude that the filaments play a significant role in
 deciding the galaxy properties and their evolution. The observed
 differences in the colour and SFR of major pairs in filaments and
 sheets can not be entirely explained by the differences in the local
 density and the stellar mass distributions. The galaxy pairs at
 smaller separations are known to trigger star formation. The
 filaments provide a favourable environment for such interaction. This
 makes the intercting galaxies bluer in filaments compared to those
 found in sheets.
 
  
\begin{acknowledgements}
BP would like to acknowledge financial support from the SERB, DST,
Government of India through the project CRG/2019/001110. BP would also
like to acknowledge IUCAA, Pune for providing support through
associateship programme. SS acknowledges DST, Government of India for
support through a National Post Doctoral Fellowship (N-PDF).

The authors acknowledge the Virgo Consortium for making their
simulation data available. The EAGLE simulations were performed using
the DiRAC-2 facility at Durham, managed by the ICC, and the PRACE
facility Curie based in France at TGCC, CEA,
Bruy\`{e}res-le-Ch\^{a}tel.

\end{acknowledgements}



\begin{thebibliography}{99}
%%%%%%%%%%%%%%%%%%%%%%%%%%%%%%%%%%%%%%%%%%%%%%%%%%%%%%%%%%%

%\bibitem[Ahumada et. al (2020)]{ahumada20} Ahumada, R., Allende Prieto, C., Almeida, A., Anders, F., Anderson, S.~F., Andrews, B.~H., Anguiano, B., et al., 2020, ApJS, 249, 3 

\bibitem[Alonso et al. (2004)]{alonso04} Alonso, M.~S., Tissera, P.~B., Coldwell, G., Lambas, D.~G., 2004, MNRAS, 352, 1081 

\bibitem[Alonso et al. (2006)]{alonso06} Alonso, M.~S., Lambas, D.~G., Tissera, P., Coldwell, G., 2006, MNRAS, 367, 1029 

%\bibitem[Alpaslan et. al (2014)]{alpaslan14} Alpaslan, M. et al., 2014,  MNRAS, 438, 177 

\bibitem[\protect\citeauthoryear{Arag{\'o}n-Calvo, et al.}{2010}]{arag10} Arag{\'o}n-Calvo M.~A., Platen E., van de Weygaert R., Szalay A.~S., 2010, ApJ, 723, 364

\bibitem[Baldry et al. (2004)]{baldry04} Baldry, I.~K., Glazebrook, K., Brinkmann, J., Ivezi{\'c}, {\v{Z}}., Lupton, R.~H., Nichol, R.~C. \& Szalay, A.~S., ApJ, 600, 681

\bibitem[Balogh, Navarro \& Morris (2000)]{balogh00} Balogh, M.~L., Navarro, J.~F., \& Morris, S.~L., 2000, ApJ, 540, 113 

\bibitem[Bharadwaj et al. (2000)]{bharadwaj00} Bharadwaj S., Sahni V., Sathyaprakash B.~S., Shandarin S.~F., Yess C., 2000, ApJ, 528, 21


\bibitem[Bamford, Nichol \& Baldry (2009)]{bamford09} Bamford, S.~P., Nichol, R.~C., Baldry, I.~K., et al., 2009,  MNRAS, 393, 1324

\bibitem[Barton et al. (2007)]{barton07} Barton, E.~J., Arnold, J.~A., Zentner, A.~R., Bullock, J.~S., Wechsler, R.~H., 2007, ApJ, 671, 1538

\bibitem[Barton, Geller \& Kenyon (2000)]{barton00} Barton, E.~J., Geller, M.~J., Kenyon, S.~J., 2000, ApJ, 530, 660 

\bibitem[Bhattacharjee, Pandey \& Sarkar (2020)]{bhattacharjee20} Bhattacharjee, S., Pandey, B., \& Sarkar, S., 2020, JCAP, 2020, 039


\bibitem[Birnboim \& Dekel (2003)]{birnboim03} Birnboim Y., Dekel A., 2003, MNRAS, 345, 349

\bibitem[Blanton et al. (2003)]{blan03} Blanton, M.~R., et al., 2003, ApJ, 594, 186

\bibitem[Bond, Kofman \& Pogosyan (1996)]{bond96} Bond, J.~R., Kofman, L., \& Pogosyan, D., 1996, \nat, 380, 603


%\bibitem[Brinchmann et. al (2004)]{brinchmann2004} Brinchmann, J., Charlot, S., White, S.~D.~M., Tremonti, C., Kauffmann, G., Heckman, T., \&  Brinkmann, J., 2004, MNRAS,351, 1151

%\bibitem[Cappellari \& Emsellem (2004)]{ppxf} Cappellari, M., \& Emsellem, E., 2004, PASP, 116, 138 

%\bibitem[Casertano \& Hut (1985)]{casertano85} Casertano, S., \& Hut, P., 1985, ApJ, 298, 80 

\bibitem[\protect\citeauthoryear{Cautun, et al.}{2014}]{cautun14} Cautun M., van de Weygaert R., Jones B.~J.~T., Frenk C.~S., 2014, MNRAS, 441, 2923

%\bibitem[Colless et al. (2001)]{colless01} Colless M., Dalton G., Maddox S., Sutherland W., Norberg P., Cole S., Bland-Hawthorn J., et al., 2001, MNRAS, 328, 1039

%\bibitem[Conroy et. al (2009)]{conroy09}  Conroy, C., Gunn, J.~E., \& White, M., 2009, ApJ, 699, 486



%\bibitem[Cooper, Gallazzi, Newman \& Yan (2010)]{cooper10} Cooper, M.~C., Gallazzi, A., Newman, J.~A., Yan, R., 2010, MNRAS, 402, 1942


%\bibitem[Cornuault et al. (2016)]{cornu18} Cornuault N., Lehnert M., Boulanger F., Guillard P., 2018, A\&A, 610, A75


\bibitem[Cox et al. (2004)]{cox04} Cox, T.~J., Primack, J., Jonsson, P., \& Somerville, R.~S., 2004, ApJL, 607, L87


\bibitem[Das, Pandey \& Sarkar (2023)]{das23} Das, A., Pandey, B \& Sarkar, S., 2023, RAA, 02, 23

\bibitem[Dav{\'e}, Finlator, \& Oppenheimer (2011)]{dave11} Dav{\'e} R., Finlator K., Oppenheimer B.~D., 2011, MNRAS, 416, 1354


\bibitem[Galarraga-Espinosa et al. (2021)]{galarraga21} Galarraga-Espinosa, D., Aghanim, N., Langer, M., Tanimura, H., 2021, A\&A, 649, A117 

\bibitem[\protect\citeauthoryear{Gal{\'a}rraga-Espinosa, Garaldi, \& Kauffmann}{2022}]{galarraga22} Gal{\'a}rraga-Espinosa D., Garaldi E., Kauffmann G., 2022, arXiv, arXiv:2209.05495, Accepted in A\&A

\bibitem[Dav{\'e}, Finlator, \& Oppenheimer)(2012)]{dave12} Dav{\'e} R., Finlator K., Oppenheimer B.~D., 2012, MNRAS, 421, 98


\bibitem[Davis \& Geller (1976)]{davis76} Davis, M., \&  Geller, M.J., 1976, ApJ, 208, 13 


\bibitem[Dekel \& Birnboim (2006)]{dekel06} Dekel A., Birnboim Y., 2006, MNRAS, 368, 2


\bibitem[Dekel, Sari, \& Coverino (2009)]{dekel09} Dekel, A., Sari, R., Coverino, D., 2009, ApJ, 703, 785

\bibitem[Doi et al. (2010)]{doi10} Doi, M et al., 2010, doi:10.1088/004-6256/139/4/1628, arxiv:1002.3701


\bibitem[Dressler (1980)]{dress80} Dressler, A., 1980, ApJ, 236, 351


\bibitem[Einasto et al. (1984)]{einasto84} Einasto J., Klypin A.~A., Saar E., Shandarin S.~F., 1984, MNRAS, 206, 529


\bibitem[Einasto et al. (2003)]{einas03}  Einasto, J., H{\" u}tsi, G., Einasto, M., Saar, E., Tucker, D.~L., M{\" u}ller, V., Hein{\" a}m{\" a}ki, P., \& Allam, S.~S. , 2003, A\&A, 405, 425 


\bibitem[Einasto, Joeveer, \& Saar (1980)]{einasto80} Einasto J., Joeveer M., Saar E., 1980, MNRAS, 193, 353

\bibitem[Ellison et al. (2008)]{ellison08} Ellison, S.~L., Patton, D.~R., Simard, L., McConnachie, A.~W., 2008, AJ, 135, 1877 


\bibitem[Ellison et al. (2010)]{ellison10} Ellison, S.~L., Patton, D.~R., Simard, L., McConnachie, A.~W.,Baldry, I.~K., Mendel, J.~T., 2010, MNRAS, 407, 1514

\bibitem[Fall \& Efstathiou (1980)]{fall80} Fall, S.~M., \&  Efstathiou, G., 1980, MNRAS, 193, 189 

\bibitem[Gabor et al. (2010)]{gabor10} Gabor J.~M., Dav{\'e} R., Finlator K., Oppenheimer B.~D., 2010, MNRAS, 407, 749


\bibitem[G{\'o}mez et al. (2003)]{gomez03} G{\'o}mez, P.~L., Nichol, R.~C., Miller, C.~J., Balogh, M.~L., Goto, T., Zabludoff, A.~I., Romer, A.~K., et al., 2003, ApJ, 584, 210

\bibitem[Goto et al. (2003)]{gotto03} Goto, T., Yamauchi, C., Fujita, Y., Okamura, S., Seikiguchi, M., Smail, I., Bernardi, M.,\&  Gomez, P.L., 2003, MNRAS, 346, 601


\bibitem[Gregory \& Thompson (1978)]{gregory78} Gregory S.~A., Thompson L.~A., 1978, ApJ, 222, 784


%\bibitem[Gunn et. al (1998)]{gunn98} Gunn, J.~E., Carr, M., Rockosi, C., Sekiguchi, M., Berry, K., Elms, B., de Haas, E., et al., 1998, AJ, 116, 3040 

%\bibitem[Gunn et. al (2006)]{gunn06} Gunn, J.~E., Siegmund, W.~A., Mannery, E.~J., Owen, R.~E., Hull, C.~L., Leger, R.~F., Carey, L.~N., et al., 2006, AJ, 131, 2332 

\bibitem[Gunn \& Gott (1972)]{gunn72} Gunn, J.~E., \& Gott, J.~R., 1972, ApJ, 176, 1 

\bibitem[Guzzo et al. (1997)]{guzo97} Guzzo, L., Strauss, M.A., Fisher,K.B., Giovanelli, R., \&  Haynes, M.P., 1997, ApJ, 489, 37


\bibitem[Heiderman et al. (2009)]{heiderman09} Heiderman, A., Jogee, S., Marinova, I., van Kampen, E., Barden, M., Peng, C.~Y., Heymans, C. et al., 2009, ApJ, 705, 1433

\bibitem[Hogg et al. (2003)]{hog03} Hogg D.~W., Blanton M.~R., Eisenstein D.~J., Gunn J.~E., Schlegel D.~J., Zehavi I., Bahcall N.~A., et al., 2003, ApJL, 585, L5

\bibitem[Joeveer \& Einasto (1978)]{joeveer78} Joeveer M., Einasto J., 1978, IAUS, 79, 241

%\bibitem[Jones, van de Weygaert, \& Arag{\'o}n-Calvo (2010)]{jones10} Jones B.~J.~T., van de Weygaert R., Arag{\'o}n-Calvo M.~A., 2010, MNRAS, 408, 897


\bibitem[Kauffmann et al. (2004)]{kauffmann04}  Kauffmann, G., White, S.~D.~M., Heckman, T.~M., et al., 2004, \mnras, 353, 713

\bibitem[Kauffmann et al. (2003)]{kauffmann03} Kauffmann, G., Heckmann, T, M., White, S, D, M., Charlot, S., Tremonti, C et al., 2003, MNRAS, 341, 54

\bibitem[Kawata \& Mulchaey (2008)]{kawata08}  Kawata, D., \& Mulchaey, J.~S., 2008, ApJL, 672, L103 


\bibitem[Kere{\v{s}} et al. (2005)]{keres05} Kere{\v{s}} D., Katz N., Weinberg D.~H., Dav{\'e} R., 2005, MNRAS, 363, 2


\bibitem[Knapen \& James (2009)]{knapen09} Knapen, J.~H., James, P.~A., 2009, ApJ, 698, 1437


\bibitem[Koyama et al. (2013)]{koyama13}  Koyama, Y., Smail, I., Kurk, J., et al., 2013, MNRAS, 434, 423

\bibitem[Lambas et al. (2008)]{lambas03} Lambas D.~G., Tissera, P.~B., Alonso, M.~S., Coldwell, G., 2008, MNRAS, 346, 1189 


\bibitem[Larson, Tinsley \& Caldwell (1980)]{larson80} Larson, R.~B., Tinsley, B.~M., \& Caldwell, C.~N., 1980, ApJ, 237, 692


%\bibitem[Lee \& Erdogdu (2007)]{lee07} Lee J., Erdogdu P., 2007, ApJ, 671, 1248

%\bibitem[Lee \& Pen (2002)]{lee02} Lee J., Pen U.-L., 2002, ApJL, 567, L111


\bibitem[Lewis et al. (2002)]{lewis02}  Lewis, I., Balogh, M., Propris, R. De., Couch, W., Bower, R., Offer, A., Bland-Hawthorn, J., et al., 2002, MNRAS, 334, 673


\bibitem[Lilly et al. (2013)]{lilly13} Lilly S. J., Carollo C. M., Pipino A., Renzini A., Peng Y., 2013, ApJ, 772, 119L


\bibitem[Martig et al. (2009)]{martig09} Martig M., Bournaud F., Teyssier R., Dekel A., 2009, ApJ, 707, 250


\bibitem[Masters et al. (2010)]{masters10} Masters K.~L., Mosleh M., Romer A.~K., Nichol R.~C., Bamford S.~P., Schawinski K., Lintott C.~J., et al., 2010, MNRAS, 405, 783


\bibitem[McAlpine et al. (2016)]{eagle16} McAlpine, S., Helly, J, C., Schaller, M., Trayford, J, W. et al., 2016, A\&C, 72, 15



\bibitem[Mesa et al. (2018)]{mesa18} Mesa V., Duplancic F., Alonso S., Mu{\~n}oz Jofr{\'e} M.~R., Coldwell G., Lambas D.~G., 2018, A\&A, 619, A24


\bibitem[Moore et al. (1996)]{moore96} Moore, B., Katz, N., Lake, G., Dressler, A., \& Oemler, A., 1996, Nature, 379, 613 

\bibitem[Moore, Lake \& Katz (1998)]{moore98}  Moore, B., Lake, G., \&  Katz, N., 1998, ApJ, 495, 139

\bibitem[Mouhcine, Baldry \& Bamford (2007)]{mocine07} Mouhcine, M., Baldry, I.~K., \& Bamford, S.~P., 2007, MNRAS, 382, 801



\bibitem[Murray, Quataert \& Thompson (2005)]{murray05} Murray, N., Quataert, E., \& Thompson, T.~A., 2005, ApJ, 618, 569



\bibitem[Nikolic, Cullen \& Alexander (2004)]{nikolic04} Nikolic, B., Cullen, H., Alexander, P., 2004, MNRAS, 355, 874 

\bibitem[Oemler(1974)]{oemler74} Oemler, A., 1974, ApJ, 194, 1 

\bibitem[Pandey \& Bharadwaj (2005)]{pandey05} Pandey B.,  \& Bharadwaj S., 2005, MNRAS, 357, 1068

\bibitem[Pandey \& Bharadwaj (2006)]{pandey06} Pandey, B., \&   Bharadwaj, S., 2006, \mnras, 372, 827 

\bibitem[Pandey \& Bharadwaj (2008)]{pandey08}  Pandey, B., \& Bharadwaj, S., 2008, \mnras, 387, 767 


\bibitem[Pandey \& Sarkar (2017)]{pandey17} Pandey, B., \&  Sarkar, S., 2017, MNRAS, 467, L6

\bibitem[Pandey \& Sarkar (2020)]{pandey20}  Pandey, B., \& Sarkar, S., 2020, MNRAS, 498, 6069

\bibitem[Pandey (2020)]{pandey20a} Pandey B., 2020, MNRAS, 499, L31

\bibitem[Park et al. (2007)]{park07} Park C., Choi Y.-Y., Vogeley M.~S., Gott J.~R., Blanton M.~R., SDSS Collaboration, 2007, ApJ, 658, 898


%\bibitem[Patton et al. (2000)]{patton00} Patton, D.~R., Carlberg, R.~G., Marzke, R.~O., Pritchet, C.~J., da Costa, L.~N., Pellegrini, P.~S., 2000, ApJ, 536, 153 

%\bibitem[Patton et al. (2011)]{patton11} Patton, D.~R., Ellison, S.~L., Simard, L., McConnachie, A.~W., Mendel, J.~T., 2011, MNRAS, 412, 591
 


\bibitem[Patton \& Atfield(2008)]{patton08} Patton, D.~R. \& Atfield, J.~E., 2008, ApJ, 685, 235


\bibitem[Peng \& Renzini (2020)]{peng20} Peng Y.-. jie ., Renzini A., 2020, MNRAS, 491, L51


\bibitem[Planck Collaboration et al. (2014)]{planck14} Planck Collaboration et al., 2014,  A\&A, 571, A1 
 
\bibitem[Porter et al. (2008)]{porter08} Porter, S.~C., Raychaudhury, S., Pimbblet, K.~A., Drinkwater, M.~J., 2008, MNRAS, 388, 1152


%\bibitem[Propris et al. (2007)]{depropris07} Propris, R. De, Conselice, C.~J., Liske, J., Driver, S.~P., Patton, D.~R., Graham, A.~W., Allen, P.~D., et al., 2007, ApJ, 666, 212 

\bibitem[\protect\citeauthoryear{Ramachandra \& Shandarin}{2015}]{ramachandra} Ramachandra N.~S., Shandarin S.~F., 2015, MNRAS, 452, 1643

\bibitem[Rees \& Ostriker(1977)]{reesostriker77} Rees, M.~J. \& Ostriker, J.~P., 1977, MNRAS,179, 541

\bibitem[Robaina et al. (2009)]{robaina09} Robaina, A.~R., Bell, E.~F., Skelton, R.~E., Mcintosh, D.~H., Somerville, R.~S., Zheng, X., Rix, H.-W. et al., 2009, ApJ, 704, 324

\bibitem[\protect\citeauthoryear{Saintonge et al.}{2012}]{saintonge12} Saintonge A., Tacconi L.~J., Fabello S., Wang J., Catinella B., Genzel R., Graci{\'a}-Carpio J., et al., 2012, ApJ, 758, 73

\bibitem[Sarkar \& Bharadwaj (2009)]{sarkar09} Sarkar, P, \& Bharadwaj, S., 2009, \mnras, 394, L66

\bibitem[Sarkar \& Pandey (2019)]{sarkar19} Sarkar, S., \& Pandey, B., 2019, MNRAS, 485, 4743 

\bibitem[Sarkar \& Pandey (2020)]{sarkar20}  Sarkar, S., \&  Pandey, B., 2020, MNRAS, 497, 4077


%\bibitem[Sarzi et. al (2006)]{gandalf} Sarzi, M., Falc{\'o}n-Barroso, J., Davies, R.~L., Bacon, R., Bureau, M., Cappellari, M., de Zeeuw, P.~T. et al., 2006, MNRAS, 366, 1151 


%\bibitem[Scudder et al. (2012)]{scudder12b} Scudder, J.~M., Ellison, S.~L., Torrey, P., Patton, D.~R., Mendel, J.~T., 2012, MNRAS, 426, 549

\bibitem[Silk (1977)]{silk77} Silk, J., 1977 ApJ, 211, 638


\bibitem[Somerville \& Primack (1999)]{somerville99} Somerville, R.~S. , \& Primack, J.~R., 1999, MNRAS, 310, 1087 


\bibitem[Springel, Matteo \& Hernquist (2005)]{springel05} Springel, V., Di Matteo, T., \&  Hernquist, L., 2005, MNRAS,361, 776 


%\bibitem[Stoughton et al. (2002)]{stout02} Stoughton C., Lupton R.~H., Bernardi M., Blanton M.~R., Burles S., Castander F.~J., Connolly A.~J., et al., 2002, AJ, 123, 485


%\bibitem[Strauss et. al (2002)]{strauss02} Strauss, M.~A., Weinberg, D.~H., Lupton, R.~H., Narayanan, V.~K., Annis, J., Bernardi, M., Blanton, M., et al., 2002, AJ, 124, 1810

\bibitem[Strateva et al. (2001)]{strateva01} Strateva I., Ivezi{\'c} {\v{Z}}., Knapp G.~R., Narayanan V.~K., Strauss M.~A., Gunn J.~E., Lupton R.~H., et al., 2001, AJ, 122, 1861

  
\bibitem[Tempel \& Tamm (2015)]{tempel15} Tempel E., Tamm A., 2015, A\&A, 576, L5

\bibitem[\protect\citeauthoryear{Thorp et al.}{2022}]{thorp22} Thorp M.~D., Ellison S.~L., Pan H.-A., Lin L., Patton D.~R., Bluck A.~F.~L., Walters D., et al., 2022, MNRAS, 516, 1462

  
\bibitem[Trayford et al. (2015)]{trayford15} Trayford, J, W et al., 2015, MNRAS, 452, 2879

\bibitem[Tuominen (2021)]{tuominen21}  Tuominen T., Nevalainen J., Tempel E., Kuutma T., Wijers N., Schaye J., Hein{\"a}m{\"a}ki P., et al., 2021, A\&A, 646, A156

\bibitem[\protect\citeauthoryear{Violino et al.}{2018}]{violino18} Violino G., Ellison S.~L., Sargent M., Coppin K.~E.~K., Scudder J.~M., Mendel T.~J., Saintonge A., 2018, MNRAS, 476, 2591

\bibitem[White \& Rees (1978)]{white78} White, S.~D.~M., \&  Rees,M.~J., 1978 \mnras, 183, 341 

\bibitem[Woods \& Geller (2007)]{woods07} Woods, D.~F., Geller, M.~J., 2007, AJ, 134, 527 

\bibitem[Woods, Geller \& Barton (2006)]{woods06}  Woods, D.~F., Geller, M.~J., Barton, E.~J., 2006, AJ, 132, 197


\bibitem[Woods et al. (2010)]{woods10} Woods, D.~F., Geller, M.~J., Kurtz, M.~J., Westra, E., Fabricant, D.~G.,  Dell'Antonio, I., 2010, AJ, 139, 1857


%\bibitem[York et. al (2000)]{york00} York, D.~G. et al., 2000, \aj, 120, 1579

\bibitem[Zehavi et al. (2002)]{zevi02} Zehavi, I., et al. 2002, ApJ, 571, 172 

\bibitem[Zeldovich \& Shandarin (1982)]{zeldovich82} Zeldovich I.~B., Shandarin S.~F., 1982, PAZh, 8, 131

%\bibitem[Zhu, Zhang, \& Feng (2022)]{zhu22} Zhu W., Zhang F., Feng L.-L., 2022, ApJ, 924, 132


\end{thebibliography}
\label{lastpage}
\end{document}


