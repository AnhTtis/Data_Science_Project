%%
%% Beginning of file 'sample62.tex'
%%
%% Modified 2018 January
%%
%% This is a sample manuscript marked up using the
%% AASTeX v6.2 LaTeX 2e macros.
%%
%% AASTeX is now based on Alexey Vikhlinin's emulateapj.cls 
%% (Copyright 2000-2015).  See the classfile for details.

%% AASTeX requires revtex4-1.cls (http://publish.aps.org/revtex4/) and
%% other external packages (latexsym, graphicx, amssymb, longtable, and epsf).
%% All of these external packages should already be present in the modern TeX 
%% distributions.  If not they can also be obtained at www.ctan.org.

%% The first piece of markup in an AASTeX v6.x document is the \documentclass
%% command. LaTeX will ignore any data that comes before this command. The 
%% documentclass can take an optional argument to modify the output style.
%% The command below calls the preprint style  which will produce a tightly 
%% typeset, one-column, single-spaced document.  It is the default and thus
%% does not need to be explicitly stated.
%%
%%
%% using aastex version 6.2
\documentclass[twocolumn]{aastex62}

%% The default is a single spaced, 10 point font, single spaced article.
%% There are 5 other style options available via an optional argument. They
%% can be envoked like this:
%%
%% \documentclass[argument]{aastex62}
%% 
%% where the layout options are:
%%
%%  twocolumn   : two text columns, 10 point font, single spaced article.
%%                This is the most compact and represent the final published
%%                derived PDF copy of the accepted manuscript from the publisher
%%  manuscript  : one text column, 12 point font, double spaced article.
%%  preprint    : one text column, 12 point font, single spaced article.  
%%  preprint2   : two text columns, 12 point font, single spaced article.
%%  modern      : a stylish, single text column, 12 point font, article with
%% 		  wider left and right margins. This uses the Daniel
%% 		  Foreman-Mackey and David Hogg design.
%%  RNAAS       : Preferred style for Research Notes which are by design 
%%                lacking an abstract and brief. DO NOT use \begin{abstract}
%%                and \end{abstract} with this style.
%%
%% Note that you can submit to the AAS Journals in any of these 6 styles.
%%
%% There are other optional arguments one can envoke to allow other stylistic
%% actions. The available options are:
%%
%%  astrosymb    : Loads Astrosymb font and define \astrocommands. 
%%  tighten      : Makes baselineskip slightly smaller, only works with 
%%                 the twocolumn substyle.
%%  times        : uses times font instead of the default
%%  linenumbers  : turn on lineno package.
%%  trackchanges : required to see the revision mark up and print its output
%%  longauthor   : Do not use the more compressed footnote style (default) for 
%%                 the author/collaboration/affiliations. Instead print all
%%                 affiliation information after each name. Creates a much
%%                 long author list but may be desirable for short author papers
%%
%% these can be used in any combination, e.g.
%%
%% \documentclass[twocolumn,linenumbers,trackchanges]{aastex62}
%%
%% AASTeX v6.* now includes \hyperref support. While we have built in specific
%% defaults into the classfile you can manually override them with the
%% \hypersetup command. For example,
%%
%%\hypersetup{linkcolor=red,citecolor=green,filecolor=cyan,urlcolor=magenta}
%%
%% will change the color of the internal links to red, the links to the
%% bibliography to green, the file links to cyan, and the external links to
%% magenta. Additional information on \hyperref options can be found here:
%% https://www.tug.org/applications/hyperref/manual.html#x1-40003
%%
%% If you want to create your own macros, you can do so
%% using \newcommand. Your macros should appear before
%% the \begin{document} command.
%%

\newcommand{\LS}[1]{{\color{magenta} [#1]}}

\newcommand{\vdag}{(v)^\dagger}
\newcommand\aastex{AAS\TeX}
\newcommand\latex{La\TeX}
\newcommand{\dd}{{\rm d}}
\usepackage{amsmath}
\usepackage{xcolor} % for the guide rule

%% Tells LaTeX to search for image files in the 
%% current directory as well as in the figures/ folder.
\graphicspath{{./}{figures/}}


%% Reintroduced the \received and \accepted commands from AASTeX v5.2
% \received{January 1, 2018}
% \revised{January 7, 2018}
% \accepted{\today}
%% Command to document which AAS Journal the manuscript was submitted to.
%% Adds "Submitted to " the arguement.
% \submitjournal{ApJL}

%% Mark up commands to limit the number of authors on the front page.
%% Note that in AASTeX v6.2 a \collaboration call (see below) counts as
%% an author in this case.
%
%\AuthorCollaborationLimit=3
%
%% Will only show Schwarz, Muench and "the AAS Journals Data Scientist 
%% collaboration" on the front page of this example manuscript.
%%
%% Note that all of the author will be shown in the published article.
%% This feature is meant to be used prior to acceptance to make the
%% front end of a long author article more manageable. Please do not use
%% this functionality for manuscripts with less than 20 authors. Conversely,
%% please do use this when the number of authors exceeds 40.
%%
%% Use \allauthors at the manuscript end to show the full author list.
%% This command should only be used with \AuthorCollaborationLimit is used.

%% The following command can be used to set the latex table counters.  It
%% is needed in this document because it uses a mix of latex tabular and
%% AASTeX deluxetables.  In general it should not be needed.
%\setcounter{table}{1}

%%%%%%%%%%%%%%%%%%%%%%%%%%%%%%%%%%%%%%%%%%%%%%%%%%%%%%%%%%%%%%%%%%%%%%%%%%%%%%%%
%%
%% The following section outlines numerous optional output that
%% can be displayed in the front matter or as running meta-data.
%%
%% If you wish, you may supply running head information, although
%% this information may be modified by the editorial offices.
\shorttitle{A new universal relation}
\shortauthors{Y. Gao et al.}
%%
%% You can add a light gray and diagonal water-mark to the first page 
%% with this command:
% \watermark{text}
%% where "text", e.g. DRAFT, is the text to appear.  If the text is 
%% long you can control the water-mark size with:
%  \setwatermarkfontsize{dimension}
%% where dimension is any recognized LaTeX dimension, e.g. pt, in, etc.
%%
%%%%%%%%%%%%%%%%%%%%%%%%%%%%%%%%%%%%%%%%%%%%%%%%%%%%%%%%%%%%%%%%%%%%%%%%%%%%%%%%

%% This is the end of the preamble.  Indicate the beginning of the
%% manuscript itself with \begin{document}.

\begin{document}

\title{A tight universal relation between the shape eccentricity and the moment of inertia for rotating neutron stars}

%% LaTeX will automatically break titles if they run longer than
%% one line. However, you may use \\ to force a line break if
%% you desire. In v6.2 you can include a footnote in the title.

%% A significant change from earlier AASTEX versions is in the structure for 
%% calling author and affilations. The change was necessary to implement 
%% autoindexing of affilations which prior was a manual process that could 
%% easily be tedious in large author manuscripts.
%%
%% The \author command is the same as before except it now takes an optional
%% arguement which is the 16 digit ORCID. The syntax is:
%% \author[xxxx-xxxx-xxxx-xxxx]{Author Name}
%%
%% This will hyperlink the author name to the author's ORCID page. Note that
%% during compilation, LaTeX will do some limited checking of the format of
%% the ID to make sure it is valid.
%%
%% Use \affiliation for affiliation information. The old \affil is now aliased
%% to \affiliation. AASTeX v6.2 will automatically index these in the header.
%% When a duplicate is found its index will be the same as its previous entry.
%%
%% Note that \altaffilmark and \altaffiltext have been removed and thus 
%% can not be used to document secondary affiliations. If they are used latex
%% will issue a specific error message and quit. Please use multiple 
%% \affiliation calls for to document more than one affiliation.
%%
%% The new \altaffiliation can be used to indicate some secondary information
%% such as fellowships. This command produces a non-numeric footnote that is
%% set away from the numeric \affiliation footnotes.  NOTE that if an
%% \altaffiliation command is used it must come BEFORE the \affiliation call,
%% right after the \author command, in order to place the footnotes in
%% the proper location.
%%
%% Use \email to set provide email addresses. Each \email will appear on its
%% own line so you can put multiple email address in one \email call. A new
%% \correspondingauthor command is available in V6.2 to identify the
%% corresponding author of the manuscript. It is the author's responsibility
%% to make sure this name is also in the author list.
%%
%% While authors can be grouped inside the same \author and \affiliation
%% commands it is better to have a single author for each. This allows for
%% one to exploit all the new benefits and should make book-keeping easier.
%%
%% If done correctly the peer review system will be able to
%% automatically put the author and affiliation information from the manuscript
%% and save the corresponding author the trouble of entering it by hand.

\correspondingauthor{Lijing Shao}
\email{lshao@pku.edu.cn}

\author[0000-0003-1390-5477]{Yong Gao}
\affiliation{Department of Astronomy, School of Physics, Peking University, Beijing 100871, China}
\affiliation{Kavli Institute for Astronomy and Astrophysics, Peking University, Beijing 100871, China}

\author[0000-0002-1334-8853]{Lijing Shao}
\affiliation{Kavli Institute for Astronomy and Astrophysics, Peking University, Beijing 100871, China}
\affiliation{National Astronomical Observatories, Chinese Academy of Sciences, Beijing 100012, China}


\author[0000-0002-1614-0214]{Jan Steinhoff}
\affiliation{Max Planck Institute for Gravitational Physics (Albert Einstein Institute),
Am M\"uhlenberg 1, Potsdam 14476, Germany}



%% Note that the \and command from previous versions of AASTeX is now
%% depreciated in this version as it is no longer necessary. AASTeX 
%% automatically takes care of all commas and "and"s between authors names.

%% AASTeX 6.2 has the new \collaboration and \nocollaboration commands to
%% provide the collaboration status of a group of authors. These commands 
%% can be used either before or after the list of corresponding authors. The
%% argument for \collaboration is the collaboration identifier. Authors are
%% encouraged to surround collaboration identifiers with ()s. The 
%% \nocollaboration command takes no argument and exists to indicate that
%% the nearby authors are not part of surrounding collaborations.

%% Mark off the abstract in the ``abstract'' environment. 
\begin{abstract}

    Universal relations that are insensitive to the equation of state (EoS) are useful in reducing the parameter space when measuring global quantities of neutron stars (NSs). In this paper, we reveal a new universal relation that connects the eccentricity to the radius and moment of inertia of rotating NSs. We demonstrate that the universality of this relation holds for both conventional NSs and bare quark stars (QSs) in the slow rotation approximation, albeit with different relations. The maximum relative deviation is approximately $1\%$ for conventional NSs and $0.1\%$ for QSs. Additionally, we show that the universality still exists for fast-rotating NSs if we use the dimensionless spin to characterize their rotation. The new universal relation will be a valuable tool to reduce the number of parameters used to describe the shape and multipoles of rotating NSs, and it may also be used to infer the eccentricity or moment of inertia of NSs in future X-ray observations.

\end{abstract}

%% Keywords should appear after the \end{abstract} command. 
%% See the online documentation for the full list of available subject
%% keywords and the rules for their use.
\keywords{dense matter --- methods: numerical --- stars: rotation}


\section{Introduction} 
\label{sec:intro}


Neutron stars (NSs) are the densest stars in the universe, offering a unique laboratory to study supranuclear matter and gravity in the strong-field regime. Currently, the equation of state (EoS) for the cores of NSs is still poorly understood. Many EoS models with varying compositions and states of dense matter have been developed, leading to significantly different predictions of global properties for NSs~\citep{Lattimer:2000nx}. Therefore, observed global properties of NSs, such as the mass and the radius, can be used to constrain EoS models. 


Despite the fact that the global properties of NSs depend sensitively on the EoS models, there exist EoS-insensitive relations that connect various quantities of NSs. These relations are said to be universal because they are insensitive to EoS models to a high degree of accuracy. 
For instance, a universal relation connecting the frequency and damping time of the quadrupolar $f$ mode to the mass and moment of inertia of NSs was discovered by \citet{Lau:2009bu}. \citet{Yagi:2013awa} found the famous I-Love-Q relation for slowly rotating NSs, which links the mass, the moment of inertia, the tidal Love number, and the spin-induced quadrupole moment. Universal relations for NSs are of great significance in both astrophysics and fundamental physics. By providing EoS-insensitive connections between different quantities, these relations allow us to extract global properties of NSs with higher accuracy, and help us study the inverse problem of determining the EoS. Moreover, universal relations can break the degeneracy between gravity theories and the uncertainties in EoS, making NSs the ideal laboratories to test gravity \citep{Shao:2022koz}. We refer readers to \citet{Yagi:2013bca}, \citet{Doneva:2017jop}, and references therein for reviews.
 
The exploration of the universal relations for rotating NSs has garnered lots of attention since the discovery of the I-Love-Q relation. The calculations of the I-Q relation were quickly extended to fast rotation by \citet{Doneva:2013rha}, it was shown that the universality of the relation is lost and becomes increasingly EoS-dependent as the spin frequency increases. However, \citet{Pappas:2013naa} and \citet{Chakrabarti:2013tca} demonstrated that the I-Q relation remains universal if dimensionless quantities are used to characterize the spin amplitude instead of the spin frequency $f$. Later, \citet{Pappas:2013naa} and \citet{Yagi:2014bxa} discovered that the first four multipole moments of rotating NSs are universal to some extent. This relation allows for a more accurate description of the spacetime geometry around a NS with fewer parameters. Additionally, \citet{Luk:2018xmt} found another universal relation connecting the radius and orbital frequency of the innermost stable circular orbit (ISCO) to the mass and spin frequency of rotating NSs.

Apart from multipoles and ISCO, eccentricity is another important parameter to describe rotating NSs. The oblateness induced by 
rotation has a large impact on the X-ray emissions from the surface of X-ray pulsars. To study this effect, \citet{Morsink:2007tv} 
parameterized the oblate shape with the compactness of NSs and discovered that the geometric effect induced by the oblateness can 
rival the Doppler effect in certain configurations. To reduce the parameter space of X-ray modeling, \citet{Baubock:2013gna} derived a 
universal relation between the eccentricity and the compactness of slowly rotating NSs. \citet{AlGendy:2014eua} also found 
an EoS-insensitive fit of the eccentricity, using a slightly different parametrization for the surface other than the one used 
in the Hartle-Thorne formalism.


In this paper, we discover a new universal relation between the surface eccentricity and the moment of inertia for rotating NSs. The paper is structured as follows. In Sec.~\ref{sec:slow}, we provide a definition of multipoles and eccentricity in the slow rotation approximation and present the universal relation for both conventional NSs and QSs. In Sec.~\ref{sec:fast}, we investigate the universal relation for fast rotating NSs. Discussion of possible applications and connections of the new universal relation to early work is shown in Sec.~\ref{sec:discussion}. Throughout the paper, we use geometric units with $G=c=1$.


\section{A new universal relation in the slow-rotation approximation}
\label{sec:slow}

\subsection{Multipole moments and shape parameters}

To study the universal relation, we first give an overview of the
structures and shape parameters of slowly rotating NSs.
Following \citet{Hartle:1967he} and \citet{Hartle:1968si}, we construct these 
stars by solving the Einstein equations perturbatively in a slow-rotation expansion to quadratic order in the 
spin. At the zeroth order in spin, we obtain the mass $M$ and the radius $R$ of the non-rotating background.
At the first order in spin, we extract the angular momentum $J$, from which we can define the moment of inertia $I$ and the 
dimensionless spin $\chi$ as 
    $I\equiv {J}/{\Omega}$ and $\chi \equiv {J}/{M^2}$,
where $\Omega$ is the angular frequency of the rotating star. 
Universal relations usually connect dimensionless quantities. The dimensionless moment of inertia $\bar I$ is usually  
defined as 
    ${\bar I} \equiv {I}/{M^3}$.
At the second order in spin, the star is deformed into an oblate shape, and we get the spin-induced quadrupole moment $Q\equiv -{J^{2}}/{M}-{8}K M^{3}/5$.
The parameter $K$ depends on the EoS of NSs and equals to zero for Kerr black holes according to the no hair theorem. 
The dimensionless quadrupole moment is defined as 
    $\bar Q\equiv -{Q}/{M^{3}\chi^{2}}$.
The I-Q relation connects the dimensionless quantities $\bar I$ and $\bar Q$.
The exterior spacetime of a slowly rotating NS can be fully described up to the quadratic order in spin by 
the mass $M$, the angular momentum $J$, and the quadrupole moment $Q$~\citep{Hartle:1968si}. 

Observationally, some observation of a rotating NS depends on the geometry of its surface.
We use the eccentricity $e_{\rm s}$ to describe the oblate shape of a NS, 
\begin{equation}
    \label{eqn:def_eccentricity}
    {e}_{\rm s}\equiv \sqrt{\left(\frac{R_{\mathrm{\rm eq}}}{R_{\mathrm{\rm p}}}\right)^{2}-1}\,,
\end{equation}
where $R_{\rm eq}$ and $R_{\rm p}$ are the equatorial and polar radii in a specific coordinate.
In the Hartle-Thorne coordinate, the isodensity surface at radial coordinate $r$ in the non-rotating star is displaced to
\begin{equation}
    \label{eqn:ht_surface}
    r \to r+\xi_{0}(r)+\xi_{2}(r) P_{2}(\cos \theta) \,,
\end{equation}
in the rotating configuration, where $\xi_{0}$ and $\xi_{2}$ are spherical and quadrupole displacements respectively, and 
$P_{2}(\cos \theta)$ is the Legendre polynomial. 
Combining Eqs.~(\ref{eqn:def_eccentricity}--\ref{eqn:ht_surface}), we get the surface eccentricity in the Hartle-Thorne coordinate as 
\begin{equation}
    \label{eqn:htshape}
    e^{\rm HT}_{\rm s}=\left[-3\left(\xi_{2}(R) / R\right)\right]^{1 / 2}\,.
\end{equation}


Equation~(\ref{eqn:ht_surface}) describes the isodensity surface in a particular coordinate system. By embedding the isodensity surface into a three-dimensional flat space (denoted by $r^{*}$, $\theta^{*}$, $\phi^{*}$), \citet{Hartle:1968si} found an invariant parametrization of the oblate surface. To the second order of the spin, the desired surface is a spheroid with 
\begin{align}
    r^{*}\left(\theta^{*}\right)&=r+\xi_{0}(r)\nonumber \\
     &+\Big\{\xi_{2}(r)+r\left[v_{2}(r)-h_{2}(r)\right]\Big\} P_{2}\left(\cos \theta^{*}\right)\,.
\end{align}
Here $v_2$ and $h_2$ are metric functions at the second order in spin. The surface eccentricity observed in flat space is then given by
\begin{equation}
    \label{eqn:flatshape}
    e^{*}_{\rm s}=\big\{-3\big[v_{2}(R)-h_{2}(R)+\xi_{2}(R)/ R\big]\big\}^{1 / 2}\,,
\end{equation}
where the superscript "$*$" denotes the eccentricity observed in the flat space.

\subsection{A universal relation for the eccentricity of NSs}

The universal relation that we discovered connects the quantity $e_{\rm s}/R\Omega$ and the dimensionless 
moment of inertia $\bar I$. 
Same as the I-Q relation~\citep{Yagi:2013awa} and the three hair relation for the multipole moments~\citep{Yagi:2014bxa},
the normalization factors $M$ and $R$ are quantities of the non-rotating background in the slow-rotation approximation. For convenience, we define 
a dimensionless radius, 
    $\hat{R}\equiv{R\Omega}$.
We have verified that the universal relation exists for both the eccentricity in the Hartle-Thorne coordinate, $e^{\rm HT}_{\rm s}$, and the eccentricity observed in flat spacetime, $e^{\rm *}_{\rm s}$. In the following, we use $e^{\rm *}_{\rm s}$ to illustrate the results.

\begin{figure}
    \centering 
    \includegraphics[width=8.5cm]{fig_mr.pdf}
    \caption{The mass-radius relation for selected EoS models of NSs ({solid}) and QSs ({dashed}). The $1$-$\sigma$ regions of the
    mass measurements of PSR~J0348$+$0432 \citep{Antoniadis1233232} and PSR~J0740$+$6620 \citep{Fonseca:2021wxt} are illustrated.}
    \label{fig:mr}
\end{figure}


We first study the universal relation for conventional NSs. Our selection of realistic EoSs includes BSK21~\citep{Goriely:2010bm}, AU~\citep{Wiringa:1988tp}, HLPS~\citep{Hebeler:2013nza}, PAL1~\citep{Prakash:1988md}, APR~\citep{Akmal:1998cf}, SLy4~\citep{Douchin:2001sv}, MS0~\citep{Mueller:1996pm}, and ENG~\citep{Engvik1994}. As shown in Fig.~\ref{fig:mr}, these models cover a wide range in the mass-radius diagram of static NSs, and all of them have maximal NS mass larger than $2\,M_{\odot}$. Although the very stiff EoSs MS0 and PAL1 have been ruled out by the tidal deformability from GW170817~\citep{GW170817}, we include them to demonstrate that the universality exists for a large family of EoSs.  

Our universal relation is described with great accuracy by 
\begin{equation}
    \label{eqn:fitting}
    \frac{e^{*}_{\rm s}}{\hat R}=\sum_{k=0}^3 a_{ k}\left(\ln \bar I\right)^k\,,
\end{equation}
where $a_{ k}$'s are fitting coefficients with $a_0 =-0.855572$, $a_1 = 2.185502$, $a_2 =-0.428061$, $a_3 = 0.051177$. 
Note that we use the dimensionless moment of inertia $\bar I \leq 100$, which corresponds to $M \gtrsim 0.5\,M_{\odot}$ for selected models.
We define the relative deviation to Eq.~(\ref{eqn:fitting}) as 
\begin{equation}
    \Delta=\frac{e^{*}_{\rm s}/ \hat{R}-(e^{*}_{\rm s}/ \hat{R})_{\rm fit}}{(e^{*}_{\rm s}/ \hat{R})_{\rm fit}} \,.
\end{equation}
As shown in Fig.~\ref{fig:universal_e}, the relative deviation to the universal relation is smaller than $1\%$ for selected models of EoSs.   NSs with $M\gtrsim 1\,M_{\odot}$ are more relevant for astrophysics, and in this case, the dimensionless moment of inertia $\bar{I}\lesssim 30$ for selected EoS models, and the universal relation takes a simpler form,
\begin{equation}
    \frac{e^{*}_{\rm s}}{\hat R} = 0.11418+1.04115\ln \bar I\,.
\end{equation}
The relative deviation to this relation is less than $\sim 1\%$.


\begin{figure}
    \centering 
    \includegraphics[width=8.5cm]{fig_slow.pdf}
    \caption{The $e_{\rm s}/\hat{R}$--$\bar I$ universal relation for slowly-rotating NSs. The upper panel shows the fitted universal relations for both NSs and QSs. The middle (lower) panel presents the relative deviation for NSs (QSs).}
    \label{fig:universal_e}
\end{figure}

For QSs, we use the phenomenological MIT bag model to describe the quark matter. This model assumes a nearly equal number of $u$, $d$, $s$ quarks and a small fraction of electrons confined within a bag of vacuum energy density $B$ \citep{Farhi:1984qu,Witten:1984rs}. We account for the mass  of the $s$ quark, $m_{\rm s}$, and include the quark-gluon interaction to the lowest order in $\alpha_c=g^2/4\pi$. To investigate the universal relation for QSs, we employ six different EoSs with varying combinations of $m_{\rm s}$, $\alpha_{\rm c}$, and $B$ in Table~\ref{tab:sqm}. The resulting mass-radius relation is displayed in Fig.~\ref{fig:mr}. 

\begin{table}
    \caption{Parameters for QSs in the MIT bag model.}
    \centering
    \begin{tabular}{lccc}
    \toprule
        Model & $B\,(\rm MeV\,fm^{-3})$  & $m_{\rm s}\,(\rm MeV)$ & $\alpha_{\rm c}$ \\ \hline
        SQM1 & 80 & 100 & 0 \\ 
        SQM2  & 80 & 50 & 0.1 \\ 
        SQM3  & 70 & 150 & 0 \\ 
        SQM4  & 70 & 50 & 0.3 \\ 
        SQM5  & 60 & 0 & 0 \\ 
        SQM6  & 60 & 100 & 0.4 \\ \hline
    \end{tabular}
    \label{tab:sqm}
\end{table}

The relation between $e^{*}_{\rm s}/\hat{R}$ and $\bar I$ is different from that of NSs, but it is still universal and can be well fitted by 
\begin{equation}
    \label{eqn:qs_fitting}
    \frac{e^{*}_{\rm s}}{\hat R}=\sum_{k=0}^4 b_{ k}\left(\ln \bar I\right)^k\,,
\end{equation}
with coefficients $b_0= -1.499749$, $b_1= 2.911859$, $b_2 =-0.749237$, $b_3 = 0.137057$, and $b_4 =-0.007801$. Interestingly, the deviation from the universal relation for QSs is much smaller than that for NSs, with relative deviation less than $\sim 0.1\%$, as shown in the lower panel of Fig.~\ref{fig:universal_e}. For QSs in our study, the condition for $M \gtrsim 1\,M_{\odot}$ corresponds to $\bar I \lesssim 20$. Within this range, the universal relation can be approximated by a simpler fitting formula, 
\begin{equation}
    \frac{e^{*}_{\rm s}}{\hat R} = -0.043372 + 1.210546\ln(\bar{I} -0.470579)\,,
\end{equation}
The relative deviation from this fitting formula is less than $\sim 0.3\%$.

\section{Universal relation for fast rotating NSs}
\label{sec:fast}

Fast rotation is relevant for sub-millisecond pulsars, nascent NSs after supernovae, and NSs formed in binary NS mergers. Rapid rotation causes NSs to develop a more obvious oblate shape. In this section, we explore the universal relation for rapidly rotating NSs using the {\tt RNS} code developed by \citet{Stergioulas:1994ea}. 

The {\tt RNS} code uses a quasi-isotropic coordinate system to represent the line element of the stationary axisymmetric spacetime,
\begin{align}
\dd s^2= & -e^{2 \nu} \dd t^2+ B^2 r^2 \sin ^2 \theta  e^{-2 \nu}(\dd \phi-\omega \dd t)^2 \nonumber \\
& +e^{2(\xi-\nu)}\left(\dd r^2+r^2 \dd \theta^2\right),
\end{align}
where $\nu$, $B$, $\omega$, and $\xi$ are metric functions that depend on $r$ and $\theta$. Assuming a perfect fluid and uniform rotation, we obtain the stellar structure and spacetime metric. The conserved angular momentum $J$ can be computed from a volume integration over the matter field. The moment of inertia $I$ and the dimensionless spin have the same definition as before. The quadrupole moment $Q$ can be obtained from the asymptotic expansion of the metric functions. The surface eccentricity is formally given by Eq.~(\ref{eqn:def_eccentricity}), with the eccentricity, equatorial and polar radii defined in the quasi-isotropic coordinate. We use the notation $\tilde{e}_{\rm s}$ to denote the eccentricity in the quasi-isotropic coordinate. Note that, unlike in the slow-rotation approximation, the normalization factor $M$ is the mass for the rotating configuration, and $\hat{R}\equiv R_{\rm eq}\Omega$.

\begin{figure}
    \centering 
    \includegraphics[width=8.5cm]{fig_fast1.pdf}
    \caption{The $\tilde{e}_{\rm s}/\hat{R}$-$\bar I$ universal relation for fast rotating NSs. 
    The first panel shows the fitted relation for $\chi=0.2$, $0.4$, and $0.6$. The second to fourth panels 
    display the relative deviations to the universal relation for $\chi=0.2$, $0.4$, and $0.6$ respectively.}
    \label{fig:universal_fast1}
\end{figure}


To study universal relations for rapidly rotating NSs, it is necessary to use a suitable parameter to characterize their spin amplitude. As demonstrated by \citet{Doneva:2013rha}, if one uses the spin frequency $f$ as the parameter, the I-Q relation for fast rotating NSs is lost. Similarly, the universal relation that we discovered also breaks down for fixed spin frequencies. However, \citet{Pappas:2013naa} and \citet{Chakrabarti:2013tca} found that the I-Q relation is still universal for fast rotating NSs if one chooses dimensionless spin parameters such as $\chi$, $Mf$, and $Rf$, instead of the dimensionful $f$. Inspired by their work, we use $\chi$ to characterize the spin amplitude and find that the $\tilde{e}_{\rm s}/\hat{R}$--$\bar I$ relation for both conventional NSs and strange QSs is still universal. 


According to \citet{Lo:2010bj}, the maximum value of the dimensionless spin parameter $\chi$ for NSs rotating at the Keplerian frequency is about 0.7 for various EoS models. This limit is nearly independent of the mass of the NS if the mass is larger than $1\,M_{\odot}$. However, for QSs in the MIT bag model, the spin parameter can be larger than unity and does not have a universal upper limit. Its value also depends strongly on the bag constant and the mass of the star. Therefore, in Fig.~\ref{fig:universal_fast1}, we display three representative cases with $\chi=0.2$, $0.4$, and $0.6$ for conventional NSs. The relative deviation from the universal relation is less than $\sim 1\%$. The cases for QSs with $\chi=0.5$ and $\chi=0.8$ are shown in Fig.~\ref{fig:universal_fast2}, and the relative deviation is on the order of $0.1\%$, which is again much tighter than conventional NSs.


\begin{figure}
    \centering 
    \includegraphics[width=8.5cm]{fig_fast2.pdf}
    \caption{The $\tilde{e}_{\rm s}/\hat{R}$--$\bar I$ universal relation for fast rotating QSs. 
    The first panel shows the fitted relation for $\chi=0.5$ and $0.8$. The second and third panels 
    display the relative deviations to the universal relation for $\chi=0.5$ and $0.8$ respectively.}
    \label{fig:universal_fast2}
\end{figure}

\section{Discussion}
\label{sec:discussion}

The relation between $e_{\rm s}/\hat{R}$ and $\bar I$ adds a new tight universal relation to the known ones. Here, $e_{\rm s}$ is the surface eccentricity formally defined in Eq.~(\ref{eqn:def_eccentricity}) without assigning a specific coordinate system. In our paper, it includes $e_{\rm s}^{\rm HT}$, $e_{\rm s}^*$, and $\tilde e_{\rm s}$, which are eccentricities defined in different coordinate systems. The effect of gauge choice on the eccentricity is very small. All of these eccentricities satisfy the universal relation very well. Since the I-Q relation~\citep{Yagi:2013awa,Chakrabarti:2013tca} connects $\bar I$ to $\bar Q$, and the three-hair relation~\citep{Pappas:2013naa,Yagi:2014bxa} connects $\bar Q$ to two other higher-order multipoles, the relation between $e_{\rm s}/\hat{R}$ and these normalized multipole moments is also universal. Combined with the I-Love relation~\citep{Yagi:2013awa}, we have a universal relation between $e_{\rm s}/\hat{R}$ and the dimensionless tidal Love number. Moreover, the universal relation between the $f$-mode oscillation and $\bar I$~\citep{Lau:2009bu} helps us connect $e_{\rm s}/\hat{R}$ to the frequency and damping time of the quadrupolar $f$ mode.

Our work is not the only universal relation for the shape parameter of rotating NSs. To incorporate the effects of the eccentricity of NSs when modelling the X-ray emission, \citet{Baubock:2013gna} derived an analytic expression connecting the eccentricity of the stellar surface to the compactness $C$, the spin angular momentum, and the quadrupole moment of the spacetime by using the analytical results in \citet{Hartle:1968si}. They also used several other universal relations to express $e_{\rm s}$ in a single parameter $C$ \citep[see Fig.~3 of][]{Baubock:2013gna}. Our $e_{\rm s}/\hat{R}$-$\bar I$ relation is tighter than the $e_{\rm s}$-$C$ relation.

Universal relations are a powerful tool to reduce modelling uncertainties and inferring NS parameters. As we discussed before, the eccentricity of rotating NSs is an observable and is an important input for X-ray modelling. The oblateness induced by rotation at frequencies above $300\,\rm Hz$ produces a geometric effect that has imprints in the pulse profile of X-ray pulsars~\citep{Morsink:2007tv}. For some emission configurations, the oblateness effect can rival the Doppler effect. Besides, the effects of oblateness need to be taken into account when measuring the radii of NSs from rotationally broadened atomic lines~\citep{Baubock:2012bj}. On one hand, our new universal relation reduces the number of parameters used to describe the shape and multipoles of rotating NSs. On the other hand, the new universal relation can be used to infer the eccentricity of NSs if the mass, radius, spin frequency, and moment of inertia are inferred in future X-ray observations. 

\begin{acknowledgments}
We thank Enping Zhou for discussion. This work was supported by the National SKA Program of China (2020SKA0120300), the National Natural Science Foundation of
China (11975027, 11991053), the
Max Planck Partner Group Program funded by the Max Planck Society, and the
High-Performance Computing Platform of Peking University.
\end{acknowledgments}

\software{RNS \citep{Stergioulas:1994ea}}

\bibliography{refs}{}
\bibliographystyle{aasjournal}


\end{document}


