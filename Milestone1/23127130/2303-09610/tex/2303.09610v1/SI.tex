\PassOptionsToPackage{square,numbers,comma,sort&compress,super}{natbib} %Comment this if superscript referencing required
\documentclass[12pt,aps,prl,letterpaper,superscriptaddress]{revtex4-2}


%\usepackage{multibbl}
\usepackage[pdftex]{graphicx}

\usepackage{amsmath}
\usepackage[latin1]{inputenc}
\usepackage{latexsym,stmaryrd}
\usepackage[dvipsnames]{xcolor}
\usepackage[pdftex]{graphicx}
\usepackage{color}
\usepackage{float}

\usepackage{multibib}
\usepackage{graphicx,amsmath,SIunits}
\usepackage{latexsym,stmaryrd}




\usepackage[dvipsnames]{xcolor}
\usepackage[normalem]{ulem}
\usepackage{float}



%Setting for figures
\renewcommand{\topfraction}{.99}
\renewcommand{\bottomfraction}{.99}
\setcounter{topnumber}{30}
\setcounter{bottomnumber}{20}
\setcounter{totalnumber}{50}
\renewcommand{\textfraction}{0.02}
\renewcommand{\floatpagefraction}{0.99}

\renewcommand{\bibname}{References}

\renewcommand{\figurename}{\textbf{Supplementary Figure}}

\makeatletter

\renewcommand*{\fnum@figure}{{\normalfont\bfseries \figurename~\thefigure}}
\renewcommand{\thefigure}{S\@arabic\c@figure}
\makeatother

\renewcommand\theequation{S\arabic{equation}}

\renewcommand\thetable{S\arabic{table}}

\renewcommand{\bibname}{References}

\renewcommand{\thesection}{S\arabic{section}}



\begin{document}


\title{Supplementary information for \\ Self-assembly of atomic-scale photonic cavities}


\author{Ali Nawaz Babar}
\email{anaba@dtu.dk}
\affiliation{Department of Electrical and Photonics Engineering, DTU Electro, Technical University of Denmark, Building 343, DK-2800 Kgs. Lyngby, Denmark}
\affiliation{NanoPhoton - Center for Nanophotonics, Technical University of Denmark, {\O}rsteds Plads 345A, DK-2800 Kgs. Lyngby, Denmark.}

\author{Thor August Schimmell Weis}
\affiliation{Department of Electrical and Photonics Engineering, DTU Electro, Technical University of Denmark, Building 343, DK-2800 Kgs. Lyngby, Denmark}

\author{Konstantinos Tsoukalas}
\affiliation{Department of Electrical and Photonics Engineering, DTU Electro, Technical University of Denmark, Building 343, DK-2800 Kgs. Lyngby, Denmark}

\author{Shima Kadkhodazadeh }
\affiliation{NanoPhoton - Center for Nanophotonics, Technical University of Denmark, {\O}rsteds Plads 345A, DK-2800 Kgs. Lyngby, Denmark.}
\affiliation{DTU Nanolab, Technical University of Denmark, Building 347,
DK-2800 Kgs. Lyngby, Denmark}

\author{Guillermo Arregui}
\affiliation{Department of Electrical and Photonics Engineering, DTU Electro, Technical University of Denmark, Building 343, DK-2800 Kgs. Lyngby, Denmark}

\author{Babak Vosoughi Lahijani}
\affiliation{Department of Electrical and Photonics Engineering, DTU Electro, Technical University of Denmark, Building 343, DK-2800 Kgs. Lyngby, Denmark}
\affiliation{NanoPhoton - Center for Nanophotonics, Technical University of Denmark, {\O}rsteds Plads 345A, DK-2800 Kgs. Lyngby, Denmark.}

\author{S{\o}ren Stobbe}
\email{ssto@dtu.dk}
\affiliation{Department of Electrical and Photonics Engineering, DTU Electro, Technical University of Denmark, Building 343, DK-2800 Kgs. Lyngby, Denmark}
\affiliation{NanoPhoton - Center for Nanophotonics, Technical University of Denmark, {\O}rsteds Plads 345A, DK-2800 Kgs. Lyngby, Denmark.}

\maketitle

\section{S1. Investigation of pull-in instabilities due to surface forces} \label{sec:surfaceforces}


\begin{figure}[t]
 \centering
\includegraphics[width=0.88
\columnwidth]{Figures/Figures_SI/S_1-1_1.pdf}
\caption[]{\textbf{Automatic image-analysis of relative scale displacement.} \textbf{a}, Top-view SEM image of a scale attached to the platform with a fabricated gap, $g_\text{f}$, of 520 nm (after underetching). \textbf{b}, Sobel edge detection. \textbf{c}, Radon transform of the image to measure the device's rotation angle relative to the image's coordinate system. $\Theta$ is the angle of rotation of an axis centered on the image. $x'$ is the distance in pixels of a given pixel from the rotated axis. \textbf{d}, Match between periodicity found in the image columns and the known periodicity of the scales. \textbf{e}, Detected scales. \textbf{f}, Scales cropped and converted to a black-and-white image. \textbf{g}, Graph obtained by summing the pixel counts of the black-and-white image, normalizing, and centering.}
\label{fig:stressrelease}
\end{figure}
% \floatbarrier

\subsection{S1.1. Displacement measurements}
\noindent The built-in stress in the device layer of silicon-on-insulator wafers causes expansion or contraction when the silicon oxide is selectively etched away to release the structures. This means that the initial gaps in the platforms we investigate, $g_\text{0}$, i.e., the gaps as they would be without the surface forces, do not exactly correspond to the fabricated gaps before underetching, $g_\text{f}$. The initial gap is modified by the stress-release displacement, $\Delta g$, which we correct for in our experiments as explained below. This displacement is experimentally obtained by including, for each spring constant and platform width, a reference platform with a fabricated 520 nm gap, which is large enough to diminish the surface forces by orders of magnitude~\cite{Gusso_DispersionForces}. We acquire scanning electron microscope (SEM) images of all such devices and perform image analysis to extract $\Delta g$. As mentioned in the Methods section of the main text, the platforms are equipped with scales on the sides, see Fig.~\ref{fig:stressrelease}a, allowing us to accurately extract their displacement from an SEM image acquired at high scan speed. The scales are analyzed by first using Sobel edge detection as shown in Fig.~\ref{fig:stressrelease}b, followed by a Radon transform as shown in Fig.~\ref{fig:stressrelease}c. The latter detects the angle of rotation of the device relative to the coordinate system of the image and compensates for it by counter-rotating with a bi-linear rotation. The pixel columns containing the scales are found by column-wise fast Fourier transforms (FFT) and identification of the Fourier components matching the known periodicity of the scales as shown in Fig.~\ref{fig:stressrelease}d. The extracted pixel columns are then limited to the rows containing periodic structures, leaving only the areas marked in Fig.~\ref{fig:stressrelease}e. The detected scales are cropped from the image and converted to black and white images as seen in Fig.~\ref{fig:stressrelease}f, from which the number of black pixels in each column is counted to give the normalized plot seen in Fig.~\ref{fig:stressrelease}g. The phase difference between the two resulting curves is found by Fourier analysis. Based on the measurements of 57 devices, we find a mean stress-release displacement of 19.4 nm with a standard deviation of 2.8 nm, independent of the platform width and the spring constant. Finally, the initial gap, $g_\text{0}$, used in Fig.~1d of the main text, is obtained by systematically acquiring SEM images of the platforms before removing the buried oxide layer to extract the fabricated gap, $g_\text{f}$, followed by subtraction of the known displacement due to stress release, $\Delta g$, such that the initial gap is extracted as $g_\text{0} = g_\text{f}-\Delta g$.
\label{subsec:stressrelease}

\begin{figure}[ht]
 \centering
\includegraphics[width=0.9\columnwidth]{Figures/Figures_SI/SF1a.pdf}
\caption[]{\textbf{Post-underetching structural state of silicon platforms with widths 2, 3, 4, and 5 \micro m, respectively, for Sample A.} The columns indicate platforms with a fabricated gap, $g_\text{f}$, measured before releasing the structures, and the rows indicate platforms with different spring constants, $k$. Each block shows the experimental data for a specific platform width, $w$. The red cells represent platforms that collapsed in-plane on the anchored silicon, and the blue cells indicate platforms that did not collapse.}
\label{fig:Raw_data_1}
\end{figure}

\begin{figure}[h!]
 \centering
\includegraphics[width=0.9\columnwidth]{Figures/Figures_SI/SF1b.pdf}
\caption[]{\textbf{Post-underetching structural state of silicon platforms with widths 8, 16, 32, and 64 \micro m, respectively, for Sample A.} The columns indicate platforms with a fabricated gap, $g_\text{f}$, measured before releasing the structures, and the rows indicate platforms with different spring constants, $k$. Each block shows the experimental data for a specific platform width, $w$. The red cells represent platforms that collapsed in-plane on the anchored silicon, and the blue cells indicate platforms that did not collapse.}
\label{fig:Raw_data_2}
\end{figure}


\begin{figure}[ht]
 \centering
\includegraphics[width=0.9\columnwidth]{Figures/Figures_SI/SF2a.pdf}
\caption[]{\textbf{Post-underetching structural state of silicon platforms with widths 2, 4, and 8 \micro m, respectively, for Sample B.} The columns indicate platforms with a fabricated gap, $g_\text{f}$, measured before releasing the structures, and the rows indicate platforms with different spring constants, $k$. Each block shows the experimental data for a specific platform width, $w$. The red cells represent platforms that collapsed in-plane on the anchored silicon, and the blue cells indicate platforms that did not collapse. The yellow cells indicate the devices excluded due to fabrication imperfections or spring failure.}
\label{fig:Raw_data_3}
\end{figure}



\begin{figure}[ht]
 \centering
\includegraphics[width=0.9\columnwidth]{Figures/Figures_SI/SF2b.pdf}
\caption[]{\textbf{Post-underetching structural state of silicon platforms with widths 16, 32, and 64 \micro m, respectively, for Sample B.} The columns indicate platforms with a fabricated gap, $g_\text{f}$, measured before releasing the structures, and the rows indicate platforms with different spring constants, $k$. Each block shows the experimental data for a specific platform width, $w$. The red cells represent platforms that collapsed in-plane on the anchored silicon, and the blue cells indicate platforms that did not collapse. The yellow cells indicate the devices excluded due to fabrication imperfections or spring failure.}
\label{fig:Raw_data_4}
\end{figure}

\begin{figure}[ht]
 \centering
\includegraphics[width=0.8\columnwidth]{Figures/Figures_SI/SampleB.pdf}
\caption[]{\textbf{Measured map of the design space for self-assembly with compliant silicon structures.} The map is obtained by characterizing 1152 platforms by SEM (Sample B). The different sizes of the circles represent the
different widths of the platforms. The dark purple circle indicate the smallest initial gap not leading to a collapsed platform and the dark green circle indicates the largest initial gap leading to a collapsed platform. All the devices below the upper bound collapse, while those above the lower bound do not.}
\label{fig:SampleB}
\end{figure}

\subsection{S1.2. Mapping the threshold for self-assembly by surface forces: Raw data}

We fabricate a total of 2688 devices distributed across two samples (Sample A: 1536 devices; Sample B: 1152 devices) with different values of platform width, $w$, fabricated gap, $g_\text{f}$, and spring constant, $k$, as discussed in the Methods section. We characterize the devices using SEM performed with a single high-speed scan to minimize charge-induced displacement of the platforms, except when imaging for illustrative purposes, as in Fig.~1c in the main text.  Figures~\ref{fig:Raw_data_1} and \ref{fig:Raw_data_2} show the resulting map of the devices that either collapsed in-plane (red) on the anchored silicon or did not collapse (blue). The devices that failed due to fabrication imperfections or spring failure, such as out-of-plane collapse (yellow), are excluded from the final dataset. The data shown in Figs.~\ref{fig:Raw_data_1} and \ref{fig:Raw_data_2} is used to plot Fig.~1d in the main text. This experiment is repeated on Sample B to verify the reproducibility and robustness of our approach, and the raw data is shown in Figs.~\ref{fig:Raw_data_3} and \ref{fig:Raw_data_4}. Figure.~\ref{fig:SampleB} shows the self-assembly design space with compliant silicon structures for Sample B (1152 devices) . In this case, we find that all platforms for which $g_0 < 4.2(k/A)^{-0.48}$ collapse and all platforms for which $g_0 > 24(k/A)^{-0.47}$ do not collapse.

\subsection{S1.3. The critical gap of Casimir-Lifshitz theory as the threshold for self-assembly by surface forces}

The van der Waals force, which is responsible for surfaces adhering when they touch, is the short-distance (non-retarded) limit of the more general Casimir-Lifshitz force that in the idealized case of perfectly reflecting infinitely extended surfaces reduces to the long-range attractive force described by Casimir~\cite{casimir1948attraction}. When the intervening material between two surfaces is vacuum, the Casimir-Lifshitz force is attractive and increases non-linearly with decreasing separation, $g$, between the surfaces. If two surfaces are initially separated by the critical gap, $g_\text{c}$, the Casimir-Lifshitz force can overwhelm the elastic forces that hold the surfaces in place, bring them in contact, and subsequently stick them onto each other. Here, we use a lumped-element model where the elastic force is described by a linear spring with spring constant $k$ to calculate the critical gap.

The total force between two parallel surfaces separated by a gap $g=g_0-x$, where $g_0$ is the initial gap between the surfaces and $x$ is their displacement due to the Casimir-Lifshitz attraction is given by
%
\begin{equation}
    \label{eq:Ftot}
    F_\text{tot}(g)= A F_\text{CL}(g) - k(g_0-g),
\end{equation}
%
where $A$ is the area and $F_\text{CL}(g)$ is the Casimir-Lifshitz force per unit area between two surfaces given by \cite{Gusso_DispersionForces} 
%
\begin{equation}
    \label{eq:CLintegral}
    \begin{split}
    F_\text{CL}(g) & = -\frac{\hbar}{2 \pi^2 c^3}  \int_1^\infty p^2 \text{d}p 
   \int_0^\infty  \xi^3 \text{d} \xi \Bigg\{    \left[ \left( \frac{K+\epsilon \left( i \xi \right)p   }{K-\epsilon \left( i \xi \right)p } \right)^2 e^{2(\xi/c)pg}-1 \right]^{-1} \\
  & + \left[ \left( \frac{K+p}{K-p } \right)^2 e^{2(\xi/c)pg}-1 \right]^{-1}
  \Bigg\},
    \end{split}
\end{equation}
%
with $K = \sqrt{p^2 -1 + \epsilon \left( i \xi \right)}$ and the dielectric function of silicon at imaginary frequencies given by $\epsilon \left( i \xi \right)= 1+ \frac{10.703}{1+ \left(  \xi*1.506*10^{-16}  \right)^{1.83}}$, where $\xi$ is measured in rad/s~\cite{moazzami2021self}. The system loses linear stability at the point $g^*$ when
\begin{equation}
    \label{eq:stab1}
    \frac{\partial  F_\text{tot}(g^*)}{ \partial g} = 0.
\end{equation}
The critical gap, $g_c$, is defined as the initial gap that leads to the instability point, $g^*$, after the surfaces have been attracted by the Casimir force, i.e.,
\begin{equation}
    \label{eq:stab2}
    g_c=\frac{A}{k}F_\text{CL}(g^*)+g^*.
\end{equation}
To calculate the critical gap $g_c$ for a given $k/A$, we first numerically solve Eq.~(\ref{eq:stab1}) for $g^*$ which we subsequently insert in Eq.~(\ref{eq:stab2}).



\section{S2. Design of nanobeam photonic-crystal cavities with bowtie unit cells}
\label{sec:cavitydesign}

\begin{figure}[t]
\centering
\includegraphics[width=1\textwidth]{Figures/Figures_SI/SF9.pdf}
\caption[]{\textbf{Design of a photonic-crystal nanobeam bowtie cavity.}  \textbf{a}, Geometric parameters of a single bowtie unit cell with an air bowtie of width $g$. \textbf{b}, Band structure (TE-like modes) for the central unit cell ($W_\text{b}$ = 116 nm, black curve), and for the mirror unit cell ($W_\text{b}$ = 139 nm, red curve). The blue region indicates the light cone. \textbf{c,d}, normalized $|\mathbf{E}|$-field of the center bowtie unit cell in the middle plane of the structure and at the edge of the Brillouin zone (\textbf{c}), and at a different wavenumber but the same frequency (\textbf{d}). White arrows indicate the electric field direction at the center of the silicon slab. \textbf{e}, Schematic of the photonic-crystal nanobeam bowtie cavity comprised of a central unit cell, tapering unit cells, $N_\text{c}$, and mirror unit cells, $N_\text{m}$.}
\label{fig:Cavity_design}
\end{figure}



\subsection{S2.1 Band structures and cavity design}
\label{subsec:bands}

The nanobeam photonic-crystal cavities we explore for self-assembly have a unit cell with a single-digit nanometer bowtie width at the center. Figure~\ref{fig:Cavity_design}a shows a triangular bowtie unit cell with a nanobeam width of $b$ = 700 nm, a lattice constant of $a$ = 400 nm, an air-bowtie width of $g$ = 2 nm, and a triangle width of $W_\text{b}$ = 116 nm. The bowtie angle is fixed to 90 degrees. The measured fabrication constraints, i.e., the smallest solid ($R_\text{S}$ = 10 nm) and void ($R_\text{V}$ = 20 nm) radii of curvature, are included in the cavity design to comply with our fabrication process, as shown in Fig.~\ref{fig:Cavity_design}a  . The band diagram of the unit cell is computed using a finite-element method in COMSOL Multiphysics and shown with solid black lines in Fig.~\ref{fig:Cavity_design}b, indicating that the second-order mode at the Brillouin-zone edge is a bowtie mode (Fig.~\ref{fig:Cavity_design}c). Fig.~\ref{fig:Cavity_design}b also includes the band structure for a unit cell with slightly modified parameters ($b$ = 700 nm, $a$ = 400 nm, $g$ = 2 nm, $W_\text{b}$ = 139 nm), which we define to be the mirror unit cell of our nanobeam cavity. We observe that the bowtie band at the edge of the Brillouin zone resides within the band gap of the mirror unit cell, which allows the generation of spatially-confined cavity modes by adiabatic tapering of the mirror unit cell into the center unit cell and back into the mirror unit cell. The band diagrams show that another mode (Fig.~\ref{fig:Cavity_design}d) is present at the same energy but at a lower wavenumber. These two modes, as shown with the superimposed vector fields in Figs.~\ref{fig:Cavity_design}c and d, have different symmetries, which ensures that a cavity design respecting the symmetry cannot couple the bowtie mode to the other modes. The exact cavity geometry is built following a well-known procedure~\cite{MarkoLoncar} in which the unit cell is continuously transformed along $N_\text{c}$ unit cells in both directions from the center to the mirror unit cell, followed by $N_\text{m}$ mirror unit-cell sections. Figure \ref{fig:Cavity_design}e shows the cavity geometry used for the waveguide-coupled photonic-crystal nanobeam cavities, which uses $N_\text{c}$ = 10 and $N_\text{m}$ = 8. Table.~\ref{tab:Waveguidecoupledcavity} shows the geometric parameters for the waveguide-coupled photonic crystal nanobeam cavity, notably the value of $W_\text{b}$ at the $i$-th defect unit cell, $W_\text{b, i}$. On the contrary, the cavity geometry used for the resonant scattering measurements of Fig. 4 in the main text uses a much shorter defect region with $N_\text{c}$ = 3 and $N_\text{m}$ = 16 and Table.~\ref{tab:Cavity_Scattering} shows the geometric parameters for that cavity.

\begin{table}[h]
\begin{tabular}{|c|c|}
\hline
\textbf{Parameter}  & \textbf{\quad Dimensions (nm) \quad} \\ \hline
Nanobeam width, $b$   & 700             \\ \hline
\quad Lattice constant, $a$ \quad & 400             \\ \hline
Air-bowtie width, $g$ & 2               \\ \hline
$W_\text{b,1}$           & 116             \\ \hline
$W_\text{b,2}$           & 116             \\ \hline
$W_\text{b,3}$           & 117             \\ \hline
$W_\text{b,4}$           & 117             \\ \hline
$W_\text{b,5}$          & 118             \\ \hline
$W_\text{b,6}$           & 119             \\ \hline
$W_\text{b,7}$          & 120             \\ \hline
$W_\text{b,8}$          & 122             \\ \hline
$W_\text{b,9}$          & 124             \\ \hline
$W_\text{b,10}$         & 126             \\ \hline
$W_\text{b,11}$         & 130             \\ \hline
$W_\text{b,12}$  - $W_\text{b,19}$         & 139             \\ \hline
\end{tabular}
\caption{Geometric parameters for the waveguide-coupled photonic crystal nanobeam cavity}
\label{tab:Waveguidecoupledcavity}
\end{table}

\begin{table}[]
\begin{tabular}{|c|c|}
\hline
\textbf{Parameter}     & \textbf{\quad Dimensions (nm) \quad} \\ \hline
\quad Nanobeam width, $b$  \quad    & 700                      \\ \hline
Lattice constant, $a$    & 400                      \\ \hline
Air-bowtie width, $g$    & 2                        \\ \hline
$W_\text{b,1}$             & 119                      \\ \hline
$W_\text{b,2}$            & 120                      \\ \hline
$W_\text{b,3}$               & 122                      \\ \hline
$W_\text{b,4}$               & 127                      \\ \hline
$W_\text{b,5}$  - $W_\text{b20}$  & 141                      \\ \hline
\end{tabular}
\caption{Geometric parameters for the photonic crystal nanobeam cavity for the resonant scattering measurements}
\label{tab:Cavity_Scattering}
\end{table}


\begin{figure}[t]
\centering
\includegraphics[width=1\textwidth]{Figures/Figures_SI/FigS_Oxide.pdf}
\caption[]{\textbf{The role of the native oxide on the modal properties of air bowtie nanocavities.} \textbf{a}, Schematic of the conformal native silicon oxide film (blue) in the bowtie region. \textbf{b}, Electric energy density in the air bowtie region of the central unit cell of a cavity with the same geometric parameters as that simulated for Fig. 5 in the main text and varying oxide thickness, $d$. \textbf{c, d, e} The wavelength, quality factor, and mode volume, respectively, of the simulated cavities as a function of $d$.}
\label{fig:Oxide}
\end{figure}

\subsection{S2.2 The role of the native oxide layer}
\label{subsec:oxide}
The simulations in Section~S2.1, which employ an idealized air-bowtie nanobeam structure, do not consider the spontaneously formed native oxide layer in the air-exposed boundaries of the fabricated structures. Through high-resolution scanning transmission electron microscopy (STEM) and electron energy-loss spectroscopy (EELS), we measure the thickness of such native oxide layer (at the bowtie tips) to be between 2 and 2.5 nm (see Subsection~S3.3), in good agreement with previous experimental observations~\cite{nasr2022effect} and ab-initio calculations~\cite{nativeoxide_bohling2016self}. The native oxide layer plays a negligible role for most photonic-crystal cavities, and has therefore generally been ignored in previous works, but the strongly localized fields of bowtie cavities make their modal properties very sensitive to the oxide layer thickness. We illustrate the strong influence of the oxide layer via simulations of the fabricated bowtie cavity for Fig.~5 in the main text with varying native oxide thickness, $d$. Figure~\ref{fig:Oxide}a shows a schematic of the geometry of the simulated bowties, the air gap of which is fixed at $g$ = 2 nm. The effect of the added oxide layer is to reduce the index contrast of the first interface from $\Delta n$ = $n_{\text{Si}}$ - $n_{\text{air}}$ to $\Delta n$ = $n_{\text{oxide}}$ - $n_{\text{air}}$, where the different indices are given by $n_{\text{Si}}$ = 3.48, $n_{\text{oxide}}$ = 1.45 and $n_{\text{air}}$ = 1. Adding the native oxide layer decreases the field intensity in the bowtie, as evidenced by the energy densities shown in Fig.~\ref{fig:Oxide}b. The impact of such field redistribution is negligible on the cavity quality factor, $Q$, but produces a pronounced blue-shift of approximately 14 nm per nm of oxide (Fig.~\ref{fig:Oxide}c) and, more importantly, leads to a more than two-fold increase in the effective mode volume evaluated at the bowtie center, $V$, between a realistic cavity with 2 nm of native oxide and the idealized cavity with no oxide (Fig.~\ref{fig:Oxide}e). We note that the oxide layer is neither included in the collapsed surfaces nor the top and bottom surfaces of the slab in order to limit the number of required mesh elements in the numerical model. 
The considerations of Fig.~\ref{fig:Oxide} were not taken into account in previous works on bowtie nanocavities and reported values of mode volumes from numerical simulations need not only to be regarded with care due to deleterious sources of error such as lightning-rod effects~\cite{Marcus} but also due to the impact of the surface oxide.

\section{S3. Self-assembly of air-bowtie cavities}
\label{sec:selfassembly}

In this section, we discuss the nanofabrication of few-nanometer gaps that can even go down to the atomic scale using conventional lithography and self-assembly.

\begin{figure}[t]
\centering
\includegraphics[width=\textwidth]{Figures/Figures_SI/SF3.png}
\caption[]{\textbf{Deterministic fabrication of few-nanometer bowtie widths.} \textbf{a, b}, Schematic of the electron-beam lithography mask for a bowtie unit cell. The exposed region is colored black and the grey region corresponds to silicon features after exposure and reactive ion-etching. The bowtie unit cell comprises two unconnected halves with a nanobeam gap, $g_\text{P}$, and a bowtie gap, $g_\text{T}$. The relative distance between point $T$ and the line $PP'$ is given by the offset $\delta$, which is 0 nm for (\textbf{a}) and 20 nm for (\textbf{b}). \textbf{c, d}, Top-view SEM images of bowtie unit cells fabricated using (\textbf{c}) $\delta$ = 0 nm and (\textbf{d}) $\delta$ = 20 nm, after lithography and plasma etching but before self-assembly. Due to the finite radius of curvature, $R_{\text{S}}$, and the critical dimension loss, $\Delta e$, associated to the fabrication process, $g_\text{P}$ becomes a fabricated gap of width $g_\text{f} = g_\text{P}+\Delta e$ and $g_\text{T}$ becomes $g_\text{b} = g_\text{T} + (\Delta e + R_\text{s})(\sqrt{2}-1)$. \textbf{e, f}, Tilted ($20 ^{\circ}$) SEM images of a bowtie unit cell with (\textbf{e}) $\delta$ = 0 nm and (\textbf{f}) $\delta$ = 20 nm after the self-assembly process. The bowtie with $\delta$ = 0 nm  leads to a pronounced air gap between the tips, and the latter self-assembles into a bowtie with tips in contact and a slot between the parallel surfaces of the nanobeam section.}
\label{fig:relativegap}
\end{figure}

\subsection{S3.1 Deterministic fabrication and size control of air bowties}
\label{sec:Bowtie}

\noindent In the cavity design described in Fig.~\ref{fig:Cavity_design}, all feature sizes except for the few-nanometer air-bowtie widths can be fabricated using conventional lithography and etching. Figures~\ref{fig:relativegap}a and b show schematics of the lithography mask used for the self-assembly of a single bowtie unit cell with a controlled air bowtie width. Relative to the final self-assembled geometry, the mask structure is composed of two unconnected regions that are separated by a gap of width $g_\text{P}$ along the flat parallel boundaries and by a gap of width $g_\text{T}$ between the two bowtie tips. We define the offset, $\delta$, as the relative distance between one of the bowtie tips ($T$) and the line defined by the flat edges ($PP'$), i.e., $\delta =(g_\text{T}-g_\text{P})/2$. For example, $\delta$ = 0 defines a bowtie unit cell where $T$, $P$ and $P'$ are co-linear, and $g_\text{T}$ and $g_\text{P}$ are equal. In contrast, $\delta$ = 20 nm defines a bowtie unit cell where the relative vertical distance between $PP'$ and $T$ is 20 nm, as shown in Figs.~\ref{fig:relativegap}a and b. The reason for having the tips closer to the central axis than the flat edges ($PP'$), i.e., $g_\text{P}>g_\text{T}$, is that the fabrication process rounds all sharp features to finite radii of curvature. We have estimated approximately $R_\text{S}$ = 10 nm for silicon features and $R_\text{V}$ = 20 nm for void features for our fabrication process~\cite{Marcus}. Therefore, the top and bottom tips of the bowties after fabrication are pushed away from each other, as shown in the top-view SEM images of Figs.~\ref{fig:relativegap}c and d, which are taken before underetching.

In addition to changing the offset $\delta$, we also adjust the bowtie dimensions to preserve its $90^{\circ}$ angle, which minimizes shot-filling and fracturing issues during electron-beam lithography. To control the size of the fabricated bowtie, we also take into account the commonly observed process-dependent uniform enlargement, $\Delta e$, of all exposed features. After self-assembly, the resulting bowtie width $g$ is given by
\begin{equation}
\label{eq:fabgap}
    g = 2((\Delta e + R_\text{s})(\sqrt{2}-1)-\delta)
\end{equation}
which determines the largest value of the offset $\delta$ that leads to the formation of a bowtie with an air gap at the tips, i.e., $g>0$, after the self-assembly process. While the solid radius of curvature is well approximated by $R_\text{S}$ = 10 nm, for our fabrication process, the feature growth $\Delta e$ varies from sample to sample with values between 10 nm and 15 nm for the samples in this work. This sets the largest offset that results in an air bowtie to a value between 8.3 nm and 10.4 nm according to Eq.~(\ref{eq:fabgap}). Since the resolution of the electron-beam lithography exposure grid for our fabrication is 1 nm, we expect self-assembled bowtie air gaps to appear for offsets below 11. Since the surface forces mainly originate from the parallel surfaces of the nanobeams, the offset is varied by keeping $g_\text{f}$ fixed at 50 nm while changing $g_\text{b}$. Examples of the resulting air bowties are shown in the tilted-view SEM images of Figs.~\ref{fig:relativegap}e and f for $\delta$ = 0 nm and $\delta$ = 20 nm. The former produces an air bowtie at the unit cell center while the tips are in contact for the latter, which showcases another type of application of the proposed self-assembly in which local protruding regions are used as stoppers, allowing the formation of few-nanometer-wide slot waveguides. In the next section, we demonstrate the deterministic fabrication of few-nanometer gaps by varying the offset $\delta$ between the two extreme cases discussed here.

\begin{figure}[t]
\centering
\includegraphics[width=0.95\textwidth]{Figures/Figures_SI/SF4.png}
\caption[]{\textbf{Characterization by SEM of few-nanometer bowtie widths.} Tilted-view ($20 ^{\circ}$) SEM images of an array of self-assembled bowties with offset $\delta$ varying from 1 nm (top-left) to 20 nm (bottom-right).}
\label{fig:Offset}
\end{figure}

\begin{figure}[t]
\centering
\includegraphics[width=0.95\textwidth]{Figures/Figures_SI/Fig_SI_SEM_v3.pdf}
\caption[]{\textbf{Extracting the relation between the mask offset, $\delta$, and the fabricated bowtie width.} \textbf{a}, Tilted-view ($20 ^{\circ}$) SEM image of a self-assembled air bowtie for $\delta$ = 10 nm. The 10 horizontal lines indicate cuts along which the gap $g$ is extracted via image analysis. \textbf{b}, Normalized intensity along the first cut shown in \textbf{a} and its derivative. The most prominent maxima and minima in the latter are used to extract the gap. \textbf{c} Histograms of the extracted gaps for offsets 0, 7, 14 and 20, from which the average gap are extracted and shown in \textbf{d}, which includes a linear fit (solid blue line).}
\label{fig:SEMextraction}
\end{figure}

\subsection{S3.2 Scanning electron microscope characterization of offset-to-width correspondence}
\label{subsec:SEMcharacterization}

\noindent We fabricate nanobeam cavities by varying $\delta$ from 0 to 20 nm in steps of one nanometer and acquire SEM images on large bowtie subsets to characterize the underlying relation between the offset $\delta$ and the bowtie width $g$ after self-assembly. Figure~\ref{fig:Offset} shows a representative high-resolution tilted SEM image of a bowtie unit cell as $\delta$ is changed from 1 nm (top-left) to 20 nm (bottom-right). First, the bowtie width monotonically narrows until a given offset $\delta_*$, where the bowtie tips touch and there is no void formation at the center. Due to the well-known systematic errors in SEM at the few-nanometer scale as well as bowtie-to-bowtie variations, the exact value of $\delta_*$ cannot be pinpointed exactly, e.g., it is between 13 nm and 15 nm for the set shown in Fig.~\ref{fig:Offset}. As $\delta$ further increases, the bowtie tips start protruding towards the centre from the parallel surfaces and act as stoppers in the directed collapse, generating a slot whose width, also denoted as $g$ for simplicicty, grows monotonically for even larger offsets. Note that, while all the structures in Fig.~\ref{fig:Offset} are fabricated and self-assembled in a single fabrication run, other fabrication runs lead to slightly different offset-to-width correspondence, as is for example the case for the structures reported in Fig. 4c in the main text. In that particular sample, the offsets are limited to $\delta$ = \{0,7,8,9,10,11,12,13,14,20\} nm, which are chosen to ensure covering the regime of few-nanometer air-bowtie cavities while including two cases in which the structure exhibits wide bowties ($\delta$ = 0 nm) and wide slots ($\delta$ = 20 nm) to help find the precise offset-to-width relation without being limited by the SEM resolution or artifacts. We image all the bowtie unit cells in the cavity region of several nanobeams fabricated with offsets 0, 7, 14 and 20 and extract the gap width in 10 positions across the device layer thickness for each image, as exemplified in Fig.~\ref{fig:SEMextraction}a. At each position, $g$ is extracted via edge detection using the maximum derivative points in a smoothed version of the SEM image intensity (Fig.~\ref{fig:SEMextraction}b). Figure~\ref{fig:SEMextraction}c depicts the histogram obtained for the 4 used offsets, while Fig.~\ref{fig:SEMextraction}d depicts a linear fit to the average gaps that we use to estimate the effective widths at all other values of $\delta$, notably the bowtie widths of 5.3 nm, 3.3 nm and 1.3 nm used for $\delta$ = 8, 9 and 10 in Fig. 4c and d in the main text. We highlight that the extracted slope of 2.03 is in excellent agreement with the expected value of 2 and that the curve intersects around $\delta_*$ = 11 nm, in good agreement with the estimate done via Eq. (\ref{eq:fabgap}).

\subsection{S3.3 Scanning transmission electron microscope characterization of bowtie widths}
\label{subsec:STEMcharacterization}

\begin{figure*}[t]
\centering
 \includegraphics[width=0.75\textwidth]{Figures/Figures_SI/SF12.pdf}
    \caption{\textbf{High-resolution STEM imaging of self-assembled nanobeam cavities.} \textbf{a}, Tilted ($20 ^{\circ}$) SEM image of a self-assembled nanobeam cavity with the surrounding frame designed for FIB-assisted lift-off. \textbf{b}, Transfer of a nanobeam to a STEM grid using a micromanipulator tip. \textbf{c}, Top-view STEM image of a self-assembled nanobeam cavity and a zoom-in into the central unit cell (blue box).
    }
    \label{fig:FIB}
\end{figure*}

 
 \begin{figure}[t]
\centering
\includegraphics[width=\textwidth]{Figures/Figures_SI/SF5.pdf}
\caption[]{\textbf{Characterization by STEM of atomic-scale bowtie widths.} Annular dark-field STEM images of the bowtie tips of the central unit cells in nanobeam cavities fabricated with $\delta$ from 8 nm to 12 nm. The (022) crystal planes of silicon are observed and indicated in the images.}
\label{fig:TEM}
\end{figure}
 


\noindent To precisely characterize the bowtie width for values of $\delta$ around $\delta_*$ and the native oxide layer, we fabricate self-assembled cavities for high-resolution STEM imaging. Since STEM requires thin layers of material to transmit electrons for analysis, individual nanobeam cavities are cut out from the wafer chip using a focused ion beam (FIB) and transferred to STEM-compatible grids (see Methods for details on the lift-off process for STEM imaging). In the sample dedicated to STEM, we employ an anchoring system that assists in cutting and lifting off the cavities from the chip (see Fig.~\ref{fig:FIB}a). We select $5$ devices from the sample with $\delta$ = 8, 9, 10, 11, and 12 nm and transfer them to the TEM grids using a motorized micro-manipulator as shown in Fig.~\ref{fig:FIB}b. Figure~\ref{fig:FIB}c shows a top-view STEM image of a self-assembled nanobeam cavity and its central bowtie unit cell. High-resolution STEM images of the central bowtie region for $\delta$ of 8 to 12 nm are shown in Fig.~\ref{fig:TEM}.

The signal intensity in STEM depends on composition, density, and thickness and it decreases gradually when transitioning from the crystalline silicon to the amorphous oxide and reducing to a background signal at the void region. Given the spatial resolution of approximately 0.1 nm of our STEM and the observations of the sidewall tilt and roughness discussed in the main text, we attribute the intensity drop to the bowtie geometry and composition, i.e., the thinner the probed thickness, the dimmer the intensity. We extract the thickness of the native oxide layer by considering its edges to be defined by the position where the crystalline lattice is no longer visible on the bottom half bowtie side and where the rate of change in intensity is maximum on the void side. Based on where the (022) lattice signal extends in the images, a 2-2.5 nm thick amorphous layer is measured at the edge of the structures, which we attribute to being the native silicon oxide layer. The exact measurement is complicated due to the dependence of signal intensity with probed thickness. As expected from the observed negative correlation between offset and bowtie width, a pronounced air gap (void region) is observed for $\delta$ = 8 nm. At offsets 11 and 12, the bowtie tips are in contact at the native oxide layers, therefore, the void region disappears. In between, the progressive change in the bowtie width is more complex than one would expect from the design rules described in Section~S3.1 due to a combination of the roughness of the bowtie tips and the bowtie shape in the vertical direction. 


\section{S4. Optical spectroscopy of air-bowtie nanocavities}

\subsection{S4.1 Far-field resonant scattering measurements}


\begin{figure}[t]
\centering
\includegraphics[width=0.85\textwidth]{Figures/Figures_SI/SF6.pdf}
\caption[]{\textbf{Cross-polarized resonant-scattering spectroscopy.} \textbf{a}, Tilted-view ($20 ^{\circ}$) SEM image of a self-assembled nanobeam cavity showing the polarization of excitation (green arrow), detection (red arrow) and the cavity mode (white arrow). \textbf{b}, Normalized measured scattered power with excitation and collection at the cavity center. The red box highlights the cavity resonance, also shown in \textbf{c}. A fit to a Fano lineshape is overlayed in red, from which we extract the resonance wavelength and quality factor.}
\label{fig:Fanofit}
\end{figure}

\noindent As detailed in the Methods section, we perform optical spectroscopy of the self-assembled nanocavities using far-field resonant scattering measurements. The probed cavity mode has a polarization in the far field that is mainly along the $y$-axis as indicated with the white arrow in the SEM image of Fig.~\ref{fig:Fanofit}a. We couple light into the cavity by exciting at normal incidence using linearly polarized light with a polarization of $45^{\circ}$ relative to the cavity mode polarization and collect light at $90^{\circ}$ relative to the excitation polarization. The excitation and collection with cross-polarization optimizes the cavity-to-background coupling efficiency. The cavity resonances appear as Fano resonances due to the interference between the high-$Q$ in-plane cavity resonance and the low-$Q$ out-of-plane resonance formed by the silicon membrane and the handle layer~\cite{Marcus}. The measured spectrum shown in Fig.~\ref{fig:Fanofit}b corresponds to one of the self-assembled cavities used for Fig.~4a in the main text ($\delta$ = 8 nm). After fitting a Fano lineshape to the resonant feature (Fig.~\ref{fig:Fanofit}c), we extract a resonant wavelength of $\lambda$ = 1521.5 nm and a quality factor of approximately $3.9\times 10^4$.


\begin{figure}[t]
\centering
\includegraphics[width=1\textwidth]{Figures/Figures_SI/SF14.png}
\caption[]{\textbf{Suspended photonic circuits with self-assembled nanobeam cavities.} Tilted ($20 ^{\circ}$) SEM image of a photonic circuit to characterize a self-assembled nanobeam cavity via in-plane transmission measurements.}
\label{fig:waveguidecoupledcavity}
\end{figure}

\subsection{S4.2 In-line transmission measurements of self-assembled nanobeam cavities}

\noindent We perform optical spectroscopy of waveguide-coupled self-assembled nanocavities using in-line transmission measurements, as detailed in the Methods section. Figure~\ref{fig:waveguidecoupledcavity} shows a tilted SEM image of an entire waveguide-coupled self-assembled nanobeam cavity device, including the two circular grating couplers used for cross-polarized and spatially resolved excitation/collection. As discussed in the main text, the spectra obtained on such cavities are normalized to that measured on a self-assembled waveguide, a characteristic SEM image of which is shown in Fig.~\ref{fig:selfassembledwaveguide}. The inset shows that the quality of the collapse is such that the interface is hardly visible. However, the observed line-edge roughness on the sidewalls also evidences that the proposed normalization is likely more appropriate than normalizing with a conventional suspended waveguide. To demonstrate the robustness of our self-assembly method, we fabricate 4 nominally identical circuits for the waveguide-coupled self-assembled nanocavity shown in the main text Fig.~5. Fig.~\ref{fig:Clones} shows the normalized transmission measurements for the set of copies, including a measurement of a nanobeam waveguide of the same length and with the unit cell corresponding to that of the cavity mirrors. For all the cavities, we observe a reduction of the quality factor and the on-resonance transmission relative to the simulated values, respectively $Q$ = $4\times10^{4}$ and $T_\text{o}$ = 0.96. We attribute the former drop to the effect of the self-assembled interface in the collapsed regions and to the sidewall roughness around the air bowties where the cavity field is most intense. That same drop partly explains the drop in $T_\text{o}$, which also occurs due to the differential bowing of the structure observed for the nanobeam cavities and the nanobeam waveguide, i.e., the normalization is not perfect. In addition, both the measured $Q$ and $T_\text{o}$ fluctuate considerably, indicating a subtle interplay between the disorder-induced losses in the structure plane and the out-of-plane direction.



\begin{figure}[t]
\centering
\includegraphics[width=0.9\textwidth]{Figures/Figures_SI/SF15.png}
\caption[]{\textbf{Suspended self-assembled nanobeam waveguides.} Tilted ($20 ^{\circ}$) SEM image of a self-assembled nanobeam waveguide, the transmission of which is used to normalize the transmission measurements on self-assembled cavities. The red box is a zoom-in of the central part of the self-assembled waveguide.}
\label{fig:selfassembledwaveguide}
\end{figure}

\begin{figure}[t]
\centering
\includegraphics[width=0.9\textwidth]{Figures/Figures_SI/SF18.pdf}
\caption[]{\textbf{Optical spectroscopy of nominally identical waveguide-coupled self-assembled nanobeam cavities.} Transmittance spectra of 4 nominally identical bowtie cavities (Cavity) and bowtie waveguides (Mirror) with an approximate bowtie width of 2 nm. On each panel, a red box highlights the fundamental cavity resonance, a Lorentzian fit to which is shown separately. The extracted loaded quality factor and resonant wavelength are indicated.}
\label{fig:Clones}
\end{figure}

\clearpage



%\bibliographystyle{naturemag-etalnoitalics.bst}% for et al in non italics
%\bibliography{ref}


\begin{thebibliography}{11}

\bibitem{Gusso_DispersionForces}
Gusso, A. \& Delben, G. J. Dispersion force for materials relevant for micro-and nanodevices fabrication. \emph{J. Phys. D} \textbf{41}, 175405 (2008).

\bibitem{casimir1948attraction}
Casimir, H. B. On the attraction between two perfectly conducting plates. In \emph{Proc. Kon. Ned. Akad. Wet}., vol. 51, 793 (1948).

\bibitem{moazzami2021self}
Moazzami Gudarzi, M. \& Aboutalebi, S. H. Self-consistent dielectric functions of materials: Toward accurate computation of Casimir-van der Waals forces. \emph{Sci. Adv.} \textbf{7}, eabg2272 (2021).

\bibitem{MarkoLoncar}
Quan, Q. \& Loncar, M. Deterministic design of wavelength scale, ultra-high Q photonic crystal nanobeam cavities. \emph{Opt. Express} \textbf{19}, 18529-18542 (2011).

\bibitem{nasr2022effect}
Nasr Esfahani, M. et al. Effect of native oxide on stress in silicon nanowires: Implications for nanoelectromechanical systems. \emph{ACS Appl. Nano Mater.} \textbf{5}, 13276-13285 (2022).

\bibitem{nativeoxide_bohling2016self}
Bohling, C. \& Sigmund, W. Self-limitation of native oxides explained. \emph{Silicon} \textbf{8}, 339-343 (2016).

\bibitem{Marcus}
Albrechtsen, M. et al. Nanometer-scale photon confinement in topology-optimized dielectric cavities. \emph{Nat. Commun.} \textbf{13}, 6281 (2022).


\end{thebibliography}

\end{document}
