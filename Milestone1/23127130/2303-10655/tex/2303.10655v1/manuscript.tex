%% LyX 2.3.6.1 created this file.  For more info, see http://www.lyx.org/.
%% Do not edit unless you really know what you are doing.
\documentclass[letterpaper, english,reprint, aps,prl]{revtex4-2}
\usepackage[T1]{fontenc}
\setcounter{secnumdepth}{3}
\usepackage{amsmath}
\usepackage{amssymb}
\pdfoutput=1
\usepackage{babel}
\usepackage{graphicx}



\begin{document}
\title{Quantum metrology, criticality, and classical brachistochrone problem}

\author{Rui Zhang$^1$}
\author{Zhucheng Zhang$^{2}$ }
\email{11936009@zju.edu.cn}
\author{Lei Shao$^2$}
\author{Yuyu Zhang$^3$}
\author{Xiaoguang Wang$^{4,}$}
\email{xgwang@zstu.edu.cn}
\affiliation{$^1$Zhejiang Institute of Modern Physics, School of Physics, Zhejiang University, Hangzhou 310027, China}
\affiliation{$^2$Graduate School of China Academy of Engineering Physics, Beijing 100193, China}
\affiliation{$^3$Department of Physics, Chongqing University, Chongqing 401330, China}
\affiliation{$^4$Key Laboratory of Optical Field Manipulation of Zhejiang Province
and Department of Physics, Zhejiang Sci-Tech University, Hangzhou 310018, China}
\date{\today}
	
\begin{abstract}
There has always been an ambiguous connection between quantum metrology and criticality, which is mysterious and fascinating. We clarify this relationship in a unitary parametrization process with a Hamiltonian governed by su(1,1) Lie algebra. Based on this type of Hamiltonian, we investigate the quantum Cram\'{e}r-Rao bound of the coupling strength in the quantum Rabi model close to the phase transition point. We show that the Hermitian operator of the unitary parametrization process can be treated as an extended brachistochrone problem on the $x-y$ plane and a linear function of time in the $z$ direction. In addition, we find that the value of quantum Fisher information is proportional to the sixth power of the evolution time when the system is close to the phase transition point.
\end{abstract}
\maketitle
	
\textit{Introduction.}--Metrology is widely used in research and with the development of quantum mechanics, quantum effects play a significant role in the process of metrology, resulting in quantum metrology \cite{fisher1923xxi,cramer1999mathematical,rao1945information,PhysRevLett.96.010401,liu2020quantum,liu2022optimal}. Specifically, quantum metrology is concerned with the highest accuracy that can be achieved in the estimation problems of various parameters.	Quantum metrology begins with the need for practice and because of the efficiency in improving measurement accuracy, it has been applied in different fields, such as detecting gravitational waves \cite{braginsky2004corner,adhikari2014gravitational,mcguirk2002sensitive}, atomic clocks \cite{andre2004stability,kessler2014heisenberg,borregaard2013near}, and quantum-enhanced positioning \cite{giovannetti2004quantum,braun2018quantum}. 

Quantum Fisher information (QFI) is a fundamental concept of quantum metrology, which is defined by maximizing the Fisher information in all possible measurements \cite{braunstein1994statistical,paris2009quantum}. However, the analytical expression of QFI is difficult to obtain in most cases. For a general unitary parametrization process with a time-dependent unitary evolution operator $U=\exp(-itH)$ and an initial state $|\psi\rangle$, where the Hamiltonian $H$ is time-independent but of parameter $\lambda$, the QFI can be derived as the variance of a Hermitian operator $\mathcal{H}\equiv i(\partial_\lambda U^{\dagger})U$ in the initial state \cite{pang2014quantum,liu2015quantum}. This implies that the analytical expression of QFI depends on the relation between the estimated parameter $\lambda$ and the Hamiltonian $H$. For $\lambda$ as an overall multiplicative factor of Hamiltonian, the Hermitian operator can be derived as $\mathcal{H}=-t\partial_{\lambda}H$ \cite{liu2015quantum}; meanwhile, Pang \emph{et al.} \cite{pang2014quantum} and Liu \emph{et al.} \cite{liu2015quantum} obtained the Hermitian operator $\mathcal{H}$ for a more general Hamiltonian parameter.

The QFI has been shown to be closely related to many physical phenomena. According to the quantum Cram\'{e}r-Rao bound, the QFI is the inverse of the variance of the estimated parameter \cite{cramer1999mathematical,rao1945information}, which means that the larger the QFI, the more accurately we can evaluate parameters. In addition, due to some excellent mathematical properties, the QFI is also connected to the geometry of quantum mechanics \cite{braunstein1994statistical,twamley1996bures,zanardi2007bures,zanardi2008quantum,facchi2010classical,liu2014fidelity,yuan2016sequential} as well as certain behaviors and phenomena in quantum dynamics \cite{toth2014quantum,taddei2013quantum,deffner2017quantum,gessner2018statistical}. Recently, the relationship between quantum metrology and criticality has attracted great attention \cite{PhysRevA.88.021801,PhysRevA.93.022103,PhysRevA.78.042105,PhysRevE.93.052118,PhysRevA.96.013817,PhysRevX.8.021022,PhysRevLett.123.173601,PhysRevLett.117.110802,PhysRevA.102.023512,PhysRevA.88.023803,PhysRevA.78.042106,PhysRevLett.124.120504,PhysRevLett.126.010502}, which mainly focuses on whether the critical point will cause a sharp increase in QFI. However, criticality-enhanced quantum metrology still has ambiguities. For example, in a time evolution with the system close to a critical point, is the value of QFI divergent or a function of time?
	
In this Letter, we clarify this problem in a unitary parametrization process with a Hamiltonian governed by su(1,1) Lie algebra. With the one-mode bosonic realization of su(1,1) algebra, we evaluate the coupling strength in the quantum Rabi model (QRM) close to the phase transition point. We analytically obtain the expression of $\mathcal{H}$ and find that it is actually a classical brachistochrone problem on the $x-y$ plane and a linear function of time in the $z$ direction. What is more, we show that the QFI of the coupling strength is a function that depends on the time and is proportional to the sixth power of time.

\textit{$\mathcal{H}$ in su(1,1) Lie algebra and the brachistochrone problem.}--We introduce the vector $\boldsymbol{\mathrm{K}}=(K_{x},K_{y},K_{z})$ as the generator
of su(1,1) Lie algebra, then a general form of Hamiltonian conforming to the structure can be written as 
\begin{equation}
	H_{\lambda}=\boldsymbol{\mathrm{\Lambda }}\cdotp\boldsymbol{\mathrm{K}}, \label{eq:su11H}
\end{equation}
where the vector $\boldsymbol{\mathrm{\Lambda}}=(\lambda _{1},\lambda _{2},\lambda _{3})$ is independent of time but of parameter $\lambda$. Partial derivatives of the Hamiltonian with respect to the parameter $\lambda$ reads $\partial_{\lambda}H_{\lambda}=\boldsymbol{\mathrm{\Gamma }}\cdotp\boldsymbol{\mathrm{K}}$ and $\boldsymbol{\mathrm{\Gamma}}=\partial_{\lambda}\boldsymbol{\mathrm{\Lambda}}$. In our paper, we redefine the cross product and the dot product as \cite{SM}: $\boldsymbol{\mathrm{a}}\boxtimes\boldsymbol{\mathrm{b}}=(a_2b_3-a_3b_2, a_3b_1-a_1b_3, a_2b_1-a_1b_2)$ and $\boldsymbol{\mathrm{a}}\boxdot\boldsymbol{\mathrm{b}}=(a_1b_1+a_2b_2-a_3b_3)$, respectively, with $\boldsymbol{\mathrm{a}}=(a_1,a_2,a_3)$	 and $\boldsymbol{\mathrm{b}}=(b_1,b_2,b_3)$. Utilizing the commutation relation, $[\boldsymbol{\mathrm{a}}\cdot\boldsymbol{\mathrm{K}},\boldsymbol{\mathrm{b}}\cdot\boldsymbol{\mathrm{K}}]=i(\boldsymbol{\mathrm{a}}\boxtimes\boldsymbol{\mathrm{b}})\cdot\boldsymbol{\mathrm{K}}$, in the process of unitary parameterization, the Hermitian operator $\mathcal{H}$ is derived as \cite{SM}
\begin{equation}
\mathcal{H} =\boldsymbol{\mathrm{\alpha}}\cdotp\boldsymbol{\mathrm{\beta}}.
\label{hermitian h}
\end{equation}
Here,  $\boldsymbol{\mathrm{\alpha}}=(x(t),y(t),z(t))$ is a coefficient vector and its specific form reads
\begin{align}
x(t) & =\left[t|\boldsymbol{\mathrm{\Lambda}}|-\sinh(t|\boldsymbol{\mathrm{\Lambda}}|)\right]/i, \label{eq:Brach1}\\
y(t) & =1-\cosh(t|\boldsymbol{\mathrm{\Lambda}}|), \label{eq:Brach2}\\
z(t) &=t,
\label{eq:Brach3}
\end{align}
with $\boldsymbol{\mathrm{|\Lambda |}}=\ensuremath{\sqrt{\lambda_{1}^{2}+\lambda_{2}^{2}-\lambda_{3}^{2}}}$. The operator $\boldsymbol{\mathrm{\beta}}=\left(-i\frac{H_\lambda^{\times 2}\left(\partial_\lambda H_\lambda\right)}{|\boldsymbol{\Lambda}|^3}, i \frac{H_\lambda^{\times 1}\left(\partial_\lambda H_\lambda\right)}{|\boldsymbol{\Lambda}|^2}, - \partial_\lambda H_\lambda\right)$, where the superoperator $H_\lambda^{\times n}\left(\partial_\lambda H_\lambda\right)$ is an nth-order nested commutator operation with  $\partial_\lambda H_\lambda$, i.e., $H_\lambda^{\times n}\left(\partial_\lambda H_\lambda\right)=\left[H_\lambda, \cdots,\left[H_\lambda, \partial_\lambda H_\lambda\right]\right]$. Note that $\boldsymbol{\mathrm{|\Lambda |}}$ can be a pure imaginary number in an actual physical system (see below), i.e., $\boldsymbol{\mathrm{|\Lambda |}}\rightarrow i\boldsymbol{\mathrm{|\Lambda |}}$. In this case, $x(t)\rightarrow t|\boldsymbol{\mathrm{\Lambda}}|-\sin(t|\boldsymbol{\mathrm{\Lambda}}|)$ and $y(t)\rightarrow 1-\cos(t|\boldsymbol{\mathrm{\Lambda}}|)$.  We can see that there is a corresponding relationship between the $x$ and $y$ components of the vector $\boldsymbol{\mathrm{\alpha}}$ and the brachistochrone problem (i.e., cycloid equation). Meanwhile, the period of this cycloid equation is $\tau=2\pi/|\boldsymbol{\mathrm{\Lambda}}|$. Thereby, the Hermitian operator $\mathcal{H}$ of the general form Hamiltonian in su(1,1) algebra can be treated as the extended brachistochrone problem on the $x-y$ plane and the linear function of time in the $z$ direction. In addition, for the Hamiltonian with su(2) algebra, one can also obtain similar results \cite{PhysRevA.92.012312}. 

In the unitary parametric process, the whole information of the system is contained in the Hermitian operator $\mathcal{H}$. With this Hermitian operator and the given initial state, the QFI of the estimated parameter can be obtained. Above we have shown that $\mathcal{H}$ in Lie algebra is essentially a brachistochrone problem. According to the definition of $\mathcal{H}$, we can find that $\mathcal{H}$ and the unitary evolution operator $U$ satisfy a Schr\"odinger-like equation, i.e., 
\begin{align}
	i\partial _\lambda U^\dagger =\mathcal{H} U^\dagger.
\end{align}
Besides, based on the Carm\'er-Rao bound and the definition of QFI, an uncertain relationship between estimated parameter $\lambda$ and $\mathcal{H}$ can be obtained as
\begin{align}
\Delta \lambda \cdotp \Delta \mathcal{H} \geq \frac{1}{2\sqrt{\nu}} ,
\end{align}
where $\Delta o=\sqrt{\left\langle o^2 \right\rangle -\left\langle o \right\rangle^2 }$ represents the standard deviation with $o=\lambda,   \mathcal{H}$, and $\nu$ is the number of measurement, as shown in the flow diagram of Fig.~\ref{fig1}(a). This uncertain relationship restricts the precision of the estimated parameter $\lambda$ and $\mathcal{H}$.

\begin{figure}
	\begin{centering}
		\includegraphics[scale=0.43]{fig1a}
		\includegraphics[scale=0.6]{fig1b}
	\end{centering}
	\caption{The image above is a flow diagram that includes a Hamiltonian $H_\lambda$ with the general form of su(1,1) Lie algebra, a Hermitian operator $\mathcal{H}$, and a parameter-dependent uncertainty relationship.  Specifically, the analytic expression of the Hermitian operator $\mathcal{H}$ can be regarded as a brachistochrone problem on the $x-y$ plane and a monotone function of time in the $z$ direction (see Eqs.~(\ref{eq:Brach1})-(\ref{eq:Brach3})). The image below provides a three-dimension graph and its projection on the $x-y$ plane for the Hermitian operator $\mathcal{H}$ in the isotropic quantum Rabi model (i.e., $\zeta=1$) with different effective coupling strengths $(\tilde{g}=0.950, 0.980,0.990,0.999)$. Here the evolution time $t$ ranges from $0$ to $\pi/\omega\sqrt{1-\tilde{g}^2}$ with $\tilde{g}=0.950$. }
	\label{fig1}
\end{figure}


% Also, $\nu$ is the time of measurement and it is worth mentioning that $\Delta \mathcal{H}$ and $\Delta \lambda $ is a normal numbers, not an operator.  
	%Two-dimension(2D) and three-dimension(3D) images about the components of the vector $\boldsymbol{\mathrm{\alpha}}$ was shown on the right in Fig.(\ref{fig1}).  
%	The picture on the $x-y$ plane about the function with respect to time can obviously display the criticality in an actual model, the value of $X(t)$ and $Y(t)$ suddenly get bigger with a  subtle change in a fixed $t$ near the critical point 1. 
%	With a general form of the Hamiltonian satisfying su(1,1) lie algebra, we easily obtain the hyperbolic coefficient function and the eigenoperator in $\mathcal{H}$ and this form is an extension cycloid equation. In addition, almost the same form of the cycloid equation can be seen in the structure of su(2), the most difference is the hyperbolic and trigonometric function, but there is commonality through transformation between them. Both of them clearly describe the relationship between QFI and quantum criticality in Lie algebra.
%We take the isotropic and anisotropic quantum Rabi models as an example to show the results about Hermitian and quantum criticality. As is well-known, the difference in the isotropic and anisotropic Rabi models is the intensity of coupling between photon and atom in the rotating wave term (RW) and anti-rotating wave term (CRW). 	
%	using a unitary transformation, $U=\exp[(g/\Delta )(a^\dagger +a)(\sigma_{+}-\sigma_{-})]$ makes the  transformed Hamitonian eliminates the coupling terms and forms a diagonal block {\cite{PhysRev.149.491,PhysRevLett.115.180404,PhysRevLett.107.100401}}. Then the effective Hamiltonian in down spin subspaces of $\left|\downarrow\right\rangle $ with the lowest energy, the effective Hamiltonian reads {\cite{PhysRevLett.119.220601,PhysRevA.104.043307}} 
	
\textit{Criticality in the quantum Rabi model.}--Many studies have shown that there is a close relationship between quantum metrology and criticality, and the criticality may be a useful resource to enhance the accuracy of parameter evaluation. In this section, we analyze the relationship between quantum metrology and criticality from the perspective of Lie algebra structure. Quantum Rabi model (QRM) is a convenient model for parameter evaluation under different algebraic structures. The Hamiltonian of QRM reads ($\hbar =1$)
	\begin{align}
		H_{\mathrm{Rabi}} & =\omega a^{\dagger}a+\frac{\Omega}{2}\sigma_{z}+H_{c}, \\
		H_{\mathrm{c}} & =g\left[(\sigma_{+}a+\sigma_{-}a^{\dagger})+\zeta(\sigma_{+}a^{\dagger}+\sigma_{-}a)\right].
	\end{align}
Here, $a^{\dagger} (a)$ is the creation (annihilation) operator of the bose mode with frequency $\omega$; $\sigma_{k}(k=x,y,z)$ and $\sigma_{\pm}$ are the Pauli operators, the raising and lowering operators of the two-level system with transition frequency $\Omega$; $g$ and $\zeta$ represent the coupling strength and the ratio between  rotating and counterrotating terms. When $\zeta=1$, this model is reduced to the isotropic QRM.

Using the Schrieffer-Wolff (SW) transformation with an operator  $U_{\mathrm{sw}}=\exp\left\{-i(\tilde{g}/\Omega)\left[(1-\zeta)\sigma_x p+(1+\zeta)\sigma_y x\right]\right\}$, and in the limit of $\Omega/\omega\rightarrow\infty$,  an effective Hamiltonian in the spin-down subspace can be derived as \cite{SM,PhysRevLett.119.220601,PhysRevLett.115.180404}, 
\begin{align}
	H_{\mathrm{eff}}^{(\downarrow)}=\frac{\omega}{2}(1-g_1^2)x^2+\frac{\omega}{2}(1-g_2^2)p^2,
	\label{sdh}
\end{align}	
where we have defined the dimensionless position and momentum operators $x=(a+a^\dagger)/\sqrt{2}$ and $p=i(a^\dagger -a)/\sqrt{2}$. Here, $g_1=\tilde{g}(1+\zeta)/2$, $g_2=\tilde{g}(1-\zeta)/2$, with $\tilde{g}=2g/\sqrt{\Omega\omega}$ denoting the effective coupling strength. With the change of $\tilde{g}$, the QRM will 
undergo quantum phase transition at the point $\tilde{g}=\tilde{g}_c=2/(1+|\zeta|)$ \cite{PhysRevLett.119.220601}, and we focus on the case of $\tilde{g}<\tilde{g}_c$ below. According to the one-mode bosonic realization of su(1,1) algebra, the effective Hamiltonian (\ref{sdh}) can be rewritten as \cite{SM}
\begin{align}
	H_{\mathrm{eff}}^{(\downarrow)} =2\omega K_z +\frac{\omega}{2}\lambda \left[K_z(1+\zeta^2)+2\zeta K_x\right]\equiv H_{\lambda}
	\label{suh}
\end{align}  
with $\lambda=-\tilde{g}^{2}$.  Based on the general form of Hamiltonian in Eq.~(\ref{eq:su11H}), we have $\boldsymbol{\mathrm{\Lambda}}=\left(\zeta\omega\lambda, 0, \omega[2+\lambda(1+\zeta^2)/2]\right)$ and $|\boldsymbol{\mathrm{\Lambda}}|=\omega \sqrt{\lambda^2 \zeta^2-[2+\lambda(1+\zeta^2)/2]^2}$. In fact, the value of $|\boldsymbol{\mathrm{\Lambda}}|$ is a pure imaginary number; for instance, $|\boldsymbol{\mathrm{\Lambda}}|=2\omega i\sqrt{1+\lambda}$ if $\zeta=1$. It can be seen from Eqs.~(\ref{eq:Brach1})-(\ref{eq:Brach3}) that the Hermitian operator $\mathcal{H}$ in this model can be treated as an extended brachistochrone problem on the $x-y$ plane and the linear function of time in the $z$ direction. For the case of $\zeta=1$, the period of cycloid equation becomes $\tau=\pi/\omega\sqrt{1+\lambda}$, which implies that the period will become infinite when the QRM operates in the phase transition point. As shown in Fig.~\ref{fig1}(b), we plot a three-dimension graph and its projection on the $x-y$ plane for the Hermitian operator $\mathcal{H}$ of the isotropic QRM with different effective coupling strengths $\tilde{g}$. We can vividly see that with the increase of the effective coupling strengths, the curve becomes shorter and shorter, which shows that we need more time to obtain an entire periodogram. In other words, the evolution time will tends to infinite when the coupling strength is close to the phase transition point.

\begin{figure}
	\begin{centering}
		\includegraphics[scale=0.5]{fig2}
	\end{centering}
	\caption{(a): Quantum Cram\'{e}r-Rao bounds of the effective coupling strength $\tilde{g}$ are plotted as a function of $\tilde{g}$ with different rations $\zeta$, where the evolution time is fixed as $t=\pi/\omega$. (b): Quantum Cram\'{e}r-Rao bound of the isotropic QRM ($\zeta=1$) is plotted as a function of the evolution time $t$ with a fixed coupling strength $\tilde{g}=0.9$ and the corresponding evolution period $\tau=\pi/\omega\sqrt{1-\tilde{g}^2}$. The inset describes the cycloid equation in the $x-y$ plane. The initial state of system is a product state between the two subsystem, that is, the two-level system is in its spin-down state $|\downarrow\rangle$ and  the bose field is $(|0\rangle+|1\rangle)/\sqrt{2}$.}\label{fig.2}
\end{figure} 

With the expression of Hamiltonian in Eq.~(\ref{suh}), the first two commutators in the superoperator $H_\lambda^{\times n}\left(\partial_\lambda H_\lambda\right)$ can be derived as
\begin{align}
	H_{\lambda}^{\times 1}(\partial_{\lambda}H_{\lambda}) & =2i\zeta\omega^{2}K_{y}, \\
	H_{\lambda}^{\times 2}(\partial_{\lambda}H_{\lambda}) & =2\zeta\omega^{3}\!\!\left\{\!\zeta \lambda K_z\!+\![2+\lambda(1+\zeta^2)/2]K_x\!\right\},\label{eq:H23}
\end{align}
respectively, with $\partial_{\lambda}H_{\lambda}=(\omega/2)[(1+\zeta^2)K_{z}+2\zeta K_{x}]$. Then, with the result of the Hermitian operator $\mathcal{H}$, the QFI about the effective coupling strength $\tilde{g}$ can be obtained as
\begin{align}
	F_{\tilde{g}}=(\partial_{\tilde{g}}\lambda)^{2}F_{\lambda}=16\tilde{g}^{2}\Delta^{2}\mathcal{H},
	\label{Fg}
\end{align}
where $F_{\lambda}$ is the QFI about the parameter $\lambda$, and the corresponding quantum Cram\'{e}r-Rao bound is given by
\begin{equation}
	\Delta \tilde{g} \geq 1/\sqrt{F_{\tilde{g}}}.
\end{equation}
Through specific analytical derivation \cite{SM}, we find that the uncertainty $\Delta \tilde{g}$ is actually a function that depends on the evolution time $t$, the effective coupling strength $\tilde{g}$ and the ratio $\zeta$. At first, we numerically simulate the relation between the uncertainty $\Delta \tilde{g}$ and the effective coupling strength $\tilde{g}$ for different $\zeta$, where the evolution time is fixed as $t=\pi/\omega$, as shown in Fig.~\ref{fig.2}(a). From the curves, we can see that for different $\zeta$, the value of the uncertainty $\Delta \tilde{g}$ decreases with the increase of $\tilde{g}$. Meanwhile, when the coupling strength $\tilde{g}$ is close to the phase transition point $\tilde{g}_c$ (i.e., $\frac{2}{3}$ for $\zeta=2.0$, $1$ for $\zeta=1.0$ and $\frac{4}{3}$ for $\zeta=0.5$), the uncertainty $\Delta \tilde{g}$ can obtain its minimum. Then, we analyze the relation of the uncertainty $\Delta \tilde{g}$ with the evolution time $t$ in the isotropic QRM ($\zeta=1$) for a given coupling strength $\tilde{g}=0.9$, as shown in Fig.~\ref{fig.2}(b). We can find that there is a close relation between the evolution of the uncertainty $\Delta \tilde{g}$ and the brachistochrone problem. Specifically, the value of $\Delta \tilde{g}$ periodically decreases with time $t$, and the period is consistent with the cycloid equation (see the inset of figure). One can also analyze the case for a bigger coupling strength $\tilde{g}$. In this case, the period $\tau$ becomes longer and tends to infinite when the coupling strength $\tilde{g}$ closes to the phase transition point $\tilde{g}_c$.

\begin{figure}
	\begin{centering}
		\includegraphics[scale=0.5]{fig3}	
	\end{centering}
	\caption{(a): QFIs $F_{\tilde{g}}$ of the isotropic QRM ($\zeta=1$) are plotted as a function the evolution time $t$ with different coupling strengths $\tilde{g}$ (i.e., $\tilde{g}=0.9,0.99,0.999,0.9999$). Here the value of the evolution period $\tau$ is consistent with that in Fig.~\ref{fig.2}. (b): Factor $A$ in the asymptotic expression of the QFI is plotted as a function of the ratio $\zeta$.}
	\label{fig.3}
\end{figure}

Above we have analyzed the evolution of the uncertainty $\Delta \tilde{g}$ with the coupling strength $\tilde{g}$ and the time $t$. Especially, we find that given a fixed coupling strength $\tilde{g}$, the evolution period of the uncertainty $\Delta \tilde{g}$ is consistent with the cycloid equation. Now, we investigate the properties of the system close to the phase transition point. With the result of the QFI $F_{\tilde{g}}$ in Eq.~(\ref{Fg}), we find that when the coupling strength $\tilde{g}$ tends to the phase transition point $\tilde{g}_c$, an asymptotic expression can be obtained as \cite{SM}
\begin{equation}
F_{\tilde{g}}\simeq A\omega^6 t^6,\quad \tilde{g}\rightarrow \tilde{g}_c,
\end{equation}
with the factor $A=320\zeta^4/9(1+\zeta)^6$. For the isotropic QRM, this asymptotic expression is reduced as $F_{\tilde{g}}\simeq 5\omega^6 t^6/9$. We can find that when the coupling strength tends to its critical value, the QFI is proportional to the sixth power of the evolution time $t$. This implies that when approaching the phase transition point, the value of the QFI \emph{does not} show divergent feature, but is only a time-dependent function. Meanwhile, This time-dependent function is also related to the ratio $\zeta$. To better illustrate this feature, we numerically simulate the evolution of the QFI of the isotropic QRM with time, as shown in Fig.~\ref{fig.3}(a). From the curves, we can vividly see that when the coupling strength tends to the phase transition point, i.e, $\tilde{g}\rightarrow 1$, the QFI $F_{\tilde{g}}$ curve fits to the asymptotic expression gradually. In addition, we also simulate the evolution of the factor $A$ in the asymptotic expression with the ratio $\zeta$, as shown in Fig.~\ref{fig.3}(b). We can see that there is a maximum for the factor $A$ when the ratio $\zeta=2$. This implies that for the anisotropic QRM with $\zeta=2$, the value of the QFI $F_{\tilde{g}}$ can obtain a maximum when the system tends to the phase transition point.

\textit{Conclusion}--In summary, we investigate a unitary parametrization process governed by an su(1,1) Hamiltonian, and clarify the relationship between quantum metrology and criticality from the perspective of Lie algebra structure. We show that the Hermitian operator $\mathcal{H}$ in the su(1,1) Hamiltonian can be treated as an extended brachistochrone problem on the $x-y$ plane and the linear function of time in the $z$ direction. In addition, we find that the uncertainty $\Delta\tilde{g}$ in the quantum Rabi model is actually a function that depends on the evolution time $t$, the effective coupling strength $\tilde{g}$ and the ratio $\zeta$. We also show that when the system tends to the phase transition point, the QFI $F_{\tilde{g}}$ is proportional to the sixth power of the evolution time, meanwhile, it can obtain a maximum when the ratio $\zeta=2$.


\begin{acknowledgments}
This work was supported by the National Natural Science Foundation of China (Grants No.~11935012).
\end{acknowledgments}
	
\bibliographystyle{apsrev4-2}
\bibliography{manuscriptRef}
\end{document}
