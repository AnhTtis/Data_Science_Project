\section{Case Study}

\subsection{Hard Samples}
\label{subsec: case study for hard samples}
Table~\ref{tab: case} showcases some \textit{hard} examples which benefits from our LLM reranking. In accordance with our intuition, we observe that the LLM rerankers correct two kinds of erroneous predictions made by LLMs. (1) The lack of external knowledge, such as the first (\textit{Triptolemus is a figure in Greek mythology}) and third examples (\textit{Minas Gerais is a state instead of city}). (2) Limited reasoning abilities, such as the second (\textit{His wife's children are his children}) and the fourth (\textit{The word "fought" in this sentence does not involve any physical violence}) examples.

\subsection{Easy Samples}
\label{subsec: case study for easy samples}

As discussed in Section~\ref{subsec: Why LLMs Fail on Easy Samples}, we attribute the inferior performance of LLMs on easy samples to their false-positive predictions for negative samples. Upon examining the incorrect cases, we speculate that two main reasons are responsible: (1) LLM hallucination and (2) span boundary mismatch. We provide details on them with specific cases below.

\noindent \textbf{LLM Hallucination}. The most distinctive mistakes made by LLMs. Without specific fine-tuning on IE tasks, LLMs (with in-context learning approaches) sometimes hallucinate and provide meaningless, even confusing explanations or answers. For instance, LLMs might predict a verb, preposition or even special symbols as an entity.

\noindent \texttt{Case 1}: LLMs predict the word \textit{die} as entity \texttt{person-other} (Ground truth: \texttt{None})

\begin{tcolorbox}[colback=white, colframe=black, boxrule=1pt, arc=0mm, boxsep=5pt, left=6pt, right=6pt, top=6pt, bottom=6pt]

\noindent \textbf{Sentence:} Flavius Rumoridus (died 5th century AD) was a Roman soldier who was appointed consul in AD 403 in the Western Roman Empire at the same time as the eastern emperor Theodosius II served in the same capacity in the East.

\vspace{0.5cm}
\noindent \textbf{Candidate Choices:}
\begin{itemize}
    \item[(a)] died do/does not belong to any known entities.
    \item[(b)] died is a kind of geopolitical entity.
    \item[(c)] died is a person, but not affiliated with following professions: actor, artist, athlete, author, director, politician, scholar, soldier.
\end{itemize}
\end{tcolorbox}

\begin{tcolorbox}[colback=white, colframe=black, boxrule=1pt, arc=0mm, boxsep=5pt, left=6pt, right=6pt, top=6pt, bottom=6pt]
\noindent \textbf{Analysis:} The sentence provides information about Flavius Rumoridus, who was a Roman soldier and appointed consul in AD 403. The sentence also mentions that he died in the 5th century AD, which is a statement about his death. Therefore, the correct answer is (c) because the sentence does not provide any information about Flavius Rumoridus being affiliated with any of the professions listed in option (c).

\vspace{0.5cm}

\noindent \textbf{Correct Answer:} (c)
\end{tcolorbox}

\noindent \texttt{Case 2}: LLMs predict the word \textit{of} as entity \texttt{organization-other} (Ground truth: \texttt{None})

\begin{tcolorbox}[colback=white, colframe=black, boxrule=1pt, arc=0mm, boxsep=5pt, left=6pt, right=6pt, top=6pt, bottom=6pt]

\noindent \textbf{Sentence:} During these years he studied at the South Kensington Schools of the Royal College of Art.

\vspace{0.5cm}
\noindent \textbf{Candidate Choices:}
\begin{itemize}
    \item[(a)] of do/does not belong to any known entities.
    \item[(b)] of is an educational institute.
    \item[(c)] of pertains to an organization that does not fall under the categories of company, educational institution, government, media, political party, religion, sports league, sports team, band or musical group.
\end{itemize}

\vspace{0.5cm}
\noindent \textbf{Analysis:} In the given sentence, \textit{of} is a preposition that indicates the relationship between \textit{South Kensington Schools} and \textit{Royal College of Art}. None of the candidate choices directly address the correct relationship between these entities or state that they are not applicable. Therefore, the correct answer is (c): \textit{of} does not pertain to any known entities mentioned in the choices provided.

\vspace{0.5cm}
\noindent \textbf{Correct Answer:} (c)
\end{tcolorbox}




\noindent \textbf{Span boundary mismatch}. Different from sentence-level approaches which take one or multiple sentences as an unit and assign each word’s label globally, LLMs under our filter-then-rerank paradigm adopt sample-level approaches. In other words, LLMs only consider each word/span candidate independently. The sample-level approaches enable LLMs to rerank candidates with the help of multi-choice question prompt. However, they also aggravate the span boundary mismatching problem in NER and ED tasks. We empirically find easy samples suffering much more from boundary mismatching problem than hard samples.

\noindent \texttt{Case 3}: LLMs predict \textit{Baron}, a subspan of \textit{Baron Carl Hurleman}, as the entity \texttt{person-other} (The ground-truth entity type of \textit{Baron} is \texttt{None}. And the ground-truth entity type of \textit{Baron Carl Hurleman} is \texttt{person-artist/author}).


\begin{tcolorbox}[colback=white, colframe=black, boxrule=1pt, arc=0mm, boxsep=5pt, left=6pt, right=6pt, top=6pt, bottom=6pt]

\noindent \textbf{Sentence:} Additions in the 1740s were by Baron Carl Hurleman (1700 - 1753).

\vspace{0.5cm}
\noindent \textbf{Candidate Choices:}
\begin{itemize}
    \item[(a)] Baron do/does not belong to any known entities.
    \item[(b)] Baron is a person, but not affiliated with following professions: actor, artist, athlete, author, director, politician, scholar, soldier.
    \item[(c)] Baron is a kind of geopolitical entity
\end{itemize}

\vspace{0.5cm}
\noindent \textbf{Analysis:} In the given sentence, \textit{Baron} is a title given to a person named Carl Hurleman, who is mentioned as the one who made additions in the 1740s. The sentence does not provide any information about connections to geopolitical entities or any other known entities mentioned in the choices provided. Therefore, the correct answer is (b): Baron is a person, but not affiliated with following professions: actor, artist, athlete, author, director, politician, scholar, soldier.

\vspace{0.5cm}
\noindent \textbf{Correct Answer:} (b)
\end{tcolorbox}


\noindent \texttt{Case 4}: LLMs predict \textit{65th}, a subspan of \textit{65th British Academy Film Awards}, as the entity \texttt{other-awards} (The ground-truth entity type of \textit{65th} is \texttt{None}. And the ground-truth entity type of \textit{65th British Academy Film Awards} is \texttt{other-awards}).

\begin{tcolorbox}[colback=white, colframe=black, boxrule=1pt, arc=0mm, boxsep=5pt, left=6pt, right=6pt, top=6pt, bottom=6pt]

\noindent \textbf{Sentence:} Ralph Fiennes was nominated for the BAFTA Award for Outstanding Debut by a British Writer, Director or Producer at the 65th British Academy Film Awards.

\vspace{0.5cm}
\noindent \textbf{Candidate Choices:}
\begin{itemize}
    \item[(a)] 65th do/does not belong to any known entities.
    \item[(b)] 65th is a kind of award.
    \item[(c)] 65th refers to some event except attack, election, natural disaster, protest, revolution and sports.
\end{itemize}

\vspace{0.5cm}
\noindent \textbf{Analysis:} The sentence mentions the \textit{65th British Academy Film Awards}, which is an event that recognizes outstanding achievements in the film industry. Therefore, option (c) is incorrect as it excludes events like award ceremonies. Option (a) is also incorrect as the number \textit{65th} is associated with the event. Hence, the correct answer is (b) as \textit{65th} refers to the specific edition of the BAFTA Awards.

\vspace{0.5cm}
\noindent \textbf{Correct Answer:} (b)
\end{tcolorbox}

\begin{table*}[]
    \centering
    \small
    \caption{Examples of the samples corrected by LLM reranking. We sample four examples from NER, RE and ED tasks, respectively. \textbf{Sentences}: The sentences in which samples locate. We color the samples (entities or trigger words) to be identified. \textbf{Before}: The prediction before LLM reranking. Based on SLM-based methods. \textbf{After}: The reranked prediction using LLMs. \textbf{Rationales}: LLM-generated Explanations.} 
    \label{tab: case}
    \begin{tabular}{p{5cm}p{1.8cm}<{\centering}p{1.8cm}<{\centering}p{5.5cm}}
        \toprule
        Sentences (entity/event) & Before & After & Rationales  \\
        \midrule
        \multicolumn{4}{c}{\cellcolor{gray!15}\textbf{Named Entity Recognition}}  \\
        According to it, King Eleusis and Cothonea (Cyntinia), parents of Triptolemus, are visited by Demeter, who rears their son, feeding him divine milk by day and placing him into the fire at night, which makes \textcolor{blue}{Triptolemus} grow faster than mortal children normally do. & person-other \quad \quad \quad \quad \textcolor{red}{\xmark} & other-god \quad \quad \quad \quad \quad \textcolor{green}{\cmark} & According to the context, Triptolemus is a figure in Greek mythology, the son of King Eleusis and Cothonea (Cyntinia). He was a demigod who was sent by Demeter to teach mankind the art of agriculture. So Triptolemus is a god in some legend/religious story. \\
        \multicolumn{4}{c}{\cellcolor{gray!15}\textbf{Relation Extraction}} \\
        Survivors include \textcolor{blue}{his} wife, Sandra; four sons, \textcolor{blue}{Jeff}, James, Douglas and Harris; a daughter, Leslie; his mother, Sally; and two brothers, Guy and Paul. & None \quad \quad \quad \quad \quad \quad \textcolor{red}{\xmark} & per:children \quad \quad \quad \quad \textcolor{green}{\cmark} & The sentence states 'Survivors include his wife, Sandra; four sons, Jeff, James, Douglas and Harris; a daughter, Leslie; his mother, Sally; and two brothers, Guy and Paul.' This indicates 'his' is the father of Jeff, James, Douglas and Harris, and the husband of Sandra. So his is the parent of Jeff. \\ 
        \\
        \textcolor{blue}{Eliza Samudio}, Bruno 's ex-girlfriend and his 4-month-old baby boy's mother, disappeared from her hometown of Belo Horizonte, \textcolor{blue}{Minas Gerais}. & \raisebox{-\height}{\shortstack[c] {per:cities\_of\\\_residence \textcolor{red}{\xmark}}} & \raisebox{-\height}{\shortstack{per:states\_of\\\_residence \textcolor{green}{\cmark}}} & The sentence states that 'Eliza Samudio, Bruno's ex-girlfriend and his 4-month-old baby boy's mother, disappeared from her hometown of Belo Horizonte, Minas Gerais.' This indicates that Eliza Samudio is from the city of Belo Horizonte, which is located in the state of Minas Gerais. So Eliza Samudio lives in the state or province Minas Gerais. \\
        
        \multicolumn{4}{c}{\cellcolor{gray!15}\textbf{Event Detection}}  
        \\
        We have \textcolor{blue}{fought} every institution, every law firm ... it 's cost us a fortune but now we 're going on a massive expansion. & Conflict.Attack \quad \quad \quad \quad \quad \textcolor{red}{\xmark} & None \quad \quad \quad \quad \quad \quad \textcolor{green}{\cmark} & In the given sentence, the word fought is used to describe the action of the speaker fighting against various institutions and law firms. This does not involve any physical violence or court proceedings, so the word fought does not trigger any known event. \\
        \bottomrule  
    \end{tabular}
\end{table*}
