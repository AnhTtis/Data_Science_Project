\section{Discussion}
%以前的视频预测方法都没有明确的讨论过实时性的问题。我们认为在视频预测领域实时的含义与其他视频任务比如video matting,video tracking是不同的,因为这些只需要与视频的fps相同或者稍快即可,即达到我们熟知的30~60fps,而对于视频预测来说,如果仅仅达到这个帧率,那么预测完的帧所耗费的时间正等于现实生活中所流逝的时间,这样子的预测毫无意义,因为事情已经发生了,所以在视频预测领域的实时应该理解为在极短的时间内可以输出一段较长预测视频,这样才可以利用预测的视频做出有意义的事情。在实际情况下,随着时间的推移我们还需要对实时预测的结果做一些修正,和第三章节一样假设预测器输入需要(1,n)帧,输(n+1,n+m)共m帧,预测器预测速率是输入视频帧的摄像机帧率的N倍,那么在预测输出过程中又有m/N帧输入,所以接下来使用(1+m/N,n+m/N)帧预测后续的(n+1+m/N,n+m+m/N)帧,以此类推,便可以像滑动窗口一样,预测更后续的序列和对之前的序列进行一些修正。
Existing studies have largely overlooked the necessity of real-time prediction speed. %Note that the meaning of real-time in the video prediction domain is different from other video tasks such as Unlike 
Some video tasks such as video matting \cite{lin2021real,ke2022modnet} and video tracking \cite{ahmed2021real,wang2022real} requires frame rates that are only slightly faster or even same with the video, i.e. 30-60 fps. However, video prediction at such frame rates can not provide the prediction result in advance with a sufficient temporal gap, making it much less valuable. Therefore, real-time prediction is a necessity to allow useful deployment of the methods. In practice, we also need to make some corrections to the real-time prediction results over time. Taking the assumption in Section~\ref{sec:model} as an illustration, the prediction model takes $\{1,2,...,n\}$ frames as input and outputs  $\{n+1,n+2,...,n+m\}$ frames. Assuming that the prediction rate is $N$ times the inputting camera frame rate. Then, the following $m/N$ frames are input by the camera during the prediction process. Hence, the next $n$ frames are predicted using $\{1+m/N,2+m/N,...,n+m/N\}$ frames and output subsequent $\{n+1+m/N,n+2+m/N,...,n+m+m/N\}$ frames. Following this ``sliding window'' process, frame sequences can be predicted with continuously periodic corrections.