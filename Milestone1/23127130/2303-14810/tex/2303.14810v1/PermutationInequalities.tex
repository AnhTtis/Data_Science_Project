\documentclass[11pt]{article}

% PACKAGES
\usepackage[utf8]{inputenc}
\usepackage{amssymb,amsmath,amsthm}
\usepackage{extarrows}
\usepackage{theoremref}
\usepackage{bm}

% LAYOUT SPECS
\oddsidemargin1cm
\evensidemargin1cm
\textwidth15cm
\topmargin-1cm
\textheight22cm

\renewcommand{\baselinestretch}{1.0}
\pagestyle{plain}

\newcount\hour \newcount\minutes \hour=\time \divide\hour by 60
\minutes=\hour \multiply\minutes by -60 \advance\minutes by \time
\def\mmmddyyyy{\ifcase\month\or Jan\or Feb\or Mar\or Apr\or May\or
Jun\or Jul\or Aug\or Sep\or Oct\or Nov\or Dec\fi \space\number\day,
\number\year}
\def\hhmm{\ifnum\hour<10 0\fi\number\hour :%
  \ifnum\minutes<10 0\fi\number\minutes} \def\Draft{{\it Draft of
\mmmddyyyy}}
\makeatother%}



% MATH SPECS
\newtheorem{definition}{Definition}%[section]
\newtheorem{theorem}[definition]{Theorem}
\newtheorem{corollary}[definition]{Corollary}
\newtheorem{lemma}[definition]{Lemma}
\newtheorem{proposition}[definition]{Proposition}
\newtheorem{conjecture}[definition]{Conjecture}
%\newtheorem{remark}[definition]{Remark}

\newtheorem{example}[definition]{Example}
\newtheorem{observation}[definition]{Observation}
\newcommand{\Int}{\mathbb{Z}}
\newcommand{\Nat}{\mathbb{N}}
\newcommand{\Real}{\mathbb{R}}
\newcommand{\Complex}{\mathbb{C}}
\newcommand{\XH}{\mathcal{H}}
\newcommand{\XI}{\mathcal{I}}
\newcommand{\lcm}{\mathrm{lcm}}
\newcommand{\rank}{\mathrm{rank}}
\renewcommand{\span}{\mathrm{span}}
\theoremstyle{remark}
\newtheorem{remark}{Remark}

% BODY OF PAPER
\title{Permutation Inequalities for Walks in Graphs\footnote{Work done in part while the first author was with University of Konstanz; material is based upon the first author's bachelor's thesis \cite{willenborg-2016}.}}
\author{{\em Nadja Willenborg}\\
Institute of Computer Science \\
University of St.Gallen, Switzerland\\
{\tt Nadja.Willenborg@unisg.ch}
\and
{\em Sven Kosub} \\
Department of Computer and Information Science \\
University of Konstanz, Germany\\
%Box 67, D-78457 Konstanz, Germany\\
{\tt Sven.Kosub@uni-konstanz.de}
}

\date{\empty}
%\date{Draft of Mar 13, 2023; LaTeXed on \hhmm, \mmmddyyyy.}

\begin{document}

\maketitle

% ===========================================
% ABSTRACT
% - ...
% -------------------------------------------

\begin{abstract}
Using spectral graph theory, we show how to obtain inequalities for the number of walks in graphs 
from nonnegative polynomials and present a new family of such inequalities. 
\end{abstract}

\section{Introduction}
Let $\mathbb{N}$ be the set of nonnegative integers, $[n]$ the set $\{1, \dots , n\}$ and $a,b,c,k,\ell,p \in \mathbb{N}$. 
By $w_{k}(G)$ we denote the number of walks of length $k$ (counted in terms of edges) in a graph $G$. 
As we will always consider the class of all simple and undirected graphs, we simply write $w_{k}$ throughout. 

Counting walks in graphs has been studied for a long time. 
In particular, pure binomial inequalities involving walk numbers, i.e., inequalities of the form 
$w_{\alpha_{1}} \cdots w_{\alpha_{n}} \leq w_{\beta_{1}} \cdots w_{\beta_{n}}$, for $\mathbf{\alpha}, \mathbf{\beta} \in \mathbb{N}^{n}$, 
are useful for many applications in graph theory (see, e.g., \cite{taeubig-2017} for an extensive discussion). 
%One important application is to exploit the relationship with the spectral radius of adjacency matrices. 
%This radius has been used to bound other graph measures such as the chromatic number, the clique number, or network-related properties.
%Another reason to study the growth of the number of walks was a paper by Feige, Peleg and Kortsarz \cite{feige2001dense} 
%on approximating the dense $k$-subgraph problem. 
The prominent inequality $w_{1}^{k} \leq w_{0}^{k-1}w_{k}$ was established by Erd\H{o}s and Simonovits in \cite{erdHos1982compactness} 
and is based on results in \cite{blakley1965holder, london1966inequalities, mulholland1959inequality}. 
Lagarias {\em et al.} showed in \cite{lagarias1984inequality} that $w_{2a+b} w_{b} \leq w_{0}w_{2(a+b)}$. 
Defining a suitable scalar product on $\mathbb{R}^{|V|}$ and using the Cauchy-Schwartz inequality, Dress and Gutman 
showed that $w_{a+b}^{2} \leq w_{2a}w_{2b}$ \cite{dress2003number}. 
All these inequalities were generalized by T\"aubig {\em et al.} in \cite{taubig2013inequalities}. 
Using spectral graph theory, they obtained the following two types of inequalities: 
\begin{equation} 
\label{eq:sandwich}
w_{2a+c}w_{2(a+b)+c} \leq w_{2a}w_{2(a+b+c)}
\end{equation} and 
\begin{equation} 
\label{eq:generalErd}
w_{2\ell +p}^{k} \leq w_{2\ell +pk}w_{2 \ell }^{k-1}.
\end{equation}
Most recently, in \cite{blekherman2022path}, Blekherman and Raymond determined the convex cone of index vectors $\mathbf{\alpha}, \mathbf{\beta} \in \mathbb{N}^{n}$ of valid pure binomial inequalities involving walk numbers, among others. Moreover, their results show that all these inequalities can be derived by a finite combination of inequalities defining the tropicalization. They also verified this for the two types of inequalities given in \cite{taubig2013inequalities}. 

In this note, we present a natural family of inequalities involving walk numbers$$ 
  w_{\alpha_{1}+\alpha_{\sigma(1)}}\cdots w_{\alpha_{n}+\alpha_{\sigma(n)}} \leq w_{2\alpha_{1}}\cdots w_{2\alpha_{n}},$$
where $\alpha \in \mathbb{N}^{n}$ and $\sigma \in S_{n}$, which can be obtained in a very simple way via positive semidefinite forms.



\section{Preliminaries} 
\label{sec:prelim}

For basic concepts about polynomials, the reader may consult, e.g., \cite{prasolov-2004}.

A multivariate polynomial $f \in \mathbb{R}[x_{1}, \dots, x_{n}]$ is called \emph{form (or homogeneous polynomial)} 
iff all monomials in $f$ have same degree.
A form $f$ is \emph{positive semidefinite (psd)} iff it takes only nonnegative values. 
We will often use $0 \leq f$ as a shorthand for nonnegative polynomials. 
Note that a psd form must be of even degree, since if $f$ is a form of odd degree $d$ then by homogeneity 
$f(-\bm{x})=(-1)^{d}f(\bm{x})$, which is nonnegative only if $f$ is the zero form. 
The set of all forms of degree $2d$ in $n$ variables is a cone, denoted by $\mathcal{P}_{n,2d}$.
A form $f$ is called a \emph{sum of squares (sos)} iff there exist other forms $h_{j}$ such that 
$f = h_{1}^{2} + \dots + h_{k}^{2}$. 
The cone of degree-$2d$ sos forms in $n$ variables is denoted by $\Sigma_{n,2d}$. 
Obviously, it holds that $\Sigma_{n,2d} \subseteq \mathcal{P}_{n,2d}$.

Hilbert proved that every degree-$2d$ psd form in $n$ variables is sos if and only if $n=2$ or $d=1$ or $(n,2d)=(3,4)$ (cf.~\cite{hilbert1970darstellung}; 
see also \cite{goel-kuhlmann-reznick-2017} for an extension to even symmetric forms).
In \cite{hurwitz1891ueber}, Hurwitz gave an explicit representation of the form $\sum_{i=1}^{2d}x_{i}^{2d}-2d \prod_{i=1}^{2d}x_{i}$ as a sum of squares. 
Later, in \cite{reznick1987quantitative}, Reznick generalized this to the following result (in a slightly different form).
 
 \begin{lemma} {\em \cite[Theorem 2.6]{reznick1987quantitative}}  
 \label{thm:reznick}
Let $f(\bm{x})=\sum_{i=1}^{n}\alpha_{i}x_{i}^{2d}-2dx_{1}^{\alpha_{1}} \cdots x_{n}^{\alpha_{n}}$ be a form such that 
$\mathbf{\alpha} = (\alpha_{1}, \dots , \alpha_{n}) \in \mathbb{N}^{n}, |\mathbf{\alpha}| =2d $. Then, $f$ is sos.
 \end{lemma}
 
 Given any point set $X \subseteq \mathbb{R}^{d}$, the \emph{convex hull} of $X$ is the set
 $$\text{conv}(X) := \left\{ \sum_{i=1}^{m}\lambda_{i}x_{i} : m \in \mathbb{N}, \lambda_{i} \in \mathbb{R}_{\geq 0}, x_{i} \in X, \sum_{i=1}^{m}\lambda_{i} = 1\right\}$$
which is the smallest convex set $C$ such that  $X \subseteq C$. 
A convex hull of a finite set is called a \emph{polytope}.
%%%%
%and is thus of the form 
%$$\text{conv}\{x_{1}, \dots , x_m \} = \left\{ \sum_{i=1}^{m}\lambda_{i}x_{i} : %\lambda_{i} \in \mathbb{R}_{\geq 0}, \sum_{i=1}^{m}\lambda_{i}=1 \right\}$$
%for some $m \in \mathbb{N}_{0}$ and $x_1, \dots , x_m \in \mathbb{R}^{n}$. 
If $P$ is a polytope then a point $x \in P$ is a \emph{vertex} (respectively \emph{extreme point}) of $P$ iff there are no $y,z \in P$ such that 
$y \neq z$ and $x = (y+z)/2$. 

 Let $f \in \mathbb{R}[x_1, \dots ,x_n]$ be such that $f(\bm{x}) = \sum_{\alpha \in \mathbb{N}^{n}}c_{\alpha}x_{1}^{\alpha_{1}} \cdots x_{n}^{\alpha_{n}}, c_{\alpha} \in \mathbb{R}$. 
 Then, the finite set $\textnormal{supp}(f) =_{\rm def} \{\alpha \in \mathbb{N}^{n} : c_{\alpha} \neq 0 \}$ is called the \emph{support} of $f$ and 
 its convex hull $N(f) =_{\rm def} \textnormal{conv}(\textnormal{supp}(f)) \subseteq \mathbb{R}^{n}$ is called the \emph{Newton polytope} of $f$. 
 Furthermore, we say that $c_{\alpha}$ is a \emph{vertex coefficient} of $f$ if there exists a vertex $\alpha = (\alpha_{1}, \dots, \alpha_{n})$ of $N(f)$ such that $c_{\alpha}x_{1}^{\alpha_{1}} \cdots x_{n}^{\alpha_{n}}$ is a term of $f$. 
 It is known %from Lemma 2.4.3 (?) 
 that every vertex of the Newton polytope of a polynomial lies in the support of the polynomial. 
 Therefore, its vertex coefficients are always $\neq 0$. 
(For more details on the Newton polytope and its geometric interpretation see, e.g., \cite{schweighofer2022real}, Section 2.4.)
  
The following necessary condition for the nonnegativity of a polynomial over the reals will give us a necessary condition for the validity of walk inequalities. 

\begin{lemma} {\em \cite[Proposition 2.1]{dressler2019approach}} 
\label{prop:newton}
 Let $f \in \mathbb{R}[x_{1}, \dots, x_{n}]$ be nonnegative on $\mathbb{R}^{n}$ and let $N(f)$ be its Newton polytope. Then all vertex coefficients of $f$ are positive and the vertices of $N(f)$ are even. 
\end{lemma}

We use the following folklore notion when considering walk inequalities in graphs.
 
 \begin{definition} 
Let $f \in \mathbb{R}[x_{1}, \dots, x_{n}]$ be a polynomial of degree $d$. 
The \emph{symmetrization} of $f$ is defined as $$f_{\textnormal{sym}}(\bm{x}) =_{\rm def} \sum_{\sigma \in S_{n}}f(x_{\sigma (1)}, \dots, x_{\sigma (n)}).$$
\end{definition}

Obviously, the nonnegativity of a polynomial remains invariant under its symmetrization. Note that generally, the reverse is not true. 
For example, consider $f(\bm{x})= x_{1}^{2}-x_{1}x_{2}$, a form that is not positive semidefinite. 
However, its symmetrization $f_{\textnormal{sym}}(\bm{x}) = x_{1}^{2} + x_{2}^{2} -2x_{1}x_{2} = (x_1-x_2)^2$ is positive semidefinite (see also Lemma \ref{thm:reznick}).



\section{Inequalities for walks via nonnegative polynomials}

For basic notions of algebraic graph theory, the reader may consult, e.g., \cite{biggs-1993}.

Let $G=(V,E)$ be an undirected graph with $n$ nodes, $m$ edges, and adjacency matrix $A=A(G)\in \Real^{n\times n}$. 
Since $A$ is symmetric, all eigenvalues of $A$ are real and $A$ is diagonalizable by an orthogonal matrix $U$; 
so,  the adjacency matrix can be written as $A(G)= UDU^{T}$, where $D= \text{diag}(\lambda_{1},\dots , \lambda_{n})$ is 
the diagonal matrix of the eigenvalues $\lambda_1\ge \dots\ge \lambda_n$ and the columns of $U$ are formed by 
an orthonormal basis of eigenvectors.

 We study (the number of) walks, i.e., sequences of nodes, where each pair of consecutive nodes is connected by an edge. Nodes and edges can be used repeatedly in the same walk. The length $k$ of a walk is counted in terms of edges. For $k \in \mathbb{N}$ and $x,y \in V$, we denote by $w_k(x,y)$ the number of walks of length $k$ starting at node $x$ and ending at node $y$. By $w_{k}(x)= \sum_{y \in V}w_{k}(x,y)$ we denote the number of all walks of length $k$ that start at node $x$. Consequently, $w_{k}= \sum_{x \in V}w_{k}(x)$ denotes the total number of walks of length $k$. For each node $v$ we have $w_{0}(v) = 1$ and $w_{1}(v)= d_{v}$, where $d_{v}$ is the degree of $v$. This implies that $w_{0}= n$ and $w_{1} = 2m$.
 
The total number of walks of length $k$ is equal to the sum of all entries of the matrix power $A(G)^{k} = UD^{k}U^{T}$ (cf.~\cite{harary-schwenk-1979,dress2003number}). Hence, $$w_{k} = \mathbf{1}_{n}^{T}A(G)^{k}\mathbf{1}_{n}= \mathbf{1}_{n}^{T}UD^{k}U^{T}\mathbf{1}_{n},$$
where $\mathbf{1}_{n}$ is the all-one vector. If we let $\mu_{1}, \dots , \mu_{n}$ be the coordinates of $\mathbf{1}_{n}$ with respect to the orthonormal basis of $n$ eigenvectors this number is given by 
 \begin{equation}\label{eq:numberOfWalks}
     w_{k}= \sum_{j=1}^{n}\lambda_{j}^{k}\mu_{j}^{2}.
 \end{equation}
 From Eq.~\eqref{eq:numberOfWalks} we get the following relationship between symmetrization and the number of walks in graphs. 
 
 \begin{proposition} \label{prop:polyIneq}
Let $G= (V,E)$ be an undirected graph, $|V|=n$. 
Let $\lambda_{1} \ge \dots \ge \lambda_{n}$ be the eigenvalues of adjacency matrix  $A$ (listed according to their multiplicities). 
Let $\mu_{1}, \dots, \mu_{n}$ be the coordinates of $\mathbf{1}_{n}$ with respect to an orthonormal basis of $n$ eigenvectors. 
Consider a multivariate polynomial $$f(\bm{x}) =_{\rm def} \sum_{\alpha \in \mathbb{N}^{k}}c_{\alpha}\cdot x_{1}^{\alpha_{1}} \cdots x_{k}^{\alpha_{k}},$$
 where $c_{\alpha} \in \mathbb{R}$ and $c_{\alpha} =0 $ for almost all $\alpha \in \mathbb{N}^{k}$.%, which satisfies that if $x_i=x_j$ then $f(x_1,\dots,x_k)=0$.
 Then, it holds that
\[\sum_{\alpha \in \mathbb{N}^{k}}c_{\alpha} \cdot w_{\alpha_{1}}\cdots\ w_{\alpha_{k}}
= \sum_{1\le i_1\le \dots \le i_k\le n}^n \gamma(i_1,\dots,i_k)\cdot f_{\textnormal{sym}}(\lambda_{i_1}, \dots , \lambda_{i_k})\cdot \mu_{i_1}^{2} \cdots \mu_{i_k}^{2}\]
for some $\gamma(i_1,\dots,i_k)>0$.
\end{proposition}
 
\begin{proof}
Applying Eq.~\eqref{eq:numberOfWalks} and rearranging terms, we obtain:
\begin{eqnarray*}
\sum_{\alpha \in \mathbb{N}^{k}}c_{\alpha}\cdot w_{\alpha_{1}} \cdots\ w_{\alpha_{k}}
&=& \sum_{\alpha \in \mathbb{N}^{k}}c_{\alpha} 
	\left( \sum_{i_1=1}^ {n}\lambda_{i_1}^{\alpha_{1}}\mu_{i_1}^{2} \right) \cdots 
	\left(  \sum_{i_k=1}^ {n}\lambda_{i_k}^{\alpha_{k}}\mu_{i_k}^{2} \right)\\
&=& \sum_{1\le i_1, \dots, i_k\le n} f(\lambda_{i_1}, \dots, \lambda_{i_k})\cdot  \mu_{i_1}^2 \cdots \mu_{i_k}^2 \\
&=& \sum_{1\le i_1\le \dots \le i_k\le n}	
	\gamma(i_1,\dots,i_k)
	\left(\sum_{\sigma \in S_k} f\bigl(\lambda_{i_{\sigma(1)}}, \dots, \lambda_{i_{\sigma(k)}}\bigr)\right) \mu_{i_1}^2 \cdots \mu_{i_k}^2 \\
&=& \sum_{1\le i_1\le \dots \le i_k\le n}	\gamma(i_1,\dots,i_k)\cdot  f_{\rm sym}(\lambda_{i_1}, \dots, \lambda_{i_k}) \cdot  \mu_{i_1}^2 \cdots \mu_{i_k}^2
\end{eqnarray*}
Here, $\gamma(i_1,\dots, i_k)^{-1}$ is the number of permutations $\sigma\in S_k$ that leave the tuple $(x_{i_1}, \dots, x_{i_k})$ unchanged.
The exact value of $\gamma(i_1,\dots, i_k)^{-1}$ is not important for our purposes; but notice that $1\le \gamma(i_1,\dots, i_k)^{-1} \le k!$.
This shows the proposition.
\end{proof}

Immediately, this gives a sufficient condition for verifying walk inequalities on graphs.

\begin{corollary} \label{cor:main}
Let $f(\textbf{x}) = \sum_{\alpha \in \mathbb{N}^{k}}c_{\alpha}x_{1}^{\alpha_{1}} \cdots x_{k}^{\alpha_{k}}$ be a polynomial such that 
$c_{\alpha} \in \mathbb{R}$ and $c_{\alpha} =0 $ for almost all $\alpha \in \mathbb{N}^{k}$.
If $f_{\textnormal{sym}}$ is nonnegative on $\mathbb{R}^{k}$ then the following inequality holds in all graphs
  \begin{equation} \label{eq:inequality}
      0 \leq  \sum_{\alpha \in \mathbb{N}^{k}} c_{\alpha}\cdot w_{\alpha_{1}} \cdots\ w_{\alpha_{k}}.
   \end{equation}
 \end{corollary}
 
\begin{example} 
\label{ex:non-binomial}
If $f$ is a univariate polynomial then $f=f_{\rm sym}$. 
Consider the even polynomial $f(x)=x^{2k}-a_1x^{2k-1}-\cdots - a_{2k-1}x$, where all numbers $a_1,\dots, a_{2k-1}$ are nonnegative and at least one of them is nonzero. Then, $f$ has root $0$ and a single positive root $p$ (note that $f$ has exactly one change of signs of the non-vanishing coefficients). 
Since $f(x)\le f(-x)$ for all $x\ge 0$, there is a unique $0<\gamma_0<p$ such that $f(\gamma_0)<0$ and $f(x)\ge f(\gamma_0)$ for all $x\in\Real$. 
Hence, $f(x)-f(\gamma_0)$ is nonnegative.
Corollary \ref{cor:main} implies that
\begin{equation}
\label{eq:arithmetic}
w_{2k}-f(\gamma_0) w_0 \ge a_1 w_{2k-1} + \cdots + a_{2k-1} w_1
\end{equation}
for all graphs. 
%Very roughly, $|f(\gamma_0)|\le f'(p)p \le 2n(1+\max \{a_1,\dots,a_{2n-1}\})^{2n}$.
Here, using the Grace-Heawood theorem, $|f(\gamma_0)|$ can be roughly bounded by
\[|f(\gamma_0)|
\le (p-\gamma_0)|h(z)| 
\le \frac{p}{2}(1+\cot (\pi/2k)) |h(z)|
\]
where $h(x)=f(x)/(x-p)$ is a polynomial of degree $2k-1$,
$z\in (\gamma_0,p)$ satisfies $f'(z)=-f(\gamma_0)/(p-\gamma_0)$, and $p\le 1+\max\{a_1,\dots,a_{2k-1}\}$.
Inequalities similar to Ineq.~\eqref{eq:arithmetic} can be easily obtained for polynomials where $a_j<0$ for some $j$.
\end{example}

\begin{proposition} \label{prop:NP}
Assume that $w_{\alpha_{1}} \cdots w_{\alpha_{k}} \leq w_{\beta_{1}} \cdots w_{\beta_{k}}$ is valid in all graphs. 
If this inequality can be verified by Corollary \ref{cor:main}, i.e., the symmetrization of $f(\textbf{x})= x_{1}^{\beta_{1}} \cdots x_{k}^{\beta_{k}}- x_{1}^{\alpha_{1}} \cdots x_{k}^{\alpha_{k}}$ is nonnegative,  
then $\beta_{i} \equiv 0 \mod 2$ for all $i \in [k].$
\end{proposition}


\begin{proof}
Since $0 \leq f_{\textnormal{sym}}$ we get by Lemma \ref{prop:newton} that the coefficients of the vertices are positive. 
Thus, $N(f_{\textnormal{sym}}) = \textnormal{conv}( \{(\beta_{\sigma(1)}, \dots, \beta_{\sigma(k)}): \sigma \in S_{k} \}).$ 
Furthermore, by the same result, all vertices of $N(f_{\textnormal{sym}})$ are even and therefore, $\beta_{i} \equiv 0 \mod 2$ for all $i \in [k]$.
\end{proof}

\begin{corollary} \label{cor:charac}
Assume that $w_{\alpha_{1}}w_{\alpha_{2}} \leq w_{\beta_{1}}w_{\beta_{2}}$ with $|\alpha| = |\beta |$ holds in all graphs. 
If this inequality can be verified by Corollary \ref{cor:main}, i.e., the symmetrization of $f(\bm{x})= x_{1}^{\beta_{1}}x_{2}^{\beta_{2}}- x_{1}^{\alpha_{1}}x_{2}^{\alpha_{2}}$ is psd. 
Then, there are $a,b,c \in \mathbb{N}$ such that
$$\alpha_{1} = 2a+c, \hspace{0.5em} \alpha_{2}= 2(a+b)+c, \hspace{0.5em} \beta_{1} = 2a, \hspace{0.5em} \beta_{2} = 2(a+b+c).$$
\end{corollary}

\begin{proof}
From Proposition \ref{prop:NP} we see that $\beta_{i} \equiv 0 \mod 2$ for $i\in\{1,2\}$. 
We also have $0 \leq f_{\textnormal{sym}}$ and 
$f_{\textnormal{sym}}(\bm{x})= x_{1}^{\beta_{1}}x_{2}^{\beta_{2}} + x_{1}^{\beta_{2}}x_{2}^{\beta_{1}}- x_{1}^{\alpha_{1}} x_{2}^{\alpha_{2}}- x_{1}^{\alpha_{2}}x_{2}^{\alpha_{1}}.$ 
This gives $\beta_1 \leq  \alpha_1 \leq \alpha_2 \leq \beta_2$ where $\alpha_1 \equiv \alpha_2 \mod 2$. 
Hence, we have $\beta_1 = 2a, \alpha_1 =2a+c, \alpha_2 = 2(a+b)+c$ and $\beta_2 = 2(a+b+c)$ with $a,b,c \in \mathbb{N}$.
\end{proof}

\begin{remark}
Corollary \ref{cor:charac} fully characterizes pure binomial inequalities with only two factors which can be derived via a psd form. In fact, Ineq.~\eqref{eq:sandwich} is the only inequality of this type that can be derived via a psd form.
\end{remark}
\begin{theorem}
Let $\alpha \in \mathbb{N}^{k}, |\alpha| =2d$. Then, $ w_{\alpha_1}\cdots w_{\alpha_k} \leq w_{0}^{k-1}w_{2d}$ holds in all graphs. 
\end{theorem}

\begin{proof}
This follows from Lemma \ref{thm:reznick} together with Corollary \ref{cor:main}.
\end{proof}

Similarly, we present an alternative proof for Ineq.~\eqref{eq:sandwich}.

\begin{theorem}
Let $a,b,c \in \mathbb{N}$. Then, $w_{2a+c}w_{2(a+b)+c} \leq w_{2a}w_{2(a+b+c)}$ holds in all graphs. 
\end{theorem}

 \begin{proof}
Consider the form $ f(\bm{x})=_{\rm def} x_1^{2a} x_2^{2(a+b+c)}-x_1^{2a+c} x_{2}^{2(a+b)+c}$ with symmetrization 
$f_{\textnormal{sym}}(\bm{x}) = x_{1}^{2a}x_{2}^{2a}(x_{1}^{2b+c}-x_{2}^{2b+c})(x_{1}^{c}-x_{2}^{c})$. 
It is easy to see that $f_{\textnormal{sym}} \in \mathcal{P}_{2,4a+2(b+c)}$ and hence, the statement follows from 
Corollary \ref{cor:main}.
\end{proof}

\begin{remark}
 More generally, given the inequality $w_{\alpha_{1}} \cdots w_{\alpha_{n}} \leq w_{\beta_{1}} \cdots w_{\beta_{n}}$ and
 $f(\textbf{x})= x_{1}^{\beta_{1}} \cdots x_{n}^{\beta_{n}}- x_{1}^{\alpha_{1}} \cdots x_{n}^{\alpha_{n}}$ such that 
 $f_{\textnormal{sym}}$ is nonnegative, we can consider the nonnegative polynomial
  $\Tilde{f}(\textbf{x}) = (x_{1}\cdots x_{n})^{2a} \cdot f_{\textnormal{sym}}(\textbf{x})$ and derive the inequality $w_{\alpha_{1}+2a} \cdots w_{\alpha_{n}+2a} \leq w_{\beta_{1}+2a} \cdots w_{\beta_{n}+2a}$.
  \end{remark}

However, considering Ineq.~\eqref{eq:generalErd} and a form $f(\bm{x}) = x_{1}^{2\ell +pk}(x_{2} \cdots x_{k})^{2\ell}-(x_{1} \cdots x_{k})^{2\ell+pk}$,
we get $\textnormal{deg}(f_{\textnormal{sym}})=k(2\ell +p)$. Since a psd form must be of even degree, we cannot validate Ineq.~\eqref{eq:generalErd} in all cases using a psd form.

\begin{theorem} \label{thm:symmetricgeneral}
Let $\alpha \in \mathbb{N}^{k}$ and $\sigma \in S_{k}$. Then, the inequality  
$$w_{\alpha_{1}+\alpha_{\sigma(1)}}\cdots w_{\alpha_{k}+\alpha_{\sigma(k)}} \leq  w_{2\alpha_{1}}\cdots w_{2\alpha_{k}}$$ holds in all graphs.  
\end{theorem}

\begin{proof}
Consider the form $ f(\bm{x})=\bigl(x_{1}^{\alpha_{1}}\cdots x_{k}^{\alpha_{k}}- x_{\sigma(1)}^{\alpha_{1}}\cdots x_{\sigma(k)}^{\alpha_{k}}\bigr)^{2}$. 
From $0 \leq f_{\textnormal{sym}}$ we derive 
$$2w_{\alpha_{1}+\alpha_{\sigma(1)}} \cdots w_{\alpha_{k}+\alpha_{\sigma(k)}} \leq w_{2\alpha_{1}} \cdots w_{2\alpha_{k}}+w_{2\alpha_{\sigma(1)}} \cdots w_{2\alpha_{\sigma(k)}}$$ and hence, $$w_{\alpha_{1}+\alpha_{\sigma(1)}}\cdots w_{\alpha_{k}+\alpha_{\sigma(k)}} \leq w_{2\alpha_{1}}\cdots w_{2\alpha_{k}}.$$
This proves the theorem.
 \end{proof}
Note that the case $k=2$ and $\sigma = (1 \, 2)$ is the inequality given by Dress and Gutman. 
Using induction on $k$ and the representation of $\sigma \in S_{k}$ as a product of disjoint cycles, 
we conclude that the only elements of $S_{k}$ that can give rise to new inequalities for walks are 
cyclic permutations. 



\section{Conclusion}

%\begin{conclusion}
Our results show that not every known pure binomial inequality for the number of walks can be derived via a nonnegative polynomial and is therefore, not as powerful as the tropicalization given by Blekherman and Raymond in \cite{blekherman2022path}. 
However, regarding the inequalities given in Theorem \ref{thm:symmetricgeneral} we assume the polynomial approach to be still a simple and strong approach for proving walk inequalities. Moreover, using the polynomial approach, we also obtained some non-binomial inequalities (see Example \ref{ex:non-binomial}),  which could not be obtained by tropicalization.
Generally,  tropicalization is suitable for characterizing universal binomial inequalities, i.e., binomial inequalities that are valid for all graphs. In contrast, the pure binomial inequality $w_0w_{r+s}\ge w_r w_s$ is not universal (see \cite{lagarias1984inequality,taeubig-2017} for counterexamples), but it is valid for regular graphs and, potentially, for subdivision graphs \cite{taeubig-2017}. It is advisable to see whether tropicalization and lightweight methods can interact in order to find a precise characterization of all these graph classes.
%\end{conclusion}


\paragraph{Acknowledgment.} We thank Markus Schweighofer (Konstanz) for helpful comments.

% =================================================
% BIBLIOGRAPHY
% -------------------------------------------------

%\newpage

{\small
\begin{thebibliography}{10}

\bibitem{blakley1965holder}
G.~R.~Blakley, P.~Roy.
 \newblock A H{\"o}lder type inequality for symmetric matrices with nonnegative entries.
\newblock {\em Proceedings of the American Mathematical Society}, 16(6):1244--1245, 1965.

\bibitem{blekherman2022path}
G.~Blekherman, A.~Raymond.
\newblock A path forward: Tropicalization in extremal combinatorics,
\newblock {\em Advances in Mathematics}, 407:108561, 2022.

\bibitem{biggs-1993}
N.~Biggs.
\newblock {\em Algebraic Graph Theory}.
\newblock 2nd edition. Cambridge University Press, Cambridge, 1993.

%\bibitem{brouwer-haemers-2012}
%A.~E.~Brouwer, W.~H.~Haemers.
%\newblock {\em Spectra of Graphs}.
%\newblock Springer, New York, NY, 2012.

%\bibitem{cvetkovic-1970}
%D.~M.~Cvetkovi\'{c}.
%\newblock  The generating function for variations with restrictions and paths of the graph and self-complementary graphs.
%\newblock {\em Univerzitet u Beogradu, Publikacije Elektrotehničkog fakulteta. Serija Matematika i Fizika}, 320/328:27--34, 1970. 

%\bibitem{cvetkovic-doob-sachs-1980}
%D.~M.~Cvetkovi\'{c}, M.~Doob, H.~Sachs.
%\newblock {\em Spectra of Graphs: Theory and Applications}.
%\newblock Academic Press, New York, NY, 1980.

%\bibitem{dress-gutman-2003}
%A.~Dress, I.~Gutman.
%\newblock The number of walks in a graph.
%\newblock {\em Applied Mathematics Letters}, 16(5):797--801, 2003.

\bibitem{dress2003number}
A.~Dress, I.~Gutman.
\newblock Asymptotic results regarding the number of walks in a graph.
\newblock {\em Applied Mathematics Letters}, 16(3):389--393, 2003.

\bibitem{dressler2019approach}
M.~Dressler, S.~Iliman, T.~de Wolff.
\newblock An approach to constrained polynomial optimization via nonnegative circuit polynomials and geometric programming.
\newblock {\em Journal of Symbolic Computation}, 91:149--172, 2019.

\bibitem{erdHos1982compactness}
P.~Erd\H{o}s, M.~Simonovits.
\newblock Compactness results in extremal graph theory.
\newblock {\em Combinatorica}, 2(3):275--288, 1982.

%\bibitem{feige2001dense}
%U.~Feige, D.~Peleg, G.~Kortsarz.
%\newblock The dense $k$-subgraph problem.
%\newblock {\em Algorithmica}, 29(3):410--421, 2001.

%\bibitem{gardner-govil-2014}
%R.~B.~Gardner, N.~K.~Govil.
%\newblock Enestr\"om-Kakeya theorem and some of its generalizations.
%\newblock In: S.~Joshi, M.~Dorff, I.~Lahiri (eds.), {\em Current Topics in Pure and Computational Complex Analysis}, chapter~8, pp.~171--199. Trends in Mathematics, Birkh\"auser, New Delhi, 2014.


\bibitem{goel-kuhlmann-reznick-2017}
C.~Goel, S.~Kuhlmann, B.~Reznick.
\newblock The analogue of Hilbert's 1888 theorem for even symmetric forms.
\newblock {\em Journal of Pure and Applied Algebra}, 221(6):1438--1448, 2017.

%\bibitem{godsil-mohar-1988}
%C.~Godsil, G.~Royle.
%\newblock {\em Algebraic Graph Theory}.
%\newblock Graduate Texts in Mathematics \#207, Springer, New York, NY, 2001.

%\bibitem{golub-loan-1996}
%G.~H.~Golub, C.~F.~van Loan.
%\newblock {\em Matrix Computations.}
%\newblock John Hopkins Studies in the Mathematical Sciences. 3rd edition, The John Hopkins University Press, Baltimore, ML, 1996.

\bibitem{harary-schwenk-1979}
F.~Harary, A.~J.~Schwenk.
\newblock The spectral approach to determining the number of walks in a graph.
\newblock {\em Pacific Journal of Mathematics}, 80(2):443--449, 1979.

%\bibitem{hazewinkel2013encyclopaedia}
%M.~Hazewinkel (ed.).
%\newblock  {\em Encyclopaedia of Mathematics}. Volume 6.
%\newblock Springer, Dordrecht, 2013.

\bibitem{hilbert1970darstellung}
D.~Hilbert.
\newblock {\"U}ber die Darstellung definiter Formen als Summe von Formenquadraten.
\newblock {\em Mathematische Annalen}, 32(3):342--350, 1888.

\bibitem{hurwitz1891ueber}
A.~Hurwitz.
\newblock \"Uber den Vergleich des arithmetischen und des geometrischen Mittels.
\newblock  {\em Journal f\"ur die reine und angewandte Mathematik}, 108:266--268, 1891.

\bibitem{lagarias1984inequality}
J.~C.~Lagarias, J.~E.~Mazo, L.~A.~Shepp, B.~D.~McKay.
\newblock An inequality for walks in a graph.
\newblock {\em SIAM Review}, 26(4):580--582, 1984.

\bibitem{london1966inequalities}
D.~London.
\newblock Inequalities in quadratic forms.
\newblock {\em Duke Mathematical Journal}, 33(3):511--522, 1966.

\bibitem{mulholland1959inequality}
H.~P.~Mulholland, C.~A.~P.~Smith.
\newblock An inequality arising in genetical theory.
\newblock {\em American Mathematical Monthly}, 66(8):673--683, 1959.

\bibitem{prasolov-2004}
V.~V.~Prasolov.
\newblock {\em Polynomials}.
\newblock Algorithms and Computation in Mathematics \#11.
Springer, Berlin, 2004.

\bibitem{reznick1987quantitative}
B.~Reznick.
\newblock A quantitative version of Hurwitz' theorem on the arithmetic-geometric inequality.
\newblock  {\em Journal f\"ur die reine und angewandte Mathematik}, 377:108--112, 1987.

\bibitem{schweighofer2022real}
M.~Schweighofer.
\newblock Real algebraic geometry, positivity and convexity.
\newblock Technical report arXiv:2205.04211 [math.HO], April 2022. 

%\bibitem{taeubig-2016}
%H.~T\"aubig.
%\newblock Inequalities for the number of walks in subdivision graphs.
%\newblock {\em MATCH Communications in Mathematical and in Computer Chemistry}, 76:61--68, 2016.

\bibitem{taeubig-2017}
H.~T\"aubig.
\newblock {\em Matrix Inequalities for Iterative Systems}.
\newblock CRC Press, Boca Raton, FL, 2017.

\bibitem{taubig2013inequalities}
H.~T\"aubig, J.~Weihmann, S.~Kosub, R.~Hemmecke, E.~W.~Mayr.
\newblock Inequalities for the number of walks in a graph.
\newblock {\em Algorithmica}, 66(4):804--828, 2013. 

\bibitem{willenborg-2016}
N.~Willenborg.
\newblock Polynomielle Wegeungleichungen in ungerichteten Graphen.
\newblock Bachelor's Thesis, Department of Mathematics and Statistics, University of Konstanz, Konstanz, 2016. In German.

\end{thebibliography}
}

\end{document}
