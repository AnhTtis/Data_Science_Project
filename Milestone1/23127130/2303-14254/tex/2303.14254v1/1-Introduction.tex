\section{Introduction}

Deep learning has been successful in various time-series applications given enormous amount of data to train. However, time-series data collected through real-world sensors is often marked by irregular samples with missing values due to collection difficulties. Such data {\em scarcity} commonly observed in time-series data can significantly degrade the performance of deep learning methods that would otherwise perform well.

A rich line of research tries to address this problem through data augmentation, that is to generate synthetic data points to augment the original dataset~\cite{um2017data,le2016data,bergmeir2016bagging,forestier2017generating,esteban2017real,che2017boosting,steven2018feature,lim2018doping,yoon2019time,nam2020data,hu2020datsing,kang2020gratis}. 
However, existing augmentation methods are mainly designed for classification, where augmented samples remain effective as long as they preserve the class labels. 
We note that augmentation designed for forecasting requires both \emph{diversity} and \emph{coherence} with the original temporal dynamics. 
Yet, existing augmentation methods generate samples that often miss one of the criteria (Figure~\ref{fig:ex}). For example, filtering-based methods are deterministic processes that produce a fixed set of synthetic samples by removing noises. Permutation-based methods change the temporal order of the original series, worsening forecasting performance.

Moreover, time-series data generated by real-life physical processes exhibit characteristics both in the time and frequency domains that are not available in other data modalities like image and text.
Therefore, temporal dynamics can be best captured through a joint consideration of time domain that carries changes over time and frequency domain that conveys periodic patterns. 
By contrast, existing augmentation methods mostly generate data in one domain, ignoring the complementary strengths of both domains.

\begin{figure}[t]
      \centering
      \begin{subfigure}[b]{0.155\textwidth}
          \centering
          \includegraphics[width=\textwidth]{figure/motivating_example_ori_2.pdf}
          \caption{Original}
          \label{fig:ex-ori}
      \end{subfigure}
      \begin{subfigure}[b]{0.155\textwidth}
          \centering
          \includegraphics[width=\textwidth]{figure/motivating_example_emd_r_2.pdf}
          \caption{Filtering}
          \label{fig:ex-ww}
      \end{subfigure}
      \begin{subfigure}[b]{0.155\textwidth}
          \centering
          \includegraphics[width=\textwidth]{figure/motivating_example_perm_2.pdf}
          \caption{Permutation}
          \label{fig:ex-perm}
      \end{subfigure}
         \caption{Visualization of original and augmented time series from ETTm2. Augmentation methods often (b) miss diversity or (c) miss coherence with the original temporal dynamics.} 
         \label{fig:ex}
\end{figure}

\begin{figure}[t]
\centering
\includegraphics[width=\linewidth]{figure/main.pdf}
\vspace{-4mm}
\caption{Overview of \our. For frequency-domain augmentation, we decompose two random series $\mathbf{S_i}, \mathbf{S_j}$ (marked in blue) with Empirical Mode Decomposition (EMD), and then reassemble the subcomponents with random weights sampled from uniform distribution $U(a,b)$ to obtain $\mathbf{S'_i}, \mathbf{S'_j}$ (marked in orange). In the time domain, we further linearly mix $\mathbf{S'_i}, \mathbf{S'_j}$ to obtain the augmented series $\mathbf{S'}$ (marked in red) for training. 
}
\label{fig:main}
\end{figure}

We propose \emph{\our} (Figure~\ref{fig:main}), by combining Spectral and Time Augmentation for time-series forecasting task.
In the frequency domain, we first apply the Empirical Mode Decomposition (EMD)~\cite{huang1998empirical}
to decompose time series into multiple subcomponents, each representing a certain pattern embedded in the data. 
We then reassemble these subcomponents with random weights to generate new synthetic series.
This offers a \emph{principled} way of augmentation as it generates diverse samples while maintaining the same basic set of subcomponents.
We adopt EMD for the frequency information as it better captures patterns for non-stationary time series compared with Fourier transform.
In the time domain, we adapt a mix-up strategy~\cite{zhang2017mixup} to learn linearly in-between randomly sampled pairs of training series, which produces varied and coherent samples. We evaluate \our on five real-world time-series datasets, and the method demonstrates state-of-the-art performance compared with existing augmentation methods.



