\section{Related Work}
Existing time-series data augmentation techniques mainly fall into four categories~\cite{wen2020time,iwana2021empirical} as follows. 

\emph{Basic random operations} include time-domain transformation, frequency-domain transformation and time-frequency domain transformation. 
Time-domain transformation contains scaling, rotation, jittering~\cite{um2017data}, slicing, warping~\cite{le2016data,ma2021joint}, etc. 
For frequency-domain transformation, RobustTAD~\cite{gao2020robusttad} in the frequency domain makes perturbations in both magnitude and phase spectra.
For time-frequency domain transformation, time-frequency features are generated from Short Fourier Transform (STFT), and then local averaging together with feature vector shuffling are applied for augmentation~\cite{steven2018feature}.
SpecAugment~\cite{park2019specaugment} proposes augmentation in Mel-Frequency by combining warping, masking frequency channels and masking time step blocks together.

\emph{Decomposition-based augmentation} leverages the Seasonal-Trend Decomposition (STL) ~\cite{cleveland1990stl} or Empirical Mode Decomposition (EMD)~\cite{huang1998empirical} to extract patterns for generating synthetic samples. 
Bagging Exponential Smoothing method~\cite{bergmeir2016bagging,kegel2018feature,bandara2021improving} uses Box–Cox transformation followed by STL decomposition, and bootstraps the reminder to assemble new series.
STL decomposition components can also be adjusted and combined with a stochastic component generated by statistical models~\cite{kegel2018feature,bandara2021improving}.
Nam et al.~\cite{nam2020data} decomposes series using EMD into components from high frequency to low frequency, and adds the residue each time one IMF occurs. This can be viewed as a special case of \our with weights equal to one for low-frequency components and zero for high-frequency components, essentially only filtering out high-frequency noise. Moreover, the method does not benefit from time-domain augmentation.

For \emph{pattern mixing method}, DBA calculates weighted average of multiple time series under DTW as new samples~\cite{forestier2017generating,bandara2021improving}. Mix-up~\cite{zhang2017mixup} constructs new examples through linear interpolation of both features and labels, but such interpolation methods mainly focus on classification tasks. 

\emph{Generative method} models underlying distribution of the dataset for generation, including both statistical generative models~\cite{cao2014parsimonious,kang2020gratis} and deep generative adversarial network (GAN)~\cite{goodfellow2014generative} based models~\cite{esteban2017real,che2017boosting,ramponi2018t,lim2018doping,yoon2019time,hu2020datsing}. 

In this work, we combine decomposition-based augmentation method in the frequency domain and pattern mixing augmentation method in the time domain to find diverse samples that preserve the original data characteristics.