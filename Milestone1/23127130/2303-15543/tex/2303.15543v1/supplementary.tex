
\documentclass[sigconf,nonacm]{acmart}

% \usepackage{amsmath}
% \usepackage{graphicx}
% \usepackage{hyperref}
% \usepackage{longtable}
\usepackage{multirow}
\usepackage{colortbl}
\usepackage{subcaption}
\usepackage[ruled,linesnumbered]{algorithm2e}
\usepackage{xcolor}
\usepackage{soul}
\usepackage{tikz}
\usepackage{pdflscape}
\usepackage{geometry}
\usepackage{tabularray}
\usepackage{adjustbox}
\usepackage{afterpage}
\usepackage{capt-of}
% \usepackage{algorithm}
% \usepackage{algpseudocode}

\title{Supplementary: The Impact of Asynchrony on Parallel Model-Based EAs}

\SetKwProg{Fn}{Function}{:}{}
\SetKwComment{Comment}{/* }{ */}


\begin{document}
\maketitle
\tableofcontents
% \listoffigures
% \listoftables
% \listofalgorithms

\section{Parallel EAs}
In the main paper we roughly describe the setup of our parallel EAs. This section serves to clarify this setup. Within our codebase we simulate the running of the algorithm such that only function evaluations cost time. Each of the evaluations is performed on one of the processors who perform work according to a queue $\mathcal{Q}$ to which the algorithms add work items. Note that work items are not necessarily a single function evaluation - in the case of GOMEA this constitutes an entire application of GOM. Once a work item is finished a new item is removed from the top of queue - or if empty, the worker waits.

A key difference between the synchronous and asynchronous approaches is how this queue is used. In the synchronous approaches the queue is used as an inner loop for which tasks are appended at the beginning of the generation, and waits until all tasks run until completion. Whereas in the asynchronous approach each task schedules a similar task to run at the end.

% We leave out the pseudocode for the standard crossover operators and how selection is performed: there is no implementation difference here.

\begin{algorithm}
    \SetKwFunction{FInit}{Initialize}
    \SetKwFunction{FStepGeneration}{StepGeneration}
    \SetKwFunction{FAppend}{Append}
    \SetKwFunction{FSynchronize}{WaitUntilAllTasksDone}
    \SetKwFunction{FTask}{PerformTask}
    \SetKwFunction{FTaskGA}{PerformTask-GA}
    \SetKwFunction{FSampleRandomly}{SampleRandom}
    \SetKwFunction{FSampleFromP}{SampleFromPopulation}
    \SetKwFunction{FCrossover}{Crossover}
    \SetKwFunction{FEvaluate}{Evaluate}
    \SetKwFunction{FSelect}{Select}
    \SetKwFunction{FRunSync}{RunSynchronous}

    \Fn{\FInit{}}{

    \ForEach{$p \in \mathcal{P}$}{
        \Comment{\FTask is not called immediately}
        $\mathcal{Q} \gets$ \FAppend{$\mathcal{Q}$, $\lambda: $\FTaskGA{true}}\;
    }
    \FSynchronize{}
    }

    \Fn{\FStepGeneration{}}{

    \ForEach{$p \in \mathcal{P}$}{
        \Comment{\FTask is not called immediately. $\lambda$ creates a function that takes no arguments and calls the function given the arguments listed.}
        $\mathcal{Q} \gets$ \FAppend{$\mathcal{Q}$, $\lambda: $\FTaskGA{false}}\;
    }
    \FSynchronize{}
    }

    \Fn{\FRunSync{}}{
        \Comment{Termination happens as condition is met.}
        \FInit{}\;
        \While{true}{
            \FStepGeneration{}\;
        }
    }

    \Fn{\FTaskGA{is-init}}{
        \eIf{is-init}{
            $s \gets $\FSampleRandomly{}\;
            $i \gets |\mathcal{P}|$\;
            $\mathcal{P} \gets$ \FAppend{$\mathcal{P}$, $s$}\;
            $s \gets $\FEvaluate{$s$}\;
            \If{$s > \mathcal{P}[i]$}{
                \Comment{In an asynchronous approach the solution may have been replaced - this allows us to backtrack the change if the initial solution turned out to be better.}
                $\mathcal{P}[i] \gets s$\;
            }
        }{
            $p_0, p_1 \gets $\FSampleFromP{}\;
            $s \gets $\FCrossover{$p_0$, $p_1$}\;
            $s \gets $\FEvaluate{$s$}\;
            $\mathcal{P} \gets $\FSelect{$\mathcal{P}$, $s$}\;
        }
    }
    \caption{Outline of a Parallel Synchronous GA}
\end{algorithm}

\begin{algorithm}
    \SetKwFunction{FTaskGA}{PerformTask-GA}
    \SetKwFunction{FTaskSGA}{PerformTask-Async-GA}
    \SetKwFunction{FRunAsynchonous}{RunAsynchronous}
    \SetKwFunction{FWaitUntilTermination}{WaitUntilTermination}
    \SetKwFunction{FAppend}{Append}
    \Fn{\FTaskSGA{is-init}}{
        \Comment{Same actions are performed as in the synchronous version.}
        \FTaskGA{is-init}\;
        \Comment{Except, next task is queued asynchronously.}
        $\mathcal{Q} \gets$ \FAppend{$\mathcal{Q}$, $\lambda: $\FTaskSGA{false}}\;
    }
    \Fn{\FRunAsynchonous{}}{
        \ForEach{$p \in \mathcal{P}$}{
            \Comment{\FTask is not called immediately}
            $\mathcal{Q} \gets$ \FAppend{$\mathcal{Q}$, $\lambda: $\FTaskSGA{true}}\;
        }
        \Comment{Now, wait until the termination criterion is hit - tasks schedule themselves.}
        \FWaitUntilTermination{}\;
    }

    
    \caption{Outline of a Parallel Asynchronous GA}
\end{algorithm}

\begin{algorithm}
    \SetKwFunction{FSelectGenerationalPooling}{SelectGenerationalPooling}
    \SetKwFunction{FSelectGenerational}{SelectGenerational}
    \SetKwFunction{FSelect}{Append}
    \SetKwFunction{FAppend}{Append}
    \Fn{\FSelectGenerationalPooling{$\mathcal{P}$, $s$}}{
        \Comment{This method has additional attached data $\mathcal{O}$ - an offspring population - that is shared between invocations}
        $\mathcal{O} \gets $\FAppend{$\mathcal{O}$, $s$}\;
        \If{$|\mathcal{O}| = |\mathcal{P}|$}{
            \Comment{Enough offspring collected to apply a generational selection operator, like tournament selection.}
            $\mathcal{P} \gets$ \FSelectGenerational{$\mathcal{P}$, $\mathcal{O}$}\;
        }
    }
    \caption{Special cases for operators of a Parallel GA}
\end{algorithm}

\begin{algorithm}
    \SetKwFunction{FTaskECGA}{PerformTask-ECGA}
    \SetKwFunction{FLearnModel}{LearnModel}
    \SetKwFunction{FSample}{Sample}
    \Fn{\FTaskECGA{is-init}}{
        \If{is-init}
        {
            \Comment{Same as GA}
            ...
        }
        
        \Comment{Check if the sampling model needs to be updated. Both uses-left and $\mathcal{M}$ are shared across tasks.}
        \If{uses-left $= 0$}{
            $\mathcal{M} \gets$ \FLearnModel{$\mathcal{P}$}\;
            \Comment{This parameter setting will effectively learn the model at the start of each generation in a synchronous approach, yet also work when transferred to an asynchronous setting.}
            uses-left $\gets |\mathcal{P}|$\;
        }
        $s \gets$ \FSample{$\mathcal{M}$}\;
        uses-left $\gets$ uses-left $-1$\;
        $s \gets $\FEvaluate{$s$}\;
        $\mathcal{P} \gets $\FSelect{$\mathcal{P}$, $s$}\;
    }
    \Comment{General framework is equivalent to the GA, except the task above is used instead.}

    \caption{Task for ECGA}
\end{algorithm}

\begin{algorithm}
    \SetKwFunction{FTaskGOMEA}{PerformTask-GOMEA}
    \SetKwFunction{FLearnFOS}{LearnFOS}
    \SetKwFunction{FSampleFromP}{SampleFromPopulation}
    \SetKwFunction{FGOMFI}{GOM+FI}
    \SetKwFunction{FGOM}{GOM}
    \Fn{\FTaskGOMEA{is-init, idx}}{
        \If{is-init}
        {
            \Comment{Same as GA}
            ...\\
            \Return{}
        }
        
        \Comment{Check if the FOS needs to be updated. Both uses-left and $\mathcal{M}$ are shared across tasks.}
        \If{uses-left $= 0$}{
            $\mathcal{F} \gets$ \FLearnFOS{$\mathcal{P}$}\;
            \Comment{Much like the model used for ECGA, this will result in a new model being learned at the start of every generation in a synchronous approach.}
            uses-left $\gets |\mathcal{P}|$\;
        }
        uses-left $\gets$ uses-left $-1$\;
        \Comment{Create a copy of the population}
        $\mathcal{P}^\prime \gets \mathcal{P}$\;
        $s \gets \mathcal{P}[idx]$\;
        $s \gets$ \FGOMFI{$s$, $idx$, $\mathcal{F}$, $\mathcal{P}^\prime$}\;
        \Comment{Solution is always updated after GOM for asynchronous. For synchronous the copy back happens generationally.}
        \If{async}{
            $\mathcal{P}[idx] \gets s$\;
        }
    }

    \Fn{\FGOM{s, idx, $\mathcal{F}$, $\mathcal{P}^\prime$}}{
        \Comment{Statistics tracking for FI has been omitted.}
        \ForEach{$v \in \mathcal{F}$}{
            $s_b \gets s$\;
            $d \gets$ \FSampleFromP{$\mathcal{P}^\prime$}\;
            $s[v] \gets d[v]$\;
            \eIf{$f(s) \geq f(s_b)$}{
                $s_b \gets s$\;
                \Comment{Used in GOMEA a/i}
                \If{update-immidiately}{
                    $\mathcal{P}[idx] \gets s$\;
                }
            }{
                $s \gets s_b$\;
            }
        }
    }

    \caption{Task for GOMEA}
\end{algorithm}

\section{Modified Golden-Section Search for Population Size for the NASBench 301 experiment}
Finding the minimum time required to find the optimum is a roughly unimodal minimization problem. Too small a population size may require additional generations to find the optimum, or may not use all the resources available to a parallel GA. While a larger population requires more work per generation, which could slow down the algorithm too.

A key problem is that for some population sizes the optimum is not found at all, given our termination criteria. For small population sizes the approach is likely to prematurely converge, as such, first the bounds need to be determined.


This search starts off with finding a population size \(P_{1}\) for which the problem is solved, followed by evaluating the time necessary for a population twice the size \(P_{2} = 2P_{1}\). If the time required to solve the problem at this population size is larger, we can immediately continue with the triplet \((P_{0} = \frac{P_{1}}{2},P_{1},P_{2})\). Otherwise, we continue doubling \(P_{1}\) until this is the case. Given the triplet \((P_{0},\ P_{1},\ P_{2})\) of population sizes, we sample \(P_{3}\) at \(\frac{1}{3}\) or \(\frac{2}{3}\) between \(P_{0}\) and \(P_{2}\), such that it lies in the longest segment between \((P_{0},\ P_{1})\) or \(\left( P_{1},\ P_{2} \right)\), respectively. If the time required for \(P_{3}\) is greater than \(P_{1}\), we continue with the triplet \((P_{0},\ P_{1},\ P_{3})\), otherwise we continue with \((P_{1},P_{3},\ P_{2})\). The algorithm terminates with \(P_{1}\ \)if there are no new unevaluated population sizes between \(P_{0}\) and \(P_{2}\).

% \section{Extended Tables for the artificial benchmark functions}
% \afterpage{
%     \clearpage
%     \thispagestyle{empty
%     \begin{landscape}
%         % \centering % Center table
%         % 
\small
\begin{tblr}{
    cell{1}{2} = {c=9}{},
    cell{2}{2} = {c=9}{},
    cell{3}{2} = {c=9}{},
    cell{4}{2} = {c=3}{},
    cell{4}{5} = {c=3}{},
    cell{4}{8} = {c=3}{},
    cell{7}{2} = {r},
    cell{7}{3} = {r},
    cell{7}{4} = {r},
    cell{7}{5} = {r},
    cell{7}{6} = {r},
    cell{7}{7} = {r},
    cell{7}{8} = {r},
    cell{7}{9} = {r},
    cell{7}{10} = {r},
    cell{8}{2} = {r},
    cell{8}{3} = {r},
    cell{8}{4} = {r},
    cell{8}{5} = {r},
    cell{8}{6} = {r},
    cell{8}{7} = {r},
    cell{8}{8} = {r},
    cell{8}{9} = {r},
    cell{8}{10} = {r},
    cell{9}{2} = {r},
    cell{9}{3} = {r},
    cell{9}{4} = {r},
    cell{9}{5} = {r},
    cell{9}{6} = {r},
    cell{9}{7} = {r},
    cell{9}{8} = {r},
    cell{9}{9} = {r},
    cell{9}{10} = {r},
    cell{10}{2} = {r},
    cell{10}{3} = {r},
    cell{10}{4} = {r},
    cell{10}{5} = {r},
    cell{10}{6} = {r},
    cell{10}{7} = {r},
    cell{10}{8} = {r},
    cell{10}{9} = {r},
    cell{10}{10} = {r},
    cell{11}{2} = {r},
    cell{11}{3} = {r},
    cell{11}{4} = {r},
    cell{11}{5} = {r},
    cell{11}{6} = {r},
    cell{11}{7} = {r},
    cell{11}{8} = {r},
    cell{11}{9} = {r},
    cell{11}{10} = {r},
    cell{12}{2} = {r},
    cell{12}{3} = {r},
    cell{12}{4} = {r},
    cell{12}{5} = {r},
    cell{12}{6} = {r},
    cell{12}{7} = {r},
    cell{12}{8} = {r},
    cell{12}{9} = {r},
    cell{12}{10} = {r},
    cell{13}{2} = {r},
    cell{13}{3} = {r},
    cell{13}{4} = {r},
    cell{13}{5} = {r},
    cell{13}{6} = {r},
    cell{13}{7} = {r},
    cell{13}{8} = {r},
    cell{13}{9} = {r},
    cell{13}{10} = {r},
    cell{14}{2} = {c=18}{},
    cell{15}{2} = {c=6}{},
    cell{15}{8} = {c=12}{},
    cell{16}{2} = {c=6}{},
    cell{16}{8} = {c=6}{},
    cell{16}{14} = {c=6}{},
    cell{17}{2} = {c=3}{},
    cell{17}{5} = {c=3}{},
    cell{17}{8} = {c=3}{},
    cell{17}{11} = {c=3}{},
    cell{17}{14} = {c=3}{},
    cell{17}{17} = {c=3}{},
    cell{20}{2} = {r},
    cell{20}{3} = {r},
    cell{20}{4} = {r},
    cell{20}{5} = {r},
    cell{20}{6} = {r},
    cell{20}{7} = {r},
    cell{20}{8} = {r},
    cell{20}{9} = {r},
    cell{20}{10} = {r},
    cell{20}{11} = {r},
    cell{20}{12} = {r},
    cell{20}{13} = {r},
    cell{20}{14} = {r},
    cell{20}{15} = {r},
    cell{20}{16} = {r},
    cell{20}{17} = {r},
    cell{20}{18} = {r},
    cell{20}{19} = {r},
    cell{21}{2} = {r},
    cell{21}{3} = {r},
    cell{21}{4} = {r},
    cell{21}{5} = {r},
    cell{21}{6} = {r},
    cell{21}{7} = {r},
    cell{21}{8} = {r},
    cell{21}{9} = {r},
    cell{21}{10} = {r},
    cell{21}{11} = {r},
    cell{21}{12} = {r},
    cell{21}{13} = {r},
    cell{21}{14} = {r},
    cell{21}{15} = {r},
    cell{21}{16} = {r},
    cell{21}{17} = {r},
    cell{21}{18} = {r},
    cell{21}{19} = {r},
    cell{22}{2} = {r},
    cell{22}{3} = {r},
    cell{22}{4} = {r},
    cell{22}{5} = {r},
    cell{22}{6} = {r},
    cell{22}{7} = {r},
    cell{22}{8} = {r},
    cell{22}{9} = {r},
    cell{22}{10} = {r},
    cell{22}{11} = {r},
    cell{22}{12} = {r},
    cell{22}{13} = {r},
    cell{22}{14} = {r},
    cell{22}{15} = {r},
    cell{22}{16} = {r},
    cell{22}{17} = {r},
    cell{22}{18} = {r},
    cell{22}{19} = {r},
    cell{23}{2} = {r},
    cell{23}{3} = {r},
    cell{23}{4} = {r},
    cell{23}{5} = {r},
    cell{23}{6} = {r},
    cell{23}{7} = {r},
    cell{23}{8} = {r},
    cell{23}{9} = {r},
    cell{23}{10} = {r},
    cell{23}{11} = {r},
    cell{23}{12} = {r},
    cell{23}{13} = {r},
    cell{23}{14} = {r},
    cell{23}{15} = {r},
    cell{23}{16} = {r},
    cell{23}{17} = {r},
    cell{23}{18} = {r},
    cell{23}{19} = {r},
    cell{24}{2} = {r},
    cell{24}{3} = {r},
    cell{24}{4} = {r},
    cell{24}{5} = {r},
    cell{24}{6} = {r},
    cell{24}{7} = {r},
    cell{24}{8} = {r},
    cell{24}{9} = {r},
    cell{24}{10} = {r},
    cell{24}{11} = {r},
    cell{24}{12} = {r},
    cell{24}{13} = {r},
    cell{24}{14} = {r},
    cell{24}{15} = {r},
    cell{24}{16} = {r},
    cell{24}{17} = {r},
    cell{24}{18} = {r},
    cell{24}{19} = {r},
    cell{25}{2} = {r},
    cell{25}{3} = {r},
    cell{25}{4} = {r},
    cell{25}{5} = {r},
    cell{25}{6} = {r},
    cell{25}{7} = {r},
    cell{25}{8} = {r},
    cell{25}{9} = {r},
    cell{25}{10} = {r},
    cell{25}{11} = {r},
    cell{25}{12} = {r},
    cell{25}{13} = {r},
    cell{25}{14} = {r},
    cell{25}{15} = {r},
    cell{25}{16} = {r},
    cell{25}{17} = {r},
    cell{25}{18} = {r},
    cell{25}{19} = {r},
    cell{26}{2} = {r},
    cell{26}{3} = {r},
    cell{26}{4} = {r},
    cell{26}{5} = {r},
    cell{26}{6} = {r},
    cell{26}{7} = {r},
    cell{26}{8} = {r},
    cell{26}{9} = {r},
    cell{26}{10} = {r},
    cell{26}{11} = {r},
    cell{26}{12} = {r},
    cell{26}{13} = {r},
    cell{26}{14} = {r},
    cell{26}{15} = {r},
    cell{26}{16} = {r},
    cell{26}{17} = {r},
    cell{26}{18} = {r},
    cell{26}{19} = {r},
    cell{27}{2} = {c=18}{},
    cell{28}{2} = {c=6}{},
    cell{28}{8} = {c=12}{},
    cell{29}{2} = {c=6}{},
    cell{29}{8} = {c=6}{},
    cell{29}{14} = {c=6}{},
    cell{30}{2} = {c=3}{},
    cell{30}{5} = {c=3}{},
    cell{30}{8} = {c=3}{},
    cell{30}{11} = {c=3}{},
    cell{30}{14} = {c=3}{},
    cell{30}{17} = {c=3}{},
    cell{33}{2} = {r},
    cell{33}{3} = {r},
    cell{33}{4} = {r},
    cell{33}{5} = {r},
    cell{33}{6} = {r},
    cell{33}{7} = {r},
    cell{33}{8} = {r},
    cell{33}{9} = {r},
    cell{33}{10} = {r},
    cell{33}{11} = {r},
    cell{33}{12} = {r},
    cell{33}{13} = {r},
    cell{33}{14} = {r},
    cell{33}{15} = {r},
    cell{33}{16} = {r},
    cell{33}{17} = {r},
    cell{33}{18} = {r},
    cell{33}{19} = {r},
    cell{34}{2} = {r},
    cell{34}{3} = {r},
    cell{34}{4} = {r},
    cell{34}{5} = {r},
    cell{34}{6} = {r},
    cell{34}{7} = {r},
    cell{34}{8} = {r},
    cell{34}{9} = {r},
    cell{34}{10} = {r},
    cell{34}{11} = {r},
    cell{34}{12} = {r},
    cell{34}{13} = {r},
    cell{34}{14} = {r},
    cell{34}{15} = {r},
    cell{34}{16} = {r},
    cell{34}{17} = {r},
    cell{34}{18} = {r},
    cell{34}{19} = {r},
    cell{35}{2} = {r},
    cell{35}{3} = {r},
    cell{35}{4} = {r},
    cell{35}{5} = {r},
    cell{35}{6} = {r},
    cell{35}{7} = {r},
    cell{35}{8} = {r},
    cell{35}{9} = {r},
    cell{35}{10} = {r},
    cell{35}{11} = {r},
    cell{35}{12} = {r},
    cell{35}{13} = {r},
    cell{35}{14} = {r},
    cell{35}{15} = {r},
    cell{35}{16} = {r},
    cell{35}{17} = {r},
    cell{35}{18} = {r},
    cell{35}{19} = {r},
    cell{36}{2} = {r},
    cell{36}{3} = {r},
    cell{36}{4} = {r},
    cell{36}{5} = {r},
    cell{36}{6} = {r},
    cell{36}{7} = {r},
    cell{36}{8} = {r},
    cell{36}{9} = {r},
    cell{36}{10} = {r},
    cell{36}{11} = {r},
    cell{36}{12} = {r},
    cell{36}{13} = {r},
    cell{36}{14} = {r},
    cell{36}{15} = {r},
    cell{36}{16} = {r},
    cell{36}{17} = {r},
    cell{36}{18} = {r},
    cell{36}{19} = {r},
    cell{37}{2} = {r},
    cell{37}{3} = {r},
    cell{37}{4} = {r},
    cell{37}{5} = {r},
    cell{37}{6} = {r},
    cell{37}{7} = {r},
    cell{37}{8} = {r},
    cell{37}{9} = {r},
    cell{37}{10} = {r},
    cell{37}{11} = {r},
    cell{37}{12} = {r},
    cell{37}{13} = {r},
    cell{37}{14} = {r},
    cell{37}{15} = {r},
    cell{37}{16} = {r},
    cell{37}{17} = {r},
    cell{37}{18} = {r},
    cell{37}{19} = {r},
    cell{38}{2} = {r},
    cell{38}{3} = {r},
    cell{38}{4} = {r},
    cell{38}{5} = {r},
    cell{38}{6} = {r},
    cell{38}{7} = {r},
    cell{38}{8} = {r},
    cell{38}{9} = {r},
    cell{38}{10} = {r},
    cell{38}{11} = {r},
    cell{38}{12} = {r},
    cell{38}{13} = {r},
    cell{38}{14} = {r},
    cell{38}{15} = {r},
    cell{38}{16} = {r},
    cell{38}{17} = {r},
    cell{38}{18} = {r},
    cell{38}{19} = {r},
    cell{39}{2} = {r},
    cell{39}{3} = {r},
    cell{39}{4} = {r},
    cell{39}{5} = {r},
    cell{39}{6} = {r},
    cell{39}{7} = {r},
    cell{39}{8} = {r},
    cell{39}{9} = {r},
    cell{39}{10} = {r},
    cell{39}{11} = {r},
    cell{39}{12} = {r},
    cell{39}{13} = {r},
    cell{39}{14} = {r},
    cell{39}{15} = {r},
    cell{39}{16} = {r},
    cell{39}{17} = {r},
    cell{39}{18} = {r},
    cell{39}{19} = {r},
    vline{2} = {-}{},
    hline{1,7} = {1-10}{},
    hline{14,20,27,33,40} = {-}{},
  }
  \textbf{selection} & \textbf{GOM}          &               &               &               &               &               &               &               &               &               &               &               &               &               &               &               &               &               \\
  \textbf{approach}  & \textbf{GOMEA}        &               &               &               &               &               &               &               &               &               &               &               &               &               &               &               &               &               \\
  \textbf{cx}        & \textbf{LL-LT}        &               &               &               &               &               &               &               &               &               &               &               &               &               &               &               &               &               \\
  \textbf{(a)sync}   & \textbf{a/e}          &               &               & \textbf{a/i}  &               &               & \textbf{s}    &               &               &               &               &               &               &               &               &               &               &               \\
  \textbf{quantile}  & \textbf{0.25}         & \textbf{0.50} & \textbf{0.75} & \textbf{0.25} & \textbf{0.50} & \textbf{0.75} & \textbf{0.25} & \textbf{0.50} & \textbf{0.75} &               &               &               &               &               &               &               &               &               \\
  \textbf{timing}    &                       &               &               &               &               &               &               &               &               &               &               &               &               &               &               &               &               &               \\
  100:1              & 36                    & 44            & 53            & 36            & 44            & 53            & 36            & 44            & 53            &               &               &               &               &               &               &               &               &               \\
  10:1               & 36                    & 44            & 53            & 36            & 44            & 53            & 36            & 44            & 53            &               &               &               &               &               &               &               &               &               \\
  2:1                & 36                    & 44            & 53            & 36            & 44            & 53            & 36            & 44            & 53            &               &               &               &               &               &               &               &               &               \\
  1:1                & 36                    & 44            & 53            & 36            & 44            & 53            & 36            & 44            & 53            &               &               &               &               &               &               &               &               &               \\
  1:2                & 36                    & 44            & 56            & 36            & 44            & 53            & 36            & 44            & 53            &               &               &               &               &               &               &               &               &               \\
  1:10               & 36                    & 44            & 56            & 36            & 44            & 53            & 36            & 44            & 53            &               &               &               &               &               &               &               &               &               \\
  1:100              & 40                    & 52            & 64            & 36            & 44            & 53            & 36            & 44            & 53            &               &               &               &               &               &               &               &               &               \\
  \textbf{selection} & \textbf{steady-state} &               &               &               &               &               &               &               &               &               &               &               &               &               &               &               &               &               \\
  \textbf{approach}  & \textbf{ECGA}         &               &               &               &               &               & \textbf{GA}   &               &               &               &               &               &               &               &               &               &               &               \\
  \textbf{cx}        & \textbf{LL-MPM}       &               &               &               &               &               & \textbf{TPX}  &               &               &               &               &               & \textbf{SFX}  &               &               &               &               &               \\
  \textbf{(a)sync}   & \textbf{a}            &               &               & \textbf{s}    &               &               & \textbf{a}    &               &               & \textbf{s}    &               &               & \textbf{a}    &               &               & \textbf{s}    &               &               \\
  \textbf{quantile}  & \textbf{0.25}         & \textbf{0.50} & \textbf{0.75} & \textbf{0.25} & \textbf{0.50} & \textbf{0.75} & \textbf{0.25} & \textbf{0.50} & \textbf{0.75} & \textbf{0.25} & \textbf{0.50} & \textbf{0.75} & \textbf{0.25} & \textbf{0.50} & \textbf{0.75} & \textbf{0.25} & \textbf{0.50} & \textbf{0.75} \\
  \textbf{timing}    &                       &               &               &               &               &               &               &               &               &               &               &               &               &               &               &               &               &               \\
  100:1              & 1787                  & 1944          & 2100          & 4024          & 4216          & 4655          & 152           & 192           & 241           & 216           & 262           & 304           & 80            & 102           & 128           & 108           & 132           & 158           \\
  10:1               & 1909                  & 2038          & 2136          & 4024          & 4216          & 4655          & 160           & 196           & 238           & 216           & 262           & 304           & 87            & 102           & 128           & 108           & 132           & 158           \\
  2:1                & 2014                  & 2166          & 2365          & 4024          & 4216          & 4655          & 160           & 208           & 246           & 216           & 262           & 304           & 76            & 104           & 116           & 108           & 132           & 158           \\
  1:1                & 3894                  & 4110          & 4612          & 3894          & 4110          & 4612          & 205           & 264           & 329           & 205           & 264           & 329           & 96            & 130           & 161           & 96            & 130           & 161           \\
  1:2                & 1975                  & 2110          & 2313          & 3955          & 4118          & 4632          & 206           & 256           & 328           & 192           & 244           & 304           & 96            & 120           & 148           & 96            & 122           & 148           \\
  1:10               & 2226                  & 2394          & 2593          & 3955          & 4118          & 4632          & 243           & 286           & 341           & 192           & 244           & 304           & 104           & 132           & 152           & 96            & 122           & 148           \\
  1:100              & 2396                  & 2562          & 2756          & 3955          & 4118          & 4632          & 239           & 288           & 356           & 192           & 244           & 304           & 100           & 128           & 156           & 96            & 122           & 148           \\
  \textbf{selection} & \textbf{generational} &               &               &               &               &               &               &               &               &               &               &               &               &               &               &               &               &               \\
  \textbf{approach}  & \textbf{ECGA}         &               &               &               &               &               & \textbf{GA}   &               &               &               &               &               &               &               &               &               &               &               \\
  \textbf{cx}        & \textbf{LL-MPM}       &               &               &               &               &               & \textbf{TPX}  &               &               &               &               &               & \textbf{SFX}  &               &               &               &               &               \\
  \textbf{(a)sync}   & \textbf{a}            &               &               & \textbf{s}    &               &               & \textbf{a}    &               &               & \textbf{s}    &               &               & \textbf{a}    &               &               & \textbf{s}    &               &               \\
  \textbf{quantile}  & \textbf{0.25}         & \textbf{0.50} & \textbf{0.75} & \textbf{0.25} & \textbf{0.50} & \textbf{0.75} & \textbf{0.25} & \textbf{0.50} & \textbf{0.75} & \textbf{0.25} & \textbf{0.50} & \textbf{0.75} & \textbf{0.25} & \textbf{0.50} & \textbf{0.75} & \textbf{0.25} & \textbf{0.50} & \textbf{0.75} \\
  \textbf{timing}    &                       &               &               &               &               &               &               &               &               &               &               &               &               &               &               &               &               &               \\
  100:1              & 1753                  & 1906          & 2085          & 3759          & 4046          & 4380          & 316           & 434           & 537           & 403           & 546           & 684           & 128           & 176           & 204           & 152           & 190           & 224           \\
  10:1               & 1824                  & 2022          & 2173          & 3759          & 4046          & 4380          & 319           & 450           & 558           & 403           & 546           & 684           & 147           & 174           & 212           & 152           & 190           & 224           \\
  2:1                & 1915                  & 2042          & 2180          & 3759          & 4046          & 4380          & 422           & 512           & 661           & 403           & 546           & 684           & 128           & 164           & 228           & 152           & 190           & 224           \\
  1:1                & 3712                  & 4052          & 4269          & 3712          & 4052          & 4269          & 435           & 496           & 646           & 435           & 496           & 646           & 143           & 182           & 248           & 143           & 182           & 248           \\
  1:2                & 1920                  & 2060          & 2222          & 3823          & 4050          & 4492          & 320           & 482           & 598           & 384           & 508           & 637           & 128           & 168           & 218           & 152           & 188           & 220           \\
  1:10               & 2149                  & 2316          & 2528          & 3823          & 4050          & 4492          & 451           & 584           & 772           & 384           & 508           & 637           & 155           & 198           & 240           & 152           & 188           & 220           \\
  1:100              & 2304                  & 2534          & 2693          & 3823          & 4050          & 4492          & 507           & 634           & 863           & 384           & 508           & 637           & 180           & 234           & 268           & 152           & 188           & 220           
  \end{tblr}
  \vspace*{1mm}
  \captionof{table}{Quantiles of minimally required population sizes for DT}\label{tab:table-dt-ext}
%         % \captionof{table}{Table caption}
%     \end{landscape}
%     \clearpage
% }
% \afterpage{%
%     \clearpage
%     \thispagestyle{empty}
    
%     \clearpage
% }
\section{Extended Tables}
As only the median on its own is not always as insightful, we have included additional tables containing the 25 and 75 percentiles as well. See Table~\ref{tab:table-dt-ext} for Deceptive Trap, Table~\ref{tab:table-ankl-ext} for Adjacent NK-Landscapes, and Table~\ref{tab:table-nasbench-ext} for NASBench 301.

\newgeometry{left=0.1cm,right=0.1cm,top=0.1cm,bottom=0.1cm}
\begin{landscape}
    % \begin{tabular}{llll}
        \centering
    %     A & B & C & D \\
    % \end{tabular}
    
\small
\begin{tblr}{
    cell{1}{2} = {c=9}{},
    cell{2}{2} = {c=9}{},
    cell{3}{2} = {c=9}{},
    cell{4}{2} = {c=3}{},
    cell{4}{5} = {c=3}{},
    cell{4}{8} = {c=3}{},
    cell{7}{2} = {r},
    cell{7}{3} = {r},
    cell{7}{4} = {r},
    cell{7}{5} = {r},
    cell{7}{6} = {r},
    cell{7}{7} = {r},
    cell{7}{8} = {r},
    cell{7}{9} = {r},
    cell{7}{10} = {r},
    cell{8}{2} = {r},
    cell{8}{3} = {r},
    cell{8}{4} = {r},
    cell{8}{5} = {r},
    cell{8}{6} = {r},
    cell{8}{7} = {r},
    cell{8}{8} = {r},
    cell{8}{9} = {r},
    cell{8}{10} = {r},
    cell{9}{2} = {r},
    cell{9}{3} = {r},
    cell{9}{4} = {r},
    cell{9}{5} = {r},
    cell{9}{6} = {r},
    cell{9}{7} = {r},
    cell{9}{8} = {r},
    cell{9}{9} = {r},
    cell{9}{10} = {r},
    cell{10}{2} = {r},
    cell{10}{3} = {r},
    cell{10}{4} = {r},
    cell{10}{5} = {r},
    cell{10}{6} = {r},
    cell{10}{7} = {r},
    cell{10}{8} = {r},
    cell{10}{9} = {r},
    cell{10}{10} = {r},
    cell{11}{2} = {r},
    cell{11}{3} = {r},
    cell{11}{4} = {r},
    cell{11}{5} = {r},
    cell{11}{6} = {r},
    cell{11}{7} = {r},
    cell{11}{8} = {r},
    cell{11}{9} = {r},
    cell{11}{10} = {r},
    cell{12}{2} = {r},
    cell{12}{3} = {r},
    cell{12}{4} = {r},
    cell{12}{5} = {r},
    cell{12}{6} = {r},
    cell{12}{7} = {r},
    cell{12}{8} = {r},
    cell{12}{9} = {r},
    cell{12}{10} = {r},
    cell{13}{2} = {r},
    cell{13}{3} = {r},
    cell{13}{4} = {r},
    cell{13}{5} = {r},
    cell{13}{6} = {r},
    cell{13}{7} = {r},
    cell{13}{8} = {r},
    cell{13}{9} = {r},
    cell{13}{10} = {r},
    cell{14}{2} = {c=18}{},
    cell{15}{2} = {c=6}{},
    cell{15}{8} = {c=12}{},
    cell{16}{2} = {c=6}{},
    cell{16}{8} = {c=6}{},
    cell{16}{14} = {c=6}{},
    cell{17}{2} = {c=3}{},
    cell{17}{5} = {c=3}{},
    cell{17}{8} = {c=3}{},
    cell{17}{11} = {c=3}{},
    cell{17}{14} = {c=3}{},
    cell{17}{17} = {c=3}{},
    cell{20}{2} = {r},
    cell{20}{3} = {r},
    cell{20}{4} = {r},
    cell{20}{5} = {r},
    cell{20}{6} = {r},
    cell{20}{7} = {r},
    cell{20}{8} = {r},
    cell{20}{9} = {r},
    cell{20}{10} = {r},
    cell{20}{11} = {r},
    cell{20}{12} = {r},
    cell{20}{13} = {r},
    cell{20}{14} = {r},
    cell{20}{15} = {r},
    cell{20}{16} = {r},
    cell{20}{17} = {r},
    cell{20}{18} = {r},
    cell{20}{19} = {r},
    cell{21}{2} = {r},
    cell{21}{3} = {r},
    cell{21}{4} = {r},
    cell{21}{5} = {r},
    cell{21}{6} = {r},
    cell{21}{7} = {r},
    cell{21}{8} = {r},
    cell{21}{9} = {r},
    cell{21}{10} = {r},
    cell{21}{11} = {r},
    cell{21}{12} = {r},
    cell{21}{13} = {r},
    cell{21}{14} = {r},
    cell{21}{15} = {r},
    cell{21}{16} = {r},
    cell{21}{17} = {r},
    cell{21}{18} = {r},
    cell{21}{19} = {r},
    cell{22}{2} = {r},
    cell{22}{3} = {r},
    cell{22}{4} = {r},
    cell{22}{5} = {r},
    cell{22}{6} = {r},
    cell{22}{7} = {r},
    cell{22}{8} = {r},
    cell{22}{9} = {r},
    cell{22}{10} = {r},
    cell{22}{11} = {r},
    cell{22}{12} = {r},
    cell{22}{13} = {r},
    cell{22}{14} = {r},
    cell{22}{15} = {r},
    cell{22}{16} = {r},
    cell{22}{17} = {r},
    cell{22}{18} = {r},
    cell{22}{19} = {r},
    cell{23}{2} = {r},
    cell{23}{3} = {r},
    cell{23}{4} = {r},
    cell{23}{5} = {r},
    cell{23}{6} = {r},
    cell{23}{7} = {r},
    cell{23}{8} = {r},
    cell{23}{9} = {r},
    cell{23}{10} = {r},
    cell{23}{11} = {r},
    cell{23}{12} = {r},
    cell{23}{13} = {r},
    cell{23}{14} = {r},
    cell{23}{15} = {r},
    cell{23}{16} = {r},
    cell{23}{17} = {r},
    cell{23}{18} = {r},
    cell{23}{19} = {r},
    cell{24}{2} = {r},
    cell{24}{3} = {r},
    cell{24}{4} = {r},
    cell{24}{5} = {r},
    cell{24}{6} = {r},
    cell{24}{7} = {r},
    cell{24}{8} = {r},
    cell{24}{9} = {r},
    cell{24}{10} = {r},
    cell{24}{11} = {r},
    cell{24}{12} = {r},
    cell{24}{13} = {r},
    cell{24}{14} = {r},
    cell{24}{15} = {r},
    cell{24}{16} = {r},
    cell{24}{17} = {r},
    cell{24}{18} = {r},
    cell{24}{19} = {r},
    cell{25}{2} = {r},
    cell{25}{3} = {r},
    cell{25}{4} = {r},
    cell{25}{5} = {r},
    cell{25}{6} = {r},
    cell{25}{7} = {r},
    cell{25}{8} = {r},
    cell{25}{9} = {r},
    cell{25}{10} = {r},
    cell{25}{11} = {r},
    cell{25}{12} = {r},
    cell{25}{13} = {r},
    cell{25}{14} = {r},
    cell{25}{15} = {r},
    cell{25}{16} = {r},
    cell{25}{17} = {r},
    cell{25}{18} = {r},
    cell{25}{19} = {r},
    cell{26}{2} = {r},
    cell{26}{3} = {r},
    cell{26}{4} = {r},
    cell{26}{5} = {r},
    cell{26}{6} = {r},
    cell{26}{7} = {r},
    cell{26}{8} = {r},
    cell{26}{9} = {r},
    cell{26}{10} = {r},
    cell{26}{11} = {r},
    cell{26}{12} = {r},
    cell{26}{13} = {r},
    cell{26}{14} = {r},
    cell{26}{15} = {r},
    cell{26}{16} = {r},
    cell{26}{17} = {r},
    cell{26}{18} = {r},
    cell{26}{19} = {r},
    cell{27}{2} = {c=18}{},
    cell{28}{2} = {c=6}{},
    cell{28}{8} = {c=12}{},
    cell{29}{2} = {c=6}{},
    cell{29}{8} = {c=6}{},
    cell{29}{14} = {c=6}{},
    cell{30}{2} = {c=3}{},
    cell{30}{5} = {c=3}{},
    cell{30}{8} = {c=3}{},
    cell{30}{11} = {c=3}{},
    cell{30}{14} = {c=3}{},
    cell{30}{17} = {c=3}{},
    cell{33}{2} = {r},
    cell{33}{3} = {r},
    cell{33}{4} = {r},
    cell{33}{5} = {r},
    cell{33}{6} = {r},
    cell{33}{7} = {r},
    cell{33}{8} = {r},
    cell{33}{9} = {r},
    cell{33}{10} = {r},
    cell{33}{11} = {r},
    cell{33}{12} = {r},
    cell{33}{13} = {r},
    cell{33}{14} = {r},
    cell{33}{15} = {r},
    cell{33}{16} = {r},
    cell{33}{17} = {r},
    cell{33}{18} = {r},
    cell{33}{19} = {r},
    cell{34}{2} = {r},
    cell{34}{3} = {r},
    cell{34}{4} = {r},
    cell{34}{5} = {r},
    cell{34}{6} = {r},
    cell{34}{7} = {r},
    cell{34}{8} = {r},
    cell{34}{9} = {r},
    cell{34}{10} = {r},
    cell{34}{11} = {r},
    cell{34}{12} = {r},
    cell{34}{13} = {r},
    cell{34}{14} = {r},
    cell{34}{15} = {r},
    cell{34}{16} = {r},
    cell{34}{17} = {r},
    cell{34}{18} = {r},
    cell{34}{19} = {r},
    cell{35}{2} = {r},
    cell{35}{3} = {r},
    cell{35}{4} = {r},
    cell{35}{5} = {r},
    cell{35}{6} = {r},
    cell{35}{7} = {r},
    cell{35}{8} = {r},
    cell{35}{9} = {r},
    cell{35}{10} = {r},
    cell{35}{11} = {r},
    cell{35}{12} = {r},
    cell{35}{13} = {r},
    cell{35}{14} = {r},
    cell{35}{15} = {r},
    cell{35}{16} = {r},
    cell{35}{17} = {r},
    cell{35}{18} = {r},
    cell{35}{19} = {r},
    cell{36}{2} = {r},
    cell{36}{3} = {r},
    cell{36}{4} = {r},
    cell{36}{5} = {r},
    cell{36}{6} = {r},
    cell{36}{7} = {r},
    cell{36}{8} = {r},
    cell{36}{9} = {r},
    cell{36}{10} = {r},
    cell{36}{11} = {r},
    cell{36}{12} = {r},
    cell{36}{13} = {r},
    cell{36}{14} = {r},
    cell{36}{15} = {r},
    cell{36}{16} = {r},
    cell{36}{17} = {r},
    cell{36}{18} = {r},
    cell{36}{19} = {r},
    cell{37}{2} = {r},
    cell{37}{3} = {r},
    cell{37}{4} = {r},
    cell{37}{5} = {r},
    cell{37}{6} = {r},
    cell{37}{7} = {r},
    cell{37}{8} = {r},
    cell{37}{9} = {r},
    cell{37}{10} = {r},
    cell{37}{11} = {r},
    cell{37}{12} = {r},
    cell{37}{13} = {r},
    cell{37}{14} = {r},
    cell{37}{15} = {r},
    cell{37}{16} = {r},
    cell{37}{17} = {r},
    cell{37}{18} = {r},
    cell{37}{19} = {r},
    cell{38}{2} = {r},
    cell{38}{3} = {r},
    cell{38}{4} = {r},
    cell{38}{5} = {r},
    cell{38}{6} = {r},
    cell{38}{7} = {r},
    cell{38}{8} = {r},
    cell{38}{9} = {r},
    cell{38}{10} = {r},
    cell{38}{11} = {r},
    cell{38}{12} = {r},
    cell{38}{13} = {r},
    cell{38}{14} = {r},
    cell{38}{15} = {r},
    cell{38}{16} = {r},
    cell{38}{17} = {r},
    cell{38}{18} = {r},
    cell{38}{19} = {r},
    cell{39}{2} = {r},
    cell{39}{3} = {r},
    cell{39}{4} = {r},
    cell{39}{5} = {r},
    cell{39}{6} = {r},
    cell{39}{7} = {r},
    cell{39}{8} = {r},
    cell{39}{9} = {r},
    cell{39}{10} = {r},
    cell{39}{11} = {r},
    cell{39}{12} = {r},
    cell{39}{13} = {r},
    cell{39}{14} = {r},
    cell{39}{15} = {r},
    cell{39}{16} = {r},
    cell{39}{17} = {r},
    cell{39}{18} = {r},
    cell{39}{19} = {r},
    vline{2} = {-}{},
    hline{1,7} = {1-10}{},
    hline{14,20,27,33,40} = {-}{},
  }
  \textbf{selection} & \textbf{GOM}          &               &               &               &               &               &               &               &               &               &               &               &               &               &               &               &               &               \\
  \textbf{approach}  & \textbf{GOMEA}        &               &               &               &               &               &               &               &               &               &               &               &               &               &               &               &               &               \\
  \textbf{cx}        & \textbf{LL-LT}        &               &               &               &               &               &               &               &               &               &               &               &               &               &               &               &               &               \\
  \textbf{(a)sync}   & \textbf{a/e}          &               &               & \textbf{a/i}  &               &               & \textbf{s}    &               &               &               &               &               &               &               &               &               &               &               \\
  \textbf{quantile}  & \textbf{0.25}         & \textbf{0.50} & \textbf{0.75} & \textbf{0.25} & \textbf{0.50} & \textbf{0.75} & \textbf{0.25} & \textbf{0.50} & \textbf{0.75} &               &               &               &               &               &               &               &               &               \\
  \textbf{timing}    &                       &               &               &               &               &               &               &               &               &               &               &               &               &               &               &               &               &               \\
  100:1              & 36                    & 44            & 53            & 36            & 44            & 53            & 36            & 44            & 53            &               &               &               &               &               &               &               &               &               \\
  10:1               & 36                    & 44            & 53            & 36            & 44            & 53            & 36            & 44            & 53            &               &               &               &               &               &               &               &               &               \\
  2:1                & 36                    & 44            & 53            & 36            & 44            & 53            & 36            & 44            & 53            &               &               &               &               &               &               &               &               &               \\
  1:1                & 36                    & 44            & 53            & 36            & 44            & 53            & 36            & 44            & 53            &               &               &               &               &               &               &               &               &               \\
  1:2                & 36                    & 44            & 56            & 36            & 44            & 53            & 36            & 44            & 53            &               &               &               &               &               &               &               &               &               \\
  1:10               & 36                    & 44            & 56            & 36            & 44            & 53            & 36            & 44            & 53            &               &               &               &               &               &               &               &               &               \\
  1:100              & 40                    & 52            & 64            & 36            & 44            & 53            & 36            & 44            & 53            &               &               &               &               &               &               &               &               &               \\
  \textbf{selection} & \textbf{steady-state} &               &               &               &               &               &               &               &               &               &               &               &               &               &               &               &               &               \\
  \textbf{approach}  & \textbf{ECGA}         &               &               &               &               &               & \textbf{GA}   &               &               &               &               &               &               &               &               &               &               &               \\
  \textbf{cx}        & \textbf{LL-MPM}       &               &               &               &               &               & \textbf{TPX}  &               &               &               &               &               & \textbf{SFX}  &               &               &               &               &               \\
  \textbf{(a)sync}   & \textbf{a}            &               &               & \textbf{s}    &               &               & \textbf{a}    &               &               & \textbf{s}    &               &               & \textbf{a}    &               &               & \textbf{s}    &               &               \\
  \textbf{quantile}  & \textbf{0.25}         & \textbf{0.50} & \textbf{0.75} & \textbf{0.25} & \textbf{0.50} & \textbf{0.75} & \textbf{0.25} & \textbf{0.50} & \textbf{0.75} & \textbf{0.25} & \textbf{0.50} & \textbf{0.75} & \textbf{0.25} & \textbf{0.50} & \textbf{0.75} & \textbf{0.25} & \textbf{0.50} & \textbf{0.75} \\
  \textbf{timing}    &                       &               &               &               &               &               &               &               &               &               &               &               &               &               &               &               &               &               \\
  100:1              & 1787                  & 1944          & 2100          & 4024          & 4216          & 4655          & 152           & 192           & 241           & 216           & 262           & 304           & 80            & 102           & 128           & 108           & 132           & 158           \\
  10:1               & 1909                  & 2038          & 2136          & 4024          & 4216          & 4655          & 160           & 196           & 238           & 216           & 262           & 304           & 87            & 102           & 128           & 108           & 132           & 158           \\
  2:1                & 2014                  & 2166          & 2365          & 4024          & 4216          & 4655          & 160           & 208           & 246           & 216           & 262           & 304           & 76            & 104           & 116           & 108           & 132           & 158           \\
  1:1                & 3894                  & 4110          & 4612          & 3894          & 4110          & 4612          & 205           & 264           & 329           & 205           & 264           & 329           & 96            & 130           & 161           & 96            & 130           & 161           \\
  1:2                & 1975                  & 2110          & 2313          & 3955          & 4118          & 4632          & 206           & 256           & 328           & 192           & 244           & 304           & 96            & 120           & 148           & 96            & 122           & 148           \\
  1:10               & 2226                  & 2394          & 2593          & 3955          & 4118          & 4632          & 243           & 286           & 341           & 192           & 244           & 304           & 104           & 132           & 152           & 96            & 122           & 148           \\
  1:100              & 2396                  & 2562          & 2756          & 3955          & 4118          & 4632          & 239           & 288           & 356           & 192           & 244           & 304           & 100           & 128           & 156           & 96            & 122           & 148           \\
  \textbf{selection} & \textbf{generational} &               &               &               &               &               &               &               &               &               &               &               &               &               &               &               &               &               \\
  \textbf{approach}  & \textbf{ECGA}         &               &               &               &               &               & \textbf{GA}   &               &               &               &               &               &               &               &               &               &               &               \\
  \textbf{cx}        & \textbf{LL-MPM}       &               &               &               &               &               & \textbf{TPX}  &               &               &               &               &               & \textbf{SFX}  &               &               &               &               &               \\
  \textbf{(a)sync}   & \textbf{a}            &               &               & \textbf{s}    &               &               & \textbf{a}    &               &               & \textbf{s}    &               &               & \textbf{a}    &               &               & \textbf{s}    &               &               \\
  \textbf{quantile}  & \textbf{0.25}         & \textbf{0.50} & \textbf{0.75} & \textbf{0.25} & \textbf{0.50} & \textbf{0.75} & \textbf{0.25} & \textbf{0.50} & \textbf{0.75} & \textbf{0.25} & \textbf{0.50} & \textbf{0.75} & \textbf{0.25} & \textbf{0.50} & \textbf{0.75} & \textbf{0.25} & \textbf{0.50} & \textbf{0.75} \\
  \textbf{timing}    &                       &               &               &               &               &               &               &               &               &               &               &               &               &               &               &               &               &               \\
  100:1              & 1753                  & 1906          & 2085          & 3759          & 4046          & 4380          & 316           & 434           & 537           & 403           & 546           & 684           & 128           & 176           & 204           & 152           & 190           & 224           \\
  10:1               & 1824                  & 2022          & 2173          & 3759          & 4046          & 4380          & 319           & 450           & 558           & 403           & 546           & 684           & 147           & 174           & 212           & 152           & 190           & 224           \\
  2:1                & 1915                  & 2042          & 2180          & 3759          & 4046          & 4380          & 422           & 512           & 661           & 403           & 546           & 684           & 128           & 164           & 228           & 152           & 190           & 224           \\
  1:1                & 3712                  & 4052          & 4269          & 3712          & 4052          & 4269          & 435           & 496           & 646           & 435           & 496           & 646           & 143           & 182           & 248           & 143           & 182           & 248           \\
  1:2                & 1920                  & 2060          & 2222          & 3823          & 4050          & 4492          & 320           & 482           & 598           & 384           & 508           & 637           & 128           & 168           & 218           & 152           & 188           & 220           \\
  1:10               & 2149                  & 2316          & 2528          & 3823          & 4050          & 4492          & 451           & 584           & 772           & 384           & 508           & 637           & 155           & 198           & 240           & 152           & 188           & 220           \\
  1:100              & 2304                  & 2534          & 2693          & 3823          & 4050          & 4492          & 507           & 634           & 863           & 384           & 508           & 637           & 180           & 234           & 268           & 152           & 188           & 220           
  \end{tblr}
  \vspace*{1mm}
  \captionof{table}{Quantiles of minimally required population sizes for DT}\label{tab:table-dt-ext}

    % \usepackage{tabularray}
\small
\begin{tblr}{
    cell{1}{2} = {c=9}{},
    cell{2}{2} = {c=9}{},
    cell{3}{2} = {c=9}{},
    cell{4}{2} = {c=3}{},
    cell{4}{5} = {c=3}{},
    cell{4}{8} = {c=3}{},
    cell{7}{2} = {r},
    cell{7}{3} = {r},
    cell{7}{4} = {r},
    cell{7}{5} = {r},
    cell{7}{6} = {r},
    cell{7}{7} = {r},
    cell{7}{8} = {r},
    cell{7}{9} = {r},
    cell{7}{10} = {r},
    cell{8}{2} = {r},
    cell{8}{3} = {r},
    cell{8}{4} = {r},
    cell{8}{5} = {r},
    cell{8}{6} = {r},
    cell{8}{7} = {r},
    cell{8}{8} = {r},
    cell{8}{9} = {r},
    cell{8}{10} = {r},
    cell{9}{2} = {r},
    cell{9}{3} = {r},
    cell{9}{4} = {r},
    cell{9}{5} = {r},
    cell{9}{6} = {r},
    cell{9}{7} = {r},
    cell{9}{8} = {r},
    cell{9}{9} = {r},
    cell{9}{10} = {r},
    cell{10}{2} = {r},
    cell{10}{3} = {r},
    cell{10}{4} = {r},
    cell{10}{5} = {r},
    cell{10}{6} = {r},
    cell{10}{7} = {r},
    cell{10}{8} = {r},
    cell{10}{9} = {r},
    cell{10}{10} = {r},
    cell{11}{2} = {r},
    cell{11}{3} = {r},
    cell{11}{4} = {r},
    cell{11}{5} = {r},
    cell{11}{6} = {r},
    cell{11}{7} = {r},
    cell{11}{8} = {r},
    cell{11}{9} = {r},
    cell{11}{10} = {r},
    cell{12}{2} = {r},
    cell{12}{3} = {r},
    cell{12}{4} = {r},
    cell{12}{5} = {r},
    cell{12}{6} = {r},
    cell{12}{7} = {r},
    cell{12}{8} = {r},
    cell{12}{9} = {r},
    cell{12}{10} = {r},
    cell{13}{2} = {r},
    cell{13}{3} = {r},
    cell{13}{4} = {r},
    cell{13}{5} = {r},
    cell{13}{6} = {r},
    cell{13}{7} = {r},
    cell{13}{8} = {r},
    cell{13}{9} = {r},
    cell{13}{10} = {r},
    cell{14}{2} = {c=24}{},
    cell{15}{2} = {c=6}{},
    cell{15}{8} = {c=18}{},
    cell{16}{2} = {c=6}{},
    cell{16}{8} = {c=6}{},
    cell{16}{14} = {c=6}{},
    cell{16}{20} = {c=6}{},
    cell{17}{2} = {c=3}{},
    cell{17}{5} = {c=3}{},
    cell{17}{8} = {c=3}{},
    cell{17}{11} = {c=3}{},
    cell{17}{14} = {c=3}{},
    cell{17}{17} = {c=3}{},
    cell{17}{20} = {c=3}{},
    cell{17}{23} = {c=3}{},
    cell{20}{2} = {r},
    cell{20}{3} = {r},
    cell{20}{4} = {r},
    cell{20}{5} = {r},
    cell{20}{6} = {r},
    cell{20}{7} = {r},
    cell{20}{8} = {r},
    cell{20}{9} = {r},
    cell{20}{10} = {r},
    cell{20}{11} = {r},
    cell{20}{12} = {r},
    cell{20}{13} = {r},
    cell{20}{14} = {r},
    cell{20}{15} = {r},
    cell{20}{16} = {r},
    cell{20}{17} = {r},
    cell{20}{18} = {r},
    cell{20}{19} = {r},
    cell{20}{20} = {r},
    cell{20}{21} = {r},
    cell{20}{22} = {r},
    cell{20}{23} = {r},
    cell{20}{24} = {r},
    cell{20}{25} = {r},
    cell{21}{2} = {r},
    cell{21}{3} = {r},
    cell{21}{4} = {r},
    cell{21}{5} = {r},
    cell{21}{6} = {r},
    cell{21}{7} = {r},
    cell{21}{8} = {r},
    cell{21}{9} = {r},
    cell{21}{10} = {r},
    cell{21}{11} = {r},
    cell{21}{12} = {r},
    cell{21}{13} = {r},
    cell{21}{14} = {r},
    cell{21}{15} = {r},
    cell{21}{16} = {r},
    cell{21}{17} = {r},
    cell{21}{18} = {r},
    cell{21}{19} = {r},
    cell{21}{20} = {r},
    cell{21}{21} = {r},
    cell{21}{22} = {r},
    cell{21}{23} = {r},
    cell{21}{24} = {r},
    cell{21}{25} = {r},
    cell{22}{2} = {r},
    cell{22}{3} = {r},
    cell{22}{4} = {r},
    cell{22}{5} = {r},
    cell{22}{6} = {r},
    cell{22}{7} = {r},
    cell{22}{8} = {r},
    cell{22}{9} = {r},
    cell{22}{10} = {r},
    cell{22}{11} = {r},
    cell{22}{12} = {r},
    cell{22}{13} = {r},
    cell{22}{14} = {r},
    cell{22}{15} = {r},
    cell{22}{16} = {r},
    cell{22}{17} = {r},
    cell{22}{18} = {r},
    cell{22}{19} = {r},
    cell{22}{20} = {r},
    cell{22}{21} = {r},
    cell{22}{22} = {r},
    cell{22}{23} = {r},
    cell{22}{24} = {r},
    cell{22}{25} = {r},
    cell{23}{2} = {r},
    cell{23}{3} = {r},
    cell{23}{4} = {r},
    cell{23}{5} = {r},
    cell{23}{6} = {r},
    cell{23}{7} = {r},
    cell{23}{8} = {r},
    cell{23}{9} = {r},
    cell{23}{10} = {r},
    cell{23}{14} = {r},
    cell{23}{15} = {r},
    cell{23}{16} = {r},
    cell{23}{17} = {r},
    cell{23}{18} = {r},
    cell{23}{19} = {r},
    cell{23}{20} = {r},
    cell{23}{21} = {r},
    cell{23}{22} = {r},
    cell{23}{23} = {r},
    cell{23}{24} = {r},
    cell{23}{25} = {r},
    cell{24}{2} = {r},
    cell{24}{3} = {r},
    cell{24}{4} = {r},
    cell{24}{5} = {r},
    cell{24}{6} = {r},
    cell{24}{7} = {r},
    cell{24}{8} = {r},
    cell{24}{9} = {r},
    cell{24}{10} = {r},
    cell{24}{14} = {r},
    cell{24}{15} = {r},
    cell{24}{16} = {r},
    cell{24}{17} = {r},
    cell{24}{18} = {r},
    cell{24}{19} = {r},
    cell{24}{20} = {r},
    cell{24}{21} = {r},
    cell{24}{22} = {r},
    cell{24}{23} = {r},
    cell{24}{24} = {r},
    cell{25}{2} = {r},
    cell{25}{3} = {r},
    cell{25}{4} = {r},
    cell{25}{5} = {r},
    cell{25}{6} = {r},
    cell{25}{7} = {r},
    cell{25}{8} = {r},
    cell{25}{9} = {r},
    cell{25}{10} = {r},
    cell{25}{14} = {r},
    cell{25}{15} = {r},
    cell{25}{16} = {r},
    cell{25}{17} = {r},
    cell{25}{18} = {r},
    cell{25}{19} = {r},
    cell{25}{20} = {r},
    cell{25}{21} = {r},
    cell{25}{22} = {r},
    cell{25}{23} = {r},
    cell{25}{24} = {r},
    cell{26}{2} = {r},
    cell{26}{3} = {r},
    cell{26}{4} = {r},
    cell{26}{5} = {r},
    cell{26}{6} = {r},
    cell{26}{7} = {r},
    cell{26}{8} = {r},
    cell{26}{9} = {r},
    cell{26}{10} = {r},
    cell{26}{14} = {r},
    cell{26}{15} = {r},
    cell{26}{16} = {r},
    cell{26}{17} = {r},
    cell{26}{18} = {r},
    cell{26}{19} = {r},
    cell{26}{20} = {r},
    cell{26}{21} = {r},
    cell{26}{22} = {r},
    cell{26}{23} = {r},
    cell{27}{2} = {c=24}{},
    cell{28}{2} = {c=6}{},
    cell{28}{8} = {c=18}{},
    cell{29}{2} = {c=6}{},
    cell{29}{8} = {c=6}{},
    cell{29}{14} = {c=6}{},
    cell{29}{20} = {c=6}{},
    cell{30}{2} = {c=3}{},
    cell{30}{5} = {c=3}{},
    cell{30}{8} = {c=3}{},
    cell{30}{11} = {c=3}{},
    cell{30}{14} = {c=3}{},
    cell{30}{17} = {c=3}{},
    cell{30}{20} = {c=3}{},
    cell{30}{23} = {c=3}{},
    cell{33}{2} = {r},
    cell{33}{3} = {r},
    cell{33}{4} = {r},
    cell{33}{5} = {r},
    cell{33}{6} = {r},
    cell{33}{7} = {r},
    cell{33}{8} = {r},
    cell{33}{9} = {r},
    cell{33}{10} = {r},
    cell{33}{11} = {r},
    cell{33}{12} = {r},
    cell{33}{13} = {r},
    cell{33}{14} = {r},
    cell{33}{15} = {r},
    cell{33}{16} = {r},
    cell{33}{17} = {r},
    cell{33}{18} = {r},
    cell{33}{19} = {r},
    cell{33}{20} = {r},
    cell{33}{21} = {r},
    cell{33}{22} = {r},
    cell{33}{23} = {r},
    cell{33}{24} = {r},
    cell{33}{25} = {r},
    cell{34}{2} = {r},
    cell{34}{3} = {r},
    cell{34}{4} = {r},
    cell{34}{5} = {r},
    cell{34}{6} = {r},
    cell{34}{7} = {r},
    cell{34}{8} = {r},
    cell{34}{9} = {r},
    cell{34}{10} = {r},
    cell{34}{11} = {r},
    cell{34}{12} = {r},
    cell{34}{13} = {r},
    cell{34}{14} = {r},
    cell{34}{15} = {r},
    cell{34}{16} = {r},
    cell{34}{17} = {r},
    cell{34}{18} = {r},
    cell{34}{19} = {r},
    cell{34}{20} = {r},
    cell{34}{21} = {r},
    cell{34}{22} = {r},
    cell{34}{23} = {r},
    cell{34}{24} = {r},
    cell{34}{25} = {r},
    cell{35}{2} = {r},
    cell{35}{3} = {r},
    cell{35}{4} = {r},
    cell{35}{5} = {r},
    cell{35}{6} = {r},
    cell{35}{7} = {r},
    cell{35}{8} = {r},
    cell{35}{9} = {r},
    cell{35}{10} = {r},
    cell{35}{11} = {r},
    cell{35}{12} = {r},
    cell{35}{13} = {r},
    cell{35}{14} = {r},
    cell{35}{15} = {r},
    cell{35}{16} = {r},
    cell{35}{17} = {r},
    cell{35}{18} = {r},
    cell{35}{19} = {r},
    cell{35}{20} = {r},
    cell{35}{21} = {r},
    cell{35}{22} = {r},
    cell{35}{23} = {r},
    cell{35}{24} = {r},
    cell{35}{25} = {r},
    cell{36}{2} = {r},
    cell{36}{3} = {r},
    cell{36}{4} = {r},
    cell{36}{5} = {r},
    cell{36}{6} = {r},
    cell{36}{7} = {r},
    cell{36}{8} = {r},
    cell{36}{9} = {r},
    cell{36}{10} = {r},
    cell{36}{11} = {r},
    cell{36}{12} = {r},
    cell{36}{13} = {r},
    cell{36}{14} = {r},
    cell{36}{15} = {r},
    cell{36}{16} = {r},
    cell{36}{17} = {r},
    cell{36}{18} = {r},
    cell{36}{19} = {r},
    cell{36}{20} = {r},
    cell{36}{21} = {r},
    cell{36}{22} = {r},
    cell{36}{23} = {r},
    cell{36}{24} = {r},
    cell{36}{25} = {r},
    cell{37}{2} = {r},
    cell{37}{3} = {r},
    cell{37}{4} = {r},
    cell{37}{5} = {r},
    cell{37}{6} = {r},
    cell{37}{7} = {r},
    cell{37}{8} = {r},
    cell{37}{9} = {r},
    cell{37}{10} = {r},
    cell{37}{11} = {r},
    cell{37}{12} = {r},
    cell{37}{13} = {r},
    cell{37}{14} = {r},
    cell{37}{15} = {r},
    cell{37}{16} = {r},
    cell{37}{17} = {r},
    cell{37}{18} = {r},
    cell{37}{19} = {r},
    cell{37}{20} = {r},
    cell{37}{21} = {r},
    cell{37}{22} = {r},
    cell{37}{23} = {r},
    cell{37}{24} = {r},
    cell{37}{25} = {r},
    cell{38}{2} = {r},
    cell{38}{3} = {r},
    cell{38}{4} = {r},
    cell{38}{5} = {r},
    cell{38}{6} = {r},
    cell{38}{7} = {r},
    cell{38}{8} = {r},
    cell{38}{9} = {r},
    cell{38}{10} = {r},
    cell{38}{11} = {r},
    cell{38}{12} = {r},
    cell{38}{13} = {r},
    cell{38}{14} = {r},
    cell{38}{15} = {r},
    cell{38}{16} = {r},
    cell{38}{17} = {r},
    cell{38}{18} = {r},
    cell{38}{19} = {r},
    cell{38}{20} = {r},
    cell{38}{21} = {r},
    cell{38}{22} = {r},
    cell{38}{23} = {r},
    cell{38}{24} = {r},
    cell{38}{25} = {r},
    cell{39}{2} = {r},
    cell{39}{3} = {r},
    cell{39}{4} = {r},
    cell{39}{5} = {r},
    cell{39}{6} = {r},
    cell{39}{7} = {r},
    cell{39}{8} = {r},
    cell{39}{9} = {r},
    cell{39}{10} = {r},
    cell{39}{11} = {r},
    cell{39}{12} = {r},
    cell{39}{13} = {r},
    cell{39}{14} = {r},
    cell{39}{15} = {r},
    cell{39}{16} = {r},
    cell{39}{17} = {r},
    cell{39}{18} = {r},
    cell{39}{19} = {r},
    cell{39}{20} = {r},
    cell{39}{21} = {r},
    cell{39}{22} = {r},
    cell{39}{23} = {r},
    cell{39}{24} = {r},
    cell{39}{25} = {r},
    vline{2} = {1-39}{},
    hline{1,7} = {1-10}{},
    hline{14,20,27,33,40} = {-}{},
}
\textbf{selection} & \textbf{GOM}            &               &               &               &               &               &               &               &               &               &               &               &               &               &               &               &               &               &               &               &               &               &               &               \\
\textbf{approach}  & \textbf{GOMEA}        &               &               &               &               &               &               &               &               &               &               &               &               &               &               &               &               &               &               &               &               &               &               &               \\
\textbf{cx}        & \textbf{LL-LT}            &               &               &               &               &               &               &               &               &               &               &               &               &               &               &               &               &               &               &               &               &               &               &               \\
\textbf{(a)sync}   & \textbf{a/e}          &               &               & \textbf{a/i}  &               &               & \textbf{s}    &               &               &               &               &               &               &               &               &               &               &               &               &               &               &               &               &               \\
\textbf{quantile}  & \textbf{0.25}         & \textbf{0.50} & \textbf{0.75} & \textbf{0.25} & \textbf{0.50} & \textbf{0.75} & \textbf{0.25} & \textbf{0.50} & \textbf{0.75} &               &               &               &               &               &               &               &               &               &               &               &               &               &               &               \\
\textbf{timing}    &                       &               &               &               &               &               &               &               &               &               &               &               &               &               &               &               &               &               &               &               &               &               &               &               \\
100:1              & 32                    & 40            & 56            & 32            & 40            & 52            & 32            & 44            & 65            &               &               &               &               &               &               &               &               &               &               &               &               &               &               &               \\
10:1               & 32                    & 40            & 60            & 32            & 36            & 52            & 32            & 48            & 60            &               &               &               &               &               &               &               &               &               &               &               &               &               &               &               \\
2:1                & 32                    & 40            & 53            & 32            & 40            & 56            & 32            & 44            & 56            &               &               &               &               &               &               &               &               &               &               &               &               &               &               &               \\
1:1                & 32                    & 48            & 64            & 32            & 48            & 64            & 32            & 44            & 64            &               &               &               &               &               &               &               &               &               &               &               &               &               &               &               \\
1:2                & 32                    & 42            & 60            & 32            & 40            & 56            & 36            & 48            & 68            &               &               &               &               &               &               &               &               &               &               &               &               &               &               &               \\
1:10               & 32                    & 40            & 56            & 32            & 40            & 56            & 32            & 48            & 57            &               &               &               &               &               &               &               &               &               &               &               &               &               &               &               \\
1:100              & 32                    & 40            & 56            & 32            & 40            & 56            & 32            & 48            & 64            &               &               &               &               &               &               &               &               &               &               &               &               &               &               &               \\
\textbf{selection} & \textbf{steady-state} &               &               &               &               &               &               &               &               &               &               &               &               &               &               &               &               &               &               &               &               &               &               &               \\
\textbf{approach}  & \textbf{ECGA}         &               &               &               &               &               & \textbf{GA}   &               &               &               &               &               &               &               &               &               &               &               &               &               &               &               &               &               \\
\textbf{cx}        & \textbf{LL-MPM}            &               &               &               &               &               & \textbf{UX}   &               &               &               &               &               & \textbf{TPX}  &               &               &               &               &               & \textbf{SFX}  &               &               &               &               &               \\
\textbf{(a)sync}   & \textbf{s}            &               &               & \textbf{a}    &               &               & \textbf{s}    &               &               & \textbf{a}    &               &               & \textbf{s}    &               &               & \textbf{a}    &               &               & \textbf{s}    &               &               & \textbf{a}    &               &               \\
\textbf{quantile}  & \textbf{0.25}         & \textbf{0.50} & \textbf{0.75} & \textbf{0.25} & \textbf{0.50} & \textbf{0.75} & \textbf{0.25} & \textbf{0.50} & \textbf{0.75} & \textbf{0.25} & \textbf{0.50} & \textbf{0.75} & \textbf{0.25} & \textbf{0.50} & \textbf{0.75} & \textbf{0.25} & \textbf{0.50} & \textbf{0.75} & \textbf{0.25} & \textbf{0.50} & \textbf{0.75} & \textbf{0.25} & \textbf{0.50} & \textbf{0.75} \\
\textbf{timing}    &                       &               &               &               &               &               &               &               &               &               &               &               &               &               &               &               &               &               &               &               &               &               &               &               \\
100:1              & 1661                  & 4004          & 6563          & 236           & 454           & 732           & 4096          & 16260         & 31800         & 1020          & 7034          & 19880         & 124           & 206           & 337           & 72            & 120           & 164           & 3252          & 11516         & 18992         & 183           & 968           & 4305          \\
10:1               & 1661                  & 4004          & 6563          & 249           & 506           & 897           & 4096          & 16260         & 31800         & 499           & 8156          & 21943         & 124           & 206           & 337           & 64            & 114           & 161           & 3252          & 11516         & 18992         & 406           & 3602          & 7900          \\
2:1                & 1661                  & 4004          & 6563          & 576           & 1426          & 3004          & 4096          & 16260         & 31800         & 1280          & 16380         & 49152         & 124           & 206           & 337           & 88            & 124           & 190           & 3252          & 11516         & 18992         & 1697          & 7160          & 14676         \\
1:1                & 1590                  & 2992          & 6667          & 1590          & 2992          & 6667          & 4096          & 17098         & 30462         &               &               &               & 115           & 178           & 260           & 115           & 178           & 260           & 4095          & 9540          & 17835         & 8184          & 23488         & 46400         \\
1:2                & 1747                  & 3760          & 6286          & 892           & 1784          & 2861          & 4096          & 15648         & 27424         &               &               &               & 123           & 182           & 264           & 112           & 194           & 277           & 4094          & 10848         & 17519         & 16270         & 49152         &               \\
1:10               & 1747                  & 3760          & 6286          & 1628          & 2924          & 4937          & 4096          & 15648         & 27424         &               &               &               & 123           & 182           & 264           & 159           & 228           & 318           & 4094          & 10848         & 17519         & 24544         & 57344         &               \\
1:100              & 1747                  & 3760          & 6286          & 1951          & 3722          & 6256          & 4096          & 15648         & 27424         &               &               &               & 123           & 182           & 264           & 128           & 218           & 305           & 4094          & 10848         & 17519         & 31392         &               &               \\
\textbf{selection} & \textbf{generational} &               &               &               &               &               &               &               &               &               &               &               &               &               &               &               &               &               &               &               &               &               &               &               \\
\textbf{approach}  & \textbf{ECGA}         &               &               &               &               &               & \textbf{GA}   &               &               &               &               &               &               &               &               &               &               &               &               &               &               &               &               &               \\
\textbf{cx}        & \textbf{LL-MPM}            &               &               &               &               &               & \textbf{UX}   &               &               &               &               &               & \textbf{TPX}  &               &               &               &               &               & \textbf{SFX}  &               &               &               &               &               \\
\textbf{(a)sync}   & \textbf{s}            &               &               & \textbf{a}    &               &               & \textbf{s}    &               &               & \textbf{a}    &               &               & \textbf{s}    &               &               & \textbf{a}    &               &               & \textbf{s}    &               &               & \textbf{a}    &               &               \\
\textbf{quantile}  & \textbf{0.25}         & \textbf{0.50} & \textbf{0.75} & \textbf{0.25} & \textbf{0.50} & \textbf{0.75} & \textbf{0.25} & \textbf{0.50} & \textbf{0.75} & \textbf{0.25} & \textbf{0.50} & \textbf{0.75} & \textbf{0.25} & \textbf{0.50} & \textbf{0.75} & \textbf{0.25} & \textbf{0.50} & \textbf{0.75} & \textbf{0.25} & \textbf{0.50} & \textbf{0.75} & \textbf{0.25} & \textbf{0.50} & \textbf{0.75} \\
\textbf{timing} &                       &               &               &               &               &               &               &               &               &               &               &               &               &               &               &               &               &               &               &               &               &               &               &               \\
100:1              & 1536                  & 3704          & 6280          & 202           & 382           & 576           & 1904          & 3682          & 6182          & 512           & 1968          & 4051          & 243           & 408           & 603           & 182           & 268           & 393           & 1531          & 2990          & 5997          & 743           & 1578          & 2689          \\
10:1               & 1536                  & 3704          & 6280          & 256           & 512           & 902           & 1904          & 3682          & 6182          & 981           & 1982          & 3958          & 243           & 408           & 603           & 137           & 254           & 432           & 1531          & 2990          & 5997          & 955           & 1752          & 3695          \\
2:1                & 1536                  & 3704          & 6280          & 635           & 1214          & 2288          & 1904          & 3682          & 6182          & 1024          & 3468          & 6749          & 243           & 408           & 603           & 236           & 396           & 573           & 1531          & 2990          & 5997          & 1745          & 3634          & 5827          \\
1:1                & 1705                  & 4012          & 6232          & 1705          & 4012          & 6232          & 2615          & 3962          & 7007          & 2909          & 4076          & 7721          & 252           & 386           & 600           & 252           & 386           & 606           & 1024          & 3058          & 5944          & 1024          & 3442          & 6684          \\
1:2                & 1390                  & 2848          & 6904          & 933           & 1756          & 2576          & 1837          & 3630          & 7217          & 2015          & 4066          & 7544          & 254           & 444           & 689           & 252           & 444           & 614           & 1791          & 3310          & 6813          & 1498          & 2872          & 5738          \\
1:10               & 1390                  & 2848          & 6904          & 1486          & 2802          & 5138          & 1837          & 3630          & 7217          & 3037          & 6260          & 11703         & 254           & 444           & 689           & 332           & 516           & 911           & 1791          & 3310          & 6813          & 2048          & 5046          & 8492          \\
1:100              & 1390                  & 2848          & 6904          & 1764          & 3654          & 6338          & 1837          & 3630          & 7217          & 3748          & 8148          & 15482         & 254           & 444           & 689           & 256           & 512           & 860           & 1791          & 3310          & 6813          & 1887          & 4096          & 8480          
\end{tblr}
\vspace*{1mm}
\captionof{table}{Quantiles of minimally required population sizes for ANKL}
\label{tab:table-ankl-ext}
\end{landscape}
\restoregeometry

\begin{table}
    \caption{Quantiles of minimally required time and corresponding population sizes for NASBench 301}
    \label{tab:table-nasbench-ext}
    \begin{tblr}{
  cell{1}{5} = {c=3}{},
  cell{1}{8} = {c=3}{},
  cell{4}{1} = {r=3}{},
  cell{4}{2} = {r=3}{},
  cell{4}{3} = {r=3}{},
  cell{4}{5} = {r},
  cell{4}{6} = {r},
  cell{4}{7} = {r},
  cell{4}{8} = {r},
  cell{4}{9} = {r},
  cell{4}{10} = {r},
  cell{5}{5} = {r},
  cell{5}{6} = {r},
  cell{5}{7} = {r},
  cell{5}{8} = {r},
  cell{5}{9} = {r},
  cell{5}{10} = {r},
  cell{6}{5} = {r},
  cell{6}{6} = {r},
  cell{6}{7} = {r},
  cell{6}{8} = {r},
  cell{6}{9} = {r},
  cell{6}{10} = {r},
  cell{7}{1} = {r=6}{},
  cell{7}{2} = {r=2}{},
  cell{7}{3} = {r=2}{},
  cell{9}{2} = {r=4}{},
  cell{9}{3} = {r=2}{},
  cell{9}{5} = {r},
  cell{9}{6} = {r},
  cell{9}{7} = {r},
  cell{9}{8} = {r},
  cell{9}{9} = {r},
  cell{9}{10} = {r},
  cell{10}{5} = {r},
  cell{10}{6} = {r},
  cell{10}{7} = {r},
  cell{10}{8} = {r},
  cell{10}{9} = {r},
  cell{10}{10} = {r},
  cell{11}{3} = {r=2}{},
  cell{11}{5} = {r},
  cell{11}{6} = {r},
  cell{11}{7} = {r},
  cell{11}{8} = {r},
  cell{11}{9} = {r},
  cell{11}{10} = {r},
  cell{12}{5} = {r},
  cell{12}{6} = {r},
  cell{12}{7} = {r},
  cell{12}{8} = {r},
  cell{12}{9} = {r},
  cell{12}{10} = {r},
  cell{13}{1} = {r=6}{},
  cell{13}{2} = {r=2}{},
  cell{13}{3} = {r=2}{},
  cell{13}{5} = {r},
  cell{13}{6} = {r},
  cell{13}{7} = {r},
  cell{13}{8} = {r},
  cell{13}{9} = {r},
  cell{13}{10} = {r},
  cell{15}{2} = {r=4}{},
  cell{15}{3} = {r=2}{},
  cell{15}{5} = {r},
  cell{15}{6} = {r},
  cell{15}{7} = {r},
  cell{15}{8} = {r},
  cell{15}{9} = {r},
  cell{15}{10} = {r},
  cell{16}{5} = {r},
  cell{16}{6} = {r},
  cell{16}{7} = {r},
  cell{16}{8} = {r},
  cell{16}{9} = {r},
  cell{16}{10} = {r},
  cell{17}{3} = {r=2}{},
  cell{17}{5} = {r},
  cell{17}{6} = {r},
  cell{17}{7} = {r},
  cell{17}{8} = {r},
  cell{17}{9} = {r},
  cell{17}{10} = {r},
  cell{18}{5} = {r},
  cell{18}{6} = {r},
  cell{18}{7} = {r},
  cell{18}{8} = {r},
  cell{18}{9} = {r},
  cell{18}{10} = {r},
  % vline{5-6} = {1}{},
  vline{5,8} = {1-18}{},
  hline{1,4,19} = {-}{},
}
                      &                   &                 & ~                & \textbf{simulation time (s) × $\mathbf{10^6}$} &               &               & \textbf{population\_size} &               &               \\
                      &                   &                 & ~                & \textbf{0.25}                        & \textbf{0.50} & \textbf{0.75} & \textbf{0.25}             & \textbf{0.50} & \textbf{0.75} \\
\textbf{selection}    & \textbf{approach} & \textbf{cx}     & \textbf{(a)sync} &                                      &               &               &                           &               &               \\
\textbf{GOM}          & \textbf{GOMEA}    & \textbf{LL-LT}  & \textbf{a/e}     & 1.591601                             & 1.677091      & 1.786227      & 38.75                     & 49.5          & 64            \\
                      &                   &                 & \textbf{a/i}     & 1.215592                             & 1.361821      & 1.64783       & 46                        & 63.5          & 79.5          \\
                      &                   &                 & \textbf{s}       & 1.462275                             & 1.652285      & 1.788804      & 48.5                      & 58.5          & 62.5          \\
\textbf{steady-state} & \textbf{ECGA}     & \textbf{LL-MPM} & \textbf{a}       &                                      &               &               &                           &               &               \\
                      &                   &                 & \textbf{s}       &                                      &               &               &                           &               &               \\
                      & \textbf{GA}       & \textbf{UX}     & \textbf{a}       & 0.75999                              & 0.982413      & 1.340313      & 208.5                     & 252.5         & 372.75        \\
                      &                   &                 & \textbf{s}       & 1.125931                             & 1.400998      & 1.790128      & 256                       & 368.5         & 512           \\
                      &                   & \textbf{TPX}    & \textbf{a}       & 3.672774                             & 4.258886      & 7.41098       & 960.25                    & 1024          & 2048          \\
                      &                   &                 & \textbf{s}       & 3.761896                             & 4.492289      & 8.453775      & 767.75                    & 1110          & 2048          \\
\textbf{generational} & \textbf{ECGA}     & \textbf{LL-MPM} & \textbf{a}       & 2.515486                             & 7.734439      & 12.016        & 1024                      & 3072          & 4096          \\
                      &                   &                 & \textbf{s}       &                                      &               &               &                           &               &               \\
                      & \textbf{GA}       & \textbf{UX}     & \textbf{a}       & 1.20572                              & 1.517906      & 1.959241      & 514.25                    & 816           & 1024          \\
                      &                   &                 & \textbf{s}       & 0.746899                             & 0.980716      & 2.065738      & 274.25                    & 480.5         & 1052.5        \\
                      &                   & \textbf{TPX}    & \textbf{a}       & 2.583693                             & 4.434104      & 6.377228      & 1024                      & 2048          & 2048          \\
                      &                   &                 & \textbf{s}       & 2.692329                             & 4.774914      & 9.726         & 1024                      & 2048          & 4096          
\end{tblr}
% \captionof{table}{Quantiles of minimally required time and corresponding population sizes for NASBench 301}
\end{table}

\end{document}