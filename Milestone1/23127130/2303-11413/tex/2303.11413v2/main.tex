\documentclass{article}
% \usepackage[nonatbib, preprint]{neurips_2021} and removing \usepackage{biblatex}
\usepackage[nonatbib, final]{nips_2017}
\usepackage[utf8]{inputenc} % allow utf-8 input
\usepackage[T1]{fontenc}    % use 8-bit T1 fonts
\usepackage{hyperref}       % hyperlinks
\usepackage{url}            % simple URL typesetting
\usepackage{booktabs}       % professional-quality tables
\usepackage{amsfonts}       % blackboard math symbols
\usepackage{nicefrac}       % compact symbols for 1/2, etc.
\usepackage{microtype}      % microtypography
\usepackage{graphicx}
\usepackage{amsmath, bm}

\title{Structural Vibration Signal Denoising \\ Using KLD Regularized Bi-Directional LSTM} % Exploration of Vibration Signal Denoising Methods

\author{%
  Youzhi Liang \\
  Department of Computer Science \\
  Stanford University \\
  Stanford, CA 94305 \\
  \texttt{youzhil@stanford.edu} \\
  \And
  Wen Liang \\
  Department of Electrical and Computer Engineering \\
  University of California San Diego \\
  La Jolla, CA 92093 \\
  \texttt{wel245@eng.ucsd.edu} \\
}


\begin{document}
% \nipsfinalcopy is no longer used

\maketitle

\begin{abstract}
Vibration signals have been increasingly utilized in various engineering fields for analysis and monitoring purposes, including structural health monitoring, fault diagnosis and damage detection, where vibration signals can provide valuable information about the condition and integrity of structures. In recent years, there has been a growing trend towards the use of vibration signals in the field of bioengineering. Activity-induced structural vibrations, particularly footstep-induced signals, are useful for analyzing the movement of biological systems such as the human body and animals. Footstep-induced signals can provide valuable information about an individual's gait, body mass, and posture, making them an attractive tool for health monitoring, security, and human-computer interaction. However, the presence of various types of noise can compromise the accuracy of footstep-induced signal analysis. In this paper, we propose a novel 'many-to-many' LSTM model with a KLD regularizer and L1 regularization, which is effective in denoising structural vibration signals, particularly for regimes with larger amplitudes. The model was trained and tested using synthetic data generated by a single degree of freedom oscillator. Our results demonstrate that the proposed approach is effective in reducing noise in the signals, particularly for regimes with larger amplitudes. The approach is promising for a wide range of applications of footstep-induced structural vibration signals, including healthcare, security, and technology.
\end{abstract}

\section{Introduction}	

The application of structural vibration signal is prevalent in various engineering domains, including civil, mechanical, and bioengineering. Its utility in civil engineering lies in monitoring the dynamic behavior of structures such as bridges and buildings subjected to loads like earthquakes and winds~\cite{speckmann2004structural}. Structural Health Monitoring (SHM) systems have been installed on such structures to ensure their safety and reliability, as aging structures may become a safety hazard~\cite{cross2013long, kaya2015real}. The SHM systems acquire structural responses to detect any abnormal conditions, allowing for timely maintenance and improved decision-making. Therefore, the accurate extraction of the underlying structural vibration signal from civil architectures is essential to provide a robust foundation for the reliability of SHM systems. In mechanical engineering, the use of structural vibration signal is vital in monitoring the health of machines, such as turbines, pumps, and engines~\cite{malekloo2022machine, fish2019dynamic, hoseinzadeh2018quantitative}. The effective monitoring of equipment condition and fault diagnosis is crucial to ensure safe operation. The accurate assessment of equipment condition necessitates high-quality data acquisition that contains significant structural vibration information and low noise levels~\cite{lei2020applications, wang2022attention}.

The use of vibration signals in bioengineering has seen a recent surge in popularity. Structural vibrations caused by activity provide a powerful tool for analyzing the movement of biological systems such as animals and humans. The study of footstep-induced signals has emerged as a crucial area of research, given that these signals contain unique information on an individual's gait, body mass, and posture, which could used as a biometric feature to identify individuals~\cite{pan2015indoor, shoji2004personal}. Footstep-induced signals have broad-ranging applications in fields such as health monitoring, security, and human-computer interaction~\cite{ekimov2006vibration, li2019smart}. The analysis of these signals can provide useful insights into an individual's physical condition, such as early indicators of movement disorders or gait patterns indicating injuries or illnesses~\cite{fagert2019gait}. Additionally, these signals can be used to develop secure and non-intrusive biometric authentication systems, which are essential in modern security applications. Finally, the study of footstep-induced signals can also revolutionize the field of human-computer interaction, as it enables the development of new natural and intuitive interfaces between humans and machines~\cite{drira2021using}. Overall, the benefits of studying footstep-induced signals are enormous, as they have potential applications in a wide range of fields, including healthcare, security, and technology~\cite{fagert2020structural}. Denoising of these vibration signals is crucial for accurate analysis and interpretation of the underlying physical phenomena. Vibration signals are often corrupted by noise from various sources, such as electrical interference, environmental factors, and measurement errors. To address this issue, various signal processing techniques have been developed, such as Fourier transform, wavelet transform, and empirical mode decomposition~\cite{kumar2021stationary, kaur2021eeg}. However, these methods have limitations in handling complex and non-stationary signals with high noise levels. As such, there is a pressing need for denoising methods that are both robust and efficient in capturing the underlying structural mechanics that govern dynamic signals.

In this paper, we propose a deep learning-based approach for vibration signal denoising using bi-directional long short-term memory (LSTM) networks with KL divergence regularization and $L_1$ regularization. The bi-directional LSTM architecture enables the model to capture both forward and backward temporal dependencies in the signal, which is essential for modeling complex dynamics. The KL divergence regularization and $L_1$ regularization are used to prevent overfitting and improve the generalization ability of the model, leveraging the prior distribution of the signal. We evaluate our proposed method on a benchmark dataset and show that it outperforms prevailing denoising methods in terms of $R^2$-score and mean squared error (MSE) using a synthetic dataset.

\section{Dataset}

\begin{figure}[ht]
  \includegraphics[width=120mm]{Figs/sample_signal.JPG}
  \centering
\caption{(a) A sample of foot-step induced floor vibration signal, normalized magnitude of vibration as a function of time~\cite{pan2017footprintid, hu2021footstep}. (b) An example of the signals generated based on Eqn.~\ref{eqn:ODE}, displacement overlaid with a high level of supplemental Gaussian noise, $w(t)$ as a function of time, $t$ [s].}
  \label{fig:sample_signal}
\end{figure}

The ground truth of a structural vibration signal can be challenging to obtain in practice due to various factors, such as the presence of noise, measurement errors, and limitations in the digitalization process~\cite{wang2022attention}. It is often impossible to measure or observe the true vibration signal directly, and hence, to generate a dataset of structural vibration time series, we opted to use a PDE/ODE solver (\href{https://docs.scipy.org/doc/scipy/reference/integrate.html#module-scipy.integrate}{scipy.integrate}) instead of conducting expensive experiments or computational-intensive simulations using Abacus. Our choice was motivated by the fact that simulations and experiments can be prohibitively expensive and time-consuming, whereas the use of a solver allowed us to quickly and efficiently generate a large dataset of 100,000 synthetic time series ($\mathcal{R}^{100,000 \times 500})$. These synthetic time series were generated by adding supplementary noise to the output of the PDE/ODE solver.

Our study focuses on the dynamics of structural vibrations induced by footsteps. Footsteps generate a transient load that causes the structure to vibrate. This vibration can be modeled as the impulse response of a Kirchhoff-Love plate, which is a widely used model in structural dynamics~\cite{reddy2020integrated, asakura2014finite}. The Kirchhoff-Love plate theory assumes that the plate is thin compared to its length and width, and that it is subject to small deformations. Additionally, we assume the linearity of isotropic plates, meaning that the material properties of the plate are uniform in all directions. Under these assumptions, the induced vibrations can be modeled as the solutions of a system of PDEs. Nguyen {\em et.~al}~\cite{nguyen2021stable} provide a detailed description of the system of PDEs that we use to model the induced vibrations, where the system consists of two coupled partial differential equations that describe the displacement and rotation of the plate. The boundary conditions are given by the Dirichlet condition, which specifies the displacement of the plate along its boundaries. By solving this system of PDEs, we are able to generate a realistic simulation of the vibrations induced by footsteps on a Kirchhoff-Love plate.

The dynamics of structural vibrations induced by footsteps can be simplified as the impulse response of a Kirchhoff-Love plate subject to Dirichlet boundary conditions. Under the assumption of linearity for isotropic plates, we model the induced vibrations as the solutions of a system of PDEs~\cite{nguyen2021stable}:

\begin{equation}
    \label{eqn:ODE}
    D_i \nabla ^2 \nabla^2 w_i(\mathbf{x}, t) - T_i \nabla ^2 w_i(\mathbf{x}, t) = \delta(\mathbf{x}, t) - \rho_i h_i \ddot{w}_i(\mathbf{x}, t) - K_i \dot{w}_i(\mathbf{x}, t),
\end{equation}

where $w_i(\mathbf{x}, t)$ is the transverse deflection, $D_i \sim \mathcal{N}(\mu_D, \sigma_D)$, $T_i \sim \mathcal{N}(\mu_T, \sigma_T)$, $\rho_i h_i \sim \mathcal{N}(\mu_{\rho h}, \sigma_{\rho h})$ and $K_i \sim \mathcal{U}(a_U, b_U)$ are all parameters of the structure, governing the dynamical response subject to impulse $\delta(\mathbf{x}, t)$. Each synthetic time series of noisy signal, $\tilde{\bm{w}}$ is obtained by $\sum w_j(\mathbf{x}, t) + \epsilon_j(t)$, where $\epsilon_jS(t)$ represents the supplementary noise, simulating the sensor noise aforementioned. We conducted additional comparisons between our synthetic data and measurements reported in prior studies~\cite{pan2017footprintid, hu2021footstep}. Figure~\ref{fig:sample_signal} (a) illustrates the floor vibration signal induced by three footsteps, which our synthetic results closely align with.

\section{Methods and Results}
\begin{figure}[ht]
  \includegraphics[width=115mm]{Figs/cs230_LSTM_Architecture.JPG}
  \centering
  \caption{(a) Architecture for bi-directional LSTM with additional frequency-domain input. (b) Table for metrics used to evaluate the unfiltered signal and filtered signal using two LSTM architectures. Bi-LSTM Fourier denotes the bi-directional LSTM with additional frequency-domain input using FFT.}
  \label{fig:RNN_Architecture1}
\end{figure}

\begin{figure}[ht]
  \includegraphics[width=120mm]{Figs/cs230_RNN_result.JPG}
  \centering
  \caption{(a) Training loss versus epoch. (Inset) At epoch = 100, an example in test set illustrating the unfiltered, filtered, and actual signal (b) Signal with noise versus actual signal. (Inset) Histogram of the error distribution. Bi-LSTM Fourier denotes the bi-directional LSTM with additional frequency-domain input using FFT.}
  \label{fig:RNN_result1}
\end{figure}

The implementation of a 'many-to-many' bi-directional LSTM architecture with a dropout rate of 0.2 was made to address the challenges associated with structural vibration signals, which are typically complex, non-stationary, and contaminated with noise (see Fig.~\ref{fig:RNN_Architecture1}). By using a bi-directional architecture, the model is able to capture both past and future temporal dependencies, which can improve the accuracy of predictions. Additionally, the use of dropout regularization helps to prevent overfitting and improve the generalization performance of the model.

To further enhance the accuracy of predictions, we utilized feature engineering techniques by extracting features in the frequency domain using the Fast Fourier Transform. These features capture the underlying physics-governed dynamics of the system, which can help to improve the accuracy of predictions~\cite{lin2023fft, li2023research}. By incorporating these features as additional inputs into the bi-directional LSTM neural network, we were able to improve the accuracy of predictions even further. The output of the bi-directioanl LSTM, the denoised signal $\hat{\bm{y}}^{[l]}$ can be expressed as

$$\hat{\bm{y}}^{[l]} =  \mathcal{F}_{\textit{LSTM}} \left( \textit{FFT}(\tilde{\bm{w}}), \tilde{\bm{w}}; \mathbf{\Theta} \right),$$

where $\mathcal{F}_{\textit{LSTM}} \left( \cdot; \mathbf{\Theta} \right) $ denotes the function of the bi-directional LSTM followed by a fully connected neural net layer, with trainable parameters $\mathbf{\Theta}$. The input to the first layer is the noisy signal, i.e. $\tilde{\bm{w}}$. To enhance the generalizability of our model, we introduce a regularized cost function that incorporates Kullback–Leibler (KL) divergence regularization through the use of a prior distribution~\cite{kingma2015variational}. The regularized cost function is formulated as the expected loss over the training set using the $L^2$-norm, along with the expected loss over the model parameters using KL divergence, expressed as follows:

$$ \mathcal{L} \left( \mathbf{\Theta} \right) = \mathbb{E}_{\bm{I}} \left\| \hat{\bm{y}} - {\bm{I}} \right\|^2 + \lambda_1 \mathbb{E} \left\| \bm{\Theta} \right\| + \lambda_{\text{KL}} \mathbb{E}_{\bm{y}} D_{\text{KL}} \left( \hat{\bm{y}} \parallel \bm{I} \right) $$

where $\mathbf{\Theta}$ denotes all the trainable parameters, $\lambda_{\text{KL}} D_{\text{KL}} \left( \bm{y}_{\mathcal{D}}^{[L]} \parallel \bm{I} \right)$ denotes the KL divergence of $\bm{y}_{\mathcal{D}}^{[L]}$ from prior distribution $\bm{I}$. The inclusion of KL divergence regularization in our model aims to prevent overfitting to a single training instance, analogous to adapting the target distribution performed by conventional backpropagation algorithms~\cite{yu2013kl, phan2020personalized}. The KLD regularizer can be especially beneficial when the structural signal is contaminated with a different type of noise, such as electrical noise, rather than the activity-induced noise. The $L_1$ regularizer functions similarly to frequency selection by reducing the number of features.

To evaluate the performance of the proposed LSTM model for denoising structural vibration signals, the reduction in loss was monitored during 100 epochs of training. The results, illustrated in Fig.\ref{fig:RNN_result1} (a), showed a decrease in loss from 0.37 to 0.16. Additionally, an inset in the figure provided an example of a filtered signal that closely matched the actual signal, suggesting the effectiveness of the proposed model, particularly for larger amplitudes of vibration. To further assess the accuracy of the denoising process, a discrete prediction of the filtered signal was plotted for larger positive amplitudes (around 5 to 7.5) in Fig.\ref{fig:RNN_result1} (b). However, the plot also revealed mild heteroskedasticity, which is caused by a larger observed error for small amplitudes (around -2.5 to 2.5). Overall, the results demonstrated that the proposed LSTM model, with a 0.2 dropout rate and a 'many-to-many' architecture, is a promising approach for denoising structural vibration signals, particularly for regimes with larger amplitudes. The loss vs. epoch and error distributions in Fig.~\ref{fig:RNN_result1} showed over 50\% error reduction using the new architecture, consistent with the quantitative metrics, such as RMSE, MAE, and $R^2$-score listed in Table 1.

\section{Conclusion}
In this paper, we presented a 'many-to-many' bi-directional LSTM model with a KLD regularizer and L1 regularization for denoising footstep-induced structural vibration signals. We evaluated the model using synthetic data generated by a single degree of freedom oscillator, and results demonstrate that the proposed model outperforms existing state-of-the-art denoising techniques. The model is particularly effective in denoising regimes with larger amplitudes, making it a valuable tool for various applications such as health monitoring, security, and human-computer interaction. The proposed approach has the potential to improve the accuracy of footstep-induced signal analysis, enabling new insights into an individual's physical condition and providing a non-intrusive biometric authentication method. Overall, the results suggest that the proposed LSTM model with KLD and L1 regularization is a promising approach for denoising structural vibration signals and has potential applications in various fields, including healthcare, security, and technology.

\bibliographystyle{unsrt}
\bibliography{reference}

\end{document}