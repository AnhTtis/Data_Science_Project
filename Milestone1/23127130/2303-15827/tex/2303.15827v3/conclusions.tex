In this work we introduce a new hybrid modelling approach, combining mechanistic knowledge with data. The knowledge we assume is in the form of a PDE family, without specific coefficient values, typically supplied by field experts. 
% The dataset we rely on is readily available in physical modelling problems, as it is simply a collection of spatio-temporal signals belonging to the same PDE family, with different coefficients. 
\balance
The problem we introduce in this work is unique because the signals at inference time correspond to PDEs with coefficients that differ from those in the training data. 
This makes the prediction problem similar to a zero-shot prediction challenge, as the model needs to generalize to unseen dynamics.
% Unlike other schemes, CONFIDE does not require knowledge of the coefficients of the PDE generating our train data, but learns how to estimate them.
We conduct extensive experiments on four well-known PDE systems, comparing our scheme to other solutions and testing its performance in different regimes. 
CONFIDE outperforms all baselines, provides reliable PDE coefficient estimations, robust to different values of hyper-parameters, and scores well even when the test signals come from out-of-distribution signals.
We further stress-test CONFIDE by removing most of the mechanistic knowledge it receives (namely, CONFIDE-0, for zero knowledge), and show that it is still able to outperform other baselines.
There are many promising future directions, such as scaling CONFIDE to real-world problems like the ones mentioned in Section~\ref{sec:related}.

% Future directions we would like to pursue include a straightforward extension to handle signals with missing datapoints, handling ``out of distribution'' signals, generated by parameters beyond the support of the dataset, and examining the robustness of predicting such signals. Another question that comes to mind is whether including multiple signals generated by the same parameters has an effect on quality of results, similar to or different from that of the context ratio. Finally, we are eager to apply CONFIDE to a real world problem like the ones mentioned in Section~\ref{sec:related}.