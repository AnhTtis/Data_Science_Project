\begin{abstract}
% We propose an explainable method for solving Partial Differential Equations by using a contextual scheme called \emph{PDExplain}. 
% During the training phase, our method is fed with data collected from a family of PDEs accompanied by the general form of this family. In the inference phase, a minimal sample collected from a phenomenon is provided, where the sample originates from the PDE family but not necessarily from the set of PDEs seen in the training phase. 
% Our algorithm can predict the PDE solution for future timesteps, while providing an explainable form of the PDE, a trait that can assist in data-driven modelling of phenomena in physical sciences. We include results of extensive experimentation, examining our method's quality both in terms of prediction error and explainability.
% We show how our algorithm can predict the PDE solution for future timesteps. Moreover, our method provides an explainable form of the PDE, a trait that can assist in modelling phenomena based on data in physical sciences. To verify our method, we conduct extensive experimentation, examining its quality both in terms of prediction error and explainability.
We introduce a method for inferring an explicit PDE from a data sample generated by  previously unseen dynamics, based on a learned context. 
The training phase integrates knowledge of the form of the equation with a differential scheme, while the inference phase yields a PDE that fits the data sample and enables both signal prediction and data explanation.
We include results of extensive experimentation, comparing our method to SOTA approaches, together with ablation studies that examine different flavors of our solution.
\end{abstract}
