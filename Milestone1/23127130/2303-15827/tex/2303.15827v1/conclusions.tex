In this work we introduce a new hybrid modelling approach, combining mechanistic knowledge with data. The knowledge we assume is in the form of a PDE family, without specific parameter values, typically supplied by field experts. 
The dataset we rely on is readily available in physical modelling problems, as it is simply a collection of spatio-temporal signals belonging to the same PDE family, with different parameters. Unlike other schemes, we do not require knowledge of the parameters of the PDE generating our train data.

We conduct extensive experiments, comparing our scheme to other solutions and testing its performance in different regimes. It achieves good results in the zero-shot learning problem, and is robust to different values of hyper-parameters.

Future directions we would like to pursue include adding support in our code for signals of higher spatial dimensions, together with a straightforward extension to handle signals with missing datapoints.

An interesting experiment we would like to conduct concerns handling ``out of distribution'' signals - generated by parameters beyond the support of the dataset, and the robustness of predicting such signals. Another question that comes to mind is whether including multiple signals generated by the same parameters has an effect on quality of results, similar to or different from that of the context ratio. Finally, we are eager to apply PDExplain to a real life problem like the ones mentioned in Section~\ref{sec:related}.