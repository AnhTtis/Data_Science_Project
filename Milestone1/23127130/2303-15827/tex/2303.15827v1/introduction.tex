Many scientific fields use the language of Partial Differential Equations (PDEs; \citealp{evans2010partial}) to describe the physical laws governing observed natural phenomena with spatio-temporal dynamics. 
Typically, a PDE system is derived from first principles and a mechanistic understanding of the problem after experimentation and data collection by domain experts of the field.
Well-known examples for such systems include Navier-Stokes and Burgers' equations in fluid dynamics, Maxwell's equations for electromagnetic theory, and Schr\"{o}dinger's equations for quantum mechanics.
Solving a PDE model could provide users with crucial information on how a signal evolves over time and space, and could be used for both prediction and control tasks.

While solving PDEs holds great value, it might still be a difficult task in many cases (we refer the reader to \citealp{zwillinger1998handbook} for an extensive handbook for analytical methods).
For many complex real-world phenomena, we might only know some of the dynamics of the system.
For example, an expert might tell us that a  heat equation PDE has a specific functional form
%such as
%\begin{align}
%\label{eq:pde_coeffs}
%        \fder{f}{t}=a(x,t,f) \frac{\partial^2 f }{\partial x^2}  + b(x,t,f)\frac{\partial f }{\partial x} + c(x,t,f),
%\end{align}
but we do not know the values of the diffusion and drift coefficient functions. In this paper we focus mainly on this case.
%In this case, the PDE system at hand is not solvable by standard methods.

There are different ways of solving PDEs when data is available. In Figure~\ref{fig:experiments.mech_dd} we illustrate the spectrum of approaches to the problem of PDEs modeling and their solutions. The horizontal axis represents the amount of mechanistic knowledge required by each approach, i.e., how much prior knowledge we have on the source of the data in terms of the PDE structure. The approaches hovering above the axis are those that employ available data, while those below the axis only use mechanistic knowledge. We describe the different approaches in detail in Section~\ref{sec:experiments}.
\begin{figure}[ht]
    \centering
    \includegraphics[width=1.0\columnwidth]{Figures/experiments/mech_dd.png}
    \label{fig:experiments.mech_dd}
    \caption{Mechanistic and data driven approaches to PDE modeling. The horizontal axis represents mechanistic knowledge and the vertical location of the approaches corresponds to the ability to utilize training data.}
\end{figure}

The current process of solving PDEs over space and time is by using numerical differentiation and integration schemes. However, numerical methods may require significant computational resources, making the PDE solving task feasible only for low-complexity problems,e.g., a small number of equations.
An alternative wide-used approach is finding simplified models that are based on certain assumptions and can roughly describe the problem's dynamics. A known example for such a model are the Reynolds-averaged Navier-Stokes equations \cite{reynolds1995dynamical}.
Building simplified models is considered a highly non-trivial task that requires special expertise, and might still not represent the phenomenon to a satisfactory accuracy.

In recent years, with the rise of Deep Learning (DL; \citealp{lecun2015deep}), novel methods for solving numerically-challenging PDEs were devised.
These methods have become especially useful thanks to the rapid development of sensors and computational power, enabling the collection of large amounts of multidimensional data related to a specific phenomenon.
In general, DL based approaches consume the observed data and learn a black-box model of the given problem that can then be used to provide predictions for the dynamics.
While this set of solutions has been shown to perform successfully on many tasks, it still suffers from two crucial drawbacks: (1) It offers no explainability as to why the predictions were made, and (2) it usually performs very poorly when extrapolating to unseen data. 

In this paper, we offer a new hybrid modelling \cite{kurz2022hybrid} approach that can benefit from both worlds: it can use the vast amount of data collected on one hand, and utilize the partially known PDEs describing the natural phenomena observed on the other hand. In addition, it can learn several contexts, therefore, employing the generalization capabilities of DL models, enabling a zero-shot learning \cite{palatucci2009zero}.
Specifically, our model is given a general functional form of the PDE (i.e., which derivatives are used and what the form of the coefficient functions is), consumes the observed data, and outputs the unknown coefficient functions. 
Then, we can then use off-the-shelf PDE solvers (e.g., PyPDE\footnote{\url{https://pypde.readthedocs.io/en/latest/}}) to solve and create predictions of the given task forward in time for any horizon.

Another key feature of our approach is that it consumes the spatio-temporal input signals required for training in an unsupervised manner, namely the coefficient functions that created the signals in the train set are unknown. This is achieved by combining an autoencoder (AE; \citealp{kramer1991nonlinear}) architecture with a loss defined using the functional form of the PDE. As a result of this feature, large amounts of training data for our algorithm can be easily acquired.
Moreover, our ability to generalize to data corresponding to a PDE whose coefficients did not appear in the train set, enables the use of synthetic data for training.
In addition, although our approach is intended to work when the PDE functional form is known, it is not limited to that scenario only.
In cases where we are given a misspecified model (when experts provide a surrogate model for instance), our model can eliminate some of the discrepancies in the extra function that is not a coefficient of one of the derivatives (the $p_0(x,t,f)$ function in Eq.~\eqref{eq:general_PDE}) 

Finally, our approach may also be feasible for  tremendously computationally intensive problems like weather prediction \cite{kang2021examination} or simulating waves \cite{lisitsa2012finite}. In such cases the PDEs are known but most researchers do not have access to High Performance Computing so a hybrid model as we propose might be handy. % We summarize the relevant use-cases of the PDE solving task in Figure~\ref{fig:intro.mapping}.

% \begin{figure}[h]
%     \centering
%     \includegraphics[width=1.0\columnwidth]{Figures/MappingPDEs.JPG}
%     \caption{Mapping the current problems and methods for modeling a spactio-temporal signal when we are given observed data that obeys a PDE.
%     }\label{fig:intro.mapping}
% \end{figure}

We summarize our contribution as follows:
\begin{enumerate}
    \item Harnessing the information contained in large datasets belonging to a phenomenon which is related to a PDE functional family in an unsupervised manner. Specifically, we propose a regression based method for doing that.
    \item Proposing a DL encoding scheme for the context conveyed in such datasets, enabling generalization for prediction of unseen samples based on minimal input, similarly to zero-shot learning.
    \item Extensive experimentation with the proposed scheme, examining the effect of context and train set size, along with a comparison to different previous methods.
\end{enumerate}

The paper is organized as follows. In Section \ref{sec:related} we review related work. In Section \ref{sec:method} we present the proposed method and in Section \ref{sec:experiments} we provide experiments to support our method. Section \ref{sec:conclusions} completes the paper with conclusions and future directions.