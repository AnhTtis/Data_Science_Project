
\subsection{Method overview} 

We introduce the \textbf{Region Conditioned Adapter Tuning} (RCA in \Cref{fig:rpa}), which enhances \textit{the vision foundation models to focus on specific visual cues for deducing inference}. 
The RCA consists of two main parts: a Regional Prompt Generator (RPG in \S \ref{sec:rpg}) for targeting specific visual areas and an Adapter$^\textbf{+}$ module (\S \ref{sec:adapter}) to transfer the frozen CLIP model for reasoning tasks. 
Finally, we replace Multi-Task Learning \cite{hessel2022abduction} with a new \textbf{Dual-Contrastive Loss} (\S \ref{sec:dual_contrast}) to bring the visual features closer to both literal description (``clue'') and hypothesis (``Inference''). 
We will elaborate on each section below.

\subsection{Regional Prompt Generator}
\label{sec:rpg}
In visual abductive reasoning, it is important to collect all relevant visual cues from the image and the region $\boldsymbol{r}$.
Therefore, we use pre-specified observation $\boldsymbol{r}$ as a ``prompt'' to guide the visual reasoning process directly. 
Our Regional Prompt Generator (RPG) creates three detailed prompts focusing on specific regions. 
These prompts harness local features (i.e., region), surrounding context, and existing visual prompts, as shown in Figure \ref{fig:prompt1}-\ref{fig:prompt3}. 
All three types of prompts go through the same process, with the only difference in being whether colors or circles \cite{shtedritski2023does} are drawn on the input images. 
To explain how it works, we'll use the ``Region+Context'' (R-CTX) as an example.

To prepare prompt and contextual tokens, we pop out patch-embedding layer $\mathcal F_{proj}$ and positional encoding \texttt{PE} from the CLIP vision tower $\mathcal F_{\text{vis}}$. 
We further resize region $\boldsymbol{r}$ and full image $\boldsymbol{i}$ into squares of the same size, then merge them vertically (or horizontally) into a combined image, i.e., \textbf{combo-image} $\boldsymbol{I}$ (\cref{eq:combo_img}). 
We apply patch-embedding on the combo image and add it to the upsampled \texttt{PE}$_{inter}$ embedding to get visual tokens $\boldsymbol{z}_0$ (\cref{eq:ir_enc3}). 
As the \texttt{PE}$_{inter}$ is twice the size of \texttt{PE}, we initialize it by inflating $\texttt{PE}$ using bilinear interpolation. 
Notably, $\boldsymbol{z}_0$, generated from the combo-image, already includes both regional prompt and global contextual tokens. 
The $\boldsymbol{z}_0$ is further fed into the remaining attention and MLP layers (denoted by $\widehat{\mathcal F}_{\text{vis}}$) to get visual representation $f_{\boldsymbol{I}}$ (\cref{eq:vis_combo}). 
We unfreeze patch-embedding $\mathcal F_{proj}$ and positional encoding \texttt{PE}$_{inter}$ to generate learnable soft prompts. 

\begin{align}
    &\boldsymbol{I}=Concat\left(\begin{bmatrix}\boldsymbol{r}\\ \boldsymbol{i}\end{bmatrix}\right)\label{eq:combo_img}\\
    &\boldsymbol{z}_{0}=\mathcal{F}_{proj}(\boldsymbol{I}) + \texttt{PE}_{inter}\label{eq:ir_enc3}\\
     &f_{\boldsymbol{I}}=\widehat{\mathcal{F}}_{\text{vis}}\left(\boldsymbol{z}_{0}\right)\label{eq:vis_combo}
\end{align}

For the ``Region + Colorful Prompt'' and ``Region + Circle Prompt'' (R-CPT \& R-CiR), we create the combo-image $\boldsymbol{I}'$ from the squared region $\boldsymbol{r}$ and a modified image $\boldsymbol{i}'$. 
In this image $\boldsymbol{i}'$, we either color the pixels inside the region with a translucent {\color{mypink}pink} rectangle or outline them with a {\color{red}red} circle \cite{shtedritski2023does}.
The rest of the process remains similar to the ``Region+Context'' prompt.

\textbf{Mixed Prompts}: During training, we randomly choose from the three types of prompts (R-CTX, R-CPT, and R-CiR) with equal chance. 
During testing, we take the average of the visual representations created by these three prompts. 
This strategy regulates the training process and allows us to obtain better generalizability.
%\textcolor{red}{Why this strategy?}

%For the text-based representation $f_{txt}$
%and the loss function, we replace the equations \ref{eq:text_enc}-\ref{eq:clip_base_loss} by the new equations \ref{eq:dc_enc}-\ref{eq:dc_comp} under the Dual-Contrastive Loss learning scheme. %We'll go into more detail about this loss in \S \ref{sec:dual_contrast}.

\subsection{Adapter$^+$ Tuning}
\label{sec:adapter}
Adapter tuning adjusts a parameter-frozen foundational model for downstream tasks by fine-tuning a few newly implanted modules (parameters). 
This strategy is widely used in NLP \cite{yang2023aim} and computer vision \cite{radford2021learning} prior works. 
Current adapters, like mini MLP \cite{yang2023aim} and tiny attention modules \cite{zhao2022tiny}, focus mainly on refining visual features. 
However, they don't consider the need to adjust the original attention maps of the base models. 
In some inference tasks such as visual abductive reasoning, it is beneficial to adapt the attention maps as well, especially to learn context-based domain knowledge and finer visual details.
To tackle this, we augment the vanilla adapter with a new \textbf{\textit{Map Adapter}}, which precisely adapts attention maps in Transformers. 
This results in the improved Adapter$^\textbf{+}$.

The Adapter$^\textbf{+}$ pipeline is illustrated in Figure \ref{fig:rpa_adapter}. 
We first include two basic adapters, referred to as Adapter (A\&M). 
They are placed after the MSHA  module and parallel to the MLP module in the $l$-th encoder of a CLIP tower (e.g., $\mathcal{F}_{\text{vis}}$ or $\mathcal{F}_{\text{txt}}$). 
These adapters are shallow and contain only two fully-connected layers to downgrade and upgrade feature dimension ($\mathbb{R}^D\leftrightarrows \mathbb{R}^d$, $d<D$) with a GELU activation in between (Equation \ref{eq:adapt_1}-\ref{eq:adapt_3}). 
The {\color{purple}light red} font indicates the parameters in the modules are tuned.
\begin{align}
&\boldsymbol{z'}_{l}={\text{MH}}{\text{SA}}\left(\boldsymbol{z}_{l-1}\right),~~~~~~~~~~~~~~~l=1,2,\cdots, L
\label{eq:adapt_1}\\
&\boldsymbol{z''}_{l}={\color{purple}\text{Adapter\_A}}\left(\boldsymbol{z'}_{l}\right) + \boldsymbol{z}_{l-1},
\label{eq:adapt_2}\\
&\boldsymbol{z}_{l}={\text{MLP}}\left(\boldsymbol{z''}_{l}\right)+{\color{purple}\text{Adapter\_M}}\left(\boldsymbol{z''}_{l}\right) + \boldsymbol{z''}_{l},
\label{eq:adapt_3}
%\nonumber &
\end{align}

The {\textbf{Map Adapter}} further refines the MSHA module by adding a small, modified attention map, labeled as {\color{purple}$\widehat{\boldsymbol{Q}}\widehat{\boldsymbol{K}}^T$} (refer to \cref{eq:map_adapt_1}). 
This additive map helps to adjust the original attention map dynamically, improving the model's ability to focus on relevant information. 
To ensure that the original attention map isn't altered too much, we use simpler $D\rightarrow d$ projections for generating the query and key (see \cref{eq:map_adapt_3}). 
Here, $d$ is smaller than $D$.
\begin{align}
&\boldsymbol{z'}_{l}={\text{Softmax}}\left(\frac{\boldsymbol{Q}\boldsymbol{K}^{T} + {\color{purple}\boldsymbol{\widehat{Q}}\boldsymbol{\widehat{K}}^{T}}}{{\sqrt{D}}}
\right)\boldsymbol{V},\label{eq:map_adapt_1}\\
&{\color{purple}\boldsymbol{\widehat{Q}}, \boldsymbol{\widehat{K}}}~= \boldsymbol{z}_{l-1} \times {\color{purple}\boldsymbol{\widehat{W}}_q, \boldsymbol{\widehat{W}}_k},~~~{\color{purple}\widehat{\boldsymbol{W}}_{q,k}}\in \mathbb{R}^{D\times d} \label{eq:map_adapt_3}\\
&\boldsymbol{Q}, \boldsymbol{K}, \boldsymbol{V} = \boldsymbol{z}_{l-1} \times \boldsymbol{W}_q, \boldsymbol{W}_k, \boldsymbol{W}_v, \boldsymbol{W}_{q,k,v}\in \mathbb{R}^{D\times D} \label{eq:map_adapt_2}
\end{align}

We compared the enhanced Adapter$^\textbf{+}$ with min MLP and Tiny Adapter counterparts (see \S \ref{sec:ab_study}) and found that our tuning method consistently performs better.

\subsection{Dual-Contrastive Loss}
\label{sec:dual_contrast}
As an observation contains three modalities, such as visual $\boldsymbol{I}$, clue sentence $\boldsymbol{c}$, and inference sentence $\boldsymbol{h}$, we comprehensively study their mutual influences by deploying contrastive loss between different modalities pairs. 
Specifically, as shown in Fig. \ref{fig:loss_all}, we deploy dual, triple, and single contrastive loss in the training phase and screen out that the \textit{Dual-Contrastive Loss works best} (Fig. \ref{fig:v2_dual_loss_fig}). 
We first elaborate on the Dual-Contrastive Loss and then compare it with the other counterparts.
 \begin{figure}[ht!]
	\centering
   	\begin{subfigure}[t]{0.09\textwidth}
		\includegraphics[width=\textwidth]{eps/d_loss.eps}
            \caption{$\mathit{Dual}$}	\label{fig:loss_dual}
	\end{subfigure}
     \begin{subfigure}[t]{0.09\textwidth}
		\includegraphics[width=\textwidth]{eps/t_loss.eps}
            \caption{$\mathit{Triple}$}	\label{fig:loss_trip}
	\end{subfigure} 
	\begin{subfigure}[t]{0.09\textwidth}
		\includegraphics[width=\textwidth]{eps/s_loss_1.eps}
            \caption{{$\mathit{Single}_{1}$}}	\label{fig:loss_inf_vis}
	\end{subfigure} 
 	\begin{subfigure}[t]{0.09\textwidth}
		\includegraphics[width=\textwidth]{eps/s_loss_2.eps}
            \caption{$\mathit{Single}_2$}	\label{fig:loss_clue_vis}
	\end{subfigure} 
  	\begin{subfigure}[t]{0.09\textwidth}
		\includegraphics[width=\textwidth]{eps/s_loss_3.eps}
            \caption{{$\mathit{MTL}$}}	\label{fig:loss_mt}
	\end{subfigure}
  	\caption{\textbf{Training Losses}: Dual (a), Triple (b) and Single (c)-(e) contrastive loss for learning to match visual $\boldsymbol{J}$ to texts $\boldsymbol{c}$ and $\boldsymbol{f}$. Solid and dashed lines represent the presence or absence of contrastive loss during training.
}\label{fig:loss_all}
\end{figure}

\begin{comment}
 \begin{figure}[ht!]
	\centering
	\begin{subfigure}[t]{0.15\textwidth}
		\includegraphics[width=\textwidth]{iccv2023AuthorKit/img/s_loss_1.png}
            \caption{{Vision-Infer Loss}}	\label{fig:loss_inf_vis}
	\end{subfigure} 
 	\begin{subfigure}[t]{0.15\textwidth}
		\includegraphics[width=\textwidth]{iccv2023AuthorKit/img/s_loss_2.png}
            \caption{{Vision-Clue Loss}}	\label{fig:loss_inf_vis}
	\end{subfigure} 
  	\begin{subfigure}[t]{0.15\textwidth}
		\includegraphics[width=\textwidth]{iccv2023AuthorKit/img/s_loss_3.png}
            \caption{{Mutli Task Loss}}	\label{fig:loss_mt}
	\end{subfigure} \\
   	\begin{subfigure}[t]{0.17\textwidth}
		\includegraphics[width=\textwidth]{iccv2023AuthorKit/img/d_loss.png}
            \caption{{Dual Contrast Loss}}	\label{fig:loss_dual}
	\end{subfigure} 
    \begin{subfigure}[t]{0.17\textwidth}
		\includegraphics[width=\textwidth]{iccv2023AuthorKit/img/t_loss.png}
            \caption{{Triple Contrast Loss}}	\label{fig:loss_trip}
	\end{subfigure} 
  	\caption{
}\label{fig:loss_all}
\end{figure}
\end{comment}

\textbf{Dual-Contrastive Loss}: Both the clue $\boldsymbol{c}$ and inference $\boldsymbol{h}$ are positively relevant to visual $\boldsymbol{I}$. 
More specifically, the former is literally equivalent, while the latter is causally related to the visual hints. 
Although their relations are in different forms, we can still deploy a Dual-Contrastive Loss, including one for ``\textit{vision-clue}'' pair and the other for ``\textit{vision-inference}'' pair (Fig. \ref{fig:loss_dual}), to regress visual features toward two textual targets. 
We use CLIP text tower $\mathcal F_{\text{txt}}$ to extract features for clue $\boldsymbol{c}$ and inference $\boldsymbol{h}$. 
We use \Cref{eq:vis_combo} to extract visual feature $f_{\boldsymbol{I}}$ for the observation $\boldsymbol{I}$.
The mathematical process is present in Equations (\ref{eq:dc_comp})-(\ref{eq:dc_enc}). 
\begin{align}
%&f_{\boldsymbol{c}}=\mathcal{F}_{\text{txt}}\left(\boldsymbol{c}\right)\label{eq:dc_enc}\\
%&f_{\boldsymbol{h}}=\mathcal{F}_{\text{txt}}\left(\boldsymbol{h}\right)\label{eq:dc_enc2}\\
&loss_{\text{dual}}=\mathcal{L}_{contrast}\left(f_{\boldsymbol{I}}, f_{\boldsymbol{c}}\right) + \mathcal{L}_{contrast}\left(f_{\boldsymbol{I}}, f_{\boldsymbol{h}}\right)\label{eq:dc_comp}\\
&f_{\boldsymbol{c}}=\mathcal{F}_{\text{txt}}\left(\boldsymbol{c}\right),~~~~~f_{\boldsymbol{h}}=\mathcal{F}_{\text{txt}}\left(\boldsymbol{h}\right)\label{eq:dc_enc}
\end{align}

\textbf{Other Loss Variants}. The rest loss functions include the \textbf{Triple} and \textbf{Single} contrastive loss. 
Particularly, compared with dual contrastive loss, the triple one newly adds the ``\textit{inference-clue}'' pair (e.g., Fig. \ref{fig:loss_trip} and Eq. \ref{eq:trip}).
\begin{align}
%\boldsymbol{I}_{inf}=\mathcal{F}_{T1}\left(f\right),~~\boldsymbol{I}_{clue}&=\mathcal{F}_{T2}\left(\boldsymbol{c}\right)\\
    loss_{\text{triple}}=\mathcal{L}_{contrast}\left(f_{\boldsymbol{I}}, {f}_{\boldsymbol{c}}\right) +& \mathcal{L}_{contrast}\left(f_{\boldsymbol{I}}, f_{\boldsymbol{h}}\right) \nonumber\\+ \mathcal{L}_{contrast}\left(f_{\boldsymbol{c}}, f_{\boldsymbol{h}}\right)\label{eq:trip}
\end{align}

We later observed that: additional \textit{inference-clue} loss in triple contrastive hurts overall performance, as the two texts (i.e., clue and inference) are not literally equivalent. 
For example, the clue sentence ``\textit{the road is wet}''$\neq$ inference sentence ``it has rained before''. 
Therefore, we can only let the two texts learn toward the same third-party feature (e.g., the visual) instead of artificially forcing them to be equivalent.

For the single contrastive loss, we have three options, namely \textit{vision-inference} (Fig. \ref{fig:loss_inf_vis}), \textit{vision-clue} (Fig. \ref{fig:loss_clue_vis}), \textit{multi-task} learning (MTL in Fig.\ref{fig:loss_mt}). 
Notably, we use an identical textual encoder for clue and inference during testing, since we only learn a single contrastive loss between a pair of modalities during training.

These three losses can be expressed in one unified form (Eq. \ref{eq:unf}), by thresholding a random probability $p$ with different values $\overline{\mathrm{T}}$. 
Specifically, when $\overline{\mathrm{T}}=1.0~\mathrm{or}~0~\mathrm{or}~0.5$, the below loss become the \textit{vision-clue}, \textit{vision-inference} and \textit{multi-task} learning loss \cite{hessel2022abduction}.
\begin{small}
\begin{align}
    \mathit{loss}=\mathcal{L}_{ctr}\left(\boldsymbol{I}_{vis}, \boldsymbol{I}_{txt}\right),\boldsymbol{I}_{txt}=\left\{
\begin{aligned}
\mathcal{F}_{T}\left(\boldsymbol{f}\right)& , p>\overline{\mathrm{T}}& \\
\mathcal{F}_{T}\left(\boldsymbol{c}\right)& , p<\overline{\mathrm{T}}&
\end{aligned}
\right.\label{eq:unf}
\end{align}
\end{small}

With the single contrastive loss, we find that only minimizing the gap between a pair, such as \textit{vision-clue} (or \textit{vision-inference}) will also shorten the gap between the other pair \textit{vision-inference} (or \textit{vision-clue}), indicating retrieval and abductive reasoning tasks are positively correlated. 
We give detailed analysis in \S \ref{sec:ab_study}.


