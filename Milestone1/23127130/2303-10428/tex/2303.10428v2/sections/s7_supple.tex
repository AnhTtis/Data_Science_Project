
This Appendix presents qualitative visualization of RGP$_s$ and baseline, additional ablation studies, and PyTorch-style algorithm implementation.

\section*{A. Qualitative Results of RGP$_s$ and CPT-Baseline}
 \begin{figure}[ht!]
	\centering
 	%\begin{subfigure}[t]{0.3\textheight}
    	\begin{subfigure}[t]{0.49\textwidth}
		\includegraphics[width=\textwidth]{eps/e1.eps}
	\end{subfigure} 
  	%\begin{subfigure}[t]{0.3\textheight}
 	%\begin{subfigure}[t]{0.48\textwidth}
	%	\includegraphics[width=\textwidth]{img/RGP.png}
	%\caption{\textbf{Regional Prompt Tuning}}
        %\label{fig:rgp_clip}
	%\end{subfigure} 

	\caption{\textbf{Top-5 Inferences Retrieved by the RGP$_s$ and CPT model.} ({\color{red}Red} font indicates the right hit with GT-inference. the figure is best viewed in color)
}\label{fig:one_example}
\end{figure}

We visualize qualitative results by our best-performed RGP$_s$ and the previous SOTA CPT-baseline \cite{hessel2022abduction} on the Sherlock validation set. Figure \ref{fig:one_example} shows one good case. We highlight regional visual hints with a translucent pink overlay and present the ground-truth clue \& inference in a yellow rectangle for viewing convenience. Our method successfully retrieves the GT-inference ``\textit{The pic was taken in Japan or UK}'' in the top-five retrieved candidates. More examples are given in Figure \ref{fig:example_vis1} and \ref{fig:example_vis2}.

\section*{B. Additional Ablations on Different Resolutions and Weighting in Loss}
\textbf{Influences of Input Resolutions.} We conduct an experiment studying ``\textit{How many tokens are needed for regional prompt and contextual tokens?}''. Denoting the length of regional prompt/contextual tokens as $L_R$/$L_G$, this length is proportional to the resolution of the region-image/full image, i.e., $L_{R~\text{or}~ G}=\frac{H\cdot W}{P^2}$. Here, [$H$, $W$], $P$ separately represent input resolution and patch-size. To study the influence of using fewer regional prompt tokens, we gradually reduce the resolution of the region-image in the range $R=[224^2, 112^2, 48^2, 16^2]$, while fixing the resolution of the full image to be $C=224^2$. We also conduct the same study for the contextual tokens ($C=[224^2, 112^2, 48^2, 16^2]$, while fixing $R=224^2$).

\begin{table}[htb!]
    \centering
    \tablestyle{2.0pt}{1.08}
    \resizebox{1.04\linewidth}{!}{
    \begin{tabular}{l|ccc|c|c}
    \multirow{2}*{\textit{RGP (Val-Set)}}&\multicolumn{3}{c|}{\textit{Retrieval}}& \textit{Localization} & \textit{Comparison} \\
    &{im$\rightarrow$txt ({\color{purple}$\downarrow$})}& {txt$\rightarrow$im
 ({\color{purple}$\downarrow$})} 
    & {P@1$_{i\rightarrow t}$ ({\color{purple}$\uparrow$})}  & GT/Auto-Box ({\color{purple}$\uparrow$})& Human Acc ({\color{purple}$\uparrow$})\\ 
    
    \hline		 
    %\rowcolor{baselinecolor}
    Context only, C=224$^2$ &{47.35}& {57.18}&{17.87}&{-} / {-}&{19.94}\\
    R=\textbf{16}$^2$,~~C=224$^2$ &{24.33}& {27.14}&{26.79}&{80.11} / {33.06}&{{21.92}}\\
    R=\textbf{48}$^2$,~~C=224$^2$ &{20.76}& {23.37}&{29.00}&{83.62} / {37.23}&{23.56}\\
    R=\textbf{112}$^2$, C=224$^2$ &{18.87}& {21.32}&{30.82}&{85.08} / {39.27}&{{24.64}}\\
    \hline
    Region only, R=224$^2$ &{25.20}& {24.79}&{28.72}&\textbf{85.54} / {41.79}&{20.90}\\
    R=224$^2$, C=\textbf{16}$^2$ &{23.89}& {24.08}&{29.03}&{85.24} / \textbf{42.06}&{{20.42}}\\
    R=224$^2$, C=\textbf{48}$^2$ &{23.69}& {25.00}&{28.81}&{84.86} / {40.86}&{21.59}\\   
    R=224$^2$, C=\textbf{112}$^2$ &{18.71}& {20.64}&{31.55}&{85.49} / {41.14}&{{23.22}}\\   
    \hline
    \rowcolor{baselinecolor}
    R=$224^2$, C=$224^2$ &\textbf{17.63}& \textbf{20.25}&\textbf{32.00}&{85.26} / {39.78}&{\textbf{25.65}}\\
    \end{tabular}
    }
    \caption{\textbf{Influcences of Input Resolutions (R=Region, C=Context Image) on Sherlock Val-Set}. We tested different settings on the \textit{General RGP} with CLIP ViT-B-16 backbone.
    The up arrow {\color{purple}$\uparrow$} (or down arrow {\color{purple}$\downarrow$}) indicates the higher (or lower), the better.
	}
    %\vspace{-10pt}
    \label{tab:res_rc}
    %\vspace{-10pt}
\end{table}
 \begin{figure}[ht!]
	\centering
 	%\begin{subfigure}[t]{0.3\textheight}
    	\begin{subfigure}[t]{0.42\textwidth}
		\includegraphics[width=\textwidth]{eps/cur.eps}
	\end{subfigure} 
	\caption{\textbf{Influence of Input Resolutions (R=Region, C=Context).} X axis is the resolution of region (R) or context (C) (i.e., $HW=X^2$), Y axis is the $P@1_{img\rightarrow txt}$ value.
}\label{fig:res_comp}
\end{figure}

We test different resolutions' combinations on the Sherlock Val-Set with our \textit{general RGP} model using CLIP ViT-B-16 backbone. As shown in Table \ref{tab:res_rc}, we find that: (1). Reducing input resolutions results in shorter tokens' length ($L_R$ or $L_G$) and then causes most metrics to become worse than before: \textit{\textbf{the smaller resolution (shorter length), the worse performance}}; (2). \textit{\textbf{The regional prompt plays a more important role than contextual tokens}}, as reducing the length of the former causes more performance drop than the latter. For a more intuitive display, we draw two curves of $P@1_{img\rightarrow text}$ with respect to the resolution $HW$=$X^2$ in Figure \ref{fig:res_comp}: \{$R=X^2$, $C=224^2$\} \textit{vs} \{$C=X^2$, $R=224^2$\}. The $P@1_{i\rightarrow t}$ drops significantly when the $R$ (resolution of the region) reduces.

\textbf{Weighting in Loss.} Since the Dual-Contrastive Loss contains two parts: ``\textit{vision-clue}''  and ``\textit{vision-inference}'' losses, we further study the balance of two parts by introducing two weights (i.e., $\alpha$, $\beta$) into the loss function (see Eq. \ref{eq:dc_comp_weight}).
\begin{align}
  loss=\alpha\cdot\mathcal{L}_{ctr}\left(\boldsymbol{I}_{vis}, \boldsymbol{I}_{clue}\right) &+ \beta\cdot\mathcal{L}_{ctr}\left(\boldsymbol{I}_{vis}, \boldsymbol{I}_{inf}\right)\label{eq:dc_comp_weight}\\
  s.t.~\alpha + \beta&=1\nonumber
\end{align}
\begin{table}[htb!]
    \centering
    \tablestyle{2.0pt}{1.08}
    \resizebox{1.04\linewidth}{!}{
    \begin{tabular}{l|ccc|c|c}
    \multirow{2}*{\textit{\textbf{RGP$_s$} (Val-Set)}}&\multicolumn{3}{c|}{\textit{Retrieval}}& \textit{Localization} & \textit{Comparison} \\
    &{im$\rightarrow$txt ({\color{purple}$\downarrow$})}& {txt$\rightarrow$im
 ({\color{purple}$\downarrow$})} 
    & {P@1$_{i\rightarrow t}$ ({\color{purple}$\uparrow$})}  & GT/Auto-Box ({\color{purple}$\uparrow$})& Human Acc ({\color{purple}$\uparrow$})\\ 
    
    \hline		 
    %\rowcolor{baselinecolor}
    ``\textit{Vision-Inference}''&\multirow{2}*{16.29}& \multirow{2}*{18.45}&\multirow{2}*{33.00}&\multirow{2}*{85.95 / {40.38}}&\multirow{2}*{{25.56}}\\
    $\alpha$=0.0, $\beta$=1.0 &&&&&\\
    \hline
    $\alpha$=0.1, $\beta$=0.9 &{16.06}& {18.11}&{33.31}&{86.04} / {40.57}&{{25.11}}\\
    \textbf{$\alpha$=0.3}, \textbf{$\beta$=0.7} &\textbf{15.95}& \textbf{18.03}&\textbf{33.32}&\textbf{86.41} / {41.02}&{\textbf{25.56}}\\
    \rowcolor{baselinecolor}
    $\alpha$=0.5, $\beta$=0.5 &{16.05}& {18.19}&{33.29}&{86.52} / \textbf{41.06}&{{25.05}}\\ 
    $\alpha$=0.7, $\beta$=0.3 &{16.85}& {19.20}&{32.44}&{85.93} / {40.62}&{{24.34}}\\ 
    $\alpha$=0.9, $\beta$=0.1 &{19.54}& {22.56}&{30.86}&{85.06} / {39.17}&{{23.98}}\\ 
    \hline
    ``\textit{Vision-Clue}''&\multirow{2}*{36.47}& \multirow{2}*{44.37}&\multirow{2}*{25.13}&\multirow{2}*{80.95 / 30.62}&\multirow{2}*{{21.51}}\\
    $\alpha$=1.0, $\beta$=0.0 &&&&&\\
    %\rowcolor{baselinecolor}
    %$\alpha$=1.0, $\beta$=1.0 &{15.78}& {17.88}&{33.67}&{86.48} / {40.94}&{{25.05}}\\
    \end{tabular}
    }
    \caption{\textbf{Impacts of Weighting on the Dual-Contrastive Loss on Sherlock Val-Set}. We test different settings on the \textit{Simplified RGP$_s$+Dual-Contrastive Loss} with CLIP ViT-B-16 backbone.
	}
    %\vspace{-10pt}
    \label{tab:weights_loss}
    %\vspace{-10pt}
\end{table}
 \begin{figure*}[ht!]
	\centering
	\begin{subfigure}[t]{0.32\textwidth}
		\includegraphics[width=\textwidth]{eps/weight_1.eps}
        \caption{Mean Rank of $img\rightarrow txt$ with respect to weight $\alpha$. (lower=better)} 
        \label{fig:weight_1}
	\end{subfigure}
	\begin{subfigure}[t]{0.32\textwidth}
		\includegraphics[width=\textwidth]{eps/weight_2.eps}
        \caption{Mean Rank of $txt\rightarrow img$ with respect to weight $\alpha$. (lower=better)}	
        \label{fig:weight_2}
	\end{subfigure} 
 	\begin{subfigure}[t]{0.32\textwidth}
		\includegraphics[width=\textwidth]{eps/weight_3.eps}
        \caption{$P@1_{img\rightarrow txt}$ with respect to weight $\alpha$. (higher=better)}	
        \label{fig:weight_3}
	\end{subfigure}\\
	\caption{\textbf{Impacts of Weighting on the Dual-Contrastive Loss on Sherlock Val-Set.} X axis is weight scalar $\alpha$.
}\label{fig:weight_loss}
\end{figure*}
We test different weights' combinations $\{\alpha, \beta\}$ on the Sherlock Val-Set with our \textit{Simplified RGP$_s$ + Dual-Contrastive Loss} using CLIP ViT-B-16 backbone. The results are shown in Table \ref{tab:weights_loss}. We find that: (1). The best performance is achieved by setting $\{\alpha=0.3, \beta=0.7\}$, which slightly bias to ``\textit{vision-inference}'' loss under the Dual-Contrastive Loss scheme. (2). Most evaluation metrics arrive at saturated states when the $\alpha \geq 0.5$, indicating our standard Dual-Contrastive Loss (Eq. \ref{eq:dc_comp}), that treats each part equally, is also an optimal choice. This trend is reflected by drawing the curves of  \textit{Retrieval} metrics with respect to the weight scalar $\alpha$ (see Figure \ref{fig:weight_loss}).

\section*{C. Algorithm of RGP$_s$ and Dual-Contrastive Loss}

We present the Algorithm of RGP$_s$ and the Dual-Contrastive Loss in PyTorch-style as below. 
\begin{algorithm}[]
\small
\caption{\small Codes for RGP$_s$ and Dual-Contrastive Loss (PyTorch-like)}
\vspace{-0.05in}
\begin{PythonA}[frame=none]
# Inputs:
#   img: input tensor of shape (H, W, C)
#   bbox: Coordinates in format [x1, y1, w, h]
#   clue: literal sentence
#   inference: hypthesis sentence
# Model Parameters:
#   Image_Enc: Init from CLIP Image Encoder
#   Text_Enc1: Init from CLIP Text Encoder
#   Text_Enc2: Init from CLIP Text Encoder

# 1. Upsampling Pos-Enc to handle (2H, W) input
Image_Enc = Resize_Pos_Embed(Image_Enc)

# 2. Create Combo Image (H, W, C)->(2H, W, C)
#  Vertically concatenate Image and Box-Image.
bbox_img  = Crop_Resize(img, bbox) # (H, W, C) 
combo_img = Vertical_Concatenate(bbox_img, img)

# 3. Dual-Contrastive Loss
I_vis  = Image_Enc(combo_img)
I_inf  = Text_Enc1(inference)
I_clue = Text_Enc2(clue)
loss_1 = Contrastive_Loss(I_vis, I_inf)
loss_2 = Contrastive_Loss(I_vis, I_clue)
total_loss = loss_1 + loss_2
total_loss.backward()

\end{PythonA}
\label{algo}
\vspace{-0.05in}
\end{algorithm}

 \begin{figure*}[ht!]
	\centering
 	%\begin{subfigure}[t]{0.3\textheight}
	\begin{subfigure}[t]{0.49\textwidth}
		\includegraphics[width=\textwidth]{eps/e2.eps}
  \caption{}	\label{fig:example_1}
	\end{subfigure} 
	\begin{subfigure}[t]{0.49\textwidth}
		\includegraphics[width=\textwidth]{eps/e3.eps}
  \caption{}	\label{fig:example_2}
	\end{subfigure} \\
	\begin{subfigure}[t]{0.49\textwidth}
		\includegraphics[width=\textwidth]{eps/e4.eps}
  \caption{}	\label{fig:example_3}
	\end{subfigure} 
	\begin{subfigure}[t]{0.49\textwidth}
		\includegraphics[width=\textwidth]{eps/e5.eps}
  \caption{}	\label{fig:example_4}
	\end{subfigure} \\
	\begin{subfigure}[t]{0.49\textwidth}
		\includegraphics[width=\textwidth]{eps/e6.eps}
  \caption{}	\label{fig:example_5}
	\end{subfigure} 
	\begin{subfigure}[t]{0.49\textwidth}
		\includegraphics[width=\textwidth]{eps/e7.eps}
  \caption{}	\label{fig:example_6}
	\end{subfigure} \\
	\begin{subfigure}[t]{0.49\textwidth}
		\includegraphics[width=\textwidth]{eps/e8.eps}
  \caption{}	\label{fig:example_7}
	\end{subfigure} 
	\begin{subfigure}[t]{0.49\textwidth}
		\includegraphics[width=\textwidth]{eps/e9.eps}
  \caption{}	\label{fig:example_8}
	\end{subfigure} 
	\caption{\textbf{Top-5 Inferences Retrieved by the RGP$_s$ and CPT model.} ({\color{red}red} font indicates right hit. the figure is best viewed in color)
}\label{fig:example_vis1}
\end{figure*}
\begin{tikzpicture}[scale=0.8,line cap=round,line join=round,>=triangle 45,x=1cm,y=1cm]
    \begin{axis}[
    x=1cm,y=1cm,
    axis lines=middle,
    xmin=-2.5,
    xmax=2.5,
    ymin=-1.8,
    ymax=1.8,
    xtick={-1,0,1},
    ytick={-1,0,1}]
    \clip(-8.590825495472101,-4.438436597195295) rectangle (9.917963951288046,4.622148812606115);
    \draw [line width=1pt,dotted,fill=black,pattern=north east lines,pattern color=black] (0,0) circle (0.7071067811865475cm);
    \draw [line width=1pt,color=qqttzz,domain=-8.590825495472101:9.917963951288046] plot(\x,{(--3-0*\x)/3});
    \draw [line width=1pt,color=qqttzz] (1,-4.438436597195295) -- (1,4.622148812606115);
    \draw [line width=1pt,color=qqwuqq,domain=-8.590825495472101:9.917963951288046] plot(\x,{(-0--1*\x)/1});
    \draw (2,-0.03) node[anchor=north west] {$x_{1}$};
    \draw (0.03,1.8) node[anchor=north west] {$x_{2}$};
\draw [line width=1pt,dash pattern=on 2pt off 2pt,domain=-10.387062729306384:1.5311150742190547] plot(\x,{(-0.479848261008447-0.7231001624628146*\x)/-1.2029484234712615});
\draw [line width=1pt,dash pattern=on 2pt off 2pt,domain=-10.387062729306384:1.5311150742190547] plot(\x,{(--0.553364676707671-1.1897124569152553*\x)/-0.6363477802075843});
    \begin{scriptsize}
    \draw [fill=black] (1,1) circle (1.5pt);
    \draw[color=qqttzz] (-2.2,0.75) node {$L_{1}$};
    \draw[color=qqttzz] (1.3,-1.5) node {$L_2$};
    \draw[color=qqwuqq] (-1.3,-1.65) node {$L_3$};
    \draw[color=black] (1.2,1.260768530226581) node {$A$};
    \end{scriptsize}
    \end{axis}
\end{tikzpicture}