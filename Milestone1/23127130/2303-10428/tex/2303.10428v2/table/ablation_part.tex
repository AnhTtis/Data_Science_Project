\begin{table}[htb!]
    \centering
    \tablestyle{2.0pt}{1.08}
    \resizebox{1.04\linewidth}{!}{
    \begin{tabular}{l|ccc|c|c}
    \multirow{2}*{\textit{Model (Val-Set)}}&\multicolumn{3}{c|}{\textit{Retrieval}}& \textit{Localization} & \textit{Comparison} \\
    &{im$\rightarrow$txt ({\color{purple}$\downarrow$})}& {txt$\rightarrow$im ({\color{purple}$\downarrow$})} 
    & {P@1$_{i\rightarrow t}$ ({\color{purple}$\uparrow$})}  & GT/Auto-Box (${\color{purple}\uparrow}$)& Human Acc (${\color{purple}\uparrow}$)\\ 
    
    \hline		 
    %\rowcolor{baselinecolor}
    \rotatebox[origin=c]{0}{$\Rsh$} Context only &47.35&57.18&17.87&- / -&19.94 \\
    \rotatebox[origin=c]{0}{$\Rsh$} Region only &25.20&24.79&28.72&85.54 / 41.79&20.90 \\
    \rotatebox[origin=c]{0}{$\Rsh$} plain Context + Region &17.88&20.86&31.72&85.11 / 38.54& \textbf{24.82} \\
    \rowcolor{baselinecolor}
    \textbf{RGP$_s$} &\textbf{16.79}&\textbf{19.01}&\textbf{32.55}&\textbf{85.98} / \textbf{40.84}&{24.64} \\
    \end{tabular}
    }
    \caption{\textbf{Comparisons of Each Part on Sherlock Validation Set}. \textit{Region} or \textit{context only} separately represent merely keeping the region prompt or contextual tokens. The \textit{plain context+region} denotes directly adding feature vectors of the two parts. }
    %we independently apply image encoders on prompt and contextual tokens like Figure \ref{fig:clip_base}, and adding output feature vectors of the two parts.}
    %\vspace{-10pt}
    \label{tab:impacts_of_parts}
    %\vspace{-10pt}
\end{table}