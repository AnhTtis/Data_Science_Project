\section{Problem Definition \& Baseline}
%We first revisit the definition and baseline with the \textit{Sherlock} benchmark \cite{hessel2022abduction} for the visual abductive reasoning task.

%\subsection{Problem Definition \& Baseline}
\label{sec:problem}
\textbf{Problem}: Hessel et al.\cite{hessel2022abduction} defines a Visual Abductive Reasoning benchmark named ``\emph{Sherlock}'' that requires a model to predict the hypothesis from observations in  ``\textbf{\textit{Observation}} $\boldsymbol{\rightarrow}$ \textbf{\textit{Hypothesis}}'' form. Specifically, visual observation refers to a pre-specified region $\boldsymbol{r}$ of an image $\boldsymbol{i}$ and is accompanied by a clue sentence $\boldsymbol{c}$. Notably, the clue is a straightforward description of real visual content and is only available during training. On the other hand, the hypothesis is defined by an inference sentence $\boldsymbol{f}$. With this, a VAR model calculates a score $\boldsymbol{s}$, which reflects the probability of deducing inference $\boldsymbol{f}$ from the region $\boldsymbol{r}$. Equation \ref{eq:similarity} shows this scoring function $\mathcal F$ and the parameters $\theta$; we call $\mathcal F$ the VAR model.



\begin{align}
    s&=\mathcal{F}(\boldsymbol{f}, \boldsymbol{i}, \boldsymbol{r}| \theta)\label{eq:similarity}
\end{align}
A good VAR model should generate a larger matching score when an inference $\boldsymbol{f}$ and observation $\{\boldsymbol{i}$, $\boldsymbol{r}\}$) are causally related, and a smaller value for wrong or non-related inferences.

 \begin{figure*}[ht!]
\centering
\subfloat[CPT\label{fig:clip_base}]{
\includegraphics[height=0.37\linewidth]{eps/v3_base.eps}
} 
\subfloat[\rpa\label{fig:rpa}]{
\includegraphics[height=0.37\linewidth]{eps/v3_rpa.eps}
} 
\hspace{-0.35cm}
\subfloat[\rpa~(Adapter$^\textbf{+}$)\label{fig:rpa_adapter}]{
\includegraphics[height=0.37\linewidth]{eps/adapter.eps}
} 
\subfloat[Dual-Contrastive Loss\label{fig:v2_dual_loss_fig}]{
\includegraphics[height=0.37\linewidth]{eps/v2_dc_loss.eps}
} 
%	\begin{subfigure}[t]{0.18\textwidth}
%		\includegraphics[width=\textwidth]{eps/v3_base.eps}
%  \caption{\textbf{Color Prompt} (CPT)}	\label{fig:clip_base}
%	\end{subfigure} 
% 	\begin{subfigure}[t]{0.18\textwidth}
%		\includegraphics[width=\textwidth]{eps/v3_base_extend.eps}
%  \caption{{CPT: ($\times 2$ \texttt{Tokens})}}	\label{fig:clip_base_extend}
%	\end{subfigure} 
%  	\begin{subfigure}[t]{0.18\textwidth}
%		\includegraphics[width=\textwidth]{eps/v3_rgp.eps}
%  \caption{\textbf{Regional Prompt}}	\label{fig:rgp_clip}
%	\end{subfigure} 
%   	\begin{subfigure}[t]{0.18\textwidth}
%		\includegraphics[width=\textwidth]{eps/v3_rgp_s.eps}
%  \caption{\textbf{Simplified RGP$_s$}}	\label{fig:rgp_clip_s}
%	\end{subfigure} 
%    	\begin{subfigure}[t]{0.23\textwidth}
%		\includegraphics[width=\textwidth]{eps/v2_dc_loss.eps}
%  \caption{\textbf{Dual-Contrastive Loss}}	\label{fig:v2_dual_loss_fig}
%	\end{subfigure} 
  	%\begin{subfigure}[t]{0.3\textheight}
 	%\begin{subfigure}[t]{0.48\textwidth}
	%	\includegraphics[width=\textwidth]{iccv2023AuthorKit/img/RGP.png}
	%\caption{\textbf{Regional Prompt Tuning}}
        %\label{fig:rgp_clip}
	%\end{subfigure} 
	\caption{\textbf{Comparing Colorful Prompt Tuning and Region-Prompted Adapter Tuning}. (a) \textit{Colorful Prompt Tuning} (CPT) from \cite{yao2021cpt} uses a semi-transparent {\color{mypink}pink} mask to highlight a designated area, marked as $\boldsymbol{r}$; (b) Our method, termed \textit{Region-Prompted Adapter Tuning} (\rpa), simultaneously generates region prompt and contextual tokens by intaking a combo-image $\boldsymbol{J}$, then tunes the frozen CLIP with Adapter$^\textbf{+}$ on top of reasoning dataset; (c) Adapter$^\textbf{+}$ includes two standard adapters and a novel Map Adapter, which separately adjust token features and the attention map; (d) Dual-Contrastive Loss simultaneously guide the visual content minimize semantic (to clue) and causal (to inference) gaps. (Note: Best viewed in color.)}%\label{fig:clip_base}
\end{figure*}

%\bff{Perhaps instead of saying baseline. We can say colorful prompt tuning}
%\bff{Now we briefly describe the prior method colorful prompt tuning and then explains our model in secion XXX..
%Before explaining our model in section 4, we present the prior model called colorful prompt tuning.
%}

\textbf{Baseline}: As presented in \cite{hessel2022abduction}, a CLIP \cite{radford2021learning} with colorful prompt tuning (CPT) \cite{yao2021cpt} is adopted as the baseline for the Sherlock benchmark (Fig. \ref{fig:clip_base}). 
Specifically, the CPT highlights the regional observation $\boldsymbol{r}$ by a translucent {\color{mypink}pink} overlay (or mask). Then the altered image $\boldsymbol{i}$ is split into left \& right parts (i.e., $\boldsymbol{i}_l$ \& $\boldsymbol{i}_r$) for feature extraction (Eq. \ref{eq:ir_enc}), and the global feature $\mathcal{I}_{vis}$ is averaged from both parts (Eq. \ref{eq:i_mean}). 
Meanwhile, the inference text $\boldsymbol{f}$ is encoded into a textual feature vector $\mathcal{I}_{text}$ by a text encoder $\mathcal{F}_{\text{T}}$ (Eq. \ref{eq:text_enc}). 
To reduce overfitting, the inference text $\boldsymbol{f}$ is randomly replaced by the clue $\boldsymbol{c}$ with a probability of 0.5 during training. 
This strategy is named \textit{Multi-Task Learning} as now the model has to predict inference text 50\% of the time and clue text 50\% of the time during training. Finally, a contrastive loss $\mathcal L_{ctr}$ bridges the gap between visual and textual features as shown in Eq. \ref{eq:clip_base_loss}. 
\begin{align}
    &\mathcal{I}_{l}=\mathcal{F}_{\text{V}}(\boldsymbol{i}_{l}),~~~ \mathcal{I}_{r}=\mathcal{F}_{\text{V}}(\boldsymbol{i}_{r})\label{eq:ir_enc}\\
    %&\boldsymbol{I}_{2}=\mathcal{F}_{\text{V}}(\boldsymbol{i}_{r})\label{eq:ir_enc}\\
    &\mathcal{I}_{vis}=\left(\mathcal{I}_{l} + \mathcal{I}_r\right)/2\label{eq:i_mean}\\
    &\mathcal{I}_{text}=\left\{\begin{aligned}
\mathcal{F}_{\text{T}}\left(\boldsymbol{f}\right)& , p\geqslant0.5& \\
\mathcal{F}_{\text{T}}\left(\boldsymbol{c}\right)& , p<0.5&
\end{aligned}
\right.\label{eq:text_enc}\\
    %&\boldsymbol{I}_{text}=\mathcal{F}_{\text{T}}(\boldsymbol{f})\label{eq:text_enc}\\
    &loss=\mathcal{L}_{ctr}\left(\mathcal{I}_{vis}, \mathcal{I}_{text}\right)\label{eq:clip_base_loss}
\end{align}
\begin{figure}[ht!]
\centering
\subfloat[R-CTX\label{fig:prompt1}]{
\includegraphics[height=0.58\linewidth]{eps/rpa.eps}
} 
\subfloat[R-CPT\label{fig:prompt2}]{
\includegraphics[height=0.58\linewidth]{eps/rgp_cpt}
} 
\subfloat[R-CiR\label{fig:prompt3}]{
\includegraphics[height=0.58\linewidth]{eps/rgp_cir}
} 
\vspace{-0.2cm}
\caption{\textbf{Three Types of Fine-Grained Region Prompts} generated by the RPG. In RPG, we assemble the combo-image I$\boldsymbol{I}$ from region $\boldsymbol{r}$ and context, colorful, or circle-prompted image $\boldsymbol{i}$. %Then, the combo-image $\boldsymbol{I}$ goes through the patch-embedding layer and is further added with positional embedding to generate input tokens. 
(a) R-CTX: \textit{Region+Context}; (b) R-CPT: \textit{Region+Colorful Prompt}; (c) R-CiP: \textit{Region+Circle Prompt}.}%\label{fig:rpg}
\vspace{-0.54cm}
\end{figure}
