\begin{table}[htb!]
    \centering
    \tablestyle{2.0pt}{1.08}
    \resizebox{1.04\linewidth}{!}{
    \begin{tabular}{l|ccc|c|c}
    \multirow{2}*{\textit{Model (Val-Set)}}&\multicolumn{3}{c|}{\textit{Retrieval}}& \textit{Localization} & \textit{Comparison} \\
    &{im$\rightarrow$txt ({\color{purple}$\downarrow$})}& {txt$\rightarrow$im
 ({\color{purple}$\downarrow$})} 
    & {P@1$_{i\rightarrow t}$ ({\color{purple}$\uparrow$})}  & GT/Auto-Box ({\color{purple}$\uparrow$})& Human Acc ({\color{purple}$\uparrow$})\\ 
    
    \hline		 
    CPT-CLIP \cite{hessel2022abduction} (our impl)&19.94 &22.15&30.05& 85.16 / 38.87&21.48   \\    
    CPT-CLIP \texttt{($\times2$ Tokens)} & 20.33& 22.32&29.74 & 84.73 / 38.70  &  22.22 \\    
    \hline
    %\rowcolor{baselinecolor}
    \textbf{RGP} &17.63& 20.25&32.00&85.26 / 39.78&\textbf{25.65}\\
    \rowcolor{baselinecolor}
    \textbf{RGP$_s$} &\textbf{16.79}&\textbf{19.01} &\textbf{32.55}&\textbf{85.98} / \textbf{40.84}&24.64\\

    \end{tabular}
    }
    \caption{\textbf{Comparisons of Region Prompt Tuning and CPT-CLIP (Baseline) on Sherlock Validation Set}. 
    The up arrow {\color{purple}$\uparrow$} (or down arrow {\color{purple}$\downarrow$}) indicates the higher (or lower), the better.
	}
    %\vspace{-10pt}
    \label{tab:rgp_vs_baseline}
    %\vspace{-10pt}
\end{table}