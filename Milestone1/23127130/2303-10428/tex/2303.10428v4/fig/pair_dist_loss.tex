 \begin{figure}[ht!]
	\centering
\subfloat[$\mathit{No Train}$\label{fig:dist_p1_no_train}]{
\includegraphics[width=0.22\linewidth]{eps/p1_no_train.eps}
}
\subfloat[$\mathit{Single}_{1}$\label{fig:dist_p3_inf_only}]{
\includegraphics[width=0.22\linewidth]{eps/p3_inf_only.eps}
}
 \subfloat[$\mathit{Single}_{2}$\label{fig:dist_p2_clue_only}]{
\includegraphics[width=0.22\linewidth]{eps/p2_clue_only.eps}
}\\
 \subfloat[$\mathit{MTL}$\label{fig:dist_p4_mkl}]{
\includegraphics[width=0.22\linewidth]{eps/p4_mkl.eps}
}
 \subfloat[$\mathit{Triple}$\label{fig:dist_p5_trip}]{
\includegraphics[width=0.22\linewidth]{eps/p5_trip.eps}
}
 \subfloat[$\mathit{Dual}$\label{fig:dist_p6_dual}]{
\includegraphics[width=0.22\linewidth]{eps/p6_dual.eps}
}
%   	\begin{subfigure}[t]{0.10\textwidth}
%		\includegraphics[width=\textwidth]{eps/p1_no_train.eps}
%            \caption{$\mathit{No Train}$}	\label{fig:dist_p1_no_train}
%	\end{subfigure}
% 	\begin{subfigure}[t]{0.10\textwidth}
%		\includegraphics[width=\textwidth]{eps/p3_inf_only.eps}
 %           \caption{{$\mathit{Single}_{1}$}}	\label{fig:dist_p3_inf_only}
%	\end{subfigure} 
%     \begin{subfigure}[t]{0.10\textwidth}
%		\includegraphics[width=\textwidth]{eps/p2_clue_only.eps}
%            \caption{$\mathit{Single}_{2}$}	\label{fig:dist_p2_clue_only}
%	\end{subfigure} \\
% 	\begin{subfigure}[t]{0.10\textwidth}
%		\includegraphics[width=\textwidth]{eps/p4_mkl.eps}
%            \caption{$\mathit{MKL}$}	\label{fig:dist_p4_mkl}
%	\end{subfigure} 
%  	\begin{subfigure}[t]{0.10\textwidth}
%		\includegraphics[width=\textwidth]{eps/p5_trip.eps}
 %           \caption{{$\mathit{Triple}$}}	\label{fig:dist_p5_trip}
%	\end{subfigure}
%   	\begin{subfigure}[t]{0.10\textwidth}
%		\includegraphics[width=\textwidth]{eps/p6_dual.eps}
%            \caption{{$\mathit{Dual}$}}	\label{fig:dist_p6_dual}
%	\end{subfigure}
  	\caption{\textbf{Contrastive Losses between Modality Pair on Sherlock Validation Set}. We used a model trained with (a) No Training, (b-c) Single, (d) MTL, (e) Triple, and Dual (f) losses. The {\color{teal}Green}/{\color{red}Red} implies the decreasing/increasing of loss values, compared with \text{No Training}. Solid and dashed lines denotes the presence or absence of contrastive loss in training.
}\label{fig:dist_all}
\vspace{-0.5cm}
\end{figure}
