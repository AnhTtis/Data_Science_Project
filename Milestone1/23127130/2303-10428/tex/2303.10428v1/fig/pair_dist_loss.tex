 \begin{figure}[ht!]
	\centering
   	\begin{subfigure}[t]{0.10\textwidth}
		\includegraphics[width=\textwidth]{eps/p1_no_train.eps}
            \caption{$\mathit{No Train}$}	\label{fig:dist_p1_no_train}
	\end{subfigure}
 	\begin{subfigure}[t]{0.10\textwidth}
		\includegraphics[width=\textwidth]{eps/p3_inf_only.eps}
            \caption{{$\mathit{Single}_{1}$}}	\label{fig:dist_p3_inf_only}
	\end{subfigure} 
     \begin{subfigure}[t]{0.10\textwidth}
		\includegraphics[width=\textwidth]{eps/p2_clue_only.eps}
            \caption{$\mathit{Single}_{2}$}	\label{fig:dist_p2_clue_only}
	\end{subfigure} \\
 	\begin{subfigure}[t]{0.10\textwidth}
		\includegraphics[width=\textwidth]{eps/p4_mkl.eps}
            \caption{$\mathit{MKL}$}	\label{fig:dist_p4_mkl}
	\end{subfigure} 
  	\begin{subfigure}[t]{0.10\textwidth}
		\includegraphics[width=\textwidth]{eps/p5_trip.eps}
            \caption{{$\mathit{Triple}$}}	\label{fig:dist_p5_trip}
	\end{subfigure}
   	\begin{subfigure}[t]{0.10\textwidth}
		\includegraphics[width=\textwidth]{eps/p6_dual.eps}
            \caption{{$\mathit{Dual}$}}	\label{fig:dist_p6_dual}
	\end{subfigure}
  	\caption{\textbf{Individual Contrastive Loss between Modality Pair on Sherlock Validation Set}. We used a model trained with (a) No Training, (b-c) Single, (d) MKL, (e) Triple, and Dual (f) losses. The {\color{teal}Green}/{\color{red}Red} implies the decreasing/increasing of loss values, compared with \text{No Training}. Solid and dashed lines represent the presence or absence of contrastive loss during training.
}\label{fig:dist_all}
\end{figure}
