 \begin{figure*}[ht!]
	\centering
	\begin{subfigure}[t]{0.18\textwidth}
		\includegraphics[width=\textwidth]{eps/v3_base.eps}
  \caption{\textbf{Color Prompt} (CPT)}	\label{fig:clip_base}
	\end{subfigure} 
 	\begin{subfigure}[t]{0.18\textwidth}
		\includegraphics[width=\textwidth]{eps/v3_base_extend.eps}
  \caption{{CPT: ($\times 2$ \texttt{Tokens})}}	\label{fig:clip_base_extend}
	\end{subfigure} 
  	\begin{subfigure}[t]{0.18\textwidth}
		\includegraphics[width=\textwidth]{eps/v3_rgp.eps}
  \caption{\textbf{Regional Prompt}}	\label{fig:rgp_clip}
	\end{subfigure} 
   	\begin{subfigure}[t]{0.18\textwidth}
		\includegraphics[width=\textwidth]{eps/v3_rgp_s.eps}
  \caption{\textbf{Simplified RGP$_s$}}	\label{fig:rgp_clip_s}
	\end{subfigure} 
    	\begin{subfigure}[t]{0.23\textwidth}
		\includegraphics[width=\textwidth]{eps/v2_dc_loss.eps}
  \caption{\textbf{Dual-Contrastive Loss}}	\label{fig:v2_dual_loss_fig}
	\end{subfigure} 
	\caption{\textbf{Comparison of Colorful and Regional Prompt Tuning}. (a). The \textit{Colorful Prompt Tuning} (CPT) \cite{yao2021cpt}, which highlights region $\boldsymbol{r}$ with a translucent ``{\color{pink}pink} mask''; (b) We extend the CPT by processing colorful prompt and visual tokens together in one Transformer; (c) Our \textit{Regional Prompt Tuning} (RGP) explicitly encodes regional information into ``{\color{pigment}prompt}'' tokens to guide the reasoning process; (d) Simplified RGP$_s$ simultaneously generating region prompt and contextual tokens by intaking a combo-image $\boldsymbol{J}$; (e) Dual-Contrastive Loss simultaneously guide the visual content minimize semantic (to clue) and causal (to inference) gaps. (the figure is best viewed in color)
}%\label{fig:clip_base}
\end{figure*}