\section{Introduction}
\IEEEPARstart{M}{ultivariate} networks~\cite{kerren2014multivariate} consisting of both topological and semantic information can model complex relations between entities.
One common analysis task performed on multivariate networks is to understand associations among structural and semantic characteristics~\cite{kerren2014multivariate,atzmueller2021mining,chetty2022social,chen2023calliopenet}.
For example, from social media usage, analysts may want to see how likely each possible combination of individual characteristics (e.g., age, gender, and extraversion) and friendship structures is related to their addiction level to social media. 
Such analysis can be more complicated when the associations underlie intertwining factors.

To aid in analysis of multivariate networks, researchers have introduced visual analytic support, including network layouts considering semantic information, interactive simplification, and incorporation of coordinated views~\cite{kerren2014multivariate,nobre2019state}. 
Among others, utilizing network representation learning (NRL)~\cite{zhang2018network,huang2023va} is one promising approach as it can capture latent features of networks and highlight essential aspects that should be examined with visualizations. 
Existing visual analytics methods~\cite{fujiwara2022network,fujiwara2020visual,martins2012multidimensional,martins2017mvn,vandenelzen2016reducing,song2022interactive} utilize NRL methods to learn networks' \textit{general} representations---overall summaries of networks (e.g., variance of node degrees).
However, to identify associations, a general-purpose NRL is not effective as it is not particularly designed to capture latent features important for associations of interest.

On the other hand, as seen in the machine learning (ML) field, NRL using neural networks (NNs) can learn representations specific to an analysis focus by using appropriate loss functions~\cite{hamilton2017inductive}.
While these representations are suitable for fully automated analyses, such as network classification, it is often difficult for analysts to interpret the extracted features.
This is not preferable when analysts want to be involved in the analysis process and derive insights into associations.

To address the aforementioned problems, we introduce a visual analytics workflow that provides (1) network representations specific to the structural and semantic associations in multivariate networks as well as (2) interpretation supports for the analysis results.

To learn such representations, we first extract essential structural features by using an NRL method and then train an NN model that is designed to classify values of a user-selected attribute (e.g., high and low addiction levels to social media).
After training, the model generates latent features that are highly related to the selected attribute. 

For the interpretation of the obtained network representations, we employ dimensionality reduction (DR), interactive visualization, and composite variables~\cite{song2013composite}.
To help analysts assess the quality of the representations, we simplify the latent features with a linear DR method (specifically, linear discriminant analysis or LDA).
We then visualize the information of the simplified features in a 2D plot, where the distribution of networks or their elements (i.e., nodes and links) is shown along a latent direction that is related to the classification.
From this plot, analysts can judge which part of networks or elements (e.g., subjects with age 20--30) likely holds clearer associations with the selected attribute.
This analysis is almost infeasible if we rely only on the classification quality measures. 
In addition, we introduce a mechanism of composite variable construction to explain the meanings of the network representations.
The mechanism first suggests network structures and attributes highly related to the representations by utilizing a model-agnostic interpretation method, the SHAP~\cite{lundberg2017shap}.
Then, after analysts interactively select structures and attributes from the suggestion, an optimization method automatically generates a composite variable that resembles the network representation.
By examining this composite variable, analysts can understand the associations among the selected attributes and the other information.

To support effective analysis of the associations, we develop a new density scatterplot, named \textit{two-class density scatterplot}, that can depict a trend of data distribution as well as inform the mixture and separation of two classes even from numerous instances (e.g., 1000 instances).
Incorporating two-class density scatterplots, we develop an interactive interface that links visualizations designed to enhance the visual analytics workflow.
We further demonstrate the capabilities of the workflow and interface with three case studies using two real-world datasets on social media usage. 
Moreover, we validate the usability of the workflow through experts' feedback.

In summary, we consider our primary contributions to be:
\begin{itemize}
    \item a visual analytics workflow considering both generation and interpretation of network representations that are expressive for user-specified analysis targets;
    \item a workflow of composite variable construction that aids in attribute selection and attribute weight optimization for interpretations with an influence from multiple attributes;
    \item a two-class density scatterplot as a versatile visualization to review individual- and group-level data patterns with class information.
\end{itemize}
We provide source code related to the workflow and a demonstration video of interactive analysis~\cite{supp}.