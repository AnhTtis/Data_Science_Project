
\section{Case Studies}
\label{sec:case_studies}

\noindent
We demonstrate the effectiveness of our workflow and interactive visualizations with three case studies using the two datasets.
For the first two case studies on Dataset I, we show analyses on network nodes.
In the third case, we analyze the egocentric networks of Dataset II.


\subsection{Study 1: Associations with Score Levels}
\label{sec:cs1}

\noindent
We present a complete version of the analysis we have performed in \autoref{sec:repr_learning} and \autoref{sec:repr_interpretation}. Our analysis target is the identification of attributes that are highly related to college students' \texttt{scorelevel}.
We classify the bottom and top \texttt{scorelevel} groups with the five-layer MLP using 128, 64, and 64 NN nodes for hidden layers.
138 and 648 students are categorized into the bottom group (\texttt{Class\,0}, red color) and top group (\texttt{Class\,1}, blue color), respectively.
The prediction results show 1.0 accuracy for Step 2 but 0.98 for Step 3 (refer to \autoref{sec:nrl} for the reason why we accept such high accuracies). 

After completing Steps 1--3, we first confirm that the extracted 1D representation provides a reasonable separation between the two classes, as shown in \autoref{fig:ui}-b1.
Also, the distribution of instances/students shows a sufficient variety in their coordinates. 
Thus, we expect that the 1D representation is not extremely overfitted for this classification.

We then review attributes' associations with the 1D representation from the view in \autoref{fig:ui}-a.
Based on the ranked order of the attributes, contributing attributes to the score level differences are highly related to many of the Big Five personality traits: \texttt{open} (openness), \texttt{consci} (conscientiousness), \texttt{extra} (extraversion), \texttt{neuro} (neuroticism), and \texttt{agree} (agreeableness).
Other highly ranked attributes include
\texttt{netaddict} (internet addiction level) and \texttt{min\_happy} (the minimum of the friends' happiness levels).
On the other hand, most network centralities are not ranked high. 
As we examined in \autoref{sec:attrib_contrib_shap}, the top 5 attributes have clear positive or negative trends with the SHAP values (e.g., higher \texttt{open} value tends to have a more negative SHAP value).

Next, we construct a composite variable with the top 5 attributes while using Spearman's as a measure of dependence.
\autoref{fig:ui}-b2 visualizes the result.
The composite variable is expressed as: $y$ = -\,0.4\,\texttt{open} +\,0.7\,\texttt{consci} -\,0.3\,\texttt{netaddict} +\,0.3\,\texttt{min\_happy} +\,0.3\,\texttt{extra}.
Each weight's sign agrees with the corresponding attribute's positive or negative influence on the 1D representation, as observed in \autoref{fig:ui}-a.
Based on the magnitudes, we can see \texttt{consci} contributes most to the 1D representation.
Also, as discussed in \autoref{sec:comp_var_construction}, we observe that \texttt{open} and \texttt{extra}---personalities that might have some overlapped aspect---likely derive a new meaningful attribute by having opposite signs with each other.
When a composite variable is constructed with these two attributes, the resultant variable, -\,0.9\,\texttt{open} +\,0.4\,\texttt{extra}, shows a 0.17 correlation coefficient. 
Thus, having both openness and introversion tends to negatively impact their score levels.

\begin{figure}[tb]
    \centering
    \includegraphics[width=\linewidth,height=0.53\linewidth]{figures/case1.pdf}
    \caption{\CaptionCaseOne{}}
    \label{fig:case1}
\end{figure}

We investigate another question raised during the above analysis: why network centralities were not listed as highly contributed attributes even though
the composite variable, \mbox{-\,0.9}\,\texttt{open} +\,0.4\,\texttt{extra}, shows a clear contribution to the 1D representation.
Openness and extraversion could be related to network centralities such as node degree (e.g., extrovert students are likely to have more friends).
To examine the relationships to network structure, we highlight students with low (-\,0.9\,\texttt{open} +\,0.4\,\texttt{extra}) in the network layout visualization, as shown in \autoref{fig:case1}-a1.
While we see several selected students (colored yellow) are located on the outskirts of the network (i.e., fewer connections to others), we cannot find any clear structural pattern. 
When selecting students with high \texttt{extra} (\autoref{fig:case1}-a2), we cannot see a clear pattern either.
Thus, we can expect that openness and extraversion in real life were difficult to capture only from the connections on Facebook.  

\begin{figure*}[tb]
    \centering
    \includegraphics[width=\linewidth]{figures/case2_1.pdf}
    \caption{\CaptionCaseTwo{}}
    \label{fig:case2}
\end{figure*}

\begin{figure}[tb]
    \centering
    \includegraphics[width=\linewidth]{figures/case2_2.pdf}
    \caption{\CaptionCaseTwoCont{}}
    \label{fig:case2_2}
\end{figure}


We further see the UI suggests \texttt{neuro} shows strongly different distributions between the selected and
non-selected students.
From the distributions of \texttt{neuro} shown in \mbox{\autoref{fig:case1}-b1} and b2, the students with high (-\,0.9\,\texttt{open} +\,0.4\,\texttt{extra}) or high $\texttt{extra}$ tend to have lower \texttt{neuro} than others. 
Therefore, \texttt{neuro} is also interrelated to \texttt{open} and \texttt{extra}.
However, based on \autoref{fig:ui}-a, \texttt{neuro} has a smaller influence on the score level and does not show consistent positive or negative influences among students.
In fact, the inclusion of \texttt{neuro} into the composite variable of the top 5 attributes does not improve the correlation coefficient.

The above observations derive several reasonable insights: The student's conscientiousness is highly related to their score level; internet addiction has a negative association with the score, especially, for those who had bad scores (refer to \autoref{sec:attrib_contrib_shap}); if all friends have sufficient happiness, the score tends to be higher, and vice versa; and the openness and extraversion show a clear combinational effect and a high openness with a low extraversion has a more negative relationship to the score.  


\subsection{Study 2: Differences in Academic Units}

\noindent
From Dataset I, we review whether students from different majors have different structural and semantic characteristics.
To perform this comparison, we select each pair of all possible different majors (\texttt{agricultural}, \texttt{business}, \texttt{engineering}, \texttt{humanities}, \texttt{physical\,sciences}, and \texttt{social\,sciences}). 
We reviewed all pairs, but here we show one representative analysis: the comparison of \texttt{business} (\texttt{Class\,0}, red, 801 students) and \texttt{engineering} (\texttt{Class\,1}, blue, 287 students).
We use the same 85 attributes and MLP as Study~1 and perform classification of \texttt{business} and \texttt{engineering}.
The prediction results show 1.0 accuracy for both Steps 2--3. 

\autoref{fig:case2}-a shows the top 8 contributed attributes to the differences between \texttt{business} and \texttt{engineering}.
Unlike Study 1, this list of the top 8 includes network centralities, \texttt{eigenvector}, \texttt{betweenness}, and \texttt{total\_degree}.
To see the associations between these centralities and the 1D representation, we generate a composite variable with these three centralities, as shown in \autoref{fig:case2}-b.
The resultant composite variable shows a moderate correlation (Spearman's: 0.511). 
Also, we can see large weights with opposite signs for \texttt{eigenvector} and \texttt{total\_degree} (+0.6 and -0.7).
Unlike degree, eigenvector centrality considers the importance of links (discussed in \autoref{sec:preprocessing}). 
Taking a subtraction of degree from eigenvector centrality can emphasize this unique characteristic of eigenvector centrality.

As \autoref{fig:case2}-a depicts a clear positive influence of high \texttt{eigenvector}, we visualize the relationships between \texttt{eigenvector} and the 1D representation.
From the result shown in \autoref{fig:case2}-c, we first notice that \texttt{eigenvector} itself has a very small correlation coefficient (\mbox{-0.047}). 
At the same time, there are two clear subgroups in a group of \texttt{engineering} as we highlight one group in yellow.
On the other hand, as shown in \autoref{fig:case2}-d, a similar but more moderate separation can be seen in \texttt{total\_degree}. 
We then construct a new composite variable with only \texttt{eigenvector} and \texttt{total\_degree}.
The result shown in \autoref{fig:case2}-e informs a clear correlation of this new composite variable to the 1D representation (+0.460).
From the visualizations in \autoref{fig:case2}-c, d, and e, we now grasp how this informative composite variable for the difference of \texttt{business} and \texttt{engineering} is generated from the two centralities. 
This result emphasizes the importance of interpretation using a combination of multiple attributes. 
It also highlights the benefits of the design of a two-class density scatterplot, which aids in identifying denser clusters and inferring trends and correlations.

We are further interested in understanding the difference between low and high \texttt{eigenvector} groups in \texttt{engineering} students.
To perform this analysis, we select these two groups as a classification target.
\autoref{fig:case2_2}-a shows the derived top 8 contributing attributes, where red and blue colors are now used for low and high \texttt{eigenvector} groups. 
Most attributes show a clear separation between the groups.
Also, many of them are related to the student's personality or mentalities, such as \texttt{consci}, \texttt{happy}, and \texttt{agree}. 
Meanwhile, the listed attributes in \autoref{fig:case2_2}-a tend to correspond with the minimum values of their friends', as indicated with the prefix, ``\texttt{min\_}''.
As the high \texttt{eigenvector} group tends to have high \texttt{total\_degree} (see \autoref{fig:case2}-d), the students in the high \texttt{eigenvector} group are likely to have many friends, resulting in a higher chance to have at least one friend with low values for these attributes.
Therefore, we consider that this result is caused by inappropriate fittings specific to these two groups.
Also, as already observed in \autoref{fig:case2}-d, the two groups have clearly different values in \texttt{total\_degree}.
As the inclusion of these attributes is problematic, we remove them during the feature extraction and rerun the remaining steps.
\autoref{fig:case2_2}-b shows the resultant top 8 contributing attributes.
By reviewing the values of each attribute (i.e., $y$-coordinates in \autoref{fig:case2_2}-b), we infer that friends of the high \texttt{eigenvector} group tend to have lower happiness levels (\texttt{happy}), worse health status (\texttt{health}), and are more addicted to Facebook (\texttt{FBaddict}), while they spend shorter time on the Internet (\texttt{nettime}).
These results highlight the importance of considering compounded factors in network analysis (e.g., more social media friends but the friends are less happy; more addicted to social media but shorter time spent on the Internet). 

From this study, we observe the difference between \texttt{business} and \texttt{engineering} students in the composite variable constructed with the network centralities. 
As the composite variable emphasizes the difference between the eigenvector and total degree centralities, \texttt{engineering} students' connections tend to be more particular to influential students in Facebook. 
Also, we demonstrate the case of identifying and resolving potential inappropriate overfitting from the visualized results.


\subsection{Study 3: Factors Promoting New Connections}
\label{sec:cs3}

\noindent
From Dataset II containing 345 egocentric networks of adults, we review contributing factors to promoting their indirect contacts (i.e., 2-hop neighbors) to direct contacts (i.e., 1-hop neighbors).
For each network, we first compute the ratio of the number of indirect contacts transformed into direct contacts during the study period.
We call this ratio the transformation rate and denote it as $\TransRate$.
We then group respondents (i.e., ego nodes) with conditions of $\TransRate\,{=}\,0$ (i.e., no transformation) as \texttt{Class\,0}  and $\TransRate\,{\geq}\,0.2$ (i.e., high transformation rate) as \texttt{Class\,1}.
\texttt{Classes\,0} and \texttt{1} consist of 56 and 136 respondents, respectively.
We then classify these classes, resulting in accuracies of 1.0 and 0.82 for Steps 2 and 3.

\begin{figure}[tb]
    \centering
    \includegraphics[width=\linewidth]{figures/case3.pdf}
    \caption{\CaptionCaseThree{}}
    \label{fig:case3}
\end{figure}

\autoref{fig:case3}-a shows the top 10 contributed attributes to the 1D representation. 
We see several clear trends between attributes' values and the SHAP values.
For example, \texttt{age*fbday} shows an increasing trend (i.e., larger \texttt{age*fbday} has a more positive impact on the transformation). 
\texttt{age*fbday} is the predefined composite variable by the existing study~\cite{lee2022indirect}.
This attribute is made by multiplying the ego's \texttt{age} and total days of Facebook use, \texttt{fbday}.
Due to this definition of \texttt{age*fbday}, all top 3 contributed attributes, \texttt{age*fbday}, \texttt{fbday}, \texttt{age}, are likely to interrelate with each other. 
Then, we construct a composite variable with these attributes.
As shown in \autoref{fig:case3}-b, the constructed composite variable, 0.6\,\texttt{age*fbday} -\,0.6\,\texttt{fbday} -\,0.5\,\texttt{age}, presents a weak correlation (Pearson's: 0.282). 
By investigating three attributes individually, we observe that they show a much weaker correlation to the 1D representation---\texttt{age*fbday}: 0.064, \texttt{fbday}: 0.052, \texttt{age}: 0.123. 
Thus, the composite variable seems to find more meaningful information by disentangling complex relationships among the three attributes.
We further construct a new composite variable by adding other highly ranked attributes, \texttt{tags/mon} (the average times of being tagged per month) and \texttt{female} (whether ego's gender is female or not).
As shown in \autoref{fig:case3}-c, the new composite variable has a significantly stronger correlation than the previous one (0.438~vs.~0.282).

For verification, we compare our findings with the existing study~\cite{lee2022indirect}.
The existing study employed a mixed-effect model to capture attributes' contributions to the transformation of indirect contacts. 
The top 5 contributed attributes suggested by their model are \texttt{female}, \texttt{extra}, \texttt{comments/mon} (the average number of comments made per month), \texttt{fbday}, and \texttt{age*fbday}. 
First of all, three attributes, \texttt{age*fbday}, \texttt{fbday}, and \texttt{female}, are seen in both their results and ours. 
Also, \texttt{tags/mon} and \texttt{comments/mon} should have semantic similarities with each other based on their meanings. 
In fact, by replacing \texttt{tags/mon} with \texttt{comments/mon}, we obtain the composite variable with 0.382 Pearson's correlation coefficient, which is considerably close to the original (Pearson's: 0.438).
These observations indicate that our NN-based model successfully captured similar information to the mixed-effect model.
The unique strength of our approach is in its ability to convey inter-relationships of attributes with their weights in the composite variable. 
While the existing work~\cite{lee2022indirect} crafted \texttt{age*fbday}, our study suggests (0.6\,\texttt{age*fbday} -\,0.6\,\texttt{fbday} -\,0.5\,\texttt{age}) or a more simplified version, (\texttt{age*fbday} -\,\texttt{fbday} -\,\texttt{age}), as a potential composite variable that better capture the influence on the transformation than \texttt{age*fbday}.

