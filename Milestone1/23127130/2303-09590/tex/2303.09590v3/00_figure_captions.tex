\newcommand{\CaptionWorkflow}{The visual analytics workflow for investigating associations in multivariate networks, where Steps 1--4 are executed with a script for machine learning and Steps 5--6 are conducted interactively with our UI.}

\newcommand{\CaptionUI}{The visual interface for facilitating interactive analysis using the workflow in \autoref{fig:workflow}. 
(a) Relating to Step 4, this view visualizes input attributes' contributions to the 1D network representation. 
(b) Composite variables generated in Step 5 are shown as a list of scatterplots, where the 1D network representation and composite variable correspond to $x$- and $y$-axes, respectively (see b2). 
As shown in b1, when a composite variable is not generated yet, a swarm-plot-like visualization presents the 1D network representation generated through Step 3 to help assess its quality.
For a and b, we employ our two-class density scatterplots.
(c) A node-link diagram and (d) a set of histograms inform the network structure and attribute distributions.
(e) Other auxiliary information is displayed, including the prediction accuracy of NNs trained for Step 2.}

\newcommand{\CaptionSHAP}{Comparison of SHAP value visualizations: (a) the default plot in the SHAP package~\cite{shap_library}; (b) the plot colored by class; and (c) our design.}

\newcommand{\CaptionScatterplots}{The comparison of scatterplot designs: (a) scatterplot with colored classes, (b) density scatterplot, (c,d) scatterplots encoding the total density and the ratio of each class's density with two different bivariate colormaps. (d) is our final design for a two-class density scatterplot.}

\newcommand{\CaptionCaseOne}{Study 1: Networks after selecting students with (a1) low (--\,0.9\,\texttt{open} +\,0.4\,\texttt{extra}) and (a2) high \texttt{extra}; distributions of \texttt{neuro} (b1, b2) corresponding to the selections of (a1, a2).}

\newcommand{\CaptionCaseTwo}{Study 2-1: The comparison of \texttt{business} and \texttt{engineering} students: (a) the SHAP values and (b-e) composite variables constructed from network centralities. Combining \texttt{eigenvector} and \texttt{total\_degree} significantly increases the correlation.}

\newcommand{\CaptionCaseTwoCont}{Study 2-2: The comparison of the high and low \texttt{eigenvector} groups of \texttt{engineering} students: the SHAP value visualizations (a) before and (b) after improving the feature extraction setting.}

\newcommand{\CaptionCaseThree}{Study 3: (a) The top contributing attributes and composite variables constructed with (b) the top 3 and (c) top 5 attributes.}