\section{Discussion}

\noindent
Through the case studies and the expert interview, we discussed the efficacy of our workflow and interactive visualizations. 
In Supplementary Materials~\cite{supp}, we further demonstrate the effectiveness of the constructed composite variables for conventional analyses, such as statistical hypothesis testing.
Besides these evaluations, to allow readers to test and evaluate our workflow and UI, we provide related source code as well as processed data~\cite{supp} from a publicly available dataset of faculty networks~\cite{wapman2022quantifying}.
Here, we provide additional discussions on our designs.

\textbf{Applicability to various data types.} 
We have designed our workflow for multivariate networks.
As multivariate networks have both high-dimensional and relational characteristics, by their nature, the workflow is even applicable to high-dimensional data and univariate network data.
For example, we can analyze high-dimensional data by skipping the extraction of network centrality-related attributes and the visualization of networks.
In addition, our workflow design can potentially adapt to advanced models of networks such as those containing meta-nodes and hyperedges. 
For example, as long as meta-nodes have the same set of node attributes as simple nodes in a network, we can still apply our workflow as is. 
Our workflow precomputes node- and link-related features before the training utilizing NNs; thus, we can deal with hyperedges during this preprocessing step (e.g., including hyperedges when applying relational functions in DeepGL).


\textbf{Other potential algorithm designs.}
While we employ the NNs, DR, and composite variable construction to support our analysis target, there are other potential designs.
One common way to understand the associations among a target and other attributes is applying a decision tree (DT)~\cite{brand2021uncovering,li2021visual}. 
When compared with a DT-based analysis, our design provides two main strengths in the interpretation step: simplicity and informativity.
A DT provides a set of attribute ranges that is useful to classify a target attribute. 
However, the number of ranges would be easily overwhelming when analyzing networks with many attributes. 
Also, from the DT results, it is difficult to numerically assess the influences from multiple attributes, such as those seen in the composite variables we have constructed for the case studies.

Another possible design is enabling the construction of more complicated composite variables, such as those with multiplications and logarithms of attributes.
Although this would be more effective to analyze complex relationships among attributes (e.g., \texttt{age*fbday} in \autoref{sec:cs3}), this construction requires much more complicated optimizations than ours.
One potential way to perform such advanced constructions while avoiding excessive computation is incorporating analysts' knowledge more actively (e.g., predetermining a part of a composite variable). 
We plan to investigate this direction in future research.

\textbf{Usability of two-class density scatterplots.}
We developed two-class density scatterplots to depict various patterns (e.g., distributions of class instances, trends, clusters, and outliers) in a single visualization.
The use of this scatterplot is not limited to the targeted analyses in this work. 
As binary classifications or group comparisons are frequently performed for ML and visual analytics, we believe two-class density scatterplots can contribute to these fields. 
As future research, we plan to evaluate our scatterplot design by conducting a comprehensive user study and designing new quality measures for multiclass scatterplots that convey both aggregated- and instance-level information (e.g., splatterplots~\cite{mayprga2013splatterplots} and winglets~\cite{lu2020winglets}).
This evaluation would also identify the two-class density scatterplot's shortcomings for further improvements.
There is one clear limitation: allowing visualization of only two classes. 
One potential way to support three or more classes is taking similar approaches to the multiclass geographical maps by Jo et al.~\cite{jo2018declarative}.
For example, we can first partition a 2D space based on the changes in the distribution of class instances and then display the distribution in each partitioned region with a bar-chart glyph.
We expect that this design can deal with several classes (e.g., five classes); however, this partition-based approach would not be suitable to reveal outliers and clusters.
Thus, we would like to investigate an extension to three or more classes in the future.

