
\section{Expert Feedback}

\noindent
To further validate our workflow’s usability, we conducted a focus group with five experts in social network studies.
The first expert (E1) is a distinguished researcher in an institute of sociology who collected Datasets I and II and also conducted research on these datasets with different focuses from ours. 
The second expert (E2) is an assistant professor in a department of sociology who formulated the guidelines for collecting the datasets.
The other three experts are researchers in institutes of sociology (E3) and statistical science (E4, E5) who also studied the same datasets.
The focus group was conducted through a video conference setup, where our workflow methods, visualization designs, and the three case studies were presented using an interactive demo combined with static screenshots.
Then, the five experts provided their comments as qualitative feedback on our workflow.

All the experts agreed that our workflow can support a wide range of their analysis targets as well as derive insights with more intuitive interpretations when compared to their current approach.
For example, E1 commented, ``The results [seen in the composite variables] are much easier to understand and intuitive than the outputs from the statistical models used in our previous research.''
All of them showed strong interest in insights that can be derived from the signed weights in composite variables.  
On the other hand, E1 noted a potential limitation of the datasets and analyses: ``The distribution of the respondents might affect the results related to the internet addiction because the students who granted the use of their data tended to be more addicted to the internet usage than others.''
E2 suggested better input attributes for the analyses, and we improved our case studies, accordingly.   

There were several discussions on our workflow usage and design.
E3 asked, ``What if there are attributes with a dominant influence on the result? What if the 1D representation does not have clear separation?'' 
For the former, the visualization shown in \autoref{fig:ui}-a can be used to identify dominant attributes and exclude them from inputs if such attributes exist.
Similarly, for the latter case, the visualization shown in \autoref{fig:ui}-b1 is useful to review the quality of the separation.
Then, based on the quality, analysts can take actions, such as adding more input attributes, reducing the ranges of the two ends of the output attribute, and improving the NN model.
E4 raised a question on our post-hoc simplification of network representations: ``Why not only use a neural network by designing the last layer with a linear activation function, instead of using LDA?''
Our answer is given in \autoref{sec:repr_simplification}. 
E5 asked, ``Why not use other centralities but these three?''
We selected degree, eigenvector, and betweenness centralities as a set of the most fundamental measures of structural characteristics.
Our workflow is flexible to employ any other centralities based on analysts' interests.

Finally, there were comments on our two-class density scatterplot design. 
E1 and E2 noted its analytical usability: ``The design of the two-class density scatterplot facilitates the identification of dense areas, providing better support for observing data distribution and correlations.''
E2 further added: ``We often wasted a considerable amount of time in validating hypotheses that are formed from misleading visual patterns of correlations and class separations in conventional scatterplots.''
E1 also commented: ``The design of the two-class density scatterplot can be particularly useful for large-scale data and has contributions to statistical science research.''