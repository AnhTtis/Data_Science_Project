
Multivariate networks~\cite{kerren2014multivariate} consisting of both topological and semantic information can model complex relations between real-world entities.
One common analysis task performed on multivariate networks is to understand associations among structural and semantic characteristics of networks~\cite{kerren2014multivariate,atzmueller2021mining,chetty2022social}.
For example, from social media usage data, analysts would want to see how likely each possible combination of individual characteristics (e.g., age, gender, and extraversion) and their friendship structures is related to their addiction level to social media. 
Such analysis can be very complicated when the associations underlie intertwining social facts.

To aid in analysis of multivariate networks, researchers have introduced visual analytic support, including network layouts considering semantic information, interactive simplification, and incorporation of coordinated views~\cite{kerren2014multivariate,nobre2019state}. 
Among others, utilizing network representation learning (NRL)~\cite{zhang2018network} is one promising approach as it can capture latent features of multivariate networks and highlight essential aspects that should be examined with visualizations. 
Existing visual analytics methods~\cite{fujiwara2020visual,fujiwara2022network,martins2012multidimensional,martins2017mvn,vandenelzen2016reducing,song2022interactive} utilize NRL methods that learn networks' \textit{general} representations---consisting of a set of features that summarizes overall network information without a certain analysis focus.
However, when analysts have a specific interest (e.g., the friendship structure's influences on the addiction level to social media), a general-purpose NRL would not be effective as it is not particularly designed for such a case, and thus may fail to capture features important for the corresponding interest.

On the other hand, as seen in the machine learning field, NRL using neural networks (NNs) can learn representations specific for an analysis focus by using appropriate loss functions~\cite{hamilton2017inductive}.
While these representations are suitable for fully automated analyses, such as network classification, it is often difficult for analysts to interpret the extracted features.
This is not preferable for the case where the analysts want to be involved in and derive insights during the analysis process.

To address the aforementioned two-folded problems, we introduce a visual analytics workflow that provides network representations specific to a task of uncovering the structural and semantic associations from multivariate networks as well as an interpretation support for the analysis results.
For NRL, we train an NN-based model that is designed to classify values of a user-selected attribute (e.g., whether the addiction level to social media is high or low).
Through the training, the model generates latent features that are highly related to the selected attribute. 

For the interpretation of the obtained network representations, we employ dimensionality reduction (DR), interactive visualization, and composite variables~\cite{song2013composite}.
To help analysts assess the quality of the representations, we simplify the latent features with a linear DR method (specifically, linear discriminant analysis or LDA).
We then visualize the information of the simplified features in a 2D plot, where the distribution of networks or their elements (i.e., nodes and links) is shown along the direction related to the classification.
From this plot, analysts can judge which part of networks or elements (e.g., subjects with age 20--30) likely holds more clear associations with the selected attribute, which is almost infeasible when only relying on the classification quality measures. 
In addition, we introduce a mechanism of composite variable construction to explain the meanings of the network representations.
The mechanism first suggests network structures and attributes highly related to the representations by utilizing a model-agnostic interpretation method, the SHAP~\cite{lundberg2017shap}.
Then, after allowing analysts to interactively select the structures and attributes from the suggestion, an optimization method automatically generates a composite variable that resembles the representation extracted by the NNs.
By examining this composite variable, analysts can understand the relationships among the selected attributes and the other related information.

To effectively analyze the associations between the composite variable and network representation, we develop a new density scatterplot, named \textit{two-class density scatterplot}, that can clearly depict a trend of the data distribution as well as inform the mixture and separation of two classes.
Incorporating the two-class density scatterplot, we develop an interactive interface linking all the visualizations to support the above visual analytics workflow, and demonstrate the capabilities of the workflow with three case studies using two real-world datasets on social media usage. 
Moreover, we conduct an expert interview to validate the usability of the workflow.

In summary, we consider our primary contributions to be:
\begin{compactitem}
    \item a visual analytics workflow considering both generation and interpretation of network representations that are expressive for user-specified analysis targets;
    \item a mechanism of composite variable construction, where we support attribute selection and attribute weight optimization to assist the interpretation with a combinational influence from multiple attributes; and
    \item a two-class density scatterplot as a versatile visualization to review individual- and group-level data patterns with binary class information.
\end{compactitem}
