
\section{Discussion}

Through the case studies and the expert interview, we have shown the effectiveness of our workflow and interactive visualizations. 
Here, we provide additional discussions on our designs.

\textbf{Applicability to various data types.} 
We have designed our workflow for multivariate networks.
As multivariate networks have both high-dimensional and relational characteristics, by its nature, the workflow can be applicable to high-dimensional data and univariate network data.
For example, we can analyze high-dimensional data by skipping the computation of the network centralities and $k$-hop neighbor statistics and the visualization of networks.
In addition, our workflow design can potentially adapt to advanced models of networks such as those containing meta-nodes and hyperedges. 
For example, as long as meta-nodes have the same set of node attributes as simple nodes in a network, we can still apply the same processes in our workflow. 
Also, our workflow precomputes node and link-related features before the training utilizing NNs; thus, we can deal with hyperedges during this preprocessing step (e.g., in DeepGL, by including hyperedges as links that should be considered for the selection of neighbor nodes).


\textbf{Other potential algorithm designs.}
While we employ the NNs, DR, and composite variable construction to support the aforementioned analysis target, there are other potential designs.
One common way to understand the associations among a target and other attributes is using a decision tree (DT)~\cite{brand2021uncovering,li2021visual}. 
When compared with a DT-based analysis, our design provides two main strengths in the interpretation step: simplicity and informativity.
A DT provides a set of conditions of attributes' ranges that can classify a target attribute. 
However, the number of conditions would be easily overwhelming when analyzing networks with many attributes. 
Also, from the DT's result, it is difficult to numerically identify the combinational influence from multiple attributes, such as those seen in the composite variables we have constructed for the case studies.

Another possible design is allowing for the construction of more complicated composite variables, for example, by allowing to take multiplications and/or logarithms of attributes.
Although this would be more effective when analyzing complex relationships among attributes (e.g., \texttt{age} and \texttt{fbday} in \autoref{sec:cs3}), this construction requires much more complicated optimizations than ours.
One potential way to perform such advanced constructions while avoiding excessive computation is incorporating analysts' knowledge by enabling them to interactively build a part of the composite variable (e.g., inputting an equation template for the composite variable). 
We plan to investigate this direction in future research.

\textbf{Usability of two-class density scatterplots.}
We have developed the two-class density scatterplot to examine various important patterns (e.g., distributions of class instances, trends, clusters, and outliers) with a single visualization.
The usage of this scatterplot is not limited to the targeted analyses in this work. 
As binary classifications or group comparisons are frequently performed for machine learning and visual analytics, we believe the two-class density scatterplot can widely contribute to these fields. 
As future research, we would like to conduct a comprehensive user study to evaluate the effectiveness of our scatterplot design as well as to identify its shortcomings for further improvements.

As its name indicates, the two-class density scatterplot is designed only for visualizing two classes. 
One potential design for three or more classes is taking similar approaches to the multiclass (geographical) maps by Jo et al.~\cite{jo2018declarative}.
For example, we can first partition a 2D space by the change in the class distribution and then display a distribution of classes in each partitioned region with a bar-chart glyph.
We expect that this design can deal with several classes (e.g., five classes); however, this partition-based approach would not be suitable to reveal outliers and clusters.
Thus, we also would like to investigate a better design for three or more classes in the future.

