\newcommand{\CaptionWorkflow}{The visual analytics workflow for understanding the relations in multivariate networks.}

\newcommand{\CaptionUI}{The visual interface for facilitating interactive analysis using the workflow in \autoref{fig:workflow}. 
(a) Relating to Step 4, this view visualizes the input attributes' contributions to the 1D network representation. 
(b) The composite variables generated in Step 5 are shown as a list of scatterplots, where the 1D network representation and the composite variable correspond to $x$- and $y$-axes, respectively (see b2). 
As shown in b1, when a composite variable is not generated yet, a swarm-plot-like visualization presents the 1D network representation generated through Step 3 to help assess its quality.
For a and b, we employ our two-class density scatterplots to depict density and class distribution information simultaneously.
As shown in the fan-shape colormap in e, color lightness and hue are used to encode the density and the ratio of each class's density at each position, respectively.
For example, in b2, we can see a high-density area mainly consisting of \texttt{Class\,1} (blue) at the right side as well as a clear separation between two classes (e.g., \texttt{Class\,0} (red) is mostly located at the left side).
(c) A node-link diagram and (d) a set of histograms inform the network structure and attribute distributions, respectively.
(e) Other auxiliary information is displayed, including the prediction accuracy of NNs trained for Step 2.}

\newcommand{\CaptionSHAP}{Design comparison of SHAP value visualizations. Across all visualizations, each row is allocated for one attribute and $x$-coordinates represent SHAP values; however, attribute values and/or class labels are encoded differently. (a) The default plot provided in the SHAP Python package~\cite{shap_library} encodes attribute values with a divergent colormap while using $y$-direction to pile up dots (similar to the swarm-plot-like visualization in \autoref{fig:ui}-b1). (b) The color of each dot encodes a class label. (c) Our design uses $y$-coordinates and colors to represent attribute values and class labels, respectively.}

\newcommand{\CaptionScatterplots}{The comparison of scatterplot designs: (a) scatterplot with colored classes, (b) density scatterplot, (c,d) scatterplots encoding the density and the ratio of each class's density with two different bivariate colormaps. (d) is our final design for a two-class density scatterplot.}

\newcommand{\CaptionCaseOne}{Study 1: The network layouts after selecting students with (a1) high ($-0.9 \mathrm(\texttt{open}) {+} 0.4 \mathrm(\texttt{extra})$) and (a2) high \texttt{extra}. The distributions of \texttt{neuro} (b1, b2) corresponding to the selections of (a1, a2).}

\newcommand{\CaptionCaseTwo}{Study 2: Visualizations for (a-e) the comparison of business and engineering students and (f) the comparison of engineering students with high and low eigenvector centralities.}

\newcommand{\CaptionCaseThree}{Study 3: (a) The top contributing attributes and composite variables constructed with (b) the top-3 and (c) top-5 attributes.}