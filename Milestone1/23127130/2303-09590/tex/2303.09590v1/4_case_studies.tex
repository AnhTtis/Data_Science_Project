
\section{Case Studies}
\label{sec:case_studies}

We demonstrate the effectiveness of our workflow and interactive visualizations with three case studies using the two datasets.
For the first two case studies, using Dataset I, we show analyses on network nodes.
In the third case, we analyze egocentric networks derived from Dataset II.


\subsection{Study 1: Associations with Score Levels}
\label{sec:cs1}

Here we describe a complete version of the analysis we have performed in \autoref{sec:repr_learning} and \autoref{sec:repr_interpretation}, where we target the identification of attributes that are highly related to college students' \texttt{scorelevel} from Dataset I, which consists of 1886 nodes/students, 64156 links, and 85 node attributes after the preprocessing (see \autoref{sec:datasets} and \autoref{sec:nrl}).
We classify the bottom and top \texttt{scorelevel} groups with the five-layer MLP using 128, 64, and 64 NN nodes for the first, second, and third hidden layers, respectively.
There are 138 students in the bottom group (\texttt{Class\,0}, red color) and 648 students in the top group (\texttt{Class\,1}, blue color). 
The prediction results show 1.0 accuracy for Step 2 but 0.98 for Step 3 (refer to \autoref{sec:nrl} for the reason for accepting such a high accuracy in Step 2). 
\autoref{fig:ui} shows the visualized results in the UI.

From the visualization of the 1D representation, as shown in \autoref{fig:ui}-b1, we first confirm that the learning result via Steps 1--3 provides a reasonable separation between two classes.
Also, from the distribution of instances/students, we expect that the 1D representation is not extremely overfitted for this classification (i.e., there is a sufficient variety for the coordinates).

We then review the attributes' associations with the 1D representation from the view in \autoref{fig:ui}-a.
Based on the ranked order of the attributes, rather than network centralities, we see that attributes more contributing to the score level differences are highly related to many of the Big Five personality traits (i.e., \texttt{open} (openness), \texttt{consci} (conscientiousness), \texttt{extra} (extraversion), \texttt{neuro} (neuroticism), and \texttt{agree} (agreeableness)) as well as certain statuses, such as
\texttt{netaddict} (internet addiction level) and \texttt{min\_happy} (the minimum of the friends' happiness levels). 
As we have already examined in \autoref{sec:attrib_contrib_shap}, the top-5 attributes have clear positive or negative trends with the SHAP values (e.g., higher \texttt{open} tends to have a more negative SHAP value).

We construct a composite variable with these top-5 attributes to maximize Spearman's correlation coefficient with the 1D representation, resulting in the visualization shown in \autoref{fig:ui}-b2.
From the equation of the composite variable, $y{=}-0.4 \mathrm(\texttt{open}) + 0.7 \mathrm(\texttt{consci}) - 0.3 \mathrm(\texttt{netaddict}) + 0.3 \mathrm(\texttt{min\_happy}) + 0.3 \mathrm(\texttt{extra})$, we see that the attributes' signs of weights match with their positive or negative influence on the 1D representation, as seen in \autoref{fig:ui}-a.
Based on the magnitudes, we can see \texttt{consci} contributes most to the 1D representation.
Also, as discussed in \autoref{sec:comp_var_construction}, we observe that \texttt{open} and \texttt{extra}---personalities that might have some overlapped aspect---likely derive a new meaningful attribute by having opposite signs with each other.
More specifically, a composite variable, $y=-0.9 \mathrm(\texttt{open}) + 0.4 \mathrm(\texttt{extra})$, shows 0.170 correlation coefficient. 
This indicates that having both openness and introversion tends to have clearer negative impacts on their score levels.

\begin{figure}[tb]
    \centering
    \includegraphics[width=\linewidth]{figures/case1.pdf}
    \caption{\CaptionCaseOne{}}
    \label{fig:case1}
\end{figure}

Even though similar information with the attribute combining \texttt{open} and \texttt{extra} could be captured by the network centralities such as node degree (e.g., the students with high \texttt{extra} could have more friends), as mentioned, the centralities do not have clear influences on the 1D representation.
To examine the relationships between the attribute and network structure, we select students who have low ($-0.9 \mathrm(\texttt{open}) + 0.4 \mathrm(\texttt{extra})$) from the UI and update the network layout visualization, as shown in \autoref{fig:case1}-a1.
While we see several selected students (colored yellow) are located on the outskirt of the network (i.e., fewer connections to others), we cannot find any clear structural pattern. 
When selecting students with high \texttt{extra}, as shown in \autoref{fig:case1}-a2, we cannot see a clear pattern either.
Thus, we can say that openness and extraversion in a real life are difficult to capture only from the connections on Facebook.  
The ordered histograms by their KS scores show that selected and non-selected students have strongly different distributions for \texttt{neuro}.
From the distributions of \texttt{neuro} shown in \autoref{fig:case1}-b1, b2, the students with high ($-0.9 \mathrm(\texttt{open}) + 0.4 \mathrm(\texttt{extra})$) or high $\texttt{extra}$ tend to have lower \texttt{neuro} than others. 
Thus, \texttt{neuro} also seems to be inter-related to \texttt{open} and \texttt{extra}; however, based on \autoref{fig:ui}-a, \texttt{neuro} has a smaller influence on the score level and does not show consistent positive or negative influences among students.
In fact, we see that the inclusion of \texttt{neuro} into the composite variable of the top-5 attributes does not improve its correlation coefficient.

The above observations derive several reasonable insights: The student's conscientiousness is highly related to their class score; internet addiction has a negative association with the score, especially, for those who had bad scores (refer to \autoref{sec:attrib_contrib_shap}); if all friends have sufficient happiness, the score tends to be higher, and vice versa; and the openness and extraversion show a clear combinational effect and a high openness with a low extraversion has a more negative relationship to the score.  

\begin{figure*}[tb]
    \centering
    \includegraphics[width=\linewidth]{figures/case2.pdf}
    \caption{\CaptionCaseTwo{}}
    \label{fig:case2}
\end{figure*}


\subsection{Study 2: Differences by Academic Units}

From Dataset I, by changing the classification target of Step 2, we review whether or not students from different majors have different structural and semantic characteristics.
To perform this comparison, we select each pair of six different majors recorded in the data (\texttt{agricultural}, \texttt{business}, \texttt{engineering}, \texttt{humanities}, \texttt{physical\,sciences}, and \texttt{social\,sciences}). 
\sloppy{While we have compared all the majors, we show one representative analysis example, where we compare students who major in \texttt{business} (\texttt{Class\,0}, red, 801 students) and \texttt{engineering} (\texttt{Class\,1}, blue, 287 students).}
We use the same 85 attributes and MLP as in Study 1.
The prediction results show 1.0 accuracy for both Steps 2--3. 

\autoref{fig:case2}-a shows the top-8 attributes contributing to the differences between \texttt{business} and \texttt{engineering}.
Unlike Study 1, we can see attributes related to the network structure such as \texttt{eigenvector}, \texttt{betweenness}, and \texttt{total\_degree}. 
To review the relationships between these centralities and the 1D representation, we generate a composite variable with these three centralities.
As shown in \autoref{fig:case2}-b, the resultant composite variable shows a moderate correlation (Spearman's: 0.511). 
Also, we can see large weights with opposite signs for \texttt{eigenvector} and \texttt{total\_degree}: +0.6 and -0.7, respectively.
As described in \autoref{sec:nrl}, unlike degree centrality, eigenvector centrality considers the importance of links. 
Therefore, taking a subtraction of node degree from eigenvector centrality emphasizes the characteristics unique to eigenvector centrality.

As we can see the strong and clear positive influence of high \texttt{eigenvector} from \autoref{fig:case2}-a, we visualize the relationships between \texttt{eigenvector} and the 1D representation, as shown in \autoref{fig:case2}-c. 
We first notice that \texttt{eigenvector} itself has a very small correlation coefficient (-0.047). 
But, we also identify two clear subgroups in a group of \texttt{engineering} as we highlight one group in yellow.
Also, a similar but more moderate separation can be seen in \texttt{total\_degree} (see \autoref{fig:case2}-d). 
We then make a new composite variable using only \texttt{eigenvector} and \texttt{total\_degree}.
The result shown in \autoref{fig:case2}-e informs a clear correlation of the composite variable to the 1D representation (0.460).
By looking at the visualizations in \autoref{fig:case2}-c, d, and e, we can observe how this composite variable informative for the difference of \texttt{business} and \texttt{engineering} is generated from the two attributes. 
Again, finding this type of information is difficult when only reviewing an influence from a single attribute.

We are also interested in the difference between low and high \texttt{eigenvector} groups in \texttt{engineering} students and select these two groups as a classification target.
\autoref{fig:case2}-f shows the top-8 contributing attributes for this classification, where red and blue colors are now used for low and high \texttt{eigenvector} groups. 
We see that most attributes show a clear separation between the groups and many of them are related to the student's personality or mentalities, such as \texttt{consci}, \texttt{happy}, and \texttt{agree}. 
However, we also see that the listed attributes in \autoref{fig:case2}-f tend to correspond with the minimum values of their friends', as indicated with the prefix, ``\texttt{min\_}''.
As the high \texttt{eigenvector} group tends to have high \texttt{total\_degree} as well (see \autoref{fig:case2}-d), the students in this group have a higher chance to have some friends with low values for any of the survey attributes.
Therefore, this result is likely from inappropriate fittings specific to these two groups, and we should further examine their differences, for example, by removing those 1-hop neighbor statistics taking the minimum or maximum. 

From this study, we observe the difference between students who major in \texttt{business} and \texttt{engineering} in the composite variable constructed with the network centralities. 
As the composite variable emphasizes the difference between the eigenvector and total degree centralities, we can say that \texttt{engineering} students tend to have connections more specific to influential students in Facebook. 
Also, we demonstrate the case of identifying potential inappropriate overfitting from the visualized results.


\subsection{Study 3: Factors Promoting New Connections}
\label{sec:cs3}


Using Dataset II, we generated multivariate egocentric networks by adding 1- and 2-hop neighbors of the survey respondent based on the contact records (i.e., 1-hop: direct contacts, 2-hop: indirect contacts) as well as network-level attributes, such as the ego node's personality, Facebook usage statistics, and network measures (e.g., clustering coefficient).
Also, we have an attribute informing whether or not an indirect contact became a direct contact from January 1, 2015 to June 30, 2017.
The resultant egocentric networks consist of 345 ego nodes, 3339 1-hop neighbors, and 19917 2-hop neighbors.

To identify contributing attributes to the transformation from indirect contacts to direct contacts, we
first compute the transformation rate $\TransRate$ for each network (i.e., the ratio of the number of indirect contacts transformed to direct contacts during the study period).
We then respectively select respondents with conditions of $\TransRate {=} 0$ (i.e., no transformation) and $\TransRate {\geq} 0.2$ (i.e., high transformation rate) as \texttt{Class\,0} (red, 56 respondents) and \texttt{Class\,1} (blue, 136 respondents).
The accuracies for Steps 2 and 3 are 1.0 and 0.82, respectively.

\begin{figure}[tb]
    \centering
    \ifarxiv
      \includegraphics[width=\linewidth]{figures/case3.pdf}
    \else
      \includegraphics[width=0.7\linewidth]{figures/case3.pdf}
    \fi
    \caption{\CaptionCaseThree{}}
    \label{fig:case3}
\end{figure}

\autoref{fig:case3}-a shows the top-10 attributes contributing to the 1D representation. 
Similar to the other studies, we see several clear trends between each attribute's values and the SHAP values.
For example, \texttt{age*fbday} (multiplication of the ego's \texttt{age} and total days of Facebook use, \texttt{fbday})---the predefined composite variable in the existing study~\cite{lee2022indirect}---shows an increasing trend (i.e., larger \texttt{age*fbday} has a more positive impact on the transformation). 
As all the top-3 attributes, \texttt{age*fbday}, \texttt{fbday}, \texttt{age}, are likely inter-related, we construct a composite variable with these attributes.
The result is shown in \autoref{fig:case3}-b, where the composite variable, $y{=}0.6 \mathrm(\texttt{age*fbday}) {-} 0.6 \mathrm(\texttt{fbday}) {-} 0.5 \mathrm(\texttt{age})$, has a weak correlation (Pearson's: 0.282). 
When we investigate three attributes individually, we notice that they have much lower correlation coefficients to the 1D representation than this composite variable (\texttt{age*fbday}: 0.064, \texttt{fbday}: 0.052, \texttt{age}: 0.123). 
Thus, the composite variable seems to find meaningful information by disentangling the complicated relationships among the three attributes. 
We further construct a composite variable by adding \texttt{tags/mon} (the average times of being tagged per month) and \texttt{female} (whether ego's gender is female or not).
As shown in \autoref{fig:case3}-c, the resultant composite variable using these attributes significantly improves the correlation coefficient from the previous one (from 0.282 to 0.438).

As a verification, we compare our findings with the existing study~\cite{lee2022indirect}, where the researchers employed a  mixed-effect model to measure each attribute's influence on the transformation of indirect contacts. 
The top-5 influential attributes suggested by their model are \texttt{female}, \texttt{extra}, \texttt{comments/mon} (the average number of comments made per month), \texttt{fbday}, and \texttt{age*fbday}. 
First of all, three attributes are seen in both their results and ours (i.e., \texttt{age*fbday}, \texttt{fbday}, and \texttt{female}). 
Also, we can expect that \texttt{tags/mon} and \texttt{comments/mon} have similar influences based on their meanings. 
In fact, when replacing \texttt{tags/mon} with \texttt{comments/mon} for the composite variable construction, we obtain the composite variable with 0.382 Pearson's correlation coefficient, which is considerably close to the original (Pearson's 0.438).
From these observations, we can expect that our NN-based model has successfully captured similar information to the statistical model used in \cite{lee2022indirect}.
Moreover, the unique strength of our approach is that we can see the inter-relationships of attributes from their weights in the composite variable. 
As discussed, we can suggest $0.6 \mathrm(\texttt{age*fbday}) {-} 0.6 \mathrm(\texttt{fbday}) {-} 0.5 \mathrm(\texttt{age})$ or a more simplified version, $ \texttt{age*fbday} {-} \texttt{fbday} {-} \texttt{age}$, as a potential composite variable to better capture the influence on the transformation than \texttt{age*fbday}, which is used in the aforementioned work.

