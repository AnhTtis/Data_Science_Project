\newcommand{\CaptionWorkflow}{The visual analytics workflow for investigating associations in multivariate networks.}

\newcommand{\CaptionUI}{The visual interface for facilitating interactive analysis using the workflow in \autoref{fig:workflow}. 
(a) Relating to Step 4, this view visualizes input attributes' contributions to the 1D network representation. 
(b) Composite variables generated in Step 5 are shown as a list of scatterplots, where the 1D network representation and composite variable correspond to $x$- and $y$-axes, respectively (see b2). 
As shown in b1, when a composite variable is not generated yet, a swarm-plot-like visualization presents the 1D network representation generated through Step 3 to help assess its quality.
For a and b, we employ our two-class density scatterplots.
% to depict density and class distribution information simultaneously.
% As shown in the fan-shape colormap in e, color lightness and hue are used to encode the density and the ratio of each class's density at each position, respectively.
% For example, in b2, we can see a high-density area mainly consisting of \texttt{Class\,1} (blue) at the right side as well as a clear separation between two classes (e.g., \texttt{Class\,0} (red) is mostly located at the left side).
(c) A node-link diagram and (d) a set of histograms inform the network structure and attribute distributions.
(e) Other auxiliary information is displayed, including the prediction accuracy of NNs trained for Step 2.}

\newcommand{\CaptionSHAP}{Comparison of SHAP value visualizations: (a) the default plot in the SHAP package~\cite{shap_library}; (b) the plot colored by class; and (c) our design.}

\newcommand{\CaptionScatterplots}{The comparison of scatterplot designs: (a) scatterplot with colored classes, (b) density scatterplot, (c,d) scatterplots encoding the total density and the ratio of each class's density with two different bivariate colormaps. (d) is our final design for a two-class density scatterplot.}

\newcommand{\CaptionCaseOne}{Study 1: Networks after selecting students with (a1) high (--\,0.9\,\texttt{open} +\,0.4\,\texttt{extra}) and (a2) high \texttt{extra}; distributions of \texttt{neuro} (b1, b2) corresponding to the selections of (a1, a2).}

\newcommand{\CaptionCaseTwo}{Study 2: The comparison of groups. (a-e) \texttt{business} and \texttt{engineering} students; (f) \texttt{engineering} students with high and low \texttt{eigenvector}.}

\newcommand{\CaptionCaseThree}{Study 3: (a) The top contributing attributes and composite variables constructed with (b) the top 3 and (c) top 5 attributes.}