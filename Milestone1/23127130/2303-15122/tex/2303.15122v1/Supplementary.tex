\newpage
\pagebreak
\onecolumn
\twocolumn
% \section{Supplementary}
% \begin{table}[!ht]
% \centering
% \begin{tabular}{@{}l|lllllllll|l@{}}
% \toprule
% \multirow{2}{*}{Method/Class} & Skin  & Nose        & E-glasses & L-Eye & R-Eye & L-Brow  & R-Brow   & L-Ear & R-Ear & \multirow{2}{*}{Mean}       \\ \cmidrule(lr){2-10}
%                               & Mouth & U-Lip       & L-Lip      & Hair  & Hat   & Earring & Necklace & Neck  & Cloth &                             \\ \cmidrule(r){1-11} \cmidrule(l){11-11} 

% \multirow{2}{*}{DML\_CSR\cite{dml_csr}}     & 95.7  & 93.9        & \textbf{92.6}       & 89.4  & 89.6  & 85.5    & \textbf{85.7}     & \textbf{88.3}  & 88.2  & \multirow{2}{*}{\textbf{86.07}} \\
%                               & 91.8  & 87.4        & 91         & 94.5  & \textbf{88.5}  & \textbf{71.4}    & 40.6     & 89.6  & 85.7  &                             \\ \cmidrule(r){1-11}
% \multirow{2}{*}{Ours}          & \textbf{96.6}  & \textbf{94} & 92.4       & \textbf{89.6}  & \textbf{89.7}  & 85.2    & 84.9     & 86.7  & 86.6  & \multirow{2}{*}{\textbf{86.07}} \\
%                               & 92.6  & \textbf{89.1}        & 91.1       & \textbf{95.2}  & 86.8  & 66.9    & \textbf{43.9}     & \textbf{91.3}  & \textbf{86.7}  &                             \\ \bottomrule
% \multirow{2}{*}{Ours$^{128\rightarrow256}$}  & \textbf{96.6}  & \textbf{93.9} & 92.4       & \textbf{89.6}  & \textbf{89.6}  & 85.2    & 84.9     & 86.7  & 86.6  & \multirow{2}{*}{86.03} \\
%                               & 92.5  & 88.9        & 91       & \textbf{95.2}  & 86.8  & 66.9    & \textbf{43.9}     & \textbf{91.3}  & \textbf{86.6}  &                             \\ \cmidrule(r){1-11}
% \multirow{2}{*}{Ours$^{192\rightarrow256}$}  & \textbf{96.6}  & \textbf{94} & 92.4       & \textbf{89.6}  & \textbf{89.7}  & 85.2    & 84.9     & 86.7  & 86.6  & \multirow{2}{*}{86.05} \\
%                               & 92.5  & \textbf{89.1}        & 91.1       & \textbf{95.2}  & 86.8  & 66.9    & \textbf{43.8}     & \textbf{91.3}  & \textbf{86.6}  &                             \\ \bottomrule                              
% \end{tabular}
% \caption{Results on CelebAMask-HQ: F1 score with low resolution output projected to 256$\times$256, compared with baselines}
%     \label{table:celeba-project}
% \end{table}
% \begin{table*}[t]
%     \centering
%     \begin{tabular}{|l|l|l|l|l|l|l|l|l|l|l|l|}
%     \hline
%         Method/Class & Skin & Hair & Nose & I-Mouth & L-Eye & R-Eye & L-Brow & R-Brow & U-Lip & L-Lip & Mean \\ \hline
%         DML\_CSR\cite{dml_csr} & 97.6 & 96.4 & \textbf{97.3} & 90.5 & 91.8 & 91.5 & 90.4 & 90.4 & 88 & 89.9 & \textbf{92.38} \\ \hline
%         Ours & 97.6 & 96 & 97.2 & 90.3 & \textbf{92} & \textbf{92.2} & \textbf{90.9} & \textbf{90.6} & 87.8 & 89.5 & \textbf{92.41} \\  \bottomrule
% Ours$^{128\rightarrow256}$ & 97.5 & 96 & 97.2 & 90.2 & {91.6} & \textbf{91.8} & \textbf{90.8} & \textbf{90.4} & 87.5 & 89.3 & {92.23} \\ \bottomrule
% Ours$^{192\rightarrow256}$ & 97.5 & 96 & 97.2 & 90.3 & \textbf{92} & \textbf{92.1} & \textbf{90.8} & \textbf{90.5} & 87.7 & 89.4 & 92.35 \\ \bottomrule
%     \end{tabular}
%     \caption{Results on Lapa: F1 score with low resolution output projected to 256$\times$256, compared with baselines}
%     \label{table:lapa-project}
% \end{table*}

\subsection{Analysis}
\begin{figure}[!ht]
\begin{center}
  \includegraphics[width=\linewidth]{images/celeb-dml-csr-better.png}
\end{center}
\caption{Few results where DML\_CSR performed better than \OURNAME on CelebAMask-HQ dataset.} 
\label{fig:celeb-dml-better}
\end{figure}
\begin{figure}[!ht]
\begin{center}
  \includegraphics[width=\linewidth]{images/lapa-dml-csr-better.png}
\end{center}
\caption{Few results where DML\_CSR performed better than \OURNAME on LaPa dataset} 
\label{fig:lapa-dml-better}
\end{figure}

The quantitative results shown in Table \ref{table:lapa_results}, \ref{table:celeba_results} points that
even though \OURNAME fares better in mean F1, the best classwise performance is scattered across multiple models. But at the same time, the gap between the best classwise scores and \OURNAME's classwise scores is marginal. Therefore, we try to further identify the problematic areas and include visualizations of \OURNAME's worst-performing results compared to DML\_CSR in F1 in Figure \ref{fig:lapa-dml-better}, \ref{fig:celeb-dml-better}. It can be seen from Figure \ref{fig:lapa-dml-better} that rows a) and c) have negligible differences, and in the remaining rows, both are performing poorly in the problematic regions of hair and face. In Figure \ref{fig:celeb-dml-better}'s rows b) and d), the F1 scores for these are debatable because of incorrect labeling in the ground truth. In the remaining rows, the underrepresented class of hat and earrings are bringing down our performance. Therefore the current setup of \OURNAME is affected by a lack of data as compared to DML\_CSR. This can also be corroborated by Table \ref{table:celeba_results}. It is also necessary to point out that the ground truth data of CelebAMask-HQ is noisy (Figure \ref{fig:gt-worse}) and can cause problems in training and testing.

From the point of view of inference time, \OURNAME could be used to generate segmentation at a lower resolution, and the generated output scaled at the required higher resolution to improve inference time and hence increase fps. The generation of lower-resolution segmentation does not require any additional training and is an outcome of being an implicit neural representation network. The 128-resolution version of \OURNAME clocked an fps of 294 compared to the regular version of resolution 256, which runs at 120 fps. This makes our model more conducive for low compute devices. 

\subsubsection{Variance in performance over mulplte runs}
We also calculate the mean and variance of our model's F1 score for Lapa, CelebAMask-HQ and Helen in Table \ref{table:mean_var}
\begin{table}[!ht]
\centering
\begin{tabular}{|l|l|l|}
\hline
        & Mean        & SD          \\ \hline
F1 Lapa & 92.35 & 0.06 \\ \hline
F1 Celeb & 85.90 & 0.20 \\ \hline
F1 Helen & 91.12 & 0.10 \\ \hline
\end{tabular}
\captionsetup{aboveskip=4pt,belowskip=-10pt}
\caption{Mean and Variance of \OURNAME}
\label{table:mean_var}
\end{table}
 It should be noted that other state-of-the-art works do not report these mean and variance over multiple runs and therefore direct comparison of these numbers is not possible.