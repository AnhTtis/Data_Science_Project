% CVPR 2023 Paper Template
% based on the CVPR template provided by Ming-Ming Cheng (https://github.com/MCG-NKU/CVPR_Template)
% modified and extended by Stefan Roth (stefan.roth@NOSPAMtu-darmstadt.de)

\documentclass[10pt,twocolumn,letterpaper]{article}

%%%%%%%%% PAPER TYPE  - PLEASE UPDATE FOR FINAL VERSION
% \usepackage[review]{cvpr}      % To produce the REVIEW version
\usepackage{cvpr}              % To produce the CAMERA-READY version
%\usepackage[pagenumbers]{cvpr} % To force page numbers, e.g. for an arXiv version

% Include other packages here, before hyperref.
\usepackage{graphicx}
\usepackage{amsmath}
\usepackage{amssymb}
\usepackage{booktabs}

\usepackage{xcolor}
\usepackage[pagebackref,breaklinks,colorlinks,bookmarks=false,urlcolor=black,citecolor={green!60!black}]{hyperref}

\usepackage{algorithm}
\usepackage{algorithmic}
\usepackage{colortbl}
\definecolor{dkgreen}{rgb}{0,0.6,0}
\definecolor{gray}{rgb}{0.5,0.5,0.5}
\definecolor{mauve}{rgb}{0.58,0,0.82}
\definecolor{LightCyan}{rgb}{0.88,1,1}

% Improves PDF readability for those with disabilities.
\usepackage[accsupp]{axessibility}

\usepackage{color}

\usepackage{tikz}
\usepackage{comment}
\usepackage{booktabs}
\usepackage{epsfig}
\usepackage{multirow}
\usepackage{enumitem}
\usepackage{bbm}
\usepackage{arydshln}

\let\proof\relax
\let\endproof\relax

\usepackage{dsfont}
\usepackage{bm}
\usepackage{mathtools}

% \usepackage{caption}
% \usepackage{subcaption}
% \usepackage{xspace}

% lei added for pseudocode
\usepackage{listings}
\usepackage{color}

% It is strongly recommended to use hyperref, especially for the review version.
% hyperref with option pagebackref eases the reviewers' job.
% Please disable hyperref *only* if you encounter grave issues, e.g. with the
% file validation for the camera-ready version.
%
% If you comment hyperref and then uncomment it, you should delete
% ReviewTempalte.aux before re-running LaTeX.
% (Or just hit 'q' on the first LaTeX run, let it finish, and you
%  should be clear).
\usepackage[pagebackref,breaklinks,colorlinks]{hyperref}


% Support for easy cross-referencing
\usepackage[capitalize]{cleveref}
\crefname{section}{Sec.}{Secs.}
\Crefname{section}{Section}{Sections}
\Crefname{table}{Table}{Tables}
\crefname{table}{Tab.}{Tabs.}


%%%%%%%%% PAPER ID  - PLEASE UPDATE
\def\cvprPaperID{5780} % *** Enter the CVPR Paper ID here
\def\confName{CVPR}
\def\confYear{2023}


\definecolor{dkgreen}{rgb}{0,0.6,0}
\definecolor{gray}{rgb}{0.5,0.5,0.5}
\definecolor{mauve}{rgb}{0.58,0,0.82}

\lstset{frame=tb,
  language=Java,
  aboveskip=3mm,
  belowskip=3mm,
  showstringspaces=false,
  columns=flexible,
  basicstyle={\small\ttfamily},
  numbers=none,
  numberstyle=\tiny\color{gray},
  keywordstyle=\color{blue},
  commentstyle=\color{dkgreen},
  stringstyle=\color{mauve},
  breaklines=true,
  breakatwhitespace=true,
  tabsize=3
}





\makeatletter
\def\ps@myheadings{%
    \let\@oddfoot\@empty\let\@evenfoot\@empty
    \def\@evenhead{\thepage\hfil\slshape\leftmark}%
    \def\@oddhead{{\slshape\rightmark}\hfil\thepage}%
    \let\@mkboth\@gobbletwo
    \let\sectionmark\@gobble
    \let\subsectionmark\@gobble
    }
  \if@titlepage
  \renewcommand\maketitle{\begin{titlepage}%
  \let\footnotesize\small
  \let\footnoterule\relax
  \let \footnote \thanks
  \null\vfil
  \vskip 60\p@
  \begin{center}%
    {\LARGE \@title \par}%
    \vskip 3em%
    {\large
     \lineskip .75em%
      \begin{tabular}[t]{c}%
        \@author
      \end{tabular}\par}%
      \vskip 1.5em%
    {\large \@date \par}%       % Set date in \large size.
  \end{center}\par
  \@thanks
  \vfil\null
  \end{titlepage}%
  \setcounter{footnote}{0}%
}
\else
\renewcommand\maketitle{\par
  \begingroup
    \renewcommand\thefootnote{\@fnsymbol\c@footnote}%
    \def\@makefnmark{\rlap{\@textsuperscript{\normalfont\color{black}\@thefnmark}}}%
    \long\def\@makefntext##1{\parindent 1em\noindent
            \hb@xt@1.8em{%
                \hss\@textsuperscript{\normalfont\@thefnmark}}##1}%
    \if@twocolumn
      \ifnum \col@number=\@ne
        \@maketitle
      \else
        \twocolumn[\@maketitle]%
      \fi
    \else
      \newpage
      \global\@topnum\z@   % Prevents figures from going at top of page.
      \@maketitle
    \fi
    \thispagestyle{plain}\@thanks
  \endgroup
  \setcounter{footnote}{0}%
}
\makeatother



\makeatletter
\newcommand\fs@nobottomruled{\def\@fs@cfont{\bfseries}\let\@fs@capt\floatc@ruled
  \def\@fs@pre{}% \hrule height.8pt depth0pt \kern2pt
  \def\@fs@post{}% Formerly \def\@fs@post{\kern2pt\hrule\relax}%
  \def\@fs@mid{\kern2pt\hrule\kern2pt}%
  \let\@fs@iftopcapt\iftrue}
\makeatother


%\renewcommand{\floatpagefraction}{0.8}
\renewcommand{\dblfloatpagefraction}{0.99}

\renewcommand{\topfraction}{0.99} % 90 percent of the page may be used by floats on top
\renewcommand{\bottomfraction}{0.99} % the same at the bottom
\renewcommand{\textfraction}{0.01} % at least 1 percent must be reserved for text
\renewcommand{\floatpagefraction}{0.99}

\setcounter{topnumber}{4}
\setcounter{bottomnumber}{4}
\setcounter{totalnumber}{8}





\makeatletter
\DeclareRobustCommand\bmvaOneDot{\futurelet\@let@token\bmv@onedotaux}
\def\bmv@onedotaux{\ifx\@let@token.\else.\null\fi\xspace}
\makeatother
\def\eg{\emph{e.g}\bmvaOneDot}
\def\Eg{\emph{E.g}\bmvaOneDot}
\def\etal{\emph{et al}\bmvaOneDot}
\def\etc{\emph{etc}\bmvaOneDot}
\def\ie{\emph{i.e}\bmvaOneDot}
\def\aka{\emph{a.k.a}\bmvaOneDot}
\def\wrt{\emph{w.r.t}\bmvaOneDot}
\def\cf{\emph{c.f}\bmvaOneDot}
\def\vs{\emph{vs}\bmvaOneDot}


% variables
%\vec{\mathcal{M}}
\renewcommand\vec[1]{\ensuremath\boldsymbol{#1}}
\renewcommand\cdots{...}
\newcommand{\tB}{\vec{\mathcal{B}}}
\newcommand{\tY}{\vec{\mathcal{Y}}}
\newcommand{\tF}{\vec{\mathcal{F}}}
\newcommand{\cB}{\mathcal{B}}
\newcommand{\mB}{\mathbf{B}}
\newcommand{\mY}{\mathbf{Y}}
\newcommand{\mZ}{\mathbf{Z}}
\newcommand{\vb}{\mathbf{b}}
\newcommand{\vy}{\mathbf{y}}
\newcommand{\valpha}{\boldsymbol{\alpha}}
\newcommand{\tA}{\vec{\mathcal{A}}}
\newcommand{\tD}{\vec{\mathcal{D}}}
\newcommand{\tX}{\vec{\mathcal{X}}}
\newcommand{\tM}{\vec{\mathcal{M}}}
\newcommand{\cX}{\mathcal{X}}
\newcommand{\mX}{\mathbf{X}}
\newcommand{\mA}{\mathbf{A}}
\newcommand{\vx}{\mathbf{x}}
\newcommand{\vq}{\mathbf{q}}
\newcommand{\mbrp}[1]{\mathbb{R}_{+}^{#1}}
\newcommand{\mbr}[1]{\mathbb{R}^{#1}}
\newcommand{\mbn}[1]{\mathbb{N}^{#1}}
\newcommand{\mbnz}[1]{\mathbb{N}_{0^+}^{#1}}
\newcommand{\mbnp}[1]{\mathbb{N}_{+}^{#1}}
\newcommand{\stackThree}{{;}_{3}}
\newcommand{\vbeta}{\vec{\beta}}
%\newcommand{\rank}[1]{\text{Rank}({#1})}
\newcommand{\tI}{\vec{\mathcal{I}}}

\newcommand{\tAnb}{\mathcal{A}}
\newcommand{\tMnb}{\mathcal{M}}
\newcommand{\tXnb}{\mathcal{X}}
\newcommand{\tYnb}{\mathcal{Y}}
\newcommand{\tInb}{\mathcal{I}}

\newcommand{\vectorise}{\text{Vec}}


%\newcommand{\tXS}{\vec{\mathcal{X}}^{*}}
\newcommand{\vv}{\mathbf{v}}
\newcommand{\tV}{\vec{\mathcal{V}}}
\newcommand{\tE}{\vec{\mathcal{E}}}
\newcommand{\tEH}{\vec{\mathcal{\hat{E}}}}
\newcommand{\tVH}{\vec{\mathcal{\bar{V}}}}
\newcommand{\tVT}{\vec{\mathcal{\hat{V}}}}
\newcommand{\idx}[1]{\mathcal{I}_{#1}}
\newcommand{\semipd}[1]{\mathcal{S}_{+}^{#1}}
\newcommand{\spd}[1]{\mathcal{S}_{++}^{#1}}

\newcommand{\tR}{\vec{\mathcal{R}}}
\newcommand{\vu}{\mathbf{u}}
\newcommand{\vup}{\mathbf{u^{'}}}
\newcommand{\vz}{\mathbf{z}}
\newcommand{\vzeta}{\boldsymbol{\zeta}}
\newcommand{\vc}{\mathbf{c}}

\newcommand{\vphi}{\boldsymbol{\phi}}
\newcommand{\vpsi}{\boldsymbol{\psi}}
\newcommand{\tPsi}{\vec{\mathcal{V}}}
\newcommand{\bigoh}{\mathcal{O}}
\newcommand{\mPsi}{\vec{\Psi}}
\newcommand{\vj}{\vec{j}}

% operators
\newcommand{\enorm}[1]{\left\|{#1}\right\|_2}
\newcommand{\fnorm}[1]{\left\|{#1}\right\|_F}
\newcommand{\lnorm}[1]{\left\|{#1}\right\|_1}
\newcommand{\riem}{\mathbf{d}_{\mathcal{R}}}
\newcommand{\spdp}[1]{\mathbb{S}_{++}^{#1}}
\newcommand{\simplex}[1]{\Delta^{#1}}
\newcommand{\set}[1]{\left\{#1\right\}}

\DeclareMathOperator*{\argmin}{arg\,min}
\DeclareMathOperator*{\argmax}{arg\,max}
\DeclareMathOperator*{\supp}{Supp}
\DeclareMathOperator*{\unique}{Unique}
\DeclareMathOperator*{\TRank}{TRank}
\DeclareMathOperator*{\rank}{Rank}
\DeclareMathOperator*{\spann}{Span}
\DeclareMathOperator*{\sym}{Sym}
% \DeclareMathOperator*{\softmingg}{SoftMin_{\bar{\gamma}}}
% \DeclareMathOperator*{\softming}{SoftMin_\gamma}
% \DeclareMathOperator*{\topminb}{TopMin_\beta}
% \DeclareMathOperator*{\topmaxbb}{TopMax_{NZ\beta}}

% \DeclareMathOperator*{\softminsel}{SoftMinSel_\gamma}

\newcommand{\flatt}[1]{\text{Flatten}\!\left({#1}\right)}
\newcommand{\unflatt}[1]{\text{Flatten}^{-1}\!\left({#1}\right)}
\newcommand{\myspan}[1]{\spann\left(#1\right)}

\DeclareMathOperator*{\trace}{Tr}
%\DeclareMathOperator*{\rank}{Rank}
\DeclareMathOperator*{\kronstack}{\uparrow\!\otimes}

\DeclareMathOperator*{\diag}{Diag}
\DeclareMathOperator*{\avg}{avg}
\DeclareMathOperator*{\sgn}{Sgn}
\DeclareMathOperator*{\hosvd}{HOSVD}
\DeclareMathOperator*{\logm}{Log}
\DeclareMathOperator*{\expm}{{Exp}}
\DeclareMathOperator*{\detm}{Det}
\DeclareMathOperator*{\fg}{g}
\newcommand{\expl}[1]{\text{e}^{#1}}
\DeclareMathOperator*{\res}{Res}
\DeclareMathOperator*{\asinh}{Asinh}
\DeclareMathOperator*{\vect}{vec}
\DeclareMathOperator*{\detach}{Detach}
%\newcommand{\exp}[1]{e^{#1}}




\newcommand{\mI}{\mathbf{I}}
\newcommand{\normvec}[1]{\frac{#1}{\|{#1}\|_2}}
\newcommand{\suptensor}[1]{\mathfrak{S}^{#1}}
\newcommand{\suptensorr}[2]{\mathfrak{S}^{#1}_{\times^{#2}}}
\newcommand{\region}{\mathcal{R}}

%\newtheorem{theorem}{Theorem}
%\newtheorem{definition}{Definition}
%\newtheorem{lemma}{Lemma}
%\newtheorem{proposition}{Proposition}
%\newtheorem{remark}{Remark}
%\newtheorem*{Proof}{Proof}

%\newcommand{\todo}[1]{{\bf \textcolor{red}{[TODO: #1]}}}


\newcommand{\mLa}{\boldsymbol{\lambda}^{*}}
\newcommand{\mLambda}{\boldsymbol{\lambda}}
\newcommand{\mU}{\mathbf{U}}
\newcommand{\mV}{\mathbf{V}}
\newcommand{\timetplone}{{(t+1)}}
\newcommand{\timet}{{(t)}}

\newcommand{\mBOvl}{{\mB^{*}}}

\newcommand{\mPi}{{\boldsymbol\Pi}}

\newcommand{\piA}{{\Pi_A}}
\newcommand{\piB}{{\Pi_B}}

\newcommand{\sigmav}{{^v\!\!\,{\sigma}}}
%\newcommand{\thickhat}[1]{\mathbf{\ddot{\text{$#1$}}}}
\newcommand{\sigmas}{{^s\!\!\,{\sigma}}}

\newcommand{\fvx}{{\boldsymbol{f}(\vx)}}
\newcommand{\fvy}{{\boldsymbol{f}(\vy)}}

\newcommand{\vsss}{\boldsymbol{s}}
\newcommand{\vw}{\boldsymbol{w}}

\newcommand{\vphibar}{\boldsymbol{\bar{\phi}}}
\newcommand{\vsigma}{\boldsymbol{\sigma}}

\def\eg{\emph{e.g.}}

\newcommand{\myg}[1]{\boldsymbol{G}\left(#1\right)}
\newcommand{\mygtwo}[1]{\boldsymbol{G}\Big(#1\Big)}
%\newcommand{\mygthree}[1]{\boldsymbol{G}\left(#1\right)}
\newcommand{\mygthree}[1]{\boldsymbol{\mathcal{G}}\!\left(\!#1\!\right)}
\newcommand{\mygthrees}[1]{\boldsymbol{\mathcal{G}^*\!}\!\left(\!#1\!\right)}
\newcommand{\mygfour}[1]{\boldsymbol{\mathcal{G}}\!\Bigg(\!#1\!\Bigg)}
\newcommand{\tG}{\boldsymbol{\mathcal{G}}}
\newcommand{\tGhat}{\widehat{\boldsymbol{\mathcal{G}}}}

\newcommand{\mygthreep}[1]{\boldsymbol{\mathcal{G}'}\!\left(\!#1\!\right)}
\newcommand{\mygthreee}[2]{\boldsymbol{\mathcal{G}}_{{\text{#1}}}\!\left(\!#2\!\right)}

\newcommand{\mygthreehat}[1]{\boldsymbol{\widehat{\mathcal{G}}}\!\left(\!#1\!\right)}
\newcommand{\mygthreephat}[1]{\boldsymbol{\widehat{\mathcal{G}}'}\!\left(\!#1\!\right)}

\newcommand{\mygthreeehat}[2]{\boldsymbol{\widehat{\mathcal{G}}_{{\text{#1}}}}\!\left(\!#2\!\right)}
\newcommand{\mygthreeep}[2]{\boldsymbol{\mathcal{G}'_{{\text{#1}}}}\!\left(\!#2\!\right)}
\newcommand{\mygthreeephat}[2]{\boldsymbol{\widehat{\mathcal{G}}'_{{\text{#1}}}}\!\left(\!#2\!\right)}

\newcommand{\vPhi}{\boldsymbol{\Phi}}
\newcommand{\invbeta}{{(1\!-\!\beta)}}
\newcommand{\invsqrtbeta}{\sqrt{1\!-\!\beta}}
\newcommand{\sqrtbeta}{\sqrt{\beta}}


\newcommand{\mIdent}{\boldsymbol{\mathds{I}}}
\newcommand{\sIdent}{\mathds{I}}
\newcommand{\vOnes}{\boldsymbol{1}}

\newcommand{\mJ}{\mathbf{J}}
\newcommand{\sXkl}{{X_{kl}}}

\newcommand{\mQ}{\mathbf{Q}}
\newcommand{\mK}{\mathbf{K}}
\newcommand{\mKb}{\bar{\mK}}
\newcommand{\mKbb}{\bar{\mKb}}
\newcommand{\Kb}{\bar{K}}
\newcommand{\Kbb}{\bar{\Kb}}
\newcommand{\mC}{\mathbf{C}}

\newcommand{\mKro}{{\mK^{q}}}
\newcommand{\mKbro}{{\mKb{\,\!}^{q}}}
\newcommand{\mKbbro}{{\mKbb^{q}}}
\newcommand{\Kbro}{{\Kb^{q}}}
\newcommand{\Kbbro}{{\Kbb^{q}}}



\newcommand{\fvxt}{{\boldsymbol{f}^{(t)}(\vx)}}
\newcommand{\Fvxt}{{\boldsymbol{F}^{(t)}(\vx)}}
\newcommand{\fvxtplusone}{{\boldsymbol{f}^{(t+1)}(\vx)}}
\newcommand{\Fvxtplusone}{{\boldsymbol{F}^{(t+1)}(\vx)}}
\newcommand{\vxzero}{\mathbf{x}_0}
\newcommand{\fvxzerotplusone}{{\boldsymbol{f}^{(t+1)}(\vxzero)}}
\newcommand{\Fvxzerotplusone}{{\boldsymbol{F}^{(t+1)}(\vxzero)}}
\newcommand{\fvxzerotplusonei}{{\boldsymbol{f}_i^{(t+1)}(\vxzero)}}

\newcommand{\tFvxt}{{{\vec{\mathcal{F}}}^{(t)}(\vx)}}
\newcommand{\tFvxtplusone}{{{\vec{\mathcal{F}}}^{(t+1)}(\vx)}}
\newcommand{\tFvxzerotplusone}{{{\vec{\mathcal{F}}}^{(t+1)}(\vxzero)}}




\newcommand{\swbar}{\bar{w}}
\newcommand{\vwbar}{\bar{\boldsymbol{w}}}

\newcommand{\vvartheta}{\boldsymbol{\vartheta}}
\newcommand{\tprim}[1]{{\uparrow T_{#1}}}


\newcommand{\vS}{\boldsymbol{S}}

\newcommand{\barM}{{\bar{M}}}
\newcommand{\barmM}{{\bar{\vec{M}}}}

\newcommand{\sN}{\vec{N}}
\newcommand{\tN}{\vec{\mathcal{N}}}
\newcommand{\tP}{\vec{\mathcal{P}}}
\newcommand{\tS}{\vec{\mathcal{S}}}
\newcommand{\tSnb}{\mathcal{S}}
\newcommand{\mS}{\vec{S}}
\newcommand{\tNnb}{\mathcal{N}}

\newcommand{\cov}{\boldsymbol{\Sigma}}
\newcommand{\covb}{\boldsymbol{\Sigma}^{(\!\diamond\!)}}
\newcommand{\covw}{\boldsymbol{\Sigma}^{(\!*\!)}}
\newcommand{\vphix}[1]{{\boldsymbol{\phi}\left({#1}\right)}}
\newcommand{\covbb}[1]{{\boldsymbol{\Sigma}^{(\diamond,{#1})}}}
\newcommand{\covww}[1]{{\boldsymbol{\Sigma}_c^{(*,{#1})}}}

\newcommand{\muw}{\boldsymbol{\mu}^{(*)}}
\newcommand{\mub}{\boldsymbol{\mu}^{(\diamond)}}

\newcommand{\mubb}[1]{{\boldsymbol{\mu}^{(\diamond,{#1})}}}
\newcommand{\muww}[1]{{\boldsymbol{\mu}_c^{(*,{#1})}}}
\newcommand{\muwww}[2]{{\boldsymbol{\mu}_{#1}^{(*,{#2})}}}

\newcommand{\mPhi}{\boldsymbol{\Phi}}
\newcommand{\mPhibar}{\boldsymbol{\bar{\Phi}}}

\newcommand{\parsmP}{\!\left(\mPhi\right)}
\newcommand{\parsmPc}{\!\left(\mPhi_c\right)}
\newcommand{\parsmPA}{\!\left(\mPhi^A\right)}
\newcommand{\parsmPB}{\!\left(\mPhi^B\right)}
\newcommand{\parsmPAB}{\!\left(\mPhi^A\!,\mPhi^B\right)}
\newcommand{\parsmPcA}{\!\left(\mPhi_c^A\right)}
\newcommand{\parsmPcB}{\!\left(\mPhi_c^B\right)}
\newcommand{\parsmPxY}[2]{{\!\left(\mPhi_{#1}^{#2}\right)}}

\newcommand{\mOmega}{\boldsymbol{\Omega}}
\newcommand{\mMu}{\boldsymbol{M}}
\newcommand{\mM}{\boldsymbol{M}}
\newcommand{\mF}{\boldsymbol{F}}
\newcommand{\parsmMu}{\!\left(\mMu\right)}
\newcommand{\mW}{\boldsymbol{W}}
\newcommand{\mD}{\boldsymbol{D}}
\newcommand{\vd}{\boldsymbol{d}}
\newcommand{\mT}{\boldsymbol{T}}
\newcommand{\mG}{\boldsymbol{G}}

\newcommand{\vp}{\boldsymbol{p}}

\newcommand{\vm}{\boldsymbol{m}}
\newcommand{\vmu}{\boldsymbol{\mu}}
\newcommand{\bvmu}{\boldsymbol{\overline{\mu}}}
\newcommand{\mP}{\boldsymbol{\Theta}}
\newcommand{\vmubar}{\boldsymbol{\bar{\mu}}}
\newcommand{\vvarphi}{\boldsymbol{\varphi}}


\newcommand{\stkout}[1]{{\ifmmode\text{\sout{\ensuremath{#1}}}\else\sout{#1}\fi}}

\newcommand{\mL}{\mathbf{L}}

\DeclareMathOperator*{\arcsinh}{arcsinh}
\newcommand{\lei}{\textcolor{black}}
\begin{document}

%%%%%%%%% TITLE - PLEASE UPDATE
\title{\lei{3Mformer: } Multi-order Multi-mode Transformer \\for Skeletal Action Recognition}

% \author{Lei Wang\\
% The Australian National University \& Data61/CSIRO\\
% Institution1 address\\
% {\tt\small lei.wang@data61.csiro.au}
% % For a paper whose authors are all at the same institution,
% % omit the following lines up until the closing ``}''.
% % Additional authors and addresses can be added with ``\and'',
% % just like the second author.
% % To save space, use either the email address or home page, not both
% \and
% Second Author\\
% Institution2\\
% First line of institution2 address\\
% {\tt\small secondauthor@i2.org}
% }

\author{Lei Wang\textsuperscript{$\dagger, \S$}\qquad Piotr Koniusz\textsuperscript{$\S,\dagger$}\\
$^{\dagger}$The Australian National University, $^\S$Data61/CSIRO\\
$^\S$firstname.lastname@data61.csiro.au
}


\maketitle

%%%%%%%%% ABSTRACT
\begin{abstract}
Many  skeletal action recognition models adopt \lei{GCNs} which  represent the human body by 3D body joints according to their physical connectivity. %These models typically have limited receptive fields as they
However, as GCNs aggregate in \lei{one- or few-hop} graph neighbourhoods, %in the skeleton graph, diffuse information of body joints locally, 
they ignore the dependency between  non-connected directly body joints. %Thus, to improve the use of skeletons, 
Thus, we propose to form hypergraph to  model hyper-edges between graph nodes (\eg, third- and fourth-order hyper-edges connect three and four nodes, respectively) which help capture higher-order motion patterns of groups of body joints. 
%
We split action sequences into temporal blocks, Higher-order Transformer (HoT) produces embeddings of each temporal block based on (i) the body joints, % (set of joints), 
(ii) pairwise links of body joints and (iii) higher-order hyper-edges of skeleton body joints. 
% 
%Our embeddings extracted from skeletal hypergraph with third-order hyper-edges outperform all existing GCN-based models by $\sim$2\% on average. 
%
Then we  combine such HoT embeddings of hyper-edges of orders $1,\cdots,r$ by a novel \lei{Multi-order Multi-mode Transformer (3Mformer)} with two modules whose order can be exchanged to achieve \lei{joint-mode} attention on joint-mode tokens based on `channel-temporal block', `\lei{order-}channel-body joint', `channel-hyper-edge (any order)' and `channel-only' pairs. The first module, called Multi-order Pooling (MP)\lei{,} additionally learns weighted aggregation along the hyper-edge mode, whereas the second module, Temporal block Pooling (TP)\lei{,} aggregates along the temporal block\footnote{For brevity, we write that we have $\tau$ temporal blocks per sequence. In fact, $\tau$ varies and so we use pooling.} mode.
%
%
Our end-to-end trainable network yields state-of-the-art results compared to  GCN-, transformer- and  existing hypergraph-based counterparts. % on several benchmarks.% , \eg, the large-scale Kinetics-Skeleton.
\end{abstract}

%===========================================================
\section{Introduction}

Action Recognition has  applications in video surveillance, human-computer interaction, sports analysis, and virtual reality. 
%
% Different from RGB video based methods, which mainly focus on modeling the spatial and temporal representations from RGB frames and/or optical flow videos,  skeleton sequences, representing a spatio-temporal evolution of 3D body joints, have been shown to be effective in action recognition, robust against sensor noises, and computationally and storage efficient. 
\lei{Different from video-based methods which mainly focus on modeling the spatio-temporal representations from RGB frames and/or optical flow,  skeleton sequences, representing a spatio-temporal evolution of 3D body joints, have been proven  robust against sensor noises and  effective in action recognition while being computationally and storage efficient.}
%
The skeleton data is usually obtained by either localization of 2D/3D coordinates of human body joints with the depth sensors or pose estimation algorithms applied to videos~\cite{Cao_2017_CVPR}. Skeleton sequences enjoy (i) simple structural connectivity of skeletal graph and (ii) temporal continuity of 3D body joints evolving in time. While temporal evolution of each body joint is highly informative, embeddings of separate body joints are insensitive to relations between  body parts. Moreover, while the links between adjacent 3D body joints (following the structural connectivity) are very informative as they model relations, these links represent highly correlated nodes in the sense of their temporal evolution. Thus, modeling larger groups of 3D body joints as hyper-edges can capture more complex spatio-temporal motion dynamics.

%Graph-based model is one of the state-of-the-art (SOTA) methods for skeletal AR due to its effective representation for the graph structure data. 
The existing graph-based models mainly differ by how they handle temporal information, \ie, 
%
\lei{Graph Neural Network (GNN) may encode spatial neighborhood of the node followed by aggregation of an LSTM~\cite{Si_2019_CVPR, 8784712}. % by Recurrent Neural Network. 
Alternatively, Graph Convolutional Network (GCN) may perform spatio-temporal convolution in the neighborhood of each node~\cite{stgcn2018aaai}.} %There are two types of GCNs, spectral GCNs and spatial GCNs. 
%In addition, spectral GCNs \cite{kipf2017semi} perform convolution in the spectral domain whereas 
Spatial GCNs perform convolution within one or two hop radius of each node, \eg, spatio-temporal GCN model called ST-GCN \cite{stgcn2018aaai} models spatio-temporal vicinity of each 3D body joint. %Each spatial temporal graph convolutional layer constructs spatial characteristics with a graph convolutional operator, and models the temporal dynamics with a convolutional operator. 
% ({\bf another version}: The earliest attempts of skeleton-based AR often encode all the human body joint coordinates in each frame to a feature vector for pattern learning~\cite{lei_tip2019}. These models rarely explore the internal dependencies between body joints, which results in missing abundant/rich actional information. To capture the joint dependencies, recent methods construct a skeleton graph whose vertices are joints and edges are bones, and apply the GCN to extract the correlected features. The ST-GCN~\cite{stgcn2018aaai} is further developed to simultaneously learn spatial and temporal features. 
%
As ST-GCN applies convolution along structural connections (links between body joints), structurally distant joints, which may cover key patterns of actions, are largely ignored. %For example, while walking or kicking a ball, joint motion dynamics of hands and feet form very informative signatures. 
ST-GCN captures ever larger neighborhoods as layers are added, but such a representation suffers from the so-called oversmoothing. %, and convolution along the structural links ignores the dependency between physically distant 3D body joints. 
%
%However, the previous GCN-based AR generally represents only the physical structure of the human body, and aggregates the 1-hop neighbors' information in a skeleton graph. Thus,
%As the receptive field of the convolution kernel is limited, it severely impairs the quality of graph representation. The methods only diffuse information in a local range and ignore the dependency between non-physical connection joints. 

Human actions are  associated with interaction groups of skeletal joints, \eg, wrist alone, head-wrist, head-wrist-ankles, \etc. The impact of these groups of joints on each action differs, and the degree of influence of each joint should be learned. Accordingly, designing a better model for skeleton data is vital given the topology of skeleton graph  is  suboptimal. % for action recognition. 
While GCN can be applied to a fully-connected graph (\ie, 3D body joints as densely connected graph nodes), Higher-order Transformer (HoT) \cite{kim2021transformers} has been proven more efficient.

Thus, we propose to use hypergraphs with hyper-edges of order $1$ to $r$ to effectively represent skeleton data for action recognition. %Attention on hyper-edges of orders $1$ to $m$ is determined in HoT.
%
% 
Compared to  GCNs, 
%
our encoder contains an MLP followed by three HoT branches that encode first-, second- and higher-order hyper-edges, \ie, set of body joints, edges between pairs of nodes, hyper-edges between triplets of nodes, \etc. Each branch  has its own learnable parameters, and processes  temporal blocks\footnote{Each temporal block enjoys a locally factored out (removed) temporal mode, which makes each block representation  compact.} one-by-one.

We notice that (i) the number of hyper-edges of $J$ joints grows rapidly with order $r$, \ie, $\binom{J}{i}$ for $i=1,\cdots,r$,  embeddings of the highest order %hyper-edges
dominate lower orders in terms of volume if such embeddings are merely concatenated, and (ii) long-range temporal dependencies of feature maps are insufficiently explored, as  sequences are split into $\tau$ temporal blocks  for computational tractability. %to be able to process each block via MLP and pass to HoT.

Merely concatenating outputs of  HoT branches of orders $1$ to $r$, and across $\tau$ blocks, is  sub-optimal.  
Thus, our \lei{Multi-order Multi-mode Transformer (3Mformer)} with two modules whose order can be exchanged, contains a variation of joint-mode tokens based on `channel-temporal block', `\lei{order-}channel-body joint', `channel-hyper-edge (any order)' and `channel-only' pairs. As HoT operates on block-by-block basis, `channel-temporal block' tokens and weighted hyper-edge aggregation in Multi-order Pooling (MP) help combine information flow block-wise. In fact, various joint-mode tokens help improve results further due to different focus of each attention mechanism. Moreover, as the \lei{block-temporal} mode needs to be aggregated (number of blocks varies across sequences), Temporal block Pooling (TP) module aggregates via popular pooling, \eg, rank-~\cite{10.1109/TPAMI.2016.2558148}, second-~\cite{Gao_2019_CVPR,NEURIPS2018_17c276c8, Girdhar_17b_AttentionalPoolingAction} or higher-order pooling~\cite{7926605, DBLP:journals/corr/abs-2110-05216}. %With the objective of maximizing the performance, the network is trained end-to-end to encourage each stream to extract rich representations. 

\vspace{0.2cm}
In summary, our main contributions are listed as follows:$\!\!\!$

\renewcommand{\labelenumi}{\roman{enumi}.}
% \begin{enumerate}[leftmargin=0.6cm]
% \item We model the skeleton data as hypergraph of orders $1$ to $r$ (set, graph and/or hypergraph), where human body joints serve as nodes. Higher-order Transformer  embeddings of such formed hyper-edges represent various groups of 3D body joints and capture various dynamics important for action recognition.
% %
% \item As HoT embeddings represent individual hyper-edge order and block, we introduce a novel Multi-mode Attention  with two modules, Multi-order Pooling and Temporal block Pooling, whose goal is to form joint-mode tokens such as `channel-temporal block', `channel-body joint', `channel-hyper-edge (any order)' and `channel', and perform weighted hyper-edge aggregation and temporal block aggregation.
% \end{enumerate}
\begin{enumerate}[leftmargin=0.6cm]
\item We model the skeleton data as hypergraph of orders $1$ to $r$ (set, graph and/or hypergraph), where human body joints serve as nodes. Higher-order Transformer embeddings of such formed hyper-edges represent various groups of 3D body joints and capture various higher-order dynamics important for action recognition.
%
\item As HoT embeddings represent individual hyper-edge order and block, we introduce a novel Multi-order Multi-mode Transformer (3Mformer) with two modules, Multi-order Pooling and Temporal block Pooling, whose goal is to form joint-mode tokens such as `channel-temporal block', `order-channel-body joint', `channel-hyper-edge (any order)' and `channel-only', and perform weighted hyper-edge aggregation and temporal block aggregation.
%
%\item Our model outperforms the state-of-the-art GCN- and hypergraph-based models on NTU-60, NTU-120 and Kinetics-Skeleton by a large margin.
\end{enumerate}
%

Our 3Mformer outperforms  other GCN- and hypergraph-based models on NTU-60, NTU-120 and Kinetics-Skeleton by a large margin.


%Do not use any additional Latex macros.

%------------------------------------------------------------------------- 

\begin{figure*}[t]%htbp % left bottom right top
% \vspace{-0.3cm}
%\centering\includegraphics[width=0.95\linewidth]{imgs/pipelinenew.pdf}
\centering\includegraphics[width=1\linewidth]{imgs/pipe4lei2.pdf}
%\vspace{-0.3cm}
\caption{Pipeline overview. Each sequence is split into $\tau$ temporal blocks $\mathbf{B}_1,\cdots,\mathbf{B}_\tau$. Subsequently, each block is embedded by a simple MLP into $\mathbf{X}_1,\cdots,\mathbf{X}_\tau$, which are passed to  Higher-order Transformers (\lei{HoT} ($n\!=\!1,\cdots,r$)) in order to obtain feature tensors $\mathbf{\Phi}_1,\cdots,\mathbf{\Phi}_\tau$. These tensors are subsequently concatenated by $\odot$ along the hyper-edge mode into a multi-order feature tensor $\boldsymbol{\mathcal{M}}$. The final step is a \lei{Multi-order Multi-mode Transformer (3Mformer} from Section \ref{sec:appr}), which contains two \lei{complementary} branches, MP$\rightarrow$TP and TP$\rightarrow$MP, whose outputs are concatenated by $\odot$ and passed to the classifier. MP and TP perform the \lei{Joint-mode} Self-Attention (JmSA) with the so-called joint-mode tokens,  based
on `channel-temporal block’, `\lei{order-}channel-body joint’, `channel-hyper-edge’ and `channel-only’ pairs. To this end, MP contains also weighted pooling along hyper-edge mode by learnable matrix $\mathbf{H}$ (and $\mathbf{H}'$ in another branch). TP contains also \lei{block-temporal} pooling denoted by $g(\cdot)$ whose role is to capture \lei{block-temporal} order with average, maximum, rank pooling, \etc. In our experiments we show that such designed MP and TP are able to efficiently process hyper-edge feature representations from HoT branches. Appendix \ref{app:3mf} shows full visualization of our 3Mformer.}
%\caption{Our Multi-mode Attention with Higher-order Transformer (MMA-HoT). It contains a simple 3-layer MLP unit, three HoT blocks,a MMA unit for both multi-order and temporal-block-wise feature fusion, and an FC layer as the classifier. The MLP unit takes $T$ neighboring frames, each with $J$ 2D/3D skeleton body joints, forming one temporal block. Each temporal block is encoded by the MLP into a $d\!\times\!J$ dimensional feature map.  Subsequently, the temporal-block-wise feature maps of $d\!\times\!J\!\times\!\tau$ form different orders of skeleton features are forwarded to HoT blocks. The produced different orders of temporal-block-wise features are then concatenated ($\oplus$) and fused by MMA unit (dashed blue), which contains (i) Multi-order Attention (MoA) with order-wise product ($\otimes$) between each order of feature map and its corresponding incidence matrix, the outputs from order-wise product are further concatenated (colored by light gray shade); and (ii) Temporal-block-wise Attention (TA) with pooling (denoted as Pool \eg, rank-, second- or higher-order pooling) are performed on concatenated features (colored by blue shade). Attn block denotes the self-attention module. Note that the reshape operations are omitted for simplicity.  Finally, the feature maps from two branches are concatenated and reshaped to vector and then passed to the classifier.}
% \vspace{-0.3cm}
\label{fig:pipeline}
\end{figure*}

\section{Related Work}
Below we describe popular action recognition models for skeletal data.

\vspace{0.1cm}
\noindent{\bf Graph-based models}. Popular GCN-based models include the Attention enhanced Graph Convolutional LSTM network (AGC-LSTM)~\cite{Si_2019_CVPR}, the Actional-Structural GCN (AS-GCN)~\cite{Li_2019_CVPR}, Dynamic Directed GCN (DDGCN)~\cite{10.1007/978-3-030-58565-5_45}, Decoupling GCN with DropGraph module~\cite{10.1007/978-3-030-58586-0_32}, Shift-GCN~\cite{cheng2020shiftgcn}, Semantics-Guided Neural Networks (SGN)~\cite{Zhang_2020_CVPR2}, AdaSGN~\cite{Shi_2021_ICCV}, Context Aware GCN (CA-GCN)~\cite{Zhang_2020_CVPR1}, Channel-wise Topology Refinement Graph Convolution (CTR-GC)~\cite{chen2021channel} and a family of Efficient GCN (EfficientGCN-Bx)~\cite{9729609}.
%
Although  GCN-based models enjoy good performance, they have shortcomings, \eg, convolution and/or pooling are applied over \lei{one- or few-hop neighborhoods, \eg, ST-GCN~\cite{stgcn2018aaai}}, according to the human skeleton graph (body joints linked up according to connectivity of human body parts). Thus,  indirect links between various 3D body joints such as hands and legs are ignored. %(ii) since each AR dataset contains various action classes and each action class has many samples with different performing subjects, the position and angle of human body joints are very different; thus it is unable to meet the needs of various situations if the traditional skeleton graph structure is fixed (iii) the edges in traditional graph structure can only connect a pair of related nodes but the joints of human skeleton are not simply pairwise connections, which fails to capture the higher-order information hidden between multiple joints.
In contrast, our model is not restricted by the structure of typical human body skeletal graph. Instead, 3D body joints are nodes which form hyper-edges of orders $1$ to $r$.

\vspace{0.1cm}
\noindent{\bf Hypergraph-based models}. Representing the human body as a hypergraph is a recent trend. A hypergraph is adopted in~\cite{ijcai2020-109} to represent the 3D human body joints to exploit the kinematic relations among the adjacent and non-adjacent joints with a semi-dynamic hypergraph neural network, which captures richer information than GCNs. A hypergraph GNN \cite{9329123}  captures both spatio-temporal information and higher-order dependencies for skeleton-based action recognition. Our work is somewhat closely related to these works, but we jointly use hypergraphs of order $1$ to $r$ to obtain rich hyper-edge embeddings based on Higher-order Transformers.

\vspace{0.1cm}
\noindent{\bf Transformer-based models}. Action recognition with transformers has become popular. Examples include self-supervised video transformer~\cite{ranasinghe2022selfsupervised} that matches the features from different views, the end-to-end trainable Video-Audio-Text-Transformer (VATT)~\cite{akbari2021vatt} for learning multi-model representations from unlabeled raw video, audio and text through the multimodal contrastive losses, and the Temporal Transformer Network with Self-supervision (TTSN)~\cite{DBLP:journals/corr/abs-2112-07338}.  Motion-Transformer~\cite{10.1145/3444685.3446289} captures the temporal dependencies via a self-supervised pre-training on human actions, Masked Feature Prediction (MaskedFeat)~\cite{DBLP:journals/corr/abs-2112-09133} pre-trained on unlabeled videos with MViT-L learns abundant visual representations, and video-masked autoencoder (VideoMAE)~\cite{DBLP:journals/corr/abs-2203-12602} with vanilla ViT uses the masking strategy. In contrast to these works, we use three HoT branches of model \cite{kim2021transformers}, and we model hyper-edges of orders $1$ to $r$ by forming several multi-mode token variations in \lei{3Mformer}.

% \noindent{\bf Transformer-based AR.} In~\cite{ranasinghe2022selfsupervised}, a self-supervised video transformer is proposed. The self-supervised objective is to match the features from different views of the same video to be invariant to spatio-temporal variations by creating local and global spatio-temporal views with varying spatial sizes and frame rates. The proposed method achieves state-of-the-art performance on Kinetics-400, UCF-101, HMDB-51 and SSv2. In~\cite{akbari2021vatt}, an end-to-end trainable Video-Audio-Text-Transformer (VATT) for learning multimodel representations from unlabeled raw video, audio and text through the multimodal constractive losses is proposed. Apart from the new state-of-the-art results on Kinetics-400, -600, -700 and Moments in Time, they also show great generalizability when transferring to image classification, despite the domain gap between videos and images. A Temporal Transformer Network with Self-supervision (TTSN) is introduced in~\cite{DBLP:journals/corr/abs-2112-07338}. TTSN consists of a temporal transformer module which models the non-linear temporal dependencies among non-local frames to enhance the complex motion feature representations, and a temporal sequence self-supervision module which reverses the sequence of video frames for robust motion extraction to improve the generalization capability of the model.
% More transformer-based models, for example,  Motion-Transformer~\cite{10.1145/3444685.3446289} captures the temporal dependencies via a self-supervised pre-training on human actions, Masked Feature Prediction (MaskedFeat)~\cite{DBLP:journals/corr/abs-2112-09133} pre-trained on unlabeled videos with MViT-L that learns abundant visual representations, and video masked autoencoder (VideoMAE)~\cite{DBLP:journals/corr/abs-2203-12602} with vanilla ViT on human actions,  are proposed recently.

\vspace{0.1cm}
\noindent{\bf Attention mechanisms.} In order to improve feature representations, attention captures relationship between tokens or semantics. Fields of natural language processing and vision have driven recent developments in attention mechanisms based on transformers~\cite{NIPS2017_3f5ee243,dosovitskiy2021an}. 
%
Examples include the hierarchical Cross Attention Transformer (CAT)~\cite{DBLP:journals/corr/abs-2106-05786}, Cross-attention
by Temporal Shift with CNNs~\cite{https://doi.org/10.48550/arxiv.2204.00452}, Cross-Attention Multi-Scale Vision Transformer (CrossViT) for image classification~\cite{DBLP:journals/corr/abs-2103-14899} and Multi-Modality Cross Attention (MMCA) Network for image and sentence matching~\cite{Wei_2020_CVPR}. 
%
In GNNs, attention can be defined over edges~\cite{velickovic2018graph,conf/uai/ZhangSXMKY18} or over nodes~\cite{10.1145/3219819.3219980}. In this work, we use the attention with hyper-edges of several orders from HoT branches serving as tokens, and joint-mode attention with joint-mode tokens based on `channel-temporal block’, `\lei{order-}channel-body joint’, `channel-hyper-edge
(any order)’ and `channel-only’ pairs formed in \lei{3Mformer}.
% Our attention mechanism aims to fuse multi-order and temporal-block-wise feature representations for graph/hypergraph data.

% A basic attention unity is typically derived from the well-known self-attention mechanisms.

\section{Background}

Below we describe  foundations necessary for our work. % on which we build our model.

\vspace{0.1cm}
\noindent{\bf Notations.} 
%$\idx{K}$ stands for the index set $\{1,2,\cdots,K\}$. Capitalized bold symbols are matrices (two-mode tensor) or higher-order tensors (more than two modes). Lowercase bold symbols denote vectors, and regular fonts are scalars. 
\lei{$\idx{K}$ stands for the index set $\{1,2,\cdots,K\}$. Regular fonts are scalars; vectors are denoted by lowercase boldface letters, \eg, $\textbf{x}$; matrices by the uppercase boldface, \eg, {\textbf{M}}; and tensors by calligraphic letters, \eg, $\tM$. 
%
An $r$th-order tensor is denoted as $\tM \in \mbr{I_1 \times I_2 \times \cdots \times I_r}$,
and the mode-$m$ matricization of $\tM$ is denoted as $\tM_{(m)}\in \mbr{I_m \times (I_1 \cdots I_{m-1}  I_{m+1}  \cdots I_{r})}$.}

%\subsection{Skeletal Graph and Hypergraph Representation}
% \subsection{Higher-order Transformer (HoT)}


% $\idx{K}$ stands for the index set $\{1,2,\cdots,K\}$. Concatenation of $\alpha_i$ is denoted by $[\alpha_i]_{i\in\idx{I}}$, whereas $\mX_{:,i}$ means we extract/access column $i$ of matrix $\mD$. Calligraphic mathcal fonts denote tensors (\eg, $\tD$), capitalized bold symbols are matrices (\eg, $\mD$), lowercase bold symbols  are vectors (\eg, $\vpsi$), and regular fonts denote scalars.

% Firstly, let $G\!=\!(\bf{V}, {\bf E})$ be a graph with the vertex set $\bf{V}$  with nodes $\{v_1, \cdots, v_n\}$, and ${\bf E}$ are edges of the graph. 

% Let ${\bf A}$ and ${\bf D}$ be the adjacency and diagonal degree matrix, respectively. Let $\tilde{\bf A}\!=\!{\bf A}\!+\!{\bf I}$ be the adjacency matrix with self-loops (identity matrix) with the corresponding diagonal degree matrix $\tilde{\bf D}$ such that $\tilde{D}_{ii}\!=\!\sum_j ({\bf A}^{ij}\!+\! {\bf I}^{ij})$. Let ${\bf S}\!=\!\tilde{\bf D}^{-\frac{1}{2}} \tilde{\bf A}\tilde{\bf D}^{-\frac{1}{2}}$ be the normalized adjacency matrix with added self-loops. For the $l$-th layer, we use ${\bf \Theta}^{(l)}$ to denote the learnt weight matrix, and ${\bf \Phi}$ to denote the outputs from the graph networks. 

% \noindent{\bf Model Overview.}
% \subsection{Model Overview}
% Fig.~\ref{fig:pipeline} shows an overview of our model. Our model contains a simple 3-layer MLP unit (FC, ReLU, FC, ReLU, Dropout, FC), three HoT blocks with each HoT for each type of input \ie, body joint feature set, skeletal graph and hypergraph features, followed by Multi-order Temporal Attention (MoTA) for both multi-order and temporal-block-wise feature fusion, and an FC layer as the classifier. 
% %

% The MLP unit takes $T$ neighboring frames, each with $J$ 2D/3D skeleton body joints, forming one temporal block. In total, depending on stride $S$, we obtain some $\tau$ temporal blocks which capture the short temporal dependency, whereas the long temporal dependency is modeled with HoT and MoTA. Each temporal block is encoded by the MLP into a $d\!\times\!J$ dimensional feature map.  
% %
% Subsequently, the 1-, 2- and $m$-order feature maps of size $d\!\times\!J\!\times\!\tau$, $d\!\times\!J^2\!\times\!\tau$ and $d\!\times\!J^m\!\times\!\tau$, and skeletal graph/hypergraph information are forwarded to HoT blocks. 
% %
% The MoTA is used to further process different orders of edge/hyperedge features as well as the temporal-block-wise skeleton features learned from HoTs, which returns some human body joint feature maps. Finally, the feature maps are reshaped to vector representation and passed to the classifier. % Below we present our HoT with MoTA.






\noindent{\bf Transformer layers}~\cite{NIPS2017_3f5ee243,dosovitskiy2021an}. A transformer encoder layer   $f\!: \mbr{J\!\times\!d} \rightarrow \mbr{J\!\times\!d}$  consists of two sub-layers: (i) a self-attention  $a\!: \mbr{J\!\times\!d} \rightarrow \mbr{J\!\times\!d}$ and (ii) an element-wise feed-forward $\text{MLP}\!:\mbr{J\!\times\!d} \rightarrow \mbr{J\!\times\!d}$. For a set of $J$ nodes with ${\bf X} \!\in\!\mbr{J\!\times\!d}$, where ${\bf x}_i$ is a feature vector of node $i$, a transformer layer\footnote{Normalizations after $a(\cdot)$ \& MLP$(\cdot)$ are omitted for simplicity.} computes:
\begin{align}
    & a({\bf x}_i)\!=\!{\bf x}_i\!+\!\sum_{h=1}^H\sum_{j=1}^J\alpha_{ij}^h{\bf x}_j{\bf W}_h^V{\bf W}_h^O, \label{eq:transformer-attn}\\
    & f({\bf x}_i)\!=\!a({\bf x}_i)\!+\!\text{MLP}(a({\bf X}))_i, \label{eq:transformer-enc}
\end{align}
where $H$ and $d_H$ denote respectively the number of heads and the head size, ${\boldsymbol{\alpha}}^h\!=\!\sigma\big({\bf X}{\bf W}_h^Q({\bf X}{\bf W}_h^{K})^\top\big)$ is the attention coefficient, ${\bf W}_h^O\!\in\!\mbr{d_H\!\times\!d}$, and ${\bf W}_h^V$, ${\bf W}_h^K$, ${\bf W}_h^Q \!\in\!\mbr{d\!\times\!d_H}$. 
%Note that the normalization after Attn$(\cdot)$ and MLP$(\cdot)$ are omitted for simplicity.

% Eq.~\eqref{eq:transformer-attn}



\vspace{0.1cm}
\noindent{\bf Higher-order transformer layers}~\cite{kim2021transformers}. Let the HoT layer be $f_{m\rightarrow n}\!:\mbr{J^m \!\times\!d}\!\rightarrow\!\mbr{J^n\!\times\!d}$ with two sub-layers: (i) a higher-order self-attention $a_{m\rightarrow n}\!:\mbr{J^m \!\times\!d}\!\rightarrow\!\mbr{J^n\!\times\!d}$ and (ii) a feed-forward  $\text{MLP}_{n\rightarrow n}\!:\mbr{J^n \!\times\!d}\!\rightarrow\!\mbr{J^n\!\times\!d}$. Moreover, let indexing vectors ${\bf i}\in\idx{J}^m\equiv\idx{J}\!\times\!\idx{J}\!\times\!\cdots\!\times\!\idx{J}$ ($m$ modes) and ${\bf j}\in\idx{J}^n\equiv\idx{J}\!\times\!\idx{J}\!\times\!\cdots\!\times\!\idx{J}$ ($n$ modes). For the input tensor ${\bf X}\!\in\!\mbr{J^m\!\times\!d}$ with hyper-edges of order $m$, a HoT layer evaluates:
%
\begin{align}
    & a_{m \rightarrow n}({\bf X})_{\boldsymbol j}\!=\!\sum_{h=1}^H\sum_\mu\sum_{\boldsymbol i}{\boldsymbol{\alpha}}_{{\boldsymbol i}, {\boldsymbol j}}^{h,\mu}{\bf X}_{\boldsymbol i}{\bf W}_{h, \mu}^V{\bf W}_{h, \mu}^O \label{eq:hot-attn}\\
    & \text{MLP}_{n\rightarrow n}(a_{m\rightarrow n}({\bf X}))\!=\!\text{L}_{n\rightarrow n}^2(\text{ReLU}(\text{L}_{n\rightarrow n}^1(a_{m\rightarrow n}({\bf X})))), \label{eq:hot-mlp}\\
    & f_{m\rightarrow n}({\bf X})\!=\!a_{m \rightarrow n}({\bf X})\!+\!\text{MLP}_{n\!\rightarrow \!n}(a_{m\rightarrow n}({\bf X})), \label{eq:hot-enc}
\end{align}
where ${\boldsymbol \alpha}^{h, \mu}\!\in\!\mbr{J^{m+n}}$ is the so-called attention coefficient tensor with multiple heads, and ${\boldsymbol \alpha}^{h, \mu}_{{\bf i},{\bf j}}\!\in\!\mbr{J}$ is a vector, ${\bf W}_{h, \mu}^V\!\in\!\mbr{d\!\times\!d_H}$ and ${\bf W}_{h, \mu}^O\!\in\!\mbr{d_H\!\times\!d}$ are learnable parameters. Moreover, $\mu$ indexes over the so-called equivalence classes of order-$(m+n)$ in the same partition of nodes,  $\text{L}_{n\rightarrow n}^1\!:\mbr{J^n\!\times\!d}\rightarrow \mbr{J^n\!\times\!d_F}$ and $\text{L}_{n\rightarrow n}^2\!:\mbr{J^n\!\times\!d_F}\rightarrow \mbr{J^n\!\times\!d}$ are equivariant linear layers and $d_F$ is the hidden dimension.
%, $H$ and $d_H$ denote the number of heads and the head size, respectively.

To compute each attention tensor ${\boldsymbol \alpha}^{h,\mu}\!\in\!\mbr{J^{m+n}}$ from the input tensor ${\bf X}\!\in\!\mbr{J^m\!\times\!d}$ of hyper-edges of order $m$, from the higher-order query and key, we obtain:
%
\begin{equation}
  {{\boldsymbol \alpha}_{{\boldsymbol i}, {\boldsymbol j}}^{h,\mu}} \!=\!
    \begin{cases}
      \frac{\sigma({\bf Q}_{\boldsymbol j}^{h,\mu}, {\bf K}_{\boldsymbol i}^{h,\mu})}{Z_{\boldsymbol j}}\;\quad({\boldsymbol i}, {\boldsymbol j}) \!\in\! \mu\\
      \quad\quad 0 \quad\quad\;\text{otherwise},
    \end{cases}
    \label{eq:att-tensor}
\end{equation}
where ${\bf Q}^\mu\!=\!\text{L}_{m\rightarrow n}^\mu({\bf X})$, ${\bf K}^\mu\!=\!\text{L}_{m\rightarrow m}^\mu({\bf X})$, and normalization constant $Z_{\boldsymbol j}\!=\!\sum_{{\boldsymbol i}:({\boldsymbol i}, {\boldsymbol j})\in \mu}\sigma({\bf Q}_{\boldsymbol j}^\mu, {\bf K}_{\boldsymbol i}^\mu)$. Finally, kernel attention  in Eq.~\eqref{eq:att-tensor} can be approximated with RKHS feature maps $\psi\in\mbrp{d_K}$ for efficacy as $d_K\ll d_H$. Specifically, we have $\sigma({\bf Q}_{\boldsymbol j}^{h,\mu}, {\bf K}_{\boldsymbol i}^{h,\mu})\approx{\boldsymbol \psi}({\bf Q}_{\boldsymbol j}^{h,\mu})^\top{\boldsymbol \psi}({\bf K}_{\boldsymbol i}^{h,\mu})$ as in \cite{pmlr-v119-katharopoulos20a,choromanski2021rethinking}. %RKHS feature maps $\psi\in\mbrp{d_K}$ and $d_K\ll d_H$ is the kernel feature map. Follow~\cite{kim2021transformers}, 
We simply choose the performer kernel~\cite{choromanski2021rethinking} given its theoretical and empirical guarantee. %, but the choice of kernel is flexible.

As the query and key tensors are computed from the input tensor ${\bf X}$ using the equivariant linear layers, the transformer encoder layer $f_{m\rightarrow n}$ satisfies the permutation equivariance.

%Finally, kernel attention  in Eq.~\eqref{eq:att-tensor} can be approximated with RKHS feature maps for efficacy, as demonstrated in \cite{pmlr-v119-katharopoulos20a,choromanski2021rethinking}:
%
%\begin{equation}
%  {\bf {\boldsymbol \alpha}_{{\boldsymbol i}, {\boldsymbol j}}^\mu} =
%    \begin{cases}
%      \frac{\psi({\bf Q}_{\boldsymbol j}^\mu)^\top\psi({\bf K}_{\boldsymbol i}^\mu)}{Z_{\boldsymbol j}} \quad({\boldsymbol i}, {\boldsymbol j}) \in \mu\\
%      \quad\quad\quad 0 \quad\quad\quad\;\text{otherwise}
%    \end{cases}
%    \label{eq:kernel-trick}
%\end{equation}
%where $\psi$: $\mbr{d_H} \! \rightarrow\!\mbrp{d_K}$ is the kernel feature map. Follow~\cite{kim2021transformers}, we simply choose Performer kernel~\cite{choromanski2021rethinking} given its theoretical and empirical guarantee, but the choice of kernel is flexible. Then the higher-order self-attention Attn$_{m\rightarrow n}({\bf X})$ in Eq.~\eqref{eq:hot-attn} becomes:

%\begin{align}
%    & \text{Attn}_{m\rightarrow n}({\bf X})_{\boldsymbol j}\!=\!\sum_\mu Z_{\boldsymbol j}^{-1}\sum_{{\boldsymbol i} | ({\boldsymbol i}, {\boldsymbol j})\in \mu}\psi({\bf Q}_j^\mu)^\top\psi({\bf K}_{\boldsymbol i}^\mu){\bf X}_{\boldsymbol i}W_\mu^VW_\mu^O \nonumber \\
%    & \qquad\qquad\qquad\!\!\! =\sum_\mu Z_{\boldsymbol j}^{-1}\psi({\bf Q}_j^\mu)^\top\sum_{{\boldsymbol i} | ({\boldsymbol i}, {\boldsymbol j})\in \mu}\psi({\bf K}_{\boldsymbol i}^\mu){\bf X}_{\boldsymbol i}W_\mu^VW_\mu^O \label{eq:kernel-trick-2}% \\
%    % & \qquad\qquad\qquad\!\!\! \text{where } Z_{\boldsymbol j}=\psi({\bf Q}_j^\mu)^\top\sum_{{\boldsymbol i} | ({\boldsymbol i}, {\boldsymbol j})\in\mu}\psi({\bf K}_{\boldsymbol i}^\mu). \nonumber
%\end{align}
%where $Z_{\boldsymbol j}=\psi({\bf Q}_j^\mu)^\top\sum_{{\boldsymbol i} | ({\boldsymbol i}, {\boldsymbol j})\in\mu}\psi({\bf K}_{\boldsymbol i}^\mu)$, and the inner-summations in Eq.~\eqref{eq:kernel-trick-2} is over the key index ${\boldsymbol i}$, which is coupled with the query index ${\boldsymbol j}$ by $({\boldsymbol i}, {\boldsymbol j})\in \mu$. To further reduce the computational bottleneck, we use the approximation to decouple the key and query indices and take the inner-summations over $\mathcal{I} =\cup_{\boldsymbol j}\{{\boldsymbol i} | ({\boldsymbol i}, {\boldsymbol j})\in \mu\}$ instead of the computation for every ${\boldsymbol j}$-th query:

%\begin{equation}
%    \text{Attn}_{m\rightarrow n}({\bf X})_{\boldsymbol j}\!\approx\!\sum_\mu Z_{\boldsymbol j}^{-1}\psi({\bf Q}_j^\mu)^\top\sum_{i \in \mathcal{I}}\psi({\bf K}_{\boldsymbol i}^\mu){\bf X}_{\boldsymbol i}W_\mu^VW_\mu^O
%\end{equation}
%where $Z_{\boldsymbol j}\!\!\approx\!\!\psi({\bf Q}_j^\mu)^\top\sum_{i \in \mathcal{I}}\psi({\bf K}_{\boldsymbol i}^\mu)$. With this approximation, we are able to reduce the cost since we only need to compute the summations $\sum_{{\boldsymbol i}\in\mathcal{I}}\psi({\bf K}_i^\mu)$ and $\sum_{{\boldsymbol i}\in\mathcal{I}}\psi({\bf K}_i^\mu){\bf X}_{\boldsymbol i}$ once and simply reuse them across all the query indices ${\boldsymbol j}$.

%As we deal with the temporal-block-wise skeleton representations, for 1-, 2- and $m$-order inputs, we concatenate the generated higher-order feature maps to form our multi-order temporal-block-wise (edge/hyperedge) feature representations. % Below we introduce our MoTA.

\section{Approach}
\label{sec:appr}

% Fig.~\ref{fig:pipeline} is our model overview. It contains a simple 3-layer MLP unit (FC, ReLU, FC, ReLU, Dropout, FC), three HoT blocks with each HoT for each type of input, \ie, body joint feature set, graph and hypergraph of body joints, followed by Multi-mode Attention (MMA) with two modules, Multi-order Pooling (MP) and Temporal block Pooling (TP), whose goal is to form joint-mode tokens such as `channel-temporal block', `channel-body joint', `channel-hyper-edge (any order)' and `channel', and perform weighted hyper-edge aggregation and temporal block aggregation. Their outputs are passed to an FC layer (the classifier).

\lei{Skeletal Graph~\cite{stgcn2018aaai} and Skeletal Hypergraph~\cite{ijcai2020-109,9329123} are very popular for modeling edges and hyper-edges. Nonetheless, in this work, we use the Higher-order Transformer (HoT)~\cite{kim2021transformers} as a backbone encoder.}
\subsection{\lei{Model Overview}}
\vspace{0.1cm}
\noindent\lei{As shown in Fig.~\ref{fig:pipeline}, our framework contains a simple 3-layer MLP unit (FC, ReLU, FC, ReLU, Dropout, FC), three HoT blocks with each HoT for each type of input (\ie, body joint feature set, graph and hypergraph of body joints), followed by Multi-order Multi-mode Transformer (3Mformer) with two modules (i) Multi-order Pooling (MP) and (ii) Temporal block Pooling (TP). 
%
The goal of 3Mformer is to form joint-mode tokens (explained later) such as `channel-temporal block', `order-channel-body joint', `channel-hyper-edge (any order)' and `channel-only', and perform weighted hyper-edge aggregation and temporal block aggregation. Their outputs are further concatenated and passed to an FC layer for classification.}

\vspace{0.1cm}
\noindent\textbf{MLP unit.} 
The MLP unit takes $T$ neighboring frames, each with $J$ 2D/3D skeleton body joints, forming one temporal block. In total, depending on stride $S$, we obtain some $\tau$ temporal blocks (a block captures the short-term temporal evolution), In contrast, the long-term temporal evolution is modeled with HoT and \lei{3Mformer}. Each temporal block is encoded by the MLP into a $d\!\times\!J$ dimensional feature map.  

\vspace{0.1cm}
\noindent\textbf{HoT branches.} 
We stack  $r$  branches of  HoT, each taking  embeddings ${\bf X}_t\in\mbr{d\!\times\!J}$ where  $t\in\idx{\tau}$ denotes a temporal block. 
HoT branches output hyper-edge feature representations of size $m\in\idx{r}$ as ${\bf\Phi}'_m\in\mbr{J^m \times d'}$ for order $m\in\idx{r}$. % Each HoT branch $m$ applies ${\bf\Phi}'_m=f_{m\rightarrow m}(\cdots f_{m\rightarrow m}(f_{1\rightarrow m}(\bf{X}_{t}))\cdots)$.

For the first-, second- and higher-order stream outputs ${\bf\Phi}'_1,\cdots,{\bf\Phi}'_r$, we (i) swap feature channel and hyper-edge modes, (ii) extract the upper triangular of tensors, and we concatenate along the block-temporal mode, so we have ${\bf\Phi}_m\in\mbr{d'\times N_{E_m}\times\tau}$, where $N_{E_m}\!=\!\binom{J}{m}$. Subsequently, we concatenate ${\bf\Phi}_1,\cdots,{\bf\Phi}_r$ along the hyper-edge mode and obtain a multi-order feature tensor $\lei{\tM}\!\in\!\mbr{d'\!\times\!N\!\times\!\tau}$ where the total number of hyper-edges across all orders is $N=\sum_{m=1}^r\binom{J}{m}$. 

\vspace{0.1cm}
\noindent\textbf{3Mformer.} Our Multi-order Multi-mode Transformer (3Mformer) with Joint-mode Self-Attention (JmSA) is used for the fusion of information flow inside the multi-order feature tensor ${\tM}$, and finally, the output from 3Mformer is passed to a classifier for classification.

\subsection{\lei{Joint-mode Self-Attention}}

\vspace{0.1cm}
\noindent\lei{\textbf{Joint-mode tokens.} We are inspired by the attentive regions of the one-class token in the standard Vision Transformer (ViT)~\cite{NIPS2017_3f5ee243} that can be leveraged to form a class-agnostic localization map. We investigate if the transformer model can also effectively capture the joint-mode attention for more discriminative classification tasks, \eg, tensorial skeleton-based action recognition by learning the joint-mode tokens within the transformer. 
%
To this end, we propose a Multi-order Multi-mode Transformer (3Mformer), which uses joint-mode tokens to jointly learn various higher-order motion dynamics among channel-, block-temporal-, body joint- and order-mode. Our 3Mformer can successfully produce joint-mode relationships from Joint-Mode Self-Attention (JmSA) mechanism corresponding to different tokens. Below we introduce our JmSA.
}

\lei{Given the order-$r$ tensor $\tM \in \mbr{I_1 \times I_2 \times \cdots \times I_r}$, to form the joint mode token, we perform the mode-$m$ matricization of $\tM$ to obtain $\textbf{M} \equiv \tM_{(m)}^\top \in \mbr{(I_1 \cdots I_{m-1}  I_{m+1}  \cdots I_{r}) \times I_m}$, and the joint-token for $\textbf{M}$ is formed. 
%
For example, for a given 3rd-order tensor that has feature channel-, hyper-edge- and temporal block-mode, we can form  `channel-temporal block', `channel-hyper-edge (any order)' and `channel-only' pairs; and if the given tensor is used as input and outputs a new tensor which produces new mode, \eg, body joint-mode, we can form the `order-channel-body joint' token.
% as `$I_1-\cdots-I_{m-1}-I_{m+1}-\cdots-I_{r}$'. 
In the following sections, for simplicity, we use reshape for the matricization of tensor to form different types of joint-mode tokens.  
%
Our JmSA is given as:
\begin{equation}
    \!\!a({\bf Q}, {\bf K}, {\bf V}) \!=\!\text{SoftMax}\left(\frac{{\bf Q}{\bf K}^\top}{\sqrt{d_{K}}}\right)\!{\bf V},\label{eq:jmsa}\!
\end{equation}
where $\sqrt{d_{K}}$ is the scaling factor, ${\bf Q}\!=\!{\bf W}^q{\bf M}$, ${\bf K}\!=\!{\bf W}^k{\bf M}$ and ${\bf V}\!=\!{\bf W}^v{\bf M}$ are the query, key and value, respectively, and $\textbf{M} \equiv \tM_{(m)}^\top$. Moreover, ${\bf Q}$, ${\bf K}$, ${\bf V}\!\in\!\mbr{(I_1 \cdots I_{m-1}  I_{m+1}  \cdots I_{r}) \times I_m}$ and ${\bf W}^q$, ${\bf W}^k$, ${\bf W}^v\!\in\!\mbr{(I_1 \cdots I_{m-1}  I_{m+1}  \cdots I_{r}) \times (I_1 \cdots I_{m-1}  I_{m+1}  \cdots I_{r})}$  are learnable weights. We notice that various joint-mode tokens have different `focus' of attention mechanisms, and we apply them in our 3Mformer for the fusion of multi-order feature representations.
}


%are $\Phi_1 \!\in\!\mbr{d'\!\times\!N_{E_1}\!\times\!\tau}$, $\Phi_2 \!\in\!\mbr{d'\!\times\! N_{E_2}\!\times\!\tau}$ and $\Phi_3 \!\in\!\mbr{d'\!\times\!N_{E_m}\!\times\!\tau}$, respectively, where $N_{E_2}\!=\!\binom{J}{2}\!=\!J(J-1)/2$ is the number of edges and $N_{E_m}\!=\!\binom{J}{m}$ is the number of hyperedges ($N_{E_m}$ depends on the order-$m$ in hypergraph setup \eg, for order-3 hypergraph, $N_{E_3}\!=\!\binom{J}{3}\!=\!J(J-1)(J-2)/6$).
%
%We concatenate the outputs of multi-order edge/hyperedge features obtained from HoT blocks into ${\bf M}\!\in\!\mbr{d'\!\times\!N\!\times\!\tau}$ as our multi-order temporal-block-wise feature representation, where $N\!=\!N_{E_1}\!+\!N_{E_2}\!+\!\cdots\!+\!N_{E_m}$ denotes the concatenated dimension among edge/hyperedge and $d'$ is the feature dimension output from HoTs.


%Subsequently, the 1-, 2- and $m$-order feature maps of size $d\!\times\!J\!\times\!\tau$, $d\!\times\!J^2\!\times\!\tau$ and $d\!\times\!J^m\!\times\!\tau$, and skeletal graph/hypergraph information are forwarded to HoT blocks. 
%
%The MoTA is used to further process different orders of edge/hyperedge features as well as the temporal-block-wise skeleton features learned from HoTs, which returns some human body joint feature maps. Finally, the feature maps are reshaped to a vector and passed to the classifier. % Below we present our HoT with MoTA.
%Below we briefly introduce the skeletal graph representation and the concept of our skeletal hypergraph. Then we present the details of the proposed HoT with MoTA. First, we introduce our notations. %for graph/hypergraph.


%\vspace{0.1cm}
%\noindent\textbf{Our hyper-edges.} 

%where ${\bf X}_v$ is the feature vector of node $v$
%where ${\bf X}_{i_1, i_2}$ is the feature of edge $(i_1, i_2)$
%where ${\bf X}_{i_1,\cdots,i_m}$ is the feature of hyperedge $(i_1,\!\cdots,\!i_m)$. Our tensor representations of skeletal hypergraph contain very rich information which are later learned by HoT.
%By ${\bf x}_e$, we denote the feature representation  of hyper-edge $e\in E_h^m$. 



%\subsection{Softmax, Second- and Third-order Pooling}

\subsection{\lei{Multi-order Multi-mode Transformer}}

% \begin{figure}[t]
% \centering%%%%
% %\vspace{-0.3cm}
% %\vspace{0.1cm}
% \begin{subfigure}[b]{0.495\linewidth}
% \includegraphics[trim=0 0 0 0, clip=true,height=1.48cm]{imgs/pipeline1.pdf}% \vspace{-0.2cm}
% \caption{\label{fig:option1} Joint attention}
% %\vspace{-0.2cm}
% \end{subfigure}
% %
% \begin{subfigure}[b]{0.495\linewidth}
% \includegraphics[trim=0 0 0 0, clip=true,height=1.48cm]{imgs/pipeline2.pdf}% \vspace{-0.2cm}
% \caption{\label{fig:option2} Separate attention}
% %\vspace{-0.2cm}
% \end{subfigure}
% \caption{Sequential and parallel attention of MoTA.
% }
% %\vspace{-0.3cm}
% \label{fig:cross-attn-options}
% \end{figure}


%
Below we introduce \lei{Multi-order Multi-mode Transformer (3Mformer) with  Multi-order Pooling (MP) block and Temporal block Pooling (TP) block, which are cascaded into two branches (i) MP$\rightarrow$TP and (ii) TP$\rightarrow$MP, to achieve different types of  joint-mode tokens.}

\subsubsection{\lei{Multi-order Pooling (MP) Module}}

\vspace{0.1cm}
\noindent{\bf \lei{JmSA} in MP.} We reshape the multi-order feature representation $\lei{\tM}\!\in\!\mbr{d'\!\times\!N\!\times\!\tau}$ into ${\bf M}\!\in\!\mbr{d'\tau\!\times\!N}$ (or reshape the output from TP explained later into ${\bf M}'\!\in\!\mbr{d'\!\times\!N}$) to let the model attend to different types of feature representations. Let us simply denote $d''\!=\!d'\tau$ (or $d''\!=\!d'$) depending on the source of input. We form an joint-mode \lei{self-attention} (if $d''=d'\tau$, we have, \ie, `channel-temporal block' token\lei{; if $d''=d'$, we have `channel-only' token}):
\begin{equation}
\!\!a_\text{MP}({\bf Q}_\text{MP}, {\bf K}_\text{MP}, {\bf V}_\text{MP}) \!=\!\text{SoftMax}\left(\frac{{\bf Q}_\text{MP}{\bf K}_\text{MP}^\top}{\sqrt{d_{K_\text{MP}}}}\right)\!{\bf V}_\text{MP},\label{eq:order-attn}\!
\end{equation}
where $\sqrt{d_{K_\text{MP}}}$ is the scaling factor, ${\bf Q}_\text{MP}\!=\!{\bf W}_\text{MP}^q{\bf M}$, ${\bf K}_\text{MP}\!=\!{\bf W}_\text{MP}^k{\bf M}$ and ${\bf V}_\text{MP}\!=\!{\bf W}_\text{MP}^v{\bf M}$  (we can use here ${\bf M}$ or ${\bf M}'$) are the query, key and value. Moreover, ${\bf Q}_\text{MP}$, ${\bf K}_\text{MP}$, ${\bf V}_\text{MP}\!\in\!\mbr{d''\times N}$ and ${\bf W}_\text{MP}^q$, ${\bf W}_\text{MP}^k$, ${\bf W}_\text{MP}^v\!\in\!\mbr{d''\times d''}$  are learnable weights. Eq.~\eqref{eq:order-attn} is a self-attention layer  which reweighs ${\bf V}_\text{MP}$ based on the correlation between  ${\bf Q}_\text{MP}$ and ${\bf K}_\text{MP}$ token embeddings of so-called joint-mode tokens. 
% The MoA produces feature representation ${\bf O}_\text{ord}\!\in\!\mbr{d'\tau\times N}$.

\vspace{0.1cm}
\noindent{\bf Weighted pooling.} Attention layer in Eq.~\eqref{eq:order-attn} produces feature representation ${\bf O}_\text{MP}\!\in\!\mbr{d''\times N}$ to enhance the relationship between for example feature channels and body joints. Subsequently, we handle the impact of hyper-edges of multiple orders by weighted pooling along hyper-edges of order $m\in\idx{r}$: % between each order of feature map and the corresponding incidence matrix, followed by a concatenation of the resulting temporal-block-wise body joint feature representations:
%
\begin{align}
    & {\bf O}_\text{MP}^{*(m)}\!=\!{\bf O}_\text{MP}^{(m)}{\bf H}^{(m)}\!\in\! \mbr{d''\times J}, %~(\text{or}~{\bf O}_\text{MP}^{*'(k)}\!=\!{\bf O}_\text{MP}^{'(k)}{\bf H}^{(k)}\!\in\!\mbr{d'\times J})
    \label{eq:ord-wise}
    %& {\bf O}^*_\text{MP}\!=\!\left[\oplus_{k\in\idx{m}}{\bf O}_\text{MP}^{*(k)}\right]\!\in\! \mbr{3d'\tau\times J}~(\text{or}~{\bf O}^{*'}_\text{MP}\!=\!\left[\oplus_{k\in\idx{m}}{\bf O}_\text{MP}^{*'(k)}\right]\!\in\! \mbr{3d'\!\times\!J})\label{eq:concat}
\end{align}
where ${\bf O}_\text{MP}^{(m)}\!\in\!\mbr{d''\times N_{E_m}}$ is simply extracted from ${\bf O}_\text{MP}$ for hyper-edges of order $m$, and  matrices ${\bf H}^{(m)}\!\in\!\mbr{N_{E_m}\times J}$ are learnable weights to perform weighted pooling along hyper-edges of order $m$. Finally, we obtain ${\bf O}_\text{MP}^{*}\!\in\!\mbr{r{d''\times J}}$ by simply concatenating ${\bf O}_\text{MP}^{*(1)},\cdots,{\bf O}_\text{MP}^{*(r)}$. If we used the input to MP from TP, then we denote the output of MP as ${{\bf O}'}_\text{MP}^{*}$.
%Finally we reshape it to form the temporal-block-wise representation ${\bf M'}\!\in\!\mbr{3d'J\!\times\!\tau}$.
%
%Note that both temporal-block-wise attention and multi-order attention produce exactly the same dimension of feature representations ${\bf O}'\!\in\!\mbr{d'\!\times\!N\!\times\!\tau}$,  between the reshaped multi-order representations ${\bf O}^*\!\in\!\mbr{d'\tau\!\times\!N}$ and the concatenated incidence matrix ${\bf H}^*\!\in\!\mbr{N\!\times\!J}$ of graph/hypergraph. We then reshape it to form the multi-order representation ${\bf M'}\!\in\!\mbr{d'J\!\times\!\tau}$.

%, and we can reshape it back to the input dimension ${\bf O}'_\text{MP}\!\in\!\mbr{d'\!\times\!N\!\times\!\tau}$.

\subsubsection{\lei{Temporal block Pooling (TP) Module}}

\vspace{0.1cm}
\noindent{\bf \lei{JmSA} in TP.} 
Firstly, we  reshape the multi-order feature representation $\lei{\tM}\!\in\!\mbr{d'\!\times\!N\!\times\!\tau}$ into ${\bf M}\!\in\!\mbr{d'N\!\times\!\tau}$ (or reshape the output from MP into ${\bf M}''\!\in\!\mbr{rd'J\!\times\!\tau}$).
% The resulting temporal-block-wise feature representation from MoA is ${\bf M'}\!\in\!\mbr{3d'J\!\times\!\tau}$, we set it to be ${\bf M}_\text{TP}\!=\!{\bf M}'$ 
For simplicity, we denote $d'''\!=\!d'N$ in the first case and $d'''\!=\!rd'J$ in the second case. As the first mode of reshaped input serves to form tokens, they are again joint-mode tokens, \eg, `channel-hyper-edge' and `order-channel-\lei{body} joint' tokens, respectively.
Moreover, TP also performs pooling along block-temporal mode (along $\tau$). We form \lei{an joint-mode self-attention}:
%
\begin{equation}
    a_\text{TP}({\bf Q}_\text{TP}, {\bf K}_\text{TP}, {\bf V}_\text{TP}) \!=\!\text{SoftMax}\left(\frac{{\bf Q}_\text{TP}{\bf K}_\text{TP}^\top}{\sqrt{d_{K_\text{TP}}}}\right)\!{\bf V}_\text{TP},
    \label{eq:temp-attn}
\end{equation}
where $\sqrt{d_{K_\text{TP}}}$ is the scaling factor, ${\bf Q}_\text{TP}\!=\!{\bf W}_\text{TP}^q{\bf M}$, ${\bf K}_\text{TP}\!=\!{\bf W}_\text{TP}^k{\bf M}$ and ${\bf V}_\text{TP}\!=\!{\bf W}_\text{TP}^v{\bf M}$ (we can use here ${\bf M}$ or ${\bf M}''$) are the query, key and value. Moreover, ${\bf Q}_\text{TP}$, ${\bf K}_\text{TP}$, ${\bf V}_\text{TP}\!\in\!\mbr{d'''\times \tau}$ %(or $\mbr{3d'J\!\times\!\tau}$) 
and ${\bf W}_\text{TP}^q$, ${\bf W}_\text{TP}^k$, ${\bf W}_\text{TP}^v\!\in\!\mbr{d'''\times d'''}$ %(or $\mbr{3d'J\times 3d'J}$) 
are learnable weights.  Eq.~\eqref{eq:temp-attn} reweighs  ${\bf V}_\text{TP}$ based on the correlation between ${\bf Q}_\text{TP}$ and ${\bf K}_\text{TP}$ token embeddings of joint-mode tokens (`channel-hyper-edge' or `order-channel-body joint'). 
%
The output of attention is the temporal representation ${\bf O}_\text{TP} \!\in\! \mbr{d'''\times \tau}$. 
If we used ${\bf M}''$ as input, we denote the output as ${\bf O}''_\text{TP}$.
% (or ${\bf O}'_\text{TP} \!\in\! \mbr{d'N\!\times\!\tau}$).
%, and we reshape it to form temporal representation ${\bf O}'_\text{TP} \!\in\! \mbr{d'\!\times\!N\!\times \tau}$.

\vspace{0.1cm}
\noindent{\bf Pooling step.} Given the temporal representation ${\bf O}_\text{TP}\!\in\!\mbr{d'''\!\times\!\tau}$ (or ${\bf O}''_\text{TP}$), %(or ${\bf O}'_\text{TP} \!\in\! \mbr{d'N\!\times\!\tau}$), 
we apply pooling along the block-temporal mode to obtain compact feature representations independent of length (block count $\tau$) of skeleton sequence. There exist many pooling operations\footnote{We do not propose pooling operators but we  select popular ones with the purpose of comparing their impact on TP.} including first-order, \eg, average, maximum, sum pooling, second-order~\cite{Gao_2019_CVPR,NEURIPS2018_17c276c8} such as attentional pooling~\cite{Girdhar_17b_AttentionalPoolingAction}, higher-order (tri-linear)~\cite{7926605, DBLP:journals/corr/abs-2110-05216} and rank pooling~\cite{10.1109/TPAMI.2016.2558148}. The output  after pooling is ${\bf O}^*_\text{TP}\!\in\!\mbr{d'''}\!$ (or ${{\bf O}''}^*_\text{TP}$). %(or ${\bf O}^{*'}_\text{TP}\!\in\!\mbr{d'N}$). 
%We simply choose some representatives \ie, avg-/max-, attentional-, trilinear- and rank-pooling for comparisons.
%, then we reshape the output to be a vector representation and pass it to the classifier.

% reshape further pass our multi-order feature representation ${\bf M}\!\in\!\mbr{d'\times N}$ into an FC layer to obtain the edge/hyperedge relationship representation of $d'\times d^*$, where ${\bf W}_\text{FC}\!\in\!\mbr{N\times d^*}$ is the learnable weights of FC.



% We also introduce the commonly used weighted average and max pooling for the fusion of different order of features and set them as our baseline for comparisons.

% \subsection{Sequential and Parallel Attentions in MoTA}
\subsubsection{\lei{Model Variants}}

We devise four model variants by different stacking of MP with TP, with the goal of exploiting attention with different kinds of joint-mode tokens:
\begin{enumerate}
    \item Single-branch: MP followed by TP, denoted MP$\rightarrow$TP, (Fig.~\ref{fig:pipeline} top right branch).
    \item Single-branch: TP followed by MP, denoted TP$\rightarrow$MP,  
    %which we first perform attention and pooling along the temporal then performing multi-order feature fusion 
    (Fig.~\ref{fig:pipeline} bottom right branch). % The outputs from (a) and (b) are exactly the same, which is ${\bf M}^\dag\!\in\!\mbr{3d'\!\times\!J}$. 
    %We denote these two MoTA variants as MoTA-SeqA and MoTA-SeqB respectively. 
    \item Two-branch (\lei{our 3Mformer,} Fig.~\ref{fig:pipeline}) which concatenates outputs of  MP$\rightarrow$TP and TP$\rightarrow$MP.
    %The final output from this parallel attention is a concatenation of both fused representations, which is ${\bf M}^\ddag\!\in\!\mbr{6d'\!\times\!J}$. 
    %We simply denote this variant as MoTA-Paral (\aka MoTA).
    \item We also investigate only MP or TP module followed by average pooling or an FC layer.
\end{enumerate}
The outputs from MP$\rightarrow$TP and TP$\rightarrow$MP have exactly the same feature dimension ($\mbr{rd'J}$, after reshaping into vector). For two-branch (\lei{our 3Mformer}), we simply concatenate these outputs ($\mbr{2rd'J}$, after concatenation). These vectors are forwarded to the FC layer to learn a classifier.
%
% In this section, we perform cross attention on two variants: (i) joint attention on ${\bf M}$, where both cross-block attention and cross-order attention are performed on one single multi-order temporal-block-wise feature representation, resulting in ${\bf M}^\dag\!\in\!\mbr{d'\!\times\!J}$ and (ii) separate attention on two copies of ${\bf M}$, where the cross-block attention is performed on the first copy ${\bf M}_\text{TP}$, cross-order attention is performed on the second copy ${\bf M}_\text{ord}$, and the resulting outputs are concatenated to form ${\bf M}^\ddag\!\in\!\mbr{2d'\!\times\!J}$ preceding the classifier. 
% Fig.~\ref{fig:cross-attn-options} shows these variants we compare in experiments.



\section{Experiments}

\subsection{Datasets and Protocols} % and Experimental Setup}

\noindent{{\bf (i) NTU RGB+D (NTU-60)}}~\cite{Shahroudy_2016_NTURGBD} contains 56,880 video sequences.% and over 4 million frames. 
This dataset has variable sequence lengths  and high intra-class variations. Each skeleton sequence has 25 joints and there are no more than two human subjects in each video. Two evaluation protocols are: (i) cross-subject (X-Sub) and (ii) cross-view (X-View).

%\vspace{0.05cm}
% \noindent
\noindent{{\bf (ii) NTU RGB+D 120 (NTU-120)}}~\cite{Liu_2019_NTURGBD120}, an extension of NTU-60, contains 120 action classes (daily/health-related), and 114,480 RGB+D video samples  captured with 106 distinct human subjects from 155 different camera viewpoints. There are also two evaluation protocols: (i) cross-subject (X-Sub) and (ii) cross-setup (X-Set).

\noindent{{\bf (iii) Kinetics-Skeleton}} is based on Kinetics~\cite{kay2017kinetics}, a large-scale dataset that contains 300,000 video clips and  up to 400 human actions collected from YouTube. This dataset involves human daily activities, sports scenes and complex human-computer interaction scenes. Since Kinetics only provides raw videos without the skeletons, ST-GCN~\cite{stgcn2018aaai} uses the publicly available OpenPose toolbox~\cite{Cao_2017_CVPR} to estimate and extract the location of 18 human body joints on every frame in the clips. We use their released skeleton data to evaluate our model. Following the standard evaluation protocol, we report the Top-1 and Top-5 accuracies on the validation set.

% \noindent{{\bf Kinetics 400}}~\cite{kay2017kinetics} dataset contains about 300,000 video clips, which includes up to 400 human actions, collected from YouTube. This dataset involves human daily activities, sports scenes and complex human-computer interaction scenes. Since Kinetics only provides raw videos without the skeleton sequences. ST-GCN~\cite{stgcn2018aaai} uses the OpenPose toolbox~\cite{Cao_2017_CVPR} to extract the human body joints. We use the skeleton data (Kinetics-Skeleton) to evaluate our model. We report the Top-1 and Top-5 accuracies on the test set.


% \subsection{Experimental Setup}
\subsection{Experimental Setup} 
We use PyTorch framework to perform our experiments. We simply use the Stochastic Gradient Descent (SGD)  with momentum 0.9 as the optimizer, select cross-entropy as the loss function, set the weight decay to be 0.0001 and batch size to be 32. The learning rate is set to 0.1 initially. On NTU-60 and NTU-120, the learning rate is divided by 10 at the 40th and 50th epoch, and the training process ends at the 60th epoch. On Kinetics-Skeleton, the learning rate is divided by 10 at the 50th and 60th epoch, and the training finishes at the 80th epoch. 
%
All models have fixed hyperparameters with 2 and 4 layers for NTU-60/NTU-120 and Kinetics-Skeleton, respectively. The hidden dimensions is set to 16 for all 3 datasets. 
We use 4 attention heads for NTU-60 and NTU-120, and 8 attention heads for Kinetics-Skeleton.
%
To form each video temporal block, we simply choose temporal block size to be 10 and stride to be 5 to allow a 50\% overlap between consecutive temporal blocks. 

\subsection{Ablation Study}
\noindent{\bf Search for the single best order $n$.} \lei{Table~\ref{tab:order} shows our analysis regarding the best order $n$. In general, increasing the order $n$ improves the performance (within $\sim$ 0.5\% on average), but causing higher computational cost, \eg, the number of hyper-edges for the skeletal hypergraph of order $n\!=\!4$ is 3060 on Kinetics-Skeleton. We also notice that combining orders 3 and 4 yields very limited improvements. The main reasons are: (i) reasonable order $n$, \eg, $n = 3$ or 4 improves accuracy as higher-order motion patterns are captured which are useful for classification-related tasks (ii) further increasing order $n$, \eg, $n = 5$ introduces patterns  in feature representations that rarely repeat even for the same action class.} Considering the cost and performance, we choose the maximum order $r\!=\!3$ ($n=1, 2, 3$) in the following experiments unless specified otherwise.

\begin{table}[t!]
\caption{Search for the single best order $n$ of hypergraph (except for $n\!=\!3\,\&\,4$ where we check if  $n\!=\!3\,\&\,4$ are complementary).}
\begin{center}
\resizebox{0.9\linewidth}{!}{\begin{tabular}{l c c c c c}
\toprule
\multirow{2}{*}{Order-$n$}& \multicolumn{2}{c}{NTU-60} & \multicolumn{2}{c}{NTU-120} & Kinetics-Skel.\\
\cline{2-6}
& {X-Sub} & {X-View} & {X-Sub} & {X-Set} & Top-1 acc.\\
\midrule
$n=1$ & 78.5 & 86.3 & 75.3 & 77.9 & 32.0\\
$n=2$ & 83.0 & 89.2 & 86.2 & 88.3 & 37.1\\
$n=3$ & 91.3 & 97.0 & 87.5 & 89.7 & 39.5\\
$n=4$ & 91.5 & 97.1 & {\bf 87.8} & 90.0 & 40.1\\
$n=5$ & 91.4 & {\bf 97.3} & {\bf 87.8} & 90.0 & 40.3\\
$n=3~\&~4$& {\bf 91.6} & 97.2 & 87.6 & {\bf 90.3} & {\bf 40.5}\\
\bottomrule
\end{tabular}}
\label{tab:order}
\end{center}
\end{table}

\begin{table}[t!]
\caption{Evaluations of our model variants with/without MP and/or TP. \lei{Baseline in the table denotes the backbone (MLP unit + HoTs) without the use of either MP or TP module.}}
\begin{center}
\resizebox{\linewidth}{!}{\begin{tabular}{l c c c c c}
\toprule
\multirow{2}{*}{Variants}& \multicolumn{2}{c}{NTU-60} & \multicolumn{2}{c}{NTU-120} & Kinetics-Skel.\\
\cline{2-6}
& {X-Sub} & {X-View} & {X-Sub} & {X-Set} & Top-1 acc.\\
\midrule
% No MMA & 91.3 & 97.0 & 87.5 & 89.7 & 39.5\\
\lei{\quad Baseline} & 89.8 & 91.4 & 86.5 & 87.0 & 38.6\\
+ TP only& 91.2 & 93.8 & 87.5 & 88.6 & 39.8\\
% TP only& 93.4 & 97.8 & 89.8 & 91.5 & 44.0\\
+ MP only& 92.0 & 94.3 & 88.7 & 89.7 & 40.3 \\
% MP only& 93.6 & 98.0 & 91.5 & 93.0 & 44.6 \\
% \rowcolor{LightCyan}
+ MP$\rightarrow$TP & 93.0 & 96.1 & 90.8 & 91.7 & 45.7 \\
+ TP$\rightarrow$MP & 92.6 & 95.8 & 90.2 & 91.1 & 44.0 \\
\rowcolor{blue!10}
+ 2-branch(3Mformer)$\!\!\!\!$ & {\bf 94.8} & {\bf 98.7} & {\bf 92.0} & {\bf 93.8} & {\bf 48.3}\\
\bottomrule
\end{tabular}}
\label{tab:cross-joint-attn}
\end{center}
\end{table}

% new global table

\begin{table*}[t!]
\caption{Experimental results on NTU-60, NTU-120 and Kinetics-Skeleton.}
\begin{center}
\resizebox{\linewidth}{!}{\begin{tabular}{l l l c c c c c c c c}
\toprule
 & \multirow{2}{*}{Method} & \multirow{2}{*}{Venue}  & \multicolumn{2}{c}{NTU-60} & & \multicolumn{2}{c}{NTU-120} & & \multicolumn{2}{c}{Kinetics-Skeleton}\\
\cline{4-5}
\cline{7-8}
\cline{10-11}
& & & {X-Sub} & {X-View} && {X-Sub} & {X-Set}  && {Top-1} & {Top-5} \\
\midrule
\multirow{7}{*}{\parbox{2.0cm}{\bf Graph-based}} 
& TCN~\cite{8014941} & CVPRW'17 & - & - & & - & - && 20.3 & 40.0\\
& ST-GCN~\cite{stgcn2018aaai} & AAAI'18 & 81.5 & 88.3 && 70.7 & 73.2 &&30.7 & 52.8 \\
& AS-GCN~\cite{Li_2019_CVPR}  & CVPR'19 & 86.8 & 94.2 && 78.3 & 79.8 && 34.8 & 56.5\\
& 2S-AGCN~\cite{2sagcn2019cvpr} & CVPR'19 & 88.5 & 95.1&& 82.5 & 84.2 && 36.1 & 58.7\\
& NAS-GCN~\cite{Peng_Hong_Chen_Zhao_2020} & AAAI'20 & 89.4 & 95.7&& -& -&& 37.1 & 60.1\\
& Sym-GNN~\cite{9334430} & TPAMI'22 & 90.1 & 96.4&& - & - && 37.2 & 58.1\\
& Shift-GCN~\cite{cheng2020shiftgcn} & CVPR'20 & 90.7 & 96.5 && 85.9 & 87.6&& - &-\\
& MS-G3D~\cite{Liu_2020_CVPR} & CVPR'20 & 91.5 & 96.2 & & 86.9 & 88.4 & & 38.0 & 60.9\\
\hline
\multirow{4}{*}{\parbox{2.0cm}{\bf Hypergraph-based}}
& Hyper-GNN~\cite{9329123} & TIP'21 & 89.5 & 95.7&& - &-  && 37.1 & 60.0\\
& DHGCN~\cite{dynamichypergraph} & CoRR'21 & 90.7 & 96.0&& 86.0 & 87.9 && 37.7 & 60.6\\
& Selective-HCN~\cite{10.1145/3512527.3531367}& ICMR'22 & 90.8 & 96.6 && - &- && 38.0 & 61.1\\
& SD-HGCN~\cite{10.1007/978-3-030-92270-2_2} & ICONIP'21 &  90.9 & 96.7&& 87.0 & 88.2 && 37.4 & 60.5\\
\hline
\multirow{9}{*}{\parbox{2.0cm}{\bf Transformer-based}}
& ST-TR~\cite{PLIZZARI2021103219}& CVIU'21 & 90.3 & 96.3&& 85.1 & 87.1 && 38.0 &60.5\\
% & 4s-MTT+Shift-GCN
& MTT~\cite{9681250}& LSP'21 & 90.8 & 96.7 && 86.1 & 87.6 && 37.9 &61.3\\
& 4s-GSTN~\cite{sym14081547} & Symmetry'22 & 91.3 & 96.6 && 86.4 & 88.7 && -&-\\
& STST~\cite{10.1145/3474085.3475473} &  ACM MM'21& 91.9 & 96.8 && -&- && 38.3 &61.2 \\
% \cline{2-11}
% & 1st-order HoT ({\em ours})&  & 78.5 & 86.3&& 75.3 & 77.9 && 32.0 & 55.1\\
% & 2nd-order HoT ({\em ours})&  & 83.0 & 89.2&& 86.2 & 88.3 && 37.1 &60.0\\
% & 3rd-order HoT ({\em ours}) & & 91.3 & 97.0&& 87.5 & 89.7 && 40.5 & 64.0\\
\cdashline{2-11}
& 3Mformer (with avg-pool, {\em ours}) & &  92.0 & 97.3&& 88.0 & 90.1 && 43.1&65.2\\
& 3Mformer (with max-pool, {\em ours}) &  & 92.1 & 97.8&& - & - && -&-\\
\rowcolor{blue!10}
% \cellcolor{gray!10}
& 3Mformer (with attn-pool, {\em ours}) & & {\bf 94.2} & {\bf 98.5}&& 89.7 & 92.4 && 45.7 & 67.6\\
% \rowcolor{LightCyan}
\rowcolor{blue!10}
& 3Mformer (with tri-pool, {\em ours}) & & 94.0 & {\bf 98.5}&& {\bf 91.2} & {\bf 92.7} && {\bf 47.7} & {\bf 71.9}\\
% \rowcolor{LightCyan}
\rowcolor{blue!10}
& 3Mformer (with rank-pool, {\em ours}) & & {\bf 94.8} & {\bf 98.7}&& {\bf 92.0} & {\bf 93.8} && {\bf 48.3} & {\bf 72.3}\\
\bottomrule
\end{tabular}}
\label{tab:globaltable}
\end{center}
\end{table*}



\begin{figure}[t]
\centering
\begin{subfigure}[t]{0.22\linewidth}
\centering\includegraphics[width=\linewidth,height=1.9cm]{imgs/temp.png}
\caption{}\label{fig:temp}
\end{subfigure}\hfill
\begin{subfigure}[t]{0.22\linewidth}
\centering\includegraphics[width=\linewidth,height=1.9cm]{imgs/channel-hyper-edge.png}
\caption{}\label{fig:channel-hyper-edge}
\end{subfigure}\hfill
\begin{subfigure}[t]{0.22\linewidth}
\centering\includegraphics[width=\linewidth,height=1.9cm]{imgs/order-channel.png}
\caption{}\label{fig:order-channel}
\end{subfigure}\hfill
\begin{subfigure}[t]{0.305\linewidth}
\centering\includegraphics[width=\linewidth,height=1.9cm]{imgs/channel-temp.png}%\hspace{-0.6cm}
\caption{}\label{fig:channel-temp}
\end{subfigure}
\caption{\lei{Visualization of attention matrices. (a) shows single-mode attention matrix of `channel-only' token, (b)--(d) show joint-mode attention matrices of `channel-hyper-edge', `order-channel-body joint' and `channel-temporal block' tokens, respectively.}}
\label{fig:attn-matrix}
\end{figure}

\begin{figure}[t]%htbp % left bottom right top
\centering\includegraphics[width=1\linewidth]{imgs/token-comp.pdf}
\caption{\lei{Evaluations of different single-mode ({\it baseline}) and joint-mode tokens. We use a 3rd-order HoT with a standard Transformer, but we replace the scaled dot-product attention with joint-mode tokens and joint-mode attention.}}
\label{fig:token-comp}
\end{figure}

\vspace{0.1cm}
\noindent\lei{{\bf Discussion on joint-mode attention.} Fig.~\ref{fig:attn-matrix} shows the visualization of some attention matrices in our 3Mformer. We notice that these attention matrices show diagonal pattern and/or vertical pattern. These patterns are consistent with the patterns of the attention matrices found in standard Transformer trained on sequences,  \eg, for natural language processing tasks~\cite{NIPS2017_3f5ee243, kovaleva-etal-2019-revealing}. We also notice that the joint-mode attention, \eg, `channel-temporal block' captures much richer information compared to single mode attention \eg, `channel-only'. Our joint-mode attention can be applied to different orders of tensor representations through simple matricization.}

\vspace{0.1cm}
\noindent{\bf \lei{Discussion on model variants}.} To show the effectiveness of the proposed MP and TP module, firstly, we compare TP only and MP only with the baseline (\lei{No MP or TP module}). We use the TP module followed by an FC layer instead of MP as in TP$\rightarrow$MP, where the FC layer takes the output from TP ($\mbr{d'N}$) and produces a vector in $\mbr{3d'J}$, passed to the classifier. Similarly, for MP only, we use the MP module followed by an average pooling layer instead of TP as in MP$\rightarrow$TP, where the average layer takes output from MP ($\mbr{3d'J\times \tau}$) and generates a vector in $\mbr{3d'J}$ (pool along $\tau$ blocks), passed to the classifier.
% \noindent{\bf 2. Evaluation of MP \& TP in MMA.} 
Table~\ref{tab:cross-joint-attn} shows the results. 
% As shown in the table, with only the TP module, we outperform the baseline (No MMA) by 1.3\% on average. With only the MP module, we outperform the baseline (No MMA) by 2.34\% on average. With MMA, we boost the performance by 6.28\%.  
\lei{With just the TP module, we outperform the baseline by 1.3\% on average. With only the MP module, we outperform the baseline by 2.34\% on average. These comparisons show that (i) JmSA in MP and TP are efficient for better performance (ii) MP performs better than TP which shows that `channel-temporal block' token contains richer information than `channel-hyper-edge' token. 
%
We also notice that MP$\rightarrow$TP slightly outperforms TP$\rightarrow$MP by $\sim$ 1\%, and the main reason is that MP$\rightarrow$TP has joint-mode tokens `channel-temporal block' and `order-channel-joint' which attend 4 modes, whereas TP$\rightarrow$MP has `channel-hyper-edge' and `channel-only' tokens which attend only 2 modes. Fig.~\ref{fig:token-comp} shows a comparison of different joint-mode tokens on 3 benchmark datasets. This also suggests that one should firstly perform attention with joint-mode `channel-block' tokens, followed by weighted pooling along the hyper-edge mode, followed by attention with joint-mode `order-channel-body joint' and finalised by block-temporal pooling. Finally, with 2-branch (3Mformer), we further boost the performance by 2--4\%, which shows that MP$\rightarrow$TP and TP$\rightarrow$MP are complementary branches. Below we use 2-branch (3Mformer) in the experiments (as in Fig.~\ref{fig:pipeline}). 
}
%

% We then compare single-branch MMA versus two-branch MMA. 
% % Table~\ref{tab:cross-joint-attn} shows the comparisons. 
% As shown in the table, MP$\rightarrow$TP outperforms TP$\rightarrow$MP by $\sim$0.6\% on average, and this suggests that one should firstly perform attention with joint-mode `channel-block' tokens, followed by weighted pooling along the hyper-edge mode, followed by attention with joint-mode `order-channel-body joint' and finalised by block-temporal pooling. With two-branch MMA, we further boost the performance by 1--2\%. Below we use MMA in the experiments (as in Fig.~\ref{fig:pipeline}).



% \noindent{\bf 3. MMA variants.} 


% \begin{table}[t!]
% \caption{Evaluations of MMA variants.}
% \begin{center}
% \resizebox{\linewidth}{!}{\begin{tabular}{l c c c c c}
% \toprule
% \multirow{2}{*}{}& \multicolumn{2}{c}{NTU-60} & \multicolumn{2}{c}{NTU-120} & Kinetics-Skel.\\
% \cline{2-6}
% & {x-sub} & {x-view} & {x-sub} & {x-set} & Top-1 acc.\\
% \midrule
% MP$\rightarrow$TP & 93.0 & 96.1 & 90.8 & 91.7 & 43.2 \\
% TP$\rightarrow$MP & 92.6 & 95.8 & 90.2 & 91.1 & 42.1 \\
% % \rowcolor{LightCyan}
% \rowcolor{blue!10}
% MMA & 94.8 & 98.7 & 92.0 & 93.8 & 45.3\\
% \bottomrule
% \end{tabular}}
% \label{tab:joint-separate-attn}
% \end{center}
% \end{table}

\vspace{0.1cm}
\noindent{\bf Comparison of pooling in TP.} As shown in Table~\ref{tab:globaltable}, average pooling (avg-pool) achieves similar performance (within $\sim$ 0.5\% difference) as maximum pooling (max-pool), second-order pooling (attn-pool) outperforms average and maximum pooling by $\sim$ 1--2\% and third-order pooling (tri-pool) outperforms second-order pooling by $\sim$ 1\%. Interestingly, rank pooling (rank-pool) achieves the best performance. We think it is reasonable as rank pooling strives to enforce the temporal order in the feature space to be preserved, \eg, it forces network to always preserve temporal progression of actions over time. With multiple attention modules, orderless statistics such as second- or third-order pooling may be too general.

\subsection{Comparisons with the State of the Arts}

We compare our model with recent state-of-the-art methods. On the NTU-60 (Tab.~\ref{tab:globaltable}), we obtain the top-1 accuracies of the two evaluation protocols during test stage. The methods in comparisons include popular graph-based~\cite{stgcn2018aaai,Li_2019_CVPR,2sagcn2019cvpr,Peng_Hong_Chen_Zhao_2020,9334430} and hypergraph-based models~\cite{9329123,dynamichypergraph,10.1145/3512527.3531367,10.1007/978-3-030-92270-2_2}. Our 3rd-order model  outperforms all graph-based methods, and also outperforms existing hypergraph-based models such as Selective-HCN and SD-HGCN by 0.45\% and 0.35\% on average on X-Sub and X-View respectively. With \lei{3Mformer} for the fusion of multi-order features, our model further boosts the performance by $\sim$ 3\% and 1.5\% on the two protocols.

It can be seen from Tab.~\ref{tab:globaltable} on NTU-60 that although some learned graph-based methods such as AS-GCN and 2S-AGCN can also capture the dependencies between human body joints, they only consider the pairwise relationship between body joints, which is the second-order interaction, and ignore the higher-order interaction between multiple body joints in form of hyper-edges, which may lose sensitivity to important groups of body joints. Our proposed \lei{3Mformer} achieves better performance by constructing a hypergraph from 2D/3D body joints as nodes for action recognition, thus capturing  higher-order interactions of body joints to further improve the performance. Note that even with the average pooling, our model still achieves competitive results compared to its counterparts.

For the NTU-120 dataset (Tab.~\ref{tab:globaltable}), we obtain the top-1 performance on X-Sub and X-Set protocols. Our 2nd-order HoT alone outperforms graph-based models by 2--2.4\% on average. For example, we outperform recent Shift-GCN by 0.3\% and 0.7\% on X-Sub and X-Set respectively. Moreover, our 3rd-order HoT alone outperforms SD-HGCN by 0.5\% and 1.5\% respectively on X-Sub and X-Set. With the \lei{3Mformer} for the fusion of multi-order feature maps, we obtain the new state-of-the-art results.

As videos from the Kinetics dataset are processed by the OpenPose, the skeletons in the Kinetics-Skeleton dataset have defects which adversely affect the performance of the model. We show both top-1 and top-5 performance in Table~\ref{tab:globaltable} to better reflect the performance of our \lei{3Mformer}. ST-GCN is the first method based on GCN, our 2nd-order HoT alone achieves very competitive results compared to the very recent NAS-GCN and Sym-GNN. The 3rd-order HoT alone outperforms Hyper-GNN, SD-HGCN and Selective-HCN by 3.4\%, 3.1\% and 2.9\% respectively for top-1 accuracies. Moreover, fusing multi-order feature maps from multiple orders of hyper-edges via \lei{3Mformer} gives us the best performance on Kinetics-Skeleton with 48.3\% for top-1, the new state-of-the-art result.




% \begin{table}[t!]
% \caption{Experimental results on NTU-60.}
% \begin{center}
% \resizebox{\linewidth}{!}{\begin{tabular}{l l l c c }
% \toprule
%  & \multirow{2}{*}{Method} & \multirow{2}{*}{Venue}  & \multicolumn{2}{c}{NTU-60} \\
% \cline{4-5}
% & & & {X-Sub} & {X-View} \\
% \midrule
% \multirow{5}{*}{\bf Graph-based}
% & ST-GCN~\cite{stgcn2018aaai} & AAAI'18 & 81.5 & 88.3\\
% & AS-GCN~\cite{Li_2019_CVPR}  & CVPR'19 & 86.8 & 94.2\\
% & 2S-AGCN~\cite{2sagcn2019cvpr} & CVPR'19 & 88.5 & 95.1\\
% & NAS-GCN~\cite{Peng_Hong_Chen_Zhao_2020} & AAAI'20 & 89.4 & 95.7\\
% & Sym-GNN~\cite{9334430} & TPAMI'22 & 90.1 & 96.4\\
% \hline
% \multirow{4}{*}{\bf Transf.-based}
% & ST-TR~\cite{PLIZZARI2021103219}& CVIU'21 & 90.3 & 96.3\\
% % & 4s-MTT+Shift-GCN
% & MTT~\cite{9681250}& LSP'21 & 90.8 & 96.7 \\
% & 4s-GSTN~\cite{sym14081547} & Symmetry'22 & 91.3 & 96.6 \\
% & STST~\cite{10.1145/3474085.3475473} &  ACM MM'21& 91.9 & 96.8 \\
% \hline
% \multirow{12}{*}{\bf Hyperg.-based}
% & Hyper-GNN~\cite{9329123} & TIP'21 & 89.5 & 95.7\\
% & DHGCN~\cite{dynamichypergraph} & CoRR'21 & 90.7 & 96.0\\
% & Selective-HCN~\cite{10.1145/3512527.3531367}& ICMR'22 & 90.8 & 96.6 \\
% & SD-HGCN~\cite{10.1007/978-3-030-92270-2_2} & ICONIP'21 &  90.9 & 96.7\\
% \cline{2-5}
% & 1st-order HoT ({\em ours})&  & 78.5 & 86.3\\
% & 2nd-order HoT ({\em ours})&  & 83.0 & 89.2\\
% & 3rd-order HoT ({\em ours}) & & 91.3 & 97.0\\
% \cdashline{2-5}
% & MoTA-HoT (avg-pool, {\em ours}) & & 92.0 & 97.3\\
% & MoTA-HoT (max-pool, {\em ours}) &  & 92.1 & 97.8\\
% & MoTA-HoT (attn-pool, {\em ours}) & & {\bf 94.2} & {\bf 98.5}\\
% % \rowcolor{LightCyan}
% \rowcolor{blue!10}
% & MoTA-HoT (tril-pool, {\em ours}) & & 94.0 & {\bf 98.5}\\
% % \rowcolor{LightCyan}
% \rowcolor{blue!10}
% & MoTA-HoT (rank-pool, {\em ours}) & & {\bf 94.8} & {\bf 98.7}\\
% \bottomrule
% \end{tabular}}
% \label{tab:ntu60}
% \end{center}
% \end{table}

% \begin{table}[t!]
% \caption{Experimental results on NTU-120.}
% \begin{center}
% \resizebox{\linewidth}{!}{\begin{tabular}{l l l c c }
% \toprule
% & \multirow{2}{*}{Method} & \multirow{2}{*}{Venue} & \multicolumn{2}{c}{NTU-120} \\
% \cline{4-5}
% & & & {X-Sub} & {X-Set} \\
% \midrule
% \multirow{2}{*}{\bf Graph-based}
% % AS-GCN+DH-TCN & 78.3 & 79.8\\
% & 2S-AGCN~\cite{2sagcn2019cvpr} & CVPR'19 & 82.5 & 84.2\\
% & Shift-GCN~\cite{cheng2020shiftgcn} & CVPR'20 & 85.9 & 87.6\\
% \hline
% \multirow{3}{*}{\bf Transf.-based}
% & ST-TR~\cite{PLIZZARI2021103219}& CVIU'21 & 85.1 & 87.1\\
% % & 2s-MTT+AGCN
% & MTT~\cite{9681250} & LSP'21 & 86.1 & 87.6 \\
% & 4s-GSTN~\cite{sym14081547} & Symmetry'22 & 86.4 & 88.7\\
% \hline
% \multirow{9}{*}{\bf Hyperg.-based}
% & DHGCN~\cite{dynamichypergraph} & CoRR'21 & 86.0 & 87.9\\
% & SD-HGCN~\cite{10.1007/978-3-030-92270-2_2} & ICONIP'21 & 87.0 & 88.2 \\
% \cline{2-5}
% & 1st-order HoT ({\em ours}) & & 75.3 & 77.9\\
% & 2nd-order HoT ({\em ours}) & & 86.2 & 88.3\\
% & 3rd-order HoT ({\em ours}) & & 87.5 & 89.7\\
% \cdashline{2-5}
% & MoTA-HoT (avg-pool, {\em ours}) & & 88.0 & 90.1\\
% & MoTA-HoT (attn-pool, {\em ours}) & & 89.7 & 92.4\\
% % \rowcolor{LightCyan}
% \rowcolor{blue!10}
% & MoTA-HoT (tril-pool, {\em ours}) & & {\bf 91.2} & {\bf 92.7}\\
% % \rowcolor{LightCyan}
% \rowcolor{blue!10}
% & MoTA-HoT (rank-pool, {\em ours}) & & {\bf 92.0} & {\bf 93.8}\\
% \bottomrule
% \end{tabular}}
% \label{tab:ntu120}
% \end{center}
% \end{table}

% \begin{table}[t!]
% \caption{Experimental results on Kinetics-Skeleton.}
% \begin{center}
% \resizebox{\linewidth}{!}{\begin{tabular}{l l l c c }
% \toprule
% & \multirow{2}{*}{Method} & \multirow{2}{*}{Venue} & \multicolumn{2}{c}{Kinetics-Skeleton} \\
% \cline{4-5}
% & & & {Top-1} & {Top-5} \\
% \midrule
% \multirow{6}{*}{\bf Graph-based}
% & TCN~\cite{8014941} & CVPRW'17 & 20.3 & 40.0\\
% & ST-GCN~\cite{stgcn2018aaai} & AAAI'18 & 30.7 & 52.8\\
% & AS-GCN~\cite{Li_2019_CVPR} & CVPR'19 & 34.8 & 56.5\\
% & 2S-AGCN~\cite{2sagcn2019cvpr} & CVPR'19 & 36.1 & 58.7\\
% & NAS-GCN~\cite{Peng_Hong_Chen_Zhao_2020} & AAAI'20 & 37.1 & 60.1\\
% & Sym-GNN~\cite{9334430} & TPAMI'22 & 37.2 & 58.1 \\
% \hline
% \multirow{3}{*}{\bf Transf.-based}
% & ST-TR~\cite{PLIZZARI2021103219}& CVIU'21 & 38.0 & 60.5\\
% & STST~\cite{10.1145/3474085.3475473} &  ACM MM'21 &  38.3 & 61.2\\
% % & MTT+AGCN
% & MTT~\cite{9681250} & LSP'22 & 37.9 & 61.3 \\
% \hline
% \multirow{10}{*}{\bf Hyperg.-based}
% & Hyper-GNN~\cite{9329123} & TIP'21 & 37.1 & 60.0\\
% & DHGCN~\cite{dynamichypergraph} & CoRR'21 & 37.7 & 60.6\\
% & SD-HGCN~\cite{10.1007/978-3-030-92270-2_2} & ICONIP'21 & 37.4 & 60.5\\
% & Selective-HCN~\cite{10.1145/3512527.3531367} & ICMR'22 & 38.0 & 61.1 \\
% \cline{2-5}
% & 2nd-order HoT ({\em ours}) & & 37.1 & 60.0\\
% & 3rd-order HoT ({\em ours}) & & 40.5 & 64.0\\
% \cdashline{2-5}
% & MoTA-HoT (avg-pool, {\em ours})&  & 43.1 & 65.2\\
% & MoTA-HoT (attn-pool, {\em ours})&  & 45.7 & 67.6\\
% % \rowcolor{LightCyan}
% \rowcolor{blue!10}
% & MoTA-HoT (tril-pool, {\em ours})&  & {\bf 47.7} & {\bf 71.9}\\
% % \rowcolor{LightCyan}
% \rowcolor{blue!10}
% & MoTA-HoT (rank-pool, {\em ours})&  & {\bf 48.3} & {\bf 72.3}\\
% \bottomrule
% \end{tabular}}
% \label{tab:kinetics}
% \end{center}
% \end{table}

% \noindent{\bf NTU-60}.
% \noindent{\bf NTU-120}.
% \noindent{\bf Kinetics-Skeleton}.

\section{Conclusions}

%
In this paper, we model the skeleton data as hypergraph to capture higher-order information formed between groups of human body joints of orders $1,\cdots,r$. We use Higher-order Transformer (HoT) to learn higher-order information on hypergraphs of  $r$-order formed over 2D/3D human body joints. We also introduce a novel \lei{Multi-order Multi-mode Transformer (3Mformer)} for the fusion of multi-order  feature representations. Our end-to-end trainable \lei{3Mformer} outperforms state-of-the-art graph- and hypergraph-based models by a large margin on several benchmarks. %including the large-scale Kinetics-Skeleton.

\lei{\section{Limitation and Future Work}}

\lei{Despite the high accuracy of our model, there are still some limitations. Firstly, as we use $r$ branches of HoT, the number of parameters and computational cost are higher than existing methods. However, our method with single branch, \eg, 3rd-order HoT only, still achieves very competitive results compared to existing graph-, transformer- and hypergraph-based models for the same parameter scale on 3 benchmarks.
%
Secondly, in this work, we only use HoT block to encode the temporal block feature representations. The more efficient way is to redesign HoT block so that it is able to encode both short-term and long-term spatio-temporal features to simplify the backbone encoder, \ie, without the need of MLP unit. Note that the design of our 3Mformer is independent of the backbone encoder. Our 3Mformer is especially suitable for tensorial data, \eg, higher-order feature representations.
%
Our future work will focus on applying our Multi-order Multi-mode Transformer (3Mformer) to other computer vision tasks with tensorial data.
}
%



%%%%%%%%% REFERENCES
{\small
\bibliographystyle{ieee_fullname}
\bibliography{egbib}
}

\clearpage
\appendix

%\pagestyle{headings}

%%%%%%%%% TITLE - PLEASE UPDATE
% \title{Multi-order Multi-mode Transformer for Skeletal Action Recognition (Supplementary Material)}
\title{\lei{3Mformer: } Multi-order Multi-mode Transformer \\for Skeletal Action Recognition \\-- Supplementary Material --}

\author{First Author\\
Institution1\\
Institution1 address\\
{\tt\small firstauthor@i1.org}
% For a paper whose authors are all at the same institution,
% omit the following lines up until the closing ``}''.
% Additional authors and addresses can be added with ``\and'',
% just like the second author.
% To save space, use either the email address or home page, not both
\and
Second Author\\
Institution2\\
First line of institution2 address\\
{\tt\small secondauthor@i2.org}
}
\maketitle



% We would like to stress that \textbf{\color{teal}we will release our code publicly}.

%\section{Details of Encoding Network.}
%\label{sec:en_det}

\section{Visualization of 3Mformer.}
\label{app:3mf}

Fig.~\ref{fig:3mformer} shows the visualization of our 3Mformer. The green and orange blocks denote the Multi-order Pooling (MP) and the Temporal block Pooling (TP) respectively, which are two basic building blocks that can be stacked to form our 3Mformer. More precisely, our 3Mformer consists of two branches: (i) MP followed by TP (denoted MP $\rightarrow$ TP, Fig.~\ref{fig:mp-tp}) and (ii) TP followed by MP (denoted TP $\rightarrow$ MP, Fig.~\ref{fig:tp-mp}). 

\begin{figure*}[ht]
  \centering
  \begin{subfigure}{\linewidth}
    \centering
    \includegraphics[trim=1.8cm 2.5cm 1.8cm 3.0cm, clip=true, width=\linewidth]{imgs/mp-tp.pdf}
    \caption{Single-branch: MP followed by TP (denoted MP$\rightarrow$TP).}
    \label{fig:mp-tp}
  \end{subfigure}
  \begin{subfigure}{\linewidth}
    \centering
    \includegraphics[trim=1.5cm 2.5cm 1.8cm 3.3cm, clip=true, width=\linewidth]{imgs/tp-mp.pdf}
    \caption{Single-branch: TP followed by MP (denoted TP$\rightarrow$MP).}
    \label{fig:tp-mp}
  \end{subfigure}  
  \caption{Visualization of 3Mformer which is a two- branch model: (a) MP$\rightarrow$TP and (b) TP$\rightarrow$MP. Green and orange blocks are Multi-order Pooling (MP) module and Temporal block Pooling (TP) module, respectively. $(m)$ inside the MP module denotes the order $m\in\idx{r}$ of hyper-edges. These two modules (MP and TP) are the basic building blocks which are further stacked to form our 3Mformer. Each module (MP or TP) uses a specific joint-mode token through matricization (we use reshape for simplicity), \eg, `channel-temporal block', `order-channel-body joint', `channel-hyper-edge (any order)' or `channel-only', and the Joint-mode Self-Attention (JmSA) is used to explore the joint-mode relationships inside the joint-mode tokens. We also form our multi-head JmSA as in standard Transformer (where the JmSA module repeats its computations multiple times in parallel and the attention module splits the query, key and value, each split is independently passed through a separate head and later combined together to produce the final joint-mode attention score). We omit the multi-head visualization for simplicity and better visualization purposes. }  
  \label{fig:3mformer}
\end{figure*} 

\section{Skeletal Graph and Hypergraph Representations}

\noindent\textbf{Skeletal Graph}~\cite{stgcn2018aaai}. Let $G\!=\!({V}, {E})$ be a skeletal graph with the vertex set ${V}$  of nodes (body joints) $\{v_1, \cdots, v_J\}$, and ${E}$ be edges (bones) of the graph, and ${E}$ consists of ${ E}_S$ and ${E}_T$. The subset ${E}_S\!=\!\{(v_{it}, v_{jt}): i,j\!\in\!\idx{J} \text{ and } t\!\in\!\idx{T}\}$ represents that at time step $t$, each pair of joints $(v_{it}, v_{jt})$ corresponding to skeletal connectivity diagram is connected; whereas ${E}_T\!=\!\{(v_{it}, v_{i(t+1)}): i\!\in\!\idx{J} \text{ and } t\!\in\!\idx{T}\}$ forms the connection of the same joint across time. The set of joints and edges together form the skeleton graph. 
% to model the human skeleton data. 
% We denote the feature matrix of the skeleton graph as ${\bf A} \!\in\!\mbr{V\!\times\!V}$. 
If two body joints are connected by an edge, the corresponding  element in the incidence matrix ${\bf H}$ is equal to 1, otherwise it is equal to 0, and the adjacency matrix ${\bf A}={\bf H}^T{\bf H}-2\mathbf{I}$ (where {\bf I} is the identity matrix). The update rule of a common GCN model at time step $t$ is defined as:
\begin{equation}
    {\bf X}_t^{(l+1)}\!=\!\sigma\big(\widetilde{\bf D}^{-\frac{1}{2}}\widetilde{\bf A}\widetilde{\bf D}^{-\frac{1}{2}}{\bf X}_t^{(l)}\mP^{(l)}\big),
\end{equation}
where $\sigma(\cdot)$ is a non-linearity, $\widetilde{\bf D}$ is the graph degree matrix, ${\bf X}_t^{(l)}$ is the input data of the convolutional layer $l$ at the time step $t$ and $\mP^{(l)}$ is the learnable parameters of layer $l$. $\widetilde{\bf A}\!=\!{\bf A}+{\bf I}$ is a normalized graph adjacency matrix.

The tensor representation of graph data can be given by ${\bf X} \!\in\!\mbr{J^2\!\times\!d}$ where $d$ is the feature channel dimension.

\vspace{0.1cm}
\noindent\textbf{Skeletal Hypergraph}~\cite{ijcai2020-109,9329123}. Hypergraph captures  complex higher-order relationships by hyper-edges that connect more than two nodes (body joints). Each hyper-edge is a subset of all nodes. Let $G_h\!=\!({ V}_h, {E}_h, {W}_h)$ where ${V}_h$, ${E}_h$ and ${W}_h$ denote respectively the set of body joints, hyper-edges and the weights of hyper-edges.
Given $v \!\in\!{V}_h$ and $e\!\in\!{E}_h$, the elements in the incidence matrix ${\bf H}_h$ of the skeleton hypergraph are defined as ${H}_{h,v,e}=1$, or simply put $h(v,e)=1$, if vertex $v$ is part of edge $e$, 0 otherwise.
%\begin{equation}
%  {\bf H}_h =
%    \begin{cases}
%      h(v, e)\!=\!1 & v\!\in\!e\\
%      h(v, e)\!=\!0 & v\!\notin\!e
%    \end{cases}
%    \label{eq:incidentmatrix}
%\end{equation}
The degree of node/body joint $v\!\in\!{V}_h$ is the number of hyper-edges passing through the node, which is defined as:
\begin{equation}
    d(v)\!=\!\sum_{e\in {E}_h}w(e)h(v, e),
\end{equation}
where $w(e)$ is the weight of hyper-edge $e$. The degree of hyper-edge $e\in {E}_h$ is the number of nodes (body joints) contained in the hyper-edge $e$ that satisfies:
\begin{equation}
    \delta(e)\!=\!\sum_{v\in {V}_h}h(v, e).
\end{equation}
Moreover, let ${\bf D}_v$ and ${\bf D}_e$ be the diagonal matrices of node degrees $d(v)$ and  the hyper-edge degrees $\delta(e)$ respectively. Let ${\bf W}$ denote the diagonal matrix of the hyper-edge weights (initially the weights of all hyper-edges are set to 1). Then the update rule of the Hypergraph Convolutional Network at the time step $t$ is given by:
\begin{equation}
    {\bf X}_t^{(l+1)}\!=\!\sigma\big({\bf D}_v^{\frac{1}{2}}{\bf H}_h{\bf W}{\bf D}_e^{-1}{\bf H}_h^\top{\bf D}_v^{\frac{1}{2}}{\bf X}_t^{(l)}\mP^{(l)}\big),
\end{equation}
%
where $\mP^{(l)}$ are learnable parameters for layer $l$.


% $\idx{K}$ stands for the index set $\{1,2,\cdots,K\}$. Concatenation of $\alpha_i$ is denoted by $[\alpha_i]_{i\in\idx{I}}$, whereas $\mX_{:,i}$ means we extract/access column $i$ of matrix $\mD$. Calligraphic mathcal fonts denote tensors (\eg, $\tD$), capitalized bold symbols are matrices (\eg, $\mD$), lowercase bold symbols  are vectors (\eg, $\vpsi$), and regular fonts denote scalars.

% Firstly, let $G\!=\!(\bf{V}, {\bf E})$ be a graph with the vertex set $\bf{V}$  with nodes $\{v_1, \cdots, v_n\}$, and ${\bf E}$ are edges of the graph. 

% Let ${\bf A}$ and ${\bf D}$ be the adjacency and diagonal degree matrix, respectively. Let $\tilde{\bf A}\!=\!{\bf A}\!+\!{\bf I}$ be the adjacency matrix with self-loops (identity matrix) with the corresponding diagonal degree matrix $\tilde{\bf D}$ such that $\tilde{D}_{ii}\!=\!\sum_j ({\bf A}^{ij}\!+\! {\bf I}^{ij})$. Let ${\bf S}\!=\!\tilde{\bf D}^{-\frac{1}{2}} \tilde{\bf A}\tilde{\bf D}^{-\frac{1}{2}}$ be the normalized adjacency matrix with added self-loops. For the $l$-th layer, we use ${\bf \Theta}^{(l)}$ to denote the learnt weight matrix, and ${\bf \Phi}$ to denote the outputs from the graph networks. 

% \noindent{\bf Model Overview.}
% \subsection{Model Overview}
% Fig.~\ref{fig:pipeline} shows an overview of our model. Our model contains a simple 3-layer MLP unit (FC, ReLU, FC, ReLU, Dropout, FC), three HoT blocks with each HoT for each type of input \ie, body joint feature set, skeletal graph and hypergraph features, followed by Multi-order Temporal Attention (MoTA) for both multi-order and temporal-block-wise feature fusion, and an FC layer as the classifier. 
% %

% The MLP unit takes $T$ neighboring frames, each with $J$ 2D/3D skeleton body joints, forming one temporal block. In total, depending on stride $S$, we obtain some $\tau$ temporal blocks which capture the short temporal dependency, whereas the long temporal dependency is modeled with HoT and MoTA. Each temporal block is encoded by the MLP into a $d\!\times\!J$ dimensional feature map.  
% %
% Subsequently, the 1-, 2- and $m$-order feature maps of size $d\!\times\!J\!\times\!\tau$, $d\!\times\!J^2\!\times\!\tau$ and $d\!\times\!J^m\!\times\!\tau$, and skeletal graph/hypergraph information are forwarded to HoT blocks. 
% %
% The MoTA is used to further process different orders of edge/hyperedge features as well as the temporal-block-wise skeleton features learned from HoTs, which returns some human body joint feature maps. Finally, the feature maps are reshaped to vector representation and passed to the classifier. % Below we present our HoT with MoTA.

\section{\lei{Skeleton Data Preprocessing}}

\lei{Before passing the skeleton sequences into MLP, we first normalize each body joint \wrt to the torso joint ${\bf v}_{f, c}$:
\begin{equation}
    {\bf v}^\prime_{f, i}\!=\!{\bf v}_{f, i}\!-\!{\bf v}_{f, c},
\end{equation}
where $f$ and $i$ are the index of video frame and human body joint respectively. After that, we further normalize each joint coordinate into  [-1, 1] range:
\begin{equation}
    \hat{{\bf v}}_{f, i}[j] = \frac{{\bf v}^\prime_{f, i}[j]}{ \text{max}([\text{abs}({\bf v}^\prime_{f, i}[j])]_{f\in{\mathcal{I}_\tau},i\in\mathcal{I}_{J} } )},
\end{equation}
%
where $j$ is for selection of the $x$, $y$ and $z$ axes, $\tau$ is the number of frames and $J$ is the number of 3D body joints per frame.}
%

\lei{For the skeleton sequences that have more than one performing subject, (i) we normalize each skeleton separately, and each skeleton is passed to MLP for learning the temporal dynamics, and (ii) for the output features per skeleton from MLP, we pass them separately to the block-temporal HoT, \eg, two skeletons from a given video sequence will have two outputs obtained from the the block-temporal HoT, and we aggregate the outputs through average pooling before passing our 3Mformer. }

\end{document}
