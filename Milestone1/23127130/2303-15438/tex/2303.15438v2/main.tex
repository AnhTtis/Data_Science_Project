
\documentclass[nohyperref]{article}

\usepackage[accepted]{icml2023}

\usepackage[textsize=tiny]{todonotes}

\usepackage{macros}

\icmltitlerunning{On the Stepwise Nature of SSL}

\begin{document}

\twocolumn[
\icmltitle{On the Stepwise Nature of Self-Supervised Learning}

\newif\ifarxiv
\arxivfalse

{
\icmlsetsymbol{equal}{*}

\begin{icmlauthorlist}
\icmlauthor{James B. Simon}{berk,gi}
\icmlauthor{Maksis Knutins}{gi}
\icmlauthor{Liu Ziyin}{tok}
\icmlauthor{Daniel Geisz}{berk}
\icmlauthor{Abraham J. Fetterman}{gi}
\icmlauthor{Joshua Albrecht}{gi}
\end{icmlauthorlist}

\icmlaffiliation{berk}{UC Berkeley}
\icmlaffiliation{gi}{Generally Intelligent}
\icmlaffiliation{tok}{University of Tokyo}

\icmlcorrespondingauthor{James Simon}{james.simon@berkeley.edu}
}

\icmlkeywords{self-supervised learning, SSL, contrastive learning, kernels, kernel methods, infinite width, NTK, deep learning, generalization, inductive bias, implicit bias, spectral bias, implicit regularization, Barlow Twins, SimCLR, CLIP}

\vskip 0.3in
]

\printAffiliationsAndNotice{}

\begin{abstract}

We present a simple picture of the training process of joint embedding self-supervised learning methods.
We find that these methods learn their high-dimensional embeddings one dimension at a time in a sequence of discrete, well-separated steps.
We arrive at this conclusion via the study of a linearized model of Barlow Twins applicable to the case in which the trained network is infinitely wide.
We solve the training dynamics of this model from small initialization, finding that the model learns the top eigenmodes of a certain contrastive kernel in a stepwise fashion, and obtain a closed-form expression for the final learned representations.
Remarkably, we then see the same stepwise learning phenomenon when training deep ResNets using the Barlow Twins, SimCLR, and VICReg losses.
Our theory suggests that, just as kernel regression can be thought of as a model of supervised learning, \textit{kernel PCA} may serve as a useful model of self-supervised learning.
\end{abstract}

\section{Introduction}

The increasing complexity of source code poses a key challenge to the reliability of large-scale software systems. Software bugs in these systems can lead to safety issues~\cite{bug_safety} for users around the world as well as cause non-negligible financial losses~\cite{bug_loss}. As such, developers have to spend a large amount of time and effort on bug fixing. Consequently, \aprfull (\apr), designed to automatically generate patches to fix software bugs, has attracted wide attention from both academia and industry~\cite{long2016prophet, legoues2012genprog, long2015spr, lou2020can, tufano2018empstudy}. 


To achieve \apr, one popular approach is known as Generate-and-Validate (G\&V)~\cite{qi2015gv, ghanbari2019prapr, lou2020can, le2016hdrepair, legoues2012genprog, wen2018capgen, hua2018sketchfix, martinez2016astor, koyuncu2020fixminder, liu2019tbar, liu2019avatar}, which is typically based on the following pipeline: First, fault localization techniques~\cite{wong2016fl, abreu2007ochiai, zhang2013injecting, papadakis2015metallaxis, li2019deepfl, li2017transforming} are applied to determine the suspicious locations in programs where bugs are likely to exist. Then, the buggy locations are used by the \apr tools to generate a list of patches that replace buggy lines with correct lines. Afterward, each patch is validated against the original test suite to identify any \emph{plausible patches} (i.e., passing all tests in the test suite). Finally, to determine the \emph{correct patches}, developers examine the list of plausible patches to see if any of them can correctly fix the bug. 

Traditional \apr tools can mainly be categorized into heuristic-based~\cite{legoues2012genprog, le2016hdrepair, wen2018capgen}, constraint-based~\cite{mechtaev2016angelix, le2017s3, demacro2014nopol, long2015spr} and \template~\cite{ghanbari2019prapr, hua2018sketchfix, martinez2016astor, liu2019tbar, liu2019avatar}. Among these traditional tools, \template \apr tools~\cite{ghanbari2019prapr, liu2019tbar, benton2020effectiveness} have been able to achieve state-of-the-art results. \Template \apr tools typically leverage pre-defined templates (e.g., adding a nullness check) for bug fixing. However, since these fix templates are typically handcrafted, the number and types of bugs they are able to fix can be limited. 



To address the limitations of traditional \apr, researchers have proposed various \learning \apr tools~\cite{li2020dlfix, chen2018sequencer, jiang2021cure, lutellier2020coconut, zhu2021recoder, ye2022rewardrepair} based on the \nmtfull (\nmt) architecture~\cite{sutskever2014mt} where the input is the buggy code snippets and the goal is to translate the buggy code snippets into a fixed version. To accomplish this, \learning \apr tools require supervised training datasets with pairs of both buggy and fixed code snippets in order to learn how to perform this translation step. These training data are usually obtained by mining historical bug fixes using heuristics/keywords~\cite{dallmeier2007benchmark}, which can be imprecise for identifying bug-fixing commits; even the actual bug-fixing commits can include irrelevant code changes, leading to further pollution in the dataset~\cite{xia2022alpharepair}.
% 
Moreover, it can be hard for such \apr tools to generalize and fix bug types unseen during training. 



To better leverage recent advances in \plmfull{s} (\plm{s}), researchers~\cite{xia2022alpharepair, xia2023repairstudy, kolak2022patch, prenner2021codexws} have directly applied \plm{s} to generate patches without bug-fixing datasets. These \llm-based \apr tools work by either directly generating a complete code function~\cite{prenner2021codexws, xia2023repairstudy} or predict/infill the correct code snippet given its surrounding context~\cite{xia2022alpharepair, xia2023repairstudy}. By directly using \llm{s} that are pre-trained on billions of open-source code snippets, \llm-based \apr tools can achieve state-of-the-art performance on many repair datasets~\cite{xia2022alpharepair}. 


% 
%
%

Traditional \apr tools have long used the insight of the \emph{plastic surgery hypothesis}~\cite{barr2014plastic} where it states that the code ingredients to fix a bug already exist within the same project. Traditional \apr tools have manually designed pattern-~\cite{ghanbari2019prapr, saha2017elixir} or heuristic-based~\cite{jiang2018simfix, legoues2012genprog} approaches to finding and using such relevant code ingredients to generate fixes for bugs. However, the plastic surgery hypothesis has been largely ignored in \llm-based \apr. In fact, \llm provides a unique opportunity to fully automate the plastic surgery hypothesis idea via fine-tuning (learning project-specific information via model updates from the buggy project) and prompting (directly providing relevant code ingredients to the model), and make it directly applicable to different languages (since the \llm{s} are typically multi-lingual).%
Moreover, despite the intensive manual efforts involved, traditional \apr tools still cannot fully leverage project-specific information due to large search space for leveraging/composing existing code ingredients. In contrast, the project-specific information can effectively leveraged by \llm{s} due to their power in code understanding/vectorization, e.g., even partial/imprecise information may still guide \llm{s} in correct patch generation!
 To this end, we ask the question: \emph{How useful is the plastic surgery hypothesis in the era of \plm{s}}?








\mypara{Our Work.} To answer the question, we present \ourtech{\xspace} -- a \llm-based approach that automatically utilizes the plastic surgery hypothesis by systematically combining multiple fine-tuning and prompting strategies for \apr. \ourtech fine-tunes \plm{s} using two novel domain-specific training strategies: \textbf{\epfinetune} -- we fine-tune using the original buggy project by aggressively masking out a high percentage of tokens, which allows \plm to learn project-specific code tokens and programming styles; and \textbf{\rofinetune} -- which only masks out a single continuous code sequence per training sample, allowing the model to get used to the final \csapr task of predicting a single continuous code sequence. Furthermore, we directly leverage the ability for \plm{s} to understand natural language instructions and introduce a novel prompting strategy, \textbf{\idprompting}, which uses information retrieval and static analysis to obtain a list of relevant identifiers for the buggy lines. While such relevant identifiers are critical for fixing some difficult bugs, they may not be seen by the \llm during inference due to limited context window size. Through the use of prompting, we directly tell the model to use these extracted identifiers (relevant code ingredients) to generate the correct code. Finally, to perform repair, we combine all four model variants (including the base model, both fine-tuned models and the base model with prompting) for the final repair.





While our insight of leveraging the plastic surgery hypothesis for \llm-based \apr is generalizable across different types of \plm{s}, to implement \ourtech, we choose a recent \plm{\xspace}, \ctfive~\cite{wang2021codet5}, which is pre-trained on millions of open-source code snippets. \ctfive is an encoder-decoder model trained using \mspfull (\msp) objective where a percentage of tokens are masked out and each continuous masked token sequence is referred to as a masked span. Also, although we only extract relevant identifiers from the current buggy project (since this paper focuses on the plastic surgery hypothesis), our work can be easily extended to obtain other code information (such as relevant statements or functions) from other sources, such as  the massive pre-training corpora~\cite{husain2020codesearchnet} or historical bug-fixing datasets~\cite{jiang2019infer}, which can provide more coding knowledge for \llm{s}. Besides, although we mainly focus on using traditional string comparison algorithms for information retrieval in this paper, these techniques can be easily replaced by other frequency-based retrieval~\cite{robertson2009probabilistic} and neural search (or embedding-based search)~\cite{reimers2019sentence}.
  In summary, this paper makes the following contributions:


%


\begin{itemize}[noitemsep, leftmargin=*, topsep=0pt]
    \item \textbf{Dimension.} This paper is the first to revisit the important plastic surgery hypothesis in the era of \llm{s}. It opens up a new dimension for \llm-based \apr to incorporate previously neglected information from the buggy project itself to boost \apr performance. Furthermore, it demonstrates the promising future of retrieval-based prompting for modern \llm-based \apr.
    \item \textbf{Implementation.} We implement \ourtech based on the recent \ctfive model. We augment the model using two novel fine-tuning strategies: \epfinetune and \rofinetune, along with a novel prompting strategy based on information retrieval and static analysis: \idprompting. We combine the patches generated by all four models together and perform patch ranking to speed up \apr.% 
    \item \textbf{Evaluation Study.} We conduct an extensive evaluation against state-of-the-art \apr tools. On the widely studied \dfj 1.2 and 2.0 datasets~\cite{just2014dfj}, \ourtech is able to achieve the new state-of-the-art results of 89 and 44 correct bug fixes (15 and 8 more than best baseline) respectively.  Furthermore, we perform a broad ablation study to justify our design. \ourtech demonstrates for the first time that the plastic surgery hypothesis can substantially boost \llm-based \apr and advance state-of-the-art \apr, while being fully automated and general. Moreover, even partial/imprecise code ingredients may still effectively guide \llm{s} for \apr!
\end{itemize}


\setlength{\tabcolsep}{1.6mm}{
\renewcommand\arraystretch{1.1}
\begin{table}[ht]
  \centering
  \scalebox{0.9}{
  \begin{tabular}{llcccc}
    \toprule
    &\multirow{2}*{Methods} & \multirow{2}*{Sal.} &   \multicolumn{2}{c}{VOC} & MS~COCO \\
    \cmidrule(r){4-5}\cmidrule(r){6-6}
    &&&\texttt{val}&\texttt{test}&\texttt{val}\\
    \hline
    \multirow{13}*{\rotatebox{90}{ResNet-50}}
    &IRN~\cite{irn}          \tiny{CVPR'19}     &              & 63.5       & 64.8          & 42.0  \\
    &LayerCAM~\cite{layercam}\tiny{TIP'21}      &              & 63.0       & 64.5          & -     \\
    &AdvCAM~\cite{advcam}    \tiny{CVPR'21}     &              & 68.1       & 68.0          & 44.2  \\
    &RIB~\cite{rib}          \tiny{NeurIPS'21}  &              & 68.3       & 68.6          & 44.2  \\
    &ReCAM~\cite{recam}      \tiny{CVPR'22}     &              & 68.5       & 68.4          & 42.9  \\
    % \rowcolor{Gray}
    &\cellcolor{Gray}IRN+\texttt{LPCAM}    &\cellcolor{Gray} & \cellcolor{Gray}68.6    & \cellcolor{Gray}68.7      & \cellcolor{Gray}44.5  \\
    &SIPE~\cite{sipe}        \tiny{CVPR'22}     &              & 68.8       & 69.7          & 40.6  \\
    &OOD~\cite{ood}+Adv      \tiny{CVPR'22}     &              & 69.8       & 69.9          & -     \\
    &AMN~\cite{amn}          \tiny{CVPR'22}     &              & 69.5       & 69.6          & 44.7  \\
    &\cellcolor{Gray}AMN+\texttt{LPCAM}    &\cellcolor{Gray} & \cellcolor{Gray}70.1    &\cellcolor{Gray} 70.4      & \cellcolor{Gray}45.5  \\ 
    &ESOL~\cite{esol}        \tiny{NeurIPS'22}  &              & 69.9$^*$   & 69.3$^*$      & 42.6  \\
    &CLIMS~\cite{clims}      \tiny{CVPR'22}     &              & 70.4$^*$   & 70.0$^*$      & -     \\
    &EDAM~\cite{edam}        \tiny{CVPR'21}     &\checkmark    & 70.9$^*$   & 71.8$^*$      & -     \\
    &\cellcolor{Gray}EDAM+\texttt{LPCAM}  &\cellcolor{Gray}\checkmark & \cellcolor{Gray}71.8$^*$ &\cellcolor{Gray} 72.1$^*$& \cellcolor{Gray}42.1\\
    \hline
    \multirow{9}*{\rotatebox{90}{WideResNet-38}}
    &Spatial-BCE~\cite{sbce} \tiny{ECCV'22}     &              & 70.0       & 71.3      & 35.2  \\
    &BDM~\cite{bdm}          \tiny{ACMMM'22}    &\checkmark    & 71.0       & 71.0      & 36.7  \\ 
    &RCA~\cite{rca}+OOA      \tiny{CVPR'22}     &\checkmark    & 71.1       & 71.6      & 35.7  \\
    &RCA~\cite{rca}+EPS      \tiny{CVPR'22}     &\checkmark    & 72.2       & 72.8      & 36.8  \\
    &HGNN~\cite{hgnn}        \tiny{ACMMM'22}    &\checkmark         & 70.5$^*$   & 71.0$^*$  & 34.5  \\ 
    &EPS~\cite{eps}          \tiny{CVPR'21}     &\checkmark         & 70.9$^*$   & 70.8$^*$  & -     \\
    &RPIM~\cite{rpim}        \tiny{ACMMM'22}    &\checkmark         & 71.4$^*$   & 71.4$^*$  & -     \\ 
    &L2G~\cite{l2g}          \tiny{CVPR'22}     &\checkmark         & 72.1$^*$   & 71.7$^*$  & 44.2  \\
    \hline
    \multirow{2}*{\rotatebox{90}{\small{DeiT-S}}}
    &MCTformer~\cite{mctformer}    \tiny{CVPR'22}     &                 & 71.9$^{\dag}$  & 71.6$^{\dag}$   & 42.0  \\
    &\cellcolor{Gray}MCTformer+\texttt{LPCAM}      &\cellcolor{Gray} & \cellcolor{Gray}72.6$^{\dag}$  & \cellcolor{Gray}72.4$^{\dag}$  &\cellcolor{Gray} 42.8 \\
    \bottomrule
  \end{tabular}}
  \vspace{-2mm}
  \caption{The mIoU results (\%) based on DeepLabV2 on VOC and MS~COCO. The side column shows three backbones of multi-label classification model. ``Sal.'' denotes using saliency maps. * denotes the segmentation model is pre-trained on MS~COCO. $^\dag$ denotes the segmentation model is pre-trained on VOC.
  }
  \vspace{-6mm}
  \label{table_related}
\end{table}
}


\section{Notation and Preliminaries}\label{sec_prel}
Let $\mathbb{Z}_{>0}$ denote the set of positive integers and let $\mathbb{Z}_{[a,b]}$ denote the set of integers in the interval $[a,b]$. The $m\times m$ identity matrix is denoted by $I_m$ and its columns by $e_i$ for $i\in\mathbb{Z}_{[1,m]}$. We use $\mathbf{0}$ to denote a vector or a matrix of zeros of appropriate dimensions. For a sequence $\{z_k\}_{k=0}^{N-1}$ with $z_k\in\mathbb{R}^\eta$, we denote its stacked vector as $z = \begin{bmatrix}z_0^\top &z_1^\top & \dots & z_{N-1}^\top\end{bmatrix}^\top$ and a stacked window of it as $z_{[l,j]} = \begin{bmatrix}z_l^\top &z_{l+1}^\top & \dots & z_{j}^\top\end{bmatrix}^\top$ with $0\leq l<j$.\par
Persistence of excitation of a sequence and its extension to multiple sequences \cite{vanWaarde20} are defined as follows.
\begin{definition} The sequence \(\{z_k\}_{k=0}^{N-1}\), $z_k\in\mathbb{R}^{\eta}$, is said to be persistently exciting of order \(L\) if \(\textup{rank}(\mathscr{H}_{L}(z))=\eta L\), where $\mathscr{H}_L(z) = \begin{bmatrix}
		z_{[0,L-1]} & z_{[1,L]} & \cdots & z_{[N-L,N-1]}
	\end{bmatrix}$.
	\label{def_PE}
\end{definition}
\begin{definition}[\cite{vanWaarde20}]\label{def_cPE}
	The sequences $\{z_k^{(j)}\}_{k=0}^{N_j-1}$, with $z_k^{(j)}\in\mathbb{R}^\eta$ and $j\in\mathbb{Z}_{[1,r]}$, are said to be \textit{collectively persistently exciting} of order $L$ if rank$(\mathcal{H}_L(\mathscr{Z}))=\eta L$, where $\mathscr{Z} = \begin{bmatrix}
		(z^{(1)})^\top & \cdots & (z^{(r)})^\top
	\end{bmatrix}^\top,$ and
	\begin{equation*}
		\mathcal{H}_L(\mathscr{Z}) = \begin{bmatrix}
			\mathscr{H}_L(z^{(1)}) & \cdots & \mathscr{H}_L(z^{(r)})
		\end{bmatrix}.
	\end{equation*}
\end{definition}
\section{Solving the dynamics of the linear model}
\label{sec:linear_th}

\subsection[The feature cross-correlation matrix]{The feature cross-correlation matrix $\mGamma$}
The task we have set up is a matrix optimization problem.
To elucidate the structure of this problem, we can simplify $\mC$ as
\begin{align}
    \mC &= \frac{1}{2n} \sum_i
    \mW \left( \vx_i {\vx_i'}\T \! + \vx_i' \vx_i\T \right) \mW\T = \mW \mGamma \mW\T
\end{align}
where we have defined the feature cross-correlation matrix $\mGamma \equiv \frac{1}{2n} \sum_i ( \vx_i {\vx_i'}\T \! + \vx_i' \vx_i\T )$.
Equation \ref{eqn:loss_fn} then becomes
\begin{equation} \label{eqn:loss_WGW}
    \L = \norm{\mW \mGamma \mW\T - \mI_d}_F^2,
\end{equation}
a form reminiscent of matrix factorization problems, and
Equation \ref{eqn:dW_dt} is
\begin{equation} \label{eqn:dW_dt_expanded}
\frac{d \mW}{d t} = - 4 \left( \mW \mGamma \mW\T - \mI_d \right) \mW \mGamma.
\end{equation}
We will denote by $\gamma_1 \ge \ldots \ge \gamma_m$ the eigenvalues of $\mGamma$ and, for any $k \in 1 \ldots m$, let $\mGamma^{(\le k)} \in \R^{k \times m}$ be the matrix containing the top $k$ eigenvectors of $\mGamma$ as rows.

\subsection{Exact solutions for aligned initialization}
It is nontrivial to solve Equation \ref{eqn:dW_dt_expanded} from arbitrary initialization.
However, as is common in matrix factorization problems, we can obtain exact trajectories starting from special initializations, and these special solutions will shed light on the general dynamics.
We first consider an ``aligned initialization" in which the right singular vectors of $\mW_0$ are the top eigenvectors of $\mGamma$.
Concretely, let
\begin{equation} \label{eqn:W0_spectral_def}
    \mW_0 = \mU \mS_0 \mGamma^{(\le d)}
\end{equation}
be the singular value decomposition of $\mW_0$ with $\mU \in \R^{d \times d}$ an arbitrary orthonormal matrix, and $\mS_0 \in \R^{d \times d}$ is a matrix of singular values given by
\begin{equation} \label{eqn:S0_def}
    \mS_0 = \diag(s_1(0), ..., s_d(0))
\end{equation}
with $s_j(0) > 0$.\footnote{We assume alignment with the top $d$ eigenvectors both for notational simplicity and because this is the solution we will ultimately care about, but our exact solution will hold for any set of $d$ eigenvectors of $\mGamma$.}
The dynamics of $\mW(t)$ under Equation \ref{eqn:dW_dt_expanded} are then given by the following Proposition:
\begin{proposition}[Trajectory of $\mW(t)$ from aligned initialization]
\label{prop:exact_dynamics}
If $\mW(0) = \mW_0$ as given by Equations \ref{eqn:W0_spectral_def} and \ref{eqn:S0_def}, then
\begin{equation} \label{eqn:aligned_W_decomp}
\mW(t) = \mU \mS(t) \mGamma^{(\le d)}
\end{equation}
with
$\mS(t) = \diag(s_1(t), ..., s_d(t))$
and
\begin{equation} \label{eqn:s_j(t)}
    s_j(t) = \frac{e^{4 \gamma_j t}}{\sqrt{s_j^{-2}(0) + (e^{8 \gamma_j t} - 1) \gamma_j}}.
\end{equation}
\end{proposition}

\textit{Proof of Proposition \ref{prop:exact_dynamics}}.
Treating Equation \ref{eqn:aligned_W_decomp} as an ansatz and inserting it into Equation \ref{eqn:dW_dt_expanded}, we find that
\begin{equation}
    \frac{d \mW}{d t} =
    4 \mU (1 - \mD \mS(t)^2) \mD \mS(t) \mGamma^{(\le d)},
\end{equation}
with $\mD = \text{diag}(\gamma_1, \ldots, \gamma_d)$.
It follows that the singular vectors of $\mW(t)$ remain fixed, and the singular values evolve according to
\begin{equation} \label{eqn:sv_ode}
    s_j'(t) = 4 \left( 1 - \gamma_j s_j^2(t) \right) \gamma_j s(t).
\end{equation}


This ODE can be solved to yield Equation \ref{eqn:s_j(t)}.
\QED

When $t \rightarrow \infty$, Proposition \ref{prop:exact_dynamics} prescribes singular values equal to
\begin{equation} \label{eqn:s_j_inf}
s_j(\infty) = \left\{\begin{array}{ll}
        \gamma_j^{-1/2} & \text{for } \gamma_j > 0, \\
        s_j(0) & \text{for } \gamma_j = 0, \\
        0 & \text{for } \gamma_j < 0.
        \end{array}\right.
\end{equation}
Each singular value thus flows monotonically towards either $\gamma_j^{-1/2}$ or zero depending on whether the corresponding eigenvalue is positive or negative.
This can be understood by noting that the loss (Equation \ref{eqn:loss_WGW}) can be rewritten as
\begin{equation} \label{eqn:L_as_eigensum}
    \L = \sum_j (1 - \gamma_j s_j^2)^2,
\end{equation}
which makes clear that if $\gamma_j > 0$, then $s_j = \gamma_j^{-1/2}$ is optimal (and achieves zero loss on the $j$-th term of the sum), but if $\gamma_j < 0$, then the model can do no better than $s_j = 0$

It is worth noting that $\lambda_j = \gamma_j s_j^2$ is the corresponding eigenvalue of $\mC$.
With this change of coordinates, the trajectories of Proposition \ref{prop:exact_dynamics} become nearly sigmoidal, with $\lambda_j \approx (1 + \lambda_j^{-1}(0) e^{-8 \gamma_j t})^{-1}$ when $|\lambda_j(0)| \ll 1$.

We will be particularly interested in the set of terminal solutions.
Accordingly, let us define the set of \textit{top spectral} $\mW$ as follows:
\begin{definition}(Top spectral $\mW$) \label{def:spectramax}
A top spectral $\mW$ is one for which $\mW = \mU \tilde{\mS} \mGamma^{(\le d)}$,
with $\mU$ an orthogonal matrix and
$\tilde{\mS} = \diag(\gamma_1^{-1/2} \mathbbm{1}_{\gamma_1 > 0}, ..., \gamma_d^{-1/2} \mathbbm{1}_{\gamma_d > 0})$.
\end{definition}
(Note that these are precisely the set of $\mW(\infty)$ found by Proposition \ref{prop:exact_dynamics} save for the edge case $\gamma_j = 0$, in which case we set $s_j = 0$.)
These solutions form an equivalence class parameterized by the rotation matrix $\mU$.
As observed by \citet{haochen:2021-provable-guarantees-for-cl}, such a rotation makes no difference for the downstream generalization of a linear probe, so we may indeed view all top spectral $\mW$ as equivalent.

Let us assume henceforth that $\gamma_1, ..., \gamma_d > 0$ and $\gamma_d > \gamma_{d+1}$.
The top spectral $\mW$ achieve $\L(\mW)=0$, but note that there generally exist other optima, such as those aligned with a different set of positive eigenvectors.
However, the top spectral $\mW$ are optimal in the following sense:
\begin{proposition} \label{prop:Fnorm_min}
The top spectral $\mW$ are precisely the solutions to
\begin{equation}
\underset{\mW}{\text{argmin}} \norm{\mW}_F
\ \ \
\text{s.t.}
\ \ \
\L(\mW) = 0.
\end{equation}
\end{proposition}
We relegate the proof to Appendix \ref{app:derivation}.
Proposition \ref{prop:Fnorm_min} implies that, of all solutions achieving $\L = 0$, the top spectral solutions have minimal $\norm{\mW}_F$.
Noting that gradient descent often has an implicit bias towards low-norm solutions \citep{gunasekar:2017-implicit-reg-in-mf, soudry:2018-implicit-bias-of-gd}, we might expect to reach this set of optima from some more general initial conditions.

\subsection{The case of small initialization}
\label{subsec:small_init}
Returning to Proposition \ref{prop:exact_dynamics}, an informative special case of our exact dynamical solution is that in which the initial singular values are small relative to their final values $(s_j(0) \ll \gamma_j^{-1/2})$.
In this case, Equation \ref{eqn:s_j(t)} states that $s_j(t)$ will remain small up to a critical time
\begin{equation} \label{eqn:tau_j_def}
\tau_j = \frac{- \log (s_j^2(0) \gamma_j)}{8 \gamma_j}
\end{equation}
after which it will rapidly grow to its final value.
Note that $\tau_j$ is principally set by $\gamma_j$
, with only a weak logarithmic dependence on initialization $s_j(0)$.
The learning dynamics can thus be understood as a \textit{stepwise process} in which $d$ orthogonal directions are each rapidly learned at their respective timescales $\tau_j$, with plateaus in between adjacent learning phases.

Proposition \ref{prop:exact_dynamics} assumed a special aligned initialization for $\mW$.
We will now give a result which generalizes this significantly, showing that the trajectory from any \textit{small} initialization closely follows that from a particular aligned initialization.

In order to state our result, we will first define the QR factorization and an ``alignment transformation."
\begin{definition}[QR factorization.]
The \textit{QR factorization} of a matrix $\mM \in \R^{a \times b}$ returns a decomposition $\mQ \mR = \mM$ such that $\mQ \in \R^{a \times a}$ is orthogonal and $\mR \in \R^{a \times b}$ is upper-triangular with nonnegative diagonal.
If $a \le b$ and $\mM$ is full rank, then the QR factorization is unique.
\end{definition}
\begin{definition}[Alignment transformation.]
The \textit{alignment transformation} $\mathcal{A}$ of a matrix $\mM$ returns a matrix $\mathcal{A}(\mM) = \mQ \tilde{\mR}$, where $\mQ\mR = \mM$ is a QR factorization and $\tilde\mR$ is $\mR$ with all off-diagonal elements set to zero.
\end{definition}

We can now state the main result of this section.
\begin{result}[Trajectory from generic small initialization] $ $
\label{res:small_init}
\vspace{-2mm}
\begin{itemize}
    \item Let $\gamma_1, ..., \gamma_d$ be unique.
    \item Let $\tilde{\mW}_0 \in \R^{d \times m}$ with $\tilde{\mW}_0 \mGamma^{(\le d)}$ full rank.
    \item Let $\mW(t)$ be the solution to Equation \ref{eqn:dW_dt_expanded} with initial condition $\mW(0) = \alpha \tilde\mW_0$.
    \item Let $\mW^*(t)$ be the aligned solution with initial condition $\mW^*(0) = \mathcal{A}(\mW(0) {\mGamma^{(\le m)}}\T) \mGamma^{(\le m)}$.
\end{itemize}
Then as $\alpha \rightarrow 0$, $\norm{\mW(t) - \mW^*(t)}_F \rightarrow 0$ for all $t$.
\end{result}
We give a derivation of this result in Appendix \ref{app:derivation}.\footnote{We style this conclusion as a Result rather than a Theorem because we give an informal derivation rather than a formal proof.
We conjecture that this result can indeed be made formal.}

Result \ref{res:small_init} states that the trajectory from generic small initialization closely follows a particular aligned trajectory.
This aligned trajectory is given in closed form by Proposition \ref{prop:exact_dynamics}, and so this result gives us equations for the dynamics from arbitrary initialization.

Some intuition for this result can be gained by examining the construction of $\mW^*(0)$.
The aligned solution $\mW^*(t) = \mU \mS^*(t) \mGamma^{(\le d)}$ is composed solely of the top $d$ eigendirections of $\mGamma$, but an arbitrary initialization will have no such preference.
How does this alignment occur?
Note that, at early times when $\mW$ is small, the quadratic term of the loss will dominate, and Equation \ref{eqn:dW_dt_expanded} reduces to
\begin{equation}
    \frac{d \mW}{d t} \approx 4 \mW \mGamma
    \ \ \ \Rightarrow \ \ \
    \mW \approx \mW(0) \, e^{4 \mGamma t}.
\end{equation}
The top eigendirections of $\mGamma$ grow faster and will quickly dominate, and after a time $\tilde{\tau} \gg (\gamma_d - \gamma_{d+1})^{-1}$, we will have
\begin{equation}
    \mW \approx \mW(0) \mPi^{(\le d)} \, e^{4 \mGamma t}
\end{equation}
where $\mPi^{(\le d)} \equiv {\mGamma^{(\le d)}}\T \mGamma^{(\le d)}$ is the projector onto the top-$d$ eigenspace of $\mGamma$.
Components aligned with eigenmodes of index $j > d$ are thus negligible compared to those of index $\le d$ and do not interfere in the learning dynamics, which converge before such later eigenmodes can grow to order one.

Having identified the relevant eigendirections, we must now determine their effective initial singular values $s^*_j(0)$.
Let $\vv_j$ be the $j$-th eigenvector of $\mGamma$ with $\norm{\vv_j} = 1$ and define $\vu_j = \mW(0) \vv_j$.
If $\vv_j$ were a right singular vector of $\mW(0)$, we would have $s^*_j(0) = \norm{\vu_j}$.
We will not be so fortunate in general, however.
Examining Equation \ref{eqn:dW_dt_expanded}, we should expect each eigenmode to only be able to grow in the subspace of $\R^d$ which has not already been filled by earlier eigenmodes, which suggests that we take
\begin{equation} \label{eqn:s_j(0)_approx}
    s^*_j(0) = \norm{\left( 1 - \sum_{k < j} \frac{\vu_k\T \vu_k}{\norm{\vu_k}^2} \right) \vu_j}.
\end{equation}
These are precisely the singular values of $\mW^*(0)$.\footnote{This can be seen by noting that the QR factorization involves the Gram-Schmidt-like process of Equation \ref{eqn:s_j(0)_approx}.}





\subsection{Numerical simulation}
\label{subsec:linear_model_sim}

We perform basic numerical simulations of our linear problem which verify Proposition \ref{prop:exact_dynamics} and Result \ref{res:small_init}.
We sample $n=500$ random images from CIFAR-10 \citep{krizhevsky:2009} and, for each, take two random crops to size $20 \times 20 \times 3$ to obtain $n$ positive pairs (which thus have feature dimension $m = 1200$).
We then randomly initialize a linear model with output dimension $d=10$ and weights drawn i.i.d. from $\mathcal{N}(0,\alpha^2)$ with $\alpha = 10^{-7}$ and train with the loss of Equation \ref{eqn:loss_fn} with learning rate $\eta = 5 \times 10^{-5}$.

During training, we track both the loss and the eigenvalues of the embedding cross-correlation matrix $\mC$.
Our stepwise dynamics predict the loss will start at $\L(0) \approx d$ and decrease by one as each mode is learned, giving
\begin{equation}
    \L(t) \approx \sum_{j | \tau_j > t} 1.
\end{equation}
The eigenvalues of $\mC$ will be $(\gamma_1 s_1^2(t), ..., \gamma_d s_d^2(t))$, with $s_j(t)$ given by Proposition \ref{prop:exact_dynamics}.
The use of Proposition \ref{prop:exact_dynamics} requires values for $s_1(0), ..., s_d(0)$, and these can be found from Equation \ref{eqn:s_j(0)_approx} and the statistics of the random initialization to be roughly
\begin{equation}
    s_j(0) \approx \sigma \sqrt{d - j + 1}.
\end{equation}
The results are plotted in Figure \ref{fig:banner_figure}(A,B).
We find excellent agreement with our theory.


\section{Kernelizing the linear model}
\label{sec:kernel_th}

Our discussion has so far dealt with an \textit{explicit} linear regression problem in which we have direct access to the data features $\vx_i$.
We used this explicit representation to construct $\mGamma \in \R^{m \times m}$, which lives in feature space.
However, many models of interest are linear with an \textit{implicit} feature space, with the output a linear function not of the input $x$ but rather of a fixed transformation of the input $\vphi(x)$.
Models of this class include kernel machines \citep{shawe:2004}, random feature models and deep neural networks in which only the last layer is trained \citep{rahimi:2007, lee:2018-nngp}, and infinite-width neural networks, the latter of which evolve under the NTK \citep{jacot:2018}.
While these models are equivalent to linear models, we do not have an explicit representation of the features $\vphi(x)$ (which may be large or even infinite-dimensional).
What we \textit{do} have access to is the model's \textit{kernel function} $\K(x,x') = \vphi(x)\T \vphi(x')$.

In our original linear model, the kernel is simply the inner product $\vx\T \vx'$.
Reinterpreting $\vx$ as $\vphi(x)$, any quantity expressed solely in terms of such inner products can still be computed after kernelization.
Our challenge, then, is to rewrite our theory of dynamics entirely in terms of such inner products so we may apply it to general kernel machines.

\subsection{Kernelized solution}

We will manage to do so.
Let our datasets be $\X \equiv \{x_1, ..., x_n\}$ and $\X' \equiv \{x'_1, ..., x'_n\}$.
Let us define the kernel matrix $\Kaa \in \R^{n \times n}$ such that $[\Kaa]_{ij} = \K(x_i,x_j)$ and define $\Kab$, $\Kba$, and $\Kbb$ analogously.
% Let us define $\mX \equiv [\vx_1, ..., \vx_n]\T$ and $\mX' \equiv [\vx'_1, ..., \vx'_n]\T$.
% Let us define $\Kaa \equiv \mX \mX\T$ and $\Kab, \Kba, \Kbb$ similarly.
Let us also define
\begin{equation}
\tilde{\mK} \equiv \begin{bmatrix}
\Kaa & \Kab\\
\Kba & \Kbb
\end{bmatrix},
\end{equation}
the kernel over the combined dataset, as well as
\begin{equation}
\!\!\!
\mZ \equiv
\frac{1}{2n}
\begin{bmatrix}
\Kab\Kaa & \Kab^2\\
\Kbb\Kaa & \Kbb\Kab 
\end{bmatrix} + \text{[transpose]}.
\end{equation}
% Finally, let us define
% $\mK_\mGamma \equiv (\tilde{\mK}^+)^{1/2} \mG (\tilde{\mK}^+)^{1/2} \in \R^{2n \times 2n}$,
% where $(\, \cdot \,)^+$ denotes a matrix pseudoinverse.
% this is confusing
Finally, let us define
$\mK_\mGamma \equiv \tilde{\mK}^{-1/2} \mZ \tilde{\mK}^{-1/2} \in \R^{2n \times 2n}$,
where $\tilde{\mK}^{-1/2}$ is interpreted as $(\tilde{\mK}^+)^{-1/2}$ if $\tilde{\mK}$ is degenerate.
The matrix $\mK_\mGamma$ is symmetric and akin to a kernelized version of $\mGamma$\footnote{Here $\mGamma = \frac{1}{2n} \sum_i \left(\vphi(x_i) \vphi(x'_i)\T + \vphi(x'_i) \vphi(x_i)\T\right)$ since we have reinterpreted $\vx$ as $\vphi(x)$. This $\mGamma$ would be sufficient to use our theory of Section \ref{sec:linear_th} were it not constructed of \textit{outer} products of $\vphi(x_i)$'s, which are inaccessible after kernelization.}.

Let $(\zeta_j, \vb_j)$ be the eigenvalues and normalized eigenvectors of $\mK_\mGamma$ indexed in descending eigenvalue order.
The kernelization of our theory is given by the following proposition.

\begin{proposition}[Kernelized solution] \label{prop:kernelized_sol} $ $
\begin{enumerate}[(a)]
\item All nonzero eigenvalues of $\mK_\mGamma$ are eigenvalues of
$\mGamma$
% \vspace{-2mm}
% \begin{equation}
%     \mGamma = \frac{1}{2n} \sum_i \left(\vphi(x_i) \vphi(x'_i)\T + \vphi(x'_i) \vphi(x_i)\T\right)
% \end{equation}
% \vspace{-2mm}
and vice versa.
\item \label{propclause:kernelized_embs}
% \todo{check matrix shapes}
The top spectral solution gives the embeddings
\begin{equation} \label{eqn:kernelized_top_spectral_embs}
    \vf(x) = \mU \tilde{\mS} [\vb_1 \ ... \ \vb_d]\T \tilde{\mK}^{-1/2} [\mK_{x \X} \ \mK_{x \X'}]\T
\end{equation}
with
$\mU$ an orthogonal matrix,
$\tilde{\mS} = \diag(\gamma_1^{-1/2}, ..., \gamma_d^{-1/2})$,
and
$\mK_{x \X}, \mK_{x \X'} \in \R^{1 \times n}$
such that 
$[\mK_{\X x}]_i = \K(x_i,x)$ and $[\mK_{\X' x}]_i = \K(x_i,x')$.
\item The top spectral solutions correspond to the embeddings $\vf(x)$ given by
\begin{equation}
    \underset{\vf}{\text{argmin}} \norm{\vf}_\K
    \ \ \ \text{s.t.} \ \ \
    \L(\vf) = 0.
\end{equation}
\end{enumerate}
\end{proposition}
% Proposition \ref{prop:kernelized_sol} allows one to 
We give the proof and an expression for $\vf(x,t)$, the embeddings over time from small (aligned) initialization, in Appendix \ref{app:proofs}.
These results allow us to predict both the training dynamics and final embeddings of our contrastive learning problem with a black-box kernel method.

\subsection{Implications of kernelized solution}
This kernelized solution has several interesting properties that we will briefly discuss here.

\textbf{Special case: $\X = \X'$.}
In the case in which the two views of the data are identical, one finds that $\mK_\mGamma = \tilde\mK$ and the model simply learns the top eigenmodes of its base (neural tangent) kernel.

% should some of this discussion be moved to the intro or to a final discussion section?
\textbf{SSL as kernel PCA.}
Proposition \ref{prop:kernelized_sol}\ref{propclause:kernelized_embs} states that the final embeddings are governed by the top $d$ eigenvectors of the kernel-like matrix $\mK_\Gamma$.
With our setup, then,
\textbf{contrastive learning with neural networks amounts to \textit{kernel PCA} in the infinite-width limit.}
This is analogous to the fact that standard supervised learning approaches kernel \textit{regression} in the infinite-width limit\footnote{This analogy can be furthered by noting that these equivalences both occur as $t \rightarrow \infty$ and both require small or zero initialization.}.
This is rather satisfying in light of the fact that kernel regression and kernel PCA are the simplest supervised and unsupervised kernel methods, respectively.

\textbf{The same theory works for multimodal SSL.}
The above solution holds for the infinite-width NTK limit of any neural network architecture.
It is worth noting that this includes multimodal setups such as CLIP \citep{radford:2021-clip} in which representations for the two datasets $\X$ and $\X'$ are generated by \textit{different} models, and the two datasets may be of different types.
We can view the two models as one combined model with two pathways, and since these pathways share no parameters, we have $\Kab = \Kba = \mathbf{0}$\footnote{In the parlance of the original linear model, the feature vectors of $\X$ and $\X'$ lie in orthogonal subspaces, and the model is tasked with discovering correlations across these subspaces.}.
Both our linear and kernelized solutions remain valid and nonvacuous in this setting.

% gah I feel like there's something here to say and I'm not saying it
% okay maybe I got it now
\textbf{Generalization on downstream tasks.}
The quality of a learned representation is often assessed by fitting a downstream function $g^*$ (such as an image classification) with a linear function of the representation as $\hat{g}(x) = \vbeta\T \vf(x)$.
Downstream task performance will be good if $g^*$ lies largely in the linear span of the components of $\vf(x)$.
Since Proposition \ref{prop:kernelized_sol}\ref{propclause:kernelized_embs} yields $\vf(x)$ in closed form, generalization can thus be assessed in our setting.
We leave this direction for future work.

\textbf{Mapping from initial to final kernel.}
Downstream task performance will be determined by the learned kernel
$\K_{\text{emb}}(x,x') \equiv \vf(x)\T \vf(x')$,
which contains all the information of $\vf$ save for the arbitrary rotation $\mU$.
We can thus think of SSL as a process which maps an initial, naive kernel $\K$ to a final kernel $\K_{\text{emb}}$ which has learned the structure of the data.
Many other poorly-understood processes in deep learning --- most notably that of feature learning --- also have the type signature \texttt{(initial kernel, data) $\rightarrow$ (final kernel)}, but there exist few closed-form algorithms with this structure.
While representation learning and supervised feature learning are different processes, it seems likely that they will prove to be related\footnote{See \citet{geirhos:2020-ssl-vs-sl} and \citet{grigg:2021-ssl-vs-sl} for evidence in this direction.}, and thus our closed-form solution for the final kernel may be useful for the study of feature learning.


We present in section~\ref{ssec:faces} an application of PnP-HVAE on face images, using a pretrained state-of-the-art hierarchical VAE. 
Next, we study the application of our framework to natural images. To that end, we introduce  in section~\ref{ssec:patchVDVAE}  a patch hierachical VAE architecture, that is able to model natural images of different resolutions. In section~\ref{ssec:app_nat}, we provide deblurring, super-resolution and inpainting experiments to demonstrate the relevance of the proposed method.

Additional results are presented in Appendix~\ref{app:add}. All experiments can be reproduced using the code available at \url{https://github.com/jprost76/PnP-HVAE}.



\subsection{Face Image restoration (FFHQ)}\label{ssec:faces}
We first demonstrate the effectiveness of PnP-HVAE on highly structured data, by performing face image restoration.
Latent variable generative models can accurately model structured images such as face images \cite{karras2019style,vahdat2020nvae,child2021very,kingma2018glow}, and then be used to produce high quality restoration of such data. 
In our experiments, we use the VDVAE model of~\cite{child2021very}, pre-trained on the FFHQ dataset~\cite{karras2019style}, as our hierarchical VAE prior.
VDVAE has $L=66$ latent variable groups in its hierarchy and generates images at resolution $256\times256$.

We compare PnP-HVAE with the intermediate layer optimization algorithm (ILO)~\cite{daras2021intermediate} that is based on a different class of generative models than HVAE. ILO is a GAN inversion method which optimizes the image latent code along with the intermediate layer representation of a StyleGAN to generate an image consistent with a degraded observation.
We use the official implementation of ILO, along with a StyleGAN2 model~\cite{karras2020analyzing, stylegan2pytorch}, that was trained for 550k iterations on images of resolution $256\times256$ from FFHQ.  
As VDVAE and StyleGAN models are not trained on the same train-test split of FFHQ, we chose to evaluate the methods on a subset of 100 images from the CelebA dataset~\cite{liu2018large}. 
For super-resolution, the degradation model corresponds to the application of a gaussian low-pass filter followed by a $\times 4$ sub-sampling, and the addition of a gaussian white noise with $\sigma=3$.
For the deblurring, we considered motion blur and  gaussian kernels, both with a noise level $\sigma=8$. %

We provide quantitative comparisons in table~\ref{table:comp_ILO}, along with a visual comparison of the results in figure~\ref{fig:face_restoration}.
PnP-HVAE has the best  PSNR and SSIM results for all the considered restoration tasks, while ILO provides better results  for the perceptual distance.
By jointly optimizing the image and its latent variable, PnP-HVAE provides  results that are both realistic and consistent with the degraded observation.
On the other hand,  ILO  only optimizes on an extended latent space. This method generates  sharp and realistic images with better LPIPS scores,   
but the results lack  of consistency with respect to the observation, which explains the overall lower PSNR performance. 






\subsection{PatchVDVAE: a HVAE for natural images}\label{ssec:patchVDVAE}
Available generative models in the literature operate on images of  fixed resolutions and
are either restrained to datasets of limited diversity, or even to registered face images~\cite{kingma2018glow,child2021very, vahdat2020nvae, karras2019style}, or requiring additional class information~\cite{brock2018large, dhariwal2021diffusion, song2020score, luhman2022optimizing}.
Fitting an unconditional model on natural images appears to be a more difficult task, as their resolution can change, and their content is highly diverse.
The complexity of the problem can be reduced by learning a prior model on patches of reduced dimension. 
For image restoration problems, the patch model can be reused on images of higher dimensions~\cite{zoran2011learning,prost2021learning,altekruger2022patchnr}. When the model is a full CNN, the prior on the set of the  patches can  be computed efficiently by applying the network on the full image~\cite{prost2021learning}.

We thus introduce  patchVDVAE, a fully convolutional hierarchical VAE.
Contrary to existing HVAE models whose resolution is constrained by the constant tensor at the input of the top-down block, patchVDVAE can generate images of different resolutions by controlling the dimension of the input latent. 
This amounts to defining a prior on patches whose dimension corresponds to the receptive field of the VAE. A similar model is used for image denoising in~\cite{prakash2021interpretable}.

 
For PatchVDVAE architecture, we use the same bottom-up and top-down blocks as VDVAE~\cite{child2021very}, and replace the constant trainable input in the first top-down block by a latent variable, to make the model fully convolutional (details on the  architecture are given in Appendix~\ref{app:details}). 
The training dataset is composed of $128\times 128$ patches extracted from a combination of DIV2K~\cite{agustsson2017ntire} and Flickr2K~\cite{Lim_2017_CVPR_workshops} datasets.
We perform data augmentation by extracting  patches at $3$ resolutions: HR-images and $\times 2$ and $\times 4$ downscaled images. 
The model is trained for $7.10^5$ iterations with a batch size of $64$. Following the recommendation of~\cite{hazami2022efficient}, we use Adamax optimizer with an exponential moving average and gradient smoothing of the variance.
We set the decoder model to be a gaussian with diagonal covariance, as in~\cite{luhman2022optimizing}.
PatchVDVAE is fully convolutional and can generate images of dimension that are multiples of $64$ as illustrated by
figure~\ref{fig:vdvae}.

\newlength{\patchwidth}
\setlength{\patchwidth}{0.135\columnwidth}
\begin{figure}[!ht]
    \centering
    \begin{subfigure}[t]{.34\columnwidth}\hspace{0.1cm}
        \setlength{\tabcolsep}{0.02pt}
\renewcommand{\arraystretch}{0}
        \begin{tabular}{*{2}{p{1.03\patchwidth}}}
            \includegraphics[width=\patchwidth]{figures_arxiv/patchVDVAE/samples/generated/64x64/setup-5-image-0018.png} &
            \includegraphics[width=\patchwidth]{figures_arxiv/patchVDVAE/samples/generated/64x64/setup-5-image-0016.png} \\
            \includegraphics[width=\patchwidth]{figures_arxiv/patchVDVAE/samples/generated/64x64/setup-5-image-0008.png} &
            \includegraphics[width=\patchwidth]{figures_arxiv/patchVDVAE/samples/generated/64x64/setup-5-image-0019.png}   
        \end{tabular}
    \end{subfigure}\hspace{-0.15cm}
    \begin{subfigure}[t]{.64\columnwidth}
\begin{tabular}{cc}\vspace{-0.1cm}
\includegraphics[width=2\patchwidth]{figures_arxiv/patchVDVAE/samples/generated/256x256/setup-2-image-0009.png}&
        \includegraphics[width=2\patchwidth]{figures_arxiv/patchVDVAE/samples/generated/256x256/setup-2-image-0002.png}\end{tabular}

    \end{subfigure}
    \caption{\label{fig:vdvae} Left: $64\times64$ patches samples from our patchVDVAE model trained on patches from natural images.
    Right: PatchVDVAE is fully convolutional and it can generate images of higher resolution (here: $128\times128$).\vspace{-0.2cm}}
\end{figure}

\subsection{Natural images restoration}\label{ssec:app_nat}
We  evaluate PnP-HVAE on natural image restoration.
For each task, we report the average value of the PSNR, the SSIM, and the LPIPS metrics on $20$ images from the test set of the BSD dataset~\cite{MartinFTM01}.\\


\noindent
{\bf Image deblurring.}
In the experiments, we consider $2$ gaussian kernels and $2$ motion blur kernels from~\cite{levin2009understanding}, with $3$ different noise levels 
$\sigma \in \{2.55, 7.65, 12.75\}$.
As a baseline we consider  EPLL~\cite{zoran2011learning}, which learns a prior on image patches with a gaussian mixture model.
We also compare PnP-HVAE  with PnP-MMO and GS-PnP, $2$ competing convergent Plug-and-Play methods based on CNN denoisers.
PnP-MMO~\cite{pesquet2021learning} restricts the denoiser to be contraction in order to guarantee the convergence of the PnP forward-backard algorithm. GS-PnP~\cite{hurault2022gradient} considers a gradient step denoiser and reaches state-of-the-art performances of non converging methods~\cite{zhang2021plug}.
We set the temperature $\tau$  in our method as $0.95$, $0.8$ and $0.6$ for noise levels $2.55$, $7.65$ and $12.75$ respectively, and we let it run for a maximum of $50$ iterations. 
For the three compared methods we use the official implementations and pre-trained models provided by the respective authors. 
Details on the choice of hyperparameters for the concurrent methods are provided in the Appendix~\ref{app:details}
Figure~\ref{fig:deblurring_bsd} illustrates that our method provides correct deblurring results. 

According to table~\ref{tab:deb}, the performance of PnP-HVAE is between those of EPLL and GS-PnP and it outperforms PnP-MMO for large noise levels.\\

\begin{table}
\begin{center}\footnotesize
    \begin{tabular}{>{\centering}m{.3cm}*{5}{c}}
    $\sigma$ &Method & PSNR$\uparrow$ & SSIM$\uparrow$ & LPIPS$\downarrow$  \\ 
    \hline
    \multirow{4}{*}{\vcell{$2.55$}}
    & PnP-HVAE & $27.75$ & $0.79$ & $0.31$\\
    & GS-PNP \cite{hurault2022gradient} & $\mathbf{29.59}$ & $\mathbf{0.84}$ & $\mathbf{0.22}$\\
    & EPLL \cite{zoran2011learning} & $26.49$ & $0.71$ & $0.36$\\ 
    & PnP-MMO \cite{pesquet2021learning} & $\underbar{29.50}$ & $\underbar{0.83}$ & $\underbar{0.20}$ \\ \hline
    \multirow{4}{*}{\vcell{$7.65$}}
    & PnP-HVAE & $\underbar{26.36}$ & $\underbar{0.72}$ & $\underbar{0.40}$\\
    & GS-PNP \cite{hurault2022gradient} & $\mathbf{27.33}$ & $\mathbf{0.77}$ & $\mathbf{0.31}$\\
    & EPLL \cite{zoran2011learning} & $24.04$ & $0.66$ & $0.45$ \\ 
    & PnP-MMO \cite{pesquet2021learning} & $25.34$ & $0.69$ & $0.34$\\
    \hline
    \multirow{4}{*}{\vcell{$12.75$}}
    & PnP-HVAE & $\underbar{25.12}$ & $\mathbf{0.73}$ & $\underbar{0.47}$\\
    & GS-PNP \cite{hurault2022gradient} & $\mathbf{26.32}$ & $\mathbf{0.73}$ & $\mathbf{0.37}$\\
    & EPLL \cite{zoran2011learning} & $23.28$ & $0.61$ & $0.51$ \\ 
    & PnP-MMO \cite{pesquet2021learning} & $22.42$ & $0.53$& $0.54$ \\
    \hline
    &\vspace*{-.3cm}\\
            \multicolumn{2}{c}{Blur and motion kernels}& \multicolumn{3}{c}{
        \includegraphics*[scale=1]{figures_arxiv/kernels/4.png}\;\includegraphics*[scale=1]{figures_arxiv/kernels/7.png}\;\includegraphics*[scale=1]{figures_arxiv/kernels/9.png}\;\includegraphics*[scale=1]{figures_arxiv/kernels/11.png}} 
    \end{tabular}
        \caption{\label{tab:deb}Comparison  of PnP-HVAE  and other restoration methods on deblurring. Results are averaged on $4$ kernels.\vspace{-0.2cm}}% on image deblurring.}
    \end{center}
\end{table}

\begin{figure}
    
    \begin{subfigure}[h]{\linewidth}
        \centering
        \includegraphics*[width=\columnwidth]{figures_arxiv/deb_s255_k7.pdf}\vspace{-0.1cm}
        \caption{Gaussian blur, $\sigma=2.55$}
    \end{subfigure}
    \begin{subfigure}[h]{\linewidth}
        \centering
        \includegraphics*[width=\columnwidth]{figures_arxiv/deb_s765_k11.pdf}\vspace{-0.1cm}
        \caption{Motion blur, $\sigma=7.65$}
    \end{subfigure}\vspace*{-0.1cm}
    \caption{\label{fig:deblurring_bsd} Natural image deblurring\vspace{-0.1cm}}
\end{figure}

\noindent {\bf Effect of the temperature.}
PnP-HVAE gives control on the temperature of the prior over the latent space.
In figure~\ref{fig:temp_effect}, we illustrate that reducing the temperature increases the strength of the regularization prior. In this example the tuning $\tau=0.7$ produces the best performance.\\
\begin{figure}[!ht]
   
    \includegraphics[width=\columnwidth]{figures_arxiv/demo_temp.pdf}\vspace{-0.15cm}
    \caption{ \label{fig:temp_effect} Effect of the temperature in PnP-VAE on a deblurring problem, with $\sigma=7.65$.\vspace{-0.15cm}}
\end{figure}


\noindent
{\bf Image inpainting.}
Next we consider the task of noisy image inpainting. 
We compose a test-set of 10 images from the validation set of BSD~\cite{MartinFTM01} and we create masks
  by occluding diverse objects of small size in the images. 
A gaussian white noise with $\sigma=3$ is added to the images.
As a comparaison, we still consider GS-PnP and EPLL.
For PnP-HVAE, the temperature is set to $\tau=0.6$, and the algorithm is run for a maximum of $200$ iterations, unless the residual $||\x_{k+1}-\x_k||$ is on a plateau.
We provide on Table~\ref{tab:inpainting_bsd} the distortion metrics with the ground truth, as well as a visual
\begin{table}



\begin{center}
    \begin{tabular}{cccc}
        & PSNR$\uparrow$ & SSIM$\uparrow$ &LPIPS$\downarrow$ \\\hline
        PnP-HVAE  & $\mathbf{29.54}$ & $\mathbf{0.93}$ & $\mathbf{0.06}$\\
        GS-PNP & $28.52$ & $\mathbf{0.93}$ & $0.09$\\
        EPLL & $\underline{29.16}$ & $\mathbf{0.93}$ & $\mathbf{0.06}$\\
    \end{tabular}
    \caption{\label{tab:inpainting_bsd}Quantitative evaluation for inpainting on BSD.}
    \end{center}
\end{table}
comparison on figure~\ref{fig:inpainting_bsd}. 
With its hierarchical structure,  PnP-HVAE outperforms the compared methods. \vspace{0.05cm}



\begin{figure}[!h]
    \includegraphics[width=\columnwidth]{figures_arxiv/demo_inp_bsd2.pdf}\vspace{-0.1cm}
    \caption{\label{fig:inpainting_bsd}Natural image inpainting\vspace{-0.3cm}}
\end{figure}











\vspace{-2mm}
\section{Conclusions}
We consider the phase-extraction problem, and we showed that, given a unitary $U = e^{i\pi H}$ and its inverse $U^{\dag}$, we could implement a block-encoding of $\phi(H)$ for some smooth function $\phi(x)$. The word `smooth' here means existence and continuity of the derivatives: the higher the number of continuous derivatives that a function has, the faster its Fourier sum (and thus the Laurent polynomial on the eigenphases) uniformly converges to that function. We are confident this can have many more applications beyond what is shown in this work. It is also worth remarking that Jackson showed that the convergence rate of a Fourier series is almost-optimal, in the sense that no trigonometric (or, equivalently, complex exponential) series can approximate the desired function faster, up to that $\log d$ factor~\cite[p.\ 21]{jacksonTheoryApproximation1930a}. Also remember that `smoothing' a function, i.e., replacing its derivative with a continuous function, does not give faster convergence for free in general, as its derivative will become steep in the points where we smooth out discontinuities, and this translates to a high Lipschitz constant: a~clear example is given by Eq.~\ref{eq:lipschitz-constant-recurrence-solution}, but in that case, fortunately, nothing depends on the size of the input $N$, and thus does not influence the asymptotic query complexity of Algorithm~\ref{alg:prop-sampling-qsp}, although the constant factor can become large even for $p = 20$. From a theoretical point of view, this work shows that, for any $\eta > 0$, there is an algorithm with query complexity 
$$\Tilde{\bigO}\left(\frac{1}{\bar{c}^{\frac{1}{2} + \eta}} \frac{1}{\epsilon^\eta} \right)$$
solving the proportional-sampling problem. This statement seems to suggest there exists an algorithm which directly solves the problem with $\eta = 0$, and an open question would be to find such algorithm.


It is also interesting to remark that Theorems~\ref{thm:haah-construction},~\ref{thm:haah-completion} indeed allow the construction for any $\phi$, even complex-valued, provided that its absolute value is reciprocal.

One could think that, in Section~\ref{sec:prop-sampling}, instead of using the linear function in the phase-extraction subroutine, we could approximate the square root and then apply the transformation directly on $e^{i \pi c(x)}$. However, in the case of proportional sampling this would be inconvenient, as the derivative of the square root function has a discontinuity with an infinite jump around 0, and we could not choose a constant $\delta$ if we had values of the oracle that are too close to $0$.


























\vspace{-3mm}
\section*{Acknowledgements}
\vspace{-2mm}
The authors thank Bobak Kiani, Randall Balestreiro, Vivien Cabannes, Kanjun Qiu, Ellie Kitanidis, Bryden Fogelman, Bartosz Wr\'{o}blewski, Nicole Seo, Nikhil Vyas, and Michael DeWeese for useful discussions and comments on the manuscript. JS gratefully acknowledges support from the National Science Foundation Graduate Fellow Research Program (NSF-GRFP) under grant DGE 1752814.


\bibliography{ml_refs}

\bibliographystyle{icml2023}

\appendix
\onecolumn

\section{Additional figures}
\label{sec:addnl_figs}

\begin{figure}[H]
  \includegraphics[width=17cm]{img/updated/fig5_realistic_centered.png}
  \vspace{-3mm}
  \caption{
    \textbf{Stepwise learning in SSL with ResNets with standard initialization and hyperparameters.}
    Embedding eigenvalues over time for \textbf{(A)} Barlow Twins, \textbf{(B)} SimCLR, and \textbf{(C)} VICReg.
    Dashed horizontal lines show rough eigenvalue thresholds separating the cluster of modes which have grown from the cluster of modes which have not yet grown and match the thresholds of Figure \ref{fig:histograms_realistic}.
  }
  % \vspace*{6.5in}
  \label{fig:dynamics_realistic}
\end{figure}

\begin{figure}[H]
  \includegraphics[width=17cm]{img/updated/fig6_hists_final.pdf}
  \vspace{-3mm}
  \caption{
    \textbf{Bimodal distribution of embedding eigenvalues shown over time for SSL with ResNets with standard initialization and hyperparameters.}.
    Histograms of embedding eigenvalues at selected times throughout training for \textbf{Top row:} Barlow Twins. \textbf{Middle row:} SimCLR. \textbf{Bottom row:} VICReg.
    Dashed vertical lines indicate at the same eigenvalue thresholds as in Figure \ref{fig:dynamics_realistic}.
  }
  % \vspace*{6.5in}
  \label{fig:histograms_realistic}
\end{figure}

\begin{figure}[H]
  \centering
  \includegraphics[width=17cm]{img/updated/fig_reps_updated.png}
  \vspace{-3mm}
  \caption{
    \textbf{Hidden representations exhibit learning steps aligned with those of embeddings.}
    Eigenvalues vs time $t = \text{[lr]}\times\text{[step]}$ for
    \textbf{(A, B)} a ResNet trained with Barlow Twins loss,
    \textbf{(C, D)} a ResNet trained with SimCLR loss, and
    \textbf{(E, F)} a ResNet trained with VICReg loss, all trained on STL-10 with small initialization. Top row: for Barlow Twins, the C matrix is used; for SimCLR and VICReg the centered PCA matrix is used.
  }
  \label{fig:rep_vals}
\end{figure}
% To hide proofs : \newcommand{\maybehide}[1]{}
% To show proofs : \newcommand{\maybehide}[1]{#1}
\newcommand{\maybehide}[1]{#1}

\section{Proofs}
\label{sec:proofs}

\subsection{Weak Open CBV}

\subsubsection{General Lemmas}

\begin{proposition}[{\bf Diamond}]
    \label{prop:diamond}
The relation $\redcbv$ enjoys the diamond property: if $t \redcbv t_i\ (i=1,2)$ and $t_1 \neq t_2$, then there exists $t_3$ such that $t_i \redcbv t_3\ i=1,2$.
\end{proposition}

\propcharnfs*

\maybehide{\begin{proof}
    We are going to show this proposition by splitting the original statement into the two following ones:
    \begin{enumerate}
        \item \label{prop:char-nfs:1} $t \not\dred$ and $\neg\isvalue{t}$ iff $t \in \neutral$.
        \item \label{prop:char-nfs:2} $t \not\dred$ iff $t \in \normal$.
    \end{enumerate}
    The proof now follows by simultaneous induction over both these statements:
    \begin{itemize}
        \item[$\Ra$)] By induction over $t$: 
        \begin{enumerate}
            \item Let $t \not\dred$ and $\neg\isvalue{t}$. We want to show that $t \in \neutral$:
            \begin{itemize}
                \item Case $t = x$ or $t = \lam x.u$. Then $\neg\isvalue{t}$ does not hold. Therefore, the statement holds vacuously.
                \item Case $t = u p$. Since $u p \not\dred$, then, in particular, it must be the case that either $\neg\isabs{u}$ or $\neg\isvalue{p}$ must hold, according to rule (\ruleBeta):
                \begin{itemize}
                    \item Assume $\neg\isabs{u}$ holds. It must be the case that $u \not\dred$, according to rule (\ruleAppL). And it also must be the case that $p \not\dred$, according to rule (\ruleAppR). Therefore, $p \in \normal$, by the \ih (\cref{prop:char-nfs}.\ref{prop:char-nfs:2}). Now, we have to consider $u$, which can be a variable, or not:
                    \begin{itemize}
                        \item Case $u = x$. Then $u p \in x \ \normal \in \neutral$.
                        \item Case $u$ is not a variable. Then $\neg\isvalue{u}$ holds. Therefore, we have $u \in \neutral$, by the \ih (\cref{prop:char-nfs}.\ref{prop:char-nfs:1}). Thus, $u p \in \neutral \ \normal \in \neutral$.
                    \end{itemize}
                    \item Assume $\neg\isvalue{p}$ holds. Then it must be the case that $u \not\dred$, according to rule (\ruleAppL). And that $p \not\dred$, according to rule (\ruleAppR). Therefore, $u \in \normal$, by the \ih (\ref{prop:char-nfs}.\ref{prop:char-nfs:2}), and $p \in \neutral$, by the \ih (\cref{prop:char-nfs}.\ref{prop:char-nfs:1}). Thus, $u p \in \normal \ \neutral \in \neutral$.
                \end{itemize}
            \end{itemize}
            \item Let $t \not\dred$. We want to show that $t \in \normal$:
            \begin{itemize}
                \item Case $t \in \val$. Then, clearly $t \in \normal$.
                \item Case $t \not\in \val$. Then, $\neg\isvalue{t}$ holds. Therefore, $t \in \neutral$, by \cref{prop:char-nfs}.\ref{prop:char-nfs:1}. Thus, in particular, $t \in \normal$.
            \end{itemize}
        \end{enumerate}
        \item[$\La$)] By induction over $t \in \normal$:
        \begin{enumerate}
            \item Let $t \in \neutral$. We want to show that $t \not\dred$ and $\neg\isvalue{t}$:
            \begin{itemize}
                \item Case $t = u p \in x \ \normal$. Then $u = x$ and $p \in \normal$. Since $u = x$, then both rules (\ruleBeta) and (\ruleAppL) cannot be applied. Since $p \in \normal$, then $p \not\dred$, by the \ih (\cref{prop:char-nfs}.\ref{prop:char-nfs:2}). Therefore, rule (\ruleAppR) also cannot be applied. Thus, $u p \not\dred$. And we can conclude, since $\neg\isvalue{u p}$ clearly holds.
                \item Case $t = u p \in \normal \ \neutral$. Then $u \in \normal$ and $p \in \neutral$. Since $u \in \normal$, then $u \not\dred$, by the \ih (\cref{prop:char-nfs}.\ref{prop:char-nfs:2}). Since $p \in \neutral$, then $p \not\dred$ and $\neg\isvalue{p}$ holds, by the \ih (\cref{prop:char-nfs}.\ref{prop:char-nfs:1}). Since $\neg\isvalue{p}$, then rule (\ruleBeta) cannot be applied. Since $u \not\dred$ and $p \not\dred$, then rules (\ruleAppL) and (\ruleAppR) cannot be applied. Therefore, $u p \not\dred$. And we can conclude since $\neg\isvalue{u p}$ clearly holds.
                \item Case $t = u p \in \neutral \ \normal$. Then $u \in \neutral$ and $p \in \neutral$. Since $u \in \neutral$, then $u \not\dred$ and $\neg\isvalue{u}$ holds, by the \ih (\cref{prop:char-nfs}.\ref{prop:char-nfs:1}). Since $p \in \normal$, then $p \not\dred$, by the \ih (\cref{prop:char-nfs}.\ref{prop:char-nfs:2}). Since $\neg\isvalue{u}$, then rule (\ruleBeta) cannot be applied. Since $u \not\dred$ and $p \not\dred$, then rules (\ruleAppL) and (\ruleAppR) cannot be applied. Therefore $u p \not\dred$. And we can conclude since $\neg\isvalue{u p}$ clearly holds.
            \end{itemize}
            \item Let $t \in \normal$. We want to show that $t \not\dred$:
            \begin{itemize}
                \item Case $t \in \val$. Then, clearly $t \not\dred$.
                \item Case $t \not\in \val$. Then, $t \in \neutral$, by definition. Thus, $t \not\dred$ holds, by~\cref{prop:char-nfs}.\ref{prop:char-nfs:1}.
            \end{itemize}
        \end{enumerate}
    \end{itemize}
\end{proof}
}
  
\begin{lemma}[Relevance]
    Let $\Phi \tr \seqi{\Gam}{t}{\tau}{(b,s)}$. Then $\dom{\Gam} \subseteq \fv{t}$.
\end{lemma}

\maybehide{\begin{proof}
    The proof following by induction over $\Phi$. Case $\Phi$ ends with rule (\ruleAx) or (\ruleLamP), then $\Phi$ is clearly relevant. The other cases following easily from the \ih.
\end{proof}}

\subsubsection{Soundness (Auxiliary Lemmas)}

\begin{lemma}
    \label{lem:values-not-neutral}
    Let $\Phi \tr \seqi{\Gam}{t}{\tau}{(b,s)}$. If $t \in \val$, then $\tau \not= \tneutral$.
\end{lemma}

\maybehide{\begin{proof}
    By case analysis on the form of $t \in \val$:
    \begin{itemize}
        \item Case $t = x$. Then we have to consider two additional cases according to the last rule used in $\Phi$:
        \begin{itemize}
            \item Case $\Phi$ ends with rule (\ruleAx), then $\tau$ is of the form $\sig \not= \tneutral$.
            \item Case $\Phi$ ends with rule (\ruleMany), then $\tau$ is of the form $\M \not= \tneutral$.
        \end{itemize}
        \item Case $t = \lam x.t$. Then we have to consider three additional cases according to the last rule used in $\Phi$:
        \begin{itemize}
            \item Case $\Phi$ ends with rule (\ruleLam), then $\tau$ is of the form $\M \ta \del \not= \tneutral$.
            \item Case $\Phi$ ends with rule (\ruleMany), then $\tau$ is of the form $\M \not= \tneutral$.
            \item Case $\Phi$ ends with rule (\ruleLamP), then $\tau = \tabs \not= \tneutral$.
        \end{itemize}
    \end{itemize}
\end{proof}}

\begin{lemma}
    \label{lem:notabs-implies-negabs}
    If $\Phi \tr \seqi{\Gam}{t}{\tau}{(b,s)}$, such that $\Gam$ is tight. If $\tau \in \nott{\tabs}$, then $\neg\isabs{t}$.
\end{lemma}

\maybehide{\begin{proof}
    By induction over $\Phi$:
    \begin{itemize}
        \item Case $\Phi$ ends with rule (\ruleAx), (\ruleApp), (\ruleAppPOne), or (\ruleAppPTwo), then $\neg\isabs{t}$ holds by definition.
        \item Case $\Phi$ ends with rule (\ruleLam), (\ruleMany), or (\ruleLamP),  then $\tau \not\in \nott{\tabs}$. Therefore, these cases do not apply.
    \end{itemize}
\end{proof}}

\begin{lemma}[{\bf Zero Steps and Normal Forms}]
    \label{lem:zero-steps-nfs}
    Let $\Phi \tr \seqi{\Gam}{t}{\tau}{(b,s)}$ be tight. $b = 0$ iff $t \in \normal$.
\end{lemma}

\maybehide{\begin{proof} \mbox{}
    \begin{itemize}
        \item[$\Ra$)] We want to show that, if $b = 0$, then $t \in \normal$. For this, we are going to split the original statement into the two following ones:
        \begin{enumerate}
            \item \label{lem:zero-steps-nfs:1} Let $\Phi \tr \seqi{\Gam}{t}{\tau}{(0,s)}$ be tight and $\neg\isvalue{t}$, then $t \in \neutral$.
            \item \label{lem:zero-steps-nfs:2} Let $\Phi \tr \seqi{\Gam}{t}{\tau}{(0,s)}$ be tight, then $t \in \normal$.
        \end{enumerate}
        The proof now follows by simultaneous induction over both these statements:
        \begin{enumerate}
            \item Let $\Phi \tr \seqi{\Gam}{t}{\tau}{(0,s)}$ be tight and $\neg\isvalue{t}$:
            \begin{itemize}
                \item Case $\Phi$ ends with rule (\ruleAx), (\ruleLam), (\ruleMany), or (\ruleLamP), then $\isvalue{t}$ holds. Therefore, these cases do not apply.
                \item Case $\Phi$ ends with rule (\ruleApp), then $b > 0$. Therefore, this case does not apply.
                \item Case $\Phi$ ends with rule (\ruleAppPOne), then $t$ is of the form $up$ and $\Phi$ is of the following form:
                \[ \begin{prooftree}
                    \hypo{\Phi_u \tr \seqi{\Gam_u}{u}{\nott{\tabs}}{(0,s_u)}}
                    \hypo{\Phi_p \tr \seqi{\Gam_p}{p}{\tightt}{(0,s_p)}}
                    \infer2[(\ruleAppPOne)]{\seqi{\Gam_u + \Gam_p}{up}{\tneutral}{(0,1+s_u+s_p)}}
                \end{prooftree} \]
                where $\tau = \tneutral$, $\Gam = \Gam_u + \Gam_p$ is tight, and $s = 1 + s_u + s_p$. Moreover, $\Gam_u$ and $\Gam_p$ are tight. By the \ih (\cref{lem:zero-steps-nfs}.\ref{lem:zero-steps-nfs:2}) over $\Phi_u$ and $\Phi_p$, we have that $u, p \in \normal$. By~\cref{lem:notabs-implies-negabs}, we have that $\neg\isabs{u}$. Therefore, either $u$ is a variable or $u \in \neutral$ by definition. So, in both cases, we can conclude that $u p \in \neutral$.
                \item Case $\Phi$ ends with rule (\ruleAppPTwo), then $t$ is of the form $up$ and $\Phi$ is of the following form:
                \[ \begin{prooftree}
                    \hypo{\Phi_u \tr \seqi{\Gam_u}{u}{\tightt}{(0,s_u)}}
                    \hypo{\Phi_p \tr \seqi{\Gam_p}{p}{\tneutral}{(0,s_p)}}
                    \infer2[(\ruleAppPTwo)]{\seqi{\Gam_u + \Gam_p}{up}{\tneutral}{(0,1+s_u+s_p)}}
                \end{prooftree} \]
                where $\tau = \tneutral$, $\Gam = \Gam_u + \Gam_p$, and $s = 1 + s_u + s_p$. Moreover, $\Gam_u$ and $\Gam_p$ are tight. By the \ih (\cref{lem:zero-steps-nfs}.\ref{lem:zero-steps-nfs:2}) over $\Phi_u$, we have that $u \in \normal$. By applying~\cref{lem:values-not-neutral} to $\Phi_p$, we have that $\neg\isvalue{p}$. By the \ih (\cref{lem:zero-steps-nfs}.\ref{lem:zero-steps-nfs:1}) over $\Phi_p$, we have that $p \in \neutral$. So, in both cases, we can conclude that $up \in \neutral$.
            \end{itemize}
            \item Let $\Phi \tr \seqi{\Gam}{t}{\tau}{(0,s)}$ be tight:
            \begin{itemize}
                \item Case $\Phi$ ends with rule (\ruleAx), (\ruleLam), or (\ruleLamP). Then, clearly $t \in \val$, so we can conclude immediately.
                \item Case $\Phi$ ends with rule (\ruleMany), then $\tau$ is of the form $\M \not\in \tightt$. Therefore, this case does not apply.
                \item In all the remaining cases $\neg\isvalue{t}$ holds. Then $t \in \neutral$, by \cref{lem:zero-steps-nfs}.\ref{lem:zero-steps-nfs:1}, so $t \in \normal$.
            \end{itemize}
        \end{enumerate}
        \item[$\La)$] We want to show that, if $t \in \normal$, then $b = 0$. The proof follows by induction over $t \in \normal$:
        \begin{enumerate}
            \item Case $t \in \neutral$. Then we have to consider the following additional cases:
            \begin{itemize}
                \item Case $t = xp$, such that $p \in \normal$. Then there are three additional cases to consider:
                \begin{itemize}
                    \item Case $\Phi$ ends with (\ruleApp), then it must be of the following form:
                    \[ \begin{prooftree}
                        \hypo{\seqi{x : \mul{\M \ta \tau}}{x}{\M \ta \tau}{(0,0)}}
                        \hypo{\Phi_p \tr \seqi{\Gam_p}{p}{\M}{(b_p,s_p)}}
                        \infer2[(\ruleApp)]{\seqi{(x : \mul{\M \ta \tau}) + \Gam_p}{xp}{\tau}{(1+b_p,s_p)}}
                    \end{prooftree} \]
                    where $\Gam = (x : \mul{\M \ta \tau}) + \Gam_p$ is tight, $b = 1+b_p$, and $s = s_p$. But, $\mul{\M \ta \tau}$ is not tight, since $\M \ta \tau \not\in \tightt$. Therefore, this case does apply.
                    \item Case $\Phi$ ends with (\ruleAppPOne), then $\Phi$ must be of the following form:
                    \[ \begin{prooftree}
                        \hypo{\seqi{(x : \mul{\tvar})}{x}{\tvar}{(0,0)}}
                        \hypo{\Phi_p \tr \seqi{\Gam_p}{p}{\tightt}{(b_p,b_p)}}
                        \infer2[(\ruleAppPOne)]{\seqi{\Gam_u + \Gam_p}{up}{\tneutral}{(b_p,1+s_u+s_p)}}
                    \end{prooftree} \]
                    where $\tau = \tneutral$, $\Gam = (x : \mul{\tvar}) + \Gam_p$ is tight, $b = b_p$, and $s = 1+ s_u + s_p$. Moreover, $\Gam_p$ is tight. By the \ih over $\Phi_p$, we have that $b_p = 0$. So we can conclude with $b = b_u + b_p = 0$.
                    \item Case $\Phi$ ends with (\ruleAppPTwo). This case is very similar to the case where $\Phi$ ends with rule (\ruleAppPOne).
                \end{itemize}
                \item Case $t = up$, such that $u \in \normal$ and $p \in \neutral$. Then there are three additional cases to consider:
                \begin{itemize}
                    \item Case $\Phi$ ends with (\ruleApp), then it must be of the following form:
                    \[ \begin{prooftree}
                        \hypo{\seqi{\Gam_u}{u}{\M \ta \tau}{(b_u,s_u)}}
                        \hypo{\Phi_p \tr \seqi{\Gam_p}{p}{\M}{(b_p,s_p)}}
                        \infer2[(\ruleApp)]{\seqi{\Gam_u + \Gam_p}{up}{\tau}{(1+b_u+b_p,s_u+s_p)}}
                    \end{prooftree} \]
                    where $\tau = \tau$, $\Gam = \Gam_u + \Gam_p$ is tight, $b = 1 + b_u + b_p$, and $s = s_u + s_p$. By~\cref{lem:tight-spreading}.\ref{lem:tight-spreading:2}, we have that $\M \in \tightt$, which is a contradiction. Therefore, this case does not apply.
                    \item Case $\Phi$ ends with (\ruleAppPOne) or (\ruleAppPTwo). These cases are very similar to the corresponding cases when $t = x p$, such that $p \in \normal$.
                \end{itemize}
                \item Case $t = up$, such that $u \in \neutral$ and $p \in \normal$. Then there are three cases to consider:
                \begin{itemize}
                    \item Case $\Phi$ ends with (\ruleApp), then it must be of the following form:
                    \[ \begin{prooftree}
                        \hypo{\seqi{\Gam_u}{u}{\M \ta \tau}{(b_u,s_u)}}
                        \hypo{\Phi_p \tr \seqi{\Gam_p}{p}{\M}{(b_p,s_p)}}
                        \infer2[(\ruleApp)]{\seqi{\Gam_u + \Gam_p}{up}{\tau}{(1+b_u+b_p,s_u+s_p)}}
                    \end{prooftree} \]
                    where $\tau = \tau$, $\Gam = \Gam_u + \Gam_p$ is tight, $b = 1 + b_u + b_p$, and $s = s_u + s_p$. By~\cref{lem:tight-spreading}.\ref{lem:tight-spreading:2} over $u \in \neutral$, we have that $\M \ta \tau \in \tightt$, which is a contradiction. Therefore, this case does not apply.
                    \item Case $\Phi$ ends with (\ruleAppPOne) or (\ruleAppPTwo). These cases are very similar to corresponding cases when $t = x p$, such that $p \in \normal$, or $t = up$, such that $u \in \normal$ and $p \in \neutral$.
                \end{itemize}
            \end{itemize}
            \item Case $t \in \normal$. Then we can consider the two following additional cases:
            \begin{itemize}
                \item Case $t \in \val$. Then $\Phi$ must end with (\ruleAx), (\ruleLam), (\ruleMany), or (\ruleLamP). With the exception of the case where $\Phi$ ends with rule (\ruleMany), we can conclude $b = 0$ immediately for every other case, by definition. Case $\Phi$ ends with rule (\ruleMany), then $\tau$ is of the form $\M \not\in \tightt$. Therefore, this case does not apply.
                \item Case $t \not\in \val$. Then, $t \in \neutral$, by definition. Therefore, $b = 0$, by \cref{lem:zero-steps-nfs}.\ref{lem:zero-steps-nfs:1}.
            \end{itemize}
        \end{enumerate}
    \end{itemize}
\end{proof}}

\begin{lemma}
    \label{lem:corr-size-counter}
    Let $\Phi \tr \seqi{\Gam}{t}{\tau}{(b,s)}$ be tight. If $b = 0$ then $s = \size{t}$.
\end{lemma}

\maybehide{\begin{proof}
    The proof follows by induction over $\Phi$:
    \begin{itemize}
        \item Case $\Phi$ ends with rule (\ruleAx) or (\ruleLamP). Then $t \in \val$ and $s = 0$. So we can conclude with $\size{t} = 0 = s$.
        \item Case $\Phi$ ends with rule (\ruleLam). Then $\tau$ is of the form $\Gam_u(x) \ta \del \not\in \tightt$, so this case does not apply.
        \item Case $\Phi$ ends with rule (\ruleApp). Then $b > 0$, so this case does not apply.
        \item Case $\Phi$ ends with rule (\ruleMany). Then $\tau$ is of the form $\M \not\in \tightt$, so this case does not apply.
        \item Case $\Phi$ ends with rule (\ruleAppPOne). Then $t = up$ and $\Phi$ must be of the following form:
        \[ \begin{prooftree}
            \hypo{\Phi_u \tr \seqi{\Gam_u}{u}{\nott {\tabs}}{(0,s_u)}}
            \hypo{\Phi_p \tr \seqi{\Gam_p}{p}{\tightt}{(0,s_p)}}
            \infer2[(\ruleAppPOne)]{\seqi{\Gam_u + \Gam_p}{up}{\tneutral}{(0,1+s_u+s_p)}}
        \end{prooftree} \]
        where $\tau = \tneutral$, $\Gam = \Gam_u + \Gam_p$, and $s = 1 + s_u + s_p$. Moreover, $\Gam_u$ and $\Gam_p$ are tight. By the \ih over $\Phi_u$ and $\Phi_p$, we have $s_u = \size{u}$ and $s_p = \size{p}$. So we can conclude with $s = 1 + \size{u} + \size{p} = \size{up}$.
        \item Case $\Phi$ ends with rule (\ruleAppPTwo). This case is very similar to the case where $\Phi$ ends with rule (\ruleAppPOne).
    \end{itemize}
\end{proof}}

\begin{lemma}[{\bf Split for Values}]
    \label{lem:split-values}
    Let $\Phi_v \tr \seqi{\Gam}{v}{\M}{(b,s)}$, such that $\M = \sqcup_{\iI} \M_i$. Then, there exist ($\Phi^i_v \tr \seqi{\Gam_i}{v}{\M_i}{(b_i,s_i)})_{\iI}$, such that $\Gam = +_{\iI} \Gam_i$, $b = +_{\iI} b_i$, and $s = +_{\iI} s_i$.
\end{lemma}

\maybehide{\begin{proof}
    We start by noting that $\Phi_v$ must end with the rule ($\ruleMany$). Therefore, we have $\Gam = +_{\jJ} \Gam_j$, $\M = \mul{\sig_j}_{\jJ}$, $b = +_{\jJ} b_j$, $s = +_{\jJ} s_j$, and $(\Phi^j_v \tr \seqi{\Gam_j}{v}{\sig_j}{(b_j,s_j)})_{\jJ}$, for some $J$. Let $\M_i = \mul{\sig_k}_{\kK_i}$, for each $\iI$, such that $J = +_{\iI} K_i$. Then, by using rule ($\ruleMany$), we can build $\Phi^i_v \tr \seqi{\Gam_i}{v}{\M_i}{(b_i, s_i)}$, for each $\iI$, such that $\Gam_i = +_{\kK_i} \Gam_k$, $b_i = +_{\kK_i} b_k$, and $s_i = +_{\kK_i} s_k$. So we can conclude with $\Gam = +_{\jJ} \Gam_j = +_{\iI} (+_{\kK_i} \Gam_k) = +_{\iI} \Gam_i$, $b = +_{\jJ} b_j = +_{\iI} (+_{\kK_i} b_k) = +_{\iI} b_i$, and $s = +_{\jJ} s_j = +_{\iI} (+_{\kK_i} s_k) = +_{\iI} s_i$.
\end{proof}}

\subsubsection{Completeness (Auxiliary Lemmas)}

\begin{lemma}[{\bf Tight Spreading}]
    \label{lem:tight-spreading}
    Let $\Phi \tr \seqi{\Gam}{t}{\tau}{(b,s)}$, such that $\Gam$ is tight:
    \begin{enumerate}
        \item \label{lem:tight-spreading:1} If $b = 0$ and $\tau$ is not an arrow type or a multi-type, then $\tau \in \tightt$.
        \item \label{lem:tight-spreading:2} If $t \in \neutral$, then $\tau \in \tightt$.
    \end{enumerate}
\end{lemma}

\maybehide{\begin{proof} \mbox{}
    \begin{enumerate}
        \item We want to show that, if $b = 0$ and $\tau$ is not an arrow type or a multiset type, then $\tau \in \tightt$. The proof follows by induction over $\Phi$:
        \begin{itemize}
            \item Case $\Phi$ ends with rule ($\ruleAx$), then it is of the following form:
            \[ \begin{prooftree}
                \infer0[(\ruleAx)]{\seqi{x : \mul{\sig}}{x}{\sig}{(0,0)}}
            \end{prooftree} \]
            such that $\tau = \sig$, $\Gam = x : \mul{\sig}$, and $s = 0$. If $x : \mul{\sig}$ is tight, then $\sig \in \{\tabs, \tvar\}$. Therefore, we can conclude with $\sig \in \{\tabs, \tvar\} \subset \tightt$.
            \item Case $\Phi$ ends with rule (\ruleLam), then $\tau$ is an arrow type. Therefore, this case does not apply.
            \item Case $\Phi$ ends with rule (\ruleApp), then $b > 0$. Therefore, this case does not apply.
            \item Case $\Phi$ ends with rule (\ruleMany), then $\tau$ is a multiset type. Therefore, this case does not apply.
            \item Case $\Phi$ ends with rule (\ruleLamP), then $\tau = \tabs \in \tightt$. 
            \item Case $\Phi$ ends with rules (\ruleAppPOne) or (\ruleAppPTwo), then $\tau = \tneutral \in \tightt$.
        \end{itemize}
        \item We want to show that, if $t \in \neutral$, then $\tau \in \tightt$. By induction over $t \in \neutral$:
        \begin{itemize}
            \item Case $t = xp$, such that $p \in \normal$. Then we have to consider the following three cases depending on the last rule in $\Phi$:
            \begin{itemize}
                \item Case $\Phi$ ends with rule (\ruleApp), then it must be of the following form:
                \[ \begin{prooftree}
                    \hypo{\seqi{x : \mul{\M \ta \tau}}{x}{\M \ta \del}{(0,0)}}
                    \hypo{\Phi_p \tr \seqi{\Gam_p}{p}{\M}{(b_p,s_p)}}
                    \infer2[(\ruleApp)]{\seqi{(x : \mul{\M \ta \tau}) + \Gam_p}{xp}{\del}{(1+b_p,s_p)}}
                \end{prooftree} \]
                where $\Gam = (x : \mul{\M \ta \del}) + \Gam_p$ is tight, $b = 1+b_p$, and $s = s_p$. But, $\mul{\M \ta \del}$ is not tight, since $\M \ta \del \not\in \tightt$. Therefore, this case does apply.
                \item Case $\Phi$ ends with rule (\ruleAppPOne) or (\ruleAppPTwo). Then $\tau = \tneutral \in \tightt$, so we can conclude immediately.
            \end{itemize}
            \item Case $t = up$, such that $u \in \normal$ and $p \in \neutral$. Then we have to consider the following three cases depending on the last rule in $\Phi$:
            \begin{itemize}
                \item Case $\Phi$ ends with rule (\ruleApp), then it must be of the following form:
                \[ \begin{prooftree}
                    \hypo{\Phi_u \tr \seqi{\Gam_u}{u}{\M \ta \tau}{(b_u, s_u)}}
                    \hypo{\Phi_p \tr \seqi{\Gam_p}{p}{\M}{(b_p, s_p)}}
                    \infer2[(\ruleApp)]{\seqi{\Gam_u + \Gam_p}{up}{\tau}{(1+b_u+b_p, s_u+s_p)}}
                \end{prooftree} \]
                where $\Gam = \Gam_u + \Gam_p$ is tight, $b = 1 + b_u + b_p$, and $s = s_u + s_p$. Moreover, $\Gam_p$ is tight. By the \ih over $\Phi_p$, we have that $\M \in \tightt$, which is a contradiction. Therefore, this case does not apply.
                \item Case $\Phi$ ends with rule (\ruleAppPOne) or (\ruleAppPTwo). Then $\tau = \tneutral \in \tightt$, so we can conclude immediately.
            \end{itemize}
            \item Case $t = up$, such that $u \in \neutral$ and $p \in \normal$. Then we have to consider the following three cases depending on the last rule in $\Phi$:
            \begin{itemize}
                \item Case $\Phi$ ends with rule (\ruleApp), then it must be of the following form:
                \[ \begin{prooftree}
                    \hypo{\Phi_u \tr \seqi{\Gam_u}{u}{\M \ta \tau}{(b_u, s_u)}}n
                    \hypo{\Phi_p \tr \seqi{\Gam_p}{p}{\M}{(b_p, s_p)}}
                    \infer2[(\ruleApp)]{\seqi{\Gam_u + \Gam_p}{up}{\tau}{(1+b_u+b_p, s_u+s_p)}}
                \end{prooftree} \]
                where $\Gam = \Gam_u + \Gam_p$ is tight, $b = 1 + b_u + b_p$, and $s = s_u + s_p$. Moreover, $\Gam_p$ is tight. By the \ih over $\Phi_p$, we have that $\M \in \tightt$, which is a contradiction. Therefore, this case does not apply.
                \item Case $\Phi$ ends with rule (\ruleAppPOne) or (\ruleAppPTwo). Then $\tau = \tneutral \in \tightt$, so we can conclude immediately.
            \end{itemize}
        \end{itemize}
    \end{enumerate}
\end{proof}}

\begin{lemma}[{\bf Typability of Normal Forms}]
    \label{lem:typ-nfs}
    If $t \in \normal$, then there exists a tight derivation $\Phi \tr \seqi{\Gam}{t}{\tau}{(b,s)}$, such that $s = \size{t}$.
\end{lemma}

\maybehide{To show this proposition we are going to need to split the original statement into the two following ones:
\begin{enumerate}
    \item \label{prop:typ-nfs:1} If $t \in \neutral$, then there exists a tight derivation $\Phi \tr \seqi{\Gam}{t}{\tneutral}{(b,s)}$, such that $s = \size{t}$.
    \item \label{prop:typ-nfs:2} If $t \in \normal$, then there exists a tight derivation $\Phi \tr \seqi{\Gam}{t}{\tightt}{(b,s)}$, such that $s = \size{t}$.
\end{enumerate}
The proof follows by simultaneous induction over both these statements:
\begin{enumerate}
    \item Let $t \in \neutral$. We want to show that there exists a tight derivation $\Phi \tr \seqi{\Gam}{t}{\tneutral}{(b,s)}$:
    \begin{itemize}
        \item Case $t = up \in x \ \normal$. Then $u = x$ and $p \in \normal$. Therefore, there exists a tight derivation $\Phi_p \tr \seqi{\Gam_p}{p}{\tightt}{(b_p,s_p)}$, by the \ih (\cref{lem:typ-nfs}.\ref{prop:typ-nfs:2}), such that $\size{p} = s_p$. Thus, we can build $\Phi$ as follows:
        \[ \begin{prooftree}
            \infer0[(\ruleAx)]{\seqi{x : \mul{\tvar}}{x}{\tvar}{(0,0)}}
            \hypo{\Phi_p \tr \seqi{\Gam_p}{p}{\tightt}{(b_p,s_p)}}
            \infer2[(\ruleAppPOne)]{\seqi{ x : \mul{\tvar} + \Gam_p}{x p}{\tneutral}{(b_p,1+s_p)}}
        \end{prooftree} \]
        And we can conclude with $\Gam = x : \mul{\tvar} + \Gam_p$, $b = b_p$, and $s = 1+s_p = 1 + \size{x} + \size{p} = \size{xp}$.
        \item Case $t = up \in \normal \ \neutral$. Then $u \in \normal$ and $p \in \neutral$. Therefore, there exists a tight derivation $\Phi_u \tr \seqi{\Gam_u}{u}{\tightt}{(b_u,s_u)}$, such that $\size{u} = s_u$, by the \ih (\cref{lem:typ-nfs}.\ref{prop:typ-nfs:2}), and there exists a tight derivation $\Phi_p \tr \seqi{\Gam_p}{p}{\tneutral}{(b_p,s_p)}$, such that $\size{p} = s_p$ by the \ih (\cref{lem:typ-nfs}.\ref{prop:typ-nfs:1}). Thus, we can build $\Phi$ as follows:
        \[ \begin{prooftree}
            \hypo{\Phi_u \tr \seqi{\Gam_u}{u}{\tightt}{(b_u,s_u)}}
            \hypo{\Phi_p \tr \seqi{\Gam_p}{p}{\tneutral}{(b_p,s_p)}}
            \infer2[(\ruleAppPTwo)]{\seqi{\Gam_u + \Gam_p}{up}{\tneutral}{(b_u+b_p,1+s_u+s_p)}}
        \end{prooftree} \]
        And we can conclude with $\Gam = \Gam_u + \Gam_p$, $b = b_u+b_p$, and $s = 1+s_u+s_p = 1 + \size{u} + \size{p} = \size{up}$.
        \item Case $t = up \in \neutral \ \normal$. Then $u \in \neutral$ and $p \in \normal$. Therefore, there exists a tight derivation $\Phi_u \tr \seqi{\Gam_u}{u}{\tneutral}{(b_u,s_u)}$, such that $\size{u} = s_u$, by the \ih (\cref{lem:typ-nfs}.\ref{prop:typ-nfs:1}), and there exists a tight derivation $\Phi_p \tr \seqi{\Gam_p}{p}{\tightt}{(b_p,s_p)}$, such that $\size{p} = s_p$, by the \ih (\cref{lem:typ-nfs}.\ref{prop:typ-nfs:2}). Thus, we can build $\Phi$ as follows:
        \[ \begin{prooftree}
            \hypo{\Phi_u \tr \seqi{\Gam_u}{u}{\tneutral}{(b_u,s_u)}}
            \hypo{\Phi_p \tr \seqi{\Gam_p}{p}{\tightt}{(b_p,s_p)}}
            \infer2[(\ruleAppPOne)]{\seqi{\Gam_u + \Gam_p}{up}{\tneutral}{(b_u+b_p,1+s_u+s_p)}}
        \end{prooftree} \]
        And we can conclude with $\Gam = \Gam_u + \Gam_p$, $b = b_u+b_p$, and $s = 1 + s_u + s_p = 1 + \size{u} + \size{p} = \size{up}$.
    \end{itemize}
    \item Case $t \in \normal$. We want to show that there exists a tight derivation $\Phi \tr \seqi{\Gam}{t}{\tightt}{(b,s)}$:
    \begin{itemize}
        \item Case $t = x$. Then we can build $\Phi$ as follows:
        \[ \begin{prooftree}
            \infer0[(\ruleAx)]{\seqi{x : \mul{\sig}}{x}{\sig}{(0,0)}}
        \end{prooftree} \]
        by picking $\sig \in \{\tabs, \tvar\}$. And we can conclude with $\Gam = \eset$, $b = 0$, and $s = 0 = \size{x}$.
        \item Case $t = \lam x.u$. Then we can build $\Phi$ as follows:
        \[ \begin{prooftree}
            \infer0[(\ruleLamP)]{\seqi{}{\lam x.u}{\tabs}{(0,0)}}
        \end{prooftree} \]
        And we can conclude with $\Gam = \eset$, $b = 0$, and $s = 0 = \size{\lam x.u}$.
        \item The remaining cases are for when $t \in \neutral$, so they are subsumed by previous cases.
    \end{itemize}
\end{enumerate} 
}

\begin{lemma}[{\bf Merge for Values}]
    \label{lem:merge-values}
    Let $(\Phi^i_v \tr \seqi{\Gam_i}{v}{\M_i}{(b_i,s_i)})_{\iI}$. Then, there exists $\Phi_v \tr \seqi{\Gam}{v}{\M}{(b,s)}$, such that $\Gam = +_{\iI} \Gam_i$, $\M = +_{\iI} \M_i$, $b = +_{\iI} b_i$, and $s = +_{\iI}$.
\end{lemma}

\maybehide{\begin{proof}
    We start by noting that each $\Phi^i_v$ must end with the rule ($\ruleMany$). Therefore, for each $\iI$, we have $\Gam_i = +_{\kK_i} \Gam_k$, $\M_i = \mul{\sig_k}_{\kK_i}$, such that $b_i = +_{\kK_i} b_k$ and $s_i = +_{\kK_i} s_k$, and the following derivations $(\Phi^k_v \tr \seqi{\Gam_k}{v}{\sig_k}{(b_k,s_k)})_{\kK_i}$. Let $J = +_{\iI} K_i$ and $\M = \mul{\sig_j}_{\jJ} = \mul{\sig_k}_{\kK_i, \iI}$. We can use rule ($\ruleMany$) to build $\Phi_v \tr \seqi{\Gam}{v}{\M}{(+_{\jJ} b_j, +_{\jJ} s_j)}$. So we can conclude with $\Gam = +_{\jJ} \Gam_j = +_{\iI} (+_{\kK_i} \Gam_k) = +_{\iI} \Gam_i$, $b = +_{\jJ} b_j = +_{\iI} (+_{\kK_i} b_k) = +_{\iI} b_i$, and $s = +_{\jJ} s_j = +_{\iI} (+_{\kK_i} s_k) = +_{\iI} s_i$.
\end{proof}}

\subsubsection{Soundness and Completeness (Main Results)}

\begin{lemma}[{\bf Substitution and Anti-Substitution}]
    \label{lem:subsantisubs}
    \begin{enumerate} \mbox{}
        \item \label{lem:subs} Let $\Phi_t \tr \seqi{\Gam_t; x : \M}{t}{\tau}{(b_t,s_t)}$ and $\Phi_v \tr \seqi{\Gam_v}{v}{\M}{(b_v,s_v)}$, then there exists $\Phi_{t \subs{x}{v}} \tr \seqi{\Gam_t + \Gam_v}{t \subs{x}{v}}{\tau}{(b_t+b_v,s_t+s_v)}$.
        \item \label{lem:antisubs} Let $\Phi_{t \subs{x}{v}} \tr \seqi{\Gam_{t \subs{x}{v}}}{t \subs{x}{v}}{\tau}{(b,s)}$. Then, there exist $\Phi_t \tr \seqi{\Gam_t; x : \M}{t}{\tau}{(b_t,s_t)}$ and $\Phi_v \tr \seqi{\Gam_v}{v}{\M}{(b_v,s_v)}$, such that $\Gam_{t \subs{x}{v}} = \Gam_t + \Gam_v$, $b = b_t + b_v$, and $s = s_t + s_v$.
    \end{enumerate}
\end{lemma}

\maybehide{\begin{proof} \mbox{}
    \begin{enumerate}
        \item %\begin{proof}
    The proof follows by induction over $\Phi_t$:
    \begin{itemize}
        \item Case $\Phi_t$ ends with rule (\ruleAx). Then $t$ must be a variable and we need to consider two cases:
        \begin{itemize}
            \item Assume $t = y = x$. Then $\Gam_t = \eset$, $\tau = \M$, $t \subs{x}{v} = v$, $b_t = 0$, and $s_t = 0$. So we can take $\Phi_{t \subs{x}{v}} = \Phi_v$ and conclude with $\Gam_t + \Gam_v = \Gam_v$, $b_t + b_v = b_v$, and $s_t + s_v = s_v$.
            \item Assume $t = y \not= x$. Then $\M = \emul$, $\Gam_v = \eset$, $t \subs{x}{v} = t$, $b_v = 0$, and $s_v = 0$. So we can take $\Phi_{t \subs{x}{v}} = \Phi_t$ and conclude with $\Gam_t + \Gam_v = \Gam_t$, $b_t + b_v = b_t$, and $s_t + s_v = s_t$.
        \end{itemize}
        \item Case $\Phi_t$ ends with rule (\ruleLam). Then $t$ must be of the form $\lam y.u$ and $\Phi_t$ must be of the following form (by $\alpha$-conversion):
        \[ \begin{prooftree}
            \hypo{\Phi_u \tr \seqi{\Gam; x : \M}{u}{\tau'}{(b_t,s_t)}}
            \infer1[(\ruleLam)]{\seqi{(\Gam \sm y); x : \M}{\lam y.u}{\Gam(y) \ta \tau'}{(b_t, s_t)}}
        \end{prooftree} \]
        where $\tau = \Gam(y) \ta \tau'$ and $\Gam_t = (\Gam \sm y)$. By the \ih, we have the following derivation $\Phi_{u \subs{x}{v}} \tr \seqi{\Gam + \Gam_v}{u \subs{x}{v}}{\tau}{(b_t + b_v, s_t + s_v)}$. Therefore, we can construct $\Phi_{t \subs{x}{v}}$ as follows:
        \[ \begin{prooftree}
            \hypo{\Phi_{u \subs{x}{v}} \tr \seqi{\Gam + \Gam_v}{u \subs{x}{v}}{\tau'}{(b_t + b_v, s_t + s_v)}}
            \infer1[(\ruleLam)]{\seqi{(\Gam + \Gam_v) \sm y}{(\lam y.u) \subs{x}{v}}{\Gam(y) \ta \tau'}{(b_t + b_v, s_t + s_v)}}
        \end{prooftree} \]
        And we can conclude with $(\Gam + \Gam_v) \sm y = (\Gam \sm y) + \Gam_v = \Gam_t + \Gam_v$, by $\alpha$-conversion.
        \item Case $\Phi_t$ ends with rule ($\ruleApp$). Then $t$ must be of the form $up$ and $\Phi_t$ must be of the following form:
        \[ \begin{prooftree}
            \hypo{\Phi_u \tr \seqi{\Gam; x : \M_1}{u}{\M' \ta \tau}{(b_u, s_u)}}
            \hypo{\Phi_p \tr \seqi{\Del; x : \M_2}{p}{\M'}{(b_p,s_p)}}
            \infer2[(\ruleApp)]{\seqi{(\Gam + \Del); x : \M_1 \sqcup \M_2}{up}{\tau}{(1+b_u+b_p, s_u+s_p)}}
        \end{prooftree} \]
        where $\Gam_t = (\Gam + \Del)$, $\M = \M_1 \sqcup \M_2$, $b_t = 1 + b_u + b_p$, and $s_t = s_u + s_p$. By~\cref{lem:split-values}, we know there exist the following derivations $(\Phi^i_v \tr \seqi{\Gam^i_v}{v}{\M_i}{(b_i,s_i)})_{i \in \{1,2\}}$, such that $\Gam_v = \Gam^1_v + \Gam^2_v$, $b_v = b_1 + b_2$, and $s_v = s_1 + s_2$. By the \ih, we know there exist $\Phi_{u \subs{x}{v}} \tr \seqi{\Gam + \Gam^1_v}{u \subs{x}{v}}{\M' \ta \tau}{(b_u+b_1, s_u+s_1)}$ and $\Phi_{p \subs{x}{v}} \tr \seqi{\Del + \Gam^2_v}{p \subs{x}{v}}{\M'}{(b_p + b_2, s_p + s_2)}$. So we can construct $\Phi_{t \subs{x}{v}}$ as follows:
        \[ \begin{prooftree}
            \hypo{\Phi_{u \subs{x}{v}} \tr \seqi{\Gam + \Gam^1_v}{u \subs{x}{v}}{\M' \ta \tau}{(b_u+b_1, s_u+s_1)}}
            \hypo{\Phi_{p \subs{x}{v}} \tr \seqi{\Del + \Gam^2_v}{p \subs{x}{v}}{\M'}{(b_p+b_2,s_p+s_2)}}
            \infer2[(\ruleApp)]{\seqi{(\Gam + \Del) + (\Gam^1_v + \Gam^2_v)}{(u p) \subs{x}{v}}{\tau}{(1+b_u+b_p+b_1+b_2, s_u + s_p + s_1 + s_2)}}
        \end{prooftree} \]
        And we can conclude with $\Gam_t + \Gam_v = (\Gam + \Del) + (\Gam^1_v + \Gam^2_v)$, $b_t + b_v = 1 + b_u + b_p + b_1 + b_2$, and $s_t + s_v = s_u + s_p + s_1 + s_2$.
        \item Case $\Phi_t$ ends with rule ($\ruleMany$). Then $t$ must be of the form $w$ and $\Phi$ must be of the following form:
        \[ \begin{prooftree}
            \hypo{(\Phi^i_w \tr \seqi{\Gam_i; x : \M_i}{w}{\sig_i}{(b_i,s_i)})_{\iI}}
            \infer1[(\ruleMany)]{\seqi{+_{\iI} \Gam_i; x : \sqcup_{\iI} \M_i}{w}{\mul{\sig_i}_{\iI}}{(+_{\iI} b_i, +_{\iI} s_i)}}
        \end{prooftree} \]
        where $\tau = \mul{\sig_i}_{\iI}$, $\Gam_t = +_{\iI} \Gam_i$, $b_t = +_{\iI} b_i$, and $s_t = +_{\iI} s_i$. By~\cref{lem:split-values}, we have the following derivations $(\Phi^i_v \tr \seqi{\Gam^i_v}{v}{\M_i}{(b^i_v, s^i_v)})_{\iI}$, such that $\Gam_v = +_{\iI} \Gam^i_v$, $b_v = +_{\iI} b^i_v$, and $s_v = +_{\iI} s^i_v$. By the \ih over each $\Phi^i_w$, we have $(\Phi^i_{w \subs{x}{v}} \tr \seqi{\Gam_i + \Gam^i_v}{w \subs{x}{v}}{\sig_i}{(b_i + b^i_v, s_i + s^i_v)})_{\iI}$. Therefore, we can construct $\Phi_{t \subs{x}{v}}$ as follows:
        \[ \begin{prooftree}
            \hypo{(\Phi^i_{w \subs{x}{v}} \tr \seqi{\Gam_i + \Gam^i_v}{w \subs{x}{v}}{\sig_i}{(b_i + b^i_v, s_i + s^i_v)})_{\iI}}
            \infer1[(\ruleMany)]{\seqi{+_{\iI} (\Gam_i + \Gam^i_v)}{w \subs{x}{v}}{\mul{\sig_i}_{\iI}}{(+_{\iI} (b_i + b^i_v), +_{\iI} (s_i + s^i_v))}}
        \end{prooftree} \]
        And we can conclude with $\Gam_t + \Gam_v = +_{\iI} \Gam_i +_{\iI} \Gam^i_v = +_{\iI} (\Gam_i + \Gam^i_v)$, $b_t + b_v = +_{\iI} b_i +_{\iI} b^i_v = +_{\iI} (b_i + b^i_v)$, and $s_t + s_v = +_{\iI} s_i +_{\iI} s^i_v = +_{\iI} (s_i + s^i_v)$.
        \item Case $\Phi_t$ ends with rule (\ruleLamP). Then $t$ must be of the form $\lam y.u$, $\Gam_t = \eset$, $\tau = \tabs$, $\M = \emul$, $\Gam_v = \eset$, $t \subs{x}{v} = \lam y.(u \subs{x}{v}) = (\lam y.u) \subs{x}{v}$, $b_t = b_v = 0$, and $s_t = s_v = 0$. So we can construct $\Phi_{t \subs{x}{v}}$ as follows:
        \[ \begin{prooftree}
            \infer0[(\ruleLamP)]{\seqi{}{(\lam y.u) \subs{x}{v}}{\tabs}{(0,0)}}
        \end{prooftree} \]
        And conclude with $\Gam_t + \Gam_v = \eset$, $b_t + b_v = 0$, and $s_t + s_v = 0$.
        \item Case $\Phi_t$ ends with rule (\ruleAppPOne) or (\ruleAppPTwo), the proof is very similar to when $\Phi_t$ ends with rule (\ruleApp).
    \end{itemize}
%\end{proof}

        \item %\begin{proof}
    The proof follows by induction over $t$:
    \begin{itemize}
        \item Case $t = y$. Then we have to consider two cases:
        \begin{itemize}
            \item Case $t = y \not= x$. Then, $t \subs{x}{v} = y$. Let $\Gam_v = \eset$, $\M = \emul$, $b_v = 0$, and $s_v = 0$. Then, $\Phi_v$ is derivable using rule ($\ruleMany$). We also take $\Phi_t = \Phi_{t \subs{x}{v}}$, so that, in particular $\Gam_t = \Gam_{t \subs{x}{v}}$. Then, we conclude with $\Gam_{t \subs{x}{v}} = \Gam_t + \Gam_v = \Gam_t$, $b = b_t + b_v = b_t$, and $s = s_t + s_v = s_t$.
            \item Case $t = y = x$. Then, $t \subs{x}{v} = v$. Let $\Gam_t = \eset$, $b_t = 0$, and $s_t = 0$. Now, we have to consider two cases depending on the last rule used in $\Phi_{t \subs{x}{v}}$: 
            \begin{itemize}
                \item Case $\Phi_{t \subs{x}{v}}$ ends with rule ($\ruleAx$), then $\tau = \sig$. Let $\Gam_v = \Gam_{t \subs{x}{v}}$, $\M = \mul{\sig}$, $b_v = b$, and $s_v = s$. Then, we can build derivation $\Phi_v$ as follows:
                \[ \begin{prooftree}
                    \hypo{\Phi_{t \subs{x}{v}} \tr \seqi{\Gam_{t \subs{x}{v}}}{v}{\sig}{(b,s)}}
                    \infer1[(\ruleMany)]{\seqi{\Gam_{t \subs{x}{v}}}{v}{\mul{\sig}}{(b,s)}}
                \end{prooftree} \]
                Let $\Gam_t = \eset$, $b_t = 0$, and $s_t = 0$. Then, $\Phi_t \tr \seqi{x : \mul{\sig}}{x}{\sig}{(0,0)}$ is given by rule ($\ruleAx$). So we can conclude with $\Gam_{t \subs{x}{v}} = \Gam_v = \Gam_t + \Gam_v$, $b = b_v = b_t + b_v$, and $s = s_v = s_t + s_v$.
                \item Case $\Phi_{t \subs{x}{v}}$ ends with rule ($\ruleMany$), then $\tau = \mul{\sig_i}_{\iI}$, for some $I$. Let $\Gam_t = \eset$, and $\M = \mul{\sig_i}_{\iI}$. Then, we can build $\Phi_t$ as follows:
                \[ \begin{prooftree}
                    \infer0[(\ruleAx)]{(\seqi{x : \mul{\sig_i}}{x}{\sig_i}{(0,0)})_{\iI}}
                    \infer1[(\ruleMany)]{\seqi{x : \mul{\sig_i}_{\iI}}{x}{\mul{\sig_i}_{\iI}}{(0,0)}}
                \end{prooftree} \] 
                Then, we can take $\Phi_v = \Phi_{t \subs{x}{v}}$, so that $\Gam_v = \Gam_{t \subs{x}{v}}$, $b_v = b$, and $s_v = s$. And we can conclude $\Gam_{t \subs{x}{v}} = \Gam_v = \Gam_t + \Gam_v$, $b = b_v = b_t + b_v$, and $s = s_v = s_t + s_v$.
            \end{itemize}
        \end{itemize}
        \item Case $t = \lam y.u$. Then $t \subs{x}{v} = (\lam y.u) \subs{x}{v} = \lam y.(u \subs{x}{v})$ and we have to consider three cases:
        \begin{itemize}
            \item Case $\Phi_{t \subs{x}{v}}$ ends with rule (\ruleLam), then it must be of the following form:
            \[ \begin{prooftree}
                \hypo{\Phi_{u \subs{x}{v}} \tr \seqi{\Gam_{u \subs{x}{v}}; y : \M'}{u \subs{x}{v}}{\tau'}{(b, s)}}
                \infer1[(\ruleLam)]{\seqi{\Gam_{u \subs{x}{v}}}{\lam y.(u \subs{x}{v})}{\M' \ta \tau'}{(b, s)}}
            \end{prooftree} \]
            where $\tau = \M' \ta \tau'$, and $\Gam_{t \subs{x}{v}} = \Gam_{u \subs{x}{v}}$. By the \ih, we have the following derivations $\Phi_u \tr \seqi{\Gam_u; y: \M'; x : \M}{u}{\del}{(b_u, s_u)}$ and $\Phi_v \tr \seqi{\Gam_v}{v}{\M}{(b_v, s_v)}$, such that $\Gam_{u \subs{x}{v}} = \Gam_u + \Gam_v$, $b = b_u + b_v$, and $s = s_u + s_v$. And we can build $\Phi_{\lam y.u}$ as follows:
            \[ \begin{prooftree}
                \hypo{\Phi_u \tr \seqi{\Gam_u; y : \M'; x : \M}{u}{\tau'}{(b_u, s_u)}}
                \infer1[(\ruleLam)]{\seqi{\Gam_u; x : \M}{\lam y.u}{\M' \ta \tau'}{(b_u, s_u)}}
            \end{prooftree} \]
            So we can pick $\Phi_t = \Phi_{\lam y.u}$, and conclude with $\Gam_{t \subs{x}{v}} = \Gam_{u \subs{x}{v}} = \Gam_u + \Gam_v$, $b = b_u + b_v$, and $s = s_u + s_v$.
            \item Case $\Phi_{t \subs{x}{v}}$ ends with rule (\ruleLamP), then is must be of the following form:
            \[ \begin{prooftree}
                \infer0[(\ruleLamP)]{\seqi{}{\lam y.(u \subs{x}{v})}{\tabs}{(0,0)}}
            \end{prooftree} \]
            where $\tau = \tabs$, $\Gam_{t \subs{x}{v}} = \eset$, $b = 0$, and $s = 0$. Let $\Gam_t = \eset$, $\M = \emul$, $b_t = 0$, and $s_t = 0$. Then, we can build $\Phi_t$ as follows:
            \[ \begin{prooftree}
                \infer0[(\ruleLamP)]{\seqi{}{\lam y.u}{\tabs}{(0,0)}}
            \end{prooftree} \]
            Let $\Gam_v = \eset$, $b_v = 0$, and $s_v = 0$. Then $\Phi_v$ can be constructed by using rule (\ruleMany) with no premises. So we can conclude with $\Gam_{t \subs{x}{v}} = \eset = \Gam_t + \Gam_v$, and $b = 0 = b_t + b_v$, and $s = 0 = s_t + s_v$.
            \item Case $\Phi_{t \subs{x}{v}}$ ends with rule ($\ruleMany$). Then $t \subs{x}{v}$ and $t$ are values, and $\Phi_{t \subs{x}{v}}$ must be of the following form:
            \[ \begin{prooftree}
                \hypo{(\Phi_i \tr \seqi{\Gam_i}{t \subs{x}{v}}{\sig_i}{(b_i,s_i)})_{\iI}}
                \infer1[(\ruleMany)]{\seqi{+_{\iI} \Gam_i}{t \subs{x}{v}}{\mul{\sig_i}_{\iI}}{(+_{\iI} b_i, +_{\iI} s_i)}}
            \end{prooftree} \]
            where $\tau = \mul{\sig_i}_{\iI}$, $\Gam_{t \subs{x}{v}} = +_{\iI} \Gam_i$, $b = +_{\iI} b_i$, and $s = +_{\iI} s_i$. By the \ih over each $\Phi_i$, we have the following derivations $\Phi^i_t \tr \seqi{\Gam^i_t; x : \M_i}{t}{\sig_i}{(b^i_t, s^i_t)}$ and $\Phi^i_v \tr \seqi{\Gam^i_v}{v}{\M_i}{(b^i_v, s^i_v)}$, such that $\Gam_i = \Gam^i_t + \Gam^i_v$, $b_i = b^i_t + b^i_v$, and $s_i = s^i_t + s^i_v$,for each $\iI$. So we can build $\Phi_t$ as follows:
            \[ \begin{prooftree}
                \hypo{(\Phi^i_t \tr \seqi{\Gam^i_t; x : \M_i}{t}{\sig_i}{(b^i_t, s^i_t)})_{\iI}}
                \infer1[(\ruleMany)]{\seqi{+_{\iI} \Gam^i_t; x : \sqcup_{\iI} \M_i}{t}{\mul{\sig_i}_{\iI}}{(+_{\iI} b^i_t, +_{\iI} s^i_t)}}
            \end{prooftree} \]
            such that $\Gam_t = +_{\iI} \Gam^i_t$, $\M = \sqcup_{\iI} \M_i$, $b_t = +_{\iI} b^i_t$, and $s_t = +_{\iI} s^i_t$. By~\cref{lem:merge-values}, we can take the following derivation $\Phi_v \tr \seqi{+_{\iI} \Gam^i_v}{v}{\M}{(+_{\iI} b^i_v, +_{\iI} s^i_v)}$. And we can conclude with $\Gam_{t \subs{x}{v}} = +_{\iI} \Gam_i = +_{\iI} (\Gam^i_t + \Gam^i_v) = +_{\iI} \Gam^i_t +_{\iI} \Gam^i_v = \Gam_t + \Gam_v$, $b = +_{\iI} b_i = +_{\iI} (b^i_t + b^i_v) = +_{\iI} b^i_t +_{\iI} b^i_v = b_t + b_v$, and $s = +_{\iI} s_i = +_{\iI} (s^i_t + s^i_v) = +_{\iI} s^i_t +_{\iI} s^i_v = s_t + s_v$.
        \end{itemize}
        \item Case $t = up$. Then $t \subs{x}{v} = (u \subs{x}{v}) (p \subs{x}{v})$ and we have to consider three cases:
        \begin{itemize}
            \item Case $\Phi_{t \subs{x}{v}}$ ends with ($\ruleApp$), then it must be of the following form:
            \[ \begin{prooftree}
                \hypo{\Phi_{u \subs{x}{v}} \tr \seqi{\Gam_{u \subs{x}{v}}}{u \subs{x}{v}}{\M' \ta \tau}{(b', s')}}
                \hypo{\Phi_{p \subs{x}{v}} \tr \seqi{\Gam_{p \subs{x}{v}}}{p \subs{x}{v}}{\M'}{(b'', s'')}}
                \infer2[(\ruleApp)]{\seqi{\Gam_{u \subs{x}{v}} + \Gam_{p \subs{x}{v}}}{(u \subs{x}{v})(p \subs{x}{v})}{\tau}{(1+b'+b'', s'+s'')}}
            \end{prooftree} \]
            where $\Gam_{t \subs{x}{v}} = \Gam_{u \subs{x}{v}} + \Gam_{p \subs{x}{v}}$, $b = 1+b'+b''$, and $s = s' + s''$. By the \ih over $\Phi_{u \subs{x}{v}}$, we have the following derivations $\Phi_u \tr \seqi{\Gam_u; x : \M_1}{u}{\M' \ta \tau}{(b_u,s_u)}$ and $\Phi^1_v \tr \seqi{\Gam^1_v}{v}{\M_1}{(b^1_v,s^1_v)}$, such that $\Gam_{u \subs{x}{v}} = \Gam_u + \Gam^1_v$, $b' = b_u + b^1_v$, and $s' = s_u + s^1_v$. And by the \ih over $\Phi_{p \subs{x}{v}}$, we have the following derivation $\Phi_{p} \tr \seqi{\Gam_p; x : \M_2}{p}{\M'}{(b_p,s_p)}$ and $\Phi^2_v \tr \seqi{\Gam^2_v}{v}{\M_2}{(b^2_v,s^2_v)}$, such that $\Gam_{p \subs{x}{v}} = \Gam_p + \Gam^2_v$, $b'' = b_p + b^2_v$, and $s'' = s_p + s^2_v$. By~\cref{lem:merge-values}, we can take the following derivation $\Phi_v \tr \seqi{\Gam^1_v + \Gam^2_v}{v}{\M_1 \sqcup \M_2}{(b^1_v+b^2_v, s^1_v + s^2_v)}$, such that $\Gam_v = \Gam^1_v + \Gam^2_v$, $b_v = b^1_v + b^2_v$, and $s_v = s^1_v + s^2_v$. And we can build $\Phi_{up}$ as follows:
            \[ \begin{prooftree}
                \hypo{\Phi_u \tr \seqi{\Gam_u; x : \mul{\sig_i}_{\iI_1}}{u}{\M' \ta \tau}{(b_u,s_u)}}
                \hypo{\Phi_{p} \tr \seqi{\Gam_p; x : \mul{\sig_i}_{\iI_2}}{p}{\M'}{(b_p,s_p)}}
                \infer2[(\ruleApp)]{\seqi{(\Gam_u + \Gam_p); x : \mul{\sig_i}_{\iI}}{up}{\tau}{(1+b_u+b_p, s_u+s_p)}}
            \end{prooftree} \]
            such that $\Gam_t = \Gam_u + \Gam_p$, $b_t = 1+b_u + b_p$, and $s_t = s_u + s_p$. So we can pick $\Phi_t = \Phi_{up}$, and conclude with $\Gam_{t \subs{x}{v}} = \Gam_{u \subs{x}{v}} + \Gam_{p \subs{x}{v}} = \Gam_u + \Gam^1_v + \Gam_p + \Gam^2_v = (\Gam_u + \Gam_p) + (\Gam^1_v + \Gam^2_v) = \Gam_t + \Gam_v$, $b = 1+b'+b'' = 1+b_u + b^1_v + b_p + b^2_v = 1 + (b_u + b_p) + (b^1_v + b^2_v) = b_t + b_v$, and $s = s_u + s^1_v + s_p + s^2_v = (s_u + s_p) + (s^1_v + s^2_v) = s_t + s_v$.
            \item Case $\Phi_{t \subs{x}{v}}$ ends with (\ruleAppPOne) and (\ruleAppPTwo). These cases are very similar to the case where $\Phi_{t \subs{x}{v}}$ ends with rule (\ruleApp).
        \end{itemize}
    \end{itemize}
%\end{proof}
    \end{enumerate}
\end{proof}}

\begin{lemma}[{\bf Split Exact Subject Reduction and Expansion}]
    \label{lem:subjred-subjexp} \mbox{}
    \begin{enumerate} 
        \item \label{lem:subj-red} Let $\Phi_t \tr \seqi{\Gam}{t}{\tau}{(b,s)}$ be tight. If $t \dred t'$, then there exists $\Phi_{t'} \tr \seqi{\Gam}{t'}{\tau}{(b-1,s)}$.
        \item \label{lem:subj-exp} Let $\Phi_{t'} \tr \seqi{\Gam}{t'}{\tau}{(b,s)}$ be tight. If $t \dred t'$, then there exists $\Phi_t \tr \seqi{\Gam}{t}{\tau}{(b+1, s)}$.
    \end{enumerate}
\end{lemma}

\maybehide{\begin{proof} \mbox{}
    \begin{enumerate}
        \item %\begin{proof}
    We will actually prove the following stronger version of the statement, which allows us to reason inductively:

    Let $\Phi_t \tr \seqi{\Gam}{t}{\tau}{(b,s)}$, such that $\Gam$ is tight, and either $\tau$ is tight or $\neg\isvalue{t}$. If $t \dred t'$, then there exists $\Phi_{t'} \tr \seqi{\Gam}{t'}{\tau}{(b-1,s)}$.

    The proof now follows by induction over $\dred$:
    \begin{itemize}
        \item Case $t = (\lam x.u) v \dred u \subs{x}{v} = t'$.Assume that $\Phi_t$ ends with rule (\ruleAppPOne). Then $\lam x.u$ must be assigned type $\nott{\tabs}$, which is not possible by~\cref{lem:notabs-implies-negabs}. Now, assume that $\Phi_t$ ends with rule (\ruleAppPTwo). Then $v$ must be assigned typed $\tneutral$, which is not possible by~\cref{lem:values-not-neutral}. Therefore, $\Phi_t$ must be of the following form:
        \[ \begin{prooftree}
            \hypo{\Phi_u \tr \seqi{\Gam_u; x : \M}{u}{\tau}{(b_u,s_u)}}
            \infer1[(\ruleLam)]{\seqi{\Gam_u}{(\lam x.u)}{\M \ta \tau}{(b_u, s_u)}}
            \hypo{\Phi_{v} \tr \seqi{\Gam_v}{v}{\M}{(b_v,s_v)}}
            \infer2[(\ruleApp)]{\seqi{\Gam_u + \Gam_{v}}{(\lam x.u) v}{\tau}{(1+b_u+b_v, s_u+s_v)}}
        \end{prooftree} \]
        where $\tau \in \tightt$, $\Gam = \Gam_u + \Gam_v$ is tight, $b = 1 + b_u + b_v$, and $s = s_u + s_v$. By~\cref{lem:subsantisubs}.\ref{lem:subs}, we know there exists the following derivation $\Phi_{u \subs{x}{v}} \tr \seqi{\Gam_u + \Gam_v}{u \subs{x}{v}}{\tau}{(b_u+b_v,s_u+s_v)}$. So we can take $\Phi_{t'} = \Phi_{u \subs{x}{v}}$ and conclude with $b - 1 = b_u + b_v$.
        \item Case $t = up \dred u'p = t'$, such that $u \dred u'$. Then $\Phi_t$ must either end with (\ruleApp), (\ruleAppPOne), or (\ruleAppPTwo):
        \begin{itemize}
            \item Case $\Phi_t$ ends with rule (\ruleApp), then it must be of the following form:
            \[ \begin{prooftree}
                \hypo{\Phi_u \tr \seqi{\Gam_u}{u}{\M \ta \tau}{(b_u,s_u)}}
                \hypo{\Phi_p \tr \seqi{\Gam_p}{p}{\M}{(b_p,s_p)}}
                \infer2[(\ruleApp)]{\seqi{\Gam_u + \Gam_p}{up}{\tau}{(1 +b_u+b_p,s_u+s_p)}}
            \end{prooftree} \]
            where $\tau = \tau \in \tightt$, $\Gam = \Gam_u + \Gam_p$ is tight, $b = 1+b_u + b_p$, and $s = s_u + s_p$. Since $u \dred u'$, it is clear that $\neg\isvalue{u}$ holds. Moreover, $\Gam_u$ is necessarily tight. Therefore, by the \ih, there exists $\Phi_{u'} \tr \seqi{\Gam_u}{u'}{\M \ta \tau}{(b_u-1, s_u)}$. Thus, we can build $\Phi_{t'}$ as follows:
            \[ \begin{prooftree}
                \hypo{\Phi_{u'} \tr \seqi{\Gam_u}{u'}{\M \ta \tau}{(b_u-1, s_u)}}
                \hypo{\Phi_p \tr \seqi{\Gam_p}{p}{\M}{(b_p,s_p)}}
                \infer2[(\ruleApp)]{\seqi{\Gam_u + \Gam_p}{u'p}{\tau}{(b_u+b_p,s_u+s_p)}}
            \end{prooftree} \]
            And we can conclude with $b - 1= b_u + b_p$.
            \item Case $\Phi_t$ ends with rule (\ruleAppPOne) or (\ruleAppPTwo), the proof are similar to the one where $\Phi_t$ ends with rule (\ruleApp).
        \end{itemize}
        \item Case $t = up \dred up' = t'$, such that $u \not\dred$ and $p \dred p'$. Then $\Phi_t$ must either end with (\ruleApp), (\ruleAppPOne), or (\ruleAppPTwo):
        \begin{itemize}
            \item Case $\Phi_t$ ends with rule (\ruleApp), then it must be of the following form:
            \[ \begin{prooftree}
                \hypo{\Phi_u \tr \seqi{\Gam_u}{u}{\M \ta \tau}{(b_u,s_u)}}
                \hypo{\Phi_p \tr \seqi{\Gam_p}{p}{\M}{(b_p,s_p)}}
                \infer2[(\ruleApp)]{\seqi{\Gam_u + \Gam_p}{up}{\tau}{(1+b_u+b_p,s_u+s_p)}}
            \end{prooftree} \]
            where $\tau \in \tightt$, $\Gam = \Gam_u + \Gam_p$ is tight, $b = 1 + b_u + b_p$, and $s = s_u + s_p$. Since $p \dred p'$, it is clear that $\neg\isvalue{p}$. Moreover, $\Gam_p$ is necessarily tight. Therefore, by the \ih, we know there exists the following derivation $\Phi_{p'} \tr \seqi{\Gam_p}{p'}{\M}{(b_p-1, s_p)}$. Thus, we can build $\Phi_{t'}$ as follows:
            \[ \begin{prooftree}
                \hypo{\Phi_u \tr \seqi{\Gam_u}{u}{\M \ta \tau}{(b_u, s_u)}}
                \hypo{\Phi_{p'} \tr \seqi{\Gam_p}{p'}{\M}{(b_p-1,s_p)}}
                \infer2[(\ruleApp)]{\seqi{\Gam_u + \Gam_p}{up'}{\tau}{(b_u+b_p,s_u+s_p)}}
            \end{prooftree} \]
            And we can conclude with $b - 1 = b_u + b_p$.
            \item Case $\Phi_t$ ends with rule (\ruleAppPOne) or (\ruleAppPTwo), the proofs are similar to the ones where $\Phi_t$ ends with rule (\ruleApp).
        \end{itemize}
    \end{itemize}
%\end{proof}

        \item %\begin{proof}
    Just like for~\cref{lem:subjred-subjexp}.\ref{lem:subj-red}, we will actually prove the following stronger version of the statement, which allows us to reason inductively:

    Let $\Phi_{t'} \tr \seqi{\Gam}{t'}{\tau}{(b,s)}$, such that $\Gam$ is tight, and either ($\tau \in \tightt$ or $\neg\isvalue{t}$). If $t \dred t'$, then there exists $\Phi_t \tr \seqi{\Gam}{t}{\tau}{(b+1,s)}$.
    
    The proof now follows by induction over $\dred$:
    \begin{itemize}
        \item Case $t = (\lam x.u) v \dred u \subs{x}{v} = t'$. Then $\Phi_{t'} \tr \seqi{\Gam}{u \subs{x}{v}}{\tau}{(b,s)}$ and, by~\cref{lem:subsantisubs}.\ref{lem:antisubs}, there exist the following derivations $\Phi_u \tr \seqi{\Gam_u; x : \M}{u}{\tau}{(b_u, s_u)}$ and $\Phi_v \tr \seqi{\Gam_v}{v}{\M}{(b_v,s_v)}$, such that $\tau \in \tightt$, $\Gam = \Gam_u + \Gam_v$ is tight, $b = b_u + b_v$, and $s = s_u + s_v$. So we can build $\Phi_t$ as follows:
        \[ \begin{prooftree}
            \hypo{\Phi_u \tr \seqi{\Gam_u; x : \M}{u}{\tau}{(b_u, s_u)}}
            \infer1[(\ruleLam)]{\seqi{\Gam_u}{\lam x.u}{\M \ta \tau}{(b_u,s_u)}}
            \hypo{\Phi_v \tr \seqi{\Gam_v}{v}{\M}{(b_v,s_v)}}
            \infer2[(\ruleApp)]{\seqi{\Gam_u + \Gam_v}{(\lam x.u)v}{\tau}{(1+b_u+b_v, s_u+s_v)}}
        \end{prooftree} \]
        And we can conclude with $b + 1 = 1 + b_u + b_v$.
        \item Case $t = up \dred u'p = t'$, such that $u \dred u'$. Then $\Phi_{t'}$ must either end with (\ruleApp), (\ruleAppPOne), or (\ruleAppPTwo):
        \begin{itemize}
            \item Case $\Phi_{t'}$ ends with rule (\ruleApp), then it must be of the following form:
            \[ \begin{prooftree}
                \hypo{\Phi_{u'} \tr \seqi{\Gam_u}{u'}{\M' \ta \tau}{(b_u, s_u)}}
                \hypo{\Phi_p \tr \seqi{\Gam_p}{p}{\M'}{(b_p, s_p)}}
                \infer2[(\ruleApp)]{\seqi{\Gam_u + \Gam_p}{u'p}{\tau}{(1 + b_u + b_p, s_u + s_p)}}
            \end{prooftree} \]
            where $\tau \in \tightt$, $\Gam = \Gam_u + \Gam_p$ it tight, $b = 1 + b_u + b_p$, and $s = s_u + s_p$. Since $u \dred u'$, it is clear that $\neg\isvalue{u}$. Moreover, $\Gam_p$ is tight. Therefore, by the \ih, there exists the following derivation $\Phi_u \tr \seqi{\Gam_u}{u}{\M' \ta \tau}{(b_u + 1, s_u)}$. Thus, we can build $\Phi_{t'}$ as follows:
            \[ \begin{prooftree}
                \hypo{\Phi_u \tr \seqi{\Gam_u}{u}{\M' \ta \tau}{(b_u + 1, s_u)}}
                \hypo{\Phi_p \tr \seqi{\Gam_p}{p}{\M'}{(b_p, s_p)}}
                \infer2[(\ruleApp)]{\seqi{\Gam_u + \Gam_p}{up}{\tau}{(1 + b_u + 1 + b_p, s_u + s_p)}}
            \end{prooftree} \]
            And we can conclude with $b + 1 = (1 + b_u + b_p) + 1 = 1 + b_u + 1 + b_p$.
            \item Case $\Phi_{t'}$ ends with rule (\ruleAppPOne) or (\ruleAppPTwo), the proofs are similar to the one where $\Phi_{t'}$ ends with rule (\ruleApp).
        \end{itemize}
        \item Case $t = up \dred up' = t'$, such that $p \dred p'$. Then $\Phi_{t'}$ must either ends with (\ruleApp), (\ruleAppPOne), or (\ruleAppPTwo):
        \begin{itemize}
            \item Case $\Phi_{t'}$ ends with rule ($\ruleApp$), then it must be of the following form:
            \[ \begin{prooftree}
                \hypo{\Phi_u \tr \seqi{\Gam_u}{u}{\M' \ta \tau}{(b_u, s_u)}}
                \hypo{\Phi_{p'} \tr \seqi{\Gam_p}{p'}{\M'}{(b_p, s_p)}}
                \infer2[(\ruleApp)]{\seqi{\Gam_u + \Gam_p}{u p'}{\tau}{(1 + b_u + b_p, s_u + s_p)}}
            \end{prooftree} \]
            where $\tau \in \tightt$, $\Gam = \Gam_u + \Gam_{p'}$ is tight, $b = 1 + b_u + b_p$, $s_t = s_u + s_p$. Since $p \dred p'$, it is clear that $\neg\isvalue{p}$ holds. Moreover, $\Gam_p$ is tight. Therefore, by the \ih, we have the following derivation $\Phi_p \tr \seqi{\Gam_p}{p}{\M' \ta \tau}{(b_p + 1, s_p)}$. Thus, we can build $\Phi_{t'}$ as follows:
            \[ \begin{prooftree}
                \hypo{\Phi_u \tr \seqi{\Gam}{u}{\M' \ta \tau}{(b_u, s_u)}}
                \hypo{\Phi_p \tr \seqi{\Gam_p}{p}{\M'}{(b_p + 1, s_p)}}
                \infer2[(\ruleApp)]{\seqi{\Gam_u + \Gam_p}{up}{\tau}{(1 + b_u + b_p + 1, s_u + s_p)}}
            \end{prooftree} \]
            And we can conclude with $b + 1 = (1 + b_u + b_p) + 1 = 1 + b_u + b_p + 1$.
            \item Case $\Phi_{t'}$ ends with rule (\ruleAppPOne) or (\ruleAppPTwo), the proofs are similar to the one where $\Phi_{t'}$ ends with rule (\ruleApp).
        \end{itemize}   
    \end{itemize}
%\end{proof}
    \end{enumerate}
\end{proof}}

\begin{theorem}[{\bf Quantitative Soundness and Completeness}]
    \label{thm:soundnesscompleteness}
  \item \label{thm:soundness} If $\Phi \tr \seqi{\Gam}{t}{\tau}{(b,s)}$ is tight, then there exists $u \in \normal$ such that  $t \drred^b u$ with $\size{u} = s$.
  \item \label{thm:completeness} If $t \drred^b u$ with  $u \in \normal$, then there exists a tight type derivation $\Phi_t \tr \seqi{\Gam}{t}{\tau}{(b, \size{u})}$.
\end{theorem}

\maybehide{\begin{proof} \mbox{}
    \begin{enumerate} 
        \item %\begin{proof}
    The proof follows by induction over $b$:
    \begin{itemize}
        \item Case $b = 0$. Then $t \in \normal$, by~\cref{lem:zero-steps-nfs}. And $d = \size{t}$, by~\cref{lem:corr-size-counter}. So we can conclude with $u = t$.
        \item Case $b > 0$. Then $t \not\in \normal$, by~\cref{lem:zero-steps-nfs}. Therefore, there exists $t'$ such that $t \dred t'$, by~\cref{prop:char-nfs}. By\cref{lem:subjred-subjexp}.\ref{lem:subj-red}, there exists $\Phi_{t'} \tr \seqi{\Gam}{t'}{\tau}{(b-1, s)}$. By the \ih, there exists $u \in \normal$, such that $t' \drred^{b-1} u$, such that $d = \size{u}$. So we can conclude with $t \dred t' \drred^{b-1} u$, which means that $t \drred^b u$, as expected.
    \end{itemize}
%\end{proof}
        \item %\begin{proof}
    The proof follows by induction over $b$:
    \begin{itemize}
        \item Case $b = 0$. Then $t = u$, which means that $t \in \normal$. Therefore, we can conclude by~\cref{lem:typ-nfs}.
        \item Case $b > 0$. Then there exists $t'$, such that $t \dred t' \drred^{b-1} u$. By the \ih, there exists a tight derivation $\Phi_{t'} \tr \seqi{\Gam}{t'}{\tau}{(b-1, \size{u})}$. By\cref{lem:subjred-subjexp}.\ref{lem:subj-exp}, there exists a tight derivation $\Phi \tr \seqi{\Gam}{t}{\tau}{(b, \size{u})}$. So, we can conclude.
    \end{itemize}
%\end{proof}
    \end{enumerate}
\end{proof}}
  

  




\subsection{A \texorpdfstring{$\lambda$}{Lambda}-Calculus with Global State}

\subsubsection{General Lemmas}

\propnormalifffinal*

\maybehide{\begin{proof}
    \begin{itemize}
    \item[$\Ra$)] Let $(t, s)$ be \final. We consider two cases:
      \begin{itemize}
            \item Case $(t,s)$ is blocked. We reason by induction on blocked configurations. \begin{itemize}
                \item Case $(t,s) = (\get{l}{x}{u}, s)$, such that $l \not\in \dom{s}$. Then $(t, s) \not\ra$ is straightforward.
                \item Case $(t,s) = (v u, s)$ and $(u,s)$ is blocked.
                  Then by the \ih, we have that $(u,s) \not\ra$. Therefore, $(v u, s) \not\ra$ holds.
            \end{itemize}
          \item Case $t \in \normal$. We reason by induction on
            $\normal$. \begin{itemize}
                \item Case $t=v \in \val$. Then $(v,s) \not\ra$  is straightforward.
                \item Case $t \in \neutral$. Then $t = v u$ and we have to consider two different  cases: \begin{itemize}
                  \item Case  $v= x$ and $u \in \normal$. Then by the \ih, we have $(u,s) \not\ra$. Therefore, $(v u, s) \not\ra$ holds.
                    \item Case $v = (\lam x.p)$ and $u \in \neutral$. Then $u \in \normal$, and by the \ih, we have that $(u,s) \not\ra$. Therefore $(v u,s) \not\ra$ holds.
                \end{itemize}
            \end{itemize}
        \end{itemize}
      \item[$\La$)] Let $t \not \ra$. We reason by
        induction on $t$: \begin{itemize}
            \item Case $t = v$. Then $t \in \normal$. Therefore $(t,s)$ is \final.
            \item Case $t = v u$. Since $(v u, s) \not\ra$, then $(u,s) \not\ra$. By the \ih, we have $(u,s)$ \final. Now, we reason
              by cases: \begin{itemize}
                \item Case $(u, s)$ is blocked. Then, $(v u, s)$ is blocked by definition. 
                \item Case $u \in \normal$. Then we have two cases: \begin{itemize}
                    \item Case $u \in \neutral$. Then $vu \in \normal$. Therefore,  $(t,s)$ is \final.
                    \item Case $u \in \val$ and $v = \lam x.p$. Then $((\lam x.p) u, s) \ra (p \subs{x}{u}, s)$, which yields a contradiction with the hypothesis $t=vu\not\ra$. Thus, this case does not apply.
                \end{itemize}
            \end{itemize}
          \item Case $t = \get{l}{x}{u}$. Since $(\get{l}{x}{u},s) \not\ra$, then $l \not\in \dom{s}$. Therefore, $(\get{l}{x}{u},s)$ is blocked, which implies
$(t,s)$ is \final. 
            \item Case $t = \set{l}{v}{u}$. Then $(\set{l}{v}{u}, s) \ra (u, \upd{l}{v}{s})$, which yields to a contraction with the hypothesis  $t\not\ra$. 
              Therefore, this case does not apply.
        \end{itemize}
    \end{itemize}
\end{proof}
} 

\proptypedunblock*

\maybehide{\begin{proof}
    By induction on $t$: \begin{itemize}
        \item Case $t \in \val $ or $t = \set{l}{v}{t}$. Then the conclusion trivially holds, since clearly $(t,s)$ is not a blocked configuration.
        \item Case $t = \get{l}{x}{t}$. We have two  cases: \begin{itemize}
            \item Case $l \in \dom{s}$. Then $(t,s)$ is clearly unblocked.
            \item Case $l \not\in \dom{s}$. Let $\stype_0 = \conj{(l : \Gam(x))} \splus \stype$. Since $t = \get{l}{x}{u}$, then $\Phi$ must be of the following form:
            \[ \begin{prooftree}
                \hypo{\Phi \tr \seqi{\Gam_u \sm x}{\get{l}{x}{u}}{\comptype{
                    \stype_0}{\ctype}}{(b_u,m_u,d_u)}}
                \hypo{\Phi_s \tr \seqi{\Del}{s}{\stype_0}{(b_s,m_s,d_s)}}
                \infer2[(\ruleConf)]{\seqi{(\Gam_u \sm x) + \Del}{(\get{l}{x}{t}, s)}{\ctype}{(b_u+b_s,1+m_u+m_s,d_u+d_s)}}
              \end{prooftree} \] 
              where $\Gam = \Gam_u \sm x$, $b = b_u+b_s$, $m = 1+m_u+m_s$, and $d = d_u + d_s$. Thus, $l \in \dom{\conj{(l : \Gam_u(x))} \splus \stype}$, and so by
            \cref{lem:states-and-state-types} we have 
          $l \in \dom{s}$, which gives a contradiction with the hypothesis $l \not\in \dom{s}$. Therefore, this case does not apply,
        \end{itemize}
        \item Case $t = v u$. Assume $\Phi_v \tr \seqi{\Gam_v}{v}{\M \ta (\comptype{\stype'}{\ctype})}{(b_v,m_v,d_v)}$ and $\Phi_u \tr \seqi{\Gam_u}{u}{\tcomptype{\stype}{\M}{\stype'}}{(b_u,m_u,d_u)}$. Then $\Phi$ must be of the following form:
        \[ \begin{prooftree}
            \hypo{\Phi_v}
            \hypo{\Phi_u}
            \infer2[(\ruleApp)]{\seqi{\Gam_v + \Gam_u}{v u}{\comptype{\stype}{\ctype}}{(1+b_v+b_u,m_v+m_u,d_v+d_u)}}
            \hypo{\Phi_s \tr \seqi{\Del}{s}{\stype}{(b_s,m_s,d_s)}}
            \infer2[(\ruleConf)]{\seqi{(\Gam_v + \Gam_u) + \Del}{(v u, s)}{\ctype}{(1+b_v + b_u + b_s, m_v + m_u + m_s, d_v + d_u + d_s)}}
        \end{prooftree} \]
        where $\Gam = (\Gam_v + \Gam_u) + \Del$, $b = 1+b_v + b_u + b_s$, $m = m_v + m_u + m_s$, and $d = d_v + d_u + d_s$. Thus, we can build the following derivation for $(u,s)$:
        \[ \begin{prooftree}
            \hypo{\Phi_u \tr \seqi{\Gam_u}{u}{\tcomptype{\stype}{\M}{\stype'}}{(b_u,m_u,d_u)}}
            \hypo{\Phi_s \tr \seqi{\Del}{s}{\stype}{(b_s,m_s,d_s)}}
            \infer2[(\ruleConf)]{\seqi{\Gam + \Del}{(u,s)}{\conftype{\M}{\stype'}}{(b_u+b_s,m_u+m_s,d_u+d_s)}}
        \end{prooftree} \]
        By the \ih, we have that $(u,s)$ is unblocked. Therefore, $(v u, s)$ also unblocked.
    \end{itemize}
\end{proof}} 

\begin{lemma}[Relevance]
    Let $\Phi \tr \seqi{\Gam}{t}{\gtype}{(b,m,d)}$ (resp. $\Phi' \tr \seqi{\Gam}{s}{\stype}{(b',m',d')}$). Then $\dom{\Gam} \subseteq \fv{t}$ (resp. $\dom{\Gam} \subseteq \fv{s}$).
\end{lemma}

\maybehide{\begin{proof}
    The proof following by induction over $\Phi$ (resp. $\Phi'$). Case $\Phi$ (resp. $\Phi'$) ends with rule (\ruleAx), (\ruleAxP), or (\ruleLamP) (resp. rule (\ruleEmp)), then $\Phi$ (resp. $\Phi'$) is clearly relevant. The other cases follow easily from the \ih.
\end{proof}}

\subsubsection{Soundness Lemmas (Auxiliary Lemmas)}

\lemzerocounters*

\maybehide{\begin{proof} \mbox{}
    \begin{enumerate}
        \item \input{proofs/lem-zero-counters}
        \item \input{proofs/lem-zero-size-store}
    \end{enumerate}
\end{proof}} 

\begin{lemma}
    \label{lem:zero-counters-normal}
    Let $\Phi \tr \seqi{\Gam}{t}{\del}{(0,0,d)}$ be tight. If $t \in \normal$, then $\del = \stype \ra \tightt \tim \stype'$ and $\stype =\stype'$.
\end{lemma}

\maybehide{\begin{proof} 
  By induction on $t \in \normal$. We consider two cases:
  \begin{itemize}
    \item Case $t \in \val$. Then such a typing derivation can only end with rule (\ruleAx) followed by rule (\ruleLift) or (\ruleLamP)followed by rule (\ruleLift), in which cases the statement is obvious.
    \item Case $t = vu \in \neutral$. Since the first counter of the derivation is $0$, $\Phi$ can only end with a persistent rule (\ruleAppPOne) or (\ruleAppPTwo). In both cases, we can conclude by applying the \ih to $u \in \normal$ or $u \in \neutral$ and their type derivations, which gives  $\stype = \stype'$.
  \end{itemize}
\end{proof}} 

\lemzeronfs*

\maybehide{\begin{proof}\
  \begin{itemize}
    \item[$\Ra$)] By point (1) of~\cref{lem:zero-counters}.
    \item[$\La$)] By induction on $t$: \begin{itemize}
    \item Case $t \in \val$. There are six cases to consider for $\Phi$:
    \begin{itemize}    
      \item $\Phi$ ends with (\ruleAx). This case does not apply since the resulting type is not a monadic type. %Then $\Phi \tr \seqi{x:\mul{\rdel}}{x}{\rdel}{(0,0,0)}$ and the conclusion holds trivially.
      \item $\Phi$ ends with (\ruleLam). This case does not apply since the resulting type is not a monadic type.
      \item $\Phi$ ends with (\ruleMany). This case does not apply since the resulting type is not a monadic type.
      \item $\Phi$ ends with (\ruleLift). This case does not apply, since $\del = \tcomptype{\stype}{\M}{\stype'}$, but $\M \not\in \tightt$.
      \item $\Phi$ ends with (\ruleAxP). Then $\Phi \tr \seqi{x:\mul{\nott{\tneutral}}}{x}{\tcomptype{\stype}{\nott{\tneutral}}{\stype}}{(0,0,0)}$, with $\stype$ tight, and the conclusion holds trivially.
      \item $\Phi$ ends with (\ruleLamP). Then $\Phi \tr \seqi{}{\lambda x.t}{\tcomptype{\stype}{\vl}{\stype}}{(0,0,0)}$, with $\stype$ tight, and the conclusion holds trivially. 
    \end{itemize}
    \item Case $t = xu$. Then $u \in \normal$, by definition and there are two cases to consider for $\Phi$:
    \begin{itemize}
      \item If $\Phi$ ends with (\ruleApp). Then $\Phi_u \tr \seqi{\Gam_u}{u}{\tcomptype{\stype}{\M}{\stype'}}{(b_u,m_u,d_u)}$, $\Phi_x \tr \seqi{x : \M \ta (\comptype{\stype'}{\ctype})}{x}{\M \ta (\comptype{\stype'}{\ctype})}{(b_x,m_x,d_x)}$, such that $\Gam =  (x:\mul{\M \ta (\comptype{\stype'}{\ctype})}) + \Gam_u$ is tight. Absurd, since $\M \ta (\comptype{\stype'}{\ctype})$ is not tight, therefore this case does not apply.
      \item If $\Phi$ ends with (\ruleAppPOne). Then $\Phi_u \tr \seqi{\Gam_u}{u}{\tcomptype{\stype}{\tightt}{\stype}}{(b_u,m_u,d_u)}$, such that $\Gam = (x: \mul{\tvar})+\Gam_u$ is tight, $b = b_u$, $m =m_u$, $d = d_u+ 1$, and $\stype$ is tight. By the \ih\ on $u$, we have $b_u=m_u=0$, therefore $b = m = 0$.
      \end{itemize}
      \item Case $t = (\lam x.p) u$. Then $u \in \neutral$, by definition and there are two cases to consider for $\Phi$:
      \begin{itemize}
        \item If $\Phi$ ends with (\ruleApp). Then $\Phi_u \tr \seqi{\Gam_u}{u}{\tcomptype{\stype}{\M}{\stype'}}{(b_u,m_u,d_u)}$, $\Phi_{\lam x.p} \tr \seqi{\Gam_{\lam x.p}}{\lam x.p}{\M \ta (\comptype{\stype'}{\ctype})}{(b_p,m_p,d_p)}$, such that $\Gam = \Gam_u + \Gam_{\lambda x.p}$ is tight, $b = 1+b_l+b_u$, $m = m_l+m_u$, $d = d_l+ d_m$. Since $\Gam_u$ is tight and $u\in\neutral$, by~\cref{lem:comp-tight-spreading}, $\M \in \tightt$, which is absurd. Therefore, this case does not apply.
        \item If $\Phi$ ends with (\ruleAppPTwo). Then $\Phi_u \tr \seqi{\Gam_u}{u}{\tcomptype{\stype}{\tneutral}{\stype}}{(b_u,m_u,d_u)}$, such that $\Gam = \Gam_u$ is tight, $b = b_u$, $m=m_u$, $d = d_u+ 1$ and $\stype_f$ is tight. By the \ih\ on $u$, we have $b_u=m_u=0$. Therefore $b = m = 0$.
      \end{itemize}
    \end{itemize}
  \end{itemize}
\end{proof}
}

\begin{lemma}
    \label{lem:states-and-state-types}
    Let $\Phi \tr \seqi{\Del}{s}{\stype}{(b,m,d)}$. If $l \in \dom{\stype}$, then $l \in \dom{s}$.
\end{lemma}
  
\maybehide{\begin{proof}
    We proceed by proving the following stronger version of the statement: 
    
    Let $\Phi_s \tr \seqi{\Del_s}{s}{\stype_s}{(b_s,m_s,d_s)}$. If $l \in \dom{\stype_s}$, then $s \equivstate \upd{l}{v}{q}$, for some value $v$ and store $q$.
    
    The proof follows by induction on $\Phi_s$: 
    \begin{itemize}
        \item Case $\Phi_s$ ends with ($\ruleEmp$). Then the conclusion is vacuously true.
        \item Case $\Phi_s$ ends with ($\ruleUpd$). Then $\Phi_s$ is of the following form: 
        \[ \begin{prooftree}
            \hypo{\Phi_v \tr \seqi{\Gam_v}{v}{\M}{(b_v,m_v,d_v)}}
            \hypo{\Phi_q \tr \seqi{\Del_q}{q}{\stype_q}{(b_q,m_q,d_q)}}
            \infer2[(\ruleUpd)]{\seqi{\Gam_v + \Del_q}{\upd{l'}{v}{q}}{\conj{l' : \M}; \stype_q}{(b_v+b_q,m_v+m_q,d_v+v_q)}}
        \end{prooftree} \]
        where $\Del_s = \Gam_v + \Del_q$, $s = \upd{l'}{v}{q}$, $\stype_s = \conj{l' : \M}; \stype_q$, $b_s = b_v + b_q$, $m_s = m_v + m_q$, and $d_s = d_v + d_q$. Now we consider two  cases: 
        \begin{itemize}
            \item Case $l = l'$. Then we are done.
            \item Case $l \not= l'$. Since we are assuming that $l \in \dom{\stype_s}$, then it must be case that $l \in \dom{\stype_q}$. But, then by the \ih, we have $q \equivstate \upd{l}{w}{q'}$, for some value $w$ and store $q'$. Therefore, $s \equivstate \upd{l'}{v}{\upd{l}{w}{q'}} \equivstate \upd{l}{w}{\upd{l'}{v}{q'}}$.
        \end{itemize}
    \end{itemize}
    The correctness of the original statement now follows easily from the fact that, clearly, if $s \equivstate \upd{l}{v}{q}$, then $l \in \dom{s}$, by Definition~\ref{def:domainS}.
\end{proof}
} 

\begin{lemma}[{\bf Split Lemma}] \mbox{} 
    \label{lem:split-values-stores}
    \begin{enumerate}
        \item {(\bf Values)} \label{lem:com-split-values}  Let $\Phi_v \tr \seqi{\Gam}{v}{\M}{(b,m,d)}$, such that $\M = \sqcup_{\iI} \M_i$. Then, there exist ($\Phi^i_v \tr \seqi{\Gam_i}{v}{\M_i}{(b_i,m_i,d_i)})_{\iI}$, such that $\Gam = +_{\iI} \Gam_i$, $b = +_{\iI} b_i$, $m = +_{\iI} m_i$, and $d = +_{\iI} d_i$.
        \item {\bf (States)} \label{lem:split-state} Let $\Phi_s \tr \seqi{\Gam}{s}{\stype}{(b,m,d)}$, such that $l \in \dom{\stype}$. Then, $s \equivstate \upd{l}{v}{q}$, $\Phi_v \tr \seqi{\Gam_v}{v}{\stype(l)}{(b_v,m_v,d_v)}$ and $\Phi_q \tr \seqi{\Gam_q}{q}{\stype'}{(b_q,m_q,d_q)}$, such that $\Gam = \Gam_v + \Gam_q$, $\stype = \conj{(l : \stype(l))}; \stype'$, $b = b_v+b_q$, $m = m_v+m_q$, and $d = d_v + d_q$.
    \end{enumerate}
\end{lemma}

\maybehide{\begin{proof}
    The proof for values is very similar to the corresponding proof for $\lam_s$, so we are only going to show the split lemma for states.
    The proof follows by induction on the structure of $s$: \begin{itemize}
        \item Case $s = \estate$. Then the statement is vacuously true.
        \item Case $s = \upd{l'}{w}{q'}$. Then $\Phi_s$ is of the form: 
        \[ \begin{prooftree}
            \hypo{\Phi_{w} \tr \seqi{\Gam_{w}}{w}{\M}{(b_w,m_w,d_w)}}
            \hypo{\Phi_{q'} \tr \seqi{\Gam_{q'}}{q'}{\stype_{q'}}{(b_{q'}, m_{q'},d_{q'})}}
            \infer2[(\ruleUpd)]{\seqi{\Gam_{w} + \Gam_{q'}}{\upd{l'}{w}{q'}}{\conj{(l' : \M)}; \stype_{q'}}{(b_w+b_{q'},m_w+m_{q'},d_w+d_{q'})}}
        \end{prooftree} \] where $\Gam = \Gam_{w} + \Gam_{q'}$, $\stype = \conj{(l' : \M)}; \stype_{q'}$, $b = b_w + b_{q'}$, $m = m_w + m_{q'}$, and $d = d_w + d_{q'}$. 
        We  consider two cases: \begin{itemize}
            \item Case $l' = l$. Then we simply take $v = w$ and $q = q'$ and we are done.
            \item Case $l' \not= l$.  Since $l \in \dom{\conj{(l' : \M)}; \stype_{q'}}$ and $l' \not= l$, then $l \in \dom{\stype_{q'}}$. By applying the \ih to $q'$, we have that  $q' \equivstate \upd{l}{w'}{q''}$, $\Phi_{w'} \tr \seqi{\Gam_{w'}}{w'}{\stype_{q'}(l)}{(b_{w'},m_{w'},d_{w'})}$ and $\Phi_{q''} \tr \seqi{\Gam_{q''}}{q''}{\stype_{q''}}{(b_{q''},m_{q''},d_{q''})}$, such that $\Gam_{q'} = \Gam_{w'} + \Gam_{q''}$, $\stype_{q'} = \conj{(l : \stype_{q'}(l))}; \stype_{q''}$, $b_{q'} = b_{w'} + b_{q''}$, $m_{q'} = m_{w'} + m_{q''}$, and $d_{q'} = d_{w'} + d_{q''}$. But $s = \upd{l'}{w}{\upd{l}{w'}{q''}} \equivstate \upd{l}{w'}{\upd{l'}{w}{q''}}$, so we can take $v = w'$, $q = \upd{l'}{w}{q''}$, and consider $\Phi_q$ to be the following derivation:
            \[ \begin{prooftree}
                \hypo{\Phi_{w} \tr \seqi{\Gam_{w}}{w}{\M}{(b_w,m_w,d_w)}}
                \hypo{\Phi_{q''} \tr \seqi{\Gam_{q''}}{q''}{\stype_{q''}}{(b_{q''}, m_{q''}, d_{q''})}}
                \infer2[(\ruleUpd)]{\seqi{\Gam_{w} + \Gam_{q''}}{\upd{l'}{w}{q''}}{\conj{(l' : \M)}; \stype_{q''}}{(b_w+b_{q''}, m_w + m_{q''}, d_w+d_{q''})}}
            \end{prooftree} \] where  $\Gam_q = \Gam_{w} + \Gam_{q''}$ and $\stype_q=\conj{(l' : \M)}; \stype_{q''}$. We can then conclude with the following observations:
            \begin{itemize}
            \item $\Gam_v + \Gam_q = \Gam_{w'} +\Gam_{w} + \Gam_{q''} =
              \Gam_{w} + \Gam_{q'} = \Gam$,
                \item Since $\stype = \conj{(l' : \M)}; \stype_{q'}$ and $l' \not= l$, then $\stype(l) = \stype_{q'}(l)$ and
                \begin{align*}
                    \stype = \conj{(l' : \M)}; \stype_{q'} & = \conj{(l': \M)}; \conj{(l : \stype_{q'}(l))}; \stype_{q''} \\
                    & = \conj{(l : \stype_{q'}(l))}; \stype_{q} \\
                    & = \conj{(l : \stype(l))}; \stype_q
                \end{align*}
              \item $b_v + b_q= b_{w'} + b_{w} + b_{q''}=
                 b_w + b_{q'} = b$, $m_v + m_q= m_{w'} + m_{w} + m_{q''}=
                 m_w + m_{q'} = b$ and
                 $d_v + d_q= d_{w'} + d_{w} + d_{q''}=
                 d_w + d_{q'} = d$.
            \end{itemize} 
        \end{itemize}
    \end{itemize}
\end{proof}
} 

\begin{lemma}
    \label{lem:comp-values-not-neutral}
    Let $\Phi \tr \seqi{\Gam}{t}{\tcomptype{\stype}{\tau}{\stype'}}{(b,m,d)}$. If $t \in \val$, then $\tau \neq \tneutral$.
\end{lemma}

\maybehide{\begin{proof}
    By case analysis on the form of $t \in \val$:
    \begin{itemize}
        \item Case $t = x$. Then we have to consider three cases according to the last rule used in $\Phi$:
        \begin{itemize}
            \item Case $\Phi$ ends with rule (\ruleAx), then $t$ can only be assigned $\sig$. Therefore, this case does not apply.
            \item Case $\Phi$ ends with rule (\ruleMany), then $\tau = \M \neq \tneutral$.
        
            \item Case $\Phi$ ends with rule (\ruleLift). Then $\tau \in \{\tvar, \tabs, \M\}$, which means that $\tau \not= \tneutral$.
        \end{itemize}
        \item Case $t = \lam x.t$. Then we have to consider three cases according to the last rule used in $\Phi$:
        \begin{itemize}
            \item Case $\Phi$ ends with rule (\ruleLam), then $t$ can only be assigned $\sig$. Therefore, this case does not apply.
            \item Case $\Phi$ ends with rule (\ruleMany), then $\tau = \M  \neq \tneutral$.
            \item Case $\Phi$ ends with rule (\ruleLamP), then $\tau = \vl$. Therefore, this case does not apply.
            \item Case $\Phi$ ends with rule (\ruleLift). $\tau \in \{\tabs, \M\}$, which means that $\tau \not= \tneutral$.
        \end{itemize}
    \end{itemize}
\end{proof}} 

\begin{lemma}
    \label{lem:comp-notabs-implies-negabs}
    Let $\Phi \tr \seqi{\Gam}{t}{\tcomptype{\stype}{\tau}{\stype'}}{(b,m,d)}$, such that $\Gam$ is tight. If $\tau \in \nott{\vl}$, then $\neg\isabs{t}$.
\end{lemma}

\maybehide{\begin{proof}
    By induction over $\Phi$:
    \begin{itemize}
        \item Case $\Phi$ ends with rule (\ruleAx), (\ruleApp), (\ruleGet), (\ruleSet), (\ruleAxP) (\ruleAppPOne), or (\ruleAppPTwo), then $\neg\isabs{t}$ holds by definition.
        \item Case $\Phi$ ends with rule (\ruleLam), (\ruleMany), or (\ruleLamP), then $\tau \in \nott{\tabs}$ does not hold. Therefore, these cases do not apply.
    \end{itemize}
\end{proof}} 

\subsubsection{Completeness (Auxiliary Lemmas)}

\begin{lemma}[{\bf Merge for Values}]
    \label{lem:comp-merge-values}
    Let $(\Phi^i_v \tr \seqi{\Gam_i}{v}{\M_i}{(b_i,m_i,d_i)})_{\iI}$. Then, there exists $\Phi_v \tr \seqi{\Gam}{v}{\M}{(b,m,d)}$, such that $\Gam = +_{\iI} \Gam_i$, $\M = +_{\iI} \M_i$, $b = +_{\iI} b_i$, $m = +_{\iI} m_i$, and $d = +_{\iI}$.
\end{lemma}
We omit this proof given its similarity with the proof for system $\syscbv$.

\lemcomtightspreading*

\maybehide{\begin{proof}
  We want to show that, if $t \in \neutral$, then $\tau \in \tightt$, for some $\stype'$. We proceed by induction on the predicate  $t \in \neutral$:
    \begin{itemize}
        \item Case $t = xu$, such that $u \in \normal$. Then we have to consider the following two cases depending on the last rule in $\Phi$:
        \begin{itemize}
            \item Case $\Phi$ ends with rule ($\ruleApp$), then it must be of the following form:
            \[ \begin{prooftree}
                \infer0[(\ruleAx)]{\seqi{x : \mul{\M \ta (\comptype{\stype'}{\ctype})}}{x}{\M \ta (\comptype{\stype'}{\ctype})}{(0,0,0)}}
                \hypo{\Phi_u \tr \seqi{\Gam_u}{u}{\tcomptype{\stype}{\M}{\stype'}}{(b_u,m_u,d_u)}}
                \infer2[(\ruleApp)]{\seqi{(x : \mul{\M \ta (\comptype{\stype'}{\ctype})}) + \Gam_u}{xu}{\comptype{\stype}{\ctype}}{(1+b_u,m_u,d_u)}}
            \end{prooftree} \]
            where $\Gam = (x : \mul{\M \ta (\comptype{\stype'}{\ctype})}) + \Gam_p$ is tight, $b = 1+b_u$, $m = m_u$, and $d = d_u$. But $\M \ta (\comptype{\stype'}{\ctype}) \not\in \tightt$, therefore $\Gam$ is not tight and we have a contraction. Thus, this case does not apply.
            \item Case $\Phi$ ends with rule (\ruleAppPOne), then $\tau = \tneutral \in \tightt$, so we can conclude immediately.
        \end{itemize}
        \item Case $t = (\lambda x.p)u$, such that $u \in \neutral$. Then we have to consider the following two cases depending on the last rule in $\Phi$:
        \begin{itemize}
            \item Case $\Phi$ ends with rule ($\ruleApp$), then it must be of the following form:
            \[ \begin{prooftree}
                \hypo{\seqi{\Gam_p}{\lam x.p}{\M \ta (\comptype{\stype'}{\ctype})}{(b_p,m_p,d_p)}}
                \hypo{\Phi_u \tr \seqi{\Gam_u}{u}{\tcomptype{\stype}{\M}{\stype'}}{(b_u,m_u,d_u)}}
                \infer2[(\ruleApp)]{\seqi{\Gam_p + \Gam_u}{(\lam x.p)u}{\comptype{\stype}{\ctype}}{(1+b_p+b_u,m_p+m_u,d_p+d_u)}}
            \end{prooftree} \]
            where $\Gam = \Gam_u + \Gam_p$ is tight, $b = 1 + b_p + b_u$, $m = m_p + m_u$, and $d = d_p + d_u$. By the \ih on $u$, we have that $\M \in \tightt$, which is a contradiction. Therefore, this case does not apply.
            \item Case $\Phi$ ends with rule (\ruleAppPTwo). Then $\tau = \tneutral \in \tightt$, so we can conclude immediately.
        \end{itemize}
    \end{itemize}
\end{proof}
} 

\typstates*

\maybehide{\begin{proof} \mbox{}
    \begin{enumerate}
        \item \input{proofs/lem-typ-states}
        \item \input{proofs/lem-comp-typ-nfs}
    \end{enumerate}
\end{proof}


} 

\subsubsection{Soundness and Completeness (Main Lemmas)}

\lemcompsubsantisubs*

\maybehide{\begin{proof} \mbox{}
    \begin{enumerate}
        \item %\begin{proof}
    We are going to generalize the original statement by replacing $\del$ with $\gtype$.
    \\ \\
    The proof now follows by induction over the structure of $\Phi_t$:
        \begin{itemize}
            \item Case $\Phi_t$ ends with rule ($\ruleAx$). Then $t$ must be a variable and we must consider two cases:
            \begin{itemize}
                \item Assume $t = y = x$. Then $\Gam_t = \eset$, $\gtype = \M$, $t \subs{x}{v} = v$, $b_t = m_t = d_t = 0$. So we can take $\Phi_{t \subs{x}{v}} = \Phi_v$ and conclude with $\Gam_t + \Gam_v = \Gam_v$, $b_t + b_v = b_v$, $m_t + m_v = m_v$, and $d_t + d_v = d_v$.
                \item Assume $t = y \not= x$. Then $\M = \emul$, $\Gam_v = \eset$, $t \subs{x}{v} = t$, $b_v = 0$, $m_v = 0$, and $d_v = 0$. So we can take $\Phi_{t \subs{x}{v}} = \Phi_t$ and conclude with $\Gam_t + \Gam_v = \Gam_t$, $b_t + b_v = b_t$, $m_t + m_v = m_t$, and $d_t + d_v = d_t$.
            \end{itemize}
            \item Case $\Phi_t$ ends with (\ruleLam). Then $t$ must be of the form $\lam y.u$ and $\Phi_t$ must be of the following form (by $\alpha$-conversion):
            \[ \begin{prooftree}
                \hypo{\Phi_u \tr \seqi{\Gam; x : \M}{u}{\comptype{\stype}{\ctype}}{(b_t,m_t,d_t)}}
                \infer1[(\ruleLam)]{\seqi{(\Gam \sm y); x : \M}{\lam y.u}{\Gam(y) \ta (\comptype{\stype}{\ctype})}{(b_t,m_t,d_t)}}
            \end{prooftree} \]
            where $\Gam_t = (\Gam \sm y)$, and $\gtype = \Gam(y) \ta (\comptype{\stype}{\ctype})$. By the \ih, we have the following derivation $\Phi_{u \subs{x}{v}} \tr \seqi{\Gam + \Gam_v}{u \subs{x}{v}}{\comptype{\stype}{\ctype}}{(b_t+b_v,m_t+m_v,d_t+d_v)}$. Therefore, we can build $\Phi_{t \subs{x}{v}}$ as follows:
            \[ \begin{prooftree}
                \hypo{\Phi_{u \subs{x}{v}} \tr \seqi{\Gam + \Gam_v}{u \subs{x}{v}}{\comptype{\stype}{\ctype}}{(b_t+b_v,m_t+m_v,d_t+d_v)}}
                \infer1[(\ruleLam)]{\seqi{(\Gam + \Gam_v) \sm y}{\lambda y.u \subs{x}{v}}{\Gam(y) \ta (\comptype{\stype}{\ctype})}{(b_t+b_v,m_t+m_v,d_t+d_v)}}
            \end{prooftree} \]
            And we conclude with $(\Gam + \Gam_v) \sm y = (\Gam \sm y) + \Gam_v = \Gam_t + \Gam_v$, by $\alpha$-conversion.
            \item Case $\Phi_t$ ends with ($\ruleApp$). Then $t$ must be of the form $wu$ and $\Phi_t$ must be of following form:
            \[ \begin{prooftree}
                \hypo{\Phi_w \tr \seqi{\Gam; x : \M_1}{w}{\M' \ta (\comptype{\stype'}{\ctype})}{(b_w,m_w,d_w)}}
                \hypo{\Phi_u \tr \seqi{\Del; x : \M_2}{u}{\tcomptype{\stype}{\M'}{\stype'}}{(b_u,m_u,d_u)}}
                \infer2[(\ruleApp)]{\seqi{\Gam + \Del; x : \M_1 \sqcup \M_2}{wu}{\comptype{\stype}{\ctype}}{(1+b_w+b_u,m_w+m_u,d_w+d_u)}}
            \end{prooftree} \]
            such that $\Gam_t = \Gam + \Del$, $\M = \M_1 \sqcup \M_2$, $\gtype = \comptype{\stype}{\ctype}$, $b_t = 1+b_w+b_u$, $m_t = m_w+m_u$, and $d_t = d_w + d_u$. By~\cref{lem:split-values-stores}.\ref{lem:com-split-values}, we know there exist the following derivations $(\Phi^i_v \tr \seqi{\Gam^i_v}{v}{\M_i}{(b_i,m_i,d_i)})_{i \in \{1,2\}}$, such that $\Gam_v = \Gam^1_v + \Gam^2_v$, $b_v = b_1 + b_2$, $m_v = m_1 + m_2$, and $d_v = d_1 + d_2$. By the \ih, we know there exist $\Phi_{w \subs{x}{v}} \tr \seqi{\Gam + \Gam^1_v}{w \subs{x}{v}}{\M' \ta (\comptype{\stype'}{\ctype})}{(b_w+b_1,m_w+m_1,d_w+d_1)}$ and $\Phi_{u \subs{x}{v}} \tr \seqi{\Del + \Gam^2_v}{u \subs{x}{v}}{\tcomptype{\stype}{\M'}{\stype'}}{(b_u+b_2,m_u+m_2,d_u+d_2)}$. %Assume $\Phi_{w \subs{x}{v}} \tr \seqi{\Gam + \Gam^1_v}{w \subs{x}{v}}{\tcomptype{\M'}{\stype'}{\ctype}}{(b_w+b_1,m_w+m_1,d_w+d_1)}$ and $\Phi_{u \subs{x}{v}} \tr \seqi{\Del + \Gam^2_v}{u \subs{x}{v}}{\stype \ta (\comptype{\M'}{\stype'})}{(b_u+b_2,m_u+m_2,d_u+d_2)}$. 
            We can build $\Phi_{t \subs{x}{v}}$ as follows:
            \[ \begin{prooftree}
                \hypo{\Phi_{w \subs{x}{v}}}
                \hypo{\Phi_{u \subs{x}{v}}}
                \infer2[(\ruleApp)]{\seqi{(\Gam + \Del) + (\Gam^1_v + \Gam^2_v)}{(wu) \subs{x}{v}}{\comptype{\stype}{\ctype}}{(1+b_w+b_u+b_1+b_2,m_w+m_u+m_1+m_2,d_w+d_u+d_1+d_2)}}
            \end{prooftree} \]
            And we can conclude with $\Gam_t + \Gam_v = (\Gam + \Del) + (\Gam^1_v + \Gam^2_v)$, $b_t + b_v = 1 + b_w+b_u+b_1+b_2$, $m_t + m_v = m_w+m_u+m_1+m_2$, and $d_t + d_v = d_w+d_u+d_1+d_2$.
            \item Case $\Phi_w$ ends with ($\ruleMany$). Then $t$ must be of the form $w$ and $\Phi_t$ must be of the following form:
            \[ \begin{prooftree}
                \hypo{(\Phi^i_w \tr \seqi{\Gam_i; x : \M_i}{w}{\rdel_i}{(b_i,m_i,d_i)})_{\iI}}
                \infer1[(\ruleMany)]{\seqi{+_{\iI} \Gam_i; x : \sqcup_{\iI} \M_i}{w}{\mul{\rdel_i}_{\iI}}{(+_{\iI}b_i, +_{\iI}m_i, +_{\iI}d_i)}}
            \end{prooftree} \]
            such that $\Gam_t = +_{\iI} \Gam_i$, $\gtype = \mul{\rdel_i}_{\iI}$, $b_t = +_{\iI} b_i$, $m_t = +_{\iI} m_i$, and $d_t = +_{\iI} d_i$. By~\cref{lem:split-values-stores}.\ref{lem:com-split-values}, $(\Phi^i_v \tr \seqi{\Gam^i_v}{v}{\M_i}{(b^i_v,m^i_v,d^i_v)})_{\iI}$, such that $\Gam_v = +_{\iI} \Gam^i_v$, $b_v = +_{\iI} b^i_v$, $m_v = +_{\iI} m^i_v$, and $d_v = +_{\iI} d^i_v$. By the \ih over each $\Phi^i_v$, we have $(\Phi^i_{w \subs{x}{v}} \tr \seqi{\Gam_i + \Gam^i_v}{w \subs{x}{v}}{\rdel_i}{(b_i+b^i_v,m_i+m^i_v,d_i+d^i_v)})_{\iI}$. Therefore, we can build $\Phi_{t \subs{x}{v}}$ as follows:
            \[ \begin{prooftree}
                \hypo{(\Phi^i_{w \subs{x}{v}} \tr \seqi{\Gam_i + \Gam^i_v}{w \subs{x}{v}}{\rdel_i}{(b_i+b^i_v,m_i+m^i_v,d_i+d^i_v)})_{\iI}}
                \infer1[(\ruleMany)]{\seqi{+_{\iI} (\Gam^i_v + \Gam^i_w)}{w \subs{x}{v}}{\mul{\tau_i}_{\iI}}{(+_{\iI}(b_i+b^i_v),+_{\iI}(m_i+m^i_v),+_{\iI}(d_i+d^i_v))}}
            \end{prooftree} \]
            And we can conclude with $\Gam_t + \Gam_v = +_{\iI} \Gam_i +_{\iI} \Gam^i_v = +_{\iI} (\Gam_i + \Gam^i_v)$, $b_t + b_v = +_{\iI} b_i +_{\iI} b^i_v = +_{\iI} (b_i + b^i_v)$, $m_t + m_v = +_{\iI} m_i +_{\iI} m^i_v = +_{\iI} (m_i + m^i_v)$, and $d_t + d_v = +_{\iI} d_i +_{\iI} d^i_v = +_{\iI} (d_i + d^i_v)$.
            \item Case $\Phi_t$ ends with (\ruleLift). Then $t$ is a variable and $\Phi_t$ must be of the following form:
            \[ \begin{prooftree}
                \hypo{\Phi_w \tr \seqi{\Gam; x : \M}{w}{\M'}{(b_t,m_t,d_t)}}
                \infer1[(\ruleLift)]{\seqi{\Gam; x : \M}{w}{\tcomptype{\stype}{\M'}{\stype}}{(b_t,m_t,d_t)}}
            \end{prooftree} \]
            where $\gtype = \tcomptype{\stype}{\M'}{\stype}$. By the \ih, we have $\Phi_{w \subs{x}{v}} \tr \seqi{\Gam + \Gam_v}{w \subs{x}{v}}{\M'}{(b_t+b_v, m_t+m_v, d_t +d_v)}$. Therefore, we can build $\Phi_{t \subs{x}{v}}$ as follows:
            \[ \begin{prooftree}
                \hypo{\Phi_{w \subs{x}{v}} \tr \seqi{\Gam + \Gam_v}{w \subs{x}{v}}{\M'}{(b_t+b_v, m_t+m_v, d_t +d_v)}}
                \infer1[(\ruleLift)]{\seqi{\Gam + \Gam_v}{w \subs{x}{v}}{\tcomptype{\stype}{\M'}{\stype}}{(b_t+b_v, m_t+m_v, d_t +d_v)}}
            \end{prooftree} \]
            And we can conclude.
            \item Case $\Phi_t$ ends with ($\ruleGet$). Then $t$ must be of the form $\get{l}{y}{u}$ and $\Phi_t$ must be of the following form:
            \[ \begin{prooftree}
                \hypo{\Phi_u \tr \seqi{\Gam_u; x : \M}{u}{\comptype{\stype}{\ctype}}{(b_u,m_u,d_u)}}
                \infer1[(\ruleGet)]{\seqi{(\Gam_u \sm y); x : \M}{\get{l}{y}{u}}{\comptype{\conj{(l : \Gam_{u}(y))} \splus \stype}{\ctype}}{(b_u,1+m_u,d_u)}}
            \end{prooftree} \]
          where $\gtype = \comptype{\conj{(l : \Gam_{u}(y))} \splus \stype}{\ctype}$, $\Gam_t = \Gam_u \sm y$, $b_t = b_u$, $m_t = 1+m_u$, and $d_t = d_u$. By the \ih, we have $\Phi_{u \subs{x}{v}} \tr \seqi{\Gam_u + \Gam_v}{u \subs{x}{v}}{\stype \ra \ctype}{(b_u+b_v,m_u+m_v,d_u+d_v)}$. Therefore, we can build $\Phi_{t \subs{x}{v}}$ as follows:
            \[ \begin{prooftree}
                \hypo{\Phi_{u \subs{x}{v}} \tr \seqi{\Gam_u + \Gam_v}{u \subs{x}{v}}{\comptype{\stype}{\ctype}}{(b_u+b_v,m_u+m_v,d_u+d_v)}}
                \infer1[(\ruleGet)]{\seqi{(\Gam_u  + \Gam_v) \sm y}{\get{l}{y}{u} \subs{x}{v}}{\comptype{\conj{(l : \Gam_u(y))} \splus \stype}{\ctype}}{(b_u+b_v,1+m_u+m_v,d_u+d_v)}}
            \end{prooftree} \]
            And we can conclude with $(\Gam_u + \Gam_v) \sm y = (\Gam \sm y) + \Gam_v = \Gam_t + \Gam_v$ by $\alpha$-conversion, $b_t + b_v = b_u+b_v$, $m_t + m_v = 1+m_u+m_v$, and $d_t + d_v = d_u +d_v$.
            \item Case $\Phi_t$ ends with ($\ruleSet$). Then $t$ must be of the form $\set{l}{w}{u}$ and $\Phi_t$ must be of the following form:
            \[ \begin{prooftree}
                \hypo{\Phi_w \tr \seqi{\Gam_w; x : \M_1}{w}{\M'}{(b_w,m_w,d_w)}}
                \hypo{\Phi_u \tr \seqi{\Gam_u; x : \M_2}{u}{\comptype{\conj{(l : \M')}; \stype}{\ctype}}{(b_u,m_u,d_u)}}
                \infer2[(\ruleSet)]{\seqi{\Gam_w + \Gam_u; x : \M_1 \sqcup \M_2}{\set{l}{w}{u}}{\comptype{\stype}{\ctype}}{(b_w+b_u,1+m_w+m_u,d_w+d_u)}}
            \end{prooftree} \]
            where $\gtype = \comptype{\stype}{\ctype}$, $\Gam_t = \Gam_w + \Gam_u$, $\del = \comptype{\stype}{\ctype}$, $b_t = b_w+b_u$, $m_t = 1+m_w + m_u$, and $d_t = d_w + d_u$. By~\cref{lem:split-values-stores}.\ref{lem:com-split-values}, we have $\Phi^1_v \tr \seqi{\Gam^1_v}{v}{\M_1}{(b^1_v,m^1_v,d^1_v)}$ and $\Phi^2_v \tr \seqi{\Gam^2_v}{v}{\M_2}{(b^2_v,m^2_v,d^2_v)}$, such that $\Gam_v = \Gam^1_v + \Gam^2_v$, $b_v = b^1_v + b^2_v$, $m_v = m^1_v + m^2_v$, and $d_v = d^1_v + d^2_v$. By the \ih, we have $\Phi_{w \subs{x}{v}} \tr \seqi{\Gam_w + \Gam^1_v}{w \subs{x}{v}}{\M'}{(b_w+b^1_v,m_w+m^1_v,d_w+d^1_v)}$ and $\Phi_{u \subs{x}{v}} \tr \seqi{\Gam_u + \Gam^2_v}{u \subs{x}{v}}{\comptype{\conj{(l : \M')}; \stype}{\ctype}}{(b_u+b^2_v,m_u+m^2_u,d_u+d^2_v)}$. Assume $\Phi_{w \subs{x}{v}} \tr \seqi{\Gam_w + \Gam^1_v}{w \subs{x}{v}}{\M'}{(b_w+b^1_v,m_w+m^1_v,d_w+d^1_v)}$ and $\Phi_{u \subs{x}{v}} \tr \seqi{\Gam_u + \Gam^2_v}{u \subs{x}{v}}{\comptype{\conj{(l : \M')}; \stype}{\ctype}}{(b_u+b^2_v,m_u+m^2_u,d_u+d^2_v)}$. We can build $\Phi_{t \subs{x}{v}}$ as follows:
            \[ \begin{prooftree}
                \hypo{\Phi_{w \subs{x}{v}}}
                \hypo{\Phi_{u \subs{x}{v}}}
                \infer2[(\ruleSet)]{\seqi{(\Gam_w + \Gam_u) + (\Gam^1_v + \Gam^2_v)}{(wu) \subs{x}{v}}{\comptype{\stype}{\ctype}}{(b_w+b_u+b^1_v+b^2_v,1+m_w+m_u+m^1_v+m^2_v,d_w+d_u+d^1_v+d^2_v)}}
            \end{prooftree} \]
            And we can conclude with $\Gam_t + \Gam_v = (\Gam_w + \Gam_u) + (\Gam^1_v + \Gam^2_v)$, $b_t + b_v = b_w+b_u+b^1_v+b^2_v$, $m_t + m_v = 1+m_w+m_u+m^1_v+m^2_v$, $d_t + d_v = d_w+d_u+d^1_v+d^2_v$.
            \item Case $\Phi_t$ ends with (\ruleAxP). Then $t$ must be a variable and we must consider two cases:
            \begin{itemize}
                \item Assume $t = y = x$. Then $\Gam_t = \eset$, $\gtype = \stype \ta (\comptype{\nott{\tneutral}}{\stype})$, $t \subs{x}{v} = v$, $b_t = m_t = d_t = 0$. Moreover, $\M = \mul{\nott{\tneutral}}$. We have to consider two cases:
                \begin{itemize}
                    \item Case $v = z$. Then $\Phi_v \tr \seqi{z : \mul{\nott{\tneutral}}}{z}{\mul{\nott{\tneutral}}}{(0,0,0)}$. So we can take $\Phi_{t \subs{x}{v}}$ as the following derivation:
                    \[ \begin{prooftree}
                        \infer0[(\ruleAxP)]{\seqi{z : \mul{\nott{\tneutral}}}{z}{\tcomptype{\stype}{\nott{\tneutral}}{\stype}}{(0,0)}}
                    \end{prooftree} \]
                    and conclude with $\Gam_t + \Gam_v = \Gam_v = (z : \mul{\nott{\tneutral}})$, $b_t + b_v = b_v = 0$, $m_t + m_v = m_v = 0$, and $d_t + d_v = d_v$.
                    \item Case $v = \lam z.p$. This case does not apply, by~\cref{lem:comp-notabs-implies-negabs}.
                \end{itemize}
                \item Assume $t = y \neq x$. Then $\M = \emul$, $\Gam_v = \eset$, $t \subs{x}{v} = t$, $b_v = 0$, $m_v = 0$, and $d_v = 0$. So we can take $\Phi_{t \subs{x}{v}} = \Phi_t$ and conclude with $\Gam_t + \Gam_v = \Gam_t$, $b_t + b_v = b_t$, $m_t + m_v = m_t$, and $d_t + d_v = d_t$.
            \end{itemize}
            \item Case $\Phi_t$ ends with (\ruleLamP). Then $t$ is of the form $\lam y.u$, $\Gam_t = \eset$, $\gtype = \tcomptype{\stype}{\vl}{\stype}$, $\M = \emul$, $\Gam_v = \eset$, $t \subs{x}{v} = \lam y.(u \subs{x}{v}) = (\lam y.u) \subs{x}{v}$, $b_t = b_v = 0$, $m_t = m_v = 0$, and $d_t = d_v = 0$. So we can build $\Phi_{t \subs{x}{v}}$ as follows:
            \[ \begin{prooftree}
                \infer0[(\ruleLamP)]{\seqi{}{(\lam y.u) \subs{x}{v}}{\tcomptype{\stype}{\vl}{\stype}}{(0,0,0)}}
            \end{prooftree} \]
            And conclude with $\Gam_t + \Gam_v = \eset$, $b_t = b_v = 0$, $m_t = m_v = 0$, and $d_t = d_v = 0$.
            \item Case $\Phi_t$ ends with (\ruleAppPOne). Then $t$ is of the form $yu$ and we have to consider two cases:
            \begin{itemize}
                \item Case $y = x$. Then $\Phi_t$ must be of the following form:
                \[ \begin{prooftree}
                    \hypo{\seqi{\Gam_u}{u}{\tcomptype{\stype}{\tightt}{\stype}}{(b_u,m_u,d_u)}}
                    \infer1[(\ruleAppPOne)]{\seqi{(x : \mul{\tvar} \sqcup \Gam_u(x)); (\Gam_u \sm x)}{x u}{\tcomptype{\stype}{\tneutral}{\stype}}{(b_u,m_u,1+d_u)}}
                \end{prooftree} \]
                such that $\Gam_t = (\Gam_u \sm x)$, $b = b_u$, $m = m_u$, and $d = 1+d_u$. Then $\M = \mul{\tvar} \sqcup \Gam_u(x)$ and, by~\cref{lem:split-values-stores}.\ref{lem:com-split-values}, we have $\Phi^1_v \tr \seqi{\Gam^1_v}{v}{\mul{\tvar}}{(b^1_v,m^1_v,d^1_v)}$ and $\Phi^2_v \tr \seqi{\Gam^2_v}{v}{\Gam_u(x)}{(b^2_v,m^2_v,d^2_v)}$, such that $\Gam_v = \Gam^1_v + \Gam^2_v$, $b_v = b^1_v + b^2_v$, $m_v = m^1_v + m^2_v$, and $d_v = d^1_v + d^2_v$. By the \ih, we know there exists $\Phi_{u \subs{x}{v}} \tr \seqi{(\Gam_u \sm x) + \Gam^2_u}{u \subs{x}{v}}{\tcomptype{\stype}{\tightt}{\stype}}{(b_u+b^2_v, m_u+m^2_v, d_u+d^2_v)}$. Now, we need to consider two cases:
                \begin{itemize}
                    \item Case $v = z$. Then $\Phi^1_v \tr \seqi{z : \mul{\tvar}}{z}{\mul{\tvar}}{(0,0,0)}$ and $\Phi^2_v \tr \seqi{z : \Gam_u(x)}{z}{\Gam_u(x)}{(0,0)}$. Therefore, we can build $\Phi_{t \subs{x}{v}} = \Phi_{v \subs{x}{v}}$ as follows:
                    \[ \begin{prooftree}
                        \hypo{\Phi_{u \subs{x}{v}} \tr \seqi{(\Gam_u \sm x) + \Gam_u(x)}{u \subs{x}{v}}{\tcomptype{\stype}{\tightt}{\stype}}{(b_u+b^2_v, m_u+m^2_v, d_u+d^2_v)}}
                        \infer1[(\ruleAppPOne)]{\seqi{(z : \mul{\tvar}) + (\Gam_u \sm x + (z : \Gam_u(x)))}{z (u \subs{x}{v})}{\tcomptype{\stype}{\tneutral}{\stype}}{(b_u+b^2_v,m_u+m^2_v,1+d_u+d^2_v)}}
                    \end{prooftree} \]
                    where $(z : \mul{\tvar}) + (\Gam_u \sm x + (z : \Gam_u(x))) = (\Gam_u \sm x) + (z : \mul{\tvar} \cup \Gam_u(x)) = \Gam_u + \Gam_v$, $b_u + b^2_v = b + b^1_v + b^2_v = b + b_v$, $m_u + m^2_v = m + m^1_v + m^2_v = m + m_v$, and $d_u + d^2_v = d + d^1_v + d^2_v = d + d_v$.
                    \item Case $v = \lam z.p$. This case does not apply, since it is not possible to assign $\tvar$ to $\lam z.p$, by~\cref{lem:comp-notabs-implies-negabs}.
                \end{itemize}
                \item Case $y \neq x$. Then, the proof is very similar to when $\Phi_t$ ends with rule ($\ruleApp$).
            \end{itemize}
            \item Case $\Phi_t$ ends with (\ruleAppPTwo), the proof is very similar to when $\Phi_t$ ends with rule (\ruleAppPOne).
        \end{itemize}
%\end{proof}

        \item %\begin{proof}
    We are going to generalize the original statement by replacing $\del$ with $\gtype$.
    \\ \\
    The proof follows by induction over $t$:
    \begin{itemize}
        \item Case $t = y$. Then we have to consider two cases:
        \begin{itemize}
            \item Let $t = y \not= x$. Then $t \subs{x}{v} = y$. Let $\Gam_v = \eset$, $\M = \emul$, $b_v = m_v = d_v = 0$. Then, $\Phi_v$ is derivable using rule ($\ruleMany$) with no premise. We also take $\Phi_t = \Phi_{t \subs{x}{v}}$, so that, in particular $\Gam_t = \Gam_{t \subs{x}{v}}$. Then, we can conclude with $\Gam_{t \subs{x}{v}} = \Gam_t + \Gam_v = \Gam_t$, $b = b_t + b_v = b_t$, $m = m_t + m_v = m_t$, and $d = d_t + d_v = d_t$.
            \item Let $t = y = x$. Then $t \subs{x}{v} = v$. Let $\Gam_t = \eset$, and $b_t = m_t = s_t = 0$. Now we will consider two cases depending on the form of $v$:
            \begin{itemize}
                \item Case $v = z$. Then $t \subs{x}{v} = z$ and we can proceed by case analysis of the last rule in $\Phi_{t\subs{x}{v}}$. In all of them, we can build $\Phi_t$ from $\Phi_{t \subs{x}{v}}$, by simply replacing $x$ with $z$, and $\Phi_v$ as follows:
                \[ \begin{prooftree}
                    \infer0[(\ruleAx)]{\seqi{z : \mul{\sig}}{z}{\sig}{(0,0,0)}}
                    \infer1[(\ruleMany)]{\seqi{z : \mul{\sig}}{z}{\mul{\sig}}{(0,0,0)}}
                \end{prooftree} \]
                And we can conclude since all the counters are zero.
                \item Case $v = \lam z.p$. Then $t \subs{x}{v} = \lam z.p$ and we can proceed by case analysis of the last rule in $\Phi_{t \subs{x}{v}}$. In all of them, we can always build $\Phi_t$ using either (\ruleAx) (case (\ruleApp)), (\ruleAxP) (case (\ruleLamP)),  (\ruleAx) plus (\ruleMany) (case (\ruleMany)), or (\ruleAx) plus (\ruleMany) plus (\ruleLift) (case (\ruleLift)). $\Phi_v$  is either $\Phi_{t \subs{x}{v}}$ (case (\ruleMany)), or it can be built from $\Phi_{t \subs{x}{v}}$ plus rule (\ruleMany) (all other cases).
            \end{itemize}
        \end{itemize}
        \item Case $t = \lam y.u$. Then $t \subs{x}{v} = (\lam y.u)\subs{x}{v} = \lam y.(u \subs{x}{v})$ and we must consider three cases:
        \begin{itemize}
            \item Case $\Phi_{t \subs{x}{v}}$ ends with rule (\ruleLam), then it must be of the following form: 
            \[ \begin{prooftree}
                \hypo{\Phi_{u \subs{x}{v}} \tr \seqi{\Gam_{u \subs{x}{v}}; y : \M'}{u \subs{x}{v}}{\comptype{\stype}{\ctype}}{(b,m,d)}}
                \infer1[(\ruleLam)]{\seqi{\Gam_{u \subs{x}{v}}}{\lam y.(u \subs{x}{v})}{\M' \ta (\comptype{\stype}{\ctype})}{(b,m,d)}}
            \end{prooftree} \]
            where $\gtype = \M' \ta (\comptype{\stype}{\ctype})$ and $\Gam_{t \subs{x}{v}} = \Gam_{u \subs{x}{v}}$. By the \ih, we have $\Phi_u \tr \seqi{\Gam_u; y : \M'; x : \M}{u}{\comptype{\stype}{\ctype}}{(b_u,m_u,d_u)}$ and $\Phi_v \tr \seqi{\Gam_v}{v}{\M}{(b_v,m_v,d_v)}$, such that $\Gam_{u \subs{x}{v}} = \Gam_u + \Gam_v$, $b = b_u + b_v$, $m = m_u + m_v$, and $d = d_u + d_v$. So we can build $\Phi_{\lam y.u}$ as follows:
            \[ \begin{prooftree}
                \hypo{\Phi_u \tr \seqi{\Gam_u; y: \M'; x : \M}{u}{\comptype{\stype}{\ctype}}{(b_u,m_u,d_u)}}
                \infer1[(\ruleLam)]{\seqi{\Gam_u; x : \M}{\lam y.u}{\M' \ta (\comptype{\stype}{\ctype})}{(b_u,m_u,d_u)}}
            \end{prooftree} \]
            And we can pick $\Phi_t = \Phi_{\lam y.u}$, and conclude with $\Gam_{t \subs{x}{v}} = \Gam_{u \subs{x}{v}} = \Gam_u + \Gam_v$, $b = b_u + b_v$, $m = m_u + m_v$, and $d = d_u + d_v$.
            \item Case $\Phi_{t \subs{x}{v}}$ ends with rule (\ruleLamP). Then it must be of the following form:
            \[ \begin{prooftree}
                \infer0[(\ruleLamP)]{\seqi{}{\lam y.(u \subs{x}{y})}{\tcomptype{\stype}{\vl}{\stype}}{(0,0,0)}}
            \end{prooftree} \]
            where $\Gam_{t \subs{x}{v}} = \eset$, $\gtype = \tcomptype{\stype}{\vl}{\stype}$, and $b = m = d = 0$. Let $\Gam_t = \eset$, $\M = \emul$, and $b_t = m_t = d_t = 0$. Then, we can construct $\Phi_t$ as follows:
            \[ \begin{prooftree}
                \infer0[(\ruleLamP)]{\seqi{}{\lam y.u}{\tcomptype{\stype}{\vl}{\stype}}{(0,0,0)}}
            \end{prooftree} \]
            Let $\Gam_v = \eset$, and $b_v = m_v = d_v = 0$. Then $\Phi_v$ can be constructed by using rule ($\ruleMany$) with no premises. So we can conclude with $\Gam_{t \subs{x}{v}} = \eset = \Gam_t + \Gam_v$, and $b = 0 = b_t + b_v$, $m = 0 = m_t + m_v$, and $d = 0 = d_t + d_v$.
            \item Case $\Phi_{t \subs{x}{v}}$ ends with rule ($\ruleMany$). Then $t \subs{x}{v}$ is a value, and $\Phi_{t \subs{x}{v}}$ must be of the following form:
            \[ \begin{prooftree}
                \hypo{(\Phi_i \tr \seqi{\Gam_i}{t \subs{x}{v}}{\rdel_i}{(b_i,m_i,d_i)})_{\iI}}
                \infer1[(\ruleMany)]{\seqi{+_{\iI} \Gam_i}{t \subs{x}{v}}{\mul{\rdel_i}_{\iI}}{(+_{\iI} b_i, +_{\iI} m_i, +_{\iI} d_i)}}
            \end{prooftree} \]
            where $\gtype = \mul{\rdel_i}_{\iI}$, $\Gam_{t \subs{x}{v}} = +_{\iI} \Gam_i$, $b = +_{\iI} b_i$, $m = +_{\iI} m_i$, and $d = +_{\iI} d_i$. By the \ih over each $\Phi_i$, we have the following derivations $\Phi^i_t \tr \seqi{\Gam^i_t; x : \M_i}{t}{\rdel_i}{(b^i_t,m^i_t,d^i_t)}$ and $\Phi^i_v \tr \seqi{\Gam^i_v}{v}{\M_i}{(b^i_v, m^i_v, d^i_v)}$, such that $\Gam_i = \Gam^i_t + \Gam^i_v$, $b = b^i_t + b^i_v$, $m = m^i_t + m^i_v$, and $d = d^i_t + d^i_v$, for each $\iI$. So we can construct $\Phi_t$ as follows:
            \[ \begin{prooftree}
                \hypo{(\Phi^i_t \tr \seqi{\Gam^i_t; x : \M_i}{t}{\rdel_i}{(b^i_t,m^i_t,d^i_t)})_{\iI}}
                \infer1[(\ruleMany)]{\seqi{+_{\iI} \Gam^i_t; x : \sqcup_{\iI} \M_i}{t}{\mul{\rdel_i}_{\iI}}{(+_{\iI} b^i_t, +_{\iI} m^i_t, +_{\iI} d^i_t)}}
            \end{prooftree} \]
            such that $\Gam_t = +_{\iI} \Gam^i_t$, $\M = \sqcup_{\iI} \M_i$, $b_t = +_{\iI} b^i_t$, $m_t = +_{\iI} m^i_t$, and $d_t = +_{\iI} d^i_t$. By~\cref{lem:comp-merge-values}, we can take the following derivation $\Phi_v \tr \seqi{+_{\iI} \Gam^i_v}{v}{\M}{(+_{\iI} b^i_v, +_{\iI} m^i_v, +_{\iI} d^i_v)}$. And we can conclude with $\Gam_{t \subs{x}{v}} = +_{\iI} \Gam_i = +_{\iI} (\Gam^i_t + \Gam^i_v) = +_{\iI} \Gam^i_t +_{\iI} \Gam^i_v = \Gam_t + \Gam_v$, $b = +_{\iI} b_i = +_{\iI} (b^i_t + b^i_v) = +_{\iI} b^i_t +_{\iI} b^i_v = b_t + b_v$, $m = +_{\iI} m_i = +_{\iI} (m^i_t + m^i_v) = +_{\iI} m^i_t +_{\iI} m^i_v = m_t + m_v$, and $d = +_{\iI} d_i = +_{\iI} (d^i_t + d^i_v) = +_{\iI} d^i_t +_{\iI} d^i_v = d_t + d_v$.
        \end{itemize}
        \item Let $t = wu$. Then $t \subs{x}{v} = (wu) \subs{x}{v} = (w \subs{x}{v})(u \subs{x}{v})$, and we have to consider three cases:
        \begin{itemize}
            \item Case $\Phi_{t \subs{x}{v}}$ ends with ($\ruleApp$). Assume $\Phi_{w \subs{x}{v}} \tr \seqi{\Gam_{w \subs{x}{v}}}{w \subs{x}{v}}{\M' \ta (\comptype{\stype'}{\ctype})}{(b',m',d')}$ and $\Phi_{u \subs{x}{v}} \tr \seqi{\Gam_{u \subs{x}{v}}}{u \subs{x}{v}}{\tcomptype{\stype}{\M'}{\stype'}}{(b'',m'',d'')}$. $\Phi_{t \subs{x}{v}}$ must be of the following form:
            \[ \begin{prooftree}
                \hypo{\Phi_{w \subs{x}{v}}}
                \hypo{\Phi_{u \subs{x}{v}}}
                \infer2[(\ruleApp)]{\seqi{\Gam_{w \subs{x}{v}} + \Gam_{u \subs{x}{v}}}{(w \subs{x}{v})(u \subs{x}{v})}{\comptype{\stype}{\ctype}}{(1+b'+b'',m'+m'',d'+d'')}}
            \end{prooftree} \]
            where $\gtype = \comptype{\stype}{\ctype}$, $\Gam_{t \subs{x}{v}} = \Gam_{w \subs{x}{v}} + \Gam_{u \subs{x}{v}}$, $b = 1+b'+b''$, $m = m'+m''$, and $d = d'+d''$. By the \ih over $\Phi_{w \subs{x}{v}}$, we have $\Phi_w \tr \seqi{\Gam_w; x : \M_1}{w}{\M' \ta (\comptype{\stype'}{\ctype})}{(b_w,m_w,d_w)}$ and $\Phi^1_v \tr \seqi{\Gam^1_v}{v}{\M_1}{(b^1_v,m^1_v,d^1_v)}$, such that $\Gam_{w \subs{x}{v}} = \Gam_w + \Gam^1_v$, $b' = b_w + b^1_v$, $m' = m_w + m^1_v$, and $d' = d_w + d^1_v$. And by the \ih over $\Phi_{u \subs{x}{v}}$, we have $\Phi_u \tr \seqi{\Gam_u; x : \M_2}{u}{\tcomptype{\stype}{\M'}{\stype'}}{(b_u, m_u,d_u)}$ and $\Phi^2_v \tr \seqi{\Gam^2_v}{v}{\M_2}{(b^2_v,m^2_v,d^2_v)}$, such that $\Gam_{u \subs{x}{v}} = \Gam_u + \Gam^2_v$, $b'' = b_u + b^2_v$, $m'' = m_u + m^2_v$, and $d'' = d_u + d^2_v$. By~\cref{lem:comp-merge-values}, we can take $\Phi_v \tr \seqi{\Gam^1_v + \Gam^2_v}{v}{\M_1 \sqcup \M_2}{(b^1_v+b^2_v, m^1_v+m^2_v, d^1_v+d^2_v)}$, such that $\Gam_v = \Gam^1_v + \Gam^2_v$, $b_v = b^1_v + b^2_v$, $m_v = m^1_v + m^2_v$, and $d_v = d^1_v + d^2_v$. And we can build $\Phi_{wu}$ as follows:
            \[ \begin{prooftree}
                \hypo{\Phi_w}
                \hypo{\Phi_u}
                \infer2[(\ruleApp)]{\seqi{(\Gam_w + \Gam_u); x : \M_1 \sqcup \M_2}{wu}{\comptype{\stype}{\kappa}}{(1+b_w+b_u,m_w+m_u,d_w+d_u)}}
            \end{prooftree} \]
            such that $\Gam_t = \Gam_w + \Gam_u$, $b_t = 1 + b_w + b_u$, $m_t = b_w + b_u$, and $d_t = d_w + d_u$. So we can pick $\Phi_t = \Phi_{wu}$, and conclude with $\Gam_{t \subs{x}{v}} = \Gam_{w \subs{x}{v}} + \Gam_{u \subs{x}{v}} = (\Gam_w + \Gam^1_v) + (\Gam_u + \Gam^2_v) = (\Gam_w + \Gam_u) + (\Gam^1_v + \Gam^2_v) = \Gam_t + \Gam_v$, $b = 1 + b' + b'' = 1 + b_w + b^1_v + b_u + b^2_v = (1 + b_w + b_u) + (b^1_v + b^2_v) = b_t + b_v$, $m = m' + m'' = m_w + m^1_v + m_u + m^2_v = (m_w + m_u) + (m^1_v + m^2_v) = m_t + m_v$, and $d = d' + d'' = d_w + d^1_v + d_u + d^2_v = (d_w + d_u) + (d^1_v + d^2_v) = d_t + d_v$.
            \item Case $\Phi_{t \subs{x}{v}}$ ends with (\ruleAppPOne) or (\ruleAppPTwo). These cases are very similar to the case where $\Phi_{t \subs{x}{v}}$ ends with (\ruleApp).
        \end{itemize}
        \item Let $t = \get{l}{y}{u}$. Then $t \subs{x}{v} = \get{l}{y}{u \subs{x}{v}}$ and $\Phi_{t \subs{x}{v}}$ must be of the following form:
        \[ \begin{prooftree}
            \hypo{\Phi_{u \subs{x}{v}} \tr \seqi{\Gam_{u \subs{x}{v}}; y : \M'}{u \subs{x}{v}}{\comptype{\stype}{\ctype}}{(b,m',d)}}
            \infer1[(\ruleGet)]{\seqi{\Gam_{u \subs{x}{v}}}{\get{l}{y}{u \subs{x}{v}}}{\comptype{\conj{(l : \M')} \splus \stype}{\ctype}}{(b,1+m',d)}}
        \end{prooftree} \]
        where $\Gam_{t \subs{x}{v}} = \Gam_{u \subs{x}{v}}$ and $m = 1+m'$. By the \ih, we have $\Phi_u \tr \seqi{\Gam_u; y : \M'; x : \M}{u}{\comptype{\stype}{\ctype}}{(b_u,m_u,d_u)}$ and $\Phi_v \tr \seqi{\Gam_v}{v}{\M}{(b_v,m_v,d_v)}$, such that $\Gam_{u \subs{x}{v}} = \Gam_u + \Gam_v$, $b = b_u + b_v$, $m' = m_u + m_v$, and $d = d_u + d_v$. So we can build $\Phi_{\get{l}{y}{u}}$ as follows:
        \[ \begin{prooftree}
            \hypo{\Phi_u \tr \seqi{\Gam_u; y : \M'; x : \M}{u}{\comptype{\stype}{\ctype}}{(b_u,m_u,d_u)}}
            \infer1[(\ruleGet)]{\seqi{\Gam_{u}; x : \M}{\get{l}{y}{u}}{\comptype{\conj{(l : \M')} \splus \stype}{\ctype}}{(b_u,1+m_u,d_u)}}
        \end{prooftree} \]
        And we can pick $\Phi_t = \Phi_{\get{l}{y}{u}}$, and conclude with $\Gam_{t \subs{x}{v}} = \Gam_{u \subs{x}{v}} = \Gam_u + \Gam_v$, $b = b_u + b_v$, $m = 1 + m' = 1 + m_u + m_v = (1 + m_u) + m_v$, and $d = d_u + d_v$.
        \item Let $t = \set{l}{w}{u}$. Then $t \subs{x}{v} = (\set{l}{w}{u}) \subs{x}{v} = \set{l}{w \subs{x}{v}}{u \subs{x}{v}}$. Assume $\Phi_{w \subs{x}{v}} \tr \seqi{\Gam_{w \subs{x}{v}}}{w \subs{x}{v}}{\M}{(b',m',d')}$ and $\Phi_{u \subs{x}{v}} \tr \seqi{\Gam_{u \subs{x}{v}}}{u \subs{x}{v}}{\comptype{\conj{(l : \M)}; \stype}{\ctype}}{(b'',m'',d'')}$. $\Phi_{t \subs{x}{v}}$ must be of the following form:
        \[ \begin{prooftree}
            \hypo{\Phi_{w \subs{x}{v}}}
            \hypo{\Phi_{t \subs{x}{v}}}
            \infer2[(\ruleSet)]{\seqi{\Gam_{w \subs{x}{v}} + \Gam_{u \subs{x}{v}}}{\set{l}{w \subs{x}{v}}{u \subs{x}{v}}}{\comptype{\stype}{\ctype}}{(b'+b'',1+m'+m'',d'+d'')}}
        \end{prooftree} \]
        where $\Gam_{t \subs{x}{v}} = \Gam_{w \subs{x}{v}} + \Gam_{u \subs{x}{v}}$, $b = b'+ b''$, $m = 1 + m'+m''$, and $d = d' + d''$. By the \ih over $\Phi_{w \subs{x}{v}}$, we have $\Phi_w \tr \seqi{\Gam_w; x : \M_1}{w}{\M}{(b_w,m_w,d_w)}$ and $\Phi^1_v \tr \seqi{\Gam^1_v}{v}{\M_1}{(b^1_v,m^1_v,d^1_v)}$, such that $\Gam_{w \subs{x}{v}} = \Gam_w + \Gam^1_v$, $b' = b_w + b^1_v$, $m'= m_w + m^1_v$, and $d' = d_w+d^1_v$. And by the \ih over $\Phi_{u \subs{x}{v}}$, we have $\Phi_u \tr \seqi{\Gam_u; x : \M_2}{u}{\comptype{\conj{(l : \M)}; \stype}{\ctype}}{(b_u,m_u,d_u)}$ and $\Phi^2_v \tr \seqi{\Gam^2_v}{v}{\M_2}{(b^2_v,m^2_v,d^2_v)}$, such that $\Gam_{u \subs{x}{v}} = \Gam_u + \Gam^2_v$, $b'' = b_u + b^2_v$, $m'' = m_u + m^2_v$, and $d'' = d_u + d^2_v$. By~\cref{lem:comp-merge-values}, we can take $\Phi_v \tr \seqi{\Gam^1_v + \Gam^2_v}{v}{\M_1 \sqcup \M_2}{(b^1_v + b^2_v, m^1_v+m^2_v, d^1_v + d^2_v)}$, such that $\Gam_v = \Gam^1_v + \Gam^2_v$, $b_v = b^1_v + b^2_v$, $m_v = m^1_v + m^2_v$, and $d_v = d^1_v + d^2_v$. And we can build $\Phi_{\set{l}{w}{u}}$ as follows:
        \[ \begin{prooftree}
            \hypo{\Phi_w \tr \seqi{\Gam_w; x : \M_1}{w}{\M}{(b_w,m_w,d_w)}}
            \hypo{\Phi_u \tr \seqi{\Gam_u; x : \M_2}{u}{\comptype{\conj{(l : \M)}; \stype}{\ctype}}{(b_u,m_u,d_u)}}
            \infer2[(\ruleSet)]{\seqi{(\Gam_w + \Gam_u); x : \M_1 \sqcup \M_2}{\set{l}{w}{u}}{\comptype{\stype}{\ctype}}{(b_w+b_u, 1+m_w+m_u,d_w+d_u)}}
        \end{prooftree} \]
        such that $\Gam_t = \Gam_w + \Gam_u$, $b_t = b_w + b_u$, $m_t = 1 + m_w + m_u$, and $d_t = d_w + d_u$. So we can pick $\Phi_t = \Gam_{\set{l}{w}{u}}$, and conclude with $\Gam_{t \subs{x}{v}} = \Gam_{w \subs{x}{v}} + \Gam_{u \subs{x}{v}} = (\Gam_w + \Gam^1_u) + (\Gam_u + \Gam^2_v) = (\Gam_w + \Gam_u) + (\Gam^1_v + \Gam^2_v) = \Gam_t + \Gam_v$, $b = b' + b'' = (b_w + b^1_v) + (b_u + b^2_v) = (b_w + b_u) + (b^1_v + b^2_v) = b_t + b_v$, $m = 1+ m' + m'' = 1+ (m_w + m^1_v) + (m_u + m^2_v) = (1 + m_w + m_u) + (m^1_v + m^2_v) = m_t + m_v$, and $d = d' + d'' = (d_w + d^1_v) + (d_u + d^2_v) = (d_w + d_u) + (d^1_v + d^2_v) = d_t + d_v$.
    \end{itemize}
%\end{proof}

    \end{enumerate}
\end{proof}}

\lemexactredexp*

\maybehide{\begin{proof} \mbox{}
    \begin{enumerate} 
        \item %\begin{proof}
  We show a stronger statement of the form:

  Let $(t,s) \red[\gname] (u,q)$. If $\Phi \tr \seqi{\Gam}{(t,s)}{\ctype}{(b,m,d)}$, $\Gam$ is tight,  and ($\ctype$ is tight or $\neg \isvalue{t}$), then $\Phi' \tr \seqi{\Gam}{(u,q)}{\ctype}{(b',m',d)}$, where $\gname =\beta$ implies $b' = b - 1$ and $m' = m$, while  $\gname \in \{\getname, \setname\}$ implies $b'=b$ and $m' = m - 1$.

  We proceed by induction on $(t,s) \ra (u,q)$: 
  \begin{itemize}
    \item Case $(t,s) = ((\lam x.p) v,s) \redbeta (p \subs{x}{v}, s) = (u,q)$. Let $\Phi_{(\lam x.p) v}$ be the sub-derivation for $(\lam x.p) v$ in $\Phi$. Assume that $\Phi_{(\lam x.p)v}$ ends with rule (\ruleAppPTwo). Then $v$ must be assigned type $\comptype{\stype}{\tneutral \tim \stype}$, which is not possible by~\cref{lem:comp-values-not-neutral}. Let $\Phi_0$ be the following derivation:
    \[ \begin{prooftree}
      \hypo{\Phi_p \tr \seqi{\Gam_{\lam x.p}; x : \M}{p}{\comptype{\stype}{\ctype}}{(b_p,m_p,d_p)}}
        \infer1[(\ruleLam)]{\seqi{\Gam_{\lam x.p}}{\lam x.p}{\M \ta (\comptype{\stype}{\ctype})}{(b_p,m_p,d_p)}}
        \hypo{\Phi_v \tr \seqi{\Gam_v}{v}{\M}{(b_v,m_v,d_v)}}
        \infer1[(\ruleLift)]{\seqi{\Gam_v}{v}{\tcomptype{\stype}{\M}{\stype}}{(b_v,m_v,d_v)}}
        \infer2[(\ruleApp)]{\seqi{\Gam_{\lam x.p}+\Gam_v}{(\lam x.p)v}{\comptype{\stype}{\ctype}}{(1+b_v+b_p,m_v+m_p,d_v+d_p)}}
    \end{prooftree} \]
    $\Phi_{(\lam x.p) v}$ must end with rule ($\ruleApp$) and $\Phi$ must be of the following form:
    \[ \begin{prooftree}
        \hypo{\Phi_0}
        \hypo{\Phi_s \tr \seqi{\Del}{s}{\stype}{(b_s,m_s,d_s)}}
        \infer2[(\ruleConf)]{\seqi{\Gam_{\lam x.p}+ \Gam_v + \Del}{((\lam x.p)v, s)}{\ctype}{(1+b_v+b_p+b_s,m_v+m_p+m_s,d_v+d_p+d_s)}}
    \end{prooftree} \]
    where  $\Gam = \Gam_{\lam x.p}+ \Gam_v + \Del$,  $b = 1+ b_v + b_p + b_s$, $m = m_v + m_p + m_s$, and $d = d_v + d_p + d_s$. By~\cref{lem:comp-subs-antisubs}.\ref{lem:comp-subs}, there exists $\Phi_{p \subs{x}{v}} \tr \seqi{\Gam_{\lam x.p} +\Gam_v}{p \subs{x}{v}}{\comptype{\stype}{\ctype}}{(b_v+b_p,m_v+m_p,d_v+d_p)}$, therefore we can build $\Phi_{(p\subs{x}{v},s)}$ as follows:
    \[ \begin{prooftree}
        \hypo{\Phi_{p \subs{x}{v}} \tr \seqi{\Gam_{\lam x.p}+\Gam_v}{p \subs{x}{v}}{\comptype{\stype}{\ctype}}{(b_v+b_p,m_v+m_p,d_v+d_p)}}
        \hypo{\Phi_s \tr \seqi{\Del}{s}{\stype}{(b_s,m_s,d_s)}}
        \infer2[(\ruleConf)]{\seqi{\Gam_{\lam x.p} + \Gam_v + \Del}{(p \subs{x}{v}, s)}{\ctype}{(b_v+b_p+b_s,m_v+m_p+m_s,d_v+d_p+d_s)}}
    \end{prooftree} \]
    We can finally conclude since the first counter is equal to $b-1$, while the second and third remain the same.
  \item Case $(t,s) = (vp,s) \ra (vp',q) = (u,q)$, such that $(p,s) \ra (p',q)$. Then we have three cases for the type derivation $\Phi_p$ of $p$ inside $\Phi$: 
    \begin{itemize}
      \item Case $\Phi_p$ ends with ($\ruleApp$). Let $\Phi_0$ be the following derivation:
      \[ \begin{prooftree}
        \hypo{\Phi_v \tr \seqi{\Gam_v}{v}{\M \ta (\comptype{\stype'}{\ctype})}{(b_v,m_v,d_v)}}
            \hypo{\Phi_p \tr \seqi{\Gam_p}{p}{\tcomptype{\stype}{\M}{\stype'}}{(b_p,m_p,d_p)}}
            \infer2[(\ruleApp)]{\seqi{\Gam_v + \Gam_p}{vp}{\comptype{\stype}{\ctype}}{(1+b_v+b_p,m_v+m_p,d_v+d_p)}}
      \end{prooftree} \]
      $\Phi$ must be of the following form:
        \[ \begin{prooftree}
            \hypo{\Phi_0}
            \hypo{\Phi_s \tr \seqi{\Del}{s}{\stype}{(b_s,m_s,d_s)}}
            \infer2[(\ruleConf)]{\seqi{\Gam_v + \Gam_p + \Del}{(vp, s)}{\kappa}{(1+b_v+b_p+b_s,m_v+m_p+m_s,d_v+d_p+d_s)}}
        \end{prooftree} \]
        where $\Gam = \Gam_v + \Gam_p + \Del$ is tight, $b = 1+b_v+b_p+b_s$, $m = m_v+m_p+m_s$, and $s = d_v+d_p+d_s$. Therefore, we can build the following derivation for $(p,s)$:
        \[ \begin{prooftree}
            \hypo{\Phi_p \tr \seqi{\Gam_p}{p}{\tcomptype{\stype}{\M}{\stype'}}{(b_p,m_p,d_p)}}
            \hypo{\Phi_s \tr \seqi{\Del}{s}{\stype}{(b_s,m_s,d_s)}}
            \infer2[(\ruleConf)]{\seqi{\Gam_p + \Del}{(p,s)}{\conftype{\M}{\stype'}}{(b_p+b_s,m_p+m_s,d_p+d_s)}}
        \end{prooftree} \]
        Since $\Gam$ is tight, then $\Gam_p + \Del$ is tight. Moreover, $(p,s) \ra (p',q)$ implies that $\neg \isvalue{p}$. Then we can apply the \ih, and thus there exists a derivation for $(p',q)$ that must be of the following form:
        \[ \begin{prooftree}
            \hypo{\Phi_{p'} \tr \seqi{\Gam_{p'}}{p'}{\tcomptype{\stype''}{\M}{\stype'}}{(b_{p'},m_{p'},d_{p'})}}
            \hypo{\Phi_q \tr \seqi{\Del_q}{q}{\stype''}{(b_q,m_q,d_q)}}
            \infer2[(\ruleConf)]{\seqi{\Gam_{p'} + \Del_q}{(p',q)}{\conftype{\M}{\stype'}}{(b_{p'}+b_q,m_{p'}+m_q,d_{p'}+d_q)}}
        \end{prooftree} \]
        where $\Gam_{p'} + \Del_q = \Gam_p + \Del$ is tight,  and the counters are related properly. Let $\Phi_0$ be the following derivation:
        \[ \begin{prooftree}
          \hypo{\Phi_v \tr \seqi{\Gam_v}{v}{\M \ta \comptype{\stype'}{\ctype'}}{(b_v,m_v,d_v)}}
            \hypo{\Phi_{p'} \tr \seqi{\Gam_{p'}}{p'}{\tcomptype{\stype''}{\M}{\stype'}}{(b_{p'},m_{p'},d_{p'})}}
            \infer2[(\ruleApp)]{\seqi{\Gam_v+\Gam_{p'}}{vp'}{\comptype{\stype''}{\ctype'}}{(1+b_v+b_{p'},m_v+m_{p'},d_v+d_{p'})}}
        \end{prooftree} \]
        We can build $\Phi_{(u,q)}$ as follows:
        \[ \begin{prooftree}
            \hypo{\Phi_0}
            \hypo{\Phi_q \tr \seqi{\Del_q}{q}{\stype''}{(b_q,m_q,d_q)}}
            \infer2[(\ruleConf)]{\seqi{\Gam_v + \Gam_{p'} + \Del_q}{(vp', q)}{\ctype'}{(1+b_v+b_{p'}+b_q,m_v+m_{p'}+m_q,d_v+d_{p'}+d_q)}}
        \end{prooftree} \]
        where $\Gam_{p'} + \Gam_v + \Del_q = \Gam_v + \Gam_p + \Del = \Gam$, $b' = 1+b_v+b_{p'}+b_q$, $m' = m_v+m_{p'}+m_q$, and $d' = d_v+d_{p'}+d_q$. We can conclude since the counters are related properly according to the \ih.  
        \item Case $\Phi_p$ ends with (\ruleAppPOne) or (\ruleAppPTwo). These two cases are very similar to the previous case.
      \end{itemize}
      \item Case $(t,s) = (\get{l}{x}{p},s) \ra (p \subs{x}{v},s) = (u,q)$, where $s \equivstate \upd{l}{v}{s'}$. Let $\Phi_0$ be the following derivation:
      \[ \begin{prooftree}
        \hypo{\Phi_{p} \tr \seqi{\Gam_{p}}{p}{\comptype{\stype}{\ctype}}{(b_p,m_p,d_p)}}
        \infer1[(\ruleGet)]{\seqi{\Gam_{p} \sm x}{\get{l}{x}{p}}{\comptype{\conj{(l : \Gam_{p}(x))} \splus \stype}{\ctype}}{(b_p,1+m_p,d_p)}}
      \end{prooftree} \]
      $\Phi$ must be of the following form:
        \[ \begin{prooftree}
        \hypo{\Phi_0}
          \hypo{\Phi_{s} \tr \seqi{\Del}{s}{\conj{(l : \Gam_{p}(x))} \splus  \stype}{(b_s,m_s,d_s)}}
          \infer2[(\ruleConf)]{\seqi{(\Gam_{p} \sm x) + \Del}{(\get{l}{x}{p}, s)}{\kap}{(b_p+b_s,1+m_p+m_s,d_p+d_s)}}
        \end{prooftree} \] 
        where $\Gam = (\Gam_{p} \sm x) + \Del$ is tight, $b = b_p + b_s$, $m = 1+ m_p + m_s$, and  $d = d_p + d_s$. Since $\Phi_{s} \tr \seqi{\Del}{s}{\conj{(l : \Gam_{p}(x))} \splus \stype}{(b_s,m_s,d_s)}$, then~\cref{lem:split-values-stores}.\ref{lem:split-state} gives $s \equivstate \upd{l}{v_0}{s'_0}$, but we necessarily have $v_0 = v$ and $s'_0 = s'$. Moreover, the lemma also gives $\Phi_v \tr \seqi{\Del_v}{v}{\Gam_{p}(x) \sqcup \stype(l)}{(b_v,m_v,d_v)}$ and $\Phi_{s'} \tr \seqi{\Del_{s'}}{s'}{\stype'}{(b_{s'},m_{s'},d_{s'})}$, where $\conj{(l : \Gam_{p}(x))} \splus \stype = \conj{(l : \Gam_{p}(x) \sqcup \stype(l))};\stype'$, $\Del = \Del_v + \Del_{s'}$, $b_s=b_v + b_{s'}$, $m_s=m_v + m_{s'}$, and $d_s=d_v + d_{s'}$. Thus, by~\cref{lem:split-values-stores}.\ref{lem:com-split-values} there exist $\Phi^1_v \tr \seqi{\Del^1_v}{v}{\Gam_{p}(x)}{(b^1_v,m^1_v,d^1_v)}$ and $\Phi^2_v \tr \seqi{\Del^2_v}{v}{\stype(l)}{(b^2_v,m^2_v,d^2_v)}$, such that $\Del_v = \Del^1_v + \Del^2_v$, $b_v = b^1_v+b^2_v$, $m_v = m^1_v+m^2_v$, and $d_v = d^1_v+d^2_v$. From $\Phi_{p} \tr \seqi{\Gam_{p}}{p}{\comptype{\stype}{\ctype}}{(b_p,m_p,d_p)}$ and $\Phi^1_v \tr \seqi{\Del^1_v}{v}{\Gam_{p}(x)}{(b^1_v,m^1_v,d^1_v)}$, we obtain $\Phi_{p \subs{x}{v}} \tr \seqi{(\Gam_{p} \sm x) +\Del^1_v}{p\subs{x}{v}}{\comptype{\stype}{\ctype}}{(b_p+b^1_v,m_p+m^1_v,d_p+d^1_v)}$, by~\cref{lem:comp-subs-antisubs}.\ref{lem:comp-subs}. We now construct an alternative type derivation for $s$ of the form:
        \[ \begin{prooftree}
            \hypo{\Phi^2_v \tr  \seqi{\Del^2_v}{v}{\stype(l)}{(b^2_v,m^2_v,d^2_v)}}
            \hypo{\Phi_{s'} \tr \seqi{\Del_{s'}}{s'}{\stype'}{(b_{s'},m_{s'},d_{s'})}}
            \infer2[(\ruleUpd)]{\seqi{\Del^2_v+ \Del_{s'}}{\upd{l}{v}{s'}}{\conj{(l:\stype(l))};\stype'}{(b^2_v+b_{s'},m^2_v+m_{s'},d^2_v+d_{s'})}}
        \end{prooftree} \]
        Let $q = s= \upd{l}{v}{s'}$ and let $\Phi_q$ be this new derivation above. Notice also that $\stype = \conj{(l:\stype(l))}; \stype'$. Then we can construct $\Phi'$ as follows:
        \[ \begin{prooftree}
            \hypo{\Phi_{p\subs{x}{v}}}
            \hypo{\Phi_q}
            \infer2[(\ruleConf)]{\seqi{(\Gam_{p} \sm x) + \Del^1_v + \Del^2_v + \Del_{s'}}{(p \subs{x}{v}, s)}{\kap}{(b,m,d)}}
        \end{prooftree} \]
        Notice that the type environment of the conclusion is $(\Gam_{p} \sm x) + \Del^1_v + \Del^2_v + \Del_{s'} = (\Gam_{p} \sm x) + \Del_v + \Del_{s'} = (\Gam_{p} \sm x) + \Del = \Gam $, and the counters are as expected.
        \item Case $(t,s) = (\set{l}{v}{p},s) \ra (p, \upd{l}{v}{s}) = (u,q)$. Let $\Phi_0$ be the following derivation:
        \[ \begin{prooftree}
          \hypo{\Phi_{v} \tr \seqi{\Gam_v}{v}{\M}{(b_v,m_v,d_v)}}
            \hypo{\Phi_{p} \tr \seqi{\Gam_{p}}{p}{\comptype{\conj{(l : \M)}; \stype}{\ctype}}{(b_p,m_p,d_p)}}
            \infer2[(\ruleSet)]{\seqi{\Gam_v + \Gam_{p}}{\set{l}{v}{p}}{\comptype{\stype}{\ctype}}{(b_v+b_p,1+m_v+m_p,s_v+s_p)}}
        \end{prooftree} \]
        $\Phi$ must be of the following form:
        \[ \begin{prooftree}
          \hypo{\Phi_0}    
            \hypo{\Phi_{s} \tr \seqi{\Gam_{s}}{s}{\stype}{(b_s,m_s,d_s)}}
            \infer2[(\ruleConf)]{\seqi{(\Gam_v + \Gam_{p}) + \Gam_{s}}{(\set{l}{v}{p}, s)}{\kap}{(b_v+b_p+b_s,1+m_v+m_p+m_s,d_v+d_p+d_s)}}
        \end{prooftree} \]
        where  $\Gam = (\Gam_v + \Gam_{p}) + \Gam_{s}$ is tight, $b = b_v+b_p+b_s$, $m=1+m_v+m_p+m_s$ and $d=d_v+d_p+d_s$. Therefore, we can build $\Phi_{\upd{l}{v}{s}}$ as follows:
        \[ \begin{prooftree}
          \hypo{\Phi_{v} \tr \seqi{\Gam_v}{v}{\M}{(b_v,m_v,d_v)}}
          \hypo{\Phi_{s} \tr \seqi{\Gam_{s}}{s}{\stype}{(b_s,m_s,d_s)}}
          \infer2[(\ruleUpd)]{\seqi{\Gam_v + \Gam_{s}}{\upd{l}{v}{s}}{\conj{(l : \M)}; \stype}{(b_v+b_s,m_v+m_s,d_v+d_s)}}
        \end{prooftree} \]
        Assume And we can build $\Phi'$ as follows:
        \[ \begin{prooftree}
            \hypo{\Phi_{p} \tr \seqi{\Gam_{p}}{p}{\comptype{\conj{(l : \M)}; \stype}{\ctype}}{(b_p,m_p,d_p)}}
            \hypo{\Phi_{\upd{l}{v}{s}}}
            \infer2[(\ruleConf)]{\seqi{\Gam_{p} + (\Gam_v + \Gam_{s})}{(p, \upd{l}{v}{s})}{\kap}{(b_v+b_v+b_s,m_v+m_v+m_s,d_v+d_v+d_s)}}
        \end{prooftree} \]
        Notice that the type environment of the conclusion is $\Gam_{p} + (\Gam_v + \Gam_{s}) = \Gam$, and the counters are as expected.
    \end{itemize}
%\end{proof}

        \item %\begin{proof}
    We show a stronger statement of the form:

    Let $(t,s) \red[\gname] (u,q)$. If  $\Phi' \tr \seqi{\Gam}{(u,q)}{\ctype}{(b',m',d)}$, $\Gam$ is tight, and ($\ctype$ is tight or $\neg \isvalue{t}$), then $\Phi \tr \seqi{\Gam}{(t,s)}{\ctype}{(b,m,d)}$, where $\gname =\beta$ implies $b' = b - 1$ and $m' = m$, while $\gname \in \{\getname, \setname\}$ implies $b'=b$ and $m' = m - 1$.

    We proceed by induction on $(t, s) \red (u,q)$: 
    \begin{itemize}
        \item Case $(t,s) = ((\lam x.p) v,s) \redbeta (p \subs{x}{v}, s) = (u,q)$. Then $\Phi'$ must be of the following form:
        \[ \begin{prooftree}
            \hypo{\Phi_{p \subs{x}{v}} \tr \seqi{\Gam_{p \subs{x}{v}}}{p \subs{x}{v}}{\comptype{\stype}{\ctype}}{(b'',m'',d'')}}
            \hypo{\Phi_s \tr \seqi{\Gam_s}{s}{\stype}{(b_s,m_s,d_s)}}
            \infer2[(\ruleConf)]{\seqi{\Gam_{p \subs{x}{v}} + \Gam_s}{(p \subs{x}{v}, s)}{\ctype}{(b''+b_s,m''+m_s,d''+d_s)}}
        \end{prooftree} \]
        such that $\Gam = \Gam_{p \subs{x}{v}} + \Gam_s$, $b' = b''+b_s$, $m' = m''+m_s$, and $d' = d''+d_s$. By~\cref{lem:comp-subs-antisubs}.\ref{lem:comp-antisubs}, there exist $\Phi_p \tr \seqi{\Gam_p; x : \M}{p}{\comptype{\stype}{\ctype}}{(b_p,m_p,d_p)}$ and $\Phi_{v} \tr \seqi{\Gam_v}{v}{\M}{(b_v,m_v,d_v)}$, such that $\Gam_{p \subs{x}{v}} = \Gam_p + \Gam_v$, $b'' = b_p+b_v$, $m'' = m_p+m_v$, and $d'' = d_p + d_v$. We can build $\Phi$ as follows:
        \[ \begin{prooftree}
            \hypo{\Phi_p \tr \seqi{\Gam_p; x : \M}{p}{\comptype{\stype}{\ctype}}{(b_p,m_p,d_p)}}
            \infer1[(\ruleLam)]{\seqi{\Gam_p}{\lam x.p}{\M \ta (\comptype{\stype}{\ctype})}{(b_p,m_p,d_p)}}
            \hypo{\Phi_{v} \tr \seqi{\Gam_v}{v}{\M}{(b_v,m_v,d_v)}}
            \infer1[(\ruleLift)]{\seqi{\Gam_v}{v}{\tcomptype{\stype}{\M}{\stype}}{(b_v,m_v,d_v)}}
            \infer2[(\ruleApp)]{\seqi{\Gam_p + \Gam_v}{(\lam x.p)v}{\comptype{\stype}{\ctype}}{(1+b_p+b_v,m_p+m_v,d_p+d_v)}}
            \hypo{\Phi_s}
            \infer2[(\ruleConf)]{\seqi{(\Gam_p + \Gam_v) + \Gam_s}{((\lam x.t')v, s)}{\ctype}{(1+b_p+b_v+b_s,m_p+m_v+m_s,d_p+d_v+d_s)}}
        \end{prooftree} \]
        such that $b = 1+b_p+b_v+b_s$, $m = m_p+m_v+m_s$, and $d = d_p+d_v+d_s$. And we can conclude with $\Gam = \Gam_{p \subs{x}{v}} + \Gam_s = (\Gam_p + \Gam_v) + \Gam_s$, $b' = b'' + b_s = b_p + b_v + b_s = (1 + b_p + b_v + b_s) - 1 = b - 1$, $m' = m'' + m_s = (m_p + m_v) + m_s = m$, and $d' = d'' + d_s = (d_p + d_v) + d_s = d$.
        \item Case $(t,s) = (vp,s) \ra (vp',q) = (u,q)$, such that $(p,s) \ra (p',q)$. Then we have three cases for the type derivation $\Phi_{p'}$ of $p'$ inside $\Phi'$:
        \begin{itemize}
            \item Case $\Phi_{vp'}$ ends with ($\ruleApp$). Let $\Phi_0$ be the following derivation:
            \[ \begin{prooftree}
                \hypo{\Phi_{v} \tr \seqi{\Gam_v}{v}{\M \ta \comptype{\stype'}{\ctype}}{(b_v,m_v,d_v)}}
                    \hypo{\Phi_{p'} \tr \seqi{\Gam_{p'}}{p'}{\tcomptype{\stype}{\M}{\stype'}}{(b'',m'',d'')}}
                    \infer2[(\ruleApp)]{\seqi{\Gam_v + \Gam_{p'}}{v p'}{\comptype{\stype}{\ctype}}{(1+b_v+b'', m_v+m'', d_v+d'')}}
            \end{prooftree} \]
            $\Phi'$ must be of the following form: 
            \[ \begin{prooftree}
                \hypo{\Phi_0}
                \hypo{\Phi_q \tr \seqi{\Gam_q}{q}{\stype}{(b_q,m_q,d_q)}}
                \infer2[(\ruleConf)]{\seqi{(\Gam_v + \Gam_{p'}) + \Gam_q}{(v p', q)}{\ctype}{(1+b_v+b''+b_q,m_v+m''+m_q,d_v+d''+d_q)}}
            \end{prooftree} \]
            such that $\Gam = (\Gam_v + \Gam_{p'}) + \Gam_q$ tight, $b' = 1+b_v+b''+b_q$, $m' = m_v+m''+m_q$, and $d' = d_v+d''+d_q$. So we can build $\Phi_{(p',q)}$ as follows:
            \[ \begin{prooftree}
                \hypo{\Phi_{p'} \tr \seqi{\Gam_{p'}}{p'}{\tcomptype{\stype}{\M}{\stype'}}{(b'',m'',d'')}}
                \hypo{\Phi_q \tr \seqi{\Gam_q}{q}{\stype}{(b_q,m_q,d_q)}}
                \infer2[(\ruleConf)]{\seqi{\Gam_{p'} + \Gam_q}{(p', q)}{\conftype{\M}{\stype'}}{(b''+b_q,m''+m_q,d''+d_q)}}
            \end{prooftree} \]
            Since $\Gam$ is tight, then $\Gam_{p'} + \Gam_q$ is tight. Moreover, $(p, s) \red (p',q)$ implies $\neg\isvalue{p}$. Then we can apply the \ih, and thus there exists a derivation for $(p,s)$ that must be of the following form:
            \[ \begin{prooftree}
                \hypo{\Phi_p \tr \seqi{\Gam_p}{p}{\tcomptype{\stype''}{\M}{\stype'}}{(b_p,m_p,d_p)}}
                \hypo{\Phi_s \tr \seqi{\Gam_s}{s}{\stype''}{(b_s,m_s,d_s)}}
                \infer2[(\ruleConf)]{\seqi{\Gam_p + \Gam_s}{(p, s)}{\conftype{\M}{\stype'}}{(b_p+b_s,m_p+m_s,d_p+d_s)}}
            \end{prooftree} \]
            where $\Gam_p + \Gam_s = \Gam_{p'} + \Gam_q$ is tight, and either (1) $b''+b_q = b_p+b_s-1$, $m''+m_q=m_p+m_s$, and $d''+d_q = d_p+d_s$, or (2) $b''+b_q = b_p+b_s$, $m''+m_q=m_p+m_s-1$, and $d''+d_q = d_p+d_s$. So, we can build $\Phi$ as follows:
            \[ \begin{prooftree}
                \hypo{\Phi_{v} \tr \seqi{\Gam_v}{v}{\M \ta (\comptype{\stype'}{\ctype})}{(b_v,m_v,d_v)}}
                \hypo{\Phi_p \tr \seqi{\Gam_p}{p}{\tcomptype{\stype''}{\M}{\stype'}}{(b_p,m_p,d_p)}}
                \infer2[(\ruleApp)]{\seqi{\Gam_v + \Gam_p}{vp}{\comptype{\stype''}{\ctype}}{(1+b_v+b_p,m_v+m_p,d_v+d_p)}}
                \hypo{\Phi_s}
                \infer2[(\ruleConf)]{\seqi{(\Gam_v + \Gam_p) + \Gam_s}{(vp, s)}{\kappa}{(1 + b_v + b_p+b_s,m_v+m_p+m_s,d_v+d_p+d_s)}}
            \end{prooftree} \]
            where $\Gam_v + \Gam_p + \Gam_s = \Gam_v + \Gam_{p'} + \Gam_q = \Gam$, $b = 1+b_v+b_p+b_s$, $m = m_v+m_p+m_s$, and $d = d_v+d_p+d_s$. We can conclude since:
            \begin{itemize}
                \item Case (1): $b' = 1 + b_v + b'' + b_q = 1 + b_v + b_p + b_s - 1 = b -1$, and the other counters are easy to check;
                \item Case (2): $m' = m_v + m'' + m_q = m_v + m_p + m_s - 1 = m - 1$, and the other counters are easy to check.
            \end{itemize}
            \item Case $\Phi_{vp'}$ ends with (\ruleAppPOne) or (\ruleAppPTwo). These two cases are very similar to the previous case.
        \end{itemize}
        \item Case $(t,s) = (\get{l}{x}{p},s) \ra (p \subs{x}{v},s) = (u,q)$, such that $s \equivstate \upd{l}{v}{s'}$. Let $\Phi_0$ be the following derivation:
        \[ \begin{prooftree}
            \hypo{\Phi^2_v \tr \seqi{\Gam^2_v}{v}{\M_2}{(b^2_v,m^2_v,d^2_v)}}
            \hypo{\Phi_{s'} \tr \seqi{\Gam_{s'}}{s'}{\stype}{(b_{s'},m_{s'},d_{s'})}}
            \infer2[(\ruleUpd)]{\seqi{\Gam^2_v + \Gam_{s'}}{\upd{l}{v}{s'}}{\conj{(l : \M_2)}; \stype}{(b^2_v+b_{s'},m^2_v+m_{s'},d^2_v+d_{s'})}}
        \end{prooftree} \]
        Then $\Phi'$ must be of the following form:
        \[ \begin{prooftree}
            \hypo{\Phi_{p \subs{x}{v}} \tr \seqi{\Gam_{p \subs{x}{v}}}{p \subs{x}{v}}{\comptype{\conj{(l : \M)}; \stype}{\ctype}}{(b'',m'',d'')}}
            \hypo{\Phi_0}
            \infer2[(\ruleConf)]{\seqi{\Gam_{p \subs{x}{v}} + (\Gam^2_v + \Gam_{s'})}{(p \subs{x}{v}, \upd{l}{v}{s'})}{\ctype}{(b''+b^2_v+b_{s'},m''+m^2_v+m_{s'},d''+d^2_v+d_{s'})}}
        \end{prooftree} \]
        such that $\Gam = \Gam_{p \subs{x}{v}} + (\Gam^2_v + \Gam_{s'})$, $b' = b'' + b^2_v + b_{s'}$, $m' = m'' + b^2_v + b_{s'}$, and $d' = d'' +d^2_v+d_{s'}$. By~\cref{lem:comp-subs-antisubs}.\ref{lem:comp-antisubs}, there exist $\Phi_p \tr \seqi{\Gam_p; x : \M_1}{p}{\comptype{\conj{(l : \M_2)}; \stype}{\ctype}}{(b_p,m_p,d_p)}$ and $\Phi^1_v \tr \seqi{\Gam^1_v}{v}{\M_1}{(b^1_v,m^1_v,d^1_v)}$, such that $\Gam_{p \subs{x}{v}} = \Gam_p + \Gam^1_v$, $b'' = b_p + b^1_v$, $m'' = m_p + m^1_v$, and $d'' = d_p + d^1_v$. Therefore, we can build $\Phi_{\get{l}{x}{p}}$ as follows:
        \[ \begin{prooftree}
            \hypo{\Phi_p \tr \seqi{\Gam_p; x : \M_1}{p}{\comptype{\conj{(l : \M_2)}; \stype}{\ctype}}{(b_p,m_p,d_p)}}
            \infer1[(\ruleGet)]{\seqi{\Gam_p}{\get{l}{x}{p}}{\comptype{\conj{(l : \M_1 \sqcup \M_2)}; \stype}{\ctype}}{(b_p,1+m_p,d_p)}}
        \end{prooftree} \]
        By~\cref{lem:comp-merge-values}, we have $\Phi_v \tr \seqi{\Gam^1_v + \Gam^2_v}{v}{\M_1 \sqcup \M_2}{(b^1_v+b^2_v,m^1_v+m^2_v,d^1_v+d^2_v)}$. Thus, we can build $\Phi_{\upd{l}{v}{s'}}$ as follows:
        \[ \begin{prooftree}
            \hypo{\Phi_v \tr \seqi{\Gam^1_v + \Gam^2_v}{v}{\M_1 \sqcup \M_2}{(b^1_v+b^2_v,m^1_v+m^2_v,d^1_v+d^2_v)}}
            \hypo{\Phi_{s'} \tr \seqi{\Gam_{s'}}{s'}{\stype}{(b_{s'},m_{s'},d_{s'})}}
            \infer2[(\ruleUpd)]{\seqi{(\Gam^1_v + \Gam^2_v) + \Gam_{s'}}{\upd{l}{v}{s'}}{\conj{(l : \M_1 \sqcup \M_2)}; \stype}{(b^1_v+b^2_v+b_{s'},m^1_v+m^2_v+m_{s'},d^1_v+d^2_v+d_{s'})}}
        \end{prooftree} \]
        Finally, we can build $\Phi$ as follows:
        \[ \begin{prooftree}
            \hypo{\Phi_{\get{l}{x}{p}}}
            \hypo{\Phi_{\upd{l}{v}{s'}}}
            \infer2[(\ruleConf)]{\seqi{\Gam_p + (\Gam^1_v + \Gam^2_v) + \Gam_{s'}}{(\get{l}{x}{p}, \upd{l}{v}{s'})}{\ctype}{(b_p+b^1_v+b^2_v+b_{s'},1+m_p+m^1_v+m^2_v+m_{s'},d_p+d^1_v+d^2_v+d_{s'})}}
        \end{prooftree} \]
        such that $b = b_p+b^1_v+b^2_v+b_{s'}$, $m = 1+m_p+m^1_v+m^2_v+m_{s'}$, and $d = d_p+d^1_v+d^2_v+d_{s'}$. And we can conclude with $\Gam = \Gam_{p \subs{x}{v}} + (\Gam^2_v + \Gam_{s'}) = \Gam_p + \Gam^1_v + \Gam^2_v + \Gam_{s'}$, $b' = b'' + b^2_v + b_{s'} = b_p + b^1_v + b^2_v + b_{s'} = b$, and $m' = m'' + m^2_v + m_{s'} = m_p + m^1_v + m^2_v + m_{s'} = (1 + m_p + m^1_v + m^2_v + m_{s'}) - 1 = m - 1$, $d' = d'' + d^2_v + d_{s'} = d_p + d^1_v + d^2_v + d_{s'} = d$.
        \item Case $(t,s) = (\set{l}{v}{p},s) \ra (p, \upd{l}{v}{s}) = (u,q)$. Let $\Phi_0$ be the following derivation:
        \[ \begin{prooftree}
            \hypo{\Phi_{v} \tr \seqi{\Gam_v}{v}{\M}{(b_v,m_v,d_v)}}
            \hypo{\Phi_{s} \tr \seqi{\Gam_s}{s}{\stype}{(b_s,m_s,d_s)}}
            \infer2[(\ruleUpd)]{\seqi{\Gam_v + \Gam_s}{\upd{l}{v}{s}}{\conj{(l : \M)}; \stype}{(b_v+b_s,m_v+m_s,d_v+d_s)}}
        \end{prooftree} \]
        $\Phi'$ must be of the following form:
        \[ \begin{prooftree}
            \hypo{\Phi_p \tr \seqi{\Gam_p}{p}{\comptype{\conj{(l : \M)}; \stype}{\ctype}}{(b_p,m_p,d_p)}}
            \hypo{\Phi_0}
            \infer2[(\ruleConf)]{\seqi{\Gam_p + (\Gam_v + \Gam_s)}{(p, \upd{l}{v}{s})}{\kap}{(b_p+b_v+b_s,m_p+m_v+m_s,d_p+d_v+d_s)}}
        \end{prooftree} \]
        such that $\Gam = \Gam_p + (\Gam_v + \Gam_s)$, $b' = b_p + b_v + b_s$, $m' = m_p + m_v + m_s$, and $d' = d_p + d_v + d_s$. Therefore, we can build $\Phi$ as follows:
        \[ \begin{prooftree}
            \hypo{\Phi_{v} \tr \seqi{\Gam_v}{v}{\M}{(b_v,m_v,d_v,)}}
            \hypo{\Phi_p \tr \seqi{\Gam_p}{p}{\comptype{\conj{(l : \M)}; \stype}{\ctype}}{(b_p,m_p,d_p)}}
            \infer2[(\ruleSet)]{\seqi{\Gam_v + \Gam_p}{\set{l}{v}{p}}{\comptype{\stype}{\ctype}}{(b_v+b_p,1+m_v+m_p,d_v+d_p)}}
            \hypo{\Phi_s}
            \infer2[(\ruleConf)]{\seqi{(\Gam_v + \Gam_p) + \Gam_s}{(\set{l}{v}{p}, s)}{\ctype}{(b_v+b_p+b_s,1+m_v+m_p+m_s,d_v+d_p+d_s)}}
        \end{prooftree} \]
        Notice that the type environment of the conclusion is $(\Gam_v + \Gam_p) + \Gam_s = \Gam$, and the counters are as expected.
    \end{itemize}
%\end{proof}
    \end{enumerate}
\end{proof}}

\compsoundness*

\maybehide{\begin{proof} \mbox{}
    \begin{enumerate}
        \item %\begin{proof}
    The proof follows by induction over $b+m$:
    \begin{itemize}
        \item Case $b+m = 0$. Then $b=m=0$, therefore $t \in \normal$, by point (1) of~\cref{lem:zero-counters}, and  $d = \size{t}$,  by point (2) of~\cref{lem:zero-counters}. Let $u = t$ and $q=s$, then  we can conclude since $\size{(u,q)} = \size{u} =\size{t} = d$.
        \item Case $b+m > 0$. Then $b>0$ or $m>0$, and in either case $t \not\in \normal$,  by~\cref{lem:zero-nfs}. Note that $(t,s)$ is not final because $t$ is unblocked by~\cref{prop:typed-unblock}. Therefore, by~\cref{prop:normal-iff-final} there exists $(t',s')$ such that $(t,s) \gsred (t',s')$. By~\cref{lem-exact-red-exp}.\ref{lem:subj-comp-red}, there exists $\Phi' \tr \seqi{\Gam}{(t',s')}{\ctype}{(b',m',d)}$, such that $b'+m'=b+m-1$. By the \ih, there exists $(u,q)$, such that $u\in \normal$, $(t',s') \gsrred^{(b',m')} (u,q)$ and $d = \size{(u,q)}$. So we can conclude with $(t,s) \gsred (t',s') \gsrred^{(b',m')} (u,q)$, which means that $(t,s) \drred^{(b,m)} (u,q)$, as expected.
    \end{itemize}
%\end{proof}

        \item %\begin{proof}
    By induction over $b + m$: \begin{itemize}
        \item Case $b + m = 0$. Then $b = m = 0$ and $(t,s) = (u,q)$. We can conclude by~\cref{lem:typestatesnfs}.\ref{lem:typ-states} and~\cref{lem:typestatesnfs}.\ref{lem:comp-typ-nfs}.
        \item Case $b + m > 0$. Then there exists $(t',s')$, such that $(t,s) \ra^{(1,0)} (t',s') \rra^{(b-1,m)} (u,q)$ or $(t,s) \ra^{(0,1)} (t',s') \rra^{(b,m-1)} (u,q)$. By the \ih, there exists $\Phi' \tr \seqi{\Gam}{(t',s')}{\kap}{(b',m',\size{(u,q)})}$ tight, such that $b' + m' = b + m - 1$. By~\cref{lem-exact-red-exp}.\ref{lem:comp-subj-exp}, we have $\Phi \tr \seqi{\Gam}{(t,s)}{\kap}{(b'',m'',\size{(u,q)})}$ tight, such that $b'' + m'' = 1+ b' + m'$. Therefore, $b'' + m'' = b + m$, since the fact that $b'' = b$, and $m'' = m$ can be easily checked by a simple case analysis.
    \end{itemize}
%\end{proof}
    \end{enumerate}
\end{proof}}


\section{Derivation of dynamics from generic small initialization}
\label{app:derivation}

Here we give an informal derivation of Result \ref{res:small_init}, which states that the solution of Equation \ref{eqn:dW_dt_expanded} from arbitrary initialization with scale $\alpha \ll 1$ closely matches the analytical solution from a certain spectrally aligned initialization.
Recall that our ingredients are the following:
\begin{itemize}
\item $\gamma_1, ..., \gamma_d$ are unique and positive.
\item $\tilde{\mW}_0 \in \R^{d \times m}$ with $\tilde{\mW}_0 \mGamma^{(\le d)}$ full rank.
\item $\mW(t)$ is the true solution with $\mW(0) = \alpha \tilde\mW_0$.
\item $\mW^*(t)$ is the spectrally aligned solution with $\mW^*(0) = \mathcal{A}(\alpha \tilde\mW_0)$ whose dynamics are given exactly by Proposition \ref{prop:exact_dynamics}.
\end{itemize}
We will show that, for sufficiently small $\alpha$, the true and aligned solutions remain arbitrarily close.

We will find it convenient to parameterize $\mW(t)$ as
\begin{align} \label{eqn:generic_W_fac}
    \mW(t) &=
    \mU
    \begin{bmatrix} s_1(t)  & a_{1 2 }(t)  & a_{1 3 }(t)  & \cdots  & a _{1 d }(t)  & \cdots  & a _{1 m }(t)  \\ a_{21}(t)  & s_2(t)  & a _{2 3 }(t)  & \cdots  & a _{2 d }(t)  & \cdots  & a _{2 m }(t)  \\ a_{31}(t)  & a_{32}(t)  & s_3(t)  & \cdots  & a _{3 d }(t)  & \cdots  & a _{3 m }(t)  \\ \vdots  & \vdots  & \vdots  & {\ddots}  & \vdots  &   & \vdots  \\ a_{d1}(t)  & a_{d2}(t) & a_{d3}(t) & \cdots  & s_d(t)  & \cdots  & a_{dm}(t)  \\  \end{bmatrix}
    \mGamma^{(\le m)} \\
    &= \mU \mA(t) \mGamma^{(\le m)},
\end{align}
where $s_j(0) > 0$ and $a_{jk}(0) = 0$ for all $j > k$.
We use the special notation $s_j$ for the diagonal elements of $\mA$ to foreshadow that these will act as the effective singular values of the dynamics.
Note that the spectrally aligned initialization $\mathcal{A}(\tilde\mW_0)$ is precisely the $\mW(0)$ of Equation \ref{eqn:generic_W_fac} but with all off-diagonal entries of $\mA(0)$ zeroed out.
Our strategy will be to show that no $a_{jk}(t)$ ever grows sufficiently large to affect the dynamics to leading order, and thus $\mW(t)$ and $\mW^*(t)$ remain close.

We will make use of big-$\O$ notation to describe the scaling of certain values with $\alpha$.
Eigenvalues $\gamma_j$ and differences $\gamma_j - \gamma_{j+1}$ will be treated as constants.
Note that, because $\mW(0) = \alpha \tilde\mW_0$, all elements of $\mA(0)$ are $\Theta(\alpha)$ if they are not zero.

\textit{Diagonalization of dynamics.}
Define $\mLambda = \text{diag}(\gamma_1, ..., \gamma_m)$.
The dynamics of Equation \ref{eqn:dW_dt_expanded} state that $\mA(t)$ evolves as
\begin{equation} \label{eqn:dA_dt_exact}
    \frac{d\mA(t)}{dt} = \left( \mI - \mA(t) \mLambda \mA\T(t) \right) \mA(t) \mLambda,
\end{equation}
where we have reparameterized $t \rightarrow t/4$ to absorb the superfluous prefactor of $4$.

\textit{Approximate solution to dynamics.}
So long as all $a_{jk}$ remain small (i.e. $o(1)$), then these dynamics are given by
\begin{equation} \label{eqn:dA_dt_approx}
    \frac{d\mA(t)}{dt} \approx \left( \mI - \fourdiag{\gamma_1 s_1^2(t)}{\gamma_2 s_2^2(t)}{\gamma_3 s_3^2(t)}{\gamma_d s_d^2(t)} \right) \mA(t) \mLambda.
\end{equation}
We will show that all $a_{jk}$ indeed remain small under the evolution of Equation \ref{eqn:dA_dt_approx}, and so Equation \ref{eqn:dA_dt_approx} remains valid.

Solving Equation \ref{eqn:dA_dt_approx} yields
\begin{align}
    \label{eqn:s_j_ideal}
    s_j(t) &= \frac{e^{\gamma_j t}}{\sqrt{s_j^{-2}(0) + (e^{2 \gamma_j t} - 1) \gamma_j}} \\
    \label{eqn:a_jk_ideal}
    a_{jk}(t) &= 
    \left\{\begin{array}{ll}
        a_{jk}(0)
        \left( \frac{s_j(t)}{s_j(0)} \right)^{\gamma_k / \gamma_j}
        = \O\left( \alpha^{1 - \gamma_k / \gamma_j} \right)
        & \text{for } j < k, \\
        0 & \text{for } j > k.
        \end{array}\right.
\end{align}
As discussed in the main text, each $s_j(t)$ remains small up to a time $\tau_j \sim - \gamma_j^{-1} \log{\alpha}$, at which it quickly grows until $\gamma_j s_j^2(t) = 1$ and saturates.
Entries of $\mA(t)$ below the diagonal remain zero.
Entries of $\mA(t)$ above the diagonal exhibit interesting dynamics: entry $a_{jk}(t)$ with $j < k$ grows exponentially at rate $\gamma_k$, but its growth is curtailed by the saturation of $s_j(t)$ before it has time to reach order one.
This is because each $s_j(t)$ grows faster than all $a_{jk}(t)$ in its row.

All $a_{jk}(t)$ thus remain $o(1)$ and all $s_j(t)$ closely follow the ideal solution of Equation \ref{eqn:s_j(t)}, and so $\norm{\mW(t) - \mW^*(t)}_F$ remains $o(1)$.
This concludes the derivation.

\textit{Numerical validation of approximation.}
While the numerical experiment presented in Figure \ref{fig:banner_figure} validates our claim regarding the trajectory of $\mW(t)$ from generic small initialization closely matching theoretical predictions from aligned initialization, here we go a step further and show agreement for individual elements of $\mA(t)$.
We analytically solve Equation \ref{eqn:dA_dt_approx} for $d = 3$ and $m = 5$ with $\gamma_j = 2^{-j}$, starting from a random upper-triangular $\mA(0)$ of scale $\alpha = 10^{-9}$.
The results, plotted in Figure \ref{fig:A_jk_traces}, closely match the unapproximated dynamics of Equations \ref{eqn:s_j_ideal} and \ref{eqn:a_jk_ideal}.

\begin{figure}
  \centering
  \includegraphics[width=11cm]{img/small_init_weight_traces.pdf}
  \vspace{-3mm}
  \caption{
  True $|s_j(t)|$ and $|a_{jk}(t)|$ compared with theoretical predictions from small init.
  Red traces show $s_j(t)$, blue traces show $a_{jk}(t)$ with $j < k$, and green traces show $a_{jk}$ with $j > k$.
  While there are no theoretical traces for $a_{jk}$ with $j > k$, these elements do remain small as predicted.
  }
  \label{fig:A_jk_traces}
\end{figure}

We present in section~\ref{ssec:faces} an application of PnP-HVAE on face images, using a pretrained state-of-the-art hierarchical VAE. 
Next, we study the application of our framework to natural images. To that end, we introduce  in section~\ref{ssec:patchVDVAE}  a patch hierachical VAE architecture, that is able to model natural images of different resolutions. In section~\ref{ssec:app_nat}, we provide deblurring, super-resolution and inpainting experiments to demonstrate the relevance of the proposed method.

Additional results are presented in Appendix~\ref{app:add}. All experiments can be reproduced using the code available at \url{https://github.com/jprost76/PnP-HVAE}.



\subsection{Face Image restoration (FFHQ)}\label{ssec:faces}
We first demonstrate the effectiveness of PnP-HVAE on highly structured data, by performing face image restoration.
Latent variable generative models can accurately model structured images such as face images \cite{karras2019style,vahdat2020nvae,child2021very,kingma2018glow}, and then be used to produce high quality restoration of such data. 
In our experiments, we use the VDVAE model of~\cite{child2021very}, pre-trained on the FFHQ dataset~\cite{karras2019style}, as our hierarchical VAE prior.
VDVAE has $L=66$ latent variable groups in its hierarchy and generates images at resolution $256\times256$.

We compare PnP-HVAE with the intermediate layer optimization algorithm (ILO)~\cite{daras2021intermediate} that is based on a different class of generative models than HVAE. ILO is a GAN inversion method which optimizes the image latent code along with the intermediate layer representation of a StyleGAN to generate an image consistent with a degraded observation.
We use the official implementation of ILO, along with a StyleGAN2 model~\cite{karras2020analyzing, stylegan2pytorch}, that was trained for 550k iterations on images of resolution $256\times256$ from FFHQ.  
As VDVAE and StyleGAN models are not trained on the same train-test split of FFHQ, we chose to evaluate the methods on a subset of 100 images from the CelebA dataset~\cite{liu2018large}. 
For super-resolution, the degradation model corresponds to the application of a gaussian low-pass filter followed by a $\times 4$ sub-sampling, and the addition of a gaussian white noise with $\sigma=3$.
For the deblurring, we considered motion blur and  gaussian kernels, both with a noise level $\sigma=8$. %

We provide quantitative comparisons in table~\ref{table:comp_ILO}, along with a visual comparison of the results in figure~\ref{fig:face_restoration}.
PnP-HVAE has the best  PSNR and SSIM results for all the considered restoration tasks, while ILO provides better results  for the perceptual distance.
By jointly optimizing the image and its latent variable, PnP-HVAE provides  results that are both realistic and consistent with the degraded observation.
On the other hand,  ILO  only optimizes on an extended latent space. This method generates  sharp and realistic images with better LPIPS scores,   
but the results lack  of consistency with respect to the observation, which explains the overall lower PSNR performance. 






\subsection{PatchVDVAE: a HVAE for natural images}\label{ssec:patchVDVAE}
Available generative models in the literature operate on images of  fixed resolutions and
are either restrained to datasets of limited diversity, or even to registered face images~\cite{kingma2018glow,child2021very, vahdat2020nvae, karras2019style}, or requiring additional class information~\cite{brock2018large, dhariwal2021diffusion, song2020score, luhman2022optimizing}.
Fitting an unconditional model on natural images appears to be a more difficult task, as their resolution can change, and their content is highly diverse.
The complexity of the problem can be reduced by learning a prior model on patches of reduced dimension. 
For image restoration problems, the patch model can be reused on images of higher dimensions~\cite{zoran2011learning,prost2021learning,altekruger2022patchnr}. When the model is a full CNN, the prior on the set of the  patches can  be computed efficiently by applying the network on the full image~\cite{prost2021learning}.

We thus introduce  patchVDVAE, a fully convolutional hierarchical VAE.
Contrary to existing HVAE models whose resolution is constrained by the constant tensor at the input of the top-down block, patchVDVAE can generate images of different resolutions by controlling the dimension of the input latent. 
This amounts to defining a prior on patches whose dimension corresponds to the receptive field of the VAE. A similar model is used for image denoising in~\cite{prakash2021interpretable}.

 
For PatchVDVAE architecture, we use the same bottom-up and top-down blocks as VDVAE~\cite{child2021very}, and replace the constant trainable input in the first top-down block by a latent variable, to make the model fully convolutional (details on the  architecture are given in Appendix~\ref{app:details}). 
The training dataset is composed of $128\times 128$ patches extracted from a combination of DIV2K~\cite{agustsson2017ntire} and Flickr2K~\cite{Lim_2017_CVPR_workshops} datasets.
We perform data augmentation by extracting  patches at $3$ resolutions: HR-images and $\times 2$ and $\times 4$ downscaled images. 
The model is trained for $7.10^5$ iterations with a batch size of $64$. Following the recommendation of~\cite{hazami2022efficient}, we use Adamax optimizer with an exponential moving average and gradient smoothing of the variance.
We set the decoder model to be a gaussian with diagonal covariance, as in~\cite{luhman2022optimizing}.
PatchVDVAE is fully convolutional and can generate images of dimension that are multiples of $64$ as illustrated by
figure~\ref{fig:vdvae}.

\newlength{\patchwidth}
\setlength{\patchwidth}{0.135\columnwidth}
\begin{figure}[!ht]
    \centering
    \begin{subfigure}[t]{.34\columnwidth}\hspace{0.1cm}
        \setlength{\tabcolsep}{0.02pt}
\renewcommand{\arraystretch}{0}
        \begin{tabular}{*{2}{p{1.03\patchwidth}}}
            \includegraphics[width=\patchwidth]{figures_arxiv/patchVDVAE/samples/generated/64x64/setup-5-image-0018.png} &
            \includegraphics[width=\patchwidth]{figures_arxiv/patchVDVAE/samples/generated/64x64/setup-5-image-0016.png} \\
            \includegraphics[width=\patchwidth]{figures_arxiv/patchVDVAE/samples/generated/64x64/setup-5-image-0008.png} &
            \includegraphics[width=\patchwidth]{figures_arxiv/patchVDVAE/samples/generated/64x64/setup-5-image-0019.png}   
        \end{tabular}
    \end{subfigure}\hspace{-0.15cm}
    \begin{subfigure}[t]{.64\columnwidth}
\begin{tabular}{cc}\vspace{-0.1cm}
\includegraphics[width=2\patchwidth]{figures_arxiv/patchVDVAE/samples/generated/256x256/setup-2-image-0009.png}&
        \includegraphics[width=2\patchwidth]{figures_arxiv/patchVDVAE/samples/generated/256x256/setup-2-image-0002.png}\end{tabular}

    \end{subfigure}
    \caption{\label{fig:vdvae} Left: $64\times64$ patches samples from our patchVDVAE model trained on patches from natural images.
    Right: PatchVDVAE is fully convolutional and it can generate images of higher resolution (here: $128\times128$).\vspace{-0.2cm}}
\end{figure}

\subsection{Natural images restoration}\label{ssec:app_nat}
We  evaluate PnP-HVAE on natural image restoration.
For each task, we report the average value of the PSNR, the SSIM, and the LPIPS metrics on $20$ images from the test set of the BSD dataset~\cite{MartinFTM01}.\\


\noindent
{\bf Image deblurring.}
In the experiments, we consider $2$ gaussian kernels and $2$ motion blur kernels from~\cite{levin2009understanding}, with $3$ different noise levels 
$\sigma \in \{2.55, 7.65, 12.75\}$.
As a baseline we consider  EPLL~\cite{zoran2011learning}, which learns a prior on image patches with a gaussian mixture model.
We also compare PnP-HVAE  with PnP-MMO and GS-PnP, $2$ competing convergent Plug-and-Play methods based on CNN denoisers.
PnP-MMO~\cite{pesquet2021learning} restricts the denoiser to be contraction in order to guarantee the convergence of the PnP forward-backard algorithm. GS-PnP~\cite{hurault2022gradient} considers a gradient step denoiser and reaches state-of-the-art performances of non converging methods~\cite{zhang2021plug}.
We set the temperature $\tau$  in our method as $0.95$, $0.8$ and $0.6$ for noise levels $2.55$, $7.65$ and $12.75$ respectively, and we let it run for a maximum of $50$ iterations. 
For the three compared methods we use the official implementations and pre-trained models provided by the respective authors. 
Details on the choice of hyperparameters for the concurrent methods are provided in the Appendix~\ref{app:details}
Figure~\ref{fig:deblurring_bsd} illustrates that our method provides correct deblurring results. 

According to table~\ref{tab:deb}, the performance of PnP-HVAE is between those of EPLL and GS-PnP and it outperforms PnP-MMO for large noise levels.\\

\begin{table}
\begin{center}\footnotesize
    \begin{tabular}{>{\centering}m{.3cm}*{5}{c}}
    $\sigma$ &Method & PSNR$\uparrow$ & SSIM$\uparrow$ & LPIPS$\downarrow$  \\ 
    \hline
    \multirow{4}{*}{\vcell{$2.55$}}
    & PnP-HVAE & $27.75$ & $0.79$ & $0.31$\\
    & GS-PNP \cite{hurault2022gradient} & $\mathbf{29.59}$ & $\mathbf{0.84}$ & $\mathbf{0.22}$\\
    & EPLL \cite{zoran2011learning} & $26.49$ & $0.71$ & $0.36$\\ 
    & PnP-MMO \cite{pesquet2021learning} & $\underbar{29.50}$ & $\underbar{0.83}$ & $\underbar{0.20}$ \\ \hline
    \multirow{4}{*}{\vcell{$7.65$}}
    & PnP-HVAE & $\underbar{26.36}$ & $\underbar{0.72}$ & $\underbar{0.40}$\\
    & GS-PNP \cite{hurault2022gradient} & $\mathbf{27.33}$ & $\mathbf{0.77}$ & $\mathbf{0.31}$\\
    & EPLL \cite{zoran2011learning} & $24.04$ & $0.66$ & $0.45$ \\ 
    & PnP-MMO \cite{pesquet2021learning} & $25.34$ & $0.69$ & $0.34$\\
    \hline
    \multirow{4}{*}{\vcell{$12.75$}}
    & PnP-HVAE & $\underbar{25.12}$ & $\mathbf{0.73}$ & $\underbar{0.47}$\\
    & GS-PNP \cite{hurault2022gradient} & $\mathbf{26.32}$ & $\mathbf{0.73}$ & $\mathbf{0.37}$\\
    & EPLL \cite{zoran2011learning} & $23.28$ & $0.61$ & $0.51$ \\ 
    & PnP-MMO \cite{pesquet2021learning} & $22.42$ & $0.53$& $0.54$ \\
    \hline
    &\vspace*{-.3cm}\\
            \multicolumn{2}{c}{Blur and motion kernels}& \multicolumn{3}{c}{
        \includegraphics*[scale=1]{figures_arxiv/kernels/4.png}\;\includegraphics*[scale=1]{figures_arxiv/kernels/7.png}\;\includegraphics*[scale=1]{figures_arxiv/kernels/9.png}\;\includegraphics*[scale=1]{figures_arxiv/kernels/11.png}} 
    \end{tabular}
        \caption{\label{tab:deb}Comparison  of PnP-HVAE  and other restoration methods on deblurring. Results are averaged on $4$ kernels.\vspace{-0.2cm}}% on image deblurring.}
    \end{center}
\end{table}

\begin{figure}
    
    \begin{subfigure}[h]{\linewidth}
        \centering
        \includegraphics*[width=\columnwidth]{figures_arxiv/deb_s255_k7.pdf}\vspace{-0.1cm}
        \caption{Gaussian blur, $\sigma=2.55$}
    \end{subfigure}
    \begin{subfigure}[h]{\linewidth}
        \centering
        \includegraphics*[width=\columnwidth]{figures_arxiv/deb_s765_k11.pdf}\vspace{-0.1cm}
        \caption{Motion blur, $\sigma=7.65$}
    \end{subfigure}\vspace*{-0.1cm}
    \caption{\label{fig:deblurring_bsd} Natural image deblurring\vspace{-0.1cm}}
\end{figure}

\noindent {\bf Effect of the temperature.}
PnP-HVAE gives control on the temperature of the prior over the latent space.
In figure~\ref{fig:temp_effect}, we illustrate that reducing the temperature increases the strength of the regularization prior. In this example the tuning $\tau=0.7$ produces the best performance.\\
\begin{figure}[!ht]
   
    \includegraphics[width=\columnwidth]{figures_arxiv/demo_temp.pdf}\vspace{-0.15cm}
    \caption{ \label{fig:temp_effect} Effect of the temperature in PnP-VAE on a deblurring problem, with $\sigma=7.65$.\vspace{-0.15cm}}
\end{figure}


\noindent
{\bf Image inpainting.}
Next we consider the task of noisy image inpainting. 
We compose a test-set of 10 images from the validation set of BSD~\cite{MartinFTM01} and we create masks
  by occluding diverse objects of small size in the images. 
A gaussian white noise with $\sigma=3$ is added to the images.
As a comparaison, we still consider GS-PnP and EPLL.
For PnP-HVAE, the temperature is set to $\tau=0.6$, and the algorithm is run for a maximum of $200$ iterations, unless the residual $||\x_{k+1}-\x_k||$ is on a plateau.
We provide on Table~\ref{tab:inpainting_bsd} the distortion metrics with the ground truth, as well as a visual
\begin{table}



\begin{center}
    \begin{tabular}{cccc}
        & PSNR$\uparrow$ & SSIM$\uparrow$ &LPIPS$\downarrow$ \\\hline
        PnP-HVAE  & $\mathbf{29.54}$ & $\mathbf{0.93}$ & $\mathbf{0.06}$\\
        GS-PNP & $28.52$ & $\mathbf{0.93}$ & $0.09$\\
        EPLL & $\underline{29.16}$ & $\mathbf{0.93}$ & $\mathbf{0.06}$\\
    \end{tabular}
    \caption{\label{tab:inpainting_bsd}Quantitative evaluation for inpainting on BSD.}
    \end{center}
\end{table}
comparison on figure~\ref{fig:inpainting_bsd}. 
With its hierarchical structure,  PnP-HVAE outperforms the compared methods. \vspace{0.05cm}



\begin{figure}[!h]
    \includegraphics[width=\columnwidth]{figures_arxiv/demo_inp_bsd2.pdf}\vspace{-0.1cm}
    \caption{\label{fig:inpainting_bsd}Natural image inpainting\vspace{-0.3cm}}
\end{figure}











\section{Predictions of embeddings from the NTK after training}
\label{app:emb_pred}

Here we demonstrate that the true embeddings learned by SSL methods using small initialization show fair agreement with predictions computed from our theory using the empirical NTK after training.
The motivation for this experiment comes from an observation of \citet{atanasov:2021-silent-alignment} regarding ResNets with small initialization trained in a supervised setting.
Though the NTK of such a model evolves dramatically during training, the final function is nonetheless well-predicted by kernel regression using the empirical NTK after training.
We find a similar result here.

We train ResNet-50 encoders from moderately small init to convergence with Barlow Twins ($\alpha=0.542$), SimCLR ($\alpha=0.284$), and VICReg ($\alpha=0.604$) losses on STL-10.
These models have only $d = 50$, which is large enough to be nontrivial but small enough to give good statistical power to our subsequent analysis.
After training, we then take a batch of $n = 1000$ random augmented image pairs $\X$ and compute both their joint empirical NTK $\tilde{\mK} \in \R^{2n \times 2n}$ and their embeddings $\vf_\text{exp}(\X) \in \R^{d \times 2n}$.
We trust that $n$ is sufficiently larger than $d$ that it is reasonable to treat $\X$ as the full population distribution.
We then compute the theoretically predicted embeddings using Equation \ref{eqn:kernelized_top_spectral_embs} as
\begin{equation}
    \vf_\text{th}(\X) = \tilde{\mS} [\vb_1 \ ... \ \vb_d]\T \tilde{\mK}^{1/2}.
\end{equation}

It is principally the subspace (in function space) spanned by a vector representation which determines the performance of a linear probe on a downstream task.
As such, we compute the right singular vectors of $\vf_\text{exp}(\X)$ and $\vf_\text{th}$, which we call $\mP_\text{exp} \in \R^{d \times 2n}$ and $\mP_\text{th} \in \R^{d \times 2n}$ and which both have rank $d$.
We then compute the \textit{normalized subspace alignment} $\a(\mP_\text{exp}, \mP_\text{th})$, where $\a(\mP, \mP’) \equiv \frac{1}{d} \norm{\mP (\mP’)\T}_F^2$.
This alignment metric attains its maximal value of $d$ when both subspaces contain the same vectors, and has an expectation of $\frac{d^2}{2 n} \ll 1$ for random subspaces\footnote{One may intuitively think of $\a(\mP, \mP’)$ as the mean fraction of a random vector from the rowspace of $\mP$ which is captured by the rowspace of $\mP'$.}.
As an additional test, we repeated this comparison replacing $\mP_\text{th}$ with $\mP_\text{NTK}$ containing the top $d$ eigenvectors of $\tilde{\mK}$ and found similar alignment.

\begin{table}
\begin{centering}
\begin{tabular}{ccccc}
  & BT & SimCLR & VICReg & (random subspaces)\\ \hline
$\a(\mP_\text{exp}, \mP_\text{th})$ & 0.615 & 0.517 & 0.592 & 0.025 \\
$\a(\mP_\text{exp}, \mP_\text{NTK})$ & 0.510 & 0.450 & 0.481 & 0.025
\end{tabular}
\caption{
\label{tab:th-exp-emb-alignment}
Alignments between true embedding subspaces and those predicted from the final NTK for different SSL methods.
}
\end{centering}
\end{table}

\begin{table}
\begin{centering}
\begin{tabular}{cccc}
$\a(\mP^\text{BT}_\text{exp}, \mP^\text{SC}_\text{exp})$ &
$\a(\mP^\text{BT}_\text{exp}, \mP^\text{VR}_\text{exp})$ &
$\a(\mP^\text{SC}_\text{exp}, \mP^\text{VR}_\text{exp})$ & (random subspaces)
\\ \hline
0.504 & 0.504 & 0.405 & 0.025\\
\end{tabular}
\caption{
\label{tab:cross-method-alignment}
Alignments between true embedding subspaces for different SSL methods.
}
\end{centering}
\end{table}

We report our observed subspace alignments in Table \ref{tab:th-exp-emb-alignment}.
For all three methods, we see an alignment between 0.5 and 0.6, which is significantly higher than expected by chance, but still misses roughly half of the span of the true embeddings\footnote{Agreement with predictions from the \textit{initial} NTK (not reported) is generally worse but still greater than chance.}.
We anticipate that much of the gap from unity is due to approximation error caused by taking a finite dataset of size $2n$.
It is perhaps surprising that our theory seems to work equally well for SimCLR and VICReg as it does for Barlow Twins.

We also report alignment scores \textit{between $\mP_\text{exp}$ from the three SSL methods} in Table \ref{tab:cross-method-alignment}, again finding alignment roughly between 0.4 and 0.5.
This finding, independent of any theoretical result, is evidence that these three methods are in some respects learning very similar things.


\section{Connection to spontaneous symmetry breaking in physical systems}
\label{app:symm}

In the main text, we noted that \citet{landau:1944-turbulence} encountered essentially the same dynamics in the study of turbulent fluid flow as we found for SSL.
This is due to the fact that both are simple processes of \textit{spontaneous symmetry breaking} (SSB), a phenomenon in which a system whose dynamics obey a symmetry spontaneously settles into an asymmetric state chosen as a result of the system's initial conditions.
In this appendix, we will explain how the dynamics of our model of SSL can be understood as a process of SSB.

Recall from Equation \ref{eqn:L_as_eigensum} that the loss of our toy model upon aligned initialization is
\begin{equation}
    \L = \sum_j \L_j = \sum_j (1 - \gamma_j s_j^2)^2 = \sum_j (1 - \bar{s}_j^2)^2,
\end{equation}
where we define $\L_j = (1 - \gamma_j s_j^2)^2$ and $\bar{s}_j = \gamma_j^{1/2} s_j$.
The quantity $\bar{s}_j$ is a (rescaled) singular value of $\mW$.
Singular values are canonically taken to be nonnegative (because one can always enforce this condition by appropriate choice of the singular vectors), but let us pretend for the present discussion that singular values may take any value along the real line.
Note that each singular value evolves via gradient descent according to its respective loss as
\begin{equation}
    \bar{s}_j'(t) = \frac{d \L_j}{d \bar{s}_j(t)}.
\end{equation}

\begin{figure}
  \centering
  \includegraphics[width=12cm]{img/ssb_cartoon_fig.png}
  \vspace{-3mm}
  \caption{
    \textbf{SSL is a process of symmetry breaking, whereas standard supervised learning is not.}
    \textbf{(A)}: Nondimensionalized loss landscape for singular value $\bar{s}_j$ in our linear model of Barlow Twins.
    The system can reach one of two minima depending on the sign of the initialization.
    \textbf{(B)}: Nondimensionalized loss landscape for an eigencoefficient in linearized \textit{supervised} learning with $c_j = 1$.
    There is only one local minimum.
    \textbf{(C)}: Example trajectories for the loss landscape in (A) from small init with different eigenvalues $\gamma_j$.
    \textbf{(D)}: Example trajectories for the loss landscape in (B) from small init with different time constants $\gamma_j$.
  }
  \label{fig:ssb_cartoon}
\end{figure}

Figure \ref{fig:ssb_cartoon}A depicts $\L_j$, with trajectories for various $\gamma_j$ shown in Figure \ref{fig:ssb_cartoon}C.
Note that this 1D landscape is symmetric and has two global minima at $\bar{s}_j = \pm 1$ and one local maximum at $\bar{s}_j = 0$.
When the system is initialized near zero, an unstable equilibrium of the dynamics, it will flow towards one basin or the other depending on its initial sign.
However, for small $\bar{s}_j(0)$, it takes a time $\tau_j = -\log(|\bar{s}_j(0)|) / 4 \gamma_j$ for $\bar{s}_j$ to escape the origin\footnote{To be more precise, it takes a time $\tau_j^{(\epsilon)} \approx -\log(\epsilon / |\bar{s}_j(0)|) / 4 \gamma_j$ to escape a ball of small constant radius $\epsilon > 0$ around the origin, and we drop the sub-leading-order contribution of $\epsilon$.}, whereas the system thereafter approaches the nearest minimum with the faster timescale $\tau'_j = 1/4\gamma_j$.
This slow escape from the origin is what leads to the sharp stepwise behavior we find.
% which is an inevitable consequence of the symmetry and differentiability of the landscape,

It should be noted that the quartic form of $\L_j$ appears in toy models of SSB across physics (see e.g. Landau-Ginzburg theory in \citet{kardar:2007-stat-phys-of-fields}) and is also the aforementioned model of \citet{landau:1944-turbulence}.
In these and other typical cases, SSB entails the breaking of a symmetry apparent at the outset of the problem such as invariance to translation, rotation, or inversion symmetry.
In the case of SSL, the symmetry is the transformation $\vf \rightarrow - \vf$, which does not change the value of the global loss\footnote{Our model in fact obeys the more general symmetry $\vf \rightarrow \mU \vf$ for any orthonormal matrix $\mU$, which is shared by SimCLR but not by VICReg or Barlow Twins with $\lambda \neq 1$.}.

Why does standard supervised learning in the NTK limit not exhibit stepwise behavior upon small initialization?
Following the analysis of \cite{jacot:2018}, the analogous modewise loss for a supervised setup takes the form $\L_j \propto (\bar{c}^*_j - \gamma_j c_j)^2$, with $c_j$ a learnable coefficient of Gram matrix eigenvector $j$ in the representer theorem coefficients of the learned function and $\bar{c}^*_j$ a constant.
We nondimensionalize in this case as $\bar{c}_j = \gamma_j c_j$.
As shown in Figure \ref{fig:ssb_cartoon}B, the landscape in the supervised case is merely quadratic and has no unstable equilibria, reflecting the lack of inversion symmetry in the loss.
Therefore $\bar{c}_j'(0)$ is not small, and the respective coefficients grow immediately rather than first undergoing a period of slow growth, as shown in Figure \ref{fig:ssb_cartoon}D.

The main surprising finding of our experiments is that SSL experiments with ResNets exhibit stepwise learning (especially upon small init) even far from the linear (i.e. lazy) regime.
It thus seems likely that one could derive the stepwise phenomenon from a much looser set of assumptions on the model class.
The view of stepwise learning as a simple consequence of SSB may be of use in the development of a more general theory in this sense.
This SSB view suggests that we may understand the growth of each new embedding direction from zero as the escape from a saddle point and use the fact that the local loss landscape around any saddle point of a given index (and with nondegenerate Hessian) is the same as any other up to rescaling of the domain.
We conjecture that stepwise learning will occur generically modulo edge cases (such as simultaneous growth of multiple directions) given an appropriate choice of loss function under minimal and realistic conditions on the empirical (i.e. time-varying) NTK of the model and assuming the final embeddings are the output of a linear readout layer.



\section{Potential modifications for speeding up SSL}
\label{app:speedup}

Compared to standard supervised learning, SSL is known in folklore to be much slower to train.
Our work presents a theory of the training dynamics of SSL, and thus it is natural to ask whether it sheds light on this point of difficulty or suggests means by which it might be addressed.
Here we suggest an explanation for the slowness of SSL training and propose various fixes which are ripe for future study.

In our picture of the training dynamics of SSL, embedding eigenvalues start small and grow sequentially, converging when $d$ eigenvalues have sufficiently grown.
Smaller eigendirections require more time to grow.
\textbf{We suggest that SSL is slow to converge because one must wait for the small eigendirections to grow.}
This hypothesis is supported by Figure \ref{fig:dynamics_realistic}, which shows that, in a realistic configuration, a significant fraction of the eigenvalues remain small even late in training.\footnote{This observation is particularly compelling in light of the finding of \citet{garrido:2022-rankme} that generalization is better when embeddings have higher rank. Their work suggests that hyperparameters should be chosen to maximize the embedding rank at the end of training.}

This suggests that SSL can be sped up by modifying training to target small eigenvalues.
Here we suggest one way this might be achieved via preconditioning of gradients and two additional ways this might be achieved by modifying the loss function itself.
We leave our results at the level of theoretical speculation.
We encourage interested SSL practitioners to try implementing the methods described and reach out with questions.

\subsection{Targeting gradients towards small PCA directions.}
One potentially promising idea is to simply compute the PCA matrix of the embeddings and ``manually" increase the gradient pull along directions which are very small, thereby encouraging the distribution to expand in directions in which it is currently flat.
Let us denote by $\mF \in \R^{2n \times d}$ the embeddings for a given data batch.
Backpropagation will compute $\nabla_\theta \L = \text{Tr}[ (\nabla_\theta \mF)\T \ \nabla_\mF \L ]$.
We may simply enact the substitution
\begin{equation}
    \nabla_\mF \L
    \longrightarrow
    (\nabla_\mF \L) (\mF\T \mF + \alpha \mI_d)^{-1}
\end{equation}
with $\alpha > 0$ to suppress $\nabla_\mF \L$ along directions of large variance and increase it along directions of small variance.
We anticipate this will encourage faster growth of small eigenmodes.
This trick could apply to any joint embedding method.

\begin{figure}
  \centering
  \includegraphics[width=12cm]{img/speedup_cartoon.pdf}
  \vspace{-3mm}
  \caption{
    \textbf{Proposed modifications for the modewise SSL loss to encourage or enable the faster growth of slow eigendirections.}
    Losses are expressed as a function of $\bar{s}_j = \gamma_j^{1/2} s_j$ and assume $\gamma_j > 0$.
    \textbf{(A)}: Original Barlow Twins modewise loss.
    \textbf{(B)}: A modified loss with a kink at zero. The origin is no longer an unstable equilibrium and nearby points will quickly escape.
    \textbf{(C, D)}: Two losses modified so as to have smaller curvature at their minima and thereby permit larger learning rates without instability.
  }
  \label{fig:speedup_cartoon}
\end{figure}

\subsection{Sharpening at zero}
This and the following method are specialized to the Barlow Twins loss.
They will make use of visual depictions of the modewise loss landscape as described in Appendix \ref{app:symm}, and so we recommend reading that appendix first.
They will both consider transformations $g(\mC)$ of the correlation matrix $\mC$, where $g$ acts pointwise on the eigenvalues of $\mC$.

Recall that we found that the singular values of the system under aligned conditions evolve under a loss of the form $\L_j = (1 - \lambda_j)^2 = (1 - \gamma_j s_j^2)^2$.
As shown in Figure \ref{fig:speedup_cartoon}A and discussed in Appendix \ref{app:symm}, this means that, when initialized near zero, they initially feel only a small gradient and must spend a long time escaping the origin, with that duration determined by just how small they are initially.
However, this is no longer the case if we change the loss so it has a kink at zero.
For example, if we change the loss to $\L_{\text{sharp}}(\mC) = \norm{g_\text{sqrt}(\mC) - \mI_d}_F$ with $g_\text{sqrt}(\lambda) = \text{sign}(\lambda) |\lambda|^{1/2}$, then all singular values with $\gamma_j > 0$ feel a $\Theta(1)$ repulsive force from the origin regardless of how close they are.
This loss is plotted in Figure \ref{fig:speedup_cartoon}B.
Interventions of this type lead to eigenvalue growth curves of the type shown in Figure \ref{fig:ssb_cartoon}D.\footnote{In fact, the particular choice $g_\text{sqrt}(\lambda) = \text{sign}(\lambda) |\lambda|^{1/2}$ gives dynamics which can be solved exactly for aligned init just like the original Barlow Twins loss studied in the main text.}

\subsection{Smoothing around minima}

The naive way to speed up the growth of slow-growing eigenmodes is simply to increase the learning rate.
This fails because the minima of the fast-growing eigenmodes will eventually become unstable.
For example, if $d=2$ and the total loss is $\L = (1 - s_1^2)^2 + (1 - 10^{-6} s_2^2)^2$, any learning rate above $\eta = 1/4$ will cause instability in $s_1$, but $s_2$ requires a larger learning rate to grow in reasonable time.
One solution here is to modify the loss landscape so that the higher eigendirections see a wider basin around their minima and can thus tolerate a larger learning rate.
One implementation might be $\L_{\text{smooth}}(\mC) = \norm{g_\text{smooth}(\mC) - \mI_d}_F$ with $g_\text{smooth}(\lambda) = \min(\lambda, 1)$.
One might replace this with e.g. $g_\text{smooth} = \min(\lambda, 1 + \epsilon (1 - \lambda))$ for some small $\epsilon > 0$.
Alternatively, one might modify the structure of the loss function itself to be e.g. $\L_j = (1 - \lambda_j)^4$, which has vanishing curvature at its minimum, or implement a similar idea with a hinge function to create a perfectly flat basin of finite width.
Two possibilities for losses modified to have wider minima are shown in Figures \ref{fig:speedup_cartoon}C,D.

Both these last ideas are specialized to Barlow Twins in the case of $\lambda = 1$, but we anticipate they could be easily generalized.
In fact, VICReg already includes a square root in the variance term which we conjecture implicitly does something similar to our proposed sharpening modification.



\end{document}