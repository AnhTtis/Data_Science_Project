\section{Derivation of dynamics from generic small initialization}
\label{app:derivation}

Here we give an informal derivation of Result \ref{res:small_init}, which states that the solution of Equation \ref{eqn:dW_dt_expanded} from arbitrary initialization with scale $\alpha \ll 1$ closely matches the analytical solution from a certain spectrally aligned initialization.
Recall that our ingredients are the following:
\begin{itemize}
\item $\gamma_1, ..., \gamma_d$ are unique and positive.
\item $\tilde{\mW}_0 \in \R^{d \times m}$ with $\tilde{\mW}_0 \mGamma^{(\le d)}$ full rank.
\item $\mW(t)$ is the true solution with $\mW(0) = \alpha \tilde\mW_0$.
\item $\mW^*(t)$ is the spectrally aligned solution with $\mW^*(0) = \mathcal{A}(\alpha \tilde\mW_0)$ whose dynamics are given exactly by Proposition \ref{prop:exact_dynamics}.
\end{itemize}
We will show that, for sufficiently small $\alpha$, the true and aligned solutions remain arbitrarily close.

We will find it convenient to parameterize $\mW(t)$ as
\begin{align} \label{eqn:generic_W_fac}
    \mW(t) &=
    \mU
    \begin{bmatrix} s_1(t)  & a_{1 2 }(t)  & a_{1 3 }(t)  & \cdots  & a _{1 d }(t)  & \cdots  & a _{1 m }(t)  \\ a_{21}(t)  & s_2(t)  & a _{2 3 }(t)  & \cdots  & a _{2 d }(t)  & \cdots  & a _{2 m }(t)  \\ a_{31}(t)  & a_{32}(t)  & s_3(t)  & \cdots  & a _{3 d }(t)  & \cdots  & a _{3 m }(t)  \\ \vdots  & \vdots  & \vdots  & {\ddots}  & \vdots  &   & \vdots  \\ a_{d1}(t)  & a_{d2}(t) & a_{d3}(t) & \cdots  & s_d(t)  & \cdots  & a_{dm}(t)  \\  \end{bmatrix}
    \mGamma^{(\le m)} \\
    &= \mU \mA(t) \mGamma^{(\le m)},
\end{align}
where $s_j(0) > 0$ and $a_{jk}(0) = 0$ for all $j > k$.
We use the special notation $s_j$ for the diagonal elements of $\mA$ to foreshadow that these will act as the effective singular values of the dynamics.
Note that the spectrally aligned initialization $\mathcal{A}(\tilde\mW_0)$ is precisely the $\mW(0)$ of Equation \ref{eqn:generic_W_fac} but with all off-diagonal entries of $\mA(0)$ zeroed out.
Our strategy will be to show that no $a_{jk}(t)$ ever grows sufficiently large to affect the dynamics to leading order, and thus $\mW(t)$ and $\mW^*(t)$ remain close.

We will make use of big-$\O$ notation to describe the scaling of certain values with $\alpha$.
Eigenvalues $\gamma_j$ and differences $\gamma_j - \gamma_{j+1}$ will be treated as constants.
Note that, because $\mW(0) = \alpha \tilde\mW_0$, all elements of $\mA(0)$ are $\Theta(\alpha)$ if they are not zero.

\textit{Diagonalization of dynamics.}
Define $\mLambda = \text{diag}(\gamma_1, ..., \gamma_m)$.
The dynamics of Equation \ref{eqn:dW_dt_expanded} state that $\mA(t)$ evolves as
\begin{equation} \label{eqn:dA_dt_exact}
    \frac{d\mA(t)}{dt} = \left( \mI - \mA(t) \mLambda \mA\T(t) \right) \mA(t) \mLambda,
\end{equation}
where we have reparameterized $t \rightarrow t/4$ to absorb the superfluous prefactor of $4$.

\textit{Approximate solution to dynamics.}
So long as all $a_{jk}$ remain small (i.e. $o(1)$), then these dynamics are given by
\begin{equation} \label{eqn:dA_dt_approx}
    \frac{d\mA(t)}{dt} \approx \left( \mI - \fourdiag{\gamma_1 s_1^2(t)}{\gamma_2 s_2^2(t)}{\gamma_3 s_3^2(t)}{\gamma_d s_d^2(t)} \right) \mA(t) \mLambda.
\end{equation}
We will show that all $a_{jk}$ indeed remain small under the evolution of Equation \ref{eqn:dA_dt_approx}, and so Equation \ref{eqn:dA_dt_approx} remains valid.

Solving Equation \ref{eqn:dA_dt_approx} yields
\begin{align}
    \label{eqn:s_j_ideal}
    s_j(t) &= \frac{e^{\gamma_j t}}{\sqrt{s_j^{-2}(0) + (e^{2 \gamma_j t} - 1) \gamma_j}} \\
    \label{eqn:a_jk_ideal}
    a_{jk}(t) &= 
    \left\{\begin{array}{ll}
        a_{jk}(0)
        \left( \frac{s_j(t)}{s_j(0)} \right)^{\gamma_k / \gamma_j}
        = \O\left( \alpha^{1 - \gamma_k / \gamma_j} \right)
        & \text{for } j < k, \\
        0 & \text{for } j > k.
        \end{array}\right.
\end{align}
As discussed in the main text, each $s_j(t)$ remains small up to a time $\tau_j \sim - \gamma_j^{-1} \log{\alpha}$, at which it quickly grows until $\gamma_j s_j^2(t) = 1$ and saturates.
Entries of $\mA(t)$ below the diagonal remain zero.
Entries of $\mA(t)$ above the diagonal exhibit interesting dynamics: entry $a_{jk}(t)$ with $j < k$ grows exponentially at rate $\gamma_k$, but its growth is curtailed by the saturation of $s_j(t)$ before it has time to reach order one.
This is because each $s_j(t)$ grows faster than all $a_{jk}(t)$ in its row.

All $a_{jk}(t)$ thus remain $o(1)$ and all $s_j(t)$ closely follow the ideal solution of Equation \ref{eqn:s_j(t)}, and so $\norm{\mW(t) - \mW^*(t)}_F$ remains $o(1)$.
This concludes the derivation.

\textit{Numerical validation of approximation.}
While the numerical experiment presented in Figure \ref{fig:banner_figure} validates our claim regarding the trajectory of $\mW(t)$ from generic small initialization closely matching theoretical predictions from aligned initialization, here we go a step further and show agreement for individual elements of $\mA(t)$.
We analytically solve Equation \ref{eqn:dA_dt_approx} for $d = 3$ and $m = 5$ with $\gamma_j = 2^{-j}$, starting from a random upper-triangular $\mA(0)$ of scale $\alpha = 10^{-9}$.
The results, plotted in Figure \ref{fig:A_jk_traces}, closely match the unapproximated dynamics of Equations \ref{eqn:s_j_ideal} and \ref{eqn:a_jk_ideal}.

\begin{figure}
  \centering
  \includegraphics[width=11cm]{img/small_init_weight_traces.pdf}
  \vspace{-3mm}
  \caption{
  True $|s_j(t)|$ and $|a_{jk}(t)|$ compared with theoretical predictions from small init.
  Red traces show $s_j(t)$, blue traces show $a_{jk}(t)$ with $j < k$, and green traces show $a_{jk}$ with $j > k$.
  While there are no theoretical traces for $a_{jk}$ with $j > k$, these elements do remain small as predicted.
  }
  \label{fig:A_jk_traces}
\end{figure}
