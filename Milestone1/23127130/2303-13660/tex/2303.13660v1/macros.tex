
% \newcommand{\mofa}{MoFA\xspace}

\newcommand*{\ourmean}[1]{\overline{#1}}

% \newcommand{\oursubsub}[1]{\vspace{10pt}\noindent\textit{#1\newline}\vspace{0pt}}
\newcommand{\oursubsub}[1]{\subsubsection{#1}}

% optional macro used to highlight things that need changing
\newcommand{\optional}[1]{\textcolor{YellowGreen}{#1}}
% \newcommand{\optional}[1]{}

% todo macro used to highlight things that need changing
\newcommand{\todo}[1]{\textcolor{red}{\textbf{#1}}}
% \newcommand{\todo}[1]{}

\newcommand{\done}[1]{\textcolor{orange}{\st{#1}}}
%\newcommand{\done}[1]{}

\newcommand{\timtext}[1]{\textcolor{black}{#1}} %added by Dave
% \newcommand{\timtext}[1]{\textcolor{olive}{\textbf{#1}}}
%\newcommand{\timtext}[1]{}

% oldtext macros
% This is meant to draw lines through the text but it doesn't
% seem to work everywhere
% \newcommand{\oldtext}[1]{\textcolor{cyan}{\st{#1}}}
% Use this to mark text in light blue for potential removal
\newcommand{\oldtext}[1]{\textcolor{cyan}{#1}}
% Use this definition to hide the old text (useful to see how long the paper
% would be if we remove the oldtext
% $\newcommand{\oldtext}[1]{}

\newcommand{\heading}[1]{\vspace{3pt}\noindent\textbf{#1 }}

\newcommand{\mysubsection}[1]{\vspace{3pt}\noindent\textbf{#1 }}

% \newcommand{\newtext}[1]{\textcolor{black}{#1}} %added by Dave
\newcommand{\newtext}[1]{\textcolor{blue}{#1}}
%\newcommand{\newtext}[1]{#1}

% list item with some vertical space removed
\newcommand{\vitem}{\vspace{-5pt}\item}


% Trying different ways to get the guidelines to stand out more
% \newenvironment{guideline}{\begin{framed}\emph{\bf GUIDELINE:} \it }{\end{framed}}
% This did not work very well, way too large a frame
% \newenvironment{guideline}{\begin{mdframed}\emph{\bf GUIDELINE:} \it }{\end{mdframed}}
\newenvironment{guideline}{\vspace{0pt} \noindent \hrulefill \\ \emph{\bf \textcolor{blue}{GUIDELINE:}} \it }{\\ \vspace{-5pt} \hrule}
% \newenvironment{guideline}{\emph{\bf GUIDELINE:} \it }{}


\newcommand{\squishbegin}{
 \begin{list}{$\bullet$}
  { \setlength{\itemsep}{0pt}
     \setlength{\parsep}{1pt}
     \setlength{\topsep}{1pt}
     \setlength{\partopsep}{0pt}
     \setlength{\leftmargin}{1.5em}
     \setlength{\labelwidth}{1em}
     \setlength{\labelsep}{0.5em} 
  } 
}

\newcommand{\squishtwobegin}{
 \begin{list}{$-$}
  { \setlength{\itemsep}{1pt}
     \setlength{\parsep}{1pt}
     \setlength{\topsep}{1pt}
     \setlength{\partopsep}{0pt}
     \setlength{\leftmargin}{1.5em}
     \setlength{\labelwidth}{1em}
     \setlength{\labelsep}{0.5em} 
  } 
}

\newcommand{\squishend}{
  \end{list}  
}

\newcommand{\experiment}{\vspace*{4pt}\noindent\textbf{Experiment Setup:\hspace{0.4em}}}
%\newcommand{\experimentend}{\vspace*{4pt}}
\newcommand{\experimentend}{}

\newcommand{\moveup}{\vspace{-8pt}}
\newcommand{\movecaptionup}{\vspace{-20pt}}
\newcommand{\movecaptionuptab}{\vspace{-17pt}}
\newcommand{\colfigwidth}{0.90\columnwidth}

% NOTE!!! Labels must come after captions.

% btable #1 - location
% etable #1 - label, #2 - caption
\newcommand{\btable}[1]{\begin{table}[#1] \begin{center} }
%\newcommand{\etable}[2]{\end{center} \movecaptionuptab \caption{#2} \label{#1} \end{table}}
\newcommand{\etable}[2]{\end{center} \vspace{-5pt} \caption{#2} \label{#1} \vspace{-15pt}\end{table}}

\newcommand{\wbtable}[1]{\begin{table*}[#1] \begin{center} }
\newcommand{\wetable}[2]{\end{center} \caption{#2} \label{#1} \end{table*}}

% Define a figure by specifying the size in the x dimension
% xfigure: #1 - location #2 - xsize, #3 - filename, #4 - label, #5 - caption
\newcommand{\xfigure}[5]{\begin{figure}[#1] \begin{center} \leavevmode \epsfxsize=#2 \epsfbox{#3} \end{center} \vspace{-12pt} \caption{#5} \label{#4} \end{figure}}
% \newcommand{\xfigure}[5]{\begin{figure}[#1] \moveup \begin{center} \leavevmode \epsfxsize=#2 \epsfbox{#3} \end{center} \movecaptionup \caption{#5} \label{#4} \end{figure}}

\newcommand{\xfigurewide}[5]{\begin{figure*}[#1] \moveup \begin{center} \leavevmode \epsfxsize=#2 \epsfbox{#3} \end{center} \movecaptionup \caption{#5} \label{#4} \end{figure*}}

% Define a figure by specifying the size in the x dimension
% yfigure: #1 - location #2 - ysize, #3 - filename, #4 - label, #5 - caption
%\newcommand{\yfigure}[5]{\begin{figure}[#1] \begin{center} \leavevmode \epsfysize=#2 \epsfbox{#3} \end{center} \vspace{-20pt} \caption{#5} \label{#4} \end{figure}}
\newcommand{\yfigure}[5]{\begin{figure}[#1] \begin{center} \leavevmode \epsfysize=#2 \epsfbox{#3} \end{center} \caption{#5} \label{#4} \end{figure}}

% Define a figure by specifying the size in the x dimension
% xyfigure: #1 - location #2 - xsize, #3 - ysize #4 - filename,
% #5 - label, #6 - caption
%\newcommand{\xyfigure}[6]{\begin{figure}[#1] \begin{center} \leavevmode \epsfxsize=#2 \epsfysize=#3 \epsfbox{#4} \end{center} \vspace{-20pt} \caption{#6} \label{#5} \end{figure}}
\newcommand{\xyfigure}[6]{\begin{figure}[#1] \begin{center} \leavevmode \epsfxsize=#2 \epsfysize=#3 \epsfbox{#4} \end{center} \caption{#6} \label{#5} \end{figure}}

\newcommand{\bfigure}[1]{\begin{figure}[#1]}
\newcommand{\efigure}[2]{\vspace{-8pt} \caption{#2} \label{#1} \end{figure}}
% \newcommand{\efigure}[2]{\caption{#2} \label{#1} \end{figure}}

