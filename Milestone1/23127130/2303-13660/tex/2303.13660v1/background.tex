\section{Background and Scope}
\label{sec:background}


\begin{figure}[t]
    \centering
    \begin{subfigure}[b]{0.85\linewidth} 
        \centering
        \includegraphics[width=\linewidth]{Figures/soccer_pitch_2teams.png}
        \caption{Soccer pitch (two teams).}
        \label{fig:soccer}
    \end{subfigure}
    \begin{subfigure}[b]{0.83\linewidth} 
        \includegraphics[width=\linewidth]{Figures/hockey_zone_crop.png}
        \caption{Ice hockey scenario (two teams) with puck.}
        \label{fig:hockey}
    \end{subfigure}
        \caption{(a) Example of a soccer (or football) pitch with two teams, and (b) an ice hockey end-zone (one-third of surface) with two teams. Players are divided into positions with individual tasks but have the overall goal of winning the game.}
        \label{fig:sports}
\end{figure}



Team sports are played in a variety of environments and governed by a diverse set of rules.
While multiagent challenges may be found in most sports, we limit the discussion of this paper to invasion games where teams of dynamically moving players participate in two-team zero-sum matches within a larger league with many teams.
This includes widely popular team sports such as soccer, ice hockey, basketball, and many more.
% General themes about this class of sports can be extracted to better define the scope of our problem space.
Matches involve two teams ($\mathcal{A}$ and $\mathcal{B}$) composed of $N$ players (agents) each.
Teams typically have substitutions, so a player might not play the entire game; thus, the set of \emph{active} players in the game at any time is $n \subseteq N$ for each of $\mathcal{A}$ and $\mathcal{B}$ (i.e., $n=6$ in ice hockey, $n=5$ in basketball, and $n=11$ in soccer).
A team can often be further divided into \emph{positions} (roles), such as defense or offense, where players within a particular position are further specialized (i.e., left and right defense).
Figure~\ref{fig:sports} shows two examples of invasion games (soccer and ice hockey) with two teams (solid and white), where players are divided into different positions.
The intermingling of players exemplifies the complexity of interactions that occur in invasion games.


We examine challenges along two axes with inter-related problem spaces: within-match coaching and team management over long horizons.
While teams aim to win individual zero-sum matches, the challenge of managing teams involves modeling a larger environment, since teams compete in leagues with many teams.
As a result, challenges at these two levels operate on different timescales, but are not mutually exclusive.
Match results impact the challenges that management faces, while management's actions impact the coaching environment.
Addressing many of the challenges in this paper requires solutions that consider impacts along both axes.
% while examining multiple time scales.




% \begin{table}[t]
% \begin{center}
% \begin{tabular}{|c|c|c|c|c|}
%  \hline
%  Game Clock & Score Diff. & Player ID & Event & Location \\
%  \hline\hline
%  10:08 & -1 & 8787 & Pass & (78.3, 34.3) \\ 
%  \hline
%  10:06 & -1 & 7175 & Shot & (72.8, 12.8) \\
%  \hline
%  10:04 & 0 & 7171 & Goal & (90.0, -2.5) \\
%  \hline
% \end{tabular}
% \end{center}
% \caption{Example of event data.}
% \label{tab:event_data}
% % \vspace{-20pt}
% \end{table}

% \begin{table}[t]
% \begin{center}
% \begin{tabular}{|c|c|c|c|}
%  \hline
%  Game Clock & Player ID & Location & Velocity \\
%  \hline\hline
%  10:08.75 & 8787 & (80.3, 32.3) & (5.4, -3.2)  \\ 
%  \hline
%  10:08.74 & 7175 & (64.2, 5.8) & (4.4, 4.2) \\
%  \hline
%  10:08.73 & 1352 & (85.7, 1.8) & (0.3, 0.1) \\
%  \hline
% \end{tabular}
% \end{center}
% \caption{Example of tracking data.}
% \label{tab:tracking_data}
% \end{table}



\subsection{Data Collection}

% \todo{Have a small section explaining the various types of data sources (i.e., event data, tracking data... Maybe this is a good place to relate this to offline RL challenges (maybe in introduction too).}

Data collected about invasion games often includes different levels of detail, from event-based records (pass, shot, or goal) to tracking player locations~\cite{rein2016big,fernandez2021framework}.
Event-based data typically records details about significant in-game events such as time, score differential, player ID, event name, and coordinates.
% (Table~\ref{tab:event_data}).
Thus, a collection of sequential events can be modeled as a Markov chain, where each event depends on the game state and action of the preceding state~\cite{schulte2017markov}.
Thousands of events are recorded each match; however, event-based data is unable to capture the complete context of invasion games since the positions of all players are not recorded.

Tracking data records the locations of all players on a playing surface multiple times per-second.
Tracking systems are currently deployed in the highest leagues of soccer~\cite{bialkowski2014large}, ice hockey~\cite{radke2022identifying}, and basketball~\cite{sampaio2015exploring}.
Each tracking sample records features such as timestep, player ID, spatial coordinates, and velocity for each player and the ball or puck, typically amounting to millions of data points in each game.
Tracking systems function through physical hardware on each player and ball or puck, or vision-based systems that extract detailed events or attributes such as hockey stick location and pose, a significant area of hardware and computer vision research itself~\cite{beetz2006camera,vats2022evaluating,rahimian2022optical}.
Tracking data can also be modeled as a Markov chain of events where the action space includes player movement within the playing surface and can be joined with event data to add additional event context to player movements~\cite{fernandez2021framework}.
Many sports datasets are freely available for download.\footnote{\url{https://www.kaggle.com/datasets?search=sports}}


