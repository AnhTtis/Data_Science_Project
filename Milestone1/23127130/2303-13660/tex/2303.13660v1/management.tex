\section{Management \& Long-Term Strategy}
\label{sec:management}

% \begin{enumerate}
%   \item ``Management" - Roster Composition
%   \begin{itemize}
%      \item Roster valuation (should be done at pair/line level; find gaps in roster)
%      \item Financial performance bonuses (should this be done at pair/line level?)
%      \item Acquisitions (where would they fit inside current roster)
%      \item Drafting (projected roster/forward thinking, where would prospects fit in future roster)
%      \item Negotiation/Markets/Salary cap through time (who gets what fraction of salary cap and for how long)
%   \end{itemize}
% \end{enumerate}


% Coaches and management, the two main decision-making stakeholders in sports, face challenges in assessing the performance and playing personality of their current and future teams.
Management considers both short-term performance with higher-level constraints and long-term planning, resulting in multi-tiered problems that are particularly interesting from an AI perspective.
Roster analysis, roster construction, and financial strategy are areas with different, but related, inter-dependent multiagent challenges.
To the best of our knowledge, using sports to develop and evaluate AI models related to financial strategies has not been explored.
% appealing to different communities within MAS.
We propose the following high-level research questions:
% \noindent
\textbf{RQ1}: How can multiagent research support management to compose a cohesive team and construct environments that promotes player development under financial restrictions?
% \noindent
% \textbf{RQ2}: How can fair allocation and game theory support roster management in the context of salary restrictions?
\textbf{RQ2}: How can team management inspire multiagent research surrounding group valuation while considering economic strategies and diverse opponents?


\subsection{Roster Analysis}
\label{sec:roster_analysis}

Management conducts recruiting, drafting, trading, and signing players with the overall objective of creating a cohesive and high performing team.
When acquiring players, management needs to accurately analyze the current state of their roster to identify areas which could be improved or those that may be overvalued.
Analyses include the structure, cohesiveness, and playing style of their current team, as well as predicting the future developmental trajectories of their prospects~\cite{schuckers2011s}.
Playing style refers to the personality or characteristics of a player, group, or entire team (i.e., defensive forwards vs. offensive forwards).
Playing style is easy for experts to define; however, is not currently easily extractable from data.

Some prior work has focused on identifying groups of players with high chemistry or performance~\cite{liemhetcharat2015applying,Ljung2018PlayerPV} or learning semantic representations of players from event data~\cite{liu2020learning}.
Analyzing playing style may be similar to the thread of research studying agent types in ad hoc teamwork~\cite{Albrecht2017ReasoningAH} or role diversity in multiagent reinforcement learning (RL)~\cite{le2017coordinated,hu2022policy,Radke2022Exploring}.
Understanding how players' styles evolve and are impacted by their environment (teammates and usage) is an area for future research that requires AI models that identify characteristics of group agency, sub-group joint policy development, and emergent role specialization.
% best learn and evolve their joint policies over time and what tasks or roles these joint policies will be good at.
This direction will require new advancements in multiagent offline RL, multiagent inverse RL, behavior cloning, and offline policy evaluation to better understand the developmental impacts of team structure and multiagent interaction on agents' policies.
% Developing tools to better understand the composition of successful teams, analyze a current roster, and predict future rosters will only become more significant as more data is collected about team sports.
These advancements will push multiagent research to further understand group agency, joint-policy evaluation, and how policies are influenced by surroundings.


% Management must not only consider performance metrics but features such as playing style, development plans, and roster composition when evaluating the current and future states of their team.
% Playing style refers to the personality or characteristics of a player, group, or entire team; thus, identifying styles of play from recorded data can recover value functions or player representations~\cite{liu2020learning}.
% Algorithms such as offline reinforcement learning, inverse reinforcement learning, or behavior cloning may be useful in learning profiles for various compositions of players or groups; however, these algorithms must be developed with consideration of the impacts of other players in the environment. 
% These features may be helpful for managers to further understand how successful teams are built and inform decisions on drafting or signing future players.

\noindent
\ding{229} Learning Agent Capabilities \\
\ding{229} Modelling and Simulation of Societies


\subsection{Roster Construction}
\label{sec:roster_construction}

% \todo{Talk about risk of losing players (i.e., team forming with robustness to failures) and trading (need to assess value and determine if team composition increases in value after trade).}

% \begin{itemize}
%     \item potential value of current roster (in response to opponents playing and in-division standings)
%     \item identify gaps in roster, filter players that could fill gaps subject to potential trades or drafting (also need to project a player's performance/style in drafting).
% \end{itemize}

%Much like how coaches must continuously assess the performance of players and groups, management must perform similar valuation at a larger scale.
%Coaches are often limited by the players currently on their roster and tasked with optimizing the performance of that group, whereas 
A key challenge of management is constructing a team that can generalize to various opponents throughout a season.
Successful teams are typically composed of players with heterogeneous and complimentary skills, placing emphasis not only on analyzing a roster, but planning for how to construct a team through drafting, developing, trading, or signing free agents.
A draft is when teams in a league select from a pool of prospective players to claim their rights, comparable to selecting objects out of a pool of items by agents with different preferences in game theory~\cite{bouveret2014manipulating}.
Draft strategies have been studied from a game theoretical perspective~\cite{brams1979prisoners} analyzing different types of utility functions based on a team's needs.
% Strategic drafting has shown to converge to Prisoner's Dilemma scenarios~\cite{brams1979prisoners} where teams may have different utility functions and draft strategies based on the needs of their team through roster analysis (Section~\ref{sec:roster_analysis}).
Whereas roster analysis can help identify areas for improvement (Section~\ref{sec:roster_analysis}), predicting opponents' draft strategies and a best response is an interesting area of future work that requires a rich understanding of game theory and multi-level planning.
% Developing models to predict other team's preferences about prospects may help when devising strategies about when to draft certain players~\cite{alcox2019applications}.


Maximizing team performance is not simply about maximizing projected utility since group performance depends on players' abilities to develop and work together.
Successful teams solve problems of anticipation, distributed intelligence, and theory of mind to work as a collective organism~\cite{williamson2014distributed,bransen2020player}.
Teams can evolve to perform greater than the sum of individual parts~\cite{williamson2014distributed}, much like how groups in multiagent RL or evolutionary game theory develop complimentary policies by training together~\cite{Radke2022Exploring}.
However, the scenarios that allow agents to develop methods to best work together are still not fully understood in multiagent research~\cite{Durugkar2020BalancingIP,radke2022importance}.

Similar to problems in roster analysis, management's drafting or player acquisition strategies may shift depending on the current or projected composition of their team.
While understanding types of agents that form chemistry has been an active area of sports analytics research~\cite{liemhetcharat2015applying}, further development in the multiagent context is required to consider long-term team strategies, planning, and group alignment or incentives.
% liemhetcharat2014weighted
% Progress on these problems will benefit the MAS community to expand group-level problems across various time or incentive dimentionalities to better understand how agents form effective joint policies.
These problems will push the MAS community to better understand how agents form joint policies.

Lastly, losing players from a roster due to player injuries, retirement, or free agency is a common occurrence in team sports which often challenge a team's robustness at certain positions.
Team forming with an emphasis on being robust to failures or outages is a common problem in MAS~\cite{schwind2021partial} that can directly support how managers construct their team.
Providing insight into different degrees of group performance and robustness will provide management with more information when accumulating risk.
Future MAS work on group robustness to failures must expand to consider long-term temporal projections of agent value, uncertainty about agent value, and contractual agreements of various lengths.
% Modeling the preferences of opponent teams and identifying environments for players develop chemistry are important for long-term planning problems.
% Supporting management with team forming strategies to mitigate the loss of players due to injury, retirement, or free agency requires models for long-term strategic planning while considering opponent team's strategies.

% areas where MAS can make significant advances in sports analytics for long-term planning problems.

%Challenges at the level of management must focus on a longer time horizon than just single matches.



%Managers may acquire future players with the intention for them to contribute to the team's disposition; to do this, they rely on insights from player development and projections within a team's culture and strategy.
% Managers may also acquire players with the intention for them to perform certain roles (i.e., a defensive forward); to do this, they must assess the needs of their current roster.
%Creating an environment for effective player development and identifying areas of a roster where performance may be improved.
% subject so some goal as defined by the team composition desired by the manager.
% However, management's team-building goals may not always be  easily extractable from data (i.e., defensive play); thus, algorithms to learn player or group contributions must extract useful information from inter-player interactions.
%Analyzing the current and future status of a team's roster also includes forecasting the expected performance and role that prospects will develop within the organization; otherwise, models will not accurately reflect the long-horizon timescale of management and player development.
% Evaluating the production of different areas of a team's roster for alternative goals requires developing models to extract player types (CITE), playing styles (CITE), and performance with other players (CITE) based on events which may not contribute to explicit reward, but makes progress on other goals (i.e., defensive play is the absence of opponents' offense).

% Further, a shared characteristic of sports teams and the development of AI algorithms is emergent behavior.
% Successful sports teams evolve to perform greater than just the sum of individual parts~\cite{williamson2014distributed}, much like how groups in multiagent RL or evolutionary game theory develop policies together~\cite{Radke2022Exploring}.
% Constructing teams that fulfill and maintain a cohesive and productive ---- are pivotal for long-term success. 



\noindent
\ding{229} Markets, Auctions, and Non-Cooperative Game Theory \\
\ding{229} Teamwork and Team Formation 
%\ding{229} Organizations and Institutions



% Finally, professional sports are a business with monetary incentives and time-restricted contracts.
% To accurately forecast the future roster composition of a team, contract duration, amount, and expiry date add another layer of complexity to roster and group valuation.





% The challenge of valuation is concerned with assessing the value of an asset.
% From the management perspective, this asset could be individual players, pairs or groups of players, or the entire roster.
% Depending on the scale of the valuation, this creates various levels of interesting problems for the development of new algorithms around performance analysis of individuals and groups.

% The AI community typically determines the value of an agents based on metrics such as accuracy achieved or reward gained.
% In multiagent scenarios, reward may be less emphasized in place of task coverage, robustness to failures, or fairness across a population depending on the domain.
% There is no single evaluation procedure by which to determine the value of AI systems, and the same is true for sports players.
% Thus, valuating different aspects of sports teams is an appropriate testbed to represent broader AI challenges.

% Beyond the previously listed advantages including an enclosed system and rich datasets with defined constraints, sport teams carry a rich signal of context.
% A cohesive and successful team must identify valuable roles to fill and compose players who successfully execute those roles.
% Executing these roles may not follow the standard definition of \emph{reward}, such as positions more focused on defense or moving the play instead of scoring.
% \todo{Talk about poor data with defense, need to get creative with how to evaluate things \emph{not} happening.}
% Developing algorithms to identify the components necessary for successful groups in specific environments, and to evaluate agents in the context of those diverse, will benefit how AI and sports teams are constructed.
% Furthermore, algorithms that identify and evaluate roles could be used to improve various areas of a team's roster that need improvement.
% This is important information for general managers (GMs) when making trades or signing players to new contracts.

% The methods in which these algorithms are developed must incorporate rich context to achieve the goals of their development.
% This context may be player-specific, such as age or previous history of injury, to determine reliability and scope in the context of time.
% At the inter-player and inter-team level, context may include the balance of roles which need to be filled within a team (and with what resources if there is a salary cap), as well as the potential and expectation for competitors to react with their \emph{best response} strategy.
% This means that roster moves are not constructed in isolation, but in the context of the league (and division) that teams play in.
% For example, signing a player to a contract must require the contributions of that player to be assessed given the current state of the team's roster and their expected role to fill, subject to the team considering how their opponents will respond.
% These challenges include various levels of game theory, individual or group valuation, and agent or player cohesion which are all important problems to the multiagent community.


\subsection{Economic Strategies}

% \begin{itemize}
%     \item May have salary cap/luxury tax. Trade off of spending to the cap/over the cap for talent but giving up roster flexibility. Must take in context of season, relative value/how good team is/odds of winning (and value of winning in short term to sell off future). Future planning/projections in temporal dimension
%     \item COMSOC problems, how much of pie do you give to subset of players, who gets it (position/age/forward thinking)?
% \end{itemize}


Managing a professional sports team is inherently coupled with economic strategies.
% within various types of constraints.
Players are given salaries and monetary incentives based on their performance; however, management is often constrained by a salary cap or luxury tax, limiting the amount of capital a team can allocate.
Further constraints include contract term limits or the percent of salary cap allocated to one player.
% of money one player makes per-season.
% or the overall duration of a player's contract.


Operating within these constraints forces management to perform resource allocation while planning for future versions of their roster.
% This emulates many inter-agent interactions and planning across areas of MAS.
Rational behavior suggests players will accept the most competitive salary they are offered, while teams wish to offer as little capital in consideration of other negotiations.
The interaction between a player and team is similar to a 2-player donation game~\cite{Santos2021SocialNI}.
% The Nash Equilibrium of a team is to offer a small amount and the player's Nash Equilibrium is to accept any offered amount.
However, players can receive offers from other teams, increasing the complexity of the strategy space to create a particularly interesting domain for behavioral game theory.


A challenge for management beyond agreeing to a single player's contract is how to allocate a team's resources to actually acquire the types of players they identify through roster analysis and construction.
Salaries across sports have been shown to be consistent with a player's performance~\cite{garner2016business}; thus, a team trying to acquire all of the best performing players may run out of available resources or experience diminishing returns on investments.
Fairly allocating capital to players based on their marginal contribution can be modeled as resource allocation problem across $N$ agents, although the value of $N$ may not be well defined.
% ~\cite{brams1996fair}
Contracts typically last for a variable number of years, meaning strategies for allocating resources needs to plan for longer time horizons.
The financial and strategic challenges of constructing and maintaining a sports team create several dimensions of challenging multiagent problems.
This provides an interesting domain to support further multiagent research where game theoretical incentives are dynamic with opponent team strategies, roster composition, and contract landscape.
% The requirement for management to plan into the future adds uncertainty about the number of players in the cake-cutting state space, since player contracts can be a variable number of years and players may retire or leave for other teams.
% As a result, management must consider fair allocation of resources to roles they find valuable within the disposition of the team they are trying to create.
Strategic analyses that inform any aspect of financial decisions must properly model downstream impacts on other areas of management and the behavioral incentives of other teams or players.
% that are of significant interest to the MAS and game theory communities.
% Progress in these layered and time-variant management issues will drive progress in how teams are managed and how economic theories and MAS are developed.


% These restrictions induce multiple levels of challenges within a single marketplace that force management to strategize about the overall financial composition of a roster into the future.
% Creating accurate models to assist management in strategizing about the financial landscape of a team must incorporate multiagent interactions.
% These models are interconnected with the roster valuation work discussed in Section~\ref{sec:management_valuation}, as team disposition and playing style ultimately impacts the financial landscape teams are capable of operating in.

% Teams must be able to identify important areas of their roster composition and allocate financial resources to those areas while also considering its impact on the remaining roster.
% This marketplace contains multiple types of incentives.
% Rational behavior suggests players wish to make as much money for their services while teams try to give players as little money as possible to have as many resources available to construct other areas of a roster.
% Such an ecosystem creates multiple types of dilemmas in which players, agents, and teams operate with the added dimension of time.
% The direct interaction between players or agents and a team deciding on a contract can be modeled as a 2-player donation game; however, interactions between players and the external marketplace of all teams may increase the value they are willing to accept.

% From the perspective of a team's management, the process of fairly allocating capital to players based on their marginal contribution can be modeled as a cake-cutting game across $N$ agents.
% Furthermore, there is uncertainty about the value of $N$ since player contracts can be a variable number of years, and players in certain roles age and may not be available in the future.
% As a result, management must consider fair allocation of resources to roles they find valuable within the disposition of the team they are trying to create.
% Progress in these layered and time-variant management issues will drive progress in how teams are managed and how economic theories and MAS are developed.

\noindent
\ding{229} Fair Allocation \\
\ding{229} Markets, Auctions, and Non-Cooperative Game Theory




% \subsection{Markets and Negotiation}

% \begin{itemize}
%     \item free agents, open market for UFAs, need to consider current cap space, market price in context of what team can afford, also potential for cap to go up in future (turn large contract into a bargain over time \todo{must be a word for this regarding interest... similar to discounting the future in RL?}).
%     \item Strategic contracts/offers based on what other teams have. This needs to consider the league AND those in your division/direct competitors (not all competition is created equal).
%     \item Lots of COMSOC questions here too regarding making offers for a large portion of cap space.
% \end{itemize}





% Consider a scenario where a team trades for an expensive superstar, but the player never reaches their full potential on the new team.
% While the player is known to be of high quality, they likely did not pair well within the other players they were paired with.
% Returning to the scenario of trading for an expensive superstar, perhaps such multiagent models could help identify a less-expensive player that flourished in the environment of the team.
