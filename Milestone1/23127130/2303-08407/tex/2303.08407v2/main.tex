\documentclass[12pt]{iopart}

% packages

\usepackage[dvips]{graphicx} % for figures
\expandafter\let\csname equation*\endcsname\relax
\expandafter\let\csname endequation*\endcsname\relax
\usepackage{amsfonts,amscd,amsmath,amsthm}
%\usepackage{iopams}
\usepackage{enumerate}
\usepackage{epsfig}
\usepackage{subfigure}
\usepackage{xcolor}
\usepackage[colorlinks = true]{hyperref}
\usepackage{physics}
\usepackage{booktabs}
\usepackage{epstopdf}
%\usepackage[expert]{mathdesign}
\usepackage{framed}
\usepackage{multirow}
\usepackage{color}
\usepackage{longtable}
\usepackage{comment}
\usepackage[ruled,vlined]{algorithm2e}
\usepackage[most]{tcolorbox}
\usepackage[T1]{fontenc}


\graphicspath{{./figure/}}

\usepackage{tikz}
\usetikzlibrary{tikzmark, calc, fit, positioning}
\usetikzlibrary{shapes,arrows}
\usetikzlibrary{quantikz2}

\newtheorem{theorem}{Theorem}
\newtheorem{lemma}{Lemma}
\newtheorem{corollary}{Corollary}
\newtheorem{claim}{Claim}
\newtheorem{conjecture}{Conjecture}
\newtheorem{postulate}{Postulate}
\newtheorem{observation}{Observation}
\newtheorem{definition}{Definition}
\newtheorem{proposition}{Proposition}
\newtheorem{presumption}{Presumption}
\newtheorem{remark}{Remark}
\newtheorem{example}{Example}
\newtheorem{criterion}{Criterion}
\newtheorem{question}{Question}
%\newenvironment{solution}{{\noindent\it \textbf{Solution}:}\quad}{\hfill $\square$\par}
\newenvironment{solution} {\begin{proof}[Solution]} {\end{proof}}

\newtcolorbox[auto counter]{mybox}[2][]{
	enhanced,
	breakable,
	colback=blue!5!white,
	colframe=blue!75!black,
	fonttitle=\bfseries,
	title=Box \thetcbcounter: #2,#1
}

\newcommand{\zxj}[1]{{\color{red} #1}}


\begin{document}
\title{Interplay among entanglement, measurement incompatibility, and nonlocality}
\author{Yuwei Zhu$^{1,2}$, Xingjian Zhang$^2$, Xiongfeng Ma$^{2,*}$}
\address{$^1$ Yau Mathematical Sciences Center, Tsinghua University, Beijing 100084, P.~R.~China}
\address{$^2$ Center for Quantum Information, Institute for Interdisciplinary Information Sciences, Tsinghua University, Beijing 100084, P.~R.~China}
\address{$^*$ Author to whom any correspondence should be addressed.}
\eads{\mailto{xma@tsinghua.edu.cn}}

\begin{abstract}
Nonlocality, manifested by the violation of Bell inequalities, indicates entanglement within a joint quantum system. A natural question is how much entanglement is required for a given nonlocal behavior. Here, we explore this question by quantifying entanglement using a family of generalized Clauser-Horne-Shimony-Holt-type Bell inequalities. Given a Bell-inequality violation, we derive analytical lower bounds on the entanglement of formation, a measure related to entanglement dilution. The bounds also lead to an analytical estimation of the negativity of entanglement. In addition, we consider one-way distillable entanglement tied to entanglement distillation and derive tight numerical estimates. With the additional assumptions of qubit-qubit systems, we find that the relationship between entanglement and measurement incompatibility is not simply a trade-off under a fixed nonlocal behavior. Furthermore, we apply our results to two realistic scenarios --- non-maximally entangled and Werner states. We show that one can utilize the nonlocal statistics by optimizing the Bell inequality for better entanglement estimation.
\end{abstract}

\noindent{\it Keywords\/}: device-independent, entanglement quantification, nonlocality, Bell inequality, incompatibility
%{\color{blue}ZXJ: Negativity? We may also mention that our results improve previous bounds}

\maketitle


\section{Introduction}
In the early development of quantum mechanics, Einstein, Podolsky and Rosen noticed that the new physical theory leads to a ``spooky action'' between separate observables that is beyond any possible classical correlation~\cite{einstein1935can}. Later, Bell formalizes such a quantum correlation via an experimentally feasible test that is now named after him~\cite{bell1964einstein}. In one of the simplest settings, the Clauser-Horne-Shimony-Holt (CHSH) Bell test~\cite{clauser1969proposed}, two distant experimentalists, Alice and Bob, each has a measurement device and share a pair of particles. While they may not know their devices and physical system \emph{a priori}, they can each take random measurements and later evaluate the Bell expression as shown in Fig.~\ref{fig:CHSH}, 
%on their own particle, labelled as $x,y\in\{0,1\}$. The measurement devices output binary results, denoted as $a,b\in\{+1,-1\}$, respectively. Afterwards, Alice and Bob evaluate the CHSH Bell expression,
\begin{equation}
 S=\sum_{a,b,x,y}ab(-1)^{xy}p(a,b|x,y)=\sum_{x,y}(-1)^{xy}\mathbb{E}(ab|x,y),
 \label{CHSH}
\end{equation}
where $x,y\in\{0,1\}$ represent their random choices of measurement settings and $a,b\in\{+1,-1\}$ denote their measurement outcomes, $p(a,b|x,y)$ denotes the outcome probability conditioned on the inputs, and $\mathbb{E}(ab|x,y)$ is the expected value of the product value, $ab$, conditioned on the tuple of inputs, $(x,y)$.
If Alice and Bob observe a value of $S>2$, then they cannot explain the observed correlation using any physical theory that follows local realism. We call such an observation the violation of a Bell inequality, and the statistics exhibit Bell nonlocality. To demonstrate such a nonlocal behavior, the physical systems must exhibit a non-classical feature, wherein quantum theory, entanglement is such an ingredient~\cite{schrodinger1935discussion}. The CHSH expression has a maximal value of $2\sqrt{2}$, which requires the maximally entangled state in a pair of qubits, $\ket{\Phi^+}=(\ket{00}+\ket{11})/\sqrt{2}$~\cite{cirelson1980quantum}. We also term this state the Bell state.

\begin{figure}[hbt!]
%\centering \includegraphics[width=0.28\textwidth]{CHSH_Bell_test.jpg}
\centering
\tikzstyle{arrow} = [thick,->,>=stealth]
\tikzstyle{block} = [rectangle, rounded corners, align=center, minimum width=1.6cm, minimum height=1cm,text centered,draw=black]
\begin{tikzpicture}[node distance = 2cm, auto]	
	\node [block]  (A) {Alice};
	\node [block, right =2cm of A] (B) {Bob};

	%% paths
	\draw[arrow,<-] (A.north) --++ (0,.5) node[above] {$x\in\{0,1\}$};
	\draw[arrow] (A.south) --++ (0,-.5) node[below] {$a\in\{\pm 1\}$};
	\draw[arrow,<-] (B.north) --++ (0,.5) node[above] {$y\in\{0,1\}$};
	\draw[arrow] (B.south) --++ (0,-.5) node[below] {$b\in\{\pm 1\}$};
%	\draw[-latex,decorate,decoration={snake,post length=2mm},thick,gray,opacity=0.5] (A.east) -- (B.west);
	\draw[-,decorate,decoration={snake},thick,gray,opacity=0.5] (A.east) -- (B.west);
\end{tikzpicture}
\caption{A diagram of the CHSH Bell test. Two space-like separated users, Alice and Bob, share an unknown quantum state and own untrusted devices. In each round of the CHSH Bell test, Alice applies the measurement determined by the random input $x\in\{0,1\}$ and outputs her measurement result, $a\in\{\pm 1\}$. The measurement process is similar on Bob’s side, with input $y$ and output $b$. As the round of tests accumulated, the CHSH Bell value $S$ in Eq.~\eqref{CHSH} can be evaluated.}
\label{fig:CHSH}
\end{figure}


Entanglement characterizes a joint physical state among multiple parties that cannot be generated through local operations and classical communication (LOCC)~\cite{guhne2009entanglement,horodecki2009quantum}. Beyond its role in understanding quantum foundations, entanglement is a useful resource in a variety of quantum information processing tasks, including quantum communication~\cite{curty2004entanglement}, quantum computation~\cite{jozsa2003role}, and quantum metrology~\cite{giovannetti2011advances}. 
With a resource-theoretic perspective, a large class of information processing operations can be interpreted as entanglement conversion processes under LOCC, wherein one may quantify the participation of entanglement using appropriate measures~\cite{vedral1997quantifying,lo1999unconditional,shor2000simple,bennett1996mixed}.
%one can interpret and quantify the participation of entanglement in particular information processing tasks~\cite{vedral1997quantifying}. 
%For instance, in analysing the communication task of quantum key distribution, where remote parties aim at sharing an identical and private key string, one can first transform the key distribution to entanglement distribution~\cite{lo1999unconditional,shor2000simple}. The number of distillable private keys is closely related to the measure of distillable entanglement in the initial distributed quantum state, which quantifies the number of Bell states that can be distilled via LOCC~\cite{bennett1996mixed}.
Therefore, the fundamental question is to detect and quantify entanglement in a system. While state tomography fully reconstructs the information of a quantum state and hence the entanglement properties~\cite{raymer1994complex,leonhardt1996discrete,leonhardt1997measuring}, 
%one can learn the full information of a quantum state and then indicate its entanglement properties via various entanglement criteria and measures. However, 
the validity of results relies on the trustworthiness of the detection probes. As detection loss and environmental noise are inevitable in practice, the realistic probes may deviate from the ideal ones~\cite{goh2019experimental}. Even worse, when the malfunction is too severe or the measurements are controlled by adversaries, tomography can lead to false entanglement detection for separable states~\cite{xu2014implementation,yuan2016reliable}.

Fortunately, quantum nonlocality provides a way to bypass the problem. Note that in the Bell test, one does not need to characterize the quantum devices \emph{a priori}, and thus, the indication of entanglement from Bell nonlocality is a device-independent (DI) conclusion~\cite{mayers1998quantum,acin2007device}. This observation leads to the question of what the minimum amount of entanglement is necessary for a given nonlocal behavior.
%We can ask the following question: What is the least amount of entanglement necessary for a given nonlocal behavior?
In other words, Bell tests can serve as a DI entanglement estimation tool. 
%In practice, as we cast entanglement into fewer parameters in a nonlocal behavior, the quantification can be more statistically robust.
In the literature, there are already endeavors into the question~\cite{verstraete2002entanglement,liang2011semi,moroder2013device,toth2015evaluating,arnon2017noise,chen2018exploring,arnon2019device}. The quantitative results provide us with tools for devising novel quantum information processing tasks. A notable investigation is the analysis of DI quantum key distribution~\cite{ekert1992quantum,mayers1998quantum,acin2007device}. With the link among nonlocality, entanglement, and secure communication, we can quantify the key privacy solely from Bell nonlocality~\cite{arnon2018practical,zhang2020efficient,zhang2021quantum}.

%{\color{red}different entanglement measures}

Despite the physical intuition for entanglement estimation via Bell nonlocality, the quantitative relation between entanglement and nonlocality can be subtle~\cite{acin2012randomness}. Above all, while the notion of Bell nonlocality arises from the observation of correlations, entanglement is defined by the opposite of a restricted state preparation process. In fact, the Bell nonlocality is a stronger notion than entanglement. Though a nonlocal behavior necessarily requires the presence of entanglement, not all entangled states can unveil a nonlocal correlation~\cite{werner1989quantum,barrett2002nonsequantial}. The conceptual difference even leads to some counter-intuitive results, where a series of works aimed at characterizing their exact relation, such as the discussions on the Peres conjecture --- whether Bell nonlocality is equivalent to distillability of entanglement~\cite{peres1999all,vertesi2014disproving}.

In addition, different entanglement measures may enjoy distinct operational meanings. In general, these measures are not identical to each other. Particularly, there exist quantum states of which the entanglement cost is strictly higher than the distillable entanglement~\cite{lami2023no}. The two measures correspond to the operations of entanglement dilution and entanglement distillation, respectively. Such a phenomenon exhibits the irreversibility of the entanglement theory. Furthermore, estimations of different entanglement measures from the same nonlocal behavior can differ. Taking the CHSH Bell expression as an example, it witnesses a non-trivial value of the negativity of entanglement as long as there is a Bell-inequality violation~\cite{moroder2013device}. On the other hand, the estimation of the negative conditional entropy of entanglement remains zero for a range of the Bell-inequality violation that is not high enough~\cite{arnon2019device}.
%within a range of CHSH Bell value starting from $2$, the CHSH local hidden variable modal upper bound, the estimation result of conditional entropy fails to $0$~\cite{arnon2019device}; when taking negativity as the entanglement measure, there is no zero lower bound over the entire $(2,2\sqrt{2}]$ interval~\cite{moroder2013device}.

%{\color{blue}ZXJ: the last sentence seems irrelevant.}

In this work, we systematically study entanglement estimation via a family of generalized CHSH-type Bell inequalities. We treat the measurement devices as black boxes. Different implementations can lead to the same observed nonlocal behavior. As depicted in Fig.~\ref{fig:interplay}, a nonlocal behavior necessarily needs both entanglement and incompatible local measurements. In other words, a system with separable states or compatible local measurements definitely fails to observe nonlocality. One may expect a trade-off relationship between state entanglement and measurement incompatibility for a given nonlocal behavior. Hence, we explore the interplay among entanglement, nonlocality, and measurement incompatibility with different entanglement measures. 

\begin{figure}[hbt!]
\centering 
\tikzstyle{arrow} = [thick,->,>=stealth]
\tikzstyle{block} = [rectangle, rounded corners, align=center, minimum width=2cm, minimum height=1cm,text centered,draw=black]
\begin{tikzpicture}[node distance = 2cm, auto]	
	\node[block]  (N) {Nonlocality};
	\node[block] (E) at ($(N)+(220:3cm)$) {Entanglement};
	\node[block] (M) at ($(N)+(-40:3cm)$) {Measurement \\ incompatibility};
	\node[below =.5cm of N] {interplay};

%% paths
	\draw[arrow,->] (N.south) -- (E.north);
	\draw[arrow,->] (N.south) -- (M.north);
	%	\draw[-latex,decorate,decoration={snake,post length=2mm},thick,gray,opacity=0.5] (A.east) -- (B.west);
	\draw[latex-latex,thick,red] (E.east) -- (M.west) node[midway,below,black] {\color{red} trade-off};
\end{tikzpicture}
\caption{The interplay among nonlocality, entanglement, and measurement incompatibility. A nonlocal behavior necessarily indicates both entanglement and incompatible local measurements. A system with separable states or compatible local measurements fails in exhibiting nonlocality. Intuitively, under a given nonlocal behavior, one may expect a trade-off relationship between entanglement and measurement incompatibility. In this work, we start from the entanglement estimation via nonlocality, from which we realize that the relation between entanglement and measurement incompatibility is subtler than a simple trade-off. We study the interplay among nonlocality, entanglement, and measurement incompatibility in detail.}

%In this work, we mainly focus on quantifying entanglement via nonlocality. Given a fixed nonlocal behavior, there may be a trade-off relation between entanglement and measurement incompatibility. For this reason, nonlocality may not fully reflect the entanglement properties in the system.
\label{fig:interplay}
\end{figure}

\begin{comment}
Entanglement acts as a resource of quantum correlations in experiments concerning quantum information theory and quantum communication. In order to quantify the amount of entanglement in a given quantum state, the concept of entanglement measure(or entanglement monotones) was proposed\cite{vedral1997quantifying} and further developed into a series of axioms\cite{vidal2000entanglement}.
Entanglement of formation(EOF) is first defined as the von Neuman entropy of subsystem for pure states, via convex roof construction to mixed states in \cite{bennett1996mixed}. EOF is later expressed as a function of concurrence, since Wooter first proposed concurrence as an convex and monotone function and thus an entanglement measure for entangled pairs of qubits in \cite{PhysRevLett.78.5022,rungta2001universal}.
One-way entanglement distillation is an entanglement measure for (....). The conditional max smooth entropy acts as a lower bound for one-way entanglement distillation of a bipartite state. By applying entropy accumulation theorem in i.i.d channel, the total max smooth entropy is bounded by the conditional entropy in a single round of experiment\cite{wilde2017converse}.

Nonlocality signifies that a resulting statistics can not be expressed by a local hidden variable model\cite{PhysicsPhysiqueFizika.1.195}. Bell inequality was first proposed by Bell to give a quantitative way to certify EPR paradox. In experiments, passing Bell test, or violation of Bell inequality implies nonlocality, and in terms of quantum mechanics, it involves entanglement and complementary measurements. Violation of a kind of Bell inequality, such as Clauser-Horne-Shimony-Holt (CHSH) Bell inequality\cite{PhysRevLett.23.880} provides a device independent(DI) approach to verify entanglement, where no assumption is made on the quantum system or measurement device. Further, the magnitude of violation gives a DI method to quantify entanglement in the quantum system\cite{verstraete2002entanglement,liang2011semi}.

A type of asymmetric CHSH inequality is used to derive entropy bound in device independent(DI) quantum key distribution\cite{Woodhead2021deviceindependent}. In DI entanglement estimation framework, this type of CHSH expression can estimate how much entanglement there is at least in this two qubit state. When a statistical expectation $S_\alpha$ violates CHSH-type inequality for some fixed $\alpha$, we give a theoretical lower bound of concurrence and a numerical upper bound of conditional entropy to quantify entanglement in the quantum system.
In actual experiments, thing happens where there are partial assumptions about the experimental device, measurement device or source, that we call a semi-DI framework. When the measurement device is trusted, the entanglement of input state can be quantified more precisely with trusted measurement device information. In this framework, the minimizing of the entanglement estimation can be mathematically and experimentally interpreted as a dual problem to the maximizing of CHSH-type violation under given trusted measurement device.

Any two qubit pure state is isometry to a ``tilted EPR state": $\cos\delta\ket{00}+\sin\delta\ket{11}$. When the two amplitude equals, it degenerates to maximally entangled state.
Werner state is the quantum state that is invariant under all local unitary operators. Two qubit Werner state can be written as maximally entangled state mixed with white noise, with probability equals to $p$, which is of the form $(1-p)\ket{\Phi^+}\bra{\Phi^+}+p\frac{I}{4}$\cite{werner1989quantum}. 
In CHSH-type test, when the input state is the two sets of state above, We observe that under some specific settings, our DI entanglement estimation results are also reasonable when $\alpha>1$, and it even performs better as an estimation of entanglement measure. This informs us that, in some DI cases, by increasing $\alpha$ to utilize output statistics, we can receive an estimated entanglement measure close to its real one. 
\end{comment}
%{\color{red}Subtle issue 2: does nonlocality indicate consistent performance for different entanglement measures? (later in our results: same amount of conditional entropy, but different concurrence, all to the same Bell value; might related to the partial order issue of entanglement measures)}
The rest of the paper is organized as follows. In Sec.~\ref{sc:Pre}, we review the necessary concepts in nonlocality and entanglement theories. In Sec.~\ref{sc:framework}, we present the general framework for estimating entanglement in the underlying system using the set of generalized CHSH-type Bell inequalities, namely the tilted CHSH Bell inequalities. Then, we consider three special entanglement measures: the entanglement of formation (EOF), the one-way distillable entanglement, and the negativity of entanglement. For a given Bell-inequality violation, we derive analytical estimation results for the EOF and negativity measures. For one-way distillable entanglement, we obtain tight numerical estimation results. In Sec.~\ref{sc:interplay}, we utilize the entanglement estimation results and investigate the interplay among nonlocality, entanglement, and measurement incompatibility. Particularly, when the underlying state is known to be a pair of qubits, we observe the relation between entanglement and measurement incompatibility under a given nonlocal behavior to be more complex than a simple trade-off. From a practical perspective, in Sec.~\ref{sc:numerical}, we also simulate statistics that arise from pure entangled states and Werner states and examine the performance of our results. 



\section{Preliminary and previous work}\label{sc:Pre}
\subsection{General CHSH-type Bell tests}

In this work, we consider the family of generalized CHSH-type Bell tests. Under quantum mechanics, the Bell expression is given by~\cite{acin2012randomness}
\begin{equation}
\label{CHSH_type}
\begin{split}
    S &=\Tr[\rho_{AB}\left(\alpha\hat{A_0}\otimes \hat{B_0}+\alpha \hat{A}_0\otimes \hat{B}_1+\hat{A}_1\otimes \hat{B}_0-\hat{A}_1\otimes \hat{B}_1\right)] \\
    &=\Tr(\rho_{AB}\hat{S}_{\alpha}),
\end{split}
\end{equation}
where $\rho_{AB}$ is the underlying bipartite quantum state, $\hat{A}_{x}$ and $\hat{B}_{y}$ are the observables measured by Alice and Bob, according to their measurement choices, $x,y\in\{0,1\}$, respectively. The family of Bell expressions is parameterized by $\alpha\geq1$, which tilts the contributions of $\hat{A}_0\otimes(\hat{B}_0+\hat{B}_1)$ and $\hat{A}_1\otimes(\hat{B}_0-\hat{B}_1)$ to the Bell value. When $\alpha=1$, Eq.~\eqref{CHSH_type} degenerates to the original CHSH expression defined by Eq.~\eqref{CHSH}~\cite{clauser1969proposed}. For simplicity, we call the expression under the fixed parameter of $\alpha$ as $\alpha$-CHSH expression and $\hat{S}_{\alpha}$ the $\alpha$-CHSH operator. If the underlying quantum state is separable or the local measurement observables are compatible, the $\alpha$-CHSH expression is upper bounded by $S(\alpha)\leq 2\alpha$, which commits a local hidden variable model to reproduce the correlation. In quantum theory, the largest value of $\alpha$-CHSH expression is $2\sqrt{\alpha^2+1}$~\cite{acin2012randomness}. Observation of a Bell value in $(2\alpha,2\sqrt{\alpha^2+1}]$, termed Bell-inequality violation, necessarily implies the existence of entanglement between two local systems and measurement incompatibility between the local measurement observables~\cite{acin2012randomness}. The generalized CHSH-type Bell inequalities have been used for parameter estimation in tasks of DI randomness generation~\cite{acin2012randomness,wooltorton2022tight} and quantum key distribution~\cite{Woodhead2021deviceindependent}, which can outperform the original CHSH inequality in certain practical cases.

In the study of Bell nonlocality, we do not put prior trust in the underlying physical systems. In particular, we do not assume a bounded system dimension. Nevertheless, the simplicity of CHSH-type Bell expressions allows us to apply Jordan's lemma to effectively reduce the system to a mixture of qubit pairs~\cite{acin2007device}. We shall explain how to apply this result when we come to the part of main results. 

\subsection{Entanglement measures}
\label{subsc:ent measures}
%{\color{blue}ZXJ: We need to give the general definition of the entanglement of formation for all dimensions, and then state that it converges to Eq. (3) in the qubit-pair case.}

%As our discussion is effectively restricted to a pair of qubits, we shall focus on a few entanglement measures that enjoy a closed form in such a case. We consider $\rho_{AB}$ as a pair of qubits here.
In our work, we study entanglement estimation in a bipartite system, $\rho_{AB}\in\mathcal{D}(\mathcal{H}_A\otimes\mathcal{H}_B)$, where we use $\mathcal{D}$ to denote the set of all density operators acting on the associating Hilbert space. 
The first measure we consider is the EOF, $E_{\mathrm{F}}(\rho_{AB})$~\cite{bennett1996mixed}. Operationally, this measure provides a computable bound on the entanglement cost, which quantifies the optimal state conversion rate of diluting maximally entangled states into the desired states of $\rho_{AB}$ under LOCC~\cite{bennett1996mixed}. For a pure state, the EOF equals the entanglement entropy, $E_{\mathrm{F}}(\ket{\phi}_{AB})=H(\rho_A)=H(\rho_B)$, where $\rho_A$ and $\rho_B$ denote the partial state of system $A$ and $B$, respectively, and $H(\cdot)$ represents the von Neumann entropy. When extended to a general state, the EOF is defined via a convex-roof construction,
\begin{equation}
    E_{\mathrm{F}}(\rho_{AB})=\min_{\{p_i,\ket{\phi}_i\}_i}\sum_{i}p_i E_{\mathrm{F}}(\ket{\phi_i}_{AB}),
    \label{EOF}
\end{equation}
where the optimization is taken over all possible pure-state decomposition, $\rho_{AB}=\sum_i p_i \ketbra{\phi_i}_{AB},\sum_ip_i=1,\forall p_i,p_i\geq0$. When restricting $\rho_{AB}$ to the region of two-qubit states, the EOF measure takes a closed form~\cite{hill1997entanglement}, 
\begin{equation}
    E_{\mathrm{F}}(\rho_{AB})=h\left(\frac{1+\sqrt{1-C^2(\rho_{AB})}}{2}\right),
\label{eof}
\end{equation}
where $h(p)=-p\log p-(1-p)\log (1-p)$ is the binary entropy function for $p\in[0,1]$, and $C(\rho_{AB})$ is the concurrence of $\rho_{AB}$, a useful entanglement monotone~\cite{hill1997entanglement,rungta2001universal}. For a general two-qubit quantum state, $\rho_{AB}$, its concurrence is given by
\begin{equation}\label{eq:concurrence}
    C(\rho_{AB})=\max\{0,\lambda_1-\lambda_2-\lambda_3-\lambda_4\},
\end{equation}
where values $\lambda_i$ are the decreasingly ordered square roots of the eigenvalues of the matrix
\begin{equation}
    X(\rho_{AB})=\sqrt{\rho_{AB}}(\sigma_y\otimes\sigma_y)\rho_{AB}^*(\sigma_y\otimes\sigma_y)\sqrt{\rho_{AB}}.
\end{equation}
Here, the density matrix of $\rho_{AB}$ is written on the computational basis of $\{\ket{00},\ket{01},\ket{10},\ket{11}\}$, where $\ket{0}$ and $\ket{1}$ are the eigenstates of $\sigma_z$, and $\rho_{AB}^*$ is the complex conjugate of $\rho_{AB}$.

%The second entanglement measure we consider is the negative conditional entropy, given by $-H(A|B)$, where $H(A|B)=H(AB)-H(B)$ is the conditional von Neuman entropy of state $\rho_{AB}$. This measure provides a lower bound on the one-way distillable entanglement of $\rho_{AB}$, which quantifies the largest possible state conversion rate of distilling Bell states from many copies of $\rho_{AB}$. [Ref]

As opposed to the entanglement dilution process, the entanglement distillation process defines another entanglement measure, the distillable entanglement~\cite{bennett1996mixed}. In this process, given sufficiently many copies of a given state, $\rho_{AB}$, the distillable entanglement is the maximal state conversion rate of distilling maximally entangled states under LOCC. While calculating this measure for a general state remains open, a well-studied lower bound is the one-way distillable entanglement, where classical communication is restricted to a one-way procedure between the two users. In the Shannon limit, where one takes infinitely many independent and identical  copies of the quantum state, the average distillation rate under one-way LOCC can be calculated by the negative conditional entropy~\cite{wilde2017converse} (see Sec.~VIB therein),
%The second entanglement measure we consider is the one-way distillable entanglement $E_D^\rightarrow(\rho_{AB})$. When Alice and Bob are allowed for local operations and only Alice is allowed to send classical information to Bob, $E_D^\rightarrow(\rho_{AB})$ quantifies the rate of distilling maximally entangled state from their shared bipartite state. When Alice and Bob share multiple independent and identical pairs of $\rho_{AB}$, a limitation of average negative conditional entropy is convergent to $E_D^\rightarrow(\rho_{AB})$
\begin{equation}
    %E_D^\rightarrow(\rho_{AB})=-\lim_{n\rightarrow\infty}\frac{1}{n}H(\rho_{AB}^{\otimes n}|\rho_{B}^{\otimes n}).
    E_D^\rightarrow(\rho_{AB})=-H(A|B)_{\rho},
\end{equation}
where $H(A|B)_{\rho}=H(\rho_{AB})-H(\rho_B)$. This result generalizes the finding in Ref.~\cite{bennett1996mixed}, which shows one-way distillable entanglement in the Shannon limit is $1-H(\rho_{AB})$ when $\rho_{AB}$ is a mixture of Bell states (see Sec.~IIIB3 therein). When the underlying state is clear from the context, we shall omit the subscript for simplicity. 

Another entanglement measure we aim to quantify is the negativity of entanglement. This measure is defined in terms of the violation of the positive partial transpose (PPT) criteria~\cite{vidal2002computable},
\begin{equation}
\mathcal{N}(\rho_{AB})=\frac{\|\rho_{AB}^{\mathrm{T}_{A}}\|_1-1}{2}=\sum_{\lambda_i(\rho_{AB}^{\mathrm{T}_{A}})<0}|\lambda_i(\rho_{AB}^{\mathrm{T}_{A}})|,
\label{negativity}
\end{equation}
where $(\cdot)^{\mathrm{T}_{A}}$ is the partial trace operation on subsystem $A$ on the computational basis and $\|\cdot\|_1$ is the trace norm of a matrix. In the second equality of Eq.~\eqref{negativity}, $\lambda_i(\cdot)$ represents the eigenvalues of a matrix. Note that a related measure, namely the logarithm of negativity, $E_{\mathcal{N}}(\rho_{AB})=\log\|\rho_{AB}^{\mathrm{T}_{A}}\|_1$, upper-bounds the distillable entanglement and is hence no less than the negative conditional entropy of entanglement~\cite{guhne2009entanglement}. Notably, the negativity of entanglement is closely related to the concurrence for a pair of qubits given in Eq.~\eqref{eq:concurrence}. Consider the underlying state to be a mixture of two-qubit Bell states,
\begin{equation}
\label{Bell_diagonal}
    \rho_{\lambda}=\lambda_{1}\ketbra{\Phi^+}+\lambda_{2}\ketbra{\Phi^-}+\lambda_{3}\ketbra{\Psi^+}+\lambda_{4}\ketbra{\Psi^-},
\end{equation}
with $\ket{\Phi^\pm}=(\ket{00}\pm\ket{11})/\sqrt{2},\ket{\Psi^\pm}=(\ket{01}\pm\ket{10})/\sqrt{2}$. We term such a state a Bell-diagonal state. Without loss of generality, we assume $\lambda_1\geq\lambda_2\geq\lambda_3\geq\lambda_4$, since we can relabel the eigenvalues corresponding to the Bell-basis states with local unitary operations. The negativity of entanglement for $\rho_{\lambda}$ is given by
\begin{equation}
    \mathcal{N}(\rho_\lambda)=\max\left\{\frac{1}{2}(\lambda_1-\lambda_2-\lambda_3-\lambda_4),0\right\}=\max\left\{\lambda_1-\frac{1}{2},0\right\}.
\end{equation}
Meanwhile, the concurrence of the state is
\begin{equation}
    C(\rho_{\lambda})=\max\{2\lambda_1-1,0\},
\end{equation}
which is exactly twice the negativity of entanglement.

%{\color{blue}ZXJ: in what special cases? (we may better state it explicitly, ``the two measures are equivalent up to a constant in the qubit-pair case, ...'')}

%We remark that the above results of entanglement measures are defined via dilution and distillation processes with infinitely many independent and identical (i.i.d.) copies of the quantum state under study or in the Shannon limit. 


\subsection{Previous work and summary of our results}
In this subsection, we briefly overview the previous findings and summarize our contributions regarding the estimation of entanglement via nonlocality in Table~\ref{tabel:previous_work}. Note that the results in Ref.~\cite{verstraete2002entanglement} and~\cite{liang2011semi} pose an additional assumption on the system dimension, and a part of the results in Ref.~\cite{toth2015evaluating} utilize steering inequalities with full trust on the measurements of one of the parties in a nonlocal setting. Most works simply deal with probabilities, where both the amount of entanglement and the Bell value are taken as expected values. There are a few exceptional works that deal with finite data, including Ref.~\cite{arnon2017noise}, where the authors consider a parallel repetition of a Bell test, and Ref.~\cite{arnon2019device}, where a single-shot estimation of one-way distillable entanglement is given.

In this work, we will estimate the three entanglement measures listed above using the family of tilted CHSH Bell expressions defined in Eq.~\eqref{CHSH_type}. For simplicity, we focus on the expected values. We utilize the tilted CHSH inequalities and obtain tight estimation results. For the negativity of entanglement, we obtain an analytical tight lower bound, proving a conjecture raised from numerical evidence in Ref.~\cite{moroder2013device}. For the EOF, in comparison to the loose estimation in Ref.~\cite{arnon2017noise}, we obtain tight estimation results for the family of tilted CHSH expressions. For the one-way distillable entanglement in the Shannon limit, we obtain tight numerical lower bounds via generalized CHSH Bell expressions in Eq.~\eqref{CHSH_type}. The result for the original CHSH expression coincides with the analytical result obtained in Ref.~\cite{arnon2019device}. 

Note that the DI estimation result is given in terms of the expected values in our study. That is, we present lower bounds on the state entanglement, given the underlying expected Bell value. When implementing entanglement estimation in an experiment, one needs to estimate the Bell value from a finite sample. Also, in a fully DI scenario, the samples may not follow an independent and identical distribution (i.i.d.). For this purpose, one needs to apply statistical methods valid for non-i.i.d. statistics. Notably, the entropy accumulation theorem (EAT) allows us to deal with entropic-based entanglement measures~\cite{arnon2019device}. Using EAT, our one-way distillable entanglement estimation result can be lifted to a finite data-size version over non-i.i.d. statistics~\cite{wilde2017converse} when the Bell test is sequentially repeated in an experiment. In addition, martingale-based techniques may also be applied~\cite{zhang2023quantum}.

% refers to estimating entanglement from the $\alpha$-CHSH Bell value. Our work does not delve into the experimental implementation aspects, such as the protocol for repeating rounds in the Bell test or the utilization of finite-size statistics for parameter estimation. For detailed accounts of experimentally estimating DI, readers are directed to Ref.~\cite{arnon2019device} for the application of the entropy accumulation theorem in entropic quantity estimation and Ref.~\cite{zhang2023quantum} for general parameter estimation using the tool of martingale.


\begin{table}[!h]
\begin{center}
\caption{ Entanglement estimation results via nonlocality. In the works that utilize the Navascu\'es-Pironio-Ac\'in-type (NPA-type) hierarchy~\cite{Navascués2008a}, a numerical method, the results numerically converged. The other works give tight bounds on entanglement, except for the results in Ref.~\cite{arnon2017noise}, which utilize the rigidity property of Bell expressions or robust self-testing. }
\label{tabel:previous_work}
%Especially, \cite{moroder2013device} gives a numerically-convergent result for the negativity of entanglement.
\resizebox{\columnwidth}{!}{
\begin{tabular}{ccccc}
\hline
\hline
 Results & Nonlocality Feature & Entanglement measure & Assumption & Main technique\\ 
 \hline
 %Verstraete2002~
 \cite{verstraete2002entanglement} & CHSH inequality & Concurrence & Dimension & Analytical \\
 %Liang2011~
 \cite{liang2011semi} & Modified CH inequalities & Concurrence & Dimension & Analytical \\ 
 %Moroder2013~
 \cite{moroder2013device} & Multipartite Bell inequalities & Negativity & DI& NPA hierarchy\\
%Chen2018~
\cite{chen2018exploring} & Multipartite Bell inequalities & Robustness of entanglement & DI&NPA hierarchy\\
%T\'{o}th2015~
\cite{toth2015evaluating} & Steering inequalities & Linear entropy & One-sided DI & NPA hierarchy\\
& Bell inequalities &Linear entropy & DI & NPA hierarchy\\
%Arnon-Friedman2017~
\cite{arnon2017noise} & Threshold quantum games
%\footnote{With nonzero classical-quantum gap} 
& EOF & DI & Rigidity\\
%Arnon-Friedman2019~
\cite{arnon2019device} & CHSH inequality & One-way distillable entanglement & DI & Analytical\\
Our results & Tilted CHSH inequalities & Concurrence & Dimension & Analytical\\
& & EOF & Dimension \& DI & Analytical\\
& & Negativity & Dimension \& DI & Analytical\\
& & One-way distillable entanglement & Dimension \& DI & Numerical\\
\hline
\end{tabular}}
\end{center}
\end{table}

%{\color{blue}In Appendix~XXX, we shall review the general definitions with a finite number of copies, namely, their definitions in the one-shot regime, and briefly explain the route for their estimation via nonlocality in Appendix~XXX.}

%{\color{red}ZXJ: Notation for the estimated value of negative conditional entropy?}

%{\color{red}Note: 
%Strictly speaking, negative conditional entropy is not an entanglement measure: it is zero for some entangled states (sufficient criterion) \\
%Nice properties: bound for distillable entanglement/entanglement of cost; convexity \\
%Peres' conjecture
%Formal definition and finite size (one-shot)
%}


\section{Device-independent entanglement estimation}\label{sc:framework}
\subsection{Entanglement estimation via optimization}
In this section, we formulate the problem of entanglement estimation via Bell nonlocality. Using the nomenclature in quantum cryptography, we also term it DI entanglement estimation. After specifying a particular entanglement measure, $E$, we ask the minimal amount of entanglement in the initial quantum system that supports the observed Bell expression value,
\begin{equation}
\label{original_optm}
\begin{split}
E_{\rm est} &=\min_{\rho_{AB},\hat{A}_0,\hat{A}_1,\hat{B}_0,\hat{B}_1}  E(\rho_{AB}), \\
\text{s.t.}\quad
S &=\Tr\left(\rho_{AB}\hat{S}_{\alpha}\right),\\
\rho_{AB}&\geq 0, \\ \Tr(\rho_{AB})&=1.
\end{split}
\end{equation}
Here, we denote the estimated entanglement measure of $E$ from Bell nonlocality as $E_{\rm est}$. As clarified above, $\hat{S}_{\alpha}$ is the $\alpha$-CHSH operator, an operator function of the measurement observables.

The optimization problem is difficult to solve directly. First, it involves multiple variables, including the underlying quantum state and the measurement observables. Second, the system dimension is unknown, as reflected in Eq.~\eqref{original_optm} where the dimension of $\rho_{AB}$ is unspecified. To address these challenges, we undertake several steps, as illustrated in Fig.~\ref{fig:flowchart}. In the original formulation of Eq.~\eqref{original_optm}, we do not make any assumption on the measurements, which are general measurements characterized by positive operator-valued measures (POVMs). Given that there is no constraint on the system's dimension, Naimark's dilation theorem~\cite{neumark1943representation} allows us to incorporate all local degrees of freedom and extend the measurements to projective ones without any loss of generality. We elaborate this further in \ref{appendix:jordan}.

\begin{figure}[hbt!]
    \centering
        \tikzstyle{arrow} = [thick,->,>=stealth]
        \tikzstyle{block} = [rectangle, rounded corners, align=center, minimum width=2cm, minimum height=1cm,text centered,draw=black]
    \begin{tikzpicture}[node distance = 2cm, auto]       
        \node [block]  (b1) {Original problem \\ Eq.~\eqref{original_optm}};
        \node [block, below =1cm of b1] (b2) {Optimal measurements \\ Eq.~\eqref{dual_optm}};
        \node [block, below =1cm of b2] (b3) {Reduction to a\\ two-qubit system};
        \node [block, below =1cm of b3] (b4) {Reduction to \\ Bell-diagonal states};
    %% paths
           \draw [arrow] (b1.south) -- node {Duality} (b2.north);
            \draw [arrow] (b2.south) -- node {Jordan's lemma} (b3.north);
            \draw [arrow] (b3.south) -- node {LOCC} (b4.north);
    \end{tikzpicture}
    \caption{Steps for estimating entanglement via CHSH-type Bell inequalities. Step 1: The original entanglement estimation problem is formulated as Eq.~\eqref{original_optm}. The only constraint is the observed Bell value, $S$. Step 2: Using a duality argument, we consider the optimization problem in Eq.~\eqref{dual_optm}, which can be interpreted as maximizing the Bell value for a given quantum state, $\rho_{AB}$.
    %is obtained from Eq.~\eqref{original_optm} by optimizing the CHSH-type violation under $\rho_{AB}$. 
    The arguments in the optimal solution are regarded as the ``optimal measurements'' that lead to the maximal Bell value for the state.
    %We obtain the measurement variables: intuitively, fix a violation $S$, the least entanglement of state $\rho_{AB}$ is achieved when the corresponding measurements are optimal; mathematically, it can be interpreted as duality; 
    Step 3: By applying Jordan's lemma, we can view the measurement process as resulting from a convex combination of pairs of qubits.
    %Meanwhile, the convexity of an entanglement measure guarantees the lower bound; 
    Step 4: We can further restrict the qubit pairs to Bell-diagonal states in solving the optimization problem. We show that in the CHSH Bell test, any two-qubit state can be transformed to a Bell-diagonal state through LOCC without changing the $\alpha$-CHSH Bell value.} %The final problem to be solved is transformed into Eq.~\eqref{final_optm}.
    \label{fig:flowchart}
\end{figure}


%In the first step, we study the dual problem of Eq.~\eqref{original_optm}. In form, we expect a solution of $S^*=f(E_{\rm est})$ for some function $f$. Duality arguments guarantee that its inverse function, $f^{-1}(S^*)$, provides a valid lower bound on the original problem by taking $S^*=S$. Furthermore, Slater's condition holds in the problem hence the strong duality guarantees that the solution to the dual problem coincides with the original solution to the primal problem~\cite{boyd2004convex}. In solving the dual problem, as the objective function is bilinear in $\rho_{AB}$ and Bell operator $\hat{S}_{\alpha}$, the optimization equals the maximization over the two arguments individually, $S^* =\max_{\rho_{AB}} \max_{\hat{A}_0,\hat{A}_1,\hat{B}_0,\hat{B}_1}  \Tr\left(\rho_{AB}\hat{S}_{\alpha}\right)$. Hence as the first level of the dual problem, for a given quantum state, $\rho_{AB}$, we can optimize over the measurements to yield the largest Bell value. In other words, we search for the ``optimal measurements'' for a fixed underlying quantum state in terms of the Bell value,

In the first step, we use duality arguments and transform Eq.~\eqref{original_optm}. Note that the objective function in Eq.~\eqref{original_optm}, $E_{\rm est}:=f(S)$, is continuous and monotonously increasing in its argument $S$, hence having a well-defined inverse function. Consider the following problem,
%By definition, the dual problem of Eq.~\eqref{original_optm} can be written as
\begin{equation}
\label{dual_optm}
\begin{split}
S^* &=\max_{\rho_{AB},\hat{A}_0,\hat{A}_1,\hat{B}_0,\hat{B}_1}  \Tr\left(\rho_{AB}\hat{S}_{\alpha}\right), \\
\text{s.t.}\quad
E(\rho_{AB})&=E_{\rm est}, \\
\rho_{AB}&\geq0, \\
\Tr(\rho_{AB})&=1,
\end{split}
\end{equation}
where the objective function in Eq.~\eqref{dual_optm} comes from the inverse function of the original optimization problem, $S^*:=f^{-1}(E_{\rm est})$. In solving Eq.~\eqref{dual_optm}, as the objective function is bilinear in $\rho_{AB}$ and Bell operator $\hat{S}_{\alpha}$, the optimization equals the maximization over the two arguments individually, $S^* =\max_{\rho_{AB}} \max_{\hat{A}_0,\hat{A}_1,\hat{B}_0,\hat{B}_1}  \Tr\left(\rho_{AB}\hat{S}_{\alpha}\right)$. 
For the inner optimization, denote $S^*(\rho_{AB})=\max_{\hat{A}_0,\hat{A}_1,\hat{B}_0,\hat{B}_1} \Tr\left(\rho_{AB}\hat{S}_{\alpha}\right)$, which can be seen as the maximal $\alpha$-CHSH Bell value that can be obtained with $\rho_{AB}$. Then, we may equivalently solve Eq.~\eqref{original_optm} with the following optimization, 
%Finally, we inverse the outer maximization back and obtain an equivalent problem of Eq.~\eqref{original_optm} where the Bell value restriction is optimized firstly over the measurements to yield the largest Bell value,
\begin{equation}
\label{equivalent_optm}
\begin{split}
E_{\rm est}&=\min_{\rho_{AB}}E(\rho_{AB}), \\
\text{s.t.}\quad
S^*(\rho_{AB}) &= S,\\
\rho_{AB}&\geq0, \\
\Tr(\rho_{AB})&=1.
\end{split}
\end{equation}
For simplicity, we call the measurements that yield the maximal $\alpha$-CHSH Bell value for $\rho_{AB}$ the ``optimal measurements''.
%Under a given Bell value, 
%In other words, we search for the optimal measurements for a fixed underlying quantum state in terms of the Bell value first. The equivalence from Eq.~\eqref{original_optm} to Eq.~\eqref{equivalent_optm} implies that the condition of searching for optimal measurements is necessary when the underlying system reaches the least quantified amount of entanglement via a given CHSH-type Bell value.
\begin{definition}
The optimal measurements of state $\rho_{AB}$ are the observables that maximize the $\alpha$-CHSH expression in Eq.~\eqref{CHSH_type} for $\rho_{AB}$, i.e., $\operatorname{argmax}_{\hat{A}_0,\hat{A}_1,\hat{B}_0,\hat{B}_1} \Tr\left(\rho_{AB}\hat{S}_{\alpha}\right)$.
\end{definition}

To bypass the dimension problem, we utilize Jordan's lemma. 
We leave the detailed analysis in \ref{appendix:jordan}. Here, we briefly state the indication of Jordan's lemma in our work. In the CHSH-type Bell test, we can effectively view the measurement process as first performing local operations to transform the underlying quantum state into an ensemble of qubit pairs, $\{p^\mu,\rho_{AB}^{\mu}\}$, with $p^{\mu}$ a probability distribution, and then measuring each pair of qubits with associate qubit observables. The measurement on each pair of qubits corresponds to a Bell value, $S^{\mu}$, and the observed Bell value is the average of these values, $S=\sum_{\mu}p^{\mu}S^{\mu}$. Guaranteed by the convexity property of an entanglement measure, we can lower-bound the amount of entanglement in the initial system by studying the average amount of entanglement in the ensemble of qubit-pairs, $\sum_{\mu}p^{\mu}E(\rho_{AB}^{\mu})$. In this way, we can essentially focus on quantifying entanglement in a pair of qubits. 

By further utilizing the non-increasing property under LOCC of an entanglement measure and choosing proper local computational bases, we may further restrict the pair of qubits to a Bell-diagonal state in Eq.~\eqref{Bell_diagonal} for simplicity. We have the following lemma.
\begin{comment}
\begin{equation}
\label{Bell_diagonal}
    \rho_{\lambda}=\lambda_{1}\ketbra{\Phi^+}+\lambda_{2}\ketbra{\Phi^-}+\lambda_{3}\ketbra{\Psi^+}+\lambda_{4}\ketbra{\Psi^-},
\end{equation}
with $\ket{\Phi^\pm}=(\ket{00}\pm\ket{11})/\sqrt{2},\ket{\Psi^\pm}=(\ket{01}\pm\ket{10})/\sqrt{2}$. We term such a state a Bell-diagonal state with respect to the computational basis. Without loss of generality, we assume $\lambda_1\geq\lambda_2\geq\lambda_3\geq\lambda_4$, since we can relabel the eigenvalues corresponding to the Bell-basis states with local unitary operations.% This result was first proved in Ref. \cite{Pironio2009device} (see Lemma 3 therein). Here, we restate the result and generalize it to the general CHSH-type Bell inequalities.
\end{comment}
\begin{lemma}
    In a CHSH Bell test, under a fixed computational basis, a two-qubit state, $\rho_{AB}$, can be transformed into a Bell-diagonal state, $\rho_\lambda$, through LOCC, with the $\alpha$-CHSH Bell value unchanged.
    %If $\rho_{AB}$ leads to value $S$ for the $\alpha$-CHSH expression with proper Bell-test observables, then there exists Bell operator $\hat{S}_\alpha$ for $\rho_{\lambda}$, such that $S=\Tr(\rho_{\lambda}\hat{S}_\alpha)$.
\label{lemma:LOCC}
\end{lemma}

The lemma indicates that an observed Bell value can always be interpreted as arising from a Bell-diagonal state. Furthermore, the operations in the lemma are restricted to LOCC and state mixing. Since these operations do not increase entanglement, we can hence restrict our analysis of lower-bounding entanglement to the set of Bell-diagonal states. The LOCC transformation in this result was first constructed in Ref. \cite{Pironio2009device} (see Lemma 3 therein). Here, we verify the unchanged $\alpha$-CHSH Bell value through the LOCC transformation. We present proof of the lemma in \ref{appendix:Bell-diagonal}. 
%A similar result first appeared in Lemma 3 of Ref. \cite{Pironio2009device}. 
%Initially, Alice's and Bob's measurements are labeled in $x-z$ plane of the Bloch sphere on their own side, respectively. By applying classical mixings and local rotations to measurements, any two-qubit state $\rho_{AB}$ can be transformed into a Bell diagonal state $\rho_{\lambda}$. The proof of lemma~\ref{lemma:LOCC} indicates that, $\rho_{AB}$ and $\rho_{\lambda}$ contain the same ``optimal violation"  $S^*$ from Eq.~\eqref{dual_optm}. Besides, the entanglement measure is non-increased under LOCC by definition. These make it valid that the solution of Eq.~\eqref{original_optm} for the Bell diagonal states acts as a lower bound of the solution for two-qubit states.

With the above simplifications, we have the following lemma in solving the problem in Eq.~\eqref{equivalent_optm} and leave the proof in \ref{appendix:Bell-diagonal}.

\begin{lemma}
%[Maximal CHSH-type violation of a Bell diagonal state]
The maximal value of the $\alpha$-CHSH expression in Eq.~\eqref{CHSH_type} for a Bell-diagonal state shown in Eq.~\eqref{Bell_diagonal}, $\rho_{\lambda}$, is given by
\begin{equation}
    S= 2\sqrt{\alpha^2(\lambda_{1}+\lambda_{2}-\lambda_{3}-\lambda_{4})^2+(\lambda_{1}-\lambda_{2}+\lambda_{3}-\lambda_{4})^2},
\end{equation}
where $\lambda_i$ is the $i$-th largest eigenvalue of $\rho_{\lambda}$.
\label{lemma:maximal_value}
\end{lemma}

%A nonlocal behavior necessarily needs both entanglement and incompatible measurements.
%To prepare for discussion in Sec.~\ref{sc:interplay}, 
In our analysis, we extend general measurements to projective ones by Naimark's dilation theorem. For two projection-valued measurements (PVMs), their incompatibility is defined as the largest inner product of their eigenvectors. Suppose two projective measurements are given by observables $\hat{M}$ and $\hat{N}$, and $\{\ket{m}\}$ and $\{\ket{n}\}$ are their eigenvectors, respectively. Then we define the incompatibility as
\begin{equation}
\label{eq:inc_overlap}
    I(\hat{M},\hat{N})=\max_{\ket{m},\ket{n}} |\braket{m}{n}|.
\end{equation}
For qubit observables, this definition can be equivalently given by the observable commutator.
Consider two qubit observables $\hat{M}=\cos\theta_1\sigma_z+\sin\theta_1\sigma_x$ and $\hat{N}=\cos\theta_2\sigma_z+\sin\theta_2\sigma_x$. Then, Eq.~\eqref{eq:inc_overlap} becomes 
\begin{equation}
\label{eq:inc_PVM}
    I(\hat{M},\hat{N})=\frac{1}{2}\abs{\cos(\theta_1-\theta_2)}+\frac{1}{2},
\end{equation}
On the other hand, the commutator between them is $[\hat{M},\hat{N}]=\sin(\theta_1-\theta_2)[\sigma_x,
    \sigma_z]$. Considering the symmetry, the commutator essentially provides a quantity as
\begin{equation}
\label{eq:com_PVM}
    I_c(\hat{M},\hat{N})=\abs{\sin(\theta_1-\theta_2)}[\sigma_x,
    \sigma_z].
\end{equation}
We can see $I(\hat{M},\hat{N})$ and $I_c(\hat{M},\hat{N})$ are of one-to-one bijection. Therefore, we use the commutator as a direct incompatibility measure in the qubit PVM case. And specifically, we use the coefficient, $\abs{\sin(\theta_1-\theta_2)}$, in Eq.~\eqref{eq:com_PVM} in the later incompatibility discussions. In proving Lemma~\ref{lemma:maximal_value}, a notable issue is that the optimal measurements may not be the most incompatible measurements. Up to a minus sign before the observables, the optimal measurements of the Bell-diagonal state in Eq.~\eqref{Bell_diagonal} are as follows,
\begin{equation}
\label{Bell_opt_meas}
 \begin{split}  
       \hat{A}_0&=\sigma_z,\\
       \hat{A}_1&=\sigma_x,\\
       \hat{B}_0&=\cos\theta\sigma_z+\sin\theta\sigma_x,\\
       \hat{B}_1&=\cos\theta\sigma_z-\sin\theta\sigma_x,
 \end{split}
\end{equation}
where $\theta$ fully determines the amount of imcompatibility of the local observables, with $\tan\theta=(\lambda_{1}-\lambda_{2}+\lambda_{3}-\lambda_{4})/[\alpha(\lambda_{1}+\lambda_{2}-\lambda_{3}-\lambda_{4})]$ determined by the Bell-diagonal state and parameter $\alpha$. While the observables on Alice's side are maximally incompatible with each other, the commutator of the observables on Bob's side is given by $[\hat{B}_0,\hat{B}_1]=\sin2\theta[\sigma_x,\sigma_z]$.
%Since $[\hat{A}_0,\hat{A}_1]=[\sigma_z,\sigma_x],[\hat{B}_0,\hat{B}_1]=\sin2\theta[\sigma_x,\sigma_z]$, the local incompatibility is fixed on one side and determined by $\sin2\theta$ on the other side. 
%When $\tan\theta$ increases from $0$ to $1$, $\theta$ increases from $0$ to $\pi/4$ and $\sin 2\theta$ from $0$ to $1$. Particularly, 
For example, when the considered state is the maximally entangled state with $\lambda_1=1,\lambda_2=\lambda_3=\lambda_4=0$ and $\alpha=1$, which corresponds to the original CHSH expression, the optimal measurements coincide with the most incompatible measurements. For more cases where $\sin2\theta$ is strictly smaller than $1$, $\hat{B}_0$ and $\hat{B}_1$ are not maximally incompatible.
%when the Bell-diagonal state is not maximally entangled or $\alpha$ is chosen strictly larger than $1$, we have $\sin2\theta$ here is strictly smaller than $1$. In general, optimal measurements are not the most incompatible measurements.

\begin{observation}
The observables that yield the largest $\alpha$-CHSH Bell value for a quantum state are not the most incompatible ones in general.
\end{observation}

Notwithstanding, a subtle issue is that we do not have access to the underlying probability distribution in the qubit-pair ensemble, $p^{\mu}$, or the underlying Bell value for each pair of qubits. As we only know the average Bell value over the ensemble, we need to be careful of convexity issues. Suppose the solution to Eq.~\eqref{original_optm} with the restriction of a pair of qubits takes the form $E_{\rm est}=E_{\rm est}(S)$. 
%We denote $E_{\rm est}(S)$ as the estimation of objective entanglement measure $E$ with respect to violation $S$, given any $\alpha\geq1$, when the input state is assumed as pairs of qubits. 
When extending the result to possibly an ensemble of qubit pairs, if $E_{\rm est}(S)$ is not concave in $S$, then 
\begin{equation}
\label{convexity}
    E_{\rm est}\left(\sum_{\mu}p^{\mu}S^{\mu}\right)\leq \sum_{\mu}p^{\mu}E_{\rm est}\left(S^{\mu}\right)\leq\sum_\mu p^{\mu} E\left(\rho_{AB}^\mu\right)\leq E(\rho_{AB}),
\end{equation}
which holds for any probability distribution $p^{\mu}$. Hence, we can directly lower-bound the amount of entanglement in the underlying state by $E_{\rm est}(S)$, where $S=\sum_{\mu}p^{\mu}S^{\mu}$ represents the observed Bell value. Yet if the function $E_{\rm est}(S)$ is concave, namely 
$[E_{\rm est}(S_1)+E_{\rm est}(S_2)]/2<E_{\rm est}[(S_1+S_2)/2]$, then the first inequality in Eq.~\eqref{convexity} no longer holds valid. Consequently, we need to take a ``convex closure'' of the function $E_{\rm est}$ to estimate the amount of entanglement from a quantum state with an unknown dimension. Here, we explain the concept of convex closure in our context.
Suppose a concave function $f(x)$ is defined on the interval $[a,b]$. Then the convex closure of $f(x)$, denoted as $f_{\mathrm{con}}(x)$, is given by
    \begin{equation}
    \label{eq:convex_closure}
        f_{\mathrm{con}}(x)=\frac{f(b)-f(a)}{b-a}(x-a)+b,
    \end{equation}
    which represents a straight line connecting points $(a,f(a))$ and $(b,f(b))$.


Another implicit issue is that we assume the entanglement measure to have a consistent definition for all dimensions, such that the last inequality in Eq.~\eqref{convexity} holds. Yet for the measure of concurrence, its definition in a high-dimensional system is subtle. 
%We know that the underlying system in a Bell test can be high dimensional, where the definition of concurrence can be subtle. 
Despite this, we may estimate the average amount of concurrence of the qubit pairs arising from the block-dephasing operation in the measurement, $\sum_{\mu}p^{\mu}C_{\rm est}\left(S^{\mu}\right)$.

%of entanglement, within all the entanglement measures we introduced in Sec.~\ref{sc:Pre}, there does not exist a well-defined concurrence for a $\rho_{AB}$ in a high dimensional Hilbert space. While it is natural to bound the EOF of a general $\rho_{AB}$, for concurrence, the $E(\rho_{AB})$ is considered as a quantity lower-bounded by the convex combination of the block-dephased states in qubit pairs, as the last inequality in  Eq.~\eqref{convexity}.}

%{\color{red}ZXJ: Please move the original detailed optimization problem under the restriction of a pair of qubits to the Appendix}
\begin{comment}
Based on the framework in Fig.~\ref{fig:flowchart}, the optimization problem for a given $\alpha$ to be solved is simplified as 
\begin{equation}
    \label{final_optm}
    \begin{aligned}
    & \underset{\rho_{AB}\in\mathcal{H}_2\otimes\mathcal{H}_2}{\text{min}}
    & &  E(\rho_{AB}) \\
    & \text{s.t.}
    & & S=\max_{\hat{A}_0,\hat{A}_1,\hat{B}_0,\hat{B}_1}\operatorname{Tr}[\rho_{AB}(\alpha \hat{A}_0\otimes \hat{B}_0+\alpha \hat{A}_0\otimes \hat{B}_1+\hat{A}_1\otimes \hat{B}_0-\hat{A}_1\otimes \hat{B}_1)],\\
    &&&\rho_{AB}\text{ is Bell diagonal.}
    %,\rho_{AB}^\dag=\rho_{AB}.
    \end{aligned}
    \end{equation}

A Bell diagonal $\rho_{AB}$ is of the form
\begin{equation}
    \rho_{AB}=\lambda_{\Phi^+}\ket{\Phi^+}\bra{\Phi^+}+\lambda_{\Phi^-}\ket{\Phi^-}\bra{\Phi^-}+\lambda_{\Psi^+}\ket{\Psi^+}\bra{\Psi^+}+\lambda_{\Psi^-}\ket{\Psi^-}\bra{\Psi^-}\label{Bell_diagonal}
\end{equation}
under Bell basis. 
\end{comment}

Following the above discussions, we study the entanglement measures of EOF and one-way distillable entanglement, which are essentially given by concurrence and conditional entropy of entanglement, respectively. 
%We deliver the results in the following subsections.


\subsection{Concurrence and entanglement of formation}
In this subsection, we take concurrence $C(\cdot)$ as the objective entanglement measure in Eq.~\eqref{original_optm}. For this measure, we have an analytical estimation result.
%We first give the entanglement estimation result with the additional dimensional restriction. 
\begin{theorem}\label{thm:concurrence_est}
%[Minimum concurrence providing a CHSH-type violation with a given $\alpha$]
    Suppose the underlying quantum state is a pair of qubits. For a given tilted CHSH expression in Eq.~\eqref{CHSH_type} parametrized by $\alpha$, if the Bell expression value is $S$, then the amount of concurrence in the underlying state can be lower-bounded,
\begin{equation}
\label{concurrence_est}
    C(\rho_{AB})\geq \sqrt{\frac{S^2}{4}-\alpha^2}.
\end{equation}
%The equality can be saturated by a rank-two or rank-one Bell-diagonal state of Eq.~\eqref{Bell_diagonal} under the optimal measurements in Eq.~\eqref{Bell_opt_meas}, with eigenvalues $\lambda_1=\frac{1}{2}+\frac{1}{2}\sqrt{\frac{S^2}{4}-\alpha^2},\lambda_2=\frac{1}{2}-\frac{1}{2}\sqrt{\frac{S^2}{4}-\alpha^2},\lambda_3=\lambda_4=0$.
The equality can be saturated when measuring a Bell-diagonal state in Eq.~\eqref{Bell_diagonal} with eigenvalues
\begin{equation}
    \begin{gathered}
        \lambda_1=\frac{1}{2}+\frac{1}{2}\sqrt{\frac{S^2}{4}-\alpha^2},\\
        \lambda_2=\frac{1}{2}-\frac{1}{2}\sqrt{\frac{S^2}{4}-\alpha^2},\\
        \lambda_3=\lambda_4=0,
    \end{gathered}
\end{equation}
using measurements in Eq.~\eqref{Bell_opt_meas} with $\theta=\arctan(\frac{1}{\alpha}\sqrt{\frac{S^2}{4}-\alpha^2})$.
\end{theorem}

We leave the detailed derivation in  \ref{appendix:concurrence_est}. %Eq.~\eqref{concurrence_est} takes equality when two conditions are satisfied: (1) the state is a rank-two or rank-one Bell-diagonal state; (2) the Bell value is obtained by the ``optimal measurements" of the state. The later condition is independent from the chosen entanglement measure.

\begin{observation}
Given a $\alpha$-CHSH Bell value, the measurements that require the minimum entanglement are not the most incompatible measurements in general.
\end{observation}

As the EOF can be expressed by concurrence in a closed form for a pair of qubits~\cite{hill1997entanglement}, this entanglement measure is directly lower-bounded by substituting Eq.~\eqref{concurrence_est} in Eq.~\eqref{eof},
\begin{equation}
\label{EOF_est_semiDI}
    E_{\rm F}(\rho_{AB})\geq h\left(\frac{1}{2}+\frac{1}{2}\sqrt{1+\alpha^2-\frac{S^2}{4}}\right).
\end{equation}

\begin{figure}[hbt!]
\centering
\includegraphics[scale=0.57]{concurrence_EOF.eps}
\caption{Diagram of concurrence and entanglement formation estimation results when the CHSH-type expression in Eq.~\eqref{CHSH_type} takes $\alpha=1.5$ and input states are two-qubit states. We plot the estimated values of concurrence and EOF
with the blue solid line and the red dashed line, respectively. The estimations are both concave in $S\in(3,2\sqrt{3.25}]$ and range from $0$ to $1$.}
\label{fig:concurrence_EOF}
\end{figure}

In Fig.~\ref{fig:concurrence_EOF}, we depict the entanglement estimation result when $\alpha=1.5$ for a pair of qubits input. Given the consistent definition of EOF across all dimensions, we extend the two-qubit EOF estimation result in Eq.~\eqref{EOF_est_semiDI} to a general-state scenario. Since the two-qubit EOF estimation result is a concave function in the Bell value, a convex closure should be taken when extending the EOF estimation result to general states with an unknown dimension. For example, suppose the underlying state already has a block-diagonal form with respect to the measurement observables, $p^{(1)}\rho_{AB}^{(1)}\oplus p^{(2)}\rho_{AB}^{(2)}$, where $p^{(1)}=p^{(2)}=1/2$, and $\rho_{AB}^{(1)},\rho_{AB}^{(2)}$ are qubit pairs. In addition, $\rho_{AB}^{(1)}$ is not entangled, while $\rho_{AB}^{(2)}$ is a Bell state. In this case, the EOF of the underlying state is $E_{\mathrm{F}}(\rho_{AB})=p^{(1)}E_{\mathrm{F}}(\rho_{AB}^{(1)})+p^{(2)}E_{\mathrm{F}}(\rho_{AB}^{(2)})=1/2$. Suppose the measurement observables are such that the expected Bell values arising from $\rho_{AB}^{(1)}$ and $\rho_{AB}^{(2)}$ are $S^{(1)}=3$ and $S^{(2)}=2\sqrt{1+1.5^2}$, respectively. Alice and Bob can only observe the average expected Bell value of $S=p^{(1)}S^{(1)}+p^{(2)}S^{(2)}$, and they will overestimate the underlying state's EOF if they directly apply Eq.~\eqref{EOF_est_semiDI}. To bypass such a problem, we take a convex closure over Eq.~\eqref{EOF_est_semiDI} according to Eq.~\eqref{eq:convex_closure} and obtain the final estimation.
%We denote the RHS of Eq.~\eqref{EOF_est_semiDI} as $E_{F,\rm est}(S)$. According to Jordan's lemma, if the violation $S$ obtains from a two-qubit nonlocal game, the lower bound of concurrence and EOF strictly follow  $C_{\rm est}(S)$ and $F_{F,\rm est}(S)$, respectively. Generally, under a complete DI setting, when any of the dimensions of Alice or Bob's Hilbert space is larger than 2, the Bell value $S$ can be viewed as the convex combination of value $S^\mu$ from each dimension 2 $\mu-$substate. 

%\textcolor{blue}{For example, suppose the underlying state already enjoys a block-diagonal form with respect to the measurement observables, $\rho_{AB}=p^{(1)}\rho_{AB}^{(1)}\oplus p^{(2)}\rho_{AB}^{(2)}$. The observed Bell value $S=p^{(1)}S^{(1)}+p^{(2)}S^{(2)}$. If $S=3.3$ is composed by $p^{(1)}=p^{(2)}=1/2$ and $S^{(1)}=3,S^{(2)}=2\sqrt{1+1.5^2}$, then the actual lower bound we can obtain is $E_{\mathrm{F}}(\rho_{AB})=p^{(1)}E_{\mathrm{F}}(\rho_{AB}^{(1)})+ p^{(2)}E_{\mathrm{F}}(\rho_{AB}^{(2)})\geq 1/2$. While Alice and Bob can only observe the average expected Bell value of $S=3.3$, and they will overestimate the underlying state's EOF if they directly apply Eq.~\eqref{EOF_est_semiDI}.}


\begin{theorem}
For a given tilted CHSH expression in Eq.~\eqref{CHSH_type}, if the Bell expression value is $S$, then the amount of entanglement of formation in the underlying state can be lower-bounded,
\begin{equation}
    E_{\rm F}(\rho_{AB})\geq\frac{S-2\alpha}{2\sqrt{1+\alpha^2}-2\alpha}.
    \label{EOF_est_DI}
\end{equation}
\end{theorem}

In the literature, Ref.~\cite{arnon2017noise} provides EOF estimation results using threshold games that have a non-zero gap between classical and quantum strategies. Translating the result to the CHSH game, the EOF estimation result in Ref.~\cite{arnon2017noise} is
\begin{equation}
\label{EOF_arnon}
    E_{\mathrm{F}}(\rho_{AB})\geq\frac{(S-2)^5}{10\cdot 180^2\cdot2^{16}}.
\end{equation}
In comparison, our EOF estimation result in Eq.~\eqref{EOF_est_DI} is much tighter. In  \ref{sc:EOF_arnon}, we briefly review the results in Ref.~\cite{arnon2017noise} and explain how to arrive at Eq.~\eqref{EOF_arnon}.


\subsection{Negative conditional entropy and one-way distillable entanglement}
In this subsection, we estimate the one-way distillable entanglement, $E_D^\rightarrow(\rho_{AB})$, depicted by the negative conditional entropy, $-H(A|B)$, via Bell nonlocality. 
%Similar to the case of concurrence, we can simplify the optimization problem following the route in Fig.~\ref{fig:flowchart}. 
For the set of Bell-diagonal states on the qubit-pair systems, since the reduced density matrix of a subsystem is a maximally mixed state, $H(B)=1$, the conditional von Neumann entropy of the state is reduced to $H(A|B)=H(AB)-H(B)=H(AB)-1$. Using the notation in Eq.~\eqref{Bell_diagonal}, the 
term of joint von Neumann entropy can be expressed by
\begin{equation}
    H(AB)=H(\Vec{\lambda})=-\sum_{i=1}^{4}\lambda_i\log\lambda_i.
\end{equation}
Thus, the lower bound of one-way distillable entanglement for a pair of qubits becomes the following optimization problem,
\begin{equation}
    \begin{split}
    E_{D\rm ,est}^\rightarrow & =\underset{\lambda_i,i=1,2,3,4}{\text{min}} 1+\sum_{i=1}^4\lambda_i\log\lambda_i, \\
    \text{s.t.}\quad
    S &=2\sqrt{\alpha^2(\lambda_1+\lambda_2-\lambda_3-\lambda_4)^2+(\lambda_1-\lambda_2+\lambda_3-\lambda_4)^2},\\
    \lambda_1 & \geq\lambda_2\geq\lambda_3\geq\lambda_4,\\
    1 & =\sum_{i=1}^4\lambda_i, \lambda_i\geq 0 \; ,i=1,2,3,4.
    \end{split}
    \label{entropy_optm}
\end{equation}
As this is a convex optimization problem, we can solve it efficiently via off-the-shelf numerical toolboxes. We present numerical results for some values of $\alpha$ in Fig.~\ref{fig:numeric_entropy}. Given any $\alpha>1$, the estimation value $E_{D\rm ,est}^\rightarrow(S)$ is a convex function on $S\in(2\alpha,2\sqrt{1+\alpha^2}]$. Following Eq.~\eqref{convexity}, the solution can be directly lifted as the lower bound on one-way distillable entanglement for a general state. 
Notably, in the special case of $\alpha=1$ that corresponds to the original CHSH expression, our numerical estimation coincides with the existing analytical result~\cite{arnon2019device},
\begin{equation}
    E_{D}^{\rightarrow}(\rho_{AB})=-H(A|B)\geq \max\left\{0,1-2 h\left(\frac{1}{2}-\frac{S}{4 \sqrt{2}}\right)\right\}.
\end{equation}

\begin{figure}[hbt!]
    \centering
    \includegraphics[scale=0.5]{numeric_entropy.eps}
    \caption{Diagram of one-way distillable entanglement estimation results. The estimation is depicted by CHSH-type Bell expressions with several discretely increasing $\alpha$. For each value of $\alpha$, the estimation result, $E_{D\rm ,est}^\rightarrow(S)$, depicted over the valid interval $S\in(2\alpha,2\sqrt{1+\alpha^2}]$ is convex. When $\alpha$ increases, $E_{D\rm ,est}^\rightarrow(S)$ at $S=2\alpha$ for each $\alpha$ increases and converges to $0$. Since $E_{D,\rm est}^\rightarrow(S)$ is convex in $S$, the estimation results hold valid without assuming the system dimension.}
    %{\color{red} terminal point converges to 1?}
    %The curves are numerical simulation results of CHSH violation when input states are assumed as pairs of qubits. The convexity of each curve certifies that $\forall\alpha\geq 1$, $E_{D\rm ,est}^\rightarrow(S)$ acts as a lower bound of one-way distillable entanglement for any input $\rho_{AB}$ without dimension restrictions. 
    \label{fig:numeric_entropy}
\end{figure}

%{\color{red}ZXJ: the quantification result fail in some range --- related to Pere's conjecture}

\subsection{Negativity of entanglement}
In this subsection, we estimate the negativity of entanglement given an $\alpha$-CHSH Bell value. In Sec.~\ref{subsc:ent measures}, we establish a connection between the concurrence and negativity for Bell-diagonal states: both measures are linear functions of eigenvalues $\lambda_i$ in Eq.~\eqref{Bell_diagonal}, and the negativity of entanglement is precisely half the concurrence for a given Bell-diagonal state. Therefore, for a pair of qubits, we can obtain the following analytical estimation result for negativity based on the concurrence estimation in Theorem~\ref{thm:concurrence_est}.

\begin{corollary}
    Suppose the underlying state is a pair of qubits. For a given tilted CHSH expression in Eq.~\eqref{CHSH_type}, if the Bell expression value is $S$, then the amount of negativity in the underlying state can be lower-bounded,
\begin{equation}
\label{negativity_est}
    \mathcal{N}(\rho_{AB})\geq \frac{1}{2}\sqrt{\frac{S^2}{4}-\alpha^2}.
\end{equation}
The equality can be saturated when measuring a Bell-diagonal state in Eq.~\eqref{Bell_diagonal} with eigenvalues
\begin{equation}
    \begin{gathered}
        \lambda_1=\frac{1}{2}+\frac{1}{2}\sqrt{\frac{S^2}{4}-\alpha^2},\\
        \lambda_2=\frac{1}{2}-\frac{1}{2}\sqrt{\frac{S^2}{4}-\alpha^2},\\
        \lambda_3=\lambda_4=0,
    \end{gathered}
\end{equation}
using measurements in Eq.~\eqref{Bell_opt_meas} with $\theta=\arctan(\frac{1}{\alpha}\sqrt{\frac{S^2}{4}-\alpha^2})$. 
\end{corollary}

In a fully DI scenario, similar to the estimation of concurrence, we take a convex closure to lower-bound the negativity of the underlying state with an unknown dimension, arriving at the following analytical result.

\begin{corollary}
    For a given tilted CHSH expression in Eq.~\eqref{CHSH_type}, if the Bell expression value is $S$, then the amount of negativity in the underlying state can be lower-bounded,
    \begin{equation}
        \label{negativity_est_DI}
            \mathcal{N}(\rho_{AB})\geq \frac{S-2\alpha}{4(\sqrt{1+\alpha^2}-\alpha)}.
        \end{equation}
\end{corollary}

Especially when $\alpha=1$, Eq.~\eqref{negativity_est_DI} becomes
\begin{equation}
\label{negativity_est_DI_alpha1}
    \mathcal{N}(\rho_{AB})\geq \frac{S-2}{4(\sqrt{2}-1)}.
\end{equation}
This analytical result proves the conjecture of Eq.~(5) in Ref.~\cite{moroder2013device}, where the authors observed a nearly linear relation between the lower bound of negativity and the underlying Bell value via the third level of an NPA-type hierarchy numerical algorithm~\cite{navascues2007bounding}. In Fig.~\ref{fig:semiDI_negativity}, we plot the negativity estimation results from the CHSH Bell value, i.e., $\alpha=1$, in the fully DI scenario in Eq.~\eqref{negativity_est_DI} and the semi-device-independent (semi-DI) case with the additional assumption of system dimensions in Eq.~\eqref{negativity_est}.

\begin{figure}[hbt!]
\centering
\includegraphics[scale=0.57]{semiDI_negativity.eps}
\caption{Diagram of negativity estimation results. The semi-DI negativity estimation when the input state is a pair of qubits is plotted with the blue solid line. The complete DI negativity estimation result is plotted with the red dashed line.}
\label{fig:semiDI_negativity}
\end{figure}

\section{Exploration of the relation among entanglement, measurement incompatibility, and Bell nonlocality within a finite-dimensional system}\label{sc:interplay}

Besides entanglement, another key ingredient behind nonlocality is measurement incompatibility. Both entanglement and measurement incompatibility can be regarded as quantum resources to unveil non-classical physical phenomena; hence a natural intuition is that for a given Bell value, there is a trade-off relation between entanglement and measurement incompatibility, where more incompatible measurement may compensate an underlying system with less entanglement and \emph{vice versa}. However, as we have discussed for the notion of optimal measurements, the observables that yield the largest Bell value for a quantum state may not correspond to the maximally incompatible ones. Particularly, as shown in Theorem~\ref{thm:concurrence_est}, for the case of the least amount of entanglement for a nonlocal behavior, the observables are generally not maximally incompatible.
%the derivations above indicate that the least amount of entanglement is obtained not necessarily at the most incompatible measurements. 
In this section, we make a detailed investigation into the relation between entanglement and measurement incompatibility under a given Bell nonlocal behavior. %In this section, we take another perspective on the relation between entanglement and nonlocality. For the subtlety of their relation, one reason is that entanglement is only one facet of the necessary conditions for nonlocal behavior. Another key ingredient to unveil nonlocality is measurement incompatibility. To better understand the issue, we study the interplay between entanglement and nonlocality under a given Bell nonlocal behavior. 

%From a quantum point of view, the violation of Bell inequality contains both entanglement in shared state and non-commutativity in local measurements. When fixing a violation in a CHSH game, the relationship between the above two quantifies is not clear. In this section, we investigate the relationship between entanglement and uncertainty.

To simplify the discussion and manifest the main factors in the interplay, we restrict our analysis with the following assumptions: (1) the underlying system is a pair of qubits, and (2) the measurement operators are qubit observables. Note this setting is consistent with the DI estimation. Namely, this is the subsystem after applying Naimark’s dilation theorem and Jordan’s lemma. Moreover, when considering projective measurements, the quantification of measurement incompatibility becomes straightforward.

%In other words, the discussions in this section are restricted to a semi-DI scenario. Note that in the fully DI scenario without any \emph{a priori} assumption, the measurement result is a mixture of basic scenarios in this form, which can be shown by Jordan's lemma. In a sense, the scenario we consider represents a typical setting for the question. To characterize a nonlocal behavior, we use the value of a particular $\alpha$-CHSH Bell expression.
%For simplicity, we present the results with the original CHSH Bell expression. 

We quantify the least amount of entanglement that is necessary for a given Bell value,
\begin{equation}
\label{interplay_optm}
    \begin{split}
     E_{\mathrm{est}}(S,\theta)&=\min_{\rho_{AB}}  E(\rho_{AB}), \\
    \text{s.t.}\quad
    \Tr\left(\rho_{AB}\hat{S}_{\alpha}\right) &= S,\\
   \hat{A}_0 &= \sigma_z,\\
   \hat{A}_1 &= \sigma_x,\\
    \hat{B}_0 &=\cos\theta\sigma_z+\sin\theta\sigma_x,\\
    \hat{B}_1 &=\cos\theta\sigma_z-\sin\theta\sigma_x,\\
    \rho_{AB}&\geq 0, \\
    \rho_{AB}&\in\mathcal{D}(\mathcal{H}_2\otimes\mathcal{H}_2), \\
    \Tr(\rho_{AB})&=1.
    \end{split}
\end{equation}
where $E$ represents a chosen entanglement measure. We still denote the solution to the optimization as $E_{\mathrm{est}}$, while it now represents the least amount of entanglement that is necessary for the nonlocal behavior under the given measurement incompatibility. In this optimization, we assume that the measurement incompatibility is parameterized by one parameter, $\theta$. On Alice's side, the two local observables are fixed to be maximally incompatible with each other. On Bob's side, when $\theta=0$ and $\pi/2$, the two local observables commute. When $\theta=\pi/4$, the local observables enjoy the maximal incompatibility, which is the other extreme. In the following discussions, we restrict the parameter to be $\theta\in[0,\pi/4]$, as other cases can be obtained via symmetry.

Before presenting the results, we make some remarks on the scenario considered in the optimization. For both parties in the Bell test, their local measurement observables need to be incompatible to violate a Bell inequality. Aside from our choice of fixing Alice's observables to be maximally incompatible while optimizing Bob's observables, one may consider alternative settings of incompatible measurements and analyze their relation with entanglement and nonlocality. The reason for our choice is that the measurement settings in Eq.~\eqref{interplay_optm} coincide with those in Eq.~\eqref{Bell_opt_meas}, which are the optimal measurements that give the largest Bell value for the Bell-diagonal states in Eq.~\eqref{Bell_diagonal}. Moreover, if we further minimize $E_{\rm est}(S,\theta)$ over $\theta$ in Eq.~\eqref{interplay_optm}, the optimization degenerates to Eq.~\eqref{equivalent_optm} with the additional assumption that the underlying system is a pair of qubits and the measurements are projective.
%measurements that yield the optimal solution will coincide with the ones obtained in Eq.~\eqref{equivalent_optm} under the system dimension constraint and the solution to  coincide as functions of $S$.
%for any given quantum state, one may encounter its optimal measurements by varying $\theta$.
\begin{comment}
{\color{blue}In fact, if the measurements in Eq.~\eqref{interplay_optm} become optimal for $\rho_{AB}$, the solution to Eq.~\eqref{interplay_optm} coincides with the solution to Eq.~\eqref{equivalent_optm} under the system dimension constraint.
In this way, the solution to Eq.~\eqref{interplay_optm} should reach the tight lower bound of device-independent entanglement estimation in a proper measurement setting.}
\end{comment}
%when one varies $\theta$ in $[0,\pi/4]$ coincides with the angle that defines the optimal measurements in Eq.~\eqref{Bell_opt_meas}, the solution to Eq.~\eqref{interplay_optm} can reach the tight lower bound of device-independent entanglement estimation. 
%The measurements we assumed in Eq.~\eqref{interplay_optm} coincides with the form of the optimal measurements \eqref{Bell_opt_meas} of Bell-diagonal states. This guarantees the solution to Eq.~\eqref{interplay_optm} can reach the tight lower bound of an entanglement estimation.

%{\color{blue}ZXJ: The readability of the statement, ``In fact... a proper measurement setting'', is not good. Try writing it more directly and briefly. (we may delete it if it is not very relevant.)}

\begin{comment}
Furthermore, we assume that the extent of measurement incompatibility on both sides is the same, where we parametrized them with one parameter, $\theta$,
\begin{equation}
\label{twoside_measure}
\begin{split}
    \hat{A}_0&=\hat{B}_0=\cos\theta\sigma_z+\sin\theta\sigma_x,\\
    \hat{A}_1&=\hat{B}_1=\cos\theta\sigma_z-\sin\theta\sigma_x.
\end{split}
\end{equation}
Here, the observables, $\sigma_z$, should be interpreted as determining the local references of each party. When $\theta=0$ and $\pi/2$, the two local observables commute with each other. When $\theta=\pi/4$, the local observables enjoy the maximal incompatibility, which is the other extreme.
\end{comment}

\subsection{Original CHSH (\texorpdfstring{$\alpha=1$}{Lg})}
To observe the interplay among entanglement, measurement incompatibility, and nonlocality, we numerically solve the optimization problem in Eq.~\eqref{interplay_optm} by taking concurrence and one-way distillable entanglement as an entanglement measure and varying $\theta\in[0,\pi/4]$ and $S$ discretely from the $\alpha$-CHSH Bell values.

In Fig.~\ref{fig:concurrence-incompatibility}, we choose the original CHSH Bell expression and present the numerical results when taking the concurrence as the entanglement measure. For a nonlocal behavior, where $S\in (2,2\sqrt{2}]$, we denote $\theta=\theta_C^*$ when the estimated concurrence reaches its minimum, $C_{\rm est}(S,\theta)=\sqrt{S^2/4-1}$. As we have derived in Theorem~\ref{thm:concurrence_est}, $\theta_C^*=\arctan\sqrt{S^2/4-1}$. 
When the amount of measurement incompatibility between the local observables is smaller than that of this point, which corresponds to $\theta<\theta_C^*$, there is a trade-off relation between concurrence and measurement incompatibility, where less entanglement is required for the given Bell value as the amount of measurement incompatibility increases.
However, when $\theta>\theta_C^*$, as the underlying state enjoys more entanglement of concurrence, larger measurement incompatibility is also required for the observed Bell value. As $S$ increases from $2$ to $2\sqrt{2}$, the range of feasible values of $(\theta,C_{\mathrm{est}})$ shrinks as $S$ grows. When $S=2\sqrt{2}$, the underlying state is maximally entangled and the local measurement observables are the most incompatible ones, and the range of possible values of $(\theta,C_{\mathrm{est}})$ degenerates to the point of $(\pi/4,1)$. This result coincides with the self-testing finding~\cite{bardyn2009device}, where the only feasible experimental setting for the maximum CHSH Bell value enjoys the above properties.

%{\color{red} (ZXJ: not understanding this sentence), and the $\theta$ for each least concurrence point converges to $\pi/2$ as $S$ grows.}

\begin{figure}[hbt!]
    \centering
    \includegraphics[scale=0.57]{concurrence-incompatibility.eps}
    \caption{Illustration of the interplay among Bell nonlocality, measurement incompatibility, and concurrence. In this figure, we consider the original CHSH Bell expression and parameterize the measurement observables as in Eq.~\eqref{Bell_opt_meas}, where incompatibility is quantified through $\theta$. We focus on the interval of $\theta\in[0,\pi/4]$, and the results elsewhere can be obtained using symmetry. 
    %The incompatibility under measurements setting~\eqref{twoside_measure} on both parties is symmetric about $\theta=\pi/4$ and increases monotonously from $\theta=0$ to $\theta=\pi/4$. We illustrate the behavior of $C_{\rm est}(S,\theta)$ for $\theta\in [0,\pi/4]$. 
    For a given value of $S$, when $\theta<\theta_C^*=\arctan\sqrt{S^2/4-1}$, there is a trade-off relation between entanglement and measurement incompatibility, where less entanglement of concurrence is required for the nonlocal behavior when the measurements become more incompatible and \emph{vice versa}. Afterward, more entanglement is required for the given Bell value as $\theta$ increases. As $S$ increases, the range of possible values of $(\theta,C_{\mathrm{est}})$ shrinks and $\theta_C^*$ gets close to $\pi/4$. 
    %{\color{red}ZXJ: the original description is not accurate}
    %For small $S$, when $\theta$ increases from $0$ to $\pi/4$, once it is large enough to support $S$, the estimated entanglement depicted by concurrence decreases to $C_{\rm est}(S)$ then increases till $\theta$ reaches $\pi/4$; when $S$ is large, the concurrence decrease monotonously when $\theta$ increases from $0$ to $\pi/4$.
    }
    \label{fig:concurrence-incompatibility}
\end{figure}
 

In Fig.~\ref{entropy-incompatibility}, we present the numerical results when taking the one-way distillable entanglement as the entanglement measure. Under a fixed Bell value, there is a strict trade-off relation between entanglement and measurement incompatibility. The more incompatible the measurement observables are, the less entanglement is necessary for the nonlocal behavior, and \emph{vice versa}. In addition, the range for the trade-off shrinks with a larger Bell violation value. In the extreme of the largest Bell violation value, $S=2\sqrt{2}$, the setting should involve both the maximally entangled state and measurement observables that are maximally incompatible, in accordance with the self-testing result. One thing to note is that the estimated negative conditional entropy reaches its minimum exactly when $\theta=\pi/4$ for all $S\in(2,2\sqrt{2}]$, which holds no longer valid in $\alpha$-CHSH inequality when $\alpha>1$.

\begin{figure}[hbt!]
    \centering
    \includegraphics[scale=0.57]{entropy-incompatibility.eps}
    \caption{Illustration of the interplay among Bell nonlocality, measurement incompatibility, and one-way distillable entanglement. In this figure, we consider the original CHSH Bell expression and parameterize the measurement observables as in Eq.~\eqref{Bell_opt_meas}.
    %The observables commute with each other when $\theta=0$ and become maximally incompatible when $\theta=\pi/4$. 
    %We illustrate the behavior of $E_{D,{\rm est}}^\rightarrow(S,\theta)$ for $\theta\in [0,\pi/4]$. 
    %We use the CHSH Bell expression in the discussion (corresponding to $\alpha=1$). 
    As $S$ increases, the range of possible values of $(\theta,E_{D,{\rm est}}^\rightarrow)$ shrinks. For a given Bell value, less entanglement is required when $\theta$ increases in the valid region.}
\label{entropy-incompatibility}
\end{figure}

%{\color{red}ZXJ: All the figures (1) single-column size (2) show just one period using periodicity}

%{\color{blue}1. Trade-off: under a fixed Bell value, more entanglement, less compatibility, and vice versa; 2. Range: when the Bell value increases, more entanglement and incompatibility are necessary, gradually converging to the self-testing limit}

\subsection{General CHSH-type (\texorpdfstring{$\alpha>1$}{Lg})}
%In general, the nonlocality can be depicted by the $\alpha$-CHSH expression in Eq.~\eqref{CHSH_type} with $\alpha>1$. For concurrence, when $\theta<\theta_1=\arctan(\frac{1}{\alpha}\sqrt{\frac{S^2}{4}-\alpha^2})$, there is a trade-off relation between concurrence and measurement incompatibility. When $\theta=\theta_1$, the estimated concurrence reaches the lower bound of $C_{\rm est}=\sqrt{\frac{S^2}{4}-\alpha^2}$. When $\theta>\theta_1$, there is an anti-trade-off relation between concurrence and measurement incompatibility. The range of the possible value of $\theta$ and $C_{\rm est}$ shrinks to a single point at $\theta=\arctan(1/\alpha)$ and $C_{\rm est}=1$ when $S$ closes $2\sqrt{1+\alpha^2}$. This accords with the self-testing result in \cite{acin2012randomness}, where the $\alpha$-CHSH expression reaches its maximal violation when measuring the maximally entangled state under non-maximally incompatible measurements. For one-way distillable entanglement, the interplay relation behaves similarly to the relation of concurrence. The well-behaved trade-off relation between measurement incompatibility and one-way distillable entanglement when $\alpha=1$ is uncommon. Numerical details and analysis for $\alpha>1$ case are illustrated in Appendix~\ref{appendix:interplay_alpha}.

Besides the original CHSH Bell expression, we also study the relation among entanglement, measurement incompatibility, and nonlocality for general $\alpha$-CHSH expressions. Fixing parameter $\alpha>1$, for any Bell value $S\in(2\alpha,2\sqrt{1+\alpha^2}]$, denote the range of plausible values of parameter $\theta$ by $\theta_{\min}\leq\theta\leq\theta_{\max}$. In Fig.~\ref{incompatibility_alpha12}, we investigate the issue under parameter $\alpha=1.2$. For both the concurrence of entanglement and one-way distillable entanglement, when $\theta$ increases from $\theta_{\min}$ to $\theta_{\max}$, the corresponding amount of estimated entanglement first monotonically decreases from $1$, which corresponds to the maximally entangled state. In this region, there is a trade-off relation between entanglement and measurement incompatibility under the given Bell value. After reaching its minimum at $\theta=\theta^*_{E}$, a point that is related to the particular entanglement measure under study, more entanglement is required as the local measurement observables become more incompatible.
%When $\theta_{\min}\leq\theta<\theta^{*}_{E}$, there is a strict trade-off relation between entanglement and measurement incompatibility. When $\theta^{*}_{E}<\theta\leq\theta_{\max}$, there is a counterintuitively anti-trade-off relation between entanglement and measurement incompatibility. 
%When $\theta=\theta^*_{E}$, the corresponding measurements in Eq.~\eqref{interplay_optm} are the optimal measurements for the state that saturates the lower bound of entanglement estimation. 
One thing worth noting is that under the same $S$, the values of $\theta_{\min}$ and $\theta_{\max}$ are the same for both entanglement measures we now study. As $S$ grows, the supported range of incompatibility and entanglement shrinks, which converges to the single point of $\theta=\arctan(1/\alpha)$ and $E_{\rm est}=1$ when $S$ approaches its maximum $2\sqrt{1+\alpha^2}$. Namely, the maximum value of the $\alpha$-CHSH expression requires a pair of non-maximally incompatible measurements on one side. This result also coincides with the self-testing findings \cite{acin2012randomness}. Another indication is that to yield a large $\alpha$-CHSH Bell value with $\alpha>1$, the measurement observables on one side cannot be too incompatible, where they lie outside the feasible region of the experimental settings.

\begin{figure}[hbt!]
    \centering
    \includegraphics[scale=0.42]{incompatibility_alpha12.eps}
    \caption{Illustration of the interplay among Bell nonlocality, measurement incompatibility, and entanglement. In this figure, we consider the $\alpha$-CHSH Bell expression with  $\alpha=1.2$. The blue curves depict the results of one-way distillable entanglement, and the red curves depict the results of concurrence. For both entanglement measures, given a Bell value, the least required amount of entanglement first monotonically decreases as $\theta$ increases. After $\theta$ is larger than a threshold value that depends on the entanglement measure, $\theta_E^*$, more entanglement is required as the measurements become more incompatible.
    The ranges of possible values of $\theta\in[\theta_{\min},\theta_{\max}]$ are the same for the two entanglement measures. 
    %Given a Bell value $S$, the amount of estimated entanglement decreases and then increases as $\theta$ grows in a supported range where $\theta$ starts and ends at the same value for both entanglement measures. 
    When $S<2.2\sqrt{2}$, $\theta_{\max}=\pi/4$. When $S\geq 2.2\sqrt{2}$, $\theta_{\max}$ is smaller than $\pi/4$. The supported range shrinks as $S$ increases. When $S$ reaches its maximum, $S=2\sqrt{1.2^2+1}$, the range degenerates to the point of $\theta=\arctan 1/1.2$. In this case, the underlying state can only be a maximally entangled state, corresponding to $E_{\rm est}=1$.}
\label{incompatibility_alpha12}
\end{figure}

%\textcolor{red}{
For concurrence, we can derive the critical points analytically. Given $\alpha$-CHSH Bell value $S$, when $\theta=\theta^{*}_{C}=\arctan(\frac{1}{\alpha}\sqrt{\frac{S^2}{4}-\alpha^2})$, the system requires the least amount of concurrence, $C_{\mathrm{est}}=\sqrt{\frac{S^2}{4}-\alpha^2}$, which can be derived from Theorem~\ref{thm:concurrence_est}. When $\theta>\theta_{C}^*$, we find there is a region of $\theta$ where the least amount of concurrence behaves differently from that of one-way distillable entanglement. That is, though more concurrence is required in the underlying system as $\theta$ grows, the system may yield less distillable entanglement. In other words, the manifestation of entanglement properties through nonlocality highly depends on the particular entanglement measure under study.
%As $S$ grows, the least amount of concurrence gradually converges to $1$ and $\theta$ converges to ,
%\begin{equation}
%    \lim_{S\rightarrow 2\sqrt{1+\alpha^2}}(\arctan(\frac{1}{\alpha}\sqrt{\frac{S^2}{4}-\alpha^2}),\sqrt{\frac{S^2}{4}-\alpha^2})=((\arctan(1/\alpha),1)).
%\end{equation}
%}

The value of $\theta_{\max}$ and the value of corresponding $E_{\mathrm{est}}$ are related to the parameter, $\alpha$. A notable issue is that under particular value of $\alpha$ and Bell value $S$, $E_{\mathrm{est}}$ at $\theta=\theta_{\max}$ can reach $1$. We find that when $1<\alpha<\sqrt{2}+1$, for $S<\sqrt{2}(\alpha+1)$, $\theta_{\max}=\pi/4$ and the corresponding least amount of entanglement, $E_{\rm est}$ is strictly smaller than $1$. For a larger Bell value, $S\geq\sqrt{2}(\alpha+1)$, $\theta_{\max}$ may be smaller than $\pi/4$, and $E_{\rm est}$ at $\theta=\theta_{\max}$ always reaches $E_{\rm est}= 1$. For Bell expressions with $\alpha\geq \sqrt{2}+1$, as long as the Bell inequality is violated, $S>2\alpha$, we have $E_{\rm est}= 1$ at $\theta=\theta_{\max}$. 
%When $\theta_{\max}\leq\pi/4$, it decreases monotonously to $\arctan(1/\alpha)$ as $S$ increases. When the supported range ends at $\theta_{\max}\leq\pi/4$ and $E_{\rm est}= 1$, it indicates that any Bell value $S$ can be reached by a maximally entangled state with either set of measurements, one with less incompatibility in Eq.~\eqref{Bell_opt_meas} parameterized by $\theta=\theta_{\min}$ and the other with more incompatibility in Eq.~\eqref{Bell_opt_meas} parameterized by $\theta=\theta_{\max}\leq\pi/4$. 
In Fig.~\ref{incompatibility_alpha0}, we illustrate the interplay relation when $\alpha=\sqrt{2}+1$. From this example, we can see that there can be two experimental settings that give rise to the same Bell value, where the underlying systems enjoy the same amount of entanglement, yet the incompatibility between the local measurements can be significantly different.


%Since the value of $\theta_1$ is related to the chosen entanglement measure. As $\alpha$ increases from $1$, $\theta_1$ of negative conditional entropy departs from $\pi/4$ continuously and closes $\theta_1$ of  concurrence. For large $\alpha$, the $\theta_1$ of negative conditional entropy coincides with the $\theta_1$ of concurrence under the same $S$. 

\begin{figure}[hbt!]
    \centering
    \includegraphics[scale=0.42]{incompatibility_alpha0.eps}
    \caption{Illustration of the interplay among Bell nonlocality, measurement incompatibility, and entanglement. In this figure, $\alpha=\sqrt{2}+1$. The blue curves depict the results of one-way distillable entanglement, and the red curves depict the results of concurrence. The relation between entanglement and measurement incompatibility is similar to that in Fig.~\ref{incompatibility_alpha12}. Nevertheless, given any Bell value $S$ that is larger than $2\alpha$, which violates the $\alpha$-CHSH Bell inequality, the least amount of entanglement in the system at $\theta=\theta_{\max}$ is $1$, corresponding to the maximally entangled state. The feasible range of $\theta\in[\theta_{\min},\theta_{\max}]$ shrinks as $S$ grows and degenerates to the point of $\theta=\arctan(\sqrt{2}-1)$, where the Bell value reaches its maximum, $S=2\sqrt{2\sqrt{2}+4}$.}
\label{incompatibility_alpha0}
\end{figure}
%{\color{red}ZXJ: we may write the results for a general $\alpha>1$ in the main text (the results are worth noticing)}



\section{Optimizing entanglement estimation in realistic settings}\label{sc:numerical}
While the full probability distribution of a nonlocal behavior gives the complete description in a Bell test, for practical purposes, one often applies a Bell expression to characterize nonlocality.
As a given Bell expression only reflects a facet of the nonlocal behavior, one may expect a better entanglement estimation result via some well-chosen Bell expressions. In particular, realistic experiments unavoidably suffer from loss and noise in state transmission and detection. The robustness of such imperfections can differ for various Bell inequalities. In this section, we aim to specify when a non-trivial choice of $\alpha$-CHSH expression leads to better estimation. 
%Furthermore, to analytically examine the most accuracy one can estimate the amount of entanglement of an underlying state from a fixed CHSH simulated statistic, we derive the optimal estimation expression under special measurements settings. 
From an experimental point of view, the investigations may benefit experimental designs of DI information processing tasks. 
%Here, we restrict our discussions within the chosen model and specialize the conditions when a choice of $\alpha>1$ yields a better result. 
%Notably, the value of a specific Bell expression gives only one facet of the nonlocal behaviour.
%With respect to the simulated statistics, we shall study if a non-trivial choice of Bell expression will benefit entanglement estimation. 
For this purpose, we simulate the nonlocal correlations that arise from two sets of states: Non-maximally entangled pure states and Werner states. The deliberate use of non-maximally entangled states has been proved beneficial for observing nonlocal correlations under lossy detectors~\cite{eberhard1993background}. The Werner states characterize the typical effect of transmission noise upon entanglement distribution through fiber links~\cite{liu2018device,li2021experimental}.

With respect to the computation bases that define Pauli operators $\sigma_z$ on each local system, the measurements are parametrized as
\begin{equation}
\label{meas_setting}
\begin{split}
    \hat{A}_0&=\sigma_z, \\
    \hat{A}_1&=\cos\theta_1\sigma_z+\sin\theta_1\sigma_x, \\
    \hat{B}_0&=\cos\theta_2\sigma_z+\sin\theta_2\sigma_x, \\
    \hat{B}_1&=\cos\theta_3\sigma_z+\sin\theta_3\sigma_x,
\end{split}
\end{equation}
for Alice and Bob, respectively. We examine the optimal choice of $\alpha$ for DI entanglement estimation if the statistics arise from the two types of states.

%{\color{red}ZXJ: is it possible to extend our result to general $\theta_1$ (also including the analysis for Alice's side)? [We may organise the results in the following manner: (1) a general requirement on when $\alpha\neq1$ gives a non-trivial result (2) special case (Alice's side: $\sigma_z,\sigma_x$) (3) discussion on the two measures (if the optimal $\alpha$ in (2) is not the same for two measures under the same setting)]}
%Here, the eigenvectors of $\sigma_z$ are the basis states of each local system, $\{\ket{0},\ket{1}\}$, and the eigenvectors of $\sigma_x$ are taken as $(\ket{0}\pm\ket{1})/\sqrt{2}$. 

%{\color{red}ZXJ: Qubit subspaces in SPDC-based photonic platforms}



%{\color{red}ZXJ: This section: numerical simulation results for special state families, examine the performance of our entanglement estimation results (1) Consider two types of states (reason?) (2) Study the $\alpha$-dependence}

%In originally-designed CHSH test, the violation provides a lower bound for a chosen entanglement measure. However, in most DI settings, this lower bound is insufficient and inaccurate comparing the real value, and it does not make the most of the outputting data. In the previous section, we design a semi-DI framework via trusted measurement devices to make sure $C_{\rm est}(S)$ does not deviate too much from the real value. To extract more information about entanglement from CHSH test data, we increase $\alpha$ to adjust output statistics. We observe that, given specific input states(here we take tilted Bell state and Werner state as examples), under some DI settings, when turning up the value of alpha, $C_{\rm est}[S(\alpha),\alpha]$ returns a higher and better estimation for entanglement measurement.
\subsection{Non-maximally entangled states}
In the first simulation model, the underlying state is a non-maximally entangled state. We express the state on its Schmidt basis,
\begin{equation}
    \ket{\phi_{AB}(\delta)}=\cos\delta\ket{00}+\sin\delta\ket{11}.
\label{pure}
\end{equation}
where parameter $\delta\in[0,\pi/2]$ fully determines the amount of entanglement in the system. We first present the estimation result through a concrete example. We specify the underlying system by $\delta=\pi/6$ and the measurements by $\theta_1=\pi/2,\theta_2=\pi/6$ and $\theta_3=-\pi/6$. As shown in Fig.~\ref{fig:pure_entropy_concurrence}, we estimate the amount of negative conditional entropy and EOF with respect to the simulated statistics. The estimation results vary with respect to the value of $\alpha$.
%, with different estimated values characterized by different values of $\alpha$ in the $\alpha$-CHSH expressions. 
%We label a few points that correspond to the special case we now consider and depict the trajectory of the estimated amount of entanglement measures in the underlying system. 
The curves show that the original CHSH expression, corresponding to $\alpha=1$, does not yield the best entanglement estimation result for the given statistics. One obtains the best estimation results with the value of $\alpha$ roughly in the range $[1.4,1.6]$ for both EOF and negative conditional entropy.

%\textcolor{red}{The $\alpha_0$ where $E_{D,\rm est}^{\rightarrow}(S)$ reaches the optimal estimation coincides with the $\alpha_0$ where $C_{\rm est,semi-DI}(S)$ reaches the optimal estimation, under the same parameter settings. Since the ``optimal measurements" is one of the  tight estimation necessary conditions, how much this condition is held can be quantified by adjusting $\alpha$.}

%{\color{red}ZXJ: concurrence and (or) entanglement of formation?}

\begin{figure}[hbt!]
    \centering
    \includegraphics[width=0.48\textwidth]{pure_entropy_EOF_example.eps}
    \caption{Entanglement estimation results for nonlocal correlations arising from non-maximally entangled states. The experimental setting is given by $\delta=\pi/6,\theta_1=\pi/2,\theta_2=\pi/6$ and $\theta_3=-\pi/6$. We depict the entanglement estimation results when using different $\alpha$-CHSH Bell expressions. We plot the estimated values of one-way distillable entanglement and EOF with the black solid line and the red dashed line, respectively.
    %{\color{red}ZXJ: no need for depicting the Bell value range? (a bit confusing in the figure)}
    }
    %when the input pure state $\ket{\phi}_{AB}$ given in Eq.~\eqref{pure} is parametrized by $\delta=\pi/6$ and the measurements given in Eq.~\eqref{meas_setting} are set with $\theta_2=\pi/6$ and $\theta_3=-\pi/6$, 
    %this diagram illustrates trajectory of maximal entanglement entropy when $\alpha$ discretely increases from $1$ to $2$, with interval equals to $0.2$. When $\alpha$ is taken at some point between $1.4$ and $1.6$, it gives the optimal estimation of $H(A|B)$.
\label{fig:pure_entropy_concurrence}
\end{figure}

To see when better entanglement estimation is obtained with $\alpha>1$ for the family of non-maximally entangled states, we analytically derive the condition of the underlying system for the measure of EOF.
%For the measure of concurrence, we analytically derive the condition of the underlying system for which the best entanglement estimation result is obtained with $\alpha>1$. 
Using Eq.~\eqref{EOF_est_DI}, we have the following result.
%In a complete DI scenario without any assumption on the input dimension, the non-convexity of the estimation for pairs of qubits \eqref{concurrence_est} induces the convex closure estimation of concurrence and entanglement of formation, demonstrated by Eq.~\eqref{EOF_est_DI}. We focus on concurrence estimation in the following discussion.

\begin{theorem}\label{thm:DI_condition_pure}
    In a Bell test experiment, suppose the underlying state of the system takes the form of Eq.~\eqref{pure}, and the observables take the form of Eq.~\eqref{meas_setting}. For EOF estimation solely from the violation values of $\alpha$-CHSH Bell inequalities, if $\theta_1,\theta_2,\theta_3$ and $\delta$ satisfy
\begin{equation}
    \sin2\delta\sin\theta_1(\sin\theta_2-\sin\theta_3)+\cos\theta_2(\sqrt{2}+1+\cos\theta_1)+\cos\theta_3(\sqrt{2}+1-\cos\theta_1)>2(1+\sqrt{2}),
    \label{eq:DI_condition_pure}
\end{equation}
then there exists $\alpha>1$, where a better estimation of $E_{\rm F,est}(S)$ can be obtained by using the $\alpha$-CHSH inequality parameterized by this value than by using the original CHSH inequality (corresponding to $\alpha=1$).
\end{theorem}
Theorem~\ref{thm:DI_condition_pure} analytically confirms the nonlocality depicted by the original CHSH Bell value does not always provide the EOF estimation that approaches the real value most. When a fixed nonlocal behavior is given in a CHSH Bell test, once the non-maximally entangled state parameter $\delta$ in Eq.~\eqref{pure} and measurement parameters $\theta_1,\theta_2,\theta_3$ in Eq.~\eqref{meas_setting} satisfy Eq.~\eqref{eq:DI_condition_pure}, it is feasible to take a CHSH-type Bell value with $\alpha>1$ to estimate the EOF of the state. We leave the proof of Theorem~\ref{thm:DI_condition_pure} in  \ref{appendix:numerical}.

\noindent \textbf{Example.}
We take a special set of parameters in Eq.~\eqref{eq:DI_condition_pure} for an example. Suppose $\theta_1=\pi/2$ and $\theta_3=-\theta_2$, which resemble the optimal measurements in Eq.~\eqref{Bell_opt_meas} in form. Under this setting, we derive an explicit expression of $\alpha_0>1$, such that the estimation $E_{\rm F,est}(S)$ is optimal when taking $\alpha=\alpha_0$ in the CHSH inequality. When $0<\theta_2<\pi/4$, any non-maximally entangled state that satisfies
\begin{equation}
    \sin2\delta>(1+\sqrt{2})\frac{1-\cos\theta_2}{\sin\theta_2}
    \label{DI_condition_pure_delta}
\end{equation}
permits a better EOF estimation characterizing with some $\alpha>1$. When the state and measurements satisfy the condition in Eq.~\eqref{DI_condition_pure_delta}, one obtains the optimally estimated EOF when the parameter $\alpha$ equals
\begin{equation}
    \alpha_E^*=\frac{1}{2}\left(T-\frac{1}{T}\right)>1,
\end{equation}
where we denote $T=\frac{\sin2\delta\sin\theta_2}{1-\cos\theta_2}$. The optimally estimated EOF is then given by
\begin{equation}
    E_{\rm F,est}|_{\alpha_E^*}=\frac{1-\cos\theta_2}{2}(T^2+1).
\end{equation}
It is worth mentioning that if we have the additional assumption that the underlying state is a pair of qubits, we can analytically derive a more accurate estimation result of EOF. We leave the detailed conclusions and examples in  \ref{appendix:numerical}.


%\begin{theorem}
%  Suppose the underlying state of the system takes the form of Eq.~\eqref{pure} and the observables take the form of Eq.~\eqref{meas_setting}. For a fixed choice of $\alpha$, the estimation result in Eq.~\eqref{concurrence_est} is tight, which can be saturated by a set of observables,
  %which the arguments that correspond to the optimal estimation in Eq.~\eqref{concurrence_est} are parameterized by
%  \begin{equation}
%  \begin{split}
%      \hat{A}_0&=\pm\sigma_z,\\
%    \hat{A}_1&=\sigma_x, \\
%    \hat{B}_0&=\pm\cos\theta\sigma_z+\sin\theta\sigma_x%,\\
%    \hat{B}_1&=\pm\cos\theta\sigma_z-\sin\theta\sigma_x%,
%  \end{split}
%  \end{equation}
%  with $\tan\theta=\sin2\delta/\alpha$.
%\end{theorem}

%{\color{red}ZXJ: is (a similar) result first derived in %Ref.~\cite{antonio2012randomness}?}

%This form of measurement coincides with our initialization when $\theta_1=\pi/2,\theta_2+\theta_3=0$. Meanwhile, it is easy to verify that concurrence of any pure state itself saturates $C(\ket{\phi_{AB}})=C_{\rm est}(S)$ as long as $S$ is the optimal violation for $\ket{\phi}_{AB}$. Thus these explains when given $\delta,\theta_1,\theta_2,\theta_3$ in previous settings, there exists an $\alpha>1$ such that $C_{\rm est}(S)=C(\ket{\phi}_{AB})$.

%For a more special case, $\theta_2+\theta_3=0$ or $\theta_2+\theta_3=2\pi$, which corresponds to $k=\sqrt{2}/2$, if $-\frac{\pi}{2}<\theta_3-\theta_2<0$, then for any $\ket{\phi_{AB}(\delta)}$ in Eq~\eqref{pure}, it satisfies
%\begin{equation}
%    \sin2\delta>\tan\theta_2>0,\text{ when }0<\theta_2<\frac{\pi}{4}\text{ or }\pi<\theta_2<\frac{5\pi}{4}.
%\end{equation}
%Under these parameters, the optimal estimation of $C_{\rm est}(S)$ of the underlying state is obtained using the $\alpha$-CHSH inequality with $\alpha=\frac{\sin2\delta}{\tan\theta_2}$. 

%\begin{enumerate}[(1)]
    
%\item Further, we take $|\theta_2+\theta_3|=0$, i.e. $k=\sqrt{2}/2$, then
%\begin{gather}
        %\theta_2+\theta_3=0,\\
        %-\frac{\pi}{2}<|\theta_3-\theta_2|<0,
%\end{gather}
%more precisely, 
%\begin{gather}
%    0<\theta_2<\frac{\pi}{4}\text{ or }\pi<\theta_2<\frac{5\pi}{4}\\
%    \theta_3=-\theta_2,
%\end{gather}

%\item When $k$ in (\ref{relation_pure}) strictly in $(\sqrt{2}/2,1)$, then $\delta,\theta_2,\theta_3$ will give $\alpha>1$ that $C_{\rm est}[S(\alpha),\alpha]$ is strictly smaller than $C(\ket{\phi}_{AB})$, but still larger than that at $\alpha=1$.
%\end{enumerate}

\begin{comment}
An example is given by the following data: input state is $\ket{\phi}_{AB}$ in Eq.~\eqref{pure}) with $\delta=0.6$ in pure state, measurements are taken as 
\begin{equation}
    \hat{A}_0=\sigma_z,\hat{A}_1=\sigma_x;
\end{equation}
and 
\begin{equation}
    \hat{B}_0=\cos\theta_2\sigma_z+\sin\theta_2\sigma_x,\hat{B}_1=\cos\theta_2\sigma_z-\sin\theta_2\sigma_x;
\end{equation}
where $\theta_2=\pi/2-1.2$, under these device settings, 
\begin{equation}
    C_{\rm est}(S)=C_{\rm est}[S(\alpha),\alpha]\doteq(-0.1313\alpha^2+0.6296\alpha+0.1141)^{\frac{1}{2}}
\end{equation}
when $\alpha=1, C_{\rm est}(S)\doteq 0.7825$; when $\alpha=\alpha_0=\frac{\sin(1.2)}{\cot(1.2)}\doteq 2.3973>1, C_{\rm est}(S)=C(\rho)=\sin1.2\doteq 0.932$, which performs as an optimal estimation of concurrence since it equals to the real value. When $\alpha$ increases from $1$ to $\alpha_0$, $C_{\rm est}(S)$ increases; when $\alpha$ exceeds $\alpha_0$, $C_{\rm est}(S)$ decreases.

\begin{figure}[hbt!]
    \centering
    \includegraphics[width=0.8\textwidth]{pure_concurrence.eps}
    \caption{A figure of $C_{\rm est}(S)$ as the function of $\alpha$ when measurements are device-independently hold as $\theta=\pi/2-1.2$ and $
    \delta=0.6$. When $\alpha$ increased to $\sin1.2/\cot1.2$, the concurrence extracted from output statistics coincides with real input concurrence.}
\end{figure}

\end{comment}

\subsection{Werner states}
In the second simulation model, we consider the set of Werner states,
\begin{equation}
\label{werner}
    \rho_{\mathrm{W}}(p)=(1-p)\ketbra{\Phi^+}+p\frac{I}{4},
\end{equation}
where we write $\ket{\Phi^+}=(\ket{00}+\ket{11})/\sqrt{2}$. The Werner state is entangled when $p<2/3$. 
%This set of states is close to the final state of entanglement distribution through fiber links, where a maximally entangled state undergoes a depolarizing channel~\cite{liu2018device,li2021experimental}. 
Similarly, for the family of Werner states, there are examples that a non-trivial choice of $\alpha$-CHSH expression gives a better estimation result. In Fig.~\ref{fig:werner_entropy_concurrence}, we present such an example. In the simulation, the underlying system is parameterized by $p=0.05$, and the measurements are parameterized by $\theta_1=\pi/2,\theta_2=\pi/6$ and $\theta_3=-\pi/6$. The optimal estimation of negative conditional entropy is obtained with the value of $\alpha$ roughly in the range of $[1.2,1.4]$, while the optimal estimation of EOF is obtained when $\alpha\in [1,1.2]$. We derive an analytical result for the feasible region of state and measurement parameters that permits a better EOF estimation for a non-trivial value of $\alpha>1$.

\begin{figure}[hbt!]
    \centering
    \includegraphics[width=0.48\textwidth]{werner_entropy_EOF_example.eps}
    \caption{Entanglement estimation results for nonlocal correlations arising from Werner states. The experimental setting is given by $p=0.05,\theta_1=\pi/2,\theta_2=\pi/6$, and $\theta_3=-\pi/6$. We depict the entanglement estimation results using different $\alpha$-CHSH Bell expressions. We plot the estimated values of one-way distillable entanglement and EOF with the black solid line and the red dashed line, respectively.}
\label{fig:werner_entropy_concurrence}
\end{figure}

\begin{theorem}
\label{thm:DI_condition_werner}
    In a Bell test experiment, suppose the underlying state of the system takes the form of Eq.~\eqref{werner}, and the observables take the form of Eq.~\eqref{meas_setting}. For EOF estimation solely from the violation values of $\alpha$-CHSH Bell inequalities, if $\theta_1,\theta_2,\theta_3$ and $p$ satisfy
    %In a CHSH game where there is no knowledge in input dimension, suppose the underlying state of the system takes the form of Eq.~\eqref{werner} and the observables take the form of Eq.~\eqref{meas_setting}. When $\theta_1,\theta_2,\theta_3$ and $p$ satisfy
\begin{equation}
    (1-p)[\sin\theta_1(\sin\theta_2-\sin\theta_3)+\cos\theta_2(\sqrt{2}+1+\cos\theta_1)+\cos\theta_3(\sqrt{2}+1-\cos\theta_1)]>2(1+\sqrt{2}),
    \label{eq:DI_relation_werner}
\end{equation}
then there exists $\alpha>1$, where a better estimation of $E_{\rm F,est}(S)$ can be obtained by using the $\alpha$-CHSH inequality parameterized by this value than by using the original CHSH inequality (corresponding to $\alpha=1$).
\end{theorem}

Theorem~\ref{thm:DI_condition_werner} indicates that the nonlocality depicted by the original CHSH Bell value does not always provide the most accurate EOF estimation of Werner states. In a Bell test experiment, for Werner states parameter $p$ in Eq.~\eqref{werner} and measurement parameters $\theta_1,\theta_2,\theta_3$ in Eq.~\eqref{meas_setting} satisfy Eq.~\eqref{eq:DI_relation_werner}, it is helpful to take a CHSH-type Bell value with $\alpha>1$ to estimate EOF of the Werner state. We leave the proof and discussion of Theorem \ref{thm:DI_condition_werner} in  \ref{appendix:numerical}.

\noindent\textbf{Example.}
As a special example, we take the measurements setting in Eq.~\eqref{eq:DI_condition_pure} with $\theta_1=\pi/2,\theta_3=-\theta_2$, the same one as we use for the case study of non-maximally entangled states. For $0<\theta_2<\pi/4$, any Werner state in Eq.~\eqref{werner} with $p$ satisfying
\begin{equation}
    p<1-\frac{1}{(\sqrt{2}-1)\sin\theta_2+\cos\theta_2}
    \label{DI_condition_werner_p}
\end{equation}
promises a better estimation result of $E_{\rm F,est}$ by using an $\alpha$-CHSH expression with $\alpha>1$ in comparison with $\alpha=1$. The right hand side of Eq.~\eqref{DI_condition_werner_p} is upper bounded by $1-(\sqrt{2}+1)/(\sqrt{4+2\sqrt{2}})$. That is, only a Werner state with $p\leq 1-(\sqrt{2}+1)/(\sqrt{4+2\sqrt{2}})\doteq 0.0761$ is possible to yield the condition in Theorem \ref{thm:DI_condition_werner}. Denote $T=\frac{(1-p)\sin\theta_2}{1-(1-p)\cos\theta_2}$. Then for the underlying system of a Werner state and measurements satisfying Eq.~\eqref{DI_condition_werner_p}, the EOF estimation result reaches its optimal value with parameter $\alpha$
\begin{equation}
    \alpha_E^*=\frac{1}{2}\left(T-\frac{1}{T}\right)>1,
\end{equation}
and the estimation result is
\begin{equation}
    E_{\rm F,est}|_{\alpha_C^*}=1-\frac{p(2-p)}{2[1-(1-p)\cos\theta_2]}.
\end{equation}
Similarly, with an additional assumption on system dimension, we can obtain a more accurate EOF estimation result. We leave the details in  \ref{appendix:numerical}.


\section{Conclusions and discussion}
In this work, we study entanglement estimation via nonlocality, where we consider several entanglement measures for a family of generalized CHSH-type expressions. This family of Bell expressions allows us to effectively reduce the dimension of an unknown system to a pair of qubits, leading to results for particular entanglement measures like negativity, EOF, and one-way distillable entanglement. Under this framework, we also investigate the interplay among entanglement, measurement incompatibility, and nonlocality. While entanglement and measurement incompatibility are both necessary conditions for a nonlocal behavior, under a given nonlocal behavior, their interplay can be subtler than a simple trade-off relation. Given a Bell value, the measurements that require the minimum entanglement are not the most incompatible measurements in general. In addition, we also apply the entanglement estimation results in realistic scenarios. For non-maximally entangled states and Werner states, we analytically show that there exist state and measurements settings where a general CHSH Bell expression with $\alpha>1$ leads to better EOF estimation of the underlying state than the original CHSH expression.

When quantifying entanglement from nonlocality, the estimation results highly depend on the specific entanglement measures. Before our work, there are similar investigations focusing on different entanglement measures~\cite{moroder2013device,toth2015evaluating,arnon2017noise,chen2018exploring,arnon2019device}. A natural question is how nonlocality reflects different entanglement properties. Among various entanglement measures, a notable one is the distillable entanglement. While the necessity of entanglement for unveiling a nonlocal behavior is well-known, whether the underlying state in such a case can always be distilled into the “gold standard” in entanglement resource theory, namely the maximally entangled states, is not clear, which is known as the Peres conjecture~\cite{peres1999all,arnon2021upper}. When considering the entanglement estimation problem, the subtlety resides in that general entanglement distillation allows two-way communication, going beyond the one-way distillable entanglement estimation results in our work and Ref.~\cite{arnon2019device}. In addition, the problem is more involved in a multipartite scenario~\cite{masanes2006asymptotic}. The exact relation between nonlocality and entanglement distillability remains to be explored.

In studying the interplay of entanglement and measurement incompatibility under a given nonlocal behavior, we make some additional assumptions on the measurement observables to ease the quantification of measurement incompatibility. In the sense of a fully DI discussion, one may consider other incompatibility measures, such as the robustness of measurement incompatibility~\cite{designolle2019incompatibility}. Despite the freedom in measuring entanglement and measurement incompatibility, we believe our results unveil the subtlety of the interplay between these nonclassical notions, where more incompatible measurements may not compensate for the absence of entanglement and vice versa. From a resource-theoretic perspective, our results may indicate restrictions on the resource transformation between entanglement and measurement incompatibility in the sense of Bell nonlocality.

When applying our results to experiments, one may consider the practical issues in more detail. For instance, the problem of entanglement estimation via nonlocality can be generalized to the one-shot regime, where one considers dilution and distillation processes with a finite number of possibly non-i.i.d. quantum states. Notably, the results in Ref.~\cite{arnon2019device} provide an approach to estimating one-shot one-way distillable entanglement via nonlocality, and the techniques in Ref.~\cite{buscemi2011entanglement} may be applicable to the estimation of one-shot entanglement cost. We leave research in this direction for future works.

%We formulate the entanglement estimation problem into several steps and quantify the least amount of entanglement contained in the underlying system. Given a fixed nonlocal behavior, we observe a trade-off relation between entanglement and measurement incompatibility. Furthermore, we apply the entanglement estimation result on the set of non-maximally entangled state and Werner state, we find out that under some special parameter conditions, there exists $\alpha>1$, such that a better estimation can be obtained by using the $\alpha$-CHSH inequality comparing with the original CHSH inequality.% Under the semi-DI assumption that the input states are pairs of qubits, the concurrence estimation is more accurate than that in the DI scenario with the same parameter condition.


 %We realize an explicit expression for concurrence and a series of numerical results for  one-way distillable entanglement. By adjusting $\alpha$, we analytically extract estimations and numerical examples where sufficiently utilizes the nonlocal statistics based on the CHSH game for tilted pure states and Werner states. Further, by constructing a proper optimization model, we investigate the relation between entanglement and measurement incompatibility under a fixed violation. %With a given CHSH violation, we observe a perfect “trade-off” relation between one-way distillable entanglement and measurement incompatibility. (The convexity property of concurrence and entanglement of formation induces a non-monotonous ``trade-off" relation to measurement incompatibility.)

%{\color{red}1. Semi-DI 2. More general Bell inequalities, or other typical sets (analytical/efficient methods; note that there are already some results) 3. The difference between various entanglement measures 4. a more careful analysis of interplay, the scenario setting (without the dimension/PVM assumption)}

%Our main result of entanglement estimation is building on DI assumption. In the application of results, we illustrate a more accurate estimation under the semi-DI assumption where the input states are pairs of qubits. This inspire us that with other explicit semi-DI assumptions in real experiments, the estimation result can be more accurate.

%Similar with the CHSH inequality, the entanglement estimation problem can be considered under nonlocality depicted by other Bell inequalities. Also, a generalization that project on a facet determined by continuous extra parameters can be applied on the other Bell inequalities.

%Our entanglement estimation method can involve distinct entanglement measures with other operational meanings.

%When observing the interplay relation between entanglement, measurement incompatibility and nonlocality, a subtle model can be constructed without the semi-DI assumption where the input states are pairs of qubits. In this way, a general structure to quantify the measurement incompatibility will be introduced.


\ack
%\section{Acknowledgement}
%We acknowledge XXX for the insightful discussions.
This work was supported by the National Natural Science Foundation of China Grant No.~12174216 and the Innovation Program for Quantum Science and Technology Grant No.~2021ZD0300804.
%, the National Key R\&D Program of China Grants No.~XXX.

Y.Z.~and X.Z.~contributed equally to this work.

\appendix

\section{Reductions of the original optimization problem}
In Sec.~\ref{sc:framework}, we reduce the original optimization problem, including the essential steps of using Jordan's lemma to bypass the dimension problem and reducing a general two-qubit state to the Bell-diagonal state. Here we explain the two steps in detail.
%\subsection{Duality}

\subsection{Jordan's lemma}\label{appendix:jordan}
\begin{comment}
We apply Jordan's lemma to bypass the system dimension problem. Without loss of generality, we can treat the two measurement observables on each side as two unitary operators with eigenvalues $\pm1$. For such a case, Jordan's lemma guarantees the existence of a proper system representation, with which the two observables are simultaneously block-diagonalized and each block is at most two-dimensional. Hence we can denote Alice's measurement observables as
\begin{equation}
    \hat{A}_x=\sum_{\mu_A}\hat{\Pi}^{\mu_A}\hat{A}_x\hat{\Pi}^{\mu_A}=\bigoplus_{\mu_A}\hat{A}_x^{\mu_A},
\end{equation}
where $\hat{\Pi}^{\mu_A}$ are projectors onto orthogonal subspaces with dimension no larger than $2$, and $\hat{A}_x^{\mu_A}$ are qubit observables with eigenvalues $\pm1$. Bob's measurement observables can be represented in a similar way with projectors $\hat{\Pi}^{\mu_B}$. For simplicity, we dilate the alphabets for $\mu_A$ and $\mu_B$ and denote them by one index, $\mu$. Therefore, we can view the measurement process as first processing quantum state $\rho_{AB}$ with a block-dephasing map on each party into 
\begin{equation}
    \bar{\rho}_{AB}=\sum_{\mu}(\hat{\Pi}^{\mu}\otimes\hat{\Pi}^{\mu})\rho_{AB}(\hat{\Pi}^{\mu}\otimes\hat{\Pi}^{\mu})=\bigoplus_{\mu}p^{\mu}\rho_{AB}^{\mu},
\end{equation}
measuring each pair of qubits $\rho_{AB}^{\mu}$ with the associate observables, and finally mixing the statistics according to probability distribution $p^{\mu}$. We write $S=\sum_{\mu}p^{\mu}S^{\mu}$, where $S^{\mu}$ is the Bell value of measuring the pair of qubits labelled by $\mu$ with associate Bell observable. Note that the block-dephasing operation is performed locally on each side, hence it does not increase entanglement in the system. \textcolor{red}{Guaranteed by the convexity property of an entanglement measurement,} we can lower-bound the amount of entanglement in the initial system by studying the average amount of entanglement in the ensemble of qubit-pairs, $\{p^{\mu},\rho_{AB}^{\mu}\}$. In this way, we can essentially focus on quantifying entanglement in a pair of qubits. 
\end{comment}
%{\color{red}By further utilising the non-increasing property under LOCC of an entanglement measure and choosing proper local computational bases, we may further restrict the pair of qubits to be a diagonal state on the Bell-state basis. We leave the detailed derivation in Appendix~XXX.}
%induces the decomposition of the bipartite quantum state Hilbert space. By utilising Jordan's lemma,
%\begin{equation}
%    \begin{aligned}
%        \operatorname{Tr}[\hat{A}_{0(1)}\rho_A]&=\operatorname{Tr}[\bigoplus_\beta\hat{A}_{0(1)}^\beta\rho_A]\\
%        &=\operatorname{Tr}[\sum_\beta(\Pi_\beta\hat{A}_i\Pi_\beta) \rho_A]\\
%        &=\operatorname{Tr}[\sum_\beta \hat{A}_{0(1)}p^\beta_A\rho_A^\beta]\\
%        &=\sum_\beta p^\beta_A\operatorname{Tr}[\hat{A}_{0(1)}\rho_A^\beta],
%    \end{aligned}
%\end{equation}
%where $\Pi_\beta$ is the projection onto the $\beta-$subspace. Same thing can do on Bob's subsystem. Therefore, the quantum violation in each round of the non-local game can be deduced to the mixture of two qubit violations. We thus can further deduce our optimization problems (\ref{dual_optm}) on $\rho_{AB}^\beta\in\mathcal{H}_2\otimes\mathcal{H}_2$.

Before applying Jordan's lemma to effectively reduce the system dimension, we first note that without loss of generality, the underlying measurements in a Bell test can be taken as projective in a DI analysis. Consider either side of the Bell test, where a measurement setting is characterized by a POVM with elements $\{\hat{M}_i\}_i$ and acts on the state $\rho$. Then, by applying Naimark's dilation theorem~\cite{neumark1943representation}, there exists a quantum channel $F$ that embeds $\rho$ with an ancillary state into a quantum state in some high-dimension Hilbert space, such that for every $\hat{M}_i$, there exists a projector $\hat{V}_i$,
\begin{equation}
    \Tr (\rho \hat{M}_i)=\Tr[\hat{V}_iF(\rho)\hat{V}_i],
\end{equation}
and $\{\hat{V}_i\}_i$ forms a valid projective measurement with $\sum_i \hat{V}_i=\hat{I}$. The dilation only takes a local ancillary state and does not change the measurement statistics. Therefore, in a fully DI setting, one can simply restrict the analysis to projective measurements. In the CHSH-type Bell test, we are dealing with binary observables. Since they can be taken as projective, they can also be described by Hermitian operators with eigenvalues $\pm1$.
% There is no assumption of the measurement we use to depict the problem. There are the general POVMs we deal with. By applying Naimark's dilation theorem, 
% without loss of generality, we can treat the two measurement observables on each side as projective ones with eigenvalues $\pm1$. The Naimark's theorem guarantees that for any POVM element, $B$, applying on a state $\rho$, there exists a projection measurement operator, $V$, such that
% \begin{equation}
%     \Tr (\rho B)=\Tr (VF(\rho)V^*),
% \end{equation}
% where $F$ projects $\rho$ to a density matrix in a higher dimensional Hilbert space by introducing only local ancillas.

We apply Jordan's lemma to bypass the system dimension problem. The description of Jordan's lemma is given below, with the proof can be found in~\cite{Pironio2009device}.

\begin{lemma}
    Suppose $\hat{A}_0$ and $\hat{A}_1$ are two Hermitian operators with eigenvalues $\pm 1$ that act on a Hilbert space with a finite or countable dimension, $\mathcal{H}$. Then there exists a direct-sum decomposition of the system, $\mathcal{H}=\bigoplus\mathcal{H}^{\mu}$, such that $\hat{A}_0=\bigoplus\hat{A}_0^{\mu}$, $\hat{A}_1=\bigoplus\hat{A}_1^{\mu}$, $\hat{A}_0^{\mu},\hat{A}_1^{\mu}\in\mathcal{L}(\mathcal{H}^{\mu})$, where the sub-systems satisfy $\operatorname{dim}\mathcal{H}^{\mu}\leq 2,\forall {\mu}$.
\end{lemma}

Jordan's lemma guarantees that the two possible observables measured by Alice can be represented as
\begin{equation}
  \hat{A}_x=\sum_{\mu}\hat{\Pi}^{\mu_A}\hat{A}_x\hat{\Pi}^{\mu_A}=\bigoplus_{\mu_A}\hat{A}_x^{\mu_A},
\end{equation}
where $x\in\{0,1\}$, $\hat{\Pi}^{\mu_A}$ are projectors onto orthogonal subspaces with dimension no larger than $2$, and $\hat{A}_x^{\mu_A}$ are qubit observables with eigenvalues $\pm1$. A similar representation applies to Bob's measurement observables. Due to the direct-sum representation, one can regard the measurement process as first applying a block-dephasing operation to the underlying quantum system. Consequently, one can equivalently regard the measurement process as measuring the following state,
\begin{equation}
    \bar{\rho}_{AB}=\sum_{\mu}(\hat{\Pi}^{\mu_A}\otimes\hat{\Pi}^{\mu_B})\rho_{AB}(\hat{\Pi}^{\mu_A}\otimes\hat{\Pi}^{\mu_B})=\bigoplus_{\mu}p^{\mu}\rho_{AB}^{\mu}.
\end{equation}
Here we relabel the indices with $\mu\equiv\{\mu_A,\mu_B\}$. As the block-dephasing operators act locally on each side, the measurement process does not increase entanglement in the system. Therefore, we can lower-bound the amount of entanglement in the initial system by studying the average amount of entanglement in the ensemble of qubit-pairs, $\{p^{\mu},\rho_{AB}^{\mu}\}$. 

Consequently, the expected CHSH Bell value in a test is the linear combination of the Bell values for the qubit pairs, $S=\sum_{\mu}p^{\mu}S^{\mu}$. Note that an observer cannot access to the probability distribution, $p^\mu$, and the Bell values for each pair of qubits, $S^\mu$, but only the expected Bell value, $S$, hence the final DI entanglement estimation result should be a function of $S$. On the other hand, we shall first derive entanglement estimation results for each pair of qubits in the form of $E_{\rm est}(S^\mu)$. It is thus essential to consider the convexity of the function, $E_{\rm est}$. If the function is not convex in its argument, i.e., $E_{\rm est}(\sum_\mu p^\mu S^\mu)$ is not smaller than $\sum_\mu p^\mu E_{\rm est}(S^\mu)$, one needs to take the convex closure of $E_{\rm est}$ to obtain a valid lower bound that holds for all possible configurations giving rise to the expected Bell value, $S$. 


\begin{comment}
\begin{proof}
It can be checked that $(\hat{A}_1\hat{A}_0)^\dag(\hat{A}_1\hat{A}_0)=I$, thus $\hat{A}_1\hat{A}_0$ is a unitary matrix. Take $\ket{{\mu_A}}$ as an eigenstate of $\hat{A}_1\hat{A}_0$, 
\begin{gather}
    \hat{A}_1\hat{A}_0\ket{{\mu_A}}=\omega_{\mu_A}\ket{{\mu_A}}, \\
    |\omega_{\mu_A}|=1,
\end{gather} 
it can also be checked that
\begin{equation}
    \hat{A}_1\hat{A}_0(\hat{A}_1\ket{\mu_A})=\hat{A}_1\hat{A}_0^\dag\hat{A}_1^\dag\ket{\mu_A}=\hat{A}_1(\hat{A}_1\hat{A}_0)^\dag\ket{\mu_A}=\omega_{\mu_A}^*\hat{A}_1\ket{\mu_A}. 
\end{equation}
The unitarity of the matrix $\hat{A}_1\hat{A}_0$ certifies that the eigenvalues of $\hat{A}_1\hat{A}_0$ span $\mathcal{H}$. Suppose $\mathcal{H}=\bigoplus\mathcal{H}^{\mu_A}_2$, where $\mathcal{H}^{\mu_A}_2:=\operatorname{Span}\{\ket{\mu_A},\hat{A}_1\ket{\mu_A}\}$ with dimension no larger than 2. $\mathcal{H}^{\mu_A}_2$ is invariant under $\hat{A}_0$ and $\hat{A}_1$ since
    \begin{gather}
        \hat{A}_0\ket{\mu_A}=\hat{A}_1(\hat{A}_1\hat{A}_0)\ket{\mu_A}=\omega_{\mu_A}(\hat{A}_1\ket{\mu_A}),\\
        \hat{A}_0(\hat{A}_1\ket{\mu_A})=\hat{A}_1(\hat{A}_1\hat{A}_0\hat{A}_1)\ket{\mu_A}=\omega^*_{\mu_A}\ket{\mu_A},\\
        \hat{A}_1\ket{\mu_A}=\hat{A}_1\ket{\mu_A},\\
        \hat{A}_1(\hat{A}_1\ket{\mu_A})=\ket{\mu_A}.
    \end{gather}
Here we end the proof.
\end{proof}
\end{comment}

\subsection{Restriction to Bell-diagonal states}
\label{appendix:Bell-diagonal}
Following the route in Fig.~\ref{fig:flowchart}, the feasible region of the state variables in an entanglement estimation problem can be effectively restricted to the set of Bell-diagonal states on the two qubit systems. We present the following lemma.
\begin{lemma}
    Suppose the underlying system in a CHSH-type Bell test lies in a two-qubit state, $\rho_{AB}$. Then there exists an LOCC that transforms $\rho_{AB}$ into an ensemble of Bell states, $\rho_\lambda$, without changing the expected Bell value.
    %Under a fixed computational basis, suppose a two-qubit state, $\rho_{AB}$, can be transformed into a Bell-diagonal state, $\rho_\lambda$, through LOCC. If $\rho_{AB}$ leads to value $S$ for the $\alpha$-CHSH expression with proper Bell-test observables, then there exists Bell operator $\hat{S}_\alpha$ for $\rho_{\lambda}$, such that $S=\Tr(\rho_{\lambda}\hat{S}_\alpha)$.
\end{lemma}
\begin{proof}
     In a CHSH-type Bell test, we transform an arbitrary pair of qubits $\rho_{AB}$ into a Bell-diagonal state $\rho_\lambda$ via three steps of LOCC. In each step, We verify that the $\alpha$-CHSH Bell values are equal for the states before and after the transformation with the same measurements.
    
    \textit{Step 1:} 
    In a CHSH Bell test, Alice and Bob fix their local computational bases, or, the axes of the Bloch spheres on each side. As there are only two observables on each side, one can represent them on the $x-z$ plane of the Bloch sphere without loss of generality. Then Alice and Bob flip their measurement results simultaneously via classical communication with probability $1/2$. This operation can be interpreted as transforming $\rho_{AB}$ into the following state,
    \begin{equation}
        \rho_1=\frac{1}{2}[\rho_{AB}+(\sigma_2\otimes\sigma_2)\rho_{AB}(\sigma_2\otimes\sigma_2)].
    \end{equation}
    %It can be checked that $\ket{\Phi^+}$ and $\ket{\Psi^-}$ are eigenstates of $\sigma_y\otimes\sigma_y$ with eigenvalue 1, while $\ket{\Phi^+},\ket{\Psi^-}$ with $-1$. 
    %Note that one should not mistake the subscripts of $x$ and $y$ for the Pauli operators with the ones for the measurement observables of Alice and Bob.
    To avoid confusion about the subscripts, we use the following convention to denote the Pauli operators in the Appendix,
    \begin{equation}
    \begin{split}
        \sigma_x&\equiv\sigma_1, \\
        \sigma_y&\equiv\sigma_2, \\
        \sigma_z&\equiv\sigma_3.
    \end{split}
    \end{equation}
    Under the Bell basis determined by the local computational bases, $\{\ket{\Phi^+},\ket{\Psi^-},\ket{\Phi^-},\ket{\Psi^+}\}$, $\rho_1$ can be denoted as
    \begin{equation}
    \rho_1=\begin{bmatrix} 
        \lambda_{\Phi^+} &l_1e^{i\phi_1} &0 &0 \\
        l_1e^{-i\phi_1} &\lambda_{\Psi^-} &0 &0 \\
        0 &0 &\lambda_{\Psi^+} &l_2e^{i\phi_2} \\
        0 &0 &l_2e^{-i\phi_2} &\lambda_{\Phi^-}
        \end{bmatrix},
    \end{equation}
where $\lambda_i,i=\Phi^\pm,\Psi^\pm$ are  eigenvalues of the corresponding Bell state and $l_1,l_2,\phi_1,\phi_2$ are off-diagonal parameters. It can be verified that the statistics of measuring $\sigma_i\otimes\sigma_j$ for $i,j=1,3$ are invariant under the operation of $\sigma_2\otimes\sigma_2$. Thus 
\begin{equation}
    \Tr [(\sigma_2\otimes\sigma_2)\rho_{AB}(\sigma_2\otimes\sigma_2)(\hat{A}_x\otimes\hat{B}_y)]=\Tr [\rho_{AB}(\hat{A}_x\otimes\hat{B}_y)]
\end{equation}
for $\hat{A}_x$ and $\hat{B}_y$, $x,y=0,1$, which indicates that $\rho_1$ and $\rho_{AB}$ share the common Bell value.

\textit{Step 2:} In this step, we apply LOCC to transform $\rho_1$ into a state where the off-diagonal terms on the Bell basis become imaginary numbers. For this purpose, Alice and Bob can each apply a local rotation around the $y$-axes of the Bloch spheres on their own systems for an angle $\theta$ by
\begin{equation}
    R_y(\theta)=\cos{\frac{\theta}{2}}I+i\sin{\frac{\theta}{2}}\sigma_2=\left(
    \begin{smallmatrix}
    \cos{\frac{\theta}{2}} & \sin{\frac{\theta}{2}} \\
    -\sin{\frac{\theta}{2}} & \cos{\frac{\theta}{2}}
    \end{smallmatrix}
    \right),
\end{equation}
with its action on a general observable characterized by $\gamma$ residing in the $x-z$ plane given by
    \begin{equation}
        R_y(\theta)(\cos\gamma\sigma_1+\sin\gamma\sigma_3)=\cos(\gamma+\frac{\theta}{2})\sigma_1+\sin(\gamma+\frac{\theta}{2})\sigma_3.
        \label{rotation}
    \end{equation}
%around the local $y$-axes on the Bloch spheres labelled by $\alpha$ and $\beta$ on the systems of Alice and Bob, respectively, to turn the off-diagonal terms of 
After applying the operation, the resulting state becomes $\rho_2=[R_y(\alpha)\otimes R_y(\beta)]\rho_1[R_y(-\alpha)\otimes R_y(-\beta)]$, where the off-diagonal terms undergo the following transformations,
\begin{gather}
    l_1e^{i\phi_1}\rightarrow\frac{1}{2}(\lambda_{\Phi^+}-\lambda_{\Psi^-})\sin(\alpha-\beta)+l_1\cos\phi_1\cos(\alpha-\beta)+l_1\sin\phi_1 i,\label{terms_rotation1}\\
    l_2e^{i\phi_2}\rightarrow\frac{1}{2}(\lambda_{\Phi^-}-\lambda_{\Psi^+})\sin(\alpha+\beta)+l_2\cos\phi_2\cos(\alpha+\beta)+l_2\sin\phi_2 i.\label{terms_rotation2}
\end{gather}
%Here, the local rotation around $y$ axis is define as %R_y(\theta)=\cos(\theta/2)I+i\sin(\theta/2)\sigma_y=\left(
%    \begin{smallmatrix}
%    \cos(\theta/2) & \sin(\theta/2) \\
%    -\sin(\theta/2) & \cos(\theta/2)
%    \end{smallmatrix}
%    \right)$, 
By choosing $\alpha$ and $\beta$ properly, the real parts in the off-diagonal terms of $\rho_2$ can be eliminated. Similarly, as in the \textit{Step 1}, the measurement of $\sigma_i\otimes\sigma_j$ for $i,j=1,3$ remains invariant under local rotations around the $y$-axes, which indicates $\rho_2$ and $\rho_1$ give the same Bell value under the same measurements.

\textit{Step 3:}
Note that $\rho_2$ and $\rho_2^*$ give the same Bell value under the given measurements,
\begin{equation}
    \Tr[\rho_2(\sigma_i\otimes\sigma_j)]=\Tr[\rho_2^*(\sigma_i\otimes\sigma_j)],i,j=1,3.
\end{equation}
Hence without loss of generality, one can take the underlying state in the Bell test as $\rho_{\lambda}=(\rho_2+\rho_2^*)/2$, which is a Bell-diagonal state. 
%To sum up, the $\alpha-$CHSH Bell value of $\rho_{AB}$ and the corresponding $\rho_{\lambda}$ are equal under the same measurements.
\end{proof}

%The states to be considered in the entanglement estimation problem are simplified to Bell-diagonal states. 
Based on the above simplification, we represent Eq.~\eqref{equivalent_optm} under Bell-diagonal states, which leads to the following lemma.
\begin{lemma}
%[Maximal CHSH-type violation of a Bell diagonal state]
The maximal value of the $\alpha$-CHSH expression in Eq.~\eqref{CHSH_type} for a Bell-diagonal state shown in Eq.~\eqref{Bell_diagonal}, $\rho_{\lambda}$, is given by
\begin{equation}
    S= 2\sqrt{\alpha^2(\lambda_{1}+\lambda_{2}-\lambda_{3}-\lambda_{4})^2+(\lambda_{1}-\lambda_{2}+\lambda_{3}-\lambda_{4})^2},
    \label{eq:maximal_value_appendix}
\end{equation}
where $\lambda_i$ is the $i$-th largest eigenvalue of $\rho_{\lambda}$.
\end{lemma}
\begin{proof}
In an $\alpha$-CHSH Bell test, measurements corresponding to non-degenerate Pauli observables can be expressed as
\begin{gather}
        \hat{A_x}=\Vec{a}_x\cdot\Vec{\sigma},\\
        \hat{B_y}=\Vec{b}_y\cdot\Vec{\sigma},
\end{gather}
where $\Vec{\sigma}=(\sigma_1,\sigma_2,\sigma_3)$ are three Pauli matrices, and $\Vec{a}_x=(a_x^1,a_x^2,a_x^3)$ and $\Vec{b}_y=(b_y^1,b_y^2,b_y^3)$ are unit vectors for $x,y=0,1$. A Bell-diagonal state shown in Eq.~\eqref{Bell_diagonal} can be expressed on the Hilbert-Schmidt basis as
\begin{equation}
    \rho_{\lambda}=\frac{1}{4}\left(I+\sum_{i,j=1}^3 T_{\lambda,ij}\sigma_i\otimes\sigma_j\right),
\end{equation}
where 
\begin{equation}
    T_{\lambda}=
    \begin{bmatrix} 
        (\lambda_1 + \lambda_3)-(\lambda_2 +\lambda_4) &0 &0  \\
        0 &(\lambda_3 + \lambda_2)-(\lambda_1 + \lambda_4) &0 \\
        0 &0 &(\lambda_1 + \lambda_2)-(\lambda_3 + \lambda_4)
    \end{bmatrix}
    \label{T_lambda}
\end{equation}
is a diagonal matrix. The $\alpha$-CHSH expression in Eq.~\eqref{CHSH_type} can be expressed in terms of $T_{\lambda}$ as
\begin{equation}
    \begin{aligned}
        %I_\alpha=
        &\Tr\left\{\alpha\rho_{\lambda}(\Vec{a}_0\cdot\Vec{\sigma})\otimes[(\Vec{b}_0+\Vec{b}_1)\cdot\Vec{\sigma}]+\rho_{\lambda}(\Vec{a}_1\cdot\sigma)\otimes[(\Vec{b}_0-\Vec{b}_1)\cdot\Vec{\sigma}]\right\}\\
        =&\alpha[\Vec{a}_0\cdot T_{\lambda}(\Vec{b}_0+\Vec{b}_1)]+[\Vec{a}_1\cdot T_{\lambda}(\Vec{b}_0-\Vec{b}_1)].
    \end{aligned}
\end{equation}
Following the method in Ref.~\cite{horodecki1995violating}, we introduce a pair of normalized orthogonal vectors, $\Vec{c}_0$ and $\Vec{c}_1$,
\begin{gather}
    \Vec{b}_0+\Vec{b}_1=2\cos\theta\Vec{c}_0,\\
    \Vec{b}_0-\Vec{b}_1=2\sin\theta\Vec{c}_1
\end{gather}
where $\theta\in[0,\pi/2]$. This gives the maximal $\alpha$-CHSH Bell value,
\begin{equation}
    \begin{aligned}
        S%\max_{\Vec{a}_0,\Vec{a}_1,\Vec{c}_0,\Vec{c}_1,\theta}\alpha(\Vec{a}_0\cdot2T_{\lambda}\cos\theta\Vec{c}_0)+(\Vec{a}_1,2T_{\lambda}\sin\theta\Vec{c}_1)\\
        &=\max_{\Vec{a}_0,\Vec{a}_1,\Vec{c}_0,\Vec{c}_1,\theta}2\alpha\cos\theta(\Vec{a}_0\cdot T_{\lambda}\Vec{c}_0)+2\sin\theta(\Vec{a}_1\cdot T_{\lambda}\Vec{c}_1).
    \end{aligned}
\end{equation}
The maximization of the Bell value is taken over parameters $\Vec{a}_x,\Vec{b}_y$ for $x,y=0,1$, with the parameters $\lambda_{i}$ fixed. We obtain
\begin{equation}
    \begin{aligned}
        S&=\max_{\Vec{c}_0,\Vec{c}_1,\theta}2\alpha\cos\theta|T_{\lambda}\Vec{c}_0|+2\sin\theta|T_{\lambda}\Vec{c}_1|\\
        &=\max_{\Vec{c}_0,\Vec{c}_1}2\sqrt{\alpha^2|T_{\lambda}\Vec{c}_0|^2+|T_{\lambda}\Vec{c}_1|^2},\label{max_CHSH_vio_Bell_c}
    \end{aligned}
\end{equation}
where the first equality in Eq.~\eqref{max_CHSH_vio_Bell_c} is saturated when $\Vec{a}_x=T_{\lambda}\Vec{c}_x/|T_{\lambda}\Vec{c}_x|, x=0,1$, and the second inequality is saturated when $\tan\theta=|T_{\lambda}\Vec{c}_1|/(\alpha|T_{\lambda}\Vec{c}_0|)$. Since $\alpha>1$ and $\Vec{c}_0$ and $\Vec{c}_1$ are orthonormal vectors, the maximum of the second line in Eq.~\eqref{max_CHSH_vio_Bell_c} is obtained when $|T_{\lambda}\Vec{c}_0|$ and $|T_{\lambda}\Vec{c}_1|$ equal to the absolute values of the largest and the second largest eigenvalues of $T_{\lambda}$, respectively. Without loss of generality, we assume $\lambda_1\geq\lambda_2\geq\lambda_3\geq\lambda_4$ in $\rho_{\lambda}$. This leads to the ordering of the absolute values of the elements of $T_{\lambda}$,
\begin{equation}
\begin{split}
    |T_{\lambda,33}|&=|(\lambda_{1}-\lambda_{4})+(\lambda_{2}-\lambda_{3})|\geq|(\lambda_{1}-\lambda_{4})-(\lambda_{2}-\lambda_{3})|=|T_{\lambda,11}|,\\
    |T_{\lambda,11}|&=|(\lambda_{3}-\lambda_{4})+(\lambda_{1}-\lambda_{2})|\geq|(\lambda_{3}-\lambda_{4})-(\lambda_{1}-\lambda_{2})|=|T_{\lambda,22}|.
\end{split}
\end{equation}
Thus, the second line in Eq.~\eqref{max_CHSH_vio_Bell_c} reaches its maximum when $|\Vec{c}_0|=(0,0,1)^T$ and $|\Vec{c}_1|=(1,0,0)^T$. Therefore, for any given Bell-diagonal state $\rho_{\lambda}$ in Eq.~\eqref{Bell_diagonal}, the maximal $\alpha$-CHSH Bell value is
\begin{equation}
    S= 2\sqrt{\alpha^2(\lambda_{1}+\lambda_{2}-\lambda_{3}-\lambda_{4})^2+(\lambda_{1}-\lambda_{2}+\lambda_{3}-\lambda_{4})^2},
\end{equation}
where measurements for $\rho_{\lambda}$ to achieve the maximal Bell value, i.e., optimal measurements, are given by
\begin{equation}
    \begin{gathered}
        \hat{A}_0=\pm\sigma_z,\\
        \hat{A}_1=\pm\sigma_x,\\
        \hat{B}_0=\pm\cos\theta\sigma_z\pm\sin\theta\sigma_x,\\
        \hat{B}_1=\pm\cos\theta\sigma_z\mp\sin\theta\sigma_x,
    \end{gathered}
    \label{opt_meas1}
\end{equation}
    or
\begin{equation}
    \begin{gathered}
        \hat{A}_0=\pm\sigma_z,\\
        \hat{A}_1=\mp\sigma_x,\\
        \hat{B}_0=\pm\cos\theta\sigma_z\mp\sin\theta\sigma_x,\\
        \hat{B}_1=\pm\cos\theta\sigma_z\pm\sin\theta\sigma_x,
    \end{gathered}
    \label{opt_meas2}
\end{equation}
with $\tan\theta=(\lambda_{1}-\lambda_{2}+\lambda_{3}-\lambda_{4})/[\alpha(\lambda_{1}+\lambda_{2}-\lambda_{3}-\lambda_{4})]$.
\end{proof}

From the proof, we see that any Bell-diagonal state $\rho_{\lambda}$ in Eq.~\eqref{Bell_diagonal} reaches its maximal Bell value of Eq.~\eqref{CHSH_type} when measurements are taken in the form Eq.~\eqref{opt_meas1} or Eq.~\eqref{opt_meas2}. In other words, measurements in Eq.~\eqref{opt_meas1} and Eq.~\eqref{opt_meas2} are the optimal measurements for $\rho_{\lambda}$ that yield the largest $\alpha$-CHSH Bell value. To solve the simplified entanglement estimation problem in Eq.~\eqref{equivalent_optm} for Bell-diagonal states, we need to solve the optimal measurements first. Given a general pair of qubits $\rho_{AB}$, the maximal $\alpha$-CHSH Bell value $S$ for $\rho_{AB}$ is expressed as a function of $T_{ij}$,
\begin{equation}
    S=[2(\alpha^2+1)(T_{11}^2+T_{13}^2+T_{31}^2+T_{33}^2)+2(\alpha^2-1)\sqrt{(T_{11}^2-T_{13}^2+T_{31}^2-T_{33}^2)^2+4(T_{11}+T_{13}+T_{31}+T_{33})^2}]^{1/2},
\label{max_CHSH_vio_gen}
\end{equation}
where $T_{ij}=\Tr[\rho_{AB}(\sigma_i\otimes\sigma_j)]$ is the coefficient of $\rho_{AB}$ under Hilbert-Schmidt basis. When $\rho_{AB}$ is Bell-diagonal, Eq.~\eqref{max_CHSH_vio_gen} degenerates to Eq.~\eqref{eq:maximal_value_appendix}.
%Now we complete the tools to prove the lemma~\ref{lemma:LOCC}. To finish the proof, an equivalent issue is to explain $\rho_{AB}$ and its corresponding $\rho_{\lambda}$ share the same maximal $\alpha$-CHSH Bell value, $S$. If this thing is true, then any violation in $(2\alpha,S]$ can be obtained from a continuous transformation on the measurement parameters, for both $\rho_{AB}$ and $\rho_{\lambda}$. Followed by Eq.~\eqref{max_CHSH_vio_gen}, the maximal $\alpha$-CHSH Bell value of any $\rho_{AB}$ consists of elements from the corresponding $T$ matrix. In \textit{Step 1} of proof of lemma~\ref{lemma:LOCC}, the value of $T_{11},T_{13},T_{22},T_{31},T_{33}$ in $T$ matrix of $\rho_{AB}$ and $\rho_1$ correspondingly equal; in \textit{Step 2}, $\rho_{AB}$ and $\rho_{2}$ reach the same maximal Bell value, with their optimal measurements merely differs from a shifting of angles; in \textit{Step 3}, $\rho_2$ and $\rho_\lambda$ share the common $T$ matrix and thus the common maximal $\alpha$-CHSH Bell value. Here we finish the proof of lemma~\ref{lemma:LOCC}.
%\begin{equation}
%    \rho_1=\frac{1}{4}\begin{bmatrix} 
%    1+t_{11}-t_{22}+t_{33} &t_{31}-t_{13}+(r_2-s_2)i &0 &0 \\
%    t_{31}-t_{13}-(r_2-s_2)i &1-t_{11}-t_{22}-t_{33} &0 &0 \\
%    0 &0 &1-t_{11}+t_{22}+t_{33} &t_{31}+t_{13}-(r_2+s_2)i \\
%    0 &0 &t_{31}+t_{13}+(r_2+s_2)i &1+t_{11}+t_{22}-t_{33}
%    \end{bmatrix},
%\end{equation}

\section{Proof of the lower bound of concurrence}
\label{appendix:concurrence_est}
In this section, we prove the analytical concurrence estimation result via the $\alpha$-CHSH Bell value. Here we restrict the underlying state as a pair of qubits.

\begin{theorem}
    Suppose the underlying quantum state is a pair of qubits. For a given $\alpha$-CHSH expression in Eq.~\eqref{CHSH_type} parametrized by $\alpha$, if the Bell value is $S$, then the amount of concurrence in the underlying state can be lower-bounded,
\begin{equation}
    C(\rho_{AB})\geq \sqrt{\frac{S^2}{4}-\alpha^2}.
\end{equation}
The equality can be saturated when measuring a Bell-diagonal state in Eq.~\eqref{Bell_diagonal} with eigenvalues
\begin{equation}
    \begin{gathered}
        \lambda_1=\frac{1}{2}+\frac{1}{2}\sqrt{\frac{S^2}{4}-\alpha^2},\\
        \lambda_2=\frac{1}{2}-\frac{1}{2}\sqrt{\frac{S^2}{4}-\alpha^2},\\
        \lambda_3=\lambda_4=0,
    \end{gathered}
\end{equation}
using measurements in Eq.~\eqref{Bell_opt_meas} with $\theta=\arctan(\frac{1}{\alpha}\sqrt{\frac{S^2}{4}-\alpha^2})$.
\end{theorem}

\begin{proof}
Given any Bell value  $S\in(2\alpha,2\sqrt{1+\alpha^2}]$, we aim to determine the least amount of concurrence that is required to support the Bell value, $S$. We solve the simplified optimization problem in Eq.~\eqref{equivalent_optm}, restricting the underlying state as a Bell-diagonal state in Eq.~\eqref{Bell_diagonal} and taking the objective entanglement measure as $C(\cdot)$,
\begin{equation}
    \begin{split}
    C_{\rm est} & =\underset{\lambda_i,i=1,2,3,4}{\text{min}} \max\{0,2\lambda_{1}-1\}, \\
    \text{s.t.}\quad
    S &=2\sqrt{\alpha^2(\lambda_1+\lambda_2-\lambda_3-\lambda_4)^2+(\lambda_1-\lambda_2+\lambda_3-\lambda_4)^2},\\
    \lambda_1 & \geq\lambda_2\geq\lambda_3\geq\lambda_4,\\
    1 & =\sum_{i=1}^4\lambda_i, \lambda_i\geq 0 \; ,i=1,2,3,4.
    \end{split}
    \label{concurrence_optm}
\end{equation}
We first reduce the number of variables to simplify the optimization in Eq.~\eqref{concurrence_optm}. Since the variables in Eq.~\eqref{concurrence_optm} are not independent with each other, we express variables $\lambda_1$ and $\lambda_4$ as functions of variables $\lambda_2$ and $\lambda_3$, 
\begin{gather}
\lambda_{\max}=\lambda_{1}=\frac{1}{2}-\frac{1}{\alpha^2+1}(\alpha^2\lambda_{2}+\lambda_{3})+\frac{1}{\alpha^2+1}\sqrt{\frac{S^2(\alpha^2+1)}{16}-\alpha^2(\lambda_{2}-\lambda_{3})^2},\label{lambda_max}\\
\lambda_{\min}=\lambda_{4}=\frac{1}{2}-\frac{1}{\alpha^2+1}(\lambda_{2}+\alpha^2\lambda_{3})-\frac{1}{\alpha^2+1}\sqrt{\frac{S^2(\alpha^2+1)}{16}-\alpha^2(\lambda_{2}-\lambda_{3})^2}\label{lambda_min}.
\end{gather}
The non-negativity of $\lambda_{\min}$ in Eq.~\eqref{lambda_min} restricts $\lambda_{2}$ and $\lambda_{3}$ outside an ellipse,
\begin{equation}
    \left(\lambda_{2}-\frac{1}{2}\right)^2+\alpha^2\left(\lambda_{3}-\frac{1}{2}\right)^2\geq\frac{S^2}{16},
\end{equation}
and the fact that $\lambda_{4}$ in Eq.~\eqref{lambda_min} is the smallest among all $\lambda_i$ restricts $\lambda_{2}$ and $\lambda_{3}$ inside an ellipse,
\begin{equation}
    \lambda_{2}^2+(4\alpha^2+1)\lambda_{3}+2\lambda_{2}\lambda_{3}-\lambda_{2}-(2\alpha^2+1)\lambda_{3}\leq\frac{S^2}{16}-\frac{\alpha^2+1}{4}.
\end{equation}
With the above derivations, the optimization in Eq.~\eqref{concurrence_optm} can be rewritten with independent variables $\lambda_{2}$ and $\lambda_{3}$ as

\begin{equation}
    \begin{split}
    C_{\rm est} & =\underset{\lambda_{2},\lambda_{3}}{\text{min}} 2\lambda_{1}-1, \\
    \text{s.t.}\quad
    \lambda_{1} &=\frac{1}{2}-\frac{1}{\alpha^2+1}(\alpha^2\lambda_{2}+\lambda_{3})+\frac{1}{\alpha^2+1}\sqrt{\frac{S^2(\alpha^2+1)}{16}-\alpha^2(\lambda_{2}-\lambda_{3})^2},\\
    0 & \leq(\lambda_{2}-\frac{1}{2})^2+\alpha^2(\lambda_{3}-\frac{1}{2})^2-\frac{S^2}{16},\\
    0 & \geq \lambda_{2}^2+(4\alpha^2+1)\lambda_{3}+2\lambda_{2}\lambda_{3}-\lambda_{2}-(2\alpha^2+1)\lambda_{3}-(\frac{S^2}{16}-\frac{\alpha^2+1}{4}),\\
    \lambda_{1}&\geq\lambda_{2}\geq\lambda_{3},\\
    0&\leq \lambda_{2},\lambda_{3},\lambda_{2}+\lambda_{3}\leq 1.
    \end{split}
    \label{concurrence_optm_solvable}
\end{equation}
The optimization in Eq.~\eqref{concurrence_optm_solvable} can be solved analytically, with the global optimal value taken at
\begin{equation}
    \begin{gathered}
        \lambda_1=\frac{1}{2}+\frac{1}{2}\sqrt{\frac{S^2}{4}-\alpha^2},\\
        \lambda_2=\frac{1}{2}-\frac{1}{2}\sqrt{\frac{S^2}{4}-\alpha^2},\\
        \lambda_3=\lambda_4=0.
        \label{optimal_state}
    \end{gathered}
\end{equation}
Therefore the estimated concurrence is lower-bounded,
\begin{equation}
    C(\rho_{AB})\geq C_{\rm est}(S)= \sqrt{\frac{S^2}{4}-\alpha^2}.
    \label{concurrence_est_appendix}
\end{equation}
The lower bound of Eq.~\eqref{concurrence_est_appendix} is saturated when the $\alpha$-CHSH Bell value $S$ is obtained by measuring the Bell-diagonal state $\rho_{\lambda}$ with the parameters in Eq.~\eqref{optimal_state} under its optimal measurements. The optimal measurements are in Eq.~\eqref{opt_meas1} and Eq.~\eqref{opt_meas2} with $\theta=\arctan(\frac{1}{\alpha}\sqrt{\frac{S^2}{4}-\alpha^2})$.
\end{proof}

\section{Entanglement of formation estimation result in Ref.~\cite{arnon2017noise}}
\label{sc:EOF_arnon}
In Ref.~\cite{arnon2017noise}, the authors use the rigidity property of nonlocal games to estimate the EOF in the underlying system. Here, we briefly review their results. The Bell test can be formulated as a nonlocal game, $G$. Under this description, the nonlocal parties win the nonlocal game if their measurement outputs satisfy a set of relations with respect to their input settings that correspond to the Bell test~\cite{elkouss2016nearly}. For instance, in the nonlocal game related to the CHSH Bell test, Alice and Bob win the game if $ab=(-1)^{xy}$. We denote the nonlocal parties' maximum winning probability with classical strategies as $\mathrm{cval}(G)$ and their maximum winning probability with quantum strategies as $\mathrm{qval}(G)$. In the CHSH game, $\mathrm{qval}(G)=(2+\sqrt{2})/4$, corresponding to the Tsirelson bound, $S=2\sqrt{2}$, and $\mathrm{cval}(G)=3/4$, corresponding to the maximum Bell value subjected to local hidden variables, $S=2$. We denote the gap between classical and quantum strategies as $\Delta=\textrm{qval}(G)-\textrm{cval}(G)$. Given the nonlocal game $G$ with a non-zero gap $\Delta$, suppose the nonlocal parties play the game for $n$ independent instances in parallel. For a given value, $\nu\in[0,\mathrm{qval}(G)]$, we say the nonlocal parties win a threshold game, $G_{\mathrm{qval}(G)-\nu}^n$, if they win a fraction of at least $[\mathrm{qval}(G)-\nu]$ instances in total. Suppose there exists a quantum strategy for the nonlocal parties such that they win $G_{\mathrm{qval}(G)-\nu}^n$ with probability $\kappa$. Then, the EOF in the underlying system of the threshold game can be lower-bounded,
\begin{equation}
    E_{\mathrm{F}}(\rho)\geq c_2\kappa^2 n,
    \label{EOF_2017}
\end{equation}
where
\begin{equation}
    c_2=\frac{(\Delta-\nu)^5}{10\cdot 180^2\log |\mathcal{A}\times\mathcal{B}|},
\end{equation}
with $\mathcal{A}$ and $\mathcal{B}$ denoting the answer alphabets of Alice and Bob, respectively.

We compare this result with ours when using the CHSH Bell test. As we estimate the EOF in the Shannon limit with the i.i.d. assumption, for a fair comparison, we take $n\rightarrow\infty$ and evaluate the average EOF per instance in Eq.~\eqref{EOF_2017}. When the Bell value is $S>2$, Alice can Bob can win the corresponding nonlocal game $G$ with probability $1/2+S/8$. Applying this strategy to each instance in the threshold game in an i.i.d. manner, Alice and Bob can win the threshold game, $G_{\mathrm{qval}(G)-\nu}^{n\rightarrow\infty}$ with $\nu=\sqrt{2}/4-S/8$, with probability $\kappa=1$. Taking these values into Eq.~\eqref{EOF_2017}, the EOF of the system per instance is lower-bounded by
%In comparison with our estimation result, we take $G$ as the CHSH game. Regarding Eq.~\eqref{EOF_2017}, we take $\kappa=1$ and $n\rightarrow\infty$ for the asymptotic EOF estimation. A CHSH Bell value, $S>2$, indicates there exists a quantum strategy of winning $G_{\textrm{qval}(G)-\nu}^{n\rightarrow\infty}$ for $\nu=\sqrt{2}/4-S/8$ and $\textrm{qval}(G)=(2+\sqrt{2})/4$. In this case, the EOF of the underlying state in the strategy is lower bounded,
\begin{equation}
\begin{split}
    E_{\mathrm{F}}(\rho_{AB})&\geq \frac{[(\frac{2+\sqrt{2}}{4}-\frac{3}{4})-(\frac{\sqrt{2}}{4}-\frac{S}{8})]^5}{10\cdot 180^2\cdot\log (2\times 2)}\\
    %&=\frac{(S/8-1/4)^5}{10\cdot 180^2\cdot2}\\
    &=\frac{(S-2)^5}{10\cdot 180^2\cdot2^{16}}.
\end{split}
\end{equation}

\begin{comment}
\textcolor{red}{
\subsection{Proof of lower bound of negativity}\label{appendix:negativity_est}
In this subsection, we prove the analytical negativity result when restricting the underlying state is a pair of qubits. In the following proof, we show the connection between explicit forms of negativity and concurrence of the Bell diagonal state. Thus the negativity estimation problem can be simply deduced as an equivalent problem of the concurrence estimation problem.
\begin{proof}
    We solve the simplified optimization problem in Eq.~\eqref{equivalent_optm} when restricting the states are Bell diagonal states in Eq.~\eqref{Bell_diagonal}. To simplify the objective function for $\rho_{\lambda}$  in Eq.~\eqref{Bell_diagonal}, under computational basis, $\rho_\lambda$ is of the form
    \begin{equation}
    \rho_{\lambda}=\frac{1}{2} 
    \begin{bmatrix} 
    \lambda_{1} + \lambda_{2} &0 &0 &\lambda_{1} - \lambda_{2} \\
    0 &\lambda_{3} + \lambda_{4} &\lambda_{3} - \lambda_{4} &0 \\
    0 &\lambda_{3} - \lambda_{4} &\lambda_{3} + \lambda_{4} &0 \\ 
    \lambda_{1} - \lambda_{2} &0 &0 &\lambda_{1} + \lambda_{2} 
    \end{bmatrix}.
\end{equation}
By partial tracing on the subspace $A$, the matrix turns out to be
    \begin{equation}
    \rho_{\lambda}^{T_A}=\frac{1}{2} 
    \begin{bmatrix} 
    \lambda_{1} + \lambda_{2} &0 &0 &\lambda_{3} - \lambda_{4} \\
    0 &\lambda_{3} + \lambda_{4} &\lambda_{1} - \lambda_{2} &0 \\
    0 &\lambda_{1} - \lambda_{2} &\lambda_{3} + \lambda_{4} &0 \\ 
    \lambda_{3} - \lambda_{4} &0 &0 &\lambda_{1} + \lambda_{2} 
    \end{bmatrix}.
\end{equation}
Since $\rho_{\lambda}$ is assumed as $\lambda_1\geq\lambda_2\geq\lambda_3\geq\lambda_4$ in Eq.~\eqref{Bell_diagonal}, the single one negativity eigenvalue of $\rho_\lambda^{T_A}$ is $(-\lambda_1+\lambda_2+\lambda_3+\lambda_4)/2$ when $\lambda_1>1/2$. Thus the objective negativity of $\rho_\lambda$ is 
\begin{equation}
    \mathcal{\rho_\lambda}=\frac{1}{2}(\lambda_1-\lambda_2-\lambda_3-\lambda_4)=\frac{1}{2}(2\lambda_1-1),\lambda_1>\frac{1}{2}.
\end{equation}
The negativity of a qubit pair is lower-bounded by the optimization problem,
\begin{equation}
    \begin{split}
    \mathcal{N}_{\rm est} & =\underset{\lambda_i,i=1,2,3,4}{\text{min}} (2\lambda_1-1)/2, \\
    \text{s.t.}\quad
    S &=2\sqrt{\alpha^2(\lambda_1+\lambda_2-\lambda_3-\lambda_4)^2+(\lambda_1-\lambda_2+\lambda_3-\lambda_4)^2},\\
    \lambda_1 & \geq\lambda_2\geq\lambda_3\geq\lambda_4,\lambda_1\geq 1/2,\\
    1 & =\sum_{i=1}^4\lambda_i, \lambda_i\geq 0 \; ,i=1,2,3,4.
    \end{split}
    \label{negativity_optm}
\end{equation}
Clearly, Eq.~\eqref{negativity_optm} coincides with Eq.~\eqref{concurrence_optm} as the objective function of the former is half of the latter when $\lambda_1\geq 1/2$. Therefore the estimation result of negativity coincides with that of concurrence as Eq.~\eqref{negativity_est}.
\end{proof}
}
\end{comment}


\section{Realistic settings in experiment}
\label{appendix:numerical}
In this section, we analyze the realistic settings and analytically derive the condition where a better concurrence estimation can be derived with a tilted CHSH Bell expression. That is, by using the family of $\alpha$-CHSH expressions, a Bell expression with parameter $\alpha>1$ gives a better estimation result than $\alpha=1$. Considering the estimation function, $C_{\rm est}(S)$, as a function parameterized by $\alpha$, then our target is to determine the condition for the following inequalities,
\begin{equation}
\begin{gathered}
    \frac{\partial C_{\rm est}(S)}{\partial \alpha}|_{\alpha=1}>0,\\
    C_{\rm est}(S)|_{\alpha=1}>0.
\end{gathered}
\label{increasing_estimation_condition}
\end{equation}
The conclusions of Theorem~\ref{thm:DI_condition_pure} and Theorem~\ref{thm:DI_condition_werner} can be directly solved from  Eq.~\eqref{increasing_estimation_condition} by substituting the corresponding estimation equation and the $\alpha$-CHSH Bell value.
\begin{comment}
We consider the non-maximally entangled state in Eq.~\eqref{pure} as an example. Under the measurements parameterized as in Eq.~\eqref{meas_setting}, the nonlocality depicted by $\alpha$-CHSH Bell value is
\begin{equation}
\begin{split}
    S&=\Tr[\rho_{AB}\hat{S}_{\alpha}]\\
    &=(\cos\theta_2+\cos\theta_3)\alpha+\sin 2\delta\sin\theta_1(\sin\theta_2-\sin\theta_3)+\cos\theta_1(\cos\theta_2-\cos\theta_3).
\end{split}
\end{equation}
When estimating concurrence from the nonlocal statistics coming out of a completely DI experiment, we apply the convex closure estimation $C_{\rm est}(S)$ from Eq.~\eqref{EOF_est_DI}. The condition
\begin{equation}
\begin{gathered}
    \frac{\partial C_{\rm est}(S)}{\partial \alpha}|_{\alpha=1}>0;\\
    C_{\rm est}(S)|_{\alpha=1}>0
\end{gathered}
\label{increasing_estimation_condition}
\end{equation}
demonstrates an increasing value of concurrence estimation when turning up $\alpha$ from $\alpha=1$. Direct computations lead to Eq.~\eqref{eq:DI_condition_pure}.
A similar derivation can be applied to Werner states \eqref{werner}, where the condition Eq.~\eqref{increasing_estimation_condition} eventually induces Eq.~\eqref{eq:DI_relation_werner}.
\end{comment}

For a better understanding of Theorem~\ref{thm:DI_condition_pure}, we take $\theta_1=\pi/2$ in Eq.~\eqref{eq:DI_condition_pure}. With straightforward derivations, we find that when the measurement parameters, $\theta_2$ and $\theta_3$, satisfy
\begin{equation}
    (\sqrt{2}+1)(\cos\theta_2+\cos\theta_3)+(\sin\theta_2-\sin\theta_3)>2(\sqrt{2}+1),
    \label{DI_condition_pure_werner}
\end{equation}
there exists a value of $\delta$ such that $\theta_1=\pi/2$, and $\theta_2,\theta_3$ and $\delta$ satisfy the inequality in  Eq.~\eqref{eq:DI_condition_pure}. In other words, when measurement parameters are set in Eq.~\eqref{meas_setting} with $\theta_1=\pi/2$ and $\theta_2,\theta_3$ follow Eq.~\eqref{DI_condition_pure_werner}, there exists a proper state, $\ket{\phi_{AB}(\delta)}$, such that concurrence estimation result $C_{\rm est}(S)$ of the $\ket{\phi_{AB}(\delta)}$ for some $\alpha>1$ is larger than that with $\alpha=1$. The conclusion from Eq.~\eqref{DI_condition_pure_werner} also applies to Werner states.


\section{Semi-device-independent optimal entanglement estimation}
In some scenarios, one may trust the functioning of the source, such as knowing the input states to be pairs of qubits, which can be seen as a semi-DI scenario. With the additional information, one may obtain a better entanglement estimation result. 
%it admits a wider choice of state and measurement parameters that yield a better estimation when $\alpha>1$ compared with $\alpha=1$. It also permits a more accurate concurrence estimation than the device-independent (DI) estimation in Section~\ref{sc:numerical}, under the CHSH-type Bell values with the same $\alpha$. 
In this section, we compare the performance of this semi-DI scenario with the fully DI scenario under a realistic setting. Suppose the underlying state is a non-maximally entangled state $\ket{\phi_{AB}(\delta)}$ in Eq.~\eqref{pure} with $\delta=0.6$, and the measurements are given by Eq.~\eqref{meas_setting} with $\theta_1=\pi/2$ and $\theta_2=-\theta_3=\pi/2-1.2$. Under this setting, the fully DI and semi-DI EOF estimation results are illustrated in Fig.~\ref{fig:DI_semi-DI_pure_concurrence}. The semi-DI estimation, $E_{\rm F,est,semi-DI}(S)$, is strictly larger than the fully DI estimation, $E_{\rm F,est,DI}(S)$, for any value of $\alpha>1$. Besides, when $\alpha$ takes the value of $\alpha^*_{E}\doteq 2.3973$, $E_{\rm F,est,semi-DI}|_{\alpha^*_E}\doteq 0.9031$ rigorously equals to the real system EOF when $\delta=0.6$.

\begin{figure}[hbt!]
    \centering    \includegraphics[scale=0.57]{DI_semi-DI_pure_EOF.eps}
    \caption{Illustration of the comparison between DI and semi-DI EOF estimation results. The experimental setting is given by $\delta=0.6,\theta_1=\pi/2,\theta_2=-\theta_3=\pi/2-1.2$. For the estimation results varying in $\alpha>1$, we plot the estimated values from the settings of semi-DI and DI with the blue solid line and the red dashed line, respectively. The semi-DI EOF estimation with confirmed knowledge of input dimensions is strictly larger than the DI EOF estimation.
    }
    \label{fig:DI_semi-DI_pure_concurrence}
\end{figure}

In addition, we also study the condition where a better entanglement estimation result using general $\alpha$-CHSH expressions is obtained under some $\alpha>1$. We present the following theorems for the state families of Werner states and non-maximally entangled states, using concurrence as the entanglement measure.

\begin{theorem}
\label{thm:semi-DI_condition_pure}
In a Bell test experiment where the input states are pairs of qubits, suppose the underlying state of the system takes the form of Eq.~\eqref{pure} and the observables take the form of Eq.~\eqref{meas_setting}. When $\theta_1,\theta_2,\theta_3$ and $\delta$ satisfy
\begin{equation}
    (\cos\theta_2+\cos\theta_3)[\sin2\delta\sin\theta_1(\sin\theta_2-\sin\theta_3)+(1+\cos\theta_1)\cos\theta_2+(1-\cos\theta_1)\cos\theta_3]> 4,
    \label{eq:semi-DI_condition_pure}
\end{equation}
there exists $\alpha>1$, where a better estimation of $C_{\rm est}(S)$ can be obtained by using the $\alpha$-CHSH inequality parameterized by this value than by using the original CHSH inequality (corresponding to $\alpha=1$).
\end{theorem}

The proof of Theorem~\ref{thm:semi-DI_condition_pure} is similar to the proof of Theorem~\ref{thm:DI_condition_pure}. Here, we alternatively apply the concurrence estimation result for pairs of qubits input, $C_{\rm est}(S)$ in Eq.~\eqref{concurrence_est}, to the condition in Eq.~\eqref{increasing_estimation_condition}. To better understand the theorem, we present an example with $\theta_1=\pi/2$ in Eq.~\eqref{eq:semi-DI_condition_pure}. When
\begin{equation}
\begin{split}
  \theta_2+\theta_3&=\arccos(\frac{4}{1+\sqrt{2}k}-1),\\
  -\frac{\pi}{4}-\arccos k&<\theta_3-\theta_2<-\frac{\pi}{4}+\arccos k,
\end{split}
\label{semi-DI_condition_pure_werner_k}
\end{equation}
where $k\in[\sqrt{2}/2,1)$, there exists $\delta$ such that $\theta_1=\pi/2,\theta_2,\theta_3$ and the $\delta$ satisfy Eq.~\eqref{eq:semi-DI_condition_pure}. In other words, when measurement parameters are set in Eq.~\eqref{meas_setting} with $\theta_1=\pi/2$ and $\theta_2,\theta_3$ following Eq.~\eqref{semi-DI_condition_pure_werner_k}, there exist non-maximally entangled states where a better semi-DI concurrence estimation result is obtained for some $\alpha>1$ in comparison with $\alpha=1$.

In Fig.~\ref{fig:DI_semi-DI_pure_concurrence}, we observe that under a well-chosen value of $\alpha$, the semi-DI concurrence estimation coincides with the real value. In many semi-DI CHSH Bell tests, the existence of $\alpha$ that yields an accurate estimation of state concurrence is ubiquitous. Theorem~\ref{thm:concurrence_est} indicates that under the assumption of qubit inputs, the lower bound of concurrence in Eq.~\eqref{concurrence_est} can be saturated at any non-maximally entangled state $\ket{\phi_{AB}(\delta)}$, once the Bell value is obtained by the optimal measurements of the $\ket{\phi_{AB}(\delta)}$. In fact, earlier research indicates that the observables,
 \begin{equation}
  \begin{split}
      \hat{A}_0&=\pm\sigma_z,\\
    \hat{A}_1&=\sigma_x, \\
    \hat{B}_0&=\pm\cos\theta\sigma_z+\sin\theta\sigma_x,\\
    \hat{B}_1&=\pm\cos\theta\sigma_z-\sin\theta\sigma_x,
  \end{split}
  \label{optimal_meas_pure}
 \end{equation}
with $\tan\theta=\sin2\delta/\alpha$, are the optimal measurements of $\ket{\phi_{AB}}$ with any fixed $\alpha$~\cite{acin2012randomness} . The form of observables in Eq.~\eqref{optimal_meas_pure} coincides with our initialization in Eq.~\eqref{meas_setting} when  $\theta_1=\pi/2,\theta_2+\theta_3=0,0<\theta_2<\pi/4$. In this case, any non-maximally entangled state $\ket{\phi_{AB}(\delta)}$ with $\delta$ satisfying $\sin2\delta>\tan\theta_2$ reaches its optimal concurrence estimation when $\alpha$ takes the value of
\begin{equation}
\alpha^*_C=\frac{\sin2\delta}{\tan\theta_2},
\end{equation}
and the optimal semi-DI estimation value is
\begin{equation}
    C_{\rm est}|_{\alpha^*_C,\rm semi-DI}=C(\ket{\phi_{AB}}).
\end{equation}


\begin{theorem}
\label{thm:semi-DI_condition_werner}
In a Bell test experiment where the input states are pairs of qubits, suppose the underlying state of the system takes the form of Eq.~\eqref{werner} and the observables take the form of Eq.~\eqref{meas_setting}. When $\theta_1,\theta_2,\theta_3$ and $p$ satisfy
\begin{equation}
    (1-p)^2(\cos\theta_2+\cos\theta_3)\cdot[\sin\theta_1(\sin\theta_2-\sin\theta_3)+(1+\cos\theta_1)\cos\theta_2+(1-\cos\theta_1)\cos\theta_3]>4,
    \label{eq:semi-DI_condition_werner}
\end{equation}
there exists $\alpha>1$, where a better estimation of $C_{\rm est}(S)$ can be obtained by using the $\alpha$-CHSH inequality parameterized by this value than by using the original CHSH inequality (corresponding to $\alpha=1$).
\end{theorem}

The proof of Theorem~\ref{thm:semi-DI_condition_werner} is similar to the proof of Theorem~\ref{thm:DI_condition_werner}. Theorem~\ref{thm:semi-DI_condition_werner} is derived from the concurrence estimation result for pairs of qubits input, $C_{\rm est}(S)$ in Eq.~\eqref{concurrence_est}, and the condition in Eq.~\eqref{increasing_estimation_condition}. Here we take $\theta_1=\pi/2$ for convenience, when $\theta_2$ and $\theta_3$ satisfy Eq.~\eqref{semi-DI_condition_pure_werner_k} for $k\in[\sqrt{2}/2,1)$, there exists Werner states $\rho_{W}$ such that, the semi-DI concurrence estimation under the settings performs better when taking an $\alpha>1$ CHSH-type Bell expression compared with $\alpha=1$.

We further interpret Theorem~\ref{thm:semi-DI_condition_werner} via a special example. In a semi-DI CHSH Bell test, if measurements in Eq.~\eqref{meas_setting} are set with $\theta_1=\pi/2,\theta_2+\theta_3=0,0<\theta_2<\pi/4$, then any Werner state $\rho_{W}$ with parameter $p$,
\begin{equation}
    p<1-\frac{1}{\sqrt{\sqrt{2}\cos\theta_2\cos(\theta_2-\pi/4)}},
    \label{semi-DI_condition_werner_p}
\end{equation}
promises a better concurrence estimation when taking an $\alpha>1$ CHSH-type Bell expression compared with $\alpha=1$. The right hand side (RHS) of Eq.~\eqref{semi-DI_condition_werner_p} is no large than $1-\frac{1}{\cos(\pi/8)2^{1/4}}$, which implies only Werner state with $p\leq 1-\frac{1}{\cos(\pi/8)2^{1/4}}\doteq 0.0898$ is possible to fit in the condition in Theorem~\ref{thm:semi-DI_condition_werner}. It is worth mentioning that for any $0<\theta_2<\pi/4$, the RHS of Eq.~\eqref{semi-DI_condition_werner_p} is strictly larger than the RHS of  Eq.~\eqref{DI_condition_werner_p}. It allows a wide choice of Werner states to promise a better estimation when taking an $\alpha>1$ CHSH-type Bell expression compared with $\alpha=1$ in the semi-DI experiment. In a semi-DI system with the Werner state and measurements satisfying Eq.~\eqref{semi-DI_condition_werner_p}, the semi-DI concurrence estimation reaches the optimal when the CHSH Bell value is taken at $\alpha$ equals to
\begin{equation}
    \alpha^*_C=\frac{(1-p)^2\cos\theta_2\sin\theta_2}{1-(1-p)^2\cos^2\theta_2}
\end{equation}
and the optimal semi-DI estimation value is
\begin{equation}
    C_{\rm est}|_{\alpha^*_C,{\rm semi-DI}}=\frac{(1-p)\sin\theta_2}{\sqrt{1-(1-p)^2\cos^2\theta_2}}.
\end{equation}
%It is worth mentioning that the right hand side of Eq.\eqref{eq:relation_p_theta_werner} is no large than $0.0896$, which means only Werner state with $p\lesssim 0.0896$ is possible to fit in the condition in Theorem~\ref{thm:semi-DI_condition_werner}. This range of $p$ is wilder than the range in Besides, unlike on the set of the non-maximally entangled states, the semi-DI estimation $C_{\rm est}(S)$ is not tight on the set of Werner states, thus the estimation can never reach the real concurrence of a $\rho_{W}$ for any $\alpha$.


\begin{comment}
\begin{equation}
\begin{split}
    S&=\Tr[\rho_{AB}\hat{S}_{\alpha}]\\
    &=(1-p)[(\cos\theta_2+\cos\theta_3)\alpha+\sin\theta_1(\sin\theta_2-\sin\theta_3)+\cos\theta_1(\cos\theta_2-\cos\theta_3)]
\end{split}
\end{equation}
when $\rho_{AB}=\rho_{W}$.

Regardless of the input dimension, $C_{\rm est}(S)$ is the convex closure estimation from Eq.~\eqref{EOF_est_DI}; when input source is trusted within the region of pairs of qubits, $C_{\rm est}(S)$ is taken as a more accurate estimation from Eq.~\eqref{concurrence_est}. We take pure state $\ket{\phi(\delta)}$ as a detailed example. The DI estimated concurrence is expressed as
\begin{equation}
    C_{\rm est}(S)=\frac{1}{2\sqrt{1+\alpha^2}-2\alpha}[(\cos\theta_2+\cos\theta_3-2)\alpha+\sin 2\delta\sin\theta_1(\sin\theta_2-\sin\theta_3)+\cos\theta_1(\cos\theta_2-\cos\theta_3)]
\end{equation}
with
\begin{equation}
    \frac{\partial C_{\rm est}(S)}{\partial \alpha}=(\cos\theta_2+\cos\theta_3-2)(2\sqrt{1+\alpha^2}-2\alpha)-(\frac{2\alpha}{\sqrt{1+\alpha^2}}-2)[(\cos\theta_2+\cos\theta_3-2)\alpha+\sin 2\delta\sin\theta_1(\sin\theta_2-\sin\theta_3)+\cos\theta_1(\cos\theta_2-\cos\theta_3)]\frac{1}{(2\sqrt{1+\alpha^2}-2\alpha)^2},
\end{equation}
and direct simplification indicates that 
It is easy to verify that in this case $\frac{\partial C_{\rm est}(S)}{\partial \alpha}|_{\alpha=1}>0;C_{\rm est}(S)|_{\alpha=1}>0$ is equivalent to 
\begin{equation}
    (\cos\theta_2+\cos\theta_3-2)(1+\sqrt{2})+\sin 2\delta\sin\theta_1(\sin\theta_2-\sin\theta_3)+\cos\theta_1(\cos\theta_2-\cos\theta_3)>0.
\end{equation}
The condition $C_{\rm est}(S)|_{\alpha=1}>0$ indicates
\begin{equation}
    (\cos\theta_2+\cos\theta_3-2)+\sin 2\delta\sin\theta_1(\sin\theta_2-\sin\theta_3)+\cos\theta_1(\cos\theta_2-\cos\theta_3)>0
\end{equation}
which is contained in Eq.~XXX. This ends the proof that condition Eq.~\eqref{increasing_estimation_condition} eventually derives Eq.~\eqref{DI_pure_condition}.
\end{comment}

%%%%%%%%%%%%%%%%%%%%%%%%%%%%%%%%%%%%%%%%
% choose a style
%\bibliographystyle{ieeetr}
\bibliographystyle{unsrt.bst}
%\bibliographystyle{apsrev}
%%%%%%%%%%%%%%%%%%%%%%%%%%%%%%%%%%%%%%%%


%%%%%%%%%%%%%%%%%%%%%%%%%%%%%%%%%%%%%%%%
% choose a .bib file
\bibliography{reference.bib}

\end{document}