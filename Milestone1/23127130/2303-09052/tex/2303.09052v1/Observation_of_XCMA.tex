%
\documentclass[%
 reprint,
 superscriptaddress,
 amsmath,amssymb,
 aps,
]{revtex4-2}
    
\usepackage{graphicx}% Include figure files
\usepackage{dcolumn}% Align table columns on decimal point
\usepackage{bm}% bold math
\usepackage{siunitx}
\usepackage{float}

\usepackage{bbm}
\usepackage{graphicx}% Include figure files
\usepackage{dcolumn}% Align table columns on decimal point
\usepackage{subfigure}
\usepackage{amsmath,bm}
\usepackage{amsfonts}
\usepackage{appendix}
\usepackage{feynmf}
\usepackage{hyperref}
\usepackage{eepic}
\usepackage{lipsum}
\usepackage{enumerate}
\usepackage{amssymb}
\usepackage[english]{babel}
\usepackage{xcolor}
\usepackage[pagewise]{lineno}

\newcommand{\ve}[1]{\mbox{\boldmath$#1$}}

\newcommand{\ua}{\uparrow} 
\newcommand{\da}{\downarrow}
\newcommand{\ra}{\rightarrow}
\newcommand{\la}{\leftarrow}
\newcommand{\bs}{\boldsymbol}
\newcommand{\ep}{\epsilon}
\newcommand{\ld}{\lambda}
\newcommand{\SRO}{Sr$_2$RuO$_4$}
\newcommand{\bk}{\mathbf k}
\newcommand{\bR}{\mathbf R}
\newcommand{\bkp}{\mathbf k^\prime}
\newcommand{\bq}{\mathbf q}
\newcommand{\br}{\mathbf r}
\newcommand{\brp}{\mathbf r^\prime}
\newcommand{\bH}{\mathbf H}
\bibliographystyle{naturemag}
\makeatletter
\renewcommand*{\@fnsymbol}[1]{\ensuremath{\ifcase#1\or \dagger\or * \else\@ctrerr\fi}}
\makeatother
%\linenumbers
\begin{document}

\preprint{APS/123-QED}

\title{Superconducting Diode Effect and Large Magnetochiral Anisotropy\\ in T$_d$-MoTe$_2$ Thin Film}

\author{Wan-Shun Du}
\thanks{These authors contributed equally to this work.}
\affiliation{Shenzhen Institute for Quantum Science and Engineering, Southern University of Science and Technology, Shenzhen 518055, China}
\affiliation{Department of Physics, Southern University of Science and Technology, Shenzhen 518055, China}
\affiliation{Department of Physics, The Hong Kong University of Science and Technology, Clear Water Bay, Kowloon, Hong Kong, China}
\author{Weipeng Chen}
\thanks{These authors contributed equally to this work.}
\affiliation{Shenzhen Institute for Quantum Science and Engineering, Southern University of Science and Technology, Shenzhen 518055, China}
\affiliation{International Quantum Academy, Shenzhen 518048, China}
\affiliation{Guangdong Provincial Key Laboratory of Quantum Science and Engineering, Southern University of Science and Technology, Shenzhen 518055, China}
\author{Yangbo Zhou}
\thanks{These authors contributed equally to this work.}
\affiliation{Department of Materials Science, School of Physics and Materials Science, Nanchang University, Nanchang, Jiangxi, 330031, China}
\affiliation{Jiangxi Key Laboratory for Two-Dimensional Materials, Nanchang University, Nanchang, Jiangxi, 330031, China}
\author{Tengfei Zhou}
\thanks{These authors contributed equally to this work.}
\affiliation{Shenzhen Institute for Quantum Science and Engineering, Southern University of Science and Technology, Shenzhen 518055, China}
\author{Guangjian Liu}
\affiliation{Department of Materials Science, School of Physics and Materials Science, Nanchang University, Nanchang, Jiangxi, 330031, China}
\affiliation{Jiangxi Key Laboratory for Two-Dimensional Materials, Nanchang University, Nanchang, Jiangxi, 330031, China}
\author{Zongteng Zhang}
\affiliation{Department of Physics, Southern University of Science and Technology, Shenzhen 518055, China}
\author{Zichuan Miao}
\affiliation{Department of Materials Science $\&$ Metallurgy, University of Cambridge, 27 Charles Babbage Road, Cambridge, CB3 0FS, United Kingdom}
\author{Hao Jia}
\affiliation{Shenzhen Institute for Quantum Science and Engineering, Southern University of Science and Technology, Shenzhen 518055, China}
\affiliation{International Quantum Academy, Shenzhen 518048, China}
\affiliation{Guangdong Provincial Key Laboratory of Quantum Science and Engineering, Southern University of Science and Technology, Shenzhen 518055, China}
\author{Song Liu}
\affiliation{Shenzhen Institute for Quantum Science and Engineering, Southern University of Science and Technology, Shenzhen 518055, China}
\affiliation{International Quantum Academy, Shenzhen 518048, China}
\affiliation{Guangdong Provincial Key Laboratory of Quantum Science and Engineering, Southern University of Science and Technology, Shenzhen 518055, China}
\author{Yue Zhao}
\affiliation{Shenzhen Institute for Quantum Science and Engineering, Southern University of Science and Technology, Shenzhen 518055, China}
\affiliation{Department of Physics, Southern University of Science and Technology, Shenzhen 518055, China}
\author{Zhensheng Zhang}
\affiliation{Shenzhen Institute for Quantum Science and Engineering, Southern University of Science and Technology, Shenzhen 518055, China}
\affiliation{International Quantum Academy, Shenzhen 518048, China}
\author{Tingyong Chen}
\affiliation{Shenzhen Institute for Quantum Science and Engineering, Southern University of Science and Technology, Shenzhen 518055, China}
\affiliation{International Quantum Academy, Shenzhen 518048, China}
\author{Ning Wang}
\affiliation{Department of Physics, The Hong Kong University of Science and Technology, Clear Water Bay, Kowloon, Hong Kong, China}
\author{Wen Huang}
\thanks{Correspondence and requests for materials should be addressed to W.H. (email: huangw3@sustech.edu.cn), Z.-B.T. (tanzb@sustech.edu.cn), J.-J.C. (email: chenjj3@sustech.edu.cn), and D.-P.Y. (email: yudp@sustech.edu.cn).}
\affiliation{Shenzhen Institute for Quantum Science and Engineering, Southern University of Science and Technology, Shenzhen 518055, China}
\affiliation{International Quantum Academy, Shenzhen 518048, China}
\affiliation{Guangdong Provincial Key Laboratory of Quantum Science and Engineering, Southern University of Science and Technology, Shenzhen 518055, China}
\author{Zhen-Bing Tan}
\thanks{Correspondence and requests for materials should be addressed to W.H. (email: huangw3@sustech.edu.cn), Z.-B.T. (tanzb@sustech.edu.cn), J.-J.C. (email: chenjj3@sustech.edu.cn), and D.-P.Y. (email: yudp@sustech.edu.cn).}
\affiliation{Shenzhen Institute for Quantum Science and Engineering, Southern University of Science and Technology, Shenzhen 518055, China}
\affiliation{International Quantum Academy, Shenzhen 518048, China}
\author{Jing-Jing Chen}
\thanks{Correspondence and requests for materials should be addressed to W.H. (email: huangw3@sustech.edu.cn), Z.-B.T. (tanzb@sustech.edu.cn), J.-J.C. (email: chenjj3@sustech.edu.cn), and D.-P.Y. (email: yudp@sustech.edu.cn).}
\affiliation{Shenzhen Institute for Quantum Science and Engineering, Southern University of Science and Technology, Shenzhen 518055, China}
\affiliation{International Quantum Academy, Shenzhen 518048, China}
\affiliation{Guangdong Provincial Key Laboratory of Quantum Science and Engineering, Southern University of Science and Technology, Shenzhen 518055, China}
\author{Da-Peng Yu}
\thanks{Correspondence and requests for materials should be addressed to W.H. (email: huangw3@sustech.edu.cn), Z.-B.T. (tanzb@sustech.edu.cn), J.-J.C. (email: chenjj3@sustech.edu.cn), and D.-P.Y. (email: yudp@sustech.edu.cn).}
\affiliation{Shenzhen Institute for Quantum Science and Engineering, Southern University of Science and Technology, Shenzhen 518055, China}
\affiliation{International Quantum Academy, Shenzhen 518048, China}
\affiliation{Guangdong Provincial Key Laboratory of Quantum Science and Engineering, Southern University of Science and Technology, Shenzhen 518055, China}


\date{December 16, 2022}

\begin{abstract}
In the absence of time-reversal invariance, metals without inversion symmetry may exhibit nonreciprocal charge transport --- a magnetochiral anisotropy that manifests as unequal electrical resistance for opposite current flow directions. If superconductivity also sets in, the charge transmission may become dissipationless in one direction while remaining dissipative in the opposite, thereby realizing a superconducting diode. Through both DC and AC magnetoresistance measurements, we study the nonreciprocal effects in thin films of the superconducting noncentrosymmetric type-II Weyl semimetal T$_d$-MoTe\textsubscript{2}. We report nonreciprocal superconducting critical currents with a diode efficiency close to 20\%~, and an extreme magnetochiral anisotropy coefficient up to $\SI{1e9}{\per\tesla\per\ampere}$, under weak out-of-plane magnetic field in the millitesla range.  Intriguingly, unlike the finding in Rashba systems, the strongest in-plane nonreciprocal effect does not occur when the field is perpendicular to the current flow direction. We develop a phenomenological theory to demonstrate that this peculiar behavior can be attributed to the asymmetric structure of spin-orbit coupling in T$_d$-MoTe\textsubscript{2}. Our study highlights how the form of spin-orbit coupling, as dictated by the underlying crystallographic symmetry, critically impacts the nonreciprocal transport. Our work demonstrates that T$_d$-MoTe\textsubscript{2} offers an accessible material platform for the practical application of superconducting diodes under a relatively weak magnetic field.
\end{abstract}

\maketitle
The nonreciprocal effect where charge transport exhibits directional dependence, such as in p-n junction, is fundamental to modern electronics ~\cite{rikken1997observation,rikken2001electrical,rikken2005magnetoelectric,avci2015unidirectional,olejnik2015electrical,yasuda2016large,ideue2017bulk,lv2018unidirectional,he2018bilinear,he2018observation}. Recently, the nonreciprocal charge responses in superconducting systems have stimulated a great deal of interest ~\cite{wakatsuki2017nonreciprocal, wakatsuki2018nonreciprocal, yasuda2019nonreciprocal, itahashi2020nonreciprocal,zhang2020nonreciprocal,lyu2021superconducting,lin2022zero}. In particular, superconducting diodes that manifest distinct critical currents in different directions have been demonstrated. Such a diode supports dissipationless charge transmission in one direction while exhibiting finite resistance in the opposite~\cite{ando2020observation, baumgartner2022supercurrent,wu2022field, pal2022josephson,hou2022ubiquitous}, paving the way for future applications in non-dissipative electronics.

The superconducting nonreciprocal effect is a descendant of a larger set of phenomena called magnetochiral anisotropy (MCA)~\cite{rikken1997observation, rikken2001electrical}. It may occur in bulk crystals when spatial inversion and time reversal symmetries are both broken~\cite{tokura2018nonreciprocal}. A case in point is metals with noncentrosymmetric structure, where time reversal invariance can be broken by the application of an external magnetic field. The nonreciprocity is described by a sample resistance that depends on the direction of the current flow \cite{rikken2005magnetoelectric},
\begin{equation}\label{eq:1}
	R = R_0(1+\gamma \mu_0H I) \,,
\end{equation}
where $R_0$ is the resistance at zero field, $I$ is the current, $\mu_0H$ represents the magnetic field, and $\gamma$ is a coefficient characterizing the strength of the MCA. While this coefficient is usually small in metals, it is argued to be markedly amplified by superconducting fluctuations slightly above the superconducting transition, as has been demonstrated in noncentrosymmetric superconductors MoS$_2$, SrTiO$_3$, NbSe$_2$, as well as in the Bi$_2$Te$_3$/FeTe heterostructure~\cite{wakatsuki2017nonreciprocal, itahashi2020nonreciprocal, yasuda2019nonreciprocal, zhang2020nonreciprocal}. And a direct observation of the diode effect with $I_{c,+} \neq |I_{c,-}|$ has been reported in a few cases~\cite{ando2020observation,baumgartner2022supercurrent,wu2022field,pal2022josephson,hou2022ubiquitous}. 

In this work, we perform transport experiments on thin films of the noncentrosymmetric superconductor T$_d$-MoTe\textsubscript{2}. This compound is a candidate type-II Weyl semimetal with a superconducting transition temperature $T_c \approx$ 0.1 K \cite{sun2015prediction,qi2016superconductivity,zhang2016raman}. Our DC measurements provide clear evidence for critical current nonreciprocity in the presence of weak out-of-plane magnetic field, with a diode efficiency $2(I_{c,+} - |I_{c,-}|)/(I_{c,+} + |I_{c,-}|)$ around 20\%. The AC measurements observe extremely strong MCA in the second harmonic resistance, with $\gamma$ reaching up to $1 \times 10^9$ T$^{-1}$A$^{-1}$ --- far exceeding previous reports in other systems. The large MCA appears under both in-plane and out-of-plane magnetic fields, and can be ascribed to the relatively smaller superconducting gap in this compound. Surprisingly, unlike in some Rashba superconductors~\cite{rikken2005magnetoelectric,wakatsuki2018nonreciprocal,yasuda2019nonreciprocal,hou2022ubiquitous}, the strongest MCA under in-plane magnetic field orientation does not occur when the field is perpendicular to the current flow direction. On the basis of a phenomenological theory, we show that this intriguing behavior has its origin in the peculiar form of the spin-orbit coupling (SOC) in T$_d$-MoTe\textsubscript{2}. 

\section*{Results}
\textbf{Superconducting properties.} Figure \ref{fig:1}a shows the scanning electron microscope (SEM) of the measured T$_d$-MoTe\textsubscript{2} device with an area of \SI{13.17}{\square\micro\meter} $\text{MoTe}_2$. Thin microflakes were obtained by mechanical exfoliation from bulk single crystals and then transferred onto a Si substrate capped with a 285 nm thick $\text{SiO}_2$ layer. The sample was pre-etched by argon (Ar) plasma for 400 s and was overall thinned from about 62 nm to 43 nm (Supplemental Material Fig. S1). Electrical contacts were then defined using electron beam lithography. To realize ohmic contact between the $\text{MoTe}_2$ sample and electrodes (5/150 nm Ti/Au), an \textit{in situ} Ar plasma etch for \SI{80}{\second} was employed prior to metal deposition. Standard four-terminal measurements were performed in a dilution refrigerator with a base temperature of 10 mK.
\begin{figure}
\includegraphics{FIG_01}% Here is how to import EPS art
\caption{\textbf{T$_d$-$\text{MoTe}_2$ device and superconductivity.} (a) False-color SEM image of the $\text{MoTe}_2$ device and the measurement setup with bias-current and monitored voltage. \textbf{x}, \textbf{y}, \textbf{z} and $\mathbf{\varphi}$ indicate the coordinate system defined in the measurement. (b) Temperature-dependent resistance from \SIrange{600}{10}{\milli\kelvin}. The red dashed line denotes the resistance reaching half the normal-state value. (c) Voltage-current ($V -I$) characteristic and the corresponding differential resistance $\mathrm{d}V/\mathrm{d}I$ at \SI{10}{\milli\kelvin}. (d) Out-of-plane magnetoresistance curve at \SI{10}{\milli\kelvin}.  \label{fig:1}}
\end{figure}


Figure \ref{fig:1}b shows the temperature dependence of the resistance $R$, measured under zero-field. The experiment was performed using standard lock-in techniques with a small AC current excitation $I_{ac} = \SI{40}{\nano\ampere}$ at a frequency of $\SI{73}{\hertz}$. As the sample is cooled down, the resistance suddenly drops at $\SI{0.23}{\kelvin}$, indicating the onset of superconductivity; zero-resistance is observed when the temperature is lower than $T_{c,0} = \SI{0.09}{\kelvin}$. Empirically, the superconducting phase transition temperature $T_{c,r}$ is determined to be \SI{0.173}{\kelvin}, selected by a reduced resistance $r = R / R_{0.6\text{K}} = 0.5$.  Note that the resistance undergoes multiple steep falls before dropping to zero, which could be attributed to the inhomogeneity in the sample due to argon etching. Figure \ref{fig:1}c shows the voltage-current ($V$ - $I$) characteristics and the corresponding differential resistance ($\mathrm{d}V/\mathrm{d}I$) at \SI{10}{\milli\kelvin}. Zero-voltage is observed when the injected current satisfies $|I|<\SI{0.9}{\micro\ampere}$, signifying a superconducting state. At larger currents ($|I|>\SI{2.5}{\micro\ampere}$), the sample returns to normal state and the $V$ - $I$ curve follows a straight line with a slope of \SI{23.8}{\ohm}. In addition, we can see multiple peaks in the $\mathrm{d}V/\mathrm{d}I$ - $I$ curve in association with the previously mentioned sample inhomogeneity. We further characterize the suppression of the superconductivity by applying an out-of-plane magnetic field $\mu_0H_z$. As can be seen in Figure \ref{fig:1}d, the sample resistance gradually develops as the field increases above \SI{3}{\milli\tesla} and does not saturate until around \SI{0.1}{\tesla}.  At the base temperature, one finds an out-of-plane upper critical field $\mu_0H_{c2,z} =$ \SI{30}{\milli\tesla}, at which half of the normal state resistance is recovered. Such a significantly broadened superconducting transition implies a relatively high defect concentration induced by argon plasma etching, although the plasma can etch the 2D materials layer-by-layer in some condition~\cite{liu2013plasma}. Furthermore, the 2D nature of the superconductivity is confirmed by checking the temperature dependences of the out-of-plane and in-plane upper critical fields, and the 2D Berezinskii–Kosterlitz–Thouless (BKT) transition~\cite{berezinskii1972destruction,kosterlitz1973ordering,halperin1979resistive} (Supplemental Material Fig. S2).
\begin{figure*}
\includegraphics{FIG_02}% Here is how to import EPS art
\caption{\textbf{Nonreciprocal superconducting transport under out-of-plane magnetic field.} (a) Differential resistance $\mathrm{d}V/\mathrm{d}I$ as a function of out-of-plane magnetic field $\mu_0 H_z$ and DC current $I$. The white contour indicates the edge of the superconducting regime where the reduced resistance $r = 0.2$. This measurement was performed by sweeping current from negative to positive values at different magnetic fields at $\SI{10}{\milli\kelvin}$. (b) Linecuts of $\mathrm{d}V/\mathrm{d}I$ at $\mu_0 H_z = $ \SIlist[list-units=single,list-final-separator={, }]{\pm 1; \pm 2; \pm 3}{\milli\tesla}, extracted from (a). The curves are slightly shifted upward for clarification. (c) Nonreciprocal critical current $I_c$, (d) difference of the critical current $\Delta I_c=I_{c,+} - |I_{c,-}|$ and (e) the supercurrent rectification efficiency $Q=\frac{\Delta I_c}{(I_{c,+} + |I_{c,-}|)/2}$, plotted as a function of magnetic field.}\label{fig:2}
\end{figure*}


\textbf{Nonreciprocal superconducting transport under out-of-plane magnetic field.} Figure \ref{fig:2}a shows the differential resistance colormap as a function of $I$ and out-of-plane magnetic field $\mu_0 H_z$. The blue region represents the superconducting state and the white contour depicts the critical current $I_c$ with reduced resistance $r = 0.2$. The $\mathrm{d}V/\mathrm{d}I$ map exhibits clear asymmetry with respect to both $I$ and $\mu_0H_z$. With the zero-field $\mathrm{d}V/\mathrm{d}I$ plotted as a reference, Figure \ref{fig:2}b shows the $\mathrm{d}V/\mathrm{d}I$ linecuts for $\mu_0 H_z$ =  \SIlist[list-units=single,list-final-separator={, }]{\pm 1; \pm 2; \pm 3}{\milli\tesla}, revealing a clear nonreciprocal magneto-transport. The nonreciprocity also manifests in the unequal superconducting critical currents in opposite directions, $I_{c,+}$ and $|I_{c,-}|$, as shown in Figure \ref{fig:2}c. This is known as the superconducting diode effect, in which an injected current in the range between $I_{c,+}$ and $| I_{c,-} |$ can only flow in one direction without dissipation. The relative magnitude between $I_{c,+}$ and $| I_{c,-} |$ reverses when the external field reverses direction.  Figure \ref{fig:2}d shows the nonreciprocal component of the critical current $\Delta I_c = I_{c,+} - |I_{c,-}|$ at different $\mu_0 H_z$. $\Delta I_c$ reaches its maximal value of $\SI{0.2}{\micro\ampere}$ at about $\SI{2}{\milli\tesla}$, and then decreases as the magnetic field increases further. The quality of the diode effect is usually described in terms of the supercurrent rectification efficiency \cite{he2022phenomenological}, namely, $Q = \frac{\Delta I_{c}}{(I_{c,+} + |I_{c,-}|)/2}$, which is presented in Figure \ref{fig:2}e. The rectification efficiency first increases with field, and then reaches a stable value of about $20\%$ for $\mu_0H_z>\SI{2}{\milli\tesla}$. Such a large value is much higher than that ($5\%$) observed in refs.~\cite{ando2020observation} and~\cite{wu2022field}, and is comparable to that in some devices reported recently ~\cite{bauriedl2022supercurrent,shin2021magnetic}.

%Now, it turns to discuss the MCA in the measurement realized by AC current and evaluate the MCA coefficient $\gamma$ with the more significant signal. 

\begin{figure*}
\includegraphics{Fig_03}% Here is how to import EPS art
\caption{\textbf{Out-of-plane magnetochiral anisotropy.} (a) First harmonic resistance $R_\omega$ and (b) second harmonic resistance $R_{2\omega}$ as a function of $\mu_0 H_z$ and AC current $I_{ac}$. {The white boundary in (a) indicates the reduced resistance $r = 0.2$.} {The green line in (b) indicates the maximal $R_{2\omega}$ extracted from the $R_{2\omega}-I_{ac}$ curves for each magnetic field.} (c) Line-cuts of (left axis) $R_{\omega}$ and (right axis) $R_{2\omega}$ at $\mu_0 H_z = $ \SI{\pm 1}{\milli\tesla} as well as the $R_{2\omega}$ under zero-field (black dots) as a reference. (d)   {$\eta$, the ratio of the nonlinear resistance $2R_{2\omega}$ to the linear resistance $R_{\omega}$,}  as a function of $I_{ac}$ at $\mu_0 H_z = $ \SIlist[list-units=single, list-final-separator={, }]{\pm 1; \pm 2; \pm 3}{\milli\tesla}. (e) MCA coefficient $\gamma$ as a function of $\mu_0 H_z$. (f) Temperature dependence of $\gamma$ at $\mu_0 H_z = \SI{1}{\milli\tesla}$. inset: Temperature dependence of magnetoresistance {R} at $\mu_0 H_z = \SI{1}{\milli\tesla}$. The red dot indicates the mean-field transition temperature {$T_{c,0.5}$}.}\label{fig:3}
\end{figure*}

\textbf{Out-of-plane magnetochiral anisotropy.} We performed AC harmonic measurements to further investigate the properties of the nonreciprocal superconducting transport. The measurement on the harmonic resistances was performed by applying an large AC current $I_{ac}$ without DC current offset under various out-of-plane magnetic field. Figure \ref{fig:3}a and \ref{fig:3}b show the colormaps of the first and second harmonic resistance, $R_\omega$ and $R_{2\omega}$, as a function of $\mu_0 H_z$ and $I_{ac}$, respectively. Here, $R_\omega$ corresponds to linear resistance $R_0$ and $R_{2\omega}$ represents the half of the nonlinear resistance which is proportional to the current and the magnetic field ($R_{2\omega} = \frac{R_0}{2}\gamma\mu_0HI$)~\cite{itahashi2020nonreciprocal,ando2020observation}. It can be seen that $R_\omega$ is symmetric with respect to the magnetic field while $R_{2\omega}$ is antisymmetric. Figure \ref{fig:3}c shows AC current dependence of $R_{\omega}$ and $R_{2\omega}$ at $\mu_0 H_z = \pm\SI{1}{\milli\tesla}$ extracted from Figure \ref{fig:3}a and \ref{fig:3}b with $R_{2\omega}$ under zero-field plotted in black dots as a reference (See Supplemental Material Fig. S3 for more line-cuts). As one can see, $R_{2\omega}$ displays a strong peak around $I_{ac} \approx$ \SI{0.75}{\micro\ampere}, which is close to the magnitude of the DC critical current. The maximal $R_{2\omega}$ for each magnetic field is shown in green line in Figure \ref{fig:3}b, peaking around $\pm\SI{2}{\milli\tesla}$. 

Figure \ref{fig:3}d plots the ratio of the nonlinear resistance to the linear resistance, $\eta$, under different magnetic fields (more data in Supplemental Material Fig. S4). The magnitude of $\eta$ can reach the limiting value of $\pm 1$ for \SI{1}{\milli\tesla} $\leq |\mu_0 H_z| \leq$ \SI{2}{\milli\tesla}, indicating a strikingly large MCA in this system. The MCA coefficient $\gamma$ which is, given by $\gamma = \frac{1}{\mu_0H_zI_{ac}}\frac{2R_{2\omega}}{R_\omega}$, is plotted as a function of magnetic field in Figure \ref{fig:3}e. In practice, at a given external field, we evaluate $\gamma$ using the peak value of $R_{2\omega}$ and the corresponding $I_{ac}$ in the $R_{2\omega}-I_{ac}$ curve. $\gamma$ reaches a maximal value of \SI{1e9}{\per\tesla\per\ampere}, far exceeding previous observations in other two-dimensional noncentrosymmetric superconductors~\cite{wakatsuki2017nonreciprocal,yasuda2019nonreciprocal,itahashi2020nonreciprocal,ando2020observation}. According to a previous theoretical analysis \cite{hoshino2018nonreciprocal}, the MCA coefficient in the superconducting fluctuation regime is enhanced from its normal state value by an order of $(\frac{\epsilon_F}{k_B T})^3$.  The exceptionally strong MCA in T$_d$-MoTe$_2$ can then be ascribed to its relatively small $T_c$ compared to other superconducting systems where nonreciprocal transport has been reported. 

The temperature dependence of $\gamma$ at $\mu_0 H_z = \SI{1}{\milli\tesla}$ is shown in Figure \ref{fig:3}f. Here, $\gamma$ at each temperature is again computed using the peak absolute value of $R_{2\omega}$ and the related $I_{ac}$ (Supplemental Material Fig. S5). It decreases with increasing temperature, and dropping rapidly above \SI{120}{\milli\kelvin}. Note that the mean-field transition temperature $T_{c,0.5}$ at \SI{1}{\milli\tesla} is determined to be \SI{152}{\milli\kelvin} (Figure \ref{fig:3}f inset), below which the phase fluctuation develops because of the free vortices and antivortices in a 2D superconductor. The marked enhancement of the MCA below the the mean-field transition indicates that the vortex dynamics plays an important role \cite{carapella2009bistable,hoshino2018nonreciprocal}.


\begin{figure*}
\includegraphics{Fig_04}% Here is how to import EPS art
\caption{\textbf{In-plane magnetochiral anisotropy.} (a) First harmonic resistance $R_\omega$ and (b) second harmonic resistance $R_{2\omega}$ as a function of in-plane perpendicular magnetic field $\mu_0 H_y$ and AC current $I_{ac}$. The white boundary in (a) indicates the reduced resistance $r = 0.2$. The green line in (b) indicates the maximal $R_{2\omega}$. (c) Line-cuts of $R_{2\omega}$ at $\mu_0 H_y = $ \SIlist[list-units=single, list-final-separator = {, }]{\pm 0.1; \pm 0.15; \pm 0.2}{\tesla}. (d) MCA coefficient $\gamma$ as a function of $\mu_0 H_y$, which is calculated using the the maximal $R_{2\omega}$ in (b) with the corresponding AC current and magnetic field. (e) Resistance $R$ as a function of in-plane magnetic field $\mu_0 H_{xy}$ at different azimuthal angle $\varphi$. The black dots indicate the critical magnetic field $\mu_0 H_{c2,\varphi}$ where the reduced resistance $r = 0.5$. The solid green line and dashed line indicate the easy axis and hard axis of the sample. (f) $R_{2\omega}$ as a function of $I_{ac}$ at different $\varphi$ with $\mu_0 H_{xy} = \SI{0.15}{\tesla}$. The solid (dotted) green lines indicate the positive (negative) values. The red, black, and blue dashed lines represent the hard axis and the two opposite directions of the easy axis in (e). (g) Line-cuts of $R_{2\omega}$ are extracted from (f) along the dashed straight lines in their respective colors.}\label{fig:4}
\end{figure*}

\textbf{In-plane magnetochiral anisotropy.} The in-plane MCA effect can be investigated following a similar procedure. Figure \ref{fig:4}a and \ref{fig:4}b show the colormaps of the first and second harmonic resistances, $R_\omega$ and $R_{2\omega}$ as a function of the in-plane perpendicular magnetic field $\mu_0 H_y$ and $I_{ac}$, respectively. It is noticeable that $R_{2\omega}$ still appears in the superconducting transition regime. Figure \ref{fig:4}c shows AC current dependence of $R_{2\omega}$ under different $\mu_0 H_y$, extracted from Figure \ref{fig:4}b. The peaks of resistances shift towards zero-current as $|\mu_0 H_y|$ increases due to the suppression of the critical current. The MCA coefficient $\gamma$ is plotted in Figure~\ref{fig:4}d, and the corresponding $\eta$ is found to reach a maximal value of 0.5 (Supplemental Material Fig. S6). We therefore see that, in terms of the magnitude of $\gamma$ and $\eta$, the rectification effect is weaker under the in-plane perpendicular magnetic field than the out-of-plane magnetic field.

Next, we rotated the in-plane magnetic field to further characterize the MCA phenomenon. Figure \ref{fig:4}e presents the colormap of the magnetoresistance as a function of in-plane field strength and field angle. The in-plane angular dependence of $\eta$ and $\gamma$ are shown in Supplemental Material Fig. S7. A significant in-plane anisotropy is observed. The anisotropy also manifests in the upper critical field $\mu_0H_{c2,\varphi}$, indicated by black dots in the figure, which displays a two-fold symmetry with easy axis at $\varphi_a =$ \SI{145}{\degree} and hard axis at $\varphi_b =$\SI{55}{\degree}.  Similar observation has been reported in a T$_d$-MoTe\textsubscript{2} device by another group \cite{cui2019transport}. In comparison to the results in Ref.~\cite{cui2019transport}, we thus identify the easy axis as the crystallographic $a$-axis which is parallel to the in-plane mirror reflection plane. Figure \ref{fig:4}f shows dependence of $R_{2\omega}$ on the AC current and the field angle $\varphi$, at a constant field strength of \SI{0.15}{\tesla}. The green curve plots the maximal values of $|R_{2\omega}|$ at different field angles $\varphi$, where the solid (dashed) segments indicate positive (negative) value of $R_{2\omega}$. The overall lineshape appears to be symmetric about the easy axis defined above. Figure \ref{fig:4}g shows the line-cuts of $R_{2\omega}$ along the dash lines in figure \ref{fig:4}f in respective colors. It is interesting to note that, the extremal values of $R_{2\omega}$ appear to be obtained for magnetic fields parallel to the easy axis, instead of for fields perpendicular to the current
flow direction as predicted and reported for Rashba superconductors~\cite{wakatsuki2018nonreciprocal,yasuda2019nonreciprocal,hou2022ubiquitous}. One may argue that the in-plane magnetic field in the easy axis can give a residual out-of-plane magnetic field if the sample was not properly aligned, leading to a $R_{2\omega}$~\cite{hou2022ubiquitous}. In this case, the $\mu_0H_{c2}$ in easy axis should be a minimum due to a residual out-of-plane magnetic field, which is contradict to our observation of a maximum $\mu_0H_{c2}$ in Figure \ref{fig:4}e. Therefore, the in-plane magnetic field MCA is not caused by the residual out-of-plane magnetic field. To the best of our knowledge, this peculiar behavior has not been reported before. In the next section, we shall show how the form of the SOC in T$_d$-MoTe$_2$ plays a critical in determining the angular dependence of the non-reciprocal transport. 

\section*{Discussion}
\begin{figure}[t]
    \includegraphics[width=8.cm]{FigDVI.eps}        
    \caption{\label{figDVI}Angular variation $\varphi$ of the nonreciprocal component, i.e. the DVI in arbitrary unit, with magnetic field orientation, for the crystallographic $a$-axis pointing at $\varphi_a=$\SI{145}{\degree} with respect to the dc current whose flow direction defines \SI{0}{\degree} (see illustration of device configuration on the right panel). The dot-dashed black line indicates the crystallographic $a$-axis parallel to the mirror symmetry plane of T$_d$-MoTe$_2$. Band parameters are set to $(t_1, t_2, \mu, \alpha_1, \alpha_2, \alpha_3)=(2, 4, 1, 0.05, 0.001, -0.01)$. 
    } 
    \label{fig:FigDVI}
\end{figure}
We now develop a phenomenological theory to explain how an anisotropic SOC may give rise to the peculiar field angle dependence of the nonreciprocal resistance $R_{2\omega}$. The nonreciprocal charge transport in inversion-symmetry breaking systems is rooted in the asymmetric Zeeman shift of the spin-orbit coupled electronic structure~\cite{rikken2001electrical,wakatsuki2018nonreciprocal,ilic2022theory,daido2022intrinsic,yuan2022supercurrent}. This asymmetry causes a disparity between left and right moving electrons, which then leads to inequivalent transport behavior for left and right injected currents. It is thus tempting to describe the nonreciprocal coefficient as proportional to the field-induced differential change in the integral of the group velocity component parallel to the current flow. For later convenience, we shall refer to such a quantity as {\it Differential Velocity Integral} (DVI). In conjunction with our experiments, let's consider a single-crystal device configuration with a dc probe current whose flow direction we set to be $0^\circ$, and with certain main crystallographic axis oriented at angle $\varphi_a$ (see Fig.~\ref{fig:FigDVI} for illustration). The field angle dependence of the DVI is then given by,
\begin{equation}\label{eqDVI}
    \gamma_{\varphi_a}(\varphi) \propto -\partial_{\bH} \left\{\sum_n \int d\bk v_{n,I}(\bk,\bH) \delta[\xi_{n}(\bk,\bH)] \right\}_{\bH=0} \,.
\end{equation}
Here, $\bH=H(\cos\varphi,\sin\varphi,0)$ denotes the in-plane external field, $\xi_n(\bk,\bH)$ is the dispersion of the $n$-th band and $v_{n,I}(\bk,\bH)$ the group velocity of the same band projected into the current flow direction. Note that the minus sign in the front is due to the negative charge carried by electrons, and that the $\delta$-function reflects the dominant contribution from states around the Fermi level. As an important remark, a serious transport theory would also need to account for the field-induced corrections to other quantities, such as the transport effective mass. However, those corrections are at best the same order as the DVI. For our illustrative purpose, it thus suffices to use the DVI to roughly capture the observed field angle dependence. 

For a simple Rashba SOC model with continuous basal plane rotational symmetry, the DVI is insensitive to $\varphi_a$, but must undergo a sign change when the in-plane magnetic field is (anti)-parallel to the direction of the injected current and reach extremal values when the field is orthogonal. This is indeed verified in the Supplementary Material, and is consistent with the original predictions~\cite{rikken2005magnetoelectric,wakatsuki2018nonreciprocal}. Also presented there is a separate analysis for Dresselhaus SOC, which behaves in the opposite manner. In systems with anisotropic band structure and more complex form of spin-orbit texture, the nonreciprocal coefficient also acquires a dependence on $\varphi_a$, and the DVI develops a more complicated dependence on the field angle $\varphi$. In the following, we turn to the T$_d$-MoTe$_2$ thin film. 

The non-centrosymmetric crystal structure of T$_d$-MoTe$_2$ is characterized by an in-plane mirror symmetry ($\mathcal{M}_y$) while breaking another in-plane and the out-of-plane mirror symmetries. This gives rise to an asymmetric SOC \cite{cui2019transport}. In a simplified description, a 2D Hamiltonian appropriate for thin films of this compound, in accordance with the crystallographic and time-reversal symmetries, can be written as~\cite{footnote1},
\begin{equation}\label{eqSOC}
    H_\bk =  \epsilon_\bk + \alpha_1 k_y \sigma_x +\alpha_2 k_x \sigma_y +\alpha_3 k_y \sigma_z \,.
\end{equation} 
Here, $\epsilon_\bk = t_1 k_x^2 + t_2 k_y^2 -\mu$ depicts the spin-independent kinetic energy, and the remaining three terms with Pauli matrices $\sigma_i$ $(i=x,y,z)$ describe the spin-orbit interaction. Assuming $\varphi_a=$\SI{145}{\degree}, Figure \ref{fig:FigDVI} shows the DVI calculated for in-plane fields and under a specific set of band parameters. We see that the lineshape roughly matches with the observed angular variation of $R_{2\omega}$ in Figure \ref{fig:4}f, with the two lobes neither perpendicular nor parallel to the dc current \cite{footnote2}. To the best of our knowledge, such a peculiar lineshape has not been reported in other material platforms. To explain this behavior, the form of the SOC is of paramount importance. For one, it is important to steer away from the Rashba limit with $\alpha_1=-\alpha_2$, so that the most pronounced nonreciprocity under in-plane fields does not occur when the field is normal to the current flow direction. For another, a finite $\alpha_3$ is crucial to obtaining unequal (absolute) nonreciprocal coefficient between arbitrary field angles $\varphi$ and $\varphi+180^\circ$, i.e. unequal lobe size as seen in our $R_{2\omega}$ plot (Figure \ref{fig:4}f). In Supplemental Material Figure S8 and S9, the angular variation of the DVI for some other $\varphi_a$ and some other sets of SOC coefficients are plotted, showing strong sensitivity to these parameters. As a side note, finite $\alpha_3$ is also necessary for the occurrence of nonreciprocal transport when the magnetic field points out of plane. 

In summary, we observed a superconducting diode effect and a large superconducting MCA in noncentrosymmetric type-II Weyl semimetal T$_d$-MoTe\textsubscript{2}. Contrary to the finding in Rashba superconductors~\cite{wakatsuki2018nonreciprocal,yasuda2019nonreciprocal,hou2022ubiquitous}, the strongest MCA effect does not occur when the in-plane magnetic field is perpendicular to the current flow direction. And a phenomenological theory based on anisotropy SOC was developed to explain the result. The MCA coefficient $\gamma$ reach up to \SI{1e-9}{\per\tesla\per\ampere} at \SI{2}{\milli\tesla}, paving a way for application of superconducting diode effect under weak magnetic field.

\section*{Method}
\textbf{Crystal growth.} 1T$^\prime$-MoTe\textsubscript{2} bulk crystals were synthesized using the conventional chemical vapor transport process \cite{empante2017chemical,wang2021superconducting}. Briefly, high purity Molybdenum (\SI{99.9}{\percent}) and Tellurium (\SI{99.9}{\percent}) powders were mixed at a stoichiometric ratio of $1:2$ with iodine as transport agent, and then sealed in a quartz tube at high vacuum ($<\num{1e-4}$ torr). The reaction was performed in a two-zone furnace with the hot zone kept at $\SI{780}{\celsius}$ and the cold zone kept at $\SI{700}{\celsius}$ for seven days. The tube was then quenched to room temperature in ice to obtain plate-like crystals. 



\textbf{Transport measurement.} The $\text{MoTe}_2$ device was put in a dilution refrigerator with the mixing chamber temperature of about \SI{10}{\milli\kelvin} for a standard four-terminal measurement. The differential resistance was measured by coupling a relatively small AC current $\mathrm{d}I=\SI{40}{\nano\ampere}$ to a DC current $I$ and recording the AC voltage $\mathrm{d}V$ and the DC voltage $V$. The DC current with small AC excitation is generated by a DC source (DC205) and a lock-in amplifier (SR830) coupled by a summing amplifier (SIM918). The AC voltage was measured by a SR830. The resistance was measured by applying a small AC current of 40 nA using standard lock-in techniques. The first harmonic and second harmonic resistances were measured by two SR830s. The magnetic field was applied by a vector superconducting magnet.

\section*{Data availability}
The data that support the findings of this study are available from the corresponding authors on reasonable request.

\section*{Acknowledgement}
This work was supported by the Key-Area Research and Development Program of GuangDong Province (Grant No.2018B030327001), Guangdong Provincial Key Laboratory (Grant No.2019B121203002), NSFC (Grant No. 11904155), the Guangdong Science and Technology Department (Grant No. 2022A1515011948),  Shenzhen Science and Technology Program (Grant No. KQTD20200820113010023), National Natural Science Foundation of China (Nos. 11864022, 62264010) and Natural Science Foundation of Jiangxi Province, China (No. 20192ACB21014).

\section*{Author contributions}
J.-J.C., Z.-B.T., and D.-P.Y. conceived and supervised this project, and designed the experiments. W.-S.D., T.-F.Z. and H.J. fabricated the devices with support from S.L, Z.-C.M.,Z.-T.Z., Y.Z. and Z.-S.Z.. J.-J.C., W.-S.D. and Z.-B.T performed the transport measurements. W.-S.D., W.H., W.-P.C., J.-J.C. and Z.-B.T. analyzed the data and wrote the paper. Y.-B.Z. and G.-J.L. grew the $\text{MoTe}_2$ crystals. T.-Y.C. and N.W. gave scientific advice. All the authors discussed the results and commented on the manuscript.

\section*{Competing interests}
The authors declare they have no competing interests.

\begin{thebibliography}{10}
\expandafter\ifx\csname url\endcsname\relax
  \def\url#1{\texttt{#1}}\fi
\expandafter\ifx\csname urlprefix\endcsname\relax\def\urlprefix{URL }\fi
\providecommand{\bibinfo}[2]{#2}
\providecommand{\eprint}[2][]{\url{#2}}

\bibitem{rikken1997observation}
\bibinfo{author}{Rikken, G.} \& \bibinfo{author}{Raupach, E.}
\newblock \bibinfo{title}{Observation of magneto-chiral dichroism}.
\newblock \emph{\bibinfo{journal}{Nature}} \textbf{\bibinfo{volume}{390}},
  \bibinfo{pages}{493--494} (\bibinfo{year}{1997}).

\bibitem{rikken2001electrical}
\bibinfo{author}{Rikken, G.}, \bibinfo{author}{F{\"o}lling, J.} \&
  \bibinfo{author}{Wyder, P.}
\newblock \bibinfo{title}{Electrical magnetochiral anisotropy}.
\newblock \emph{\bibinfo{journal}{Phys. Rev. Lett.}}
  \textbf{\bibinfo{volume}{87}}, \bibinfo{pages}{236602}
  (\bibinfo{year}{2001}).

\bibitem{rikken2005magnetoelectric}
\bibinfo{author}{Rikken, G.} \& \bibinfo{author}{Wyder, P.}
\newblock \bibinfo{title}{Magnetoelectric anisotropy in diffusive transport}.
\newblock \emph{\bibinfo{journal}{Phys. Rev. Lett.}}
  \textbf{\bibinfo{volume}{94}}, \bibinfo{pages}{016601}
  (\bibinfo{year}{2005}).

\bibitem{avci2015unidirectional}
\bibinfo{author}{Avci, C.~O.} \emph{et~al.}
\newblock \bibinfo{title}{Unidirectional spin hall magnetoresistance in
  ferromagnet/normal metal bilayers}.
\newblock \emph{\bibinfo{journal}{Nat. Phys.}}
  \textbf{\bibinfo{volume}{11}}, \bibinfo{pages}{570--575}
  (\bibinfo{year}{2015}).

\bibitem{olejnik2015electrical}
\bibinfo{author}{Olejn{\'\i}k, K.}, \bibinfo{author}{Nov{\'a}k, V.},
  \bibinfo{author}{Wunderlich, J.} \& \bibinfo{author}{Jungwirth, T.}
\newblock \bibinfo{title}{Electrical detection of magnetization reversal
  without auxiliary magnets}.
\newblock \emph{\bibinfo{journal}{Phys. Rev. B}}
  \textbf{\bibinfo{volume}{91}}, \bibinfo{pages}{180402}
  (\bibinfo{year}{2015}).

\bibitem{yasuda2016large}
\bibinfo{author}{Yasuda, K.} \emph{et~al.}
\newblock \bibinfo{title}{Large unidirectional magnetoresistance in a magnetic
  topological insulator}.
\newblock \emph{\bibinfo{journal}{Phys. Rev. Lett.}}
  \textbf{\bibinfo{volume}{117}}, \bibinfo{pages}{127202}
  (\bibinfo{year}{2016}).

\bibitem{ideue2017bulk}
\bibinfo{author}{Ideue, T.} \emph{et~al.}
\newblock \bibinfo{title}{Bulk rectification effect in a polar semiconductor}.
\newblock \emph{\bibinfo{journal}{Nat. Phys.}}
  \textbf{\bibinfo{volume}{13}}, \bibinfo{pages}{578--583}
  (\bibinfo{year}{2017}).

\bibitem{lv2018unidirectional}
\bibinfo{author}{Lv, Y.} \emph{et~al.}
\newblock \bibinfo{title}{Unidirectional spin-hall and rashba- edelstein
  magnetoresistance in topological insulator-ferromagnet layer
  heterostructures}.
\newblock \emph{\bibinfo{journal}{Nat. Commun.}}
  \textbf{\bibinfo{volume}{9}}, \bibinfo{pages}{1--7} (\bibinfo{year}{2018}).

\bibitem{he2018bilinear}
\bibinfo{author}{He, P.} \emph{et~al.}
\newblock \bibinfo{title}{Bilinear magnetoelectric resistance as a probe of
  three-dimensional spin texture in topological surface states}.
\newblock \emph{\bibinfo{journal}{Nat. Phys.}}
  \textbf{\bibinfo{volume}{14}}, \bibinfo{pages}{495--499}
  (\bibinfo{year}{2018}).

\bibitem{he2018observation}
\bibinfo{author}{He, P.} \emph{et~al.}
\newblock \bibinfo{title}{Observation of out-of-plane spin texture in a
  {SrTiO}$_3$ (111) two-dimensional electron gas}.
\newblock \emph{\bibinfo{journal}{Phys. Rev. Lett.}}
  \textbf{\bibinfo{volume}{120}}, \bibinfo{pages}{266802}
  (\bibinfo{year}{2018}).

\bibitem{wakatsuki2017nonreciprocal}
\bibinfo{author}{Wakatsuki, R.} \emph{et~al.}
\newblock \bibinfo{title}{Nonreciprocal charge transport in noncentrosymmetric
  superconductors}.
\newblock \emph{\bibinfo{journal}{Sci. Adv.}}
  \textbf{\bibinfo{volume}{3}}, \bibinfo{pages}{e1602390}
  (\bibinfo{year}{2017}).

\bibitem{wakatsuki2018nonreciprocal}
\bibinfo{author}{Wakatsuki, R.} \& \bibinfo{author}{Nagaosa, N.}
\newblock \bibinfo{title}{Nonreciprocal current in noncentrosymmetric rashba
  superconductors}.
\newblock \emph{\bibinfo{journal}{Phys. Rev. Lett.}}
  \textbf{\bibinfo{volume}{121}}, \bibinfo{pages}{026601}
  (\bibinfo{year}{2018}).

\bibitem{yasuda2019nonreciprocal}
\bibinfo{author}{Yasuda, K.} \emph{et~al.}
\newblock \bibinfo{title}{Nonreciprocal charge transport at topological
  insulator/superconductor interface}.
\newblock \emph{\bibinfo{journal}{Nat. Commun.}}
  \textbf{\bibinfo{volume}{10}}, \bibinfo{pages}{1--6} (\bibinfo{year}{2019}).

\bibitem{itahashi2020nonreciprocal}
\bibinfo{author}{Itahashi, Y.~M.} \emph{et~al.}
\newblock \bibinfo{title}{Nonreciprocal transport in gate-induced polar
  superconductor {SrTiO}$_3$}.
\newblock \emph{\bibinfo{journal}{Sci. Adv.}}
  \textbf{\bibinfo{volume}{6}}, \bibinfo{pages}{eaay9120}
  (\bibinfo{year}{2020}).

\bibitem{zhang2020nonreciprocal}
\bibinfo{author}{Zhang, E.} \emph{et~al.}
\newblock \bibinfo{title}{Nonreciprocal superconducting {NbSe}$_2$ antenna}.
\newblock \emph{\bibinfo{journal}{Nat. Commun.}}
  \textbf{\bibinfo{volume}{11}}, \bibinfo{pages}{1--9} (\bibinfo{year}{2020}).

\bibitem{lyu2021superconducting}
\bibinfo{author}{Lyu, Y.-Y.} \emph{et~al.}
\newblock \bibinfo{title}{Superconducting diode effect via conformal-mapped
  nanoholes}.
\newblock \emph{\bibinfo{journal}{Nat. Commun.}}
  \textbf{\bibinfo{volume}{12}}, \bibinfo{pages}{1--7} (\bibinfo{year}{2021}).

\bibitem{lin2022zero}
\bibinfo{author}{Lin, J.-X.} \emph{et~al.}
\newblock \bibinfo{title}{Zero-field superconducting diode effect in
  small-twist-angle trilayer graphene}.
\newblock \emph{\bibinfo{journal}{Nat. Phys.}}
  \textbf{\bibinfo{volume}{18}}, \bibinfo{pages}{1221--1227}
  (\bibinfo{year}{2022}).

\bibitem{ando2020observation}
\bibinfo{author}{Ando, F.} \emph{et~al.}
\newblock \bibinfo{title}{Observation of superconducting diode effect}.
\newblock \emph{\bibinfo{journal}{Nature}} \textbf{\bibinfo{volume}{584}},
  \bibinfo{pages}{373--376} (\bibinfo{year}{2020}).

\bibitem{baumgartner2022supercurrent}
\bibinfo{author}{Baumgartner, C.} \emph{et~al.}
\newblock \bibinfo{title}{Supercurrent rectification and magnetochiral effects
  in symmetric josephson junctions}.
\newblock \emph{\bibinfo{journal}{Nat. Nanotechnol}}
  \textbf{\bibinfo{volume}{17}}, \bibinfo{pages}{39--44}
  (\bibinfo{year}{2022}).

\bibitem{wu2022field}
\bibinfo{author}{Wu, H.} \emph{et~al.}
\newblock \bibinfo{title}{The field-free josephson diode in a van der waals
  heterostructure}.
\newblock \emph{\bibinfo{journal}{Nature}} \textbf{\bibinfo{volume}{604}},
  \bibinfo{pages}{653--656} (\bibinfo{year}{2022}).

\bibitem{pal2022josephson}
\bibinfo{author}{Pal, B.} \emph{et~al.}
\newblock \bibinfo{title}{Josephson diode effect from cooper pair momentum in a
  topological semimetal}.
\newblock \emph{\bibinfo{journal}{Nat. Phys.}} \bibinfo{pages}{1--6}
  (\bibinfo{year}{2022}).

\bibitem{hou2022ubiquitous}
\bibinfo{author}{Hou, Y.} \emph{et~al.}
\newblock \bibinfo{title}{Ubiquitous superconducting diode effect in
  superconductor thin films}.
\newblock \emph{\bibinfo{journal}{arXiv preprint arXiv:2205.09276}}
  (\bibinfo{year}{2022}).

\bibitem{tokura2018nonreciprocal}
\bibinfo{author}{Tokura, Y.} \& \bibinfo{author}{Nagaosa, N.}
\newblock \bibinfo{title}{Nonreciprocal responses from non-centrosymmetric
  quantum materials}.
\newblock \emph{\bibinfo{journal}{Nat. Commun.}}
  \textbf{\bibinfo{volume}{9}}, \bibinfo{pages}{1--14} (\bibinfo{year}{2018}).

\bibitem{sun2015prediction}
\bibinfo{author}{Sun, Y.}, \bibinfo{author}{Wu, S.-C.}, \bibinfo{author}{Ali,
  M.~N.}, \bibinfo{author}{Felser, C.} \& \bibinfo{author}{Yan, B.}
\newblock \bibinfo{title}{Prediction of weyl semimetal in orthorhombic
  {MoTe}$_2$}.
\newblock \emph{\bibinfo{journal}{Phys. Rev. B}}
  \textbf{\bibinfo{volume}{92}}, \bibinfo{pages}{161107}
  (\bibinfo{year}{2015}).

\bibitem{qi2016superconductivity}
\bibinfo{author}{Qi, Y.} \emph{et~al.}
\newblock \bibinfo{title}{Superconductivity in weyl semimetal candidate
  {MoTe}$_2$}.
\newblock \emph{\bibinfo{journal}{Nat. Commun.}}
  \textbf{\bibinfo{volume}{7}}, \bibinfo{pages}{1--7} (\bibinfo{year}{2016}).

\bibitem{zhang2016raman}
\bibinfo{author}{Zhang, K.} \emph{et~al.}
\newblock \bibinfo{title}{Raman signatures of inversion symmetry breaking and
  structural phase transition in type-{II} weyl semimetal {MoTe}$_2$}.
\newblock \emph{\bibinfo{journal}{Nat. Commun.}}
  \textbf{\bibinfo{volume}{7}}, \bibinfo{pages}{1--6} (\bibinfo{year}{2016}).

\bibitem{liu2013plasma}
\bibinfo{author}{Liu, Y.} \emph{et~al.}
\newblock \bibinfo{title}{Layer-by-Layer Thinning of MoS$_2$ by Plasma}.
\newblock \emph{\bibinfo{journal}{ACS Nano}} \textbf{\bibinfo{volume}{7}},
  \bibinfo{pages}{4202--4209} (\bibinfo{year}{2013}).

\bibitem{berezinskii1972destruction}
\bibinfo{author}{Berezinskii, V.}
\newblock \bibinfo{title}{Destruction of long-range order in one-dimensional
  and two-dimensional systems possessing a continuous symmetry group. ii.
  quantum systems}.
\newblock \emph{\bibinfo{journal}{Sov. Phys. JETP}}
  \textbf{\bibinfo{volume}{34}}, \bibinfo{pages}{610--616}
  (\bibinfo{year}{1972}).

\bibitem{kosterlitz1973ordering}
\bibinfo{author}{Kosterlitz, J.~M.} \& \bibinfo{author}{Thouless, D.~J.}
\newblock \bibinfo{title}{Ordering, metastability and phase transitions in
  two-dimensional systems}.
\newblock \emph{\bibinfo{journal}{J. Phys.C}}
  \textbf{\bibinfo{volume}{6}}, \bibinfo{pages}{1181} (\bibinfo{year}{1973}).

\bibitem{halperin1979resistive}
\bibinfo{author}{Halperin, B.} \& \bibinfo{author}{Nelson, D.~R.}
\newblock \bibinfo{title}{Resistive transition in superconducting films}.
\newblock \emph{\bibinfo{journal}{J. Low Temp. Phys.}}
  \textbf{\bibinfo{volume}{36}}, \bibinfo{pages}{599--616}
  (\bibinfo{year}{1979}).

\bibitem{he2022phenomenological}
\bibinfo{author}{He, J.~J.}, \bibinfo{author}{Tanaka, Y.} \&
  \bibinfo{author}{Nagaosa, N.}
\newblock \bibinfo{title}{A phenomenological theory of superconductor diodes}.
\newblock \emph{\bibinfo{journal}{ New J. Phys}}
  \textbf{\bibinfo{volume}{24}}, \bibinfo{pages}{053014}
  (\bibinfo{year}{2022}).

\bibitem{bauriedl2022supercurrent}
\bibinfo{author}{Bauriedl, L.} \emph{et~al.}
\newblock \bibinfo{title}{Supercurrent diode effect and magnetochiral
  anisotropy in few-layer {NbSe}$_2$}.
\newblock \emph{\bibinfo{journal}{Nat. Commun.}}
  \textbf{\bibinfo{volume}{13}}, \bibinfo{pages}{1--7} (\bibinfo{year}{2022}).

\bibitem{shin2021magnetic}
\bibinfo{author}{Shin, J.} \emph{et~al.}
\newblock \bibinfo{title}{Magnetic proximity-induced superconducting diode
  effect and infinite magnetoresistance in van der waals heterostructure}.
\newblock \emph{\bibinfo{journal}{arXiv preprint arXiv:2111.05627}}
  (\bibinfo{year}{2021}).

\bibitem{hoshino2018nonreciprocal}
\bibinfo{author}{Hoshino, S.}, \bibinfo{author}{Wakatsuki, R.},
  \bibinfo{author}{Hamamoto, K.} \& \bibinfo{author}{Nagaosa, N.}
\newblock \bibinfo{title}{Nonreciprocal charge transport in two-dimensional
  noncentrosymmetric superconductors}.
\newblock \emph{\bibinfo{journal}{Phys. Rev. B}}
  \textbf{\bibinfo{volume}{98}}, \bibinfo{pages}{054510}
  (\bibinfo{year}{2018}).

\bibitem{carapella2009bistable}
\bibinfo{author}{Carapella, G.}, \bibinfo{author}{Granata, V.},
  \bibinfo{author}{Russo, F.} \& \bibinfo{author}{Costabile, G.}
\newblock \bibinfo{title}{Bistable abrikosov vortex diode made of a {Py}--{Nb}
  ferromagnet-superconductor bilayer structure}.
\newblock \emph{\bibinfo{journal}{Appl. Phys. Lett.}}
  \textbf{\bibinfo{volume}{94}}, \bibinfo{pages}{242504}
  (\bibinfo{year}{2009}).

\bibitem{cui2019transport}
\bibinfo{author}{Cui, J.} \emph{et~al.}
\newblock \bibinfo{title}{Transport evidence of asymmetric spin--orbit coupling
  in few-layer superconducting 1{T}$_d$-{MoTe}$_2$}.
\newblock \emph{\bibinfo{journal}{Nat. Commun.}}
  \textbf{\bibinfo{volume}{10}}, \bibinfo{pages}{1--8} (\bibinfo{year}{2019}).

\bibitem{ilic2022theory}
\bibinfo{author}{Ili{\'c}, S.} \& \bibinfo{author}{Bergeret, F.}
\newblock \bibinfo{title}{Theory of the supercurrent diode effect in rashba
  superconductors with arbitrary disorder}.
\newblock \emph{\bibinfo{journal}{Phys. Rev. Lett.}}
  \textbf{\bibinfo{volume}{128}}, \bibinfo{pages}{177001}
  (\bibinfo{year}{2022}).

\bibitem{daido2022intrinsic}
\bibinfo{author}{Daido, A.}, \bibinfo{author}{Ikeda, Y.} \&
  \bibinfo{author}{Yanase, Y.}
\newblock \bibinfo{title}{Intrinsic superconducting diode effect}.
\newblock \emph{\bibinfo{journal}{Phys. Rev. Lett.}}
  \textbf{\bibinfo{volume}{128}}, \bibinfo{pages}{037001}
  (\bibinfo{year}{2022}).

\bibitem{yuan2022supercurrent}
\bibinfo{author}{Yuan, N.~F.} \& \bibinfo{author}{Fu, L.}
\newblock \bibinfo{title}{Supercurrent diode effect and finite-momentum
  superconductors}.
\newblock \emph{\bibinfo{journal}{P.N.A.S.}} \textbf{\bibinfo{volume}{119}}, \bibinfo{pages}{e2119548119}
  (\bibinfo{year}{2022}).

\bibitem{footnote1}
\bibinfo{title}{While a more accurate microscopic model is necessarily more complicate, 
  the simplification does not significantly alter
  our main conclusion that the anisotropic band structure and asymmetric
  spin-orbit coupling are critical for the peculiar nonreciprocal responses
  observed in thin films of {T}$_d$-{MoTe}$_2$} .

\bibitem{footnote2}
\bibinfo{title}{Note that the lineshape is actually symmetric about an in-plane
  axis that is a few degrees off the crystallographic $a$-axis, i.e. the easy axis.} .

\bibitem{empante2017chemical}
\bibinfo{author}{Empante, T.~A.} \emph{et~al.}
\newblock \bibinfo{title}{Chemical vapor deposition growth of few-layer mote2
  in the 2{H}, 1{T}$^\prime$, and 1{T} phases: tunable properties of {MoTe}$_2$
  films}.
\newblock \emph{\bibinfo{journal}{ACS nano}} \textbf{\bibinfo{volume}{11}},
  \bibinfo{pages}{900--905} (\bibinfo{year}{2017}).

\bibitem{wang2021superconducting}
\bibinfo{author}{Wang, Z.} \emph{et~al.}
\newblock \bibinfo{title}{Superconducting 2{D} {NbS}$_2$ grown epitaxially by
  chemical vapor deposition}.
\newblock \emph{\bibinfo{journal}{ACS Nano}} \textbf{\bibinfo{volume}{15}},
  \bibinfo{pages}{18403--18410} (\bibinfo{year}{2021}).

\end{thebibliography}

\end{document}
