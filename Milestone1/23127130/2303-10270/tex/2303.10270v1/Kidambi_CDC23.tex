%%%%%%%%%%%%%%%%%%%%%%%%%%%%%%%%%%%%%%%%%%%%%%%%%%%%%%%%%%%%%%%%%%%%%%%%%%%%%%%%
%2345678901234567890123456789012345678901234567890123456789012345678901234567890
%        1         2         3         4         5         6         7         8

\documentclass[letterpaper, 10 pt, conference]{ieeeconf}  % Comment this line out if you need a4paper

%\documentclass[a4paper, 10pt, conference]{ieeeconf}      % Use this line for a4 paper

\IEEEoverridecommandlockouts                              % This command is only needed if 
                                                          % you want to use the \thanks command

\overrideIEEEmargins                                      % Needed to meet printer requirements.

%In case you encounter the following error:
%Error 1010 The PDF file may be corrupt (unable to open PDF file) OR
%Error 1000 An error occurred while parsing a contents stream. Unable to analyze the PDF file.
%This is a known problem with pdfLaTeX conversion filter. The file cannot be opened with acrobat reader
%Please use one of the alternatives below to circumvent this error by uncommenting one or the other
%\pdfobjcompresslevel=0
%\pdfminorversion=4

% See the \addtolength command later in the file to balance the column lengths
% on the last page of the document

% The following packages can be found on http:\\www.ctan.org
\usepackage{graphics} % for pdf, bitmapped graphics files
\usepackage{epsfig} % for postscript graphics files
%\usepackage{mathptmx} % assumes new font selection scheme installed
%\usepackage{times} % assumes new font selection scheme installed
\usepackage{amsmath} % assumes amsmath package installed
\usepackage{theoremref}
\usepackage{epstopdf}
\usepackage{amssymb}  % assumes amsmath package 
\newtheorem{remark}{Remark}
\newtheorem{assumption}{Assumption}
\newtheorem{theorem}{Theorem}
\newtheorem{definition}{Definition}
\newtheorem{condition}{Condition}
\newtheorem{lemma}{Lemma}
\title{\LARGE \bf

Adaptive Modified RISE-based Quadrotor Trajectory Tracking with Actuator Uncertainty Compensation}


\author{Krishna Bhavithavya Kidambi$^{1}$, Madhur Tiwari$^{2}$, Emmanuel Ogbanje Ijoga$^{3}$ and William MacKunis$^{4}$% <-this % stops a space
% \thanks{*This work was not supported by any organization}% <-this % stops a space
\thanks{$^{1}$Krishna Bhavithavya Kidambi is an Assistant Professor in the Department of Mechanical and Aerospace Engineering, University of Dayton, Dayton, OH 45469.
	{\tt\small kkidambi1@udayton.edu}}
\thanks{$^{2}$Madhur Tiwari is an Assistant Professor in the Department of Aerospace, Physics and Space Sciences, Florida Institute of Technology, Melbourne, FL 32901.
	{\tt\small mtiwari1@fit.edu}}
\thanks{$^{3}$Emmanuel Ogbanje Ijoga, William MacKunis are with Physical Sciences Department, Embry-Riddle Aeronautical University, Daytona
	Beach, FL 32114.
	{\tt\small ijogae@my.erau.edu}, {\tt\small mackuniw@erau.edu}}
}


\begin{document}



\maketitle
\thispagestyle{empty}
\pagestyle{empty}


%%%%%%%%%%%%%%%%%%%%%%%%%%%%%%%%%%%%%%%%%%%%%%%%%%%%%%%%%%%%%%%%%%%%%%%%%%%%%%%%
\begin{abstract}

This paper presents an adaptive robust nonlinear control method, which achieves reliable trajectory tracking control for a quadrotor unmanned aerial vehicle in the presence of gyroscopic effects, rotor dynamics, and external disturbances. Through novel mathematical manipulation in the error system development, the quadrotor dynamics are expressed in a control-oriented form, which explicitly incorporates the uncertainty in the gyroscopic term and control actuation term. An adaptive robust nonlinear control law is then designed to stabilize both the position and attitude loops of the quadrotor system. A rigorous Lyapunov-based analysis is utilized to prove asymptotic trajectory tracking, where the region of convergence can be made arbitrarily large through judicious control gain selection. Moreover, the stability analysis formally addresses gyroscopic effects and actuator uncertainty. To illustrate the performance of the control law, comparative numerical simulation results are provided, which demonstrate the improved closed-loop performance achieved under varying levels of parametric uncertainty and disturbance magnitudes.

\end{abstract}


%%%%%%%%%%%%%%%%%%%%%%%%%%%%%%%%%%%%%%%%%%%%%%%%%%%%%%%%%%%%%%%%%%%%%%%%%%%%%%%%
\section{INTRODUCTION}

Unmanned aerial vehicles (UAVs) (or Quadrotors) have the potential to transport people and things to locations not traditionally served by current modes of air transportation. The UAV market can potentially proliferate, and one can sense that the lack of regulation is currently preventing this. Hundreds of exciting and promising applications in various domains such as air delivery, surveying, agriculture, architecture, security, and video entertainment are being developed \cite{idrissi2022review, sonugur2022review}. 

% Unfortunately, most are not viable today because they fall out of the regulation frame.
% International organizations are just beginning to discuss regulations and certification procedures for
% autonomously flying vehicles (drones or personal aerial vehicles). Examples include the Global UTM Association, the Federal Aviation Administration (FAA), the European Aviation Safety Agency (EASA), NASA UAM Grand Challenge, and the Technical Committee ISO 020/SC 16 for ‘Unmanned aircraft systems’, to name a few.

Developing reliable and safe control algorithms is still a challenging task in a myraid of quadrotor control applications \cite{rinaldi2023comparative}. Over the last decade several methods have been proposed to compensate for uncertain and time-varying external operating conditions; however, there remain numerous open challenges in the design of robust and adaptive nonlinear quadrotor UAV control systems, which are rigorously proven to simultaneously compensate for external disturbances (e.g., due wind gusts and aerodynamic anomalies) and internal anomalies, disturbances, and unmodeled effects in the UAV actuation dynamics (e.g., due to electromechanical propeller motor dynamics and/or actuator faults).

Numerous recent approaches to quadrotor control are based on classical control methods applied to linearized models of the UAV dynamics \cite{yu2021unified, cohen2020finite,miranda2020robust,foehn2018}. Popular linear control methods include proportional-integral-derivative (PID) control \cite{miranda2020robust} and linear quadratic regulators (LQR) \cite{foehn2018}. While PID and LQR methods benefit from simple implementation and intuitive control gain-tuning, they have limited capability to compensate for the inherent nonlinearities in the quadrotor dynamics; and they tend to have low robustness to external disturbances.

In an effort to overcome the limitations of linear control methods, nonlinear control techniques have been widely shown to achieve an increased level of reliability in quadrotor flight under various uncertain and adversarial operating conditions. Common nonlinear control design approaches include feedback linearization \cite{martins2022inner, mahmood2017}, backstepping \cite{wang2023adaptive, wen2021optimized, liu2021non}, adaptive control \cite{patnaik2023adaptive, bai2021computational} and robust control \cite{dhadekar2021robust, lin2022neural,baek2023synthesized}. 
Advantages to feedback linearization include mathematical simplicity and easy implementation; however, they often incur reduced robustness to disturbances and sensitivity to model uncertainty. To address this, feedback linearization-based linear control designs often employ robust nonlinear compensators to achieve the desired disturbance-rejection objectives \cite{mahmood2017}. The authors in \cite{liu2021non} present an adaptive back-stepping controller to track time-varying trajectories with parameter uncertainties and under-actuation. The control algorithm is extended to include velocity input and motor coefficients, and geometric parameters are handled by the adaptive laws.  

% Nonlinear adaptive control methods have also been presented to compensate for disturbances, but these results can often be limited to local convergence results \cite{mahmood2017}.


 In \cite{dhadekar2021robust}, a robust control scheme based on nonlinear dynamic inversion (NDI) for
disturbance rejection is proposed. However, to achieve performance in agile
quadcopters, accounting for actuator uncertainty is vital.
The authors in \cite{jia2022agile} develop an agile control subjected to aerodynamic drag, the center of gravity shift, and motor dynamics. It is shown that accounting for uncertainties in the actuator dynamics is essential to achieve the desired trajectory-tracking performance.


While standard sliding mode control approaches are widely known for their disturbance-rejection capabilities, their implementation can be challenging due to the use of high-bandwidth switching \cite{li2018model, yu2021novel}. Motivated by this, robust integral of the sign of the error (RISE)-based control can be used to achieve reliable disturbance rejection using continuous control \cite{xian2004continuous}. RISE has been widely shown to achieve superior rejection of norm-bounded disturbances in various recent results \cite{yang2023output, wang2019velocity, shao2018rise}. In addition, our recent result in \cite{kidambi2020robust} presents a modified RISE (MRISE)-based quadrotor control design, which is proven to reject external disturbances of varying magnitudes and compensate for varying levels of parametric uncertainty in the UAV dynamics. While the MRISE result achieved significant performance improvement in trajectory tracking compared to traditional RISE, the work does not formally incorporate the  effects of actuator uncertainty and gyroscopic effects in the quadrotor dynamic model.

Motivated by this and future demands of building robust and agile quadrotors, we propose an adaptive MRISE-based trajectory tracking control method for a quadrotor system subjected to parametric uncertainty and unmodeled gyroscopic effects and uncertainties in the actuator dynamics in addition to external disturbances. The result is achieved through a non-trivial reworking of the error system development and stability analysis in \cite{kidambi2020robust}, which now formally incorporates unmodeled effects in the actuation dynamic model. The primary contributions of this result can be summarized as follows:
\begin{enumerate}
    \item An augmented adaptive MRISE control design for both the inner and outer loop of the quadrotor tracking control system, which formally incorporates feedback compensators for the actuator uncertainty, gyroscopic effects, model uncertainties and external disturbances. The control algorithm accounts for varying magnitudes of uncertainty ($\pm$ 15\%)
   with little or no gain tuning.
    \item A rigorous closed-loop stability analysis using a Lyapunov-based approach, which provides a detailed derivation of semi-global asymptotic tracking and the specific control gain conditions under which convergence can be guaranteed.
    \item A detailed comparative numerical simulation study to demonstrate the performance of the proposed control design.
\end{enumerate}
The rest of the paper is organized as follows: Section \ref{mathmodel} describes the nonlinear dynamic model of the quadrotor, especially Section \ref{uncertainmodel} describes a detailed model incorporating the gyroscopic effects and the actuator dynamics. The open-loop and closed-loop error systems are described in Section \ref{openlooperror} and \ref{closedlooperror}, respectively. In Section \ref{stabilityanaly}, the closed-loop stability analysis is presented. Finally, Section \ref{simresults} presents the numerical simulations to demonstrate the effectiveness of the proposed method, followed by conclusions in Section \ref{conclusion}.


\section{MATHEMATICAL MODEL}\label{mathmodel}
\subsection{Quad Dynamics}
The mathematical model of quadrotor dynamics is well studied in the literature \cite{tang2018,mahony2012multirotor}. The explicit form of the dynamics by using the Newton–Euler method with position and orientation can be written as \cite{svacha2020imu}
\begin{subequations}\label{sys1}
\begin{eqnarray}
	m \dot{V}&=& F_g+F_T+F_d+d_1(t) \\
	I \dot{\Omega} &=& -\Omega^\times I \Omega + \tau -\tau_a + \tau_g + d_2(t)
\end{eqnarray}
\end{subequations}
where $m \in \mathbb{R}^+$ is the mass of the quadrotor, $V = [V_x,V_y,V_z]^T$ and $\Omega = [p,q,r]^T$ denote the translational and angular velocity defined in the inertial frame and body frame, respectively. $I = diag(I_x,I_y,I_z) \in \mathbb{R}^{3\times 3}$ is a positive definite diagonal moment of inertia matrix, and $d_1(t), d_2(t)$ are the external disturbances in the translation and rotational dynamics respectively. 

The remaining forces in the translational dynamics are given as $F_g = [0,0,-mg]^T$, where g = 9.81 $\frac{m}{s^2}$ is the gravitational acceleration. $F_d$ is the drag due to the translational motion and is given by \begin{equation}
    F_d = -diag(K_{d_x},K_{d_y},K_{d_z})V 
\end{equation}
where $K_{d_x},K_{d_y},K_{d_z}$ are translational drag coefficients. The control input $F_T \in \mathbb{R}$ is the total thrust generated by all four motors and is given as 
\begin{equation}\label{eq3}
F_T = R_1
    \sum_{i=1}^{4}F_i
\end{equation}
where $F_i = K_T u_i^2$, and  $K_T$ is the thrust coefficient and $u_i$ is the speed of $i^{th}$ motor. The term $R_1$ is given as 
\begin{equation}\label{R1}
R_1 = 
\begin{bmatrix}
    \cos(\psi) \sin(\theta) \cos(\phi)+\sin(\psi) \sin(\phi)\\
\sin(\phi)\sin(\theta)\sin(\psi)-\cos(\psi)\sin(\phi)\\
    \cos(\theta)\cos(\phi)
\end{bmatrix}
\end{equation}

The gyroscopic torque $\tau_g$ is given as 
\begin{equation}
    \tau_g = \sum_{i=1}^{4} \Omega^\times I_r \begin{bmatrix}
        0 \\0 \\
        (-1)^{i-1}u_i
    \end{bmatrix}
\end{equation}
where $I_r$ is the rotor inertia. The aerodynamic frictional torque $\tau_a$ is given as \begin{equation}
     \tau_a = -diag(K_{a_x},K_{a_y},K_{a_z})\Vert \Omega^2 \Vert
\end{equation}
where $K_{a_x},K_{a_y},K_{a_z}$ are the coefficients of aerodynamic friction. The torque $\tau$ applied on the quadrotor is given by 

\begin{eqnarray}
    \tau = \begin{bmatrix}
        \tau_{\phi} \\
        \tau_{\theta} \\
        \tau_{\psi}
    \end{bmatrix} = \begin{bmatrix}
        0 & -lK_T & 0 & lK_T \\
        lK_T & 0 & -lK_T & 0 \\
        K_D & -K_D & K_D & -K_D 
    \end{bmatrix}
    \begin{bmatrix}
        u_1^2 \\
        u_2^2 \\
        u_3^2 \\
        u_4^2 \\
    \end{bmatrix}
\end{eqnarray}
where $\tau_{\phi},\tau_{\theta},\tau_{\psi}$ denotes the torque along $x,y,z$ axis respectively, $l$ is the distance between each rotor to the quadrotor center of mass, and $K_D$ is the drag coefficient.

\subsection{Actuator Dynamics}
The control input commands to the quadrotor are applied through a DC motor and when tracking aggressive trajectories, actuator dynamics cannot be ignored for the sake of precise tracking performance. The dynamics of the $i^{th}$ motor speed is given as \cite{jia2022agile}
\begin{equation}
    \dot{u}_i = \frac{k_\tau}{I_r}(u_{ci}-u_i)
\end{equation}
where $k_{\tau}$ is the motor coefficient and $u_{ci}$ is the commanded speed of the $i^{th}$ motor and $u_i$ is introduced in (\ref{eq3}).
\subsection{Uncertain Quadrotor Model}\label{uncertainmodel}
The dynamic equations in (\ref{sys1}) are recast by incorporating parametric and actuator uncertainties and are simplified as 
\begin{subequations}
\begin{eqnarray}
	\dot{P} &=& V \label{x1}\\
	\dot{V} &=& f_1(V)+F_V+\Delta_V F_V+d_V(t) \label{x2}\\
	\dot{\omega} &=& \Omega  \label{x3} \\
	\dot{\Omega} &=& f_2(\Omega)+\tau_{\Omega}+\Delta_\Omega \tau_{\Omega}+d_{\Omega}(t)  \label{x4}
	\end{eqnarray}
 \end{subequations}
 where $P = [x,y,z]^T$, $V = [V_x,V_y,V_z]^T$, $\omega = [\phi,\theta,\psi]^T$ and $\Omega = [p,q,r]^T$ are the position, velocity, angular position and angular velocity in 3-D; and the following definitions have been made:
 \begin{equation*}
         F_V = \frac{F_T}{m} ; \qquad  \Delta_V F_V = \frac{R_1}{m}
    \sum_{i=1}^{4}\Delta F_i 
 \end{equation*} \vspace{-0.4cm}
 \begin{eqnarray*}
    f_1(V) &=& \frac{\bar{F}_g}{m}+ \frac{\bar{F}_d}{m} \\
     d_V(t) &=& -\frac{\Delta m \dot{V}}{m}+\frac{\Delta F_g}{m}+\frac{\Delta F_d}{m}+\frac{d_1(t)}{m} \\
    \tau_{\Omega} &=& R_2\times [u_1^2,u_2^2,u_3^2,u_4^2]^T \\ \Delta_\Omega \tau_{\Omega} &=& \Delta R_2 \times [u_1^2,u_2^2,u_3^2,u_4^2]^T \\
    f_2(\Omega) &=& I^{-1}\Big[- \overline{\Omega^\times I \Omega}-\bar{\tau}_a+\bar{\tau}_g\Big] 
 \end{eqnarray*}
 \begin{equation*}
     d_{\Omega}(t) = I^{-1}\Big[-\overline{\Omega^\times \Delta I \Omega}-\Delta I \dot{\Omega} -\Delta \tau_a + \Delta \tau_g +d_2(t) \Big]
 \end{equation*}
 
\section{Control Model}
The control objective is to design the control signal to regulate the quadrotor states to a given desired time-varying reference trajectory that is sufficiently smooth, despite the presence of uncertainties in the dynamic and actuator model respectively. Thus, the control objective can be mathematically stated as 
	\begin{equation} \label{e}
	\Vert \xi(t) \Vert \rightarrow 0
	\end{equation}
	where $\xi$ in (\ref{e}) is the difference between the current state and desired state of the quadrotor and $\Vert . \Vert$ in (\ref{e}) denotes standard Euclidean norm. 
 \begin{assumption} \label{destraj}
		The desired trajectory profile $P_{d}\left(
		t\right) $ and its first three-time derivatives are bounded in the sense that $P_{d}\left( t\right) ,\dot{P}_{d}\left( t\right) ,\ddot{P}_{d}\left(
		t\right) ,\dddot{P}_{d}\left( t\right) \in \mathcal{L}_{\infty }$ $\forall $ 
		$t\geq 0$.
\end{assumption}
\begin{assumption} \label{disass}
		The distrubances in the quadtrotor dynamics in (\ref{x2}) and (\ref{x4}) are smooth enough such that 
  \begin{equation}
      \Vert \dot{d}_V \Vert \leq \epsilon_1, \qquad \Vert \dot{d}_\Omega \Vert \leq \epsilon_2 
  \end{equation}
  where $\epsilon_1, \epsilon_2$ are some unknown positive constants.
	\end{assumption}
\subsection{Open-loop Error System}\label{openlooperror}
\subsubsection{Position Loop}
To assess the controller performance and to enable the subsequent stability analysis, the tracking error $\xi_1(t) \in \mathbb{R}^3$ and auxiliary tracking errors $\xi_2(t), \eta_1(t) \in \mathbb{R}^3$ for the position loop are defined as 
\begin{eqnarray}
    \xi_1 = P-P_{d},  \label{xi1} \\
    \xi_2 = \dot{\xi}_1+k_1\xi_1  = V-V^*_d \label{xi2} \\
    V^*_d = V_d-k1\xi_1 \\
    \eta_1 = \dot{\xi}_2 + k_2 \xi_2  \label{eta1}
\end{eqnarray}
where $P_d = [x_d,y_d,z_d]^T \in \mathbb{R}^3$, $V_d = [V_{d_x},V_{d_y},V_{d_z}]^T \in \mathbb{R}^3$ are the desired position and velocity; and $k_1,k_2 \in \mathbb{R}$ denote positive control gains. Taking the derivative of (\ref{eta1}) and using the definition in (\ref{xi1}) and (\ref{xi2}) the open loop error dynamics can be expressed as 
\begin{eqnarray}
    % \dot{\eta}_1 = \ddot{\xi}_2 + k_2 \dot{\xi}_2  
    % = \ddot{V}-\ddot{V}^*_{d}+ k_2 \dot{\xi}_2 \label{eta1dot1} \\
    \dot{\eta}_1 = \ddot{V}-\ddot{V}_{d}+k_1(\dot{\xi}_2-k_1\dot{\xi}_1)+k_2\dot{\xi}_2 \label{eta1dot2}
\end{eqnarray}
By using the definition in (\ref{x2}) the tracking error in (\ref{eta1dot2}) is expanded as 
\begin{eqnarray}\label{eta1dot3}
    \dot{\eta}_1 &=& \dot{f}_1(V)+\dot{F}_V(t)+\dot{\Delta}_V F_V(t)+\dot{d}_V(t)-\ddot{V}_{d} \notag \\ &+&k_1(\dot{\xi}_2-k_1\dot{\xi}_1)+k_2\dot{\xi}_2.
\end{eqnarray}
The error dynamics in (\ref{eta1dot3}) can be expressed as 
\begin{equation}
    \dot{\eta}_1 = \tilde{\Gamma}_1+\Gamma_{1d}+\dot{F}_V(t)+\dot{\Delta}_V F_V(t)-\xi_2 \label{eta1dot4}
\end{equation}
where the unknown auxiliary functions, $\tilde{\Gamma}_1(t), \Gamma_{1d}(t) \in \mathbb{R}^3$ are defined as \begin{eqnarray}\label{gamma1tilde1}
\tilde{\Gamma}_1(t) &=& \dot{f}_1(V)+k_1(\dot{\xi}_2-k_1\dot{\xi}_1)+k_2\dot{\xi}_2+\xi_2 \\
\Gamma_{1d} &=& \dot{d}_V(t)-\ddot{V}_{d}.  \label{gamma1d}
\end{eqnarray}
 \begin{assumption}\label{delta1assu}
Approximate model knowledge is available such that $\dot{\Delta}_V(t)$ satisfies%
		\begin{equation} \label{delta1ineq}
		\left\Vert \dot{\Delta}_V(t) \right\Vert _{i\infty }<\delta_1 <1,  
		\end{equation}%
		where $\delta_1 \in \mathbb{R}^{+}$ is a known bounding constant, and $%
		\left\Vert \cdot \right\Vert _{i\infty }$ denotes the induced infinity norm. 
 \end{assumption}
The motivation for the separation of terms in  (\ref{gamma1tilde1}) and (\ref{gamma1d}) is based on the fact that the following inequalities can be developed 
	\begin{equation}\label{gamma1tilde2}
	\hspace{-0.2cm}	\Vert \tilde{\Gamma}_1\Vert \leq \rho_1 \left( \left\Vert \mu_1 \right\Vert \right)
	\left\Vert \mu_1\right\Vert ,\:\:\: \Vert \Gamma_{1d}\Vert \leq \kappa _{\Gamma_{1d}},\:\:\:
	\Vert \dot{\Gamma}_{1d}\Vert \leq \kappa _{\dot{\Gamma}_{1d}} 
	\end{equation}
	where $\kappa _{\Gamma_{1d}}$, $\kappa _{\dot{\Gamma}_{1d}} \in \mathbb{R}^{+}$ are known
	bounding constants; $\rho \left( \cdot \right) $ is a positive, globally
	invertible, non-decreasing function; and $\mu_1(t)\in \mathbb{R}^{9}$ is
	defined as 
	\begin{equation}\label{mu1def}
	\mu_1\left( t\right) \triangleq \left[ 
	\begin{array}{ccc}
	\xi_1^{T}\left( t\right) & \xi_2^{T}\left( t\right) & \eta_1^{T}\left( t\right)%
	\end{array}%
	\right] ^{T}. 
	\end{equation}

\subsubsection{Attitude Loop}
The tracking error in the attitude loop $\xi_3(t) \in \mathbb{R}^3$ and auxiliary tracking errors $\xi_4(t), \eta_2(t) \in \mathbb{R}^3$ are defined as
\begin{eqnarray}
    \xi_3 = \omega-\omega_{d},  \label{xi3} \\
    \xi_4 = \dot{\xi}_3+k_3\xi_3  = \Omega-\Omega^*_d \label{xi4} \\
    \Omega^*_d = \Omega_d-k_3\xi_3 \\
    \eta_2 = \dot{\xi}_4 + k_4 \xi_4  \label{eta2}
\end{eqnarray}
where $\omega_d = [\phi_d,\theta_d,\psi_d]^T \in \mathbb{R}^3$, $\Omega_d = [p_d,q_d,r_d]^T \in \mathbb{R}^3$ are the desired angular position and angular velocity, $k_3,k_4 \in \mathbb{R}$ denote positive control gains. 

Since the quadrotor is an underactuated system with 4 control inputs and 6 outputs and using the differential flatness property, the control input can be obtained from four flat outputs $[x_d,y_d,z_d,\psi_d]^T$ (user-defined). In addition, the desired trajectories of roll ($\phi_d$) and pitch ($\theta_d$) depend on these flat outputs and are defined according to the dynamics (\ref{x1}) and (\ref{x2}) as  \cite{liu2021non,shao2018rise}:
	\begin{eqnarray}
	\phi_d = \sin^{-1} \bigg[\frac{m}{U_1}(U_1\sin(\psi_d)-U_2\cos(\psi_d)) \bigg], \\
	\theta_d = \tan^{-1} \bigg[\frac{1}{U_3} (U_1\cos(\psi_d)+U_2\cos(\psi_d)) \bigg], 
	\end{eqnarray}
	where $F_V$ is the thrust force, which is defined as 
	\begin{equation}
	F_V = m\sqrt{U_1^2+U_2^2+U_3^2}
	\end{equation}
where $U_i = F_V \times R_1(i)$ for $i=1,2,3$ and $R_1(i)$ is the $i^{th}$ row of matrix $R_1$ defined in (\ref{R1}).

By taking the derivative of (\ref{eta2}) and using the definition in (\ref{xi3}) and (\ref{xi4}) the open loop error dynamics in the attitude loop can be expressed as 
\begin{eqnarray}
    % \dot{\eta}_2 = \ddot{\xi}_4 + k_4 \dot{\xi}_4  
    % = \ddot{\Omega}-\ddot{\Omega}^*_{d}+ k_4 \dot{\xi}_4 \label{eta2dot1} \\
    \dot{\eta}_2 = \ddot{\Omega}-\ddot{\Omega}_{d}+k_3(\dot{\xi}_4-k_3\dot{\xi}_3)+k_4\dot{\xi}_4 \label{eta2dot2}
\end{eqnarray}

By using the definition in (\ref{x4}) the tracking error in (\ref{eta2dot2}) is expanded as 
\begin{eqnarray}\label{eta2dot3}
    \dot{\eta}_2 &=& \dot{f}_2(\Omega)+\dot{\tau}_{\Omega}(t)+\dot{\Delta}_\Omega \tau_{\Omega}(t)+\dot{d}_{\Omega}(t)-\ddot{\Omega}_{d} \notag \\ &+&k_3(\dot{\xi}_4-k_3\dot{\xi}_3)+k_4\dot{\xi}_4.
\end{eqnarray}
The error dynamics in (\ref{eta2dot3}) can be expressed as 
\begin{equation}
    \dot{\eta}_2 = \tilde{\Gamma}_2+\Gamma_{2d}+\dot{\tau}_\Omega(t)+\dot{\Delta}_\Omega \tau_\Omega(t)-\xi_4 \label{eta2dot4}
\end{equation}
where the unknown auxiliary functions, $\tilde{\Gamma}_2(t), \Gamma_{2d}(t) \in \mathbb{R}^3$ are defined as \begin{eqnarray}\label{gamma2tilde1}
\tilde{\Gamma}_2(t) &=& \dot{f}_2(\Omega)+k_3(\dot{\xi}_4-k_3\dot{\xi}_3)+k_4\dot{\xi}_4+\xi_4 \\
\Gamma_{1d} &=& \dot{d}_{\Omega}(t)-\ddot{\Omega}_{d}. \label{gamma2d}
\end{eqnarray}
 \begin{assumption}\label{delta2assu}
Approximate model knowledge is available such that $\dot{\Delta}_\Omega(t)$ satisfies%
		\begin{equation} \label{delta2ineq}
		\left\Vert \dot{\Delta}_\Omega(t) \right\Vert _{i\infty }<\delta_2 <1,  
		\end{equation}%
		where $\delta_2 \in \mathbb{R}^{+}$ is a known bounding constant, and $%
		\left\Vert \cdot \right\Vert _{i\infty }$ denotes the induced infinity norm. 
 \end{assumption}
The motivation for the separation of terms in  (\ref{gamma2tilde1}) and (\ref{gamma2d}) are based on the fact that the following inequalities can be developed 
	\begin{equation}\label{gamma2tilde2}
\hspace{-0.2cm}	\Vert \tilde{\Gamma}_2\Vert \leq \rho_2 \left( \left\Vert \mu_2 \right\Vert \right)
	\left\Vert \mu_2\right\Vert , \Vert \Gamma_{2d}\Vert \leq \kappa _{\Gamma_{2d}},
	\Vert \dot{\Gamma}_{2d}\Vert \leq \kappa _{\dot{\Gamma}_{2d}} 
\end{equation}
	where $\kappa _{\Gamma_{2d}}$, $\kappa _{\dot{\Gamma}_{2d}} \in \mathbb{R}^{+}$ are known
	bounding constants; $\rho \left( \cdot \right) $ is a positive, globally
	invertible, non-decreasing function; and $\mu_2(t)\in \mathbb{R}^{9}$ is
	defined as 
	\begin{equation}\label{mu2def}
	\mu_2\left( t\right) \triangleq \left[ 
	\begin{array}{ccc}
	\xi_3^{T}\left( t\right) & \xi_4^{T}\left( t\right) & \eta_2^{T}\left( t\right)%
	\end{array}%
	\right] ^{T}.  
	\end{equation}

\subsection{Closed-loop Error System}\label{closedlooperror}
\subsubsection{Position Loop}
Based on the open-loop error system dynamics in (\ref{eta1dot4}), the control
	term $F_V(t)$ is designed as: 
	\begin{eqnarray}\label{FV}
	 \dot{F}_V(t)=-\alpha_{F} \Vert F_V(t)\Vert sgn(\eta_1)-(\alpha_1+1)\eta_1\notag \\ -\hat{\lambda}_1 sgn(\eta_1), 
	\end{eqnarray}
	where $\alpha_{F}, \alpha_1 \in \mathbb{R}$ are positive, control gains. $\hat{\lambda}_1 \in \mathbb{R}^3$ is a adaptive nonlinear feedback gain, which is defined as 
 \begin{equation}\label{lamda1hatdot}
     \dot{\hat{\lambda}}_{1i} = \beta_1 sgn(\eta_1)\eta_1
 \end{equation}
 where $i=1,2,3$, $\beta_1$ is a positive adaptation gain. 
	\begin{remark}
		Note that the theoretical design and stability analysis in this paper are based on the discontinuous signum function, i.e., \vspace{-0.2cm}
		\begin{eqnarray}
		sgn(\sigma) = \Bigg \{
		\begin{array}{c}
		1 \qquad \sigma > 0 \\
		0 \qquad  \sigma = 0 \\
		-1  \qquad \sigma < 0
		\end{array}
		\qquad \forall \sigma \in \mathbb{R}.
		\end{eqnarray}
	\end{remark}
	After substituting (%
	\ref{FV}) into (\ref{eta1dot4}), the closed-loop error dynamics is obtained as 
	\begin{eqnarray} \label{eta1dot5}	
\dot{\eta}_1 &=& \tilde{\Gamma}_1+\Gamma_{1d}-\alpha_{F} \Vert F_V(t)\Vert sgn(\eta_1)-(\alpha_1+1)\eta_1\notag \\ &-&\hat{\lambda}_1 sgn(\eta_1)+\dot{\Delta}_V F_V(t)-\xi_2
	\end{eqnarray} 
 \subsubsection{Attitude Loop}
Based on the open-loop error system dynamics in (\ref{eta2dot4}), the control
	term $\tau_{\Omega}(t)$ is designed as: 
	\begin{eqnarray}\label{tauomega}
	 \dot{\tau}_{\Omega}(t)=-\alpha_{\Omega} \Vert \tau_{\Omega}(t)\Vert sgn(\eta_2)-(\alpha_2+1)\eta_2\notag \\ -\hat{\lambda}_2 sgn(\eta_2), 
	\end{eqnarray}
	where $\alpha_{\Omega}, \alpha_{2} \in \mathbb{R}$ are positive, control gains. $\hat{\lambda}_2 \in \mathbb{R}^3$ is a adaptive nonlinear feedback gain, which is defined as 
 \begin{equation}\label{lamda2hatdot}
     \dot{\hat{\lambda}}_{2i} = \beta_2 sgn(\eta_2)\eta_2
 \end{equation}
 where $i=1,2,3$; $\beta_2$ is a positive adaptation gain.
 
	After substituting (%
	\ref{tauomega}) into (\ref{eta2dot4}), the closed-loop error dynamics is obtained as 
	\begin{eqnarray} \label{eta2dot5}	
\dot{\eta}_2 &=& \tilde{\Gamma}_2+\Gamma_{2d}-\alpha_{\Omega} \Vert \tau_{\Omega}(t)\Vert sgn(\eta_2)-(\alpha_2+1)\eta_2\notag \\ &-&\hat{\lambda}_2 sgn(\eta_2)+\dot{\Delta}_\Omega \tau_\Omega(t)-\xi_4
	\end{eqnarray} 

\section{Stability Analysis}\label{stabilityanaly}
Before providing the stability analysis of the closed-loop quadrotor system, the following conditions, definition and lemma will be utilized in the proof of \thref{theorem1}. 
\begin{condition}[$\lambda_{ji}$-Gain]\label{lambdagain_condition}
		To facilitate the following stability proof, the adaptive gain $\lambda_{ji}$ for $j=1,2$ and $i=1,2,3$ must satisfy the following condition 
  \begin{eqnarray}
      \lambda_{1i} = \Gamma_{1d}+\frac{1}{k_2} \dot{\Gamma}_{1d} \label{lamda1gaincon}; \qquad 
      \lambda_{2i} = \Gamma_{2d}+\frac{1}{k_4} \dot{\Gamma}_{2d} \label{lamda2gaincon}
  \end{eqnarray}
\end{condition}

\begin{definition}\label{Qdef}
To facilitate the Lyapunov-based proof of \thref{theorem1}, two functions $
Q_1(t)\in \mathbb{R}$ and $
Q_2(t)\in \mathbb{R}$ are defined as 
\begin{eqnarray}
Q_1(t) \triangleq \lambda_1 \left\vert \xi_2(0) \right\vert
-\xi_2^{T}(0) \Gamma_{1d}(0) -\int_{0}^{t} R_1(\upsilon) d\upsilon  \label{Q1} \\
Q_2(t) \triangleq \lambda_2 \left\vert \xi_4(0) \right\vert
-\xi_4^{T}(0) \Gamma_{2d}(0) -\int_{0}^{t} R_2(\upsilon) d\upsilon  \label{Q2}
\end{eqnarray}
where the auxiliary function $R_1(t)\in \mathbb{R}$ and $R_2(t)\in \mathbb{R}$ are defined as 
\begin{eqnarray}
R_1(t)=\eta_1^{T}(t) \big[\Gamma_{1d}(t)-\lambda_1 sgn(\eta_1) \big]  \label{R1} \\
R_2(t)=\eta_2^{T}(t) \big[\Gamma_{2d}(t)-\lambda_2 sgn(\eta_2) \big]  \label{R2}
\end{eqnarray}
\end{definition}
\begin{lemma} \label{lemmaR}
 Provided the sufficient conditions in (\ref{lamda1gaincon})  are satisfied, the auxiliary functions in the translational and rotational loop are defined as 
\begin{eqnarray}
\int_{0}^{t} R_1(\upsilon) d\upsilon \leq \lambda_1 \left\vert \xi_2(0) \right\vert
-\xi_2^{T}(0) \Gamma_{1d}(0)   \label{R1ineq} \\
\int_{0}^{t} R_2(\upsilon) d\upsilon \leq \lambda_2 \left\vert \xi_4(0) \right\vert
-\xi_4^{T}(0) \Gamma_{2d}(0) \label{R2ineq}
\end{eqnarray}
Hence, (\ref{R1ineq}) and (\ref{R2ineq}) can be used to prove that $Q_1(t) \geq 0$ and $Q_1(t) \geq 0$. Proof of Lemma \ref{lemmaR} can be found in \cite{xian2004continuous} and is omitted here for brevity. 
\end{lemma}

\begin{condition}[Control Gain]\label{gain_condition}
		The control gains defined in (\ref{FV}) and (\ref{tauomega}) are selected according to the condition
		\begin{equation} \label{alpha1gaincon}
		\alpha_{1} > \frac{\rho^2(\Vert \mu_1 \Vert)}{4min\{(k_1-\frac{1}{2}),(k_2-\frac{1}{2}),1\}}  
		\end{equation}
		\begin{equation} \label{alpha2gaincon}
		\alpha_{2} > \frac{\rho^2(\Vert \mu_2 \Vert)}{4min\{(k_3-\frac{1}{2}),(k_4-\frac{1}{2}),1\}}  		
		\end{equation}
		\begin{equation}
		\alpha_{F} \geq \delta_1, \qquad   \alpha_{\tau} \geq \delta_2 . \label{gaincon3}
		\end{equation}
  \end{condition}


\begin{theorem}\thlabel{theorem1}
For the quadrotor dynamics described in (\ref{x1})-(\ref{x4}) and for a given sufficiently smooth desired trajectory $[x_d,y_d,z_d,\psi_d]^T$, the robust nonlinear control law given in (\ref{FV}) and (\ref{tauomega}) along with the adaptive parameters updated based on (\ref{lamda1hatdot}) and (\ref{lamda2hatdot}) ensure the tracking error is asymptotically regulated and all the signals remain bounded throughout the closed-loop operation in the sense that 
\begin{equation}
		\Vert \xi(t) \Vert \rightarrow 0 \qquad \mbox{for} \qquad t \geq t_n < \infty,
		\end{equation} 
  where $t_n \in \mathcal{L}_\infty$.
\end{theorem}
\begin{proof}
Let $\mathcal{D}\subset \mathbb{R}^{6n+6+2}$ be a
domain containing $w\left( t\right) =0$, where $w\left( t\right) \in \mathbb{%
R}^{6n+1}$ is defined as 
\begin{equation} 
w\left( t\right) \triangleq \left[ 
\begin{array}{ccccc}
\mu^{T}\left( t\right) & \tilde{\lambda}_1^T& \tilde{\lambda}_2^T & \sqrt{Q_1(t)} & \sqrt{Q_2(t)}
\end{array}
\right] ^{T}  \label{wdef}
\end{equation}%
where $\tilde{\lambda}_i = \hat{\lambda}_i-\lambda_i, (i=1,2)$ in (\ref{wdef}) represent the difference between the estimated parameter and its true value defined in Condition \ref{lambdagain_condition} and $\mu(t) \in \mathbb{R}^{6n}$ is defined as 
		\begin{equation}
		\mu\left( t\right) \triangleq \left[ 
		\begin{array}{cc}
		\mu_1^{T}\left( t\right) & \mu_2^{T}\left( t\right) %
		\end{array}%
		\right] ^{T},  \label{mudef}
		\end{equation}
$Q_1(t)$ and $Q_2(t)$ are defined in Definition \ref{Qdef}.
Let $V\left( w,t\right) :\mathcal{D}\times \lbrack 0,\infty )\rightarrow 
\mathbb{R}$ be a radially unbounded, positive definite function defined as 
\begin{eqnarray}\label{vdef}
V&=&\frac{1}{2}\xi_1^{T}\xi_1+\frac{1}{2}\xi_2^{T}\xi_2+\frac{1}{2}\xi_3^{T}\xi_3+\frac{1}{2}\xi_4^{T}\xi_4+\frac{1}{2}\eta_1^{T}\eta_1 \notag \\ &+&\frac{1}{2}\eta_2^{T}\eta_2+ Q_1+Q_2+\frac{1}{2\beta_1}\tilde{\lambda}_1^T \tilde{\lambda}_1+\frac{1}{2\beta_2}\tilde{\lambda}_2^T \tilde{\lambda}_2  
\end{eqnarray}
which satisfies the inequalities%
\begin{equation}
U_{1}\left( w\right) \leq V\left( w,t\right) \leq U_{2}\left( w\right) ,
\label{U1U2_bnds}
\end{equation}
provided the gain Condition \ref{lambdagain_condition} are satisfied. In (\ref%
{U1U2_bnds}), the continuous positive definite functions $U_{1}\left(
w\right) $, $U_{2}\left( w\right) \in \mathbb{R}$ are defined as%
\begin{equation}
U_{1}\left( w\right) \triangleq \textstyle{\frac{1}{2}}\left\Vert w\right\Vert
^{2},\qquad U_{2}\left( w\right) \triangleq \left\Vert w\right\Vert ^{2}.
\label{U1U2_defs}
\end{equation}%

After taking the time derivative of (\ref{vdef}) and using (\ref{xi1}), (\ref{xi2}), (\ref{xi3}), (\ref{xi4}), (\ref{eta1dot5}), (\ref{eta2dot5}), $\dot{V}$ can be expressed as 
\begin{eqnarray}\label{vdot1}
\dot{V} &=& \xi_1^T[\xi_2-k_1\xi_1]+\xi_2^T[\eta_1-k_2\xi_2]+\xi_3^T[\xi_4-k_3\xi_3] \notag \\ &+&\xi_4^T[\eta_2-k_4\xi_4]+ \eta_1^T[\tilde{\Gamma}_1+\Gamma_{1d}-\alpha_{F} \Vert F_V\Vert sgn(\eta_1)\notag \\ &-&(\alpha_1+1)\eta_1\notag -\hat{\lambda}_1 sgn(\eta_1)+\dot{\Delta}_V F_V(t)-\xi_2] \notag \\ &+& \eta_2^T[\tilde{\Gamma}_2+\Gamma_{2d}-\alpha_{\Omega} \Vert \tau_{\Omega}\Vert sgn(\eta_2)-(\alpha_2+1)\eta_2\notag \\ &-&\hat{\lambda}_2 sgn(\eta_2)+\dot{\Delta}_\Omega \tau_\Omega(t)-\xi_4]+\dot{Q}_1+\dot{Q}_2\notag \\ &+& \frac{1}{\beta_1} \tilde{\lambda}_1^T \dot{\hat{\lambda}}_1+\frac{1}{\beta_2}\tilde{\lambda}_2^T \dot{\hat{\lambda}}_2. 
\end{eqnarray}
By using the bounding inequalities in (\ref{gamma1tilde2}), (\ref{gamma2tilde2}) and the defition of $Q_1,Q_2$ from (\ref{Q1}), (\ref{Q2}), and the definition of $\dot{\hat{\lambda}}_1$, $\dot{\hat{\lambda}}_2$ from (\ref{lamda1hatdot}), (\ref{lamda2hatdot}), the expression in (\ref{vdot1}) can be upper bounded as 
\begin{eqnarray}\label{vdot2}
\dot{V} &\leq& \xi_1^T\xi_2+\xi_3^T\xi_4-k_1 \Vert \xi_1 \Vert^2-k_2 \Vert \xi_2 \Vert^2-k_3 \Vert \xi_3 \Vert^2  \notag \\ &-& k_4 \Vert \xi_4 \Vert^2+ 
 \Vert \eta_1 \Vert \rho(\Vert \mu_1 \Vert) \Vert \mu_1 \Vert+ \Vert \eta_1 \Vert \Vert \Gamma_{1d} \Vert \notag \\ &-&\Vert \eta_1 \Vert (\alpha_{F}) \Vert F_V \Vert-(\alpha_1+1)\Vert \eta_1 \Vert^2 - \Vert \eta_1 \Vert \hat{\lambda}_1 sgn(\eta_1) \notag \\ &+& \delta_1 \Vert \eta_1 \Vert \Vert F_V \Vert + \Vert \eta_2 \Vert \rho(\Vert \mu_2 \Vert) \Vert \mu_2 \Vert+\Vert \eta_2 \Vert \Vert \Gamma_{2d} \Vert\notag \\ &-&\Vert \eta_2 \Vert (\alpha_{\tau}) \Vert \tau_\Omega \Vert -(\alpha_2+1)\Vert \eta_2 \Vert^2- \Vert \eta_2 \Vert  \Vert \hat{\lambda}_2 sgn(\eta_2) \notag \\ &+&\delta_2 \Vert \eta_2 \Vert \Vert \tau_\Omega \Vert - \Vert \eta_1 \Vert \Vert \Gamma_{1d} \Vert + \Vert \eta_1 \Vert \lambda_1 sgn(\eta_1)\notag \\ &-&\Vert \eta_2 \Vert \Vert \Gamma_{2d} \Vert + \Vert \eta_2 \Vert \lambda_2 sgn(\eta_2) + \tilde{\lambda}_1^T sgn(\eta_1)\eta_1 \notag \\ &+& \tilde{\lambda}_2^T sgn(\eta_2)\eta_2.
\end{eqnarray}
		But $\xi_1^T\xi_2$ and $\xi_3^T\xi_4$ can be upper bounded as 
		\begin{equation}\label{e1ineq}
		\xi_1^T\xi_2 \leq \frac{1}{2} \Vert \xi_1 \Vert^2+\frac{1}{2} \Vert \xi_2 \Vert^2 ; \qquad \xi_3^T \xi_4 \leq \frac{1}{2} \Vert \xi_3 \Vert^2+\frac{1}{2} \Vert \xi_4 \Vert^2. 
		\end{equation}
  By using (\ref{e1ineq}), and the gains $\alpha_F, \alpha_\tau, \alpha_1$ and $\alpha_2$ satisfy the gain condition in Condition \ref{gain_condition}, the upper bound in (\ref{vdot2}) can be further simplified as
% \begin{eqnarray}\label{vdot3}
% \dot{V} &\leq& -(k_1-\frac{1}{2}) \Vert \xi_1 \Vert^2 -(k_2-\frac{1}{2}) \Vert \xi_2 \Vert^2 - \Vert \eta_1 \Vert^2- \alpha_1 \Vert \eta_1 \Vert^2 \notag \\ &+& \rho(\Vert \mu_1 \Vert) \Vert \mu_1 \Vert \Vert \eta_1 \Vert -[\alpha_F-\delta_1]\Vert \eta_1 \Vert \Vert F_V \Vert\notag \\&-&(k_3-\frac{1}{2}) \Vert \xi_3 \Vert^2 -(k_4-\frac{1}{2}) \Vert \xi_4 \Vert^2- \Vert \eta_2 \Vert^2 - \alpha_2 \Vert \eta_2 \Vert^2 \notag \\&+& \rho(\Vert \mu_2 \Vert) \Vert \mu_2 \Vert \Vert \eta_2 \Vert -[\alpha_\tau-\delta_2]\Vert \eta_2 \Vert \Vert \tau_\Omega \Vert
%  \end{eqnarray}
 \begin{eqnarray}\label{vdot4}
\dot{V} &\leq& -(k_1-\frac{1}{2}) \Vert \xi_1 \Vert^2 -(k_2-\frac{1}{2}) \Vert \xi_2 \Vert^2 - \Vert \eta_1 \Vert^2  \\ &-&\alpha_1 \bigg [ \Vert \eta_1 \Vert - \frac{\rho(\Vert \mu_1 \Vert)}{2\alpha_1} \Vert \mu_1 \Vert \bigg]^2 + \frac{\rho ^2(\Vert \mu_1 \Vert) }{4\alpha_1} \Vert \mu_1\Vert^2\notag \\ &-&(k_3-\frac{1}{2}) \Vert \xi_3 \Vert^2 -(k_4-\frac{1}{2}) \Vert \xi_4 \Vert^2- \Vert \eta_2 \Vert^2 \notag \\ &-&\alpha_2 \bigg [ \Vert \eta_2 \Vert - \frac{\rho(\Vert \mu_2 \Vert)}{2\alpha_2} \Vert \mu_2 \Vert \bigg]^2 + \frac{\rho ^2(\Vert \mu_2 \Vert) }{4\alpha_2} \Vert \mu_2\Vert^2 \notag 
 \end{eqnarray}
% \begin{eqnarray}\label{vdot5}
% \dot{V} &\leq& - \bigg[ min\{(k_1-\frac{1}{2}),(k_2-\frac{1}{2}),1\} - \frac{\rho ^2(\Vert \mu_1 \Vert) }{4\alpha_1} \bigg] \Vert z_1 \Vert^2 \notag \\  &-& \bigg[ min\{(k_3-\frac{1}{2}),(k_4-\frac{1}{2}),1\} - \frac{\rho ^2(\Vert \mu_2 \Vert) }{4\alpha_2} \bigg] \Vert \mu_2 \Vert^2, \notag 
% \end{eqnarray}
% \begin{equation}\label{vdot6}
% \dot{V} \leq -\bigg[ min\{c_1,c_2\}-\frac{\rho^2(\Vert \mu \Vert)}{4min\{\alpha_1,\alpha_2}\} \bigg] \Vert \mu \Vert^2,
% \end{equation}
\begin{equation}\label{vdot6}
\dot{V} \leq -\bigg[ c_1-\frac{\rho^2(\Vert \mu \Vert)}{4c_2} \bigg] \Vert \mu \Vert^2,
\end{equation}
where $c_1 \triangleq \min\Big[ min\{(k_1-\frac{1}{2}),(k_2-\frac{1}{2}),1\}, min\{(k_3-\frac{1}{2}),(k_4-\frac{1}{2}),1\}\Big]$ and $c_2 \triangleq min\{\alpha_1,\alpha_2\}$. The following expression
can be obtained from (\ref{vdot6})
\begin{equation}\label{vdot7}
\dot{V}\leq -U\left( w\right) ,  
\end{equation}
where $U(w) = c\left\Vert \mu \right\Vert ^{2}$, for some positive
constant $c\in \mathbb{R}$ is a continuous positive semi-definite function
that is defined on the domain%
\begin{equation}
\mathcal{K} \triangleq \left\{ w\left( t\right) \in \mathbb{R}%
^{6n+2}|\left\Vert w\right\Vert \leq \rho ^{-1}\left( 2\sqrt{c_1c_2}\right) \right\} .  \label{Domain_D}
\end{equation}

The expressions in (\ref{U1U2_bnds}) and (\ref{vdot6}), along with (\ref{alpha1gaincon}), (\ref{alpha2gaincon}) can be used to prove
that $V\left( w,t\right) \in \mathcal{L}_{\infty }$ in $\mathcal{K}$; hence, 
$\xi_1(t),\xi_2(t),\xi_3(t),\xi_4(t)$, $\eta_1\left( t\right), \eta_2(t) \in \mathcal{L}_{\infty }$ in $%
\mathcal{K}$. Given that $\xi_1(t),\xi_2(t),\xi_3(t),\xi_4(t)$, $\eta_1\left( t\right), \eta_2(t) \in \mathcal{L}_{\infty }$, a standard linear analysis technique can be used along with (%
\ref{xi2}), (\ref{eta1}), (\ref{xi4}), (\ref{eta2}) to show that $\dot{\xi}_1(t), \dot{\xi}_2(t),\dot{\xi}_3(t),\dot{\xi}_4(t) \in \mathcal{L}_{\infty }$
in $\mathcal{K}$. Since $\xi_1(t),\xi_2(t),\xi_3(t),\xi_4(t)$, $\dot{\xi}_1(t), \dot{\xi}_2(t),\dot{\xi}_3(t),\dot{\xi}_4(t)\in 
\mathcal{L}_{\infty }$, (\ref{xi1}) and (\ref{xi3}) can be used along with the Assumption \ref{destraj} 
 to prove that $P(t)$, $V(t) \in 
\mathcal{L}_{\infty }$ in $\mathcal{K}$. Given that $P(t)$, $V(t) \in 
\mathcal{L}_{\infty }$, (\ref{x1})-(\ref{x4}) along with Assumption \ref{delta1assu} and \ref{delta2assu} to prove
that the control input $F_V(t), \tau_\Omega(t) \in \mathcal{L}_{\infty }$ in $%
\mathcal{K}$. Since $\eta_1\left( t\right), \eta_2(t) \in \mathcal{L}_{\infty }$, can be used to prove that $\dot{F}_V\left( t\right), \dot{\tau}_\Omega \in \mathcal{L}_{\infty
} $ in $\mathcal{K}$. Given that $\eta_1\left( t\right), \eta_2(t)$, $\xi_1(t),\xi_2(t),\xi_3(t),\xi_4(t) \in 
\mathcal{L}_{\infty }$, the bounding inequalities in (\ref{gamma1tilde2}), (\ref{gamma2tilde2}) can be used to prove that $\dot{\eta}_1
\left( t\right), \dot{\eta}_2(t) \in \mathcal{L}_{\infty }$ in $\mathcal{D}$. Since $\dot{\xi}_1(t), \dot{\xi}_2(t),\dot{\xi}_3(t),\dot{\xi}_4(t) $, $\dot{\eta}_1
\left( t\right), \dot{\eta}_2(t)  \in \mathcal{L}_{\infty }$, (\ref{mu1def}) and (\ref{mu2def}) can be used to prove that $\mu\left( t\right) $ is uniformly
continuous in $\mathcal{K}$. Hence, the definitions of $U\left( w\right) $
and $\mu\left( t\right) $ can be used to prove that $U\left( w\right) $ is
uniformly continuous in $\mathcal{K}$.

Let $\mathcal{S}\subset \mathcal{K}$ denote a set defined as follows:%
\begin{equation}
\mathcal{S\triangleq }\left\{ w\left( t\right) \subset \mathcal{K}|U\left(
w\left( t\right) \right) \leq \textstyle{\frac{1}{2}}\left( \rho ^{-1}\left( 2\sqrt{%
c_1c_2}\right) \right) ^{2}\right\} .  \label{Sdef}
\end{equation}%
Theorem $8.4$ of \cite{khalil2015nonlinear} can now be invoked to state that 
\begin{equation*}
c\left\Vert \mu \left( t\right) \right\Vert ^{2}\rightarrow 0\text{\qquad
as\qquad }t\rightarrow \infty \text{\qquad }\forall ~w\left(0\right)
\in \mathcal{S}\text{.}
\end{equation*}%
Based on the definition of $\mu \left( t\right) $, (\ref{Sdef}) can be used to
show that 
\begin{equation}
\left\Vert \xi \left( t\right) \right\Vert \rightarrow 0\text{\qquad as\qquad }%
t\rightarrow \infty \text{\qquad }\forall ~w\left( 0\right) \in \mathcal{%
S}\text{.}  \label{e_Convergence}
\end{equation}%
Hence, asymptotic regulation of the quadrotor states is achieved, provided the initial conditions lie within the set $\mathcal{S}$,
where $\mathcal{S}$ can be made arbitrarily large by increasing the control
gain $\alpha$ - a semi-global result.
\end{proof}

\section{Simulation Results}\label{simresults}
In this section, numerical examples are illustrated by conducting extensive simulations to validate the efficiency of the proposed adaptive modified RISE control algorithm for both the position and attitude loop with varying magnitudes of the model, actuator uncertainties, and disturbances. During the implementation of the control law in the current numerical simulation, the discontinuous signum function is replaced with the continuously differentiable tanh(·) function. This is a standard approximation, which relates to the well-accepted definition of an “equivalent value operator” of a discontinuous function \cite{drakunov1992sliding}.

The reader is referred to \cite{kidambi2020robust} for a detailed description of the physical parameters, uncertainties, and external disturbance acting on the quadrotor. 
% The simulations are performed in MATLAB $R2021a$, on a Intel 11$^{th}$ Gen Core i9 processor with 32GB of RAM. 
The simulation results are summarized in Fig.\ref{Fig1}-\ref{bargraph}. The results show a comparison between an MRISE-based control formulation in \cite{kidambi2020robust} and an adaptive modified RISE formulation presented in this paper. The comparison is for a specific case, with an uncertainty of $ 10\%$ deviation from true values and a disturbance magnitude of 5.

\begin{figure}[!h]
    \centering 
    \includegraphics[width=0.5\textwidth,height = 0.5\textwidth]{Fig1.eps}
    \caption{[Left] Time evolution of the position of the quadrotor, [right] Error between the desired and current state}
	\label{Fig1}
\end{figure}
 A minimum snap trajectory proposed in \cite{mellinger2011minimum} can be utilized to generate feasible trajectories for quadrotors, but in this current work, the desired trajectories are generated using the following definitions 
\begin{eqnarray}\label{des_tra}
\begin{bmatrix}
x_{des},y_{des},z_{des},\psi_{des}
\end{bmatrix}^T = 
\begin{bmatrix}
3\sin(t),3\cos(t),t,\sin(t)
\end{bmatrix}^T.   
\end{eqnarray}  
 % The reader is refereed to the uncertainties and external disturbances considered in the paper that are acting on the model and actuator dynamics of the quadrotor are given as: 
	% \begin{equation}
	% m_{\Delta} = unc*m, \qquad J_{\Delta} = unc*J, \qquad \Delta l = unc*l,  \notag 
	% \end{equation}
	% \begin{equation}
	% K_{F\Delta} =  unc*\bar{K}_F, \qquad  K_{T\Delta} = unc*\bar{K}_T \notag
	% \end{equation}
	% \begin{eqnarray}
	% %\end{align}
	% %\begin{eqnarray}
	% \begin{bmatrix}
	% d_{1}
	% \end{bmatrix} = j* 
	% \begin{bmatrix}
	% \sin(t) \\
	% \cos(t) \\
	% \sin(t)\cos(t)
	% \end{bmatrix}; \notag 
	% \begin{bmatrix}
	% d_{2}
	% \end{bmatrix} = j* 
	% \begin{bmatrix}
	% \sin(0.5t) \\
	% \cos(0.5t) \\
	% \sin(0.5t)\cos(0.5t)
	% \end{bmatrix} \notag  
	% \end{eqnarray}
 % where $\bar{(\cdot)}$ denotes the nominal values and $(\cdot)_{\Delta}$ denotes the quantity of  uncertainty incorporated in the respective parameters and $unc$ is the uncertainty value. The $unc$ is varied from -15$\%$ to 15$\%$. $d_{1}$ and $d_{2}$ are the disturbance magnitudes in the position and attitude loop respectively, where the magnitude of the disturbance $j$ takes the values ranging between 1,$\hdots$, 10. respectively. The disturbance is introduced at $t=15$sec during the simulation.

 The control gains associated with the outer loop in (\ref{FV}) are given as $\alpha_{F}$ = 0.01, $\alpha_1$= 2, $k_1$ = 5, $k_2$= 2.5, and those of the inner loop in (\ref{tauomega}) are given as $\alpha_\tau$ = 0.01, $\alpha_2$= 5, $k_3$= 20, $k_4$ = 9,  respectively. 

 \begin{figure}[h!]
    \centering 
    \includegraphics[width=0.5\textwidth,height = 0.5\textwidth]{Fig2.eps}
    \caption{Time evolution of the error between the desired and measured translational position during the closed loop operation}
    \label{Fig2}
\end{figure}

 \begin{figure}[h!]
    \centering 
    \includegraphics[width=0.5\textwidth,height = 0.5\textwidth]{Fig3.eps}
    \caption{Time evolution of the error between the desired and measured angular position during the closed loop operation}
		\label{Fig3}
\end{figure} 
Fig. \ref{Fig1} shows the positional state $x,y,z, \phi, \theta, \psi$ (blue) of the quadrotor tracking using the control structure presented in this paper, is compared with the quadrotor trajectory (red) using modified RISE-based control in \cite{kidambi2020robust} with the desired (black) trajectory defined in (\ref{des_tra}). The disturbance is introduced at $t=15$sec; the proposed adaptive control structure reacted more robustly thereby reducing the error between the current and desired states. Fig \ref{Fig2} shows the corresponding errors associated with the positional states $x,y,z$ during the closed-loop control operation. Fig. \ref{Fig3} shows the tracking error in the translational states of  the quadrotor $\phi, \theta, \psi$ orientation during the closed-loop. 

Fig. \ref{adapgains} shows the time evolution of the elements of the adaptive parameter estimate
$\hat{\lambda}_1$ and $\hat{\lambda}_2$ during closed-loop controller operation. Fig. \ref{control} shows the evolution of the control magnitudes during closed-loop operation. The magnitude of these control signals closely resembles the experimental results presented
in \cite{jia2022agile} for agile trajectory tracking. Fig. \ref{bargraph} shows 
the bar graph of the mean RMS error in the states $P, V, \omega, \Omega$ over varying uncertainty levels. The results in Figs.\ref{Fig2}, \ref{Fig3} and \ref{bargraph} clearly demonstrate the improvement in the closed-loop performance that is achieved by the proposed adaptive modified RISE-based control formulation that compensates under varying levels of model and actuator uncertainties as well as disturbance magnitudes.
  \begin{figure}[h!]
    \centering 
    \includegraphics[width=0.55\textwidth,height = 0.5\textwidth]{Adaptive_Gains.eps}
    \caption{Time response of the adaptive parameter estimates $\hat{\lambda}_{1i}$ and $\hat{\lambda}_{2i}$ }
		\label{adapgains}
	\end{figure}
   \begin{figure}[h!]
    \centering 
    \includegraphics[width=0.55\textwidth,height = 0.45\textwidth]{Control.eps}
    \caption{Commnaded control inputs during the closed-loop operation}
		\label{control}
	\end{figure}

 \begin{figure}[h!]
    \centering 
    \includegraphics[width=0.55\textwidth,height = 0.45\textwidth]{BarGraph_Uncertainty.eps}
    \caption{A bar graph showing the mean RMS error during the closed-loop operation for the states $P,V,\omega, \Omega$ under varying uncertainty levels}
		\label{bargraph}
	\end{figure}

 \section{conclusion} \label{conclusion}
 In this paper, an adaptive MRISE state feedback controller is proposed that achieves asymptotic trajectory tracking under gyroscopic effects, model, and actuator uncertainties and at the same time mitigates external disturbances acting on the quadrotor. The proposed nonlinear control algorithms adapt to varying magnitudes of uncertainties, disturbances and track trajectories of varying speeds. A rigorous Lyapunov-based analysis is utilized to prove asymptotic trajectory tracking of both the position and attitude loops, where the region of convergence can be made arbitrarily large through judicious control gain selection. Detailed numerical simulations are provided to evaluate the proposed control design. Future
work will focus on evaluating the proposed robust control methods through real-time experiments on a quadrotor.
% \section*{APPENDIX}


% \section*{ACKNOWLEDGMENT}


%%%%%%%%%%%%%%%%%%%%%%%%%%%%%%%%%%%%%%%%%%%%%%%%%%%%%%%%%%%%%%%%%%%%%%%%%%%%%%%%
\bibliographystyle{ieeetr}
\bibliography{Kidambi_CDC23}

\end{document}
