\section{Methods}\label{sec:meth}
To investigate the effect of ET on the evolution of contact binaries, we compute binary evolution models using the stellar evolution code MESA \citep{paxtonModulesExperimentsStellar2011, paxtonModulesExperimentsStellar2013, paxtonModulesExperimentsStellar2015, paxtonModulesExperimentsStellar2018, paxtonModulesExperimentsStellar2019, jermynModulesExperimentsStellar2023}, version r22.11.1.
% , with the \texttt{mesasdk-x86\_64-macos-22.6.2} SDK.
These models are the first massive binary evolution models that include ET in contact layers.
We follow the evolution of binaries from the ZAMS until the least massive component overflows the second Lagrangian point.

\subsection{Physical assumptions in MESA}
The microphysical setup of our stellar evolution models remains mostly the same as in \citetalias{fabryModelingOvercontactBinaries2022}, with some additions from the newer MESA version.
In summary, the nuclear net contains the eight isotopes of $\ce{^{1}H, ^{3}He, ^{4}He, ^{12}C, ^{14}N, ^{16}O, ^{20}Ne}$ and $\ce{^{24}Mg}$, sufficient for the main sequence and nuclear burning rates are taken from the JINA \citep{cyburtJINAREACLIBDatabase2010} and NACRE libraries, with weak interaction rates from \citet{fullerStellarWeakInteraction1985, odaRateTablesWeak1994} and \citet{langankeShellmodelCalculationsStellar2000}.
Plasma screening is included following \citet{chugunovCoulombTunnelingFusion2007} and thermal neutrino losses are computed in \citet{itohNeutrinoEnergyLoss1996}.
For the equation of state, a blend is used between different tables by \citet{saumonEquationStateLowMass1995, timmesAccuracyConsistencySpeed2000, rogersUpdatedExpandedOPAL2002, potekhinThermodynamicFunctionsDense2010} and \citet{jermynSkyeDifferentiableEquation2021}, specified in \citet{jermynModulesExperimentsStellar2023}.
Radiative opacities are also blended from CO-enhanced tables of OPAL opacities \citep{iglesiasRadiativeOpacitiesCarbon1993, iglesiasUpdatedOpalOpacities1996} and \citet{fergusonLowTemperatureOpacities2005} for lower temperatures.
At high temperatures, Compton scattering opacity is from \citet{poutanenRosselandFluxMean2017}, and electron conduction opacities are taken from \citet{cassisiUpdatedElectronconductionOpacities2007} and \citet{blouinNewConductiveOpacities2020}.
In all simulations in this work, we set the metallicity of stars to the solar value $Z_\odot$, where $Z_\odot = 0.0142$, with metal fractions as determined from \citet{asplundChemicalCompositionSun2009}.
\par
Mass loss through winds is accounted for by following the prescription of \citet{brottRotatingMassiveMainsequence2011}.
If the surface hydrogen fraction is $X > 0.7$, the mass loss rate is taken either from \citet{vinkMasslossPredictionsStars2001} for temperatures above the iron bi-stability jump \citep[calibrated also by ][]{vinkMasslossPredictionsStars2001} or the maximum of the rates \citet{vinkMasslossPredictionsStars2001} and \citet{nieuwenhuijzenAtmosphericAccelerationsStability1995} below this temperature.
For $X < 0.4$, the wind prescription of \citet{hamannSpectralAnalysesGalactic1995} is used, although decreased by a factor of ten.
When $0.4 < X < 0.7$, the wind is linearly interpolated between the above results.
We allow part of the wind launched by a star to be accreted by its companion using the Bondi-Hoyle mechanism \citep{bondiMechanismAccretionStars1944} as implemented by \citet{hurleyEvolutionBinaryStars2002}.
\par
For mixing, we use the Ledoux criterion to determine convectively mixed regions, where the mixing length theory of \citet{bohm-vitenseUberWasserstoffkonvektionszoneSternen1958}, in the version described by \citet{coxPrinciplesStellarStructure1968}, is applied with a mixing length parameter of $\alpha = 2$.
The convective core is allowed to overshoot its boundary.
Following the calibration of \citet{brottRotatingMassiveMainsequence2011}, we use a step overshoot where the diffusion coefficient 0.01 pressure scale heights into the convective layer is kept constant out to 0.335 pressure scale heights beyond the boundary. 
Semiconvection is included following the model of \citet{langerSemiconvectiveDiffusionEnergy1983} with a high efficiency of $\alpha_{\rm sc} = 100$ as calibrated by \citet{schootemeijerConstrainingMixingMassive2019}.
We include thermohaline mixing as developed by \citet{kippenhahnTimeScaleThermohaline1980} with an efficiency parameter $\alpha_{\rm th} = 1$.
All other mixing processes, in particular rotational mixing, are ignored in order to isolate as best as possible the effect of energy transfer.
\par
At all times, the rotation of the stars is synchronized to the orbital period, as part of the requirement to use the Roche potential as the deformation geometry (Sect. \ref{ssec:roche}).
Additionally, we artificially diffuse the total angular momentum throughout the interior of the star to enforce solid body rotation.
The treatment of mass transfer (MT) is explained in Sect. \ref{ssec:mt}, while the implementation of energy transfer (ET) is shown in Sect. \ref{ssec:et}.

\subsection{Shellularity and Roche Lobe geometry}\label{ssec:roche}
Since we deal with highly tidally deformed stars, we use the modifications to the stellar structure equations from \citetalias{fabryModelingOvercontactBinaries2022} to incorporate the RL geometry into a one-dimensional (1D) stellar evolution code.
The stars are therefore modeled as hydrostatic structures living in the Roche potential $\Psi$ of a fully synchronized binary \citepalias[see Eq. 19 of][]{fabryModelingOvercontactBinaries2022}.
Furthermore, we assume stellar layers to be shellular.
Shellularity is reached when all intensive quantities, in particular the temperature, pressure and mass density are constant along a stellar layer, and that such a layer coincides with a unique equipotential surface.
\par
It should be emphasized that, in context of 1D models, full shellularity of layers of a (single) star is an assumption, and not a self-consistently modeled feature as of course one needs a 3D structure to explicitly test shellularity of a stellar layer.
However, overflowing layers of components in a contact binary can be tested to be shellular with each other if the temperature, density, etc. at the 1D cells are equal for equal values of Roche potential. 
Given the method we used to split the common envelope of the contact binary (Sect. 2.4 of \citetalias{fabryModelingOvercontactBinaries2022}), the computation of the tidal deformation corrections in \citetalias{fabryModelingOvercontactBinaries2022} based on equipotential surfaces and the usage of the corresponding outer boundary condition (Sect. 4 of \citetalias{fabryModelingOvercontactBinaries2022}), the final ingredient of ET in the common envelope developed here will ensure shellularity of one component with the other, again, defined as having similar cell values of pressure, temperature, etc. for similar values of potential.
\par
In contact, stellar layers are shared between the components and a choice must be made as to what part of the envelope belongs to each component.
To this end we constructed what we call a splitting surface emanating from the Lagrangian point ${\rm L}_1$, by following the gradient $\nabla\Psi$ outward in all directions (see \citetalias{fabryModelingOvercontactBinaries2022}, Sect. 2.4).
With this setup, the volume of both components is uniquely defined, and in \citetalias{fabryModelingOvercontactBinaries2022} we computed integrals in the Roche potential necessary to represent distorted shells in 1D.
The volume equivalent radius then acts as the new independent variable of a stellar shell, and is defined as:
\begin{equation}
    V_\Psi \equiv \frac{4\pi}{3}r_\Psi^3.
\end{equation}
Given a certain splitting strategy, and the requirement of the stellar surface being on a common equipotential, a relation of the form
\begin{equation}\label{eq:contact}
    r_{\Psi, 2} = F(q; r_{\Psi, 1}),
\end{equation}
with $q = M_2 / M_1$, constrains the radii of the overflowing components 1 and 2. 
This is a statement of the contact condition, and is at the basis of Kuiper's paradox (Sect. \ref{sec:ettheory}).
For vertical splitting surfaces through ${\rm L}_1$, \citet{marchantNewRouteMerging2016} found the fit:
\begin{equation}
    \frac{r_{\Psi, 2} - r_{\rm RL, 2}}{r_{\Psi, 2}} \approx q^{-0.52}  \frac{r_{\Psi, 1} - r_{\rm RL, 1}}{r_{\Psi, 1}}, 
\end{equation}
where we have denoted $r_{\rm RL} = r_{\Psi_{{\rm L}_1}}$ the volume equivalent RL radius of a component.
With our setup of the splitting surface however, this approximation is no longer accurate for significantly overflowing shells of mass ratios away from unity.
Therefore, in this work, we evaluate the function $F$ of Eq. \eqref{eq:contact} by interpolating the results of the Roche integrations obtained in \citetalias{fabryModelingOvercontactBinaries2022}.
\par
Retaining the form of the fit of \citet{marchantRoleMassTransfer2021}, for the radius of a component overflowing to its outer Lagrangian point ${\rm L}_{\rm out}$, which is ${\rm L}_2$ for the less massive and ${\rm L}_3$ for the more massive component, we construct a new fit from the integration results of \citetalias{fabryModelingOvercontactBinaries2022}, which is:
\begin{subequations}\label{eq:l2fit}
    \begin{align}
        \frac{r_{L_{\rm out}} - r_{\rm RL}}{r_{\rm RL}} &= \frac{3.3752}{1+\left(\frac{\ln q + 1.0105}{\sigma}\right)^2}\cdot \frac{1}{9.0087+q^{-0.4022}},\\
        \sigma &= \frac{62.9237}{15.9839 + q^{0.2240}}.
    \end{align}
\end{subequations}

This fit has an error smaller than 0.1\% in the range $-7 \leq \log q \leq 7$, compared to 1\% if we used the fit of \citet{marchantRoleMassTransfer2021}.
We use this fit then to determine when the least massive component overflows to ${\rm L}_2$, after which we stop the evolutionary simulation.
\par
% In Appendix \ref{app:convergence}, we test for convergence of our models given 
\subsection{Mass transfer}\label{ssec:mt}
The components of a contact binary are, per definition, in mechanical contact with each other. Thus, stellar material can be exchanged throughout the evolution of such a system.
However, MT cannot occur at an arbitrary rate, since the surface of both components are constrained by the contact condition of Eq. \eqref{eq:contact}.
If this were not satisfied, pressure gradients would arise across the surface of the contact binary, and we expect this would induce strong, horizontal\footnote{Horizontal in the context of this work means perpendicular to the local effective gravity $\vec{g} = \nabla\Psi$. Conversely, vertical means parallel to $\nabla\Psi$.} flows equilibrating the surface.
Thus, in our simulations, we implicitly adjust the MT rate either to keep the donor just below its RL radius in a semidetached configuration, or so the component radii satisfy the contact condition of Eq. \eqref{eq:contact} in a contact configuration \citep[see also][]{marchantNewRouteMerging2016}.
\par

\subsection{Energy transfer}\label{ssec:et}
Not only are the contact layers of both stellar components in mechanical contact, facilitating MT, there is also thermal contact between these layers, allowing for ET.
As elaborated in Sect. \ref{ssec:ctctdis}, we model ET in contact binaries as the transfer of luminosity of one component to the other (Eq. \eqref{eq:etransfer}), at the location of the RL.
It should be mentioned that no further corrections to the ET need to be computed since the splitting surface we use, computed in \citetalias{fabryModelingOvercontactBinaries2022}, is parallel to the local effective gravity and thus the radiative flux.
Therefore, since the regular energy transport by radiation (or by convection) in the interior of either component is vertical, no vertical energy flux crosses over toward the other component.
Our splitting strategy thus nicely decouples horizontal and vertical energy flows and in this approximation, the ET computed here is not affected by the vertical energy transport present in the components.
\par
The stability and performance of numerical simulations are severely impacted by introducing sharp discontinuities at the RL radius, in this case that of luminosity.
To remedy these issues, we use the following approach.
\par
First, for numerical stability over evolutionary steps, we smooth the amount of ET by computing
\begin{equation}
    \tilde{L}_{\rm trans}^{n} = p L_{\rm trans}^{n} + (1-p) \tilde{L}_{\rm trans}^{n-1},
\end{equation}
where $\tilde{L}_{\rm trans}^{n}$ is the smoothed transferred luminosity at evolutionary step $n$, $L_{\rm trans}$ is the luminosity to be transferred as per the Roche geometry (Eq. \eqref{eq:etransfer}), and $p$ is the smoothing factor.
In all simulations including ET, we use a moderate smoothing of $p=0.5$.
\par
Then, we implement the luminosity transfer as a constant extra source of specific heat $\varepsilon_{\rm RL, 1, 2}$ in both stars, occurring in the cells with a radius within 1.00 and 1.01 times its respective RL radius, a choice so as to reflect the estimate of \citet{shuStructureContactBinaries1979} (see Eq. \eqref{eq:etthickness} below and also Appendix \ref{app:et} for further details).
The magnitude of $\varepsilon_{\rm extra, 1, 2}$ is determined from the required luminosity:
\begin{subequations}\label{eq:heats}
\begin{align}
    \varepsilon_{\rm RL, 1} \Delta m_1 &= \tilde{L}_{\rm trans},\\
    \varepsilon_{\rm RL, 2} \Delta m_2 &= -\tilde{L}_{\rm trans},
\end{align}
\end{subequations}
where $\Delta m_{1, 2}$ is the mass of the cells we put the extra heat in.
\par
Since MT is modeled as removing stellar material from the top of the donor and putting it on top of the accretor, during MT the common layers are out of thermal equilibrium due to expansion or compression.
To correct for this effect and regain shellularity, especially at the surface, we transport additional heat in the common envelope according to Eq. \eqref{eq:etransfer}, where now we evaluate $S\langle g\rangle $ at the surface layer of both components instead of the RL.
The constant heat source $\varepsilon_{\rm surf, 1, 2}$ is then computed similarly to Eq. \eqref{eq:heats}, only now $\Delta m$ encompasses all shells from 1.01 times the RL up to the surface.
\par
As a final approximation, we linearly scale the necessary ET for shallow contact systems, defined as when $0 \leq \frac{R-R_{\rm RL}}{R_{\rm RL}} \leq 0.01$ for either star:
\begin{equation}
    \tilde{L}_{\rm trans, shallow} = \tilde{L}_{\rm trans} \frac{\min\left(\frac{R_1 - R_{\rm RL, 1}}{R_{\rm RL, 1}}, \frac{R_2 - R_{\rm RL, 2}}{R_{\rm RL, 2}}\right)}{0.01}.
\end{equation}
\par

We recognize that using our ET prescription in fast MT phases is the weakest element of our simulations.
When the MT timescale is shorter than the thermal timescale, the star falls out of thermal equilibrium, which is characterized by strong (vertical) gradients in the luminosity profile.
During such phases then, the energy budget of the star is redistributed vertically in order to regain as best as possible thermal equilibrium.
Adding to this process the horizontal energy transport in the layers near the RL could induce complex interactions between both energy flows.
Even though we mentioned above the horizontal and vertical energy flows to be decoupled, this is an idealization and the approximation may break down in quickly evolving phases.
Furthermore, while our computation of how much energy is transferred depends only on the geometry and the theorem of von Zeipel, which does not require thermal equilibrium of the layers, hydrodynamical effects break the validity of von Zeipel's theorem \citep{vonzeipelRadiativeEquilibriumRotating1924}, which can be significant near the Lagrangian point ${\rm L}_1$.
\par
