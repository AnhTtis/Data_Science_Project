\section{Energy transfer location} \label{app:et}
Because our ET formalism is implemented explicitly, and the radius of a stellar model can change during the solver iterations, which is also coupled to mass added or subtracted due to MT, a non negligible mismatch between the location of energy input and the RL can occur.
In particular, when mass is removed due to MT, the ET location is then evaluated on the remaining cells, without those cells being adjusted since the previous step.
Therefore, in phases with a high MT rate, it is possible that all mass above the RL is removed in one step (although after the solver performs the integration, the star will expand due to thermal relaxation).
This means that the location of ET in terms of radius is rather poorly defined.
To mitigate this, instead of choosing directly to put energy in cells with radii
\begin{equation}
    r \in [1.00, 1.01]R_{\rm RL},
\end{equation}
we opt to do following.
At the beginning of the evolutionary step, we evaluate:
\begin{equation}
    x_{\rm depth} = 1-m(r=R_{\rm RL})/M
\end{equation}
as the depth fraction of the current RL in mass coordinate.
Then during the next integration, the location of energy input is cells
\begin{equation}
    r \in [1.00, 1.01]r(x=x_{\rm depth}).
\end{equation}
This treatment mitigates mismatches in ET locations introduced by MT during evolutionary steps to the degree that the depth fraction changes from step to step.