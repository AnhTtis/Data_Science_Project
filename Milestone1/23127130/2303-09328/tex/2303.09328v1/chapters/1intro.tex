\section{Introduction}
Massive stars $(M_{\rm initial} \gtrsim 8 M_\odot)$ are predominantly found in binaries that interact during their lives \citep{sanaBinaryInteractionDominates2012}, which means that the impact of binarity on the evolution of massive stars must be taken into account in stellar structure and evolution models.
This is supported by population synthesis calculations \citep{vanbeverenWROtypeStar1998, deminkIncidenceStellarMergers2014}.
An extreme form of binary interaction happens when the stellar companions enter a contact configuration, where both stars overflow their respective Roche lobes (RLs).
The stars connect at the so-called neck and form a peanut like shape, as determined by the geometry of the Roche potential.
\par
Understanding the evolution of contact systems is of great importance in the study of, among others, stellar mergers, as these are all preluded by an (unstable) contact phase.
Mergers could be the potential source of strong magnetic fields observed in about 7\% of massive stars \citep{ferrarioOriginMagnetismUpper2009, wickramasingheMostMagneticStars2014, fossatiFieldsOBStars2015, schneiderRejuvenationStellarMergers2016, grunhutMiMeSSurveyMagnetism2017b, schneiderStellarMergersOrigin2019, frost2023}.
Furthermore, close binaries are expected to rotate rapidly due to the high Keplerian velocity and quick tidal synchronization.
This might lead to the physical conditions necessary to create long-duration gamma-ray bursts and magnetar-driven superluminous supernovae \citep{yoonEvolutionRapidlyRotating2005, yoonSingleStarProgenitors2006, aguilera-denaRelatedProgenitorModels2018, aguilera-denaPrecollapsePropertiesSuperluminous2020}.
The high rotational velocity could also lead to chemically homogeneous evolution, which is a proposed channel to produce gravitational wave progenitors \citep{marchantNewRouteMerging2016, mandelMergingBinaryBlack2016}.
\par
In the past, the low-mass contact systems, called W Ursae Majoris (W UMa) systems, have been extensively studied.
These systems have an orbital period of less than a day with a total mass on the order of $1 M_\odot$.
Observational campaigns from \citet{eggenContactBinariesII1967} and \citet{binnendijkOrbitalElementsUrsae1970} identified the first of these stars and found that a significant fraction had mass ratios away from unity.
Now, thousands of W UMa systems have been identified thanks to large surveys like the OGLE survey \citep{szymanskiContactBinariesOGLEI2001}.
Observations of massive contact systems however are much rarer, but fortunately, two dozen or so systems have been identified from the VLT Flames Tarantula Survey (VFTS) \citep{evansVLTFLAMESTarantulaSurvey2011}, the MACHO survey \citep{alcockMACHOProjectLMC1997} and OGLE in the Magellanic clouds, as well as studies of galactic targets \citep[e.g.][]{lorenzoMYCamelopardalisVery2014, lorenzoMassiveMultipleSystem2017, yangComprehensiveStudyThree2019}.
\par
Also on the modeling side, most work in the literature is dedicated to understanding the W UMa systems as opposed to massive contact systems.
\citet{kuiperInterpretationLyraeOther1941} initiated theoretical considerations of contact binaries with the argument that due to the stark difference between RL radii and the mass-radius relation of main sequence stars, contact systems with mass ratio away from unity cannot be stable.
The argument of Kuiper transformed into what is now referred to as Kuiper's paradox once a multitude of contact systems were observed with unequal mass components.
However, it should be clear immediately that this paradox is only apparent as the mass-radius relation of single main sequence stars need not apply for stars in a contact configuration.
\par

\citet{lucyStructureContactBinaries1968} first considered energy transfer (ET) in common convective envelopes of stellar components, and computed the first approximated structure models for W UMa stars.
Later, \citet{lucyUrsaeMajorisSystems1976, flanneryCyclicThermalInstability1976, hazlehurstDissipationFactorContact1985} and \citet{kahlerStructureEquationsContact1989} calculated models that are out of thermal equilibrium and show cyclic behavior.
\citet{shuStructureContactBinaries1976, shuStructureContactBinaries1979} and \citet{lubowStructureContactBinaries1977, lubowStructureContactBinaries1979} (collectively \citetalias{shuStructureContactBinaries1976}) constructed models of contact binaries where they dropped the requirement of a continuous structure at the layer coinciding with the equipotential surface of the first Lagrangian point ($L_1$).
These models resolve Kuiper's paradox regardless of the thermal structure of the envelopes, be they radiative or convective.
Despite much criticism \citep[see e.g.][]{hazlehurstEquilibriumContactBinary1993, kahlerStructureEquationsContact1989}, this is the simplest model of ET in radiative envelopes present in the literature.
\par

Accurately modeling massive, long-lived contact systems has been attempted in the past.
\citet{marchantNewRouteMerging2016} computed detailed evolution model grids of massive binaries with initial periods down to $\SI{0.5}{d}$, which includes the regime of contact binaries at the zero age main sequence.
Those models however did not include energy transfer between contact components, as the models concerned binaries of mass ratio close to unity $M_2/M_1 = \numrange{0.8}{1}$ and the effect was thought to be small.
\par
\citet{senDetailedModelsInteracting2022} used the models of \citet{marchantImpactTidesMass2016} to study the semi-detached Algol systems, although these models also included contact phases.
\citet{menonDetailedEvolutionaryModels2021} extended these grids by computing models with initial mass ratios down to $M_2/M_1 = 0.6$ with the study of massive, long-lived contact binaries in mind.
They found, also without including energy transfer in contact phases, a strong correlation between observed mass ratio and period in contact systems, broadly in agreement with observation.
However, the mass ratio distribution they derive is heavily skewed toward values close to unity, which is not supported observationally.
They suggest that including energy transfer in contact phases of unequal mass components could alleviate this discrepancy.
\par

% Since we lose spherical symmetry by rotation and even axial symmetry by the presence of a companion, calculating models of binary stars is an arduous task, as the stellar structure and evolution equations should in principle be solved fully in three dimensions (3D).
% Therefore, simplifying assumptions are made in order to reduce the problem back to 1D, which current computational resources are capable of handling.
% For example, rotation can be modeled using the methodology developed by \citet{kippenhahnSimpleMethodSolution1970} and \citet{endalEvolutionRotatingStars1976}. 
% In this method, correction factors enter in the stellar structure equations of hydrostatic equilibrium and radiative energy transport, and are computed from the deformed geometry of the stellar shells.
% This method is developed further for tidal correction in a synchronized Roche binary, and is the subject of the first paper in this series \citep{fabryModelingcontactBinaries2022}.
% \par

% The theory of mass and angular momentum transfer is well developed and prescriptions exist to apply them in stellar evolution calculations.
% For example, in RL overflowing donor stars, the method of \citet{ritterTurningMassTransfer1988} and \citet{kolbComparativeStudyEvolution1990} calculate the mass transfer (MT) rate by taking into account the conditions of the donor star near $L_1$.
% This method was recently extended for heavily overflowing donor stars (such as in contact systems) by \citet{marchantRoleMassTransfer2021}.
% However, in addition to mechanical contact, the common layers of the components of a contact binary are in thermal contact as well, which can give rise to an energy flux from one to the other.
% This effect is ignored in evolution models, like those of \citet{menonDetailedEvolutionaryModels2021}.
% \par

In this work, we apply the ET model of \citetalias{shuStructureContactBinaries1976} in common stellar layers to modern stellar structure models, to further advance our evolutionary modeling of massive contact binaries.
In Sect. \ref{sec:ettheory}, we describe and discuss the theory of ET.
Section \ref{sec:meth} describes the physical setup of the stellar evolution code, along with our ET implementation.
In Sect. \ref{sec:models}, we compare models computed with and without ET in contact layers and discuss the difference in the observable properties of the models.
Lastly, in Sect. \ref{sec:conc}, we provide our concluding remarks.
