\section{Stellar models}\label{sec:models}

\subsection{Detailed example of ET evolution}\label{ssec:detailed}
Here we detail the evolution of a $25 M_\odot$ primary and a $20 M_\odot$ secondary (so $q_{\rm ini}=0.8$) in an initial orbit of 1.4 days.
This initial configuration is chosen so that, at the onset of the long lasting contact phase, this model mimics the observed galactic contact binary V382 Cyg studied in \citet{abdul-masihConstrainingOvercontactPhase2021}.
We simulate both cases where ET, as implemented in Sect. \ref{ssec:et}, is included and excluded.
\par
In Fig. \ref{fig:hrd}, we show the Hertzsprung-Russel diagram (HRD) broken up into four distinct phases of the evolution, while in Fig. \ref{fig:mdot}, we show the MT rate and mass ratio evolution.
First, we have the pre-interaction phase (leftmost column) followed by a short lived, fast case A MT phase from the primary to the secondary when the donor experiences RL overflow (second column of Fig. \ref{fig:hrd}).
During this MT phase, the components briefly enter into contact, as the accretor falls highly out of thermal equilibrium and swells beyond ${\rm L}_1$.
After thermal relaxation of the accretor, the system detaches, after which slow case A MT is initiated (third column of Fig. \ref{fig:hrd}), still from the primary to the secondary, although the mass ratio has inverted by this point.
This phase of nuclear timescale MT occurs in a semi-detached configuration and is referred to as an Algol phase.
Finally, the evolution of the now more massive secondary catches up to that of the stripped primary, contact is engaged again and MT inverts (rightmost column of Fig. \ref{fig:hrd}), moving material from the secondary back to the primary (however see below).
This contact configuration evolves on the nuclear timescale and ends when ${\rm L}_2$ overflow occurs.

\begin{figure*}
    \centering
    \includegraphics[width=1.5\columnwidth]{img/hrd.png}
    \caption{HRD of a 25 + 20 $M_\odot$ binary with an initial period of $1.4\si{d}$. Red lines indicate a simulation with the inclusion of ET, while blue lines ignore ET. Moving left to right, the columns highlight four subsequent phases in the evolution (see Sect. \ref{ssec:detailed} for details). Contact phases are highlighted with thick outlines.}
    \label{fig:hrd}
\end{figure*}
\begin{figure}
    \centering
    \includegraphics{img/mdot.png}
    \caption{MT rate (top panel) and mass ratio (bottom panel) evolution of the 25 + 20 $M_\odot$ binary studied in Sect. \ref{ssec:detailed}. Contact phases are highlighted with thick outlines.}
    \label{fig:mdot}
\end{figure}
\par

Qualitatively, the evolution of this system is similar with and without inclusion of ET.
Up to and including the slow case A MT phase, the evolution of this system is nearly identical. 
One exception is the different HRD tracks during the fast MT case.
In the second column of Fig. \ref{fig:hrd}, the red tracks exhibit a `hook', while the blue tracks do not.
This is a direct consequence of ET in the short contact phase changing the outer structure of the components.
The secondary donates energy to the primary and cools a bit, jumping to the right of the HRD (and vice-versa for the primary).
The point at which the primary and secondary tracks touch (at roughly $\log L/L_\odot=4.8$ and $\log T_{\rm eff}/K=4.545$) corresponds to the situation where $M_1 = M_2$.
This is consistent with the theory since we expect that for $M_1=M_2$ we have similar luminosities per Eq. \eqref{eq:contactML} and also similar effective temperatures following the law of Stefan-Boltzmann.
When contact is disengaged, the secondary (at this point more massive than the primary) jumps back to the left as the energy sink to the primary disappears.
Given the timescale of this MT phase however, it is unlikely to observe contact binaries in this phase and therefore constrain the effects of ET.
\par 
Still, we observe quantitative differences between the energy transferring and non energy transferring models during the final, inverted MT phase, which is long-lived and therefore more likely to be observed.
In particular, from the $q(t)$ curves in the bottom panel of Fig, \ref{fig:mdot}, we see that the ET model spends more time at mass ratios $q \gtrsim 1.4$ versus the no ET model. 
Furthermore, the no ET model monotonically approaches a mass ratio of unity without further inverting the mass ratio or the MT direction.
In contrast, the ET model experiences two inversions of the mass ratio at around $\tau=3.2$ and $\SI{4.0}{\mega yr}$ and the MT is inverted at around $\tau=\SI{3.3}{\mega yr}$.
\par
In Fig. \ref{fig:dtdq}, in the top panel we show the differential duration of the long-lived contact phase, plotted against $\bar{q} = \min\left(\frac{M_1}{M_2}, \frac{M_2}{M_1}\right)$\footnote{We use $\bar{q}$ here since in an observed system, one has no a priori information on what the most massive component is.}, while the cumulative duration is shown in the bottom panel.
The top panel shows three distinct peaks in the red $\bar{q}$ distribution, two of which correspond to the periods where the ET model is inverting its MT direction (at $q\approx1.4$ and $q\approx0.85$) and the last corresponds to a mass ratio close unity $\bar{q}\approx0.97$ at the end of the simulation.
Comparing the two former peaks to the no ET curve at those $\bar{q}$ means that the ET model spends more time at these mass ratios than the no ET model.
As a cumulative distribution, from the bottom panel of Fig. \ref{fig:dtdq}, we deduce that it is about four times more likely to observe the ET model in a configuration more extreme than $\bar{q} = 0.8$ than the no ET model, even though the total length of the contact phase is similar to within \SI{0.05}{\mega yr}.

\begin{figure}
    \centering
    \includegraphics{img/dtdq.png}
    \caption{Top panel: Differential duration of contact (in bins of $d\bar{q}=0.005$) as function of observed mass ratio $\bar{q} = \min\left(\frac{M_1}{M_2}, \frac{M_2}{M_1}\right)$. Bottom panel: Cumulative duration of contact in configurations more extreme than $\bar{q}$.}
    \label{fig:dtdq}
\end{figure}
\par

When comparing the profiles of the components in both models, we note a considerable difference in the structure of the outer layers of the components.
In Fig. \ref{fig:profiles}, the temperature, density and luminosity profiles of the models at the start of inverted MT phase ($\tau=\SI{2.7}{\mega yr}$) are shown.
The independent coordinate is chosen to be $\tilde{r} = \frac{r - R_{\rm RL}}{R  - R_{\rm RL}}$, so as to map the overflowing layers to the interval [0, 1].
At this point in the evolution, in both cases of ET and no ET, the stars are in a contact configuration and the primary is stripped to $M_1 = 18.3M_\odot$, while the secondary accreted mass to $M_2 = 26.4M_\odot$.
Note that these masses are close to the component masses \citet{abdul-masihConstrainingOvercontactPhase2021} found for V382 Cyg.
We see that in the system without ET, the overflowing shells do not satisfy our definition of shellularity since the temperature and density profiles of the primary and secondary do not match for constant Roche potential (which approximately corresponds to constant $\tilde{r}$). 
The ET models however match significantly better, as seen in the left column of Fig. \ref{fig:profiles} by the joining of the temperature and density profiles of the primary and secondary.
In particular, at the surface, the density of the ET components agree to within $0.2\%$, while the temperature to within $0.5\%$, whereas the surface properties of the no ET components vary more than $10\%$ (as expected for models of differing mass).
\begin{figure}
    \centering 
    \includegraphics{img/rhot.png}
    \caption{Profiles of outer layers of the binary components at the onset of inverse MT. It shows (top to bottom) the temperature, density and luminosity profiles for both components in the ET (red, left column) and no ET (blue, right column) cases, all as function of the scaled radius coordinate $\tilde{r} = \frac{r - R_{\rm RL}}{R  - R_{\rm RL}}$. The gray vertical lines show the location of the RL.}
    \label{fig:profiles}
\end{figure}
\par

An important shortcoming of our model is the assumed thickness of the ET layer.
\citet{shuStructureContactBinaries1979} argue that the thickness $d$ of the ET layer is on the order of:
\begin{equation}\label{eq:etthickness}
    \frac{d}{a} \sim \delta^{0.4},\quad \delta = \frac{L_{\rm trans}/a^2}{\rho h c_s}
\end{equation}
with $a$ the binary separation, and $\rho$, $h$ and $c_s$ the density, specific enthalpy and local sound speed evaluated at the RL, respectively.
\citet{lubowStructureContactBinaries1979} compute this number to be $d/a \sim 10^{-2}$ for stars of masses around 4-8$M_\odot$ which justifies that \citetalias{shuStructureContactBinaries1976} modeled the layer as a discontinuity in the stellar profile, located at the RL radius.
\par
However, direct computation of Eq. \eqref{eq:etthickness} for our $25 + 20 M_\odot$ model show that this estimation breaks down for higher masses, see the red line in Fig. \ref{fig:deltashu}.
This suggests that the energy redistribution flow, modeled as a discontinuity by \citetalias{shuStructureContactBinaries1976}, is not sufficient in these higher mass stars.
Except when the binary has reached considerable overflow, where $d/a \lesssim 10^{-2}$, we find that the thickness needed can be a significant fraction of the binary separation, even surpassing it in the early stages of the contact phase.
\par
Another way to compute an upper limit on the thickness of the ET layer is to use Bernouilli's equation.
If we consider fluid motion from far away from ${\rm L}_1$ on the primary star (location $i$) toward ${\rm L}_1$ (location $f$), we have
\begin{equation}
    \frac{1}{2}v_f^2 + \int_i^f\frac{dP}{\rho} = 0,
\end{equation}
where we have already canceled the potential terms $\Psi_i, \Psi_f$ since we move along an equipotential surface, and the initial velocity $v_i$ is assumed to be negligible.
Making then the estimation:
\begin{equation}
    \int_i^f\frac{dP}{\rho} \approx \left(\frac{1}{\rho_f}+\frac{1}{\rho_i}\right)\left(P_f-P_i\right),
\end{equation}
and assuming that the transferred energy through the binary neck of width $b$ by a mass flow $\dot{M} = \rho_i v_f b^2$ is
\begin{equation}
    L_{\rm trans} = \rho_i v_f b^2 (c_{p, 1} + c_{p, 2}) (T_f - T_i),
\end{equation}
we compute for the minimal thickness of the ET layer:
\begin{equation}\label{eq:width}
    b \approx \sqrt{\frac{L_{\rm trans}}{\rho_i (c_{p, 1} + c_{p, 2}) (T_f-T_i) \sqrt{\left(\frac{1}{\rho_f}+\frac{1}{\rho_i}\right)\left(P_f-P_i\right)}}}.
\end{equation}
Finally the thickness of the layer at the neck is related to the thickness far away from ${\rm L}_1$ by
\begin{equation}\label{eq:thickbernoui}
    \frac{d}{a} \approx \left(\frac{b}{a}\right)^2,
\end{equation}
since the Roche potential varies quadratically near ${\rm L}_1$ and linearly elsewhere.
Equation \eqref{eq:width} gives a lower limit on the width of ET layer so that a balanced mass flow $\dot{M}$ in the contact binary can carry a to be transferred luminosity $L_{\rm trans}$.
Conversely, it can be interpreted as $L_{\rm trans}$ being the maximal luminosity the mass flow can carry in a layer of fixed width $b$. 
\par
We plot also the Bernouilli computed thickness of Eq. \eqref{eq:thickbernoui} in Fig. \ref{fig:deltashu}.
Similarly, we see that the required thickness $d/a$ is much larger than what the radius of the primary star allows room for.
Only at later times, when the mass ratio has equilibrated and deeper contact is engaged is the estimated width smaller than the overflow rate of $R-R_{\rm RL}$ of the primary.

\begin{figure}
    \centering
    \includegraphics{img/deltashu.png}
    \caption{Thickness of the ET layer $d$ with respect to the binary separation $a$ as function of the radius of the primary star during the long lived contact phase. The gray line gives the physical size of the overflowing layers, and corresponds to the maximal width the ET layer can assume.}
    \label{fig:deltashu}
\end{figure}
\par

\subsection{Mass vs luminosity ratios}
During the nuclear timescale, inverted MT phase, contact is engaged so that our ET scheme acts to move luminosity from one component to the other. 
As mentioned in Sect. \ref{sec:ettheory}, we expect the luminosity ratio of contact binaries to follow the mass ratio $L \propto M$, as opposed to detached stars following a single star mass-luminosity relation $L \propto M^\alpha$ with $\alpha \simeq \numrange{2}{3}$.
Figure \ref{fig:ML} shows the evolution of the luminosity ratio as function of the mass ratio during the slow MT phase of the $25 + 20 M_\odot$ explored in Sect. \ref{ssec:detailed}.
In this graph, the models evolve from the top right at $q\approx 1.4$ to near equal mass ratio on the left.
We see that the model not including ET follow closely a $q^{2.2}$ relation, appropriate for single stars in the mass range of $\numrange{10}{30}M_\odot$.
As the models including ET engage into deep contact however, their mass-luminosity ratio changes drastically from the $q^{2.2}$ line in near contact to the $q^1$ relation in full contact.
\par
Overplotted on Fig. \ref{fig:ML} are measurements of several observed massive contact binaries of \citet{abdul-masihConstrainingOvercontactPhase2021, yangComprehensiveStudyThree2019} and \citet{lorenzoMYCamelopardalisVery2014} \citep[see also Fig. 2 of][]{langerOpenQuestionsMassive2022}.
Curiously, the luminosity ratios from \citet{abdul-masihConstrainingOvercontactPhase2021}, although following an $L \propto M$ trend, are offset to lower $L_2/L_1$ than predicted.
Either the luminosity of the primary is overestimated, or the uncertainties are underestimated.
The systems included from \citet{mahyTarantulaMassiveBinary2020} were categorized as `uncertain configurations' since the measurement of the radius was consistent with being above as well as below the RL.
In context of the mass-luminosity relation however, we expect that the system from \citet{mahyTarantulaMassiveBinary2020} at $q\approx 1.3$ (VFTS 563) is a true contact system, while the one at $q\approx 1.2$ (VFTS 217) is not (although within 1$\sigma$ it could be either).

\begin{figure}
    \centering
    \includegraphics{img/lumratio.png}
    \caption{Luminosity ratio versus mass ratio during the nuclear timescale MT of the $19 + 14 M_\odot$ systems of Sect. \ref{ssec:detailed}.
    Overplotted are measurements from observed massive (near-)contact systems from \citet{abdul-masihConstrainingOvercontactPhase2021}, \citet{yangComprehensiveStudyThree2019} and \citet{lorenzoMYCamelopardalisVery2014}. The systems from \citet{mahyTarantulaMassiveBinary2020} were classified as `uncertain configurations'.}
    \label{fig:ML}
\end{figure}
