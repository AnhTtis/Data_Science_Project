\section{Energy transfer in fast mass transfer phases}
Given the caveat stated in Sect. \ref{ssec:et}, we address whether including ET only when the star is in thermal equilibrium changes the evolution.
To test this, we use the setup for the $19+14M_\odot$ binary as described above, but require for ET to occur the additional condition that:
\begin{equation}\label{eq:slowET}
    \left|\frac{M}{\dot{M}}\right|_{\rm donor} > 10 \tau_{\rm thermal, donor},
\end{equation}
meaning that the mass transfer timescale is at least an order of magnitude larger than the thermal timescale of the donor star.
This will shut of our ET implementation during the fast case A MT phase, while retain it during the slow case A phase.
Figure \ref{fig:slowET} shows the MT rate and mass ratio evolution from the point of RLOF to $L_2$ overflow.
By comparing the ET track with the `slow ET' track, which requires the condition of Eq. \eqref{eq:slowET}, we note that including ET during fast MT phases only slightly changes the evolution of this binary model.
For full ET, the final mass ratio after fast case A is slightly higher, which delays reattachment of the primary, but the slow evolving contact phase is nearly unchanged.
\begin{figure}
    \centering
    \includegraphics{img/slowET.png}
    \caption{Mass transfer rate and mass ratio evolution of the $19+14M_\odot$ binary studied in Sect. \ref{ssec:detailed}. For the purple line labeled `slow ET', we required the additional condition of Eq. \eqref{eq:slowET} for ET to occur.}
    \label{fig:slowET}
\end{figure}
\par