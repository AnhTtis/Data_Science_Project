\section{Theory of energy transfer}\label{sec:ettheory}

\subsection{Simple considerations}
The theoretical modeling of contact stars started with \citet{kuiperInterpretationLyraeOther1941}, who stated that stable contact systems of uniform composition with unequal masses cannot exist as a result of differing mass-radius relationships. 
For the galactic, zero age main sequence (ZAMS) models of \citet{brottRotatingMassiveMainsequence2011}, we derive the mass radius relation of single stars to be approximately
\begin{equation}\label{eq:massradius}
    \left.\frac{R_2}{R_1}\right|_{\rm ZAMS} = \left(\frac{M_2}{M_1}\right)^{0.57} = q^{0.57},
\end{equation}
where we defined $q\equiv M_2/M_1$.
However, the condition of contact in a binary following the Roche geometry constrains the surface of the stars to same equipotential, which leads to
\begin{equation}\label{eq:contactapprox}
    \left.\frac{R_2}{R_1}\right|_{\rm Roche} \approx \left(\frac{M_2}{M_1}\right)^{0.46} = q^{0.46},
\end{equation}
for stars not overflowing their RL too much.
Clearly, both these equations \eqref{eq:massradius}-\eqref{eq:contactapprox} cannot be satisfied simultaneously unless $M_1 = M_2$.
However, while the contact condition needs to be satisfied from dynamical arguments, the stellar structure need not, a priori, follow a single star model.
Even though the dense stellar core will be largely unperturbed due to the companion, the outer layers of contact components are highly distorted, causing different total radii with respect to spherical models, as seen explicitly in \citet{fabryModelingOvercontactBinaries2022} (henceforth \citetalias{fabryModelingOvercontactBinaries2022}).
This result did not yet take energy transfer into account, and we expect further changes under the consideration that a hotter gas under similar pressure takes up more volume.
Therefore, the ZAMS mass-radius relation of Eq. \eqref{eq:massradius} is not expected to be satisfied under general contact conditions, and so Kuiper's paradox can be resolved by providing alternative stellar models that have a mass-radius relation closer to the contact condition of Eq. \eqref{eq:contactapprox}.
\par
Following the Roche lobe geometry, \citet{lucyStructureContactBinaries1968} finds the ratio of the surface areas $S_2/S_1$ of two stars in contact to be proportional to $(M_2/M_1)^{\beta}$, with $\beta=0.96$. 
Using the approximation $\beta \simeq 1$, combined with von Zeipel's theorem of gravity darkening $T_{\rm eff}^4 \propto g$ \citet{vonzeipelRadiativeEquilibriumRotating1924}, and the Stefan-Boltzmann law $L \propto S T_{\rm eff}^4$, results in the simple expectation that in contact binaries, the luminosity ratio follows the mass ratio \citep{lucyStructureContactBinaries1968, tassoulStellarRotation2000}:
\begin{equation}\label{eq:contactML}
    \frac{L_2}{L_1} \simeq \frac{M_2}{M_1} = q.
\end{equation}
Single main sequence stars, on the other hand, follow the well known mass-luminosity relation 
\begin{equation}\label{eq:singleML}
    \frac{L_{\rm s, 2}}{L_{\rm s, 1}} \simeq \left(\frac{M_2}{M_1}\right)^{\alpha} = q^\alpha,
\end{equation} 
with $\alpha\simeq\numrange{2}{3}$ for the upper main sequence \citep{kohlerEvolutionRotatingVery2015, grafenerEddingtonFactorKey2011}, and we use the symbol $\simeq$ to denote these relations are approximations to simple power laws.
We see that this leads to a difference between the luminosity of a single star $L_{s,1}$ and the luminosity $L_1$ of a star of the same mass in a contact binary of
\begin{equation}\label{eq:dL1}
    \Delta L_1 = L_1 - L_{1,s} = -f L_1,
\end{equation}
and for the companion
\begin{equation}\label{eq:dL2}
    \Delta L_2 = L_2 - L_{2,s} = f L_1 \simeq \frac{f}{q} L_2,
\end{equation}
since $L_2\simeq q L_1$ and we require $\Delta L_1 + \Delta L_2 = 0$ to conserve energy.
This defines $f$ as
\begin{equation}\label{eq:dLf}
f\simeq\frac{q-q^{\alpha}}{1+q^{\alpha}}.
\end{equation}
Therefore, the two stars in a contact binary can fulfill the single star mass-luminosity relation of Eq. \eqref{eq:singleML} in their cores and the contact binary mass-luminosity condition of Eq. \eqref{eq:contactML} at their surfaces if the amount of energy per time given by Eqs. \eqref{eq:dL1} and \eqref{eq:dL2} are transferred from the more massive to the less massive star in their common envelope.

\subsection{Models of energy transfer}
The general solution to Kuiper's paradox is to consider detailed stellar models with the inclusion of energy transfer (ET) between the binary components.
There exist several models of ET in the literature.
\par
\citet{lucyStructureContactBinaries1968} and \citet{biermannModelsContactBinaries1972} provided a first solution by adjusting the adiabatic constants of convective envelopes in contact components.
However, this is an unsatisfactory solution, since this setup requires the stars to be burning hydrogen through different nuclear chains or cycles in the case of \citet{lucyStructureContactBinaries1968}, or that the models exhibit inaccurate light curves in \citet{biermannModelsContactBinaries1972}.
Other models like those of \citet{lucyUrsaeMajorisSystems1976} or \citet{ flanneryCyclicThermalInstability1976} relaxed the requirement of thermal equilibrium and constructed models of W UMa stars that exhibited thermal cycles. 
\citet{kahlerStructureEquationsContact1989} presented a detailed model that required turbulent motions in the common envelope to explain early type (radiative) W UMa binaries.
\par
Meanwhile, \citetalias{shuStructureContactBinaries1976} presented the contact discontinuity model of contact binaries, by relaxing the requirement of continuous structural quantities across the RL.
This is the only model that treats the common envelope as a single volume of the binary structure, at the price of hiding a heat engine in a very thin region around the RL.
One peculiar feature is that this model necessitated a temperature inversion at the RL layer in one of the components as otherwise they would not be able to construct thermally stable contact models of uniform composition (in order to model binaries at zero age). 
This feature has received criticism in that the proposed heat engine violates the second law of thermodynamics and cannot be stable over thermal timescales \citep[see][and references therein]{hazlehurstEquilibriumContactBinary1993, kahlerStructureEquationsContact1989}.
\par
Later, \citet{kahlerStructureContactBinaries2004} concludes from all collected theoretical arguments that internal circulation currents must exist in the less luminous component that reduces the radiative temperature gradient since the luminosity carried by radiation is reduced by the circulation luminosity.
\par
Given the complexity of the theoretical problem of the structure of contact binaries, especially with radiative envelopes, it is beyond the scope of this work to further develop the analytic theory.
Instead, we apply an ET model in modern stellar structure calculations.
With shellularity as our base assumption of the stellar structure (see Sect. \ref{ssec:roche} for the precise definition), the model of \citetalias{shuStructureContactBinaries1976} is a natural choice, though we recognize that this comes with the apparent thermodynamical problems stated above.
However, we believe we avoid the most fundamental one, as we do not explicitly require a temperature inversion in our models.
We only use the model of \citetalias{shuStructureContactBinaries1976} to compute the amount of energy transferred (see Sect. \ref{ssec:ctctdis}).

\subsection{Energy transfer in the Roche geometry}\label{ssec:ctctdis}
The work of \citetalias{shuStructureContactBinaries1976} provides a general model of ET, by introducing the notion of an energy flow at the base of the common envelope.
If the contact layers are shellular, and satisfy von Zeipel's gravity darkening, conservation of energy at the RL implies:
\begin{subequations}\label{eq:newL}
\begin{align}
    L_1' &= (L_1+L_2) \frac{S_1\langle g\rangle_1}{S_1\langle g\rangle_1 + S_2\langle g\rangle_2},\\
    L_2' &= (L_1+L_2) \frac{S_2\langle g\rangle_2}{S_1\langle g\rangle_1 + S_2\langle g\rangle_2}.
\end{align}
\end{subequations}
Here, $S$ is the surface area of the RL, $\langle g\rangle$ the surface averaged effective gravity at the RL, and the primed quantities specify the state just above the ET layer, while unprimed those just below.
This equation specifies that the fraction of the total luminosity that each component radiates is proportional to $S\langle g\rangle$.
The transferred luminosity then equals
\begin{equation}\label{eq:etransfer}
    L_{\rm trans} = L_1 - L_1' = \frac{L_1 S_2\langle g\rangle_2 - L_2 S_1\langle g\rangle_1}{S_1\langle g\rangle_1 + S_2\langle g\rangle_2}.
\end{equation}
Comparing Eq. \eqref{eq:newL} against Eqs. \eqref{eq:dL1}-\eqref{eq:dL2}, we find for the fraction $f$:
\begin{equation}
    f = \frac{L_1S_2\langle g\rangle_2-L_2S_1\langle g\rangle_1}{(L_1+L_2)S_1\langle g\rangle_1},
\end{equation}
which is consistent with Eq. \eqref{eq:dLf} as $\frac{S_2\langle g\rangle}{S_1\langle g\rangle} \simeq q$ and $\frac{L_2}{L_1}\simeq q^\alpha$.
\par

% \citet{shuStructureContactBinaries1979} argue that the thickness $d$ of the ET layer is of order:
% \begin{equation}\label{eq:etthickness}
%     \frac{d}{a} = \delta^{0.4},\quad \delta = \frac{L/a^2}{\rho h c_s}
% \end{equation}
% with $a$ the binary separation, and $L$, $\rho$, $h$ and $c_s$ the luminosity, density, specific enthalpy and local sound speed evaluated at the RL, respectively.
% \citet{lubowStructureContactBinaries1979} compute this number to be $d/a \sim 1 \cdot 10^{-2}$ for stars of masses around 4-8$M_\odot$.
% In this layer, a heat engine operates to move a net amount of energy $L_{\rm trans}$ from the more luminous component to the other. 
% Given that the ET layer is small compared to the sizes of the stars and the orbit, \citetalias{shuStructureContactBinaries1976} modeled it as a discontinuity in the stellar profile, located at the RL radius.
% \par

% \subsection{The mathematical problem}
% The mathematical problem of stellar structure and evolution (in 1D) is in its most fundamental form a coupled initial value and a boundary value problem.
% A set of nonlinear partial differential equations are supplied by boundary conditions (BCs), as well as an initial setup of the stellar material, and is then solved numerically to create a sequence of stellar models that show its evolution.
% The general, time dependent problem is extremely complex and hard to analyze.
% However, in the present work, we are interested in the structure of a contact binary in hydrostatic and thermal equilibrium only, meaning we need only consider the following time independent stellar structure equations:
% \begin{align}
%     \frac{dr_\Psi}{dM_\Psi} &= \frac{1}{4\pi r_\Psi^2 \rho}, \label{eq:continuity}\\
%     \frac{dP}{dM_\Psi} &= -\frac{GM_\Psi}{4\pi r_\Psi^4}f_P, \label{eq:hydro}\\
%     \frac{dT}{dM_\Psi} &= -\frac{GM_\Psi T}{4\pi r_\Psi^4}\nabla_{\rm rad}f_T \quad{\rm or}\quad \frac{ds}{dM_\Psi} = 0,\label{eq:trans}\\
%     \frac{dL_\Psi}{dM_\Psi} &= \epsilon.\label{eq:energy}
% \end{align}
% In writing these equations, we have immediately changed variables to take into account a conservative deforming potential $\Psi$, as is the case in a binary modeled by the Roche potential (see \citetalias{fabryModelingcontactBinaries2022}, Sect. 2).
% For a single star of total mass $M_\star$, this set of equations is to be solved on the interval $[0, M_\star]$, and as four first order differential equations, four  BCs are to be supplied to arrive at a set of solutions for that star\footnote{The solution to the boundary value problem is not necessarily unique, as for example also pointed out by \citet{kippenhahnStellarStructureEvolution2013}.}.
% \par
% In a contact binary, the set of differential equations \eqref{eq:continuity}-\eqref{eq:energy} need to be solved for two stars simultaneously.
% Furthermore, the condition that the layers that are in contact need to be shellular splits the integration volume of both stars in two parts, one below the Roche lobe, where the stars are separated, and one above that is part of the shared envelope.
