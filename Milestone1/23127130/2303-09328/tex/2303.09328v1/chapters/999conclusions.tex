\section{Conclusions}\label{sec:conc}
In this work we have taken a step forward in the detailed modeling of contact binaries, by implementing a model of ET in detailed stellar structure and evolution models.
From Fig. \ref{fig:profiles}, we see that this relatively simple model (that is, the inclusion of a heat source or sink in the stellar model as proxy for the ET) is capable of shellularizing the common layers of stellar components in a contact configuration.
Models without such ET do not exhibit their common layers to be shellular, which, from theoretical arguments, would drive strong horizontal flows equilibrating all gradients, in particular pressure.
\par

From Figs. \ref{fig:mdot} and \ref{fig:dtdq}, we saw that the time spent in deep contact at mass ratios between $\bar{q}=\numrange{0.7}{0.8}$ is extended if ET was included, versus when it was ignored.
This is a promising result, in that if this trend persists across the parameter space of total mass, initial mass ratio and initial period, ET could provide an answer to the discrepancy between the observed mass ratio distribution of massive contact systems and its predicted distribution.
The computation of a full grid of models and performing population synthesis is the topic of future work.
\par
