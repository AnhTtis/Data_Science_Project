\section{Convergence study}\label{app:convergence}
This appendix details the convergence study we used to validate our stellar model with and without energy transfer.
We have implemented the ET method in both MESA version r15140, as well as version r22.11.1.
\par
\subsection{Explicit evaluation}\label{ssec:expl}
In MESA r15140, both the tidal deformation corrections of \citetalias{fabryModelingOvercontactBinaries2022} and the ET of Sect. \ref{ssec:et} are evaluated explicitly in the differential equation solver, meaning only information of the previous evolutionary step is used.
\par
We perform a time resolution study in which we take a base template, and increase the time resolution by successive factors of two.
Figure \ref{fig:convergence15140} shows the evolution of the MT rate, ET rate, the mass ratio and the donor radius during the fast case A MT phase.
While the time of the onset of MT, which coincides with the donor reaching the RL, remains roughly constant for the different resolutions, the MT rate evolves significantly differently.
For higher time resolutions, the MT rate is higher, resulting in slightly shorter contact phase and a higher final mass ratio.
This effect is independent of our implementation of ET since the changes occur before the contact phase (signified by the bump in the donor radius).
Furthermore, these models do not yet converge to a certain value for the quantities of interest, which are the contact phase duration and the mass ratio after the MT phase.
This means we would need to increase the time resolution further to get to a convergent solution.
This is prohibitive however keeping in mind the aim is to run a full population of models in later work.
We therefore elect to implement the tidal deformation corrections in MESA version 22.11.1 to allow for implicit evaluation during integrations.
\par
\begin{figure}
    \centering
    \includegraphics{img/convergence15140.png}
    \caption{Evolution of the MT rate and ET rate (top panel) and the fractional RL radius and mass ratio (bottom panel) during fast case A MT for a $20 + 14 M_\odot$ binary, computed using the explicit evaluation of the tidal deformation corrections in MESA r15140. The color indicates what time resolution scale has been used.}
    \label{fig:convergence15140}
\end{figure}

\subsection{Implicit evaluation}
In MESA r22.11.1, infrastructure is provided to supply numerically approximated derivatives necessary for the solver to implicitly evaluate the tidal deformation corrections during an integration step.
Generally, implicit evaluation is beneficial for numerical stability, and allows for longer time steps.
Note that the amount of energy to be transferred and its location in the stellar structure is still evaluated explicitly, as detailed in Sect. \ref{ssec:et}.
\par
We do a similar resolution study as in Sect. \ref{ssec:expl} of a $20+14M_\odot$ binary, were we progressively increase the time resolution between simulations.
Compared to version r15140, we observe much better convergence of the MT rate evolution, as pictured in Fig. \ref{fig:convergence22}.
\par
Finally, we perform a spacial resolution convergence study for the same initial conditions as before.
Figure \ref{fig:spaceconvergence} shows that our models are well converged in the spacial coordinate already at the 1x level.
Even during the contact phase where ET occurs, and a sharp increase in luminosity is introduced, the models depend very weakly on the spacial resolution.
\par
\begin{figure}
    \centering
    \includegraphics{img/convergence22.11.1.png}
    \caption{Same as Fig. \ref{fig:convergence15140}, but now the models were computed using implicit evaluation of the tidal deformation corrections in MESA r22.11.1.}
    \label{fig:convergence22}
\end{figure}
\begin{figure}
    \centering
    \includegraphics{img/space_convergence.png}
    \caption{Same as Fig. \ref{fig:convergence22}, but now the color indicates the increase in spacial resolution.}
    \label{fig:spaceconvergence}
\end{figure}