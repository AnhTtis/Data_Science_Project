%====================================================================
\section{Introduction}
 \label{sec:introduction}
 %====================================================================
 
Hadron-hadron scattering processes have long been used as a tool to explore strong interaction physics. The amplitudes which describe these processes as a function of energy and angle can be expanded in partial-waves, and examination of these yields information about the resonance content of quantum chromodynamics. Scattering of the lightest hadron, the pion, off the pion cloud around a proton or nucleus offers the simplest such process, being unburdened by complications of spin.

Experimentally, $\pi\pi$ scattering in the lowest partial-waves shows very different behavior across the three possible isospins. Isospin--2 is found to be weak and repulsive, with the lack of resonances being an early motivator of the $q\bar{q}$ quark model. Isospin--1 houses the narrow $\rho$ resonance, appearing as a rapid rise in the phase-shift of the $P$--wave amplitude, which is otherwise featureless at low energy. Isospin--0 is found to be attractive, but shows only a slow rise in phase-shift with energy across the elastic scattering region. Typically, this rise is associated with a \emph{broad} scalar resonance, the $\sigma$. A state with these quantum numbers has historically been included in models of the nucleon-nucleon potential to describe intermediate-range effects. Precise determination of the pole location of the $\sigma$ remained a problem until recently, with a range of amplitude parameterizations applied to experimental data generating a spread of pole locations~\cite{ParticleDataGroup:2022pth}. This problem was solved by applying dispersion relations which implement analyticity and crossing symmetry in a consistent way, providing additional constraint beyond that given by the isospin--0 scattering data on the real energy axis alone~\cite{Ananthanarayan:2000ht,Colangelo:2001df,Kaminski:2006yv,Kaminski:2006qe,GarciaMartin:2011cn,Moussallam:2011zg}.


The nature of the $\sigma$ within QCD is somewhat unclear~\cite{Jaffe:1976ig,Maiani:2004uc,tHooft:2008rus,RuizdeElvira:2010cs,Guo:2015daa}. It is often partnered with the $f_0(980),\, a_0(980)$ and $\kappa$ states into a `scalar nonet', despite the very different appearances of these resonances (narrow states at $K\bar{K}$ threshold versus very broad states away from any threshold). An association of this type for the lightest vector resonances, $\rho, \omega, \phi, K^*$, is quite natural, given their common properties, and is often used as motivation for a $q\bar{q}$ quark-model assignment of these states, with their modest differences being due to the mild breaking of an approximate $SU(3)$ flavor symmetry that leads to states with strange content being heavier. The scalar nonet has no such simple interpretation~\cite{Morgan:1974cm,Tornqvist:1982yv,Tornqvist:1995ay,Tornqvist:1995kr}.


Recently, $\pi\pi$ scattering in QCD has been considered making use of the first-principles lattice approach. The discrete spectrum of states with the quantum numbers of $\pi\pi$ in the finite periodic spatial volume of the lattice can be related to $\pi\pi$ scattering amplitudes through the L\"uscher relation~\cite{Luscher:1986pf,Briceno:2017max}. Computations have taken place at several values of the light-quark mass for all three isospins~\cite{
Sharpe:1992pp,Gupta:1993rn,Kuramashi:1993ka,Fukugita:1994ve,JLQCD:2002mgw,Du:2004ib,CP-PACS:2004dtj,Beane:2005rj,Beane:2007xs,CLQCD:2007rcz,Feng:2009ij,Yagi:2011jn,Sasaki:2013vxa,Fu:2013ffa,ETM:2015bzg,Dudek:2010ew,NPLQCD:2011htk,Dudek:2012gj,Bulava:2016mks,Horz:2019rrn,Culver:2019qtx,Fischer:2020jzp,CP-PACS:2007wro, Feng:2010es, Lang:2011mn, CS:2011vqf, Dudek:2012xn, Pelissier:2012pi, Wilson:2015dqa, Bali:2015gji, Bulava:2016mks, Alexandrou:2017mpi,Andersen:2018mau,ExtendedTwistedMass:2019omo,Fischer:2020yvw,Briceno:2016mjc,Fu:2017apw,Guo:2018zss,RBC:2021acc}.
By parameterizing the elastic scattering amplitudes the resonance pole content can be investigated through analytic continuation into the complex energy plane.



%
Isospin--2 is found to be weak and repulsive, as in experiment, and the variation of the scattering length with changing quark mass has been explored in the context of chiral perturbation theory~\cite{NPLQCD:2011htk,Fischer:2020jzp}. Isospin--1 is found to feature a $\rho$--like resonance whose mass increases and width decreases with increasing quark mass until it becomes stable at a pion mass near 400 MeV. Isospin--0 shows a much more dramatic evolution with changing quark mass: at $m_\pi \sim 391$ MeV, a clear \emph{stable} bound-state $\sigma$ is observed, while at $m_\pi \sim 239$ MeV, a slow variation of the phase-shift appears to indicate a broad resonance $\sigma$, albeit with a large degree of amplitude parameterization dependence in the pole position~\cite{Briceno:2016mjc,Briceno:2017qmb}.

The possibility that the $\sigma$ could undergo a transition from being a broad resonance into being bound, as the light quark mass is increased, was previously explored in a unitarized version of chiral perturbation theory~\cite{Hanhart:2008mx,Guo:2018zss}. By making assumptions about the quark mass dependence of certain low-energy coefficients, it was found that over a relatively small variation in pion mass, the $\sigma$ undergoes a rapid transition from being a bound state, to being a virtual bound state (a pole on the real energy axis below threshold on the unphysical Riemann sheet), to being a broad resonance. These results provide a particular manifestation of the general framework for scalar resonance pole trajectory discussed in Ref.~\cite{Hanhart:2014ssa}.

In this paper we will report the results of a calculation determining $\pi\pi$ scattering amplitudes in all three isospins at two previously unconsidered light quark masses, corresponding to $m_\pi \sim $ 283 and 330 MeV. These values lie between the points at which the $\sigma$ has been observed in lattice calculations to be bound, and where it appears as a broad resonance, so that we aim to be able to close in on the region where the transition takes place.

The scatter of $\sigma$ pole positions under reasonable variation of amplitude parameterization in the previous lattice calculation at $m_\pi \sim 239$ MeV indicated that the same issue present in analysis of experimental scattering data may plague attempts to pin down with precision the pole location in these first-principles QCD efforts. A possible mechanism to overcome this might be to apply dispersion relations to the results of lattice QCD computations. Such an approach would require input of computed $\pi\pi$ scattering amplitudes in all three isospins in low partial waves, which is what we provide in this paper.


We will show that the isospin--2 amplitude evolves smoothly with changing light quark mass, and we will explore the role of the `Adler zero' predicted by the broken chiral symmetry of QCD. The evolution of the $\rho$ resonance which dominates the isospin--1 amplitude is presented, with a confirmation of the near independence of its coupling to $\pi\pi$ to variations in the light quark mass.   The isospin--0 $S$-wave amplitude is found to undergo a dramatic transition between ${m_\pi \sim  330}$ MeV and 283 MeV, from a behavior compatible with an only-just-bound $\sigma$ at the heavier mass to a mild energy dependence compatible with being either a virtual bound state or a subthreshold resonance at the lighter mass. We will conclude that to make more precise statements about the $\sigma$ in cases that it is unbound, we require additional constraints of the type offered by dispersion relations\footnote{While this paper was in the final stages of production, a preprint, Ref.~\cite{Cao:2023ntr}, appeared applying dispersion relations to lattice QCD data, focusing at $m_\pi \sim 391$ MeV, where the $\sigma$ is a well-determined bound-state.}.   


