%====================================================================
\section{$\pi \pi \to \pi \pi$ $I=0$}
 \label{sec:I0pipi}
%====================================================================

Isospin--0 $\pi\pi$ scattering in $S$--wave is not simple to describe, being neither weak nor dominated by a single narrow resonance. At the physical pion mass, despite there being relevant scattering data available for over forty years, it is only recently that the role of the broad $\sigma$ resonance has been confirmed with certainty~\cite{Caprini:2005zr,Garcia-Martin:2011nna,Moussallam:2011zg}. Recent consideration of this scattering channel using first-principles lattice QCD showed a clear $\sigma$ \emph{bound-state} when $m_\pi \sim 391$ MeV, and evidence for a broad $\sigma$ resonance (albeit with significant parameterization dependence) when $m_\pi \sim 239$ MeV~\cite{Briceno:2016mjc}. We will observe in the current calculation that between the two pion masses considered here the $\sigma$ undergoes a dramatic change in form.

%figure 11
\begin{figure*}[!ht]
\begin{center}
%
\resizebox{\textwidth}{!}{
  \includegraphics{fig11a_spectrum_isospin0_850_Dwave.pdf}%{../figures/850-I0-spec2.pdf} 
  \includegraphics{fig11b_spectrum_isospin0_856_Dwave.pdf}%{../figures/856-I0-spec2.pdf} 
}
%
\resizebox{\textwidth}{!}{
  \includegraphics{fig11c_spectrum_isospin0_850_SDwave.pdf}%{../figures/850-I0-spec1.pdf} 
  \includegraphics{fig11d_spectrum_isospin0_856_SDwave.pdf}%{../figures/856-I0-spec1.pdf} 
}
%
\caption{
$I=0$ finite-volume spectra, $a_t E_\mathsf{cm}$, by irrep, against $L/a_s$, for $m_\pi \sim 330$ MeV (left) and $m_\pi \sim 283$ MeV (right). Upper panel irreps have $D$-wave as their leading partial-wave, while those in the lower panel have $S$-wave leading. Red/green curves show $\pi\pi$/$K\bar{K}$ non-interacting energies.
}
\label{fig:I0}   
\end{center}
\end{figure*}


%====================================================================
\subsection{$I=0$ finite-volume spectra}
\label{subsec:I0FV}
%====================================================================

Spectra were extracted from correlator matrices computed using a basis of single-hadron operators, $\pi\pi$-like operators, $K\bar{K}$-like operators, and some $\eta\eta$-like operators (for the lighter pion mass, larger volume lattice). The energies are shown in~\cref{fig:I0}, where it is clear that there are large departures from the non-interacting $\pi\pi$ energies in those irreps containing subduction of the $\pi\pi$ $S$--wave suggesting strong scattering, while those irreps having $D$--wave as their leading partial-wave show only small downward shifts indicative of mild attraction. \mbox{$D$--wave} resonances, $f_2, f_2'$, are expected to lie at significantly higher energy, well into the coupled-channel region~\cite{Briceno:2017qmb}.

Even though the $I=0$ correlation functions receive vital contributions from relatively noisy diagrams featuring complete quark-line annihilation, the use of distillation leads to high-quality signals, and the extracted energy levels are of high statistical precision. The error bars on the points plotted in~\cref{fig:I0} also include systematic errors originating from fitting variations, which are modest in most cases.



%figure 12
\begin{figure}
\begin{center}
%
  \includegraphics[width=\columnwidth]{fig12a_delta_isospin0_850_Dwave.pdf}%{../figures/850-I0-Dwave.pdf}
  
  \includegraphics[width=\columnwidth]{fig12b_delta_isospin0_856_Dwave.pdf}%{../figures/856-I0-Dwave.pdf}
%  
\caption{
$I=0$ $D$-wave phase-shift for $m_\pi \sim 330$ MeV (top) and $m_\pi \sim 283$ MeV (bottom). Discrete data points come from irreps in which $\ell=2$ is the lowest subduced partial-wave, assuming $\ell\ge 4$ scattering to be negligible. 
Curves show two illustrative parameterizations fitted to the full set of energy levels plotted: an effective range expansion with two terms (red) and a conformal mapping with two terms (black). The shaded region indicates energies above the $K \bar K$ threshold.
}
\label{fig:I0Dfit}   
\end{center}
\end{figure}


In total, there are 23 levels for $m_\pi \sim 330$ MeV and $74$ levels for $m_\pi \sim 283$ MeV. For the $D$--wave dominated irreps, there is no evidence of coupling between $\pi\pi$ and $K\bar{K}$, and a description in terms of purely elastic $\pi\pi$ scattering, even above the $K\bar{K}$ threshold will prove to be successful. The $S$--wave dominated irreps on the other hand cannot be described so simply, and we consider only energy levels lying some way below the $K\bar{K}$ threshold, where channel coupling is expected to turn on rapidly. For $m_\pi \sim 330$ MeV we use 18 energy levels, and 48 for $m_\pi \sim 283$ MeV.






%====================================================================
\subsection{$\pi \pi \to \pi \pi$ $I=0$ scattering}
\label{subsec:I0}
%====================================================================



There is no evidence in the computed spectra that amplitudes with $\ell > 2$ are required over the energy region we are considering, and as seen in~\cref{fig:I0Dfit}, even the $D$-wave amplitude is only very mildly attractive. 


%figure 13
\begin{figure*}[ht]
\begin{center}
%
\resizebox{\textwidth}{!}{
  \includegraphics{fig13a_delta_isospin0_850_Swave.pdf}%{../figures/850-I0-Swave.pdf} 
  \hspace{.1cm} 
  \includegraphics{fig13b_delta_isospin0_856_Swave.pdf}%{../figures/856-I0-Swave.pdf} 
}
%
\resizebox{\textwidth}{!}{
  \includegraphics{fig13c_kcotd_isospin0_850_Swave.pdf}%{../figures/850-I0-Swave_kcot.pdf} 
  \hspace{.1cm} 
  \includegraphics{fig13d_kcotd_isospin0_856_Swave.pdf}%{../figures/856-I0-Swave_kcot.pdf}
}
%
\caption{
$I=0$ $S$--wave scattering for $m_\pi\sim330$ MeV (left) and $m_\pi\sim283$ MeV (right). Four example parameterizations are shown: a two-term conformal mapping (black), an effective range expansion with two terms (green), and two choices with an Adler zero fixed at the leading order $\chi$PT location, a two-term conformal mapping (red), and an effective range expansion with two terms (orange). In the bottom panels, the gray curves show $\mp \sqrt{-k^2}$ below threshold -- intersection of the $k \cot \delta^0_0$ curves with these indicate the location of a bound-state or a virtual bound-state respectively.}
\label{fig:I0Sfit}   
\end{center}
\end{figure*}

The $S$--wave amplitudes, shown in~\cref{fig:I0Sfit}, provide our first example of amplitudes whose description is not obvious, and where the behavior changes dramatically between the two pion masses considered. The lighter pion mass shows a phase-shift increasing with a moderate slope from threshold, and when plotted as $k \cot \delta^0_0$, a crossing of threshold at a small positive value, indicating a large positive scattering length. The heavier pion mass shows a qualitatively different energy dependence, having an approximately flat phase-shift above threshold, and a $k \cot \delta^0_0$ threshold crossing at a small negative value, indicating a large negative scattering length.

The plots of $k \cot \delta^0_0$ show clearly the presence of a systematic variation with choice of parameterization. A wide range of forms of the type described in Section~\ref{sec:amplitude-analysis} is used, including cases with and without an Adler zero. As was the case for $I=2$ $S$--wave analysis, we explore Adler zeroes fixed at the leading order location, $s_A = \tfrac{1}{2}m_\pi^2$, and varying between the extremes suggested in Ref.~\cite{Garcia-Martin:2011iqs} (appropriately scaled for the changed pion mass), and finally, allowing the zero location to float as a free parameter. 

The four illustrative cases presented in~\cref{fig:I0Sfit} (right) indicate a slight sensitivity to the presence of an Adler zero, likely reduced relative to the $I=2$ case by virtue of the zero being further below threshold. The energy levels below threshold do not obviously suggest a preference either way for an Adler zero.


%figure 14
\begin{figure}
\begin{center}
%
\includegraphics[width=\columnwidth]{fig14a_scatlen_isospin0_850_Swave.pdf}%{../figures/850_S0_SL} 

\includegraphics[width=\columnwidth]{fig14b_scatlen_isospin0_856_Swave.pdf}%{../figures/856_S0_SL.pdf} 
%
\caption{Extracted scattering length for a range of $I=0$ $S$--wave amplitude parameterizations for $m_\pi\sim330 \, \mathrm{MeV}$ (top) and ${m_\pi\sim283\, \mathrm{MeV}}$ (bottom). Each amplitude is labeled by the $\chi^2/N_\mathrm{dof}$ with which it describes the finite-volume spectrum. Red points correspond to amplitudes containing an Adler zero at some location, while blue points lack any enforced subthreshold zero.
}
\label{fig:I0S_SL}   
\end{center}
\end{figure}


Figure~\ref{fig:I0S_SL} shows the scattering length for each parameterization choice that successfully describes the finite-volume spectra. It is clear that at the heavier pion mass, the presence, or not, of an Adler zero has no impact on the value of the scattering length, and we will discuss this further in the next section in the context of there being a bound-state pole dominating the amplitude. At the lighter pion mass, there is a slight tendency to a larger scattering length for amplitudes that lack an Adler zero, but the effect is barely significant. Our estimates at these two pion masses and those determined in Ref.~\cite{Briceno:2016mjc} are plotted in~\cref{fig:I0_SL_quarkmass}. An explanation of the observed behavior would be that these pion masses straddle a rapid divergence near $m_\pi \sim 300$ MeV, where the scattering length tends to $\pm \infty$ on either side of the divergence. In the next section, we will discuss how this can be related to the $\sigma$ pole undergoing a transition between Riemann sheets by passing through the $\pi\pi$ threshold.




%figure 15
\begin{figure}[!ht]
\begin{center}
%
\resizebox{\columnwidth}{!}{
  \includegraphics{fig15_scatlen_isospin0_mpi.pdf}%{../figures/I0_SL_quark_mass.pdf} }
}
%
\caption{ $I=0$ $S$--wave scattering length at four pion masses (this paper and Ref.~\cite{Briceno:2016mjc}). Red points correspond to parameterizations featuring an Adler zero, while blue points have no enforced subthreshold zero. 
The result of dispersive analysis applied to experimental data~\cite{Garcia-Martin:2011iqs} is shown by the gray point.
}
\label{fig:I0_SL_quarkmass}   
\end{center}
\end{figure}


%====================================================================
\subsection{The $\sigma$ pole}
 \label{subsec:sigma}
%====================================================================

The presence of singularities on the real axis below threshold can be inferred rather directly from graphs of $k \cot \delta$ against $k^2$. Since
%
\begin{equation}
t^I_\ell(s) = \frac{1}{\rho(s)} \frac{1}{\cot \delta^I_\ell(s) - i} = \frac{\tfrac{1}{2}\sqrt{s}}{k \cot \delta^I_\ell(s) - i k} \, ,
\end{equation}
%
it follows that pole singularities are present whenever the graph of $k\cot \delta$ intersects a curve, $ik = \pm \sqrt{-k^2}$ below threshold. The negative sign corresponds to a pole on the physical Riemann sheet, a \emph{bound state}, while the positive sign corresponds to the second Riemann sheet and a \emph{virtual bound state}.

In Figure~\ref{fig:I0Sfit} (left), for the heavier pion mass, all amplitude parameterizations intersect with $- \sqrt{-k^2}$ only slightly below threshold, with a parameterization dependence below the statistical uncertainty. This indicates the presence of a bound-state lying very close to threshold, and indeed numerical determination shows it to be statistically compatible with being at threshold, see Figure~\ref{fig:I0Spoles}. Restricting amplitude fits to only describing levels in a small region around threshold does not change this conclusion. 


As a bound-state pole approaches threshold, the value of $k\cot \delta^0_0$ at the pole location tends to $1/a^0_0$, and hence the scattering length must diverge to $-\infty$ as was suggested in the previous section. An argument due to Weinberg~\cite{Weinberg:1962hj} suggests that the scattering length ($a_0^0$), effective range ($r^0_0$), and binding energy ($\epsilon=2 m_\pi-m_\sigma$) together can be used to determine the degree to which this bound-state is of ``$\pi\pi$-molecular'' versus ``compact'' nature,

%
\begin{equation}
\begin{aligned}
& a^0_0=-2 \frac{1-Z}{2-Z} \frac{1}{\sqrt{m_\pi \epsilon}}\,, \quad r^0_0=-\frac{Z}{1-Z} \frac{1}{\sqrt{m_\pi \epsilon}},
\end{aligned}
\end{equation}
%
where $Z$ is interpreted as the probability to find the state in a compact configuration. Compatible values of $Z$ are obtained from each of these two equations suggesting that (suppressed) corrections are modest, and the resulting $Z = 0.07(4)$ suggests dominance of a $\pi\pi$ component over any compact component in the $\sigma$ at this pion mass.


%figure 16
\begin{figure*}[ht]
\begin{center}
\resizebox{\textwidth}{!}{
  \raisebox{-0.5\height}{\includegraphics{fig16a_poles_isospin0_850.pdf}}%{../figures/850-I0-Swave-poles.pdf}} 
  \hspace{.1cm} 
  \raisebox{-0.5\height}{\includegraphics{fig16b_poles_isospin0_856.pdf}}%{../figures/856-I0-Swave-poles.pdf}} 
}
%
\caption{
Extracted $\sigma$ pole location for each $I=0$ $S$--wave parameterization found capable of describing the finite-volume spectra for $m_\pi\sim330$ MeV (left) and $m_\pi\sim283$ MeV (right). Left panel shows the physical sheet housing a bound-state pole, right panel shows the lower half-plane of the unphysical sheet housing either a virtual bound-state pole or a subthreshold resonance pole.
}
\label{fig:I0Spoles}   
\end{center}
\end{figure*}


In Figure~\ref{fig:I0Sfit} (right), for the lighter pion mass, there are now two classes of parameterization. Many parameterizations capable of describing the finite-volume spectrum cross the curve $+\sqrt{-k^2}$ below threshold, indicating the presence of a virtual bound state, but some do not. Upon searching these latter amplitudes for poles off the real axis, complex conjugate pairs of poles are found off the real axis below threshold. As can be seen in Figure~\ref{fig:I0Sfit} (right), there is not a significant qualitative difference in the amplitude above threshold between the effect of a virtual bound-state and a sub-threshold resonance. The locations of these poles are shown in Figure~\ref{fig:I0Spoles}.

In the case of a virtual bound state lying at threshold, a similar logic to that presented above for a bound state indicates that the scattering length must diverge to $+\infty$ as the pole reaches threshold. The transition, as the pion mass increases, from scattering length diverging to $+\infty$, to reducing from $-\infty$ would therefore correspond to a pole on the second Riemann sheet moving onto the physical Riemann sheet by passing through the threshold.



Figure~\ref{fig:sigma-quark-mass} summarizes the $\sigma$ pole positions extracted from calculations at $m_\pi \sim 391, 330, 283$ and $239$ MeV using the same lattice action. At the heaviest two pion masses, the $\sigma$ is a stable bound-state, at $283$ MeV it is either a virtual bound-state or a subthreshold resonance (depending upon parameterization), while at $239$ MeV it appears to be a broad resonance. Qualitatively this evolution in quark mass conforms to the general scheme presented in Ref.~\cite{Hanhart:2008mx,Pelaez:2010fj,Hanhart:2014ssa} -- as the pion mass is increased from its physical value, where the $\sigma$ is a broad resonance, the complex conjugate pole pairs on the second Riemann sheet move toward the real energy axis, eventually meeting at a point below threshold. One pole then moves away towards negative infinity, becoming less relevant, while the other moves toward threshold as a virtual bound state. When this pole reaches threshold, it moves onto the physical sheet as a bound state, which becomes more deeply bound as the pion mass increases further. 

%figure 17
\begin{figure*}[ht]
\begin{center}
%
\resizebox{\textwidth}{!}{
\includegraphics{fig17a_poles_isospin0_mpi.pdf}%{../figures/I0-Swave-poles-quark-mass-mpi.pdf} 
\hspace{.1cm} 
\includegraphics{fig17b_couplings_isospin0_mpi.pdf}%{../figures/I0-Swave-residues-quark-mass.pdf} 
}
%
\caption{
Left: $\sigma$ pole location with varying pion mass from this calculation and from calculations on lattices with the same action~\cite{Dudek:2012xn, Wilson:2015dqa}. Green ($m_\pi \sim 391$ MeV) and blue ($m_\pi \sim 330$ MeV) points lie on the physical sheet while red ($m_\pi \sim 283$ MeV) and orange ($m_\pi \sim 239$ MeV) points lie on the unphysical sheet (additional parameterizations have been applied to the energy levels published in~\cite{Briceno:2016mjc} to generate the spread of orange points). Right: Magnitude of the $\sigma$ pole coupling, as defined in Eqn.~\ref{gNorm} at four pion masses, with different parameterizations displaced horizontally for clarity. The points at $m_\pi \sim 283$ MeV are separated into two groupings according to whether the pole in that case is a subthreshold resonance or a virtual bound-state. The dashed vertical lines locate each one of our pion masses. The result of dispersive analyses~\cite{Caprini:2005zr,Garcia-Martin:2011nna,Moussallam:2011zg} of experimental data is shown in gray in each plot.
}
\label{fig:sigma-quark-mass}   
\end{center}
\end{figure*}

The behavior of the amplitude in the pion mass region where the conjugate pole pair meet on the real axis indicates the origin of the large statistical errors in the right panel of Figure~\ref{fig:I0Spoles}. At this point, the derivative of the pole location with respect to amplitude parameters can diverge, leading to an inability to propagate a statistical error~\footnote{
For example, with an effective range parameterization, the dependence on the pole locations on the scattering length and effective range is given by 
$$
\frac{\partial k_R}{\partial a} = \frac{1}{a^2} \frac{1}{r k_R - i}, \;
\frac{\partial k_R}{\partial r} = -\frac{1}{2} \frac{k_R^2}{r k_R - i}, 
$$
and the location where the two poles coincide is $k_R = i/r$ with $r$ negative and with $a = -2 r$.
}. The fact that our $m_\pi \sim 283$ MeV choice appears to be close to this point generates larger statistical uncertainties on the pole position than might otherwise be expected.

The lack of a reliable determination of a \emph{second} subthreshold pole (as expected by the pole evolution argument described above) for the heavier pion mass considered in our calculation likely reflects the insensitivity of the amplitude near and above threshold (which determines the finite volume spectrum) to such a rather distant pole. Some additional constraints below threshold would be required to pin it down with certainty.


The reduction in the value of $|g_{\pi\pi}|$ observed in Figure~\ref{fig:sigma-quark-mass} for $m_\pi \sim 330$ MeV is expected on general grounds if this point is close to the pion mass where the $\sigma$ pole passes through the threshold. Kinematic constraints on the amplitude at threshold ensure that as the pole approaches threshold, the $S$--wave coupling must behave like $g_{\pi\pi}^2 \propto \sqrt{s_R - 4 m_\pi^2}$, and hence must vanish as the pole crosses the threshold. 

As was previously observed in a lattice calculation with $m_\pi \sim 239$ MeV~\cite{Briceno:2016mjc}, the results of this paper indicate that in those cases where the $\sigma$ is unbound, even with the use of large numbers of high-precision finite-volume energy levels, the $\sigma$ pole location cannot be precisely pinned down. Different parameterizations which describe the real-energy data equally well lead to pole locations and couplings scattered well outside the statistical uncertainty, and we conclude that to reduce this systematic error it will be necessary to introduce a greater level of theoretical constraint into the determination of the scattering amplitudes. In the next section, we will describe a plausible approach to achieve this which makes use of the full set of partial-wave amplitudes across three isospins computed in this paper.








