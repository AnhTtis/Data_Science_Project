%figure 6
\begin{figure*}[ht]
\begin{center}
%
\resizebox{\textwidth}{!}{
  \includegraphics{fig6a_spectrum_isospin1_850.pdf}%{../figures/850-I1-spec.pdf} 
  \includegraphics{fig6b_spectrum_isospin1_856.pdf}%{../figures/856-I1-spec.pdf} 
}
%
\caption{
$I=1$ finite-volume spectra, $a_t E_\mathsf{cm}$, by irrep, against $L/a_s$, for $m_\pi \sim 330$ MeV (left) and $m_\pi \sim 283$ MeV (right). Red/green curves show $\pi\pi$/$K\bar{K}$ non-interacting energies.
}
\label{fig:I1spec}   
\end{center}
\end{figure*}





%====================================================================
\section{$\pi \pi \to \pi \pi$ $I=1$}
 \label{sec:I1pipi}
%====================================================================

The $I=1$ channel contains the $P$--wave $\rho$ resonance which appears below the $K\bar{K}$ threshold, while the \mbox{$F$--wave} amplitude is expected to be featureless and very weak across the elastic region. In order to reliably determine the finite-volume spectrum up to slightly above the $K\bar{K}$ threshold, we make use of a large basis of single hadron operators, $\pi\pi$--like operators, and $K\bar{K}$--like operators.

%====================================================================
\subsection{$I=1$ finite-volume spectra}
\label{subsec:I1FV}
%====================================================================

Figure~\ref{fig:I1spec} shows the extracted spectra for the two pion masses considered in this calculation, where large departures from the $\pi\pi$ non-interacting energies (red curves) are observed, indicative of strong interactions. The isolated `extra' levels near $a_t E_\mathsf{cm} \sim 0.14$ suggest a narrow resonance in that energy region. At higher energies, the extracted finite-volume spectra lie very close to the non-interacting energies (including those corresponding to $K\bar{K}$) suggesting that the scattering amplitude may be featureless above the resonance. 



In total, we extracted 23 levels for $m_\pi \sim 330$ MeV and $95$ levels for $m_\pi \sim 283$ MeV, of which 17 and 42 are below the $K \bar K$ threshold, respectively. Examination of the operator overlaps for states above the $K\bar{K}$ threshold suggests that there appears to be no significant coupling between the $\pi\pi$ and $K\bar{K}$ channels, indicating that an analysis of elastic scattering above threshold, retaining only those levels with overlap onto $\pi\pi$ operators, may be successful. This appears to be essentially the same situation as was observed for $m_\pi \sim 239$ MeV in Ref.~\cite{Wilson:2015dqa}~\footnote{Referred to in that paper as $m_\pi \sim 236$ MeV. An improved extraction of the pion mass and the $\Omega$ baryon mass used to set the scale provide the newer value.}, where coupled channel analysis showed negligible $\pi\pi, K\bar{K}$ channel coupling over a significant energy region above threshold.





%====================================================================
\subsection{$\pi \pi \to \pi \pi$ $I=1$ scattering}
\label{subsec:I1}
%====================================================================

%figure 7
\begin{figure}[h]
\begin{center}
  \includegraphics[width=\columnwidth]{fig7a_delta_isospin1_850_Pwave.pdf}%{../figures/850-I1-Pwave.pdf}
  
  \includegraphics[width=\columnwidth]{fig7b_delta_isospin1_856_Pwave.pdf}%{../figures/856-I1-Pwave.pdf}
\caption{
$I=1$ $P$--wave phase-shift for $m_\pi \sim 330$ MeV (top) and $m_\pi \sim 283$ MeV (bottom). A parameterization using a conformal mapping with a resonance enforcing $F^I_\ell(s)$ factor shown by the black curve, and a $K$--matrix with a single pole plus a constant shown by the red curve. Discrete `data' points with large uncertainties have been removed from the plot for clarity. The shaded region indicates  energies above the $K \bar K$ threshold.
}
\label{fig:I1Pfit}   
\end{center}
\end{figure}




As explained above, we restrict ourselves to an elastic analysis in this manuscript, and the extracted spectra indicate that the $F$-wave amplitude is negligible relative to the $P$-wave in the region of interest. The $\ell=3$ angular momentum barrier factor suppresses the low-energy interactions, and the only resonance with those quantum numbers that decays to $\pi \pi$ is a $\rho_3$, which is expected to appear far above the energy region we consider. Thus, in this case, each energy level can be used to determine a discrete value of $\delta^1_1(s)$, as plotted in~\cref{fig:I1Pfit}. The behavior for each pion mass is clearly that of a narrow resonance, and we consider elastic amplitude parameterizations which describe the finite-volume spectra up to $a_t E_\mathsf{cm} = 0.19$.



A Breit-Wigner form, Eqn.~\ref{eq:bw}, is found to describe the finite-volume spectra reasonably, with parameters
%
\begin{center}
	\begin{tabular}{rll}
	$m_\mathrm{BW} =$                         & $0.13978\, (51) \cdot a_t^{-1}$   &
	\multirow{2}{*}{ $\begin{bmatrix*}[r] 1 &  0.08  \\
					                    & 1    \end{bmatrix*}$ } \\
			$g_\mathrm{BW} = $                  & $5.664\,(104)$   & \\[1.3ex]
			\multicolumn{2}{r}{$\chi^2/N_{\text{dof}}=\frac{14.42}{17-2}=0.96$.}
		\end{tabular}
	\end{center}
	\vspace{-1.0cm}
	\begin{equation}\label{eq:bw_850}\end{equation}
%
for $m_\pi \sim 330 \,\mathrm{MeV}$, and
%
\begin{center}
	\begin{tabular}{rll}
	$m_\mathrm{BW} =$                         & $0.13444\, (34) \cdot a_t^{-1}$   &
	\multirow{2}{*}{ $\begin{bmatrix*}[r] 1 & 0.00  \\
					                    & 1    \end{bmatrix*}$ } \\
			$g_\mathrm{BW} = $                  & $5.564\,(61)$   & \\[1.3ex]
			\multicolumn{2}{r}{$\chi^2/N_{\text{dof}}=\frac{41.20}{49-2}=0.88$.}
		\end{tabular}
	\end{center}
	\vspace{-1.0cm}
	\begin{equation}\label{eq:bw_856}\end{equation}
%
for $m_\pi \sim 283 \,\mathrm{MeV}$. In both cases, the Breit-Wigner mass and coupling parameters are essentially uncorrelated.

Considering a wider variety of amplitude parameterizations, including $K$--matrix forms and conformal expansions, we can find examples that describe the data with slightly improved $\chi^2/N_\mathrm{dof}$, but all successful descriptions show compatible phase-shift energy dependence in the region of the resonance. In the next section, we will examine the variation of the $\rho$ resonance pole with parameterization choice.



The amplitude at threshold is characterized by a scattering `length', defined via $k^3 \cot \delta^1_1 =  \frac{1}{a^1_1}$, and as seen in~\cref{fig:I1S_SL}, amplitudes capable of describing the finite-volume spectrum with the smallest $\chi^2$ values have compatible values of this parameter. The first entry plotted for each pion mass corresponds to the Breit-Wigner fit, and such a form is not expected to provide a faithful description of amplitudes away from the resonance that is being parameterized, and hence this form may not describe the threshold behavior accurately. 


%figure 8
\begin{figure}
\begin{center}
  \includegraphics[width=\columnwidth]{fig8a_scatlen_isospin1_850.pdf}%{../figures/850_P1_SL} 
  
  \includegraphics[width=\columnwidth]{fig8b_scatlen_isospin1_856.pdf}%{../figures/856_P1_SL}
\caption{
Extracted scattering length for a range of $I=1$ $P$--wave amplitude parameterizations for ${m_\pi\sim330 \,\mathrm{MeV}}$ (top) and ${m_\pi\sim283 \,\mathrm{MeV}}$ (bottom). Each amplitude is labeled by the $\chi^2/N_\mathrm{dof}$ with which it describes the finite-volume spectrum. 
}
\label{fig:I1S_SL}   
\end{center}
\end{figure}






%====================================================================
\subsection{The $\rho$ resonance}
 \label{subsec:rho}
%====================================================================

In the case of the $I=1$ $P$--wave, for both pion masses, a pole singularity lying near the real axis is found on the second Riemann sheet for every parameterization that successfully describes the finite-volume spectra. The pole location for each parameterization is plotted in~\cref{fig:I1Ppoles} where we observe very little scatter, indicating that the lattice spectra precisely determine the mass and width of the $\rho$ resonance at these pion masses without significant amplitude parameterization dependence.  



%figure 9
\begin{figure*}[!hbt]
\begin{center}
%
\resizebox{\textwidth}{!}{
  \includegraphics{fig9a_poles_isospin1_850.pdf}%{../figures/850-I1-Pwave-poles.pdf} 
  \hspace{.1cm} 
  \includegraphics{fig9b_poles_isospin1_856.pdf}%{../figures/856-I1-Pwave-poles.pdf} 
}
%
\caption{
Extracted $\rho$ resonance pole location for each $I=1$ $P$-wave parameterization found capable of describing the finite-volume spectra for $m_\pi\sim330$ MeV (left) and $m_\pi\sim283$ MeV (right).
}
\label{fig:I1Ppoles}   
\end{center}
\end{figure*}


These $\rho$ pole results supplement those obtained in Refs.~\cite{Dudek:2012xn,Wilson:2015dqa} at $m_\pi \sim 391, 239$ MeV, and in~\cref{fig:rhopoles} we present the evolution of the pole position and pole residue coupling (defined in Eqn.~\ref{gNorm}) with varying pion mass. As expected the $\rho$ becomes heavier as the light quark mass increases and narrower as the phase-space for decay to two pions decreases. The coupling appears to be consistent with being constant across the range of pion masses considered. These results agree with the expectations for an `ordinary $q \bar q$ meson' as defined in Ref.~\cite{Pelaez:2006nj}, and agree with predictions made for the quark mass trajectory of the $\rho$  in unitarized chiral perturbation theory models~\cite{Hanhart:2008mx,Pelaez:2010fj,Chen:2012rp}.



% figure 10
\begin{figure*}[ht]
\begin{center}
%
\resizebox{\textwidth}{!}{
  \includegraphics{fig10a_poles_isospin1_mpi.pdf}%{../figures/I1-Pwave-poles-quark-mass-mpi.pdf} 
  \hspace{.1cm} 
  \includegraphics{fig10b_couplings_isospin1_mpi.pdf}%{../figures/I1-Pwave-residues-quark-mass.pdf} 
}
%
\caption{ Left: $\rho$ resonance pole location with varying pion mass from this calculation (blue and red points) and from calculations on lattices with the same action~\cite{Dudek:2012xn, Wilson:2015dqa} (green, orange).  Right: Magnitude of the complex $\rho$ resonance pole coupling, as defined in Eqn.~\ref{gNorm}, and the real coupling, $g_\mathrm{BW}$, extracted when a Breit-Wigner, Eqn~\ref{eq:bw}, is used to describe the spectrum. The uncertainties on the pole location and pole couplings quoted from Ref.~\cite{Wilson:2015dqa} (orange points) are a rather conservative average over a large number of parameterizations, including several which include the $K\bar{K}$ coupled-channel region. For pole properties, the result of dispersive analysis of experimental data~\cite{Garcia-Martin:2011nna} is shown in gray, while the physical value of $g_\mathrm{BW}$ comes from the neutral modes listed in the PDG~\cite{ParticleDataGroup:2022pth}.
}
\label{fig:rhopoles}   
\end{center}
\end{figure*}




