%====================================================================
\section{Extracting scattering amplitudes from finite-volume spectra}
 \label{sec:amplitude-analysis}
 %====================================================================

The relationship between two-body scattering amplitudes and the discrete spectrum of states in a finite periodic volume is well established~\cite{Luscher:1986pf, Luscher:1990ux, Luscher:1991cf, Rummukainen:1995vs, He:2005ey, Christ:2005gi, Kim:2005gf, Guo:2012hv, Hansen:2012tf, Briceno:2012yi, Briceno:2014oea}. For an irrep $\Lambda$ of total momentum $\vec{P}$, the discrete spectrum in an $L \times L \times L$ box corresponds to the solutions of
%
 \begin{equation}
\det \left[ \mathbf{1} + i \boldsymbol{\rho}(E) \cdot \mathbf{t}(E) \cdot \Big( \mathbf{1} + i \boldsymbol{\mathcal{M}}^{\vec{P}, \Lambda}(E,L)   \Big)\right] = 0 \,,
\label{eq:qcond}
\end{equation}
%
where $\boldsymbol{\mathcal{M}}(E,L)$ is a matrix of known kinematic functions which characterize the cubic spatial volume, while $\mathbf{t}(E)$ contains the partial-wave scattering amplitudes. In general, these are matrices in the spaces of coupled-channels and partial-wave angular momentum, $\ell$, but for elastic scattering they reduce to being dense and diagonal matrices respectively in $\ell$.

Our approach follows from parameterizing the energy-dependence of the partial-wave amplitudes $t^I_\ell(s)$ for those lowest values of $\ell$ which subduce into the irrep $\vec{P}, \Lambda$. In practice for $I=1$, only $\ell=1$ is relevant over the elastic region, while for $I=0,2$, both $\ell=0$ and $\ell=2$ are considered. For a given set of parameter values in these parameterizations, the solution of Eqn.~\ref{eq:qcond} yields a discrete spectrum that can be compared to the lattice QCD computed spectrum via a correlated $\chi^2$. We form this $\chi^2$ by considering energy levels in all irreps which constrain the partial-waves for each choice of isospin, and take as the amplitude results those which minimize the $\chi^2$. Explicit expressions for the subduction of partial-waves into irreps of the relevant symmetry group are presented in Refs.~\cite{Dudek:2016cru} and \cite{Thomas:2011rh}, and further discussion of the approach, and implementation details can be found in Refs.~\cite{Briceno:2017max, woss:2020cmp, Dudek:2016cru,Briceno:2017qmb}.


The elastic scattering partial-wave amplitudes appearing in Eqn.~\ref{eq:qcond} can be parameterized by compact forms, allowing for a description of the entire lattice QCD computed spectrum in terms of a few free parameters. The resulting amplitudes can be analytically continued into the complex energy plane to search for pole singularities. A range of parameterization forms is typically used, with the spread of amplitude behaviors and pole locations providing an estimate of systematic error. Each relevant partial-wave amplitude $t^I_\ell(s)$ is parameterized in a way that respects elastic unitarity exactly, but may not necessarily respect other fundamental constraints. 

In terms of the \emph{phase-shift}, $\delta_{\ell}^{I}(s)$, elastic amplitudes can be written as
%
\begin{equation}
t_{\ell}^{I}(s)=\frac{1}{\rho(s)} \, e^{i \delta^{I}_{\ell}(s)} \sin \delta_{\ell}^{I}(s)=\frac{1}{\rho(s)} \frac{1}{\cot \delta_{\ell}^{I}(s)-i},
\label{eq:tamp}
\end{equation}
%
where $\rho(s) = 2 k/\sqrt{s}$ is the $\pi \pi$ phase-space, with ${k = \tfrac{1}{2}\sqrt{ s - 4m_\pi^2}}$ being the scattering momentum. 

In those cases where a single partial-wave only dominates Eqn.~\ref{eq:qcond}, or where the amplitudes for higher partial waves are fixed, each discrete finite-volume energy can be used to obtain a discrete value of the phase-shift at that energy. This approach is used to make the discrete phase-shift `data points' that will appear in plots later in this document. The amplitude curves are not obtained by fitting these `data', but rather using the spectrum $\chi^2$ approach described above.


At low scattering energies, the slow variation of the $S$--wave and $D$--wave can be well described by a low-order expansion in the square of the scattering momentum, typically called the \emph{effective range expansion},
%
\begin{equation}
k^{2\ell+1} \cot \delta_{\ell}^{I} =  F^I_{\ell}(s)\left( \tfrac{1}{a^I_\ell}+\tfrac{1}{2}r^I_\ell k^2+\dots\right),
\label{eq:cotere}
\end{equation}
%
where the conventional choice has $F^I_{\ell}(s) = 1$, $a_\ell^I$ interpreted as the \emph{scattering length} and  $r_\ell^I$ as the \emph{effective range}. Additional desired features can be built into the amplitude with other choices, such as $F^I_\ell(s)= \frac{4m_\pi^2 - s_A}{s-s_A}$ to ensure a zero of the amplitude, like those predicted by broken chiral symmetry known as `Adler zeroes'.

An alternative expansion follows from defining
%
\begin{equation}
\Phi^I_{\ell}(s) = \tfrac{2}{\sqrt{s}} k^{2 \ell+1} \cot \delta^I_{\ell}(s),
\end{equation}
%
which must be real analytic between the elastic threshold and the inelastic threshold. One can engineer the presence of an effective inelastic threshold ($s_0$), and the opening of the \emph{left-hand-cut} at $s = 0$ (required by crossing-symmetry), by using a \emph{conformal mapping variable}~\cite{Cherry:2000ut,Pelaez:2016tgi},
%
\begin{equation}
\omega(s) = \frac{\sqrt{s}-\alpha\sqrt{s_0-s}}{\sqrt{s}+\alpha\sqrt{s_0-s}} \, .
\end{equation}
%
In this expression $\alpha$ and $s_0$ are fixed parameters that determine what energy region is mapped into a unit disk of $\omega$~\footnote{In practice, we will set $s_0 = 0.09 \,a_t^{-2}$ and $\alpha=0.8$ for the $I=2$ waves  and the $I=0$ $D$--wave, as they do not exhibit any inelastic behavior up to high energies. We use $s_0 = 0.05 \,a_t^{-2}$ and $\alpha=1$ for the $I=1$ $P$--wave. For the $I=0$ $S$--wave, where we expect the inelasticity to become significant at a lower energy, we set $\alpha=1$, and we consider two values of $s_0 = 0.032 \,a_t^{-2},\, 0.04 \,a_t^{-2}$.}. The convergence of the conformal expansion is expected to be rapid,
%
\begin{equation}
\Phi^I_\ell(s)= F^I_{\ell}(s) \sum_{n=0}^N B_{n} \, \omega^{n} \, ,
\end{equation}
%
where, again, one may build additional properties into the amplitude by suitable choices for $F^I_{\ell}(s)$, for example $F^I_\ell(s) = \frac{s-m^2_R}{m_R^2}$, to force a resonance. As suggested in Ref.~\cite{Yndurain:2007qm}, spurious singularities introduced below threshold by this conformal expansion can be removed by adding a suitable function,
%
\begin{equation}
\Phi^I_\ell(s)= F^I_{\ell}(s) \left(\gamma^I_\ell(s)+\sum_{n=0}^N B_{n} \, \omega^{n} \right)\, .
\end{equation}



Partial-waves that contain a narrow resonance and no other features, like the experimental $I=1$ $P$--wave, are usually well-described over a limited energy region by a Breit-Wigner form, which effectively parameterizes a single nearby pole,
%
\begin{equation}
t_{\ell=1}(s)=\frac{1}{\rho(s)} \frac{\sqrt{s}\,  \Gamma(s)}{m_\mathrm{BW}^2 - s - i \sqrt{s}\,  \Gamma(s)} \, ,
\label{eq:bw}
\end{equation}
%
with the energy-dependent width, $\Gamma(s) = \tfrac{g_\mathrm{BW}^2}{6\pi} \tfrac{k^3}{s}$. The width form can be elaborated to damp out the threshold behavior at high energies (sometimes called barrier factors) at the cost of including at least one extra parameter and possibly spurious singularities.


A rather flexible parameterization scheme which generalizes nicely to the case of \emph{coupled-channel} amplitudes, uses the $K$-matrix defined in
%
\begin{equation}
\left( t^I_\ell(s) \right)^{-1} = \left( K^I_\ell(s) \right)^{-1} - i \rho(s) \, ,
\end{equation}
%
where a common parameterization choice is a sum of poles plus a finite-order polynomial,
%
\begin{equation}
K^I_\ell(s) = (2k)^{2\ell} \left[ \sum_r \frac{g_r^2}{m_r^2 - s} + \sum_p \gamma_p s^p \right] \, .
\end{equation}
%
This form can be modified to ensure an Adler zero by taking $K(s) \to (s-s_A)\,  K(s)$, and the unphysical singularity in the phase-space at $s=0$ can be removed from the physical sheet by replacing $-i\rho(s)$ with the Chew-Mandelstam function, which we present subtracted at threshold, as
%
\begin{equation}
I(s)=  I\left(4 m_\pi^2\right) +\frac{\rho(s)}{\pi} \log \left[\frac{\rho(s) +1}{\rho(s) - 1}\right] \, ,
\label{eq:chewman}
\end{equation}
which has $\mathrm{Im} \, I(s) = - \rho(s)$ as required by unitarity. 

For each partial wave, we will consider a large number of parameterizations based on the forms above, reporting all those which are found to be capable of describing the computed finite-volume spectra as established by the spectrum $\chi^2$ value.

For every amplitude parameterization, once the parameters are constrained by describing the lattice QCD spectra, we search the second Riemann sheet for poles that we interpret as being due to resonances. The pole locations provide a model-independent definition of a mass and width for the resonance, $\sqrt{s_R} = m_R-i\,\Gamma_R/2$, and the corresponding residue in $t^I_\ell(s) \sim c^2/(s_R - s )$, gives a coupling of the resonance to $\pi\pi$. An alternative definition of the $\pi\pi$ coupling, as presented in Ref.~\cite{Garcia-Martin:2011nna}, is related to ours by,
%
\begin{equation}
g_{\pi\pi}^2 = 16\pi \frac{2\ell+1}{(2k_R)^{2\ell}} \, c^2 . \label{gNorm}
\end{equation}
%

We will find, as has been observed in previous lattice calculations~\cite{Dudek:2012xn, Wilson:2015dqa, Briceno:2017qmb, Wilson:2019wfr, Woss:2019hse, Johnson:2020ilc}, and in amplitude analyses of experimental data~\cite{Au:1986vs,JPAC:2017dbi,JPAC:2018zyd,Rodas:2021tyb}, that when a narrow resonance is present the pole position and coupling typically show very little scatter over a range of sufficiently flexible parameterizations, but when a resonance pole lies far into the complex plane, different amplitudes which behave similarly in a limited energy region on the real energy axis (and which describe the finite-volume spectra equally well in the lattice case) can lead to quite different pole locations~\cite{Pelaez:2015qba,Ropertz:2018stk,Pelaez:2020gnd}. We will return to this point later when discussing the $\sigma$ pole appearing in the isospin--0 \mbox{$S$--wave}.

