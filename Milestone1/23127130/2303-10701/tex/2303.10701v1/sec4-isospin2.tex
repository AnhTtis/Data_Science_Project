%====================================================================
\section{$\pi \pi \to \pi \pi$ $I=2$}
 \label{sec:I2pipi}
%====================================================================


%%% I2 spectrum out of section to handle floats
% 
\begin{figure*}
\begin{center}
%
\resizebox{\textwidth}{!}{
  \includegraphics{fig1a_spectrum_isospin2_850_Dwave.pdf}%{../figures/850-I2-spec2.pdf} 
  \includegraphics{fig1b_spectrum_isospin2_856_Dwave.pdf}%{../figures/856-I2-spec2.pdf} 
}
%
\resizebox{\textwidth}{!}{
  \includegraphics{fig1c_spectrum_isospin2_850_SDwave.pdf}%{../figures/850-I2-spec1.pdf} 
  \includegraphics{fig1d_spectrum_isospin2_856_SDwave.pdf}%{../figures/856-I2-spec1.pdf} 
}
%
\caption{
$I=2$ finite-volume spectra, $a_t E_\mathsf{cm}$, by irrep, against $L/a_s$, for $m_\pi \sim 330$ MeV (left) and $m_\pi \sim 283$ MeV (right). Upper panel irreps have $D$-wave as their leading partial-wave, while those in the lower panel have $S$-wave leading. Red curves show $\pi\pi$ non-interacting energies.
}
\label{fig:I2}   
\end{center}
\end{figure*}
%



As in experiment, previous determinations in lattice QCD at various pion masses (e.g. ~\cite{Beane:2007xs,Dudek:2010ew,NPLQCD:2011htk,Dudek:2012gj,Kurth:2013tua,Fu:2013ffa,Bulava:2016mks}) have found $\pi\pi$ scattering in isospin--2 to be weak and repulsive. Lattice calculations of this channel typically use a basis of operators resembling a pair of pions only, since $q\bar{q}$ operators cannot access $I=2$. The lowest inelastic channel is $\pi\pi\pi\pi$, but expectations from experiment are that the coupling of this channel to $\pi\pi$ turns on very slowly~\cite{Losty:1973et,Durusoy:1973aj,Hoogland:1977kt}.





 %====================================================================
\subsection{$I=2$ finite-volume spectra}
\label{subsec:I2FV}
%====================================================================

Using bases of $\pi\pi$ operators as described in Section~\ref{sec:latt}, matrices of correlation functions are computed, and variational analysis leads to the spectra shown in Figure~\ref{fig:I2}. Departure of the discrete energy levels from the values for non-interacting $\pi\pi$ pairs can be observed, being much larger in those irreps which feature a subduction of the $S$--wave. 


The first inelastic threshold here is $\pi\pi\pi\pi$, indicated in Figure~\ref{fig:I2} by the horizontal dashed line. We have not included any $\pi\pi\pi\pi$-like operator constructions in our basis, so the spectrum presented above the inelastic threshold will only be a correct subset of the complete true spectrum in the case that the $\pi\pi$ and $\pi\pi\pi\pi$ channels are decoupled. In experiment, this is indeed the case until quite high energies, well above those considered here~\cite{Losty:1973et,Durusoy:1973aj,Hoogland:1977kt}.



The determined energies have fractional errors typically at the 0.5\% level, where this includes an estimate of systematic error coming from varying fitting details, the precise set of high-lying operators used in the variational diagonalization, and whether a ``weighting-shifting'' step (see Ref.~\cite{Dudek:2012gj}) is applied to cancel mild finite-time-extent effects. These systematic variations impact at a level below the statistical error on most points. Across all irreps, we extract 50 energy levels for $m_\pi \sim 330$ MeV, and 98 for $m_\pi \sim 283$ MeV, of which 19 and 31 are below the $\pi\pi\pi\pi$ threshold, respectively.



%
%figure 2
%
\begin{figure}[h]
%
  \includegraphics[trim=0 6 0 3, width=\columnwidth, clip]{fig2a_delta_isospin2_850_Dwave.pdf}
  %{../figures/850-I2-Dwave.pdf}
%
  \includegraphics[trim=0 10 0 3, width=\columnwidth, clip]{fig2b_delta_isospin2_856_Dwave.pdf}
  %{../figures/856-I2-Dwave.pdf}
%
\caption{
$I=2$ $D$-wave phase-shift for $m_\pi \sim 330$ MeV (top) and $m_\pi \sim 283$ MeV (bottom). Discrete data points come from irreps in which $\ell=2$ is the lowest subduced partial-wave, assuming $\ell\ge 4$ scattering to be negligible. Curves show two illustrative parameterizations: a scattering length form (red) and a conformal mapping with two terms (black), fitted to the full set of energy levels below $\pi\pi\pi\pi$ threshold. The shaded region indicates energies above the $\pi \pi \pi \pi$ threshold. }
%
\label{fig:I2Dfit}   
%
\end{figure}
%


%
%figure 3
%
\begin{figure*}
\begin{center}
%
\resizebox{\textwidth}{!}{
  \includegraphics{fig3a_delta_isospin2_850_Swave.pdf}%{../figures/850-I2-Swave.pdf} 
  \hspace{.1cm} 
  \includegraphics{fig3b_delta_isospin2_856_Swave.pdf}%{../figures/856-I2-Swave.pdf} 
  }
\resizebox{\textwidth}{!}{
  \includegraphics{fig3c_kcotd_isospin2_850_Swave.pdf}%{../figures/850-I2-Swave_kcot.pdf} 
  \hspace{.1cm} 
  \includegraphics{fig3d_kcotd_isospin2_856_Swave.pdf}%{../figures/856-I2-Swave_kcot.pdf} 
}
%  
\caption{$I=2$ $S$--wave scattering for $m_\pi\sim330$ MeV (left) and $m_\pi\sim283$ MeV (right). Four example parameterizations are shown: a two-term conformal mapping (black), an effective range expansion with two terms (green), and two choices with an Adler zero fixed at the leading order $\chi$PT location, a two-term conformal mapping (red), and an effective range expansion with two terms (orange).  The enforced presence of the Adler zero can be seen in the deviation of the red and orange curves from nearly flat behavior at threshold in the lower panels. Discrete `data' points with large uncertainties have been removed from the plot for clarity. The shaded region indicates energies above the $\pi \pi \pi \pi$ threshold. }
\label{fig:I2Sfit}   
\end{center}
\end{figure*}




 %====================================================================
\subsection{$\pi \pi \to \pi \pi$ $I=2$ scattering}
\label{subsec:I2}
%====================================================================


The spectra presented in the previous section can be used to constrain $S$--wave and $D$--wave elastic scattering amplitudes~\footnote{The spectra are compatible with the amplitudes for $\ell \ge 4$ being zero throughout the energy region considered.} using the approach described in~\cref{sec:amplitude-analysis}. Examining energy levels in those irreps whose lowest subduced partial-wave is $\ell=2$, we observe extremely small energy shifts from the non-interacting curves that suggest a very weak interaction. 

Descriptions in terms of parameterizations featuring only a single free parameter, such as a scattering length, lead to good descriptions of the spectra, and as can be seen in Figure~\ref{fig:I2Dfit}, clearly describe a very weak $D$--wave interaction. Adding further parameter freedom does not lead to an improved description of the spectra. 


The spectra shown in the lower row of Figure~\ref{fig:I2} are for those irreps in which the $S$--wave is present. These are included together with the spectra in the top row in a $\chi^2$ to obtain descriptions of the $S$-- and $D$--wave amplitudes simultaneously. The $S$--wave amplitudes for several sample parameterization choices are shown in Figure~\ref{fig:I2Sfit}. The principal difference between these various descriptions of the finite-volume spectrum, most of which have $\chi^2/N_\mathrm{dof}\sim 1$, can be observed to be at threshold where the slope of the phase-shift curve, and hence the scattering length, appears to be poorly constrained. This is more clearly seen in the plots of $k \cot \delta^2_0$, where for both pion masses a spread of behaviors at threshold, well outside the statistical uncertainty, is observed. The behaviors fall into two broad categories -- amplitudes where $k \cot \delta^2_0$ is fairly flat at threshold correspond to those which have not been engineered to have an Adler zero below threshold, unlike those which fall at threshold, where an Adler zero was included at the tree-level $\chi$PT location, $s_A = 2 m_\pi^2$. 


Given that the pion masses used in this study are further from the chiral limit than the experimental pion mass, we expect corrections to the tree-level location of an Adler zero that may be significant. As was shown in Ref.~\cite{Garcia-Martin:2011iqs}, dispersive analyses of experimental data suggest that even for the physical pion mass the Adler zero may be displaced from the tree-level location. Motivated by this result, we take the range produced by the ``CFD" dispersive predictions in Ref.~\cite{Garcia-Martin:2011iqs}, and extrapolate it to the pion masses used herein using $s_A = s_A^\mathrm{phys} \left(m_\pi/m_\pi^\mathrm{phys}\right)^2$. We consider descriptions of the finite-volume spectra using amplitudes with Adler zeros fixed at the extremes suggested by this approach, together with some amplitudes for which the Adler zero is allowed to float freely, although these latter choices lead to statistically imprecise results for the amplitude. 



We plot in~\cref{fig:I2S_SL} the values of $S$-wave scattering length extracted from all parameterizations which provide a reasonable description of the finite-volume spectra, separated between those parameterizations with an Adler zero (of varying location) and those without -- a clear systematic difference is observed, indicating a strong correlation between the location of a subthreshold zero and the value of the scattering length when constrained by only finite-volume energy levels above threshold.


In Figure~\ref{fig:I2ChPT} we show the pion mass evolution of the $I=2$ $S$-wave scattering length across four pion masses computed with the same lattice action. The rightmost point, taken from Ref.~\cite{Dudek:2012gj} reflects an average over several parameterizations in which Adler zeroes were not enforced. The leftmost points, at $m_\pi \sim 239$ MeV, show an analysis of the same type followed by this paper that has not been previously published~\footnote{Details of this analysis will be provided in a forthcoming publication.}. It is clear that the slope of the variation with pion mass is very sensitive to the existence, or not, of an Adler zero. We return in the conclusions of this paper to the question of whether the presence and exact location of the Adler zero, which lies far from the region of constraint provided by finite-volume energy levels, can be resolved using only lattice QCD data.




%figure 4
\begin{figure}
\begin{center}
%
  \includegraphics[width=\columnwidth]{fig4a_scatlen_isospin2_850_Swave.pdf}%{../figures/850_S2_SL} 
  
  \includegraphics[width=\columnwidth]{fig4b_scatlen_isospin2_856_Swave.pdf}%{../figures/856_S2_SL.pdf} 
%  
\caption{ Extracted scattering length for a range of $I=2$ $S$--wave amplitude parameterizations for $m_\pi\sim330 \, \mathrm{MeV}$ (top) and ${m_\pi\sim283\, \mathrm{MeV}}$ (bottom). Each amplitude is labelled by the $\chi^2/N_\mathrm{dof}$ with which it describes the finite-volume spectrum. Red points correspond to amplitudes containing an Adler zero at some location, while blue points lack any enforced subthreshold zero. }
\label{fig:I2S_SL}   
\end{center}
\end{figure}



%figure 5
\begin{figure}[!ht]
\begin{center}
%
\includegraphics[width=\columnwidth]{fig5_scatlen_isospin2_mpi.pdf}%{../figures/I2_SL_quark_mass.pdf}
% 
\caption{$I=2$ $S$--wave scattering length extracted from parameterizations describing finite-volume spectra at four pion masses. Red points indicate amplitudes that feature an Adler zero, while blue points lack an enforced subthreshold zero. The result of dispersive analysis applied to experimental data~\cite{Garcia-Martin:2011iqs} is shown by the gray point.
}
\label{fig:I2ChPT}   
\end{center}
\end{figure}
