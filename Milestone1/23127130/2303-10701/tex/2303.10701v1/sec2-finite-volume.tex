%====================================================================
\section{Lattices and operator construction}
 \label{sec:latt}
 %====================================================================

\begin{table*}
\begin{tabular}{rrlccc|lllll|c}
 
  $-a_t m_\ell$ & $(L/a_s)^3$ & $\times \; T/a_t$ & $N_{\mathrm{cfgs}}$ & $N_\mathrm{vecs}$ & $N_\mathrm{t_{src}}$   & \multicolumn{1}{c}{$a_t m_\pi$} & \multicolumn{1}{c}{$a_t m_K$} & \multicolumn{1}{c}{$a_t m_\eta$} & \multicolumn{1}{c}{$a_t m_\Omega$} & \multicolumn{1}{c|}{$\xi$} & $m_\pi$/MeV\\[0.2ex]
\hline

 $0.0850$ & $24^3$ &$\times$ 256 & 473 & 160 & 4--16  & 0.05635(14) & 0.09027(15) & 0.09790(100)         & 0.2857(8)  & 3.467(8)     & 330  \\
 $0.0856$ & {\footnotesize {$24^3$,\! $32^3$}} &$\times$ 256 & {\footnotesize 392,\! 475} & {\footnotesize 160,\! 256} & 4--8  & 0.04720(11) & 0.08659(14) & 0.09602(70) & 0.2793(8)  & 3.457(6)     & 283  \\
\end{tabular}
\caption{
Anisotropic three-flavor lattices used in this paper. Anisotropy values, $\xi$, are obtained from the pion dispersion relation. $N_{\text{vecs}}$ indicates the number of distillation vectors used in the construction of correlation functions, and $N_\mathrm{t_{src}}$ the number of $0 \to t$ perambulator time-sources averaged over~\cite{HadronSpectrum:2009krc}.
}
\label{tab:lattices}
\end{table*}



The calculations described in this manuscript make use of anisotropic Clover lattices~\cite{Edwards:2008ja, HadronSpectrum:2008xlg} whose parameters are presented in Table~\ref{tab:lattices}. These three-flavor lattices, which have $a_s \sim 0.12 \, \mathrm{fm}$, degenerate light quarks, and a strange quark mass approximately tuned to the physical strange quark mass, were previously used in calculations of the $\pi K$ system~\cite{Wilson:2019wfr,Radhakrishnan:2022ubg}~\footnote{
The pion masses have been recomputed with greater statistics since that previous paper, and herein are referred to as 283, 330 MeV, which correspond to the 284, 327 MeV lattices therein.
}.


 

In order to determine $\pi\pi$ scattering amplitudes, we require the spectra of states with the appropriate quantum numbers in the finite spatial volume of the lattice. These spectra are extracted using variational analysis of matrices of two-point correlation functions computed using a basis of operators at source and sink. A basis that has proven successful in prior calculations~\cite{Dudek:2012gj,Dudek:2012xn,Dudek:2014qha,Wilson:2014cna,Briceno:2016mjc,Briceno:2017qmb} makes use of `single-hadron' operators (in isospin--0 and isospin--1) of fermion bilinear type, $\bar{\psi} \Gamma \overleftrightarrow{D} \ldots \overleftrightarrow{D} \psi$, supplemented by operators resembling a pair of mesons having definite total momentum, and magnitude of relative momentum,
%
\begin{equation}
	\big( \Omega\,\Omega \big)_{\vec{P}, \Lambda,\mu}^{\dag\,\,[\vec{p}_1, \vec{p}_2]} = \sum
	_{\substack{ \vec{p}_1 + \vec{p}_2 = \vec{P} }}  \mathcal{C}(\vec{P},\Lambda,\mu; \vec{p}_1; \vec{p}_2 )\, \Omega^\dag(\vec{p}_1)\, \Omega^\dag(\vec{p}_2).
\end{equation} 
%
The operators appearing in the product on the right-hand side are selected to be those linear combinations of `single-hadron' operators that optimally overlap with the pion states in the variational analysis of correlation functions with the quantum numbers of a single pion. The `lattice Clebsch-Gordan coefficients' in this equation ensure that the operator transforms in a definite irreducible representation, $\Lambda$, of the relevant lattice symmetry group. Systems of definite angular momentum subduce into these `irreps' according to Table~II of Ref.~\cite{Dudek:2012gj} for $I=0,\,2$, and according to Table~III of Ref.~\cite{Dudek:2012xn} for $I=1$. In this paper, we will consider irreps with total momentum $|\vec{P}|^2 \le 4 \left(2\pi/L\right)^2$.


Use of the distillation framework~\cite{Peardon:2009gh} allows for efficient computation of a large number of correlation functions, and in particular allows diagrams featuring quark-antiquark annihilation (of which there are many in the isospin--0 case) to be evaluated without further approximation~\cite{Briceno:2016mjc,Briceno:2017qmb}.


While our primary focus is on $\pi \pi$ elastic scattering, in order to have a reliable evaluation of the finite-volume spectra in the energy region where the $K\bar{K}$ and $\eta\eta$ thresholds open, we have included, where relevant, also $K\bar{K}$--like and $\eta\eta$--like operators into our basis. We are guided as to which relative momentum combinations to include in the basis by the \emph{non-interacting} energy of the operator, $E_\mathrm{n.i.} = \sqrt{ m^2 + |\vec{p}_1|^2} + \sqrt{ m^2 + |\vec{p}_2|^2}$, where $m$ is the mass of the meson ($\pi, K, \eta$). All operator constructions are included which have non-interacting energy in the energy region we intend to study. 




The matrices of correlation functions computed in the large basis indicated above are analyzed using a variational approach based upon solving a generalized eigenvalue problem. Our primary interest is in the spectrum which is obtained by fitting the exponential time-dependence of the extracted eigenvalues. In order to account for the impact of choice of fitting window and number of exponentials, we implement a version of the ``model averaging" technique proposed in Ref.~\cite{Jay:2020jkz}, as described in Ref.~\cite{Radhakrishnan:2022ubg}. In addition, the sensitivity of the extracted energy levels to the choice of the reference timeslice $t_0$ in the generalized eigenvalue problem, and to the precise choice of operators in the basis is explored and reflected in the quoted energy values and uncertainties.
 

When dimensionful quantities are required, the lattice scale is set using the $\Omega$ baryon mass computed on the relevant lattice, $a_t=\frac{a_t m_{\Omega}}{m_{\Omega}^{\text {phys }}}$, where the physical mass of the $\Omega$ baryon is $m_{\Omega}^{\text {phys }}=1672.5$ MeV. The quoted pion mass in MeV for each lattice follows from use of this prescription.


