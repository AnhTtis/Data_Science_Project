%====================================================================
\section{Conclusions and outlook}
 \label{sec:summary}
 %====================================================================



We have reported on the extraction of $\pi\pi$ elastic scattering amplitudes across all three isospins for low partial-waves, supplementing earlier calculations on anisotropic lattices with $m_\pi \sim 391, 239$ MeV, with new calculations at two interpolating pion masses, 330 and 283 MeV. The new pion masses explore the region where we expected the $\sigma$ state appearing in the $I=0$ \mbox{$S$--wave} to transition from a bound state into a resonance. We continued the philosophy used in prior publications~\cite{Dudek:2012gj,Dudek:2012xn, Dudek:2014qha,Wilson:2014cna,Wilson:2015dqa, Dudek:2016cru, Briceno:2016mjc, Briceno:2017qmb, Woss:2018irj, Wilson:2019wfr, Woss:2019hse, Woss:2020ayi, Johnson:2020ilc} of exploring a wide range of amplitude parameterizations to test the uniqueness of the lattice QCD spectrum constrained amplitudes and their resonance content.


The isospin--2 $S$--wave at both new pion masses is found to be weak and repulsive, and we isolated a sensitivity in the extracted value of the scattering length to whether amplitude parameterizations contain a subthreshold zero like an Adler zero. The isospin--1 $P$--wave is dominated by an isolated narrow $\rho$ resonance, and we were able to establish the trajectory of the corresponding resonance pole through the complex plane as the pion mass varies. The corresponding coupling of the resonance to $\pi\pi$ was found to be essentially quark mass independent.

The isospin--0 $S$--wave, which houses the $\sigma$, shows the most dramatic change between the two pion masses considered. At $m_\pi \sim 330$ MeV the phase-shift is relatively flat over the entire region, and is found to feature a bound-state $\sigma$ with a binding energy of only about \mbox{3 MeV}, while at $m_\pi \sim 283$ MeV the phase-shift rises slowly from $0^\circ$ caused by the $\sigma$ being either a virtual bound-state or a subthreshold resonance. We conclude that the $\sigma$ undergoes a transition from being a bound state to being a virtual bound state somewhere between $m_\pi \sim 283$ MeV and $m_\pi \sim 330$ MeV. 

The very different quark mass evolutions observed for the vector $\rho$ and the scalar $\sigma$ agree with the general arguments that a $P$--wave resonance can only become stable by having the complex conjugate resonance pole pair coalesce at the threshold, while an $S$--wave state need not meet this requirement. Once the pair of $S$--wave poles meet on the real axis below threshold, they evolve differently, with one of them approaching threshold as the quark mass increases. In those lattices where we find a bound $\sigma$, the pole closest to threshold determines the low energy behavior of the partial-wave.



The fact that we are unable to state with certainty whether the $\sigma$ at $m_\pi \sim 283$ MeV is a virtual bound-state or a subthreshold resonance reflects the same problem that was previously reported for the $\sigma$ at $m_\pi \sim 239$ MeV in Ref.~\cite{Briceno:2016mjc}, where equally good amplitude descriptions of the finite-volume spectrum have poles in locations scattered across the complex plane.

The inability of even large numbers of high-precision lattice QCD energy levels to uniquely pin down the $\sigma$ pole location, and also to determine the location of Adler zeroes in the $I=0$ and $I=2$ $S$--wave amplitudes, are problems that most likely have a common origin. In both cases we are required to analytically continue relatively far from where the amplitudes are constrained, which is over a limited section of the real energy axis mainly above threshold. 

We propose that a solution is to apply additional theoretical constraints onto the amplitudes. In particular, the behavior of any fixed isospin partial-wave amplitude for $s<0$ is controlled by partial waves in all isospins by virtue of \emph{crossing symmetry}, and dispersion relations allow us to make practical use of this symmetry, while also ensuring good analytic properties of the amplitudes. Since we have computed amplitudes with all isospins on the same lattices in this paper, we can envisage applying a dispersion relation analysis approach to more accurately constrain partial-wave amplitudes. We are pursuing such an approach, and a publication is in preparation.


