% !TeX root = paper.tex
% !Tex program=pdflatex
% !TeX spellcheck = en_US

\section{Conclusion}
This paper has studied the task of learning a deterministic finite automaton from a sample of unlabeled words that could be used to separate normal form anomalous words.
We proposed three unsupervised learning settings, studied their properties (e.g., their computational complexity), and developed  learning algorithms that utilize off-the-shelf constraint optimization tools.
In addition, we have shown how regularization can improve the interpretability of the learned DFAs.
Our empirical evaluation has demonstrated practical feasibility in the context of three anomaly detection benchmarks.

We see various promising directions for future research.
First, the analysis of the complexity of the third learning setting remains an open problem to be tackled in the future. 
Second, we plan to develop heuristics that sacrifice the optimality of a solution in favor of computational efficiency.
Third, we want to extend our approach to more expressive automata classes, for instance, register automata, to handle data over continuous domains.