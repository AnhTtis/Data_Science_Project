\documentclass[twocolumn,superscriptaddress]{revtex4-2}
\usepackage[utf8]{inputenc}
\usepackage{times}
\usepackage[dvips]{graphicx}
\usepackage[usenames,dvipsnames]{xcolor}
\usepackage{amsmath}
\usepackage{amssymb}
\usepackage{bm}

\DeclareUnicodeCharacter{3000}{\textcolor{red}{BAD!!}}

\newcommand{\fmcomment}[1]{{\color{blue}[#1]}}
\newcommand{\fmmodify}[1]{{\color{magenta}#1}}
\newcommand{\figref}[1]{{\color{orange}\ref{#1}}}
\newcommand{\Revised}[1]{{\color{black}#1}}
\newcommand\hl[1]{{\color{red} #1}}

\DeclareMathOperator{\erfc}{erfc}

\begin{document}

\title{Brownian yet non-Gaussian diffusion of a light particle in heavy gas: Lorentz gas based analysis}

\author{Fumiaki Nakai}
\email{nakai.fumiaki.c7@s.mail.nagoya-u.ac.jp}
\affiliation{Department of Materials Physics, Graduate School of Engineering, Nagoya University, Furo-cho, Chikusa, Nagoya 464-8603, Japan}

\author{Takashi Uneyama}
\affiliation{Department of Materials Physics, Graduate School of Engineering, Nagoya University, Furo-cho, Chikusa, Nagoya 464-8603, Japan}

\begin{abstract}
Non-Gaussian diffusion was recently observed in a gas mixture with mass and fraction contrast [F. Nakai et al, Phys. Rev. E {\bf 107}, 014605 (2023)].
The mean square displacement of a minor gas particle with a small mass is linear in time, while the displacement distribution deviates from the Gaussian distribution, which is called the Brownian yet non-Gaussian diffusion.
In this work, we theoretically analyze this case where the mass contrast is sufficiently large.
Major heavy particles can be interpreted as immobile obstacles, and a minor light particle behaves like a Lorentz gas particle within an intermediate time scale.
Despite the similarity between the gas mixture and the conventional Lorentz gas system, the Lorentz gas description cannot fully describe the Brownian yet non-Gaussian diffusion.
A successful description can be achieved through a \Revised{canonical} ensemble average of the statistical quantities of the Lorentz gas over the initial speed.
\end{abstract}
\maketitle

\section{Introduction}
\label{sec:introduction}

Gas diffusion is a classical problem \cite{chapman1990, jeans1921, mazenko2008, dorfman2021}, and it may be considered to be fully understood nowadays.
However, recent work revealed that gas diffusion is not that simple nor fully understood \cite{nakai2023fluctuating}.
The authors\cite{nakai2023fluctuating} numerically investigated the diffusion of a light gas particle in gas mixtures with mass and fraction contrast and found that the minor light molecule exhibits Brownin yet non-Gaussian diffusion: the mean square displacement (MSD) is linear in time \Revised{$\mathrm{MSD}\sim t^1$}, while the displacement distribution deviates from the Gaussian distribution.
\Revised{(We do not consider the anomalous diffusion $\mathrm{MSD}\sim t^{\delta}$ ($\delta\ne 1$ ) accompanied by a non-Gaussian displacement distribution, in what follows. Such diffusion is non-Brownian and non-Gaussian, while it is often observed in glassy liquids \cite{chaudhuri2007universal}, polymeric liquids\cite{ge2018nanorheology}, or some complex systems.)}
The Brownian yet non-Gaussian diffusion has been widely observed in complex systems with heterogeneous environments and/or conformational degrees of freedom, such as glass-forming liquids \Revised{\cite{rusciano2022fickian, chaudhuri2007universal, miotto2021length}}, polymeric fluids \cite{miyaguchi2017elucidating, uneyama2015fluctuation}, colloidal suspensions \cite{kim2013simulation, guan2014even}, \Revised{confined systems \cite{alexandre2023non}}, biological systems \cite{wang2009anomalous, he2016dynamic, jeon2016protein}, and active matters \cite{leptos2009dynamics, kurtuldu2011enhancement}.
In contrast to these complex systems, there is no heterogeneity nor internal degrees of freedom in the gas mixture.
In the previous work \cite{nakai2023fluctuating}, the origin of the non-Gaussian behavior was attributed to the fluctuating diffusivity which arises from a \Revised{separation of two relaxation timescales of the minor light particle velocity.
The relaxation timescale for speed (the magnitude of the velocity)
can be much longer than that for the direction of the velocity.
Namely, when the velocity is described in the spherical coordinates, the polar and azimuthal angle components rapidly relax at the time scale where the radial component almost remains unchanged.
}


The dynamics of a light particle in heavier particles have often been approximated as the Lorentz gas \cite{machta1983diffusion, andersen1964relaxation, dorfman2021, lebowitz1978transport}, which is composed of a mobile particle and immobile particles.
Although Lorentz gas was originally constructed for the dynamics of an electron in metal, later, the model has been regarded as the simple model for the transport phenomena of gas \cite{andersen1964relaxation} and also for some classical dynamical systems \cite{klages2019normal, zeitz2017active}.
Various transport properties of the Lorentz gas including the diffusion coefficient \cite{bruin1974computer, machta1983diffusion} or the relaxation of the velocity \cite{andersen1964relaxation, machta1983diffusion} are analyzed \Revised{even for dense cases \cite{hofling2006localization, hofling2007crossover, hofling2008critical, hofling2013anomalous}}.
Diverse extended models \Revised{\cite{klages2019normal, zeitz2017active, leitmann2017time, sanvee2022normal} including the experimental systems \cite{schnyder2017dynamic, siboni2018nonmonotonic, skinner2013localization}, glass forming liquids\cite{krakoviack2009tagged,voigtmann2009double}} have also been extensively studied.
Naively, we expect that the diffusion of a light gas particle in the heavy gas particles can be described using Lorentz gas.
To the best of the authors' knowledge, the relation between the simple dilute Lorentz gas and diffusion of a light particle in a matrix of heavy gas particles is not clear, and whether the non-Gaussian behavior can be described by the Lorentz gas is not clear, neither.


In the current work, we theoretically analyze the diffusion of a light gas particle in heavy gas particles.
We employ the dilute random Lorentz gas to describe the Brownian yet non-Gaussian diffusion of a light particle in heavy gas.
We first derive analytical expressions for the MSD and the non-Gaussian parameter of the Lorentz gas using the point process \cite{cox1980point}.
\Revised{(One can also calculate these correlation functions using the Boltzmann equation \cite{dufty2005brownian,brey1999brownian,garzo2019granular} and will reach the same results.)}
Afterward, we calculate the \Revised{canonical} ensemble average of them over the initial speed, which obeys the Maxwell-Boltzmann distribution.
The averaged result quantitatively reproduces the Brownian yet non-Gaussian diffusion of a light and minor gas particle in binary gas mixtures within an intermediate timescale.


\begin{figure}
    \centering
    \includegraphics[width=0.35\textwidth]{fig_lorentz_gas.eps}
    \caption{
Binary gas mixture as a Lorentz gas, which consists of a single mobile particle (point mass) in immobile spherical particles. $\bm{v}_i$ denotes the mobile particle velocity after $i$-th collision, and $\bm{u}_i$ is the unit vector from the mobile particle to the colliding immobile particle.}
    \label{fig:lorentz_gas}
\end{figure}


\section{Model}
\label{sec:model}

We consider a system that consists of the mobile particle with mass $m$ and size $0$ (point mass) in the fixed spherical obstacles with radius $\sigma$.
Note that we introduce the mass $m$ to analyze the binary gas mixture, although the conventional Lorentz gas model is independent of $m$.
We limit ourselves to the case where the fixed obstacles are dilute.
The fixed obstacles are randomly distributed in space, and there is no statistical correlation between different obstacles.
We describe the number density of the obstacles as $\rho$.
The mobile particle ballistically moves until it collides with an immobile obstacle.
When the mobile particle collides with the obstacle, the velocity of the mobile particle instantaneously changes via the hard-core repulsion potential \cite{dorfman2021}.
If we describe the velocities of the mobile particle before and after the $i$-th collision as $\bm{v}_{i}$ and $\bm{v}_{i + 1}$, they are related as
\begin{equation}
    \bm{v}_{i+1}=(\bm{1}-2\bm{u}_i \bm{u}_i) \cdot \bm{v}_{i}
    \label{eq:velocity_change}
\end{equation}
where $\bm{u}_i$ is the unit vector connecting the mobile particle to the colliding immobile particle, and $\bm{1}$ is the unit tensor.
See Fig.~\ref{fig:lorentz_gas} for a schematic representation of our model.
From Eq.~\eqref{eq:velocity_change}, the speed of the mobile particle $v$ remains unchanged: $v = |\bm{v}_i|$ (for any $i$).
\Revised{Here, we assume that the stochastic process of the mobile particle obeys the Markovian process; the successive collisions are uncorrelated.
This assumption is sufficient to describe the Brownian yet non-Gaussian diffusion that emerges in the binary gas mixture \cite{nakai2023fluctuating}.
(Due to this Markovian assumption, our model cannot reproduce so-called the
long-time tail, which is caused by correlated collisions \cite{alder1978long,brey1983long,ernst1971long,machta1984long}.)}
If we choose $m$, $v$, and $1/\rho\sigma^2$ to define the dimensionless units, there are no parameters in the random dilute Lorentz gas.

The model explained above can be interpreted as an approximate description for a minor and light particle in binary gas mixtures with sufficiently large mass and fraction contrast.
In such a binary gas mixture, the heavy particles are not entirely immobile but can move very slowly.
This means that the speed of the mobile particle is not exactly constant but just approximately constant at a specific time scale.
\Revised{In this sense, Eq.~\eqref{eq:velocity_change} is not exact
for a light particle but it should be interpreted as an approximation.}
Later, we will discuss a relation between the Lorentz gas and the \Revised{minor light molecule in the binary gas mixture.}


\section{Theory and Discussions}

\subsection{Lorentz Gas}
\label{subsec:lorentz_gas}

The dynamics of the mobile particle are described using $\bm{v}_i$ and $t_i$, where $t_i$ is the collision time at the $i$-th collision.
Using $\bm{v}_i$ and $t_i$, the position of the mobile particle at the time $t$ after the $n$-th collision, $\bm{r}(n,t)$, can be described as
\begin{equation}
 \bm{r}(n, t)=\Revised{(t-t_n)\bm{v}_n+\sum_{i=0}^{n-1} (t_{i+1}-t_i)\bm{v}_i}.
 \label{eq:position}
\end{equation}
To describe the dynamics, we require the collision statistics between successive collisions.
In the current case, the collision frequency density $f(\bm{u})$ for a collision with a direction vector $\bm{u}$ thoroughly characterizes the collision statistics.
$f(\bm{u})$ is obtained from the collision statistics for the binary gas mixture\cite{nakai2023fluctuating, mazenko2008} by setting the surrounding gas velocity to be $0$:
\begin{equation}
    f(\bm{u}_i)=\rho\sigma^2\bm{v}_i\cdot\bm{u}_i
    \Theta (\bm{v}_i\cdot\bm{u}_i),
    \label{eq:frequency_n}
\end{equation}
where $\Theta(x)$ is the Heaviside step function.
Eq.~\eqref{eq:frequency_n} can be rewritten into a simple form in the spherical coordinates.
Without loss of generality, we can take the Cartesian coordinates for $\bm{v}_{i}$ as $\bm{v}_i=(0,0,v)$ and express $\bm{u}_{i}$ as $\bm{u}_i=(\sin\theta_i\cos\phi_i, \sin\theta_i\sin\phi_i, \cos\theta_i)$ with $\theta_i\in [0, \pi/2)$ and $\phi_i\in[0, 2\pi]$.
Then Eq.~\eqref{eq:frequency_n} reduces to
\begin{equation}
    f(\bm{u}_i)=\rho\sigma^2 v\cos\theta_i
    \label{eq:frequency_n_reduce}
\end{equation}
By integrating Eq.~\eqref{eq:frequency_n_reduce} over $\bm{u}_{i}$, we obtain the collision frequency as
\begin{equation}
    \int f(\bm{u}_i) d\bm{u}_i = \frac{v}{\lambda}
    \label{eq:frequency}
\end{equation}
where $\lambda=1/\pi\rho\sigma^2$ is the mean free path.
Combining Eqs.~\eqref{eq:frequency_n_reduce} and  \eqref{eq:frequency}, we obtain the probability density where the mobile particle collides at $\bm{u}_i$ at the time interval $t_{i+1}-t_i$ for a given $v$, $P(t_{i+1}-t_i, \bm{u}_i; v)$ as \cite{nakai2023fluctuating}
\begin{equation}
    P(t_{i+1}-t_i, \bm{u}_i; v)=f(\bm{u}_i)\exp\left[-(t_{i+1}-t_i)\frac{v}{\lambda}\right].
    \label{eq:collision_statistics}
\end{equation}


The probability density at time $t$ with the number of collisions $n$ and the speed $v$ can be calculated using Eq.~\eqref{eq:collision_statistics}:
\begin{equation}
 \label{eq:collision_probability_full}
\begin{split}
    P(\bm{r}, \{\bm{u}_i \}, \{t_i\}; n, t, v)=&\delta[\bm{r}- \bm{r}(n, t)] \\
    &\times \prod_{i=0}^{n}P(t_{i+1}-t_i, \bm{u}_i, v).
\end{split}
\end{equation}
Integrating Eq.~\eqref{eq:collision_probability_full} over $\lbrace \bm{u}_{i} \rbrace$ and $\lbrace t_{i} \rbrace$, we have the probability density for $\bm{r}$ under given $n$, $t$, and $v$:
\begin{equation}
 \label{eq:collision_probability_reduced}
\begin{split}
    &P(\bm{r}; n, t, v)\\
    =&\int d\bm{v}_0
    \int_t^{\infty}dt_{n+1}
    \int_0^{t}dt_{n}
    \int_0^{t_n}dt_{n-1}
    \cdots
    \int_0^{t_2}dt_{1}\\
    &\times \int d\bm{u}_{n} \cdots \int d\bm{u}_{0}
    P(\bm{r}, \{\bm{u}_i \}, \{t_i\}; n, t, v)
    P(\bm{v}_0)\\
    =&e^{- vt/\lambda}
    \int d\bm{v}_0
    \int_0^{t}dt_{n}
    \int_0^{t_n}dt_{n-1}
    \cdots
    \int_0^{t_2}dt_{1}\\
    & \times \int d\bm{u}_{n-1} \cdots \int d\bm{u}_{0}
    \delta[\bm{r}-\bm{r}(n, t)]
    \prod_{i=0}^{n-1}f(\bm{u}_i)
    P(\bm{v}_0; v),
\end{split}
\end{equation}
where $P(\bm{v}_0; v) = \delta(|\bm{v}_{0}| - v) / 4 \pi v^{2}$ is the initial velocity distribution of the mobile particle.


To proceed with the calculation, we introduce the characteristic function, $C(\bm{k}; n, t, v)=\int d\bm{r} e^{i\bm{k}\cdot\bm{r}} P(\bm{r}; n, t, v)$.
From Eq.~\eqref{eq:collision_probability_reduced}, we can calculate $C(\bm{k}; n, t, v)$ as
\begin{equation}
\begin{split}
    &C(\bm{k}; n, t, v)\\
    =&e^{-vt/\lambda}
    \int d\bm{v}_0
    \int_0^{t}dt_{n}
    \int_0^{t_n}dt_{n-1}
    \cdots
    \int_0^{t_2}dt_{1}\\
    &\times \int d\bm{u}_{n-1} \cdots \int d\bm{u}_{0}
    \prod_{i=0}^{n-1} f(\bm{u}_i)
    P(\bm{v}_0;v)\\ &\times \exp\left\{i\bm{k}\cdot\left[\bm{v}_n(t-t_n)+\sum_{i=0}^{n-1}\bm{v}_i(t_{i+1}-t_i)\right]\right\} .
\end{split}
\label{eq:chacteristic_function}
\end{equation}
Here we consider the Laplace transform of Eq.~\eqref{eq:chacteristic_function}: $\hat{C}(\bm{k}; n, s, v) = \mathcal{L} [{C}(\bm{k}; n, \cdot, v)](s)
= \int_{0}^{\infty} dt \, e^{-t s} C(\bm{k}; n, t, v)$.
Then we have
\begin{equation}
\begin{split}
 \hat{C}(\bm{k}; n, s, v)
    = &\int d\bm{v}_{0}
    \frac{P(\bm{v}_0;v)}{s-i\bm{k}\cdot\bm{v}_{0}+v/\lambda} \\
 & \times \prod_{i=0}^{n-1} \int d\bm{u}_{i}
    \frac{f(\bm{u}_i)}{s-i\bm{k}\cdot\bm{v}_{i+1}+v/\lambda}.
\label{eq:chacteristic_function_laplace}
\end{split}
\end{equation}
The integrals over $\bm{v}_{0}$ and $\lbrace \bm{u}_{i} \rbrace$ in Eq~\eqref{eq:chacteristic_function_laplace}
can be analytically calculated. The result is
\begin{equation}
    \hat{C}(\bm{k}; n, s, v) %\mathcal{L}[C(\bm{k}; n, t, v)](s)
    =\frac{\lambda}{v}\left[\frac{1}{k\lambda}
    \arctan\left(\frac{kv}{s+v/\lambda}\right)\right]^{n+1} ,
\label{eq:van_hove_result_n}
\end{equation}
where $k = |\bm{k}|$. To obtain the probability density for $\bm{r}$ under given $s$ and $v$, we need to consider all the contributions from different $n$.
This can be easily calculated by taking the summation over $n$: $\hat{C}(\bm{k};s,v) = \sum_{n = 0}^{\infty} \hat{C}(\bm{k};n,s,v)$.
From Eq.~\eqref{eq:van_hove_result_n}, we have
\begin{equation}
 \hat{C}(\bm{k};s,v)
    =\frac{\displaystyle 
    \arctan\left(\frac{kv}{s+v/\lambda}\right)}
 {\displaystyle (v / \lambda) \left[ k\lambda - 
    \arctan\left(\frac{kv}{s+v/\lambda}\right) \right]}.
    \label{eq:van_hove_ft_lt}
\end{equation}
Eq.~\eqref{eq:van_hove_ft_lt} corresponds to the Fourier-Laplace transform of the self part of the van Hove correlation function, and thus any quantities which characterize the diffusion behavior can be calculated from Eq.~\eqref{eq:van_hove_ft_lt}.
Note that Eq.~\eqref{eq:van_hove_ft_lt} satisfies the normalization condition of the probability density: $\hat{C}(\bm{k}; s,v) =s^{-1}$ at $k=0$.


The MSD is calculated as the second-order moment for the van Hove correlation function.
From Eq.~\eqref{eq:van_hove_ft_lt}, we have the Laplace transform of the MSD under a given $v$ as follows:
\begin{equation}
\begin{split}
    \mathcal{L}[\langle \bm{r}^2(\cdot)\rangle_{v}](s)
    =&-\left.\frac{\partial^2}{\partial\bm{k}^2} \hat{C}(\bm{k}; s, v)\right|_{\bm{k}=\bm{0}}\\
    =&\frac{2v^2}{s^2(s+v/\lambda)} ,
\end{split}
\label{eq:msd_lt}
\end{equation}
where $\langle\cdots\rangle_{v}$ denotes the statistical average under a given $v$.
Similarly, the Laplace transform of the fourth-order moment becomes
\begin{equation}
\begin{split}
    \mathcal{L}[\langle \bm{r}^4(\cdot)\rangle_{v}](s)
 =&\frac{\partial^2}{\partial\bm{k}^2}\frac{\partial^2}{\partial\bm{k}^2} \left. \hat{C}(\bm{k}; s, v) \right|_{\bm{k}=\bm{0}}\\
    =&\frac{8v^4(9s+5v/\lambda)}{3s^3(s+v/\lambda)^3} .
\end{split}
\label{eq:m4d_lt}
\end{equation}
The inverse Laplace transforms of Eqs.~\eqref{eq:msd_lt} and \eqref{eq:m4d_lt} give
\begin{align}
    \frac{\langle \bm{r}^2(t)\rangle_{v}}{\lambda^2}
    =&2 \left(-1+vt/\lambda+e^{-vt/\lambda}\right) ,
    \label{eq:msd_lorentz}\\
\begin{split}
    \frac{\langle \bm{r}^4(t)\rangle_{v}}{\lambda^4}
    =&
    \frac{4v^2t^2}{3\lambda^2}
    \left(5+4e^{-vt/\lambda}\right)-\\
    &\frac{8vt}{\lambda}
    \left(2-e^{-vt/\lambda}\right)
    +8\left(1-e^{-vt/\lambda}\right) .
\end{split}
    \label{eq:m4d_lorentz}
\end{align}
It is straightforward to show that at the short-time limit ($t \to 0$) Eqs.~\eqref{eq:msd_lorentz} and \eqref{eq:m4d_lorentz} approach $\langle \bm{r}^2(t)\rangle_{v} \to v^2t^2$ and $\langle \Revised{\bm{r}^4(t)}\rangle_{v} \to v^4t^4$.
These reflect the ballistic motion.
At the long-time limit $t\to \infty$, the MSD approaches $\langle \bm{r}^2(t)\rangle_{v} \to 2v\lambda t$, which corresponds to the normal diffusion.
The diffusion coefficient, $D$, defined as $\langle \bm{r}^2(t; v)\rangle=6Dt$, becomes $D=v\lambda/3$, which is consistent with the well-established result \cite{dorfman2021} in the gas kinetic theory.
From Eqs.~\eqref{eq:msd_lorentz} and \eqref{eq:m4d_lorentz}, the analytic expression of the non-Gaussian parameter (NGP) under a given $v$, $\alpha(t;v)$, is calculated to be
\begin{equation}
\begin{split}
    &\alpha(t; v)=\frac{3\langle \bm{r}^4(t)\rangle_{v}}
    {5\langle \bm{r}^2(t)\rangle_{v}^2}-1\\
    &=\frac{4e^{-vt/\lambda}(v^2t^2/\lambda^2-vt/\lambda+1)+1-2vt/\lambda-5e^{-2vt/\lambda}}
    {5(-1+vt/\lambda+e^{-vt/\lambda})^2}.
\end{split}
\label{eq:ngp_lorentz}
\end{equation}

Fig.~\ref{fig:Lorentz-gas-msd-ngp} displays the MSD (Eq.~\eqref{eq:msd_lorentz}) and the absolute value of the NGP (Eq.~\eqref{eq:ngp_lorentz}) (the NGP by Eq.~\eqref{eq:ngp_lorentz} is always negative) of the Lorentz gas.
MSD shows simple ballistic and diffusive behaviors at the short and long timescales, respectively.
The crossover time is approximately equal to the mean free time, $vt/\lambda \approx 1$.
NGP becomes $-2/5$ at the short timescale and approaches $0$ at the long time scale.
The decay of $-\alpha(t;v)$ starts around $vt / \lambda \approx 1$, where the MSD switches from ballistic motion to normal diffusion.
The result that the NGP approaches zero at the long-time scale means that the dynamics of Lorentz gas can be reasonably described by the Gaussian process for $t \gtrsim \lambda / v$.
The non-Gaussianity at $t=0$ originates from the energy conservation of the Lorentz gas.
At the short timescale, the mobile particle ballistically moves, and thus NGP reflects the non-Gausianity of the velocity distribution.
We can easily evaluate the NGP at the short-time limit:
\begin{equation}
    \alpha(t;v) =\frac{3\langle \bm{v}^4t^4\rangle_{v}}{5\langle \bm{v}^2t^2\rangle_{v}^2}-1
    =-\frac{2}{5},
\end{equation}
which is consistent with the theoretical prediction at the short timescale in Fig.~\ref{fig:Lorentz-gas-msd-ngp}.

\Revised{
Before proceeding to the next section, we should mention the long-time tails, although they are beyond the scope of this work.
In general, molecules in matrices exhibit long-time correlated dynamics \cite{alder1978long,brey1983long,ernst1971long,machta1984long} and exhibit long-time tails at a long-time scale.
This means that the history of collisions (or called ring collisions) can not be strictly ignored even in any long timescale.
Such a correlation leads to some characteristic power laws in time correlation functions\cite{ernst1971long, machta1984long}.
The current system does not show such long-time tails because of the Markovian process assumption.
}

\begin{figure}
    \centering
    \includegraphics[width=0.3\textwidth]{fig_msd_ngp_lorentz.eps}
    \caption{
    Theoretical predictions of the scaled mean square displacement (Eq.~\eqref{eq:msd_lorentz}) and non-Gaussian parameter (Eq.~\eqref{eq:ngp_lorentz}) against reduced time $v t / \lambda$ for the dilute random Lorentz gas.
}
    \label{fig:Lorentz-gas-msd-ngp}
\end{figure}



\subsection{Binary Gas Mixture}
\label{subsec:binary_gas_mixture}

The Lorentz gas has often been regarded as the model that describes a light particle in heavy particles.
Thus, one may expect that the Brownian yet non-Gaussian diffusion, which is observed for a minor light particle in heavy gas \cite{nakai2023fluctuating}, can be predicted from the Lorentz gas model.
However, the theoretical result of the random dilute Lorentz gas (Fig.~\ref{fig:Lorentz-gas-msd-ngp}) does not exhibit the non-Gaussian diffusion in the normal diffusion regime (MSD$\propto t$).

\Revised{
This apparent inconsistency between the Lorentz gas model and a binary gas mixture comes from the fact that the speed (or kinetic energy) of a light gas particle in a binary gas mixture fluctuates at a very long timescale.
(We may interpret the results based on the Lorentz gas model are for the microcanonical ensemble, and we should use the canonical ensemble for the binary gas mixture.)
Within a certain timescale shorter than the relaxation timescale of the speed of the light particle, we can interpret the self-part of the van-Hove correlation function of the light particle in the gas mixture $G(\bm{r}, t)$ as the canonical ensemble average of that for the Lorentz gas model.
Using Eq.~\eqref{eq:collision_probability_reduced}, $G(\bm{r}, t)$ is described as
\begin{equation}
    G(\bm{r}, t)= \sum_{n=0}^\infty \int dv P(\bm{r}; n, t, v) P_{\text{MB}}(v),
    \label{eq:van-Hove_full}
\end{equation}
where $P_{\text{MB}}(v)$ is the Maxwell-Boltzmann distribution for the speed:
\begin{equation}
    P(v)=4\pi v^2\left(\frac{m}{2\pi k_BT}\right)^{3/2}
    \exp\left(-\frac{mv^2}{2k_BT}\right).
    \label{eq:Maxwell-Boltzmann}
\end{equation}
Here, $k_B$ is the Boltzmann constant.
While the calculation of Eq.~\eqref{eq:van-Hove_full} itself is difficult,
some moments for the displacement can be analytically obtained.
From Eqs.~\eqref{eq:msd_lorentz} and \eqref{eq:Maxwell-Boltzmann}, the second moment of $G(\bm{r}, t)$ is
}
\begin{equation}
\begin{split}
    \frac{\langle \bm{r}^2(t)\rangle}{\lambda^2}
    =& \int \frac{\langle \bm{r}^2(t) \rangle_{v}}{\lambda^2} P(v) dv\\
    =&2
    \left[-1+\frac{2\gamma t}{\sqrt{\pi}}
    +(1+2\gamma^2t^2)e^{\gamma^2t^2}\erfc(\gamma t)\right],
\end{split}
\label{eq:msd_ensemble}
\end{equation}
where $\gamma$ is a characteristic frequency defined as $\gamma=\sqrt{k_BT/2m}/\lambda$.
Fig.~\ref{fig:msd_ensemble} displays the prediction for the MSD of a light particle in a binary gas mixture by Eq.\eqref{eq:msd_ensemble}.
For comparison, the mean square displacements of the light particle in heavier particles, calculated from the kinetic Monte Carlo (KMC) simulations \cite{nakai2023fluctuating}, are also presented with various mass ratios \Revised{$\mu=m/M$ where $M$ is the mass of the major gas particle}.
Our theoretical prediction quantitatively agrees with the MSD from the KMC simulations when $\mu$ is sufficiently small ($\mu\ll 1$).

\begin{figure}
    \centering
    \includegraphics[width=0.3\textwidth]{fig_msd_ensemble.eps}
    \caption{
Theoretical prediction for MSD of binary gas mixtures with different mass ratios.
The red curve is the theoretical prediction by Eq.~\eqref{eq:msd_ensemble}, and symbols are the results of KMC simulations\cite{nakai2023fluctuating}.
}
    \label{fig:msd_ensemble}
\end{figure}


In a similar manner, the NGP for a binary gas mixture can be analytically calculated.
The ensemble average of the fourth-order moment is obtained using Eqs.~\eqref{eq:m4d_lorentz} and \eqref{eq:Maxwell-Boltzmann} as
\begin{equation}
\begin{split}
    &\frac{\langle \bm{r}^4(t)\rangle}{\lambda^4}
    =\int \frac{\langle \bm{r}^4(t)\rangle_{v}}{\lambda^4} P(v) dv\\
    =&\frac{4}{3}
    \left[
    6-\frac{12\gamma t}{\sqrt{\pi}}+30\gamma^2t^2
    -\frac{56\gamma^3t^3}{\sqrt{\pi}}
    -\frac{32\gamma^5t^5}{\sqrt{\pi}} \right.\\
    &\left. - (6+24\gamma^2t^2-72\gamma^4t^4-32\gamma^6t^6)
    e^{\gamma^2t^2}
    \erfc(\gamma t)
    \right] .
\end{split}
\label{eq:m4d_ensemble}
\end{equation}
Fig.~\ref{fig:ngp_ensemble} displays the NGP $\alpha(t)$ calculated from Eqs.~\eqref{eq:msd_ensemble} and \eqref{eq:m4d_ensemble}.
For comparison, the NGPs from the KMC simulations \cite{nakai2023fluctuating} are also shown with various $\mu$.
Our theoretical prediction successfully describes the non-Gaussian parameter from the KMC simulation for sufficiently small $\mu$ except for the decay of the NGP at the very long time scale.
We will discuss this discrepancy later.
The NGP by Eqs.~\eqref{eq:msd_ensemble} and \eqref{eq:m4d_ensemble} approaches $0$ at $t \to 0$, unlike the case of the Lorentz gas.
This reflects the fact that the probability density of the speed of the mobile particle obeys the Maxwell-Boltzmann distribution, which is a Gaussian distribution.
At the long-time limit ($t\to\infty$), the NGP approaches $3\pi/8-1$, and this quantity corresponds to the plateau of the NGP from the KMC simulations.

\begin{figure}
    \centering
    \includegraphics[width=0.4\textwidth]{fig_ngp_ensemble.eps}
    \caption{
Theoretical prediction for NGP of binary gas mixtures with different mass ratios.
The red curve is the theoretical prediction by Eqs.~\eqref{eq:msd_ensemble} and \eqref{eq:m4d_ensemble}, and symbols are the results from the KMC simulations\cite{nakai2023fluctuating}.
}
    \label{fig:ngp_ensemble}
\end{figure}


At last, we discuss the self part of the van Hove correlation function $G_{s}(\bm{r}, t)$ at the limit of the long time scale which is sufficiently larger than the mean free time, based on the Lorentz gas model.
At the long-time limit in the Lorentz gas ($s\ll v/\lambda$), Eq.~\eqref{eq:van_hove_ft_lt} reduces to
\begin{equation}
     \hat{C}(\bm{k};s,v)
    \approx \frac{\displaystyle 
    \arctan(k\lambda)-\frac{ks}{k^2v+v/\lambda^2}}
 {\displaystyle (v / \lambda) \left[ k\lambda - 
    \arctan(k\lambda)+\frac{ks}{k^2v+v/\lambda^2}\right]}.
\end{equation}
Its inverse-Laplace-transform is
\begin{equation}
\begin{split}
    &C(\bm{k};t,v)\\
    \approx    &(1+k^2\lambda^2)\exp\left[-\frac{(1+k^2\lambda^2)[k\lambda-\arctan(k\lambda)]}{k\lambda}\frac{tv}{\lambda}\right].
    \label{eq:van-hove_ft_limit}
\end{split}
\end{equation}
Here, we should note that Eq.~\eqref{eq:van-hove_ft_limit} is valid for the long-time scale, $t\gg \lambda/v$. In this timescale, we expect that only the small wavelength component ($k\lambda\ll 1$) becomes dominant.
Then Eq.\eqref{eq:van-hove_ft_limit} can be reduced to 
\begin{equation}
    C(\bm{k};t,v)     \approx  \exp\left(-\frac{v t\lambda k^2}{3}\right).
     \label{eq:van-hove_gaussian_limit}
\end{equation}
Eq.~\eqref{eq:van-hove_gaussian_limit} is nothing but the characteristic function of the Gaussian distribution.
Thus, the van-Hove correlation function of the Lorentz gas for a given speed $v$ at the long-time limit is
\begin{equation}
    G(x; t, v)     \approx  \sqrt{\frac{3}{4\pi v\lambda t}}\exp\left(-\frac{3 x^2}{4v \lambda t}\right) .
    \label{eq:van_hove_v}
\end{equation}
From Eqs.~\eqref{eq:Maxwell-Boltzmann} and \eqref{eq:van_hove_v}, we obtain the van-Hove correlation function for a light gas particle in a binary gas mixture as
\begin{equation}
\begin{split}
    G(x;t) = &\int_0^{\infty} G(x; t, v)P(v)dv,
\end{split}
    \label{eq:van_hove_approximation_ensemble}
\end{equation}
\Revised{
Although Eq.~\eqref{eq:van_hove_approximation_ensemble} can be analytically calculated, the result is rather complex.
Because the Brownian yet non-Gaussian diffusion often has striking features in tails in the van-Hove correlation function \cite{sposini2018random, jeon2016protein, Chechkin2017}, we attempt to obtain the approximate form for the tail for the large displacement region.
The saddle point approximation for Eq.~\eqref{eq:van_hove_approximation_ensemble} gives
\begin{equation}
\begin{split}
    G(x; t, v)     \approx
    &\sqrt{\frac{3m}{4\pi k_BT}}\frac{|x|}{\lambda t}
    \exp\left[-3\left(\sqrt{\frac{9m}{128k_BT}}\frac{x^2}{\lambda t} \right)^{\frac{2}{3}}\right]\\
    &\mathrm{for} \ x^2\gg \lambda t\sqrt{k_BT/m}) .
\end{split}
\label{eq:van-hove_tail}
\end{equation}
Eq.~\eqref{eq:van-hove_tail} is the same as the form obtained phenomenologically in our previous work \cite{nakai2023fluctuating}.
It would be worth stressing here that Eq.~\eqref{eq:van-hove_tail} is not exponential nor stretched Gaussian distributions.
}


As we mentioned, our theoretical prediction for the NGP $\alpha(t)$
converges to a constant value at the long-time limit.
Therefore, at least in the theoretical framework shown above, our theory does not predict the Gaussian behavior at the long-time region observed in the KMC simulations.
The discrepancy between the theoretical prediction and the KMC data at the long-time limit can be attributed to the lack of speed relaxation in our analysis.
The KMC simulations revealed that there are two characteristic relaxation time scales in a binary gas mixture: the direction and speed relaxations \cite{nakai2023fluctuating}.
In our analysis, we considered the direction relaxation via hard-core collisions, while the speed relaxation is not explicitly considered.
We only assumed that the speed relaxation makes the initial ensemble with the Maxwell-Boltzmann distribution.
The speed relaxation affects the long-time dynamics, but it is totally ignored.
Therefore, the diffusion coefficient for the particle with a given initial speed $v$ remains constant even at the long-time limit: $D = v \lambda / 3$.
This means that the long-time diffusion behavior reflects the initial speed distribution. This is why the NGP does not approach zero even at the long-time limit.


The diffusion coefficient of a light particle in a binary gas mixture fluctuates in time due to speed relaxation.
Therefore, if we describe the diffusion of the light particle at the long timescale, we need to incorporate another stochastic process into the model.
The diffusing diffusivity model \cite{Chechkin2017} in which the diffusion coefficient obeys the Langevin equation would be employed to incorporate the fluctuation of the speed.
Some analytical results of the gas kinetic theory \cite{pagitsas1979kinetic} can be utilized to design the stochastic process for the speed of the particle.


\section{conclusion}
In this work, we theoretically analyzed the dynamics of a light particle in heavy gas particles with large mass contrast, based on the dilute Lorentz gas model.
\Revised{
We derived the analytical expressions for the MSD and NGP of the Lorentz gas for a given speed $v$ via the point process approach.
The result with constant $v$ does not exhibit the Brownian yet non-Gaussian diffusion observed in the binary gas mixtures with mass and fraction contrast.
We found that the Brownian yet non-Gaussian diffusion can be reproduced through the canonical ensemble average of statistical quantities for the Lorentz gas over the initial speed}, except for the very long-time region.
This work revealed a relation between the conventional Lorentz gas and the binary gas mixture, and it will provide fresh insight into the theoretical modeling for gas diffusion.


\section*{Acknowledgments}
FN was supported by Grant-in-Aid for JSPS (Japan Society for the Promotion of Science) Fellows (Grant No. JP21J21725).

\appendix
\setcounter{figure}{5}

%\Revised{
%\section{Velocity auto-correlation function and Burnett coefficient}
%\label{appendix_vac_burnett}
%Our Lorentz gas system is under the Markovian process and thus does not show the long-time tails \cite{machta1984long, ernst1971long} which originate from the correlated collisions between the moving particle and the fixed scatters.
%To show the absence of the long-time tails, we calculate the velocity auto-correlation function $\langle\bm{v}(t)\cdot\bm{v}(0)\rangle_v$ and the Burnett coefficient $(d/dt)[\langle x^4(t)\rangle_v-3\langle x^2(t)\rangle_v^2]/24$ in the Lorentz gas with the Markovian process.
%
%In stochastically steady states, the velocity auto-correlation function is the same as the half of the second time-derivative of the mean square displacement:
%\begin{equation}
%    \langle\bm{v}(t)\cdot\bm{v}(0)\rangle_v
%    =\frac{\partial^2 \langle \bm{r}^2(t)\rangle_v }{2\partial t^2}
%    =v^2\exp\left(-\frac{vt}{\lambda}\right)
%\end{equation}
%where it becomes $v^2$ at $t=0$ and decays with exponential function.
%
%To calculate the Burnett coefficient, we require the second and fourth moments of the probability density of the moving particle displacement with a single degree of freedom, e.g. x-direction.
%The dynamics of the moving particle is statistically isotropic, and then we find the simple relations $\langle \bm{r}^2(t)\rangle_{v}=3\langle x^2(t)\rangle_{v}$ and $\langle \bm{r}^4(t)\rangle_{v}=5\langle x^4(t)\rangle_{v}$ for the second and fourth moments, respectively.
%Using these relations and Eqs.~\eqref{eq:msd_lorentz} and \eqref{eq:m4d_lorentz}, the Burnett coefficient is obtained as
%\begin{equation}
%\begin{split}
%    &\frac{\partial}{\partial t}
%    \frac{\langle x^4(t)\rangle_v-3\langle x^2(t)\rangle_v^2}{24}\\
%    =&\frac{v\lambda^3}{45}\left[
%    \left(-\frac{2t^2v^2}{\lambda^2}+\frac{6tv}{\lambda}-4+5e^{-tv/\lambda}\right)e^{-tv/\lambda}
%    -1
%    \right].
%\end{split}
%\end{equation}
%From the result, the Burnett coefficient does not exhibit a power-law decay in our system, as we expected.
%}

\Revised{
\section{Kinetic Monte Carlo simulation}\label{appendix_kmc}
We briefly explain the kinetic Monte Carlo simulation for the dynamics of a minor particle in the binary gas mixture with fraction contrast (the details are shown in Ref.~\cite{nakai2023fluctuating}).
In this method, we employ the following assumptions:
\begin{enumerate}
    \item \label{hardcore_assumption}
	  The minor particles interact with the major particle via the hard-core potential.
    \item \label{dilute_assumption}
	  The minor particles are infinitely diluted; the minor particles interact only with the major ones.
    \item \label{markovian_assumption}
	  The dynamics of the minor particle obey the Markovian process.
    \item \label{equilibrium_assumption}
	  The system is in an equilibrium state; the positions of the particles are homogeneously distributed, and the statistics of the velocity obey the Maxwell-Boltzmann distributions.
\end{enumerate}
These assumptions have often been employed in gas kinetics \cite{dorfman2021,mclennan1989introduction}.
From Assumption~\ref{hardcore_assumption}, the minor particle velocity $\bm{v}$ changes to $\bm{v}'$ by a collision with a major one as
\begin{equation}
    \bm{v}'=\bm{v}-\frac{2M}{m+M}(\bm{v}-\bm{V})\cdot\hat{\bm{r}}\hat{\bm{r}},
    \label{eq:appendix_velocity_change}
\end{equation}
where $\hat{\bm{r}}$ is the direction unit vector connecting the center of the minor particle to that of the colliding major particle and $\bm{V}$ is the velocity of the colliding major particle.
On the basis of Assumptions \ref{dilute_assumption}-\ref{equilibrium_assumption}, we can calculate
the collision statistics. The probability density where a minor particle collides with a major particle having $\bm{V}$ at $\hat{\bm{r}}$ and at the time interval $\tau$, for a given minor particle's velocity $\bm{v}$ is
\begin{equation}
\begin{split}
    &P(\bm{V}, \hat{\bm{r}}, s | \bm{v})\\
    & = \rho\sigma^2(\bm{v}-\bm{V})\cdot \hat{\bm{r}}
    \left(\frac{M}{2\pi k_BT}\right)^{3/2}
    \exp\left( -\frac{MV^2}{2k_BT} \right)\\
    &\times\exp\left[-F(v) s\right]
    \Theta[(\bm{v}-\bm{V})\cdot\hat{\bm{r}}],
\end{split}
\label{eq:appendix_collision_statistics}
\end{equation}
where $\Theta(x)$ is the Heaviside function and $F(v)$ is the collision frequency of the minor particle for a given $v$ \cite{nakai2023fluctuating}:
\begin{equation}
    F(v)=\pi\rho\sigma^2\left[
    \left(v+\frac{\xi^2}{2v}\right)\mathrm{erf}\left(\frac{v}{\xi}\right)
    +\frac{\xi}{\sqrt{\pi}}\exp\left(-\frac{v^2}{\xi^2}\right)
    \right],
\label{eq_appendix_collision_frequency}
\end{equation}
where we defined the characteristic major particle speed $\xi\equiv \sqrt{2k_BT/M}$.
The stochastic process of the minor particle is 
fully characterized by Eqs.~\eqref{eq:appendix_velocity_change}
and \eqref{eq:appendix_collision_statistics}.
At the large the mass ratio case ($\mu=m/M\ll 1$), which is the interest of this work, Eqs.~\eqref{eq:appendix_velocity_change} and \eqref{eq:appendix_collision_statistics} reduce to Eqs.~\eqref{eq:velocity_change} and \eqref{eq:frequency_n}.
}

\bibliographystyle{apsrev4-2}
\bibliography{ref}

% \clearpage 
% \newcommand{\AUTHOR}[1]{Authors response: {\color{MidnightBlue}#1}\\[0mm]{}}
% \newcommand{\REFEREE}[1]{Reviewer: \emph{#1}\\[0mm]{}} 
% \parindent 0mm

% \subsection*{[EC12561] Response to Editor} 

% Dear Dr. Juan-Jose Lietor-Santos, \\

% Thank you for having provided very useful reports. 
% All changes have been highlighted in \Revised{blue} color
% in the revised manuscript. Please find enclosed a point-by-point response to both referees.

% \subsection*{Response to report of the first referee} 

% \REFEREE{The manuscript provides an analytical description based on the Lorentz gas model
% for the recently observed Brownian yet non-Gaussian (BNG) diffusion in a gas
% mixture with large mass difference discussed by the same authors in a previous
% publication in PRE. I find the discussed topic timely and the presentation very
% clear, thus I believe this work is worthy of publication in PRE. However, before
% publication I would like to ask the authors to address one point:}

% \AUTHOR{We thank this referee for the careful reading and positive assessment.
% We revised the manuscript based on the comments.}

% \REFEREE{(1)
% The main finding of the paper is that the Lorentz gas model alone cannot
% describe the BnG in gas mixtures with large mass difference as the (possible)
% long relaxation time of the speed of the mobile particles is not take into
% account. The authors overcome this issues by introducing a statistical average
% over the speed probability distribution, that is of course described as the
% Maxwell-Boltzmann distribution. However this average is computed directly at the
% level of the moments of the self Van-Hove distribution function (second- and
% forth-order moment). I wonder what would be the result if the Maxwell-Boltzmann
% distribution of the speed was introduced already in the calculations reported in
% Eq. (8)? I expect this would lead directly to Eq. (20) and (21)?}

% \AUTHOR{
% We appreciate your useful comment.
% We consider that Eq.~\eqref{eq:collision_probability_reduced} works as a
% sort of a propagator, and thus if we introduce the equilibrium Maxwell-Boltzmann distribution into Eq.~\eqref{eq:collision_probability_reduced},
% As you pointed out, the results simply become the same with Eqs.\eqref{eq:msd_ensemble} and \eqref{eq:m4d_ensemble}.
% We added Eq.~\eqref{eq:van-Hove_full} and revised the text just before and after Eq.~\eqref{eq:van-Hove_full}.
% }

% \REFEREE{(2)
% Small typo:
% -page 3, after Eqs. (15) and (16): "$\langle r^2(t) \rangle_v \to v^2 t^2$ and  $\langle r^2(t)\rangle_v
% \to v^4 t^4$"I believe in the second limit there should be the fourth-moment and
% not the second one.}

% \AUTHOR{
% We appreciate for pointing out the typo.
% We corrected it as the text following Eq.~\eqref{eq:m4d_lorentz} on page 3.
% }

% \subsection*{Response to report of the second referee}

% \REFEREE{Nakai and Uneyama discuss the transport properties of a light particle
% moving in a random and diluted arrangement of obstacles (Lorentz gas).  The
% authors argue that after a suitable ensemble average the model describes
% also the case of a light particle in a gas of heavy particles and they
% claim that then the transport exhibits the phenomenon of Brownian yet
% non-Gaussian diffusion (BNGD), which has received a lot of interest
% recently. Within the approximation of uncorrelated collisions, they
% calculate analytically the 2nd and 4th moments of the van Hove function for
% the two cases and corroborate their findings with data from previous work
% using kinetic Monte Carlo simulations.}

% \AUTHOR{We thank this referee for useful suggestions, which we took into account during the revision of the manuscript.}

% \REFEREE{The question of the manuscript is timely. Despite the half-century long
% history of kinetic theory and numerous rigorous and elaborate results on
% the transport coefficients, it seems that the tails of the van Hove
% function have received little attention. Given this knowledge gap the crude
% approximation of a Markovian sequence of collisions may be pardoned.
% However, the presentation repeatedly lacks sharp facts and important
% details in favor of convoluted hand waving arguments. It remains somewhat
% opaque what are the assumptions and the precise claims to be proven. The
% important question whether the transport in the Lorentz gas emerges as the
% large-mass limit of the binary mixture remains unanswered.}

% \AUTHOR{
% We appreciate your valuable comments.
% In short, the rigorous relation between the Lorentz gas and the
% binary gas mixture with a large mass contrast limit is currently not fully
% clear. Our binary gas mixture model seems to reduce to the Lorentz gas based on {\em some physical assumptions} which we believe reasonable.
% The employed assumptions are described
% in the 1st paragraph in Sec.~\ref{sec:model}.
% We also revised the manuscript to emphasize the main claim in the text in the conclusion.
% We believe that the physical arguments shown in the manuscript
% will be sufficient for our purpose.
% }

% \REFEREE{Also what is the
% added value of the simulations, given that they make use of the same
% approximations?}

% \AUTHOR{Thank you for the comment.
% The simulation is used to validate our analytic results.
% In the simulation, the minor particle collides with the major moving particle with {\em finite mass ratios.}
% The simulation data show that our analytic results are reasonable for binary gas systems with \em{finite but large mass contrasts}.
% We explained the KMC simulation method, adding Appendix \ref{appendix_kmc}.
% }

% \REFEREE{In addition, it seems that the authors have overlooked a considerable body of the literature on the topic. The manuscript requires a
% substantial revision and improvements before it can be reconsidered for
% publication in Physical Review E. The following remarks and questions
% should be addressed.}

% \AUTHOR{We highly appreciate your useful advice.
% Based on your comments, we revised the manuscript and incorporated the suggested literature into this paper.}

% \REFEREE{(1) For the case of heavy, but mobile obstacles, the collision law (1)
% needs to be modified such that it depends also on the momentum of the
% obstacle. The key question of the problem is whether the limit of infinite
% obstacle mass commutes with the canonical ensemble average and with the
% long-time limit.  This is essentially the claim of the authors but a
% rigorous proof is missing.}

% \AUTHOR{
% We appreciate your useful comment.
% We agree that we did not give rigorous proof for this point.
% Honestly speaking, we were not able to obtain mathematically rigorous
% proof. Nonetheless, we believe that the {\em assumption} that
% the canonical ensemble is realized at the very long time scale is reasonable.
% Firstly, the light particle changes its momentum by a collision with a heavy gas particle. If the mass ratio is very large, the momentum change by a single
% collision is very small and Eq.~\eqref{eq:velocity_change} works as {\em a very
% good approximation}. At a very long time scale, the light particle collides
% with large particles many times, and ultimately the momentum is equilibrated. Physically, the equilibration should be realized as long as
% the mass ratio is finite. Thus the {\em assumption} that the momentum
% distribution obeys the canonical ensemble should be also reasonable.
% Secondly, we performed the kinetic Monte Carlo simulations for the binary gas mixture with finite mass, and analytical expressions quantitatively reproduce the Brownian yet non-Gaussian diffusion as shown in Figs.~\ref{fig:msd_ensemble} and \ref{fig:ngp_ensemble} (except for the long timescale).
% Thus, we believe that the canonical ensemble average of the time-correlation functions for the Lorentz gas successfully explains the Brownian yet non-Gaussian diffusion in the binary gas mixture observed at large mass ratios, at least as a very good approximation.
% }

% \REFEREE{(1)
% Specifically, I expect that the calculations performed for the Lorentz gas,
% based on eq. (1), are repeated for mobile obstacles using the modified
% collision law. Then, the obstacle mass can be sent to infinity to verify
% whether the Lorentz gas results are reproduced in this limit.}

% \AUTHOR{
% Thank you for the useful comment. 
% We agree that the calculations for the mobile obstacles are ideal
% to justify our results.
% However, the collision rule by Eq.~\eqref{eq:velocity_change} greatly simplifies calculations.
% We think that we cannot perform analytic calculations with a
% different collision rule for mobile obstacles.
% Instead, to verify our theoretical calculation is correct, we calculated the dynamics of a minor component particle in the binary gas mixture using the KMC simulations with the appropriate collision statistics \eqref{eq:appendix_collision_statistics} and collision law \eqref{eq:appendix_velocity_change}.
% From Figs.~\ref{fig:msd_ensemble} and \ref{fig:ngp_ensemble}, our theoretical results agree well with the KMC simulation with large mass ratios, and the simulation data seem to agree with the theoretical results at the limit of $m/M\ll 1$.
% }

% \REFEREE{(2)
% The argument given before eq. (19) suggests that the average should not
% be taken over the initial velocity of the light particle but over the
% momentum of the heavy particle at the time of collision. Please clarify.}

% \AUTHOR{
% We appreciate this comment.
% For clarity, We modified the manuscript as in the text above Eq.~\eqref{eq:van-Hove_full}.
% }

% \REFEREE{(2)
% In addition, I ask the authors to include a description of how the kinetic
% Monte Carlo simulations for the Lorentz gas and for the binary mixture
% differ. Are different collision rules and/or statistics used?}

% \AUTHOR{We appreciate the valuable comment.
% As you pointed out, the collision rules are different. At the short-time
% scale, the Lorentz gas and the binary gas mixture behave almost in the
% same way, but their behaviors become different at the very long timescale.
% We added the explanation for the KMC simulation in Appendix.\ref{appendix_kmc}.
% The collision rule and statistics of the Lorentz gas are obtained as the limits of those for the binary gas mixture, as also explained in Appendix.\ref{appendix_kmc}.
% }

% \REFEREE{(3)
% The present work uses a Markovian approximation of the dynamics in the
% spirit of the stosszahlansatz. Such an approach may be admissible given the
% specific question at hand, albeit it is expected to describe the actual
% dynamics in binary mixtures and in the Lorentz gas only at extremely low
% densities. Nevertheless, readers should be alerted about this fact and a
% brief discussion of hallmark features of the dynamics due to correlated
% collisions is in place. In particular, the long-time tails of the velocity
% autocorrelation function and the (super-)Burnett coefficient (related to
% non-Gaussianity) should be mentioned.}

% \AUTHOR{We appreciate this comment.
% The current system is under the Markovian nature, and thus it
% does not show the long-time tail related to the non-Markovian
% multi-particle correlation.
% We added the paragraph which describes the long-time behavior in our system, referring to the suggested literature, in the last paragraph in Subsec.\ref{subsec:lorentz_gas}.
% }

% \REFEREE{(4) The non-vanishing long-time limit of the non-Gaussian parameter appears
% to be an artifact of the ensemble average. For the original Lorentz gas and
% also for the hard sphere fluid, theoretical work for the corresponding
% Burnett coefficient B(t) predicts $B(t) \sim t^{-1/2}$ in $d=3$ dimensions and
% thus $\alpha(t) \sim t^{-3/2}$. See
% Brey, J. Chem. Phys. 79, 4585 (1983)
% Machta et al., J. Stat. Phys. 35, 413 (1984)
% }

% \AUTHOR{
% We appreciate this useful comment.
% First of all, we should explicitly state that our model does not
% have the long-time tail, thus any behavior related to the long-time
% tail cannot be reproduced.
% We are interested in the non-Gaussian diffusion behavior {\em at the intermediate-time scale} where long-time tail is negligible. And this non-Gaussian behavior
% originates from the mass contrast in gas mixtures.
% {\em At the intermediate-time scale where the speed is not relaxed}, the non-Gaussianity obtained from the ensemble average is not an artifact.
% We added the texts in the last paragraph in Subsec.\ref{subsec:lorentz_gas} and the 2nd paragraph in Subsec.\ref{subsec:binary_gas_mixture} for clarity.
% }

% \REFEREE{(5) In the literature, the phenomenon of Brownian yet non-Gaussian
% diffusion (BNGD) is also associated with the appearance of exponential
% tails in the van Hove function. The mere observation of deviations from a
% Gaussian displacement distribution is not a sufficient criterion for BNGD.
% (The definition of BNGD should be corrected in the abstract and in the
% introduction.) The author's numerical work (ref. [5] of the manuscript)
% points at the possibility that exponential tails exist in the present
% model. Now it would be of interest to deduce the asymptotics of the van
% Hove function from the analytic result, eq. (12), at least, by doing the
% Fourier backtransform numerically.}

% \AUTHOR{
% Thank you for this comment.
% The ``Brownian yet non-Gaussian (BNGD)'' diffusion is defined as the
% diffusion behavior which shows the normal (Brownian) diffusion for
% the ensemble-averaged MSD and the non-Gaussian displacement distribution
% (self van Hove correlation function)\cite{sposini2018random, jeon2016protein}.
% Therefore, the deviation from the Gaussian displacement distribution
% {\em should be sufficient} to define BNGD.
% Although the exponential tails for the self-van Hove correlation functions
% are widely reported, but the exponential tails are not essential.
% Various types of tails such as stretched Gaussian or some functions are possible for BNGD.
% In our previous work \cite{nakai2023fluctuating}, the van-Hove correlation function shows a tail that is not an exponential function.
% We calculated the asymptotic behavior (tail) for the van-Hove correlation function as in Eq.~\eqref{eq:van-hove_tail}.
% The obtained function is not an exponential function.}

% \REFEREE{(6)
% The introduction and also the discussion miss a large body of the
% literature on the subject; in particular, readers may get the impression
% that there was virtually no development in the past twenty years. I suggest
% that the author consider putting their work in the appropriate context:}

% \AUTHOR{We appreciate your useful comments.
% We added discussions and incorporated the literature that the referee suggested.}

% \REFEREE{(6)
% BNGD was observed first in glass-forming liquids by
% Chaudhuri et al., Phys. Rev. Lett. 99, 060604 (2007).
% Miotto et al., Phys. Rev. X 11, 031002 (2021).
% Both references should appear next to [6].}

% \AUTHOR{
% We appreciate useful info.
% We added these papers to the first paragraph in Sec.\ref{sec:introduction}.
% }

% \REFEREE{(6)
% For precise simulation data for the diffusion constant of the Lorentz gas and
% for the velocity correlation function, please see the works by
% Hoefling, Franosch et al.:
% Phys. Rev. Lett. 96, 165901 (2006)
% Phys. Rev. Lett. 98, 140601 (2007)
% J. Chem. Phys. 128, 164517 (2008)
% Rep. Prog. Phys. 76, 046602 (2013)}

% \AUTHOR{Thank you for your valuable comments.
% We included these works in the 2nd paragraph in Sec.\ref{sec:introduction}.
% }

% \REFEREE{(6)
% Experimental realizations of the Lorentz gas were carried out by
% Dullens et al., Phys. Rev. Lett. 111, 128301 (2013); Phys. Rev. E 95, 032602
% (2017).
% See also Siboni et al., Phys. Rev. Lett. 120, 056601 (2018).}

% \AUTHOR{We appreciate this valuable comment.
% We added these works in the 2nd paragraph in Sec.\ref{sec:introduction}.
% }

% \REFEREE{(6)
% For active matter variants of the Lorentz gas, see
% Irani et al., Phys. Rev. Lett. 128, 144501 (2022) and references therein.}

% \AUTHOR{
% Thank you for the information on an interesting paper. We have
% read the paper by Irani et al and found that the target system is interesting
% but much different from our target (BNGD of gas). In order to avoid possible
% confusion for readers, we do not cite the work by Irani et al.
% }

% \REFEREE{(6)
% A comparison of the results to those for the wind tree model would be of
% theoretical interest:
% Sanvee et al., Phys. Rev. E 106, 024104 (2022)}

% \AUTHOR{
% Thank you for the information on another interesting paper.
% The Wind-tree model by Sanvee et al seems to be interesting if
% we want to compare some results for the Lorentz gas model with it.
% However, our target in this work is a binary gas mixture, and
% the direct comparison of our results with the Wind-tree model
% seems not to be informative.
% }

% \REFEREE{(6)
% And closely related to spherical obstacles is a planar wall:
% Alexandre et al., Phys. Rev. Lett. 130, 077101 (2023).}

% \AUTHOR{We appreciate this useful info.
% We added this paper to the 2nd paragraph in Sec.\ref{sec:introduction}.
% }

% \REFEREE{(6)
% Finally, there has been progress on the theory side, most notably by
% Krakoviack, Phys. Rev. Lett. 94 065703 (2005);
% Phys. Rev. E 79 061501 (2009); Phys. Rev. E 82, 061501 (2010)
% Voigtmann, EPL 96, 36006 (2011); Phys. Rev. Lett. 103, 205901 (2009)}

% \AUTHOR{
% Thank you for the information on some papers. The analyses of the non-Gaussian diffusion behavior based on the mode-coupling theory (MCT) are interesting.
% We added some of the listed papers to the 2nd paragraph in Sec.\ref{sec:introduction}.
% }

% \REFEREE{(6)
% On kinetic theory treatments of mixtures of light and heavy particles under
% inelastic collisions, I suggest to consider
% Dufty et al., Phys. Rev. E 60, 7174 (1999); New J. Phys. 7, 20 (2005).}

% \AUTHOR{
% Thank you for the valuable comment.
% We included these papers in the 3rd paragraph in Sec.\ref{sec:introduction}.
% }

% \REFEREE{(7)
% Further remarks: 
% In the last sentence of the very first paragraph ("In the previous work ... minor light particle."), it is unclear which timescales are compared. What is the timescale of the "velocity direction"? Please rephrase.
% }

% \AUTHOR{
% Thank you for this comment.
% In our binary gas mixture, there are two characteristic relaxation times for the light gas particle. One is the relaxation time of the speed
% (the magnitude of the velocity), and
% another is the relaxation time of the direction of the velocity.
% The timescale of the velocity direction is the time scale where the latter relaxation takes place.
% For clarity, we added the following explanation to the first paragraph in Sec.\ref{sec:introduction}.

% "In the previous work \cite{nakai2023fluctuating}, the origin of the non-Gaussian behavior was attributed to the fluctuating diffusivity which arises from a separation of two relaxation timescales of the minor light particle velocity.
% The relaxation timescale for speed (the magnitude of the velocity)
% can be much longer than that for the direction of the velocity.
% Namely, when the velocity is described in the spherical coordinates, the polar and azimuthal angle components rapidly relax at the time scale where the radial component almost remains unchanged."
% }

% \REFEREE{(7)
% For the Lorentz gas, the post-average over the initial velocity is equivalent
% to switching from the microcanonical to the canonical ensemble of light particles. I suggest to use these standard terms in the text and to make the
% connection clear.}

% \AUTHOR{
% We appreciate this comment.
% In some aspects, the averaging over the initial velocity may be
% interpreted as the switching from the microcanonical to the canonical ensemble,
% as you pointed out. But in this work, we rather consider that what we did was to separate the canonical average into two averages; the average with respect to the constant initial speed, and the average with respect to the initial speed.
% We revised the expressions as in the texts in the abstract, the first paragraph in Sec.~\ref{subsec:binary_gas_mixture}, and the conclusion.
% }

% \REFEREE{(7)
% In eq. (2) I suggest to write the factors $v_n$ and $v_i$ after the
% parentheses to avoid a misinterpretation of the time differences as
% functional arguments.}

% \AUTHOR{We appreciate your advice. We corrected it as Eq.\eqref{eq:position}.}

\end{document}