\begin{abstract}
We investigate different natural language processing (NLP) approaches based on contextualised word representations for the problem of early prediction of lung cancer using free-text patient medical notes of Dutch primary care physicians.
Because lung cancer has a low prevalence in primary care, we also address the problem of classification under highly imbalanced classes.
Specifically, we use large Transformer-based pretrained language models (PLMs) and investigate: 
1) how \textit{soft prompt-tuning}---an NLP technique used to adapt PLMs using small amounts of training data---compares to standard model fine-tuning;
2) whether simpler static word embedding models (WEMs) can be more robust compared to PLMs in highly imbalanced settings; and
3) how models fare when trained on notes from a small number of patients.
We find that
1) soft-prompt tuning is an efficient alternative to standard model fine-tuning;
2) PLMs show better discrimination but worse calibration compared to simpler static word embedding models as the classification problem becomes more imbalanced; and
3) results when training models on small number of patients are mixed and show no clear differences between PLMs and WEMs.
All our code is available open source in \url{https://bitbucket.org/aumc-kik/prompt_tuning_cancer_prediction/}.\footnote{A short version of this paper has been published at the 21st International Conference on Artificial Intelligence in Medicine (AIME 2023).}

\keywords{Prediction models  \and Natural Language Processing \and Cancer.}
\end{abstract}