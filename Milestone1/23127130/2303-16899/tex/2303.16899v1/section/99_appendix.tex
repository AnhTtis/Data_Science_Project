\renewcommand{\thefigure}{A.\arabic{figure}} % \thesection instead of A would make it A.1, B.1...
\setcounter{figure}{0} 
\renewcommand{\thetable}{A.\arabic{table}}
\setcounter{table}{0} 

\appendix

% Overview
We first show the details of the AD collection pipeline (Sec.~\ref{app:sec:details})
with qualitative text examples (Sec.~\ref{app:sec:qualitative-text}).
Then we describe additional implementation details (Sec.~\ref{app:sec:implementation})
with extra qualitative movie AD examples (Sec.~\ref{app:sec:qualitative-ad}).
Finally, we list the movie IDs used in our MAD-v2 split (Sec.~\ref{app:sec:ids}).

\section{AD Collection Pipeline Additional Details }
\label{app:sec:details}

\begin{figure}[h!]
    \centering
    \includegraphics[width=\textwidth]{img/pipeline.pdf}
    \caption{\textbf{A schematic of our AD collection pipeline.}
    The pipeline takes the audio file (with mixed AD and movie audio) as input,
    and automatically outputs the AD in text form with corresponding timestamps. 
    % ** from az: great figure! ** :)
    }
    \label{fig:pipeline}
\end{figure}

\subsection{AD Collection Pipeline for MAD-v2}
Collecting movie AD has two main challenges.
First, in the audio files (\eg~from AudioVault)
the movie AD is \emph{fused} with the original movie audio,~\ie~on the same audio track. The pipeline needs to identify the AD speaker among the movie characters accurately.
Second, for the same movie,
the audio files from AudioVault is usually not synchronised with the movie from which the MAD visual features were extracted,
mainly due to the varied durations of intro and outro of different movie source.
Since we rely on the MAD visual features, the synchronisation is an essential step.

The automated data collection pipeline is briefly introduced in Sect.~\ref{sec:mad} of the main paper.
A schematic is shown in Fig.~\ref{fig:pipeline}, in detail:


\begin{enumerate}
\item We transcribe the mixed audio file using \textit{WhisperX}~\cite{bain2022whisperx} which provides accurate punctuated transcriptions with word-level timestamps.
\item The transcript is tokenized into sentences using the nltk python toolbox~\cite{bird2006nltk},  resulting in transcription sentences and their corresponding temporal segments (inferred the start and end time of the first and last word in the sentence respectively).
\item Each sentence segment is assigned a single speaker identity (\eg~\texttt{SPEAKER\_00}, \texttt{SPEAKER\_01}, \etc) by performing speaker diarization on the mixed audio, whereby each sentence timestamp is provided as oracle voice activity detection. Specifically,  we use SpeechBrain ECAPA-TDNN voice embeddings~\cite{desplanques2020ecapa} trained on VoxCeleb~\cite{nagrani2017voxceleb} and Agglomerative Clustering with a threshold of 0.95. \item To automatically identify the cluster associated with the AD speaker, we exploit the third-person nature of AD narrations and select the cluster with the lowest proportional occurrence of first- \& second-person pronouns,~\eg~``I'' and ``you'' with 95 or more speaker segments.
\item To synchronise the segment timestamps with the original audio track from which the MAD visual features were extracted, we follow~\cite{soldan2022mad} and calculate the time delay $\tau$ between the original movie audio files and the mixed audio files via FFT cross-correlation. The timestamps of the identified AD segments are shifted according $\tau$ in order to synchronise them to the visual features and subtitles collected in MAD.
\end{enumerate}

\subsection{AD Collection Pipeline for AudioVault}

The collection pipeline for AudioVault is introduced in  Sect.~\ref{sec:audiovault_textonly} of the main paper, 
we provide more details here.
To collect text-only AD annotations from AudioVault, the final synchronisation step is unnecessary.
Therefore, we follow steps 1-4 of the MAD denoising pipeline as described above,
which takes as input the mixed audio tracks and
outputs the ASR with timestamps from the possible AD speaker.

The large-scale collection from AudioVault audio files is noisy,
\eg~some ADs are of lower-quality or are sourced from short movies. Therefore, we apply a stricter filtering step that 
removes movies containing fewer than 100 AD narrations 
or a word frequency of first- \& second-person pronouns larger than 5\%.

\subsection{Comparison with MAD-v1}
% Overview of the differences between MAD-v1 and MAD-v2
The key advantages of our pipeline are three-fold:
(1) it relies on \textit{audio-based} speaker separation to identify the AD speaker among the movie characters, 
whereas the pipeline in the original MAD work~\cite{soldan2022mad} relies on \textit{text-based} speaker separation by using the timestamps from the DVD subtitles and assumes any ASR transcription outside of these timestamps is AD. The error is propagated because the official subtitles are non-exhaustive (some dialogue is missed by the official subtitles). (2) It requires only the mixed audio as input, whereas MAD must also source the official DVD subtitles and align them -- presenting additional scaling costs and challenges. (3) It uses an advanced ASR model Whisper~\cite{radford2022whisper} which gives much more accurate transcriptions than previous methods, especially for punctuation and the spelling of names and other identities.

% \clearpage
\section{Qualitative Examples of MAD-v2 vs MAD-v1.}
\label{app:sec:qualitative-text}
More qualitative examples 
of MAD-v2 and MAD-v1 are shown in Fig.~\ref{fig:happyness} and~\ref{fig:sully}.
It is clear that our pipeline produces more accurate AD
compared to the original MAD-v1,
particularly in the spelling of names and 
the exclusion of dialogue.

\begin{figure}[h!]
    \centering
    \includegraphics[width=\textwidth]{img/ThePursuitofHappyness.pdf}
    \caption{\textbf{Comparison of the AD quality from MAD-v2 with MAD-v1.}
    The erroneous transcriptions are marked in {\color{red} red text}. 
    `Manual Verification' means we manually transcribe the AD narration from the audio track.
    The sample is originally from \emph{The Pursuit of Happyness} (2006).
    The failure mode of MAD-v1 in each example is 
    \textbf{(a)} dialogue leakage, \textbf{(b)} incorrect ASR, \textbf{(c)} missing words,
    \textbf{(d)} dialogue leakage, \textbf{(e)} incorrect ASR and name spelling,
    \textbf{(f)} incorrect name spelling.
    }
    \label{fig:happyness}
\end{figure}

\begin{figure}[h!]
    \centering
    \includegraphics[width=\textwidth]{img/Sully.pdf}
    \caption{\textbf{(continue) Comparison of the AD quality from MAD-v2 with MAD-v1.}
    The erroneous transcriptions are marked in {\color{red} red text}. 
    `Manual Verification' means we manually transcribe the AD narration from the audio track.
    The sample is originally from \emph{Sully: Miracle on the Hudson} (2016).
    The failure mode of MAD-v1 in each example is 
    \textbf{(a)} dialogue leakage, \textbf{(b)} dialogue leakage, 
    \textbf{(c)} dialogue leakage and incorrect ASR,
    \textbf{(d)} missing words, \textbf{(e)} number spelling and sentence partitioning,
    \textbf{(f)} missing words.
    }
    \label{fig:sully}
\end{figure}

\clearpage
{
% {\color{blue}
\section{Quantitative Comparison between MAD-v2 vs MAD-v1 on Grounding}
{
We re-purpose the CLIP zero-shot video-language grounding (VLG) performance from~\cite{soldan2022mad} as an indicator of dataset quality.
In detail, for both MAD-v2 and MAD-v1, 
we randomly choose a set of 5 movies from the \emph{training split},
and compute the VLG performance with frozen CLIP visual and textual encoders.
The AD textual quality and timestamps are the only factors that differ in this comparison.
We use the MAD training split because we did not modify the val/test splits, 
which are from LSMDC annotations.
The code to compute VLG performance is from~{\small\url{https://github.com/Soldelli/MAD}}.
The result in Table~\ref{tab:vlg} shows MAD-v2 annotations also benefit the VLG task.
}
\vspace{-2mm}
\begin{table}[h]
\centering
\small
\begin{tabular}{l|lll}
\hline
R@50 & IoU@0.1 & IoU@0.3 & IoU@0.5 \\ \hline
MAD-v1-\texttt{Unnamed} & 32.08 & 22.85 & 14.26 \\
MAD-v2-\texttt{Unnamed} & \textbf{33.25} & \textbf{24.22} & \textbf{15.58} \\ \hline
\end{tabular}
\vspace{-3mm}
\caption{CLIP zero-shot VLG performance on MAD-v1 and MAD-v2.}
\label{tab:vlg}
\vspace{-2mm}
\end{table}
}


\section{Additional Implementation Details}
\label{app:sec:implementation}
{
\paragraph{Design Choices.}
\begin{itemize}
\setlength\itemsep{-1mm}
    \item Number of frames per movie clip $N$: We choose $N=8$. Most AD annotations have a time duration of 1-3 seconds, equivalent to 5-15 frames (features) under 5 FPS -- the sampling rate provided by the MAD dataset. Therefore $N=8$ is a reasonable choice.
    \item Number of AD sentences as context $K$: We experiment $K\in\{1,2,3,6\}$ in Figure~\ref{fig:length}.
    \item Number of subtitles $L$: As described in Sect.~\ref{sec:exp:arch}, for simplicity we take the most recent 4 dialogues within a 1-minute time window. Note that the time distribution of subtitles varies a lot -- the most recent 4 dialogues could span just a few seconds or up to minutes before the current AD timestamp.
\end{itemize}
}

\paragraph{Evaluation Metrics.}
We use the \texttt{pycocoeval} package from~{\small \url{https://github.com/tylin/coco-caption}}
to compute the ROUGE-L, CIDEr, SPICE and METEOR.
The package post-processes both the predicted text and ground-truth text internally to remove the punctuation and make them lowercase.
To compute the BertScore, 
we use the package from~{\small \url{https://github.com/Tiiiger/bert_score}}.
Note that before computing the BertScore, 
both the predicted text and the ground-truth text are 
converted to lowercase without any punctuation,
as these are factors that the BertScore is sensitive to.

{
\paragraph{Alternative approach for vision-language fusion.}
We investigate an alternative vision \& language fusion mechanism whereby the context AD sentence prompts are fed as \emph{language features} rather than \emph{raw text
tokens}. 
Empirically, we observe that raw text inputs outperform language features 
(e.g. 12.6 CIDEr in Table~\ref{table:context} vs. about 8.0 CIDEr when feeding language features).
}

\section{Additional Qualitative Examples}
\label{app:sec:qualitative-ad}

\begin{figure}[t!]
    \centering
    \includegraphics[width=\textwidth]{img/HP_Gatsby.pdf}
    \caption{\textbf{Qualitative examples of AutoAD model.}
    We show the ground-truth AD and the AD predictions 
    under both the oracle and recurrent settings. 
    Previous AD context is shown in {\color{gray}gray}.
    Samples are taken from \emph{Harry Potter and the Order of the Phoenix} (2007) and \emph{The Great Gatsby} (2013).
    }
    \label{fig:more_qualitative}
\end{figure}

More qualitative examples are shown in Fig.~\ref{fig:more_qualitative}.
It shows that the AutoAD model gives reasonable descriptions
for the movie domain, 
like the actions (swim, dance), and face expression (eyes widen).
Note that under the oracle setting, the model is capable of learning character names (sample \textbf{a, c, d, f}) mainly due to the extra information from the ground-truth context. 
The model is still limited in its ability to identify characters accurately,~\eg in sample \textbf{f}, the movie shows Nick and \emph{Daisy} are dancing. Whereas the oracle prediction describes that Nick and \emph{Gatsby} are dancing, and the recurrent model simply predicts that a man and woman are dancing. {Also the pronouns often appear in the recurrent prediction, such as the word `his' in sample \textbf{a} and \textbf{b}, which shows the model learns the bias of pronouns but cannot recognize characters correctly.}
\section{Dataset Splits}
\label{app:sec:ids}

To clarify the dataset split of the movies in MAD and LSMDC, 
we list the movie IDs of each split used (and not used) in this paper.
The splits can also be found on the website~\small{\url{https://www.robots.ox.ac.uk/~vgg/research/autoad/}}.

\paragraph{MAD-v2.} 
It consists of 488 movies and all of them are used for training.
We provide the cleaner ADs for these movies using the automated pipeline described above.
This set is the same as the training set of movies proposed in MAD~\cite{soldan2022mad}. 
The movie IDs are:\\
{\scriptsize\texttt{
[2723, 2730, 2731, 2735, 2738, 2745, 2750, 2758, 2768, 2778, 2787, 2800, 2801, 2814, 2818, 2854, 2869, 2870, 2873, 2911, 2913, 2928, 2934, 2944, 2948, 2970, 2986, 2992, 2996, 3001, 3014, 3020, 3021, 3023, 3033, 3040, 3049, 3050, 3059, 3060, 3066, 3070, 3103, 3106, 3113, 3114, 3117, 3129, 3138, 3146, 3153, 3160, 3170, 3171, 3209, 3239, 3253, 3276, 3277, 3295, 3314, 3339, 3340, 3354, 3376, 3393, 3401, 3408, 3414, 3417, 3447, 3464, 3480, 3482, 3500, 3509, 3510, 3513, 3521, 3548, 3575, 3590, 3599, 3611, 3625, 3720, 3743, 3759, 3773, 3820, 3834, 3837, 3858, 3905, 3911, 3922, 3977, 4001, 4007, 4010, 4017, 4031, 4043, 4053, 4061, 4071, 4080, 4082, 4143, 4156, 4200, 4204, 4210, 4253, 4266, 4299, 4303, 4305, 4368, 4377, 4378, 4390, 4423, 4434, 4451, 4455, 4460, 4480, 4489, 4528, 4535, 4551, 4576, 4578, 4587, 4596, 4597, 4608, 4611, 4618, 4634, 4635, 4638, 4644, 4664, 4670, 4671, 4684, 4702, 4709, 4719, 4728, 4740, 4741, 4753, 4772, 4778, 4797, 4798, 4813, 4815, 4839, 4880, 4884, 4888, 4901, 4902, 4914, 4925, 4929, 4933, 4936, 4950, 4962, 4970, 4977, 4982, 4992, 5014, 5041, 5055, 5063, 5074, 5093, 5101, 5118, 5139, 5144, 5217, 5236, 5237, 5257, 5259, 5265, 5270, 5283, 5293, 5308, 5335, 5366, 5367, 5369, 5417, 5420, 5432, 5449, 5461, 5469, 5473, 5477, 5494, 5506, 5510, 5511, 5522, 5563, 5565, 5568, 5574, 5575, 5577, 5583, 5594, 5605, 5607, 5634, 5641, 5649, 5677, 5678, 5682, 5685, 5700, 5735, 5737, 5743, 5749, 5752, 5758, 5762, 5792, 5807, 5814, 5818, 5819, 5828, 5852, 5865, 5872, 5873, 5898, 5900, 5913, 5923, 5950, 5958, 6012, 6013, 6022, 6048, 6055, 6057, 6076, 6086, 6090, 6137, 6153, 6154, 6156, 6177, 6186, 6194, 6224, 6232, 6319, 6334, 6394, 6402, 6491, 6521, 6607, 6613, 6617, 6629, 6636, 6655, 6656, 6672, 6685, 6701, 6706, 6741, 6769, 6770, 6775, 6810, 6811, 6816, 6819, 6832, 6833, 6837, 6859, 6869, 6870, 6878, 6890, 6952, 6959, 6992, 6994, 7001, 7005, 7007, 7026, 7036, 7050, 7055, 7131, 7195, 7196, 7243, 7682, 7882, 8152, 8276, 8295, 8346, 8496, 8578, 8587, 8589, 8593, 8598, 8601, 8608, 8616, 8618, 8637, 8734, 8766, 8767, 8811, 9110, 9277, 9380, 9384, 9386, 9387, 9419, 9421, 9451, 9456, 9460, 9461, 9462, 9481, 9482, 9488, 9502, 9504, 9509, 9510, 9515, 9519, 9526, 9528, 9529, 9535, 9552, 9555, 9575, 9576, 9583, 9595, 9606, 9615, 9617, 9618, 9619, 9620, 9638, 9642, 9644, 9647, 9654, 9659, 9676, 9689, 9719, 9724, 9732, 9733, 9735, 9737, 9738, 9741, 9747, 9750, 9751, 9754, 9756, 9761, 9773, 9774, 9785, 9799, 9846, 9896, 9906, 9920, 9952, 10142, 10149, 10202, 10322, 10527, 10536, 10784, 10813, 10836, 10861, 10894, 10965, 11003, 11010, 11099, 11129, 11139, 11140, 11143, 11147, 11148, 11154, 11318, 11321, 11345, 11396, 11430, 11438, 11530, 11620, 11727, 11796, 11962, 12010, 12079, 12090, 12125, 12131, 12132, 12144, 12147, 12148, 12186, 12211, 12220, 12222, 12263, 12273, 12294, 12324, 12358, 12504, 12563, 12585, 12618, 12653, 12658, 12743, 12852, 12869, 12900, 12906, 12911, 12923, 12958, 13018, 13027, 13031, 13045, 13140, 13146, 13159, 13165, 13187, 13191, 13201]
}}

\paragraph{MAD-eval Evaluation Set.}
{
It consists of 10 movies,
which are obtained by 
$ set(\text{MAD val/test}) \cap set(\text{LSMDC val}) $,
that excluding LSMDC train movies for the ease of future comparison.
The annotations are inherited from the LSMDC dataset,
and we use both the \emph{named} and \emph{unnamed} version of it,
where the \emph{named} version can be downloaded from the LSMDC website\footnote{https://sites.google.com/site/describingmovies/download?authuser=0}.
The movie IDs are:\\
{\scriptsize\texttt{
[1005\_Signs,
1026\_Legion,
1027\_Les\_Miserables,
1051\_Harry\_Potter\_and\_the\_goblet\_of\_fire,
3009\_BATTLE\_LOS\_ANGELES,
3015\_CHARLIE\_ST\_CLOUD,
3031\_HANSEL\_GRETEL\_WITCH\_HUNTERS,
3032\_HOW\_DO\_YOU\_KNOW,
3034\_IDES\_OF\_MARCH,
3074\_THE\_ROOMMATE]}}
}

{
\paragraph{Unused MAD movies.}
151 movies from MAD val/test are \emph{not used} in either training or testing in our paper.
They are the intersection $ set(\text{MAD val/test}) \cap set(\text{LSMDC train/test}) $.
The movie IDs are:\\
{\scriptsize\texttt{
[0001\_American\_Beauty, 0002\_As\_Good\_As\_It\_Gets, 0003\_CASABLANCA, 0004\_Charade, 0005\_Chinatown, 0006\_Clerks, 0007\_DIE\_NACHT\_DES\_JAEGERS, 0008\_Fargo, 0009\_Forrest\_Gump, 0010\_Frau\_Ohne\_Gewissen, 0011\_Gandhi, 0012\_Get\_Shorty, 0013\_Halloween, 0014\_Ist\_das\_Leben\_nicht\_schoen, 0016\_O\_Brother\_Where\_Art\_Thou, 0017\_Pianist, 0019\_Pulp\_Fiction, 0020\_Raising\_Arizona, 0021\_Rear\_Window, 0022\_Reservoir\_Dogs, 0023\_THE\_BUTTERFLY\_EFFECT, 0026\_The\_Big\_Fish, 0027\_The\_Big\_Lebowski, 0028\_The\_Crying\_Game, 0029\_The\_Graduate, 0030\_The\_Hustler, 0031\_The\_Lost\_Weekend, 0032\_The\_Princess\_Bride, 0033\_Amadeus, 0038\_Psycho, 0041\_The\_Sixth\_Sense, 0043\_Thelma\_and\_Luise, 0046\_Chasing\_Amy, 0049\_Hannah\_and\_her\_sisters, 0050\_Indiana\_Jones\_and\_the\_last\_crusade, 0051\_Men\_in\_black, 0053\_Rendezvous\_mit\_Joe\_Black, 1001\_Flight, 1002\_Harry\_Potter\_and\_the\_Half-Blood\_Prince, 1003\_How\_to\_Lose\_Friends\_and\_Alienate\_People, 1004\_Juno, 1006\_Slumdog\_Millionaire, 1007\_Spider-Man1, 1008\_Spider-Man2, 1009\_Spider-Man3, 1010\_TITANIC, 1011\_The\_Help, 1012\_Unbreakable, 1014\_2012, 1015\_27\_Dresses, 1017\_Bad\_Santa, 1018\_Body\_Of\_Lies, 1019\_Confessions\_Of\_A\_Shopaholic, 1020\_Crazy\_Stupid\_Love, 1028\_No\_Reservations, 1031\_Quantum\_of\_Solace, 1033\_Sherlock\_Holmes\_A\_Game\_of\_Shadows, 1034\_Super\_8, 1035\_The\_Adjustment\_Bureau, 1037\_The\_Curious\_Case\_Of\_Benjamin\_Button, 1038\_The\_Great\_Gatsby, 1039\_The\_Queen, 1040\_The\_Ugly\_Truth, 1042\_Up\_In\_The\_Air, 1043\_Vantage\_Point, 1045\_An\_education, 1046\_Australia, 1047\_Defiance, 1048\_Gran\_Torino, 1050\_Harry\_Potter\_and\_the\_deathly\_hallows\_Disk\_One, 1052\_Harry\_Potter\_and\_the\_order\_of\_phoenix, 1054\_Harry\_Potter\_and\_the\_prisoner\_of\_azkaban, 1055\_Marley\_and\_me, 1057\_Seven\_pounds, 1058\_The\_Damned\_united, 1059\_The\_devil\_wears\_prada, 1060\_Yes\_man, 1061\_Harry\_Potter\_and\_the\_deathly\_hallows\_Disk\_Two, 1062\_Day\_the\_Earth\_stood\_still, 3001\_21\_JUMP\_STREET, 3002\_30\_MINUTES\_OR\_LESS, 3003\_40\_YEAR\_OLD\_VIRGIN, 3004\_500\_DAYS\_OF\_SUMMER, 3005\_ABRAHAM\_LINCOLN\_VAMPIRE\_HUNTER, 3007\_A\_THOUSAND\_WORDS, 3008\_BAD\_TEACHER, 3012\_BRUNO, 3013\_BURLESQUE, 3014\_CAPTAIN\_AMERICA, 3016\_CHASING\_MAVERICKS, 3017\_CHRONICLE, 3018\_CINDERELLA\_MAN, 3020\_DEAR\_JOHN, 3022\_DINNER\_FOR\_SCHMUCKS, 3023\_DISTRICT\_9, 3024\_EASY\_A, 3025\_FLIGHT, 3026\_FRIENDS\_WITH\_BENEFITS, 3028\_GHOST\_RIDER\_SPIRIT\_OF\_VENGEANCE, 3030\_GROWN\_UPS, 3033\_HUGO, 3035\_INSIDE\_MAN, 3036\_IN\_TIME, 3037\_IRON\_MAN2, 3038\_ITS\_COMPLICATED, 3039\_JACK\_AND\_JILL, 3040\_JULIE\_AND\_JULIA, 3041\_JUST\_GO\_WITH\_IT, 3042\_KARATE\_KID, 3043\_KATY\_PERRY\_PART\_OF\_ME, 3045\_LAND\_OF\_THE\_LOST, 3046\_LARRY\_CROWNE, 3047\_LIFE\_OF\_PI, 3048\_LITTLE\_FOCKERS, 3049\_MORNING\_GLORY, 3050\_MR\_POPPERS\_PENGUINS, 3051\_NANNY\_MCPHEE\_RETURNS, 3052\_NO\_STRINGS\_ATTACHED, 3053\_PARENTAL\_GUIDANCE, 3054\_PERCY\_JACKSON\_LIGHTENING\_THIEF, 3055\_PROMETHEUS, 3056\_PUBLIC\_ENEMIES, 3058\_RUBY\_SPARKS, 3060\_SANCTUM, 3061\_SNOW\_FLOWER, 3062\_SORCERERS\_APPRENTICE, 3063\_SOUL\_SURFER, 3066\_THE\_ADVENTURES\_OF\_TINTIN, 3067\_THE\_ART\_OF\_GETTING\_BY, 3069\_THE\_BOUNTY\_HUNTER, 3070\_THE\_CALL, 3071\_THE\_DESCENDANTS, 3072\_THE\_GIRL\_WITH\_THE\_DRAGON\_TATTOO, 3073\_THE\_GUILT\_TRIP, 3075\_THE\_SITTER, 3076\_THE\_SOCIAL\_NETWORK, 3077\_THE\_VOW, 3078\_THE\_WATCH, 3079\_THINK\_LIKE\_A\_MAN, 3081\_THOR, 3082\_TITANIC1, 3083\_TITANIC2, 3084\_TOOTH\_FAIRY, 3085\_TRUE\_GRIT, 3086\_UGLY\_TRUTH, 3087\_WE\_BOUGHT\_A\_ZOO, 3088\_WHATS\_YOUR\_NUMBER, 3089\_XMEN\_FIRST\_CLASS, 3090\_YOUNG\_ADULT, 3091\_ZOMBIELAND, 3092\_ZOOKEEPER]}}

}