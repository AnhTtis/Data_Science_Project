% ****** Start of file apssamp.tex ******
%
%   This file is part of the APS files in the REVTeX 4.2 distribution.
%   Version 4.2a of REVTeX, December 2014
%
%   Copyright (c) 2014 The American Physical Society.
%
%   See the REVTeX 4 README file for restrictions and more information.
%
% TeX'ing this file requires that you have AMS-LaTeX 2.0 installed
% as well as the rest of the prerequisites for REVTeX 4.2
%
% See the REVTeX 4 README file
% It also requires running BibTeX. The commands are as follows:
%
%  1)  latex apssamp.tex
%  2)  bibtex apssamp
%  3)  latex apssamp.tex
%  4)  latex apssamp.tex
%
\documentclass[%
 reprint,
%superscriptaddress,
%groupedaddress,
unsortedaddress,
%runinaddress,
%frontmatterverbose, 
%preprint,
%preprintnumbers,
%nofootinbib,
%nobibnotes,
%bibnotes,
 amsmath,amssymb,
 aps,
%pra,
%prb,
%rmp,
%prstab,
%prstper,
%floatfix,
]{revtex4-2}
%\usepackage{physics}
\usepackage{graphicx}% Include figure files
\usepackage{dcolumn}% Align table columns on decimal point
\usepackage{bm}% bold math
\usepackage{xcolor}
%\usepackage{hyperref}% add hypertext capabilities
%\usepackage[mathlines]{lineno}% Enable numbering of text and display math
%\linenumbers\relax % Commence numbering lines

%\usepackage[showframe,%Uncomment any one of the following lines to test 
%%scale=0.7, marginratio={1:1, 2:3}, ignoreall,% default settings
%%text={7in,10in},centering,
%%margin=1.5in,
%%total={6.5in,8.75in}, top=1.2in, left=0.9in, includefoot,
%%height=10in,a5paper,hmargin={3cm,0.8in},
%]{geometry}

\newcommand{\ket}[1]{\left|#1\right\rangle}
\newcommand{\bra}[1]{\left\langle #1\right|}
% \newcommand{\ket}[1]{\left|#1\right\rangle}
\newcommand{\braket}[3]{\left\langle #1\left| #2 \right| #3\right\rangle}

\begin{document}

%\preprint{APS/123-QED}

\title{Degeneracy in excited state quantum phase transitions of two-level bosonic models and its influence on system dynamics}% Force line breaks with \\
%\thanks{A footnote to the article title}%

\author{J. Khalouf-Rivera}
\affiliation{Departamento de Ciencias Integradas y Centro de Estudios Avanzados en Física, Matemáticas y Computación, Universidad de Huelva, Huelva 21071, Spain}

\author{Qian Wang}
\affiliation{Department of Physics, Zhejiang Normal University, Jinhua 321004, China}
\altaffiliation[Also at ]{CAMTP-Center for Applied Mathematics and Theoretical Physics, University of Maribor, 
Mladinska 3, SI-2000 Maribor, Slovenia}

\author{Lea F. Santos}
\affiliation{Department of Physics, University of Connecticut, Storrs, CT, USA}

\author{J.E.~García Ramos}
\affiliation{Departamento de Ciencias Integradas y Centro de Estudios Avanzados en Física, Matemáticas y Computación, Universidad de Huelva, Huelva 21071, Spain}
\altaffiliation[Also at ]{Instituto Carlos I de F\'{\i}sica Te\'orica y Computacional, Universidad de Granada, Fuentenueva s/n, 18071 Granada, Spain}

\author{M.~Carvajal}
\affiliation{Departamento de Ciencias Integradas y Centro de Estudios Avanzados en Física, Matemáticas y Computación, Universidad de Huelva, Huelva 21071, Spain}
\altaffiliation[Also at ]{Instituto Carlos I de F\'{\i}sica Te\'orica y Computacional, Universidad de Granada, Fuentenueva s/n, 18071 Granada, Spain}

\author{F.~Pérez-Bernal}
\affiliation{Departamento de Ciencias Integradas y Centro de Estudios Avanzados en Física, Matemáticas y Computación, Universidad de Huelva, Huelva 21071, Spain}
\altaffiliation[Also at ]{Instituto Carlos I de F\'{\i}sica Te\'orica y Computacional, Universidad de Granada, Fuentenueva s/n, 18071 Granada, Spain}


%\collaboration{CLEO Collaboration}%\noaffiliation

\date{\today}% It is always \today, today,
             %  but any date may be explicitly specified

\begin{abstract}
Excited state quantum phase transitions in collective many-body quantum systems influence the system spectral properties due to changing degeneracy patterns in different phases. We report a fundamental difference in the nature of the degeneracy for boson two-level models, depending on the dimension of the space where the models were defined. To illustrate the consequences, we assess the validity of an out-of-time-order correlator as a possible order operator for excited state quantum phase transitions in different models.
%% \begin{description}
%% \item[Usage]
%% Secondary publications and information retrieval purposes.
%% \item[Structure]
%% You may use the \texttt{description} environment to structure your abstract;
%% use the optional argument of the \verb+\item+ command to give the category of each item. 
%% \end{description}
\end{abstract}

%\keywords{Suggested keywords}%Use showkeys class option if keyword
                              %display desired
\maketitle

%\tableofcontents

%\section{\label{sec:level1}First-level heading:\protect\\ The line
%break was forced \lowercase{via} \textbackslash\textbackslash}

Quantum phase transitions (QPTs) are zero-temperature phase transitions driven by quantum instead of thermal fluctuations. In QPTs the system's ground state undergoes abrupt changes once one or several Hamiltonian parameters (control parameters) go through particular (critical) values \cite{Carr2010}. 
In collective systems, true discontinuities only happen in the large system size limit (also called mean-field or thermodynamic limit) though QPT precursors can be found in finite-size systems \cite{Iachello2004}.  Particular attention has been paid to such transitions in algebraic models of application to the study of nuclear structure \cite{Cejnar2009, *Casten2009, *Cejnar2010}. The QPT concept was later extended to encompass the realm of excited states, introducing the excited state quantum phase transition (ESQPT) concept \cite{Caprio2008, *Cejnar2008}. An extensive review paper on ESQPTs has been recently published where the interested reader can find the main developments in the field \cite{[][{, and references therein.}]Cejnar2021}.

Two-level bosonic models have been extensively used as approximations to widely different physical systems. In the present work we focus on two such models: the Lipkin-Meshkov-Glick (LMG) model and  the two-dimensional limit of the vibron model (2DVM). We also extend our results to other two models,  the vibron model (VM) and the interacting boson model (IBM) in App.~\ref{appa}. These four models have a similar algebraic structure, with a unitary Lie algebra $u(n+1)$ as their dynamical algebra, where $n$ is the dimension of the problem under study \cite{general0}. In the four cases, the dynamical algebra generators can be built as bilinear products of creation and annihilation boson operators. In accordance with the algebraic approach, the different subalgebra chains that start in the dynamical algebra and end in the system's symmetry algebra are called \textit{dynamical symmetries} and they provide solvable example of physical limits of interest, as well as a basis to carry out calculations \cite{bookalg,frank}. The ground state QPTs for the aforesaid four models, among others, were studied in Ref.~\cite{Cejnar2007} considering a general system Hamiltonian including one- and two-body interactions.

The LMG model is a one dimensional model ($n = 1$) introduced in the 60's as a toy model for the study of the validity of approximate methods in nuclear structure studies   \cite{Lipkin1965,*Meshkov1965,*Glick1965}. Besides its simplicity, due to its rich physical content, the LMG model has been extensively used in the study of  QPTs \cite{Dusuel2004,*Ribeiro2008,*Engelhardt2013,*Campbell2016,*Heyl2018,Castanos2006,Romera2014} and ESQPTs \cite{Relano2008, *PFernandez2009, *PFernandez2011, Yuan2012, Santos2016, Wang2019a, Wang2019b, Nader2021,Gamito2022}. The interest on the LMG model has  been further fostered by the existence of different realizations for this model \cite{Morrison2008, Larson2010, Zibold2010, AFerreira2013, Jurcevic2014, *Jurcevic2017, *Muniz2020, Makhalov2019,Cervia2021,Hlatshwayo2022,liarxiv}. This model can be realized with a chain of $N_2$ spins  subject to all-to-all long range interactions. In this case, introducing a collective pseudospin, $\hat J_\alpha =\frac{1}{2} \sum_i^{N_2}\sigma_{i,\alpha}$, where $\alpha = x, y, z$ and $\sigma_{i,\alpha}$ is the  $\alpha$ Pauli spin matrix for the $i$-th spin in the chain.  The general LMG model has first-, second-, and third-order ground-state QPTs \cite{Castanos2006, Romera2014}. We can define a simplified LMG model Hamiltonian
\begin{equation}
  \hat H_{LMG} = (1-\xi)\left(\frac{\hat N_{2}}{2} + \hat J_z\right) - \frac{4\xi}{N_{2}}  \hat J_x^2~,
  \label{HLMGM}
\end{equation}
\noindent with a control parameter \(\xi\in[0,1]\). This Hamiltonian is written in an intensive form dividing the second term, a two-body term, by the system size, $N_2$, to make possible to study the large size limit of the system. The algebraic structure of this model is clearer in a bosonic realization, introducing a scalar boson, \(s^\dagger (s)\), and a pseudoscalar boson, \(t^\dagger (t)\). The four generators of the $u(n+1) = u(2)$ dynamical algebra are the bilinear products \(\{s^\dagger s, t^\dagger t, t^\dagger s, s^\dagger t\}\) and, making use of a Schwinger transform, the generators can be recast in the form of the quasispin components with \(\{\hat N, \hat J_x, \hat J_y, \hat J_z\}\) \cite{frank} 
\begin{align*}
  \hat N_2 &= t^\dagger t + s^\dagger s~, &\hat J_x &= \frac{1}{2}(t^\dagger s + s^\dagger t)~,\\
  \hat J_y &=  \frac{1}{2i}(t^\dagger s - s^\dagger t) ~, &\hat J_z &=  \frac{1}{2}(t^\dagger t - s^\dagger s)~.
\end{align*}
\noindent There are two possible dynamical symmetries starting from the $u(2)$ dynamical algebra
\begin{align}
&u(2)\supset u(1)& \mbox{Chain (LMG-I)}~,\label{lmgmi}\\
&u(2)\supset so(2) &\mbox{Chain (LMG-II)}~.\label{mgmii}
\end{align}
The \(\hat N_2\) operator in this case is a constant and it is equal to the total number of \(s\) and \(t\) bosons, denoting the totally symmetric \(u(2)\) irrep that spans the system's Hilbert space. The Hamiltonian Eq.~\eqref{HLMGM} can be recast as 
\begin{equation}
  \hat H_{LMG} = (1-\xi)\hat n_t - \frac{4\xi}{N_{2}}  \hat P_t~,
  \label{HLMGM_boson}
\end{equation}
\noindent where $\hat n_t = t^\dagger t$ and $\hat P_t = \frac{1}{4}(t^\dagger s + s^\dagger t)^2$.  Hence, the model Hamiltonian combines the first order Casimir operator of $u(1)$ and the second order Casimir operator of $so(2)$. The LMG-I(II) dynamical symmetries are recovered for \(\xi = 0(1)\). The simplified \(\hat H_{LMG}\) can be split into even and odd symmetry blocks, as it conserves the parity \(\hat \Pi = e^{\imath \pi \hat{n}_t }\). The model has a second-order ground-state QPT for a critical value of the control parameter \(\xi_c = 0.2\) and an associated ESQPT  \cite{Relano2008, Santos2016}. The high density of states associated with the ESQPT along the separatrix line can be clearly seen in the correlation energy diagram shown in the panel (a) of Fig.~\ref{lmg_2dvm_ced} for a LMG model with a system size \(N_2 = 50\). It is also apparent how even and odd parity states (depicted with full blue and dashed red lines, respectively) are degenerate in the broken-symmetry region ($\xi > \xi_c$) for states below the critical energy marked by the separatrix. This degeneracy is broken for states at energies above the critical ESQPT energy. This fact has been used to define a constant of the motion able to identify ESQPT phases in quantum collective models in one dimension \cite{Corps2021,Corps2022}.

\begin{figure*}
  \includegraphics[width=0.75\textwidth, angle=-90]{Ediff_U2-U3-U4-U6_N50_test.pdf}
\caption{\label{lmg_2dvm_ced} (Color online) Panel (a): Excitation energy scaled by the system size ($\varepsilon = (E-E_0)/N_2$) as a function of the control parameter $\xi$ for the LMG model with model Hamiltonian Eq.~\eqref{HLMGM_boson} and system size \(N_2 = 50\). Full blue (dashed red) lines mark even (odd) parity levels.  Panel (b): Excitation energy scaled by the system size ($\varepsilon = (E-E_0)/N_3$) as a function of the control parameter $\xi$ for the 2DVM model Hamiltonian Eq.~\eqref{H2DVM} with a system size \(N_3 = 50\). Full blue (dashed red) lines mark levels with angular momentum \(\ell = 0 (1)\). In both panels the color-filled area marks the energy difference  between selected states with different parity (panel (a)) or angular momentum (panel (b)). Panel (c): Energy difference between selected pairs of states of the LMG model Hamiltonian Eq.~\eqref{HLMGM_boson} having different parity as a function of the control parameter $\xi$. Panel (d): Energy difference between selected states of Hamiltonian Eq.~\eqref{H2DVM} with angular momentum \(\ell = 0\) and \(1\) as a function of the control parameter $\xi$. In both cases, the labels of the selected pairs of levels are provided in the legend of the panels and the color used for each pair of states is the same color used to fill the corresponding area in the upper panel.}
\end{figure*}

We compare the LMG model results with results obtained for the 2DVM, a model that was first introduced for the study of vibrational bending degrees of freedom in molecules \cite{Iachello1996}. Molecular bending vibrations are  two-dimensional, thus \(n = 2\) and the system dynamical algebra is the $u(3)$ Lie algebra. In this approach, bending vibrations are treated as collective bosonic excitations  and the model building blocks are a scalar boson operator \(\sigma^\dagger (\sigma)\) and two circular bosons \(\tau_i^\dagger (\tau_i)\) with \(i = +,-\) \cite{PBernal2008}. As in the LMG model case, the nine  \(u(3)\) generators are the bilinear product of creation and annihilation operators \cite{Iachello1996, PBernal2008}. The 2DVM is the simplest two-level model with nontrivial angular momentum, a fact that has made it a convenient model for QPT and ESQPT studies \cite{Iachello2003, PBernal2005, Caprio2008, PBernal2008,a2DVM_2010, isomersU3,KRivera2022,a2DVM_2022}. In fact, the first experimental signatures of ESQPT precursors have been found in the molecular bending spectrum of nonrigid molecules \cite{Larese2011,*Larese2013,*KRivera2020}.

As in the LMG model, the 2DVM has two dynamical symmetries that,  due to angular momentum conservation, converge in the system's symmetry algebra, $so(2)=\text{span}\left\{\hat{\ell}=\tau_+^{\dagger}\tau_+-\tau_-^{\dagger}\tau_-\right\}$,
\begin{align}
&u(3)\supset u(2) \supset so(2)&\mbox{Chain (2DVM-I)}~,\label{2dvmi}\\
&u(3)\supset so(3)\supset so(2) &\mbox{Chain (2DVM-II)}~.\label{2dvmii}
\end{align}

In this case, the total number of \(\sigma\) and \(\tau\) bosons is \(N_3\), a constant that denotes the totally symmetric \(u(3)\) irrep that spans the system's Hilbert space. The dynamical symmetry in Eq.~\eqref{2dvmi} is a convenient approximation to model the bending degrees of freedom of a linear molecule, while Eq.~\eqref{2dvmii} dynamical symmetry is applied to the modeling of rigidly-bent molecular species \cite{Iachello1996, PBernal2008, KRivera2020}.  In the same spirit as we have done for the LMG model, we introduce a simple model Hamiltonian that allows for the transition between the two limiting cases associated with the dynamical symmetries, using \(\hat n_\tau\), the first order Casimir operator of the \(u(2)\) subalgebra in Eq.~\eqref{2dvmi} and a pairing operator \(\hat P_\tau\), built with the second order Casimir operator of \(so(3)\) in Eq.~\eqref{2dvmii}

\begin{equation}
  \hat H_{2DVM} = (1-\xi) \hat n_\tau + \frac{\xi}{N_{3}}  \hat P_\tau~,
  \label{H2DVM}
\end{equation}
\noindent with a control parameter \(\xi\in[0,1]\), \(\hat n_\tau = \tau^\dagger_+\tau_+ + \tau^\dagger_-\tau_-\), and \(\hat P_\tau =  N_3^2 - \hat W^2\) where \(\hat W^2\) is the second order Casimir operator of \(so(3)=\text{span}\left\{\hat{D}_{\pm}=\sqrt{2}\left(\pm \tau_\pm^{\dagger}\sigma \mp \sigma^{\dagger}\tau_{\mp}\right),\hat{\ell}\right\}\)~\cite{PBernal2008}. This model Hamiltonian is block-diagonal for states belonging to different irreps of the symmetry algebra \(so(2)\), as expected considering the conservation of the system angular momentum, \(\ell\) \footnote{We used the notation $\ell$ for the \(so(2)\) quantum number, that is commonly used in molecular spectroscopy for the vibrational angular momentum in the doubly-degenerate bending degree of freedom of linear species.}. In molecular bending vibrations this conserved quantity 
is identified either with the vibrational angular momentum (in the \(u(2)\) dynamical symmetry) or with the projection of the angular momentum on the molecular figure axis (in the \(so(3)\) dynamical symmetry) \cite{PBernal2008, KRivera2020}. The correlation energy diagram for the 2DVM model is shown in panel (b) of Fig.~\ref{lmg_2dvm_ced}, where excitation energies  for \(\ell = 0\) (full blue lines) and \(1\) (dashed red lines) states are plot as a function of the control parameter \(\xi\) for a system size \(N_3 = 50\). At first sight, the results are completely equivalent to the ones obtained for the LMG model and shown in  panel (a) of Fig.~\ref{lmg_2dvm_ced}. As in the LMG model, the 2DVM has a second order ground state QPT and an associated ESQPT \cite{PBernal2008}. In this case the degeneration in the broken-symmetry phase occurs for eigenstates with different angular momentum values with energies under the ESQPT critical energy, instead of states with different parity. Again, the critical energy is marked by a separatrix with a high density of states. States with angular momentum \(\ell > 1\) and energy under the ESQPT critical energy also become degenerate and they have not been shown in the figure for the sake of clarity.


Despite the different $n$ values of the models considered, their underlying algebraic structure is very similar. Both models have a \(u(n)\) dynamical algebra and undergo a second-order ground-state QPT between \(u(n-1)-so(n)\) and its associated ESQPT. In fact, the same happens for the two models considered in App.~\ref{appa}, with $n=3,5$. From the energy functional obtained using the coherent state formalism, the four cases considered reduce to the cusp catastrophe \cite{Cejnar2007}. Therefore, it could be expected that the ESQPT precursors and their influence on system dynamics would be common to the different models. This fact has been verified in some instances, e.g.~the participation ratio of the system's eigenstates expressed in the basis associated with the \(u(n-1)\) dynamical symmetry has been proposed as a precursor of the ESQPT obtaining similar results in the  LMG model \cite{Santos2016}, the 2DVM \cite{PBernal2017}, and the three dimensional vibron model \cite{Santos2015}. In the present work, we pay heed to a particular aspect that implies a qualitative difference between the one-dimensional LMG model and the 2DVM or other models in higher dimensions (see results in App.~\ref{appa}). From the (a) and (b) panels in   Fig.~\ref{lmg_2dvm_ced}, it is clear that states under the critical energy of the ESQPT in the broken-symmetry region are degenerate. This implies a degeneracy of states with even and odd parity in the LMG model or states from different rotational bands in the 2DVM case. However, the nature of this degeneracy in the LMG model is utterly different to the degeneracy in the 2DVM (or in the models included in App.~\ref{appa}). This fact was already pointed out by Caprio, Cejnar, and Iachello in the ESQPT seminal reference \cite{Caprio2008} (see Fig.~6 and the second paragraph in page 1120 of this reference). The difference of energy between pairs of selected levels belonging to different Hamiltonian matrix blocks is highlighted in  panels (a) and (b) of Fig.~\ref{lmg_2dvm_ced} by filling the energy interval with certain colors. As expected, the width of the colored surface decreases as the system gets closer to the critical ESQPT energy and it disappears once the pair of levels become degenerate as they go under the corresponding ESQPT critical energy. This is what has recently been called \textit{level kissing} and experimentally shown using a Kerr oscillator realized with circuit QED \cite{Frattiniarxiv, Chavezarxiv}.  The value of the energy difference between the selected pairs of levels for the LMG model and the 2DVM are depicted, in logarithmic scale and using the same colors, in panels (c) and (d) of Fig.~\ref{lmg_2dvm_ced}.

%% \begin{figure*}
%% \includegraphics[width=0.85\textwidth, angle=-90]{Ediff_U2-U3-U4-U6_N50.pdf}
%% \caption{\label{fig:ediffs} (Color online) }
%% \end{figure*}

From panel (c) of Fig.~\ref{lmg_2dvm_ced}, it is clear that once that a pair of states cross the ESQPT separatrix in the LMG model they  become degenerate up to the numerical precision of our calculation even for finite-size systems. The situation in the 2DVM is utterly different, as shown in panel (d) of Fig.~\ref{lmg_2dvm_ced}. In the 2DVM case, the degeneracy is not complete, the energy differences decrease to a value close to \(10^{-2}\), thus levels are not completely degenerate until the control parameter $\xi$ is equal to unity. The comparison of the results in panels (c) and (d) of Fig.~\ref{lmg_2dvm_ced} indicates a fundamental difference between the models. Energy differences for adjacent states with different parity in the LMG model are already zero  in the finite size case, while in the 2DVM we expect that the corresponding energy differences decrease with the system size, and will be zero only in the thermodynamic limit. 

The different nature of the degeneracy in the two models considered can be understood from the panels in Fig.~\ref{figdeltaE}. In both panels energy differences between the first four pair of states with different symmetry for a control parameter $\xi= 0.5$ as a function of the system size are shown. Panel (a) includes results for the LMG model, computing the difference between even and odd states. However, data shown in panel (b) are for the 2DVM corresponding states with angular momentum  $\ell = 0$ and $\ell = 1$. The abscissa axis is linear in the case of the LMG model, while it is logarithmin in the 2DVM case. This implies that the energy splitting between even and odd states in the LMG model decreases exponentially, while in the 2DVM the splitting follows a power law. In the LMG case, to achieve the required accuracy in the calculations, the use library for real and complex floating-point arithmetic with arbitrary precision has been required \cite{mpmath}.



%\begin{figure}
%\includegraphics[width=0.5\textwidth]{delta_E_lmg_xi05_i03.pdf}
%\caption{\label{figdeltaE} (Color online) Panel (a) Differences of enegy  between even and odd energy levels in a LMG model \(\Delta E_i = E_i^- - E_i^+\), \(i = 0, 1, 2, 3\). The %control parameter value is \(\xi = 0.5\) and the results are depicted as a function of the system size \(N_2\) in logarithmic scale.}
%\end{figure}

\begin{figure}
\includegraphics[width=0.5\textwidth]{LMG_2DVM_degeneracy_OTOC_xi05.pdf}
\caption{\label{figdeltaE}   (Color online) Panel (a): Differences of energy  between even and odd energy levels in the LMG model \(\Delta E_i = E_i^- - E_i^+\) with \(i = 0, 1, 2, 3\) as a function of the system size, \(N_2\), using log-lin axes. Black lines are the result of fitting the depicted data to an exponential law. Panel (b): Differences of energy  between $\ell = 0$ and $\ell =1 $ energy levels in the 2DVM model, \(\Delta E_i = E_i^{\ell = 1} - E_i^{\ell = 0}\) with \(i = 0, 1, 2, 3\) as a function of the system size, \(N_3\), using log-log axes. Black lines are the result of fitting the depicted data to a power law. In both cases the control parameter value is \(\xi = 0.5\).}
\end{figure}


Such a striking difference on the degeneracy patterns in the two models is expected to have a noticeable influence over the system dynamics. In order to illustrate this point, we consider the time evolution of an out-of-time-order correlator (OTOC). 
%\textcolor{red}{Example of consequence on the dynamics of the systems: long-time limit average value of an OTOC as an ESQPT order parameter for finite size systems \cite{Wang2019b}. Only valid in the LMGM case, in the rest of the cases this quantity is zero. Evince the difference between phases with the evolution with time of the OTOC.}
An OTOC is a four-point temporal correlation function, originally introduced in the context of superconductivity studies~\cite{Larkin1969}. In recent times, OTOCs have been in the limelight  for two main reasons \cite{Swingle2018}. On the first hand, they were suggested as an efficient quantum chaos probe as it was shown that OTOCs grow exponentially, in accordance with the system Lyapunov exponent value, at early times in non-integrable systems \cite{kitaev, Shenker2014, Hashimoto2017, *Roberts2015, *Maldacena2016,  *Akutagawa2020, *Zhao2021}. Also, it has been found that in chaotic systems they approach their long-time limit saturation value in a characteristic way \cite{IGarcia2018, *Rammensee2018, *Fortes2019}. On the second hand, they can be used to quantify information scrambling in quantum systems, as they depend on the system entanglement spread \cite{Swingle2016a, *Lewis2019, *Xu2019, *Niknam2020}. Despite the fact that the unusual time ordering of its constituents operators hinders the experimental access to OTOCs using local operators, several approaches using different platforms have successfully provided OTOC data \cite{Li2017, *Garttner2017, *Wei2018, *Landsman2019, *Pegahan2021, *Green2022, *Braumuller2022}. OTOCs have also been recently used to characterize QPTs \cite{Shen2017, *Heyl2018, *Ceren2019, *Nie2020, *Lewis2020, *Bin2023}. Finally, it has been recently found that they  are also valuable ESQPT probes because the instability associated with ESQPTs' critical points fosters an exponential OTOC increase at short times, even in integrable systems \cite{ChavezC2019, *Pilatowsky2020, *Hashimoto2020, *Xu2020, Chavezarxiv}. 

%%%%%%%%%%%%%%%%%%%%%%%%%%%%%%%%%%%%%%%%%%%%%%%%%%%%%%%%%%%%%%%%%%%%%%%%%%%%%%%%%%%%%%%%%%%%%%%%%%%%%%%%%%%%%
Given any two operators, $\hat{W}$ and $\hat{V}$, where $\hat{W}(t)$ is the operator $\hat{W}$ in the Heisenberg's representation, $\hat{W}(t)=e^{\imath\hat{H}t}\hat{W}e^{-\imath\hat{H}t}$, the spread of $\hat{W(t)}$ with $\hat{V}$ can be obtained through the expectation value of the squared commutator \cite{Swingle2018, Swingle2016a, Hashimoto2017}

\begin{equation}
{\cal C}_{w,v}(t)=\left\langle \left[\hat{W}(t),\hat{V}(0)\right]^{\dagger}\left[\hat{W}(t),\hat{V}(0)\right]\right\rangle~.
\label{C}
\end{equation}

The expectation value is usually computed in the canonical ensemble, however, it can also be computed over the system eigenstates (microcanonical OTOC) \cite{Hashimoto2017, Hashimoto2020}. The squared commutator Eq.~(\ref{C}) can be expressed as ${\cal C}_{w,v}(t)= {\cal A}_{w,v}(t) -2\Re{[{\cal F}_{w,v}(t)]}$: the sum of a two-point correlator,  $ {\cal A}_{w,v}(t) = \left\langle \hat{W}^{\dagger}(t)\hat{V}^{\dagger}(0)\hat{V}(0)\hat{W}(t)\right\rangle + \left\langle \hat{V}^{\dagger}(0)\hat{W}^{\dagger}(t)\hat{W}(t)\hat{V}(0)\right\rangle$ plus  a four-point correlator, ${\cal F}_{w,v}(t)$, where out-of-time order effects take place
\begin{equation}
{\cal F}_{w,v}(t)=\left\langle \hat{W}^{\dagger}(t)\hat{V}^{\dagger}(0)\hat{W}(t)\hat{V}(0)\right\rangle.
\label{OTOC}
\end{equation}

Two of the authors (QW and FPB) have recently shown for the LMG model that the long time average value of an OTOC operator is a valid ESQPT order operator~\cite{Wang2019b}. A similar OTOC has proved also a valid order parameter in an anharmonic LMG model \cite{KRivera2022b}. In both cases, the average value of the OTOC at the infinite time limit is able to distinguish between phases below and above the ESQPT critical energy. In the present work, our aim is to show how different are the results obtained for the long time OTOC values depending on the model dimensionality, something that stems from the different nature of the degeneracy below the ESQPT critical energy shown in Fig.~\ref{lmg_2dvm_ced}. For this purpose, we derive the formulas needed to compute the long-time average value of the microcanonical OTOC. Using Eq.~\eqref{OTOC} and the closure for the system eigenstates, \(|\psi_k\rangle\) with \(k = 1, \ldots, D\), the microcanonical OTOC for a system \(j\)-th eigenstate, \({\cal F}^{(j)}_{VW}(t)\), can be expressed as follows 
\begin{widetext}
\begin{equation}
  {\cal F}^{(j)}_{VW}(t)  = \sum_{j_1,j_2,j_3}^D e^{i\omega(j, j_1,j_2,j_3)t} \langle\psi_j|\hat W^\dagger|\psi_{j_1}\rangle\langle\psi_{j_1}|\hat V^\dagger|\psi_{j_2}\rangle\langle\psi_{j_2}|\hat W|\psi_{j_3}\rangle\langle\psi_{j_3}|\hat V|\psi_{j}\rangle = 
\sum_{j_1,j_2,j_3}^D e^{i\omega(j, j_1,j_2,j_3)t}N^{(j)}_{j_1,j_2,j_3}~,\label{Fjt}
\end{equation}
\end{widetext}
where \(\omega(j, j_1,j_2,j_3) =  E_{j} + E_{j_2} - E_{j_1} - E_{j_3}\).

The time-averaged value of  \({\cal F}^{(j)}_{VW}(t)\) in the infinite time limit, \(\overline{{\cal F}^{(j)}_{VW}}\), is the equilibrium value of this four-point correlator

\begin{align}
  \overline{{\cal F}^{(j)}_{VW}}=&\lim_{T\to\infty}\frac{1}{T}\int_0^T {\cal F}^{(j)}_{VW}(t)dt=\lim_{T\to\infty}\overline{{\cal F}^{(j)}_{VW}}(T)\nonumber\\
  =& \sum_{j_1,j_2,j_3}^DN^{(j)}_{j_1,j_2,j_3}\delta_{\omega(j,j_1,j_2,j_3), 0}~.\label{AvgF}
\end{align}



\begin{figure*}
\includegraphics[width=0.9\textwidth]{figure_paper_ESQPT_bar_F_T_LMG_2DVM.pdf}
\caption{\label{figFT} (Color online) Panel (a): Time-averaged value of  \({\cal F}^{(j)}_{VW}(t)\) for the LMG model with \(\hat V =\hat W = \hat J_x\) as a function of the system's excitation energy scaled by the system size ($E/N_2$) for even parity eigenstates of Eq.~\eqref{HLMGM} Hamiltonian with a system size \(N_2 = 300\). Panel (b): Time-averaged value of  \({\cal F}^{(j)}_{VW}(t)\) with  \(\hat V = \hat D_-\) and \(\hat W = \hat D_+\) as a function of the system's excitation energy scaled by the system size ($(E-E_{gs})/N_3$) for \(\ell = 0\) eigenstates of the 2DVM Hamiltonian in Eq.~\eqref{H2DVM} with \(N_3 = 300\). All panels:  The control parameter value is \(\xi = 0.6\) and the full red line is the stationary value obtained with Eq.~\eqref{AvgF}. Dashed lines are the result of averaging for different time interval values  (see panel legend). The inset in the left panel is a zoom to the vicinity of the ESQPT critical energy. A vertical dot-dashed pink line marks the ESQPT critical energy in the mean field limit.}
\end{figure*}


Taking Eq.~\eqref{Fjt} into consideration, it is clear that a nonzero average value of \({\cal F}^{(j)}_{VW}(t)\) in the long-time limit implies that one or more instances of the  \(\omega(j,j_1,j_2,j_3)\) quantity should be equal to zero, to avoid oscillations. In order to check this, we have performed calculations for the LMG and 2DVM cases. Following Ref.~\cite{Wang2019b}, in the LMG model we define \(\hat V =\hat W = \hat J_x\). The use of Eq.~\eqref{AvgF} allows for a direct calculation of the long-time average value of the correlator, avoiding the evaluation of computationally expensive oscillatory time integrals. The results obtained  in the stationary \(t\to\infty\) limit  for the even parity states of the LMG Hamiltonian \eqref{HLMGM_boson} with a system size \(N_2 = 300\) and a control parameter \(\xi = 0.6\) are shown in panel (a) of Fig.~\ref{figFT} with a full red line. This result is in agreement with the results obtained in  Ref.~\cite{Wang2019b}. If one computes intermediate average values for finite times, depicted with dashed blue lines in panel (a) of Fig.~\ref{figFT}, one can see how the results tend to the stationary solution as the averaging time increases. This is clearly seen in the inset included in this panel with a zoom to the results in the vicinity of the ESQPT critical energy. As already noticed in \cite{Wang2019b}, only eigenstates whose energy is less than the ESQPT critical energy have a nonzero value of \(\overline{{\cal F}^{(j)}_{VW}}\). This can be explained taking into account that the operator \(\hat J_x\) transforms even parity states into odd parity ones (see App.~\ref{app:atels}) and, from Eq.~\eqref{Fjt} for  \(\hat V =\hat W = \hat J_x\), it is clear that levels \(j\) and \(j_2\) are even parity states while levels \(j_1\) and \(j_3\) are odd parity states. Hence, the \(\omega(j,j_1,j_2,j_3) = 0\) condition is only fulfilled for even and odd parity eigenstates that are degenerate. As shown in panel (c) of Fig.~\ref{lmg_2dvm_ced}, this only occurs for LMG states below the critical energy of the ESQPT, whose value in the mean field limit is marked by a  dot-dashed pink vertical line in panel (a) of Fig.~\ref{figFT}. 


 In the 2DVM case, an equivalent choice for the OTOC operators would be \(\hat V = \hat D_-\) and \(\hat W = \hat D_+\). These two generators enter into the pairing operator and connect states with different values of the vibrational angular momentum,  \(\ell\) (see App.~\ref{app:atels}). If this is the case, assuming we compute the microcanonical OTOC for \(\ell = 0\) states, the fulfilment of the  \(\omega(j,j_1,j_2,j_3) = 0\) condition in Eq.~\eqref{AvgF} implies that the sum of energies of the \(j\)-th and \(j_2\)-th \(\ell = 0\) eigenstates should be equal to the sum of energies for the \(j_1\)-th and \(j_3\)-th \(|\ell| = 1\) eigenstates. From the results in panel (d) Fig.~\ref{lmg_2dvm_ced} it is clear that  2DVM  eigenstates belonging to different Hamiltonian blocks are not degenerate for finite size systems though eigenstates with energies below the ESQPT critical energy can be close in energy. Due to this lack of zero \(\omega(j,j_1,j_2,j_3)\) terms, the stationary value of the four-point correlator, \(\overline{{\cal F}^{(j)}_{VW}}\), is zero for all possible values of \(j\) (see full red line in panel (b) of Fig.~\ref{figFT}).  The smaller energy differences for eigenstates with energies below the ESQPT critical energy imply that it takes longer for such states to reach the zero stationary limit value of the microcanonical OTOC, compared to what happens for states with energies above the ESQPT critical energy. This is clearly seen in panel (b) of Fig.~\ref{figFT}, where for shorter times (e.g.~the dashed lighter green shade  lines) eigenstates above and below the critical energy can be distinguished. As in the previous case, the critical ESQPT energy is marked by a dot-dashed pink vertical line. As the time over which the average value is computed is larger,  \(\overline{{\cal F}^{(j)}_{D_-D_+}}\to 0\) for all eigenstates. However, the small energy gap between levels with different angular momentum below the critical energy of the ESQPT indicates that \(\overline{{\cal F}^{(j)}_{D_-D_+}}\) could be considered as an approximate order parameter for the ESQPT, computing the average time value of \({\cal F}^{(j)}_{VW}(t)\) for finite time values and taking profit of the much longer time needed for states below the critical energy  to reach the OTOC stationary value.

To check our previous results we have performed a second set of calculations, shown in Fig.~\ref{figFN}, where we display the \(\overline{{\cal F}^{(j)}_{VW}}(T)\) OTOC for the two models under study, fixing the averaging time to a constant value, \(T= 1000\), and including results for different system sizes. As in the previous figure, the results for the LMG model are shown in panel (a) with different shades of blue and the results for the 2DVM model are shown in panel (b), with different shades of green. Again, as in Fig.~\ref{figFT}, we have marked the mean field limit value of the critical energy with a vertical dot-dashed pink line. In the LMG case, there is no dependence on the system size for eigenstates with energies below the ESQPT critical energy, something that can be explained because such states are degenerate up to the numerical precision used in this calculation for all system sizes. In the panel (a) inset, with a zoom to the critical energy region, it can be clearly appreciated how, as expected, for larger system sizes, states go to zero closer to the expected mean field critical value marked with a vertical dot-dashed pink line. The  \(\overline{{\cal F}^{(j)}_{VW}}(T)\) OTOC value, for eigenstates with energies larger than the ESQPT critical enery are closer to the expected stationary limit zero value for lower system sizes. This can be easily explained because we are performing the averaging of the four-point OTOC over a finite \(T = 1000\) value and this time is large enough to cancel the oscillatory integral for the smaller system sizes, but not for the larger ones. The results for the 2DVM case are shown in panel (b) of Fig.~\ref{figFN}. In this case the smaller the system size, the closest to the zero stationary result, for all eigenstates. The difference with the stationary value increases for increasing system sizes, in particular for energies smaller than the critical ESQPT energy. As in the LMG case, as the system size grows, the microcanonical OTOC at energies less than the critical energy goes to zero closer to the mean field critical energy. The differences between the stationary limit are larger for states below the critical energy than for states with energies above the critical energy, due to the decreasing energy gap between states with different angular momentum values in this zone that is evinced in Fig.~\ref{lmg_2dvm_ced}. As shown in App.~\ref{appa}, the results obtained for \(\overline{{\cal F}^{(j)}_{VW}}\) in the VM and IBM cases are akin to the ones shown in  panel (b) of Fig.~\ref{figFT}) for the 2DVM.

\begin{figure*}
\includegraphics[width=0.9\textwidth]{figure_paper_ESQPT_bar_F_N_LMG_2DVM.pdf}
\caption{\label{figFN} (Color online) Panel (a): Time-averaged value of  \({\cal F}^{(j)}_{VW}(t)\) for the LMG model with \(\hat V =\hat W = \hat J_x\) as a function of the system's excitation energy scaled by the system size ($(E-E_{gs})/N_2$) for even parity eigenstates of Eq.~\eqref{HLMGM} Hamiltonian for various system size values.  Panel (b): Time-averaged value of  \({\cal F}^{(j)}_{VW}(t)\) with  \(\hat V = \hat D_-\) and \(\hat W = \hat D_+\) as a function of the system's excitation energy scaled by the system size ($(E-E_{gs})/N_3$) for \(\ell = 0\) eigenstates of the 2DVM Hamiltonian in Eq.~\eqref{H2DVM} for various system size values, \(N_3\).  
All panels: All calculations were carried out for a control parameter value \(\xi = 0.6\)  and the time average is performed over a time interval $T = 1000$. Dashed lines are the result of averaging for different system size values (see panel legend).  A vertical  dot-dashed pink line marks the ESQPT critical energy in the mean field limit.}
\end{figure*} 

The main conclusion of the present work is that the degeneracy for eigenstates belonging to different symmetries below the critical energy of the ESQPT is different in systems defined in two-level bosonic models in one dimension compared to models defined for higher dimensions. In fact, in one dimension the states are degenerate even for finite size systems and the energy splitting decreases exponentially with the system size, while in dimensions above one the degeneracy decreases following a potential law and states are only truly degenerate in the thermodynamic limit. We have checked the consequences of this different degeneracy in the system dynamics, computing an out-of-time-order correlator as a possible order parameter for the ESQPT.  In the LMG system (and similar 1D systems) \(\overline{{\cal F}^{(j)}_{VW}}\), the four-point OTOC component, can be used as an ESQPT order parameter, even for finite size systems. This is not true for the ESQPT higher dimension models. For such models the degeneracy is not complete for finite size systems. However, in this last case, the microcanonical OTOC is an order parameter in the mean field limit. We have also checked that, with an appropriate combination of the system size and the averaging time, one could also define with an OTOC an approximate order parameter for ESQPTs in models with dimension larger than one.    

% \textcolor{blue}{
% \begin{itemize} 
% \item Mention in the appendix that the results for the VM and the IBM will be similar to those obtained for the 2DVM. Include the Figures for the correlation energy diagram and the energy differences in these two cases in the App.? (\textbf{Jamil}) 
%\item Is it possible to obtain the stationary value \(\overline{{\cal F}^{(j)}_{VW}}\) in the thermodynamic limit? %It seems so, making use of the QDS concept and assuming that states under(above) the ESQPT can be approximated to %$u(n-1)$ ($so(n)$) basis states. Check this. (\textbf{JEnrique})
%\\
%To obtain the stationary value of $\overline{F_{VM}^{(j)}}$, please note the following identity
%\begin{align}
%     \overline{e^{i(E_j-E_{j_1}+E_{j_2}-E_{j_3})t}}=\delta_{jj_1}\delta_{j_2j_3}+\delta_{jj_3}\delta_{j_1j_2}-\delt%a_{jj_1j_2j_3}
%\end{align}
% \end{itemize}
% }


%% \begin{figure*}
%% \includegraphics[width=\textwidth]{LMG_N50.pdf}
%% \caption{\label{fig:lmg} (Color online) Left panel: Excitation energy scaled by the system size ($\varepsilon = (E-E_0)/2S$) as a function of the control parameter $\xi$ for the LMGM in Eq.~\eqref{HLMGM}. Full blue (dashed red) lines mark even (odd) parity levels. The color-filled area mark the energy difference  between selected even and odd parity levels (see right panel caption). Right panel: }
%% \end{figure*}

%% \begin{figure*}
%% \includegraphics[width=\textwidth]{2DVM_N50.pdf}
%% \caption{\label{fig:2dvm} (Color online) Left panel: Excitation energy scaled by the system size ($\varepsilon = (E-E_0)/2S$) as a function of the control parameter $\xi$ for the LMGM in Eq.~\eqref{lmgm}. Full blue (dashed red) lines mark even (odd) parity levels. The color-filled area mark the energy difference  between selected even and odd parity levels (see right panel caption). Right panel: }
%% \end{figure*}

%% \begin{figure}
%% \includegraphics[width=0.5\textwidth]{U4VM_N50.pdf}
%% \caption{\label{fig:vibron} (Color online) Left panel: Excitation energy scaled by the system size ($\varepsilon = (E-E_0)/N_4$) as a function of the control parameter $\xi$ for the VM Hamiltonian in Eq.~\eqref{HVM}. Full blue (dashed red) lines mark angular momentum \(J = 0 (1)\) levels. The color-filled area mark the energy difference  between selected levels (see right panel legend). Right panel: }
%% \end{figure}

%% \begin{figure}
%% \includegraphics[width=0.5\textwidth]{IBM_N50.pdf}
%% \caption{\label{fig:ibm} (Color online) Left panel: Excitation energy scaled by the system size ($\varepsilon = (E-E_0)/N_6$) as a function of the control parameter $\xi$ for the IBM Hamiltonian in Eq.~\eqref{HIBM}. Full blue (dashed red) lines mark seniority \(\tau = 0(1)\) levels. The color-filled area mark the energy difference  between selected levels with different seniority (see right panel caption). Right panel: }
%% \end{figure}


\begin{acknowledgments}
We wish to acknowledge useful discussions with José Miguel Arias, Pedro
  Pérez Fernández, Jorge Dukelsky, Armando Relaño, Ángel López Corps, Jorge Hirsch, and Saúl Pilatowski-Cameo.

  This project has received funding from the European Union's Horizon
  2020 research and innovation program under the Marie
  Sk{\l}odowska-Curie grant agreement No 872081 and from grant
  PID2019-104002GB-C21 funded by MCIN/AEI/ 10.13039/501100011033 and,
  as appropriate, by “ERDF A way of making Europe”, by the “European
  Union” or by the “European Union NextGenerationEU/PRTR”.  This work
  has also been partially supported by the Consejer\'{\i}a de
  Conocimiento, Investigaci\'on y Universidad, Junta de Andaluc\'{\i}a
  and European Regional Development Fund (ERDF), UHU-1262561 (JKR and
  FPB) and US-1380840 (JKR), and PY2000764, by the Ministerio de Ciencia, Innovaci\'on y
  Universidades (ref.COOPB20364) (MC). JKR also thanks to Margarita Salas Fellowship. Computing resources supporting
  this work were provided by the CEAFMC and Universidad de Huelva High
  Performance Computer (HPC@UHU) located in the Campus Universitario
  el Carmen and funded by FEDER/MINECO project UNHU-15CE-2848.  
\end{acknowledgments}

\appendix

\section{Extension to models in \(3\) and \(5\) dimensions}
\label{appa}
  The vibron model (VM) was introduced in the 80s by Iachello \cite{Iachello:81}, extending the algebraic approach to the study of molecular structure \cite{frank,bookmol}. In particular, ro-vibrational excitations for a diatomic molecule are treated as bosonic collective excitations \cite{Iachello:81, bookmol}. Due to the three-dimensional nature of the problem associated with the dipole degree of freedom in diatomic molecules, this case has \(n=3\) and a \(u(4)\) Lie algebra as its dynamical algebra.  The bosonic operators requested to build the sixteen generators of the dynamical algebra  are a scalar boson operator \(s^\dagger (s)\) and an angular momentum one boson \(p_\mu^\dagger (p_\mu)\) with \(\mu = \pm 1, 0\). As in the previous cases, the \(u(4)\) generators are built as bilinear products of creation and annihilation operators \cite{frank, Iachello1996,bookmol, Iachello:81}.  As in the two previous cases, the VM has two dynamical symmetries %\footnote{There is another $so(4)$ dynamical symmetry, denoted as \(\overline{so(4)}\), which doesn't introduce new phenomenolosymmetriesconverginggy.} 
   converging in $so(3)$, the system's symmetry algebra, associated with the conservation of the angular momentum
\begin{align}
&u(4)\supset u(3) \supset so(3)&\mbox{Chain (VM-I)}~,\label{vmi}\\
&u(4)\supset so(4)\supset so(3) &\mbox{Chain (VM-II)}~.\label{vmii}
\end{align}
\noindent In this case, the total number of \(s\) and \(p\) bosons is denoted as \(N_4\) and it denotes the totally symmetric \(u(4)\) irrep that spans the system's Hilbert space. The dynamical symmetry in Eq.~\eqref{vmi} is a convenient approximation to model vibration of floppy, weakly-bent molecules and the Eq.~\eqref{vmii} dynamical symmetry provides a Morse-like spectrum and it has been applied to many molecular species \cite{bookmol, Iachello:81}. A model Hamiltonian defined in the same way as in the previous two cases can be built using the first order Casimir operator of the \(u(3)\) subalgebra in Eq.~\eqref{vmi} and the \(so(4)\) pairing operator, built with the second order Casimir operator of \(so(4)\) in Eq.~\eqref{vmii}

\begin{equation}
  \hat H_{VM} = (1-\xi) \hat n_p + \frac{\xi}{N_{4}}  \hat P_p~,
  \label{HVM}
\end{equation}
\noindent with a control parameter \(\xi\in[0,1]\), \(\hat n_\tau = \sum_\mu p^\dagger_\mu p_\mu \), and \(\hat P_p =  N_4^2 - \hat D^2 - \hat J^2\) where \(\hat D^2 + \hat J^2\) is the second order Casimir operator of the \(so(4)\) subalgebra \cite{frank, Iachello1996,bookmol, bookalg}. In this case the symmetry algebra is \(so(3)\), due to conservation of angular momentum,  and Hamiltonian \eqref{H2DVM} is split into different blocks, one for each angular momentum value, \(J\) considered \cite{bookmol}.

The model Hamiltonian \eqref{HVM} has a second-order ground-state QPT between the \(u(3)\) and \(so(4)\) dynamical symmetries, and an associated ESQPT.  The correlation energy diagram is displayed in the panel (a) of Fig.~\ref{VM-IBM_spectra} for levels with angular momentum $J=0$ (full blue lines) and $J=1$ (dashed red lines). The correlation energy diagram is akin than the correlation energy diagrams for the LMG and the 2DVM shown in Fig.~\ref{lmg_2dvm_ced}. Once again, the difference between levels with different angular momentum is hightlighted. In the panel (c) of Fig.~\ref{VM-IBM_spectra} these differences are plotted using semi-log scale. As we observed in the 2DVM, the order of the degeneracy doesn't reach the numerical precision until $\xi=1$.


\begin{figure*}
  \includegraphics[width=0.75\textwidth, angle=-90]{VM_IBM_N50.pdf}
\caption{\label{VM-IBM_spectra} (Color online) Panel (a): Excitation energy scaled by the system size ($\varepsilon = (E-E_0)/N_4$) as a function of the control parameter $\xi$ for the VM with model Hamiltonian Eq.~\eqref{HVM} and system size \(N_4 = 50\). Full blue (dashed red) lines mark levels with angular momumtum $J=0 (1)$.  Panel (b): Excitation energy scaled by the system size ($\varepsilon = (E-E_0)/N_6$) as a function of the control parameter $\xi$ for the IBM model Hamiltonian Eq.~\eqref{HIBM} with a system size \(N_6 = 50\). Full blue (dashed red) lines mark levels with seniority \(\nu_s=\nu_d=0 (\nu_s=\nu_d=1)\). In both panels the color-filled area marks the energy difference  between selected states with different angular momentum (panel (a)) or seniority (panel (b)). Panel (c): Energy difference between selected pairs of states of the VM Hamiltonian Eq.~\eqref{HIBM} having different angular momentum as a function of the control parameter $\xi$. Panel (d): Energy difference between selected states of Hamiltonian Eq.~\eqref{HIBM} with different seniority as a function of the control parameter $\xi$. In both cases, the labels of the selected pairs of levels are provided in the legend of the panels and the color used for each pair of states is the same color used to fill the corresponding area in the upper panel.}
\end{figure*}


The IBM has a very successful history in the study of nuclear structure using algebraic methods. This  model was introduced in the 70's of the XX-th century by Arima and Iachello \cite{Arima1975, Arima1976, Arima1978, Arima1979}. Since then, it has become a standard tool in the study of nuclear structure from a collective point of view \cite{booknuc}, and it has been enriched in many different ways, including an odd fermion, IBFM \cite{ibfm_book}, or considering explicitly the difference between protons and neutrons, IBM-2 \cite{Arima1977}, so the original model is sometimes known as IBM-1. In this case, \(n = 5\) due to the  five dimensions that characterize the nuclear collective problem   and the model dynamical algebra is \(u(6)\)  \cite{booknuc}. The nuclear excitations are treated as bosons that can be traced back to coupled nucleons with angular momentum zero or two. Therefore the model building blocks are a scalar boson, \(s\) and a boson with angular momentum two, \(d_\mu\), with \(\mu = \pm 2, \pm 1, 0\) and the 36 generators of \(u(6)\) are expressed as the bilinear products of creation and annihilation boson operators \cite{booknuc,frank,bookalg}. As in the previous case, the angular momentum is conserved in this case and, therefore, the system's symmetry algebra is \(so(3)\). In this case there are at least three relevant dynamical symmetries, but we concentrate in this work on the two subalgebra chains that are equivalent to the cases previously mentioned
\begin{align}
&u(6)\supset u(5) \supset so(5)\supset so(3)&\mbox{Chain (IBM-I)}~,\label{ibi}\\
&u(6)\supset so(6)\supset so(5)\supset so(3) &\mbox{Chain (IBM-II)}~.\label{ibii}
\end{align}
\noindent In this case, the total number of \(s\) and \(d\) bosons is denoted as \(N_6\) and it defines the totally symmetric \(u(6)\) irrep that spans the system's Hilbert space. The dynamical symmetry in Eq.~\eqref{ibi} is a convenient approximation to model nuclear structure in spherically symmetric (vibrational) nuclides, while the dynamical symmetry Eq.~\eqref{vmii} provides a way to model the so called gamma-unstable nuclei \cite{frank, booknuc}. Hence, we define a model Hamiltonian in the same way as in the previous cases.  The first term, \(\hat n_d\), is the number operator of \(d\) bosons, which is the first order Casimir operator of \(u(5)\)  in Eq.~\eqref{ibi}. The second term is the Casimir operator of the $so(6)$ subalgebra in Eq.~\eqref{vmii}, \(\hat P_d = 2\left(N_6(N_6+4) - \left[ d^\dagger\cdot d^\dagger- s^\dagger s^\dagger \right]\left[\tilde d\cdot\tilde d-\tilde s\tilde s\right]\right)\),
\begin{equation}
  \hat H_{IBM} = (1-\xi) \hat n_d + \frac{\xi}{N_{6}}  \hat P_d~.
  \label{HIBM}
\end{equation}
As in the three previous cases, the Hamiltonian has one control parameter, \(\xi\in[0,1]\) \cite{frank, booknuc, bookalg}. If we do not consider other possible dynamical symmetries, the symmetry algebra is \(so(5)\) instead of \(so(3)\), and the conserved quantity is seniority ($\nu_s$ and $\nu_d$), which is related with the label \(\tau\) of the irrep. This implies that Hamiltonian Eq.~\eqref{HIBM} can be split into seniority blocks \cite{frank, booknuc, bookalg}. As in the previous cases, the model Hamiltonian \eqref{HIBM} has a second-order ground-state QPT between the \(u(5)\) and \(so(6)\) dynamical symmetries, and the corresponding ESQPT.  The correlation energy diagram is plotted in panel (b) of Fig.~\ref{VM-IBM_spectra}, for \(\tau = 0\) and \(1\)  levels. On more time, a very similar spectra is obtained, where the eigenvalues seem to be degenerate in the broken-symmetry phase. However, the highlighted differences in the panel (b) are plotted using log scale in panel (d) of Fig.~\ref{VM-IBM_spectra} versus the control parameter. As it was expected, the degeneracy  is not achieved until the system is in the dynamical symmetry $so(6)$.


\section{Matrix element of the relevant operators of the $u(2)$ and $u(3)$ models}
\label{app:atels}
In this section we provide of the matrix elements that are necessaries to develop the calculations presented in the present manuscript.
\subsection{Matrix elements of the LMG model}
The chain I of the LMG model, introduced in Eq.~\eqref{lmgmi}, provides of one label $n_t$ to name the states of the basis $\left\{\ket{[N_2]n_t},\;n_t=0,1...,N_2\right\}$. The elements of this basis conserve the parity, $\hat{\Pi}\ket{[N]n_t}=(-1)^{n_t\%2}\ket{[N]n_t}$, where the symbol $\%$ denotes the modulo-$2$ operation. The expected value of the operator $\hat{J}_x$ in this basis is
\begin{align}\label{Jxmatel}
\braket{[N]n_t'}{\hat{J}_x}{[N]n_t}=&\frac{1}{2}\sqrt{(N-n_t)(n_t+1)}\delta_{n_t',n_t+1} \nonumber \\
+&\frac{1}{2}\sqrt{(N-n_t-1)n_t}\delta_{n_t',n_t-1} ~.
\end{align}
 From Eq.~\eqref{Jxmatel} it is trivial to realize that operator $\hat{J}_x$ mixes elements with different parity, however $\hat{J}_x^2$ connects the state $\ket{[N]n_t}$ with itself and with $\ket{[N]n_t\pm 2}$, both with the same parity as it.

\subsection{Matrix element of the 2DVM}
The relevant operators in this model are $\hat{n}$, $\hat{D}_{\pm}$, $\hat{\ell}$, and $\hat{W}^2$, the latter can be expressed as $\hat{W}^2=\frac{1}{2}\left(\hat{D}_+\hat{D}_-+\hat{D}_-\hat{D}_+\right)+\hat{\ell}^2$. Let's choose the basis provided by the chain I, Eq.~\eqref{2dvmi}. The element of this basis can be labeled using the vibrational quantum number $n$ and the vibrational angular momentum $\ell$ as $\left\{\ket{[N]n,\ell}\equiv \ket{n^{\ell}}\right\}$ for $n=N, N-1, N-2, ..., 0$ and  $\ell = \pm n, \pm(n-2), ..., \pm (n\%2)$. The matrix elements of the operators $\hat{n}$ and $\hat{\ell}$ are trivial, and the matrix element of $\hat{D}_{\pm}$ are
\begin{align}\label{Dpmmatel}
    \braket{n'^{\ell'}}{\hat{D}_\pm}{n^{\ell}}=\pm\sqrt{(N-n)(n\pm\ell+2)} \delta_{\ell',\ell\pm1}\delta_{n',n+1}~.
\end{align}
From Eq.~\eqref{Dpmmatel} it is  understood how $\hat{D}_{\pm}$ connect a state with angular momentum $\ell$ to another with $\ell'=\ell\pm 1$. The matrix element of $\hat{W}^2$ can be easily derived from Eq.~\eqref{Dpmmatel}. 

%% To start the appendixes, use the \verb+\appendix+ command.
%% This signals that all following section commands refer to appendixes
%% instead of regular sections. Therefore, the \verb+\appendix+ command
%% should be used only once---to setup the section commands to act as
%% appendixes. Thereafter normal section commands are used. The heading
%% for a section can be left empty. For example,
%% \begin{verbatim}
%% \appendix
%% \section{}
%% \end{verbatim}
%% will produce an appendix heading that says ``APPENDIX A'' and
%% \begin{verbatim}
%% \appendix
%% \section{Background}
%% \end{verbatim}
%% will produce an appendix heading that says ``APPENDIX A: BACKGROUND''
%% (note that the colon is set automatically).

%% If there is only one appendix, then the letter ``A'' should not
%% appear. This is suppressed by using the star version of the appendix
%% command (\verb+\appendix*+ in the place of \verb+\appendix+).

%% \section{A little more on appendixes}

%% Observe that this appendix was started by using
%% \begin{verbatim}
%% \section{A little more on appendixes}
%% \end{verbatim}

%% Note the equation number in an appendix:
%% \begin{equation}
%% E=mc^2.
%% \end{equation}

%% \subsection{\label{app:subsec}A subsection in an appendix}

%% You can use a subsection or subsubsection in an appendix. Note the
%% numbering: we are now in Appendix~\ref{app:subsec}.

%% Note the equation numbers in this appendix, produced with the
%% subequations environment:
%% \begin{subequations}
%% \begin{eqnarray}
%% E&=&mc, \label{appa}
%% \\
%% E&=&mc^2, \label{appb}
%% \\
%% E&\agt& mc^3. \label{appc}
%% \end{eqnarray}
%% \end{subequations}
%% They turn out to be Eqs.~(\ref{appa}), (\ref{appb}), and (\ref{appc}).

% The \nocite command causes all entries in a bibliography to be printed out
% whether or not they are actually referenced in the text. This is appropriate
% for the sample file to show the different styles of references, but authors
% most likely will not want to use it.
%% \nocite{*}

\bibliography{letter_OTOC}% Produces the bibliography via BibTeX.

\end{document}
%
% ****** End of file apssamp.tex ******
