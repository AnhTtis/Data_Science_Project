\section{Stopped subalgebras} 
\label{sec:stopped}
In this section we prove our main result, which provides simplified models of
the singular Chekanov--Eliashberg dg-algebra in $\R^{3}$.
\subsection{Bordered Legendrians}
We will be working with bordered Legendrians, of which we here give an
overview. A more detailed account of this topic can
be found in \cite{ABS22}.   

\begin{definition}
A \emph{bordered Legendrian} in $\R^{3}$ is a subset
$\Gamma \subset  \R^{3}$ of the form $\Gamma = \Lambda \cap \{ x \in \R^{3} : | x | \leq M\}$
where $\Lambda$ is a proper embedding of a singular Legendrian skeleton of a Weinstein hypersector 
$V \subset \R^{3}$ such that for $| x | \geq M$, each point of $\Lambda$ has 
$y$-coordinate equal to $0$ (possibly with finitely many two-valent singularities). If $\Lambda$ does not have any singularities
inside $\{  x \in \R^{3} : | x | \leq M\}$ we call $\Gamma$ \emph{smooth} and
otherwise we say that
$\Gamma$ is \emph{singular}. If $\{ x \in \R^{3} : x=-M\} \cap \Gamma = \emptyset$ or 
$\{ x \in \R^{3} : x=M\} \cap \Gamma = \emptyset$ we say that $\Gamma$ has no
\emph{left ends} or \emph{right ends}, respectively.
\end{definition}

\begin{definition}
Let $\Gamma$ be a bordered Legendrian with no left ends. Let $\Lambda'$
be the singular Legendrian obtained by closing up the left ends of $\Gamma$
into a singularity consisting of a single $0$-cell, as shown to the left in
\autoref{fig:closing}. Let $V'$ be the
thickening of $\Lambda'$ to a Weinstein surface with skeleton $\Lambda'$.
The \emph{bordered Chekanov--Eliashberg dg-algebra} of $\Gamma$,
denoted by $CE^{*}( \Gamma;V_{0};\R^{4} )$, is the unital dg-subalgebra of 
the Chekanov--Eliashberg dg-algebra $CE^{*}( \Lambda';V_{0}';\R^{4} )$
generated by all chords which through the Ng resolution correspond to crossings
in the front of $\Gamma$, and the chords of $\partial \Lambda'$ which belong to
singularities of $\Gamma$. 
\begin{figure}[!htb]
    \centering
    \incfig{closing}
    \caption{The closure $\Lambda'$ of $\Lambda$, illustrated in the front
    projection to the left and the Lagrangian projection to the right.}
    \label{fig:closing}
\end{figure}
\end{definition}
\begin{lemma}
	The algebra $CE^{*}( \Gamma;V_{0};\R^{4} )$ is well-defined in the sense
	that it is closed under the differential.
\end{lemma}
\begin{proof}
	Using the Ng resolution, one sees that the chords of $\Lambda'$
	which are excluded from $CE^{*}( \Gamma;V_{0};\R^{4} )$ are all of
	higher action than the ones included, and can therefore not occur in
	the differential of any chord in $CE^{*}( \Gamma;V_{0};\R^{4} )$. The
	chords of $\partial \Lambda'$ at the singularity in
	\autoref{fig:closing} can not occur in the differential as no
	disks contributing to the differential of $CE^{*}( \Gamma;V_{0};\R^{4}
	)$ can cross the twist to the right in \autoref{fig:closing}.
\end{proof}

This algebra was first introduced in \cite{Siv11} for smooth bordered Legendrians and
generalized to singular bordered Legendrians in \cite{ABS22}. 
In \cite{ABS22}, it is shown that
$CE^{*}(\Gamma;V_{0};\R^{4} )$ is invariant up to quasi-isomorphism
under Legendrian isotopy of $\Gamma$, by which we mean an isotopy of $\Lambda$ which is
constant outside $\{  x \in \R^{3} : | x | < M\}$. 

\begin{theorem}[{\cite[Theorem 3.3.15]{ABS22}}]
	The dg-algebra $CE^{*}( \Gamma;V_{0};\R^{4})$ is up to quasi-isomorphism
	invariant under Legendrian isotopy of $\Gamma$ supported in some fixed
	compact subset.
\end{theorem}
\begin{proof}
	See {\cite[Theorem 3.3.15]{ABS22}}.
\end{proof}

\begin{definition}
Let $\Gamma$ be a bordered Legendrian. The \emph{reflection} of
$\Gamma$ is the bordered Legendrian ${}^{*}\Gamma$ whose front is the
reflection of the front of $\Gamma$ over the horizontal axis.
\end{definition}

\begin{lemma}
\label{res:push-twist}
	Let $\Gamma$ be a bordered Legendrian. Let $\Lambda$
	be a Legendrian whose front contains $\Gamma$ as a
	tangle, along with a positive or negative half-twist 
	as illustrated to the left in \autoref{fig:half-twists}. 
	Then the operations 
	in \autoref{fig:half-twists} preserve the Legendrian isotopy type of
	$\Lambda$.
\begin{figure}[!htb]
    \centering
    \incfig{half-twists}
    \caption{Pushing $\Gamma$ through the cusp of a half-twist.}
    \label{fig:half-twists}
\end{figure}
\end{lemma}
\begin{remark}
	Note that $\Gamma$ need not have the same number of left ends as right
	ends, or may lack left or right ends altogether. 
\end{remark}
\begin{proof}	
	We perform Reidemeister moves and push the left cusps, right cusps,
	crossings, and singularities of $\Gamma$ through the half-twist, as shown in
	\autoref{fig:elementary-permutation}. Doing this turns left cusps into
	right cusps and vice versa, flips the singularities, and reverses the
	order, so we get the bordered Legendrian ${}^{*}\Gamma$ on the other
	side. See the proof of \cite[Lemma 5.1.4]{Ng01} for a similar
	construction.
\begin{figure}[!htb]
    \centering
    \incfig{elementary-permutation}
    \caption{Pushing cusps, crossings, and singularities through a half-twist.}
    \label{fig:elementary-permutation}
\end{figure}
\end{proof}

\subsection{Stopped subalgebras}
Let $\Lambda$ be the singular Legendrian skeleton of some Weinstein
hypersurface $V \subset \R^{3}$. 

\begin{definition}
If there are two values $x_0 < x_1$ such that all singularities of $\Lambda$ have
$x$-coordinates equal to either $x_0$ or $x_1$ and all other points of $\Lambda$ have
$x$-coordinates in the interval $( x_0,x_1 )$ we say that $\Lambda$ has 
\emph{all singularities to the sides}. We call the singularities with
$x$-coordinate equal to $x_0$ \emph{left singularities} and those with
$x$-coordinate equal to $x_{1}$ \emph{right singularities}.
\end{definition}

Every singular Legendrian in $\R^{3}$ is isotopic to a Legendrian with all
singularities to the sides and we now assume that $\Lambda$ is in such a
position. We construct a bordered Legendrian $\Lambda
\cup \Omega$ by attaching one strand $\Omega_{i}$ to each left singularity of
$\Lambda$, numbering from bottom to top, as shown in
\autoref{fig:exact-stops}. 
\begin{figure}[!htb]
    \centering
    \incfig{exact-stops}
    \caption{Adding strands $\Omega = \Omega_{1} \cup
	    \Omega_{2} \cup \ldots \cup \Omega_{k}$ to $\Lambda$. There are no
    Reeb chords of $\Lambda \cup \Omega$ with endpoints on $\Omega$.}
    \label{fig:exact-stops}
\end{figure}
Note that $\partial \Lambda$ consists of two-point
spheres and $\partial \Omega$ of one-point stops.
The bordered Legendrian $\Lambda \cup \Omega$ has the property that there are
no Reeb chords
of $\Lambda \cup \Omega$ with an endpoint on $\Omega$ in the algebra 
$CE^{*}( \Lambda \cup \Omega;V_{0};\R^{4} )$. Note however that there are Reeb
chords of $\partial \Lambda \cup \partial \Omega$ with endpoints on $\partial
\Omega$.

\begin{definition}
The \emph{stopped subalgebra} of $CE^{*}( \Lambda \cup \Omega;V_{0};\R^{4} )$, denoted by 
$CE^{*}(\Lambda;V_{0,\partial \Omega};\R^{4} )$, is the unital dg-subalgebra generated by all
chords of $\Lambda$, and the chords of $\partial \Lambda$ which do not pass
though $\partial \Omega$.
\end{definition}
\begin{lemma}
	The stopped subalgebra is a subcomplex, and hence itself a unital
	dg-algebra.
\end{lemma}
\begin{proof}
	By \autoref{res:zero-dim-diff}, the differential of a chord of
	$\partial \Lambda \cup \partial \Omega$ which does not pass through
	$\partial \Omega$, nor has any endpoint on $\partial \Omega$, 
	will not contain any chords which
	pass through or have endpoints on $\partial \Omega$. For
	a chord $c$ of $\Lambda$, it is clear from the Ng resolution that 
	$\partial c$ does not contain any chords of $\partial \Lambda \cup
	\partial \Omega$ which pass through or have endpoints on $\partial
	\Omega$
\end{proof}
If the base points of $\partial V_{0}$ are placed near $\partial \Omega$ then 
the chords of $\partial \Lambda$ which do not pass though $\partial \Omega$
are precisely the chords of the form $c_{ij}^{0}$. In particular, $CE^{*}(
\Lambda \cup \Pi;V_{0};W )$ is finitely generated. Note that $CE^{*}(
\Lambda;V_{0,\partial \Omega};W )$ also embeds as subalgebra of $CE^{*}(
\Lambda;V_{0};W )$.

The motivation for the notation comes from considering $CE^{*}(
\Lambda;V_{0,\partial \Omega};W )$ as being a
version of the relative algebra of \autoref{ssec:ceforsing}, 
but where $\Sigma( \Sigma )$ is a
non-compact stop at which we attach a 'non-compact half-handle', as shown in 
\autoref{fig:singular-definition-stop}. 
\begin{figure}[!htb]
    \centering
    \incfig{singular-definition-stop}
    \caption{The cobordism $W_{V,\Sigma(\Omega)}^{0}$, where $L_{\Sigma( \Omega )}$ 
    is the core of the half-handle attached at $\Sigma(\Omega)$.}
    \label{fig:singular-definition-stop}
\end{figure}
We do not give the full details of the
geometric construction, but note that attaching the non-compact half-handle
corresponds to putting a stop diffeomorphic to a Legendrian arc in the boundary
of the Weinstein cobordism $W^0_V$.

\begin{lemma}
\label{res:stopped-subalgebra-prop}
	The canonical inclusion	
	\[
		CE^{*}( \Lambda;V_{0,\partial \Omega};\R^{4} ) 
		\hookrightarrow
		CE^{*}( \Lambda \cup \Omega;V_{0};\R^{4} )
	\]
	is a quasi-isomorphism onto 
	$CE^{*}( \Lambda \cup \Omega;V_{0};\R^{4} )[\Lambda,\Lambda]$.	
\end{lemma}
\begin{remark}
	Note that this lemma says 
	something different than \autoref{res:sing-cobordism-map}, the reason being 
	that $V_{0,\partial\Omega}$ is a Weinstein hypersector and not a hypersurface.
\end{remark}
\begin{proof}
	We will use an argument similar to that in the proof of
	\autoref{res:surgery-map-surface}.
	We filter $CE^{*}( \Lambda \cup \Omega;V_{0};\R^{4} )$ by Reeb
	chord action and let $\widetilde{\partial}$ be the action preserving
	component of the differential $\partial$. Since $\partial$ is strictly
	action decreasing on
	chords of $\Lambda$, $\widetilde{\partial}$ will act as
	$\partial_{0}$ from \autoref{res:zero-dim-diff} on chords of $\partial
	\Lambda \cup \partial \Omega$ and vanish on chords of $\Lambda \cup
	\Omega$. Let 
\[
	\tilde{\iota}:(CE^{*}( \Lambda;V_{0,\partial 
	\Omega};\R^{4} ),\widetilde{\partial}) \to 
	(CE^{*}( \Lambda \cup \Omega;V_{0};\R^{4}),\widetilde{\partial}) 
\] 
	be the inclusion considered as going between the same algebras as
	$\iota$ but with $\widetilde{\partial}$ as differential.
	It follows from \autoref{res:short-chords-generate} that the inclusion
	\[
		(CE^{*}( \partial \Lambda;V_{0,\partial \Omega}
		),\partial_{0} )
		\hookrightarrow
		(CE^{*}( \partial \Lambda \cup \partial \Omega;V_{0}
		),\partial_{0})
	\]
	is a quasi-isomorphism onto 
	$(CE^{*}( \partial \Lambda \cup \partial \Omega;V_{0}
	),\partial_{0})[\partial \Lambda,\partial \Lambda]$. 
	Since $\tilde{\iota}$ simply extends this inclusion by the identity on
	the remaining generators, it follows that
	the whole inclusion $\tilde{\iota}$ is a
	quasi-isomorphism onto $(CE^{*}( \Lambda \cup \Omega;V_{0};\R^{4}
	)[\Lambda,\Lambda],\widetilde{\partial})$. If we
	consider the mapping cone of $\iota$ we see that the
	first page of the spectral sequence arising from the action filtration
	of the cone is isomorphic to the mapping cone of
	$\tilde{\iota}$. Consequently, the sequence vanishes on the second
	page and $\iota$ is a quasi-isomorphism
	onto $CE^{*}( \Lambda \cup \Omega;V_{0};\R^{4}
	)[\Lambda,\Lambda]$. 
\end{proof}
\begin{lemma}
\label{res:stopped-lemma}
	There is a quasi-isomorphism
	\[
		CE^{*}( \Lambda \cup \Omega;V_{0};\R^{4} ) \to
		CE^{*}( \Lambda;V_{0};\R^{4} ),
	\] 
	and moreover, the restriction of this 
	quasi-isomorphism to $CE^{*}( \Lambda \cup \Omega;V_{0};\R^{4}
	)[\Lambda,\Lambda]$ is also a quasi-isomorphism.
\end{lemma}
\begin{proof}
	First, we order the strands of $\Omega$ as
	$\Omega_{1},\ldots,\Omega_{k}$ going from top to bottom. 
	By applying \autoref{res:push-twist} twice we can perform an isotopy of 
	$\Lambda \cup \Omega$ to get a bordered Legendrian of the form shown
	in \autoref{fig:exact-stops-isotopy}. 
\begin{figure}[!htb]
    \centering
    \incfig{exact-stops-isotopy}
    \caption{The bordered Legendrian $\Lambda \cup \Omega$ after a Legendrian
	    isotopy which introduces Reeb chords from $\Omega$ to itself.}
    \label{fig:exact-stops-isotopy}
\end{figure}
	This isotopy introduces one Reeb chord
	generator $a_{ij}$ from $\Omega_{i}$ to $\Omega_{j}$ 
	for each pair $1 \leq i,j \leq k$. The differential
	acts on these by $\partial a_{ij} = \pm \delta_{ij}e_{i} + 
	\sum_{m > i} \pm a_{mj}a_{im}$, where $e_{i}$ is the idempotent
	corresponding to $\Omega_{i}$. In particular, we have 
	$\partial a_{kk} = \pm e_{k}$. Moreover, $a_{kk}$ does
	not occur in the differential of any other chords. 
	By \autoref{res:exact-removal}, there is then a quasi-isomorphism 
	\[
		CE^{*}( \Lambda \cup \Omega;V_{0};\R^{4} ) \isomto
		CE^{*}( \Lambda \cup \Omega_{1} \cup \ldots \cup
		\Omega_{k-1};V_{0};\R^{4} ).
	\]
	By induction we then get the desired quasi-isomorphism 
	$CE^{*}( \Lambda \cup \Omega;V_{0};\R^{4} ) \isomto
	CE^{*}( \Lambda;V_{0};\R^{4} )$. 
	By construction (see the proof of
	\autoref{res:exact-removal}) all words with an endpoint on $\Omega$
	are sent to zero. As a chain complex,
	$CE^{*}( \Lambda \cup \Omega;V_{0};\R^{4} )$ splits into a direct sum of
	the subcomplex $CE^{*}( \Lambda \cup \Omega;V_{0};\R^{4}
	)[\Lambda,\Lambda]$ and the subcomplex of words with an endpoint on
	$\Omega$. This implies that the restriction to 
	$CE^{*}( \Lambda \cup \Omega;V_{0};\R^{4})[\Lambda,\Lambda]$ is also a
	quasi-isomorphism.
\end{proof}

Combining these results, we get that the stopped subalgebra is quasi-isomorphic
to the Chekanov--Eliashberg dg-algebra of $\Lambda$.

\begin{theorem}
\label{res:finite-stopped-models}
	Let $\Lambda \subset \R^{3}$ be a singular Legendrian. 
	Then the canonical inclusion 
	\[
		CE^{*}( \Lambda;V_{0,\partial \Omega};\R^{4})
		\hookrightarrow
		CE^{*}( \Lambda;V_{0};\R^{4} )
	\] 
	is a quasi-isomorphism.
\end{theorem}
\begin{proof}
	Composing the maps \autoref{res:stopped-subalgebra-prop} and
	\autoref{res:stopped-lemma} we get a quasi-isomorphism,
	\[
		CE^{*}( \Lambda;V_{0,\partial \Omega};\R^{4})
		\isomto
		CE^{*}( \Lambda \cup \Omega;V_{0};\R^{4} )[\Lambda,\Lambda]
		\isomto
		CE^{*}( \Lambda;V_{0};\R^{4} ),
	\]
	and it is clear from the construction of these maps that 
	the composition is the canonical inclusion.
\end{proof}
\begin{remark}
\label{rm:omega-subset}
	To simplify the exposition, we have only considered the case when we
	attach one strand $\Omega_i$ at each left singularity. However, one can
	equally well attach them only at a subset of the left singularities.
	If one does this and lets $\Omega$ be the union of the strands attached
	at this subset, one can in the same way define the stopped subalgebra 
	by excluding the chords of $\partial \Lambda$ passing though the $\partial \Omega$.
	The results \autoref{res:stopped-subalgebra-prop}, \autoref{res:stopped-lemma}, and
	\autoref{res:finite-stopped-models} can also be applied in this more
	general setting, and the proofs are
	the same.
\end{remark}

\subsection{Bordered Legendrians from opening up singular Legendrians}
The stopped subalgebra can be realized as the Chekanov--Eliashberg dg-algebra of 
a bordered Legendrian. This allows us to exploit the isotopy invariance and
further simplify the algebra.
\begin{definition}
\label{def:opening}
Let $\Lambda \subset \R^{3}$ be a singular Legendrian 
with all singularities to the sides. 
The \emph{opening} of $\Lambda$ is the
smooth bordered Legendrian $\Lambda^{\circ}$ obtained by removing the singularities of 
$\Lambda$ and separating the strands, as shown in 
\autoref{fig:singular-resolution}.
Given a left or right singularity $t$ of
$\Lambda$, we define $\Lambda^{\circ,t}$ to be the (possibly singular) bordered
Legendrian obtained by opening only at $t$.
\begin{figure}[!htb]
    \centering
    \incfig{opening}
    \caption{The front projection of a singular Legendrian $\Lambda$ and its
    opening $\Lambda^{\circ}$.}
    \label{fig:opening}
\end{figure}
\end{definition}

\begin{definition}
\label{def:resolution}
Let $\Lambda \subset \R^{3}$ be a singular Legendrian 
with all singularities
to the sides. The \emph{resolution} of $\Lambda$ is the
smooth bordered Legendrian $\Lambda^{\bullet}$ constructed as shown in
\autoref{fig:singular-resolution}. It is obtained from the opening
$\Lambda^{\circ}$ by performing a negative half-twist of the ends corresponding
to each left singularity of $\Lambda$, and a positive half twist followed by a
negative half-twist of the ends corresponding to each right singularity, in
such a way that no new chords are introduced between ends corresponding to different
singularities. We are free to choose the order of the twists, as well as
whether they go up or down. Given a left or right singularity $t$ of
$\Lambda$ we define $\Lambda^{\bullet,t}$ to be the (possibly singular) 
bordered Legendrian obtained by resolving only at $t$.
\begin{figure}[!htb]
    \centering
    \incfig{singular-resolution}
    \caption{The front projection of the resolution $\Lambda^{\bullet}$ of the
	    Legendrian $\Lambda$ in
    \autoref{fig:opening}.}
    \label{fig:singular-resolution}
\end{figure}
\end{definition}
\begin{remark}
	Note that due to the choices involved in the definition, the 
	resolution $\Lambda^{\bullet}$ is not canonically defined up
	to Legendrian isotopy. However, for any two different choices of order
	and direction of the twists in the construction of $\Lambda^{\bullet}$,
	there is a canonical identification of the
	respective Reeb chords. The Chekanov--Eliashberg dg-algebra 
	$CE^{*}( \Lambda^{\bullet};\R^{4} )$ is thus well-defined up to
	isomorphism.
\end{remark}
The idea of the following theorem is the easily verified fact that, in the case
when the singularities are stopped, one can replace the finitely many internal
generators by Reeb chords introduced by removing the singularities and wrapping
the strands.
\begin{theorem}
\label{res:resolution}
	Let $\Lambda \subset \R^{3}$ be a singular Legendrian 
	with all singularities to the sides. Then there is a
	quasi--isomorphism 
	\[
		CE^{*}( \Lambda;V_{0};\R^{4} ) \cong 
		CE^{*}( \Lambda^{\bullet};\R^{4} ).
	\]
	Moreover, if $t$ is an arbitrary singularity of $\Lambda$ there is a
	quasi-isomorphism
	\[
		CE^{*}( \Lambda;V_{0};\R^{4} ) \cong 
		CE^{*}( \Lambda^{\bullet,t};V^{\bullet,t}_{0};\R^{4} ).
	\]
\end{theorem}
\begin{proof}
	We prove the first quasi-isomorphism; the proof of the second is
	similar in light of \autoref{rm:omega-subset}.
	By performing a positive half twist at each right singularity of
	$\Lambda$, using Reidemeister VI moves, 
	we can obtain a Legendrian $\Lambda'$ with only left singularities such
	that $(\Lambda')^{\bullet} = \Lambda^{\bullet}$. Since
	$\Lambda$ and $\Lambda'$ are Legendrian isotopic by construction, 
	we get a quasi-isomorphism
	\[
		CE^{*}( \Lambda;V_{0};\R^{4} ) \cong 
		CE^{*}( \Lambda';V_{0};\R^{4} ).
	\]	
	It is thus sufficient to prove the theorem assuming that 
	$\Lambda$ only has left singularities.

	For each left singularity $t$ of valency $n$ of $\Lambda$ the stopped
	subalgebra $CE^{*}(\Lambda;V_{0,\partial \Omega};\R^{4})$ has one Reeb
	chord generator $t_{ij}^{0}$ for each $1 \leq i < j \leq n$, labeling
	counter-clockwise from a base point placed to the left.  The
	differential acts by
	\[
		\partial t_{ij}^{0} = \sum_{i < k < j} \pm t_{kj}^{0}t_{ik}^{0}.
	\] 
	Looking at the corresponding negative half-twist in
	$\Lambda^{\bullet}$ one sees that there is 
	one Reeb chord generator $a_{ij}$ for each $1 \leq i < j \leq n$, using
	the same labeling of the strands as for the points of $\partial
	\Lambda$. The differential applied to these chords are given by 
	 \[
		\partial a_{ij}= \sum_{i < k < j} \pm a_{ik}a_{kj},
	\]
	so the canonical identification of generators, mapping $a_{ij}$ to 
	$t_{ij}^{0}$ is a unital isomorphism of algebras. Using the Ng
	resolution to compute the differential it is clear that this is
	compatible with the differential, and we thus have a dg-algebra
	isomorphism 
	\[
		CE^{*}(\Lambda;V_{0,\partial \Omega};\R^{4}) \cong 
		CE^{*}( \Lambda^{\bullet};\R^{4} ).
	\] 
	The result now follows from
	\autoref{res:finite-stopped-models}.
\end{proof}
\begin{remark} 
	Any singular Legendrian can be moved by a Legendrian isotopy into a position with all
	singularities to the sides, so \autoref{res:resolution} produces finitely
	generated models for all singular Legendrians in $\R^{3}$. The algebra 
	$CE^{*}( \Lambda^{\bullet};\R^{4})$ will depend up to
	quasi-isomorphism on the choice of isotopy.
\end{remark}
\begin{theorem}
\label{res:opening}
	Let $\Lambda \subset \R^{3}$ be a singular Legendrian and let $t$ be
	an arbitrary right singularity of $\Lambda$. Then there is an isotopy
	\[
		\Lambda^{\bullet,t} \sim \Lambda^{\circ,t}
	\] 
	and in particular, there are then quasi-isomorphisms
	\[
		CE^{*}( \Lambda;V_{0};\R^{4} ) \cong 
		CE^{*}(\Lambda^{\bullet,t};\R^{4}) \cong
		CE^{*}(\Lambda^{\circ,t};\R^{4}).
	\]
\end{theorem}
\begin{proof}
	The resolution $\Lambda^{\bullet,t}$ is obtained from
	$\Lambda^{\circ,t}$ by performing two half-twists. By applying 
	\autoref{res:push-twist} (with $\Gamma = \Lambda^{\circ,t}$) we
	get an isotopy $\Lambda^{\bullet,t} \sim (^{*}\Lambda)^{\bullet,t}$. Doing
	this again (with $\Gamma = {}^{*} (\Lambda^{\circ,t} )$) we obtain
	an isotopy $(^{*}\Lambda)^{\bullet,t} \sim \Lambda^{\circ,t}$.
	Compare \autoref{fig:exact-stops} and \autoref{fig:exact-stops-isotopy}.
	The result then follows by \autoref{res:resolution}.
\end{proof}
It is important to note that the above theorem is not true if one opens at more
than one right singularity, as it is then not possible to perform the isotopy.
