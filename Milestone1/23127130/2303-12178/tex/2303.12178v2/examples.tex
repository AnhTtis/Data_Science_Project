\section{Examples and computations}
\label{sec:examples}
We here give a number of examples of how the results in the previous sections can be used to
compute the cohomology and minimal model of the Chekanov–Eliashberg dg-algebra. 

\begin{example}
\label{ex:unknot}
	Let $\Lambda$ be the standard Legendrian unknot with a single top
	handle, illustrated in \autoref{fig:unknot}.
\begin{figure}[!htb]
    \centering
    \incfig{unknot}
    \caption{The front projection of the standard Legendrian unknot $\Lambda$
	    with a single top handle, and its opening $\Lambda^{\circ}$.}
    \label{fig:unknot}
\end{figure}
	The opening of $\Lambda$ lacks Reeb chords, so its Chekanov--Eliashberg
	dg-algebra is isomorphic to the ground field $\textbf{k}$.
	\autoref{res:opening} then says that there is a quasi-isomorphism
	$CE^{*}( \Lambda;V_{0};\R^{4} )\cong \textbf{k}$. This has been shown
	in \cite[Example 7.4]{AE21} using other methods. The result is expected
	in light of the surgery formula, as the corresponding co-core consists
	of a single section in $W_{V} \cong T^{*}\R^{2}$ whose wrapped Floer
	cohomology is generated by a single self-intersection point, see
	\cite[Section 1.1]{AE21}.
\end{example}

\subsection{Rainbow connected sums}
We here describe a way of connecting two singular Legendrians so that 
the Chekanov--Eliashberg dg-algebra of the resulting Legendrian is
quasi-isomorphic to the direct product of the Chekanov--Eliashberg dg-algebras of the
initial Legendrians.
\begin{definition}
Let $\Lambda ^{1}$ and $\Lambda^{2}$ be two unlinked singular Legendrians in
$\R^{3}$, such that $\Lambda^{1}$ and $\Lambda^{2}$
both have $k$ left singularities. The
\textit{rainbow connected sum} of $\Lambda^{1}$ and $\Lambda^{2}$ is the
Legendrian $\Lambda^{1} \# \Lambda^{2}$ obtained by connecting $\Lambda^{1}$ and
$\Lambda^{2}$ by $k$ handles $\Pi$ as in \autoref{fig:rainbow-sum}.
\begin{figure}[!hbt]
    \centering
    \incfig{rainbow-sum}
    \caption{The Rainbow connected sum of two singular Legendrians
    $\Lambda^{1}$ and $\Lambda^{2}$ in the front, with $k=4$.}
    \label{fig:rainbow-sum}
\end{figure}
\end{definition}
\begin{remark}
\label{rm:rainbow-one}
In particular, if one connects two unlinked Legendrians $\Lambda^{1}$ and
$\Lambda^{2}$ with only one
handle in some arbitrary way, the resulting 
Legendrian will always be isotopic to a rainbow connected
sum of two Legendrians isotopic to $\Lambda^{1}$ and $\Lambda^{2}$.
\end{remark}

\begin{proposition}
\label{res:rainbow-map}
	There is a quasi-isomorphism 
	\[
		CE^{*}( \Lambda^{1} \# \Lambda^{2};V_{0}^{1} \cup
		V_{0}^{2};\R^{4} ) \isomto 
		CE^{*}( \Lambda^{1};V_{0}^{1};\R^{4}) \times 
		CE^{*}( \Lambda^{2};V_{0}^{2};\R^{4}) 
	\]
	where '$\times$' denotes the direct product of dg-algebras.
\end{proposition}
\begin{proof}
	Similarly to in the proof of \autoref{res:stopped-lemma}, one can
	perform an isotopy of $\Lambda^{1} \# \Lambda^{2}$ introducing 
	chords as in \autoref{fig:exact-stops-isotopy} and use 
	\autoref{res:exact-removal} to successively remove the handles in $\Pi$. 
	We then obtain a quasi-isomorphism 
	$CE^{*}( \Lambda^{1}\#\Lambda^{2};V_{0};\R^{4} )  \isomto
	CE^{*}( \Lambda^{1}\cup \Lambda^{2};V_{0}^{1} \cup 
	V_{0}^{2};\R^{4} )$, and since $\Lambda^{1}$ and $\Lambda^{2}$ are
	unlinked there is a natural isomorphism $CE^{*}( \Lambda^{1}\cup
	\Lambda^{2};V_{0}^{1}\cup V_{0}^{2};\R^{4} ) \cong
	CE^{*}( \Lambda^{1};V_{0}^{1};\R^{4}) \times 
	CE^{*}( \Lambda^{2};V_{0}^{2};\R^{4})$.
\end{proof}

Using this, we can compute the cohomology of the handcuff graph.
\begin{example}
	Let $\Lambda$ be the singular Legendrian illustrated in
	\autoref{fig:handcuff-graph},
	obtained by connecting two unknots with a single handle. 
\begin{figure}[!htb]
    \centering
    \incfig{handcuff-graph}
    \caption{The front projection of two unlinked unknots connected by a handle.}
    \label{fig:handcuff-graph}
\end{figure}
	As noted in \autoref{rm:rainbow-one}, $\Lambda$ is isotopic to a
	rainbow connected sum of two unknots. By \autoref{res:rainbow-map} and
	\autoref{ex:unknot} we then have
	\[
		CE^{*}( \Lambda;V_{0};\R^{4} ) \cong \textbf{k} \times
		\textbf{k}.
	\] 
	This result was expected by An--Bae in \cite[Section 8]{AB20}.
\end{example}
\subsection{Finite dimensional models}
We here construct an infinite class of Legendrians for which 
\autoref{res:opening} gives finite dimensional models.

\begin{definition}
\label{def:permutation-legendrians}
	A \emph{handle permutation of order $n$} is a word
	$\sigma=\sigma_1\sigma_2\ldots\sigma_2$ of elements $\sigma_{i} \in
	\{1,2,\ldots,n\}$ such that each $k \in \{1,2,\ldots,n\}$ appears
	precisely two times in $\sigma$. We write $\sigma( i^{-} )$ and
	$\sigma( i^{+} )$ for the unique numbers such that 
	$\sigma_{\sigma( i^{-} )}=\sigma_{\sigma( i^{+} )}=i$ and $\sigma(
	i^{-} ) < \sigma( i^{+} )$. 
	If two handle permutations are related by a permutation (i.e.
	relabeling) of
	$\{1,2,\ldots,n\}$ we consider them to be equivalent.
\end{definition}
\begin{definition}
	Let the \emph{standard unknot of size $l$} be the knot whose front
	diagram is obtained by a uniform scaling of the front of the standard
	unknot illustrated to the left in \autoref{fig:unknot}, such that the
	length of the Reeb chord becomes $l$. Given a handle permutation
	$\sigma$ of order $n$, we define $\Lambda^{\circ}_{\sigma}$ to be the
	smooth bordered Legendrian consisting of $n$ strands
	$\Lambda_1^{\circ}, \ldots ,\Lambda_n^{\circ}$ such that each strand
	$\Lambda_i^{\circ}$ is a copy of the left half of the standard unknot
	of size $\sigma( i^{+} ) - \sigma( i^{-} )$ translated so that the
	upper and lower ends of $\Lambda_{i}^{\circ}$ have the respective front
	coordinates $( 0,\sigma( i^{+} ))$ and $( 0,\sigma( i^{-} ))$. We then
	define $\Lambda_{\sigma}$ to be the unique singular Legendrian with
	opening $\Lambda^{\circ}_{\sigma}$ such that $\Lambda_{\sigma}$ only
	has one singularity, and call it the \emph{permutation Legendrian} of
	$\sigma$. We denote by $V_{\sigma}$ the Weinstein thickening of
	$\Lambda_{\sigma}$. Note that this construction is independent of the
	choice of representative of the handle permutation, as defined above.
\end{definition}

The algebra $CE^{*}( \Lambda_{\sigma}^{ \circ};\R^{4} )$ has $n$
idempotents and one Reeb chords generator $a_{ij}$ from $i$ to $j$ for each pair 
$i$ and $j$ such that $\sigma( i^{-} ) < \sigma( j^{-} ) < \sigma( i^{+} ) <
\sigma( j^{+} )$. If one
gives the handles the same Maslov potential (under the canonical identification
by translation and rescaling) then $| a_{ij} | = 1$ for all $i$ and $j$.
The differential is given by
\[
	\partial a_{ij} = \sum \pm a_{kj}a_{ik}
\] 
where the sum is taken over all $1 < k < n$ for which the chords $a_{kj}$ and
$a_{ik}$ exist, i.e. all $k$ such that $\sigma( i^{-} ) < \sigma( k^{-} ) <
\sigma( i^{+} ) < \sigma( k^{+} )$ and $\sigma( k^{-} ) < \sigma( j^{-} ) <
\sigma( k^{+} ) < \sigma( j^{+} )$. The
finite dimensionality implies that the cohomology can be readily computed.
\begin{example}
\label{ex:standard-an}
	Let $n > 0$ and let $\sigma$ be the handle permutation of order $n$ given by
	$\sigma(i^{-}) = i$ and $\sigma(i^{+}) = 2i$. We call $\Lambda_{A}^{n}
	:= \Lambda_{\sigma}$ the \emph{standard $A_{n}$-Legendrian}. 
\begin{figure}[!htb]
    \centering
    \incfig{an-pos-neg-sing}
    \caption{The front projection of the 
	    standard $A_{n}$-Legendrian $\Lambda_{A}^{n}$ and its opening
    $\Lambda_{A}^{n,\circ}$.} 
    \label{fig:an-pos-neg-sing}
\end{figure}
	The algebra $CE^{*}( \Lambda_{A}^{n,\circ};\R^{4} )$ has one
	generator $a_{ij}$ for each $1 \leq i < j \leq n$, with  
	\[
		\partial a_{ij} = \sum_{i < k < j} \pm a_{kj}a_{ik}.
	\] 
	\begin{proposition}
	The minimal model of $CE^{*}( \Lambda_{A}^{n};V_{A,0}^{n};\R^{4} )$ is
	isomorphic to the path algebra of the $A_{n}$-quiver
\[\begin{tikzcd}
	{\stackrel{1}{\bullet}} & {\stackrel{2}{\bullet}} & \stackrel{3}{\bullet} & 
	{\ldots} & {\stackrel{n-1}{\bullet}} & 
	{\stackrel{n}{\bullet}} 
	\arrow["{\alpha_1}", from=1-1, to=1-2]
	\arrow["{\alpha_2}", from=1-2, to=1-3]
	\arrow["{\alpha_3}", from=1-3, to=1-4]
	\arrow["{\alpha_{n-1}}", from=1-4, to=1-5]
	\arrow["{\alpha_{n-2}}", from=1-5, to=1-6]
\end{tikzcd}\]
	bound by the relations $\alpha_{i+1}\alpha_{i}=0$ for $1 \leq i \leq
	n-2$, without any higher operations.
	\end{proposition}
	\begin{proof}
		The differential of $CE^{*}( \Lambda_{A}^{n,\circ};\R^{4} )$ 
		is essentially the same differential as $\partial_{0}$ in
		\autoref{ssec:surgery-surfaces} and one can use the same argument as in
		\autoref{res:short-chords-generate} to see that the cohomology is
		generated by the residues of the chords of the form $a_{i,i+1}$ for $1
		\leq i < n$, with zero multiplication. The cohomology is thus
		isomorphic to the path algebra. There is a dg-algebra quasi-isomorphism
		$CE^{*}( \Lambda_{A}^{n,\circ};\R^{4} ) \to
		H^{*}CE( \Lambda_{A}^{n,\circ};\R^{4} )$ given by
		$a_{i,i+1} \mapsto [a_{i,i+1}]$ and $a_{ij} \mapsto 0$ for 
		$j-i > 1$, which shows that this indeed is the minimal model
		of $CE^{*}( \Lambda_{A}^{n,\circ};\R^{4} )$, and 
		then by \autoref{res:opening} also the minimal model of $CE^{*}(
		\Lambda_{A}^{n};V_{A,0}^{n};\R^{4} )$
	\end{proof}
	
	Partial computations of this example were done by An--Bae in
	\cite[Section 8]{AB20}, where they expected this result. These
	computations are related to the computations using the technology of
	microlocal sheaf theory; see Nadler's computations for the $A_n$-arboreal
	Legendrian from \cite{Nad17}.

	As a geometric application, we can use this computation to obstruct the
	existence of certain isotopies, compare \cite{Mis03}.
	\begin{corollary}
		A Legendrian isotopy from $\Lambda_{A}^{n}$ to itself cannot
		produce a non-trivial permutation of the handles. 
	\end{corollary}
	\begin{proof}
		An isotopy from $\Lambda_{A}^{n}$ to itself permuting 
		the handles induces an algebra automorphism of $H^{*}CE(
		\Lambda_{A}^{n};V_{A,0}^{n};\R^{4} )$ permuting the corresponding 
		idempotents in the same way, and no non-trivial
		permutations of the idempotents of 
		$H^{*}CE( \Lambda_{A}^{n};V_{A,0}^{n};\R^{4} )$ can be produced by
		an algebra automorphism.
	\end{proof}
\end{example}

\subsubsection{Mutations}
Consider an abstract Weinstein surface $V$ such that $V_{0}$ is connected
and let $\partial \Lambda \subset V_{0}$ be the critical attaching spheres. By
fixing a base point in $\partial V_{0}$ and following the Reeb flow around
$\partial V_{0}$ we get a handle permutation $\sigma$ such that $V$
and $V_{\sigma}$ are isomorphic as Weinstein manifolds and
there is an embedding $V \hookrightarrow \R^{3}$ whose image is $V_{\sigma}$. The
handle permutation $\sigma$ depends on the base point and will in general result in
non-Legendrian isotopic $V_{\sigma}$. However, given a fixed handle
decomposition, $\sigma$ is unique up to cyclic
permutations and $V_{\sigma}$ is unique up to Weinstein isotopy.  

\begin{definition}
	Let $\sigma$ be a handle permutation of order $n$ and let $k \in
	\Z$. The \emph{shift of $\sigma$ by $k$} is the
	permutation
	\[
		\sigma[k]:= \sigma_{1-k}\ldots\sigma_{i-k}\ldots\sigma_{n-k},
	\] 
	counting mod $n$ in the indices.
\end{definition}

\begin{proposition}
\label{res:mutation}
	Let $\sigma$ be a handle permutation of order $n$ and let $k \in \Z$.
	There is a Weinstein isotopy
	\[
		V_{\sigma} \sim
		V_{\sigma[k]}.
	\] 
\end{proposition}
\begin{proof}
	It is sufficient to show this for $k=1$. The isotopy is illustrated in
	\autoref{fig:weinstein-isotopy-proof} and is performed as follows. We
	relabel so that $\sigma_{2n} = n$ and 
	give the ends of the handles $\Lambda_i$ of $V_{\sigma}$ the labeling
	$i^{\pm}$ where $i=1,\ldots,n$, as specified in
	\autoref{def:permutation-legendrians}. In the first step, we perform a
	Weinstein isotopy introducing a handle with a right cusp as in
	\autoref{res:weinstein-lemma}, which splits
	the singularity in two singularities such that an end $i^{\pm}$
	belongs to the lower singularity if $\sigma( i^{\pm} ) \leq \sigma(
	n^{-} )$ and to the upper singularity if $\sigma( i^{\pm}) > \sigma(
	n^{-} )$. In the second and third steps, we perform
	Reidemeister IV moves at the ends $n^{-}$ and $n^{+}$ which moves the
	handle $\Lambda_{n}$ into a position slightly below the introduced
	handle, obtaining the Legendrian in the fourth picture in
	\autoref{fig:weinstein-isotopy-proof}. This
	Legendrian can be obtained from $V_{\sigma[1]}$ by an analogous
	Weinstein isotopy, as shown in the two last pictures of the figure. 
	We thus have $V_{\sigma} \sim V_{\sigma[1]}$.
\begin{figure}[!htb]
    \centering
    \incfig{weinstein-isotopy-proof}
    \caption{A Weinstein isotopy between the permutation Legendrians
	    $\Lambda_{ 121323 }$ and $\Lambda_{ 312132 }$. 
	    The figure should be read from left to right and top
    	    to bottom.}
    \label{fig:weinstein-isotopy-proof}
\end{figure}
\end{proof}

Viewing the Chekanov--Eliashberg dg-algebras as path algebras,
the quiver of $CE^{*}( \Lambda_{ \sigma[1] }^{\circ};\R^{4} )$ can be obtained
from that of $CE^{*}( \Lambda_{\sigma}^{\circ};\R^{4} )$ by flipping all
arrows with target on the vertex corresponding to the $\sigma_{2n}$:th handle.
The isotopy can therefore be viewed as geometric realization of the 
Bernstein--Gelfand--Ponomarev reflection functors \cite{BGP73}.
\begin{example}
\label{ex:an-no-relations}
	Consider the handle permutation $\sigma = 121323$. 
\begin{figure}[!htb]
    \centering
    \incfig{source-mutation-example}
    \caption{The openings of the three Legendrians obtained by shifting
    $\sigma=121323$.}
    \label{fig:source-mutation-example}
\end{figure}
	The openings of 
	$\Lambda_{\sigma}$ and the shifts $\Lambda_{ \sigma[1] }$ and
	$\Lambda_{ \sigma[2] }$ are illustrated in \autoref{fig:source-mutation-example}.
	The Chekanov--Eliashberg dg-algebras of the openings of all three have
	vanishing differential and are isomorphic to the respective path
	algebras of the following quivers:
\[\begin{tikzcd}
	\bullet & \bullet & \bullet, && \bullet & \bullet & \bullet, && \bullet &
	\bullet & \bullet.
	\arrow[from=1-1, to=1-2]
	\arrow[from=1-2, to=1-3]
	\arrow[from=1-7, to=1-6]
	\arrow[from=1-5, to=1-6]
	\arrow[from=1-10, to=1-11]
	\arrow[from=1-10, to=1-9]
\end{tikzcd}\]
\end{example}

The Weinstein isotopies in \autoref{res:mutation} are, however, not all
Weinstein isotopies which exist between the permutation Legendrians, as the
following example shows.
\begin{example}
	Let $n > 1$ and let $\sigma$ the handle permutation of order $n$,
	given by
	\begin{align*}
		\sigma( 1^{-} ) = 1, && \sigma( i^{-} ) = 2i-2, &&& \sigma(
		n^{-} ) = 2n-2,\\
		\sigma( 1^{+} ) = 3, && \sigma( i^{-} ) = 2i+1, &&& \sigma(
		n^{-} )
		= 2n,
	\end{align*}
	for $i \neq 1,n$. The permutation Legendrian $\Lambda_{A'}^{n}:=
	\Lambda_{\sigma}$ and its opening are
	illustrated in
	\autoref{fig:an-no-relations}.
\begin{figure}[!htb]
    \centering
    \incfig{an-no-relations}
    \caption{The front projection of the Legendrian $\Lambda_{A'}^{n}$ and its
	    opening $\Lambda_{A'}^{n, \circ}$.}
    \label{fig:an-no-relations}
\end{figure}
The Chekanov--Eliashberg dg-algebra of $\Lambda_{A'}^{n, \circ}$ 
has differential zero and is isomorphic to the path
algebra of the $A_{n}$-quiver 
\[\begin{tikzcd}
	{\stackrel{1}{\bullet}} & {\stackrel{2}{\bullet}} & \stackrel{3}{\bullet} & 
	{\ldots} & {\stackrel{n-1}{\bullet}} & 
	{\stackrel{n}{\bullet}} 
	\arrow["{\alpha_1}", from=1-1, to=1-2]
	\arrow["{\alpha_2}", from=1-2, to=1-3]
	\arrow["{\alpha_3}", from=1-3, to=1-4]
	\arrow["{\alpha_{n-1}}", from=1-4, to=1-5]
	\arrow["{\alpha_{n-2}}", from=1-5, to=1-6]
\end{tikzcd}\]
without any relations. We expect that this Weinstein surface can be naturally 
realized a Milnor fiber of the $A_{n}$-singularity in the boundary of $B^{4}$,
see \cite[Corollary 1.17]{GPS18}\cite{Nad17}.
\begin{proposition}
\label{res:an-isotopy}
	There is a Weinstein isotopy,
	\[
		V_{A}^{n} \sim V_{ A' }^{n}.
	\] 
\end{proposition}
\begin{proof}
	The isotopy is illustrated in \autoref{fig:an-isotopy} for $n=4$. The
	procedure is the same for all $n$. 
\begin{figure}[!htb]
    \centering
    \incfig{an-isotopy}
    \caption{A Weinstein isotopy between $V_{A}^{n}$ and $V_{ A' }^{n}$ for
	    $n=4$. The figure should be read from left to right and top to bottom.}
    \label{fig:an-isotopy}
\end{figure}
	The first step is a Weinstein handle
	introduction as in \autoref{res:weinstein-lemma}. The second step is a
	Reidemeister VI move. The third step is another handle introduction.
	The fourth step is an application of \autoref{res:push-twist}. The
	fifth, sixth, and seventh steps are performed in the same way as the
	first four: by introducing a handle, performing a Reidemeister VI move,
	introducing another handle, and then applying 
	\autoref{res:push-twist}. By repeating this procedure for each
	handle and then performing a series of Weinstein handle contractions one 
	then obtains the Legendrian in the ninth picture. The second to last step is
	then a Reidemeister IV move, followed by another application of 
	\autoref{res:push-twist}, and the final step is a Weinstein
	handle contraction.
\end{proof}
\end{example}

Examples \ref{ex:standard-an} and \ref{ex:an-no-relations} 
demonstrate that the Chekanov--Eliashberg dg-algebra can undergo
significant changes under Weinstein isotopy. Recall that this corresponds to
different choices of generators of the partially wrapped Fukaya category
obtained by stopping at the Weinstein surface. It
is interesting to note that the path algebra of the $A_{n}$-quiver with all
quadratic relations quotiented out is the Koszul dual of the path algebra of
the same quiver without relations.  We intend to investigate
bifurcations of the Chekanov--Eliashberg dg-algebra under Weinstein isotopy further in the future.
\subsection{Infinite dimensional models}
In addition to the permutation Legendrians, there are several families of
singular Legendrians with infinite dimensional models for which the 
cohomology can be computed. 
\begin{example}
\label{ex:theta}
	Let $n > 1$ and let $\Lambda_{\theta}^{n}$ be the singular
	Legendrian with one right
	singularity and one left singularity such that when opening
	$\Lambda_{\theta}^{n}$ at both singularities one obtains a bordered Legendrian
	consisting of $n$ strands without cusps, crossings, or singularities.
	It is illustrated for $n=4$ in \autoref{fig:an-plus-one-construction}.
\begin{figure}[!htb]
    \centering
    \incfig{an-plus-one-construction}
    \caption{The front projection of the Legendrian $\Lambda_{\theta}^{n}$ for
    $n=4$.}
    \label{fig:an-plus-one-construction}
\end{figure}
	We call $\Lambda_{\theta}^{n}$ the \emph{$\theta_{n}$-Legendrian}. From
	the proof of \autoref{res:an-isotopy} we see that
	$\Lambda_{\theta}^{n}$ is Weinstein isotopic to $\Lambda_{A}^{n-1}$ and
	$\Lambda_{A'}^{n-1}$.  
	\begin{proposition}
	\label{res:theta-minimal}
	The minimal model of $CE^{*}(
	\Lambda_{\theta}^{n};V_{\theta,0}^{n};\R^{4} )$ is isomorphic to
	the $A_{\infty}$-algebra whose underlying module is the path algebra of
	the quiver with one vertex for each $i \in \Z_{n}$ and one arrow
	$\alpha_{i}$ from $i$ to $i+1$ for each $i$, with all compositions
	quotiented out, and whose $A_{\infty}$-operations $\mu_{k}$ act
	by
	 \[
		 \mu_{n}( \alpha_{i+n-1} \otimes\ldots \otimes \alpha_{i+1}
		 \otimes \alpha_{i} ) = \varepsilon_{i}
	 \] 
	for each $i \in \Z_{n}$, and vanish on all other words.
	\end{proposition}
	\begin{proof}
		By \autoref{res:opening}, we have a quasi-isomorphism 
		\[
			CE^{*}(
			\Lambda;V_{0};\R^{4} ) \cong CE^{*}(
			(\Lambda_{\theta}^{n})^{\circ,t};V_{0};\R^{4} ),
		\] 
		where $t$ is
		the right singularity. The algebra $CE^{*}(
		(\Lambda_{\theta}^{n})^{\circ,t};V_{0};\R^{4} )$ is isomorphic
		to its internal algebra, i.e. the Chekanov--Eliashberg dg-algebra
		of $n$ one-point stops in the boundary of the disk. The result
		then follows from 
		\autoref{res:zero-dim-leg-min-mod}.
	\end{proof}
	We thus have a third, non-formal representative of the
	Chekanov--Eliashberg dg-algebras of the Weinstein isotopy class of 
	$\Lambda_{A}^{n}$ and $\Lambda_{A'}^{n}$.

	\begin{remark}
	Geometrically, the fact that $\Lambda_{\theta}^{n}$ and the $n$-point
	stop have quasi-isomorphic Chekanov--Eliashberg dg-algebras can be
	explained by noting that $\Lambda_\theta^{n}$ is the $(
	T^{*}D^{1},T^{*}S^{0} )$-stabilization of the $n$-point stop. This
	stabilization construction is a special case of the product
	construction defined in \cite[Section 3.2]{Eli18}, and for which
	Ganatra--Pardon--Shende \cite{GPS18} have constructed a Künneth
	embedding of the corresponding partially wrapped Fukaya categories. In
	light of \autoref{res:sing-surgery-map}, one therefore expects there to
	be a Künneth formula for Chekanov--Eliashberg dg-algebras, from which the
	above quasi-isomorphism would follow.  
	\end{remark}

	Similarly to in \autoref{ex:standard-an}, we can use the above computation 
	to obstruct certain isotopies. 

	\begin{corollary}
		There exists a Legendrian isotopy from $\Lambda_\theta^{n}$ to
		itself realizing a
		given permutation of the handles if and only if the
		permutation is cyclic.
	\end{corollary}	
	\begin{proof}
		The only permutations of the idempotents of the minimal model
		in \autoref{res:theta-minimal} which can be realized by an
		algebra automorphism are the cyclic ones. This implies that
		there are no isotopies of $\Lambda$ producing non-cyclic
		permutations of the handles. The cyclic permutations are easy
		to produce using the Reidemeister moves.
	\end{proof}
\end{example}
\begin{example}
Let $n > 0$. We construct a singular Legendrian $\Lambda_{\text{cyc}}^{n}$ as
follows. 
\begin{figure}[!htb]
    \centering
    \incfig{vanishing-example}
    \caption{The singular Legendrian $\Lambda_{\text{cyc}}^{n}$, defined as the
    union of $n$ unknots successively more squeezed in the $x$-direction and
    stretched in the $z$-direction.} 
    \label{fig:vanishing-example}
\end{figure}
First, we consider a standard unknot with a single top handle and with the
singularity and right cusp placed at the origin in $\R^{3}$. We construct a
second unknot by applying a transformation of the form $( x,z ) \to ( x \slash
M, M z)$ for some $M > 1$ to the front diagram of the first unknot.  We
construct a third unknot by applying the same transformation to the second
unknot, and continue like this until we have $n$ unknots. Then
$\Lambda_{\text{cyc}}^{n}$ is defined to be the union of these unknots, see
\autoref{fig:vanishing-example}.

\begin{proposition}
	The minimal model of $CE^{*}(
	\Lambda_{\textup{cyc}}^{n};V_{\textup{cyc},0}^{n};\R^{4} )$ is 
	isomorphic to
	the path algebra of the quiver with $n$ vertices $1,\ldots,n$ and one
	arrow from $i$ to $j$ for each ordered pair $( i,j ) \in
	\{1,\ldots,n\}^{2}$ such that $i \neq j$.
\end{proposition}
\begin{proof}
	For each ordered pair $(i,j)$ of distinct components of
	$\Lambda_{\textup{cyc}}^{n,\circ}$, there is one
	Reeb chord $a_{ij}$ from $i$ to $j$. If one gives the handles the same
	Maslov potential then each $a_{ij}$ has degree zero, and the
	differential thus vanishes. The result then follows by \autoref{res:opening}.
\end{proof}
\begin{proposition}
	The Legendrian $\Lambda_{\textup{cyc}}^{n}$ is not Weinstein
	isotopic to any permutation Legendrian.
\end{proposition}
\begin{proof}
	As noted in \cite[Remark 1.5]{AE21}, the Hochschild homology of the
	singular Chekanov--Eliashberg dg-algebra is a Weinstein isotopy invariant.
	The Hochschild homology is derived invariant, and in particular,
	invariant under quasi-isomorphism. Recall that the Hochschild homology
	in degree $0$ of an associate algebra $A$ is the quotient $A \slash
	\text{Span}\{ab - ba \mid a,b \in A\}$. Since $CE^{*}( \Lambda^{n,
	\circ}_{\text{cyc}};\R^{4} )$ is the path algebra of a cyclic quiver
	this implies that $HH_{0}(CE^{*}( \Lambda^{n,
	\circ}_{\text{cyc}};\R^{4} ) )$ is infinite dimensional (a
	linearly independent infinite subset 
	is given by e.g. $\{[(a_{21}a_{12})^{i}] \mid i
	\in \N \}$). On the other
	hand, for any permutation Legendrian $\Lambda_{\sigma}$, the algebra
	$CE^{*}( \Lambda_\sigma^{\circ};\R^{4} )$ is a semi-free finite
	dimensional dg-algebra and therefore homologically smooth and compact,
	see \cite[Section 8]{KS09}. By \cite[Proposition 8.10]{KS09}, it
	then follows that the Hochschild homology of $CE^{*}(
	\Lambda_\sigma^{\circ};\R^{4} )$ is finite dimensional. Thus, such a
	Weinstein isotopy cannot exist.
\end{proof}
\end{example}

\begin{example}
	Let $n > 1$ and let $\Lambda_{\theta'}^{n}$ be the Legendrian
	illustrated in \autoref{fig:stabilized-theta}. 
\begin{figure}[H]
    \centering
    \incfig{stabilized-theta}
    \caption{The front projection the singular Legendrian knot
    $\Lambda_{\theta'}^{n}$ for $n=3$, and its construction from
    $\Lambda_{\theta}^{n}$.}
    \label{fig:stabilized-theta}
\end{figure}
	 It is obtained from the Legendrian $\Lambda_{\theta}^{n}$ from 
	 \autoref{ex:theta} by performing a
	 Legendrian isotopy using \autoref{res:push-twist} so that we get two
	 right singularities, and then attaching and contracting a handle with
	 core $\Pi$, as shown in the figure.
\begin{proposition}
	 There is a quasi-isomorphism
	 \[
		 CE^{*}( \Lambda_{\theta'}^{n};V_{\theta',0}^{n};\R^{4} ) \cong 
		 CE^{*}( \Lambda_{\theta}^{n};V_{\theta,0}^{n};\R^{4} ).
	 \] 
\end{proposition}
\begin{proof}
	The Reeb chord corresponding to the right cusp of $\Pi$ in
	\autoref{fig:stabilized-theta} has
	differential $\pm e_{\Pi}$. The result then follows by
	\autoref{res:exact-removal} and \autoref{res:weinstein-isotopy-cor}
\end{proof}
\begin{remark}
	This provides an alternative proof for a result by Etg\"{u}--Lekili.
	Let $CE^{*}( \partial \Omega;D_{2} )$ be the Chekanov--Eliashberg
	dg-algebra of $n$ one-point stops in the boundary of the disk $D^{2}$.  
	In \cite[Theorem 12]{EL19}, it was shown that the inclusion of the
	dg-subalgebra of all chords $c_{ij}^{p}$ with $p \leq 1$ into $CE^{*}(
	\partial \Omega;D_{2} )$ is a quasi-isomorphism. This subalgebra is
	isomorphic to $CE^{*}( \Lambda_{\theta'}^{n, \circ};\R^{4} )$. The
	chords with $p=0$ correspond to the chords in the triangle shape to
	the left in \autoref{fig:stabilized-theta}, and the chords with $p=1$
	to the chords in the diamond shape in the center. Using
	\autoref{res:opening} we thus recover \cite[Theorem 12]{EL19}.
\end{remark}
\end{example}
