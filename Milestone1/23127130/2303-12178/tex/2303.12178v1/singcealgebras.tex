\section{Chekanov--Eliashberg algebras for singular Legendrians}
In this section, we give an overview of the Asplund--Ekholm construction of the
Chekanov--Eliashberg dg-algebra for singular Legendrians, and how to compute it
in $\R^{3}$ using the combinatorial description by An--Bae \cite{AB20}. The
definition immediately gives singular versions of the surgery and cobordism
maps for smooth Legendrians. We also review the singular Legendrian
Reidemeister moves and the Ng resolution. 

\subsection{Chekanov--Eliashberg algebras for singular Legendrians}
\label{ssec:ceforsing}
This material follows \cite{AE21}, to which we refer for details.  Let $W$ be a
$2n$-dimensional Weinstein manifold, and let $V \subset \partial W$ be a
$(2n-2)$-dimensional Weinstein hypersurface. We will also allow the case
$W=B^{2n} \setminus \text{pt}$ for some $\text{pt} \in \partial B^{2n} =
S^{2n-1}$. The latter corresponds to the Weinstein manifold with a stop at a
point in the boundary, and $\partial W = \R^{2n+1}$ can be identified with the
standard contact vector space, i.e. the standard contact Darboux ball. We
denote the singular Legendrian skeleton of $V$ as $\Lambda$, and write $V_{0}$
for the subcritical part of $V$ and $\partial \Lambda \subset \partial V_{0}$
for the Legendrian attaching attaching spheres in the handle decomposition of
$V$. We construct a cobordism $W_{V}^{0}$ as follows.  Let
$D^{*}_{\epsilon}[-1,1]$ be an $\epsilon$-disk subbundle of $T^{*}[-1,1]$. Then
$W_{V}^{0}$ is obtained by attaching the handle $V_{0}\times
D^{*}_{\epsilon}[-1,1]$ to $W \sqcup (\R\times ( \R\times V ))$ along
$V_{0}\times D^{*}_{\epsilon}[-1,1]\mid_{\{-1,1\}} \subset \partial W \sqcup (
\R\times V )$. We have $\partial_{-}W_{V_{0}} = \R \times V$.  We construct a
smooth Legendrian link $\Sigma$ of $n$-spheres in $\partial W_{V}^{0}$ by
joining the two copies of the top strata $\Lambda \setminus V_{0}$ in $\partial
W$ and $\R \times V$ across the handle by $\partial \Lambda \times [-1,1]$, as
shown in \autoref{fig:singular-definition}. For each Legendrian handle $\Pi$ in
$\Lambda$ we denote the corresponding Legendrian attaching sphere by $\Sigma(
\Pi ) \subset \Sigma$.
\begin{figure}[!htb]
    \centering
    \incfig{singular-definition}
    \caption{The cobordism $W_{V}^{0}$ in the Asplund--Ekholm construction.}
    \label{fig:singular-definition}
\end{figure}

\begin{definition}[{\cite[Definition 3.2]{AE21}}]
The \emph{singular Chekanov--Eliashberg dg-algebra} of $\Lambda$ in $\partial
W$, denoted by $CE^{*}( \Lambda;V_{0};W)$, is the ordinary smooth
Chekanov--Eliashberg algebra as in \autoref{ssec:smooth-ce} of $\Sigma$ in
$\partial W_{V}^{0}$, i.e.
\[
	CE^{*}( \Lambda;V_{0};W ):=CE^{*}( \Sigma;W_{V}^{0} ).
\] 
\end{definition}

\begin{lemma}[\cite{AE21}]
\label{res:ce-sing-subalg}
	The Chekanov--Eliashberg algebra 
	$CE^{*}( \Lambda;V_{0};W )$ contains $CE^{*}( \partial
	\Lambda;V_{0} )$ as a dg-subalgebra.
\end{lemma}
\begin{proof}
	See \cite[Section 3.3]{AE21}.
\end{proof}
The definition immediately gives an analogue of the surgery formula for
singular Legendrians.  We write $W_{V}:=W_{V,\Sigma}^{0}$ for the Weinstein
manifold obtained by handle attachment to $W_{V}^{0}$ to along $\Sigma$, and
denote the co-cores of $W_{V}$ as $C$. 

\begin{theorem}[{\cite[Theorem 1.1]{AE21}}]
\label{res:sing-surgery-map}
	There is an $A_{\infty}$-quasi-isomorphism	
	\[
		CE^{*}( \Lambda;V_{0};W ) \isomto CW^{*}( C;W_{V} ).
	\] 
\end{theorem}
\begin{proof}
	Follows from \autoref{res:surgery-map} and the definition of $CE^{*}(
	\Lambda;V_{0};W )$.
\end{proof}
	This surgery formula is the same as the one in \cite[Conjecture
	3]{EL17} \cite{Syl19}. The wrapped Floer cohomology of $W_{V}$, can
	equivalently be thought of as the partially wrapped Floer cohomology of
	$W$ stopped at $V$. 
\begin{remark}	
	A smooth Legendrian $\Lambda$ can be considered as a singular
	Legendrian with a single top handle, and the singular
	Chekanov--Eliashberg algebra is then quasi-isomorphic to the
	Chekanov--Eliashberg algebra with based loop space coefficients, see
	\cite[Theorem 1.2, Section 4]{AE21}. 
\end{remark}

	We will also consider a relative version of the algebra. 

	\begin{definition}
	Let $\Pi \subset \Lambda$ be a subset of the Legendrian cores of the
	top handles of $V$. Let $W_{V,\Sigma( \Pi )}^{0}$ be the cobordism
	obtained by handle attachment along $\Sigma( \Pi )$ in $W_{V}^{0}$. The
	Chekanov--Eliashberg algebra of $\Lambda \setminus \Pi$ \emph{relative}
	$\Pi$ is 
	\[
		CE^{*}( \Lambda \setminus \Pi;V_{0,\partial \Pi};W ):= 
		CE^{*}( \Sigma \setminus \Sigma_{\Pi};W_{V,\Sigma_\Pi}^{0}).  
	\]
	\end{definition}
	This is a slightly different and less general construction than in
	\cite[Section 6.1]{AE21}. However, since there are no Reeb chords in
	the negative part of $\partial W^{0}_{V,\Sigma( \Pi )}$ that
	corresponds to $V \times \R$, and no pseudo-holomorphic curves can
	enter there (\cite[Lemma 3.5]{AE21}), it results in an isomorphic
	algebra. This definition gives an analogue of the cobordism map.

\begin{theorem}
\label{res:sing-cobordism-map}
	There is a dg-algebra morphism
	\[
		CE^{*}( \Lambda \setminus \Pi;V_{0,\partial \Pi};W ) \to 
		CE^{*}( \Lambda;V_{0};W )
	\]
	which is a quasi-isomorphism onto 
	$CE^{*}( \Lambda;V_{0};W )[\Lambda \setminus \Pi,\Lambda \setminus
	\Pi]$.
\end{theorem}
\begin{proof}
	Follows from \autoref{res:cobordism-map} and the definition of $CE^{*}(
	\Lambda;V_{0};W )$ and $CE^{*}( \Lambda \setminus \Pi;V_{0,\partial
	\Pi};W )$.
\end{proof}
As suggested by the notation, there is also a relative version of
\autoref{res:ce-sing-subalg}.
\begin{lemma}[\cite{AE21}]
	The Chekanov--Eliashberg algebra $CE^{*}( \Lambda \setminus
	\Pi;V_{0,\partial \Pi};W )$ contains $CE^{*}( \partial \Lambda
	\setminus \partial \Pi;V_{0, \partial \Pi} )$ as a dg-subalgebra.
\end{lemma}
\begin{proof}
	See \cite[Section 3.3]{AE21}.
\end{proof}
Asplund--Ekholm give a surgery presentation of this algebra. Since
\autoref{res:cobordism-map} fails to be an isomorphism for $n=2$ this case
needs to be treated separately.

\begin{lemma}[{\cite[Proposition 6.2]{AE21}}]
\label{res:surgery-presentation}
	For $n > 2$, the algebra $CE^{*}( \Lambda \setminus \Pi;V_{0,\partial
	\Pi};W)$ is generated by composable words of Reeb chords of $\Lambda$
	in $\partial W$ and $\partial \Lambda$ in $V_{0}$, without an endpoint
	on $\Pi$ or $\partial \Pi$. The subalgebra $CE^{*}( \partial \Lambda
	\setminus \partial \Pi; V_{0,\partial \Pi})$ is generated by composable
	words of Reeb chords of $\partial \Lambda$ in $V_{0}$, without
	endpoints on $\partial \Pi$. For $n=2$, the algebra is quasi-isomorphic
	to the algebra described above, and isomorphic to the algebra generated
	by composable words of Reeb chords of $\Lambda$ in $\partial W$ and
	$\partial \Lambda \setminus \partial \Pi$ in $V_{0,\partial \Pi}$,
	without an endpoint on $\Pi$.  
\end{lemma}

\subsection{The singular algebra in $\R^{3}$}	
In \cite[Section 7.1]{AE21}, Asplund--Ekholm show that in $\R^{3}$, their
invariant is isomorphic to the combinatorial version by An--Bae in \cite{AB20}.
We will use this below, and give a brief summary of the method. 

Let $V \subset \R^{3}$ be a Weinstein hypersurface and let $\pi: \R^{3} \to
\R^{2}$, $( x,y,z ) \to ( x,y )$ be the Lagrangian projection. We assume that
$V$ is in generic position so that $\pi|_{V_{0}}$ is an embedding and let
$\R^{2,\circ} := (\R^{2} \setminus \pi( V_{0} ) )\sqcup_{\partial V_{0}}
(\partial V_{0} \times (-\infty, 0 ] )$, i.e. the surface obtained by cutting
out the image of the subcritical part of the hypersurface and attaching the
negative end of the symplectization of $\partial V_{0}$ instead.  The image
$\widetilde{\Lambda} := \pi( \Lambda \setminus (\Lambda \cap V_{0}))$ of the
cores of the top handles of $V$ forms an exact Lagrangian, with negative end
$\partial \Lambda$ in $\partial_{-} \R^{2, \circ} \cong \partial V_{0}$. 

We can then consider the version of the Legendrian dg-algebra $CE^{*}(
\widetilde{\Lambda};\R \times \R^{2,\circ})$ as described by An--Bae
\cite{AB20}. It is generated by double points of $\widetilde{\Lambda}$ and Reeb
chords of $\partial \Lambda$, and its differential is given by counting disks
in $\R^{2,\circ}$ with boundary on $\widetilde{\Lambda}$, whose punctures
converge to either a double point or to a chord in the boundary, similarly to
in the classical Chekanov--Eliashberg algebra of smooth knots in $\R^{3}$.

The grading is easiest to compute in the front projection. Let $S$ be the
subset of $\Lambda$ consisting of all singularities along with the points
which are cusps in the front (i.e. the points with $y$-coordinate zero). We
choose a \emph{Maslov potential} $m:\Lambda \setminus S \to \Z$, such that $m$
is locally constant on $\Lambda \setminus S$, and increases with one when we
pass an up cusp and decreases with one when we pass a down cusp. The degree of
Reeb chord $a$ of $\Lambda$ with starting point $a^{-}$ and endpoint $a^{+}$
then has grading $ | a | = -m( a^{+} ) + m( a^{-} )$. The chords of
$c_{ij}^{p}$ of $\Lambda$ are graded by $| c_{ij}^{p} | = q + 1 + (- m( j ) +
m( i ))$, where $q$ is the number of times $c_{ij}^{p}$ passes through the
$z$-axis in the front (or equivalently, the $y$-axis in the Lagrangian
projection).

\begin{proposition}
\label{res:ce-in-contactization}
	There is an isomorphism of dg-algebras,
	\[
		CE^{*}(\widetilde{\Lambda};\R \times \R^{2,\circ}) 
		\cong CE^{*}( \Lambda;V_{0};\R^{4}  ) 
	\] 
\end{proposition}
\begin{proof}
	This is a special case of \cite[Lemma 7.1]{AE21}.	
\end{proof}
We give an example of how to compute the algebra. It is adapted from
\cite[Section 4.6.2]{AB20}.

\begin{example}
	Let $\Lambda$ be the singular Legendrian illustrated in
	\autoref{fig:pinched-figure-eight}. 
\begin{figure}[!htb]
    \centering
    \incfig{pinched-figure-eight}
    \caption{To the left is the Lagrangian projection of a singular Legendrian
    $\Lambda$. In the center are the disks contributing to $\partial a_1$. To
    the right is the front projection, with a choice of Maslov potential.}
    \label{fig:pinched-figure-eight}
\end{figure}
	It has six Reeb chords
	$a_1,a_2,b,c_1,c_2$, and $d$, and an infinite number of
	chords $t_{ij}^{p}$ of $\partial \Lambda$. We label the points of
	$\partial \Lambda$ as $1,2,3,4,$ going counter-clockwise from the top. 
	Giving the strands of the front
	projections Maslov potentials as shown in
	\autoref{fig:pinched-figure-eight}, we get the degrees
	\begin{equation*}
	\begin{aligned}[c]
		| a_1 | &= -1,\\
		| a_2 | &= -1,\\
		| b | &= -1,
	\end{aligned}
	\quad\quad\quad\quad
	\begin{aligned}[c]
		| c_1 | &= -1,\\
		| c_2 | &= 1,\\
		| d | &= 0,
	\end{aligned}
	\quad\quad\quad\quad
	\begin{aligned}[c]
		| t_{ij}^{p} | &= -2p+1.
	\end{aligned}
	\end{equation*}
	Counting the disks in the Lagrangian projection we get the differential,
	\begin{equation*}
	\begin{aligned}[c]
		\partial a_1  &= c_1c_2t_{12}^{0} + t_{12}^{0} + e_1,\\
		\partial a_2  &= dc_2c_1 + t_{23}^{0}c_1 + d + e_2,\\
		\partial b  &= dt_{34}^{0} + e_2,
	\end{aligned}
	\quad\quad\quad\quad
	\begin{aligned}[c]
		\partial  c_1  &= 0,  \\
		\partial  c_2  &= 0, \\
		\partial  d  &= 0,
	\end{aligned}
	\end{equation*}
	and as before
	\[
		\partial t_{ij}^{p} = \sum_{k,l} t_{kj}^{p-l}t_{ik}^{l}.
	\] 
	The disks contributing to $\partial a_{1}$ are illustrated in the
	\autoref{fig:pinched-figure-eight}. 
\end{example}
\subsection{Weinstein isotopy and invariance}
\label{ssec:isotopy}
There are two notions of isotopy of $V$. The first is \emph{Weinstein isotopy},
which is a smooth isotopy (not necessarily preserving the contact distribution) of 
$V$ through Weinstein hypersurfaces, see
\cite{Eli18}. For a generic Weinstein isotopy, the skeleton undergoes an
isotopy, together with handle-slides and introduction and contraction of
handles. See \autoref{fig:weinstein-reidemeister}. 
The second notion is that of \emph{Legendrian isotopy} of a Weinstein skeleton,
which is an ambient contact isotopy of this singular Legendrian.
In particular, a Legendrian isotopy of a Weinstein skeleton induces a Weinstein
isotopy of the corresponding thickening, preserving the handle
decomposition. Recall that any Weinstein hypersurface
has a naturally defined
skeleton that is independent of the contact form; see \cite[Section 2]{Eli18}.
The important difference between these notions
is that the quasi-isomorphism type of $CE^{*}( \Lambda;V_{0};W )$ is invariant
under Legendrian isotopy, but not Weinstein isotopy. 

\begin{theorem}[Asplund--Ekholm \cite{AE21}]
\label{res:invariance}
	The Chekanov--Eliashberg algebra $CE^{*}( \Lambda;V_{0};W )$ is up to
	quasi-isomorphism invariant under Legendrian isotopy of $\Lambda$.
\end{theorem}
\begin{proof}
	See \cite[Section 2.1]{AE21}. Legendrian isotopies of $V$ induce
	Legendrian isotopies of the link $\Sigma$ in the definition of $CE^{*}(
	\Lambda;V_{0};W )$, so the result follows from the isotopy invariance
	of the Chekanov--Eliashberg algebra for smooth Legendrians.  
\end{proof}

\begin{remark}
For $\partial W=\R^{3}$, An--Bae have introduced the stronger notion of
\emph{generalized stable tame isomorphism} and shown that two singular
Legendrian knots in $\R^{3}$ are Legendrian isotopic if and only if their
Chekanov--Eliashberg algebras are generalized stable tame isomorphic
\cite[Theorem 5.1]{AB20}.
\end{remark}

Below we will explore several concrete examples of how invariance fails for
Weinstein isotopies. Also see \cite[Examples 7.4 and 7.5]{AE21}. A
Weinstein isotopy will typically alter the handle decomposition of $W_{V}$.
Since the co-cores $C$ generate the wrapped Fukaya category
$\mathcal{W}(W_{V})$ \cite{CRGG17, GPS18} one expects the category
$\text{Perf}(CE^{*}( \Lambda;V_0;W ))$ of perfect complexes of dg-modules, i.e.
semi-free dg-modules over $CE^{*}( \Lambda;V_{0};W )$, to be invariant up to
quasi-equivalence under Weinstein isotopy. See also \cite[Remark 1.5]{AE21}.

When $\partial W=\R^{3}$, one can see that the algebra changes in a predicable way when
performing a handle contraction.

\begin{corollary}
\label{res:weinstein-isotopy-cor}
	Let $V \subset \R^{3}$ be a Weinstein hypersurface with skeleton
	$\Lambda$ and let $V' \subset \R^{3}$ be a Weinstein isotopic
	hypersurface with skeleton $\Lambda'$, obtained by performing a
	Legendrian isotopy and contracting a single handle $\Pi$ of $V$ as
	in \autoref{fig:weinstein-reidemeister}. Then there is a dg-algebra
	morphism
	\[
		CE^{*}( \Lambda';V_{0}';\R^{4} ) \to CE^{*}(
		\Lambda;V_{0};\R^{4} )
	\]
	which is a quasi-isomorphism onto $CE^{*}( \Lambda;V_{0};\R^{4}
	)[\Lambda\setminus\Pi,\Lambda\setminus\Pi]$.
\end{corollary}
\begin{proof}
	By \autoref{res:invariance}, Legendrian isotopy preserves the
	quasi-isomorphism type, so we need only consider the handle
	contraction. It follows from \autoref{res:ce-in-contactization} and
	\autoref{res:surgery-presentation} that $CE^{*}(
	\Lambda';V_{0}';\R^{4} )$ is isomorphic to the relative algebra
	$CE^{*}( \Lambda \setminus \Pi;V_{0,\partial \Pi};\R^{4} )$, and by
	\autoref{res:sing-cobordism-map} there is a dg-algebra morphism
	$CE^{*}( \Lambda \setminus \Pi;V_{0,\partial \Pi};\R^{4} ) \to CE^{*}(
	\Lambda;V_{0};\R^{4} )$ which is a quasi-isomorphism onto $CE^{*}(
	\Lambda;V_{0};\R^{4})[\Lambda\setminus\Pi,\Lambda\setminus\Pi]$. The result follows.
\end{proof}

\subsection{Reidemeister moves and the Ng resolution} 
In $\R^{3}$, there are versions of the Reidemeister moves for singular
Legendrians, extending the moves for smooth Legendrians introduced in
\cite{Swi92}.

\begin{proposition}[\cite{BI09}]
	Two singular Legendrians in $\R^{3}$ are Legendrian isotopic (after
	deforming the angles of the strands at the singular points) if and
	only if their front diagrams are related by a sequence of planar
	isotopies and singular Legendrian Reidemeister moves as shown in
	\autoref{fig:sing-legendrian-reidemeister}.
\end{proposition}
\begin{proof}
	See \cite[Section 4]{BI09}.
\end{proof}
\begin{figure}[!htb]
    \centering
    \incfig{sing-legendrian-reidemeister}
    \caption{The singular Legendrian Reidemeister moves, introduced in
	    \cite{BI09}. Reflections of these moves, with the crossings adjusted
	    accordingly, are also allowed. In the
	    moves involving singularities, any number of strands are allowed.}
    \label{fig:sing-legendrian-reidemeister}
\end{figure}
\begin{remark}
	The set of moves in \autoref{fig:sing-legendrian-reidemeister} is not
	minimal; IV is a special case of VI.
\end{remark}
There is an analogous result for Weinstein isotopies.
\begin{proposition}
\label{res:weinstein-lemma}
	Two singular Legendrians in $\R^{3}$ have Weinstein isotopic
	thickenings if and only
	if their front diagrams are related by a sequence of planar isotopies,
	singular Legendrian Reidemeister moves as shown in
	\autoref{fig:sing-legendrian-reidemeister}, and Weinstein handle
	introductions and contractions as shown in
	\autoref{fig:weinstein-reidemeister}.
\end{proposition}
\begin{remark}
	The set of moves in \autoref{fig:weinstein-reidemeister} is not minimal
	either; any one of the three types of moves can can be
	obtained from any other by composing with Legendrian isotopies. 
\end{remark}
\begin{figure}[!htb]
    \centering
    \incfig{weinstein-reidemeister}
    \caption{Introduction and contraction of Weinstein handles. The
	    singularities can have any valency, including zero.}
    \label{fig:weinstein-reidemeister}
\end{figure}
\begin{proof}
	We first prove that the moves applied to the skeleton can be realized
	by a Weinstein isotopy of the thickening. Consider a small deformation
	of a singular Legendrian $\Lambda^0$ with
	thickening $V^0$ by one of the moves in \autoref{fig:weinstein-reidemeister} 
	to produce a singular
	Legendrian $\Lambda^1$. A sufficiently small thickening may be assumed to be
	Weinstein for the restriction $\lambda^0$ of the standard contact form $dz-ydx$
	(see \cite[Lemma 12.1]{CE12}). One
	can readily find a smooth deformation $\lambda^1=\lambda^0+dh_t$ of the
	Liouville form of the thickening $V^{0}$ of the original skeleton that induces a
	family of Weinstein structures that connects $\Lambda^0$ to $\Lambda^1$. We can
	now deform $V^0$ to $V^1$ simply by adding the function $h_t$ to its
	$z$-coordinate.

	We now conversely show that a generic Weinstein isotopy can be assumed to
	induce a deformation of the skeleton that can be realized by the moves.
	For a Weinstein isotopy $(V^t,\lambda^t,\phi^t)$ for which $\phi^t$ remains
	Morse, one can readily deform the skeleton so that it is induced by an
	ambient contact isotopy. One can perform an explicit deformation near
	the 0-skeleton to ensure that there is an ambient contact isotopy that
	induces that deformation. One can then deform the remaining part of the
	skeleton by using the standard result that a Legendrian isotopy is
	induced by an ambient contact isotopy (see \cite{Gei08}). Such
	deformations are induced by the Reidemeister moves.

	A generic Weinstein isotopy consists of generic times when $\phi^t$
	remains Morse, together with a finite number of singular moments that
	are handle-slides, or births or deaths. Note that both these singular
	moments can be realized by the moves from Figure 12 together with a
	Legendrian isotopy. The statement follows from this.
\end{proof}
Since one typically computes the Chekanov--Eliashberg algebra in the Lagrangian
projection but constructs isotopies in the front projection, it is useful to be able to
go back and forth between them. This is done using the so called
Ng resolution, which was introduced by Ng for smooth Legendrians in
\cite{Ng03} and generalized to singular Legendrians by An--Bae--Su in
\cite{ABS22}.
\begin{definition}
	Let $\Lambda \subset \R^{3}$ be a singular Legendrian. The \emph{Ng
	resolution} of $\Lambda$ is the singular Legendrian whose Lagrangian
	projection (up to an isotopy of $\R^{2}$ which corrects
	the areas bounded by the different regions in the projections and the
	angles of the strands at the singular points)
	is obtained  by performing the operations illustrated in
	\autoref{fig:legendrian-to-lagrangian} to the front diagram of
	$\Lambda$.
	\begin{figure}[!htb]
	    \centering
	    \incfig{legendrian-to-lagrangian}
	    \caption{Obtaining the Lagrangian projection of an isotopic
		    Legendrian from the front projection. In the operation to the 
		    upper right, the singularity is allowed to have any
	    	valency, and the operation consists of performing a twist of the
    		strands to left of the singularity.}
	    \label{fig:legendrian-to-lagrangian}
	\end{figure}
\end{definition}
\begin{proposition}[\cite{ABS22}]
	The Ng resolution is well defined in the sense that, after a planar
	isotopy, the Lagrangian projection indeed lifts to a Legendrian.
	Moreover, every singular Legendrian in $\R^{3}$ is Legendrian isotopic
	to its Ng resolution after deforming the angles at the singularities.
\end{proposition}
\begin{proof}
	This is \cite[Lemma 2.2.16]{ABS22}.
\end{proof}
