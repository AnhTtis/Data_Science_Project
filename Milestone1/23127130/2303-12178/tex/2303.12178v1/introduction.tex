\section{Introduction}
The Chekanov--Eliashberg dg-algebra is an invariant of Legendrian submanifolds
of contact manifolds. In recent years, its definition has been extended to
singular Legendrian skeleta of Weinstein hypersurfaces. This was first done
combinatorially by An--Bae in \cite{AB20} for singular Legendrians in $\R^{3}$,
and given a geometric-analytic generalization by Asplund--Ekholm in \cite{AE21}
to skeleta of Weinstein hypersurfaces in the boundary of a general Weinstein
manifold.  For a singular Legendrian $\Lambda$, which is the skeleton of some
Weinstein hypersurface $V$, living in the ideal contact boundary of some
Weinstein manifold $W$, we denote this algebra by $CE^{*}( \Lambda;V_{0};W)$,
where $V_{0}$ is the subcritical part of $V$. Through the surgery formula
\cite[Theorem 1.1]{AE21}, this algebra corresponds to the partially wrapped
Floer cohomology of the Weinstein sector obtained by stopping the manifold at
the hypersurface $V$. When $\Lambda$ is smooth, $CE^{*}( \Lambda;V_{0};W )$ is
quasi-isomorphic to the Chekanov--Eliashberg algebra with based loop space
coefficients \cite[Theorem 1.2]{AE21}. In $\R^{3}$, this dg-algebra is also
quasi-isomorphic to the more classical Chekanov--Eliashberg algebra with
non-central coefficients in $\Z[t,t^{-1}]$.

In this paper, we describe a way of constructing simpler models of the
Chekanov--Eliashberg algebra for singular Legendrians in $\R^{3}$, and obtain
many new examples in which the cohomology can be computed.  These include the
infinite families of the $\theta$-Legendrian and its generalizations, and the
standard $A_{n}$-Legendrians. There is also an infinite class of singular
Legendrians for which this method produces finite dimensional models.

The algebra $CE^{*}( \Lambda;V_{0};W )$ contains the Chekanov--Eliashberg
algebra $CE^{*}( \partial \Lambda;V_{0} )$ of the attaching spheres $\partial
\Lambda \subset \partial V_{0}$ as a dg-subalgebra. We compute the cohomology
and minimal model of this subalgebra when $\dim V=2$, where it is also known as
the internal algebra of Ekholm--Ng. This finishes the proof of the
Bourgeois--Ekholm--Eliashberg surgery formula in dimension two, which
previously was not considered in the literature. In contrast to the situation
in higher dimensions, the surgery map for surfaces fails to be an
$A_{\infty}$-isomorphism while it still is a quasi-isomorphism.

\subsection*{Outline}
In the remaining part of Section 1, we state our main results. Section 2 and
Section 3 are mainly a review of basic definitions and results about
Chekanov--Eliashberg algebras and singular Legendrians. The exception is
Section 2.2, which deals with the surgery formula for Weinstein surfaces. In
Section 4, we prove our main results.  Finally, in Section 5 we give examples
of how one can apply these results to compute the cohomology of the
Chekanov--Eliashberg algebra. 

\subsection{Chekanov--Eliashberg algebras in the boundary of a Weinstein
surface}
Let $V$ be a compact Weinstein surface and let $\partial \Lambda \subset
\partial V$ be an embedding which is a $0$-dimensional Legendrian, partitioned
into two-point spheres and one-point stops. The Reeb chords are
counter-clockwise arcs with endpoints on $\partial \Lambda$.  We fix a base
point in $\partial V \setminus \partial \Lambda$ for each component of
$\partial V$ and denote the chords as $c_{ij}^{p}$, where $i$ is the starting
point, $j$ is the endpoint, and $p$ is the number of times the chord passes
through the base point. See \autoref{fig:internal-algebra}.  
\begin{figure}[!htb]
    \centering
    \incfig{internal-algebra}
    \caption{The Chekanov--Eliashberg algebra in the boundary of a disk.}
    \label{fig:internal-algebra}
\end{figure}
The Chekanov--Eliashberg algebra $CE^{*}( \partial \Lambda;V )$ is then an
infinitely generated semi-free unital dg-algebra over the ground field
$\bf{k}$, whose underlying associate algebra is the path algebra of the quiver
with one vertex for each sphere and stop of $\partial \Lambda$, and the Reeb
chords as arrows. The results in this paper can be shown to hold over any field
$\bf{k}$, but in order to avoid worrying about signs, our computations will be
carried out in the case $\textbf{k}=\Z_2$.  However, the results remain true in
all characteristics. The differential is given by $\partial = \partial_{0} +
\partial_{-1}$, where
\[
	\partial_{0}( c_{ij}^{p} )  = \sum \pm c_{kj}^{p}c_{ik}^{p-l}
\]
with the sum taken over all $k$ and $l$ for which the chords on the right hand
side exist, and $\partial_{-1}( c_{ij}^{p} ) = e_{k}$ if $c_{ij}^{p}$ is a
chord with $i=j$, $p=1$, and which lives in the boundary of a disk, where
$e_{k}$ is the idempotent corresponding to the sphere or stop containing $i$,
and $\partial_{-1}(c_{ij}^{p} )=0$ otherwise. The differential comes from a
count of pseudo-holomorphic curves as usual, see \autoref{res:zero-dim-diff}.

We attach Weinstein handles at the spheres and half-handles at the stops to
obtain a Weinstein sector $V_{\partial \Lambda}$. Let $C \subset V_{\partial
\Lambda}$ be the co-cores of the handles and half-handles, and let $CW^{*}(
C;V_{\partial \Lambda} )$ be the partially wrapped Floer cohomology of $C$. The
following theorem combines the surgery formula for smooth Legendrians by
Bourgeois--Ekholm--Eliashberg and the surgery formula for singular Legendrians
by Asplund--Ekholm.  In this paper we prove the case when $\dim V_{\partial
\Lambda}=2$, which previously was not treated.  
\begin{theorem}[\autoref{res:surgery-map-surface}]
	There is an $A_{\infty}$-quasi-isomorphism
	\[
		CW^{*}( C;V_{\partial \Lambda} ) \isomto CE^{*}( \partial \Lambda;V ).
	\] 
\end{theorem}
A proof of the surgery formula in higher dimensions has appeared in
\cite[Appendix B]{EL17}\cite{Ekh19}, and there the map is an
$A_{\infty}$-isomorphism.  In the two-dimensional case, all Reeb chords are
degenerate, which prevents the map from being a chain level isomorphism. We
prove that it is a quasi-isomorphism by explicitly computing the cohomology of
both sides.  As noted in \cite[Example 7.3]{AE21}, when $\partial \Lambda$
consists entirely of one-point stops, the two-dimensional case follows from
computations done in \cite[Corollary 11]{EL19}.

Using the surgery formula, one can then also describe the higher operations of
the minimal model of $CE^{*}( \partial \Lambda;V)$. We call a word in $CE^{*}(
\partial \Lambda;V)$ \emph{unconcatable} if there are no adjacent Reeb chords
in the word whose respective starting point and endpoint coincide. We call a
chord \emph{short} if it is not the concatenation of any other two chords.
Finally, we call a word $c_k \ldots c_1$ of Reeb chords (of which none is an
idempotent) a \emph{disk sequence} if all adjacent chords in the word have
respective starting points and endpoints which coincide, and if when
concatenating the chords of the word into a single chord, one obtains a chord
of the form $c_{ii}^{1}$ which bounds a disk in $V$. For two short chords $c_2$
and  $c_1$ we define their concatenation $c_2 * c_1$ as the geometric
concatenation to a Reeb chord, if it makes sense, and as $0$ otherwise.

\begin{theorem}
\label{res:zero-dim-leg-min-mod}
	Let $\partial \Lambda$ be a $0$-dimensional Legendrian in the boundary
	of a Weinstein surface $V$ and suppose that no co-core in $C$ is
	null-homotopic. As a vector space, $H^{*}CE( \partial \Lambda;V )$ has
	a basis consisting of the unconcatable words of short chords. The
	$A_{\infty}$-operations $\mu_{k}$ of the minimal model structure on
	$H^{*}CE( \partial \Lambda;V )$ act as follows. The multiplication is
	associative and unital, and determined by the relations 
	\begin{align*}
		\mu_{2}( c_2 \otimes c_1 ) = \partial_{-1}( c_{2} * c_{1} )
	\end{align*}
	for any short chords $c_2$ and  $c_{1}$, neither of which is an
	idempotent. The higher operations are for $k \ge 3$ given by
	\begin{align*}
		\mu_{k}( c_{k}\otimes\ldots\otimes c_{1}c ) = c,\\
		\mu_{k}( cc_{k}\otimes \ldots \otimes c_{1} ) = c, 
	\end{align*}
	on all tensor products of words of the form $c_{k}\otimes\ldots\otimes
	c_{1}c$ and $cc_{k}\otimes \ldots \otimes c_{1}$ such that $c_{k}\ldots
	c_{1}$ is a disk sequence, and  $c_{1}c$ and $cc_{k}$ are unconcatable,
	and vanish on all other words. The chord $c$ is also allowed to be an
	idempotent.

	On the other hand, if $\partial \Lambda$ contains a $0$-sphere
	$\partial \Pi$ for which $C_{\Pi}$ is null-homotopic, then there is a
	quasi-isomorphism $CE^{*}( \partial \Lambda \setminus \partial
	\Pi;V_{\partial \Pi} ) \isomto CE^{*}( \partial \Lambda;V )$.
\end{theorem}
\subsection{Chekanov--Eliashberg algebras for singular Legendrians}
We here state our main result. Let $V$ be a Weinstein hypersurface in $\R^{3}$
with singular Legendrian skeleton $\Lambda$.  Recall that a Weinstein
hypersurface is a hypersurface $V \subset (\R^3,\ker (dz-ydx))$ such that there
exists a contact form $e^f(dz-ydx)$ that restricts to a Liouville form on $V$
that is compatible with a Weinstein structure. In \cite[Section 2]{Eli18}, it
was shown that the skeleton of a Weinstein hypersurface is independent on the
choice of contact form (as long as the restriction is Liouville).  We assume
that the skeleton is smooth away from the subcritical part $V_{0} \subset V$,
where the latter is a union of balls. Further we assume that the skeleton
consists of the cone of a finite number of points in the boundary of $V_{0}$.
We let $\partial \Lambda \subset \partial V_{0}$ be the attaching spheres in
the handle decomposition of $V$.  The Chekanov--Eliashberg algebra $CE^{*}(
\Lambda;V_{0};\R^{4})$ is, as an associative algebra, isomorphic to the path
algebra over $\bf{k}$ of the quiver with one idempotent for each top handle of
$V$, one arrow for each Reeb chord of $\Lambda$ in $\R$, and one arrow for each
Reeb chord of $\partial \Lambda$ in $\partial V_0$. Note that the algebra is
infinitely generated. The differential is defined by counting
pseudo-holomorphic curves in the symplectization. In $\R^{3}$, it can be
explicitly computed by counting admissible disks in the Lagrangian projection
\cite[Section 7.1]{AE21}, similarly to in the classical definition due to
Chekanov \cite{Che02}.

Our main theorem states that for each singular Legendrian $\Lambda$, there is a
so-called \emph{bordered} or open Legendrian $\Lambda^{\bullet}$ in the sense
of \cite{Siv11, ABS22, ABS19, ABK22}, with Chekanov--Eliashberg algebra
quasi-isomorphic to that of $\Lambda$.  
\begin{figure}[!htb]
    \centering
    \incfig{intro-resolution}
    \caption{To the left is the front projection of a singular Legendrian
	    $\Lambda$ and to the right is
    its bordered Legendrian resolution $\Lambda^{\bullet}$.}
    \label{fig:intro-resolution}
\end{figure}
We call $\Lambda^{\bullet}$ the \emph{resolution} of $\Lambda$; see
\autoref{def:resolution}. We need to warn the reader $\Lambda^{\bullet}$ is not
canonically defined up to compactly supported Legendrian isotopy. Despite this,
however, we can associate an invariant $CE^{*}( \Lambda^{\bullet};\R^{4} )$ to
$\Lambda^{\bullet}$ which is quasi-isomorphic to the Chekanov--Eliashberg
algebra of $\Lambda$.  In the front projection, $\Lambda^{\bullet}$ is
constructed by first moving $\Lambda$ by an isotopy into a position such that
the singularities all have the same $x$-coordinate, and the rest of the
Legendrian lies to the right of the singularities. We then replace each
singularity with a negative half-twist as in \autoref{fig:intro-resolution}.
It is immaterial in which order the twists are performed, and if the ends of
the twists go above or below the Legendrian.  
\begin{theorem}[\autoref{res:resolution}]
\label{res:intro-res}
	Let $\Lambda \subset \R^{3}$ be a singular Legendrian. There is a
	quasi-isomorphism
	 \[
		 CE^{*}( \Lambda;V_{0};\R^{4} ) \cong 
		 CE^{*}( \Lambda^{\bullet};\R^{4} ).
	\] 
\end{theorem}
 
The Chekanov--Eliashberg algebra of $\Lambda^{\bullet}$ has one generator for
each crossing and right cusp in the front, and the differential is given by
counting disks, just as for ordinary compact Legendrians. In particular,
$CE^{*}( \Lambda^{\bullet};\R^{4} )$ is finitely generated.

If $\Lambda$ only has one singularity, and the rest of $\Lambda$ lies to the
left of the singularity in the front projection, there is a related
construction which gives an even simpler bordered Legendrian $\Lambda^{\circ}$,
which we call the \emph{opening} of $\Lambda$; see \autoref{def:opening}.  In
the front, $\Lambda^{\circ}$ is obtained from $\Lambda$ by removing the
singularity and separating the strands, as in \autoref{fig:intro-opening}.
\begin{theorem}[\autoref{res:opening}]
\label{res:intro-opening}
	Let $\Lambda \subset \R^{3}$ be a singular Legendrian with only one
	singularity, such that the rest of $\Lambda$ lies to the left of the
	singularity in the front projection. Then there is a quasi-isomorphism
	 \[
		 CE^{*}( \Lambda;V_{0};\R^{4} ) \cong 
		 CE^{*}( \Lambda^{\circ};\R^{4} ).
	\] 
\end{theorem}
\begin{figure}[!htb]
    \centering
    \incfig{intro-opening}
    \caption{To the left is the front projection of a singular Legendrian
	 $\Lambda$ and to the right is its bordered Legendrian opening
    	$\Lambda^{\circ}$.}
    \label{fig:intro-opening}
\end{figure}
In Section 5 we explore a number of examples where the Chekanov--Eliashberg
algebra of $\Lambda^{\circ}$ is simple enough to allow one to compute the
minimal $A_{\infty}$-model. 

\subsection*{Acknowledgments}
This paper is based on the author's bachelor and master theses, which were
supervised by Georgios Dimitroglou Rizell. The author would like to express his
gratitude to him for introducing him to this research area, many helpful
discussions, and for his careful reading of earlier drafts of this paper. The
author is supported by the grant KAW 2021.0300 from the Knut and Alice
Wallenberg Foundation.
