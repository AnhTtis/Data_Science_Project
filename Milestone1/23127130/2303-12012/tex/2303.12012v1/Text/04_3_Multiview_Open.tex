\begin{table}[htbp]
    \small
    \centering
    \begin{tabular}{c|c|c|c|c|c}
        \hline
        & CD ($\times 10^{-3}$) $\downarrow$ & Ours & NeuS & IDR & HFS \\
        \hline
        \hline
        \multirow{9}{*}{D3D} 
        &long slv upper & \textbf{4.483} & $6.864$ & $11.494$ & $9.695$\\
        &short slv upper & \textbf{4.517} & $6.048 $ & $9.043$ & $7.800$\\
        &no slv upper & \textbf{3.418} & $4.856 $ & $17.710$ & $8.576$\\
        &long slv dress & \textbf{4.843} & $6.135$ & $9.203$ & $8.235$\\
        &short slv dress & \textbf{4.276} & $7.951$ & $8.506$ & $7.705$\\
        &no slv dress & \textbf{3.706} & $5.406$ & $6.785$ & $7.565$\\
        &pants & \textbf{5.391} & $11.847 $ & $10.880$ & $16.205$\\
        &dress & \textbf{3.889} & $5.673 $ & $6.983$ & $11.644$\\
        \cline{2-6}
        &average & \textbf{4.315} & $6.847$ & $10.075$ & $9.678$\\
        \hline
        \hline
        \multirow{6}{*}{MGN}
        &LongCoat & \textbf{7.601} & $8.038$ & $12.058$ & $10.398$\\
        &TShirtNoCoat & \textbf{8.481} & $9.910$ & $15.709$ & $13.128$\\
        &ShirtNoCoat & \textbf{5.281} & $8.084$ & $9.509$ & $11.299$\\
        &ShortPants & \textbf{15.324} & $15.480$ & $16.329$ & $18.332$\\
        &Pants & \textbf{9.191} & $12.188$ & $19.931$ & $19.414$\\
        \cline{2-6}
        &average & \textbf{9.176} & $10.740$ & $14.707$ & $14.514$\\
        \hline
    \end{tabular}
    \vspace{-0.2em}
    \caption{Quantitative evaluation on \textit{Deep Fashion 3D
    Dataset}~(D3D)~\cite{zhu2020deep}~with chamfer distance averaged over five examples per category, and \MGN~(MGN)~\cite{bhatnagar2019mgn} with chamfer distance averaged on two examples per category.}
    \vspace{-1em}
    \label{table:comparison_open_d3d}
\end{table}
% \begin{table}[htbp]
    \centering
    \begin{tabular}{|c|c|c|c|}
        \hline
        CD ($\times 10^{-3}$) & Ours & NeuS~\cite{wang2021neus} & IDR~\cite{yariv2020idr} \\
        \hline
        LongCoat & \textbf{5.561} & $7.361$ & $10.348$\\
        TShirtNoCoat & \textbf{TODO} & $TODO$ & $14.784$\\
        ShirtNoCoat & \textbf{4.843} & $6.135$ & $10.091$\\
        ShortPants & \textbf{TODO} & $TODO$ & $14.203$\\
        Pants & \textbf{TODO} & $TODO$ & $15.574$\\
        \hline
    \end{tabular}
    \caption{Quantitative evaluation on \MGN.}
    \label{table:comparison_open_mgn}
\end{table}
\vspace{-0.5em}
\subsection{Multiview Reconstruction on Open Surfaces}
\vspace{-0.5em}
% We investigate if our method is applicable to multi-view reconstruction of open surfaces. For this experiment, we do not condition our model and train one model per object.

We conduct this experiment on eight categories from Deep Fashion 3D~\cite{zhu2020deep} and five categories from the MGN dataset~\cite{bhatnagar2019mgn}. {We compare our approach with two state-of-the-art volume rendering based methods -- NeuS~\cite{wang2021neus} and HFS~\cite{wang2022hfneus}, and a surface rendering based method -- IDR~\cite{yariv2020idr}.}

We report the Chamfer Distance averaged on five examples for each category from \DFD~\cite{zhu2020deep} and report the Chamfer Distance averaged on two examples for each category from \MGN~in Table~\ref{table:comparison_open_d3d}. \modelName~generally provides lower numerical errors compared with the state-of-the-arts.
We show qualitative results in Fig.~\ref{fig:comparison_open}. {\modelName~also provides lower numerical errors in F-score. Please refer to the supplemental for the comparisons.}

% \xiaoxu{
In most cases, NeuS~\cite{wang2021neus} and IDR~\cite{yariv2020idr} are able to reconstruct the geometry with thick, watertight surfaces. While, for the pants in Figure~\ref{fig:comparison_open}, NeuS fails to recover the shape of the waist. \modelName{} is able to reconstruct high-fidelity open surfaces with consistent normals, including the thin straps of the camisoles and dresses.
% }
% \brandon{Add some discussions here.}