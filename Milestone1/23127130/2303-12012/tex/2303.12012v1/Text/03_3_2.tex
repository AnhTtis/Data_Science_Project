\vspace{-1em}
\paragraph{Construction of Opaque Density Function.}
According to NeuS\cite{wang2021neus}, the weight function $\wt$ should have two properties: unbiased and occlusion-aware. Similarly, we define unbiased rendering weight $\wt$ with Equation~\ref{equ:define_wt_unbiased}~ and define an occlusion-aware weight function based on the opaque density $\rho(t)$ with Equation~\ref{equ:define_wt_occlusion_aware}.

\vspace{-1em}
\begin{empheq}
    [left=\empheqlbrace]{align}
    \wt & = \frac{
            \phi_{s}(-Sign(\mathbf{v}\cdot\mathbf{n})f(\pt(t)))
    }{
    \int_{-\infty}^{+\infty}\phi_{s}(-Sign(\mathbf{v}\cdot\mathbf{n})f(\pt(t)))
    }
    \label{equ:define_wt_unbiased}\\
    \wt &= \exp(-\int_{0}^{t}\rho(u)du)\rho(t)
    \label{equ:define_wt_occlusion_aware}
\end{empheq}

% % \xiaoxu{
% Following the rules of unbiasedness\cite{wang2021neus}, we define a weight function $w(t)$ on the ray based on the SDF of the scene:
% % }

% \begin{equation}
%     w(t) = 
%     \frac{
%         \phi_{s}(f(\pt(t))\cdot \gamma(\pt(t)))
%     }{
%         \int_{-\infty}^{+\infty}\phi_{s}(f(\pt(t))\cdot \gamma(\pt(t)))
%     }
%     \label{equ:define_wt_unbiased}
% \end{equation}

% Equation~\ref{equ:define_wt_unbiased} is naturally unbiased,
% % \weikai{Why is naturally unbiased? We should provide an proof here or in the supplemental.}
% but not occlusion aware. Accordingly, we further define an opaque density function $\rho(t)$, which is the counterpart of the volume density in the standard volume rendering formulation~\cite{mildenhall2020nerf}, and we compute the weight by
% \begin{equation}
%     w(t) = T(t)\rho(t)
%     \label{equ:define_wt_occlusion_aware}
% \end{equation}
% where $T(t)=\exp(-\int_{0}^{t}\rho(u)du)$ is the accumulated transmittance along the ray.

Solving Equation~\ref{equ:define_wt_unbiased}~and Equation~\ref{equ:define_wt_occlusion_aware}, we get
\vspace{-0.5em}
\begin{equation}
    \rho(t) = \frac{
                -\frac{
                    d\Phi_{s}
                }{
                    dt
                }(-Sign(\mathbf{v}\cdot\mathbf{n})f(\pt(t)))
            }{
                \Phi_{s}(-Sign(\mathbf{v}\cdot\mathbf{n})f(\pt(t))))
            }
\vspace{-0.5em}
\end{equation}
\noindent Please checkout the supplemental for the derivation.