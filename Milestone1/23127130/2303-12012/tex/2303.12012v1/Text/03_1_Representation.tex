\begin{figure}[t!]
\centering
    \includegraphics[width=0.9\linewidth]{Figure/surface_representation/SDF+V.pdf}
\caption{(a) is the signed distance function (SDF); (b) is the validity probability function $\vldty$; (c) is the watertight surface extracted from (a) SDF; (d) is the open surface extracted from (a) SDF and (b) validity probability. In our mesh extraction process, we set the SDF of the 3D query points with {low validity (here $\vldty < 0.5$)} to \textit{NAN} and extract the open surface with the Marching Cubes algorithm.
%\weikai{1) Better to use a more interesting shape instead of a hand posture for illustrating our key idea. Hand posture sometimes has special meanings (like copy/paste icons=D). 2) For signed distance function (b), why there is a narrow band around the surface (it seems to be as thick as that of validity probability function in (c))? The narrow band has a special meaning, as it keeps the region valid instead of null. So it is better to visualize the surface boundary in (b) as thin curve, while widening the narrow band in (c) a little bit to distinguish them.}
}
\vspace{-1.5em}
\label{fig:surface_representation}
\end{figure}

\begin{figure*}[t]
    \centering
    \includegraphics[width=\textwidth]{figures/Overview_STG.pdf}
    %\includegraphics[width=\textwidth]{figures/pipeline2.pdf}
    \vspace{-20pt}
    \caption{\textbf{Spatio-temporal grounding approach.} 
    % We incorporate both spatial and temporal information in the training process including three modalities. 
    (a)~We want to select frames with possible groundable objects and tasks. To this end, projected word features are matched with respective frame features. (b)~Sinkhorn-knopp optimal transport is then leveraged to ensure the variety of our selected frames. (c)~Based on the selected frames, a global representation is learned to allow for temporal localization as well as (d)~a local representation to ground the action description to the spatial region. 
    %Local contrastive loss on video spatio-temporal and text features to learn multimodal interactions between finer-grained features. Global pairwise contrastive loss on video and text features to pull the features close across modalities in a high-level semantic space. 
    }
    \label{fig:pipeline}
    %\vspace{-10pt}
\end{figure*}
Our representation is able to reconstruct 3D surfaces with arbitrary topologies without 3D ground-truth data for training. As shown in Figure~\ref{fig:surface_representation}, by taking the SDF (Figure~\ref{fig:surface_representation} (a)) and validity probability (Figure~\ref{fig:surface_representation} (b)) into consideration together, we acquire additional information that the bottom line in Figure~\ref{fig:surface_representation} (a) is invalid.
{We discard parts of the reconstructed surface according to the validity score and extract an open surface as shown in Figure~\ref{fig:surface_representation} (d) with the Marching Cubes algorithm~\cite{marching_cubes}. }
% By setting $\vldty(\textbf{p})<0.5$ as \nan, we can extract an open surface as shown in Figure~\ref{fig:surface_representation} (d) with the Marching Cubes algorithm~\cite{marching_cubes}. 
% Hence, the validity probability function acts as a surface eliminator. We describe the details of our method in this section.
%\brandon{I think Figure~\ref{fig:surface_representation}(b) is very confusing. I understand you want to demonstrate the effects of validity, but revierers may not understand this. We claim we deal with arbitrary surfaces, but how to we get a closed SDF field? If I were the reviewer, I would have this question.} \xiaoxu{What I want to demonstrate is that SDF $+$ Validity is able to represent arbitrary shapes. ``How to we get a closed SDF field" is another question that we will answer in the following sections.}