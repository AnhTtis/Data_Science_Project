% \vspace{-0.5em}
\subsection{Ablation Studies}
\vspace{-0.5em}
\begin{figure}[htb]
    \centering
        \includegraphics[width=1.0\linewidth]{Figure/ablation_validity_regularization/image.pdf}
    \vspace{-1.5em}
    \caption{Ablation study on the regularizations about validity.
    }
\vspace{-2em}
\label{fig:ablation_validity}
\end{figure}

\paragraph{Regularizations on validity.}
We conduct an ablation study on the regularizations about validity, i.e. $\mathcal{L}_{bce}$ and $\mathcal{L}_{sparse}$.
As shown in Figure~\ref{fig:ablation_validity} (c), by setting $\mathcal{L}_{bce}=0$, the renderer tends to generate rendering probability between 0 and 1, thus resulting in noisy faces in the output mesh; as shown in Figure~\ref{fig:ablation_validity} (d), by setting $\mathcal{L}_{sparse}=0$, the renderer will keep the redundant surfaces, instead of learning a validity space as sparse as possible.


% Xiaoxu: sigmoid_factor doesn't affect that much
% \paragraph{Sigmoid factor (enforce consistency between meshing and rendering)}
% Goal: Consistency between the result from regression and from classification
% \input{Figure/ablation_sigmoid_factor/ablation_sigmoid_factor}

\begin{figure}[htb]
    \begin{minipage}[t]{.09\textwidth}
        \centering
        \subfloat[GT]{\includegraphics[width=\textwidth]{Figure/ablation_n_views/320/gt.png}}
    \end{minipage}
    \begin{minipage}[t]{.09\textwidth}
        \centering
        \subfloat[64 views\\CD = 0.00568]{\includegraphics[width=\textwidth]{Figure/ablation_n_views/320/66views.png}}
    \end{minipage}
    \begin{minipage}[t]{.09\textwidth}
        \centering
        \subfloat[32 views\\CD = 0.00632]{\includegraphics[width=\textwidth]{Figure/ablation_n_views/320/34views.png}}
    \end{minipage}
    \begin{minipage}[t]{.09\textwidth}
        \centering
        \subfloat[16 views\\CD = 0.00763]{\includegraphics[width=\textwidth]{Figure/ablation_n_views/320/18views.png}}
    \end{minipage}
    \begin{minipage}[t]{.09\textwidth}
        \centering
        \subfloat[8 views\\CD = 0.01326]{\includegraphics[width=\textwidth]{Figure/ablation_n_views/320/10views.png}}
    \end{minipage}
    \vspace{-1em}
    \caption{Ablation study on multi-view reconstruction with different number of views.}
\vspace{-1em}
\label{fig:ablation_n_views}
\end{figure}

\paragraph{Reconstruct with different number of views.}
We additionally show results on reconstruction with different number of views. As shown in Figure~\ref{fig:ablation_n_views}, our method is able to reconstruct open surfaces even with sparse viewpoints. The reconstruction quality improves with the increase of views, quantitively and qualitatively. 



% Not that important
% \paragraph{Reconstructing thin structures (w. vs. w/o mask regularization)}
% As described in Section~\ref{sec:method_training}, if a pixel is sensitive to dilation or erosion, we consider this pixel as part of the sensitive region. We use higher mask weight for the sensitive regions.
% \begin{figure}[htb]
    \begin{minipage}[t]{.2\textwidth}
        \centering
        \includegraphics[width=\textwidth]{Figure/placeholder.pdf}
        % \subcaption{Image 1.}
    \end{minipage}
    \hfill
    \begin{minipage}[t]{.2\textwidth}
        \centering
        \includegraphics[width=\textwidth]{Figure/placeholder.pdf}
        % \subcaption{Image 2.}
    \end{minipage}  
    \label{fig:ablation_mask}
    \caption{Ablation study on mask regularization.}
\end{figure}
