\begin{figure}[t!]
\centering
    \includegraphics[width=0.9\linewidth]{Figure/rendering_weights/image_render_weights.pdf}
\vspace{-0.5em}
\caption{Illustration of the rendering with validity probability.}
\vspace{-1em}
\label{fig:render_weights}
\end{figure}


\paragraph{Rendering with Validity Probability.}
To render both closed and open surfaces, we multiply the validity probability of the 3D query points to their opacity value in the rendering process. The discrete opacity value $\beta_{i}$ of the $i$-th sampled point is

\vspace{-0.5em}
\begin{equation}
\beta_{i} = \alpha_{i}\cdot \vldty(\pt(t_{i}))
\end{equation}

We show a 2D illustration of rendering two objects with open boundaries in Figure~\ref{fig:render_weights}.
% One example is shown in Figure~\ref{fig:render_weights}. Object 1 and Object 2 are two open-rectangles.
% Ray 1 has four intersections with the objects and Ray 2 has two intersections due to the existence of the gaps (marked with dotted lines).
Ray 2 only has two intersections with the objects due to the existences of open gaps (marked as dotted lines). 
Ray 1 and Ray 2 share the same SDF $f(\pt(t_{i}))$ and discrete opacity value $\alpha_{i}$.
However, according to the validity branch $\vldty(\pt(t_{i}))$, Ray 1 has four valid regions while Ray 2 only has two. By considering the validity probabilities, the discrete opacity value $\beta_{i}$ of the gaps in Ray 2 are set to zero, avoiding generating false surfaces in reconstruction.

Therefore, the final rendered pixel color of a surface is
\begin{equation}
    \imagePred(\mathbf{o}, \mathbf{v})=\sum_{i=1}^{n}\Pi_{j=1}^{i-1}(1 - \beta_{j})\beta_{i}c_{i}
    \vspace{-1em}
    \label{equ:discrete_color_isat}
\end{equation}