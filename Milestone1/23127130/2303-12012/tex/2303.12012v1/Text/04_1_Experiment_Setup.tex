\subsection{Experiment Setup}

\paragraph{Tasks and Datasets.}
We validate \modelName~using three types of experiments. 
We first conduct multi-view reconstruction for real-world watertight objects to ensure that \modelName~achieves comparable reconstruction quality on watertight surfaces. We conduct this experiment on 10 scenes from the \textit{DTU Dataset}~\cite{dtu}. Each scene contains $49$ or $64$ RGB images and masks with a resolution of $1600\times 1200$.
Second, we reconstruct open surfaces from multi-view images. We run this experiment on eight categories from the \DFD~\cite{zhu2020deep} and five categories from the \MGN~\cite{bhatnagar2019mgn}, which contain clothes with a wide variety of materials, appearance, and geometry, including challenging cases for reconstruction algorithms, such as camisoles.
Finally, we construct an autoencoder, which takes a single image as the input and provides validation on the challenging task of single-view reconstruction on open surfaces. We conduct this experiment on the \textit{dress} category from the \DFD~\cite{zhu2020deep}. We randomly select 116 objects as the training set and 25 objects as the test set.
All experiments are compared with the SOTA methods for better verification.
{To avoid thin closed reconstructions during the training process, we employ a smaller learning rate for the SDF-Net and a larger learning rate for the Validity-Net.}
Please refer to the implementation of \netName{} in the supplementary.
\vspace{-3mm}

\begin{figure*}[htb]
	\vspace{0.0mm}
	\begin{minipage}[t]{0.13\textwidth}
		\centering
		\includegraphics[width=0.9\textwidth]{Figure/comparison_open_d3d/long_sleeve_upper/252/crop/gt_marked.png}
	\end{minipage}
	\begin{minipage}[t]{0.06\textwidth}
		\centering
		\includegraphics[width=1.0\textwidth]{Figure/comparison_open_d3d/long_sleeve_upper/252/crop/gt.png}
	\end{minipage}
	\begin{minipage}[t]{0.13\textwidth}
		\centering
		\includegraphics[width=0.9\textwidth]{Figure/comparison_open_d3d/long_sleeve_upper/252/crop/ours_marked.png}
	\end{minipage}
	\begin{minipage}[t]{0.06\textwidth}
		\centering
		\includegraphics[width=1.0\textwidth]{Figure/comparison_open_d3d/long_sleeve_upper/252/crop/ours.png}
	\end{minipage}
	\begin{minipage}[t]{0.13\textwidth}
		\centering
		\includegraphics[width=0.9\textwidth]{Figure/comparison_open_d3d/long_sleeve_upper/252/crop/neus_marked.png}
	\end{minipage}
	\begin{minipage}[t]{0.06\textwidth}
		\centering
		\includegraphics[width=1.0\textwidth]{Figure/comparison_open_d3d/long_sleeve_upper/252/crop/neus.png}
	\end{minipage}
	\begin{minipage}[t]{0.13\textwidth}
		\centering
		\includegraphics[width=0.9\textwidth]{Figure/comparison_open_d3d/long_sleeve_upper/252/crop/idr_marked.png}
	\end{minipage}
	\begin{minipage}[t]{0.06\textwidth}
		\centering
		\includegraphics[width=1.0\textwidth]{Figure/comparison_open_d3d/long_sleeve_upper/252/crop/idr.png}
	\end{minipage}
	\begin{minipage}[t]{0.13\textwidth}
		\centering
		\includegraphics[width=0.9\textwidth]{Figure/comparison_open_d3d/long_sleeve_upper/252/crop/HFS_marked.png}
	\end{minipage}
	\begin{minipage}[t]{0.06\textwidth}
		\centering
		\includegraphics[width=1.0\textwidth]{Figure/comparison_open_d3d/long_sleeve_upper/252/crop/HFS.png}
	\end{minipage}
	\\
	\vspace{0.0mm}
	\begin{minipage}[t]{0.13\textwidth}
		\centering
		\includegraphics[width=0.9\textwidth]{Figure/comparison_open_d3d/no_sleeve_upper/323/crop/gt_marked.png}
	\end{minipage}
	\begin{minipage}[t]{0.06\textwidth}
		\centering
		\includegraphics[width=1.0\textwidth]{Figure/comparison_open_d3d/no_sleeve_upper/323/crop/gt.png}
	\end{minipage}
	\begin{minipage}[t]{0.13\textwidth}
		\centering
		\includegraphics[width=0.9\textwidth]{Figure/comparison_open_d3d/no_sleeve_upper/323/crop/ours_marked.png}
	\end{minipage}
	\begin{minipage}[t]{0.06\textwidth}
		\centering
		\includegraphics[width=1.0\textwidth]{Figure/comparison_open_d3d/no_sleeve_upper/323/crop/ours.png}
	\end{minipage}
	\begin{minipage}[t]{0.13\textwidth}
		\centering
		\includegraphics[width=0.9\textwidth]{Figure/comparison_open_d3d/no_sleeve_upper/323/crop/neus_marked.png}
	\end{minipage}
	\begin{minipage}[t]{0.06\textwidth}
		\centering
		\includegraphics[width=1.0\textwidth]{Figure/comparison_open_d3d/no_sleeve_upper/323/crop/neus.png}
	\end{minipage}
	\begin{minipage}[t]{0.13\textwidth}
		\centering
		\includegraphics[width=0.9\textwidth]{Figure/comparison_open_d3d/no_sleeve_upper/323/crop/idr_marked.png}
	\end{minipage}
	\begin{minipage}[t]{0.06\textwidth}
		\centering
		\includegraphics[width=1.0\textwidth]{Figure/comparison_open_d3d/no_sleeve_upper/323/crop/idr.png}
	\end{minipage}
	\begin{minipage}[t]{0.13\textwidth}
		\centering
		\includegraphics[width=0.9\textwidth]{Figure/comparison_open_d3d/no_sleeve_upper/323/crop/HFS_marked.png}
	\end{minipage}
	\begin{minipage}[t]{0.06\textwidth}
		\centering
		\includegraphics[width=1.0\textwidth]{Figure/comparison_open_d3d/no_sleeve_upper/323/crop/HFS.png}
	\end{minipage}
	\\
	\vspace{0.0mm}
	\begin{minipage}[t]{0.13\textwidth}
		\centering
		\includegraphics[width=0.9\textwidth]{Figure/comparison_open_d3d/short_sleeve_dress/63/crop/gt_marked.png}
	\end{minipage}
	\begin{minipage}[t]{0.06\textwidth}
		\centering
		\includegraphics[width=1.0\textwidth]{Figure/comparison_open_d3d/short_sleeve_dress/63/crop/gt.png}
	\end{minipage}
	\begin{minipage}[t]{0.13\textwidth}
		\centering
		\includegraphics[width=0.9\textwidth]{Figure/comparison_open_d3d/short_sleeve_dress/63/crop/ours_marked.png}
	\end{minipage}
	\begin{minipage}[t]{0.06\textwidth}
		\centering
		\includegraphics[width=1.0\textwidth]{Figure/comparison_open_d3d/short_sleeve_dress/63/crop/ours.png}
	\end{minipage}
	\begin{minipage}[t]{0.13\textwidth}
		\centering
		\includegraphics[width=0.9\textwidth]{Figure/comparison_open_d3d/short_sleeve_dress/63/crop/neus_marked.png}
	\end{minipage}
	\begin{minipage}[t]{0.06\textwidth}
		\centering
		\includegraphics[width=1.0\textwidth]{Figure/comparison_open_d3d/short_sleeve_dress/63/crop/neus.png}
	\end{minipage}
	\begin{minipage}[t]{0.13\textwidth}
		\centering
		\includegraphics[width=0.9\textwidth]{Figure/comparison_open_d3d/short_sleeve_dress/63/crop/idr_marked.png}
	\end{minipage}
	\begin{minipage}[t]{0.06\textwidth}
		\centering
		\includegraphics[width=1.0\textwidth]{Figure/comparison_open_d3d/short_sleeve_dress/63/crop/idr.png}
	\end{minipage}
	\begin{minipage}[t]{0.13\textwidth}
		\centering
		\includegraphics[width=0.9\textwidth]{Figure/comparison_open_d3d/short_sleeve_dress/63/crop/HFS_marked.png}
	\end{minipage}
	\begin{minipage}[t]{0.06\textwidth}
		\centering
		\includegraphics[width=1.0\textwidth]{Figure/comparison_open_d3d/short_sleeve_dress/63/crop/HFS.png}
	\end{minipage}
	\\
	\vspace{0.0mm}
	\begin{minipage}[t]{0.13\textwidth}
		\centering
		\includegraphics[width=0.9\textwidth]{Figure/comparison_open_d3d/pants/315/crop/gt_marked.png}
	\end{minipage}
	\begin{minipage}[t]{0.06\textwidth}
		\centering
		\includegraphics[width=1.0\textwidth]{Figure/comparison_open_d3d/pants/315/crop/gt.png}
	\end{minipage}
	\begin{minipage}[t]{0.13\textwidth}
		\centering
		\includegraphics[width=0.9\textwidth]{Figure/comparison_open_d3d/pants/315/crop/ours_marked.png}
	\end{minipage}
	\begin{minipage}[t]{0.06\textwidth}
		\centering
		\includegraphics[width=1.0\textwidth]{Figure/comparison_open_d3d/pants/315/crop/ours.png}
	\end{minipage}
	\begin{minipage}[t]{0.13\textwidth}
		\centering
		\includegraphics[width=0.9\textwidth]{Figure/comparison_open_d3d/pants/315/crop/neus_marked.png}
	\end{minipage}
	\begin{minipage}[t]{0.06\textwidth}
		\centering
		\includegraphics[width=1.0\textwidth]{Figure/comparison_open_d3d/pants/315/crop/neus.png}
	\end{minipage}
	\begin{minipage}[t]{0.13\textwidth}
		\centering
		\includegraphics[width=0.9\textwidth]{Figure/comparison_open_d3d/pants/315/crop/idr_marked.png}
	\end{minipage}
	\begin{minipage}[t]{0.06\textwidth}
		\centering
		\includegraphics[width=1.0\textwidth]{Figure/comparison_open_d3d/pants/315/crop/idr.png}
	\end{minipage}
	\begin{minipage}[t]{0.13\textwidth}
		\centering
		\includegraphics[width=0.9\textwidth]{Figure/comparison_open_d3d/pants/315/crop/HFS_marked.png}
	\end{minipage}
	\begin{minipage}[t]{0.06\textwidth}
		\centering
		\includegraphics[width=1.0\textwidth]{Figure/comparison_open_d3d/pants/315/crop/HFS.png}
	\end{minipage}
	\\
	\vspace{0.0mm}
	\begin{minipage}[t]{0.13\textwidth}
		\centering
		\subfloat[GT]{\includegraphics[width=0.9\textwidth]{Figure/comparison_open_mgn/TShirtNoCoat/125611500935128/crop/gt_marked.png}}
	\end{minipage}
	\begin{minipage}[t]{0.06\textwidth}
		\centering
		\includegraphics[width=1.0\textwidth]{Figure/comparison_open_mgn/TShirtNoCoat/125611500935128/crop/gt.png}
	\end{minipage}
	\begin{minipage}[t]{0.13\textwidth}
		\centering
		\subfloat[Ours]{\includegraphics[width=0.9\textwidth]{Figure/comparison_open_mgn/TShirtNoCoat/125611500935128/crop/ours_marked.png}}
	\end{minipage}
	\begin{minipage}[t]{0.06\textwidth}
		\centering
		\includegraphics[width=1.0\textwidth]{Figure/comparison_open_mgn/TShirtNoCoat/125611500935128/crop/ours.png}
	\end{minipage}
	\begin{minipage}[t]{0.13\textwidth}
		\centering
		\subfloat[NeuS]{\includegraphics[width=0.9\textwidth]{Figure/comparison_open_mgn/TShirtNoCoat/125611500935128/crop/neus_marked.png}}
	\end{minipage}
	\begin{minipage}[t]{0.06\textwidth}
		\centering
		\includegraphics[width=1.0\textwidth]{Figure/comparison_open_mgn/TShirtNoCoat/125611500935128/crop/neus.png}
	\end{minipage}
	\begin{minipage}[t]{0.13\textwidth}
		\centering
		\subfloat[IDR]{\includegraphics[width=0.9\textwidth]{Figure/comparison_open_mgn/TShirtNoCoat/125611500935128/crop/idr_marked.png}}
	\end{minipage}
	\begin{minipage}[t]{0.06\textwidth}
		\centering
		\includegraphics[width=1.0\textwidth]{Figure/comparison_open_mgn/TShirtNoCoat/125611500935128/crop/idr.png}
	\end{minipage}
	\begin{minipage}[t]{0.13\textwidth}
		\centering
		\subfloat[HFS]{\includegraphics[width=0.9\textwidth]{Figure/comparison_open_mgn/TShirtNoCoat/125611500935128/crop/HFS_marked.png}}
	\end{minipage}
	\begin{minipage}[t]{0.06\textwidth}
		\centering
		\includegraphics[width=1.0\textwidth]{Figure/comparison_open_mgn/TShirtNoCoat/125611500935128/crop/HFS.png}
	\end{minipage}
\vspace{-1em}
\caption{Comparisons on open surface reconstruction. Row 1 -- 4 are evaluated on \DFD~\cite{zhu2020deep} and Row 5 is evaluated on \MGN~\cite{bhatnagar2019mgn}. \modelName~is able to reconstruct high-fidelity open surfaces while NeuS~\cite{wang2021neus}, IDR~\cite{yariv2020idr} and HFS~\cite{wang2022hfneus} fail to recover the correct topologies.}
\vspace{-1em}
\label{fig:comparison_open}
\end{figure*}

\vspace{-0.5em}


% \paragraph{Implementation of \netName.} 
% We implement \netName~as follows.
% \textbf{SDF-Net}: We borrowed the implementation of the SDF network in DeepSDF~\cite{deepsdf}, which consists of 8 layers
% with hidden layers of width 512, and a single skip connection from the input to the middle layer. We initialize the parameters of the MLP with geometric initialization~\cite{igr}.
% \noindent\textbf{Color-Net}: We borrowed the implementation of the renderer MLP in IDR~\cite{yariv2020idr}, which consists of 4 layers, with hidden layers of
% width $512$. We apply positional encoding~\cite{mildenhall2020nerf} to improve the learning of high-frequencies.
% \noindent \textbf{Validity-Net}: The MLP consists of 8 layers with Xavier initialization. We used the \textit{ReLU} activation between hidden layers and \textit{Sigmoid} for the output.
% \weikai{Move this paragraph to supplemental.}

\vspace{-0.5mm}


\paragraph{Implementation details.}
For the reconstruction experiments on open surfaces, we render the  ground truth point clouds from \DFD~\cite{zhu2020deep} with Pytorch3D~\cite{ravi2020pytorch3d} at a resolution of $256^2$. To get diverse supervision data, we uniformly sample 648 and 64 viewpoints on the unit sphere for \DFD~and \MGN~(MGN), respectively.
For the single view reconstruction experiment, we uniformly sample 64 viewpoints on the unit sphere as the camera positions.
We use an ResNet-18~\cite{He_2016_CVPR} encoder to predict a latent code $\textbf{z}$ describing the surface's geometry and color. We then use the concatenation of $\{\textbf{z}, \pt\}$ as the input to \netName~(decoder) to evaluate the SDF, validity, and color at the query positions. We optimize the autoencoder by comparing the 2D rendering and the ground truth image. 
In the evaluation stage, we accept a single image as the input and directly export the evaluated SDF and validity as 3D mesh.

\vspace{-1em}
\paragraph{Evaluations.} For multiview reconstruction on watertight surfaces, we measure the Chamfer Distance (CD) with \textit{DTU MVS 2014 evaluation toolkit}~\cite{dtu}. For the reconstruction experiments on open surfaces, we measure the CD with the \textit{PCU} Library~\cite{point-cloud-utils}. For all the experiments, we evaluate the result meshes at resolution $512^3$. 