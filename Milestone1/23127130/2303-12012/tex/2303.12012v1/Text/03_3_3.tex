\vspace{-1em}
\paragraph{Discretization.}
% \brandon{We use too many ``like xxx'', ``the same as xxx'' in this and next subsections. After reading, this makes me feel that we only invent $\gamma(\pt)$, nothing else. We should NOT write paragraphs evenly according to each step of the algorithm, but should emphasize more for our NEW parts, like the validity; even if it is quite simple, it is worth to expand much more. For similar steps like Neus parts, even if it is complex, we still should shrink them. 3PSDF is also a simple idea, but Weikai organized well with good emphasis; Resnet is even simpler, but the paper writes well. Please refer those as examples.} 


We adopt the classic discretization scheme in differentiable volumetric rendering\cite{mildenhall2020nerf, wang2021neus} for the opacity and weight function. For a set of sampled points along the ray
$
\{p_{i} = \textbf{o} + t_{i}\textbf{v} |i=1, ..., n, t_{i} < t_{i + 1}\}
$, the rendered pixel color is
\vspace{-1.0em}
\begin{equation}
    \imagePred(\mathbf{o}, \mathbf{v})=\sum_{i=1}^{n}\Pi_{j=1}^{i-1}(1 - \alpha_{j})\alpha_{i}c_{i}
    \label{equ:discrete_color}
    \vspace{-0.5em}
\end{equation}
where $c_i$ is the estimated color for the $i$-th sampling point; $\alpha_{i}$ is the discrete opacity value in SDF rendering
\begin{equation}
\small
\alpha_{i}=\frac{
                \Phi_{s}(-Sign(\mathbf{v}\cdot\mathbf{n})f(\pt(t_{i})))
                 - \Phi_{s}(-Sign(\mathbf{v}\cdot\mathbf{n})f(\pt(t_{i+1})))
            }{
                \Phi_{s}(-Sign(\mathbf{v}\cdot\mathbf{n})f(\pt(t_{i})))
            }
\end{equation}

% where $\alpha_{j}$ is the discrete opacity value, which can be derived from Equation~\ref{equ:discrete_alpha}.
% \begin{align}
% \begin{split}
%     \alpha_{i} &= 1 - \exp(-\int_{t_{i}}^{t_{i+1}}\rho(t)dt)\\
%               &= 1 - \exp(-\int_{t_{i}}^{t_{i+1}}
%                 \frac{
%                     -\frac{
%                         d\Phi_{s}
%                     }{
%                         dt
%                     }(f(\pt(t))\cdot \gamma(\pt(t)))
%                 }{
%                     \Phi_{s}(f(\pt(t)))\cdot \gamma(\pt(t)))
%                 }dt)\\
%               &= 1 - e^{(ln(\Phi_{s}(f(\pt(t_{i+1}))\cdot \gamma(\pt(t_{i+1}))) - ln(\Phi_{s}(f(\pt(t_{i}))\cdot \gamma(\pt(t_{i}))))}\\
%               &= 1 - \frac{
%                 \Phi_{s}(f(\pt(t_{i+1}))\cdot \gamma(\pt(t)))
%               }{
%                 \Phi_{s}(f(\pt(t_{i}))\cdot \gamma(\pt(t)))
%               }\\
%               &=\frac{
%                 \Phi_{s}(f(\pt(t_{i}))\cdot \gamma(\pt(t))) - \Phi_{s}(f(\pt(t_{i+1}))\cdot \gamma(\pt(t)))
%               }{
%                 \Phi_{s}(f(\pt(t_{i}))\cdot \gamma(\pt(t)))
%               }
%     \label{equ:discrete_alpha}
% \end{split}
% \end{align}
% \brandon{Too many equation label numbers: from 7 to 11. maybe 8, 9, 10 or at least 9, 10 are not necessary.}

Now we have built an unbiased and occlusion-aware volume weight function that supports rendering the front and back faces with the SDF representation.
