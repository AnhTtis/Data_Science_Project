\vspace{-1.5mm}
\subsection{Multiview Reconstruction on Closed Surfaces}
\vspace{-0.5em}
% We investigate if our method is applicable to multi-view reconstruction in real-world scenarios. For this experiment, we do not condition our model and train one model per object.

% \weikai{We only have quantitative comparisons with NeRF?}\xiaoxu{Yes}
% \xiaoxu{NeuS compared with NerF in their paper. I copied the results.}

We compare our approach with the state-of-the-art volume and surface rendering based methods - HFS~\cite{wang2022hfneus}, NeuS~\cite{wang2021neus} and IDR~\cite{yariv2020idr}, and a classic mesh reconstruction and novel view synthesis method -- NeRF~\cite{mildenhall2020nerf}. We report the quantitative results in Table~\ref{table:comparison_watertight}.

We also show visual comparison with a widely-used MVS method: COLMAP~\cite{schoenberger2016sfm, schoenberger2016mvs}. 
We show qualitative results in Fig.~\ref{fig:comparison_watertight}. The results reconstructed with the proposed method show comparable quality compared with the state-of-the-art. 


% To evaluate our approach and baseline methods, we use 8 categories from the DeepFashion3D Dataset~\cite{zhu2020deep}: long sleeve upper, short-sleeve upper, long sleeve dress, short sleeve dress, no sleeve dress, long pants, short pants, and skirt. And we use TODO categories from the MGN Dataset~\cite{bhatnagar2019mgn}: TShirtNoCoat, ShirtNoCoat, LongCoat, Pants, ShortPants. The dataset contain clothes a wide variety of materials, appearance and geometry, including challenging cases for reconstruction algorithms. We render the meshes to generate 216 images with the image resolution of $256 \times 256$. 

% To validate that our approach could represent closed surfaces. We use 5 scenes from the DTU dataset~\cite{jensen2014large}. Each scene contains 49 or 64 images with the image resolution of $1600 \times 1200$.

% 122: neus: 0.53
\begin{table}[h]
    \small
    \centering
    \begin{tabular}{c|c|c|c|c|c}
        \hline
        CD$\downarrow$ & Ours & NeuS & IDR & NeRF & HFS\\
        \hline
        \hline
        % scan 37 & $1.86$ & $\textbf{0.98}$ & $1.87$ & $2.39$\\
        % scan 24 & $1.75$ & $\textbf{0.83}$ & $1.63$ & $1.15$\\
        scan 55 & $0.47$ & $0.38$ & $0.48$ & $0.66$ & $\textbf{0.37}$\\
        % scan 65 & $1.08$ & $\textbf{0.60}$ & $0.79$ & $1.44$\\
        scan 69 & $0.84$ & $\textbf{0.60}$ & $0.77$ & $1.50$ & $0.66$\\
        scan 83 & $1.28$ & $1.43$ & $1.33$ & $\textbf{1.20}$ & $1.27$\\
        scan 97 & $1.09$ & $\textbf{0.96}$ & $1.16$ & $1.96$ & $1.00$\\
        scan 105 & $\textbf{0.75}$ & $0.78$ & $0.76$ & $1.27$ & $0.86$\\
        scan 106 & $0.76$ & $\textbf{0.52}$ & $0.67$ & $0.66$ & $0.57$\\
        scan 110 & $\textbf{0.80}$ & $1.44$ & $0.90$ & $2.61$ & $1.24$\\
        scan 114 & $0.38$ & $\textbf{0.36}$ & $0.42$ & $1.04$ & $0.41$\\
        scan 118 & $0.56$ & $\textbf{0.46}$ & $0.51$ & $1.13$ & $0.52$\\
        scan 122 & $0.55$ & $\textbf{0.49}$ & $0.53$ & $0.99$ & $\textbf{0.49}$\\
        \hline
        \hline
        average & $0.749$ & $0.742$ & $0.753$ & $1.302$ & $\textbf{0.741}$\\
        \hline
    \end{tabular}
    \caption{Quantitative evals on real-world object reconstruction.
    % \brandon{we have the same performance as Neus in the last row; only bolding ours are OK or not? Also, I think we should add average for both tables.}
    }
    % \vspace{-1.5em}
    \label{table:comparison_watertight}
\end{table}

