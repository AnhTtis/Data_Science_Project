\vspace{-1em}
\section{Introduction}
\vspace{-0.5em}
% \note{- What's the problem? What's our motivation?
% \begin{itemize}
%     \item DR is an important technique for 3D unsupervised learning of 3D shapes. However, existing DR techniques are limited to either specific topology (mesh-based) or watertight surfaces (SDF).
%     \item New advances in neural implicit functions, e.g. 3PSDF and GIFS, have enabled the modeling of surface with arbitrary topologies. However, no existing techniques can support differentiable rendering of these representations. 
% \end{itemize}
% }

3D reconstruction from multi-view images is a fundamental problem in computer vision and computer graphics. 
Recent advances in neural implicit functions~\cite{yariv2020idr,liu2019learning,wang2021neus,niemeyer2020differentiable} have brought impressive progress in achieving high-fidelity reconstruction of complex geometry even with sparse views. 
They use differentiable rendering to render the inferred implicit surface into images which are compared with the input images for network supervision.
This provides a promising alternative of learning 3D shapes directly from 2D images without 3D data.
However, existing neural rendering methods represent surfaces as signed distance function (SDF)~\cite{yariv2020idr,dist} or occupancy field~\cite{niemeyer2020differentiable, oechsle2021unisurf}, limiting their output to \textit{closed} surfaces.
This leads to a barrier in reconstructing a large variety of real-world objects with open boundaries, such as 3D garments, walls of a scanned 3D scene, \textit{etc}.
% 3D modeling and reconstruction is a fundamental problem in computer vision and computer graphics. Existing learning-based approaches need a large corpus of 3D data for model training, which requires laborious efforts for data capturing and labeling. 
%Differentiable rendering (DR) provides a promising alternative of learning 3D shapes directly from 2D images, without relying on 3D ground truths. However, existing DR techniques based on Signed Distance Function (SDF) are limited to watertight shapes and are not able to reconstruct shapes with open boundaries. 
The recently proposed NDF~\cite{chibane2020ndf}, 3PSDF~\cite{chen_2022_3psdf} and GIFS~\cite{Ye_2022_CVPR} introduce new implicit representations supporting 3D geometry with arbitrary topologies, including both closed and open surfaces. 
However, none of these representations are compatible with existing neural rendering frameworks.
Leveraging neural implicit rendering to reconstruct \textit{non-watertight} shapes, i.e., shapes with \textit{open} surfaces, from multi-view images remains a virgin land.
% existing DR methods is able to learn the neural representations of arbitrary shapes directly from multi-view images. 
%\brandon{This intro of 3PSDF is unnatural. I think we should say like some recent works propose approaches to reconstruct arbitrary topologies, but they are not easily compatible with existing DR framework.}

% \note{- What do we introduce?
% \begin{itemize}
%     \item A new differentiable rendering framework for neural implicit representation that is able to reconstruct surfaces with arbitrary topologies from multi-view images without 3D ground truth.
% \end{itemize}
% }

% \note{- How does it work? What's unique?
% \begin{itemize}
%     \item Introduce a validity branch to infer the valid region of SDF. At rendering, regions with low validity will not be rendered, enabling the possibility of rendering non-watertight surfaces on top of well recognized SDF rendering.
%     \item Introduce a new rendering mechanism that enables both front and back surfaces to be rendered.
%     \item A regularization framework that promotes the rendering of open surfaces while ensuring plausible surface connectivity and avoiding artifacts. 
% \end{itemize}
% }

% We propose a method to learn Implicit Surfaces with Arbitrary Topologies (\modelName) from multi-view images, generating high-quality implicit fields from 2D images and hence achieving the reconstruction of 3D surfaces with arbitrary topologies without requiring 3D ground-truth data. 
%\brandon{This sentence may be confusing to reviewers, as by default, implicit fields cannot reconstruct open surfaces, and we did not mention revised implicit fields could do this before here. Maybe we should put 3PSDF and other similar work like HSDF ahead of our work, but keep as concise as possible.}

We fill this gap by presenting \textit{\modelName{}}, a \textit{Ne}ural rendering framework that reconstructs surfaces with \textit{A}rbitrary \textit{T}opologies using multi-view supervision.
%As the state-of-the-art multi-view reconstruction approaches represent the 3D space as “inside” and “outside”, the implicit representations are limited to closed surfaces. 
Unlike previous neural rendering frameworks only using color and SDF predictions, we propose a validity branch to estimate the surface existence probability at the query positions, thus avoiding rendering 3D points with low validity as shown in Figure \ref{fig:surface_representation}. In contrast to 3PSDF \cite{chen_2022_3psdf} and GIFS \cite{Ye_2022_CVPR}, our validity estimation is a differentiable process. It is compatible with the volume rendering framework while maintaining its flexibility in representing arbitrary 3D topologies.
To correctly render both closed and open surfaces, we introduce a sign adjustment scheme to render both sides of surfaces, while maintaining unbiased weights and occlusion-aware properties as previous volume renderers.
In addition, to reconstruct intricate geometry, a specially tailored regularization mechanism is proposed to promote the formation of open surfaces.
% while ensuring plausible surface connectivity and avoiding artifacts.
By minimizing the difference between the rendered and the ground-truth pixels, we can faithfully reconstruct both the validity and SDF field from the input images. 
At reconstruction time, the predicted validity value along with the SDF value can be readily converted to 3D mesh with the classic field-to-mesh conversion techniques, e.g., the Marching Cubes Algorithm~\cite{marching_cubes}. 

We evaluate \modelName{} in the task of multi-view reconstruction on a large variety of challenging shapes, including both closed and open surfaces. 
\modelName{} can consistently outperform the current state-of-the-art methods both qualitatively and quantitatively.  
We also show that \modelName{} can provide efficient supervision for learning complex shape priors that can be used for reconstructing non-watertight surface only from a single image.
% To the best of our knowledge, our paper is the first approach that can learn arbitrary surface topologies containing open boundaries from multi-view images without any 3D supervision. 
Our contributions can be summarized as:
\begin{itemize}
    \item A neat neural rendering scheme of implicit surface, coded \emph{\modelName}, that introduces a novel validity branch, and, \emph{for the first time}, can faithfully reconstruct surfaces with arbitrary topologies from multi-view images.
    \vspace{-2mm}
    \item A specially tailored learning paradigm for \modelName{} with effective regularization for open surfaces.
    \vspace{-2mm}
    \item \modelName~ sets the new state-of-the-art on multi-view reconstruction on open surfaces across a wide range of benchmarks.
\end{itemize}

% \note{- What do we show? What's the impact?
% \begin{itemize}
%     \item We evaluate \modelName{} on a large variety of challenging shapes including both closed and open surfaces in the task of multi-view reconstruction.
%     \item Our method is even able to reconstruct shapes of arbitrary topologies from a single image, indicating that our method can provide effective supervision for learning priors encoding non-watertight shapes. 
% \end{itemize}
% }

% Reconstructing surfaces from multi-view images is a fundamental problem in computer vision and computer graphics. 
% In recent years, learning-based approaches has gained popularity in solving this problem with differentialble rendering systems based on rasterization~\cite{chen2019learning, genova2018unsupervised, kato2018neural, kundu20183drcnn, liu2019softras, opendr} and ray casting/tracing\cite{Paschalidou_2018_CVPR}. 
% Neural implicit representations~\cite{sdf0, Occupancy_Networks} are gaining popularity because of their flexibility in surface representation with arbitrary topology and resolution, thus avoiding discretization artifacts. Recent works~\cite{mildenhall2020nerf, dvr, yariv2020multiview, wang2021neus} model the 3D surface as occupancy~\cite{dvr, oechsle2021unisurf} or signed distance field (SDF)~\cite{mildenhall2020nerf, yariv2020multiview, wang2021neus}. They train a neural network to model the signed distance or occupancy. However, current approaches built upon SDFs can only model closed surfaces that naturally support in/out test for extracting the zero-level set. For open surfaces without well-defined continuous SDF, reconstructing surfaces from multi-view images is still an open question.

% In this paper, we solve this problem by representing an open-surface as a level set of a signed distance field (SDF) with a valid sign, which indicates the possibility of the validity of the space. The SDF of the points with high validity will be considered in the reconstructed 3D object, while the SDF value of the points with low validity will be assigned with a nan value~\cite{3PSDF}, which prevents the decision boundary to be formed between these points and the neighborhood. Therefore, MuNOS supporting easy field-to-mesh conversion using the classic Marching Cubes algorithm.

% By introducing space validity, we make it possible to apply the volume rendering approach to learn the implicit neural representation of open surfaces.
% In the volume rendering process, the space validity also represents the rendering possibility. Points with high validity will have high opacity, leading to higher rendering weight in the $\alpha$-composition of the colors to produce the output pixel. 
% We use individual neural networks to train the valid sign and the the signed distance field separately. 

