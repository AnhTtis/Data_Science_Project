% CVPR 2022 Paper Template
% based on the CVPR template provided by Ming-Ming Cheng (https://github.com/MCG-NKU/CVPR_Template)
% modified and extended by Stefan Roth (stefan.roth@NOSPAMtu-darmstadt.de)

% \documentclass[10pt,twocolumn,letterpaper]{article}

%%%%%%%%% PAPER TYPE  - PLEASE UPDATE FOR FINAL VERSION
% \usepackage[review]{cvpr}      % To produce the REVIEW version
\usepackage{cvpr}              % To produce the CAMERA-READY version
%\usepackage[pagenumbers]{cvpr} % To force page numbers, e.g. for an arXiv version

% Include other packages here, before hyperref.
\usepackage{url}            % simple URL typesetting
\usepackage{amsfonts}       % blackboard math symbols
\usepackage{amsmath}
\usepackage{amssymb}
\usepackage{booktabs}       % professional-quality tables
\usepackage{float}
\usepackage{graphicx}
\usepackage{microtype}      % microtypography
\usepackage{nicefrac}       % compact symbols for 1/2, etc.
\usepackage{xcolor}         % colors
\usepackage{mathrsfs}
\usepackage{empheq}
\usepackage{array,multirow}
% \usepackage[accsupp]{axessibility}  % Improves PDF readability for those with disabilities.

% It is strongly recommended to use hyperref, especially for the review version.
% hyperref with option pagebackref eases the reviewers' job.
% Please disable hyperref *only* if you encounter grave issues, e.g. with the
% file validation for the camera-ready version.
%
% If you comment hyperref and then uncomment it, you should delete
% ReviewTempalte.aux before re-running LaTeX.
% (Or just hit 'q' on the first LaTeX run, let it finish, and you
%  should be clear).
\usepackage[pagebackref,breaklinks,colorlinks]{hyperref}

\makeatletter
\@namedef{ver@everyshi.sty}{}
\makeatother
\usepackage{tikz}

%%%%%%%%% PAPER ID  - PLEASE UPDATE
\def\cvprPaperID{2230} % *** Enter the CVPR Paper ID here
\def\confName{CVPR}
\def\confYear{2023}



\ifdefined\siggraph
\usepackage{times}
\fi

%\usepackage{parskip}
\usepackage{color}
\usepackage{ifthen}
\usepackage{float}
\usepackage{alltt}
\usepackage{newlfont} % for Box
\usepackage{array}
\usepackage{wrapfig}
\usepackage{booktabs}
\usepackage{multirow}
\usepackage{amsfonts}
\usepackage{dsfont}
\usepackage[linesnumbered,ruled,vlined]{algorithm2e}
 
%%% Coloring the comment as blue
\newcommand\mycommfont[1]{\footnotesize\ttfamily\textcolor{blue}{#1}}
\SetCommentSty{mycommfont}
 


\definecolor{purple}{rgb}{1, 0, 1}

\newcommand{\ie}{\emph{i.e.,}\xspace}
\newcommand{\eg}{\emph{e.g.,}\xspace}
\newcommand{\abr}{\emph{abbr.}\xspace}
\newcommand{\ea}{\emph{et al.}\xspace}
\newcommand{\gensync}{\emph{GenSync}\xspace}
\newcommand{\colosseum}{\emph{Colosseum}\xspace}
\newcommand{\srep}{\emph{SREP}\xspace} % Set Reconciliation Enhances
\newcommand{\srepsim}{\emph{SREPSim}\xspace}
% Propagation
\newcommand{\esrep}{\emph{E-SREP}\xspace}
\newcommand{\epsrep}{\emph{EP-SREP}\xspace}
\newcommand{\mesrep}{\emph{ME-SREP}\xspace}
\newcommand{\mempoolsync}{\emph{MempoolSync}}

\newcommand{\fref}[1]{Fig.~\ref{#1}}
\newcommand{\tref}[1]{Table~\ref{#1}}
\newcommand{\aref}[1]{Algorithm~\ref{#1}}
\newcommand{\procref}[1]{Procedure~\ref{#1}}
\newcommand{\sref}[1]{Section~\ref{#1}}
\newcommand{\lineref}[1]{line~\ref{#1}}
\newcommand{\appref}[1]{Appendix~\ref{#1}}

% Change \eqref
\LetLtxMacro{\originaleqref}{\eqref}
\renewcommand{\eqref}{Eq.~\originaleqref}

% Theorems and corollaries
\newcounter{theoremcount}
\setcounter{theoremcount}{0}
\DeclareRobustCommand{\theorem}[1]{%
  \refstepcounter{theoremcount}%
  \noindent\textit{\textbf{Theorem \thetheoremcount\label{theorem:#1}: }}%
}
\DeclareRobustCommand{\theoremref}[1]{Theorem~\ref{theorem:#1}}

\DeclareRobustCommand{\proof}{\emph{Proof:}\xspace}
\DeclareRobustCommand{\qqed}{\hfill$\blacksquare$}

\newcounter{corollcount}
\setcounter{corollcount}{0}
\DeclareRobustCommand{\coroll}[1]{%
  \refstepcounter{corollcount}%
  \noindent\textit{\textbf{Corollary \thecorollcount\label{coroll:#1}: }}%
}
\DeclareRobustCommand{\corollref}[1]{Corollary~\ref{coroll:#1}}

\newcounter{lemmacount}
\setcounter{lemmacount}{0}
\DeclareRobustCommand{\lemma}[1]{%
  \refstepcounter{lemmacount}%
  \noindent\textit{\textbf{Lemma \thelemmacount\label{lemma:#1}: }}%
}
\DeclareRobustCommand{\lemmaref}[1]{Lemma~\ref{lemma:#1}}

\newcounter{definitioncount}
\setcounter{definitioncount}{0}
\DeclareRobustCommand{\definition}[1]{%
  \refstepcounter{definitioncount}%
  \noindent\textit{\textbf{Definition \thedefinitioncount\label{definition:#1}: }}%
}
\DeclareRobustCommand{\defref}[1]{Definition~\ref{definition:#1}}

%notes of different authors
\newif\ifnotes
\notestrue
\notesfalse

\newif\ifdiff
\difftrue
\difffalse

\newcommand{\anote}[1]{\ifnotes $\ll$\textsf{\textcolor{purple}{Ari: {#1}}}$\gg$ \fi}
\newcommand{\nnote}[1]{\ifnotes $\ll$\textsf{\textcolor{orange}{Novak: {#1}}}$\gg$ \fi}
\newcommand{\diff}[1]{\ifdiff\textcolor{orange}{#1}\else#1\fi}

%%% Local Variables:
%%% mode: latex
%%% TeX-master: "main"
%%% End:

\renewcommand{\Pr}{\field{P}}

\usepackage{MnSymbol}
\usepackage{comment}

\DeclareMathOperator{\Tr}{Tr}
\newcommand{\bA}{\boldsymbol{A}}
\newcommand{\ba}{\boldsymbol{a}}
\newcommand{\bx}{\boldsymbol{x}}
\newcommand{\bxi}{\boldsymbol{\xi}}
\newcommand{\bc}{\boldsymbol{c}}
\newcommand{\bC}{\boldsymbol{C}}
\newcommand{\bq}{\boldsymbol{q}}
\newcommand{\bd}{\boldsymbol{d}}
\newcommand{\bX}{\boldsymbol{X}}
\newcommand{\bM}{\boldsymbol{M}}
\newcommand{\bone}{\boldsymbol{1}}
\newcommand{\bI}{\boldsymbol{I}}
\newcommand{\bu}{\boldsymbol{u}}
\newcommand{\bb}{\boldsymbol{b}}
\newcommand{\by}{\boldsymbol{y}}
\newcommand{\bY}{\boldsymbol{Y}}
\newcommand{\bhatY}{\boldsymbol{\hat{Y}}}
\newcommand{\bbary}{\boldsymbol{\bar{y}}}
\newcommand{\bg}{\boldsymbol{g}}
\newcommand{\bz}{\boldsymbol{z}}
\newcommand{\bZ}{\boldsymbol{Z}}
\newcommand{\bbarZ}{\boldsymbol{\bar{Z}}}
\newcommand{\bbarz}{\boldsymbol{\bar{z}}}
\newcommand{\bhatZ}{\boldsymbol{\hat{Z}}}
\newcommand{\bhatz}{\boldsymbol{\hat{z}}}
\newcommand{\bhatx}{\boldsymbol{\hat{x}}}
\newcommand{\haty}{\hat{y}}
\newcommand{\barz}{\bar{z}}
\newcommand{\bS}{\boldsymbol{S}}
\newcommand{\bbarS}{\boldsymbol{\bar{S}}}
\newcommand{\bw}{\boldsymbol{w}}
\newcommand{\bhatw}{\hat{\boldsymbol{w}}}
\newcommand{\bW}{\boldsymbol{W}}
\newcommand{\bU}{\boldsymbol{U}}
\newcommand{\bv}{\boldsymbol{v}}
\newcommand{\bzero}{\boldsymbol{0}}
\newcommand{\balpha}{\boldsymbol{\alpha}}
\newcommand{\sA}{\mathcal{A}}
\newcommand{\sC}{\mathcal{C}}
\newcommand{\sD}{\mathcal{D}}
\newcommand{\sX}{\mathcal{X}}
\newcommand{\sY}{\mathcal{Y}}
\newcommand{\sS}{\mathcal{S}}
\newcommand{\sT}{\mathcal{T}}
\newcommand{\sZ}{\mathcal{Z}}
\newcommand{\sL}{\mathcal{L}}
\newcommand{\sI}{\mathcal{I}}
\newcommand{\bbO}{\mathbb{O}}
\newcommand{\sbarZ}{\bar{\mathcal{Z}}}
\newcommand{\fbag}{\bold{F}}


\begin{document}

%%%%%%%%% TITLE - PLEASE UPDATE
\title{\modelName{}: Learning Neural Implicit Surfaces with Arbitrary Topologies\\ from Multi-view Images}
\author{Xiaoxu Meng~~~~~~~~Weikai Chen~~~~~~~~Bo Yang\\
Digital Content Technology Center, Tencent Games\\
{\tt\small \{xiaoxumeng,weikaichen,brandonyang\}@global.tencent.com}
% For a paper whose authors are all at the same institution,
% omit the following lines up until the closing ``}''.
% Additional authors and addresses can be added with ``\and'',
% just like the second author.
% To save space, use either the email address or home page, not both
% \and
% Second Author\\
% Institution2\\
% First line of institution2 address\\
% {\tt\small secondauthor@i2.org}
}

% We show three groups of shape reconstruction results generated by NDF [10] (in cyan) and our proposed 3PSDF (in gold)
% respectively. Our method is able to faithfully reconstruct high-fidelity, intricate geometric details including both the closed and open
% surfaces, while NDF suffers from the meshing problems. Each NDF result is reconstructed from a dense point cloud containing 1 million
% points while ours are reconstructed using an equivalent resolution.

\twocolumn[{%
    \renewcommand\twocolumn[1][]{#1}%
    \maketitle
    \begin{center}
    \centering
    \captionsetup{type=figure}
    \includegraphics[width=0.95\textwidth]{Figure/teaser/teaser.pdf}
    \captionof{figure}{
    We show three groups of surface reconstruction from multi-view images. The front and back faces are rendered in blue and yellow respectively. Our method (left) is able to  reconstruct high-fidelity and intricate surfaces of arbitrary topologies, including those non-watertight structures, e.g. the thin single-layer shoulder strap of the top (middle). In comparison, the state-of-the-art NeuS~\cite{wang2021neus} method (right) can only generate watertight surfaces, resulting in thick, double-layer geometries.
    }
    \end{center}%
}]

\input{Text/99_Commands}

\begin{abstract}

What will Wi-Fi 8 be? 
Driven by the strict requirements of emerging applications, next-generation Wi-Fi is set to prioritize Ultra High Reliability (UHR) above all. 
In this paper, we explore the journey towards IEEE 802.11bn UHR, the amendment that will form the basis of Wi-Fi\,8. 
%[Lorenzo: \footnote{At the time of writing, 802.11bn is the name most likely to be adopted for the new Wi-Fi amendment. This will be confirmed prior to final publication.}]
%After providing an overview of the nearly completed Wi-Fi\,7 standard, 
We first present new use cases calling for further Wi-Fi evolution and also outline current standardization, certification, and spectrum allocation activities, sharing updates from the newly formed UHR Study Group.
We then introduce the disruptive new features envisioned for Wi-Fi\,8 %802.11bn
and discuss the associated research challenges. Among those, we focus on access point coordination and demonstrate that it could build upon 802.11be multi-link operation to make Ultra High Reliability a reality in Wi-Fi\,8.
\end{abstract}
\section{Introduction}\label{section:Introduction}
\glspl{cps} integrate real-time computing and communication capabilities with monitoring and control actions over components in the physical world~\cite{shi_survey_2011}. To face the harshness of the space environment, modern space systems such as satellites and spacecraft require tight coupling between onboard processing, communication (cyber), sensing, and actuation (physical) elements~\cite{klesh_cyber-physical_2012}. The orbit determination and control subsystems on a small spacecraft or in satellites' constellations provide a clear link between onboard processing and sensing elements of the spacecraft's physical environment~\cite{di_mascio_-board_2021}, becoming increasingly critical as small spacecrafts become ever more capable~\cite{klesh_cyber-physical_2012}. In this scenario, digital computing systems representing the decisional part of a \gls{scps} must be designed to be reliable and tolerate faults induced by cosmic radiation. Radiation-induced soft errors such as \glspl{set} and \glspl{seu} can occur more frequently in space than at ground level, creating the need for additional hardware to mitigate detrimental effects on the system~\cite{wachter_survey_2019}.

Various solutions exist to protect electronics from the adverse effects of radiation~\cite{wachter_survey_2019}. Costly radiation-hardened technologies, insulating techniques~\cite{alles_radiation_2011}, and polymer shielding~\cite{shahzad_views_2022} help mitigate soft errors. It is also possible to enhance the fault tolerance capabilities of digital systems by introducing redundancy at different levels in their design flow. Temporal redundancy techniques rely on repeated executions of the same work to determine the correct result~\cite{feng_shoestring_2010}. Spatial or modular redundancy techniques rely on multiple hardware blocks executing the same task and comparing the results~\cite{ginosar_survey_2012}. These approaches rely on rigid schemes for repetition in space and time of redundant blocks or tasks, hence they can severely impact the \gls{ppa} of computing platforms.

The increasing demand for strong processing capabilities in space~\cite{xie_resource-cost-aware_2018} is pushing researchers toward lower-overhead solutions. In recent years, the advent of RISC-V and open-source hardware has encouraged the development of high-performance \glspl{soc} for various domains without licensing or other restrictions. This includes the space domain, where custom modifications to improve properties such as reliability~\cite{di_mascio_open-source_2021} and fault tolerance are often required. Among proposed architectures, heterogeneous systems with multi-core computing clusters have gained traction in the space industry~\cite{ginosar_rc64_2016} due to increased performance and flexibility for computation and \gls{dsp} workloads~\cite{kurth_hero_2018}.
While multiple processors offer increased performance for parallelizable tasks, they also provide a unique opportunity for reliability enhancements: multiple cores can execute identical tasks, comparing their results to detect and react to faults.

In this paper, we introduce a space-ready multi-core RISC-V-based computing system featuring a \gls{hmr} approach. We leverage the independent cores available in a multi-core RISC-V cluster for redundant execution in a dynamically runtime configurable manner and introduce \gls{dcls} and \gls{tcls} modes, extending the On-Demand Redundancy Grouping with \gls{tcls} configurable under reset presented in~\cite{rogenmoser_-demand_2022}.
Our design allows each application to configure its reliability setting according to its requirements, possibly decided at runtime.
% Without sacrificing performance in the general case, this architecture allows for the safe execution of a safety-critical section at a 2.3\% area overhead.
Furthermore, we implement two recovery alternatives, software and hardware-assisted, comparing their impact on the hardware resources and performance in case of a fault. The checking, voting, and switching hardware in the implemented design does not affect the internals of the processor core, allowing for the use of verified RISC-V processor cores without requiring any internal (potentially erroneous) modifications to rapidly build a reliable system.

% The key contributions of this paper are:
% \begin{itemize}
%     \item Combined Dual Core Lockstep and Triple Core Lockstep extensions within a RISC-V-based multi-core cluster.
%     \item Design of a split-lock mechanism to enable runtime-selectable redundant configurations switching in just 550 clock cycles between the available redundant modes. 
%     % \item Exploring the trade-offs for \gls{dmr} and \gls{tmr} on a core-level perspective of the implemented cluster and relating this to classical redundancy mechanisms.
%     \item Design of hardware extensions for fast fault recovery in just 24 clock cycles with only $\sim9\%$ area impact and no timing and performance impacts over the original architecture.
% \end{itemize}
% We propose the first system integrating these functionalities on an open-source RISC-V-based multi-core cluster for fine-tunable reliability vs. performance trade-offs. The key contributions of this paper are:
In summary, we introduce the following key contributions:
\begin{itemize}
    \item A re-configurable computing cluster for Dual-Core Lockstep and Triple-Core Lockstep execution capable of tackling compute-intensive and safety-critical applications. The proposed cluster can be configured so that the computing cores can operate independently if the application requires high-performance capabilities or in Dual/Triple-Core Lockstep mode, depending on the criticality of the executed task.
    \item Robust hardware support for fast fault recovery execution, featuring dedicated Error-Correcting Codes-protected registers to restore the state of the computing cores to the closest reliable state in time. This feature allows the cores to perform cycle-by-cycle backups of their internal state in the protected registers, reducing by $15\times$ the required time to recover from a fault  over the software-based approach.
    \item A runtime-programmable split-lock mechanism allowing for fast switching and re-configuration between the available redundant modes. With these features, it is possible to explicitly define portions of code within a \textit{safety-critical section}, configuring the cores for safe lockstep execution with minimum configuration switching overhead.
\end{itemize}

To validate our proposed approach, we implemented the RISC-V cluster in Global Foundries 22~\si{\nano\meter} technology, achieving up to \SI{430}{\mega\hertz} operating frequency and \SI{1160}{\mega OPS} when configured in independent mode and 617 and 414 MOPS in \gls{dmr} and \gls{tmr} mode, respectively. With only software-based recovery features, the proposed cluster occupies \SI{0.612}{\milli\meter\squared} with just 1.3\% area overhead over the non-redundant configuration, featuring 363 clock cycles time-to-recovery in triple mode. When enhanced with hardware-based recovery features, it provides rapid fault recovery in just 24 clock cycles occupying \SI{0.660}{\milli\meter\squared}, $\sim$9.4\% area overhead over the baseline RISC-V cluster. The proposed split-lock mechanism allows for entering and exiting a redundant mode in just 390 clock cycles for safety-critical code execution.
To foster future research in space-ready computer architecture, we release the proposed architecture as the first fully open-source RISC-V-based multi-core cluster with a finely tunable trade-off between reliability and performance.
\section{Related work}
\label{sec:relatedwork}

Since its initial presentation in Antol \etal~\cite{antol2015vqa}, VQA has thoroughly advanced. Initial developments focused on multimodal fusion modules, which combine visual and text embeddings~\cite{nam2017dual,cadene2019murel}. From basic concatenation and summation~\cite{antol2015vqa}, to more complex fusion mechanisms that benefit from projecting the embeddings to different spaces, numerous approaches have been proposed~\cite{fukui2016multimodal,kim2016hadamard,ben2017mutan}. The addition of attention mechanisms~\cite{kim2018bilinear, nam2017dual,cadene2019murel} and subsequently transformer architectures~\cite{vaswani2017attention} has also contributed to the creation of transformer-based vision-language models, such as LXMERT, which have shown state-of-the-art performances~\cite{tan2019lxmert}. 

More recently, methods have proposed to improve other aspects of VQA, including avoiding shortcut learning and biases~\cite{dancette2021beyond,han2021greedy}, improving 3D spatial reasoning~\cite{banerjee2021weakly}, Out-Of-Distribution (OOD) generalization~\cite{cao2021linguistically,teney2020unshuffling}, improving transformer-based vision-language models~\cite{yang2021auto,zhou2021trar}, external knowledge integration~\cite{ding2022mukea,gao2022transform} and model evaluation with visual and/or textual perturbations~\cite{gupta2022swapmix,walmer2022dual}. With the awareness of bias in VQA training data some works have also addressed building better datasets (\eg, v2.0~\cite{goyal2017making}, VQA-CP~\cite{agrawal2018don}, CLEVR~\cite{johnson2017clevr} and GCP~\cite{hudson2019gqa}).

Furthermore, these developments have now given rise to VQA methods in specific domains. For instance, the VizWiz challenge~\cite{gurari2018vizwiz,gurari2019vizwiz,chen2022grounding} aims at creating VQA models that can help visually impaired persons with routine daily tasks, while there is a growing number of medical VQA works with direct medicine applications~\cite{nguyen2019overcoming,gupta2021hierarchical,vu2020question,zhan2020medical}. 

\paragraph{Consistency in VQA}
Consistency in VQA can be defined as the ability of a model to produce answers that are not contradictory. This is, given a pair of questions about an image, the answers predicted by a VQA model should not be contrary (\eg answering ``Yes" to ``Is it the middle of summer?" and ``Winter" to ``What season is it?"). Due to its significance in reasoning, consistency in VQA has become a focus of study in recent years~\cite{ribeiro2019red,shah2019cycle,gokhale2020vqa,selvaraju2020squinting,jing2022maintaining}. Some of the first approaches for consistency enhancement focused on creating re-phrasings of questions, either by dataset design or at training time~\cite{shah2019cycle}. Along this line, entailed questions were proposed~\cite{ribeiro2019red,gokhale2020vqa}, such that a question generation module was integrated into a VQA model~\cite{ray2019sunny,goel2021iq}, used as a benchmarking method to evaluate consistency~\cite{yuan2021perception} or as a rule-based data-augmentation technique~\cite{ribeiro2019red}. Other approaches tried to shape the embedding space by imposing constraints in the learned representations~\cite{teney2019incorporating} and by imposing similarities between the attention maps of pairs of questions~\cite{selvaraju2020squinting}. Another work~\cite{tascon2022consistency} assumed entailment relations between pairs of questions to regularize training. A more recent approach attempts to improve consistency by using graph neural networks to simulate a dialog in the learning process~\cite{jing2022maintaining}. 

While these approaches show benefits in some cases, they typically only consider that a subset of logical relationships exists between pairs of question-answers or assume that a single relation holds for all QA pairs. Though true in the case of re-phrasings, other question generation approaches cannot guarantee that the produced questions preserve unique relations or that grammatical structure remains valid. Consequently, these methods often rely on metrics that either over or under estimate consistency by relying on these assumptions. In the present work, we propose a strategy to alleviate these limitation by considering all logical relations between pairs of questions and answers. 

\paragraph{Entailment prediction} Natural Language Inference (NLI), or Recognizing Textual Entailment (RTE), is the task of predicting how two input sentences (namely \textit{premise} and \textit{hypothesis}) are related, according to three pre-established categories: entailment, contradiction and neutrality~\cite{maccartney2008modeling}. For example, if the premise is ``A soccer game with multiple males playing" and the hypothesis is ``Some men are playing a sport," then the predicted relation should be an entailment, because the hypothesis logically follows from the premise. Several benchmarking datasets (\eg, SNLI~\cite{young2014image}, MultiNLI~\cite{williams2017broad}, SuperGLUE~\cite{wang2019superglue}, WIKI-FACTCHECK~\cite{sathe2020automated} and ANLI~\cite{nie2019adversarial}) have contributed to the adaption of general-purpose transformer-based models like BERT~\cite{devlin2018bert}, RoBERTa~\cite{liu2019roberta} and DeBERTa~\cite{he2020deberta} for this task. In this work, we will leverage these recent developments to build a model capable of inferring relations between propositions. 

 





\section{Volumetric Rendering with \modelName}
\begin{figure}[t!]
\centering
    \includegraphics[width=0.9\linewidth]{Figure/surface_representation/SDF+V.pdf}
\caption{(a) is the signed distance function (SDF); (b) is the validity probability function $\vldty$; (c) is the watertight surface extracted from (a) SDF; (d) is the open surface extracted from (a) SDF and (b) validity probability. In our mesh extraction process, we set the SDF of the 3D query points with {low validity (here $\vldty < 0.5$)} to \textit{NAN} and extract the open surface with the Marching Cubes algorithm.
%\weikai{1) Better to use a more interesting shape instead of a hand posture for illustrating our key idea. Hand posture sometimes has special meanings (like copy/paste icons=D). 2) For signed distance function (b), why there is a narrow band around the surface (it seems to be as thick as that of validity probability function in (c))? The narrow band has a special meaning, as it keeps the region valid instead of null. So it is better to visualize the surface boundary in (b) as thin curve, while widening the narrow band in (c) a little bit to distinguish them.}
}
\vspace{-1.5em}
\label{fig:surface_representation}
\end{figure}

\begin{figure*}[t]
    \centering
    \includegraphics[width=\textwidth]{figures/Overview_STG.pdf}
    %\includegraphics[width=\textwidth]{figures/pipeline2.pdf}
    \vspace{-20pt}
    \caption{\textbf{Spatio-temporal grounding approach.} 
    % We incorporate both spatial and temporal information in the training process including three modalities. 
    (a)~We want to select frames with possible groundable objects and tasks. To this end, projected word features are matched with respective frame features. (b)~Sinkhorn-knopp optimal transport is then leveraged to ensure the variety of our selected frames. (c)~Based on the selected frames, a global representation is learned to allow for temporal localization as well as (d)~a local representation to ground the action description to the spatial region. 
    %Local contrastive loss on video spatio-temporal and text features to learn multimodal interactions between finer-grained features. Global pairwise contrastive loss on video and text features to pull the features close across modalities in a high-level semantic space. 
    }
    \label{fig:pipeline}
    %\vspace{-10pt}
\end{figure*}
Our representation is able to reconstruct 3D surfaces with arbitrary topologies without 3D ground-truth data for training. As shown in Figure~\ref{fig:surface_representation}, by taking the SDF (Figure~\ref{fig:surface_representation} (a)) and validity probability (Figure~\ref{fig:surface_representation} (b)) into consideration together, we acquire additional information that the bottom line in Figure~\ref{fig:surface_representation} (a) is invalid.
{We discard parts of the reconstructed surface according to the validity score and extract an open surface as shown in Figure~\ref{fig:surface_representation} (d) with the Marching Cubes algorithm~\cite{marching_cubes}. }
% By setting $\vldty(\textbf{p})<0.5$ as \nan, we can extract an open surface as shown in Figure~\ref{fig:surface_representation} (d) with the Marching Cubes algorithm~\cite{marching_cubes}. 
% Hence, the validity probability function acts as a surface eliminator. We describe the details of our method in this section.
%\brandon{I think Figure~\ref{fig:surface_representation}(b) is very confusing. I understand you want to demonstrate the effects of validity, but revierers may not understand this. We claim we deal with arbitrary surfaces, but how to we get a closed SDF field? If I were the reviewer, I would have this question.} \xiaoxu{What I want to demonstrate is that SDF $+$ Validity is able to represent arbitrary shapes. ``How to we get a closed SDF field" is another question that we will answer in the following sections.}
\subsection{Formulation}
\label{sec:method_overview}
Given N images ${\imageGT (k)}_{k=1}^N$ with a resolution of $(W,\ H)$ together with corresponding camera intrinsics, extrinsics, and object masks ${\maskGT (k)}_{k=1}^N$, our goal is to reconstruct the surface of the object. 
%\brandon{In other papers and the dataset, do all camera parameters available or estimated? How to they deal with wild images? Can we also do it? If yes, we need to revise the input formulation.}
%\xiaoxu{Some methods are able to estimate camera parameters in the optimization process, such as NeuS and IDR. However, the visual effect is not as good as the reconstruction with cameras.As suggested by Weikai, We didn't try the reconstruction without camera parameters, since the conclusion will be similar to SOTA.}
The framework of our method is shown in Figure~\ref{fig:pipeline}. Given a sampled pixel $\mathbf{o}$ on an input image, we project it to the 3D space and get the sampled 3D points on the ray emitting from the pixel as 
$\left\{\mathbf{p}\left(t\right)=\mathbf{o}+t\mathbf{v}\ \middle|\ t\geq0\right\}$,
where $\mathbf{o}$ is the center of the camera and $\mathbf{v}$ is the unit direction vector of the ray. 
Then, we predict the signed distance value $f(\mathbf{p}(t))$, validity probability $\vldty(\mathbf{p}(t))$, 
and the RGB value $\clr(\mathbf{p}(t))$ of the points by our fully connected neural networks called \netName. Specifically, \netName~includes:
\begin{itemize}
	\item SDF-Net: a mapping function $f(\cdot):\mathbf{R}^3\rightarrow \mathbf{R}$ to represent the signed distance field. 
% 	We borrowed the implementation of the sdf prediction net of IDR~\cite{yariv2020idr}.
	\item Validity-Net: a mapping function $\vldty(\cdot):\mathbf{R}^3\rightarrow \mathbf{R}$ to represent the validity probability; 
% 	the multilayer perceptron consists of 8 layers. We used the non-linear maps of [10] to improve the learning of high-frequency details. In Validity-Net, we used the ReLU activation between hidden layers and Sigmoid activation for the output.
	\item Color-Net: a mapping function $\clr(\cdot):\mathbf{R}^3\rightarrow \mathbf{R}^3$ to predict the per-point color of the 3D space.
% 	We borrowed the implementation of the color prediction net of IDR~\cite{yariv2020idr}.
\end{itemize}

The outputs of the three networks are delivered to our novel \modelName~renderer to render images and masks from the implicit representations. Our renderer supports both open and closed surfaces, and therefore it provides the capability of reconstructing arbitrary shapes.

{
The predicted mask $\maskPred$ could be inferred from the rendering weights $\wt$ for each sampling point, and the predicted image $\imagePred$ could be calculated from the RGB $\clr(\pt(t))$ and the rendering weights $\wt$:
\vspace{-1em}
\begin{equation}
    \label{equ:pred_imgs}
    \begin{aligned}
        &\maskPred(\mathbf{o}, \mathbf{v}) = \int_{0}^{+\infty}\wt dt,
        \\
        &\imagePred(\mathbf{o}, \mathbf{v}) = \int_{0}^{+\infty}\wt \clr(\pt(t)))dt.
    \end{aligned}
    \vspace{-0.5em}
\end{equation}
}
The predicted masks and images are used for loss calculation during training, which will be illustrated in Section \ref{sec:method_training}. After training is completed, we go through the testing module as shown in Figure \ref{fig:pipeline}; we set the SDFs to NAN for 3D points with $\vldty(\textbf{p})$ less than 0.5, and feed them to Marching Cubes algorithm to produce the final mesh.
\subsection{Construction of \modelName{} Renderer}
\label{sec:method_rendering_model}
According to Equation \ref{equ:pred_imgs}, one key issue in the rendering process is to find an appropriate weight function $\wt$. We split this task into two steps: 1) building a probability density function to estimate volume density from SDF; 2) estimating the weight function $\wt$ from the volume density and the validity probability.

\vspace{-1em}
\paragraph{Construction of Probability Density Function.}
Due to aiming at building arbitrary surfaces, we first introduce the difference between rendering watertight and open surfaces. 
The first difference is the rendering of back-faces. The state-of-the-art watertight surface reconstruction approaches~\cite{wang2021neus,dvr, yariv2020idr} only render the front faces of the surface and ignores the back faces. 
% In volume rendering~\cite{wang2021neus}, the 3D query points with surface normals pointing toward the ray direction share zero-weights; in surface rendering~\cite{dvr, yariv2020idr}, the renderer only triggers the rendering when the ray enters the surface from outside to inside. 
Such a scheme would fail for open surfaces: as shown in Figure~\ref{fig:render_bothsides} (L), the back camera receives an empty rendering of the open surface. While, we render each surface point with ray intersections, as shown in Figure~\ref{fig:render_bothsides} (R).

\input{Figure/render_bothside/render_bothsides}


% The ignorance is safe as the surfaces have already been rendered when the ray entered the surface from outside to inside.
%\weikai{This sentence is very confusing.}

% \brandon{The logic is not reasonable between this and the above paragraph. In above, we talk about rendering both sides; it is about intersections but has nothing related with SDF. More smooth transition needed here.} 
The second difference is the definition of ``inside" and ``outside", which do not exist for non-watertight surfaces. Therefore, we leverage the local surface normal to determine the sign of the distance as in 3PSDF~\cite{chen_2022_3psdf}. For a local region around a surface, we use positive normal direction as pseudo ``outside" with positive-signed distance, and vice versa. 
% \xiaoxu{For points outside the local region around any surface,  
% the SDF still form a continuous field. The SDF of these points will not be rendered in the training stage and will not be exported in the testing stage because of low validity probability.}.
% \weikai{This paragraph is very difficult to understand. We should write from a higher level -- as pointed out by Brandon: SDF only need to render outside surface while open surface needs to render both sides. Use some figures to help explain.}

% A ray can directly exit the surface from the “pseudo” inside to the “pseudo” outside without entering the “pseudo” inside. In order to render all the surfaces, we render each valid surface point if the ray enters the surface from the “pseudo” outside to the “pseudo” inside, and render each valid surface point if the ray exits the surface from the “pseudo” inside to the “pseudo” outside. 
%\brandon{This paragraph is confusing. I think we should explain in such order: 1) We use SDF for open surface; 2) give definitions of "pseudo" inside/outside; 3) explain that in some views, only "pseudo" inside of open surfaces are visible; 4) therefore, we cannot ignore pseudo inside points.} \xiaoxu{Revised}

We expect that the rendering behaves the same when the ray crosses a surface from either direction. 
The state-of-the-art volume rendering work, NeuS~\cite{wang2021neus}, uses logistic density distribution $\phi_{s}(f(\textbf{p}))$, also known as the derivative of the sigmoid mapping function
${\Phi}_\mathbf{s}\left(f(\textbf{p})\right)$, as the probability density function. 
However, it is not applicable in our scenario -- for surfaces with opposite normal directions, $\Phi_{s}\left(f(\textbf{p})\right)$ will lead to different density values as $\Phi_{s}\left(f(\textbf{p})\right)\neq\Phi_{s}\left(-f(\textbf{p})\right)$.
%However, ${\Phi}_\mathbf{s}\left(f(\textbf{p})\right)$ does not meet our requirement: since $\Phi_{s}\left(f(\textbf{p})\right)\neq\Phi_{s}\left(-f(\textbf{p})\right)$, for surfaces with different normal directions, $\Phi_{s}\left(x\right)$ will lead to different density values.

%\brandon{I revised this part, but I feel use $f(p)$ may be better than x, as later equations use $f(p)$. Use different notations for SDF definition may lead to confusion.}\xiaoxu{Revised}

{
We therefore modify the SDF value by flipping its sign in the regions where the SDF value increases along the camera ray. The probability density function is defined as
\begin{equation}
\sigma(\pt) = \phi_{s}(-Sign(\mathbf{v}\cdot\mathbf{n})f(\pt)),
\vspace{-0.5em}
\end{equation}
}
% We therefore introduce a sign adjustment function
% \vspace{-0.5em}
% \begin{equation}
% \gamma(\pt)\ =\ -Sign(cos(\mathbf{v},\mathbf{n})) \in \{-1, 1\}
% \vspace{-0.5em}
% \end{equation}

\noindent where $\mathbf{v}$ is the unit direction vector of the ray and $\mathbf{n}$ is the gradient of the signed distance function. Such definition assures the same rendering behaviors when ray enters the surface from either direction. 

% For any query points $\pt_1$, $\pt_2$ with opposite normals directions and $f\left(\pt_1\right)=-f\left(\pt_2\right)$, we have
% \vspace{-0.5em}
% \begin{equation}
% \left\{
% \begin{array}{lr}
%     \Phi_{s}\left(f(\pt_{1})\cdot\gamma(\pt_{1})\right)
%     =
%     \Phi_{s}\left(f(\pt_{2})\cdot\gamma(\pt_{2}))\right)\\
%     \phi_{s}\left(f(\pt_{1})\cdot\gamma(\pt_{1}))\right)
%     =
%     \phi_{s}\left(f(\pt_{2})\cdot\gamma(\pt_{2}))\right)
% \end{array}
% \right.
% \label{equ:equal_with_gamma}
% \vspace{-0.5em}
% \end{equation}

% Such definition assures the same rendering behaviors when ray enters the surface from either direction. Then our probability density function is defined as $\phi_{s}\left(f(\pt)\cdot\gamma(\pt)\right)$.

% \weikai{We should explain what does the $Sign(.)$ function return. Based on the explanation below, $\gamma(\pt)$ simply returns the opposite sign of the dot product between $\mathbf{V}$ and $\mathbf{n}$?}

% When the ray enters the surface from the “pseudo” outside to the “pseudo” inside, we have $\gamma(\pt)=1$
% $$\mathbf{n}\cdot\mathbf{v}<0\Longrightarrow\gamma(\pt)=-Sign(cos(\mathbf{v}, \mathbf{n})) = 1.$$ 

% When the ray exits the surface from the “pseudo” inside to the “pseudo” outside, we have
% $$\mathbf{n}\cdot\mathbf{v}>0\Longrightarrow\gamma(\pt)=-Sign(cos(\mathbf{v}, \mathbf{n})) = -1.$$ 
% %\brandon{We need careful proof of the differentiable property. This issue was challenged before in CAS' paper.}

% For any query points $\pt_1$, $\pt_2$ with opposite normals directions and $f\left(\pt_1\right)=-f\left(\pt_2\right)$, we have
% \begin{equation}
% \left\{
% \begin{array}{lr}
%     \Phi_{s}\left(f(\pt_{1})\cdot\gamma(\pt_{1})\right)
%     =
%     \Phi_{s}\left(f(\pt_{2})\cdot\gamma(\pt_{2}))\right)\\
%     \phi_{s}\left(f(\pt_{1})\cdot\gamma(\pt_{1}))\right)
%     =
%     \phi_{s}\left(f(\pt_{2})\cdot\gamma(\pt_{2}))\right)
% \end{array}
% \right.
% \label{equ:equal_with_gamma}
% \end{equation}

% % \brandon{No definitions what $\Phi$ and $\phi$ mean.}
% Such definition assures the same rendering behaviors when ray enters the surface from the “pseudo” outside to the “pseudo” inside or the opposite. 

% Then we taking
% $\phi_{s}\left(f(\pt)\cdot\gamma(\pt)\right)$
% as the mapping function from the signed distance field to the probability density field.
%\weikai{The previous context is discussing about the sigmoid function itself. Why we use the derivative of the sigmoid function as the mapping function in the end? Also, the symbol $\Phi$ and $\phi$ are the same symbol just with different cases. Make sure we use two different symbols.}
\vspace{-1em}
\paragraph{Construction of Opaque Density Function.}
According to NeuS\cite{wang2021neus}, the weight function $\wt$ should have two properties: unbiased and occlusion-aware. Similarly, we define unbiased rendering weight $\wt$ with Equation~\ref{equ:define_wt_unbiased}~ and define an occlusion-aware weight function based on the opaque density $\rho(t)$ with Equation~\ref{equ:define_wt_occlusion_aware}.

\vspace{-1em}
\begin{empheq}
    [left=\empheqlbrace]{align}
    \wt & = \frac{
            \phi_{s}(-Sign(\mathbf{v}\cdot\mathbf{n})f(\pt(t)))
    }{
    \int_{-\infty}^{+\infty}\phi_{s}(-Sign(\mathbf{v}\cdot\mathbf{n})f(\pt(t)))
    }
    \label{equ:define_wt_unbiased}\\
    \wt &= \exp(-\int_{0}^{t}\rho(u)du)\rho(t)
    \label{equ:define_wt_occlusion_aware}
\end{empheq}

% % \xiaoxu{
% Following the rules of unbiasedness\cite{wang2021neus}, we define a weight function $w(t)$ on the ray based on the SDF of the scene:
% % }

% \begin{equation}
%     w(t) = 
%     \frac{
%         \phi_{s}(f(\pt(t))\cdot \gamma(\pt(t)))
%     }{
%         \int_{-\infty}^{+\infty}\phi_{s}(f(\pt(t))\cdot \gamma(\pt(t)))
%     }
%     \label{equ:define_wt_unbiased}
% \end{equation}

% Equation~\ref{equ:define_wt_unbiased} is naturally unbiased,
% % \weikai{Why is naturally unbiased? We should provide an proof here or in the supplemental.}
% but not occlusion aware. Accordingly, we further define an opaque density function $\rho(t)$, which is the counterpart of the volume density in the standard volume rendering formulation~\cite{mildenhall2020nerf}, and we compute the weight by
% \begin{equation}
%     w(t) = T(t)\rho(t)
%     \label{equ:define_wt_occlusion_aware}
% \end{equation}
% where $T(t)=\exp(-\int_{0}^{t}\rho(u)du)$ is the accumulated transmittance along the ray.

Solving Equation~\ref{equ:define_wt_unbiased}~and Equation~\ref{equ:define_wt_occlusion_aware}, we get
\vspace{-0.5em}
\begin{equation}
    \rho(t) = \frac{
                -\frac{
                    d\Phi_{s}
                }{
                    dt
                }(-Sign(\mathbf{v}\cdot\mathbf{n})f(\pt(t)))
            }{
                \Phi_{s}(-Sign(\mathbf{v}\cdot\mathbf{n})f(\pt(t))))
            }
\vspace{-0.5em}
\end{equation}
\noindent Please checkout the supplemental for the derivation.
\vspace{-1em}
\paragraph{Discretization.}
% \brandon{We use too many ``like xxx'', ``the same as xxx'' in this and next subsections. After reading, this makes me feel that we only invent $\gamma(\pt)$, nothing else. We should NOT write paragraphs evenly according to each step of the algorithm, but should emphasize more for our NEW parts, like the validity; even if it is quite simple, it is worth to expand much more. For similar steps like Neus parts, even if it is complex, we still should shrink them. 3PSDF is also a simple idea, but Weikai organized well with good emphasis; Resnet is even simpler, but the paper writes well. Please refer those as examples.} 


We adopt the classic discretization scheme in differentiable volumetric rendering\cite{mildenhall2020nerf, wang2021neus} for the opacity and weight function. For a set of sampled points along the ray
$
\{p_{i} = \textbf{o} + t_{i}\textbf{v} |i=1, ..., n, t_{i} < t_{i + 1}\}
$, the rendered pixel color is
\vspace{-1.0em}
\begin{equation}
    \imagePred(\mathbf{o}, \mathbf{v})=\sum_{i=1}^{n}\Pi_{j=1}^{i-1}(1 - \alpha_{j})\alpha_{i}c_{i}
    \label{equ:discrete_color}
    \vspace{-0.5em}
\end{equation}
where $c_i$ is the estimated color for the $i$-th sampling point; $\alpha_{i}$ is the discrete opacity value in SDF rendering
\begin{equation}
\small
\alpha_{i}=\frac{
                \Phi_{s}(-Sign(\mathbf{v}\cdot\mathbf{n})f(\pt(t_{i})))
                 - \Phi_{s}(-Sign(\mathbf{v}\cdot\mathbf{n})f(\pt(t_{i+1})))
            }{
                \Phi_{s}(-Sign(\mathbf{v}\cdot\mathbf{n})f(\pt(t_{i})))
            }
\end{equation}

% where $\alpha_{j}$ is the discrete opacity value, which can be derived from Equation~\ref{equ:discrete_alpha}.
% \begin{align}
% \begin{split}
%     \alpha_{i} &= 1 - \exp(-\int_{t_{i}}^{t_{i+1}}\rho(t)dt)\\
%               &= 1 - \exp(-\int_{t_{i}}^{t_{i+1}}
%                 \frac{
%                     -\frac{
%                         d\Phi_{s}
%                     }{
%                         dt
%                     }(f(\pt(t))\cdot \gamma(\pt(t)))
%                 }{
%                     \Phi_{s}(f(\pt(t)))\cdot \gamma(\pt(t)))
%                 }dt)\\
%               &= 1 - e^{(ln(\Phi_{s}(f(\pt(t_{i+1}))\cdot \gamma(\pt(t_{i+1}))) - ln(\Phi_{s}(f(\pt(t_{i}))\cdot \gamma(\pt(t_{i}))))}\\
%               &= 1 - \frac{
%                 \Phi_{s}(f(\pt(t_{i+1}))\cdot \gamma(\pt(t)))
%               }{
%                 \Phi_{s}(f(\pt(t_{i}))\cdot \gamma(\pt(t)))
%               }\\
%               &=\frac{
%                 \Phi_{s}(f(\pt(t_{i}))\cdot \gamma(\pt(t))) - \Phi_{s}(f(\pt(t_{i+1}))\cdot \gamma(\pt(t)))
%               }{
%                 \Phi_{s}(f(\pt(t_{i}))\cdot \gamma(\pt(t)))
%               }
%     \label{equ:discrete_alpha}
% \end{split}
% \end{align}
% \brandon{Too many equation label numbers: from 7 to 11. maybe 8, 9, 10 or at least 9, 10 are not necessary.}

Now we have built an unbiased and occlusion-aware volume weight function that supports rendering the front and back faces with the SDF representation.

\input{Figure/rendering_weights/render_weights}

\paragraph{Rendering with Validity Probability.}
To render both closed and open surfaces, we multiply the validity probability of the 3D query points to their opacity value in the rendering process. The discrete opacity value $\beta_{i}$ of the $i$-th sampled point is

\vspace{-0.5em}
\begin{equation}
\beta_{i} = \alpha_{i}\cdot \vldty(\pt(t_{i}))
\end{equation}

We show a 2D illustration of rendering two objects with open boundaries in Figure~\ref{fig:render_weights}.
% One example is shown in Figure~\ref{fig:render_weights}. Object 1 and Object 2 are two open-rectangles.
% Ray 1 has four intersections with the objects and Ray 2 has two intersections due to the existence of the gaps (marked with dotted lines).
Ray 2 only has two intersections with the objects due to the existences of open gaps (marked as dotted lines). 
Ray 1 and Ray 2 share the same SDF $f(\pt(t_{i}))$ and discrete opacity value $\alpha_{i}$.
However, according to the validity branch $\vldty(\pt(t_{i}))$, Ray 1 has four valid regions while Ray 2 only has two. By considering the validity probabilities, the discrete opacity value $\beta_{i}$ of the gaps in Ray 2 are set to zero, avoiding generating false surfaces in reconstruction.

Therefore, the final rendered pixel color of a surface is
\begin{equation}
    \imagePred(\mathbf{o}, \mathbf{v})=\sum_{i=1}^{n}\Pi_{j=1}^{i-1}(1 - \beta_{j})\beta_{i}c_{i}
    \vspace{-1em}
    \label{equ:discrete_color_isat}
\end{equation}

% \subsection{Implementation of \netName}
% \label{sec:method_isat_net}
% \brandon{My preference would be shrinking this subsection and merging it into 4.1 ``implementation details'' part.}

\subsection{Training}
\label{sec:method_training}
We supervise the training of \netName{} with five losses. The first three are \textbf{RGB Loss}, \textbf{Mask Loss}, and \textbf{Eikonal Loss}, the same as used in previous neural rendering works~\cite{wang2021neus,yariv2020idr}.
% \brandon{Add ref here}. 
They are defined as
\vspace{-0.5em}
\begin{equation}
    \mathcal{L}_{rgb} = \sum_{i, j}||\imagePred(i,j) - \imageGT(i,j)||\cdot \maskGT(i,j)
    \vspace{-3mm}
\end{equation}
\begin{equation}
    \mathcal{L}_{mask} = \sum_{i, j}BCE(\maskPred(i,j), \maskGT(i,j))
    \vspace{-2mm}
\end{equation}
\begin{equation}
    \mathcal{L}_{eikonal} = \frac{1}{N}\sum_{\textbf{p}}(|\frac{\partial f(\textbf{p})}{\partial \textbf{p}}| - 1)^{2}
    \vspace{-1.5mm}
\end{equation}
where BCE is the binary cross entropy.

% \weikai{Add more insights to the following loss functions.}

\vspace{-0.8em}
\paragraph{Rendering Probability Loss}
In the physical world, the existence of surfaces is binary (exist/not exist). As a result, the validity probability of a 3D sampling point is either 0 (with no surface) or 1 (with surface). We therefore add the binary cross entropy of $\vldty(\textbf{p})$ as an extra regularization:
\vspace{-2mm}
\begin{equation}
    \mathcal{L}_{bce} = \frac{1}{N}\sum_{\textbf{p}} BCE(\vldty(\textbf{p}), \vldty(\textbf{p})).
    \vspace{-3mm}
\end{equation}

% 122: neus: 0.53
\begin{table}[h]
    \small
    \centering
    \begin{tabular}{c|c|c|c|c|c}
        \hline
        CD$\downarrow$ & Ours & NeuS & IDR & NeRF & HFS\\
        \hline
        \hline
        % scan 37 & $1.86$ & $\textbf{0.98}$ & $1.87$ & $2.39$\\
        % scan 24 & $1.75$ & $\textbf{0.83}$ & $1.63$ & $1.15$\\
        scan 55 & $0.47$ & $0.38$ & $0.48$ & $0.66$ & $\textbf{0.37}$\\
        % scan 65 & $1.08$ & $\textbf{0.60}$ & $0.79$ & $1.44$\\
        scan 69 & $0.84$ & $\textbf{0.60}$ & $0.77$ & $1.50$ & $0.66$\\
        scan 83 & $1.28$ & $1.43$ & $1.33$ & $\textbf{1.20}$ & $1.27$\\
        scan 97 & $1.09$ & $\textbf{0.96}$ & $1.16$ & $1.96$ & $1.00$\\
        scan 105 & $\textbf{0.75}$ & $0.78$ & $0.76$ & $1.27$ & $0.86$\\
        scan 106 & $0.76$ & $\textbf{0.52}$ & $0.67$ & $0.66$ & $0.57$\\
        scan 110 & $\textbf{0.80}$ & $1.44$ & $0.90$ & $2.61$ & $1.24$\\
        scan 114 & $0.38$ & $\textbf{0.36}$ & $0.42$ & $1.04$ & $0.41$\\
        scan 118 & $0.56$ & $\textbf{0.46}$ & $0.51$ & $1.13$ & $0.52$\\
        scan 122 & $0.55$ & $\textbf{0.49}$ & $0.53$ & $0.99$ & $\textbf{0.49}$\\
        \hline
        \hline
        average & $0.749$ & $0.742$ & $0.753$ & $1.302$ & $\textbf{0.741}$\\
        \hline
    \end{tabular}
    \caption{Quantitative evals on real-world object reconstruction.
    % \brandon{we have the same performance as Neus in the last row; only bolding ours are OK or not? Also, I think we should add average for both tables.}
    }
    % \vspace{-1.5em}
    \label{table:comparison_watertight}
\end{table}

\vspace{-1em}
\paragraph{Rendering Probability Regularization}
For real-world objects with open structures, the surfaces are sparsely distributed in the 3D space. To prevent \netName{} from predicting redundant surfaces, we introduce a sparsity loss to promote the formation of open surfaces:
\vspace{-2mm}
\begin{equation}
    \mathcal{L}_{sparse} = \frac{1}{N}\sum_{\textbf{p}}\vldty(\textbf{p}).
    \vspace{-3mm}
\end{equation}

% \weikai{We need a definition of the final loss function, with detailed weight for each loss component.}
\noindent We optimize the following loss function
\vspace{-1.5mm}
\begin{align}
\begin{split}
\mathcal{L} &= \mathcal{L}_{rgb}
            + \lambda_{mask} \cdot \mathcal{L}_{mask}
            + \lambda_{eikonal} \cdot \mathcal{L}_{eikonal}\\
            & + \lambda_{bce} \cdot \mathcal{L}_{bce} 
            + \lambda_{sparse} \cdot \mathcal{L}_{sparse}.
\end{split}
\vspace{-1.5mm}
\end{align}
% where $\lambda_{mask} = 0.3$, $\lambda_{eikonal} = 0.1$, $\lambda_{bce} = 0.1$, $\lambda_{sparse} = 0.05$. We use the same parameters in all experiments unless otherwise mentioned.


\section{Experiments}
\subsection{Experiment Setup}

\paragraph{Tasks and Datasets.}
We validate \modelName~using three types of experiments. 
We first conduct multi-view reconstruction for real-world watertight objects to ensure that \modelName~achieves comparable reconstruction quality on watertight surfaces. We conduct this experiment on 10 scenes from the \textit{DTU Dataset}~\cite{dtu}. Each scene contains $49$ or $64$ RGB images and masks with a resolution of $1600\times 1200$.
Second, we reconstruct open surfaces from multi-view images. We run this experiment on eight categories from the \DFD~\cite{zhu2020deep} and five categories from the \MGN~\cite{bhatnagar2019mgn}, which contain clothes with a wide variety of materials, appearance, and geometry, including challenging cases for reconstruction algorithms, such as camisoles.
Finally, we construct an autoencoder, which takes a single image as the input and provides validation on the challenging task of single-view reconstruction on open surfaces. We conduct this experiment on the \textit{dress} category from the \DFD~\cite{zhu2020deep}. We randomly select 116 objects as the training set and 25 objects as the test set.
All experiments are compared with the SOTA methods for better verification.
{To avoid thin closed reconstructions during the training process, we employ a smaller learning rate for the SDF-Net and a larger learning rate for the Validity-Net.}
Please refer to the implementation of \netName{} in the supplementary.
\vspace{-3mm}

\begin{figure*}[htb]
	\vspace{0.0mm}
	\begin{minipage}[t]{0.13\textwidth}
		\centering
		\includegraphics[width=0.9\textwidth]{Figure/comparison_open_d3d/long_sleeve_upper/252/crop/gt_marked.png}
	\end{minipage}
	\begin{minipage}[t]{0.06\textwidth}
		\centering
		\includegraphics[width=1.0\textwidth]{Figure/comparison_open_d3d/long_sleeve_upper/252/crop/gt.png}
	\end{minipage}
	\begin{minipage}[t]{0.13\textwidth}
		\centering
		\includegraphics[width=0.9\textwidth]{Figure/comparison_open_d3d/long_sleeve_upper/252/crop/ours_marked.png}
	\end{minipage}
	\begin{minipage}[t]{0.06\textwidth}
		\centering
		\includegraphics[width=1.0\textwidth]{Figure/comparison_open_d3d/long_sleeve_upper/252/crop/ours.png}
	\end{minipage}
	\begin{minipage}[t]{0.13\textwidth}
		\centering
		\includegraphics[width=0.9\textwidth]{Figure/comparison_open_d3d/long_sleeve_upper/252/crop/neus_marked.png}
	\end{minipage}
	\begin{minipage}[t]{0.06\textwidth}
		\centering
		\includegraphics[width=1.0\textwidth]{Figure/comparison_open_d3d/long_sleeve_upper/252/crop/neus.png}
	\end{minipage}
	\begin{minipage}[t]{0.13\textwidth}
		\centering
		\includegraphics[width=0.9\textwidth]{Figure/comparison_open_d3d/long_sleeve_upper/252/crop/idr_marked.png}
	\end{minipage}
	\begin{minipage}[t]{0.06\textwidth}
		\centering
		\includegraphics[width=1.0\textwidth]{Figure/comparison_open_d3d/long_sleeve_upper/252/crop/idr.png}
	\end{minipage}
	\begin{minipage}[t]{0.13\textwidth}
		\centering
		\includegraphics[width=0.9\textwidth]{Figure/comparison_open_d3d/long_sleeve_upper/252/crop/HFS_marked.png}
	\end{minipage}
	\begin{minipage}[t]{0.06\textwidth}
		\centering
		\includegraphics[width=1.0\textwidth]{Figure/comparison_open_d3d/long_sleeve_upper/252/crop/HFS.png}
	\end{minipage}
	\\
	\vspace{0.0mm}
	\begin{minipage}[t]{0.13\textwidth}
		\centering
		\includegraphics[width=0.9\textwidth]{Figure/comparison_open_d3d/no_sleeve_upper/323/crop/gt_marked.png}
	\end{minipage}
	\begin{minipage}[t]{0.06\textwidth}
		\centering
		\includegraphics[width=1.0\textwidth]{Figure/comparison_open_d3d/no_sleeve_upper/323/crop/gt.png}
	\end{minipage}
	\begin{minipage}[t]{0.13\textwidth}
		\centering
		\includegraphics[width=0.9\textwidth]{Figure/comparison_open_d3d/no_sleeve_upper/323/crop/ours_marked.png}
	\end{minipage}
	\begin{minipage}[t]{0.06\textwidth}
		\centering
		\includegraphics[width=1.0\textwidth]{Figure/comparison_open_d3d/no_sleeve_upper/323/crop/ours.png}
	\end{minipage}
	\begin{minipage}[t]{0.13\textwidth}
		\centering
		\includegraphics[width=0.9\textwidth]{Figure/comparison_open_d3d/no_sleeve_upper/323/crop/neus_marked.png}
	\end{minipage}
	\begin{minipage}[t]{0.06\textwidth}
		\centering
		\includegraphics[width=1.0\textwidth]{Figure/comparison_open_d3d/no_sleeve_upper/323/crop/neus.png}
	\end{minipage}
	\begin{minipage}[t]{0.13\textwidth}
		\centering
		\includegraphics[width=0.9\textwidth]{Figure/comparison_open_d3d/no_sleeve_upper/323/crop/idr_marked.png}
	\end{minipage}
	\begin{minipage}[t]{0.06\textwidth}
		\centering
		\includegraphics[width=1.0\textwidth]{Figure/comparison_open_d3d/no_sleeve_upper/323/crop/idr.png}
	\end{minipage}
	\begin{minipage}[t]{0.13\textwidth}
		\centering
		\includegraphics[width=0.9\textwidth]{Figure/comparison_open_d3d/no_sleeve_upper/323/crop/HFS_marked.png}
	\end{minipage}
	\begin{minipage}[t]{0.06\textwidth}
		\centering
		\includegraphics[width=1.0\textwidth]{Figure/comparison_open_d3d/no_sleeve_upper/323/crop/HFS.png}
	\end{minipage}
	\\
	\vspace{0.0mm}
	\begin{minipage}[t]{0.13\textwidth}
		\centering
		\includegraphics[width=0.9\textwidth]{Figure/comparison_open_d3d/short_sleeve_dress/63/crop/gt_marked.png}
	\end{minipage}
	\begin{minipage}[t]{0.06\textwidth}
		\centering
		\includegraphics[width=1.0\textwidth]{Figure/comparison_open_d3d/short_sleeve_dress/63/crop/gt.png}
	\end{minipage}
	\begin{minipage}[t]{0.13\textwidth}
		\centering
		\includegraphics[width=0.9\textwidth]{Figure/comparison_open_d3d/short_sleeve_dress/63/crop/ours_marked.png}
	\end{minipage}
	\begin{minipage}[t]{0.06\textwidth}
		\centering
		\includegraphics[width=1.0\textwidth]{Figure/comparison_open_d3d/short_sleeve_dress/63/crop/ours.png}
	\end{minipage}
	\begin{minipage}[t]{0.13\textwidth}
		\centering
		\includegraphics[width=0.9\textwidth]{Figure/comparison_open_d3d/short_sleeve_dress/63/crop/neus_marked.png}
	\end{minipage}
	\begin{minipage}[t]{0.06\textwidth}
		\centering
		\includegraphics[width=1.0\textwidth]{Figure/comparison_open_d3d/short_sleeve_dress/63/crop/neus.png}
	\end{minipage}
	\begin{minipage}[t]{0.13\textwidth}
		\centering
		\includegraphics[width=0.9\textwidth]{Figure/comparison_open_d3d/short_sleeve_dress/63/crop/idr_marked.png}
	\end{minipage}
	\begin{minipage}[t]{0.06\textwidth}
		\centering
		\includegraphics[width=1.0\textwidth]{Figure/comparison_open_d3d/short_sleeve_dress/63/crop/idr.png}
	\end{minipage}
	\begin{minipage}[t]{0.13\textwidth}
		\centering
		\includegraphics[width=0.9\textwidth]{Figure/comparison_open_d3d/short_sleeve_dress/63/crop/HFS_marked.png}
	\end{minipage}
	\begin{minipage}[t]{0.06\textwidth}
		\centering
		\includegraphics[width=1.0\textwidth]{Figure/comparison_open_d3d/short_sleeve_dress/63/crop/HFS.png}
	\end{minipage}
	\\
	\vspace{0.0mm}
	\begin{minipage}[t]{0.13\textwidth}
		\centering
		\includegraphics[width=0.9\textwidth]{Figure/comparison_open_d3d/pants/315/crop/gt_marked.png}
	\end{minipage}
	\begin{minipage}[t]{0.06\textwidth}
		\centering
		\includegraphics[width=1.0\textwidth]{Figure/comparison_open_d3d/pants/315/crop/gt.png}
	\end{minipage}
	\begin{minipage}[t]{0.13\textwidth}
		\centering
		\includegraphics[width=0.9\textwidth]{Figure/comparison_open_d3d/pants/315/crop/ours_marked.png}
	\end{minipage}
	\begin{minipage}[t]{0.06\textwidth}
		\centering
		\includegraphics[width=1.0\textwidth]{Figure/comparison_open_d3d/pants/315/crop/ours.png}
	\end{minipage}
	\begin{minipage}[t]{0.13\textwidth}
		\centering
		\includegraphics[width=0.9\textwidth]{Figure/comparison_open_d3d/pants/315/crop/neus_marked.png}
	\end{minipage}
	\begin{minipage}[t]{0.06\textwidth}
		\centering
		\includegraphics[width=1.0\textwidth]{Figure/comparison_open_d3d/pants/315/crop/neus.png}
	\end{minipage}
	\begin{minipage}[t]{0.13\textwidth}
		\centering
		\includegraphics[width=0.9\textwidth]{Figure/comparison_open_d3d/pants/315/crop/idr_marked.png}
	\end{minipage}
	\begin{minipage}[t]{0.06\textwidth}
		\centering
		\includegraphics[width=1.0\textwidth]{Figure/comparison_open_d3d/pants/315/crop/idr.png}
	\end{minipage}
	\begin{minipage}[t]{0.13\textwidth}
		\centering
		\includegraphics[width=0.9\textwidth]{Figure/comparison_open_d3d/pants/315/crop/HFS_marked.png}
	\end{minipage}
	\begin{minipage}[t]{0.06\textwidth}
		\centering
		\includegraphics[width=1.0\textwidth]{Figure/comparison_open_d3d/pants/315/crop/HFS.png}
	\end{minipage}
	\\
	\vspace{0.0mm}
	\begin{minipage}[t]{0.13\textwidth}
		\centering
		\subfloat[GT]{\includegraphics[width=0.9\textwidth]{Figure/comparison_open_mgn/TShirtNoCoat/125611500935128/crop/gt_marked.png}}
	\end{minipage}
	\begin{minipage}[t]{0.06\textwidth}
		\centering
		\includegraphics[width=1.0\textwidth]{Figure/comparison_open_mgn/TShirtNoCoat/125611500935128/crop/gt.png}
	\end{minipage}
	\begin{minipage}[t]{0.13\textwidth}
		\centering
		\subfloat[Ours]{\includegraphics[width=0.9\textwidth]{Figure/comparison_open_mgn/TShirtNoCoat/125611500935128/crop/ours_marked.png}}
	\end{minipage}
	\begin{minipage}[t]{0.06\textwidth}
		\centering
		\includegraphics[width=1.0\textwidth]{Figure/comparison_open_mgn/TShirtNoCoat/125611500935128/crop/ours.png}
	\end{minipage}
	\begin{minipage}[t]{0.13\textwidth}
		\centering
		\subfloat[NeuS]{\includegraphics[width=0.9\textwidth]{Figure/comparison_open_mgn/TShirtNoCoat/125611500935128/crop/neus_marked.png}}
	\end{minipage}
	\begin{minipage}[t]{0.06\textwidth}
		\centering
		\includegraphics[width=1.0\textwidth]{Figure/comparison_open_mgn/TShirtNoCoat/125611500935128/crop/neus.png}
	\end{minipage}
	\begin{minipage}[t]{0.13\textwidth}
		\centering
		\subfloat[IDR]{\includegraphics[width=0.9\textwidth]{Figure/comparison_open_mgn/TShirtNoCoat/125611500935128/crop/idr_marked.png}}
	\end{minipage}
	\begin{minipage}[t]{0.06\textwidth}
		\centering
		\includegraphics[width=1.0\textwidth]{Figure/comparison_open_mgn/TShirtNoCoat/125611500935128/crop/idr.png}
	\end{minipage}
	\begin{minipage}[t]{0.13\textwidth}
		\centering
		\subfloat[HFS]{\includegraphics[width=0.9\textwidth]{Figure/comparison_open_mgn/TShirtNoCoat/125611500935128/crop/HFS_marked.png}}
	\end{minipage}
	\begin{minipage}[t]{0.06\textwidth}
		\centering
		\includegraphics[width=1.0\textwidth]{Figure/comparison_open_mgn/TShirtNoCoat/125611500935128/crop/HFS.png}
	\end{minipage}
\vspace{-1em}
\caption{Comparisons on open surface reconstruction. Row 1 -- 4 are evaluated on \DFD~\cite{zhu2020deep} and Row 5 is evaluated on \MGN~\cite{bhatnagar2019mgn}. \modelName~is able to reconstruct high-fidelity open surfaces while NeuS~\cite{wang2021neus}, IDR~\cite{yariv2020idr} and HFS~\cite{wang2022hfneus} fail to recover the correct topologies.}
\vspace{-1em}
\label{fig:comparison_open}
\end{figure*}

\vspace{-0.5em}


% \paragraph{Implementation of \netName.} 
% We implement \netName~as follows.
% \textbf{SDF-Net}: We borrowed the implementation of the SDF network in DeepSDF~\cite{deepsdf}, which consists of 8 layers
% with hidden layers of width 512, and a single skip connection from the input to the middle layer. We initialize the parameters of the MLP with geometric initialization~\cite{igr}.
% \noindent\textbf{Color-Net}: We borrowed the implementation of the renderer MLP in IDR~\cite{yariv2020idr}, which consists of 4 layers, with hidden layers of
% width $512$. We apply positional encoding~\cite{mildenhall2020nerf} to improve the learning of high-frequencies.
% \noindent \textbf{Validity-Net}: The MLP consists of 8 layers with Xavier initialization. We used the \textit{ReLU} activation between hidden layers and \textit{Sigmoid} for the output.
% \weikai{Move this paragraph to supplemental.}

\vspace{-0.5mm}


\paragraph{Implementation details.}
For the reconstruction experiments on open surfaces, we render the  ground truth point clouds from \DFD~\cite{zhu2020deep} with Pytorch3D~\cite{ravi2020pytorch3d} at a resolution of $256^2$. To get diverse supervision data, we uniformly sample 648 and 64 viewpoints on the unit sphere for \DFD~and \MGN~(MGN), respectively.
For the single view reconstruction experiment, we uniformly sample 64 viewpoints on the unit sphere as the camera positions.
We use an ResNet-18~\cite{He_2016_CVPR} encoder to predict a latent code $\textbf{z}$ describing the surface's geometry and color. We then use the concatenation of $\{\textbf{z}, \pt\}$ as the input to \netName~(decoder) to evaluate the SDF, validity, and color at the query positions. We optimize the autoencoder by comparing the 2D rendering and the ground truth image. 
In the evaluation stage, we accept a single image as the input and directly export the evaluated SDF and validity as 3D mesh.

\vspace{-1em}
\paragraph{Evaluations.} For multiview reconstruction on watertight surfaces, we measure the Chamfer Distance (CD) with \textit{DTU MVS 2014 evaluation toolkit}~\cite{dtu}. For the reconstruction experiments on open surfaces, we measure the CD with the \textit{PCU} Library~\cite{point-cloud-utils}. For all the experiments, we evaluate the result meshes at resolution $512^3$. 
\vspace{-1.5mm}
\subsection{Multiview Reconstruction on Closed Surfaces}
\vspace{-0.5em}
% We investigate if our method is applicable to multi-view reconstruction in real-world scenarios. For this experiment, we do not condition our model and train one model per object.

% \weikai{We only have quantitative comparisons with NeRF?}\xiaoxu{Yes}
% \xiaoxu{NeuS compared with NerF in their paper. I copied the results.}

We compare our approach with the state-of-the-art volume and surface rendering based methods - HFS~\cite{wang2022hfneus}, NeuS~\cite{wang2021neus} and IDR~\cite{yariv2020idr}, and a classic mesh reconstruction and novel view synthesis method -- NeRF~\cite{mildenhall2020nerf}. We report the quantitative results in Table~\ref{table:comparison_watertight}.

We also show visual comparison with a widely-used MVS method: COLMAP~\cite{schoenberger2016sfm, schoenberger2016mvs}. 
We show qualitative results in Fig.~\ref{fig:comparison_watertight}. The results reconstructed with the proposed method show comparable quality compared with the state-of-the-art. 


% To evaluate our approach and baseline methods, we use 8 categories from the DeepFashion3D Dataset~\cite{zhu2020deep}: long sleeve upper, short-sleeve upper, long sleeve dress, short sleeve dress, no sleeve dress, long pants, short pants, and skirt. And we use TODO categories from the MGN Dataset~\cite{bhatnagar2019mgn}: TShirtNoCoat, ShirtNoCoat, LongCoat, Pants, ShortPants. The dataset contain clothes a wide variety of materials, appearance and geometry, including challenging cases for reconstruction algorithms. We render the meshes to generate 216 images with the image resolution of $256 \times 256$. 

% To validate that our approach could represent closed surfaces. We use 5 scenes from the DTU dataset~\cite{jensen2014large}. Each scene contains 49 or 64 images with the image resolution of $1600 \times 1200$.

% 122: neus: 0.53
\begin{table}[h]
    \small
    \centering
    \begin{tabular}{c|c|c|c|c|c}
        \hline
        CD$\downarrow$ & Ours & NeuS & IDR & NeRF & HFS\\
        \hline
        \hline
        % scan 37 & $1.86$ & $\textbf{0.98}$ & $1.87$ & $2.39$\\
        % scan 24 & $1.75$ & $\textbf{0.83}$ & $1.63$ & $1.15$\\
        scan 55 & $0.47$ & $0.38$ & $0.48$ & $0.66$ & $\textbf{0.37}$\\
        % scan 65 & $1.08$ & $\textbf{0.60}$ & $0.79$ & $1.44$\\
        scan 69 & $0.84$ & $\textbf{0.60}$ & $0.77$ & $1.50$ & $0.66$\\
        scan 83 & $1.28$ & $1.43$ & $1.33$ & $\textbf{1.20}$ & $1.27$\\
        scan 97 & $1.09$ & $\textbf{0.96}$ & $1.16$ & $1.96$ & $1.00$\\
        scan 105 & $\textbf{0.75}$ & $0.78$ & $0.76$ & $1.27$ & $0.86$\\
        scan 106 & $0.76$ & $\textbf{0.52}$ & $0.67$ & $0.66$ & $0.57$\\
        scan 110 & $\textbf{0.80}$ & $1.44$ & $0.90$ & $2.61$ & $1.24$\\
        scan 114 & $0.38$ & $\textbf{0.36}$ & $0.42$ & $1.04$ & $0.41$\\
        scan 118 & $0.56$ & $\textbf{0.46}$ & $0.51$ & $1.13$ & $0.52$\\
        scan 122 & $0.55$ & $\textbf{0.49}$ & $0.53$ & $0.99$ & $\textbf{0.49}$\\
        \hline
        \hline
        average & $0.749$ & $0.742$ & $0.753$ & $1.302$ & $\textbf{0.741}$\\
        \hline
    \end{tabular}
    \caption{Quantitative evals on real-world object reconstruction.
    % \brandon{we have the same performance as Neus in the last row; only bolding ours are OK or not? Also, I think we should add average for both tables.}
    }
    % \vspace{-1.5em}
    \label{table:comparison_watertight}
\end{table}


\begin{table}[htbp]
    \small
    \centering
    \begin{tabular}{c|c|c|c|c|c}
        \hline
        & CD ($\times 10^{-3}$) $\downarrow$ & Ours & NeuS & IDR & HFS \\
        \hline
        \hline
        \multirow{9}{*}{D3D} 
        &long slv upper & \textbf{4.483} & $6.864$ & $11.494$ & $9.695$\\
        &short slv upper & \textbf{4.517} & $6.048 $ & $9.043$ & $7.800$\\
        &no slv upper & \textbf{3.418} & $4.856 $ & $17.710$ & $8.576$\\
        &long slv dress & \textbf{4.843} & $6.135$ & $9.203$ & $8.235$\\
        &short slv dress & \textbf{4.276} & $7.951$ & $8.506$ & $7.705$\\
        &no slv dress & \textbf{3.706} & $5.406$ & $6.785$ & $7.565$\\
        &pants & \textbf{5.391} & $11.847 $ & $10.880$ & $16.205$\\
        &dress & \textbf{3.889} & $5.673 $ & $6.983$ & $11.644$\\
        \cline{2-6}
        &average & \textbf{4.315} & $6.847$ & $10.075$ & $9.678$\\
        \hline
        \hline
        \multirow{6}{*}{MGN}
        &LongCoat & \textbf{7.601} & $8.038$ & $12.058$ & $10.398$\\
        &TShirtNoCoat & \textbf{8.481} & $9.910$ & $15.709$ & $13.128$\\
        &ShirtNoCoat & \textbf{5.281} & $8.084$ & $9.509$ & $11.299$\\
        &ShortPants & \textbf{15.324} & $15.480$ & $16.329$ & $18.332$\\
        &Pants & \textbf{9.191} & $12.188$ & $19.931$ & $19.414$\\
        \cline{2-6}
        &average & \textbf{9.176} & $10.740$ & $14.707$ & $14.514$\\
        \hline
    \end{tabular}
    \vspace{-0.2em}
    \caption{Quantitative evaluation on \textit{Deep Fashion 3D
    Dataset}~(D3D)~\cite{zhu2020deep}~with chamfer distance averaged over five examples per category, and \MGN~(MGN)~\cite{bhatnagar2019mgn} with chamfer distance averaged on two examples per category.}
    \vspace{-1em}
    \label{table:comparison_open_d3d}
\end{table}
% \begin{table}[htbp]
    \centering
    \begin{tabular}{|c|c|c|c|}
        \hline
        CD ($\times 10^{-3}$) & Ours & NeuS~\cite{wang2021neus} & IDR~\cite{yariv2020idr} \\
        \hline
        LongCoat & \textbf{5.561} & $7.361$ & $10.348$\\
        TShirtNoCoat & \textbf{TODO} & $TODO$ & $14.784$\\
        ShirtNoCoat & \textbf{4.843} & $6.135$ & $10.091$\\
        ShortPants & \textbf{TODO} & $TODO$ & $14.203$\\
        Pants & \textbf{TODO} & $TODO$ & $15.574$\\
        \hline
    \end{tabular}
    \caption{Quantitative evaluation on \MGN.}
    \label{table:comparison_open_mgn}
\end{table}
\vspace{-0.5em}
\subsection{Multiview Reconstruction on Open Surfaces}
\vspace{-0.5em}
% We investigate if our method is applicable to multi-view reconstruction of open surfaces. For this experiment, we do not condition our model and train one model per object.

We conduct this experiment on eight categories from Deep Fashion 3D~\cite{zhu2020deep} and five categories from the MGN dataset~\cite{bhatnagar2019mgn}. {We compare our approach with two state-of-the-art volume rendering based methods -- NeuS~\cite{wang2021neus} and HFS~\cite{wang2022hfneus}, and a surface rendering based method -- IDR~\cite{yariv2020idr}.}

We report the Chamfer Distance averaged on five examples for each category from \DFD~\cite{zhu2020deep} and report the Chamfer Distance averaged on two examples for each category from \MGN~in Table~\ref{table:comparison_open_d3d}. \modelName~generally provides lower numerical errors compared with the state-of-the-arts.
We show qualitative results in Fig.~\ref{fig:comparison_open}. {\modelName~also provides lower numerical errors in F-score. Please refer to the supplemental for the comparisons.}

% \xiaoxu{
In most cases, NeuS~\cite{wang2021neus} and IDR~\cite{yariv2020idr} are able to reconstruct the geometry with thick, watertight surfaces. While, for the pants in Figure~\ref{fig:comparison_open}, NeuS fails to recover the shape of the waist. \modelName{} is able to reconstruct high-fidelity open surfaces with consistent normals, including the thin straps of the camisoles and dresses.
% }
% \brandon{Add some discussions here.}
\begin{figure}[htbp]
\centering
    \includegraphics[width=0.95\linewidth]{Figure/comparison_singleview_reconstruction/image.pdf}
\vspace{-0.5em}
\caption{With given single-view images, ours predicts
accurate 3D geometry of arbitrary shapes with the autoencoder. \modelName{} achieves CD = $0.0771$ averaged on the 25
objects from the test set, which outperforms NeuS~\cite{wang2021neus} (CD =
$0.0778$) and DVR~\cite{dvr} (CD = $0.0789$). }
\vspace{-1.5em}
\label{fig:comparison_singleview_reconstruction}
\end{figure}

\vspace{-0.5em}
\subsection{Single View Reconstruction on Open Surfaces}
\vspace{-0.5em}
% We investigate if our method is applicable to arbitrary 3D shape reconstruction from single-view images. 
We construct an autoencoder, which accepts a single image as the input, and exports the 3D mesh as the output. For this experiment, we compare our approach against the state-of-the-art single-view reconstruction method: DVR~\cite{dvr} and the volume rendering based method: NeuS~\cite{wang2021neus}.

% We use the dress subset from \textit{Deep Fashion 3D Dataset}~\cite{zhu2020deep} for this experiment. We randomly select $116$ meshes and $28$ meshes as the training and testing sets, respectively. We render $648$ RGB images and the corresponding masks of resolution $256\times256$ per object as the supervision.
% We randomly sample the viewpoint on the unit-sphere to get diverse supervision data.

The qualitative results is shown in Fig~\ref{fig:comparison_singleview_reconstruction}.
% , and the quantitative results in reported in Table~\ref{}.
Our method is able to infer accurate 3D shape representations from single-view images when only using 2D multi-view images and object masks as supervision.
Qualitatively, in contrast to the DVR~\cite{dvr} and NeuS~\cite{wang2021neus} autoencoder, our method is able to reconstruct open surfaces.
Quantitatively, our method achieves CD = $0.0771$, which outperforms NeuS (CD = $0.0778$) and DVR (CD = $0.0789$) averaged on all the 25 objects from the test set.
% Our results rivals the quality of the conventional multi-view reconstruction approach.

% \vspace{-0.5em}
\subsection{Ablation Studies}
\vspace{-0.5em}
\begin{figure}[htb]
    \centering
        \includegraphics[width=1.0\linewidth]{Figure/ablation_validity_regularization/image.pdf}
    \vspace{-1.5em}
    \caption{Ablation study on the regularizations about validity.
    }
\vspace{-2em}
\label{fig:ablation_validity}
\end{figure}

\paragraph{Regularizations on validity.}
We conduct an ablation study on the regularizations about validity, i.e. $\mathcal{L}_{bce}$ and $\mathcal{L}_{sparse}$.
As shown in Figure~\ref{fig:ablation_validity} (c), by setting $\mathcal{L}_{bce}=0$, the renderer tends to generate rendering probability between 0 and 1, thus resulting in noisy faces in the output mesh; as shown in Figure~\ref{fig:ablation_validity} (d), by setting $\mathcal{L}_{sparse}=0$, the renderer will keep the redundant surfaces, instead of learning a validity space as sparse as possible.


% Xiaoxu: sigmoid_factor doesn't affect that much
% \paragraph{Sigmoid factor (enforce consistency between meshing and rendering)}
% Goal: Consistency between the result from regression and from classification
% \input{Figure/ablation_sigmoid_factor/ablation_sigmoid_factor}

\begin{figure}[htb]
    \begin{minipage}[t]{.09\textwidth}
        \centering
        \subfloat[GT]{\includegraphics[width=\textwidth]{Figure/ablation_n_views/320/gt.png}}
    \end{minipage}
    \begin{minipage}[t]{.09\textwidth}
        \centering
        \subfloat[64 views\\CD = 0.00568]{\includegraphics[width=\textwidth]{Figure/ablation_n_views/320/66views.png}}
    \end{minipage}
    \begin{minipage}[t]{.09\textwidth}
        \centering
        \subfloat[32 views\\CD = 0.00632]{\includegraphics[width=\textwidth]{Figure/ablation_n_views/320/34views.png}}
    \end{minipage}
    \begin{minipage}[t]{.09\textwidth}
        \centering
        \subfloat[16 views\\CD = 0.00763]{\includegraphics[width=\textwidth]{Figure/ablation_n_views/320/18views.png}}
    \end{minipage}
    \begin{minipage}[t]{.09\textwidth}
        \centering
        \subfloat[8 views\\CD = 0.01326]{\includegraphics[width=\textwidth]{Figure/ablation_n_views/320/10views.png}}
    \end{minipage}
    \vspace{-1em}
    \caption{Ablation study on multi-view reconstruction with different number of views.}
\vspace{-1em}
\label{fig:ablation_n_views}
\end{figure}

\paragraph{Reconstruct with different number of views.}
We additionally show results on reconstruction with different number of views. As shown in Figure~\ref{fig:ablation_n_views}, our method is able to reconstruct open surfaces even with sparse viewpoints. The reconstruction quality improves with the increase of views, quantitively and qualitatively. 



% Not that important
% \paragraph{Reconstructing thin structures (w. vs. w/o mask regularization)}
% As described in Section~\ref{sec:method_training}, if a pixel is sensitive to dilation or erosion, we consider this pixel as part of the sensitive region. We use higher mask weight for the sensitive regions.
% \begin{figure}[htb]
    \begin{minipage}[t]{.2\textwidth}
        \centering
        \includegraphics[width=\textwidth]{Figure/placeholder.pdf}
        % \subcaption{Image 1.}
    \end{minipage}
    \hfill
    \begin{minipage}[t]{.2\textwidth}
        \centering
        \includegraphics[width=\textwidth]{Figure/placeholder.pdf}
        % \subcaption{Image 2.}
    \end{minipage}  
    \label{fig:ablation_mask}
    \caption{Ablation study on mask regularization.}
\end{figure}

\section{Discussion}
\label{sec:discussion}
In the following, we discuss our main findings (\S\ref{main-findings}) and our work's implications and recommendations for achieving ```Fairness by design'' in UbiComp work (\S\ref{implications}).
\subsection{Main Findings\label{main-findings}}
By screening 523 papers published at IMWUT between 2018 and 2022, we found that only a small portion of 5\% adhered to fairness reporting, while the overwhelming majority thereof focused on accuracy or error metrics. By delving into the smaller number of 49 papers, we surfaced biases in machine learning data and models across several sensitive attributes and application domains that would otherwise remain scattered in the UbiComp literature. Yet, the identified lack of diverse datasets in IMWUT publications could result in biases remaining undetected in the absence of heterogeneous demographics. To quantify such biases, included papers primarily employed performance evaluation instead of fairness metrics, while challenges in fairness assessment were found in regression and multi-class classification scenarios. Similar to other communities, defining fairness in UbiComp was not a simple task and involved considering its sociotechnical context, its ethical risks, and opportunities. Nevertheless, in an effort to employ fairness in practice ---sensitive attributes aside--- we found that the community has been striving for generalizability through ablation studies, real-world deployments, and personalization.


\subsection{Implications and Recommendations\label{implications}}
Drawing from these findings and borrowing from the ``Privacy by design'' literature \cite{fjeld2020principled,cavoukian2009privacy}, we propose a ``Fairness by design'' equivalent, requiring AI developers and researchers to consider data and model fairness concerns from the very beginning of any AI project or system design. It is, thus, a proactive and preventative approach that prioritizes fairness as a core value in the development and implementation of UbiComp technologies, products, and services. To facilitate the community achieving ``Fairness by design'', we next discuss recommendations for integrating fairness into the entire ML pipeline of UbiComp studies. These recommendations span two fronts, one concerning the data and the other the model. \\


\begin{figure}[tb!]
  \centering
  \includegraphics[width=0.75\linewidth]{figures/recommendations.pdf}
  \caption{\textbf{Recommendations for ``Fairness by design'' in UbiComp.} Actions to be taken by researchers for performing fairness assessments in both data and models in UbiComp works. Fairness needs to be considered from the very start of a project.\label{fig:recommendations}} 
\end{figure}


    






\noindent \textbf{Data Collection.}  
Prior to the problem definition, researchers should identify the types of fairness-related harms relevant to their work (e.g., quality-of-service, allocation, stereotyping, and erasure harms \cite{crawford2017trouble}). For example, in an AFib detection application, quality-of-service harms could occur if the model had a substantially different performance for different ages, while allocation harms could occur if such difference led to one group unfairly receiving better care than another. Additionally, it is important to consider the demographic groups ---including historically marginalized groups (e.g., based on gender, race, and ethnicity)--- that might be harmed. We should also consider groups that are relevant to a particular scenario or deployment setting. For example, in a depression screening application, gender could be relevant as a sensitive attribute due to reported gender differences in the disorder's signals \cite{parker2010gender}. Relevant attributes can be identified either from the theoretical literature or through fairness literature related to the target application domain.

When defining the problem statement, researchers should also have in mind the generalizability of the prediction task (e.g., applicable across demographic groups). To achieve that, it is of prime importance to consider an adequate enough sample size that would enable fairness to be studied (e.g., through sub-group analyses). For example, an AFib detection system showed poor performance in people with abnormal heart rhythms other than AFib, most likely because its data annotation scheme assigned Normal sinus rhythm (NSR) and other types of heart rhythms to the same Non-AFib category (i.e., binary classification), due to the limited number of subjects with different types of heart rhythms \cite{10.1145/3397313}. Additionally, datasets in UbiComp are either self-collected or well-established benchmarks (e.g., those found in the UCI repository\footnote{UCI Machine Learning Repository: \url{https://archive.ics.uci.edu/ml/index.php}}) used to evaluate new models. For self-collected data, researchers should strive for a diverse representation of human participants in both the recruitment and the data annotation phase. Considering that the models encode the biases of the labels, they should not only be assessed by multiple people to ensure agreement but also strive for demographic diversity amongst them. For benchmark data, researchers should think carefully about the pre-processing stage. Unlike other fields where the datasets are provided out-of-the-box, in UbiComp, it is not uncommon to require further slicing or windowing in order to be used for predictive modeling. For example, applying a sliding window method can generate thousands of samples from a sensor signal that belongs to a single user. This carries the risk of providing virtually ``enough'' samples for training, which, however, come from a handful of users. As a result, the model does not learn generalizable patterns. 

As with any data science project, data validation methods  play an important role in ensuring the results' robustness. The same holds true when it comes to fairness. Typical data validation methods, therefore, should also be applied across sensitive attributes. For example, inspecting outliers that can consistently fall into particular demographic groups, data that are not missing at random and affect certain groups, or other kinds of data anomalies (e.g., measurement error due to a device malfunction or device differences). Regarding the latter, devices such as smartwatches offer model-based estimates for many well-being features. For instance, the measurement error of a heart-rate prediction model can propagate to every downstream application. If the original device has not been validated across different groups, this can affect every possible application that is using such data. More broadly, visualization tools (e.g., What-If Tool\footnote{\url{https://pair-code.github.io/what-if-tool/}}, FairLens\footnote{\url{https://www.synthesized.io/fairlens}}, Tensorflow's Fairness Indicators\footnote{\url{https://github.com/tensorflow/fairness-indicators}}) may help surface any potential data anomalies and help correct them before they creep into models.

At the same time, the community itself could implement a mandatory data statement policy, requiring authors to report sensitive attributes concerning their participant samples. This builds on recent quests that advocate for data excellence~\cite{sambasivan2021everyone}, for example, by making data statements and datasheets for datasets mandatory for authors submitting their work.\\

\noindent \textbf{Model Training and Evaluation.} Until recently, most ML-based UbiComp applications employed some sort of feature engineering in order to extract statistical summaries from sensor data. However, during the past couple of years, this step has been automated since we have witnessed a remarkable consolidation of deep-learning models and architectures such as Convolutional Neural Networks \cite{krizhevsky2017imagenet} and Transformers  \cite{vaswani2017attention}. As a result, such models are being used as generic feature extractors for different data types -- be it images, text, time-series, or video. A side effect of this consolidation is that recently proposed mitigation methods can be applied across a wide range of models, regardless of input data types. For example, one approach modifies the weights of training samples or changes features and labels based on these attributes \cite{calmon2017optimized}. Another approach learns fair representations that remove correlations between sensitive and non-sensitive attributes \cite{zemel2013learning}, while a third approach involves dividing the training data into subgroups and modifying them to have similar feature distributions across subgroups \cite{feldman2015certifying}. Additionally, some methods operate on the latent space of the models by obfuscating information about protected attributes \cite{zemel2013learning}. Overall, these techniques aim to promote fairness and reduce bias by focusing on various aspects of data and model architectures.

Yet enhancing fairness in machine learning requires a means to quantify it. As UbiComp systems blend into the real world, we realize that single evaluation metrics struggle to reflect the success criteria of ML models. As such, monitoring a multitude of metrics becomes the norm, and this is where we believe that monitoring and reporting fairness metrics across different groups should become standard practice. However, we acknowledge that sometimes it might not be feasible to collect data from representative demographics, especially for smaller pilot studies. In these situations, researchers should aim for a diverse user sample based on assumptions about relevant sensitive attributes. This approach can help uncover potential biases in the data and models, which can then be addressed in later stages of development. Alternatively, researchers can leverage advances in generative models to synthesize data covering multiple sensitive attributes and potential intersections \cite{chaudhari2022fairgen}. 

This is where the concept of intersectional fairness comes in. Intersectional fairness means designing and training algorithms to account for the complex ways that different social identities can intersect and impact a person's experiences and outcomes. UbiComp technologies for diagnosing heart disease and monitoring vital signs provide an exemplary case. As reports suggest, differences in coronary heart disease are based on gender \cite{maas2010gender}, socioeconomic status \cite{schultz2018socioeconomic}, and race \cite{fincher2004racial}. In such cases, it is important to ensure that the models do not perpetuate existing biases and inequalities by failing to account for intersectional differences in health outcomes and access to healthcare---the biases encountered by a Black woman from a low socioeconomic background may not be the same as those experienced by a White woman from a high socioeconomic background.

Beyond traditional notions of fairness, such as directly discriminating based on sensitive attributes, we should also consider indirect notions of fairness. For example, within the paradigm of distributed/federated learning, the resource allocation of participating devices may also reflect the demographic and socio-economic information of owners, which makes the exclusion of such clients unfair in terms of participation.  Cheaper devices cannot support the execution of large models and are either excluded or dropped together with their unique data \cite{horvath2021fjord, cho2022flame}. Last, as models are being deployed in real applications, we should monitor their performance in real time and adjust for data and fairness drift \cite{ghosh2022faircanary} by ensuring that models produce fair predictions independent of changes in input data and demographics. 
























\section{Conclusion}
\label{sec:conclusion}
The field of mobile, wearable and ubiquitous computing (UbiComp) faces significant challenges in ensuring fairness in the development of ML-based UbiComp technologies. Although efforts have been made to address biases, only a small percentage of publications in the Proceedings of the ACM IMWUT journal focus on fairness reporting and enhancement mechanisms. Sensitive attributes such as race, nationality, and language are often overlooked, while it is evident that there is a need for more diverse sample recruitment to ensure that the benefits of these technologies are shared equally across all members of society. The lack of a universal fairness definition, metric, or ``fair'' threshold that applies to different applications poses a sociotechnical challenge. UbiComp researchers must be explicit and transparent about their fairness priorities, definitions, and assumptions, making trade-offs between competing priorities, ethical risks, and opportunities. Despite these challenges, the UbiComp community strives for ``fairer'' models by conducting and reporting ablation studies, in-the-wild vs. in-the-lab experiments, and personalized model development. Ultimately, the UbiComp community must continue to prioritize fairness to ensure that the development of these technologies leads to just and equitable outcomes.

{\small
\bibliographystyle{ieee_fullname}
\bibliography{egbib}
}

\end{document}
