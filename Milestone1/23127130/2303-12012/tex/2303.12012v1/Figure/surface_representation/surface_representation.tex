\begin{figure}[t!]
\centering
    \includegraphics[width=0.9\linewidth]{Figure/surface_representation/SDF+V.pdf}
\caption{(a) is the signed distance function (SDF); (b) is the validity probability function $\vldty$; (c) is the watertight surface extracted from (a) SDF; (d) is the open surface extracted from (a) SDF and (b) validity probability. In our mesh extraction process, we set the SDF of the 3D query points with {low validity (here $\vldty < 0.5$)} to \textit{NAN} and extract the open surface with the Marching Cubes algorithm.
%\weikai{1) Better to use a more interesting shape instead of a hand posture for illustrating our key idea. Hand posture sometimes has special meanings (like copy/paste icons=D). 2) For signed distance function (b), why there is a narrow band around the surface (it seems to be as thick as that of validity probability function in (c))? The narrow band has a special meaning, as it keeps the region valid instead of null. So it is better to visualize the surface boundary in (b) as thin curve, while widening the narrow band in (c) a little bit to distinguish them.}
}
\vspace{-1.5em}
\label{fig:surface_representation}
\end{figure}