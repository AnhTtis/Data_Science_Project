\documentclass[12pt]{article}
\usepackage{amssymb}
\usepackage{amsmath}
\usepackage{graphicx}
\usepackage{indentfirst}
\usepackage{cite}
\usepackage{appendix}

\linespread{1.6}

\topmargin=0in
\headheight=0in
\headsep=0in
\oddsidemargin=0pt
\evensidemargin=0pt
\marginparwidth=0in
\marginparsep=0in
\textheight=235mm
\textwidth=160mm

\allowdisplaybreaks

\begin{document}

\title{Cross sections for inelastic $K$+$\phi$ scattering}
\author{Yi-Hao Pan and Xiao-Ming Xu$^1$}
\date{}
\maketitle \vspace{-1cm}
\centerline{$^1$Department of Physics, Shanghai University, Baoshan,
Shanghai 200444, China}

\begin{abstract}
In the first Born approximation we study the reactions $K\phi\to\pi K$, 
$\rho K$, $\pi K^*$, and $\rho K^*$ with quark-antiquark annihilation and
creation. Transition amplitudes are derived with the development in spherical
harmonics of the relative-motion wave functions of the two initial mesons and
of the two final mesons so that parity is conserved and the total angular
momentum of the final mesons equals the one of the initial mesons. 
Unpolarized cross sections are calculated from the transition amplitudes that
also contain mesonic quark-antiquark relative-motion wave functions and 
transition potentials for quark-antiquark annihilation and creation. Notable 
temperature dependence of the cross sections is shown. While the cross sections
for $K\phi\to\rho K$, $K\phi\to\pi K^*$, and $K\phi\to\rho K^*$ may be of the
millibarn scale, the cross section for $K\phi\to\pi K$ is very small.
\end{abstract}

\noindent
Keywords: meson-meson scattering, quark-antiquark annihilation and creation, 
quark potential model.

\noindent
PACS: 25.75.-q; 24.85.+p; 12.38.Mh

\clearpage
\vspace{0.5cm}
\leftline{\bf I. INTRODUCTION}
\vspace{0.5cm}

Since enhanced $\phi$ yield was suggested as a signature for the 
formation of quark-gluon plasmas \cite{RM,AS}, many measurements on $\phi$ 
mesons have been made in relativistic heavy-ion collisions such as Au-Au 
collisions at the BNL Relativistic Heavy Ion Collider 
\cite{STAR2007,STAR2009PRC,STAR2009PLB,PHENIX2011,STAR2013,STAR2016,STAR2020,
STAR2022,PHENIX2023} 
or Pb-Pb collisions at the CERN Large Hadron Collider 
\cite{ALICE2015,ALICE2017,ALICE2020,ALICE2022}. Measured ratios like 
$\phi/\pi$, $\phi/K$, and $\Omega/\phi$ and the $\phi$ nuclear 
modification factor as a function of transverse momentum show
enhancement of $\phi$ mesons produced in relativistic heavy-ion collisions 
relative to $p$+$p$ collisions. This indicates that strange quarks and strange 
antiquarks are produced in parton-parton scattering in initial nucleus-nucleus
collisions and deconfined matter. Combination of a strange quark and a strange 
antiquark forms a $\phi$ meson at hadronization of the quark-gloun plasma.
The $\phi$ meson in collisions with hadrons in hadronic matter may be broken,
and this changes the $\phi$ yield.
Therefore, studying inelastic hadron-$\phi$ scattering is a fundamental issue
in relativistic heavy-ion collisions.

Hadron-$\phi$ reactions can be studied in hadron degrees of freedom
\cite{BR,KS,Haglin,SH,CLK,AK,MKAN} or quark degrees of freedom \cite{LX,XW}.
Starting from an effective meson Lagrangian, Feynman diagrams with one-kaon
exchange were considered, and squared invariant amplitudes for 
$\pi \phi \to K\bar{K}^* +K^*\bar{K}$, $\rho \phi \to K\bar K$, and 
$\phi \phi \to K\bar K$ were provided in Ref. \cite{KS}. 
Using a Lagrangian that is based on an effective
theory in which vector mesons are identified as the dynamical gauge bosons
of the hidden $U(3)_V$ local symmetry in the $U(3)_L \times U(3)_R/U(3)_V$
nonlinear sigma model, large cross sections for $K^* \phi \to \pi K$ and
$K\phi \to \pi K^*$ are shown in Ref. \cite{MKAN}. With a $\pi \phi \rho$
coupling cross sections for $\phi N \to \pi N$, $\phi N \to \rho N$, and
$\phi N \to \pi \Delta$ were obtained in Ref. \cite{CLK}. With a 
$N\Lambda K$ coupling cross sections for $\phi N \to K\Lambda$ are shown to be
much larger than those for $\phi N \to \pi N$, $\phi N \to \rho N$, and
$\phi N \to \pi \Delta$. Experimental efforts to extract the $\phi +N$ total 
cross section from $d(\gamma,pK^+K^-)n$ have been made by the CLAS 
Collaboration \cite{CLAS}. The $\phi +N$ cross section may lead to a
difference in $\phi$ production between $\pi^-$-induced reactions on C
and W targets \cite{HADES}. Inelastic $\pi+\phi$ scattering and inelastic 
$\rho+\phi$ scattering were studied in Ref. \cite{LX} in the quark interchange 
mechanism \cite{BS,Swanson}. 
Adopting temperature dependence in a quark potential, mesonic quark-antiquark
wave functions, and meson masses, prominent temperature-dependent cross 
sections for the inelastic $\pi+\phi$ and $\rho+\phi$ scattering in hadronic 
matter were obtained \cite{LX}. Besides pions and rho mesons, kaons in 
hadronic matter also interact with $\phi$
mesons. However, quark-level study of inelastic $K+\phi$ scattering has not 
been done. Moreover, temperature dependence of the inelastic $K+\phi$ 
scattering
is unexplored both experimentally and theoretically. Therefore, the present
work aims to study the inelastic $K+\phi$ scattering and their temperature
dependence.

Some meson-meson reactions may be dominated by the process where a quark and an
antiquark annihilate into a gluon and subsequently the gluon creates another 
quark-antiquark pair. Such quark-antiquark annihilation and creation has been
used in Refs. \cite{SXW,WX} to obtain unpolarized cross sections for 
the reactions:
$\pi \pi \to \rho \rho$, $K \bar
{K} \to K^* \bar {K}^\ast$, $K \bar{K}^\ast \to K^* \bar{K}^\ast$, $K^\ast
\bar{K} \to K^* \bar{K}^\ast$, $\pi \pi \to K \bar K$, $\pi \rho \to
K \bar {K}^\ast$, $\pi \rho \to K^* \bar{K}$, $K \bar {K} \to \rho \rho$,
$K \bar {K} \to K \bar {K}^\ast,~ K \bar{K} \to K^* \bar{K},
~\pi K \to \pi K^\ast,~ \pi K \to \rho K,
~\pi \pi \to K \bar{K}^\ast,~ \pi \pi \to K^\ast \bar{K},
~\pi \pi \to K^\ast \bar{K}^\ast, ~\pi \rho \to K \bar{K},
~\pi \rho \to K^\ast \bar{K}^\ast, ~\rho \rho \to K^\ast \bar{K}^\ast,
~K \bar{K}^\ast \to \rho \rho$, and $K^* \bar{K} \to \rho \rho$.
The $s$ quark (or the $\bar s$ antiquark) of a kaon may annihilate with the
$\bar s$ antiquark (or the $s$ quark) of a $\phi$ meson to produce a gluon, and
subsequently the gluon splits into a $u\bar u$ or $d\bar d$ pair. The $u\bar u$
or $d\bar d$ pair
combines with spectator constituents of the $K$ and $\phi$
mesons to form two mesons that are not $\phi$ mesons. Quark-antiquark 
annihilation and creation thus leads to inelastic $K+\phi$ scattering. By 
contrast quark interchange does not cause inelastic $K+\phi$ scattering. The 
mechanism that governs the inelastic $K+\phi$ scattering completely differs
from the
mechanism that governs the inelastic $\pi+\phi$ and $\rho+\phi$ scattering.
Therefore, with quark-antiquark annihilation and creation in the first Born 
approximation, in the present work we study the reactions: $K\phi\to\pi K$, 
$K\phi\to\rho K$, $K\phi\to\pi K^*$, and $K\phi\to\rho K^*$. 

This paper is organized as follows. In the next section we derive 
transition-amplitude formulas for 2-to-2 meson-meson scattering that are 
driven by quark-antiquark annihilation and creation. Numerical results and 
relevant discussions are given in Sec. III. A summary is in the last 
section.

\vspace{0.5cm}
\leftline{\bf II. FORMALISM}
\vspace{0.5cm}

The reaction $A(q_1\bar{q}_1)+B(q_2\bar{q}_2) \to
C(q_{3}\bar{q}_{1})+D(q_{2}\bar{q}_{4})$ 
($A(q_1\bar{q}_1)+B(q_2\bar{q}_2) \to 
C(q_{1}\bar{q}_{4})+D(q_{3}\bar{q}_{2})$) 
takes place due to that quark $q_1$ ($q_2$) and antiquark $\bar{q}_2$ 
($\bar{q}_1$) in the initial mesons
annihilate into a gluon and subsequently the gluon creates quark $q_3$ 
and antiquark $\bar{q}_4$. The two processes $q_1+\bar{q}_2 \to q_{3}
+\bar{q}_4$ and $\bar{q}_1+q_2 \to q_{3}+\bar{q}_4$ give rise to
the two transition potentials
$V_{{\rm a}q_1\bar{q}_2}$ and $V_{{\rm a}\bar{q}_1q_2}$, respectively.
Denote by $E_{\rm i}$ and $\vec{P}_{\rm i}$ ($E_{\rm f}$ and $\vec{P}_{\rm f}$)
the total energy and the total momentum of the two initial (final) mesons,
respectively. Let $E_A$ ($E_B$, $E_C$, $E_D$) be the energy of
meson $A$ ($B$, $C$, $D$), and
$V$ the volume where every meson wave function is normalized.
The $S$-matrix element for $A+B \to C+D$ is
\begin{equation}
S_{\rm fi}  = \delta_{\rm fi} -(2\pi)^4 i \delta (E_{\rm f} - E_{\rm i})
\delta^3 (\vec{P}_{\rm f} - \vec{P}_{\rm i})
\frac {{\cal M}_{{\rm a}q_1\bar{q}_2}+{\cal M}_{{\rm a}\bar{q}_1q_2}}
{{V^2}\sqrt{2E_A2E_B2E_C2E_D}},
\end{equation}
where ${\cal M}_{{\rm a}q_1\bar{q}_2}$ and
${\cal M}_{{\rm a}\bar{q}_1q_2}$
are the transition amplitudes given by
\begin{eqnarray}\label{amp_q1qb2}
{\cal M}_{{\rm a}q_1 {\bar q}_2}&=&\frac 
{{(m_{q_3}+m_{\bar{q}_1})}^3}{{m^{3}_{\bar{q}_1}}}\sqrt {2E_A2E_B2E_C2E_D}
\nonumber\\
&&
\times \int d\vec{r}_{q_1\bar{q}_1} d\vec{r}_{q_2\bar{q}_4} 
d\vec{r}_{q_3\bar{q}_1, q_2\bar{q}_4} \psi_{CD}^+ V_{{\rm a}q_1\bar{q}_2}
\nonumber\\
&&
\times \psi_{AB} e^{i\vec{p}_{q_1\bar{q}_1,q_2\bar{q}_2}\cdot
\vec{r}_{q_1\bar{q}_1,q_2\bar{q}_2}-i\vec{p}_{q_3\bar{q}_1, q_2\bar{q}_4}\cdot
\vec{r}_{q_3\bar{q}_1,q_2\bar{q}_4}},
\end{eqnarray}
\begin{eqnarray}\label{amp_qb1q2}
{\cal M}_{{\rm a}\bar{q}_1q_2}&=&\frac 
{{(m_{q_1}+m_{\bar{q}_4})}^3}{m_{q_1}^3}\sqrt {2E_A2E_B2E_C2E_D}
\nonumber\\
&&
\times \int d\vec{r}_{q_1\bar{q}_1} d\vec{r}_{q_3\bar{q}_2} 
d\vec{r}_{q_1\bar{q}_4, q_3\bar{q}_2} \psi_{CD}^+ V_{{\rm a}\bar{q}_1q_2}
\nonumber\\
&&
\times \psi_{AB} e^{i\vec{p}_{q_1\bar{q}_1,q_2\bar{q}_2}\cdot
\vec{r}_{q_1\bar{q}_1,q_2\bar{q}_2}-i\vec{p}_{q_1\bar{q}_4, q_3\bar{q}_2}\cdot
\vec{r}_{q_1\bar{q}_4,q_3\bar{q}_2}},
\end{eqnarray}
where $m_{q_{1}}$ ($m_{\bar{q}_{1}}$, $m_{q_{3}}$, $m_{\bar{q}_{4}}$) is the 
mass of $q_{1}$ ($\bar{q}_{1}$, $q_{3}$, $\bar{q}_{4}$); $\vec {r}_{ab}$ the 
relative coordinate of constituents $a$ and $b$;
$\vec{r}_{q_1\bar {q}_1,q_2\bar {q}_2}$ 
($\vec {r}_{q_3\bar {q}_1,q_2\bar {q}_4}$, 
$\vec {r}_{q_1\bar {q}_4, q_3\bar {q}_2}$) the relative coordinate
of $q_1\bar {q}_1$ and $q_2\bar {q}_2$ ($q_3\bar {q}_1$ and $q_2\bar {q}_4$, 
$q_1\bar {q}_4$ and $q_3\bar {q}_2$);
$\vec {p}_{q_1\bar {q}_1,q_2\bar {q}_2}$ 
($\vec {p}_{q_3\bar {q}_1, q_2\bar {q}_4}$, 
$\vec {p}_{q_1\bar {q}_4, q_3\bar {q}_2}$) the relative momentum 
of $q_1\bar {q}_1$ and $q_2\bar {q}_2$ ($q_3\bar {q}_1$ and $q_2\bar {q}_4$, 
$q_1\bar {q}_4$ and $q_3\bar {q}_2$); $\psi_{AB}$ ($\psi_{CD}$) the wave 
function of mesons $A$ and $B$ ($C$ and $D$), and $\psi_{AB}^+$ 
($\psi_{CD}^+$) the Hermitean conjugate of $\psi_{AB}$ ($\psi_{CD}$).
The wave function of mesons $A$ and $B$ is
\begin{equation}\label{eq_6}
\psi_{AB} =\phi_{A\rm color} \phi_{B\rm color} \phi_{A\rm rel} \phi_{B\rm rel}
\chi_{S_A S_{Az}} \chi_{S_B S_{Bz}} \varphi_{AB\rm flavor},
\end{equation}
and the wave function of mesons $C$ and $D$ is
\begin{equation}\label{teacher_1}
\psi_{CD} =\phi_{C\rm color} \phi_{D\rm color} \phi_{C\rm rel} \phi_{D\rm rel}
\chi_{S_C S_{Cz}} \chi_{S_D S_{Dz}} \varphi_{CD\rm flavor},
\end{equation}
where $S_A$ ($S_B$, $S_C$, $S_D$) is the spin of meson $A$ ($B$, $C$, $D$) with
its
magnetic projection quantum number $S_{Az}$ ($S_{Bz}$, $S_{Cz}$, $S_{Dz}$); 
$\phi_{A\rm color}$
($\phi_{B\rm color}$, $\phi_{C\rm color}$, $\phi_{D\rm color}$),
$\phi_{A\rm rel}$
($\phi_{B\rm rel}$, $\phi_{C\rm rel}$, $\phi_{D\rm rel}$), 
and $\chi_{S_A S_{Az}}$ ($\chi_{S_B S_{Bz}}$, $\chi_{S_C S_{Cz}}$, 
$\chi_{S_D S_{Dz}}$)
are the color wave function, the quark-antiquark relative-motion wave function,
and the spin wave function of meson $A$ ($B$, $C$, $D$), respectively;
$\varphi_{AB\rm flavor}$ ($\varphi_{CD\rm flavor}$) is the flavor wave
function of mesons $A$ and $B$ ($C$ and $D$).

The development in spherical harmonics of the relative-motion wave function
of mesons $A$ and $B$ (aside from a normalization constant) is given by
\begin{eqnarray}\label{teacher_5,6}
e^{i\vec{p}_{q_1\bar{q}_1,q_2\bar{q}_2} \cdot
\vec{r}_{q_1\bar{q}_1,q_2\bar{q}_2}} & = & 4\pi
\sum\limits_{L_{\rm i}=0}^{\infty}
\sum\limits_{M_{\rm i}=-L_{\rm i}}^{L_{\rm i}}
i^{L_{\rm i}} j_{L_{\rm i}} (\mid \vec{p}_{q_1\bar{q}_1,q_2\bar{q}_2} \mid
r_{q_1\bar{q}_1,q_2\bar{q}_2})
\nonumber\\
& & Y_{L_{\rm i}M_{\rm i}}^\ast
(\hat{p}_{q_1\bar{q}_1,q_2\bar{q}_2}) Y_{L_{\rm i}M_{\rm i}}
(\hat{r}_{q_1\bar{q}_1,q_2\bar{q}_2}),
\end{eqnarray}
and the development in spherical harmonics of the relative-motion wave function
of mesons $C$ and $D$ leads to
\begin{eqnarray}
e^{-i\vec{p}_{q_3\bar{q}_1,q_2\bar{q}_4} \cdot
\vec{r}_{q_3\bar{q}_1,q_2\bar{q}_4}} & = & 4\pi
\sum\limits_{L_{\rm f}=0}^{\infty}
\sum\limits_{M_{\rm f}=-L_{\rm f}}^{L_{\rm f}}
i^{L_{\rm f}} (-1)^{L_{\rm f}} j_{L_{\rm f}} 
(\mid \vec{p}_{q_3\bar{q}_1,q_2\bar{q}_4} \mid r_{q_3\bar{q}_1,q_2\bar{q}_4})
\nonumber\\
& & Y_{L_{\rm f}M_{\rm f}}^\ast (\hat{p}_{q_3\bar{q}_1,q_2\bar{q}_4})
Y_{L_{\rm f}M_{\rm f}} (\hat{r}_{q_3\bar{q}_1,q_2\bar{q}_4}),
\end{eqnarray}
in ${\cal{M}}_{aq_{1}\bar{q}_{2}}$, and
\begin{eqnarray}
e^{-i\vec{p}_{q_1\bar{q}_4,q_3\bar{q}_2} \cdot
\vec{r}_{q_1\bar{q}_4,q_3\bar{q}_2}} & = & 4\pi
\sum\limits_{L_{\rm f}=0}^{\infty}
\sum\limits_{M_{\rm f}=-L_{\rm f}}^{L_{\rm f}}
i^{L_{\rm f}} (-1)^{L_{\rm f}} j_{L_{\rm f}} 
(\mid \vec{p}_{q_1\bar{q}_4,q_3\bar{q}_2} \mid r_{q_1\bar{q}_4,q_3\bar{q}_2})
\nonumber\\
& & Y_{L_{\rm f}M_{\rm f}}^\ast (\hat{p}_{q_1\bar{q}_4,q_3\bar{q}_2})
Y_{L_{\rm f}M_{\rm f}} (\hat{r}_{q_1\bar{q}_4,q_3\bar{q}_2}),
\end{eqnarray}
in ${\cal{M}}_{a\bar{q}_{1}q_{2}}$, where $Y_{L_{\rm i}M_{\rm i}}$ 
($Y_{L_{\rm f}M_{\rm f}}$) are the spherical harmonics with
the orbital-angular-momentum quantum number $L_{\rm i}$ ($L_{\rm f}$) and the 
magnetic
projection quantum number $M_{\rm i}$ ($M_{\rm f}$), $j_{L_{\rm i}}$ and
$j_{L_{\rm f}}$ are the spherical Bessel
functions, and $\hat{p}_{q_1\bar{q}_1,q_2\bar{q}_2}$
($\hat{p}_{q_3\bar{q}_1,q_2\bar{q}_4}$, $\hat{p}_{q_1\bar{q}_4,q_3\bar{q}_2}$,
$\hat{r}_{q_1\bar{q}_1,q_2\bar{q}_2}$,
$\hat{r}_{q_3\bar{q}_1,q_2\bar{q}_4}$, $\hat{r}_{q_1\bar{q}_4,q_3\bar{q}_2}$)
denotes the polar angles of $\vec{p}_{q_1\bar{q}_1,q_2\bar{q}_2}$
($\vec{p}_{q_3\bar{q}_1,q_2\bar{q}_4}$, $\vec{p}_{q_1\bar{q}_4,q_3\bar{q}_2}$,
$\vec{r}_{q_1\bar{q}_1,q_2\bar{q}_2}$,
$\vec{r}_{q_3\bar{q}_1,q_2\bar{q}_4}$, $\vec{r}_{q_1\bar{q}_4,q_3\bar{q}_2}$).

Let $\chi_{SS_{z}}$ ($\chi_{S^{\prime}S^{\prime}_{z}}$) stand for the spin wave
function of mesons $A$ and $B$ ($C$ and $D$),
which has the total spin $S$ ($S^{\prime}$) and its $z$ component $S_z$ 
($S^{\prime}_{z}$). The
Clebsch-Gordan coefficients $(S_{A}S_{Az}S_{B}S_{Bz}|SS_{z})$ relate
$\chi_{SS_z}$ to $\chi_{S_AS_{Az}}\chi_{S_BS_{Bz}}$, 
and $(S_{C}S_{Cz}S_{D}S_{Dz}|S^{\prime}S^{\prime}_{z})$ relate
$\chi_{S^{\prime}S^{\prime}_{z}}$ to $\chi_{S_CS_{Cz}}\chi_{S_DS_{Dz}}$:
\begin{eqnarray}
\chi_{S_{A}S_{A_{z}}}\chi_{S_{B}S_{B_{z}}}&=&
\sum^{S_{\rm max}}_{S=S_{\rm min}}\sum^{S}_{S_{z}=-S}
(S_{A}S_{Az}S_{B}S_{Bz}|SS_{z})\chi_{SS_{z}},\\
\chi_{S_{C}S_{C_{z}}} \chi_{S_{D}S_{D_{z}}}
&=&
\sum^{S^{\prime}_{\rm max}}_{S^{\prime}=S^{\prime}_{\rm min}}
\sum^{S^{\prime}}_{S^{\prime}_z=-S^{\prime}}
(S_{C}S_{Cz}S_{D}S_{Dz}|S^{\prime}S^{\prime}_z) \chi_{S^{\prime}S^{\prime}_z},
\end{eqnarray}
where $S_{\rm min}=\mid S_A-S_B \mid$, $S_{\rm max}=S_A+S_B$, 
$S^{\prime}_{\rm min}=\mid S_C-S_D \mid$, and $S^{\prime}_{\rm max}=S_C+S_D$. 
$Y_{L_{\rm i}M_{\rm i}}$ and $\chi_{SS_{z}}$ ($Y_{L_{\rm f}M_{\rm f}}$ and 
$\chi_{S^{\prime}S^{\prime}_z}$) are coupled to the wave function
$\varphi^{\rm in}_{JJ_z}$ ($\varphi^{\rm final}_{J^\prime J_z^\prime}$) which 
has the total angular momentum $J$ ($J^{\prime}$) of mesons
$A$ and $B$ ($C$ and $D$) and its $z$ component $J_z$ ($J^{\prime}_z$),
\begin{eqnarray}
Y_{L_{\rm{i}}M_{\rm{i}}}\chi_{SS_{z}} &=&
\sum^{J_{\rm max}}_{J=J_{\rm min}}\sum^J_{J_z=-J}
(L_{\rm{i}}M_{\rm{i}}SS_{z}|JJ_{z})\varphi^{\rm{in}}_{JJ_{z}},\\
Y_{L_{\rm f}M_{\rm f}}\chi_{S^{\prime}S^{\prime}_z} &=&
\sum^{J^{\prime}_{\rm max}}_{J^{\prime}=J^{\prime}_{\rm min}} 
\sum^{J^{\prime}}_{J^{\prime}_z=-J^{\prime}}
(L_{\rm f}M_{\rm f}S^{\prime}S^{\prime}_z|J^{\prime}J^{\prime}_z) 
{\varphi}^{\rm final}_{J^{\prime}J^{\prime}_z},
\end{eqnarray}
where $J_{\rm min}=\mid L_{\rm i}-S \mid$, $J_{\rm max}=L_{\rm i}+S$, 
$J^{\prime}_{\rm min}=\mid L_{\rm f}-S^{\prime} \mid$, and
$J^{\prime}_{\rm max}=L_{\rm f}+S^{\prime}$.
$(L_{\rm{i}}M_{\rm{i}}SS_{z}|JJ_{z})$ and 
$(L_{\rm{f}}M_{\rm{f}}S^{\prime}S^{\prime}_{z}|J^{\prime}J^{\prime}_{z})$ are 
the Clebsch-Gordan coefficients. It
follows from Eqs. (6)-(12) that the transition amplitude given in 
Eq. (\ref{amp_q1qb2}) becomes
\begin{eqnarray}
{\cal M}_{{\rm a}q_{1}\bar{q}_{2}}&=&
\frac{(m_{q_3}+m_{\bar{q}_1})^3}{{m^3_{\bar{q}_1}}} 
\sqrt{2E_{A}2E_{B}2E_{C}2E_{D}} (4\pi)^2
\sum^{\infty}_{L_{\rm i}=0} \sum^{L_{\rm i}}_{M_{\rm i}=-L_{\rm i}} 
i^{L_{\rm i}}
Y^*_{L_{\rm i}M_{\rm i}}(\hat{p}_{q_{1}\bar{q}_{1},q_{2}\bar{q}_{2}})
\nonumber\\
&&
\sum^{\infty}_{L_{\rm f}=0} \sum^{L_{\rm f}}_{M_{\rm f}=-L_{\rm f}} 
i^{L_{\rm f}} (-1)^{L_{\rm f}}
Y^*_{L_{\rm f}M_{\rm f}}(\hat{p}_{q_{3}\bar{q}_{1},q_{2}\bar{q}_{4}})
{\phi}^{+}_{C{\rm color}} {\phi}^{+}_{D{\rm color}} 
{\varphi}^{+}_{CD{\rm flavor}}
\nonumber\\
&&
\int{d\vec{r}_{q_{1}\bar{q}_{1}} d\vec{r}_{q_{2}\bar{q}_{4}} 
d\vec{r}_{q_{3}\bar{q}_{1},q_{2}\bar{q}_{4}}}
{\phi}^{+}_{C{\rm rel}} {\phi}^{+}_{D{\rm rel}} 
\sum_{S^{\prime}S^{\prime}_{z}}
(S_{C}S_{Cz}S_{D}S_{Dz}|S^{\prime}S^{\prime}_{z})
\nonumber\\
&&
\sum_{J^{\prime}J^{\prime}_{z}}
(L_{\rm f}M_{\rm f}S^{\prime}S^{\prime}_{z}|J^{\prime}J^{\prime}_{z}) 
\varphi^{\rm final}_{J^{\prime}J^{\prime}_{z}}
V_{{\rm a}q_{1}\bar{q}_{2}} \sum_{SS_{z}}(S_{A}S_{Az}S_{B}S_{Bz}|SS_{z})
\nonumber\\
&&
\sum_{JJ_{z}}(L_{\rm i}M_{\rm i}SS_{z}|JJ_{z}) \varphi^{\rm in}_{JJ_{z}} 
{\phi}_{A{\rm rel}} {\phi}_{B{\rm rel}} {\varphi}_{AB{\rm flavor}}
\phi_{A{\rm color}}\phi_{B{\rm color}} 
\nonumber\\
&&
j_{L_{\rm i}}(\mid \vec{p}_{q_1\bar{q}_1,q_2\bar{q}_2} 
\mid r_{q_1\bar{q}_1,q_2\bar{q}_2}) j_{L_{\rm f}}
(\mid \vec{p}_{q_3\bar{q}_1,q_2\bar{q}_4} \mid r_{q_3\bar{q}_1,q_2\bar{q}_4}),
\end{eqnarray}
and the transition amplitude given in Eq. (\ref{amp_qb1q2}) does
\begin{eqnarray}
{\cal M}_{{\rm a}\bar{q}_{1}q_{2}}&=&
\frac{(m_{q_1}+m_{\bar{q}_4})^3}{{m^3_{q_1}}} 
\sqrt{2E_{A}2E_{B}2E_{C}2E_{D}} (4\pi)^2
\sum^{\infty}_{L_{\rm i}=0} \sum^{L_{\rm i}}_{M_{\rm i}=-L_{\rm i}} 
i^{L_{\rm i}}
Y^*_{L_{\rm i}M_{\rm i}}(\hat{p}_{q_{1}\bar{q}_{1},q_{2}\bar{q}_{2}})
\nonumber\\
&&
\sum^{\infty}_{L_{\rm f}=0} \sum^{L_{\rm f}}_{M_{\rm f}=-L_{\rm f}} 
i^{L_{\rm f}} (-1)^{L_{\rm f}}
Y^*_{L_{\rm f}M_{\rm f}}(\hat{p}_{q_{1}\bar{q}_{4},q_{3}\bar{q}_{2}})
{\phi}^{+}_{C{\rm color}} {\phi}^{+}_{D{\rm color}} 
{\varphi}^{+}_{CD{\rm flavor}}
\nonumber\\
&&
\int{d\vec{r}_{q_{1}\bar{q}_{1}} d\vec{r}_{q_{3}\bar{q}_{2}} 
d\vec{r}_{q_{1}\bar{q}_{4},q_{3}\bar{q}_{2}}}
{\phi}^{+}_{C{\rm rel}} {\phi}^{+}_{D{\rm rel}} 
\sum_{S^{\prime}S^{\prime}_{z}}
(S_{C}S_{Cz}S_{D}S_{Dz}|S^{\prime}S^{\prime}_{z})
\nonumber\\
&&
\sum_{J^{\prime}J^{\prime}_{z}}
(L_{\rm f}M_{\rm f}S^{\prime}S^{\prime}_{z}|J^{\prime}J^{\prime}_{z}) 
\varphi^{\rm final}_{J^{\prime}J^{\prime}_{z}}
V_{{\rm a}q_{1}\bar{q}_{2}} \sum_{SS_{z}}
(S_{A}S_{Az}S_{B}S_{Bz}|SS_{z})
\nonumber\\
&&
\sum_{JJ_{z}}(L_{\rm i}M_{\rm i}SS_{z}|JJ_{z}) \varphi^{\rm in}_{JJ_{z}} 
{\phi}_{A{\rm rel}} {\phi}_{B{\rm rel}} {\varphi}_{AB{\rm flavor}}
\phi_{A{\rm color}}\phi_{B{\rm color}} 
\nonumber\\
&&
j_{L_{\rm i}}(\mid \vec{p}_{q_1\bar{q}_1,q_2\bar{q}_2} 
\mid r_{q_1\bar{q}_1,q_2\bar{q}_2}) j_{L_{\rm f}}
(\mid \vec{p}_{q_1\bar{q}_4,q_3\bar{q}_2} \mid r_{q_1\bar{q}_4,q_3\bar{q}_2}).
\end{eqnarray}
Conservation of the total angular momentum implies that $J$ equals 
$J^{\prime}$ and $J_z$ equals $J^{\prime}_{z}$. This leads to
\begin{eqnarray}
{\cal M}_{{\rm a}q_{1}\bar{q}_{2}}&=&
\frac{(m_{q_3}+m_{\bar{q}_1})^3}{{m^3_{\bar{q}_1}}} 
\sqrt{2E_{A}2E_{B}2E_{C}2E_{D}} (4\pi)^2
\sum^{\infty}_{L_{\rm i}=0} \sum^{L_{\rm i}}_{M_{\rm i}=-L_{\rm i}} 
i^{L_{\rm i}}
Y^*_{L_{\rm i}M_{\rm i}}(\hat{p}_{q_{1}\bar{q}_{1},q_{2}\bar{q}_{2}})
\nonumber\\
&&
\sum^{\infty}_{L_{\rm f}=0} \sum^{L_{\rm f}}_{M_{\rm f}=-L_{\rm f}} 
i^{L_{\rm f}} (-1)^{L_{\rm f}}
Y^*_{L_{\rm f}M_{\rm f}}(\hat{p}_{q_{3}\bar{q}_{1},q_{2}\bar{q}_{4}})
{\phi}^{+}_{C{\rm color}} {\phi}^{+}_{D{\rm color}} 
{\varphi}^{+}_{CD{\rm flavor}}
\nonumber\\
&&
\int{d\vec{r}_{q_{1}\bar{q}_{1}} d\vec{r}_{q_{2}\bar{q}_{4}} 
d\vec{r}_{q_{3}\bar{q}_{1},q_{2}\bar{q}_{4}}}
{\phi}^{+}_{C{\rm rel}} {\phi}^{+}_{D{\rm rel}} 
\sum_{S^{\prime}S^{\prime}_{z}}
(S_{C}S_{Cz}S_{D}S_{Dz}|S^{\prime}S^{\prime}_{z})
\nonumber\\
&&
\sum_{JJ_{z}}(L_{\rm f}M_{\rm f}S^{\prime}S^{\prime}_{z}|JJ_{z}) 
\varphi^{\rm final}_{JJ_{z}}
V_{{\rm a}q_{1}\bar{q}_{2}} \sum_{SS_{z}}(S_{A}S_{Az}S_{B}S_{Bz}|SS_{z})
\nonumber\\
&&
(L_{\rm i}M_{\rm i}SS_{z}|JJ_{z}) \varphi^{\rm in}_{JJ_{z}} 
{\phi}_{A{\rm rel}} {\phi}_{B{\rm rel}} {\varphi}_{AB{\rm flavor}}
\phi_{A{\rm color}}\phi_{B{\rm color}} 
\nonumber\\
&&
j_{L_{\rm i}}(\mid \vec{p}_{q_1\bar{q}_1,q_2\bar{q}_2} 
\mid r_{q_1\bar{q}_1,q_2\bar{q}_2}) j_{L_{\rm f}}
(\mid \vec{p}_{q_3\bar{q}_1,q_2\bar{q}_4} \mid r_{q_3\bar{q}_1,q_2\bar{q}_4}),
\\
{\cal M}_{{\rm a}\bar{q}_{1}q_{2}}&=&
\frac{(m_{q_1}+m_{\bar{q}_4})^3}{{m^3_{q_1}}} 
\sqrt{2E_{A}2E_{B}2E_{C}2E_{D}} (4\pi)^2
\sum^{\infty}_{L_{\rm i}=0} \sum^{L_{\rm i}}_{M_{\rm i}=-L_{\rm i}} 
i^{L_{\rm i}}
Y^*_{L_{\rm i}M_{\rm i}}(\hat{p}_{q_{1}\bar{q}_{1},q_{2}\bar{q}_{2}})
\nonumber\\
&&
\sum^{\infty}_{L_{\rm f}=0} \sum^{L_{\rm f}}_{M_{\rm f}=-L_{\rm f}} 
i^{L_{\rm f}} (-1)^{L_{\rm f}}
Y^*_{L_{\rm f}M_{\rm f}}(\hat{p}_{q_{1}\bar{q}_{4},q_{3}\bar{q}_{2}})
{\phi}^{+}_{C{\rm color}} {\phi}^{+}_{D{\rm color}} 
{\varphi}^{+}_{CD{\rm flavor}}
\nonumber\\
&&
\int{d\vec{r}_{q_{1}\bar{q}_{1}} d\vec{r}_{q_{3}\bar{q}_{2}} 
d\vec{r}_{q_{1}\bar{q}_{4},q_{3}\bar{q}_{2}}}
{\phi}^{+}_{C{\rm rel}} {\phi}^{+}_{D{\rm rel}} 
\sum_{S^{\prime}S^{\prime}_{z}}
(S_{C}S_{Cz}S_{D}S_{Dz}|S^{\prime}S^{\prime}_{z})
\nonumber\\
&&
\sum_{JJ_{z}}(L_{\rm f}M_{\rm f}S^{\prime}S^{\prime}_{z}|JJ_{z}) 
\varphi^{\rm final}_{JJ_{z}}
V_{{\rm a}\bar{q}_{1}q_{2}} \sum_{SS_{z}}(S_{A}S_{Az}S_{B}S_{Bz}|SS_{z})
\nonumber\\
&&
(L_{\rm i}M_{\rm i}SS_{z}|JJ_{z}) \varphi^{\rm in}_{JJ_{z}} 
{\phi}_{A{\rm rel}} {\phi}_{B{\rm rel}} {\varphi}_{AB{\rm flavor}}
\phi_{A{\rm color}}\phi_{B{\rm color}} 
\nonumber\\
&&
j_{L_{\rm i}}(\mid \vec{p}_{q_1\bar{q}_1,q_2\bar{q}_2} 
\mid r_{q_1\bar{q}_1,q_2\bar{q}_2}) j_{L_{\rm f}}
(\mid \vec{p}_{q_1\bar{q}_4,q_3\bar{q}_2} \mid r_{q_1\bar{q}_4,q_3\bar{q}_2}).
\end{eqnarray}
Using the relation
\begin{eqnarray}
\varphi^{\rm{in}}_{JJ_{z}} &=&
\sum_{\bar{M}_{\rm i}\bar{S}_{z}}
(L_{\rm{i}}\bar{M}_{\rm{i}}S\bar{S}_{z}|JJ_{z})
Y_{L_{\rm{i}}\bar{M}_{\rm{i}}}\chi_{S\bar{S}_{z}},\\
\varphi^{\rm final}_{JJ_{z}} &=&
\sum_{\bar{M}_{\rm f}\bar{S}^{\prime}_{z}}
(L_{\rm f}\bar{M}_{\rm f}S^{\prime}\bar{S}^{\prime}_{z}|JJ_{z})
Y_{L_{\rm f}\bar{M}_{\rm f}} \chi_{S^{\prime}S^{\prime}_{z}},
\end{eqnarray}
where $(L_{\rm{i}}\bar{M}_{\rm{i}}S\bar{S}_{z}|JJ_{z})$ and 
$(L_{\rm{f}}\bar{M}_{\rm{f}}S^{\prime}\bar{S}^{\prime}_{z}|J_{}J_{z})$ are the
Clebsch-Gordan coefficients, we get
\begin{eqnarray}
{\cal M}_{{\rm a}q_{1}\bar{q}_{2}}&=&
\frac{(m_{q_3}+m_{\bar{q}_1})^3}{{m^3_{\bar{q}_1}}} 
\sqrt{2E_{A}2E_{B}2E_{C}2E_{D}} (4\pi)^2
\sum^{\infty}_{L_{\rm i}=0} \sum^{L_{\rm i}}_{M_{\rm i}=-L_{\rm i}} 
i^{L_{\rm i}}
Y^*_{L_{\rm i}M_{\rm i}}(\hat{p}_{q_{1}\bar{q}_{1},q_{2}\bar{q}_{2}})
\nonumber\\
&&
\sum^{\infty}_{L_{\rm f}=0} \sum^{L_{\rm f}}_{M_{\rm f}=-L_{\rm f}} 
i^{L_{\rm f}} (-1)^{L_{\rm f}}
Y^*_{L_{\rm f}M_{\rm f}}(\hat{p}_{q_{3}\bar{q}_{1},q_{2}\bar{q}_{4}})
\sum_{S^{\prime}S^{\prime}_{z}}
(S_{C}S_{Cz}S_{D}S_{Dz}|S^{\prime}S^{\prime}_{z})
\nonumber\\
&&
\sum_{JJ_{z}}(L_{\rm f}M_{\rm f}S^{\prime}S^{\prime}_{z}|JJ_{z})
\sum_{\bar{M}_{\rm f}\bar{S}^{\prime}_{z}}
(L_{\rm f}\bar{M}_{\rm f}S^{\prime}\bar{S}^{\prime}_{z}|JJ_{z})
\sum_{SS_{z}}(S_{A}S_{Az}S_{B}S_{Bz}|SS_{z}) 
\nonumber\\
&&
(L_{\rm i}M_{\rm i}SS_{z}|JJ_{z}) 
\sum_{\bar{M}_{\rm i}\bar{S}_{z}}(L_{\rm i}\bar{M}_{\rm i}S\bar{S}_{z}|JJ_{z})
{\phi}^{+}_{C{\rm color}} {\phi}^{+}_{D{\rm color}} 
{\varphi}^{+}_{CD{\rm flavor}}
\chi^{+}_{S^{\prime}\bar{S}^{\prime}_{z}}
\nonumber\\
&&
\int{d\vec{r}_{q_{1}\bar{q}_{1}} d\vec{r}_{q_{2}\bar{q}_{4}} 
d\vec{r}_{q_{3}\bar{q}_{1},q_{2}\bar{q}_{4}}}
j_{L_{\rm f}}(\mid \vec{p}_{q_3\bar{q}_1,q_2\bar{q}_4} 
\mid r_{q_3\bar{q}_1,q_2\bar{q}_4})
Y_{L_{\rm f}\bar{M}_{\rm f}}(\hat{r}_{q_{3}\bar{q}_{1},q_{2}\bar{q}_{4}})
\nonumber\\
&&
{\phi}^{+}_{C{\rm rel}} {\phi}^{+}_{D{\rm rel}} 
V_{{\rm a}q_{1}\bar{q}_{2}}\phi_{A{\rm rel}}\phi_{B{\rm rel}} 
j_{L_{\rm i}}(\mid \vec{p}_{q_1\bar{q}_1,q_2\bar{q}_2} 
\mid r_{q_1\bar{q}_1,q_2\bar{q}_2})%
Y_{L_{\rm i}\bar{M}_{\rm i}}(\hat{r}_{q_{1}\bar{q}_{1},q_{2}\bar{q}_{2}})
\nonumber\\
&&
\chi_{S\bar{S}_{z}} {\varphi}_{AB{\rm flavor}}
\phi_{A{\rm color}}\phi_{B{\rm color}}, \\
{\cal M}_{{\rm a}\bar{q}_{1}q_{2}}&=& 
\frac{(m_{q_1}+m_{\bar{q}_4})^3}{{m^3_{q_1}}} 
\sqrt{2E_{A}2E_{B}2E_{C}2E_{D}} (4\pi)^2
\sum^{\infty}_{L_{\rm i}=0} \sum^{L_{\rm i}}_{M_{\rm i}=-L_{\rm i}} 
i^{L_{\rm i}}
Y^*_{L_{\rm i}M_{\rm i}}(\hat{p}_{q_{1}\bar{q}_{1},q_{2}\bar{q}_{2}})
\nonumber\\
&&
\sum^{\infty}_{L_{\rm f}=0} \sum^{L_{\rm f}}_{M_{\rm f}=-L_{\rm f}} 
i^{L_{\rm f}} (-1)^{L_{\rm f}}
Y^*_{L_{\rm f}M_{\rm f}}(\hat{p}_{q_{1}\bar{q}_{4},q_{3}\bar{q}_{2}})
\sum_{S^{\prime}S^{\prime}_{z}}
(S_{C}S_{Cz}S_{D}S_{Dz}|S^{\prime}S^{\prime}_{z})
\nonumber\\
&&
\sum_{JJ_{z}}(L_{\rm f}M_{\rm f}S^{\prime}S^{\prime}_{z}|JJ_{z})
\sum_{\bar{M}_{\rm f}\bar{S}^{\prime}_{z}}
(L_{\rm f}\bar{M}_{\rm f}S^{\prime}\bar{S}^{\prime}_{z}|JJ_{z})
\sum_{SS_{z}}(S_{A}S_{Az}S_{B}S_{Bz}|SS_{z}) 
\nonumber\\
&&
(L_{\rm i}M_{\rm i}SS_{z}|JJ_{z}) 
\sum_{\bar{M}_{\rm i}\bar{S}_{z}}(L_{\rm i}\bar{M}_{\rm i}S\bar{S}_{z}|JJ_{z})
{\phi}^{+}_{C{\rm color}} {\phi}^{+}_{D{\rm color}} 
{\varphi}^{+}_{CD{\rm flavor}}
\chi^{+}_{S^{\prime}\bar{S}^{\prime}_{z}}
\nonumber\\
&&
\int{d\vec{r}_{q_{1}\bar{q}_{1}} d\vec{r}_{q_{3}\bar{q}_{2}} 
d\vec{r}_{q_{1}\bar{q}_{4},q_{3}\bar{q}_{2}}}
j_{L_{\rm f}}(\mid \vec{p}_{q_1\bar{q}_4,q_3\bar{q}_2} 
\mid r_{q_1\bar{q}_4,q_3\bar{q}_2})
Y_{L_{\rm f}\bar{M}_{\rm f}}(\hat{r}_{q_{1}\bar{q}_{4},q_{3}\bar{q}_{2}})
\nonumber\\
&&
{\phi}^{+}_{C{\rm rel}} {\phi}^{+}_{D{\rm rel}} 
V_{{\rm a}\bar{q}_{1}q_{2}}\phi_{A{\rm rel}}\phi_{B{\rm rel}} 
j_{L_{\rm i}}(\mid \vec{p}_{q_1\bar{q}_1,q_2\bar{q}_2} 
\mid r_{q_1\bar{q}_1,q_2\bar{q}_2})%
Y_{L_{\rm i}\bar{M}_{\rm i}}(\hat{r}_{q_{1}\bar{q}_{1},q_{2}\bar{q}_{2}})
\nonumber\\
&&
\chi_{S\bar{S}_{z}} {\varphi}_{AB{\rm flavor}}
\phi_{A{\rm color}}\phi_{B{\rm color}}.
\end{eqnarray}

Furthermore, we need the identity
\begin{equation}
j_l(pr)Y_{lm}(\hat{r})=\int \frac{d^3p^\prime}{(2\pi)^3}\frac{2\pi^2}{p^2}
\delta (p-p^\prime) i^l (-1)^l Y_{lm}(\hat{p}^\prime)
e^{i\vec{p}^{~\prime} \cdot \vec{r}} ,
\end{equation}
which is obtained with the help of
$\int_0^\infty j_l(pr)j_l(p^\prime r)r^2dr=\frac{\pi}{2p^2}
\delta (p-p^\prime)$ \cite{AW,joachain}, and where $\hat{r}$ ($\hat{p}^\prime$)
denotes the polar angles of $\vec r$ ($\vec{p}^{~\prime}$). Substituting Eq. 
(21) in Eqs. (19) and (20), we get
\begin{eqnarray}
{\cal M}_{{\rm a}q_{1}\bar{q}_{2}}&=&
\frac{(m_{q_3}+m_{\bar{q}_1})^3}{{m^3_{\bar{q}_1}}} 
\sqrt{2E_{A}2E_{B}2E_{C}2E_{D}} (4\pi)^2
\sum^{\infty}_{L_{\rm i}=0} \sum^{L_{\rm i}}_{M_{\rm i}=-L_{\rm i}} 
i^{L_{\rm i}}
Y^*_{L_{\rm i}M_{\rm i}}(\hat{p}_{q_{1}\bar{q}_{1},q_{2}\bar{q}_{2}})
\nonumber\\
&&
\sum^{\infty}_{L_{\rm f}=0} \sum^{L_{\rm f}}_{M_{\rm f}=-L_{\rm f}} 
i^{L_{\rm f}} (-1)^{L_{\rm f}}
Y^*_{L_{\rm f}M_{\rm f}}(\hat{p}_{q_{3}\bar{q}_{1},q_{2}\bar{q}_{4}})
\sum_{S^{\prime}S^{\prime}_{z}}
(S_{C}S_{Cz}S_{D}S_{Dz}|S^{\prime}S^{\prime}_{z})
\nonumber\\
&&
\sum_{JJ_{z}}(L_{\rm f}M_{\rm f}S^{\prime}S^{\prime}_{z}|JJ_{z})
\sum_{\bar{M}_{\rm f}\bar{S}^{\prime}_{z}}
(L_{\rm f}\bar{M}_{\rm f}S^{\prime}\bar{S}^{\prime}_{z}|JJ_{z})
\sum_{SS_{z}}(S_{A}S_{Az}S_{B}S_{Bz}|SS_{z}) 
\nonumber\\
&&
(L_{\rm i}M_{\rm i}SS_{z}|JJ_{z}) 
\sum_{\bar{M}_{\rm i}\bar{S}_{z}}(L_{\rm i}\bar{M}_{\rm i}S\bar{S}_{z}|JJ_{z})
{\phi}^{+}_{C{\rm color}} {\phi}^{+}_{D{\rm color}} 
{\varphi}^{+}_{CD{\rm flavor}}
\chi^{+}_{S^{\prime}\bar{S}^{\prime}_{z}}
\nonumber\\
&&
\int{\frac{d^{3}p_{\rm frm}}{(2\pi)^{3}}} 
\frac{2\pi^{2}}{\vec{p}^{2}_{{q_{3}}\bar{q}_1,q_{2}\bar{q}_{4}}}
\delta(\mid \vec{p}_{q_{3}\bar{q}_1,q_{2}\bar{q}_{4}} 
\mid-\mid \vec{p}_{\rm frm} \mid) i^{L_{\rm f}} (-1)^{L_{\rm f}}%
Y_{L_{\rm f}\bar{M}_{\rm f}} (\hat{p}_{\rm frm})
\nonumber\\
&&
\int{\frac{d^{3}p_{\rm irm}}{(2\pi)^{3}}} 
\frac{2\pi^{2}}{\vec{p}^{2}_{{q_{1}}\bar{q}_1,q_{2}\bar{q}_{2}}}
\delta(\mid \vec{p}_{q_{1}\bar{q}_1,q_{2}\bar{q}_{2}} 
\mid-\mid \vec{p}_{\rm irm} \mid) i^{L_{\rm i}} (-1)^{L_{\rm i}}%
Y_{L_{\rm i}\bar{M}_{\rm i}} (\hat{p}_{\rm irm})
\nonumber\\
&&
\int{d\vec{r}_{q_{1}\bar{q}_{1}} d\vec{r}_{q_{2}\bar{q}_{4}} 
d\vec{r}_{q_{3}\bar{q}_{1},q_{2}\bar{q}_{4}}}
{\phi}^{+}_{C{\rm rel}} {\phi}^{+}_{D{\rm rel}} 
V_{{\rm a}q_{1}\bar{q}_{2}}\phi_{A{\rm rel}}\phi_{B{\rm rel}} 
\nonumber\\
&&
e^{i\vec{p}_{\rm frm}\cdot\vec{r}_{q_{3}\bar{q}_1,q_{2}\bar{q}_4}}
e^{i\vec{p}_{\rm irm}\cdot\vec{r}_{q_{1}\bar{q}_1,q_{2}\bar{q}_2}}
\chi_{S\bar{S}_{z}} \varphi_{AB{\rm flavor}}
\phi_{A{\rm color}}\phi_{B{\rm color}},
\end{eqnarray}
\begin{eqnarray}
{\cal M}_{{\rm a}\bar{q}_{1}q_{2}}&=& 
\frac{(m_{q_1}+m_{\bar{q}_4})^3}{{m^3_{q_1}}} 
\sqrt{2E_{A}2E_{B}2E_{C}2E_{D}} (4\pi)^2
\sum^{\infty}_{L_{\rm i}=0} \sum^{L_{\rm i}}_{M_{\rm i}=-L_{\rm i}} 
i^{L_{\rm i}}
Y^*_{L_{\rm i}M_{\rm i}}(\hat{p}_{q_{1}\bar{q}_{1},q_{2}\bar{q}_{2}})
\nonumber\\
&&
\sum^{\infty}_{L_{\rm f}=0} \sum^{L_{\rm f}}_{M_{\rm f}=-L_{\rm f}} 
i^{L_{\rm f}} (-1)^{L_{\rm f}}
Y^*_{L_{\rm f}M_{\rm f}}(\hat{p}_{q_{1}\bar{q}_{4},q_{3}\bar{q}_{2}})
\sum_{S^{\prime}S^{\prime}_{z}}
(S_{C}S_{Cz}S_{D}S_{Dz}|S^{\prime}S^{\prime}_{z})
\nonumber\\
&&
\sum_{JJ_{z}}(L_{\rm f}M_{\rm f}S^{\prime}S^{\prime}_{z}|JJ_{z})
\sum_{\bar{M}_{\rm f}\bar{S}^{\prime}_{z}}
(L_{\rm f}\bar{M}_{\rm f}S^{\prime}\bar{S}^{\prime}_{z}|JJ_{z})
\sum_{SS_{z}}(S_{A}S_{Az}S_{B}S_{Bz}|SS_{z}) 
\nonumber\\
&&
(L_{\rm i}M_{\rm i}SS_{z}|JJ_{z}) 
\sum_{\bar{M}_{\rm i}\bar{S}_{z}}(L_{\rm i}\bar{M}_{\rm i}S\bar{S}_{z}|JJ_{z})
{\phi}^{+}_{C{\rm color}} {\phi}^{+}_{D{\rm color}} 
{\varphi}^{+}_{CD{\rm flavor}}
\chi^{+}_{S^{\prime}\bar{S}^{\prime}_{z}}
\nonumber\\
&&
\int{\frac{d^{3}p_{\rm frm}}{(2\pi)^{3}}} 
\frac{2\pi^{2}}{\vec{p}^{2}_{{q_{1}}\bar{q}_{4},q_{3}\bar{q}_{2}}}
\delta(\mid \vec{p}_{q_{1}\bar{q}_{4},q_{3}\bar{q}_{2}} 
\mid-\mid \vec{p}_{\rm frm} \mid) 
i^{L_{\rm f}} (-1)^{L_{\rm f}}%
Y_{L_{\rm f}\bar{M}_{\rm f}} (\hat{p}_{\rm frm})
\nonumber\\
&&
\int{\frac{d^{3}p_{\rm irm}}{(2\pi)^{3}}} 
\frac{2\pi^{2}}{\vec{p}^{2}_{{q_{1}}\bar{q}_1,q_{2}\bar{q}_{2}}}
\delta(\mid \vec{p}_{q_{1}\bar{q}_1,q_{2}\bar{q}_{2}} 
\mid-\mid \vec{p}_{\rm irm} \mid) i^{L_{\rm i}} (-1)^{L_{\rm i}}%
Y_{L_{\rm i}\bar{M}_{\rm i}} (\hat{p}_{\rm irm})
\nonumber\\
&&
\int{d\vec{r}_{q_{1}\bar{q}_{1}} d\vec{r}_{q_{3}\bar{q}_{2}} 
d\vec{r}_{q_{1}\bar{q}_{4},q_{3}\bar{q}_{2}}}
{\phi}^{+}_{C{\rm rel}} {\phi}^{+}_{D{\rm rel}} 
V_{{\rm a}\bar{q}_{1}q_{2}}\phi_{A{\rm rel}}\phi_{B{\rm rel}} 
\nonumber\\
&&
e^{i\vec{p}_{\rm frm}\cdot\vec{r}_{q_{1}\bar{q}_4,q_{3}\bar{q}_2}}
e^{i\vec{p}_{\rm irm}\cdot\vec{r}_{q_{1}\bar{q}_1,q_{2}\bar{q}_2}}
\chi_{S\bar{S}_{z}} \varphi_{AB{\rm flavor}}
\phi_{A{\rm color}}\phi_{B{\rm color}}.
\end{eqnarray}
Let $\vec{r}_c$ be the position vector of constituent $c$. $\phi_{A\rm{rel}}$
and $\phi_{B\rm{rel}}$ are functions of
the relative coordinate of the quark and
the antiquark inside mesons $A$ and $B$, respectively. 
We take the Fourier transform of
$V_{{\rm a}q_1\bar{q}_2}$, $V_{{\rm a}\bar{q}_1q_2}$, $\phi_{A\rm{rel}}$,
and $\phi_{B\rm{rel}}$:
\begin{equation}
V_{{\rm a}q_1\bar{q}_2}(\vec{r}_{q_3}-\vec{r}_{q_1}) =
\int \frac {d^3k}{(2\pi)^3} V_{{\rm a}q_1\bar{q}_2} (\vec {k})
e^{i\vec {k} \cdot (\vec{r}_{q_3}-\vec{r}_{q_1})},
\end{equation}
\begin{equation}
V_{{\rm a}\bar{q}_1q_2}(\vec{r}_{q_3}-\vec{r}_{q_2}) =
\int \frac {d^3k}{(2\pi)^3} V_{{\rm a}\bar{q}_1q_2} (\vec {k})
e^{i\vec {k} \cdot (\vec{r}_{q_3}-\vec{r}_{q_2})},
\end{equation}
\begin{equation}
\phi_{A\rm rel}(\vec{r}_{q_1\bar{q}_1}) =
\int \frac {d^3p_{q_1\bar{q}_1}}{(2\pi)^3} \phi_{A\rm rel}
(\vec {p}_{q_1\bar{q}_1})
e^{i\vec {p}_{q_1\bar{q}_1} \cdot \vec {r}_{q_1\bar{q}_1}},
\end{equation}
\begin{equation}
\phi_{B\rm rel}(\vec{r}_{q_2\bar{q}_2}) =
\int \frac {d^3p_{q_2\bar{q}_2}}{(2\pi)^3} \phi_{B\rm rel}
(\vec {p}_{q_2\bar{q}_2})
e^{i\vec {p}_{q_2\bar{q}_2} \cdot \vec {r}_{q_2\bar{q}_2}}.
\end{equation}
In Eq. (24)-(25) $\vec k$ is the gluon momentum,
and in Eqs. (26)-(27)
$\vec{p}_{ab}$ is the relative momentum of constituents $a$ and $b$. 
In momentum space the normalizations are
\begin{displaymath}
\int \frac{d^3p_{q_1\bar{q}_1}}{(2\pi)^3}
\phi_{A\rm rel}^+(\vec{p}_{q_1\bar{q}_1})
\phi_{A\rm rel}(\vec{p}_{q_1\bar{q}_1})=1,
\end{displaymath}
\begin{displaymath}
\int \frac{d^3p_{q_2\bar{q}_2}}{(2\pi)^3}
\phi_{B\rm rel}^+(\vec{p}_{q_2\bar{q}_2})
\phi_{B\rm rel}(\vec{p}_{q_2\bar{q}_2})=1.
\end{displaymath}
The spherical polar coordinates of $\vec{p}_{\rm irm}$ and $\vec{p}_{\rm frm}$ 
are expressed as
$(\mid \vec{p}_{\rm irm} \mid, \theta_{\rm irm}, \phi_{\rm irm})$ and 
$(\mid \vec{p}_{\rm frm} \mid, \theta_{\rm frm}, \phi_{\rm frm})$, 
respectively. Integration over
$\mid \vec{p}_{\rm irm} \mid$, $\mid \vec{p}_{\rm frm} \mid$, 
$\vec{r}_{q_1\bar{q}_1}$, $\vec{r}_{q_2\bar{q}_4}$, and 
$\vec{r}_{q_3\bar{q}_1,q_2\bar{q}_4}$ in Eq. (22) yields
\begin{eqnarray}
{\cal M}_{{\rm a}q_{1}\bar{q}_{2}}&=&
\sqrt{2E_{A}2E_{B}2E_{C}2E_{D}}
\sum^{\infty}_{L_{\rm i}=0} \sum^{L_{\rm i}}_{M_{\rm i}=-L_{\rm i}}
Y^*_{L_{\rm i}M_{\rm i}}(\hat{p}_{q_{1}\bar{q}_{1},q_{2}\bar{q}_{2}})
\nonumber\\
&&
\times \sum^{\infty}_{L_{\rm f}=0} \sum^{L_{\rm f}}_{M_{\rm f}=-L_{\rm f}} 
(-1)^{L_{\rm f}}
Y^*_{L_{\rm f}M_{\rm f}}(\hat{p}_{q_{3}\bar{q}_{1},q_{2}\bar{q}_{4}})
\sum_{S^{\prime}S^{\prime}_{z}}
(S_{C}S_{Cz}S_{D}S_{Dz}|S^{\prime}S^{\prime}_{z})
\nonumber\\
&&
\times \sum_{JJ_{z}}(L_{\rm f}M_{\rm f}S^{\prime}S^{\prime}_{z}|JJ_{z})
\sum_{\bar{M}_{\rm f}\bar{S}^{\prime}_{z}}
(L_{\rm f}\bar{M}_{\rm f}S^{\prime}\bar{S}^{\prime}_{z}|JJ_{z})
\sum_{SS_{z}}(S_{A}S_{Az}S_{B}S_{Bz}|SS_{z}) 
\nonumber\\
&&
\times (L_{\rm i}M_{\rm i}SS_{z}|JJ_{z}) 
\sum_{\bar{M}_{\rm i}\bar{S}_{z}}(L_{\rm i}\bar{M}_{\rm i}S\bar{S}_{z}|JJ_{z})
{\phi}^{+}_{C{\rm color}} {\phi}^{+}_{D{\rm color}} 
{\varphi}^{+}_{CD{\rm flavor}}
\chi^{+}_{S^{\prime}\bar{S}^{\prime}_{z}}
\nonumber\\
&&
\times \int{d\theta_{\rm frm}d\phi_{\rm frm}} 
\sin{\theta}_{\rm frm} Y_{L_{\rm f}\bar{M}_{\rm f}}(\hat{p}_{\rm frm})
\int{d\theta_{\rm irm}d\phi_{\rm irm}} 
\sin{\theta}_{\rm irm} Y_{L_{\rm i}\bar{M}_{\rm i}}(\hat{p}_{\rm irm})
\nonumber\\
&&
\times \int{\frac{d^{3}p_{q_{1}\bar{q}_{1}}}{(2\pi)^3}}
\int{\frac{d^{3}p_{q_{2}\bar{q}_{2}}}{(2\pi)^3}}
{\phi}^{+}_{C{\rm rel}} (\vec{p}_{q_{1}\bar{q}_{1}}
-\frac{m_{\bar{q}_1}}{m_{q_{1}}+m_{\bar{q}_{1}}}\vec{p}_{\rm irm}
-\frac{m_{\bar{q}_1}}{m_{q_{3}}+m_{\bar{q}_{1}}}\vec{p}_{\rm frm})
\nonumber\\
&&
\times {\phi}^{+}_{D{\rm rel}} (\vec{p}_{q_{2}\bar{q}_{2}}
-\frac{m_{q_{2}}}{m_{q_{2}}+m_{\bar{q}_{2}}}\vec{p}_{\rm irm}
-\frac{m_{q_{2}}}{m_{q_{2}}+m_{\bar{q}_{4}}}\vec{p}_{\rm frm})
\nonumber\\
&&
\times  V_{{\rm a}q_{1}\bar{q}_{2}}
[\vec{p}_{q_{1}\bar{q}_{1}}-\vec{p}_{q_{2}\bar{q}_{2}}
+(\frac{m_{q_1}}{m_{q_{1}}+m_{\bar{q}_{1}}}
-\frac{m_{\bar{q}_{2}}}{m_{q_{2}}+m_{\bar{q}_{2}}})\vec{p}_{\rm irm}]
\nonumber\\
&&
\times \phi_{A{\rm rel}}(\vec{p}_{q_{1}\bar{q}_{1}})
\phi_{B{\rm rel}}(\vec{p}_{q_{2}\bar{q}_{2}})
\chi_{S\bar{S}_{z}} \varphi_{AB{\rm flavor}}
\phi_{A{\rm color}}\phi_{B{\rm color}},
\end{eqnarray}
in which $\mid \vec{p}_{\rm irm} \mid = \mid
\vec{p}_{q_1\bar{q}_1,q_2\bar{q}_2} \mid$ and $\mid \vec{p}_{\rm frm} \mid = 
\mid \vec{p}_{q_3\bar{q}_1,q_2\bar{q}_4} \mid$; $\hat{p}_{\rm irm}$
($\hat{p}_{\rm frm}$) denotes the polar angles of $\vec{p}_{\rm irm}$ 
($\vec{p}_{\rm frm}$); $m_{q_2}$ and 
$m_{\bar{q}_2}$ are the $q_2$ and $\bar{q}_2$ masses, respectively.
Integration over
$\mid \vec{p}_{\rm irm} \mid$, $\mid \vec{p}_{\rm frm} \mid$, 
$\vec{r}_{q_1\bar{q}_1}$, $\vec{r}_{q_3\bar{q}_2}$, and 
$\vec{r}_{q_1\bar{q}_4,q_3\bar{q}_2}$
in Eq. (23) yields
\begin{eqnarray}
{\cal M}_{{\rm a}\bar{q}_{1}q_{2}}&=&
\sqrt{2E_{A}2E_{B}2E_{C}2E_{D}}
\sum^{\infty}_{L_{\rm i}=0} \sum^{L_{\rm i}}_{M_{\rm i}=-L_{\rm i}}
Y^*_{L_{\rm i}M_{\rm i}}(\hat{p}_{q_{1}\bar{q}_{1},q_{2}\bar{q}_{2}})
\nonumber\\
&&
\sum^{\infty}_{L_{\rm f}=0} 
\sum^{L_{\rm f}}_{M_{\rm f}=-L_{\rm f}} (-1)^{L_{\rm f}}
Y^*_{L_{\rm f}M_{\rm f}}(\hat{p}_{q_{1}\bar{q}_{4},q_{3}\bar{q}_{2}})
\sum_{S^{\prime}S^{\prime}_{z}}
(S_{C}S_{Cz}S_{D}S_{Dz}|S^{\prime}S^{\prime}_{z})
\nonumber\\
&&
\sum_{JJ_{z}}(L_{\rm f}M_{\rm f}S^{\prime}S^{\prime}_{z}|JJ_{z})
\sum_{\bar{M}_{\rm f}\bar{S}^{\prime}_{z}}
(L_{\rm f}\bar{M}_{\rm f}S^{\prime}\bar{S}^{\prime}_{z}|JJ_{z})
\sum_{SS_{z}}(S_{A}S_{Az}S_{B}S_{Bz}|SS_{z}) 
\nonumber\\
&&
(L_{\rm i}M_{\rm i}SS_{z}|JJ_{z}) 
\sum_{\bar{M}_{\rm i}\bar{S}_{z}}(L_{\rm i}\bar{M}_{\rm i}S\bar{S}_{z}|JJ_{z})
{\phi}^{+}_{C{\rm color}} {\phi}^{+}_{D{\rm color}} 
{\varphi}^{+}_{CD{\rm flavor}}
\chi^{+}_{S^{\prime}\bar{S}^{\prime}_{z}}
\nonumber\\
&&
\int{d\theta_{\rm frm}d\phi_{\rm frm}} \sin{\theta}_{\rm frm} 
Y_{L_{\rm f}\bar{M}_{\rm f}}(\hat{p}_{\rm frm})
\int{d\theta_{\rm irm}d\phi_{\rm irm}} \sin{\theta}_{\rm irm} 
Y_{L_{\rm i}\bar{M}_{\rm i}}(\hat{p}_{\rm irm})
\nonumber\\
&&
\int{\frac{d^{3}p_{q_{1}\bar{q}_{1}}}{(2\pi)^3}}
\int{\frac{d^{3}p_{q_{2}\bar{q}_{2}}}{(2\pi)^3}}
{\phi}^{+}_{C{\rm rel}} (\vec{p}_{q_{1}\bar{q}_{1}}
+\frac{m_{q_1}}{m_{q_{1}}+m_{\bar{q}_{1}}}\vec{p}_{\rm irm}
+\frac{m_{q_1}}{m_{q_{1}}+m_{\bar{q}_{4}}}\vec{p}_{\rm frm})
\nonumber\\
&&
{\phi}^{+}_{D{\rm rel}} (\vec{p}_{q_{2}\bar{q}_{2}}
+\frac{m_{\bar{q}_{2}}}{m_{q_{2}}+m_{\bar{q}_{2}}}\vec{p}_{\rm irm}
+\frac{m_{\bar{q}_{2}}}{m_{q_{3}}+m_{\bar{q}_{2}}}\vec{p}_{\rm frm})
\nonumber\\
&&
V_{{\rm a}\bar{q}_{1}q_{2}}
[\vec{p}_{q_{2}\bar{q}_{2}}-\vec{p}_{q_{1}\bar{q}_{1}}
-(\frac{m_{q_2}}{m_{q_{2}}+m_{\bar{q}_{2}}}
-\frac{m_{\bar{q}_{1}}}{m_{q_{1}}+m_{\bar{q}_{1}}})\vec{p}_{\rm irm}]
\nonumber\\
&&
\phi_{A{\rm rel}}(\vec{p}_{q_{1}\bar{q}_{1}})
\phi_{B{\rm rel}}(\vec{p}_{q_{2}\bar{q}_{2}})
\chi_{S\bar{S}_{z}} \varphi_{AB{\rm flavor}}
\phi_{A{\rm color}}\phi_{B{\rm color}},
\end{eqnarray}
in which $\mid \vec{p}_{\rm irm} \mid = \mid
\vec{p}_{q_1\bar{q}_1,q_2\bar{q}_2} \mid$ and $\mid \vec{p}_{\rm frm} \mid = 
\mid
\vec{p}_{q_1\bar{q}_4,q_3\bar{q}_2} \mid$. So far, we have obtained new
expressions of the transition amplitudes from Eqs. (2) and (3).

With the transition amplitudes the unpolarized cross section for 
$A+B \to C+D$ is
\begin{eqnarray}
\sigma^{\rm unpol}(\sqrt{s},T) &=&
\frac{1}{(2J_{A}+1)(2J_{B}+1)}\frac{1}{32\pi s}
\frac{\mid \vec{P}^{~\prime}(\sqrt{s}) \mid}{\mid \vec{P}(\sqrt{s}) \mid} 
\nonumber \\
&& 
\times{\int^{\pi}_{0} d\theta \sum_{J_{A_{z}}J_{B_{z}}J_{C_{z}}J_{D_{z}}} 
{\mid {\cal M}_{{\rm a}q_1\bar{q}_2} +{\cal M}_{{\rm a}\bar{q}_1q_2} \mid}^{2}
\sin{\theta}},\label{eq:unpolcs}
\end{eqnarray}
where $s$ is the Mandelstam variable obtained from the four-momenta $P_A$ and
$P_B$ of mesons $A$ and $B$ by $s=(P_A+P_B)^2$; $T$ is the temperature;
$J_A$ ($J_B$, $J_C$, $J_D$) and $J_{Az}$ ($J_{Bz}$, $J_{Cz}$, $J_{Dz}$) 
of meson $A$ ($B$, $C$, $D$) are the total angular momentum and its $z$ 
component, respectively; $\theta$ is the angle between $\vec{P}$ and $\vec{P}'$
which are the three-dimensional momenta of mesons $A$ and $C$ in the 
center-of-mass frame, respectively.
We calculate the cross section in the center-of-mass frame of the two initial 
mesons.


\vspace{0.5cm}
\leftline{\bf III. NUMERICAL CROSS SECTIONS AND DISCUSSIONS }
\vspace{0.5cm}

The quark-antiquark relative-motion wave functions, $\phi_{A\rm rel}$ and
$\phi_{B\rm rel}$ in Eq. (4) as well as $\phi_{C\rm rel}$ and
$\phi_{D\rm rel}$ in Eq. (5), are solutions of the Schr\"odinger equation with
a temperature-dependent quark potential. The potential is given in Ref. 
\cite{SXW}, and originates from perturbative quantum chromodynamics (QCD) 
at short distances and
lattice QCD at intermediate and long distances. The transition potentials
$V_{{\rm a}q_1\bar{q}_2}$ and $V_{{\rm a}\bar{q}_1q_2}$ are derived from
perturbative QCD in Ref.
\cite{SXW}. From the wave functions and the transition potentials we get the
transition amplitudes ${\cal M}_{{\rm a}q_{1}\bar{q}_{2}}$ and 
${\cal M}_{{\rm a}\bar{q}_{1}q_{2}}$. In practical calculations the summations
over $L_{\rm i}$ and $L_{\rm f}$ in the transition amplitudes (see
Eqs. (28) and (29)) are from 0 to 3.
The orbital-angular-momentum quantum numbers $L_{\rm i}$ and $L_{\rm f}$
are selected to satisfy that parity is conserved and that the total angular
momentum of the two final mesons equals the total angular momentum of the two 
initial mesons. Values of $L_{\rm i}$ and $L_{\rm f}$ are listed in Table 1.

\begin{table*}[htbp]
\caption{\label{table1} Total spin and orbital-angular-momentum quantum 
number.}
\tabcolsep=5pt
\begin{tabular}{ccccc}
  \hline
  \hline
reaction & $S$ & $S^\prime$ & $L_{\rm i}$ & $L_{\rm f}$\\
  \hline
$K\phi \to \pi K$ & 1 & 0 & 1 & 1\\
                  & 1 & 0 & 2 & 2\\
                  & 1 & 0 & 3 & 3\\
  \hline
$K\phi \to \rho K$ & 1 & 1 & 0 & 0,2\\
                   & 1 & 1 & 1 & 1,3\\
                   & 1 & 1 & 2 & 0,2\\
                   & 1 & 1 & 3 & 1,3\\
  \hline
$K\phi \to \pi K^*$ & 1 & 1 & 0 & 0,2\\
                    & 1 & 1 & 1 & 1,3\\
                    & 1 & 1 & 2 & 0,2\\
                    & 1 & 1 & 3 & 1,3\\
  \hline
$K\phi \to \rho K^*$ & 1 & 0 & 1 & 1\\
                     & 1 & 0 & 2 & 2\\
                     & 1 & 0 & 3 & 3\\
                     & 1 & 1 & 0 & 0,2\\
                     & 1 & 1 & 1 & 1,3\\
                     & 1 & 1 & 2 & 0,2\\
                     & 1 & 1 & 3 & 1,3\\
                     & 1 & 2 & 0 & 2\\
                     & 1 & 2 & 1 & 1,3\\
                     & 1 & 2 & 2 & 0,2\\
                     & 1 & 2 & 3 & 1,3\\
  \hline
  \hline
\end{tabular}
\end{table*}

We consider the four $K$+$\phi$ reactions:
$K\phi\to\pi K$, $K\phi\to\rho K$, $K\phi\to\pi K^*$, and $K\phi\to\rho K^*$. 
${\cal M}_{{\rm a}q_1\bar{q}_2}$ and ${\cal M}_{{\rm a}\bar{q}_1q_2}$ are 
proportional to flavor matrix elements. Since
the flavor matrix elements for the $K$+$\phi$ reactions with total isospin
$I = \frac{1}{2}$ are zero for ${\cal M}_{{\rm a}q_1\bar{q}_2}$ and 
$-\frac{\sqrt{6}}{2}$ for ${\cal M}_{{\rm a}\bar{q}_1q_2}$, only the process 
$\bar{q}_1+q_2 \to q_{3}+\bar{q}_4$ contributes to these reactions.

The reaction $K\phi\to\rho K^*$ is endothermic at $T/T_c=0$ and exothermic at 
$T/T_c=0.65$, 0.75, 0.85, 0.9, and 0.95. The other three reactions are 
exothermic. Cross sections for exothermic reactions are infinite at threshold 
energies. We thus start calculations of the cross sections for exothermic
reactions at $\sqrt{s}=m_K+m_\phi+10^{-4}$ GeV, where $m_K$ and $m_\phi$ are
the masses of the kaon and the $\phi$ meson, respectively.
Numerical unpolarized cross sections for $K\phi\to\pi K$, $K\phi\to\rho K$, 
$K\phi\to\pi K^*$, and $K\phi\to\rho K^*$ are plotted in Fig. 1 through Fig. 4.

\begin{figure}[htbp]
\centering
\includegraphics[scale=0.65]{kphipik.eps}
\caption{Cross sections for $K \phi \to \pi K$ at various temperatures.}
\label{fig1}
\end{figure}

\begin{figure}[htbp]
\centering
\includegraphics[scale=0.65]{kphirhok.eps}
\caption{Cross sections for $K \phi \to \rho K$ at various temperatures.}
\label{fig2}
\end{figure}

%\newpage
\begin{figure}[htbp]
\centering
\includegraphics[scale=0.65]{kphipika.eps}
\caption{Cross sections for $K \phi \to \pi K^*$ at various temperatures.}
\label{fig3}
\end{figure}

%\newpage
\begin{figure}[htbp]
\centering
\includegraphics[scale=0.65]{kphirhoka.eps}
\caption{Cross sections for $K \phi \to \rho K^*$ at various temperatures.}
\label{fig4}
\end{figure}

The numerical cross sections for endothermic reactions are parametrized as
\begin{eqnarray}
\sigma^{\rm unpol}(\sqrt {s},T)
&=&a_1 \left( \frac {\sqrt {s} -\sqrt {s_0}} {b_1} \right)^{c_1}
\exp \left[ c_1 \left( 1-\frac {\sqrt {s} -\sqrt {s_0}} {b_1} \right) \right]
\nonumber \\
&&+ a_2 \left( \frac {\sqrt {s} -\sqrt {s_0}} {b_2} \right)^{c_2}
\exp \left[ c_2 \left( 1-\frac {\sqrt {s} -\sqrt {s_0}} {b_2} \right) \right],
\label{enparaEq}
\end{eqnarray}
where $\sqrt{s_0}$ is the threshold energy, and $a_1$, $b_1$, $c_1$, $a_2$,
$b_2$, and $c_2$ are parameters. The numerical cross sections
for exothermic reactions are parametrized as
\begin{eqnarray}
\sigma^{\rm unpol}(\sqrt {s},T)
&=&\frac{\vec{P}^{\prime 2}}{\vec{P}^2}
\left\{a_1 \left( \frac {\sqrt {s} -\sqrt {s_0}} {b_1} \right)^{c_1}
\exp \left[ c_1 \left( 1-\frac {\sqrt {s} -\sqrt {s_0}} {b_1} \right) \right]
\right.
\nonumber \\
&&+ \left.
a_2 \left( \frac {\sqrt {s} -\sqrt {s_0}} {b_2} \right)^{c_2}
\exp \left[ c_2 \left( 1-\frac {\sqrt {s} -\sqrt {s_0}} {b_2} \right) \right]
\right\},
\label{exparaEq}
\end{eqnarray}
where
\begin{eqnarray}
{\vec{P}}^2(\sqrt{s})&=&\frac{1}{4s} \left[ \left(s-m^2_A-m^2_B \right) ^2
-4m^2_Am^2_B \right],
\\
{\vec{P}}^{\prime 2}(\sqrt{s})&=&\frac{1}{4s}\left[ \left(s-m^2_C-m^2_D 
\right) ^2-4m^2_Cm^2_D \right],
\end{eqnarray}
in which $m_A$, $m_B$, $m_C$, and $m_D$ are the masses of mesons $A$, $B$, $C$,
and $D$, respectively. The parameter values are listed in Table 2. $d_0$ is the
separation between the peak's location on the $\sqrt s$-axis and the threshold
energy, and $\sqrt{s_z}$ is the square root of the Mandelstam variable at which
the cross section is 1/100 of the peak cross section. For the endothermic
reaction $K\phi \to \rho K^*$ at $T=0$ a peak is displayed in Fig. 4, and 
$d_0=0.25$ GeV and $\sqrt{s_z}=3.04$ GeV are obtained from the numerical cross 
section for $K\phi \to \rho K^*$.
About exothermic reactions we may not see peak cross sections, but 
$\vec{P}^2/\vec{P}^{~\prime 2}$ times numerical cross sections for exothermic
reactions must show peak cross sections. Hence, for exothermic reactions $d_0$ 
and $\sqrt{s_z}$ are obtained from
$\vec{P}^2/\vec{P}^{~\prime 2}$ times the numerical cross sections.

\begin{table*}[htbp] 
\centering\caption{Values of the parameters. $a_1$ and $a_2$ are in units of 
millibarns; $b_1$, $b_2$, $d_0$, and $\sqrt{s_{z}}$ are in units of GeV; $c_1$ 
and $c_2$ are dimensionless.}
\tabcolsep=5pt
\begin{tabular}{*{12}{c}}
\hline
reaction & $T/T_{c}$ & $a_{1}$ & $b_{1}$ & $c_{1}$ & $a_{2}$ & $b_{2}$ & 
$c_{2}$ & $d_{0}$ & $\sqrt{s_{z}}$ \\
\hline
$K\phi\to\pi K$
& 0 & 0.0000026 & 0.17 & 0.62 & 0.0000046 & 0.27 & 1.95 & 0.3 & 2.83 \\ 
& 0.65 & 0.000002 & 0.13 & 0.54 & 0.000003 & 0.24 & 1.2 & 0.15 & 2.57 \\
& 0.75 & 0.0000002 & 0.1 & 0.25 & 0.0000037 & 0.19 & 0.86 & 0.2 & 2.46 \\
& 0.85 & 0.00000076 & 0.239 &  8.37 & 0.00000124 & 0.136 & 0.58 & 0.2 & 2.15\\
& 0.9 & 0.00000042 &  0.271 & 4.7 & 0.00000094 & 0.237 & 0.59 & 0.3 & 1.97 \\
& 0.95 & 0.0000008 & 0.43 & 5.9 & 0.0000009 & 0.23 & 0.55 & 0.45 & 2.17 \\
$K\phi\to\rho K$
& 0 & 0.64 & 0.125 & 2.21 & 0.6 & 0.107 & 0.53 & 0.1 &  2.87 \\
& 0.65 & 0.11 & 0.12 & 0.52 & 0.07 & 0.14 & 2.15 & 0.15 &  2.65 \\
& 0.75 & 0.04 & 0.137 & 0.49 & 0.03 & 0.148 & 1.93 & 0.15 & 1.4 \\
& 0.85 & 0.004 & 0.183 & 2.54 & 0.005 & 0.167 & 0.48 & 0.25 & 2.81 \\
& 0.9 & 0.00233 & 0.202 & 0.5 & 0.00115 & 0.2582 & 4.2 & 0.25 & 2.2 \\
& 0.95 & 0.0009 & 0.371 & 6.63 & 0.0014 & 0.258 & 0.53 & 0.4 & 3.52 \\
$K\phi\to\pi K^*$ 
& 0 & 0.1 & 0.153 & 0.34 & 0.33 & 0.096 & 0.95 & 0.1 & 2.94 \\ 
& 0.65 & 0.121 & 0.132 & 0.51 & 0.068 & 0.121 & 1.84 & 0.15 & 2.69 \\
& 0.75 & 0.05 & 0.147 & 0.5 & 0.027 & 0.127 & 1.65 & 0.15 & 2.52 \\
& 0.85 & 0.0025 & 0.2 & 3.4 & 0.0044 & 0.18 & 0.55 & 0.2 &  2.14 \\
& 0.9 & 0.00089 & 0.211 & 0.54 & 0.00145 & 0.256 & 2.24 & 0.35 & 2.27 \\
& 0.95 & 0.0014 & 0.358 & 3.9 & 0.0016 & 0.153 & 0.55 & 0.3 & 1.86 \\
$K\phi\to\rho K^*$
& 0 & 0.29 & 0.17 & 0.72 & 1.29 & 0.22 & 6.18 & 0.25 & 3.04 \\
& 0.65 & 0.062 & 0.14 & 0.49 & 0.096 & 0.226 & 3.9 & 0.25 & 4.19 \\
& 0.75 & 0.013 & 0.202 & 3.04 & 0.019 & 0.123 & 0.55 & 0.2 & 2.48 \\
& 0.85 & 0.00018 & 0.34 & 0.28 & 0.00059 & 0.11 & 0.91 & 0.15 & 2.16 \\
& 0.9 & 0.000087 & 0.199 & 1.86 & 0.000105 & 0.153 & 0.54 & 0.25 & 1.48 \\
& 0.95 & 0.00007 & 0.26 & 2.7 & 0.00013 & 0.17 & 0.62 & 0.2 & 18.02 \\
\hline
\end{tabular}
\end{table*}

The threshold energy for each of the exothermic reactions 
$K\phi\to\pi K$, $K\phi\to\rho K$, and $K\phi\to\pi K^*$ are the sum of the 
$K$ and $\phi$ masses. When the
temperature increases, decreases of the masses lead to a decrease in the 
threshold energy as shown in Figs. 1-3. As $\sqrt s$ increases 
near the threshold energy, the cross 
sections for these reactions decrease rapidly due to the factor 
$\mid \vec{P}^{~\prime} \mid / \mid \vec{P} \mid$ in Eq. (30).
The threshold energy of the endothermic reaction 
$K\phi\to\rho K^*$ at $T=0$ is the sum of the $\rho$ and $K^*$ masses.
As $\sqrt s$ increases from the threshold energy, the cross section for 
$K\phi\to\rho K^*$ at $T=0$ increases rapidly from zero, reaches a peak value 
of about 1.66 mb, and then decreases.

The total spin of the two initial mesons in the reaction $K\phi\to\pi K$ is 1,
and the total spin of the two final mesons is 0. Because the two total spins 
are unequal, the cross section for $K\phi\to\pi K$ is very small. About
$K\phi\to\rho K$, $K\phi\to\pi K^*$, and $K\phi\to\rho K^*$ the total spin of
the two initial mesons may equal the one of the two final mesons, and the cross
sections may be a few millibarns when $\sqrt s$ is not at the threshold energy.


\vspace{0.5cm}
\leftline{\bf IV. SUMMARY }
\vspace{0.5cm}

With the development in spherical harmonics of the relative-motion wave 
functions of the two initial mesons and of the two final mesons, new 
expressions
of the transition amplitudes have been obtained. With the transition amplitudes
we have
calculated unpolarized cross sections for $K\phi\to\pi K$, 
$K\phi\to\rho K$, $K\phi\to\pi K^*$, and $K\phi\to\rho K^*$, which are governed
by quark-antiquark annihilation and creation. Both parity conservation and
total-angular-momentum conservation are maintained. To use the numerical 
cross sections conveniently, we have parametrized the cross 
sections. Each of the exothermic reactions $K\phi\to\pi K$, $K\phi\to\rho K$, 
and $K\phi\to\pi K^*$ exhibits a rapid decrease first and then a slow decrease
in cross section with 
increasing $\sqrt{s}$ from the threshold energy. That
the reaction $K\phi\to\rho K^*$ is endothermic or exothermic depends on
temperature. The temperature-dependent cross sections are related to 
temperature dependence of the quark potential, the quark-antiquark 
relative-motion wave functions, and the meson masses.


\vspace{0.5cm}
\leftline{\bf ACKNOWLEDGEMENTS}
\vspace{0.5cm}

This work was supported by the National Natural Science Foundation of China
under Grant No. 11175111.


\begin{thebibliography}{00}
\bibitem{RM} J. Rafelski and B. M$\ddot{\rm u}$ller, Phys. Rev. Lett. 48, 1066 
(1982).
\bibitem{AS} A. Shor, Phys. Rev. Lett. 54, 1122 (1985).
\bibitem{STAR2007} B. I. Abelev {\it et al.}, Phys. Rev. Lett. 99, 112301 
(2007).
\bibitem{STAR2009PLB} B. I. Abelev {\it et al.}, Phys. Lett. B 673, 183 (2009).
\bibitem{STAR2009PRC} B. I. Abelev {\it et al.}, Phys. Rev. C 79, 064903 
(2009).
\bibitem{PHENIX2011} A. Adare {\it et al.}, Phys. Rev. C 83, 024909 (2011).
\bibitem{STAR2013} L. Adamczyk {\it et al.}, Phys. Rev. C 88, 014902 (2013).
\bibitem{STAR2016} L. Adamczyk {\it et al.}, Phys. Rev. C 93, 021903(R) (2016).
\bibitem{STAR2020} J. Adam {\it et al.}, Phys. Rev. C 102, 034909 (2020).
\bibitem{STAR2022} M. S. Abdallah {\it et al.}, Phys. Rev. C 105, 064911 
(2022).
\bibitem{PHENIX2023} N. J. Abdulameer {\it et al.}, Phys. Rev. C 107, 014907
(2023).
\bibitem{ALICE2015} B. Abelev {\it et al.}, Phys. Rev. C 91, 024609 (2015).
\bibitem{ALICE2017} J. Adam {\it et al.}, Phys. Rev. C 95, 064606 (2017).
\bibitem{ALICE2020} S. Acharya {\it et al.}, Phys. Lett. B 802, 135225 (2020).
\bibitem{ALICE2022} S. Acharya {\it et al.}, Phys. Rev. C 106, 034907 (2022).
\bibitem{BR} P.-Z. Bi and J. Rafelski, Phys. Lett. B 262, 485 (1991).
\bibitem{KS} C. M. Ko and D. Seibert, Phys. Rev. C 49, 2198 (1994).
\bibitem{Haglin} K. Haglin, Nucl. Phys. A 584, 719 (1995).
\bibitem{SH} W. Smith and K. L. Haglin, Phys. Rev. C 57, 1449 (1998).
\bibitem{CLK} W. S. Chung, G. Q. Li, and C. M. Ko, Nucl. Phys. A 625, 347 
(1997).
\bibitem{AK} L. Alvarez-Ruso and V. Koch, Phys. Rev. C 65, 054901 (2002).
\bibitem{MKAN} A. Mart\'{i}nez Torres, K. P. Khemchandani, L. M. Abreu, 
F. S. Navarra, and M. Nielsen, Phys. Rev. D 97, 056001 (2018).  
\bibitem{LX} Z.-F. Luo and X.-M. Xu, Chin. Phys. C 36, 836 (2012).
\bibitem{XW} X.-M. Xu and H. J. Weber, Mod. Phys. Lett. A 35, 2030016 (2020).
\bibitem{CLAS} X. Qian {\it et al.}, Phys. Lett. B 680, 417 (2009).
\bibitem{HADES} J. Adamczewski-Musch {\it et al.}, Phys. Rev. Lett. 123, 022002
(2019).
\bibitem{BS} T. Barnes and E. S. Swanson, Phys. Rev. D 46, 131 (1992).
\bibitem{Swanson} E. S. Swanson, Ann. Phys. (N.Y.) 220, 73 (1992).
\bibitem{SXW} Z.-Y. Shen, X.-M. Xu, and H. J. Weber, Phys. Rev. D 94, 034030 
(2016).
\bibitem{WX} T.-T. Wang and X.-M. Xu, Chin. Phys. C 43, 024102 (2019). 
\bibitem{AW} G. B. Arfken and H. J. Weber, {\it Mathematical Methods for 
Physicists} (Elsevier, Amsterdam, 2006).
\bibitem{joachain} C. J. Joachain, {\it Quantum Collision Theory} 
(North-Holland Publishing Company, Amsterdam, 1983).
\end{thebibliography}

\end{document}
\endinput
