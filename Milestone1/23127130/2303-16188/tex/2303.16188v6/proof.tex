\section{The Proof of Theorem~\ref{thm:srklinear}}
\label{sec:srklinear}
We first present a lemma from~Lin et al.~\cite{lin2021greedy}.
\begin{lemma}[\!\!{\cite[Lemma 25]{lin2021greedy}}]
\label{lm:strong_self}
If the twice differentiable function $f:\BR^d\to\BR$ is $M$-strongly self-concordant and $\mu$-strongly convex, and the positive-definite matrix~$\G\in\RB^{d\times d}$ and the vector $\vx\in\BR^d$ satisfy $\nabla^2 f(\x)\preceq \G\preceq \eta\nabla^2 f(\x)$ for some~$\eta>1$, then we have
\begin{align}
& \nabla^2 f(\x_{+})\preceq\tilde{\G}\preceq \eta(1+Mr)^2\nabla^2 f(\x_{+}) \label{eq:tilde_G_geq} \\
& \text{and}~~~\tau_{\nabla^2f(\x_{+})}(\tilde{\G})\leq (1+Mr)^2\left(\frac{\tau_{\nabla^2 f(\x)}(\G)}{\trcommon{\nabla^2f(\x)}}+2Mr\right)\tr{\nabla^2f(\x_{+})}    \label{eq:delta_SRK_lin}
\end{align}
for any $\vx_+\in\BR^d$, where $\tilde{\G}=(1+Mr)\G$ and $r=\|\x-\x_{+}\|_{\x}$.
%and the notation of $\tau_\H(\G)$ follows the expressions of~\eqref{eq:measure_srk}.
\end{lemma}





Now, we prove Theorem~\ref{thm:srklinear} by using Lemma~\ref{lm:sr1good}, \ref{lm:linear-quadra}, \ref{lm:linear}, and \ref{lm:strong_self}.



\begin{proof}
Denote $\tilde{\eta}_t \triangleq \min_{\G_t\preceq \eta\nabla^2 f(\x_t)}\eta$. We first prove that for all $t\geq 0$, it holds
\begin{align}\label{psd:G_t}
    \nabla^2 f(\x_t)\preceq \G_t\preceq \tilde{\eta}_t\nabla^2 f(\x_t).
\end{align}
The result of $\G_t\preceq \tilde{\eta}_t\nabla^2 f(\x_t)$ holds by the definition of $\tilde{\eta}_t$, and we only need to prove~$\nabla^2 f(\x_t)\preceq \G_t$ by induction.
For the case of $t=0$, it holds naturally by initial condition. Suppose it holds for $t=0,1,\dots,\hat{t}$. Then for the case of $t=\hat{t}+1$, we have
\begin{align*}
\G_{\hat{t}+1}=\srk(\tilde{\G}_{\hat{t}},\nabla^2 f(\x_{\hat{t}+1}),\U_{\hat{t}})\overset{\eqref{eq:srk_good}}{\succeq} \tilde{\G}_{\hat{t}} \overset{\eqref{eq:tilde_G_geq}}{\succeq} \nabla^2  f(\x_{\hat{t}+1}),
\end{align*}
which finishes the induction.

Let $\lambda_t\triangleq\lambda(\x_t)$. 
% and $\tilde{\eta}_t \triangleq \min_{\G_t\preceq \eta\nabla^2 f(\x_t)}\eta$,
% % \begin{align*}
% % \lambda_t\triangleq\lambda(\x_t)\qquad\text{and}\qquad\tilde{\eta}_t \triangleq \min_{\nabla^2 f(\x_t)\preceq \eta\G_t}\eta,
% % \end{align*}
% which means 
% \begin{align*}
%    \nabla^2f(\x_t)\preceq \G_t\preceq \tilde{\eta}_t\nabla^2 f(\x_t).
% \end{align*}
Applying equation (\ref{psd:G_t}) and Lemma~\ref{lm:linear-quadra}, we achieve
\begin{align*}
        \lambda_{t+1}\leq \left(1-\frac{1}{\tilde{\eta}_t}\right)\lambda_t + \frac{M\lambda_t^2}{2} + \frac{M^2\lambda_t^3}{4\tilde{\eta}_t}~~~\text{for all $\lambda_t$ such that $M\lambda_t\leq 2$}.
\end{align*}
We then figure out the relation between $\tilde{\eta}_{t+1}$ and $\tilde{\eta}_t$. 
Let $r_t=\|\x_{t+1}-\x_t\|_{\x_t}$.
Applying equation (\ref{psd:G_t}) and Lemma~\ref{lm:strong_self} with $\G=\G_t$, $\x=\x_t$, $\x_{+}=\x_{t+1}$, $\eta = \tilde{\eta}_t$, and $r=r_t$, we achieve
\begin{align}
\label{eq:tildeG_t}
      \nabla^2 f(\x_{t+1})\preceq\tilde{\G}_t\preceq \tilde{\eta}_t(1+Mr_t)^2\nabla^2 f(\x_{t+1}).
\end{align}
Using Lemma~\ref{lm:sr1good} with $\G=\tilde{\G}_t$, $\A= \nabla^2 f(\x_{t+1})$, and $\eta=\tilde{\eta}_t(1+Mr_t)^2$, we have
\begin{align*}   
& \G_{t+1}
\overset{\eqref{eq:srk_good},\,\eqref{eq:tildeG_t}}{\preceq} \tilde{\eta}_t(1+Mr_t)^2\nabla^2 f(\x_{t+1})\overset{\eqref{eq:linear-quadra}}{\preceq} \tilde{\eta}_t(1+M\lambda_t)^2\nabla^2 f(\x_{t+1}),
\end{align*}
which means
$\tilde{\eta}_{t+1}=\min_{\G_{t+1}\preceq \eta \nabla^2 f(\x_{t+1})}\eta\leq \tilde{\eta}_t(1+M\lambda_t)^2$.
Hence, sequences $\{\tilde{\eta}_t\}$ and~$\{\lambda_t\}$ satisfy conditions of Lemma~\ref{lm:linear} with $m_1=m_2=M$. We then apply Lemma~\ref{lm:linear} to obtain results
$\nabla^2 f(\x_t)\preceq \tilde{\G}_t \preceq ({3\tilde{\eta}_0}/{2})\nabla^2f(\x_t)\preceq({3{\eta}_0}/{2})\nabla^2 f(\x_t)$
and $\lambda(\x_t)\leq \left(1-{1}/{2\tilde{\eta}_0}\right)^t\lambda(\x_0)\leq\left(1-{1}/{2{\eta}_0}\right)^t\lambda(\x_0)$, which finishes the proof.
% \begin{align*}
%   & \nabla^2 f(\x_t)\preceq \tilde{\G}_t \preceq \frac{3\tilde{\eta}_0}{2}\nabla^2f(\x_t)\preceq\frac{3{\eta}_0}{2}\nabla^2 f(\x_t)\\
%  \text{and}~~~~~ &  \lambda(\x_t)\leq \left(1-\frac{1}{2\tilde{\eta}_0}\right)^t\lambda(\x_0)\leq\left(1-\frac{1}{2{\eta}_0}\right)^t\lambda(\x_0).
% \end{align*}
\end{proof}


\section{The Proof of Theorem~\ref{thm:srk}}
\label{sec:srk_proof}
We first provide some auxiliary lemmas which will be used in our later proof.

\begin{lemma}
\label{lm:GHneq}
For any positive definite symmetric matrices $\G,\H\in\RB^{d\times d}$ such that~$\H\preceq\G$, 
it holds that
\begin{align}
\label{eq:GHneq}
\G\preceq \left(1+\frac{\hat\varkappa d \tau_{\H}(\G)}{\trcommon{\H}}\right)\H,
\end{align}
where $\hat\varkappa$ is the condition number of $\H$ and the notation of $\tau_\H(\G)$ follows the expression of \eqref{eq:measure_srk}.
If it further satisfies that $\G\preceq\eta\H$ for some $\eta\geq 1$, then we have
\begin{align}
\label{eq:GHneq_2}
  \qquad\frac{\tau_{\H}(\G)}{{\rm tr}(\H)} \leq \eta -1.
\end{align}
\end{lemma}
\begin{proof}
We obtain inequality \eqref{eq:GHneq} by statement 3) of equation (42) from Lin et al. \cite[Lemma 25]{lin2021greedy}.
We prove inequality \eqref{eq:GHneq_2} by the definition of $\tau_{\H}(\G)$ as follows
\begin{align*}
     \frac{\tau_{\H}(\G)}{\trcommon{\H}} =\frac{ \trcommon{\G-\H}}{\trcommon{\H}}{\leq} \frac{\trcommon{(\eta-1)\H}} {\trcommon{\H}}= \eta-1.
\end{align*}

\end{proof}



\begin{lemma}[\!\!{\cite[Lemma 26]{lin2021greedy}}]
\label{lm:randomsequence_lin}
Suppose the non-negative random sequence $\{X_t\}$
satisfies $\EB[X_t]\leq c\left(1-{1}/{\alpha}\right)^t$ for all $t\geq0$ and some constants $c\geq 0$ and $\alpha>1$. 
Then for any $\delta\in(0,1)$, we have $  X_t\leq {c\alpha^2}\left(1-{1}/{(1+\alpha)}\right)^t/\delta$
% \begin{align*}
%     X_t\leq \frac{c\alpha^2}{\delta}\cdot\left(1-\frac{1}{1+\alpha}\right)^t
% \end{align*}
for all $t$ with probability at least $1-\delta$.
\end{lemma}

Now, we provide the proof of Theorem~\ref{thm:srk}.

\begin{proof}
We denote 
$\EB_{t}[\,\cdot\,]\triangleq \EB[\,\cdot\,|\,\U_0,\cdots,\U_{t-1}]$, 
$g_t={\tau_{\nabla^2 f(\x_t)}(\G_t)}/{\trcommon{\nabla^2 f(\x_t)}}$, 
$\lambda_t=\lambda(\x_t)$, $r_t=\|\x_{t+1}-\x_t\|_{\x_t}$,
% g_t=\tau_{\nabla^2 f(\x_t)}(\G_t)}/\tr{\nabla^2 f(\x_t)}$, 
and $\delta_t=d\varkappa g_t$.
The initial condition \eqref{eq:initial} means we can apply
Theorem~\ref{thm:srklinear} to achieve
\begin{align}\label{eq:linearrate}
    \nabla^2 f(\x_t)\preceq\G_t
    \qquad\text{and}\qquad
    \lambda_{t}\leq \left(1-\frac{1}{2\eta_0}\right)^{t}\lambda_0.
\end{align}
According to Theorem~\ref{thm:matrix}, we have
\begin{align}
\label{eq:srk_G-H}
    \EB_{t}\left[\tau_{\nabla^2 f(\x_{t+1})}(\G_{t+1})\right] \overset{\eqref{eq:measure_srk}}{\leq} \left(1-\frac{k}{d}\right) \tau_{\nabla^2 f(\x_{t+1})}(\tilde{\G}_{t}).
\end{align}
According to Lemma~\ref{lm:strong_self} with $\G=\G_t$,  $\x=\x_t$, $\x_{+}=\x_{t+1}$, and $r=r_t$, we have
\begin{align}
\label{eq:tau_neq}
    \EB_t[\tau_{\nabla^2f(\x_{t+1})}(\tilde{\G}_t)]\overset{\eqref{eq:delta_SRK_lin}}{\leq} (1+Mr_t)^2(g_t + 2Mr_t)\trcommon{\nabla^2f(\x_{t+1})},
\end{align}
which means
\begin{align}
\label{eq:delta_t_condi}
\begin{split}
      & \EB_{t}[\delta_{t+1}] = \EB_{t}\left[\frac{d\varkappa \tau_{\nabla^2 f(\x_{t+1})}(\G_{t+1})}{\trcommon{\nabla^2 f(\x_{t+1})}}\right] 
       \overset{\eqref{eq:srk_G-H}}{\leq} \left(1-\frac{k}{d}\right)\EB_t\left[\frac{d\varkappa \tau_{\nabla^2 f(\x_{t+1})}(\tilde{\G}_t)}{\trcommon{\nabla^2 f(\x_{t+1})}}\right]\\
      &\overset{\eqref{eq:tau_neq}}{\leq} \left(1-\frac{k}{d}\right)\EB_t\left[\frac{d\varkappa(1+Mr_t)^2(g_t + 2Mr_t)\trcommon{\nabla^2f(\x_{t+1})} }{\trcommon{\nabla^2 f(\x_{t+1})}}\right]\\
      &\overset{\eqref{eq:linear-quadra}}{\leq}\left(1-\frac{k}{d}\right)(1+M\lambda_t)^2(\delta_t+2\varkappa dM\lambda_t).
\end{split}
\end{align}
% Next, we prove 
% \begin{align}
% \label{eq:GtsucceqHt}
%     \G_{t}\succeq\nabla^2 f(\x_t)
% \end{align}
% by induction. \eqref{eq:GtsucceqHt} holds for $t=0$ by the initial condition. Suppose it holds for $t=0,\cdots,t'$, then for $t=t'+1$, according to Lemma~\ref{lm:strong_self}, we have $\tilde{\G}_{t'}\succeq\nabla^2 f(\x_{t'})$. Combining with Lemma~\ref{lm:sr1good}, we have
% \begin{align*}
%     \G_{t'+1}\overset{\eqref{eq:srk_good}}{\succeq}\nabla^2 f(\x_{t'+1}),
% \end{align*}
% which finishes the induction.
According to Lemma~\ref{lm:GHneq} with $\G=\G_t$ and $\H=\nabla^2 f(\x_t)$, we have
\begin{align*}
    \nabla^2 f(\x_{t})\overset{\eqref{eq:linearrate}}{\preceq} \G_t \overset{\eqref{eq:GHneq}}{\preceq} (1+\delta_t)\nabla^2 f(\x_t).
\end{align*}
According to Lemma~\ref{lm:linear-quadra} with $\eta_t=1+\delta_t$, we have
 \begin{align}
 \label{eq:lambda_t_condi}
\begin{split}
        & \lambda_{t+1}
        \overset{\eqref{eq:linear-quadra}}{\leq} \left(1-\frac{1}{1+\delta_t}\right)\lambda_t +\frac{M\lambda_t^2}{2} + \frac{M^2\lambda_t^3}{4(1+\delta_t)}\\ 
        &= \left(1+\frac{M\lambda_t}{2}\right)\cdot\left(\frac{\delta_t + {M\lambda_t}/{2}}{1+\delta_t}\right)\lambda_t  \,\,\leq\left(1+M\lambda_t\right)^2\left(\delta_t+\frac{M\lambda_t}{2} \right)\lambda_t.
\end{split}
 \end{align}
According to Lemma \ref{lm:GHneq} with $\G=\G_0$, $\H=\nabla^2 f(\x_0)$, $\eta=\eta_0$, and the initial condition \eqref{eq:initial}, we have
\begin{align}
\label{eq:delta_0_bound}
    &\delta_0=\frac{d\varkappa\tau_{\nabla^2 f(\x_0)}(\G_0)}{\trcommon{\nabla^2 f(\x_0)}}\overset{\eqref{eq:GHneq_2}}{\leq}d\varkappa(\eta_0-1)\\
    \label{eq:initial_condi_theta}
    & \text{and}~~~ \delta_0+2d\varkappa M\lambda_0\overset{\eqref{eq:delta_0_bound}}{\leq} d\varkappa (\eta_0 -1) +2d\varkappa M\lambda_0\overset{\eqref{eq:initial}}{\leq}  d\varkappa\eta_0. 
\end{align}
Then we apply Lemma~\ref{lm:superlinear} 
on the random sequences $\{\lambda_t\}$ and $\{\delta_t\}$ with
$c_1=M$, $c_2={M}/{2}$, $c_3=2\varkappa dM$, $\alpha={d}/{k}$, $\beta=2\eta_0$, and $s=d\varkappa\eta_0$
% \begin{align*}
%     c_1=M,\quad c_2=\frac{M}{2}, \quad c_3=2\varkappa dM, 
%      \quad \alpha=\frac{d}{k}, \quad {\color{red}\beta=2\eta_0},\quad \text{and}\quad s=d\varkappa\eta_0 
% \end{align*}
to achieve inequality~\eqref{eq:E_lambda_srk}, where we can verify conditions \eqref{eq:supercondi} and \eqref{eq:supercondi_2} by equations \eqref{eq:delta_t_condi}, \eqref{eq:lambda_t_condi} and \eqref{eq:linearrate}, \eqref{eq:initial_condi_theta} respectively.
% Noticing that equations \eqref{eq:delta_t_condi} and \eqref{eq:lambda_t_condi} corresponds to condition \eqref{eq:supercondi} is satisfied, \eqref{eq:linearrate} and \eqref{eq:initial_condi_theta} means condition~\eqref{eq:supercondi_2} is satisfied.).

Now, we prove the two-stage convergence of \srk~methods as follows. 
\begin{enumerate}[label=(\alph*),topsep=0.05cm, itemsep=0.1cm, leftmargin=0.9cm] 
\item For \srk~method with randomized strategy such that $\left[\U_{t}\right]_{ij}\overset{\rm{i.i.d}}{\sim}{\fN(0,1)}$, we apply Lemma~\ref{lm:randomsequence_lin} with $X_t=\lambda_{t+1}/\lambda_t$, $\alpha = d/k$ and $c=2d\varkappa\eta_0$ to obtain 
 \begin{align}
 \label{eq:rasrk_super}
    \frac{ \lambda_{t+1}}{\lambda_t}  \leq \frac{2d^3\varkappa\eta_0}{k^2\delta}\left(1-\frac{k}{d+k}\right)^{t}
 \end{align}
holds for all $t$ with probability at least $1-\delta$. 
We take $t_0= \OM(d\ln(d\varkappa \eta_0/\delta)/k)$, which leads to
\begin{align}
\label{eq:rasrk_t0}
    \frac{2d^3\varkappa\eta_0}{k^2\delta}\left(1-\frac{k}{d+k}\right)^{t_0}\leq\frac{1}{2}.
\end{align}
Together with the linear convergence \eqref{eq:srklinear} provided by Theorem \ref{thm:srklinear}, we have
\begin{equation*}
\begin{split}
    &\lambda_{t+t_0}
    \overset{\eqref{eq:rasrk_super}}{\leq} \frac{2d^3\varkappa\eta_0}{k^2\delta}\cdot\left(1-\frac{k}{d+k}\right)^{t+t_0-1}\cdot\lambda_{t+t_0-1} \\
    & \!\!\overset{\eqref{eq:rasrk_t0}}{\leq} \left(1-\frac{k}{d+k}\right)^{t-1}\!\!\cdot\frac{\lambda_{t+t_0-1}}{2}\leq \cdots \overset{\eqref{eq:rasrk_super},\,\eqref{eq:rasrk_t0}}{\leq}\!\!\left(1-\frac{k}{d+k}\right)^{t(t-1)/2}\!\!\cdot\left(\frac{1}{2}\right)^{t}\lambda_{t_0}\\
    &\overset{\eqref{eq:srklinear}}{\leq} \left(1-\frac{k}{d+k}\right)^{t(t-1)/2}\cdot\left(\frac{1}{2}\right)^t\cdot\left(1-\frac{1}{2\eta_0}\right)^{t_0}\lambda_0
\end{split}
\end{equation*}
with probability at least $1-\delta$.

\item 
For \srk~method with greedy strategy such that $\U_t=\E_k(\tilde{\G}_t-\nabla^2 f(\x_{t+1}))$, we can take  $t_0=\OM\left(d\ln(\eta_0d\varkappa )/k\right)$ such that
\begin{align}
\label{eq:grsrk_t0}
  2d\varkappa\eta_0  \left(1-\frac{k}{d}\right)^{t_0}\leq \frac{1}{2}.
\end{align}
Together with the linear convergence \eqref{eq:srklinear} provided by Theorem \ref{thm:srklinear}, we have
\begin{align*}
    & \lambda_{t+t_0} \overset{\eqref{eq:E_lambda_srk}}{\leq} 2d\varkappa\eta_0\left(1-\frac{k}{d}\right)^{t+t_0-1}\lambda_{t+t_0-1} \\
    & \overset{\eqref{eq:grsrk_t0}}{\leq} \left(1-\frac{k}{d}\right)^{t-1}\cdot\frac{\lambda_{t+t_0-1}}{2}\leq \cdots \overset{\eqref{eq:E_lambda_srk},\,\eqref{eq:grsrk_t0}}{\leq} \left(1-\frac{k}{d}\right)^{t(t-1)/2}\cdot\left(\frac{1}{2}\right)^{t}\lambda_{t_0}\\
    &\,\overset{\eqref{eq:srklinear}}{\leq} \left(1-\frac{k}{d}\right)^{t(t-1)/2}\cdot\left(\frac{1}{2}\right)^t\cdot\left(1-\frac{1}{2\eta_0}\right)^{t_0}\lambda_0.
\end{align*}
\end{enumerate}
 \end{proof}
 



