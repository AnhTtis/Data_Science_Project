\documentclass{article}

%\usepackage[nonatbib]{neurips_2022}
\usepackage[utf8]{inputenc}
\usepackage{amsmath}
\usepackage{amssymb}
\usepackage{multirow, hhline}
\usepackage[table]{xcolor}
\usepackage[para,online,flushleft]{threeparttable}

%arxiv version
\usepackage[letterpaper,margin=1.0in]{geometry}
\usepackage{graphicx}


%%%%%%%%%%%%%%%%%%
\usepackage{hyperref}       % hyperlinks
\usepackage{url}            % simple URL typesetting
\usepackage{booktabs}       % professional-quality tables
\usepackage{amsfonts}       % blackboard math symbols
\usepackage{bbding}
\usepackage{xcolor}         % colors
\usepackage{algorithm}
\usepackage{algorithmic}
\usepackage{graphicx}
\usepackage{enumitem}
%\usepackage{multirow}
\usepackage[numbers,sort]{natbib}
\usepackage{color}
\newcommand{\pc}{PC}

\newcommand{\inputraw}{x}
\newcommand{\bow}{\textbf{x}}
\newcommand{\vocabulary}{\mathcal{V}}
\newcommand{\bowlen}{N}

\newcommand{\assignment}{\mathsf{A}}

\newcommand{\corpus}{D}
\newcommand{\corpusdoc}{\bow}
\newcommand{\corpusdoclen}{\left|\corpusdoc\right|}

\newcommand{\topicwordprior}{\beta}
\newcommand{\topicdocumentprior}{\alpha}

\newcommand{\pdf}{\text{PDF}}
\newcommand{\submission}{\bow}
\newcommand{\submissions}{\mathcal{S}}

\newcommand{\reviewersset}{\mathcal{R}}
\newcommand{\reviewersubset}{R}
\newcommand{\surroundingreviewers}{R}
\newcommand{\reviewer}{r}
\newcommand{\archive}{A}

\newcommand{\reviewerload}{\mathsf{L}_{\reviewer}}
\newcommand{\paperload}{\mathsf{L}_{\submission}}

\newcommand{\select}{\text{sel}}
\newcommand{\reject}{\text{rej}}

\newcommand{\requestedreviewers}{\reviewersubset_{\select}}
\newcommand{\rejectedreviewers}{\reviewersubset_{\reject}}

\newcommand{\bid}{b}


\newcommand{\reviewerwindow}{\omega}
\newcommand{\revieweroffset}{\upsilon}
\newcommand{\reviewerwordsmax}{\nu}

\newcommand{\reviewerwordsmass}{Q}
\newcommand{\reviewerwords}{\hat{\topicworddist}}
\newcommand{\reviewertopics}{\tau}


\newcommand{\topic}{t}
\newcommand{\topics}{T}
\newcommand{\topicssubset}{\tau}
\newcommand{\topicspace}{\mathcal{T}}

\newcommand{\stepsize}{k}
\newcommand{\nosuccessors}{M}

\newcommand{\loss}{\ell}

\newcommand{\cardinalitytopicssubset}{k}
\newcommand{\cardinalitytopics}{\left| \topics \right|}

\newcommand{\topicsubsetmarginprob}{\lambda}
\newcommand{\topicsubsetminprob}{\nu}

\newcommand{\extractor}{\Phi}
\newcommand{\topicextractor}{\Gamma}
\newcommand{\topicworddist}{\phi}
\newcommand{\topicworddisthat}{\hat{\topicworddist}}
\newcommand{\topicdocumentdist}{\theta}
\newcommand{\ldadocument}{d}
\newcommand{\word}{w}

\newcommand{\prob}{P}

\newcommand{\beamwidth}{B}
\newcommand{\maxitr}{I}
\newcommand{\clusters}{C}
\newcommand{\switches}{S}

\newcommand{\attackbudgetscale}{\sigma}

\newcommand{\modifications}{\delta}
\newcommand{\modificationsmannorm}{{\left| \left| \modifications \right| \right|}_1}
\newcommand{\modificationsinfnorm}{{\left| \left| \modifications \right| \right|}_\infty}

\newcommand{\maxmannorm}{L_1^\text{max}}
\newcommand{\maxinfnorm}{L_\infty^\text{max}}

\newcommand{\margin}{\gamma}

\newcommand{\submissionpdf}{\submission}

% Everything related with problem space
\newcommand{\Dom}{\ensuremath{\mathcal{Z}}\xspace}
\newcommand{\F}{\ensuremath{\mathcal{F}}\xspace}
\newcommand{\preprocessing}{\ensuremath{\rho}\xspace}
\newcommand{\inputpdf}{\ensuremath{z}\xspace}

% Code transformations, search strategy
\newcommand{\transformation}{\ensuremath{\omega}\xspace}
\newcommand{\transformations}{\ensuremath{\Omega}\xspace}

%\newcommand{\pc}{PC}

\newcommand{\inputraw}{x}
\newcommand{\bow}{\textbf{x}}
\newcommand{\vocabulary}{\mathcal{V}}
\newcommand{\bowlen}{N}

\newcommand{\assignment}{\mathsf{A}}

\newcommand{\corpus}{D}
\newcommand{\corpusdoc}{\bow}
\newcommand{\corpusdoclen}{\left|\corpusdoc\right|}

\newcommand{\topicwordprior}{\beta}
\newcommand{\topicdocumentprior}{\alpha}

\newcommand{\pdf}{\text{PDF}}
\newcommand{\submission}{\bow}
\newcommand{\submissions}{\mathcal{S}}

\newcommand{\reviewersset}{\mathcal{R}}
\newcommand{\reviewersubset}{R}
\newcommand{\surroundingreviewers}{R}
\newcommand{\reviewer}{r}
\newcommand{\archive}{A}

\newcommand{\reviewerload}{\mathsf{L}_{\reviewer}}
\newcommand{\paperload}{\mathsf{L}_{\submission}}

\newcommand{\select}{\text{sel}}
\newcommand{\reject}{\text{rej}}

\newcommand{\requestedreviewers}{\reviewersubset_{\select}}
\newcommand{\rejectedreviewers}{\reviewersubset_{\reject}}

\newcommand{\bid}{b}


\newcommand{\reviewerwindow}{\omega}
\newcommand{\revieweroffset}{\upsilon}
\newcommand{\reviewerwordsmax}{\nu}

\newcommand{\reviewerwordsmass}{Q}
\newcommand{\reviewerwords}{\hat{\topicworddist}}
\newcommand{\reviewertopics}{\tau}


\newcommand{\topic}{t}
\newcommand{\topics}{T}
\newcommand{\topicssubset}{\tau}
\newcommand{\topicspace}{\mathcal{T}}

\newcommand{\stepsize}{k}
\newcommand{\nosuccessors}{M}

\newcommand{\loss}{\ell}

\newcommand{\cardinalitytopicssubset}{k}
\newcommand{\cardinalitytopics}{\left| \topics \right|}

\newcommand{\topicsubsetmarginprob}{\lambda}
\newcommand{\topicsubsetminprob}{\nu}

\newcommand{\extractor}{\Phi}
\newcommand{\topicextractor}{\Gamma}
\newcommand{\topicworddist}{\phi}
\newcommand{\topicworddisthat}{\hat{\topicworddist}}
\newcommand{\topicdocumentdist}{\theta}
\newcommand{\ldadocument}{d}
\newcommand{\word}{w}

\newcommand{\prob}{P}

\newcommand{\beamwidth}{B}
\newcommand{\maxitr}{I}
\newcommand{\clusters}{C}
\newcommand{\switches}{S}

\newcommand{\attackbudgetscale}{\sigma}

\newcommand{\modifications}{\delta}
\newcommand{\modificationsmannorm}{{\left| \left| \modifications \right| \right|}_1}
\newcommand{\modificationsinfnorm}{{\left| \left| \modifications \right| \right|}_\infty}

\newcommand{\maxmannorm}{L_1^\text{max}}
\newcommand{\maxinfnorm}{L_\infty^\text{max}}

\newcommand{\margin}{\gamma}

\newcommand{\submissionpdf}{\submission}

% Everything related with problem space
\newcommand{\Dom}{\ensuremath{\mathcal{Z}}\xspace}
\newcommand{\F}{\ensuremath{\mathcal{F}}\xspace}
\newcommand{\preprocessing}{\ensuremath{\rho}\xspace}
\newcommand{\inputpdf}{\ensuremath{z}\xspace}

% Code transformations, search strategy
\newcommand{\transformation}{\ensuremath{\omega}\xspace}
\newcommand{\transformations}{\ensuremath{\Omega}\xspace}
\usepackage{tcolorbox}
\usepackage{pifont}
\definecolor{mydarkgreen}{RGB}{39,130,67}
\definecolor{mydarkred}{RGB}{192,25,25}
\definecolor{bgcolor}{rgb}{0.93,0.99,1}
%\definecolor{bgcolor}{rgb}{0.8,1,1}
\definecolor{bgcolor2}{rgb}{0.8,1,0.8}
\definecolor{bgcolor3}{rgb}{0.50,0.90,0.50}


\newcommand{\green}{\color{mydarkgreen}}
\newcommand{\red}{\color{mydarkred}}
\newcommand{\cmark}{\green\ding{51}}%
\newcommand{\xmark}{\red\ding{55}}%
\newcommand{\cI}{\mathcal{I}}
\newcommand{\cL}{\mathcal{L}}
\newcommand{\cO}{\mathcal O}
\newcommand{\cC}{\mathcal C}


\usepackage{relsize} %SEB: feel free to revert, I think the font looked a bit too big otherwise
\newcommand{\algname}[1]{{\sf\green\relscale{0.90}#1}\xspace}
\newcommand{\algnameS}[1]{{\sf\green\relscale{0.90}#1}\xspace}
\newcommand{\dataname}[1]{{\tt\color{blue}#1}\xspace}
\begin{document}

\title{Symmetric Rank-$k$ Methods}
% \author{
%     Chengchang Liu\thanks{The Chinese Unversity of Hong Kong; 7liuchengchang@gmail.com} \qquad \qquad Cheng Chen\thanks{East China Normal University; jackchen1990@gmail.com} \qquad \qquad Luo Luo\thanks{Fudan University; luoluo@fudan.edu.cn}
%  }
\author{
    Chengchang Liu\thanks{The Chinese Unversity of Hong Kong; 7liuchengchang@gmail.com} \qquad \qquad Cheng Chen\thanks{East China Normal University; jackchen1990@gmail.com} \qquad \qquad Luo Luo\thanks{Fudan University; luoluo@fudan.edu.cn}
 }
 \date{}
 
% \date{Received: date / Accepted: date}


\maketitle

 \begin{abstract}
This paper proposes a novel class of block quasi-Newton methods for convex optimization which we call symmetric rank-$k$ (SR-$k$) methods.
Each iteration of SR-$k$ incorporates the curvature information with~$k$ Hessian-vector products achieved from the greedy or random strategy.
We prove that SR-$k$ methods have the local superlinear convergence rate of $\OM\big((1-k/d)^{t(t-1)/2}\big)$ for minimizing smooth and strongly convex function, where $d$ is the problem dimension and $t$ is the iteration counter.
This is the first explicit superlinear convergence rate for block quasi-Newton methods, 
and it successfully explains why block quasi-Newton methods converge faster than ordinary quasi-Newton methods in practice.
We also leverage the idea of SR-$k$ methods to study the block BFGS and block DFP methods, showing their superior convergence rates.
 \end{abstract}
 

 

\section{Introduction}
We study quasi-Newton methods for solving the minimization problem
\begin{align}\label{prob:main}
    \min_{\x\in\RB^d} f(\x),
\end{align}
where $f:\BR^d\to\BR$ is smooth and strongly convex. 
Quasi-Newton methods~\cite{broyden1970convergence2,broyden1970convergence,shanno1970conditioning,broyden1967quasi,davidon1991variable,byrd1987global,yuan1991modified,asl2024aj} are widely recognized for their fast convergence rates and efficient updates, which attracts growing attention in many fields such as statistics~\cite{jamshidian1997acceleration, zhang2011quasi,bishwal2007parameter},  economics~\cite{ludwig2007gauss,li2013dynamic} and machine learning~\cite{goldfarb2020practical, hennig2013quasi,liu2022quasi,liu2022partial,lee2018distributed,qiu2023quasi}.
Unlike standard Newton methods which need to compute the Hessian and its inverse, quasi-Newton methods go along the descent direction by the following scheme
\begin{align*}
    \x_{t+1}=\x_t-\G_t^{-1}\nabla f(\x_t),
\end{align*}
where $\G_t\in\BR^{d\times d}$ is an estimator of the Hessian  $\nabla^2 f(\x_t)$.
The most popular ways to construct the Hessian estimator are the Broyden family updates, including the the Broyden--Fletcher--Goldfarb--Shanno (BFGS) method~\cite{broyden1970convergence2,broyden1970convergence,shanno1970conditioning}, the Davidon--Fletcher--Powell (DFP) method~\cite{davidon1991variable,fletcher1963rapidly}, and the symmetric rank-one (SR1) method~\cite{broyden1967quasi,davidon1991variable}. 

The classical quasi-Newton methods with Broyden family updates \cite{broyden1970convergence2,broyden1970convergence} find the Hessian estimator $\G_{t+1}$ for the next round by the secant equation 
\begin{align}\label{eq:sec}
\G_{t+1}(\x_{t+1}-\x_t)=\nabla f(\x_{t+1})-\nabla f(\x_t).    
\end{align}
These methods have been proven to exhibit local superlinear convergence in 1970s~\cite{powell1971on,Dennis1974A,broyden1973on,dai2002convergence},
and their non-asymptotic superlinear rates were established in recent years~\cite{rodomanov2021rates,rodomanov2021new,ye2022towards,jin2022non,jin2022sharpened}.
For example, Rodomanov and Nesterov~\cite{rodomanov2021rates} showed classical BFGS method enjoys the local superlinear rates of $\OM\big((d\varkappa/t)^{t/2}\big)$, where $\varkappa>1$ is the condition number. They also improved this result to $\OM\big((\exp{d\ln(\varkappa)/t}-1)^{t/2}\big)$ in consequent work \cite{rodomanov2021new}. 
Later, Ye et al.~\cite{ye2022towards} showed classical SR1 method converges with a local superlinear rate of $\OM\big((d\ln(\varkappa)/t)^{t/2}\big)$.


Some recent work \cite{gower2017randomized,rodomanov2021greedy,lin2021greedy,ji2023greedy} proposed another type of quasi-Newton methods, which construct the Hessian estimator by the following equation
\begin{equation}
\label{eq:grracondi}
    \G_{t+1}\u_t=\nabla^2 f(\x_{t+1})\u_t,
\end{equation}
where $\u_t\in\RB^{d}$ is chosen by greedy or randomized strategy.
% \begin{align*}
%     \color{red} \text{write down the expressions}.
% \end{align*}
Rodomanov and Nesterov~\cite{rodomanov2021greedy} established a local superlinear rate of $\OM\big((1-{1}/{(\varkappa d)})^{t(t-1)/2}\big)$ for greedy quasi-Newton methods with Broyden family updates.
Later, Lin et al.~\cite{lin2021greedy} provided a condition-number free superlinear rate of~$\OM\big((1-{1}/{d})^{t(t-1)/2}\big)$ for the greedy and randomized quasi-Newton methods with the specific BFGS and SR1 updates.

Block quasi-Newton methods construct the Hessian estimator along multiple directions at each iteration. 
The study of these methods dates back to 1980s. Schnabel \cite{schnabel1983quasi} proposed the first block BFGS method by extending equation (\ref{eq:sec}) to multiple secant equations, i.e.,
$\G_{t+1}(\x_{t+1}-\x_{t+1-j})=\nabla f(\x_{t+1})-\nabla f(\x_{t+1-j})$
% \begin{align*}
%     \G_{t+1}(\x_{t+1}-\x_{t+1-j})=\nabla f(\x_{t+1})-\nabla f(\x_{t+1-j})
% \end{align*}
for all~$j=1,\dots,k$.
However, the resulting update of this method cannot guarantee that the Hessian estimator is symmetric, and it requires additionally modification to ensure the positive definiteness~\cite{schnabel1983quasi,o1994linear,gao2018block,lee2023almost}. 
Another line of research in 2010s~\cite{gao2018block,gower2016stochastic,gower2017randomized,kovalev2020fast,boutet2020secant}  consider the variants of block BFGS methods that generalizes condition~\eqref{eq:grracondi} to
\begin{align}
\label{eq:blockcondi}
    \G_{t+1}\U_t=\nabla^2 f(\x_{t+1})\U_t,
\end{align}
where $\U_t=[\u_t^{(1)},\cdots,\u_t^{(k)}]\in\RB^{d\times k}$ indicates multiple directions constructed by deterministic or randomized ways. 
Since the Hessian-matrix product $\nabla^2 f(\x_{t+1})\U_t$ can be efficiently achieved  by accessing the Hessian-vector products $\{\nabla^2 f(\x_{t+1})\u_t^{(j)}\}_{j=1}^k$ in parallel~\cite{gao2018block}, these block BFGS methods usually exhibit better empirical performance than the ordinary quasi-Newton methods without block fashion updates.
Gao and Goldfarb~\cite{gao2018block}, Kovalev et al.~\cite{kovalev2020fast} provided asymptotic local superlinear rates for block BFGS methods based on the condition~\eqref{eq:blockcondi}, 
but the advantage of block quasi-Newton methods over ordinary quasi-Newton methods is still unclear in theory. 
%The reason why block quasi-Newton methods enjoy faster convergence behavior than ordinary quasi-Newton methods in practice remains open.
% $\nabla^2 f(\x_{t+1})\U_t = \big[\nabla^2 f(\x_{t+1}) \u_{t}^{(1)},\cdots, \nabla^2 f(\x_{t+1})\u_{t}^{(k)}\big]$
% }
% Although block quasi-Newton methods usually have better empirical performance than ordinary quasi-Newton methods without block fashion updates~\cite{schnabel1983quasi,o1994linear, gao2018block,gower2016stochastic,gower2017randomized,kovalev2020fast}, 
% their theoretical guarantees are mystery until Gao and Goldfarb~\cite{gao2018block}  proved a modified version of block BFGS method has asymptotic local superlinear convergence.
% On the other hand, \textcolor{blue}{\cite{gower2016stochastic,gower2017randomized,kovalev2020fast}} introduced the randomized block BFGS by generalizing condition~\eqref{eq:grracondi} to
% \begin{align*}
%     \G_{t+1}\U_t=\nabla^2 f(\x_{t+1})\U_t,
% \end{align*}
% where $\U_t\in\RB^{d\times k}$ is some randomized matrix. 
% Their empirical studies showed randomized block BFGS performs well on real-world applications.
% In addition, Kovalev et al.~\cite{kovalev2020fast} proved randomized block BFGS method also has asymptotic local superlinear convergence, but its advantage over ordinary BFGS methods is still unclear in theory. 
% The existing theoretical analysis for block quasi-Newton methods lacks the non-asymptotic convergence results, which naturally leads to the following question:
% \begin{center}
% \color{blue}
% \textit{Can we explain why block quasi-Newton method enjoy faster convergence behavior in practice?}
% \end{center} 
% The theoretical results of existing block quasi-Newton methods cannot explain why they enjoy faster convergence behavior than ordinary quasi-Newton methods in practice. 
This naturally leads to the following question:
\textit{Can we provide a block quasi-Newton method with explicit superior convergence rate over the ordinary ones?}


In this paper, we give an affirmative answer by proposing symmetric rank-$k$ \mbox{(SR-$k$)} methods.
%for $k\in[d-1]$ and the quadratic rate of Newton methods for $k=d$. 
We design a novel block update to construct the Hessian estimator $\mG_{t+1}\in\BR^{d\times d}$ that satisfies equation~\eqref{eq:blockcondi}.
% {\color{red}
% We provide a novel block fashion update to constructions of Hessian estimators in \srk~methods are based on an extension of symmetric rank-1 methods (SR1)~\cite{broyden1967quasi,davidon1991variable,ye2022towards} and the equation of the form~\eqref{eq:blockcondi}.} 
We also provide the randomized and greedy strategies to determine directions in $\mU_t\in\BR^{d\times k}$ for SR-$k$ methods, which leads to the explicit local superlinear convergence rate of~$\OM\big((1-{k}/{d})^{t(t-1)/2}\big)$, where $k$~is the number of directions used to approximate Hessian at per iteration.
For the case of~$k=1$, our methods reduce to randomized and greedy SR1 methods~\cite{lin2021greedy}.
For the case of~$k\geq 2$, it is clear that SR-$k$ methods have faster superlinear rates than existing greedy and randomized quasi-Newton methods~\cite{lin2021greedy,rodomanov2021greedy}.
For the case of $k=d$, it recovers the quadratic convergence rate like standard Newton's method.
We also follow the design of SR-$k$ to propose the variants of randomized block BFGS methods~\cite{gower2016stochastic,gower2017randomized,kovalev2020fast} and a randomized block DFP method, resulting the faster explicit superlinear rates.
We compare the proposed methods with existing quasi-Newton methods in Table~\ref{tbl:main-minmax}. 

% We shall also leverage the idea of SR-$k$ to solve the smooth saddle point problem (SPP)
% \begin{align}
% \label{eq:minmax}
%     \min_{\x\in\RB^{d_\x}}\max_{\y\in\RB^{d_\y}}f(\x,\y),
% \end{align}
% where $f(\cdot,\cdot)$ is strongly-convex in $\x$ and strongly-concave in $\y$. 
% We apply the iteration
% \begin{align*}
% [\x_{t+1}^{\top},\y_{t+1}^{\top}]^{\top}=    [\x_{t}^{\top},\y_{t}^{\top}]^{\top} -\G_{t}^{-1}\nabla^2 f(\x_{t},\y_t)\nabla f(\x_t,\y_t) 
% \end{align*}
% to update the variables, where $\G_t$ is constructed by using the \srk~update to estimate the square of the Hessian $\left[\nabla^2 f(\x_t,\y_{t})\right]^2$. 
% This method generalizes the recent proposed quasi-Newton methods~\cite{liu2022quasi} for SPP and achieves superior convergence rates.



% We introduce the iteration
% \begin{align*}
%     \vz_{t+1} = \vz_t - \mH_t^{-1}\mJ(\vz_t)^{\top}\mF(\vz_t),
% \end{align*}
% where $\mJ(\cdot)\in\BR^{d\times d}$ is the Jacobians of $\mF(\cdot)$ and $\mH_t$ is established by SR-$k$ update to estimate $\mJ(\vz_t)^\top\mJ(\vz_t)$. 
% This method generalizes the recent proposed quasi-Newton methods~\cite{liu2022quasinewton} for solving nonlinear equation and achieves superior convergence rates under regularity condition.

% \paragraph{Paper Organization}
% The remainder of this paper is organized as follows.
% In Section~\ref{sec:pre}, we introduce notations and assumptions throughout this paper. 
% In Section~\ref{sec:update}, we propose the SR-$k$ update in the view of matrix approximation. 
% In Section~\ref{sec:algorithm}, we propose the quasi-Newton methods with SR-$k$ updates for minimizing smooth strongly convex function and provide their the superior local superlinear convergence rates. 
% In Section~\ref{sec:Block BFGS}, we propose the new randomized block BFGS methods and randomized block DFP method with explicit local superlinear convergence rates.
% % In Section~\ref{sec:minmax}, we extend the idea of \srk~method to solve the saddle point problems.
% In Section~\ref{sec:exp}, we conduct numerical experiments to show the outperformance of proposed methods.
% Finally, we conclude our work in Section~\ref{sec:conclusion}.

\begin{table*}[!t]
	\centering
	\caption{\small We summarize local convergence rates of existing and proposed quasi-Newton methods for strongly convex optimization, where $\varkappa$ denotes the condition number of the objective and $t$ is the iteration counter. 
    The second column displays the number of secant equations used to construct the Hessian estimator.
    \label{tbl:main-minmax}} 
%
\begin{threeparttable}
\setlength\tabcolsep{5.pt}
\begin{tabular}{c c c c}
\toprule[.1em]
\begin{tabular}{c} 
    \bf Methods  
\end{tabular} &
\bf  \#Equations & 
\begin{tabular}{c}  \bf Convergence \end{tabular} & \bf Reference \\
\toprule[.1em]

% \begin{tabular}{c} 
%   Newton\\  
% \end{tabular} 
% & $d$ 
% & quadratic & \cite{nesterov2018lectures,nocedal1999numerical}
% \\  
% \midrule 

% \begin{tabular}{c}  DFP \\      
% \cite{rodomanov2021rates,rodomanov2021new} 
% \end{tabular}
% & $1$ or $2$ 
% & $\displaystyle{\OM\left(\big(\frac{d\varkappa^2}{t}\big)\right)}$  
% \\  
% \midrule 


\begin{tabular}{c}  Classical BFGS     
\end{tabular}
& $1$ 
&\begin{tabular}{c}
$\displaystyle{\OM\big(\big({d\varkappa}/{t}\big)^{t/2}\big)}$\\ [0.1cm]
$\displaystyle{\OM\big(\big(\exp{{d\ln(\varkappa)}/{t}}-1\big)^{t/2}\big)}$  
% \\  [0.1cm]
% $\displaystyle{\OM\big(\big (1/{t}\big)^{t/2}\big)}$~\tnote{(*)}  
\end{tabular}   
& \begin{tabular}{c}
  \cite{rodomanov2021rates}  \\[0.2cm] 
  \cite{rodomanov2021new} 
  %\\[0.2cm]
 % \cite{jin2022non}
\end{tabular} 
\\  
\midrule 

\begin{tabular}{c}  Classical BFGS/DFP   
\end{tabular}
& $1$ 
&\begin{tabular}{c}
% $\displaystyle{\OM\big(\big({d\kappa}/{t}\big)^{t/2}\big)}$\\ [0.1cm]
% $\displaystyle{\OM\big(\big(\exp{{d\ln(\varkappa)}/{t}}-1\big)^{t/2}\big)}$  
$\displaystyle{\OM\big(\big (1/{t}\big)^{t/2}\big)}$~\tnote{(*)}  
\end{tabular}   
& \begin{tabular}{c}
 \cite{jin2022non}
\end{tabular} 
\\  
\midrule 
\begin{tabular}{c}  Classical DFP   
\end{tabular}
& $1$ 
&\begin{tabular}{c}
$\displaystyle{\OM\big(\big({d\varkappa^2}/{t}\big)^{t/2}\big)}$
\end{tabular}   
& \begin{tabular}{c}
\cite{rodomanov2021rates}
\end{tabular} 
\\  
\midrule 


% \begin{tabular}{c}  BFGS/DFP \\      
% \cite{jin2022non} 
% \end{tabular}
% & $2$ 
% & $\displaystyle{\OM\big(\big({1}/{t}\big)^t\big)}$~\tnote{(*)}  
% \\  
% \midrule 


% \begin{tabular}{c} Classical SR1 \\      
% \end{tabular}
% & $1$ 
% &  $\displaystyle{\OM\big(\big({d\ln(\varkappa)}/{t}\big)^{t/2}\big)}$  
% &  \cite{ye2022towards}
% \\  
% \midrule           
% \begin{tabular}{c}
%     Greedy/Randomized Broyden
% \end{tabular} 
% & ~~$1$~~ 
% & ~~$\displaystyle{\OM\big((1-1/(\varkappa d))^{{t(t-1)}/{2}}\big)}$~~  
% &\cite{rodomanov2021greedy,lin2021greedy}
% \\  
% \midrule 

\begin{tabular}{c} Classical SR1 \\      
\end{tabular}
& $1$ 
&  $\displaystyle{\OM\big(\big({d\ln(\varkappa)}/{t}\big)^{t/2}\big)}$  
&  \cite{ye2022towards}
\\  
\midrule           
\begin{tabular}{c}
    Greedy/Randomized Broyden
\end{tabular} 
& ~~$1$~~ 
& ~~$\displaystyle{\OM\big((1-1/(\varkappa d))^{{t(t-1)}/{2}}\big)}$~~  
&\cite{rodomanov2021greedy,lin2021greedy}
\\  
\midrule 


\begin{tabular}{c}
    Randomized BFGS
\end{tabular}
& $1$ 
& $\displaystyle{\OM\big((1-1/d)^{{t(t-1)}/{2}}\big)}$     
& \cite{lin2021greedy}
\\
\midrule 

\begin{tabular}{c}
    Greedy/Randomized SR1 
\end{tabular}
& $1$
& $\displaystyle{\OM\big((1-1/d)^{{t(t-1)}/{2}}\big)}$  
&\cite{lin2021greedy}
\\
\midrule 


\begin{tabular}{c}
Block-BFGS 
\end{tabular} 
& $k\in[d]$ 
& asymptotic superlinear  
&  \cite{gao2018block}
\\  
\midrule 
\begin{tabular}{c}
Randomized Block-BFGS (v1) 
\end{tabular} 
& $k\in[d]$ 
& asymptotic superlinear 
&   \cite{kovalev2020fast,gower2017randomized} 
\\  
\midrule
\cellcolor{bgcolor}	
Greedy/Randomized SR-$k$ 
\cellcolor{bgcolor}	
& 
\begin{tabular}{c}
    ~$k\in[d-1]$   \\[0.1cm]
    $k=d$  
\end{tabular}
\cellcolor{bgcolor}	
& \cellcolor{bgcolor}	
\begin{tabular}{c}
$\OM\big((1-{k}/{d})^{{t(t-1)}/{2}}\big)$ \\[0.1cm]
quadratic
\end{tabular}
% $\displaystyle{\begin{cases}\OM\left((1-{k}/{d})^{{t(t-1)}/{2}}\right), & k\in[d-1] \\[0.15cm]
% \OM\left(\lambda_0^{2^t}\right), & k=d 
% \end{cases}}$  
&  \cellcolor{bgcolor} Alg.~\ref{alg:SRK}
\\[0.25cm]
\midrule
\cellcolor{bgcolor}	
\begin{tabular}{c}
Randomized Block-BFGS/DFP  
\end{tabular}
&\cellcolor{bgcolor}	 \begin{tabular}{c}
    ~$k\in[d-1]$   \\[0.1cm]
    $k=d$  
\end{tabular}
& \cellcolor{bgcolor}	\begin{tabular}{c}
$\OM\big((1-{k}/{(d\varkappa)})^{{t(t-1)}/{2}}\big)$ \\[0.1cm]
quadratic
\end{tabular}
&  \cellcolor{bgcolor}	   Alg.~\ref{alg:bfgs} 
% \\  
% \midrule
% \cellcolor{bgcolor}	
% \begin{tabular}{c}
%     Randomized Block-DFP
% \end{tabular}
% &\cellcolor{bgcolor}	 \!$k\in[d]$  
% & \cellcolor{bgcolor}	$\displaystyle{\OM\big((1-{k}/{(\varkappa d)})^{{t(t-1)}/{2}}\big)}$  
% & \cellcolor{bgcolor}	 Alg.~\ref{alg:bfgs} 
\\
\midrule
\cellcolor{bgcolor}	
Faster Randomized Block-BFGS  
\cellcolor{bgcolor}	
& 
\begin{tabular}{c}
    ~$k\in[d-1]$   \\[0.1cm]
    $k=d$  
\end{tabular}
\cellcolor{bgcolor}	
& \cellcolor{bgcolor}	
\begin{tabular}{c}
$\OM\big((1-{k}/{d})^{{t(t-1)}/{2}}\big)$ \\[0.1cm]
quadratic
\end{tabular}
% $\displaystyle{\begin{cases}\OM\left((1-{k}/{d})^{{t(t-1)}/{2}}\right), & k\in[d-1] \\[0.15cm]
% \OM\left(\lambda_0^{2^t}\right), & k=d 
% \end{cases}}$  
&  \cellcolor{bgcolor} Alg.~\ref{alg:fasterbfgs}
\\[0.25cm]
\bottomrule[.1em]
\end{tabular} % default value: 6pt
\begin{tablenotes}
    	{\scriptsize     
  			\item [{(*)}] This convergence rate requires an additional assumption that the initial Hessian estimator be sufficiently close to the exact Hessian~\cite[Corollary 3]{jin2022non}. \\
%     \item [{($\dag$)}] This method requires QR decomposition of rank-1 change matrix, which is not practical when $d$ is large.
     }
\end{tablenotes}
\end{threeparttable}\vskip-0.55cm	
\end{table*}  



\section{Preliminaries}
\label{sec:pre}
We first introduce the notations used in this paper. We use~$\{\e_1,\cdots,\e_d\}$ to present the standard basis in space $\RB^d$ and let $\I_d\in\RB^{d\times d}$ be the identity matrix. 
We use $\S^{\dag}$ to denote the Moore-Penrose inverse of a matrix $\S$.
We denote the trace of a square matrix by ${\rm tr}(\cdot)$. 
We use $\|\cdot\|$ to present the spectral norm of a matrix and the Euclidean norm of a vector. Given a positive definite matrix $\A\in\BR^{d\times d}$, we denote the corresponding weighted norm as~$\|\vx\|_\mA\triangleq(\vx^{\top}\mA\vx)^{1/2}$ for some $\vx\in\BR^d$. 
We use the notation~$\Norm{\vx}_\vz$ to present~$\|\vx\|_{\nabla^2 f(\vz)}$ for positive definite Hessian $\nabla^2 f(\vz)$, if there is no ambiguity for the reference function~$f(\cdot)$.
We also define
$\mE_{k}(\mA)\triangleq[\ve_{i_1};\cdots;\ve_{i_k}]\in\BR^{d\times k}$,
% \begin{align}
%     \label{eq:EA}
%     \mE_{k}(\mA)\triangleq[\ve_{i_1};\cdots;\ve_{i_k}]\in\BR^{d\times k},
% \end{align}
where $i_1,\dots,i_k$ are the indices for the largest~$k$ diagonal entries of matrix $\mA\in\BR^{d\times d}$.

% When 
% The expectation $\EB_{\U}[\,\cdot\,]$ or $\EB[\,\cdot\,]$ consider the randomness of the random matrix $\U$ for one iteration or all the randomness during the updates, we can view it with no randomness for the same notation when applied to deterministic updates.

Throughout this paper, we suppose the objective in problem (\ref{prob:main}) satisfies the following assumptions.
\begin{assumption}
\label{ass:smooth}
We assume the objective $f:\BR^d\to\BR$ is $L$-smooth, i.e., there exists some constant~$L\geq 0$ such that $\|\nabla f(\vx)-\nabla f(\vy)\|\leq L\|\vx-\vy\|$ for any $\vx,\vy\in\BR^d.$
%$\nabla^2 f(\vx) \preceq L \I_d$.
\end{assumption}

\begin{assumption}
\label{ass:strongconvex}
We assume the objective $f:\BR^d\to\BR$ is $\mu$-strongly convex, i.e., there exists some constant~$\mu>0$ such that 
\begin{align*}
f(\lambda\vx+(1-\lambda)\vy)\leq \lambda f(\vx) + (1-\lambda)f(\vy)-\frac{\lambda(1-\lambda)\mu}{2}\|\vx-\vy\|^2.  
\end{align*}
%$  $ 
for any $\vx,\vy\in\BR^d$ and $\lambda\in[0,1]$.
% , i.e., there exists some constant~$\mu>0$ such that 
% \begin{align*}
% f(\lambda\vx+(1-\lambda)\vx')\leq \lambda f(\vx) + (1-\lambda)f(\vx')-\frac{\lambda(1-\lambda)\mu}{2}\|\vx-\vx'\|^2    
% \end{align*}
% for any $\vx,\vy\in\BR^d$ and $\lambda\in[0,1]$.
\end{assumption}

We define the condition number as $\varkappa \triangleq L/\mu$. 
The following proposition shows the twice differentiable  function has bounded Hessian under Assumptions \ref{ass:smooth} and~\ref{ass:strongconvex}~\cite{nesterov2018lectures}.
\begin{proposition}
Suppose the objective $f:\BR^d\to\BR$ is twice differentiable and satisfies Assumptions~\ref{ass:smooth} and \ref{ass:strongconvex}, then it holds $\mu\mI_d \preceq \nabla^2 f(\x) \preceq L\mI_d$
% \begin{align}
% \mu\mI_d \preceq \nabla^2 f(\x) \preceq L\mI_d
% \end{align}
for any $\x\in\BR^d$.
\end{proposition}


We also impose the assumption of strongly self-concordance~\cite{rodomanov2021greedy,lin2021greedy} as follows.
\begin{assumption}
\label{ass:strongself}
We assume the objective $f:\RB^d\to\RB$ is $M$-strongly self-concordant, i.e., there exists some constant $M\geq 0$ such that 
\begin{align*}
   \nabla^2 f(\y)-\nabla^2 f(\x)\preceq M\|\y-\x\|_\z\nabla^2 f(\w) 
\end{align*}
holds for all $\x,\y,\w,\z\in\RB^d$.
% there exists some constant $M\geq 0$ such that
% \begin{align}
%     \nabla^2 f(\x')-\nabla^2 f(\x)\preceq M\|\x'-\x\|_\z\nabla^2 f(\w)
% \end{align}
% for any $\x,\x',\w,\z\in\RB^d$.
\end{assumption}

Noticing that the strongly convex function with Lipschitz continuous Hessian is strongly self-concordant~\cite{rodomanov2021greedy}.

\begin{proposition}
Suppose the objective $f:\BR^d\to\BR$ satisfies Assumption \ref{ass:strongconvex} and has $L_2$-Lipschitz continuous Hessian, i.e., there exists some constant $L_2\geq 0$ such that~$\|\nabla^2 f(\x)-\nabla^2 f(\y)\|\leq L_2\|\x-\y\|$ holds
% \begin{align*}
% \| \nabla^2 f(\x)-\nabla^2 f(\x')\|\leq L_2\|\x-\x'\|,    
% \end{align*}
for all $\x$, $\y\in\RB^d$, then the function~$f(\cdot)$ is $M$-strongly self-concordant with $M={L_2}/{\mu^{3/2}}$.
\end{proposition}

\section{Symmetric Rank-$k$ Updates}
\label{sec:update}

We propose the symmetric rank-$k$ (SR-$k$) update for matrix approximation as follows.

\begin{definition}[SR-$k$ Update]
Let $\mA\in\BR^{d\times d}$ and $\mG\in\BR^{d\times d}$ be two positive-definite matrices with~$\A\preceq \G$. 
For any matrix $\U\in\RB^{d\times k}$, we define
\begin{align}
\label{eq:srk}    
    {\text{\rm SR-$k$}}(\G,\A,\U) \triangleq \G-(\G-\A)\U\big(\U^{\top}(\G-\A)\U\big)^{\dag}\U^{\top}(\G-\A).
\end{align}
\end{definition}

Noticing that the formula of \srk~update contains a term of Moore-Penrose inverse, since the matrix $\U^{\top}(\G-\A)\U$ is possibly singular even if matrices $\mA,\mG\in\BR^{d\times d}$, and~$\mU\in\BR^{d\times k}$ are full rank.
This leads to its design and analysis be quite different from other block-type updates (i.e., block BFGS/DFP updates) we study in Section~\ref{sec:Block BFGS}.

% \begin{remark}
% Our \srk~update \eqref{eq:srk} reduces to SR1 update 
% \cite{broyden1967quasi,davidon1991variable,ye2022towards} by taking $k=1$. However, the formular 
% \end{remark}

In this section, 
% we denote  $\R\triangleq \G-\A\succeq\0$, 
we always use $\G_{+}$ to denote the output of \srk~update such that 
\begin{align}\label{iter:srk}
\G_{+}\triangleq\srk(\G,\A,\U).     
\end{align}

We provide the following lemma to show that \srk~update does not increase the deviation from target matrix $\A$. %which is similar to ordinary Broyden family updates~\cite{rodomanov2021greedy}.
\begin{lemma}
\label{lm:sr1good}
Given any positive-definite matrices $\mA\in\BR^{d\times d}$ and $\mG\in\BR^{d\times d}$ with~$\A\preceq\G\preceq\eta \A$ for some~$\eta\geq 1$, it holds that
\begin{align}\label{eq:srk_good}
   \A\preceq\G_{+}\preceq \eta\A. 
\end{align}
\end{lemma}
\begin{proof}
According to the update rule (\ref{eq:srk}), we have
\begin{align*}
    &\G_{+}-\A= (\G-\A)-(\G-\A)\U(\U^{\top}(\G-\A)\U)^{\dag}\U^{\top}(\G-\A)\\
    &=\left(\I_d\!-\!(\G\!-\!\A)\U(\U^{\top}(\G\!-\!\A)\U)^{\dag}\U^{\top}\right)(\G-\A)\left(\I_d\!-\!\U(\U^{\top}(\G\!-\!\A)\U)^{\dag}\U^{\top}(\G\!-\!\A)\right)\\
    &\succeq\0, 
\end{align*}
which indicates $\G_{+}\succeq \A$.
% \begin{align*}
%     \G_{+}\succeq\A.
% \end{align*}
On the other hand, the condition $\mG\preceq\eta\mA$ indicates
\begin{align*}
    \G_{+}&\preceq \eta\A - (\G-\A)\U(\U^{\top}(\G-\A)\U)^{\dag}\U^{\top}(\G-\A)\preceq \eta\A,
\end{align*}
where the last inequality is by the condition $\G\succeq \A$.
Hence, we finish the proof.
\end{proof}


To evaluate the convergence of SR-$k$ update, we introduce the quantity~\cite{lin2021greedy,ye2022towards}
 \begin{align}
 \label{eq:measure_srk}
     \tau_\A(\G)\triangleq\trcommon{\G-\A},
 \end{align}
to describe the difference between the target matrix $\mA$ and the current estimator $\mG$.
% The randomized or greedy SR1 update~\cite{lin2021greedy} omits the rate of 
% \begin{align*}
%     \EBP{\tau_{\A}(\G_{+})}\leq \left(1-\frac{1}{d}\right)\tau_{\A}(\G),
% \end{align*}
% when take $k=1$ in \eqref{eq:srk} and choose a proper $\U\in\RB^{d\times 1}$.

In the remainder of this section, we aim to establish the convergence guarantee  
\begin{align}
\label{eq:srk-aim}
    \EBcommon{\tau_{\A}(\G_{+})}\leq \left(1-\frac{k}{d}\right)\tau_{\A}(\G)
\end{align}
for approximating the target matrix $\mA\in\BR^{d\times d}$ by \srk~iteration (\ref{iter:srk}) with the appropriate choice of (random) matrix $\mU\in\BR^{d\times k}$.
Note that if we take $k=1$, the equation (\ref{eq:srk-aim}) will reduce to $ \EBcommon{\tau_{\A}(\G_{+})}\leq \left(1-{1}/{d}\right)\tau_{\A}(\G)$, which corresponds to the convergence result of randomized and greedy SR1 updates~\cite{lin2021greedy}.
%which exhibit the advantage of using block updates in terms of approximating the object matrix.
% {\color{red}The following theorem} show that \srk~updates with randomized or greedy strategies enjoy explicit faster convergence rate than SR1~updates for estimating $\mA$ in terms of the measure $\tau_\A(\cdot)$.



Observe that we can split $\tau_{\A}(\G_{+})$ as follows
\begin{align}
\label{eq:tauAG+}
\begin{split}
    \tau_{\A}(\G_{+}) &\!\overset{\eqref{eq:srk},\, \eqref{iter:srk}}{=}\tr{\G-\A - (\G-\A)\U(\U^{\top}(\G-\A)\U)^{\dag}\U^{\top}(\G-\A)} \\
    &\,\,\,\,\,\,\,\,=\,\,\,\,\,\,\,\,\underbrace{\trcommon{\G-\A}}_{\text{Part}~{\rm \uppercase\expandafter{\romannumeral1}}} - \underbrace{\tr{(\G-\A)\U(\U^{\top}(\G-\A)\U)^{\dag}\U^{\top}(\G-\A)}}_{\text{Part}~{\rm \uppercase\expandafter{\romannumeral2}}}.
    \end{split}
\end{align}
% where Part I is equal to $\tau_{\A}(\G)$ and Part II encourages the \srk~update to decrease the value of $\tau_\A(\cdot)$.
% To obtain the result of \eqref{eq:srk-aim}, we only need to prove Part II satisfies
Since Part I is equal to $\tau_{\A}(\G)$ and Part II is nonnegative, the \srk~update can reduce the estimation error with regard to $\tau_\A(\cdot)$.
To obtain \eqref{eq:srk-aim}, we only need to prove 
\begin{align*}
\EBP{\tr{(\G-\A)\U(\U^{\top}(\G-\A)\U)^{\dag}\U^{\top}(\G-\A)}}\geq \frac{k}{d}\trcommon{\G-\A}.
\end{align*}
We first provide the following lemma to bound Part II (in the view of $\mR=\mG-\mA$). 
% the Part~{\rm \uppercase\expandafter{\romannumeral2} is lower bounded. This guarantees a sufficient decrease of $\tau_\A(\cdot)$.
\begin{lemma}
\label{lm:pdneq2}
For symmetric positive semi-definite matrix $\R\in\RB^{d\times d}$ and full rank matrix $\U\in\RB^{d\times k}$ with $k\leq d$, it holds that
\begin{align}
\label{eq:keyeq}  \tr{\R\U\left(\U^{\top}\R\U\right)^{\dag}\U^{\top}\R} \geq \tr{\U\left(\U^{\top}\U\right)^{-1}\U^{\top}\R}.
\end{align}
\end{lemma}
\begin{proof}
Let $\U=\Q\mSigma\V^{\top}$ be the reduced SVD of $\mU\in\BR^{d\times k}$, where $\Q\in\RB^{d\times k}$, $\V\in\RB^{k\times k}$ are (column) orthonormal (i.e. $\Q^{\top}\Q=\I_k$ and $\V^{\top}\V=\I_k$) and~$\mSigma\in\RB^{k\times k}$ is diagonal, then the right-hand side of inequality (\ref{eq:keyeq}) can be written as
\begin{align*}
\tr{\U\left(\U^{\top}\U\right)^{-1}\U^{\top}\R}& = \tr{\Q\mSigma\V^{\top}\left(\V\mSigma^2\V^{\top}\right)^{-1}\V\mSigma\Q^{\top}\R}  =\tr{\Q\Q^{\top}\R}.
\end{align*}
Consequently, we upper bound the left-hand side of inequality (\ref{eq:keyeq}) as follows 
% \begin{align*}
%     &\tr{\R\U\left(\U^{\top}\R\U\right)^{\dag}\U^{\top}\R} = \tr{\R\Q\mSigma\V^{\top}\left(\V\mSigma\Q^{\top}\R\Q\mSigma\V^{\top}\right)^{\dag}\V\mSigma\Q^{\top}\R}\\
%     &\color{red}{=\tr{\R\Q\mSigma\V^{\top}(\V^{\top})^{\dag}\left(\mSigma\Q^{\top}\R\Q\mSigma\right)^{\dag}\V^{\dag}\V\mSigma\Q^{\top}\R}
%     }\\    &=\tr{\R\Q\mSigma\left(\mSigma\Q^{\top}\R\Q\mSigma\right)^{\dag}\mSigma\Q^{\top}\R}\\    &=\tr{\Q^{\top}\R\Q\mSigma\left(\mSigma\Q^{\top}\R\Q\mSigma\right)^{\dag}\mSigma\Q^{\top}\R\Q}\\
%     &~~~+\tr{(\I_d-\Q\Q^{\top})^{1/2}\R\Q\mSigma\left(\mSigma\Q^{\top}\R\Q\mSigma\right)^{\dag}\mSigma\Q^{\top}\R(\I_d-\Q\Q^{\top})^{1/2}}
%     \\
%     &\geq \tr{\Q^{\top}\R\Q\mSigma\left(\mSigma\Q^{\top}\R\Q\mSigma\right)^{\dag}\mSigma\Q^{\top}\R\Q}\\
% &=\tr{\mSigma^{-1}\mSigma\Q^{\top}\R\Q\mSigma\left(\mSigma\Q^{\top}\R\Q\mSigma\right)^{\dag}\mSigma\Q^{\top}\R\Q\mSigma\mSigma^{-1}}\\
% &=   \tr{\mSigma^{-1}\mSigma\Q^{\top}\R\Q\mSigma\mSigma^{-1}} 
% = \tr{\Q^{\top}\R\Q},
%     % &=\tr{\R\Q\left(\Q^{\top}\R\Q\right)^{\dag}\Q^{\top}\R}\\
%     % &=\tr{\Q^{\top}\R\Q\left(\Q^{\top}\R\Q\right)^{\dag}\Q^{\top}\R\Q}+\tr{\R\Q\left(\Q^{\top}\R\Q\right)^{\dag}\Q^{\top}\R(\I-\Q\Q^{\top})}\\
%     % &=\tr{\Q^{\top}\R\Q\left(\Q^{\top}\R\Q\right)^{\dag}\Q^{\top}\R\Q}+\tr{(\I_d-\Q\Q^{\top})^{1/2}\R\Q\left(\Q^{\top}\R\Q\right)^{\dag}\Q^{\top}\R(\I_d-\Q\Q^{\top})^{1/2}}
%     % \\
%     % &\geq\tr {\Q^{\top}\R\Q},
% \end{align*}
\begin{align*}
    &\tr{\R\U\left(\U^{\top}\R\U\right)^{\dag}\U^{\top}\R} = \tr{\R\Q\mSigma\V^{\top}\left(\V\mSigma\Q^{\top}\R\Q\mSigma\V^{\top}\right)^{\dag}\V\mSigma\Q^{\top}\R}\\    &=\tr{\Q^{\top}\R\Q\mSigma\V^{\top}\left(\V\mSigma\Q^{\top}\R\Q\mSigma\V^{\top}\right)^{\dag}\V\mSigma\Q^{\top}\R\Q}\\
    &~~~+\tr{\big(\I_d-\Q\Q^{\top}\big)^{1/2}\R\Q\mSigma\V^{\top}\left(\V\mSigma\Q^{\top}\R\Q\mSigma\V^{\top}\right)^{\dag}\V\mSigma\Q^{\top}\R\big(\I_d-\Q\Q^{\top}\big)^{1/2}}
    \\
    &\geq \tr{\Q^{\top}\R\Q\mSigma\V^{\top}\left(\V\mSigma\Q^{\top}\R\Q\mSigma\V^{\top}\right)^{\dag}\V\mSigma\Q^{\top}\R\Q}\\
&=\tr{(\V\mSigma)^{-1}\V\mSigma\Q^{\top}\R\Q\mSigma\V^{\top}\left(\V\mSigma\Q^{\top}\R\Q\mSigma\V^{\top}\right)^{\dag}\V\mSigma\Q^{\top}\R\Q\mSigma\V^\top\big(\mSigma\V^\top\big)^{-1}}\\
&=  \tr{(\V\mSigma)^{-1}\V\mSigma\Q^{\top}\R\Q\mSigma\V^\top\big(\mSigma\V^\top\big)^{-1}} 
= \tr{\Q^{\top}\R\Q},
\end{align*}
where %\textcolor{blue}{the second equality is due to $\V$ is orthonormal.}  
the inequality is due to  % $\I_d-\Q\Q^{\top} = \I_d-\Q(\Q^{\top}\Q)^{-1}\Q^{\top} \succeq \0$ exists 
$\Q$ is column orthonormal (thus $\I_d-\Q\Q^{\top}\succeq \0$) and  $\R\Q\mSigma\V^\top\left(\V\mSigma\Q^{\top}\R\Q\mSigma\V^\top\right)^{\dag}\V\mSigma\Q^{\top}\R$ is positive semi-definite.
% \begin{align*}
%     \I_d-\Q\Q^{\top} = \I_d-\Q(\Q^{\top}\Q)^{-1}\Q^{\top} \succeq \0.
% \end{align*}
We connect above results to finish the proof. 
\end{proof}
We  provide two strategies for selecting matrix $\mU\in\BR^{d\times k}$ of the \srk~update:
\begin{enumerate}[label=(\alph*),topsep=1.5pt, leftmargin=1.2cm,itemsep=0.12cm]
\item For the randomized strategy, we construct matrix $\mU\in\BR^{d\times k}$ by sampling each of its entries  
 according to the standard normal distribution independently, i.e., $ [\mU]_{ij} \overset{{\rm i.i.d}}{\sim} \fN(0,1)$ for all $i\in[d]$ and $j\in[k]$.
% \begin{align*}
% %\label{eq:U_greedy}
%     [\mU]_{ij} \overset{{\rm i.i.d}}{\sim} \fN(0,1).
% \end{align*}
\item For the greedy strategy, we construct matrix  $\mU\in\BR^{d\times k}$ as $\U=\mE_{k}(\G-\A)$,
% \begin{align*}
% %\label{eq:U_random}
% \U=\mE_{k}(\G-\A),     
% \end{align*}
where $\E_k(\cdot)$ follows the definition in Section \ref{sec:pre}. 
\end{enumerate}
% \begin{remark}
% For $k=1$, SR-$k$ updates with above two strategies reduce to randomized or greedy SR1 updates~\cite{rodomanov2021rates,lin2021greedy}.
% \end{remark}
The following lemma indicates that the term of  $\tr{\U(\U^{\top}\U)^{-1}\U\R}$ in inequality \eqref{eq:keyeq} can be further lower bounded when we choose $\U\in\BR^{d\times k}$ by the above randomized or greedy strategy, which guarantees a sufficient decrease of~$\tau_{\A}(\cdot)$ for \srk~updates.

% The remain of this section shows the multiple directions in $\mU\in\BR^{d\times k}$ provably make \srk~update has the advantage over SR1 update in the view of estimating the target matrix $\mA$. The following lemma indicates that constructing $\U$ by randomized or greedy strategy can lower bound $\tr{\U(\U^{\top}\U)^{-1}\U^{\top}\R}$ by~$\tau_\A(\G)=\tr{\R}$.

\begin{lemma}
\label{lm:explicitbound}
\srk~updates with randomized and greedy strategies  have the following properties:
\begin{enumerate}[label=(\alph*),topsep=0pt, leftmargin=1.2cm,itemsep=0.15cm]
\item If $\U\in\BR^{d\times k}$ is chosen as $[\mU]_{ij} \overset{{\rm i.i.d}}{\sim} \fN(0,1)$, then we have    
    % If $\U$ is chosen by randomized strategy, i.e., each entry of $\mU$ is independently distributed to $\fN(0,1)$, then 
\begin{align}\label{eq:random-up}         \EBP{\tr{\U\left(\U^{\top}\U\right)^{-1}\U^{\top}\R}} = \frac{k}{d} \trcommon{\R}
\end{align}
for any matrix $\R\in\BR^{d\times d}$.     
\item If $\U\in\BR^{d\times k}$ is chosen as $\U=\E_k(\R)$, then we have
\begin{align}\label{eq:greedy-up}
{\tr{\U\left(\U^{\top}\U\right)^{-1}\U^{\top}\R}} \geq \frac{k}{d} \trcommon{\R}
\end{align}
for any matrix $\R\in\BR^{d\times d}$.
\end{enumerate}
\end{lemma}
\begin{proof}
We first consider the randomized strategy such that each entry of the matrix $\mU\in\BR^{d\times k}$ is independently sampled from~$\fN(0,1)$.
We use $\fV_{d,k}$ to present the Stiefel manifold which is the set of all $d\times k$ column orthogonal matrices and denote~$\fP_{k,d-k}$ as the set of all $m\times m$ orthogonal projection matrices of rank $k$.

According to Theorem 2.2.1 (iii) and Theorem 2.2.2 (iii) of Chikuse \cite{chikuse2003statistics}, the random matrix $ \Z=\U(\U^{\top}\U)^{-1/2}$
% \begin{align*}
%     \Z=\U(\U^{\top}\U)^{-1/2}
% \end{align*}
is uniformly distributed on $\fV_{d,k}$ and the random matrix
$  \Z\Z^\top=\U(\U^{\top}\U)^{-1}\U^{\top} $
% \begin{align*}
%     \Z\Z^\top=\U(\U^{\top}\U)^{-1}\U^{\top} 
% \end{align*}
is uniformly distributed on $\fP_{k,d-k}$.
Then, applying Theorem~2.2.2~(i) of Chikuse \cite{chikuse2003statistics} on matrix $\Z\Z^{\top}$ achieves
\begin{align}\label{eq:EP}    \EB\left[\U(\U^{\top}\U)^{-1}\U^{\top}\right] = \frac{k}{d}\I_d.
\end{align}
Consequently, we have
\begin{align*}
    \EB \left[\tr{\U\left(\U^{\top}\U\right)^{-1}\U^{\top}\R}\right]&=\tr{\EB\left[\U\left(\U^{\top}\U\right)^{-1}\U^{\top}\right]\R}\overset{\eqref{eq:EP}}{=}\frac{k}{d}\trcommon{\R}.
\end{align*}

% We first prove the following fact for the random matrix $\U$ that
% \begin{align}
% \label{eq:EP}    \EB\left[\U(\U^{\top}\U)^{-1}\U^{\top}\right] = \frac{k}{d}\I_d.
% \end{align}
% We use $\fV_{d,k}$ to present the Stiefel manifold which is the set of all $d\times k$ column orthogonal matrices.
% We denote $\fP_{k,d-k}$ as the set of all $m\times m$ orthogonal projection matrices of rank $k$.

% According to Theorem 2.2.1 (iii) of \citet{chikuse2003statistics}, the random matrix
% \begin{align*}
%     \Z=\U(\U^{\top}\U)^{-1/2}
% \end{align*}
% is uniformly distributed on the Stiefel manifold $\fV_{d,k}$.
% Applying Theorem 2.2.2 (iii) of \citet{chikuse2003statistics}, the random matrix
% \begin{align*}
%     \Z\Z^\top=\U(\U^{\top}\U)^{-1}\U^{\top} 
% \end{align*}
% is uniformly distributed on $\fP_{k,d-k}$. 
% Combining above results with Theorem 2.2.2 (i) of \citet{chikuse2003statistics} on $\Z\Z^{\top}$ achieves~\eqref{eq:EP}.

% The proof of \eqref{eq:EP} is formally presented in  Lemma~\ref{lm:EP} in Appendix~\ref{sec:auli}. 

Then we consider the greedy strategy such that $\U=\E_k(\R)$.
%We use $\{r_{i}\}_{i=1}^{d}$ to denote the diagonal entries of $\R$ such that $ r_{1}\geq r_{2} \geq \cdots \geq r_{d},$
%which implies
We use $\{r_{i}\}_{i=1}^{k}$ to denote $k$ largest diagonal entries of $\R$ with $ r_{1}\geq r_{2} \geq \cdots \geq r_{k}$, then we have
% \begin{align*}
%      r_{1}\geq r_{2} \geq \cdots \geq r_{d},
% \end{align*}
\begin{align}
\label{eq:bigeq}
    %\sum_{i=1}^d r_{i} =\tr{\R}\qquad{\text{and}}\qquad 
    \sum_{i=1}^{k}r_{i}\geq\frac{k}{d}\trcommon{\R}.
\end{align}
We let $\vu_i\in\BR^d$ be the $i$-th column of $\mU\in\BR^{d\times k}$, then we have
\begin{align*}  
&\tr{\U\big(\U^{\top}\U\big)^{-1}\U^{\top}\R} 
=\tr{\big(\U^{\top}\U\big)^{-1}\U^{\top}\R\U}
=\tr{\I_k\U^{\top}\R\U} =\tr{\U^{\top}\R\U}\\
&= \sum_{i=1}^{k} \u_{i}^{\top}\R \u_{i} =\sum_{i=1}^{k}r_{i}\!\!\overset{\eqref{eq:bigeq}}{\geq} \frac{k}{d}\trcommon{\R}.
\end{align*}
Hence, we finish the proof.
\end{proof}

% \begin{remark}
% The results of Lemma \ref{lm:pdneq2} and Lemma \ref{lm:explicitbound}(a) in the special case of~$k=1$ can be directly proved by using Cauchy--Schwarz inequality and properties of the uniform distribution on the unit sphere \cite{lin2021explicit}. 
% However, our analysis for the general case of~$k\geq 1$ are more challenging and have not been explored before.
% \end{remark}

\begin{remark}
The proof of result (\ref{eq:random-up}) in Lemma~\ref{lm:explicitbound}  uses some statistical results on manifold \cite{chikuse2003statistics}. For readers who are not familiar with manifold theory,
we also provide an elementary proof for equation (\ref{eq:random-up}) in Appendix~\ref{appen:addiproof}.
\end{remark}




Now, we formally present the convergence result~\eqref{eq:srk-aim} for \srk~update.
\begin{theorem}\label{thm:matrix}
Let $ \G_{+}={\text{\rm SR-$k$}}(\G,\A,\U)$
% \begin{align*}
% %\label{eq:block-update}
%  \G_{+}={\text{\rm SR-$k$}}(\G,\A,\U)
% \end{align*}
with $\G,\mA\in\RB^{d\times d}$ such that $\mG\succeq\mA$ and select $\U\in\RB^{d\times k}$ by the randomized strategy $[\mU]_{ij} \overset{{\rm i.i.d}}{\sim} \fN(0,1)$ or the greedy strategy~$\mU=\mE_k(\G-\A)$, where $k\leq d$. 
% one of the following strategies:
% \begin{enumerate}
% \item Sample each entry of $\mU$ according to $\fN(0,1)$ independently.
% \item Construct $\U=\E_k(\mG-\mA)$.
% \end{enumerate}
Then we have $\EB\left[\tau_{\A}(\G_{+})\right]\leq \left(1-k/d\right)\tau_\A(\G).  $
% \begin{align*}
% \EB\left[\tau_{\A}(\G_{+})\right]\leq \left(1-\frac{k}{d}\right)\tau_\A(\G).  
% \end{align*}
% , we have
% \begin{align}\label{ieq:srk-matrix}
%     \EB\left[\tau_{\A}(\G_{+})\right]\leq \left(1-\frac{k}{d}\right)\tau_\A(\G).
% \end{align}
\end{theorem}
\begin{proof}
Applying Lemma~\ref{lm:pdneq2} and Lemma~\ref{lm:explicitbound} by taking $\R=\G-\A$, we have
\begin{align*}
    \EBcommon{\tau_{\A}(\G_{+})} & ~~~\overset{\eqref{eq:tauAG+}}{=}~~~\tau_{\A}(\G)-\EBP{\tr{(\G-\A)\U(\U^{\top}\big(\G-\A)\U\big)^{\dag}\U^{\top}(\G-\A)}}\\
    & ~~~\overset{\eqref{eq:keyeq}}{\leq}~~~\tau_{\A}(\G) -\EBP{\tr{\U\left(\U^{\top}\U\right)^{-1}\U^{\top}(\G-\A)}}\\
    & \overset{\eqref{eq:random-up},\,\eqref{eq:greedy-up}}{\leq} \tau_{\A}(\G) -\frac{k}{d}\tr{(\G-\A)} 
    = \left(1-\frac{k}{d}\right)\tau_{\A}(\G).
\end{align*}
\end{proof}
%The term $(1-k/d)$ before $\tau_\mA(\mG)$ in the result of Theorem \ref{thm:matrix} reveals the advantage of \srk~update, since the larger $k$ leads to the faster decay on the measure $\tau_{\A}(\cdot)$.
%As a comparison, the results of ordinary randomized or greedy SR1 updates~\cite{lin2021greedy} can only match the special case of Theorem \ref{thm:matrix} when $k=1$.
Theorem \ref{thm:matrix} reveals the advantage of the \srk~update, i.e., we can achieve faster convergence with respect to $\tau_{\A}(\cdot)$ by increasing the block size $k$.
As a comparison, the results of ordinary randomized or greedy SR1 updates~\cite{lin2021greedy} are exactly the special case of Theorem \ref{thm:matrix} when $k=1$.

%\section{Minimizing of Strongly Self-Concordant Function}
\section{Minimizing Strongly Convex Function}\label{sec:srk-opt}


By leveraging the proposed \srk{ } updates, we introduce a novel block quasi-Newton method, referred to as the \srk{} method.
The specifics of the \srk~method are outlined in Algorithm~\ref{alg:SRK},
where $M>0$ is the self-concordant parameter that follows the notation in Assumption~\ref{ass:strongself}.
% \cc{By leveraging the proposed \srk~update, we study a novel block quasi-Newton method for strongly self-concordant function, which we call \srk~method. We present the \srk~method in Algorithm~\ref{alg:SRK}, where $M>0$ follows the notation in Assumption~\ref{ass:strongself}.}
% %We propose the \srk~method for minimizing strongly self-concordant function in Algorithm~\ref{alg:SRK}, where $M>0$ follows the notation in Assumption~\ref{ass:strongself}.

We shall consider the convergence rate of \srk~method (Algorithm~\ref{alg:SRK}) and show its superiority to existing quasi-Newton methods. 
Our convergence analysis is based on the measure of local gradient norm \cite{nesterov2018lectures,rodomanov2021greedy,lin2021greedy}, which is defined as 
\begin{align*}
\lambda(\x)\triangleq \sqrt{\nabla f(\x)^{\top}(\nabla^2f(\x))^{-1}\nabla f(\x)}.
\end{align*}
% \begin{align}
% \label{eq:convergemeasure}
%     \lambda(\x)\triangleq \sqrt{\nabla f(\x)^{\top}(\nabla^2f(\x))^{-1}\nabla f(\x)}.
% \end{align}
The analysis starts from the following result for quasi-Newton iterations.

\begin{lemma}[\!\!{\cite[Lemma 4.3]{rodomanov2021greedy}}]
\label{lm:linear-quadra}
Suppose that the twice differentiable objective $f:\BR^d\to\BR$ is strongly self-concordant with constant $M\geq0$ and the positive definite matrix $\G_t\in\BR^{d\times d}$ satisfies $ \nabla^2 f(\x_t)\preceq \G_t\preceq \eta_t \nabla^2 f(\x_t)$
% \begin{align}
% \label{eq:Gbound}
%     \nabla^2 f(\x_t)\preceq \G_t\preceq \eta_t \nabla^2 f(\x_t)
% \end{align}
for some $\eta_t\geq 1$ and $M\lambda(\x_t)\leq 2$.
Then the update formula 
\begin{align} \label{eq:iterbyG}
\x_{t+1}=\x_t-\G_t^{-1}\nabla f(\x_t)    
\end{align}
holds that 
\begin{align}
\label{eq:linear-quadra}
r_{t}\leq \lambda(\x_t)~~~\text{and}~~~  \lambda(\x_{t+1})\leq \left(1-\frac{1}{\eta_t}\right)\lambda(\x_t) + \frac{M(\lambda(\x_t))^2}{2}+\frac{M^2(\lambda (\x_t))^3}{4\eta_t},
\end{align}
where $r_t  = \|\x_{t+1}-\x_t\|_{\x_t}$.
\end{lemma}



Note that the value of $\eta_t$ in equation (\ref{eq:linear-quadra}) is crucial to describe the local convergence rates of different types of quasi-Newton methods.
Applying Lemma \ref{lm:linear-quadra} by setting $\eta_t = 3\eta_0/2$ and combining with Lemma~\ref{lm:sr1good}, we can establish the linear convergence rate of \srk~methods as follows (see Appendix~\ref{sec:srklinear} for the detailed proof).
 
\begin{theorem}\label{thm:srklinear}
Under Assumptions \ref{ass:smooth}, \ref{ass:strongconvex}, and \ref{ass:strongself}, if we run SR-k method (Algorithm \ref{alg:SRK}) with $\vx_0\in\BR^d$ and $\mG_0\in\BR^{d\times d}$ such that $\nabla^2 f(\x_0)\preceq \G_0\preceq \eta_0 \nabla^2f(\x_0)$ and  $M\lambda(\vx_0)\leq {\ln(3/2)}/(4\eta_0)$
% \begin{align*}
%     M\lambda(\vx_0)\leq \frac{\ln(3/2)}{4\eta_0}
% \qquad\text{and}\qquad
%     \nabla^2 f(\x_0)\preceq \G_0\preceq \eta_0 \nabla^2f(\x_0)
% \end{align*} 
for some $\eta_0\geq 1$. Then it holds that
\begin{align}
\label{eq:srklinear}
  \nabla^2 f(\x_{t})\preceq \G_t\preceq \frac{3\eta_0}{2}\nabla^2 f(\x_t)\qquad\text{and}\qquad\lambda(\x_t)\leq \left(1-\frac{1}{2\eta_0}\right)^t\lambda(\x_0).
\end{align}
\end{theorem}


\label{sec:algorithm}
\begin{algorithm}[t]
\caption{Symmetric Rank-$k$ Method}\label{alg:SRK}
\begin{algorithmic}[1]
\STATE \textbf{Input:} $\vx_0, \G_0$, $M$, and $k$ \\
\STATE \textbf{for} $t=0,1\dots$ \\
\STATE \quad $\x_{t+1}=\x_t-\G_t^{-1}\nabla f(\x_t)$ \\ 
\STATE \quad $r_t=\|\x_{t+1}-\x_{t}\|_{\x_t}$ \\
\STATE \quad  $\tilde{\G}_{t}=(1+Mr_t)\G_t$ \\
\STATE \quad Construct $\U_t\in\RB^{d\times k}$ by  \\
\quad\quad  (a) randomized strategy: $\left[\U_{t}\right]_{ij}\overset{\rm{i.i.d}}{\sim}{\fN(0,1)}$ \\
\quad\quad  (b) greedy strategy: $\U_t=\E_k(\tilde{\G}_t-\nabla^2 f(\x_{t+1}))$ \\
\STATE \quad $\G_{t+1}= \srk(\tilde{\G}_t,\nabla^2f(\x_{t+1}),\U_t)$ \\
\STATE \textbf{end for}
\end{algorithmic}
\end{algorithm}\vskip-0.3cm


Furthermore, we can obtain the superlinear rate for iteration \eqref{eq:iterbyG} if there exists some sequence $\{\eta_t\}$ such that $\eta_t\geq 1$ for all $t\geq 1$ and $\lim_{t\to\infty}\eta_t=1$.
For example, the randomized and greedy SR1 methods~\cite{lin2021greedy} leads to some $\{\eta_t\}$ such that~$\eta_t\geq 1$ and 
$\EB[\eta_t-1]\leq \OM((1-1/d)^{t})$.
As the results shown in Theorem \ref{thm:matrix}, the proposed \srk~updates have the superiority in matrix approximation. 
So we desire to construct some~$\{\eta_t\}$ for \srk~methods with the tighter bound
$\EB[\eta_t-1]\leq \OM((1-{k}/{d})^{t})$.

Based on above intuition, we derive the faster local superlinear convergence rate for \srk{} methods in the following theorem (see Appendix~\ref{sec:srk_proof} for the detailed proof), which is explicitly sharper than the convergence rates of existing randomized and greedy quasi-Newton methods~\cite{lin2021greedy,rodomanov2021greedy}.
% for the SR-$k$ method by incorporating the local linear-quadratic rate of the $\lambda(\x_t)$ and the linear convergence rate of matrix approximation provided in section~\ref{sec:update}.
\begin{theorem}
\label{thm:srk}
Under Assumptions \ref{ass:smooth}, \ref{ass:strongconvex}, and \ref{ass:strongself}, we run SR-k method (Algorithm~\ref{alg:SRK}) with $\vx_0\in\BR^d$ and $\mG_0\in\BR^{d\times d}$ such that 
\begin{align}
\label{eq:initial}
   M \lambda(\x_0)\leq \frac{\ln 2}{2} \cdot \frac{d-k}{ \eta_0 d^2\varkappa}\qquad\text{and}\qquad\nabla^2 f(\x_0)\preceq\G_0\preceq \eta_0\nabla^2f(\x_0)
\end{align} 
for some $k<d$ and $\eta_0\geq 1$. 
Then it holds that
\begin{align}
\label{eq:E_lambda_srk}
     \BE\left[\frac{\lambda(\x_{t+1})}{\lambda(\x_t)}\right]\leq 2d\varkappa\eta_0\left(1-\frac{k}{d}\right)^{t},
\end{align}
which naturally indicates the following two stage convergence behaviors:
\begin{enumerate}[label=(\alph*),topsep=0.05cm, itemsep=0.1cm, leftmargin=0.9cm]
\item For \srk~method with randomized strategy, we have
    \begin{align*}
    \lambda(\x_{t_0+t})\leq\left(1-\frac{k}{d+k}\right)^{t(t-1)/2}\cdot\left(\frac{1}{2}\right)^t\cdot\left(1-\frac{1}{2\eta_0}\right)^{t_0}\lambda(\x_0),
\end{align*}
with probability at least $1-\delta$ for some $\delta\in(0,1)$, where $t_0=\OM(d\ln(\varkappa d\eta_0/\delta)/k)$. 
\item For \srk~method with greedy strategy, we have
    \begin{align*}
    \lambda(\x_{t_0+t})\leq\left(1-\frac{k}{d}\right)^{t(t-1)/2}\cdot\left(\frac{1}{2}\right)^t\cdot\left(1-\frac{1}{2\eta_0}\right)^{t_0}\lambda(\x_0),
\end{align*}
where $t_0=\OM\left(d\ln(\eta_0d\varkappa )/k\right)$.
\end{enumerate}
\end{theorem}
% \begin{proof}
% We present the detailed proof in Appendix~\ref{sec:srk_proof}.
% \end{proof}

\begin{remark}
The condition $\nabla^2 f(\x_0)\preceq\G_0\preceq \eta_0\nabla^2f(\x_0)$ in Theorem~\ref{thm:srklinear} and \ref{thm:srk} can be satisfied by simply setting $\mG_0=L\mI_d$, then we can efficiently implement the update on $\mG_t$ by Woodbury identity~\cite{woodbury1950inverting}.
In a follow-up work, Liu et al.~\cite{liu2023block} analyzed block quasi-Newton methods for solving nonlinear equations. However, their convergence guarantees require the Jacobian estimator at the initial point be sufficiently accurate, which leads to potentially expensive cost in the step of initialization. 
\end{remark}

We can also set $k=d$ for \srk~methods, which leads to $\eta_t=1$ almost surely for all $t\geq 1$ and achieves the quadratic convergence rate like standard Newton methods. %We formally present this result as follows.
% \srk~methods with $k=d$ have the local quadratic convergence rate because we have $\eta_t=1$ almost surely for all $t\geq 1$.
\begin{corollary}
\label{cor:recoverNewton}
Under Assumptions~\ref{ass:smooth}, \ref{ass:strongconvex}, and \ref{ass:strongself}, we run SR-$k$ method (Algorithm~\ref{alg:SRK}) with $k=d$ and $\vx_0\in\BR^d$ such that~$M\lambda(\x_0)\leq 2$,
then it holds that~$\lambda(\x_{t+1})\leq M(\lambda(\x_t))^2$ almost surely 
% \begin{align}
%   \lambda(\x_{t+1})\leq M(\lambda(\x_t))^2
% \end{align}
for all  $t\geq 1$. 
\end{corollary}
\begin{proof}
    The update rule  $\G_{t}=\srk(\tilde{\G}_{t-1},\nabla^2 f(\x_{t}),\U_{t-1})$ implies the equality 
    $\U_{t-1}^{\top}\G_{t}\U_{t-1} = \U_{t-1}^{\top}\nabla^2 f(\x_{t})\U_{t-1} $.
    % \begin{align*}
    %  &\U_{t-1}^{\top}\G_{t}\U_{t-1} \\
    %  &= \U_{t-1}^{\top}\tilde{\G}_{t-1}\U_{t-1}-\U_{t-1}^{\top}(\tilde{\G}_{t-1}-\nabla^2 f(\x_{t}))\U_{t-1}\big(\U^{\top}(\tilde{\G}_{t-1}-\nabla^2 f(\x_{t}))\U_{t-1}\big)^{\dag}\U_{t-1}^{\top}(\tilde{\G}_{t-1}-\nabla^2 f(\x_{t}))\U_{t-1}\\
    %  &=\U_{t-1}^{\top}\nabla^2 f(\x_{t})\U_{t-1}.   
    % \end{align*}
    Our choice of $\U_t\in\RB^{d\times d}$ guarantees that it is non-singular almost surely, so we have $  \G_{t}=\nabla^2 f(\x_{t})$
    for all $t\geq 1$. 
    Lemma~\ref{lm:linear-quadra} with~$\eta_{t}=1$ implies
    $ \lambda(\x_{t+1})\leq {M(\lambda(\x_t))^2}/{2} + {M^2(\lambda(\x_t))^3}/{2} \leq M(\lambda(\x_t))^2$
    % \begin{align*}
    %     \lambda(\x_{t+1})\leq \frac{M(\lambda(\x_t))^2}{2} + \frac{M^2(\lambda(\x_t))^3}{4} \leq M(\lambda(\x_t))^2
    % \end{align*}
    % 
    holds for all $t\geq 1$ almost surely, which finishes the proof.
\end{proof}






%  \begin{proof}
% According to Theorem~\ref{thm:matrix}, we have
% \begin{align*}
%     \BE[\tau_{\nabla^2 f(\x_{t+1})}(\G_{t+1})]{=}0.
% \end{align*}
% According to Theorem~\ref{thm:srklinear} and Lemma~\ref{lm:GHneq} in Appendix~\ref{sec:srk_proof}, we have
% \begin{align}
% \label{eq:k=dlambdaneq}
%   \nabla^2f(\x_{t+1})\overset{\eqref{eq:srklinear}}{\preceq}  \G_{t+1}\preceq \underbrace{\left(1+\frac{d\varkappa\tau_{\nabla^2 f(\x_{t+1})}(\G_{t+1})}{\tr{\nabla^2 f(\x_{t+1})}}\right)}_{a_{t+1}}\nabla^2 f(\x_{t+1})~~~\text{and}~~~\lambda_{t+1}\leq \lambda_0,
% \end{align}
% where 
% \begin{align*}
%     \EB[a_{t+1}]=1+\frac{d\varkappa}{\tr{\nabla^2 f(\x_{t+1})}}\EBP{\tau_{\nabla^2 f(\x_{t+1})}(\G_{t+1})}=1.
% \end{align*}
% According to Lemma~\ref{lm:linear-quadra}, we have
% \begin{align}
% \label{eq:k=dlambda}
%     \lambda_{t+2}\leq \underbrace{\left(1-\frac{1}{a_{t+1}}\right)}_{b_{t+1}}\lambda_{t+1} + \frac{M\lambda_t^2}{2} + \frac{M^2\lambda_{t+1}^3}{4a_{t+1}}.
% \end{align}
% It also holds that
% \begin{align*}
%    0\leq \EB[b_{t+1}]=1 -\EB[1/a_{t+1}]{\leq} 1-1/\EB[a_{t+1}]= 0,
% \end{align*}
% where the first inequality comes from the fact that $a_{t+1}\geq 1$ and the second inequality comes from the fact 
% $\EB[1/X]\geq 1/\EB[X]$ for positive random variable $X>0$ by using Jensen's inequality.

% Thus, we have
% \begin{align*}
%     \EB[\lambda_{t+2}]\overset{\eqref{eq:k=dlambda}}{\leq} \frac{M\lambda_{t+1}^2}{2}+ \frac{M^2\lambda_{t+1}^3}{4}\overset{\eqref{eq:k=dlambdaneq}}{\leq}  \frac{M\lambda_{t+1}^2}{2}+ \frac{M\lambda_{t+1}^2}{2}\cdot\frac{M\lambda_{0}}{2}\overset{\eqref{eq:k=dinitial}}{\leq} M\lambda_{t+1}^2,
% \end{align*}
% for all $t\geq0$.
%  
% % \begin{align}
% %     \EBP{}
% % \end{align}
%  \end{proof}



\section{Improved Results for Block BFGS and DFP Methods}\label{sec:Block BFGS}
% DFP
% Unified Broyden
% Greedy version of BFGS/DFP

\begin{algorithm}[t]
\caption{Randomized Block BFGS/DFP}\label{alg:bfgs}
\begin{algorithmic}[1]
\STATE \textbf{Input:} $\x_0$, $\G_0$, $M$, and $k$ \\
\STATE \textbf{for} $t=0,1\dots$\\
\STATE \quad $\x_{t+1}=\x_t-\G_t^{-1}\nabla f(\x_t)$ \\
\STATE \quad $r_t=\|\x_{t+1}-\x_{t}\|_{\x_t}$ \\[0.08cm]
\STATE \quad $\tilde{\G}_{t}=(1+Mr_t)\G_t$ \label{line:correction} \\
\STATE \quad Construct $\U_t$ by $\left[\U_{t}\right]_{ij}\overset{\rm{i.i.d}}{\sim} {\fN(0,1)}$ \\[0.03cm]
\STATE \quad Update $\G_t$ by  \\[0.05cm]
\quad \quad  (a) $\G_{t+1}=\bfgs(\tilde{\G}_t,\nabla^2f(\x_{t+1}),\U_t)$\\[0.05cm]
\quad \quad (b) $\G_{t+1}= \dfp(\tilde{\G}_t,\nabla^2f(\x_{t+1}),\U_t)$ \\
\STATE \textbf{end for} 
\end{algorithmic}
\end{algorithm}


Following our our investigation on \srk~methods, 
we can also achieve the non-asymptotic superlinear convergence rates of randomized block BFGS and randomized block DFP methods \cite{gower2016stochastic,gower2017randomized}.
%In this section, we provide the non-asymptotic superlinear convergence rates of randomized block BFGS method \cite{gower2016stochastic,gower2017randomized} and randomized block DFP method by following the idea of \srk.
The block BFGS~\cite{schnabel1983quasi,gower2017randomized,gower2016stochastic} and DFP~\cite{gower2017randomized} updates are defined as follows.

\begin{definition}
%[\citet{schnabel1983quasi,gower2016stochastic,gower2017randomized}]
Let $\A\in\RB^{d\times d}$ and $\G\in \RB^{d\times d}$ be two positive-definite symmetric matrices with $\A\preceq \G$. For any full rank matrix $\U\in\RB^{d\times k}$ with $k\leq d$, we define 
%{\color{red}quadratic convergence when $k=d$}
\begin{align}\label{update:RaBFGS}
   {\text{\rm BlockBFGS}}(\G,\A,\U) 
   \triangleq \G\!-\!\G\U\left(\U^{\top}\G\U\right)^{-1}\U^{\top}\G+\A\U\left(\U^{\top}\A\U\right)^{-1}\U^{\top}\A,
\end{align}
and
\begin{align}
\label{eq:dfp-update}
\begin{split}
    &{\text{\rm BlockDFP}}(\G,\A,\U)\\
    & \triangleq \!\A\U\big(\U^{\top}\!\A\U\big)^{-1}\U^{\top}\!\A\!+\!\big(\I_d\!-\!\A\U\big(\U^{\top}\!\A\U\big)^{-1}\U^{\top}\big)\G\big(\I_d\!-\!\U\!\big(\U^{\top}\!\A\U\big)^{-1}\U^{\top}\!\A\!\big)\!.
  \end{split}
\end{align}
\end{definition}
In previous work, Gower et al. \cite{gower2016stochastic} and Kovalev et al. \cite{kovalev2020fast} proposed randomized block BFGS method 
 by constructing the Hessian estimator with formula (\ref{update:RaBFGS}) and showed it has asymptotic local superlinear convergence rate. 
On the other hand, Gower and Richt{\'a}rik~\cite{gower2017randomized} studied the randomized block DFP update (\ref{eq:dfp-update}) for matrix approximation but they did not consider solving optimization problems.




To achieve the explicit superlinear convergence rate of block BFGS and block DFP methods, we provide some properties of block BFGS and block DFP updates which are similar to our observations on \srk~update.
We first show that block BFGS and DFP updates also have the non-increasing deviation from the target matrix.


\begin{lemma}
\label{lm:bfgsnofar}
Given any positive-definite matrices $\mA\in\BR^{d\times d}$ and $\mG\in\BR^{d\times d}$ with~$\A\preceq\G\preceq\eta \A$
% \begin{align}
% \label{eq:blockneq-condi}
% $\A\preceq\G\preceq\eta \A$
% \end{align}
for some~$\eta\geq 1$, both of updates $\G_{+}={ \text{\rm BlockBFGS}}(\G,\A,\U)$ and $\G_{+}={ \text{\rm BlockDFP}}(\G,\A,\U)$ for full rank matrix~$\U\in\RB^{d\times k}$ hold that $ \A\preceq\G_{+}\preceq \eta\A. $
% \begin{align}
% \label{eq:blockneq}
%    \A\preceq\G_{+}\preceq \eta\A. 
% \end{align}
\end{lemma}
\begin{proof}
%\textbf{Block BFGS Update.}
We first consider the block BFGS update $\G_{+}={ \text{\rm BlockBFGS}}(\G,\A,\U)$. 
According to the Woodbury identity~\cite{woodbury1950inverting}, we have
% \begin{align}    \label{update:invRaBFGS}
% \G_{+}^{-1}\!\!=\!\!\U(\U^{\top}\!\A\U)^{-1}\!\U^{\top}\!\!+\!\!\left(\I_d\!-\!\U(\U^{\top}\!\A\!\U)^{-1}\!\U^{\top}\A\right)\!\G^{-1}\!\!\left(\I_d\!-\!\A\U(\U^{\top}\!\A\U)^{-1}\!\U^{\top}\!\right)\!.
% \end{align}
\begin{align}\label{update:invRaBFGS}
\begin{split}    
\G_{+}^{-1} = & \,\U\big(\U^{\top}\A\U\big)^{-1}\U^{\top} \\
& + \big(\I_d-\U\big(\U^{\top}\A\U\big)^{-1}\U^{\top}\A\big)\G^{-1}\big(\I_d-\A\U\big(\U^{\top}\A\U\big)^{-1}\U^{\top}\big).
\end{split}
\end{align}
The condition $\A\preceq\G\preceq\eta \A$ means $\eta^{-1}\A^{-1}\preceq\G^{-1}\preceq \A^{-1}$,
% \begin{align}
% \label{eq:blockneq1}
%     \frac{1}{\eta}\A^{-1}\preceq\G^{-1}\preceq \A^{-1},
% \end{align}
which implies
% \begin{align*}
%     \G_{+}^{-1}\!\!\!\overset{\textcolor{red}{\eqref{update:invRaBFGS}}}{\preceq} \!\!\U(\U^{\top}\!\A\U)^{-1}\U^{\top}\!\!+\!\left(\I_d\!-\!\U(\U^{\top}\!\A\U)^{-1}\U^{\top}\!\A\right)\!\A\!^{-1}\!\left(\I_d\!-\!\A\U(\U^{\top}\!\A\U)^{-1}\U^{\top}\right)\!\preceq\!\A\!^{-1}
% \end{align*}
\begin{align*}
    \G_{+}^{-1}\overset{\eqref{update:invRaBFGS}}{\preceq} & \U\big(\U^{\top}\A\U\big)^{-1}\U^{\top} \\
    & + \big(\I_d-\U\big(\U^{\top}\A\U\big)^{-1}\U^{\top}\A\big)\A^{-1}\big(\I_d-\A\U\big(\U^{\top}\A\U\big)^{-1}\U^{\top}\big) = \A^{-1}
\end{align*}
and
\begin{align*}
\G_{+}^{-1}\!\!
&\overset{\eqref{update:invRaBFGS}}{\succeq}\!\!\U\big(\U^{\top}\!\A\U\big)^{-1}\U^{\top}\!\!+\!\eta^{-1}\!\big(\I_d\!-\!\U\big(\U^{\top}\!\A\U\big)^{-1}\U^{\top}\!\A\big)\!\A\!^{-1}\!\big(\I_d\!-\!\A\U\big(\U^{\top}\!\A\U\big)^{-1}\U^{\top}\big)\\
    &~=\eta^{-1}\A ^{-1}+(1-\eta^{-1})\U\big(\U^{\top}\A\U\big)^{-1}\U^{\top}\succeq \eta^{-1}\A^{-1}.
\end{align*}
Thus we have $ {\eta}^{-1}\A^{-1}\preceq\G^{-1}_{+}\preceq \A^{-1}$, which finishes the proof for the block BFGS update.
% which is equivalent to the desired result \eqref{eq:blockneq}.
We then consider the DFP update $\G_{+}={\text{\rm BlockDFP}}(\G,\A,\U)$.
The condition $\A\preceq\G\preceq\eta \A$ means
\begin{align*}
\G_{+}\!\!\!\overset{\eqref{eq:dfp-update}}{\succeq}\!\!\!
\A\U\big(\U^{\top}\!\A\U\big)\!^{-1}\U^{\top}\!\A \!+\!\big(\I_d\!-\!\A\U\big(\U^{\top}\!\A\U\big)\!^{-1}\U^{\top}\big)\!\A\!\big(\I_d\!-\!\U\big(\U^{\top}\!\A\U\big)^{-1}\U^{\top}\!\A\big)\!=\!\A
\end{align*}
and
\begin{align*}
 \G_{+}\!\!&\overset{\eqref{eq:dfp-update}}{\preceq} \!
\A\U\big(\U^{\top}\!\A\U\big)^{-1}\U^{\top}\!\A \!+\!\eta\big(\I_d\!-\!\A\U\big(\U^{\top}\!\A\U\big)^{-1}\U^{\top}\big)\!\A\!\big(\I_d\!-\!\U\big(\U^{\top}\!\A\U\big)^{-1}\U^{\top}\A\big)\\
&\,\,\,= \eta\A +(1-\eta)\A\U\big(\U^{\top}\A\U\big)^{-1}\U^{\top}\A \preceq \eta\A,
\end{align*}
where the last inequality is due to the facts $\eta\geq 1$ and $\A\U\big(\U^{\top}\A\U\big)^{-1}\U^{\top}\A\succeq\0$, which finishes the proof for block DFP update.
\end{proof}

Then we introduce the following quantity~\cite{rodomanov2021greedy}
\begin{align}
\label{eq:measurebfgs}
    \sigma_{\A}(\G)\triangleq\tr{\A^{-1}(\G-\A)}
\end{align}
to measure the difference between the target matrix $\mA\in\BR^{d\times d}$ and the current estimator~$\mG\in\BR^{d\times d}$. 
In the following theorem, we show that randomized block BFGS and DFP updates converge to the target matrix with a faster rate than the ordinary randomized BFGS and DFP updates~\cite{rodomanov2021greedy,lin2021greedy}.


\begin{theorem}\label{thm:bfgs}
Consider the randomized block update
\begin{align}\label{eq:bfgsupdate}
    \G_{+}={\text{\rm BlockBFGS}}(\G,\A,\U)\qquad\text{or}\qquad\G_{+}={\text{\rm BlockDFP}}(\G,\A,\U),
\end{align}
where $\G,\A\in\RB^{d\times d}$ and $\G\succeq\A$. If $\mu\I_d\preceq\A\preceq L\I_d$ and  
$\U\in\BR^{d\times k}$ is chosen as~$[\mU]_{ij} \overset{{\rm i.i.d}}{\sim} \fN(0,1)$, then we have
\begin{align}
\label{eq:bfgssigma}
  \EB\left[\sigma_{\A}(\G_{+})\right]\leq \left(1-\frac{k}{d\varkappa}\right)\sigma_\A(\G).
\end{align}
\end{theorem}




\begin{proof}
% We first show that the inequality
% \begin{align}
% \label{eq:trineq2}
% \begin{split}
% &\tr{\big(\U^{\top}\A\U\big)^{-1}\U^{\top}\G\U}-\tr{\A\U\big(\U^{\top}\A\U\big)^{-1}\U^{\top}}\\
% &\geq \frac{\mu}{L}\tr{\A^{-1}(\G-\A)\U\big(\U^{\top}\U\big)^{-1}\U^{\top}}
% \end{split}
% \end{align}
% holds for all $\G\in\BR^{d\times d}$, $\A\in\BR^{d\times d}$, and $\U\in\BR^{d\times k}$ satisfying the conditions of this theorem. 
The condition $\mu\I_d\preceq\A\preceq L\I_d$ means we have
\begin{align}
    \label{eq:Aneq}   \left(\U^{\top}\A\U\right)^{-1} \succeq \frac{1}{L}\cdot\left(\U^{\top}\U\right)^{-1} \qquad\text{and}\qquad \frac{1}{\mu}\cdot\I_d-\A^{-1}\succeq\0,
\end{align}
which leads to the following inequality
\begin{align}
\label{eq:trineq2}
\begin{split}           
    &{\rm tr}\big(\!\left(\U^{\top}\A\U\right)^{-1}\U^{\top}\G\U\big)-{\rm tr}\big(\A\U\left(\U^{\top}\A\U\right)^{-1}\U^{\top}\big)\\
    &={\rm tr}\big(\!\left(\U^{\top}\A\U\right)^{-1}\U^{\top}(\G-\A)\U\big)
    \overset{\eqref{eq:Aneq}}{\geq} \frac{1}{L}\tr{\big(\U^{\top}\U\big)^{-1}\U^{\top}(\G-\A)\U}
  \\
  &=\frac{1}{L}\tr{(\G-\A)\U\left(\U^{\top}\U\right)^{-1}\U^{\top}}= \frac{\mu}{L}{\rm tr}\left(\frac{\I_d}{\mu}\cdot(\G-\A)\U\left(\U^{\top}\U\right)^{-1}\U^{\top}\right) 
     %&~=~\frac{\mu}{L}{\rm tr}\bigg(\!\!\left(\frac{\I_d}{\mu}-\A^{-1}\right)^{1/2}\!\!\!(\G-\A)^{1/2}\U\left(\U^{\top}\U\right)^{-1}\U^{\top}(\G-\A)^{1/2}\left(\frac{\I_d}{\mu}-\A^{-1}\right)^{1/2}\bigg) \\
     %&~~~~~+ \frac{\mu}{L}\tr{\A^{-1}(\G-\A)\U\left(\U^{\top}\U\right)^{-1}\U^{\top}} \\
\\
&\!\!\overset{\eqref{eq:Aneq}}{\geq} \frac{\mu}{L}\tr{\A^{-1}(\G-\A)\U\left(\U^{\top}\U\right)^{-1}\U^{\top}}.
\end{split}
\end{align}

For the block BFGS update, we set $\P=\A^{1/2}\U$. Then it holds
\begin{align}\label{eq:bfgs-proof-psd}
\begin{split}
  &\U^{\top}\G{\big(\A^{-1}-\U\left(\U^{\top}\A\U\right)^{-1}\U^{\top}\big)}\G\U\\
  &= \U^{\top}\G\A^{-1/2}\big(\I_d -\A^{1/2}\U\left(\U^{\top}\A\U\right)^{-1}\U^{\top}\A^{1/2}\big)\A^{-1/2}\G\U\\
    &= \U^{\top}\G\A^{-1/2}\big(\I_d-\P\left(\P^{\top}\P\right)^{-1}\P^{\top}\big)\A^{-1/2}\G\U
    \succeq \0,
\end{split}
\end{align}
which implies 
\begin{align}
\label{eq:trineq}
\begin{split}               
& \tr{\U\G\big(\U^{\top}\G\U\big)^{-1}\U^{\top}\G\A^{-1}}-\tr{\big(\U^{\top}\A\U\big)^{-1}\U^{\top}\G\U} \\
&=\tr{\big(\U^{\top}\G\U\big)^{-1}\U^{\top}\G\A^{-1}\G\U}-\tr{\big(\U^{\top}\A\U\big)^{-1}\U^{\top}\G\U}\\    &=\tr{\big(\U^{\top}\G\U\big)^{-1}\big(\U^{\top}\G\A^{-1}\G\U - \U^{\top}\G\U\big(\U^{\top}\A\U\big)^{-1}\U^{\top}\G\U\big)}\\    &=\text{tr}\big(\big(\U^{\top}\!\G\U\big)\!^{-1/2}\U^{\top}\!\G{\big(\A\!^{-1}\!-\!\U\left(\U^{\top}\!\A\U\right)^{-1}\!\U^{\top}\big)}\G\U\!\left(\U^{\top}\!\G\U\right)\!^{-1/2}\big)\!\overset{\eqref{eq:bfgs-proof-psd}}{\geq}\! 0.
\end{split}
\end{align}
The equation \eqref{update:RaBFGS} implies
\begin{align*}
&\tr{(\G_{+}-\A)\A^{-1}} \\ &~\overset{\eqref{update:RaBFGS}}{=}\tr{(\G-\A)\A^{-1}}-\tr{\G\U\left(\U^{\top}\G\U\right)^{-1}\U^{\top}\G\A^{-1}}+\tr{\A\U\left(\U^{\top}\A\U\right)^{-1}\U^{\top}}\\
 &\overset{\eqref{eq:trineq}}{\leq} \tr{(\G-\A)\A^{-1}}- \big(\tr{\big(\U^{\top}\A\U\big)^{-1}\U^{\top}\G\U}-\tr{\A\U\big(\U^{\top}\A\U\big)^{-1}\U^{\top}}\big).
\end{align*}
Taking expectation on both sides of above result, we obtain 
\begin{align}
\label{eq:sigmaupdate}
\begin{split}
& \EBcommon{\sigma_{\A}(\G_{+})}
 \overset{\eqref{eq:measurebfgs}}{\leq} \sigma_{\A}(\G)-\EBP{\tr{\big(\U^{\top}\A\U\big)^{-1}\U^{\top}\G\U-\A\U\big(\U^{\top}\A\U\big)^{-1}\U^{\top}}} \\
   &\overset{\eqref{eq:trineq2}}{\leq}\sigma_{\A}(\G) - \BE\left[\frac{\mu}{L}\tr{\A^{-1}(\G-\A)\U\big(\U^{\top}\U\big)^{-1}\U^{\top}}\right]\\
   &~=\sigma_{\A}(\G)- \frac{k}{d\varkappa}\sigma_\A(\G) = \left(1-\frac{k}{d\varkappa}\right)\sigma_\A(\G),
\end{split}
\end{align}
where the last line is achieved by applying  Lemma~\ref{lm:explicitbound}(a) with $\R=\A^{-1}(\G-\A)$. 

For the block DFP update, the equation \eqref{eq:dfp-update} indicates 
% \begin{align*}
%       & \tr{ (\G_{+}-\A)\A^{-1}}\\
%    &\overset{\eqref{eq:dfp-update}}{=} \tr{(\G-\A)\A^{-1}} - \big(\tr{\big(\U^{\top}\A\U\big)^{-1}(\U^{\top}\G\U)}-\tr{\A\U\big(\U^{\top}\A\U\big)^{-1}\U^{\top}}\big).
% \end{align*}
\begin{align*}
\begin{split}
        & \tr{ (\G_{+}-\A)\A^{-1}}\\
   &\overset{\eqref{eq:dfp-update}}{=} \tr{\A\U(\U^{\top}\A\U)^{-1}\U^{\top}}+\tr{(\G-\A)\A^{-1}}\\    &~~~~~~+\tr{\A\U(\U^{\top}\A\U)^{-1}\U^{\top}\G\U(\U^{\top}\A\U)^{-1}\U^{\top}} \\
   &~~~~~~ -\tr{\A\U(\U^{\top}\A\U)^{-1}\U^{\top}\G\mA^{-1}}-\tr{\G\U(\U^{\top}\A\U)^{-1}\U^{\top}}\\
   &~=~\tr{(\G-\A)\A^{-1}} - \big(\tr{(\U^{\top}\A\U)^{-1}(\U^{\top}\G\U)}-\tr{\A\U\big(\U^{\top}\A\U\big)^{-1}\U^{\top}}\big).
\end{split}   
\end{align*}
Take the expectation on both sides of above equation, then we can follow the analysis of the block BFGS update from the step of~\eqref{eq:sigmaupdate} to prove desired result for block DFP update. Hence, we finish the proof.
\end{proof}
\begin{remark}   
Notice that Gower and Richt{\'a}rik~\cite{gower2017randomized} also established convergence rates for randomized block BFGS and DFP updates, but their results do not reveal the relationship between the convergence rates and the block size $k$.
\end{remark}

% Then we have
% \begin{align}
% \label{eq:trneq3}
% \begin{split}
%     &\EB\left[\tr{\G\U\left(\U^{\top}\G\U\right)^{-1}\U^{\top}\G\A^{-1}}-\tr{\A\U\left(\U^{\top}\A\U\right)^{-1}\U^{\top}}\right]\\
%     &=~~~\EB\left[\tr{\G\U\left(\U^{\top}\G\U\right)^{-1}\U^{\top}\G\A^{-1}}-\tr{\left(\U^{\top}\A\U\right)^{-1}\left(\U^{\top}\G\U\right)}\right] \\
%     &~~~~~~+ \EB\left[\tr{\left(\U^{\top}\A\U\right)^{-1}\left(\U^{\top}\G\U\right)}-\tr{\A\U\left(\U^{\top}\A\U\right)^{-1}\U^{\top}}\right]\\
%     &\!\!\!\!\overset{\eqref{eq:trineq},\eqref{eq:trineq2}}{\geq}\frac{\mu}{L}\EB\left[\tr{\A^{-1}(\G-\A)\U\left(\U^{\top}\U\right)^{-1}\U^{\top}}\right]=\frac{k}{d\varkappa}\sigma_\A(\G),
% \end{split}
% \end{align}
% where the last equality is by taking $\R=\A^{-1}(\G-\A)$ in Lemma~\ref{lm:explicitbound} (a).
% Finally, we have
% \begin{align*}
%     &\EB[\sigma_\A(\G_{+})]\\
%    & \overset{\eqref{eq:sigmaupdate}}{=}\sigma_\A(\G)-\EB_\U\left[\tr{\G\U\left(\U^{\top}\G\U\right)^{-1}\U^{\top}\G\A^{-1}}-\tr{\A\U\left(\U^{\top}\A\U\right)^{-1}\U^{\top}}\right]
%    \\
%    &\overset{\eqref{eq:trneq3}}{\leq} \left(1-\frac{k}{d\varkappa}\right)\sigma_\A(\G).
% \end{align*}

% which implies
% \begin{align*}
%     \EBP{\sigma_{\A}(\G_{+})} 
%     &\leq~ \sigma_{\A}(\G) - \EBP{\tr{(\U^{\top}\A\U)^{-1}(\U^{\top}\G\U)}-\tr{\A\U\left(\U^{\top}\A\U\right)^{-1}\U^{\top}}}\\
%     &\!\!\overset{\eqref{eq:trineq2}}{\leq}\sigma_{\A}(\G)-\frac{1}{\varkappa}\EBP{\tr{\A^{-1}(\G-\A)\U\left(\U^{\top}\U\right)^{-1}\U^{\top}}}\\
%     &=~\sigma_{\A}(\G)- \frac{k}{d\varkappa}\sigma_\A(\G).
% \end{align*}

% \begin{remark}
% \label{rmk:BFGS}
%     For $k=d$, the update rules of block BFGS and block DFP imply that $\G_{+}\U=\A\U$ for $\U\in\RB^{d\times d}$. Since $\U$ is non-singular a.s, we have $\G_{+}=\A$ a.s. Thus, we have a sharper rate over \eqref{eq:bfgsupdate} such that
%     \begin{align*}
%         \EBP{\sigma_{\A}(\G_{+})} = 0.
%     \end{align*}
% \end{remark}



We propose randomized block BFGS and DFP methods in Algorithm~\ref{alg:bfgs}.
Using the results of Theorem~\ref{thm:bfgs}, we establish the explicit superlinear convergence rate for our methods as follows  (see Appendix \ref{sec:bfgs_proof} for the detailed proof).

\begin{theorem}
\label{thm:BFGS}
Under Assumption~\ref{ass:smooth}, \ref{ass:strongconvex}, and \ref{ass:strongself}, we run randomized block BFGS/DFP method (Algorithm~\ref{alg:bfgs}) with $\vx_0\in\BR^d$ and $\mG_0\in\BR^{d\times d}$ such that
\begin{align}
\label{eq:bfgsini}
 M \lambda(\x_0)\leq \frac{\ln 2}{4}\cdot \frac{d\varkappa - k}{\eta_0 d^2\varkappa} 
\qquad\text{and}\qquad
\nabla^2 f(\x_0)\preceq\G_0\preceq \eta_0\nabla^2f(\x_0)
\end{align}
for some $k<d$ and $\eta_0\geq 1$. 
Then we have 
% $    \EBP{\frac{\lambda(\x_{t+1})}{\lambda(\x_t)}}\leq 2d\eta_0\left(1-\frac{k}{d\varkappa}\right)^{t}.$
\begin{align*}
    \BE\left[\frac{\lambda(\x_{t+1})}{\lambda(\x_t)}\right]\leq 2d\eta_0\left(1-\frac{k}{d\varkappa}\right)^{t}.
\end{align*}
\end{theorem}
% \begin{proof}
% We present the proof details in Appendix~\ref{sec:bfgs_proof}.
% \end{proof}

% \begin{corollary}
% \label{thm:bfgs-1}
% Consider the block BFGS update
% \begin{align}
% \label{eq:bfgsupdate-1}
%     \G_{+}={ \text{\rm BlockBFGS}}(\G,\A,\LL^{\top}\tilde{\U})
% \end{align}
% where $\A\in\RB^{d\times d}$ is positive definite satisfies
% $\mu\I\preceq\A\preceq L\I$, 
% $\G\succeq\A$, the square matrix $\LL$ satisfies that $\LL^{\top}\LL=\G^{-1}$, and the  elements of random matrix $\tilde{\U}\in\RB^{d\times k}$ are independent and identically distributed as $\fN(0,1)$.
% Then,we have
% \begin{align}
% \label{eq:bfgssigma-1}
%     \EB_{\tilde{\U}}\left[\sigma_{\A}(\G_{+})\right]\leq \left(1-\frac{k}{d}\right)\sigma_\A(\G).
% \end{align}
% \end{corollary}


For randomized block BFGS method, we can introduce the scaled directions to eliminate the condition number in the results of Theorem \ref{thm:bfgs} and \ref{thm:BFGS}. We first improve the result for matrix approximation as follows.

\begin{theorem}\label{thm:bfgs-1}
Consider the block BFGS update
\begin{align}
\label{eq:bfgsupdate-1}
    \G_{+}={ \text{\rm BlockBFGS}}(\G,\A,\LL^{\top}{\U})
\end{align}
where $\mG\succeq\A\in\RB^{d\times d}$.
If $\mu\I\preceq\A\preceq L\I$,  $\LL\in\BR^{d\times d}$ satisfies that $\LL^{\top}\LL=\G^{-1}$, and $\U\in\BR^{d\times k}$ is chosen as $[\mU]_{ij} \overset{{\rm i.i.d}}{\sim} \fN(0,1)$.
Then we have
\begin{align}
\label{eq:fasterbfgssigma-1}
    \EB\left[\sigma_{\A}(\G_{+})\right]\leq \left(1-\frac{k}{d}\right)\sigma_\A(\G).
\end{align}
\end{theorem}
\begin{proof}
According to the block-BFGS update \eqref{update:RaBFGS}, we have
\begin{align*}
   &\tr{ (\G_{+}-\A)\A^{-1}}\\
   &=\tr{(\G-\A)\A^{-1}}-\tr{\LL^{-1}\U\big(\U^{\top}\U\big)^{-1}\U^{\top}\LL^{-\top}\A^{-1}}\\
&~~~~~~~~~~~~~~~~~~+\tr{\A\LL^{\top}\U\big(\U\LL^{\top}\A\LL^{\top}\U\big)^{-1}\U^{\top}\LL}\\
   &=\tr{(\G-\A)\A^{-1}}-\tr{\LL^{-1}\U\big(\U^{\top}\U\big)^{-1}\U^{\top}\LL^{-\top}\A^{-1}}+k.
\end{align*}
The second term on the right-hand side holds that
\begin{align*}   
    & \EBP{\tr{\LL^{-1}\U\big(\U^{\top}\U\big)^{-1}\U^{\top}\LL^{-\top}\A^{-1}}}\\
    &=\tr{\EBP{\U(\U^{\top}\U)^{-1}\U^{\top}}\LL^{-\top}\A^{-1}\LL^{-1}}
    \overset{\eqref{eq:EP}}{=}\frac{k}{d}\tr{\A^{-1}\G}.
\end{align*} 
Thus, we combine the above results and get
\begin{align*}
   & \EB\left[\sigma_{\A}(\G_{+})\right] 
   =\sigma_\A(\G)-\frac{k}{d}\tr{\A^{-1}\G}+k\\
   &=\sigma_\A(\G)-\frac{k}{d}\tr{\A^{-1}(\G-\A)}
   =\left(1-\frac{k}{d}\right)\sigma_\A(\G).
\end{align*}
\end{proof}
For given matrices $\mG,\mA,\mL,\mU\in\BR^{d\times d}$ satisfying the condition of Theorem \ref{thm:bfgs-1}, we can access $\LL_+\in\BR^{d\times d}$ such that $\G_{+}^{-1}=\LL
_{+}^\top\LL_{+}$ by a closed form expression~\cite[Section 9]{gower2017randomized}.
% \textcolor{blue}{Liu: shall we add some discussion with the results in \cite{gower2017randomized}, where they achieves a result of $\OM(1-\frac{1}{d})$ for approximating the inverse of a matrix by applying such strategy?}

\begin{proposition}
\label{prop:efficientL}
Under the setting of Theorem \ref{thm:bfgs-1}, the matrix 
\begin{align*}\begin{split}
    &\LL_{+} = {\text{\rm UpdateL}}(\LL,\A,\U)\\
    & \triangleq\LL +\big(\U\big(\U^{\top}\U\big)^{-1/2}- \LL\A\LL^{\top}\U\big(\U^{\top}\LL\A\LL^{\top}\U\big)^{-1/2}\big)\big(\U^{\top}\LL\A\LL^{\top}\U\big)^{-1/2}\U^{\top}\LL,
\end{split}
\end{align*}
holds that $\G_{+}^{-1}=\LL_{+}^{\top}\LL_+$.
% The line 9 of Algorithm~\ref{alg:fasterbfgs} can be done by taking $\LL$
\end{proposition}
We can achieve Proposition \ref{prop:efficientL} by applying equation (9.11) of  Gower et al. \cite{gower2017randomized} with~$\tilde{\S}_k=\U$ and $\LL_k=\LL^{\top}$, 
where $\tilde{\S}_k$ and $\mL_k$ follow notations of Gower et al. \cite{gower2017randomized}.
This result means we can access the scaling matrix $\mL_+$ efficiently during iterations.

Based on the BFGS update with scaling directions shown in Theorem \ref{thm:bfgs-1} and the efficient update rule shown in Proposition~\ref{prop:efficientL}, we propose the faster randomized block BFGS method in Algorithm~\ref{alg:fasterbfgs},
which has the following local convergence rate matching \srk~methods (see Appendix~\ref{app:fasterBFGSproof} for the detailed proof).
%
\section{Extension for Solving Nonlinear Equations}
\begin{algorithm}[t]
\caption{Symmetric Rank-$k$ Method for Nonlinear Equation}\label{alg:SRK-NE}
\begin{algorithmic}[1]
\STATE \textbf{Input:} $\H_0$, $M$ and $k$. \\[0.15cm]
\STATE \textbf{for} $t=0,1\dots$\\[0.15cm]
\STATE \quad  $\z_{t+1}=\z_t-\H_t^{-1}\J(\z_t)^{\top}  \F(\z_t)$ \\[0.15cm]
\STATE \quad $r_t=\|\z_{t+1}-\z_{t}\|_2$ \\[0.15cm]
\STATE \quad $\tilde{\H}_{t}=(1+Mr_t)\H_t$ \\[0.1cm]
\STATE \quad construct $\U_t$ by $\left[\U_{t}\right]_{ij}\overset{\rm{i.i.d}}{\sim} {\fN(0,1)}$ \\[0.15cm]
\STATE \quad  $\H_{t+1}= \srk(\tilde{\H}_t, \J(\z_{t+1})^\top\J(\z_{t+1}),\U_t)$\\[0.15cm]
\STATE \textbf{end for}
\end{algorithmic}
\end{algorithm}

In this section, we apply \srk~methods to solve the nonlinear equations
\begin{align}
\label{eq:NE}
    \F(\z) =\0,
\end{align}
where $\F:\RB^{d}\to\RB^{d}$ is a differentiable vector-valued function. We use $\J(\vz)$ to represent the Jacobian of $\F(\cdot)$ at $\z\in\RB^d$ and impose the following assumptions.

\begin{assumption}
\label{ass:NElip}
We assume the vector-valued function
$F:\RB^{d}\to\RB^d$ is differentiable and its Jacobian is $\tilde L_2$-Lipschitz continuous, i.e., there exists some $\tilde L_2\geq 0$ such that 
\begin{align}
    \|\J(\z)-\J(\z')\|\leq \tilde L_2\|\z-\z'\|.
\end{align}
for any $\vz,\vz'\in\BR^d$.
\end{assumption}

\begin{assumption}
\label{ass:nonde}
We assume there exists equation (\ref{eq:NE}) has a solution $\z^*$ such that $\J(\z^*)$ is non-degenerate.
\end{assumption}

According to Assumption~\ref{ass:nonde}, we denote
\begin{align*}
    \tilde\mu \triangleq \frac{\sigma_{\min}(\J(\z^*))}{\sqrt{2}},\qquad \tilde L \triangleq 2\sigma_{\max}(\J(\z^*)) \qquad \text{and} \qquad {\tilde\kappa}\triangleq \frac{\tilde L}{\tilde \mu},
\end{align*}
where $\sigma_{\min}(\cdot)$ and $\sigma_{\max}(\cdot)$ are the smallest and the largest singular values of given matrix respectively. 

We present \srk~methods for solving nonlinear equations in Algorithm~\ref{alg:SRK-NE}.
The design of this algorithm is inspired by the recent work of~\citet{liu2022quasi}, which applies the quasi-Newton methods to estimate the information of non-degenerate indefinite matrix by its square.
We use the Euclidean norm $\tilde\lambda(\z)\triangleq\norm{\F(\z)}$ to measure the convergence of our algorithm. 
The advantage of block updates in \srk~updates results a faster superlinear convergence than~\citet{liu2022quasi}'s methods.
Following the analysis of \srk~methods for convex optimization, we obtain the results for solving nonlinear equations as follows.


\begin{theorem}
\label{thm:srk-NE}
Under {Assumption \ref{ass:NElip} and \ref{ass:nonde}}, we run Algorithm~\ref{alg:SRK-NE} with $k<d$, $\tilde M={2\tilde{\varkappa}^2 \tilde 
 L_2}/{\tilde L}$ and set the initial~$\vz_0$ and $\H_0$ such that
\begin{align*}
    \tilde \lambda(\z_0)\leq \frac{\ln 2}{8}\cdot \frac{ (d-k)\tilde{\mu}}{\tilde  M\eta_0 d^2{\tilde \varkappa}^2} 
    \qquad \text{and} \qquad 
    \J(\z_0)^{\top}\J(\z_0)\preceq\H_0\preceq \eta_0\J(\z_0)^{\top}\J(\z_0)
\end{align*} 
for some $\eta_0\geq 1$. 
Then we have
\begin{align*}
     \EBP{\frac{\tilde{\lambda}({\z}_{t+1})}{\tilde{\lambda}(\z_t)}}\leq 2d\tilde{\varkappa}^2\eta_0\left(1-\frac{k}{d}\right)^{t}.
\end{align*}

\end{theorem}
\begin{theorem}
\label{thm:fasterBFGS}
Under Assumption~\ref{ass:smooth}, \ref{ass:strongconvex}, and \ref{ass:strongself}, we run faster randomized block BFGS method (Algorithm~\ref{alg:fasterbfgs}) with $\vx_0\in\BR^d$ and $\mL_0,\mG_0\in\BR^{d\times d}$ such that
\begin{align}
\label{eq:fasterbfgsini}
M\lambda_0 \leq \frac{\ln 2}{4}\cdot \frac{d-k}{\eta_0 d^2},  \quad
\nabla^2 f(\x_0)\preceq\G_0\preceq \eta_0\nabla^2f(\x_0), \quad \text{and} \quad \G_0^{-1}=\LL_0^{\top}\LL_0
\end{align}
for some $k<d$ and $\eta_0\geq 1$. 
Then we have %$    \EBP{\frac{\lambda(\x_{t+1})}{\lambda(\x_t)}}\leq 2d\eta_0\left(1-\frac{k}{d}\right)^{t}.$
\begin{align*}
    \BE\left[\frac{\lambda(\x_{t+1})}{\lambda(\x_t)}\right]\leq 2d\eta_0\left(1-\frac{k}{d}\right)^{t}.
\end{align*}
\end{theorem}

\begin{remark}
The condition on $\mG_0\in\BR^{d \times d}$ in Theorem~\ref{thm:BFGS} and \ref{thm:fasterBFGS} can be satisfied by simply setting $\mG_0=L\mI_d$. 
In addition, we can set $\mL_0=L^{-1/2}\mI_d$ to satisfy the condition on $\mL_0\in\BR^{d \times d}$ in Theorem \ref{thm:fasterBFGS}, then we can efficiently implement the update on $\LL_t$ by Proposition \ref{prop:efficientL}.
\end{remark}


\begin{remark}
For $k=1$, the convergence rates provided by Theorem~\ref{thm:BFGS}~and~\ref{thm:fasterBFGS} match the results of the ordinary randomized BFGS/DFP methods~\cite{rodomanov2021greedy,lin2021greedy} and fast randomized BFGS methods~\cite{lin2021greedy} respectively. 
For $k=d$, the updates \eqref{eq:bfgsupdate} and \eqref{eq:bfgsupdate-1} holds $\mG_+=\mA$ almost surely. This leads to the local quadratic convergence rate of our (faster) randomized block BFGS/DFP methods, which is similar to the behaviors of \srk~methods shown in Corollary \ref{cor:recoverNewton}. 
\end{remark}

\begin{algorithm}[t]
\caption{Faster Randomized Block BFGS}\label{alg:fasterbfgs}
\begin{algorithmic}[1]
\STATE \textbf{Input:} $\x_0$, $\G_0$, $\mL_0$, $M$, and $k$ \\
\STATE \textbf{for} $t=0,1\dots$\\
\STATE \quad $\x_{t+1}=\x_t-\G_t^{-1}\nabla f(\x_t)$ \\
\STATE \quad $r_t=\|\x_{t+1}-\x_{t}\|_{\x_t}$ \\[0.05cm]
\STATE \quad $\tilde{\G}_{t}=(1+Mr_t)\G_t$ \\[0.05cm]
\STATE \quad $\tilde{\LL}_{t}=\LL_t/\sqrt{1+Mr_t}$ \\
\STATE \quad Construct $\U_t$ by $\left[\U_{t}\right]_{ij}\overset{\rm{i.i.d}}{\sim} {\fN(0,1)}$ \\[0.05cm]
\STATE \quad $\G_{t+1}= \bfgs(\tilde{\G}_t,\nabla^2f(\x_{t+1}),\tilde{\LL}_{t}^{\top}\U_t)$ \label{line:fastBFGS} \\[0.05cm]
\STATE \quad ${\LL}_{t+1} = {\text{\rm UpdateL}}(\tilde{\LL}_t,\nabla^2 f(\x_{t+1}), \U_t)$
\STATE \textbf{end for}
\end{algorithmic}
\end{algorithm} 



% \begin{remark}
% % For $k=1$, the convergence rates provided by Theorem \ref{thm:bfgs} matches the results for ordinary randomized BFGS and DFP updates~\cite{lin2021greedy}. 
% % On the other hand, Gower and Richt{\'a}rik~\cite{gower2017randomized} established the convergence of randomized block BFGS and DFP updates with respect to the measure of Frobenius norm, while rates cannot be sharper even if the value of $k$ is increased.   
% % \end{remark}


% \begin{remark}
% For $k=1$, Theorem \ref{thm:BFGS} matches the results of ordinary randomized BFGS and DFP methods~\cite{lin2021greedy}. For~$k=d$, we have $\G_{+}\U=\A\U$ for both $\G_{+}={ \text{\rm BlockBFGS}}(\G,\A,\U)$ and $\G_{+}={ \text{\rm BlockDFP}}(\G,\A,\U)$, which means the output of block BFGS and block DFP update are identical to the target matrix almost surely and Algorithm~\ref{alg:bfgs} can achieve the local quadratic convergence rate like standard Newton method.
% \end{remark}

% 
\section{Extension for Solving Nonlinear Equations}
\begin{algorithm}[t]
\caption{Symmetric Rank-$k$ Method for Nonlinear Equation}\label{alg:SRK-NE}
\begin{algorithmic}[1]
\STATE \textbf{Input:} $\H_0$, $M$ and $k$. \\[0.15cm]
\STATE \textbf{for} $t=0,1\dots$\\[0.15cm]
\STATE \quad  $\z_{t+1}=\z_t-\H_t^{-1}\J(\z_t)^{\top}  \F(\z_t)$ \\[0.15cm]
\STATE \quad $r_t=\|\z_{t+1}-\z_{t}\|_2$ \\[0.15cm]
\STATE \quad $\tilde{\H}_{t}=(1+Mr_t)\H_t$ \\[0.1cm]
\STATE \quad construct $\U_t$ by $\left[\U_{t}\right]_{ij}\overset{\rm{i.i.d}}{\sim} {\fN(0,1)}$ \\[0.15cm]
\STATE \quad  $\H_{t+1}= \srk(\tilde{\H}_t, \J(\z_{t+1})^\top\J(\z_{t+1}),\U_t)$\\[0.15cm]
\STATE \textbf{end for}
\end{algorithmic}
\end{algorithm}

In this section, we apply \srk~methods to solve the nonlinear equations
\begin{align}
\label{eq:NE}
    \F(\z) =\0,
\end{align}
where $\F:\RB^{d}\to\RB^{d}$ is a differentiable vector-valued function. We use $\J(\vz)$ to represent the Jacobian of $\F(\cdot)$ at $\z\in\RB^d$ and impose the following assumptions.

\begin{assumption}
\label{ass:NElip}
We assume the vector-valued function
$F:\RB^{d}\to\RB^d$ is differentiable and its Jacobian is $\tilde L_2$-Lipschitz continuous, i.e., there exists some $\tilde L_2\geq 0$ such that 
\begin{align}
    \|\J(\z)-\J(\z')\|\leq \tilde L_2\|\z-\z'\|.
\end{align}
for any $\vz,\vz'\in\BR^d$.
\end{assumption}

\begin{assumption}
\label{ass:nonde}
We assume there exists equation (\ref{eq:NE}) has a solution $\z^*$ such that $\J(\z^*)$ is non-degenerate.
\end{assumption}

According to Assumption~\ref{ass:nonde}, we denote
\begin{align*}
    \tilde\mu \triangleq \frac{\sigma_{\min}(\J(\z^*))}{\sqrt{2}},\qquad \tilde L \triangleq 2\sigma_{\max}(\J(\z^*)) \qquad \text{and} \qquad {\tilde\kappa}\triangleq \frac{\tilde L}{\tilde \mu},
\end{align*}
where $\sigma_{\min}(\cdot)$ and $\sigma_{\max}(\cdot)$ are the smallest and the largest singular values of given matrix respectively. 

We present \srk~methods for solving nonlinear equations in Algorithm~\ref{alg:SRK-NE}.
The design of this algorithm is inspired by the recent work of~\citet{liu2022quasi}, which applies the quasi-Newton methods to estimate the information of non-degenerate indefinite matrix by its square.
We use the Euclidean norm $\tilde\lambda(\z)\triangleq\norm{\F(\z)}$ to measure the convergence of our algorithm. 
The advantage of block updates in \srk~updates results a faster superlinear convergence than~\citet{liu2022quasi}'s methods.
Following the analysis of \srk~methods for convex optimization, we obtain the results for solving nonlinear equations as follows.


\begin{theorem}
\label{thm:srk-NE}
Under {Assumption \ref{ass:NElip} and \ref{ass:nonde}}, we run Algorithm~\ref{alg:SRK-NE} with $k<d$, $\tilde M={2\tilde{\varkappa}^2 \tilde 
 L_2}/{\tilde L}$ and set the initial~$\vz_0$ and $\H_0$ such that
\begin{align*}
    \tilde \lambda(\z_0)\leq \frac{\ln 2}{8}\cdot \frac{ (d-k)\tilde{\mu}}{\tilde  M\eta_0 d^2{\tilde \varkappa}^2} 
    \qquad \text{and} \qquad 
    \J(\z_0)^{\top}\J(\z_0)\preceq\H_0\preceq \eta_0\J(\z_0)^{\top}\J(\z_0)
\end{align*} 
for some $\eta_0\geq 1$. 
Then we have
\begin{align*}
     \EBP{\frac{\tilde{\lambda}({\z}_{t+1})}{\tilde{\lambda}(\z_t)}}\leq 2d\tilde{\varkappa}^2\eta_0\left(1-\frac{k}{d}\right)^{t}.
\end{align*}

\end{theorem}

\section{Numerical Experiments}\label{sec:exp}



% \begin{figure}[t]
% \centering
% \begin{tabular}{cccc}
% \includegraphics[scale=0.28]{Graph_grsrk/MNIST.res.pdf} &
% \includegraphics[scale=0.28]{Graph_grsrk/sido0.res.pdf} &
% \includegraphics[scale=0.28]{Graph_grsrk/gisette.res.pdf}
% \\[-0.1cm]
% \small (a) MNIST (iteration) & \small  (b) sido0 (iteration) &\small (c) gisette (iteration) \\[0.2cm]
% \includegraphics[scale=0.28]{Graph_grsrk/MNIST.time.pdf} & 
% \includegraphics[scale=0.28]{Graph_grsrk/sido0.time.pdf} &
% \includegraphics[scale=0.28]{Graph_grsrk/gisette.time.pdf}
% \\[-0.1cm]
% \small  (d) MNIST (time) & \small  (e) sido0 (time) &\small (f) gisette (time)
% \end{tabular}\vskip-0.15cm
% \caption{We demonstrate ``\#iteration vs. $\|\nabla f(\x)\|$'' and ``running time (s) vs. $\|\nabla f(\x)\|$'' on datasets ``MNIST'', ``sido0'', and ``gisette'' with different $k=\{1,80,200,500,1000\}$ for Gr\srk.}\label{fig:diff_k_gr}
% \end{figure}
We conduct the experiments on the model of regularized logistic regression, which can be formulated as 
\begin{align}
\label{prob:logstic}
    \min_{\vx\in\BR^d} f(\x)\triangleq \frac{1}{n}\sum_{i=1}^{n}\ln\left(1+\exp{-b_i\va_i^{\top}\x}\right)+\frac{\gamma}{2}\|\x\|^2,
\end{align}
where $\va_i\in\RB^d$ and $b_i\in\{-1,+1\}$ are the feature and the corresponding label of the $i$-th sample respectively, and $\gamma>0$ is the regularization hyperparameter.

We refer to \srk~methods with randomized/greedy strategies (Algorithm~\ref{alg:SRK})~as R-\srk/G-\srk.
The SR1 methods with randomized/greedy strategies are referred as R-SR1/G-SR1 \cite[Algorithm 4]{lin2021greedy}.
We refer to the existing randomized block BFGS method \cite[Algorithm 1]{kovalev2020fast} as RB-BFGSv1 and our randomized block BFGS/DFP methods (Algorithm~\ref{alg:bfgs}) as RB-BFGSv2/RB-DFP.
We denote our faster block BFGS method (Algorithm~\ref{alg:fasterbfgs}) as
FRB-BFGS.
We compare the proposed R-\srk, G-\srk, RB-BFGSv2, RB-DFP, and FRB-BFGS with baselines on problem~(\ref{prob:logstic}).
We do not include the empirical results of classical SR1/BFGS/DFP methods since G-SR1 has the competitive performance with them \cite{rodomanov2021greedy}.
For all methods, we tune $\G_0$ and $M$ from~$\{\I_d, 10\I_d, 10^2\I_d, 10^3\I_d, 10^4\I_d\}$ and $\{1,10,10^2,10^3,10^4\}$ respectively.
We evaluate the performance on datasets ``MNIST'' ($d= 780$), ``sido0'' ($d=4,932$), and ``gisette'' ($d=5,000$). 
We conduct experiments by Python 3.8.12 on a PC with Apple M1.

We present the results of ``iteration numbers vs. gradient norm'' and ``running time (second) vs. gradient norm'' in Figure \ref{fig:experiment-10}, where we take $k=200$ for block quasi-Newton methods (R-\srk, G-\srk, RB-BFGSv1, RB-BFGSv2, and FRB-BFGS).
We observe the proposed R-\srk~and G-\srk~significantly outperform other methods.
The proposed block BFGS methods (RB-BFGSv2 and FRB-BFGS) outperform the block DFP method (RB-DFP), which is similar to the advantage of the BFGS method over the DFP method~\cite{rodomanov2021greedy}.
% Besides, our FRB-BFGS performs better than RB-BFGSv2, which is consistent to our theoretical analysis in Section~\ref{sec:Block BFGS}.

%which is similar to the advantage of the greedy BFGS method over the greedy DFP method~\cite{rodomanov2021greedy}.

We also demonstrate the impact of parameter $k$ for  \srk~methods. 
It is natural that Figure~\ref{fig:diff_k_ra}(a), 2(b), and \ref{fig:diff_k_ra}(c) show the larger $k$ leads to faster convergence in terms of iteration numbers, which validates our theoretical analysis in Section \ref{sec:srk-opt}.
In addition, Figure~\ref{fig:diff_k_ra}(d), 2(e), and \ref{fig:diff_k_ra}(f) show SR-$k$ methods with $k>1$ significantly outperform SR$1$ in terms of the running time. 
This is because the block update can reduce cache miss rate and take the advantage of parallel computing.  
However, increasing~$k$ does not always result less running time because the acceleration caused by the block update is limited by the cache size. 
It seems difficult to find a universal way to select the best $k$, because it depends on the the specific experimental hardware platform.


\begin{figure}[t]
\centering
\begin{tabular}{cccc}
\includegraphics[scale=0.28]{Graph/MNIST.res.pdf} &
\includegraphics[scale=0.28]{Graph/sido0.res.pdf} &
\includegraphics[scale=0.28]{Graph/gisette.res.pdf}
\\[-0.2cm]
\footnotesize~~~~~ (a) MNIST (iteration) & \footnotesize~~~~~  (b) sido0 (iteration) & \footnotesize~~~~~  (c) gisette (iteration) \\[0.15cm]
\includegraphics[scale=0.28]{Graph/MNIST.time.pdf} & 
\includegraphics[scale=0.28]{Graph/sido0.time.pdf} &
\includegraphics[scale=0.28]{Graph/gisette.time.pdf}
\\[-0.15cm]
\footnotesize~~~~~   (d) MNIST (time) & \footnotesize~~~~~  (e) sido0 (time) & \footnotesize~~~~~  (f) gisette (time)
\end{tabular}\vskip-0.15cm
\caption{We show ``\#iteration vs. $\|\nabla f(\x)\|$'' and ``running time (s) vs. $\|\nabla f(\x)\|$'' on datasets ``MNIST'', ``sido0'', and ``gisette'', where  we take $k=200$ for all of block quasi-Newton methods.}\label{fig:experiment-10} \vskip-0.4cm
\end{figure}
\begin{figure}[t]
\centering
\begin{tabular}{cccc}
\includegraphics[scale=0.28]{Graph_rasrk/MNIST.res.pdf} &
\includegraphics[scale=0.28]{Graph_rasrk/sido0.res.pdf} &
\includegraphics[scale=0.28]{Graph_rasrk/gisette.res.pdf}
\\[-0.2cm]
\footnotesize~~~~~ (a) MNIST (iteration) & \footnotesize~~~~~   (b) sido0 (iteration) &\small (c) gisette (iteration) \\[0.15cm]
\includegraphics[scale=0.28]{Graph_rasrk/MNIST.time.pdf} & 
\includegraphics[scale=0.28]{Graph_rasrk/sido0.time.pdf} &
\includegraphics[scale=0.28]{Graph_rasrk/gisette.time.pdf}
\\[-0.2cm]
\footnotesize~~~~~   (d) MNIST (time) & \footnotesize~~~~~   (e) sido0 (time) & \footnotesize~~~~~  (f) gisette (time)
\end{tabular}\vskip-0.15cm
\caption{We show ``\#iteration vs. $\|\nabla f(\x)\|$'' and ``running time (s) vs. $\|\nabla f(\x)\|$'' for proposed R-\srk~and G-\srk~with different $k\in\{1,80,200,500\}$ on datasets ``MNIST'', and $k\in\{1,80,200,500,1000\}$ on datasets ``sido0'' and ``gisette''.}\label{fig:diff_k_ra} \vskip-0.5cm
\end{figure}
% In our experimental environment and candidates of $k$, the Ra\srk~method with $k=200$ has the best performance for ``MNIST'' and ``sido0'' (Figure~\ref{fig:diff_k_ra}(d), \ref{fig:diff_k_ra}(e)) and with $k=500$ has the best performance for ``gisette'' (Figure~\ref{fig:diff_k_ra}(f)); the Gr\srk~method with $k=200$ has the best performance for ``MNIST'' (Figure~\ref{fig:diff_k_gr}(d)) and with $k=500$ has the best performance  for ``sido0'' and ``gisette'' (Figure~\ref{fig:diff_k_gr}(e), \ref{fig:diff_k_gr}(f)).
% \textcolor{blue}{Some empirical findings on how to choose a proper $k$?}
\section{Conclusion}
\label{sec:conclusion}
In this paper, we have proposed rank-$k$ (\srk) methods for convex optimization.
We have proved \srk~methods enjoy the explicit local superlinear convergence rate of $\OM\left((1-k/d)^{t(t-1)/2}\right)$.
Our result successfully reveals the advantage of block-type updates in quasi-Newton methods, building a bridge between the theories of ordinary quasi-Newton methods and standard Newton method.
We also provide the convergence rate of $\OM\left((1-k/(\varkappa d))^{t(t-1)/2}\right)$ for randomized block BFGS/DFP methods and design the faster randomized block BFGS method to match the rate of \srk~methods.
In future work, 
it would be interesting to study the global convergence of block quasi-Newton methods ~\cite{jiang2023online,jiang2023accquasi,rodomanov2024global,jin2024nona,jin2024nonb} and study their limited memory~\cite{berahas2022limited,liu1989limited,gao2023limited} as well as stochastic variants~\cite{mokhtari2018iqn,wang2017stochastic,wang2019stochastic,yang2022stochastic}.
% 
% 
 


\appendix

% \begin{align*}
%    \G_{+}= \G-\underbrace{(\G-\A)\U}_{\P}\underbrace{\left(\U^{\top}(\G-\A)\U\right)^{\dag}}_{\C}\underbrace{\U^{\top}(\G-\A)}_{\P^{\top}},
% \end{align*}
% which implies that
% \begin{align*}
%     \G_{+}^{-1} = \G^{-1}-\G^{-1}\P(\C+\P^{\top}\G\P)^{-1}\P^{\top}\G^{-1}
% \end{align*}
% \section{Auxiliary Lemmas}
% \label{sec:auli}
% \begin{lemma}
% \label{lm:trmulti}
% For any positive semi-definite matrices $\B,\P_1,\P_2\in\BR^{d\times d}$ such that $\P_1\succeq \P_2$, we have
% \begin{align}
%     \tr{\P_1\B}\geq\tr{\P_2\B}.
% \end{align}
% \end{lemma}
% \begin{proof}
% $\P_1-\P_2\succeq\0$ means
% \begin{align*}
%    (\P_1-\P_2)^{1/2}\B (\P_1-\P_2)^{1/2}\succeq\0,
% \end{align*}
% which implies 
% \begin{align*}
%  & \tr{\P_1\B}-\tr{\P_2\B} \\
%  = & \tr{(\P_1-\P_2)\B}\\
% =&\tr{(\P_1-\P_2)^{1/2}\B(\P_1-\P_2)^{1/2}}\\
% \geq & 0.
% \end{align*}
% % means 
% % \begin{align*}
% %     \tr{(\P_1-\P_2)^{1/2}\B(\P_1-\P_2)^{1/2}}\geq 0.
% % \end{align*}
% % So we have
% % \begin{align*}
% %  & \tr{\P_1\B}-\tr{\P_2\B} \\
% %  = & \tr{(\P_1-\P_2)\B}\\
% % =&\tr{(\P_1-\P_2)^{1/2}\B(\P_1-\P_2)^{1/2}}\\
% % \geq & 0.
% % \end{align*}
% 
% \end{proof}

% \begin{lemma}
% For positive semi-definite matrix $\S\in\RB^{d\times d}$ and the column orthonormal matrix $\Q\in\RB^{d\times k}$, we have
% \begin{align}
%     \label{eq:trace}
%     \tr{\Q^{\top}\S\Q}\leq \tr{\S}.
% \end{align}
% \end{lemma}
% \begin{proof}
% Since matrix $\Q$ is column orthonormal, we have
% \begin{align*}
%     \Q\Q^{\top}=\Q(\Q^{\top}\Q)^{-1}\Q^{\top}\preceq\I_d.
% \end{align*}
% According to Lemma~\ref{lm:trmulti}, we have
% \begin{align*}
%       \tr{\Q^{\top}\S\Q}=\tr{\S\Q\Q^{\top}}\leq \tr{\S}.
% \end{align*}
% 
% \end{proof}


% \begin{lemma}
% \label{lm:EP}
% Let $\U\in\RB^{d\times k}$ be a random matrix and each of its entry is independent and identically distributed according to $\fN(0,1)$, then it holds that
% \begin{align*}
%    \EB\left[\U(\U^{\top}\U)^{-1}\U^{\top}\right] = \frac{k}{d}\I_d.
% \end{align*}
% \end{lemma}
% \begin{proof}
% We use $\fV_{d,k}$ to present the Stiefel manifold which is the set of all $d\times k$ column orthogonal matrices.
% We denote $\fP_{k,d-k}$ as the set of all $m\times m$ orthogonal projection matrices idempotent of rank $k$.

% According to Theorem 2.2.1 (iii) of \citet{chikuse2003statistics}, the random matrix
% \begin{align*}
%     \Z=\U(\U^{\top}\U)^{-1/2}
% \end{align*}
% is uniformly distributed on the Stiefel manifold $\fV_{d,k}$.
% Applying Theorem 2.2.2 (iii) of \citet{chikuse2003statistics}, the random matrix
% \begin{align*}
%     \P=\Z\Z^\top=\U(\U^{\top}\U)^{-1}\U^{\top} 
% \end{align*}
% is uniformly distributed on $\fP_{k,d-k}$. 
% Combining above results with Theorem 2.2.2 (i) of \citet{chikuse2003statistics} on $\mP$ achieves
% \begin{align*}
%     \EB[\P]= \frac{k}{d}\I_d.
% \end{align*}
% 
% \end{proof}
% \begin{remark}
% The above proof requires the knowledge for statistics on manifold. 
% For the readers who are not familiar with this, we also present a elementary proof of Lemma~\ref{lm:EP} by induction in Appendix~\ref{appen:addiproof}.
% \end{remark}

%{\color{blue} Luo has finished editing for above paragraphs.}


\section{An Elementary Proof of Lemma~\ref{lm:explicitbound}(a)}
\label{appen:addiproof}
Before proving Lemma \ref{lm:explicitbound}(a), we first provide the following lemma.
\begin{lemma}\label{lem:1d_gauss}
% Given column orthonormal matrix $\mP\in\BR^{d\times k}$  
% with $k\le d$ and $d$-dimensional random vector $\vv\sim\mathcal{N}_d(\vzero,\mP\mP^{\top})$, then we have $\BE\left[{\vv\vv^\top}/{(\vv^\top \vv)}\right]=\mP\mP^\top/k$.
Assume matrix $\mP\in\BR^{d\times r}$ is column orthonormal with $r\le d$ and the $d$-dimensional random vector $\vv$ is distributed according to $\mathcal{N}_d(\vzero,\mP\mP^{\top})$, then we have $\BE\left[{\vv\vv^\top}/{(\vv^\top \vv)}\right]=\mP\mP^\top/r$.
% \begin{align*}
% \BE\left[\frac{\vv\vv^\top}{\vv^\top \vv}\right]=\frac{1}{k}\mP\mP^\top. 
% \end{align*}
\end{lemma}
\begin{proof}
The distribution $\vv\sim\mathcal{N}_d(\vzero,\mP\mP^{\top})$ implies there exists a $r$-dimensional normal distributed random vector $\vw\sim\mathcal{N}_r(\vzero,\mI_r)$ such that $\vv=\mP\vw$. Thus, we have
\begin{align*}
\BE\left[\frac{\vv\vv^\top}{\vv^\top \vv}\right] 
= & \BE\left[\frac{(\mP\vw)(\mP\vw)^\top}{(\mP\vw)^\top (\mP\vw)}\right] 
=  \BE\left[\frac{\mP\vw\vw^\top\mP^\top}{\vw^\top\mP^\top\mP\vw}\right] 
=  \mP\BE\left[\frac{\vw\vw^\top}{\vw^\top\vw}\right]\mP^\top 
 = \frac{1}{r}\mP\mP^\top,
\end{align*}
where the last step is because $\vw/\norm{\vw}$ is uniform distributed on the $r$-dimensional unit sphere and its covariance matrix is $\mI_r/r$.
\end{proof}

Now we prove statement (a) of Lemma~\ref{lm:explicitbound}.
\begin{proof}    
We only need to prove equation \eqref{eq:EP} by using induction on $k$.
The induction base $k=1$ have been verified by Lemma~\ref{lem:1d_gauss}. 
Now we assume equation~\eqref{eq:EP}
% \begin{align*}
% \BE\left[\mU(\mU^\top\mU)^{-1}\mU^\top\right]=\frac{k}{d}\mI_d 
% \end{align*}
holds for any $\mU\in\BR^{d\times k}$ such that each of its entries are independently distributed according to $\fN(0,1)$. 
We define the random matrix 
\begin{align*}
\hat{\mU} = \begin{bmatrix}
\mU & \vv 
\end{bmatrix}\in\BR^{d\times(k+1)},
\end{align*}
where $\vv\sim\fN_d(\vzero,\mI_d)$ is independent distributed to $\mU$. We define $\B=\mU(\mU^\top\mU)^{-1}\mU^\top$, then we use block matrix inversion formula to rewrite $(\hat{\mU}^\top\hat{\mU})^{-1}$ as follows
% \begin{align*}
%   &(\hat{\mU}^\top\hat{\mU})^{-1} 
%   =   \left(\begin{bmatrix}\mU^\top \\ \vv^\top \end{bmatrix} \begin{bmatrix}\mU & \vv \end{bmatrix} \right)^{-1} = \left(\begin{bmatrix}
%       \U^{\top}\U&\U^{\top}\v\\
%       \v^{\top}\U&\v^{\top}\v
%   \end{bmatrix}\right)^{-1}\\
%   &=\begin{bmatrix}
%   \S  & -\dfrac{(\U^{\top}\U)^{-1}\U^{\top}\v}{\v^{\top}(\I-\B)\v}\\[0.2cm]
%    -\dfrac{\v^{\top}\U(\U^{\top}\U)^{-1}}{\v^{\top}(\I-\B)\v}  &~ \dfrac{1}{\v^{\top}(\I-\B)\v}
%   \end{bmatrix},
% \end{align*}
\begin{align*}
  &(\hat{\mU}^\top\hat{\mU})^{-1} 
  =   \left(\begin{bmatrix}\mU^\top \\ \vv^\top \end{bmatrix} \begin{bmatrix}\mU & \vv \end{bmatrix} \right)^{-1} = \left(\begin{bmatrix}
      \U^{\top}\U&\U^{\top}\v\\
      \v^{\top}\U&\v^{\top}\v
  \end{bmatrix}\right)^{-1}\\[0.15cm]
  &=\begin{bmatrix}
  \S  & -(\U^{\top}\U)^{-1}\U^{\top}\v(\v^{\top}(\I_d-\B)\v)^{-1}\\
   -(\v^{\top}(\I_d-\B)\v)^{-1}\v^{\top}\U(\U^{\top}\U)^{-1}  &~ (\v^{\top}(\I_d-\B)\v)^{-1}
  \end{bmatrix},
\end{align*}
where $\S= (\U^{\top}\U)^{-1}+(\U^{\top}\U)^{-1}\U^{\top}\v(\v^{\top}(\I_d-\B)\v)^{-1}\v^{\top}\U(\U^{\top}\U)^{-1}$.
Thus, we have
\begin{align*}
     & \hat{\mU}(\hat{\mU}^\top\hat{\mU})^{-1}\hat{\mU}^\top
     = \begin{bmatrix}\mU & \vv \end{bmatrix}\left(\begin{bmatrix}\mU^\top \\ \vv^\top \end{bmatrix} \begin{bmatrix}\mU & \vv \end{bmatrix} \right)^{-1} \begin{bmatrix}\mU^\top \\ \vv^\top \end{bmatrix} \\
     &= \B+\frac{\B\v\v^{\top}\B}{\v^{\top}(\I_d-\B)\v}-\frac{\B\v\v^{\top}}{\v^{\top}(\I_d-\B)\v}-\frac{\v\v^{\top}\B}{\v^{\top}(\I_d-\B)\v}+\frac{\v\v^{\top}}{\v^{\top}(\I_d-\B)\v}\\
     &=\B+\frac{(\mI_d-\B)\vv\vv^\top(\mI_d-\B)}{\vv^\top(\mI_d-\B)\vv}.
\end{align*}
% and compute $\hat{\mU}(\hat{\mU}^\top\hat{\mU})^{-1}\hat{\mU}^\top$ by using block matrix inversion formula and Woodbury matrix identity:
% \begin{align*}
%     & \hat{\mU}(\hat{\mU}^\top\hat{\mU})^{-1}\hat{\mU}^\top \\
%     &=  \begin{bmatrix}\mU & \vv \end{bmatrix}\left(\begin{bmatrix}\mU^\top \\ \vv^\top \end{bmatrix} \begin{bmatrix}\mU & \vv \end{bmatrix} \right)^{-1} \begin{bmatrix}\mU^\top \\ \vv^\top \end{bmatrix}
%     &= 
%     %\\
%    % = & \mA+\frac{(\mI_d-\mA)\vv\vv^\top(\mI_d-\mA)}{\vv^\top(\mI_d-\mA)\vv},
% \end{align*}
% where $\mA=\mU(\mU^\top\mU)^{-1}\mU^\top$.
Since the rank of the projection matrix $\mI_d-\B$ is $d-k$, we can write  $\mI_d-\B=\mQ\mQ^\top$, where $\mQ\in\BR^{d\times (d-k)}$ is column orthonormal. 
Thus, we achieve
\begin{align*}
 & \EBP{\hat{\mU}(\hat{\mU}^\top\hat{\mU})
 \hat{\mU}^\top}
=\frac{k}{d}\mI_d +\BE_{\mU}\left[\BE_{\vv}\left[\frac{(\mI_d-\B)\vv\vv^\top(\mI_d-\B)}{\vv^\top(\mI_d-\B)\vv}\,\Big|\,\mU\right]\right]\\
&=\frac{k}{d}\mI_d +\BE_{\mU}\left[\BE_{\vv}\left[\frac{(\mQ\mQ^\top\vv)(\vv^\top\mQ\mQ^\top)}{(\vv^\top\mQ\mQ^\top)(\mQ\mQ^\top\vv)}\,\Big|\,\mU\right]\right]
=\frac{k}{d}\mI_d +\frac{1}{d-k}\BE_{\mU}[\mQ\mQ^\top] \\
&=\frac{k}{d}\mI_d +\frac{1}{d-k}\BE_{\mU}[\mI_d-\B] 
=\frac{k}{d}\mI_d +\frac{1}{d-k}\left(\mI_d-\frac{k}{d}\mI_d\right) 
=\frac{k+1}{d}\mI_d,
\end{align*}
which completes the induction. 
In above derivation, the first and the fifth equalities come from the inductive hypothesis; third equality is achieved by applying Lemma~\ref{lem:1d_gauss} with $\mP=\mQ\mQ^\top$ and $r=d-k$, and the fact~$\mQ\mQ^{\top}\vv\sim\mathcal{N}_d(\vzero,\mQ\mQ^{\top})$.
\end{proof}


\section{Auxiliary Lemmas for Non-negative Sequences}
The following lemmas on non-negative sequences
%which will be useful to obtain the explicit convergence rates of the proposed block quasi-Newton methods.
extend Theorem 23 of Lin et al.~\cite{lin2021greedy} and Theorem 4.7 of Rodomanov and Nesterov \cite{rodomanov2021greedy}, leading to a general framework for the analysis of (block) quasi-Newton methods.

\begin{lemma}
\label{lm:superlinear}
Let $\{\lambda_t\}$ and $\{\delta_t\}$ be two non-negative random sequences that satisfy
\begin{align}
\label{eq:supercondi}
  &\BE_{t}\left[\delta_{t+1}\right]\leq \left(1-\frac{1}{\alpha}\right)(1+c_1\lambda_t)^2(\delta_t+c_2\lambda_t), ~~\lambda_{t+1}\leq (1+c_1\lambda_t)^2(\delta_t+c_3\lambda_t)\lambda_t,\\
\label{eq:supercondi_2}
 &\delta_0+c\lambda_0 \leq s,~~~\text{and}~~~ \lambda_{t}\leq \left(1-\frac{1}{\beta}\right)^t\lambda_0,
\end{align}
for some $c_1, c_2, c_3\geq 0$ with $c=\max\{c_2,c_3\}$, $s\geq 0$,  $\alpha>1$, and $\beta>1$, where $\EB_{t}[\,\cdot\,]\triangleq \EB[\,\cdot\,|\,\delta_0,\cdots,\delta_{t},\lambda_0,\cdots,\lambda_{t}]$. 
If $\lambda_0>0$ is sufficient small such that 
\begin{align}
\label{eq:supercondiini}
    \lambda_0\leq \frac{\ln 2}{\beta (2c_1+c(\alpha/(\alpha-1)))},
\end{align}
we have $\BE\left[{\lambda_{t+1}}/{\lambda_t}\right]\leq 2s\left(1-{1}/{\alpha}\right)^{t}.$
% \begin{align*}
%     \BE\left[\frac{\lambda_{t+1}}{\lambda_t}\right]\leq 2\left(1-\frac{1}{\alpha}\right)^{t}s.
% \end{align*}
\end{lemma}

\begin{proof}
We denote $  \theta_t\triangleq \delta_t+c \lambda_t.$
% \begin{align}
% \label{eq:thetadef}
%     \theta_t\triangleq \delta_t+c \lambda_t.
% \end{align}
Noticing that we have $\exp{x}\geq 1+x$ for all $x\geq 0$. Applying this fact with $x=c_1\lambda_t$ and the definition $c=\max\{c_2,c_3\}$, we have
\begin{align}
\label{eq:lambdat_leq_thetat}
  &\EB_{t}[\delta_{t+1}] \overset{\eqref{eq:supercondi}}{\leq}\left(1-\frac{1}{\alpha}\right)(\delta_t+c_2\lambda_t)(1+c_1\lambda_t)^2\leq \left(1-\frac{1}{\alpha}\right)\theta_t\exp{2c_1\lambda_t}\\
\label{eq:lambda_t_succ}
    &\text{and} \quad \lambda_{t+1} \overset{\eqref{eq:supercondi}}{\leq} (\delta_t+c_3\lambda_t)(1+c_1\lambda_t)^2\lambda_t \leq \theta_t\lambda_t\exp{2c_1\lambda_t}.
\end{align}
We denote $\tilde{c}\triangleq2c_1+c\alpha/(\alpha-1)$, then it holds that
\begin{align}
\label{eq:thetat+1leqs}
\begin{split}
    &\EB_{t}[\theta_{t+1}]= \BE_t[\delta_{t+1}+c \lambda_{t+1}] \\
    &\overset{\eqref{eq:lambdat_leq_thetat},\eqref{eq:lambda_t_succ}}{\leq} 
    \left(1-\frac{1}{\alpha}\right)\theta_t\exp{2c_1\lambda_t} + c\theta_t\lambda_t\exp{2c_1\lambda_t}\\
    &~~~~~=\left(1-\frac{1}{\alpha}\right)\left(1+\frac{c\alpha \lambda_t}{\alpha-1}\right)\theta_t \exp{2c_1\lambda_t}
    \\
    &~~~~~\leq\left(1-\frac{1}{\alpha}\right)\theta_t\exp{\left(2c_1 + \frac{c\alpha}{\alpha-1}\right)\lambda_t}\\
    &~~~~~=\left(1-\frac{1}{\alpha}\right)\theta_t\exp{\tilde{c}\lambda_t}
    \overset{\eqref{eq:supercondi_2}}{\leq} \left(1-\frac{1}{\alpha}\right)\theta_t\exp{\tilde{c}\left(1-\frac{1}{\beta}\right)^t\lambda_0},
    \end{split}
\end{align}
where the second inequality comes from $\exp{x}\geq 1+x$ with $x=c\alpha\lambda_t/(\alpha-1)$.
Taking expectation on the both sides of equation \eqref{eq:thetat+1leqs}, we have
\begin{align}
\label{eq:expthetat}
    \EB[\theta_{t+1}]\leq \left(1-\frac{1}{\alpha}\right)\exp{\tilde{c}\left(1-\frac{1}{\beta}\right)^t\lambda_0}\EB[\theta_t],
\end{align}
where we use the fact $\EB[\EB_{t}[\delta_{t+1}]] = \EB[\delta_{t+1}]$.
% \textcolor{red}{Due to the initial condition, we have
% \begin{align}
% \label{eq:lessexp}
%     \exp{2c_1\lambda_t}\overset{\eqref{eq:supercondi}}{\leq} \exp{2c_1\lambda_0}\overset{\eqref{eq:supercondiini}}{\leq} 1.
% \end{align}}
Therefore, we finish the proof as follows
\begin{align*}    
& \BE\left[\frac{\lambda_{t+1}}{\lambda_t}\right]
\overset{\eqref{eq:lambda_t_succ}}{\leq} \BE[\theta_{t}\exp{2c_1\lambda_t}] 
\overset{\eqref{eq:supercondi_2}}{\leq} \EBcommon{\theta_{{t}}}\exp{\tilde{c}\big(1-\frac{1}{\beta}\big)^t\lambda_0} 
\\
&\overset{\eqref{eq:expthetat}}{\leq}  \left(1-\frac{1}{\alpha}\right)\EBcommon{\theta_{t-1}}\exp{\tilde{c}\big(1-\frac{1}{\beta}\big)^{t-1}\lambda_0+\tilde{c}\big(1-\frac{1}{\beta}\big)^t\lambda_0}\\
&\overset{\eqref{eq:expthetat}}{\leq}\left(1-\frac{1}{\alpha}\right)^{t} \EBcommon{\theta_0}\exp{\tilde{c}\sum_{i=0}^{{t}}\big(1-\frac{1}{\beta}\big)^i\lambda_0}\\
&~~{\leq}~~  \left(1-\frac{1}{\alpha}\right)^t \EBcommon{\delta_0+c\lambda_0}\exp{\tilde{c}\beta\lambda_0}\overset{\eqref{eq:supercondi_2},\,\eqref{eq:supercondiini}}{\leq} \left(1-\frac{1}{\alpha}\right)^t 2s.
\end{align*}

\end{proof}


\begin{lemma}
\label{lm:linear}
Let~$\{\lambda_t\}$ and $\{\tilde{\eta}_t\}$ be two positive sequences with $\tilde{\eta}_t\geq 1$ and satisfy %$\tilde{\eta}_t\geq 1$ for all $t\geq 0$ and 
\begin{align}
\label{eq:lambda_t_1}
    &\lambda_{t+1}\leq \left(1-\frac{1}{\tilde{\eta}_t}\right)\lambda_t + \frac{m_1\lambda_t^2 }{2} + \frac{m_1^2\lambda_t^3}{4\tilde{\eta}_t}~~~\text{for all $\lambda_t$ such that $m_1\lambda_t\leq 2$}\\
    \label{eq:etat+1}
    &{\text {and}}~~~\tilde{\eta}_{t+1}\leq (1+m_2\lambda_t)^2\tilde{\eta}_t
\end{align}
for some $m_1,m_2>0$.
If
\begin{align}
    \label{eq:linear_initial}
    m\lambda_0\leq \frac{\ln ({3}/{2})}{4\tilde{\eta}_0},
\end{align}
where $m\triangleq\max\{m_1,m_2\}$,
then it holds that
\begin{align}
& \tilde{\eta}_t\leq\tilde{\eta}_0 \exp{2m\sum_{i=0}^{t-1}\lambda_i}\leq \frac{3\tilde{\eta}_0}{2} \label{eq:tilde_eta} \\
&\text{and}\quad
\lambda_t\leq\left(1-\frac{1}{2\tilde{\eta}_0}\right)^t\lambda_0. \label{eq:lambda_t_linear}
\end{align}
\end{lemma}
\begin{proof}
%see \citet{Lin2021greedy} Theorem 23
% Our analysis follows the proof of Theorem 4.7 of Rodomanov and Nesterov \cite{rodomanov2021greedy} and Theorem 23 of Lin et al.~\cite{lin2021greedy}.
We prove the results of~\eqref{eq:tilde_eta} and \eqref{eq:lambda_t_linear} by induction. 
In the case of $t=0$, inequalities \eqref{eq:tilde_eta} and  \eqref{eq:lambda_t_linear} are satisfied naturally by conditions $\tilde\eta_t\geq 1$ and condition (\ref{eq:etat+1}), where we define $\sum_{i=0}^{t}\lambda_i=0$ if $t<0$. 
Now we suppose inequalities~\eqref{eq:tilde_eta} and \eqref{eq:lambda_t_linear} holds for $t=0,\dots,\hat{t}$, then we have
\begin{align}    m\sum_{i=0}^{\hat{t}}\lambda_i\overset{\eqref{eq:lambda_t_linear}}{\leq} m\lambda_0\sum_{i=0}^{\hat{t}}\left(1-\frac{1}{2\tilde{\eta}_0}\right)^{i} \leq  m \lambda_0 \cdot 2\tilde{\eta}_0 \overset{\eqref{eq:linear_initial}}{\leq} 1.
\end{align}
In the case of $t=\hat{t}+1$, we use induction to achieve
\begin{equation}
\label{eq:simple_induct}
\begin{split}
  \frac{ 1-{m_1\lambda_{\hat{t}}}/{2}}{\tilde{\eta}_{\hat{t}}}&\geq\frac{\exp{-m_1\lambda_{\hat{t}}}}{\tilde{\eta}_{\hat{t}}}\overset{\eqref{eq:tilde_eta}}{\geq} \frac{\exp{-m_1\lambda_{\hat{t}}}\cdot \exp{-2m\sum_{i=0}^{\hat{t}-1}\lambda_i}}{\tilde{\eta}_0}\\
  &\geq
  \frac{\exp{-2m\sum_{i=0}^{\hat{t}}\lambda_i}}{\tilde{\eta}_0}\overset{\eqref{eq:tilde_eta}}{\geq} \frac{2}{3\tilde{\eta}_0},
\end{split}
\end{equation}  
where the first step is due to the fact $\exp{-2x}\leq 1-x$ with $x=m_1\lambda_{\hat t}/2\in[0,1/2]$ and the last second step is based on $m\geq m_1$. We also have 
\begin{equation} 
\label{eq:lambda_t_bound}
m_1\lambda_{\hat t} \leq m\lambda_{\hat{t}}\overset{\eqref{eq:lambda_t_linear}}{\leq} m\lambda_0\overset{\eqref{eq:linear_initial}}{\leq} \frac{1}{8\tilde{\eta}_0}\leq 2,
\end{equation}
where the last step is due to $\tilde\eta_0\geq 1$.
According to the condition \eqref{eq:lambda_t_1}, we have
\begin{align*}
&  \lambda_{\hat{t}+1}\overset{\eqref{eq:lambda_t_1}}{\leq} 
     \left(1-\frac{1}{\tilde{\eta}_{\hat{t}}}\right)\lambda_{\hat{t}} + \frac{m_1\lambda_{\hat{t}}^2 }{2} + \frac{m_1^2\lambda_{\hat{t}}^3}{4\tilde{\eta}_{\hat{t}}} =\left(1+\frac{m_1\lambda_{\hat{t}}}{2}\right)\cdot\left(1-\frac{1-{m_1\lambda_{\hat{t}}}/{2}}{\tilde{\eta}_{\hat{t}}}\right)\lambda_{\hat{t}}\\
&~\overset{\eqref{eq:simple_induct},\,\eqref{eq:lambda_t_bound}}{\leq}  \left(1+\frac{1}{16\tilde{\eta}_0}\right)\cdot\left(1-\frac{2}{3\tilde{\eta}_0}\right)\lambda_{\hat{t}} \leq \left(1-\frac{1}{2\tilde{\eta}_0}\right)\lambda_{\hat{t}}
\overset{\eqref{eq:lambda_t_linear}}{\leq} \left(1-\frac{1}{2\tilde{\eta}_0}\right)^{{\hat{t}}+1}\lambda_0.
\end{align*}
% where the inequality $(*)$ is due to the fact that
% $(1+1/(16\tilde{\eta}_0))(1-2/(3\tilde{\eta}_0)) = 1 - 29/(48\tilde{\eta}_0)-1/(24\tilde{\eta}_0^2) < 1-1/(2\tilde{\eta}_0)$.
The induction also implies
\begin{align*}
   \tilde{ \eta}_{\hat{t}+1}\overset{\eqref{eq:etat+1}}{\leq}(1+m_2\lambda_{\hat{t}})^2\tilde{\eta}_t\leq \tilde{\eta}_{\hat{t}} \exp{2m\lambda_{\hat{t}}}\overset{\eqref{eq:tilde_eta}}{\leq} \tilde{\eta}_0\exp{2m\sum_{i=0}^{\hat{t}}\lambda_{\hat{t}}}\overset{\eqref{eq:tilde_eta}}{\leq } \frac{3\tilde{\eta}_0}{2},
\end{align*}
where the second inequality is due to the fact $\exp{x}\geq 1+x$ with $x = m_2\lambda_{\hat{t}}\geq 0$. Thus, we finish the proof.
\end{proof}

\section{Proof of the main theorem}

Let $p$ be an odd prime and let $V$ be an $n$-dimensional vector space over $\mathbb{F}_p$ with basis $v_1,v_2,\dots,v_n$. The groups $G$ in (a),(b) are precisely those in (\ref{Gpi}), with $n=3$, associated to the linear maps
\begin{enumerate}[label = (\alph*)]
\item $\pi : V\rightarrow \Lambda^2V;\,$ $v_2^\pi = v_3^\pi  =1 $ and $v_1^\pi = (v_1\wedge v_2)$,
\item $\pi : V\rightarrow \Lambda^2V;\,$ $v_2^\pi = v_3^\pi =1 $ and $v_1^\pi = (v_2\wedge v_3)$.
\end{enumerate}
Similarly, the groups $G$ in (c),(d),(e) are precisely those in  (\ref{Gpi}), with $n=4$, associated to the linear maps
\begin{enumerate}[label = (\alph*)]\setcounter{enumi}{+2}
\item $\pi : V\rightarrow \Lambda^2V;\,$ $v_2^\pi = v_3^\pi = v_4^\pi =1 $ and $v_1^\pi = (v_1\wedge v_2)$,
\item $\pi : V\rightarrow \Lambda^2V;\,$ $v_2^\pi = v_3^\pi = v_4^\pi =1 $ and $v_1^\pi = (v_3\wedge v_4)$,
\item $\pi : V\rightarrow \Lambda^2V;\,$ $v_2^\pi = v_3^\pi = v_4^\pi =1 $ and $v_1^\pi = (v_1\wedge v_2)(v_3\wedge v_4)$.
\end{enumerate}
By Propositions \ref{rank one prop'} and \ref{rank one prop}, for $n=3,4$ and up to a change of basis, these are the only linear maps $\pi$ of rank one.

In this section, let us take $n=3,4$ and the symbol $\pi$ denotes one of the five linear maps above. As explained in Section \ref{group section}, we may identify
\[ G/G' = V\mbox{ and }G'=\Lambda^2V.\]
Moreover, we have a natural isomorphism
\[ \Aut^c(G) \simeq \Aut^c(\pi).\]
With these identifications, we may rephrase (\ref{Delta2}) as
\begin{equation}\label{Delta3}
\Delta(u^\alpha,v^\alpha) = \Delta(u,v)^{\hat{\alpha}}
\end{equation}
for all $u,v\in V$ and $\alpha\in\Aut^c(\pi)$. The $S$ and $S'$ in Section \ref{bilinear form sec} become
\begin{align*}
S &=  \{\mbox{symmetric bilinear $\Delta :V\times V\rightarrow \Lambda^2V$ satisfying (\ref{Delta3})}\}\\
S' &= \{\mbox{anti-symmetric bilinear  $\Delta :V\times V\rightarrow \Lambda^2V$ satisfying (\ref{Delta3})}\}
\end{align*}
in the current setting. The group $\Aut^c(\pi)$ was computed in Section \ref{group section}. Let $P$ and $Q$ denote the subgroups defined there. Then, we have
\[\Aut^c(\pi) = P\rtimes Q.\]
We shall also make the following assumption.

\begin{assume}Assume that $p\geq 5$ in the cases (a),(c),(e).
\end{assume}

We first show that the groups $G$ in question satisfy Assumption \ref{assumption} so that the discussion thereafter applies.  
\begin{lemma}\label{gamma lemma}
Let $\gamma : V\rightarrow\Aut^c(\pi)$ be an $\Aut^c(\pi)$-equivariant homomorphism and let $1\leq i,j\leq n$. Suppose that
\begin{enumerate}[label = $(\arabic*)$]
\item $\gamma(v_i)=1$,
\item $v_i^\alpha = v_iv_j$ for some $\alpha\in \Aut^c(\pi)$.
\end{enumerate}
Then $\gamma(v_j)=1$ also holds.
\end{lemma}

\begin{proof}Indeed, we have
\[ 1 = \gamma(v_i)^\alpha = \gamma(v_i^\alpha) = \gamma(v_i)\gamma(v_j) = \gamma(v_j)\]
by the hypotheses.
\end{proof}

\begin{prop}\label{gamma prop}There is no non-trivial $\Aut^c(\pi)$-equivariant homomorphism from $V$ to $\Aut^c(\pi)$.
\end{prop}

\begin{proof}Let $\gamma : V\rightarrow\Aut^c(\pi)$ be an $\Aut^c(\pi)$-equivariant homomorphism and observe that $\gamma(V)$ must be a normal $p$-subgroup of $\Aut^c(\pi)$. But
 \[ Q \simeq \begin{cases}
\mathbb{F}_p^\times\times \mathbb{F}_p^\times &\mbox{in case (a)}\\
\GL_2(\mathbb{F}_p)&\mbox{in cases (b) and (e)}\\
\mathbb{F}_p^\times \times \GL_2(\mathbb{F}_p)&\mbox{in cases (c) and (d)}
\end{cases}\]
has no non-trivial normal $p$-subgroup. Since $\Aut^c(\pi) = P\rtimes Q$, we see that $\gamma(V)$ must lie inside $P$. We now deal with each case separately.
\begin{enumerate}[label=(\alph*), wide=0pt]
\item It is clear from Proposition \ref{auto1'} that
\[ v_1^{\alpha_{12}} = v_1v_2\]
for some $\alpha_{12} \in P$, and so it is enough to show that $\gamma(v_1)=\gamma(v_3)=1$ by Lemma \ref{gamma lemma}, it. Let us put
\[ \gamma(v_1) = \begin{bmatrix}
1 & b_1 & 0 \\
0 & 1 & 0 \\
0 & c_1 & 1
\end{bmatrix}\mbox{ and }\gamma(v_3)= \begin{bmatrix}
1 & b_3 & 0 \\
0 & 1 & 0 \\
0 & c_3 & 1
\end{bmatrix} \]
From $\gamma(v_1^\alpha) = \gamma(v_1)^\alpha$ for $\alpha\in Q$ of the shape
\[ \alpha = \begin{bmatrix}s & 0 & 0 \\
0 & 1 & 0\\
0 & 0 & s\end{bmatrix} \mbox{ with } s\in \mathbb{F}_p^\times,\]
we get that $\gamma(v_1^\alpha) = \gamma(v_1)^s$ and
\[  \begin{bmatrix}
1 & sb_1 & 0 \\
0 & 1 & 0 \\
0 & sc_1 & 1
\end{bmatrix}= \begin{bmatrix}
1 & s^{-1}b_1 & 0 \\
0 & 1 & 0 \\
0 & s^{-1}c_1 & 1
\end{bmatrix}.\]
Since $p\geq 5$, there exists $s\in \mathbb{F}_p^\times$ with $s^2\neq 1$, and so $b_1=c_1=0$. We may obtain $b_3 = c_3 =0$ by the exact same calculation.
\item It is clear from Proposition \ref{auto2'} that
\[ v_1^{\alpha_{12}} = v_1v_2\mbox{ and } v_1^{\alpha_{13}} = v_1v_3\]
for some $\alpha_{12},\alpha_{13} \in P$, so it suffices to show that $\gamma(v_1)=1$ by Lemma \ref{gamma lemma}. Let us put
\[ \gamma(v_1) = \begin{bmatrix}1 & b_1 & c_1\\0& 1 & 0 \\ 0 & 0 & 1\end{bmatrix}.\]
From $\gamma(v_1^\alpha) = \gamma(v_1)^\alpha$ for $\alpha\in Q$ of the shape
\[\begin{bmatrix}
1 & 0 & 0\\
0 & s & 0\\
0 & 0 &s^{-1}
\end{bmatrix}\mbox{ with }s\in\mathbb{F}_p^\times,\]
we get that $\gamma(v_1^\alpha) = \gamma(v_1)$ and
\[  \begin{bmatrix}1 & b_1 & c_1\\0& 1 & 0 \\ 0 & 0 & 1\end{bmatrix}
=  \begin{bmatrix}1 & sb_1 & s^{-1}c_1\\0& 1 & 0 \\ 0 & 0 & 1\end{bmatrix}.\]
This yields $b_1=c_1=0$.
\item It is clear from Proposition \ref{auto1} that
\[ v_1^{\alpha_{12}} = v_1v_2\mbox{ and }v_3^{\alpha_{34}} = v_3v_4\]
for some $\alpha_{12}\in P, \alpha_{34}\in Q$, so it suffices to show that $\gamma(v_1)=\gamma(v_3)=1$ by Lemma \ref{gamma lemma}. Let us put
\[ \gamma(v_1) = \begin{bmatrix}
1 & b_1 & 0 & 0\\
0 & 1 & 0 & 0\\
0 & c_1 & 1 & 0\\
0 & d_1 & 0 & 1
\end{bmatrix}\mbox{ and }
 \gamma(v_3) = \begin{bmatrix}
1 & b_3 & 0 & 0\\
0 & 1 & 0 & 0\\
0 & c_3 & 1 & 0\\
0 & d_3 & 0 & 1
\end{bmatrix}.\]
From $\gamma(v_1^\alpha) = \gamma(v_1)^\alpha$ for $\alpha\in  Q$ of the shape
\[ \alpha =\begin{bmatrix}
s & 0 & 0 &0\\
0 & 1 & 0 & 0\\
0 & 0 & s & 0\\
0 & 0 & 0 & s
\end{bmatrix} \mbox{ with } s\in \mathbb{F}_p^\times,\]
we get that $\gamma(v_1^\alpha) = \gamma(v_1)^s$ and
\[ \begin{bmatrix}
1 & sb_1 & 0 & 0\\
0 & 1 & 0 & 0\\
0 & sc_1 &1 & 0\\
0 & sd_1 & 0 & 1
\end{bmatrix} = \begin{bmatrix}
1 & s^{-1}b_1 & 0 & 0\\
0 & 1 & 0 & 0\\
0 & s^{-1}c_1 &1 & 0\\
0 & s^{-1}d_1 & 0 & 1
\end{bmatrix} .\]
Since $p\geq 5$, there exists $s\in \mathbb{F}_p^\times$ with $s^2\neq 1$, and so $b_1=c_1=d_1=0$. We may obtain $b_3=c_3=d_3=0$ by the exact same calculation.
\item It is clear from Proposition \ref{auto2} that
\[ v_1^{\alpha_{12}} = v_1v_2,\,\ v_1^{\alpha_{13}} = v_1v_3,\,\ v_1^{\alpha_{14}} = v_1v_4.\]
for some $\alpha_{12},\alpha_{13},\alpha_{14}\in P$, and so it suffices to show that $\gamma(v_1)=1$ by Lemma \ref{gamma lemma}. Let us put
\[ \gamma(v_1) = \begin{bmatrix}
1 & b_1 & c_1 & e_1\\
0 & 1 & d_1 & f_1\\
0 & 0 & 1 & 0\\
0 & 0 & 0 & 1
\end{bmatrix}.\]
From $\gamma(v_1^\alpha) = \gamma(v_1)^\alpha$ for $\alpha\in \Aut^c(\pi)$ of the shape
\[ \alpha =\begin{bmatrix}
1 & 0 & 0 & 0\\
0 & 1 & g & 0\\
0 & 0 & s & 0\\
0 & 0 & 0 & s^{-1}
\end{bmatrix} \mbox{ with  } s\in \mathbb{F}_p^\times\mbox{ and }g\in \mathbb{F}_p,\]
we get that $\gamma(v_1^\alpha) = \gamma(v_1)$ and
\[ \begin{bmatrix}
1 & b_1 & c_1 & e_1\\
0 & 1 & d_1 & f_1\\
0 & 0 &1 & 0\\
0 & 0 & 0 & 1
\end{bmatrix} = \begin{bmatrix}
1 & b_1 & gb_1 + sc_1 & s^{-1}e_1\\
0 & 1 & sd_1 & s^{-1}f_1\\
0 & 0 & 1 & 0\\
0 & 0 & 0 &1 
\end{bmatrix}.\]
This yields $b_1 = c_1 = d_1=e_1=f_1=0$. 
\item It is clear from Proposition \ref{auto3} that
\[v_1^{\alpha_{12}} = v_1v_2,\,\ v_1^{\alpha_{13}} = v_1v_3,\,\ v_1^{\alpha_{14}} = v_1v_4\]
for some $\alpha_{12},\alpha_{13},\alpha_{14}\in P$, and so it suffices to show that $\gamma(v_1)=1$ by Lemma \ref{gamma lemma}. Let us put
\[ \gamma(v_1) = \begin{bmatrix}
1 & b_1 & -d_1 & c_1\\
0 & 1 & 0 & 0\\
0 & c_1 & 1 & 0\\
0 & d_1 & 0 & 1
\end{bmatrix}.\]
From $\gamma(v_1^\alpha) = \gamma(v_1)^\alpha$ for $\alpha\in Q$ of the shape
\[ \alpha =\begin{bmatrix}
s& 0 & 0 & 0\\
0 & 1 & 0 & 0\\
0 & 0 & s & 0\\
0 & 0 & 0 &1
\end{bmatrix} \mbox{ with } s\in \mathbb{F}_p^\times,\]
we get that $\gamma(v_1^\alpha) = \gamma(v_1)^{s}$ and
\[\begin{bmatrix}
1 & sb_1 & -sd_1 & sc_1\\
0 & 1 & 0 & 0\\
0 & sc_1 & 1 & 0\\
0 & sd_1 & 0 & 1
\end{bmatrix}= \begin{bmatrix}
1 & s^{-1}b_1 & -d_1 & s^{-1}c_1\\
0 & 1 & 0 & 0\\
0 & s^{-1}c_1 & 1 & 0\\
0 & d_1 & 0 & 1\end{bmatrix}.\]
This implies that $d_1=0$. Since $p\geq 5$, there exists $s\in \mathbb{F}_p^\times$ with $s^2\neq 1$, and we see that $b_1=c_1=0$ as well.
\end{enumerate}
In all cases, we have shown that $\gamma$ is trivial.
 \end{proof}
 
 Therefore, we may apply Theorem \ref{pre thm} to obtain
 \begin{equation}\label{T(G)} T(G) \simeq S \rtimes \res(\mathcal{S}').\end{equation}
It remains to determine the structure of $S$ and $\res(\mathcal{S}')$.

 \subsection{A module-theoretic approach} 
 
Observe that by the universal property of $S^2V$, the symmetric square of $V$, there is a natural correspondence between
\begin{itemize}
\item symmetric bilinear forms $V\times V\rightarrow\Lambda^2V$,
\item linear maps $S^2V\rightarrow \Lambda^2V$.
\end{itemize}
Similarly, there is a natural correspondence between
\begin{itemize}
\item anti-symmetric bilinear forms $V\times V\rightarrow\Lambda^2V$,
\item linear maps $\Lambda^2V\rightarrow \Lambda^2V$.
\end{itemize}
Since we are writing addition in $V$ multiplicatively, let us denote multiplication in $S^2V$ by $*$ to avoid confusion. Then, both $S^2V$ and $\Lambda^2V$ are naturally $\Aut^c(\pi)$-modules via the action
\[ (u* v)^{\alpha} = u^\alpha * v^\alpha\mbox{ and }(u\wedge v)^\alpha = u^\alpha \wedge v^\alpha\]
for all $u,v\in V$ and $\alpha\in \Aut^c(\pi)$. Taking (\ref{Delta3}) into account, it follows that elements of $S$ and $S'$, respectively, correspond to $\Aut^c(\pi)$-module homomorphisms $S^2V\rightarrow \Lambda^2V$ and $\Lambda^2V\rightarrow\Lambda^2V$.

Let us first restrict the action to $Q$. An $\Aut^c(\pi)$-module homomorphism is in particular a $Q$-module homomorphism. The latter is easier to understand because matrices in $Q$ are all block diagonal, and so we easily see that both $S^2V$ and $\Lambda^2V$, as $Q$-modules, are decomposable as a direct sum of irreducible submodules. In the tables below, we list a basis for each irreducible component, and we indicate the action of an arbitrary $\alpha\in Q$ in matrix form with respect to the given basis. Here
\[ \alpha = \begin{bmatrix} s & 0 & 0 \\ 0 & 1 & 0 \\ 0 & 0 &t\end{bmatrix},\begin{bmatrix}
|A| &  \begin{matrix} 0 & 0 \end{matrix}\\
 \begin{matrix} 0 \\ 0 \end{matrix} & A
\end{bmatrix}\]
in cases (a),(b), respectively, while 
\[ \alpha = \begin{bmatrix}
s & 0 & \begin{matrix} 0 & 0 \end{matrix}\\
0 & 1 & \begin{matrix} 0 & 0 \end{matrix}\\
\begin{matrix} 0 \\ 0 \end{matrix} & \begin{matrix} 0 \\ 0 \end{matrix} & A
\end{bmatrix},\begin{bmatrix}
|A| & 0 & \begin{matrix} 0 & 0 \end{matrix}\\
0 & s & \begin{matrix} 0 & 0 \end{matrix}\\
\begin{matrix} 0 \\ 0 \end{matrix} & \begin{matrix} 0 \\ 0 \end{matrix} & A
\end{bmatrix},\begin{bmatrix}
|A| & 0 & \begin{matrix} 0 & 0 \end{matrix}\\
0 & 1 & \begin{matrix} 0 & 0 \end{matrix}\\
\begin{matrix} 0 \\ 0 \end{matrix} & \begin{matrix} 0 \\ 0 \end{matrix} & A
\end{bmatrix}\]
in cases (c),(d),(e), respectively. The variables $s,t$ here range over $\mathbb{F}_p^\times$, and $A$ ranges over $\GL_2(\mathbb{F}_p)$.

 \begingroup
\setlength{\tabcolsep}{10pt} % Default value: 6pt
\renewcommand{\arraystretch}{1.15}
%\captionof{table}{}
 \begin{center}
  \begin{longtable}{ |c|c|}
  \multicolumn{2}{c}{Case (a)}\\
 \hline
 \hline
\multicolumn{2}{|c|}{Components of $S^2V$} \\
\hline
 Basis & Action of $\alpha\in Q$\\ \hline
 $v_1*v_1 $ & $s^2$ \\ 
 $v_1*v_2$ & $s$ \\ 
 $v_1*v_3$ & $st$ \\ 
 $v_2*v_2$ & $1$\\
 $v_2*v_3$ & $t$ \\
 $v_3*v_3$ & $t^2$\\
\hline\hline
\multicolumn{2}{|c|}{Components of $\Lambda^2V$}\\
\hline
 Basis & Action of $\alpha\in Q$ \\ \hline
 $v_1\wedge v_2 $ & $s$\\ 
 $v_1\wedge v_3$ & $st$  \\ 
 $v_2\wedge v_3$ & $t$\\ 
\hline
\end{longtable} 
 \begin{longtable}{ |c|c|}
  \multicolumn{2}{c}{Case (b)}\\
 \hline
 \hline
\multicolumn{2}{|c|}{Components of $S^2V$}\\
\hline
 Basis & Action of $\alpha\in Q$  \\ \hline
 $v_1*v_1 $ & $|A|^2$ \\ 
 $v_1*v_2,v_1*v_3$ & $|A|A$  \\ 
 $v_2*v_2, v_2*v_3,v_3*v_3$ & omitted  \\ 
\hline
\hline
\multicolumn{2}{|c|}{Components of $\Lambda^2V$}\\
\hline
 Basis & Action of $\alpha\in Q$ \\ \hline
 $v_1\wedge v_2 ,v_1\wedge v_3$ & $|A|A$  \\ 
 $v_2\wedge v_3$ & $|A|$  \\ 
\hline
\end{longtable}
 \begin{longtable}{ |c|c| }
  \multicolumn{2}{c}{Case (c)}\\
 \hline
 \hline
\multicolumn{2}{|c|}{Components of $S^2V$}\\
\hline
 Basis & Action of $\alpha\in Q$ \\ \hline
 $v_1*v_1 $ & $s^2$ \\ 
 $v_1*v_2$ & $s$ \\ 
 $v_1*v_3,v_1*v_4$ & $sA$  \\ 
 $v_2*v_2$ & $1$ \\
 $v_2*v_3,v_2*v_4$ & $A$ \\
 $v_3*v_3,v_3*v_4,v_4*v_4$ & omitted \\
\hline
\hline
\multicolumn{2}{|c|}{Components of $\Lambda^2V$} \\
\hline
 Basis & Action of $\alpha\in Q$  \\ \hline
 $v_1\wedge v_2 $ & $s$  \\ 
 $v_1\wedge v_3,v_1\wedge v_4$ & $sA$  \\ 
 $v_2\wedge v_3,v_2 \wedge v_4$ & $A$  \\ 
 $v_3\wedge v_4$ & $|A|$ \\
\hline
\end{longtable}
%\captionof{table}{The case when $v_1^\pi = v_1\wedge v_2$}\label{a sym}
 \begin{longtable}{ |c|c| }
 \multicolumn{2}{c}{Case (d)}\\
 \hline
 \hline
\multicolumn{2}{|c|}{Components of $S^2V$}\\
\hline 
Basis & Action of $\alpha\in Q$ \\ \hline
 $v_1*v_1 $ & $|A|^2$ \\ 
 $v_1*v_2$ & $s|A|$  \\ 
 $v_1*v_3,v_1*v_4$ & $|A|A$  \\ 
 $v_2*v_2$ & $s^2$\\
 $v_2*v_3,v_2*v_4$ & $sA$  \\
 $v_3*v_3,v_3*v_4,v_4*v_4$ & omitted \\
\hline
\hline
\multicolumn{2}{|c|}{Components of $\Lambda^2V$}\\
\hline 
Basis & Action of $\alpha\in Q$  \\ \hline
 $v_1\wedge v_2 $ & $s|A|$ \\ 
 $v_1\wedge v_3,v_1\wedge v_4$ & $|A|A$  \\ 
 $v_2\wedge v_3,v_2 \wedge v_4$ & $sA$  \\ 
 $v_3\wedge v_4$ & $|A|$ \\
\hline
\end{longtable}
%\captionof{table}{The case when $v_1^\pi = v_3\wedge v_4$}\label{a sym}
 \begin{longtable}{ |c|c| }
 \multicolumn{2}{c}{Case (e)}\\
 \hline
 \hline
\multicolumn{2}{|c|}{Components of $S^2V$}\\
\hline 
Basis & Action of $\alpha\in Q$ \\ \hline
 $v_1*v_1 $ & $|A|^2$ \\ 
 $v_1*v_2$ & $|A|$  \\ 
 $v_1*v_3,v_1*v_4$ & $|A|A$  \\ 
 $v_2*v_2$ & $1$ \\
 $v_2*v_3,v_2*v_4$ & $A$  \\
 $v_3*v_3,v_3*v_4,v_4*v_4$ & omitted \\
\hline
\hline
\multicolumn{2}{|c|}{Components of $\Lambda^2V$}\\
\hline 
Basis & Action of $\alpha\in Q$  \\ \hline
 $v_1\wedge v_2 $ & $|A|$ \\ 
 $v_1\wedge v_3,v_1\wedge v_4$ & $|A|A$  \\ 
 $v_2\wedge v_3,v_2 \wedge v_4$ & $A$  \\ 
 $v_3\wedge v_4$ & $|A|$  \\
\hline
\end{longtable}
%\captionof{table}{The case when $v_1^\pi = v_3\wedge v_4$}\label{a sym}
\end{center} 
\endgroup

\vspace{-0.55cm}
 
Under a $Q$-module homomorphism, an irreducible component of the domain either lies in the kernel or gets mapped to an isomorphic irreducible component of the codomain. From the stated action of $Q$, we can easily compare the isomorphism classes of the irreducible components of $S^2V$ and $\Lambda^2V$. Note that the omitted action does not matter because $\Lambda^2V$ does not have any $3$-dimensional irreducible component. The next two propositions are then immediate. 
 
 \begin{prop}\label{prelim prop sym}For any $\Delta\in S$, the following holds.
 \begin{enumerate}[label= $(\arabic*)$]
 \item In case (a), we have
\begin{align*}\Delta(v_1,v_1)&=1,\\
\Delta(v_2,v_2) &=1,\\
 \Delta(v_3,v_3)&=1.
\end{align*}
\item In case (b), we have
\begin{align*}
\Delta(v_1,v_1)& = 1,\\
\Delta(v_2,v_2) &= \Delta(v_2,v_3) =\Delta(v_3,v_3)=1.
\end{align*}
\item In cases (c),(d), and (e), we have
\begin{align*}
\Delta(v_1,v_1) &=1,\\
 \Delta(v_2,v_2) &=1,\\
 \Delta(v_3,v_3) &= \Delta(v_3,v_4)=\Delta(v_4,v_4) =1.
 \end{align*}
 \end{enumerate}
 \end{prop}
 
 \begin{prop}\label{prelim prop anti}
For any $\Delta\in S'$, the following holds.
 \begin{enumerate}[label= $(\arabic*)$]
\item In case (a), we have
\begin{align*}
\Delta(v_1,v_2) & \in \langle v_1\wedge v_2\rangle,\\
\Delta(v_1,v_3) & \in \langle v_1\wedge v_3\rangle,\\
\Delta(v_2,v_3) & \in \langle v_2\wedge v_3\rangle.
\end{align*}
\item In case (b), we have
\begin{align*}
\Delta(v_1,v_2),\Delta(v_1,v_3)& \in \langle v_1\wedge v_2,v_1\wedge v_3\rangle,\\
\Delta(v_2,v_3) & \in \langle v_2\wedge v_3\rangle.
\end{align*}
\item In cases (c) and (d), we have
\begin{align*}
 \Delta(v_1,v_2) & \in \langle v_1\wedge v_2\rangle,\\\
 \Delta(v_1,v_3),\Delta(v_1,v_4) & \in \langle v_1\wedge v_3, v_1\wedge v_4 \rangle,\\
 \Delta(v_2,v_3),\Delta(v_2,v_4) & \in \langle v_2\wedge v_3, v_2\wedge v_4 \rangle, \\
 \Delta(v_3,v_4) & \in \langle v_3\wedge v_4\rangle.
\end{align*}
\item In case (e), we have
\begin{align*}
 \Delta(v_1,v_2),\Delta(v_3,v_4) & \in \langle v_1\wedge v_2, v_3\wedge v_4\rangle,\\
 \Delta(v_1,v_3),\Delta(v_1,v_4) & \in \langle v_1\wedge v_3, v_1\wedge v_4 \rangle,\\
 \Delta(v_2,v_3),\Delta(v_2,v_4) & \in \langle v_2\wedge v_3, v_2\wedge v_4 \rangle
\end{align*}
\end{enumerate}  
 \end{prop}
  
We may refine parts of Proposition \ref{prelim prop anti} as follows.

 \begin{prop}\label{scalar prop} For any $\Delta\in S'$, the following holds.
 \begin{enumerate}[label= $(\arabic*)$]
 \item In case (b), there exists $\lambda\in\mathbb{F}_p$ such that
 \[ \begin{cases}
 \Delta(v_1,v_2) = (v_1\wedge v_2)^\lambda,\\
 \Delta(v_1,v_3) = (v_1\wedge v_3)^\lambda.
 \end{cases}\]
 \item In cases (c),(d), and (e), there exist $\lambda_1,\lambda_2\in \mathbb{F}_p$ such that
\[\begin{cases}
\Delta(v_1,v_3) = (v_1\wedge v_3)^{\lambda_1} \\
\Delta(v_1,v_4) = (v_1\wedge v_4)^{\lambda_1}
\end{cases}\,\
\begin{cases}
\Delta(v_2,v_3) = (v_2\wedge v_3)^{\lambda_2},\\
\Delta(v_2,v_4) = (v_2\wedge v_4)^{\lambda_2}.
\end{cases}\]
 \end{enumerate}
  \end{prop}
 
 \begin{proof} Consider case (b). We know from Proposition \ref{prelim prop anti} that $\Delta$ has to induce a $Q$-module endomorphism 
 \[ \delta : \langle v_1\wedge v_2,v_1\wedge v_3\rangle \rightarrow \langle v_1\wedge v_2,v_1\wedge v_3 \rangle.\]
If $\delta$ is trivial, then simply take $\lambda=0$. If $\delta$ is non-trivial, then it has to be invertible because $\langle v_1\wedge v_2,v_1\wedge v_3\rangle$ is irreducible. Say $\delta$ is given by the matrix $M\in \GL_2(\mathbb{F}_p)$. But $M$ must commute with the action of $Q$ and observe that $Q$ restricts to an $\SL_2(\mathbb{F}_p)$-action on $\langle v_1\wedge v_2,v_1\wedge v_3\rangle$.
Since the only matrices that centralize $\SL_2(\mathbb{F}_p)$ are the scalar multiples of the identity, it follows that $M = \left[\begin{smallmatrix} \lambda& 0\\ 0 & \lambda\end{smallmatrix}\right]$
for some $\lambda\in\mathbb{F}_p^\times$. The proves (1), and the same argument may be applied to prove (2).\end{proof}

\subsection{Computation of $S$ and $S'$} We shall now compute $S$ and $S'$ by taking the action of $P$ into account.

First, notice that a symmetric bilinear form $\Delta : V\times V\rightarrow \Lambda^2V$ is uniquely determined by
  \[ \Delta(v_i,v_j)\mbox{ for }1\leq i \leq j \leq n.\]
The next observation shall also be useful.

\begin{lemma}\label{sym lemma}Let $\Delta \in S$ and let $1\leq i, j \leq n$. If
\begin{enumerate}[label = $(\arabic*)$]
\item $\Delta(v_i,v_i) = \Delta(v_j,v_j)=1$,
\item  $v_i^{\alpha }= v_iv_j$ for some $\alpha\in \Aut^c(\pi)$,
\end{enumerate}
then $\Delta(v_i,v_j) =\Delta(v_j,v_i)= 1$ also holds.
\end{lemma}

\begin{proof}By the hypothesis and the condition (\ref{Delta3}), we have
\begin{align*}
1 & = \Delta(v_i,v_i)^{\hat{\alpha}}\\
& = \Delta(v_i^{\alpha} ,v_i^{\alpha})\\
& = \Delta(v_iv_j,v_iv_j)\\
& = \Delta(v_i,v_i)\Delta(v_i,v_j)\Delta(v_j,v_i)\Delta(v_j,v_j)\\
&=\Delta(v_i,v_j)\Delta(v_j,v_i)\\
& = \Delta(v_i,v_j)^2,
\end{align*}
where the last equality holds because $\Delta$ is symmetric. Since $p$ is odd, we may take the square root and so $\Delta(v_i,v_j)=\Delta(v_j,v_i)=1$.
\end{proof}

\begin{prop}\label{S=1} We have $S=1$ in all cases (a),(b),(c),(d), and (e).
\end{prop}

\begin{proof}Let $\Delta\in S$ be arbitrary. We consider each case separately.
\begin{enumerate}[label=(\alph*),wide=0pt]
\item It is clear from Proposition \ref{auto1'} that
\[ v_1^{\alpha_{12}} = v_1v_2\mbox{ and } v_3^{\alpha_{23}} = v_2v_3\]
for some $\alpha_{12},\alpha_{23}\in P$. We then have
\[ \Delta(v_i,v_j) = 1\mbox{ for all }1\leq i \leq j\leq 3\mbox{ with }(i,j)\neq (1,3) \]
by Proposition \ref{prelim prop sym} and Lemma \ref{sym lemma}. Comparing the irreducible components of $S^2V$ and $\Lambda^2V$ as $Q$-modules, we also see that
\[ \Delta(v_1,v_3) = (v_1\wedge v_3)^\lambda\]
for some $\lambda\in\mathbb{F}_p$. But consider the action of $\alpha\in P$ given by
\[ \alpha = \begin{bmatrix} 1 & 1 & 0 \\ 0 & 1 & 0 \\ 0 & 1 & 1\end{bmatrix}.\]
By the condition (\ref{Delta3}), we have
\begin{align*}
\Delta(v_1,v_3)^{\hat{\alpha}}  & = \Delta(v_1^\alpha,v_3^\alpha)\\
& = \Delta(v_1v_2,v_2v_3)\\
& = \Delta(v_1,v_2)\Delta(v_1,v_3)\Delta(v_2,v_2)\Delta(v_2,v_3)\\
& = \Delta(v_1,v_3).
\end{align*}
But the left hand side is equal to
\[(v_1v_2\wedge v_2v_3)^\lambda =  (v_1\wedge v_2)^\lambda  (v_2\wedge v_3)^\lambda\Delta(v_1,v_3).\]
It follows that $\lambda=0$ and so $\Delta(v_1,v_3)=1$ also holds.
\item It is clear from Proposition \ref{auto2'} that
\[ v_1^{\alpha_{12}} = v_1v_2\mbox{ and } v_1^{\alpha_{13}} = v_1v_3\]
for some $\alpha_{12},\alpha_{13}\in P$. We then have
\[ \Delta(v_i,v_j) = 1\mbox{ for all }1\leq i \leq j\leq 3 \]
by Proposition \ref{prelim prop sym} and Lemma \ref{sym lemma}. 
\item It is clear from Proposition \ref{auto1} that
\[ v_1^{\alpha_{12}} = v_1v_2,\, 
v_3^{\alpha_{23}} = v_2v_3,\, v_4^{\alpha_{24}} = v_2v_4\]
for some $\alpha_{12},\alpha_{23},\alpha_{24}\in P$. We then have
 \[ \Delta(v_i,v_j)=1\mbox{ for all }1\leq i \leq j \leq 4 \mbox{ with }(i,j)\not\in\{(1,3),(1,4)\}\]
by Proposition \ref{prelim prop sym} and Lemma \ref{sym lemma}. Comparing the irreducible components of $S^2V$ and $\Lambda^2V$ as $Q$-modules, we also see that
\[ \Delta(v_1,v_3),\Delta(v_1,v_4)\in \langle v_1\wedge v_3,v_1\wedge v_4\rangle\]
has to hold. Let us write
\[ \Delta(v_1, v_3) = (v_1\wedge v_3)^{\lambda}(v_1\wedge v_4)^{\kappa},\]
and consider the action of $\alpha_1\in P$ defined by
\[\alpha_1 =  \begin{bmatrix}1 & 1 & 0 & 0 \\
0 & 1 & 0 & 0\\
 0& 1 & 1 & 0\\
 0 & 0 & 0 & 1\end{bmatrix}.
 \]
Since $\Delta$ satisfies the condition (\ref{Delta3}), we get that
\begin{align*}
\Delta(v_1,v_3)^{\hat{\alpha}_1}
& = \Delta(v_1^{\alpha_1},v_3^{\alpha_1})\\
& = \Delta(v_1v_2,v_2v_3)\\
& =\Delta(v_1,v_2)\Delta(v_1,v_3)\Delta(v_2,v_2)\Delta(v_2,v_3)\\
& = \Delta(v_1,v_3).\end{align*}
But explicitly, the left hand side is given by
\[(v_1v_2\wedge v_2v_3)^\lambda (v_1v_2\wedge v_4)^{\kappa}  = (v_1\wedge v_2)^{\lambda} (v_2\wedge v_3)^\lambda(v_2\wedge v_4)^\kappa\Delta(v_1,v_3).\]
This shows that $\lambda = \kappa = 0$ and hence $\Delta(v_1,v_3) =1$. Since there exists $\alpha_2\in Q$ for which $v_1^{\alpha_2} = v_1$ and $v_3^{\alpha_2} = v_4$, we have
\[ 1 = \Delta(v_1,v_3)^{\hat{\alpha}_2} = \Delta(v_1^{\alpha_2},v_3^{\alpha_2} ) = \Delta(v_1,v_4).\]
We have thus shown that $\Delta(v_1,v_3) = \Delta(v_1,v_4)=1$ also holds.
\item It is clear from Proposition \ref{auto2} that 
\[ v_1^{\alpha_{12}} = v_1v_2,\,
v_1^{\alpha_{13}} = v_1v_3,\,
v_1^{\alpha_{14}} = v_1v_4,\,
v_2^{\alpha_{23}} = v_2v_3,\,
v_2^{\alpha_{24}} = v_2v_4\]
for some $\alpha_{12},\alpha_{13},\alpha_{14},\alpha_{23},\alpha_{24}\in P$. We then have
\[ \Delta(v_i,v_j) = 1\mbox{ for all }1\leq i \leq j\leq 4 \]
by Proposition \ref{prelim prop sym} and Lemma \ref{sym lemma}.  
\item It is clear from Proposition \ref{auto3} that
\[ v_1^{\alpha_{12}} = v_1v_2,\,
v_1^{\alpha_{13}} = v_1v_3,\,
v_1^{\alpha_{14}} = v_1v_4,\,
v_3^{\alpha_{23}} = v_2v_3,\,
v_4^{\alpha_{24}}= v_2v_4\]
for some $\alpha_{12},\alpha_{13},\alpha_{14},\alpha_{23},\alpha_{24}\in P$. We then have
\[ \Delta(v_i,v_j) = 1\mbox{ for all }1\leq i \leq j\leq 4 \]
by Proposition \ref{prelim prop sym} and Lemma \ref{sym lemma}.  
\end{enumerate}
In all cases, we have shown that $\Delta=1$, and so indeed $S=1$.
  \end{proof}
 
Next, note that an anti-symmetric bilinear form $\Delta :V \times V \rightarrow \Lambda^2V$ is uniquely determined by
\[ \Delta(v_i,v_j) \mbox{ for }1\leq i < j\leq n.\]
We also make the following observation.

\begin{lemma}\label{anti lemma}
Let $\Delta\in S'$ and let $1\leq i,j,k\leq n$ with $i\neq j,k$. If
\begin{enumerate}[label = $(\arabic*)$]
\item $\Delta(v_i,v_j) = (v_i\wedge v_j)^{\lambda_1}$ or equivalently $\Delta(v_j,v_i) = (v_j\wedge v_i)^{\lambda_1}$,
\item $\Delta(v_i,v_k) = (v_i\wedge v_k)^{\lambda_2}$ or equivalently $\Delta(v_k,v_i) = (v_k\wedge v_i)^{\lambda_2}$,
\item $v_i^\alpha = v_i,\, v_j^\alpha =v_jv_k$ for some $\alpha\in \Aut^c(\pi)$,\end{enumerate}
then $\lambda_1 = \lambda_2$ has to hold.
\end{lemma}

%Note that the equivalence holds because $\Delta$ is anti-symmetric.

\begin{proof}By the condition (\ref{Delta3}), we have
\[
\Delta(v_i,v_j)^{\hat{\alpha}} = \Delta(v_i^\alpha,v_j^\alpha)
= \Delta(v_i,v_jv_k)
=\Delta(v_i,v_j)\Delta(v_i,v_k). \]
Using the hypothesis, we rewrite this as
\[ (v_i\wedge v_j)^{\lambda_1}(v_i\wedge v_k)^{\lambda_1} = (v_i\wedge v_j)^{\lambda_1}( v_i\wedge v_k)^{\lambda_2},\] 
which implies that $\lambda_1 =\lambda_2$, as claimed.
\end{proof}

For each $\lambda\in\mathbb{F}_p$, as noted in Remark \ref{remark}, clearly
\[ \Delta_{[\lambda]} : V \times V\rightarrow \Lambda^2V;\,\ \Delta_{[\lambda]}(u,v) = (u\wedge v)^\lambda\]
is an anti-symmetric bilinear form satisfying (\ref{Delta3}), namely $\Delta_{[\lambda]}\in S'$.

\begin{prop}\label{S' prop}We have
\[ S' =\begin{cases}
  \{ \Delta_{[\lambda]} : \lambda\in \mathbb{F}_p\}&\mbox{in cases (a),(b),(c), and (d)},\\
  \{ \Delta_{[\lambda]}\Delta_{[\kappa]}^* : \lambda,\kappa\in \mathbb{F}_p\}&\mbox{in case (e)},
  \end{cases} \] 
where $\Delta_{[\kappa]}^* : V\times V\rightarrow \Lambda^2V$ denotes the anti-symmetric form defined by
\begin{align*}
\Delta_{[\kappa]}^*(v_1,v_2) & = (v_3\wedge v_4)^\kappa,&\Delta_{[\kappa]}^*(v_2,v_3) & = (v_2\wedge v_3)^{-\kappa},\\
\Delta_{[\kappa]}^*(v_1,v_3) & = (v_1\wedge v_3)^{-\kappa},&\Delta_{[\kappa]}^*(v_2,v_4)& = (v_2\wedge v_4)^{-\kappa},\\
\Delta_{[\kappa]}^*(v_1,v_4)& = (v_1\wedge v_4)^{-\kappa},&\Delta_{[\kappa]}^*(v_3,v_4) & = (v_1\wedge v_2)^{\kappa}.\end{align*}
\end{prop}

\begin{proof}Let $\Delta\in S'$ be arbitrary. We consider each case separately.
\begin{enumerate}[label=(\alph*),wide=0pt]
\item[(a),(b)] By Propositions \ref{prelim prop anti} and \ref{scalar prop}, we know that
\begin{align*} \Delta(v_1,v_2) &= (v_1\wedge v_2)^{\lambda_1}\\ 
\Delta(v_1,v_3) &= (v_1\wedge v_3)^{\lambda_2}\\
\Delta(v_2,v_3) &= (v_2\wedge v_3)^{\lambda_3}
\end{align*}
for some $\lambda_1,\lambda_2,\lambda_3 \in \mathbb{F}_p$. In case (a), by Proposition \ref{auto1'}, we have
\[ \begin{cases}
v_1^{\alpha_{12}} = v_1\\
v_3^{\alpha_{12}} = v_2v_3
\end{cases}\,\ \begin{cases}
v_3^{\alpha_{23}} = v_3\\
v_1^{\alpha_{23}} = v_1v_2
\end{cases}\]
for some $\alpha_{12},\alpha_{23}\in P$. In case (b), we already know from Proposition  \ref{scalar prop} that $\lambda_1=\lambda_2$, and by Proposition \ref{auto2'}, we have
\[ \begin{cases}
v_3^{\alpha_{23}} = v_3\\
v_1^{\alpha_{23}} = v_1v_2
\end{cases}\]
for some $\alpha_{23}\in P$. In both cases, we get that
\[\lambda :=\lambda_1 = \lambda_2 = \lambda_3\]
by Lemma \ref{anti lemma}. This shows that $\Delta = \Delta_{[\lambda]}$, as claimed.
\item[(c),(d)] By Propositions \ref{prelim prop anti} and \ref{scalar prop}, we know that
\begin{align*} \Delta(v_1,v_2) &= (v_1\wedge v_2)^{\lambda_1}&\Delta(v_2,v_3) & = (v_2\wedge v_3)^{\lambda_3}\\ 
\Delta(v_1,v_3) &= (v_1\wedge v_3)^{\lambda_2} & \Delta(v_2,v_4) &= (v_2\wedge v_4)^{\lambda_3}\\
\Delta(v_1,v_4) &= (v_1\wedge v_4)^{\lambda_2}&\Delta(v_3,v_4) &= (v_3\wedge v_4)^{\lambda_4}
\end{align*}
for some $\lambda_1,\lambda_2,\lambda_3,\lambda_4 \in \mathbb{F}_p$. In case (c), by Proposition \ref{auto1}, we have
\[ \begin{cases}
v_1^{\alpha_{12}} = v_1\\
v_3^{\alpha_{12}} = v_2v_3
\end{cases}\,\
\begin{cases}
v_3^{\alpha_{23}} = v_3\\
v_1^{\alpha_{23}} = v_1v_2
\end{cases}
\,\
\begin{cases}
v_4^{\alpha_{34}} = v_4\\
v_3^{\alpha_{34}} = v_2v_3
\end{cases}\]
for some $\alpha_{12},\alpha_{23},\alpha_{34}\in P$. In case (d), by Proposition \ref{auto2}, we have
\[ \begin{cases}
v_1^{\alpha_{12}} = v_1\\
v_2^{\alpha_{12}} = v_2v_3
\end{cases}\,\
\begin{cases}
v_3^{\alpha_{23}} = v_3\\
v_1^{\alpha_{23}} = v_1v_2
\end{cases}
\,\
\begin{cases}
v_4^{\alpha_{34}} = v_4\\
v_2^{\alpha_{34}} = v_2v_3
\end{cases}\]
for some $\alpha_{12},\alpha_{23},\alpha_{34}\in P$. In both cases, we get that 
\[\lambda :=\lambda_1 = \lambda_2 = \lambda_3= \lambda_4\]
by Lemma \ref{anti lemma}. This shows that $\Delta = \Delta_{[\lambda]}$, as claimed.
\item[(e)] By Propositions \ref{prelim prop anti} and \ref{scalar prop}, we know that 
\begin{align*} \Delta(v_1,v_2) &= (v_1\wedge v_2)^{\lambda_1}(v_3\wedge v_4)^{\kappa_1}&\Delta(v_2,v_3) & = (v_2\wedge v_3)^{\lambda_3}\\ 
\Delta(v_1,v_3) &= (v_1\wedge v_3)^{\lambda_2} & \Delta(v_2,v_4) &= (v_2\wedge v_4)^{\lambda_3}\\
\Delta(v_1,v_4) &= (v_1\wedge v_4)^{\lambda_2}&\Delta(v_3,v_4) &= (v_1\wedge v_2)^{\kappa_4}(v_3\wedge v_4)^{\lambda_4}
\end{align*}
 for some $\lambda_1,\lambda_2,\lambda_3,\lambda_4,\kappa_1,\kappa_4\in \mathbb{F}_p$. Consider $\alpha\in P$ given by
\[ \alpha = \begin{bmatrix} 1 & 0 & 0 & 1\\
0 & 1 & 0 & 0\\
 0 & 1 & 1 & 0 \\ 
 0 & 0 & 0 & 1\end{bmatrix},\]
and we compute that
 \begin{align*}
\Delta(v_1,v_2)^{\hat{\alpha}}& = (v_1v_4\wedge v_2)^{\lambda_1}(v_2v_3\wedge v_4)^{\kappa_1} \\
&= \Delta(v_1,v_2)(v_4\wedge v_2)^{\lambda_1-\kappa_1},\\
\Delta(v_1^\alpha,v_2^\alpha) & = \Delta(v_1v_4,v_2) \\
&=\Delta(v_1,v_2)(v_4\wedge v_2)^{\lambda_3},\\
\Delta(v_1,v_3)^{\hat{\alpha}} & = (v_1v_4\wedge v_2v_3)^{\lambda_2} \\
&= \Delta(v_1,v_3)(v_1\wedge v_2)^{\lambda_2}(v_4\wedge v_2)^{\lambda_2}(v_4\wedge v_3)^{\lambda_2},\\
\Delta(v_1^\alpha,v_3^\alpha) & = \Delta(v_1v_4,v_2v_3) \\
&= \Delta(v_1,v_3)(v_1\wedge v_2)^{\lambda_1-\kappa_4}(v_4\wedge v_2)^{\lambda_3}(v_4\wedge v_3)^{\lambda_4-\kappa_1},\\
\Delta(v_3,v_4)^{\hat{\alpha}} & = (v_1v_4\wedge v_2)^{\kappa_4}(v_2v_3\wedge v_4)^{\lambda_4}\\
& = \Delta(v_3,v_4)(v_2\wedge v_4)^{\lambda_4-\kappa_4},\\
\Delta(v_3^\alpha,v_4^\alpha) & = \Delta(v_2v_3,v_4)\\
&= \Delta(v_3,v_4)(v_2\wedge v_4)^{\lambda_3}.
\end{align*}
Since the condition (\ref{Delta3}) has to hold, we deduce that
\[ \lambda_3= \lambda_1 - \kappa_1,\,\
\lambda_2 = \lambda_1-\kappa_4 =\lambda_3=\lambda_4-\kappa_1,\,\ \lambda_3 = \lambda_4-\kappa_4.\]
Solving this system of equations, we get that
\[ \lambda := \lambda_1 = \lambda_4 ,\,\ \kappa:=\kappa_1=\kappa_4,\,\ \lambda_2 =\lambda_3 = \lambda -\kappa.\]
This shows that $\Delta = \Delta_{[\lambda]}\Delta_{[\kappa]}^*$. Conversely, for any $\lambda,\kappa\in\mathbb{F}_p$, we know that $ \Delta_{[\lambda]}\in S'$ already and it is straightforward to check that $\Delta_{[\kappa]}^*$ also satisfies (\ref{Delta3}), so then $\Delta_{[\lambda]}\Delta_{[\kappa]}^*\in S'$.
 \end{enumerate}
 This completes the proof.
\end{proof}
   
\subsection{The structure of $T(G)$} We shall now prove Theorem \ref{thm1}. We already know from (\ref{T(G)}) and Proposition \ref{S=1} that
\[ T(G) \simeq \res(\mathcal{S}').\]
In cases (a),(b),(c), and (d), the theorem follows because we have
\[ \res(\mathcal{S}') \simeq \mathbb{F}_p^\times\]
by Remark \ref{remark} and Proposition \ref{S' prop}. In case (e), by Proposition \ref{S' prop}, the elements of $S'$  are precisely the bilinear forms
\[ \Delta_{[\sigma]}: V\times V\rightarrow\Lambda^2V ;\,\  \Delta_{[\sigma]}(u,v) = (u\wedge v)^\sigma.\]
Here $\sigma$ is any endomorphism on $\Lambda^2V$ of the form
\begin{equation}\label{tau}
 \begin{bmatrix}
\lambda & &  & &&\kappa\\
 & \lambda-\kappa & & &&\\
 & & \lambda-\kappa & & &\\
 & & & \lambda-\kappa & &\\
 & & & &\lambda-\kappa &\\
\kappa & & & &&\lambda
\end{bmatrix} \mbox{ with }\lambda,\kappa\in \mathbb{F}_p,\end{equation}
written with respect to the basis
\[ v_1\wedge v_2, v_1\wedge v_3, v_1\wedge v_4,v_2\wedge v_3,v_2\wedge v_4,v_3\wedge v_4\]
of $\Lambda^2V$. By \cite[Example 3.4]{LMH}, we know that $N_{\Delta_{[\sigma]}}\simeq G$ occurs only for $1+2\sigma\in \GL(\Lambda^2V)$. Let us make a change of variables $\tau = 1+2\sigma$, and consider $\tau_{\lambda,\kappa}\in \GL(\Lambda^2V)$ of the form (\ref{tau}) but with the restriction $\kappa\neq\pm\lambda$. Observe that then
\[
\eta_{\lambda,\kappa}   = \begin{bmatrix}
\lambda+\kappa &&&\\
&(\lambda+\kappa)^{-1} &&\\
&&1&\\
&&&1
\end{bmatrix},\]
written with respect to the basis $v_1,v_2,v_3,v_4$ of $V$, in which case
\[\hat{\eta}_{\lambda,\kappa} = \begin{bmatrix}
1 &&&&&\\
 &\lambda+\kappa&&&&\\
 &&\lambda+\kappa&&&\\
 &&&(\lambda+\kappa)^{-1}&&\\
 &&&&(\lambda+\kappa)^{-1}&\\
 &&&&&1
\end{bmatrix},\]
yields a solution to $\pi\hat{\eta}_{\lambda,\kappa}\tau_{\lambda,\kappa} = \eta_{\lambda,\kappa}\pi$. From (\ref{S'}), we deduce that
\[ \res(\mathcal{S}') \simeq \{(\eta_{\lambda,\kappa},\hat{\eta}_{\lambda,\kappa}\tau_{\lambda,\kappa}) : \lambda,\kappa\in\mathbb{F}_p\mbox{ with }\kappa\neq\pm\lambda\}.  \]
It is straightforward to verify that
\[ \eta_{\lambda_{1},\kappa_1} \eta_{\lambda_{2},\kappa_{2}} = \eta_{\lambda,\kappa},\,\
 \hat{\eta}_{\lambda_1,\kappa_1}\tau_{\lambda_1,\kappa_1}\hat{\eta}_{\lambda_2,\kappa_2}\tau_{\lambda_2,\kappa_2} = \hat{\eta}_{\lambda,\kappa}\tau_{\lambda,\kappa}\]
 for any $\lambda_1,\lambda_2,\kappa_1,\kappa_2\in \mathbb{F}_p$ with $\kappa_1\neq\pm\lambda_1$ and $\kappa_2\neq\pm\lambda_2$, where
 \[\begin{bmatrix} \lambda & \kappa\\ \kappa &\lambda \end{bmatrix}= \begin{bmatrix} \lambda_1 &\kappa_1\\\kappa_1&\lambda_1\end{bmatrix}
 \begin{bmatrix}\lambda_2& \kappa_2\\ \kappa_2 & \lambda_2\end{bmatrix}
 =\begin{bmatrix} \lambda_1\lambda_2 + \kappa_1\kappa_2 & \lambda_1\kappa_2 +\lambda_2\kappa_1\\\lambda_1\kappa_2 +\lambda_2\kappa_1&\lambda_1\lambda_2 + \kappa_1\kappa_2 \end{bmatrix}.\]
It follows that $\res(\mathcal{S}')$ is isomorphic to the subgroup
\[ \left\{ \begin{bmatrix}\lambda & \kappa \\ \kappa & \lambda\end{bmatrix}  : \lambda,\kappa\in\mathbb{F}_p\mbox{ with }\kappa\neq\pm\lambda\right\}\]
of $\GL_2(\mathbb{F}_p)$, or conjugating by $\left[\begin{smallmatrix}1 & -1\\ 1 & 1 \end{smallmatrix}\right]$, the subgroup
\[ \left\{ \begin{bmatrix}\lambda + \kappa& 0 \\  0 & \lambda- \kappa\end{bmatrix}  : \lambda,\kappa\in\mathbb{F}_p\mbox{ with }\kappa\neq\pm\lambda\right\}\]
of $\GL_2(\mathbb{F}_p)$. This decomposes as
\[ \left\{ \begin{bmatrix} \lambda & 0 \\ 0 & 1 \end{bmatrix}: \lambda\in \mathbb{F}_p^\times \right\}\times  \left\{ \begin{bmatrix} 1 & 0 \\ 0 & \kappa \end{bmatrix}: \kappa\in \mathbb{F}_p^\times \right\}\]
and so is isomorphic to $\mathbb{F}_p^\times \times \mathbb{F}_p^\times$, as claimed in (e).
%We then have
%\begin{align*}
% \res(\mathcal{S'}) = & \{ (\eta,\hat{\eta}\tau)\Gamma(G) : \eta\in \GL(V),\, \tau\in \GL(\Lambda^2V)\\
% &\hspace{2.5cm}\mbox{of the shape (\ref{tau}) with $\kappa \neq \lambda,\pm2\lambda$,}\\
% &\hspace{2.5cm}\mbox{and the equation }\pi\hat{\eta}\tau = \eta\pi\mbox{ holds}\}
% \end{align*}
%by (\ref{res(S')}). Let us solve $\pi\hat{\eta}\tau = \eta\pi$ for such $\eta\in \GL(V)$ and $\tau\in \GL(\Lambda^2V)$ in a manner very similar to the proof of Proposition \ref{auto3}. Write 
%\begin{align*}
%v_1^\eta & = v_1^{n_{11}} v_2^{n_{12}} v_3^{n_{13}} v_4^{n_{14}}, \\
%v_2^\eta & = v_1^{n_{21}} v_2^{n_{22}} v_3^{n_{23}} v_4^{n_{24}},\\
%v_3^\eta & = v_1^{n_{31}} v_2^{n_{32}} v_3^{n_{33}} v_4^{n_{34}},\\
%v_4^\eta & = v_1^{n_{41}} v_2^{n_{42}} v_3^{n_{43}} v_4^{n_{44}}.
%\end{align*}
%Since $v_2^\pi =v_3^\pi = v_4^\pi = 1$, necessarily $n_{21} = n_{31} = n_{41} = 0$ and $n_{11}\neq0$. We may then simplify $v_1^{\pi\hat{\eta}\tau} = v_1^{\eta\pi}$ as
%\begin{align*}
%&((v_1^{n_{11}} v_2^{n_{12}} v_3^{n_{13}} v_4^{n_{14}} \wedge v_2^{n_{22}} v_3^{n_{23}} v_4^{n_{24}})( v_2^{n_{32}} v_3^{n_{33}} v_4^{n_{34}}\wedge v_2^{n_{42}} v_3^{n_{43}} v_4^{n_{44}}))^{\tau} \\
%&\hspace{7.25cm}= (v_1\wedge v_2)^{n_{11}} (v_3\wedge v_4)^{n_{11}}.\end{align*}
%Since $\langle v_1\wedge v_3\rangle$ and $\langle v_1\wedge v_4\rangle$
%are eigenspaces of $\tau$, which is taken to be invertible here, by comparing exponents, we see that $n_{23} = n_{24}=0$. and $n_{22}\neq0$. The above equation then becomes
%\begin{align*}
%&((v_1^{n_{11}} v_2^{n_{12}} v_3^{n_{13}} v_4^{n_{14}} \wedge v_2^{n_{22}})( v_2^{n_{32}} v_3^{n_{33}} v_4^{n_{34}}\wedge v_2^{n_{42}} v_3^{n_{43}} v_4^{n_{44}}))^{\tau} \\
%&\hspace{7.25cm}= (v_1\wedge v_2)^{n_{11}} (v_3\wedge v_4)^{n_{11}}.\end{align*}
%Since $\langle v_2\wedge v_3\rangle$ and $\langle v_2\wedge v_4\rangle$
%are eigenspaces of $\tau$, by comparing exponents, we similarly deduce that
%\[ -n_{13}n_{22} + \begin{vmatrix} n_{32} & n_{33} \\ n_{42} & n_{43} \end{vmatrix}
%= -n_{14}n_{22} + \begin{vmatrix} n_{32} & n_{34} \\ n_{42} & n_{44} \end{vmatrix} = 0.\]
%Finally, by comparing the $v_1\wedge v_2$ and $v_3\wedge v_4$ terms, we obtain
%\[ \begin{bmatrix}\lambda & \kappa \\ \kappa &\lambda\end{bmatrix}
%\begin{bmatrix}n_{11}n_{22} \\[4pt] \lvert\begin{smallmatrix} n_{33}&n_{34}\\n_{43}&n_{44}\end{smallmatrix}\rvert\end{bmatrix} 
%=\begin{bmatrix} n_{11}\\n_{11}\end{bmatrix}.\]
%The matrix on the left is taken to be invertible, so equivalently
%\[ \begin{bmatrix}n_{22}\\[4pt] n_{11}^{-1}\lvert\begin{smallmatrix} n_{33}&n_{34}\\n_{43}&n_{44}\end{smallmatrix}\rvert\end{bmatrix}=
%\begin{bmatrix} \lambda & \kappa \\ \kappa & \lambda\end{bmatrix}^{-1}\begin{bmatrix}1\\1\end{bmatrix} = \begin{bmatrix}(\lambda+\kappa)^{-1} 
%\\ (\lambda + \kappa)^{-1} 
%\end{bmatrix}.\]
%Put $s_{\lambda,\kappa} =\lambda +\kappa$. Then $\pi\hat{\eta}\tau = \eta\pi$ holds if and only if
%\begin{align*}
%\eta &= \begin{bmatrix}
%s_{\lambda,\kappa} \lvert\begin{smallmatrix}n_{33} & n_{34}\\n_{43} & n_{44}\end{smallmatrix}\rvert & n_{12} & s_{\lambda,\kappa} \lvert\begin{smallmatrix}n_{32} & n_{33} \\ n_{42} & n_{43}\end{smallmatrix}\rvert & s_{\lambda,\kappa} \lvert\begin{smallmatrix} n_{32} & n_{34} \\ n_{42} & n _{44}\end{smallmatrix}\rvert\\
%0 & s_{\lambda,\kappa}^{-1} & 0 & 0\\
%0 & n_{32} & n_{33} & n_{34}\\
%0 & n_{42} & n_{43} & n_{44}
%\end{bmatrix}\\
%& = \begin{bmatrix} s_{\lambda,\kappa} & 0 & 0& 0 \\
%0 & s_{\lambda,\kappa}^{-1}& 0 & 0 \\
% 0 & 0 & 1 & 0\\
% 0 & 0 & 0 & 1\end{bmatrix} \begin{bmatrix}
%\lvert\begin{smallmatrix}n_{33} & n_{34}\\n_{43} & n_{44}\end{smallmatrix}\rvert & s_{\lambda,\kappa}^{-1}n_{12} &\lvert\begin{smallmatrix}n_{32} & n_{33} \\ n_{42} & n_{43}\end{smallmatrix}\rvert & \lvert\begin{smallmatrix} n_{32} & n_{34} \\ n_{42} & n _{44}\end{smallmatrix}\rvert \\
%0 & 1 & 0 & 0\\
%0 & n_{32} & n_{33} & n_{34}\\
%0 & n_{42} & n_{43} & n_{44}
%\end{bmatrix},
%\end{align*}
%where this last matrix lies in $\Aut^c(\pi)$ by Proposition \ref{auto3}. The class of $(\eta,\hat{\eta}\tau)$ modulo $\Gamma(G)$ is not affected when $\eta$ is multiplied by an element of $\Aut^c(\pi)$. Thus, we may take
%\[ \eta = \begin{bmatrix} s_{\lambda,\kappa} &  & &  \\
% & s_{\lambda,\kappa}^{-1}&  &  \\
%  &  & 1 & \\
%  &  &  & 1\end{bmatrix},\,\ \hat{\eta} =
%\begin{bmatrix}
%  1 & & & & & \\
%  & s_{\lambda,\kappa}& &  &\\
%  &  & s_{\lambda,\kappa} & & &\\
%  & & & s_{\lambda,\kappa}^{-1} & &\\
%   &  & & &s_{\lambda,\kappa}^{-1} & \\
%  & &  & & & 1
%  \end{bmatrix}.\]
%% in which case we have
%% \begin{align*}
%%  \hat{\eta}\tau  =
%%\begin{bmatrix}
%%  \lambda & 0 & 0 & 0 & 0 &\kappa\\
%%  0 & (\lambda-\kappa)s_{\lambda,\kappa}& 0 & 0 &0 & 0\\
%%  0 & 0 & (\lambda-\kappa)s_{\lambda,\kappa} & 0 & 0 &0\\
%%  0 & 0 & 0 & (\lambda-\kappa)s_{\lambda,\kappa}^{-1}   & 0 & 0\\
%%   0 & 0 & 0 & 0 &(\lambda-\kappa)s_{\lambda,\kappa}^{-1} & 0 \\
%%\kappa & 0 & 0 & 0 & 0 & \lambda
%%\end{bmatrix}.
%%   \end{align*}
%To simplify notation, let us put
%\begin{align*}
%M_{\lambda,\kappa} & =  \left[ \begin{smallmatrix} \lambda + \kappa&&&\\ & (\lambda+\kappa)^{-1} &&\\ &&1&&\\&&&1\end{smallmatrix}\right],\\
% N_{\lambda,\kappa} & =  \left[\begin{smallmatrix}
% \lambda &&&&&\kappa\\
% & (\lambda-\kappa)(\lambda+\kappa) &&&&\\
% &&(\lambda-\kappa)(\lambda+\kappa) &&&\\\
% &&&(\lambda-\kappa)(\lambda+\kappa)^{-1}&&\\
% &&&&(\lambda-\kappa)(\lambda+\kappa)^{-1}&\\
%\kappa &&&&&\lambda
%\end{smallmatrix} \right].
%\end{align*}
%We then deduce that
%\[ \res(\mathcal{S}') \simeq \{ (M_{\lambda,\kappa} ,N_{\lambda,\kappa}) : \lambda,\kappa\in \mathbb{F}_p\mbox{ with }\kappa\neq \lambda,\pm2\lambda\},\]
%and it is not hard to show that this is isomorphic to 




\section{The Proof of Theorem~\ref{thm:BFGS}}
\label{sec:bfgs_proof}
We first provide the linear convergence for the proposed block BFGS and DFP methods (Algorithm~\ref{alg:bfgs}).
\begin{lemma}
\label{lm:BFGSlinear}
Under the setting of Theorem~\ref{thm:BFGS},  Algorithm~\ref{alg:bfgs} holds that 
\begin{align}
\label{eq:BFGSlinear}
    \lambda(\vx_t)\leq \left(1-\frac{1}{2\eta_0}\right)^t\lambda(\vx_0)~~~\text{and}~~~ \nabla^2 f(\x_t)\preceq \G_t\preceq \frac{3\eta_0}{2}\nabla^2 f(\x_t)
\end{align}
for all $t\geq 0$.
\end{lemma}

\begin{proof}
We can obtain this result by following the proof of Theorem~\ref{thm:srklinear} by replacing the steps of using Lemma~\ref{lm:sr1good} with using Lemma~\ref{lm:bfgsnofar}.
\end{proof}

We then provide some auxiliary lemmas for later analysis.%~\cite{rodomanov2021greedy,lin2021greedy}.
\begin{lemma}[\!\!{\cite[Lemma 4.8]{rodomanov2021greedy}}]
\label{lm:strong_self_bfgs}
If the twice differentiable function $f:\BR^d\to\BR$ is $M$-strongly self-concordant and $\mu$-strongly convex and the positive definite matrix $\G\in\RB^{d\times d}$ and $\vx\in\BR^d$ satisfy $\nabla^2 f(\x)\preceq \G\preceq \eta\nabla^2 f(\x)$ for some $\eta>1$, then we have
\begin{align}
 \sigma_{\nabla^2f(\x_{+})}(\tilde{\G})\leq (1+Mr)^2 (\sigma_{\nabla^2f(\x)}({\G})+2dMr)   \label{eq:delta_BFGS_lin}
\end{align}
for any $\vx,\vx_{+}\in\BR^d$ where $\tilde{\G}=(1+Mr)\G$ and $r=\|\x-\x_{+}\|_{\x}$. % and $\sigma_\H(\G)$ follow the expression of \eqref{eq:measurebfgs}.
\end{lemma}

\begin{lemma}\label{lm:GHneq_bfgs}
For any positive definite symmetric matrices $\G,\H\in\RB^{d\times d}$ such that~$\H\preceq\G$, 
we have
\begin{align}
\label{eq:GHneq_bfgs}
\G\preceq (1+\sigma_{\H}(\G))\H.
\end{align}
%where the notation of $\sigma_\H(\G)$ follow the expression of \eqref{eq:measurebfgs}.
If it further satisfies that $\G\preceq\eta\H$ for some $\eta \geq 1$, then we have
\begin{align}
\label{eq:GHneq_2_bfgs}
  \sigma_{\H}(\G)\leq d(\eta -1).
\end{align}
\end{lemma}
\begin{proof}
We obtain inequality \eqref{eq:GHneq_bfgs} by statement 1) of Equation (42) by Lin et al. \cite[Lemma 25]{lin2021greedy}.
We prove inequality \eqref{eq:GHneq_2_bfgs} by the definition of $\sigma_{\H}(\G)$ as follows
\begin{align*}
      &\sigma_{\H}(\G)=\tr{\H^{-1}(\G-\H)}=\tr{\H^{-1/2}(\G-\H)\H^{-1/2}}\\
      &\leq \tr{(\eta-1)\H^{-1/2}\H\H^{-1/2}} = d(\eta-1).
\end{align*}
\end{proof}


Now, we present the proof for Theorem~\ref{thm:BFGS}.
\begin{proof}
Denote $\delta_t\triangleq\sigma_{\nabla^2 f(\x_t)}(\G_t)$ and $\lambda_t\triangleq \lambda(\x_t)$. 
According to Lemma~\ref{lm:BFGSlinear} and Lemma~\ref{lm:GHneq_bfgs} with $\mG=\mG_t$ and $\mH=\nabla^2 f(\vx_t)$, we have
%$\nabla^2 f(\x_t)\preceq\G_t\overset{{\eqref{eq:GHneq_bfgs}}}{\preceq} (1+\delta_t) \nabla^2 f(\x_t).$ 
\begin{align}
\label{eq:BFGScondi_GH}
   \nabla^2 f(\x_t)\overset{\eqref{eq:BFGSlinear}}{\preceq} \G_t \overset{\eqref{eq:GHneq_bfgs}}{\preceq} (1+\delta_t) \nabla^2 f(\x_t).
\end{align}
According to Theorem~\ref{thm:bfgs} with $\mA=\nabla^2 f(\vx_{t+1})$ and $\mG=\tilde\mG_t$, we have
\begin{align}
\label{eq:BFGSconverge1}
    \EB_t[\delta_{t+1}] = \EB_t\big[\sigma_{\nabla^2 f(\x_{t+1})}(\G_{t+1})\big] \overset{\eqref{eq:bfgssigma}}{\leq} \left(1-\frac{k}{d\varkappa}\right)\sigma_{\nabla^2 f(\x_{t+1})}(\tilde{\G}_t).
\end{align}
According to Lemma~\ref{lm:strong_self_bfgs} with $\mG=\mG_t$ and $\vx=\vx_t$, we have
\begin{align}
\label{eq:BFGSconverge2}
  \sigma_{\nabla^2 f(\x_{t+1})}(\tilde{\G}_t) \overset{\eqref{eq:delta_BFGS_lin}}{\leq}(1+Mr_t)^2(\delta_t+2dMr_t).
\end{align}
Thus, we can obtain following result
\begin{align}
\label{eq:delta_t_condi_BFGS}
\begin{split}
    &\EB_{t}[\delta_{t+1}] \overset{\eqref{eq:BFGSconverge1}}{\leq} \left(1-\frac{k}{d\varkappa}\right)\sigma_{\nabla^2 f(\x_{t+1})}(\tilde{\G}_{t})\\
    &\overset{\eqref{eq:BFGSconverge2}}{\leq}
    \left(1-\frac{k}{d\varkappa}\right) (1+Mr_t)^2(\delta_t+2dMr_t)\\
    &\overset{\eqref{eq:linear-quadra}}{\leq} \left(1-\frac{k}{d\varkappa}\right)(1+M\lambda_t)^2(\delta_t+2dM\lambda_t).
\end{split}
\end{align}
% According to Lemma~\ref{lm:linear-quadra} with $\eta = 1+\delta_t$, we achieve
We follow Lemma~\ref{lm:linear-quadra} with $\eta_t = 1+\delta_t$ and the derivation of equation \eqref{eq:lambda_t_condi} to achieve
\begin{align}\label{eq:lambda_t_condi_BFGS}
    \lambda_{t+1} \leq (1+M\lambda_t)^2\left(\delta_t+\frac{M\lambda_t}{2}\right)\lambda_t.
\end{align}
According to Lemma~\ref{lm:BFGSlinear}, we have 
\begin{align}
\label{eq:lambda_linear_condi_BFGS}
    \lambda_{t}\leq \left(1-\frac{1}{2\eta_0}\right)^{t}\lambda_0.
\end{align}
According to Lemma~\ref{lm:GHneq_bfgs} with $\G=\G_0$,  $\H=\nabla^2 f(\x_0)$, $\eta=\eta_0$, and the initial condition (\ref{eq:bfgsini}), we have 
\begin{align}
\label{eq:theta_0_condi_BFGS}
    \delta_0=\sigma_{\nabla^2 f(\x_0)}(\G_0) \overset{\eqref{eq:GHneq_2_bfgs}}{\leq} d(\eta_0-1)\qquad\text{and}\qquad\delta_0+2dM\lambda_0\overset{\eqref{eq:bfgsini}}{\leq} d\eta_0.
\end{align}
Then we apply Lemma~\ref{lm:superlinear} 
on the random sequences $\{\lambda_t\}$ and $\{\delta_t\}$ with
$c_1=M$, $c_2=M/2$, $c_3=2dM$, $\alpha=d\varkappa/k$, $\beta={2\eta_0}$, and $s=d\eta_0$ to finish the proof, where we can verify conditions \eqref{eq:supercondi} and \eqref{eq:supercondi_2} by equations \eqref{eq:delta_t_condi_BFGS}, \eqref{eq:lambda_t_condi_BFGS} and \eqref{eq:lambda_linear_condi_BFGS}, \eqref{eq:theta_0_condi_BFGS} respectively.
% Hence, the random sequences of $\{\lambda_t\}$ and $\{\delta_t\}$ satisfy the conditions of Lemma~\ref{lm:superlinear} with
% $c_1=M$,$c_2=2 dM$, $c_3=2 dM$, $\alpha=d\varkappa/k$, $\beta={2\eta_0}$, and $s=d\eta_0$,
% % \begin{align*}
% % c_1=M,\quad c_2=2 dM,\quad c_3=2 dM,\quad
% % \alpha=\frac{d\varkappa}{k},\quad
% % \beta={2\eta_0}, \quad\text{and}\quad s=\eta_0 d,
% % \end{align*}
% which means we have proved Theorem~\ref{thm:BFGS}.
% \textcolor{blue}{(\eqref{eq:delta_t_condi_BFGS} and \eqref{eq:lambda_t_condi_BFGS} means condition~\eqref{eq:supercondi} is satisfied, \eqref{eq:lambda_linear_condi_BFGS} and \eqref{eq:theta_0_condi_BFGS} means condition~\eqref{eq:supercondi_2} is satisfied.)}
\end{proof}


\section{The proof of Theorem \ref{thm:fasterBFGS}}
\label{app:fasterBFGSproof}
\begin{proof}
We define $\delta_t\triangleq\sigma_{\nabla^2 f(\x_{t})}(\G_{t})$ and $\lambda_t\triangleq \lambda(\x_t)$.   
The proof of this theorem can follow the counterpart of Theorem~\ref{thm:BFGS} by using  $\tilde{\LL}_{t}^{\top}\U_t$ to replace $\mU_t$ (line \ref{line:fastBFGS} in Algorithm~\ref{alg:fasterbfgs}). 
This leads to the faster rate for matrix approximation, i.e., applying Theorem \ref{thm:bfgs-1} with $\G=\tilde{\G}_t$, $\A = \nabla^2 f(\x_{t+1})$, and $\mU=\mU_t$ to achieve
% The proof of this theorem is almost identical to the counterpart of Theorem~\ref{thm:BFGS}. 
% The only difference is that we use directions $\tilde{\LL}_{t}^{\top}\U_t$ to replace $\mU_t$ (line \ref{line:fastBFGS} in Algorithm~\ref{alg:fasterbfgs}) to achieve the faster rate for matrix approximation, i.e., applying Theorem \ref{thm:bfgs-1} with {\color{red} $\G=\tilde{\G}_t$, $\A = \nabla^2 f(\x_{t+1})$}, and $\mU=\mU_t$ to achieve
\begin{align}\label{eq:fasterBFGSconverge1}
    \EB_t[\delta_{t+1}] = \EB_t[\sigma_{\nabla^2 f(\x_{t+1})}(\G_{t+1})] \overset{\eqref{eq:fasterbfgssigma-1}}{\leq} \left(1-\frac{k}{d}\right)\sigma_{\nabla^2 f(\x_{t+1})}(\tilde{\G}_t).
\end{align}
We follow the proof of Theorem~\ref{thm:BFGS}
by replacing equation~\eqref{eq:BFGSconverge1} with  \eqref{eq:fasterBFGSconverge1} and replacing all the terms of $\varkappa d$ with $d$ to finish the proof of Theorem \ref{thm:fasterBFGS}.
% The random sequences of $\{\lambda_t\}$ and $\{\delta_t\}$ satisfy the conditions of Lemma~\ref{lm:superlinear} with
% \begin{align*}
% c_1=M,\quad c_2=2 dM,\quad c_3=2 dM,\quad
% \alpha=\frac{d}{k},\quad
% \beta={2\eta_0}, \quad\text{and}\quad s=d\eta_0,
% \end{align*}
% which means we have proved Theorem~\ref{thm:fasterBFGS}.
\end{proof}


%
\section{Extension for Solving Nonlinear Equations}
\begin{algorithm}[t]
\caption{Symmetric Rank-$k$ Method for Nonlinear Equation}\label{alg:SRK-NE}
\begin{algorithmic}[1]
\STATE \textbf{Input:} $\H_0$, $M$ and $k$. \\[0.15cm]
\STATE \textbf{for} $t=0,1\dots$\\[0.15cm]
\STATE \quad  $\z_{t+1}=\z_t-\H_t^{-1}\J(\z_t)^{\top}  \F(\z_t)$ \\[0.15cm]
\STATE \quad $r_t=\|\z_{t+1}-\z_{t}\|_2$ \\[0.15cm]
\STATE \quad $\tilde{\H}_{t}=(1+Mr_t)\H_t$ \\[0.1cm]
\STATE \quad construct $\U_t$ by $\left[\U_{t}\right]_{ij}\overset{\rm{i.i.d}}{\sim} {\fN(0,1)}$ \\[0.15cm]
\STATE \quad  $\H_{t+1}= \srk(\tilde{\H}_t, \J(\z_{t+1})^\top\J(\z_{t+1}),\U_t)$\\[0.15cm]
\STATE \textbf{end for}
\end{algorithmic}
\end{algorithm}

In this section, we apply \srk~methods to solve the nonlinear equations
\begin{align}
\label{eq:NE}
    \F(\z) =\0,
\end{align}
where $\F:\RB^{d}\to\RB^{d}$ is a differentiable vector-valued function. We use $\J(\vz)$ to represent the Jacobian of $\F(\cdot)$ at $\z\in\RB^d$ and impose the following assumptions.

\begin{assumption}
\label{ass:NElip}
We assume the vector-valued function
$F:\RB^{d}\to\RB^d$ is differentiable and its Jacobian is $\tilde L_2$-Lipschitz continuous, i.e., there exists some $\tilde L_2\geq 0$ such that 
\begin{align}
    \|\J(\z)-\J(\z')\|\leq \tilde L_2\|\z-\z'\|.
\end{align}
for any $\vz,\vz'\in\BR^d$.
\end{assumption}

\begin{assumption}
\label{ass:nonde}
We assume there exists equation (\ref{eq:NE}) has a solution $\z^*$ such that $\J(\z^*)$ is non-degenerate.
\end{assumption}

According to Assumption~\ref{ass:nonde}, we denote
\begin{align*}
    \tilde\mu \triangleq \frac{\sigma_{\min}(\J(\z^*))}{\sqrt{2}},\qquad \tilde L \triangleq 2\sigma_{\max}(\J(\z^*)) \qquad \text{and} \qquad {\tilde\kappa}\triangleq \frac{\tilde L}{\tilde \mu},
\end{align*}
where $\sigma_{\min}(\cdot)$ and $\sigma_{\max}(\cdot)$ are the smallest and the largest singular values of given matrix respectively. 

We present \srk~methods for solving nonlinear equations in Algorithm~\ref{alg:SRK-NE}.
The design of this algorithm is inspired by the recent work of~\citet{liu2022quasi}, which applies the quasi-Newton methods to estimate the information of non-degenerate indefinite matrix by its square.
We use the Euclidean norm $\tilde\lambda(\z)\triangleq\norm{\F(\z)}$ to measure the convergence of our algorithm. 
The advantage of block updates in \srk~updates results a faster superlinear convergence than~\citet{liu2022quasi}'s methods.
Following the analysis of \srk~methods for convex optimization, we obtain the results for solving nonlinear equations as follows.


\begin{theorem}
\label{thm:srk-NE}
Under {Assumption \ref{ass:NElip} and \ref{ass:nonde}}, we run Algorithm~\ref{alg:SRK-NE} with $k<d$, $\tilde M={2\tilde{\varkappa}^2 \tilde 
 L_2}/{\tilde L}$ and set the initial~$\vz_0$ and $\H_0$ such that
\begin{align*}
    \tilde \lambda(\z_0)\leq \frac{\ln 2}{8}\cdot \frac{ (d-k)\tilde{\mu}}{\tilde  M\eta_0 d^2{\tilde \varkappa}^2} 
    \qquad \text{and} \qquad 
    \J(\z_0)^{\top}\J(\z_0)\preceq\H_0\preceq \eta_0\J(\z_0)^{\top}\J(\z_0)
\end{align*} 
for some $\eta_0\geq 1$. 
Then we have
\begin{align*}
     \EBP{\frac{\tilde{\lambda}({\z}_{t+1})}{\tilde{\lambda}(\z_t)}}\leq 2d\tilde{\varkappa}^2\eta_0\left(1-\frac{k}{d}\right)^{t}.
\end{align*}

\end{theorem}

\bibliographystyle{plainnat}
\bibliography{ref}
 
\end{document}


