\section{Results}
\label{sec:results}

We present the results of our analysis by organizing the papers according to the robustness-related data property they discuss:  % (Sections~\ref{sec:results-number-of-samples}-~\ref{sec:results-domain-specific}).
number of samples,
dimensionality,
type of distribution,
density,
concentration,
separation,
label quality, and
domain specific properties.
Papers that discuss more than one data property are presented in all corresponding sections.
That is, in what follows, a paper can be discussed in more than one section.
Section~\ref{sec:results-summary} summarizes our findings.  
%Furthermore, we define each non-trivial data property under its corresponding section before presenting our findings.

To ease navigation, for each discussed data property,
we also include a map showing how the relevant papers relate to each other via their citation information.
We further annotate each paper with its \emph{applicability} and \emph{explainability} categories.
Specifically,
%we denote with the \roundrect{C} symbol papers that present a correlation between a data property and adversarial robustness;
we annotate with an \roundrect{A} symbol papers that propose an actionable technique to modify or measure a robustness-related property;
we annotate with \roundrect{E} papers that put extra emphasis on explaining the correlation between a data property and robustness rather than establishing such a correlation.
%\ad{In each section, we also categorized papers by their topics, and annotate with \roundrect{\#} in the figures to indicate the grouping.}

\subsection{Number of Samples}
\label{sec:results-number-of-samples}

\emph{Number of samples} simply means the quantity of samples available in the training dataset.
For the example in Fig.~\ref{fig:samples_guideline}, where circles represent training samples for a two-class dataset,
the left dataset has fewer samples than the right dataset.

The term \emph{sample complexity} refers to the number of training samples required to achieve a certain model performance,
e.g., 90\%, in terms of either robust or standard generalization.
Then, \emph{sample complexity gap} refers to the difference in the number of samples required to achieve the same model performance for robust generalization as for standard generalization.

\begin{figure*}[b!]
  \centering
  \includegraphics[width=0.5\linewidth]{images/number_of_samples_overview_no_distr_info.pdf}
  \vspace{-0.2in}
  \caption{Number of samples illustration.}
  \label{fig:samples_guideline}
  \vspace{-0.1in}
\end{figure*}


\begin{figure*}[t!]
	\centering
	\includegraphics[width=0.9\linewidth]{images/number_of_samples_citation_graph.pdf} 
	\vspace{-0.1in}
	\caption{Papers discussing the number of samples.}
	\label{fig:samples}
	\vspace{-0.15in}
\end{figure*}


Papers studying the relationship between the number of training samples and the robustness of the resulting model
are shown in Fig.~\ref{fig:samples}. They can roughly be divided into
\roundrect{1} papers discussing sample complexity for robust generalization,
\roundrect{2} papers proposing techniques to resolve the sample complexity gap between the number of samples required to achieve the same level of robust and standard generalization, and
\roundrect{3} papers proposing techniques to deal with data imbalance, i.e., an unequal number of samples in different classes.


\noindent
\roundrect{1} {\bf Sample Complexity}.
Schmidt et al.~\cite{Schmidt:Santurkar:Tsipras:Talwar:Madry:NeurIPS:2018} observe that the number of training samples required for robust generalization is larger than the number of samples required for the equivalent-level standard generalization, i.e., that there exists a \emph{sample complexity gap} between the standard and robust generalization.
Specifically, for linear classifiers trained on a mixture of Gaussian distributions
(referred to as \emph{Schmidt's Gaussian mixture} in the remainder of this paper),
the authors prove that standard generalization requires a constant number of samples while
equivalent-level robust generalization requires a number of samples proportional to the data dimensionality
($O(\sqrt{d})$).
The gap in sample complexity persists for this data distribution in nonlinear classifiers as well.
Yet, the sample complexity gap disappears for nonlinear classifiers trained on a mixture of Bernoulli distributions;
these distributions also need substantially fewer samples than Gaussian mixtures.
The authors conclude that sample complexity for robust generalization depends on the distribution,
even when the same type of classifiers is considered.
Their experimental validation with the \mnist~\cite{MNST:1998}, \cifarten~\cite{CIFAR:ten:hundred:2009}, and \svhn~\cite{SVHN:dataset:2011} image datasets shows that \mnist, which is closer to a Bernoulli mixture, indeed requires a smaller number of training samples to achieve a reasonable robust generalization 
than the \cifarten and \svhn datasets, which are \mbox{closer to a Gaussian mixture}.


In follow-up work, Dan et al.~\cite{Dan:Wei:Ravikumar:ICML:2020} provide reasons for why robust generalization
requires more samples than standard generalization, focusing, again, on Gaussian mixture distributions.
Departing from the Signal-to-Noise ratio (SNR) metric that is based on the distance
between two Gaussian distributions and is known to capture the hardness of standard classification,
the authors propose a new Adversarial SNR (AdvSNR) metric,
defined as the minimum SNR for standard and adversarially perturbed data,
to capture the hardness of robust classification.
They then show that, given a dataset of a particular dimensionality,
the number of samples required to achieve the theoretically optimal, accurate classifier is
inversely proportional to SNR.
Likewise, the number of samples required to achieve the theoretically optimal, robust classifier
is inversely proportional to AdvSNR.
Because AdvSNR is never greater than SNR for a given dataset, it follows that
achieving the same robust generalization as standard generalization requires at least the same amount of samples.

Bhattacharjee et al.~\cite{Bhattacharjee:Jha:Chaudhuri:PMLR:2021} study the sample complexity gap for linear classifiers,
as a factor of data dimensionality (the number of features representing samples ) and 
separation (the distance between samples from different classes).
The authors show that the sample complexity gap is directly proportional to the dimensionality of the data when
the allowed perturbation radius of adversarial samples is similar to the distance between classes.
%That is, in this setup, the higher the dimensionality of the data, the more samples are needed for robust generalization.
However, such a gap no longer exists in well-separated data, when the perturbation radius is much smaller than 
\mbox{the distance between classes}.


Similarly, Gourdeau et al.~\cite{Gourdeau:Kanade:Kwiatkowska:Worrell:JMLR:2021, Gourdeau:Kanade:Kwiatkowska:Worrell:IJCAI:2022} show that,
for simple classifiers based on feature conjunctions and $\alpha$-log-Lipschitz distributions laying on a boolean hyper-cube, 
the sample complexity is a function of the data dimensionality $d$ and the adversarial perturbation budget.
Specifically, when the adversarial perturbation size is bounded by $log (d)$, the sample complexity is polynomial to the dimensionality;
when the perturbation size is at least $log (d)$, the sample complexity becomes superpolynomial to dimensionality.
Javanmard et al.~\cite{Javanmard:Soltanolkotabi:Hassani:COLT:2020} focus on adversarially-trained linear regression models
for standard Gaussian distributions.
The authors show that, when the number of samples is greater than the data dimensionality,
there exists a trade-off between adversarial and standard risks.
Moreover, this trade-off improves as the number of samples per \mbox{dimension increases}.


Cullina et al.~\cite{Cullina:Bhagoji:Mittal:NeurIPS:2018} give an upper bound on the number of samples needed
for robust generalization for the binary classification problem with linear classifiers in a distribution-agnostic setup,
with $L_p$ norm-bounded adversaries.
The authors derive the upper bound using the classifier VC dimension~--
a common measure of the capacity and the expressive power of the classifier, shown earlier
to be useful to determine the upper limit of sample complexity for standard generalization~\cite{Shalev-Shwartz:Ben-David:2014}.
They show that the VC dimension for learning adversarially-robust models remains the same as that for learning accurate models, which means that the upper bound of sample complexity is identical for standard and robust generalization
in this setup.
However, the authors demonstrate that this conclusion does not generalize to other types of classifiers and types of adversaries.


Similarly, Montasser et al.~\cite{Montasser:Hanneke:Srebro:NeurIPS:2022} study binary classifiers
constructed using 
the one-inclusion graph algorithm~\cite{Haussler:Littlestone:Warmuth:IC:1994}. 
They show that both the lower and the upper bound of sample complexity is finite, i.e., 
one can achieve robust generalization using a finite number of samples in this setup.

Xu and Liu~\cite{Xu:Liu:NeurIPS:2022} study sample complexity bounds in a multi-class setup. 
As VC dimension is defined only for the binary case, the authors propose 
Adversarial Graph dimension and Adversarial Natarajan dimension metrics, 
which extend their corresponding counterparts, 
Graph dimension~\cite{BenDavid:CesaBianchi:Haussler:Long:JCSS:1995} and Natarajan dimension~\cite{Natarajan:ML:2004}, 
commonly used in multi-class learning. 
The authors show that sample complexity is upper-bounded by the former and lower-bounded by the latter metric. 
 
\vspace{0.05in}
\noindent
\roundrect{2} {\bf Resolving the Sample Complexity Gap}.
As the number of labeled samples required to achieve robust generalization could be large and not readily available,
researchers explore cheaper alternatives, such as, unlabeled data and generated (fake) data.
Uesato et al.~\cite{Uesato:Alayrac:Huang:Stanforth:Fawzi:Kohli:NeurIPS:2019} and
Carmon et al.~\cite{Carmon:Raghunathan:Schmidt:Duchi:Liang:NeurIPS:2019}
concurrently proposed to use pseudo-labeling~\cite{Scudder:ITIT:1965}~--
a process of assigning labels to unlabeled samples using a classifier trained on a set of labeled samples,
assessing the effectiveness of their approaches on Schmidt's Gaussian mixture.
The main result of both works is that closing the sample complexity gap requires a number of unlabeled samples
proportional to the dimensionality of the data, albeit with a higher quantity than for the labeled samples,
likely due to the ``noise'' in generating labels.
The main difference between the works is that while Uesato et al. show that in their setup (a specific linear classifier)
the quantity of the required unlabeled samples only depends on data dimensionality,
Carmon et al.~\cite{Carmon:Raghunathan:Schmidt:Duchi:Liang:NeurIPS:2019} use a less restrictive setup and show that the quantity of unlabeled samples also depends on the original sample complexity for standard generalization.
Both of these works empirically evaluate the effectiveness of their proposed approaches on the \cifarten and \svhn datasets showing that unlabeled data could be a much cheaper alternative to labeled data for enhancing the robustness of models.


Najafi et al.~\cite{Najafi:Maeda:Koyama:Miyato:NeurIPS:2019}
note that the biggest risk of using a mixture of labeled and unlabeled datasets for learning adversarially robust models is
the uncertainty in sample labels.
Given an estimate of the quality of pseudo-labels,
the authors derive the minimum ratio between labeled and unlabeled samples required to avoid the additional
adversarial risk induced by label uncertainties. 

Instead of using unlabeled data, which might also be hard to find,
Gowal et al.~\cite{Gowal:Rebuffi:Wiles:Stimberg:Calian:Mann:NeurIPS:2021} suggest
using Generative Adversarial Networks (GANs) to generate labeled data.
The authors show that GANs are more effective than other methods, e.g., image cropping, when
producing additional samples.
This is because such models result in a more diverse dataset, which is beneficial for increasing robust accuracy.
Using images from \cifarten, \cifarhundred~\cite{CIFAR:ten:hundred:2009}, \svhn, and \tinyset~\cite{Torralba:Fergus:Freeman:TPAMI:TinyImageSet:2008}, the authors show that their proposed approach can significantly increase robust accuracy without the need for additional real samples.

Xing et al.~\cite{Xing:Song:Cheng:NeurIPS:2022} reason about the effects of real unlabeled and 
generated data on robustness of adversarially trained models. 
As real data is more informative for building decision boundaries, and as 
both real unlabeled and generated data need pseudo-labeling, 
the difference between real unlabeled and generated data boils down to the quality of data generators. 
Following this reasoning, the authors propose a strategy that assigns lower weights to the loss from generated samples compared to real samples during adversarial training, 
with the exact weights (i.e., the representation of the generator quality) being determined through cross-validation.

\vspace{0.05in}
\noindent
\roundrect{3} {\bf Effects of Data Imbalance}.
Wu et al.~\cite{Wu:Liu:Huang:Wang:Lin:CVPR:2021} analyze the adversarial robustness of DNNs on long-tail distributions: setups where the training data contains a large number of classes with few samples.
They show that robust generalization is harder to achieve on such distributions and compare the performance of
multiple adversarially trained classifiers that use learning algorithms specifically designed for such setups.
The comparisons show that scale-invariant classifiers~\cite{Wang:Wang:Zhou:Ji:Gong:Zhou:Li:Liu:CVPR:2018,Pang:Yang:Dong:Xu:Zhu:Su:NeurIPS:2020} result in higher robust accuracy as they
avoid assigning smaller weights to minority classes, which, in turn, promotes robust generalization by reducing bias in the decision boundary.

Both Wang et al.~\cite{Wang:Xu:Han:Xiaorui:Yaxin:Thuraisingham:Tang:ArrXiv:2021} and 
Qaraei et al.~\cite{Qaraei:Babbar:ML:2022}
propose to use re-weighted loss functions to improve robustness of under-represented classes.
Specifically, Wang et al. show that, 
for Gaussian mixture distributions, the robustness gap between classes depends on the amount of imbalance and the overall separation of a dataset. 
The authors thus propose modifying the loss function to assign weights correlated with data imbalance
while also promoting separation. 
Qaraei et al. focus on extreme multilabel text classification, where the output space is extremely large and the data follows a strongly imbalanced distribution. 
They also recommend using re-weighted loss functions to mitigate the robustness issue of the under-represented classes. 

\subsection{Dimensionality}
\label{sec:results-dimensionality}

\begin{figure*}[h]
\centering
  \begin{minipage}{0.55\textwidth}
  \centering
  \includegraphics[width=\linewidth]{images/dimensionality_overview_3d.pdf}
  \vspace{-0.25in}
  \caption{Dimensionality illustration.}
  \label{fig:dimensionality_illustration}
  \vspace{-0.1in}
  \end{minipage}
  \begin{minipage}{0.44\textwidth}
  \vspace{0.1in}
  \centering
    \subcaptionbox{Actual. \label{fig:intrinsic_dimensionality_original}}{
     \includegraphics[width=0.42\linewidth]{images/dimensionality_original_dim.pdf}
   }
   \subcaptionbox{Intrinsic. 	 \label{fig:intrinsic_dimensionality_intrinsic}}{
    \includegraphics[width=0.42\linewidth]{images/dimensionality_intrinsic_dim.pdf}
    }
  \vspace{-0.05in}
  \caption{Actual and intrinsic dimensionality.}
  \label{fig:intrinsic_dimensionality}
  \vspace{-0.1in}
  \end{minipage}
\end{figure*}

%\begin{figure*}[h]
%  \centering
% % \vspace{-0.1in}
%  \includegraphics[width=0.99\linewidth]{images/dimensionality_vertical_legend_wide_updated.pdf}
%  \vspace{-0.1in}
%  \caption{Papers discussing dimensionality. }
%  \label{fig:dimensionality}
%  \vspace{-0.1in}
%\end{figure*}

\begin{figure*}[h]
  \centering
 % \vspace{-0.1in}
  \includegraphics[width=0.99\linewidth]{images/dimensionality_citation_graph.pdf}
  \vspace{-0.1in}
  \caption{Papers discussing dimensionality. }
  \label{fig:dimensionality}
  \vspace{-0.1in}
\end{figure*}

\emph{Dimensionality} refers to the number of features used to represent the data, e.g.,
features $f_1$, $f_2$, $f_3$ in Fig.~\ref{fig:dimensionality_illustration}.  %, \ldots, f_d$
For illustration purposes, we show a dataset with a dimensionality of three on the left-hand side of the figure and
a dataset with a dimensionality of one on the right-hand side.
% as in Figure~\ref{fig:IntrinsicDimensionality}(b) for three-dimensional data.
\emph{Intrinsic dimensionality} refers to the number of features used in a minimal representation of the data.
Fig.~\ref{fig:intrinsic_dimensionality} shows an example of a case where the intrinsic dimensionality
is smaller than the actual dimensionality:
the samples in Fig.~\ref{fig:intrinsic_dimensionality}a are lying on a three-dimensional ``swiss roll''.
``Unwrapping'' the roll into a plain sheet, as shown in Fig.~\ref{fig:intrinsic_dimensionality}b,
makes it possible to distinguish between the samples using only two dimensions.



Papers studying the relationship between data dimensionality and adversarial robustness are shown in Fig.~\ref{fig:dimensionality}.
We divide them into papers
\roundrect{1} characterizing the hardness of robust generalization due to high dimensionality, % under specific assumptions of the dataset or classifier chosen,
\roundrect{2} suggesting robust model types and configurations for high-dimensional data,
\roundrect{3} discussing the impact of high dimensionality on existing defense techniques, and
\roundrect{4} utilizing dimensionality reduction techniques for improving robustness.
%Several works correlate adversarial vulnerability to high dimensionality of the input data.
%To analyze this phenomenon, some works base their study under specific assumptions about the data distribution while others make assumptions about the classifier or the defense mechanism chosen.

\vspace{0.05in}
\noindent
\roundrect{1} {\bf Effects of Dimensionality}.
A number of authors show that adversarial examples are inevitable in high-dimensional space.
Specifically, Gilmer et al.~\cite{Gilmer:Metz:Faghri:Schoenholz:Raghu:Wattenberg:Goodfellow:ICLR:2018} prove
this for a synthetic binary dataset composed of two concentric multi-dimensional spheres
(a.k.a., hyperspheres) in high-dimensional space ($>$100), showing that samples are, on average, closer to their nearest adversarial examples than to each other.
They also prove that the adversarial risk
of a model trained on this dataset only depends on its standard accuracy and dimensionality.
A similar result is shown by Diochnos et al.~\cite{Diochnos:Mahloujifar:Mahmoody:2018},
for a uniformly distributed boolean hypercube dataset, and
Shafahi et al.~\cite{Shafahi:Huang:Studer:Feizi:Goldstein:ICLR:2019}, for unit-hypersphere and unit-hypercube datasets.
\revadd{De Palma et al.~\cite{DePalma:Kiani:Lloyd:ICML:2021} prove that irrespective of the model architecture, 
for a dataset with dimensionality $d$, the perturbation required to fool a classifier is inversely proportional 
to $\sqrt{d}$. This means that it gets easier to generate adversarial examples with an increase in dimensionality.}

%%of the data, and not on the learning algorithm.
%%
%%The authors show that, for a fixed standard accuracy, the upper bound of the perturbation required to generate adversarial examples is inversely proportional to $\sqrt{d}$.
%%Specifically, as dimensionality increases, the perturbation required to generate adversarial examples decreases.
%%derive the relationship among the error rate $\mu(E)$ of resulting model, the dimension $d$ of the dataset, and the adversarial robustness of the model, which is estimated by the average $L_2$ distance $d(E)$ between the samples and their closest error (i.e., misclassified when perturbed).
%%Furthermore, they note that adversarial examples are inevitable for high-dimensional data
%%, which they define as $d \geq 100$,
%%consistently successfully attacking models with high standard accuracy.
%%as the samples are, on average, closer to their closest adversarial examples than to each other.
%%the average distance $d(E)$ between the training samples and their closest adversarial examples to be smaller than the average distance between the training samples.
%%
%%
%Diochnos et al.~\cite{Diochnos:Mahloujifar:Mahmoody:2018} investigate a uniformly distributed boolean hypercube dataset and also show that the number of perturbations required to create an adversarial example depends on dimensionality.
%Specifically, for any classifier trained on such a dataset of dimensionality $d$, at most $\mathcal{O}{\sqrt{d}}$ perturbations are enough to significantly increase its adversarial risk.
%Following these works, Shafahi et al.~\cite{Shafahi:Huang:Studer:Feizi:Goldstein:ICLR:2019} investigate continuous unit-hypersphere and unit-hypercube datasets, also proving that adversarial examples are inevitable in high dimension.
%However, they suggest that class density is even a stronger predictor of adversarial robustness.
%Hence, low-dimensional datasets with high class density are favorable for training robust models.
%% They show that dimensionality and class density are both predictors of the resulting model's adversarial robustness.
%%Specifically, dimensionality negatively affects robustness under a constant upper bound for class density, and the upper bound of class density positively affects robustness under a constant dimensionality.
%% The asymtotic relation: accuracy follows $V_c * exp(-d * \epsilon^2)$ --> dimensionality worsen the adversarial robustness at an exponential rate
%%Their analysis show adversarial robustness to be negatively correlated with the dimensionality of the data distribution when the upper bound of class density is constant. %\gx{removed `only' before when, is it necessary?}
%%It is also shown that robustness is positively correlated with the upper bound of class density for the data distribution when dimensionality is constant.
%%Despite pointing out that dimensionality is a weaker predictor of robustness than class density, the authors find low-dimensional datasets with high class density favorable for training robust model.
%%\gx{They pointed this out in the intro, and also expermented with b-MNIST to show}

%\gx{Model specific analysis}

Another line of work analyzes the effect of dimensionality on the robustness of specific types of classifiers.
In particular,
%\jr{Shubhraneel-P38}
Simon-Gabriel et al.~\cite{Simon-Gabriel:Ollivier:Scholkopf:Bottou:Lopez-Paz:ICML:2019} study feedforward neural networks
with ReLU activation functions and He-initialized weights, %~\cite{He:Zhang:Ren:Sun:ICCV:2015}.
% \de{and satisfy the symmetry assumption, i.e., all nodes in the same layer have equal in-degrees and there are no skip-layer connections.}
%%\gx{I updated it because the symmetry assumption is only required to prove some additional results, the weight initialization scheme is the main assumption for the results regarding dimensionality.}
showing that a higher input dimensionality increases the success rate of adversarial attacks, regardless of the topology of the network.
%Specifically, their work suggests that adversarial damage, which measures the attack success rate on correctly classified samples, increases proportionally with $\sqrt{d}$.
The authors, however, demonstrate that regularizing the gradient norms of the network decreases the impact of the input dimension on adversarial vulnerability, thereby improving model robustness on high-dimensional inputs.
Daniely et al.~\cite{Daniely:Schacham:NeurIPS:2020} study the effect of dimensionality on ReLU networks
with random weights and with layers having decreasing dimensions.
Like Simon-Gabriel et al., the authors prove that the robustness of ReLU networks degrades proportionally to dimensionality.

%\jr{Shubhraneel-P37}
Amsaleg et al.~\cite{Amsaleg:Bailey:Barbe:Erfani:Furon:Houle:Radovanovic:Nguyen:TIFS:2021}
focus on $k$-NNs and other non-parametric models that base predictions on the proximity of samples.
The authors use the Local Intrinsic Dimensionality metric to represent the intrinsic dimensionality
in the neighborhood of a particular sample $x$.
The authors build up on the observation that when this metric is high, there are more samples in close proximity of $x$
(as, otherwise, a more sparse neighborhood could be encoded in fewer dimensions).
Thus,
%one can change the neighborhood ranking of samples surrounding $x$ with only a small perturbation,
it is possible to arbitrarily change the neighborhood ranking of the nearest neighbor of $x$ using a small perturbation.
As predictions of proximity-based models are based on the nearest neighbor ranking, the adversarial risks increase in this setup.

All the aforementioned works are also in agreement with a number of papers discussed
in Section~\ref{sec:results-number-of-samples},
i.e., ~\cite{Dan:Wei:Ravikumar:ICML:2020,Bhattacharjee:Jha:Chaudhuri:PMLR:2021,Gourdeau:Kanade:Kwiatkowska:Worrell:JMLR:2021},
which show, in their respective settings, that sample complexity for robust generalization is proportional to dimensionality.
%\jr{TODO: why the last two papers are not in Figure 11? They also talk about dimensionality. Same question for separation.}
% $d$, with a rate of $\sqrt{d}$.


%Amsaleg et al.~\cite{Amsaleg:Bailey:Barbe:Erfani:Furon:Houle:Radovanovic:Nguyen:TIFS:2021} show that the perturbation required to change the ranks of nearest neighbors for $k$-NN classifiers is inversely proportional to the Local Intrinsic Dimensionality (LID) of the data.
%LID characterizes intrinsic dimensionality in the neighborhood of a reference sample,
%i.e., a high LID indicates that a large number of samples lie near the reference sample.
%Specifically, they prove that, if the LID around a given sample is high, it is possible to arbitrarily change the order of its nearest neighbors using a small perturbation.
%This is relevant since arbitrarily changing the nearest neighbors of a given sample represents an adversarial attack scenario for $k$-NNs and similar non-parametric models.


%Focusing on intrinsic dimensionality,
%Amsaleg et al.~\cite{Amsaleg:Bailey:Barbe:Erfani:Furon:Houle:Radovanovic:Nguyen:TIFS:2021} show that the perturbation required to change the ranks of nearest neighbors for $k$-NN classifiers is inversely proportional to the Local Intrinsic Dimensionality (LID) of the data.
%LID characterizes intrinsic dimensionality in the neighborhood of a reference sample,
%i.e., a high LID indicates that a large number of samples lie near the reference sample.
%Specifically, they prove that, if the LID around a given sample is high, it is possible to arbitrarily change the order of its nearest neighbors using a small perturbation.
%This is relevant since arbitrarily changing the nearest neighbors of a given sample represents an adversarial attack scenario for $k$-NNs and similar non-parametric models.

\vspace{0.05in}
\noindent
\roundrect{2} {\bf Model Selection and Configuration}.
%\jr{Jaskeerat-P41}
Wang et al.~\cite{Wang:Jha:Chaudhuri:ICML:2018} prove that the optimal $k$ for producing robust $k$-NN classifiers depends on the dimensionality $d$ and number of samples $n$ of the given dataset ($k = \Omega(\sqrt{dn \text{ log}(n)})$).
However, they note that for high-dimensional data, the optimal $k$ might be too large to use in practice.
The authors thus focus on improving the robustness of 1-NN algorithms through sample selection,
showing the effectiveness of their approach on %that it improves the robustness of adversarially trained 1-NN on
the \halfmoon, \mnistv, and \abalone datasets.

Yin et al.~\cite{Yin:Kannan:Bartlett:ICML:2019} show that transferring a robust solution found on training data
to test data gets more difficult as the dimensionality of data increases.
However, constraining the classifier weights mitigates this problem.
Specifically, the authors prove that constraining the weights by $L_p$ norm, for $p > 1$, leads to
a performance gap between training and test data that has a polynomial dependence on dimensionality;
when the weights are constrained by $L_1$ norm, the performance gap has no dependence on dimensionality.
\revadd{Li et al.~\cite{Li:Jin:Zhong:Hopcroft:Wang:NeurIPS:2022} rather focus on model configuration.
The authors show that robust generalization in networks with ReLU activation requires  
the network size to be exponential in original and intrinsic data dimensionalities, 
even in the simplest case when the underlying distribution is linearly separable.}
%Furthermore, the same exponential relationship also exists for intrinsic data dimensionality.}

%\jr{Michael-P2}
Carbone et al.~\cite{Carbone:Wicker:Laurenti:Patane:Bortolussi:Sanguinetti:NeurIPS:2020} study neural networks,
showing that adversarial vulnerability arises due to the gap between the actual and intrinsic dimensionality,
a.k.a., degeneracy.
The authors show that adversarial example generations in high-dimensional degenerate data can be performed by
using gradient information of a neural network, to move the samples in the direction normal to the data manifold.
As such, example generation exploits the additional dimensions without changing the ``semantics'' of the perturbed sample.
The authors then show that Bayesian Neural Networks are more robust than other neural networks to gradient-based attacks:
due to their randomness, they make gradients less effective for crafting attacks.
%The authors validated their findings using the \mnist and \fmnist~\cite{Xiao:Rasul:Vollgraf:ArXiv:FashionMNIST:2017} datasets.

%<Randomized smoothing> is not robust to <high dimensoinality>
%\jr{Michael-P4}
\vspace{0.05in}
\noindent
\roundrect{3} {\bf Effects of Dimensionality on Defense Techniques}.
High dimensionality also poses challenges to defense techniques that aim to improve robustness.
%, such as,
%randomized smoothing, %which results in certifiably robust classifier by first creating multiple noisy instances of the input and aggregating their classification results,
%adversarial training, and %which learns robust classifier by iteratively optimizing with respect to adversarial examples generated on-the-fly,
%data augmentation. % which improves resulting model's robustness by augmenting the training set.
%
Specifically, Blum et al.~\cite{Blum:Dick:Manoj:Zhang:JMLR:2020} focus on randomized smoothing~--
a technique that improves robustness by generating noisy instances of a (possibly perturbed) sample
and then making predictions for the sample based on an aggregation of predictions for its noisy instances.
The authors show that the amount of noise required to defend against $L_{p}$ adversaries, for $p > 2$,
is proportional to dimensionality.
They further demonstrate that, for high-dimensional images, randomized smoothing indeed fails to generate instances that preserve semantic image information.
In a similar line of work, Kumar et al.~\cite{Kumar:Levine:Goldstein:Feizi:ICML:2020} show that the certified radius decreases
as the dimensionality increases when using randomized smoothing for certifying robustness for a given $L_{p}$ radius.
% a similar result:
%for $p > 2$, the largest $L_{p}$ radius that can be certified with randomized smoothing decreases inversely with $d^{(1/2-1/p)}$ as dimensionality $d$ increases.
%% for $L_1$ the largest radius is O(1/d) and for $L_\infty$ the largest radius is O(d^{(1-1/p)}}

Adversarial training~-- a defense technique that improves model robustness by adaptively training a model
against possible adversarial examples~-- often incurs a trade-off between standard and adversarial accuracy~\cite{Tsipras:Santurkar:Engstrom:Turner:Madry:ICLR:2019, Zhang:Yu:Jiao:Xing:Ghaoui:Jordan:ICML:2019}:
optimizing for high robust accuracy results in a drop in standard accuracy and vice versa.
%%\jr{Gabby-P58} %\jr{Gabby-P15}
%For instance,
Mehrabi et al.~\cite{Mehrabi:Javanmard:Rossi:Rao:Mai:ICML:2021} build up on the work of Javanmard et al.~\cite{Javanmard:Soltanolkotabi:Hassani:COLT:2020},
discussed in Section~\ref{sec:results-number-of-samples}, which 
showed that, for a finite number of training samples, % with the number of samples greater than data dimensionality,
the trade-off between adversarial and standard accuracy improves as the number of samples per dimension increases.
Mehrabi et al. further extend this result for unlimited training data and computational power, observing that,
for an unlimited number of training samples, the trade-off between adversarial and standard accuracy improves as the dimension of the data decreases.

%Different from Javanmard et al., they consider distributional adversarial training, which perturbs the entire training data distribution instead of perturbing individual instances.
%Nonetheless, they observe a similar trend as Javanmard et al.~-- the trade-off improves as the dimension of the data decreases.
%\gx{Should we mention their other non-dimensionality related findings here?}
%In addition, they find feature dependency (i.e., correlation between features) to have a more nuanced influence on the trade-off for both Linear Regression and Binary Classification.
%Specifically, they find that increasing feature dependency worsens the trade-off upto a certain value for feature dependency, i.e., if one increases feature dependency beyond this point, the trade-off starts to improve.


%	\item ``Fundamental Tradeoffs in Distributionally Adversarial Training~\cite{Mehrabi:Javanmard:Rossi:Rao:Mai:ICML:2021}''
%	theoretically studied the fundamental trade-off (given unlimited computational power and training data) between standard and adversarial risk for binary classification with adversarial training.
%	Showed the trade-offs depends on adversary's power, feature dimension, and features correlation.
%	\textbf{Include},  as it quantitatively discusses the trade-off in terms of properties of data (e.g., feature dimension, feature correlations).
%	\textbf{Actionable Item}: this paper cited Javanmard et al.~\cite{Javanmard:Soltanolkotabi:Hassani:COLT:2020}, which discussed on the tradeoffs in adversarial learning for linear regression (currently identified as NotSure), check how their conclusions compare to each other.

%\jr{Shubhraneel-P33}
Data augmentation is another common defense technique that aims to improve robustness of a model by creating
perturbed samples at radius $r$ from a certain subset of original samples in training data.
Rajput et al.~\cite{Rajput:Feng:Charles:Loh:Papailiopoulos:ICML:2019}
prove, for linear and certain nonlinear classifiers, that the number of augmentations required for robust generalization depends on the dimensionality of data, % and constraints on the perturbation radius,
i.e., it is at least linearly proportional to dimensionality for any fixed radius $r$.
Thus, data augmentation becomes more expensive for high-dimensional data.

%no constraint on $r$ --> $d+1$
%$r$ is restricted --> at least linearly proportional to $d$
%$r$ is restricted and the perturbation is smaller than the desired margin --> exponential in $d$
%
%They show that, for linear classifiers on a linearly separable $d$-dimensional dataset, $d+1$ augmented data points suffice to achieve the optimal margin given no constraint on the perturbation radius $r$.
%When the radius of perturbation $r$ is restricted, the number of augmentations required is at least linearly proportional to $d$. %and the number of original samples.
%In particular, if the perturbation is smaller than the desired margin, the number of augmentations required is exponential in $d$.
%The authors also extend their results to non-linear classifiers that assign the same label to all data points within small convex hulls (e.g., nearest neighbor classifiers) on both linearly and nonlinearly separable datasets.

\vspace{0.05in}
\noindent
\roundrect{4} {\bf Reducing Dimensionality}.
%\jr{Gabby-P19}
Following the idea that the gap between the actual and intrinsic dimensionality contributes to adversarial vulnerability,
Awasthi et al.~\cite{Awasthi:Jain:Rawat:Vijayaraghavan:NeurIPS:2020} propose to use Principal Component Analysis (PCA)~\cite{Jolliffe:2002}
%a representation learning technique that reduces dimensionality,
to decrease the dimensionality of data before applying randomized smoothing.
As a result, a larger amount of noise can be injected to perturb samples, thus improving robustness without compromising accuracy.
%The large amount of noise leads to classifier that is less sensitive to perturbations, and hence achieves higher certified accuracy across a wide range of radii.
The authors apply the proposed ideas to image data, showing that the combination of PCA and randomized smoothing is more beneficial than using randomized smoothing alone.
%images can be compressed with little information loss using only one fifth of their original dimension.
%the actual dimension required to represent the information on natural images is much smaller than what is currently used.
%
%This is in line with the concept of degeneracy introduced by Carbone et al.~\cite{Carbone:Wicker:Laurenti:Patane:Bortolussi:Sanguinetti:NeurIPS:2020}.
%They show that one can compress images with little information loss using only one fifth of its original dimension.
%Based on such an observation, they propose to use Principal Component Analysis (PCA), a representation learning technique that reduces dimensionality, to project data from its original dimension $d$ to a lower dimension $r$ prior to applying adversarial training with randomized smoothing.
%As a result, a magnitude of $\sqrt{d/r}$ larger amount of noise can be injected during the adversarial training process, and a larger radius for certified robustness can be achieved.
%
%Decrease <dimensionality> by <Euclidean to Hyperbolic space transformation> For Heirarchical data
%\jr{Jaskeerat-P39}
Weber et al.~\cite{Weber:Zaheer:Rawat:Menon:Kumar:NeurIPS:2020} show, for hierarchical data,
that changing the representation from Euclidean to hyperbolic space reduces the dimensionality
without sacrificing semantic information embedded in the input data.
%\ad{This, in turn, allows for more efficient training of robust \de{large-margin} classifiers.} \jr{how margins are relevant here. Why robust is not enough.} \gx{Even though their main message is supposed to hold for all classifiers, they only validated with large-margin classifiers. }

%They note that this change in representation is practically a geometrical transformation that reduces the dimensionality of hierarchical data without sacrificing its separability or semantic information, which leads to improved robustness.
%They further provide an algorithm to learn a large-margin classifier in the hyperbolic space.
%Their experiments show that their algorithm achieves superior performance in hyperbolic
%space at significantly lower dimensions. 
\subsection{Distribution}
\label{sec:results-distribution}

\begin{figure*}[h]
\centering
%\vspace{-0.1in}
  \begin{minipage}{0.52\textwidth}
  \centering
  \vspace{0.2in}
  \includegraphics[width=\linewidth]{images/distribution_gaus_bern_illustration.pdf}
  \vspace{-0.22in}
  \caption{Distribution illustration.}
  \label{fig:distribution_illustration}
  \vspace{-0.1in}
  \end{minipage}
  \hspace{0.1in}
  \begin{minipage}{0.44\textwidth}
	\centering
	\vspace{-0.1in}
	\includegraphics[width =\textwidth]{images/distribution_citation_graph.pdf}
	\vspace{-0.25in}
    \caption{Papers discussing distribution.}
  \label{fig:distribution}
  \end{minipage}
  \vspace{-0.1in}
\end{figure*}

\emph{Distribution} refers to a function that encodes how samples lie in space, usually by giving the probabilities of their occurrence in particular regions.
Common types of distributions, such as uniform, Bernoulli, and Gaussian
are introduced in Section~\ref{sec:background_distribution}.
Fig.~\ref{fig:distribution_illustration} shows examples of datasets that follow a Gaussian distribution (left) and a Bernoulli distribution (right).
The term \emph{variance} refers to a measure of dispersion that takes into account the spread of all data points in a dataset.
Specifically, the \emph{variance of a distribution} measures the dispersion of samples from the mean;
\emph{feature variance} measures the dispersion of samples over a particular feature only.
We say that a distribution satisfies \emph{symmetry} when distributions on either side of the mean mirror each other.


Papers that discuss how distribution properties, including variance and symmetry, influence models' robustness
are shown in Fig.~\ref{fig:distribution}.
They can be categorized into:
\roundrect{1} papers showing that model robustness depends on the underlying data distribution,
%Fawzi et al.  Ding et al.
\roundrect{2} papers identifying properties of distributions that improve robustness, and
 %Izmailov et al, Lee et al. , Richardson et al.  and
\roundrect{3} papers introducing techniques to transform distributions into ones that are more optimal for robustness.
% Pang et al. and Wan et al.
%Fawzi et al.  show data distribution with some characteristics makes it impossible for training adversarially robust classifiers,
%Ding et al.  show different distributions results in different adversarial robustness

\vspace{0.05in}
\noindent
\roundrect{1} {\bf Types of Distributions}.
%Some works show that some data distributions are more robust than others.
As discussed in Section~\ref{sec:results-number-of-samples},
Schmidt et al.~\cite{Schmidt:Santurkar:Tsipras:Talwar:Madry:NeurIPS:2018} prove, for nonlinear classifiers,
that a mixture of Gaussian distributions incurs higher sample complexity
for robust generalization than a mixture of Bernoulli distributions.
%\jr{TODO: Update figure 13 (schmidt et al. citation number)?}
%\jr{Shubhraneel-P35}
Likewise, Ding et al.~\cite{Ding:Lui:Jin:Wang:Huang:ICLR:2019} show that
a distribution shift alone can affect robust accuracy while retaining the same standard accuracy.
Specifically, the authors prove that uniform data lying on a unit cube results in more robust models than
uniform data lying on a unit sphere.
They further experiment with \mnist and \cifarten datasets,
applying existing semantically lossless transformations, namely \emph{smoothing} and \emph{saturation},
to cause the distribution shift.
The results of this experiment show that robustness decreases gradually when transforming \mnist from
a unit-cube-like to a unit-sphere-like distribution and increases for \cifarten when going the opposite way;
in both cases, the models retain \mbox{their standard accuracy}.

Fawzi et al.~\cite{Fawzi:Fawzi:Fawzi:NeurIPS:2018} study the robustness of data distributions modeled by
a smooth generative model~-- a type of generative model that maps samples from input space to output
space while preserving their relative distances, e.g., to compress data.
% satisfying modulus continuity property~\cite{Anastassiou:Gal:2012}. \jr{?}
The authors show that smooth generative models with high-dimensional input space
produce data distributions that make any classifier trained on this data inherently vulnerable.
The authors conclude that non-smoothness and low input space dimensionality
are desirable when modeling data with generative models.
%This is because the robustness radius is directly proportional to the smoothness of the generative model, irrespective of dimensionality of the input space.
%It follows that, for a high-dimensional input space, the
%robustness radius becomes negligible compared to the magnitude of the inputs, and corresponds to adversarial vulnerability}
%\de{This vulnerability is directly proportional to the dimensionality of the input space, the smoothness of the generative model, among others.}
%\jr{why not in the dimensionality section as well?},
%\jr{because a smooth generative model is not a subject to an attack and thus its input space is not in scope for our survey.}

%The authors prove that any classifier trained on a dataset generated by a smooth generative model with high dimensional latent space  %from which input data points are sampled,
%is susceptible to small perturbations.

\vspace{0.05in}
\noindent
\roundrect{2} {\bf Properties of Distributions}.
Izmailov et al.~\cite{Izmaliov:Sugrim:Chadha:McDaniel:Swami:MILCOM:2018} show that, in a binary classification setting,
features with small variance in both classes and means close to each other
cause adversarial vulnerability.
Moreover, a feature with a small variance in one class can still cause vulnerability
even if the means of this feature in both classes are farther separated
but the second class has a larger feature variance.
Intuitively, that is because models tend to assign non-zero weights to such features,
which can be leveraged by attackers to shift the classification into the wrong class.
That is, even small perturbations in such features can shift data points to another class.
To increase robustness, the authors suggest removing such features, either based on domain knowledge
% (e.g., ignoring the pixels in the peripheral regions of images)
or based on feature evaluation metrics, such as, mutual information~\cite{Shannon:1949}.

%study different cases where a feature may contribute to adversarial vulnerability in a binary classification setting.
%In particular, they show that the features that have small variances and close means among classes cause adversarial vulnerability,  as small perturbations on these features suffice to shift data points from one class distribution to another.
%In addition, they demonstrate that vulnerability can arise from features that have separated means among classes if they have significantly different variances.
%In this case, the class with a smaller feature variance becomes more vulnerable, as equivalent perturbations may induce more significant changes in the feature values for that class, resulting in a higher chance of misleading predictions.
%To increase robustness, the authors suggest removing such features, either based on domain knowledge
%% (e.g., ignoring the pixels in the peripheral regions of images)
%or based on feature evaluation metrics, such as, mutual information~\cite{Shannon:1949}.

Similarly,
Lee et al.~\cite{Lee:Lee:Yoon:CVPR:2020} prove that decreasing feature variance in individual classes
can increase robustness for Schmidt's Gaussian mixtures.
These mixtures have equivalent feature variances for all classes and separated means.
In such a setting, low feature variance implies that the feature has a strong correlation with the class and
perturbing this feature will unlikely result in a vulnerability 
(i.e., will likely result in a semantically-meaningful change).
However, even when features have low variance, if these features are
non-robust~\cite{Ilyas:Santurkar:Tsipras:Engstrom:Tran:Madry:NeurIPS:2019}, i.e., hold no semantic information,
and have a smaller variance in the training data than
in the underlying true population,
%adversarially trained models tend to overfit to them,
they will still cause adversarial vulnerability as adversarially trained models tend to overfit to them.
As a countermeasure, the authors propose a label-smoothing-based data augmentation technique which uses
continuous instead of discrete values for labels and acts like a regularization method that prevents the model from overfitting to such features.
%\jr{features with low variance? or also 2 and 3?
%1) low variance
%2) non robust
%3) different from true population


\begin{wrapfigure}{r}{0.30\textwidth}
  \vspace{-0.35in}
  \begin{center}
    \includegraphics[width=0.26\textwidth]{images/gaussianMixFig.pdf}
    %includegraphics[scale = 0.2]{images/gaussianMixFig.pdf}
  \end{center}
    \vspace{-0.15in}
  \caption{Asymmetrical dataset.}
  \label{fig:asymetric_data}
  \vspace{-0.1in}
\end{wrapfigure}
Richardson and Weiss~\cite{Richardson:Weiss:JMLR:2021} claim that adversarial vulnerability can be caused by
sub-optimal data distributions and/or sub-optimal training methods.
The authors define synthetic binary datasets (of images) that use Gaussian distributions with separated means
and say that a dataset is symmetric if and only if classes have the same variance.
They further prove that even an optimal classifier is non-robust when the underlying dataset has strong asymmetry,
as in the
example in Fig.~\ref{fig:asymetric_data}.
If the dataset is symmetric the optimal classifier is provably robust,
even though a sub-optimal training method can still cause vulnerability when trained on this dataset.


\vspace{0.05in}
\noindent
\roundrect{3} {\bf Transforming Distributions}.
%Pang et al. and Wan et al. proposed techniques to transform the input latent representation to particular distributions that are more likely to results in robust classifiers.
%\jr{Gabby-P22}
%As data may not be suitably distributed for robust generalization, some works suggest techniques for learning more optimal distributions in latent feature space.
Both Pang et al.~\cite{Pang:Du:Zhu:ICML:2018} and Wan et al.~\cite{Wan:Chen:Yu:Wu:Zhong:Yang:TPAMI:2022}
change the latent DNN feature representation to be similar to Gaussian mixtures. % \jr{why not Bernoulli?}.
Specifically, Pang et al. show that,
for Linear Discriminant Analysis (LDA) classifiers trained on Gaussian mixtures, the robustness radius of LDA
is proportional to the distance between the Gaussian centers.
The robustness of LDA is further maximized for symmetric Gaussian mixtures.
The authors thus modify the DNN loss function to create a latent feature representation similar to symmetric Gaussian mixtures
and further replace the last layer of DNN from commonly used Softmax Regression~\cite{Cramer:2002} to LDA.
%  (which is good news because we want to estimate and maximize the robustness radius).
To achieve the desired robustness radius, the authors compute the coordinates of the desired
Gaussian centers (as a function of the number of classes and the dimensionality of the input data)
and feed this data to the loss function.
Departing from the assumption that symmetric Gaussian mixtures are advantageous for the underlying model robustness,
Wan et al. modify the DNN loss function to compute the centers of the Gaussians directly while generating symmetric \mbox{Gaussian feature distributions}.
%The authors show that their method improves robustness by promoting symmetry in the resulting distributions.
%\ad{The authors mention that their loss function promotes symmetry in the resulting Gaussian feature distribution, which has been previously shown to improve adversarial robustness. }
%\jr{if they are not using LDA, why are they doing this? why changing to Gaussian mixtures is a good idea? what is their cool idea that makes it work!}

\subsection{Density}
\label{sec:results-density}

\begin{figure*}[h]
	\centering
	\vspace{-0.15in}
	\begin{minipage}{0.53\textwidth}
		\centering
		\includegraphics[width=\linewidth]{images/density_overview.pdf}
		\vspace{-0.14in}
		\caption{Density illustration.}
		\label{fig:density_guideline}
		\vspace{-0.1in}
	\end{minipage}
	\hspace{0.1in}
	\begin{minipage}{0.42\textwidth}
		\centering
		\includegraphics[width =\textwidth]{images/density_citation_graph.pdf}
		\vspace{-0.25in}
		\caption{Papers discussing density. }
		\label{fig:density}
	\end{minipage}
\end{figure*}


\emph{Density} measures the closeness of samples in a particular bounded region.
For continuous data, it is mathematically described by the probability density function,
which gives the probability for a variable to take a certain range of values.
For discrete data, it is described as the probability mass function, which
gives the probability for a variable to take a particular value.
We say that an area is dense when there is a high probability that random samples lie in the same area, i.e.,
close to each other.
For example, the dataset on the right-hand side of Fig.~\ref{fig:density_guideline} contains a larger number of samples in close proximity and, thus, is more dense than the dataset on the left-hand side of the figure.
Furthermore, density can be defined over samples from one class, in which case, it is referred to as \emph{class density}.

Papers that study how density influences adversarial robustness are shown in Fig.~\ref{fig:density}. They can roughly be divided into
\roundrect{1}~papers discussing the effect of class density on robustness and
\roundrect{2}~papers proposing attacks and defenses using density information.


\vspace{0.05in}
\noindent
\roundrect{1} {\bf Effects of Class Density}.
Shafahi et al.~\cite{Shafahi:Huang:Studer:Feizi:Goldstein:ICLR:2019} 
show that datasets with a higher upper bound of class density lead to better robustness.
In particular, for image datasets, the authors show that images of lower complexity,
e.g., with simple objects on plain backgrounds, have a higher correlation among adjacent pixels.
Datasets comprised of such images have a higher density, as pixel values are more frequently repeated, and, thus, lead to better robustness.
The authors confirm this observation by showing that classifiers trained on \mnist, which has a lower image complexity and thus higher density than \cifarten, are more robust than those trained on \cifarten.
Furthermore, the authors state that class density is a better predictor of robustness than dimensionality:
even after up-scaling \mnist to the same dimensionality as \cifarten,
it still has a higher density and thus results in more robust classifiers than \cifarten.

Naseer et al.~\cite{Naseer:Prabakaran:Hasan:Shafique:ML:2023} show that imbalance in class densities 
is a more substantial predictor of robustness bias among classes than the difference in the number of samples.
The authors further propose a two-step strategy to remove this bias through data augmentation.
First, they gradually increase the perturbation size in samples from all classes and identify which classes get misclassified with the smallest perturbation size, treating this as an indication of low density. 
They then generate realistic and diversified samples for these classes, 
to reduce imbalance, which in turn leads to improved robustness.

\vspace{0.05in}
\noindent
\roundrect{2} {\bf Attacks and Defenses Using Density}.
Several works note that adversarial examples are commonly found in low-density regions of the training dataset, as models are unable to learn accurate decision boundaries using a small number of samples from these regions.
Zhang et al.~\cite{Zhang:Chen:Song:Boning:Dhillon:Hsieh:ICLR:2019}
propose an attack strategy that retrieves candidate samples from low-density regions and
perturbs them to generate adversarial examples.
The authors demonstrate that, even after adversarial training, models will not be robust to adversarial attacks that target these low-density regions.


A similar finding by Zhu et al.~\cite{Zhu:Sun:Li:ICLR:2022} suggests that adversarial examples from low-density regions have a higher probability of being transferable between different models trained on the same dataset.
Based on this observation, the authors propose an attack that increases the transferability of adversarial examples by
identifying perturbation directions that maximize both the adversarial risk and
the alignment with the direction of density decrease for the underlying data distribution,
i.e., move samples towards regions with lower density.

Departing from the same idea that low-density regions are prone to adversarial attacks,
Song et al.~\cite{Song:Kim:Nowozin:Ermon:Kushman:ICLR:2017} focus on creating a defense mechanism
that uses generative models to detect if a sample comes from a low-density region when making predictions.
If so, the sample is moved towards a more dense region of the training data as a ``purification'' step.

To harden models directly, Pang et al.~\cite{Pang:Xu:Dong:Du:Chen:Zhu:ICLR:2020} propose a new loss function for DNNs,
to learn dense latent feature representations.
The authors first show that the commonly used Softmax Cross-Entropy loss function induces sparse representations
(i.e., with low class density),
which lead to vulnerable models. This is because a low number of samples in close proximity to each other prevent a model from learning reliable decision boundaries.
They then propose a loss function that explicitly encourages feature representations to concentrate around class centers;
like in their earlier work~\cite{Pang:Du:Zhu:ICML:2018}, the authors compute the coordinates of the desired
class centers (as a function of the number of classes and the dimensionality of the input data)
to maximize the distances between the centers.
The authors demonstrate that the proposed approach improves robustness under both standard and adversarial training.

\subsection{Separation}
\label{sec:results-separation}

Closely related to density, \emph{separation} refers to the distance between classes.
% \de{(a.k.a. inter-class distance)} \ad{that can be quantified using different metrics, e.g., inter-class distance and optimal transport}.\js{Inter-class distance generally refers to a specific way to compute separation,but our section includes other different metrics as well}
Fig.~\ref{fig:separation_guideline} shows examples of not well-separated (top) and well-separated (bottom) datasets.
Intuitively, learning an accurate classifier is easier when data is well-separated as samples from different classes are farther apart and samples from the same class are closer together.
Different metrics to quantify separation include
the \emph{optimal transport distance},
which computes the minimum distance required to transport samples from one class to another, and
\emph{inter-class distance}, which computes the distance between samples in different classes.


%\begin{figure*}[h]
%\centering
%  \vspace{-0.35in}
%  \begin{minipage}{0.43\textwidth}
%  \centering
%  \vspace{0.05in}
%  \includegraphics[width=0.68\linewidth]{images/separation_overview.pdf}
%  \vspace{0.07in}
%  \caption{Separation illustration.}
%  \label{fig:separation_guideline}
%  \vspace{-0.15in}
%  \end{minipage}
%  \begin{minipage}{0.5\textwidth}
%  %\vspace*{0.1in}
%  \centering
%  \includegraphics[width =\textwidth]{images/separation_vertical_legend_updated.pdf}
%  \vspace{-0.3in}
%  \caption{Papers discussing separation.}
%  \label{fig:separation}
%  \end{minipage}
%%  \vspace{-0.1in}
%\end{figure*}

% New figure
\begin{figure*}[h]
	\centering
	\vspace{-0.25in}
	\begin{minipage}{0.43\textwidth}
		\centering
		\vspace{0.25in}
		\includegraphics[width=0.68\linewidth]{images/separation_overview.pdf}
		\vspace{0.07in}
		\caption{Separation illustration.}
		\label{fig:separation_guideline}
		\vspace{-0.3in}
	\end{minipage}
	\begin{minipage}{0.5\textwidth}
		\vspace*{0.10in}
		\centering
 		\includegraphics[width =\textwidth]{images/separation_citation_graph.pdf}
		\vspace{-0.25in}
		\caption{Papers discussing separation.}
		\label{fig:separation}
	\end{minipage}
    \vspace{-0.1in}
\end{figure*}



Papers that discuss data separation in relation to adversarial robustness are shown in Fig.~\ref{fig:separation}.
They can roughly be divided into
\roundrect{1} papers showing the effect of separation on robustness and
\roundrect{2} papers proposing techniques to promote separation and, thus, increase robustness.


\vspace{0.05in}
\noindent
\roundrect{1} {\bf Effects of Separation}.
%Another line of work presents classifier-agnostic bounds on adversarial risk by using inter-class distance.
%For example,
%\jr{Jaskeerat-P27}
Bhagoji et al.~\cite{Bhagoji:Cullina:Mittal:NeurIPS:2019} calculate lower bounds for adversarial risk 
in a binary classification setting using the optimal transport distance.
The authors show
%, i.e., the minimum distance used to transport samples from one distribution to another.
that the lower bound decreases as the distance between the two class distributions increases,
i.e., a classifier becomes more robust with better separation.
Based on this result, they estimate the minimum adversarial risks for image datasets, like \mnist and \cifarten,
showing that the theoretically calculated risks are lower than the empirical values achieved by the state-of-the-art defense models.
The authors conclude that there is still room for improving existing techniques.
%\mt{Rephrase as: Their results show that the empirical adversarial risk from state-of-the-art defense models is still higher than what can be achieved.}

%\jr{Gabby-P47 / P8}
Pydi and Jog~\cite{Pydi:Jog:ICML:2020, Pydi:Jog:NeurIPS:2021} arrive at a similar conclusion~-- %as Bhagoji et al.~\cite{Bhagoji:Cullina:Mittal:NeurIPS:2019}
%in the same setup but ussing a different proof strategy.
%The authors also
that robustness improves as separation between classes increases.
The authors further focus on datasets with simple univariate distributions, such as Gaussian and uniform.
They propose a technique to construct classifiers that can achieve the optimal,
lowest possible adversarial risk for a given separation between classes.
The main idea behind this technique is to analyze the optimal way to
transport samples from one class to another 
(which represents the smallest perturbation needed to create adversarial examples) and 
further use this information to identify the decision boundary that induces the maximal distance required to transport 
samples between classes. 
% as it maximizes the required perturbation size. .
That is, the approach maximizes the distance between samples of each class and the decision boundary, 
resulting in an optimally robust classifier.
% it should be far from any class to get to the decision boundary
%The authors show that the decision boundaries for optimally robust classifiers are sensitive \jr{?} to the maximum	 perturbation size and use this information to propose a decision boundary
%which makes it harder \jr{vague} to transport samples from one class to another.
%\js{Pydi et al show that as separation increases between classes, adversarial risk decreases i.e. robustness improves same as Bhagoji et al. That is the connection to separation. Then they go a step further and propose a method to get the optimal robust classifier for some given data which would have a fixed separation between classes.
%The intuition behind finding the optimal robust classifier is, analyzing the optimal (shortest) way to transport samples from one class to another i.e. the easiest perturbation which would make adversarial examples, and drawing a decision boundary which makes it harder to transport samples from one class to another}

%\de{They demonstrate that  }\jr{how is this relevant to separation?} \de{and they often differ from the decision boundary optimal for accuracy.}

Bhattacharjee et al.~\cite{Bhattacharjee:Chaudhuri:ICML:2020}
prove that certain non-parametric models, such as k-NNs, % and kernel classifiers with fast-decaying kernel functions,
are inherently robust when trained on a large number of well-separated samples.
This is because these classifiers make predictions based on neighborhoods and well-separated data ensures that
samples in close proximity to each other share the same labels.
In their later work, discussed in Section~\ref{sec:results-number-of-samples}~\cite{Bhattacharjee:Jha:Chaudhuri:PMLR:2021},
the authors
%Bhattacharjee et al.~\cite{Bhattacharjee:Jha:Chaudhuri:PMLR:2021},
%,
show that, in well-separated data, robust accuracy is independent of dimensionality and a robust linear classifier can be learned without the need for a large number of training samples.
This result shows that adversarial vulnerability can be efficiently tamed by increasing separation.
%introduced by high dimensionality can be fought by increasing separation}




\vspace{0.05in}
\noindent
\roundrect{2} {\bf Promoting Separation}.
%\jr{Gabby-P171}
Yang et al.~\cite{Yang:Rashtchian:Wang:Chaudhuri:AISTATS:2020} propose a sample-selection-based technique to improve the adversarial robustness of non-parametric models by increasing the separation among the training data.
In particular,
as non-parametric models tend to learn complex decision boundaries when the training samples from different classes are close to each other,
the authors propose to remove the smallest subset of samples so that all pairs of differently labeled samples
remain separated even when perturbed by the maximum perturbation size. 
Wang et al.~\cite{Wang:Jha:Chaudhuri:ICML:2018}, already discussed in Section~\ref{sec:results-dimensionality}, 
focus on improving robustness of 1-NN classifiers. 
Such classifiers struggle to take advantage of points close together with opposite labels, resulting in worse robustness.
Hence, the authors propose retaining the largest subset of training samples that are
(i) well-separated and (ii) in high agreement on labels with their nearby samples
(a.k.a., highly confident). 
The authors show that their approach outperforms adversarially trained 1-NNs. 

For non-parametric classifiers, a more effective strategy is to enforce separation in the latent representations. 
% to improve the adversarial robustness of DNNs.
%Some papers suggest modifying loss functions to promote separation in latent representation.
Specifically, Mustafa et al.~\cite{Mustafa:Khan:Hayat:Goecke:Shen:Shao:TPAMI:2020}
attribute the cause of adversarial vulnerability to close proximity of classes in latent space.
Hence, they propose a loss function to learn intermediate feature representations that
separate different classes into convex polytopes, i.e., polyhedra in higher dimensions,
that are maximally separated.
%This enforces the model to learn well-separated decision regions for each class and improves robustness.
% \js{Polytopes is the term for polygons (2-dimensional) in higher dimensions. Often times machine learning algorithms separate classes by making planes/hyperplanes. Having classes separated in distant (well separated) polyhedra/polytopes ensures that classifiers can easily learn the decision boundaries.}
\revadd{
Mygdalis et al.~\cite{Mygdalis:Pitas:PR:2022} propose a loss function to separate classes into hyperspheres, 
such that samples in a class have minimum distance from their hypersphere center and 
maximum distance from the remaining hyperspheres. 
The authors demonstrate that their approach outperforms that of
Mustafa et al.~\cite{Mustafa:Khan:Hayat:Goecke:Shen:Shao:TPAMI:2020} and other baselines 
w.r.t. standard and robust accuracy for \cifarten, \cifarhundred and \svhn.}


%\mt{Michael- P173}

Bui et al.~\cite{Bui:Le:Zhao:Montague:deVel:Abraham:Phung:ECCV:2020} observe that the adversarial vulnerability of DNNs arises from a large difference in intermediate layer values between clean and adversarial data.
They thus propose to modify the loss function so that it results in an intermediate latent representation 
that has high similarity between clean and their corresponding adversarial samples,
while promoting large inter-class and small intra-class distances and 
increased margins from class centers to decision boundaries.
%prevents decision boundaries from going through high-density regions, which effectively increases the margin from class centers to decision boundaries in high-density setup.
% \jr{why? it increases in some cases but not others.}
Likewise,
Pang et al.~\cite{Pang:Du:Zhu:ICML:2018} and Wan et al.~\cite{Wan:Chen:Yu:Wu:Zhong:Yang:TPAMI:2022} discussed in Section~\ref{sec:results-distribution},
as well as Pang et al.~\cite{Pang:Xu:Dong:Du:Chen:Zhu:ICLR:2020} discussed in Section~\ref{sec:results-density},
improve DNN robustness by separating centers of the produced latent distributions, which, in turn, 
increases the separation \mbox{between classes}.
\revadd{
Similarly, Cheng et al.~\cite{Cheng:Zhu:Zhang:Liu:PR:2023} propose improving separation by 
enforcing equal variance in all directions for all classes (Distribution Normalization) and 
maximizing the minimum margin between any two classes (Margin Balance). 
%The authors compare their approach with standard adversarial training and other baselines on \mnist, \cifarten and \cifarhundred, demonstrating improved robustness.
}

Yang et al.~\cite{Yang:Feng:Du:Du:Xu:ICDM:2021} propose a representation-learning technique
to learn feature representations that
bring samples of class $C$ and adversarial examples generated for $C$ into close proximity
while separating the samples of $C$ from both
(i) adversarial examples generated for other classes and misclassified as class $C$ and
(ii) samples from other classes.
These separations are enforced by the loss function proposed by the authors.
The authors show that their approach improves the resulting model robustness compared with standard DNNs.

%\jr{Michael-P32}
%In the same topic of representation learning,
%Other than learning robust latent representations,
%\ad{Separate from works using loss functions to encourage separation,}
Garg et al.~\cite{Garg:Sharan:Zhang:Hu:NeurIPS:2018} propose an approach to generate well-separated features for a dataset using graph theory.
%In graph theory, the Laplacian matrix of a graph reveals many useful properties, for example, the sparsest cut of a graph can be approximated through the second eigenvector, i.e., the eigenvector corresponds to the second smallest eigenvalue of the graph Laplacian.
%Based on such insight, the authors propose to identify robust features from the graph Laplacian of a dataset.
Specifically, they convert the input dataset into a graph, where vertices correspond to the input data points and
edges represent the similarity between the data points (e.g., calculated using Euclidean distance).
%Afterwards, the Laplacian matrix of the graph which is defined as the difference between the diagonal in-degree matrix and the adjacency matrix is computed.
The authors prove that features extracted using the eigenvectors of the Laplacian matrix capturing the structure of the graph
will have significant variation across the data points,
while being robust to small perturbations. These qualities make them good candidates for robust features.
%\jr{is this in agrement with small/large feature variation discussion in distribution: 
%maybe they are talking here about varience for the entire dataset, which is different from feature varience discussed there.}
%\js{We want features to have variation across data points (so a model can distinguish the points) and we want features to be robust to small perturbations. So the authors are essentially saying, that these qualities make them excellent candidates for robust features}
%The authors then demonstrate that the spectral properties
%(properties of the eigenvectors and eigenvalues of the laplacian matrix)
%of the graph correlated to adversarial robustness. \jr{why? how?}
The authors then demonstrate that a linear model trained on the \mnist dataset with 20 features generated using their approach is more robust to $L_2$-norm-based transfer attacks than a fully connected neural network trained on the full pixel values of the \mnist dataset.

%\jr{why transfer attack?} \js{to simulate a black box setting as white box attacks are unrealistic, I assume, the authors don't say this, they just cite a paper and say that transfer attacks can successfully fool most models} \ad{ }

\subsection{Concentration}
\label{sec:results-concentration}

\emph{Concentration} of a dataset refers to the ``concentration of measure'' phenomenon from measure theory~\cite{Talagrand:1996:AnnalsProbability}.
In a nutshell, concentration is the minimum value of a measured function over all valid measurable sets,
after an $\epsilon$-expansion.  
More formally,
for a metric probability space $(\mathcal{X}, \mu, d)$ with instance space $\mathcal{X}$, probability measure $\mu$, and distance metric $d$,
the concentration function $h$ is defined as: $h(\mu, \alpha, \epsilon) = $ inf$_{A \subseteq \mathcal{X}} \{\mu(A_\epsilon) : \mu(A) \geq \alpha\}$ for any $\alpha \in (0,1)$ and $\epsilon \geq 0$~\cite{Mahloujifar:Zhang:Mahmoody:Evans:NeurIPS:2019}.
Here $A_\epsilon$ refers to the $\epsilon$-expansion of set $A$, defined as $A_\epsilon = \{ x : d(x, A) \leq \epsilon\}$.


\begin{figure*}[h]
  \centering
  \begin{minipage}{0.40\textwidth}
  \centering
  \vspace{-0.08in}
  \includegraphics[width =0.82\textwidth]{images/concentration_guideline.pdf}
  \vspace{-0.12in}
  \caption{Concentration illustration.}
  \label{fig:concentrationFig}
  \end{minipage}
  \begin{minipage}{0.56\textwidth}
  \centering
  \includegraphics[width =0.9\textwidth]{images/concentration_vertical_legend_wide.pdf} 
  \vspace{-0.1in}
  \caption{Papers discussing concentration. }
  \label{fig:concentration}
  \end{minipage}
  \vspace{-0.1in}
\end{figure*}

Fig.~\ref{fig:concentrationFig} shows how the concentration of measure phenomenon can be used to determine the classification error after adversarial perturbation.
By modeling the classification error set as measurable set $A$
and adversarial errors from perturbation budget $\epsilon$ as $A_{\epsilon}$,
one can relate the concentration of the data to the minimum adversarial risk for any imperfect classifier with error rate $\mu(A) \geq \alpha$.
Using this formulation, a dataset being highly concentrated implies that, for some non-zero initial error, the minimum adversarial risk from an $\epsilon$-expansion on the error set is very large.
We refer to such datasets as datasets with low \emph{intrinsic robustness}~--
a measure that represents the maximal achievable robustness for any classifier on a dataset.


Fig.~\ref{fig:concentration} shows the papers that relate data concentration to adversarial robustness.
They can roughly be divided into:
\roundrect{1} papers discussing the effect of concentration on robustness and
\roundrect{2} papers proposing techniques to estimate robustness through calculating concentrations.


\vspace{0.05in}
\noindent
\roundrect{1} {\bf Effects of Concentration}.
A number of papers prove the inevitability of adversarial examples
using the concentration of measure phenomenon. 
In particular, Dohmatob~\cite{Dohmatob:ICML:2019} investigates datasets
conforming to uniform, Gaussian, and several other distributions
that satisfy $W_2$ transportation-cost inequality~\cite{Talagrand:1996}.
The author proves that data distributions satisfying such inequality have high concentration,
which results in a rapid robustness decrease, beyond a critical perturbation size~--
a value that depends on the standard error of the classifier and the natural noise level of the dataset, which, in turn,
is defined as the largest variance in the case of Gaussian distribution.
Even though \mnist might not satisfy the $W_2$ transportation-cost inequality,
the author experiments with this dataset, observing a sudden drop in robustness
as the perturbation size increases.
As such, the author suggests that the \mnist dataset may also have high concentration and be
governed by the concentration of measure phenomena.


Mahloujifar et al.~\cite{Mahloujifar:Diochnos:Mahmoody:AAAI:2019} focus on a collection of data distributions with high concentration called L\'evy families~\cite{Levy:1951}, which include unit sphere, unit cube, and  isotropic n-Gaussian
(i.e., Gaussian with independent variables with the same variance).
The authors prove that classifiers trained on such highly-concentrated data distributions admit adversarial examples with perturbation
$\mathcal{O}(\sqrt{d})$  for dimensionality $d$.
This implies that a relatively small perturbation can mislead models trained on these data distributions with high dimensional inputs.


\vspace{0.05in}
\noindent
\roundrect{2} {\bf Estimating Robustness Through Concentration}.
Several approaches utilize the connection between concentration and adversarial risk to estimate 
the intrinsic robustness of datasets by calculating their concentrations.
Mahloujifar et al.~\cite{Mahloujifar:Zhang:Mahmoody:Evans:NeurIPS:2019} are the first to propose an approach for estimating
dataset concentration using subsets of samples.
Specifically, the authors propose a technique that searches for the minimum expansion set based on a collection of subsets carefully chosen according to the perturbation norm (e.g., a union of balls for $L_2$ norm).
They prove that
the estimated concentration value converges to the true value for the underlying distribution as the sample size and the quality/representativeness of the chosen subsets increase.
The authors apply their approach for estimating the maximum achievable robustness for the \mnist and \cifarten datasets,
observing a gap between the derived theoretical values and values observed empirically by the state-of-the-art models.

In follow-up work, Prescott et al.~\cite{Prescott:Zhang:Evans:ICLR:2021} propose
an alternative approach to estimate concentration based on half space expansion using \emph{Gaussian Isoperimetric Inequality}
for the $L_2$ norm~\cite{Borell:InvMath:1975}.
The authors further generalize their results to $L_p$ norms, where $p \geq 2$.
Compared with Mahloujifar et al.~\cite{Mahloujifar:Zhang:Mahmoody:Evans:NeurIPS:2019},
their approach yields higher achievable robustness on \mnist and \cifarten, revealing a larger gap between the theoretical robustness and the state-of-the-art.
As the theoretically achievable robustness derived from a concentration perspective is shown to be high,
the authors suggest that factors other than concentration may contribute to this gap.


Zhang and Evans~\cite{Zhang:Evans:ICLR:2022} assume access to information about label uncertainty,
i.e., a function that assigns the level of label uncertainties for any data point.
Such a function can use, e.g., labeling results from multiple human annotators or confidence scores from an ML classifier.
The authors suggest that considering regions with high label uncertainty can guide the concentration estimation
as these are the regions where a classifier is more likely to make mistakes and be vulnerable to attacks.
They thus propose an approach to estimate concentration by identifying the smallest set after
$\epsilon$-expansion with an average uncertainty level greater than a pre-set value. 
The evaluation results show that the maximum achievable robustness estimated with their approach
is closer to the robustness values observed for CNN models on the \cifarten dataset than in
any of the aforementioned works, implying that the
room for improvement is smaller than assumed earlier.

\subsection{Label Quality}
\label{sec:results-label-quality}

%\begin{figure*}[h]
%\vspace{-0.12in}
%	\begin{minipage}[t]{0.55\textwidth}
%	  %\centering
%	  \includegraphics[valign=t, width=0.95\linewidth]{images/label_noise_illustration.pdf}
%	  \vspace{-0.18in}
%	  \caption{Label noise illustration.}
%	  \label{fig:label_quality_guideline}
%	\end{minipage}
%    \begin{minipage}[t]{0.36\textwidth}
%       \centering
%       \includegraphics[valign=t, width=0.72\textwidth]{images/label_vertical_legend.pdf} %0.41
%      \vspace{0.02in}
%      \caption{Papers discussing label quality.}
%      \label{fig:label}
%    \end{minipage}
%\vspace{-0.2in}%
%\end{figure*}

% New figure
\begin{figure*}[h]
	\vspace{-0.12in}
	\begin{minipage}[t]{0.55\textwidth}
		%\centering
		\includegraphics[valign=t, width=0.95\linewidth]{images/label_noise_illustration.pdf}
		\vspace{-0.18in}
		\caption{Label noise illustration.}
		\label{fig:label_quality_guideline}
	\end{minipage}
	\begin{minipage}[t]{0.4\textwidth}
		\centering
		\includegraphics[valign=t, width=0.9\textwidth]{images/label_quality_citation_graph.pdf} %0.41
		\vspace{-0.04 in}
		\caption{Papers discussing label quality.}
		\label{fig:label}
	\end{minipage}
	\vspace{-0.1in}%
\end{figure*}


\emph{Label quality} refers to the correctness and informativeness of the set of labels assigned to a training dataset.
Label correctness or, inversely, the presence of inaccurate labels (shown as the highlighted dots on the left-hand side of Fig.~\ref{fig:label_quality_guideline}) is typically referred to as \emph{label noise}.
The granularity of the labels is typically referred to as \emph{label informativeness}.
Papers that  discuss the relationship between label quality and model robustness are outlined in Fig.~\ref{fig:label}.

%and the quantity of labels for each input to be associated with model's adversarial robustness.
%Assuming an oracle that record the true labels for all samples, label noise refer to having an arbitrary portion of input samples with incorrect labels.

%\js{Add to Table}
Mao et al.~\cite{Mao:Gupta:Nitin:Ray:Song:Yang:Vondrick:ECCV:2020} show that training a model
simultaneously for multiple tasks, e.g., to simultaneously locate and estimate the distance of objects in images
(an approach also referred to as \emph{multi-task learning}),
%such as semantic segmentation, depth estimation, and object detection
improves robustness.
This is because in multi-task learning, a model learns a shared feature representation by training on data
with labels from several tasks.
As a result, perturbations required to attack multiple tasks at the same time, e.g.,
to sabotage an autonomous driving system by misleading the model in both object identification and distance estimation,
cancel each other out.
While the authors prove that the model robustness to adversarial attacks is proportional to the number of tasks
that it is trained on,
the benefits of multi-task learning disappear when the concurrently trained tasks are highly correlated with each other,
as it reduces the chances for the perturbations to cancel each other.
% \jr{I would expect the reasons to relate to feature representation, not perturbations}. \gx{Here we followed the intuition introduced by the paper.
%One hypothesis we had when discussing this paper was that training with multiple task labels reduce the possibility of picking up spurious signals from data, hence resulting in more robust representation. However, this is not mentioned in this paper.}
The authors further show that training with multiple tasks also improves model robustness against single-task attacks.
%The authors suggest to train models with a diverse set of tasks to better reap the benefit of multi-task learning in creating robust models.

%\jr{Gabby-P172}
\begin{wrapfigure}{r}{0.4\textwidth}
  \vspace{-0.25in}
    \begin{minipage}[t]{0.45\textwidth}
	 \centering
   	 \subcaptionbox{Overfit}{
     \includegraphics[width=0.4\linewidth]{images/label_noise_guideline_overfit.pdf}
     \vspace{-0.15in}
   	 }
     \subcaptionbox{Not overfit.}{
     \includegraphics[width=0.4\linewidth]{images/label_noise_guideline_not_overfit.pdf}
     \vspace{-0.15in}
     }
     \vspace{-0.15in}
     \caption{Influence of overfitting to label noise.}
     \label{fig:label_noise_influence}
    \end{minipage}
    \vspace{-0.15in}
\end{wrapfigure}
Sanyal et al.~\cite{Sanyal:Dokania:Kanade:Torr:ICLR:2021} hypothesize that label noise and coarse labels are the reasons for adversarial vulnerability.
The authors prove that, given a large training set with random label noise, any classifier that overfits to that set is likely to be vulnerable to adversarial attacks.
This is because overfitting leads to overly complex decision boundaries that leave more room for attacks, as illustrated in  Fig.~\ref{fig:label_noise_influence}.

The authors also demonstrate that adversarial risk increases as the level of label noise increases.
Defense mechanisms, such as early stopping and adversarial training, enhance robustness by preventing models from overfitting to noisy samples.
In the absence of label noise, using coarse labels
(e.g., one label for the entire class of dogs rather than labels for each individual dog breed)
results in ``sub-optimal'' latent feature representations and also contributes to the adversarial vulnerability.
\revadd{A similar result was also shown by Shamir et al.~\cite{Shamir:Safran:Ronen:Dunkelman:ArXiv:2019} 
for high-dimensional data and piecewise linear classifiers, such as DNN with ReLU activation: 
the number of perturbations required to generate successful adversarial examples is proportional to the number of classes, 
making models trained on data with fine-grained labels more robust. }

\revadd{Dong et al.~\cite{Dong:Liu:Shang:NeurIPS:2022} show that label noise is an inherent part of adversarial training 
as labels assigned to the adversarial examples may not always match their ``correct'' labels. 
The authors show that the amount of label noise introduced in adversarial training is proportional 
to the perturbation radius and the confidence of the model prediction for an adversarial example. 
They further propose to mitigate this issue by filtering low-confidence examples generated with a high perturbation radius and 
demonstrate that their approach can achieve higher robust accuracy than in standard adversarial training for 
\cifarten, \cifarhundred, and \tinyset. 

Zhang et al.~\cite{Zhang:Jiang:Hou:Wei:Han:Li:Cheng:TIP:2021} utilize soft labels, 
i.e., labels that capture the probability of a data point belonging to a certain class, 
to learn the relationship between classes. This encourages a model 
to learn representations that group similar samples together, thus 
increasing intra-class density and, as a result, increasing robustness. }
%\jr{check if has any connections with papers in either density or separation. If so, move accordingly.}}
%\js{checked, leaving this summary here}

%Such suggestion align with the one from Mao et al. that learning with multi-task can improve the representaiton learned.



\subsection{Domain-Specific}
\label{sec:results-domain-specific}

Papers in this category provide insights into the correlation between domain-specific data properties and adversarial robustness.
Among our collected papers, all the domain-specific studies focused on the same topic: understanding the adversarial vulnerabilities of image classifiers based on image \emph{frequency}~--
how fast the intensity of pixel values changes with respect to space (i.e., images with intensive color changes have high frequency).
As shown in Fig.~\ref{fig:image_freq}, the skin of a zebra has higher image frequency than a horse, because of the black-and-white stripes.
%Frequency information of images is widely used in applications, such as image reconstruction, filtering, and compression.
%
%Discrete Fourier Transform is the technique commonly applied to transform an image from the spatial domain to the frequency domain.
%This transformation outputs the \emph{amplitude} and \emph{phase} of the frequency in the image.
%The amplitude represents the magnitude of the different frequencies in the image and captures the geometrical structure of its features.
%%(i.e. sharp corners in the spatial domain).
%Phase encodes the locations of those features.
%\jr{most definitions in this paragraph are never used, besides in the last paper. I removed it.}
%
Papers studying image frequency are listed in Fig.~\ref{fig:domain}.
They can be roughly divided into:
\roundrect{1} papers discussing the influence of frequency distribution on the model adversarial robustness,
and~\roundrect{2}~papers explaining adversarial vulnerabilities using perceptual differences between human and models. % frequency information.
%Yin et al.~\cite{Yin:Lopes:Shlens:Cubuk:Gilmer:NeurIPS:2019} and Ortiz-Jimenez et al.~\cite{Ortiz-Jimenez:Modas:Moosavi-Dezfooli:Frossard:NeurIPS:2020}
%We found two papers discuss how the distribution of frequency among training images influence resulting visual model's adversarial robustness.

\begin{figure*}[t]
  \centering
    \begin{minipage}{0.5\textwidth}
  \centering
  \vspace{0.05in}
  \includegraphics[valign=t, width =\textwidth]{images/domain_specific_frequency.pdf}
  \caption{Image frequency.}
  \label{fig:image_freq}
  \end{minipage}
  \begin{minipage}{0.49\textwidth}
  \centering
  \includegraphics[valign=t, width =0.9\textwidth]{images/domain_specific_vertical_legend.pdf} %0.56
  \vspace{-0.02in}
  \caption{Papers discussing domain-specific properties. }
  \label{fig:domain}
  \end{minipage}
  \vspace{-0.15in}
\end{figure*}

\vspace{0.05in}
\noindent
\roundrect{1} {\bf Image Frequency Distribution.}
%\jr{Gabby-P23}
Yin et al.~\cite{Yin:Lopes:Shlens:Cubuk:Gilmer:NeurIPS:2019} show that frequency distribution of the inputs generated from data augmentation techniques explains the resulting model sensitivity to adversarial attacks.
%Data augmentation are commonly used to improve image classifier's robust generalization ability, previous work~\cite{Gilmer:Ford:Carlini:Cubuk:ICML:2019} showed training with Gaussian augmentation or adversarial training improves robustness against most types of corruptions (e.g., \emph{blurring}, \emph{Gaussian noise}), but worsens robustness against \emph{fog} and \emph{contrast} corruptions.
%They explain the differences in robustness against different corruption strategies using the distribution of frequency among the data.
In particular, the authors show that Gaussian augmentation,
as well as adversarial training techniques that rely on data augmentation,
generate perturbations with high-frequency components.
Augmenting training data with those augmented inputs makes the resulting model more robust against perturbations in
high-frequency domains while, at the same time, more vulnerable to perturbations concentrated in low-frequency domains.
%
%Augmenting training data with those samples makes resulting models become more robust against noise from high-frequency domains.
%However, corruptions like \emph{fog} and \emph{contrast} generate perturbations that are more concentrated at low-frequency domains, which makes models trained with Gaussian augmentation or adversarial training less robust against these corruptions.
To mitigate this issue, the authors propose to avoid biasing the model towards/against certain frequency ranges by increasing the diversity of frequency distribution in augmentations.

%\jr{All-P48}
Ortiz-Jimenez et al.~\cite{Ortiz-Jimenez:Modas:Moosavi-Dezfooli:Frossard:NeurIPS:2020}
analyze CNN robustness through the classifier margins along particular frequencies, which the authors define
%the margin along a particular frequency range
as the minimal perturbation in that frequency required to change the model prediction.
The authors show that CNN models tend to have smaller margins along low-frequency vs. high-frequency ranges,
likely because, for most image datasets, one can differentiate classes mainly using features from low-frequency ranges.
This finding implies that models are more sensitive to attacks that modify low-frequency components.
The authors thus suggest training more robust models that enlarge margins along low-frequency ranges by augmenting training datasets with perturbations concentrated in those ranges.
%This implies that models are more sensitive to the perturbations on low-frequency components than those on high-frequency components.
%The authors also show that adversarial training can enlarge margins along low-frequency ranges by augmenting training datasets with perturbations more concentrated in the low frequency ranges.
%, and as a by-product of the model's inductive bias, the margins along high-frequency ranges also increase.

%\textbf{``High-Frequency Component Helps Explain the Generalization of Convolutional Neural Networks''}: \gx{Data property: frequency of the images, especially high-frequency components}
%Wang et al.~\cite{Wang:Wu:Huang:Xing:CVPR:2020} and Chen et al.~\cite{Chen:Peng:Ma:Li:Du:Tian:ICCV:2021}
%We found another two papers explain the existence of adversarial examples using the differences in what human and CNN models' capture from the images.
\vspace{0.05in}
\noindent
\roundrect{2} {\bf Perceptual Differences.}
%\mt{Add to table}
Wang et al.~\cite{Wang:Wu:Huang:Xing:CVPR:2020} attribute the origin of adversarial examples to the perceptual differences of humans and CNNs in frequency ranges.
In particular, humans classify images based on low-frequency components as high-frequency components are not visible to the human eye.
CNNs, on the other hand, are able to `see' the full frequency spectrum which allows them to exploit high-frequency components for better generalization.
This implies that adversarial examples generated by perturbing high-frequency components can mislead CNNs while being imperceptible to humans.
%In the case of adversarial attacks, this implies that adversarial examples generated by perturbing high-frequency components can mislead CNNs while being imperceptible to humans.
The authors show that adversarially robust models depend less on high-frequency components and propose to use smoother convolutional filters to reduce a model's attention to these components.
%In addition, they demonstrate a connection between the model's reduced attention to high frequency components and the use of smoother Convolutional filters.
%Their experiments show that using such filters can easily improve the robustness of standardly trained models, however, the improvement is only apparent for larger perturbations for adversarially trained models.

%\js{Add to table}
Unlike Wang et al.~\cite{Wang:Wu:Huang:Xing:CVPR:2020},
Chen et al.~\cite{Chen:Peng:Ma:Li:Du:Tian:ICCV:2021} posit that the adversarial vulnerability of CNNs
results from their over-reliance on amplitude information of images~--
the magnitude of the different frequencies in the image.
%Humans, considered as robust classifiers, rely more on phase information to recognize an object.
The authors show that replacing the amplitude information of an image with information from another image can successfully mislead CNNs but not humans, who rather rely on phase information~-- the locations of the features, to recognize objects.
Based on this observation, the authors propose to strengthen CNNs' attention to phase information through a data augmentation technique that fuzzes amplitude while preserving the same phase information.

%\de{replaces the amplitude information of images with ones from multiple other images while maintaining the same labels.} \jr{how does this give more attention to phase information?}
% \ad{creates multiple copies of an image by retaining the phase information while replace the amplitude information with the ones from other images to help strengthening the correlation between phase information and labels.}
%%combines the phase spectrum of the current image with amplitude spectrum of another image and labels it as the current image only.

%\gx{answer the questions in bullet points}
%\section{Questions our survey answers}
%\begin{enumerate}
%     \item What properties of data influence the resulting model’s robustness?
%	\item How to select and represent data to improve robustness?
%	\item How to select and configure a classifier to be robust based on the properties data?
%\end{enumerate} 


\subsection{Summary of Results}
\label{sec:results-summary}

Overall, the surveyed papers are mostly in agreement on how each of the identified data property influences adversarial robustness. The main findings are given below.

\vspace{0.05in}
\noindent {\bf Number of samples.} More training samples are needed for robust than for standard generalization.
For a variety of training setups (i.e., different types of classifiers and data distributions), the number of training samples required to achieve robust generalization is proportional to the dimensionality of the training data.
Unlabeled samples or generated data can be used to fulfill the need for more samples needed for robust generalization, i.e., to close the sample complexity gap.
Class imbalance, i.e., having an imbalanced number of samples across different classes, hurts robust generalization due to the model bias towards over-represented samples.

\vspace{0.05in}
\noindent
{\bf Dimensionality.} Dimensionality captures the size of the feature set.
Higher dimensionality correlates with
higher adversarial risk, worse standard-to-adversarial risk trade-off, difficulty in robustness certification, and difficulty in applying common defense techniques.
This is because adversarial attacks can exploit the excessive dimensions to construct adversarial examples.

\vspace{0.05in}
\noindent
{\bf Distribution.} Some data distributions are more robust than others, e.g., mixtures of Bernoulli distributions are more robust than mixtures of Gaussian distributions.
%Small variance in data distributions within a class helps robust generalization .
%distributions observed in common image datasets, such as \mnist and \cifarten.
Learning feature representations that resemble robust distributions can improve robustness.

\vspace{0.05in}
\noindent
{\bf Density.} Density reflects the closeness of samples in a particular bounded region (inter-class distance).
Adversarial examples are commonly found in low-density regions of data, where samples are far apart from each other.
This is because models cannot accurately learn decision boundaries near low-density regions due to the small number of samples available. As such, high data density for each class correlates with lower adversarial risk.

\vspace{0.05in}
\noindent
{\bf Separation.} Separation characterizes %the closeness of samples from the same class (density!) and
the distance of samples from different classes to each other (inter-class distance).
Greater separation between classes decreases adversarial risk as it is harder to generate perturbation that will
cross the boundaries between classes.
Most papers that provide techniques for improving separations, e.g., by feature selection, also ensure that it does not come
at the expense of decreasing density, as these two concepts are closely related.

\vspace{0.05in}
\noindent
{\bf Concentration.} Given a function defined over a non-empty set,
concentration (from the phenomenon of concentration of measure~\cite{Talagrand:1996:AnnalsProbability})
is the minimum value of the function after expanding the input set by $\epsilon$ in all dimensions.
%It is used to measure the minimum achievable error considering all epsilon perturbations of sample set.
For example, expanding the set of misclassified samples by a certain $\epsilon$ gives a set of possible samples that can be
misclassified with an $\epsilon$-size perturbation (candidate adversarial examples).
Concentration, in this case, measures the minimal possible size of this set, which provides the upper bound of the achievable model robustness.
As some datasets tend to exhibit inherently high concentration,
e.g., datasets that lie on unit hypersphere~\cite{Mahloujifar:Diochnos:Mahmoody:AAAI:2019},
achieving high robust generalization is harder for these datasets.
The impact of high concentration on adversarial robustness is further magnified for high-dimensional data.



\vspace{0.05in}
\noindent
{\bf Label quality.} High label noise correlates with higher adversarial risk.
More specific labels, e.g., ``cat'' and ``dog'' instead of ``animal'', are more robust than coarse labels, as they allow the model to extract more distinct features.
Learning for different tasks concurrently, e.g., to simultaneously locate and estimate the distance of objects in images,
improves robustness of the learned models, as the model can utilize the information from multiple sources of data.

\vspace{0.05in}
\noindent
{\bf Domain specific.} Image frequency, i.e., the rate of change in pixel value is shown to be correlated with robustness.
Specifically, a diverse distribution of frequencies in training data results in lower adversarial risk.
Most image datasets have a low image frequency, which results in learned models having smaller margins and, thus,
a lower distance of samples to the decision boundary for features corresponding to the low-frequency image components.
This increases the risk for adversarial perturbation that utilize these features.


%We present this information in Fig. \ref{fig:summaryofresults}.
%On the other hand, papers that study dimensionality all note that high dimensionality leads to adversarial vulnerability.
%Some papers that study intrinsic dimensionality and degeneracy also argue that dimensionality is a source of adversarial vulnerability when the dataset is not only high dimensional but can also be represented using a smaller dimension.
%In addition, some works suggest that some distributions such as the Gaussian distribution may be more robust than others.
%Hence, one can increase robustness by using data transformation techniques that make datasets resemble specific type of distributions such as the Gaussian.
%
%%%% These may need to be separated
%Papers that deal with distance related properties note that a larger separation of the datasets, that corresponds to a larger inter-class distance, reduces adversarial vulnerability.
%Similarly, a smaller intra-class distance, that corresponds to a larger class density, also reduces adversarial vulnerability.
%High concentration of samples near the boundary region of the classes increases adversarial vulnerability.
%However, the problem may not be resolved by merely removing such samples from the dataset as these samples are essential to learning an accurate decision boundary by characterizing the uncertain region between classes.
%Adding regularization terms in loss functions and changing the feature representation are among the approaches proposed to promote high inter-class distance and low intra-class distance in feature representations.
%Other reasons for adversarial vulnerability include mislabeled samples and coarse labels.




