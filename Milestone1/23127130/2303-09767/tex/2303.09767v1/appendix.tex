\section{Appendices}
\label{sec:appendix}

\subsection{Google Scholar Search Constraints}
\label{sec:constraints}

We used Google Scholar's \emph{site} or \emph{source} filtering constraints to perform a targeted, per-venue search.
Specifically, for venues that have their publications on custom websites,
e.g., \emph{``proceedings.neurips.cc''}~\cite{NeurIPS:Conf:Site:2022} for proceedings from NeurIPS, 
we performed a \emph{site}-constrained search.
For venues that have their publications hosted on a shared website, 
e.g., PMLR~\cite{PMLR:Site:2022}) for proceedings of ICLR,  
we performed a \emph{source}-constrained search.
Table~\ref{tbl:constraints} shows a complete set of constraints we used; the ``+'' symbol indicates a union of the results.

\begin{table}[t]
  \caption{The constraints applied to limit the Google Scholar search}
  \vspace{-0.1in}
\scalebox{0.85}{
\begin{tabular}{|l|l|}
  \hline
  \textbf{Venue} & \textbf{Search Constraints with Publication Source / Site}                             \\ \hline \hline
  ACL        & source:``Association for Computational Linguistic''+ source: ``ACL''                      \\ \hline
  CL         & source:``Computational Linguistics''+ source: ``CL''                      \\ \hline 
  COLT       & source:``International Conference on Learning Theory''+ source: ``COLT''                     \\ \hline
  ICLR       & source:``International Conference on Learning Representations'' + source:``ICLR'' \\ \hline
  ICML       & source:``International Conference on Machine Learning'' + source:``ICML''        \\ \hline
  NDSS       & source:``Network and Distributed System Security'' + source:``NDSS''             \\ \hline
  Artificial Intelligence & source: ``Artificial Intelligence" \\ \hline  
  Neural Networks & source: ``Neural Networks"  \\ \hline
  Pattern Recognition & source: ``Pattern Recognition"  \\ \hline
  Knowledge Based System & source: ``Knowledge Based System"  \\ \hline
  JISA & source: ``Journal of Information and Security Applications"  \\ \hline
  AAAI       & site:aaai.org                                                             \\ \hline
  IJCAI      & site:ijcai.org                                                              \\ \hline
  JMLR       & site:jmlr.org                                                            \\ \hline  
  NeurIPS       & site:proceedings.neurips.cc                            \\ \hline
  USENIX     & site:usenix.org                                                               \\ \hline
  ArXiv       & site:arxiv.org                                                             \\ \hline
\end{tabular}
}
  \label{tbl:constraints}
%  \vspace{-0.15in}
\end{table}


\subsection{Detailed Paper Categorization}
\label{sec:appendix-categorization}
We include the detailed categorization tables for the papers collected in the following tables.
In the table, we used abbreviation to denote the datasets used in the papers: \rectangled{M} for \mnist, \rectangled{FM} for \fmnist~\cite{Xiao:Rasul:Vollgraf:ArXiv:FashionMNIST:2017}, 
\rectangled{S} for \svhn, 
\rectangled{C-10} for \cifarten, 
\rectangled{C-100} for \cifarhundred,
\rectangled{IN} for \imagenet~\cite{Krizhevsky:Sutskever:Hinton:ComACM:ImageNet:2012},
\rectangled{TI} for \tinyset,
\rectangled{CA} for \celeba~\cite{Liu:Luo:Wang:Tang:CelebA:2015},
\rectangled{HM} for \halfmoon,
\rectangled{M1V7} for \mnistv,
\rectangled{A} for \abalone~\cite{Abalone:1995},
\rectangled{L} for \lsun~\cite{Yu:Zhang:Song:Seff:Xiao:ArXiv:LSUN:2015},
\rectangled{CS} for \cityscapes~\cite{Cordts:Cityscapes:2016},
\rectangled{TO} for \taskonomy~\cite{Amir:Taskonomy:2018}.

%\begin{sidewaystable}
\centering
\scalebox{0.54}{
\begin{tabular}{|c|ccccccc|cccccccc|}
\hline
\multicolumn{1}{|l|}{}                                                                                                & \multicolumn{7}{c|}{Problem Setup}                                                                                                                                                                                                                                                                                                                                                                                                                                                                                                                                                                                                                                                                                         & \multicolumn{8}{c|}{Applicability}                                                                                                                                                                                                                                                                                                                                                                                                                                                                                                                                                                                                                                      \\ \cline{2-16} 
\multicolumn{1}{|l|}{}                                                                                                & \multicolumn{1}{c|}{}                                                                               & \multicolumn{2}{c|}{Model}                                                                                                                                                                          & \multicolumn{4}{c|}{Robustness Setting}                                                                                                                                                                                                                                                                                                                                                                        & \multicolumn{2}{c|}{Actionability}                                                               & \multicolumn{6}{c|}{Type of Evidence}                                                                                                                                                                                                                                                                                                                                                                                                                                                                                                                                \\ \cline{3-16} 
\multicolumn{1}{|l|}{}                                                                                                & \multicolumn{1}{c|}{}                                                                               & \multicolumn{1}{c|}{}                                                                           & \multicolumn{1}{c|}{}                                                                             & \multicolumn{1}{c|}{}                                                                                      & \multicolumn{1}{c|}{}                                                                                  & \multicolumn{1}{c|}{}                                                                                  &                                                                                 & \multicolumn{1}{l|}{}                         & \multicolumn{1}{l|}{}                            & \multicolumn{1}{l|}{}                         & \multicolumn{5}{c|}{Empirical}                                                                                                                                                                                                                                                                                                                                                                                                                                                                                       \\ \cline{12-16} 
\multicolumn{1}{|l|}{\multirow{-4}{*}{Paper}}                                                                         & \multicolumn{1}{c|}{\multirow{-3}{*}{Target Distribution}}                                          & \multicolumn{1}{c|}{\multirow{-2}{*}{\begin{tabular}[c]{@{}c@{}}Learning \\ Task\end{tabular}}} & \multicolumn{1}{c|}{\multirow{-2}{*}{\begin{tabular}[c]{@{}c@{}}Classifier \\ Type\end{tabular}}} & \multicolumn{1}{c|}{\multirow{-2}{*}{\begin{tabular}[c]{@{}c@{}}Definition of \\ Robustness\end{tabular}}} & \multicolumn{1}{c|}{\multirow{-2}{*}{\begin{tabular}[c]{@{}c@{}}Attacker's \\ Knowledge\end{tabular}}} & \multicolumn{1}{c|}{\multirow{-2}{*}{\begin{tabular}[c]{@{}c@{}}Attacker's \\ Technique\end{tabular}}} & \multirow{-2}{*}{\begin{tabular}[c]{@{}c@{}}Perturbation \\ Bound\end{tabular}} & \multicolumn{1}{l|}{\multirow{-2}{*}{Metric}} & \multicolumn{1}{l|}{\multirow{-2}{*}{Technique}} & \multicolumn{1}{l|}{\multirow{-2}{*}{Formal}} & \multicolumn{1}{c|}{Dataset}                                                                                                                                   & \multicolumn{1}{c|}{Attack}                                                                       & \multicolumn{1}{c|}{\begin{tabular}[c]{@{}c@{}}Classifier \\ Type\end{tabular}} & \multicolumn{1}{c|}{\begin{tabular}[c]{@{}c@{}}Training \\ Procedure\end{tabular}}      & \begin{tabular}[c]{@{}c@{}}Definition of \\ Robustness\end{tabular} \\ \hline
 Schmidt et al.~\cite{Schmidt:Santurkar:Tsipras:Talwar:Madry:NeurIPS:2018}    & \multicolumn{1}{c|}{\begin{tabular}[c]{@{}c@{}}Gaussian-mixture, \\ Bernoulli-mixture\end{tabular}} & \multicolumn{1}{c|}{\begin{tabular}[c]{@{}c@{}}Binary\\ Classification\end{tabular}}            & \multicolumn{1}{c|}{Any}                                                                          & \multicolumn{1}{c|}{\begin{tabular}[c]{@{}c@{}}Error-rate \\ based\end{tabular}}                           & \multicolumn{1}{c|}{White Box}                                                                         & \multicolumn{1}{c|}{Any}                                                                               & $L_{\infty}$                                                                              & \multicolumn{1}{c|}{Yes}                      & \multicolumn{1}{c|}{No}                          & \multicolumn{1}{c|}{\checkmark}     & \multicolumn{1}{c|}{\begin{tabular}[c]{@{}c@{}}\rectangled{M} , \rectangled{C-10} , \\ \rectangled{SVHN} \end{tabular}}                          & \multicolumn{1}{c|}{PGD}                                                                          & \multicolumn{1}{c|}{DNNs}                                                       & \multicolumn{1}{c|}{Adversarial}                                                        & \begin{tabular}[c]{@{}c@{}}Error-rate \\ based\end{tabular}         \\ \hline
 Dan et al.~\cite{Dan:Wei:Ravikumar:ICML:2020}                            & \multicolumn{1}{c|}{Gaussian-mixture}                                                               & \multicolumn{1}{c|}{\begin{tabular}[c]{@{}c@{}}Binary\\ Classification\end{tabular}}            & \multicolumn{1}{c|}{Any}                                                                          & \multicolumn{1}{c|}{\begin{tabular}[c]{@{}c@{}}Error-rate \\ based\end{tabular}}                           & \multicolumn{1}{c|}{White Box}                                                                         & \multicolumn{1}{c|}{Any}                                                                               & $Lp,p\ge 1$                                                           & \multicolumn{1}{c|}{Yes}                      & \multicolumn{1}{c|}{No}                          & \multicolumn{1}{c|}{\checkmark}     & \multicolumn{1}{c|}{N/A}                                                                                                                                       & \multicolumn{1}{c|}{N/A}                                                                          & \multicolumn{1}{c|}{N/A}                                                        & \multicolumn{1}{c|}{N/A}                                                                & N/A                                                                 \\ \hline
Cullina et al.~ \cite{Cullina:Bhagoji:Mittal:NeurIPS:2018}                    & \multicolumn{1}{c|}{Any}                                                                            & \multicolumn{1}{c|}{\begin{tabular}[c]{@{}c@{}}Binary\\ Classification\end{tabular}}            & \multicolumn{1}{c|}{Any}                                                                          & \multicolumn{1}{c|}{\begin{tabular}[c]{@{}c@{}}Error-rate \\ based\end{tabular}}                           & \multicolumn{1}{c|}{White Box}                                                                         & \multicolumn{1}{c|}{Any}                                                                               & $Lp$                                                                              & \multicolumn{1}{c|}{Yes}                      & \multicolumn{1}{c|}{No}                          & \multicolumn{1}{c|}{\checkmark}     & \multicolumn{1}{c|}{N/A}                                                                                                                                       & \multicolumn{1}{c|}{N/A}                                                                          & \multicolumn{1}{c|}{N/A}                                                        & \multicolumn{1}{c|}{N/A}                                                                & N/A                                                                 \\ \hline
Bhattacharjee et al.~\cite{Bhattacharjee:Jha:Chaudhuri:PMLR:2021}                  & \multicolumn{1}{c|}{Well-separated}                                                                 & \multicolumn{1}{c|}{\begin{tabular}[c]{@{}c@{}}Binary\\ Classification\end{tabular}}            & \multicolumn{1}{c|}{Linear}                                                                       & \multicolumn{1}{c|}{\begin{tabular}[c]{@{}c@{}}Error-rate \\ based\end{tabular}}                           & \multicolumn{1}{c|}{White Box}                                                                         & \multicolumn{1}{c|}{Any}                                                                               & $Lp,p\ge 2$                                                             & \multicolumn{1}{c|}{Yes}                      & \multicolumn{1}{c|}{No}                          & \multicolumn{1}{c|}{\checkmark}     & \multicolumn{1}{c|}{N/A}                                                                                                                                       & \multicolumn{1}{c|}{N/A}                                                                          & \multicolumn{1}{c|}{N/A}                                                        & \multicolumn{1}{c|}{N/A}                                                                & N/A                                                                 \\ \hline
Gourdeau et al.~ \cite{Gourdeau:Kanade:Kwiatkowska:Worrell:JMLR:2021}          & \multicolumn{1}{c|}{Any}                                                                            & \multicolumn{1}{c|}{Any}                                                                        & \multicolumn{1}{c|}{Any}                                                                          & \multicolumn{1}{c|}{\begin{tabular}[c]{@{}c@{}}Error-rate \\ based\end{tabular}}                           & \multicolumn{1}{c|}{White Box}                                                                         & \multicolumn{1}{c|}{Any}                                                                               & Any                                                                             & \multicolumn{1}{c|}{Yes}                      & \multicolumn{1}{c|}{No}                          & \multicolumn{1}{c|}{\checkmark}     & \multicolumn{1}{c|}{N/A}                                                                                                                                       & \multicolumn{1}{c|}{N/A}                                                                          & \multicolumn{1}{c|}{N/A}                                                        & \multicolumn{1}{c|}{N/A}                                                                & N/A                                                                 \\ \hline
Carmon et al.~ \cite{Carmon:Raghunathan:Schmidt:Duchi:Liang:NeurIPS:2019}    & \multicolumn{1}{c|}{Gaussian-mixture}                                                               & \multicolumn{1}{c|}{\begin{tabular}[c]{@{}c@{}}Binary\\ Classification\end{tabular}}            & \multicolumn{1}{c|}{Any}                                                                          & \multicolumn{1}{c|}{Radius-based}                                                                          & \multicolumn{1}{c|}{White Box}                                                                         & \multicolumn{1}{c|}{\begin{tabular}[c]{@{}c@{}}Gradient \\ based\end{tabular}}                         & L2, $L_{\infty}$                                                                          & \multicolumn{1}{c|}{Yes}                      & \multicolumn{1}{c|}{Yes}                         & \multicolumn{1}{c|}{\checkmark}     & \multicolumn{1}{c|}{\cifarten, \svhn}                                                                                            & \multicolumn{1}{c|}{PGD}                                                                          & \multicolumn{1}{c|}{CNN}                                                        & \multicolumn{1}{c|}{Adversarial}                                                        & Radius-based                                                        \\ \hline
Uesato et al.~ \cite{Uesato:Alayrac:Huang:Stanforth:Fawzi:Kohli:NeurIPS:2019}& \multicolumn{1}{c|}{Gaussian-mixture}                                                               & \multicolumn{1}{c|}{\begin{tabular}[c]{@{}c@{}}Binary\\ Classification\end{tabular}}            & \multicolumn{1}{c|}{Any}                                                                          & \multicolumn{1}{c|}{\begin{tabular}[c]{@{}c@{}}Error-rate \\ based\end{tabular}}                           & \multicolumn{1}{c|}{White Box}                                                                         & \multicolumn{1}{c|}{Any}                                                                               & $L_{\infty}$                                                                              & \multicolumn{1}{c|}{Yes}                      & \multicolumn{1}{c|}{Yes}                         & \multicolumn{1}{c|}{\checkmark}     & \multicolumn{1}{c|}{\cifarten, \svhn}                                                                                            & \multicolumn{1}{c|}{PGD, FGSM}                                                                    & \multicolumn{1}{c|}{DNNs}                                                       & \multicolumn{1}{c|}{Adversarial}                                                        & \begin{tabular}[c]{@{}c@{}}Error-rate \\ based\end{tabular}         \\ \hline
 Najafi et al.~\cite{Najafi:Maeda:Koyama:Miyato:NeurIPS:2019}                & \multicolumn{1}{c|}{Any}                                                                            & \multicolumn{1}{c|}{Any}                                                                        & \multicolumn{1}{c|}{Any}                                                                          & \multicolumn{1}{c|}{\begin{tabular}[c]{@{}c@{}}Error-rate \\ based\end{tabular}}                           & \multicolumn{1}{c|}{White Box}                                                                         & \multicolumn{1}{c|}{\begin{tabular}[c]{@{}c@{}}Gradient \\ based\end{tabular}}                         & L2, $L_{\infty}$                                                                          & \multicolumn{1}{c|}{Yes}                      & \multicolumn{1}{c|}{Yes}                         & \multicolumn{1}{c|}{\checkmark}     & \multicolumn{1}{c|}{\begin{tabular}[c]{@{}c@{}}\mnist, \cifarten, \\ \svhn\end{tabular}}                          & \multicolumn{1}{c|}{PGD}                                                                          & \multicolumn{1}{c|}{DNNs}                                                       & \multicolumn{1}{c|}{Adversarial}                                                        & \begin{tabular}[c]{@{}c@{}}Error-rate \\ based\end{tabular}         \\ \hline
Gowal et al.~\cite{Gowal:Rebuffi:Wiles:Stimberg:Calian:Mann:NeurIPS:2021}  & \multicolumn{1}{c|}{Any}                                                                            & \multicolumn{1}{c|}{Any}                                                                        & \multicolumn{1}{c|}{Any}                                                                          & \multicolumn{1}{c|}{\begin{tabular}[c]{@{}c@{}}Error-rate \\ based\end{tabular}}                           & \multicolumn{1}{c|}{White Box}                                                                         & \multicolumn{1}{c|}{Any}                                                                               & $Lp$                                                                              & \multicolumn{1}{c|}{Yes}                      & \multicolumn{1}{c|}{Yes}                         & \multicolumn{1}{c|}{-}                         & \multicolumn{1}{c|}{\begin{tabular}[c]{@{}c@{}}\mnist, \cifarten, \\ \svhn, \tinyset\end{tabular}} & \multicolumn{1}{c|}{AutoAttack}                                                                   & \multicolumn{1}{c|}{DNNs}                                                       & \multicolumn{1}{c|}{Adversarial}                                                        & \begin{tabular}[c]{@{}c@{}}Error-rate \\ based\end{tabular}         \\ \hline
Wu et al.~\cite{Wu:Liu:Huang:Wang:Lin:CVPR:2021}                        & \multicolumn{1}{c|}{Not mentioned}                                                                  & \multicolumn{1}{c|}{Any}                                                                        & \multicolumn{1}{c|}{DNNs}                                                                         & \multicolumn{1}{c|}{\begin{tabular}[c]{@{}c@{}}Error-rate \\ based\end{tabular}}                           & \multicolumn{1}{c|}{White Box}                                                                         & \multicolumn{1}{c|}{\begin{tabular}[c]{@{}c@{}}Gradient \\ based\end{tabular}}                         & $L_{\infty}$                                                                              & \multicolumn{1}{c|}{Yes}                      & \multicolumn{1}{c|}{Yes}                         & \multicolumn{1}{c|}{-}                         & \multicolumn{1}{c|}{\begin{tabular}[c]{@{}c@{}}\cifarten, \\ \cifarhundred\end{tabular}}                                         & \multicolumn{1}{c|}{\begin{tabular}[c]{@{}c@{}}PGD, C\&W, \\ Transfer-based attacks\end{tabular}} & \multicolumn{1}{c|}{DNNs}                                                       & \multicolumn{1}{c|}{\begin{tabular}[c]{@{}c@{}}Standard \& \\ Adversarial\end{tabular}} & \begin{tabular}[c]{@{}c@{}}Error-rate \\ based\end{tabular}         \\ \hline
\end{tabular}
}
\caption{Categorization Table for Papers related to Number of Samples}
\end{sidewaystable}
\begin{sidewaystable}
\caption{Categorization Table for Papers - Part 1}
\label{tbl:categorization1}
\centering
\scalebox{0.57}{
\begin{tabular}{|l|l|l|lllllll|ccccllll|}
\hline
\multicolumn{1}{|c|}{\multirow{4}{*}{ID}} & \multicolumn{1}{c|}{\multirow{4}{*}{Paper}}                                                   & \multicolumn{1}{c|}{\multirow{4}{*}{Data Property}}                          & \multicolumn{7}{c|}{Problem Setup}                                                                                                                                                                                                                                                                                                                                                                                                                                                                                                                                                                                                                                                                                                           & \multicolumn{8}{c|}{Practicality}                                                                                                                                                                                                                                                                                                                                                                                                                                                                                                                                                                              \\ \cline{4-18} 
\multicolumn{1}{|c|}{}                                                                       & \multicolumn{1}{c|}{}                                                                         & \multicolumn{1}{c|}{}                                                        & \multicolumn{1}{c|}{\multirow{3}{*}{\begin{tabular}[c]{@{}c@{}}Target \\ Distribution\end{tabular}}}            & \multicolumn{2}{c|}{Model}                                                                                                                                                                         & \multicolumn{4}{c|}{Robustness Setting}                                                                                                                                                                                                                                                                                                                                                                               & \multicolumn{2}{c|}{Applicability}                                                        & \multicolumn{1}{c|}{\multirow{3}{*}{Exp.}} & \multicolumn{5}{c|}{Type of Evidence}                                                                                                                                                                                                                                                                                                                                                                                                                                 \\ \cline{5-12} \cline{14-18} 
\multicolumn{1}{|c|}{}                                                                       & \multicolumn{1}{c|}{}                                                                         & \multicolumn{1}{c|}{}                                                        & \multicolumn{1}{c|}{}                                                                                           & \multicolumn{1}{c|}{\multirow{2}{*}{\begin{tabular}[c]{@{}c@{}}Learning \\ Task\end{tabular}}} & \multicolumn{1}{c|}{\multirow{2}{*}{\begin{tabular}[c]{@{}c@{}}Classifier \\ Type\end{tabular}}}  & \multicolumn{1}{c|}{\multirow{2}{*}{\begin{tabular}[c]{@{}c@{}}Definition \\ of \\ Robustness\end{tabular}}} & \multicolumn{1}{c|}{\multirow{2}{*}{\begin{tabular}[c]{@{}c@{}}Attacker's \\ Knwl.\end{tabular}}} & \multicolumn{1}{c|}{\multirow{2}{*}{\begin{tabular}[c]{@{}c@{}}Attacker's \\ Tech.\end{tabular}}} & \multicolumn{1}{c|}{\multirow{2}{*}{\begin{tabular}[c]{@{}c@{}}Perturb \\ Bound\end{tabular}}} & \multicolumn{1}{c|}{\multirow{2}{*}{Metr.}} & \multicolumn{1}{c|}{\multirow{2}{*}{Tech.}} & \multicolumn{1}{c|}{}                      & \multicolumn{1}{c|}{Fml.}       & \multicolumn{4}{c|}{Empirical}                                                                                                                                                                                                                                                                                                                                                                                                      \\ \cline{14-18} 
\multicolumn{1}{|c|}{}                                                                       & \multicolumn{1}{c|}{}                                                                         & \multicolumn{1}{c|}{}                                                        & \multicolumn{1}{c|}{}                                                                                           & \multicolumn{1}{c|}{}                                                                          & \multicolumn{1}{c|}{}                                                                             & \multicolumn{1}{c|}{}                                                                                        & \multicolumn{1}{c|}{}                                                                             & \multicolumn{1}{c|}{}                                                                             & \multicolumn{1}{c|}{}                                                                          & \multicolumn{1}{c|}{}                       & \multicolumn{1}{c|}{}                       & \multicolumn{1}{c|}{}                      & \multicolumn{1}{c|}{}           & \multicolumn{1}{c|}{Dataset}                                                                                                                                    & \multicolumn{1}{c|}{\begin{tabular}[c]{@{}c@{}}Classifier \\ Type\end{tabular}}            & \multicolumn{1}{l|}{\begin{tabular}[c]{@{}l@{}}Training \\ Proc.\end{tabular}} & Attacks                                                                             \\ \hline \midrule \hline
1                                                                                            & Amsaleg et   al.~\cite{Amsaleg:Bailey:Barbe:Erfani:Furon:Houle:Radovanovic:Nguyen:TIFS:2021} & Dimensionality                                                               & \multicolumn{1}{l|}{Any}                                                                                        & \multicolumn{1}{l|}{Any}                                                              & \multicolumn{1}{l|}{Any}                                                                          & \multicolumn{1}{l|}{Radius based}                                                                              & \multicolumn{1}{l|}{White box}                                                                    & \multicolumn{1}{l|}{Any}                                                                          & $L_2$                                                                                               & \multicolumn{1}{c|}{\checkmark}             & \multicolumn{1}{c|}{\xmark}           & \multicolumn{1}{c|}{\xmark}      & \multicolumn{1}{c|}{\checkmark}               & \multicolumn{1}{l|}{\rectangled{C-10},  \rectangled{IN}}                                                                                                        & \multicolumn{1}{l|}{$k$-NN}                                                                & \multicolumn{1}{l|}{Standard}                                                  & N/A  \\ \hline
2                                                                                            & Awasthi et   al.~\cite{Awasthi:Jain:Rawat:Vijayaraghavan:NeurIPS:2020}                       & Dimensionality                                                               & \multicolumn{1}{l|}{Any}                                                                                        & \multicolumn{1}{l|}{Any}                                                              & \multicolumn{1}{l|}{DNNs}                                                                         & \multicolumn{1}{l|}{Radius based}                                                                            & \multicolumn{1}{l|}{White box}                                                                    & \multicolumn{1}{l|}{Any}                                                                          & $L_2, L_{\infty}$                                                                                   & \multicolumn{1}{c|}{\checkmark}             & \multicolumn{1}{c|}{\xmark}             & \multicolumn{1}{c|}{\xmark}    & \multicolumn{1}{c|}{\xmark}              & \multicolumn{1}{l|}{\rectangled{C-10},  \rectangled{C-100}}                                                                                                    & \multicolumn{1}{l|}{DNNs}                                                                  & \multicolumn{1}{l|}{Adversarial}                                               & PGD                                                                                \\ \hline
3                                                                                            & Bhagoji et al.~\cite{Bhagoji:Cullina:Mittal:NeurIPS:2019}                                    & Separation                                                                   & \multicolumn{1}{l|}{Any}                                                                                        & \multicolumn{1}{l|}{\begin{tabular}[c]{@{}l@{}}Binary \\ Classif.\end{tabular}} & \multicolumn{1}{l|}{Any}                                                                          & \multicolumn{1}{l|}{\begin{tabular}[c]{@{}l@{}}Error-rate \\ based\end{tabular}}                             & \multicolumn{1}{l|}{White box}                                                                    & \multicolumn{1}{l|}{\begin{tabular}[c]{@{}l@{}}Gradient \\ based\end{tabular}}                    & $L_2$                                                                                               & \multicolumn{1}{c|}{\checkmark}             & \multicolumn{1}{c|}{\xmark}            & \multicolumn{1}{c|}{\xmark}        & \multicolumn{1}{c|}{\checkmark}             & \multicolumn{1}{l|}{\begin{tabular}[c]{@{}l@{}}\rectangled{C-10},  \rectangled{M}, \\ \rectangled{FM}\end{tabular}}                                           & \multicolumn{1}{l|}{DNNs}                                                                  & \multicolumn{1}{l|}{Adversarial}                                               & PGD, FGSM                                \\ \hline
4                                                                                            & Bhattacharjee et   al.~\cite{Bhattacharjee:Chaudhuri:ICML:2020}                             & Separation                                                                   & \multicolumn{1}{l|}{Any}                                                                                        & \multicolumn{1}{l|}{\begin{tabular}[c]{@{}l@{}}Binary \\ Classif.\end{tabular}} & \multicolumn{1}{l|}{\begin{tabular}[c]{@{}l@{}}Non-\\ parametric \\ classifiers\end{tabular}}     & \multicolumn{1}{l|}{Radius based}                                                                            & \multicolumn{1}{l|}{White box}                                                                    & \multicolumn{1}{l|}{\begin{tabular}[c]{@{}l@{}}Distance\\ based\end{tabular}}                     & $L_2$                                                                                               & \multicolumn{1}{c|}{\checkmark}             & \multicolumn{1}{c|}{\checkmark}     & \multicolumn{1}{c|}{\checkmark}         & \multicolumn{1}{c|}{\checkmark}              & \multicolumn{1}{l|}{\rectangled{HM}}                                                                                                                            & \multicolumn{1}{l|}{\begin{tabular}[c]{@{}l@{}}Histogram, \\ 1-NN\end{tabular}}            & \multicolumn{1}{l|}{Standard}                                                  & \begin{tabular}[c]{@{}l@{}}Distance-\\ based \\ attacks\end{tabular}                                 \\ \hline
5                                                                                            & Bhattacharjee et   al.~\cite{Bhattacharjee:Jha:Chaudhuri:PMLR:2021}                          & \begin{tabular}[c]{@{}l@{}}Number of samples, \\ Dimensionality, \\ Separation \end{tabular}                                                            & \multicolumn{1}{l|}{Well-separated}                                                                             & \multicolumn{1}{l|}{\begin{tabular}[c]{@{}l@{}}Binary \\ Classif.\end{tabular}} & \multicolumn{1}{l|}{Linear}                                                                       & \multicolumn{1}{l|}{\begin{tabular}[c]{@{}l@{}}Error-rate\\ based\end{tabular}}                              & \multicolumn{1}{l|}{White box}                                                                    & \multicolumn{1}{l|}{Any}                                                                          & $L_p, p > 2$                                                                                        & \multicolumn{1}{c|}{\checkmark}             & \multicolumn{1}{c|}{\xmark}            & \multicolumn{1}{c|}{\checkmark}           & \multicolumn{1}{c|}{\checkmark}              & \multicolumn{1}{l|}{N/A}                                                                                                                                        & \multicolumn{1}{l|}{N/A}                                                                   & \multicolumn{1}{l|}{N/A}                                                       & N/A                                                                                         \\ \hline
6                                                                                            & Blum et al.~\cite{Blum:Dick:Manoj:Zhang:JMLR:2020}                                           & Dimensionality                                                               & \multicolumn{1}{l|}{Any}                                                                                        & \multicolumn{1}{l|}{Any}                                                              & \multicolumn{1}{l|}{\begin{tabular}[c]{@{}l@{}}Randomized \\ smoothed \\ classifier\end{tabular}} & \multicolumn{1}{l|}{Radius based}                                                                            & \multicolumn{1}{l|}{White box}                                                                    & \multicolumn{1}{l|}{Any}                                                                          & $L_p, p > 2$                                                                                        & \multicolumn{1}{c|}{\checkmark}             & \multicolumn{1}{c|}{\xmark}              & \multicolumn{1}{c|}{\xmark}       & \multicolumn{1}{c|}{\checkmark}             & \multicolumn{1}{l|}{\rectangled{C-10}}                                                                                                                          & \multicolumn{1}{l|}{\begin{tabular}[c]{@{}l@{}}Smoothed \\ DNN\end{tabular}}               & \multicolumn{1}{l|}{Adversarial}                                               & \begin{tabular}[c]{@{}l@{}}Gaussian \\ noise \end{tabular}               \\ \hline
7                                                                                            & Bui et   al.~\cite{Bui:Le:Zhao:Montague:deVel:Abraham:Phung:ECCV:2020}                       & Separation  & \multicolumn{1}{l|}{Any}                                                                                        & \multicolumn{1}{l|}{Any}                                                              & \multicolumn{1}{l|}{DNNs}                                                                         & \multicolumn{1}{l|}{\begin{tabular}[c]{@{}l@{}}Error-rate \\ based\end{tabular}}                             & \multicolumn{1}{l|}{White box}                                                                    & \multicolumn{1}{l|}{\begin{tabular}[c]{@{}l@{}}Gradient \\ based\end{tabular}}                    & $L_p$                                                                                               & \multicolumn{1}{c|}{\xmark}                 & \multicolumn{1}{c|}{\checkmark}        & \multicolumn{1}{c|}{\xmark}           & \multicolumn{1}{c|}{\xmark}              & \multicolumn{1}{l|}{\rectangled{C-10},  \rectangled{M}}                                                                                                        & \multicolumn{1}{l|}{CNNs}                                                                  & \multicolumn{1}{l|}{Adversarial}                                               & PGD                                \\ \hline
8                                                                                            & Carbone et   al.~\cite{Carbone:Wicker:Laurenti:Patane:Bortolussi:Sanguinetti:NeurIPS:2020}   & Dimensionality                                                               & \multicolumn{1}{l|}{Any}                                                                                        & \multicolumn{1}{l|}{Any}                                                              & \multicolumn{1}{l|}{\begin{tabular}[c]{@{}l@{}}Bayesian \\ neural \\ network\end{tabular}}        & \multicolumn{1}{l|}{Radius based}                                                                            & \multicolumn{1}{l|}{White box}                                                                    & \multicolumn{1}{l|}{\begin{tabular}[c]{@{}l@{}}Gradient \\ based\end{tabular}}                    & $L_{\infty}$                                                                                        & \multicolumn{1}{c|}{\xmark}                 & \multicolumn{1}{c|}{\xmark}           & \multicolumn{1}{c|}{\checkmark}          & \multicolumn{1}{c|}{\checkmark}           & \multicolumn{1}{l|}{\begin{tabular}[c]{@{}l@{}}\rectangled{M},   \rectangled{FM}, \\ \rectangled{HM}\end{tabular}}                                            & \multicolumn{1}{l|}{\begin{tabular}[c]{@{}l@{}}Bayesian \\ neural \\ network\end{tabular}} & \multicolumn{1}{l|}{Adversarial}                                               & PGD,FGSM                                \\ \hline
9                                                                                            & Carmon et al.~\cite{Carmon:Raghunathan:Schmidt:Duchi:Liang:NeurIPS:2019}                     & Number of samples                                                            & \multicolumn{1}{l|}{\begin{tabular}[c]{@{}l@{}}Gaussian-mixture \\ (theory), \\ Any (application)\end{tabular}} & \multicolumn{1}{l|}{\begin{tabular}[c]{@{}l@{}}Binary \\ Classif.\end{tabular}} & \multicolumn{1}{l|}{Any}                                                                          & \multicolumn{1}{l|}{Radius based}                                                                            & \multicolumn{1}{l|}{White box}                                                                    & \multicolumn{1}{l|}{\begin{tabular}[c]{@{}l@{}}Gradient \\ based\end{tabular}}                    & $L_2, L_{\infty}$                                                                                   & \multicolumn{1}{c|}{\checkmark}             & \multicolumn{1}{c|}{\checkmark}            & \multicolumn{1}{c|}{\xmark}     & \multicolumn{1}{c|}{\checkmark}             & \multicolumn{1}{l|}{\rectangled{C-10},   \rectangled{S}}                                                                                                       & \multicolumn{1}{l|}{CNNs}                                                                  & \multicolumn{1}{l|}{Adversarial}                                               & PGD \\ \hline
10                                                                                           & Chen et al.~\cite{Chen:Peng:Ma:Li:Du:Tian:ICCV:2021}                                         & Domain-Specific                                                              & \multicolumn{1}{l|}{Any}                                                                                        & \multicolumn{1}{l|}{Any}                                                              & \multicolumn{1}{l|}{CNN}                                                                          & \multicolumn{1}{l|}{\begin{tabular}[c]{@{}l@{}}Error-rate \\ based\end{tabular}}                             & \multicolumn{1}{l|}{White box}                                                                    & \multicolumn{1}{l|}{Any}                                                                          & $L_2$                                                                                               & \multicolumn{1}{c|}{\checkmark}             & \multicolumn{1}{c|}{\checkmark}          & \multicolumn{1}{c|}{\xmark}        & \multicolumn{1}{c|}{\xmark}               & \multicolumn{1}{l|}{\begin{tabular}[c]{@{}l@{}}\rectangled{C-10},   \rectangled{C-100}, \\ \rectangled{S}, \rectangled{IN}, \\ \rectangled{L}\end{tabular}} & \multicolumn{1}{l|}{CNNs}                                                                  & \multicolumn{1}{l|}{Standard}                                                  & PGD, FGSM                                 \\ \hline
11                                              & Chen et al.~\cite{Chen:Ren:Yan:NeurIPS:2022}                                & Domain-Specific                                      & \multicolumn{1}{l|}{Image Data}                                                                                                        & \multicolumn{1}{l|}{Any}                                                 & \multicolumn{1}{l|}{CNNs}                                                                                        & \multicolumn{1}{l|}{\begin{tabular}[c]{@{}l@{}}Error-rate \\ based\end{tabular}}                           & \multicolumn{1}{l|}{White Box}                                                                & \multicolumn{1}{l|}{\begin{tabular}[c]{@{}l@{}}Gradient \\ based\end{tabular}}                          & $L_p$                                                           & \multicolumn{1}{c|}{\xmark}                      & \multicolumn{1}{c|}{\xmark}                   & \multicolumn{1}{c|}{\checkmark}                      & \multicolumn{1}{c|}{\xmark}                   & \multicolumn{1}{l|}{\begin{tabular}[c]{@{}l@{}}\rectangled{C-10},   \rectangled{C-100}, \\ \rectangled{TI}, \rectangled{IN}, \\ \rectangled{L}\end{tabular}}  & \multicolumn{1}{l|}{CNNs}            & \multicolumn{1}{l|}{\begin{tabular}[c]{@{}l@{}}Standard,\\ Adversarial\end{tabular}}       & \begin{tabular}[c]{@{}l@{}}FGSM, BIM\\ PGD, C\&W \end{tabular} \\ \hline
12                                              & Cheng et al.~\cite{Cheng:Zhu:Zhang:Liu:PR:2023}                             & Separation                                          & \multicolumn{1}{l|}{Gaussian Mixture}                                                                           & \multicolumn{1}{l|}{Any}                                                                                       & \multicolumn{1}{l|}{DNNs}                                                                                        &  \multicolumn{1}{l|}{\begin{tabular}[c]{@{}l@{}}Error-rate \\ based\end{tabular}}                           & \multicolumn{1}{l|}{Any}                                   & \multicolumn{1}{l|}{Any}                                   & $L_2$                                                          & \multicolumn{1}{c|}{\xmark}                 & \multicolumn{1}{c|}{\checkmark}                  & \multicolumn{1}{c|}{\checkmark}                     & \multicolumn{1}{c|}{\xmark}                  & \multicolumn{1}{l|}{\begin{tabular}[c]{@{}l@{}}\rectangled{C-10}, \rectangled{C-100},\\ \rectangled{M}\end{tabular}}                 & \multicolumn{1}{l|}{DNNs}            & \multicolumn{1}{l|}{\begin{tabular}[c]{@{}l@{}}Standard,\\ Adversarial\end{tabular}} & \begin{tabular}[c]{@{}l@{}}FGSM, PGD, \\ C\&W \end{tabular}     \\ \hline
13                                                                                           & Cullina et al.~\cite{Cullina:Bhagoji:Mittal:NeurIPS:2018}                                    & Number of samples                                                            & \multicolumn{1}{l|}{Any}                                                                                        & \multicolumn{1}{l|}{\begin{tabular}[c]{@{}l@{}}Binary \\ Classif.\end{tabular}} & \multicolumn{1}{l|}{Any}                                                                          & \multicolumn{1}{l|}{\begin{tabular}[c]{@{}l@{}}Error-rate \\ based\end{tabular}}                             & \multicolumn{1}{l|}{White box}                                                                    & \multicolumn{1}{l|}{Any}                                                                          & $L_p$                                                                                               & \multicolumn{1}{c|}{\checkmark}             & \multicolumn{1}{c|}{\xmark}                 & \multicolumn{1}{c|}{\checkmark}        & \multicolumn{1}{c|}{\checkmark}          & \multicolumn{1}{l|}{N/A}                                                                                                                                        & \multicolumn{1}{l|}{N/A}                                                                   & \multicolumn{1}{l|}{N/A}                                                       & N/A                                                                                         \\ \hline
14                                                                                           & Dan et al.~\cite{Dan:Wei:Ravikumar:ICML:2020}                                                & \begin{tabular}[c]{@{}l@{}}Number of samples, \\ Dimensionality\end{tabular} & \multicolumn{1}{l|}{Gaussian-mixture}                                                                           & \multicolumn{1}{l|}{\begin{tabular}[c]{@{}l@{}}Binary \\ Classif.\end{tabular}} & \multicolumn{1}{l|}{Any}                                                                          & \multicolumn{1}{l|}{\begin{tabular}[c]{@{}l@{}}Error-rate \\ based\end{tabular}}                             & \multicolumn{1}{l|}{White box}                                                                    & \multicolumn{1}{l|}{Any}                                                                          & $L_p, p\ge 1$                                                                                       & \multicolumn{1}{c|}{\checkmark}             & \multicolumn{1}{c|}{\xmark}                 & \multicolumn{1}{c|}{\checkmark}                & \multicolumn{1}{c|}{\checkmark}               & \multicolumn{1}{l|}{N/A}                                                                                                                                        & \multicolumn{1}{l|}{N/A}                                                                   & \multicolumn{1}{l|}{N/A}                                                       & N/A                                                                                         \\ \hline
15                                                                                           & Daniely et al.~\cite{Daniely:Schacham:NeurIPS:2020}                                           & Dimensionality                                                               & \multicolumn{1}{l|}{Any}                                                                                        & \multicolumn{1}{l|}{Any}                                                              & \multicolumn{1}{l|}{\begin{tabular}[c]{@{}l@{}}ReLU \\ networks\end{tabular}}                     & \multicolumn{1}{l|}{Radius-based}                                                                            & \multicolumn{1}{l|}{White box}                                                                    & \multicolumn{1}{l|}{Any}                                                                          & $L_2$                                                                                               & \multicolumn{1}{c|}{\checkmark}             & \multicolumn{1}{c|}{\xmark}            & \multicolumn{1}{c|}{\xmark}         & \multicolumn{1}{c|}{\checkmark}           & \multicolumn{1}{l|}{N/A}                                                                                                                                        & \multicolumn{1}{l|}{N/A}                                                                   & \multicolumn{1}{l|}{N/A}                                                       & N/A                                                                                         \\ \hline
16                                           & De Palma et al.~\cite{DePalma:Kiani:Lloyd:ICML:2021}                        & Dimensionality                                      & \multicolumn{1}{l|}{Image Data}                                                                                 & \multicolumn{1}{l|}{Binary Classif.}                                                                           & \multicolumn{1}{l|}{DNNs}                                                                                        & \multicolumn{1}{l|}{Radius-based}                              & \multicolumn{1}{l|}{White Box}                             & \multicolumn{1}{l|}{Any}                                   & $L_1$                                                          & \multicolumn{1}{c|}{\checkmark}                     & \multicolumn{1}{c|}{\xmark}              & \multicolumn{1}{c|}{\xmark}                         & \multicolumn{1}{c|}{\checkmark}              & \multicolumn{1}{l|}{\rectangled{M}, \rectangled{C-10}}                                          & \multicolumn{1}{l|}{DNNs}            & \multicolumn{1}{l|}{Standard}                & Others              \\ \hline
17                                              & Deng and Karam~\cite{Deng:Karam:ECCVWorkshops:2020}                         & Domain-Specific                                     & \multicolumn{1}{l|}{Image Data}                                                                                 & \multicolumn{1}{l|}{Any}                                                                                       & \multicolumn{1}{l|}{CNNs}                                                                                        &  \multicolumn{1}{l|}{\begin{tabular}[c]{@{}l@{}}Error-rate \\ based\end{tabular}}                           & \multicolumn{1}{l|}{White Box}                             & \multicolumn{1}{l|}{GANs-based}                            & $L_{\infty}$                                                   & \multicolumn{1}{c|}{\checkmark}                 & \multicolumn{1}{c|}{\checkmark}              & \multicolumn{1}{c|}{\checkmark}                     & \multicolumn{1}{c|}{\xmark}                  & \multicolumn{1}{l|}{\rectangled{IN}}                                                            & \multicolumn{1}{l|}{CNNs}            & \multicolumn{1}{l|}{Standard}                & FTUAP               \\ \hline
18                                              & Deng and Karam~\cite{Deng:Karam:TIP:2022}                                  & Domain-Specific                                     & \multicolumn{1}{l|}{Image Data}                                                                                 & \multicolumn{1}{l|}{Any}                                                                                       & \multicolumn{1}{l|}{CNNs}                                                                                        &  \multicolumn{1}{l|}{\begin{tabular}[c]{@{}l@{}}Error-rate \\ based\end{tabular}}                           & \multicolumn{1}{l|}{White Box}                             & \multicolumn{1}{l|}{GANs-based}                            & $L_{\infty}$                                                   & \multicolumn{1}{c|}{\checkmark}                 & \multicolumn{1}{c|}{\checkmark}              & \multicolumn{1}{c|}{\checkmark}                     & \multicolumn{1}{c|}{\xmark}                  &  \multicolumn{1}{l|}{\begin{tabular}[c]{@{}l@{}} \rectangled{MC}, \rectangled{G} \\ etc. \end{tabular}}  & \multicolumn{1}{l|}{CNNs}            & \multicolumn{1}{l|}{Standard}                & FTUAP               \\ \hline
19                                                                                           & Ding et al.~\cite{Ding:Lui:Jin:Wang:Huang:ICLR:2019}                                         & Distribution                                                                 & \multicolumn{1}{l|}{Any}                                                                                        & \multicolumn{1}{l|}{Any}                                                              & \multicolumn{1}{l|}{Any}                                                                          & \multicolumn{1}{l|}{\begin{tabular}[c]{@{}l@{}}Error-rate \\ based\end{tabular}}                             & \multicolumn{1}{l|}{White box}                                                                    & \multicolumn{1}{l|}{Any}                                                                          & Any                                                                                                 & \multicolumn{1}{c|}{\xmark}                 & \multicolumn{1}{c|}{\xmark}                 & \multicolumn{1}{c|}{\checkmark}         & \multicolumn{1}{c|}{\checkmark}       & \multicolumn{1}{l|}{\rectangled{C-10},   \rectangled{M}}                                                                                                       & \multicolumn{1}{l|}{DNNs}                                                                  & \multicolumn{1}{l|}{Adversarial}                                               & PGD                                 \\ \hline
20                                                                                           & Diochnos et al.~\cite{Diochnos:Mahloujifar:Mahmoody:2018}                                     & Dimensionality                                                               & \multicolumn{1}{l|}{\begin{tabular}[c]{@{}l@{}}Uniform distribu \\ -tion on boolean \\ hypercube \end{tabular}}                                                                          & \multicolumn{1}{l|}{Any}                                                              & \multicolumn{1}{l|}{Any}                                                                          & \multicolumn{1}{l|}{Radius based}                                                                            & \multicolumn{1}{l|}{White box}                                                                    & \multicolumn{1}{l|}{Any}                                                                          & $L_0$                                                                                               & \multicolumn{1}{c|}{\checkmark}             & \multicolumn{1}{c|}{\xmark}                & \multicolumn{1}{c|}{\xmark}      & \multicolumn{1}{c|}{\checkmark}                & \multicolumn{1}{l|}{N/A}                                                                                                                                        & \multicolumn{1}{l|}{N/A}                                                                   & \multicolumn{1}{l|}{N/A}                                                       & N/A                                                                                         \\ \hline
21                                                                                           & Dohmatob~\cite{Dohmatob:ICML:2019}                                                           & Concentration                                                                & \multicolumn{1}{l|}{Any}                                                                                        & \multicolumn{1}{l|}{Any}                                                              & \multicolumn{1}{l|}{Any}                                                                          & \multicolumn{1}{l|}{Radius based}                                                                            & \multicolumn{1}{l|}{White box}                                                                    & \multicolumn{1}{l|}{Any}                                                                          & \begin{tabular}[c]{@{}l@{}}$L_p$, \\ Geodesic\end{tabular}                                          & \multicolumn{1}{c|}{\checkmark}             & \multicolumn{1}{c|}{\xmark}                 & \multicolumn{1}{c|}{\xmark}      & \multicolumn{1}{c|}{\checkmark}          & \multicolumn{1}{l|}{\rectangled{M}}                                                                                                                             & \multicolumn{1}{l|}{DNNs}                                                                  & \multicolumn{1}{l|}{Adversarial}                                               & \begin{tabular}[c]{@{}l@{}}Not\\ mentioned \end{tabular}                                \\ \hline
22                                             & Dong et al.~\cite{Dong:Liu:Shang:NeurIPS:2022}                              & Label Quality                                       & \multicolumn{1}{l|}{Any}                                                                                        & \multicolumn{1}{l|}{Any}                                                                                       & \multicolumn{1}{l|}{DNNs}                                                                                        &  \multicolumn{1}{l|}{\begin{tabular}[c]{@{}l@{}}Error-rate \\ based\end{tabular}}                           & \multicolumn{1}{l|}{White Box}                             & \multicolumn{1}{l|}{\begin{tabular}[c]{@{}l@{}}Gradient- \\ based\end{tabular}}                         & $L_2$                                                          & \multicolumn{1}{c|}{\xmark}                 & \multicolumn{1}{c|}{\checkmark}                  & \multicolumn{1}{c|}{\checkmark}                     & \multicolumn{1}{c|}{\checkmark}              & \multicolumn{1}{l|}{\begin{tabular}[c]{@{}l@{}}\rectangled{C-10}, \rectangled{C-100},\\ \rectangled{TI}\end{tabular}}  & \multicolumn{1}{l|}{DNNs}            & \multicolumn{1}{l|}{Adversarial}             & \begin{tabular}[c]{@{}l@{}}Square\\ RayS \end{tabular} \\ \hline

\end{tabular}}
\end{sidewaystable}
\begin{sidewaystable}
\caption{Categorization Table for Papers - Part 2}
\centering
\scalebox{0.57}{
\begin{tabular}{|l|l|l|lllllll|ccccllll|}
\hline
\multicolumn{1}{|c|}{\multirow{4}{*}{ID}} & \multicolumn{1}{c|}{\multirow{4}{*}{Paper}}                                                   & \multicolumn{1}{c|}{\multirow{4}{*}{Data Property}}                          & \multicolumn{7}{c|}{Problem Setup}                                                                                                                                                                                                                                                                                                                                                                                                                                                                                                                                                                                                                                                                                                           & \multicolumn{8}{c|}{Practicality}                                                                                                                                                                                                                                                                                                                                                                                                                                                                                                                                                                              \\ \cline{4-18} 
\multicolumn{1}{|c|}{}                                                                       & \multicolumn{1}{c|}{}                                                                         & \multicolumn{1}{c|}{}                                                        & \multicolumn{1}{c|}{\multirow{3}{*}{\begin{tabular}[c]{@{}c@{}}Target \\ Distribution\end{tabular}}}            & \multicolumn{2}{c|}{Model}                                                                                                                                                                         & \multicolumn{4}{c|}{Robustness Setting}                                                                                                                                                                                                                                                                                                                                                                               & \multicolumn{2}{c|}{Applicability}                                                        & \multicolumn{1}{c|}{\multirow{3}{*}{Exp.}} & \multicolumn{5}{c|}{Type of Evidence}                                                                                                                                                                                                                                                                                                                                                                                                                                 \\ \cline{5-12} \cline{14-18} 
\multicolumn{1}{|c|}{}                                                                       & \multicolumn{1}{c|}{}                                                                         & \multicolumn{1}{c|}{}                                                        & \multicolumn{1}{c|}{}                                                                                           & \multicolumn{1}{c|}{\multirow{2}{*}{\begin{tabular}[c]{@{}c@{}}Learning \\ Task\end{tabular}}} & \multicolumn{1}{c|}{\multirow{2}{*}{\begin{tabular}[c]{@{}c@{}}Classifier \\ Type\end{tabular}}}  & \multicolumn{1}{c|}{\multirow{2}{*}{\begin{tabular}[c]{@{}c@{}}Definition \\ of \\ Robustness\end{tabular}}} & \multicolumn{1}{c|}{\multirow{2}{*}{\begin{tabular}[c]{@{}c@{}}Attacker's \\ Knwl.\end{tabular}}} & \multicolumn{1}{c|}{\multirow{2}{*}{\begin{tabular}[c]{@{}c@{}}Attacker's \\ Tech.\end{tabular}}} & \multicolumn{1}{c|}{\multirow{2}{*}{\begin{tabular}[c]{@{}c@{}}Perturb \\ Bound\end{tabular}}} & \multicolumn{1}{c|}{\multirow{2}{*}{Metr.}} & \multicolumn{1}{c|}{\multirow{2}{*}{Tech.}} & \multicolumn{1}{c|}{}                      & \multicolumn{1}{c|}{Fml.}       & \multicolumn{4}{c|}{Empirical}                                                                                                                                                                                                                                                                                                                                                                                                      \\ \cline{14-18} 
\multicolumn{1}{|c|}{}                                                                       & \multicolumn{1}{c|}{}                                                                         & \multicolumn{1}{c|}{}                                                        & \multicolumn{1}{c|}{}                                                                                           & \multicolumn{1}{c|}{}                                                                          & \multicolumn{1}{c|}{}                                                                             & \multicolumn{1}{c|}{}                                                                                        & \multicolumn{1}{c|}{}                                                                             & \multicolumn{1}{c|}{}                                                                             & \multicolumn{1}{c|}{}                                                                          & \multicolumn{1}{c|}{}                       & \multicolumn{1}{c|}{}                       & \multicolumn{1}{c|}{}                      & \multicolumn{1}{c|}{}           & \multicolumn{1}{c|}{Dataset}                                                                                                                                    & \multicolumn{1}{c|}{\begin{tabular}[c]{@{}c@{}}Classifier \\ Type\end{tabular}}            & \multicolumn{1}{l|}{\begin{tabular}[c]{@{}l@{}}Training \\ Proc.\end{tabular}} & Attacks                                                                             \\ \hline \midrule \hline
22                                        & Izmailov et   al.~\cite{Izmaliov:Sugrim:Chadha:McDaniel:Swami:MILCOM:2018}               & Distribution                                        & \multicolumn{1}{l|}{Any}                                                                             & \multicolumn{1}{l|}{\begin{tabular}[c]{@{}l@{}}Binary \\ Classif.\end{tabular}}                & \multicolumn{1}{l|}{\begin{tabular}[c]{@{}l@{}}Linear SVM, \\ RBF SVM, \\ NNs\end{tabular}}                      & \multicolumn{1}{l|}{\begin{tabular}[c]{@{}l@{}}Error-rate \\ based\end{tabular}}                             & \multicolumn{1}{l|}{White box}                                                                    & \multicolumn{1}{l|}{\begin{tabular}[c]{@{}l@{}}Gradient \\ based\end{tabular}}                          & $L_{\infty}$                                                                                  & \multicolumn{1}{c|}{\checkmark}             & \multicolumn{1}{c|}{\checkmark}             & \multicolumn{1}{c|}{\xmark}          & \multicolumn{1}{c|}{\xmark}        & \multicolumn{1}{l|}{\rectangled{M}}                                                                                                                             & \multicolumn{1}{l|}{\begin{tabular}[c]{@{}l@{}}Linear SVM, \\ RBF SVM,\\ DNN\end{tabular}}   & \multicolumn{1}{l|}{Standard}                                                         & FGSM \\ \hline
23                                        & Javanmard et   al.~\cite{Javanmard:Soltanolkotabi:Hassani:COLT:2020}                     & \begin{tabular}[c]{@{}l@{}} Number of samples, \\ Dimensionality \end{tabular}   & \multicolumn{1}{l|}{Any}                                                                             & \multicolumn{1}{l|}{Regres.}                                                                      & \multicolumn{1}{l|}{\begin{tabular}[c]{@{}l@{}}Linear \\ Regres.\end{tabular}}                                & \multicolumn{1}{l|}{\begin{tabular}[c]{@{}l@{}}Error-rate \\ based\end{tabular}}                             & \multicolumn{1}{l|}{Black box}                                                                    & \multicolumn{1}{l|}{Any}                                                                                & $L_2$                                                                                         & \multicolumn{1}{c|}{\checkmark}             & \multicolumn{1}{c|}{\xmark}               & \multicolumn{1}{c|}{\xmark}      & \multicolumn{1}{c|}{\checkmark}                          & \multicolumn{1}{l|}{N/A}                                                                                                                                        & \multicolumn{1}{l|}{N/A}                                                                     & \multicolumn{1}{l|}{N/A}                                                              & N/A                                                                                         \\ \hline
24                                        & Kumar et al.~\cite{Kumar:Levine:Goldstein:Feizi:ICML:2020}                               & Dimensionality                                      & \multicolumn{1}{l|}{Any}                                                                             & \multicolumn{1}{l|}{Any}                                                                             & \multicolumn{1}{l|}{Any}                                                                                         & \multicolumn{1}{l|}{Radius based}                                                                            & \multicolumn{1}{l|}{Any}                                                                          & \multicolumn{1}{l|}{Any}                                                                                & $L_p, p > 2$                                                                                  & \multicolumn{1}{c|}{\checkmark}             & \multicolumn{1}{c|}{\xmark}            & \multicolumn{1}{c|}{\xmark}          & \multicolumn{1}{c|}{\checkmark}                         & \multicolumn{1}{l|}{\rectangled{C-10}, \rectangled{IN}}                                                                                                        & \multicolumn{1}{l|}{DNNs}                                                                    & \multicolumn{1}{l|}{DNNs}                                                             & \begin{tabular}[c]{@{}l@{}}Gaussian \\ noise \\ (certification)\end{tabular} \\ \hline
25                                        & Lee et al.~\cite{Lee:Lee:Yoon:CVPR:2020}                                                 & Distribution                                        & \multicolumn{1}{l|}{Any}                                                                             & \multicolumn{1}{l|}{Any}                                                                             & \multicolumn{1}{l|}{DNNs}                                                                                        & \multicolumn{1}{l|}{\begin{tabular}[c]{@{}l@{}}Error-rate \\ based\end{tabular}}                             & \multicolumn{1}{l|}{\begin{tabular}[c]{@{}l@{}}White box,  \\ Black box\end{tabular}}             & \multicolumn{1}{l|}{\begin{tabular}[c]{@{}l@{}}Gradient \\ based,\\ Non- \\ gradient \\ based\end{tabular}} & $L_{\infty}$                                                                                  & \multicolumn{1}{c|}{\xmark}                 & \multicolumn{1}{c|}{\checkmark}             & \multicolumn{1}{c|}{\checkmark}        & \multicolumn{1}{c|}{\checkmark}        & \multicolumn{1}{l|}{\begin{tabular}[c]{@{}l@{}}\rectangled{C-10}, \rectangled{C-100}, \\ \rectangled{S},   \rectangled{TI}\end{tabular}}                     & \multicolumn{1}{l|}{DNNs}                                                                    & \multicolumn{1}{l|}{\begin{tabular}[c]{@{}l@{}}Standard, \\ Adversarial\end{tabular}} & \begin{tabular}[c]{@{}l@{}}PGD, FGSM,\\ C\&W,\\ Transfer-based \\ attacks \end{tabular}                                  \\ \hline
26                                        & Mahloujifar et   al.~\cite{Mahloujifar:Diochnos:Mahmoody:AAAI:2019}                       & Concentration                                       & \multicolumn{1}{l|}{\begin{tabular}[c]{@{}l@{}}Distributions \\ in L\'evy Families\end{tabular}}     & \multicolumn{1}{l|}{Any}                                                                             & \multicolumn{1}{l|}{Any}                                                                                         & \multicolumn{1}{l|}{\begin{tabular}[c]{@{}l@{}}Error-rate \\ based\end{tabular}}                             & \multicolumn{1}{l|}{White box}                                                                    & \multicolumn{1}{l|}{Any}                                                                                & $L_0$                                                                                         & \multicolumn{1}{c|}{\checkmark}             & \multicolumn{1}{c|}{\xmark}          & \multicolumn{1}{c|}{\xmark}               & \multicolumn{1}{c|}{\checkmark}                      & \multicolumn{1}{l|}{N/A}                                                                                                                                        & \multicolumn{1}{l|}{N/A}                                                                     & \multicolumn{1}{l|}{N/A}                                                              & N/A                                                                                         \\ \hline
27                                        & Mahloujifar et   al.~\cite{Mahloujifar:Zhang:Mahmoody:Evans:NeurIPS:2019}                & Concentration                                       & \multicolumn{1}{l|}{Any}                                                                             & \multicolumn{1}{l|}{Any}                                                                             & \multicolumn{1}{l|}{Any}                                                                                         & \multicolumn{1}{l|}{\begin{tabular}[c]{@{}l@{}}Error-rate \\ based\end{tabular}}                             & \multicolumn{1}{l|}{White box}                                                                    & \multicolumn{1}{l|}{Any}                                                                                & $L_2, L_{\infty}$                                                                             & \multicolumn{1}{c|}{\checkmark}             & \multicolumn{1}{c|}{\xmark}                 & \multicolumn{1}{c|}{\xmark}                 & \multicolumn{1}{c|}{\xmark}                 & \multicolumn{1}{l|}{\rectangled{C-10}, \rectangled{M}}                                                                                                         & \multicolumn{1}{l|}{DNNs}                                                                    & \multicolumn{1}{l|}{Adversarial}                                                      & PGD                               \\ \hline
28                                        & Mao et   al.~\cite{Mao:Gupta:Nitin:Ray:Song:Yang:Vondrick:ECCV:2020}                     & Label Quality                                       & \multicolumn{1}{l|}{Any}                                                                             & \multicolumn{1}{l|}{Any}                                                                             & \multicolumn{1}{l|}{DNNs}                                                                                        & \multicolumn{1}{l|}{\begin{tabular}[c]{@{}l@{}}Error-rate \\ based\end{tabular}}                             & \multicolumn{1}{l|}{White box}                                                                    & \multicolumn{1}{l|}{\begin{tabular}[c]{@{}l@{}}Gradient\\ based\end{tabular}}                           & $L_{\infty}$                                                                                  & \multicolumn{1}{c|}{\checkmark}             & \multicolumn{1}{c|}{\xmark}                 & \multicolumn{1}{c|}{\checkmark}             & \multicolumn{1}{c|}{\checkmark}    & \multicolumn{1}{l|}{\rectangled{CS}, \rectangled{TO}}                                                                                                          & \multicolumn{1}{l|}{DNNs}                                                                    & \multicolumn{1}{l|}{Standard}                                                         & \begin{tabular}[c]{@{}l@{}}PGD, FGSM\\ MIM, Houdini\end{tabular}                                  \\ \hline
29                                        & Mehrabi et   al.~\cite{Mehrabi:Javanmard:Rossi:Rao:Mai:ICML:2021}                        & Dimensionality                                      & \multicolumn{1}{l|}{Gaussian-mixture}                                                                & \multicolumn{1}{l|}{\begin{tabular}[c]{@{}l@{}}Regres., \\ Binary \\ Classif.\end{tabular}} & \multicolumn{1}{l|}{\begin{tabular}[c]{@{}l@{}}Linear \\ Regres.,\\ Linear\\ classifiers\end{tabular}}        & \multicolumn{1}{l|}{\begin{tabular}[c]{@{}l@{}}Error-rate \\ based\end{tabular}}                             & \multicolumn{1}{l|}{White box}                                                                    & \multicolumn{1}{l|}{Any}                                                                                & $L_p, p \ge 1$                                                                                & \multicolumn{1}{c|}{\checkmark}             & \multicolumn{1}{c|}{\xmark}          & \multicolumn{1}{c|}{\xmark}                & \multicolumn{1}{c|}{\checkmark}                     & \multicolumn{1}{l|}{N/A}                                                                                                                                        & \multicolumn{1}{l|}{N/A}                                                                     & \multicolumn{1}{l|}{N/A}                                                              & N/A                                                                                         \\ \hline
30                                        & Mustafa et   al.~\cite{Mustafa:Khan:Hayat:Goecke:Shen:Shao:ICCV:2019}                    & Separation                                          & \multicolumn{1}{l|}{Any}                                                                             & \multicolumn{1}{l|}{Any}                                                                             & \multicolumn{1}{l|}{DNNs}                                                                                        & \multicolumn{1}{l|}{\begin{tabular}[c]{@{}l@{}}Error-rate \\ based\end{tabular}}                             & \multicolumn{1}{l|}{White box}                                                                    & \multicolumn{1}{l|}{\begin{tabular}[c]{@{}l@{}}Gradient \\ based\end{tabular}}                          & $L_p$                                                                                         & \multicolumn{1}{c|}{\xmark}                 & \multicolumn{1}{c|}{\checkmark}              & \multicolumn{1}{c|}{\xmark}       & \multicolumn{1}{c|}{\checkmark}                      & \multicolumn{1}{l|}{\begin{tabular}[c]{@{}l@{}}\rectangled{C-10}, \rectangled{C-100}, \\ \rectangled{M},   \rectangled{FM}, \\ \rectangled{S}\end{tabular}} & \multicolumn{1}{l|}{CNNs}                                                                    & \multicolumn{1}{l|}{\begin{tabular}[c]{@{}l@{}}Standard,\\ Adversarial\end{tabular}}  & \begin{tabular}[c]{@{}l@{}}PGD, FGSM, \\ BIM, MIM, \\C\&W \end{tabular}                                  \\ \hline
31                                        & Najafi et al.~\cite{Najafi:Maeda:Koyama:Miyato:NeurIPS:2019}                             & Number of samples                                   & \multicolumn{1}{l|}{Any}                                                                             & \multicolumn{1}{l|}{Any}                                                                             & \multicolumn{1}{l|}{Any}                                                                                         & \multicolumn{1}{l|}{\begin{tabular}[c]{@{}l@{}}Error-rate \\ based\end{tabular}}                             & \multicolumn{1}{l|}{White box}                                                                    & \multicolumn{1}{l|}{\begin{tabular}[c]{@{}l@{}}Gradient \\ based\end{tabular}}                          & $L_2, L_{\infty}$                                                                             & \multicolumn{1}{c|}{\checkmark}             & \multicolumn{1}{c|}{\checkmark}          & \multicolumn{1}{c|}{\xmark}      & \multicolumn{1}{c|}{\checkmark}                           & \multicolumn{1}{l|}{\begin{tabular}[c]{@{}l@{}}\rectangled{C-10}, \rectangled{M}, \\ \rectangled{S}\end{tabular}}                                             & \multicolumn{1}{l|}{DNNs}                                                                    & \multicolumn{1}{l|}{Adversarial}                                                      & PGD                                 \\ \hline
32                                        & Oritz-Jimenez et   al.~\cite{Ortiz-Jimenez:Modas:Moosavi-Dezfooli:Frossard:NeurIPS:2020} & Domain Specific                                     & \multicolumn{1}{l|}{Any}                                                                             & \multicolumn{1}{l|}{Any}                                                                             & \multicolumn{1}{l|}{CNNs}                                                                                         & \multicolumn{1}{l|}{Radius based}                                                                            & \multicolumn{1}{l|}{White box}                                                                    & \multicolumn{1}{l|}{\begin{tabular}[c]{@{}l@{}}Gradient \\ based\end{tabular}}                          & $L_2$                                                                                         & \multicolumn{1}{c|}{\checkmark}             & \multicolumn{1}{c|}{\xmark}          & \multicolumn{1}{c|}{\checkmark}        & \multicolumn{1}{c|}{\xmark}                 & \multicolumn{1}{l|}{\begin{tabular}[c]{@{}l@{}}\rectangled{C-10}, \rectangled{M}, \\ \rectangled{IN}\end{tabular}}                                            & \multicolumn{1}{l|}{CNNs}                                                                     & \multicolumn{1}{l|}{\begin{tabular}[c]{@{}l@{}}Standard,\\ Adversarial\end{tabular}}  & PGD \\ \hline
33                                        & Pang et al.~\cite{Pang:Du:Zhu:ICML:2018}                                                 & \begin{tabular}[c]{@{}l@{}}Distribution, \\ Separation \end{tabular}                                       & \multicolumn{1}{l|}{Any}                                                                             & \multicolumn{1}{l|}{Any}                                                                             & \multicolumn{1}{l|}{DNNs}                                                                                        & \multicolumn{1}{l|}{Radius based}                                                                            & \multicolumn{1}{l|}{White box}                                                                    & \multicolumn{1}{l|}{\begin{tabular}[c]{@{}l@{}}Gradient \\ based\end{tabular}}                          & $L_2$                                                                                         & \multicolumn{1}{c|}{\checkmark}             & \multicolumn{1}{c|}{\checkmark}             & \multicolumn{1}{c|}{\checkmark}         & \multicolumn{1}{c|}{\checkmark}        & \multicolumn{1}{l|}{\begin{tabular}[c]{@{}l@{}}\rectangled{C-10}, \rectangled{M}, \\ \rectangled{IN}\end{tabular}}                                            & \multicolumn{1}{l|}{DNNs}                                                                    & \multicolumn{1}{l|}{Standard}                                                         & \begin{tabular}[c]{@{}l@{}}FGSM, BIM\\ ILCM, JSMA\end{tabular}                                  \\ \hline
34                                        & Pang et al.~\cite{Pang:Xu:Dong:Du:Chen:Zhu:ICLR:2020}                                    & \begin{tabular}[c]{@{}l@{}} Density, \\ Separation \end{tabular}                                                                                & \multicolumn{1}{l|}{Any}                                                                             & \multicolumn{1}{l|}{Any}                                                                             & \multicolumn{1}{l|}{DNNs}                                                                                        & \multicolumn{1}{l|}{\begin{tabular}[c]{@{}l@{}}Error-rate \\ based\end{tabular}}                             & \multicolumn{1}{l|}{\begin{tabular}[c]{@{}l@{}}White box,\\ Black box\end{tabular}}               & \multicolumn{1}{l|}{\begin{tabular}[c]{@{}l@{}}Gradient \\ based,\\ Non-\\ gradient \\ based\end{tabular}} & $L_2, L_{\infty}$                                                                             & \multicolumn{1}{c|}{\checkmark}             & \multicolumn{1}{c|}{\checkmark}          & \multicolumn{1}{c|}{\xmark}         & \multicolumn{1}{c|}{\checkmark}          & \multicolumn{1}{l|}{\begin{tabular}[c]{@{}l@{}}\rectangled{C-10}, \rectangled{C-100}, \\ \rectangled{M}\end{tabular}}                                         & \multicolumn{1}{l|}{DNNs}                                                                    & \multicolumn{1}{l|}{\begin{tabular}[c]{@{}l@{}}Standard,\\ Adversarial\end{tabular}}  & \begin{tabular}[c]{@{}l@{}}PGD, FGSM, \\ Transfer-based \\ attacks\end{tabular}                                  \\ \hline
35                                        & Prescott et al.~\cite{Prescott:Zhang:Evans:ICLR:2021}                                    & Concentration                                       & \multicolumn{1}{l|}{\begin{tabular}[c]{@{}l@{}}Gaussian (theory), \\ Any (application)\end{tabular}} & \multicolumn{1}{l|}{Any}                                                                             & \multicolumn{1}{l|}{Any}                                                                                         & \multicolumn{1}{l|}{\begin{tabular}[c]{@{}l@{}}Error-rate \\ based\end{tabular}}                             & \multicolumn{1}{l|}{White box}                                                                    & \multicolumn{1}{l|}{Any}                                                                                & $L_p, p\ge 2$                                                                                 & \multicolumn{1}{c|}{\checkmark}             & \multicolumn{1}{c|}{\xmark}                     & \multicolumn{1}{c|}{\xmark}    & \multicolumn{1}{c|}{\checkmark}        & \multicolumn{1}{l|}{\begin{tabular}[c]{@{}l@{}}\rectangled{C-10}, \rectangled{M}, \\ \rectangled{FM},   \rectangled{S}\end{tabular}}                         & \multicolumn{1}{l|}{N/A}                                                                     & \multicolumn{1}{l|}{N/A}                                                              & N/A                                \\ \hline
36                                        & Pydi \& Jog~\cite{Pydi:Jog:ICML:2020}                                                 & Separation                                          & \multicolumn{1}{l|}{Any}                                                                             & \multicolumn{1}{l|}{\begin{tabular}[c]{@{}l@{}}Binary \\ Classif.\end{tabular}}                & \multicolumn{1}{l|}{Any}                                                                                         & \multicolumn{1}{l|}{\begin{tabular}[c]{@{}l@{}}Error-rate \\ based\end{tabular}}                             & \multicolumn{1}{l|}{White box}                                                                    & \multicolumn{1}{l|}{\begin{tabular}[c]{@{}l@{}}Gradient \\ based\end{tabular}}                          & $L_2, L_{\infty}$                                                                             & \multicolumn{1}{c|}{\checkmark}             & \multicolumn{1}{c|}{\xmark}                 & \multicolumn{1}{c|}{\xmark}   & \multicolumn{1}{c|}{\checkmark}           & \multicolumn{1}{l|}{\begin{tabular}[c]{@{}l@{}}\rectangled{C-10}, \rectangled{M}, \\ \rectangled{FM},   \rectangled{S}\end{tabular}}                         & \multicolumn{1}{l|}{DNNs}                                                                    & \multicolumn{1}{l|}{Adversarial}                                                      & N/A                                 \\ \hline
37                                        & Pydi \& Jog~\cite{Pydi:Jog:NeurIPS:2021}                                                 & Separation                                          & \multicolumn{1}{l|}{Any}                                                                             & \multicolumn{1}{l|}{\begin{tabular}[c]{@{}l@{}}Binary \\ Classif.\end{tabular}}                & \multicolumn{1}{l|}{Any}                                                                                         & \multicolumn{1}{l|}{\begin{tabular}[c]{@{}l@{}}Error-rate \\ based\end{tabular}}                             & \multicolumn{1}{l|}{White box}                                                                    & \multicolumn{1}{l|}{\begin{tabular}[c]{@{}l@{}}Gradient \\ based\end{tabular}}                          & $L_2, L_{\infty}$                                                                             & \multicolumn{1}{c|}{\checkmark}             & \multicolumn{1}{c|}{\xmark}                    & \multicolumn{1}{c|}{\xmark}   & \multicolumn{1}{c|}{\checkmark}         & \multicolumn{1}{l|}{N/A}                                                                                                                                        & \multicolumn{1}{l|}{N/A}                                                                     & \multicolumn{1}{l|}{N/A}                                                              & N/A                                                                                         \\ \hline
38                                        & Rajput et   al.~\cite{Rajput:Feng:Charles:Loh:Papailiopoulos:ICML:2019}                  & Dimensionality                                      & \multicolumn{1}{l|}{Any}                                                                             & \multicolumn{1}{l|}{Any}                                                                             & \multicolumn{1}{l|}{\begin{tabular}[c]{@{}l@{}}Linear \\ classifiers, \\ non-linear \\ classifiers\end{tabular}} & \multicolumn{1}{l|}{Radius based}                                                                            & \multicolumn{1}{l|}{Any}                                                                          & \multicolumn{1}{l|}{Any}                                                                                & $L_2$                                                                                         & \multicolumn{1}{c|}{\checkmark}             & \multicolumn{1}{c|}{\xmark}               & \multicolumn{1}{c|}{\xmark}         & \multicolumn{1}{c|}{\checkmark}        & \multicolumn{1}{l|}{N/A}                                                                                                                                        & \multicolumn{1}{l|}{N/A}                                                                     & \multicolumn{1}{l|}{N/A}                                                              & N/A                                                                                         \\ \hline
\end{tabular}
}
\end{sidewaystable}
\begin{sidewaystable}
\caption{Categorization Table for Papers - Part 3}
\centering
\scalebox{0.57}{
\begin{tabular}{|l|l|l|lllllll|ccccllll|}
\hline
\multicolumn{1}{|c|}{\multirow{4}{*}{ID}} & \multicolumn{1}{c|}{\multirow{4}{*}{Paper}}                                                   & \multicolumn{1}{c|}{\multirow{4}{*}{Data Property}}                          & \multicolumn{7}{c|}{Problem Setup}                                                                                                                                                                                                                                                                                                                                                                                                                                                                                                                                                                                                                                                                                                           & \multicolumn{8}{c|}{Practicality}                                                                                                                                                                                                                                                                                                                                                                                                                                                                                                                                                                              \\ \cline{4-18} 
\multicolumn{1}{|c|}{}                                                                       & \multicolumn{1}{c|}{}                                                                         & \multicolumn{1}{c|}{}                                                        & \multicolumn{1}{c|}{\multirow{3}{*}{\begin{tabular}[c]{@{}c@{}}Target \\ Distribution\end{tabular}}}            & \multicolumn{2}{c|}{Model}                                                                                                                                                                         & \multicolumn{4}{c|}{Robustness Setting}                                                                                                                                                                                                                                                                                                                                                                               & \multicolumn{2}{c|}{Applicability}                                                        & \multicolumn{1}{c|}{\multirow{3}{*}{Exp.}} & \multicolumn{5}{c|}{Type of Evidence}                                                                                                                                                                                                                                                                                                                                                                                                                                 \\ \cline{5-12} \cline{14-18} 
\multicolumn{1}{|c|}{}                                                                       & \multicolumn{1}{c|}{}                                                                         & \multicolumn{1}{c|}{}                                                        & \multicolumn{1}{c|}{}                                                                                           & \multicolumn{1}{c|}{\multirow{2}{*}{\begin{tabular}[c]{@{}c@{}}Learning \\ Task\end{tabular}}} & \multicolumn{1}{c|}{\multirow{2}{*}{\begin{tabular}[c]{@{}c@{}}Classifier \\ Type\end{tabular}}}  & \multicolumn{1}{c|}{\multirow{2}{*}{\begin{tabular}[c]{@{}c@{}}Definition \\ of \\ Robustness\end{tabular}}} & \multicolumn{1}{c|}{\multirow{2}{*}{\begin{tabular}[c]{@{}c@{}}Attacker's \\ Knwl.\end{tabular}}} & \multicolumn{1}{c|}{\multirow{2}{*}{\begin{tabular}[c]{@{}c@{}}Attacker's \\ Tech.\end{tabular}}} & \multicolumn{1}{c|}{\multirow{2}{*}{\begin{tabular}[c]{@{}c@{}}Perturb \\ Bound\end{tabular}}} & \multicolumn{1}{c|}{\multirow{2}{*}{Metr.}} & \multicolumn{1}{c|}{\multirow{2}{*}{Tech.}} & \multicolumn{1}{c|}{}                      & \multicolumn{1}{c|}{Fml.}       & \multicolumn{4}{c|}{Empirical}                                                                                                                                                                                                                                                                                                                                                                                                      \\ \cline{14-18} 
\multicolumn{1}{|c|}{}                                                                       & \multicolumn{1}{c|}{}                                                                         & \multicolumn{1}{c|}{}                                                        & \multicolumn{1}{c|}{}                                                                                           & \multicolumn{1}{c|}{}                                                                          & \multicolumn{1}{c|}{}                                                                             & \multicolumn{1}{c|}{}                                                                                        & \multicolumn{1}{c|}{}                                                                             & \multicolumn{1}{c|}{}                                                                             & \multicolumn{1}{c|}{}                                                                          & \multicolumn{1}{c|}{}                       & \multicolumn{1}{c|}{}                       & \multicolumn{1}{c|}{}                      & \multicolumn{1}{c|}{}           & \multicolumn{1}{c|}{Dataset}                                                                                                                                    & \multicolumn{1}{c|}{\begin{tabular}[c]{@{}c@{}}Classifier \\ Type\end{tabular}}            & \multicolumn{1}{l|}{\begin{tabular}[c]{@{}l@{}}Training \\ Proc.\end{tabular}} & Attacks                                                                             \\ \hline \midrule \hline
39                                        & Richardson \& Weiss ~\cite{Richardson:Weiss:JMLR:2021}                                    & Distribution                                        & \multicolumn{1}{l|}{Gaussian-mixture}                                                                           & \multicolumn{1}{l|}{\begin{tabular}[c]{@{}l@{}}Binary \\ Classif.\end{tabular}} & \multicolumn{1}{l|}{\begin{tabular}[c]{@{}l@{}}Bayes \\ optimal, \\ SVM,  \\ CNNs\end{tabular}} & \multicolumn{1}{l|}{Radius based}                                                                            & \multicolumn{1}{l|}{White box}                                                                    & \multicolumn{1}{l|}{Any}                                                                          & $L_2$                                                                                              & \multicolumn{1}{c|}{\xmark}                 & \multicolumn{1}{c|}{\xmark}                 & \multicolumn{1}{c|}{\xmark}                  & \multicolumn{1}{c|}{\xmark}                  & \multicolumn{1}{l|}{\rectangled{M}}                                                                                                         & \multicolumn{1}{l|}{\begin{tabular}[c]{@{}l@{}}Linear SVM, \\ Kernel SVM,\\ CNNs\end{tabular}}         & \multicolumn{1}{l|}{\begin{tabular}[c]{@{}l@{}}Standard,\\ Adversarial\end{tabular}}  & C\&W \\ \hline
40                                        & Sanyal et al.~\cite{Sanyal:Dokania:Kanade:Torr:ICLR:2021}                                 & Label Quality                                       & \multicolumn{1}{l|}{Any}                                                                                        & \multicolumn{1}{l|}{\begin{tabular}[c]{@{}l@{}}Binary \\ Classif.\end{tabular}} & \multicolumn{1}{l|}{Any}                                                                                  & \multicolumn{1}{l|}{\begin{tabular}[c]{@{}l@{}}Error-rate \\ based\end{tabular}}                             & \multicolumn{1}{l|}{White box}                                                                    & \multicolumn{1}{l|}{Any}                                                                          & Any                                                                                                & \multicolumn{1}{c|}{\xmark}                 & \multicolumn{1}{c|}{\xmark}                 & \multicolumn{1}{c|}{\checkmark}               & \multicolumn{1}{c|}{\checkmark}       & \multicolumn{1}{l|}{\rectangled{C-10},   \rectangled{M}}                                                                                   & \multicolumn{1}{l|}{DNNs}                                                                              & \multicolumn{1}{l|}{\begin{tabular}[c]{@{}l@{}}Standard, \\ Adversarial\end{tabular}} & PGD                                \\ \hline
41                                        & Schmidt et   al.~\cite{Schmidt:Santurkar:Tsipras:Talwar:Madry:NeurIPS:2018}               & \begin{tabular}[c]{@{}l@{}}Number of samples, \\ Distribution \end{tabular}                                                                         & \multicolumn{1}{l|}{\begin{tabular}[c]{@{}l@{}}Gaussian-mixture, \\ Bernoulli-mixture\end{tabular}}             & \multicolumn{1}{l|}{\begin{tabular}[c]{@{}l@{}}Binary \\ Classif.\end{tabular}} & \multicolumn{1}{l|}{Any}                                                                                  & \multicolumn{1}{l|}{\begin{tabular}[c]{@{}l@{}}Error-rate \\ based\end{tabular}}                             & \multicolumn{1}{l|}{White box}                                                                    & \multicolumn{1}{l|}{Any}                                                                          & $L_{\infty}$                                                                                       & \multicolumn{1}{c|}{\checkmark}             & \multicolumn{1}{c|}{\xmark}            & \multicolumn{1}{c|}{\xmark}       & \multicolumn{1}{c|}{\checkmark}               & \multicolumn{1}{l|}{\begin{tabular}[c]{@{}l@{}}\rectangled{C-10},   \rectangled{M}, \\ \rectangled{S}\end{tabular}}                       & \multicolumn{1}{l|}{DNNs}                                                                              & \multicolumn{1}{l|}{Adversarial}                                                      & PGD                                \\ \hline
42                                        & Shafahi et   al.~\cite{Shafahi:Huang:Studer:Feizi:Goldstein:ICLR:2019}                    & {\begin{tabular}[c]{@{}l@{}}Dimesionality, \\ Density\end{tabular}}                             & \multicolumn{1}{l|}{\begin{tabular}[c]{@{}l@{}}N-dimensional\\ hypercube\end{tabular}}                          & \multicolumn{1}{l|}{Any}                                                              & \multicolumn{1}{l|}{Any}                                                                                  & \multicolumn{1}{l|}{Radius-based}                                                                            & \multicolumn{1}{l|}{White box}                                                                    & \multicolumn{1}{l|}{Any}                                                                          & \begin{tabular}[c]{@{}l@{}}$L_p$, \\ Geodesic\end{tabular}                                         & \multicolumn{1}{c|}{\xmark}                 & \multicolumn{1}{c|}{\xmark}                 & \multicolumn{1}{c|}{\checkmark}         & \multicolumn{1}{c|}{\checkmark}      & \multicolumn{1}{l|}{\rectangled{C-10},   \rectangled{M}}                                                                                   & \multicolumn{1}{l|}{CNN}                                                                               & \multicolumn{1}{l|}{Adversarial}                                                      & PGD                              \\ \hline
43                                        & Simon-Gabriel et   al.~\cite{Simon-Gabriel:Ollivier:Scholkopf:Bottou:Lopez-Paz:ICML:2019} & Dimensionality                                      & \multicolumn{1}{l|}{Any}                                                                                        & \multicolumn{1}{l|}{Any}                                                              & \multicolumn{1}{l|}{DNNs}                                                                                 & \multicolumn{1}{l|}{\begin{tabular}[c]{@{}l@{}}Error-rate \\ based\end{tabular}}                             & \multicolumn{1}{l|}{White box}                                                                    & \multicolumn{1}{l|}{Any}                                                                          & Any                                                                                                & \multicolumn{1}{c|}{\checkmark}             & \multicolumn{1}{c|}{\xmark}             & \multicolumn{1}{c|}{\xmark}      & \multicolumn{1}{c|}{\checkmark}               & \multicolumn{1}{l|}{\rectangled{C-10}}                                                                                                      & \multicolumn{1}{l|}{DNNs}                                                                              & \multicolumn{1}{l|}{Adversarial}                                                      & PGD                                \\ \hline
44                                        & Song et al.~\cite{Song:Kim:Nowozin:Ermon:Kushman:ICLR:2017}                               & Density                                             & \multicolumn{1}{l|}{Any}                                                                                        & \multicolumn{1}{l|}{Any}                                                              & \multicolumn{1}{l|}{Any}                                                                                  & \multicolumn{1}{l|}{\begin{tabular}[c]{@{}l@{}}Error-rate \\ based\end{tabular}}                             & \multicolumn{1}{l|}{Any}                                                                          & \multicolumn{1}{l|}{Any}                                                                          & Any                                                                                                & \multicolumn{1}{c|}{\checkmark}             & \multicolumn{1}{c|}{\checkmark}             & \multicolumn{1}{c|}{\xmark}             & \multicolumn{1}{c|}{\xmark}                 & \multicolumn{1}{l|}{\begin{tabular}[c]{@{}l@{}}\rectangled{C-10},   \rectangled{M}, \\ \rectangled{FM}\end{tabular}}                      & \multicolumn{1}{l|}{CNNs}                                                                              & \multicolumn{1}{l|}{Adversarial}                                                      & \begin{tabular}[c]{@{}l@{}}FGSM, BIM \\ C\&W, \\ DeepFool \end{tabular}                            \\ \hline
45                                        & Uesato et   al.~\cite{Uesato:Alayrac:Huang:Stanforth:Fawzi:Kohli:NeurIPS:2019}            & Number of samples                                   & \multicolumn{1}{l|}{\begin{tabular}[c]{@{}l@{}}Gaussian-mixture \\ (theory), \\ Any (application)\end{tabular}} & \multicolumn{1}{l|}{\begin{tabular}[c]{@{}l@{}}Binary \\ Classif.\end{tabular}} & \multicolumn{1}{l|}{Any}                                                                                  & \multicolumn{1}{l|}{\begin{tabular}[c]{@{}l@{}}Error-rate \\ based\end{tabular}}                             & \multicolumn{1}{l|}{White box}                                                                    & \multicolumn{1}{l|}{Any}                                                                          & $L_{\infty}$                                                                                       & \multicolumn{1}{c|}{\checkmark}             & \multicolumn{1}{c|}{\checkmark}           & \multicolumn{1}{c|}{\xmark}       & \multicolumn{1}{c|}{\checkmark}                     & \multicolumn{1}{l|}{\rectangled{C-10},   \rectangled{S}}                                                                                   & \multicolumn{1}{l|}{DNNs}                                                                              & \multicolumn{1}{l|}{Adversarial}                                                      & PGD, FGSM                                 \\ \hline
46                                        & Wan et al.~\cite{Wan:Chen:Yu:Wu:Zhong:Yang:TPAMI:2022}                                    & \begin{tabular}[c]{@{}l@{}}Distribution, \\ Separation \end{tabular}                                        & \multicolumn{1}{l|}{Any}                                                                                        & \multicolumn{1}{l|}{Any}                                                              & \multicolumn{1}{l|}{Any}                                                                                  & \multicolumn{1}{l|}{\begin{tabular}[c]{@{}l@{}}Error-rate \\ based\end{tabular}}                             & \multicolumn{1}{l|}{White box}                                                                    & \multicolumn{1}{l|}{Any}                                                                          & $L_{\infty}$                                                                                       & \multicolumn{1}{c|}{\xmark}                 & \multicolumn{1}{c|}{\checkmark}          & \multicolumn{1}{c|}{\xmark}            & \multicolumn{1}{c|}{\checkmark}                        & \multicolumn{1}{l|}{\begin{tabular}[c]{@{}l@{}}\rectangled{C-10},   \rectangled{M}, \\ \rectangled{IN}\end{tabular}}                      & \multicolumn{1}{l|}{DNNs}                                                                             & \multicolumn{1}{l|}{Standard}                                                         &                                 \begin{tabular}[c]{@{}l@{}}FGSM,  BIM, \\ ILCM, C\&W\end{tabular} \\ \hline
47                                        & Wang et al.~\cite{Wang:Wu:Huang:Xing:CVPR:2020}                                           & Domain Specific                                     & \multicolumn{1}{l|}{Any}                                                                                        & \multicolumn{1}{l|}{Any}                                                              & \multicolumn{1}{l|}{CNNs}                                                                                 & \multicolumn{1}{l|}{\begin{tabular}[c]{@{}l@{}}Error-rate \\ based\end{tabular}}                             & \multicolumn{1}{l|}{White box}                                                                    & \multicolumn{1}{l|}{\begin{tabular}[c]{@{}l@{}}Gradient\\ based\end{tabular}}                     & $L_2$                                                                                              & \multicolumn{1}{c|}{\checkmark}             & \multicolumn{1}{c|}{\xmark}                & \multicolumn{1}{c|}{\checkmark}    & \multicolumn{1}{c|}{\xmark}                              & \multicolumn{1}{l|}{\rectangled{C-10}}                                                                                                      & \multicolumn{1}{l|}{CNNs}                                                                              & \multicolumn{1}{l|}{\begin{tabular}[c]{@{}l@{}}Standard,\\ Adversarial\end{tabular}}  & PGD, FGSM                                 \\ \hline
48                                        & Wang et al.~\cite{Wang:Jha:Chaudhuri:ICML:2018}                                           & \begin{tabular}[c]{@{}l@{}}Dimensionality, \\ Separation \end{tabular}                                       & \multicolumn{1}{l|}{Any}                                                                                        & \multicolumn{1}{l|}{\begin{tabular}[c]{@{}l@{}}Binary \\ Classif.\end{tabular}} & \multicolumn{1}{l|}{kNN}                                                                                  & \multicolumn{1}{l|}{Radius based}                                                                            & \multicolumn{1}{l|}{White box}                                                                    & \multicolumn{1}{l|}{Any}                                                                          & $L_2$                                                                                              & \multicolumn{1}{c|}{\checkmark}             & \multicolumn{1}{c|}{\checkmark}             & \multicolumn{1}{c|}{\xmark}                               & \multicolumn{1}{c|}{\xmark}   & \multicolumn{1}{l|}{\begin{tabular}[c]{@{}l@{}}\rectangled{M},   \rectangled{M1V7} , \\ \rectangled{HM}\end{tabular}}                     & \multicolumn{1}{l|}{$k$-NN}                                                                               & \multicolumn{1}{l|}{Adversarial}                                                      & \begin{tabular}[c]{@{}l@{}}Direct attack,\\ Transfer-based \\ attacks\end{tabular}                             \\ \hline
49                                        & Weber et   al.~\cite{Weber:Zaheer:Rawat:Menon:Kumar:NeurIPS:2020}                         & Dimensionality                                      & \multicolumn{1}{l|}{Hierarchial data}                                                                           & \multicolumn{1}{l|}{Any}                                                              & \multicolumn{1}{l|}{Any}                                                                                  & \multicolumn{1}{l|}{\begin{tabular}[c]{@{}l@{}}Error-rate \\ based\end{tabular}}                             & \multicolumn{1}{l|}{White box}                                                                    & \multicolumn{1}{l|}{Any}                                                                          & Check                                                                                              & \multicolumn{1}{c|}{\checkmark}             & \multicolumn{1}{c|}{\checkmark}               & \multicolumn{1}{c|}{\checkmark}        & \multicolumn{1}{c|}{\xmark}                           & \multicolumn{1}{l|}{\rectangled{IN}}                                                                                                        & \multicolumn{1}{l|}{\begin{tabular}[c]{@{}l@{}}Hyperbolic\\ perceptron\end{tabular}}                   & \multicolumn{1}{l|}{Adversarial}                                                      & \begin{tabular}[c]{@{}l@{}}Gradient \\ based\end{tabular}                                 \\ \hline
50                                        & Wu et al.~\cite{Wu:Liu:Huang:Wang:Lin:CVPR:2021}                                          & Number of samples                                   & \multicolumn{1}{l|}{Any}                                                                                        & \multicolumn{1}{l|}{Any}                                                              & \multicolumn{1}{l|}{DNNs}                                                                                 & \multicolumn{1}{l|}{\begin{tabular}[c]{@{}l@{}}Error-rate \\ based\end{tabular}}                             & \multicolumn{1}{l|}{White box}                                                                    & \multicolumn{1}{l|}{\begin{tabular}[c]{@{}l@{}}Gradient \\ based\end{tabular}}                    & $L_{\infty}$                                                                                       & \multicolumn{1}{c|}{\checkmark}             & \multicolumn{1}{c|}{\checkmark}             & \multicolumn{1}{c|}{\xmark}                             & \multicolumn{1}{c|}{\xmark}      & \multicolumn{1}{l|}{\rectangled{C-10},   \rectangled{C-100}}                                                                               & \multicolumn{1}{l|}{DNNs}                                                                              & \multicolumn{1}{l|}{\begin{tabular}[c]{@{}l@{}}Standard,\\ Adversarial\end{tabular}}  & \begin{tabular}[c]{@{}l@{}}PGD C\&W \\ Transfer-based \\attacks\end{tabular}                                 \\ \hline
51                                        & Yang et al.~\cite{Yang:Feng:Du:Du:Xu:ICDM:2021}                                           & Separation                                           & \multicolumn{1}{l|}{Any}                                                                                        & \multicolumn{1}{l|}{Any}                                                              & \multicolumn{1}{l|}{DNNs}                                                                                 & \multicolumn{1}{l|}{\begin{tabular}[c]{@{}l@{}}Error-rate \\ based\end{tabular}}                             & \multicolumn{1}{l|}{White box}                                                                    & \multicolumn{1}{l|}{\begin{tabular}[c]{@{}l@{}}Gradient \\ based\end{tabular}}                    & $L_2$                                                                                              & \multicolumn{1}{c|}{\checkmark}             & \multicolumn{1}{c|}{\checkmark}                & \multicolumn{1}{c|}{\xmark}           & \multicolumn{1}{c|}{\checkmark}                     & \multicolumn{1}{l|}{\begin{tabular}[c]{@{}l@{}}\rectangled{C-10},   \rectangled{C-100}, \\ \rectangled{M}, \rectangled{TI}\end{tabular}} & \multicolumn{1}{l|}{DNNs}                                                                              & \multicolumn{1}{l|}{Adversarial}                                                      & PGD                                  \\ \hline
52                                        & Yang et   al.~\cite{Yang:Rashtchian:Wang:Chaudhuri:AISTATS:2020}                          & Separation                                          & \multicolumn{1}{l|}{Any}                                                                                        & \multicolumn{1}{l|}{\begin{tabular}[c]{@{}l@{}}Binary \\ Classif.\end{tabular}} & \multicolumn{1}{l|}{\begin{tabular}[c]{@{}l@{}}Non-\\ parametric \\ classifiers\end{tabular}}             & \multicolumn{1}{l|}{Radius based}                                                                            & \multicolumn{1}{l|}{White box}                                                                    & \multicolumn{1}{l|}{\begin{tabular}[c]{@{}l@{}}Distance\\ based\end{tabular}}                     & $L_2$                                                                                              & \multicolumn{1}{c|}{\checkmark}             & \multicolumn{1}{c|}{\checkmark}            & \multicolumn{1}{c|}{\xmark}      & \multicolumn{1}{c|}{\checkmark}              & \multicolumn{1}{l|}{\rectangled{HM}}                                                                                                        & \multicolumn{1}{l|}{\begin{tabular}[c]{@{}l@{}}Histogram,\\ 1-NN\end{tabular}}                         & \multicolumn{1}{l|}{Standard}                                                         & \begin{tabular}[c]{@{}l@{}}Distance\\ based\end{tabular}                                  \\ \hline
53                                                                                           & Yin et al.~\cite{Yin:Lopes:Shlens:Cubuk:Gilmer:NeurIPS:2019}                                & Domain Specific                                                              & \multicolumn{1}{l|}{Any}                                                                                        & \multicolumn{1}{l|}{Any}                                                              & \multicolumn{1}{l|}{Any}                                                                          & \multicolumn{1}{l|}{\begin{tabular}[c]{@{}l@{}}Error-rate \\ based\end{tabular}}                             & \multicolumn{1}{l|}{White box}                                                                    & \multicolumn{1}{l|}{Any}                                                                          & $L_2$                                                                                               & \multicolumn{1}{c|}{\checkmark}             & \multicolumn{1}{c|}{\xmark}                 & \multicolumn{1}{c|}{\xmark}             & \multicolumn{1}{c|}{\xmark}          & \multicolumn{1}{l|}{\rectangled{C-10},  \rectangled{IN}}                                                                                                       & \multicolumn{1}{l|}{DNNs}                                                                  & \multicolumn{1}{l|}{Adversarial}                                               & \begin{tabular}[c]{@{}l@{}} Corruptions, \\ PGD \end{tabular}                                 \\ \hline
54                                        & Yin et al.~\cite{Yin:Kannan:Bartlett:ICML:2019}                                           & Dimensionality                                      & \multicolumn{1}{l|}{Any}                                                                                        & \multicolumn{1}{l|}{Any}                                                              & \multicolumn{1}{l|}{\begin{tabular}[c]{@{}l@{}}Linear \\ classifiers, \\ DNNs\end{tabular}}               & \multicolumn{1}{l|}{\begin{tabular}[c]{@{}l@{}}Error-rate \\ based\end{tabular}}                             & \multicolumn{1}{l|}{White box}                                                                    & \multicolumn{1}{l|}{Any}                                                                          & $L_{\infty}$                                                                                       & \multicolumn{1}{c|}{\checkmark}             & \multicolumn{1}{c|}{\xmark}              & \multicolumn{1}{c|}{\xmark}         & \multicolumn{1}{c|}{\checkmark}                     & \multicolumn{1}{l|}{\rectangled{M}}                                                                                                         & \multicolumn{1}{l|}{\begin{tabular}[c]{@{}l@{}}Linear \\ classifiers,\\ ReLU \\ networks\end{tabular}} & \multicolumn{1}{l|}{Adversarial}                                                      & PGD                               \\ \hline
55                                        & Zhang et   al.~\cite{Zhang:Chen:Song:Boning:Dhillon:Hsieh:ICLR:2019}                      & Density                                             & \multicolumn{1}{l|}{Any}                                                                                        & \multicolumn{1}{l|}{Any}                                                              & \multicolumn{1}{l|}{Any}                                                                                  & \multicolumn{1}{l|}{\begin{tabular}[c]{@{}l@{}}Error-rate \\ based\end{tabular}}                             & \multicolumn{1}{l|}{White box}                                                                    & \multicolumn{1}{l|}{Any}                                                                          & $L_2, L_{\infty}$                                                                                  & \multicolumn{1}{c|}{\checkmark}             & \multicolumn{1}{c|}{\xmark}                 & \multicolumn{1}{c|}{\xmark}                & \multicolumn{1}{c|}{\xmark}        & \multicolumn{1}{l|}{\begin{tabular}[c]{@{}l@{}}\rectangled{C-10},   \rectangled{M}, \\ \rectangled{FM}\end{tabular}}                      & \multicolumn{1}{l|}{DNNs}                                                                              & \multicolumn{1}{l|}{Adversarial}                                                      & C\&W                                 \\ \hline
56                                        & Zhang \& Evans~\cite{Zhang:Evans:ICLR:2022}                                               & Concentration                                       & \multicolumn{1}{l|}{\begin{tabular}[c]{@{}l@{}}Gaussian (theory), \\ Any (application)\end{tabular}}            & \multicolumn{1}{l|}{Any}                                                              & \multicolumn{1}{l|}{Any}                                                                                  & \multicolumn{1}{l|}{\begin{tabular}[c]{@{}l@{}}Error-rate \\ based\end{tabular}}                             & \multicolumn{1}{l|}{White box}                                                                    & \multicolumn{1}{l|}{Any}                                                                          & $L_2, L_{\infty}$                                                                                  & \multicolumn{1}{c|}{\checkmark}             & \multicolumn{1}{c|}{\xmark}                & \multicolumn{1}{c|}{\xmark}    & \multicolumn{1}{c|}{\checkmark}               & \multicolumn{1}{l|}{\rectangled{C-10}}                                                                                                      & \multicolumn{1}{l|}{DNNs}                                                                              & \multicolumn{1}{l|}{\begin{tabular}[c]{@{}l@{}}Standard, \\ Adversarial\end{tabular}} & AutoAttack                               \\ \hline
57                                        & Zhu et al.~\cite{Zhu:Sun:Li:ICLR:2022}                                                    & Density                                             & \multicolumn{1}{l|}{Any}                                                                                        & \multicolumn{1}{l|}{Any}                                                              & \multicolumn{1}{l|}{DNNs}                                                                                 & \multicolumn{1}{l|}{\begin{tabular}[c]{@{}l@{}}Error-rate \\ based\end{tabular}}                             & \multicolumn{1}{l|}{Any}                                                                          & \multicolumn{1}{l|}{Any}                                                                          & $L_{\infty}$                                                                                       & \multicolumn{1}{c|}{\checkmark}             & \multicolumn{1}{c|}{\xmark}               & \multicolumn{1}{c|}{\xmark}          & \multicolumn{1}{c|}{\checkmark}            & \multicolumn{1}{l|}{\rectangled{IN}}                                                                                                        & \multicolumn{1}{l|}{CNNs}                                                                              & \multicolumn{1}{l|}{Adversarial}                                                      & \begin{tabular}[c]{@{}l@{}} PGD \\ Transfer-based \\ attacks \end{tabular}                                 \\ \hline
\end{tabular}
}
\end{sidewaystable}
%\input{fullCategorizationTbl_p1c}

%\nocite{Xiao:Rasul:Vollgraf:ArXiv:FashionMNIST:2017}
%\nocite{Krizhevsky:Sutskever:Hinton:ComACM:ImageNet:2012}