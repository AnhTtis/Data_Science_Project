\section{Paper Organization}
\label{sec:organization}
\jr{where do we explain how we came up with these dimensions? Should probably be part of the methodology.}

\jr{this is not actually "paper organization", as the paper is organized into very different subsections. Will think later where this section belongs. Perhaps it should rather be part of methodology, to explain how we categorized the papers.}


The categorization dimensions we use to analyze the collected papers are depicted in Fig.~\ref{fig:categorization}.
We characterize the papers using these dimensions so that it is easier for the reader to understand paper contributions and compare related works.
The resulting table that contains specific entries for each paper is included in the supplementary material.
\begin{figure*}[th!]
	\centering
	\includegraphics[width=13.5cm, height=2.8cm] {images/smaller_categorization.pdf}
	\caption{Paper categorization dimensions.}
	\label{fig:categorization}
	\vspace{-0.1in}
\end{figure*}

This section explains the dimensions themselves and the kind of attributes they represent 


\subsection{Problem Setup}
Problem setup contains information about the adversarial robustness setting the authors consider in their work.
In other words, it is the kind of problem the authors analyze and present their findings for.
As shown in fig.~\ref{fig:categorization} \jr{Figure, Table, Section, should always be capitalized}, it contains information about the target distribution, attack, robustness setting and model.

\begin{itemize}
	\item \textbf{Attack}: This dimension records information pertaining to how the attack is constructed and bounded, namely, the attack technique which can be either gradient-based or non-gradient based, and the perturbation bound which can be any of the $L_p$ norms.
%		\begin{itemize}
%			\item \textbf{Techique}
%			\item \textbf{Perturbation Bound}
%		\end{itemize}
	\item \textbf{Robustness Setting}: \mt {Need to update. This is related to perturbation bound information.} \jr{is this still incomplete?}
	This dimension records information about how the paper characterizes robustness.
	Namely, the robustness definition that can be one of radius-based, error-rate based and rank-based, and the attacker's knowledge that can be either white-box, grey-box or black-box.
%		\begin{itemize}
%			\item \textbf{Robustness Definition}
%			\item \textbf{Attacker's Knowledge}
%		\end{itemize}
	\item \textbf{Target Distribution}: This dimension contains information about the type of data the papers analyze. Based on the collected papers, the common target data distributions include Gaussian mixtures, Bernoulli mixtures, Boolean Hypercube, Volume Hypercubes, Heirarchical data, Concentrated distributions and Smooth generative model based distributions.
	\item \textbf{Model}: This dimension captures information about the classification problem and the classifier.
	One sub-dimension is the classifier type which can be one of Linear, DNN, ReLU networks, and non-parametric models such as KNNs and Kernel SVMs.
	The second sub-dimension is the classification type that can be one of Binary, Multi-class, Regression or Inference. \mt{The name should be changed since "regression" does not fall under classification"} 
%		\begin{itemize}
%			\item \textbf{Classifier Type}
%			\item \textbf{Classification Type}
%		\end{itemize}
\end{itemize}

\subsection{Data Property}
 \gx{We also need to briefly describe the data properties we identified?}

\subsection{Applicability}
The Applicability dimension contains information about actionability and evaluation methods of the papers.
This dimension is slightly blurred in Fig.~\ref{fig:categorization} to indicate that it will be discussed in more detail under the Discussion section.
\begin{itemize}
	\item \textbf{Actionability}: This dimension collects actionable information the paper presents. A typical example is a paper presenting a metric or modification technique of particular data property of a dataset.
	Hence, this dimension can be a technique or an algorithm.
	\item \textbf{Type of Evidence}: This dimension contains information about the empirical evaluation the authors perform, if it exists.
	Hence, it contains information about the dataset, classifier, training algorithm, attack, and the robustness definition used by the authors.
\end{itemize}

