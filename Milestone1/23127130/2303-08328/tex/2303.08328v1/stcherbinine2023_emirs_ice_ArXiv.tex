%%%%%%%%%%%%%%%%%%%%%%%%%%%%%%%%%%%%%%%%%%%%%%%%%%%%%%%%%%%%%%%%%%%%%%%%%%%%
% AGUJournalTemplate.tex: this template file is for articles formatted with LaTeX
%
% This file includes commands and instructions
% given in the order necessary to produce a final output that will
% satisfy AGU requirements, including customized APA reference formatting.
%
% You may copy this file and give it your
% article name, and enter your text.
%
%
% Step 1: Set the \documentclass
%
%

%% To submit your paper:
%\documentclass[draft]{agujournal2019}
\documentclass{agujournal2019}
\usepackage{url} %this package should fix any errors with URLs in refs.
\usepackage{lineno}
\usepackage[inline]{trackchanges} %for better track changes. finalnew option will compile document with changes incorporated.
\usepackage{soul}
\usepackage{upgreek}
%\linenumbers
\newcommand{\mum}{$\upmu$m}
\newcommand{\coo}{CO\textsubscript{2}}
\newcommand{\hoo}{H\textsubscript{2}O}
\newcommand{\Tcoo}{$T$\textsubscript{CO\textsubscript{2}}}
% \newcommand{\Tcoo}{$T_\mathrm{CO\textsubscript{2}}$}

\newcommand{\td}[1]{\textcolor{red}{\bfseries #1}}
%%%%%%%
% As of 2018 we recommend use of the TrackChanges package to mark revisions.
% The trackchanges package adds five new LaTeX commands:
%
%  \note[editor]{The note}
%  \annote[editor]{Text to annotate}{The note}
%  \add[editor]{Text to add}
%  \remove[editor]{Text to remove}
%  \change[editor]{Text to remove}{Text to add}
%
% complete documentation is here: http://trackchanges.sourceforge.net/
%%%%%%%

\draftfalse

%% Enter journal name below.
%% Choose from this list of Journals:
%
% JGR: Atmospheres
% JGR: Biogeosciences
% JGR: Earth Surface
% JGR: Oceans
% JGR: Planets
% JGR: Solid Earth
% JGR: Space Physics
% Global Biogeochemical Cycles
% Geophysical Research Letters
% Paleoceanography and Paleoclimatology
% Radio Science
% Reviews of Geophysics
% Tectonics
% Space Weather
% Water Resources Research
% Geochemistry, Geophysics, Geosystems
% Journal of Advances in Modeling Earth Systems (JAMES)
% Earth's Future
% Earth and Space Science
% Geohealth
%
% ie, \journalname{Water Resources Research}

\journalname{Geophysical Research Letters}


\begin{document}

%% ------------------------------------------------------------------------ %%
%  Title
%
% (A title should be specific, informative, and brief. Use
% abbreviations only if they are defined in the abstract. Titles that
% start with general keywords then specific terms are optimized in
% searches)
%
%% ------------------------------------------------------------------------ %%

% Example: \title{This is a test title}

% \title{Martian Diurnal and Seasonal Ice Mapping from EMIRS Surface Temperature Retrievals}
\title{Diurnal and Seasonal Mapping of Martian Ices with EMIRS}

%% ------------------------------------------------------------------------ %%
%
%  AUTHORS AND AFFILIATIONS
%
%% ------------------------------------------------------------------------ %%

% Authors are individuals who have significantly contributed to the
% research and preparation of the article. Group authors are allowed, if
% each author in the group is separately identified in an appendix.)

% List authors by first name or initial followed by last name and
% separated by commas. Use \affil{} to number affiliations, and
% \thanks{} for author notes.
% Additional author notes should be indicated with \thanks{} (for
% example, for current addresses).

% Example: \authors{A. B. Author\affil{1}\thanks{Current address, Antartica}, B. C. Author\affil{2,3}, and D. E.
% Author\affil{3,4}\thanks{Also funded by Monsanto.}}

\authors{Aur\'elien Stcherbinine\affil{1}, Christopher S. Edwards\affil{1},
    Michael D. Smith\affil{2},
    Michael J. Wolff\affil{3}, 
    Christopher Haberle\affil{1},
    Eman Al Tunaiji\affil{4},
    Nathan M. Smith\affil{1},
    Kezman Saboi\affil{1},
    Saadat Anwar\affil{5},
    Lucas Lange\affil{6},
    Philip R. Christensen\affil{5}
    }


% \affiliation{1}{First Affiliation}
% \affiliation{2}{Second Affiliation}
% \affiliation{3}{Third Affiliation}
% \affiliation{4}{Fourth Affiliation}

\affiliation{1}{Department of Astronomy and Planetary Science, Northern Arizona University, 
    Flagstaff, AZ, USA}
\affiliation{2}{NASA Goddard Space Flight Center, Greenbelt, MD, USA}
\affiliation{3}{Space Science Institute, Boulder, CO, USA}
\affiliation{4}{Mohammed Bin Rashid Space Centre, Dubai, UAE}
\affiliation{5}{School of Earth and Space Exploration, Arizona State University, Tempe, AZ, USA}
\affiliation{6}{Laboratoire de Météorologie Dynamique (LMD/IPSL), Sorbonne Université, ENS, École Polytechnique, CNRS, Paris, France}

%(repeat as many times as is necessary)

%% Corresponding Author:
% Corresponding author mailing address and e-mail address:

% (include name and email addresses of the corresponding author.  More
% than one corresponding author is allowed in this LaTeX file and for
% publication; but only one corresponding author is allowed in our
% editorial system.)

% Example: \correspondingauthor{First and Last Name}{email@address.edu}

\correspondingauthor{Aur\'elien Stcherbinine}{aurelien.stcherbinine@nau.edu}

%% Keypoints, final entry on title page.

%  List up to three key points (at least one is required)
%  Key Points summarize the main points and conclusions of the article
%  Each must be 140 characters or fewer with no special characters or punctuation and must be complete sentences

% Example:
% \begin{keypoints}
% \item	List up to three key points (at least one is required)
% \item	Key Points summarize the main points and conclusions of the article
% \item	Each must be 140 characters or fewer with no special characters or punctuation and must be complete sentences
% \end{keypoints}

\begin{keypoints}
\item We monitor the seasonal growth and retreat of both polar caps over MY 36.
\item \coo\ ice appears at the surface at equatorial latitudes during the second half of the night
    around the equinoxes.
% \item enter point 3 here (spectral?)
\end{keypoints}

%% ------------------------------------------------------------------------ %%
%
%  ABSTRACT and PLAIN LANGUAGE SUMMARY
%
% A good Abstract will begin with a short description of the problem
% being addressed, briefly describe the new data or analyses, then
% briefly states the main conclusion(s) and how they are supported and
% uncertainties.

% The Plain Language Summary should be written for a broad audience,
% including journalists and the science-interested public, that will not have 
% a background in your field.
%
% A Plain Language Summary is required in GRL, JGR: Planets, JGR: Biogeosciences,
% JGR: Oceans, G-Cubed, Reviews of Geophysics, and JAMES.
% see http://sharingscience.agu.org/creating-plain-language-summary/)
%
%% ------------------------------------------------------------------------ %%

%% \begin{abstract} starts the second page

\begin{abstract}    % 150 words max
Condensation and sublimation of ices at the surface of the planet is a key part of both the Martian \hoo\ and \coo\ cycles, either from a seasonal or diurnal aspect.
While most of the ice is located within the polar caps, surface frost is known to be formed during nighttime down to equatorial latitudes.
Here, we use data from the Emirates Mars Infrared Spectrometer (EMIRS) onboard the Emirates Mars Mission (EMM) to monitor the diurnal and seasonal evolution of the ices at the surface of Mars over almost one Martian year.
The unique local time coverage provided by the instrument allows us to observe the apparition of equatorial \coo\ frost in the second half of the Martian night around the equinoxes, to its sublimation at sunrise.

\end{abstract}

\section*{Plain Language Summary}   % 200 words max
% [ enter your Plain Language Summary here or delete this section]
The \hoo\ and \coo\ ices that form at the surface on Mars play an important role in the exchange between the atmosphere and the surface of the planet.
While most of the ice is located within the two polar caps that grew and shrink seasonally, ice is also known to condensate as surface frost during the night and sublimate during the day. This nighttime surface frost deposition can be observed even at equatorial latitudes.
In this paper we use data from the Emirates Mars Infrared Spectrometer onboard the Emirates Mars Mission to detect the \hoo\ and \coo\ ices at the surface of the planet at all local times over almost one Martian Year, which allows us to monitor both the seasonal and diurnal evolution of the distribution of ices at the surface of Mars. We observe that nighttime \coo\ frost forms equatorial latitudes in the second half of the night to disappear at sunrise around the Martian equinoxes.


%% ------------------------------------------------------------------------ %%
%
%  TEXT
%
%% ------------------------------------------------------------------------ %%

\section{Introduction}  \label{sec:intro}
    The Martian polar caps are the main reservoirs of both \hoo\ and \coo\ ices on the surface of the red planet, at the interface between the surface and the atmosphere with an active exchange of volatiles by sublimation/condensation.
    Their seasonal growth and retreat \cite<e.g.,>{kieffer_2000, kieffer_2001, langevin_2005a, langevin_2007, appere_2011, calvin_2015, calvin_2017, oliva_2022} are an important process in both the \hoo\ and \coo\ cycles on the present-day Mars, which are two majors features of the global Martian atmospheric circulation \cite{forget_1999, montmessin_2017, titus_2017}.
    % Every year, a significant fraction of the atmospheric \coo\ condensates into the seasonal polar caps which results in annual variations of the atmospheric pressure of about 3~mbar recorded under North polar latitudes by the Viking landers \cite{leighton_1966, hourdin_1993, forget_1998}, i.e., one-third of the mean surface pressure.
    Every year, a significant fraction of the atmospheric \coo\ condensates into the seasonal polar caps which results in annual variations of about one-third of the global atmospheric mass \cite{leighton_1966, james_1992, hourdin_1993, hourdin_1995, forget_1998}.
    In addition, the presence of ice at the surface of the planet also changes the albedo and the surface thermal inertia, thus affecting the energy budget at a local but also planetary scale.

    While the ices are essentially located in the polar regions, both \coo\ \& \hoo\ ices can also be observed seasonally under low latitudes, in the shadows of pole-facing slopes or at the bottom of some craters \cite{schorghofer_2006, brown_2008, carrozzo_2009, vincendon_2010, conway_2012, lange_2022}.
    From a more transient perspective, frost has also been observed to be deposed on the surface during nighttime, and remaining until early morning \cite{landis_2007, piqueux_2016}.
    The presence of this daily cycle of \coo\ frost has been shown to have an impact on surface processes such as gullies or slope streaks formation \cite{pilorget_2016, khuller_2021a, lange_2022}.
    Thus, better constraining this diurnal frost cycle and the properties of the \coo\ ice in these regions is of importance to better understand current active processes at the surface of Mars.
    In addition, as the observations of the seasonal growth and retreat of the polar caps over the past years have shown interannual variations \cite{piqueux_2015a}, it is of interest to monitor the evolution of both the North Polar Cap (NPC) and South Polar Cap (SPC).

    In this paper, we use for the first time data from the Emirates Mars Mission to map and monitor the evolution of the Martian polar caps along with the nighttime surface \coo\ frost at low latitudes.
    First, we describe in Section~\ref{sec:methods} the dataset and methods used in this study for the detection and mapping of the surface ices. Then, Section~\ref{sec:results} presents the seasonal and diurnal monitoring of the polar caps and the midlatitude nighttime \coo\ frost. Finally, Section~\ref{sec:ccl} summarizes the main points of the study.


\section{Dataset and methods}   \label{sec:methods}
    \subsection{Dataset}    \label{sec:data}
        The Emirates Mars InfraRed Spectrometer (EMIRS) instrument onboard the Emirates Mars Mission (EMM) "Hope" probe is a Fourier Transform Infrared spectrometer that is observing the Martian surface and atmosphere between 6 and 100~\mum\ with a selectable spectral resolution of 5~cm$^{-1}$ or 10~cm$^{-1}$ from February 2021 \cite{edwards_2021, amiri_2022}. The unique orbit of EMM allows EMIRS to observe the whole Martian surface across all local times in $\sim4$~orbits, which corresponds to $\sim5^\circ$ of $L_s$, or 10 Earth days.

        With a pixel size typically between 100 and 300~km, it is important to consider the spatial extent of each pixel to compute accurate maps, especially for the study of the polar regions where the emission angles are high.
        However, although polygons shapes of the pixels footprints as computed by the SPICE kernels \cite{acton_1996, acton_2018} are provided to users, their use to generate maps is not straightforward and significantly time-consuming.
        Thus, we have developed a new Python module called "SPiP" (\emph{Spacecraft Pixel footprint Projection}) that generates an approximation of the pixel footprints projected on a regular longitude/latitude map using 3D trigonometry and assuming (for now) a spherical planet \cite{stcherbinine_2023a}. 
        % The module's source code is freely available on GitHub at \url{https://github.com/NAU-PIXEL/spip}.

        In this study, we use the Martian surface temperature retrieved using a multiple-step algorithm applied on a large portion of the EMIRS spectra between 7.6 and 40~\mum\, excluding the strong CO$_2$ absorption band at 15~\mum\ \cite{smith_2022}.
        % For the CO$_2$ ice detections, we also require the surface pressure information, but as it cannot be reliably retrieved from EMIRS data alone, it is taken from the Mars Climate Database (MCD) \cite{forget_1999, millour_2018} and computed to match the observational parameters of each EMIRS observation \cite{smith_2022}.
        
        \subsubsection*{Data filtering}
        Then, in order to prevent our processing for instrumental bias and data artifacts, we only consider in our maps the data from pixels that prove the conditions below:
        \begin{itemize}
            \item The emission angle is lower than $80^\circ$.
            \item The retrieved surface temperature is between 140~K and 300~K.
            \item The entire field of view of the pixel is within the Martian disk.
        \end{itemize}

    \subsection{Diurnally stable ice maps processing}  \label{sec:daily_maps_method}
        As ice has higher thermal inertia than the regular Martian soil (typically $>2000$~J.K$^{-1}$.m$^{-2}$.s$^{-1/2}$ vs $\sim 200$~J.K$^{-1}$.m$^{-2}$.s$^{-1/2}$ \cite{putzig_2007}), we can detect the presence of surface ice that is stable across the day from the low amplitude of the surface temperature diurnal variations of these regions.
        Thus, we produce surface temperature maps from EMIRS retrievals at all Martian local times and compute the daily variations to convert them into maps of the presence of diurnally stable surface ice. 
        We converged to this temperature variations method instead of using the absolute surface temperatures in order to \hoo\ ice in our retrievals. Indeed, if the presence of \coo\ ice can confidently be derived from the absolute surface temperature (see section~\ref{sec:loct_maps_method}), it is more challenging for \hoo\ ice as we have to take into account the presence of available water, and some places can exhibit temperature compatible with water ice while being actually ice-free.
        One can note that this method does not allow us to retrieve any information on the composition of the ice. Thus, for these maps, the term "ice" can refer to either \hoo\ or \coo\ ice.

        A noteworthy point: in the following, the data are duplicated on both sides of the maps (i.e., in terms of longitude) when running the interpolation and smoothing to prevent edge effects and any dependence on the choice of the center longitude.

        First, we gather all the surface temperature values retrieved from EMIRS that prove the data filtering conditions described in section~\ref{sec:data} over 4 consecutive EMM orbits to have a full spatial coverage for all local times.
        Then, we compute surface temperature maps for bins of 1 hour of local time with a spatial resolution of $0.5^\circ \times 0.5^\circ$ by (1) projection of the pixel footprints, (2) linear interpolation of the data over the entire longitude/latitude grid with \texttt{scipy.interpolate.griddata} \cite{virtanen_2020}, and (3) gaussian smoothing with a standard deviation $\sigma=5$.
        We also compute for each map the gaussian kernel-density estimation (KDE) of our data for each map via \texttt{scipy.stats.gaussian\_kde} \cite{virtanen_2020}, and we flag as "low data density" all the pixels of the final maps associated with a KDE lower than $5\cdot10^{-6}$, this value has been tuned experimentally by comparing with the pixel footprints maps.

        Next, in order to increase the data coverage and prevent possible spurious temperature retrievals from some pixels, we bin the data into 3-hour maps. To do so, we crop each 1-hour map to keep only the pixels with a high data density and compute the median between the 3 consecutive maps. Then, we apply again a linear interpolation and a gaussian smoothing filter (with $\sigma=10$) to reconstruct the data over the whole longitude/latitude grid, and we flag as "low data density" the pixels that have been reconstructed by the interpolation (i.e., without any data from one of the 3 initial cropped 1-hour maps).
        This gives us a set of 8 maps of the surface temperature for local times ranging from 00:00 to 24:00 with the following binning: 00/03/06/09/12/15/18/21/24. 

        Then, we compute a map of the amplitude of diurnal temperature variations using only the pixels for which we have at least 7 data points (over 8) across the day. We exclude the maximum and minimum values for each pixel to prevent our results from data artifacts that may affect one map and compute the difference between the maximum and minimum temperature values from the remaining data $\Delta T$. This provides us a map of the amplitude of the temperature variation between daytime and nighttime. 
        % This $\Delta T$ map is then converted into a surface ice map from a calibration made from a comparison with EXI images acquired during the same period, where we can identify the latitudinal extent of the polar caps.
        This $\Delta T$ map is then converted into a map of the ice diurnally stable at the surface from a calibration made from a comparison with EXI images.
        To do so, we compared for 9 different values of $L_s$ spanning from $L_s=58^\circ$ to $L_s=290^\circ$ the latitudinal extent of the polar caps derived from several EXI images, with the corresponding $\Delta T$ maps computed with EMIRS.
        This ice map distinguishes 3 categories:% (see Figure~\ref{fig:dailyicemaps}): 
        \begin{itemize}
            \item "Ice" for pixels associated with $\Delta T \leq25~\mathrm{K}$
            \item "Maybe ice" for pixels associated with $25~\mathrm{K} < \Delta T \leq 35~\mathrm{K}$
            \item "Not ice" for pixels associated with $\Delta T > 35~\mathrm{K}$
        \end{itemize}

        These $\Delta T$ thresholds may seem relatively high compared to what will be expected for surfaces continuously covered by ice, but due to the large size of the EMIRS pixel footprints, especially at high latitudes, some of the pixels may include a mix of icy and non-icy regions. This will tend to increase the retrieved temperature when some portion of the pixel footprints 

        Finally, we flag as "low data density" the pixels of the ice map associated with less than 7 "high data density" pixels from the 3-hours binned surface temperature maps; and we flag the entire map as "low quality" if more than 50\% of its data points are flagged as "low data density".

        These maps are also included in the L3 products released by the EXI instrument as ice masks associated with the clouds' optical depth, as the retrievals cannot distinguish between atmospheric and surface ice \cite{wolff_2022}.

        % \begin{figure}%[h]
        %     \centering
        %     \includegraphics[width=\textwidth]{Figures/2dailyicemaps_orb80D-248D_v08-01_HR.png}
        %     \caption{Example of two maps of the daily stable surface ice obtained from EMIRS retrievals during Northern summer (left, orbits 80 -- 83) and winter (right, orbits 248 -- 251). The shaded regions represent the pixels with a low data density.}
        %     \label{fig:dailyicemaps}
        % \end{figure}

    \subsection{Local time CO\textsubscript{2} ice maps}    \label{sec:loct_maps_method}
        % Similarly to what is done for the daily ice maps (see section~\ref{sec:daily_maps_method}), we select all the EMIRS surface temperature values previously retrieved \cite{smith_2022} over 4 consecutive orbits, along with the MCD surface pressures for the corresponding pixels.
        % Then, following the methodology of \citeA{piqueux_2016}, we compute for each EMIRS pixel the \coo\ freezing temperature (\Tcoo) according to Clapeyron's law:
        Similarly to what is done for the diurnally stable ice maps (see section~\ref{sec:daily_maps_method}), we select all the EMIRS surface temperature values previously retrieved \cite{smith_2022} over 4 consecutive orbits, and compare them to the freezing temperature of \coo\ (\Tcoo) computed for each pixel according to Clapeyron's law \cite{piqueux_2016}:
        \begin{equation}    \label{eq:Tco2}
            \ln P = \alpha - \frac{\beta}{T\textsubscript{\coo}}
        \end{equation}
        with $\alpha=23.3494$, $\beta=3182.48$, and $P$ the \coo\ partial pressure taken as 96\% of the total surface pressure (in mbar in equation~(\ref{eq:Tco2})) \cite{piqueux_2016}.
        As P cannot be reliably retrieved from EMIRS data alone, it is taken from the Mars Climate Database (MCD) \cite{forget_1999, millour_2018} and computed to match the observational parameters of each EMIRS observation \cite{smith_2022}.
        In the following, as \coo\ is the principal component of the Martian atmosphere, the formation of \coo\ frost on the surface is not limited by the presence of gaseous \coo\ (unlike \hoo\ frost), so we will consider that \coo\ frost is present at the surface anywhere the retrieved surface temperature $T$ is lower than the predicted \Tcoo.

        % We project the results over the footprints of every pixel into slices of 3 hours of local time, which results in 8 maps containing 3 values (see Figure~\ref{fig:locticemaps}): either "No \coo\ ice", "\coo\ ice", or "Maybe \coo\ ice" if pixels footprints with and without \coo\ ice detections are overlapping. The spatial resolution of the maps is $0.5^\circ \times 0.5^\circ$.
        We project the results over the footprints of every pixel into slices of 3 hours of local time, centered on every hour of the day.
        This results in 24 maps containing 3 values:% (see Figure~\ref{fig:locticemaps}): %either "No \coo\ ice", "\coo\ ice", or "Maybe \coo\ ice": if pixels footprints with and without \coo\ ice detections are overlapping. 
        \begin{itemize}
            \item "\coo\ ice" if $T \leq \Tcoo$
            \item "No \coo\ ice" if $T > \Tcoo$
            \item "Maybe \coo\ ice" if there is an overlap between "\coo\ ice" and "No \coo\ ice" pixels footprints when sampling on the new longitude/latitude grid
        \end{itemize}
        
        The spatial resolution of the maps is $0.5^\circ \times 0.5^\circ$.
        Then, we use a nearest-pixel interpolation to reconstruct a whole map of the presence of \coo\ ice using \texttt{scipy.interpolate.griddata} \cite{virtanen_2020}.
        Finally, we compute the gaussian KDE of our data for each map via \texttt{scipy.stats.gaussian\_kde} \cite{virtanen_2020}, and we flag as "low data density" all the pixels of the final maps associated with a KDE lower than $5\cdot10^{-6}$, this value has been tuned experimentally by comparing with the pixel footprints maps.
        One may note that consecutive maps have a temporal overlap of 1 or 2 hours, but this allows us to have a smoother view to better see when the \coo\ ice starts to form at the surface, without being biased by the choice of the local time binning.

        % \begin{figure}%[h]
        %     \centering
        %     \includegraphics[width=\textwidth]{Figures/diurnal_maps_orb164D_v08-01_vertical_hideLowDensity_HR.png}
        %     \caption{Local time surface \coo\ ice maps for orbits 164 -- 167 ($L_s=162^\circ-167^\circ$, MY 36). Here the pixels with a low data density are hidden for each map. We observe that \coo\ frost is detected here only between local solar times of 03:00 and 06:00.}
        %     \label{fig:locticemaps}
        % \end{figure}

\section{Results and discussion}    \label{sec:results}
    \subsection{Seasonal variations}    \label{sec:seasonal_variations}

        \begin{figure}[h]
            \centering
            \includegraphics[width=\textwidth]{Figures/emirs_seasonal_surface_ice_med_v08-01_2023-02-08_qual1_all_contours_legend_vGCM_2.pdf}
            \caption{Seasonal variations of the latitudinal median extent of the Martian polar caps
                from EMM/EMIRS retrievals for diurnally stable surface ice. Only maps with a high-quality flag have been considered here.
                The red dotted line shows the limits of the polar caps derived from the Mars Planetary Climate Model (PCM) with a nominal dust scenario \cite{forget_1999, navarro_2014a}, the green and violet dashed lines show the limits of the NPC derived from MOC and OMEGA observations respectively acquired during MY~24-26 \cite{benson_2005} and MY~27-28 \cite{appere_2011}, and the yellow dotted lines show the limits of the SPC during its MY~27 recession derived from OMEGA observations \cite{schmidt_2010}.}
            \label{fig:seasonal_variations}
        \end{figure}

        Figure~\ref{fig:seasonal_variations} shows the seasonal evolution of the median latitudes of the North and South polar caps between $L_s=57^\circ$ (MY 36) and $L_s=11^\circ$ (MY 37) derived from EMIRS observations. This figure has been obtained by computing the median value of the diurnally stable ice maps (cf. section~\ref{sec:daily_maps_method}) over all longitudes for observations with a high-quality flag. 
        The temporal range encompasses the Northern Summer and the NPC recession/SPC progression along with part of the Northern Winter.
        In addition, we also include in Figure~\ref{fig:seasonal_variations} the edges of the polar caps as derived from other orbital instruments (OMEGA \& MOC) for previous Martian years \cite{benson_2005, schmidt_2010, appere_2011}, and predicted by the Mars PCM version 6 numeric model assessing a "nominal dust scenario" \cite{forget_1999, navarro_2014a, naar_2021, forget_2022} as a comparison with our MY~36 EMIRS retrievals.
        For the Mars PCM, the caps areas are defined as the regions where the surface \hoo\ ice layer at LST=12 averaged over all longitudes is greater than $10^{-3}$~kg.m$^{-2}$.

        \subsubsection*{North Polar Cap}
        We observe that the edge of the NPC remains stable between $70^\circ$N and $75^\circ$N between $L_s=58^\circ$ and $L_s=143^\circ$ (i.e., during the Northern Summer, when only the perennial NPC remains), then moves progressively equatorward to reach $\sim40^\circ$N at $L_s=250^\circ$ and remains there until $L_s=290^\circ$. Unlike most of the previous studies of the evolution of the NPC, we do not observe here the recession of the cap but its growth, which will provide a noticeable contribution to our overall understanding of the annual cycle of the NPC.

        Considering uncertainties of $\sim3^\circ$ in latitude due to the spatial extend of the EMIRS pixels at these latitudes our NPC retrievals match previous measurements by MOC and OMEGA at $L_s\sim57^\circ$ (MY 36) and $L_s\sim11^\circ$ (MY 37) \cite{benson_2005, appere_2011}.
        However, we can see that after the Norther Summer solstice ($L_s=90^\circ$) a larger discrepancy occurs between our EMIRS retrievals that identify the edge of the cap around $70^\circ$N -- $75^\circ$N and the previous MOC observations that report a limit around $80^\circ$N \cite{benson_2005}.
        By looking at images of the perennial NPC we can see that it barely reaches $80^\circ$N, but we also observe the presence of an additional region of perennial water ice between latitudes $74^\circ$N and $80^\circ$N for longitudes ranging from $95^\circ$E to $245^\circ$E \cite<e.g.,>{langevin_2005a, stcherbinine_2021b}.
        Thus, as this area represents $\sim40\%$ of the longitudes below the polar cap and the EMIRS pixels can span over a few tenths of degrees in longitude under these latitudes, it is likely that the cap boundary that is detected here includes these icy deposits, which explains the mismatch with the MOC data that only consider the center part of the NPC.

        % The comparison with the MCD predictions shows an asymmetry over the year: the MCD NPC boundary is $\sim10^\circ$ lower compared to previous OMEGA \& MOC observations during the NPC retreat phase ($L_s \sim 330^\circ - 80^\circ$) and compared to EMIRS retrievals for $L_s = 58^\circ - 74^\circ$, then it is $\sim5^\circ$ above the EMIRS boundary during the expanding phase from $L_s=190^\circ$ to $L_s=275^\circ$.
        % The MCD predicts a fast growth from $L_s=134^\circ$ to $L_s=170^\circ$ followed by a plateau until $L_s=195^\circ$ and a second expansion phase until $L_s=290^\circ$, while our results show a linear expansion from $L_s=140^\circ$ to $L_s=250^\circ$ follow by a plateau.
        % Otherwise, the model matches the EMIRS and OMEGA observations during the Northern Winter ($L_s\sim290^\circ$) and the EMIRS retrievals during Northern Summer, thus also including the Northern icy deposits outside the primary perennial polar cap with the longitudinal average.
        
        The comparison with the Mars PCM predictions shows an asymmetry over the year: the PCM NPC boundary matches previous OMEGA \& MOC observations during the NPC retreat phase ($L_s \sim 330^\circ - 80^\circ$) and EMIRS retrievals for $L_s = 330^\circ - 11^\circ$ and $L_s = 58^\circ - 74^\circ$, then it is $\sim5^\circ$ to $10^\circ$ above the EMIRS boundary during the expanding phase from $L_s=185^\circ$ to $L_s=270^\circ$.
        The PCM predicts a linear growth from $L_s=134^\circ$ to $L_s=170^\circ$ followed by a plateau until $L_s=195^\circ$ and a second expansion phase until $L_s=290^\circ$, while our results show a linear expansion from $L_s=140^\circ$ to $L_s=250^\circ$ follow by a plateau. The presence of this plateau between $L_s=170^\circ$ and $L_s=190^\circ$ in the model which is not observed in the EMIRS retrievals leads to the smaller extent of the NPC predicted by the PCM compared to our retrievals during the second half of the expansion phase.
        One may also consider that the PCM has been run here for a "nominal dust scenario" \cite{forget_1999, forget_2022, millour_2022, montabone_2015}, i.e., not including the specificities of MY 36. Previous studies \cite<e.g.,>{calvin_2015, piqueux_2015a} have shown the presence of interannual variations of a few degrees in latitude (typically up to 3-4$^\circ$) in the extent of the seasonal polar cap deposits for the same values of $L_s$. Thus, considering the uncertainties of $\sim3^\circ$ in latitude in our retrievals due to the spatial extent of EMIRS pixels footprints, the discrepancies observed between our retrievals and the PCM may be reflecting interannual variations of the evolution of the NPC, with a faster expanding phase in MY 36 compared to the nominal scenario.
        Otherwise, the model matches the EMIRS and OMEGA observations during the Northern Winter ($L_s\sim290^\circ$) and the EMIRS retrievals during Northern Summer ($L_s\sim90^\circ$), thus also including the Northern icy deposits outside the primary perennial polar cap with the longitudinal average.

        \subsubsection*{South Polar Cap}
        The Mars SPC is highly asymmetric and can be divided into two regions: the "cryptic" and the "anti-cryptic" \cite{kieffer_2000, schmidt_2010}. They do not have the same sublimation rate during the polar cap recession and do not extend to the same latitudes. In particular, during the Southern Summer, the perennial SPC is only present in the "anti-cryptic" region \cite{langevin_2007, schmidt_2010}. Thus, considering the methodology used here to map the seasonal evolution of the polar caps with EMIRS, we expect our EMIRS latitudinal boundary for the SPC to be located between the "cryptic" and "anti-cryptic" ones derived from OMEGA observations \cite{schmidt_2010}.

        Indeed, we observe a good agreement between the EMIRS results and the OMEGA observations during the Southern Winter ($L_s\sim 95^\circ - 130^\circ$) and the second half of the SPC recession ($L_s \sim 200^\circ - 295^\circ$).
        Between $L_s=133^\circ$ and $L_s=190$ the EMIRS SPC boundary is detected up to $6^\circ$ northern than the external OMEGA edge ("outer anti-cryptic"). However, the boundaries derived by OMEGA are here "crocus" lines \cite{schmidt_2010}, i.e, the limits of the \coo\ deposits, while our methods capture equally all the ices that may be present on the surface (\coo\ or \hoo).
        Figure~12 from \citeA{langevin_2007} shows that between $L_s=130^\circ$ and $L_s=155^\circ$ the seasonal \hoo\ deposits extend outside the \coo\ ones, with detections up to $\sim 45^\circ$S which is coherent with our observations.

        Regarding the limits predicted by the Mars PCM, we observe that it matches our detections from $L_s=58^\circ$ to $L_s=92^\circ$ and for $L_s \sim 150^\circ - 225^\circ$. 
        Then, the PCM line is at slightly more equatorial latitudes (up to 5 degrees) compared to our detections but matches the "outer anti-cryptic" limit derived by OMEGA \cite{schmidt_2010}.
        % Aside from these periods, the MCD line is constantly at more equatorial latitudes by 5 to 10 degrees. 
        % The recession of the SPC starts earlier in the MCD than in our observations ($L_s\sim160^\circ$ vs $L_s\sim170^\circ$) but occurs at a slower rate than what is observed by EMIRS, which leads to an increase of the gap between both lines from $L_s=170^\circ$ to $L_s=260^\circ$.

        % \td{Attention: Crocus lines pour \cite{schmidt_2010} $\rightarrow$ \coo\ vs \hoo\ peut expliquer différences ? $\rightarrow$ oui cf \cite{langevin_2007} fig 12, \hoo\ s'étend un peu plus loin que \coo\ et valeurs cohérentes}

        % \td{
        % \begin{itemize}
        %     % \item Seasonal variations map (lat vs $L_s$)
        %     % \item comparison with MCD
        %     \item \cite{langevin_2007} South polar cap, \cite{appere_2011} North polar cap OMEGA
        %     \item \cite{brown_2010} South polar cap, \cite{brown_2012} North polar cap CRISM
        %     \item Interannual variations of the latitudinal extent of the polar caps about a few degrees \cite{benson_2005, appere_2011, calvin_2015, calvin_2017, acharya_2022}
        %     \item South $\rightarrow$ strong asymmetry of the polar cap, may explain some discrepancies observed with the MCD
        %     \item + large footprints of EMIRS pixels
        %     % \item NPC vers $L_s\sim90^\circ$, différence avec MOC: perennial polar cap vers $\sim80^\circ$ (MOC) mais ice deposits jusqu'à $\sim75^\circ$ entre lon 90 et 260 (cf cartes OMEGA \cite<e.g.,>{langevin_2005a, appere_2011, stcherbinine_2021b})
        % \end{itemize}
        % }

    \subsection{Diurnal variations}     \label{sec:diurnal_variations}
        \begin{figure}
            \centering
            \includegraphics[width=\textwidth]{Figures/emirs_co2_diurnal_ice_detections_orb0156-0195_density0-30_v08-01_cmoice1_HR.png}
            \caption{Density of detections of surface \coo\ ice from $L_s=152^\circ$ to $L_s=203^\circ$ (EMM orbits 156 to 195) for bins of 3 hours of local solar time.
            Regions without data coverage are in white.
            We observe that \coo\ frost is detected here mostly between local solar times of 03:00 and 06:00.}
            \label{fig:diurnal_density_co2frost}
        \end{figure}

        \begin{figure}
            \centering
            \includegraphics[width=\textwidth]{Figures/emirs_co2_diurnal_ice_detections_multiLs_lt3-6_density0-50_v08-01_cmoice1_HR.png}
            \caption{Density of detections of surface \coo\ ice between local solar times 03:00 and 06:00 for bins of $\sim 50^\circ$ of $L_s$ over MY 36.
            Regions without data coverage are in white.}
            \label{fig:seasonal_density_co2frost}
        \end{figure}

        From the surface \coo\ ice maps generated for bins of 3 hours of LST on a 4-orbits basis as described in section~\ref{sec:loct_maps_method}, we compute for each bin of local time a map of the percentage of \coo\ frost detections over wider ranges of $L_s$, considering only the pixels flagged as "high data density".

        Figure~\ref{fig:diurnal_density_co2frost} shows the \coo\ surface ice detections for each local time between $L_s=152^\circ$ and $L_s=203^\circ$. We can see that apart from the polar caps, \coo\ ice is detected at night under equatorial and mid-latitude regions mostly around $100^\circ$W (Tharsis region), and also around $40^\circ$E (Arabia Terra), which corresponds to low thermal inertia areas \cite{putzig_2007} where nighttime surface \coo\ frost have been detected through MCS or THEMIS measurements \cite{piqueux_2016, khuller_2021a, lange_2022}.
        Regarding the daily evolution of these \coo\ frost deposits, we observe that it starts to appear at midnight (panel a) and last after 6~am to disappear during daytime, with a maximum of intensity between LST of 03:00 and 06:00 (panel b). 
        Condensation of the \coo\ occurs during the second half of the Martian night until sunrise.
        % The 3~am MCS observations of \coo\ frost \cite{piqueux_2016} thus correspond to the optimal time for the study of the \coo\ surface frost.

        Figure~\ref{fig:seasonal_density_co2frost} shows the density of detections of surface \coo\ ice between 3~am and 6~am by EMIRS from $L_s=6^\circ$ to $L_s=290^\circ$ for 6 temporal bins of $\sim 50^\circ$ of $L_s$.
        We can see that our detections of non-polar surface \coo\ frost vary with the $L_s$: while they remain located in the same two regions mentioned above, the density of detections along with the size of the area covered by the frost evolve.
        Indeed, equatorial and mid-latitudes \coo\ frost is mostly detected for $6^\circ \leq L_s \leq 19^\circ$ and $120^\circ \leq L_s \leq 203^\circ$ (panels a, c \& d).
        Panel b ($49^\circ \leq L_s \leq 100^\circ$) corresponds to the end of Spring and early Summer in the Northern hemisphere, i.e., the period of the year with the highest temperatures in the Northern hemisphere. Thus, as the regions where nighttime \coo\ frost is usually detected under midlatitudes are located in the Northern hemisphere, there are almost no detections outside the South polar cap during this period. Even the retrieved nighttime temperatures are not low enough to allow the formation of \coo\ ice at the surface of the planet in these regions.
        This formation of \coo\ frost late during the night with a quick sublimation at sunrise was expected from the models \cite<e.g.,>{lange_2022} with implications on the formation of gullies and slope streaks \cite{pilorget_2016, khuller_2021a, lange_2022}. But as previous nighttime observations were conducted only around 03:00, this is the first time we are able to confirm it with direct observations of the diurnal cycle of the surface \coo\ frost at low latitudes.

        Similarly, we also report almost no midlatitudes \coo\ frost detections between $L_s=203^\circ$ and $L_s=290^\circ$ (panels e \& f). This period corresponds to the Southern spring and summer, but also to the part of the Martian Year where the planet is closer to the Sun as the perihelion is reached at $L_s=251^\circ$. Thus, because of the eccentricity of the Martian orbit, the global temperature at the surface of the planet is significantly higher during the second half of the year \cite<e.g.,>{smith_2006a,bell_2008}, which explains the fewer nighttime \coo\ frost detections at equatorial and midlatitudes.

        This seasonal trend was previously observed by MCS and THEMIS \cite{piqueux_2016, khuller_2021a}. However, one may note the fewer frost detections in our EMIRS results, which may be related to the difference in terms of spatial resolution between the instruments.
        Indeed, the spatial resolution of the EMIRS pixels is comprised between 100 and 300~km \cite{edwards_2021}, which is much larger than the few km of the MCS footprints \cite{mccleese_2007} or the 100~m of THEMIS \cite{christensen_2004}.
        Plus, the detections made by both MCS and THEMIS reveal that the \coo\ frost detections may 
        be localized to areas that are sub-pixel at the EMIRS resolution \cite{piqueux_2016, khuller_2021a}. Thus, if an EMIRS pixel is only partially covered by surface \coo\ frost, the retrieved temperature, averaged over the entire footprint, will be higher than the \coo\ freezing temperature as a portion of the footprint will be unfrozen.
        This results in a lower sensibility in the surface \coo\ frost retrieving process, but also strengthen our detections as they testify of the presence of \coo\ ice over all our pixel footprints.

        % \td{
        % \begin{itemize}
        %     % \item Diurnal variations: 1 fig with all local times + 1 figure with 1 local time at multiple $L_s$
        %     \item Sud, ellipse proche du Soleil + chaud
        %     \item Less detections than \citeA{piqueux_2016} or \citeA{khuller_2021a} for instance
        %         $\rightarrow$ large EMIRS footprints ? Spatial mix
        %     \item MCS \cite{mccleese_2007} THEMIS \cite{christensen_2004}
        % \end{itemize}
        % }
        
    % \subsection{Spectral comparisons}   \label{sec:spectra}

    %     \td{
    %     \begin{itemize}
    %         \item Spectra comparison (1 avg polar cap \emph{vs} 1 avg frost)
    %         \item Particle size?
    %         \item Large size of EMIRS pixels
    %         \item Dust emissivity $\sim 1$ can mask the \coo\ signatures
    %         \item Aerosols emissivity?
    %         \item \coo\ ice as cold trap for \hoo ?
    %     \end{itemize}
    %     }

\section{Conclusion}    \label{sec:ccl}
    % \td{
    % \begin{itemize}
    %     \item Automatic monitoring of the seasonal variations of the polar caps by EMIRS (continuously over the duration of EMM)
    %     \item Simultaneous observation of both the North and South caps
    % \end{itemize}
    % }

    In this paper, we present the results of our study on the monitoring of the Martian surface ices with the EMIRS instrument onboard EMM. From the variation of the surface temperature over the day, we developed a method to automatically map the presence of diurnally stable ice at the surface, which allows us to monitor the seasonal variations of the two polar caps. Then, based on the method previously developed for MCS data \cite{piqueux_2016}, we use the surface temperature to detect and map the presence of \coo\ frost at the surface of the planet, and monitor for the first time its evolution as a function of the local time thanks to the unique orbit of the EMM probe.

    We monitor the evolution of the seasonal polar caps from $L_s=57^\circ$ (MY 36) to $L_s=11^\circ$ (MY 37), with a temporal resolution of $5^\circ$ of $L_s$ (10 Earth days).
    Plus, the large-scale view of EMIRS and the automatization of our method allow us to continuously and simultaneously monitor the annual variations of both polar caps.
    % Comparisons with previous observations of the growth and recession of the polar caps by OMEGA and MOC instruments show a good agreement. 

    Also, we are able to observe for the first time the apparition and disappearance of low-latitude nighttime \coo\ frost at the surface of the planet, thanks to the unique ability of EMM instruments to provide full coverage in terms of local time. \coo\ ice is detected at the surface down to the equator around spring and fall equinoxes in the second half of the night (essentially between 3~am and 6~am) with a quick sublimation at sunrise, which confirms previous model expectations.

    % To conclude, [...]

\section*{Data availability Statement}
The SPiP module is freely available on GitHub at \url{https://github.com/NAU-PIXEL/spip} \cite{stcherbinine_2023a}.

Data from the Emirates Mars Mission (EMM) are freely and publicly available on the EMM Science Data Center (SDC, \url{http://sdc.emiratesmarsmission.ae}).
This location is designated as the primary repository for all data products produced by the EMM team and is designated as long-term repository as required by the UAE Space Agency.
The data available (\url{http://sdc.emiratesmarsmission.ae/data}) include ancillary spacecraft data, instrument telemetry, Level 1 (raw instrument data) to Level 3 (derived science products), quicklook products, and data users guides (\url{https://sdc.emiratesmarsmission.ae/documentation}) to assist in the analysis of the data.
Following the creation of a free login, all EMM data are searchable via parameters such as product file name, solar longitude, acquisition time, sub-spacecraft latitude \& longitude, instrument, data product level, etc.

Data products can be browsed within the SDC via a standardized file system structure that follows the convention:
\texttt{/emm/data/\textless Instrument\textgreater /\textless DataLevel\textgreater /\textless Mode\textgreater /\textless Year\textgreater /\textless Month\textgreater }

Data product filenames follow a standard convention:\linebreak
\texttt{emm\_\textless Instrument\textgreater \_\textless DataLevel\textgreater \textless StartTimeUTC\textgreater \_\textless OrbitNumber\textgreater \_\textless Mode\textgreater \_\textless Description\textgreater\linebreak
\_\textless Kernel-Level\textgreater \_\textless Version\textgreater .\textless FileType\textgreater }

EMIRS data and users guides are available at: \url{https://sdc.emiratesmarsmission.ae/data/emirs}

% The Mars PCM v6 is available here: \url{http://svn.lmd.jussieu.fr/Planeto/trunk/LMDZ.MARS}
The Mars PCM v6 and the MCD are available from \url{http://www-mars.lmd.jussieu.fr}

\acknowledgments
% We thank Lucas Lange (LMD) for providing the Mars PCM simulations for the surface ice for MY 36.
The authors want to thank Sylvain Piqueux (JPL) for his helpful discussion regarding the \coo\ ice retrievals.

\bibliography{biblio_stcherbinine2023_emirs_ice}

\end{document}


%% ------------------------------------------------------------------------ %%
%% ------------------------------------------------------------------------ %%

%%% Suggested section heads:
% \section{Introduction}
%
% The main text should start with an introduction. Except for short
% manuscripts (such as comments and replies), the text should be divided
% into sections, each with its own heading.

% Headings should be sentence fragments and do not begin with a
% lowercase letter or number. Examples of good headings are:

% \section{Materials and Methods}
% Here is text on Materials and Methods.
%
% \subsection{A descriptive heading about methods}
% More about Methods.
%
% \section{Data} (Or section title might be a descriptive heading about data)
%
% \section{Results} (Or section title might be a descriptive heading about the
% results)
%
% \section{Conclusions}


\section{= enter section title =}
%Text here ===>>>


%%

%  Numbered lines in equations:
%  To add line numbers to lines in equations,
%  \begin{linenomath*}
%  \begin{equation}
%  \end{equation}
%  \end{linenomath*}



%% Enter Figures and Tables near as possible to where they are first mentioned:
%
% DO NOT USE \psfrag or \subfigure commands.
%
% Figure captions go below the figure.
% Table titles go above tables;  other caption information
%  should be placed in last line of the table, using
% \multicolumn2l{$^a$ This is a table note.}
%
%----------------
% EXAMPLE FIGURES
%
% \begin{figure}
% \includegraphics{example.png}
% \caption{caption}
% \end{figure}
%
% Giving latex a width will help it to scale the figure properly. A simple trick is to use \textwidth. Try this if large figures run off the side of the page.
% \begin{figure}
% \noindent\includegraphics[width=\textwidth]{anothersample.png}
%\caption{caption}
%\label{pngfiguresample}
%\end{figure}
%
%
% If you get an error about an unknown bounding box, try specifying the width and height of the figure with the natwidth and natheight options. This is common when trying to add a PDF figure without pdflatex.
% \begin{figure}
% \noindent\includegraphics[natwidth=800px,natheight=600px]{samplefigure.pdf}
%\caption{caption}
%\label{pdffiguresample}
%\end{figure}
%
%
% PDFLatex does not seem to be able to process EPS figures. You may want to try the epstopdf package.
%

%
% ---------------
% EXAMPLE TABLE
% Please do NOT include vertical lines in tables
% if the paper is accepted, Wiley will replace vertical lines with white space
% the CLS file modifies table padding and vertical lines may not display well
%
 \begin{table}
 \caption{Time of the Transition Between Phase 1 and Phase 2$^{a}$}
 \centering
 \begin{tabular}{l c}
 \hline
  Run  & Time (min)  \\
 \hline
   $l1$  & 260   \\
   $l2$  & 300   \\
   $l3$  & 340   \\
   $h1$  & 270   \\
   $h2$  & 250   \\
   $h3$  & 380   \\
   $r1$  & 370   \\
   $r2$  & 390   \\
 \hline
 \multicolumn{2}{l}{$^{a}$Footnote text here.}
 \end{tabular}
 \end{table}

%% SIDEWAYS FIGURE and TABLE
% AGU prefers the use of {sidewaystable} over {landscapetable} as it causes fewer problems.
%
% \begin{sidewaysfigure}
% \includegraphics[width=20pc]{figsamp}
% \caption{caption here}
% \label{newfig}
% \end{sidewaysfigure}
%
%  \begin{sidewaystable}
%  \caption{Caption here}
% \label{tab:signif_gap_clos}
%  \begin{tabular}{ccc}
% one&two&three\\
% four&five&six
%  \end{tabular}
%  \end{sidewaystable}

%% If using numbered lines, please surround equations with \begin{linenomath*}...\end{linenomath*}
%\begin{linenomath*}
%\begin{equation}
%y|{f} \sim g(m, \sigma),
%\end{equation}
%\end{linenomath*}

%%% End of body of article

%%%%%%%%%%%%%%%%%%%%%%%%%%%%%%%%
%% Optional Appendix goes here
%
% The \appendix command resets counters and redefines section heads
%
% After typing \appendix
%
%\section{Here Is Appendix Title}
% will show
% A: Here Is Appendix Title
%
%\appendix
%\section{Here is a sample appendix}

%%%%%%%%%%%%%%%%%%%%%%%%%%%%%%%%%%%%%%%%%%%%%%%%%%%%%%%%%%%%%%%%
%
% Optional Glossary, Notation or Acronym section goes here:
%
%%%%%%%%%%%%%%
% Glossary is only allowed in Reviews of Geophysics
%  \begin{glossary}
%  \term{Term}
%   Term Definition here
%  \term{Term}
%   Term Definition here
%  \term{Term}
%   Term Definition here
%  \end{glossary}

%
%%%%%%%%%%%%%%
% Acronyms
%   \begin{acronyms}
%   \acro{Acronym}
%   Definition here
%   \acro{EMOS}
%   Ensemble model output statistics
%   \acro{ECMWF}
%   Centre for Medium-Range Weather Forecasts
%   \end{acronyms}

%
%%%%%%%%%%%%%%
% Notation
%   \begin{notation}
%   \notation{$a+b$} Notation Definition here
%   \notation{$e=mc^2$}
%   Equation in German-born physicist Albert Einstein's theory of special
%  relativity that showed that the increased relativistic mass ($m$) of a
%  body comes from the energy of motion of the body—that is, its kinetic
%  energy ($E$)—divided by the speed of light squared ($c^2$).
%   \end{notation}



\section{Open Research}
AGU requires an Availability Statement for the underlying data needed to understand, evaluate, and build upon the reported research at the time of peer review and publication.

Authors should include an Availability Statement for the software that has a significant impact on the research. Details and templates are in the Availability Statement section of the Data and Software for Authors Guidance: \url{https://www.agu.org/Publish-with-AGU/Publish/Author-Resources/Data-and-Software-for-Authors#availability}

It is important to cite individual datasets in this section and, and they must be included in your bibliography. Please use the type field in your bibtex file to specify the type of data cited. Some options include Dataset, Software, Collection, ComputationalNotebook. Ex: 
\\
\begin{verbatim}

@misc{https://doi.org/10.7283/633e-1497,
  doi = {10.7283/633E-1497},
  url = {https://www.unavco.org/data/doi/10.7283/633E-1497},
  author = {de Zeeuw-van Dalfsen, Elske and Sleeman, Reinoud},
  title = {KNMI Dutch Antilles GPS Network - SAB1-St_Johns_Saba_NA P.S.},
  publisher = {UNAVCO, Inc.},
  year = {2019},
  type = {dataset}
}

\end{verbatim}

For physical samples, use the IGSN persistent identifier, see the International Geo Sample Numbers section:
\url{https://www.agu.org/Publish-with-AGU/Publish/Author-Resources/Data-and-Software-for-Authors#IGSN}
%%%%%%%%%%%%%%%%%%%%%%%%%%%%%%%%%%%%%%%%%%%%%%%

\acknowledgments
This section is optional. Include any Acknowledgments here.
The acknowledgments should list:\\
All funding sources related to this work from all authors\\
Any real or perceived financial conflicts of interests for any author\\
Other affiliations for any author that may be perceived as having a conflict of interest with respect to the results of this paper.\\
It is also the appropriate place to thank colleagues and other contributors. AGU does not normally allow dedications.


%% ------------------------------------------------------------------------ %%
%% References and Citations

%%%%%%%%%%%%%%%%%%%%%%%%%%%%%%%%%%%%%%%%%%%%%%%
%
% \bibliography{<name of your .bib file>} don't specify the file extension
%
% don't specify bibliographystyle

% In the References section, cite the data/software described in the Availability Statement (this includes primary and processed data used for your research). For details on data/software citation as well as examples, see the Data & Software Citation section of the Data & Software for Authors guidance
% https://www.agu.org/Publish-with-AGU/Publish/Author-Resources/Data-and-Software-for-Authors#citation

%%%%%%%%%%%%%%%%%%%%%%%%%%%%%%%%%%%%%%%%%%%%%%%

%\bibliography{enter your bibtex bibliography filename here}



%Reference citation instructions and examples:
%
% Please use ONLY \cite and \citeA for reference citations.
% \cite for parenthetical references
% ...as shown in recent studies (Simpson et al., 2019)
% \citeA for in-text citations
% ...Simpson et al. (2019) have shown...
%
%
%...as shown by \citeA{jskilby}.
%...as shown by \citeA{lewin76}, \citeA{carson86}, \citeA{bartoldy02}, and \citeA{rinaldi03}.
%...has been shown \cite{jskilbye}.
%...has been shown \cite{lewin76,carson86,bartoldy02,rinaldi03}.
%... \cite <i.e.>[]{lewin76,carson86,bartoldy02,rinaldi03}.
%...has been shown by \cite <e.g.,>[and others]{lewin76}.
%
% apacite uses < > for prenotes and [ ] for postnotes
% DO NOT use other cite commands (e.g., \citet, \citep, \citeyear, \citealp, etc.).
% \nocite is okay to use to add references from your Supporting Information
%



\end{document}



More Information and Advice:

%% ------------------------------------------------------------------------ %%
%
%  SECTION HEADS
%
%% ------------------------------------------------------------------------ %%

% Capitalize the first letter of each word (except for
% prepositions, conjunctions, and articles that are
% three or fewer letters).

% AGU follows standard outline style; therefore, there cannot be a section 1 without
% a section 2, or a section 2.3.1 without a section 2.3.2.
% Please make sure your section numbers are balanced.
% ---------------
% Level 1 head
%
% Use the \section{} command to identify level 1 heads;
% type the appropriate head wording between the curly
% brackets, as shown below.
%
%An example:
%\section{Level 1 Head: Introduction}
%
% ---------------
% Level 2 head
%
% Use the \subsection{} command to identify level 2 heads.
%An example:
%\subsection{Level 2 Head}
%
% ---------------
% Level 3 head
%
% Use the \subsubsection{} command to identify level 3 heads
%An example:
%\subsubsection{Level 3 Head}
%
%---------------
% Level 4 head
%
% Use the \subsubsubsection{} command to identify level 3 heads
% An example:
%\subsubsubsection{Level 4 Head} An example.
%
%% ------------------------------------------------------------------------ %%
%
%  IN-TEXT LISTS
%
%% ------------------------------------------------------------------------ %%
%
% Do not use bulleted lists; enumerated lists are okay.
% \begin{enumerate}
% \item
% \item
% \item
% \end{enumerate}
%
%% ------------------------------------------------------------------------ %%
%
%  EQUATIONS
%
%% ------------------------------------------------------------------------ %%

% Single-line equations are centered.
% Equation arrays will appear left-aligned.

Math coded inside display math mode \[ ...\]
 will not be numbered, e.g.,:
 \[ x^2=y^2 + z^2\]

 Math coded inside \begin{equation} and \end{equation} will
 be automatically numbered, e.g.,:
 \begin{equation}
 x^2=y^2 + z^2
 \end{equation}


% To create multiline equations, use the
% \begin{eqnarray} and \end{eqnarray} environment
% as demonstrated below.
\begin{eqnarray}
  x_{1} & = & (x - x_{0}) \cos \Theta \nonumber \\
        && + (y - y_{0}) \sin \Theta  \nonumber \\
  y_{1} & = & -(x - x_{0}) \sin \Theta \nonumber \\
        && + (y - y_{0}) \cos \Theta.
\end{eqnarray}

%If you don't want an equation number, use the star form:
%\begin{eqnarray*}...\end{eqnarray*}

% Break each line at a sign of operation
% (+, -, etc.) if possible, with the sign of operation
% on the new line.

% Indent second and subsequent lines to align with
% the first character following the equal sign on the
% first line.

% Use an \hspace{} command to insert horizontal space
% into your equation if necessary. Place an appropriate
% unit of measure between the curly braces, e.g.
% \hspace{1in}; you may have to experiment to achieve
% the correct amount of space.


%% ------------------------------------------------------------------------ %%
%
%  EQUATION NUMBERING: COUNTER
%
%% ------------------------------------------------------------------------ %%

% You may change equation numbering by resetting
% the equation counter or by explicitly numbering
% an equation.

% To explicitly number an equation, type \eqnum{}
% (with the desired number between the brackets)
% after the \begin{equation} or \begin{eqnarray}
% command.  The \eqnum{} command will affect only
% the equation it appears with; LaTeX will number
% any equations appearing later in the manuscript
% according to the equation counter.
%

% If you have a multiline equation that needs only
% one equation number, use a \nonumber command in
% front of the double backslashes (\\) as shown in
% the multiline equation above.

% If you are using line numbers, remember to surround
% equations with \begin{linenomath*}...\end{linenomath*}

%  To add line numbers to lines in equations:
%  \begin{linenomath*}
%  \begin{equation}
%  \end{equation}
%  \end{linenomath*}



