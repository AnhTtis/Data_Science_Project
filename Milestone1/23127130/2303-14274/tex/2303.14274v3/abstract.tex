\begin{abstract}
This dissertation aims to provide a comprehensive account of set theory with urelements. In Chapter 1, I present mathematical and philosophical motivations for studying urelement set theory and lay out the necessary technical preliminaries. Chapter 2 is devoted to the axiomatization of urelement set theory, where I introduce a hierarchy of axioms and discuss how ZFC with urelements should be axiomatized. The breakdown of this hierarchy of axioms in the absence of the Axiom of Choice is also explored. In Chapter 3, I investigate forcing with urelements and develop a new approach that addresses a drawback of the existing machinery. I demonstrate that forcing can preserve, destroy, and recover the axioms isolated in Chapter 2 and discuss how Boolean ultrapowers can be applied in urelement set theory. Chapter 4 delves into class theory with urelements. I first discuss the issue of axiomatizing urelement class theory and then explore the second-order reflection principle with urelements. In particular, assuming large cardinals, I construct a model of second-order reflection where the principle of limitation of size fails.  %本篇论文旨在全面分析带有“Urelements”的集合论。第1章介绍了研究“Urelement”集合论的数学和哲学动机,并阐述了必要的技术前提。第2章致力于“Urelement”集合论的公理化,其中我考虑了一类公理体系,并讨论了如何给带有“Urelements”的ZFC公理化。此外,我还探讨了在缺乏选择公理的情况下这种公理体系的分解。在第3章中,我研究了带有“Urelements”的力迫法,通过提出一种新的力迫法解决了现有机制的缺陷。我证明了强制法可以保留、破坏和恢复第2章中分离出的公理,并讨论了如何在带有“Urelements”的集合论中应用布尔超幂。第4章深入探讨了带有“Urelements”的类论。我首先讨论了公理化“Urelement”类论的问题,然后探讨了带有“Urelements”的二阶反射原理。特别是,在假设具有大基数的情况下,我构建了一个满足二阶反射的模型,其中限制原理失效。
\end{abstract}