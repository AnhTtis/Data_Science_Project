

\chapter{Introduction}
Section \ref{section:UrelementsinSetTheory} presents mathematical and philosophical motivations for studying set theory with urelements. Section \ref{section:basicaxioms} introduces the basic axioms of set theory with urelements. Section \ref{section:interpretingUinV} reviews a well-known method of interpreting urelements in pure set theory, through which various versions of urelement set theory can be interpreted in pure set theory.


\section{Urelements in set theory}\label{section:UrelementsinSetTheory}
 Urelements are members of sets that are not themselves sets. While urelements, as non-reducible mathematical objects, were included in the earlier development of set theory (e.g., in Zermelo's original presentation \cite{zermelo1930grenzzahlen}), most contemporary set theorists decided that their role is superfluous. This is because within a reasonable pure set theory such as ZFC, all mathematical objects can be recovered \textit{up to isomorphism}. However, set theory with urelements are still of mathematical and philosophical interests.

\subsection{The mathematics}
In the pre-forcing era, permutation models were used to establish the independence of the Axiom of Choice in the Zermelo-Fraenkel set theory (ZF) with urelements. This technique was developed by Fraenkel, Mostowski, and Specker \cite{fraenkel1922begriff,Mostowski1939-MOSBDB,Specker1957-SPEZAD}. With the invention of forcing, permutation models became a flexible method of obtaining independence results concerning choice principles, thanks to the Jech-Sochor Embedding Theorem. Urelements play an essential role in alternative set theories such as Kripke–Platek set theory \cite{barwise2017admissible}, Quine's New Foundations \cite{jensen1968consistency}, versions of constructive set theory \cite{Cantini2008-CROCST-4}, and non-well founded set theory \cite{barwise1996vicious}.

However, many questions regarding urelement set theory remain unexplored. Most existing studies of ZF with urelements, such as \cite{jech2008axiom} and \cite{potter2004set}, assume as an axiom that the urelements form a set.\footnote{In \cite{barwise2017admissible}, Barwise studies Kripke–Platek set theory that allows a proper class of urelements.} This assumption is highly unnatural and raises several issues. Firstly, whether the urelements form a set should not be settled by an axiom of set theory. According to many philosophical arguments, certain abstract entities, such as propositions and possible worlds, cannot form a set by their nature.\footnote{For arguments for no set of all propositions, see \cite{Grim1984-GRITIN-5} and \cite{Menzel2012-MENSAW}. The principle that for every cardinal $\kappa$, there is a set of urelements of size $\kappa$ also appears in the discussion of \textit{recombination principles} in modal metaphysics. See \cite{Forrest1984-FORAAA} and \cite{nolan1996recombination}.} Secondly, permitting only a set of urelements conceals a great deal of subtleties of urelement set theory, while allowing a proper class of urelements is needed to understand many set-theoretic axioms and constructions fully. Surprisingly, there does not seem to be a systematic study of ZF(C) with a class of urelements, and many fundamental questions are thus left unanswered. This dissertation aims to address some of these questions by focusing on the following three:
\begin{enumerate}
\item What is the most general formalization of set theory (and class theory) with urelements?
\item How do standard set-theoretic constructions, such as forcing, behave in the context of urelements?
\item Can the existence of urelements affect the strength of strong axioms of infinity?
\end{enumerate}


\subsection{The philosophy}\label{subsection:philosophy}
One philosophical motivation for studying urelement set theory comes from the potential applications of set theory to domains containing non-sets. For example, since the mass-function maps the physical objects to the real numbers, they can only be reduced to sets if set theory allows urelements. Similarly, philosophers often give arguments involving \textit{sets} of propositions, mereological fusions, and possible worlds. Set theory with urelements thus provides a foundation for studying these debates.

Urelements can also have implications on the philosophical conceptions of set. The Naive Comprehension Principle, which says that for every condition $\varphi$ there is a set of things satisfying $\varphi$, seems to be the most natural conception of set. By Russell's paradox, we know that this naive conception is simply contradictory because the Russell set $\{x: x \notin x\}$ cannot exist. If the commonly accepted axioms of ZF can be seen as a response to Russell's paradox, they must be justified on the basis of different conceptions. Three conceptions of set are often discussed in the literature (\cite{maddy1988believing}): \textit{the iterative conception}, \textit{limitation of size}, and \textit{reflection}. 

According to the iterative conception of set, sets are formed in \textit{stages}: at each stage sets of things available on the earlier stages are formed. The Russell set simply cannot exist in this picture because it cannot be formed at any stage. Boolos \cite{boolos1971iterative} shows that a faithful formulation of the iterative conception recovers almost all the axioms of Zermelo set theory. A more general version of the iterative conception also asserts that there are as many stages as possible (see \cite{Martin2001-MARMUO}, which can be adopted to justify the Axiom of Replacement. 

According to the limitation-of-size conception (\cite{hallett1986cantorian}), a collection of objects form a set as long as they are not ``too many''. This is often seen as ``one step back from disaster'' since one might think that the Russell set is paradoxical percisely because it contains too many objects. When formulated in class theory, limitation of size provides justifications for the Axiom of Replacement and the Axiom of Choice. 

The reflection conception (\cite{tait2005constructing}) aims to articulate the idea that the universe of sets is so enormous that it is \textit{indescribable}, i.e., any true statement about the universe of sets is already true in an initial segment of the universe. Reflection principles are often taken as a form of \textit{maximality principle}, and when formulated as an axiom scheme, they provide justifications for many axioms of ZF including the Axiom of Inifinity and Replacement. In the second-order context, reflection principles also produce large cardinals (\cite{bernays1976problem}).

The three conceptions are by no means in conflict, and people often appeal to them simultaneously when justifying the axioms of ZFC. But how do they relate to each other, and which one, if any, is a more robust conception? In the context of pure set theory without urelements, these questions seem to have easy answers. For example, limitation of size is also viewed as a robust maximality principle (e.g., G\"odel \cite{godel1986collected} holds this view) because it allows sets to form as much as possible. Furthermore, since first-order reflection follows as a theorem scheme in ZF, the reflection conception appears to be a consequence of the other two. However, when urelements are included, the situation becomes completely different. In Chapter 4, we will see that limitation of size, when formulated as an axiom in class theory with urelements, becomes highly unnatural. Instead, it is its negation that may count as a maximality principle. Additionally, with urelements, the first-order reflection principle no longer follows from the basic axioms of ZF. In fact, in Chapter 2, I will present evidence suggesting that reflection is a more robust conception of set than the other two.


\section{Basic axioms}\label{section:basicaxioms}
The language of urelement set theory, in addition to $\in$, contains a unary predicate $\A$ for urelements. $Set(x)$ abbreviates $\neg\A(x)$. The standard axioms (and axiom schemes) of ZFC, modified to allow urelements, are as follows.

\begin{itemize}
\item[] (Axiom $\A$) $\forall x (\A(x) \rightarrow \neg \exists y (y \in x))$.
\item[] (Extensionality) $\forall x, y (Set(x) \land Set(y) \land \forall z (z \in y \leftrightarrow z \in x) \rightarrow x = y)$
\item[] (Foundation) $\forall x (\exists y (y \in x) \rightarrow \exists z\in x \ (z \cap x = \emptyset))$
\item[] (Pairing) $\forall x, y \exists z \forall v (v \in z \leftrightarrow v = x \lor v = y )$
\item[] (Union) $\forall x \exists y \forall z (z \in y \leftrightarrow \exists w \in x \ ( z \in w))$.
\item[] (Powerset) $\forall x \exists y \forall z (z \in y \leftrightarrow Set(z) \land z \subseteq x)$
\item [] (Separation) $\forall x, u \exists y \forall z (z \in y \leftrightarrow z \in x \land \varphi(z, u))$
\item[] (Infinity) $\exists s (\exists y \in s \ (Set(y) \land \forall z (z \notin y)) \land \forall x \in s \ (x \cup \{x\} \in s))$
\item[] (Replacement) $\forall w, u (\forall x \in w \ \exists ! y \varphi(x, y, u)   \rightarrow \exists v \forall x \in w \ \exists y \in v \ \varphi(x, y, u))$
\item[] (Collection) $\forall w, u (\forall x \in w \ \exists y \varphi(x, y, u)   \rightarrow \exists v \forall x \in w\  \exists y \in v\  \varphi(x, y, u))$.
\item[] (AC) Every set is well-orderable.\footnote{Over $\ZFUR$, this form of AC is equivalent to other variants such as the principle that every family of non-empty sets has a choice function and Zorn's Lemma. The proofs are all standard and hence omitted.}
\end{itemize}
\begin{definition}
\ \newline ZU = Axiom $\A$ + Extensionality + Foundation + Pairing + Union + Powerset + Infinity + Separation.\\
$\ZFUR = $ ZU + Replacement. \\
$\ZFCUR = $ $\ZFUR$ + AC.\\
ZF = $\ZFUR$ + $\forall x \neg \A (x)$.\\
ZFC =  ZF + AC.
\end{definition}
\noindent Note that there is no axiom asserting that the urelements form a set. The subscript R indicates that the correponding theories are only formulated with Replacement rather than Collection. In fact, it is folklore that $\ZFCUR$ cannot prove Collection when a proper class of urelements is allowed. In the next chapter, the issue of axiomatizing ZFC with urelements will be discussed in depth, and we shall see that it is $\ZFCUR$ + Collection that is a more robust urelement set theory.

\begin{definition}\label{def:kernel,cardinal}
    A set is \textit{transitive} if every member of it is a subset of it. The transitive closure of a set $x$, $trc(x)$, is the smallest transitive set that contains $x$. The \textit{kernel} of an object $x$, $ker(x)$, is the set of the urelements in $trc(\{x\})$. A set is \textit{pure} if its kernel is empty. A set $\alpha$ is an ordinal if it is a transitive pure set well-ordered by $\in$. The powerset of a set $x$, $P(x)$, is the set of all \textit{subsets} of $x$. $x \sim y$ abbreviates ``$x$ is equinumerous with $y$''. A set $\kappa$ is a cardinal if it is an ordinal that is not equinumerous with any ordinal below $\kappa$.
\end{definition}
\noindent Under this definition, the kernel of a urelement is its singleton, which is somewhat nonstandard but will be useful for our purpose. In $\ZFUR$, we shall also (informally) talk about first-order parametrically definable classes.
\begin{definition}
  $U$ is the class of all objects; $V$ is the class of all pure sets; $Ord$ is the class of all ordinals; and $\A$ also denotes the class of all urelements. $A \subseteq \A$ thus means ``$A$ is a set of urelements''.  
\end{definition}
 It is routine to check that $\ZFUR$ proves the transfinite recursion theorem (it is understood that all classes are parametrically definable classes).
\begin{theorem}[$\ZFUR$]
Let $R$ be a well-founded and set-like class relation on a class $X$. Given a class function $F: X \times U \rightarrow U$, there is a unique class function $G: X \rightarrow U$ such that for every $ x \in X $, $G(x) = F(x, G\restriction \{y \in X : \<y, x> \in R\})$.  \qed
\end{theorem}
In $\ZFUR$, every set $x$ has a transitive closure and hence a kernel. For any set of urelements $A$, by transfinite recursion on $Ord$ we define the $V_{\alpha}(A)$-hierarchy as follows.
\begin{itemize}
    \item [] $V_0(A) = A$;
    \item [] $V_{\alpha+1}(A) = P(V_{\alpha}(A)) \cup V_{\alpha}(A)$;
    \item [] $V_{\gamma}(A) = \bigcup_{\alpha < \gamma} V_\alpha(A)$, where $\gamma$ is a limit;
    \item [] $V(A) = \bigcup_{\alpha \in Ord} V_\alpha(A)$.
\end{itemize}
\noindent For every $x$ and set $A$ of urelements, $x \in V(A)$ if and only if $ker(x) \subseteq A$. Every permutation $\pi$ of a set of urelements $A$ can be extended to a definable permutation of the universe $U$ in a canonical way: we let $\pi a = a$ for every urelement $a \notin A$ and $\pi x = \{\pi y : y \in x \}$ for every set $x$ by transfinite recursion. Such $\pi$ preserves $\in$ and is thus an automorphism of the universe $U$. For every set $x$, whenever $\pi$ \textit{point-wise fixes} $ker(x)$, i.e., $\pi a = a$ for every $a \in ker(x)$, $\pi$ also point-wise fixes $x$. Finally, it is a useful fact that $\ZFUR$ proves the following restricted version of Collection.
\begin{itemize}
   \item[] (Collection$^-$) $\forall w, u (\exists A \subseteq \A \ \forall x \in w \ \exists y \in V(A)\ \varphi(x, y, u)   \rightarrow \exists v \forall x \in w\  \exists y \in v\  \varphi(x, y, u))$.
\end{itemize}
\begin{prop}\label{weakcollection}
$\ZFUR \vdash$ Collection$^-$.
\end{prop}
\begin{proof}
For every $x \in w$, let $\alpha_x$ be the least $\alpha$ such that there is some $y \in V_\alpha (A)$ with $\varphi(x, y, u)$ and let $\alpha = \bigcup_{x \in w} \alpha_x$. $V_\alpha (A)$ is then the desired collection set $v$.
\end{proof}






\section{Interpreting $U$ in $V$}\label{section:interpretingUinV}
There is a canonical way of interpreting urelement set theory in pure set theory, which seems to appear first in the appendix of \cite{barwise2017admissible}.

\begin{definition}\label{barwiseinterpretation1}
Let $V$ be a model of ZF and $X$ be a class of $V$. In $V$, we define $V \llbracket X \rrbracket$ by recursion as follows. 
\begin{itemize}
    \item [] $ V \llbracket X \rrbracket = (\{0\} \times X) \cup \{\barx \in V : \exists x (\barx = \<1 , x> \land x \subseteq V \llbracket X \rrbracket)\}.$
\end{itemize}
For every $\barx, \bary \in V \llbracket X \rrbracket$,
\begin{itemize}
    \item [] $\barx \ \bar{\in} \ \bary \text{ if and only if } \exists y (\bary = \<1, y> \land  \barx  \in y);$
    \item [] $\bar{\A}(\barx) \text{ if and only if } \barx \in \{0\} \times X.$
\end{itemize}
$V \llbracket X \rrbracket$ will also denote the model $\<V \llbracket X \rrbracket,\ \bar{\A},\ \bar{\in}>$ for the language of urelement set theory. For every $x \in V$, define $\hat{x} = \<1, \{\hat{y} : y \in x\}>$ and let $\hat{V} = \{\hat{x} : x \in V\}$. For any $\barx. \bary \in V \llbracket X \rrbracket$, let $\overline{\{\barx, \bary\}} = \<1, \{\barx, \bary \}>$, which codes the pair of $\barx$ and $ \bary$ in $V\llbracket X \rrbracket$.
\end{definition}
\noindent That is, we treat $\{0\} \times X$ as the class of urelements and then generate sets by closing under the operation: if $x \subseteq V \llbracket X \rrbracket$, then $\<1, x> \in V \llbracket X \rrbracket$. By an easy induction, one can show that the map $x \mapsto \hat{x}$ is an isomorphism from $\<V, \in>$ to $\<\hat{V}, \bar{\in}>$.


\begin{theorem}\label{thm:V[X]modelsZFU}
Let $V$ be a model of ZF and $X$ be a class of $V$. $V \llbracket X \rrbracket \models$ $\ZFUR$ + Collection.
\end{theorem}
\begin{proof}
For Extensionality, if $\bar{x}, \bar{y}$ are sets in $V \llbracket X \rrbracket$ with the same $\bar{\in}$-members, then $x = \<1, \bar{x}> = \<1, \bar{y}> = y$. Foundation holds because $\bar{\in}$ is a well-founded relation in $V$. 

\textit{Pairing.} For any $\barx, \bary \in V \llbracket X \rrbracket$, $\overline{\{\barx, \bary\}}$ will be the pair of $\barx, \bary$ in $V \llbracket X \rrbracket$.


\textit{Union.} Given any set $\barx$ in $V \llbracket X \rrbracket$, let $\bary = \<1, \{\barz : \exists \bar{w} \ \bar{\in} \ \barx \  (\barz \ \bar{\in} \ \bar{w}) \}>$. $V \llbracket X \rrbracket$ then thinks that $\bary$ is the  union of $\barx$.


\textit{Powerset.}
Given a $\barx = \<1, x>$ with $x \subseteq V \llbracket X \rrbracket$. Define $\bary = \langle 1, \{\langle 1, v\rangle : v \subseteq x\}\rangle$, which is the powerset of $\barx$ in $V \llbracket X \rrbracket$.

\textit{Infinity.} It is easy to check that $\hat{\omega}$ is an inductive set in $V \llbracket X \rrbracket$.

\textit{Separation.} Given a $\barx = \<1, x>$ with $x \subseteq V \llbracket X \rrbracket$ and a parameter $\bar{u} \in V \llbracket X \rrbracket$. Let $y =\{ \barz \in x : V \llbracket X \rrbracket \models \varphi(\barx, \barz, \bar{u} ) \}$. Then $\<1, \bary>$ is the desired subset of $\barx$ in $V \llbracket X \rrbracket$.

\textit{Collection.} Suppose that for every $\barx \ \bar{\in} \ \barw$, there is some $\bary$ such that $V \llbracket X \rrbracket \models \varphi (\barx, \bary, \bar{u})$, where $\bar{u} \in V \llbracket X \rrbracket$. By Collection and Separation in $V$, there is some $v \subseteq V \llbracket X \rrbracket$ such that for every $\barx \ \bar{\in} \ \bar{w}$, there is some $\bary \in v$ with $V \llbracket X \rrbracket \models \varphi (\barx, \bary, \bar{u})$. Then $\<1, v>$ is a desired collection set in $V \llbracket X \rrbracket$.
\end{proof}


\begin{lemma}\label{vhatisov}
Let $V$ be a model of ZF and $X$ be a class of $V$. Then $\hat{V}$ is the class of \textit{all} pure sets in $V \llbracket X \rrbracket$. Therefore, $V$ is isomorphic to $V^{V\llbracket X \rrbracket}$, i.e., the class of all pure sets in $V\llbracket X \rrbracket$.
\end{lemma}
\begin{proof}
I first show that $\hat{V} \subseteq V^{V\llbracket X \rrbracket}$ by an $\in$-induction. Assume that for all $y \in x$, $\hat{y}$ is a pure set in $V \llbracket X \rrbracket$. Since if $V \llbracket X \rrbracket \models \bar{z} \in trc(\hat{x})$ then there is some $y \in x$ such that $V \llbracket X \rrbracket \models \bar{z} \in  trc(\{\hat{y}\})$, it follows that $\hat{x}$ is a pure set in $V \llbracket X \rrbracket$. To show $V^{V\llbracket X \rrbracket} \subseteq \hat{V}$, we used an $\bar{\in}$-induction. Suppose that $\barx$ is a pure set in $V \llbracket X \rrbracket$ and for all $\bar{y} \ \bar{\in} \ \bar{x}$, $\bar{y} \in \hat{V}$. Let $v = \{z \in V : \exists \bar{y} \ \bar{\in} \ \bar{x}(\bar{y} = \hat{z})\}$, which is a set in $V$ because the map $z \mapsto \hat{z}$ is 1-1. Then $\barx = \hat{v}$.
\end{proof} 
\begin{theorem}\label{thm:ACholdsViffACholdsinV[X]}
Let $V$ be a model of ZF and $X$ be a class of $V$. Then
\begin{enumerate}
\item If a set $x \subseteq V \llbracket X \rrbracket$ is well-orderable in $V$, then $\barx = \<1, x>$ is well-oderable in $V \llbracket X \rrbracket$;
\item $V \models $ AC if and only if $V \llbracket X \rrbracket \models $ AC.
\end{enumerate}
\end{theorem}
\begin{proof}
(1) First note that if $x \sim v$ for some set $v \in V$, then $V \llbracket X \rrbracket \models \barx \sim \hat{v}$, where  $\barx = \<1, x>$. This is because any bijection $f$ between $x$ and $v$ in $V$ can be coded by $\bar{f} = \<1, f'>$, where $f' = \{\overline{\<\bary, \hat{w}>} : \bary \in x \land w \in v \land f(\bary) = w\}$ and $\overline{\<\bary, \hat{w}>} = \<1, \{\overline{\{\bary\}}, \overline{\{\bary, \hat{w}\}}\}>$. Then $\bar{f}$ will be a bijection between $\barx$ and $\hat{v}$ in $V\llbracket X \rrbracket$. Thus, if $x \sim \alpha$ for some ordinal $\alpha$, $V \llbracket X \rrbracket \models \bar{x} \sim \hat{\alpha}$ and $\hat{\alpha}$ is an ordinal in $V \llbracket X \rrbracket$ by Lemma \ref{vhatisov}.

(2) The left-to-right direction follows from (1). Suppose that in $V$, some set $x$ cannot be well-ordered. Then $\hat{V}$ thinks that $\hat{x}$ cannot be well-ordered by Lemma \ref{vhatisov}, so AC fails in $\hat{V}$ and hence in $V \llbracket X \rrbracket$.
\end{proof}

\begin{theorem}\label{con(zfc)->con(zfcu)}
The following theories are mutually interpretable.
\begin{enumerate}
    \item ZF.
    \item $\ZFCUR$ + Collection +  $\A \sim \omega$.
    \item $\ZFCUR$ + Collection +  $\A \sim \omega_1$.
    \item $\ZFCUR$ + Collection +  ``for every cardinal $\kappa$, there is a set of $\kappa$-many urelements''.
\end{enumerate}
\end{theorem}
\begin{proof}
It is clear that in any model of $\ZFCUR$, the class of pure sets is a model of ZFC. And if $V$ is a model of ZF, we can first go to its $L$ to have a model of ZFC. To get exactly $\omega$-many urelements, consider the model $L \llbracket  \omega \rrbracket$ in which the set of all urelements $\bar{\A}$ is $\<1, \{0\}\times \omega>$ and $L\llbracket  \omega \rrbracket \models \bar{\A} \sim \hat{\omega}$. Similarly, $L \llbracket  \omega_1 \rrbracket$ will be a model of $\ZFCUR$  with exactly $\omega_1$-many urelements. To get unboundedly many urelements, consider $L\llbracket  Ord \rrbracket$. For every $\omega_\alpha$, $\bar{A} = \<1, \{0\} \times \omega_\alpha>$ will then be a set of urelements of size $\omega_\alpha$ in $L \llbracket Ord \rrbracket$, because $L \llbracket Ord \rrbracket \models \hat{\omega_\alpha} = \omega_\alpha$ by Lemma \ref{vhatisov}.\end{proof}

In fact, we have shown that ZF is semi-bi-interpretable (see \cite{FreireForthcoming-FREBIW-2} for a definition of bi-interpretation) with the other urelement theories listed above since the map $x \mapsto \hat{x}$ is a definable isomorphism from $V$ to the class of pure sets in $V \llbracket  X \rrbracket$ for any class $X$. Also, if $U$ is a model of $\ZFCUR$  where the class of urelements $\A$ is a set, then in $U$ the urelements can be enumerated by a pure set $x$ so there is a definable isomorphism between $U$ and $V\llbracket x \rrbracket$. It is shown in \cite{HamkinsForthcoming-HAMRIS} that no model of $\ZFCUR$ + Collection with a proper class of urelements can be bi-interpretable with a model of ZFC. 