\chapter{Forcing with Urelements}
In this chapter, I investigate forcing in the context of urelement set theory. In Section \ref{section:twomethods}, I review two standard methods of forcing: forcing via countable transitive models and forcing via Boolean-valued models. Section \ref{section:forcingoverctm} explores forcing over countable transitive models of $\ZFUR$. To overcome a significant drawback of the existing approach regarding the fullness property, I develop a new forcing machinery. The main results are as follows: (i) Over $\ZFCUR$, Collection is equivalent to the principle that every forcing relation, defined in the new way, is full. (ii) Forcing over $\ZFUR$ preserves $\ZFUR$ together with several axioms introduced in Section \ref{section:additionalaxioms}. (iii) Forcing can also destroy the $\DCK$ and recover Collection. (iv) Ground model definability fails when the ground model contains a proper class of urelements. (v) The new forcing machinery generates the same forcing extensions as the old one. In Section \ref{section:forcingBVM}, I provide a brief overview of some fundamental results about Boolean-valued models with urelements, established in joint work with Wu \cite{wu2022}. Based on these results, I consider how Boolean ultrapowers can be applied to arbitrary models of ZFCU.



\section{Two methods}\label{section:twomethods}
Forcing is a widely used technique in contemporary set theory. It has led to landmark results such as the independence of the Continuum Hypothesis and philosophical analyses such as the multiverse conception of set (\cite{Hamkins2012-HAMTSM} and \cite{Antos2015-ANTMCI}) and set-theoretic potentialism \cite{Hamkins2022-HAMTML}.

In the presence of urelements, it is natural to ask how forcing behaves. For example, will a forcing extension of a model of $\ZFUR$ always be a model of $\ZFUR$? Which of the axioms isolated in Chapter 2 will be preserved by forcing? Will the presence of urelements affect the forcing machinery? These questions are the focus of this chapter.

There are two main approaches to forcing. The first is the countable transitive model approach (CTM), which involves starting with a countable transitive model $M$ of set theory and a forcing poset $\P$ in $M$, and then extending $M$ by an $M$-generic filter $G$ over $\P$. The resulting forcing extension $M[G]$ is a countable transitive model of set theory where various set-theoretic statements, such as the Continuum Hypothesis, may fail or hold depending on the choice of the poset $\P$.

The second approach is based on Boolean-valued models (BVM). Given a complete Boolean algebra $\B$, a Boolean-valued model $M^\B$ for a first-order language $\mathcal{L}$ consists of a domain of $\B$-names together with a $\B$-valued truth assignment $\llbracket \ \rrbracket_\B$, which assigns a $\B$-value to each assertion in $\mathcal{L}$ about the $\B$-names in a way that obeys the axioms of first-order logic. If $V$ is an arbitrary model of set theory, we can form $V^\B$ as a definable class in $V$ by a complete Boolean algebra $\B$ in $V$. By the fundamental theorem of $\VB$, all the axioms of set theory will have value $1$ in $V^\B$ for every $\B$. This allows us to prove the independence of a statement $\varphi$ from set theory by finding some $\B$ such that $\llbracket \varphi \rrbracket_\B \neq 1$.

The CTM approach to forcing assumes the existence of a countable transitive model of ZFC, which is a stronger assumption than the mere consistency of ZFC. This drawback is usually addressed by working with a countable transitive model of a finite fragment of ZFC and appealing to the reflection principle in the meta-theory. On the other hand, the BVM approach to forcing does not require the ground model to be either transitive or countable. From any model of ZFC, one can construct a concrete model of ZFC + $\varphi$ by taking the quotient structure of certain Boolean-valued models. The fullness of $V^\B$ ensures that the Łoś Theorem holds, and the quotient structure can be seen as a definable class in $V$. As a result, the BVM approach is able to establish mutual interpretability between ZFC (see \cite{FreireForthcoming-FREBIW-2} for more on this) and its various extensions, while the CTM approach only establishes equiconsistency. Furthermore, the BVM approach provides a naturalistic account of forcing and allows one to force over any model of ZFC, avoiding the need for countable transitive models (see \cite{hamkins2012well}).

However, in the context of forcing with urelements, the theoretic virtues of the BVM approach are not as clear as in the classical case. While $\ZFUR$ is sufficient for the basic machinery of forcing, if the ground model $U$ is only a model of $\ZFUR$, the quotient structure may fail to be constructed due to the essential role of AC. For instance, if AC fails in $U$, there may not be any non-principal ultrafilters on a complete Boolean algebra $\B \in U$. Moreover, even if such an ultrafilter $F$ exists, the Boolean-valued model $U^\B$ may not be full without AC, and therefore the Łoś theorem may not hold for $U^\B/F$. In fact, even if $U$ is a model of $\ZFCUR$, $U^\B$ may still fail to be full, as the fullness of every properly defined $U^\B$ is equivalent to Collection (see \cite{wu2022}). Thus, a naturalist account of forcing with urelements is only possible when the ground model satisfies ZFCU. Consequently, the CTM approach becomes useful when we are interested in forcing over $\ZFUR$. An investigation of both approaches of forcing in the context of urelements is thus justified.


\section{Forcing over countable transitive models of $\ZFUR$}\label{section:forcingoverctm}
In this section, I will investigate poset forcing over countable transitive models of $\ZFUR$. Basic knowledge of forcing in ZFC covered in \cite[Ch.\ VII]{kunen2014set} will be assumed. Our meta theory, accordingly, will be some suitable urelement set theory such as $\ZFUR$. Notably, forcing with urelements has been studied in several places including \cite{blass1989freyd}, \cite{hall2002characterization}, and \cite{Hall2007-ERIPMA}. However, in all of these studies it is assumed that the urelements form a set,\footnote{For instance, in \cite{hall2002characterization}, Hall shows that if $N \subseteq M$ are countable transitive model of $\ZFUR$ with a set of urelements, then $N$ is a permutation model of $M$ only if $M$ is a certain forcing extension of $N$.} so it only needs some trivial adjustments to show that forcing preserves the axioms. But when a proper class of urelements is allowed, it becomes interesting to see which of the axioms introduced in Section \ref{section:additionalaxioms} are preserved by forcing. And this, as we shall see, will require some new arguments based on the earlier results.





\subsection{The existing approach and its problem}
 In pure set theory, given a forcing poset $\P$ with the maximal element $1_\P$, by transfinite recursion we define: $\dotx $ is a $\P$-name if and only if $\dotx$ is a set of ordered-pairs $\<\doty, p>$, where $\doty$ is a $\P$-name and $p \in \P$. Then every set $x$ in $V$ will have a canonical name  $\check{x} = \{ \<\check{y}, 1_\P> : y \in x \}$. In particular, $\emptyset$ be the its own name. To generalize this definition in urelement set theory, a natural idea, adopted in  \cite{blass1989freyd}, \cite{hall2002characterization} and \cite{Hall2007-ERIPMA}, is to treat each urelement as a different copy of $\emptyset$, which yields the following definition.
%Old P-names%
\begin{definition}\label{oldpnames}
Let $\P$ be a forcing poset. $\dot{x}$ is a $\P$-name$_\#$ if and only if either $\dot{x}$ is an urelement, or $\dot{x}$ is a set of ordered-pairs $\langle \dot{y}, p \rangle$, where $\dot{y}$ is a $\P$-name$_\#$ and $p \in \P$. $U^{\P}_\# = \{ \dot{x} \in U : \dotx \text{ is a } \P\text{-name}_\#\}$.
\end{definition}
\noindent This definition turns out to have a drawback, and the subscript $\#$ is meant to indicate this fact. To reveal its problem, we need to develop the basics of this approach. Let $\mathcal{L}^\P_\#$ be forcing language which contains $\{=, \in, \A \}$ as the non-logical symbols and each $\P$-name$_\#$ as a constant symbol. For each formula $\varphi (v_1, ..., v_n) \in \mathcal{L}^\P_\#$ and $\dotx_1, ..., \dotx_n \in U^{\P}_\#$, one can define the forcing relation $p \forces^\P_\# \varphi(\dotx_1, ..., \dotx_n)$ by recursion as follows (the superscript $\P$ will be omitted when it is clear from the context).

\begin{definition}\label{oldforcinglrelation}
Let $\P$ be a forcing poset. For every $\dotx, \doty, ... \in U^{\P}_\#$ and $p \in \P$,
\begin{enumerate}
    \item $p \forces_\# \dotx \in \doty$ if and only if $\{ q \in \P : \exists \<\dotz, r> \in \doty (q \leq r \land q \forces_\# \dotz = \dotx)\}$ is dense below $p$.
    \item $p \forces_\# \dotx \subseteq \doty$ if and only if whenever $\<\dotz, r> \in \dotx$ and $q \leq p, r$, then $q \forces_\# \dotz \in \doty$.
    \item $p \forces_\# \dotx = \doty$ if and only if either (i) $\dotx$ and $\doty$ are the same urelement, or (ii) $p \forces_\# \dotx \subseteq \doty$ and $p \forces_\# \doty \subseteq \dotx$.
    \item $p \forces_\# \A (\dotx)$ if and only if $\dotx$ is an urelement.
    \item $p \forces_\# \neg \varphi$ if and only if there is no $q \leq p$ such that $q \forces_\# \varphi$.
    \item $p \forces_\# \varphi \land \psi$ if and only if $p \forces_\# \varphi$ and $p \forces_\# \psi$.
    \item $p \forces_\# \exists x \varphi$ if and only if $\{q \in \P : \text{ there is some } \dotz \in U^{\P}_\# \text{ such that }q \forces_\# \varphi(\dotz)\}$ is dense below $p$.
\end{enumerate}
\end{definition}


Now let $M$ be a countable transitive model of $\ZFUR$ and $\P \in M$ be a forcing poset. Given an $M$-generic filter $G$, for every $\dotx \in M^\P_\# = M \cap U^\P_\#$ , we define 

\begin{equation*}
       \dotx_G =
    \begin{cases*}
       \dotx & if $\A(x)$ \\
\{\dot{y}_G : \exists p \in G \langle \dot{y}, p \rangle \in \dot{x} \}     & otherwise 
    \end{cases*}
\end{equation*}
$M[G]_\# = \{\dotx_G : \dotx \in M^\P_\#  \}$ is then a transitive model that includes $M$ with the same ordinals and urelements of $M$. Moreover, one can easily prove the forcing theorem for $(\forces_\#)^M$.

\begin{theorem}[The Forcing Theorem for $\forces_\# $]
Let $M$ be a countable transitive model of $\ZFUR$, $\P \in M$ be a forcing poset, and $G$  be an $M$-generic filter over $\P$. For every $\dotx_0, ..., \dotx_n \in M^\P_\#$, 
\begin{align*}
    M[G]_\# \models \varphi (\dotx_{0_{G}}, ..., \dotx_{n_{G}})\text{ if and only if } \exists p \in G  (p \forces_\# \varphi (\dot{x}_0, ..., \dot{x}_n))^M.
\end{align*}
\end{theorem}
\begin{proof}
By the definition of $\forces_\#$, $M[G]_\# \models \A (\dotx_{G})$ just in case  $p \forces_\#  \A(\dotx)$ for every $p \in \P$, so the urelement predicate causes no problem. And when one of $\dotx_0$ and $\dotx_1$ is an urelement, their $G$-valuations are identical only if $\dotx_0$ and $\dotx_1$ are the same urelement. The rest of the theorem can then be proved by standard text-book arguments as in Kunen\cite[Ch.VII]{kunen2014set}.
\end{proof}
\noindent In fact, one can proceed to show that  $M[G]_\# \models \ZFUR$. 

However, one important feature of forcing is missing in this approach. The following is a standard theorem of ZFC (i.e., $\ZFCUR$ + ``there is no urelements'').

\begin{theorem}[ZFC]
Let $\P$ be a forcing poset. Its forcing relation is \textit{full}, i.e., whenever $p \forces \exists y \varphi(y,\dotx_0, ..., \dotx_n)$, then $p \forces \varphi (\doty, \dotx_0, ..., \dotx_n)$ for some $\doty \in V^\P$. \qed
\end{theorem}
\begin{remark}
If $U$ contains two urelements and $\P$ contains a maximal antichain with at least two elements, its forcing relation $\forces_\#$ is not full.
\end{remark}
\begin{proof}
Suppose that $\<p_i : i \in I>$, where $I$ has at least two elements, is a maximal antichain, and let $\<a_i : i \in I>$ be some urelements such that at least two of them are distinct. Consider the $\P$-name$_\#$ $\dotx = \{\<a_i, p_i> : i \in I \}$. It is routine to check that $1_\P \forces_\# \exists y (y \in \dotx)$. Suppose \textit{for reductio} that $1_\P \forces_\# \doty \in \dotx$ for some $\doty \in U^\P_\#$. Then there will be two distinct urelements $a_i$ and $a_j$ such that both ``$\doty = a_i$'' and ``$\doty = a_j$'' are forced, which is impossible by the definition of $\forces_\#$.
\end{proof}
\noindent The reason of why this happened is that $U^\P_\#$ contains too few names. Recall the following standard theorem in ZF.
\begin{theorem}[ZF]
Let $\P$ be a forcing poset. Then if $f$ is a function from an antichain of a forcing poset $\P$ to $V^\P$, then there is a $\doty \in V^\P$, called \textit{a} \textit{mixture} of $f$, such that $p \forces f(p) = \doty$ for every $p \in dom(f)$. \qed
\end{theorem}
\noindent As we have seen, this does not hold for $U^\P_\#$ because we cannot even mix two urelements.
\subsection{A new forcing machinery with urelements}
To have mixtures, we would want all names to be sets of ordered pairs. Also, in any forcing extension, if a name is collapsed into an urelement, there must be a unique one; furthermore, no name should be collapsed into a member of this urelement. This motivates the following new definition of names.
\begin{definition}\label{newpnames}
Let $\P$ be a forcing poset. $\dot{x}$ is a $\P$-name if and only if (i) $\dot{x}$ is a set of ordered-pairs $\langle y, p \rangle$ where $p\in \P$ and $y$ is either a $\P$-name or an urelement, and (ii) whenever $\langle a, p \rangle, \langle y, q\rangle \in \dot{x}$, where $a$ is an urelement and $a \neq y$, then $p$ and $q$ are incompatible (written as $p \bot q$). For every urelement $a$, $\check{a} = \{\<a, 1_\P>\}$; for every set $x$, $\check{x} = \{\<\check{y}, 1_\P> : y \in x\}$. $U^\P = \{ \dotx : \dotx \text{ is a } \P \text{-name}\}$.
\end{definition}
\noindent Note that $U^\P$, unlike $U^\P_\#$, contains no urelements. In particular, $\{\<a, 1>\}$ is not the canonical name of $\{a\}$ but the canonical name of $a$ itself. And when $\<a, p> \in \dotx$ for some urelement $a$, this indicates that $a$ will be \textit{identical to}, rather than \textit{a member of}, $\dotx_G$ for any generic filter $G$ containing $p$. Now we proceed to define the new forcing relation.

\begin{definition}\label{def:newforcingrelation}
Let $\P$ be a forcing poset. The forcing language $\mathcal{L}^\P$ contains $\{\subseteq, =, \in, \A, \overset{\mathscr{A}}{=}\}$ as the non-logical symbols and every $\P$-name in $U^\P$ as a constant symbol. For every $\dotx_1, \dotx_2, ... \in U^\P$, $p \in P$ and $\varphi \in \mathcal{L}^\P$,
\begin{enumerate}
    \item $p \forces \A(\dotx_1)$ if and only if $\{q \in \P : \exists \<a , r> \in \dotx_1 \ (\A(a) \land q \leq r)\}$ is dense below $p$.
    \item $p \forces \dotx_1 \Aeq \dotx_2$ if and only if $\{q \in \P : \exists a, r_1, r_2 (\A(a) \land \<a, r_1> \in \dotx_1 \land \<a, r_2> \in \dotx_2 \land q \leq r_1, r_2 )\} \cup \{q \in \P : \forall \<a_1, r_1> \in \dotx_1 \ (\A(a_1) \rightarrow q \bot r_1) \land \forall \<a_2, r_2> \in \dotx_2 \ (\A(a_2) \rightarrow q \bot r_2)\}$ is dense below $p$. 
    \item $p \forces \dotx_1 \in \dotx_2$ if and only if $\{ q \in \P : \exists \<\doty, r> \in \dotx_2 (\doty \in U^\P \land q \leq r \land q \forces \doty = \dotx_1)\}$ is dense below $p$.
    \item $p \forces \dotx_1 \subseteq \dotx_2$ if and only if for every $\doty \in U^\P$ and $r, q \in \P$, if $\<\doty, r> \in \dotx_1$ and $q \leq p, r$, then $q \forces \doty \in \dotx_2$.
    \item $p \forces \dotx_1 = \dotx_2 $ if and only if $p \forces \dotx_1 \subseteq \dotx_2$, $p \forces \dotx_2 \subseteq \dotx_1$ and $p \forces \dotx_1 \Aeq \dotx_2$.
    \item $p \forces \neg \varphi$ if and only if there is no $q \leq p$ such that $q \forces \varphi$.
    \item $p \forces \varphi \land \psi$ if and only if $p \forces \varphi$ and $p \forces \psi$.
    \item $p \forces \exists x \varphi$ if and only if $\{q \in \P : \text{ there is some } \dotz \in U^{\P} \text{ such that }q \forces \varphi(\dotz)\}$ is dense below $p$.
\end{enumerate}
\end{definition}
\begin{lemma}\label{forcingbasic}
Let $\P$ be a forcing notion and $p, q \in \P$.
\begin{enumerate}
    \item If $p \forces \varphi$ and $q \leq p$, then $q \forces \varphi$.
    \item If $\{r \in \P : r \forces \varphi \}$ is dense below $p$, $p \forces \varphi$.
    \item $1_\P \forces \dotx = \dotx$ for every $\dotx \in U^\P$.
    \item If $\dotx \in U^\P$ and $\<\doty, r> \in \dotx$, where $\doty$ is a $\P$-name, then $r \forces \dotx \in \doty$.
    \item If $p \forces \varphi(\dotx, \dotu_1, ..., \dotu_n)$ and $p \forces \dotx = \doty$, then $p \forces \varphi(\doty, \dotu_1, ..., \dotu_n)$.
    \item Let  $\varphi$ be an atomic formula in the language of urelement set theory. For every $x, y$, $\varphi(x, y)$ if and only $p \forces \varphi(\check{x}, \check{y})$. \qed
\end{enumerate}
\end{lemma}
\noindent The following lemma verifies that $U^\P$, unlike $U^P_\#$, is closed under mixtures.
\begin{lemma}\label{mixinglemmaforUP}
Let $\P$ be a forcing poset. Then for every function $f : dom(f) \rightarrow U^\P$, where $dom(f)$ is an antichain in $\P$, there is a $\dotx \in U^\P$ ( \textit{a} \textit{mixture} of $f$) such that $p \forces f(p) = \dotx$ for every $p \in dom(f)$.
\end{lemma}
\begin{proof}
Define $\dotx$ as follows.
\begin{align*}
    \dotx = \bigcup_{p \in dom(f)}\{\<y, r> \in dom(f(p)) \times \P :  \exists q \  ( \<y, q> \in f(p) \land r \leq p, q )\}.
\end{align*}
\noindent Let us first check that $\dotx$ is a $\P$-name satisfying the incompatibility condition. Suppose that $\<a, r_1>, \<y, r_2> \in \dotx$ and $a \neq y$ for some urelement $a$. Then there is some $p \in dom(f)$ and some $q$ such that $\<a, q_1> \in f(p)$ and $r_1 \leq p, q_1$. If $y$ is also in $dom(f(p))$, then there is some $q_2$ such that $\<y, q_2> \in f(p)$ and $r_2 \leq q_2$, which means $r_1$ and $r_2$ are incompatible. If $y \in dom(p')$ for some $p' \in dom(f)$ distinct from $p$, then $\<y, q> \in f(p')$ for some $q'$ and $r_2 \leq p', q'$. Then $r_2$ is incompatible with $r_1$ because $dom(f)$ is an antichain.

Consider any $p \in dom(f)$. It remains to show that $p \forces \dotx = f(p)$. $p \forces \dotx \subseteq f(p)$ because if $\<\doty, r> \in \dotx$ and $q \leq p, r$, then $q \forces \doty \in f(p)$ since $r \forces \doty \in f(p)$. To show $p \forces f(p) \subseteq \dotx$, suppose that $\<\doty, q> \in f(p)$ and $r \leq p, q$. Then $\<\doty, r> \in \dotx$ and hence $r \forces \doty \in \dotx$. 

Finally, we show that $p \forces \dotx \Aeq f(p)$. Fix a condition $s\leq p$. 

\noindent \textit{Case 1.} There is some $\<b, r> \in \dotx$ for some urelement $b$ such that $s$ and $r$ are compatible. Then for some $p' \in dom(f)$ and $\<b, r'> \in f(p')$, $r \leq p', r'$. It follows that $p$ and $p'$ are compatible and hence $p = p'$. Thus, $s$ has an extension in the set $\{q \in \P : \exists a, r_1, r_2 (\A(a) \land \<a, r_1> \in \dotx \land \<a, r_2> \in f(p) \land q \leq r_1, r_2 )\}$. 

\noindent \textit{Case 2.} There is no $\<a, r> \in \dotx$ such that $a$ is an urelement and $s$ is compatible with $r$. Then for every $\<a, r'> \in f(p)$, $s$ cannot be compatible with $r'$ either since otherwise $\<a, s'>$ will be in $\dotx$ for some $s' \leq s, r'$. Thus, $s$ is in the set $\{q \in \P : \forall \<a_1, r_1> \in \dotx \ (\A(a_1) \rightarrow q \bot r_1) \land \forall \<a_2, r_2> \in f(p) \ (\A(a_2) \rightarrow q \bot r_2)\}$. This shows that  $p \forces \dotx \Aeq f(p)$.\end{proof}



\begin{theorem}\label{thm:fulness<->collection}
Over $\ZFCUR$, the following are equivalent.
\begin{enumerate}
    \item Collection.
    \item Fullness Principle: for every forcing poset $\P$, its forcing relation $\forces$ is full.
\end{enumerate}
\end{theorem}
\begin{proof}
The argument for $(1) \rightarrow  (2)$ is standard given that we know $U^\P$ is closed under mixtures for every $\P$. So fix some forcing poset $\P$ and suppose that $p \forces \exists y \varphi$ (with parameters suppressed). By AC, there is a maximal antichain $I$ in the subposet $\{q \in \P : q \leq p \land \exists \dotx \in U^\P q \forces \varphi(\dotx) \}$. By Collection and AC, there is a function $f : I \rightarrow U^\P$ such that for every $q \in I$, $q \forces \varphi(f(q))$. It follows from Lemma \ref{mixinglemmaforUP} that there is some $\doty \in U^\P$ such that $q \forces \doty = f(q)$ and hence $q \forces \varphi(\doty)$ for every $q \in I$. It is then routine to check that $p \forces \varphi(\doty)$.

$(2) \rightarrow (1).$ Assume $(2)$ and suppose that $\forall x \in w \exists y \varphi (x, y, u)$ for some set $w$ and parameter $u$. Define the forcing poset $\mathbb{W}$ to be $w \cup \{w\}$, where for every $p, q \in \mathbb{W}$, $p \leq q$ if and only if $p = q $ or $q = w$.

\begin{claim}
For every $\doty \in U^{\mathbb{W}}$ and $p \in w$, there is some $y$ such that $ker(y) \subseteq ker(\mathbb{W}) \cup ker(\doty)$ and $p \forces \doty = \check{y}$.
\end{claim}
\begin{claimproof}
By induction on the rank of $\doty$. We may assume that there is no urelement $a$ such that $\<a, p> \in \doty$ or $\<a, 1_{\mathbb{W}}> \in \doty$, since otherwise $p \forces \doty = \check{a}$ for such $a$. Define
\begin{align*}
  y = \{z : \exists \dotz \in dom(\doty) \cap U^\mathbb{W} \ (ker(z) \subseteq ker(\dotz) \cup ker(\mathbb{W}) \land p \forces \dotz \in \doty \land p \forces \dotz = \check{z})\}.
\end{align*}
$y$ is a set by Lemma \ref{forcingbasic} (6), and it is clear that $ker(y) \subseteq ker(\mathbb{W}) \cup ker(\doty)$. $p \forces \doty \Aeq \check{y}$ by the assumption. To show that $p \forces \doty \subseteq \check{y}$, observe that if $\<\dotz, r> \in \doty$ and $q \leq p, r$, then $p = q$ and $p \forces \dotz \in \doty$; and since by the induction hypothesis $p \forces \dotz = \check{z}$ for some $z \in y$, it follows that $p \forces \dotz \in \check{y}$. $p \forces \check{y} \subseteq \doty$ because for any $z \in y$,  $p \forces \dotz = \check{z}$ and $p \forces \dotz \in \doty$ for some $\mathbb{W}$-name $\dotz \in dom(\doty)$, and hence $p \forces \check{z} \in \doty$.
\end{claimproof}

\noindent By induction on the complexity of formulas and Lemma \ref{forcingbasic} (6), it follows that for every $p \in \mathbb{W}$ and $x_1, ..., x_n$, $\psi(x_1, ..., x_n)$ if and only if $p \forces \psi(\check{x_1}, ..., \check{x_n})$ for any formula $\psi$ in the language of urelement set theory.

Next we define a $\P$-name $\dotx = \{\<\check{z}, p> : \neg \A (p) \land z \in p \land p \in w\} \cup \{\<a, a> : a \in w \land \A(a) \}$.
\begin{claim}
For every $p \in w$, $p \forces \check{p} = \dotx.$
\end{claim}
\begin{claimproof}
If $p = a $ for some urelement $a$, then it is clear from the definition that $p \forces \check{a} = \dotx$. If $p$ is a set, then $p \forces \check{p} \Aeq \dotx$ because if $\<a, a> \in \dotx$ then $p \bot a$. $p \forces \check{p} \subseteq \dotx$ because whenever $z \in p$, $p \forces \check{z} \in \dotx$. $p \forces \dotx \subseteq \check{p}$ because if $\<\check{z}, r> \in \dotx$ and $q \leq p, r$, then $r= q= p$ and $z \in p$, so $p \forces \check{z} \in \check{p}$.
\end{claimproof}


Now for every $p \in w$, there is some $y$ such that $\varphi(p, y, u)$; so $p \forces \varphi(\check{p}, \check{y}, \check{u})$ and hence $p \forces \varphi(\dotx, \check{y}, \check{u})$. By the definition of $\forces$, this means $1_\mathbb{W} \forces \exists y \varphi(\dotx, y, \check{u})$. Since $\forces$ is full for $U^\mathbb{W}$, it follows that there is a $\doty \in U^\mathbb{W}$ with $1_\mathbb{W} \forces \varphi(\dotx, \doty, \check{u})$; so for every $p \in w$, $p \forces \varphi(\check{p}, \doty, \check{u})$. By the first claim $p \forces \doty = \check{y}$ for some $y$ such that $ker(y) \subseteq ker(\doty) \cup ker(\mathbb{W})$. Let $A = ker(\doty) \cup ker(\mathbb{W})$. For every $p \in w$, there is some $y \in V(A)$ such that $p \forces \varphi( \check{p}, \check{y}, \check{u})$ and hence $\varphi(p, y, u)$. This suffices for Collection by Proposition \ref{weakcollection}.
\end{proof}
\noindent It is folklore that Fullness Principle implies AC. I include a proof of this for completeness.

\begin{theorem}\label{thm:fulness->AC}
$\ZFUR \vdash$ Fullness Principle $\rightarrow$ AC.
\end{theorem}
\begin{proof}
Let $w$ be a set of non-empty sets. Define the forcing poset $\mathbb{W}$ to be $w \cup \{w\}$, where for every $p, q \in \mathbb{W}$, $p \leq q$ if and only if $p = q $ or $q = w$. As before, since $\mathbb{W}$ is trivial, for every $\doty \in U^{\mathbb{W}}$ and $p \in w$, there is some $y$ such that $p \forces \doty = \check{y}$. And consequently, for every $x_1, ... x_n$, $\psi(x_1, ..., x_n)$ if and only if $p \forces \psi(\check{x_1}, ..., \check{x_n})$. Define $\dotx = \{\<\check{z}, p> :  z \in p \land p \in w\}$. The same argument as before shows that $p \forces \check{p} = \dotx $ for every $p \in w$. For every $p \in w$, since it is non-empty, there is some $z \in p$ such that $p \forces \check{z} \in \check{p}$ and hence $p \forces \check{z} \in \dotx$. This shows that $1_\mathbb{W} \forces \exists y (y \in \dotx)$. By Fullness Principle, there is a $\doty \in U^\mathbb{W}$ such that $p \forces \doty \in \dotx$ for every $p \in w$. Thus, $p \forces \doty \in \check{p}$ for every $p \in w$. Now define a function $f$ on $w$ such that $f(p) = z$ if and only if $p \forces \check{z} = \doty$. It follows that $f$ is a choice function on $w$.\end{proof}


\begin{corollary}
Over $\ZFUR$,
\begin{enumerate}
    \item Fullness Principle $\rightarrow$ RP;
    \item RP $\nrightarrow$ Fullness Principle.
\end{enumerate}
\end{corollary}
\begin{proof}
The implication follows from Theorem \ref{thm:fulness->AC}, Theorem \ref{thm:fulness<->collection} and Lemma \ref{easyimplication} (6). The implication cannot be reversed by the Basic Fraenkel Model (Example \ref{exp:BasicFModel}).
\end{proof}






\subsection{Forcing extensions and the forcing theorem}
\begin{definition}\label{m[g]def}
Let $M$ be a countable transitive model of $\ZFUR$, $\P \in M$ be a forcing poset and $G$ be an $M$-generic filter over $\P$.
\begin{enumerate}
    \item $M^\P = U^\P \cap M$
   \item For every $\dot{x} \in M^\P$, 
        \subitem (i) $\dotx_G = a$ if  $\mathcal{A}(a)$ and  $\langle a, p \rangle \in \dot{x}$ for some $p \in G$; 
        \subitem (ii) $\dotx_G = \{ \dot{y}_G: \langle \doty , p \rangle \in \dotx \text{ for some } \doty \in M^\P \text{ and } p \in G \}$ otherwise.
     \item $M[G] = \{\dotx_G : \dot{x} \in M^\P \}$. 
\end{enumerate}
\end{definition}
\noindent Note that $\dotx_G$ is well-defined by clause (ii) in Definition \ref{newpnames}. It is shown in \ref{subsection:m[g]=m[g]sharp} that $M[G]$ is in fact the same as $M[G]_\#$.
\begin{lemma}
Let $M$ be a countable transitive model of $M$, $\P \in M$ be a forcing poset, and $G$ be an $M$-generic filter over $\P$. Then
\begin{enumerate}
    \item  $M \subseteq M[G]$;
    \item  $G \in M[G]$;
    \item  $M[G]$ is transitive;
    \item  $Ord \cap M = Ord \cap M[G]$;
    \item  For every transitive model $N$ of $\ZFUR$ such that $G\in N$ and $M \subseteq N$, $M[G] \subseteq N$;
    \item $\mathcal{A} \cap M = \mathcal{A} \cap M[G]$.
\end{enumerate}
\end{lemma}
\begin{proof}
(1)--(5) are all proved by standard text-book arguments as in \cite[Ch.VII]{kunen2014set}. (6) is clear by the construction of $M[G]$ because every urelement in $M[G]$ must come from $ker(\dotx)$ for some $\dotx \in M^\P$. 
\end{proof}


\begin{lemma}\label{McoversM[G]}
$ker(\dot{x}_G) \subseteq ker(\dot{x})$, for every $\dot{x} \in M^{\P}$. Hence, every set of urelements in $M[G]$ is a subset of some set of urelements in $M$.
\end{lemma}
\begin{proof}
By induction on the rank of $\dot{x}$, and we may assume that $\dotx_G$ is a set. Since $ker(\dot{x}_G) \subseteq \bigcup \{ ker(\dot{y}_G) : \dot{y} \in dom(\dot{x})\}$ and by the induction hypothesis $ker(\dot{y}_G) \subseteq ker(\dot{y}) \subseteq ker(\dot{x})$ for every $\dot{y} \in dom(\dot{x})$, the result follows.
\end{proof}

\begin{theorem}[The Forcing Theorem for $\forces$]
Let $M$ be a countable transitive model of $\ZFUR$, $\P \in M$ be a forcing poset. Then for every $\dotx_1, ..., \dotx_n \in M^\P$,

\begin{enumerate}
    \item For every $M$-generic filter $G$ over $\P$, $M[G] \models \varphi (\dotx_{1_G}, ..., \dotx_{n_G})$ if and only if $\exists p \in G (p \forces \varphi(\dotx_1, ..., \dotx_n))^M$.
    \item For every $p \in \P$, $(p \forces \varphi(\dotx_1, ..., \dotx_n))^M$ if and only if for every $M$-generic filter $G$ over $\P$ such that $p \in G$, $M[G] \models \varphi (\dotx_{1_G}, ..., \dotx_{n_G})$. 
\end{enumerate}
\end{theorem}
\begin{proof}
(2) is an easy consequence of (1) and the proof of the Boolean cases and quantifier case of (1) is the same as in \cite[Chapter VII. Theorem 3.5]{kunen2014set}. So it remains to show that the atomic cases for (1) hold.

\noindent \textit{Case 1}. $\varphi(\dotx_1, \dotx_2)$ is $\dotx_1 \in \dotx_2$. The argument is the same as in \cite[Chapter VII, Theorem 3.5]{kunen2014set}.


\noindent \textit{Case 2}. $\varphi(\dotx)$ is $\A(\dotx)$. Suppose that $\dotx_G$ is some urelement $b$. Then $\<b, p> \in \dotx$ for some $p \in G$, so $\{q \in \P : \exists \<a , r> \in \dotx \ (\A(a) \land q \leq r)\}$ is dense below $p$ and hence $p \forces \A(\dotx)$. Suppose that $p \forces \A(\dotx)$ for some $p \in G$. Then there is some $q \in G$ such that $\<b, r> \in \dotx$ for some $r \geq q$ and urelement $b$. Thus, $\dotx_G = b$.

\noindent \textit{Case 3}. $\varphi(\dotx_1, \dotx_2)$ is $\dotx_1 = \dotx_2$. For the left-to-right direction of (1), suppose that $\dotx_{1_G} = \dotx_{2_G}$.


\textit{Subcase 3.1}. $\dotx_{1_G} = \dotx_{2_G} = b$ for some urelement $b$. Then $\<b, s_1> \in \dotx_1$ and $\<b, s_2> \in \dotx_2$ for some $s_1, s_2 \in G$. Fix some $p \in G$ such that $p \leq s_1, s_2$. Observe first that $p \forces \dotx_1 \subseteq \dotx_2$ and $p \forces \dotx_2 \subseteq \dotx_1$ trivially hold: for any $\P$-name $\doty$ and $r \in \P$ such that $\<\doty, r> \in \dotx_1 (\text{or } \dotx_2)$, $p$ must be incompatible with $r$ because $r$ is incompatible with $s_1(\text{or }s_2)$. Moreover, $p \forces \dotx_1 \Aeq \dotx_2$ because $\{q \in \P : \exists a, r_1, r_2 (\A(a) \land \<a, r_1> \in \dotx_1 \land \<a, r_2> \in \dotx_2 \land  q \leq r_1, r_2 )\}$ is clearly dense below $p$. Hence, $p \forces \dotx_1 = \dotx_2$.

\textit{Subcase 3.2}. $\dotx_{1_G}$ is a set.  We first use a standard text-book argument to show that there is some $p \in G$ such that $p \forces \dotx_1 \subseteq \dotx_2$ and $p \forces \dotx_2 \subseteq \dotx_1$. Define:
\begin{itemize}
    \item [] $D_1= \{p \in \P : p \forces \dotx_1 \subseteq \dotx_2 \land p \forces \dotx_2 \subseteq \dotx_1 \}$
    \item [] $D_2 = \{p \in \P : \exists \<\doty_1, q_1> \in \dotx_1 \ (p \leq q_1 \land \forall \<\doty_2, q_2> \in \dotx_2 \ \forall r \leq q_2 \ (r \forces \doty_1 = \doty_2 \rightarrow p \bot r )) \}$
    \item [] $D_3 = \{p \in \P : \exists \<\doty_2, q_2> \in \dotx_2 \ (p \leq q_2 \land \forall \<\doty_1, q_1> \in \dotx_1 \ \forall r \leq q_1 \  (r \forces \doty_2 = \doty_1 \rightarrow p \bot r )) \}$
\end{itemize}
 If $p \nVdash \dotx_1 \subseteq \dotx_2$, then there are $\<\doty_1, q_1> \in \dotx_1$ and $r \leq p, q_1$ such that $r \nforces \doty_1 \in \dotx_2$; so there is an $s \leq r$ such that for every $\<\doty_2, q_2> \in \dotx_2$ and $s' \leq q_2$, if $s' \forces \doty_1 = \doty_2$, then $s \bot s'$. Hence, $s \leq p$ and $s \in D_2$. Similarly, if $p \nforces \dotx_2 \subseteq \dotx_1$, then $p$ will have an extension in $D_3$. This shows that $D_1 \cup D_2 \cup D_3$ is dense. However, $G \cup (D_2 \cup D_3)$ must be empty. Suppose \textit{for reductio} that $p \in G\cap D_2$. Fix some $\<\doty_1, q_1> \in \dotx_1$ with $p \leq q_1$ that witnesses $p \in D_2$. It follows that $\doty_1{_G} = \doty_2{_G}$ for some $\<\doty_2, q_2> \in \dotx_2$ with $q_2 \in G$. By the induction hypothesis, there is some $r \in G$ such that $r \leq q_2$ and $r \forces \doty_1 = \doty_2$. But $p$ must be incompatible with such $r$, which is a contradiction. The same argument shows that $G \cap D_3$ is empty. Therefore, there is some $p \in G$ such that $p \forces \dotx_1 \subseteq \dotx_2$ and $p \forces \dotx_2 \subseteq \dotx_1$.






Now I wish to find some $q \in G$ such that $q \forces \dotx_1 \Aeq \dotx_2$. Define:
\begin{itemize}
    \item [] $E_1 = \{ q \in \P : \forall r \leq q \ [\forall \<a_1, s_1> \in \dotx_1 \ (\A(a) \rightarrow r \bot s_1) \land \forall \<a_2, s_2> \in \dotx_2 \ (\A (a_2) \rightarrow r \bot s_2)]\}$.
    \item [] $E_2 = \{ q \in \P : \exists \<a, r> \in \dotx_1 \ (\A(a) \land q \leq r) ) \}$.
    \item [] $E_3 = \{ q \in \P : \exists \<a, r> \in \dotx_2 \ (\A(a) \land q \leq r)\}$.
\end{itemize}
$E_1 \cup E_2 \cup E_3$ is dense. But if there is some $q \in G \cap (E_2 \cup E_3)$, either $\dotx_{1_G}$ or $\dotx_{2_G}$ would be an urelement. Thus there is some $q \in G \cap E_1$ such that the set
\begin{itemize}
    \item [] $\{r \in \P : \forall \<a_1, s_1> \in \dotx_1 \ (\A(a_1) \rightarrow r \bot s_1) \land \forall \<a_2, s_2> \in \dotx_2 \ (\A (a_2) \rightarrow r \bot s_2)\}$
\end{itemize}
is dense below $q$. Therefore, $q \forces \dotx_1 \Aeq \dotx_2$. A common extension of $p$ and $q$ in $G$ will then force $\dotx_1 = \dotx_2$.

For the right-to-left direction of (1) in Case 3, suppose that for some $p \in G$, $p \forces \dotx_1 = \dotx_2$.

\textit{Subcase 3.3}. $\dotx_{1_G} = b$ for some urelement $b$. Then $\<b, r> \in \dotx_1$ for some $r \in G$. Define:
\begin{itemize}
    \item [] $F_1 = \{ q \in \P : \exists a, s_1, s_2 (\A (a) \land \<a, s_1> \in \dotx_1 \land \<a, s_2> \in \dotx_2 \land q \leq s_1, s_2) \}$.
    \item [] $F_2 = \{q \in \P : \forall \<a, s_1> \in \dotx_1 \ (\A(a) \rightarrow q \bot s_1) \land \forall \<a, s_2> \in \dotx_2 \ (\A(a) \rightarrow q \bot s_2 )\}$.
\end{itemize}
Since $p \forces \dotx_1 \Aeq \dotx_2$, $F_1 \cup F_2$ is dense below $p$. But clearly $F_2 \cap G$ is empty as $\<b, r> \in \dotx_1$, so there is some $q \in F_1 \cap G$. It follows that $\<b, s_1> \in \dotx_1$ and $\<b, s_2> \in \dotx_2$ for some $s_1, s_2 \in G$. Therefore, $\dotx_{2_G} = b = \dotx_{2_G}$.

\textit{Subcase 3.4}. $\dotx_{1_G}$ is a set. Suppose \textit{for reductio} that $\dotx_{2_G}$ is some urelement $b$ and so $\<b, r> \in \dotx_2$ for some $r \in G$. Since $p \forces \dotx_1 \Aeq \dotx_2$, it follows that there are some urelement $a$ and  $s \in G$ such that $\<a, s> \in \dotx_1$. This implies that $\dotx_{1_G} = a$, which is a contradiction. Hence, $\dotx_{2_G}$ is a set, so it remains to show that $\dotx_{1_G}$ and $\dotx_{2_G}$ have the same members. If $\doty_G \in \dotx_{1_G}$, then $\<\doty, r> \in \dotx_1$ for some $r \in G$. So there is some $q \in G$ with $q \leq p, r$, and since $p \forces \dotx_1 \subseteq \dotx_2$, $q \forces \doty \in \dotx_2$. By the induction hypothesis, $\doty_G \in \dotx_{2_G}$. The same argument will show that $\dotx_{2_G} \subseteq \dotx_{1_G}$. \end{proof}

\subsection{The fundamental theorem of forcing with urelements}
\begin{theorem}\label{forcingpreservesZFCU}
Let $M$ be a countable transitive model of $\ZFUR$, $\P \in M$ be a forcing poset, and $G$ be an $M$-generic filter over $\P$. Then
\begin{enumerate}
    \item $M[G]$ is a countable transitive model of ZU;
    \item $M[G] \models$ AC if $M \models$ AC;
    \item $M[G] \models \ACA$ if  $M \models \ACA$;
    \item $M[G] \models$ Collection if $M \models$ Collection.
\end{enumerate}
\end{theorem}
\begin{proof}
The proof of (1) and (2) are the same as in Kunen \cite[Ch.VII]{kunen2014set} and hence omitted. (3) follows from Lemma \ref{McoversM[G]} that every set of urelements in $M[G]$ is covered by some set of urelements in $M$. 

For (4), suppose that $M[G] \models \forall v \in \dotw_G\ \exists y \varphi(v, y, \dot{u}_G)$ for some $\dotw_G$ and $\dot{u}_G$. In $M$, define 
\begin{align*}
    x = \{\langle \dotx, p \rangle \in (dom(\dotw) \cap M^\P) \times \P :  \exists \dot{y} \in M^\P p \forces \varphi(\dotx, \dot{y}, \dot{u}) \}.
\end{align*}
By Collection in $M$, there is a set of $\P$-names $v$ such that for every $\langle \dotx, p \rangle \in x$, there is a $\dot{y} \in v$ with $p \forces \varphi (\dotx, \dot{y}, \dot{u})$. Define $\dot{v}$ to be $v \times \{1_\P\}$. It is now routine to check that $M[G] \models \forall x \in \dot{w}_G \ \exists y \in \dot{v}_G \ \varphi(x, y, \dot{u}_G)$.
\end{proof}
A more difficult question is whether forcing preserves Replacement when the ground model $M$ does not satisfy Collection. When $M$ is a model of ZF, the standard argument for $M[G] \models $ Replacement appeals to Collection in $M$. But this move is not allowed when $M$ only satisfies $\ZFCUR$. A new argument is thus needed.


\begin{definition}\label{purification}
Let $\P$ be a forcing poset and $A$ be a set of urelements. For every urelement $a$, let $\overset{A}{a} = a$. For every $\dotx \in U^\P$, we define the $A$\textit{-purification of }$\dotx$, $\overset{A}{\dot{x}}$, as follows.
\begin{align*}
   \overset{A}{\dot{x}} = \{\langle \overset{A}{y}, p \rangle : \langle y , p \rangle \in \dot{x} \land ( y \in U^\P \lor y \in A) \}. 
\end{align*}
\end{definition}
\noindent That is, we get $\overset{A}{\dot{x}}$ by hereditarily throwing out the urelements used to build $\dot{x}$ that are not in $A$.
\begin{prop}
Let $\P$ be a forcing poset and $A$ be a set of urelements such that $ker(\P) \subseteq A$. For every $\dotx \in U^\P$, $\overset{A}{\dot{x}} \in U^\P$ and $ker(\overset{A}{\dot{x}}) \subseteq A$.
\end{prop}
\begin{proof}
By induction on the rank of $\dotx$. To show that $\overset{A}{\dot{x}}$ is always a $\P$-name, we only need to check the incompatibility condition in Defnition \ref{newpnames} holds. Suppose that $\<a, p>, \<y, q> \in \overset{A}{\dot{x}}$, where $a$ is an urelement and $y \neq a$. If $y$ is another urelement in $dom(\dotx)$, then $p$ and $q$ are incompatible; otherwise $y$ is some $\overset{A}{\dotz}$, where $\<\dotz, q> \in \dotx$ and $\dotz$ is a $\P$-name, then $p$ and $q$ are incompatible because no urelement is a $\P$-name. $ker(\overset{A}{\dot{x}}) \subseteq A$ because $ker(\overset{A}{\dot{x}})$ is contained in $\bigcup_{y \in dom(\dotx)}ker(\overset{A}{\dot{y}}) \cup ker(\P)$, which is a subset of $A$ by the induction hypothesis.
\end{proof}

\begin{lemma}\label{iso->id}
Let $\bar{x}$ and $\bar{y}$ be transitive sets. Every surjective $\in$-isomorphism from $\barx$ to $\bary$ that fixes $\emptyset$ and every urelement is the identity map.
\end{lemma}
\begin{proof}
Let $f$ be such surjective $\in$-ismorphism. We show that $x = f(x)$ for every set $x \in \barx$ by $\in$-induction. Fix some $x \in \bar{x}$. By the induction hypothesis, $x \subseteq f(x)$. If $y\in f(x)$, then $y = f(w)$ for some $w \in x$ so by the induction hypothesis $y \in x$. Therefore, $x = f(x)$.
\end{proof}


\begin{theorem}\label{forcingpreservesreplacement}
Let $M$ be a countable transitive model of $\ZFUR$, $\P \in M$ be a forcing poset and $G$ be $M$-generic over $\P$. Then $M[G] \models$ Replacement.
\end{theorem}
\begin{proof}
Suppose that for some $\dot{w}_G$ and $\dot{u}_G$ in $M[G]$, $M[G] \models \forall x \in \dot{w}_G\ \exists ! y \varphi(x, y, \dot{u}_G)$. Let $A = ker(\dot{w}) \cup ker(\P) \cup ker(\dot{u})$. By Theorem \ref{forcingpreservesZFCU}, we may assume $M$ does not satisfy Collection and hence has a proper class of urelements. 

\begin{lemma}\label{keylemmarep}
For every $\dot{v}_G \in \dot{w}_G$, there exist $p \in G$ and $\mu' \in M^{\P}$ such that $p \forces \varphi(\dot{v}, \mu', \dot{u})$ and $ker(\mu') \subseteq A$.
\end{lemma}
\begin{proof}
Fix a $\dot{v}_G \in \dot{w}_G$ for some $\dot{v} \in dom (\dot{w}) \cap M^\P$. Since $M[G] \models \exists ! y (\dot{v}_G, y, \dot{u}_G)$, there is a $\P$-name $\mu$ and a $p \in G$ such that $ p \forces \varphi (\dot{v}, \mu, \dot{u}) \land \forall z (\varphi (\dot{v}, z, \dot{u}) \rightarrow \mu = z)$.
\begin{claim}\label{claim1}
 For every $M$-generic filter $H$ over $\P$ such that $p \in H$, $ker(\mu_H) \subseteq A$.
\end{claim}
\begin{claimproof}
Suppose not. Then there is some $b \in ker(\mu_H) \setminus A$. Since $M$ has a proper class of urelements, there is some urelement $c \in M$ such that $c \notin A \cup ker(\mu)$. In $M$, let $\pi$ be the automorphism that only swaps $b$ and $c$. Since $\pi$ point-wise fixes $A$, it follows that 
\begin{align*}
    p \forces \varphi (\dot{v}, \pi \mu, \dot{u}) \land \forall z (\varphi (\dot{v}, z, \dot{u}) \rightarrow \pi \mu = z).
\end{align*}
Thus, $M[H] \models \mu_H = (\pi\mu)_H$. Since $b \in ker(\mu_H)$, $\pi b  \in ker(\pi\mu_H)$($\pi$ is viewed as an automorphism of the background universe); but $\pi b = c \notin ker(\mu)$ and $ker(\mu_H) \subseteq ker(\mu)$, so $\pi b \notin ker(\mu_H)$, which is a contradiction.
\end{claimproof}


\noindent Note that we cannot hope to show that $ker(\mu) \subseteq A$ in general. For if $\mu^*$ is some $\P$-name such that  $\mu^* = \mu \cup \{\langle \{\<b, 1_\P>\}, q \rangle\}$, where $b$ is an urelement not in $A$ and $q$ is not compatible with $p$, we would still have $p \forces \mu = \mu^*$.


\begin{claim}\label{claim2}
Let $H$ be an $M$-generic filter over $\P$ such that $p \in H$. For every $\dot{x}, \dot{y} \in M^{\P}$, if $\dot{x}_H, \dot{y}_H \in trc(\{\mu_H\})$, then $\dot{x}_H = \dot{y}_H$ if and only if $(\overset{A}{\dot{x})}_H = (\overset{A}{\dot{y}})_H$.
\end{claim}
\begin{claimproof}
If $\dotx_H = \doty_H = a$ for some urelement $a$, then by Claim \ref{claim1} $a \in A$. Then it is easy to check that $(\overset{A}{\dot{y})}_H = (\overset{A}{\dot{x}})_H = a$. If $(\overset{A}{\dot{y})}_H = (\overset{A}{\dot{x}})_H = b$ for some urelement $b$, then $b \in A$ and it follows that $\dotx_H = \doty_H = b$.


So suppose $\dotx_H = \doty_H$ are sets in $trc(\{\mu_H\})$ and the claim holds for every $\dot{z} \in dom(\dot{x}) \cup dom(\dot{y})$. Clearly, $(\overset{A}{\dot{x})}_H$ and $(\overset{A}{\dot{y}})_H$ must also be sets. If $\overset{A}{\dot{z}}_H \in \overset{A}{\dot{x}}_H$ for some $\dotz \in M^\P \cap dom(\dotx)$, we have $\dot{z}_H \in \dot{y}_H = \dot{x}_H$. So there is some $\dot{w} \in M^\P \cap dom(\doty)$ such that $\dot{w}_H = \dot{z}_H$. $\dot{z}_H \in trc(\{\mu_H\})$ so by the induction hypothesis $\overset{A}{\dot{z}}_H = \overset{A}{\dot{w}}_H \in (\overset{A}{\dot{y}})_H$. This shows that $\overset{A}{\dot{x}}_H \subseteq \overset{A}{\dot{y}}_H$, and we will have $\overset{A}{\dot{x}}_H = \overset{A}{\dot{y}}_H$ by the same argument.

Now suppose that $\dotx_H, \doty_H \in trc(\{\mu_H\})$ and $\overset{A}{\dot{x}}_H = \overset{A}{\dot{y}}_H$ are sets. Then $\dotx_H$ and $\doty_H$ must be sets. For if, say, $\dotx_H = a$ for some urelement $a$, then $a \in A$ by Claim \ref{claim1}, which implies that $\overset{A}{\dot{x}}_H = a$. Let $\dot{z}_H \in \dot{x}_H$ for some $\dotz \in M^\P \cap dom(\dotx)$. Then $\overset{A}{\dot{z}}_H \in \overset{A}{\dot{y}}_H$ and so $\overset{A}{\dot{z}}_H = \overset{A}{\dot{w}}_H$ for some $\dot{w}_H \in \dot{y}_H$. By the induction hypothesis, it follows that $\dot{z}_H = \dot{w}_H$. This shows that $\dot{x}_H \subseteq \dot{y}_H$ and consequently, $\dot{x}_H = \dot{y}_H$.
\end{claimproof}



\begin{claim}\label{claim3}
$p \forces \overset{A}{\mu} = \mu$.
\end{claim}
\begin{claimproof}
 Let $H$ be an $M$-generic filter on $\P$ that contains $p$. We show that $\overset{A}{\mu}_H = \mu_H$. Let $f$ be the function on $trc(\{\mu_H\})$ that sends every $\dot{y}_H$ to $\overset{A}{\dot{y}}_H$, which is is well-defined by Claim \ref{claim2}. By Lemma \ref{iso->id}, it suffices to show that $f$ maps $trc(\{\mu_H\})$ onto $trc(\{\overset{A}{\mu}_H\})$, preserves $\in$ and fixes all the urelements.

\textit{ $f$ preserves $\in$}. Consider any $\dot{y}{_H}, \dot{x}{_H} \in trc(\{\mu_H\})$. Suppose that $\dot{y}{_H} \in \dot{x}{_H}$. Then $\dot{y}{_H} = \dot{z}_H$ for some $\dot{z} \in M^\P \cap dom(\dot{x})$ so $ \overset{A}{\dot{z}}_H \in \overset{A}{\dot{x}}_H$; by Claim \ref{claim2}, it follows that $\overset{A}{\dot{y}}_H = \overset{A}{\dot{z}}_H \in \overset{A}{\dot{x}}_H$. Suppose that $\overset{A}{\dot{y}}_H \in \overset{A}{\dot{x}}_H$. Then $\overset{A}{\dot{y}}_H = \overset{A}{\dot{z}}_H$ for some $\dot{z}_H \in \dot{x}_{H}$ so $\dot{y}{_H} = \dot{z}_H \in \dot{x}_{H}$ by Claim \ref{claim2} again.

\textit{$f$ maps $trc(\{\mu_H\})$ onto $trc(\{\overset{A}{\mu}_H\})$}. If $\dot{y}_H \in trc(\{\mu_H\})$, then $\dot{y}_H \in \dot{y}_1{_H} \in ... \in \dot{y}_n{_H} \in \mu_H$ for some $n$. Since $f$ is $\in$-preserving, it follows that $\overset{A}{\dot{y}}_H \in \overset{A}{\dot{y}_1}_H \in ... \in \overset{A}{\dot{y}_n}_H \in \overset{A}{\mu}_H$ and hence $\overset{A}{\dot{y}}_H \in trc(\{{\overset{A}{\mu} }_H\})$. To see it is onto, let $x \in x_1 \in ... \in x_n \in \overset{A}{\mu}_H$. Then $x = \overset{A}{\dot{y}}_H \in \overset{A}{\dot{y}_1}_H  \in ... \in \overset{A}{\dot{y}_n}_H \in \overset{A}{\mu}_H$, but then $\dot{y}_H \in \dot{y}_1{_H} \in ... \in \dot{y}_n{_H} \in \mu_H$ and hence  $\dot{y}_H \in trc(\{\mu_H\})$.

\textit{$f$ fixes all the urelements in  $trc(\{\mu_H\})$.} Suppose $\dotx_H = a \in trc(\{\mu_H\})$ for some urelement $a$. Then by Claim \ref{claim1}, $a \in A$ and hence $\overset{A}{\dot{x}}_H = a$. \end{claimproof}

\noindent The lemma is now proved by letting $\mu'$ be $\overset{A}{\mu}$.
\end{proof}
Now in M, we define
\begin{align*}
    \bar{w} = \{\langle \dot{v}, p\rangle \in (dom(\dot{w}) \cap M^\P) \times \P : \exists \mu \in M^{\P} (ker(u) \subseteq A \land p \forces \varphi(\dot{v}, \mu, \dot{u})) \}.
\end{align*}
For every $\langle \dot{v}, p \rangle \in \bar{w}$, let $\alpha_{ \dot{v}, p }$ be the least $\alpha$ such that there is some $\mu \in V_\alpha(A) \cap M^{\P}$ such that $p \forces \varphi(\dot{v}, \mu, \dot{u})$. Let $\beta = Sup_{\langle \dot{v}, p \rangle \in \bar{w}} \alpha_{\dot{v}, p}$ and set $\rho = (V_\beta(A) \cap M^{\P}) \times \{1_\P\}$. It remains to show that $M[G] \models \forall x \in \dot{w}_G\ \exists y \in \rho_G\ \varphi(x, y, \dot{u}_G)$. Let $\dot{v}_G \in \dot{w}_G$. By Lemma \ref{keylemmarep}, there is some $p \in G$ such that $\langle \dot{v}, p \rangle \in \bar{w}$. So there is some  $\P$-name $\mu \in dom(\rho)$ such that $p \forces \varphi(\dot{v}, \mu, \dot{u})$. Thus, $M[G] \models \varphi(\dot{v}_G, \mu_G, \dot{u}_G)$ and $\mu_G \in \rho_G$.\end{proof}









\begin{theorem}[The Fundamental Theorem of Forcing with Urelements]\label{fundamentalthmofforcing}
Let $M$ be a countable transitive model of $\ZFUR$, $\P \in M$ be a forcing poset and $G$ be an $M$-generic fitler over $\P$. Then
\begin{enumerate}
    \item $M[G] \models$ $\ZFUR$.
    \item $M[G] \models \ZFCUR$ if $M\models$ $\ZFCUR$.
    \item $M[G] \models$ ZFCU if  $M \models $ ZFCU. 
    \item $M[G] \models$ Plenitude if $M \models $ Plenitude.
    \item $M[G] \models $ Duplication if $M \models $ Duplication.
    \item $M[G] \models$ Plenitude$^+$ if $M \models $ Plenitude$^+$.
    \item $M[G] \models$ Tail if $M \models$ Tail.
    \item $M[G] \models $ DC$_{<Ord}$ if $M \models $ DC$_{<Ord}$.
    \item $M[G] \models$ RP$^-$ if $M \models$ RP$^-$.
    \item $M[G] \models$ RP if $M \models$ RP.
    \item $M[G] \models $ Closure if $M \models $ Closure + $\ACA$.
\end{enumerate}
\end{theorem}
\begin{proof} 
(1), (2), and (3) are Theorem \ref{forcingpreservesZFCU} and Theorem \ref{forcingpreservesreplacement}.

(4) is clear because $M \subseteq M[G]$ and $M[G]$ and $M$ has the same ordinals. 

(5) follows easily from Lemma \ref{McoversM[G]}.

For (6), suppose that $M \models $ Plenitude$^+$ and $\dotx_G \in M[G]$. Then in $M$, there is a bijection $f$ from $dom(\dotx)$ to a set of urelements. Using $f$ we can code an injective function in $M[G]$ from $\dotx_G$ to $\A$. So Plenitude$^+$ holds in $M[G]$.

For (7), suppose that $M\models$ Tail and $A \subseteq \A$ is in $M[G]$. Then let $A' \in M$ be a set of urelements containing $A$ and $B'$ be a tail of $A'$. It is not hard to check that $A'\setminus A \cup B'$ is a tail of $A$ in $M[G]$.  



(8) Suppose that $M \models $ DC$_{<Ord}$. It is a standard result that $\forall \kappa \text{DC}_\kappa$ implies AC, so $M \models$ AC and hence $M[G] \models$ AC. Since  DC$_{<Ord}$ implies that either $\A$ is a set, or Plenitude holds, it follows that $M[G] \models (\A \text{ is a set} \lor \text{Plenitude})$ by (4). By Theorem \ref{maintheorem1}, we have $M[G] \models$ DC$_{<Ord}$. 

(9) Suppose that $M \models$ RP$^-$ and $M[G] \models \varphi(\dotx{_1}_G, ..., \dotx{_n}_G)$. Let $p\in G$ be such that $p \forces \varphi(\dotx_1, ..., \dotx_n)$. By RP$^-$ in $M$, there is a transitive set $m$ containing $\P$ and $\dotx_1, ..., \dotx_n$ such that $(p \forces \varphi(\dotx_1, ..., \dotx_n))^m$ and $m$ satisfies some finite fragment of $\ZFUR$ that suffices for the construction of $\P$-names inside $m$. It then follows that $m[G] \models \varphi(\dotx{_1}_G, ..., \dotx{_n}_G)$. $m[G]$ is a transitive set in $M[G]$ because $\dot{m}= \{\<\doty, 1_\P> : \doty \in m \cap M^\P\}$ is a $\P$-name for $m[G]$. Therefore, $M[G] \models$ RP$^-$.

(10) Suppose that $M \models$RP. Given a formula $\varphi (v_1, ..., v_n)$ and some $\dotu_G \in M[G]$, let $\psi (p, \P, v_1, ..., v_n)$ be the formula asserting that $p$ is a forcing condition in $\P$ and $p \forces \varphi(v_1, ..., v_n)$ for $\P$-names $v_1, ..., v_n$. By RP in $M$, there will a transitive set $m$ containing $\{\P, \dotu \}$ that reflects $\psi$ and satisfies some finite fragment of $\ZFUR$ sufficient for forcing. Then as in the last paragraph, $m[G]$ is a transitive set containing $\dotu_G$ in $M[G]$. If $M[G] \models \varphi (\dotx{_1}_G, ..., \dotx{_n}_G)$ for some $\dotx{_1}_G, ..., \dotx{_n}_G$ in $m[G]$, then there will be $p \in G$ such that $(p \forces \varphi(\dotx{_1}, ..., \dotx{_n}))^m$, and so $m[G] \models \varphi (\dotx{_1}_G, ..., \dotx{_n}_G)$. And if $M[G] \models \varphi (\dotx{_1}_G, ..., \dotx{_n}_G)^{m[G]}$, then there is some $p \in G$ such that $(p\forces \varphi(\dotx{_1}, ..., \dotx{_n}))^m$, so $(p\forces \varphi(\dotx{_1}, ..., \dotx{_n}))^M $ and hence $M[G] \models \varphi (\dotx{_1}_G, ..., \dotx{_n}_G)$. This shows that $M[G] \models $ RP.

(11)  Suppose that in $M$, Closure holds and every set of urelements is well-orderable. Let $x \in M[G]$ be a set of realized cardinals whose supermum is some limit cardinal $\lambda$. It suffices to show that in $M$, every cardinal $\kappa < \lambda$ is realized. Since $\lambda$ is a limit cardinal in $M$, for every $\kappa < \lambda$, there is some cardinal $\kappa'$ in $M[G]$ with $\kappa < \kappa' < \lambda$ that is realized by some $B\subseteq \A$ in $M[G]$; so by Lemma \ref{McoversM[G]} $B \subseteq B'$ for some set of urelements $B' \in M$. Then $(\kappa' \leq |B'|)^M$ since otherwise it would contradict the fact that $\kappa'$ is a cardinal in $M[G]$. Thus, $\kappa$ is realized in $M$.
\end{proof}
\noindent It is unclear if forcing preserves Closure if the ground model does not have $\ACA$.














\subsection{Destroying the $\DCK$ and recovering Collection}\label{dck&collection}
I now move on to the preservation of the $\DCK$. A forcing poset $\P$ is $\kappa$-closed if in $\P$ every infinite descending chain of length less than $\kappa$ has a lowerbound. It is a text-book result that $\kappa$-closed forcing posets preserve cardinalities $\leq \kappa$.
\begin{theorem}\label{kclosedforcingpreservedck}
Let $M$ be a countable transitive model of $\ZFCUR$ + DC$_\kappa$-scheme, $\P \in M$ be such that $(\P \text{ is } \kappa^+\text{-closed})^M$ and $G$ be an $M$-generic fitler over $\P$. Then $M[G] \models$ $\ZFCUR$ + DC$_\kappa$-scheme.
\end{theorem}
\begin{proof}

We first make some definitions. For every $\alpha$-sequence $s$ of $\P$-names, let $\dot{s}^{(\alpha)}$ denote the canonical $\P$-name such that $\dot{s}^{(\alpha)}_G$ is an $\alpha$-sequence in $M[G]$ with $\dot{s}^{(\alpha)}_G(\eta) = s(\eta)_G$ for all $\eta < \alpha$. Given a $p \in \P$ and a suitable formula $\varphi$, a $\kappa$-sequence of the form $\<\<p_\alpha, \dot{x}_\alpha> : \alpha < \kappa>$, where $\<p_\alpha, \dot{x}_\alpha> \in \P \times M^\P$, is said to be a \textit{ $\varphi$-chain below} $p$ if $\<p_\alpha : \alpha < \kappa>$ is a descending chain below $p$ and for every $\alpha < \kappa$, $p_\alpha \forces \varphi(\dot{s}^{(\alpha)}, \dot{x}_{\alpha+1})$ where $s = \<\dot{x}_\eta : \eta < \alpha>$.

Suppose that $M[G] \models \forall x \exists y \varphi(x. y, u)$. There is some $p \in G$ such that $p \forces \forall x \exists y \varphi (x, y, \dot{u})$. Let $D$ be the set of forcing conditions that are a lower bound of some $\varphi$-chain below $p$. We claim that $D$ is dense below $p$. If $r \leq p$, let $\psi(x, y, r, \P)$ be the formula defined as follows.
\begin{itemize}
    \item [] $\psi(x, y, \P, \dot{u}) =_{df}$ if $x = \<\<p_\eta, \dot{x}_\eta> : \eta < \alpha>$, where  $\<p_\eta : \eta < \alpha>$ is a descending chain of length $\alpha$ for some $\alpha < \kappa$, then $y = \<q, \dot{x}> \in \P \times M^\P$ such that $q$ bounds $\<p_\eta : \eta < \alpha>$ and $q \forces \varphi(\dot{s}^{(\alpha)}, \dot{x}, \dot{u})$.
\end{itemize}
Let $\P\downarrow r$ denote the set of conditions in $\P$ below $r$. In $M$, for every $x \in (\P \downarrow r \times M^\P)^{<\kappa}$, since $\P$ is $\kappa$-closed, there is some $y \in \P \downarrow r \times M^\P$ such that $\psi(x, y, \P. \dot{u})$. By DC$_\kappa$-scheme in $M$, there exists a $\varphi$-chain $\<\<p_\alpha, \dot{x}_\alpha> : \alpha < \kappa>$, where $\<p_\alpha : \alpha < \kappa>$ is below $r$ and hence below $p$. $\P$ is $\kappa^+$-closed, so there is some $q$ that bounds this $\varphi$-chain below $p$. Thus, $D$ is dense below $p$. It then follows that there is $q \in G$ that bounds a $\varphi$-chain, $\<\<p_\alpha, \dot{x}_\alpha> : \alpha < \kappa>$, below $p$. Let $s = \<\dot{x}_\alpha : \alpha < \kappa>$ and $f = \dot{s}^{(\kappa)}_G$. $f$ is then a $\kappa$-sequence in $M[G]$ and as $\P$ is $\kappa$-closed, $\kappa$ is the same cardinal in $M[G]$ as in $M$. Moreover, $M[G] \models \varphi (f\restriction \alpha, f(\alpha), u)$ for all $\alpha < \kappa$ because $q \forces \varphi(\dot{s}^{(\alpha)}, \dot{x}_\alpha, \dot{u})$. 
\end{proof}
For any infinite cardinals $\kappa$ and $\lambda$ with $\kappa < \lambda$, $\textup{Col}(\kappa, \lambda)$ is the forcing poset consisting of all partial functions from $\kappa$ to $\lambda$ of size less than $\kappa$ (ordered by reverse inclusion). Forcing with $\textup{Col}(\kappa, \lambda)$ collapses $\lambda$ to $\kappa$.
\begin{theorem}
Forcing with urelements does not preserve the DC$_{\omega_1}$-scheme in general even if the ground model satisfies ZFCU. 
\end{theorem}
\begin{proof}
Consider a countable transitive model $M$ of $\ZFCUR$ where every set of urelements has tail cardinal $\omega_1$. By Theorem \ref{tail->collection} and Lemma \ref{Tailkappa->DCkappa}, both Collection and the DC$_{\omega_1}$-scheme hold in $M$.  In $M$, let $\P = \textup{Col}(\omega, \omega_1)$ and $G$ be $M$-generic over $\P$. Then in $M[G]$, every set of urelements is countable, because every $A \in M[G]$ is a subset of some $A' \in M$ such that $|A'| \leq \omega_1{^M}$ but $\omega_1{^M}$ is collapsed to $\omega$ in $M[G]$. As a result, every set of urelements will have tail cardinal $\omega$. By an usual argument as in Theorem \ref{zfcurindependece}, this implies that the DC$_{\omega_1}$-scheme fails in $M[G]$.
\end{proof}
Note that since ZFCU proves the DC$_\omega$-scheme (Theorem \ref{maintheorem1}), forcing over ZFCU preserves DC$_\omega$-scheme as it preserves ZFCU.
\begin{question}
Does forcing over $\ZFCUR$ preserve the DC$_\omega$-scheme?
\end{question}



\begin{lemma}\label{recovercollection}
Let $M$ be a countable transitive model of $\ZFCUR$ where for every set of urelements, there is another infinite disjoint set of urelements. Then there is a forcing extension of $M$ which satisfies ZFCU.
\end{lemma}
\begin{proof}
By Theorem \ref{maintheorem1} and \ref{forcingpreservesZFCU}, we may assume that in $M$, there is a least cardinal $\kappa$ not realized since otherwise Collection holds in every forcing extension of $M$. Let $G$ be an $M$-generic filter over $\textup{Col}(\omega, \kappa)$. As $\kappa$ is collapsed to $\omega$ in $M[G]$, every set of urelements in $M[G]$ is countable. If $A$ is  a set of urelements in $M[G]$, let $A' \in M$ be such that $A \subseteq A'$. By assumption, there is another infinite $B \in M$ disjoint from $A'$. Since $B$ has size $\omega$ in $M[G]$, Tail holds in $M[G]$ and hence Collection holds in $M[G]$ by Lemma \ref{tail->collection}. \end{proof}
The next theorem says that ZFCU (in particular, Collection) is necessarily ``forceble'' when the ground model satisfies $\ZFCUR$ + DC$_\omega$-scheme.
\begin{theorem}
If $M$ is countable transitive model of $\ZFCUR$ + DC$_\omega$-scheme and $M[G]$ is a forcing extension of $M$, then $M[G]$ has a forcing extension that satisfies ZFCU.
\end{theorem}
\begin{proof}
First, we may assume that in $M$ $\A$ is a proper class. By the DC$_\omega$-scheme in $M$, for every set of urelements in $M$, there is an infinite set of urelements disjoint from it. But notice that this fact is preserved by forcing by Lemma \ref{McoversM[G]}. So we can apply Lemma \ref{recovercollection} to $M[G]$.
\end{proof}
\noindent Not every model of $\ZFCUR$ has a forcing extension which satisfies ZFCU. For example, if in $M$ every set of urelements is finite but there is a proper class of them, then this will remain the case in every forcing extension of $M$.

\subsection{Ground model definability}
Laver \cite{Laver2007-LAVCVL} and Woodin \cite{woodin2011continuum} proved independently the ground model definability for ZFC: every model of ZFC is definable in its forcing extensions with parameters. Here I first show that the ground model definability fails badly when the gound model contains a proper class of urelements. I then generalize Laver's argument, which is also attributed to Hamkins \cite{Hamkins2003ExtensionsWT}, to consider when ground model definability will hold.

For any infinite set of $x \in M$, $\textup{Fn}(x, 2)$ is the forcing poset consisting of all finite partial functions from $x$ to $2$ ordered by reversed inclusion. If $G$ is an $M$-generic filter over $\textup{Fn}(x, 2)$, then for every set $y \in M$ that is equinumerous with $x$, $M[G]$ contains a new subset of $y$.
\begin{theorem}
Let $M$ be a countable transitive model of $\ZFUR$.
\begin{enumerate}
    \item If $M \models$ DC$_\omega$-scheme + ``$\A$ is a proper class'', then $M$ has a forcing extension in which $M$ is not definable with parameters;
    \item if $M \models $ Plentitude + AC, then $M$ is not definable in any of its non-trivial forcing extensions. 
\end{enumerate}
\end{theorem}
\begin{proof}
(1) Suppose that $M \models$ DC$_\omega$-scheme + ``$\A$ is a proper class''. Let $\P \in M$ be $\textup{Fn}(\omega, 2)$ and $G$ be an $M$-generic filter over $\P$. Suppose \textit{for reductio} that $M$ is definable in $M[G]$ with a parameter $\dot{u}_G \in M[G]$ such that $M = \{ x \in M[G] : M[G] \models \varphi (x,\dot{u}_G)\}.$ Let $B' \in M$ be an infinite set of urelements disjoint from $ker(\dot{u})$, which exists by the DC$_\omega$-scheme. Since DC$_\omega$ implies that $B'$ must have an $\omega$-subset, $M[G]$ contains a new countable subset $B$ of $B'$ which is not in $M$. Fix another $\omega$-set of urelements $C \in M$ disjoint from $ker(\dot{u}) \cup B'$. In $M[G]$, there will be an automorphism that swaps $C$ and $B$ while point-wise fixing $ker(\dot{u})$. Since $M[G] \models \neg \varphi (B,\dot{u}_G)$ and $ker(\dot{u}_G) \subseteq ker(\dot{u})$, it follows that $M[G] \models \neg \varphi (C,\dot{u}_G)$ and hence $C \notin M$, which is a contradiction. 


(2) Suppose that $M \models$ Plentitude$^+$ and consider any $M[G]$ such that $M \subsetneq M[G]$. First observe that there must be some set of urelements $B$ such that $B \in M[G] \setminus M$. Fix some $\dot{x}_G \in M[G] \setminus M$ of the least rank so that $\dot{x}_G \subseteq M$. Let $A = ker(\dot{x})$. It follows that $\dot{x}_G \subseteq V_\alpha (A)^M$ for some $\alpha$. By Plenitude and AC in $M$, there is a bijection $f$ from $V_\alpha (A)^M$ to a set of urelements. $f[\dot{x}_G]$ will then be a new set of urelements in $M[G]$.

For \textit{reductio}, suppose that $M = \{ x \in M[G] : M[G] \models \varphi(x,  \dot{u}_G)\}$ for some formula $\varphi$ with parameter $ \dot{u}_G$. Fix some $B \in M[G]\setminus M$ and $B' \in M$ such that $B \subseteq B'$. In $M$, $B'$ has a duplicate $E$ that is disjoint from $ker(\dot{u})$. Then $E$ has a new subset $D$ in $M[G]$ that is disjoint from $ker(\dot{u})$. By AC and Plenitude in $M$, we can again find a duplicate $C \in M$ of $D$ that is disjoint from $ker(\dot{u})$. So there will be an automorphism in $M[G]$ that swaps $C$ and $D$ while point-wise fixing $ker(\dot{u})$. As $M[G] \models \neg \varphi (D, \dot{u}_G)$, it follows that $M[G] \models \neg \varphi(C, \dot{u}_G)$ and hence $C \notin M$, which is a contradiction. %Can we just use Plenitude+ in the ground model?
\end{proof}

The ground model definability holds for models of ZFCU when there is only a set of urelements. This is proved by making some adjustments to the ZFC arguments as in \cite{Laver2007-LAVCVL} and \cite{Reitz2007-JONTGA}. I shall include a proof of this for completeness.
\begin{definition}
 Let $\mathcal{L}_\delta$ be the language extending the language of urelement set theory with a constant symbol $\delta$. ZFCU$_\delta$ is the theory in $\mathcal{L}_\delta$ consisting of ZU, AC, the axiom that $\delta$ is a regular cardinal, $\leq \delta$-Replacement, which states that Replacement holds for every definable function with a domain of size less than $\delta$, and the following axiom.
\begin{itemize}
    \item [](*) Every well-ordering is isomorphic to $\langle \alpha, \in \rangle$ for some ordinal $\alpha$.
\end{itemize}
\end{definition}
\noindent Given a set of urelements $A$, a regular cardinal $\delta$ and a Beth-fixed point $\lambda$ with cf$(\lambda) > \delta$, $V_\lambda(A) \models$ ZFCU$_\delta$.
\begin{lemma}\label{M=M'}
Let $M$ and $M'$ be two transitive models of ZFCU$_\delta$. If $M$ and $M'$ have the same sets of ordinals and the same sets of urelements, then $M = M'$.
\end{lemma}
\begin{proof}
Fix any set $x \in M$. Since $M \models$ZFCU$_\delta$, every set in $M$ has a transitive closure and hence a kernel. By AC in $M$, there is some $\alpha$ and $R \subseteq \alpha \times \alpha$ such that $\langle \alpha, R \rangle$ is isomorphic to $\langle trc(\{x\}), \in \rangle$. Let $W$ well-order $(\alpha +1  )\times (\alpha + 1)$ by G\"odel's pairing function, which will be isomorphic to some ordinal $\beta$ by (*). Let $G$ be the injective function on $\alpha \times \alpha$ such that $G(\eta, \zeta)$ is the order-type of the initial segment of $W$ up to $\langle \eta, \zeta \rangle$. Now we can code $\langle \alpha, R \rangle$ into a set of ordinals $\overline{x} = \{G(\alpha, G(\eta, \zeta)) : \langle \eta, \zeta \rangle \in R\}$. Then $x$ is coded by some $\langle \overline{x}, ker(x) \rangle$, which is also in $M'$ by assumption. Since the definition of $G$ is absolute, we can then decode $\langle \overline{x}, ker(x) \rangle$ in $M'$ and hence $x \in M'$.\end{proof}



\begin{definition}[Hamkins]
Let $N \subseteq M$ be transitive models of $\ZFCUR$ and $\delta$ is a regular cardinal in $M$. 
\begin{enumerate}
    \item ( $\<N, M>$ has the $\delta$-cover property if and only if for each $ x \in M$ with $x \subseteq N$ and $(|x| < \delta)^M$, there is a $y \in N$ such that $x \subseteq y$ and $(|y| < \delta)^N$.
    \item $\<N, M>$ has the $\delta$-approximation property if and only if for each $y \in M$ with $y \subseteq N$, if $y \cap z \in N$ for every $z \in N$ with $(|z|<\delta)^N$, then $y \in N$.
\end{enumerate}
\end{definition}
\begin{theorem}\label{Hamkinsuniquenessthm}
Let $N, N' \subseteq M$ be transitive models of ZFCU$_\delta$ for some regular cardinal $\delta$ in $M$. If both $\langle N, M\rangle$ and $\langle N', M\rangle$ have the $\delta$-cover and $\delta$-approximation properties and $(\delta^+)^N = (\delta^+)^{N'} = (\delta^+)^M$ and $P(\delta)^N = P(\delta)^{N'}$, then $N$ and $N'$ have the same sets of ordinals.
\end{theorem}
\begin{proof}
The proof is the same as in Laver \cite[Theorem 1]{Laver2007-LAVCVL}, attributed to Hamkins.
\end{proof}
\noindent When $N$ and $N'$ are transitive models of ZFC$_\delta$, we will in fact have $N' = N$ as every pure set can be coded by a set of ordinals. But it is possible for $N$ and $N'$ to have the same sets of ordinals but different urelements (or sets of urelements).
\begin{lemma}\label{M[G]coversM}
Let $M$ be a transitive model of $\ZFCUR$, $\P \in M$ be such that $(|\P| < \delta)^M$ for some regular cardinal $\delta$ in $M$ and $G$ be an $M$-generic filter over $\P$. Then $\langle M, M[G] \rangle$ has the $\delta$-cover and $\delta$-approximation properties. Moreover, for any set of urelemetns $A \in M$, there are unboundedly many Beth-fixed points $\lambda$ such that $\langle V_\lambda(A)^M, V_\lambda(A)^{M[G]} \rangle$ has $\delta$-cover and $\delta$-approximation properties.
\end{lemma}
\begin{proof}
For the $\delta$-cover property, suppose that $x \in M[G]$, $x \subseteq M$ and $(|x| < \delta)^{M[G]}$. Then $ker(x) \subseteq A$ for some $A \in M$. So $x \subseteq V_\alpha (A)^M$ for some $\alpha$. In $M[G]$, fix a bijection $f$ from $\kappa$ to $x$ for some $\kappa < \delta$. Since $\P \text{ has } \delta\text{-}c.c.$,\footnote{A forcing poset $\P$ has $\kappa\textup{-}c.c.$ if every antichain in $\P$ has size less than $\kappa$.} by the standard argument there is some function $g \in M$ from $\kappa$ to $P(V_\alpha (A)^M)$ such that $f(\beta) \in g(\beta)$ and $|g(\beta)| < \delta$ for every $\beta < \kappa$. Then in $M$, $\bigcup_{\beta < \kappa} F(\beta)$ has size $< \delta$ and covers $x$. 

For the $\delta$-approximation property, by following Laver's argument in \cite{Laver2007-LAVCVL} we can first show that the $\delta$-approximation property holds sets of ordinals in $M[G]$. Fix an $x \in M[G] \setminus M$ with $x \subseteq M$. Since $x$ is a subset of some $V_\lambda(A)^M$ for some $A \in M$, in $M$ we can fix a bijection $f$ between some ordinal $\alpha$ and $V_\lambda(A)^M$. $f[x]$ is then a set of ordinals not in $M$, so there is a set of ordinals $z \in M$ with $(|z| < \delta)^M$ such that $f[x] \cap  z \notin M$. Using $f$ and $z$ we can find a $y \in M$ such that $(|y| < \delta)^M$ and $y\cap x \notin M$. Hence, $\langle M, M[G] \rangle$ has the $\delta$-approximation property.


Moreover, fix some $\alpha$ and let $\lambda > \alpha$ be a Beth-fixed point with $ \textup{cf} (\lambda) > \delta$. We show that $ V_\lambda(A)^M$ and $V_\lambda(A)^{M[G]}$ have the $\delta$-cover and $\delta$-approximation properties. For the $\delta$-cover property, if $x \in V_\lambda(A)^{M[G]}$ is such that $(|x| < \delta)^{M[G]}$ and $x \subseteq  V_\lambda(A)^M$, then $x \in V_\beta (A)^{M[G]}$ for some $\beta < \lambda$ and so  $x \subseteq V_\beta (A)^M$. By the $\delta$-cover property of $\langle M, M[G] \rangle$, $x$ is covered by some $y \in M$ with $(|y| < \delta)^M$. As a result, $y \cap V_\beta (A)^M$ covers $x$ in $V_\lambda(A)^M$.

For the $\delta$-approximation property, let $x \in V_\lambda(A)^{M[G]}$ be such that $x \subseteq  V_\lambda(A)^M$ and suppose that for every $y \in  V_\lambda(A)^M$  with $(|y| < \delta)^M$, $x \cap y \in V_\lambda(A)^M$. For any $\overline{y} \in M$ with cardinality less than $\delta$, as cf$(\lambda) > \delta$, $\overline{y} \cap V_\lambda(A)^M \in V_\lambda(A)^M$; so $\overline{y} \cap x = \overline{y} \cap V_\lambda(A)^M \cap x$ is in $ V_\lambda(A)^M$ and hence in $M$. By the $\delta$-approximation property of $\langle M, M[G] \rangle$, this implies that $x \in M$. Since $x \subseteq  V_\beta (A)^M$ for some $\beta < \lambda$, $x \in V_\lambda(A)^M$. \end{proof}



\begin{theorem}
Let $M$ be a transitive model of ZFCU + ``$\A$ is a set'', $\P \in M$ and $G$ be an $M$-generic filter over $\P$. $M$ is definable in $M[G]$ with parameters.
\end{theorem}
\begin{proof}
In $M[G]$, let $\A$ be the set of all urelements, $\gamma = |\P|$ and $\delta$ be $\gamma^+$. For any set $N$ and cardinal $\lambda$ in $M[G]$, $N$ is said to be is \textit{good for} $\lambda$ if 
\begin{itemize}
    \item [] (i) $N$ is transitive and $\langle N, \in \rangle \models$ZFCU$_\delta$;
    \item [] (ii) $\delta^+{^N} = \delta^+{^{M[G]}}$ and $P(\delta)^N = P(\delta)^M$; %It is important that here N agrees with "M" on the powerset of delta%
    \item [] (iii) $\lambda = Ord \cap N$;
    \item [] (iv) $\langle N, V_\lambda (\A)^{M[G]} \rangle \text{ has the } \delta \text{-cover and } \delta \text{-approximation properties}$.
\end{itemize}
For any cardinal $\lambda$, we say $\lambda$ is \textit{tall} if $\lambda$ is a Beth-fixed point with cf$(\lambda) > \delta$.
Let $\varphi (x, P(\A)^M, P(\delta)^{M})$ be the following formula with parameters $P(\A)^M$ and $P(\delta)^{M}$.
\begin{itemize}
    \item [] $\varphi (x, P(\A)^M, P(\delta)^{M})$ if and only if $\exists N, \lambda ( x \in N \land N \text{ is good for } \lambda \land \lambda \text{ is tall} \land P(\A)^N = P(\A)^M )$ 
\end{itemize}
We claim that $M = \{x \in M[G] : M[G] \models \varphi (x, P(\A)^M, P(\delta)^{M})\}$. 

If $x \in M$, then $x$ is in some tall $\lambda$ and so $V_\lambda(\A)^M \models$ZFCU$_\delta$. $\langle V_\lambda(\A)^M , V_\lambda (\A)^{M[G]} \rangle$ has the $\delta$-cover and $\delta$-approximation properties by Theorem \ref{M[G]coversM}, and $P(\A) \cap V_\lambda(\A)^M = P(\A)^M$. Therefore,  $M[G] \models \varphi (x, P(\A)^M, P(\delta)^{M})$.

Suppose that $M[G] \models \varphi (x, P(\A)^M, P(\delta)^{M})$. Then $x$ is in some $N$ such that $N$ is good for $\lambda$, so $N$ and $V_\lambda (\A)^{M[G]}$ have the $\delta$-cover and $\delta$-approximation properties. Since $\lambda$ is tall, $V_\lambda(\A)^M$ and $V_\lambda (\A)^{M[G]}$ satisfy ZFCU$_\delta$ and have the $\delta$-cover and $\delta$-approximation properties. $V_\lambda(\A)^M$ computes $\delta^+$ correctly with respect to $V_\lambda (\A)^{M[G]}$ because $\P$ has $\delta$-c.c. in $M$ and $\lambda$ is tall; it is also clear that $P(\delta)^{V_\lambda(\A)^M} = P(\delta)^M$. Theorem \ref{Hamkinsuniquenessthm} applies, so $V_\lambda(\A)^M$ and $N$ have the same sets of ordinals. Since they also have the same sets of urelements, it follows from Lemma \ref{M=M'} that $V_\lambda(\A)^M = N $. Therefore, $x$ is in $M$. \end{proof}

\begin{corollary}
Let $M$ be a transitive model of $\ZFCUR$ where some cardinal $\kappa$ is not realized. If $\P \in M$ is such that $(|\P|$ is $\kappa$-closed $)^M$ and $G$ is $M$-generic over $\P$, $M$ is definable in $M[G]$.
\end{corollary}
\begin{proof}
First, $M$ and $M[G]$ have the same sets of urelements because a $\kappa$-closed forcing does not add small subsets. Hence, $P(A)^M = P(A)^{M[G]}$ for every $A \in M$.
Let $\delta$ be a regular cardinal in $M$ such that $(|\P| < \delta)^M$. Then we can verify that $M = \{x \in M[G] : M[G] \models \exists A \varphi (x, P(A), P(\delta)^{M})\}$, where $\varphi (x, P(A), P(\delta)^{M})$ is defined as in the last theorem with parameter $P(\delta)^M$.
\end{proof}

\subsection{$M[G] = M[G]_\#$}\label{subsection:m[g]=m[g]sharp}
Finally, I show that the two ways of defining $\P$-names, as in Definition \ref{oldpnames} and \ref{newpnames}, give rise to the same forcing extension when we force over a countable transitive model of $\ZFCUR$.

\begin{definition}
Let $M$ be a countable transitive model of $\ZFCUR$ and $\P \in M$ be a forcing poset. We define a map $\sim : M^\P_\# \rightarrow M^\P$ by recursion as follows. For every $\dotx \in M^\P_\#$,
\begin{equation*}
    \widetilde{\dotx} =
    \begin{cases*}
     \{\<a, 1_\P>\} & if $\A(\dotx)$  \\
     \{\<\widetilde{\doty}, p> : \<\doty, p> \in \dot{x}\}        & otherwise 
    \end{cases*}
  \end{equation*}
\end{definition}


\begin{lemma}\label{tilde1-1}
For any $M$-generic filter $G$ over $\P$ and $\dot{y}, \dot{x} \in M^\P_\# $, $\dot{y}_G = \dot{x}_G$ if and only if $\widetilde{\dot{y}}_{G} = \widetilde{\dot{x}}_{G}$.
\end{lemma}
\begin{proof}
We prove it by an induction on the rank of $\dot{y}$ and $\dot{x}$. The lemma holds easily when $\dot{y}$ and $\dot{x}$ are urelements. Suppose $\dot{x}$ and $\dot{y}$ are sets. Then $\widetilde{\dot{x}}$ and $\widetilde{\dot{y}}$ don't contain any urelements in their domains, so $\widetilde{\dot{y}}_{G}$ and $\widetilde{\dot{x}}_{G}$ must be sets. If $\dot{y}_G = \dot{x}_G$, then for any $\widetilde{\dotz}_{G}  \in \widetilde{\dot{y}}_{G}$, $\dotz_G \in \dot{x}_G$ so $\dotz_G = \dot{v}_G$ for some $\dot{v} \in dom(\dot{x})$; by the induction hypothesis, $\widetilde{\dotz}_{G} = \widetilde{\dot{v}}_{G} \in \widetilde{\dot{x}}_{G}$ so $\widetilde{\dot{y}}_{G} \subseteq \widetilde{\dot{x}}_{G}$ and hence $\widetilde{\dot{y}}_{G} = \widetilde{\dot{x}}_{G}$ by the same argument. If $\widetilde{\dot{y}}_{G} = \widetilde{\dot{x}}_{G}$, then for any $\dotz_G \in \dot{y}_G$, $\widetilde{\dotz}_{G} \in \widetilde{\dot{x}}_{G}$ so $\widetilde{\dotz}_{G} = \widetilde{\dot{v}}_{G}$ for some $\dot{v} \in dom(\dot{x})$; by the induction hypothesis, $\dotz_G = \dot{v}_G \in \dot{x}_G$; so $\dot{y}_G \subseteq \dot{x}_G$ and hence the same argument shows that $\dot{y}_G = \dot{x}_G$.
\end{proof}
\noindent The next lemma shows that every $\P$-name in $M^\P$ is a mixture of the $\sim$-image of some $\P$-names in $M^\P_\#$.

\begin{lemma}\label{MPbarmixMP}
Let $M$ be a countable transitive model of $\ZFCUR$ and $\P \in M$ be a forcing poset. For every $\P$-name $\dot{x}$ in $M^\P$,  there is a function $f : dom(f) \rightarrow M^\P_\# $ in $M$ such that (i) $ker(f) \subseteq ker(\dot{x}) \cup ker(\P)$; (ii) $dom(f)$ is a maximal antichain of $\P$; and (iii) for every $p \in dom(f)$, $p \forces \dot{x} = \widetilde{f(p)}$.
\end{lemma}
\begin{proof}
By induction on the rank of $\dot{x}$. Suppose the lemma holds for all the $\P$-names in the domain of $\dot{x}$. Condition (i) allows us to find (without using Collection) some $\alpha$ that is big enough such that for every $\doty \in dom(\dot{x})$, there is some $f$ as in the lemma that lives in $V_\alpha(ker(\dot{x}) \cup ker(\P))$. Then by AC in $M$, we can choose an $f_{\doty}$ for each $\doty \in dom(\dot{x})$. In $M$, define
\begin{align*}
    \dot{w} = \{\<f_{\doty} (p), r> : \doty \in dom(\dot{x}) \cap M^\P \land \exists q (\<\doty, q> \in \dot{x} \land p \in dom(f_{\doty}) \land r \leq p, q)\}.
\end{align*}
It is clear that $\dot{w} \in M^\P_\# $ and $ker(\dot{w}) \subseteq ker(\dot{x}) \cup ker(\P)$. Define $Z = \{ p \in \P : \exists a, q \in \mathcal{A} (\<a, q> \in \dot{x} \land p \leq q) \}$. Let $Y$ be a maximal antichain in $Z$ and let $X$ be a maximal antichain in $\P$ extending $Y$. Note that for every $p \in Y$, there is a unique urelement $a_p \in dom(\dot{x})$ such that $p \leq q$ and $\<a_p, q> \in \dot{x}$ for some  $q$. Now we define $f: X \rightarrow (\mathcal{A} \cap dom(\dot{x})) \cup \{\dotw \}$ as follows.
 \begin{equation*}
    f(p) =
    \begin{cases*}
       a_p & if $p \in Y$  \\
       \dot{w}    & otherwise 
    \end{cases*}
\end{equation*}
It is clear that $ker(f) \subseteq ker(\dot{x}) \cup ker(\P)$. 

It remains to show that for every $p \in X$, $p \forces \dot{x} = \widetilde{f(p)}$. Fix a $p \in X$ and an $M$-generic filter $G$ over $\P$ that contains $p$.


Case 1: $p \in Y$. Then $\widetilde{f(p)} = \{\<a_p, 1_\P>\}$. And since there is a $q$ such that $\<a_p, q> \in \dot{x}$ and $p \leq q$, it follows that $\dot{x}_{G} = a_p = \widetilde{f(p)}_{G}$.

Case 2: $p \notin Y$.
\begin{claim}
$\dot{x}_{G}$ is a set.
\end{claim}
\begin{claimproof}
Suppose $\dot{x}_{G}$ is an urelement. Then for some urelemen $a$ and $q \in G$, $\<a, q> \in \dot{x}$. So there is a $r$ which extends both $p$ and $q$; as $r \in Z$, there is some $s \in Y$ such that $s$ and $r$ are compatible because $Y$ is maximal in $Z$. But this means that $p$ is compatible with some $s \in Y$, which is a contradiction because $X$ is an antichain.
\end{claimproof}


\noindent Then $\widetilde{f(p)}_{G} = \widetilde{\dot{w}}_{G}$. Note that $\widetilde{\dot{w}}_{G}$ is a set by the construction of $\widetilde{\dot{w}}$. So it remains to show that $\dot{x}_{G} \subseteq \widetilde{\dot{w}}_{G}$ and $\widetilde{\dot{w}}_{G} \subseteq \dot{x}_{G}$. Consider any $\doty_{G} \in \dot{x}_{G}$ with $\<\doty, q> \in \dot{x}$ and $q \in G$. Since $dom(f_{\doty})$ is a maximal antichain, there is some $p' \in dom(f_{\doty})$ and $r \in G$ such that $p' \in G$ and $r \leq q, p'$. So $\<f_{\doty} (p'), r> \in \dot{w}$ and $p' \forces \doty = \widetilde{f_{\doty}(p')}$. It follows that $ \doty_{G} = \widetilde{f_{\doty}(p')}_{G} \in \widetilde{\dot{w}}_{G}$ and hence $\dot{x}_{G} \subseteq \widetilde{\dot{w}}_{G}$.


To show that $\widetilde{\dot{w}}_{G} \subseteq \dot{x}_{G}$, fix some $\widetilde{f_{\doty}(p')}_{G} \in \widetilde{\dot{w}}_{G}$ such that $\doty \in dom(\dot{x})$, $p' \in dom(f_{\doty})$ and  $\<f_{\doty}(p'), r> \in \dot{w}$ for some $r \in G$. Then there is some $q$ such that $\<\doty, q> \in \dot{x}$ and $r \leq p', q$, which implies $\doty_{G} \in \dot{x}_{G}$. As $p' \forces \doty = \widetilde{f_{\doty}(p')}$, we have $\widetilde{f_{\doty}(p')}_{G} = \doty_{G} \in \dot{x}_{G}$, as desired.
\end{proof}



%M[G] = M[G] bar%
\begin{theorem}
Let $M$ be a countable transitive model of $\ZFCUR$, $\P \in M$ be a forcing poset and $G$ be an $M$-generic filter over $\P$. There is an elementary embedding from $M[G]_\#$ to $M[G]$. Hence, $M[G] = M[G]_\#$.
\end{theorem}
\begin{proof}
 We prove that the map $\dot{x}_G  \mapsto \widetilde{\dot{x}}_{G}$ is elementary by an induction on formulas. Lemma \ref{tilde1-1} shows that this map is well-defined and 1-1. It is easy to check that the map preserves membership. Also, it is clear that $\dot{x}_G$ is an urelement just in case $\widetilde{\dot{x}}_{G}$ is. The Boolean cases are trivial. If $M[G] \models  \exists x \varphi (x)$, then $M[G] \models \varphi (\dotx_{G})$ for some $\dotx \in M^\P$. Fix a function $f$ for $\dotx$ as in Lemma \ref{MPbarmixMP}. Then for some $p \in dom(f)$, $p \in G$ and $p \forces \dot{x} = \widetilde{f(p)}$, and so $\dot{x}_{G} = \widetilde{\dot{y}}_{G}$ where $\dot{y} = f(p) \in M^\P_\# $. By the induction hypothesis, $M[G]_\# \models \varphi (\dot{y}_G)$ and hence $M[G]_\# \models \exists x \varphi(x)$. Therefore, $M[G] \models $ ZFCU$_R$, and by the minimality of $M[G]_\#$ and $M[G]$ it follows that $M[G]_\# = M[G]$.\end{proof}
The assumption $M \models$ AC is not necessary for the conclusion that $M[G]_\# = M[G]$. This is because one can show that $M[G]_\# \models \ZFUR$ whenever $M$ does (the argument is the same as the proof of Theorem \ref{forcingpreservesreplacement}), and so $M[G]_\# = M[G]$ by the minimality of both forcing extensions. However, the proof presented here clarifies the relationship between these two kinds of $\P$-names.





















%BVM%
\section{Boolean-valued models with urelements}\label{section:forcingBVM}
In this section, I discuss Boolean-valued models of set theory with urelements. I first summarize some key results regarding Boolean-valued models of $\ZFCUR$ proved in my joint work with Wu \cite{wu2022}. Based on these results, I further investigate Boolean-valued ultrapowers of models of ZFCU. Basic knowledge of Boolean-valude models of ZFC, which is covered in the first three chapters of \cite{LBell2005-LBESTB}, will be assumed.

\subsection{An overview of $\UB$}
If $V$ is a model of ZFC, then given a complete Boolean algebra $\B \in V$, by transfinite recursion in $V$ a $\B$-name is defined to be a function from a set of $\B$-names to $\B$. As in Definition \ref{oldpnames}, there is also a straightforward generalization of $V^\B$, adopted in  \cite{blass1989freyd}, in urelement set theory. Namely, we treat each urelement as its own $\B$-name. And given what have seen earlier, it is unsurprising that this approach faces the same problem as $\P$-names$_\#$.
\begin{definition}
    A Boolean-valued model for a language $\mathscr{L}$ is said to be \textit{full} just in case for any formula $\varphi$ in $\mathscr{L}$ and $\tau_1, ..., \tau_n \in M^\B$, there is some $x \in M^\B$ such that $\llbracket \exists v \varphi(v, \tau_1, ..., \tau_n) \rrbracket = \llbracket  \varphi(x, \tau_1, ..., \tau_n) \rrbracket$.  
\end{definition}
\noindent It is straightforward to show that if urelements are treated as their own names, almost all Boolean-valued models of ZFCU are not \textit{full} (see \cite{wu2022}). However, fullness, as I have mentioned in the beginning of this chapter, is crucial for applications of Boolean-valued models in set theory. In the joint work \cite{wu2022} with Wu, we provided the following new definition of Boolean-valued models with urelements.
\begin{definition}[$\ZFUR$ \cite{wu2022}]\label{newbnames}
Let $\B$ be a complete Boolean algebra.
\begin{enumerate}
    \item A function $\tau: dom(\tau) \rightarrow \mathbb{B}$ is a $\BB$-name if and only if for any $x \in dom(\tau)$, $x$ is either an urelement or a $\BB$-name, and for any urelement $a \in dom(\tau)$ and $x \in dom(\tau)$, $\tau(a) \land \tau(x) = 0_\B$ whenever $x \neq a$.
    
    \item Let $\tau$ be a $\BB$-name. 
    \subitem $dom^{\mathscr{A}}(\tau) = \{ a \in dom(\tau) : a \in \mathscr{A} \}$;
    \subitem $dom^{\mathbb{B}}(\tau) = \{ \eta \in dom(\tau) : \eta \text{ is a $\BB$-name}\}$;
    \subitem $\tau(a) = 0_\B$ whenever $a$ is an urelement not in $dom^{\mathscr{A}}(\tau)$.
    
    \item $\UBB = \{ \tau \in U : \tau \text{ is a } \BB \text{-name}\}$.
    
     \item $\pazocal{L}_{\BB}$ contains $\{\subseteq, =, \in, \A, \overset{\mathscr{A}}{=}\}$ as the non-logical symbols and each $\B$-name as a constant symbol. $\pazocal{AL}_{\BB}$ is the class of all atomic formulas in $\mathcal{L}_{\BB}$. The Boolean evaluation function $\llbracket\ \  \rrbracket: \pazocal{AL}_{\BB} \rightarrow \B$ is defined as follows.
     \begin{align*}
     & \llbracket \mathscr{A}(\tau) \rrbracket = \bigvee\limits_{a \in \mathscr{A}} \tau(a) \\
     & \llbracket \tau \overset{\mathscr{A}}{=} \sigma \rrbracket = \bigwedge\limits_{a \in \mathscr{A}} (\tau(a) \Leftrightarrow \sigma (a) )\\
     & \llbracket \tau \in \sigma \rrbracket = \bigvee\limits_{\mu \in dom^{\mathbb{B}}(\sigma)}\llbracket \tau = \mu \rrbracket \land \sigma(\mu) \\
    & \llbracket \tau \subseteq \sigma \rrbracket= \bigwedge\limits_{\eta \in dom^{\mathbb{B}}(\tau)} \tau(\eta) \Rightarrow \llbracket \eta \in \sigma \rrbracket\\
     & \llbracket \tau = \sigma \rrbracket = \llbracket \tau \subseteq \sigma \rrbracket \land \llbracket \sigma \subseteq \tau \rrbracket \land \llbracket \tau \overset{\mathscr{A}}{=} \sigma \rrbracket
     \end{align*}
     \item $\llbracket\ \  \rrbracket$ is extended to $\mathcal{L}_\B$ in the standard way. Namely,
\begin{enumerate}
    \item [] $\llbracket \varphi \land \psi \rrbracket = \llbracket \varphi \rrbracket\land \llbracket \psi \rrbracket$;
    \item [] $\llbracket \neg \varphi \rrbracket = \neg \llbracket \varphi \rrbracket$;
    \item []  $\llbracket \exists x \varphi \rrbracket = \bigvee\limits_{\tau \in \UB}\llbracket \varphi(\tau) \rrbracket$.
\end{enumerate}
\end{enumerate}
\end{definition}
\noindent Note that no urelement is a $\BB$-name since every $\BB$-name is a set, and each urelement $a$ will be represented canonically by $\{\<a, 1_\B>\}$ in $\UBB$ instead of itself. For any $\tau \in \UBB$ and urelement $a$, $\tau(a)$ will be the $\B$-degree to which $\{\<a, 1_\B>\}$ \textit{is identical }to $\tau$, as opposed to the value of $\{\<a, 1_\B>\}$'s membership to $\tau$. This motivates the incompatibility condition: if $a, b \in dom(\tau)$ are two urelements, then $\tau(a) \land \tau(b)$ must be $0_\B$ because this is the degree to which $\tau$ is both of them; if $a, \sigma \in dom(\tau)$, where $\sigma$ is a $\BB$-name, then $\tau(a) \land \tau(\sigma)$ must be $0_\B$ as well because this is the degree to which $\tau$ is an urelement with a member. In fact, it is this restriction that ensures $\llbracket \text{no urelement has any members} \rrbracket =1_\B$. To see this, consider any $\tau \in \UB$. Since for any urelement $a$ and $\B$-name $\mu \in dom(\tau)$, $\tau(a) \leq \neg \tau(\mu)$, we have
\begin{align*}
   \llbracket \mathscr{A}(\tau) \rrbracket & = \bigvee\limits_{a \in \mathscr{A}} \tau(a) \\
    & \leqslant \bigwedge\limits_{\sigma \in \UB}\bigwedge\limits_{\mu \in dom^{\mathbb{B}}(\tau)} \llbracket \sigma \neq \eta \rrbracket \lor \neg\tau(\mu) \\
    & = \llbracket \forall y (y \notin \tau) \rrbracket.
\end{align*}
Finally, $ \llbracket \tau \overset{\mathscr{A}}{=} \sigma \rrbracket^{\UBB}$ is the degree to which $\tau$ and $\sigma$ are identical when they are taken as urelements.
\begin{theorem}[\cite{wu2022}]\label{collection<->fullness}
Over $\textup{ZFCU}_\text{R}$, the following are equivalent.
\begin{enumerate}
    \item Collection.
    \item For every complete Boolean algebra $\mathbb{B}$, $\UBB$ is full.\qed
\end{enumerate}
\end{theorem}
\begin{theorem}[The Fundamental Theorem of $\UB$ [\cite{wu2022}]]\label{fundamentalthm}
Let $U$ be a model of $\ZFCUR$ and $\B$ be a complete Boolean-algebra in $U$. Then ($\UB \models \varphi$ abbreviates $\llbracket \varphi \rrbracket = 1_\B$)
\begin{enumerate}
    \item $\UB \models $ $\ZFCUR$;
    \item $\UB \models $ Collection if $U \models$ Collection.\qed
\end{enumerate}
\end{theorem}


\subsection{Boolean ultrapowers with urelements}
In this section, I consider how the construction of Boolean ultrapowers studied in \cite{hamkins2012well} can be carried out for models of ZFCU. Let $U$ be any model of ZFCU and $\B$ be a complete Boolean algebra in $U$. Theorem \ref{collection<->fullness} and \ref{fundamentalthm} allow us to transform $U^\B$ into a classical two-valued model with respect to \textit{any} ultrafilter $F$ on $\B$. For every $\sigma, \tau \in \UBB$, we define:
\begin{itemize}
    \item [] $\sigma =_F \tau$ if and only if  $\llbracket \sigma =\tau \rrbracket \in F$;
    \item []  $\sigma \in_F \tau$ if and only if  $\llbracket \sigma \in \tau \rrbracket \in F$;
    \item [] $\A_F(\sigma)$ if and only if $\llbracket \A(\sigma) \rrbracket \in F$.
\end{itemize}
For each $\tau \in U^\B$, let $[\tau]_F$ be the equivalence class $\{\sigma \in \UBB : \sigma =_F \tau \}$ and $\UBB/F = \{[\tau]_F : \tau \in \UBB \}$. $\in_F$ and $\A_F$ are well-defined on the corresponding equivalent classes because $=_F$ is a congruence with respect to them. This generates a two-valued model $\<\UBB/F, \A_F, \in_F>$ for the language of urelement set theory (which will also be denoted by $\UBB/F$.)

\begin{theorem}[Łoś Theorem]\label{losthm}
Let $U$ be any model of ZFCU, $\B$ be a complete Boolean algebra in $U$, and $F$ be an ultrafilter on $\B$. For every $\tau_1, ..., \tau_n \in \UB$, $\UBB/F \models \varphi([\tau_1]_F, ..., [\tau_n]_F)$ if and only if $\llbracket \varphi(\tau_1, ..., \tau_n) \rrbracket \in F$. Hence, $\UB/F \models$ ZFCU.
\end{theorem}
\begin{proof}
Atomic formulas and Boolean connectives are easy to check. For the quantifier case, suppose $\varphi$ is some $\exists x \psi$. The left-to-right dIrection is immediate. If $\llbracket \exists x \psi (x, \tau_1, ..., \tau_n) \rrbracket$ is in $F$, since  $\UBB$ is full by Theorem \ref{collection<->fullness}, it follows that $\llbracket \psi (\tau, \tau_1, ..., \tau_n) \rrbracket \in F$ for some $\tau \in \UB$ and hence $\UBB/F \models \exists x \psi (x, [\tau_1], ..., [\tau_n])$ by the induction hypothesis. $\UBB/F \models$ ZFCU by Theorem \ref{fundamentalthm}.
\end{proof}
\noindent Therefore, to show that a certain statement $\varphi$ is not provable from ZFCU it suffices to find a $\B$ such that $\llbracket \varphi \rrbracket \neq 0_\B$ and then consider an ultrafilter $F \subseteq \B$ that contains $\llbracket \varphi \rrbracket$. And one can further establish the mutual interpretability of various extensions of ZFCU. However, we should note that since forcing preserves Plenitude, this way of taking quotient models directly from a model of ZFCU is much less flexible than the method described in Theorem \ref{zfcumutualinter}.

Following \cite{hamkins2012well}, let us further consider how $U$ sits inside $U^\B$. Define $\Ucheck$ as a new unary predicate such that $\llbracket \tau \in \Ucheck \rrbracket = \bigvee\limits_{x \in U}\llbracket \tau =  \xcheck \rrbracket$, which clearly obeys the law of identity. Define $\tau \in_F \Ucheck_F$ as $\llbracket \tau \in \Ucheck \rrbracket \in F$. $=_F$ remains a congruence with respect to $\Ucheck_F$. In $\UB$, $\Ucheck$ represents the class of all objects the ground model $U$. For any formula $\varphi$ in the language of urelement set theory, let $\varphi^{\Ucheck}$ be the result of restricting all the quantifiers to $\Ucheck$. Then by the same argument as in \cite[Lemma 4]{hamkins2012well}, one can show that $U \models \varphi(x_1, ..., x_n)$ if and only if $\llbracket \varphi^{\Ucheck} (\xcheck_1, ..., \xcheck_n) \rrbracket = 1_\B$. Let $\Ucheck_F = \{[\tau]_F : \llbracket \tau \in \Ucheck \rrbracket \in F\}$. As a submodel of $U^\B/F$, by the same argument as in \cite[Lemma 12]{hamkins2012well}, one can show that $\Ucheck_F \models \varphi ([\tau_1]_F, ..., [\tau_n]_F)$ if and only if $\llbracket \varphi^{\Ucheck}(\tau_1, ..., \tau_n)\rrbracket \in F$. This yields the Boolean ultrapower embedding.

\begin{theorem} \label{ultrapowermap}
Let $U$ be a model of ZFCU, $\B$ be a complete Boolean algebra in $U$ and $F$ be an ultrafilter on $\B$. The Boolean ultrapower map $j_F : U \rightarrow \Ucheck_F $ defined by $j_F(x) = [\xcheck]_F$ is an elementary embedding. \qed
\end{theorem}
\noindent Note that the map $j_F$ is not necessarily onto: the same argument in \cite[Theorem 16]{hamkins2012well} will show that $j_F$ is onto just in case $F$ is $U$-generic.


Furthermore, following \cite{hamkins2012well}, we can show that the quotient structure $U^\B/F$ is in fact a forcing extension of $\Ucheck_F$ in the sense of Definition \ref{m[g]def}. Let $\dot{G} = \{\<\check{p}, p> : p \in \B\}$. It is not hard to check that (see \cite[Lemma 8]{hamkins2012well})
$$U^\B \models \dot{G} \text{ is a } \Ucheck\text{-generic ultrafilter on } \check{\B}.$$
If $\tau \in U^\B$ and $F$ is a filter on $\B$, we can define the $F$-valuation of $\tau$, $\tau_F$, as follows (similar to Definition \ref{m[g]def}). 
\begin{itemize}
    \item [] $\tau_F = a$ if $\mathcal{A}(a)$ and  $\langle a, p \rangle \in \tau$ for some $p \in F$;
    \item [] $\tau_F = \{ \sigma_F: \langle \sigma , p \rangle \in \tau \text{ for some } \sigma \in U^\B \text{ and } p \in F \}$ otherwise.
\end{itemize}
 \noindent Note that this is well-defined by the incomptability condition in Definition \ref{newbnames}. Then by induction one can check that for any $\tau \in U^\B$,
$$\UB \models \tau \text{ is the } \dot{G}\text{-valuation of } \check{\tau}.$$
In other words, $\UB$ sees itself as the forcing extension $\Ucheck[\dot{G}]$. Let $G$ be $[\dot{G}]_F$. By Theorem \ref{losthm}, $\UB/F \models \forall x \exists y \in \check{U}_F(x = y_G)$ and hence $\UB/F=\Ucheck_F[G]$. 

I conclude this chapter with an application of this machinery.
\begin{theorem}
If $U$ is a model of ZFCU + $\neg$Plenitude, then there is a model $W$ of ZFCU in which every set of urelements is countable, and $W$ is a forcing extension of an elementary extension of $U$.
\end{theorem}
\begin{proof}
Let $\kappa$ be the least cardinal not realized in $U$ and $\B = RO(\kappa^{\omega})$ be the complete Boolean-algebra which consists of all the regular open sets of the product topology $\kappa^\omega$, where $\kappa$ is assigned the discrete topology. It is a classical result that $\UB \models \check{\kappa} \sim \omega$. For every $\tau \in \UB$, observe that $\llbracket \tau \subseteq \A \rrbracket = \llbracket  \tau \subseteq \check{A_\tau} \rrbracket$ and $\UB \models |\check{A_\tau}| \leq |\check{\kappa}|$, where $A_\tau$ is the set of urelements in $dom(\tau)$. It follows that $\llbracket \tau \subseteq \A \rrbracket \leq \llbracket |\tau| \leq \omega \rrbracket$ for every $\tau \in \UB$, i.e., $\UB \models $ ``every set of urelements is countable''. Let $F$ be an ultrafilter on $\B$. By Theorem \ref{losthm}, it follows that $\UB/F \models $ ZFCU + ``every set of urelements is countable''. And by the previous disucssion, $\UB/F$ is a forcing extension of $\Ucheck_F$, which is an elementary extension of $U$ by Theorem \ref{ultrapowermap}.
\end{proof}