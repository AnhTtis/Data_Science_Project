The ARMBench dataset presents: 1) a collection of sensor data acquired by a robotic manipulation workcell performing pick-and-place operation, 2) metadata and reference images for objects in containers, 3) a set of annotations acquired either automatically, by virtue of the system design, or via manual labeling, and 4) tasks and metrics to benchmark perception algorithms for robotic manipulation. Fig.\ \ref{fig:contributions} illustrates the benchmark tasks and variety of objects captured in the dataset. The dataset captures diversity in objects with respect to Amazon product categories as well as physical characteristics such as size, shape, material, deformability, appearance, fragility, etc. 

The data collection platform is a robotic manipulation workcell performing pick-and-place operation in a warehouse \cite{Sparrow2022}. The workcell contains a robotic arm mounted with a vacuum-based end-effector. It is presented with a heterogeneous collection of objects placed in unstructured configurations within a container (storage tote). The robotic arm is tasked with picking one object at a time (singulation) and place it on moving trays until the container is empty. The empty container ejects the workcell and is replaced by a new container. While the operation is completely autonomous, it includes a human-in-the-loop to monitor the status of each pick-and-place activity, annotate, and resolve any defects during manipulation. Multiple imaging sensors are placed in the workcell to facilitate and validate the pick-and-place operation. Following is a list of sensor data (Fig.\ \ref{fig:intro}) associated with each pick activity:
\begin{itemize}
\item Pick-image: A 5\,MP camera is used to capture a top-down image of the container.
% \item Pick-3D: Two Ensenso sensors capture the 3D point cloud of the source container.
\item Transfer-images: Multiple 5\,MP cameras are placed on different sides in the workcell to capture the moving object from different viewpoints.
% \item Transfer-Barcode: Multiple Cognex barcode sensors are used to scan the barcode of the object during transfer.
\item Place-image: A top-down view of the object is captured once it is placed on the tray.
\item Video: A camera is mounted to capture 720p videos of pick-and-place manipulation processes at 30\,FPS
\end{itemize}
Additionally, the following metadata (Fig.\ \ref{fig:contributions} (b)) is available by virtue of a warehouse tracking system:
\begin{itemize}
\item Container-manifest: A list of objects present in the container along with data such as product description, coarse dimensions, and weight.
\item Reference images: One or more images of objects from previous operations within the warehouse.
\end{itemize}
The sensor data and metadata were consumed by perception algorithms required to autonomously operate the robotic workcell. Benchmarking against these algorithms would not only optimize a manipulation task such as the one used for data collection but also enable more complex and intentional manipulation. This work considers a subset of such perception tasks namely object segmentation, object identification, and defect detection. These are critical not only to make informed grasping and motion decisions but also to track the state of the objects and containers within the warehouse. The following sections will describe these tasks and present the challenges using annotations, baseline algorithms, and evaluation metrics.

\begin{figure}
    \centering
    \vspace{-5pt}
    \includegraphics[width=0.9\linewidth]{images/modediversity.png}
    \caption{\textbf{Visualization of different grasp modes on the same laptop.} We show 3 different grasp modes for opening a laptop. Both the starting and ending hand poses of the grasps are shown. }
    \label{fig:result-modediversity}
\end{figure}

\begin{figure}
    \centering
    \vspace{-5pt}
    \includegraphics[width=0.9\linewidth]{images/comparison.png}
    \caption{\textbf{Qualitative results compared with GraspTTA \cite{grasptta} and ManipNet \cite{zhang2021manipnet}.} The results generated by our method are more realistic.}
    \label{fig:result-comparison}
    \vspace{-10pt}
\end{figure}


\label{sec:exp}
In this section, we
% explain details of our framework and
apply our method to synthesize HOM, and evaluate the effect of our method with various metrics. We first introduce the experimental settings including dataset (Section~\ref{sec:data}), baselines (Section~\ref{sec:baselines}), and evaluation metrics (Section~\ref{sec:metrics}), while the experimental details are presented in our supplementary material.
% , and experimental details including baselines from other methods in Section~\ref{sec:detail}
We then show that our method could handle shape diversity and manipulation diversity and achieve a surpassing synthesis performance compared with other methods in Section~\ref{sec:compare}.
% Finally, we propose ablation studies in Section~\ref{sec:ablation}.

\subsection{Dataset}
\label{sec:data}

We utilize HOI4D Dataset \cite{hoi4d} in our experiment. HOI4D is a real-world dataset that contains dynamic HOM data spanning various rigid and articulated object categories. We select five object categories with different functionalities in our experiment, as shown in (\tabref{tab:dataset}). % Especially, we specify several different manipulation types in \emph{Laptop} category with \emph{Open the Laptop} task, and use \emph{Scissors} to show that we can generate complex manipulation. 
The selected HOM data contains both various manipulation types (\eg opening a laptop with different human preferences) and complex manipulation processes (\eg opening a small scissor by putting fingers through the hole) to improve the diversity and complexity of our synthesis results. To improve the usability of HOI4D under our setting, we applied some data cleaning and augmentation methods to the raw data (see our supplementary material for more details).
% Note that all of our articulated objects consist of two parts, whereas it doesn't mean that our method could not be generalized to a more complicated articulated object category.

% \textbf{Data Processing}
% \modify{move to the supplementary material?}

% \subsection{Experimental Details}
% \label{sec:detail}

% \modify{can we move this whole subsection to the supplementary-material? now there's little space to write (8-page limit)}

% Given triangular meshes $\{\mathbf{M}_i\}_{i=1}^N$ of the manipulated object, we sample 1000 points for each mesh, and each part of the object is normalised to our CAMS. We use a batch size of 64 for training, and the training contains 1000 epochs. We use $\lambda_{flag}=0.1$, $\lambda_{pos}=500$, $\lambda_{dir}=100$, $\lambda_{tip}=100$, $\lambda_{vec}=1$, and $\lambda_{kld}=5$ during training, we will further explain how to balance loss in (Section~\ref{sec:ablation}).

% \lxmodify{As for the synthesizer, we empirically set $\lambda_{\mathrm{tip}}=50$, $\lambda_{\mathrm{joint}}=1$, and $\lambda_{\mathrm{smooth}}=0.05$ and $1000$ respectively for the MANO pose parameters and hand transition part in $\theta$ for fitting finger embedding. Our training contains 2000 epochs in this step. When optimizing contact and penetration, we empirically set $\lambda_{\mathrm{contact}}=80$, $\lambda_{\mathrm{trans}}=1$, $\lambda_v=5$ and $\lambda_a=20$. This step runs for $6$ iterations and $\lambda_{\mathrm{smooth}}$ in this step is set to $1$ in the first $2$ iterations, $10$ in the following $2$ iterations and $500$ in the last $2$ iterations. Each iteration runs for $500$ epochs.}

 \section{Benchmarks and Evaluation}
\label{sec:eval}

We evaluate \krakenSpace to answer the following set of questions:
\begin{itemize}
\item How much improvement does partial evaluation and our implemented compiler optimizations give \kraken? %(\S \ref{sec:eval2})
\item How much faster is our purely functional f-expr language, \krakenSpace, compared to other implementations of fexprs? %(\S \ref{sec:eval1} - \ref{sec:eval2})
\item How does \kraken's performance, with its fexprs, compare to macros? %(\S \ref{sec:eval1}, \S \ref{sec:eval3})
\item How do the different partial evaluation mechanisms/optimizations in \krakenSpace contribute towards reduction in overall runtime?
%\item What does \krakenSpace do internally when we create a data structure and evaluate it for some function? (\S \ref{sec:casestudy})
\end{itemize}

\textbf{Experimental Setup}: 
We ran these experiments in a reproducible Nix environment on a NixOS install \cite{10.1145/1411203.1411255} (Kernel 6.0.0) on a laptop with 8 cores / 16 threads and 64 GB of RAM.
Our code contains the scripts and Nix Flakes needed to reproduce the exact set of dependencies to run our tests.
%The code can be found at \url{https://github.com/limvot/kraken}.

The Kraken benchmarks were run using both the Wasmtime and WAVM WebAssembly engines for most benchmarks.
The Wasmtime WebAssembly engine is one of the most popular, developed by the Bytecode Alliance itself, and uses the CraneLift code generation backend.
The WAVM WebAssembly engine is interesting for its use of LLVM, and it often produces the fastest code on benchmarks but has a higher startup time.
We eliminated the Cfold Wasmtime benchmark due to problems running out of stack space (a known property of the Cfold benchmark).

\textbf{Benchmarks}: 
To showcase the capability of Kraken, we created benchmarks that are commonly implemented in functional languages and have been used as benchmarks in other papers \cite{reinking2021perceus, 10.1145/3547646}.
The benchmarks are
\begin{itemize}
\item Fib - Calculating the nth Fibonacci number
\item RB-Tree - Inserting n items into a red-black tree, then traversing the tree to sum its values
\item Deriv - Computing a symbolic derivative of a large expression
\item Cfold - Constant-folding a large expression
\item NQueens - Placing n number of queens on the board such that no two queens are diagonal, vertical, or horizontal from each other
\end{itemize}
All benchmarks besides Fibonacci use the fexpr version of match for pattern matching in \kraken, which is equivalent to the macro version in NewLisp. We also RB-Tree using NewLisp's~\cite{mueller2018newlisp} version of fexpr match. We modified the sizes of the problems presented to the benchmark to account for the longer running times of some of the less-optimized implementations.
The code for Kraken and NewLisp is very similar, and we should note that it is very unidiomatic NewLisp.
Our goal was not to compare Kraken and NewLisp as implementation languages for Red-Black Trees, but to stress test a single reasonably complex fexpr/macro, namely pattern matching.
% \textbf{Comparison with other languages}: We evaluated \krakenSpace against a language that contains f-exprs, as well as against itself with various optimizations disabled. The only other language we could find which contains a real f-expr mechanism is NewLisp~\cite{mueller2018newlisp} and so we ported \kraken's benchmark implementation to NewLisp.

%The six state-of-the-art languages are Java 17.0.1, Swift 5.4.2, Koka 2.3.2, C++, Haskell 8.10.7, and OCaml 4.12.
%The language choices were taken directly from Perceus reference-counting paper \cite{reinking2021perceus}.
%The Fibonacci benchmark additionally tests Python 3.9.11 and Chez Scheme 9.5.4.
%Koka, Ocaml and Haskell are good comparison points as statically-typed, compiled, functional programming languages, while Chez Scheme is a good comparison point as a mature and industrial strength dynamically-typed Scheme implementation known for its performance. 
%\subsection{Basic Level Comparison}
\subsection{The Effect of Partial Evaluation on Eval Calls}

\begin{table}[h]
\caption{Number of eval calls with no partial evaluation for Fexprs}
	\begin{tabular}{||c | c c c c c ||} 
		\hline
		&Evals & Eval w1 Calls & Eval w0 Calls & Comp Dyn & Comp Dyn\\ 
        & & & & w1 Calls & w0 Calls\\ [0.5ex] 
		\hline\hline
		Cfold 5 & 10897376 & 2784275 & 879066  & 1 & 0 \\ 
		\hline
		  Deriv 2  & 11708558 & 2990090 & 946500 & 1 & 0 \\ 
        \hline
		  NQueens 7 & 13530241 & 3429161 & 1108393 & 1 & 0 \\ 
    \hline
		  Fib 30 & 119107888 & 30450112 & 10770217 & 1 & 0 \\ 
    \hline
		  RB-Tree 10 & 5032297 & 1291489 & 398104 & 1 & 0 \\ 
		\hline
	\end{tabular}
    \label{npe:calls}
 \end{table}

As mentioned before, using fexprs without partial evaluation will prelude optimization and cause a massive amount of repeated work. Table \ref{npe:calls} and Table \ref{pe:calls} show the number of calls to the \krakenSpace runtime's eval function, the number of times the runtime's eval function executed a call to an applicative with wrap\_level=1, the number of times the runtime's eval function executed a call to an operative with wrap\_level=0, the number of compiled dynamic calls to applicatives with wrap\_level=1, and the number of compiled dynamic calls to operatives with wrap\_level=0.
These are shown for \krakenSpace test cases with partial evaluation turned off and turned on. 
\begin{table}[h]
\caption{Number of eval calls in Partially Evaluated Fexprs}
	\begin{tabular}{||c | c c c c c ||} 
		\hline
		&Evals & Eval w1 Calls & Eval w0 Calls & Comp Dyn & Comp Dyn\\ 
        & & & & w1 Calls & w0 Calls\\ [0.5ex] 
		\hline\hline
		Cfold 5 & 0 & 0 & 0  & 0 & 0 \\ 
		\hline
		  Deriv 2  & 0 & 0 & 0 & 2 & 0 \\ 
        \hline
		  NQueens 7 & 0 & 0 & 0 & 0 & 0 \\ 
    \hline
		  Fib 30 & 0 & 0 & 0 & 0 & 0 \\ 
    \hline
		  RB-Tree 10 & 0 & 0 & 0 & 10 & 0 \\ 
		\hline
	\end{tabular}
    \label{pe:calls}
 \end{table}

\begin{table}[h]
\caption{Number of calls to the runtime's eval function for RB-Tree. The table shows the non-partial evaluation numbers -> partial evaluation numbers.}
	\begin{tabular}{||c | c c c c c ||} 
		\hline
		&Evals & Eval w1 Calls & Eval w0 Calls & Comp Dyn & Comp Dyn\\ 
        & & & & w1 Calls & w0 Calls\\ [0.5ex] 
		\hline\hline
		  RB-Tree 7 & 2952848 -> 0 & 757932 -> 0 & 233513 -> 0 & 1 -> 7 & 0 -> 0\\ 
        \hline
		  RB-Tree 8 & 3532131 -> 0 & 906548 -> 0 & 279379 -> 0 & 1 -> 8 & 0 -> 0\\ 
        \hline
		  RB-Tree 9 & 4278001 -> 0 & 1097965 -> 0 & 3383831 -> 0 & 1 -> 9 & 0 -> 0\\ 
		\hline
	\end{tabular}
    \label{pe:rb}
    \vspace{-4mm}
 \end{table}

Without partial evaluation, no compilation can be done because it is impossible to tell if arguments to calls will be evaluated. In all benchmarks, partial evaluation removed all calls to the runtime's eval function, resulting in a completely compiled program. Looking at RB-Tree, there are over a million calls to combiners with wrap level 1 (normal functions), and 398,000 calls to combiners with wrap level 0 (operatives replacing macros). This massive blowup in the number of calls is due to the repeated and exponential re-execution of macro-like-combiners in the definition of other macro-like-combiners, as discussed in the Introduction.

The non-partially-evaluated benchmarks show 1 compiled dynamic call to an applicative (its the first call into eval) and 0 compiled dynamic calls to operatives, because there is no compilation at all. For the partially evaluated benchmarks, there are a few compiled dynamic calls to applicatives due to higher-order function use in the benchmarks, and there are no compiled dynamic calls to operatives, as all operative use has been eliminated.
We also varied the inputs for RB-Tree shown in Table \ref{pe:rb} to give a sense for how the number scale with respect to input size.

The incredible slowdown implied by these tables comes to full fruition in our RB-Tree test in Fig.~\ref{fig:kraken_nqueens_rbtree}.
We kept this run shorter because Kraken's non-partial-evaluating interpreter takes an incredibly long time even for 100 insertions (40 minutes).
The compounding layers of repeated macro-like operative calls in the non-partially-evaluated Kraken version cause a ~70,000x slowdown relative to the partial evaluated, optimized, and compiled version.
For the remaining benchmarks, we remove the naive interpreted \krakenSpace version, as in each case its performance is so bad as to blow out the graph and make it impossible to do any comparison.
In our optimized Kraken, our partial evaluation algorithm is able to fully collapse these levels of inefficiency, evaluate and inline the results, and give the backend more specialized code to optimize, emitting a compiled version that handily beats not only the NewLisp-fexpr implementation but even the NewLisp-macro implementation, as can be seen in Fig.~\ref{fig:kraken_vs_world_fib}.
We kept the benchmark sizes small in this test because the stack limits of NewLisp prevent sizes larger then ~880, while the Tail Call Elimination performed by the \krakenSpace compiler allows us to run much larger benchmarks, including the run of 4,800,000 inserts to the RB-Tree.
This result shows the dramatic effect of partial evaluation and compiler optimizations on runtime for \kraken. Our technique takes the performance of a fully fexpr based language from being completely infeasible to being faster than a macro-based dynamic scripting language currently in use.
% \begin{center}
% \begin{table}[ht]
% \caption{Number of call to the runtime's eval function for Fib. The table shows the non-partial evaluation numbers -> partial evaluation numbers}
% 	\begin{tabular}{||c | c c c c c ||} 
% 		\hline
% 		&Evals & Eval w1 Calls & Eval w0 Calls & Comp Dyn w1 Calls & Comp Dyn w0 Calls\\ [0.5ex] 
% 		\hline\hline
% 		Fib 10 & 8468 -> 0 & 2167 -> 0  & 777 -> 0 & 1 -> 0 & 0 -> 0 \\ 
% 		\hline
% 		  Fib 15  & 87916 -> 0 & 22478 -> 0 & 7961 -> 0 & 1 -> 0 & 0 -> 0 \\ 
%         \hline
% 		  Fib 20 & 969010 -> 0 & 247731 -> 0 & 87633 -> 0 & 1 -> 0 & 0 -> 0 \\ 
%     \hline
% 		  Fib 25 & 10740492 -> 0 & 2745825 -> 0  & 971209 -> 0 & 1 -> 0 & 0 -> 0 \\ 
% 		\hline
% 	\end{tabular}
%     \label{pe:fib}
%  \end{table}
% \end{center}

\begin{figure}[h]
\caption{Constant Fold and Deriv}
\includegraphics[width=0.45\textwidth]{cfold_table.csv_}
\includegraphics[width=0.45\textwidth]{deriv_table.csv_}
\label{fig:kraken_const_deriv}
\vspace{-6mm}
\end{figure}
\subsection{Comparison between Kraken Versions}
Beyond the massive speedup from partial-evaluation, Fig. \ref{fig:kraken_const_deriv} and \ref{fig:kraken_nqueens_rbtree} show the effect of the various compiler optimizations we described by disabling them one by one.
 Our main four optimizations have a strong positive effect on runtime, with the exception of lazy environment instantiation. Lazy environment instantiation helps massively on fib, and some on Deriv, but generally hurts the rest slightly.


\begin{figure}[h]
\caption{N-Queens}
\includegraphics[width=0.45\textwidth]{nqueens_table.csv_}
\includegraphics[width=0.45\textwidth]{slow_rbtree_table.csv_}
\label{fig:kraken_nqueens_rbtree}
\vspace{-4mm}
\end{figure}


\subsection{Comparison against Others}


To give a general idea of our current performance, we also show a Fibonacci benchmark that mostly exercises pure function-call speed and inlining as seen in Fig. ~\ref{fig:kraken_vs_world_fib}.
We include Python and Chez Scheme to give a general idea for where an exemplar slow and an exemplar fast dynamic language would fall.
With the benefit of our partial evaluation, compilation, and leaning upon mature WebAssembly implementations, we beat both, but this should be taken with a grain of salt, as this is a very limited micro-benchmark only meant to give a general sense of the order of magnitude of our performance.



\label{sec:eval1}
\begin{figure}[h]
\caption{Kraken vs. Others. Ordered by fastest to slowest}
\includegraphics[width=0.45\textwidth]{fib_table.csv_}
\includegraphics[width=0.45\textwidth]{rbtree_table.csv_}
\label{fig:kraken_vs_world_fib}
\end{figure}

%\label{sec:eval_nqueens}
%\begin{figure}[h]
%\caption{N-Queens}
%\includegraphics[width=0.45\textwidth]{nqueens_table.csv_}
%\includegraphics[width=0.45\textwidth]{slow_nqueens_table.csv_}
%\label{fig:kraken_nqueens}
%\end{figure}

%\label{sec:eval_nqueens}
%\begin{figure}[h]
%\caption{Kraken, N-Queens, absolute value and log-scale}
%\includegraphics[width=0.45\textwidth]{nqueens_table.csv_}
%\includegraphics[width=0.45\textwidth]{nqueens_table.csv_log}
%\label{fig:kraken_nqueens}
%\end{figure}
%\label{sec:eval_nqueensp}
%\begin{figure}[h]
%\caption{Kraken, N-Queens, absolute value and log-scale}
%\includegraphics[width=0.45\textwidth]{slow_nqueens_table.csv_}
%\includegraphics[width=0.45\textwidth]{slow_nqueens_table.csv_log}
%\label{fig:kraken_nqueensp}
%\end{figure}

%\label{sec:eval_cfold}
%\begin{figure}[h]
%\caption{C-Fold}
%\includegraphics[width=0.45\textwidth]{cfold_table.csv_}
%\includegraphics[width=0.45\textwidth]{slow_cfold_table.csv_}
%\label{fig:kraken_cfold}
%\end{figure}
%\label{sec:eval_cfold}
%\begin{figure}[h]
%\caption{Kraken, C-Fold, absolute value and log-scale}
%\includegraphics[width=0.45\textwidth]{cfold_table.csv_}
%\includegraphics[width=0.45\textwidth]{cfold_table.csv_log}
%\label{fig:kraken_cfold}
%\end{figure}
%\label{sec:eval_cfoldp}
%\begin{figure}[h]
%\caption{Kraken, C-Fold, absolute value and log-scale}
%\includegraphics[width=0.45\textwidth]{slow_cfold_table.csv_}
%\includegraphics[width=0.45\textwidth]{slow_cfold_table.csv_log}
%\label{fig:kraken_cfoldp}
%\end{figure}

%\label{sec:eval_deriv}
%\begin{figure}[h]
%\caption{Deriv}
%\includegraphics[width=0.45\textwidth]{deriv_table.csv_}
%\includegraphics[width=0.45\textwidth]{slow_deriv_table.csv_}
%\label{fig:kraken_deriv}
%\end{figure}
%\label{sec:eval_deriv}
%\begin{figure}[h]
%\caption{Kraken, Deriv, absolute value and log-scale}
%\includegraphics[width=0.45\textwidth]{deriv_table.csv_}
%\includegraphics[width=0.45\textwidth]{deriv_table.csv_log}
%\label{fig:kraken_deriv}
%\end{figure}
%\label{sec:eval_derivp}
%\begin{figure}[h]
%\caption{Kraken, Deriv, absolute value and log-scale}
%\includegraphics[width=0.45\textwidth]{slow_deriv_table.csv_}
%\includegraphics[width=0.45\textwidth]{slow_deriv_table.csv_log}
%\label{fig:kraken_derivp}
%\end{figure}

%\subsection{Comparison against state-of-the-art languages}
%\label{sec:eval3}

%\begin{figure}[h]
%\caption{Kraken vs. S.o.t.A.}
%\includegraphics[width=0.45\textwidth]{cfold_table.csv_}
%\includegraphics[width=0.45\textwidth]{rbtree_table.csv_}
%\label{fig:kraken_vs_world1}
%\end{figure}

%\begin{figure}[h]
%\caption{Kraken vs. S.o.t.A.}
%\includegraphics[width=0.45\textwidth]{deriv_table.csv_}
%\includegraphics[width=0.45\textwidth]{nqueens_table.csv_}
%\label{fig:kraken_vs_world2}
%\end{figure}

% \begin{figure}[h]
% \caption{Kraken vs. S.o.t.A. (Log)}
% \includegraphics[width=0.45\textwidth]{cfold_table.csv_log}
% \includegraphics[width=0.45\textwidth]{rbtree_table.csv_log}
% \label{fig:kraken_vs_world_log_1}
% \end{figure}
% \begin{figure}[h]
% \caption{Kraken vs. S.o.t.A. (Log)}
% \includegraphics[width=0.45\textwidth]{deriv_table.csv_log}
% \includegraphics[width=0.45\textwidth]{nqueens_table.csv_log}
% \label{fig:kraken_vs_world_log_2}
% \end{figure}

%As we noted before with the Fib(30) microbenchmark in Section \ref{sec:eval1}, we remain significantly slower than state-of-the-art compiled languages.
%This is particularly true for memory-intensive benchmarks due to our naive reference-counting and malloc/free implementations.
%However, our results are of a similar order of magnitude to the difference between the state-of-the-art compiled languages and dynamic scripting languages, like Python's results in the Fib(30) microbenchmark.
%We assert that is not a fundamental limitation because the classic f-expr slowness is being eliminated, as shown by Fig. \ref{fig:kraken_vs_newlisp1} and Fig. \ref{fig:kraken_vs_newlisp2}.
%In future work, we plan to expand our compile-time analysis and optimization to implement a modified, dynamic-language version of Perceus reference counting.
%With this change, we belive \krakenSpace can be competitive with these state-of-the-art languages.

%\subsection{Case Study: Red-Black Tree}
%\label{sec:casestudy}

%\begin{figure}[h]
%\caption{Kraken vs. S.o.t.A. - RB-Tree Focus}
%\includegraphics[width=0.4\textwidth]{rbtree_table.csv_}
%\includegraphics[width=0.4\textwidth]{rbtree_table.csv_log}
%\label{fig:kraken_vs_world_rbtree}
%\end{figure}


%To evaluate our partial evaluation algorithm and compiler, we extracted the benchmarks used by the Koka language project from their code repository and added Kraken versions, as well as implementing a naive Fibonacci microbenchmark ourselves to evaluate pure function call speed.\\
%With partial evaluation and the compiler optimizations listed above, we get fairly strong performance on purely numerical computations, such as the naive Fibonacci microbenchmark.
%Unfortunately, the overhead of our unsophisticated reference counting, dynamic type checking, and bounds checking causes poor performance on benchmarks involving data structures relative to mainstream programming language implementations.
%This is not a fundamental limitation, and will be addressed in future work, as recounted in the next section.
%It should be noted, however, that while the performance relative to established language implementations is very poor for the memory-intensive benchmarks (600-900x slower), we still realize a massive speedup compared to an unoptimized and non-partial-evaluated f-expr implementation (100,000x faster)!


\begin{table}[ht!]
% \vspace{-12pt}
\caption{\textbf{User study results on the reconstructed shape and details.} {\name} achieves the best results in coarse shape and details according to human perception compared to prior art~\protect\citesupp{feng2021learningsupp,wood2022densesupp}.}\label{tab:userstudy}
\vspace{2pt}
\resizebox{1\linewidth}{!} (Ours) & 35.06\% (Dense~\citesupp{wood2022densesupp})   & 10.39\% (\citesupp{MICAsupp,deng2019accuratesupp,wu2021synergysupp,guo2020towardssupp})      \\ \midrule[1pt]
Detail         & \textbf{80.52\%} (Ours) & 11.69\% (DECA~\citesupp{feng2021learningsupp})    & 7.79\% (\citesupp{danvevcek2022emocasupp,wang2022faceversesupp,chen2020selfsupp,yang2020facescapesupp})       \\ \bottomrule[1pt]
\end{tabular}
}
\vspace{-10pt}
\end{table}

\subsection{Evaluation Metrics}
\label{sec:metrics}

We report several evaluation metrics on our experiments. We use these metrics to quantize the human-likeness and physical plausibility of motion synthesis results.

% \textbf{Contact Space Human-Likeness} We evaluate the smallest distance between the synthesized manipulation and all collected ground truth data on the same object instance. We calculate the distance under a contact space, i.e., we extract the contact positions with normal directions, and then calculate a $L_2$ loss on them.
% 
% \textbf{Pose Space Human-Likeness} This metric is similar to the contact space human-likeness, but the $L_2$ distance is calculated directly on the MANO joint positions.

\textbf{Contact-Movement Consistency} We evaluate whether the object's movement can align with the contact forces produced by hand-object contacts, using the same physics model as in ManipNet\cite{zhang2021manipnet}. For articulated objects, we assume there are free opposite forces at the spin axis of the object. We calculate the proportion of frames that the contacts align with the object motion.

\textbf{Articulation Consistency} For articulated objects, we evaluate whether the hand pose can manipulate the object in a human-like manner. In particular, for each part of the object in each frame, we compute the torque of all contact points w.r.t. the object's spin axis if a unit force is applied along the normal direction of the contact point. If the maximal attitude of the torque with the same direction of object rotation exceeds a threshold, we regard such frame as qualified. We calculate the proportion of qualified frames.


\textbf{Penetration Rate} We compute the mean penetration proportion of hand vertices for each sequence. We regard penetrations within a small threshold $\lambda$=5mm as not penetrated since small penetrations can be seen as contacts made by a soft hand in real.

\textbf{Perceptual Score} We collect human perceptual scores to judge the naturalness of the motion sequences. We ask people not familiar with motion synthesis to give discrete perceptual scores for the results and calculate a mean score for each category using each baseline method. The detailed approach is left to the supplementary material.


\subsection{Baselines}
\label{sec:baselines}
As introduced in Section~\ref{sec:relatedwork}, there are only a few works about dynamic HOM generation using a learning-based method, while there are various techniques that could have the potential to be used in this task. Consequently, we design the following baselines.

\textbf{GraspTTA\cite{grasptta}:} GraspTTA proposed a strong baseline in static grasp generation. We use it to generate static grasps of several key snapshots in manipulation and then refine the result at test time using the TTA loss (We do not refine the network at test time). The dynamic HOM animation is thus generated from interpolation based on these snapshots.

\textbf{ManipNet\cite{zhang2021manipnet}:} Benefiting from carefully designed geometric sensors, ManipNet has shown a strong ability to generalization on generic object manipulation synthesis tasks. Different from our setting, ManipNet assumes additional input of wrist trajectory. To compare it with other methods, we provide it with the wrist trajectory generated by CAMS as input (and thus it only differs at fingers).

Though both GraspTTA and ManipNet have their own mechanisms to improve synthesis quality (\eg reduce penetration), they can also benefit from our optimization-based motion synthesizer (optimizing $\mathcal L_{penetr}$ to reduce penetration). We combine all baselines with an optimization stage (denoted as ``w/ opt'') and compare them with the CAMS-CVAE motion planner.

\subsection{Ablation Studies}
\label{sec:ablation}

\textbf{Remove Contact Optimization:} To demonstrate the advantages of contact optimizations in motion synthesis, we train a baseline model with the contact optimizations removed (CAMS-).

Besides removing the contact optimization in the synthesizer, we also did several ablation studies using different representation spaces of fingers. These experiments are left to our supplementary material.

\subsection{Comparison}
\label{sec:compare}

\textbf{Quantitative Results} Table~\ref{tab:quantitative_result} and Table~\ref{tab:user_study} show the quantitative results of our method and all baselines. Our method outperforms previous work on all tasks, even if contact optimization is applied to them.

An observation is that after applying the offline optimization stage, the performance gain of our method is significantly higher than the baselines. This can be explained by that the $\mathcal L_{contact}$ term in contact optimization takes the contact targets from our planner as input, and it is the key to producing gradients guiding the finger placement. Without intermediate contact target information, the optimization can only leverage the $\mathcal L_{penetr}$ loss term, and it may push the finger out of the object in unpredictable directions.

\textbf{Qualitative Results} Besides \figref{fig:Head}, \figref{fig:result-modediversity} shows that our method can generate diverse grasp modes on a single object instance. \figref{fig:result-comparison} shows that our method can generate reasonable poses given complex object shapes. We also show full result demonstrations in our video, including the generated whole HOM processes for different tasks, robustness to different object shapes and sizes, comparison between baseline methods, and diversity of generated manipulation styles.

% \begin{figure}
%     \centering
%     \includegraphics[width=\linewidth]{images/motionseq.png}
%     \caption{\textbf{Visualization of the hand motion sequence manipulating scissors.} The sequence is split into 3 stages, where 4 frames are shown in each stage. }
%     \label{fig:result-motionsequence}
%     \vspace{-5pt}
% \end{figure}




% \label{sec:ablation}

% \textbf{Importance of Normal Vector in CAMS}
% We argue that the normal vector $\mathbf{V}_i$ in $\mathbf{R}=\{(\mathbf{C}_i, \mathbf{V}_i, \mathbf{N}_i)\}_{i=1}^5$ is % essential not only for finding correct contact using Synthesizer, but also for encoding the manipulation. We conduct % several experiments to study the effect of normal vector in Planner, by using different $\lambda_{vec}$. We find that % normal vector plays a deterministic role in manipulation diversity encoding. As shown in \emph{fig!!}, the % $\mathcal{L}_{KLD}$ decreases significantly when we lower the $\lambda_{vec}$ from 200 to 0. We know that a lower % $\mathcal{L}_{KLD}$ doesn't always creates a better result, on the one hand, a lower $\mathcal{L}_{KLD}$ means that the % sampled latent code from the standard Gaussian distribution is more similar to the latent code sampled from the real % latent space, on the other hand, a lower $\mathcal{L}_{KLD}$ might come from KL-vanishing, resulting in the degradation % of the network. In our experiment, when the $\mathcal{L}_{KLD}$ decreases under 0.18, the network starts to overfit and % there is less and less variation for generated manipulation. There are original three manipulation types for % \emph{Laptop} and we could generate them randomly by sampling a latent code from the standard Gaussian distribution with % $\lambda_{vec}=100$, as shown in \emph{fig!!}, however, when we lower the $\lambda_{vec}$ to 1, or just abandon it, there % is only one manipulation type and it is just an average of three manipulation types, causing lots of problems, as shown % in \emph{fig!!}. It is reasonable considering that different manipulation types look similar from the perspective of % contact position: there is no difference when we grasp different surfaces of the \emph{Laptop} and grasp different % positions of the same surface if they have similar $L_2$ distance. However, the normal vector of different surfaces % changes drastically, and it's an important clue when we decide the way to grasp something.

% \textbf{Finger Embedding in CAMS}
% joint conflicts?

% \textbf{contact reference along the whole sequence}
% more like a projection of contact