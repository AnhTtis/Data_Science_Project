\section{Conclusions}
\label{sec:conclusion}

We propose \backronym, a method to address the challenging task of dense correspondences across images of an object or object category captured in-the-wild. \backronym utilizes noisy and sparse pseudo-correspondences in pre-trained ViT feature space to build an accurate and dense consistent mapping from image to a canonical space. Extensive qualitative and quantitative experiments show that \backronym works in low-shot settings and can deal with extreme variations in pose, background, occlusion, and object deformations.
We also propose a new metric $\kcycle$ to evaluate the consistency of keypoint predictions over a set of images beyond pair-wise consistency.
%A limitation of our work is its dependence on the quality of features learned by the pre-trained ViTs. A way to mitigate this limitation could be to manually annotate seed matches. Despite this limitation, 
\backronym can obtain consistent dense mappings competitive with supervised counterparts with just a few images. In future work, we will explore applications of \backronym in other few-shot downstream tasks such as reconstruction, pose estimation and tracking.
