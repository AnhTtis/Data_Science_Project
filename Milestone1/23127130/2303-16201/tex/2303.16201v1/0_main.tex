\documentclass[10pt,twocolumn,letterpaper]{article}
\usepackage[numbers,sort&compress]{natbib}
\usepackage{iccv}
\usepackage{times}
\usepackage{epsfig}
\usepackage{graphicx}
\usepackage{amsmath}
\usepackage{amssymb}

% Include other packages here, before hyperref.
% Ours
\usepackage{booktabs}
\usepackage{multirow}
\usepackage{enumitem}
\usepackage[hypcap=false,font=small]{caption}
% \usepackage{subcaption}
% \usepackage{tabu}
\usepackage[table, svgnames]{xcolor}
\usepackage{microtype}
\usepackage{xspace}

% If you comment hyperref and then uncomment it, you should delete
% egpaper.aux before re-running latex.  (Or just hit 'q' on the first latex
% run, let it finish, and you should be clear).
\usepackage[pagebackref=true,breaklinks=true,letterpaper=true,colorlinks,bookmarks=false]{hyperref}

% Comments
\newcommand{\option}[1]{{\color{black}{\small\bf\sf [#1]}}}
\newcommand{\kg}[1]{{\color{blue}{\small\bf\sf [KG: #1]}}}
\newcommand{\todo}[1]{{\color{red}{\small\bf\sf [TODO: #1]}}}
\newcommand{\vjnote}[1]{{\color{magenta}{\small\bf\sf [Varun: #1]}}}
\newcommand{\aknote}[1]{{\color{cyan}{\small\bf\sf [Abhishek: #1]}}}

% Math shortcuts
\newcommand{\backronym}{ASIC\xspace}
\newcommand{\myparagraph}[1]{\medskip\noindent\textbf{#1}}

\newcommand{\bbox}{\text{bbox}}
\newcommand{\alphapck}{\alpha_\bbox}
\newcommand{\kcycle}{\text{k-CyPCK}}
\newcommand{\cycle}{\text{-CyPCK}}

\newcommand{\I}{\mathbf{I}}
\newcommand{\Ia}{\I^\text{a}}
\newcommand{\Ib}{\I^\text{b}}
\newcommand{\Iatob}{\I^\text{a $\rightarrow$ b}}
\newcommand{\F}{\mathbf{F}}
\newcommand{\Fa}{\F^\text{a}}
\newcommand{\Fb}{\F^\text{b}}
\newcommand{\f}{\mathbf{f}}
\newcommand{\fa}{\f^\text{a}}
\newcommand{\fb}{\f^\text{b}}
\newcommand{\p}{\mathbf{p}}
\newcommand{\pa}{\p^\text{a}}
\newcommand{\pb}{\p^\text{b}}
\newcommand{\A}{\boldsymbol{\Phi}_\text{align}}
\newcommand{\G}{\mathbf{G}}
\newcommand{\C}{\mathbf{C}}
\newcommand{\Ca}{\C^\text{a}}
\newcommand{\Cb}{\C^\text{b}}
\newcommand{\cc}{\mathbf{c}}
\newcommand{\cca}{\cc^\text{a}}
\newcommand{\ccb}{\cc^\text{b}}
\newcommand{\Irec}{\I_\text{Recon}}
\newcommand{\M}{\mathbf{M}}
\newcommand{\Mrec}{\M_\text{Recon}}
\newcommand{\loss}{\mathcal{L}}
\newcommand{\T}{\mathcal{T}}
\newcommand{\W}{\mathcal{W}}
\newcommand{\Id}{\mathcal{I}}

\linepenalty=1000

\makeatletter
\g@addto@macro{\endtabular}{\rowfont{}}% Clear row font
\makeatother
\newcommand{\rowfonttype}{}% Current row font
\newcommand{\rowfont}[1]{% Set current row font
\gdef\rowfonttype{#1}#1\ignorespaces%
}
\makeatother
% Support for easy cross-referencing
\usepackage[capitalize]{cleveref}
\crefname{section}{Sec.}{Secs.}
\Crefname{section}{Section}{Sections}
\Crefname{table}{Table}{Tables}
\crefname{table}{Tab.}{Tabs.}

\iccvfinalcopy % *** Uncomment this line for the final submission

\def\iccvPaperID{4552} % *** Enter the ICCV Paper ID here
\def\httilde{\mbox{\tt\raisebox{-.5ex}{\symbol{126}}}}

% Pages are numbered in submission mode, and unnumbered in camera-ready
\ificcvfinal\pagestyle{empty}\fi

\begin{document}

%%%%%%%%% TITLE
\title{\vspace{-2em} ASIC: Aligning Sparse in-the-wild Image Collections}

% \author{First Author\\
% Institution1\\
% Institution1 address\\
% {\tt\small firstauthor@i1.org}
% For a paper whose authors are all at the same institution,
% omit the following lines up until the closing ``}''.
% Additional authors and addresses can be added with ``\and'',
% just like the second author.
% To save space, use either the email address or home page, not both
% \and
% Second Author\\
% Institution2\\
% First line of institution2 address\\
% {\tt\small secondauthor@i2.org}
% }
\author{Kamal Gupta\textsuperscript{\rm 1,2}, Varun Jampani\textsuperscript{\rm 1}, Carlos Esteves\textsuperscript{\rm 1}, \\
Abhinav Shrivastava\textsuperscript{\rm 2}, Ameesh Makadia\textsuperscript{\rm 1}, Noah Snavely\textsuperscript{\rm 1},  Abhishek Kar\textsuperscript{\rm 1}\hfill \\ \\
\textsuperscript{\rm 1}Google \quad \quad
% {\tt\small \{varunjampani,machc,snavely,makadia,abhiskar\}@google.com}
\textsuperscript{\rm 2}University of Maryland, College Park\\
% {\tt\small \{kampta,abhinav\}@cs.umd.edu}
}

\twocolumn[{%
\renewcommand\twocolumn[1][]{#1}%
\maketitle
\begin{center}
    \centering
    \vspace{-1.2em}
    \includegraphics[width=0.97\textwidth]{imgs/Teaser.pdf}
    \vspace{-0.5em}
    \captionof{figure}{\textbf{Globally consistent and dense aligments with ASIC.} Given a small set ($\sim$10-30) of images of an object or object category captured in-the-wild, our framework computes a dense and consistent mapping between all the images in a self-supervised manner. \textbf{First row:} Unaligned sets of images from the SAMURAI (Keywest) and SPair-71k (Cow) datasets. \textbf{Second row:} Dense correspondence maps produced by our method. \textbf{Third row:} Image in the first column warped to the images in columns 2-5. %\textbf{Fourth row:} An edit propagation application. We can propagate edits on any one image to all other images in the collection via our dense correspondence maps.
    }
    \label{fig:teaser}
\end{center}
}] 

\maketitle
% Remove page # from the first page of camera-ready.
\ificcvfinal\thispagestyle{empty}\fi


%%%%%%%%% ABSTRACT
\begin{abstract}
\vspace{-3mm}
  We present a method for joint alignment of sparse in-the-wild image collections of an object category. Most prior works assume either ground-truth keypoint annotations or a large dataset of images of a single object category. However, neither of the above assumptions hold true for the long-tail of the objects present in the world. We present a self-supervised technique that directly optimizes on a sparse collection of images of a particular object/object category to obtain consistent dense correspondences across the collection. We use pairwise nearest neighbors obtained from deep features of a pre-trained vision transformer (ViT) model as noisy and sparse keypoint matches and make them dense and accurate matches by optimizing a neural network that jointly maps the image collection into a learned canonical grid. Experiments on CUB and SPair-71k benchmarks demonstrate that our method can produce globally consistent and higher quality correspondences across the image collection when compared to existing self-supervised methods.
  Code and other material will be made available at \url{https://kampta.github.io/asic}.
\vspace{-5mm}
\end{abstract}

%%%%%%%%% BODY TEXT
%%%%%%%%% BODY TEXT
\section{Introduction}
\label{sec:intro}
\begin{figure}[t]
\begin{center}
    \includegraphics[width=1\linewidth]{figures/teaser.pdf}
\end{center}
\vspace{-0.1in}
\caption{\textbf{{\em Foggy} vs {\em Clear} NeRF.} Our \ournerf gets rid of reconstruction errors manifested as foggy ``floaters" in the density volume without additional input or significant computational overhead. 
%
Below are density profiles along a given ray before and after our geometry correction procedure, where we discard density peaks corresponding to floaters.
}
\label{fig:teaser}
\vspace{-0.2in}
\end{figure}



%The emergence of 
Neural Radiance Fields (NeRFs)~\cite{mildenhall2020nerf}  %and its variants 
have made revolutionary contributions in %photo-realistic 
novel view synthesis~\cite{barron2021mip,barron2022mip}, 
autonomous driving~\cite{rematas2022urban,tancik2022block}, digital human~\cite{hong2022headnerf,zhao2022humannerf}, and 3D content generation~\cite{eg3d,poole2022dreamfusion,lin2022magic3d}.
%by leveraging a multi-layer perceptron (MLP) to implicitly model the mapping from input 5D coordinates (i.e., 3D coordinates $\mathbf{x} = (x,y,z)$ and 2D viewing directions $\mathbf{d}=(\theta,\phi)$) to volume density $\sigma$ and view-dependent emitted radiance color $\mathbf{c} = (r,g,b)$. 
%
%They then use traditional volume rendering mechanisms on the obtained continuous 5D function (i.e., MLP) to generate novel views. 
To date, unfortunately, most NeRF-based methods encounter challenges when tackling large-scale cluttered scenes (e.g., Fig.~\ref{fig:teaser}):
\begin{enumerate}[leftmargin=0.16in, topsep=2pt,itemsep=-1ex,partopsep=1ex,parsep=1ex]
\item Input observations used for NeRF are often too sparse  compared to forward-facing or synthetic looking-inward scenes;
%\item Recovering fine-grained objects within a large volume is challenging for NeRF; %in capturing details accurately.
\item View-dependent visual effects give rise to ambiguity, resulting in a ``foggy" density field as shown in Fig.~\ref{fig:teaser}. 
%
Such artifacts are particularly pronounced in indoor scenes strewn with view-dependent appearances, such as specular highlights, glossy surface reflections from man-made objects. 
\end{enumerate}

Despite attempts to enhance NeRF's rendering quality given suboptimal input, such as using 3D conical frustums~\cite{barron2021mip,barron2022mip}, physically-grounded augmentations~\cite{chen2022aug}, and misalignment correction~\cite{jiang2022alignerf},  these challenges have yet to be fully resolved.
%
Depth supervision~\cite{deng2022depth, wei2021nerfingmvs} or proxy geometry~\cite{xu2021scalable,wu2022scalable} images can help alleviate the challenges in handling large-scale with sparse input, at the expense of %but they come at the cost of requiring 
expensive pre-processing or additional input.
%
Another line of work~\cite{wang2021neus, oechsle2021unisurf, wang2022neuris} achieves better reconstruction of surface geometry by using signed distances instead of volume density as scene representation. However, they sacrifice the ability to synthesize photo-realistic novel views.

%We observe that NeRF has been suffering from foggy ``floater" artifacts in large-scale cluttered scenes.
%
%Such artifacts are particularly pronounced in indoor scenes strewn with view-dependent appearances from man-made objects. 
%
To address the above issues, we propose an extension to NeRF, dubbed as {\bf \ournerf}, which enforces effective {\em appearance} and {\em geometry} constraints conducive to accurate colors and 3D densities estimation. We believe \ournerf can contribute beyond novel view synthesis, such as NeRF object detection~\cite{hu2022nerf}, NeRF object segmentation~\cite{zhi2021place, liu2022unsupervised, fan2022nerf,ren2022neural}, and NeRF registration~\cite{goli2022nerf2nerf}, where the rooms for improvement are substantial if more accurate color and density estimation are available.

Correspondingly, there are two steps in \ournerf. First, for appearance correction, the view-independent and view-dependent color components are predicted from the underlying 3D scene, which is combined to produce the final color estimation (Fig.~\ref{fig:toaster}).
%
The view-independent component (diffuse color and shading) captures the overall scene color, while the view-dependent component (highlights or reflections) captures color variations due to changes in viewing angle.
%
\ournerf then discards these view-dependent appearances in the training views to prevent them from interfering with the density estimation.
%
Second, a simple and effective geometry correction procedure will be performed to further eliminate the foggy ``floaters" or density errors. This geometry correction procedure is based on an assumption in line with traditional ray tracing in computer graphics.
\begin{comment}
% xh: basically copying method
On the other hand, ClearNeRF performs a geometric correction procedure performed on each traced ray during inference to refine the density estimation and better tackle the floater artifacts. 
%
The geometry correction procedure assumes that there should only be one salient peak along each traced ray during NeRF inference. 
Only the salient peak closest to the ray origin (the camera center) corresponds to  true geometry while the others will be manifested as foggy floaters hovering in the density volume. 
%
This assumption is in line with traditional ray tracing in computer graphics where in the absence of noise, only one intersection per ray should be returned to indicate the closest ray-object intersection.
%
\end{comment}
%%%%%%%%%%%
%As shown in Fig.~\ref{fig:teaser}, when reconstructing an indoor scene with sparse input and highly view-dependent objects, NeRF produces severe floating artifacts due to its attempt to explain view-dependent appearances.
%
Experiments verify that our proposed \ournerf can effectively get rid of floater artifacts without additional input.% or significant computational overhead. 


In summary, our contributions include the following:
\begin{itemize}[leftmargin=0.16in, topsep=2pt,itemsep=-1ex,partopsep=1ex,parsep=1ex]
    \item We propose a concise method for decomposing view-independent and view-dependent appearance during NeRF training and eliminate the interference of view-dependent appearance.
    \item We propose a geometric correction procedure performed on each traced ray during inference to refine the density estimation and better tackle the floater artifacts.
    \item Extensive experiments and ablations verify the effectiveness of our core designs and results in improvements over the vanilla NeRF and other state-of-the-art alternatives.
    %without additional computational resources or other inputs.
\end{itemize}




\section{Related work}

In recent years, large language models have improved significantly in various NLP areas, especially in generative tasks.
A lot of new concepts were introduced, starting from attention mechanism~\cite{bahdanau2014neural}, transformers~\cite{vaswani2017attention} to multitask, learning from instructions~\cite{wang2022super} and human feedback~\cite{wang2021putting}.
The last becomes extremely popular in the generative context including machine translation. 
% new architectures were proposed~\cite{radford2019language,brown2020language}, and, 
Consequently, the usage of machine translation tools has become a necessary compound for understanding a foreign language. 
Unfortunately, like other neural network-based algorithms, these tools are vulnerable to adversarial examples~\cite{DBLP:journals/corr/GoodfellowSS14}. 
Starting from text classification \cite{li-etal-2020-bert-attack,DBLP:conf/acl/EbrahimiRLD18,Li2018TextBuggerGA}, vulnerability and robustness received a lot of attention in the NLP community. 
For MT systems one of the pioneering works was~\cite{ebrahimi2018adversarial}, where authors proposed a character-level approach to generate adversarial examples.
% that neural MT systems are vulnerable to character-level perturbations, where only a few symbols in an input query are subject to change. 
Inheriting HotFlip~\cite{ebrahimi-etal-2018-hotflip} there were considered white-box and black-box settings, where only a few symbols in an input query are subject to change imitating typos.

While white-box optimization may yield stronger adversarial perturbations it implies access to the model's architecture and weights which is impractical in the case of online MT tools. 
In~\cite{wallace} there was considered a white-box universal approach to a targeted attack on conditional text generation. 
The authors modeled perturbation as an insertion of a trigger, a token sequence of small length, that results in a generated sequence similar to the target set of sentences. 
While during experiments certain triggers cause a model to produce sensitive racist output, they are generally meaningless and similarly to character-level attacks are easy to detect. 
Authors of~\cite{guo-etal-2021-gradient,9747475} reported high attack transferability making this approach promising for black-box setup, however,  the research is limited only to the GPT-2 model for generation task. 
The above papers use greedy techniques to walk through the searching space during the optimization, on the other hand, attacks on NLP models could be found via projection onto embeddings~\cite{wallace}, and for MT task this was discovered in~\cite{Seq2Sick,Sadrizadeh2023TargetedAA,sadrizadeh2023transfool}. 
In~\cite{zhang2021crafting}, it was shown that black-box optimization may yield transferable word-level attack that fools online translation tools, for example Baidu and Bing translators. 
This work proposed to use the word saliency as the measure of uncertainty. 
Masking candidates the saliency was estimated via additional BERT model~\cite{devlin2018bert}  which lead to strong readable and imperceptible adversaries, however, neither human evaluation was performed nor quantities results for online tools were given. In~\cite{wan2022paeg}, a gradient-based approach to generate phrase-level adversarial examples for neural MT systems was proposed. Similarly to~\cite{zhang2021crafting}, it is proposed to estimate the vulnerable word positions are estimated in an input phrase with the use of gradient information and replace corresponding words by the candidates computed with an auxiliary model.

% \mynote{actually we may underline that we do not generate adversarial examples per se (we arent aimed at misclassification), but rather generate inputs that are been translated though they should not}

% \mynote{TODO: Maybe add more criticism of zhang2021crafting and point out the differences in our approach.}

% \todopa{}{}{
% https://www.semanticscholar.org/paper/AdvAug\%3A-Robust-Adversarial-Augmentation-for-Neural-Cheng-Jiang/1e7d3a9846da556bc7b84ae1410d257b89448c30
% }

%\todopa{}{}{
%https://www.semanticscholar.org/paper/A-Targeted-Attack-on-Black-Box-Neural-Machine-with-Xu-Wang/2a46eb47e8742be29b16a5b83dc1a38616b24ce6
%}

%\todopa{}{}{https://www.semanticscholar.org/paper/PAEG\%3A-Phrase-level-Adversarial-Example-Generation-Wan-Yang/a6dd2a8debb5d5324c4f2be7fb7bb52ce109cbaf}

% \todopa{}{}{
% https://download.huan-zhang.com/events/srml2022/accepted/bhandari22lost.pdf
% }

%\todopa{}{}{http://fan-yao.com/paper/2021_SEED_nmtstroke.pdf}

% \todopa{}{}{https://arxiv.org/pdf/2303.01068v1.pdf}

%\todopa{kosinski2023theory}
%    {Theory of mind may have spontaneously emerged in large language models}
%    {https://arxiv.org/pdf/2302.02083.pdf}
%    {We can say that large language models are very clever now, etc...}

% \todopa{ebrahimi2018adversarial}
%     {On adversarial examples for character-level neural machine translation}
%     {https://arxiv.org/pdf/1806.09030.pdf}
%     {Very related work (see beamsearch in the text also)...}

% \todopa{zhang2021crafting}
%     {Crafting adversarial examples for neural machine translation}
%     {https://github.com/JHL-HUST/AdvNMT-WSLS}
%     {Very related work. See: ``Besides, WSLS exhibits strong transferability on attacking Baidu and Bing online translators.''}

% \todopa{sadrizadeh2023transfool}
%     {TransFool: An Adversarial Attack against Neural Machine Translation Models}
%     {https://arxiv.org/pdf/2302.00944.pdf}
%     {Very related work!}
\section{Method}
\label{sec: method}
% This section introduces the rendering pipeline of our proposed hierarchical compositional scene. 
% our pipeline consists of three processes, including decomposing the text into editable 3D layout, rendering the compositional views with local (object) NeRFs and global (scene) NeRF and the joint optimization on these hierarchical 3D representations.

% Note that the transformation between the object and the scene frame is defined by ${p}_o$ and ${D}_o$. 
%
% Next, we build a residual connection to add ${\sigma}_o$ and the referenced global color, and the rendering result will be used to calculate the SDS loss based on the global text.  
% Fig.~\ref{fig:framework} illustrates our pipeline, which consists of three main components, including the editable 3D scene layout based on multi-object text (Sec.~\ref{ssec:layout}), the scene rendering pipeline that composites the predictions from all local NeRFs (Sec.~\ref{ssec:render}), and the joint optimization on both local and global representation models (Sec.~\ref{sec:optimization}).
% To elaborate, our editable 3D scene layout represents a global frame of the scene by decomposing it into a set of local frames, where each is parameterized by a local NeRF, a 3D bounding box, and a corresponding local text prompt.
% For instance, the text prompt `A teddy bear and a stuffed monkey sit side by side' is interpreted as a 3D scene layout, as shown in Fig.~\ref{fig:framework}.  
% The whole 3D layout, \ie, scene frame, consists of two 3D bounding boxes, \ie local frames \#1 and \#2, with specific local text prompts, \ie, `a teddy bear' and `a stuffed monkey'. 
% %
% To render the scene view, we first calculate the ray-box intersections between the boxes and rays $({\boldsymbol{r}}_o, \boldsymbol{\phi}_d, {\boldsymbol{\theta}}_d)$, where the ${\boldsymbol{r}}_o$ is the ray origin and the $({\boldsymbol{r}}_o, \boldsymbol{\phi}_d)$ is its direction.
% Then, to infer each object's properties in local NeRFs, we sample the global points $({\boldsymbol{x}}_g, {\boldsymbol{y}}_g, {\boldsymbol{z}}_g)$ in the global frame within the ray-box intersection intervals and project them into the normalized local location $({\boldsymbol{x}}_l, {\boldsymbol{y}}_l, {\boldsymbol{z}}_l)$ in the local frame.
% %
% Given the local sampling points $({\boldsymbol{x}}_l, {\boldsymbol{y}}_l, {\boldsymbol{z}}_l)$, the implicit local NeRF ${\boldsymbol{\theta}}_l$ outputs four pseudo-color channels ${\boldsymbol{C}}_l$ and density $\boldsymbol{\sigma}$, which can be used to render a local view of the local frame to match its local text prompt.
% %
% We further calibrate the predicted pseudo-color $\boldsymbol{C}_l$ from local frames by adding the global embeddings ${\boldsymbol{emb}}_g$ to improve the global view consistency.
% Then, the calibrated predictions after composition are used to reconstruct the scene view by volumetric rendering along the rays.
% %
% Lastly, the rendered views based on local and global frames are guided by score distillation sampling loss $\nabla \mathcal{L}_{\text{SDS}}$~\cite{poole2022dreamfusion} to optimize all the learnable parameters. 
To resolve the issue of guidance collapse, our principal strategy is to \textit{decompose the scene into reusable components and compose/recompose them into a unified and consistent one}.
This enables flexible control over the generated content with direct use of prompts and box layouts, as illustrated in \cref{fig:teaser}.
%
Our proposed CompoNeRF confers several key benefits:
1) \textbf{Semantic Coherence}: It reliably creates 3D objects with detailed textures and global consistency, exemplified by authentic light interactions, such as reflections on the bed surface.
2) \textbf{Modularity and Reusability}: CompoNeRF functions as an ensemble of independently trained NeRF models. These can be efficiently stored and later retrieved from a cached dataset, enabling their reuse in various cases.
3) \textbf{Editability}: Our approach allows for flexible scene modification, such as interchanging the lamp for a vase filled with sunflowers or altering its scale, by simply adjusting the box dimensions for later finetuning. This feature enhances flexibility and creative possibilities. 


% Furthermore, the usage of layout boxes enables more flexible control over the generated content compared with the intricate sketch shape in Latent-NeRF\cite{metzer2022latent}. 
\begin{figure*}[t]
    \centering
    \includegraphics[width=0.9\linewidth]{figures/method.pdf}
    % \vspace{-12pt}
    \caption{\textbf{Framework Overview}.
The CompoNeRF model unfolds in three stages: 1) Editing 3D scene, which initiates the process by structuring the scene with 3D boxes and textual prompts; 2) Scene rendering, which encapsulates the composition/recomposition process, facilitating the transformation of NeRFs to a global frame, ensuring cohesive scene construction. Here, we specify design choices between density-based or color-based(without refining density) composition; 3) Joint Optimization, which leverages textual directives to amplify the rendering quality of both global and local views, while also integrating revised text prompts and NeRFs for refined scene depiction.
  % The model is structured into three components: Composition, Decomposition, and Recomposition. Composition deals with the foundational setup, detailed with choices for density-based and color-based composition. Decomposition utilizes the modularity of the CompoNeRF feature, caching each NeRF module offline for efficient recalibration. Recomposition reuses these cached NeRFs and adjusts the semantic context, providing a revised output with the inclusion of the offline NeRF enhancements.
    % Our model consists of two branches where the upper part is individual NeRFs, and the lower part denotes global calibration with our tailored composition model. The specific designs for density-based and color-based composition modules are highlighted. 
    % CompoNeRF consists of three parts: 1). The editable 3D scene layout configures the scene representations with 3D boxes and text prompts; 2).  The scene rendering includes the global calibration and the compositional process; 3). The joint optimization applies global and local text guidance on global and local render views.
    % The global frame (scene space) contains a set of local frames. Each is  represented by a local NeRF associated with a 3D box and text prompt defined by the editable 3D layout.
    % The scene view is volumetric rendered by sampling the points $({\boldsymbol{x}}_g, \boldsymbol{y}_g, \boldsymbol{z}_g)$ intersected with any local frame along the ray $(\boldsymbol{r}_o, {\boldsymbol{\phi}}_d, \boldsymbol{\theta}_d)$.
    % The sampling points are first inferred through the local NeRF with the local frame locations $({\boldsymbol{x}}_l, \boldsymbol{y}_l, \boldsymbol{z}_l)$ projected from the global location $({\boldsymbol{x}}_g, \boldsymbol{y}_g, \boldsymbol{z}_g)$.
    % And then, all the local predictions are calibrated by a global MLP with conditional input to render the scene view.
    % During the optimization, the text guidance is applied to both local views predicted by local frames only and global views predicted by the composition of all local frame predictions.
    }
    \label{fig:framework}
    % \vspace{-8pt}
\end{figure*}

\subsection{Preliminaries}
Defining individual object bounding boxes as \textit{local frames} and the overall scene coordinate system as the \textit{global frame}, we build the foundation of NeRF and diffusion processes.

\label{sec:background}
\noindent \textbf{3D Representation in Latent Space.}
Our methodology capitalizes on the state-of-the-art text-to-image generative model—Stable Diffusion as described by Rombach et al\cite{rombach2022high}.
We build upon the Latent-NeRF framework~\cite{metzer2022latent}, which computes latent colors for individual objects by considering their sample positions within a localized frame. Specifically, it maps a three-dimensional point in local coordinates \(\boldsymbol{x}_l = (x_l, y_l, z_l)\) to a volumetric density \(\boldsymbol{\sigma}_l\) and an associated color \(\boldsymbol{C}_l\), expressed as \((\boldsymbol{C}_l, \boldsymbol{\sigma}_l) = f_{\boldsymbol{\theta}_l}(x_l, y_l, z_l)\). Here, \(f\) represents a Multi-Layer Perceptron (MLP) characterized by parameters \(\boldsymbol{\theta}_l\).
 This NeRF-generated color is then assessed in the context of the Stable Diffusion model, using text prompts to guide NeRF toward spatially coherent inference with intricate context.
% to infer pseudo-color for each object using local NeRF.
% Specifically, the representation maps a point $\boldsymbol{x}_l = \left({x}_l, {y}_l, {z}_l\right)\in [-1, 1]$ in the local frame to its corresponding volumetric density $\boldsymbol{\sigma}_l$ and emitted color $\boldsymbol{C}_l$, \ie,  $\left(\boldsymbol{C}_l, {\boldsymbol{\sigma}_l}\right)=\boldsymbol{\theta}_{_l}\left({x_l}, {y}_l, {z}_l\right)$.
% The predicted pseudo-color is fed forward into the decoder of the Stable Diffusion model to obtain the final rendering result.

\noindent \textbf{Volume Rendering with Multiple Objects.}
% For each local frame $j$ with NeRF parameterized as $\theta_j$, we follow original NeRF design\cite{nerf} to integrate $(\boldsymbol{C}_l, \boldsymbol{\sigma}_l)$ of   sampled points from any hit ray $r_l=(\boldsymbol{o}_l, \boldsymbol{d}_l)$ by,
% For consistent scene rendering, object transmittance $T_k$ must be recalculated in the global frame based on independent properties inferred from local NeRFs. Hence, we sort predictions according to their distance to $\boldsymbol{o}_g$. 
% Similar to \cref{eq:volrend}, global color $\hat{\boldsymbol{C}}_g$ of ray $\boldsymbol{r}_g=(\boldsymbol{o}_g, \boldsymbol{d}_g)$ is predicted by the volumetric rendering integrating over $m$ objects,
We extend the volume rendering process to accommodate multiple objects by assigning each a local frame, denoted as $j$, with NeRF parameters $\boldsymbol{\theta}_{l, j}$. Drawing from the foundational NeRF approach \cite{nerf}, in each local frame, we integrate the color $\boldsymbol{C}_l$ and density $\boldsymbol{\sigma}_l$ for points $\boldsymbol{x}_l$ sampled along a ray $\boldsymbol{r}_l$, emanates from the camera origin $\boldsymbol{o}_l$ in direction $\boldsymbol{d}_l$. This is formalized in the predicted color integration for $\hat{\boldsymbol{C}}_l$ as:
{\setlength\abovedisplayskip{2pt}
\setlength\belowdisplayskip{2pt}
\begin{equation}
\label{eq:volrend}
{\hat{\boldsymbol{C}}_l}({\boldsymbol{r}_l})=\sum_{k=1}^{N} T_{l, k} \left(1-\exp \left(-\sigma_{l, k} \delta_k\right) \right) {\boldsymbol{C}}_{l,k},
\end{equation}}where $T_{l, k}=\exp \left(-\sum_{j=1}^{k-1} \sigma_{l,j} \delta_j\right)$ represents the transmittance to the $k$-th of total $N$ sample, calculated exponentially over the cumulative density along $\boldsymbol{r}_l$, and $\delta_k$ is the interval between adjacent samples.
%
To synthesize a coherent scene, we transition from processing individual local frames to a collective global frame. Within this global context, we reconcile object attributes inferred from their individual local NeRFs for refined $\boldsymbol{\sigma}_g, \boldsymbol{C}_g$ along with $T_{g, k}$. The samples $\boldsymbol{x}_g$ are ordered based on their spatial distances from the origin $\boldsymbol{o}_g$ following the coordinate transformation. We then express the volumetric rendering of a ray $\boldsymbol{r}_g$ integrating $m$ objects within the global frame as follows:
{
\setlength\abovedisplayskip{2pt}
\setlength\belowdisplayskip{2pt}
\begin{equation}
\label{eq:multi_volrend}
{\hat{\boldsymbol{C}}_g}({\boldsymbol{r}_g})=\sum_{k=1}^{m*N} T_{g, k} \left(1-\exp \left(-\sigma_{g, k} \delta_k\right) \right) {\boldsymbol{C}}_{g,k}. 
\end{equation}}

\noindent \textbf{Score Distillation Sampling.}
% During the SDS process, a noise image $\boldsymbol{X}_t$ is first generated by adding a sampled noise $\epsilon \sim \mathcal{N}(0, I)$ in noise level $t$ into a rendered view $\boldsymbol{X}$ from a NeRF.
To facilitate the conversion from text descriptions to 3D models, DreamFusion~\cite{poole2022dreamfusion} utilizes Score Distillation Sampling (SDS), leveraging the generative capabilities of a diffusion model, denoted as $\phi$, to guide the optimization of NeRF parameters, symbolized as $\boldsymbol{\theta}$.
%
Initially, SDS creates a noisy image $\boldsymbol{X}_t$ by infusing a randomly sampled noise $\epsilon$, which follows a normal distribution $\mathcal{N}(0, I)$, into a NeRF-rendered image $\boldsymbol{X}$ at a given noise level $t$.
The diffusion model $\phi$ then estimates the noise $\epsilon_\phi\left(\boldsymbol{X}_t, t, T\right)$ from this noisy image, conditioned by the noise level $t$ and an optional text prompt $T$. 
The key step in SDS involves calculating the gradient of the loss function, which measures the discrepancy between the estimated noise and the originally added noise:
{\setlength\abovedisplayskip{2pt}
\setlength\belowdisplayskip{2pt}
\begin{equation}
\label{eq:sds_loss}
\nabla_\theta \mathcal{L}_{\text{SDS}}(\boldsymbol{X}_t, T)=  w(t)\left(\epsilon_\phi\left(\boldsymbol{X}_t, t, T\right)-\epsilon\right),
\end{equation}}where $w(t)$ is a weighting function that adjusts the influence of the gradient based on the noise level. 
The gradients across all rendered views direct the update of $\boldsymbol{\theta}$, ensuring that the NeRF-generated images align with the text descriptions. Additionally, we incorporate the 'perturb and average' technique from SJC for more robust $\mathcal{L}_{\text{SDS}}$. For a comprehensive understanding of these methods, the reader is directed to the detailed explanations provided in \cite{poole2022dreamfusion,wang2022score}.

%
%
% \subsection{Editable 3D Scene Layout}
% \label{ssec:layout}
% The 3D scene layout explicitly combines language structures with 3D layouts in an editable way.
% Given the input text prompt $T$, the attribute-object pairs can be easily obtained based on user control.
% Note that the text prompt indicates the multi-object text prompt by default.
% % available for free in many structured representations, such as the constituency tree.
% As shown in Fig.~\ref{fig:framework}, we can extract multiple noun phrases with their binding attributes and map these local text prompts into corresponding regions.
% Specifically, we define the scene structure with $m$ local frames, each employs a local NeRF $\boldsymbol{\theta}_l$ as representation, the local text prompt $T_{l} \subseteq{T}$ and its spatial layout with 3D boxes $\mathbf{b} = \{\mathbf{p}, \mathbf{s}\} \in  \mathbb{R}^6$ of each object entity, where $\mathbf{p}=\{p_x, p_y, p_z\}$ refers to the center point and $\mathbf{s}=\{s_x, s_y, s_z\}$ denotes the box scale. 
% \textit{Our editable 3D layout is easy to be collected and edited with its simplicity, allowing for versatile and interactive user control by modifying the box's or text's properties to define a new scene}.
% Moreover, as depicted in Fig.~\ref{fig:teaser}, each component in a 3D scene layout can be replaced or re-composited with other trained local NeRFs, which is more friendly for flexible user editions compared with using only text prompts.
% We fine-tuned the new layout by global rendering, which enables scalable re-editing.
% Each relationship $r_k \in R$ is a triplet in a <subject-predictive object> format, where a subject node is. After we generate the scene graph from the complex prompts, we can sample the closest relationship with the 2d spatial layout as the initial 3D position. fine-tuned the new layout by global rendering, which enables scalable re-editing
%
% \subsection{Scene Rendering Pipeline}
% \label{ssec:render}
% In CompoNeRF, the scene images are rendered by a ray-casting approach following the design of NeRF.
% % Each ray to be cast is generated based on the camera pose, intrinsic, and transformation.
% The camera is defined by a pinhole camera model, casting a set of rays $(\boldsymbol{r}_o, \boldsymbol{\phi}_d, {\boldsymbol{\theta}}_d)=\boldsymbol{o}+t\boldsymbol{d}$ through each pixel on the frame of size $H \times W$, where the $\boldsymbol{r}_o \in  \mathbb{R}^3$ is the origin and the $(\boldsymbol{\phi}_d, \boldsymbol{\theta}_d)$ is the viewing direction.
% Along this ray, we sample all the points intersected with any layout box of local frames.
% For each hit sampled point, the color and volumetric density are computed through the local NeRF of the hit local frame.
% The ray color perdition is calculated by the differentiable integration applied on all the point-predicted colors and volumetric density along the ray.
%
% \noindent \textbf{Ray-box Intersection with Local Frames.}
% Given a ray $\boldsymbol{r}_i$, each box $\boldsymbol{b}_j$ of the local frame is applied with the AABB ray intersection test algorithm to check the intersections.
% When the ray $r_i$ is hit with a box $\boldsymbol{b}_j$ of the local frame, we use the entrance and exit points as near $\boldsymbol{t}_{in}$ and far $\boldsymbol{t}_{out}$ bounds to sample $N$ equidistant quadrature points, $
% \boldsymbol{t}_{i,j,n}=\frac{n-1}{N-1}\left(\boldsymbol{t}_{out}-\boldsymbol{t}_{in}\right)+\boldsymbol{t}_{in} , n \in \left[1, N\right]$
% % Despite each local frame only having a small number of hit rays compared to the scene, we observe that it is enough to represent each object accurately while maintaining short rendering times.
% Note that the coordinates of sampled points are first projected into normalized coordinates using the box scale of local frames to enable each local NeRF to learn the scale-independent representation.
% The bounding box $\mathbf{b}$ of the local frame in global coordinate can be transformed into a canonical bounding box by ${(\mathbf{b}} - \boldsymbol{p}) / \mathbf{s}$.
% Considering the rendering efficiency, we only calculate the valid points, interacted with the boxes, and set all the empty points with a constant background color.
%
% The appearance of a set object representations depends on its interaction with the scene and illumination which should be decided by the local frame location.
% To ensure the volumetric consistency, we only calibrate the emitted color with scene location, while the gradient still can be propagated.
% Since the overall color depends on both the global  positions $({x}_w, {y}_w, {z}_w)$ and ray directions $({\phi}_d, {\theta}_d)$, the global color embedding is learned based on both the positions and ray directions.
% Since the overall color depends on both the global  positions $({x}_w, {y}_w, {z}_w)$ and ray directions $({\phi}_d, {\theta}_d)$, the global color embedding is learned based on both the positions and ray directions.
% \subsection{The Proposed CompoNeRF}
% \subsubsection{Composition Module}
% CompoNeRF aims to composite multiple NeRFs to reconstruct multi-object scenes with both box and prompt guidance.
% %
% Our framework, as shown in \cref{fig:framework}, applies the AABB ray intersection test algorithm to check for intersections on each box in the global frame. We then samples $\boldsymbol{x}_g$ within the ray box intervals, and project them to $\boldsymbol{x}_l$ to infer  $\left(\boldsymbol{C}_l, {\boldsymbol{\sigma}_l}\right)$ in separate NeRF models. 
% %
% We then utilize volume rendering to obtain rendered views for each local frame respectively. 
% %
% After that, they would be passed on to our tailored composition Module to infer 
% $\left(\boldsymbol{C}_g, {\boldsymbol{\sigma}_g}\right)$
% for global rendering. 
% Next, we match local and global texts with their corresponding image outputs by SDS losses. 
% We also support recomposition by passing samples from cached models into $\boldsymbol{x}_l$ to continue the above process.
\begin{figure}[t!]
    \centering
    \includegraphics[width=\linewidth]{figures/abls.pdf}
    % \vspace{-22pt}
    % \caption{Ablation study on text guidance. (a) without local SDS losses. (b) without global SDS losses. (c) vanilla SDS losses without perturb and average scoring~\cite{wang2022score}. (d) full model.}
    \caption{\textbf{Design Impact Comparison: Density vs. Color-based Methods.} The top row illustrates the density-based approach's detailed rendering and quick convergence in the 'table wine' scene. The bottom row highlights the color-based method's enhancements and its drawbacks, such as geometric and shadow inaccuracies, particularly in close-up views and slow convergence.
    % \textbf{(a)} global text guidance(integrating local frames by \cref{eq:multi_volrend}) and global calibration(integrating local frames, then aligning the rendering result directly with the full text). 
    }
    \label{fig:abls}
    % \vspace{-20pt}
\end{figure}
\subsection{The Proposed CompoNeRF}
\subsubsection{Composition Module}
CompoNeRF is designed to composite multiple NeRFs to reconstruct scenes featuring multiple objects, utilizing guidance from both bounding boxes and textual prompts. Within our framework, depicted in \cref{fig:framework}, the Axis-Aligned Bounding Box (AABB) ray intersection test algorithm is applied to ascertain intersections across each box in the global frame. Subsequently, we sample points \(\boldsymbol{x}_g\) within the intervals of the ray-box and project them to \(\boldsymbol{x}_l\) to deduce the corresponding color \(\boldsymbol{C}_l\) and density \(\boldsymbol{\sigma}_l\) within individual NeRF models.
%
These properties are processed through our composition module to infer the global color \(\boldsymbol{C}_g\) and density \(\boldsymbol{\sigma}_g\), crucial for the global rendering.
%
Volume rendering techniques~\cite{kajiya1984ray} are then employed to procure the rendered views for both local and global frames. We propose dual SDS losses to ensure coherence between the image outputs and their corresponding textual descriptions. Additionally, our approach facilitates recomposition by channeling samples from cached models back into local frames along with the text revision, thereby streamlining the integration.

% As shown in \cref{fig:abls}(a), we verify its necessity by dropping $\nabla \mathcal{L}_{\text{SDS}_g}$. 
% %
% Compared with our full model, its layout does not fit our shared sense of a room, \ie, \emph{nightstand} is usually lower than \emph{bed}; \emph{lamp} needs a base to support it. Additionally,  it lacks global consistency, such as light reflection, to make it more realistic. 
% %
% Therefore, we leverage the full text semantics to ensure consistent global rendering across local frames. 
% %
% Instead of conditioning the global rendering view with the full prompt directly, we note that global calibration is necessary for geometry and color to be learned sufficiently.
% For example, we observe that geometric completeness and texture of \emph{nightstand} are not ideal. Although reflection appears around \emph{nightstand}, \emph{bed} is stripped of the light. 
% %
% Therefore, we opt to leverage the correlation between the rendering output of the combined NeRFs and the overall semantics to perform multi-object scene reconstruction.  
%

\noindent\textbf{Global Composition.}
The independent optimization of each local frame may inadvertently result in a lack of global coherence within the scene. To address this, our scene composition process is designed to integrate these frames, thereby achieving a more consistent result.
%
Before exploring the specifics of the module, it is imperative to discuss two critical design decisions within the composition module, as depicted in \cref{fig:framework}.
%
Upon integrating the properties inferred from \(\boldsymbol{x}_g\) into the composition module, they are fine-tuned through gradients derived from the global SDS loss.  This process leads to a critical consideration: the necessity and implications of refining the global density \(\boldsymbol{\sigma}_g\). This can be divided into two approaches: \textbf{1) Density-based:} The advantage of adjusting \(\boldsymbol{\sigma}_g\) is that it can adjust geometry, thus yielding a scene more congruent with the global text prompt. 
However, this comes at the cost of potentially compromising the optimal color \(\boldsymbol{C}_g\), as calibrating \(\boldsymbol{\sigma}_g\) introduces more uncertainty for subsequent color refinement as it requires prior density features $\boldsymbol{h}$ as shown at \cref{fig:compo}. 
\textbf{2) Color-based:} Conversely, directly employing \(\boldsymbol{\sigma}_l\) mitigates this uncertainty but at the expense of reduced geometric control, presenting a challenging balance to strike in the pursuit of precise scene composition.
% , which may lead to suboptimal outcomes.
%
After thorough experiments, exemplified in \cref{fig:abls}, we have opted for the density-based approach to refine \(\boldsymbol{\sigma}_g\)  prioritizing both \textbf{accuracy and efficiency}. The test revealed that it excels in rendering intricate details, such as enhanced wood grain textures and more naturally contoured 'salad', as accentuated by boxes. This method also demonstrated a swifter convergence rate. Conversely, while the color-based improved reflections and reduced flickering on the 'wine cup', it was plagued by issues such as sparse density, which adversely brings holes at the base of the 'cup' and the corner of the 'table'.
Furthermore, upon close examination, it becomes evident that shadow artifacts of 'wine' on the 'table' are pronounced, suggesting that its disadvantages outweigh its advantages.
%  in this context
% \textbf{Global Composition.}
% Each local frame is optimized independently, causing a lack of global connections for scene composition.
% Before delving into module details, there are two choices (see \cref{fig:framework}) on the composition module design we need to elaborate on first. 
% %
% In \cref{fig:framework}, by taking $\boldsymbol{x}_g$ into the composition module, their inferred properties are calibrated with gradients propagated from the global SDS loss. 
% However, it remains unclear whether $\boldsymbol{\sigma}_g$ should be refined or not. 
% %
% The trade-off on its usage is the density adjustment bringing a more reasonable layout and more geometric details that fit the global text prompt. While its potential downside is that $\boldsymbol{C}_g$ may not be optimal as $\boldsymbol{\sigma}_g$ has more uncertainty compared to $\boldsymbol{\sigma}_l$, bringing sub-optimal rendering results. 

% We choose the density-based method after comparing them with the experiment shown in \cref{fig:abls}. 
% %
% Specifically, we test both designs on the scene \emph{table wine} and discover that the density-based design provides more intrinsic details(as indicated by green boxes), \eg, enriched wood grains, and a more natural shape for \emph{salad} and has much faster convergence speed. In contrast, the color-based method enhances the reflection and smooths flickering on \emph{wine cup}, (as indicated by red boxes), but it suffers from 1) sparse density, resulting in poorly generated geometry at the base of  \emph{cup} and the wood \emph{table} corner. Additionally, shadow artifacts appeared on \emph{table} when viewed up close, outweighing benefits of the color-based method.

\begin{figure}[t!]
    \centering
    \includegraphics[width=\linewidth]{figures/compo_module.pdf}
    % \vspace{-24pt}
    % \caption{Ablation study on text guidance. (a) without local SDS losses. (b) without global SDS losses. (c) vanilla SDS losses without perturb and average scoring~\cite{wang2022score}. (d) full model.}
    \caption{\textbf{Detail of Composition module}: density-based design. 
    }
    \label{fig:compo}
    % \vspace{-18pt}
\end{figure}
\noindent\textbf{Network Design.}
The compositional framework of our network, as delineated in \cref{fig:compo}, is predicated on an architecture that employs a suite of MLPs, represented as \(\{\boldsymbol{\theta}_l\}_{l=1}^{m}\),  each dedicated to a distinct local frame. To harmonize \(\boldsymbol{\sigma}_l\) and \(\boldsymbol{C}_l\), we incorporate global MLPs, including density calibrator $f_{\boldsymbol{\theta}_{g_d}}$ and color calibrator $f_{\boldsymbol{\theta}_{g_c}}$.
%
A transformation module complements this system, tasked with maintaining the spatial coherence between the global and local frames. It governs the transformation of sampling points $\boldsymbol{x}$, ray directions $\boldsymbol{d}$, and adjacent sampling distances $\delta$. This module also orders the points $\{\boldsymbol{x}_{g,j}\}_j$ by their distance to the global camera origin $\boldsymbol{o}_g$, ensuring that each local point $\boldsymbol{x}_l$ is accurately matched with its corresponding global point $\boldsymbol{x}_g$ for subsequent volume rendering. 
%
The network design is:
{
\setlength\abovedisplayskip{4.5pt}
\setlength\belowdisplayskip{4.5pt}
\begin{align}
\label{eq:g_c_d}
{\boldsymbol{\sigma}_g}  &= \alpha_d f_{\boldsymbol{\theta}_{g_d}}({\boldsymbol{x}_g}) + \boldsymbol{\sigma}_l, \\
{\boldsymbol{C}_g}  &= \alpha_c f_{\boldsymbol{\theta}_{g_c}}(\boldsymbol{h}, {\boldsymbol{d}_g}) + \boldsymbol{C}_l. 
\end{align}}In contrast to the local frames, the global frame's color output $\boldsymbol{C}_g$ is inferred based on $\boldsymbol{h}$ and conditional on $\boldsymbol{d}_g$ to enable a view-dependent lighting effect.
% Denote the density features as $\boldsymbol{h}$. 
%
%
Residual learning is leveraged here, where \(\boldsymbol{\sigma}_l, \boldsymbol{C}_l\) serve as foundational elements that support the learning of global density \(\boldsymbol{\sigma}_g\) and color \(\boldsymbol{C}_g\). The parameters \(\alpha_d, \alpha_c\) are adjustable, allowing fine-tuning of the influence that local components exert on the global outputs.
%
It is imperative to acknowledge that in our color-based method, density calibration is intentionally excluded to concentrate solely on the refinement of color dynamics as shown at \cref{fig:framework}. This is achieved by conditioning the process on both spatial and directional global inputs \((\boldsymbol{x}_g, \boldsymbol{d}_g)\), as demonstrated in the following equations:
\begin{align}
\setlength\abovedisplayskip{4.5pt}
\setlength\belowdisplayskip{4.5pt}
\label{eq:g_c_c}
\boldsymbol{\sigma}_g = \boldsymbol{\sigma}_l, \quad
{\boldsymbol{C}_g} = \alpha_c f_{\boldsymbol{\theta}_{g_c}}({\boldsymbol{x}_g}, {\boldsymbol{d}_g}) + \boldsymbol{C}_l.
\end{align}
The integration of extra $\boldsymbol{x}_g$ aims to facilitate a fair comparison under same inputs with the density-based. It enhances the visual appeal of effects like the wine cup's reflection, as demonstrated in \cref{fig:abls}. However, this method is not without its compromises. It tends to produce artifacts and is characterized by a slower convergence rate. Additionally, this approach limits the ability to precisely control density, subsequently impacting the intricate geometric details.


\begin{figure*}[t!]
    \centering
    \includegraphics[width=\linewidth]{figures/sota.pdf}
    % \vspace{-24pt}
    \caption{\textbf{Qualitative comparison with other text-to-3D methods using multi-object text prompts}. Cases 1-3 demonstrate simpler settings characterized by compositions involving two objects. In contrast, Cases 4-8 delve into more intricate scenarios featuring compositions with more than two objects. Smaller images are presented to illustrate the generated local NeRFs(partially shown in Cases 4-8).}
    \label{fig:sota}
    % \vspace{-5pt}
\end{figure*}
%
% \begin{table*}[t!]
% \centering
% \resizebox{\textwidth}{!}
% {
% \begin{tabular}{cccccccc}
% \toprule
% Method            & \rotatebox{60}{table wine}  & \rotatebox{60}{teddy monkey} & \rotatebox{60}{computer mouse} & \rotatebox{60}{bed room}  & \rotatebox{60}{chess} & \rotatebox{60}{pisa tower} & \rotatebox{60}{astronaut} & \rotatebox{60}{tesla}  \\ \midrule
% LatentNeRF  & 21.55 & 27.38 & 17.13 & 21.86 & 31.19 & 24.31 & 27.07 & 25.16 \\
% SJC & 23.33 & 27.37 & 18.00 & 22.54 & 30.53 & \textbf{26.18 }& 27.84 & 23.55 \\
% CompoNeRF & \textbf{32.68} & \textbf{28.57}	 &\textbf{ 22.34} &\textbf{ 28.65} & \textbf{31.45} & \textbf{28.96} & 25.82 & 25.95 & 24.42 & \textbf{32.71} & \textbf{26.13 }& \textbf{26.38} & \textbf{30.98} & \textbf{33.37} \\
% \bottomrule
% \end{tabular}
% }
% \vspace{-10pt}
% \caption{Performance of our CompoNeRF in different 3D scenes. We use CLIP score \cite{parmar2023zero,zhang2023sine,wang2023imagen} as our evaluation metric, which is a common evaluation metric in text-to-image generation tasks to evaluate the similarity of the generated image to the text prompt. }
% \label{perclass}
% \end{table*}
%
\begin{table*}[t!]
% \scalebox{0.8}
\renewcommand{\arraystretch}{1.2}
\fontsize{4pt}{4pt}
\selectfont 
\centering
% \vspace{-8pt}
\resizebox{\textwidth}{!}
{
% \begin{tabular}{lcccccccc}
% \hline
% Method     & table\_wine    & tesla          & pyramid        & chess          & apple and banana      & astronaut      & glass\_balls   & Eiffel\_tower    \\ \hline
% LatentNeRF & 21.55          & 25.16          & 27.43          & 31.19          & 27.69          & 27.07          & 29.51          & 26.32          \\
% SJC        & 23.33          & 23.55          & 25.62          & 30.53          & 28.21          & 27.84          & 28.76          &27.41 \\
% \textbf{CompoNeRF(Ours)}     & \textbf{32.68} & \textbf{26.13} & \textbf{28.96} & \textbf{31.45} & \textbf{33.37} & \textbf{32.71} & \textbf{30.98} & \textbf{28.44}          \\ \hline
% \end{tabular}
\begin{tabular}{lcccccccc}
\hline
Method                   & Case 1         & Case 2         & Case 3         & Case 4         & Case 5         & Case 6         & Case 7         & Case 8         \\ 
\hlineB{1.1}
LatentNeRF               & 25.16          & 27.07          & 27.69          & 31.19          & 21.55          & 26.32          & 27.43          & 29.51          \\
SJC                      & 23.55          & 27.84          & 28.21          & 30.53          & 23.33          & 27.41          & 25.62          & 28.76          \\
\textbf{CompoNeRF (Ours)} & \textbf{26.13} & \textbf{32.71} & \textbf{33.37} & \textbf{31.45} & \textbf{36.06} & \textbf{28.44} & \textbf{28.96} & \textbf{30.98} \\ \hlineB{1.1}
\end{tabular}
}

% \vspace{-6pt}
\caption{\textbf{Performance comparison of our CompoNeRF in different 3D scenes}. For our evaluation metric, we utilize the average of CLIP scores~\cite{parmar2023zero,zhang2023sine,wang2023imagen} across different views, which serve to assess the similarity between the generated images and the global text prompt. }
\label{tb:perclass}
\end{table*}
% \cref{fig:framework} depicts the network architecture of the composition module. Denote $m$ as local MLP $\{\boldsymbol{\theta}_l\}_{l=1}^{m}$ for each local frame. Then, we introduce the global MLPs including density $\boldsymbol{\theta}_{g_d}$ and $\boldsymbol{\theta}_{g_c}$ calibrators to refine $\boldsymbol{\sigma}_l$ and $\boldsymbol{C}_l$. 
% %
% In detail, the network design is, 
% {
% % \setlength\abovedisplayskip{4.5pt}
% % \setlength\belowdisplayskip{4.5pt}
% \begin{align}
% \label{eq:g_c_d}
% {\boldsymbol{\sigma}_g}  &= \alpha_d \boldsymbol{\theta}_{g_d}({\boldsymbol{\sigma}_l}) + \boldsymbol{\sigma}_l, \\  
% {\boldsymbol{C}_g}  &= \alpha_c \boldsymbol{\theta}_{g_c}({\boldsymbol{C}_l},  {\boldsymbol{d}_g}) + \boldsymbol{C}_l, 
% \end{align}}
% %
% where residual $\boldsymbol{\sigma}_l, \boldsymbol{C}_l$ assist in learning $\boldsymbol{\sigma}_g$ and $\boldsymbol{C}_g$, while $\alpha_d, \alpha_c$ balance their contribution as learnable parameters.
% %
% Note that the color-based omits density calibration, and simply uses the shared color refinement.



% The 3D boxes are only used for the spatial configuration of local NeRFs, while the implicit representation of local NeRFs is inferred by the canonical samples inside the local frame without considering the global relationship across different objects.
% To relieve such location-dependent effects, we further calibrate the output color and density from the local NeRF with global coordinates $({\boldsymbol{x}}_g, {\boldsymbol{y}}_g, {\boldsymbol{z}}_g)$ and ray directions $\left({\boldsymbol{\phi}}_{d}, {\boldsymbol{\theta}}_{d}\right)$ as the conditional input.
% % to inject the global visual clues.
% %
% %
% Specifically, we adopt a shared MLP $\boldsymbol{\theta}_{g}$ to calibrate all the predicted object colors, that is,
% {\setlength\abovedisplayskip{4.5pt}
% \setlength\belowdisplayskip{4.5pt}
% \begin{align}
% \label{eq:MLP_dyn_2}
% {\boldsymbol{C}_g} = {\boldsymbol{C}_l} + \boldsymbol{emb}_{g} &= {\boldsymbol{C}_l} + \boldsymbol{\theta}_{g}({\boldsymbol{x}}_g, {\boldsymbol{y}}_g, {\boldsymbol{z}}_g, {\boldsymbol{\phi}}_{d}, {\boldsymbol{\theta}}_{d}),
% \end{align}}
% where ${\boldsymbol{C}_l}$ is the color predicted by the local NeRF.
% Therefore, the scene color can preserve the view-consistent behavior from the original architecture and add consistency across poses for the volumetric density.
% Since the color and density values share the same latent expression in $({\boldsymbol{x}}_l, {\boldsymbol{y}}_l, {\boldsymbol{z}}_l)$, we only calibrate the emitted scene color explicitly with the scene location, as the densities of local NeRFs also are implicitly adjusted during optimization.

% \noindent \textbf{Global and Local Volumetric Rendering.}
% After compositing all the interacted points, each ray $\boldsymbol{r}_i$ collects a set sampling points by $\{\boldsymbol{t}_{i,j,n} \}_{j=1, n=1}^{m_j, N}$, where $m_j$ is the number of the hit object.
% For each sampling point, the inference results with the respective 3D representations are the local color $\boldsymbol{c}_{l}$, global color $\boldsymbol{c}_{g}$, and density $\sigma$.

% In fact, the local view $\hat{C}_{l,j}$ of single object $j$ also can be rendered by the sampled points  belongs to the same local frames as shown at Fig.~\ref{fig:framework}.

\subsubsection{Recomposition}
Our architecture advances scene reconstruction by providing an intuitive interface for layout manipulation.  This capability is crucial for the reconfiguration of scene elements into novel scenes, as depicted in \cref{fig:framework}. Here, the input panel allows for adjustments in the attributes of bounding boxes, such as modifying the position and scale of the 'apple' bounding box prior to composition. The refinement process further involves sampling ray-box intervals from the global frame, leading to transformed coordinates with the corresponding ray samples that are then incorporated into the pipeline, as demonstrated in \cref{fig:compo}.
%
Each bounding box represents an individual NeRF, providing the flexibility to move, scale, or remove elements as needed. CompoNeRF's capabilities also extend to textual edits, exemplified by the transformation of 'wine' into 'juice'.
%
Since NeRFs have been well trained, we only finetune \(\theta_g, \theta_l\) to align text prompts to promote consistency of both local and global views.
%
Moreover, the NeRFs once retrained within the edited scene, are also structured to be decomposable and cacheable in future scene compositions.
% Our CompoNeRF architecture facilitates the seamless reconstruction of scenes leveraging existing models. It enables precise editing of bounding boxes parameterized by \(\{\boldsymbol{\theta}_l\}_{l=1}^{m}\), allowing for their reconfiguration into new layouts. Refer to \cref{fig:framework}, the input panel permits the modification of attributes such as the position and scale of the 'apple' node's bounding box prior to composition. The process is further refined by sampling from the updated ray-box intervals within the global frame, which are then projected onto \(\boldsymbol{x}_l\), ensuring a streamlined reconstruction that integrates the 'apple' effectively. This addition is executed with careful attention to color consistency, positioning the 'apple' adjacent to the 'French bread' to complement the scene's overall palette. Each bounding box represents an individual NeRF, which means they can be manipulated through moving, scaling, and removal operations. CompoNeRF also extends its editing prowess to textual modifications, as evidenced by the 'wine cup' now appearing filled with juice—a change propagated through both subtexts and the global test. 
% %
% Since NeRFs have been well trained, we only finetune $\theta_g, \theta_l$ to align text prompts to promote consistency of both local and global views . 
% %
% Moreover, the NeRFs, once retrained within the reimagined scene, are also structured to be decomposable and cacheable for subsequent scene compositions.

% , as shown in Fig.~\ref{fig:framework}.
% For each scene described by the multi-object text prompt $T$, we
% To enhance the guidance of local representations, we use the local text prompt $T_l \subseteq T$ of a single object to optimize the local NeRFs in local views.
% The scene views $\hat{\boldsymbol{X}}_g=\{\hat{\boldsymbol{C}}_{g,i}\}_{i=1}^{H\times W}$ is obtained from the predicted pixel values of $H \times W$ rays by compositing all the ray-box interaction values.
% Similarly, the rendered view $\hat{\boldsymbol{X}}_{l,j}$ of the local frame $\boldsymbol{\theta}_j$ without compositing other objects can be calculated by $\hat{\boldsymbol{C}}_{l,j}$, as depicted in Sec.~\ref{ssec:render}.
% We use the local color instead of the globally calibrated color to obtain a local view because the local NeRF should learn the object identity unrelated to its placed position, as the position can be different during user edition.
% % Compared to cropping the local region from a global view for training, separate rendering can avoid the undesired information from other objects brought by the occlusion and resolution adjustments.
% Formally, we employ the following loss as the learning objective,
\begin{figure*}[t!]
    \centering
    \includegraphics[width=\linewidth]{figures/editing.pdf}
    % \vspace{-23pt}
    \caption{\textbf{Scene Editing Outcome:} Demonstrated here are the stages of our recomposition, utilizing cached source scenes. Each NeRF is individually identified by colorful labels. These decomposed nodes are then positioned in the initial layout and subsequently calibrated to form the final composition. The detailed description of the ambient environment is underscored, enhancing the scene's realism.}
    \label{fig:app}
    % \vspace{-12pt}
\end{figure*} 
\subsubsection{Optimization}
\label{sec:optimization}
During optimization, our method employs dual text guidance to align rendering results with both global and local textual descriptions. The optimization objective is:
{
\small
\setlength\abovedisplayskip{2pt}
\setlength\belowdisplayskip{2pt}
\begin{equation}
\label{eqn:loss_f}
\mathcal{L}= {\alpha_g}\nabla\mathcal{L}_{\text{SDS}}(\hat{\boldsymbol{X}}_{g}, T) + {\alpha_l}\sum_{j=1}^{m} \nabla\mathcal{L}_{\text{SDS}}(\hat{\boldsymbol{X}}_{l,j}, T_{l,j}) + \beta\mathcal{L}_{\text{sparse}},\nonumber
\end{equation}
}where $T$ signifies the global text prompt, while $T_{l}$ pertains to a specific object within the global context. The hyperparameters $\alpha_{g}, \alpha_{l}$, and $\beta$ modulate the respective loss weights. 
% $\nabla \mathcal{L}_{\text{SDS}}$ is the score distillation sampling loss, as described in Sec.~\ref{sec:background}.
As suggested in~\cite{metzer2022latent}, we use $L_{\text{sparse}}$ included to penalize the binary entropy of local NeRFs' densities, thereby mitigating the issue of extraneous floating radiance.
Additionally, incorporating directional cues such as "front view" or "side view" into the input text, as suggested by \cite{poole2022dreamfusion,metzer2022latent} proves beneficial in specifying camera poses during the training phase, further enhancing the alignment of our generated scenes with the intended perspectives.
% Note that the global calibration in the scene frame can adaptively revise both $({C}_l, {\sigma})$ in local NeRF with $\nabla \mathcal{L}_{SDS}$ along with the back-propagating gradient.


\begin{figure*}[!ht]
  \centering
  \includegraphics[width=\linewidth]{imgs/Warped.pdf}
  \caption{\textbf{Dense warping} from a source image (top row) to a target image (second row). We warp all foreground pixels (highlighted by a red overlay in the source image). Our methods produce dense and semantically more meaningful warps from the source to the target.}
  \label{fig:warp}
  \vspace{-1em}
\end{figure*}

\section{Experiments}
\label{sec:experiments}

We evaluate our method on several real-world in-the-wild image collections of both rigid and non-rigid object categories. For all datasets, we use a fixed set of hyperparameters (provided in the appendix) unless specified otherwise. 
%and train our model on a single 16GB GPU for a fixed 20,000 iterations (batch size of 20). 

%In \S\ref{sec:qual_consistent}, we visualize the canonical space mapping learned by the model. In \S\ref{sec:qual_dense}, we evaluate the dense correspondence learned by the model qualitatively.
% In \S\ref{sec:quant_pair}, we show the performance of our approach using standard quantitative metrics on datasets where ground truth keypoint annotations are available. In \S\ref{sec:quant_consistent}, we propose a new metric to evaluate the consistency of keypoint predictions as we propagate keypoints over multiple images in a sequence.

\myparagraph{Datasets.}
\textbf{SPair-71k}~\citep{min2019spair} consists of 1,800 images from 18 
%PASCAL VOC~\citep{everingham2015pascal} 
categories. We optimize over image collections derived from the SPair-71k test set for each category independently and report results
%We train a model on the test images for each category independently ($\sim$25 images each) using no ground truth annotations. 
 on each individual category, as well as aggregate results over all 18 categories. In case of \textbf{PF-Willow}~\citep{ham2016}, we consider all 4 categories of the dataset containing $\sim$30 images. 
% We also report results on the test set of the first three individual categories of 
\textbf{CUB-200}~\citep{wah2011caltech} datasets consists of over 200 fine-grained categories. We optimized our model on the test sets of first 3 categories of the dataset, consisting of 15-20 images each.
%contains about 12K images from 200 bird categories. We report our results on 
We also show qualitative results on 4 objects from \textbf{SAMURAI} dataset~\citep{boss2022samurai}.
%We don't provide quantitative results on the SAMURAI data since the groundtruth keypoints annotations are unavailable.

%to demonstrate the effectiveness of our approach for collections where only few images are available. SAMURAI consists of images of objects taken under a variety of background and lighting conditions. While the images are taken in less `in-the-wild' conditions as compared to SPair, objects in SAMURAI are relatively uncommon (and not present in datasets used to train large-scale SSL models).


\subsection{Canonical Space Alignment}
\label{sec:qual_consistent}

% A convenient 
One simple way to visualize the alignment of an image set when mapped to the canonical space  $\G$
% mapping of the image set 
is to define a colormap over the canonical grid and color the image pixels according to their mapped location in the canonical grid. 
In \cref{fig:canon}, the first row for each collection contains sample input images. The second row shows discrete parts obtained via parts co-segmentation using \citep{amir2021deep}. 
While these parts are also consistent across the image set, our canonical space mapping (third row) can be seen as a dense and continuous co-segmentation. We show the results for six datasets: CUB-200 birds; Dogs, Cats, and Train from SPair-71k; and Robot and Shoe from SAMURAI. 
The colormap used for the canonical space is provided in the supplement.
We observe that our method can find dense correspondences across highly varying poses, backgrounds, and lighting.
It also maps common parts of objects in a dataset to nearby regions of the canonical space.
This is evident in \cref{fig:canon} where, for instance, the faces of different cats are colored similarly.
% always 
%While our method can reason about semantically similar parts of objects, it struggles with left-right ambiguity in symmetric objects. This is partially an artifact of the part consistency loss (\cref{eq:parts}). Since the parts obtained using part co-segmentation do not disambiguate between left and right parts (hands, legs, \etc), these errors are propagated to our method as well.
% A simple way to deal with this ambiguity is to estimate the total variation (TV) loss of the output of alignment network for both original and flipped image, and use the one with minimum TV loss.

\subsection{Visualizing Dense Correspondences}
\label{sec:qual_dense}
We can also find dense correspondences between a pair of images $\Ia$ and $\Ib$ using our framework. Recall that $\A$ outputs canonical space coordinates $\cca$ and $\ccb$ for each pixel location $\pa$ and $\pb$.
%Since our mapping is unconstrained, it is not trivially invertible.
In order to warp the source image $\Ia$ to a target image $\Ib$, for every foreground pixel $\pa_i$ in $\Ia$, we need to find its nearest neighbor among the set of points in $\pb$ in the canonical space:
%%%%%%
% \begin{align*}
    % \Iatob(\pa_i) \approx \arg \min_j \parallel \cca_i - \ccb_j \parallel^2
% \end{align*}
%%%%%%
We perform this action for all the foreground pixels in the source image, and splat according to the nearest neighbor mapping to get our desired warped image. 
\cref{fig:warp} shows qualitative results 
% of our method in the last row 
for 10 different datasets. The top row is the source image with foreground mask highlighted. The second row is the target image. We show results for two other pairwise image optimization approaches, and then our method in the last row. NBB~\citep{aberman2018neural} computes nearest neighbors using VGG-19 and applies a Moving Least Squares (MLS) optimization~\citep{schaefer2006image} to compute a dense flow from source to target. While the flow computed via MLS is smooth, it usually does not respect semantic correspondences, as evident in the figure. 
We extend their technique to use DINO features as well.
With DINO and the nearest neighbor approach (DVD)~\citep{amir2021deep}, the semantic correspondences are arguably better, but since this approach relies on the output of a Vision Transformer (ViT), which has lower resolution than the image, it produces a sparse flow. Our method produces both dense and consistent flow between an image pair.

\begin{table*}[t]
\centering
% \footnotesize
% \renewcommand{\arraystretch}{1.1}
% \renewcommand{\tabcolsep}{2pt}
\caption{\textbf{Evaluation on SPair-71k.} Per-class and average PCK@$0.10$ on test split. Highest PCK among \textit{weakly supervised} methods in bold, second highest underlined. Scores marked with ($\star$) means the paper uses a fixed image from the test set as canonical image. Our method is competitive against other weak supervised approaches and often outperforms them.}
% \vspace{-0.5em}
\label{tab:spair_pck}
\resizebox{\textwidth}{!}{
\begin{tabular}{@{}clccccccccccccccccccc@{}}
% \begin{tabular}{*{5}{>{\rowfonttype}clccccccccccccccccccc}<{\rowfont{}}}
\toprule
\textbf{Supervision} & \textbf{Method}  & Aero & Bike & Bird & Boat &Bottle& Bus  & Car  & Cat  & Chair& Cow  & Dog  & Horse& Motor&Person& Plant& Sheep & Train & TV & \textbf{All}\\  
\midrule
\multirow{1}{3.2cm}{\centering{Strong Supervision}}
% & HPF~\citep{min2019hyperpixel}         & 25.2 & 18.9 & 52.1 & 15.7 & 38.0 & 22.8 & 19.1 & 52.9 & 17.9 & 33.0 & 32.8 & 20.6 & 24.4 & 27.9 & 21.1 & 15.9 & 31.5 & 35.6 & 28.2\\
% & SCOT~\citep{liu2020semantic}          & 34.9 & 20.7 & 63.8 & 21.1 & 43.5 & 27.3 & 21.3 & 63.1 & 20.0 & 42.9 & 42.5 & 31.1 & 29.8 & 35.0 & 27.7 & 24.4 & 48.4 & 40.8 & 35.6\\
% & DHPF~\citep{min2020learning}          & 38.4 & 23.8 & 68.3 & 18.9 & 42.6 & 27.9 & 20.1 & 61.6 & 22.0 & 46.9 & 46.1 & 33.5 & 27.6 & 40.1 & 27.6 & 28.1 & 49.5 & 46.5 & 37.3\\
% & PMD~\citep{li2021probabilistic}       & 38.5 & 23.7 & 60.3 & 18.1 & 42.7 & 39.3 & 27.6 & 60.6 & 14.0 & 54.0 & 41.8 & 34.6 & 27.0 & 25.2 & 22.1 & 29.9 & 70.1 & 42.8 & 37.4\\
% & MMNet~\citep{zhao2021multi}           & 43.5 & 27.0 & 62.4 & 27.3 & 40.1 & 50.1 & 37.5 & 60.0 & 21.0 & 56.3 & 50.3 & 41.3 & 30.9 & 19.2 & 30.1 & 33.2 & 64.2 & 43.6 & 40.9\\
% & CHM~\citep{min2021convolutional}      & 49.6 & 29.3 & 68.7 & 29.7 & 45.3 & 48.4 & 39.5 & 64.9 & 20.3 & 60.5 & 56.1 & 46.0 & 33.8 & 44.3 & 38.9 & 31.4 & 72.2 & 55.5 & 46.3\\
% & CATs~\citep{cho2021cats}              & 52.0 & 34.7 & 72.2 & 34.3 & 49.9 & 57.5 & 43.6 & 66.5 & 24.4 & 63.2 & 56.5 & 52.0 & 42.6 & 41.7 & 43.0 & 33.6 & 72.6 & 58.0 & 49.9\\
% & PMNC~\citep{lee2021patchmatch}        & 54.1 & 35.9 & 74.9 & 36.5 & 42.1 & 48.8 & 40.0 & 72.6 & 21.1 & 67.6 & 58.1 & 50.5 & 40.1 & 54.1 & 43.3 & 35.7 & 74.5 & 59.9 & 50.4\\
& SCorrSAN~\citep{huang2022learning}    & 57.1 & 40.3 & 78.3 & 38.1 & 51.8 & 57.8 & 47.1 & 67.9 & 25.2 & 71.3 & 63.9 & 49.3 & 45.3 & 49.8 & 48.8 & 40.3 & 77.7 & 69.7 & 55.3\\
\rowfont{\color{Black}}
\multirow{1}{3.2cm}{\centering{GAN supervision}}
& GANgealing~\citep{peebles2022gan}     &    - & 37.5 &    - &    - &    - &    - &    - & 67.0 &    - &    - & 23.1 &    - &    - &    - &    - &    - &    - & 57.9 &    -\\
\hline
\multirow{7}{3.2cm}{\centering{Weak supervision \\ (train/test)}}
& CNNGeo~\citep{rocco2017convolutional} & 23.4 & 16.7 & 40.2 & 14.3 & 36.4 & 27.7 & 26.0 & 32.7 & 12.7 & 27.4 & 22.8 & 13.7 & 20.9 & 21.0 & 17.5 & 10.2 & 30.8 & 34.1 & 20.6\\
& A2Net~\citep{seo2018attentive}        & 22.6 & 18.5 & 42.0 & 16.4 & 37.9 & \textbf{30.8} & 26.5 & 35.6 & 13.3 & 29.6 & 24.3 & 16.0 & 21.6 & 22.8 & \textbf{20.5} & 13.5 & 31.4 & 36.5 & 22.3\\
& WeakAlign~\citep{rocco2018end}        & 22.2 & 17.6 & 41.9 & 15.1 & 38.1 & 27.4 & 27.2 & 31.8 & 12.8 & 26.8 & 22.6 & 14.2 & 20.0 & 22.2 & 17.9 & 10.4 & 32.2 & 35.1 & 20.9\\
& NCNet~\citep{rocco2018neighbourhood}  & 17.9 & 12.2 & 32.1 & 11.7 & 29.0 & 19.9 & 16.1 & 39.2 &  9.9 & 23.9 & 18.8 & 15.7 & 17.4 & 15.9 & 14.8 &  9.6 & 24.2 & 31.1 & 20.1\\
& SFNet~\citep{lee2019sfnet}            & 26.9 & 17.2 & 45.5 & 14.7 &  \underline{38.0} & 22.2 & 16.4 & \textbf{55.3} & 13.5 & 33.4 & 27.5 & 17.7 & 20.8 & 21.1 & 16.6 & 15.6 & 32.2 & 35.9 & 26.3 \\
& PMD~\citep{li2021probabilistic}       & 26.2 & 18.5 & 48.6 & 15.3 & \textbf{38.0} & 21.7 & 17.3 & 51.6 & 13.7 & 34.3 & 25.4 & 18.0 & 20.0 & 24.9 & 15.7 & 16.3 & 31.4 & \textbf{38.1} & 26.5\\
& PSCNet-SE~\citep{jeon2021pyramidal}   & 28.3 & 17.7 & 45.1 & 15.1 & 37.5 &  \underline{30.1} & \underline{27.5} & 47.4 & 14.6 & 32.5 & 26.4 & 17.7 & 24.9 & 24.5 & \underline{19.9} & 16.9 & 34.2 & \underline{37.9} & 27.0\\
\hline
\multirow{5}{3.2cm}{\centering{Weak supervision \\ (test-time optimization)}}
& VGG+MLS~\citep{aberman2018neural}     & 29.5 & 22.7 & 61.9 &  \textbf{26.5} & 20.6 & 25.4 & 14.1 & 23.7 & 14.2 & 27.6 & 30.0 & 29.1	& \underline{24.7} & 27.4 &	19.1 & 19.3 & 24.4 & 22.6 & 27.4\\
& DINO+MLS~\citep{aberman2018neural,caron2020unsupervised}& 49.7 & 20.9 & 63.9 & 19.1 & 32.5 & 27.6 & 22.4 & 48.9 & 14.0 & 36.9 & 39.0 & \underline{30.1} & 21.7 & \underline{41.1} & 17.1 & 18.1 & \underline{35.9} & 21.4  & 31.1\\
& DINO+NN~\citep{amir2021deep}          & \underline{57.2} &  24.1 &  \underline{67.4} & 24.5 & 26.8 & 29.0 & 27.1 & 52.1 &  \underline{15.7} &  \underline{42.4} &  \underline{43.3} & 30.1  & 23.2 & 40.7 & 16.6 & \underline{24.1} & 31.0 & 24.9 & \underline{33.3}\\
& NeuCongeal~\citep{ofri2023neural}& -    & \textbf{29.1$^\star$}    &    - &    - &    - &    - &    - & 53.3    &    - &    - & 35.2   &    - &    - &    - &    - &    - &    - & -    &    -\\
& \backronym~(Ours)                    & \textbf{57.9} & \underline{25.2} & \textbf{68.1} &  \underline{24.7}	& 35.4 & 28.4 & \textbf{30.9} &  \underline{54.8}	& \textbf{21.6} & \textbf{45.0} &	\textbf{47.2} & \textbf{39.9} & \textbf{26.2} & \textbf{48.8} &	14.5 & \textbf{24.5} & \textbf{49.0} & 24.6 & \textbf{36.9}\\

\bottomrule
\end{tabular}}
\vspace{-1em}
\end{table*}

% \begin{table}[!ht]
% \centering
%     \caption{\textbf{Evaluation on CUB-200 and PF-Willow.} PCK for three CUB categories. We outperform other approaches at both coarse precision ($\alphapck$=0.10) and fine precision ($\alphapck$=0.01).}
%     \vspace{-0.8em}
%     \label{tab:cub_pck}
%     \resizebox{\linewidth}{!}{
%     \begin{tabular}{ lcc|cc|cc } 
%     \toprule
%      & \multicolumn{2}{c|}{CUB-001} & \multicolumn{2}{c|}{CUB-002} & \multicolumn{2}{c}{CUB-003} \\
%      \midrule
%     \textbf{Method} & PCK@$0.1$ & PCK@$0.01$ & PCK@$0.1$ & PCK@$0.01$ & PCK@$0.1$ & PCK@$0.01$  \\
%     \midrule
%     NBB~\citep{aberman2018neural} & 22.1 & 0.3 & 21.2 & 2.1  & 34.2 & \textbf{10.2} \\
%     DVD~\citep{amir2021deep}      & 66.8 & 7.8 & 68.5 & 8.2  & 69.5 & 6.9 \\
%     \backronym~(Ours)             & \textbf{71.8} & \textbf{9.6} & \textbf{78.6} & \textbf{10.9} & \textbf{77.4} & 9.9 \\
%     \bottomrule
%     \end{tabular}
%     }
%     \vspace{-1.2em}
% \end{table}

\begin{table}[!ht]
\centering
    \caption{\textbf{CUB-200 and PF-Willow.} PCK@$0.10$ for three CUB categories and four PF-Willow categories.}
    % \vspace{-0.8em}
    \label{tab:cub_pck}
    \resizebox{\linewidth}{!}{
    \begin{tabular}{ lcc|cc } 
    \toprule
     & \multicolumn{2}{c|}{CUB-200 (3 categories)} & \multicolumn{2}{c}{PF-Willow (4 categories)} \\
     \midrule
    \textbf{Method} & PCK@$0.1$ & PCK@$0.05$ & PCK@$0.1$ & PCK@$0.05$  \\
    \midrule
    PMD~\citep{li2021probabilistic}        & - & - & 74.7 & 40.3 \\
    PSCNet-SE~\citep{jeon2021pyramidal}    & - & - & 75.1 & 42.6  \\
    VGG+MLS~\citep{aberman2018neural}      & 25.8 & 18.3 & 63.2 & 41.2   \\
    DINO+MLS~\citep{aberman2018neural,caron2020unsupervised} & 67.0 & 52.0 & 66.5 & 45.0 \\
    DINO+NN~\citep{amir2021deep}               & 68.3 & 52.8 & 60.1 & 40.1 \\
    \backronym~(Ours)                      & \textbf{75.9} & \textbf{57.9} & \textbf{76.3} & \textbf{53.0} \\
    \bottomrule
    \end{tabular}
    }
    \vspace{-1.2em}
\end{table}

\subsection{Pairwise Correspondence}
\label{sec:quant_pair}
\myparagraph{Metric.}
For evaluating accuracy of pairwise correspondence, we use the PCK metric~\citep{yang2012articulated} (percentage of correct keypoints) on the SPair, CUB, and PF-Willow datasets.
%We report PCK computed across all pair of images in all datasets, unless mentioned otherwise. 

%While most existing works show the result at $\alphapck=0.10$, which corresponds to roughly $20$ pixels (for a bounding box of size $200 \times 200$ pixels), we also report results for our method at higher precision thresholds.

\myparagraph{Baselines.}
We categorize prior works based on the supervision used: 
(1) \emph{Strong supervision} methods utilize human-annotated keypoints to learn pairwise image correspondence and achieve the best performance (on average). We include the numbers from a recent work~\citep{huang2022learning} for reference purposes.
(2) \emph{GAN supervision} methods like~\citep{peebles2022gan} use a category-specific GAN pre-trained with large external datasets. While this method works well, it is restricted to only the categories for which large datasets are available and GAN training is feasible.
(3) \emph{Weak supervision} methods use category-level supervision (\ie they assume that given pair/collection of images are from same category). They often resort to fine-tuning a large ImageNet~\citep{deng2009imagenet} pre-trained network using a self-supervised loss function (\eg with synthetic transformations) and optionally use additional information such as foreground masks or matching image pairs for training.
%Some works, such as GANgealing~\citep{peebles2022gan} use large external datasets for training GAN models that allow them to sample from highly diverse category-specific images.
%Most importantly, these works follow a train/test setting where they (usually) fine-tune  on the SPair-71k train set and evaluate on SPair-71k test. 
Some of these works follow a \emph{train/test} setting, where the network is fine-tuned on a separate set of training images.  Note that in our work, we train a much smaller network from scratch instead of fine-tuning a large network. Some approaches (including ours) directly perform \emph{test-time optimization} without additional training data or annotations.

%Note that these works are not directly comparable to our method which trains a much smaller network from scratch on SPair-71k test per object category and does not aim for generalization but rather for consistent dense correspondences. For the sake of completion (and lack of other closely related works), we include some of these recent weakly-supervised works in \cref{tab:spair_pck}.
%(3) \emph{Weak supervision (test-time optimization)} refers to the approaches which simply optimize for test images without additional training data or annotations. Our method is closest to these works. 

NBB \cite{aberman2018neural} optimizes a flow from one image to another using mutual nearest neighbors as control points~\citep{schaefer2006image}. While~\cite{aberman2018neural} shows the results by computing nearest neighbors from a VGG network, we further extend their work to utilize a DINO network. \cite{amir2021deep} simply computes nearest neighbors in DINO feature space. 
A concurrent work, Neural Congealing~\cite{ofri2023neural}, is closest to our work, in that they also perform test-time training using a canonical atlas. 
However, for objects with large deformations (such as in SPair-71k), they need to apply category specific accommodations (for instance, fixing the atlas for bicycle category). 
Our canonical grid allows for large deformations and is learned in all cases with a fixed set of hyperparameters.
We obtain scores for other models from their respective papers (whenever available) or from~\citep{huang2022learning}. 
Scores for~\cite{aberman2018neural,amir2021deep,caron2020unsupervised} are computed using official code. 
The official code of~\cite{ofri2023neural} did not converge on several objects in our experiments, hence we report the quantitative results from the paper.

\myparagraph{Discussion.} \cref{tab:spair_pck} shows PCK@$0.1$ for all SPair-71k~\citep{min2019spair} categories.
It is evident that having groundtruth keypoint annotations
%(and/or large datasets)
during training is highly beneficial; approaches that lack keypoints 
% available
during training lag behind.
%and have difficulty generalizing to unseen images. Test-time optimization approaches, such as ours, attempt to bridge the gap between these two approaches.
We also observe that for categories with rigid objects (or less extreme deformations) such as `Bottle' or `Bus', weakly supervised approaches attain a similar performance as ours. However, in the objects with extreme variations such as animals/birds, our method outperforms other baselines. In our experiments, we observed that per-category hyperparameters can increase PCK performance further by $\sim 2\%$. This strategy is similar to Neural Congealing~\cite{ofri2023neural} where specific accommodations are made per category (\eg tailored training regime for bicycle). However we report our numbers with a fixed set of hyperparameters for consistency.

\cref{tab:cub_pck} shows average results for the first 3 categories of the CUB dataset, and 4 categories of the PF-Willow dataset. Note that PF-Willow is % a relatively `easier' dataset as 
an easier dataset
compared to SPair-71k since it consists of rigid objects with little variation. 
Our method has performance similar to PSCNet-SE~\cite{jeon2021pyramidal} when we compute PCK using threshold $\alphapck=0.1$ (which corresponds to a $\sim$20-pixels margin of error). However at higher precision ($\alphapck=0.05$), our method provides much larger gains compared to the baselines.

%Note that our work combines NBB~\citep{aberman2018neural} and DVD~\citep{amir2021deep} to derive pseudo-groundtruth keypoint pairs for self-supervision. 
%Since these pseudo-groundtruth keypoints need not align with the annotated keypoints (used for testing), our optimization scheme indeed performs worse than the simpler DVD approach in some categories. 
%However, we outperform DVD on average for 18 SPair classes. Our method also achieves higher PCK than NBB and DVD for all three fine-grained CUB categories (refer \cref{tab:cub_pck}).

% \begin{figure}[!t]
%   \centering
%   \includegraphics[width=\linewidth]{imgs/precise_pck.pdf}
%   \caption{PCK at varying values of $\alphapck$.}
%   \label{fig:precise_pck}
%   \vspace{-.7em}
% \end{figure}

%In addition to PCK@$0.1$, we also report finer grained PCK metrics varying $\alphapck$ between 0.01 and 0.1 to better evaluate dense correspondence accuracy. \cref{fig:precise_pck} shows the performance for our model on two SPair and one CUB categories (full results available in the supplement). While the performance of all methods drops drastically at smaller thresholds, our approach performs slightly better than DVD. \cref{tab:cub_pck} shows the PCK results for both $\alphapck=0.01$ and $\alphapck=0.10$ on the CUB dataset.

\begin{figure}[t]
  \centering
  \includegraphics[width=\linewidth]{imgs/cycle_pck.pdf}
%   \vspace{-1.2em}
  \caption{\footnotesize \textbf{Image Set Correspondence.} $\kcycle$ at varying $\alphapck$ (higher is better). Our method outperforms DVD baseline at both large and small values of $\alphapck$.}
  \label{fig:cycle_pck}
%   \vspace{-1.2em}
\end{figure}

\subsection{Image Set Correspondence}
\label{sec:quant_consistent}
Our goal in this work is to recover dense and \textit{consistent} correspondences. A shortcoming of the PCK metric is that it is only computed between image pairs. %In our experiments, we observed that for all methods, 
However, the errors in keypoint prediction tend to accumulate when transferring keypoints over a sequence of images. To address this limitation, we propose a new metric to measure consistency across multiple images, called $\kcycle$. 
Given a set of $k$ images $\{\I^1, \I^2, \dots, \I^k\}$ and an annotated keypoint in the first image $\p^1$ visible in \textbf{each} of the $k$ images, we propagate $\p$ from $\I^1 \rightarrow \I^2, \I^2 \rightarrow \I^3,\dots, \I^{k-1} \rightarrow \I^{k}, \I^{k} \rightarrow \I^{1}$ and get the corresponding predictions $\p^{1\rightarrow2}, \p^{1\rightarrow2\rightarrow3},\dots$ and so on. As before, $\p^{1\rightarrow \dots \rightarrow j}$ is considered to be predicted correctly if it is within a threshold $\alphapck\cdot\max{(H_\bbox, W_\bbox)}$ of the ground truth keypoints $\hat{\p}^{1\rightarrow \dots \rightarrow j}$. We sum up all the correct predictions and plot scores at different values of $\alphapck$ in \cref{fig:cycle_pck}. 
We choose $k=4$ for all experiments (with additional results for other values of $k$ provided in the supplement). Note that since the number of possible permutations of $k$-length sequences can be very large, we randomly sample 200 sequences in our experiments.

\cref{fig:cycle_pck} shows that our method significantly outperforms the DINO+NN baseline for both small and large values of $\alphapck$ across all datasets
% consistently 
(complete results in the supplemental material). We attribute this result to having a consistent canonical space across the image collection that 
% doesn't allow 
prevents errors in keypoint transfer from accumulating to large values. 

% \begin{figure}[t]
%   \centering
%   \includegraphics[width=\linewidth]{imgs/edit.pdf}
%   \caption{\footnotesize \textbf{Edit Propagation.} We demonstrate edit propagation applications using our model for `Cow' category of SPair-71k and `Firetruck' from SAMURAI. \backronym allows us to propagate edits from a random image to other images in the dataset.}
%   \label{fig:edit}
%   \vspace{-1.2em}
% \end{figure}

% \subsection{Edit Propagation}
% \label{sec:edit}
% \backronym can enable a number of image editing applications. In \cref{fig:edit}, we show results of a model trained on `Cow' category of SPair-71k and `Firetruck' from SAMURAI. We added a small edit on the nose of the cow, and propagated it to other images of the dataset. ASIC is able to propagate the edits despite of changes in the object appearance, occlusion, \etc. We make a similar observation in the case of `Firetruck'. Here edit applied to the window of firetruck was reliable transferred to other images.

% \begin{table}[!ht]
% \centering
%     \caption{Consistent Correspondence Results PCK$(\kcycle, \alphapck)$.}
%     \resizebox{\linewidth}{!}{
%     \begin{tabular}{ lccccccc } 
%     \toprule 
%     \textbf{Dataset} & \textbf{Method} & $\mathbf{(2, 0.1)}$ & $\mathbf{(3, 0.1)}$ & $\mathbf{(4, 0.1)}$ & $\mathbf{(2, 0.01)}$ & $\mathbf{(3, 0.01)}$ & $\mathbf{(4, 0.01)}$ \\
%     \hline
%     \multirow{2}{*}{Bicyle} & DVD++~\citep{amir2021deep} & 65.3 & 47.1 & 38.3 & 14.6 & 5.6 & 2.0 \\
%                             & \backronym~ (Ours)                       & 92.8 & 88.9 & 92.0 & 59.3 & 43.3 & 40.9  \\
%     \hline
%     \multirow{2}{*}{Cat} & DVD++~\citep{amir2021deep} & 76.3 & 70.7 & 70.1 & 18.4 & 9.1 & 8.4 \\
%                          & \backronym~ (Ours)                       & 84.7 & 84.3 & 84.2 & 54.2 & 42.7 & 33.9  \\
%     \hline
%     \multirow{2}{*}{Dog} & DVD++~\citep{amir2021deep} & 79.2 & 65.0 & 66.7 & 16.8 & 7.2 & 8.4 \\
%                          & \backronym~ (Ours)                       & 91.5 & 90.0 & 87.9 & 47.3 & 31.5 & 27.4  \\
%     \hline
%     \multirow{2}{*}{TV} & DVD++~\citep{amir2021deep} & 52.7 & 31.6 & 23.4 & 8.8 & 4.8 & 2.2 \\
%                         & \backronym~ (Ours)                       & 83.9 & 73.2 & 71.9 & 30.6 & 15.1 & 12.1  \\
%     \hline
%     \multirow{2}{*}{CUB1} & DVD++~\citep{amir2021deep} & 84.3 & 70.7 & 63.9 & 21.8 & 8.7 & 7.2 \\
%                           & \backronym~ (Ours)                       & 94.1 & 94.4 & 94.1 & 58.3 & 51.5 & 48.8  \\
%     \hline
%     \multirow{2}{*}{CUB2} & DVD++~\citep{amir2021deep} & 79.1 & 67.4 & 62.4 & 13.9 & 8.3 & 5.6 \\
%                           & \backronym~ (Ours)                       & 94.6 & 94.0 & 94.8 & 57.6 & 53.0 & 44.9  \\
%     \hline
%     \multirow{2}{*}{CUB3} & DVD++~\citep{amir2021deep} & 80.8 & 68.6 & 63.9 & 17.4 & 9.4 & 5.6 \\
%                           & \backronym~ (Ours)                       & 94.8 & 94.8 & 92.9 & 62.9 & 56.2 & 49.7  \\
%     \bottomrule
%     \end{tabular}
%     }
% \end{table}

% \myparagraph{Improving SSL backbone.}


\subsection{Ablations}
\label{sec:ablation}
We perform an ablation study on our various proposed losses proposed, summarized in \cref{tab:ablation}. We report average PCK@$0.10$ results for first 3 categories of the CUB-200 dataset and all 4 categories of the PF-Willow dataset. As expected, the keypoint loss $\mathcal{L}_\text{KP}$ plays the most important role in our overall framework. We also found the total variation regularization $\mathcal{L}_\text{TV}$ to be crucial for network training convergence. $\mathcal{L}_\text{Equi}$ is necessary for learning dense correspondence. Finally, $\mathcal{L}_\text{Recon}$ and $\mathcal{L}_\text{Parts}$ provide comparatively small improvements.

%As expected, too high or too low of the temperature values deteriorate the PCK performance. At high values of temperature, the model tends to ignore the pseudo correspondences and at lower values, the model starts overfitting on the the pseudo correspondences. We observe a similar trend for the coefficients of other loss functions $\mathcal{L}_\text{Equi}$ , $\mathcal{L}_\text{Parts}$, and $\mathcal{L}_\text{Recon}$. While our method works out of the box for a fixed set of hyperparameters for all the categories, to achieve the best results for image sets, we perform a hyperparameter search over different hyperparameters using $\kcycle$ (for pseudo correspondences) as the criterion.

\begin{table}[!ht]
\centering
    % \footnotesize
    \caption{\textbf{Ablation Study.} Average PCK@$0.10$ (for 3 and 4 categories respectively) in CUB and PF-Willow datasets.}
    \vspace{-1em}
    \label{tab:ablation}
    \resizebox{0.9\linewidth}{!}{
    \begin{tabular}{ lcc } 
    \toprule
    \textbf{Ablation} & CUB-200 & PF-Willow \\
                      & (3 categories) & (4 categories) \\
    \midrule
    % Temperature in $\mathcal{L}_\text{KP}$ ($\tau=0.1$)    & 66.8 & 31.4 \\
    % Temperature in $\mathcal{L}_\text{KP}$ ($\tau=0.5$)    & 70.7 & 38.4 \\
    % Temperature in $\mathcal{L}_\text{KP}$ ($\tau=1.0$)    & 58.7 & 30.8 \\
    % Coefficient of $\mathcal{L}_\text{Equi}$  ($\lambda=0$)  & 65.2 & 31.2 \\
    % Coefficient of $\mathcal{L}_\text{Equi}$  ($\lambda=1$)  & 70.7 & 38.4 \\
    % Coefficient of $\mathcal{L}_\text{Equi}$  ($\lambda=10$) & 68.3 & 35.5 \\
    % Coefficient of $\mathcal{L}_\text{Parts}$  ($\lambda=0$)  & 68.1 & 30.3 \\
    % Coefficient of $\mathcal{L}_\text{Parts}$  ($\lambda=1$)  & 70.7 & 30.1 \\
    % Coefficient of $\mathcal{L}_\text{Parts}$  ($\lambda=10$) & 68.9 & 38.4 \\
    % Coefficient of $\mathcal{L}_\text{Recon}$ ($\lambda=0$)  & 69.3 & 32.5\\
    % Coefficient of $\mathcal{L}_\text{Recon}$ ($\lambda=1$)  & 70.7 & 38.4 \\
    % Coefficient of $\mathcal{L}_\text{Recon}$ ($\lambda=10$) & 67.2 & 37.2\\
    % Complete objective (with $\tau=0.5$) & 70.7 & 38.4 \\
    Complete objective & 75.9 & 76.3 \\
    %Lower temperature ($\tau=0.1$)  & 66.8  & 31.4 \\
    %Higher temperature ($\tau=1.0$) & 58.7 & 30.8 \\
    No  $\mathcal{L}_\text{KP}$     & 22.8 & 36.2 \\
    No  $\mathcal{L}_\text{TV}$     & 43.9 & 40.4 \\
    No  $\mathcal{L}_\text{Equi}$   & 64.8 & 65.6 \\
    No  $\mathcal{L}_\text{Recon}$  & 73.3 & 74.2 \\
    No  $\mathcal{L}_\text{Parts}$  & 73.6 & 73.5 \\
    \bottomrule
    \end{tabular}
    }
    % \vspace{-1.2em}
\end{table}


\begin{figure}[!ht]
 \centering
%  \vspace{-1em}
 \includegraphics[width=\linewidth,trim={0 0 0 0}, clip]{imgs/Challenges.pdf}
 \vspace{-1em}
 \caption{\footnotesize \textbf{Limitations.} Top row shows that our model can map left part of object in source to right part of object in target when object is symmetric. Bottom row shows that our model fails for very large out-of-plane rotations.}
 \label{fig:challenges}
 \vspace{-1em}
\end{figure}
\subsection{Limitations}
\noindent\textbf{Left-right ambiguity:}
One shortcoming of our approach is that it cannot differentiate well between left and right parts well for symmetric objects. We attribute this problem to the SSL models being invariant to left-right flips during their training. The top row of \cref{fig:challenges} shows that our model matches the left part of the cow torso in the source image to the right part of the torso in the target image (note left part of cow is not visible in the target image). Some heuristics used in prior works, such as flipping the source and target images and picking a combination that provides minimum total variation loss, could be used in our work as well. For brevity, we provide results without this heuristic.
\noindent\textbf{Large shape changes:}
Our model doesn't handle large viewpoint changes well, especially when there are few intermediate viewpoints. In the bottom row of \cref{fig:challenges}, we see that model is unable to warp the source cow image to target image even for the co-visible portions.

\section{Conclusions}
\label{sec:conclusion}

We propose \backronym, a method to address the challenging task of dense correspondences across images of an object or object category captured in-the-wild. \backronym utilizes noisy and sparse pseudo-correspondences in pre-trained ViT feature space to build an accurate and dense consistent mapping from image to a canonical space. Extensive qualitative and quantitative experiments show that \backronym works in low-shot settings and can deal with extreme variations in pose, background, occlusion, and object deformations.
We also propose a new metric $\kcycle$ to evaluate the consistency of keypoint predictions over a set of images beyond pair-wise consistency.
%A limitation of our work is its dependence on the quality of features learned by the pre-trained ViTs. A way to mitigate this limitation could be to manually annotate seed matches. Despite this limitation, 
\backronym can obtain consistent dense mappings competitive with supervised counterparts with just a few images. In future work, we will explore applications of \backronym in other few-shot downstream tasks such as reconstruction, pose estimation and tracking.


\clearpage
%%%%%%%%% REFERENCES
{\small
\bibliographystyle{ieee_fullname}
\bibliography{egbib}
}

\clearpage
\appendix
\newpage
\appendix

\section*{\LARGE Appendix}


\section{Dataset Details}
\label{sec:dataset_details}

This section describes the details about the dataset we used in experiments (Section~\ref{sec:experiment}).

We use "tiny" version of Taskonomy dataset provided by \citep{taskonomy2018}, which consists of images and labels collected from 35 different buildings.
We use the train and val split for training and early-stopping, respectively, and use the "muleshoe" building included in the test split for evaluation.

To demonstrate our universal few-shot learner, we use ten dense prediction tasks in Taskonomy dataset~\citep{taskonomy2018}, which are semantic segmentation (SS), surface normal (SN), Euclidean distance (ED), Z-buffer depth (ZD), texture edge (TE), occlusion edge (OE), 2D keypoints (K2), 3D keypoints (K3), reshading (RS), and principal curvature (PC).
All labels are normalized into $[0, 1]$ with task-specific pre-processing.
For details on the pre-processing, we refer readers to \cite{taskonomy2018}.
Based on the annotations provided by Taskonomy, we preprocess some tasks to increase the diversity of tasks.
Specifically, we modify three single-channel tasks that can be easily augmented: Euclidean distance, texture edge, and occlusion edge.
\begin{enumerate}[leftmargin=0.5cm]
    \item 
    \textbf{Texture edge} (TE) labels are generated by applying Sobel edge detector~\citep{kanopoulos1988design} to RGB images, which consists of a Gaussian filter and image gradient computation.
    The Gaussian filter has two hyper-parameters, namely kernel size and the standard deviation, where adjusting those hyper-parameters yield different \emph{thickness} of detected edges.
    We use three different sets of hyper-parameters -- $(3, 1), (11, 2), (19, 3)$ -- to produce $3$-channel labels.
    We give an example of each channel of TE task in Figure~\ref{fig:texture_edge_augmentation}.
    
    \item
    \textbf{Euclidean distance} (ED) labels consists of pixel-wise depth map, where the depth is computed by the Euclidean distance from each image pixel to the camera's optical center.
    As this task is very similar to the Z-buffer depth prediction (ZD) whose label pixels are the distance from each image pixel to the camera plane, we augment the ED task by segmenting the depth range and re-normalizing within each segment.
    Specifically, we compute the $5$-quantiles of the pixel-wise depth labels in the whole dataset, then use each quantile as different channels after re-noramlization into $[0, 1]$.
    Thus the objective of each channel of the augmented ED task is to predict Euclidean distance within a specific range, where the ranges are disjoint for different channels.
    We give an example of each channel of ED task in Figure~\ref{fig:euclidean_distance_augmentation}.
    To visualize 5-channel labels, we average the first and the second channels as "R"-channel, the third and the fourth channels as "G"-channel, and use the fifth channel as "B"-channel.
    
    \item
    \textbf{Occlusion edge} (OE) labels are similar to texture edge, but they are constructed to depend on only the 3D geometry rather than color or lighting~\citep{taskonomy2018}.
    We observe that the channel augmentation by quantiles (that we apply to Euclidean distance task) can fairly diversify the labels.
    Therefore, we augment the OE labels into 5-channel labels, where we visualize them similar to the ED labels.
    We give an example of each channel of OE task in Figure~\ref{fig:occlusion_edge_augmentation}.
\end{enumerate}

Also, for semantic segmentation, we exclude three classes ("bottle", "toilet", "book"), as little images of the classes are included in the Taskonomy dataset.
The 12 classes we used in experiments are: "chair", "couch", "plant", "bed", "dining table", "tv", "mircrowave", "oven", "sink", "fridge", "clock", and "base".

\begin{figure}[ht!]
    \centering
    \includegraphics[width=0.8\textwidth]{figure_files/Texture_Edge_Augmentation.pdf}
    \caption{Channel augmentation on texture edge prediction (TE) task. We apply three different sets of hyper-parameters (kernel size, standard deviation) in Sobel edge detector to generate a 3-channel edge task.
    Second to Fourth columns show the augmented channel with different kernel size and standard deviation, where the last column shows the 3-channel label visualized as RGB.}
    \label{fig:texture_edge_augmentation}
\end{figure}
\begin{figure}[ht!]
    \centering
    \includegraphics[width=\textwidth]{figure_files/Euclidean_Distance_Augmentation.pdf}
    \caption{Channel augmentation on Euclidean distance prediction (ED) task. We compute 5-quantiles of the pixel-wise label distribution, and use each $p$-th 5-quantile as each channel after re-normalizing into $[0, 1]$.
    Second to Fifth columns show the augmented channel with different quantile, where the last column shows the 5-channel label visualized as RGB.}
    \label{fig:euclidean_distance_augmentation}
\end{figure}
\begin{figure}[ht!]
    \vspace{-0.2cm}
    \centering
    \includegraphics[width=\textwidth]{figure_files/Occlusion_Edge_Augmentation.pdf}
    \caption{Channel augmentation on occlusion edge prediction (OE) task. We compute 5-quantiles of the pixel-wise label distribution, and use each $p$-th 5-quantile as each channel after re-normalizing into $[0, 1]$.
    Second to Fifth columns show the augmented channel with different quantile, where the last column shows the 5-channel label visualized as RGB.}
    \label{fig:occlusion_edge_augmentation}
\end{figure}


\clearpage
\section{Implementation Details}
\label{sec:implementation_details}

This section describes the implementation details in our experiments (Section~\ref{sec:experiment}).

\subsection{Architecture Details of VTM}
\label{sec:arch-vtm}
\paragraph{Encoders and Decoders}
We employ BEiT-B architecture~\citep{bao2021beit} pretrained on Imagenet-22k dataset~\citep{deng2009imagenet} with $224 \times 224$ resolution as our image encoder.
For our label encoder and decoder, we follow the DPT-B architecture~\citep{ranftl2021vision}.
Specifically, we use a randomly initialized ViT-B~\citep{dosovitskiy2020image} as label encoder $g$ and extract features from $3, 6, 9, 12$-th layers of the encoder to form multi-level label features (label tokens).
Similarly, we extract multi-level image features (image tokens) from $3, 6, 9, 12$-th layers of the image encoder (BEiT).
As the DPT-B architecture decodes four-level features using RefineNet-based decoder~\citep{lin2017refinenet}, we pass the predicted query label features from matching module at each layer to the decoder.
As the label values of tasks in Taskonomy are normalized to $[0, 1]$, we use a sigmoid activation function at the head of the decoder to produce values in $[0, 1]$.
To predict semantic segmentation task whose label values are discrete (either $0$ or $1$), we discretize the predicted label with threshold $0.1$.

\paragraph{Matching Modules}
In the implementation of the matching module with multihead attention, we adopt three conventions in vision transformer~\citep{dosovitskiy2020image} which slightly modifies the equations described in Section~\ref{sec:architecture}.
Recall that the matching module is computed on three input matrices $\mathbf{q}\in\mathbb{R}^{M\times d}$ and $\mathbf{k},\mathbf{v}\in\mathbb{R}^{NM\times d}$ as follows:
\begin{align}
    \text{MHA}(\mathbf{q},\mathbf{k},\mathbf{v}) &= \text{Concat}(\mathbf{o}_1, ..., \mathbf{o}_H)w^O, \\
    \text{where }\mathbf{o}_h &= \text{Softmax}\left(\frac{\mathbf{q}w_h^Q(\mathbf{k}w_h^K)^\top}{\sqrt{d_H}}\right)\mathbf{v}w_h^V,
\end{align}
where $H$ is number of heads, $d_H$ is head size, and $w_h^Q,w_h^K,w_h^V\in\mathbb{R}^{d\times d_H}$, $w^O\in\mathbb{R}^{Hd_H\times d}$.
First, we perform layer normalization~\citep{ba2016layer} before each input projection matrices $w_h^Q, w_h^K, w_h^V$ and after the output projection matrix $w^O$, where we share the layer normalization parameters for $w_h^Q$ and $w_h^K$.
Second, we add a residual connection with GELU non-linearity~\citep{hendrycks2016gaussian} after gathering the outputs from multiple heads as follows:
\begin{align}
    \text{MHA}(\mathbf{q}, \mathbf{k}, \mathbf{v}) &= \mathbf{o} + \text{GELU}(\mathbf{o}w^O), \\
    \text{where}~\mathbf{o} &= \text{Concat}(\mathbf{o}_1, \mathbf{o}_2, \cdots, \mathbf{o}_H).
\end{align}
Finally, we apply Dropout~\citep{srivastava2014dropout} with rate 0.1 in the attention scores.

\subsection{Architecture Details of Baselines}
\label{sec:arch-baseline}
\paragraph{Encoders and Decoders}
For the supervised learning baselines based on transformer encoder (DPT and InvPT), we use the same encoder backbone with ours (BEiT pretrained on ImageNet-22k).
We use the decoder of DPT-B configuration in \cite{ranftl2021vision} for DPT as ours, and use the original multi-task decoder implementation provided by \cite{ye2022inverted} for InvPT.
For few-shot learning baselines (HSNet, VAT, DGPNet), we use ResNet-101~\citep{he2016deep} pretrained on ImageNet-1k~\citep{deng2009imagenet} as their encoder backbones, which is their best configuration.
For the other architectural details, we follow the original implementation of each method provided by \cite{min2021hypercorrelation} (HSNet), \cite{hong2022cost} (VAT), and \cite{johnander2021dense} (DGPNet).

\paragraph{Modification on Few-shot Baselines}
As HSNet and VAT are designed for semantic segmentation, we slightly modify their architectures to train them on general dense prediction tasks.
Specifically, both models involve a binary masking operation to filter out support image features using their labels (which are assumed to be binary), before computing 4D correlation tensor between support and query feature pixels.
For continuous labels of general dense prediction tasks, the binary masking becomes pixel-wise multiplication with labels.
However, as the correlation is computed by cosine similarity between feature pixels that is norm-invariant, all non-zero feature pixels with the same direction are treated in the same manner.
This make them unable to discriminate different non-zero label values, \emph{e.g.}, correlation between query and support feature pixels would be the same regardless of the assigned support label values. 
Therefore, we move the masking operation to after computing the cosine-similarity, so that the models can recognize different non-zero label values through different norms of the masked features by (non-binary) labels.

We use the DGPNet without modification as it is based on a regression method (Gaussian Processes) which is inherently applicable to general dense prediction tasks with continuous labels.


\subsection{Training Details}

\paragraph{Training}
We train all models with 300,000 iterations using the Adam optimizer~\citep{kingma2015adam}, and use \emph{poly} learning rate schedule~\citep{liu2015parsenet} with base learning rates $10^{-5}$ for pre-trained parameters and $10^{-4}$ for parameters trained from scratch.
The models are early-stopped based on the validation metric.
At each episodic training of iteration, we sample a batch of episodes with size 8.
In each episode, we construct a 5-channel task from the training tasks $\mathcal{T}_\text{train}$ by first splitting all channels of training tasks and randomly sample 5 channels among them.
Then support and query sets are sampled for the selected channels, where we use support and query size of 4 for Ours and DGP, while using 1 for HSNet and VAT as they only supports 1-shot training.
To train DPT, we construct a batch of each target task $\mathcal{T}_\text{test}$, whose channels are given at once, with batch size $64$.
To train InvPT, we construct a batch of all ten tasks, whose channels are all given at once, while using batch size $16$ due to its large memory consumption.

\paragraph{Data Augmentation}
We apply random crop (from $256 \times 256$ resolution to $224 \times 224$) and random horizontal flip to images, where the random horizontal flip is applied except for surface normal labels as their values are sensitive to the horizontal direction (flipping images and labels together changes the semantics of the task).
As we apply random crop during training, the resolution of test images ($256 \times 256$) differs from the training images.
To evaluate the models with consistent resolution, we perform five-crop (cropping the four corners and center of an image) to test query images so that the model also predicts five-cropped labels, then aggregate them by averaging the overlapping regions to produce final prediction for evaluation of resolution $(256 \times 256)$.
For few-shot models, we apply center crop to support images at test-time.

\paragraph{Task Augmentation}
For episodic training of few-shot models, we further apply two kinds of task augmentation.
First, for each channel of $C$-channel labels sampled at each episode ($C=5$ in our experiments), we apply random jittering and gaussian blur on each channel independently.
Then we apply MixUp~\citep{zhang2018mixup} on the augmented channels and auxiliary channels which are additionally sampled from the training tasks $\mathcal{T}_\text{train}$, to create a linearly interpolated label of two channels.
We apply the task augmentation consistently in each episode to preserve the task identity.


\clearpage
\section{Additional Results}
\label{sec:additional_results}

This section provides additional results on our experiments (Section~\ref{sec:experiment}).


\subsection{Additional Results on Ablation Study}
\label{sec:additional_results_on_ablation_study}

\subsubsection{Sensitivity to the Choice of Support Set}
\label{sec:support_set_sensitivity}
As discussed in Section~\ref{sec:experiment}, we evaluate the $10$-shot performance of our VTM with four different support sets that are disjointly sampled from the training data $\mathcal{D}_\text{train}$.
We report the results in Table~\ref{tab:support_set_choice}, which shows that our model is robust to the choice of support set.
We use the first support set ($\#1$) in Table~\ref{tab:support_set_choice} for comparison with other baselines or ablated variants in Section~\ref{sec:experiment}, due to the huge computational cost for evaluating few-shot baselines HSNet and VAT.

\begin{table}[ht]
\caption{Ablation study on the choice of support set. We disjointly sample four different support sets and report the $10$-shot performance on each set, with the mean and standard deviation.}
\label{tab:support_set_choice}
\begin{center}
    \renewcommand{\arraystretch}{1.5}
    \renewcommand{\aboverulesep}{0pt}
    \renewcommand{\belowrulesep}{0pt}
    \setlength\tabcolsep{2pt}
    \small
    \begin{tabular}{c|cc|cc|cc|cc|cc}
        \toprule
        \multirow{4}{*}{Support set} &
        \multicolumn{10}{c}{Tasks} \\
        
        \cmidrule{2-11}
        &
        \multicolumn{2}{c|}{Fold 1} & \multicolumn{2}{c|}{Fold 2} & \multicolumn{2}{c|}{Fold 3} & 
        \multicolumn{2}{c|}{Fold 4} & \multicolumn{2}{c}{Fold 5} \\
        
        \cmidrule{2-11}
        &
        SS & SN & ED & ZD & TE & OE & K2 & K3 & RS & PC \\
        &
        mIoU ↑ & mErr ↓ & RMSE ↓ & RMSE ↓ & RMSE ↓ & RMSE ↓ & RMSE ↓ & RMSE ↓ & RMSE ↓ & RMSE ↓ \\
        
        \midrule
        \# 1 &
        0.4097 & 11.4391 & 0.0741 & 0.0316 & 0.0791 & 
		0.0912 & 0.0639 & 0.0519 & 0.1089 & 0.0420 \\

        \# 2 &
        0.4190 & 11.8860 & 0.0845 & 0.0338 & 0.0839 & 
		0.0926 & 0.0629 & 0.0497 & 0.1131 & 0.0437 \\

        \# 3 &
        0.3781 & 11.7418 & 0.0776 & 0.0343 & 0.0807 & 
		0.0944 & 0.0656 & 0.0494 & 0.1101 & 0.0425 \\

        \# 4 &
        0.4017 & 11.6203 & 0.0794 & 0.0362 & 0.0799 & 
		0.0908 & 0.0672 & 0.0502 & 0.1158 & 0.0424 \\
		
	\midrule
        Mean &
        0.4021 & 11.6718 & 0.0789 & 0.0340 & 0.0809 & 
		0.0922 & 0.0649 & 0.0503 & 0.1120 & 0.0427 \\

        Std. &
        0.0152 & 0.1640 & 0.0038 & 0.0016 & 0.0018 & 
		0.0014 & 0.0016 & 0.0010 & 0.0027 & 0.0006 \\
		
        \bottomrule
        
    \end{tabular}
\end{center}
\end{table}


% \paragraph{Ablation Study on Training Procedure}
\subsubsection{Ablation Study on Training Procedure}
\label{sec:training_procedure}
To understand the source of the generalization performance of our method more clearly, we conduct an ablation study on training procedure.
We compare four models based on DPT architecture with different training procedures as follows.
\begin{itemize}[leftmargin=0.5cm]
    \item \textbf{M1}: Randomly initialized DPT, 10-shot trained.
    \item \textbf{M2}: DPT with BEiT pre-trained encoder, 10-shot fine-tuned.
    \item \textbf{M3} (Ours w/o Matching): DPT with BEiT pre-trained encoder, multi-task trained with task-specific bias tuning, and then 10-shot fine-tuned.
    \item \textbf{M4} (Ours): DPT with BEiT pre-trained encoder, meta-trained with task-specific bias tuning, and then 10-shot fine-tuned.

\end{itemize}

\begin{table}[ht]
\vspace{-0.2cm}
\caption{10-shot learning performance of ablated variants of DPT and Ours.}
\vspace{-0.2cm}
\label{tab:training_procedure_ablation}
\begin{center}
    \renewcommand{\arraystretch}{1.5}
    \renewcommand{\aboverulesep}{0pt}
    \renewcommand{\belowrulesep}{0pt}
    \setlength\tabcolsep{2pt}
    \small
    \begin{tabular}{c|cc|cc|cc|cc|cc}
        \toprule
        \multirow{4}{*}{Model} &
        \multicolumn{10}{c}{Tasks} \\
        
        \cmidrule{2-11}
        &
        \multicolumn{2}{c|}{Fold 1} & \multicolumn{2}{c|}{Fold 2} & \multicolumn{2}{c|}{Fold 3} & 
        \multicolumn{2}{c|}{Fold 4} & \multicolumn{2}{c}{Fold 5} \\
        
        \cmidrule{2-11}
        &
        SS & SN & ED & ZD & TE & OE & K2 & K3 & RS & PC \\
        &
        mIoU ↑ & mErr ↓ & RMSE ↓ & RMSE ↓ & RMSE ↓ & RMSE ↓ & RMSE ↓ & RMSE ↓ & RMSE ↓ & RMSE ↓ \\
        
        \midrule
        M1 &
        0.0644 & 21.0976 & 0.1959 & 0.0711 & 0.0995 & 
		0.1842 & 0.0670 & 0.0600 & 0.2335 & 0.0431 \\
		
        M2 &
        0.0582 & 15.8135 & 0.1615 & 0.0530 & 0.1136 & 
		0.1480 & 0.0948 & 0.0606 & 0.1858 & 0.0431 \\
        
        M3 &
        0.2681 & 13.0704 & 0.1111 & 0.0404 & \textbf{0.0778} & 
		0.1061 & \textbf{0.0613} & 0.0537 & 0.1559 & 0.0445 \\
        
        M4 &
        \textbf{0.4097} & \textbf{11.4391} & \textbf{0.0741} & \textbf{0.0316} & 0.0791 & 
		\textbf{0.0912} & 0.0639 & \textbf{0.0519} & \textbf{0.1089} & \textbf{0.0420} \\
        
        \bottomrule
        
    \end{tabular}
\end{center}
\vspace{-0.1cm}
\end{table}
\begin{figure}[ht]
    \centering
    \vspace{-0.3cm}
    \includegraphics[width=\textwidth]{figure_files/Fine-Tuning_Visualization.pdf}
    \caption{Qualitative comparison of Ours and its ablated variants in training procedure. All models use 10 labeled examples for each target task, where M3 and M4 observe additional labeled examples of training tasks (different from the target task) in each fold.
    }
    \label{fig:training_procedure_qualitative}
    \vspace{-0.3cm}
\end{figure}

We summarize the quantitative result in Table~\ref{tab:training_procedure_ablation} and qualitative comparison in Figure~\ref{fig:training_procedure_qualitative}.
First, as expected, we observe that DPT with naive 10-shot training (M1) fails to generalize to the test examples in most of the tasks, except for two 2D texture-related tasks (TE, K2). We conjecture that TE and K2 are “easy” cases in terms of few-shot learning, as they are defined as low-level computational algorithms on RGB images, while other high-level tasks require knowledge about semantics (SS) or 3D space (SN, ED, ZD, OE, K3, RS, PC).
Second, we note that BEiT pretraining (M2) largely improves the few-shot generalization performance, allowing the model to produce coarse predictions of the dense labels. However, it still cannot capture object-level fine-grained details in many tasks.
Third, we observe that multi-task training and few-shot adaptation, combined with an efficient parameter-sharing strategy of bias tuning (M3, M4), further improves the performance with a clear gap with M2 where the predictions are also qualitatively finer than M2’s.
Finally, as discussed in Section~\ref{sec:ablation_study}, M4 still further improves over M3 with a clear gap. This shows that in a few-shot learning setting, our matching framework and episodic training are more effective than simple multi-task pretraining employed in M3.
In summary, we may conclude that the fast generalization of Ours is benefitted from episodic training of various tasks followed by parameter-efficient few-shot adaptation as well as powerful pre-training of the encoder (BEiT).



\subsubsection{Fine-tuning with Full Supervision}
To further explore how our method scales well when a large labeled dataset is given, we also fine-tuned our VTM with full supervision of test tasks.
For the fine-tuning, we used the same training dataset as the fully-supervised DPT and employed the episodic fine-tuning objective (Section 3.3). For evaluation, since providing the entire training data as the support set for the matching module is infeasible, we provide a random subset of the training data as the support set to the model.
We summarize the result in Figure~\ref{fig:performance_on_shots_with_full}, which extends Figure~\ref{fig:performance_on_shots} in Section~\ref{sec:experiment}.
In most tasks, our model consistently improves when more supervision is given.
With full supervision at test tasks, our model performs slightly worse than the DPT baseline in seven tasks and performs better or similarly in the other three tasks.
We conjecture that the performance degradation comes from two aspects: (1) the absence of direct input-output connection, \emph{i.e.}, the matching module serves as a bottleneck, and (2) negative transfer from meta-training tasks to test tasks.

\begin{figure}[ht!]
    \centering
    \vspace{-0.3cm}
    \includegraphics[width=\textwidth]{figure_files/Performance_on_Shots_v3.pdf}
    \caption{Performance of VTM on various shots.
    In general, VTM consistently improves performance as more supervision is given, and even surpasses fully supervised baselines on many tasks.
    }
    \label{fig:performance_on_shots_with_full}
    \vspace{-0.3cm}
\end{figure}

\subsubsection{Effect of Number of Training Tasks}
\label{sec:number_of_training_tasks}
The amount of meta-training tasks is an important factor that can affect the performance of the universal few-shot learner.
To verify this, we fixed two test tasks (SS, SN) and trained our VTM on five different subsets of the original eight training tasks (three different subsets with two tasks and two different subsets with five tasks).
We summarize the results in the Table~\ref{tab:training_tasks_ablation}.
As expected, the performance consistently improves as we increase the number of training tasks.
We also note that the few-shot performance becomes sensitive to the choice of training tasks when their number is small (two), presumably as the model becomes reliant on training tasks more correlated to test tasks, while the variance decreases substantially when more training tasks are added.
In addition, the experiment with incomplete training data (Appendix~\ref{sec:incomplete_experiment}) shows the potential ability of our methods in more realistic settings where the training dataset is formed by a combination of different task-specific datasets.
From these results, we expect that our model can further enhance its universality on few-shot learning by utilizing a combined training dataset of much more diverse tasks, which we leave as future work.

\begin{table}[ht]
\caption{10-shot learning performance of Ours with various number of training tasks.}
\vspace{-0.2cm}
\label{tab:training_tasks_ablation}
\begin{center}
    \renewcommand{\arraystretch}{1.5}
    \renewcommand{\aboverulesep}{0pt}
    \renewcommand{\belowrulesep}{0pt}
    \setlength\tabcolsep{6pt}
    \small
    \begin{tabular}{c|cc}
        \toprule
        \multirow{4}{*}{Number of Training Tasks} &
        \multicolumn{2}{c}{Tasks} \\
        
        \cmidrule{2-3}
        &
        \multicolumn{2}{c}{Fold 1} \\
        
        \cmidrule{2-3}
        &
        SS & SN \\
        &
        mIoU ↑ & mErr ↓ \\
        
        \midrule
        2 &
        0.2878 ± 0.0565 & 17.3947 ± 4.8742 \\
		
        5 &
        0.3919 ± 0.0132 & 12.6769 ± 0.1235 \\
        
        8 &
        \textbf{0.4097} & \textbf{11.4391} \\
        
        \bottomrule
        
    \end{tabular}
\vspace{-0.2cm}
\end{center}
\end{table}


\subsubsection{Episodic Training with Incomplete Dataset}
\label{sec:incomplete_experiment}
It would make our method more practical if the model could learn from an incomplete dataset where images are not associated with whole training task labels.
To see how our framework extends to such incomplete settings, we conducted an additional experiment.
We simulate the extreme case of incomplete data by partitioning the training images, such that each image is associated with only a single task out of 8 training tasks.
Specifically, we partitioned the buildings in Taskonomy into eight groups – each corresponds to a different training task.
As this reduces the effective size of training data by the number of training tasks (1/8 in our case), we also train a baseline where we use complete data but use only 1/8 of the training images (for each building, we discard 7/8 of the images).
The results are summarized in Table~\ref{tab:incomplete_dataset}.
We can see that the performance degradation is marginal when we give incomplete data, which implies that our method can be promising in handling realistic scenarios where the training data is a collection of heterogeneous datasets with different label annotations.

\begin{table}[ht]
\caption{10-shot learning performance of Ours trained with incomplete and complete multi-task dataset.}
\vspace{-0.2cm}
\label{tab:incomplete_dataset}
\begin{center}
    \renewcommand{\arraystretch}{1.5}
    \renewcommand{\aboverulesep}{0pt}
    \renewcommand{\belowrulesep}{0pt}
    \setlength\tabcolsep{6pt}
    \small
    \begin{tabular}{c|cc}
        \toprule
        \multirow{4}{*}{Training Data} &
        \multicolumn{2}{c}{Tasks} \\
        
        \cmidrule{2-3}
        &
        \multicolumn{2}{c}{Fold 1} \\
        
        \cmidrule{2-3}
        &
        SS & SN \\
        &
        mIoU ↑ & mErr ↓ \\
        
        \midrule
        Incomplete (one task per building) &
        0.3559 & 13.6207 \\

        Complete (1/8 training data) &
        0.3980 & 12.1633 \\
        
        Complete (whole training data) &
        0.4097 & 11.4391 \\
        
        \bottomrule
        
    \end{tabular}
\vspace{-0.2cm}
\end{center}
\end{table}


\subsection{Further Analysis}

\subsubsection{Parameter-Efficiency Analysis}
We report the number of task-specific and shared parameters of our VTM and two supervised baselines, DPT and InvPT, to compare how our task adaptation is parameter-efficient.
As DPT is a single-task learning model, no parameters are shared across tasks and the whole network should be trained independently for every new task.
InvPT, which is a multi-task learning model, shares a large portion of its parameters across tasks (\emph{e.g.}, encoder backbone), still consumes many parameters for each task in the decoder.

Due to the extensive amount of parameter-sharing, our method is also promising in continual learning setting.
As all task-specific knowledge is included in the bias parameters of the image encoder, the knowledge acquired from past tasks can be recalled without forgetting by keeping the corresponding bias parameters and switching to them whenever a past model is needed.
We especially note that the size of bias parameters is fairly small (288 KB, which amounts to keeping about 3 labeled images of 256x256 resolution for each task).
This allows our model to retain past knowledge very efficiently by keeping the tuned bias parameters plus a few-shot support set, whose external memory requirement is far less compared to memory-based approaches in continual learning that keep hundreds of images~\citep{bang2021rainbow,wang2022continual}.
While the continual learning setting is not our main focus, applying our method to a continual learning setting would be an interesting future direction.

\begin{table}[ht]
\caption{Number of task-specific and shared parameters for a single-channel task (in million).}
\vspace{-0.2cm}
\label{tab:number_of_parameters}
\begin{center}
    \renewcommand{\arraystretch}{1.5}
    \small
    \begin{tabular}{cccc}
        \toprule
        Model & Task-Specific & Shared \\
        \midrule
        DPT (supervised learning) & 110.55 & 0 \\
        InvPT (multi-task learning) & 24.57 & 106.75 \\
        Ours (few-shot learning) & 0.0703 & 202.95 \\
        \bottomrule
        
    \end{tabular}
\vspace{-0.2cm}
\end{center}
\end{table}

\subsubsection{Computation Cost Analysis}
To analyze how our method is computationally efficient compared to supervised DPT, we measured the MACs (multiply–accumulate operations) of our model and DPT using an open-source python library thop~\footnote{https://github.com/Lyken17/pytorch-OpCounter}.
We report the results in Table~\ref{tab:computation_cost}.
Having encoded the support set (e.g., 10-shot), we can see that the computational cost of our model’s inference on a single query image is about 30\% larger than the cost of DPT’s, due to the Matching part.

\begin{table}[ht]
\caption{MACs of Ours and DPT on a single-query inference for a single-channel task.}
\vspace{-0.2cm}
\label{tab:computation_cost}
\begin{center}
    \renewcommand{\arraystretch}{1.5}
    \small
    \begin{tabular}{ccc}
        \toprule
        Model & MACs (G) \\
        \midrule
        DPT & 30.15 \\
        Ours after encoding support (10-shot) & 38.79 \\
        \bottomrule
        
    \end{tabular}
\vspace{-0.2cm}
\end{center}
\end{table}

\subsubsection{Role of Attention Heads}
To analyze the role of attention heads, in Figure~\ref{fig:multihead_attention}, we visualized the attention maps for each head over support images for a given query patch, feature level (3rd level in this example), and task (RS in this example).
The figure shows that each head attends to different regions of the support images.
Moreover, we can find some patterns in heads; for example, the first head tends to attend to flat areas of the scene, such as the floor or ceiling (low-frequency features), while the third head tends to attend to objects, such as couch or plant (high-frequency features).
To further verify the benefit of multi-head attention in the matching module, we also trained our VTM with single head in the matching modules.
The result is summarized in the table below and Table~\ref{tab:attention_heads}.
We can see the performance drop in both SS and SN tasks, which supports that exploiting multiple heads benefits our matching framework.

\begin{figure}[ht]
    \centering
    \includegraphics[width=\textwidth]{figure_files/Multihead_Attention_Visualization.pdf}
    \caption{Visualization of multi-head attention maps of VTM. Here we visualize the matching module at 3rd level for reshading (RS) task.
    }
    \label{fig:multihead_attention}
\end{figure}
\begin{table}[ht]
\caption{10-shot learning performance of Ours with different number of attention heads in Matching module.}
\label{tab:attention_heads}
\begin{center}
    \renewcommand{\arraystretch}{1.5}
    \renewcommand{\aboverulesep}{0pt}
    \renewcommand{\belowrulesep}{0pt}
    \setlength\tabcolsep{6pt}
    \small
    \begin{tabular}{c|cc}
        \toprule
        \multirow{4}{*}{Number of Attention Heads} &
        \multicolumn{2}{c}{Tasks} \\
        
        \cmidrule{2-3}
        &
        \multicolumn{2}{c}{Fold 1} \\
        
        \cmidrule{2-3}
        &
        SS & SN \\
        &
        mIoU ↑ & mErr ↓ \\
        
        \midrule
        1 &
        0.3702 & 12.5936 \\
        
        4 &
        \textbf{0.4097} & \textbf{11.4391} \\
        
        \bottomrule
        
    \end{tabular}
\end{center}
\end{table}


\clearpage
\subsection{Additional Qualitative Comparison with Baselines}
\label{sec:additional_qualitative_comparison_with_baselines}

We provide additional results on the qualitative evaluation of our model and the baselines.
Figure~\ref{fig:appendix_comparison_1}-\ref{fig:appendix_comparison_4} show visualizations on different query image and support set, where we vary the class of semantic segmentation task included in each support.
The result shows consistent trends of that we discussed in Section~\ref{sec:experiment}.
Ours is competitive to the fully supervised baselines (DPT and InvPT), while the other few-shot baselines (HSNet, VAT, DGPNet) fail to learn different dense prediction tasks.


In Figure~\ref{fig:appendix_comparison_2}, even the GT label for semantic segmentation ("couch" class) is noisy as it is a pseudo-label generated by a pre-trained segmentation model~\citep{taskonomy2018}, our model successfully segments two couches present in the figure.
This can be attributed to the task-agnostic architecture of VTM based on non-parametric matching.

\begin{figure}[ht]
    \centering
    \includegraphics[width=\textwidth]{figure_files/Appendix_Comparison_1.pdf}
    \caption{Additional results of qualitative comparison between Ours and the baselines.
    }
    \label{fig:appendix_comparison_1}
\end{figure}

\begin{figure}[ht]
    \centering
    \includegraphics[width=\textwidth]{figure_files/Appendix_comparison_2.pdf}
    \caption{Additional results of qualitative comparison between Ours and the baselines.
    }
    \label{fig:appendix_comparison_2}
\end{figure}

\begin{figure}[ht]
    \centering
    \includegraphics[width=\textwidth]{figure_files/Appendix_comparison_3.pdf}
    \caption{Additional results of qualitative comparison between Ours and the baselines.
    }
    \label{fig:appendix_comparison_3}
\end{figure}

\begin{figure}[ht]
    \centering
    \includegraphics[width=\textwidth]{figure_files/Appendix_comparison_4.pdf}
    \caption{Additional results of qualitative comparison between Ours and the baselines.
    }
    \label{fig:appendix_comparison_4}
\end{figure}


\clearpage
\subsection{Additional Qualitative Comparison with Our Variants}
\label{sec:additional_qualitative_comparison_with_our_variants}

We also provide additional results on the qualitative evaluation of our model and our ablated variants, Ours w/o Matching and Ours w/o Adaptation.
Figure~\ref{fig:appendix_ablation_1}-\ref{fig:appendix_ablation_4} show visualizations on different query image and support set. 
The results show a consistent trend with the quantitative results in Table~\ref{tab:main_table}.
Interestingly, our method without adaptation already exhibits some degree of adaptation to the unseen tasks even without fine-tuning and task-specific components, showing that the non-parametric architecture of our model and the parameter sharing derived from is appropriate to learn generalizable knowledge to understand the novel tasks.
On the other hand, adding a task-specific component and adaptation mechanism to the model allows more dramatic improvement in understanding novel tasks from few-shot examples, showing the importance of the adaptation mechanism in our task.
Finally, we observe that equipping the matching mechanism with the adaptation module provides much sharper and fast adaptation to the unseen tasks, which verifies our claims.


\begin{figure}[ht]
    \centering
    \includegraphics[width=\textwidth]{figure_files/Appendix_abaltion_1.pdf}
    \caption{Additional results of qualitative comparison between Ours and its ablated variants.
    }
    \label{fig:appendix_ablation_1}
\end{figure}

\begin{figure}[ht]
    \centering
    \includegraphics[width=\textwidth]{figure_files/Appendix_abaltion_2.pdf}
    \caption{Additional results of qualitative comparison between Ours and its ablated variants.
    }
    \label{fig:appendix_ablation_2}
\end{figure}

\begin{figure}[ht]
    \centering
    \includegraphics[width=\textwidth]{figure_files/Appendix_abaltion_3.pdf}
    \caption{Additional results of qualitative comparison between Ours and its ablated variants.
    }
    \label{fig:appendix_ablation_3}
\end{figure}

\begin{figure}[ht]
    \centering
    \includegraphics[width=\textwidth]{figure_files/Appendix_abaltion_4.pdf}
    \caption{Additional results of qualitative comparison between Ours and its ablated variants.
    }
    \label{fig:appendix_ablation_4}
\end{figure}

\end{document}