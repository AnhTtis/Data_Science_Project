\section{Conclusion}
In this paper, we propose UTSP, an Unsupervised Learning method to solve the TSP. 
We build a surrogate loss that encourages the GNN to find the shortest path and satisfy the constraint that the path should be a Hamiltonian Cycle. The surrogate loss function does not rely on any labelled ground truth solution and helps alleviate sparse reward problems in RL.
UTSP uses a two-phase strategy. We first build a heat map based on the GNN's output. The heat map is then fed into a search algorithm. Compared with RL/SL, our method vastly reduces training cost and takes fewer training samples. We further show that our UL training helps reduce the search space. This helps explain why the generated heat maps can guide the search algorithm.  
On the model side, our results indicate that a low-pass GNN will produce an indistinguishable representation due to the oversmoothing issue, which results in  unfavorable heat maps and fails to reduce the search space. Instead, after incorporating band-pass operators into GNN, we can build efficient heat maps that successfully reduce search space. Our findings show that the expressive power of GNNs is critical for generating a non-smooth representation that helps find the solution. 


In conclusion, UTSP is competitive with or outperforms other learning-based TSP heuristics in terms of solution quality and running speed. In addition, UTSP  takes $\sim$ 10\% of the
number of parameters and $\sim$ 0.2\% of (unlabelled) training samples, compared with RL or SL methods. Our UTSP framework demonstrates that by providing a surrogate loss and a GNN which encourages a non-smooth representation, we can learn the hidden patterns in TSP instances without supervision and further reduce the search space. This allows us to build a heuristic by exploiting a small amount of unlabelled data.
Future directions include designing more expressive GNNs (such as adding edge features)  and using different surrogate loss functions. We anticipate that these concepts will extend to more combinatorial problems.
