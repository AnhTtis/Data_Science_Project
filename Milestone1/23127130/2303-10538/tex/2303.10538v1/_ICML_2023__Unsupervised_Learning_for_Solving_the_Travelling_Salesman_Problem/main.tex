%%%%%%%% ICML 2022 EXAMPLE LATEX SUBMISSION FILE %%%%%%%%%%%%%%%%%

\documentclass[nohyperref]{article}

% Recommended, but optional, packages for figures and better typesetting:
\usepackage{microtype}
\usepackage{graphicx}
\usepackage{subfigure}
\usepackage{booktabs} % for professional tables
\usepackage{xcolor}

\newcommand{\Carla}[1]{\textcolor{blue}{ #1}}

\setlength{\tabcolsep}{4pt} % setting column spaces



% hyperref makes hyperlinks in the resulting PDF.
% If your build breaks (sometimes temporarily if a hyperlink spans a page)
% please comment out the following usepackage line and replace
% \usepackage{icml2022} with \usepackage[nohyperref]{icml2022} above.
\usepackage{hyperref}


% Attempt to make hyperref and algorithmic work together better:
\newcommand{\theHalgorithm}{\arabic{algorithm}}

% Use the following line for the initial blind version submitted for review:
% \usepackage{icml2023}
% \usepackage{float}
% If accepted, instead use the following line for the camera-ready submission:
\usepackage[accepted]{icml2023}
% For theorems and such
\usepackage{amsmath}
\usepackage{amssymb}
\usepackage{mathtools}
\usepackage{amsthm}

% if you use cleveref..
\usepackage[capitalize,noabbrev]{cleveref}

%%%%%%%%%%%%%%%%%%%%%%%%%%%%%%%%
% THEOREMS
%%%%%%%%%%%%%%%%%%%%%%%%%%%%%%%%
\theoremstyle{plain}
\newtheorem{theorem}{Theorem}[section]
\newtheorem{proposition}[theorem]{Proposition}
\newtheorem{lemma}[theorem]{Lemma}
\newtheorem{corollary}[theorem]{Corollary}
\theoremstyle{definition}
\newtheorem{definition}[theorem]{Definition}
\newtheorem{assumption}[theorem]{Assumption}
\theoremstyle{remark}
\newtheorem{remark}[theorem]{Remark}
% \newcommand{\Yiwei}[1]{\textcolor{red}{ #1}}
% Todonotes is useful during development; simply uncomment the next line
%    and comment out the line below the next line to turn off comments
%\usepackage[disable,textsize=tiny]{todonotes}
\usepackage[textsize=tiny]{todonotes}


% The \icmltitle you define below is probably too long as a header.
% Therefore, a short form for the running title is supplied here:

%%%%add by Yimeng
\usepackage{multirow}
\newcommand{\tabincell}[2]{\begin{tabular}{@{}#1@{}}#2\end{tabular}}


%%%%%

% \icmltitlerunning{Submission and Formatting Instructions for ICML 2023}

\begin{document}

\twocolumn[
\icmltitle{Unsupervised Learning for Solving the Travelling Salesman Problem}

% It is OKAY to include author information, even for blind
% submissions: the style file will automatically remove it for you
% unless you've provided the [accepted] option to the icml2022
% package.

% List of affiliations: The first argument should be a (short)
% identifier you will use later to specify author affiliations
% Academic affiliations should list Department, University, City, Region, Country
% Industry affiliations should list Company, City, Region, Country

% You can specify symbols, otherwise they are numbered in order.
% Ideally, you should not use this facility. Affiliations will be numbered
% in order of appearance and this is the preferred way.
\icmlsetsymbol{equal}{*}

\begin{icmlauthorlist}
\icmlauthor{Yimeng Min}{equal,yyy}
\icmlauthor{Yiwei Bai}{equal,yyy}
\icmlauthor{Carla P. Gomes}{yyy}
% \icmlauthor{Firstname4 Lastname4}{sch}
% \icmlauthor{Firstname5 Lastname5}{yyy}
% \icmlauthor{Firstname6 Lastname6}{sch,yyy,comp}
% \icmlauthor{Firstname7 Lastname7}{comp}
%\icmlauthor{}{sch}
% \icmlauthor{Firstname8 Lastname8}{sch}
% \icmlauthor{Firstname8 Lastname8}{yyy,comp}
%\icmlauthor{}{sch}
%\icmlauthor{}{sch}
\end{icmlauthorlist}

\icmlaffiliation{yyy}{Department of Computer Science, Cornell University, Ithaca, NY, 14850, USA}
% \icmlaffiliation{comp}{Company Name, Location, Country}
% \icmlaffiliation{sch}{School of ZZZ, Institute of WWW, Location, Country}
\icmlcorrespondingauthor{Yimeng Min}{min@cs.cornell.edu}
% \icmlcorrespondingauthor{Firstname2 Lastname2}{first2.last2@www.uk}

% You may provide any keywords that you
% find helpful for describing your paper; these are used to populate
% the "keywords" metadata in the PDF but will not be shown in the document
\icmlkeywords{Travelling Salesman Problem, GNN}

\vskip 0.3in
]

% this must go after the closing bracket ] following \twocolumn[ ...

% This command actually creates the footnote in the first column
% listing the affiliations and the copyright notice.
% The command takes one argument, which is text to display at the start of the footnote.
% The \icmlEqualContribution command is standard text for equal contribution.
% Remove it (just {}) if you do not need this facility.

% \printAffiliationsAndNotice{}  % leave blank if no need to mention equal contribution
%
\printAffiliationsAndNotice{\icmlEqualContribution}
% otherwise use the standard text.

\begin{abstract}


Over the past few years, there has been a significant amount of research focused on studying the ReLU activation function, with the aim of achieving neural network convergence through over-parametrization. However, recent developments in the field of Large Language Models (LLMs) have sparked interest in the use of exponential activation functions, specifically in the attention mechanism.

Mathematically, we define the neural function $F: \R^{d \times m} \times  \mathbb{R}^d \rightarrow \mathbb{R}$ using an exponential activation function. Given a set of data points with labels $\{(x_1, y_1), (x_2, y_2), \dots, (x_n, y_n)\} \subset \mathbb{R}^d \times \mathbb{R}$ where $n$ denotes the number of the data. Here $F(W(t),x)$ can be expressed as $F(W(t),x) := \sum_{r=1}^m a_r \exp(\langle w_r, x \rangle)$, where $m$ represents the number of neurons, and $w_r(t)$ are weights at time $t$. It's standard in literature that $a_r$ are the fixed weights and it's never changed during the training. We initialize the weights $W(0) \in \mathbb{R}^{d \times m}$ with random Gaussian distributions, such that $w_r(0) \sim \mathcal{N}(0, I_d)$ and initialize $a_r$ from random sign distribution for each $r \in [m]$.

Using the gradient descent algorithm, we can find a weight $W(T)$ such that $\| F(W(T), X) - y \|_2 \leq \epsilon$ holds with probability $1-\delta$, where $\epsilon \in (0,0.1)$ and $m = \Omega(n^{2+o(1)}\log(n/\delta))$. To optimize the over-parametrization bound $m$, we employ several tight analysis techniques from previous studies [Song and Yang arXiv 2019, Munteanu, Omlor, Song and Woodruff ICML 2022]. 

 

% This document provides a basic paper template and submission guidelines.
% Abstracts must be a single paragraph, ideally between 4--6 sentences long.
% Gross violations will trigger corrections at the camera-ready phase.
\end{abstract}



% \twocolumn[
% % \onecolumn
\begin{table*}[htb]
\footnotesize
\centering
\caption{Results of SAG + Local Search  w.r.t. existing baselines, tested on 10,000 instances with $n$ = 20, 50 and 100.}
%\Yiwei{[Yiwei: I keep the two results here only for comparison, I think finally we only keep the best one. I did not tune too much parameters (2 - 4 sets for each TSP variant). I would like to tune and search a longer one for TSP10000.]}
\vskip 0.15in
\label{table:Exps01}
    \begin{tabular}{lllllllllll}
    \toprule
    %   \toprule[2pt]
       \multirow{2}{*}{Method} & \multirow{2}{*}{Type}  & \multicolumn{3}{c}{TSP20} & \multicolumn{3}{c}{TSP50} & \multicolumn{3}{c}{TSP100}\\
         &  & Length & Gap (\%)  & Time & Length & Gap (\%) & Time & Length & Gap (\%)  & Time \\
        \hline
        Concorde  & Solver & {3.8303} & {0.0000 } & {2.31m} & {5.6906} & {0.0000 } & {13.68m} & {7.7609} & {0.0000 } & {1.04h}  \\
        Gurobi  & Solver & {3.8302} & {-0.0001 } & {2.33m} & {5.6905} & {0.0000 } & {26.20m} & {7.7609} & {0.0000 } & {3.57h}  \\
        LKH3  & Heuristic & {3.8303} & {0.0000 } & {20.96m} & {5.6906} & {0.0008 } & {26.65m} & {7.7611} & {0.0026 } & {49.96m}  \\
        % \hline
         GAT \citep{deudon2018learning}  & RL, S &  {3.8741} &  {1.1443 } &   {10.30m} &  {6.1085} &  {7.3438 } &   {19.52m} &  {8.8372} &  {13.8679 } &   {47.78m}  \\

        GAT \citep{deudon2018learning}  & \tabincell{c}{RL, S\\ 2-OPT} &  {3.8501} &  {0.5178 } &  {15.62m} & {5.8941} & {3.5759 } & {27.81m} & {8.2449} & {6.2365 } & {4.95h}  \\

        GAT \citep{kool2018attention}  & RL, S &  {3.8322} &  {0.0501 } &   {16.47m}  & {5.7185} & {0.4912 } &  {22.85m} & {7.9735} & {2.7391 } & {1.23h}  \\

        GAT \citep{kool2018attention}  & RL, G & {3.8413} & {0.2867 } & {6.03s} & {5.7849} & {1.6568 } & {34.92s} & {8.1008} & {4.3791 } & {1.83m} \\

        GAT \citep{kool2018attention}  & RL, BS &  {3.8304} &  {0.0022 } &  {15.01m} & {5.7070} & {0.2892 } & {25.58m} & {7.9536} & {2.4829 } & {1.68h}  \\

        GCN \citep{joshi2019efficient} & SL, G & {3.8552} & {0.6509 } &  {{19.41s}} & {5.8932} & {3.5608 } &  {2.00m} & {8.4128} & {8.3995 } &  {11.08m}  \\

        GCN \citep{joshi2019efficient} & SL, BS &  {3.8347} &  {0.1158 } &   {21.35m} &  {5.7071} &  {0.2905 } &   {35.13m} &  {7.8763} &  {1.4828 } &   {31.80m}  \\

        GCN \citep{joshi2019efficient} & SL, BS* & {3.8305} & {0.0075 } & {22.18m} & {5.6920} & {0.0251 } & {37.56m} & {7.8719} & {1.4299 } & {1.20h}  \\

        \hline
        \multirow{2}{*} {Att-GCRN\citep{fu2021generalize}} & \multirow{2}{*}{\tabincell{c}{SL+RL \\MCTS}} & \multirow{2}{*}{{3.8300}} & \multirow{2}{*}{\textbf{-0.0074}} & {23.33s} +  & \multirow{2}{*}{5.6908} & \multirow{2}{*}{{0.0032 }} & {2.59m} +  & \multirow{2}{*}{{7.7616}} & \multirow{2}{*}{{0.0096 }} & {3.94m} +   \\
        & & & & 1.05m & & & 2.63m & & & 5.25m\\ \hline

        % \multirow{2}{*} {\tabincell{c}{Random Init ($T = 0.01$)}} & \multirow{2}{*}{MCTS} & \multirow{2}{*}{{NA}} & \multirow{2}{*}{{NA}} & {NA} +  & \multirow{2}{*}{NA} & \multirow{2}{*}{{NA}} & {NA} +  & \multirow{2}{*}{{7.8752}} & \multirow{2}{*}{1.473 } & {0+}   \\
        % & & & & NA & & & NA & & & 6m\\     \hline
        %  \multirow{2}{*} {\tabincell{c}{Random Init ($T = 0.02$)}} & \multirow{2}{*}{MCTS} & \multirow{2}{*}{{NA}} & \multirow{2}{*}{{NA}} & {NA} +  & \multirow{2}{*}{NA} & \multirow{2}{*}{{NA}} & {NA} +  & \multirow{2}{*}{{7.8483}} & \multirow{2}{*}{1.126 } & {0+}   \\
        % & & & & NA & & & NA & & & 12m\\     \hline
        %   \multirow{2}{*} {\tabincell{c}{Uniform Init ($T = 0.01$)}} & \multirow{2}{*}{MCTS} & \multirow{2}{*}{{NA}} & \multirow{2}{*}{{NA}} & {NA} +  & \multirow{2}{*}{NA} & \multirow{2}{*}{{NA}} & {NA} +  & \multirow{2}{*}{{7.8332}} & \multirow{2}{*}{0.932 } & {0+}   \\
        % & & & & NA & & & NA & & & 6m\\     \hline     
        %   \multirow{2}{*} {\tabincell{c}{Uniform Init ($T = 0.02$)}} & \multirow{2}{*}{MCTS} & \multirow{2}{*}{{NA}} & \multirow{2}{*}{{NA}} & {NA} +  & \multirow{2}{*}{NA} & \multirow{2}{*}{{NA}} & {NA} +  & \multirow{2}{*}{{7.8084}} & \multirow{2}{*}{0.612 } & {0+}   \\
        % & & & & NA & & & NA & & & 12m\\     \hline
        %\usepackage{supertabular}??
        \multirow{2}{*} {{\bf{\small{UTSP} (ours)}} } & \multirow{2}{*}{UL, Search} & \multirow{2}{*}{{3.8303}} & \multirow{2}{*}{-0.0009} & {38.23s} +  & \multirow{2}{*}{5.6894} & \multirow{2}{*}{\textbf{-0.0200}} & {1.34m} +  & \multirow{2}{*}{{7.7608}} & \multirow{2}{*}{\textbf{-0.0011 }} & {5.68m+}   \\
        & & & & 1.04m & & & 2.60m & & & 5.21m\\ 
        % \multirow{1}{*} {\textbf{Ours (Restart)} ($T = 0.01$s)} & \multirow{1}{*}{UL, Search} & \multirow{1}{*}{{N/A}} & \multirow{1}{*}{\textbf{N/A}} & {38.23s} +  & \multirow{1}{*}{N/A} & \multirow{1}{*}{\textbf{-0.02}} & {1.34m} +  & \multirow{1}{*}{{7.760}} & \multirow{1}{*}{\textbf{-0.011 }} & {5.68m+}   \\
        % & & & & 1.04m & & & 2.60m & & & 5.21m\\ 
\bottomrule
        % \bottomrule[2pt]
    \end{tabular}
\end{table*}
% \twocolumn
% ]


% In the unusual situation where you want a paper to appear in the
% references without citing it in the main text, use \nocite
\section{Introduction}
\label{sec:introduction}
% \begin{itemize}
%     % Diffusion of FL
%     \item {\st{Diffusion of FL}}
%     % Security threats to FL
%     \item {\st{Security threats to FL with particular focus on model poisoning}}
%     % Limitations of existing countermeasures
%     \item {\st{Current countermeasures (e.g., KRUM) and their limitations}}
%     % Proposed method and its advantages
%     \item {\st{Intuitive description of the proposed method and its difference (i.e., advantages) w.r.t. state of the art}}
%     % Main contributions
%     \item {\st{Summary of the main contributions of this work}}
%     % Paper's structure and organization
%     \item {\st{Paper's structure and organization}}
% \end{itemize}

% Diffusion of FL
Recently, {\em federated learning} (FL) has emerged as the leading paradigm for training distributed, large-scale, and privacy-preserving machine learning (ML) systems~\cite{mcmahan2017googleai,mcmahan2017aistats}. 
The core idea of FL is to allow multiple edge clients to collaboratively train a shared, global model without disclosing their local private training data.
%Specifically, an FL system consists of a central server and many edge clients; 
A typical FL round involves the following steps: {\em(i)} the server randomly picks some clients and sends them the current, global model; {\em(ii)} each selected client locally trains its model with its own private data; then, it sends the resulting local model to the server;\footnote{Whenever we refer to global/local model, we mean global/local model {\em parameters}.} {\em(iii)} the server updates the global model by computing an \emph{aggregation function}, usually the average (FedAvg), on the local models received from clients.
% \begin{enumerate}
%     \item[{\em(i)}] the server sends the current, global model to the clients and appoints some of them for training;
%     \item[{\em(ii)}] each selected client locally trains its copy of the global model with its own private data; then, it sends the resulting local model back to the server;\footnote{Whenever we refer to global/local model, we mean global/local model {\em parameters}.}
%     \item[{\em(iii)}] the server updates the global model by computing an \emph{aggregation function} on the local models received from clients (by default, the average, also referred to as FedAvg~\cite{mcmahan2017aistats}).
% \end{enumerate}
This process goes on until the global model converges. %(e.g., after a certain number of rounds or other similar stopping criteria).
%\\
% The advantages of FL over the traditional, centralized learning paradigm are undoubtedly clear in terms of flexibility/scalability (clients can join/disconnect from the FL network dynamically), network communications (only model weights\footnote{We will use \textit{parameters} and \textit{weights} interchangeably.} are exchanged between clients and server), and privacy (each client's private training data is kept local at the client's end and not uploaded to the server).
\\
% Security threats to FL
%However, the growing adoption of FL also raises security concerns~\cite{costa2022covert}, particularly about its confidentiality, integrity, and availability.
Although its advantages over standard ML, FL also raises security concerns~\cite{costa2022covert}. %, particularly about its confidentiality, integrity, and availability~\cite{costa2022covert}.
% OLD, LONG VERSION
% Indeed, some work deals with privacy leakage that may expose the local data of some clients~\cite{melis2019sp}. 
% A large body of work, instead, investigates attacks that usually aim to detriment the predictive accuracy of the learned global model. For instance, \emph{data poisoning} attacks achieve this goal by letting an adversary pollute the training set of some corrupt FL clients with maliciously crafted examples~\cite{jagielski2018sp}.
% Similarly, in \emph{model poisoning} the attacker attempts to tweak the global model weights~\cite{bhagoji2019pmlr} by directly perturbing the local model's weights of some infected FL clients before these are sent to the central server for aggregation, usually via so-called Byzantine attacks. 
% It turns out that Byzantine model poisoning attacks severely impact standard FedAvg; therefore, more robust aggregation functions must be designed to make FL systems secure.
Here, we focus on \emph{untargeted model poisoning} attacks~\cite{bhagoji2019pmlr}, where an adversary attempts to tweak the global model weights %\footnote{We will use the terms \textit{parameters} and \textit{weights} interchangeably.} 
by directly perturbing the local model's parameters of some infected clients before these are sent to the central server for aggregation.
In doing so, the adversary aims to jeopardize the global model \textit{indiscriminately} at inference time.
Such model poisoning attacks severely impact standard FedAvg; therefore, more robust aggregation functions must be designed to secure FL systems.
\\
% In this paper, we focus on designing a novel robust aggregation scheme at the server's end to contrast the effect of Byzantine model poisoning attacks.
%
% Current countermeasures and their limitations
%Several countermeasures have been proposed in the literature to combat model poisoning attacks on FL systems.
% Some methods use simple statistics more robust than plain average to smooth the impact of malicious updates (e.g., Trimmed Mean and FedMedian~\cite{yin2018icml}). 
% Other defenses implement outlier detection techniques to discard malicious updates from the aggregation performed at the server's end. Those are either based on heuristics (e.g., Krum/Multi-Krum~\cite{blanchard2017nips} and Bulyan~\cite{mhamdi2018pmlr}) or data-driven approaches (e.g., K-means clustering~\cite{shen2016acm} or DnC via spectral analysis~\cite{shejwalkar2021ndss}). 
% Finally, some strategies rely on a centralized ``source of trust'' to spot potential malicious updates (e.g., FLTrust~\cite{cao2020fltrust}).
% Several countermeasures have been proposed in the literature to combat model poisoning attacks on FL systems, i.e., to discard possible malicious local updates from the aggregation performed at the server's end. 
% These techniques range from simple statistics more robust than plain average (e.g., Trimmed Mean and FedMedian~\cite{yin2018icml}) to outlier detection heuristics (e.g., Krum/Multi-Krum~\cite{blanchard2017nips} and Bulyan~\cite{mhamdi2018pmlr}) or data-driven approaches (e.g., spectral analysis via K-means clustering~\cite{shen2016acm} or spectral analysis), or methods based on ``source of trust'' (e.g., FLTrust~\cite{cao2020fltrust}).
% OLD, LONG VERSION
%Several countermeasures have been proposed in the literature to combat Byzantine model poisoning attacks on FL systems.
% Descriptive statistics
% For example, Trimmed Mean and FedMedian aggregate local model updates using more robust statistics than standard average~\cite{yin2018icml}.
%
% % Heuristics for outlier detection
% Many existing Byzantine-resilient strategies implement some outlier detection heuristics to discard the model updates sent by potentially malicious clients from the input of the aggregation function.
% One of the most popular heuristics is Krum~\cite{blanchard2017nips}.
% This strategy tries to mitigate the impact of Byzantine attacks by selecting as a global model the local model with the smallest sum of Euclidean distances to {\em all} the other local models.
% Although powerful, Krum requires the server to know (or, at least, estimate) the number of malicious FL clients upfront, which is generally impossible in a realistic attack scenario. %
% Moreover, Krum may become ineffective for complex, high-dimensional model parameter spaces due to the curse of dimensionality.
% Bulyan~\cite{mhamdi2018pmlr} tries to overcome this issue by combining Krum with a variant of Trimmed Mean.
% % Data-driven outlier detection
% Other strategies use data-driven outlier detection techniques -- e.g., via K-means clustering~\cite{shen2016acm} -- to spot potential malicious local model updates. 
% %For instance, Shen et al. propose to cluster local model updates with K-means and thus identify outliers.
%
% % Other techniques
% As far as the server is concerned, any local model received can be from a potential malicious client. 
% FLTrust~\cite{cao2020fltrust} assumes the server acts as a client, i.e., trains a local model on an additional {\em trustworthy} dataset at the server's end and compares it against all the local models from other clients. 
% This way, the server can rely on some ``source of trust'' when discarding potentially malicious clients.
%\\
% Limitations of existing Byzantine-resilient strategies
Unfortunately, existing defense mechanisms either rely on simple heuristics (e.g., Trimmed Mean and FedMedian by~\cite{yin2018icml}) or need strong and unrealistic assumptions to work effectively (e.g., foreknowledge or estimation of the number of malicious clients in the FL system, as for Krum/Multi-Krum~\cite{blanchard2017nips} and Bulyan~\cite{mhamdi2018pmlr}, which, however, cannot exceed a fixed threshold).
Furthermore, outlier detection methods using K-means clustering~\cite{shen2016acm} or spectral analysis like DnC~\cite{shejwalkar2021ndss} do not directly consider the temporal evolution of local model updates received.
Finally, strategies like FLTrust~\cite{cao2020fltrust} require the server to collect its own dataset and act as a proper client, thereby altering the standard FL protocol.
\\
% OLD, LONG VERSION
% Overall, existing Byzantine-resilient strategies are either simple heuristics (e.g., FedMedian) or, if they are more complex, they rely on strong and unrealistic assumptions to work effectively (e.g., knowing the number of malicious clients in the FL system in advance, as for Krum and alike).
% Furthermore, data-driven outlier detection methods do not consider the temporary evolution of local model updates received (e.g., K-means clustering). 
% Finally, strategies like FLTrust requires the server to collect its own dataset and act as a proper client, thereby altering the standard FL protocol.
%
% Description of the proposed method
This work introduces a novel pre-aggregation \textit{filter} robust to untargeted model poisoning attacks. Notably, this filter $(i)$ operates without requiring prior knowledge or constraints on the number of malicious clients and $(ii)$ inherently integrates temporal dependencies. 
The FL server can employ this filter as a preprocessing step before applying \textit{any} aggregation function, be it standard like FedAvg or robust like Krum or Bulyan.
Specifically, we formulate the problem of identifying corrupted updates as a multidimensional (i.e., matrix-valued) time series anomaly detection task. 
The key idea is that legitimate local updates, resulting from well-calibrated iterative procedures like stochastic gradient descent (SGD) with an appropriate learning rate, show \textit{higher predictability} compared to malicious updates. This hypothesis stems from the fact that the sequence of gradients (thus, model parameters) observed during legitimate training exhibit regular patterns, as validated in Section~\ref{subsec:intuition}. %until convergence. 
%This regularity may be more pronounced for smooth convex loss functions, but it can still be captured within an appropriate time window, even for more complex and convoluted loss surfaces. 
%We provide evidence of this claim in Appendix~B, where we show that the average mutual information (i.e., ``predictability''), calculated over pairs of legitimate model updates sent at different FL rounds, is significantly higher than the corresponding computation for a malicious client.
\\
Inspired by the matrix autoregressive (MAR) framework for multidimensional time series forecasting~\cite{chen2021je}, we propose the FLANDERS ({\em \textbf{F}ederated \textbf{L}earning meets \textbf{AN}omaly \textbf{DE}tection for a \textbf{R}obust and \textbf{S}ecure}) filter.
The main advantages of FLANDERS over existing strategies like FLDetector~\cite{zhao2020multivariate} are its resilience to large-scale attacks, where $50\%$ or more FL participants are hostile, and the capability of working under realistic non-iid scenarios.
We attribute such a capability to two key factors: $(i)$ FLANDERS works without knowing a priori the ratio of corrupted clients, and $(ii)$ it embodies temporal dependencies between intra- and inter-client updates, quickly recognizing local model drifts caused by evil players. Below, we summarize our main contributions:

\begin{itemize}
\item[{\em(i)}]
We provide empirical evidence that the sequence of models sent by legitimate clients is more predictable than those of malicious participants performing untargeted model poisoning attacks.
\\
\item[{\em(ii)}] 
We introduce FLANDERS, the first pre-aggregation filter for FL robust to untargeted model poisoning based on multidimensional time series anomaly detection.
\\
\item[{\em(iii)}] 
We integrate FLANDERS into Flower,\footnote{\scriptsize{\url{https://flower.dev/}}} a popular FL simulation framework for reproducibility.
\\
\item[{\em(iv)}] 
We show that FLANDERS improves the robustness of the existing aggregation methods under multiple settings: different datasets, client's data distribution (non-iid), models, and attack scenarios.
\\
\item[{\em(v)}] 
We publicly release all the implementation code of FLANDERS along with our experiments.\footnote{\scriptsize{\url{https://anonymous.4open.science/r/flanders_exp-7EEB}}}
\end{itemize}

% Paper's structure and organization
The remainder of the paper is structured as follows. %some related work and the current state-of-the-art solutions to security issues that FL entails. 
Section~\ref{sec:background} covers background and preliminaries. 
In Section~\ref{sec:related}, we discuss related work.
Section~\ref{sec:problem} and Section~\ref{sec:method} describe the problem formulation and the method proposed. % to tackle it. 
Section~\ref{sec:experiments} gathers experimental results. %, and Section~\ref{sec:limitations} discusses some limitations of this work.
Finally, we conclude in Section~\ref{sec:conclusion}.
 %discusses the limitations of this work and draws future research directions.
%reports conclusions and draws perspectives for future research directions.

%%%%%%% OLD %%%%%%%
%to overcome the resilience of Byzantine failures in distributed Stochastic Gradient Descent computations. 
% The strength of Krum is its time complexity, which is linear in the gradient dimension. 
% However, the robustness of the approach is guaranteed for gradient-based learning applications only when the majority of the clients are not compromised. 
% Besides, the aggregation mechanism of Krum, as well as that of similar methods, is robust from a coarse-grained perspective and does not provide solutions to errors and perturbations that may occur at inference time.
%A related approach to~\cite{blanchard2017nips} is the work of Su et al.~\cite{su2016dc}. Here, the authors propose an iterated approximate agreement to tackle a multi-layer scenario attacked by Byzantine agents. 
%However, the method works efficiently on the sole discrete context and it is inapplicable to continuous state environments.
%\gabri{Maybe, we should just talk about the main limitations of existing countermeasures without digging into their details (or, we can just mention Krum as this is the most popular one). I will move the description of all these methods to the Related Work section.}
\section{Model}
In this paper, we study symmetric TSP on 2D plane. Given $n$ cities and the  coordinates $(x_i,y_i) \in \mathbb{R}^2$ of these cities, our goal is to find the shortest possible route that visits each city exactly once and returns to the origin city, where $i\in \{1,2,3,...,n\}$ is the index of the city.
\subsection{Graph Neural Network}
Given a TSP instance, let $\mathbf{D}_{i,j}$ denote the Euclidean distance between city $i$ and city $j$. $\mathbf{D} \in \mathbb{R}^{n \times n}$ is the distance matrix. We first build adjacency matrix $\mathbf{W} \in \mathbb{R}^{n \times n}$ with $\mathbf{W}_{i,j} = e^{-\mathbf{D}_{i,j}/\tau}$ and node feature  $\mathbf{F} \in \mathbb{R}^{n \times 2}$ based on the input coordinates, where $\mathbf{F}_{i} = (x_i,y_i)$ and $\tau$ is the temperature.
The node feature matrix $\mathbf{F}$ and the weight matrix $\mathbf{W}$ are then fed into a GNN to generate a transition matrix $\mathbb{T} \in \mathbb{R}^{n\times n}$. 

In our model, we use Scattering Attention GNN (SAG), SAG has both low-pass and band-pass filters and can build adaptive representations by implicitly learning node-wise weights for combining multiple different channels in the network using attention-based architecture.
Recent studies show that SAG can output expressive representations for graph combinatorial problems such as maximum clique while remaining lightweight~\cite{min2022can}. 

Let $\mathcal{S} \in \mathbb{R}^{n\times n}$  denote the output of SAG,  we first apply a column-wise Softmax activation to 
 the GNN's output and we can summarize
this operation in matrix notation as $\mathbb{T}_{i,j} = {e^{\mathcal{S}_{i,j}}}/{\sum_{k=1}^n e^{\mathcal{S}_{k,j}}}$. This ensures that each element in $\mathbb{T}$ is greater than zero and the summation of each column is 1. 
We then use $\mathbb{T}$ to build a heat map $\mathcal{H}$, where $\mathcal{H} \in   \mathbb{R}^{n\times n}$. 
In our model, we use $\mathcal{H}$  to estimate the probability of each edge
 belonging to the optimal solution and use
 $\mathbb{T}$ to build a surrogate loss of the Hamiltonian Cycle constraint. 
 As illustrated in Figure~\ref{fig:Transition},  our approach aims to generate an expressive transition matrix $\mathbb{T}$ which assigns large weights (close to 1) on the transition elements and small weights (close to 0) on others.
This will allow us to build a non-smooth heat map $\mathcal{H}$ and improve the performance of the local search.
\begin{figure}[htb]
\vskip 0.2in
\begin{center}
\centerline{\includegraphics[width=\columnwidth]{Figures/diagramtransition.png}}
\caption{We use a SAG to generate a non-smooth transition matrix $\mathbb{T}$. The SAG model is a function of the coordinates and the weighted adjacency matrix.}
\label{fig:Transition}
\end{center}
\vskip -0.2in
\end{figure}
\subsection{Building the  Heat Map using the Transition Matrix}
We build the heat map $\mathcal{H}$ based on $\mathbb{T}$. As mentioned, $\mathcal{H}_{i,j}$ is the probability for edge ($i$,$j$) to belong to the optimal TSP solution or not. We define $\mathcal{H}$ as:
\begin{equation}\label{eq:Hij}
\mathcal{H} = \mathbb{T} \mathbb{V} \mathbb{T}^T,
\end{equation}
where $$\mathbb{V}  = 
\begin{pmatrix}
0 & 1 & 0 & 0 & \cdots & 0  & 0 & 0 \\
0 & 0 & 1 & 0 & \cdots & 0  & 0 & 0\\
0 & 0 & 0 & 1 & \cdots & 0  & 0 & 0\\

\vdots  & \vdots  & \vdots & \ddots  & \ddots & \vdots & \vdots  & \vdots  \\
0 & 0 & 0 & 0 & \ddots & 1 & 0 & 0\\
0 & 0 & 0 & 0 & \cdots & 0 & 1 & 0\\
0 & 0 & 0 & 0 & \cdots & 0 & 0 & 1\\
1 & 0 & 0 & 0 & \cdots & 0 & 0 & 0
\end{pmatrix}$$ is the Sylvester shift matrix~\cite{sylvester1909collected}, $\mathbb{V} \in \mathbb{R}^{n\times n}$. We can interpret $\mathbb{V}$ as a cyclic permutation operator that performs a circular shift. The elements in $\mathcal{H}$ can be written as:
$
\mathcal{H}_{i,j} = \sum_{k=1}^n \mathbb{T}_{i,k} \mathbb{T}_{j,k+1 (\mathrm{mod}\; n)}.
$
$\mathcal{H}_{i,j}$ is the sum of element-wise multiplication on two terms:
the $i$-th row of the $\mathbb{T}$ and the $j$-th row of the $\mathbb{T}$ with left translation. For example, given the transition matrix in Figure~\ref{fig:Transition}, Figure~\ref{fig:TSP} illustrates the corresponding heat map and the TSP solution\footnote{We build $\mathcal{H}$ using Equation~\ref{eq:Hij} instead of directly using $\mathbb{T}$ because, under binary assumption, the output of SAG does not satisfy the Hamiltonian Cycle constraint. An example is shown in Figure~\ref{fig:Transition}. }.
\begin{figure}[htb]
\vskip 0.2in
\begin{center}
\centerline{\includegraphics[width=0.9\columnwidth]{Figures/diagramTSP.png}}
\caption{Left: the heat map corresponds to the transition matrix in Figure~\ref{fig:Transition}; right: the corresponding TSP path.}
\label{fig:TSP}
\end{center}
\vskip -0.2in
\end{figure}

The first row in $\mathcal{H}$ is the probability distribution of directed edges start from city $1$, and since the third element is the only non-zero one in the first row, we then add directional edge $1 \rightarrow 3 $ to our TSP solution. Similarly, the first column in $\mathcal{H}$ can be regarded as the probability distribution of directed edges which end in city $1$.  Ideally, given a graph $\mathcal{G}$ with $n$ nodes, we want to build a transition matrix where  each row and column are assigned with one value 1 (True) and $n-1$ values 0 (False), so that the heat map will only contain one valid solution. 
In practice, we will build a transition matrix $\mathbb{T}$ whose heat map $\mathcal{H}$ assigns large probabilities to the edges in the TSP solution and small probabilities to the other edges.
\begin{figure}[h]
\vskip 0.2in
\begin{center}
\centerline{\includegraphics[width=0.8\columnwidth]{Figures/HamiltonianCycle.png}}
\caption{Illustration of builing a cycle from transition matrix $\mathbb{T}$. Here $q_{k-1} = 3$, $q_k = n -2$, $q_{k+1} = 2$, $q_n = n$ and $q_1 = 4$, this means $\mathcal{H}$ contains the following four directed edges: $3 \rightarrow n -2$, $n-2 \rightarrow 2$, $1 \rightarrow n$ and $n \rightarrow 4$.  }
\label{fig:HamiltonianCycle}
\end{center}
\vskip -0.2in
\end{figure}
\begin{lemma}\label{lem:bijection}
Let $q_i$ denote the row index of the non-zero element in $i$-th column in $\mathbb{T}$, $\mathbb{T}_{q_i,i}=1$, $q_i \in \{1,2,3,4,...,n\}$.  When each row and column in  $\mathbb{T} $ have one value 1 (True) and $n-1$ value 0 (False), then $q_i = q_j$ if and only if $i = j$.
\end{lemma}
\begin{proof}
    If there exist a $(i,j)$ pair where $q_i = q_j$ when $i \neq j$, then $\mathbb{T}_{q_i,i} = 1$ and $\mathbb{T}_{q_j,j} = \mathbb{T}_{q_i,j} = 1$. This means $q_i$-th row has two non-zero elements, which leads to a contradiction.
\end{proof}
\begin{lemma}\label{lem:binary}
Consider graph $\mathcal{G}$ with $n$ nodes, for any $\mathbb{T} \in \mathbb{R}^{n \times n}$ with $\mathbb{T}_{i,j} \geq 0$, when each row and column in  $\mathbb{T} $ have one value 1 (True) and $n-1$ value 0 (False). Then, each row and column in $\mathcal{H} $ also have one value $1$ (True) and $n-1$ value $0$ (False), which means each city corresponds to only one beginning point and one ending point.
\end{lemma}

\begin{proof}
First, it is clear that for $\forall a,b$, $\mathcal{H}_{a,b} \in  \mathbb{Z}_{\geq 0}$. Let's assume $\mathbb{T}_{a,l} = \mathbb{T}_{b,m} =  1 (a \neq b, l \neq m)$, since
$
\mathcal{H}_{i,j} = \sum_{k=1}^n \mathbb{T}_{i,k} \mathbb{T}_{j,k+1 (\mathrm{mod}\; n)},
$
we then have
\begin{align*}
\sum_{j=1}^n \mathcal{H}_{a,j}  & = \sum_{j=1}^n  \sum_{k=1}^n \mathbb{T}_{a,k} \mathbb{T}_{j,k+1 (\mathrm{mod}\; n)}  \\ & =
\sum_{k=1}^n  \mathbb{T}_{a,k} \{ \sum_{j=1}^n \mathbb{T}_{j,k+1 (\mathrm{mod}\; n)}   \} \\ 
& =  \sum_{k=1}^n  \mathbb{T}_{a,k}   = 1.
\end{align*}
This implies that the summation of each row in $\mathcal{H}$ is 1. Similarly,
\begin{align*}
\sum_{i=1}^n \mathcal{H}_{i,b} & =   \sum_{i=1}^n \sum_{k=1}^n   \mathbb{T}_{i,k} \mathbb{T}_{b,k+1 (\mathrm{mod}\; n)}    \\ 
& = \sum_{k=1}^n    \mathbb{T}_{b,k+1 (\mathrm{mod}\; n)} \{\sum_{i=1}^n \mathbb{T}_{i,k}\}   \\ 
& = \sum_{k=1}^n    \mathbb{T}_{b,k+1 (\mathrm{mod}\; n)}  = 1.
\end{align*}
This suggests the summation of each column in $\mathcal{H}$ is $1$. Since each element in $\mathcal{H}_{i,j} \in  \mathbb{Z}_{\geq 0}$, we can then conclude that each row and column in  $\mathcal{H} $ have one value $1$ (True) and $n-1$ value $0$ (False). 
Also,
$\mathcal{H}_{ii} = \sum_{k=1}^n  \sum_{i=1}^n   \mathbb{T}_{i,k+1 (\mathrm{mod}\; n)} \mathbb{T}_{i,k}  = 0$, this indicates that the elements in the main diagonal of $\mathcal{H}$ are 0, which implies no self-loops.
As mentioned, the $i$-th row in $\mathcal{H}$ is the probability distribution of directed edges start from city $i$, and the $j$-th column is the probability distribution of directed edges end in city $j$. Because each row and column in  $\mathcal{H} $ have one value 1 (True) element and $\mathcal{H}$'s diagonal entries are all zero, this means that each city is the beginning point of one 
 directed edge and is also the ending point of another different directed edge.
 \end{proof}
\begin{lemma}\label{lem:HamiltonianCycle}
There is at least one cycle in $\mathcal{H}$ which contains $n$ edges and visits all cities when each row and column in  $\mathbb{T} $ have one value 1 (True) and $n-1$ value 0 (False). 
\end{lemma}

\begin{proof}
\textbf{Lemma}~\ref{lem:bijection} indicates that $q_i \neq q_j$ when $i \neq j$ and $q_i \in \{1,2,3,4,...,n\}$, then $\cup_{i=1}^n q_i = \{1,2,3,4,...,n\}$. From Equation~\ref{eq:Hij}, we have
\begin{align*}
\mathcal{H}_{q_i,q_{i+1}} & = \sum_{k=1}^n \mathbb{T}_{q_i,k} \mathbb{T}_{q_{i+1},k+1 (\mathrm{mod}\; n)} \\
& \geq \mathbb{T}_{q_i,i} \mathbb{T}_{q_{i+1},i+1 (\mathrm{mod}\; n)} \\
& \geq 1.
\end{align*}
Using \textbf{Lemma}~\ref{lem:binary}, since each row and column in $\mathcal{H}$ have one value $1$ (True) and $n-1$ value $0$ (False),  it suffices to show that $1 \geq \mathcal{H}_{q_i,q_{i+1}} \geq 1$, therefore $\mathcal{H}_{q_i,q_{i+1}} = 1$. This suggests that there is a directed edge from city $q_i$ to $q_{i+1}$. We can then construct a cycle $\mathcal{C}$ from $\mathcal{H}$,  we can write $\mathcal{C}$ as
\begin{align*}
    q_1 \rightarrow q_2 \rightarrow q_3 \rightarrow q_4 \rightarrow ... \rightarrow q_n \rightarrow q_1,
\end{align*}
where $\rightarrow$ is a directed edge. Since $\cup_{i=1}^n q_i = \{1,2,3,4,...,n\}$, cycle $\mathcal{C}$ visits all $n$ cities and have $n$ edges. One example of how to build $\mathcal{H}_{q_i,q_{i+1}}$ from $\mathbb{T}$ is shown in Figure~\ref{fig:HamiltonianCycle}.
\end{proof}
\begin{corollary}\label{corly:1}
$\mathcal{H}$ represents one Hamiltonian Cycle when each row and column in  $\mathbb{T} $ have one value 1 (True) and $n-1$ value 0 (False). 
\end{corollary}
\begin{proof}
    From \textbf{Lemma}~\ref{lem:HamiltonianCycle}, cycle $\mathcal{C}$ contains $n$
edges and visits all cities, if there exists another edge $(i,j)$ which does not belong to $\mathcal{C}$, then city $i$ is the starting point of at least two edges and  city $j$ is the ending point of at least two edges. This results in a contradiction with \textbf{Lemma}~\ref{lem:binary}. Thus, it suffices to conclude that $\mathcal{C}$  visits each city exactly once and  $\mathcal{H}$ only contains the edges in $\mathcal{C}$. This implies that $\mathcal{H}$ represents one Hamiltonian Cycle.
\end{proof}
 \subsection{Unsupervised Loss}
In order to generate such an expressive transition matrix $\mathbb{T}$, we minimize the following objective function:
\begin{equation} \label{eq:loss}
\begin{aligned}
    \mathcal{L} =  & 
 \lambda_1 \underbrace{\sum_{i=1}^n (\sum_{j=1}^n \mathbb{T}_{i,j} - 1)^2}_{\text{Row-wise constraint}}  +  \lambda_2   \underbrace{\sum_{i}^n \mathcal{H}_{i,i}}_{\text{No self-loops}} \\
     & + \underbrace{\sum_{i=1}^n \sum_{j=1}^n \mathbf{D}_{i,j} \mathcal{H}_{i,j}}_{\text{Minimize the distance }}. 
\end{aligned}
\end{equation}

The first term in $\mathcal{L}$ encourages the summation of each row in $\mathbb{T}$ to be close to 1. As mentioned, we normalize each column of $\mathbb{T}$ using  Softmax activation. So when the first term is minimized to zero, each row and column in $\mathbb{T}$ are normalized. The second term penalizes the weight on the main diagonal of $\mathcal{H}$, this discourages self-loops in  TSP solutions. The third term can be regarded as the expectation TSP length of the heat map $\mathcal{H}$, where $\mathbf{D}_{i,j}$ is the distance between city $i$ and $j$. As mentioned, since $\mathcal{H}$ corresponds to one Hamiltonian Cycle given an ideal transition matrix with one value $1$ (True)
and $n-1$ value $0$ (False) in each row and column. Then the minimum value of $\sum_{i=1}^n \sum_{j=1}^n \mathbf{D}_{i,j} \mathcal{H}_{i,j}$ is the shortest Hamiltonian Cycle on the graph, which corresponds to the optimal solution of TSP. 

Given a heat map $\mathcal{H}$, we consider  $M$ largest elements in each row (without diagonal elements) and set other $n-M$ elements as 0. Let $\Tilde{H}$ denote the new heat map, we then symmetrize the new heat map by $\mathcal{H}' = \Tilde{H} + \Tilde{H}^T$. 
Let $\mathbf{E}_{ij} \in \{0,1\}$ denote whether an undirected edge ($i,j$)  is in our prediction or not. Without loss of generality, we can assume $0<i<j\leq n$ and  define $\mathbf{E}_{ij}$ as :
$$
\mathbf{E}_{ij} = 
    \begin{cases}
      1, & \text{if}\ \mathcal{H}'_{ij} = \mathcal{H}'_{ji} > 0 \\
      0, & \text{otherwise}
    \end{cases}.
$$
Let $\Pi$ denote the set of undirected edges $(i,j)$ with $\mathbf{E}_{ij} = 1$. Ideally, we would build a prediction edge set $\Pi$ with a small $M$ value, and $\Pi$ can cover all the ground truth edges so that we are able to reduce search space size from $n(n-1)/2$ to  $|\Pi|$. In practice, generating small-size edge sets $\Pi$ which always cover the ground truth solutions is very difficult. We aim to let  $\Pi$ cover as many ground truth edges as possible and use $\mathcal{H}'$ to guide the local search process. 

\input{Sections/MCTS}
\section{Experiments}
\subsection{Dataset}
Our dataset contains 2,000 samples for training and 1,000 samples for validation. We use the same test dataset in~\cite{fu2021generalize}. The test dataset contains $10,000$ 2D-Euclidean TSP instances for $n = 20,50,100$, 128 instances for $n = 200,500,1,000$. We train our models on TSP instances with 20, 50, 100, 200, 500, and 1,000 vertices. We then build the corresponding heat maps based on these trained models.  
% \onecolumn
\begin{table*}[htb]
\footnotesize
\centering
\caption{Results of SAG + Local Search  w.r.t. existing baselines, tested on 10,000 instances with $n$ = 20, 50 and 100.}
%\Yiwei{[Yiwei: I keep the two results here only for comparison, I think finally we only keep the best one. I did not tune too much parameters (2 - 4 sets for each TSP variant). I would like to tune and search a longer one for TSP10000.]}
\vskip 0.15in
\label{table:Exps01}
    \begin{tabular}{lllllllllll}
    \toprule
    %   \toprule[2pt]
       \multirow{2}{*}{Method} & \multirow{2}{*}{Type}  & \multicolumn{3}{c}{TSP20} & \multicolumn{3}{c}{TSP50} & \multicolumn{3}{c}{TSP100}\\
         &  & Length & Gap (\%)  & Time & Length & Gap (\%) & Time & Length & Gap (\%)  & Time \\
        \hline
        Concorde  & Solver & {3.8303} & {0.0000 } & {2.31m} & {5.6906} & {0.0000 } & {13.68m} & {7.7609} & {0.0000 } & {1.04h}  \\
        Gurobi  & Solver & {3.8302} & {-0.0001 } & {2.33m} & {5.6905} & {0.0000 } & {26.20m} & {7.7609} & {0.0000 } & {3.57h}  \\
        LKH3  & Heuristic & {3.8303} & {0.0000 } & {20.96m} & {5.6906} & {0.0008 } & {26.65m} & {7.7611} & {0.0026 } & {49.96m}  \\
        % \hline
         GAT \citep{deudon2018learning}  & RL, S &  {3.8741} &  {1.1443 } &   {10.30m} &  {6.1085} &  {7.3438 } &   {19.52m} &  {8.8372} &  {13.8679 } &   {47.78m}  \\

        GAT \citep{deudon2018learning}  & \tabincell{c}{RL, S\\ 2-OPT} &  {3.8501} &  {0.5178 } &  {15.62m} & {5.8941} & {3.5759 } & {27.81m} & {8.2449} & {6.2365 } & {4.95h}  \\

        GAT \citep{kool2018attention}  & RL, S &  {3.8322} &  {0.0501 } &   {16.47m}  & {5.7185} & {0.4912 } &  {22.85m} & {7.9735} & {2.7391 } & {1.23h}  \\

        GAT \citep{kool2018attention}  & RL, G & {3.8413} & {0.2867 } & {6.03s} & {5.7849} & {1.6568 } & {34.92s} & {8.1008} & {4.3791 } & {1.83m} \\

        GAT \citep{kool2018attention}  & RL, BS &  {3.8304} &  {0.0022 } &  {15.01m} & {5.7070} & {0.2892 } & {25.58m} & {7.9536} & {2.4829 } & {1.68h}  \\

        GCN \citep{joshi2019efficient} & SL, G & {3.8552} & {0.6509 } &  {{19.41s}} & {5.8932} & {3.5608 } &  {2.00m} & {8.4128} & {8.3995 } &  {11.08m}  \\

        GCN \citep{joshi2019efficient} & SL, BS &  {3.8347} &  {0.1158 } &   {21.35m} &  {5.7071} &  {0.2905 } &   {35.13m} &  {7.8763} &  {1.4828 } &   {31.80m}  \\

        GCN \citep{joshi2019efficient} & SL, BS* & {3.8305} & {0.0075 } & {22.18m} & {5.6920} & {0.0251 } & {37.56m} & {7.8719} & {1.4299 } & {1.20h}  \\

        \hline
        \multirow{2}{*} {Att-GCRN\citep{fu2021generalize}} & \multirow{2}{*}{\tabincell{c}{SL+RL \\MCTS}} & \multirow{2}{*}{{3.8300}} & \multirow{2}{*}{\textbf{-0.0074}} & {23.33s} +  & \multirow{2}{*}{5.6908} & \multirow{2}{*}{{0.0032 }} & {2.59m} +  & \multirow{2}{*}{{7.7616}} & \multirow{2}{*}{{0.0096 }} & {3.94m} +   \\
        & & & & 1.05m & & & 2.63m & & & 5.25m\\ \hline

        % \multirow{2}{*} {\tabincell{c}{Random Init ($T = 0.01$)}} & \multirow{2}{*}{MCTS} & \multirow{2}{*}{{NA}} & \multirow{2}{*}{{NA}} & {NA} +  & \multirow{2}{*}{NA} & \multirow{2}{*}{{NA}} & {NA} +  & \multirow{2}{*}{{7.8752}} & \multirow{2}{*}{1.473 } & {0+}   \\
        % & & & & NA & & & NA & & & 6m\\     \hline
        %  \multirow{2}{*} {\tabincell{c}{Random Init ($T = 0.02$)}} & \multirow{2}{*}{MCTS} & \multirow{2}{*}{{NA}} & \multirow{2}{*}{{NA}} & {NA} +  & \multirow{2}{*}{NA} & \multirow{2}{*}{{NA}} & {NA} +  & \multirow{2}{*}{{7.8483}} & \multirow{2}{*}{1.126 } & {0+}   \\
        % & & & & NA & & & NA & & & 12m\\     \hline
        %   \multirow{2}{*} {\tabincell{c}{Uniform Init ($T = 0.01$)}} & \multirow{2}{*}{MCTS} & \multirow{2}{*}{{NA}} & \multirow{2}{*}{{NA}} & {NA} +  & \multirow{2}{*}{NA} & \multirow{2}{*}{{NA}} & {NA} +  & \multirow{2}{*}{{7.8332}} & \multirow{2}{*}{0.932 } & {0+}   \\
        % & & & & NA & & & NA & & & 6m\\     \hline     
        %   \multirow{2}{*} {\tabincell{c}{Uniform Init ($T = 0.02$)}} & \multirow{2}{*}{MCTS} & \multirow{2}{*}{{NA}} & \multirow{2}{*}{{NA}} & {NA} +  & \multirow{2}{*}{NA} & \multirow{2}{*}{{NA}} & {NA} +  & \multirow{2}{*}{{7.8084}} & \multirow{2}{*}{0.612 } & {0+}   \\
        % & & & & NA & & & NA & & & 12m\\     \hline
        %\usepackage{supertabular}??
        \multirow{2}{*} {{\bf{\small{UTSP} (ours)}} } & \multirow{2}{*}{UL, Search} & \multirow{2}{*}{{3.8303}} & \multirow{2}{*}{-0.0009} & {38.23s} +  & \multirow{2}{*}{5.6894} & \multirow{2}{*}{\textbf{-0.0200}} & {1.34m} +  & \multirow{2}{*}{{7.7608}} & \multirow{2}{*}{\textbf{-0.0011 }} & {5.68m+}   \\
        & & & & 1.04m & & & 2.60m & & & 5.21m\\ 
        % \multirow{1}{*} {\textbf{Ours (Restart)} ($T = 0.01$s)} & \multirow{1}{*}{UL, Search} & \multirow{1}{*}{{N/A}} & \multirow{1}{*}{\textbf{N/A}} & {38.23s} +  & \multirow{1}{*}{N/A} & \multirow{1}{*}{\textbf{-0.02}} & {1.34m} +  & \multirow{1}{*}{{7.760}} & \multirow{1}{*}{\textbf{-0.011 }} & {5.68m+}   \\
        % & & & & 1.04m & & & 2.60m & & & 5.21m\\ 
\bottomrule
        % \bottomrule[2pt]
    \end{tabular}
\end{table*}
% \twocolumn
\subsection{Results}
\begin{figure}[!htb]
\vskip 0.2in
\begin{center}
\centerline{\includegraphics[width=\columnwidth]{Figures/TrainingLoss.png}}
\caption{TSP $100$ training curve using  unsupervised learning surrogate loss. We compare two GNN models: GCN~\cite{kipf2016semi} 
 and SAG~\cite{min2022can}, where GCN is a low-pass model and SAG is a low-pass + band-pass model.}
\label{fig:trainingloss}
\end{center}
\vskip -0.2in
\end{figure}
% \onecolumn
\begin{table*}[t]
\footnotesize
\centering
\caption{Results of SAG + Local Search w.r.t. existing baselines, tested on 128 instances with $n$ = 200, 500 and 1000.}
\vskip 0.15in
\label{table:Exps02}
    \begin{tabular}{lllllllllll}
    \toprule
    %   \toprule[2pt]
       \multirow{2}{*}{Method} & \multirow{2}{*}{Type}  & \multicolumn{3}{c}{TSP200} & \multicolumn{3}{c}{TSP500} & \multicolumn{3}{c}{TSP1000}\\
         &  & Length & Gap (\%)  & Time & Length & Gap (\%)  & Time & Length & Gap (\%) & Time \\
        \hline
                Concorde &Solver & {10.7191} & {0.0000 } & {3.44m} & {16.5458} & {0.0000 } & {37.66m} & {23.1182} & {0.0000 } & {6.65h} \\
        Gurobi &Solver & {10.7036} & {-0.1446 } & {40.49m} & {16.5171} & {-0.1733 } & {45.63h} & {-} & {-} & {-}  \\
        LKH3 & Heuristic & {10.7195} & {0.0040 } & {2.01m} & {16.5463} & {0.0029 } & {11.41m} & {23.1190} & {0.0036 } & {38.09m}  \\
        % \hline
        GAT \citep{deudon2018learning} & RL, S & {13.1746} & {22.9079 } &  {4.84m} & {28.6291} & {73.0293 } &  {20.18m} & {50.3018} & {117.5860 } & {37.07m}  \\

        GAT \citep{deudon2018learning} & \tabincell{c}{RL, S\\ 2-OPT} & {11.6104} & {8.3159 } & {9.59m} & {23.7546} & {43.5687 } & {57.76m} & {47.7291} & {106.4575 } & {5.39h}  \\

        GAT \citep{kool2018attention}  & RL, S & {11.4497} & {6.8160 } &  {4.49m} & {22.6409} & {36.8382 } & {15.64m} & {42.8036} & {85.1519 } &  {63.97m}  \\

        GAT \citep{kool2018attention}  & RL, G & {11.6096} & {8.3081 } & {5.03s} & {20.0188} & {20.9902 } & {1.51m} & {31.1526} & {34.7539 } & {3.18m}  \\

        GAT \citep{kool2018attention}  & RL, BS & {11.3769} & {6.1364 } & {5.77m} & {19.5283} & {18.0257 } & {21.99m} & {29.9048} & {29.2359 } & {1.64h}  \\

        GCN \citep{joshi2019efficient} & SL, G & {17.0141} & {58.7272 } &  {59.11s} & {29.7173} & {79.6063 } & {6.67m} & {48.6151} & {110.2900 } &  {28.52m}  \\

        GCN \citep{joshi2019efficient} & SL, BS & {16.1878} & {51.0185 } &  {4.63m} & {30.3702} & {83.5523 } &  {38.02m} & {51.2593} & {121.7278 } &  {51.67m}  \\

        GCN \citep{joshi2019efficient} & SL, BS* & {16.2081} & {51.2079 } & {3.97m} & {30.4258} & {83.8883 } & {30.62m} & {51.0992} & {121.0357 } & {3.23h}  \\

        \hline
        \multirow{2}{*} {Att-GCRN\citep{fu2021generalize}} & \multirow{2}{*}{\tabincell{c}{SL+RL \\MCTS}} & \multirow{2}{*}{{10.7358}} & \multirow{2}{*}{{0.1563 }} & {20.62s} +  & \multirow{2}{*}{16.7471} & \multirow{2}{*}{{1.2169 }} & {31.17s} +  & \multirow{2}{*}{{23.5153}} & \multirow{2}{*}{{1.7179 }} & {43.94s} +   \\
        & & & & 1.33m & & & 3.33m & & & 6.68m\\ \hline

        % \multirow{2}{*} {\tabincell{c}{Random Init ($T = 0.01$)}} & \multirow{2}{*}{MCTS} & \multirow{2}{*}{{NA}} & \multirow{2}{*}{{NA}} & {NA} +  & \multirow{2}{*}{NA} & \multirow{2}{*}{{NA}} & {NA} +  & \multirow{2}{*}{{7.8752}} & \multirow{2}{*}{1.473 } & {0+}   \\
        % & & & & NA & & & NA & & & 6m\\     \hline
        %  \multirow{2}{*} {\tabincell{c}{Random Init ($T = 0.02$)}} & \multirow{2}{*}{MCTS} & \multirow{2}{*}{{NA}} & \multirow{2}{*}{{NA}} & {NA} +  & \multirow{2}{*}{NA} & \multirow{2}{*}{{NA}} & {NA} +  & \multirow{2}{*}{{7.8483}} & \multirow{2}{*}{1.126 } & {0+}   \\
        % & & & & NA & & & NA & & & 12m\\     \hline
        %   \multirow{2}{*} {\tabincell{c}{Uniform Init ($T = 0.01$)}} & \multirow{2}{*}{MCTS} & \multirow{2}{*}{{NA}} & \multirow{2}{*}{{NA}} & {NA} +  & \multirow{2}{*}{NA} & \multirow{2}{*}{{NA}} & {NA} +  & \multirow{2}{*}{{7.8332}} & \multirow{2}{*}{0.932 } & {0+}   \\
        % & & & & NA & & & NA & & & 6m\\     \hline     
        %   \multirow{2}{*} {\tabincell{c}{Uniform Init ($T = 0.02$)}} & \multirow{2}{*}{MCTS} & \multirow{2}{*}{{NA}} & \multirow{2}{*}{{NA}} & {NA} +  & \multirow{2}{*}{NA} & \multirow{2}{*}{{NA}} & {NA} +  & \multirow{2}{*}{{7.8084}} & \multirow{2}{*}{0.612 } & {0+}   \\
        % & & & & NA & & & NA & & & 12m\\     \hline  
        
        \multirow{2}{*} {\textbf{UTSP (Ours)}} & \multirow{2}{*}{UL, Search} & \multirow{2}{*}{{10.7289}} & \multirow{2}{*}{\textbf{0.0918}} & {0.56m} +  & \multirow{2}{*}{16.6846} & \multirow{2}{*}{\textbf{0.8394}} & {1.37m} +  & \multirow{2}{*}{{23.3903}} & \multirow{2}{*}{\textbf{1.1770}} & {3.35m+}   \\
         & & & & 1.11m & & & 1.33m & & & 2.67m\\ 
         
        % \multirow{1}{*} {\textbf{Ours (Restart)} ($T = 0.04$s)} & \multirow{1}{*}{UL, Search} & \multirow{1}{*}{{10.7405}} & \multirow{1}{*}{\textbf{0.1996}} & {0.56m} +  & \multirow{1}{*}{16.6847} & \multirow{1}{*}{\textbf{0.8394}} & {1.37m} +  & \multirow{1}{*}{{23.3903}} & \multirow{1}{*}{\textbf{1.1770}} & {3.35m+}   \\
        %  & & & & 0.53m & & & 1.33m & & & 2.67m\\
\bottomrule
        % \bottomrule[2pt]
    \end{tabular}
\end{table*}
% \twocolumn
Table~\ref{table:Exps01} and Table~\ref{table:Exps02} present model's performance on TSP 20, 50, 100, 200, 500 and 1,000. The first three lines in Table~\ref{table:Exps01} and Table~\ref{table:Exps02}  summarize the performance of two exact solvers (Concorde and Gurobi) and LKH3 heuristic~\cite{helsgaun2017extension}. The learning-based me
thods can be divided into RL sub-category and SL sub-category. 
Greedy decoding (G), Sampling (S), Beam Search (BS), and Monte Carlo Tree Search are the decoding schemes used in RL/SL. The 2-OPT is a greedy local search heuristic. 
%Given a neural network output, 2-OPT may generate a better solution.

We compare our model with existing solvers as well as different learning-based algorithms. The performance of our method is averaged of four runs with different random seeds. The running time for our method is divided into two parts: the inference time (building the heat map $\mathcal{H}$) and the search time (running search algorithm). 
\begin{figure}[t]
\vskip 0.2in
\begin{center}
\centerline{\includegraphics[width=\columnwidth]{Figures/GCNHeatMap.png}}
\caption{The heat map $\mathcal{H}$ generated using GCN on TSP 100. The diagonal elements are set to 0. $X$-axis and $y$-axis are the
city indices.}
\label{fig:GGNheat}
\end{center}
\vskip -0.2in
\end{figure}
\begin{figure}[t]
\vskip 0.2in
\begin{center}
\centerline{\includegraphics[width=\columnwidth]{Figures/SCTHeatMap.png}}
\caption{The heat map $\mathcal{H}$ generated using SAG on TSP 100. The diagonal elements are set to 0. $X$-axis and $y$-axis are the
city indices.}
\label{fig:SCTheat}
\end{center}
\vskip -0.2in
\end{figure}
On small instances, our results match the ground-truth solutions and generate average gaps of $\textbf{-0.00009\%}$, $\textbf{-0.002\%}$ and $\textbf{-0.00011\%}$ respectively on instances with $n = 20,50,100$, where the negative values are the results of the rounding problem. The total runtime of our method remains competitive w.r.t. all other learning baselines. On larger instances with $n$ = $200,500$ and $1,000$, we notice that traditional solvers and heuristics (Concorde,Gurobi and LKH3) fail to generate the optimal solutions within reasonable time when the size of problems grows. For RL/SL baselines, they  generate results far away from ideal solutions, particularly for cases with  $n=1,000$.   Our UTSP method is able to obtain $\textbf{0.0918\%}$, $\textbf{0.8394\%}$ and $\textbf{1.1770\%}$ on TSP $200,500$ and $1,000$, respectively. We remark that the UTSP takes a shorter total running time  (inference + search) and  outperform the existing learning baselines on these large instances. The gap between running time becomes more pronounced when the size increases to 1,000. More discussion between \cite{fu2021generalize} and  UTSP can be found in Appendix~\ref{sec:append_runtime}.





Our model also takes less training time because we require very few training instances. Taking TSP 100 as an example, RL/SL needs 1 million training instances, and the total training time can take one day using a  
NVIDIA V100 GPU, while our method only takes about 40 minutes with 2,000 training instances.  The training data size does not increase w.r.t. TSP size. Our training data consists of 2,000 instances for TSP 200, 500 and 1,000. At the same time, the UTSP model also remains very lightweight. On TSP 100, we use a 2-layer SAG with 64 hidden units and the model consists of 44,392 trainable parameters. In contrast, RL method in~\cite{kool2018attention} takes approximately 700,000 parameters and the SL method in~\cite{joshi2022learning} takes approximately 350,000 parameters.  


Our results indicate the UTSP algorithm is able to generate better solutions within a reasonable time.
Our UL pipeline also generalizes well to unseen examples without requiring a large number of training samples. This is because the loss function in Equation~\ref{eq:loss} is fully differentiable w.r.t the parameters in SAG and we are able to train the model in an end-to-end fashion. 
In other words, given a heat map $\mathcal{H}$, the model learns to assign large weights to more promising edges and small weights to less promising ones through backpropagation without any prior knowledge of the ground truth or any exploration step. However, when using SL, the model learns from the TSP solutions, which fails when multiple solutions exist or the solutions are not optimal. While for RL, the model often encounters an exploration dilemma and is not guaranteed to converge~\cite{bengio2021machine}\cite{joshi2019learning}.  Overall, UTSP requires fewer training samples and has better generalization comparing to SL/RL models. 

\subsection{Expressive Power of GNNs}
We aim to generate a non-smooth transition matrix $\mathbb{T}$  and build an expressive heat map $\mathcal{H}$ to guide the search algorithm.
However, most GNNs aggregate information  from adjacent nodes and these aggregation
steps usually consist of local averaging operations, which can be interpreted as a low-pass filter and causes the oversmoothing problem~\cite{wenkel2022overcoming}.  
The low-pass model generates a smooth transition matrix $\mathbb{T}$, which finally makes the elements $\mathcal{H}$ become indistinguishable. So it becomes difficult to discriminate whether the edges belong to the optimal solution or not.  In our model,  we assume all nodes in the graph are connected, so every node has $n-1$ neighboring nodes. This means every node receives messages from all other nodes and we have a global averaging operation over the graph, this can lead to severe oversmoothing issue.

To avoid oversmoothing, one solution is to use shallow GNNs.  However, this would result in narrow receptive fields and create the problem of underreaching~\cite{barcelo2020logical}. In our model, we use SAG because this scattering-based method helps  overcome the oversmoothing problem by combining band-pass wavelet filters with GCN-type filters~\cite{min2022can}. Figure~\ref{fig:trainingloss} illustrates the training loss on TSP $100$ and the differences between our SAG model and the graph convolutional network (GCN)~\cite{kipf2016semi}, where GCN 
  only performs low-pass filtering on graph signals~\cite{nt2019revisiting}. 
When using GCN, the training loss decreases slowly, and the validation loss reaches a plateau after we train the model for 20 epochs. This is because the low-pass model generates a smooth $\mathbb{T}$. Such a smooth $\mathbb{T}$ results in an indistinguishable $\mathcal{H}$, which harms the training process. Instead, we observe lower training and validation loss when using SAG; this suggests that SAG generates a more expressive representation which facilitates the training process. 



Figure~\ref{fig:GGNheat} and Figure ~\ref{fig:SCTheat} illustrate the generated heat maps using GCN and SAG on a TSP 100 instance, we choose this instance from the validation set randomly. When using the GCN, due to the oversmoothing problem, the model generates a smooth representation and $\mathcal{H}$ becomes indistinguishable. The elements in $\mathcal{H}$ have a small variance and most of them are $\sim$ 0.01. Instead, the SAG generates a discriminative representation and the elements in the heat map have a larger variance.




Here, we train both GCN and SAG with the same loss function. So the differences illustrated in Figure~\ref{fig:GGNheat} and~\ref{fig:SCTheat} are the direct result of overcoming the oversmoothing problem. 



\section{Search Space Reduction}
To understand what happens during our training process, we study how the prediction edge set $\Pi$ changes with training time. As mentioned, let $\Pi$ denote undirected edge set in $\mathcal{H}'$, and let $\Gamma$ denote the ground truth edge set, $
\eta = |\Gamma \cap \Pi|/|\Gamma|
$ is the extent of how good our prediction set $\Pi$ covers the solution $\Gamma$. If $\eta = 1$, then $\Gamma$ is a subset of $\Pi$, which means our prediction edge set successfully covers all ground truth edges. Similarly, $\eta = 0.95$ means we cover 95\% ground truth edges.

\begin{figure}[t]
\vskip 0.2in
\begin{center}
\centerline{\includegraphics[width=\columnwidth]{Figures/Edgefloat.png}}
\caption{Average edge overlap coefficient $\eta$ w.r.t. training epochs using SAG and GCN on TSP 100 ($M=10$).}
\label{fig:edgefloat}
\end{center}
\vskip -0.2in
\end{figure}
Figure~\ref{fig:edgefloat} shows how the average overlap coefficient $\eta$  changes with training epochs. We calculate the coefficient based on 1,000 validation instances in TSP 100. We notice that the coefficient quickly increases to $\sim 98\%$  after we train SAG for 10 epochs. This suggests that the surrogate loss successfully encourages the SAG to put more weights on the more promising edges. We also compare the performance with GCN. Since the loss does not decrease significantly during our training when using GCN (shown in Figure~\ref{fig:trainingloss}), it is not surprising to see the average overlap coefficient of GCN always maintains at a relatively low level. After training the model for 100 epochs, SAG model has an average coefficient of $99.756\%$ while GCN only has $33.893\%$.

Overall, the unsupervised learning  training reduces the search space from $4950$ edges to $583.134$ edges with over $99\%$ overlap accuracy. 
This helps explain why our search algorithm is able to perform well within reasonable time. 

\begin{figure}[t]
\vskip 0.2in
\begin{center}
\centerline{\includegraphics[width=\columnwidth]{Figures/OverlapHis.png}}
\caption{Number of fully covered instances w.r.t. training epochs using SAG and GCN on TSP 100. The validation set consists of 1,000 samples ($M=10$).}
\label{fig:overlaphis}
\end{center}
\vskip -0.2in
\end{figure}

We then study how many cases where our prediction edge set $\Pi$ covers the ground truth solution. Figure~\ref{fig:overlaphis} illustrates how the number of fully covered instances ($\eta = 1$) changes with time. After training the model for 100 epochs, we observed 780 fully covered instances in 1,000 validation samples using SAG while 0 instances using GCN. Finally, we  calculate the average of size $|\Pi|$. Our results show that SAG has an average size of $583.134$ edges, while for GCN, the number is $738.739$.  

These results also indicate that there is a correspondence between the loss and the quality of our prediction. 
In most SL tasks such as classification or regression tasks, a smaller validation loss usually means we achieve better performance and the minimum of the loss corresponds to the global optimal solution (100\% accuracy). However, it is no theoretical guarantee that our loss in Equation~\ref{eq:loss} is also a measure of the solution quality.
Our empirical results demonstrate that a lower surrogate loss encourages the model to assign larger weights on the promising edges and reduces the search space. This implies that we can assess the quality of the generated heat maps using our loss in Equation~\ref{eq:loss}.


We also compare the prediction edge sets and our results demonstrate that smooth representations fail to reduce the search space. Figure~\ref{fig:GGNheattop10} and Figure~\ref{fig:SCTheattop10} illustrate the difference of prediction edge sets between GCN and SAG. Figure~\ref{fig:GGNheattop10} and Figure~\ref{fig:SCTheattop10} are generated using the heat map in Figure~\ref{fig:GGNheat} and Figure~\ref{fig:SCTheat} with $M=10$, respectively. The light green regions correspond to the prediction edge set $\Pi$.  The $x$-axis  and $y$-axis  are the city indices, a light green box with position $(i,j)$ means edge $(i,j)$ belongs to $\Pi$. 
\begin{figure}[!htb]
\vskip 0.2in
\begin{center}
\centerline{\includegraphics[width=\columnwidth]{Figures/GCNHeatMapTop10.png}}
\caption{Illustration of how the prediction edge set covers the ground truth edges using GCN. Light green: the prediction edge set $\Pi$; blue: $\Pi$ contains the ground truth edge, red: $\Pi$ misses the ground truth edge.}
\label{fig:GGNheattop10}
\end{center}
\vskip -0.2in
\end{figure}
When using GCN, as shown in Figure~\ref{fig:GGNheattop10}, we observe more continuous light green regions comparing to Figure~\ref{fig:SCTheattop10}. As mentioned before, a low-pass model will enforce similarity on neighboring nodes and lead to unfavorable representations.  The continuous regions in Figure~\ref{fig:GGNheattop10} are the direct result of oversmoothness. We observe fewer continuous light green regions when using SAG, this suggest that the model helps alleviate the oversmoothing problem and generates a more distinguishable representation.  
\begin{figure}[!htb]
\vskip 0.2in
\begin{center}
\centerline{\includegraphics[width=\columnwidth]{Figures/SCTHeatMapTop10.png}}
\caption{Illustration of how the prediction edge set covers the ground truth edges using SAG. Light green: the prediction edge set $\Pi$; blue: $\Pi$ contains the ground truth edge, red: $\Pi$ misses the ground truth edge.}
\label{fig:SCTheattop10}
\end{center}
\vskip -0.2in
\end{figure}
We further study how the prediction sets cover the ground truth solution. In Figure~\ref{fig:GGNheattop10} and~\ref{fig:SCTheattop10}, blue and red boxes are the egdes in ground truth solution. A blue box with position $(i,j)$ corresponds to the condition that our prediction set $\Pi$ covers the right edge $(i,j)$,  while a red box at position $(i,j)$ means there is a ground truth edge $(i,j)$ but $\Pi$ fails to cover it. When using GCN, we observe 118 red boxes and 82 blue boxes, this means the GCN misses 59 correct edges, while SAG's prediction set successfully covers all the right edges.

Overall, the GCN's prediction set has 875 edges (1,750 light green boxes) and  SAG's prediction set $\Pi$ has 614 edges (1,228 light green boxes).  Although the low-pass model has a larger prediction set $\Pi$, it still falls short of covering the right edges. This emphasizes the importance of including band-pass filters and overcoming the oversmoothness problem.





 




\section{Conclusion}\label{sec:conclusion}
In this work, we focus on addressing the fundamental challenge of OOD detection tasks, which is how to fully understand the semantic discrepancy between the ID/OOD samples. We reveal that the key to success in the realistic SCOOD task is to allocate as many ID samples in the unlabeled set correctly as possible. To this end, we propose a novel uncertainty-aware optimal transport scheme that introduces class-specific energy scores as guidance for effective label assignment. Experimental results show that our method achieves better performance than previous state-of-the-art methods on SCOOD benchmarks.

\textbf{Limitations.} In addition to temperature scaling, other techniques such as feature clipping applied in ReAct~\cite{sun2021react} also enhance the performance of energy score, so how to obtain an OOD score that best fits the SCOOD task can be further explored. Moreover, a setting highly related to SCOOD has been proposed in \cite{katz2022training} and formulated as a constrained optimization problem. We will also theoretically analyze these practical OOD settings in our feature work.

% \section*{Acknowledgments}
\textbf{Acknowledgments.} 
This work is supported by National Key R\&D Program of China under Grant 2020AAA0105701, National Natural Science Foundation of China (NSFC) under Grants 61872327, Major Special Science and Technology Project of Anhui, National Natural Science Foundation of China (62033012) and Ant Group through Ant Research Intern Program.

% \nocite{langley00}
% \newpage
%\bibliography{reference}
\bibliographystyle{icml2023}
\begin{thebibliography}{22}
\providecommand{\natexlab}[1]{#1}
\providecommand{\url}[1]{\texttt{#1}}
\expandafter\ifx\csname urlstyle\endcsname\relax
  \providecommand{\doi}[1]{doi: #1}\else
  \providecommand{\doi}{doi: \begingroup \urlstyle{rm}\Url}\fi

\bibitem[Applegate et~al.(2006)Applegate, Bixby, Chvatal, and
  Cook]{applegate2006concorde}
Applegate, D., Bixby, R., Chvatal, V., and Cook, W.
\newblock Concorde tsp solver, 2006.

\bibitem[Barcel{\'o} et~al.(2020)Barcel{\'o}, Kostylev, Monet, P{\'e}rez,
  Reutter, and Silva]{barcelo2020logical}
Barcel{\'o}, P., Kostylev, E.~V., Monet, M., P{\'e}rez, J., Reutter, J., and
  Silva, J.-P.
\newblock The logical expressiveness of graph neural networks.
\newblock In \emph{8th International Conference on Learning Representations
  (ICLR 2020)}, 2020.

\bibitem[Bengio et~al.(2021)Bengio, Lodi, and Prouvost]{bengio2021machine}
Bengio, Y., Lodi, A., and Prouvost, A.
\newblock Machine learning for combinatorial optimization: a methodological
  tour d’horizon.
\newblock \emph{European Journal of Operational Research}, 290\penalty0
  (2):\penalty0 405--421, 2021.

\bibitem[Bresina(1996)]{random2}
Bresina, J.~L.
\newblock Heuristic-biased stochastic sampling.
\newblock In \emph{AAAI/IAAI, Vol. 1}, pp.\  271--278, 1996.

\bibitem[Croes(1958)]{2opt}
Croes, G.~A.
\newblock A method for solving traveling-salesman problems.
\newblock \emph{Operations research}, 6\penalty0 (6):\penalty0 791--812, 1958.

\bibitem[Deudon et~al.(2018)Deudon, Cournut, Lacoste, Adulyasak, and
  Rousseau]{deudon2018learning}
Deudon, M., Cournut, P., Lacoste, A., Adulyasak, Y., and Rousseau, L.-M.
\newblock Learning heuristics for the tsp by policy gradient.
\newblock In \emph{International Conference on the Integration of Constraint
  Programming, Artificial Intelligence, and Operations Research}, pp.\
  170--181. Springer, 2018.

\bibitem[Fu et~al.(2021)Fu, Qiu, and Zha]{fu2021generalize}
Fu, Z.-H., Qiu, K.-B., and Zha, H.
\newblock Generalize a small pre-trained model to arbitrarily large tsp
  instances.
\newblock \emph{Proceedings of the AAAI Conference on Artificial Intelligence},
  35\penalty0 (8):\penalty0 7474--7482, 2021.

\bibitem[Gomes et~al.(1998)Gomes, Selman, Kautz, et~al.]{random1}
Gomes, C.~P., Selman, B., Kautz, H., et~al.
\newblock Boosting combinatorial search through randomization.
\newblock \emph{AAAI/IAAI}, 98:\penalty0 431--437, 1998.

\bibitem[Gomes et~al.(2000)Gomes, Selman, Crato, and Kautz]{random3}
Gomes, C.~P., Selman, B., Crato, N., and Kautz, H.
\newblock Heavy-tailed phenomena in satisfiability and constraint satisfaction
  problems.
\newblock \emph{Journal of automated reasoning}, 24\penalty0 (1):\penalty0
  67--100, 2000.

\bibitem[Helsgaun(2000)]{helsgaun2000effective}
Helsgaun, K.
\newblock An effective implementation of the lin--kernighan traveling salesman
  heuristic.
\newblock \emph{European journal of operational research}, 126\penalty0
  (1):\penalty0 106--130, 2000.

\bibitem[Helsgaun(2017)]{helsgaun2017extension}
Helsgaun, K.
\newblock An extension of the lin-kernighan-helsgaun tsp solver for constrained
  traveling salesman and vehicle routing problems.
\newblock \emph{Roskilde: Roskilde University}, pp.\  24--50, 2017.

\bibitem[Joshi et~al.(2019{\natexlab{a}})Joshi, Laurent, and
  Bresson]{joshi2019efficient}
Joshi, C.~K., Laurent, T., and Bresson, X.
\newblock An efficient graph convolutional network technique for the travelling
  salesman problem.
\newblock \emph{arXiv preprint arXiv:1906.01227}, 2019{\natexlab{a}}.

\bibitem[Joshi et~al.(2019{\natexlab{b}})Joshi, Laurent, and
  Bresson]{joshi2019learning}
Joshi, C.~K., Laurent, T., and Bresson, X.
\newblock On learning paradigms for the travelling salesman problem.
\newblock \emph{arXiv preprint arXiv:1910.07210}, 2019{\natexlab{b}}.

\bibitem[Joshi et~al.(2022)Joshi, Cappart, Rousseau, and
  Laurent]{joshi2022learning}
Joshi, C.~K., Cappart, Q., Rousseau, L.-M., and Laurent, T.
\newblock Learning the travelling salesperson problem requires rethinking
  generalization.
\newblock \emph{Constraints}, pp.\  1--29, 2022.

\bibitem[Kingma \& Ba(2014)Kingma and Ba]{kingma2014adam}
Kingma, D.~P. and Ba, J.
\newblock Adam: A method for stochastic optimization.
\newblock \emph{arXiv preprint arXiv:1412.6980}, 2014.

\bibitem[Kipf \& Welling(2016)Kipf and Welling]{kipf2016semi}
Kipf, T.~N. and Welling, M.
\newblock Semi-supervised classification with graph convolutional networks.
\newblock \emph{arXiv preprint arXiv:1609.02907}, 2016.

\bibitem[Kool et~al.(2019)Kool, van Hoof, and Welling]{kool2018attention}
Kool, W., van Hoof, H., and Welling, M.
\newblock Attention, learn to solve routing problems!
\newblock In \emph{International Conference on Learning Representations}, 2019.
\newblock URL \url{https://openreview.net/forum?id=ByxBFsRqYm}.

\bibitem[Min et~al.(2022)Min, Wenkel, Perlmutter, and Wolf]{min2022can}
Min, Y., Wenkel, F., Perlmutter, M., and Wolf, G.
\newblock Can hybrid geometric scattering networks help solve the maximum
  clique problem?
\newblock \emph{arXiv preprint arXiv:2206.01506}, 2022.

\bibitem[Nt \& Maehara(2019)Nt and Maehara]{nt2019revisiting}
Nt, H. and Maehara, T.
\newblock Revisiting graph neural networks: All we have is low-pass filters.
\newblock \emph{arXiv preprint arXiv:1905.09550}, 2019.

\bibitem[Qiu et~al.(2022)Qiu, Sun, and Yang]{qiu2022dimes}
Qiu, R., Sun, Z., and Yang, Y.
\newblock Dimes: A differentiable meta solver for combinatorial optimization
  problems.
\newblock \emph{arXiv preprint arXiv:2210.04123}, 2022.

\bibitem[Sylvester(1909)]{sylvester1909collected}
Sylvester, J.~J.
\newblock \emph{The Collected Mathematical Papers of James Joseph
  Sylvester...}, volume~3.
\newblock University Press, 1909.

\bibitem[Wenkel et~al.(2022)Wenkel, Min, Hirn, Perlmutter, and
  Wolf]{wenkel2022overcoming}
Wenkel, F., Min, Y., Hirn, M., Perlmutter, M., and Wolf, G.
\newblock Overcoming oversmoothness in graph convolutional networks via hybrid
  scattering networks.
\newblock \emph{arXiv preprint arXiv:2201.08932}, 2022.

\end{thebibliography}



%%%%%%%%%%%%%%%%%%%%%%%%%%%%%%%%%%%%%%%%%%%%%%%%%%%%%%%%%%%%%%%%%%%%%%%%%%%%%%%
%%%%%%%%%%%%%%%%%%%%%%%%%%%%%%%%%%%%%%%%%%%%%%%%%%%%%%%%%%%%%%%%%%%%%%%%%%%%%%%
% APPENDIX
%%%%%%%%%%%%%%%%%%%%%%%%%%%%%%%%%%%%%%%%%%%%%%%%%%%%%%%%%%%%%%%%%%%%%%%%%%%%%%%
%%%%%%%%%%%%%%%%%%%%%%%%%%%%%%%%%%%%%%%%%%%%%%%%%%%%%%%%%%%%%%%%%%%%%%%%%%%%%%%
\newpage
\section{Appendix}


\subsection{Training and Search Details}
We train our model using Adam~\cite{kingma2014adam}. All models are trained using Nvidia V100 GPU. 
All the search-related parameters are listed in Table \ref{tb:sp}. $M$ refers to the size of the candidate set of each city. $K$ is the maximal number of edges we can remove in one action, and for each round of local search, we randomly select one number from the listed interval. $T$ is the total number of actions we will try to expand one node. Here we keep the $\alpha = 0$ to show that the quality of our unsupervised learned heat map is high. Lower $\alpha$ means the local search algorithm focuses more on the edges with higher heat map value. Actually, in the experiments, we find the results are similar with $\alpha \leq 1$.
\begin{table}[h]
    \centering
    \begin{tabular}{lllllll}
        \toprule
        & $\alpha$ & $\beta$ & $M$ & $K$ & $T$  \\
        
        \midrule
        TSP-20 & 0 & 10 & 8 & 10 & 60  \\
        TSP-50 & 0 & 10 & 8 & [5, 15) & 150  \\
        TSP-100 & 0 & 10 & 8 & [5, 35) & 300  \\
        TSP-200 & 0 & 10 & 8 & [10, 90) & 600  \\
        TSP-500 & 0 & 50 & 5 & [30, 130) & 1000  \\
        TSP-1000 & 0 & 50 & 5 & [10, 110) & 2000  \\
        \bottomrule
    \end{tabular}
    \caption{Search parameters for all the TSP experiments.}
    \label{tb:sp}
\end{table}
%We select $M = 8,10,10$ for TSP 20, 50 and 100, $M = 20$ for TSP 200, 500 and 1,000 and $M = 100$ for TSP 10,000. All the models are trained using Nvidia V100 GPU. 

\subsection{Running Time Discussion}
\label{sec:append_runtime}
As discussed in \cite{kool2018attention}, running time is important but hard to compare since it is affected by many factors. We report the clock time for solving all the test instances  in Table \ref{table:Exps01}.
For the UTSP (our method) and the state-of-the-art learning-based method Att-GCRN \cite{fu2021generalize}, we run the search algorithm on exactly the same environment (one Intel  Xeon Gold 6326) for a fair comparison.  And for other baselines, we directly refer to the results from \cite{fu2021generalize}. So the time there are only for indicative purpose since the computing hardware is not the same.


%\appendix

%\section{Appendix for Proofs}

\paragraph{Proof of Theorem \ref{thm:main}.}

\begin{proof}
\label{proof:main}
Our proof has two steps. In Step 1, we will show that SimCLR is equivalent to minimizing the cross entropy loss defined in Eqn.~(\ref{eqn:cross-entropy}). 
In Step 2, we will show  that minimizing the cross-entropy loss 
is equivalent to spectral clustering on $\bfpi$. 
Combining the two steps together, we have proved our theorem. 

\textbf{Step 1: } SimCLR is equivalent to minimizing the cross entropy loss.

The cross-entropy loss takes expectation over 
$\bfW_\bfX\sim \mathbb{P}(\cdot ; \bfpi)$, 
which means $\bfW_\bfX$ has exactly one non-zero entry in each row $i$. By Lemma~\ref{lem:multinomial}, we know every row $i$ of $\bfW_\bfX$ is independent of other rows. Moreover, 
$\bfW_{\bfX,i}\sim \mathcal{M}(1, \bfpi_i/\sum_j \bfpi_{i,j})=\mathcal{M}(1, \bfpi_i)$, because $\bfpi_i$ itself is a probability distribution.
Similarly, we know $\bfW_\bfZ$ also has the row-independent property by sampling over $\mathbb{P}(\cdot;\bfK_\bfZ)$.
Therefore, by Lemma~\ref{lem:cross_split}, we know Eqn.~(\ref{eqn:cross-entropy}) is equivalent to:
\[
 -\sum_{i=1}^n \mathbb{E}_{\bfW_{\bfX,i}}[\log \mathbb{P}(\bfW_{\bfZ,i}=\bfW_{\bfX,i};\bfK_\bfZ)],
\]

This expression takes expectation over $\bfW_{\bfX,i}$ for the given row $i$. Notice that 
$\bfW_{\bfX,i}$ has exactly one non-zero entry, which equals $1$ (same for $\bfW_{\bfZ,i}$). 
As a result
we expand the above expression to be:
\begin{equation}
 -\sum_{i=1}^n \sum_{j\neq i} \Pr(\bfW_{\bfX,i,j}=1)\log \Pr(\bfW_{\bfZ,i,j}=1).
\label{eqn:detailed-expansion}    
\end{equation}


By Lemma~\ref{lem:multinomial}, $\Pr(\bfW_{\bfZ,i,j}=1)=\bfK_{\bfZ,i,j}/\|\bfK_{\bfZ,i}\|_1$ for $j\neq i$. Recall that $\bfK_\bfZ=(k(\bfZ_i-\bfZ_j))_{(i,j)\in[n]^2}$, which means 
$\bfK_{\bfZ,i,j}/\|\bfK_{\bfZ,i}\|_1=\frac{\exp(-\|\bfZ_i-\bfZ_j\|^2/{2\tau})}{\sum_{k\neq i}
\exp(-\|\bfZ_i-\bfZ_k\|^2/{2\tau})
}$ for $j\neq i$, when $k$ is the Gaussian kernel with variance $\tau$. 

Notice that $\bfZ_i=f(\bfX_i)$, so we know
\begin{equation}
-\log \Pr(\bfW_{\bfZ,i,j}=1)=
-\log \frac{\exp(-\|f(\bfX_i)-f(\bfX_j)\|^2/{2\tau})}{\sum_{k\neq i}
\exp(-\|f(\bfX_i)-f(\bfX_k)\|^2/{2\tau}),
}
\label{eqn:infonce-equivalence}    
\end{equation}


The right hand side is exactly the InfoNCE loss defined in Eqn.~(\ref{eqn:infonce}).
Inserting Eqn.~(\ref{eqn:infonce-equivalence}) into Eqn.~(\ref{eqn:detailed-expansion}), we get the SimCLR algorithm, which first samples augmentation pairs $(i,j)$ with $\Pr(\bfW_{\bfX,i,j}=1)$ for each row $i$, and then optimize the InfoNCE loss. 

\textbf{Step 2: } minimizing the cross entropy loss 
is equivalent to spectral clustering on $\bfpi$.


By Lemma~\ref{lem:convert_to_spectral}, we may further convert the loss to 
\begin{equation}
\label{eqn:main-theorem-repul-attr}
\min_{\bfZ}
-\sum_{(i,j)\in [n]^2} \mathbf{P}_{i,j}
\log k (\bfZ_i-\bfZ_j)+\log \mathbf{R}(\bfZ).
\end{equation}
Since $k$ is the Gaussian kernel, this reduces to \[
\min_\bfZ \mathrm{tr}(\bfZ^\top \mathbf{L}(\bfpi) \bfZ)
+\log \mathbf{R}(\bfZ),
\]

where we use the fact that $\mathbb{E}_{\bfW_\bfX\sim \mathbb{P}(\cdot; \bfpi)}[\mathbf{L}(\bfW_\bfX)]
=\mathbf{L}(\bfpi)
$, because the Laplacian operator is linear and $
\mathbb{E}_{\bfW_\bfX\sim \mathbb{P}(\cdot; \bfpi)}(\bfW_\bfX)=\bfpi
$.
\end{proof}

\paragraph{Proof of Theorem \ref{thm:clip}.}
\begin{proof}
Since $\bfW_\bfX\sim \mathbb{P}(\cdot;\bfpi_{\mathbf{A}, \mathbf{B}})$, we know 
$\bfW_\bfX$ has exactly one non-zero entry in each row, denoting the pair that got sampled. 
A notable difference compared to the previous proof is we now have $n_\mathcal{A}+n_\mathcal{B}$ objects in our graph. CLIP deals with this by taking a mini-batch of size $2N$, 
such that $n_\mathcal{A}=n_\mathcal{B}=N$, and adding the $2N$ InfoNCE losses together. We label the objects in $\mathcal{A}$ as $[n_\mathcal{A}]$, and the objects in $\mathcal{B}$ as $\{n_\mathcal{A}+1, \cdots, n_\mathcal{A}+n_\mathcal{B}\}$. 

Notice that $\bfpi_{\mathbf{A}, \mathbf{B}}$ is a bipartite graph, so the edges of objects in $\mathcal{A}$ will only connect to object in $\mathcal{B}$ and vice versa. We can define the similarity matrix in $\cZ$ as $\bfK_\bfZ$, 
where $\bfK_\bfZ(i, j+n_\mathcal{A})=\bfK_\bfZ(j+n_\mathcal{A},i)= k(\bfZ_i-\bfZ_j)$ for $i\in [n_\mathcal{A}], j\in [n_\mathcal{B}]$, and otherwise we set $\bfK_\bfZ(i,j)=0$. 
The rest is same as the previous proof. 
\end{proof}

\paragraph{Proof of Theorem \ref{thm:exponential}.}

\begin{proof}
\label{proof:exponential}
Since the objective function consists of a linear term combined with an entropy regularization, which is a strongly concave function, the maximization problem is a convex optimization problem. Owing to the implicit constraints provided by the entropy function, the problem is equivalent to having only the equality constraint. We then introduce the Lagrangian multiplier $\lambda$ and obtain the following relaxed problem:

$$
\widetilde{E}(\boldsymbol{\alpha})=\psi_{1}-\sum_{i=1}^n \alpha_{i} \psi_{i}+\tau \sum_{i=1}^n \alpha_{i}\log \alpha_{i}+\lambda\left(\boldsymbol{\alpha}^{\top} \mathbf{1}_n-1\right).
$$

As the relaxed problem is unconstrained, taking the derivative with respect to $\alpha_{i}$ yields

$$
\frac{\partial \widetilde{E}(\boldsymbol{\alpha})}{\partial \alpha_{i}}=-\psi_{i}+\tau\left(\log \alpha_{i}+\alpha_{i} \frac{1}{\alpha_{i}}\right)+\lambda=0.
$$

Solving the above equation implies that $\alpha_{i}$ takes the form
$
\alpha_{i}=\exp \left(\frac{1}{\tau} \psi_{i}\right) \exp \left(\frac{-\lambda}{\tau}-1\right).
$ Since $\alpha_{i}$ lies on the probability simplex, the optimal $\alpha_{i}$ is explicitly given by
$
\alpha^{*}_{i}=\frac{\exp \left(\frac{1}{\tau} \psi_{i}\right)}{\sum_{i^{\prime}=1}^n \exp \left(\frac{1}{\tau} \psi_{i^{\prime}}\right)} .
$ Substituting the optimal point into the objective function, we obtain
$$
\begin{aligned}
E\left(\boldsymbol{\alpha}^*\right)  &=\psi_1-\sum_{i=1}^n \frac{\exp \left(\frac{1}{\tau} \psi_{i}\right)}{\sum_{i^{\prime}=1}^n \exp \left(\frac{1}{\tau} \psi_{i^{\prime}}\right)} \psi_{i}+\tau \sum_{i=1}^n \frac{\exp \left(\frac{1}{\tau} \psi_{i}\right)}{\sum_{i^{\prime}=1}^n \exp \left(\frac{1}{\tau} \psi_{i^{\prime}}\right)}\log \frac{\exp \left(\frac{1}{\tau} \psi_{i}\right)}{\sum_{i^{\prime}=1}^n \exp \left(\frac{1}{\tau} \psi_{i^{\prime}}\right)} \\
& =\psi_1 - \tau \log \left(\sum_{i=1}^n \exp \left(\frac{1}{\tau} \psi_{i}\right)\right).
\end{aligned}
$$
Thus, the Lagrangian dual function is given by
\begin{equation*}
-E\left(\boldsymbol{\alpha}^*\right)= -\tau \log \frac{\exp \left(\frac{1}{\tau} \psi_{1}\right)}{\sum_{i=1}^n \exp \left(\frac{1}{\tau} \psi_{i}\right)}.\qedhere
\end{equation*}
\end{proof}



\section{More on Experiments} \label{section: experiment_details}

\paragraph{CIFAR-10 and CIFAR-100} CIFAR-10 ~\citep{krizhevsky2009learning} and CIFAR-100 ~\citep{krizhevsky2009learning} are well-known classic image classification datasets. Both CIFAR-10 and CIFAR-100 contain a total of 60k $32 \times 32$ labeled images of different classes, with 50k for training and 10k for testing. CIFAR-10 is similar to CIFAR-100, except there are 10 different classes in CIFAR-10 and 100 classes in CIFAR-100.

\paragraph{TinyImageNet} TinyImageNet ~\citep{le2015tiny} is a subset of ImageNet ~\citep{deng2009imagenet}. There are 200 different object classes in TinyImageNet, with 500 training images, 50 validation images, and 50 test images for each class. All the images in TinyImageNet are colored and labeled with a size of $64 \times 64$.

\textbf{Pseudo-code.} Algorithm \ref{alg:Training Procedure} presents the pseudo-code for our empirical training procedure.

\begin{algorithm}[!htbp]
\caption{Training Procedure}
\label{alg:Training Procedure}
\begin{algorithmic}[1]
\REQUIRE trainable encoder network $f$, batch size $N$, augmentation strategy \textit{aug}, loss function $L$ with hyperparameters \textit{args}
\FOR {sampled minibatch ${x_i}_{i=1}^N$}
\FORALL{$i \in { 1, ..., N }$}
\STATE draw two augmentations $t_i = \textit{aug}\left(x_i\right) $, $t_i' = \textit{aug}\left(x_i\right) $
\STATE $z_i = f\left(t_i\right)$, $z_i' = f\left(t_i'\right)$
\ENDFOR
\STATE compute loss $\mathcal{L} = L(N, z, z', \textit{args})$
\STATE update encoder network $f$ to minimize $\mathcal{L}$
\ENDFOR
\STATE \textbf{Return} encoder network $f$
\end{algorithmic}
\end{algorithm}

We also provide the pseudo-code for our core loss function used in the training procedure in Algorithm \ref{alg:Core loss}. The pseudo-code is almost identical to SimCLR's loss function, with the exception of an extra parameter $\gamma$.

\begin{algorithm}[!htbp]
\caption{Core loss function $\mathcal{C}$}
\label{alg:Core loss}
\begin{algorithmic}[1]
\REQUIRE batch size $N$, two encoded minibatches $z_1, z_2$, $\gamma$, temperature $\tau$
\STATE $z = \textit{concat}\left(z_1, z_2\right)$
\FOR {$i \in {1, ..., 2N }, j \in {1, ..., 2N}$ }
\STATE $s_{i,j} = \Vert z_i - z_j \Vert_2^{\gamma}$
\ENDFOR
\STATE \textbf{define} $l(i, j)$ \textbf{as} $l(i, j) = - \log \frac{exp\left(s_{i,j}/\tau \right)}{\sum_{k=1}^{2N} \mathbf{1}{[k \ne i]} exp\left(s{i, j} / \tau \right)} $
\STATE \textbf{Return} $\frac{1}{2N} \sum_{k=1}^N\left[l(i, i+N) + l(i+N, i)\right]$
\end{algorithmic}
\end{algorithm}

Utilizing the core loss function $\mathcal{C}$, we can define all kernel loss functions used in our experiments in Table \ref{table: loss definition}. For all $z_i \in z$ with even dimensions $n$, we define $z_{L_i} = z_i\left[0:n/2\right]$ and $z_{R_i} = z_i\left[n/2:n\right]$.

\begin{table}[ht]
\centering
\begin{tabular}{{@{}l|l@{}}}
Kernel  &  Loss function \\ \midrule
Laplacian & $\mathcal{C}\left(N, z, z', \gamma=1, \tau\right)$\\ \midrule
Sum       & $\lambda * \mathcal{C}\left(N, z, z', \gamma=1, \tau_1\right) + (1-\lambda) * \mathcal{C}\left(N, z, z', \gamma=2, \tau_2\right)$  \\ \midrule
Concatenation Sum&$\lambda * \mathcal{C}\left(N, z_L, z'_L, \gamma=1, \tau_1\right) + (1-\lambda) * \mathcal{C}\left(N, z_R, z'_R, \gamma=2, \tau_2\right)$\\ \midrule
$\gamma = 0.5$ & $\mathcal{C}\left(N, z, z', \gamma=0.5, \tau\right)$          \\ 

\end{tabular}

\caption{Definition of kernel loss functions in our experiments}
\label {table: loss definition}
\end{table}

\textbf{Baselines.} We reproduce the SimCLR algorithm using PyTorch Lightning~\citep{PytorchLightning}.

\textbf{Encoder details.}
The encoder $f$ consists of a backbone network and a projection network. We employ ResNet50~\citep{ResNet} as the backbone and a 2-layer MLP (connected by a batch normalization~\citep{ioffe2015batch} layer and a ReLU \cite{nair2010rectified} layer) with hidden dimensions 2048 and output dimensions 128 (or 256 in the concatenation kernel case).

\textbf{Encoder hyperparameter tuning.}
For each encoder training case, we randomly sample 500 hyperparameter groups (sample details are shown in Table \ref{table: Hyperparameter sample}) and train these samples simultaneously using Ray Tune ~\citep{RayTune}, with the ASHA scheduler~\citep{li2018massively}. Ultimately, the hyperparameter group that maximizes the online validation accuracy (integrated in PyTorch Lightning) within 5000 validation steps is chosen for the given encoder training case.

\begin{table}[ht]
\centering

\begin{tabular}{@{}l|l|l@{}}
\midrule
Hyperparameter  & Sample Range & Sample Strategy \\ \midrule
start learning rate & $\left[10^{-2}, 10\right]$ & log uniform \\ \midrule
$\lambda$       & $\left[0, 1\right]$ & uniform \\ \midrule
$\tau$, $\tau_1$, $\tau_2$ & $\left[0, 1\right]$ & log uniform \\ \midrule
\end{tabular}

\caption{Hyperparameters sample strategy}
\label {table: Hyperparameter sample}
\end{table}

\textbf{Encoder training.} 
We train each encoder using the LARS optimizer~\citep{LARSOptimizer}, LambdaLR Scheduler in PyTorch, momentum 0.9, weight decay $10^{-6}$, batch size 256, and the aforementioned hyperparameters for 400 epochs on a single A-100 GPU.

\textbf{Image transformation.} The image transformation strategy, including augmentation, is identical to the default transformation strategy provided by PyTorch Lightning.

\textbf{Linear evaluation.}
The linear head is trained using the SGD optimizer with a cosine learning rate scheduler, batch size 64, and weight decay $10^{-6}$ for 100 epochs. The learning rate starts at $0.3$ and ends at $0$.

\textbf{Moco Experiments.} We also tested our method based on MoCo~\citep{he2019moco}. The results are summarized in Table \ref{tab:results-moco}. Here we choose ResNet18~\citep{ResNet} as the backbone and set a temperature of $0.1$ as default. For our simple sum kernel, we set $\lambda=0.8$. The results show that our method outperforms the original MoCo method.

\begin{table}[thb]
\centering
\caption{MoCo Experiment Results on CIFAR-10 and CIFAR-100.}
\label{tab:results-moco}
\resizebox{\textwidth}{!}{%
\begin{tabular}{@{}c|ccc|ccc@{}}
\toprule
\multirow{3}{*}{Method} & \multicolumn{3}{c|}{CIFAR-10} & \multicolumn{3}{c}{CIFAR-100} \\ \cmidrule(lr){2-4} \cmidrule(lr){5-7} 
                        & 200 epochs & 400 epochs    & 1000 epochs   & 200 epochs & 400 epochs & 1000 epochs         \\ \midrule
MoCo (repro.)         & $76.41 \pm 0.12$    & $80.01 \pm 0.15$          & $84.45 \pm 0.08$    & $\mathbf{47.02 \pm 0.11}$ & $52.50 \pm 0.07$ & $57.62 \pm 0.15$            \\
\midrule
Laplacian Kernel        & ${78.09 \pm 0.10}$    & $\mathbf{83.85 \pm 0.09}$          & $\mathbf{88.34 \pm 0.16}$    & $46.12 \pm 0.22$   & $53.44 \pm 0.17$ & $59.10 \pm 0.14$        \\
Simple Sum Kernel & $\mathbf{78.12 \pm 0.15}$   & $83.23 \pm 0.18$ & $87.50 \pm 0.20$ & $46.65 \pm 0.06$ & $\mathbf{53.62 \pm 0.19}$ & $\mathbf{59.83 \pm 0.12}$\\
\bottomrule
\end{tabular}
}
\end{table}



\section{More Experiments on Synthetic Data}


Consider a scenario with $n$ clusters, each containing $k$ vertices. Let the probability of vertices $u$ and $v$ from the same cluster belonging to $\bfpi$ be $p$. Conversely, for vertices $u$ and $v$ from different clusters, let the probability of belonging to $\pi$ be $q$. We generate the graph $\bfpi$ randomly, based on $p$ and $q$. We experiment with values of $k=100$ and $n=6$ for ease of visualization, embedding all points in a two-dimensional space. Each vertex's initial position originates from a normal distribution. In each iteration, we sample a subgraph of $\bfpi$ uniformly, ensuring each vertex has an out-degree of $1$. We then optimize the corresponding vectors using InfoNCE loss with an SGD optimizer and iterate until convergence. Our experimental setup consists of an SGD learning rate of $1$, an InfoNCE loss temperature of $0.5$, and a batch size of $50$. We evaluate two scenarios with different $p$ and $q$ values: $p=1$, $q=0$, and $p=0.75$, $q=0.2$. The results of these experiments are visualized in Figure \ref{fig:vis-spectral-cluster}. The obtained embeddings exhibit the hallmark pattern of spectral clustering of graph $\bfpi$.

\begin{figure}[!tb]
\centering
\subfigure{
\includegraphics[width=1\textwidth]{Figures/cluster_pi.png}
\label{fig:vis-cluster}
}
\subfigure{
\includegraphics[width=1\textwidth]{Figures/noised_cluster_pi.png}
\label{fig:vis-noised-cluster}
}
\caption{Visualizations of the optimization process using InfoNCE Loss on the vectors corresponding to $\bfpi$. Points of identical color belong to the same cluster within $\bfpi$. To showcase the internal structure of $\bfpi$, we randomly select 10 vertices from each cluster to display the edge distribution of $\bfpi$.}
\label{fig:vis-spectral-cluster}
\end{figure}


%%%%%%%%%%%%%%%%%%%%%%%%%%%%%%%%%%%%%%%%%%%%%%%%%%%%%%%%%%%%%%%%%%%%%%%%%%%%%%%
%%%%%%%%%%%%%%%%%%%%%%%%%%%%%%%%%%%%%%%%%%%%%%%%%%%%%%%%%%%%%%%%%%%%%%%%%%%%%%%

\end{document}


% This document was modified from the file originally made available by
% Pat Langley and Andrea Danyluk for ICML-2K. This version was created
% by Iain Murray in 2018, and modified by Alexandre Bouchard in
% 2019 and 2021 and by Csaba Szepesvari, Gang Niu and Sivan Sabato in 2022. 
% Previous contributors include Dan Roy, Lise Getoor and Tobias
% Scheffer, which was slightly modified from the 2010 version by
% Thorsten Joachims & Johannes Fuernkranz, slightly modified from the
% 2009 version by Kiri Wagstaff and Sam Roweis's 2008 version, which is
% slightly modified from Prasad Tadepalli's 2007 version which is a
% lightly changed version of the previous year's version by Andrew
% Moore, which was in turn edited from those of Kristian Kersting and
% Codrina Lauth. Alex Smola contributed to the algorithmic style files.
