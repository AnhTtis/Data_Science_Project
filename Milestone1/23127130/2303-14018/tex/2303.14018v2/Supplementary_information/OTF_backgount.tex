\section*{OTF background}

To find the extent of an OTF, we need to find where the crossing from signal to backgound happens. To find this cutoff power, we use the assumed composition of our signal:

\begin{align}
    X &= P(C_p) + G(\sigma_g) + b \\
    X &= C_p + \eta_p + \eta_g + b
    \label{eq:signal}
\end{align}

where $P$ is a poisson distribution of the photon count $C_p$ resulting in noise $\eta_p$, $G$ is gaussian noise of $\sigma_g$ resulting in noise $\eta_g$, and $b$ is a bias offset. The noise $\eta_p$ and $\eta_g$ are assumed to be independent and spectrally white. We can then find the variance of the signal using:

\begin{equation}
    Var(X_i) = \cancelto{0}{Var(C_{p,i})} + Var(\eta_{p,i}) + Var(\eta_{g,i}) + \cancelto{0}{Var(b_i)} = \sigma_{p,i}^2 + \sigma_{g,i}^2
\end{equation}

where $i$ is the pixel index. As the noise is spectrally white the variance for each pixel is independent. The variance of the entire signal is then given by:

\begin{equation}
    Var(X) = \frac{\sum_i Var(X_i) + \cancelto{0}{\sum_{i\neq j}Cov(X_i,X_j)}}{N_c^3} = \overline{\sigma_p^2} + \overline{\sigma_g^2}
\end{equation}

where $\overline{\sigma_p^2}$ and $\overline{\sigma_g^2}$ are the average variance of the poisson and gaussian noise respectively. We know that the variance of the poisson noise is equal to the mean, and the variance of the gaussian noise is given by the camera manufacturer by the electron RMS value. This means we can find the variance of the signal:

\begin{equation}
    Var(X) = \overline{C_p} + \sigma_{RMS}^2
\end{equation}

where $\overline{C_p}$ is the average signal power without the bias offset. To correlate this variance to the OTF, we assume the signal can be decomposed as in equation \ref{eq:signal}. As the Fourier transform is a linear operation, we can then assume the OTF can be decomposed as:

\begin{equation}
    \hat{X} = \hat{C_p} + \hat{\eta_p} + \hat{\eta_g} + \hat{b}
\end{equation}

where $\hat{X}$ is the Fourier transform of $X$. We then assume the signal ground truth transforms into a perfect OTF with no background and DC term equal $\overline{C_p}$. The bias offset will only have a frequency component in the DC term equal to $b$. As the noise terms are spectrally white, they will have a constant power across all frequencies. This power can be found using Wiener-Khinchin theorem\cite{stein2000computer}:

\begin{equation}
    \overline{S}_X = |\hat{X}|^2
\end{equation}

where $\overline{S}_X$ is the average power spectral density of $X$. We can find the average power of the Fourier transform of the noise terms:

\begin{equation}
    \overline{S}_{\eta} = \sum_{h} \gamma_{eta}(h)e^{-2\pi i \nu h}
\end{equation}

where $\gamma_{eta}(h)$ is the autocovariance of the noise term $\eta$, and $\nu$ is the frequency. As the noise is spectrally white, this only leads to the variance of the noise, and we get:

\begin{align}
    \overline{S}_{\eta} &= \overline{S}_{\eta_p} + \overline{S}_{\eta_g} = C_p + \sigma_{RMS}^2 \\
    |\hat{\eta}| &= \sqrt{\overline{C_p} + \sigma_{RMS}^2}
\end{align}

as we don't know $C_p$, we can estimate it using the DC term of the OTF. As the DC term is composed of the signal and the bias offset, we get:

\begin{equation}
    \overline{C_p} = DC_{\hat{X}}-b 
\end{equation}

where $DC_{\hat{X}}$ is the DC term of the OTF. Using this we can find the average spectral power of the background as:

\begin{equation}
    |\hat{\eta}| = \sqrt{DC_{\hat{X}} - b + \sigma_{RMS}^2}
\end{equation}

By subtracting this background level from the OTF, we define the cutoff frequency as the first zero-crossing of the resulting OTF. Then, assuming the OTF is an ellipsoid, we can find the PSF volume and are from the resulting resolution limits.
