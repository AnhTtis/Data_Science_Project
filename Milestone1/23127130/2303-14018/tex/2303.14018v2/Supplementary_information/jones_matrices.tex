\section*{Tracing matrices}

In all the simulations in this paper, we have used the Jones calculus to trace the electric field through the system. The Jones calculus is a matrix formalism that describes the polarization of light. For our case, we have modeled the system using the following matrices:

\begin{align}
    \textbf{T}_{sys}(\theta_1,\phi,\alpha) = &\textbf{R}_{\textbf{z}}^{-1}(\phi_{\alpha}) \textbf{M}_3(\theta_3) \textbf{R}_\textbf{y}(\theta_3) \textbf{F}_\textbf{T} \textbf{R}_\textbf{y}(-\theta_2) \textbf{R}_\textbf{z}(\phi_{\alpha}) \textbf{R}_\textbf{x}(\alpha) \textbf{R}_\textbf{z}^{-1}(\phi) \overline{\textbf{M}}_2(\theta_2) \textbf{M}_1(\theta_1) \textbf{R}_\textbf{z}(\phi) \\
    \textbf{M}_i(\theta_i) = &\textbf{T}_i(\theta_i') \textbf{O}_i(\theta_i) \Gamma_i(\theta_i) A_i(\theta_i) \text{\quad , \quad}
    \overline{\textbf{M}}_i(\theta_i) = \textbf{O}_i(\theta_i) \textbf{T}_i(\theta_i') \Gamma_i(\theta_i) A_i(\theta_i)
\end{align}

All the matrices, except $\textbf{F}_\textbf{T}$, are simply rotation matrices. These matrices will map the electric field into a new coordinate system, without altering the amplitude of the field. In this section, we will describe the matrices and their role in the system.

The first matrix applied to the electric field is $\textbf{R}_\textbf{z}(\phi)$, and is defined as:

\begin{equation}
    \textbf{R}_\textbf{z}(\phi) = \begin{bmatrix}
        \cos(\phi) & \sin(\phi) & 0 \\
        -\sin(\phi) & \cos(\phi) & 0 \\
        0 & 0 & 1
    \end{bmatrix}
\end{equation}

where $\phi$ is the angle of rotation around the z-axis. This matrix is used to map the field from Cartesian coordinates into a the meridional and sagittal planes. This way, the electric field can eaily be traces through a lens using $\textbf{L}(\theta)$, which is defined as:

\begin{equation}
    \textbf{L}(\theta) = \begin{bmatrix}
        \cos(\theta) & 0 & \sin(\theta) \\
        0 & 1 & 0 \\
        -\sin(\theta) & 0 & \cos(\theta)
    \end{bmatrix}
\end{equation}

where $\theta$ is the angle of the ray relative to the optical axis. Both $\textbf{O}_(\theta)$ and $\textbf{T}_(\theta)$ are identical to $\textbf{L}(\theta)$, but distinguished for clarity to separate objectives and tube lenses. This matrix is used to simulate a lens, which is done by either collimating or focusing the electric field. The matrix $\boldsymbol{\Gamma}(\theta)$ is used to map the transmission of the lens, and is defined as:

\begin{equation}
    \boldsymbol{\Gamma}(\theta) = \begin{bmatrix}
        T_p & 0 & 0 \\
        0 & T_s & 0 \\
        0 & 0 & 1
    \end{bmatrix}
\end{equation}

where $T_p$ and $T_s$ is the transmission of the lens for p- and s-polarized light respectively. In this paper, the transmission was measured experimentally. The next element is not a matrix, but simply a scalar function. This function, $A(\theta)$, is the apodization function of the lens. This can be found using the Abbe-Sine condition, and is given by:

\begin{equation}
    A(\theta) = \sqrt{\frac{n}{\cos(\theta)}} \text{\quad\quad} A'(\theta) = \sqrt{\frac{\cos(\theta)}{n}}
\end{equation}

where $A(\theta)$ is for a collimating lens, and $A'(\theta)$ is for a focusing lens. The next matrix is $\textbf{R}_\textbf{x}(\alpha)$, which is used to rotate the electric field around the x-axis. In this paper, this matrix is used to simulate the rotation of the optical axis between O2 and O3. This matrix is given by:

\begin{equation}
    \textbf{R}_\textbf{x}(\alpha) = \begin{bmatrix}
        1 & 0 & 0 \\
        0 & \cos(\alpha) & -\sin(\alpha) \\
        0 & \sin(\alpha) & \cos(\alpha)
    \end{bmatrix}
\end{equation}

where $\alpha$ is the tilt of the optical axis. The next matrix is $\textbf{R}_\textbf{y}(\theta)$, which is used to rotate the electric field around the y-axis. This is used to map the electric field from the meridional and sagittal planes into s- and p-polarized light. This matrix is given by:

\begin{equation}
    \textbf{R}_\textbf{y}(\theta) = \begin{bmatrix}
        \cos(\theta) & 0 & -\sin(\theta) \\
        0 & 1 & 0 \\
        \sin(\theta) & 0 & \cos(\theta)
    \end{bmatrix}
\end{equation}

where $\theta$ is the angle of the rays relative to the optical axis. 

The last matrix is $\textbf{F}_\textbf{T}$, which is a transmission matrix for a refractive index change. This matrix is identical in form to $\boldsymbol{\Gamma}(\theta)$. However, the transmissions for this matrix are calculated using the Fresnel equations. For our case, where the transmission through the refractive index change was measured along with the O3 transmission, this matrix was substituded with an identity matrix of equal size.
