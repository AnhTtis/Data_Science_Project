\section*{Sampling criterion}

For the simulations to be valid, the electric field traced through the system needs to be properly sampled. For this to be true we need to sample fine enough for the phase to not change by more than $\pi$ between two pixels. For this to be true, we can use the sampling condition outlined by Leutenegger et al.:

\begin{equation}
    N_m > \frac{2\text{NA}_f^2}{\sqrt{n_f^2-\text{NA}_f^2}}\frac{|z|}{\lambda_0}
\end{equation}

where $N_m$ is the minimum needed sampling points, $n_f$ is the refractive index in the image space, $z$ is the maximum axial distance from the focal point, $\text{NA}_f$ is the numerical aperture of the focusing lens, and $\lambda_0$ is the wavelength in vacuum. In our simulations, the sampling volume is a cube with isotropic resolution in the sample space. This means our $|z|$ can be found using:

\begin{equation}
    |z| = \frac{v N_c M_a}{M_t}
\end{equation}

where $v$ is the voxel size, $N_c$ is the number of voxels in the axial direction, $M_a$ is the axial magnification of the system, and $M_t$ is the transverse magnification of the system. Assuming a silicon immersion $O_1$, dry $O_2$, and a snouty $O_3$, we get $M_t=56$ and $M_a=2240$. Using a matched 0.025NA $T_3$ we get:

\begin{equation}
    N_m \gtrsim 0.05 \frac{v}{\lambda_0} N_c
\end{equation}

We know that $N_m$ must exist on the interval $N_m\in [0,N_c]$. As we will always sample with the full array of simulated voxels, we can set $N_m$ to its maximum value $N_c$, leading to the condition $v \lesssim 20\lambda_0$. Assuming the sample should be Nyquist sampled, we also get the condition:

\begin{equation}
    v \leq n_v < \frac{\lambda_0}{2 \text{NA}_{O_1}}\frac{M_t}{2}
\end{equation}

where $n_v$ is the Nyquist voxel. Assuming a high NA $O_1$ (NA$\in(1,1.4)$), we get the condition:

\begin{equation}
    v < \frac{\lambda_0 M_t}{4 \text{NA}_{O_1}} < 10\lambda_0
\end{equation}

Assuming Nyquist sampling, we can then conclude that our sampling rate is sufficient to avoid phase changes of more than $\pi$. The voxel size can also be determined using the equation outlined in\cite{leutenegger2006fast}:

\begin{equation}
    v = \frac{N_c \lambda_0}{N_p \text{NA}_f}
\end{equation}

where $N_p$ is the total number of pixels in one axis after padding the image to avoid aliasing. Redefining $N_p/2N_c = s$ where s is the scaling factor of the padding. To avoid aliasing, the scaling factor should be at least 2\cite{leutenegger2006fast}. This results in the condition:

\begin{equation}
    v = \frac{\lambda_0}{2 s \text{NA}_f} < 10\lambda_0
\end{equation}

This means the requirement for our voxel size will always be met for a suitable scaling factor.
