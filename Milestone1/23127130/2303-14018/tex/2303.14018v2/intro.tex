\section*{Introduction}

Volumetric imaging is an important tool for studying biological phenomena, and has thus given rise to a broad set of techniques with various benefits and drawbacks\cite{fischer2011microscopy}. One of these techniques is light-sheet fluorescent microscopy (LSFM)\cite{mappes2012invention,voie1993orthogonal,stelzer1994fundamental,stelzer1995new,chen2014lattice,huisken2009selective,santi2011light}. LSFM offers a combination of high spatial- and temporal-resolution, optical sectioning, and low phototoxicity, which makes the technique highly suitable for live cell imaging. LSFM works by having two orthogonally oriented objectives with a shared foci. One objective illuminates the focus plane of the second objective with a light-sheet during imaging. This technique has given rise to several designs over the last decades \cite{voie1993orthogonal,stelzer1994fundamental,stelzer1995new,chen2014lattice,huisken2009selective}. However, because the objectives are placed orthogonal to each other while maintaining a shared foci, the combined Numerical Aperture (NA) of the two objectives is spatially limited. Additionally, the physical design of the two-objective setup generally prevents the system from utilizing conventional sample mounting. This makes the technique challenging for certain sample types like slides, dishes, and multiwell plates.

\begin{figure}[hbt!]
    \centering
    \fbox{\includegraphics[width=0.9\linewidth]{figures/setup.png}}
    \caption{Schematic setup of an OPM system. O and T represent objectives and tube lenses respectively. The lens train from O1 to O2 forms a perfect 3D imaging system, and O3 and T3 forms a tertiary tilted microscope. The sample is placed above O1, and illuminated by a light-sheet (blue) inclined by a tilt $\upalpha$. The illuminated area is relayed onto the focus of O3, and re-imaged onto the camera.}
    \label{fig:OPM_setup}
\end{figure}

\textbf{Oblique Plane Microscopy} (OPM)\cite{dunsby2008optically,bouchard2015swept,kumar2018integrated,sparks2020dual,glaser2022hybrid,yang2022daxi,sapoznik2020versatile} is a technique that uses the same objective lens for both illumination and imaging, thus overcoming the main limitations of LSFM. A schematic OPM setup is shown in Figure \ref{fig:OPM_setup}. In this setup, the light-sheet is formed by illuminating the pupil of the imaging objective off center. The center point of the pupil illumination will determine the inclination of the light-sheet, while the extent of the illumination will determine the thickness, width, and length of the focused sheet. This will generate a light-sheet at an oblique angle to the focal plane of O1. Where O represents a microscope objective, and T represents a tube lens.

Due to the oblique illumination, most of the illuminated area will be out of the traditional focus plane. To get this plane in focus, the sample volume needs to be relayed to a remote image space using a perfect 3D imaging system\cite{botcherby2008optical}, seen in Figure \ref{fig:OPM_setup} as the lens train from O1 to O2. To avoid aberrations and distortions, the magnification of a perfect imaging system should be close to unity\cite{mohanan2022sensitivity}. The remote image space is then re-imaged by a third microscope, seen in Figure \ref{fig:OPM_setup} as O3 and T3. This microscope is oriented so the remote image plane is perfectly in the traditional focus plane of O3. 

Here, we assume the effective pupil function of this system will then be limited by the overlap of the O2 and O3 pupil. This effective pupil will limit the achievable resolution of the system. To maximize the effective pupil, a high NA O3 can be used. However, due to the tilted geometry of the tertiary microscope, an immersion medium is needed to fit a high NA O3. A specialized glass immersion objective (P/N: AMS-AGY), often called “snouty”\cite{alfred_millett_sikking_2019}, has been created for this purpose, although water immersion objectives using glass cover slips have also been used\cite{yang2019epi}. This change in RI from O2 to O3 can make the light transmission highly polarization dependent. This dependency will be especially noticeable for samples with high fluorescent anisotropy\cite{lakowicz2006principles}. 

Fluorescent anisotropy is a measure of how much of the excitation polarization is preserved in the emission. In this paper we model fluorophores as ensembles of rotating dipoles. If a dipole has low rotational diffusion, e.g. from a chemical bond or high mass, the anisotropy of the dipole is increased. In this paper, we explore the extreme case of low rotational diffusion, corresponding to maximum retention of the excitation polarization.