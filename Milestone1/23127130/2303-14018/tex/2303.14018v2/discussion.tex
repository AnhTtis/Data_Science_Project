\section*{Discussion}

In the results shown in Figure \ref{fig:resolution_plot}, we see an increase in PSF area with increased angle. This is expected, as the overlap of the O2 and O3 acceptance cone will decrease with increased angle. This results in a decrease in the effective aperture of the system, thus increasing the PSF area. The PSF volume, however, is seen to decrease with increased angle. As the light-sheet inclination increases, the aperture available for making the light-sheet increases, thinning the light-sheet, and thus decreasing the effective PSF volume. However, a system can use a fixed light-sheet thickness for any inclination. In this case, the length of the light-sheet will be contant. This allows for imaging deeper into a sample, but at the cost of volumetric resolution. The light-sheet volume will then follow the trend of the simulated PSF area.

% If, however, a system used a fixed light-sheet thickness for any inclination, the PSF volume will follow the trend of the PSF area. Such a configuration would trade volumetric resolution for field of view. This trend is seen for both p- and s-polarized light-sheets.

The effect of light-sheet polarization is seen in Figure \ref{fig:OTF_CS}. Here, we see that p-polarized light-sheets give rise to side lobes of low power in the OTF. These side lobes are likely a result of a sign change in the OTF and are prone to blend into the background more quickly. This effect is not present for s-polarized light-sheets, making the higher frequency components of the OTF more visible for s-polarized light-sheets. Also found in Figure \ref{fig:OTF_CS} are artifacts from the Fourier transform along the kz axis (seen in panels c) and d)). These artifacts are a result of PSF truncation, and can be minimized by increasing the sampling volume.

The results shown in Figure \ref{fig:resolution_plot} are simulated using an AMS-AGY v1.0 style O3. As seen in Figure \ref{fig:objective_transmission}, the light throughput of this objective is minimally affected by the polarization, which is not the case for a water immersion O3 with coverslip. Note that an anti-reflection coating could be applied to a coverslip to provide similar resilience to polarization and improved transmission performance. In this case, the light throughput will be highly dependent on the polarization of the light-sheet. This will likely alter the results shown in Figure \ref{fig:resolution_plot}. 

Further, the immersion medium of O3 will affect the light transmission of the system, as seen in Figure \ref{fig:light_efficiency}. Here we see that s-polarized light-sheets have a higher optical efficiency than p-polarized light-sheets. This effect is illustrated in panel b) of Figure \ref{fig:light_efficiency}. As an s-polarized light-sheet is the most effective at exciting a dipole oriented along the y-axis, the collection efficiency of the emission will be higher for s-polarized light-sheets. However, due to better transmission of p-polarized light after O1, as seen in Figure \ref{fig:objective_transmission}, this gap in optical efficiency might decrease for certain system configurations. This increased transmission is not, however, sufficient to overcome the increased collection efficiency of s-polarized light-sheets. This means that s-polarized light-sheets will almost always have a higher optical efficiency than p-polarized light-sheets. This effect has only been simulated for a non-scattering medium. In scattering samples, other effects might come into play. For instance, p-polarized light is more likely to be scattered in the plane of the light-sheet, thus reducing unwanted background light excitation in dense samples.

The results also show that a AMS-AGY v1.0 style O3 gives lower light-loss than the Fresnel reflections caused by a water immersion O3 using a glass coverslip. This can be explained by the anti-reflection coating on the AMS-AGY v1.0. Further, the light transmission of the AMS-AGY v1.0 has been measured experimentally, while the water immersion O3 has only been simulated using Fresnel transmissions. This means the transmission loss will likely be higher for the water immersion O3 than shown in Figure \ref{fig:light_efficiency}, which would further increase the gap in optical efficiency between the two objectives. If no anti-reflection coated coverslip is used with a water-immersion lens, we conclude that the AMS-AGY O3 is the better choice for most OPM systems