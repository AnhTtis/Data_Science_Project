\section*{Results}

\begin{figure}[hbt!]
  \centering
  \fbox{\includegraphics[width=0.9\linewidth]{figures/PSF_volume_new.png}}
  \caption{Panels a) and b) shows the PSF volume and area respectively for all system configurations using an AMS-AGY v1.0 O3. The blue and red bars represent p- and s-polarized light-sheets respectively. The left axis represents a range of light-sheet inclinations (angle from focus plane to light-sheet plane), while the right axis represents the photon count at the brightest pixel. The PSF area and volume were calculated assuming an elipsoidal PSF. All results are derived from ten simulations with identical system configuration and an ensemble of 100 dipoles.}
  \label{fig:resolution_plot}
\end{figure}

\begin{table}[hbt!]
  \centering
  \caption{\bf Resolution limits, sectioning, and light-sheet length for all tested system configurations}
  \begin{tabular}{cccccccc}
    \hline
    & \multicolumn{2}{c}{x {[}nm{]}} & \multicolumn{2}{c}{y {[}nm{]}} & z {[}nm{]} & S {[}nm{]} & $\mathrm{L}_{ls}$ [\SI{}{\um}] \\
    $\upalpha$ & p & s & p & s & p/s & p/s &  \\
    \hline
    20$^\circ$ & 188 & 185 & 216 & 205 & 571 & 1714 & 57.3  \\
    25$^\circ$ & 185 & 182 & 212 & 209 & 457 & 980  & 14.4  \\
    30$^\circ$ & 185 & 187 & 222 & 214 & 403 & 623  & 6.5 \\
    35$^\circ$ & 189 & 186 & 223 & 219 & 343 & 490  & 3.7 \\
    40$^\circ$ & 198 & 188 & 235 & 231 & 286 & 403  & 2.4 \\
    \hline
  \end{tabular}
  \label{tab:resolution}
\end{table}

\begin{figure}[htb!]
  \centering
  \fbox{\includegraphics[width=0.9\linewidth]{figures/OTF_CS.png}}
  \caption{The OTF of a system configuration using 30° light-sheet inclination, 1.25NA water immersion O1, 0.95NA dry O2, 1.0 NA glass immersion O3, and SNR=100. Panels a), c), and e) show the OTF for an s-polarized light-sheet, while panels b), d), and f) show the OTF for a p-polarized light-sheet. Panels a) and b) show the XY cross section of the OTF, c) and d) show the YZ cross section, and e) and f) show a 3D rendering of the OTF.}
  \label{fig:OTF_CS}
\end{figure}

\begin{figure}[hbt!]
  \centering
  \fbox{\includegraphics[width=0.9\linewidth]{figures/optical_efficiency.pdf}}
  \caption{Optical efficiency of OPM. a) shows the amount of light emitted by the fluorophore that is transmitted through the system, normalized to the maximum transmission. The black lines represent a 1NA dry O3 (physically unrealized, but simulated to remove Fresnel reflections). The blue lines represent an AMS-AGY v1.0 style O3. The yellow lines represent water immersion O3 using a 1.5 RI cover slip. The green lines show the transmission of a collimated laser transmitted through O2 and O3, measured experimentally. The dotted lines represent p-polarized input light, and solid lines represent s-polarized light. b) shows pictogram representing the excitation and emission of the dipole emitters. The light-sheet (blue) can be p-polarized or s-polarized, indicated by the axes drawn on the bottom left and top right of the light-sheet respectively. The green shapes above the matrix represent the emission pattern of a dipole oriented along the z-, x- and y-axis going from left to right. The acceptance cone of the objective is indicated by the dark-shaded area. In the matrix, each cell consist of two elements. On the left of the cell is the excitation efficiency of the light-sheet on the dipole, and on the right is the collection efficiency of the dipole emission. A red cross represent low efficiency, and a green check mark represent high efficiency.}
  \label{fig:light_efficiency}
\end{figure}

Figure \ref{fig:resolution_plot} shows the PSF volume and is using a 1.35NA silicone immersion O1, 0.95NA dry O2, and a 1NA glass immersion O3. The results use an ensemble of 100 dipoles to make the effective PSF. To minimize the effects of the random noise added to the PSF, the OTF was calculated for ten noise samples drawn from the same distribution. The resolution limits were then found for all ten noise samples, and averaged to find a single value. The axis specific resolution limits, as well as the sectioning thickness and length of the light-sheet, are given in Table \ref{tab:resolution}. This table shows the resolution limits for the highest simulated SNR value (using $10^4$ photons at the brightest pixel). The results show an increase in PSF area and a decrease in PSF volume when increasing the light-sheet inclination. The results also show an effect on both PSF area and volume with respect to photon count. Depending on the system configuration and SNR, s-polarized light can give anything from 2\% to 30\% decreased PSF area and volume.

Figure \ref{fig:OTF_CS} shows an OTF for a system configuration using a 30° light-sheet inclination, 1.25NA water immersion O1, 0.95NA dry O2, AMS-AGY v1.0 O3, and SNR=100. The left panels (a, c, and e) show the OTF for an s-polarized light-sheet, while the right panels (b, d, and f) show the OTF for a p-polarized light-sheet. From the images, we can see a ring feature in the OTF for a p-polarized light-sheet that is not present in the s-polarized case.

Figure \ref{fig:light_efficiency}, panel a) shows the optical efficiency for different system configurations. All system configurations use a 1.35NA silicone immersion O1, 0.95NA dry O2, and a 1NA O3. The black lines represent a system with a dry O3. Albeit a 1.0 NA dry objective does not exist, it is included to show the theoretically optimal case. The yellow lines represent a water immersion O3 with a 1.5 RI glass coverslip to hold in the water\cite{yang2019epi}. The blue lines represent an AMS-AGY v1.0 O3. From this, we see that an AMS-AGY v1.0 objective gives higher optical efficiency than its water immersion counterpart thanks to its anti-reflection coating. We also find that s-polarized light-sheets have a higher optical efficiency than p-polarized light for most light-sheet inclinations. Only at extreme tilt angles do p-polarized light-sheets outperform s-polarized illumination in terms of light-efficiency.

Panel b) in Figure \ref{fig:light_efficiency} shows a pictogram describing the light-sheet excitation and dipole emission. The light-sheet (blue) can be p-polarized or s-polarized, indicated by the axes drawn on the light-sheet. The green shapes above the matrix represent the emission pattern of a dipole in various orientations. The excitation efficiency of the light-sheet on the dipole is given as a inner product of the polarization of the dipole and light-sheet. The collection efficiency of the dipole emission is given as the overlap between the emission spectrum and the acceptance cone of the objective lens. For a system configuration to be light efficient, both the excitation and collection efficiency must be good. As shown in the figure, this is only achieved for s-polarized light-sheets, with a dipole oriented along the y-axis.
