\section*{Methods}

In order to investigate the impact of polarization on the system, a simulation software was developed to model the vectorial diffraction of the system as outlined by Kim, et al\cite{kim2018calculation}. In the simulations, an emission source was modelled as a point-like dipole and placed in the focal point of O1. This was done to immitate the emission pattern of a small fluorophore. The electric field of the dipole emission was then characterized at discreet points within the collection cone of O1. In order to trace the fields through the optical system, each optical component was modelled as a 3$\times$3 Jones matrix. To find the complete transformation of the system, the matrices were then multiplied to find a system transformation. For our case, the system model can be described as:

\begin{align}
    \textbf{T}_{sys}(\theta_1,\phi,\alpha) = &\textbf{R}_{\textbf{z}}^{-1}(\phi_{\alpha}) \textbf{M}_3(\theta_3) \textbf{R}_\textbf{y}(\theta_3) \textbf{F}_\textbf{T} \textbf{R}_\textbf{y}(-\theta_2) \textbf{R}_\textbf{z}(\phi_{\alpha}) \textbf{R}_\textbf{x}(\alpha) \textbf{R}_\textbf{z}^{-1}(\phi) \overline{\textbf{M}}_2(\theta_2) \textbf{M}_1(\theta_1) \textbf{R}_\textbf{z}(\phi) \underline{E}_0 \\
    \textbf{M}_i(\theta_i) = &\textbf{T}_i(\theta_i') \textbf{O}_i(\theta_i) \boldsymbol{\Gamma}_i(\theta_i) A_i(\theta_i) \text{\quad , \quad}
    \overline{\textbf{M}}_i(\theta_i) = \textbf{O}_i(\theta_i) \textbf{T}_i(\theta_i') \boldsymbol{\Gamma}_i(\theta_i) A_i(\theta_i)
\end{align}

where $\textbf{R}_\textbf{i}$ denotes the rotation matrix around axis \textbf{i}, \textbf{O} and \textbf{T} are both lens matrices but separated to distinguish between tube lenses and objectives, $\boldsymbol{\Gamma}_i$ is the light transmission of the objective at $\theta_i$, $A_i$ is the apodization of the microscope at $\theta_i$, $\textbf{F}_\textbf{T}$ is the Fresnel transmission of the refractive index (RI) change, and $\underline{E}_0$ is the electric field of the dipole emission. The Fresnel transmission is only applied if the transmission to a new immersion medium is not altered by a anti-reflection coating. A full description of the matrices can be found in the supplementary information.

For the glass immersion objective (AMS-AGY v1.0), and the dry objective used for O2, the transmission was measured experimentally using a single-axis stage and a power meter. A laser, beam expander, tube lens, polarizer, and half-wave plate was built into a single unit and mounted on a translation stage. The laser beam was expanded to fill a tube lens, then focused onto the back focal plane (BFP) of the objective. By moving the focus of the beam radially from the center of the BFP, the output angle of the beam was changed. The power of the beam was then measured at a range of angles for both p- and s-polarized light. The results were then curve fitted to 5th degree polynomials. This polynomial can be assumed to be a taylor approximation of the true transmission function. The function was then used to make a transmission mask for the corresponding objective. The data used for this mask is found in Figure \ref{fig:objective_transmission}.

\begin{figure}[ht!]
    \centering
    \fbox{\includegraphics[width=0.9\linewidth]{figures/transmission_function.png}}
    \caption{The top panel shows the light transmission for a 0.95NA dry objective (CFI Plan Apochromat Lambda D 40X), while the bottom panel shows the light transmission for a 1NA glass immersion objective (AMS-AGY v1.0). The blue and red lines indicate p- and s-polarized light respectively. The transmission of the objectives was normalized to the maximum value and curve fitted using 5th degree polynomials.}
    \label{fig:objective_transmission}
    \end{figure}

Further, for every microscope the lenses are assumed to share an identical BFP. Assuming all lenses follow the abbe sine condition, the angles of the simulated rays can be related using\cite{gu2000advanced}:

\begin{equation}
    \theta_i' = \arcsin\left(\frac{\text{NA}'}{\text{NA}}n\sin(\theta_i)\right)
\end{equation}

where $\text{NA}$ is the Numerical Aperture of the objective, $\text{NA}'$ is the Numerical Aperture of the tube lens, $\theta_i$ is the angles of rays at the objective, $\theta_i'$ is the anlges of the rays at the tube lens, and $n$ is the refractive index of the objective immersion medium.

For two lenses sharing an image space, the lenses are assumed to share an identical focal point. To get the input angles of the collimating lens, the output angles of the focusing lens is altered using either Snell's law for a immersion medium change, or a change in the optical axis for the transition from O2 to O3. 

To ensure the validity of the model, some assumptions were made for the system: the dipole emitter is located perfectly in the focal plane of the first objective, the lenses are rotationally symmetric and centered on the optical axis, and there are no spherical or chromatic aberrations.

When the electric field has been traced through the system, we end up with the final electric field converging towards the focal point. This electric field can then be evaluated using a version of the Debye integral\cite{debye1909behavior}. Owing to the computational inefficiency of numerical integration, the integral was evaluated using a Fourier transform version of the Debye integral. The version of the Debye integral used here was based on the method outlined by Leutenegger et al.\cite{leutenegger2006fast}, then modified and implemented by James Manton\cite{Manton2022debye}. To get sufficient sampling of our signal, the pixel size, $v$, of our camera needs to be sufficiently small. To fulfill this condition, Nyquist sampling is normally sufficient. For an in depth derivation of this condition, see the supplementary information.

% To fulfill this condition, using 56$\times$ magnification and a 1.35NA O1, the pixel size is limited by:

% \begin{equation}
%     v < 10\lambda_0
% \end{equation}

% where $\lambda_0$ is the emission wavelength of the fluorescence. For a full derivation of this condition, see the supplementary information.

Having set a suitable pixel size, we need to find a sampling volume that is large enough to avoid significant PSF truncation. As the maximum extent of the PSF is in the axial direction, we can find the sampling volume by the axial extent of the PSF. Assuming the PSF can be modeled as an Airy disc, the PSF will extent until the signal is below the noise floor. Although the mathematical formulation of the Airy disc can be used to determine this, it is more practical to test different sampling volumes until the results converge. For our simulations, a sampling size of 240$\upmu$m in each axis was found to be mostly sufficient.

% Having set a suitable pixel size, we need to find a sampling volume that is large enough to avoid significant PSF truncation. As the maximum extent of the PSF is in the axial direction, we can approximate the volume using a gaussian beam waist. As the beam waist is coupled to the confocal parameter, we can approximate the maximum light-sheet thickness using a confocal parameter equal to the FoV of an experimental setup. Using the afformentioned magnification, and realistic pixel size and camera pixel count of $v_e=6.5\upmu$m and $N_e=2048$, we get $\text{FoV}\approx240\upmu$m. Thus, we can approximate the maximum light-sheet thickness using:

% \begin{equation}
%     \omega_0 = \sqrt{\frac{\text{FoV} \lambda_0}{2\pi}} \approx 4.3\upmu\text{m}
% \end{equation}

% where $\omega_0$ is the beam waist of a Gaussian beam. To find this beam waist in number of voxels, we can transform it using:

% \begin{equation}
%     N_c = \frac{\omega_0 M_a}{v} \approx \frac{240\upmu\text{m}}{v} > \frac{24\upmu\text{m}}{\lambda_0}
% \end{equation}

Now we have set constrains on the pixel size and count. The remaining simulation parameters are only constrained by the specific setup, and can thus be freely chosen without impacting the validity of the simulation. The remaining simulation parameters are: the light-sheet inclination and polarization, the fluorescent anisotropy and brightness of the sample, the excitation and emission wavelength, the lenses used, and the size of the dipole ensemble used for the PSF calculation. The light-sheet thickness is determined by its opening angle. As we are, in this paper, interested in the limits of each system configuration, we assume the light-sheet is made using the maximum available aperture. The parameters used for the simulations will be given in the results section, alongside the corresponding result.

To predict the Point Spread Function (PSF) that will be read from a camera, the noise-less PSF was normalized to achieve the desired photon count. Photon shot noise was then added by passing the voxels through a Poisson distribution, followed by the addition of Gaussian noise to simulate camera readout noise. Once noise was added, a bias offset was included and the voxels were discretized.

To determine the Optical Transfer Function (OTF), the PSF was Fourier transformed. The OTF contains the resolvable frequencies of an image, and by identifying the maximum extent of the OTF in each axis, the corresponding resolution limit can be defined. Assuming a PSF with shot noise, Gaussian readout noise, and a bias offset, the background of the OTF can be approximated using:

%When the PSF was calculated, the results are modified to include noise. The final PSF was then normalized to the desired photon count to simulate the desired signal to noise ratio (SNR). The pixels were then passed through a Poisson distribution to simulate photon shot noise. After adding shot noise, a layer of Gaussian background noise was added to simulate camera readout noise. After noise was added, the pixels were discretized and a bias offset was added, the PSF was then saved as a 3D TIFF file. The PSF was then Fourier transformed to find the Optical Transfer Function (OTF). An OTF contains the resolvable frequencies of a system. By finding the maximum extent of the OTF in each axis, the corresponding resolution limit can be defined. 

%To find the extent of the OTF, we need to estimate the background of the Fourier transform. The derivation of this background is given in the supplementary information. Assuming a PSF with shot noise, Gaussian readout noise, and a bias offset, the background of the OTF can be approximated using:

\begin{equation}
    |\hat{\eta}_X| = \sqrt{DC_{\hat{X}} + \sigma_{RMS}^2 - b}
\end{equation}

where $|\hat{\eta}_X|$ is the average background power of the Fourier transform, $DC_{\hat{X}}$ is the DC component of the Fourier transform, $\sigma_{RMS}$ is the RMS of the readout noise, and $b$ is the bias offset. For a rigorous derivation of this background, see the supplementary information.

Subtracting this background from the OTF, the cutoff frequency is given by the first zero crossing of the modified OTF. The resolution limit is then found using the cutoff frequency and the base frequency of the Fourier transform. However, in $\hat{z}$-direction the OTF will have two different cutoff frequencies, one along the $\hat{z}$-axis and a second for the maximum extent in the $\hat{z}$ direction. The cutoff along the $\hat{z}$-axis will give a sectioning thickness, while the maximum extent will give the axial resolution limit. Then, assuming the PSF is an ellipsoid, we can find the in-focus area and volume of the PSF. For this paper, we will use the sectioning thickness to calculate the PSF volume.

