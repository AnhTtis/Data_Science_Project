\documentclass[9pt,twocolumn]{article}

\usepackage{graphicx} % Required for inserting images
\usepackage{lineno}
\usepackage{siunitx}
\usepackage{upgreek}
\usepackage{varwidth}
\usepackage[utf8]{inputenc}	
\usepackage{amsmath,amsfonts,amssymb,amsthm,mathtools} % Most people will need these for mathematics

%Pakker for figurer
\usepackage{graphicx}	%Graphics package
%\usepackage[center]{caption}
\usepackage{subcaption}
\usepackage{wrapfig}  % Wrap text around figures	
\usepackage{float} % To place figures correctly

\usepackage{makeidx} % Probably important.
\usepackage{listings} % To include code-snippets
\usepackage{color}
\usepackage{hyperref} % Clickable links to urls, internal and external references etc.
\usepackage[
backend=biber,
style=ieee,
]{biblatex}
\addbibresource{sample.bib}

\usepackage{siunitx}  %Write units easily \SI{value}

\lstset{
basicstyle=\footnotesize\ttfamily,
identifierstyle=\bfseries\color{green!40!black},
commentstyle=\itshape\color{purple!40!black},
keywordstyle=\color{blue},
stringstyle=\color{orange},
}

\usepackage{pdfpages}	% Packages to handle certain issues with the pdf conversion and landscape figures/tables
\usepackage{lscape} 

\usepackage{footnote}
\usepackage{appendix} % Appendix + add to table of contents
\usepackage{sidecap}

% Table packages
\usepackage{tabularx}
\makesavenoteenv{tabular}
\makesavenoteenv{table}
\usepackage{array}
\usepackage{booktabs}
\newcommand*\rot{\rotatebox{90}}

%Special in-document vector graphics:
\usepackage{tikz}
\usetikzlibrary{shapes.geometric,calc,positioning,3d,intersections,arrows}
\usetikzlibrary{decorations.markings}
\usepackage{pgfplots}
\pgfplotsset{compat=1.14}
\usepackage{circuitikz}  %Brukes til å enkelt tegne kretser

\usepackage{titling}
\setlength{\droptitle}{-8ex}
\pretitle{\begin{flushleft}\Large\bfseries}
\posttitle{\par\end{flushleft}}
\preauthor{\begin{flushleft}\Large}
\postauthor{\end{flushleft}}
\predate{\begin{flushleft}}
\postdate{\end{flushleft}}


\title{S-polarized light-sheets improve resolution and light-efficiency in oblique plane microscopy}
\author{%
	\large
	\textsc{Jon-Richard Sommernes$^{1}$, Alfred Millett-Sikking$^{2}$, Florian Ströhl$^{1,*}$}\\
	\normalsize	$^1$Department of Physics and Technology, UiT – The Arctic University of Norway, Tromsø, Norway \\
    \normalsize	$^2$Calico Life Sciences LLC, South San Francisco, CA, USA \\
    \normalsize $^*$Corresponding author: florian.strohl@uit.no
	}

\begin{document}

\maketitle

 \noindent\textbf{Oblique plane microscopy (OPM) offers 3D optically sectioned imaging with high spatial- and temporal-resolution while enabling conventional sample mounting. The technique uses two microscopes for remote focusing and a tilted tertiary microscope, usually including an immersion objective, to image an oblique sample plane. This design induces Fresnel reflections and a reduced effective aperture, thus impacting the resolution and light efficiency of the system. Using vectorial diffraction simulations, the system performance was characterized based on illumination angle and polarization, signal to noise ratio, and refractive index of the tertiary objective immersion. We show that for samples with high fluorescent anisotropy, s-polarized light-sheets yield higher average resolution for most system configurations, as well as higher light-efficiency. We also provide a tool for performance characterization of arbitrary light-sheet imaging systems.}
 \newline


\textbf{Volumetric imaging} is an important tool for studying biological phenomena, and has thus given rise to a broad set of techniques with various benefits and drawbacks\cite{fischer2011microscopy}. One of these techniques is light-sheet microscopy (LSFM)\cite{mappes2012invention,voie1993orthogonal,stelzer1994fundamental,stelzer1995new,chen2014lattice,huisken2009selective,santi2011light}. LSFM offers a combination of high spatial- and temporal-resolution, optical sectioning, and low phototoxicity, which makes the technique highly suitable for live cell imaging. LSFM works by having two orthogonally oriented objectives with identical foci. One objective illuminates the focus plane of the second objective with a light-sheet during imaging. This technique has given rise to several designs over the last decades \cite{voie1993orthogonal,stelzer1994fundamental,stelzer1995new,chen2014lattice,huisken2009selective}. However, because the objectives are placed orthogonal to each other while maintaining identical foci, the combined aperture of the two objectives is spatially limited. Additionally, the physical design of the two-objective setup generally prevents the system from utilizing conventional sample mounting. This makes the technique challenging for certain sample types.

\begin{figure}[ht!]
\centering
\fbox{\includegraphics[width=0.95\linewidth]{setup.png}}
\caption{Schematic setup of an OPM system. O and T represent objectives and tube lenses respectively. The lens train from O1 to O2 forms a perfect imaging system, and O3 and T3 forms a tertiary tilted microscope. The sample is placed over O1, and illuminated by a light-sheet (blue) inclined by a tilt $\upalpha$. The illuminated area is relayed onto the focus of O3, and re-imaged onto the camera.}
\label{fig:OPM_setup}
\end{figure}

\textbf{Oblique Plane Microscpy} (OPM)\cite{dunsby2008optically,bouchard2015swept,kumar2018integrated,sparks2020dual,glaser2022hybrid,yang2022daxi,sapoznik2020versatile} is a technique that uses the same objective lens for both illumination and imaging, thus overcoming the main limitations of LSFM. A schematic OPM setup is shown in figure \ref{fig:OPM_setup}. In this setup, the light-sheet is formed by illuminating the pupil of the imaging objective with a line of collimated light. The center point of the light will determine the inclination of the light-sheet and the length of the line will determine the thickness and length of the focused sheet. This will generate a light-sheet at an oblique angle to the focal plane of O1.

Due to this oblique illumination, most of the illuminated area will be out of focus. To get this plane in focus, the sample volume needs to be relayed to a remote image space using a perfect imaging system\cite{botcherby2008optical}, seen in figure \ref{fig:OPM_setup} as the lens train from O1 to O2. The magnification of a perfect imaging system is limited close to unity. The remote image space is therefore re-imaged by a third microscope, seen in figure \ref{fig:OPM_setup} as O3 and T3. This microscope is oriented so the remote image plane is perfectly in the focus plane of O3. 

The effective pupil function of this system will then be limited by the overlap of the O2 and O3 pupil. This effective pupil will limit the achievable resolution of the system. To maximize the effective pupil, a high NA O3 can be used. However, due to the tilted geometry of the tertiary microscope, an immersion medium is needed to fit a high NA O3. A specialized glass immersion objective, often called “snouty”\cite{alfred_millett_sikking_2019}, has been created for this purpose, although water immersion objectives using glass cover slips has also been used\cite{yang2019epi}. This induces a RI change between O2 and O3, making the light transmission highly polarization dependent. This dependency is especially noticeable for samples with high fluorescent anisotropy. 

Fluorescent anisotropy is a measure of how much of the excitation polarization is preserved in the emission. In this paper we model fluorophores as ensembles of rotating dipoles. If a dipole has low rotational diffusion, e.g. from a chemical bond or high mass, the anisotropy of the dipole is increased. In this paper, we explore the two extreme cases of high rotational diffusion and low rotational diffusion, corresponding to no retention and maximum retention of excitation polarization respectively.

\textbf{PSF calculation}. To trace the emission of the dipole while preserving the polarization information, an OPM system was modelled using vectorial raytracing with non-paraxial 3x3 Jones matrices\cite{kim2018calculation}. To find the PSF, the field was then evaluated using vectorial Debye diffraction integral\cite{leutenegger2006fast}. Evaluating the field at different offsets from the focus, we found the 3D PSF. This was done for an ensemble of 100 dipoles in each simulated system configuration to find the system performance. Using the traced field, we calculated the power of the field before and after the trace, giving us the relative power-loss of the system.

\begin{figure}[ht!]
\centering
\fbox{\includegraphics[width=0.95\linewidth]{analysis.png}}
\caption{a) shows the histogram of a full 3D OTF. b) shows a XZ cross section of the same OTF. The black vertical line in the histogram shows the cutoff value separating signal and noise. The red lines in the OTF cross section shows where the pixel intensity crosses the noise threshold.}
\label{fig:analysis}
\end{figure}

\begin{figure*}[ht!]
\centering
\fbox{\includegraphics[width=\linewidth]{resolution_plot.png}}
\caption{The measured PSF size for all system configurations using an O3 of 1 NA. The red bars show the in focus PSF area, and the blue bars show the PSF volume. The standard deviation of the results are represented by black error bars. The columns show SNR of 2, 5, 20, and 100 going from left to right. Within each box is four clusters of three bars. The clusters show the results for a light-sheet tilt of 25, 30, 35, and 40 degrees, while each cluster contains the result for a p- and s-polarized light-sheet, along with the case of zero fluorescent anisotropy (here referred to as u-polarization).}
\label{fig:resolution_plot}
\end{figure*}

Using the vectorial tracing, the light-sheet was also calculated. The effective PSF of the system is given as a product of the dipole PSF and the light-sheet. For the purposes of this study, the light-sheet was made as thin as possible. As the light-sheet inclination increases, more of the pupil is available for illumination. This means the light-sheet will become shorter and thinner at higher inclinations. 

\textbf{Analysis}. To find the resolution limit of the system, we first transformed the effective PSF of the system into a OTF. The OTF contains the resolvable frequencies of the system. By finding the maximum extent of the OTF in each axis, we determined the corresponding resolution limit. The maximum extent of the OTF was found by calculating the transition from signal to background as shown in figure \ref{fig:analysis}. Assuming the signal and background are given by separate distributions, a threshold value was found to separate them. It was found experimentally that the pixel value corresponding to the highest gradient of the logarithm of the histogram is a reliable threshold value. This value is shown as the black line in figure \ref{fig:analysis} a), and the cutoff is shown in figure \ref{fig:analysis} b). For noise-less data, this results in the resolution limits given in table \ref{tab:resolution}.

\begin{table}[htbp!]
\centering
\caption{\bf Resolution limit and light-sheet length for illumination configurations}
\begin{tabular}{cccccccc}
\hline
 & \multicolumn{2}{c}{x {[}nm{]}} & \multicolumn{2}{c}{y {[}nm{]}} & \multicolumn{2}{c}{z {[}nm{]}} & $\mathrm{L}_{ls}$ [\SI{}{\um}] \\
$\upalpha$ & p & s & p & s & p & s &  \\
 \hline
25$^\circ$ & 201 & 211 & 219 & 221 & 747 & 751 & 97.8  \\
30$^\circ$ & 201 & 211 & 227 & 229 & 602 & 599 & 24.6 \\
35$^\circ$ & 204 & 214 & 236 & 237 & 499 & 489 & 11.0 \\
40$^\circ$ & 216 & 214 & 250 & 249 & 428 & 428 & 6.2 \\
\hline
\end{tabular}
  \label{tab:resolution}
\end{table}

To find the effect of the signal to noise ratio, Poisson noise was added to the PSF before calculating the OTF. This was done ten times for each SNR value to find the variability of the resolution when adding noise. The resolution limits was then used to represent the PSF area and volume. The product of the resolution limit in x and y will be referred to as the PSF area, and the product of all resolution limits will be the PSF volume. The PSF area and volume for all system configurations is presented in figure \ref{fig:resolution_plot}.

We find an increase in PSF area and a decrease in PSF volume for higher light-sheet inclinations. This trend stems from the inverse proportionality of the effective aperture in the transition from O2 to O3 to the light-sheet inclination. Regardless of light-sheet polarisation or fluorophore anisotropy, increasing the light-sheet inclination also increases the aperture available for creating the light-sheet, resulting in a thinner beam, thus decreasing PSF volume. 

The effects of the light-sheet polarization is seen in figure \ref{fig:OTF_CS}. Here we see the p-polarized light-sheets give rise to side lobes of low power in the OTF. These side lobes are a result of a sign change in the OTF. This effect is not present for s-polarized light-sheets. These low-intensity side lobes will blend into the background more quickly. This makes the OTF for s-polarized light-sheets more stable for low SNR cases, and thus better suited for light sensitive samples. 

Fresnel transmission for p and s is heavily RI dependent, yet we find close to no impact on the achievable resolution for different front-lens materials of O3 when maintaining identical NA. However, the RI will affect the light transmission of the system, as seen in figure \ref{fig:light_efficiency}.

\begin{figure}[htb!]
\centering
\fbox{\includegraphics[width=0.95\linewidth]{OTF_CS.png}}
\caption{The OTF of a system configuration using 30° light-sheet inclination, 1.0 NA O3, and SNR=100. Panels a), c), and e) show the OTF for an s-polarized light-sheet, and panels b), d), and f) show the OTF for a p-polarized light-sheet. Panels a) and b) show the XY cross section of the OTF, c) and d) show the YZ cross section, and e) and f) show a 3D rendering of the OTF.}
\label{fig:OTF_CS}
\end{figure}


In figure \ref{fig:light_efficiency} a), the black lines represent a system with no Fresnel reflections. This means all light loss is due to the reduction of the effective pupil of the system. The yellow and blue lines represent water immersion O3 with and without a glass cover slip respectively, and the red lines represent a 1.7 RI glass immersion O3. From this, we can see the immersion of O3 can reduce the light transmission up to 32\%. We also find that s-polarized light-sheets have a higher optical efficiency than p-polarized light for most light-sheet inclinations. For a light-sensitive sample, this can then give rise to higher power, and better SNR. Note that in highly scattering samples, p-polarized light might give better sectioning as it will scatter more in the plane of the light-sheet.


In figure \ref{fig:light_efficiency} b) experimental and simulated data for glass immersion O3 light transmission is given. The simulations show both an uncoated interface and a single layer index matching coating. In this experiment, O1 was removed and a collimated beam was given as input along the optical axis. From this data we see two separate maxima for the simulations and experiments. While the simulated results have maximum throughput at 0$^\circ$ light-sheet inclination, the experiments have maximum throughput around 20$^\circ$. The exact coating of this objective (AMS-AGY v2, ASI) is undisclosed but was optimized for 20 degree incidence angle by design (therefore presumably a multi-layer interference coating).

The results also show that a lower front lens RI of O3 immersion O3 will give better optical efficiency than the currently available 1.7 RI glass immersion objective, even for high-transmission water-immersion objectives using a glass cover-slip to hold the water in place. This transmission gain can be further increased using a low RI polymer cover slip in combination with a O2 with no cover slip requirement. However, the high RI glass used in "snouty" allows for equal NA to water immersion using a lower collection cone. This allows the design of the objective to be simplified, requiring fewer optical interfaces. Therefore, in a practical scenario, the inherently high transmission due to fewer optical surfaces inside a solid immersion objective and its simpler alignment might yet render it preferable over a water-immersion lens.

From the results we have shown regarding light-sheet polarisation in OPM systems, we find that s-polarized light-sheets offer a more stable OTF, resulting in better resolution at lower SNR for samples with high fluorescent anisotropy. We have also shown s-polarized excitation light to give an increased optical efficiency over a range of light-sheet inclinations, making s-polarized light-sheets yield more consistent

\clearpage


\begin{figure}[H]
\centering
\fbox{\includegraphics[width=0.95\linewidth]{optical_efficiency.png}}
\caption{Optical efficiency of OPM. a) amount of excitation power detected by the camera, normalized to the theoretical maximum shown by the black lines. The blue lines represent a water immersion O3 without a glass cover slip, the yellow lines represent water immersion O3 using a 1.5 RI cover slip, and the red lines represent a 1.7 RI glass immersion O3. b) shows the transmitted power of an expanded laser beam transmitted through O2 and O3. The gray lines show experimental data, the red lines show simulated data on a glass immersion O3 using no anti-reflection coatings, and the magenta lines show simulated data using a single layer index matching coating. In both plots, the dotted lines represent p-polarized input light, and solid lines represent s-polarized light. c) shows a pictogram representing the excitation and emission of the dipole emitters. The light-sheet (blue) can be p-polarized or s-polarized, indicated by the axes on the bottom left and top right of the light-sheet respectively. The green shapes in the matrix represent the emission pattern of a dipole oriented along the z-, x- and y-axis going from left to right. Light-collection is indicated by the dark-shaded area. The excitation efficiency of the light-sheet on the dipole is given as a inner product of the dipole and light-sheet polarisation. The excitation and collection efficiency of the dipole is then dependent on the light-sheet polarization, as seen in the left and right side of each matrix cell respectively. A red cross represent low efficiency, and a green check mark represent high efficiency.}
\label{fig:light_efficiency}
\end{figure}

 \noindent results for all SNR. We also provide a software tool for simulating any arbitrary light-sheet imaging system (see data availability statement). This software will allow users to find the PSF, OTF, and light transmission of a system as given in this paper.

\textbf{Funding} The Research Council of Norway (\#314546) and the Centre for Digital Life Norway.

\textbf{Disclosures} The authors declare no conflicts of interest.

\textbf{Data Availability Statement} Simulation software including user interface is found at https://doi.org/10.18710/YAYNNL


\printbibliography

\end{document}
