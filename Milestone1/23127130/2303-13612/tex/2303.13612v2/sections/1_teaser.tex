\twocolumn[
\maketitle

\newlength{\teaserheight}
\setlength\teaserheight{1.9cm}
\newlength{\probaheight}

\setlength\probaheight{2.0cm}
\centering
\setlength\lineskip{1.5pt}
\setlength\tabcolsep{1.5pt} 
{\footnotesize
\begin{tabular}{cr}
\begin{tabular}{
>{\centering\arraybackslash}m{\teaserheight}
>{\centering\arraybackslash}m{\teaserheight}
>{\centering\arraybackslash}m{\teaserheight}
>{\centering\arraybackslash}m{\teaserheight}
>{\centering\arraybackslash}m{1.cm}
>{\centering\arraybackslash}m{\teaserheight}
>{\centering\arraybackslash}m{\teaserheight}
>{\centering\arraybackslash}m{\teaserheight}
>{\centering\arraybackslash}m{\teaserheight}
}
 
 Reference & Query & Predicted pose & Pose distribution &
& Reference & Query & Predicted pose & Pose distribution \\
\frame{\includegraphics[height=\teaserheight, ]{figures/teaser/bus/3_ref_1_crop.png}}&
\frame{\includegraphics[height=\teaserheight, ]{figures/teaser/bus/1_query_crop.png}}&
\frame{\includegraphics[height=\teaserheight, ]{figures/teaser/bus/0_pred_crop.png}}&
\frame{\includegraphics[height=\teaserheight, ]{figures/teaser/bus/proba.png}} &
& \frame{\includegraphics[height=\teaserheight, ]{figures/teaser/skateboard_3/5_ref_3_crop.png}} &
\frame{\includegraphics[height=\teaserheight, ]{figures/teaser/skateboard_3/1_query_crop.png}}&
\frame{\includegraphics[height=\teaserheight, ]{figures/teaser/skateboard_3/0_pred_crop.png}}&
\frame{\includegraphics[height=\teaserheight, ]{figures/teaser/skateboard_3/proba.png}}\\


% \frame{\includegraphics[height=\teaserheight, ]{figures/teaser/mug2/ref.png}} &
% \frame{\includegraphics[height=\teaserheight, ]{figures/teaser/mug2/query.png}}&
% \frame{\includegraphics[height=\teaserheight, ]{figures/teaser/mug2/pred.png}}&
% \frame{\includegraphics[height=\teaserheight, ]{figures/teaser/mug/proba2.png}}\\
\frame{\includegraphics[height=\teaserheight, ]{figures/teaser/dishwasher_167_3_10/ref.png}} &
\frame{\includegraphics[height=\teaserheight, ]{figures/teaser/dishwasher_167_3_10/1_query.png}}&
\frame{\includegraphics[height=\teaserheight, ]{figures/teaser/dishwasher_167_3_10/0_pred.png}}&
\frame{\includegraphics[height=\teaserheight, ]{figures/teaser/dishwasher_167_3_10/proba.png}} &

& 
\frame{\includegraphics[height=\teaserheight, ]{figures/teaser/bottle_v3_3/3_ref_1_crop.png}} &
\frame{\includegraphics[height=\teaserheight, ]{figures/teaser/bottle_v3_3/1_query_crop.png}} &
\frame{\includegraphics[height=\teaserheight, ]{figures/teaser/bottle_v3_3/0_pred_crop.png}} &
\frame{\includegraphics[height=\teaserheight, ]{figures/teaser/bottle_v3_3/proba.png}}\\


\frame{\includegraphics[height=\teaserheight, ]{figures/teaser/mug/ref.png}} &
\frame{\includegraphics[height=\teaserheight, ]{figures/teaser/mug/query.png}}&
\frame{\includegraphics[height=\teaserheight, ]{figures/teaser/mug/pred.png}}&
\frame{\includegraphics[height=\teaserheight, ]{figures/teaser/mug/proba.png}} &
& \frame{\includegraphics[height=\teaserheight, ]{figures/experiments/qualitative_occlusion/clock_142_35_7/ref.png}} &
\frame{\includegraphics[height=\teaserheight, ]{figures/experiments/qualitative_occlusion/clock_142_35_7/query.png}} &
\frame{\includegraphics[height=\teaserheight, ]{figures/experiments/qualitative_occlusion/clock_142_35_7/0_pred_crop.png}}&
\frame{\includegraphics[height=\teaserheight, ]{figures/experiments/qualitative_occlusion/clock_142_35_7/proba.png}}\\
\end{tabular}

\end{tabular}
}
\vspace{-10pt}
\captionof{figure}{
Given as input a single reference view of a novel object, our method predicts the relative 3D pose (rotation) of a query view and its ambiguities. \textbf{We visualize the predicted pose by rendering the object from this pose, but the 3D model is only used for visualization purposes, not as input to our method.} Our method works by estimating a probability distribution over the space of 3D poses, visualized here on a sphere centered on the object. \textbf{We use the canonical pose of the 3D model to visualize this distribution, but not as input to our method.} From this distribution, we can also identify the pose ambiguities: For example, in the case of the bottle, any pose with the same pitch and roll is possible; in the case of the mug, a range of poses are possible as the handle is not visible in the query image. Our method is also robust to partial occlusions, as shown on the clock hidden in part by a rectangle in the query image.}
\label{fig:teaser}
\vspace{2pt}
\bigbreak]
