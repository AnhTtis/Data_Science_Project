\begin{figure}[t]
% \newlength{\plotHeight}
\setlength\plotHeight{1.8cm}
\centering
\setlength\lineskip{1.5pt}
\setlength\tabcolsep{1.5pt} 
{\footnotesize
\begin{tabular}{c}
\begin{tabular}{
>{\centering\arraybackslash}m{\plotHeight}
>{\centering\arraybackslash}m{\plotHeight}
>{\centering\arraybackslash}m{\plotHeight}
>{\centering\arraybackslash}m{\plotHeight}
}
\frame{\includegraphics[width=\plotHeight, ]{figures/experiments/occlusions/bottle.png}}&
\frame{\includegraphics[width=\plotHeight, ]{figures/experiments/occlusions/cup.png}}&
\frame{\includegraphics[width=\plotHeight, ]{figures/experiments/occlusions/dishwasher.png}}&
\frame{\includegraphics[width=\plotHeight, ]{figures/experiments/occlusions/train.png}}\\

\end{tabular}

% \begin{tabular}{@{\;\;\;\;}
% >{\arraybackslash}m{5cm}
% }
% \includegraphics[height=0.35\teaserheight, ]{figures/teaser/colorbar.png}\\
% \end{tabular}\\

\end{tabular}}
\caption{\nguyen{{\bf Samples of occluded test sets.} We use Random Erasing \cite{zhong2020random} to generate occlusions on objects for testing robustness of our methods to occlusions. }}
\label{fig:occlusionSamples}
\end{figure} 