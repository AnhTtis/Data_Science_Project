\vspace{-6pt}

In this section, we first describe our experimental setup in Section~\ref{sec:experimental setup}. We then compare our method to others \cite{nguyen_pizza_2022, mariotti_semi-supervised_2020, mariotti_viewnet_2021, 3dim, sundermeyer-cvpr20-multipathlearning, nguyen2022templates} on both synthetic and real-world datasets in Section~\ref{sec:main_results}. Section~\ref{sec:occlusions} reports an evaluation of the robustness to partial occlusions. We  provide the run-time in Section~\ref{sec:run_time}. Finally, we discuss  failure cases in Section~\ref{sec:failureCases}. An ablation study is provided in the supp.~mat.

 % We then conduct an ablation study to investigate the effectiveness of our method in different settings in Section~\ref{sec:ablation}, and provide the run-time in Section~\ref{sec:run_time}. Finally, we discuss  failure cases in Section~\ref{sec:failureCases}.




\subsection{Experimental setup}
\label{sec:experimental setup}


\section{The \textsf{AIOZ-GDANCE} Dataset}

Since we want to develop a large-scale dataset with in-the-wild videos, setting up a MoCap system is not feasible. However, manually annotating 3D groundtruth for millions of frames from dancing videos is also an extremely tedious job. Therefore, we propose a semi-automatic labeling method with humans in the loop to produce a large-scale group dance dataset. 


\subsection{Data Collection and Preprocessing}
\label{sec:tracking}
\textbf{Video Collection.} We collect the in-the-wild, public domain group dancing videos along with the music from Youtube, Tiktok, and Facebook.   
All group dance videos are processed at $1920 \times 1080$ resolution and 30FPS.

\textbf{Human Tracking.} We perform tracking for all humans in the videos using the state-of-the-art multi-object tracker~\cite{sun2022_dancetrack} to obtain the tracking bounding boxes. Note that although the tracker can produce reasonable results, there are failure cases in some frames. Therefore, we manually correct the bounding box of the incorrect cases. This tracking correction is crucial since we want the trajectory of each person to be accurately tracked in order to reconstruct their motion in latter stages.

\textbf{Pose Estimation.} Given the bounding boxes of each person in the video, we leverage the recent 2D pose estimation method~\cite{alphapose1} to generate the initial 2D poses for each person. 
In practice, there exist some inaccurately detected keypoints due to motion blur and partial occlusion. We manually fix the incorrect cases to obtain the 2D keypoints of each human bounding box. 


\begin{figure*}[ht]
    \centering
    \includegraphics[width=\textwidth,keepaspectratio=true]{images/dataset_pipeline102.2.pdf}
    \vspace{-3ex}
    \caption{The pipeline of making our \textsf{AIOZ-GDANCE} dataset. Blue boxes denote manual correction/annotation steps.}
    \vspace{-1ex}
    \label{fig:overview_dataset_pipeline}
\end{figure*}


\subsection{Group Motion Fitting}
To construct 3D group dance motion, we first reconstruct the full body motion for each dancer by fitting the 3D mesh. We then jointly optimize all dancer motions to construct the globally-coherent group motion. Finally, we post-process and remove wrong cases from the optimization results.

\textbf{Local Mesh Fitting.}
\label{sec:mesh-fitting} 
We use SMPL model~\cite{SMPL:2015} to represent the 3D human. The SMPL model is a differentiable function that maps the pose parameters $\mathbf{\theta}$, the shape parameters $\mathbf{\beta}$, and the root translation $\mathbf{\tau}$ into a set of 3D human body mesh vertices $\mathbf{V}\in \mathds{R}^{6890\times3}$ and 3D joints $\mathbf{X}\in \mathds{R}^{J\times3}$, where $J$ is the number of body joints.
Our optimizing motion variables for each individual dancer consist of a sequence of SMPL joint angles $\{\mathbf{\theta}_t\}_{t=1}^T$, a sequence of the root translation $\{\mathbf{\tau}_t\}_{t=1}^T$, and a single SMPL shape parameter $\mathbf{\beta}$. We fit the sequence of SMPL motion variables to the tracked 2D keypoints (obtained from Section \ref{sec:tracking}) by extending SMPLify-X~\cite{SMPL-X:2019} across the whole video sequence:
\begin{equation}
\label{eq:mesh_fitting}
E_{\rm local} = E_{\rm J} + \lambda_{\theta}E_{\theta} + \lambda_{\beta} E_{\beta} + \lambda_{\rm S}E_{\rm S} + \lambda_{\rm F}E_{\rm F},
\end{equation}
where $E_{\rm J}$, $E_{\theta}$ and $E_{\beta}$ are as in~\cite{SMPL-X:2019} but calculated across every frames of the video sequence. The smoothness term $E_{\rm S} = \sum_{t=1}^{T-1}\Vert \mathbf{\theta}_{t+1} - \mathbf{\theta}_{t} \Vert^2 + \sum_{j=1}^J\sum_{t=1}^{T-1}\Vert \mathbf{X}_{j,t+1} - \mathbf{X}_{j,t} \Vert^2$ encourages the temporal smoothness of the motion. The term $E_{\rm F} =  \sum_{t=1}^{T-1} \sum_{j \in \mathcal{F}} c_{j,t}\Vert \mathbf{X}_{j,t+1} - \mathbf{X}_{j,t} \Vert^2$ ensures feet joints to stay stationary when in contact (zero velocity). $\mathcal{F}$ is the set of feet joint indexes, $c_{j,t}$ is the feet contact label of joint $j$ at time $t$ produced by a contact estimation network~\cite{zou2020contact_net}. 


\textbf{Global Optimization.}
Given the 3D motion sequence of each dancer $p$: $\{\mathbf{\theta}^p_t, \mathbf{\tau}^p_t\}$, we further resolve the motion trajectory problems in group dance by solving the following objective: 
\begin{align}
\label{eq:global_opt}
E_{\rm global} &= E_{\rm J} + \lambda_{\rm pen}E_{\rm pen} + \lambda_{\rm reg}\sum_{p}E_{\rm reg}(p) \notag\\ &+ \lambda_{\rm dep}\sum_{p,p',t}E_{\rm dep}(p,p',t) + \lambda_{\rm gc}\sum_{p}E_{\rm gc}(p),
\end{align}
where 
$E_{\rm pen}$ is the Signed Distance Function penetration term based on~\cite{jiang2020_coherent_reconstruction} to prevent the overlapping of reconstructed motions between dancers.
${E_{\rm reg}(p) =\sum_{t=1}^T\Vert \mathbf{\theta}^p_t - \hat{\mathbf{\theta}}^p_t\Vert^2}$ is the regularization term that prevents the motion from deviating too much from the prior optimized individual motion $\{\hat{\mathbf{\theta}}^p_t\}$ obtained by optimizing Equation \ref{eq:mesh_fitting} for dancer $p$. 


In practice, we find that the relative depth ordering of dancers in the scene can be inconsistent due to the ambiguity of the 2D projection. To ensure the group motion quality, we watch the videos and manually provide the ordinal depth relation information of all dancers in the scene at each frame $t$ as follows:
\begin{equation}
r_t(p,p') =
\begin{cases}
1, &\text{if dancer } p \text{ is closer than } p' \\ 
-1, &\text{if dancer } p \text{ is farther than } p' \\
0, &\text{if their depths are roughly equal}
\end{cases}
\end{equation}

Given the relative depth information provided by human annotators, we derive the depth relation term $E_{\rm dep}$ inspired by~\cite{chen2016_depth_ranking}. This term encourages consistent ordinal depth relation between the motion trajectories of multiple dancers, especially when dancers partially occlude each other:  
\begin{equation}
\small
\label{eq:global_depth1}
E_{\rm dep}(p,p',t) =
\begin{cases}
     \log(1+\exp(z^p_t - z^{p'}_t)), &r_t(p,p')=1 \\
     \log(1+\exp(-z^p_t + z^{p'}_t)), &r_t(p,p')=-1 \\
     (z^p_t - z^{p'}_t)^2, &r_t(p,p')=0 \\
\end{cases}
\end{equation}
where $z^p_t$ is the depth component of the root translation $\mathbf{\tau}^p_t$ of the person $p$ at frame $t$.

Finally, we apply the global ground contact constraint $E_{\rm gc}$ to further ensure consistency between the motion of every person and the environment based on the ground contact information. This contact term is also needed to reduce the artifacts such as foot-skating, jittering, and penetration under the ground.
\begin{equation}
\label{eq_Egc}
\small
E_{\rm gc}(p) = \sum_{t=1}^{T-1} \sum_{j \in \mathcal{F}} c^p_{j,t}\Vert \mathbf{X}^p_{j,t+1} - \mathbf{X}^p_{j,t} \Vert^2 + c^p_{j,t} \Vert  (\mathbf{X}^p_{j,t} - \mathbf{f})^\top \mathbf{n}^* \Vert^2,
\end{equation}
where $\mathcal{F}$ is the set of feet joint indexes, $\mathbf{n}^*$ is the estimated plane normal and $\mathbf{f}$ is a 3D fixed point on the ground plane. The first term in Equation~\ref{eq_Egc} is the zero velocity constraint when the feet are in contact with the ground, while the second term encourages the feet position to stay near the ground when in contact. To obtain the ground plane parameters, we initialize the plane point $\mathbf{f}$ as the weighted median of all contact feet positions. The plane normal $\mathbf{n}^*$ is obtained by optimizing a robust Huber objective:
\begin{equation}
\small
\mathbf{n}^* = \arg\min_{\mathbf{n}} \sum_{\mathbf{X}_{\rm feet}} \mathcal{H}\left((\mathbf{X}_{\rm feet} - \mathbf{f})^\top \frac{\mathbf{n}}{\Vert\mathbf{n}\Vert}\right) + \Vert \mathbf{n}^\top\mathbf{n} - 1 \Vert^2,
\end{equation}
where $\mathcal{H}$ is the Huber loss function~\cite{huber1992robust},  $\mathbf{X}_{\rm feet}$ is the 3D feet positions of all dancers across the whole sequence that are labelled as in contact (i.e., $c^p_{j,t} = 1$) . 















\textbf{Post Processing.}
Although our optimization process produces relatively good results, there are some extreme cases that it fails to handle. We recheck all the results and fix the cases with minor problems. Other severely wrong cases are simply discarded. More details can be found in our Supplementary Material.































\begin{figure*}[ht]
    \centering
    \includegraphics[width=\textwidth]{images/baseline.pdf}
    \caption{
    Architecture of our Music-driven 3D \textbf{G}roup \textbf{Dance} generato\textbf{r} (GDanceR). Our model takes in a music sequence and a set of initial positions, and then auto-regressively generates coherent group dance motions that are attuned to the input music.
    }
    \label{fig:GMG}
\end{figure*}



\subsection{How will \textbf{\textsf{AIOZ-GDANCE}} be useful to the community?}


We bring up some interesting research directions that can be benefited from our dataset:

\begin{itemize}
    \item Group Dance Generation: While single-person choreography is a hot research topic~\cite{survey-dancing-deep-metaverse,li2021AIST++,siyao2022_bailando,li2022_phantomdance,chen2021_choreomaster}, group dance generation has not yet well investigated. We hope that the release of our dataset will foster more this research direction.
    
    
    \item Human Pose Tracking: By having SMPL groundtruth motion, our dataset can be used in many human pose/motion tracking tasks such as in~\cite{yang2021learning,sun2022_dancetrack,doering2022posetrack21}. 
    
\end{itemize}

Apart from these tasks, we believe our dataset can be used in other scenarios such as dance education~\cite{papillon2022pirounet,ferreira2021_learn2dance_gcn}, dance style transfer~\cite{zhang2021dance,yin2022dance,thangstyle}, or human behavior analysis~\cite{men2022gan,le2022global,lee2022human,le2023uncertainty}. The research community is free to explore other applications of our dataset.








\begin{table*}[!t]
\addtolength{\tabcolsep}{1pt}
\centering
    \scalebox{.82}{
    \begin{tabular}{@{}l l | ccccccccccc | c@{}}
	\toprule
	& Method &  novel inst. & bottle$^*$ & bus & clock & dishwasher & guitar & mug & pistol & skateboard & train & washer & mean \\
	\midrule
	\parbox[t]{2mm}{\multirow{7}{*}{\rotatebox[origin=c]{90}{$\bm{\AccThirty}\uparrow$}}} 
	& ViewNet \cite{mariotti_viewnet_2021} & \bf 80.35 &  42.46 &  39.08 &  35.12 &  35.13 &  23.09 &  50.42 &  32.13 &  36.67 &  51.35 &  45.03 &  42.80  \\
	& SSVE \cite{mariotti_semi-supervised_2020} & 78.16 &  55.59 &  41.13 &  53.42 &  40.03 &  32.07 &  57.45 &  51.03 &  \bf 65.11 &  48.03 &  70.69 &  53.88  \\
	& PIZZA \cite{nguyen_pizza_2022} &  75.16 &  70.01 &  41.49 &  50.13 &  51.41 &  16.14 &  54.69 &  54.67 &  61.08 &  59.45 &  80.92 &  55.92  \\
	& 3DiM \cite{3dim} & 80.13 &  89.12 &  46.35 &  35.17 &  43.28 &  21.35 &  51.12 &  46.13 &  53.07 &  43.57 &  76.06 &  53.21	\\
	& Ours (top 1)  & 78.37 &  \bf 90.04 &  \bf 56.49 &  \bf 59.62 &  \bf 66.72 &  \bf 34.31 &  \bf 63.70 &  \bf 83.23 &   60.58 &  \bf 61.78 &  \bf 80.53 &  \bf 66.85	\\
	\cmidrule{2-14}
	& Ours (top 3) & 91.23 &  99.04 &  90.26 &  78.55 &  93.59 &  60.28 &  75.60 &  91.23 &  83.59 &  83.41 &  91.83 &  85.33 \\
	& Ours (top 5) & 92.97 &  99.04 &  94.35 &  83.23 &  98.44 &  66.71 &  81.85 &  95.31 &  89.84 &  93.57 &  95.91 &  90.11 \\
    \midrule
    \parbox[t]{2mm}{\multirow{7}{*}{\rotatebox[origin=c]{90}{$\bm{\MedErr}\downarrow$}}}  & ViewNet \cite{mariotti_viewnet_2021} & 6.72 &  30.56 &  75.02 &  18.09 &  32.07 &  70.43 &  36.01 &  38.04 &  43.01 &  26.07 &  41.39 &  37.95  \\
    & SSVE \cite{mariotti_semi-supervised_2020} & 6.25 &  27.63 &  84.42 &  19.75 &  26.17 &  67.52 &  34.01 &  19.25 &  \bf 19.06 &  24.13 &  45.13 &  33.94  \\
    & PIZZA     \cite{nguyen_pizza_2022} &  5.95 &  29.40 &  65.64 &  21.03 &  16.36 &  60.12 &  29.92 &  18.59 &  22.90 &  19.59 &  35.52 &  29.55 \\
    & 3DiM \cite{3dim} & \bf 5.86 &  \bf 5.67 &  59.02 &  25.15 &  40.57 &  55.09 &  20.07 &  16.38 &  18.03 &  16.82 &  16.28 &  25.36\\
	& Ours (top 1) &  8.25 &  5.70 &  \bf 57.64 &  \bf 17.67 &  \bf 13.36 &  \bf 51.52 &  \bf 19.92 &  \bf 8.59 &   21.10 &  \bf 17.59 &  \bf 5.52 &  \bf 20.62\\
	\cmidrule{2-14}
	& Ours (top 3) & 5.43 &  1.07 &  5.43 &  8.26 &  5.50 &  19.06 &  11.62 &  6.41 &  7.19 &  7.08 &  4.28 &  7.39\\
	& Ours (top 5) & 4.64 &  0.87 &  4.56 &  7.45 &  4.29 &  15.58 &  8.83 &  5.44 &  5.86 &  5.10 &  4.22 &  6.08\\
	\bottomrule
	\end{tabular}
    }
    \vspace*{1mm}
    \caption{{\bf 3D pose estimation on novel instances and novel categories from our dataset.} We treat “bottle" as a symmetric category, i.e., the error is only the difference of elevation angle. Since the quality of prediction may depend on the reference image, we report the score as the average of 5 runs with 5 different reference images.  
    }
    % We provide the std of this score in the supplementary material. 
    
    %\nguyenrmk{Yes, it's time reason. And yes again, 'unseen ins.' corresponds to unseen instances of training categories. It's out of scope, I agree but since we re-implemented all these methods, it's important to show that it works correctly for seen categories so the reviews can know we re-implement them successfully.}
  \label{tab:shapeNet}
  \end{table*} 
To the best of our knowledge, we are the first method addressing the problem of object pose estimation from a single image when the object belongs to a category not seen during training: PIZZA~\cite{nguyen_pizza_2022} evaluated on the DeepIM refinement benchmark, which is made of pairs of images with a small relative pose; SSVE~\cite{mariotti_semi-supervised_2020} and ViewNet~\cite{mariotti_viewnet_2021} evaluated only on objects from categories seen during training.  We therefore  had to create a new benchmark to evaluate our method. 

\vspace{-12pt}
\paragraph{Synthetic dataset.} 
We created a dataset as in FORGE~\cite{forge_jiang} using the same ShapeNet~\cite{chang2015shapenet} object categories.
For the training set, we randomly select 1000 object instances from each of the 13 categories as done in FORGE~(\textit{airplane, bench, cabinet, car, chair, display, lamp, loudspeaker, rifle, sofa, table, telephone, and vessel}), resulting in a total of 13,000 instances. We build two separate test sets for evaluation. The first test set is the ``novel instances'' set, which contains \nguyen{50 new instances for each training category}. The second test set is the ``novel category'' set, which includes \nguyen{100 models per category for} the 10 unseen categories selected by FORGE~(\textit{bus, guitar, clock, bottle, train, mug, washer, skateboard, dishwasher, and pistol}). For each 3D model, we randomly select camera poses to produce five reference images and five query images. We use BlenderProc~\cite{denninger2019blenderproc} as rendering engine. 

Figure~\ref{fig:shapeNet} illustrates the categories used for training our architecture and the categories used for testing it. The shapes and appearances of the categories in the test set are very different from the shapes and appearances of the categories in the training set, and thus constitute a good test set for generalization to unseen categories.

\vspace{-12pt}
\paragraph{Real-world dataset.}
We evaluate on the T-LESS dataset \cite{hodan-wacv17-tless} following the evaluation protocol of \cite{sundermeyer-cvpr20-multipathlearning}: we train only on objects 1-18 and test on the full PrimeSense test set using the ground-truth masks. At inference, we randomly sample a non-occluded reference image either from \emph{all views} or only from \emph{front views}~(-45\textdegree $\leq$ azimuth $\leq$ 45\textdegree), which often offers more information on the object and illustrates the influence of the reference view.

%The visualization for the shape and appearance difference between training and testing objects can be found in \cite{hodan2018bop}.}

\vspace{-12pt}
\paragraph{Metrics.}
For the ShapeNet dataset, we report two different metrics based on relative camera pose error as done in \cite{mariotti_semi-supervised_2020}. Specifically, we provide the median pose error across instances for each category in the test set, and the accuracy $\bm{\AccThirty}$ for which a prediction is treated as correct when the pose error is $\le 30^{\circ}$. Additionally, we present the results of our method for the top 3 and 5 nearest neighbors retrieved by template matching.

For the T-LESS dataset, as most objects are symmetric, we report the recall VSD metric as done in \cite{sundermeyer-cvpr20-multipathlearning}. Please note that for the evaluation on the T-LESS dataset, we also predict the translation by using the same formula ``projective distance estimation'' as SSD-6D~\cite{kehl-iccv17-ssd6d}, as done in~\cite{sundermeyer-eccv18-implicit3dorientationlearning, sundermeyer-cvpr20-multipathlearning}. This translation is deduced from the retrieved template and the relative scale factor between the two input images, as detailed in Section~8 of~\cite{nguyen2022templates}.

% \vincentrmk{In table 2, you say that you treat bottle as a symmetric category. What does it mean? How do you change the metrics for bottle, exactly? } \nguyenrmk{It means I change the metric for bottle: The error is only the difference of elevation angle}

\vspace{-12pt}
\paragraph{Baselines.}

We compare our work with all previous methods that aim to predict a pose from a single view: PIZZA \cite{nguyen_pizza_2022}, a regression-based approach that directly predicts the relative pose, as well as SSVE~\cite{mariotti_semi-supervised_2020} and ViewNet~\cite{mariotti_viewnet_2021}, which employ semi-supervised and self-supervised techniques to treat viewpoint estimation as an image reconstruction problem using conditional generation. We also compare our method with the recent diffusion-based method 3DiM~\cite{3dim}, which generates pixel-level view synthesis. Since 3DiM originally only targets view-synthesis and is not designed for 3D object pose, we use it to generate templates and perform nearest neighbor search to estimate a 3D object pose. To make 3DiM work in the same setting as us, we retrain it using relative pose conditioning instead of canonical pose conditioning.

\vspace{-12pt}
\paragraph{Implementation.}
Only the code of PIZZA is available. The other methods did not release their code at the time of writing, however we re-implemented them. We use a ResNet18 backbone  as in~\cite{nguyen_pizza_2022} for PIZZA, SSVE, and ViewNet. We train all models on input images with a resolution of 256$\times$256 except for 3DiM for which we use a resolution of 128$\times$128 since 3DiM performs view synthesis in pixel space, which takes much more memory. Our re-implementations achieve similar performance as the original papers when evaluated on the same data for seen categories, as shown in Table~\ref{tab:shapeNet}, which validates our comparisons. 
Our method also uses the frozen encoder from \cite{stable-diffusion} to encode the input images into embeddings of size 32$\times$32$\times$8. \nguyen{In all settings, we train the baselines and our method using the same training set and  AdamW~\cite{loshchilov_decoupled_2017} with an initial learning rate of $5\,{\times}\,10^{-5}$. % 5e-5
Training takes about 20 hours on 4 V100 GPUs for each method.

} %We refer the reader to the supplementary material for more implementation details.


\begin{table}[t]
\centering
\resizebox{\linewidth}{!}{
\begin{tabular}{@{\,}l | l l c c c@{\,}}
\toprule
 & \multirow{2}{*}{\raisebox{-1mm}{\bf Method}} & 
 \multirow{2}{*}{\begin{tabular}{@{}c@{}} \textbf{Ref.\ image}\raisebox{4mm}{} \\ \textbf{sampling} \end{tabular}}
 & \multicolumn{3}{c}{\textbf{Recall VSD}} \\
 \cmidrule(l){4-6}
& & %\textbf{sampling} 
& \textbf{Seen obj.} % \textbf{Obj. 1-18} 
& \!\!\!\!\textbf{Novel obj.}\!\!\!\! % \textbf{Obj. 19-30} 
& \textbf{Avg} \\
 \midrule
  \multirow{2}{*}{\rotatebox[origin=c]{90}{\stackbox[c][b]{\baselineskip=12pt GT\\CAD\par}}}

  & Nguyen et al.~\cite{nguyen2022templates} & -  & \bf 60.15 & \bf 58.70 & \bf 59.57\\
  & MultiPath~\cite{sundermeyer-cvpr20-multipathlearning} & - & 43.17 & 43.33 & 43.24\\
  \midrule
  \multirow{2}{*}{\rotatebox[origin=c]{90}{\stackbox[c][b]{\baselineskip=12pt 1 ref.~image~~\\(avg 5 runs)~~\par}}}
  & PIZZA~\cite{nguyen_pizza_2022} & all views & 20.05 & 15.90 & 18.39  \\
  & Ours & all views & \bf 47.03 & \bf 45.69 & \bf 46.49 \\
 \cmidrule(lr){2-6}
  & PIZZA~\cite{nguyen_pizza_2022} & front views & 21.63 & 15.55 & 19.19 \\
  & Ours & front views & \bf 49.30 & \bf 48.46 & \bf 48.96 \\
\bottomrule
\end{tabular}}
% \vspace*{-10pt}
\caption{{\nguyen{\bf Comparison to PIZZA~\cite{nguyen_pizza_2022} and CAD-based methods~\cite{nguyen2022templates, sundermeyer-cvpr20-multipathlearning}} on seen (obj.~1-18) and novel (obj.~19-30) objects of T-LESS.
% , following the protocol of~\cite{sundermeyer-cvpr20-multipathlearning}. 
We report  numbers averaged over 5 different samplings and runs.}}
\label{tab:tless}
\end{table}


\begin{figure}[t]
    \centering
    \begin{tabular}{cc}
    \footnotesize{Seen objects: \#4, \#14} & \footnotesize{Novel objects: \#20, \#22} \\
    \includegraphics[width=0.46\linewidth]{rebuttal/obj4.png} & \includegraphics[width=0.46\linewidth]{rebuttal/obj20.png}\\
    \includegraphics[width=0.46\linewidth]{rebuttal/obj14.png} & \includegraphics[width=0.46\linewidth]{rebuttal/obj22.png} \\[-1mm]
    \footnotesize{Reference$\;\;\;\;$Query$\;\;\;\;$Prediction} & \footnotesize{Reference$\;\;\;\;$Query$\;\;\;\;$Prediction}\\
    \end{tabular}
    % \vspace*{-10pt}
    \caption{{\bf Qualitative results on real images of T-LESS.} For each sample, we show in the last column the predicted poses. }
    % \textbf{We visualize the predicted poses by rendering the object from these poses, but the 3D model is only used for visualization purposes, not as input to our method.}
    % \vspace*{-4mm}
    \label{fig:tless}
\end{figure}

\begin{figure}[t]
%\newlength{\plotHeight}
\setlength\plotHeight{2.0cm}
\centering
\setlength\lineskip{1.5pt}
\setlength\tabcolsep{1.5pt} 
{\footnotesize
\begin{tabular}{cr}
\begin{tabular}{
>{\centering\arraybackslash}m{\plotHeight}
>{\centering\arraybackslash}m{\plotHeight}
>{\centering\arraybackslash}m{\plotHeight}
>{\centering\arraybackslash}m{\plotHeight}
}
\frame{\includegraphics[height=\plotHeight, ]{figures/experiments/failure_cases/final_dishwasher.png}}&
\frame{\includegraphics[height=\plotHeight, ]{figures/experiments/failure_cases/final_clock.png}}&

% \frame{\includegraphics[height=\plotHeight, ]{figures/experiments/failure_cases/black_busv2.png}}&
\frame{\includegraphics[height=\plotHeight, ]{figures/experiments/failure_cases/final_mug.png}}&
\frame{\includegraphics[height=\plotHeight, ]{figures/experiments/failure_cases/guitar3.png}}\\
\footnotesize{\;\;\;\;\;Dishwasher$\;\;\;\;\;\;\;\;\;\;\;\;\;$Clock$\;\;\;\;\;\;\;\;\;\;\;\;\;\;\;\;\;\;\;$Mug\;\;\;\;\;\;\;\;\;\;\;\;\;\;\;\;\;\;\;Guitar}
\end{tabular}
\end{tabular}}
% \vspace*{-10pt}
\caption{\nguyen{{\bf Failure cases.} ``Dishwashers'', ``clocks'', and ``dishwashers'' are ``nearly symmetrical'' while ``guitars'' are barely visible from some viewpoints. This makes the pose estimation very challenging, and all the methods perform poorly on these categories.}}
\label{fig:failureCases}
\end{figure}

\begin{figure*}
\newlength{\imageheight}
\setlength\imageheight{1.48cm}
\centering
\setlength\lineskip{1.pt}
\setlength\tabcolsep{1.pt} 
{\small
\begin{tabular}{c}
\begin{tabular}{
>{\centering\arraybackslash}m{0.5cm}
>{\centering\arraybackslash}m{\imageheight}
>{\centering\arraybackslash}m{\imageheight}
>{\centering\arraybackslash}m{\imageheight}
>{\centering\arraybackslash}m{\imageheight}
>{\centering\arraybackslash}m{\imageheight}
>{\centering\arraybackslash}m{1.cm}
>{\centering\arraybackslash}m{\imageheight}
>{\centering\arraybackslash}m{\imageheight}
>{\centering\arraybackslash}m{\imageheight}
>{\centering\arraybackslash}m{\imageheight}
>{\centering\arraybackslash}m{\imageheight}
}
\hline
\multicolumn{12}{c}{\textbf{without occlusions}}\\
\hline
\\[-0.2cm]
\parbox[t]{0mm}{\rotatebox[origin=c]{90}{Bottle}} & \frame{\includegraphics[height=\imageheight, ]{figures/experiments/qualitative/bottle_2_10_10/4_ref_2_crop.png}} &
\frame{\includegraphics[height=\imageheight, ]{figures/experiments/qualitative/bottle_2_10_10/1_query_crop.png}} &
\frame{\includegraphics[height=\imageheight, ]{figures/experiments/qualitative/bottle_2_10_10/pizza.png}}&
\frame{\includegraphics[height=\imageheight, ]{figures/experiments/qualitative/bottle_2_10_10/0_pred_crop.png}}&
\frame{\includegraphics[height=\imageheight, ]{figures/experiments/qualitative/bottle_2_10_10/proba2.png}}&
&
\frame{\includegraphics[height=\imageheight, ]{figures/experiments/qualitative/bottle_6_14_1/4_ref_2_crop.png}} &
\frame{\includegraphics[height=\imageheight, ]{figures/experiments/qualitative/bottle_6_14_1/1_query_crop.png}} &
\frame{\includegraphics[height=\imageheight, ]{figures/experiments/qualitative/bottle_6_14_1/pizza.png}}&
\frame{\includegraphics[height=\imageheight, ]{figures/experiments/qualitative/bottle_6_14_1/0_pred_crop.png}}&
\frame{\includegraphics[height=\imageheight, ]{figures/experiments/qualitative/bottle_6_14_1/proba2.png}}\\

% \frame{\includegraphics[height=\imageheight, ]{figures/experiments/qualitative/bus_85_50_5/0_pred_crop.png}} &
% \frame{\includegraphics[height=\imageheight, ]{figures/experiments/qualitative/bus_85_50_5/1_query_crop.png}} &
% \frame{\includegraphics[height=\imageheight, ]{figures/experiments/qualitative/bus_85_50_5/pizza.png}}&
% \frame{\includegraphics[height=\imageheight, ]{figures/experiments/qualitative/bus_85_50_5/0_pred_crop.png}}&
% \frame{\includegraphics[height=\imageheight, ]{figures/experiments/qualitative/bus_85_50_5/proba.png}}&
% &
% \frame{\includegraphics[height=\imageheight, ]{figures/experiments/qualitative/bus_104_68_15/3_ref_1_crop.png}} &
% \frame{\includegraphics[height=\imageheight, ]{figures/experiments/qualitative/bus_104_68_15/1_query_crop.png}} &
% \frame{\includegraphics[height=\imageheight, ]{figures/experiments/qualitative/bus_104_68_15/pizza.png}}&
% \frame{\includegraphics[height=\imageheight, ]{figures/experiments/qualitative/bus_104_68_15/0_pred_crop.png}}&
% \frame{\includegraphics[height=\imageheight, ]{figures/experiments/qualitative/bus_104_68_15/proba.png}}\\

\parbox[t]{0mm}{\rotatebox[origin=c]{90}{Clock}} & \frame{\includegraphics[height=\imageheight, ]{figures/experiments/qualitative/clock_138_31_9/0_pred_crop.png}} &
\frame{\includegraphics[height=\imageheight, ]{figures/experiments/qualitative/clock_138_31_9/1_query_crop.png}} &
\frame{\includegraphics[height=\imageheight, ]{figures/experiments/qualitative/clock_138_31_9/pizza.png}}&
\frame{\includegraphics[height=\imageheight, ]{figures/experiments/qualitative/clock_138_31_9/0_pred_crop.png}}&
\frame{\includegraphics[height=\imageheight, ]{figures/experiments/qualitative/clock_138_31_9/proba2.png}}&
&
\frame{\includegraphics[height=\imageheight, ]{figures/experiments/qualitative/clock_159_50_6/2_ref_0_crop.png}} &
\frame{\includegraphics[height=\imageheight, ]{figures/experiments/qualitative/clock_159_50_6/1_query_crop.png}} &
\frame{\includegraphics[height=\imageheight, ]{figures/experiments/qualitative/clock_159_50_6/pizza.png}}&
\frame{\includegraphics[height=\imageheight, ]{figures/experiments/qualitative/clock_159_50_6/0_pred_crop.png}}&
\frame{\includegraphics[height=\imageheight, ]{figures/experiments/qualitative/clock_159_50_6/proba2.png}}\\

\parbox[t]{0mm}{\rotatebox[origin=l]{90}{Dishwasher}} & \frame{\includegraphics[height=\imageheight, ]{figures/experiments/qualitative/dishwasher_167_3_8/0_pred_crop.png}} &
\frame{\includegraphics[height=\imageheight, ]{figures/experiments/qualitative/dishwasher_167_3_8/1_query_crop.png}} &
\frame{\includegraphics[height=\imageheight, ]{figures/experiments/qualitative/dishwasher_167_3_8/pizza.png}}&
\frame{\includegraphics[height=\imageheight, ]{figures/experiments/qualitative/dishwasher_167_3_8/0_pred_crop.png}}&
\frame{\includegraphics[height=\imageheight, ]{figures/experiments/qualitative/dishwasher_167_3_8/proba2.png}}&
&
\frame{\includegraphics[height=\imageheight, ]{figures/experiments/qualitative/dishwasher_167_3_10/2_ref_0_crop.png}} &
\frame{\includegraphics[height=\imageheight, ]{figures/experiments/qualitative/dishwasher_167_3_10/1_query_crop.png}} &
\frame{\includegraphics[height=\imageheight, ]{figures/experiments/qualitative/dishwasher_167_3_10/pizza.png}}&
\frame{\includegraphics[height=\imageheight, ]{figures/experiments/qualitative/dishwasher_167_3_10/0_pred_crop.png}}&
\frame{\includegraphics[height=\imageheight, ]{figures/experiments/qualitative/dishwasher_167_3_10/proba2.png}}\\

\parbox[t]{0mm}{\rotatebox[origin=c]{90}{Guitar}} & \frame{\includegraphics[height=\imageheight, ]{figures/experiments/qualitative/guitar_189_24_0/2_ref_0_crop.png}} &
\frame{\includegraphics[height=\imageheight, ]{figures/experiments/qualitative/guitar_189_24_0/1_query_crop.png}} &
\frame{\includegraphics[height=\imageheight, ]{figures/experiments/qualitative/guitar_189_24_0/pizza.png}}&
\frame{\includegraphics[height=\imageheight, ]{figures/experiments/qualitative/guitar_189_24_0/0_pred_crop.png}}&
\frame{\includegraphics[height=\imageheight, ]{figures/experiments/qualitative/guitar_189_24_0/proba2.png}}&
&
\frame{\includegraphics[height=\imageheight, ]{figures/experiments/qualitative/guitar_193_28_15/3_ref_1_crop.png}} &
\frame{\includegraphics[height=\imageheight, ]{figures/experiments/qualitative/guitar_193_28_15/1_query_crop.png}} &
\frame{\includegraphics[height=\imageheight, ]{figures/experiments/qualitative/guitar_193_28_15/pizza.png}}&
\frame{\includegraphics[height=\imageheight, ]{figures/experiments/qualitative/guitar_193_28_15/0_pred_crop.png}}&
\frame{\includegraphics[height=\imageheight, ]{figures/experiments/qualitative/guitar_193_28_15/proba2.png}}\\

\parbox[t]{0mm}{\rotatebox[origin=c]{90}{Mug}} & \frame{\includegraphics[height=\imageheight, ]{figures/experiments/qualitative/mug2/2_ref_0_crop.png}} &
\frame{\includegraphics[height=\imageheight, ]{figures/experiments/qualitative/mug2/1_query_crop.png}} &
\frame{\includegraphics[height=\imageheight, ]{figures/experiments/qualitative/mug2/pizza.png}}&
\frame{\includegraphics[height=\imageheight, ]{figures/experiments/qualitative/mug2/0_pred_crop.png}}&
\frame{\includegraphics[height=\imageheight, ]{figures/experiments/qualitative/mug2/proba2.png}}&
&
\frame{\includegraphics[height=\imageheight, ]{figures/experiments/qualitative/mug11/2_ref_0_crop.png}} &
\frame{\includegraphics[height=\imageheight, ]{figures/experiments/qualitative/mug11/1_query_crop.png}} &
\frame{\includegraphics[height=\imageheight, ]{figures/experiments/qualitative/mug11/pizza.png}}&
\frame{\includegraphics[height=\imageheight, ]{figures/experiments/qualitative/mug11/0_pred_crop.png}}&
\frame{\includegraphics[height=\imageheight, ]{figures/experiments/qualitative/mug11/proba2.png}}\\

\parbox[t]{0mm}{\rotatebox[origin=c]{90}{Pistol}} & \frame{\includegraphics[height=\imageheight, ]{figures/experiments/qualitative/pistol1/2_ref_0_crop.png}} &
\frame{\includegraphics[height=\imageheight, ]{figures/experiments/qualitative/pistol1/1_query_crop.png}} &
\frame{\includegraphics[height=\imageheight, ]{figures/experiments/qualitative/pistol1/pizza.png}}&
\frame{\includegraphics[height=\imageheight, ]{figures/experiments/qualitative/pistol1/0_pred_crop.png}}&
\frame{\includegraphics[height=\imageheight, ]{figures/experiments/qualitative/pistol1/proba2.png}}&
&
\frame{\includegraphics[height=\imageheight, ]{figures/experiments/qualitative/pistol9/4_ref_2_crop.png}} &
\frame{\includegraphics[height=\imageheight, ]{figures/experiments/qualitative/pistol9/1_query_crop.png}} &
\frame{\includegraphics[height=\imageheight, ]{figures/experiments/qualitative/pistol9/pizza.png}}&
\frame{\includegraphics[height=\imageheight, ]{figures/experiments/qualitative/pistol9/0_pred_crop.png}}&
\frame{\includegraphics[height=\imageheight, ]{figures/experiments/qualitative/pistol9/proba2.png}}\\

\parbox[t]{1mm}{\rotatebox[origin=l]{90}{Skateboard}} & \frame{\includegraphics[height=\imageheight, ]{figures/experiments/qualitative/skateboard1/3_ref_1_crop.png}} &
\frame{\includegraphics[height=\imageheight, ]{figures/experiments/qualitative/skateboard1/1_query_crop.png}} &
\frame{\includegraphics[height=\imageheight, ]{figures/experiments/qualitative/skateboard1/pizza.png}}&
\frame{\includegraphics[height=\imageheight, ]{figures/experiments/qualitative/skateboard1/0_pred_crop.png}}&
\frame{\includegraphics[height=\imageheight, ]{figures/experiments/qualitative/skateboard1/proba2.png}}&
&
\frame{\includegraphics[height=\imageheight, ]{figures/experiments/qualitative/skateboard13/4_ref_2_crop.png}} &
\frame{\includegraphics[height=\imageheight, ]{figures/experiments/qualitative/skateboard13/1_query_crop.png}} &
\frame{\includegraphics[height=\imageheight, ]{figures/experiments/qualitative/skateboard13/pizza.png}}&
\frame{\includegraphics[height=\imageheight, ]{figures/experiments/qualitative/skateboard13/0_pred_crop.png}}&
\frame{\includegraphics[height=\imageheight, ]{figures/experiments/qualitative/skateboard13/proba2.png}}\\

\parbox[t]{1mm}{\rotatebox[origin=c]{90}{Train}} & \frame{\includegraphics[height=\imageheight, ]{figures/experiments/qualitative/train1/3_ref_1_crop.png}} &
\frame{\includegraphics[height=\imageheight, ]{figures/experiments/qualitative/train1/1_query_crop.png}} &
\frame{\includegraphics[height=\imageheight, ]{figures/experiments/qualitative/train1/pizza.png}}&
\frame{\includegraphics[height=\imageheight, ]{figures/experiments/qualitative/train1/0_pred_crop.png}}&
\frame{\includegraphics[height=\imageheight, ]{figures/experiments/qualitative/train1/proba2.png}}&
&
\frame{\includegraphics[height=\imageheight, ]{figures/experiments/qualitative/train7/4_ref_2_crop.png}} &
\frame{\includegraphics[height=\imageheight, ]{figures/experiments/qualitative/train7/1_query_crop.png}} &
\frame{\includegraphics[height=\imageheight, ]{figures/experiments/qualitative/train7/pizza.png}}&
\frame{\includegraphics[height=\imageheight, ]{figures/experiments/qualitative/train7/0_pred_crop.png}}&
\frame{\includegraphics[height=\imageheight, ]{figures/experiments/qualitative/train7/proba2.png}}\\[0.1cm]


\parbox[t]{1mm}{\rotatebox[origin=c]{90}{Washer}} & \frame{\includegraphics[height=\imageheight, ]{figures/experiments/qualitative/washer0/2_ref_0_crop.png}} &
\frame{\includegraphics[height=\imageheight, ]{figures/experiments/qualitative/washer0/1_query_crop.png}} &
\frame{\includegraphics[height=\imageheight, ]{figures/experiments/qualitative/washer0/pizza.png}}&
\frame{\includegraphics[height=\imageheight, ]{figures/experiments/qualitative/washer0/0_pred_crop.png}}&
\frame{\includegraphics[height=\imageheight, ]{figures/experiments/qualitative/washer0/proba2.png}}&
&
\frame{\includegraphics[height=\imageheight, ]{figures/experiments/qualitative/washer8/2_ref_0_crop.png}} &
\frame{\includegraphics[height=\imageheight, ]{figures/experiments/qualitative/washer8/1_query_crop.png}} &
\frame{\includegraphics[height=\imageheight, ]{figures/experiments/qualitative/washer8/pizza.png}}&
\frame{\includegraphics[height=\imageheight, ]{figures/experiments/qualitative/washer8/0_pred_crop.png}}&
\frame{\includegraphics[height=\imageheight, ]{figures/experiments/qualitative/washer8/proba2.png}}\\

\hline
\multicolumn{12}{c}{\textbf{with partial occlusions}}\\
\hline
\\[-0.2cm]


& \frame{\includegraphics[height=\imageheight, ]{figures/experiments/qualitative_occlusion/bus_66_33_0/ref.png}} &
\frame{\includegraphics[height=\imageheight, ]{figures/experiments/qualitative_occlusion/bus_66_33_0/query.png}} &
\frame{\includegraphics[height=\imageheight, ]{figures/experiments/qualitative_occlusion/bus_66_33_0/pizza.png}}&
\frame{\includegraphics[height=\imageheight, ]{figures/experiments/qualitative_occlusion/bus_66_33_0/0_pred_crop.png}}&
\frame{\includegraphics[height=\imageheight, ]{figures/experiments/qualitative_occlusion/bus_66_33_0/proba2.png}}&
&

\frame{\includegraphics[height=\imageheight, ]{figures/experiments/qualitative_occlusion/dishwasher_165_1_1/ref.png}} &
\frame{\includegraphics[height=\imageheight, ]{figures/experiments/qualitative_occlusion/dishwasher_165_1_1/query.png}} &
\frame{\includegraphics[height=\imageheight, ]{figures/experiments/qualitative_occlusion/dishwasher_165_1_1/pizza.png}}&
\frame{\includegraphics[height=\imageheight, ]{figures/experiments/qualitative_occlusion/dishwasher_165_1_1/0_pred_crop.png}}&
\frame{\includegraphics[height=\imageheight, ]{figures/experiments/qualitative_occlusion/dishwasher_165_1_1/proba2.png}}\\

& \frame{\includegraphics[height=\imageheight, ]{figures/experiments/qualitative_occlusion/clock_124_19_11/ref.png}} &
\frame{\includegraphics[height=\imageheight, ]{figures/experiments/qualitative_occlusion/clock_124_19_11/query.png}} &
\frame{\includegraphics[height=\imageheight, ]{figures/experiments/qualitative_occlusion/clock_124_19_11/pizza.png}}&
\frame{\includegraphics[height=\imageheight, ]{figures/experiments/qualitative_occlusion/clock_124_19_11/0_pred_crop.png}}&
\frame{\includegraphics[height=\imageheight, ]{figures/experiments/qualitative_occlusion/clock_124_19_11/proba2.png}}&
&

\frame{\includegraphics[height=\imageheight, ]{figures/experiments/qualitative_occlusion/clock_142_35_7/ref.png}} &
\frame{\includegraphics[height=\imageheight, ]{figures/experiments/qualitative_occlusion/clock_142_35_7/query.png}} &
\frame{\includegraphics[height=\imageheight, ]{figures/experiments/qualitative_occlusion/clock_142_35_7/pizza.png}}&
\frame{\includegraphics[height=\imageheight, ]{figures/experiments/qualitative_occlusion/clock_142_35_7/0_pred_crop.png}}&
\frame{\includegraphics[height=\imageheight, ]{figures/experiments/qualitative_occlusion/clock_142_35_7/proba2.png}}\\

& \frame{\includegraphics[height=\imageheight, ]{figures/teaser/bottle_4_12_0/ref.png}} &
\frame{\includegraphics[height=\imageheight, ]{figures/teaser/bottle_4_12_0/query.png}}&
\frame{\includegraphics[height=\imageheight, ]{figures/teaser/bottle_4_12_0/pizza.png}}&
\frame{\includegraphics[height=\imageheight, ]{figures/teaser/bottle_4_12_0/pred.png}}&
\frame{\includegraphics[height=\imageheight, ]{figures/teaser/bottle_4_12_0/proba2.png}}&

&
\frame{\includegraphics[height=\imageheight, ]{figures/experiments/qualitative_occlusion/bottle_0_0_5/ref.png}} &
\frame{\includegraphics[height=\imageheight, ]{figures/experiments/qualitative_occlusion/bottle_0_0_5/query.png}} &
\frame{\includegraphics[height=\imageheight, ]{figures/experiments/qualitative_occlusion/bottle_0_0_5/pizza.png}}&
\frame{\includegraphics[height=\imageheight, ]{figures/experiments/qualitative_occlusion/bottle_0_0_5/0_pred_crop.png}}&
\frame{\includegraphics[height=\imageheight, ]{figures/experiments/qualitative_occlusion/bottle_0_0_5/proba2.png}}\\

&Reference  & Query & PIZZA~\cite{nguyen_pizza_2022} & Ours & Pose distribution & &Reference  & Query & PIZZA~\cite{nguyen_pizza_2022} & Ours & Pose distribution\\
\end{tabular}
\end{tabular}
}
\vspace{-3mm}
\caption{\label{fig:qualitative} {\bf Visual results on unseen categories} from ShapeNet. An arrow indicates the pose with the highest probability as recovered by our method. We visually compare with PIZZA, which is the method with the second best performance. \textbf{We visualize the predicted poses by rendering the object from these poses, but the 3D model is only used for visualization purposes, not as input to our method. Similarly, we use the canonical pose of the 3D model to visualize this distribution, but not as input to our method.}
} 
\end{figure*}

% \vspace{-4pt}
\subsection{Comparison with the state of the art}
\label{sec:main_results}
% \vspace{-4pt}
\subsubsection{Results on ShapeNet}
Table~\ref{tab:shapeNet} summarizes the results of our method compared with the baselines discussed above.
%other methods~\cite{mariotti_semi-supervised_2020, mariotti_viewnet_2021, nguyen_pizza_2022, 3dim}. 
Under both the Acc30 and Median metrics, our method consistently achieves the best overall performance, outperforming the baselines by more than 10\% in Acc30 and 10$^o$ in Median. 
In particular, while other works produce reasonable results on unseen instances of seen training categories, they often struggle to estimate the 3D pose of objects from unseen categories. By contrast, our method works well in this case, demonstrating a better generalization ability on unseen categories.

% thanks its efficient novel synthesis in latent space.

Figure~\ref{fig:qualitative} shows some visualization results of our method on unseen categories, with and without symmetries. Our method  produces more accurate 3D poses than the baselines when there is a symmetry axis.

% \vspace*{-10pt}
\subsubsection{Results on T-LESS}

% \vspace*{-4pt}
Table~\ref{tab:tless} shows our comparison with~\cite{nguyen_pizza_2022,sundermeyer-cvpr20-multipathlearning,nguyen2022templates} on real images of T-LESS. While our method focuses on the more challenging case of using \textit{a single reference image}, \cite{nguyen2022templates,sundermeyer-cvpr20-multipathlearning} rely on ground-truth CAD models. Our method consistently outperforms the baseline PIZZA by a large margin. Interestingly, although there is still a gap compared to the SOTA~\cite{nguyen2022templates}, our method outperforms MultiPath~\cite{sundermeyer-cvpr20-multipathlearning}. Figure~\ref{fig:tless} shows results on seen and unseen objects of T-LESS.


% \vspace*{-7pt}
\subsection{Robustness to occlusions}
\label{sec:occlusions}
% \vspace*{-6pt}
To evaluate the robustness of our method against occlusions, we added random rectangle filled with Gaussian noise to the query images over the objects, in a  similar way to  Random Erasing~\cite{zhong2020random}. We vary the size of the rectangles to cover a range betwen 0\% to 25\% of the bounding box of the object.  Figures~\ref{fig:teaser} and \ref{fig:qualitative} show several examples.



\begin{table}[!t]
    \addtolength{\tabcolsep}{-2pt}
    \centering
    \scalebox{0.85}{
    \begin{tabular}{@{}l l c c  c  c  c  c }
    \toprule
    \parbox[t]{2mm}{\multirow{2}{*}{\rotatebox[origin=c]{90}{$\bm{\AccThirty}\uparrow$}}} & Method &  0\% & 5\% & 10\% & 15\%  & 20\% & 25\% \\
    \cmidrule{2-8}
    & PIZZA \cite{nguyen_pizza_2022} & 48.9 &  44.6 &  33.3 &  24.5 &  18.2 &  14.6 \\%&  33.05\\
    & NOPE (ours) & \bf 59.8 & \bf 54.3 &  \bf 48.4 &  \bf 45.1 &  \bf 43.7 &  \bf 40.5  \\% \bf 53.42\\
    \bottomrule
    \end{tabular}}
    % \vspace{-8pt}
    \caption{{\bf Robustness to partial occlusions. } We add rectangles of Gaussian noise to the query image, and vary the ratio between the area of the rectangle and the area of the object's 2D bounding box. 
    % We compare with PIZZA, the method with the second best performance in Table~\ref{tab:shapeNet}. 
    Our method remains robust under large occlusions, while PIZZA's performance decreases significantly.
    }
    \label{tab:occlusion}
\end{table}

Table~\ref{tab:occlusion} compares PIZZA, the best second performing method in our previous evaluation, to our method for different occlusion rates. Our method remains robust even under large occlusions, thanks to embedding matching. Figure~\ref{fig:qualitative} shows that our pose probabilities remain peaked on the correct maximum and shows clearly the symmetries.

\begin{table}[t]
	\centering
	\resizebox{0.88\linewidth}{!}{
	\begin{tabular}{@{}l r c c@{}}
	\toprule
    \multirow{2}{*}{\bf Method} 
    &\multirow{2}{*}{\bf \parbox{1.3cm}{Memory}}
     & 
    \multicolumn{2}{c}{\textbf{Run-time}}\\
	\cmidrule(lr){3-4}                                
	& & \textbf{Processing} & \textbf{Neighbors search} \\
	\midrule
	3DiM \cite{3dim} & 358.6 MB &  13 min &   0.31 s\\
	NOPE (ours) &  22.4 MB &  1.01 s &  0.18 s\\
    \bottomrule
	\end{tabular}}
 % \vspace{-10pt}
\caption{{\bf Average run-time}  of our method and 3DiM \cite{3dim} on a single GPU V100. We report the memory used for storing novel views, the time taken to generate novel views, and the time taken for nearest neighbor search to obtain the final prediction. }
    \label{tab:runtime}
\end{table}

\subsection{Runtime analysis} 
\label{sec:run_time}
% \vspace*{-5pt}
We report the running time of NOPE and 3DiM in Table~\ref{tab:runtime}. Our method is significantly faster than 3DiM, thanks to our strategy of predicting the embedding of novel viewpoints with a single step instead of multiple diffusion steps.


\subsection{Failure cases}
\label{sec:failureCases}

% \vspace*{-4pt}

All the methods fail to yield accurate results when evaluated on ``clock'', ``dishwasher'', ``guitar'', and ``mug'' categories, as indicated by the high median errors. As shown in Figure~\ref{fig:failureCases}, these categories except ``guitar'' are ``almost symmetric'', in the sense that only small details make the pose non-ambiguous. Our predictions using the top-3 and top-5 nearest neighbors significantly improves median errors for 90-symmetrical, 180-symmetrical objects, but not circular-symmetrical as mug objects. Additionally, guitar objects  can appear very thin under certain viewpoints.

%After visual inspection~(see Figure~\ref{fig:failureCases}), it appears that 3D models from the guitar category can be very thin under some viewpoints. 

% To check if this is really the reason for the poor performance on these two categories, we re-run the evaluation by taking the reference view for the ``guitar" category to be the view with the largest silhouette and by treating both categories as having a 180-degree symmetry. Table~\ref{tab:guitar_and_bus} presents the results of this new evaluation; the metrics are significantly better. This shows that the failures are indeed caused by views where the objects appear to be very thin, and by `quasi-symmetries'.

% \begin{table}[!t]
%     \addtolength{\tabcolsep}{-2pt}
%     \centering
%     \scalebox{0.8}{
%     \begin{tabular}{@{}l l c  c  c  c  c@{}}
%     \toprule
%     & Category & ViewNet \cite{mariotti_viewnet_2021} & SSVE \cite{mariotti_semi-supervised_2020} & PIZZA \cite{nguyen_pizza_2022}  & 3DiM \cite{3dim} & Ours \\
%     \midrule
%     \parbox[t]{2mm}{\multirow{4}{*}{\rotatebox[origin=c]{90}{$\bm{\MedErr}\downarrow$}}} & guitar & 70.43 & 67.52 & 60.12 & 55.09 & \bf 51.52\\
%     & bus    & 75.02 & 84.42 & 65.64 & 59.02 & \bf 57.64\\
%     \cmidrule{2-7}
%     & \textdagger{guitar} & 31.05 & 25.46 & 20.87 & 20.05 & \bf 18.65\\
%     & \textdagger{bus}    & 19.26 & 23.45 & 17.03 & 13.25 & \bf 12.51\\
%     \bottomrule
%     \end{tabular}}
%     \caption{{\bf Additional experiments on the ``guitar'' and ``bus'' categories. } We compare the base results from Table~\ref{tab:shapeNet} (\textbf{top})  with an evaluation in which we fix the reference viewpoints of the objects from the ``guitar'' category, and we allow a 180$^{\circ}$ symmetry in the error metric  (\textbf{bottom} with \textdagger{}). We report the median of the errors for each method. The values are significantly better than the values reported in Table~\ref{tab:shapeNet}, which validate the reasons for the poor results on these two categories in the general case.
%     }
%     \label{tab:guitar_and_bus}
%     \vspace{-2pt}
% \end{table}
