\subsection{Motivation}
To motivate our approach, we evaluate the use of the state-of-the-art 3D model-based image generation method (3DiM)~\cite{3dim} for pose estimation. 3DiM is a diffusion-based method for view synthesis in pixel space. To apply it to pose prediction given a reference image of an object, we experimented with using it to generate a set of novel views from many different viewpoints and match a query image of the object to these views in pixel space. Since the generated views are annotated with the corresponding pose, this gives us a pose estimate. 


As shown in Figure~\ref{fig:motivation}, the images generated by 3DiM look very realistic. However, the recovered poses are not very accurate. This can be explained in part by the fact that diffusion can ``invent'' details that disturb image matching. The limitations of this approach will further be quantitatively demonstrated in our experiments (Table~\ref{tab:shapeNet}). This motivates us to learn to directly generate a discriminative representation of the views.


\begin{figure}[t]
\newlength{\plotheight}
\setlength\plotheight{1.5cm}
\centering
\setlength\lineskip{2.0pt}
\setlength\tabcolsep{2.0pt} 
{\small
\begin{tabular}{
% 
>{\centering\arraybackslash}m{\plotheight}
>{\centering\arraybackslash}m{\plotheight}
>{\centering\arraybackslash}m{\plotheight}%|
>{\centering\arraybackslash}m{\plotheight}%|
>{\centering\arraybackslash}m{\plotheight}
}
 & & Generated & Recovered &  \\
 & & view from & pose by & Estimated \\
 & & the query & template &  pose \\
Reference & Query & GT pose & matching & distribution \\
\frame{\includegraphics[height=\plotheight, ]{figures/method/motivation/sample1/ref.png}}&
\frame{\includegraphics[height=\plotheight, ]{figures/method/motivation/sample1/query.png}}&
\frame{\includegraphics[height=\plotheight, ]{figures/method/motivation/sample1/3dim.png}}&
\frame{\includegraphics[height=\plotheight, ]{figures/method/motivation/sample1/pose.png}}&
\frame{\includegraphics[height=\plotheight, ]{figures/method/motivation/sample1/proba.png}}\\[0em]
%\hline
\frame{\includegraphics[height=\plotheight, ]{figures/method/motivation/sample2/ref.png}}&
\frame{\includegraphics[height=\plotheight, ]{figures/method/motivation/sample2/query.png}}&
\frame{\includegraphics[height=\plotheight, ]{figures/method/motivation/sample2/3dim.png}}&
\frame{\includegraphics[height=\plotheight, ]{figures/method/motivation/sample2/pose.png}}&
\frame{\includegraphics[height=\plotheight, ]{figures/method/motivation/sample2/proba.png}}\\
\end{tabular}
% \begin{tabular}{
% >{\centering\arraybackslash}m{0.3\teaserheight}
% }
% % \includegraphics[height=3\teaserheight, ]{figures/teaser/colorbar.png}\\
% \end{tabular}
}
\vspace{-4mm}
\caption{
\textbf{The limit of image generation with diffusion models for pose prediction.} While the images generated by 3DiM look very realistic, they may change the appearance of the object, impairing the similarity computation between the query image and the generated view, and hence the pose estimation when used for our problem. Notice that the probability distributions does not peak on the right pose but shows many wrong local maxima.
}
\label{fig:motivation}
\end{figure}
