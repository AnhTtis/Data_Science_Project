% based on the CVPR template provided by Ming-Ming Cheng (https://github.com/MCG-NKU/CVPR_Template)
% modified and extended by Stefan Roth (stefan.roth@NOSPAMtu-darmstadt.de)

\documentclass[10pt,twocolumn,letterpaper]{article}
\usepackage[review]{cvpr} % To force page numbers, e.g. for an arXiv version
\usepackage{graphicx}
% Include other packages here, before hyperref.
\usepackage{stfloats}

\usepackage{mathtools} % \coloneqq :=
\usepackage{graphicx}
\usepackage{amsmath}
\usepackage{amssymb}
\usepackage{booktabs}
\usepackage{dsfont}
\usepackage{multirow}
\usepackage{multicol}
\usepackage[pagebackref,breaklinks,colorlinks]{hyperref}
\usepackage[table]{xcolor}% http://ctan.org/pkg/xcolor

% Support for easy cross-referencing
\usepackage[capitalize]{cleveref}
\crefname{section}{Sec.}{Secs.}
\Crefname{section}{Section}{Sections}
\Crefname{table}{Table}{Tables}
\crefname{table}{Tab.}{Tabs.}
% TODOs
\newcommand{\TODO}[1]{\textcolor{magenta}{TODO: #1}}
\newcommand{\UPDATE}[1]{\hl{magenta}{TODO: #1}}
\newcommand{\placeholder}[1]{\textcolor{lightgray}{#1}}

% Comments
\newcommand{\SR}[1]{\textcolor{blue}{SR: #1}}
\newcommand{\WW}[1]{\textcolor{violet}{WW: #1}}
\newcommand{\TK}[1]{\textcolor{red}{TK: #1}}
\newcommand{\TA}[1]{\textcolor{orange}{#1}}  % TH already defined, so used TA


% Sections
\renewcommand*\sectionautorefname{\Snospace}
\def\sectionautorefname{Sec.}
\def\subsectionautorefname{Sec.}
\def\subsubsectionautorefname{Sec.}
\def\figureautorefname{Fig.}


% Figures
\newcommand{\insertFigure}[5]{
    \begin{figure}[t]
      \centering
      \includegraphics[width=#3\linewidth]{figure/#1}
      \vspace{#4}
      \caption{#2}
      \label{fig:#1}
      \vspace{#5}
    \end{figure}
}

\newcommand{\insertFigureWide}[5]{
    \begin{figure*}[t]
      \centering
      \includegraphics[width=#3\linewidth]{figure/#1}
      \vspace{#4}
      \caption{#2}
      \label{fig:#1}
      \vspace{#5}
    \end{figure*}
}


% List
\newcommand{\squishlist}{
 \begin{list}{$\bullet$}
  { \setlength{\itemsep}{0pt}
     \setlength{\parsep}{3pt}
     \setlength{\topsep}{3pt}
     \setlength{\partopsep}{0pt}
     \setlength{\leftmargin}{1.5em}
     \setlength{\labelwidth}{1em}
     \setlength{\labelsep}{0.5em} } }
     
\newcommand{\squishend}{
  \end{list}  }


% Equations 
\newcommand{\insertEquation}[1]{%
\vskip -0.20in
\begin{gather}
    #1
\end{gather}
% \vskip -0.12in
}

\newcommand{\insertEqNoNum}[1]{%
\vskip -0.20in
\begin{gather*}
    #1
\end{gather*}
% \vskip -0.12in
}

% circled numbers
\newcommand*\circled[1]{\tikz[baseline=(char.base)]{\node[shape=circle,fill,inner sep=0.7pt] (char) {\textcolor{white}{#1}};}}

%%%%%%%%% PAPER ID  - PLEASE UPDATE
\def\cvprPaperID{6994} % *** Enter the CVPR Paper ID here
\def\confName{ICCV}
\def\confYear{2023}
\title{Unseen Object Pose Estimation from a Single Image}
\title{NOPE: Novel Object Pose Estimation from a Single Image \\


Supplementary Material
} 

\author{Van Nguyen Nguyen$^{1}$,  Thibault Groueix$^{2}$, Yinlin Hu$^{3}$, Mathieu Salzmann$^{4}$,  Vincent Lepetit$^{1}$\\ % To be defined later
{$^{1}$LIGM, Ecole des Ponts}, {$^{2}$Adobe}, {$^{3}$MagicLeap}, {$^{4}$EPFL}\\
}

\begin{document}
\definecolor{DarkMagenta}{rgb}{0.7, 0.0, 0.7}
\newcommand{\nguyen}[1]{{\color{DarkMagenta}#1}}
\newcommand{\nguyenrmk}[1]{{\color{DarkMagenta} {\bf [VN: #1]}}}

\definecolor{DarkBlue}{rgb}{0.0, 0.0, 0.8}
\newcommand{\thibault}[1]{{\color{DarkBlue} #1}}
\newcommand{\thibaultrmk}[1]{{\color{DarkBlue} {\bf [TG: #1]}}}

\definecolor{DarkOrange}{rgb}{1.0, 0.55, 0.0}
\newcommand{\ms}[1]{{\color{DarkOrange}#1}}
\newcommand{\mathieurmk}[1]{{\color{DarkOrange} {\bf [MS: #1]}}}

\definecolor{DarkGreen}{rgb}{0.0, 0.5, 0.0}
\newcommand{\vincent}[1]{{\color{DarkGreen} #1}}
\newcommand{\vincentrmk}[1]{{\color{DarkGreen} {\bf [VL: #1]}}}

\newcommand{\was}[1]{}

\definecolor{Random}{rgb}{0.1, 0.5, 0.9}
\newcommand{\yinlin}[1]{{\color{Random} #1}}
\newcommand{\yinlinrmk}[1]{{\color{Random} {\bf [YH: #1]}}}


% Uncomment for submitted version:
\renewcommand{\nguyen}[1]{#1}
\renewcommand{\nguyenrmk}[1]{}
\renewcommand{\thibault}[1]{#1}
\renewcommand{\thibaultrmk}[1]{}
\renewcommand{\yinlin}[1]{#1}
\renewcommand{\yinlinrmk}[1]{#1}
\renewcommand{\ms}[1]{#1}
\renewcommand{\mathieurmk}[1]{}
\renewcommand{\vincent}[1]{#1}
\renewcommand{\vincentrmk}[1]{}

\maketitle

\begin{table*}[!b]
\addtolength{\tabcolsep}{1pt}
\centering
    \scalebox{.65}{
    \begin{tabular}{@{}l l | ccccccccccc | c@{}}
	\toprule
	& Method &  novel inst. & bottle$^*$ & bus & clock & dishwasher & guitar & mug & pistol & skateboard & train & washer & mean \\
	\midrule
	\parbox[t]{2mm}{\multirow{7}{*}{\rotatebox[origin=c]{90}{$\bm{\AccThirty}\uparrow$}}} 
	& ViewNet \cite{mariotti_viewnet_2021} & {\bf 77.5} $\pm$ 4.2 &  48.4 $\pm$ 2.4 &  36.2 $\pm$ 4.8 &  23.5 $\pm$ 3.1 &  16.4 $\pm$ 4.1 &  37.8 $\pm$ 8.1 &  31.3 $\pm$ 5.2 &  17.9 $\pm$ 4.1 &  33.9 $\pm$ 4.8 &  44.8 $\pm$ 4.1 &  25.1 $\pm$ 4.8 &  35.7 $\pm$ 4.4  \\
	& SSVE \cite{mariotti_semi-supervised_2020} & 75.3 $\pm$ 3.2 &  61.5 $\pm$ 2.2 &  38.2 $\pm$ 3.7 &  41.8 $\pm$ 3.1 &  21.3 $\pm$ 2.9 &  46.8 $\pm$ 8.7 &  38.4 $\pm$ 4.6 &  36.8 $\pm$ 3.6 & {\bf 62.3} $\pm$ 4.1 &  41.5 $\pm$ 3.5 &  50.8 $\pm$ 3.5 &  46.8 $\pm$ 4.6  \\
	& PIZZA \cite{nguyen_pizza_2022} &  72.3 $\pm$ 3.5 &  76.0 $\pm$ 2.8 &  38.6 $\pm$ 5.1 &  38.5 $\pm$ 2.7 &  32.6 $\pm$ 5.1 &  30.8 $\pm$ 8.6 &  35.6 $\pm$ 4.2 &  40.4 $\pm$ 2.3 &  58.3 $\pm$ 3.1 &  52.9 $\pm$ 5.1 &  {\bf 61.0} $\pm$ 4.6 &  48.8 $\pm$ 3.8 \\
	& 3DiM \cite{3dim} & 77.3 $\pm$ 2.1 &  95.1 $\pm$ 1.9 &  43.5 $\pm$ 4.1 &  23.6 $\pm$ 3.2 &  24.5 $\pm$ 2.6 &  36.0 $\pm$ 5.3 &  32.0 $\pm$ 3.1 &  31.9 $\pm$ 2.5 &  50.3 $\pm$ 3.5 &  37.0 $\pm$ 3.2 &  56.1 $\pm$ 2.9 &  46.1 $\pm$ 3.6	\\
	
	& Ours (top 1) & 75.5 $\pm$ 1.3 &  {\bf 96.0 } $\pm$ 1.1 &  {\bf 53.6} $\pm$ 3.2 &  {\bf 48.0} $\pm$ 2.1 &  {\bf 48.0} $\pm$ 1.8 &  {\bf 49.0} $\pm$ 5.1 &  {\bf 44.6} $\pm$ 2.3 &  {\bf 69.0} $\pm$ 3.1 &  57.8 $\pm$ 2.8 &  {\bf 55.2} $\pm$ 2.3 &  60.6 $\pm$ 2.0 &  {\bf  59.8} $\pm$ 3.1 \\
	\cmidrule{2-14}
	& Ours (top 3) & 92.0 $\pm$ 1.5 & 97.4 $\pm$ 0.9 & 83.8 $\pm$ 2.5 & 73.4 $\pm$ 3.6 & 78.5 $\pm$ 2.2 & 66.8 $\pm$ 4.5 & 56.0 $\pm$ 2.0 & 83.8 $\pm$ 3.1 & 86.2 $\pm$ 2.2 & 86.0 $\pm$ 1.9 & 84.4 $\pm$ 2.0 & 80.8 $\pm$ 2.1 \\
	& Ours (top 5) & 95.5 $\pm$ 0.8 & 97.8 $\pm$ 0.2 & 89.8 $\pm$ 2.0 & 80.4 $\pm$ 3.0 & 88.2 $\pm$ 2.0 & 74.6 $\pm$ 4.6 & 62.8 $\pm$ 2.1 & 88.4 $\pm$ 2.6 & 92.8 $\pm$ 2.0 & 95.4 $\pm$ 1.7 & 93.4 $\pm$ 2.1 & 87.1 $\pm$ 1.7 \\
    \midrule
    \parbox[t]{2mm}{\multirow{7}{*}{\rotatebox[origin=c]{90}{$\bm{\MedErr}\downarrow$}}}  & ViewNet \cite{mariotti_viewnet_2021} &  6.6 $\pm$ 5.0 & 26.7 $\pm$ 3.5 & 35.8 $\pm$ 5.1 & 40.3 $\pm$ 4.6 & 96.3 $\pm$ 3.0 & 50.6 $\pm$ 6.9 & 51.6 $\pm$ 4.1 & 42.8 $\pm$ 4.1 & 37.4 $\pm$ 3.7 & 26.8 $\pm$ 3.1 & 44.3 $\pm$ 2.9 & 41.7 $\pm$ 4.9 \\
    & SSVE \cite{mariotti_semi-supervised_2020} & 6.1 $\pm$ 3.1 & 23.8 $\pm$ 2.6 & 45.2 $\pm$ 3.0 & 41.9 $\pm$ 3.3 & 90.4 $\pm$ 2.3 & 47.6 $\pm$ 7.6 & 49.6 $\pm$ 4.1 & 24.0 $\pm$ 2.9 &  13.5 $\pm$ 3.8 & 24.9 $\pm$ 4.1 & 48.1 $\pm$ 3.8 & 37.7 $\pm$ 4.2\\
    & PIZZA     \cite{nguyen_pizza_2022} &5.8 $\pm$ 2.6 & 25.5 $\pm$ 3.4 & 26.4 $\pm$ 4.1 & 43.2 $\pm$ 4.1 & 80.6 $\pm$ 2.7 & 40.2 $\pm$ 5.1 & 45.5 $\pm$ 4.5 & 23.4 $\pm$ 2.1 & 17.3 $\pm$ 3.1 & 20.3 $\pm$ 5.1 & 38.5 $\pm$ 5.1 & 33.3 $\pm$ 4.7
 \\
    & 3DiM \cite{3dim} & {\bf 5.7} $\pm$ 3.1 & {\bf 1.8} $\pm$ 2.4 & 19.8 $\pm$ 3.5 & 47.3 $\pm$ 3.7 & 98.8 $\pm$ 2.9 & 35.2 $\pm$ 5.8 & 35.7 $\pm$ 2.5 & 21.2 $\pm$ 2.8 & {\bf 12.5} $\pm$ 2.9 & {\bf 17.6} $\pm$ 2.5 & 19.2 $\pm$ 2.6 & 28.6 $\pm$ 3.0 \\
	& Ours (top 1) & 8.1 $\pm$ 1.3 & {\bf 1.8} $\pm$ 1.1 & {\bf 18.4} $\pm$ 3.2 & {\bf 39.9} $\pm$ 2.1 & {\bf 77.6} $\pm$ 1.8 & {\bf 31.6} $\pm$ 5.1 & {\bf 35.5} $\pm$ 2.3 & {\bf 13.4} $\pm$ 3.1 & 15.5 $\pm$ 2.8 & 18.3 $\pm$ 2.3 & {\bf 8.5} $\pm$ 2.0 & {\bf 24.4} $\pm$ 3.1\\
	\cmidrule{2-14}
	& Ours (top 3) & 5.0 $\pm$ 1.0 & 1.3 $\pm$ 0.7 & 5.8 $\pm$ 3.6 & 9.1 $\pm$ 2.1 & 4.8 $\pm$ 2.2 & 16.0 $\pm$ 4.1 & 22.6 $\pm$ 3.1 & 8.1 $\pm$ 1.8 & 6.5 $\pm$ 1.6 & 6.7 $\pm$ 2.6 & 5.7 $\pm$ 2.6 & 8.3 $\pm$ 2.4\\
	& Ours (top 5) & 4.5 $\pm$ 0.8 & 1.2 $\pm$ 0.6 & 4.5 $\pm$ 3.0 & 7.1 $\pm$ 2.0 & 4.4 $\pm$ 2.1 & 11.6 $\pm$ 3.1 & 18.4 $\pm$ 2.6 & 6.1 $\pm$ 2.4 & 5.6 $\pm$ 1.7 & 4.9 $\pm$ 2.0 & 5.0 $\pm$ 2.8 & 6.6 $\pm$ 2.1\\
	\bottomrule
	\end{tabular}
    }
    % \vspace*{-5mm}
    \caption{{\bf Quantitative results on ShapeNet dataset.} We treat “bottle" as a symmetric category, i.e., the error is only the difference of elevation angle. Since the quality of prediction may depend on the reference image, we report \nguyen{the average and the standard deviation} of 5 runs with 5 different reference images.  
    }
    % We provide the std of this score in the supplementary material. 
    
    %\nguyenrmk{Yes, it's time reason. And yes again, 'unseen ins.' corresponds to unseen instances of training categories. It's out of scope, I agree but since we re-implemented all these methods, it's important to show that it works correctly for seen categories so the reviews can know we re-implement them successfully.}
  \label{tab:resultStd}
  \end{table*} 

\section{Implementation details}
Our implementation of U-Net is mainly based on denoising-diffusion-pytorch \cite{lucidrains2022}. 
%The U-Net architecture is composed of two primary components: the encoder block and the decoder block. 
The encoder receives as input an image embedding of size $8\times32\times32$ and encodes it into a $1024\times4\times4$ representation. The decoder takes as input  a this $1024\times4\times4$ representation and intermediate representations from the encoder through skip connections, as inputs and decodes them into an embedding for a new view of the original size $8\times32\times32$.

Both the encoder and decoder consist of four layers, each of which comprises a residual block, a cross-attention layer that facilitates the injection of pose embeddings into feature maps, and a convolution followed by group normalization and the SiLU activation function.

\begin{figure}[!t]
    \begin{center}
    \includegraphics[width=1.0\linewidth]{figures/supplementary/u_net_architecture.png}
    \end{center}
    \caption{
        \label{fig:uNetArchitecture}
        {Architecture of our U-Net.} 
        \vspace{-5pt} 
        }
\end{figure}

We show in Figure~\ref{fig:uNetArchitecture} an overview of the U-Net architecture we use to generate the novel views' embeddings.


\section{Additional results}

We provide in Table~\ref{tab:resultStd} the 3D pose estimation results of the baselines \cite{nguyen_pizza_2022, mariotti_semi-supervised_2020, mariotti_viewnet_2021, 3dim} and our method on the ShapeNet dataset with standard deviation computed on 5 runs with 5 different reference images.

We also show in Figures~\ref{fig:resultsNoOcclusion} and \ref{fig:resultsOcclusion} some additional results on \textbf{unseen categories} without and with occlusions.



\begin{figure*}
\newlength{\imageheight}
\setlength\imageheight{1.50cm}
\centering
\setlength\lineskip{3.pt}
\setlength\tabcolsep{2.pt} 
{\small
\begin{tabular}{c}
\begin{tabular}{
>{\centering\arraybackslash}m{0.5cm}
>{\centering\arraybackslash}m{\imageheight}
>{\centering\arraybackslash}m{\imageheight}
>{\centering\arraybackslash}m{\imageheight}
>{\centering\arraybackslash}m{\imageheight}
>{\centering\arraybackslash}m{\imageheight}
>{\centering\arraybackslash}m{0.5cm}
>{\centering\arraybackslash}m{\imageheight}
>{\centering\arraybackslash}m{\imageheight}
>{\centering\arraybackslash}m{\imageheight}
>{\centering\arraybackslash}m{\imageheight}
>{\centering\arraybackslash}m{\imageheight}
}
\hline
\multicolumn{12}{c}{\textbf{without occlusions}}\\
\hline
\\[-0.2cm]
\parbox[t]{0mm}{\rotatebox[origin=c]{90}{Bottle}} & \frame{\includegraphics[height=\imageheight, ]{figures/supplementary/bottle_41_0_10/ref.png}} &
\frame{\includegraphics[height=\imageheight, ]{figures/supplementary/bottle_41_0_10/1_query_crop.png}} &
\frame{\includegraphics[height=\imageheight, ]{figures/supplementary/bottle_41_0_10/pizza.png}}&
\frame{\includegraphics[height=\imageheight, ]{figures/supplementary/bottle_41_0_10/0_pred_crop.png}}&
\frame{\includegraphics[height=\imageheight, ]{figures/supplementary/bottle_41_0_10/proba.png}}&
&
\frame{\includegraphics[height=\imageheight, ]{figures/supplementary/bottle_53_3_9/ref.png}} &
\frame{\includegraphics[height=\imageheight, ]{figures/supplementary/bottle_53_3_9/1_query_crop.png}} &
\frame{\includegraphics[height=\imageheight, ]{figures/supplementary/bottle_53_3_9/pizza.png}}&
\frame{\includegraphics[height=\imageheight, ]{figures/supplementary/bottle_53_3_9/0_pred_crop.png}}&
\frame{\includegraphics[height=\imageheight, ]{figures/supplementary/bottle_53_3_9/proba.png}}\\

\parbox[t]{0mm}{\rotatebox[origin=c]{90}{Bus}} & \frame{\includegraphics[height=\imageheight, ]{figures/supplementary/bus_5_2_12/ref.png}} &
\frame{\includegraphics[height=\imageheight, ]{figures/supplementary/bus_5_2_12/1_query_crop.png}} &
\frame{\includegraphics[height=\imageheight, ]{figures/supplementary/bus_5_2_12/pizza.png}}&
\frame{\includegraphics[height=\imageheight, ]{figures/supplementary/bus_5_2_12/0_pred_crop.png}}&
\frame{\includegraphics[height=\imageheight, ]{figures/supplementary/bus_5_2_12/proba.png}}&
&
\frame{\includegraphics[height=\imageheight, ]{figures/supplementary/bus_79_3_13/ref.png}} &
\frame{\includegraphics[height=\imageheight, ]{figures/supplementary/bus_79_3_13/1_query_crop.png}} &
\frame{\includegraphics[height=\imageheight, ]{figures/supplementary/bus_79_3_13/pizza.png}}&
\frame{\includegraphics[height=\imageheight, ]{figures/supplementary/bus_79_3_13/0_pred_crop.png}}&
\frame{\includegraphics[height=\imageheight, ]{figures/supplementary/bus_79_3_13/proba.png}}\\

\parbox[t]{0mm}{\rotatebox[origin=c]{90}{Clock}} & \frame{\includegraphics[height=\imageheight, ]{figures/supplementary/clock_43_0_8/ref.png}} &
\frame{\includegraphics[height=\imageheight, ]{figures/supplementary/clock_43_0_8/1_query_crop.png}} &
\frame{\includegraphics[height=\imageheight, ]{figures/supplementary/clock_43_0_8/pizza.png}}&
\frame{\includegraphics[height=\imageheight, ]{figures/supplementary/clock_43_0_8/0_pred_crop.png}}&
\frame{\includegraphics[height=\imageheight, ]{figures/supplementary/clock_43_0_8/proba.png}}&
&
\frame{\includegraphics[height=\imageheight, ]{figures/supplementary/clock_52_2_12/ref.png}} &
\frame{\includegraphics[height=\imageheight, ]{figures/supplementary/clock_52_2_12/1_query_crop.png}} &
\frame{\includegraphics[height=\imageheight, ]{figures/supplementary/clock_52_2_12/pizza.png}}&
\frame{\includegraphics[height=\imageheight, ]{figures/supplementary/clock_52_2_12/0_pred_crop.png}}&
\frame{\includegraphics[height=\imageheight, ]{figures/supplementary/clock_52_2_12/proba.png}}\\


\parbox[t]{0mm}{\rotatebox[origin=l]{90}{Dishwasher}} & \frame{\includegraphics[height=\imageheight, ]{figures/supplementary/dishwasher_18_0_4/ref.png}} &
\frame{\includegraphics[height=\imageheight, ]{figures/supplementary/dishwasher_18_0_4/1_query_crop.png}} &
\frame{\includegraphics[height=\imageheight, ]{figures/supplementary/dishwasher_18_0_4/pizza.png}}&
\frame{\includegraphics[height=\imageheight, ]{figures/supplementary/dishwasher_18_0_4/0_pred_crop.png}}&
\frame{\includegraphics[height=\imageheight, ]{figures/supplementary/dishwasher_18_0_4/proba.png}}&
&
\frame{\includegraphics[height=\imageheight, ]{figures/supplementary/dishwasher_19_4_2/ref.png}} &
\frame{\includegraphics[height=\imageheight, ]{figures/supplementary/dishwasher_19_4_2/1_query_crop.png}} &
\frame{\includegraphics[height=\imageheight, ]{figures/supplementary/dishwasher_19_4_2/pizza.png}}&
\frame{\includegraphics[height=\imageheight, ]{figures/supplementary/dishwasher_19_4_2/0_pred_crop.png}}&
\frame{\includegraphics[height=\imageheight, ]{figures/supplementary/dishwasher_19_4_2/proba.png}}\\

\parbox[t]{0mm}{\rotatebox[origin=c]{90}{Guitar}} & \frame{\includegraphics[height=\imageheight, ]{figures/supplementary/guitar_38_7_8/ref.png}} &
\frame{\includegraphics[height=\imageheight, ]{figures/supplementary/guitar_38_7_8/1_query_crop.png}} &
\frame{\includegraphics[height=\imageheight, ]{figures/supplementary/guitar_38_7_8/pizza.png}}&
\frame{\includegraphics[height=\imageheight, ]{figures/supplementary/guitar_38_7_8/0_pred_crop.png}}&
\frame{\includegraphics[height=\imageheight, ]{figures/supplementary/guitar_38_7_8/proba.png}}&
&
\frame{\includegraphics[height=\imageheight, ]{figures/supplementary/guitar_55_4_13/ref.png}} &
\frame{\includegraphics[height=\imageheight, ]{figures/supplementary/guitar_55_4_13/1_query_crop.png}} &
\frame{\includegraphics[height=\imageheight, ]{figures/supplementary/guitar_55_4_13/pizza.png}}&
\frame{\includegraphics[height=\imageheight, ]{figures/supplementary/guitar_55_4_13/0_pred_crop.png}}&
\frame{\includegraphics[height=\imageheight, ]{figures/supplementary/guitar_55_4_13/proba.png}}\\

\parbox[t]{0mm}{\rotatebox[origin=c]{90}{Mug}} & \frame{\includegraphics[height=\imageheight, ]{figures/supplementary/mug_16_4_11/ref.png}} &
\frame{\includegraphics[height=\imageheight, ]{figures/supplementary/mug_16_4_11/1_query_crop.png}} &
\frame{\includegraphics[height=\imageheight, ]{figures/supplementary/mug_16_4_11/pizza.png}}&
\frame{\includegraphics[height=\imageheight, ]{figures/supplementary/mug_16_4_11/0_pred_crop.png}}&
\frame{\includegraphics[height=\imageheight, ]{figures/supplementary/mug_16_4_11/proba.png}}&
&
\frame{\includegraphics[height=\imageheight, ]{figures/supplementary/mug_37_7_1/ref.png}} &
\frame{\includegraphics[height=\imageheight, ]{figures/supplementary/mug_37_7_1/1_query_crop.png}} &
\frame{\includegraphics[height=\imageheight, ]{figures/supplementary/mug_37_7_1/pizza.png}}&
\frame{\includegraphics[height=\imageheight, ]{figures/supplementary/mug_37_7_1/0_pred_crop.png}}&
\frame{\includegraphics[height=\imageheight, ]{figures/supplementary/mug_37_7_1/proba.png}}\\

\parbox[t]{0mm}{\rotatebox[origin=c]{90}{Pistol}} & \frame{\includegraphics[height=\imageheight, ]{figures/supplementary/pistol_20_0_3/ref.png}} &
\frame{\includegraphics[height=\imageheight, ]{figures/supplementary/pistol_20_0_3/1_query_crop.png}} &
\frame{\includegraphics[height=\imageheight, ]{figures/supplementary/pistol_20_0_3/pizza.png}}&
\frame{\includegraphics[height=\imageheight, ]{figures/supplementary/pistol_20_0_3/0_pred_crop.png}}&
\frame{\includegraphics[height=\imageheight, ]{figures/supplementary/pistol_20_0_3/proba.png}}&
&
\frame{\includegraphics[height=\imageheight, ]{figures/supplementary/pistol_21_2_9/ref.png}} &
\frame{\includegraphics[height=\imageheight, ]{figures/supplementary/pistol_21_2_9/1_query_crop.png}} &
\frame{\includegraphics[height=\imageheight, ]{figures/supplementary/pistol_21_2_9/pizza.png}}&
\frame{\includegraphics[height=\imageheight, ]{figures/supplementary/pistol_21_2_9/0_pred_crop.png}}&
\frame{\includegraphics[height=\imageheight, ]{figures/supplementary/pistol_21_2_9/proba.png}}\\

\parbox[t]{1mm}{\rotatebox[origin=l]{90}{Skateboard}} & \frame{\includegraphics[height=\imageheight, ]{figures/supplementary/skateboard_3_6_3/ref.png}} &
\frame{\includegraphics[height=\imageheight, ]{figures/supplementary/skateboard_3_6_3/1_query_crop.png}} &
\frame{\includegraphics[height=\imageheight, ]{figures/supplementary/skateboard_3_6_3/pizza.png}}&
\frame{\includegraphics[height=\imageheight, ]{figures/supplementary/skateboard_3_6_3/0_pred_crop.png}}&
\frame{\includegraphics[height=\imageheight, ]{figures/supplementary/skateboard_3_6_3/proba.png}}&
&
\frame{\includegraphics[height=\imageheight, ]{figures/supplementary/skateboard_14_2_5/ref.png}} &
\frame{\includegraphics[height=\imageheight, ]{figures/supplementary/skateboard_14_2_5/1_query_crop.png}} &
\frame{\includegraphics[height=\imageheight, ]{figures/supplementary/skateboard_14_2_5/pizza.png}}&
\frame{\includegraphics[height=\imageheight, ]{figures/supplementary/skateboard_14_2_5/0_pred_crop.png}}&
\frame{\includegraphics[height=\imageheight, ]{figures/supplementary/skateboard_14_2_5/proba.png}}\\

\parbox[t]{1mm}{\rotatebox[origin=c]{90}{Train}} & \frame{\includegraphics[height=\imageheight, ]{figures/supplementary/train_9__15/ref.png}} &
\frame{\includegraphics[height=\imageheight, ]{figures/supplementary/train_9__15/1_query_crop.png}} &
\frame{\includegraphics[height=\imageheight, ]{figures/supplementary/train_9__15/pizza.png}}&
\frame{\includegraphics[height=\imageheight, ]{figures/supplementary/train_9__15/0_pred_crop.png}}&
\frame{\includegraphics[height=\imageheight, ]{figures/supplementary/train_9__15/proba.png}}&
&
\frame{\includegraphics[height=\imageheight, ]{figures/supplementary/train_46__6/ref.png}} &
\frame{\includegraphics[height=\imageheight, ]{figures/supplementary/train_46__6/1_query_crop.png}} &
\frame{\includegraphics[height=\imageheight, ]{figures/supplementary/train_46__6/pizza.png}}&
\frame{\includegraphics[height=\imageheight, ]{figures/supplementary/train_46__6/0_pred_crop.png}}&
\frame{\includegraphics[height=\imageheight, ]{figures/supplementary/train_46__6/proba.png}}\\

\parbox[t]{1mm}{\rotatebox[origin=c]{90}{Washer}} & \frame{\includegraphics[height=\imageheight, ]{figures/supplementary/washer_12_2_12/ref.png}} &
\frame{\includegraphics[height=\imageheight, ]{figures/supplementary/washer_12_2_12/1_query_crop.png}} &
\frame{\includegraphics[height=\imageheight, ]{figures/supplementary/washer_12_2_12/pizza.png}}&
\frame{\includegraphics[height=\imageheight, ]{figures/supplementary/washer_12_2_12/0_pred_crop.png}}&
\frame{\includegraphics[height=\imageheight, ]{figures/supplementary/washer_12_2_12/proba.png}}&
&
\frame{\includegraphics[height=\imageheight, ]{figures/supplementary/washer_69_5_9/ref.png}} &
\frame{\includegraphics[height=\imageheight, ]{figures/supplementary/washer_69_5_9/1_query_crop.png}} &
\frame{\includegraphics[height=\imageheight, ]{figures/supplementary/washer_69_5_9/pizza.png}}&
\frame{\includegraphics[height=\imageheight, ]{figures/supplementary/washer_69_5_9/0_pred_crop.png}}&
\frame{\includegraphics[height=\imageheight, ]{figures/supplementary/washer_69_5_9/proba.png}}\\

&Reference  & Query & PIZZA~\cite{nguyen_pizza_2022} & Ours & Pose distribution & &Reference  & Query & PIZZA~\cite{nguyen_pizza_2022} & Ours & Pose distribution\\
\end{tabular}
\end{tabular}
}
% \vspace{3mm}
\caption{\label{fig:resultsNoOcclusion} {\bf Visual results on unseen categories and occlusions}. The arrow indicates the pose with the highest probability as recovered by our method. We visually compare to PIZZA, which is the method with the second best performance. \textbf{We visualize the predicted poses by rendering the object from these poses, but the 3D model is only used for visualization purposes, not as input to our method. Similarly, we use the canonical pose of the 3D model to visualize this distribution, but not as input to our method.}
} 
\end{figure*}

\begin{figure*}
\newlength{\imageheight}
\setlength\imageheight{1.50cm}
\centering
\setlength\lineskip{3.pt}
\setlength\tabcolsep{2.pt} 
{\small
\begin{tabular}{c}
\begin{tabular}{
>{\centering\arraybackslash}m{0.5cm}
>{\centering\arraybackslash}m{\imageheight}
>{\centering\arraybackslash}m{\imageheight}
>{\centering\arraybackslash}m{\imageheight}
>{\centering\arraybackslash}m{\imageheight}
>{\centering\arraybackslash}m{\imageheight}
>{\centering\arraybackslash}m{0.5cm}
>{\centering\arraybackslash}m{\imageheight}
>{\centering\arraybackslash}m{\imageheight}
>{\centering\arraybackslash}m{\imageheight}
>{\centering\arraybackslash}m{\imageheight}
>{\centering\arraybackslash}m{\imageheight}
}
\hline
\multicolumn{12}{c}{\textbf{with occlusions}}\\
\hline
\\[-0.2cm]
\parbox[t]{0mm}{\rotatebox[origin=c]{90}{Bottle}} & \frame{\includegraphics[height=\imageheight, ]{figures/supplementary_occlusion/bottle_11_1_6/ref.png}} &
\frame{\includegraphics[height=\imageheight, ]{figures/supplementary_occlusion/bottle_11_1_6/1_query_crop.png}} &
\frame{\includegraphics[height=\imageheight, ]{figures/supplementary_occlusion/bottle_11_1_6/pizza.png}}&
\frame{\includegraphics[height=\imageheight, ]{figures/supplementary_occlusion/bottle_11_1_6/0_pred_crop.png}}&
\frame{\includegraphics[height=\imageheight, ]{figures/supplementary_occlusion/bottle_11_1_6/proba.png}}&
&
\frame{\includegraphics[height=\imageheight, ]{figures/supplementary_occlusion/bottle_41_0_8/ref.png}} &
\frame{\includegraphics[height=\imageheight, ]{figures/supplementary_occlusion/bottle_41_0_8/1_query_crop.png}} &
\frame{\includegraphics[height=\imageheight, ]{figures/supplementary_occlusion/bottle_41_0_8/pizza.png}}&
\frame{\includegraphics[height=\imageheight, ]{figures/supplementary_occlusion/bottle_41_0_8/0_pred_crop.png}}&
\frame{\includegraphics[height=\imageheight, ]{figures/supplementary_occlusion/bottle_41_0_8/proba.png}}\\

\parbox[t]{0mm}{\rotatebox[origin=c]{90}{Bus}} & \frame{\includegraphics[height=\imageheight, ]{figures/supplementary_occlusion/bus_10_7_11/ref.png}} &
\frame{\includegraphics[height=\imageheight, ]{figures/supplementary_occlusion/bus_10_7_11/1_query_crop.png}} &
\frame{\includegraphics[height=\imageheight, ]{figures/supplementary_occlusion/bus_10_7_11/pizza.png}}&
\frame{\includegraphics[height=\imageheight, ]{figures/supplementary_occlusion/bus_10_7_11/0_pred_crop.png}}&
\frame{\includegraphics[height=\imageheight, ]{figures/supplementary_occlusion/bus_10_7_11/proba.png}}&
&
\frame{\includegraphics[height=\imageheight, ]{figures/supplementary_occlusion/bus_74_5_10/ref.png}} &
\frame{\includegraphics[height=\imageheight, ]{figures/supplementary_occlusion/bus_74_5_10/1_query_crop.png}} &
\frame{\includegraphics[height=\imageheight, ]{figures/supplementary_occlusion/bus_74_5_10/pizza.png}}&
\frame{\includegraphics[height=\imageheight, ]{figures/supplementary_occlusion/bus_74_5_10/0_pred_crop.png}}&
\frame{\includegraphics[height=\imageheight, ]{figures/supplementary_occlusion/bus_74_5_10/proba.png}}\\

\parbox[t]{0mm}{\rotatebox[origin=c]{90}{Clock}} & \frame{\includegraphics[height=\imageheight, ]{figures/supplementary_occlusion/clock_7_6_9/ref.png}} &
\frame{\includegraphics[height=\imageheight, ]{figures/supplementary_occlusion/clock_7_6_9/1_query_crop.png}} &
\frame{\includegraphics[height=\imageheight, ]{figures/supplementary_occlusion/clock_7_6_9/pizza.png}}&
\frame{\includegraphics[height=\imageheight, ]{figures/supplementary_occlusion/clock_7_6_9/0_pred_crop.png}}&
\frame{\includegraphics[height=\imageheight, ]{figures/supplementary_occlusion/clock_7_6_9/proba.png}}&
&
\frame{\includegraphics[height=\imageheight, ]{figures/supplementary_occlusion/clock_22_7_5/ref.png}} &
\frame{\includegraphics[height=\imageheight, ]{figures/supplementary_occlusion/clock_22_7_5/1_query_crop.png}} &
\frame{\includegraphics[height=\imageheight, ]{figures/supplementary_occlusion/clock_22_7_5/pizza.png}}&
\frame{\includegraphics[height=\imageheight, ]{figures/supplementary_occlusion/clock_22_7_5/0_pred_crop.png}}&
\frame{\includegraphics[height=\imageheight, ]{figures/supplementary_occlusion/clock_22_7_5/proba.png}}\\


\parbox[t]{0mm}{\rotatebox[origin=l]{90}{Dishwasher}} & \frame{\includegraphics[height=\imageheight, ]{figures/supplementary_occlusion/dishwasher_18_0_3/ref.png}} &
\frame{\includegraphics[height=\imageheight, ]{figures/supplementary_occlusion/dishwasher_18_0_3/1_query_crop.png}} &
\frame{\includegraphics[height=\imageheight, ]{figures/supplementary_occlusion/dishwasher_18_0_3/pizza.png}}&
\frame{\includegraphics[height=\imageheight, ]{figures/supplementary_occlusion/dishwasher_18_0_3/0_pred_crop.png}}&
\frame{\includegraphics[height=\imageheight, ]{figures/supplementary_occlusion/dishwasher_18_0_3/proba.png}}&
&
\frame{\includegraphics[height=\imageheight, ]{figures/supplementary_occlusion/dishwasher_25_6_2/ref.png}} &
\frame{\includegraphics[height=\imageheight, ]{figures/supplementary_occlusion/dishwasher_25_6_2/1_query_crop.png}} &
\frame{\includegraphics[height=\imageheight, ]{figures/supplementary_occlusion/dishwasher_25_6_2/pizza.png}}&
\frame{\includegraphics[height=\imageheight, ]{figures/supplementary_occlusion/dishwasher_25_6_2/0_pred_crop.png}}&
\frame{\includegraphics[height=\imageheight, ]{figures/supplementary_occlusion/dishwasher_25_6_2/proba.png}}\\

\parbox[t]{0mm}{\rotatebox[origin=c]{90}{Guitar}} & \frame{\includegraphics[height=\imageheight, ]{figures/supplementary_occlusion/guitar_26_0_1/ref.png}} &
\frame{\includegraphics[height=\imageheight, ]{figures/supplementary_occlusion/guitar_26_0_1/1_query_crop.png}} &
\frame{\includegraphics[height=\imageheight, ]{figures/supplementary_occlusion/guitar_26_0_1/pizza.png}}&
\frame{\includegraphics[height=\imageheight, ]{figures/supplementary_occlusion/guitar_26_0_1/0_pred_crop.png}}&
\frame{\includegraphics[height=\imageheight, ]{figures/supplementary_occlusion/guitar_26_0_1/proba.png}}&
&
\frame{\includegraphics[height=\imageheight, ]{figures/supplementary_occlusion/guitar_27_1_7/ref.png}} &
\frame{\includegraphics[height=\imageheight, ]{figures/supplementary_occlusion/guitar_27_1_7/1_query_crop.png}} &
\frame{\includegraphics[height=\imageheight, ]{figures/supplementary_occlusion/guitar_27_1_7/pizza.png}}&
\frame{\includegraphics[height=\imageheight, ]{figures/supplementary_occlusion/guitar_27_1_7/0_pred_crop.png}}&
\frame{\includegraphics[height=\imageheight, ]{figures/supplementary_occlusion/guitar_27_1_7/proba.png}}\\

\parbox[t]{0mm}{\rotatebox[origin=c]{90}{Mug}} & \frame{\includegraphics[height=\imageheight, ]{figures/supplementary_occlusion/mug_17_0_8/ref.png}} &
\frame{\includegraphics[height=\imageheight, ]{figures/supplementary_occlusion/mug_17_0_8/1_query_crop.png}} &
\frame{\includegraphics[height=\imageheight, ]{figures/supplementary_occlusion/mug_17_0_8/pizza.png}}&
\frame{\includegraphics[height=\imageheight, ]{figures/supplementary_occlusion/mug_17_0_8/0_pred_crop.png}}&
\frame{\includegraphics[height=\imageheight, ]{figures/supplementary_occlusion/mug_17_0_8/proba.png}}&
&
\frame{\includegraphics[height=\imageheight, ]{figures/supplementary_occlusion/mug_56_2_2/ref.png}} &
\frame{\includegraphics[height=\imageheight, ]{figures/supplementary_occlusion/mug_56_2_2/1_query_crop.png}} &
\frame{\includegraphics[height=\imageheight, ]{figures/supplementary_occlusion/mug_56_2_2/pizza.png}}&
\frame{\includegraphics[height=\imageheight, ]{figures/supplementary_occlusion/mug_56_2_2/0_pred_crop.png}}&
\frame{\includegraphics[height=\imageheight, ]{figures/supplementary_occlusion/mug_56_2_2/proba.png}}\\

\parbox[t]{0mm}{\rotatebox[origin=c]{90}{Pistol}} & \frame{\includegraphics[height=\imageheight, ]{figures/supplementary_occlusion/pistol_0_3_8/ref.png}} &
\frame{\includegraphics[height=\imageheight, ]{figures/supplementary_occlusion/pistol_0_3_8/1_query_crop.png}} &
\frame{\includegraphics[height=\imageheight, ]{figures/supplementary_occlusion/pistol_0_3_8/pizza.png}}&
\frame{\includegraphics[height=\imageheight, ]{figures/supplementary_occlusion/pistol_0_3_8/0_pred_crop.png}}&
\frame{\includegraphics[height=\imageheight, ]{figures/supplementary_occlusion/pistol_0_3_8/proba.png}}&
&
\frame{\includegraphics[height=\imageheight, ]{figures/supplementary_occlusion/pistol_2_4_10/ref.png}} &
\frame{\includegraphics[height=\imageheight, ]{figures/supplementary_occlusion/pistol_2_4_10/1_query_crop.png}} &
\frame{\includegraphics[height=\imageheight, ]{figures/supplementary_occlusion/pistol_2_4_10/pizza.png}}&
\frame{\includegraphics[height=\imageheight, ]{figures/supplementary_occlusion/pistol_2_4_10/0_pred_crop.png}}&
\frame{\includegraphics[height=\imageheight, ]{figures/supplementary_occlusion/pistol_2_4_10/proba.png}}\\

\parbox[t]{1mm}{\rotatebox[origin=l]{90}{Skateboard}} & \frame{\includegraphics[height=\imageheight, ]{figures/supplementary_occlusion/skateboard_32_0_3/ref.png}} &
\frame{\includegraphics[height=\imageheight, ]{figures/supplementary_occlusion/skateboard_32_0_3/1_query_crop.png}} &
\frame{\includegraphics[height=\imageheight, ]{figures/supplementary_occlusion/skateboard_32_0_3/pizza.png}}&
\frame{\includegraphics[height=\imageheight, ]{figures/supplementary_occlusion/skateboard_32_0_3/0_pred_crop.png}}&
\frame{\includegraphics[height=\imageheight, ]{figures/supplementary_occlusion/skateboard_32_0_3/proba.png}}&
&
\frame{\includegraphics[height=\imageheight, ]{figures/supplementary_occlusion/skateboard_14_2_8/ref.png}} &
\frame{\includegraphics[height=\imageheight, ]{figures/supplementary_occlusion/skateboard_14_2_8/1_query_crop.png}} &
\frame{\includegraphics[height=\imageheight, ]{figures/supplementary_occlusion/skateboard_14_2_8/pizza.png}}&
\frame{\includegraphics[height=\imageheight, ]{figures/supplementary_occlusion/skateboard_14_2_8/0_pred_crop.png}}&
\frame{\includegraphics[height=\imageheight, ]{figures/supplementary_occlusion/skateboard_14_2_8/proba.png}}\\

\parbox[t]{1mm}{\rotatebox[origin=c]{90}{Train}} & \frame{\includegraphics[height=\imageheight, ]{figures/supplementary_occlusion/train_29__1/ref.png}} &
\frame{\includegraphics[height=\imageheight, ]{figures/supplementary_occlusion/train_29__1/1_query_crop.png}} &
\frame{\includegraphics[height=\imageheight, ]{figures/supplementary_occlusion/train_29__1/pizza.png}}&
\frame{\includegraphics[height=\imageheight, ]{figures/supplementary_occlusion/train_29__1/0_pred_crop.png}}&
\frame{\includegraphics[height=\imageheight, ]{figures/supplementary_occlusion/train_29__1/proba.png}}&
&
\frame{\includegraphics[height=\imageheight, ]{figures/supplementary_occlusion/train_51__7/ref.png}} &
\frame{\includegraphics[height=\imageheight, ]{figures/supplementary_occlusion/train_51__7/1_query_crop.png}} &
\frame{\includegraphics[height=\imageheight, ]{figures/supplementary_occlusion/train_51__7/pizza.png}}&
\frame{\includegraphics[height=\imageheight, ]{figures/supplementary_occlusion/train_51__7/0_pred_crop.png}}&
\frame{\includegraphics[height=\imageheight, ]{figures/supplementary_occlusion/train_51__7/proba.png}}\\

\parbox[t]{1mm}{\rotatebox[origin=c]{90}{Washer}} & \frame{\includegraphics[height=\imageheight, ]{figures/supplementary_occlusion/washer_59_1_10/ref.png}} &
\frame{\includegraphics[height=\imageheight, ]{figures/supplementary_occlusion/washer_59_1_10/1_query_crop.png}} &
\frame{\includegraphics[height=\imageheight, ]{figures/supplementary_occlusion/washer_59_1_10/pizza.png}}&
\frame{\includegraphics[height=\imageheight, ]{figures/supplementary_occlusion/washer_59_1_10/0_pred_crop.png}}&
\frame{\includegraphics[height=\imageheight, ]{figures/supplementary_occlusion/washer_59_1_10/proba.png}}&
&
\frame{\includegraphics[height=\imageheight, ]{figures/supplementary_occlusion/washer_64_3_6/ref.png}} &
\frame{\includegraphics[height=\imageheight, ]{figures/supplementary_occlusion/washer_64_3_6/1_query_crop.png}} &
\frame{\includegraphics[height=\imageheight, ]{figures/supplementary_occlusion/washer_64_3_6/pizza.png}}&
\frame{\includegraphics[height=\imageheight, ]{figures/supplementary_occlusion/washer_64_3_6/0_pred_crop.png}}&
\frame{\includegraphics[height=\imageheight, ]{figures/supplementary_occlusion/washer_64_3_6/proba.png}}\\

&Reference  & Query & PIZZA~\cite{nguyen_pizza_2022} & Ours & Pose distribution & &Reference  & Query & PIZZA~\cite{nguyen_pizza_2022} & Ours & Pose distribution\\
\end{tabular}
\end{tabular}
}
% \vspace{3mm}
\caption{\label{fig:resultsOcclusion} {\bf Visual results on unseen categories} from ShapeNet. The arrow indicates the pose with the highest probability as recovered by our method. We visually compare with PIZZA, which is the method with the second best performance. \textbf{We visualize the predicted poses by rendering the object from these poses, but the 3D model is only used for visualization purposes, not as input to our method. Similarly, we use the canonical pose of the 3D model to visualize this distribution, but not as input to our method.}
} 
\end{figure*}


{\small
\bibliographystyle{ieee_fullname}
% \bibliography{references, zotero}
\bibliography{cleaned_refs}
}
%%%%%%%%% REFERENCES


\end{document}
