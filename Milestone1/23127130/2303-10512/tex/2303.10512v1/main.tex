\documentclass{article} % For LaTeX2e
\usepackage{iclr2023_conference,times}
% Optional math commands from https://github.com/goodfeli/dlbook_notation.
\newcommand{\bbox}{\text{bbox}}
\newcommand{\alphapck}{\alpha_\bbox}
\newcommand{\kcycle}{\text{k-CyPCK}}
\newcommand{\cycle}{\text{-CyPCK}}

\newcommand{\I}{\mathbf{I}}
\newcommand{\Ia}{\I^\text{a}}
\newcommand{\Ib}{\I^\text{b}}
\newcommand{\Iatob}{\I^\text{a $\rightarrow$ b}}
\newcommand{\F}{\mathbf{F}}
\newcommand{\Fa}{\F^\text{a}}
\newcommand{\Fb}{\F^\text{b}}
\newcommand{\f}{\mathbf{f}}
\newcommand{\fa}{\f^\text{a}}
\newcommand{\fb}{\f^\text{b}}
\newcommand{\p}{\mathbf{p}}
\newcommand{\pa}{\p^\text{a}}
\newcommand{\pb}{\p^\text{b}}
\newcommand{\A}{\boldsymbol{\Phi}_\text{align}}
\newcommand{\G}{\mathbf{G}}
\newcommand{\C}{\mathbf{C}}
\newcommand{\Ca}{\C^\text{a}}
\newcommand{\Cb}{\C^\text{b}}
\newcommand{\cc}{\mathbf{c}}
\newcommand{\cca}{\cc^\text{a}}
\newcommand{\ccb}{\cc^\text{b}}
\newcommand{\Irec}{\I_\text{Recon}}
\newcommand{\M}{\mathbf{M}}
\newcommand{\Mrec}{\M_\text{Recon}}
\newcommand{\loss}{\mathcal{L}}
\newcommand{\T}{\mathcal{T}}
\newcommand{\W}{\mathcal{W}}
\newcommand{\Id}{\mathcal{I}}


\usepackage[colorlinks, linkcolor=blue, anchorcolor=blue, citecolor=blue]{hyperref}
\usepackage{url}
\usepackage{microtype}
\usepackage{graphicx}
\usepackage{booktabs}
\usepackage{multirow} 
\usepackage{subcaption}
\usepackage{amsmath}
\usepackage{amssymb}
\usepackage{mathtools}
\usepackage{amsthm}
\usepackage{letltxmacro}
\usepackage{algorithm}
\usepackage{algorithmic}
\usepackage[english]{babel}
\usepackage{amsthm}
\usepackage{mathrsfs}
\usepackage{lineno}

\newcommand{\W}{W}
\newcommand{\Wpre}{W^{(0)}}
\newcommand{\kk}{k}
\newcommand{\CC}{\mathcal{C}}
\newcommand{\DeltaW}{\Delta}
\newcommand{\DeltaWk}{\DeltaW_{\kk}}
\newcommand{\DeltaWki}{\DeltaW_{\kk,i}}
\newcommand{\A}{A}
\newcommand{\B}{B}
\newcommand{\Ai}{\A_{\cdot i}}
\newcommand{\Bi}{\B_{i \cdot}}
\newcommand{\PP}{P}
\newcommand{\PPk}{\PP_{\kk}}
\newcommand{\PPki}{\PP_{\kk,*i}}
\newcommand{\PPt}{\PP^{(t)}}
\newcommand{\PPtk}{\PPt_{\kk}}
\newcommand{\PPtki}{\PPt_{\kk,*i}}
\newcommand{\PPcal}{\mathcal{\PP}}
\newcommand{\PPcalt}{\PPcal^{(t)}}
\newcommand{\QQ}{Q}
\newcommand{\QQk}{\QQ_{\kk}}
\newcommand{\QQki}{\QQ_{\kk,i*}}
\newcommand{\QQt}{\QQ^{(t)}}
\newcommand{\QQtk}{\QQt_{\kk}}
\newcommand{\QQtki}{\QQt_{\kk,i*}}
\newcommand{\QQcal}{\mathcal{\QQ}}
\newcommand{\QQcalt}{\QQcal^{(t)}}
\newcommand{\Lam}{\Lambda}
\newcommand{\Lamk}{\Lam_{\kk}}
\newcommand{\Lamt}{\Lam^{(t)}}
\newcommand{\Lamtk}{\Lamt_{\kk}}
\newcommand{\Lamtpk}{\Lam^{(t+1)}_{\kk}}
\newcommand{\Lami}{\Lam_{ii}}
\newcommand{\Lamki}{\Lam_{\kk,ii}}
\newcommand{\Lamtki}{\Lamt_{\kk,ii}}
\newcommand{\Lamcal}{\mathcal{E}}
\newcommand{\Lamcalt}{\Lamcal^{(t)}}
\newcommand{\tLam}{\tilde{\Lam}}
\newcommand{\tLamt}{\tilde{\Lam}^{(t)}}
\newcommand{\tLamti}{\tilde{\Lam}^{(t)}_{ii}}
\newcommand{\tLamk}{\tLam_{\kk}}
\newcommand{\tLamki}{\tLam_{\kk,ii}}
\newcommand{\tLamtk}{\tLamt_{\kk}}
\newcommand{\lambdaki}{\lambda_{\kk,i}}
\newcommand{\bLam}{\bm{\Lam}}
\newcommand{\blam}{\bm{\lambda}}
\newcommand{\blamt}{\blam^{(t)}}
\newcommand{\blamtp}{\blam^{(t+1)}}
\newcommand{\Bu}{b}
\newcommand{\But}{\Bu^{(t)}}
\newcommand{\BuT}{\Bu^{(T)}}
\newcommand{\Buinit}{\Bu^{(0)}}
\newcommand{\rinit}{r^{(0)}}
\newcommand{\rat}{r^{(t)}}
\newcommand{\raT}{r^{(T)}}
\newcommand{\ramt}{r_{m}^{(t)}}
\newcommand{\ramT}{r_{m}^{(T)}}
\newcommand{\rkt}{r_{k}^{(t)}}
\newcommand{\rkT}{r_{k}^{(T)}}
\newcommand{\rbar}{\bar{r}}
\newcommand{\rbart}{\rbar^{(t)}}
\newcommand{\rbarT}{\rbar^{(T)}}
\newcommand{\bx}{\bm{x}}
\newcommand{\bh}{\bm{h}}
\newcommand{\Reg}{R}
\newcommand{\DeltaT}{\Delta_{T}}
\newcommand{\Sc}{S}
\newcommand{\Sct}{\Sc^{(t)}}
\newcommand{\Sci}{\Sc_{i}}
\newcommand{\Scki}{\Sc_{\kk,i}}
\newcommand{\Scti}{\Sc^{(t)}_{i}}
\newcommand{\Sctk}{\Sct_{\kk}}
\newcommand{\Sctki}{\Sct_{\kk,i}}
\newcommand{\scf}{s}
\newcommand{\scft}{\scf^{(t)}}
\newcommand{\I}{I}
\newcommand{\Ibar}{\overline{I}}
\newcommand{\Ibart}{\Ibar^{(t)}}
\newcommand{\Ubar}{\overline{U}}
\newcommand{\Ubart}{\Ubar^{(t)}}
\newcommand{\LL}{\mathcal{L}}
\newcommand{\X}{X}
\newcommand{\He}{H}
\newcommand{\Wq}{\W_{q}}
\newcommand{\Wk}{\W_{k}}
\newcommand{\Wv}{\W_{v}}
\newcommand{\Wqi}{\W_{q_i}}
\newcommand{\Wki}{\W_{k_i}}
\newcommand{\Wvi}{\W_{v_i}}
\newcommand{\Wo}{\W_{o}}
\newcommand{\Wfp}{\W_{f_1}}
\newcommand{\Wfq}{\W_{f_2}}
\newcommand{\bb}{\bm{b}}
\newcommand{\Gcal}{\mathcal{G}}
\newcommand{\Gcali}{\Gcal_i}
\newcommand{\Gcalk}{\Gcal_{\kk}}
\newcommand{\Gcalki}{\Gcal_{\kk,i}}

\newcommand{\Proj}{\mathcal{T}}

\newcommand{\norm}[1]{\lVert#1 \rVert}
\newcommand{\normlarge}[1]{\left\lVert#1\right\rVert}

\newcommand{\ouralg}{AdaLoRA} 

\newtheorem*{remark}{Remark}


\title{Adaptive Budget Allocation for Parameter-Efficient Fine-Tuning}
\author{Qingru Zhang$^\dagger$\thanks{Work was done during Qingru Zhang's internship at Microsoft Azure AI.}, \ Minshuo Chen$^\ddagger$, \ Alexander Bukharin$^\dagger$, \ Pengcheng He$^\diamond$, \ Yu Cheng$^\diamond$, \\
\textbf{Weizhu Chen$^\diamond$} and \ \textbf{Tuo Zhao$^\dagger$}\\
$^\dagger$Georgia Institute of Technology \ \ 
$^\ddagger$Princeton University \ \
$^\diamond$Microsoft Azure AI \\
\texttt{\{qingru.zhang,abukharin3,tourzhao\}@gatech.edu} \\
\texttt{mc0750@princeton.edu} \\
\texttt{\{penhe,yu.cheng,wzchen\}@microsoft.com}
}

\newcommand{\fix}{\marginpar{FIX}}
\newcommand{\new}{\marginpar{NEW}}

\iclrfinalcopy % Uncomment for camera-ready version, but NOT for submission.
\begin{document}


\maketitle

\begin{abstract}
    Fine-tuning large pre-trained language models on downstream tasks has become an important paradigm in NLP. However, common practice fine-tunes all of the parameters in a pre-trained model, which becomes prohibitive when a large number of downstream tasks are present. Therefore, many fine-tuning methods are proposed to learn incremental updates of pre-trained weights in a parameter efficient way, e.g., low-rank increments. These methods often evenly distribute the budget of incremental updates across all pre-trained weight matrices, and overlook the varying importance of different weight parameters. As a consequence, the fine-tuning performance is suboptimal. To bridge this gap, we propose {\ouralg}, which adaptively allocates the parameter budget among weight matrices according to their importance score. In particular, {\ouralg} parameterizes the incremental updates in the form of singular value decomposition. Such a novel approach allows us to effectively prune the singular values of unimportant updates, which is essentially to reduce their parameter budget but circumvent intensive exact SVD computations. We conduct extensive experiments with several pre-trained models on natural language processing, question answering, and natural language generation to validate the effectiveness of {\ouralg}. Results demonstrate that {\ouralg} manifests notable improvement over baselines, especially in the low budget settings. Our code is publicly available at \url{https://github.com/QingruZhang/AdaLoRA}.   	
\end{abstract}

The advance of Pre-trained Language Models (PLMs) like GPT-3 \cite{brown2020language} and LLaMA \cite{DBLP:journals/corr/abs-2302-13971} has substantially improved the performance of deep neural networks across a variety of Natural Language Processing (NLP) tasks. Various language models, based on the Transformer \cite{vaswani2017attention} architecture,  have been proposed, leading to state-of-the-art (SOTA) performance on the fundamental discrimination tasks. These models are first trained with self-supervised training objectives (e.g., predicting masked tokens according to surrounding tokens) on massive unlabeled text data, then fine-tuned on annotated data to adapt to downstream tasks of interest.  However, annotated data is usually limited to a wide range of downstream tasks, which results in overfitting and a lack of generalization to unseen data.

One straightforward way to deal with this data scarcity problem is data augmentation , and incorporating generative models to perform data augmentation has been widely adopted recently . Despite its popularity, the generated text can easily deviate from the real data distribution without exploiting any of the signals passed back from the discrimination task. In previous studies, generative data augmentation and discrimination have been well studied as separate problems, but it is less clear how these two can be leveraged in one framework and how their performances can be improved simultaneously. \looseness=-1

Generative Adversarial Networks (GANs) \cite{https://doi.org/10.48550/arxiv.1406.2661} are good attempts to couple generative and discriminative models in an adversarial manner, where a two-player minimax game between learners is carefully crafted. GANs have achieved tremendous success in domains such as image generation , and related studies have also shown their effectiveness in semi-supervised learning. However,  in the text field, GANs are difficult to train, most training objectives work well for only one model, either the discriminator or the generator, so rarely both learners can be optimal at the same time. This essentially arises from the adversarial nature of GANs, that during the process, optimizing one learner can easily destroy the learning ability of the other, making GANs fail to converge.

Another limitation of simultaneously optimizing the generator and the discriminator comes from the discrete nature of text in NLP, as no gradient propagation can be done from discriminators to generators. One theoretically sound attempt is to use reinforcement learning (RL), but the sparsity and the high variance of the rewards in NLP make the training particularly unstable \cite{caccia2019language}. 

To address these shortcomings, we novelly introduce a self-consistent learning framework based on one generator and one discriminator: the generator and the discriminator are alternately trained by way of cooperation instead of competition, and the selected samples are used as the medium to pass the feedback signal from the discriminator. Specifically, in each round of training, the samples generated by the generator are synthetically labeled by the discriminator, and then only part of them would be selected based on dynamic thresholds and used for the training of the discriminator and the generator in the next round. Several benefits can be discovered from this cooperative training process. First, a closed-loop form of cooperation can be established so that we can get the optimal generator and discriminator at the same time. Second, this framework helps improve the generation quality while ensuring the domain specificity of generator, which in turn contributes to training. Third, a steady stream of diverse synthetic samples can be added to the training in each round and lead to continuous improvement of the performance of all learners. Finally, we can start the training with only domain-related corpus and obtain strong results, while these data can be easily sampled with little cost or supervision. Also, the performance on labeled datasets can be further boosted based on the strong baselines. As an example to demonstrate the effectiveness of our framework in the text field, we examine it on four downstream text generation benchmarks, including AFQMC, CHIP-STS, QQP, and MRPC. The experiments show that our method significantly improves over standalone state-of-the-art discriminative models on zero-shot and full-data settings.

Our contributions are summarized as follows,

$\bullet$ We propose a self-consistent learning framework in the text field that incorporates the generator and the discriminator, in which both achieve remarkable performance gains simultaneously.

$\bullet$ We propose a dynamic selection mechanism such that cooperation between the generator and the discriminator drives the convergence to reach their scoring consensus.

$\bullet$ Experimental results show that the generator in our framework can continuously adjust its generation samples based on the performance of downstream tasks, while the discriminator can outperform the strong baselines.


\section{Background and Motivation}
This section first introduces mini-batch GNN training, 
% in Section~\ref{sec:background}.
and then 
% discusses the data loading-induced performance bottleneck and 
elaborates on the limitations of existing optimizations. 
% , such as caching and hybrid parallelism, 
% in Section~\ref{sec:problem}.

\begin{figure}[ht]
    \centering
    \includegraphics[width=\linewidth]{figures/mini-batch.pdf}
    \caption{Example of a mini-batch.}
    \label{fig:minibatch}
\end{figure}

\subsection{Mini-Batch GNN Training and Data Parallelism} 
A GNN model is defined as a sequence of \emph{GNN layers}.\footnote{In the following, the term ``layer'' will refer to GNN layers, not to neural network layers unless otherwise stated.}
During each mini-batch training iteration, there are three phases: \textit{sampling}, \textit{loading}, and \textit{training}.
The sampling phase randomly selects a mini-batch starting from the target vertices.
A mini-batch with two target vertices is shown in Figure~\ref{fig:comparison}(a). 
In the loading phase, the input features of the vertices in the bottom layer of the mini-batch are loaded into the GPUs.
During forward propagation, each GNN layer $l > 0$ aggregates and transforms the features of the vertices in the layer $l-1$ of the sample and produces the features of the vertices in the layer $l$ (see Figure~\ref{fig:minibatch}).
The last GNN layer computes the features of the target vertices, which are then used to calculate the loss. 
During backward propagation, the layers are executed in reverse order to compute gradients. 
Finally, all GPUs aggregate and apply the computed gradients.


\begin{figure}[ht]
    \centering
    \begin{subfigure}[t]{0.45\linewidth}
        \centering
        \includegraphics[width=\textwidth]{results/breakdown/orkut_epoch_breakdown.pdf}
        \caption{The fraction of sampling, loading, and training time per epoch of DGL, P3* and Quiver on Orkut with the GAT model.}
        \label{fig:orkut_epoch_breakdown}
    \end{subfigure}
    \hfill
    \begin{subfigure}[t]{0.45\linewidth}
        \centering
        \includegraphics[width=\textwidth]{results/breakdown/quiver_epoch_breakdown.pdf}
        \caption{The percentage of sampling, loading, and training time per epoch of Quiver on Orkut and Papers100M with the GraphSage model. }
        \label{fig:quiver_epoch_breakdown}
    \end{subfigure}
    \caption{Epoch time breakdown of existing systems on a four-GPU host with NVLink.}
    \label{fig:epoch_breakdown}
\end{figure}

% \begin{figure}[h!]
%     \centering
%     \advance\leftskip-1cm
% % \advance\rightskip-3cm
% \includegraphics[keepaspectratio=true,width=.5\textwidth]{results/epoch_breakdown/sage_epoch_breakdown_arxiv.png}
%     % \includegraphics[width=0.85\columnwidth]{results/epoch_breakdown/sage_epoch_breakdown.png}
%      \caption{The fraction of sampling, loading, and training time per epoch of existing systems when training the GraphSage model on a four GPU host with NVLink. \ms{New figures. This figure is out of bounds. Larger fonts. Edit text as needed}
%      %DGL and P3 cache no feature data in all experiments. Quiver caches 44\% for Papers100M, and 12\% percent for Friendster.  All systems use the GAT model with fanouts [15, 15, 15] and batch size 1024. 
%      }
%     \label{fig:epoch_breakdown}
% \end{figure}

Data parallelism is the most commonly used training strategy for mini-batch GNN training. 
In data parallel training, the target vertices are partitioned among GPUs, where each partition corresponds to a separate \textit{micro-batch} (see Figure~\ref{fig:comparison}(a)). 
Each GPU independently loads the input features of all the vertices in the bottom layer of its micro-batch and trains on it.
This approach has two limitations: a high cost of data loading and a high degree of computational and data loading redundancy.

% Instead of creating multiple independent and overlapping micro-batches as done by data parallelism, \tname generates a single mini-batch for all the target vertices and splits it without overlaps, avoiding redundant computation and loads.

\mypar{Data loading bottleneck}
Input feature loading is a major overhead in data-parallel GNN training, which contributes to a large fraction of the total training time and prevents a good utilization of the GPUs.
% When training a GNN using a mini-batch with N target vertices, the GPUs need to load the feature vectors of the vertices in the k-hop neighborhood of these target vertices into their memory. 
% 
Figure \ref{fig:orkut_epoch_breakdown} shows the time breakdown of sampling, feature loading, and forward/backward pass per epoch with three GNN systems: DGL, Quiver, and P3*.
We will initially focus on DGL, which is a standard data-parallel baseline, and discuss optimizations shortly.
% We observe that data loading can take up to 69\% of the epoch time for DGL, 48\% for P3*, and 79\% for Quiver. 
We observe that data loading can take more than 60\% of the epoch time for DGL. 
This data loading-induced performance bottleneck is also reported in other GNN training literature~\citep{quiver, pagraph, wholegraph,ugache}. 

\mypar{Redundant loading and computation}
Table~\ref{tab:redundancy} further reports the degree of computational and data loading redundancy in data-parallel training.
With 4 GPUs, data parallelism creates 4 separate micro-batches (``Micro''), causing up to $1.2\times$ compute and $2.5\times$ feature loading compared to having only a single mini-batch (``Mini'').

% \begin{table}[t]
% \centering 
% \tabcolsep=0.08cm
% \begin{tabular}{|l||c|c|c|}
% \hline 
% \textbf{Dataset} & \textbf{4x Micro} & \textbf{1x Mini} & \textbf{\% redundancy} \\ \hline \hline
% % ogbn-products & 195M& 155M & 25.5\% \\
% orkut & 926M & 751M & 23\% \\
% % papers100M & 327M & 131M & 148.9\% \\
% papers100M & 452M & 389M & 16\% \\
% % friendster & 452M & 389M & 16\% \\
% % amazon & 473M & 263M & 79.2\% \\
% \hline
% \end{tabular}
% \caption{Redundant computation. The total number of edges computed over one epoch when each mini-batch is sampled as 4 micro-batches of size 1024 (\texttt{4x Micro}) vs. 1 mini-batch of size 4096 (\texttt{1x Mini}). \ms{Numbers for the other graphs?}}
% \label{tab:overlap-edges} 
% \end{table}

\begin{table}[ht]
\centering
% \resizebox{\columnwidth}{!}{%
\small 
\tabcolsep=0.02cm
\begin{tabular}{|c|ccc|ccc|}
\hline
\multirow{2}{*}{\textbf{Graph}} & \multicolumn{3}{c|}{\textbf{\# Edges Computed}}                                                 & \multicolumn{3}{c|}{\textbf{\# Feature Vectors Loaded}}                                                \\ \cline{2-7} 
                                & \multicolumn{1}{c|}{\textbf{ Micro}} & \multicolumn{1}{c|}{\textbf{ Mini}} & \textbf{Ratio} & \multicolumn{1}{c|}{\textbf{ Micro}} & \multicolumn{1}{c|}{\textbf{ Mini}} & \textbf{Ratio} \\ \hline
\textbf{Orkut}                  & \multicolumn{1}{c|}{926M}              & \multicolumn{1}{c|}{751M}             & 1.2x           & \multicolumn{1}{c|}{422M}              & \multicolumn{1}{c|}{169M}             & 2.5x          \\ \hline
\textbf{Papers100M}             & \multicolumn{1}{c|}{452M}              & \multicolumn{1}{c|}{389M}             & 1.2x          & \multicolumn{1}{c|}{231M}              & \multicolumn{1}{c|}{154M}             & 1.5x           \\ \hline
\textbf{Friendster}             & \multicolumn{1}{c|}{13.4B}             & \multicolumn{1}{c|}{13.1B}            & 1.0x          & \multicolumn{1}{c|}{11.4B}             & \multicolumn{1}{c|}{9.4B}             & 1.2x           \\ \hline
\end{tabular}%
% }
\caption{Redundant computation and data loading. The total number of edges computed and feature data loaded over one epoch when each mini-batch is sampled as 4 micro-batches of size 1024 (Micro) vs. 1 mini-batch of size 4096 (Mini).}
\vspace{-0.2cm}
\label{tab:redundancy}
\end{table}




% For example, in Figure~\ref{fig:comparison}(a), although there are only two target vertices in that mini-batch, a total of eight input features need to be loaded into the GPU memory for forward and backward propagation on a 2-layer GNN. 

% The data loading phase can take up a significant portion of GNN training time and even become the performance bottleneck, depending on the system, the graph characteristics, and the GNN model.


\subsection{Limitations of Existing Optimizations} \label{sec:problem}

\begin{comment}
Table~\ref{tab:loading-bottleneck} shows the time spent on the three phases, sampling, data loading, and training, when training two GraphSAGE and GAT models on two graph datasets, Papers100M and Orkut, using DGL~\citep{dgl} on a four-GPU server with NVLink.
DGL is one of the well-established GNN training frameworks that support mini-batch data parallel training. 
Details on the experiment settings are in Section~\ref{sec:settings}.
% DGL doesn't support caching only a part of a graph in GPU memory and both graphs we consider are too big to fit. 
From the table, we observe that data loading can take up to XX\% of the epoch time. \ms{update number}
This data loading-induced performance bottleneck is also reported in other GNN training literature~\citep{quiver, pagraph}. 

\begin{table*}[t]
\small 
\begin{tabular}{|c||c|c||c|c|c|c||c|c|c|c|}
\hline 
\multirow{2}{*}{ Graph }  & \multirow{2}{*}{ System}  & \multirow{2}{*}{ Cache \%} &\multicolumn{4}{|c||}{SAGE} & \multicolumn{4}{|c|}{GAT} \\
\cline{4-11}
 & & & S  & L  & FB & Total & S  & L  & FB  & Total  \\  
\hline \hline  
 & DGL & no cache & 6.57 & 14.46 & 11.89 & 32.92 & 6.36 & 14.42 & 31.36 & 52.14\\
Papers100M & Quiver & SAGE: 51\% - GAT: 44\%  &11.17 & 17.23 & 8.99 & 37.38 & 11.19 & 22.09 & 25.06 & 58.33 \\
& $P^{3*}$ & OOM & OOM & OOM & OOM & OOM & OOM & OOM & OOM & OOM \\
\hline
& DGL & no cache &0.96 & 96.59 & 13.29 & 110.84 & 0.98 & 96.32 & 17.55 & 114.85 \\
Orkut & Quiver & 100\% & 4.08 & 6.85 & 12.81 & 23.74 & 3.83 & 6.91 & 17.44 & 28.18\\
& $P^{3*}$ & 100\% & 0.97 & 2 & 15.72 & 18.68 & 1.01 & 1.83 & 25.89 & 28.73 \\
\hline
\end{tabular}
\caption{Data loading bottleneck on a system with NVLink. 
\ms{Explain S/L/T once finalized. make it look less similar to the evaluation results (since they are the same numbers). }
}
\label{tab:loading-bottleneck}
\end{table*}
% 
\end{comment}

% \begin{figure}
%     \centering
%     \includegraphics[width=0.49\columnwidth]{results/epoch_breakdown/Papers100M_epoch_breakdown.png}
%     \includegraphics[width=0.49\columnwidth]{results/epoch_breakdown/Friendster_epoch_breakdown.png}

%      \caption{The fraction of sampling, loading, and training time per epoch of existing systems when training GAT on a 4 GPU host with NVLink. 
%      %DGL and P3 cache no feature data in all experiments. Quiver caches 44\% for Papers100M, and 12\% percent for Friendster.  All systems use the GAT model with fanouts [15, 15, 15] and batch size 1024. 
%      }
%     \label{fig:epoch_breakdown}
% \end{figure}


% \subsection{Existing Optimizations} \label{sec:limitations}
% \mypar{Existing optimizations}
Many approaches have been proposed to address the data loading-induced performance bottleneck of mini-batch training. 
These approaches fall broadly into three categories: caching~\citep{pagraph, gnnlab, quiver, wholegraph}, hybrid parallelism~\citep{gandhi2021p3}, and algorithmic optimizations~\citep{dong2021global,twolevel, ramezani2020gcn,liu2023bgl}. 
In this work, 
% we do not consider algorithmic optimizations that impose approximation choices onto the users and impact model accuracy. 
we focus on optimizations that can be used in general-purpose GNN training systems that scale to multiple GPUs without imposing modeling choices that can impact the model accuracy or semantics, such as using specific sampling algorithms or relaxing synchrony.

We now discuss caching and hybrid parallelism approaches 
% for single-host mini-batch training 
and show that they still suffer from high data loading costs.
Besides caching, none of these solutions addresses the fundamental problem of redundant loading and computation that is inherent in data parallelism.

\mypar{Limitations of data-parallel caching}
To reduce data loading time, several systems maintain a static cache in the main memory of the GPUs. 
This cache is populated offline with frequently-accessed input features~\citep{pagraph, gnnlab}.
The latest systems use a distributed shared memory to enable GPUs to fetch features from other GPUs' memory using fast GPU-to-GPU interconnects like NVLink \citep{quiver, dsp, ugache}.
As shown in Figure~\ref{fig:orkut_epoch_breakdown}, the distributed shared-memory caching mechanism in Quiver~\citep{quiver} can reduce loading time for the Orkut graph, whose features are too large to fit in a single GPU memory but can be fully cached across multiple GPUs. 

Distributed caching, however, does not fully address the performance issue of data loading, which can still take a significant fraction of the epoch time as shown in Figure~\ref{fig:quiver_epoch_breakdown}.
For the GraphSage model, data loading over NVLink can be relatively expensive for Quiver with a graph that can be fully cached like Orkut.
For larger graphs such as Papers100M, only a part of input features can be cached on GPUs. 
Data loading can still stress the PCIe bus between the host and devices, resulting in unsatisfactory training performance. 
With Papers100M, only up to 60\% of the input features can be cached and the data loading time remains high.
%We fixed this bug of Quiver and termed the improved system DistCache. DGL can still outperform DistCache}
%Even with Orkut, which can be fully cached in the distributed GPU memory, Quiver still needs to spend up to XX\% of the time loading data.\ms{update numbers}

\begin{comment}
Table~\ref{tab:loading-bottleneck} reports the maximum percentage of features that can be cached on a four-GPU server with NVLink interconnect. 
With Friendster, only up to XX\% of the input features can be cached and the data loading time remains high (up to 14\% of the epoch time).  
Even with Orkut, which can be fully cached in the distributed GPU memory, Quiver still needs to spend up to XX\% of the time loading data.\ms{update numbers}
\end{comment}


\begin{comment}    
These experiments consider a system equipped with NVLink.
Without NVLink, data loads from the host memory or other GPUs always need to occur over PCIe.
This makes distributed GPU caching not effective, as we will show in our experiments.
\ms{Review this later. PCIe numbers are still inconclusive.}
\end{comment}

% % \begin{table}[t]
% \centering 
% \tabcolsep=0.08cm
% \begin{tabular}{|l||c|c|c|}
% \hline 
% \textbf{Dataset} & \textbf{4x Micro} & \textbf{1x Mini} & \textbf{\% redundancy} \\ \hline \hline
% % ogbn-products & 195M& 155M & 25.5\% \\
% orkut & 926M & 751M & 23\% \\
% % papers100M & 327M & 131M & 148.9\% \\
% papers100M & 452M & 389M & 16\% \\
% % friendster & 452M & 389M & 16\% \\
% % amazon & 473M & 263M & 79.2\% \\
% \hline
% \end{tabular}
% \caption{Redundant computation. The total number of edges computed over one epoch when each mini-batch is sampled as 4 micro-batches of size 1024 (\texttt{4x Micro}) vs. 1 mini-batch of size 4096 (\texttt{1x Mini}). \ms{Numbers for the other graphs?}}
% \label{tab:overlap-edges} 
% \end{table}

\begin{table}[ht]
\centering
% \resizebox{\columnwidth}{!}{%
\small 
\tabcolsep=0.02cm
\begin{tabular}{|c|ccc|ccc|}
\hline
\multirow{2}{*}{\textbf{Graph}} & \multicolumn{3}{c|}{\textbf{\# Edges Computed}}                                                 & \multicolumn{3}{c|}{\textbf{\# Feature Vectors Loaded}}                                                \\ \cline{2-7} 
                                & \multicolumn{1}{c|}{\textbf{ Micro}} & \multicolumn{1}{c|}{\textbf{ Mini}} & \textbf{Ratio} & \multicolumn{1}{c|}{\textbf{ Micro}} & \multicolumn{1}{c|}{\textbf{ Mini}} & \textbf{Ratio} \\ \hline
\textbf{Orkut}                  & \multicolumn{1}{c|}{926M}              & \multicolumn{1}{c|}{751M}             & 1.2x           & \multicolumn{1}{c|}{422M}              & \multicolumn{1}{c|}{169M}             & 2.5x          \\ \hline
\textbf{Papers100M}             & \multicolumn{1}{c|}{452M}              & \multicolumn{1}{c|}{389M}             & 1.2x          & \multicolumn{1}{c|}{231M}              & \multicolumn{1}{c|}{154M}             & 1.5x           \\ \hline
\textbf{Friendster}             & \multicolumn{1}{c|}{13.4B}             & \multicolumn{1}{c|}{13.1B}            & 1.0x          & \multicolumn{1}{c|}{11.4B}             & \multicolumn{1}{c|}{9.4B}             & 1.2x           \\ \hline
\end{tabular}%
% }
\caption{Redundant computation and data loading. The total number of edges computed and feature data loaded over one epoch when each mini-batch is sampled as 4 micro-batches of size 1024 (Micro) vs. 1 mini-batch of size 4096 (Mini).}
\vspace{-0.2cm}
\label{tab:redundancy}
\end{table}

\begin{figure*}[ht]
    \centering
    \includegraphics[width=\textwidth]{figures/overview.drawio.pdf}
    \vspace{-5mm}
    \caption{Overview of the \name training pipeline.}
    \label{fig:overview}
\end{figure*}

\mypar{Limitations of existing hybrid parallelism} 
The alternative to data-parallelism for mini-batch GNN training is called \emph{push-pull parallelism} proposed by the P3 system~\citep{gandhi2021p3}.
P3 targets distributed multi-host systems and aims to avoid transferring input features among hosts.
Each host keeps a slice of the feature vector of each vertex in host memory.
Like in data-parallel training, each GPU is associated with a micro-batch.
However, in push-pull parallelism, each GPU computes the input layer of \emph{all} micro-batches on its feature slice.
GPUs then exchange partial activations and continue the iteration in a data-parallel fashion.

$P^3$ is a multi-host system that does not use GPUs for caching, so it was not previously evaluated in a single-host multi-GPU setting.
To fill this gap, we have implemented its push-pull parallelism approach in a single-host multi-GPU setting.
We call this implementation P3*.
Figure~\ref{fig:comparison}(b) illustrates how P3* applies hybrid parallelism. 
Like the original $P^3$, this implementation partitions the cached features among GPUs at the cost of a cross-GPU push-pull shuffle.
For the Orkut graph, which can be fully cached, P3* does not exchange features among GPUs during the loading time, as Quiver does.
Instead, it pushes the bottom layer of all micro-batches to all GPUs, which induces a much lower loading cost.
The additional overhead of push-pull shuffling outweighs the gains in terms of loading time during the forward and backward pass, which results in higher overall training costs (see Figure~\ref{fig:orkut_epoch_breakdown}).
For other graphs that cannot be fully cached, P3* loads all the features in the mini-batch.
The training time still is higher than the other two systems due to the cost of shuffling.

\begin{comment}
\mypar{Implications for Training Cost}
One of the most important reasons for optimizing the training time of ML models is to reduce the dollar cost of training.
This depends not only on the training time but also on the type of server that is used for training.
GNN models are very small and they can be easily replicated at each GPU, unlike for example large language models that need to be partitioned among multiple GPUs.
For example, the models we consider range around XXX \ms{report size}.
The computation in GNN training is also relatively lightweight.
Therefore, the benefit of using high-end servers with NVLink mainly comes from mitigating the data loading bottleneck.
However, are these speedups sufficient to justify the additional cost of using high-end GPU servers for GNN training?

To answer this question, we consider the cost of the two AWS instance types we used for our evaluation and evaluate the cost per epoch of each system.
We consider a p3.8xlarge instance with NVLink, which currently costs \$12.24 per hour and a g4dn.12xlarge without NVLink, which costs \$3.912 per hour.
The cost per epoch is a good metric to measure the overall training cost because all the systems we consider run the models unmodified using synchronous training and have similar convergence rates per epoch, as we validated experimentally.

Figure XXX shows the cost per epoch achieved by different systems. 
For the Papers100M graph, which cannot be entirely cached in the distributed GPU cache, running DGL on an instance \emph{without} NVLink results in a much lower cost per epoch than using a caching-based system on NVLink.
Using a cheaper instance is preferable unless minimizing the absolute training time is the main goal.

The Orkut graph can be entirely cached in the distributed GPU cache.
In this case, using NVLink has a similar cost per epoch as PCIe, so using a high-end server is preferable since it can speed up training at no extra cost.
\end{comment}

% \mypar{Summary and motivation}
% Data-parallel training suffers from a fundamental problem, which is \emph{redundant data loading}. 
% In each training iteration of data-parallel training, two micro-batches (prepared for two GPUs separately) can share the same input vertices due to the interconnected nature of graphs. 
% The input features of these overlapped vertices need to be loaded twice, one for each GPU. 
% Caching speeds up some of these data loads but it does not fundamentally solve the redundant loading problem.
% The push-pull parallelism approach proposed by P3 eliminates redundant data loads but it introduces an expensive shuffle during training that adds a significant cost to the end-to-end training time.
% This cost often outweighs its benefits compared to data parallelism with caching.

\begin{figure*}[t]
    \centering
    \includegraphics[width=\linewidth]{fig/ModelStructure.pdf}
    \caption{Overall architecture of \name. The left part represents different modality-specific encoders to extract latent features and the multimodal fusion module to integrate multimodal representations. The right part represents the contextual relational model decoders to get the similarity score and the decision fusion module to make the final prediction on all modalities.}
    \label{fig:model}
\end{figure*}

\section{Methodology}

Formally, a knowledge graph is defined as $\mathcal{G} = \langle \mathcal{E}, \mathcal{R}, \mathcal{T} \rangle$, where $\mathcal{E}$ and $\mathcal{R}$ indicate sets of entities and relations, respectively. 
$\mathcal{T} = \{(h, r, t) | h, t \in \mathcal{E}, r \in \mathcal{R}\}$ represents relational triples of the KG.
In multimodal KGs, each entity in KGs is represented by multiple features from different modalities.
Here, we define the set of modalities $\mathcal{K} = \{s, v, t, m\}$ where $s, v, t, m$ denote structural, visual, textual and multimodal modality, respectively.
Due to the complexity of real-world knowledge, it is almost impossible to take all the triples into account.
Therefore, given a well-formulated KG, the \emph{Link Prediction} task aims at predicting missing links between entities.
Specifically, link prediction models expect to learn a score function of relational triples to estimate the likelihood of a triple, which is always formulated as $\psi : \mathcal{E} \times \mathcal{R} \times \mathcal{E} \to \mathbb{R}$.


\subsection{Overall Architecture}

In order to fully exploit the complicated interaction between different modalities, we propose a two-stage fusion model instead of simply considering the multimodal information separately in a unified vector space.
As shown in Figure~\ref{fig:model}, \name consists of four key components:
\begin{itemize}[leftmargin=*]
	\item[1] The Modality-Specific Encoders are used for extracting structural, visual and textual features as the input of multimodal fusion stage.
	\item[2] The Multimodal Fusion Module, which is the first fusion stage, effectively models bilinear interactions between different modalities based on \textit{Tucker} decomposition and contrastive learning.
	\item[3] The Contextual Relational Model calculates the similarity of contextual entity representations to formulate triple scores as modality-specific predictions for decision fusion stage.
	\item[4]  The Decision Fusion Module, which is the second fusion stage, takes all the similarity scores from structural, visual, textual and multimodal models into account to make the final prediction.
\end{itemize}

\subsection{Modality-Specific Encoders}
In this subsection, we first introduce the pre-trained encoders used for different modalities.
These encoders are not fine-tuned during training and we treat them as fixed feature extractors to obtain the modality-specific entity representations.
Note that \name is a general framework and it is straightforward to replace them with other up-to-date encoders or add ones for new modalities into \name.

\subsubsection{Structural Encoder}

From the most basic view, the structural information of KG, we employ a Graph Attention Network (GAT)\footnote{https://github.com/Diego999/pyGAT}~\cite{DBLP:conf/iclr/VelickovicCCRLB18} with TransE loss.

Specifically, our GAT encoder takes L1 distance of neighbor aggregated representations as energy function of triples, which is $E(h, r, t) = ||\mathbf{h}+\mathbf{r}-\mathbf{t}||$.
In the training process, we minimize the following Hinge loss~\eqref{eq-gat-loss}:
\begin{equation}\label{eq-gat-loss}
    \begin{split}
        \mathcal{L}_{GAT} = & \sum_{(h,r,t) \in \mathcal{T}}\sum_{(h', r, t') \in \mathcal{T'}} \mathrm{max} \{0,  \\
        &\gamma + E(h,r,t) - E(h',r,t')\}
    \end{split}
\end{equation}
where $\gamma$ is margin hyper-parameter and $\mathcal{T'}$ denotes set of negative triples derived from $\mathcal{T}$. 
$\mathcal{T'}$ is created by randomly replacing head or tail entities of triples in $\mathcal{T}$, which is~\eqref{eq-gat-neg}:
\begin{equation}\label{eq-gat-neg}
    \mathcal{T'} = \{(h',r,t)|h' \in \mathcal{E} \backslash h\} \cup \{(h,r,t')|t' \in \mathcal{E} \backslash t\}
\end{equation}

\subsubsection{Visual Encoder} 
Visual features are greatly expressive while providing different views of knowledge from traditional KGs. 
To effectively extract visual features, we utilize VGG16\footnote{https://github.com/machrisaa/tensorflow-vgg} pre-trained on \textit{ImageNet}\footnote{https://image-net.org/} to get image embeddings of corresponding entities following~\cite{DBLP:conf/esws/LiuLGNOR19}.
Specifically, we take outputs of the last hidden layer before softmax operation as visual features, which are 4096-dimensional vectors.

\subsubsection{Textual Encoder} 
Entity descriptions contain much richer but more complex knowledge than pure KGs.
To fully extract the complex knowledge, we employ BERT~\cite{DBLP:conf/naacl/DevlinCLT19} as the textual encoder, which is very expressive to get description embeddings of corresponding entities.
The textual features are 768-dimensional vectors, i.e., pooled outputs of pre-trained BERT-Base model\footnote{https://github.com/huggingface/transformers}.

\subsection{Multimodal Fusion}
The multimodal fusion stage aims to effectively get multimodal representations, which fully capture the complex interactions between different modalities.
Many existing multimodal fusion methods have achieved promising results in many tasks like VQA (Visual Question Answering).
However, most of them aim at finding the commonality to get more precise representations by modality projecting~\cite{DBLP:conf/nips/FromeCSBDRM13,DBLP:conf/aaai/CollellZM17} or cross-modal attention~\cite{DBLP:conf/aaai/PerezSVDC18}.
These types of methods will suffer from the loss of unique information in different modalities and can not achieve sufficient interaction between modalities.
To this end, we propose to employ the bilinear models, which have a strong ability to realize full parameters interaction as the cornerstone to perform the fusion of multimodal information.
Specifically, we extend the \textit{Tucker} decomposition, which decomposes the tensor into a core tensor transformed by a matrix along with each mode to 4-mode factors as expressed in Equation~\eqref{eq-tucker}:
\begin{equation}\label{eq-tucker}
    \mathcal{P} = (((\mathcal{P}_c \times \mathbf{M}_s) \times \mathbf{M}_v) \times \mathbf{M}_t) \times \mathbf{M}_d
\end{equation}
where $\mathbf{M}_s \in \mathbb{R}^{d_s \times t_s}$, $\mathbf{M}_v \in \mathbb{R}^{d_v \times t_v}$, $\mathbf{M}_t \in \mathbb{R}^{d_t \times t_t}$,  $\mathbf{M}_d \in \mathbb{R}^{\mathcal{D} \times t_d}$ denotes transformation matrix and $\mathcal{P}_c \in \mathbb{R}^{t_s \times t_v \times t_t \times t_d}$ denotes a smaller core tensor.

In such a situation, entity embeddings are first projected into a low-dimensional space and then fused with the core tensor $\mathcal{P}_c$.
Following~\cite{DBLP:conf/iccv/Ben-younesCCT17}, we further reduce the computation complexity by decomposing the core tensor $\mathcal{P}_c$ to merge representations of all modalities into a unified space with element-wise product.
The detailed calculation process is expressed as Equation~\eqref{eq-fusion}:
\begin{equation}\label{eq-fusion}
    \mathbf{e}_m = \tilde{\mathbf{e}}_s^\mathsf{T} \mathbf{M}_d^s * \tilde{\mathbf{e}}_v^\mathsf{T} \mathbf{M}_d^v * \tilde{\mathbf{e}}_t^\mathsf{T} \mathbf{M}_d^t
\end{equation}
where $\tilde{\mathbf{e}}_k = \mathrm{ReLU}(\mathbf{e}_k\mathbf{M}_k) \in \mathbb{R}^{t_k}$ denotes latent representations and $\mathbf{e}_k \in \mathbb{R}^{d_k}$ is the original embedding representations and $\mathbf{M}_d^k \in \mathbb{R}^{t_k \times t_d}$ is decomposed transformation matrix for each modality $k \in \{s, v, t\}$.

However, the multimodal bilinear fusion has no bound limitation while the gradient produced by the final prediction result can only implicitly guide parameter learning.
To alleviate this problem, we add constraints to limit the correlation between different modality representations of the same entity to be stronger.
Therefore, we further leverage contrastive learning~\cite{DBLP:conf/icml/ChenK0H20,DBLP:conf/nips/LiSGJXH21,DBLP:conf/cvpr/Yuan0K0WMKF21} between different entities and modalities as an additional learning objective for regularization.
In the settings of contrastive learning, we take the pairs of representations of the same entity of different modalities as positive samples and the pairs of representations of different entities as negative samples.
As shown in Figure~\ref{fig:cl}, we aim at limiting the distance of negative samples to be larger than positive samples to enhance multimodal fusion, which is:
\begin{equation}
    d(f(x), f(x^+)) << d(f(x), f(x^-))
\end{equation}
where $d(\cdot, \cdot)$ denotes the distance measure and $f(\cdot)$ denotes the embedding function. The superscript $+, -$ represent the positive and negative samples, respectively.

\begin{figure}
    \centering
    \includegraphics[width=\linewidth]{fig/ContrastiveLearning.pdf}
    \caption{Example of multimodal contrastive learning. The distance between the representations of the same entity in different modalities is minimized, while the distance between the representations of different entities is maximized.}
    \label{fig:cl}
\end{figure}

Specifically, we randomly sample $N$ entities from the entity set as a minibatch and define contrastive learning loss upon it.
The positive pairs are naturally obtained with the same entities while the negative pairs are constructed by negative sharing~\cite{DBLP:conf/kdd/ChenSSH17} of all other entities.
We take the latent representations $\tilde{\mathbf{e}}_k = \mathrm{ReLU}(\mathbf{e}_k\mathbf{M}_k) \in \mathbb{R}^{t_k}$ and leverage cosine similarity $d(u, v) = - \mathbf{u}^\mathsf{T}\mathbf{v}/||\mathbf{u}||\mathbf{v}||$ as distance measure.
Then we have the following contrastive loss function for each entity $i$:
\begin{equation}\label{eq-cl}
    \mathcal{L}_{CLi} = \frac{1}{3N} \sum_{p,q \in \mathcal{M}} \sum_{j=1}^N  d(e_i^{p}, e_i^{q}) - d(e_i^{p}, e_j^{q}) + 2
\end{equation}
where $\mathcal{M} = \{(s, v), (s, t), (v, t)\}$ is set of modality pairs.

\subsection{Contextual Relational Model}
After obtaining representations of each modality and multimodal, we then design a contextual relational model, which takes relations in triples as contextual information for scoring, to get the predictions.
Note that this relational model can be easily replaced by any scoring function like TransE.

Due to the variety and complexity of relations in KGs, we argue that improving the degree of parameter interaction~\cite{DBLP:conf/aaai/VashishthSNAT20} is crucial for better modeling the relational triples.
The degree of parameter interaction means the calculation ratio of each parameter to all other parameters. 
For example, dot product could achieve $1/d$ degree while cross product could achieve $(d-1)/d$ degree.
Based on this assumption, we propose to use bilinear outer product between entity and relation embeddings to incorporate contextual information into entity representations.
Instead of taking relations as input as in previous studies, our contextual relational model utilizes relations to provide context in the transformation matrix of entity embeddings.
Then, entity embeddings are projected using the contextual transformation matrix to get \emph{contextual embeddings}, which are used for calculating similarity with all candidate entities.
The learning objective is to minimize the binary cross-entropy loss.
For each modality $k \in \mathcal{K}$, the computation details are shown as Equation~\eqref{eq-crm} to Equation~\eqref{eq-loss}:
\begin{gather}
    \hat{\mathbf{e}}_k = \mathbf{e}_k^\mathsf{T}\mathbf{W}_k^r  + \mathbf{b} = \mathbf{e}_k^\mathsf{T}\mathbf{W}_k\mathbf{r} + \mathbf{b}_k \label{eq-crm} \\
    \mathbf{y}_k = \sigma(\mathrm{cosine}(\mathbf{e}_k, \hat{\mathbf{e}}_k)) = \sigma (\frac{\mathbf{e}_k \cdot \hat{\mathbf{e}}_k}   
    {|\mathbf{e}_k| |\hat{\mathbf{e}}_k|}) \label{eq-sim} \\
    \mathcal{L}_k = -\frac{1}{N} \sum_{i=1}^N (t_i \cdot \mathrm{log}(y_{i,k})+(1-t_i) \cdot \mathrm{log}(1-y_{i,k})) \label{eq-loss}
\end{gather}
where $\mathbf{e}_k$ and $\hat{\mathbf{e}}_k$ are original and contextual entity embeddings respectively;
$\mathbf{W}_k^r = \mathbf{W}_k \mathbf{r}$ denotes contextual transformation matrix which is obtained by matrix multiplication of weight matrix $\mathbf{W}_k$ and relation vectors $\mathbf{r}$ while $\mathbf{b}_k$ is a bias vector;
$\sigma$ is sigmoid function and $\mathbf{y}_k = [y_{1,k},y_{2,k},...,y_{N,k}]$ is final prediction of modality $k$.

\subsection{Decision Fusion}
Existing multimodal approaches mainly focus on projecting different modality representations into a unified space and predicting with commonality between modalities, which will fail to preserve the modality-specific knowledge.
We alleviate this problem in the decision fusion stage by joint learning and combining predictions of different modalities to further leverage the complementarity.

Under the multimodal settings, we assign different contextual relational models for each modality and utilize their own results for training in different views.
Recall the contrastive learning loss in Equation~\eqref{eq-cl}, the overall training objective is to minimize the joint loss shown in Equation~\eqref{eq-mmloss}:
\begin{equation}\label{eq-mmloss}
    \mathcal{L}_{Joint} = \gamma_s \mathcal{L}_s + \gamma_v \mathcal{L}_v + \gamma_t \mathcal{L}_t + \gamma_m \mathcal{L}_{m} + \mathcal{L}_{CL}
\end{equation}
where $\mathcal{L}_k$ denotes binary cross entropy loss for modality $k$ as Equation~\eqref{eq-loss} and $\gamma_k$ is a learned weight parameter.

\begin{algorithm}[t]
\caption{Optimization Algorithm.}\label{alg:optim}
\begin{algorithmic}[1]
\STATE \textbf{Input:} Multimodal Knowledge Graph $\mathcal{G}$
\STATE \textbf{Output:} Trained Model $\mathcal{M}$
\STATE Pre-train structural encoder GAT on $\mathcal{G}$ with the loss in Equation(1)
\STATE Obtain pre-trained visual encoder VGG16 and textual encoder BERT-base
\STATE Initialize the entity embeddings $\mathbf{E}_s, \mathbf{E}_v, \mathbf{E}_t$ in $\mathcal{M}$ with the outputs of pre-trained encoders
\WHILE{not converge}
    \STATE Sample a batch of entities from $\mathcal{G}$
    \FOR{Entity $e$ in batch}
    \STATE Obtain the structural, visual, textual embeddings $\mathbf{e}_s, \mathbf{e}_v, \mathbf{e}_t$ of entity $e$
    \STATE Compute the multimodal fused embeddings $\mathbf{e}_m$ of entity $e$ with Equation (4)
    \STATE Compute the contrastive learning loss $\mathcal{L}_{CL}$ with Equation (6)
    \STATE Compute the loss $\mathcal{L}_s, \mathcal{L}_v, \mathcal{L}_t, \mathcal{L}_m$ with modality-specific scorers via Equation (7) - Equation (9)
    \STATE Compute the joint loss $\mathcal{L}_{Joint}$ with the above losses $\mathcal{L}_s, \mathcal{L}_v, \mathcal{L}_t, \mathcal{L}_m, \mathcal{L}_{CL}$ via Equation (10)
    \STATE Update model parameters of $\mathcal{M}$ by minimizing $\mathcal{L}_{Joint}$
    \ENDFOR
\ENDWHILE
\RETURN $\mathcal{M}$
\end{algorithmic}
\end{algorithm}

To better illustrate the whole training process of \name, we describe it via the pseudo-code of the optimization algorithm.
As shown in Algorithm~\ref{alg:optim}, we first obtain the pre-trained encoders of structural, visual and textual and utilize them for entity embeddings (line 3-5).
Since the pre-trained models are much larger and more complex than \name, they are not fine-tuned and their outputs are directly used as inputs of \name.
The multimodal embeddings are obtained by multimodal fusion while contrastive learning is applied to further enhance the fusion stage (line 9-11).
During training, each modality delivers its own prediction and loss via the modality-specific scorers (line 12), and then the joint prediction and loss are computed based on all modalities including multimodal ones (line 14).

For inference, we propose to jointly consider the predictions of each modality as well as multimodal ones.
Specifically, the overall predictions are shown in Equation~\eqref{eq-df}:
\begin{equation}\label{eq-df}
    \mathbf{y}_{Joint} = \frac{\gamma_s \mathbf{y}_s + \gamma_v \mathbf{y}_v + \gamma_t \mathbf{y}_t + \gamma_m \mathbf{y}_m} {\gamma_s + \gamma_v + \gamma_t + \gamma_m}
\end{equation}
where $\gamma_k$ denotes weight for modality $k$ as same as Equation~\eqref{eq-mmloss} while the values in $\mathbf{y}$ are in [0, 1].



\section{Experimental Results}
In this section, we validate the effectiveness of our proposal. We first introduce datasets, metrics and implementation details involved in our evaluation. Then, we compare \netname{} with state-of-the-art methods, conduct an ablation study on our model and, finally, discuss its limitations.


\begin{table*}[htbp] \scriptsize
	\renewcommand\tabcolsep{2.3pt} 
	\centering
	\scalebox{0.85}{
	\begin{tabular}{@{}ccccccccccccccccc@{}}
		\toprule
		 Dataset & Scale & Metrics & GF~\cite{he2010guided} & SD~\cite{ham2017robust}  & GSRPT~\cite{lutio2019guided} & MSG~\cite{hui2016depth} & DKN~\cite{kim2021deformable} & FDKN~\cite{kim2021deformable} & PMBANet~\cite{ye2020pmbanet} & FDSR~\cite{he2021towards} & JIIF~\cite{tang2021joint} & DCTNet~\cite{zhao2022discrete} & LGR~\cite{de2022learning} & DADA~\cite{metzger2022guided} & DSR-EI & DSR-EI$^+$ \\ \midrule
		\multirow{6}{*}{\rotatebox[origin=l]{90}{\scriptsize \textbf{Middlebury}}} & \multirow{2}{*}{$4\times$} 
		& MSE & 33.3 & 24.9 & 39.8 & 4.13 & 4.29 & 3.60 & 4.72 & 7.72 & 2.70 & 5.00 & 3.04 & \bronze{2.58} & \gold{2.46} & \silver{2.56} \\
		& & MAE & 1.27 & 0.46 & 0.79 & 0.22 & 0.18 & 0.16 & 0.25 & 0.35 & \bronze{0.11} & 0.24 & 0.13 & \bronze{0.11} & \silver{0.08} & \gold{0.07} \\ \cline{2-17}
		& \multirow{2}{*}{$8\times$} 
		& MSE & 40.5 & 82.5 & 32.7 & 10.5 & 11.2 & 10.4 & 9.48 & 23.2 & 8.01 & 15.1 & 7.26 & \silver{5.68} & \bronze{6.20} & \gold{5.13} \\
		& & MAE & 1.49 & 0.86 & 0.82 & 0.43 & 0.38 & 0.37 & 0.38 & 0.69 & 0.27 & 0.57 & 0.24 & \bronze{0.20} & \gold{0.18} & \gold{0.18} \\ \cline{2-17}
		& \multirow{2}{*}{$16\times$} 
		& MSE & 67.4 & 511 & 41.5 & 34.2 & 47.6 & 38.5 & 30.6 & 55.4 & 37.5 & 52.3 & 24.7 & \silver{16.3} & \gold{15.8} & \bronze{16.6}  \\
		& & MAE & 2.21 & 1.73 & 1.24 & 1.06 & 1.42 & 1.18 & 0.89 & 1.51 & 0.98 & 1.50 & 0.67 & \bronze{0.48} & \silver{0.47} & \gold{0.40} \\ \hline\hline
	    % middlebury end
		\multirow{6}{*}{\rotatebox[origin=l]{90}{\scriptsize \textbf{NYUv2}}} & \multirow{2}{*}{$4\times$}
		& MSE & 114 & 36.0 & 112 & 6.85 & 11.4 & 9.07 & 10.8 & 10.1 & \bronze{3.28} & 3.63 & 6.45 & 4.83 & \silver{2.82} & \gold{2.75}\\
		& & MAE & 3.91 & 1.31 & 3.61 & 0.81 & 1.03 & 0.85 & 0.93 & 0.94 & \bronze{0.52} & 0.68 & 0.73 & 0.64 & \silver{0.49} & \gold{0.47}\\ \cline{2-17}
		& \multirow{2}{*}{$8\times$} 
		& MSE & 142 & 105 & 122 & 24.1 & 29.8 & 29.9 & 17.2 & 19.5 & \bronze{15.2} & 20.9 & 19.6 & 16.6 & \gold{11.8} & \gold{11.8}\\
		& & MAE & 4.47 & 2.57 & 3.86 & 1.66 & 1.82 & 1.80 & 1.38 & 1.38 & \bronze{1.29} & 1.79 & 1.42 & 1.30 & \silver{1.12} & \gold{1.09}\\ \cline{2-17}
		& \multirow{2}{*}{$16\times$} 
		& MSE & 249 & 533 & 219 & 84.5 & 115 & 113 & 84.9 & 86.4 & 59.9 & 77.0 & 67.5 & \bronze{59.0} & \silver{47.8} & \gold{47.1} \\
		& & MAE & 6.34 & 5.07 & 5.40 & 3.35 & 4.01 & 3.95 & 3.26 & 3.35 & 2.81 & 3.61 & 2.90 & \bronze{2.64} & \silver{2.48} & \gold{2.40}\\ \hline\hline
		% NYU end
		\multirow{6}{*}{\rotatebox[origin=l]{90}{\scriptsize \textbf{DIML}}} & \multirow{2}{*}{$4\times$}
		& MSE & 25.6 & 10.5 & 20.7 & 1.73 & 3.47 & 2.20 & 3.05 & 2.75 & \bronze{1.19} & 2.09 & 1.68 & 1.33 & \silver{0.70} & \gold{0.65} \\
		& & MAE & 1.45 & 0.40 & 1.15 & 0.22 & 0.33 & 0.23 & 0.31 & 0.29 & \bronze{0.16} & 0.31 & 0.20 & 0.17 & \silver{0.13} & \gold{0.12} \\ \cline{2-17}
		& \multirow{2}{*}{$8\times$} 
		& MSE & 34.1 & 44.9 & 23.0 & 4.13 & 5.47 & 5.95 & 5.87 & 8.40 & 3.65 & 7.08 & 3.51 & \bronze{2.93} & \silver{2.12} & \gold{2.09} \\
		& & MAE & 1.77 & 0.83 & 1.26 & 0.40 & 0.45 & 0.47 & 0.47 & 0.66 & 0.32 & 0.65 & 0.31 & \bronze{0.28} & \gold{0.22} & \gold{0.22} \\ \cline{2-17}
		& \multirow{2}{*}{$16\times$} 
		& MSE & 66.3 & 41.1 & 39.3 & 13.0 & 19.3 & 20.8 & 13.8 & 32.9 & 11.7 & 23.4 & 9.45 & \bronze{7.61} & \gold{6.29} & \silver{6.31} \\
		& & MAE & 2.74 & 1.91 & 1.78 & 0.93 & 1.20 & 1.24 & 0.87 & 1.66 & 0.81 & 1.75 & 0.68 & \bronze{0.59} & \silver{0.52} & \gold{0.50} \\
		% DIML end
    \bottomrule
	\end{tabular}}
    \vspace{-0.3cm}
	\caption{\textbf{Results on Middlebury, NYUv2 and DIML datasets.} The lower the MSE and MAE, the better.}
	\label{sota_comparison_mid_nyu_diml}
\end{table*}



\begin{table*}[t] \footnotesize
	\renewcommand\tabcolsep{1.5pt} 
	\centering
	\scalebox{0.85}{
	\begin{tabular}{@{}ccccccccccccccccc@{}}
		\toprule
		 Scale & SDF~\cite{li2016deep} & SVLRM~\cite{pan2019spatially} & DJF~\cite{li2016deep} & DJFR~\cite{li2019joint} & PAC~\cite{su2019pixel} & CUNet~\cite{deng2020deep} & FDKN~\cite{kim2021deformable} & DKN~\cite{kim2021deformable} & FDSR~\cite{he2021towards} & DCTNet~\cite{zhao2022discrete} & RSAG~\cite{yuan2023recurrent} & DSR-EI & DSR-EI$^+$ \\ \midrule
		$4\times$ & 2.00 & 3.39 & 3.41 & 3.35 & 1.25 & 1.18 & 1.18 & 1.30 & 1.16 & \bronze{1.07} & 1.14 & \gold{0.91} & \gold{0.91} \\
		$8\times$ & 3.23 & 5.59 & 5.57 & 5.57 & 1.98 & 1.95 & 1.91 & 1.96 & 1.82 & 1.78 & \bronze{1.75} & \gold{1.37} & \silver{1.38} \\
		$16\times$ & 5.16 & 8.28 & 8.15 & 7.99 & 3.49 & 3.45 & 3.41 & 3.42 & 3.06 & 3.18 & \bronze{2.96} & \gold{2.10} & \gold{2.10}  \\
    \bottomrule
	\end{tabular}}
	\vspace{-0.3cm}
	\caption{\textbf{Results on the RGBDD dataset.} We report RMSE, the lower the better.}
	\label{sota_comparison_rgbdd}
\end{table*}



\subsection{Datasets and Metrics}
We evaluate \netname{} on four datasets, compared with existing methods when super-solving depth maps by three different upsampling factors: $4\times,\ 8\times$, and $16\times$. 

\textbf{Middlebury}\cite{scharstein2003high,scharstein2007learning,hirschmuller2007evaluation,scharstein2014high}. We train all learning-based methods using 50 RGB-D images with ground truth from Middlebury 2005, 2006 and 2014 datasets. As in~\cite{de2022learning}, we retain 5 for validation and 5 for testing. 

\textbf{NYUv2}\cite{silberman2012indoor}. It contains 1449 RGB-D images in total. Following \cite{de2022learning}, we randomly split it into 849 RGB-D images for the training set, 300 for the validation set and 300 for the test set. Compared to \cite{ye2020pmbanet,liu2022pdr}, it comes with a validation set to make the comparison fairer.

\textbf{DIML}\cite{kim2016structure,kim2017deep,kim2018deep,cho2021deep} consists of 2 million color images and corresponding depth maps from indoor and outdoor scenes. We adopt the same strategy outlined in \cite{de2022learning}, i.e., considering only the indoor data subset, and use 1440 for training, 169 for validation, and 503 for testing.

\textbf{RGBDD}\cite{he2021towards} is a new real-world dataset for GDSR, which consists of 4811 image pairs. For evaluation, we follow the protocol described in \cite{he2021towards}, using 2215 images (1586 portraits, 380 plants, 249 models) as the training set and 405 images (297 portraits, 68 plants, 40 models) as the test set. 

\textbf{Metrics.} Following \cite{de2022learning}, we compute mean square error (MSE / $cm^2$) and mean absolute error (MAE / $cm$) as metrics on Middlebury, NYUv2 and DIML. For RGBDD, we use root mean square error (RMSE / $cm$) as in \cite{he2021towards}. 

\subsection{Implementation Details}
During training, the HR depth maps and the color images are randomly cropped into $256\times 256$ patches. LR depth patches are generated by bicubic interpolation at $64\times 64$, $32\times 32$, $16\times 16$ resolution for $4\times$, $8\times$ and $16\times$ factors, respectively. We randomly extract about 75K, 168K, 223K and 232K patches from Middlebury, NYUv2, DIML and RGBDD for training. Before being fed to the network, depth maps and images are normalized in the [0, 1] range.

We use Pytorch \cite{paszke2019pytorch} to implement and train \netname{}, on a single Nvidia RTX 3090 GPU. The batch size is set to 4, using Adam as the optimizer. The learning rate is initialized to $1\times 10^{-4}$, then performing a 5-epoch warm-up and cosine annealing. We use random rotation, horizontal/vertical flipping as data augmentation. According to the size of the four datasets, we train our network for 1505, 198, 155 and 109 epochs on Middlebury, NYUv2, DIML and RGBDD, respectively. 
When evaluating results on a specific dataset, we do not perform any pre-training on the others. Following \cite{de2022learning}, testing is performed by processing $256\times256$ patches at a time on Middlebury, NYUv2 and DIML for fairness, while full-resolution images are processed for RGBDD.

\begin{figure*}[t] 
	\centering
	\renewcommand\tabcolsep{1.5pt} 
	\begin{tabular}{cccccccccccc}
	\vspace{-0.1cm}
    \rotatebox[origin=l]{90}{\scriptsize \quad \textbf{Middlebury}} & \includegraphics[height=0.6in]{./figs/sota_comp_middlebury/389/Middlebury_389_img.pdf}
        \hspace{-1.8mm} & \includegraphics[height=0.6in]{./figs/sota_comp_middlebury/389/Middlebury_389_source.pdf}
	\hspace{-1.8mm} &  \includegraphics[height=0.6in]{./figs/sota_comp_middlebury/389/Middlebury_389_GT.pdf}
	\hspace{-1.8mm} & \includegraphics[height=0.6in]{./figs/sota_comp_middlebury/389/Middlebury_389_PMBA.pdf}
	\hspace{-1.8mm} & \includegraphics[height=0.6in]{./figs/sota_comp_middlebury/389/Middlebury_389_FDSR.pdf}
	\hspace{-1.8mm} & \includegraphics[height=0.6in]{./figs/sota_comp_middlebury/389/Middlebury_389_JIIF.pdf}
	\hspace{-1.8mm} & \includegraphics[height=0.6in]{./figs/sota_comp_middlebury/389/Middlebury_389_DCTnet.pdf}
	\hspace{-1.8mm} & \includegraphics[height=0.6in]{./figs/sota_comp_middlebury/389/Middlebury_389_LGR.pdf}
	\hspace{-1.8mm} & \includegraphics[height=0.6in]{./figs/sota_comp_middlebury/389/Middlebury_389_MSS.pdf}
        
        \hspace{-1.8mm} & \includegraphics[height=0.6in]{./figs/sota_comp_middlebury/389/Middlebury_389_ours.pdf}
    \\ \vspace{-0.1cm}
    
    \rotatebox[origin=l]{90}{\scriptsize \quad \textbf{NYUv2}} & \includegraphics[height=0.6in]{./figs/sota_comp_nyu/357/NYU_357_img.pdf}
	\hspace{-1.8mm} & \includegraphics[height=0.6in]{./figs/sota_comp_nyu/357/NYU_357_source.pdf}
	\hspace{-1.8mm} & \includegraphics[height=0.6in]{./figs/sota_comp_nyu/357/NYU_357_GT.pdf}
	\hspace{-1.8mm} & \includegraphics[height=0.6in]{./figs/sota_comp_nyu/357/NYU_357_PMBA.pdf}
	\hspace{-1.8mm} & \includegraphics[height=0.6in]{./figs/sota_comp_nyu/357/NYU_357_FDSR.pdf}
	\hspace{-1.8mm} & \includegraphics[height=0.6in]{./figs/sota_comp_nyu/357/NYU_357_JIIF.pdf}
	\hspace{-1.8mm} & \includegraphics[height=0.6in]{./figs/sota_comp_nyu/357/NYU_357_DCTnet.pdf}
	\hspace{-1.8mm} & \includegraphics[height=0.6in]{./figs/sota_comp_nyu/357/NYU_357_LGR.pdf}
	\hspace{-1.8mm} & \includegraphics[height=0.6in]{./figs/sota_comp_nyu/357/NYU_357_MSS.pdf}
 
	\hspace{-1.8mm} & \includegraphics[height=0.6in]{./figs/sota_comp_nyu/357/NYU_357_ours.pdf}
	\\ 
	
    \rotatebox[origin=l]{90}{\scriptsize \quad \textbf{DIML}} & \includegraphics[height=0.6in]{./figs/sota_comp_diml/856/DIML_856_img.pdf}
	\hspace{-1.8mm} & \includegraphics[height=0.6in]{./figs/sota_comp_diml/856/DIML_856_source.pdf}
	\hspace{-1.8mm} & \includegraphics[height=0.6in]{./figs/sota_comp_diml/856/DIML_856_GT.pdf}
	\hspace{-1.8mm} & \includegraphics[height=0.6in]{./figs/sota_comp_diml/856/DIML_856_PMBA.pdf}
	\hspace{-1.8mm} & \includegraphics[height=0.6in]{./figs/sota_comp_diml/856/DIML_856_FDSR.pdf}
	\hspace{-1.8mm} & \includegraphics[height=0.6in]{./figs/sota_comp_diml/856/DIML_856_JIIF.pdf}
	\hspace{-1.8mm} & \includegraphics[height=0.6in]{./figs/sota_comp_diml/856/DIML_856_DCTnet.pdf}
	\hspace{-1.8mm} & \includegraphics[height=0.6in]{./figs/sota_comp_diml/856/DIML_856_LGR.pdf}
	\hspace{-1.8mm} & \includegraphics[height=0.6in]{./figs/sota_comp_diml/856/DIML_856_MSS.pdf}
 
	\hspace{-1.8mm} & \includegraphics[height=0.6in]{./figs/sota_comp_diml/856/DIML_856_ours.pdf}
 \\
	& \scriptsize \textbf{(a)} RGB & \scriptsize \textbf{(b)} Bicubic & \scriptsize \textbf{(c)} GT & \scriptsize \textbf{(d)} PMBA & \scriptsize \textbf{(e)} FDSR & \scriptsize \textbf{(f)} JIIF & \scriptsize \textbf{(g)} DCTNet & \scriptsize \textbf{(h)} LGR & \scriptsize \textbf{(i)} \netname{} & \scriptsize \textbf{(j)} \netname{} (depth)
	\end{tabular}
    \vspace{-0.3cm}
	\caption{\textbf{Qualitative comparison on Middlebury, NYUv2 and DIML datasets (scaling factor $8\times$).} From left to right: (a) RGB image, (b) Bicubic upsampled depth map, (c) GT; then, error maps achieved by selected methods: (d) PMBA~\cite{ye2020pmbanet}, (e) FDSR~\cite{he2021towards}, (f) JIIF~\cite{tang2021joint}, (g) DCTNet~\cite{zhao2022discrete}, (h) LGR~\cite{de2022learning}; finally, (i) error maps and (j) predictions by \netname.} 
	\label{qualitative}
\end{figure*}


\begin{table*}[htbp] \footnotesize
	\renewcommand\tabcolsep{1.5pt} 
	\centering
	\scalebox{0.85}{
	\begin{tabular}{@{}ccccccccccccccccc@{}}
		\toprule
		 Testing Dataset & Metric & GF\cite{he2010guided} & SD~\cite{ham2017robust}  & GSRPT~\cite{lutio2019guided} & MSG~\cite{hui2016depth} & FDKN~\cite{kim2021deformable} & PMBANet~\cite{ye2020pmbanet} & FDSR~\cite{he2021towards} & JIIF~\cite{tang2021joint} & DCTNet~\cite{zhao2022discrete} & LGR~\cite{de2022learning} & \netname$^+$ \\ \midrule
		\multirow{2}{*}{DIML}
		& MSE & 34.1 & 44.9 & 23.0 & 5.76 & 6.74 & 7.35 & 7.73 & \silver{4.10} & 5.64 & \bronze{4.95} & \gold{3.72} \\
		& MAE & 1.77 & 0.83 & 1.26 & 0.51 & 0.53 & 0.59 & 0.74 & \silver{0.38} & 0.77 & \bronze{0.40} & \gold{0.36} \\ \hline
		\multirow{2}{*}{Middlebury\textit{-HR}}
		& MSE & 40.5 & 82.5 & 32.7 & 11.0 & \bronze{10.0} & \silver{9.62} & 18.4 & 19.3 & 17.5 & \gold{8.25} & 14.6 \\
		& MAE & 1.49 & 0.86 & 0.82 & 0.54 & \silver{0.43} & \bronze{0.46} & 0.73 & 0.74 & 0.77 & \gold{0.35} & 0.54  \\ \hline
		\multirow{2}{*}{Middlebury\textit{-LR}}
		& MSE & 25.6 & 28.8 & 15.8 & 8.89 & 5.54 & 4.16 & 6.92 & 4.40 & 6.96 & 5.94 & \gold{3.44} \\
		& MAE & 2.31 & 2.07 & 1.73 & 1.62 & 0.99 & \silver{0.91} & 1.09 & \bronze{0.92} & 1.15 & 1.11 & \gold{0.87}  \\
        \bottomrule
	\end{tabular}}
	\vspace{-0.3cm}
	\caption{\textbf{Cross-dataset generalization.} All methods are trained on NYUv2 and tested on DIML/Middlebury with factor $8\times$. Middlebury\textit{-HR} is the test set defined in \cite{de2022learning}, Middlebury\textit{-LR} is the one from \cite{tang2021joint}. The lower MSE and MAE, the better. }
	\label{cross-data_comparison}
\end{table*}

\subsection{Comparison with State-of-the-Art}
We compare \netname{} to GF \cite{he2010guided}, SD \cite{ham2017robust}, GSRPT \cite{lutio2019guided}, MSG \cite{hui2016depth}, DKN and its fast implementation FDKN \cite{kim2021deformable}, PMBANet \cite{ye2020pmbanet}, FDSR \cite{he2021towards}, JIIF \cite{tang2021joint}, DCTNet \cite{zhao2022discrete}, LGR \cite{de2022learning}, and finally to DADA~\cite{metzger2022guided} on Middlebury, NYUv2 and DIML datasets. We could not compare with PDRNet \cite{liu2022pdr} under the same setting because the source code is unavailable at the time of writing. For the other methods, we use the results from \cite{de2022learning} or the officially published codes, and results from \cite{yuan2023recurrent,metzger2022guided} for concurrent works. On the RGBDD dataset, the proposed network is compared to SDF~\cite{li2016deep}, SVLRM \cite{pan2019spatially}, DJF~\cite{li2016deep}, DJFR~\cite{li2019joint}, PAC~\cite{su2019pixel}, CUNet~\cite{deng2020deep}, FDKN~\cite{kim2021deformable}, DKN~\cite{kim2021deformable}, FDSR~\cite{he2021towards}, DCTNet~\cite{zhao2022discrete} and RASG~\cite{yuan2023recurrent}. To be fair with DCTNet~\cite{zhao2022discrete}, we downsample depth maps as the LR input.  
When reporting results, we highlight \gold{absolute}, \silver{second} and \bronze{third} best methods for each metric on each dataset.

\textbf{Quantitative Comparison.} Tabs. \ref{sota_comparison_mid_nyu_diml} and \ref{sota_comparison_rgbdd} report the accuracy of super-solved depth maps at factors $4\times$, $8\times$ and $16\times$ on the four datasets. As expected, learning-based methods show a significant improvement over traditional methods \cite{he2010guided,ham2017robust,lutio2019guided}. \netname{} vastly outperforms any existing network, with larger gaps in accuracy with the increasing of the upsampling factor. This can be attributed to the limitations affecting existing methods, i.e., 1) the guidance of either explicit or implicit RGB features alone being insufficient; 2) multi-modal information fusion on a single scale being not flexible enough to deal with complex scenes. Both limitations are fully addressed by \netname, which consistently outperforms concurrent works \cite{metzger2022guided,yuan2023recurrent}. 


The margin is consistent both on perfect (Middlebury) and noisy datasets (NYUv2, DIML, RGBDD), with the latter being a more challenging, realistic benchmark. Although \netname$^+$ is definitely the absolute best, its margin over \netname{} is negligible, with tiny gains yielded by NLSPN with respect to our main modules. Indeed, \netname{} alone consistently outperforms any other approach already.

       
\textbf{Qualitative Comparison.}
Fig. \ref{qualitative} shows qualitative comparisons of $8\times$ super-solved depth maps on Middlebury, NYUv2 and DIML datasets, respectively. From left to right, we show, the RGB image and LR depth map, followed by the ground truth HR depth and error maps obtained by several state-of-the-art frameworks, concluding with ours in the second-to-last columns. In each of the three examples, the lower error magnitude produced by \netname{}$^+$ further demonstrates its superior accuracy. 

\textbf{Cross-dataset Generalization.}
We conclude the comparison with existing methods by conducting cross-dataset experiments with $8\times$ factor. All methods are trained on the NYUv2 dataset and directly evaluated on DIML and Middlebury. Table \ref{cross-data_comparison} collects quantitative results for the 11 selected methods. Again, CNN-based methods attain better performance than traditional approaches, despite the domain gap playing a significant role in performance -- as evident by comparing results with Table \ref{cross-data_comparison}. Nonetheless, \netname{} outperforms any other framework on DIML. 


\begin{figure}	
	\centering	
	\captionsetup[subfigure]{font=footnotesize,textfont=footnotesize}
	\subfloat[RGB]{	
		\centering	
		\label{cross_dataset} 
		\includegraphics[height=0.8in]{./figs/ablation_figure/cross_dataset/receptive_field/cross_dataset.pdf}}	
	\hspace{-2mm}
	\subfloat[$D_{hr}$]{	
		\centering	
		\label{HR}
		\includegraphics[width=0.8in]{./figs/ablation_figure/cross_dataset/receptive_field/HR.pdf}}
	\hspace{-2mm}
	\subfloat[$D_{lr}$]{	
		\centering	
		\label{LR}
		\includegraphics[width=0.8in]{./figs/ablation_figure/cross_dataset/receptive_field/LR.pdf}}
		\vspace{-0.3cm}
	\caption{\textbf{Image context processed on Middlebury -- HR vs LR.} (a) RGB image and depth patches $D$ processed when testing on (b) Middlebury\textit{-HR} and (c) Middlebury\textit{-LR}. }	
	\label{hr-lr} 
\end{figure}

When considering the Middlebury dataset, we evaluate using the setting proposed in \cite{de2022learning} -- Middlebury\textit{-HR} in the table. In this case, our results are slightly less accurate compared to a few existing methods. However, given the very high resolution of Middlebury images, we argue that this testing protocol -- i.e., consisting of processing $256\times 256$ crops at a time -- penalizes our network's ability to leverage the global context in the input that results irremediably reduced to a very local area in these images. Therefore, we also evaluate on Middlebury test set defined by~\cite{tang2021joint} -- Middlebury-\textit{LR} in the table. Note that different subsets of images are used in Middlebury\textit{-HR} and Middlebury-\textit{LR} splits. Besides, Middlebury-\textit{LR} images are resized and processed without cropping, i.e., used at full-size after resizing, allowing to fully exploit global context, while this is not feasible with Middlebury-\textit{HR} due to memory constraints. In this case, \netname{} attains the best performance again, confirming our previous analysis, as shown in Tab. \ref{cross-data_comparison}. Such a difference in terms of context is highlighted in Fig. \ref{hr-lr}.

\begin{table}[t]
    \centering
	\renewcommand\tabcolsep{3pt} 
    \scalebox{0.5}{
    \begin{tabular}{ccc}

    \begin{tabular}{@{}ccccccc@{}} %\label{hf_infomation}
		\toprule
		\textbf{No.} & \textbf{Gradient} & \tabincell{c}{\textbf{Shallow} \\ \textbf{Feature}} & \textbf{LCF} & \textbf{ResBlock} & \textbf{MSE} & \textbf{MAE}\\
		\midrule
		(\uppercase\expandafter{\romannumeral1}) & \XSolidBrush &  \Checkmark     &  \Checkmark &  & 13.1 & 1.19 \\
		(\uppercase\expandafter{\romannumeral2}) & \Checkmark &    \XSolidBrush   &   &  & 12.4 & 1.14 \\
		(\uppercase\expandafter{\romannumeral3}) & \Checkmark &    \Checkmark     &   & \Checkmark & 12.3 & 1.15 \\
		\rowcolor{LightYellow}
		(\uppercase\expandafter{\romannumeral4}) & \Checkmark &    \Checkmark     & \Checkmark  &  & \gold{11.8} & \gold{1.12} \\
		\bottomrule
	\end{tabular}
	
	& \quad &
	
	\begin{tabular}{@{}clcc@{}} %\label{edge_types}
		\toprule
		\specialrule{0em}{3pt}{3pt}
		\multicolumn{1}{c}{\textbf{No.}} & 
		\tabincell{l}{\textbf{HF Information} \textbf{ \quad\quad\quad\quad}} & \textbf{MSE} & \textbf{MAE}\\
		\specialrule{0em}{3pt}{2pt}
		\midrule
		(\uppercase\expandafter{\romannumeral1}) & 
		{Canny Edge} & 12.0 & 1.13 \\
		(\uppercase\expandafter{\romannumeral2}) & 
		{Gaussian Edge} & 12.1 & 1.16 \\
		(\uppercase\expandafter{\romannumeral3}) & 
		{DCT} & 12.1 & 1.15 \\
		(\uppercase\expandafter{\romannumeral4}) & 
		{Wavelet Transform} & 12.1 & 1.15  \\
		\rowcolor{LightYellow}
		(\uppercase\expandafter{\romannumeral5}) & 
		{Gradient Map} & \gold{11.8} & \gold{1.12} \\
		\bottomrule
	\end{tabular}
	
	\\
	\textbf{(a)} & \quad & \textbf{(b)} 
	\\
	\\
	
	
	\begin{tabular}{@{}clcccc@{}} %\label{dsp_ablation}
		\toprule
		\textbf{No.} & \textbf{Config.} & \textbf{Params (M)} & \textbf{Flops (G)} & \textbf{MSE} & \textbf{MAE}\\
		\midrule
		(\uppercase\expandafter{\romannumeral1}) & EdgeNet \cite{liu2021multi}    & 5.78 &  95.6  & 12.0 & \gold{1.12} \\
		(\uppercase\expandafter{\romannumeral2}) & SCPA \cite{zhao2020efficient}  & 0.29 &  13.1  & 12.5 & 1.16 \\
		\rowcolor{LightYellow}
		(\uppercase\expandafter{\romannumeral3}) & HFEB       & \gold{0.27} & \gold{11.6}  & \gold{1.18} & \gold{1.12} \\
		\rowcolor{white}
		\bottomrule
		\multicolumn{4}{c}{\quad\quad\textbf{(c)}} \\
		\\
		\toprule
		\textbf{No.} & \textbf{Config.} & \textbf{Params (M)} & \textbf{MSE} & \textbf{MAE}\\
		\midrule
		(\uppercase\expandafter{\romannumeral1}) & 
		w/o AFFM        & -   & 12.7 & 1.16 \\
		(\uppercase\expandafter{\romannumeral2}) & 
		w/o att         & 1.3 & 12.2 & 1.13 \\
		(\uppercase\expandafter{\romannumeral3}) & 
		Concat.  & 4.5 & 12.2 & 1.13 \\
		\rowcolor{LightYellow}
		(\uppercase\expandafter{\romannumeral4}) & 
		AFFM & 3.0 & \gold{11.8} & \gold{1.12} & \\
		\rowcolor{white}
		\bottomrule
		\multicolumn{4}{c}{\quad\quad\textbf{(e)}} \\
	\end{tabular}
	
	
	& \quad &
	
	
	\begin{tabular}{@{}clccc@{}} %\label{affm_setting}
		\toprule
		\textbf{No.} & \textbf{Scales} & \textbf{Params (M)} & \textbf{MSE} & \textbf{MAE}\\
		\midrule
		(\uppercase\expandafter{\romannumeral1}) & 
		H1              & 1.5 & 12.3 & 1.14 \\
		\rowcolor{LightYellow}
		(\uppercase\expandafter{\romannumeral2}) & 
		H1, H2       & 3.0 & \gold{11.8} & \gold{1.12} \\
		\rowcolor{white}
		(\uppercase\expandafter{\romannumeral3}) & 
		H1, H2, H3              & 4.5 & \gold{11.8} & \gold{1.12} \\
		\bottomrule
		\multicolumn{4}{c}{\textbf{(d)}} \\
% 		\\
% 		\\
        \specialrule{0em}{5.4pt}{5.4pt} %
		\toprule
		\specialrule{0em}{1.7pt}{1.7pt} %
		\textbf{No.} & \textbf{Stages} & \textbf{Params (M)} & \textbf{MSE} & \textbf{MAE}\\
		\specialrule{0em}{1.7pt}{1.7pt} %
		\midrule
		\specialrule{0em}{1.8pt}{1.8pt} %
		(\uppercase\expandafter{\romannumeral1}) & 
		$1$   & 14.2 & 13.3 & 1.19 \\
		\specialrule{0em}{1.8pt}{1.8pt} %
		\rowcolor{LightYellow}
		(\uppercase\expandafter{\romannumeral2}) & 
		$2$   & 25.0 & 11.8 & 1.12 \\
		\specialrule{0em}{1.8pt}{1.8pt} %
		\rowcolor{white}
		(\uppercase\expandafter{\romannumeral3}) & 
		$3$   & 37.5 & \gold{11.6} & \gold{1.10} \\
		\specialrule{0em}{1.8pt}{1.8pt} %
		\bottomrule
		\multicolumn{4}{c}{\quad\quad\textbf{(f)}} \\
	\end{tabular}
	
    \end{tabular}}
    \vspace{-0.3cm}
    \caption{\textbf{Ablation study (NYUv2 test set, $8\times$ factor).} We measure the impact of (a) explicit vs implicit HR features, (b) different kinds of HF supervision, (c) different sub-networks for explicit HF features extraction, (d) scales at which AFFM is applied, (e) modules building AFFM, (f) number of stages in GDRB. In yellow, configurations corresponding to our final model without NLSPN.}
    \label{tab:ablations}
\end{table}


\subsection{Ablation Study}
We now perform a series of ablation experiments to measure the impact of key components and parameters in \netname. Tab. \ref{tab:ablations} collects the outcome of these studies, conducted on NYUv2 test set with $8\times$ factor. Without loss of fairness, NLSPN is never used here -- to fully focus on the impact of single components. 

\textbf{(a) Implicit vs Explicit High-Frequency Features.}
To measure the impact of both implicit and explicit HR features, we compare the performance of the proposed network and its variants when extracting either only one of the two. The quantitative results are collected in Tab.~\ref{tab:ablations}(a). Without the help of gradient maps (I), the performance of the network significantly degrades. We believe this is caused by the difficulty in effectively extracting fine structures or salient edges required for LR depth maps from implicit HF features alone. Moreover, explicit features highlight regions in the image that need to be focused on, avoiding \netname{} to learn to localize them and easing its task. 


Nonetheless, explicit HF features alone as guidance (II) are insufficient as well. We argue that the explicit information might neglect some RGB features, whereas implicit HF feature extraction can recover them. Furthermore, to verify the effectiveness of LCF, we replace it with ResBlock~\cite{he2016deep} (III) to extract shallow features from RGB images, highlighting a negative impact on implicit features extraction -- i.e., it results less accurate than (II). 

\textbf{(b) Ablation on Explicit High-Frequency Features.}
We now investigate which kind of HF information is more effective for our framework. Purposely, we train HFEB with supervision coming from five different HF features used as ground truth edge maps $E_{gt}$. Tab.~\ref{tab:ablations}(b) collects results from this experiment, highlighting that Canny edges (I) and Gradient maps (V) lead to slightly better results. 


\textbf{(c) Impact of HFEB.}
To verify the effectiveness of HFEB, we replace it with EdgeNet~\cite{liu2021multi} -- based on the widely-used U-net structure -- and SCPA~\cite{zhao2020efficient}, which inspires our scaling strategy. As shown in Tab.~\ref{tab:ablations}(c), EdgeNet (I) achieves lower MSE and MAE than SCPA (II), yet needs more parameters -- 5.78M vs. 0.29M. HFEB (III) yields the same accuracy as EdgeNet, with fewer parameters than SCPA, thus being both more accurate and efficient. 



\textbf{(d -- e) Impact of AFFM.}
We now measure the effectiveness of AFFM. Tab.~\ref{tab:ablations}(d) shows results obtained by deploying AFFM at different scales, respectively the highest (I), the first two (II) and all of the three scales. We can notice how performing fusion at the highest scale alone results insufficient, whereas using multi-scale features for fusion yields improvements, despite saturating already when using two scales, with the lowest one not providing additional, meaningful details to be taken into account.

Furthermore, we ablate AFFM in its single components. Tab.~\ref{tab:ablations}(e) resumes the outcome of this evaluation. 
We first test the performance of \netname{} without AFFM (I), highlighting a large drop in accuracy. By adding dynamic fusion, yet without using attention (II) vastly improves the results already, while replacing the weighted sum in the upper of Fig.~\ref{affm} with concatenation and a ResBlock~\cite{he2016deep} (III) yields worse results compared to our full AFFM (IV). 

\textbf{(f) Impact of Stages Number.}
To conclude, we evaluate the impact of the multi-stage design.
As shown in Tab.~\ref{tab:ablations}(f), a single-stage architecture (I) is vastly outperformed by deploying two stages (II), yet at the expense of doubling the number of parameters. Furthermore, while the three-stage architecture (III) still yields some improvement, the benefit is minor in comparison to the significant increase in parameters. Hence, we choose two stages as the default configuration to balance accuracy and efficiency.


\begin{table}[t] \footnotesize
	\renewcommand\tabcolsep{1.5pt} 
	\centering
	\scalebox{0.8}{
	\begin{tabular}{@{}lcccccc@{}}
		\toprule
		 & PMBANet~\cite{ye2020pmbanet} & FDSR~\cite{he2021towards} & JIIF~\cite{tang2021joint} & DCTNet~\cite{zhao2022discrete} & LGR~\cite{de2022learning} & Ours \\ 
		 \midrule
		 Runtime (ms)
		 & 26.9 & 1.03 & 89.8 & 9.03 & 26.4 & 51.5\\
		 Memory Peak (GB)
		 & 3.07 & 2.05 & 2.36 & 0.26 & 0.19 & 18.6 \\ 
		\bottomrule
	\end{tabular}}
	\vspace{-0.3cm}
	\caption{\textbf{Computational requirements}. Experiments on Nvidia RTX 3090 GPU, with $256\times256$ input and $8\times$ factor.}
	\label{runtime_memory}
    
\end{table}

\subsection{Limitations}
We conclude by listing a few limitations of \netname. As previously pointed out, global context is crucial for it to achieve the best performance. When this is unavailable, some accuracy is lost when generalizing across datasets. Moreover, the significant improvements over existing methods are paid for in terms of time/memory requirements. Tab. \ref{runtime_memory} highlights the higher runtime and, more evidently, peak memory usage. Future work will aim at reducing the overhead, while minimizing the drop in accuracy.


\section{Conclusion}
This paper proposed \netname{}, a depth super-resolution network, which includes a high-frequency extraction branch (HFEB) and a guided depth restoration branch (GDRB). Specifically, implemented as an efficient transformer, HFEB extracts explicit HF features. Then, GDRB deploys a two-stage encoder-decoder network to recover HR depth maps progressively, by adaptively fusing discriminative features while supplementing additional, implicit HF information. Exhaustive experiments demonstrate that \netname{} sets a new state-of-the-art for guided depth super-resolution.

\bibliography{ref}
\bibliographystyle{iclr2023_conference}


\section{Model Details}
\label{model training}
The 5.0B Transformer-XL is pre-trained on 32 A100s with 40G memory for 45 days, the batch size is set to 32*8=256. After running 445k steps, the final validation loss reduces to about 2.4. The 2.7B OPT is incrementally trained on the basis of the open-source model.



\bgroup
\setlength{\tabcolsep}{1.3mm}
\begin{tabular}{lrrrrrcl}
    \toprule
    \cthead{Dataset}                                           & \cthead{\( n \)} & \cthead{\( v \)} & \cthead{\( k \)} & \cthead{\( n_\text{small} \)} & \cthead{\( n_\text{big} \)} & \cthead{Dim.\ }                    & \cthead{Licence} \\ \cmidrule(lr){1-1} \cmidrule(lr){2-8}
    NoisyMNIST~\cite{lecunGradientbasedLearningApplied1998}   & \( 70000 \)     & \( 2 \)         & \( 10 \)        & \( 6313 \)                   & \( 7877 \)                 & \( (28 \times 28)^{2} \)          & CC BY-SA 3.0 \\
    NoisyFashion~\cite{xiaoFashionMNISTNovelImage2017}        & \( 70000 \)     & \( 2 \)         & \( 10 \)        & \( 7000 \)                   & \( 7000 \)                 & \( (28 \times 28)^{2} \)          & MIT \\
    EdgeMNIST~\cite{lecunGradientbasedLearningApplied1998}    & \( 70000 \)     & \( 2 \)         & \( 10 \)        & \( 6313 \)                   & \( 7877 \)                 & \( (28 \times 28)^{2} \)          & CC BY-SA 3.0 \\
    EdgeFashion~\cite{xiaoFashionMNISTNovelImage2017}         & \( 70000 \)     & \( 2 \)         & \( 10 \)        & \( 7000 \)                   & \( 7000 \)                 & \( (28 \times 28)^{2} \)          & MIT \\
    COIL-20~\cite{neneColumbiaObjectImage1996}               & \( 480 \)       & \( 3 \)         & \( 20 \)        & \( 24 \)                     & \( 24 \)                   & \( (64 \times 64)^{3} \)          & None \\
    Caltech7~\cite{fei-feiLearningGenerativeVisual2007}       & \( 1474 \)      & \( 6 \)         & \( 7 \)         & \( 34 \)                     & \( 798 \)                  & \( 48, 40, 254, 1984, 512, 928 \) & CC BY 4.0 \\
    Caltech20~\cite{fei-feiLearningGenerativeVisual2007}      & \( 2386 \)      & \( 6 \)         & \( 20 \)        & \( 33 \)                     & \( 798 \)                  & \( 48, 40, 254, 1984, 512, 928 \) & CC BY 4.0 \\
    PatchedMNIST~\cite{lecunGradientbasedLearningApplied1998} & \( 21770 \)     & \( 12 \)        & \( 3 \)         & \( 6903 \)                   & \( 7877 \)                 & \( (28 \times 28)^{12} \)         & CC BY-SA 3.0 \\
    \bottomrule
\end{tabular}

\egroup


 
During the pre-training of the generator model, we utilize the memory-cache mechanism of Transformer-XL and design a special attention mask to concatenate the multiple input sentences into one sample, to reduce the number of the padding token in a batch and therefore increase the number of effective tokens. To make the generation more robust, we add noise to the original sentences by randomly replacing or discarding tokens with a 5\% probability. In addition, the prompts that we use for Chinese generation and English generation are as follows,

\begin{itemize}
\item Chinese prompt: \begin{CJK}{UTF8}{gbsn} “$s^a$”的相似句是“$s^b$” \end{CJK} (en: A similar sentence to ``$s^a$" is ``$s^b$".)
\item English prompt: ``$s^a$" is similar to ``$s^b$"
\end{itemize}

When training the discriminator, following the usage of special tokens in BERT \cite{DBLP:conf/naacl/DevlinCLT19}, we use $[SEP]$ to concatenate two sentences and take the embedding at the $[CLS]$ position to represent the whole sentence to predict the label. Moreover, we utilize the mask method in BERT to randomly mask 15\% of the input tokens.

\section{Dataset Details}
\label{dataset setting}
The statistics of the experimental datasets are reported in Table~\ref{dataset}.

Other Chinese datasets (LCQMC \cite{liu-etal-2018-lcqmc}, OPPO, PAWS-X-zh \cite{yang-etal-2019-paws}, BQ \cite{chen-etal-2018-bq}, CCKS, Chinese-STS-B \cite{wang-etal-2018-glue}) and English datasets (QQP \cite{wang-etal-2018-glue}, STS-B \cite{wang-etal-2018-glue}, PAWS-X-en \cite{yang-etal-2019-paws}) are collected and used as the corpus of similar sentence pairs for pre-training the generator. 

The QQP-ZH dataset contains 9000 pieces of data randomly selected and translated from the English QQP dataset, which is then divided into training set and test set in a ratio of 3:2.

\section{Parameter Details}
\label{parameters}
The training parameters of zero-shot are shown in Table~\ref{zero-shot-parm}. The three thresholds are used to select positive and negative examples for training the discriminator and positive examples for training the generator, respectively. We adopt cosine annealing learning rate decay strategy during training.

\begin{table}
\small
\centering
\setlength\tabcolsep{2pt} % 列间距
\caption{Parameter Settings of Zero-Shot.}
\label{zero-shot-parm}
\begin{tabular}{l|cccc}
\toprule
 & AFQMC & CHIP-STS & QQP-ZH & MRPC \\ \midrule
\begin{tabular}[l]{@{}r@{}}Max Threshold\\ \textit{-negative}\end{tabular} & 0.8 & 0.9 & 0.95 & 0.95 \\ 
\begin{tabular}[l]{@{}r@{}}Min Threshold\\ \textit{-negative}\end{tabular} & 0.6 & 0.7 & 0.8 & 0.8 \\ 
\begin{tabular}[l]{@{}r@{}}Max Threshold\\ \textit{-positive}\end{tabular} & 0.8 & 0.9 & 0.95 & 0.95 \\ 
\begin{tabular}[l]{@{}r@{}}Min Threshold\\ \textit{-positive}\end{tabular} & 0.6 & 0.7 & 0.8 & 0.8 \\ 
\begin{tabular}[l]{@{}r@{}}Max Threshold\\ \textit{-generator}\end{tabular} & 0.6 & 0.9 & 0.95 & 0.95 \\ 
\begin{tabular}[l]{@{}r@{}}Min Threshold\\ \textit{-generator}\end{tabular} & 0.6 & 0.7 & 0.8 & 0.8 \\ 
Threshold Increase & 0.07 & 0.1 & 0.05 & 0.05 \\
Sentence Num & 6000 & 6000 & 3000 & 4000 \\ 
Learning Rate & \multicolumn{4}{c}{2e-5} \\ 
Warm Up Steps & \multicolumn{4}{c}{40} \\ 
Early Stopping & \multicolumn{4}{c}{1} \\ 
\begin{tabular}[l]{@{}r@{}}$\mathcal{G}$ Batch Size\\ \textit{-training}\end{tabular} & \multicolumn{3}{c}{2(concat 30 samples)} & 24 \\ 
\begin{tabular}[l]{@{}r@{}}$\mathcal{G}$ Batch Size\\ \textit{-predicting}\end{tabular} & \multicolumn{3}{c}{512} & 100 \\ 
\begin{tabular}[l]{@{}r@{}}$\mathcal{D}$ Batch Size\\ \textit{-training}\end{tabular} & \multicolumn{3}{c}{64} & 32 \\ 
\begin{tabular}[l]{@{}r@{}}$\mathcal{D}$ Batch Size\\ \textit{-predicting}\end{tabular} & \multicolumn{3}{c}{384} & 96 \\ \bottomrule
\end{tabular}
\end{table}

The training parameters of fine-tuning are shown in Table~\ref{fine-tune-parm}.

\begin{table}
\small
\centering
\setlength\tabcolsep{2pt} % 列间距
\caption{Parameter Settings of Fine-Tune.}
\label{fine-tune-parm}
\begin{tabular}{l|cccc}
\toprule
 & AFQMC & CHIP-STS & QQP-ZH & MRPC \\\midrule
\begin{tabular}[l]{@{}r@{}}Max Threshold\\ \textit{-negative}\end{tabular} & 0.98 & 0.98 & 0.84 & 0.8 \\ 
\begin{tabular}[l]{@{}r@{}}Min Threshold\\ \textit{-negative}\end{tabular} & 0.9 & 0.7 & 0.6 & 0.6 \\ 
\begin{tabular}[l]{@{}r@{}}Max Threshold\\ \textit{-positive}\end{tabular} & 0.98 & 0.98 & 0.98 & 0.8 \\
\begin{tabular}[l]{@{}r@{}}Min Threshold\\ \textit{-positive}\end{tabular} & 0.9 & 0.7 & 0.9 & 0.6 \\ 
\begin{tabular}[l]{@{}r@{}}Max Threshold\\ \textit{-generator}\end{tabular} & 0.98 & 0.98 & 0.98 & 0.8 \\ 
\begin{tabular}[l]{@{}r@{}}Min Threshold\\ \textit{-generator}\end{tabular} & 0.9 & 0.7 & 0.9 & 0.6 \\ 
Threshold Increase & 0.07 & 0.07 & 0.07 & 0.2 \\
Sentence Num & 6000 & 6000 & 3000 & 3000 \\
Learning Rate & \multicolumn{4}{c}{5e-6} \\
Warm Up Steps & \multicolumn{4}{c}{40} \\
Early Stopping & \multicolumn{4}{c}{1} \\
\begin{tabular}[c]{@{}r@{}}$\mathcal{G}$ Batch Size\\\textit{ -training}\end{tabular} & \multicolumn{3}{c}{2(concat 30 samples)} & 24 \\
\begin{tabular}[l]{@{}r@{}}$\mathcal{G}$ Batch Size\\\textit{ -predicting}\end{tabular} & \multicolumn{3}{c}{512} & 100 \\
\begin{tabular}[l]{@{}r@{}}$\mathcal{D}$ Batch Size\\ \textit{-training}\end{tabular} & 32 & 64 & 32 & 32 \\ 
\begin{tabular}[l]{@{}r@{}}$\mathcal{D}$ Batch Size\\ \textit{-predicting}\end{tabular} & 256 & 384 & 256 & 96 \\ \bottomrule
\end{tabular}
\end{table}

\section{The Figure of Different Threshold Functions}
\label{thre_figure}

\begin{figure*}[ht] %htbp
\begin{center}
\includegraphics[width=1.0\textwidth]{Figure/zero_shot.pdf}
\end{center}
\caption{Results of contrast experiments on Cosine(green), Constant(orange), Quadratic(blue) and Linear(red) function in Zero-Shot Setting.} 
\label{filter_fig}
\end{figure*}


\begin{table*}[htbp]
\caption{Generated Samples in English}
\label{samples_en}
\centering
\begin{tabular}{c}
\hline
MRPC(News) \\ \hline
\begin{tabular}[c]{@{}c@{}}\textit{origin:} Hong Kong was flat, Australia, Singapore and South Korea lost 0.2-0.4 percent.\\ \textit{first round:} the market is flat, the US is flat, Europe is flat, Japan is flat, \\ ... Malaysia is flat, Philippines is flat, Vietnam is flat, Singapore is\\ \textit{last round:} Hong Kong was unchanged, Australia, Singapore and South Korea \\ were down 0.2-0.4 percent.\end{tabular} \\ \hline
\begin{tabular}[c]{@{}c@{}}\textit{origin:} Ballmer has been vocal in the past warning that Linux is a threat to Microsoft.\\ \textit{first round:} Microsoft is a threat to Linux.\\ \textit{last round:} Ballmer has been vocal in the past about the threat Linux poses to Microsoft.\end{tabular} \\ \hline
\begin{tabular}[c]{@{}c@{}}\textit{origin:} Last year, Congress passed similar, though less expensive, buyout\\ legislation for peanut farmers, ending that Depression-era program.\\ \textit{first round:} The government is going to buy out your farm and give you a check\\ for \$1,000,000.00. You can't refuse.\\ \textit{last round:} Congress passed a similar bill last year to end a Depression-era\\ program for peanut farmers, ending that program.\end{tabular} \\ \hline
\begin{tabular}[c]{@{}c@{}}\textit{origin:} Licensing revenue slid 21 percent, however, to \$107.6 million.\\ \textit{first round:} The company's net income fell to \$1.1 million, or 2 cents per share,\\ in the quarter ended Sept. 30, from \$1.3 million, or 3 cents per share, a year earlier.\\ \textit{last round:} Licensing revenue fell 21 percent to \$107.6 million \\ in the third quarter.\end{tabular} \\ \hline
\begin{tabular}[c]{@{}c@{}}\textit{origin:} SCO says the pricing terms for a license will not \\ be announced for weeks.\\ \textit{first round:} we're not going to tell you how much it costs to buy a car, but\\ we're going to tell you how much it costs to rent one.\\ \textit{last round:} The pricing terms for a license for the SCO software will not \\ be announced for several weeks.\end{tabular} \\ \hline
\end{tabular}
\end{table*}


Figure~\ref{filter_fig} dipicts the comparison results in each round. The linear function (red line) is significantly better than the other functions on both CHIP-STS and MRPC datasets. In the AFQMC and QQP-ZH datasets, the quadratic function (blue line) is slightly more effective than the linear function. In general, we can intuitively see that all functions show a significant increase relative to the starting point.

\section{The Table of KL Divergence}
Table~\ref{consistent} shows the values of KL divergence in the first and last rounds. Numerically, it is more evident that the distances are significantly reduced on the four datasets. 

\begin{table}
\small
\centering
\setlength\tabcolsep{8pt} % 列间距
\caption{The KL divergence in the First and Last rounds of Zero-Shot.}
\label{consistent}
\sc
\begin{tabular}{l|cc}
\toprule
 & \begin{tabular}[l]{@{}r@{}}KL Divergence\\ \textit{-first round}\end{tabular} & \begin{tabular}[l]{@{}r@{}}KL Divergence\\ \textit{-last round}\end{tabular} \\ \midrule
AFQMC & 0.29 & 0.26 \\
CHIP-STS & 0.16 & 0.08 \\
QQP-ZH & 0.18 & 0.06 \\
MRPC &0.22 & 0.02 \\ \bottomrule
\end{tabular}
\end{table}

\section{Generated Samples}
\label{generate samples}
We use nucleus sampling\cite{Holtzman2020The} (with $p = 0.9$) to generate similar sentences.
Generated examples in English are shown in Table~\ref{samples_en} and in Chinese in Table~\ref{samples_zh}.





\begin{table*}[htbp]
\caption{Generated Samples in Chinese.}
\label{samples_zh}
\small
\centering
\begin{CJK}{UTF8}{gbsn}
\begin{tabular}{lcc}
\toprule
\multirow{15}{*}{\begin{tabular}[c]{@{}c@{}}AFQMC\\ (Financial)\end{tabular}} & \begin{tabular}[c]{@{}c@{}}\textit{origin:}可以用自己的支付宝为别人的花呗还款吗\\(Can I use my Alipay to repay someone else's Huabei)\\ \textit{first round:}花呗的钱可以让其他人代还吗\\(Can someone else pay for the money from Huabei)\\ \textit{last round:}花呗能不能帮别人还\\(Can my Huabei help others repay?)\end{tabular} \\ \cmidrule{2-2} 
 & \begin{tabular}[c]{@{}c@{}}\textit{origin:}花呗怎么还欠费\\(How does Huabei pay off the arrears)\\ \textit{first round:}花呗怎么还有电费\\(Why does Huabei still have electricity bills)\\ \textit{last round:}花呗还款怎么还\\(How to pay off Huabei)\end{tabular} \\ \cmidrule{2-2} 
 & \begin{tabular}[c]{@{}c@{}}\textit{origin:}我的花呗收款二维码\\(QR code for my Huabei payment)\\ \textit{first round:}商家花呗的二维码\\(Merchant Huabei's QR code)\\ \textit{last round:}花呗收款二维码在哪里\\(Where is the QR code for Huabei payment)\end{tabular} \\ \midrule
\multirow{15}{*}{\begin{tabular}[c]{@{}c@{}}CHIP-STS\\ (Medical)\end{tabular}} & \begin{tabular}[c]{@{}c@{}}\textit{origin:}艾滋病的病因是什么\\(What is the cause of AIDS)\\ \textit{first round:}艾滋病毒是什么?\\(What is HIV?)\\ \textit{last round:}艾滋病是什么原因引起的?\\(What causes AIDS?)\end{tabular} \\ \cmidrule{2-2} 
 & \begin{tabular}[c]{@{}c@{}}\textit{origin:}高血压总是流口水是怎么回事\\(High blood pressure is always drooling what is going on)\\ \textit{first round:}高血压怎么回事\\(What about high blood pressure)\\ \textit{last round:}高血压为什么会流口水?\\(Why does high blood pressure cause drooling?)\end{tabular} \\ \cmidrule{2-2} 
 & \begin{tabular}[c]{@{}c@{}}\textit{origin:}得了糖尿病,现在越来越瘦了怎么回事\\(Why am I getting thinner and thinner now that I have diabetes)\\ \textit{first round:}糖尿病现在怎么回事?\\(What's going on with diabetes now?)\\ \textit{last round:}糖尿病患者为什么会瘦?\\(Why do people with diabetes lose weight?)\end{tabular} \\ \midrule
\multirow{15}{*}{\begin{tabular}[c]{@{}c@{}}QQP-ZH\\ (Common)\end{tabular}} & \begin{tabular}[c]{@{}c@{}}\textit{origin:}如何从此网站删除我的帐户?\\(How do I delete my account from this site?)\\ \textit{first round:}怎么删除网站\\(How to delete a website)\\ \textit{last round:}如何才能删除我的帐户?\\(How can I delete my account?)\end{tabular} \\ \cmidrule{2-2} 
 & \begin{tabular}[c]{@{}c@{}}\textit{origin:}关于电子产品的一些好书是什么?\\(What are some good books on electronics?)\\ \textit{first round:}有什么好的电子产品推荐\\(Any good electronics recommendations)\\ \textit{last round:}有哪些关于电子产品的好书?\\(What are some good books about electronics?)\end{tabular} \\ \cmidrule{2-2} 
 & \begin{tabular}[c]{@{}c@{}}\textit{origin:}为什么没有人看到无尽和无限之间的区别?\\(Why does no one see the difference between endless and infinite?)\\ \textit{first round:}为什么宇宙中没有极限的存在\\(Why is there no limit in the universe)\\ \textit{last round:}为什么没有人知道无限和有限之间的区别?\\(Why does no one know the difference between infinite and finite?)\end{tabular} \\ \bottomrule
\end{tabular}
\end{CJK}
\end{table*}




\end{document}
