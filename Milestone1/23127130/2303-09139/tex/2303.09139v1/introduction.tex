\section{Introduction}
In recent years, autonomous vehicles have been deployed in complex, city-scale scenarios to accomplish various tasks such as food delivery, warehouse administration, and public transportation. These vehicles routinely travel on highly regulated paths, such as highways, crossroads, and sidewalks, or in spaces with large open areas including shopping malls, school libraries, etc. Most prior works \cite{6225166,bareiss2015generalized,9196799,8715479} build navigation algorithms on one of these assumptions. In reality, however, autonomous vehicles must also be prepared for unexpected and unregulated scenarios or spaces with narrow passages. Dealing with narrow spaces is inevitable when two food delivery robot meets in the aisle of a hotel or an autonomous truck travels downtown to reach a warehouse. Narrow passages are notoriously difficult to handle, even when navigating a single robot \cite{szkandera2020narrow}, and scaling to multiple agents is still an open problem.

\begin{figure}[h]
\centering
\scalebox{.8}{\includegraphics[width=1\linewidth]{figs/illus.pdf}}
\caption{\label{fig:illus} \small{Two agents travel in an environment with one large and one narrow space. (a): Top picture: if the two agents travel at the same speed, their meeting point (defined as the POI and illustrated as the red dot) will be in the narrow space where local navigation techniques can fail. (b)Bottom picture: our method shifts the POI to the large space so that local navigation can successfully generate collision-free trajectories by yielding.}}
\vspace*{-10px}
\end{figure}
Prior methods for navigating multiple agents are classified into decentralized local techniques and centralized global techniques, each having its pros and cons. Local navigation methods \cite{6225166,bareiss2015generalized} assume agents move towards their goal positions along some local directions without communicating with each other. When obstacles or other agents get in the way, heuristic behaviors, such as yielding \cite{van2011reciprocal}, grouping\cite{6630970}, and following \cite{10.5555/2982818.2982838,7487147} are used to avoid collisions. However, local techniques can fail in the face of narrow passages where agents form deadlock configurations as illustrated in \prettyref{fig:illus} (a). On the other hand, global navigation methods \cite{Berg-RSS-09,yu2018effective,he2022multi} coordinate agent motions in a central node to avoid collisions. Although these methods can handle many agents in complex environments with narrow passages, they rely on strong assumptions such as the environment being grid-like, agents moving on discrete graph-like structures, or the agents being holonomic. However, actions such as constructing such discrete structures or generalizing to nonholonomic agents are non-trivial and cannot be used in time-critical applications due to a high computational cost.

\TE{Main Result:} We propose an improved decentralized algorithm for nonholonomic multi-agent navigation, which incorporates ideas from centralized techniques to alleviate the deadlock problem. We observe that yielding behaviors used by prior local navigation approaches \cite{6225166,bareiss2015generalized} can have high success rates in large open areas while being less successful in narrow spaces as illustrated in \prettyref{fig:illus} (b). As a result, we propose shifting the yielding areas to large open spaces of the environment to increase the success rate. Specifically, our algorithm relies on the construction of a medial axis for the free space. By mapping agent positions and their trajectories to the medial axis, we can estimate their Positions-Of-Impacts (POIs), which are positions where agents get close enough for local navigation techniques to generate yielding behaviors. We then estimate the surrounding space required by such yielding behaviors. If the space around a POI is not large enough for the yielding to be successful, we search for nearby large spaces and re-plan agent trajectories to move the POI. We show that such re-planning can be accomplished at a relatively low-cost without communication with other agents, preserving the decentralized nature of our method.

We evaluate our method in 4 challenging scenarios with 5-15 robots. The results show that our method exhibits real-time performance, taking up to 20 ms and 43 ms on average to plan the POIs. Compared with local navigation alone, our method achieves up to a $100\%$ higher success rate in some scenarios.