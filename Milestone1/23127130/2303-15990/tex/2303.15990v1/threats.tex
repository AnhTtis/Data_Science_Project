%%%%%%%%%%%%%%%%%%%%%%%%%%%%%%%%%%%%%%%%%%%
%%%%%%%%%%%%%%%%%%%%%%%%%%%%%%%%%%%%%%%%%%%
\section{Threats to Validity}
%%%%%%%%%%%%%%%%%%%%%%%%%%%%%%%%%%%%%%%%%%%
%%%%%%%%%%%%%%%%%%%%%%%%%%%%%%%%%%%%%%%%%%%

\textit{Threats to Construct Validity} concern the correct operationalization of the concepts being studied. First, the inference method we used for extracting requirements from Dockerfiles works on a series of assumptions (\eg the presence of comments) that might not always be completely satisfied in practice. To mitigate this crucial threat, we carefully tested our parser and manually checked examples until we were satisfied with the procedure used to achieve this goal. In total, we were able to infer 113,442 unique \nlRecipes, with 31,990 unique combination of dependencies, which gives us confidence on the fact that our parser works as intended for most of the Dockerfiles considered. Also, we made sure to exploit all the instances (also the ones without comments) in the training procedure (\ie by using them for pre-training).
The choice of the fields that compose our \nlRecipe could exclude requirements that developers might be interested in specifying (such as \texttt{USER}). As we explained in \secref{sec:approach}, we exclude only the requirements derived from Dockerfile instructions that do not appear very often.
The best candidate Dockerfile we selected for each \nlRecipe (\secref{sec:approach}) might not be representative of the group of Dockerfiles. In our methodology, we relied on Jaccard similarity to mitigate this risk.

\textit{Threats to Internal Validity} concern factors internal to our study that might have influenced our findings.
The percentage of marching layers (\RQ{3}) might not accurately capture the structural similarity between two Docker images: If a layer is different, many or even all the subsequent layers will be different.
However, to the best of our knowledge, the only alternative is to perform a diff on Docker containers \cite{web:containerdiff}, which, however, only works at the level of installed packages.
Considering the layers allows knowing from which point the two images started to differ, we believe this is the best choice.

\textit{Threats to External Validity} concern the generalizability of our findings.
We relied on the largest collection of Dockerfiles available in the literature \cite{eng2021revisiting}. More Dockerfiles might have been created between 2020 and now. However, we believe that the considered dataset allows us to provide reasonably generalizable.
