\documentclass[10pt,twocolumn,letterpaper]{article}

\usepackage{iccv}
\usepackage{times}
\usepackage{epsfig}
\usepackage{graphicx}
\usepackage{amsmath}
\usepackage{amssymb}
\usepackage{booktabs, array}
\usepackage{multirow}
\usepackage{comment}
\usepackage{color, colortbl}
\usepackage{enumitem}
\usepackage{bbding}
\usepackage{pifont}
\usepackage{caption}
\usepackage{subcaption}\usepackage{wasysym}
\newcommand{\tabincell}[2]{\begin{tabular}{@{}#1@{}}#2\end{tabular}}
\usepackage{bbm}

\setcounter{figure}{1}
\definecolor{LightYellow}{rgb}{1,1,0.7}


\definecolor{gold}{rgb}{1.0, 0.874, 0}
\definecolor{silver}{rgb}{0.77,0.77,0.77}
\definecolor{brown}{rgb}{0.95, 0.678, 0.4}

\newcommand{\gold}[1]{\colorbox{gold}{\textbf{#1}}}
\newcommand{\silver}[1]{\colorbox{silver}{\textbf{#1}}}
\newcommand{\bronze}[1]{\colorbox{brown}{\textbf{#1}}}



\usepackage[pagebackref=true,breaklinks=true,letterpaper=true,colorlinks,bookmarks=false]{hyperref}

\iccvfinalcopy 

\def\httilde{\mbox{\tt\raisebox{-.5ex}{\symbol{126}}}}

\newcommand{\netname}{DSR-EI}

\begin{document}

\title{Depth Super-Resolution from Explicit and Implicit High-Frequency Features}

\author{Xin Qiao$^{1}$ \hspace{0.5cm} Chenyang Ge$^{1}$ \hspace{0.5cm} Youmin Zhang$^{2}$ \hspace{0.5cm}
Yanhui Zhou$^{1}$ \\  Fabio Tosi$^{2}$ \hspace{0.5cm} Matteo Poggi$^{2}$ \hspace{0.5cm} Stefano Mattoccia$^{2}$ \\ \vspace{-0.3cm} \\
$^{1}$ Xi'an Jiaotong University \hspace{1cm} $^{2}$ University of Bologna \\
}

\twocolumn[{
\renewcommand\twocolumn[1][]{#1}
\maketitle
\begin{center}
    \vspace{-0.5cm}
    \includegraphics[width=0.95\textwidth]{figs/teaser/teaser.pdf} 
    \label{fig:teaser}
\end{center}
\vspace{-0.3cm}
\small \hypertarget{fig:teaser}{Figure 1.} \textbf{Depth Super-Resolution exploiting explicit and implicit high-frequency features.} On the left, an overview of our framework, combining the power of both explicit and implicit high-frequency information extracted from the inputs. On the right, qualitative examples with (a) RGB images, (b) ground truth depth and error maps by existing methods (c -- d) and ours (e).
\vspace{0.5cm}
}]
%%%%%%%%%%%%%%%%%%%%%%%%%%%%%%%%%%%%%%%  figure end
\maketitle

%%%%%%%%% ABSTRACT
\begin{abstract}

We propose a novel multi-stage depth super-resolution network, which progressively reconstructs high-resolution depth maps from explicit and implicit high-frequency features. The former are extracted by an efficient transformer processing both local and global contexts, while the latter are obtained by projecting color images into the frequency domain. Both are combined together with depth features by means of a fusion strategy within a multi-stage and multi-scale framework. Experiments on the main benchmarks, such as NYUv2, Middlebury, DIML and RGBDD, show that our approach outperforms existing methods by a large margin ($\sim20\%$ on NYUv2 and DIML against the contemporary work DADA, with $16\times$ upsampling), establishing a new state-of-the-art in the guided depth super-resolution task.

\end{abstract}


%%%%%%%%% BODY TEXT
The advance of Pre-trained Language Models (PLMs) like GPT-3 \cite{brown2020language} and LLaMA \cite{DBLP:journals/corr/abs-2302-13971} has substantially improved the performance of deep neural networks across a variety of Natural Language Processing (NLP) tasks. Various language models, based on the Transformer \cite{vaswani2017attention} architecture,  have been proposed, leading to state-of-the-art (SOTA) performance on the fundamental discrimination tasks. These models are first trained with self-supervised training objectives (e.g., predicting masked tokens according to surrounding tokens) on massive unlabeled text data, then fine-tuned on annotated data to adapt to downstream tasks of interest.  However, annotated data is usually limited to a wide range of downstream tasks, which results in overfitting and a lack of generalization to unseen data.

One straightforward way to deal with this data scarcity problem is data augmentation , and incorporating generative models to perform data augmentation has been widely adopted recently . Despite its popularity, the generated text can easily deviate from the real data distribution without exploiting any of the signals passed back from the discrimination task. In previous studies, generative data augmentation and discrimination have been well studied as separate problems, but it is less clear how these two can be leveraged in one framework and how their performances can be improved simultaneously. \looseness=-1

Generative Adversarial Networks (GANs) \cite{https://doi.org/10.48550/arxiv.1406.2661} are good attempts to couple generative and discriminative models in an adversarial manner, where a two-player minimax game between learners is carefully crafted. GANs have achieved tremendous success in domains such as image generation , and related studies have also shown their effectiveness in semi-supervised learning. However,  in the text field, GANs are difficult to train, most training objectives work well for only one model, either the discriminator or the generator, so rarely both learners can be optimal at the same time. This essentially arises from the adversarial nature of GANs, that during the process, optimizing one learner can easily destroy the learning ability of the other, making GANs fail to converge.

Another limitation of simultaneously optimizing the generator and the discriminator comes from the discrete nature of text in NLP, as no gradient propagation can be done from discriminators to generators. One theoretically sound attempt is to use reinforcement learning (RL), but the sparsity and the high variance of the rewards in NLP make the training particularly unstable \cite{caccia2019language}. 

To address these shortcomings, we novelly introduce a self-consistent learning framework based on one generator and one discriminator: the generator and the discriminator are alternately trained by way of cooperation instead of competition, and the selected samples are used as the medium to pass the feedback signal from the discriminator. Specifically, in each round of training, the samples generated by the generator are synthetically labeled by the discriminator, and then only part of them would be selected based on dynamic thresholds and used for the training of the discriminator and the generator in the next round. Several benefits can be discovered from this cooperative training process. First, a closed-loop form of cooperation can be established so that we can get the optimal generator and discriminator at the same time. Second, this framework helps improve the generation quality while ensuring the domain specificity of generator, which in turn contributes to training. Third, a steady stream of diverse synthetic samples can be added to the training in each round and lead to continuous improvement of the performance of all learners. Finally, we can start the training with only domain-related corpus and obtain strong results, while these data can be easily sampled with little cost or supervision. Also, the performance on labeled datasets can be further boosted based on the strong baselines. As an example to demonstrate the effectiveness of our framework in the text field, we examine it on four downstream text generation benchmarks, including AFQMC, CHIP-STS, QQP, and MRPC. The experiments show that our method significantly improves over standalone state-of-the-art discriminative models on zero-shot and full-data settings.

Our contributions are summarized as follows,

$\bullet$ We propose a self-consistent learning framework in the text field that incorporates the generator and the discriminator, in which both achieve remarkable performance gains simultaneously.

$\bullet$ We propose a dynamic selection mechanism such that cooperation between the generator and the discriminator drives the convergence to reach their scoring consensus.

$\bullet$ Experimental results show that the generator in our framework can continuously adjust its generation samples based on the performance of downstream tasks, while the discriminator can outperform the strong baselines.


% \documentclass[main.tex]{subfiles}
\label{sec:relatedwork}
% \begin{document}

\paragraph{Transient faults in asynchronous circuits}
Several studies have explored the effects of transient faults on asynchronous circuits. Detection and mitigation techniques with some form of redundancy have been proposed alongside.

The authors in~\cite{lafrieda2004fault} perform a thorough analysis of single-event transient (SET) effects, among other types of faults, in QDI circuits. The fault's impact is first presented at the gate level, then on communication channels, translating the fault to a deadlock. They also discuss other possible errors (synchronization failure, token generation, and token consumption).
An efficient failure detection method for QDI circuits is presented in~\cite{peng2005efficient}. The method brings the circuit to a fail-safe state in the presence of hard and soft errors. The authors investigate the probability for a glitch to propagate through a state-holding element in asynchronous circuits.
In~\cite{monnet2007formal}, the authors propose a formal method to model the behavior of QDI circuits in the presence of transient faults. They use symbolic simulation to provide an exhaustive list of possible effects and analyze which of these cases are theoretically reachable. Their model, however, does not support delay parameters, which potentially reduces the set of reachable states, further improving the resistance of a design against single-event upsets (SEUs).
They also discuss in~\cite{monnet2005asynchronous} the Muller C-element fault sensitivity and specify a global sensitivity criterion to SETs for asynchronous circuits. The work provides a behavioral analysis of QDI circuits in the presence of faults.
With the help of signal transition graphs (STGs), the authors in~\cite{bainbridge2009glitch} informally analyze SEUs due to glitches on QDI network-on-chip links. Several mitigation techniques are proposed with a focus on reducing the latch's sensitive window to a glitch.
Some of these techniques are tested and compared against other proposed variations in \cite{huemer2020QDIwindows}, \cite{behal2021towards}, and \cite{tabassam2022set}. The assessment there is based on extensive fault injection simulations into different QDI buffer styles, in order to identify the main culprits of the circuit. The authors provide a quantitative analysis to determine the windows of vulnerability to SETs and the impact of certain parameter choices on the resilience of the circuit. However, the analysis is done based on a regular timing grid, which causes linear complexity in time and in resolution, and cannot exclude the potential of overlooking relevant windows between the grid points.

\paragraph{Hazards in PRSs}
QDI circuits can be modeled on different levels. The Production Rule Set (PRS), introduced by Martin~\cite{martin1986compiling}, is a widely-used low-level representation that can be directly translated to a CMOS transistor implementation. PRSs do not normally support hazards, and by guaranteeing \emph{stability} and \emph{non-interference} characteristics~\cite{jang2005seu}, a PRS execution is assumed to be hazard-free. The authors consider an SEU as flipping of a variable's value and model it in so called transition graphs to identify deadlock or abnormal behavior.
\cite{katelman2012rewriting} extends the semantics of PRSs in order to be able to address hazards as circuit failures, but it is limited to checking the hazard-freedom property of a circuit. 

These papers are focused on the \emph{possibility} of failure and are restricted to precedence of events, without explicitly considering timing.
Our work enables further propagation of what we define as a glitch in order to check whether it has reached the final outputs of a circuit and, based on actual timing information, \emph{quantify} this proportion of failure.

% \end{document}
\section{\netname{} Framework}

In GDSR, HF information in color images -- complementary to depth maps -- is essential for achieving high performance, which motivates us to seek an efficient method to extract it. In this section, we present our framework that exploits explicit and implicit HF information for depth super-resolution. Then, we introduce the two branches in our network: the High-Frequency Extraction Branch (HFEB) and the Guided Depth Restoration Branch (GDRB). 

\begin{figure}	
	\centering	
	\captionsetup[subfigure]{font=footnotesize,textfont=footnotesize}
	\subfloat{	
		\centering	
		\includegraphics[width=3.2in]{./figs/hf_compare/hf_compare.pdf}}
	\vspace{-0.cm}
	\caption{\textbf{High-frequency information loss (factor $4\times$).} From left to right, HR depth map and its corresponding gradient map, followed by the gradient map from bicubic upsampled LR depth map and LR depth map itself. HF information is mostly lost in the second gradient map.}
	\label{hf_compare}
\end{figure}


Fig.~\ref{framework} shows an overview of our architecture. Given the LR depth map $D_{lr} \in \mathbb{R}^{h\times w\times 1}$ and the corresponding HR color image $I_{hr} \in \mathbb{R}^{H\times W\times 3}$, we aim at restoring HR depth map $\tilde{D}_{sr}$. Note that $H=s\times h$ and $W=s\times w$, where $s$ denotes the upsampling factor -- e.g., $4\times, 8\times$ or even $16\times$. In our proposed network, the input depth map is firstly upsampled with bicubic interpolation to the same size as $I_{hr}$. At different scales, we denote the corresponding depth maps and color images as $D_{lr}^{i}$ and $I_{hr}^{i}$, respectively, with $s=2^i$. Then, according to the above notation, the input images $D_{lr}^{0}$ and $I_{hr}^{0}$ are fed into the two branches, respectively.  Before being sent to GDRB, both the RGB and depth images are processed by a channel-attention block (CAB) \citep{zhang2018image} and a low-cut filtering (LCF) module, which will be explained in detail in Sec. \ref{GDRB}.


\subsection{High-Frequency Extraction Branch (HFEB)}
We argue HF information is crucial for effective super-solving depth and is often lost by upsampling. The primary goal of HFEB is to produce an accurate gradient map from an LR depth map, with the support of a color image jointly processed with it. 

Indeed, as pointed out in \cite{wang2020depth}, networks for GDSR tend to focus more on depth discontinuities or object boundaries. However, from Fig.~\ref{hf_compare}, we can notice that even with a $4\times$ factor, most high-frequency information vanishes, as shown by the gradient maps extracted from HR and upsampled LR depth maps, leading to severe degradation of the super-solved depth map. Traditional methods tend to transfer texture to depth maps rather than structural details, failing to extract accurate edges. Moreover, methods extracting binary edges \citep{wang2020depth} gather insufficient high-frequency information, yielding sub-optimal results. 

\begin{figure}	
	\centering	
	\captionsetup[subfigure]{font=footnotesize,textfont=footnotesize}
	\subfloat{	
		\centering	
		\includegraphics[width=3.2in]{./figs/framework/DSP.pdf}}	
    \vspace{-0.cm}		
	\caption{\textbf{DSP architecture.} Differently from SCPA \citep{zhao2020efficient}, our module processes features at different scales, allowing to extract explicit HF information more effectively.}	
	\label{dsp} 
\end{figure}

The work~\citep{pu2022edter} has shown that transformer-based networks can extract clear and meaningful edges by leveraging both global and local features simultaneously. Considering the sparsity of edge maps, we design an efficient transformer, inspired by dynamic scale policy \citep{wang2019elastic} and self-attention \citep{vaswani2017attention}, to obtain strong HF priors for guiding depth super-resolution. Specifically, our transformer consists of a stack of blocks called dynamic self-calibrated convolution with pixel attention (DSP) and one LightViT block~\citep{huang2022lightvit}. 
To better extract HF features, we design the DSP block, which is inspired by SCPA \citep{zhao2020efficient} and performs self-calibrated convolution with two branches at a single scale. However, unlike SCPA, our DSP block includes an additional branch that enables the processing of features at different scales without incurring extra computational burden, as we will demonstrate empirically in our experiments. Specifically, stacked DSP blocks can be expressed as:
\begin{equation}
    \label{eq_dsp1}
    \Phi_{M}=\mathcal{F}_{DSP}^{M}(\mathcal{F}_{DSP}^{M-1}(\cdot\cdot\cdot\mathcal{F}_{DSP}^{1}(\Phi_0)\cdot\cdot\cdot))
\end{equation}
where $\mathcal{F}_{DSP}^{m}$ denotes the mapping of the $m$-th DSP block, $m\in[1, M]$, $\Phi_0$ and $\Phi_M$ are the input/output features, respectively. As shown in Fig.~\ref{dsp}, each DSP block includes three branches: the upper is the dynamic scale branch, the middle is the flat convolution branch, and the lower is the pixel attention branch. Specifically, we employ three convolutions with $1\times1$ kernel to split the channels, which are further processed by each branch. Note that the dynamic scale branch needs to be downsampled before $1\times1$ convolution. 
{Given the input $\Phi_{m-1}$, we obtain:
\begin{gather}
    \Phi_{m-1}^{1}=Conv_{1\times1}((\Phi_{m-1})\downarrow)  \label{eq_dsp2_1}
    \\
    \Phi_{m-1}^{k}=Conv_{1\times1}(\Phi_{m-1})  \label{eq_dsp2_2}
\end{gather}
where $\Phi_{m-1}^{1}$ is the output from the upper dynamic scale branch, $k=2,3$ denotes the features of the other two branches, $Conv_{1\times1}$ is $1\times1$ convolution, and $\downarrow$ is the downsampling operation.} 
Except for the pixel attention branch, which has features with half the total channels, the other two branches process features with $\frac{1}{4}$ of the channels each. Next, the pixel attention branch obtains features through the pixel attention scheme~\citep{zhao2020efficient}. In contrast, the other two branches extract spatial information with a $3\times3$ flat convolution, followed by a $1\times1$ convolution to restore the number of channels to be the same as the pixel attention branch. Note that the dynamic scale branch needs upsampling after $1\times1$ convolution. Then, the features from the dynamic scale and the flat convolution branches can be fused by summation. After concatenation of the features followed by a $1\times1$ convolution, the DSP finally generates the output features $\Phi_{m}$ in a residual learning fashion. {It can be written as follows:
\begin{gather}
    \Phi_{m}^{1}=Conv_{1\times1}(Conv_{3\times3}(\Phi_{m-1}^{1}))\uparrow     \label{eq_dsp3_1}
    \\
    \Phi_{m}^{2}=Conv_{1\times1}(Conv_{3\times3}(\Phi_{m-1}^{2}))               \label{eq_dsp3_2}
    \\ 
    \Phi_{m}^{3}=Conv_{3\times3}(\Phi_{m-1}^{3}) \odot \sigma(Conv_{1\times1}(\Phi_{m-1}^{3}))
    \label{eq_dsp3_3}
    \\
    \Phi_{m}^{'}=Conv_{3\times3}(\Phi_{m}^{3})
    \label{eq_dsp3_5}
    \\
    \Phi_{m}^{''}=Conv_{3\times3}(\Phi_{m}^{1}\oplus\Phi_{m}^{2})  \label{eq_dsp3_4}
\end{gather}
where $\sigma$ is the sigmoid function, $\odot$ and $\oplus$ are element-wise multiplication and element-wise summation, respectively, and $\uparrow$ denotes the upsampling operation. After concatenation of the features $\Phi_{m}^{'}$ and $\Phi_{m}^{''}$ followed by a $1\times1$ convolution, the DSP finally generates the output features $\Phi_{m}$ in a residual learning manner. This process can be expressed as follows:
\begin{equation}
    \label{eq_dsp1_4}
    \Phi_{m}=Conv_{1\times1}([\Phi_{m}^{'}, \Phi_{m}^{''}]) \oplus \Phi_{m-1}
\end{equation}
where $[\cdot]$ perform concatenation.}

To further enhance the feature representation of the subnetwork, we incorporate LightViT \citep{huang2022lightvit} as the tail module, which utilizes local-global attention broadcast to aggregate information from all tokens, allowing for the efficient integration of global dependencies of local tokens into each image token. Finally, considering that the supervised attention module (SAM) \citep{zamir2021multi} can restore information progressively between stages/branches, we employ it to output the gradient map $E\in \mathbb{R}^{H\times W\times 1}$ and high-frequency features $F_{edge}\in \mathbb{R}^{H\times W\times C}$, used respectively as intermediate output -- allowing for explicit supervision over edges -- and as guidance for GDRB. Under this lightweight design, HFEB can effectively still extract meaningful structural information with different scale receptive fields.


\subsection{Guided Depth Restoration Branch (GDRB)}
\label{GDRB}
As shown in Fig.~\ref{framework}, GDRB is composed of two stages, and each one processes features at three scales, following a coarse-to-fine strategy \citep{gao2019dynamic,sarlin2019coarse}. The two stages are implemented with standard U-net architectures~\citep{ronneberger2015u}. More specifically, a cross-stage feature fusion module~\citep{zamir2021multi} is deployed between the two, which proved to be effective in image restoration and, in our design, allows GDRB to benefit from the intermediate features extracted by HFEB. To prevent aliasing in downsampling, we employ content-aware filtering layers~\citep{zou2022delving} in the encoders. Besides, GDRB deploys some further SAM blocks \citep{zamir2021multi}, allowing valuable features to propagate to the next stage. In addition to depth features, the SAMs of the two stages also output depth maps $\tilde{D}_{sr}^{'}$ and $\tilde{D}_{sr}^{''}$, to which intermediate supervision is provided. Note that input images are downsampled to the lower stage using pixel unshuffling to prevent information loss. Subsequently, the depth map output of this stage is restored at high resolution by employing pixel shuffling.

Based on the above structure, we propose two novel modules: AFFM and LCF. The former fuses gradient features between each encoder/decoder, while the latter supplements additional HF information in an implicit manner. 

\begin{figure}	
	\centering	
	\captionsetup[subfigure]{font=footnotesize,textfont=footnotesize}
	\subfloat{	
		\centering	
		\includegraphics[width=2.9in]{./figs/framework/AFFM.pdf}}
	\vspace{-0.cm}
	\caption{\textbf{AFFM architecture, operating at middle scale.} AFFMs for the remaining scales follow the same design.}
	\label{affm} 
\end{figure}

\textbf{Adaptive feature fusion module.} Recent networks such as \cite{ye2020pmbanet,tang2021joint} typically concatenate RGB and depth features directly during feature fusion,  followed by additional operations such as channel attention~\citep{zhang2018image} to capture useful information. In contrast, inspired by \cite{liu2021multi}, we run adaptive feature fusion through AFFM in two steps to strengthen the reconstruction of HF cues, as illustrated in Fig.~\ref{affm}. We differentiate from \cite{liu2021multi} by using dynamic convolution~\citep{chen2020dynamic} to better aggregate depth and HR features. In the first step, we generate dynamic weights $\pi_i, i=0,1,2$, which are then assigned to features from different scales within the current stage. Finally, we perform element-wise summation to obtain the feature maps $F^{'}$. For clarity, the figure shows the module working at the middle scale of the network as an example, with the others sharing the same design. 
{The process is defined as follows:
\begin{gather}
    F_{cat}=Conv_{3\times3}(Conv_{1\times1}([F^0\downarrow,F^1,F^2\uparrow]))       \label{eq_affm1_0}
    \\
    \{\pi_0,\pi_1,\pi_2\}=\sigma(Avgpool(F_{cat})) \label{eq_affm1_1}
    \\
    F^{'}=\pi_0\cdot F^0\downarrow+\pi_1\cdot F^1+\pi_2\cdot F^2\uparrow  \label{eq_affm1_2}
\end{gather}
where $F^{i}, i=0,1,2$ denotes the feature maps from the three scales, and $\downarrow$, $\uparrow$ are respectively downsampling and upsampling operators.}

In the second step, gradient features $F_{edge}$ from HFEB are concatenated with $F^{'}$. Then, per-pixel attention maps $F_{att}$ are generated by a ResBlock \citep{he2016deep} followed by an average pooling operation. These attention maps are then applied directly to the adaptively fused features $F^{'}$ through element-wise multiplication operation. Finally, after $1\times 1$ convolution, the attention-guided features $F_{out}^{i}$ are delivered to the corresponding scale of the current stage. In Fig. \ref{affm}, the output is passed to the middle scale of the decoder. AFFMs working at the other scales send their output to the corresponding scale in the decoder.  
{This step can be formalized as follows:
\begin{gather}
    F^{''}=[Avgpool(Conv_{3\times3}(F_{edge})),F^{'}]  \label{eq_affm2_0}
    \\
    F_{att}=\sigma(Avgpool(ResBlock(F^{''}))) \label{eq_affm2_1}
    \\
    F_{out}^{1}=con_{1\times1}(F^{'}\otimes F_{att})  \label{eq_affm2_2}
\end{gather}
where $\otimes$ is an element-wise multiplication operation and $F_{out}^{1}$ denotes the output features at the middle scale.}


\begin{figure}	
	\centering	
    \includegraphics[width=0.45\textwidth]{./figs/framework/LCF.pdf}
    \vspace{-0.cm}
	\caption{\textbf{Low-cut filtering module (LCF).} LF features are extracted through DCT and multi-spectral channel attention, and subtracted from the input to retain HF features.}	
	\label{lcf}
\end{figure}

\textbf{Low-cut filtering module.} 
The performance of our method greatly benefits from the explicit gradient information, but some valuable high-frequency information still vanishes. This fact motivates us to consider extracting complementary information in the frequency domain. As a common practice \citep{chang2007reversible,lin2010improving}, we use the low-frequency information of the discrete cosine transform (DCT) to compress images. Based on the design approach proposed in \cite{qin2021fcanet}, we develop a filtering module utilizing feature decomposition in the frequency domain to extract low-frequency components from the input. Specifically, we apply a $1\times1$ convolution followed by a channel split to the input color image $I_{hr}^{0}$. Then, we can obtain assigned frequency components from the output features $[{f}_0, {f}_1, \cdot\cdot\cdot, {f}_{n-1}]$ after DCT. Thus, the multi-spectral channel attention maps are generated by a fully connected layer and sigmoid activation. According to \cite{qin2021fcanet}, the low-frequency information is first assured to pass. Thus, we subtract such a low-frequency component from the input features producing the complementary high-frequency features $F_{rgb}$. Fig.~\ref{lcf} illustrates LCF in detail.  The high-frequency cues extracted from these features enable GDRB to progressively super-resolve LR depth maps into HR ones.


\textbf{Refinement.} To enhance the depth quality further, we optionally feed our final output into NLSPN \citep{park2020non} for refinement. This variant of the method is referred to as \netname$^+$.


\subsection{Training Loss}
Our network is trained in an end-to-end fashion using two loss terms: depth loss $L_d$ and gradient loss $L_g$. The depth loss is defined as:
\begin{equation}
\begin{aligned}
\label{lossd}
    L_d\!=\!\parallel\!(\!\tilde{D}_{sr}\!-\!D_{gt})\!\odot\!\mathbb{I}\!\parallel_1\!&+\lambda_d\cdot\!\parallel\!(\!\tilde{D}_{sr}^{'}\!-\!D_{gt})\!\odot\!\mathbb{I}\!\parallel_1\! \\
        &+\lambda_d\cdot\!\parallel\!(\!\tilde{D}_{sr}^{''}\!-\!D_{gt})\!\odot\!\mathbb{I}\!\parallel_1
\end{aligned}
\end{equation}
where $D_{gt}$ is the ground truth depth, $\tilde{D}_{sr}$, $\tilde{D}_{sr}^{'}$ and $\tilde{D}_{sr}^{''}$ are predicted depth maps from different stages, and $\mathbb{I}$ is pixel validity, as defined in \cite{de2022learning}. We empirically set $\lambda_d=0.2$.
Gradient loss $L_g$ is computed on HEFB output, as:
\begin{equation}
    L_g=\parallel\tilde{E}-E_{gt}\parallel_1
\end{equation}
where $\tilde{E}$ is the predicted gradient map and $E_{gt}$ is the ground truth one, extracted according to \cite{liu2021multi}. Thus, the total loss can be defined as:
\begin{equation}
    L_{total}=L_d+\lambda_g\cdot L_g
\end{equation}
with $\lambda_g$ empirically set to 0.01.
\section{Experimental Results}
In this section, we validate the effectiveness of our proposal. We first introduce datasets, metrics and implementation details involved in our evaluation. Then, we compare \netname{} with state-of-the-art methods, conduct an ablation study on our model and, finally, discuss its limitations.


\begin{table*}[htbp] \scriptsize
	\renewcommand\tabcolsep{2.3pt} 
	\centering
	\scalebox{0.85}{
	\begin{tabular}{@{}ccccccccccccccccc@{}}
		\toprule
		 Dataset & Scale & Metrics & GF~\cite{he2010guided} & SD~\cite{ham2017robust}  & GSRPT~\cite{lutio2019guided} & MSG~\cite{hui2016depth} & DKN~\cite{kim2021deformable} & FDKN~\cite{kim2021deformable} & PMBANet~\cite{ye2020pmbanet} & FDSR~\cite{he2021towards} & JIIF~\cite{tang2021joint} & DCTNet~\cite{zhao2022discrete} & LGR~\cite{de2022learning} & DADA~\cite{metzger2022guided} & DSR-EI & DSR-EI$^+$ \\ \midrule
		\multirow{6}{*}{\rotatebox[origin=l]{90}{\scriptsize \textbf{Middlebury}}} & \multirow{2}{*}{$4\times$} 
		& MSE & 33.3 & 24.9 & 39.8 & 4.13 & 4.29 & 3.60 & 4.72 & 7.72 & 2.70 & 5.00 & 3.04 & \bronze{2.58} & \gold{2.46} & \silver{2.56} \\
		& & MAE & 1.27 & 0.46 & 0.79 & 0.22 & 0.18 & 0.16 & 0.25 & 0.35 & \bronze{0.11} & 0.24 & 0.13 & \bronze{0.11} & \silver{0.08} & \gold{0.07} \\ \cline{2-17}
		& \multirow{2}{*}{$8\times$} 
		& MSE & 40.5 & 82.5 & 32.7 & 10.5 & 11.2 & 10.4 & 9.48 & 23.2 & 8.01 & 15.1 & 7.26 & \silver{5.68} & \bronze{6.20} & \gold{5.13} \\
		& & MAE & 1.49 & 0.86 & 0.82 & 0.43 & 0.38 & 0.37 & 0.38 & 0.69 & 0.27 & 0.57 & 0.24 & \bronze{0.20} & \gold{0.18} & \gold{0.18} \\ \cline{2-17}
		& \multirow{2}{*}{$16\times$} 
		& MSE & 67.4 & 511 & 41.5 & 34.2 & 47.6 & 38.5 & 30.6 & 55.4 & 37.5 & 52.3 & 24.7 & \silver{16.3} & \gold{15.8} & \bronze{16.6}  \\
		& & MAE & 2.21 & 1.73 & 1.24 & 1.06 & 1.42 & 1.18 & 0.89 & 1.51 & 0.98 & 1.50 & 0.67 & \bronze{0.48} & \silver{0.47} & \gold{0.40} \\ \hline\hline
	    % middlebury end
		\multirow{6}{*}{\rotatebox[origin=l]{90}{\scriptsize \textbf{NYUv2}}} & \multirow{2}{*}{$4\times$}
		& MSE & 114 & 36.0 & 112 & 6.85 & 11.4 & 9.07 & 10.8 & 10.1 & \bronze{3.28} & 3.63 & 6.45 & 4.83 & \silver{2.82} & \gold{2.75}\\
		& & MAE & 3.91 & 1.31 & 3.61 & 0.81 & 1.03 & 0.85 & 0.93 & 0.94 & \bronze{0.52} & 0.68 & 0.73 & 0.64 & \silver{0.49} & \gold{0.47}\\ \cline{2-17}
		& \multirow{2}{*}{$8\times$} 
		& MSE & 142 & 105 & 122 & 24.1 & 29.8 & 29.9 & 17.2 & 19.5 & \bronze{15.2} & 20.9 & 19.6 & 16.6 & \gold{11.8} & \gold{11.8}\\
		& & MAE & 4.47 & 2.57 & 3.86 & 1.66 & 1.82 & 1.80 & 1.38 & 1.38 & \bronze{1.29} & 1.79 & 1.42 & 1.30 & \silver{1.12} & \gold{1.09}\\ \cline{2-17}
		& \multirow{2}{*}{$16\times$} 
		& MSE & 249 & 533 & 219 & 84.5 & 115 & 113 & 84.9 & 86.4 & 59.9 & 77.0 & 67.5 & \bronze{59.0} & \silver{47.8} & \gold{47.1} \\
		& & MAE & 6.34 & 5.07 & 5.40 & 3.35 & 4.01 & 3.95 & 3.26 & 3.35 & 2.81 & 3.61 & 2.90 & \bronze{2.64} & \silver{2.48} & \gold{2.40}\\ \hline\hline
		% NYU end
		\multirow{6}{*}{\rotatebox[origin=l]{90}{\scriptsize \textbf{DIML}}} & \multirow{2}{*}{$4\times$}
		& MSE & 25.6 & 10.5 & 20.7 & 1.73 & 3.47 & 2.20 & 3.05 & 2.75 & \bronze{1.19} & 2.09 & 1.68 & 1.33 & \silver{0.70} & \gold{0.65} \\
		& & MAE & 1.45 & 0.40 & 1.15 & 0.22 & 0.33 & 0.23 & 0.31 & 0.29 & \bronze{0.16} & 0.31 & 0.20 & 0.17 & \silver{0.13} & \gold{0.12} \\ \cline{2-17}
		& \multirow{2}{*}{$8\times$} 
		& MSE & 34.1 & 44.9 & 23.0 & 4.13 & 5.47 & 5.95 & 5.87 & 8.40 & 3.65 & 7.08 & 3.51 & \bronze{2.93} & \silver{2.12} & \gold{2.09} \\
		& & MAE & 1.77 & 0.83 & 1.26 & 0.40 & 0.45 & 0.47 & 0.47 & 0.66 & 0.32 & 0.65 & 0.31 & \bronze{0.28} & \gold{0.22} & \gold{0.22} \\ \cline{2-17}
		& \multirow{2}{*}{$16\times$} 
		& MSE & 66.3 & 41.1 & 39.3 & 13.0 & 19.3 & 20.8 & 13.8 & 32.9 & 11.7 & 23.4 & 9.45 & \bronze{7.61} & \gold{6.29} & \silver{6.31} \\
		& & MAE & 2.74 & 1.91 & 1.78 & 0.93 & 1.20 & 1.24 & 0.87 & 1.66 & 0.81 & 1.75 & 0.68 & \bronze{0.59} & \silver{0.52} & \gold{0.50} \\
		% DIML end
    \bottomrule
	\end{tabular}}
    \vspace{-0.3cm}
	\caption{\textbf{Results on Middlebury, NYUv2 and DIML datasets.} The lower the MSE and MAE, the better.}
	\label{sota_comparison_mid_nyu_diml}
\end{table*}



\begin{table*}[t] \footnotesize
	\renewcommand\tabcolsep{1.5pt} 
	\centering
	\scalebox{0.85}{
	\begin{tabular}{@{}ccccccccccccccccc@{}}
		\toprule
		 Scale & SDF~\cite{li2016deep} & SVLRM~\cite{pan2019spatially} & DJF~\cite{li2016deep} & DJFR~\cite{li2019joint} & PAC~\cite{su2019pixel} & CUNet~\cite{deng2020deep} & FDKN~\cite{kim2021deformable} & DKN~\cite{kim2021deformable} & FDSR~\cite{he2021towards} & DCTNet~\cite{zhao2022discrete} & RSAG~\cite{yuan2023recurrent} & DSR-EI & DSR-EI$^+$ \\ \midrule
		$4\times$ & 2.00 & 3.39 & 3.41 & 3.35 & 1.25 & 1.18 & 1.18 & 1.30 & 1.16 & \bronze{1.07} & 1.14 & \gold{0.91} & \gold{0.91} \\
		$8\times$ & 3.23 & 5.59 & 5.57 & 5.57 & 1.98 & 1.95 & 1.91 & 1.96 & 1.82 & 1.78 & \bronze{1.75} & \gold{1.37} & \silver{1.38} \\
		$16\times$ & 5.16 & 8.28 & 8.15 & 7.99 & 3.49 & 3.45 & 3.41 & 3.42 & 3.06 & 3.18 & \bronze{2.96} & \gold{2.10} & \gold{2.10}  \\
    \bottomrule
	\end{tabular}}
	\vspace{-0.3cm}
	\caption{\textbf{Results on the RGBDD dataset.} We report RMSE, the lower the better.}
	\label{sota_comparison_rgbdd}
\end{table*}



\subsection{Datasets and Metrics}
We evaluate \netname{} on four datasets, compared with existing methods when super-solving depth maps by three different upsampling factors: $4\times,\ 8\times$, and $16\times$. 

\textbf{Middlebury}\cite{scharstein2003high,scharstein2007learning,hirschmuller2007evaluation,scharstein2014high}. We train all learning-based methods using 50 RGB-D images with ground truth from Middlebury 2005, 2006 and 2014 datasets. As in~\cite{de2022learning}, we retain 5 for validation and 5 for testing. 

\textbf{NYUv2}\cite{silberman2012indoor}. It contains 1449 RGB-D images in total. Following \cite{de2022learning}, we randomly split it into 849 RGB-D images for the training set, 300 for the validation set and 300 for the test set. Compared to \cite{ye2020pmbanet,liu2022pdr}, it comes with a validation set to make the comparison fairer.

\textbf{DIML}\cite{kim2016structure,kim2017deep,kim2018deep,cho2021deep} consists of 2 million color images and corresponding depth maps from indoor and outdoor scenes. We adopt the same strategy outlined in \cite{de2022learning}, i.e., considering only the indoor data subset, and use 1440 for training, 169 for validation, and 503 for testing.

\textbf{RGBDD}\cite{he2021towards} is a new real-world dataset for GDSR, which consists of 4811 image pairs. For evaluation, we follow the protocol described in \cite{he2021towards}, using 2215 images (1586 portraits, 380 plants, 249 models) as the training set and 405 images (297 portraits, 68 plants, 40 models) as the test set. 

\textbf{Metrics.} Following \cite{de2022learning}, we compute mean square error (MSE / $cm^2$) and mean absolute error (MAE / $cm$) as metrics on Middlebury, NYUv2 and DIML. For RGBDD, we use root mean square error (RMSE / $cm$) as in \cite{he2021towards}. 

\subsection{Implementation Details}
During training, the HR depth maps and the color images are randomly cropped into $256\times 256$ patches. LR depth patches are generated by bicubic interpolation at $64\times 64$, $32\times 32$, $16\times 16$ resolution for $4\times$, $8\times$ and $16\times$ factors, respectively. We randomly extract about 75K, 168K, 223K and 232K patches from Middlebury, NYUv2, DIML and RGBDD for training. Before being fed to the network, depth maps and images are normalized in the [0, 1] range.

We use Pytorch \cite{paszke2019pytorch} to implement and train \netname{}, on a single Nvidia RTX 3090 GPU. The batch size is set to 4, using Adam as the optimizer. The learning rate is initialized to $1\times 10^{-4}$, then performing a 5-epoch warm-up and cosine annealing. We use random rotation, horizontal/vertical flipping as data augmentation. According to the size of the four datasets, we train our network for 1505, 198, 155 and 109 epochs on Middlebury, NYUv2, DIML and RGBDD, respectively. 
When evaluating results on a specific dataset, we do not perform any pre-training on the others. Following \cite{de2022learning}, testing is performed by processing $256\times256$ patches at a time on Middlebury, NYUv2 and DIML for fairness, while full-resolution images are processed for RGBDD.

\begin{figure*}[t] 
	\centering
	\renewcommand\tabcolsep{1.5pt} 
	\begin{tabular}{cccccccccccc}
	\vspace{-0.1cm}
    \rotatebox[origin=l]{90}{\scriptsize \quad \textbf{Middlebury}} & \includegraphics[height=0.6in]{./figs/sota_comp_middlebury/389/Middlebury_389_img.pdf}
        \hspace{-1.8mm} & \includegraphics[height=0.6in]{./figs/sota_comp_middlebury/389/Middlebury_389_source.pdf}
	\hspace{-1.8mm} &  \includegraphics[height=0.6in]{./figs/sota_comp_middlebury/389/Middlebury_389_GT.pdf}
	\hspace{-1.8mm} & \includegraphics[height=0.6in]{./figs/sota_comp_middlebury/389/Middlebury_389_PMBA.pdf}
	\hspace{-1.8mm} & \includegraphics[height=0.6in]{./figs/sota_comp_middlebury/389/Middlebury_389_FDSR.pdf}
	\hspace{-1.8mm} & \includegraphics[height=0.6in]{./figs/sota_comp_middlebury/389/Middlebury_389_JIIF.pdf}
	\hspace{-1.8mm} & \includegraphics[height=0.6in]{./figs/sota_comp_middlebury/389/Middlebury_389_DCTnet.pdf}
	\hspace{-1.8mm} & \includegraphics[height=0.6in]{./figs/sota_comp_middlebury/389/Middlebury_389_LGR.pdf}
	\hspace{-1.8mm} & \includegraphics[height=0.6in]{./figs/sota_comp_middlebury/389/Middlebury_389_MSS.pdf}
        
        \hspace{-1.8mm} & \includegraphics[height=0.6in]{./figs/sota_comp_middlebury/389/Middlebury_389_ours.pdf}
    \\ \vspace{-0.1cm}
    
    \rotatebox[origin=l]{90}{\scriptsize \quad \textbf{NYUv2}} & \includegraphics[height=0.6in]{./figs/sota_comp_nyu/357/NYU_357_img.pdf}
	\hspace{-1.8mm} & \includegraphics[height=0.6in]{./figs/sota_comp_nyu/357/NYU_357_source.pdf}
	\hspace{-1.8mm} & \includegraphics[height=0.6in]{./figs/sota_comp_nyu/357/NYU_357_GT.pdf}
	\hspace{-1.8mm} & \includegraphics[height=0.6in]{./figs/sota_comp_nyu/357/NYU_357_PMBA.pdf}
	\hspace{-1.8mm} & \includegraphics[height=0.6in]{./figs/sota_comp_nyu/357/NYU_357_FDSR.pdf}
	\hspace{-1.8mm} & \includegraphics[height=0.6in]{./figs/sota_comp_nyu/357/NYU_357_JIIF.pdf}
	\hspace{-1.8mm} & \includegraphics[height=0.6in]{./figs/sota_comp_nyu/357/NYU_357_DCTnet.pdf}
	\hspace{-1.8mm} & \includegraphics[height=0.6in]{./figs/sota_comp_nyu/357/NYU_357_LGR.pdf}
	\hspace{-1.8mm} & \includegraphics[height=0.6in]{./figs/sota_comp_nyu/357/NYU_357_MSS.pdf}
 
	\hspace{-1.8mm} & \includegraphics[height=0.6in]{./figs/sota_comp_nyu/357/NYU_357_ours.pdf}
	\\ 
	
    \rotatebox[origin=l]{90}{\scriptsize \quad \textbf{DIML}} & \includegraphics[height=0.6in]{./figs/sota_comp_diml/856/DIML_856_img.pdf}
	\hspace{-1.8mm} & \includegraphics[height=0.6in]{./figs/sota_comp_diml/856/DIML_856_source.pdf}
	\hspace{-1.8mm} & \includegraphics[height=0.6in]{./figs/sota_comp_diml/856/DIML_856_GT.pdf}
	\hspace{-1.8mm} & \includegraphics[height=0.6in]{./figs/sota_comp_diml/856/DIML_856_PMBA.pdf}
	\hspace{-1.8mm} & \includegraphics[height=0.6in]{./figs/sota_comp_diml/856/DIML_856_FDSR.pdf}
	\hspace{-1.8mm} & \includegraphics[height=0.6in]{./figs/sota_comp_diml/856/DIML_856_JIIF.pdf}
	\hspace{-1.8mm} & \includegraphics[height=0.6in]{./figs/sota_comp_diml/856/DIML_856_DCTnet.pdf}
	\hspace{-1.8mm} & \includegraphics[height=0.6in]{./figs/sota_comp_diml/856/DIML_856_LGR.pdf}
	\hspace{-1.8mm} & \includegraphics[height=0.6in]{./figs/sota_comp_diml/856/DIML_856_MSS.pdf}
 
	\hspace{-1.8mm} & \includegraphics[height=0.6in]{./figs/sota_comp_diml/856/DIML_856_ours.pdf}
 \\
	& \scriptsize \textbf{(a)} RGB & \scriptsize \textbf{(b)} Bicubic & \scriptsize \textbf{(c)} GT & \scriptsize \textbf{(d)} PMBA & \scriptsize \textbf{(e)} FDSR & \scriptsize \textbf{(f)} JIIF & \scriptsize \textbf{(g)} DCTNet & \scriptsize \textbf{(h)} LGR & \scriptsize \textbf{(i)} \netname{} & \scriptsize \textbf{(j)} \netname{} (depth)
	\end{tabular}
    \vspace{-0.3cm}
	\caption{\textbf{Qualitative comparison on Middlebury, NYUv2 and DIML datasets (scaling factor $8\times$).} From left to right: (a) RGB image, (b) Bicubic upsampled depth map, (c) GT; then, error maps achieved by selected methods: (d) PMBA~\cite{ye2020pmbanet}, (e) FDSR~\cite{he2021towards}, (f) JIIF~\cite{tang2021joint}, (g) DCTNet~\cite{zhao2022discrete}, (h) LGR~\cite{de2022learning}; finally, (i) error maps and (j) predictions by \netname.} 
	\label{qualitative}
\end{figure*}


\begin{table*}[htbp] \footnotesize
	\renewcommand\tabcolsep{1.5pt} 
	\centering
	\scalebox{0.85}{
	\begin{tabular}{@{}ccccccccccccccccc@{}}
		\toprule
		 Testing Dataset & Metric & GF\cite{he2010guided} & SD~\cite{ham2017robust}  & GSRPT~\cite{lutio2019guided} & MSG~\cite{hui2016depth} & FDKN~\cite{kim2021deformable} & PMBANet~\cite{ye2020pmbanet} & FDSR~\cite{he2021towards} & JIIF~\cite{tang2021joint} & DCTNet~\cite{zhao2022discrete} & LGR~\cite{de2022learning} & \netname$^+$ \\ \midrule
		\multirow{2}{*}{DIML}
		& MSE & 34.1 & 44.9 & 23.0 & 5.76 & 6.74 & 7.35 & 7.73 & \silver{4.10} & 5.64 & \bronze{4.95} & \gold{3.72} \\
		& MAE & 1.77 & 0.83 & 1.26 & 0.51 & 0.53 & 0.59 & 0.74 & \silver{0.38} & 0.77 & \bronze{0.40} & \gold{0.36} \\ \hline
		\multirow{2}{*}{Middlebury\textit{-HR}}
		& MSE & 40.5 & 82.5 & 32.7 & 11.0 & \bronze{10.0} & \silver{9.62} & 18.4 & 19.3 & 17.5 & \gold{8.25} & 14.6 \\
		& MAE & 1.49 & 0.86 & 0.82 & 0.54 & \silver{0.43} & \bronze{0.46} & 0.73 & 0.74 & 0.77 & \gold{0.35} & 0.54  \\ \hline
		\multirow{2}{*}{Middlebury\textit{-LR}}
		& MSE & 25.6 & 28.8 & 15.8 & 8.89 & 5.54 & 4.16 & 6.92 & 4.40 & 6.96 & 5.94 & \gold{3.44} \\
		& MAE & 2.31 & 2.07 & 1.73 & 1.62 & 0.99 & \silver{0.91} & 1.09 & \bronze{0.92} & 1.15 & 1.11 & \gold{0.87}  \\
        \bottomrule
	\end{tabular}}
	\vspace{-0.3cm}
	\caption{\textbf{Cross-dataset generalization.} All methods are trained on NYUv2 and tested on DIML/Middlebury with factor $8\times$. Middlebury\textit{-HR} is the test set defined in \cite{de2022learning}, Middlebury\textit{-LR} is the one from \cite{tang2021joint}. The lower MSE and MAE, the better. }
	\label{cross-data_comparison}
\end{table*}

\subsection{Comparison with State-of-the-Art}
We compare \netname{} to GF \cite{he2010guided}, SD \cite{ham2017robust}, GSRPT \cite{lutio2019guided}, MSG \cite{hui2016depth}, DKN and its fast implementation FDKN \cite{kim2021deformable}, PMBANet \cite{ye2020pmbanet}, FDSR \cite{he2021towards}, JIIF \cite{tang2021joint}, DCTNet \cite{zhao2022discrete}, LGR \cite{de2022learning}, and finally to DADA~\cite{metzger2022guided} on Middlebury, NYUv2 and DIML datasets. We could not compare with PDRNet \cite{liu2022pdr} under the same setting because the source code is unavailable at the time of writing. For the other methods, we use the results from \cite{de2022learning} or the officially published codes, and results from \cite{yuan2023recurrent,metzger2022guided} for concurrent works. On the RGBDD dataset, the proposed network is compared to SDF~\cite{li2016deep}, SVLRM \cite{pan2019spatially}, DJF~\cite{li2016deep}, DJFR~\cite{li2019joint}, PAC~\cite{su2019pixel}, CUNet~\cite{deng2020deep}, FDKN~\cite{kim2021deformable}, DKN~\cite{kim2021deformable}, FDSR~\cite{he2021towards}, DCTNet~\cite{zhao2022discrete} and RASG~\cite{yuan2023recurrent}. To be fair with DCTNet~\cite{zhao2022discrete}, we downsample depth maps as the LR input.  
When reporting results, we highlight \gold{absolute}, \silver{second} and \bronze{third} best methods for each metric on each dataset.

\textbf{Quantitative Comparison.} Tabs. \ref{sota_comparison_mid_nyu_diml} and \ref{sota_comparison_rgbdd} report the accuracy of super-solved depth maps at factors $4\times$, $8\times$ and $16\times$ on the four datasets. As expected, learning-based methods show a significant improvement over traditional methods \cite{he2010guided,ham2017robust,lutio2019guided}. \netname{} vastly outperforms any existing network, with larger gaps in accuracy with the increasing of the upsampling factor. This can be attributed to the limitations affecting existing methods, i.e., 1) the guidance of either explicit or implicit RGB features alone being insufficient; 2) multi-modal information fusion on a single scale being not flexible enough to deal with complex scenes. Both limitations are fully addressed by \netname, which consistently outperforms concurrent works \cite{metzger2022guided,yuan2023recurrent}. 


The margin is consistent both on perfect (Middlebury) and noisy datasets (NYUv2, DIML, RGBDD), with the latter being a more challenging, realistic benchmark. Although \netname$^+$ is definitely the absolute best, its margin over \netname{} is negligible, with tiny gains yielded by NLSPN with respect to our main modules. Indeed, \netname{} alone consistently outperforms any other approach already.

       
\textbf{Qualitative Comparison.}
Fig. \ref{qualitative} shows qualitative comparisons of $8\times$ super-solved depth maps on Middlebury, NYUv2 and DIML datasets, respectively. From left to right, we show, the RGB image and LR depth map, followed by the ground truth HR depth and error maps obtained by several state-of-the-art frameworks, concluding with ours in the second-to-last columns. In each of the three examples, the lower error magnitude produced by \netname{}$^+$ further demonstrates its superior accuracy. 

\textbf{Cross-dataset Generalization.}
We conclude the comparison with existing methods by conducting cross-dataset experiments with $8\times$ factor. All methods are trained on the NYUv2 dataset and directly evaluated on DIML and Middlebury. Table \ref{cross-data_comparison} collects quantitative results for the 11 selected methods. Again, CNN-based methods attain better performance than traditional approaches, despite the domain gap playing a significant role in performance -- as evident by comparing results with Table \ref{cross-data_comparison}. Nonetheless, \netname{} outperforms any other framework on DIML. 


\begin{figure}	
	\centering	
	\captionsetup[subfigure]{font=footnotesize,textfont=footnotesize}
	\subfloat[RGB]{	
		\centering	
		\label{cross_dataset} 
		\includegraphics[height=0.8in]{./figs/ablation_figure/cross_dataset/receptive_field/cross_dataset.pdf}}	
	\hspace{-2mm}
	\subfloat[$D_{hr}$]{	
		\centering	
		\label{HR}
		\includegraphics[width=0.8in]{./figs/ablation_figure/cross_dataset/receptive_field/HR.pdf}}
	\hspace{-2mm}
	\subfloat[$D_{lr}$]{	
		\centering	
		\label{LR}
		\includegraphics[width=0.8in]{./figs/ablation_figure/cross_dataset/receptive_field/LR.pdf}}
		\vspace{-0.3cm}
	\caption{\textbf{Image context processed on Middlebury -- HR vs LR.} (a) RGB image and depth patches $D$ processed when testing on (b) Middlebury\textit{-HR} and (c) Middlebury\textit{-LR}. }	
	\label{hr-lr} 
\end{figure}

When considering the Middlebury dataset, we evaluate using the setting proposed in \cite{de2022learning} -- Middlebury\textit{-HR} in the table. In this case, our results are slightly less accurate compared to a few existing methods. However, given the very high resolution of Middlebury images, we argue that this testing protocol -- i.e., consisting of processing $256\times 256$ crops at a time -- penalizes our network's ability to leverage the global context in the input that results irremediably reduced to a very local area in these images. Therefore, we also evaluate on Middlebury test set defined by~\cite{tang2021joint} -- Middlebury-\textit{LR} in the table. Note that different subsets of images are used in Middlebury\textit{-HR} and Middlebury-\textit{LR} splits. Besides, Middlebury-\textit{LR} images are resized and processed without cropping, i.e., used at full-size after resizing, allowing to fully exploit global context, while this is not feasible with Middlebury-\textit{HR} due to memory constraints. In this case, \netname{} attains the best performance again, confirming our previous analysis, as shown in Tab. \ref{cross-data_comparison}. Such a difference in terms of context is highlighted in Fig. \ref{hr-lr}.

\begin{table}[t]
    \centering
	\renewcommand\tabcolsep{3pt} 
    \scalebox{0.5}{
    \begin{tabular}{ccc}

    \begin{tabular}{@{}ccccccc@{}} %\label{hf_infomation}
		\toprule
		\textbf{No.} & \textbf{Gradient} & \tabincell{c}{\textbf{Shallow} \\ \textbf{Feature}} & \textbf{LCF} & \textbf{ResBlock} & \textbf{MSE} & \textbf{MAE}\\
		\midrule
		(\uppercase\expandafter{\romannumeral1}) & \XSolidBrush &  \Checkmark     &  \Checkmark &  & 13.1 & 1.19 \\
		(\uppercase\expandafter{\romannumeral2}) & \Checkmark &    \XSolidBrush   &   &  & 12.4 & 1.14 \\
		(\uppercase\expandafter{\romannumeral3}) & \Checkmark &    \Checkmark     &   & \Checkmark & 12.3 & 1.15 \\
		\rowcolor{LightYellow}
		(\uppercase\expandafter{\romannumeral4}) & \Checkmark &    \Checkmark     & \Checkmark  &  & \gold{11.8} & \gold{1.12} \\
		\bottomrule
	\end{tabular}
	
	& \quad &
	
	\begin{tabular}{@{}clcc@{}} %\label{edge_types}
		\toprule
		\specialrule{0em}{3pt}{3pt}
		\multicolumn{1}{c}{\textbf{No.}} & 
		\tabincell{l}{\textbf{HF Information} \textbf{ \quad\quad\quad\quad}} & \textbf{MSE} & \textbf{MAE}\\
		\specialrule{0em}{3pt}{2pt}
		\midrule
		(\uppercase\expandafter{\romannumeral1}) & 
		{Canny Edge} & 12.0 & 1.13 \\
		(\uppercase\expandafter{\romannumeral2}) & 
		{Gaussian Edge} & 12.1 & 1.16 \\
		(\uppercase\expandafter{\romannumeral3}) & 
		{DCT} & 12.1 & 1.15 \\
		(\uppercase\expandafter{\romannumeral4}) & 
		{Wavelet Transform} & 12.1 & 1.15  \\
		\rowcolor{LightYellow}
		(\uppercase\expandafter{\romannumeral5}) & 
		{Gradient Map} & \gold{11.8} & \gold{1.12} \\
		\bottomrule
	\end{tabular}
	
	\\
	\textbf{(a)} & \quad & \textbf{(b)} 
	\\
	\\
	
	
	\begin{tabular}{@{}clcccc@{}} %\label{dsp_ablation}
		\toprule
		\textbf{No.} & \textbf{Config.} & \textbf{Params (M)} & \textbf{Flops (G)} & \textbf{MSE} & \textbf{MAE}\\
		\midrule
		(\uppercase\expandafter{\romannumeral1}) & EdgeNet \cite{liu2021multi}    & 5.78 &  95.6  & 12.0 & \gold{1.12} \\
		(\uppercase\expandafter{\romannumeral2}) & SCPA \cite{zhao2020efficient}  & 0.29 &  13.1  & 12.5 & 1.16 \\
		\rowcolor{LightYellow}
		(\uppercase\expandafter{\romannumeral3}) & HFEB       & \gold{0.27} & \gold{11.6}  & \gold{1.18} & \gold{1.12} \\
		\rowcolor{white}
		\bottomrule
		\multicolumn{4}{c}{\quad\quad\textbf{(c)}} \\
		\\
		\toprule
		\textbf{No.} & \textbf{Config.} & \textbf{Params (M)} & \textbf{MSE} & \textbf{MAE}\\
		\midrule
		(\uppercase\expandafter{\romannumeral1}) & 
		w/o AFFM        & -   & 12.7 & 1.16 \\
		(\uppercase\expandafter{\romannumeral2}) & 
		w/o att         & 1.3 & 12.2 & 1.13 \\
		(\uppercase\expandafter{\romannumeral3}) & 
		Concat.  & 4.5 & 12.2 & 1.13 \\
		\rowcolor{LightYellow}
		(\uppercase\expandafter{\romannumeral4}) & 
		AFFM & 3.0 & \gold{11.8} & \gold{1.12} & \\
		\rowcolor{white}
		\bottomrule
		\multicolumn{4}{c}{\quad\quad\textbf{(e)}} \\
	\end{tabular}
	
	
	& \quad &
	
	
	\begin{tabular}{@{}clccc@{}} %\label{affm_setting}
		\toprule
		\textbf{No.} & \textbf{Scales} & \textbf{Params (M)} & \textbf{MSE} & \textbf{MAE}\\
		\midrule
		(\uppercase\expandafter{\romannumeral1}) & 
		H1              & 1.5 & 12.3 & 1.14 \\
		\rowcolor{LightYellow}
		(\uppercase\expandafter{\romannumeral2}) & 
		H1, H2       & 3.0 & \gold{11.8} & \gold{1.12} \\
		\rowcolor{white}
		(\uppercase\expandafter{\romannumeral3}) & 
		H1, H2, H3              & 4.5 & \gold{11.8} & \gold{1.12} \\
		\bottomrule
		\multicolumn{4}{c}{\textbf{(d)}} \\
% 		\\
% 		\\
        \specialrule{0em}{5.4pt}{5.4pt} %
		\toprule
		\specialrule{0em}{1.7pt}{1.7pt} %
		\textbf{No.} & \textbf{Stages} & \textbf{Params (M)} & \textbf{MSE} & \textbf{MAE}\\
		\specialrule{0em}{1.7pt}{1.7pt} %
		\midrule
		\specialrule{0em}{1.8pt}{1.8pt} %
		(\uppercase\expandafter{\romannumeral1}) & 
		$1$   & 14.2 & 13.3 & 1.19 \\
		\specialrule{0em}{1.8pt}{1.8pt} %
		\rowcolor{LightYellow}
		(\uppercase\expandafter{\romannumeral2}) & 
		$2$   & 25.0 & 11.8 & 1.12 \\
		\specialrule{0em}{1.8pt}{1.8pt} %
		\rowcolor{white}
		(\uppercase\expandafter{\romannumeral3}) & 
		$3$   & 37.5 & \gold{11.6} & \gold{1.10} \\
		\specialrule{0em}{1.8pt}{1.8pt} %
		\bottomrule
		\multicolumn{4}{c}{\quad\quad\textbf{(f)}} \\
	\end{tabular}
	
    \end{tabular}}
    \vspace{-0.3cm}
    \caption{\textbf{Ablation study (NYUv2 test set, $8\times$ factor).} We measure the impact of (a) explicit vs implicit HR features, (b) different kinds of HF supervision, (c) different sub-networks for explicit HF features extraction, (d) scales at which AFFM is applied, (e) modules building AFFM, (f) number of stages in GDRB. In yellow, configurations corresponding to our final model without NLSPN.}
    \label{tab:ablations}
\end{table}


\subsection{Ablation Study}
We now perform a series of ablation experiments to measure the impact of key components and parameters in \netname. Tab. \ref{tab:ablations} collects the outcome of these studies, conducted on NYUv2 test set with $8\times$ factor. Without loss of fairness, NLSPN is never used here -- to fully focus on the impact of single components. 

\textbf{(a) Implicit vs Explicit High-Frequency Features.}
To measure the impact of both implicit and explicit HR features, we compare the performance of the proposed network and its variants when extracting either only one of the two. The quantitative results are collected in Tab.~\ref{tab:ablations}(a). Without the help of gradient maps (I), the performance of the network significantly degrades. We believe this is caused by the difficulty in effectively extracting fine structures or salient edges required for LR depth maps from implicit HF features alone. Moreover, explicit features highlight regions in the image that need to be focused on, avoiding \netname{} to learn to localize them and easing its task. 


Nonetheless, explicit HF features alone as guidance (II) are insufficient as well. We argue that the explicit information might neglect some RGB features, whereas implicit HF feature extraction can recover them. Furthermore, to verify the effectiveness of LCF, we replace it with ResBlock~\cite{he2016deep} (III) to extract shallow features from RGB images, highlighting a negative impact on implicit features extraction -- i.e., it results less accurate than (II). 

\textbf{(b) Ablation on Explicit High-Frequency Features.}
We now investigate which kind of HF information is more effective for our framework. Purposely, we train HFEB with supervision coming from five different HF features used as ground truth edge maps $E_{gt}$. Tab.~\ref{tab:ablations}(b) collects results from this experiment, highlighting that Canny edges (I) and Gradient maps (V) lead to slightly better results. 


\textbf{(c) Impact of HFEB.}
To verify the effectiveness of HFEB, we replace it with EdgeNet~\cite{liu2021multi} -- based on the widely-used U-net structure -- and SCPA~\cite{zhao2020efficient}, which inspires our scaling strategy. As shown in Tab.~\ref{tab:ablations}(c), EdgeNet (I) achieves lower MSE and MAE than SCPA (II), yet needs more parameters -- 5.78M vs. 0.29M. HFEB (III) yields the same accuracy as EdgeNet, with fewer parameters than SCPA, thus being both more accurate and efficient. 



\textbf{(d -- e) Impact of AFFM.}
We now measure the effectiveness of AFFM. Tab.~\ref{tab:ablations}(d) shows results obtained by deploying AFFM at different scales, respectively the highest (I), the first two (II) and all of the three scales. We can notice how performing fusion at the highest scale alone results insufficient, whereas using multi-scale features for fusion yields improvements, despite saturating already when using two scales, with the lowest one not providing additional, meaningful details to be taken into account.

Furthermore, we ablate AFFM in its single components. Tab.~\ref{tab:ablations}(e) resumes the outcome of this evaluation. 
We first test the performance of \netname{} without AFFM (I), highlighting a large drop in accuracy. By adding dynamic fusion, yet without using attention (II) vastly improves the results already, while replacing the weighted sum in the upper of Fig.~\ref{affm} with concatenation and a ResBlock~\cite{he2016deep} (III) yields worse results compared to our full AFFM (IV). 

\textbf{(f) Impact of Stages Number.}
To conclude, we evaluate the impact of the multi-stage design.
As shown in Tab.~\ref{tab:ablations}(f), a single-stage architecture (I) is vastly outperformed by deploying two stages (II), yet at the expense of doubling the number of parameters. Furthermore, while the three-stage architecture (III) still yields some improvement, the benefit is minor in comparison to the significant increase in parameters. Hence, we choose two stages as the default configuration to balance accuracy and efficiency.


\begin{table}[t] \footnotesize
	\renewcommand\tabcolsep{1.5pt} 
	\centering
	\scalebox{0.8}{
	\begin{tabular}{@{}lcccccc@{}}
		\toprule
		 & PMBANet~\cite{ye2020pmbanet} & FDSR~\cite{he2021towards} & JIIF~\cite{tang2021joint} & DCTNet~\cite{zhao2022discrete} & LGR~\cite{de2022learning} & Ours \\ 
		 \midrule
		 Runtime (ms)
		 & 26.9 & 1.03 & 89.8 & 9.03 & 26.4 & 51.5\\
		 Memory Peak (GB)
		 & 3.07 & 2.05 & 2.36 & 0.26 & 0.19 & 18.6 \\ 
		\bottomrule
	\end{tabular}}
	\vspace{-0.3cm}
	\caption{\textbf{Computational requirements}. Experiments on Nvidia RTX 3090 GPU, with $256\times256$ input and $8\times$ factor.}
	\label{runtime_memory}
    
\end{table}

\subsection{Limitations}
We conclude by listing a few limitations of \netname. As previously pointed out, global context is crucial for it to achieve the best performance. When this is unavailable, some accuracy is lost when generalizing across datasets. Moreover, the significant improvements over existing methods are paid for in terms of time/memory requirements. Tab. \ref{runtime_memory} highlights the higher runtime and, more evidently, peak memory usage. Future work will aim at reducing the overhead, while minimizing the drop in accuracy.


\section{Conclusion}
This paper proposed \netname{}, a depth super-resolution network, which includes a high-frequency extraction branch (HFEB) and a guided depth restoration branch (GDRB). Specifically, implemented as an efficient transformer, HFEB extracts explicit HF features. Then, GDRB deploys a two-stage encoder-decoder network to recover HR depth maps progressively, by adaptively fusing discriminative features while supplementing additional, implicit HF information. Exhaustive experiments demonstrate that \netname{} sets a new state-of-the-art for guided depth super-resolution.
\section*{Acknowledgments}
We acknowledge supports by the National Natural Science Foundation of China (No.61627811), Natural Science Foundation of Shaanxi Province (No. 2021JZ-04), Joint project of key R$\&$D universities in Shaanxi Province (No. 2021GXLH-Z-093), and the China Scholarship Council (CSC). 


%%%%%%%%% REFERENCES
{\small
\bibliographystyle{ieee_fullname}
\bibliography{reference}
}

\end{document}