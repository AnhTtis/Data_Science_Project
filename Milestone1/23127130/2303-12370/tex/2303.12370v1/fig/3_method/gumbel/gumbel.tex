% \begin{figure}[h]
% \centering
% \centerline{\includegraphics[width=0.45\textwidth]{fig/3_method/gumbel/gumbel.pdf}}%%图像路径:pic是文件夹名。
% \caption{Visualization of Gumbel-Align Contrastive learning.
% }
% \label{fig:gumbel}
% \end{figure}


% \begin{figure}[ht!]
%     \centering
%     \begin{subfigure}{0.3\textwidth}
%       \centering   
%       \includegraphics[width=1\linewidth]{fig/3_method/gumbel/gumbel1.pdf}
%        \caption{Prediction}
%     \end{subfigure}      
%     \hfill
%     \vline
%     \hfill  
%     \begin{subfigure}{0.5\textwidth}
%         \centering   
%         \includegraphics[width=\linewidth]{fig/3_method/gumbel/gumbel2.pdf}
%         \caption{Dataset Part-B}
%     \end{subfigure}
%     \hfill
%     \vline
%     \hfill
%     \begin{subfigure}{0.5\textwidth}
%         \centering   
%         \includegraphics[width=\linewidth]{fig/3_method/gumbel/gumbel3.pdf}
%         \caption{Data analysis}
%         % \label{fig:1c}
%     \end{subfigure}
%     \hfill
%     \vline
%     \hfill
%     \begin{subfigure}{0.5\textwidth}
%         \centering   
%         \includegraphics[width=\linewidth]{fig/3_method/gumbel/gumbel4.pdf}
%         \caption{Data analysis}
%         % \label{fig:1c}
%     \end{subfigure}
%     \caption{
%     \textbf{The summary of proposed benchmark RepCount}: The first two columns represent the part-A and part-B respectively, the right column shows the statistics of video length and repetition count of our dataset.
%     }
% \label{fig:datashow}
% \end{figure}


\begin{figure}[ht!]
  \centering
    \begin{subfigure}{0.23\textwidth}
      \centering   
      \includegraphics[width=\textwidth]{fig/3_method/gumbel/gumbel1.pdf}
        \caption{The Output of Gumbel-Softmax}
        \label{fig:1a}
    \end{subfigure}        
    \hfill 
    \begin{subfigure}{0.23\textwidth}
      \centering   
      \includegraphics[width=\textwidth]{fig/3_method/gumbel/gumbel2.pdf}
        \caption{Maximum-index Sorting}
        \label{fig:1b}
    \end{subfigure}
    \hfill
    \begin{subfigure}{0.23\textwidth}
      \centering   
      \includegraphics[width=\linewidth]{fig/3_method/gumbel/gumbel3.pdf}
        \caption{Viterbi Algorithm}
        \label{fig:1c}
    \end{subfigure}
    \hfill
    \begin{subfigure}{0.23\textwidth}
      \centering   
      \includegraphics[width=\linewidth]{fig/3_method/gumbel/gumbel4.pdf}
        \caption{Splitting}
        \label{fig:1d}
    \end{subfigure}
\caption{
\textbf{Visualization} of fine-grained contrastive loss. The upper left figure shows the similarity matrix with Gumbel-Softmax. The other three figures show three kinds of pseudo-labels generation methods respectively: 1) maximum-index sorting; 2) Viterbi algorithm; 3) splitting.
}
\label{fig:gumbel}
\vspace{-1em}
\end{figure}