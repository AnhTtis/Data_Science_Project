\documentclass[twocolumn,superscriptaddress,amsmath,amssymb,aps,sort&compress,prl]{revtex4-2}
\usepackage[utf8]{inputenc}
%\usepackage[columnwise]{lineno}
%\linenumbers
\usepackage{times,color,amsthm,graphics,graphicx,bm,bbm,dcolumn}
\usepackage{epsfig}
\usepackage{graphicx}
\usepackage{siunitx}
\usepackage{xcolor}
\usepackage[colorlinks,urlcolor=blue,citecolor=blue,linkcolor=teal]{hyperref}
%\usepackage{multibib}

\setlength{\belowcaptionskip}{-10pt}
\setcitestyle{super}
\newcommand{\tocheck}[1]{\textcolor{red}{#1}}
\newcommand{\blue}[1]{\textcolor{blue}{#1}}

\begin{document}

\title{Universality of the superfluid Kelvin-Helmholtz instability by single-vortex tracking}

%\date{\today}
\author{D. Hern\'andez-Rajkov}
\email[E-mail: ] {rajkov@lens.unifi.it}
\affiliation{European Laboratory for Nonlinear Spectroscopy (LENS), University of Florence, 50019 Sesto Fiorentino, Italy}
\affiliation{Istituto Nazionale di Ottica del Consiglio Nazionale delle Ricerche (CNR-INO) c/o LENS, 50019 Sesto Fiorentino, Italy}

\author{N.~Grani}
\affiliation{European Laboratory for Nonlinear Spectroscopy (LENS), University of Florence, 50019 Sesto Fiorentino, Italy}
\affiliation{Istituto Nazionale di Ottica del Consiglio Nazionale delle Ricerche (CNR-INO) c/o LENS, 50019 Sesto Fiorentino, Italy}
\affiliation{Department of Physics, University of Florence, 50019 Sesto Fiorentino, Italy}


\author{F.~Scazza}
\affiliation{Department of Physics, University of Trieste, 34127 Trieste, Italy}
\affiliation{European Laboratory for Nonlinear Spectroscopy (LENS), University of Florence, 50019 Sesto Fiorentino, Italy}
\affiliation{Istituto Nazionale di Ottica del Consiglio Nazionale delle Ricerche (CNR-INO) c/o LENS, 50019 Sesto Fiorentino, Italy}

\author{G.~Del~Pace}
\affiliation{Department of Physics, University of Florence, 50019 Sesto Fiorentino, Italy}
%\affiliation{Istituto Nazionale di Ottica del Consiglio Nazionale delle Ricerche (CNR-INO), c/o LENS, via N. Carrara 1 50019 Sesto Fiorentino, Italy}

\author{W.~J.~Kwon}
\affiliation{Department of Physics, Ulsan National Institute of Science and Technology (UNIST), Ulsan 44919, Republic of Korea}

\author{M.~Inguscio}
\affiliation{European Laboratory for Nonlinear Spectroscopy (LENS), University of Florence, 50019 Sesto Fiorentino, Italy}
\affiliation{Istituto Nazionale di Ottica del Consiglio Nazionale delle Ricerche (CNR-INO) c/o LENS, 50019 Sesto Fiorentino, Italy}
\affiliation{Department of Engineering, Campus Bio-Medico University of Rome, 00128 Rome, Italy}

\author{K.~Xhani}
\affiliation{Istituto Nazionale di Ottica del Consiglio Nazionale delle Ricerche (CNR-INO) c/o LENS, 50019 Sesto Fiorentino, Italy}

\author{C.~Fort}
\affiliation{Istituto Nazionale di Ottica del Consiglio Nazionale delle Ricerche (CNR-INO) c/o LENS, 50019 Sesto Fiorentino, Italy}
\affiliation{Department of Physics, University of Florence, 50019 Sesto Fiorentino, Italy}

\author{M.~Modugno}
\affiliation{Department of Physics, University of the Basque Country UPV/EHU, 48080 Bilbao, Spain}
\affiliation{IKERBASQUE, Basque Foundation for Science, 48013 Bilbao, Spain}
\affiliation{EHU Quantum Center, University of the Basque Country UPV/EHU, 48940 Leioa, Biscay, Spain}

\author{F.~Marino}
\affiliation{Istituto Nazionale di Ottica del Consiglio Nazionale delle Ricerche (CNR-INO) c/o LENS, 50019 Sesto Fiorentino, Italy}
\affiliation{Istituto Nazionale di Fisica Nucleare, Sez. di Firenze, 50019 Sesto Fiorentino, Italy}

\author{G.~Roati}
\affiliation{European Laboratory for Nonlinear Spectroscopy (LENS), University of Florence, 50019 Sesto Fiorentino, Italy}
\affiliation{Istituto Nazionale di Ottica del Consiglio Nazionale delle Ricerche (CNR-INO) c/o LENS, 50019 Sesto Fiorentino, Italy}


\begin{abstract}
At the interface between two fluid layers in relative motion, 
infinitesimal fluctuations can be exponentially amplified, inducing vorticity and the breakdown of the laminar flow. This process, known as the Kelvin-Helmholtz instability \cite{helmholtz, kelvin}, is responsible for many familiar phenomena observed in the atmosphere \cite{luce, fukao}, and the oceans \cite{li,smyth}, as well as in astrophysics \cite{McNally2012}, and it is one of the paradigmatic routes to turbulence in fluid mechanics \cite{klaassen, mashayek1, thorpe1, thorpe2}. While in classical hydrodynamics the instability is ruled by universal scaling laws, to what extent universality emerges in quantum fluids is yet to be fully understood. Here, we shed light on this matter by triggering the Kelvin-Helmholtz instability in atomic superfluids across widely different regimes, ranging from weakly-interacting bosonic to strongly-correlated fermionic pair condensates. Upon engineering two counter-rotating flows with tunable relative velocity, we observe how their contact interface develops into an ordered circular array of quantized vortices, which loses stability and rolls up into clusters in close analogy with classical Kelvin-Helmholtz dynamics \cite{thorpe4}. 
We extract the instability growth rates by tracking the position of individual vortices and find that they follow universal scaling relations, predicted by both classical hydrodynamics \cite{drasin, charru} and a microscopic point-vortex model \cite{aref, havelock}. 
Our results connect quantum and classical fluids revealing how the motion of quantized vortices mirrors the interface dynamics and open the way for exploring a wealth of out-of-equilibrium phenomena, from vortex-matter phase transitions \cite{Johnstone2019, Gauthier2019} to the spontaneous emergence of two-dimensional quantum turbulence \cite{barenghi, kobyakov, Neely2013}.
\end{abstract}

\maketitle

A close relationship exists between vortices and shear-flow instabilities in fluid mechanics. In classical hydrodynamics, the interface between two fluid layers in relative motion is identified by an ideal surface containing an infinite number of line vortices, namely a vortex sheet \cite{charru}. More than one century ago, Lord Kelvin and von Helmholtz predicted the dynamical instability of such a vortex sheet \cite{helmholtz, kelvin}. 
The Kelvin-Helmholtz instability (KHI) initially manifests itself as a wave-like deformation of the interface, exponentially growing in time with a rate proportional to the relative velocity between the two fluids (Fig.~\ref{fig:Fig1}a).
It quickly leads to the twisting of the vortex sheet \cite{drasin,charru} and eventually to a turbulent mixing of spiraling structures \cite{thorpe2}.
Starting from the seminal experiments by Reynolds in 1883 \cite{reynolds}, the KHI has been the subject of extensive research and experimentation \cite{thorpe, kent, thorpe3, sullivan, shearer}.

The last few decades have seen exciting advances in the field of superfluidity \cite{novelSF2} -- in particular in ultracold atomic systems \cite{bloch2008} -- owing to our ever-increasing capability to manipulate pristine quantum systems in which vorticity is quantized, and dissipation occurs through different channels from those in ordinary fluids. 
A natural question is whether these key differences affect the onset and the microscopic nature of flow instabilities and whether the universal KHI scaling relations established in classical hydrodynamics is sufficiently general to include superfluids, particularly in the presence of strong interactions.
%
Theoretical investigations have so far been focused mostly on the stability of the interface between distinct sliding fluid components, such as superfluid or normal phases of the same liquid \cite{korshunov,Volovik2002} or immiscible components in binary Bose-Einstein condensates \cite{takeuchi, suzuki, lundh} (BECs). Experimental observations remain limited to the interface between the A and B phases of ${}^3$He in a rotating cryostat \cite{blaauwgeers,finne2006}. All these systems involve phases with distinct physical properties, which complicate the picture due to the presence of additional Rayleigh-Taylor\cite{kobyakov} and counter-superflow instabilities\cite{suzuki}.

\begin{figure}[t!]%[htbp]
\centering
\vspace{0 pt}
\includegraphics[width=1\columnwidth]{Images/Fig1_v6.pdf}
\caption{\textbf{Kelvin-Helmholtz instability in classical and quantum fluids.} \textbf{a}, The interface (horizontal dashed line) separating two counter-propagating classical flows undergoes KHI, developing a wave-like deformation with a characteristic wavenumber $k$. \textbf{b}, In single-component superfluids, the interface modes are replaced by the collective excitations of a one-dimensional vortex lattice. The dotted sinusoidal line illustrates one of these modes, with the vortices displaced with respect to their initial positions, as indicated by the red arrows. In both \textbf{a} and \textbf{b} panels, we display on the right vertical cuts of the velocity profiles along the transverse direction.
\textbf{c}, Left panel: in counter-rotating flows, the KHI arises as a deformation of a circular vortex necklace. The circle of radius $R_0$ (dashed line) interpolates the initial vortex positions. The winding number $m$ of the deformation is related to the azimuthal wavenumber $k$ via $m = k R_0$. As an example, the mode $m=4$ is displayed (sinusoidal dotted curve). The central and right panels display the later stages of vortex pairing and irregular mixing, together with the corresponding vortex trajectories (red curved arrows). In all panels, the background colors indicate the magnitude of the velocity component tangential to the shear layer, 
with the flow directions specified by white arrows.}
\label{fig:Fig1}
\end{figure}


We follow a conceptually simpler, minimal approach to the KHI by creating a shear flow in a single-component superfluid \cite{baggaley,carusotto}. In such a system, the continuous vortex sheet at the layer between the two flows is replaced by an array of quantized vortices\cite{baggaley,zwierlein} (Fig.~\ref{fig:Fig1}b). The tangential flow varies smoothly everywhere except at the position of the vortex cores, resulting in a finite-thickness layer, i.e.~an effective interface. 
%
In a ring-shaped geometry with two counter-rotating flows, the classical circular vortex sheet transforms into a circular array of quantized vortices, or a vortex \emph{necklace}, with radius $R_0$.
While this geometry is equivalent to the planar case for what concerns the instability mechanism, it has the advantage of genuinely imposing periodic boundary conditions, preventing possible fluid accumulation at the system boundaries. 
%
By imaging the individual vortex motion, we show that the vortex necklace is unstable:~infinitesimal fluctuations in the system rapidly break the periodicity and trigger complex dynamics, with nearby vortices rolling up and eventually displaying an irregular, albeit deterministic, motion (Fig.~\ref{fig:Fig1}c).
We experimentally demonstrate that this behaviour corresponds to a \textit{discrete} version of Kelvin-Helmholtz dynamics, where collective excitations of the vortex lattice exhibit universal KHI scalings across different superfluid regimes. 
Our observations establish homogeneous superfluid rings of ultracold fermions as a versatile laboratory for quantum fluid-dynamics experiments. We overcome typical difficulties that complicate experiments with helium quantum fluids \cite{blaauwgeers,finne2006}, especially in terms of precise initial flow control, suppression of the normal fraction, and single-vortex detection.

Our experiment starts with two thin and uniform Fermi superfluids comprising $N_p \simeq 3\times 10^4$ pairs of fermionic $^6$Li atoms (see Methods). Interactions between atoms forming the pairs are encoded in the $s$-wave scattering length $a$. This can be tuned through a broad Feshbach resonance, entering different superfluid regimes ranging from weakly interacting BECs of tightly bound molecules ($1/k_Fa>1$) to strongly correlated unitary Fermi gases (UFG, $1/k_Fa\simeq0$) and Bardeen-Cooper-Schrieffer (BCS, $1/k_Fa < 0$) superfluids. Here, $k_F = (ME_F/\hbar^2)^{1/2}$ is the Fermi wavevector, estimated from the system global Fermi energy $E_F$, and $M$ is the mass of a fermion pair. The superfluids are confined into concentric annular optical traps initially separated by a narrow potential barrier, resulting in two reservoirs with equal density $n_{2D}=\SI{4.96(2)}{\mu m^{-2}}$ (see Fig.~\ref{fig:Fig2}a). Sample temperatures are well below the critical temperature $T_c$ of the superfluid transition, $T=0.3(1)\,T_c$, corresponding to near-unity superfluid fractions in all interaction regimes. The superfluid healing length $\xi$ is always smaller than the vertical cloud size, making the superfluid dynamics three-dimensional. In contrast, vortex dynamics remain two-dimensional since only a few Kelvin modes of vortex lines are expected to be thermally and geometrically accessible at our temperature \cite{supp}. 

We excite persistent flows in each reservoir by optically imprinting a dynamical phase onto the superfluid rings \cite{delpace}. In particular, we drive independent currents by shining oppositely oriented, azimuthal light gradients onto the internal and external rings (see Methods).
With this procedure, we set integer quantized circulations $w_{i,e}$ with corresponding velocity fields $v_{i,e}(r)=\hbar w_{i,e}/(M r)$, where $r$ is the distance from the ring center, and the indices $i,e$ refer to the internal and external rings, respectively.
The relative velocity at the interface equals $\Delta v = \hbar (w_e-w_i)/ (M R_0)$, where $R_0$ is the radius of the circular potential barrier separating the reservoirs. 
In all interaction regimes, we fix $w_e = - w_i$ and only consider velocities $\Delta v_\mathrm{max} < 0.7\,c_s$, where $c_s$ is the measured speed of sound in bulk \cite{supp}.

\begin{figure}[t!]%[htbp]
\centering
\vspace{0 pt}
\includegraphics[width=0.98\columnwidth]{Images/Fig2_v4.pdf}
\caption{\textbf{Flow preparation and Kelvin-Helmholtz dynamics.} \textbf{a}, \textit{In situ} density profile of two concentric counter-rotating superfluids (BEC),
set in relative motion by a phase-imprinting technique while being separated by a potential barrier with FWHM $=\SI{1.4(1)}{\mu m}$. \textbf{b}, Single-shot TOF images in the BEC regime for $\Delta w=8$, as the barrier height $V_0$ is gradually lowered from $V_0\approx 3.5\,\mu$ to 0, where $\mu$ is the superfluid chemical potential (Methods). \textbf{c--e}, Typical single-shot TOF images of the vortex patterns obtained at different times $t\geq0$ after merging the two superfluids in distinct interaction regimes: %and for values of $\Delta w$: 
\textbf{c.} BEC at $1/k_Fa=4.3(1)$ with $\Delta w=6$; \textbf{d}, UFG at $1/k_Fa=0.0(1)$ with $\Delta w=10$; and \textbf{e}, BCS superfluid at $1/k_Fa=-0.3(1)$ with $\Delta w=10$. In all interaction regimes, the vortex necklace destabilizes for $t>0$, and few-vortex clusters form.}
\label{fig:Fig2}
\end{figure}

We merge the two counter-rotating superfluids by gradually lowering the barrier potential. To follow the evolution of the flow, we image
the atomic density profile after a short time-of-flight (TOF) expansion at varying times during the barrier removal.
%
As shown in Fig.~\ref{fig:Fig2}b, as long as the barrier separates the two superfluid rings, a spiral interference pattern is observed, due to the presence of an azimuthal phase gradient between the rings. The number of spiral arms matches the number $\Delta w \equiv|w_e-w_i|$ of $2\pi$ phase-slips at the circular interface \cite{eckel,delpace}. Eventually, when the barrier is completely removed and the two superfluids come into full contact ($t=0$), we observe a necklace of $N_v=\Delta w$ singly-charged, quantized vortices \cite{carusotto,baggaley} with the same circulation sign determined by vorticity\cite{baggaley}. Soon after the vortex necklace emerges, its periodicity breaks down and vortices start pairing in a quasi-synchronous process. As time progresses, metastable clusters of increasingly larger size form, following trajectories reminiscent of the characteristic Kelvin-Helmholtz roll-up dynamics (Fig.~\ref{fig:Fig2}c-e).

\begin{figure*}[ht!]%[htbp]
\centering
\vspace{0 pt}
\includegraphics[width=\textwidth]{Images/Fig3_v5.pdf}\vspace{-5 pt}
\caption{\textbf{Analysis of Kelvin-Helmholtz instability across the BEC-BCS crossover.}
\textbf{a}, Normalized angular structure factor $s(m, t)$ for a UFG superfluid and $\Delta w=12$. Solid red lines represent vertical cuts $s(m)$, corresponding to angular spectra at different times, and the dashed horizontal line indicates the expected most unstable mode. 
\textbf{b}, Time evolution of $s$ for the most unstable mode, $m^*=6$. An exponential fit of the data (line) is used to extract the growth rate $\sigma^*$. Error bars denote the standard error of the mean over $\sim 20$ experimental realizations.
\textbf{c--e}, Normalized dispersion relations, $\sigma_m/\sigma^*$, as a function of $m/\Delta w$ for: \textbf{c}, BEC [$1/k_Fa=4.3(1)$]; \textbf{d}, UFG [$1/k_Fa=0.0(1)$]; and \textbf{e}, BCS [$1/k_Fa=-0.3(1)$] superfluids. Rates are shown for different $\Delta w=8,10,12$, respectively denoted by circles, squares, and triangles. Solid lines show the rates predicted by the PVM in Eq.~(\ref{eq:pvm}) (black line) and by Rayleigh's Eq.~(\ref{eq:RayleighFormula}) using $\delta=0.8\hbar/M\Delta v$ (magenta line).
\textbf{f}, Experimental and theoretical scaling of $\sigma^*$ against $\Delta v^2$ in different interaction regimes. Filled symbols corresponds to experimental data: BCS ($1/k_Fa=-0.5(1)$, red diamonds), ($1/k_Fa=-0.3(1)$, orange diamonds); UFG ($1/k_Fa=0.0(1)$, blue triangles); and BEC ($1/k_Fa=4.3(1)$, dark green stars), ($1/k_Fa=8.3(3)$, light green circles). Open symbols refer to GP simulations at $T=0$ (open green squares) and cZNG simulations at $T/Tc=0.4$ (open red circles) for $1/k_Fa=4.3(1)$. Solid lines refer to  Rayleigh's (magenta) and PVM (black) predictions.
\textbf{g}, Experimental growth rates rescaled by the interaction-dependent constant $\nu$, shown in the inset as a function of $1/k_Fa$. In panels, \textbf{c}--\textbf{g}, vertical error bars denote fitting $1\sigma$-errors, while horizontal ones reflect the experimental uncertainty on the initial vortex number.
}
\label{fig:Fig3}
\end{figure*}


The stability of the vortex necklace can be analyzed using the point vortex model\cite{supp} (PVM), where each vortex is treated as a point particle advected by the velocity field generated by all other vortices. The vortex motion is thus determined by the background flow which, in turn, emerges as a phenomenon associated with the vortex dynamics. The KHI appears as a departure from the initial ordered vortex configuration with characteristic rates given by \cite{aref, havelock}
\begin{equation}\label{eq:pvm}
\sigma_\mathrm{PVM}(k, \Delta v) = \frac{\Gamma k}{2d_v}\left(1-\frac{kd_v}{2\pi}\right),
\end{equation}
where $k$ is the perturbation mode wavenumber, $\Gamma=h/M$ is the quantized vortex circulation, and $d_v=h/(M\Delta v)$ is the initial inter-vortex separation. The maximum growth rate $\sigma^*$, i.e.~the growth rate of the most unstable mode, is reached for $k^*=\pi/d_v$. This particular wavenumber sets a fundamental scaling law, $\sigma^*\propto {\Delta v}^2$, providing a direct hallmark of the instability. The mode $k^*$ is also the largest wavenumber that can be supported in a one-dimensional array of lattice constant $d_v$ (i.e.~a mode with a wavelength equal to $2d_v$). As such, the model can be meaningfully applied to describe our data only for $k \leq k^*$. In the limit of closely packed vortices $d_v\rightarrow 0$, Eq.~(\ref{eq:pvm}) reduces to $\sigma_\mathrm{PVM} \sim \frac{1}{2}k \Delta v$, which equals Kelvin's growth rate of a continuous vortex sheet\cite{charru} (i.e., a zero-width interface). The PVM thus extends the traditional KHI result to the case of a \textit{discrete} vortex sheet formed by a finite number of vortices, providing thus a \emph{microscopic} description of the instability. 
It predicts also a smooth change of the tangential flow between the two layers, except at the position of the vortices, resulting in a finite-size effective interface with half-width $\delta = \hbar/(M \Delta v)$ (Fig.~\ref{fig:Fig1}b, and Ref. \citenum{supp}).  

In classical fluid mechanics, the problem of the stability of a finite-width shear layer was first analyzed by Rayleigh \cite{rayleigh} who derived an interface-dependent growth rate as:
\begin{equation}\label{eq:RayleighFormula}
    \sigma_\mathrm{R}(k, \Delta v) = \mathrm{Im} \frac{\Delta v}{4\delta} \sqrt{(2k\delta-1)^2-e^{-4k\delta}}.
\end{equation}

Here, $\delta$ is the interface width and depends on the fluid's specifics and on the flow shear velocity. According to Eq.~(\ref{eq:RayleighFormula}), the instability only occurs for $k \delta \leq 0.64$, while the system is stable against perturbations with higher wavenumbers \cite{charru}.

In the experiment, we reconstruct the evolution of the vortex necklace by tracking vortex positions over time and characterizing it through the vortex structure factor $S$. This provides a quantitative description of the lattice structure, and its deformation from the initial periodic configuration \cite{Warren}. In our circular geometry, although vortices move both in radial and azimuthal directions, the vortex crystal modes are purely azimuthal (see dotted line in Fig.~\ref{fig:Fig1}c). Therefore, to decompose the motion, it is natural to introduce the angular structure factor (see Methods), defined as $S(m, t)=(1/N_v)\sum_{j,l} \exp[i m(\theta_j(t)-\theta_l(t)]$. Here, $\theta_j(t)$ is the angular position of the $j$-th vortex, and $m$ is the integer winding number of the mode, defined as $m = k R_0$. Figure \ref{fig:Fig3}a displays $s(m, t)=S(m, t)/N_v$ measured for $1/k_Fa=0.0(1)$ and $\Delta w=12$. The spectral peak at $m=\Delta w$, characteristic of a periodic necklace, evolves towards lower angular modes while it simultaneously broadens (see angular spectra plotted in red color). Similar behaviour is found in all interaction regimes. This indicates the breakdown of the necklace structure while an increasing number of modes become populated. We extract the instability rate of each mode $m$ by analyzing the initial time evolution of $s(m, t)$. For most modes, we observe that $s(m,t)$ grows exponentially in time as $s(m,t) \sim e^{2 \sigma_m t}$, where $\sigma_m$ is the growth rate of the $m$-th mode, signaling the onset of a linearly unstable flow \cite{supp, charru}. As an example, in Fig.~\ref{fig:Fig3}b we present the behavior for $m=6$.

In Fig.~\ref{fig:Fig3}c-e, we show the normalized dispersion relations $\sigma_m/\sigma^*$ as a function of $m/\Delta w$ for BEC, UFG, and BCS superfluids. All the measured rates are in good agreement with the expected trends from the PVM [Eq.~(\ref{eq:pvm})] and Rayleigh [Eq.~(\ref{eq:RayleighFormula})] formulas using $\delta = 0.8 \hbar /( M \Delta v)$ (see Ref.~\citenum{supp} for the determination of $\delta$). We see that the instability occurs for $m \lesssim 0.8 \,\Delta w$ matching the value predicted by the Rayleigh formula. The most unstable mode is found at $m^* \sim 0.5\,\Delta w$, as analytically derived from Eq.~(\ref{eq:pvm}) and confirmed by Eq.~(\ref{eq:RayleighFormula}) using the above value for $\delta$.
%
In Fig.~\ref{fig:Fig3}f, the extracted $\sigma^*$ for different superfluid regimes is plotted as a function of the relative velocity. The measured growth rates display a power law increase, $\sigma^*\propto {\Delta v}^{\alpha}$, with fitted exponent $\alpha=1.98(3)$ in agreement with the expected $\alpha=2$ within the experimental uncertainty \cite{supp}. Upon rescaling the fitted slopes with an interaction-dependent factor $\nu$ (see Fig.~\ref{fig:Fig3}g), all data collapse on a single line over more than one decade. These scaling properties together with the normalized dispersion relations in Fig.~\ref{fig:Fig3}c-e provide clear evidence of KHI dynamics in superfluids and of its universality across the BEC-BCS crossover.

The data in Fig.~\ref{fig:Fig3}f are compared with Eqs.~(\ref{eq:pvm}) and (\ref{eq:RayleighFormula}), and with 3D numerical simulations based on Gross-Pitaevskii (GP) equation and collisionless Zaremba-Nikuni-Griffin (cZNG) model\cite{ZNGbook,nick_book}. While theoretical rates -- analytical and numerical -- agree quantitatively with each other, the measured ones show systematically lower values. In classical fluids, dissipation effects (e.g. surface tension, viscosity) typically tend to stabilize the system, leading to smaller growth rates \cite{charru, drasin}. In our system, temperature effects and technical imperfections such as spurious vortices and other density excitations originated from the barrier removal, as well as from a non-ideal imprinting procedure, \cite{delpace, supp} are possible sources of dissipation.
These effects are encapsulated phenomenologically in the $\nu$ parameter shown in the inset of Fig.~\ref{fig:Fig3}g. 
For $1/k_Fa \ge 0$, $\nu$ show a nearly constant value of $0.29(1)$, while BCS superfluids with $1/k_Fa < 0$ display a stronger deviation from the ideal case ($\nu =1$). 
The numerical simulations of the cZNG model in the BEC limit suggest no significant role of the small thermal fraction. On the other hand, in the BCS regime, we expect the combined effects of the thermal component and Andreev vortex-bound states to enter into play, opening a further dissipative channel \cite{Kwon,barresi}. A quantitative understanding of the microscopic mechanisms connecting non-local vortex dynamics to dissipation, especially in presence of strong correlations, remains an open problem \cite{Sonin2015,mehdi}. 

\begin{figure}[t!]%[htbp]
\centering
\includegraphics[width=1\columnwidth]{Images/Fig4_v5_O_trasp.pdf}\vspace{-5pt}
\caption{\textbf{Nonlinear vortex dynamics.} 
\textbf{a}, Single-shot TOF images of vortex patterns acquired after $36$\,ms of evolution, showing $2-$, $3-$, and $4-$fold vortex cluster symmetries. Images are acquired for a BEC superfluid starting from nominally identical necklaces with $N=16$.
\textbf{b}, Time evolution of the mean winding number $\langle m \rangle$ (circles) and its variance $\langle \Delta m^2 \rangle$ (squares) normalized to $\Delta w$ for $1/k_Fa=4.3(1)$ and $\Delta w=12$, averaged over $\sim\,$20 experimental realizations. The data are compared with PVM simulations (black-solid and red-dashed lines).
\textbf{c}, Trajectories of three vortices (colored lines) belonging to nearly-identical necklaces with $N=12$. A set of 40 initial vortex positions are picked randomly, each within a range of one healing length $\xi \approx 0.5\mu$m around their reference value, for which the necklace is perfectly periodic. We show the vortex pattern at $t=0,20,60$ms for one realization.
\textbf{d}, Mean-distance between each-vortex trajectory calculated as a function of time by averaging over $40$ nearly-identical random initial conditions, as described above, and over $N_v=12$ vortices. The distances are normalized to the mean separation of any given two points in the ring geometry (Methods). At short times, $t<20$ ms (shaded area), the separation between trajectories grows exponentially (dashed line) with a characteristic maximal Lyapunov exponent of $\Lambda=0.111(5)\,\text{ms}^{-1}$, in agreement with the maximum growth rate obtained for the KHI with $\Delta w=12$, $\sigma_{PVM}^*=0.127(5)\,\text{ms}^{-1}$. 
}
\label{fig:Fig4}
\end{figure}

After the initial stage of the instability, with characteristic time $1/\sigma^*$, the vortex dynamics enter a nonlinear regime with vortex clusters forming and fragmenting as time progresses while conserving the total vortex number. Starting from nominally identical initial conditions, vortices cluster in arrangements displaying different symmetries, as shown in Fig.~\ref{fig:Fig4}a. 
%
Which of these symmetries appears at a given time hinges on the initial conditions and fluctuations in the system. 
Averaging over different realizations, we observe the system exploring widely different configurations, losing information about the initial order. In particular, the mean winding number $\langle m \rangle$ is nearly constant in time around $\Delta w/2$, while the variance $\langle \Delta m^2 \rangle$ saturates to the vortex number $\Delta w$ (see Fig.~\ref{fig:Fig4}b). This indicates that the system reaches a steady state, where all modes are significantly populated.
PVM simulations starting with nearly identical configurations reproduce quantitatively the experimental data, suggesting that vortices tend to spread over the whole system volume (Fig.~\ref{fig:Fig4}c).

This extreme sensitivity to the initial conditions, where arbitrarily small differences lead to exponential divergence of trajectories, is typically quantified by an average divergence rate known as the maximal Lyapunov exponent \cite{pikovsky2016}, $\Lambda$. We calculate $\Lambda$ by comparing the vortex trajectories of the different PVM simulations, computing their relative separation as a function of time (see Fig.~\ref{fig:Fig4}c-d and Methods). 
Interestingly, we observe that the average separation between same-vortex trajectories grows exponentially, $\sim e^{\Lambda t}$, with a rate $\Lambda$ similar to the maximal KHI growth rate, $\sigma_\mathrm{PVM}^*$ (Fig.~\ref{fig:Fig4}d). While $\sigma^*$ quantifies the collective motion of the necklace, $\Lambda$ refers to the trajectory of a single vortex. 
This connection clarifies the role of quantized vortices in defining the interface dynamics and the KHI as the mechanism behind the necklace breakdown. For $t \gg 1/ \Lambda$, the distance between trajectories tends to saturate to the average separation of two random points in the system (set by the value $1$ in the graph). This is due to the finite-size effects that constrain the trajectories divergence, eventually leading to the attainment of boundary-dominated dynamical equilibrium. 
%
We remark that a positive maximal Lyapunov exponent implies the unpredictability of a deterministic system \cite{pikovsky2016}, and is generally associated with chaotic advection, and turbulent flows \cite{babiano}. Classically, the KHI drives the system into a turbulent state portrayed by an irregular -- sensitive to initial conditions -- mixing of spiraling structures at different scales \cite{thorpe2}. The intertwined vortex trajectories shown in Fig.~\ref{fig:Fig4}c are reminiscent of this scenario.

%--------------------------------------------------
%------------------ Conclusions -------------------
%--------------------------------------------------

Our observation of the superfluid Kelvin-Helmholtz instability showcases a pristine example of an emergent phenomenon in which quantized vortices play the role of elementary constituents: the instability arises from the coherent motion of vortices, here acting simultaneously as sources and probes of the unstable flow. 
The same microscopic mechanism operates in all superfluid regimes, and it underlies the observed universal behaviour, which interestingly belongs to the class of classical inviscid fluids with finite-size shear layers. 
%
We anticipate our results to be of relevance for diverse non-equilibrium phenomena in strongly correlated quantum matter, ranging from rapidly rotating quantum gases \cite{zwierlein} to pulsar glitches \cite{Haskell2015} and neutron star mergers \cite{Price2006}. Our findings also set the starting point to explore a variety of vortex matter phase-transitions in fermionic superfluids \cite{Sachkou2019},  including negative temperature and non-trivial cluster states \cite{Simula, Johnstone2019, Gauthier2019, Reeves2022}, even in presence of dissipative mechanisms resulting from vortex-vortex and vortex-quasiparticles interactions \cite{Heyl,mehdi}.
An exciting direction for future experiments concerns the cascade of secondary instabilities towards the spontaneous onset of quantum turbulence \cite{baggaley, kobyakov, finne2006}, exploring a route complementary to external forcing \cite{Henn2009, Navon2016} to probe its underlying microscopic mechanisms from the few- to the many-vortex perspective.

%--------------------------------------------------
%--------------------- Methods --------------------
%-------------------------------------------------
\section{Acknowledgements}
We thank Iacopo Carusotto, Nigel Cooper, and Giovanni Modugno for valuable comments on the manuscript, and the Quantum Gases group at LENS for fruitful discussions. This work was supported by the European Research Council under Grant Agreement No.~307032, the Italian Ministry of University and Research under the PRIN2017 project CEnTraL, and European Union’s Horizon 2020 research and innovation program under the Qombs project FET Flagship on Quantum Technologies Grant Agreement No.~820419 and Marie Skłodowska-Curie Grant Agreement No.~843303, and the Research Fund (1.220137.01) of UNIST (Ulsan National Institute of Science and Technology). Acknowledge financial support from: PNRR MUR project PE0000023-NQSTI M.M.~acknowledges support from Grant No.~PID2021-126273NB-I00 funded by MCIN/AEI/10.13039/501100011033 and ``ERDF - A way of making Europe'', and from the Basque Government through Grant No.~IT1470-22. 

%--------------------------------------------------
%--------------------- Methods --------------------
%--------------------------------------------------

\section{Methods}

\subsection{Sample preparation}
We prepare fermionic superfluid samples by evaporating a balanced mixture of the two lowest hyperfine spin states $|F,m_F\rangle= |1/2,\pm1/2\rangle$ of $^6$Li, near their scattering Feshbach resonance at $\SI{832}{G}$ in an elongated, elliptic optical dipole trap, formed by horizontally crossing two infrared beams at a $14^{\circ}$ angle. At the end of the evaporation, we sweep the magnetic field to the desired interaction regime. A repulsive $\mathrm{TEM_{01}}$-like optical potential at $\SI{532}{nm}$ with a short waist of about $\SI{13}{\mu m}$ is then adiabatically ramped up before the end of the evaporation to provide strong vertical confinement, $\omega_z\simeq 2\pi\times400$ Hz. 
Successively, a box-like potential is turned on to trap the resulting sample in a circular region of the $x$–$y$ plane. This circular box is tailored using a Digital Micromirror Device (DMD). When both potentials have reached their final configuration, the infrared lasers forming the crossed dipole trap are adiabatically extinguished, completing the transfer into the final uniform pancake trap\cite{Kwon}. Finally, to create the pair of superfluid rings at rest we dynamically change the DMD-tailored potential. 
We first create the hole at the center of the initial disk and then dynamically increase its size until reaching a radius of $R_i=\SI{10.0\pm 0.2}{\mu m}$.
Finally, an optical barrier separating the two superfluid rings is adiabatically raised at $R_0=\SI{27.5\pm 0.2}{\mu m}$. A residual radial harmonic potential of $\SI{2.5}{Hz}$ is present due to the combined effect of an anti-confinement provided by the $\mathrm{TEM_{01}}$ laser beam in the horizontal plane and the confining curvature of the magnetic field used to tune the Feshbach resonance. This weak confinement has a negligible effect on the sample over the $R_e=\SI{45.0\pm 0.2}{\mu m}$ radius of our box trap, resulting in an essentially homogeneous density.

\vspace{-10pt}
\subsection{Phase-imprinting procedure}
We excite controllable persistent current states in each of the two rings by using the phase imprinting protocol described in Ref.~\citenum{delpace}. Using the DMD we create an optical gradient along the azimuthal direction, namely $U(r, \theta) = U_0 \,\theta/2\pi \times \mathrm{sign}(r-R_0)$. By projecting such a potential over a time $t_I<\hbar /\mu$, we imprint a phase $\phi(r,\theta)= U(r,\theta) \,t_I/h$ onto the superfluid wavefunction. By suitably tuning the imprinting time $t_I$ and the gradient intensity $U_0$, we excite well-defined winding number states in each of the two rings in a reproducible way. We measure the imprinted circulations using an interferometric probe: we let the two rings expand for $3$ ms of time of flight (TOF) and then image the resulting spiral-shaped interference pattern. The first panel of Fig. \ref{fig:Fig2}b shows such an interferogram for $\Delta w = 8$. In particular, the number of spirals in the interferogram yields the relative winding number $\Delta w$ between the two rings \cite{eckel,delpace}. Additionally, we independently check that before the imprinting procedure, the inner ring is in the $w=0$ state by realizing a similar experimental protocol now in a geometry similar to the reported in Ref. \citenum{delpace}. All circulation states excited in the two rings have been observed to persist for several hundreds of ms \cite{delpace}, except for $\Delta w > \mathrm{12}$ in the BCS regimes. Nevertheless, we observe these states to not decay for the typical timescale of the observed KHI $t <\SI{40}{ms}$. To reduce the effect of extra density excitation in the system on the KHI dynamics, we wait $\SI{300}{ms}$ after imprinting before removing the barrier between the superfluids.

\subsection{Vortex imaging and tracking}
We trigger the KHI dynamics by removing the circular barrier separating the two ring superfluids. In particular, we lower down its intensity by opportunely changing the DMD pattern. The barrier removal process takes 28 ms and brings the system into the vortex necklace configuration of Fig \ref{fig:Fig2}. We confirm that the duration of the barrier removal does not affect significantly the dynamics. Removing the barrier over time scales faster than 10\,ms creates unwanted excitations such as solitonic structures.
To image the vortices in the BEC regime, we acquire the TOF image of the superfluid density, where vortices appear as clear holes. In particular, we abruptly switch off the vertical confinement and at the same time, we start to ramp down the DMD potential, removing it completely in $\SI{1}{ms}$.
Then, we let the system evolve further for $\SI{2.2}{ms}$ of TOF and then acquire the absorption image. This modified TOF method allows for to maximization of the vortex visibility. However, the small condensed fraction in the strongly-interacting regime makes it impossible to detect vortices with this simple method.
Therefore, in the UFG and BCS regimes, we employ the technique developed in Ref. \citenum{Kwon}: we add a linear magnetic field ramp of $4-5$ ms to $\SI{700}{G}$ before the imaging to map the system in a BEC superfluid.

The position of the vortices is tracked manually in each acquired image. The error on the position of the vortex is limited by the size of the vortex after the TOF sequence. To estimate it we perform a Gaussian fit of the vortex density hole and obtain a waist of $\sim 1.0-\SI{1.4}{\mu m}$ for all interaction regimes.

\vspace{-12pt}
\subsection{Point Vortex Model (PVM)}
We consider a two-dimensional fluid containing N point vortices with quantized circulations $\Gamma=h/M$. When the inter-vortex separation is greater than a few healing lengths, vortices are advected by the velocity field created by other vortices. The equation of motion of each vortex is $d\vec{r}_i/dt = \vec{v_i}^0$, where $\vec{v_i}^0$ is the velocity field created by all the other vortices. 
If we consider a 1D array of equispaced vortices at coordinates $(x_n,y_n)=(d_v/2+n d_v, 0)$, moving in a 2D space of coordinates $(x,y)$ without boundaries, the tangential velocity of the superfluid flow can be written as: 
\begin{equation*}
    v_{x}(x, y) = -\frac{\Gamma}{2d_v} \frac{\sinh \left(2\pi y/d_v\right)}{\cosh \left(2\pi y/d_v\right)+\cos \left(2\pi x/d_v\right)}.
\end{equation*}
From this relation, the finite-size width of the effective interface  is naturally expressed in units of $\delta=d_v/(2\pi)$, or equivalently $\delta=\hbar/(M\Delta v)$.

When considering the ring geometry, $\vec{v_i}^0$ must take into account the boundary conditions, namely that the flow must have a zero radial component at both the internal ($R_i$) and external ($R_e$) radii. 
We include the boundary conditions by using the method of image vortices \cite{supp}, and solve the equation of motions for the vortex necklace configuration in the ring with the Runge-Kutta method of fourth order. From the obtained trajectories of the vortices, $\vec{r}_i (t)$ we compute the normalized angular structure factor $s (m, t)$.

\subsection{Angular structure factor analysis}
At $t=0$, the one-dimensional angular structure factor of a finite array of $N$ vortices placed in a perfect necklace arrangement, with angular coordinates $\theta_j^0 = 2\pi j/N$, is $S^0(m)=\sin^2(\pi m)/(N\sin^2(\pi m /N))$. The departure from the necklace configuration can be modeled through the small fluctuations in the vortex positions at $t=0$: $\theta_j = \theta_j^0 + \delta \theta$. In crystals, small fluctuations ($\delta \theta \ll 2\pi/N$) are considered as disorder of the first kind \cite{Warren}, and they modify the structure factor as $S(m,t) \approx S^0(m)-m^2\langle \delta \theta^2 \rangle(t)\,S^0(m)$. Here, the temporal dependence of $S(m,t)$ is entirely provided by the term $\langle \delta \theta^2 \rangle(t)$. In the context of the PVM \cite{aref, havelock}, the motion of the vortices is linked to the KHI growth rate. In particular, the deviation from their initial position grows as $\delta \theta\sim e^{\sigma_m t}$, where $\sigma_m$ is given by Eq.~(\ref{eq:pvm}). Therefore, the temporal evolution of the structure factor is $S(m,t) \sim e^{2 \sigma_m t}$.

\vspace{-5pt}
\subsection{Maximum growth rate $\sigma^{*}$}
To obtain experimentally the maximum growth rate $\sigma^*$, we fit the dispersion relation of the measured rates, Fig. \ref{fig:Fig2} c-e, using the following function $f(x, \sigma^*) = \sigma^* \frac{\sqrt{e^{-4\eta x}-(2\eta x-1)^2}}{A}$, with $x=m/\Delta w$, and $A=\max\left[\sqrt{e^{-4\eta x}-(2\eta x-1)^2}\right]=(W(e^{-1}) + 1)/(2\eta)\approx 0.639/\eta$, where $W(x)$ is the Lambert W-function and $\eta=0.8$ (see Supplementary Information\cite{supp} for details). The function $f(x,1)$ corresponds to Eq.~\eqref{eq:RayleighFormula} normalized to the maximum value shown as the magenta line in Fig.~\ref{fig:Fig2}c-e. We perform the fit of the dispersion relation letting $\sigma^*$ as the only free parameter.

\subsection{Maximal Lyapunov exponent}
To extract the Lyapunov exponent of the system we perform 40 PVM simulations under nearly-identical conditions for a necklace with $N = 12$. The initial positions of each vortex are taken randomly within a range of one healing length ($0.5\,\mu$m) around their reference values for a perfectly periodic necklace. We then define the function $\mathcal{L}_k = \langle |\vec{r_i}^k-\vec{r_j}^k| \rangle_{i,j}$, as the average distance between two simulated trajectories of the $k$-th vortex in the necklace. Here, we denote by  $\left\langle \cdot \right \rangle_{i,j}$ the average over different simulations, i.e., $i,j=0,...,40$. Then, we compute the average over the $N$ vortices $\langle \Delta \vec{x} \rangle = \left\langle \mathcal{L}_k \right \rangle_{k}$, which we report in Fig.~\ref{fig:Fig4}d after normalizing it to 
%The values of $\langle \Delta \vec{x} \rangle$ reported in Fig. \ref{fig:Fig4} are normalized to 
the mean separation of any two points in the ring geometry $\mathrm{\bar{d}} = \int_{\Omega}\int_{\Omega}\sqrt{(x-x')^2+(y-y')^2} \,dx\,dy\,dx'dy' / A^2$, where $\Omega$ is the ring region with area $A=\pi (R_e^2-R_i^2)$. Although $\mathrm{\bar{d}}$ is straightforward to write, computing the integrals to obtain an analytical result for it is quite involved. For this reason, we numerically evaluate it by taking $10^5$ random points uniformly distributed inside the ring geometry delimited by $R_i$ and $R_e$. Then, we compute the $10^{10}$ possible combinations for the point-to-point distances and calculate their average value to estimate the mean separation %of two points to be 
$\mathrm{\bar{d}} \approx \SI{41.78\pm 0.02}{\mu m}$. 
We extract the characteristic rate $\Lambda$ from a fit of $\langle \Delta \vec{x}\, \rangle$ initial trend over the first \SI{14}{ms} from the starting time of the instability.


%--------------------------------------------------
%------------------ Bibliography ------------------
%--------------------------------------------------
% This must be in the first 5 lines to tell arXiv to use pdfLaTeX, which is strongly recommended.
\pdfoutput=1
% In particular, the hyperref package requires pdfLaTeX in order to break URLs across lines.

\documentclass[11pt]{article}

% Remove the "review" option to generate the final version.
%\usepackage[review]{ACL2023}
\usepackage{ACL2023}

% Standard package includes
\usepackage{times}
\usepackage{latexsym}

% For proper rendering and hyphenation of words containing Latin characters (including in bib files)
\usepackage[T1]{fontenc}
% For Vietnamese characters
% \usepackage[T5]{fontenc}
% See https://www.latex-project.org/help/documentation/encguide.pdf for other character sets

% This assumes your files are encoded as UTF8
\usepackage[utf8]{inputenc}

% This is not strictly necessary, and may be commented out.
% However, it will improve the layout of the manuscript,
% and will typically save some space.
\usepackage{microtype}

% This is also not strictly necessary, and may be commented out.
% However, it will improve the aesthetics of text in
% the typewriter font.
\usepackage{inconsolata}


% If the title and author information does not fit in the area allocated, uncomment the following
%
%\setlength\titlebox{10cm}
%
% and set <dim> to something 5cm or larger.

%%%%%%%%%%%%%%%%%%%%%%%%%%%%%%%%%%
\usepackage{graphicx}
\usepackage{amsfonts}
\usepackage{amsmath}
\usepackage{bigdelim}
\usepackage{diagbox}
\usepackage{amsthm}
\usepackage{makecell}
\usepackage{mathtools}
\usepackage{booktabs}
\usepackage[shortlabels]{enumitem}
\graphicspath{ {figs/} }

\theoremstyle{remark}
\newtheorem*{question}{Question}

\newcommand{\tk}[1]{\textcolor{blue}{{#1}}}
\newcommand{\sy}[1]{\textcolor{red}{{#1}}}
\newcommand{\mg}[1]{\textcolor{purple}{{#1}}}
\newcommand{\lh}[1]{\textcolor{green}{{#1}}}
\newcommand{\lc}[1]{\textcolor{green}{{#1}}}

% Rounded color box
\definecolor{light_blue}{HTML}{cfdfff}
\usepackage[most]{tcolorbox}
\tcbset{on line, 
        boxsep=1pt, left=0pt,right=0pt,top=0pt,bottom=0pt,
        colframe=white,colback=light_blue,  
        highlight math style={enhanced}
        }

\newcommand{\quash}[1]{}  %Anything in \quash is ignored
\newcommand{\gpt}{\textsc{GPT-2}}
\newcommand{\bert}{\textsc{BERT}}
\newcommand{\bertlarge}{\textsc{BERT-large}}
\newcommand{\mask}{\texttt{[MASK]}}
\newcommand{\cls}{\texttt{[CLS]}}
\newcommand{\sep}{\texttt{[SEP]}}
\newcommand{\mat}{\texttt{mat}}
\newcommand{\id}{\texttt{id}}
\newcommand{\matl}{\texttt{mat}_{\ell \rightarrow \ell'}}
\newcommand{\matattnl}{\texttt{mat\_attn}_{\ell \rightarrow \ell'}}
\newcommand{\matffl}{\texttt{mat\_ffn}_{\ell \rightarrow \ell'}}
\newcommand{\matlnl}{\texttt{mat\_ln1\_ln2}_{\ell \rightarrow \ell'}}
\newcommand{\idl}{\texttt{id}_{\ell \rightarrow \ell'}}
\newcommand{\matlL}{\texttt{mat}_{\ell \rightarrow L}}
\newcommand{\matattnlL}{\texttt{mat\_attn}_{\ell \rightarrow L}}
\newcommand{\matfflL}{\texttt{mat\_ffn}_{\ell \rightarrow L}}
\newcommand{\matlnlL}{\texttt{mat\_ln1\_ln2}_{\ell \rightarrow L}}
\newcommand{\idlL}{\texttt{id}_{\ell \rightarrow L}}

\definecolor{blue(munsell)}{rgb}{0.0, 0.5, 0.69}
%%%%%%%%%%%%%%%%%%%%%%%%%%%%%%%%%%

\title{Jump to Conclusions: Short-Cutting Transformers\\With Linear Transformations}

% Author information can be set in various styles:
% For several authors from the same institution:
% \author{Author 1 \and ... \and Author n \\
%         Address line \\ ... \\ Address line}
% if the names do not fit well on one line use
%         Author 1 \\ {\bf Author 2} \\ ... \\ {\bf Author n} \\
% For authors from different institutions:
% \author{Author 1 \\ Address line \\  ... \\ Address line
%         \And  ... \And
%         Author n \\ Address line \\ ... \\ Address line}
% To start a seperate ``row'' of authors use \AND, as in
% \author{Author 1 \\ Address line \\  ... \\ Address line
%         \AND
%         Author 2 \\ Address line \\ ... \\ Address line \And
%         Author 3 \\ Address line \\ ... \\ Address line}

\author{Alexander Yom Din$^{1}$ ~~~~~ Taelin Karidi$^{1}$ ~~~~~ Leshem Choshen$^{1}$ ~~~~~
Mor Geva$^{2}$ 
\vspace{0.2cm} \\
$^1$Hebrew University of Jerusalem ~~~ $^2$Google Research \\
\small{\texttt{\{alexander.yomdin, taelin.karidi, leshem.choshen\}@mail.huji.ac.il}}, \small{\texttt{pipek@google.com}}}

\quash{
\author{Alexander Yom Din \\
  Hebrew University of Jerusalem \\ \texttt{alexander.yomdin@mail.huji.ac.il} \\\And
  Taelin Karidi \\
  Hebrew University of Jerusalem \\
  \texttt{taelin.karidi@mail.huji.ac.il} \\\And
  Leshem Choshen \\
  Hebrew University of Jerusalem \\ \texttt{leshem.choshen@mail.huji.ac.il} \\\And
  Mor Geva \\
  Google Research \\
  \texttt{pipek@google.com} \\}
}

\begin{document}
\maketitle



\begin{abstract}
% \vspace{-1em}
The diffusion-based generative models have achieved remarkable success in text-based image generation. However, since it contains enormous randomness in generation progress, it is still challenging to apply such models for real-world visual content editing, especially in videos. 
In this paper, we propose \texttt{FateZero}, a zero-shot text-based editing method on real-world videos without per-prompt training or use-specific mask. 
\RM{Specifically, different from a pipeline of two independent inversion and then generation stages, we find the intermediate attention maps during inversions store better structure and motion information. We thus reform them to temporally casual attention and replace them in the generation progress. To further reduce the unnecessary semantic leakage of source video and enhance the editing quality, we then remix the temporally casual attentions via the cross-attention features of the source prompt as the mask.}
To edit videos consistently, we propose several techniques based on the pre-trained models. Firstly, in contrast to the straightforward DDIM inversion technique, our approach captures intermediate attention maps during inversion, which effectively retain both structural and motion information. These maps are directly fused in the editing process rather than generated during denoising. To further minimize semantic leakage of the source video, we then fuse self-attentions with a blending mask obtained by cross-attention features from the source prompt. Furthermore, we have implemented a reform of the self-attention mechanism in denoising UNet by introducing spatial-temporal attention to ensure frame consistency.
Yet succinct, our method is the first one to show the ability of zero-shot text-driven video style and local attribute editing from the trained text-to-image model. We also have a better zero-shot shape-aware editing ability based on the text-to-video model~\cite{tuneavideo}. \RM{Besides video, our unified method also achieves state-of-the-art performance in zero-shot image editing.\chenyang{Need exp or remove the zero-shot image}} Extensive experiments demonstrate our superior temporal consistency and editing capability than previous works.
% The code will be released.
% \chenyang{emphasize: our observation at inversion time} \xiaodong{replacing the bold part to the actual pipeline: \textbf{Specifically, we work on replacing and mixing the attention maps between the inversion and generation since the self-attention map keeps the structure of the original natural image and the cross-attention is semantic-related, after remixing, we replace them in the corresponding generation steps for denoising.}}
% \footnote{Since there is no general video diffusion model is publicly available, we use one-shot video generation method~(Tune-A-Video~\cite{tuneavideo}) as the pretrained video diffusion model for zero-shot video editing\xiaodong{can be removed if we actually zero-shot on video}.}.
\end{abstract}
\section{Introduction}

The ability to reason about plans is critical for performing long-horizon tasks \citep{erol1996hierarchical, sohn2018hierarchical, sharma-etal-2022-skill}, compositional generalization \citep{corona-etal-2021-modular} and generalization to unseen tasks and environments \citep{shridhar2020alfred}.
Consider a simple long-horizon planning scenario where a robot is tasked with preparing a meal and serving it on the table. 
This presents a non-trivial planning problem since the agent needs to understand the sequence of operations required to perform the task and search for the relevant objects in the unfamiliar environment by interacting with various objects. %



Large language models have been recently shown to possess commonsense knowledge about the world such as object affordances and physical dynamics \citep{ouyang2022training,chowdhery2022palm}.
Early approaches considered text based environments and fine-tuned PLMs to predict actions given the history of past observations and actions \citep{jansen-2020-visually,micheli-fleuret-2021-language,yao-etal-2020-keep}.
Recent work has used this ability to reason about plans from text instructions in simulated household environments with simplifying assumptions such as text-only environment observations or feedback \citep{huang2022language,ahn2022can,li2022pre,logeswaran-etal-2022-shot}.


We focus on \emph{visually grounded planning} with PLMs --- the ability to adapt plans based on interaction and visual feedback from the environment.
While PLMs have strong planning commonsense priors, predictions from a PLM may not be directly realizable in the environment since the observation and action spaces are unknown.
This requires \emph{grounding} the PLM in the environment and adapting it to observe visual feedback, which is highly non-trivial.
Some prior works assume the availability of a pre-trained affordance function \citep{ahn2022can} or a success detector \citep{mirchandani2021ella}.
Notably, SayCan \citep{ahn2022can} completely decouples the PLM from observation information by selecting actions that have both high affordability (through a pre-trained affordance model) and high PLM likelihood.
Although this partially addresses the grounding problem, the use of visual feedback for action affordance alone is limited.
Often an agent must choose one of many affordable actions using information from observations.
For example, a driving agent should re-navigate and possibly turn around when encountering a ``road closed'' sign, but both turning around and driving forward are indistinguishable to SayCan because they are both affordable and the PLM is blind to observations.

Another workaround explored in prior work is translating the information in the visual observations to text using a pre-trained captioning system \citep{shridhar2021alfworld,huang2022language}.
However, it can be difficult to faithfully describe an image in words and information is lost in this inherently noisy process, which limits the information available to the planner.



Recent work shows that PLMs can be adapted for various natural language tasks by inserting tunable embeddings or soft prompts at the input of the PLM (also called prompt tuning or prefix tuning)~\citep{li-liang-2021-prefix,lester-etal-2021-power}.
This approach also extends to multi-modal understanding tasks such as image captioning \citep{mokady2021clipcap} and VQA \citep{tsimpoukelli2021multimodal} where images are encoded as soft prompts and finetuned for the target task.
Transformer based architectures have also been successfully applied to offline Reinforcement Learning in recent work \citep{chen2021decision,janner2021offline,li2022pre,reid2022can}.

Taking inspiration from these works, we propose the simple approach of embedding visual observations (`visual prompts') and \textit{directly inserting them as PLM input embeddings}.
The visual encoder and PLM are jointly trained for the target task, an approach we call \textbf{\oursfull}~(\ours).
By teaching the PLM to use observations for planning in an end to end manner, we remove the dependency on external data such as captions and affordability information that was used in prior work.
We show that this simple approach performs better than prior PLM-based planning approaches on two embodied planning benchmarks based on ALFWorld~\citep{shridhar2021alfworld} and Virtualhome~\cite{puig2018virtualhome}.



\section{Related Work}

%Here we summarize prior work on transfer learning and property inference.

%\shortsection{Transfer Learning}
%%Transfer learning reuses features learned by pre-trained models for new tasks, with the pretext that inherent similarities in the generic features will be useful for the downstream tasks and hence reducing their cost of downstream training. Specifically, the downstream model trainer will use a pre-trained upstream model as the starting point for the downstream training, with inclusion of (or replacement with) the task-specific classification layer/module. The downstream model is then trained by either updating all layers of the model (including ones reused from upstream model) or freezing some earlier layers of the reused parts as the ``feature extractor'' and only updating the rest. The latter approach is more popular as the reused feature extractors can already learn useful feature representations and the training cost is also much lower and affordable for individuals with limited computational resources. We study the vulnerability of the latter transfer learning approach in this paper. 


%\shortsection{Transfer Learning} 
Several works have demonstrated risks associated with transfer learning across a variety of attack goals. Wang et al.~\cite{wang2018great} and Yao et al.~\cite{yao2019latent} consider manipulating the upstream model such that the fine-tuned downstream models contain backdoors, misclassifying test inputs that contain predefined backdoor triggers. These transfer manipulations are tailored to their particular attack goals and cannot be applied for the property inference goal considered in this paper. Zou et al.~\cite{zou2020privacy} study the threat of membership inference attacks on transfer learning, but with normally trained upstream models.  
%\dnote{its clear that the goals are different for these attacks, but how similar are the methods?} \ynote{similarity of the methods? more details about the methods? do not know what is expected here}
%In contrast, we investigate the possibility of boosting the effectiveness of property inference by manipulating the upstream model training. % Schuster et al.~\cite{schuster2020humpty} show that the attacker can modify the corpus on which the word embedding is trained such that the downstream NLP models which use that embedding will behave abnormally.

%\shortsection{Property Inference}
The risk of property inference was introduced by Ateniese et al.~\cite{ateniese2015hacking}, % introduces the threat of inferring properties of the training data from pre-trained models, 
and several subsequent works have developed property inference (also known as distribution inference) attacks~\cite{Wang2022GroupPI, suri2022formalizing, Jurez2022BlackBoxAF, Hartmann2022DistributionIR}.
% Ganju et al.~\cite{ganju2018property} and Suri and Evans~\cite{suri2022formalizing} 
These works study property inference against normally trained models, and they launch attacks using a variety of black-box and white-box attacks. All the white-box attacks use meta-classifiers, which take the permutation-invariant representation~\cite{ganju2018property} of the model parameters as the features. We use the state-of-the-art white-box attack~\cite{suri2022formalizing} in our experiments.
%We will use the state-of-the-art white-box method proposed by Ganju et al.~\cite{ganju2018property} and later extended by suri et al.~\cite{suri2022formalizing} in this paper.
%\dnote{do we use these attacks?} 
Melis et al.~\cite{melis2019exploiting} and Zhang et al.~\cite{zhang2021leakage} focus on property inference in distributed training scenarios. In their settings, the attacker is a participant in the global model training and conducts property inference using meta-classifiers that are trained on model outputs or gradients. Similarly, Suri et al.~\cite{suri2022subject} focus on federated learning settings where the attacker is a participant (or the central server) that utilizes black-box attacks for inferring membership of data from particular subjects. %\dnote{if we use black-box attacks, explain which ones, or how ours are related to previous ones} 
For our experiments, We improve the black-box meta-classifier proposed by Zhang et al.~\cite{zhang2021leakage} using the ``query tuning'' technique in Xu et al.~\cite{xu2019detecting}. 

The closest works to ours are Chase et al.~\cite{saeed} and Chaudhari et al.~\cite{Chaudhari2022SNAPEE}, which both consider a scenario where the attacker can manipulate some of the training data of the model to induce a model that significantly increases property inference risk.
% \dnote{it enables precise property inference attacks?}.
These works assume an adversary with the ability to poison the victim's training data, while the adversary in our scenario has no access to the victim's training data, and therefore, their methods are not applicable.
% \dnote{example how different from ours, and why the methods are not applicable}
%Thus, their methods are not applicable to our transfer learning scenario.
%Their methods rely on inducing certain behavior correlated with the properties to be inferred, and thus are not applicable to our transfer learning scenario. \anote{Still a bit unclear why that is the case.}
%
There are also works similar to ours that leverage ``adversarial initializations'' for attack purposes.
% \cite{grosse2019adversarial, boenisch2021curious, wen2022fishing, fowl2021robbing}.
Grosse et al.~\cite{grosse2019adversarial} focus on scenarios where the attacker can control the parameter initialization of a model, and demonstrate that the attacker can use special initializations to damage the performance of the trained model. %This attack is orthogonal to ours.
Other works \cite{boenisch2021curious, wen2022fishing, fowl2021robbing} show that the malicious central server in a federated learning protocol can reconstruct some training samples via falsifying the global model in some training rounds and then analyzing the submitted gradients. These kinds of attacks do not apply to our transfer-learning scenario since the attacker cannot access the downstream gradients, and can only manipulate the upstream training.

\iffalse %%%%%%%%%%%%%%%%%%%%%%%%%%%%%%%%

In this section, we provide the background and also the summary of prior attacks on transfer learning (Section~\ref{sec:transfer_learning}) and property inference (Section~\ref{sec:property_inference}). Then, we introduce the closely related manipulation attacks against machine learning models to boost different privacy risks in Section~\ref{sec:active_inference_attacks}.

%\anote{Do we really need a dedicated section for this? It's barely 2 paragraphs right now.}

%\dnote{the most closely related work to ours are works that attempt to amplify inference attacks by poisoning models, the two most relevant I know of are \url{https://www.computer.org/csdl/proceedings-article/sp/2022/131600b569/1CIO8nmuota} and \url{https://arxiv.org/abs/2204.00032}, but need to look thoroughly for others. We should definitely be describing this and relating it to our work, probably in the introduction. Most of what is here is Background, but should be clear what this section is for (not muddling background and related work)}

\subsection{Transfer Learning} \label{sec:transfer_learning}
Transfer learning reuses features learned by pre-trained models for new tasks, with the pretext that inherent similarities in generic features can be useful for downstream tasks, thus reducing the cost of downstream training. Specifically, the downstream model trainer uses a pre-trained upstream model as the starting point for downstream training, with the inclusion (or replacement) of task-specific classification layers/modules. The downstream model is then trained by either updating all layers of the model (including ones reused from the upstream model) or freezing some earlier layers of the reused parts as the ``feature extractor'' and only updating the rest. The latter approach is more popular as the reused feature extractors can already learn useful feature representations and the training cost is also much lower and affordable for individuals with limited computational resources. We study the vulnerability of the latter transfer learning approach in this paper. 
%mainly in two ways:  1) all the layers (including ones reused from ) and tune the full model; the other one is to freeze some earlier layers of the model as the feature extractor and only tune the rest later layers. The second update strategy could achieve better efficiency since the frozen layers can already produce meaningful feature representations~\cite{wang2018great,yao2019latent}, and we will study the transfer learning using this strategy. 

Recently, various attacks have been proposed for the transfer learning setting, but with different attack goals from ours. Wang et al.~\cite{wang2018great} generate adversarial examples against black-box student models that transfer knowledge from publicly available teacher models without repeated queries. Yao et al.~\cite{yao2019latent} propose to manipulate the upstream model such that the downstream models derived from the upstream model contain backdoors, which would misclassify test inputs that contain some predefined backdoor triggers. Zou et al.~\cite{zou2020privacy} study the threat of membership inference attacks on transfer learning and the upstream models are trained normally. In contrast, we investigate the possibility of boosting the effectiveness of property inference by manipulating the upstream model training. Schuster et al.~\cite{schuster2020humpty} show that the attacker can modify the corpus on which the word embedding is trained such that the downstream NLP models which use that embedding will behave abnormally.

%This additionally allows model trainers to achieve satisfactory performance with limited training samples, leading to reduced computational costs. The most common approach reuses parameters in the earlier layers of the pre-trained model, either by fixing them as the feature extractor or just using them for initialization, to conduct downstream training.

\subsection{Property Inference} \label{sec:property_inference}

\shortsection{Property Inference Attacks} In property inference attacks, the adversary aims to infer some sensitive properties of some data, given a model trained on it. For example, the adversary may be interested in sensitive properties like the presence of people of a specific race in the dataset~\cite{ateniese2015hacking, melis2019exploiting}), or even be curious about the 
the statistics of the training set (e.g, the ratio of people with a specific gender~\cite{saeed, ganju2018property, suri2022formalizing, zhang2021leakage}).


Ateniese et al.~\cite{ateniese2015hacking} were the first to identify the threat of inferring properties of the training data from pre-trained models. Ganju et al.~\cite{ganju2018property} and Suri and Evans~\cite{suri2022formalizing} 
study property inference against normally trained models, and they launch attacks using white-box meta-classifiers, which utilize the permutation-invariance representation~\cite{ganju2018property} of the model parameters, while other works focus on distributed training~\cite{zhang2021leakage} where the attacker is a participant in the global model training and conducts property inference using meta-classifiers trained on model outputs. Similarly, Suri et al.~\cite{suri2022subject} focus on federated learning, where the attacker is a participant (or the central server) that utilizes black-box attacks for inferring membership of data from particular subjects. Chase et al.~\cite{saeed} propose an active property inference attack for data poisoning scenarios, which we will cover and compare to in Section~\ref{sec:active_inference_attacks}.

%The closest work to ours are by Chase et al.~\cite{saeed} and Tramer et al.~\cite{tramer2022truth}. In their work, the attacker can manipulate some of the training data of the model such that a model trained (from scratch) on the poisoned data has an increased inference risk. However, their methods are not applicable to the transfer learning scenario. 
%In this work, we will focus on the property inference in transfer learning scenarios in which the attacker releases the upstream model and infer sensitive properties of the downstream models tuned from that upstream model.
% 

\shortsection{Defenses}
Defending against property inference attacks is an open problem. There are no studies in the current literature on active adversaries, and only a couple on passive ones. Ma et. al.~\cite{ma2021nosnoop} propose a defense against property inference attacks on data batches in the  collaborative learning setting. However, adversaries in the transfer-learning setting do not have access to batch-wise gradients of the downstream trainer. Chen and Ohrimenko~\cite{chen2022protecting} utilize mechanisms that add carefully-crafted noise to features to provide theoretical guarantees against inference adversaries, but focus on query-based access to the underlying dataset, not a machine learning model trained on it. These existing defenses thus do not apply to our threat model.

%propose a framework that reduces property inference to Boolean functions of individual members, posing the ratio of members satisfying the given function in a dataset as the property. These property inference attacks have since then been proposed as distribution inference attacks~\cite{suri2022formalizing}, presenting such attacks as inferring properties of the distributions used to sample datasets, differentiating them from exact inference attacks like dataset inference~\cite{maini2021dataset}. Nearly all property inference attacks use meta-classifiers to perform inference: training models on versions of datasets with and without the target property, followed by training a meta-classifier on top of these classifiers's model representations. These representations can take several forms: using model weights themselves with permutation-invariance~\cite{ganju2018property}, or model activations or logits for a generated set of query points~\cite{xu2019detecting}. However, the capability of such approaches is limited: the most that these attacks have been shown to work is medium-sized convolutional networks on the CelebA dataset~\cite{suri2022formalizing}.


\subsection{Active Privacy Attacks} \label{sec:active_inference_attacks}
% Perhaps the closely related works to ours as ones that proactively enhance the effectiveness of privacy attacks by manipulating the model training process in certain ways~\cite{saeed, melis2019exploiting, nasr2019comprehensive, tramer2022truth}. 
%shown that the adversary can, by using proactive ways, achieve stronger attacks that infer private information from deep learning systems~\cite{nasr2019comprehensive, melis2019exploiting, tramer2022truth, saeed}. In this section, we introduce the ones that are close to ours.

In the decentralized federated learning training, by submitting specially crafted gradients to the central server, malicious agents can increase membership inference risk~\cite{nasr2019comprehensive} and property inference risks~\cite{melis2019exploiting} of other benign agents' training data. However, these attacks do not apply to transfer learning scenario, as the attacker cannot control model gradients of downstream training. In the centralized setting, researchers propose attacks to poison the victim's training data such that the impacts of attribute inference and membership inference~\cite{tramer2022truth} and property inference~\cite{saeed} attacks are amplified on the poisoned model.
The ability to poison the victim's data is a threat model orthogonal to ours, since we have no access to the victim's downstream data. While there is scope to combine such approaches for stronger attacks (albeit with stronger access assumptions), we choose to focus on the scenario with no read/write access to the victim's data.

\fi %%%%%%%%%%%%%%%%%%%%%%%%%%%%%%%%

\section{Linear Shortcut Across Blocks}
\label{sec:layer_jump}

To use a hidden representation from layer $\ell<L$ as a final representation, we propose to cast it using linear regression, while skipping the computation in-between these layers. More generally, this approach can be applied to cast any $\ell$-th hidden representation to any subsequent layer $\ell'>\ell$.


\subsection{Method}
\label{subsec:methodology_linear_shortcut}

Given a source layer $\ell$ and a target layer $\ell'$ such that $0 \leq \ell < \ell' \leq L$, our goal is to learn a mapping
%$A_{\ell', \ell} \in \mathbb{R}^{d_h \times d_h}$
from hidden representations at layer $\ell$ to those at layer $\ell'$. To this end, we first collect a set of corresponding hidden representation pairs $(h^\ell, h^{\ell'})$. Concretely, we run a set $\mathcal{T}$ of input sequences through the model, and for each input $s$, we extract the hidden representations $h_{i_s}^{\ell}, h_{i_s}^{\ell'}$, where $i_s$ is a random position in $s$.
Next, we learn a matrix $A_{\ell', \ell} \in \mathbb{R}^{d_h \times d_h}$ by fitting linear regression over $\mathcal{T}$, i.e., $A_{\ell', \ell}$ is a numerical minimizer for:
$$ A \mapsto \sum_{s \in \mathcal{T}} || A \cdot h_{i_s}^\ell - h_{i_s}^{\ell'} ||^2,$$ 
and define the mapping of a representation $h$ from layer $\ell$ to layer $\ell'$ as:
\begin{equation}
\label{eq:linear_jump}
    \matl{} (h) \coloneqq A_{\ell', \ell} \cdot h.
\end{equation}


\subsection{Baseline}
\label{subsec:baseline}

We evaluate 
% our method against 
the prevalent approach of ``reading'' hidden representations directly, without any transformation. 
Namely, the propagation of a hidden representation from layer $\ell$ to layer $\ell'$ is given by the identity function, dubbed \id{}:

$$ \idl{} (h) \coloneqq h.$$

% Notably, 
This baseline 
assumes that representations at different layers operate in the same linear space.

\subsection{Quality of Fit}
\label{subsec:experiments_r2}

We first evaluate our method by measuring how well the learned linear mappings approximate the representations at the target layer. To this end, we calculate the (coordinate-averaged) $r^2$-score of our mapping's outputs with respect to the representations obtained from a full inference pass, and compare to the same for the \id{} baseline.


\paragraph{Models.}

We use \gpt{} \cite{radford2019language}, a decoder-only auto-regressive LM, with $L = 48$, $d_h = 1600$, and \bert{} \cite{devlin-etal-2019-bert}, an encoder-only model trained with masked language modeling, with $L=24$, $d_h=1024$.
% \footnote{\label{footnote:hf}We use models and data from Huggingface \cite{wolf-etal-2020-transformers,lhoest-etal-2021-datasets}.}
%For masked token prediction, we use a masked LM head pre-trained for our \bert{} model.

% \footnote{Specifically, we use the Huggingface Transformers \cite{wolf-etal-2020-transformers} implementations of all these models.}

%\sy{We use \gpt{} \cite{radford2019language}, a decoder-only auto-regressive LM, coming in four scales; $\texttt{gpt2}$ ($L = 12$, $d_h = 768$), $\texttt{gpt2-medium}$ ($L = 24$, $d_h = 1024$), $\texttt{gpt2-large}$ ($L = 36$, $d_h = 1280$) and $\texttt{gpt2-xl}$ ($L = 48$, $d_h = 1600$). Also, we use \bert{} \cite{devlin-etal-2019-bert}, an encoder-only model trained with masked language modeling, coming in two scales;  \texttt{bert-base-uncased} ($L=12$, $d_h=768$) and \texttt{bert-large-uncased} ($L=24$, $d_h=1024$). For masked token prediction, we use masked LM heads pre-trained for our models. Specifically, we use the Huggingface Transformers \cite{wolf-etal-2020-transformers} implementations of all these models. The plots presented in this section are for $48$-layered \gpt{} and $24$-layered \bert{}.}

%\sy{We use \gpt{} \cite{radford2019language}, a decoder-only auto-regressive LM, in the Huggingface \cite{wolf-etal-2020-transformers} implementation\footnote{\url{https://huggingface.co/gpt2}}, coming in four scales; $\texttt{gpt2}$ ($L = 12$, $d_h = 768$), $\texttt{gpt2-medium}$ ($L = 24$, $d_h = 1024$), $\texttt{gpt2-large}$ ($L = 36$, $d_h = 1280$) and $\texttt{gpt2-xl}$ ($L = 48$, $d_h = 1600$). Also, we use \bert{} \cite{devlin-etal-2019-bert}, an encoder-only model trained with masked language modeling, in the Hugginface implementation, coming in two scales;  \texttt{bert-base-uncased}\footnote{\url{https://huggingface.co/bert-base-uncased}} ($L=12$, $d_h=768$) and \texttt{bert-large-uncased}\footnote{\url{https://huggingface.co/bert-large-uncased}} ($L=24$, $d_h=1024$). For masked token prediction, we use the \texttt{BertForMaskedLM} heads from Huggingface, pretrained for these models. The plots presented in this section are for $48$-layered \gpt{} and $24$-layered \bert{}.}

\paragraph{Data.}
We sample random sentences from Wikipedia,
% \footref{footnote:hf} 
collecting 9,000 (resp. 3,000) sentences for the training set $\mathcal{T}$ (resp. validation set $\mathcal{V}$).\footnote{We use sentences rather than full documents to simplify the analysis.}
%\sy{We use two data sources to evaluate our method. One is Wikiepdia \cite{lhoest-etal-2021-datasets}\footnote{\url{https://huggingface.co/datasets/wikipedia}}; we use \texttt{spaCy}\footnote{\url{https://spacy.io/}} to divide documents into sentences\footnote{We use sentences rather than full documents to simplify the analysis.}\footnote{We pick randomly a Wikipedia document and then pick randomly a sentence ending in a newline character in it. \sy{[maybe this footnote is not needed?]}}, collecting 9,000 (resp. 3,000) random sentences for the training set $\mathcal{T}$ (resp. validation set $\mathcal{V}$). The second is a news article sentences dataset, the 10K English 2020 news sentences corpus
% \footnote{\url{https://downloads.wortschatz-leipzig.de/corpora/eng_news_2020_10K.tar.gz}} from the Leipzig Corpora Collection \cite{goldhahn-etal-2012-building}, which we randomly divide into a training set $\mathcal{T}$ consisting of 9,000 examples and a validation set $\mathcal{V}$ consisting of 1,000 examples.
% We truncate sentences to the maximal token length allowed by the model \mg{do we ever need to truncate? a sentence has about 10 words and the max. input len is thousands} \sy{[I surely did not need to in Leipzig, but discovered (via a transformers runtime warning) that I do need to for some (probably a minority) of the Wikipedia sentences. This probably has to do with that it is not really ``sentences" necessarily, for example, I noticed that it has some listings or something like that (bulleted items)... So some minority might get very long I guess...]}.
For each example $s$, we select a random position $i_s$ and extract the hidden representations $h_{i_s}^{\ell}$ at that position from all the layers.
For \bert{}, we first replace the input token at position $i_s$ with a \mask{} token, as our motivation is interpreting predictions, which are obtained via masked tokens in \bert{} (see \S\ref{subsec:BERT}).
Thus, in this case, the hidden representations we consider
%in the case of \bert{}
are of \mask{} tokens only.
%As we observed highly similar results for the two data sources across all our experiments, throughout the paper we will mainly report results for Wikipedia (except for \S\ref{sec:robustness}, where we cross-validate).


\begin{figure}[t]
\includegraphics[scale=0.2]{figs/r2_scores_48.pdf}
% \includegraphics[width=\columnwidth]{figs/r2_scores_48.pdf}
\caption{The coordinate-averaged $r^2$-score of $\matl{}$ (left) and $\idl{}$ (right) (\gpt{}).}
\label{fig:r2_scores}
\end{figure}


\begin{figure}[t]
\setlength{\belowcaptionskip}{-10pt}
\includegraphics[scale=0.2]{figs/bertmask_r2_scores_24.pdf}
% \includegraphics[width=\columnwidth]{figs/bertmask_r2_scores_24.pdf}
\caption{The coordinate-averaged $r^2$-score of $\matl{}$ (left) and $\idl{}$ (right) (\bert{}).}
\label{fig:bertmask_r2_scores}
\end{figure}



\paragraph{Evaluation.}
For every pair of layers $\ell, \ell'$, such that $0 \leq \ell < \ell' \leq L$, we use the training set $\mathcal{T}$ to fit linear regression as described in \S\ref{subsec:methodology_linear_shortcut}, and obtain a mapping $\matl{}$. 
Next, we evaluate the quality of $\matl{}$ as well as of $\idl{}$ using the $r^2$-coefficient, uniformly averaged over all coordinates. Concretely, we compute the $r^2$-coefficient of each of the predicted representations $\matl{} (h_{i_s}^{\ell})$ and $\idl{} (h_{i_s}^{\ell})$ versus the true representations $h_{i_s}^{\ell'}$
over all $s \in \mathcal{V}$.
%as we vary $s \in \mathcal{V}$.
%for every $s \in \mathcal{V}$.



\paragraph{Results.}
Results for \gpt{} and \bert{} are presented in Figs.~\ref{fig:r2_scores} and~\ref{fig:bertmask_r2_scores}, respectively.
In both models, \mat{} consistently yields better approximations than \id{}, as it obtains higher $r^2$-scores (in blue) across the network. 
This gap between \mat{} and \id{} is especially evident in \bert{}, where \id{} completely fails to map the representations between most layers, suggesting that hidden representations are modified  substantially by every transformer block.
Overall, this highlights the shortcoming of existing practices to inspect representations in the same linear space, and the gains from using our method to approximate future layers.
% in the network.
\section{Linear Shortcut for Language Modeling}
\label{sec:prediction}

We saw that our method approximates future hidden representations substantially better than a naive propagation. 
In this section, we will show that this improvement also translates to better predictive abilities from earlier layers. Specifically, we will use our method to estimate how often intermediate representations encode the final prediction, in the context of two fundamental LM tasks; next token prediction and masked token prediction.

\paragraph{Evaluation Metrics.}
Let $h, h' \in \mathbb{R}^{d_h}$ be a final representation and a substitute final representation obtained by some mapping, and denote by $\delta (h), \delta (h') \in \mathbb{R}^{d_v}$ their corresponding output probability distributions (obtained through projection to the output vocabulary -- see details below). 
We measure the prediction quality of $h'$ with respect to $h$ using two metrics:
\begin{itemize}
[leftmargin=*,topsep=1pt,parsep=1pt]
    \item \textbf{Precision@$k$} ($\uparrow$ is better): This checks whether the token with the highest probability according to $\delta(h')$ appears in the top-$k$ tokens according to $\delta(h)$. Namely, we sort $\delta(h)$ and assign a score of $1$ if $\arg\max(\delta(h'))$ appears in the top-$k$ tokens by $\delta(h)$, and $0$ otherwise.
    
    \item \textbf{Surprisal} ($\downarrow$ is better): We measure the minus log-probability according to $\delta(h)$, of the highest-probability token according to $\delta(h')$. Intuitively, low values mean that the model sees the substitute result as probable and hence not surprising.
\end{itemize}

\noindent We report the average Precision@$k$ and Surprisal over the validation set $\mathcal{V}$.



\subsection{Next Token Prediction}
\label{subsec:next_token_prediction_task}

Auto-regressive LMs output for every position a probability distribution over the vocabulary for the next token. Specifically, the output distribution for every position $i$ is given by $\delta (h_i^L)$, where:
\begin{equation}\label{eq:output_distribution}
    \delta (h) = \texttt{softmax} ( E^\top \cdot h) \in \mathbb{R}^{d_v}
\end{equation}
For some LMs, including \gpt{}, a layer normalization $\texttt{ln\_f}$ is applied to the final layer representation before this conversion (i.e., computing $\delta (\texttt{ln\_f}(h))$ rather than $\delta (h)$).

Recall that our goal is to measure how well this distribution can be estimated from intermediate representations, i.e. estimating $\delta (h_i^L)$ from $\delta (h_i^\ell)$ where $\ell<L$. To this end, we first run examples from the validation set through the model, while extracting for each example $s$ the hidden representation of a random position $i_s$ at every layer. Next, we apply our mappings $\matlL{}$ and the $\idlL{}$ baseline to cast the hidden representations of every layer $\ell$ to final layer substitutes (see \S\ref{sec:layer_jump}). Last, for each layer, we convert its corresponding final-layer substitute to an output distribution (Eq.~\ref{eq:output_distribution}) and compute the average Precision@$k$ (for $k=1,5,10$) and Surprisal scores with respect to the final output distribution, over the validation set.

\paragraph{Results.}
Figs.~\ref{fig:pre} and~\ref{fig:surp} show the average Precision@$k$ and Surprisal scores per layer in $48$-layered \gpt{}, respectively (the plots for the other \gpt{} models are presented in \S\ref{sec:app_scale}). Across all layers, \mat{} outperforms \id{} in terms of both scores, often by a large margin (e.g. till layer $44$ the Precision@$1$ achieved by \mat{} is bigger than that of $\id{}$ by more than $0.2$). 
This shows that linear mappings enable not just better estimation of final layer representations, but also of the predictions they induce. Moreover, the relatively high Precision@$k$ scores of \mat{} in early layers ($0.62$-$0.82$ for $k=10$, $0.52$-$0.74$ for $k=5$, and $0.28$-$0.45$ for $k=1$) suggest that early representations already encode a good estimation of the final prediction. Also, the substantially lower Surprisal scores of \mat{} compared to \id{} imply that our method allows for a more representative reading into the layer-wise prediction-formation of the model than allowed through direct projection to the vocabulary.

\begin{figure}[t]
\centering
\includegraphics[scale=0.4]{figs/pre_48.pdf}
\caption{Precision@$k$ ($k = 1,5, 10$) of $\matlL{}$ and $\idlL{}$ for next token prediction in $48$-layered \gpt{}.}
\label{fig:pre}
\end{figure}

\begin{figure}[t]
\centering
\includegraphics[scale=0.35]{figs/surp_48.pdf}
\caption{Surprisal for $\matlL$ and the baseline $\idlL{}$ ($48$-layered \gpt{} next token prediction task). A 95\% confidence interval surrounds the lines.}
\label{fig:surp}
\end{figure}

\subsection{Masked Token Prediction}
\label{subsec:BERT}

We now conduct the same experiment for the task of masked language modeling, where the model predicts a probability distribution of a masked token in the input rather than the token that follows the input. Unlike next token prediction, where the output distribution is computed from representations of varying input tokens, in masked token prediction the output is always obtained from representations of the same input token (i.e. \texttt{[MASK]}).

For this experiment, we use \bert{}, on top of which we use a pretrained masked language model head $\delta$; given a token sequence $s$, a \mask{} token inside it and its final representation $h$, $\delta (h) \in \mathbb{R}^{d_v}$
 is a probability distribution over tokens giving the model's assessment
 of the likelihood of tokens to be fitting in place of the \mask{} token in $s$.


\begin{figure}[t]
\centering
\includegraphics[scale=0.4]{figs/bertmask_pre_24.pdf}
\caption{Precision@$k$ ($k = 1,5, 10$) for  $\matlL{}$ and the baseline $\idlL{}$ ($24$-layered \bert{} masked token prediction task).}
\label{fig:bertmask_pre}
\end{figure}

\begin{figure}[t]
\centering
\includegraphics[scale=0.35]{figs/bertmask_surp_24.pdf}
\caption{Surprisal for $\matlL{}$ and the baseline $\idlL{}$ ($24$-layered \bert{} masked token prediction task). A 95\% confidence interval surrounds the lines.}
\label{fig:bertmask_surp}
\end{figure}

\paragraph{Results.}
Figs.~\ref{fig:bertmask_pre} and~\ref{fig:bertmask_surp} present the average Precision@$k$ and Surprisal scores per layer in $24$-layered \bert{} (the plots for the $12$-layered \bert{} model are presented in \S\ref{sec:app_scale}), overall showing trends similar to those observed for next token prediction in \gpt{} (\S\ref{subsec:next_token_prediction_task}). This is despite the differences between the two tasks and the considerable architectural differences between \bert{} and \gpt{}.
Notably, the superiority of \mat{} over \id{} in this setting is even more prominent; 
while \mat{}'s precision is between $0.2-0.6$ in the first ten layers (Fig.~\ref{fig:bertmask_pre}), \id{}'s precision for all values of $k$ is close to zero, again strongly indicating that our method allows for better reading into early layer hidden representations. 
More generally, \mat{} improves the Precision@$1$ of \id{} by more than $17\%$ at most layers, and unveils that a substantial amount of predictions ($>25\%$ starting from layer $3$) appear already in the very first layers.
Interestingly, the (rough) divide between the first half of layers and last half of layers for $\id{}$ in Figs.~\ref{fig:bertmask_pre},~\ref{fig:bertmask_surp} seems to align with the two-hump shape of the blue region for $\mat{}$ in Fig.~\ref{fig:bertmask_r2_scores}.

\paragraph{Analysis.}
We manually compare the predictions of our mapping $\matlL{}$ with $\idlL{}$, for a $24$-layered \bert{} model.  Concretely, we select 50 random sentences from the Leipzig dataset. Next, for each layer $\ell$, we manually analyze how many of the top-$5$ tokens according to $\matlL{}$ and $\idlL{}$ fit into context. We consider a token to fit into context if it is grammatically plausible within the sentence (see Tab.~\ref{tab:manual} for concrete examples).
In the resulting $1250$ instances (i.e. $50$ sentences $\times$ $25$ representations), we observe a substantially higher plausibility rate of $85.36\%$ for \mat{} compared to $52.8\%$ for \id{}. In fact, only in less than $4.3\%$ of the instances there are more plausible tokens among the top-$5$ tokens according to \id{} than among the top-$5$ tokens according to \mat{}, further supporting the Surprisal results above.

\begin{table*}
\footnotesize
\setlength{\belowcaptionskip}{-15pt}
\begin{tabular}{p{0.3\linewidth}ccccc}
& $\texttt{id}_{4 \rightarrow 24}$ & $\texttt{mat}_{4 \rightarrow 24}$ & $\texttt{id}_{12 \rightarrow 24}$ & $\texttt{mat}_{12 \rightarrow 24}$ & $\texttt{id}_{24 \rightarrow 24}$ \\ \midrule
\multirow{5}{=}{aldridge had shoulder surgery in \mask{}.} & fellowship & \tcbox{time} & cyclist & \tcbox{2009} & \tcbox{september} \\
& employment & \tcbox{it} & emergencies & \tcbox{2008} & \tcbox{november} \\
& agreement & her & seniors & \tcbox{2010} & \tcbox{december} \\
& \#\#ostal & them & cycling & \tcbox{2006} & \tcbox{august} \\
& \#\#com & work & \tcbox{pennsylvania} & \tcbox{2007} & \tcbox{july} \\ \midrule
\multirow{5}{=}{on your next view you will be asked to \mask{} continue reading.} & \#\#com & be & be & be & \tcbox{please} \\
& accreditation & get & undergo & \tcbox{please} & \tcbox{simply} \\ 
& $	\copyright$ & go & spartans & help & \tcbox{also} \\ 
& fellowship & \tcbox{help} & seniors & \tcbox{simply} & \tcbox{again} \\ 
& summer & have & * & say & \tcbox{immediately} \\ \bottomrule
\end{tabular}
\caption{Examples of top-$5$ predictions at layers $4$, $12$ and $24$, under the mappings $\matlL{}$ and $\idlL{}$, for a $24$-layered \bert{} model. Grammatically plausible predictions (according to a human annotator) are marked in \tcbox{blue}. Note that at layer $24$ the predictions of $\matlL{}$ and $\idlL{}$ are the same (by definition).} 
\label{tab:manual}
\end{table*}

\section{Implication to Early Exiting}
\label{sec:applications}

%The fact that it is often possible to approximate
The possibility of approximating
the final prediction already in the early layers has important implications for efficiency; applying our linear mapping instead of executing transformer blocks of quadratic time complexity, could save a substantial portion of the computation. In this section, we demonstrate this in the context of early exiting.

When 
% performing transformer model inference under 
using an early exit strategy \cite{schwartz-etal-2020-right, xin-etal-2020-deebert, schuster2022confident}, one aims at deciding dynamically at which layer to stop the computation and ``read'' the prediction from the hidden representation of that layer.
More precisely, under a confidence measure paradigm, one decides to stop the computation for a position $i$ at layer $\ell$ based on a confidence criterion, that is derived from casting the hidden representation $h_i^\ell$ as a final-layer representation and converting it to an output probability distribution. Specifically, following \citet{schuster2022confident}, a decision to exit is made if the difference between the highest and the second highest probabilities is bigger than $$ 0.9 \cdot \lambda + 0.1 \cdot {\rm exp} (-4 i / N),$$
where $N$ is the average length of the input until position $i_s$ for $s \in \mathcal{V}$, and $\lambda$ is a hyper-parameter.

\begin{figure}[t]
\setlength{\belowcaptionskip}{-10pt}
\centering
\includegraphics[width=\columnwidth]{figs/ee_gpt2bert.pdf}
\caption{Precision@$1$ with early exit and ``fixed exit'', applied to the $24$-layer \gpt{} for next token prediction (left) and the $24$-layer \bert{} for masked token prediction (right). Varying the confidence parameter $\lambda$, the $x$-coordinate is the average number of layers processed before an early exit decision is reached.}
\label{fig:ee_gpt2bert}
\end{figure}

\quash{
\begin{figure}[t]
\setlength{\belowcaptionskip}{-10pt}
\centering
\includegraphics[scale=0.35]{figs/ee_pre1_24.pdf}
\caption{Precision@$1$ for the various early exit methods, and previous ``fixed exit'' methods for comparison ($24$-layer \gpt{} next token prediction task). Varying the confidence parameter $\lambda$, the $x$-coordinate is the average number of layers processed before an early exit decision is reached.}
\label{fig:ee_pre1}
\end{figure}
}

\paragraph{Experiment.}
We assess the utility of our mapping $\matlL{}$ for early exit as a plug-and-play replacement for $\idlL{}$, through which intermediate representations are cast into final-layer representations.
We use \gpt{} for the next token prediction and \bert{} for masked token prediction (both with 24 layers).
We run each of the models over the validation set examples, while varying the confidence parameter $\lambda$ and using either $\idlL{}$ or $\matlL{}$ for casting intermediate representations.
Furthermore, we compare these early exit variants to the ``fixed exit'' strategy from \S\ref{sec:prediction}, where the computation is stopped after a pre-defined number of layers rather than relying on a dynamic decision.
We evaluate each variant in terms of both prediction's accuracy, using the Precision@$1$ metric (see \S\ref{sec:prediction}), and efficiency, measured as the average number of transformer layers processed during inference.


\paragraph{Results.}
%Figs.~\ref{fig:ee_pre1} and~\ref{fig:bertmask_ee_pre1}
Fig.~\ref{fig:ee_gpt2bert}
plots the average Precision@$1$ score against the average number of layers processed, for $24$-layer \gpt{} and $24$-layer \bert{}. For both models, under an early exit strategy our mapping \mat{} again provides a substantial improvement over \id{}.
For example, aiming at $95\%$ average precision, \mat{} saves $\sim3.3$ ($13.8$\%) layers in \gpt{} compared to only $\sim1.4$ ($5.9$\%) layers by \id{}, and $\sim4.8$ ($20$\%) layers in \bert{} versus $\sim3.5$ ($14.6$\%) layers by \id{}.
These results highlight the potential gains prominent early exit methods can obtain by using our method.
Notably, in both models and for each of the mapping methods, early exit obtains better results than fixed layer exit, as expected. 

\quash{
\begin{figure}[t]
\setlength{\belowcaptionskip}{-10pt}
\centering
\includegraphics[scale=0.35]{figs/bertmask_ee_pre1_24.pdf}
\caption{Precision@$1$ for the various early exit methods, and previous ``fixed exit'' methods for comparison ($24$-layer \bert{} masked token prediction task). Varying the confidence parameter $\lambda$, the $x$-coordinate is the average number of layers processed before an early exit decision is reached.}
\label{fig:bertmask_ee_pre1}
\end{figure}
}
\section{Linear Shortcut Across Sub-Modules}
\label{sec:submodules}

% Our experiments show that
% , despite the commonly-applied simplification by interpretability works, transformer layers do not operate in the same linear space and 
% there is a major gap in approximating future representations using an identity mapping (\S\ref{sec:layer_jump}, \S\ref{sec:prediction}).
% Here, 
In this section, we investigate whether discrepancies across layers result from specific sub-modules or are a general behaviour of all sub-modules in the network.  
This is done by extending our approach to test how well particular components in transformer blocks can be linearly approximated. 


\paragraph{Method.}

Consider \gpt{} for definiteness, then:
% we have 
$$ \texttt{b}_{\ell} = \texttt{b}_{\ell}^{\texttt{ffn}} \circ \texttt{b}_{\ell}^{\texttt{attn}}$$ 
% with
\begin{equation}\label{eq:attn} \texttt{b}^{\texttt{attn}}_{\ell} (H) = \texttt{attn}_{\ell} (\texttt{ln1}_{\ell} (H)) + H,\end{equation} 
where $\texttt{attn}_{\ell}$ is
%a multi-head self-attention
a MHSA
layer and \texttt{ln1} is a layer normalization (LN), and 
$$ \texttt{b}^{\texttt{ffn}}_{\ell} (H) = \texttt{ffn}_{\ell} (\texttt{ln2}_{\ell} (H)) + H,$$  
where $\texttt{ffn}_{\ell}$ is
%a feed-forward network
an FFN
layer and $\texttt{ln2}$ is a LN.
\quash{
Given a block $\texttt{b}_\ell$ and one of its sub-modules $\texttt{ln1}_\ell, \ \texttt{attn}_\ell, \ \texttt{ln2}_\ell$, or $\texttt{ffn}_\ell$, we fit linear regression approximating the output of the sub-module given its input and then use it in order to define mappings, as we now describe.
}
Given a block $\texttt{b}_\ell$ and one of its sub-modules $\texttt{ln1}_\ell, \ \texttt{attn}_\ell, \ \texttt{ln2}_\ell$, or $\texttt{ffn}_\ell$, we fit linear regression approximating the output of the sub-module given its input, and then use it to define mappings $\matattnl{}$, $\matlnl{}$ and $\matffl{}$.
%We provide the definition of $\matattnl{}$ below, and that of the other two in App. \ref{sec:app_submodule_skip_description}.
We provide the formal definitions of these mappings in App. \ref{sec:app_submodule_skip_description}.
\iffalse
\paragraph{$\matattnl{}$.}
%Illustrating this on $\texttt{attn}_\ell$ for definiteness,
For an input $s$, let $v^\ell_{i_s}$ be the vector at position $i_s$ in the output of $\texttt{attn}_\ell (\texttt{ln1}_\ell (H^{\ell - 1}))$. We denote by $A_\ell^{\texttt{attn}} \in \mathbb{R}^{d_h \times d_h}$ the matrix numerically minimizing 
$$ A \mapsto \sum_{s \in \mathcal{T}} || A \cdot \texttt{ln1}_\ell (h^{\ell-1}_{i_s}) - v^\ell_{i_s}||^2,$$
and define an attention sub-module replacement (Eq.~\ref{eq:attn}) by $$
\texttt{b}^{\overline{\texttt{attn}}}_\ell (h) \coloneqq A_{\ell}^{\texttt{attn}} \cdot \texttt{ln1}_\ell (h) + h. $$
We then define a mapping between two layers ${\ell \rightarrow \ell'}$ by:
$$ \matattnl{} (h) \coloneqq $$
$$ \texttt{b}^{\texttt{ffn}}_{\ell'} ( \texttt{b}^{\overline{\texttt{attn}}}_{\ell'} ( \ldots (\texttt{b}^{\texttt{ffn}}_{\ell+1} ( \texttt{b}^{\overline{\texttt{attn}}}_{\ell+1} (h)))\ldots)).$$ 
Namely, when applying each $\ell''$-th block, $\ell < \ell'' \leq \ell'$, we replace its attention sub-module $\texttt{attn}_{\ell''}$ by its linear approximation.
%In an analogous way, we consider the mappings $\matffl{}$ and $\matlnl{}$, where in the latter we perform the linear shortcut both for \texttt{ln1} and for \texttt{ln2} (see~\S\ref{sec:app_submodule_skip_description} for precise descriptions).
Importantly, unlike the original attention module, the approximation $\texttt{b}^{\overline{\texttt{attn}}}_\ell$ operates on each position independently, and therefore applying $\matattnl{}$ disables any contextualization between the layers $\ell$ and $\ell'$. Note that this is not the case for $\matffl{}$ and $\matlnl{}$, which retain the self-attention sub-modules and operate contextually.
\fi

\paragraph{Evaluation.}


We analyze the $24$-layered \gpt{}, and proceed completely analogously to \S\ref{subsec:next_token_prediction_task}, evaluating the Precision@$1$ and Surprisal metrics for the mappings $\matattnlL{}$, $\matfflL{}$ and $\matlnlL{}$.

\begin{figure}[t]
\setlength{\belowcaptionskip}{-0pt}
\centering
%\includegraphics[scale=0.2]
\includegraphics[width=\columnwidth]{figs/parts_presurp_24.pdf}
\caption{Precision@$1$ and Surprisal for the various sub-module linear mappings, and $\matlL{}$ for comparison ($24$-layer \gpt{} next token prediction task). A 95\% confidence interval surrounds the Surprisal lines.}
\label{fig:parts_presurp}
\end{figure}

\quash{
\begin{figure}[t]
\centering
\includegraphics[scale=0.4]{figs/parts_pre1_24.pdf}
\caption{Precision@$1$ for the various sub-module linear shortcut mappings, and the mapping $\matlL{}$ for comparison (\gpt{} next token prediction task).}
\label{fig:parts_pre1}
\end{figure}

\begin{figure}[t]
\centering
\includegraphics[scale=0.35]{figs/parts_surp_24.pdf}
\caption{Surprisal for the various sub-module linear shortcut mappings, and the mapping $\matlL{}$ for comparison (\gpt{} next token prediction task). A 95\% confidence interval surrounds the lines.}
\label{fig:parts_surp}
\end{figure}
}

\paragraph{Results.}
Fig.~\ref{fig:parts_presurp} shows the average Precision@$1$ and Surprisal scores per layer.
From a certain layer (\textasciitilde$7$), all sub-module mappings achieve better results than the full-block mapping $\matlL{}$. Thus, it is not just the cumulative effect of all the sub-modules in the transformer block that is amenable to linear approximation, but also individual sub-modules can be linearly approximated. 
Furthermore, the linear approximation of attention sub-modules is less harmful than that of the FFN or LN sub-modules. 
% Hypothetically, 
A possible reason is that the linear replacement of FFN or LN ``erodes'' the self-attention computation after a few layers. 
Moreover, the good performance of $\matattnlL{}$ suggests that contextualization often exhausts itself in early layers; speculatively, it is only in more delicate cases that the self-attention of late layers adds important information. Last, remark the sharp ascent of the scores for layer normalization in layers $5$-$8$, for which we do not currently see a particular reason. To conclude, we see that the possibility of linear approximation permeates
%the various
transformer components.


\section{Related Work}

Recently, there was a lot of interest in utilizing intermediate representations in transformer-based LMs, both for interpretability and for efficiency.

In the direction of interpretability, one seeks to understand the prediction construction process of the model \cite{tenney-etal-2019-bert, voita-etal-2019-bottom}.

More recent works use mechanistic interpretability and view the inference pass as a residual stream of information \cite{dar2022analyzing,geva-etal-2022-transformer}. Additionally, there are works on probing, attempting to understand what features are stored in the hidden representations \cite{adi2017finegrained, conneau-etal-2018-cram,liu-etal-2019-linguistic}. Our work is different in that it attempts to convert intermediate representations into a final-layer form, which is interpretable by design.

In the direction of efficiency, there is the thread of work on early exit, where computation is cut at a dynamically-decided earlier stage \cite{schwartz-etal-2020-right,xin-etal-2020-deebert,schuster2022confident}. Other works utilize a fixed early stage network to parallelize inference \citep{leviathan2022fast, chen2023accelerating}. However, intermediate representations are directly propagated in these works, which we show is substantially worse than our approach. Moreover, our method requires training considerably less parameters than methods such as \citet{schuster-etal-2021-consistent}, that learn a different output softmax for each intermediate layer.  

More broadly, skipping transformer layers and analyzing the linearity properties of transformer components have been discussed in prior works \cite{Zhao2021of,mickus-etal-2022-dissect,wang-etal-2022-skipbert,lamparth2023analyzing}.


\section{Conclusion and Future Work}

We present a simple and effective method for enhancing utilization of hidden representations in transformer-based LMs, that uses 
pre-fitted context-free and token-uniform linear mappings.
Through a series of experiments on different data sources, model architectures and scales, we show that our method consistently outperforms the prevalent practice of interpreting representations in the final-layer space of the model, yielding better approximations of succeeding representations and the predictions they induce, thus allowing a more faithful interpretation of the model's prediction-formation.
We demonstrate the practicality of our method for improving computation efficiency, saving a substantial amount of compute on top of prominent early exiting approaches. 
Also, by extending our method to sub-modules, 
% more specifically the attention sub-modules, 
we observe that replacing a part of the transformer inference by a non-contextual linear computation often results in a small deterioration of the prediction.
This opens new research directions for improving model efficiency,
% and parallelizability.
% including breaking the computation into several parallelizable tasks.
including breaking the computation into parallel tasks.

\section*{Limitations}

Although we see in this work that there is more linear structure to transformer inference than could be explained solely by the residual connection, we do not elucidate a reason for that. We also do not try to formulate formal criteria according to which to judge, in principle, the quality of ways of short-cutting transformer inference in-between layers. In addition, our experiments cover only English data.


%\section*{Ethics Statement}
%Scientific work published at ACL 2023 must comply with the ACL Ethics Policy.\footnote{\url{https://www.aclweb.org/portal/content/acl-code-ethics}} We encourage all authors to include an explicit ethics statement on the broader impact of the work, or other ethical considerations after the conclusion but before the references. The ethics statement will not count toward the page limit (8 pages for long, 4 pages for short papers).

\section*{Acknowledgements}

We thank Tal Schuster for constructive comments.

% Entries for the entire Anthology, followed by custom entries
\bibliography{anthology,custom}
\bibliographystyle{acl_natbib}

\appendix

\section{Descriptions of $\matattn{}$, $\matff{}$ and $\matln{}$}
\label{sec:app_submodule_skip_description}

Here we detail the definitions of the mappings $\matattnl{}$, $\matffl{}$ and $\matlnl{}$ utilized in \S\ref{sec:submodules}.

\paragraph{Description of $\matattnl{}$.}
%Illustrating this on $\texttt{attn}_\ell$ for definiteness,
For an input $s$, let $v^\ell_{i_s}$ be the vector at position $i_s$ in the output of $\texttt{attn}_\ell (\texttt{ln1}_\ell (H^{\ell - 1}))$. We denote by $A_\ell^{\texttt{attn}} \in \mathbb{R}^{d_h \times d_h}$ the matrix numerically minimizing 
$$ A \mapsto \sum_{s \in \mathcal{T}} || A \cdot \texttt{ln1}_\ell (h^{\ell-1}_{i_s}) - v^\ell_{i_s}||^2,$$
and define an attention sub-module replacement (Eq.~\ref{eq:attn}) by $$
\texttt{b}^{\overline{\texttt{attn}}}_\ell (h) \coloneqq A_{\ell}^{\texttt{attn}} \cdot \texttt{ln1}_\ell (h) + h. $$
We then define a mapping between two layers ${\ell \rightarrow \ell'}$ by:
$$ \matattnl{} (h) \coloneqq $$
$$ \texttt{b}^{\texttt{ffn}}_{\ell'} ( \texttt{b}^{\overline{\texttt{attn}}}_{\ell'} ( \ldots (\texttt{b}^{\texttt{ffn}}_{\ell+1} ( \texttt{b}^{\overline{\texttt{attn}}}_{\ell+1} (h)))\ldots)).$$ 
Namely, when applying each $\ell''$-th block, $\ell < \ell'' \leq \ell'$, we replace its attention sub-module $\texttt{attn}_{\ell''}$ by its linear approximation.
%In an analogous way, we consider the mappings $\matffl{}$ and $\matlnl{}$, where in the latter we perform the linear shortcut both for \texttt{ln1} and for \texttt{ln2} (see~\S\ref{sec:app_submodule_skip_description} for precise descriptions).
Importantly, unlike the original attention module, the approximation $\texttt{b}^{\overline{\texttt{attn}}}_\ell$ operates on each position independently, and therefore applying $\matattnl{}$ disables any contextualization between the layers $\ell$ and $\ell'$. Note that this is not the case for $\matffl{}$ and $\matlnl{}$, which retain the self-attention sub-modules and operate contextually.

\paragraph{Description of $\matffl{}$.}
Let $v^\ell_{i_s}$ be the vector at position $i_s$ in the output of $\texttt{ln2}_{\ell} (\texttt{b}_\ell^{\texttt{attn}} (H^{\ell - 1}))$, for a given input $s$. We denote by $A_\ell^{\texttt{ffn}} \in \mathbb{R}^{d_h \times d_h}$ the matrix numerically minimizing 
$$ A \mapsto \sum_{s \in \mathcal{T}} || A \cdot v^{\ell}_{i_s} - \texttt{ffn}_{\ell} (v^\ell_{i_s})||^2,$$
and define a replacement of the feed-forward sub-module $\texttt{b}_{\ell}^{\texttt{ffn}}$ by $$ \texttt{b}^{\overline{\texttt{ffn}}}_\ell (H) \coloneqq A_{\ell}^{\texttt{ffn}} \cdot \texttt{ln2}_\ell (H) + H.$$
We then define a mapping between two layers ${\ell \rightarrow \ell'}$ by:
$$ \matffl{} (H) \coloneqq $$
$$ \texttt{b}^{\overline{\texttt{ffn}}}_{\ell'} ( \texttt{b}^{\texttt{attn}}_{\ell'} ( \ldots (\texttt{b}^{\overline{\texttt{ffn}}}_{\ell+1} ( \texttt{b}^{\texttt{attn}}_{\ell+1} (H))\ldots)).$$

\paragraph{Description of $\matlnl{}$.}
Let $v^\ell_{i_s}$ be the vector at position $i_s$ in the output of $\texttt{b}^{\texttt{attn}}_{\ell} (H^{\ell - 1})$, for a given input $s$. We denote by $A_\ell^{\texttt{ln1}} \in \mathbb{R}^{d_h \times d_h}$ the matrix numerically minimizing 
$$ A \mapsto \sum_{s \in \mathcal{T}} || A \cdot h^{\ell}_{i_s} - \texttt{ln1}_{\ell} (h^\ell_{i_s})||^2$$ and we denote by $A_\ell^{\texttt{ln2}} \in \mathbb{R}^{d_h \times d_h}$ the matrix numerically minimizing $$ A \mapsto \sum_{s \in \mathcal{T}} || A \cdot v^{\ell}_{i_s} - \texttt{ln2}_{\ell} (v^\ell_{i_s})||^2.$$ We define a replacement of the block $\texttt{b}^{\texttt{attn}}_{\ell}$ by \begin{equation} \texttt{b}^{\overline{\texttt{ln1}}}_\ell (H) \coloneqq \texttt{attn}_{\ell} (A_{\ell}^{\texttt{ln1}} \cdot H) + H\end{equation} and we define a replacement of the block $\texttt{b}^{\texttt{ffn}}_{\ell}$ by \begin{equation} \texttt{b}^{\overline{\texttt{ln2}}}_\ell (H) \coloneqq \texttt{ffn}_{\ell} (A_{\ell}^{\texttt{ln2}} \cdot H) + H.\end{equation}
We then define a mapping between two layers ${\ell \rightarrow \ell'}$ by:
$$ \matlnl{} (H) \coloneqq $$
$$ \texttt{b}^{\overline{\texttt{ln2}}}_{\ell'} ( \texttt{b}^{\overline{\texttt{ln1}}}_{\ell'} ( \ldots (\texttt{b}^{\overline{\texttt{ln2}}}_{\ell+1} ( \texttt{b}^{\overline{\texttt{ln1}}}_{\ell+1} (H))\ldots)).$$


\end{document}


%\renewcommand{\figurename}{Supplementary Figure}
\renewcommand{\thefigure}{S.\arabic{figure}}
\setcounter{figure}{0}
\renewcommand{\theequation}{S.\arabic{equation}}
\setcounter{equation}{0}
\renewcommand{\thesection}{S.\arabic{section}}
\setcounter{section}{0}
\renewcommand{\thetable}{S.\arabic{table}}
\setcounter{table}{0}

\setlength{\tabcolsep}{18pt}

\onecolumngrid

%\newpage

\setcounter{equation}{0}
\setcounter{figure}{0}
\setcounter{table}{0}

%%%%%%%%%%%%%%%%%%%%%%%%%%%%%%%%%%%%%%%%%%%%%%%%%%%%
\clearpage
\setcounter{page}{1}

\begin{center}
\textbf{\large Supplemental Information\\[4mm] 
\Large Universality of the superfluid Kelvin-Helmholtz instability by single-vortex tracking}\\[4mm]
D. Hern\'andez-Rajkov,$^{1,2,\ast}$
N. Grani,$^{1, 2, 3}$
F. Scazza,$^{4, 1, 2}$
G. Del Pace,$^{3}$
W. J. Kwon,$^{5}$
M. Inguscio,$^{1,2,6}$
K. Xhani,$^{2}$
C. Fort,$^{2,3}$
M. Modugno,$^{7,8,9}$
M. Marino,$^{2,10}$
and
G. Roati$^{1, 2}$
\\[2mm]
\emph{\small $^1$ European Laboratory for Nonlinear Spectroscopy (LENS), University of Florence, 50019 Sesto Fiorentino, Italy}\\
\emph{\small $^2$ Istituto Nazionale di Ottica del Consiglio Nazionale delle Ricerche (CNR-INO) c/o LENS, 50019 Sesto Fiorentino, Italy}\\
\emph{\small $^3$ Department of Physics, University of Florence, 50019 Sesto Fiorentino, Italy}\\
\emph{\small $^4$ Department of Physics, University of Trieste, 34127 Trieste, Italy}\\
\emph{\small $^5$ Department of Physics, Ulsan National Institute of Science and Technology (UNIST), Ulsan 44919, Republic of Korea}\\
\emph{\small $^6$ Department of Engineering, Campus Bio-Medico University of Rome, 00128 Rome, Italy}\\
\emph{\small $^7$ Department of Physics, University of the Basque Country UPV/EHU, 48080 Bilbao, Spain}\\
\emph{\small $^8$ IKERBASQUE, Basque Foundation for Science, 48013 Bilbao, Spain}\\
\emph{\small $^9$ EHU Quantum Center, University of the Basque Country UPV/EHU, 48940 Leioa, Biscay, Spain}\\
\emph{\small $^{10}$ Istituto Nazionale di Fisica Nucleare, Sez. di Firenze, 50019 Sesto Fiorentino, Italy}
%
\end{center}
\vspace*{-10pt}
\begin{center}
\emph{\small ${}^\ast$ E-mail: rajkov@lens.unifi.it}
\end{center}


\normalsize 
%--------------------------------------------------
%-------------- Experimental Methods --------------
%--------------------------------------------------
\section{Experimental Methods}

\paragraph{$\sigma^{*}$ scaling law}
To obtain experimentally the scaling between $\sigma^{*}$ and $\Delta v$ we fit the measured rates, reported in Fig.~3f, using the function $\sigma^*=A\, {\Delta v} ^{\alpha}$, leaving $A$ and $\alpha$ as free parameters. The results are shown in Fig.~\ref{fig:ExtendedFig1}. For all interaction regimes, we find the value of $\alpha$ to be consistent with the one expected from Eqs.~(1) and (2), namely $\alpha=2$. The fitted exponent averaged over all interaction regimes is $\alpha=1.98(3)$, where the error corresponds to the standard deviation of the mean.

\begin{figure}[ht!]%[htbp]
\centering
\includegraphics[width=0.5\columnwidth]{Images/ExtendedFig1.pdf}\vspace{-5pt}
\caption{KHI power-law scaling exponent $\alpha$ in the different interaction regimes, with $\sigma^*=A \Delta v ^{\alpha}$. The horizontal line denote the expected exponent $\alpha=2$ obtained from Eqs.~(1) and (2).}
\label{fig:ExtendedFig1}
\end{figure}

\paragraph{Thermodynamic properties}
The thermodynamic properties of the superfluid are obtained from analytical calculations based on the polytropic approximation for trapped gas, for which $\mu = g_{\gamma} n^{\gamma}$, where $\gamma$ is the effective polytropic index $\gamma = \partial \log \mu /\partial \log n$ \cite{Heiselberg2004,Haussmann2008}. In the BEC limit, $\gamma=1$, while at unitarity and in the BCS limit, $\gamma=2/3$. Within this approximation, the chemical potential and the Fermi energy take the form \cite{delpace2022imprinting}:
\begin{equation}\label{eq:chemicalpotential}
\mu_0  =  \left[\frac{ \Gamma(\frac{1}{\gamma} +\frac{3}{2})}{\pi^{3/2}\Gamma(\frac{1}{\gamma}+1)} \frac{\sqrt{M/2}\,\omega_z N g_{\gamma}^{1/\gamma}}{R_\mathrm{e}^2-R_\mathrm{i}^2}\right]^{\frac{2\gamma}{\gamma+2}},\quad E_F = 2\hbar\left[\frac{\hbar\omega_z N}{m(R_\mathrm{e}^2-R_\mathrm{i}^2)}\right]^{1/2}
\end{equation}
where $\Gamma$ is the Gamma function, $g_{\gamma}$ is a pre-factor that depends on the interaction regime: (i) for $(k_Fa)^{-1}>1$, $g_\mathrm{BEC} = 4\pi\hbar^2a_M/M$ with $a_M = 0.6 \, a$, $M = 2 m$, and $m$ is the mass of the ${}^6$Li atom; (ii) for $(k_Fa)^{-1}<-1$, $g_\mathrm{BCS} = \frac{\hbar^2}{2m}(6\pi^2)^{2/3}$; (iii) for $|(k_Fa)^{-1}|<1$, $g_\mathrm{UFG} = \xi_B \frac{\hbar^2}{2m}(6\pi^2)^{2/3}$, where $\xi_B$ is the Bertsch parameter taking the value $\xi_B \simeq 0.37$ at unitarity for $T=0$ \cite{Haussmann2008}. 
The Fermi wave number (see main text) is $k_F = \sqrt{2m\,E_F/\hbar^2}$.

\paragraph{Speed of sound}

\begin{figure}[h!]
	\centering	
 \vspace{10pt}
        \includegraphics[width=\textwidth]{Images/FigExp1.pdf}
            \caption{
            \textbf{a}, Image of the optical potential projected onto the atomic sample after being shaped by the DMD. \textbf{b}, Density wave propagating in the radial direction after abruptly enlarging and restoring the inner radius of the ring trap by $\SI{1.2(2)}{\mu m}$.
            \textbf{c}, Black circles correspond to the measured propagation velocity of the density wave as a function of the interaction parameter. Green square symbols correspond to the maximum relative velocity $\Delta v_{max}$ at the interface explored in this work. The solid lines display the expected speed of sound $c_s/v_F$ in the BEC and crossover regimes (see text). The dashed line corresponds to the critical velocity for pair breaking in the BCS regime \cite{Weimer2015}. 
            }
	\label{fig:ExpMeth1}
\end{figure}
\vspace{10pt}
 We measured the speed of sound by preparing the system in the same ring geometry of Fig. \ref{fig:ExpMeth1}a without the optical barrier separating the ring in two regions. We excite a sound wave propagating along the radial direction by abruptly enlarging and subsequently restoring the inner radius of the box potential. This procedure creates a density burst that travels at constant velocity along the radial direction toward the external radius. We measured the speed of sound $c_s$ by fitting the perturbed radial density profile with a Gaussian function, as shown in Fig. \ref{fig:ExpMeth1}b. The measured speed of propagation is reported in Fig. \ref{fig:ExpMeth1}c.
To calculate the speed of sound analytically, we calculate $c_s = \sqrt{\frac{n}{M}\frac{\partial \mu}{\partial n}} =\sqrt{\gamma \mu/M}$, and take the ratio with the Fermi velocity $v_F = \hbar k_F$.
In the BEC limit and for our geometry, the ratio $c_s/v_F$ can be expressed analytically in terms of $k_Fa$ and is given by $\frac{c_s}{v_F} = \left(\frac{3k_Fa_M}{2^{13/2}}\right)^{1/3}$, and for the BCS and crossover regimes as $\frac{c_s}{v_F}=\sqrt{{1 \over 3}\xi^{3/4}}$, where we assumed $\xi\equiv \xi(k_Fa)$ at $T=0$ \cite{Kwon2020}. The expected behavior of $\frac{c_s}{v_F}$ is shown in Fig. \ref{fig:ExpMeth1}c as a solid line.
For $1/k_Fa<0$, the measured speed is below the expected speed of sound. However, the measured propagation speed agrees with Leggett 3D homogeneous critical velocity for superfluid pair breaking \cite{Weimer2015}, suggesting this method fails to capture the speed of sound in the BCS regime. Nonetheless, the critical velocity for superfluid pair breaking is above the range of superfluid tangential velocities explored in this work.

\paragraph{Quasi-2D vortex dynamics}
The vertical confinement provided by the $\mathrm{TEM_{1,0}}$ laser beam is such that in the BEC regimes the ratio $\mu/(\hbar \omega_z) \gtrapprox 1.5$, and in the UFG and BCS regimes $E_F/(\hbar \omega_z) \approx 6$; making the system collisionally three dimensional. However, vortex dynamics behave as a quasi-two-dimensional system since only a few Kelvin modes can be populated. In fact, the standard Kelvin dispersion \cite{Fetter, Rooney2011} is:
\begin{equation}
    \omega(k) = -\frac{\hbar k^2}{2M} \log\left(\xi k\right),
\end{equation}
where $\xi$ is the healing length. Due to geometrical restrictions \cite{Rooney2011}, only modes with wavelength larger than the healing length can be effectively populated in a superfluid. Under our experimental condition, this translates into the fact that only the lowest wavenumber Kelvin mode with $k=\pi/R_z$ can be populated in the BEC regime, where $R_z=\sqrt{2\mu/(M\omega_z^2)}$ is the Thomas-Fermi radius in the $z$-direction. On the other hand, in the UFG and BCS regimes, due to higher Thomas-Fermi radius and smaller healing length, only the first three Kelvin modes with $k_n=(\pi+2\pi n)/R_z$ can be populated. Anyway, in all the interaction regimes explored in this work, the number of possibly populated Kelvin modes remains so small that we can assume a 2D dynamics of the vortex motion. 

\paragraph{Preparation of the vortex necklace}
We remove the circular barrier (Fig.~\ref{fig:ExpMeth1}a) between the two rings by lowering its intensity using a sequence of 15 different DMD patterns. To obtain a clear initial condition of the vortex crystal and to prevent the formation of other excitation in the system \cite{kanai2019merging}, we set the duration of the barrier removal to $\tau = \SI{28}{ms}$. 

\begin{figure}[h!]
	\centering	
        \includegraphics[width=\textwidth]{Images/Fig2Exp.pdf}
            \caption{ \textbf{a}, Infidelity in creating the target circulation state state, $\langle\Delta w\rangle _M - {\Delta w}_T$. \textbf{b}, Number of spurious vortices observed before removing the optical barrier, and \textbf{c}, Deviation of the total number of vortices from the target state, $\langle N_v \rangle _M - \Delta w_T$. All three panels were generated from 100 experimental repetitions for each of the two target states $\Delta w_T=6,12$ (blue and orange, respectively). \textbf{d}, Total number of vortices detected after removing the barrier, $t=0$ of vortex dynamics, as a function of the imprinted winding number difference $\Delta w_T$. The red dashed line is the identity line, $N_v=\Delta w$.
            }
	\label{fig:ExpMeth2}
\end{figure}

After the complete barrier removal, we observe the creation of a vortex necklace with a number of vortices given by the relative circulation $\Delta w$, as illustrated in Fig.~2a. The phase imprinting method allows to excite circulation states in the two superfluids in a highly reproducible way, but experimental imperfections can lead to shot-to-shot fluctuations in the circulation state of the rings. This leads to fluctuations in the initial configuration of vortices, which we estimate by analyzing the statistics of the relative circulation and vortex number in datasets of 100 experimental realizations on a BEC superfluid at $1/k_Fa_s=4.1(1)$. 
Figure~\ref{fig:ExpMeth2}a shows the distribution of the measured relative circulation between the two rings $\langle \Delta w\rangle_M$ with respect to the target $\Delta w_T$, measured from interferograms acquired before removing the circular barrier for $\Delta w_T = 6$ and $\Delta w_T=12$. 
In Fig.~\ref{fig:ExpMeth2}b, the number of spurious vortices introduced by the phase-imprinting protocol is displayed, measured from the TOF expansion of the two rings before the barrier removal. Finally, Fig.~\ref{fig:ExpMeth2}c shows the distribution of the total number of vortices in the superfluid detected in the TOF expansion after removing the circular barrier. 
Despite the high reliability in producing the desired circulation states in the two rings(Fig.~\ref{fig:ExpMeth2}a), we observe that the distribution of the total number of vortices detected after the barrier removal is augmented and broadened by the presence of  spurious vortices. This leads to residual fluctuations of the initial configurations of the vortex necklace (Fig.~\ref{fig:ExpMeth2}d), which determine the experimental uncertainty on the initial relative velocity $\Delta v$ (see horizontal error bars in Figs.~3f,g). They also contribute to the experimental noise on the extracted exponential growth rate for a given $\Delta w$.

%--------------------------------------------------
%--------------- Numerical Methods ----------------
%--------------------------------------------------
\section{Numerical Methods}
\subsection{Gross-Pitaevskii simulations}
In the BEC regime, at $T=0$, we simulate our system by solving the three-dimensional Gross-Pitaevskii (GP) equation
\begin{equation}
i\hbar \frac{\partial \psi(\bm{r},t)}{\partial t}= \left [  -\frac{\hbar^2}{2M}\nabla^2 + V(\bm{r},t) +g n(\bm{r},t) \right]\psi (\bm{r},t),
\label{eq:GPE}
\end{equation}
where $ V(\mathbf{r},t)$ is the external potential, $g n(\bm{r},t)$ is the mean-field potential due to the interaction strength $g=4\pi\hbar^2a_M/M$, the density is $n(\bm{r},t)=\int|\psi(\bm{r},t)|^2d^3\bm{r}$, and $\int n(\bm{r},t) d^3\bm{r}=N$ is the total number of molecules. In particular, the expression for the external potential we employed is: 

\begin{equation}\label{Vtrap}
V(r_\perp,z,t)=V_G(t) e^{-2\frac{(r_\perp-R_0)^2}{\sigma^2}} +V_r \left [ \tanh \left(\frac{r_\perp-R_{e}}{\epsilon} +1\right) +  \tanh \left(\frac{R_{i}-r_\perp}{\epsilon} +1\right)  \right] + 
\frac{1}{2} M (\omega^2_x x^2+\omega^2_y y^2+\omega^2_z z^2),
\end{equation}

where $r_\perp\equiv\sqrt{x^2 +y^2}$, $V_G(t)$ and $\sigma$ are the height and the size of the circular Gaussian barrier located at a distance $R_0=(R_{i}+R_{e})/2$ from the center in $xy$ plane, and $V_r$ and $\epsilon$ are the height and the stiffness of the ring potential in the $xy$ plane limited by internal and external radii $R_{i}$ and $R_{e}$ respectively. 
The simulation parameters are $a_M=1010\, a_0$, $N=3\times 10^4$. We used a grid of equal size along $x$ and $y$-directions equal to $\left[-60, 60\right]\mu$m and $ \left[-10,10\right]\mu$m along $z$ axis, based on up to $384 \times 384 \times 32$ grid points.
The harmonic potential frequencies are the same as in the experiment, $\omega_x=\omega_y=2\pi \times 2.5$~Hz and $\omega_z=2\pi \times 396$~Hz. The parameters of the ring potential are $V_r/h=2500$~Hz, $\epsilon=1.5$~$\mu$m, $R_{i}=10$~$\mu$m and $R_{e}=45$~$\mu$m respectively. 
The barrier size (FWHM) is $\sigma=1.4$~$\mu$m, whereas its height is linearly decreased in time during the barrier removal, starting from the initial value of $V_G(t=0)=V_0=h\times 5000$~Hz. 


We first compute the ground-state of the condensate (with the circular barrier at $V_0$) by numerically minimizing the GP energy functional corresponding to Eq. (\ref{eq:GPE}) by means of a conjugate gradient algorithm \cite{Press2007,Modugno2003}. The ground-state chemical potential is $\mu/h=887$~Hz. Then, we imprint an opposite circulation in the two rings, adding a phase term $\exp( \pm i w \theta)=\exp[\pm i w  \arctan (y/x)]$ in the outer (+) and inner (-) ring, respectively, being $w$ the integer winding number. This phase imprinting results in an opposite velocity field in each of the two rings with $v=\pm\hbar/M\nabla \theta=\pm\hbar w/(Mr_\perp)$. 
The dynamical evolution is then triggered by linearly lowering the height $V_G(t)$ of the internal barrier at 28~ms, as done in the experiment. At the end of this phase, $N_v=\Delta w$ vortices are nucleated at $r_\perp=R_0$ forming a regular array with angular periodicity $\Delta \theta=2\pi/N_v$.

\subsection{Finite-temperature kinetic model}
At finite temperature, the system is partially condensed, and its wavefunction can be written as the sum of condensate and thermal components. Here, we use the collisionless Zaremba-Nikuni-Griffin model \cite{ZNGbook, JacksonPRA2002, NickJPhysB2008, nick_book} (cZNG) to investigate the effect of the thermal component. This model has already been successfully applied in the study of different phenomena such as the  collective modes \cite{JacksonPRL2002,JacksonPRL2001,XhaniPRL2020,XhaniPRR2022}, soliton and vortex dynamics \cite{JacksonPRA2007,JacksonPRA2009,AllenPRA2013,AllenPRA2014,XhaniPRL2020}. 
The condensate  wavefunction $\psi$  evolves according to the generalized Gross-Pitaevskii equation:
\begin{equation}
\label{gped}
i \hbar \frac{\partial \psi (\bm{r},t)}{\partial t}=\left[ - \frac{\hbar^2 \nabla ^2}{2M}+V(\bm{r},t)+g(n(\bm{r},t)+2n_\mathrm{th}(\bm{r},t))\right] \psi(\bm{r},t) \;,
\end{equation}
which includes an additional term with respect to the Eq.~\eqref{eq:GPE}: the mean-field potential of the thermal cloud ($2g n_{\mathrm th}$ where $n_\mathrm{th}$ is the thermal cloud density). The thermal cloud dynamics is instead described through the phase-space distribution function $f$, which satisfies the collisionless Boltzmann equation:
\begin{equation}
\label{bolt}
\frac{\partial f}{\partial t}+ \frac{\mathbf{p}}{M} \cdot \nabla_{\mathbf{r}} f-\nabla_{\mathbf{r}}V^{\rm eff} _{\rm th} \cdot \nabla_{\mathbf{p}}f=0
\end{equation}
where $V^{\rm eff}_{\rm th}=V+2g (n+n_{\rm th})$ is the generalized mean-field potential felt by the thermal particles. The thermal cloud density instead is found as $n_{\rm th}=1/(2\pi\hbar)^3 \int d\mathbf{p} ~ f(\mathbf{p},\mathbf{r},t)$). These two equations are solved self-consistently in a grid of equal size along the $x$ and $y$ directions, equal to  $\left[-106,106\right]\mu$m, and to $ \left[-22,22\right]\mu$m along the $z$ axis, based on $768 \times 768 \times 160$ grid points. The total particle number is equal to $N=3 \times 10^4$. 

We first find the condensate equilibrium density by solving the time-independent generalized GP equation:
\begin{equation}\label{gped2}
\mu \psi _0=\left( -\frac{\hbar ^2}{2M} \nabla ^2  + V + g (n_0 + 2 n^0_\mathrm{th})\right)  \psi_0,
\end{equation}
with  $n_0$ and $n_0^\mathrm{th}$ being the equilibrium condensate and thermal density, respectively, in the presence of the circular Gaussian barrier. The initial thermal cloud density Ansatz is based on a  Gaussian profile obtained for a certain temperature \cite{ZNGbook}. 
Then, we imprint a phase on the initial condensate wavefunction, as it was done for $T=0$ GP simulations, having  opposite signs in the outer (+) and inner (-) rings. After that, the generalized GP equation and the Boltzmann equation are solved self-consistently in order to describe the system dynamics, in the presence of a time-dependent circular barrier whose potential height is removed in 28\,ms. 
%
Due to the repulsive interaction between condensate and thermal particles, the thermal density $n^{\rm th}$ is maximum where the condensate density $n^{\rm BEC}$ is minimum, as it occurs e.g.~at the barrier position, at the edges of the condensate and at the vortex cores.

\subsection{Time evolution and stability analysis from the numerical results}

 \begin{figure}[ht!]
    \begin{center}
        \includegraphics[width=0.9\columnwidth]{Images/Fig2Num3.pdf}
            \caption{Heat map of the angular structure factor $S(m,t)$ for the $\Delta w=6$ configuration for: \textbf{a}, $T=0$ (GP) and \textbf{b}, $T\simeq 0.4 T_c$ (cZNG). The dashed line corresponds to $m=\Delta w/2$, whose amplitudes in logarithmic scales as a function of time for both temperatures are shown in panel \textbf{c}. \textbf{d}, Comparison between the growth rate extracted at $T=0$ and $T\simeq 0.4 T_c$ as a function of the number of vortices.}
        \label{fig:NumMeth3}
    \end{center}
\end{figure}

From the simulations performed either using the GP equations for the $T=0$ case, or the cZNG model for the $T\neq 0$ case, we extract the density profile integrated along $z$. Then we identify the positions of the vortices ad different evolution times $t$, from which we calculate the structure factor $S(m,t)$ for different modes $m$. The result is shown in Fig. \ref{fig:NumMeth3} a for the case of $\Delta w=6$ (indicated by the color). Fig.~\ref{fig:NumMeth3}c, shows the time-dependence of the structure factor for the mode  $m=\Delta w/2$ mode both for the $T=0$ (black) and $T=53nK$ (red) which is characterized by an initial exponential growth. Similarly, Fig. \ref{fig:NumMeth3} b shows the behaviour of the structure factor at temperature $T=53\, \mbox{nK} \simeq 0.4 \,  T_c$, close to the experimental value. We first observe that similar to the experimental results, at finite temperature, the growth of the $m=3$ mode amplitude starts almost immediately after the vortex array is generated, while at $T=0$ the instability starts much later. Furthermore, at a finite temperature, more modes are populated. 
We then extract the growth rate of the instability, and this procedure is repeated for different initial velocities, with the  results  shown in Fig. \ref{fig:NumMeth3}d. Interestingly, even though the starting time of the instability and the population of different modes are strongly affected by the presence of the thermal cloud, the growth rate of the $m=\Delta w/2$ mode turns out to be only slightly affected, being in average larger than the one at $T=0$. We also note that at longer time evolution, the presence of the thermal cloud significantly affects the vortices dynamics, consistent with the results of \cite{JacksonPRA2009, AllenPRA2013}.

Some additional tests have been performed to prove the stability of the GP results presented in this work. We verified that by doubling or halving the number of particles $N$ or the width of the internal barrier $\sigma$, the growth rates do not change. We also included in the GP simulations some noise in the density distribution of the ground state of the BEC. This noise term only affects the time when the instability starts without changing the growth rates of the most unstable mode. 
Moreover, cZNG simulations show that fixing the condensate number of the finite temperature case to be the same as the simulation at $T=0$ provides a similar growth rate, opposite to what is found in a Josephson junction \cite{XhaniPRR2022} where the superfluid dynamics strongly depends on the condensate number.


Finally, we tried to include in the simulation an \textit{imperfect} phase imprinting. This has been done by adding a phase to the ground state wavefunction distributed around a mean value $2\pi w$ with fluctuations smaller than $\pi$. This phase imprinting leads to the creation of density waves, which affect the dynamics of the vortex crystal. However, by averaging on different realizations, we found that the most unstable mode's growth rate has a value that is only $18\%$ smaller than the one of the \textit{clean} configuration. Nevertheless, we do not exclude that the presence of additional vortices due to the experimental imprinting techniques could have a larger effect on the observed growth rates.

\subsection{Thickness of the interface layer and correction factor}\label{sec:EffectiveWidth}

\begin{figure}[ht!]
    \begin{center}
        \includegraphics[width=0.7\columnwidth]{Images/Fig.S3.pdf}
            \caption{Gross-Pitaevskii simulations. \textbf{a}, Density distribution $n(x,y,0)$; \textbf{b}, phase distribution$\phi(x,y,0)$; \textbf{c}, angular velocity distribution $v_{\theta}(x,y,0)$ for the case $\Delta w=10$ at $t=0$.}
        \label{fig:NumMeth1}
    \end{center}
\end{figure}

To estimate the interface thickness $\delta$ in Eq.~(2), we analyzed the results of the GP simulations in the following way. From the phase $\phi$ of the condensate wavefunction, we numerically compute the velocity field as $v=\hbar/M \nabla \phi$. In Fig. \ref{fig:NumMeth1}, we show, from left to right, the density distribution $n(x,y,0)$, the phase $\phi(x,y,0)$, and the tangential velocity field $v_{\theta}(x,y,0)$, for the case with $\Delta w=10$ at $t=0$, namely immediately after the removal of the circular barrier. Considering that the tangential velocity field $v_{\theta}$ changes sign at $r_{\perp}=R_0$, to extract the width of the interface layer between the two regions, we fit the radial velocity profile with the function $v_{\theta}(r_{\perp})=a\tanh \left [ (r_{\perp}-R_0)/\delta(\theta) \right]/r_{\perp}$.
The behaviour of $v_{\theta}(r_{\perp})$ is shown in Fig. \ref{fig:NumMeth2}a for different values of $\theta$ between two adjacent vortices. 
The value of $\delta (\theta)$ extracted from the fit is shown in Fig. \ref{fig:NumMeth2}b. The interface thickness exhibits a sinusoidal behavior with minima in correspondence with the vortex cores and maxima in the middle of each vortex pair.

\begin{figure}[h!]
    \begin{center}
    \vspace{10pt}
        \includegraphics[width=0.95\columnwidth]{Images/Fig2Num.pdf}
            \caption{\textbf{a}, Plot of $v_{\theta}$ for different angular positions between two adjacent vortices for the case $N_v=10$. The vertical gray dashed line corresponds to $r=R_0$, and the dashed red line represents the imprinted velocity in the two rings with modulus $\hbar N_v /(2 M r_{\perp})$.
            \textbf{b}, Values of the interface half-width $\delta$ (gray data point) for different angular positions between two consecutive vortices located at $\theta=0$~rad and $\theta=0.63$~rad ($N_v$=10). This behaviour is fitted with a sinusoidal function $\delta= a \cos (N_v \theta)+b$ (red line) to extract the mean value $\delta_\text{av}=b$.
            \textbf{c}, Values of $\delta_\text{av}$ as a function of $\Delta v$, plotted together with the expected behavior $\delta=\hbar/(M \Delta v)$ (red line).
            \textbf{d--e}, Extracted growth rate from the GP simulations (blue circles) as a function of $\Delta v$ for the experimentally explored velocities (\textbf{d}), and all simulated velocities (\textbf{e}). The PVM maximum growth rate is shown as a black dashed line, while the solid lines correspond to Rayleigh's formula for the maximum growth rate with $\eta=0.8$ (green line) and $\eta=1$ (orange line).
            }
        \label{fig:NumMeth2}
    \end{center}
\end{figure}

We define the effective value of the interface half-width $\delta_\text{av}$ as the average value of $\delta (\theta)$, and extract it by fitting such quantity with a sinusoidal function $\delta= a \cos (N_v \theta)+\delta_\text{av}$. 
In Fig.~\ref{fig:NumMeth2}c we report the obtained values of $\delta_\text{av}$ as a function of $\Delta v= \hbar \Delta w/ (MR_0)$. 
A similar study of the effective thickness of the interface can be performed within the PVM. As commented in the Methods, the equation of motion of a linear array of equispaced vortices can be analytically solved in a 2D space $x-y$ without boundaries, providing Eq. (3) of the main text for the velocity field. Plotting $v_x (x_0, y)$ for various $x_0 \in [-0.5 d_v, \, 0.5\, d_v]$ between two neighboring vortices provides similar trends of Fig.~\ref{fig:NumMeth2} a-b. We then employ the same procedure used to extract $\delta_\text{av}$ from the GP results for the PVM prediction, obtaining results in agreement with those of GP simulations. In particular, the estimated average interface thickness with both methods is well in agreement with the formula $\delta_\text{av} = \hbar /(M \Delta v)$, which we plot in Fig.~\ref{fig:NumMeth2}c as a red line.


On the other hand, Rayleigh's formula involves a linearly varying velocity between the two merging superfluids. We account for this difference by introducing an effective interface thickness $\delta = \eta \hbar /( M \Delta v)$, where $\eta$ is a phenomenological parameter of the order of $1$. We then fix the value of $\eta$ asking that the expressions of the most unstable mode under Rayleigh's formula [Eq.~(2)] and PVM are the same, namely:

\begin{equation}
    \sigma_R^*(k, \Delta v) =  \left(\frac{\sqrt{e^{-2\eta}-(\eta-1)^2}}{\eta}\right)\frac{M \Delta v^2}{4\hbar} = \sigma_{PVM}(k^*, \Delta v)= \frac{1}{2} \frac{M\Delta v^2}{4\hbar}.
\label{Eq_SM:sigma}
\end{equation}
For the two expressions to be the same, we have to fix $\left(\sqrt{e^{-2\eta}-(\eta-1)^2}\right)/\eta = \frac{1}{2}$, yielding the value of $\eta \approx 0.80465$. 

Finally, we verified that such a value of $\delta$ is also consistent with GP simulation results of the most unstable mode $\sigma_\text{GP}$. In particular, in Fig.~\ref{fig:NumMeth2}e, we report the values of $\sigma_\text{GP}$ as extracted from GP simulations (symbols) and compare it with the one given by Eq.~(\ref{Eq_SM:sigma}) with $\eta = 1$ (orange line) and $\eta = 0.8$ (green line). The latter is observed to well reproduce the GP results for low velocities, whereas for larger ones $\eta = 1$ matches the GP results better. The transition between these two regimes happens at velocities where the PVM is no longer applicable, i.e., when the inter-vortex distance is in the same order of magnitude as the healing length. The transition approximately occurs when $h/(M\Delta v) \approx 15~\xi$, with the associated velocity limit of $\Delta v= h/(15\,\xi M)\approx 3.16$~mm/s, or $\Delta w \approx 16$. For these reasons, we use the value $\eta \approx 0.8$ for $\Delta w < 16$ for all the measurements reported in this work.

\subsection{Point-Vortex Model (PVM) simulations}
We consider a two-dimensional fluid containing $N_v$ point vortices with quantized circulations $\Gamma=h/M$. When the inter-vortex separation is greater than a few healing lengths, vortices are advected by the velocity field created by other vortices. The equation of motion of each vortex is $d\vec{r}_i/dt = \vec{v_i}^0$, where $\vec{v_i}^0$ is the velocity field created by all the other vortices. When considering the ring geometry, $\vec{v_i}^0$ must take into account the boundary conditions, namely that the flow must have a zero radial component at both the internal ($R_i$) and external ($R_e$) radii. We include the boundary conditions by using the method of image vortices \cite{Martikainen}:

\begin{align}
    \vec{v_i}^0 &= \frac{\Gamma}{2\pi} \sum_{i\neq j}^{N_v} \kappa_j \hat{z} \times \frac{\vec{r}_i-\vec{r}_j}{\left|\vec{r}_i-\vec{r}_j\right|^2} +\frac{1}{2}\left(\frac{\Gamma}{2\pi} \sum_{j=1}^{N_v} \kappa_j \hat{z} \times \frac{\vec{r}_i-\vec{0}}{\left|\vec{r}_i-\vec{0}\right|^2}\right)\\ \nonumber
    &+\frac{\Gamma}{2\pi} \sum_{j=1}^{N_v} \sum_{n=1}^{\infty}\kappa_j \hat{z} \times \left( \frac{\vec{r}_i-\left(\frac{R_e}{R_i}\right)^{2n}\vec{r}_j}{\left|\vec{r}_i-\left(\frac{R_e}{R_i}\right)^{2n}\vec{r}_j\right|^2}
    + \frac{\vec{r}_i-\left(\frac{R_i}{R_e}\right)^{2n}\vec{r}_j}{\left|\vec{r}_i-\left(\frac{R_i}{R_e}\right)^{2n}\vec{r}_j\right|^2}\right)\\ \nonumber
    &-\frac{\Gamma}{2\pi} \sum_{j=1}^{N_v} \sum_{n=0}^{\infty}\kappa_j \hat{z} \times \left(\frac{\vec{r}_i-\left(\frac{R_e}{R_i}\right)^{2n}\left(\frac{R_e^2}{|\vec{r}_j|^2}\right)\vec{r}_j}{\left|\vec{r}_i-\left(\frac{R_e}{R_i}\right)^{2n}\left(\frac{R_e^2}{|\vec{r}_j|^2}\right)\vec{r}_j\right|^2}
    +\frac{\vec{r}_i-\left(\frac{R_i}{R_e}\right)^{2n}\left(\frac{R_i^2}{|\vec{r}_j|^2}\right)\vec{r}_j}{\left|\vec{r}_i-\left(\frac{R_i}{R_e}\right)^{2n}\left(\frac{R_i^2}{|\vec{r}_j|^2}\right)\vec{r}_j\right|^2}\right)
\end{align}

\bigskip

We solve the equations of motions for the vortex necklace configuration in the ring with the Runge-Kutta method of fourth order. From the obtained trajectories of the vortices, $\vec{r}_i (t)$ we compute the normalized angular structure factor $s (m, t)$.

%--------------------------------------------------
%------------------ Bibliography ------------------
%--------------------------------------------------

Below we first briefly describe the selected models and then their implementation details during pre-training.

% Traditional convolutional action recognition networks before 2017 are mostly built to process single frame or multiple consecutive frames; however, such simple structures overlook the importance of long-range temporal context in action recognition, which somehow underestimates the intrinsic temporal information within videos. 
Temporal segment networks (TSN) proposes segment-based sampling to learn temporal information across frames. 
Specifically, in TSN, a video is evenly divided into several temporal segments, which one random frame is sampled from. 
Then the output from each segment will be aggregated via pooling to obtain the final prediction. 
Temporal Shift Module (TSM) shifts feature channels along the temporal axis, which facilitates information exchanged among neighboring frames. 
It can be plug-and-played in 2D networks to enable stronger temporal modeling at zero computation and zero parameters.
Thus, TSM can achieve the performance of heavy 3D CNNs while maintaining the efficiency of 2D CNNs.
% TSM introduces stronger temporal learning capacity to 2D networks while maintaining light-weight. 

Inflated 3D ConvNet (I3D) is designed to bootstrap from the corresponding 2D network since (1) the architecture of 2D network is well designed and (2) the  weights of 2D network is well pre-trained, e.g., Inception~\cite{inception} $\rightarrow$ Inception-I3D~\cite{carreira2017quo}. 
% utilize pre-trained weights from the corresponding 2D network since these 2D weights have been well-designed and trained to perceive visual concepts.
I3D initializes its 3D kernels by duplicating the 2D ones along the temporal dimension, which helps the convergence of 3D CNNs. 
Inspired by~\cite{vaswani2017attention}, non-local networks (NL) adapts the non-local operation (i.e., self-attention~\cite{vaswani2017attention}) in its building block to model long-range dependency.
For video action recognition, its goal is to relate the same object, or person-object interaction within a distant time interval in videos.
Similar to TSM, non-local block is compatible to most convolutional networks.


TimeSformer is a pure transformer-based model, which is an extension of ViT~\cite{dosovitskiy2020image} to the spatiotemporal space. 
Given the quadratic complexity of self-attention, TimeSformer compares several attention strategies when considering temporal dimention in videos.
Finally, TimeSformer introduces the divided space-time attention to greatly reduce the computation burden but achieves promising results.
% on most video action recognition datasets. 
% This structure shows both effectiveness and efficiency in their reported results. 
Continuing this modeling shift from CNNs to Transformers, VideoSwin extends Swin Transformer~\cite{liu2021swin} by adding the inductive bias of locality in video transformers. 
Simply speaking, it adapts the idea of 2D shifted window self-attention to 3D space, which results in better speed-accuracy trade-off compared to previous approaches~\cite{bertasius2021space,arnab2021vivit}.
% Similarly, VideoSwin is an extension of Swin Transformer~\cite{liu2021swin}, by adapting the 2D shifted window self-attention to 3D.
% And shifted window ensure the connection across distant regions in the spatiotemporal tensors.


\begin{figure}[t]
\centering
    \includegraphics[width=8cm]{figures/radar_new.pdf}
    \caption{The rank of the averaged performance within different data domains for the 6 models in different settings. The most outside in these radar images means the highest performance. For each domain, we average the top-1 accuracy as the scores in finetuning and average the top-1 accuracy of 16-shot results in few-shot learning. Complete results are shown in Table~\ref{tab:finetune} and Figure~\ref{fewshot}.}
    \label{radar}
\end{figure}

\end{document}
