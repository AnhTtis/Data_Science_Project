\documentclass[10pt,twocolumn,letterpaper]{article}

\usepackage{iccv}
\usepackage{times}
\usepackage{epsfig}
\usepackage{graphicx}
\usepackage{amsmath}
\usepackage{amssymb}
\usepackage{subfigure}
\usepackage{color}
\usepackage{multirow}
\usepackage{textcomp, gensymb}
\usepackage{bbding}
\usepackage{booktabs}
\usepackage{bm, eucal}
\usepackage{pifont}
\usepackage{enumitem}

% Include other packages here, before hyperref.
\usepackage{caption, multirow, overpic, textpos}
\usepackage{booktabs}
\usepackage[table]{xcolor}

\usepackage{xspace}
\usepackage{balance}
\usepackage{numprint}
\usepackage{sidecap}
\usepackage{float}
\usepackage{siunitx}
\usepackage{pifont}

% If you comment hyperref and then uncomment it, you should delete
% egpaper.aux before re-running latex.  (Or just hit 'q' on the first latex
% run, let it finish, and you should be clear).
\usepackage[pagebackref=true,breaklinks=true,letterpaper=true,colorlinks,bookmarks=false]{hyperref}

\newcommand{\titlepicture}[2][]{%
  \renewcommand\placetitlepicture{%
    \includegraphics[#1]{#2}\par\medskip
  }%
}
\newcommand{\placetitlepicture}{} % initialization


% \usepackage[pagebackref,breaklinks,colorlinks]{hyperref}
\newcommand{\minus}{\scalebox{0.75}[1.0]{$-$}}


% Support for easy cross-referencing
\usepackage[capitalize]{cleveref}
\crefname{section}{Sec.}{Secs.}
\Crefname{section}{Section}{Sections}
\Crefname{table}{Table}{Tables}
\crefname{table}{Tab.}{Tabs.}

\iccvfinalcopy % *** Uncomment this line for the final submission

\def\iccvPaperID{2663} % *** Enter the ICCV Paper ID here
\def\httilde{\mbox{\tt\raisebox{-.5ex}{\symbol{126}}}}

% Pages are numbered in submission mode, and unnumbered in camera-ready
% \ificcvfinal\pagestyle{empty}\fi

\def\Ours{{Zolly}\xspace}
\def\Oursp{{$\rm Zolly^{\cP}$}\xspace}

\newcommand{\name}{Zolly\xspace}

\newcommand{\cP}{\mathcal{P}}

\newcommand{\misscite}{\textcolor{red}{[C]~}}
\newcommand{\missref}{\textcolor{red}{[R]~}}
\newcommand{\missvalue}{\textcolor{red}{[V]~}}
\newcommand{\needcheck}[1]{\textcolor{red}{#1}}
\newcommand{\todo}[1]{\textcolor{red}{#1}}


\begin{document}

% \normalsize Wenjia Wang$^{1}$\thanks{xxdfdsfasdgfx}
 % The University of HongKong
% Institution1 address\\
% {\tt\small wwj2022@connect.hku.hk}
% For a paper whose authors are all at the same institution,
% omit the following lines up until the closing ``}''.
% Additional authors and addresses can be added with ``\and'',
% just like the second author.
% To save space, use either the email address or home page, not both
% \and \normalsize Yongtao Ge$^{2 *}$ \and\normalsize Haiyi Mei$^3$\and\normalsize Zhongang Cai$^4$ \and\normalsize Qingping Sun$^3$ \and\normalsize Yanjun Wang$^3$ \and\normalsize Chunhua Shen$^5$ \and\normalsize Lei Yang$^{4, \dagger}$ \and\normalsize Taku Komura$^1$\\\
% \normalsize $^1$ The University of HongKong \quad
% \normalsize $^2$ The University of Adelaide \quad
% \normalsize $^3$ Shanghai AI Laboratory \\
% \normalsize $^4$ SenseTime Research \quad
% \normalsize $^5$ Zhejiang University \quad\\
% % First line of institution2 address\\
% {\small \href{wwj2022@connect.hku.hk}, \href{yongtao.ge@adelaide.edu.au}, \href{yanglei@sensetime.com}
% }

%%%%%%%%% TITLE
\title{Zolly: Zoom Focal Length Correctly for Perspective-Distorted \\Human Mesh Reconstruction}

\author{Wenjia Wang$^{1, 4}$ \quad \; Yongtao Ge$^{2}$ \quad \; Haiyi Mei$^3$ \quad \; Zhongang Cai$^{3, 4}$ \quad \; \\ Qingping Sun$^3$ \quad \; Yanjun Wang$^3$ \quad \;  Chunhua Shen$^5$ \quad \; Lei Yang$^{3, 4, \dagger}$ \quad \; Taku Komura$^1$\\[1.5mm]
\normalsize $^1$ The University of Hong Kong \quad
\normalsize $^2$ The University of Adelaide \quad
\normalsize $^3$ SenseTime Research \\
\normalsize $^4$ Shanghai AI Laboratory \quad
\normalsize $^5$ Zhejiang University \quad\\
}
% \maketitle
% Remove page # from the first page of camera-ready.
\ificcvfinal\thispagestyle{empty}\fi

\twocolumn[{%
	\renewcommand\twocolumn[1][]{#1}%
	\maketitle
	\begin{center}
		\newcommand{\teaserwidth}{\textwidth}
		\centerline{\includegraphics[width=\teaserwidth,clip]{pictures/teaser_bright.pdf}
		}
		\vspace{-1ex}
		\captionof{figure}{\label{fig:teaser}
 Close-up photography can make it difficult to discern 3D human pose in perspective-distorted images, while state-of-the-art methods often struggle with weak-perspective camera models or inaccurate focal length estimates. Our method overcomes these challenges, accurately recovering 3D human mesh from an approximate distance for fine-grained reconstruction. We could thus get focal length by our proposed approach.
  }
	\end{center}%
}]
% \saythanks
{
  \renewcommand{\thefootnote}%
    {\fnsymbol{footnote}}
  \footnotetext[1]{LY$^\dagger$ is the corresponding author.}
}

\begin{abstract}
As models continue to grow in size, the development of memory optimization methods (MOMs) has emerged as a solution to address the memory bottleneck encountered when training large models. To comprehensively examine the practical value of various MOMs, we have conducted a thorough analysis of existing literature from a systems perspective. 
% Furthermore, we have evaluated the most widely adopted MOMs employed in mainstream frameworks for both vision and language models.
Our analysis has revealed a notable challenge within the research community: the absence of standardized metrics for effectively evaluating the efficacy of MOMs. The scarcity of informative evaluation metrics hinders the ability of researchers and practitioners to compare and benchmark different approaches reliably. Consequently, drawing definitive conclusions and making informed decisions regarding the selection and application of MOMs becomes a challenging endeavor.
To address the challenge, this paper summarizes the scenarios in which MOMs prove advantageous for model training. We propose the use of distinct evaluation metrics under different scenarios. By employing these metrics, we evaluate the prevailing MOMs and find that their benefits are not universal. We present insights derived from experiments and discuss the circumstances in which they can be advantageous.

\end{abstract}
\section{Introduction}


Recent years have witnessed the rise of human digitization~\cite{habermannDeepCapMonocularHuman2020,alexanderCREATINGPHOTOREALDIGITAL,pengNeuralBodyImplicit2021,alldieckDetailedHumanAvatars2018, rajANRArticulatedNeural2020}. This technology greatly impacts the entertainment, education, design, and engineering industry.
There is a well-developed industry solution for this task.
High-fidelity reconstruction of humans can be achieved either with full-body laser scans~\cite{saitoSCANimateWeaklySupervised2021}, dense synchronized multi-view cameras~\cite{xiangModelingClothingSeparate2021a,xiangDressingAvatarsDeep2022a}, or light stages~\cite{alexanderCREATINGPHOTOREALDIGITAL}.
However, these settings are expensive and tedious to deploy and consist of a complex processing pipeline, preventing the technology's democratization.

Another solution is to view the problem as inverse rendering and learn digital humans directly from custom-collected data.
Traditional approaches directly optimize explicit mesh representation~\cite{loperSMPLSkinnedMultiperson2015, fangRMPERegionalMultiperson2018, pavlakosExpressiveBodyCapture2019} which suffers from the problems of smooth geometry and coarse textures~\cite{prokudinSMPLpixNeuralAvatars2020,alldieckVideoBasedReconstruction2018}. Besides, they require professional artists to design human templates, rigging, and unwrapped UV coordinates.
Recently, with the help of volumetric-based implicit representations~\cite{mildenhallNeRFRepresentingScenes2020, parkDeepSDFLearningContinuous2019, meschederOccupancyNetworksLearning2019} and neural rendering~\cite{laineModularPrimitivesHighPerformance2020, liuSoftRasterizerDifferentiable2019, thiesDeferredNeuralRendering2019}, 
one can easily digitize a quality-plausible human avatar from video footage~\cite{jiangNeuManNeuralHuman2022,wengHumanNeRFFreeviewpointRendering}.
Particularly, volumetric-based implicit representations~\cite{mildenhallNeRFRepresentingScenes2020, pengNeuralBodyImplicit2021} can reconstruct scenes or objects with much higher fidelity against previous neural renderer~\cite{thiesDeferredNeuralRendering2019,prokudinSMPLpixNeuralAvatars2020}, and is more user-friendly as it does not need any human templates, pre-set rigging, or UV coordinates.
Captured visual footage and corresponding skeleton tracking are enough for training.
However, better reconstructions and more friendly usability are at the expense of the following factors.
1) \textbf{Inefficiency:}
They require longer optimization times (typically tens of hours or days) and inference slowly.
Volume rendering~\cite{mildenhallNeRFRepresentingScenes2020,lombardiNeuralVolumesLearning2019} formulates images by querying the densities and colors of millions of spatial coordinates. 
In the training stage, due to memory constraints, only a small fraction of points are sampled which leads to slow convergence speed.
2) \textbf{Entangled representations}:
The geometry, materials, and motion dynamics are entangled in the neural networks. 
Due to the implicit nature of neural nets, one can hardly edit one property without touching the others~\cite{yuanNeRFEditingGeometryEditing2022a,liuEditingConditionalRadiance2021}.
3) \textbf{Graphics incompatibility}:
Volume rendering is incompatible with the current popular graphic pipeline,
which renders triangular/quadrilateral meshes efficiently with the rasterization technique.
Many downstream applications require mesh rasterization in their workflow (\eg, editing~\cite{foundationBlenderOrgHome}, simulation~\cite{benderPositionBasedSimulationMethods2015}, real-time rendering~\cite{akenine2019real}, ray-tracing~\cite{waldRTXRayTracing}).
Although there are approaches~\cite{lorensenMarchingCubesHigh,labelleIsosurfaceStuffingFast2007} can convert volumetric fields into meshes, the gaps from discrete sampling degrade the output quality in terms of both meshes and textures.


To address these issues, we present \textbf{EMA}, a method based on \textbf{E}fficient \textbf{M}eshy neural fields to reconstruct animatable human \textbf{A}vatars.
Our method enjoys flexibility from implicit representations and efficiency from explicit meshes, yet still maintains high-fidelity reconstruction quality.
Given video sequences and the corresponding pose tracking, our method digitizes humans in terms of canonical triangular meshes, physically-based rendering (PBR) materials, and skinning weights \textit{w.r.t.} skeletons.
We jointly learn the above components via inverse rendering~\cite{laineModularPrimitivesHighPerformance2020,chenDIBRLearningPredict2021,chenLearningPredict3D2019} in an end-to-end manner.
Each of them is derived from a separate neural field, which relaxes the requirements of a preset human template, rigging, or UV coordinates.
Specifically, we predict a canonical mesh out of a signed distance field (SDF) by differentiable marching tetrahedra~\cite{shenDeepMarchingTetrahedra2021,gaoGET3DGenerativeModel,gaoLearningDeformableTetrahedral2020,munkbergExtractingTriangular3D2022}, then we extend the marching tetrahedra~\cite{shenDeepMarchingTetrahedra2021} for spatial-varying materials by utilizing a neural field to predict PBR materials \textit{on the mesh surfaces} after rasterization~\cite{munkbergExtractingTriangular3D2022,hasselgrenShapeLightMaterial2022,laineModularPrimitivesHighPerformance2020}.
To make the canonical mesh animatable, we take another neural field to model the forward linear blend skinning for the meshes. 
Given a posed skeleton, the canonical mesh is then transformed into the corresponding poses.
Finally, we shade the mesh with a rasterization-based differentiable renderer~\cite{laineModularPrimitivesHighPerformance2020} and train our models with a photo-metric loss.
After training, we export the mesh with materials and discard the neural fields.

\looseness=-1
There are several merits of our method design.
1) \textbf{Efficiency}:
Powered by efficient mesh rendering, our method can render in real-time.
Besides, the training speed is boosted as well, 
since we compute loss holistically on the whole image and the gradients only flow on the mesh surface. In contrast, volume rendering takes limited pixels for loss computation and back-propagates the gradients in the whole space.
Our method only needs about an hour of training and minutes of optimization are enough for plausible avatar reconstruction.
2) \textbf{Disentangled representations}:
Our shape, materials, and motion modules are disentangled naturally by design, which facilitates editing. 
Besides, Canonical meshes with forward skinning modeling handle the out-of-distribution poses better.
3) \textbf{Graphics compatibility}:
Our derived mesh representation is compatible with 
the prominent graphic pipeline, which leads to instant downstream applications (\eg, the shape and materials can be edited directly in design software~\cite{foundationBlenderOrgHome}).
To further improve reconstruction quality, we additionally optimize image-based environment lights and non-rigid motions.


We conduct extensive experiments on standards benchmarks H36M~\cite{ionescuHuman36MLarge2014b} and ZJU-MoCap~\cite{pengNeuralBodyImplicit2021}.
Our method achieves very competitive performance for novel view synthesis, generalizes better for novel poses, 
and significantly improves both training time and inference speed against previous arts.
Our research-oriented code reaches real-time inference speed (100+ FPS for rendering $512\times512$ images).
We in addition showcase applications including novel pose synthesis, material editing, and relighting.
\section{Related Work} \label{sec:related work}
\vspace{-0.2cm}
{\noindent \bf Vision-Language Pre-training.} In the early literature, \cite{Mori99,Frome13,Weston11} explore jointly training image-text embeddings using paired text documents. Recently, some studies have further scaled up the training with large-scale web data to form ``the \textbf{foundation} models'', {\em e.g.}, CLIP~\cite{Radford21}, ALIGN~\cite{Jia21}, Florence~\cite{yuan2021florence}, FILIP~\cite{yao2021filip}, VideoCLIP~\cite{xu2021videoclip}, and LiT~\cite{zhai2022lit}. These foundation models usually contain one visual encoder and one textual encoder, which are trained using simple noise contrastive learning for powerful cross-modal representations. They have shown promising potential in many tasks, such as image classification and detection, action recognition, and retrieval. In this paper, we use CLIP for low-shot temporal action localization, but the same technique should be applicable to other foundation models as well.



\vspace{0.1cm}
{\noindent \bf Prompting} refers to leveraging input instructions to steer foundation models for desired outputs. In the NLP domain, early papers~\cite{Gao21,Jiang20,Timo21,Shin20} focus on handcrafted prompt templates. To avoid labor and increase flexibility, some studies~\cite{Lester21,li21-prefixtuning,li2021prefix} propose learnable prompt tuning at the textual stream, showing strong low-shot generalization. In the CV domain, some recent papers~\cite{zhou2019learn,zhou2022conditional,ju2022prompting} introduce such randomly initialized prompt tuning to handle visual tasks, {\em e.g.}, image understanding~\cite{zhu2022prompt,lu2022prompt,yang2022learning,ma2023diffusionseg} and video understanding~\cite{jia2022visual,nag2022zero,ni2022expanding}. However, these studies ignore lexical ambiguity of category names, and cases that are not easy to describe in text. This paper designs novel conditional prompt tuning and language descriptions from LLMs, to solve these issues. 



\vspace{0.1cm}
{\noindent \bf Closed-set Temporal Action Localization} considers to detect and classify action instances from one pre-defined category list. Specifically, existing methods can be divided into two popular supervisions, {\em i.e.}, strong~\cite{zeng2019graph,lin2021learning,qing2021temporal} and weak~\cite{wang2017untrimmednets,ju2023constraint,ju2020point,yudistira2022weakly}. Strong supervision gives precise boundary labels and category labels for training. There are two detailed pipelines: the top-down framework~\cite{shou2016temporal,shou2017cdc,gao2017turn,chao2018rethinking,lin2017single,xu2017r,tan2021relaxed,zhu2021enriching,wang2022rcl,xu2020g} pre-defines extensive anchors, adopts fixed-length sliding windows to produce initial proposals, then regresses to refine boundaries; the bottom-up framework~\cite{zhao2017temporal,lin2018bsn,lin2019bmn,vo2023aoe,zhao2020bottom,bai2020boundary} learns frame-wise boundary detectors for the boundary frames, then groups extreme frames or estimates action lengths for proposal generation. In addition, several works~\cite{gao2018ctap,liu2019multi,yang2020revisiting} used various fusion strategies to complement these frameworks. On the other hand, weak supervision trains without boundary labels to alleviate annotation costs. The video-level setting learns from category labels~\cite{paul2018w,ju2022distilling}, the CAS-based framework~\cite{liu2019completeness,ju2021adaptive,min2020adversarial,narayan2021d2,lee2019background,lee2021weakly,zhao2021soda} and attention-based framework~\cite{nguyen2018weakly,luo2021action,nguyen2019weakly,shi2020weakly,gao2022fine,he2022asm,huang2021foreground,luo2020weakly,ma2022weakly} have been well studied. To generate better results from CAS or attention, some studies~\cite{shou2018autoloc,liu2019weakly} improved post-processing. To balance cost and performance, some papers introduced single-frame annotations~\cite{ju2021divide,ma2020sf,lee2021learning,yang2021background,mettes2019pointly} or instance-number annotations~\cite{narayan20193c,xu2019segregated}. 

Nevertheless, all the above methods assume that action categories remain identical for training and testing, which is an over-simplification of real application scenarios, limiting practical uses of the vision system.



\vspace{0.1cm} 
{\noindent \bf Low-Shot Temporal Action Localization} considers more realistic scenarios: generalize TAL towards action categories that are unseen (zero-shot) or with several support samples (few-shot). Existing methods~\cite{ju2022prompting,nag2022zero,zhang2022ow,bao2022opental} most rely on foundational models pre-trained on large-scale image-caption pairs for help. Typically, E-Prompt~\cite{ju2022prompting} is the first to construct wide baselines with popular prompt tuning~\cite{Lester21,li21-prefixtuning} and vanilla temporal modeling. STALE~\cite{nag2022zero} explores the one-stage framework to further simplify usage. Although promising, all above methods meet two main challenges: (1) For category semantics, the definition may be vague, inaccurate, or incomplete. (2) For visual motions, temporal modeling may be insufficient. In this paper, for detailed category understanding, we design novel language descriptions from LLMs and vision-conditional prompt tuning; for clearer motion understanding, we introduce optical flows to provide explicit motion inputs. 




\section{Methods}



This paper aims to utilize pre-trained diffusion generation models for downstream tasks by proposing a two-stage synthesis-exploitation framework. 
In \secref{Problem Formulation}, we start by describing the preliminary. 
In \secref{Synthesizing Labeled Data}, we detail the synthesis stage to generate sufficient labeled data. 
In \secref{Diffusion Features and Synthetic Data Supervision}, we detail the exploitation stage to close the structural gap between generative models and discriminative tasks.










\subsection{Preliminary and Overview}
\label{Problem Formulation}
{\noindent \bf Problem Definition.} Object Discovery (OD), \ie, saliency segmentation and object localization, as a fundamental and typical discriminative task, is studied in this paper. 
Concretely, object discovery aims to train one pixel-level segmentation model $\Phi_{\mathrm{OD}}$ that partitions one image $\mathcal{I}$ into two disjoint groups, namely, foreground and background.
\vspace{-0.5em}
\begin{equation}
    \mathcal{M}_{\mathrm{seg}} = \Phi_{\mathrm{OD}}(\mathcal{I}) \in\{0,1\}^{H \times W \times 1}, \  \mathcal{I} \in \mathbb{R}^{H \times W \times 3}, 
\vspace{-0.5em}
\end{equation}
where $\mathcal{M}_{\mathrm{seg}}$ refers to the binary segmentation mask. 

Here, to clearly evaluate the effectiveness of our method, we focus on the strict \textit{unsupervised} setting, {\em i.e.}, the model is trained \textit{without} any manually annotated data. 


\vspace{0.1cm}
{\noindent \bf Motivation.} 
\hspace{1pt} This paper aims to exploit pixel-level visual knowledge from pre-trained diffusion generation models, for downstream discriminative tasks, \eg, OD. To achieve this goal, we design a novel synthesis-exploitation framework (\figref{fig:framework}). Specifically, at the synthesis stage, we explicitly construct one free (infinite-size) discriminative synthetic dataset, to obtain sufficient labeled samples. At the exploitation stage, we enable diffusion to be compatible with OD tasks, by extracting implicit diffusion features, and training one discovery decoder with the synthetic dataset.

























\vspace{0.2cm}
{\noindent \bf Diffusion}~\cite{sohl2015deep,ddpm} is one recently popular generative idea, containing forward and reverse processes. 
The \textit{forward process} is a Markov chain where noise is gradually added to the data.
The \textit{reverse process} is a denoising procedure that can be decomposed into a linear combination of a noisy image $\boldsymbol{x}_t$ and a noise approximator $\epsilon_\theta(\cdot)$. $t=1,\dots,T$ refers to the denoising timesteps.
The key to diffusion models is to learn the function $\epsilon_\theta(\cdot)$, typically using a UNet~\cite{ronneberger2015u}.


Particularly, we build on a variant of the text-to-image diffusion model, namely, Stable Diffusion~\cite{ldm}. During the synthesis process, it's sampled by iteratively denoising $\boldsymbol{x}_t$ conditioned on the input text prompt $y$ for timestep $t=1,\dots,T$. The conditional denoising UNet $\boldsymbol\epsilon_\theta(\boldsymbol{x}_t, t, y)$ stacks layers of self- and cross-attentions. 
$y$ is first encoded to text embeddings by a pre-trained text encoder, then text embeddings are mapped to intermediate layers as $K$ and $V$ via the attention mechanism, and the noisy image $\boldsymbol{x}_t$ is mapped as $Q$. 
For step $t$ and layer $l$, we call cross-attention as $\mathcal{A}_c^{t,l}$, self-attention as $\mathcal{A}_s^{t,l}$, and intermediate features as $\mathcal{F}^{t,l}$.
{\bf Note that}, this paper freezes Stable Diffusion pre-trained on LAION-5B~\cite{schuhmann2022laion} (5 billion image-text pairs), as a knowledge provider. This diffusion model involves both low-level object details and high-level class semantics, enabling us to achieve unsupervised object discovery.

































\subsection{Synthesis Stage:  Free Data Generation}
\label{Synthesizing Labeled Data}
As illustrated in \figref{fig:framework} (1), this stage aims to synthesize large and free image-mask pairs through Stable Diffusion, solving the lack of labeled training data under unsupervised settings. 
We detail image synthesis in \secref{sec:Image Generation}, and mask generation in \secref{synthetic mask generation}.
 





\vspace{-0.25cm}
\subsubsection{Image Generation}
\label{sec:Image Generation}
For one pre-trained text-to-image Stable Diffusion~\cite{ldm}, we here freeze it, then generate images through inputting random Gaussian noise and class text prompts. Class names are sampled from ImageNet~\cite{imagenet}. 



For text input, a simple way is to simply use class names, but this may limit diversity and cause bottlenecks for downstream tasks. 
Hence, to adaptively generate various text prompts for each class, we interact with ChatGPT~\cite{chatGPT}. 
For example, we ask ChatGPT to list prompts about ``aeroplane'', then it could give some generative-style prompts like: ``\textit{A aeroplane soaring through a vibrant sunset sky, fluffy clouds, warm lighting, viewed from a low angle, realistic style.}'' 
The generated prompts introduce richer context, thus can better unleash the potential of the Stable Diffusion to synthesis high-fidelity, more diverse images. One noise reduction strategy is also applied following~\cite{he2022synthetic}.


\vspace{-0.25cm}
\subsubsection{Mask Generation}
\label{synthetic mask generation}




















Here, we generate high-quality masks by leveraging attentions in pre-trained diffusion models as clues, following two non-trivial observations.
(1) Cross-attention $\mathcal{A}_c$ indicates locality between the conditioning text and noisy image, thus $\mathcal{A}_c$ can coarsely describe \textit{objectness}.
(2) Self-attention $\mathcal{A}_s$ inside one image indicates pairwise semantic similarity between pixels, thus $\mathcal{A}_s$ could roughly describe \textit{coherence}.
Inspired by these, we propose AttentionCut, a training-free strategy to generate masks guided by attention maps.







\vspace{0.2cm}
{\noindent \bf Preparations.}
We first extract $\mathcal{A}_c$ and $\mathcal{A}_s$ at the position of category token in the prompt sentence, then aggregate different resolutions and timesteps considering multi-scale objects and avoiding focus shift during diffusion.
Formally,
\vspace{-0.5em}
\begin{equation}
    \mathcal{A}_c=\frac{1}{kT}\sum_{l=1}^{k}\sum_{t=0}^{T-1} \mathcal{A}_c^{t,l}; \
    \mathcal{A}_s=\frac{1}{LT}\sum_{l=1}^{L}\sum_{t=0}^{T-1} \mathcal{A}_s^{t,l}, 
    \vspace{-0.5em}
\end{equation}
where $t=T-1, \dots, 0$ is for each reverse step and $l = 1, \dots, L$ is for intermediate layers. $\mathcal{A}_c$ is averaged among the top-$k$ of the standard variation from all $\mathcal{A}_c^l$, while $\mathcal{A}_s$ is averaged among all layers and time steps.


\vspace{0.2cm}
{\noindent \bf Objectness.}
Intuitively, the pixel-level cross-attention $\mathcal{A}_c$ under a specific category can roughly be seen as segmentation masks, as it indicates how likely a pixel belongs to the category. However, in practice we found $\mathcal{A}_c$ is sparse and inattentive near the boundary, which can seriously damage segmentation results. 
To handle this issue, we improve $\mathcal{A}_c$ by strengthening the edge area with the self-attention $\mathcal{A}_s$. It indicates semantic connectivity, \ie, how semantically two pixels belong to one group. 
Specifically, we first randomly select a set of initial seeds $\mathcal{B}$ from the boundary of the binary mask $\left[\mathcal{A}_c>\tau\right]$. 
Then each selected seed $b \in \mathcal{B}$ can expand as a confidence map $\mathcal{A}_s(b, \cdot)$, which is the self-attention between $b$ and other pixels, indicating weights of the boundary area. 
We assume $\mathcal{A}_s(\cdot, b)=\mathcal{A}_s(b, \cdot)$, as $\mathcal{A}_s$ is symmetric theoretically.
For pixel $p$, these maps are averaged as a refined map $r(p)$, to reinforce the boundary pixels: 
\vspace{-0.5em}
\begin{equation}
    r(p) = 1/{|\mathcal{B}|}\cdot \sum\nolimits_{b \in {B}} \mathcal{A}_s(p,b).
\vspace{-0.5em}
\end{equation}

Combining cross-attention $\mathcal{A}_c$ and the refined map $r(p)$ with a balance weight $\lambda_\phi$, the pixel-level objectness $\phi$ are:
\vspace{-0.5em}
\begin{equation}
    \phi(p) = \left\{
    \begin{aligned}
        -\log(\mathcal{A}_c(p)+\lambda_\phi r(p)),\  \text{if }p \in\text{foreground},\\
        \log(1-\mathcal{A}_c(p)-\lambda_\phi r(p)),\  \text{if }p\in\text{background},
    \end{aligned}
    \right.
    \vspace{-0.5em}
\end{equation}
where $\mathcal{A}_c(p)$ is the cross-attention at pixel $p$.


\vspace{0.2cm}
{\noindent \bf Inner Coherence.}
With only objectness, we found that the masks tend to lose local information, for example, irregular corners, mis-segmented holes, or jagged contours.
This can be solved by taking local consistency into account, \ie, how likely two neighboring pixels belong to one group.
Here we design an inner coherence term that can help to enforce continuity, proximity and smoothness of segments belonging to the same object, and penalize those who deviate.




The proposed inner coherence consists of two parts: semantic and spatial.
As mentioned above, $\mathcal{A}_s$ can indicate semantic coherence, as self-attention is calculated in semantic feature space.
Spatial coherence is designed to indicate pixels pairwise distance in both RGB and Euclidian space.
This coherence is obtained by absorbing the form of geodesic distance on the surface of image intensity, then by negative exponential transformation.
The inner coherence $\psi$ can be formalized as:
\vspace{-0.5em}
\begin{equation}
    \begin{aligned}
        \psi(p,q) &= \mathcal{A}_s(p,q) + \lambda_\psi e^{-\mathcal{D}(p,q)},\\
        \mathcal{D}(p,q) &= \min_{P}\int_0^1\|\nabla I\left(P(s)\right)\cdot v(s)\|ds,
    \end{aligned}
    \label{eq:psi}
    \vspace{-0.5em}
\end{equation}
where for pixel $p$ and $q$, $\mathcal{A}_s(p,q)$ is the self-attention and $\mathcal{D}(p,q)$ is the geodesic distance;
$P$ is an arbitrary path from $p$ to $q$ parameterized by $s\in[0,1]$;
$v(s)$ denotes the unit vector $P'(s)/\|P'(s)\|$ that is tangent to the path direction; $I(\cdot)$ is image RGB intensity.






\vspace{0.2cm}
{\noindent \bf Calculating Mask.}
Given objectness and inner coherence, we define an energy function $E$ for each potential mask $\mathcal{M}$:
\vspace{-0.5em}
\begin{equation}
E(\mathcal{M}) = \sum\nolimits_p\phi(p)+\lambda\sum\nolimits_{\mathcal{M}(p)\neq\mathcal{M}(q)}\psi(p,q),
\vspace{-0.5em}
\end{equation}
where $\lambda$ denotes the weight between $\phi$ and $\psi$; $\mathcal{M}(\cdot)\in\{0,1\}$ means the pixel in this mask. The binary mask $\mathcal{M}$ is generated by minimizing $E(\mathcal{M})$, \ie, use Ford-Fulkerson algorithm~\cite{ford1956maximal} to find a minimum cut in the image graph. And after further post-processing and denoising~\cite{barron2016fast, zhang2021datasetgan,li2022bigdatasetgan}, we can obtain the final synthetic mask (see \figref{fig:teaser} Right for some examples).









\vspace{0.2cm}
{\noindent \bf Discussion.}
Compared with other training-free mask generation methods like NCut~\cite{shi2000normalized} and K-means~\cite{lloyd1982},
they only consider pairwise similarly, thus cannot decide fore/background for each partition. 
Compared with DenseCRF~\cite{NIPS2011_beda24c1},
AttentionCut has well-designed objectness and inner coherence terms, which is more suitable for diffusion models and guarantees convergence.
In \tabref{tab:raw_cut}, we have conducted experiments to validate the superiority of AttentionCut.



\subsection{Exploitation Stage: Diffusion Knowledge }
\label{Diffusion Features and Synthetic Data Supervision}


This stage aims to bridge the architectural gap between pre-trained diffusion models and discriminative tasks, \eg, object discovery. 
As shown in \figref{fig:framework} (2), we achieve this in two steps: in \secref{Extracting Diffusion Features}, we treat diffusion models as a universal feature extractor to distill explicit visual knowledge; in \secref{Segment Decoder}, we feed diffusion features into one flexible decoder, and train with ``infinite'' synthetic data.









\subsubsection{Extracting Diffusion Knowledge}
\label{Extracting Diffusion Features}






For diffusion models, they are fed with noise and text to output synthesis images; while for object discovery models, they are fed with images to output pixel-level masks. Such an architectural gap blocks direct feature extraction from diffusion. 
To solve this, given one image, we are required to find the corresponding input noise of diffusion models under some conditioning text, then features can be extracted through diffusion reverse process.
To get input noise, we combine diffusion inversion~\cite{ddim}
with the conditional UNet. To get the conditioning text, we simply classify images by CLIP~\cite{clip}.











\vspace{0.2cm}
{\noindent \bf Diffusion Inversion and Feature Extraction.} 
Given pre-trained diffusion models, we here inverse one image back to its corresponding noise under the conditioning text. 
This diffusion inversion can be seen as a special forward process.





One trivial solution is to use the typical DDPM~\cite{ddpm}. Although it can yield latent variables (\ie, noise) through the forward process, 
these variables are stochastic and cannot reconstruct the image through the reverse process.
So it is not suitable for feature extraction.
Inspired by DDIM~\cite{ddim}, we modify each step by combining it with conditional denoising UNet $\boldsymbol\epsilon_\theta(\boldsymbol{x}_t, t, y)$ in Stable Diffusion, making the forward/reverse non-Markovian to enjoy deterministic. 
Now the forward/reverse process for each step is:
\vspace{-0.5em}
\begin{equation}
\resizebox{0.90\linewidth}{!}{$
\begin{aligned}
    &\boldsymbol{x}_{t+1}=\sqrt{\alpha_{t+1}} \boldsymbol{f}_\theta\left(\boldsymbol{x}_t, t, y\right)+\sqrt{1-\alpha_{t+1}} \boldsymbol{\epsilon}_\theta\left(\boldsymbol{x}_t, t, y\right),
    \\
    &\boldsymbol{x}_{t-1}=\sqrt{\alpha_{t-1}} \boldsymbol{f}_\theta\left(\boldsymbol{x}_t, t,y\right)+\sqrt{1-\alpha_{t-1}} \boldsymbol{\epsilon}_\theta\left(\boldsymbol{x}_t, t,y\right),
    \end{aligned}$
    }
    \label{equ: deterministic reverse}
    \vspace{-0.5em}
\end{equation}
where $\boldsymbol{f}_\theta\left(\boldsymbol{x}_t, t, y\right)=\left({\boldsymbol{x}_t-\sqrt{1-\bar{\alpha}_t} \boldsymbol{\epsilon}_\theta\left(\boldsymbol{x}_t, t, y\right)}\right)\,/\, {\sqrt{\bar{\alpha}_t}}$, $\alpha_t=1-\beta_t$, $\bar{\alpha}_t=\prod_{s=1}^t\left(1-\beta_s\right)$, $\beta_t$ is a variance schedule. $y$ denotes the conditional text, and $t$ means timesteps.

After diffusion inversion, to get the corresponding noise, features $\mathcal{F}^{t,l}$ can be extracted from $\boldsymbol\epsilon_\theta(\boldsymbol{x}_t, t, y)$ during each reverse step $t=T-1, \dots, 0$ and intermediate layer $l = 1, \dots, L$.
To cover long range and multi-level features of multi-scale objects, they are aggregated in all time steps:
\vspace{-0.5em}
\begin{equation}
\mathcal{F}^l=1/T\cdot\sum\nolimits_{t=0}^{T-1} \mathcal{F}^{t,l}.
\vspace{-0.5em}
\end{equation}
In practice, we choose the output of the ``SpatialTransformer'' block in Stable Diffusion, where $L=6$ with resolutions $16 \times 16$, $32 \times 32$, and $64 \times 64$, two of each.


\begin{table*}
\begin{center}
\caption{Comparison with \sota\ methods on the public crowd analysis benchmarks: \jhu, ShanghaiTech, UCF, and \nwpu. 
The best results are shown in \first{red}. The second-best results are shown in \second{blue}. 
}
\vspace{\tablegap}
\resizebox{0.95\textwidth}{!}{
\begin{tabular}{l c c c c c c c c c c c c c}
\toprule
 \multirow{2}{*}{Method} & \multirow{2}{*}{Venue} &\multicolumn{2}{c}{\jhu} &\multicolumn{2}{c}{\shha} &\multicolumn{2}{c}{\shhb} &\multicolumn{2}{c}{\ucf} &\multicolumn{2}{c}{\qnrf} &\multicolumn{2}{c}{\nwpu}\\[0.2ex]
 \cmidrule(lr){3-4}\cmidrule(lr){5-6}\cmidrule(lr){7-8}\cmidrule(lr){9-10}\cmidrule(lr){11-12}\cmidrule(lr){13-14}
& & MAE$\downarrow$ & MSE$\downarrow$ & MAE$\downarrow$ & MSE$\downarrow$ & MAE$\downarrow$ & MSE$\downarrow$ & MAE$\downarrow$ & MSE$\downarrow$ & MAE$\downarrow$ & MSE$\downarrow$ & MAE$\downarrow$ & MSE$\downarrow$\\[0.2ex]
\midrule\midrule
TopoCount \cite{abousamra2021localization}	& AAAI'21	& {60.9}	& {267.4}	& {61.2}	& {104.6}	& {7.8}	& {13.7}	& {184.1}	& {258.3}	& {89.0}	& {159.0}	& {107.8}	& {438.5}	\\[0.2ex]
SUA \cite{meng2021spatial}	& ICCV'21	& {80.7}	& {290.8}	& {68.5}	& {121.9}	& {14.1}	& {20.6}	& {-}	& {-}	& {130.3}	& {226.3}	& {111.7}	& {443.2}	\\[0.2ex]
ChfL \cite{shu2022crowd}	& CVPR'22	& {57.0}	& {235.7}	& {57.5}	& {94.3}	& {6.9}	& {11.0}	& {-}	& {-}	& {80.3}	& {137.6}	& {76.8}	& {343.0}	\\[0.2ex]
MAN \cite{lin2022boosting}	& CVPR'22	& {53.4}	& \second{209.9}	& {56.8}	& {90.3}	& {-}	& {-}	& {-}	& {-}	& {77.3}	& {131.5}	& {76.5}	& {323.0}	\\[0.2ex]
GauNet \cite{cheng2022rethinking}	& CVPR'22	& {58.2}	& {245.1}	& {54.8}	& {89.1}	& {6.2}	& {9.9}	& {186.3}	& {256.5}	& {81.6}	& {153.7}	& {-}	& {-}	\\[0.2ex]
CLTR \cite{liang2022end}	& ECCV'22	& {59.5}	& {240.6}	& {56.9}	& {95.2}	& {6.5}	& {10.6}	& {-}	& {-}	& {85.8}	& {141.3}	& {74.3}	& {333.8}	\\[0.2ex]
CrwodHat \cite{wu2023boosting}	& CVPR'23	& \second{52.3}	& {211.8}	& {51.2}	& {81.9}	& \first{5.7}	& {9.4}	& {-}	& {-}	& {75.1}	& \second{126.7}	& {68.7}	& \second{296.9}	\\[0.2ex]
STEERER \cite{han2023steerer}	& ICCV'23	& {54.3}	& {238.3}	& {54.5}	& {86.9}	& {5.8}	& \second{8.5}	& {-}	& {-}	& {74.3}	& {128.3}	& \second{63.7}	& {309.8}	\\[0.2ex]
PET \cite{liu2023point}	& ICCV'23	& {58.5}	& {238.0}	& \second{49.3}	& \second{78.8}	& {6.2}	& {9.7}	& {-}	& {-}	& {79.5}	& {144.3}	& {74.4}	& {328.5}	\\[0.2ex]
\rowcolor{black!10}\method\	& 	& \first{47.3}	& \first{198.9}	& \first{47.4}	& \first{75.0}	& \first{5.7}	& \first{8.2}	& \first{160.8}	& \first{225.0}	& \first{68.9}	& \first{125.6}	& \first{57.8}	& \first{221.2}	\\[0.2ex]
\bottomrule
\end{tabular}
}
\vspace{\tablegap}
\label{table: crowd counting performance}
\end{center}
\end{table*}


\vspace{0.2cm}
{\noindent \bf CLIP-classifiable Prior.}
Notice that in \equref{equ: deterministic reverse}, the diffusion inversion should be done under some conditional text $y$.
We choose $y$ to be the CLIP-classified category of the input image, because of the following observations:
(1) humans take pictures by naturally framing an object of interest near the center of the image~\cite{judd2009learning} (center prior); 
(2) most background regions can be easily connected to image boundaries, while difficult for object regions~\cite{wei2012geodesic} (background prior); 
(3) CLIP is pre-trained on a large corpus of web-curated data, and most of which is human-token images with saliency objects~\cite{clip} (source prior). 
It is easy to classify images with the center and background priors, and the source prior enables us to classify using CLIP~\cite{clip}.
We summarize this as \textit{CLIP-classifiable prior}.

In practice, we choose the label set in ImageNet~\cite{imagenet}, and combine semantically similar classes, \eg, poodle and Chihuahua as dogs, etc.
Besides, multiple prompt templates are used, \eg, ``A photo of \{category\}'' to boost performance.



\subsubsection{Segment Decoder}
\label{Segment Decoder}
To enable diffusion compatible with object discovery, we here propose two options for preference. 
One is to attach a flexible decoder to the pre-trained diffusion models, and train using the synthesised data to achieve object discovery. This option costs many parameters and rich training data, bringing superior performance, and we denote it as {\textit{DiffusionSeg}} in  {\tabref{table:main_all}}. 
The other is to extract cross- and self-attention during diffusion inversion, and generate pseudo-masks using AttentionCut in \secref{synthetic mask generation}.
Such an option costs no trainable parameters and data, thus showing faster inference speeds, and we call it {\textit{AttentionCut}} in \tabref{table:main_all}. 

 
















\subsection{Discussion}
This paper uses pre-trained diffusion models for unsupervised object discovery.
Comparing with discriminative pre-training~\cite{lost,tokencut,selfmask}, generative pre-training has additional pixel-level understanding, which is more suitable for object discovery. 
Compared with MAE-style~\cite{mae} generative pre-training, which learns reconstruction representations to help object discovery, diffusion models show a clear advantage, \ie, synthesis abundant data, which is valuable to improve performance (see \tabref{tab:train_syn} and \figref{fig:scale}). 
Comparing with GANs in image synthesizing, diffusion models have significant advantages in higher sample quality and diversity, more stability and robustness~\cite{dhariwal2021diffusion}.
Compared to a few early GAN-based works that struggle to synthesise  mask with manual annotations~\cite{zhang2021datasetgan,li2022bigdatasetgan}, diffusion model can obtain mask using AttentionCut, without manually labeling.














\section{Experiments}


\subsection{Datasets}


\begin{figure}[t]
    \centering
  \includegraphics[width=0.93\linewidth]{pictures/dataset.pdf}
    \caption{\textbf{Demonstration of distortion image.} The three columns from left to right are our PDHuman dataset, HuMMan dataset, and SPEC-MTP dataset, respectively. The value from the arrow (yellow) indicates the distortion scale of the pixel.}
    \label{fig:dataset}
    % \vspace{-10pt}
\end{figure}

\noindent\textbf{PDHuman.}
%
Despite perspective distortion being a common problem, no existing public dataset is specifically designed for this task.
%
Inspired by recent synthetic datasets~\cite{synbody,gta,bedlam,Patel:CVPR:2021:AGORA}, we introduce a synthetic dataset named PDHuman. The dataset contains $126,198$ images in the training split and $27,448$ images in the testing split, with annotations including camera intrinsic matrix, 2D/3D keypoints, SMPL parameters ($\boldsymbol{\theta}$, $\boldsymbol{\beta}$), and translation for each image. The testing split is further divided into 5 protocols by the max distortion scale of each image sample.
We define the max distortion scale for each sample as $\tau$; this value will be used in splitting protocols.
% \todo{define max distortion scale with $\tau$?}

We use $630$ human models from RenderPeople~\cite{renderpeople} and $1,710$ body pose sequences from Mixamo~\cite{mixamo}, with 500 HDRi images with various lighting conditions as backgrounds. 
% For data generation, we collected 630 photogrammetry-scanned human models from
% % Renderpeople\footnote[2]{\url{https://renderpeople.com}}
% Renderpeople \cite{renderpeople}, with SMPL parameters well fitted. 
% We collected body pose sequence from a real MoCap sequence in 
% % Mixamo\footnote[3]{\url{https://www.mixamo.com/}},
% Mixamo \cite{mixamo},
% then re-targeted the pose to SMPL skeleton. We use HDRi images as backgrounds. 
We use the dolly-zoom effect to generate random camera extrinsic and intrinsic matrices with random rotations, translations, and focal lengths.
The distance from the human body to the cameras is set from 0.5m to 10m, so our dataset contains severely distorted, slightly distorted, and nearly non-distorted images.
Then we use Blender~\cite{blender} to render the RGB images. See \cref{fig:dataset} for brief demonstration.
For detailed rendering procedures and more image demonstrations, please refer to Sup. Mat.

% For test split, we calculated the max distortion scale of each sample and divided the images into five protocols by distortion scale.

% For better re-targeting results, we saved the 3d vertices of every frame and used SGD loop optimization to optimize the SMPL parameters.
% Our pipeline is inspired by recent works on synthetic data \cite{Patel:CVPR:2021:AGORA,hspace,varol17_surreal}. 


\noindent\textbf{SPEC-MTP and HuMMan Datasets.}
SPEC-MTP dataset~\cite{spec} is proposed to test human pose reconstruction in world coordinates. It includes many close-up shots, mostly taken from below or overhead views, leading to images with distorted human bodies. HuMMan dataset~\cite{humman} is captured by multi-view RGBD cameras and has accurate ground truth because the SMPL parameters are fitted based on 3D keypoints and point clouds. HuMMan also contains images with distorted human bodies since the actors were close to the cameras, all less than 3 meters away. In our paper, we extend both datasets into real-world perspective-distorted datasets. SPEC-MTP is used only for testing. For HuMMan, we split it into training and testing parts. When testing, we divide these two datasets into three protocols based on their maximum distortion scale $\tau$. See \cref{fig:dataset} for brief demonstration.

% The SPEC-MTP dataset is proposed in \cite{spec} mainly used to test the human pose reconstruction in world coordinates. It contains many close-up shot, taken from below or overhead views. This leads to many images of distorted human bodies.
% The HuMMan dataset is proposed in \cite{humman}, it is an indoor dataset captured by multi-view RGBD cameras, and the SMPL parameters were fitted using triangulated 3D key-points and point-clouds. So its ground-truth was very accurate. It also contains some images with distorted human bodies since the actors were very close to the cameras, often less than 3 meters. 
% We extend the SPEC-MTP~\cite{spec} and HuMMan~\cite{humman} dataset as real world perspective distorted datasets.
% Following SPEC~\cite{spec}, SPEC-MTP is only used for testing in our paper. And HuMMan is split into training and testing parts following its original set.
% For SPEC-MTP and HuMMan, we calculate the max distortion scale of each image sample and divide the images into three protocols by distortion scale.


\noindent\textbf{Non-distorted Datasets.}
For non-distorted datasets, we use Human3.6M~\cite{h36m}, COCO~\cite{coco}, MPI-INF-3DHP ~\cite{mpi-inf3dhp} and LSPET~\cite{lspet} as our training data.
Following~\cite{meshgraphormer, fastmetro}, we also report the results fine-tuned on 3DPW~\cite{3dpw} training data.

\subsection{Evaluation Metrics}
To measure the accuracy of reconstructed human mesh, we follow the previous works~\cite{hmr,spec,cliff} by adopting MPJPE (Mean Per Joint Position Error), PA-MPJPE (Procrustes Analysis Mean Per Joint Position Error) and PVE (Per Vertex Error) as our 3D evaluation metrics. They all measure the Euclidean distances of 3D points or vertices between the predictions and ground truth in millimeters (mm).

To measure the re-projection results in perspective distorted datasets such as PDHuman, SPEC-MTP and HuMMan, we leverage metrics widely used in segmentation tasks, MeanIoU~\cite{pascal} as our 2D metric. We both report foreground and background MeanIoU marked as mIoU and body part MeanIoU marked as P-mIoU. We use the 24-part vertex split provided by official SMPL~\cite{smpl} for body part segmentation. During the evaluation, for weak-perspective methods like HMR, we will render the predicted segmentation masks with a focal length of $5,000$ pixels. And we use the corresponding focal length on methods with specific camera models, such as SPEC~\cite{spec}, CLIFF~\cite{cliff}, and proposed \Ours.


\begin{table*}[t]
    \centering
    \scalebox{0.61}{\begin{tabular}{lccccc|ccccc|ccccc}
\toprule
\multirow{2}{*}{\textbf{Methods}} & \multicolumn{5}{c}{PDHuman~($\tau=3.0$)}& \multicolumn{5}{c}{SPEC-MTP~($\tau=1.8$)}
&\multicolumn{5}{c}{HuMMan~($\tau=1.8$)}\\ \cmidrule{2-16}
&PA-MPJPE$\downarrow$ & MPJPE$\downarrow$  & PVE$\downarrow$ & mIoU$\uparrow$ &P-mIoU$\uparrow$& PA-MPJPE$\downarrow$ & MPJPE$\downarrow$ & PVE$\downarrow$ & mIoU$\uparrow$ &P-mIoU$\uparrow$& PA-MPJPE$\downarrow$ & MPJPE$\downarrow$ & PVE$\downarrow$ &  mIoU$\uparrow$  & P-mIoU$\uparrow$ \\ \midrule 

\rule{0pt}{10pt} HMR~(R50)~\cite{hmr} &62.5    &91.5   & 106.7  & 48.9 & 21.7  & 73.9& 121.4  & 145.6  & 48.8  & 16.0  & 30.2 &  43.6  & 52.6 & 65.1   & 39.5 \\

\rule{0pt}{10pt} HMR-$f$~(R50)~\cite{hmr}& 61.6 & 90.2 &105.5 & 45.2& 20.4 & 72.7 & 123.2 & 145.1 & 52.3  & 20.1  & 29.9 & 43.6& 53.4 & 62.7& 34.9  \\

\rule{0pt}{10pt} SPEC~(R50)~\cite{spec}  & 65.8&  94.9 &109.6 & 43.4   & 19.6    & 76.0 & 125.5 &144.6 &  49.9 & 18.8   & 31.4  & 44.0 & 54.2 & 51.4 & 25.6       \\

\rule{0pt}{10pt} CLIFF~(R50)~\cite{cliff} &66.2 & 99.2   &  115.2  &51.4  & 24.8   & 74.3   &  115.0 & 132.4 & 53.6  &23.7  &28.6 &42.4 &50.2  &68.8   & 44.7\\



\rule{0pt}{10pt} PARE~(H48)~\cite{pare}   & 66.3& 95.9   & 116.7 & 48.2    & 20.9   &74.2& 121.6     & 143.6 & 55.8   & 23.2    & 32.6 & 53.2& 65.5  & 66.5 & 38.3  \\

% CLIFF~(H48)~\cite{cliff} &   &  &  &    &   &  &   &  &   &  &   &   &   & \\
% PARE (H48) &   &  &    & &   &  &   & & \\

% PyMarf~\cite{pymaf}   &   &  &   &  &   &   &   &  &   &  &   &    &   &  \\
% HybrIK~\cite{hybrik}   &   &  &   &  &   &   &   &  &   &  &   &  &   &    \\
\midrule
\rule{0pt}{10pt} GraphCMR~(R50)   & 62.0  & 85.8  & 98.4  & 47.9    & 21.5  &  76.1 & 121.4    & 141.6  &  53.5 & 22.0  & 29.5 &  40.6 & 48.4  & 61.6 & 37.5   \\

\rule{0pt}{10pt} FastMETRO (H48)~\cite{fastmetro}   & 58.6 & 83.6   & 95.4 & 50.1  & 22.5   & 75.0   & 123.1  & 137.0 & 53.5  & 20.5  & 26.3  & 38.8 & 45.5  & 68.3 & 45.2    \\

\midrule
\rule{0pt}{10pt} \Oursp (R50)  & 54.3  & 80.9  & 93.9  & \textbf{54.5} &  \textbf{27.4} & 72.9 & 117.7  &  138.2 & 54.7 & 22.4 &24.4  &  36.7& 45.9 & 70.4 & \textbf{45.4}   \\
\rule{0pt}{10pt} \Ours (R50)  & 54.3&  76.4  & 87.6 & 51.4 &  24.0  & 74.0& 122.1 & 135.5  & 58.9  & 24.9  & 25.5 & 36.7 & 43.4 & 67.0 & 38.4  \\
\rule{0pt}{10pt} \Ours (H48)  & \textbf{49.9} &\textbf{70.7}  & \textbf{82.0}  & 53.0 & 26.5 &  \textbf{67.4} & \textbf{114.6} &  \textbf{126.7}& \textbf{62.3}  & \textbf{30.4} & \textbf{22.3} &\textbf{32.6} &  \textbf{40.0} & \textbf{71.2} & 45.1      \\
\bottomrule
\end{tabular}
}
    
    \caption{Results of SOTA methods on PDHuman, SPEC-MTP~\cite{spec} and HuMMan~\cite{humman} datasets. Here we report the largest distortion protocol. 
    % $\rm TH_{d}^{3.0}$ represents the threshold of 3.0 distortion scale, which means the maximum distortion scale in all the images are larger than 3.0. 
    R50 terms ResNet-50~\cite{resnet}, and H48 terms HRNet-w48~\cite{hrnet} here.}
    \label{tab:sota}
\end{table*}

\subsection{Implementation Details}
Unless specified, we use ResNet-50~\cite{resnet} and HRNet-w48~\cite{hrnet} backbones for model-free \Ours. 
%
We also design a model-based variant, \Oursp ($\cP$ stands for parametric), by changing the mesh reconstruction module to a model-based pose and shape estimation module. The details of \Oursp can be found in the Sup. Mat.
% For our , we mainly report results using ResNet-50~\cite{resnet}. 
%
All backbones are initialized by COCO~\cite{coco} key-point dataset pre-trained models.
%
We use Adam~\cite{adam} optimizer with a fixed learning rate of $2e^{-4}$. 
%
All experiments of \Ours are conducted on 8 A100 GPUs for around 160 epochs, 14$\sim$18 hours. 
%
 Our training pipeline was built based on MMHuman3D~\cite{mmhuman3d} code base.
For samples with ground-truth focal length and translations, we render IUV and distortion images online during the training by PyTorch3D~\cite{pytorch3d}.

For comparison on PDHuman, SPEC-MTP, and HuMMan, we follow the official codes of HMR~\cite{hmr}, SPEC~\cite{spec}, PARE~\cite{pare}, GraphCMR\cite{graphcmr}, FastMETRO\cite{fastmetro}. We re-implemented CLIFF~\cite{cliff} since the authors have not released the training codes. All SOTA methods are trained on 8 A100 GPUs until convergence, following the officially released hyper-parameters. 
% To ensure a fair comparison, all methods use the pre-trained backbone weights provided by HMR-Benchmark~\cite{hmr-benchmark}
All methods are trained on the same datasets with the same proportion, \eg, 
Human3.6M~\cite{h36m} (40\%), PDHuman (20\%), HuMMan~\cite{humman} (10\%), MPI-INF-3DHP~\cite{mpi-inf3dhp} (10\%), COCO~\cite{coco} (10\%), LSPET~\cite{lspet} (5\%).

% occupies a proportion of 40\%, 20\%, 15\%, 10\%, 10\%, 5\%, respectively.

% For \Ours, we use the ResNet-50~\cite{resnet} and HRNet-w48~\cite{hrnet} backbone both. For our model-based variant \Oursp, we mainly report the ResNet-50~\cite{resnet} results. Following
% HMR-Benchmark~\cite{hmr-benchmark}, we use the backbone weights pre-trained on COCO~\cite{coco} both for ResNet-50 and HRNet-w48 backbones. The Adam~\cite{adam} optimizer with a fixed learning rate of $2e^{-4}$ is used. With 8 A100 GPUs, total training takes around 160 epochs, 14$\sim$18 hours.
% During training, for datasets with ground-truth focal length and translations, we will render the ground-truth uv image and distortion image by PyTorch3D~\cite{pytorch3d}. We use the uv topology provided by DecoMR~\cite{decomr}.

% For comparison of PDHuman, SPEC-MTP and HuMMan, we conducted experiments on HMR~\cite{hmr}, SPEC~\cite{spec} PARE~\cite{pare}, GraphCMR~\cite{graphcmr}, FastMetro~\cite{fastmetro} by modifying their official codes. For SPEC~\cite{spec}, we load their pre-traind CamCalib network and estimate the focal length of the full image, then we transform the focal length to the cropped image. Since we compare all the 3d metrics in camera coordinates, we set the extrinsic rotation matrix estimated by CamClib as identity. We also re-implemented CLIFF~\cite{cliff} following their paper's configuration since they have not released their training codes.
% For fair comparison, all the methods are trained on the same dataset with the same proportion as \Ours. Our training pipeline was built based on MMHuman3D~\cite{mmhuman3d} code base. We use the same pre-trained backbone weights provided by HMR-Benchmark~\cite{hmr-benchmark} for all the methods we compare.
% All the SOTA methods are trained on 8 A100 GPUs until convergence, following the original hyper-parameters in the official released codes.


% MPJPE (Mean Per Joint Position Error) first aligns the predicted and ground-truth 3D joints at the pelvis, and then calculates their distances, which comprehensively evaluates the predicted poses and shapes, including the global rotations.

% PA-MPJPE (Procrustes-Aligned Mean Per Joint Position Error, or reconstruction error) performs Procrustes alignment before computing MPJPE, which mainly measures the articulated poses, eliminating the discrepancies in scale and global rotation.

% PVE (Per Vertex Error, or MVE used in the AGORA evaluation) does the same alignment as MPJPE at first, but calculates the distances of vertices on the human mesh surfaces.


% We also extend our method to a parametric-based method by simply changing the mesh reconstruction head to the pose and shape estimation head. We call this variation \Oursp. We also report results of $\rm POET^{\cP}$ in the experiments. The detailed pipeline will be demonstrated in Sup. Mat..

For distorted datasets, we only report the results on the protocols with the largest distortion scales. The full results of all protocols are shown in Sup. Mat.

\subsection{Main Results}
\noindent\textbf{Results on PDHuman, SPEC-MTP, and HuMMan.}
%
We report PA-MPJPE, MPJPE, PVE, mIoU, and P-mIoU on these three datasets. For model-based methods, we compare with HMR~\cite{hmr}, SPEC~\cite{spec}, CLIFF~\cite{cliff}, PARE~\cite{pare}. We compare model-free methods with GraphCMR~\cite{graphcmr} and FastMETRO~\cite{fastmetro}.
From ~\cref{tab:sota}, we can see that SPEC~\cite{spec} performs poorly on these distorted datasets. This is mainly due to their wrong focal length assumption, which has negative rather than positive effects on their supervision. (Note that our re-implemented SPEC has higher performance than the official code, see in Sup. Mat.) CLIFF performs well on SPEC-MTP, while badly on PDHuman. Because their focal length assumption is about $53^{\circ}$ for 16:9 images, close to SPEC-MTP images. Although HMR-$f$ is trained with the same focal length as \Ours, it improves little compared to HMR since they have not encoded the distortion or distance feature into their network. \Ours-H48 outperforms SOTA methods on most metrics, especially the 3D ones. Some 2D re-projection metrics,  \eg mIoU and P-mIoU, of \Ours-H48 are lower than~\Oursp-R50 version. We conjecture that model-based methods have better reconstructed shapes.
Please refer to \cref{fig:sota_demo} for qualitative results.
More qualitative results and failure cases can be found in Sup. Mat.


\begin{figure*}[htp]
  \centering
\includegraphics[width=0.97\linewidth]{pictures/sota.pdf}
  \caption{\textbf{Qualitative results of SOTA methods.} Besides \Ours, we visualize the results of three methods with specific camera models: HMR~\cite{hmr}, SPEC~\cite{spec}, CLIFF~\cite{cliff}. \Oursp terms our model-based variance. We show results come from different data sources. Row 1: PDHuman test. Row 2, 3: web images. Row 4: SPEC-MTP. The number under each image represents predicted/ground-truth $f$, FoV angle, and $T_{z}$. The ground-truth $f$ and $T_{z}$ for SPEC-MTP are pseudo labels. The focal lengths here are all transformed to pixels in full image.}
  \label{fig:sota_demo}
\end{figure*}

\begin{table}
    \centering
     \scalebox{0.7}{
        \begin{tabular}{lccccc}
        \toprule
        \multirow{2}{*}{\textbf{Methods}} &\multirow{2}{*}{Backbone}&
        \multirow{2}{*}{w. 3DPW} 
        & \multicolumn{3}{c}{Metrics}\\ \cmidrule{4-6}
        & &  & PA-MPJPE$\downarrow$ & MPJPE$\downarrow$ & PVE$\downarrow$  \\ 
        \midrule 

        HybrIK\cite{hybrik} &  ResNet-34 & $\times$ & 48.8  & 80.0  & 94.5  \\
        HybrIK\cite{hybrik} &  ResNet-34 &$\checkmark$ & 45.0  & 74.1  & 86.5  \\
        GraphCMR~\cite{graphcmr} & ResNet-50  & $\times$  &70.2  & - &    - \\
        HMR~\cite{hmr} & ResNet-50 & $\times$ &72.6  &  116.5 & -         \\
        SPIN~\cite{spin}  &  ResNet-50 &$\times$ & 59.2     & 96.9    &116.4\\
        PyMAF~\cite{pymaf}  & ResNet-50  & $\times$ & 58.9 & 92.8  & 110.1   \\
        SPEC~\cite{spec}  &  ResNet-50  & $\checkmark$ & 52.7  & 96.4  & -  \\
        PARE~\cite{pare}  &  ResNet-50 & $\times$ &  52.3 & 82.9  & 99.7\\


        FastMETRO~\cite{fastmetro} &  ResNet-50  &$\checkmark$ & 48.3  & 77.9  & 90.6  \\
        CLIFF~\cite{cliff}  &  ResNet-50   &$\checkmark$ & 45.7  & 72.0  & 85.3   \\
        PARE~\cite{pare}  &  HRNet-w32 & $\checkmark$ & 46.5 &  74.5 &88.6 \\
        CLIFF~\cite{cliff}  &  HRNet-w48 & $\checkmark$ & 43.0 & 69.0 & 81.2        \\
        Graphormer~\cite{meshgraphormer}  &  HRNet-w64 &$\checkmark$& 45.6  & 74.7  & 87.7   \\
        FastMETRO~\cite{fastmetro} &  HRNet-w64  &$\checkmark$ & 44.6  & 73.5  & 84.1    \\


        \midrule
        $\rm\Ours^\cP$ & ResNet-50  & $\times$&    48.9      & 80.0 & 92.3\\
        % $\rm\Ours^\cP$ & ResNet-50  & $\checkmark$&   \\
         \Ours & ResNet-50  & $\times$&    49.2     & 79.6 & 92.7 \\
             \Ours & ResNet-50  & $\checkmark$ & 44.1 & 72.5 & 84.3  \\

        \Ours w/o PD & HRNet-w48  & $\times$ & 48.3 & 78.0 & 92.0 \\
        \Ours w/o PD & HRNet-w48  & \checkmark & 40.9   &   67.2 & 78.4  \\
        \Ours   & HRNet-w48  & $\times$ & 47.9 & 76.2        & 89.8\\
        \Ours  & HRNet-w48  & \checkmark & \textbf{39.8}   &    \textbf{65.0} & \textbf{76.3}  \\
        \bottomrule
        \end{tabular} 
    }
    \vspace{-0.05in}
    \caption{Results of SOTA methods on 3DPW. \Oursp terms our parametric-based variant.
    `\Ours w/o PD' terms trained without distorted data.
    }
    \vspace{-2ex}
 %   \vspace{-1ex}
    \label{tab:3dpw}
\end{table}
\noindent\textbf{Results on 3DPW.}
This study compares our proposed method, \Ours, with SOTA methods~\cite{hybrik, graphcmr, hmr, spin, pymaf, spec, pare, fastmetro, cliff, meshgraphormer}, including both model-based and model-free approaches. As shown in~\cref{tab:3dpw}, \Ours-R50 achieves comparable results to the SOTA method FastMETRO-R50 even without being fine-tuned on the 3DPW training set. Moreover, after fine-tuning, \Ours with both backbone structures show a significant improvement in performance. Furthermore, when using HRNet-w48, our approach outperforms all SOTA methods in all three metrics, surpassing model-based SOTA method CLIFF~\cite{cliff} and model-free SOTA method FastMETRO~\cite{fastmetro}. This superiority can be attributed to two main factors: on one hand, in training, we use ground-truth focal length and translation from 3DPW raw data to supervise the rendering of IUV images and distortion images; on the other hand, given that 96\% of the 3DPW images were captured within a distance of 1.2m to 10m, with more than half of them captured within 4m, there exist many perspective-distorted images. We provide a detailed analysis of the results for samples captured from different distances in the 3DPW dataset in the Sup. Mat.

\begin{figure}[ht]
    \centering
    \includegraphics[width=0.95\linewidth]{pictures/supp_pw3d.pdf}
    \caption{Qualitative results for 3DPW. \Ours achieves good alignment with the characters in the original image, but other SOTA methods have difficulty aligning images that suffer from distortion caused by overhead shots, which causes upper body dilation and lower body shrinkage. 
    The number under each image represents predicted/ground-truth $f$, FoV angle, and $T_{z}$. }
    \label{fig:demo_pw3d}
\end{figure}



\noindent\textbf{Results on Human3.6M:}
During training, we get the ground-truth focal length and translation from Human3.6M~\cite{h36m} training set for our supervision. When evaluating Human3.6M, we follow HybrIK~\cite{hybrik} by using SMPL joints as the ground truth for evaluation. As shown in~\cref{tab:h36m}, our method performs well on Human3.6M through it is not a perspective-distorted dataset. \Ours-H48 achieves the best result on the PA-MPJPE metric and achieves comparable results on the MPJPE metric. CLIFF achieves the best results on MPJPE while they also need ground-truth bounding boxes during testing.

\begin{table}[ht]
    \centering
    \scalebox{0.75}{\begin{tabular}{lccc}
        \toprule
        \multirow{2}{*}{\textbf{Methods}} &\multirow{2}{*}{Backbone}
        & \multicolumn{2}{c}{Metrics}\\ \cmidrule{3-4}
        &  & PA-MPJPE$\downarrow$ & MPJPE$\downarrow$   \\ 
        \midrule 

        HybrIK\cite{hybrik} &  ResNet-34 & 34.5 & 54.4 \\
        HMR~\cite{hmr} & ResNet-50  & 56.8 &88.0 \\
        GraphCMR~\cite{graphcmr} & ResNet-50  &50.1 & - \\
        SPIN~\cite{spin}  &  ResNet-50 & 41.1 &  62.5 \\
        PyMAF~\cite{pymaf}  & ResNet-50  &    40.5 &57.7\\
        FastMETRO~\cite{fastmetro} &  ResNet-50  &  37.3 & 53.9\\
        CLIFF~\cite{cliff}  &  ResNet-50  &35.1  &50.5 \\
        CLIFF~\cite{cliff}  &  HRNet-w48 & 32.7 &\textbf{47.1} \\
        Graphormer~\cite{meshgraphormer}  &  HRNet-w64 & 34.5 & 51.2  \\
        FastMETRO~\cite{fastmetro} &  HRNet-w64  & 33.7 & 52.2 \\
        \midrule    
        \Oursp  & ResNet-50  & 34.7 & 54.0 \\
        \Ours  & ResNet-50  & 34.2 & 52.7 \\
        \Ours  & HRNet-w48  & \textbf{32.3} & 49.4 \\
        \bottomrule
        \end{tabular} 
}
    \vspace{-0.05in}
    \caption{Results of SOTA methods on Human3.6M~\cite{h36m}.
    }
    \vspace{-2ex}
 %   \vspace{-1ex}
    \label{tab:h36m}
\end{table}


\subsection{Ablation study}
\begin{table}
\centering
% \Large
\scalebox{0.7}{
\begin{tabular}{cccccc}
    \toprule
\multirow{2}{*}{\textbf{w/ PD}} &\multirow{2}{*}{\textbf{w/ 3DPW}} &   \multirow{2}{*}{\textbf{w/ gt $f$}} & \multicolumn{3}{c}{Metrics} \\ \cmidrule{4-6}
    &  &  & PA-MPJPE & MPJPE  & PVE  \\ 
    \midrule 

$\times$   & $\times$  &   - &   48.3     &  78.0   & 92.0  \\

$\times$   & $\checkmark$  &    $\times$  & 41.3        & 67.4     &  78.9   \\ 
$\times$  & $\checkmark$ &  $\checkmark$  & 40.9   & 67.2  & 78.4  \\ \hline

$\checkmark$  &  $\times$  &   - &      47.9  & 76.2    &  89.8\\

$\checkmark$     &  $\checkmark$  &   $\times$    & 40.9         & 66.4    & 78.3 \\ 
$\checkmark$   &    $\checkmark$   &   $\checkmark$    & \textbf{39.8}       & \textbf{65.0}       & \textbf{76.3}  \\ 

    \bottomrule
\end{tabular}}
\caption{\small Ablation study of \Ours-H48 of different training settings on 3DPW dataset. w/ PD means whether trained on perspective-distorted datasets (PDHuman, HuMMan). w/ 3DPW means whether fine-tuned on 3DPW~\cite{3dpw} dataset. w/ gt $f$ means using ground-truth focal length when fine-tuned on 3DPW.}
\label{tab:abl_pw3d}
\vspace{-10pt}
\end{table}
\noindent\textbf{Ablation on training settings on the standard benchmark 3DPW.}
In Table~\ref{tab:abl_pw3d}, we present the results of our ablation study on the standard 3DPW benchmark~\cite{3dpw}, where we investigate the impact of different training settings on the performance of our method. By controlling two different variables, we show that introducing perspective-distorted datasets and fine-tuning with ground-truth focal length both lead to a slight improvement in performance. Notably, our method \Ours-H48 still outperforms the current state-of-the-art methods even without using perspective-distorted data or ground-truth focal length.


% Please see the ablation study on loss function, module structure, and training data in Supp.Mat.

This study evaluates the effectiveness of the distortion feature and the hybrid re-projection loss function. The evaluation is conducted on the PDHuman ($\tau=3.0$), as this  exhibits the highest degree of distortion. More experimental results are provided in the Sup. Mat.

\noindent\textbf{Effect of distortion feature.}
In \cref{tab:ablation}, w/o $w(I_{d})$ terms without warp distortion image into UV space, and w/o $c(F_{d})$ terms without concatenating distortion feature to per-vertex feature. We can see that, while the mIoU and P-mIoU change a little, the 3D metrics increase significantly with the correct distortion feature. This study validates our intuition that distortion information helps the network predict more accurate vertex coordinates.

\noindent\textbf{Effect of the hybrid re-projection loss function.}
We experimented with different re-projection loss configurations and found that relying solely on weak-perspective loss significantly decreases 2D alignment. Incorporating perspective loss improved 3D metrics slightly but increased the 2D segmentation error significantly. Moreover, using per-joint distortion weight to supervise the weak-perspective camera improved the alignment of the human mesh and resulted in more accurate 3D supervision without increasing the 2D segmentation error. 


% \noindent\textbf{Effect of training settings on the standard benchmark 3DPW.}
% In Table~\ref{tab:abl_pw3d}, we present the results of our ablation study on the standard 3DPW benchmark~\cite{3dpw}, where we investigate the impact of different training settings on the performance of our method. By controlling two different variables, we show that introducing perspective-distorted datasets and fine-tuning with ground-truth focal length both lead to a slight improvement in performance. Notably, our method \Ours-H48 still outperforms the current state-of-the-art methods even without using perspective-distorted data or ground-truth focal length.
% \begin{table}
\centering
% \Large
\scalebox{0.7}{
\begin{tabular}{cccccc}
    \toprule
\multirow{2}{*}{\textbf{w/ PD}} &\multirow{2}{*}{\textbf{w/ 3DPW}} &   \multirow{2}{*}{\textbf{w/ gt $f$}} & \multicolumn{3}{c}{Metrics} \\ \cmidrule{4-6}
    &  &  & PA-MPJPE & MPJPE  & PVE  \\ 
    \midrule 

$\times$   & $\times$  &   - &   48.3     &  78.0   & 92.0  \\

$\times$   & $\checkmark$  &    $\times$  & 41.3        & 67.4     &  78.9   \\ 
$\times$  & $\checkmark$ &  $\checkmark$  & 40.9   & 67.2  & 78.4  \\ \hline

$\checkmark$  &  $\times$  &   - &      47.9  & 76.2    &  89.8\\

$\checkmark$     &  $\checkmark$  &   $\times$    & 40.9         & 66.4    & 78.3 \\ 
$\checkmark$   &    $\checkmark$   &   $\checkmark$    & \textbf{39.8}       & \textbf{65.0}       & \textbf{76.3}  \\ 

    \bottomrule
\end{tabular}}
\caption{\small Ablation study of \Ours-H48 of different training settings on 3DPW dataset. w/ PD means whether trained on perspective-distorted datasets (PDHuman, HuMMan). w/ 3DPW means whether fine-tuned on 3DPW~\cite{3dpw} dataset. w/ gt $f$ means using ground-truth focal length when fine-tuned on 3DPW.}
\label{tab:abl_pw3d}
\vspace{-10pt}
\end{table}

We conclude that utilizing dense distortion features and an accurate camera model greatly improves the performance of our proposed method
\begin{table}[ht]
\centering
% \Large
\scalebox{0.62}{
\begin{tabular}{ccccccc}
    \toprule
\multirow{2}{*}{\textbf{Architecture}} & \multirow{2}{*}{\textbf{Loss}} & \multicolumn{5}{c}{Metrics} \\ \cmidrule{3-7}
     &  & PA-MPJPE & MPJPE  & PVE & mIoU  & P-mIoU \\ 
    \midrule 

% \rule{0pt}{10pt}  ${N_{v}}$ & $L_{w}$ & 59.3  &  83.4 &  96.0 & 49.3 & 24.9 \\
  %v_head
 \rule{0pt}{10pt} \Ours w/o$\text{ }w(I_{d}), c(F_{d})$   & $\Sigma d_{J}L_{W}^{J} + L_{P}$  & 60.2  & 86.8  & 99.0 & 52.0 & 24.9     \\
  % wo_warp_d
 \rule{0pt}{10pt} \Ours w/o$\text{ }c(F_{d})$  & $\Sigma d_{J}L_{W}^{J} + L_{P}$  & 57.0 & 83.3 & 95.0   & 51.2 & 23.6    \\ 
 %wo_cat_d
 \rule{0pt}{10pt} \Ours & $L_{W}$  & 56.4  & 80.0  & 92.2 &  47.3 & 21.2     \\ 
  \rule{0pt}{10pt} \Ours & $L_{W} + L_{P}$  & 56.1 & 79.1 & 91.0 & 52.5 & 25.5    \\ 
 %loss_hmr
 \midrule
 \rule{0pt}{10pt} \Ours & $\Sigma d_{J}L_{W}^{J} + L_{P}$  & 54.3 & 76.4 & 87.6  & 51.4 & 24.0    \\ 
    \bottomrule
\end{tabular}}
\caption{\small Ablation study of \Ours-H48 structure on PDHuman~($\rm \tau=3.0$). $L_{W}$ indicates weak-perspective re-projection loss. $L_{P}$ indicates perspective re-projection loss. $\sum d_{J}L_{W}^{J}$ terms dividing the per-joint distortion weight in our weak-perspective loss.}
\label{tab:ablation}
\vspace{-10pt}
\end{table}

%%%%%%%%%%%%%%%%%%%%%%%%%%%%%%%%%%%%%%%%%%%%%%%%%

\section{Conclusion} 
We introduce \algname{}, a novel object detection method that achieves highly efficient inference speed while also improving zero-shot generalization compared with existing methods. The prompt-based decoding approach reduces the computational burden of object queries. The RoI-based masked attention and RoI pruning techniques allow us to efficiently leverage a large ViT-based CLIP model, enhancing detection performance through classification prediction ensembling. Comprehensive experiments show that \algname{} is $21.2$ times faster than OV-DETR while achieving comparable or higher APs on base and novel classes compared to two-stage OVD methods. %We believe that our work will inspire future work to explore the benefits of using Transformers.


%\paragraph{Ethics Statement.} 
%The focus of this paper is on open-vocabulary object detection. Our approach involves the integration of Transformer-based object detector and CLIP. We have not identified any foreseeable negative social impact associated with our work to share our findings with the scientific community. Nonetheless, we will continue to monitor and consider any potential concerns that may arise. 
% \clearpage


%%%%%%%%% REFERENCES
% \clearpage
% \clearpage
% \newpage
\appendix
% \onecolumn


% PDHuman Dataset consists of two parts: synthetic dataset and real dataset. The synthetic dataset contains training and testing set, and real dataset are for testing.


\section{Details of Perspective-distorted Datasets.}
\subsection{PDHuman}
\label{chapter:pdhuman-syn}
Our pipeline is inspired by recent works on synthetic data~\cite{Patel:CVPR:2021:AGORA,hspace,varol17_surreal}. 
A photogrammetry-scanned human model with a unique body pose will be rendered with a random viewpoint in an HDRi background. The detailed statistics of PDHuman are illustrated in \cref{tab:pdhuman_stat}.

\begin{table}[htp]
    \centering
    \scalebox{0.75}{
    
\begin{tabular}{l|p{25pt}<{\raggedright}p{25pt}<{\raggedright}p{25pt}<{\raggedright}p{25pt}<{\raggedright}p{25pt}<{\raggedright}p{25pt}<{\raggedright}}
\toprule
Protocol             & Train & 5    & 4    & 3    & 2     & 1     \\ \midrule
Number             &  126198 & 2821 & 4166 & 6601 & 12225 & 27448 \\ 
$\tau$ & 1.0 & 3.0  & 2.6  & 2.2  & 1.8   & 1.4   \\ 


Mean $f$ &  245    &   174  & 176     &     180 & 191  &    230   \\ 
% Max $f$ &  1449    &  970 &970 \\ 
% Min $f$ &   94 & 93  & 93\\ 
% Maximum FoV       &  $\rm 139.9^\circ$       &      &      &      &       &       \\ 
% Minimum FoV       &  $\rm 20.1^\circ$       &      &      &      &       &       \\ 
Mean FoV       &  $\rm 98.6^\circ$       &     $\rm 113.7^\circ$  &  $\rm 112.0^\circ$     &  $\rm 112.0^\circ$    &    $\rm 110.0^\circ$     &   $\rm 102.0^\circ$    \\ 
Mean $T_{z}$      &  1.3  &    0.8  &    0.8  &   0.8   &    0.9   &    1.1   \\
% Max $T_{z}$      &   8.0 &  1.5 & 1.6\\
% Min $T_{z}$      &   0.7  & 0.5 & 0.5 \\

\bottomrule
\end{tabular}
}
    \caption{Statistical information of PDHuman. $\tau$ denotes maximum diisortion scale in the main text.}
    \label{tab:pdhuman_stat}
    \vspace{-5pt}
\end{table}


\noindent\textbf{Human model.}
We use a corpus of $630$ photogrammetry-scanned human models from Renderpeople~\cite{renderpeople},
with well-fitted SMPL parameters. 
Initially, the body pose is sampled from a collection of high-quality motion sequences obtained from
% Mixamo\footnote[3]{\url{https://www.mixamo.com/}},
Mixamo~\cite{mixamo}.
The the pose is converted to SMPL skeleton using a re-targeting approach.
%We export the vertices from Blender for each frame to get the optimized SMPL parameters.
Finally, we use a SGD optimizer to optimize the chamfer distance between the SMPL vertices and RenderPeople vertices to refine the pose and shape parameter.

\noindent\textbf{Camera.}
% To cover most of the scenarios in real life,  
% we randomly sample a perspective camera 
% with a focal length ranging from \SI{7}{\mm} to \SI{102}{\mm}.
% Then the body model is placed in the center of a sphere 
% with a random radius relying on the focal length of the camera,
% and the camera is placed on the surface of the same sphere
% in random elevation and azimuth angle while facing the center.
In order to simulate a wide range of real-world scenarios, a perspective camera is randomly sampled with a focal length that spans from 7mm to 102mm. The corresponding FoV angle is from 10\degree to 140\degree.
The human mesh is then positioned at the center of a sphere, whose radius is chosen randomly and dependent on the camera's focal length.
The camera, facing the center of the sphere, is then placed on the surface of the sphere with a randomized elevation and azimuth angle.

\noindent\textbf{Rendering pipeline.} 
To increase the diversity of data, each frame contains ambient lighting calculated by path tracing in Blender~\cite{blender} and diverse background generated by HDRi images from PolyHaven~\cite{polyhaven}. The size of all rendered images is $512\times512$.


% We use  based on $500$ HDRi images obtained from PolyHaven \cite{polyhaven},
% and Blender \cite{blender} with path tracing (cycles) are utilized to render photo-realistic images.
%
% In addition, we use the SMPL~\cite{smpl} parameters and camera matrices to render distortion maps and IUV images by Pytorch3D~\cite{pytorch3d}.


\begin{figure}
    \centering
    \includegraphics[width=1.0\linewidth]{pictures/supp_pdhuman.pdf}
    \vspace{-5pt}
    \caption{More examples from PDHuman dataset.}
    \label{fig:pdhuman_demo}
    \vspace{-5pt}
\end{figure}

\subsection{SPEC-MTP}
\label{chapter:spec-mtp}

\begin{table}
    \centering
    \scalebox{0.8}{\begin{tabular}{l|p{50pt}<{\raggedright}p{50pt}<{\raggedright}p{50pt}<{\raggedright}}
\toprule
Protocol              & 3    & 2     & 1     \\ \midrule
Number                & 713 &2609  & 6083  \\ 
$\tau$    & 1.8   & 1.4   & 1.0\\ 
Mean $f$  & 935     &     976  &     1114  \\ 
Mean FoV     &     $69.3^\circ$      &   $67.7^\circ$    &       $62.4^\circ$      \\ 
Mean $T_{z}$       &1.1        & 1.1 &     1.4  \\ \bottomrule
\end{tabular}}
    \caption{Statistical information of SPEC-MTP~\cite{spec}. $\tau$ denotes the maximum distortion scale in the main text.}
    \label{tab:spec_mtp_stat}
    \vspace{-5pt}
\end{table}

SPEC-MTP~\cite{spec} is a real-world image dataset with calibrated focal lengths and well-fitted SMPL parameters (including $\theta$, $\beta$), and translation. The images were taken at relatively close-up distances, leading to noticeable perspective distortion in the limbs and torsos of subjects. We use it as one of the evaluation datasets for our task. The detailed statistics of SPEC-MTP are illustrated in \cref{tab:spec_mtp_stat}.

% The dataset contains $5,356$ images, and is used as an evaluation dataset in our task.
% $4,212$ of the images have the shape of $1920\times1080$ and 1144 of the images have the shape of $1280\times720$.
% This dataset is used as an evaluation dataset in our task. 
% , as shown in \cref{fig:supp_specmtp_sota}

% When utilizing SPEC-MTP, it is important to consider that the dataset has several limitations. (1). The limited diversity of human individuals and background environments in the dataset may affect its generalizability to other datasets. (2). The dataset restricts poses as subjects were asked to mimic a template pose, which may restrict its applicability in certain research domains. (3). The inaccurate calibration of images could lead to sub-optimal estimations of body pose and shape parameters, which may affect the output of image processing applications.
\subsection{HuMMan}
The HuMMan dataset, as proposed in \cite{humman}, is a real-world image dataset that utilizes 10 calibrated RGBD kinematic cameras to capture shots for each frame. From these shots, segmented point clouds are extracted from depth images, resulting in a comprehensive dataset. The SMPL parameters were registered upon triangulated 3D keypoints and point clouds, providing ground-truth data that is highly useful for mocap tasks. We reshape all the images to $360\times 640$ pixels. The detailed statistics of HuMMan are illustrated in~\cref{tab:humman_stat}.

% It is also important to consider the limitations of the dataset which include a limited actor diversity, with all poses being predefined for actors to mimic, and each camera having similar focal lengths. Additionally, the capture environment is only limited to one indoor laboratory, which could impact the generalization of the dataset's findings to other environments.

\begin{table}
    \centering
    \scalebox{0.75}{

\begin{tabular}{l|p{40pt}<{\raggedright}p{40pt}<{\raggedright}p{40pt}<{\raggedright}p{40pt}<{\raggedright}}
\toprule
Protocol   &Train            & 3    & 2     & 1     \\ \midrule
Number                & 84170  & 926 & 2696 & 6550 \\ 
$\tau$    & 1.0 & 1.8   & 1.4   & 1.0\\ 
Mean $f$ &    318  & 318       &   318&  318  \\ 
Mean FoV              &  127    &127 &  127     & 127      \\ 
Mean $T_{z}$           &  1.9    &  1.9 & 1.9      &  1.9     \\ \bottomrule
\end{tabular}}
\caption{Statistical information of HuMMan~\cite{humman}. $\tau$ denotes the maximum distortion scale in the main text.}
\label{tab:humman_stat}
\vspace{-5pt}
\end{table}

\section{Analysis of 3DPW dataset}
We divide the 3DPW dataset into three protocols based on the maximum distortion scale $\tau$ in~\cref{tab:supp_pw3d_info}. We report the results of our re-implemented HMR-R50~\cite{hmr} and \Ours-H48 on each protocol in \cref{tab:supp_pw3d}.
%The results reveal that \Ours achieves larger improvement as the distortion scale becomes bigger. This phenomenon demonstrates that \Ours's improvement on 3DPW is largely attributed to the improvement in distorted images.
The experiments indicate that \Ours outperforms HMR-R50, and this improvement is more pronounced as the distortion scale increases.
This observation serves as compelling evidence that \Ours's success on 3DPW can be primarily attributed to its superior performance on distorted images.
%
% We also present qualitative results in \cref{fig:demo_pw3d}.
%
% Interestingly, CLIFF~\cite{cliff} adjusted the focal length of its camera model to match the 3DPW dataset, which could have resulted in exploiting the properties of 3DPW to achieve higher performance.

\begin{table}[ht]
    \centering
    \scalebox{0.75}{\begin{tabular}{l|p{45pt}<{\raggedright}p{45pt}<{\raggedright}p{45pt}<{\raggedright}}\toprule
 Protocol  &            1  &   2         &      3      \\\midrule
Number                &     35115         &            19016        &   4657           \\

$\tau$ &        1.0      &      1.08      &    1.16              \\
mean $f$                & 1966             &      1966       &      1966           \\
mean FoV              &  $52^\circ$            & $52^\circ$                 &        $52^\circ$ \\
mean $T_{z}$    & 4.6 & 3.5 & 2.9\\
\bottomrule       
\end{tabular}}
    \caption{3 protocols of 3DPW divided by $\tau$. The larger the value of $\tau$, the greater the degree of distortion.}
    \label{tab:supp_pw3d_info}
\end{table}


\begin{table}
    \centering
    \scalebox{0.75}{
    \begin{tabular}{l|p{55pt}<{\centering}p{55pt}<{\centering}p{55pt}<{\centering}}
\toprule
\multirow{2}{*}{\textbf{Protocol/Method}} & \multicolumn{3}{c}{3DPW Test}  \\ \cmidrule{2-4}
&PA-MPJPE$\downarrow$ & MPJPE$\downarrow$  & PVE$\downarrow$ \\ \midrule 

\rule{0pt}{10pt} p1(HMR-R50)  &  50.2&  80.9  & 94.5 \\
\rule{0pt}{10pt} p1(\Ours-H48)  &39.8  & 65.0 &76.3 \\
\rule{0pt}{10pt} Improvement  &+10.4   & +15.9& +18.2\\\midrule

\rule{0pt}{10pt} p2(HMR-R50)  &51.3   &82.3  &96.2\\
\rule{0pt}{10pt} p2(\Ours-H48)  & 39.9  & 64.8 & 76.8 \\
\rule{0pt}{10pt} Improvement  & +11.4 & +17.5 &+19.4\\\midrule

\rule{0pt}{10pt} p3(HMR-R50)  & 58.3 &93.9  &107.8  \\
\rule{0pt}{10pt} p3(\Ours-H48)  &44.7  & 71.7 & 84.6\\
\rule{0pt}{10pt} Improvement  & + 13.6  & +22.2   & +23.2 \\

% \rule{0pt}{10pt} p5(HMR-R50)  &  &   &  &  \\
% \rule{0pt}{10pt} p5(\Ours-H48)  &  &  &  & \\
% \rule{0pt}{10pt} Improvement  &  &  &  & \\\midrule

% \rule{0pt}{10pt} Full(HMR-R50)  &  &   &  &  \\
% \rule{0pt}{10pt} Full(\Ours-H48)  &  &  &  & \\
% \rule{0pt}{10pt} Improvement  &  &  &  & \\



\bottomrule

\end{tabular}
}
    \caption{Reults of our re-implemented HMR~\cite{hmr} and \Ours-H48 on different protocols of 3DPW. Mainly for showing the correlation between performance improvement and distortion.}
    \label{tab:supp_pw3d}
\end{table}


% \begin{figure}[ht]
%     \centering
%     \includegraphics[width=0.95\linewidth]{pictures/supp_pw3d.pdf}
%     \caption{Qualitative results for 3DPW. \Ours achieves good alignment with the characters in the original image, but other SOTA methods have difficulty aligning images that suffer from distortion caused by overhead shots, which causes upper body dilation and lower body shrinkage. 
%     The number under each image represents predicted/ground-truth $f$, FoV angle, and $T_{z}$. }
%     \label{fig:demo_pw3d}
% \end{figure}



\section{Details of Model-based Variant of \Ours}
\begin{figure*}[ht]
    \centering
    \includegraphics[width=0.95\linewidth]{pictures/pipeline-p.pdf}

    \caption{\textbf{\Oursp pipeline overview}. Compared to \Ours, the main difference is the reformulated mesh reconstruction module.}
    \label{fig:pipeline_p}
\end{figure*}
As shown in~\cref{fig:pipeline_p}, we introduce a model-based \Oursp by changing the mesh reconstruction module. Different from \Ours, we regress SMPL parameters rather than 3D vertex coordinates through a transformer decoder in \Oursp. We warp the grid Feature $F_{grid}$ into UV space ($F_{grid-w}$) via $I_{IUV}$ to eliminate the spatial distortion of each part of the features, then concatenate the warped distortion feature $F_{grid-w}$ and regress SMPL parameters from it. We represent the 24 rotations of joints $\theta$ and body shape parameters $\beta$ as 25 learnable tokens. The translation estimation module and supervision are exactly the same as \Ours.



\section{More about Cameras}
\noindent\textbf{Affine Transformation.}
Our affine transformation of translation is the same as SPEC~\cite{spec}.
$T_{x}, T_{y}$ and $t_{x},t_{y}$ should satisfy the following equation for every $x, y$ by connecting the re-projected coordinates in the cropped image coordinate system and original image coordinate system in screen space.
\begin{equation}
\small
\begin{bmatrix}  
\frac{w}{2}\left[ s_{x}(x+t_{x})) +1\right]+c_{x}-\frac{w}{2}\\[6pt]
\frac{h}{2}\left[ s_{y}(y+t_{y})) +1\right]+c_{y}-\frac{h}{2}
  \end{bmatrix}=
\begin{bmatrix}\frac{W}{2}\left[S_{W}(x+T_{x}) + 1\right]\\[6pt]
\frac{H}{2}\left[ S_{H}(y+T_{y}) + 1\right] \end{bmatrix}.
\end{equation}
where $S_{W}=s_{x}/(\frac{W}{w})$, $S_{H}=s_{y}/(\frac{H}{h})$, so we can get the transform by:

\begin{equation}
\small
\begin{bmatrix}
T_{x}\\ 
T_{y}\\ 
1
\end{bmatrix}=\begin{bmatrix}
 1& 0 & (2c_{x}-W)/ws_{x}\\ 
 &1  & (2c_{y}-H)/hs_{y}\\ 
 &  & 1
\end{bmatrix}\begin{bmatrix}
t_{x}\\ 
t_{y}\\ 
1
\end{bmatrix}
\end{equation}
$c_{x}, c_{y}$ terms the bounding-box center coordinate in original image. $h, w$ terms the cropped image size, where $H, W$ terms original image size. During training, we will expand every bounding box of the human body to a square and resize the cropped image to $224\times224$ pixels.

\noindent\textbf{Analysis of dolly zoom.}
%A dolly zoom is an in-camera effect where you dolly towards or away from a subject while zooming in the opposite direction.
The dolly zoom is an optical effect performed in-camera, whereby the camera moves towards or away from a subject while simultaneously zooming in the opposite direction.
It was first proposed in the film JAW~\cite{jaws1975}. In this section, we simulate the effect on CMU-MOSH~\cite{mosh} data.
%
First, we get all the vertices in CMU-MOSH~\cite{mosh} by feeding the SMPL~\cite{smpl} parameters to the body model. 
We further obtain 3D joints by multiplying the joint regressor matrix to the vertices. 
%
%Last, we fix the weak-perspective camera parameters $(s, t_{x}, t_{y})$ and adjust the distance. This results in approximating fixed human body location and size, with increased distortion as the camera gets closer.
We then apply weak-perspective camera parameters $(s, t_{x}, t_{y})$ while adjusting the distance to approximate the human body's location and size, producing increased distortion as the camera approaches.
% With this approach, we can calculate the focal length as $f=shT_{z}/2$. 
The weak-perspectively projected 2D joints are $s(x+t_{x}, y+t_{y})$, where $x, y$ are the corresponding 3D joint coordinates. We set the image height to 224 pixels and re-project the 2D joints and compare the error between the weak-perspective and perspective projection results.
As shown in \cref{tab:supp_dolly}, when the subject is located over 4 meters away, the re-projected error is only 1.76 pixels, which is negligible on a 224-pixel image. When the subject is further than 8 meters, the error is less than 1 pixel, indicating non-distortion of the images. 

% This demonstrates why and how we manage the settings of our camera model.


 \begin{table}
    \centering
    \scalebox{0.7}{\begin{tabular}{l|p{20pt}
<{\raggedright}p{20pt}<{\raggedright}p{20pt}<{\raggedright}p{20pt}<{\raggedright}p{20pt}<{\raggedright}p{20pt}<{\raggedright}p{20pt}<{\raggedright}p{20pt}<{\raggedright}p{20pt}<{\raggedright}p{20pt}<{\raggedright}}\toprule
$T_{z}$  &      0.5 &0.75&1.0 & 2.0 & 4.0 & 8.0 & 12.0 & 16.0 &20       \\\midrule
$\tau$  &    3.06&1.69&1.41&1.16&1.07&1.04&1.02&1.02&1.01          \\
$Error$  & 30.56&10.46&7.38&3.54&1.76&0.88&0.59&0.44&0.35              \\

\bottomrule       
\end{tabular}}
    \caption{Distortion and re-projection error caused by distance in CMU-MOSH~\cite{mosh}. The re-projected error is measured in pixels.}
    \label{tab:supp_dolly}
    \vspace{-10pt}
\end{table}

\begin{figure}[t]
    \centering
    \includegraphics[width=0.9\linewidth]{pictures/supp_dollyzoom.pdf}
    \caption{Distortion and re-projection error caused by distance. The vertical axis is measured in pixels, and the horizontal axis is measured in meters.}
    \label{fig:supp_dolly}
    \vspace{-5pt}
\end{figure}  

\section{More quantitative results}

% \subsection{Ablation Studies.}

% \begin{table}[ht]
\centering
% \Large
\scalebox{0.62}{
\begin{tabular}{ccccccc}
    \toprule
\multirow{2}{*}{\textbf{Architecture}} & \multirow{2}{*}{\textbf{Loss}} & \multicolumn{5}{c}{Metrics} \\ \cmidrule{3-7}
     &  & PA-MPJPE & MPJPE  & PVE & mIoU  & P-mIoU \\ 
    \midrule 

% \rule{0pt}{10pt}  ${N_{v}}$ & $L_{w}$ & 59.3  &  83.4 &  96.0 & 49.3 & 24.9 \\
  %v_head
 \rule{0pt}{10pt} \Ours w/o$\text{ }w(I_{d}), c(F_{d})$   & $\Sigma d_{J}L_{W}^{J} + L_{P}$  & 60.2  & 86.8  & 99.0 & 52.0 & 24.9     \\
  % wo_warp_d
 \rule{0pt}{10pt} \Ours w/o$\text{ }c(F_{d})$  & $\Sigma d_{J}L_{W}^{J} + L_{P}$  & 57.0 & 83.3 & 95.0   & 51.2 & 23.6    \\ 
 %wo_cat_d
 \rule{0pt}{10pt} \Ours & $L_{W}$  & 56.4  & 80.0  & 92.2 &  47.3 & 21.2     \\ 
  \rule{0pt}{10pt} \Ours & $L_{W} + L_{P}$  & 56.1 & 79.1 & 91.0 & 52.5 & 25.5    \\ 
 %loss_hmr
 \midrule
 \rule{0pt}{10pt} \Ours & $\Sigma d_{J}L_{W}^{J} + L_{P}$  & 54.3 & 76.4 & 87.6  & 51.4 & 24.0    \\ 
    \bottomrule
\end{tabular}}
\caption{\small Ablation study of \Ours-H48 structure on PDHuman~($\rm \tau=3.0$). $L_{W}$ indicates weak-perspective re-projection loss. $L_{P}$ indicates perspective re-projection loss. $\sum d_{J}L_{W}^{J}$ terms dividing the per-joint distortion weight in our weak-perspective loss.}
\label{tab:ablation}
\vspace{-10pt}
\end{table}

% \begin{table}
\centering
% \Large
\scalebox{0.7}{
\begin{tabular}{cccccc}
    \toprule
\multirow{2}{*}{\textbf{w/ PD}} &\multirow{2}{*}{\textbf{w/ 3DPW}} &   \multirow{2}{*}{\textbf{w/ gt $f$}} & \multicolumn{3}{c}{Metrics} \\ \cmidrule{4-6}
    &  &  & PA-MPJPE & MPJPE  & PVE  \\ 
    \midrule 

$\times$   & $\times$  &   - &   48.3     &  78.0   & 92.0  \\

$\times$   & $\checkmark$  &    $\times$  & 41.3        & 67.4     &  78.9   \\ 
$\times$  & $\checkmark$ &  $\checkmark$  & 40.9   & 67.2  & 78.4  \\ \hline

$\checkmark$  &  $\times$  &   - &      47.9  & 76.2    &  89.8\\

$\checkmark$     &  $\checkmark$  &   $\times$    & 40.9         & 66.4    & 78.3 \\ 
$\checkmark$   &    $\checkmark$   &   $\checkmark$    & \textbf{39.8}       & \textbf{65.0}       & \textbf{76.3}  \\ 

    \bottomrule
\end{tabular}}
\caption{\small Ablation study of \Ours-H48 of different training settings on 3DPW dataset. w/ PD means whether trained on perspective-distorted datasets (PDHuman, HuMMan). w/ 3DPW means whether fine-tuned on 3DPW~\cite{3dpw} dataset. w/ gt $f$ means using ground-truth focal length when fine-tuned on 3DPW.}
\label{tab:abl_pw3d}
\vspace{-10pt}
\end{table}
% 

% This study evaluates the effectiveness of the distortion feature and the hybrid re-projection loss function. The evaluation is conducted on the PDHuman ($\tau=3.0$), as this exhibits the highest degree of distortion. More experimental results are provided in the Sup. Mat.

% \noindent\textbf{Ablation on distortion feature.}
% In \cref{tab:ablation}, w/o $w(I_{d})$ terms without warp distortion image into UV space, and w/o $c(F_{d})$ terms without concatenating distortion feature to per-vertex feature. We can see that, while the mIoU and P-mIoU change a little, the 3D metrics increase significantly with the correct distortion feature. This study validates our intuition that distortion information helps the network predict more accurate vertex coordinates.

% \noindent\textbf{Ablation on the hybrid re-projection loss function.}
% We experimented with different re-projection loss configurations and found that relying solely on weak-perspective loss significantly decreases 2D alignment. Incorporating perspective loss improved 3D metrics slightly but increased the 2D segmentation error significantly. Moreover, using per-joint distortion weight to supervise the weak-perspective camera improved the alignment of the human mesh and resulted in more accurate 3D supervision without increasing the 2D segmentation error. 


% \noindent\textbf{Ablation on training settings on the standard benchmark 3DPW.}
% In Table~\ref{tab:abl_pw3d}, we present the results of our ablation study on the standard 3DPW benchmark~\cite{3dpw}, where we investigate the impact of different training settings on the performance of our method. By controlling two different variables, we show that introducing perspective-distorted datasets and fine-tuning with ground-truth focal length both lead to a slight improvement in performance. Notably, our method \Ours-H48 still outperforms the current state-of-the-art methods even without using perspective-distorted data or ground-truth focal length.



% \noindent\textbf{Results on standard benchmark Human3.6M:}
% During training, we get the ground-truth focal length and translation from Human3.6M~\cite{h36m} training set for our supervision. When evaluating Human3.6M, we follow HybrIK~\cite{hybrik} by using SMPL joints as the ground truth for evaluation. As shown in~\cref{tab:h36m}, our method performs well on Human3.6M through it is not a perspective-distorted dataset. \Ours-H48 achieves the best result on the PA-MPJPE metric and achieves comparable results on the MPJPE metric. CLIFF achieves the best results on MPJPE while they also need ground-truth bounding boxes during testing.

% \begin{table}[ht]
%     \centering
%     \scalebox{0.75}{\begin{tabular}{lccc}
        \toprule
        \multirow{2}{*}{\textbf{Methods}} &\multirow{2}{*}{Backbone}
        & \multicolumn{2}{c}{Metrics}\\ \cmidrule{3-4}
        &  & PA-MPJPE$\downarrow$ & MPJPE$\downarrow$   \\ 
        \midrule 

        HybrIK\cite{hybrik} &  ResNet-34 & 34.5 & 54.4 \\
        HMR~\cite{hmr} & ResNet-50  & 56.8 &88.0 \\
        GraphCMR~\cite{graphcmr} & ResNet-50  &50.1 & - \\
        SPIN~\cite{spin}  &  ResNet-50 & 41.1 &  62.5 \\
        PyMAF~\cite{pymaf}  & ResNet-50  &    40.5 &57.7\\
        FastMETRO~\cite{fastmetro} &  ResNet-50  &  37.3 & 53.9\\
        CLIFF~\cite{cliff}  &  ResNet-50  &35.1  &50.5 \\
        CLIFF~\cite{cliff}  &  HRNet-w48 & 32.7 &\textbf{47.1} \\
        Graphormer~\cite{meshgraphormer}  &  HRNet-w64 & 34.5 & 51.2  \\
        FastMETRO~\cite{fastmetro} &  HRNet-w64  & 33.7 & 52.2 \\
        \midrule    
        \Oursp  & ResNet-50  & 34.7 & 54.0 \\
        \Ours  & ResNet-50  & 34.2 & 52.7 \\
        \Ours  & HRNet-w48  & \textbf{32.3} & 49.4 \\
        \bottomrule
        \end{tabular} 
}
%     \vspace{-0.05in}
%     \caption{Results of SOTA methods on Human3.6M~\cite{h36m}.
%     }
%     \vspace{-2ex}
%  %   \vspace{-1ex}
%     \label{tab:h36m}
% \end{table}


\noindent\textbf{Full results on PDHuman:} As shown in~\cref{tab:sota_supp_pdhuman} and \cref{tab:sota_supp_pdhuman2}, we report results on all 5 protocols in the PDHuman test dataset. Our proposed methods, \Ours(H48) and \Oursp (R50) outperform the other methods in all metrics by a large margin. 
% Specifically, our model-based approach, \Oursp, achieves some of the highest 2D mIoU metrics.

\noindent\textbf{Full results on SPEC-MTP:} As illustrated in~\cref{tab:sota_supp_specmtp}, we report the results of all the 3 protocols in SPEC-MTP dataset. In this real-world dataset, \Ours (H48) largely outperforms other methods in all metrics. In column $\tau=1.0$, Note that our re-implemented SPEC$*$ achieves higher performance than the official implementation.
 
\noindent\textbf{Full results on HuMMan:} As shown in \cref{tab:sota_sup_humman}, \Ours (H48) largely outperforms other methods in all metrics. By contrast, although CLIFF~\cite{cliff} performs comparably well on the HuMMan dataset, it demonstrates poor performance on the PDHuman dataset. We conjecture the focal length assumption of CLIFF is suitable for datasets captured by fixed and similar camera settings, \eg HuMMan dataset, while not valid for the PDHuman dataset with varied camera settings.


\begin{table*}[hb]
    \centering
    \scalebox{0.61}{\begin{tabular}{lccccc|ccccc|ccccc}
\toprule
\multirow{2}{*}{\textbf{Methods}} & \multicolumn{5}{c}{PDHuman~($\tau=3.0$)}  & \multicolumn{5}{c}{PDHuman~($\tau=2.6$)}
&\multicolumn{5}{c}{PDHuman~($\tau=2.2$)} \\ \cmidrule{2-16}
&PA-MPJPE$\downarrow$ & MPJPE$\downarrow$  & PVE$\downarrow$ & mIoU$\uparrow$ &P-mIoU$\uparrow$& PA-MPJPE$\downarrow$ & MPJPE$\downarrow$ & PVE$\downarrow$ & mIoU$\uparrow$ &P-mIoU$\uparrow$& PA-MPJPE$\downarrow$ & MPJPE$\downarrow$ & PVE$\downarrow$ &  mIoU$\uparrow$  & P-mIoU$\uparrow$ \\ \midrule 

\rule{0pt}{10pt} HMR~(R50)~\cite{hmr}  & 62.5 & 91.5 & 106.6 & 48.9& 21.7 & 59.9 &87.8 & 102.4 & 50.0 & 22.5 & 57.4 & 84.0 & 98.1 & 51.4 & 23.6 \\

\rule{0pt}{10pt} HMR-$f$~(R50)~\cite{hmr}& 61.6 & 90.2 & 105.5 & 45.2 & 20.4 & 59.2 & 86.6 & 101.3 & 46.5 & 21.4 &56.8 & 82.9 & 97.2 & 48.1 & 22.7  \\

\rule{0pt}{10pt} SPEC~(R50)~\cite{spec}  &  65.8 & 94.9 & 109.6 & 43.4 & 19.6 & 63.2 & 91.5 & 105.8 &43.3 & 19.5 &60.6 & 87.3 & 101.3 & 42.2 & 18.7   \\

\rule{0pt}{10pt} CLIFF~(R50)~\cite{cliff} & 66.2 & 99.2 & 115.2 & 51.4 & 24.8 & 63.4 & 94.4 & 109.8 & 52.7 & 25.9 & 60.6 & 89.6 & 104.3 & 54.2 & 27.1 \\



\rule{0pt}{10pt} PARE~(H48)~\cite{pare} & 66.3 & 95.9 & 116.7 & 48.2 & 20.9 &63.6& 92.3 & 112.7 & 49.3 & 21.7 & 60.6 & 88.7 & 108.6 & 50.7 & 22.7   \\

\midrule
\rule{0pt}{10pt} GraphCMR~(R50)   &  62.1 & 85.8 & 98.4 & 47.9 & 21.5 & 59.5 & 82.6 & 94.8 & 49.1 & 22.4 & 56.8 & 78.8 & 90.4 & 50.5 & 23.6\\

\rule{0pt}{10pt} FastMetro(H48)~\cite{fastmetro}   &58.6 & 83.6 & 95.4 & 50.1 & 22.5 & 55.8 & 79.9 & 91.4 & 51.4 & 23.5  & 53.1 & 75.9 & 86.7 & 52.9 & 24.9 \\

\midrule
\rule{0pt}{10pt} $\rm \Ours^{P}$ (R50)  &  54.3 & 80.9 & 93.9 & \textbf{54.5} & \textbf{27.4} & 52.4 & 77.5 & 90.2 & \textbf{55.7} & 28.5 & 50.0 & 74.0 & 86.4 & \textbf{56.9} & \textbf{29.5} \\
\rule{0pt}{10pt} \Ours (R50)  &  54.3 & 76.4&87.6 & 51.4 & 24.0 & 51.8 & 73.3 & 84.1 & 52.4 & 24.8 &  49.3 & 70.1 & 80.6 & 53.3 & 25.7 \\
\rule{0pt}{10pt} \Ours (H48)  & \textbf{49.7} & \textbf{70.2} & \textbf{81.2} & 50.5 & 23.8 & \textbf{47.6} & \textbf{64.3} & \textbf{74.4} & 55.3 & \textbf{28.5} & \textbf{44.9} & \textbf{64.3} & \textbf{74.7} & 55.3 & 28.5     \\
\bottomrule

\end{tabular}
}
    
    \caption{Results of SOTA methods on PDHuman ($\tau=3.0$, $\tau=2.6$, $\tau=2.2$ protocols). HMR-$f$ terms HMR~\cite{hmr} model trained with same focal length as \Ours.}
    \label{tab:sota_supp_pdhuman}
\end{table*}

\begin{table*}[h]
    \centering
    \scalebox{0.61}{\begin{tabular}{lp{57pt}<{\centering}p{47pt}<{\centering}p{47pt}<{\centering}p{47pt}<{\centering}p{47pt}<{\centering}|p{57pt}<{\centering}p{47pt}<{\centering}p{47pt}<{\centering}p{47pt}<{\centering}p{47pt}<{\centering}}
\toprule
\multirow{2}{*}{\textbf{Methods}} & \multicolumn{5}{c}{PDHuman~($\tau=1.8$)} & \multicolumn{5}{c}{PDHuman~($\tau=1.4$)}
 \\ \cmidrule{2-11}
&PA-MPJPE$\downarrow$ & MPJPE$\downarrow$  & PVE$\downarrow$ & mIoU$\uparrow$ &P-mIoU$\uparrow$& PA-MPJPE$\downarrow$ & MPJPE$\downarrow$ & PVE$\downarrow$ & mIoU$\uparrow$ &P-mIoU$\uparrow$ \\ \midrule 

\rule{0pt}{10pt} HMR~(R50)~\cite{hmr} & 53.9 &  79.0 & 92.4 & 53.6 & 25.1 & 49.2 & 73.3 & 85.9 & 57.3 & 28.2\\

\rule{0pt}{10pt} HMR-$f$~(R50)~\cite{hmr}&  53.4 & 78.3 & 91.8 & 50.4 & 24.6  & 48.8 & 72.7 & 85.3 & 54.6 & 28.1 \\

\rule{0pt}{10pt} SPEC~(R50)~\cite{spec}  &  56.8 & 81.8 & 95.1 & 40.1 & 17.1 & 51.8 & 75.4 & 87.9 & 37.4 & 15.3   \\

\rule{0pt}{10pt} CLIFF~(R50)~\cite{cliff} & 56.7 & 83.6 & 97.3 & 56.5 & 29.1 & 51.6 & 76.9 & 89.7 & 60.2 & 32.7\\



\rule{0pt}{10pt} PARE~(H48)~\cite{pare}  & 56.8 & 83.9 & 103.0 & 52.8 & 24.3 & 51.8 & 78.5 & 96.6 & 56.6 & 27.5 \\

\midrule
\rule{0pt}{10pt} GraphCMR~(R50)   &  53.2 & 74.2 & 85.2 & 52.7 & 25.3 & 48.7 & 69.1 & 79.4 & 56.4 & 28.6 \\

\rule{0pt}{10pt} FastMetro(H48)~\cite{fastmetro}   &  49.4 & 71.1 & 81.1 & 55.5 & 27.0 & 45.0 & 65.8 & 75.2 & 59.7 & 31.0  \\

\midrule
\rule{0pt}{10pt} $\rm \Ours^{P}$ (R50)  & 47.1 & 69.8 & 81.6 & \textbf{58.7} & \textbf{30.7} & 43.2 & 65.2 & 76.5 & \textbf{61.3} & \textbf{32.6}  \\
\rule{0pt}{10pt} \Ours (R50)  & 45.9 & 66.0 & 75.9 & 54.8 & 26.8 & 41.9 & 61.5 & 70.9 & 57.2 & 28.2 \\
\rule{0pt}{10pt} \Ours (H48)  & \textbf{42.1} & \textbf{60.7} & \textbf{70.4} & 56.8 & 29.5 & \textbf{39.4} & \textbf{56.6} & \textbf{69.6} & 58.3 & 29.9    \\
\bottomrule
\end{tabular}
}
    
    \caption{Results of SOTA methods on PDHuman ($\tau=1.8$, $\tau=1.4$ protocols).}
    \label{tab:sota_supp_pdhuman2}
\end{table*}


% \subsection{SPEC-MTP}
\begin{table*}[h]
    \centering
    \scalebox{0.61}{\begin{tabular}{lccccc|ccccc|ccccc}
\toprule
\multirow{2}{*}{\textbf{Methods}} & \multicolumn{5}{c}{SPEC-MTP~($\tau=1.8$)}& \multicolumn{5}{c}{SPEC-MTP~($\tau=1.4$)}
&\multicolumn{5}{c}{SPEC-MTP~($\tau=1.0$)}\\ \cmidrule{2-16}
&PA-MPJPE$\downarrow$ & MPJPE$\downarrow$  & PVE$\downarrow$ & mIoU$\uparrow$ &P-mIoU$\uparrow$& PA-MPJPE$\downarrow$ & MPJPE$\downarrow$ & PVE$\downarrow$ & mIoU$\uparrow$ &P-mIoU$\uparrow$& PA-MPJPE$\downarrow$ & MPJPE$\downarrow$ & PVE$\downarrow$ &  mIoU$\uparrow$  & P-mIoU$\uparrow$ \\ \midrule 

\rule{0pt}{10pt} HMR~(R50)~\cite{hmr}  & 73.9 & 121.4 & 145.6 & 48.8 & 16.0 & 73.1 & 112.5 & 135.7 & 51.1 & 20.0 & 69.6 & 111.8 & 135.7 & 50.5 & 21.8\\

\rule{0pt}{10pt} HMR-$f$~(R50)~\cite{hmr}& 72.7 & 123.2 & 145.1 & 52.3 & 21.0 & 72.1 & 113.3 & 135.5 & 51.9 & 21.9 & 69.1 & 112.8 & 136.3 & 52.5 & 24.8 \\

\rule{0pt}{10pt} SPEC~(R50)~\cite{spec}  &   76.0 & 125.5 & 144.6 & 49.9 & 18.8 & 72.4 & 114.0 & 134.3 & 49.3 & 19.5 & 67.4 & 110.6& 132.5 & 49.1 & 21.2  \\

\rule{0pt}{10pt} SPEC~*~(R50)~\cite{spec}  &  -& -& -& -& -& -& -& -& -& -& 71.8& 116.1&  136.4 & -& - \\


\rule{0pt}{10pt} CLIFF~(R50)~\cite{cliff} & 74.3 & 115.0 & 132.4 & 53.6 & 23.7 & 70.2 & 107.0 & 126.8 & 52.0 & 22.1 & 67.4 & 108.7 & 130.4 & 51.9 & 23.4\\



\rule{0pt}{10pt} PARE~(H48)~\cite{pare} & 74.2 & 121.6 & 143.6 & 55.8 & 23.2 & 71.6 & 112.7 & 137.2 & 55.1 & 22.4 & 68.5 & 113.5 & 139.6 & 55.3 & 25.1  \\

\midrule
\rule{0pt}{10pt} GraphCMR~(R50)   & 76.1 & 121.1 & 133.1 & 56.3 & 23.4 & 74.4 & 114.9 & 129.5 & 52.6 & 20.8 &  70.2 & 112.7 & 127.8 & 51.7 & 22.0\\

\rule{0pt}{10pt} FastMetro(H48)~\cite{fastmetro}   & 75.0   & 123.1  & 137.0 & 53.5  & 20.5 & 70.8 & 112.3 & 128.0 & 52.4 & 20.6 & 66.3 & 110.2 & 126.5 & 51.8 & 22.6   \\

\midrule
\rule{0pt}{10pt} $\rm \Ours^{P}$ (R50)  & 72.9 & 117.7 & 138.2 & 54.7 & 22.4 & 70.5 & 108.1 & 129.4 & 53.9 & 21.5 &  68.4 & 110.2 & 134.3 & 54.7 & 24.2 \\
\rule{0pt}{10pt} \Ours (R50)  & 74.0 & 122.1 & 135.6 & 58.9 & 24.9 & 70.3 & 111.1 & 126.0 & 56.9 & 22.0 & 66.9 & 109.6 & 124.4 & 56.5 & 23.4 \\
\rule{0pt}{10pt} \Ours (H48)  & \textbf{67.4} & \textbf{114.6} & \textbf{126.7} & \textbf{62.6} & \textbf{30.4}   & \textbf{66.5} & \textbf{106.1} & \textbf{120.1} & \textbf{59.9} & \textbf{26.6} & \textbf{65.8} & \textbf{108.2} & \textbf{121.9} & \textbf{58.5} & \textbf{27.0} \\
\bottomrule
\end{tabular}
}
    
    \caption{Results of SOTA methods on SPEC-MTP ($\tau=1.8$, $\tau=1.4$, $\tau=1.0$ protocols). SPEC-MTP~($\tau=1.0$) indicates the original SPEC-MTP~\cite{spec} dataset. SPEC~* terms the results reported in SPEC~\cite{spec}.}
    \label{tab:sota_supp_specmtp}
\end{table*}

% \subsection{HuMMan}

\begin{table*}[h]
    \centering
    \scalebox{0.61}{\begin{tabular}{lccccc|ccccc|ccccc}
\toprule
\multirow{2}{*}{\textbf{Methods}} & \multicolumn{5}{c}{HuMMan~($\tau=1.8$)}& \multicolumn{5}{c}{HuMMan~($\tau=1.4$)}
&\multicolumn{5}{c}{HuMMan~($\tau=1.0$)}\\ \cmidrule{2-16}
&PA-MPJPE$\downarrow$ & MPJPE$\downarrow$  & PVE$\downarrow$ & mIoU$\uparrow$ &P-mIoU$\uparrow$& PA-MPJPE$\downarrow$ & MPJPE$\downarrow$ & PVE$\downarrow$ & mIoU$\uparrow$ &P-mIoU$\uparrow$& PA-MPJPE$\downarrow$ & MPJPE$\downarrow$ & PVE$\downarrow$ &  mIoU$\uparrow$  & P-mIoU$\uparrow$ \\ \midrule 

\rule{0pt}{10pt} HMR~(R50)~\cite{hmr} & 30.2 & 43.6 & 52.6 & 65.1 & 39.5 & 31.9 & 45.0 & 39.5 & 66.6 & 39.9 & 30.0 & 44.1 & 50.7 & 66.6 & 39.5 \\

\rule{0pt}{10pt} HMR-$f$~(R50)~\cite{hmr}& 29.9 & 43.6 & 53.4 & 62.7 & 34.9 & 31.3 & 45.0 & 53.3 & 66.6 & 39.9 & 29.8 & 44.1 & 50.7 & 66.6 & 39.5 \\

\rule{0pt}{10pt} SPEC~(R50)~\cite{spec}  & 31.4 & 44.0 & 54.2 & 51.4 & 24.6 & 33.1 & 46.1 & 41.7 & 46.0 & 19.2 &   31.2 & 44.8 & 51.6 & 42.2 & 16.6 \\

\rule{0pt}{10pt} CLIFF~(R50)~\cite{cliff} &  28.6 & 42.4 & 50.2 & 68.9 & 44.7 & 30.3 & 43.3 & 51.2 & 70.2 & 44.9 & 28.3 & 42.3 & 48.5 & 70.6 & 44.5 \\



\rule{0pt}{10pt} PARE~(H48)~\cite{pare} & 32.6 & 53.2 & 65.5 & 64.5 & 38.3  & 33.6 & 53.3 & 66.2 & 65.1 & 38.0 & 32.2 & 53.1 & 64.6 & 65.0 & 37.6\\

\midrule
\rule{0pt}{10pt} GraphCMR~(R50)   & 29.5 & 40.6 & 48.4 & 61.6 & 37.5 & 30.3 & 40.6 & 48.2 & 62.6 & 37.6 & 29.3 & 40.2 & 46.3 & 62.8 & 37.0 \\

\rule{0pt}{10pt} FastMetro(H48)~\cite{fastmetro}   & 26.3 & 38.8 & 45.6 & 68.3 & 45.2 &   27.8 & 39.9 & 46.6 & 69.9 & 45.7 & 26.5 & 38.5 & 43.6 & 70.0 & 45.3 \\

\midrule
\rule{0pt}{10pt} $\rm \Ours^{P}$ (R50)  &  24.4 & 36.7 & 45.9 & 70.4 & \textbf{45.5} & 26.2 & 37.6 & 45.6 & 70.4 & 45.3 & 25.6 & 37.7 & 43.7 & 70.8 & 45.2 \\
\rule{0pt}{10pt} \Ours (R50)  &  25.5 & 36.7 & 43.4 & 67.0 & 38.4 & 25.6 & 36.5 & 42.5 & 70.4 & 42.7 & 24.2 & 35.2 & 40.4 & 70.7 & 42.4 \\
\rule{0pt}{10pt} \Ours (H48)  & \textbf{22.3} & \textbf{32.6} & \textbf{40.0} & \textbf{71.2} & 45.1 & \textbf{24.1} & \textbf{33.8} & \textbf{40.7} & \textbf{72.2} & \textbf{47.9} & \textbf{23.0} & \textbf{33.0} & \textbf{38.7} & \textbf{73.2} & \textbf{47.4}    \\
\bottomrule
\end{tabular}
}
    
    \caption{Results of SOTA methods on HuMMan ($\tau=1.8$, $\tau=1.4$, $\tau=1.0$ protocols).}
    \label{tab:sota_sup_humman}
\end{table*}

\section{Qualitative results.}
% \ding{177}


\noindent\textbf{Qualitative results on Human3.6M~\cite{h36m} dataset.}
We show qualitative results of \Ours on Human3.6M dataset in \cref{fig:h36m}
\begin{figure}[ht]
    \centering
    \includegraphics[width=1.\linewidth]{pictures/h36m.pdf}
    \caption{Qualitative results on Human3.6M dataset. The number under each image represents predicted/ground-truth focal length $f$, FoV angle, and z-axis translation $T_{z}$. Our method could predict an approximate translation for non-distorted images as well.}
    \label{fig:h36m}
    \vspace{-10pt}
\end{figure}



\begin{figure}[h]
    \centering
    \includegraphics[width=1\linewidth]{pictures/failure_cases.pdf}
    \caption{Failure cases. The left part is input, and the right part is our prediction.}
    \label{fig:failure}
    \vspace{-10pt}
\end{figure}

\noindent\textbf{Failure cases.}
% \subsection{Human3.6M}
% Our methods do not hold well in some extreme circumstances. 
%  As in \cref{fig:failure}, since our training data lack extremely big hands as (1)(2)(6) and extremely big feet as (7)(8), our method could not perform well on these images. And other circumstance as characters with extremely strong body shape as in (2)(3)(7). Our method also performs not well on self-occluded images like (4). 
Although our methodology is generally effective, it has trouble under certain extreme circumstances. As demonstrated in~\cref{fig:failure}, due to the lack of training data containing characters with large hands ((1), (2), and (6)), and large feet ((7) and (8)), \Ours produce sub-optimal results on such images. 
%
Similarly, our approach may not perform well on characters with exceptional body shapes, as exemplified by (2), (3), and (7), where the athletes have muscular bodies. Additionally, it is difficult for \Ours to reconstruct self-occluded human bodies, as depicted in (4). We are actively exploring strategies to address these limitations and improve the robustness of our methodology.

\noindent\textbf{More qualitative results on distorted images.}
We show more qualitative results of \Ours comparing with SOTA methods for perspective-distorted images on PDHuman (\cref{fig:supp_pdhuman_sota}), Web images (\cref{fig:supp_real_sota}), and SPEC-MTP (\cref{fig:supp_specmtp_sota}).

\clearpage



\begin{figure*}
    \centering
    \includegraphics[width=1.0\linewidth]{pictures/supp_pdhuman_sota.pdf}
    \caption{Qualitative results on PDHuman dataset. The number under each image represents predicted/ground-truth focal length $f$, FoV angle, and z-axis translation $T_{z}$. }
    \label{fig:supp_pdhuman_sota}
\end{figure*}


\begin{figure*}[htp]
    \includegraphics[width=1.0\linewidth]{pictures/supp_real.pdf}
\caption{Qualitative results on in-the-wild images. The number under each image represents the predicted focal length $f$, FoV angle, and z-axis translation $T_{z}$. Images are collected from \url{https://pexels.com} and \url{https://yandex.com}.
}
\label{fig:supp_real_sota}
\end{figure*}

\begin{figure*}
    \includegraphics[width=1.0\linewidth]{pictures/supp_specmtp.pdf}
    \caption{Qualitative results on SPEC-MTP dataset. The number under each image represents predicted/ground-truth focal length $f$, FoV angle, and z-axis translation $T_{z}$. The ground-truth $T_{z}$ and focal length $f$ for SPEC-MTP are pseudo labels.}
    \label{fig:supp_specmtp_sota}
\end{figure*}
% \subsection{HuMMan}

% \subsection{PDHuman}


% \section{Downstream tasks.}
% \paragraph{Differentiable rendering.}
% \paragraph{Test time optimization for fixed cameras.}

\clearpage
% \newpage
{\small
\bibliographystyle{ieee_fullname}
\bibliography{main}
}


\end{document}
