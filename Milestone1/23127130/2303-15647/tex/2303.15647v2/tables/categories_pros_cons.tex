\newcommand{\hug}{\includegraphics[height=1.3em]{img/hf-logo.png}}
\newcommand{\adapterhub}{\includegraphics[height=1.3em]{img/adapters-logo.png}}

\begin{table}[t]
\begin{small}
    \centering
    \setlength{\tabcolsep}{4pt} % Default value: 6pt
    \begin{tabular}{lc|ccc|cc|c|c}
        \toprule
        & & \multicolumn{3}{c|}{Additive} & \multicolumn{2}{c|}{Selective} & \multirow{2}{*}{Reparam.} & \multirow{2}{*}{Hybrid} \\
        & & Adapters & Soft prompts & Other & Structured & Sparse & & \\
        \midrule
% new row
\multicolumn{2}{l|}{Storage efficiency} & \yes & \yes & \yes & \yes & \yes & \yes & \yes \\
% new row
\midrule
\multirow{2}{*}{Training}& RAM
& \yes & \yes & \yes & \yes & \no & \yes & \orange{maybe} \\
% new row
& Speed &
\no & \no & \orange{maybe} & \orange{maybe} & \no & \yes & \orange{maybe} \\
% new row
\midrule
\multirow{2}{*}{Inference}&\footnotesize{Single-task}
&\multicolumn{3}{c|}{\no\ Adds overhead} &
\multicolumn{2}{c|}{\yes\ No overhead} & \yes    & \orange{maybe} \\
% new row
&\footnotesize{Mutli-task}&
\multicolumn{3}{c|}{\yes Allows efficient inference} & \orange{maybe} & \no & \orange{maybe} & \orange{maybe} \\
% new row
\midrule
\multicolumn{2}{l|}{Implemented in} & \adapterhub & \hug \adapterhub & \hug \adapterhub & \no & \no & \hug \adapterhub & \adapterhub\\
        \bottomrule
    \end{tabular}
    \caption{
    % PEFT category summary and comparison in terms of training and inference efficiency. For training, we compare if methods from the corresponding category are more memory-efficient (RAM) or faster (Speed) than full training. For inference, we indicate if the methods generally add overhead after training (e.g., having to inference original network and the added adapters). We also indicate if the method could be efficiently applied to multi-task inference. Specifically, if it's possible to have one backbone network with multiple task-specific adapters.
    % Finally, we incate if methods of this category are implemented in popular opensource libraries: HuggingFace PEFT (\hug) and in AdapterHub (\adapterhub).
Comparison of PEFT categories in terms of the axes outlined in Section \ref{sec:peft_comparison}. For training, we evaluate methods by their memory (RAM) efficiency and speed relative to full training. \yes\ means more effective than regular training, \no\ means less effective, and \orange{maybe} suggests it might vary based on the specific method within the category. For inference, the table shows whether a method introduces single-task overhead and if it facilitates efficient multi-task inference. Lastly, the table highlights if these methods are implemented in popular open-source libraries: HuggingFace PEFT (\hug) and AdapterHub (\adapterhub).
    }
    \label{tab:type_comparison}
\end{small}
\end{table}
