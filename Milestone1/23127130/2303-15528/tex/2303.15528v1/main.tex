\documentclass{bmvc2k}

%% Enter your paper number here for the review copy
% \bmvcreviewcopy{620}
\def\thefootnote{*}\footnotetext{Equal contribution}

\usepackage{multicol}
\usepackage{multirow}
\usepackage{amsfonts}
\usepackage{booktabs}
% \usepackage{float}
% \usepackage[caption = false]{subfig}
\usepackage{xfrac}    % for \xfrac macro
\usepackage{nicefrac} % for \nicefrac macro
\usepackage{amsmath}  % for \dfrac macro
\usepackage{url}

% \usepackage[utf8]{inputenc}
\usepackage{paralist}
\usepackage[inline]{enumitem}

\makeatletter
\def\thickhline{%
  \noalign{\ifnum0=`}\fi\hrule \@height \thickarrayrulewidth \futurelet
   \reserved@a\@xthickhline}
\def\@xthickhline{\ifx\reserved@a\thickhline
               \vskip\doublerulesep
               \vskip-\thickarrayrulewidth
             \fi
      \ifnum0=`{\fi}}
\makeatother

\newlength{\thickarrayrulewidth}
\setlength{\thickarrayrulewidth}{2\arrayrulewidth}

\title{Few-Shot Domain Adaptation for Low Light RAW Image Enhancement}

% Enter the paper's authors in order
% \addauthor{Name}{email/homepage}{INSTITUTION_CODE}
% \addauthor{K Ram Prabhakar}{http://www.vision.inst.ac.uk/~ss}{1}
\addauthor{K Ram Prabhakar}{ramprabhakar@iisc.ac.in}{1}
\addauthor{Vishal Vinod\thefootnote{}}{vishal114186@gmail.com}{1}
\addauthor{Nihar Ranjan Sahoo\thefootnote{}}{niharsahooigit@gmail.com}{1}
\addauthor{R. Venkatesh Babu}{venky@iisc.ac.in}{1}

% Enter the institutions
% \addinstitution{Name\\Address}
\addinstitution{
 Video Analytics Lab,\\
 Department of Computational and Data Sciences,\\
 Indian Institute of Science,\\
 Bangalore, India
} 
% \addinstitution{
%  Collaborators, Inc.\\
%  123 Park Avenue,\\
%  New York, USA
% }

\runninghead{Ram, Vishal, Nihar, Babu}{FSDA for Low Light RAW Enhancement}

% Any macro definitions you would like to include
% These are not defined in the style file, because they don't begin
% with \bmva, so they might conflict with the user's own macros.
% The \bmvaOneDot macro adds a full stop unless there is one in the
% text already.
\def\eg{\emph{e.g}\bmvaOneDot}
\def\Eg{\emph{E.g}\bmvaOneDot}
\def\etal{\emph{et al}\bmvaOneDot}

%-------------------------------------------------------------------------
% Document starts here
\begin{document}

\maketitle

\begin{abstract}
Enhancing practical low light raw images is a difficult task due to severe noise and color distortions from short exposure time and limited illumination. Despite the success of existing Convolutional Neural Network (CNN) based methods, their performance is not adaptable to different camera domains. In addition, such methods also require large datasets with short-exposure and corresponding long-exposure ground truth raw images for each camera domain, which is tedious to compile. To address this issue, we present a novel few-shot domain adaptation method to utilize the existing source camera labeled data with few labeled samples from the target camera to improve the target domain's enhancement quality in extreme low-light imaging. Our experiments show that only ten or fewer labeled samples from the target camera domain are sufficient to achieve similar or better enhancement performance than training a model with a large labeled target camera dataset. To support research in this direction, we also present a new low-light raw image dataset captured with a Nikon camera, comprising short-exposure and their corresponding long-exposure ground truth images. The code is available at \url{https://val.cds.iisc.ac.in/HDR/BMVC21/index.html}.
\end{abstract}

%-------------------------------------------------------------------------
\section{Introduction}
\label{sec:intro}
    \section{Introduction} \label{sec:intro}

Large amounts of time and effort are devoted to
verification and validation of every microprocessor design project.
Broadly, design verification can be broken into two large categories:
(1) functional and (2) performance verification, which is to identify design bugs that degrade performance without affecting functionality. Performance bugs are different from performance bottleneck as the former is due to design mistakes while the later is caused by tight resource constraints. Performance loss due to performance bugs  can 
be very significant, with recent reported cases shown to be
$>10\%$~\cite{mccalpin2018hpl}. This demonstrates a critical 
need for automated mechanisms for performance debugging.  As 
recent designs from Intel~\cite{corei7-11}, AMD~\cite{ryzen-9},
ARM~\cite{cortex-a}, and others place an even greater emphasis
on core performance, design complexity has scaled
dramatically, likewise scaling the difficulty in all forms of
verification.


%Functional verification has received extensive attention from researchers and, although complex, it benefits from the availability of known correct outputs that can be used to compare against.

Performance verification at microarchitecture level ensures that a
design correctly achieves expected performance in terms of execution
time or cycle count.  The main challenge in this task is that, unlike
functional verification, there is no exact golden reference to compare
against.  This is because of the high difficulty of modeling all the
interactions between the different units in complex microprocessor
designs, and accurately represent how they affect the overall system
performance.  %This task also suffers from 
%the lack of a good debugging infrastructure, as well as from 
%limited visibility into intermediate points in the design, which are mostly exposed through performance counters. Although useful for estimating the performance of the system, these counters are very difficult to use for manual debugging because of their complex relationship with processor performance and due to the large amounts of data they generate.  
Traditionally, performance
verification is conducted mostly through manual techniques which rely
on rough estimations of performance gain expected by
microarchitectural changes~\cite{Singhal2004}. Such manual processes
are not only very lengthy but also error-prone.



The process of performance verification and debugging roughly consists of two steps: (1)~detection, which determines whether a
design achieves expected performance or not, and (2)~localization,
which identifies the microarchitectural units causing the performance
issues and is the focus of this work.

There are few previous studies on automating detection of
microprocessor performance bugs~\cite{Bose1994,
  surya1994architectural,carvajal2021detection}. 
The majority of those~\cite{Bose1994, surya1994architectural} relies on capturing
design intentions using a bespoke performance model as a golden
reference, this  entails long development time and may contain
errors by itself. Recently, a data driven and machine learning
(ML)-based approach~\cite{carvajal2021detection} was developed for
automatic performance bug detection with high accuracy. Although
significant, these works do not solve the 
problem of performance bug localization.
%pressing problem of identifying where the performance bug is.

Works in automating microprocessor performance bug localization
are even scarcer.  Adir \emph{et al.}~\cite{adir2005generic}
propose perhaps the only partially related work of which we are
aware.  Their work focuses on formal planning of test program
generation for individual units, such as issue queues. This strategy
follows conventional functional verification, involving heavy
manual effort, costing significant engineer-time to develop a test
plan, and as much as ten days of computer runtime per functional
unit. To the best of our knowledge, there has been no systematic study
on automatic performance bug localization for microarchitecture
designs.

Performance bug localization is a complicated task, which is currently
solved using mostly manual techniques.
Even
in the more widely studied area of functional validation, the industry
lacks a standardized mechanism to automate bug localization, it has
been only recently that academic efforts have attempted to automate
this task~\cite{BugMD}. Considering this, it is important to note that
any type of design automation which successfully reduces the 
time and effort required by engineers to debug their designs is highly
valuable. Since automatic performance debug for microprocessors is
a huge yet under-studied challenge, it is very difficult, if not impossible, to find a perfect solution in a single work. Although our work is not perfect, it serves a key stepping stone 
 toward solving the problem.

This work tackles the performance bug localization problem by
using ML to generate a ranked list of most likely mi\-cro\-ar\-chi\-tec\-tur\-al units that 
might contain the bug.  This list may be used
to prioritize the debugging order, as well as to identify
teams with the right expertise to perform further debug. Two different methodologies are
proposed, evaluated, and contrasted. These data-driven
techniques achieve high
accuracy, while being fully automated. Further, they
consider intra- and inter-unit interactions, as opposed to other
techniques proposed in the partially related previous work~\cite{adir2005generic} which
considered only intra-unit behavior.

%Our methods are based on ML, wherein our models are
%trained using data from legacy designs.
%To take the full advantage of
%these approaches, we assume that architectural changes in a new design
%are incremental when compared to its previous
%generations. Examining recent processors from major vendors including
%Intel, ARM, and AMD, we find this assumption holds true, since the generational change in microarchitectures
%is largely incremental. Thus, the methodologies proposed here provide
%alue for a multitude of upcoming designs.  However, even when
%disruptive changes occur, the methodologies can still be beneficial for bug localization on structures that conform to previous microarchitectures, using workloads that
%do not exercise new functionalities. Further, as general purpose microarchitectures become ever more mature, and the inter-generational performance gains decrease, 
%it is even more important to retain as much performance as possible, making performance debugging ever more important.

The major contributions of this work include the following:
\begin{compactitem}
\itemsep0em 
\item This is the first systematic study on fully automatic
  performance bug localization for microarchitecture designs, to the
  best of our knowledge.

\item Two ML-based approaches to tackle performance bug
  localization, as well as a hybrid of both, are evaluated
  and contrasted.

\item For bugs with an average IPC impact greater than 1\%, our best
  performing methodology identifies the correct bug location as the
  most likely unit in $\sim77\%$ of the cases, and achieves over 90\%
  accuracy when the three most likely options (out of 11 possible) are
  considered.  

\item One of the proposed methodologies is not only very accurate localizing
performance bugs, but it can also be applied to confirm the results
of performance bug detection with high accuracy.

\item Although the focus of this work is on microprocessor core,
we evaluated our methodologies on the processor memory hierarchy. This evaluation
uses a different experimental setup, showing the robustness of the proposed techniques.

\end{compactitem}

As an early work on performance bug localization, the design of this study is subject to potential limitations, however, we feel it still represents a good first step towards solving the problem. The scope of our work and its limitations are as follows:
\begin{compactitem}

\item Legacy
designs with identified performance bugs are required, so that the ML
models can be trained. Bug-free legacy designs are required only 
in one of the methodologies, yet, if available, the other can take advantage of the additional data.
However, thanks to the thorough pre- and post-silicon
debug to which the designs are submitted, these legacy designs are
generally available.

\item We assume that only one bug is present at a time, 
in parallel to the single-fault model which is common practice
in VLSI testing works. As explained in Section~\ref{subsec:impl_bugs}, we still expect 
our methodologies to work well in the presence of multiple bugs in a single design.

\item Our methodologies do not provide a quantitative coverage guarantee.  
In general, performance bug
coverage is extremely difficult to define and is a potential
research topic on its own. We know of no prior work which presents a
definition of such coverage. Nonetheless, the evaluated bugs are based on published errata, cover a large amount of microarchitectural units and affect the system in a variety of ways. Thus, we feel these bugs represent a reasonable start for early work in this area.

\item We assume that there are no dramatic structural
microarchitectural changes between the legacy designs and the
designs under debug. Examining recent processors from major vendors, including Intel, ARM, and AMD, we find this assumption holds true, since the generational change in microarchitectures
is largely incremental. That said, even when larger shifts occur, the
methodologies can be partially reused. For example, consider the
introduction of the AVX instructions with Intel's Sandy Bridge
architecture in 2011.  Initially there would be no available data to
test these instructions using our methodologies, however the rest of
the Sandy Bridge design could be debugged with our methodology,
leveraging workloads that do not exercise the new instructions.  In
future implementations, data from Sandy Bridge can be used to build
the models required to use our methods for debugging AVX. 
%Further, as general purpose microarchitectures become even more mature, and the inter-generational performance gains decrease, it is even more important to retain as much performance as possible, making performance debugging even more important.}

\item We limit our evaluation to a pre-silicon setup, because
it is infeasible for us to inject known design bugs in silicon to
evaluate the methodologies.  Further, should our methodologies be
applied to a commercially available design, and an actual bug be
found and localized, we would not be able to verify that such
localization is correct without prior knowledge of its existence so
as to verify our findings. However, our methodologies can be applied in both pre- and post-silicon scenarios. During pre-silicon stages fixing performance 
bugs is easier and cheaper, 
the availability of performance counters is greater (due to the usage of a
simulator) and the counters can be sampled at a much faster rate. 
By using only counters available in-silicon, and adjusting the sampling frequency, we could use the proposed
methodologies during post-silicon stages. In post-silicon analysis the methodology
could be applied to longer workloads, providing a way to exercise complicated bugs that
are not possible to trigger with short pre-silicon traces.
%Further, we can follow hybrid approaches where the ML model training is performed using simulations, and the techniques are applied to data obtained from microprocessors during post-silicon debug. }

\end{compactitem}

Despite the aforementioned, we present a first, useful, yet attainable,
step towards the goal of performance bug localization, and we hope this work can draw the attention of the research community
to the broader performance validation domain.

\iffalse{
In Section~\ref{sec:scope} we describe the problem
formulation and outline the scope of this work.  We note that, to
date, very little work exists in automating performance bug
localization. 

Section~\ref{sec:methodology} 
describes the approaches developed to tackle the performance bug
localization task.  Section~\ref{sec:experimental_setup} provides
details of the architectures, and performance bugs used for
evaluation. Section~\ref{sec:evaluation} presents results obtained in
several experiments developed to evaluate the methodologies. A brief
review of previous work related to performance debugging is presented
in Section~\ref{sec:related_work}.  And finally,
Section~\ref{sec:conclusion} concludes the paper.
}
\fi
\vspace{-0.5cm}\section{Related Work}
\label{sec:related}
    \section{Related Work}
\label{sec:related}

\noindent \textbf{Action Recognition.}
%
Most recent approaches for action recognition are to exploit appearance and motion cues jointly and achieve remarkable success~\cite{feichtenhofer2019slowfast,i3d,lin2019tsm,huang2021tada,qing2022learning,wang2021oadtr,pei2022learning}.
%
Typically, two-stream networks~\cite{two-stream,two-stream-2,TSN} consist of two branches that explore spatial information and temporal dynamics, respectively.
%
% To overcome the limitation of 2D CNN in modeling long-range dependencies, some attempts~\cite{lin2019tsm,r2+1d,TDN} begin to introduce additional temporal mining operations.
Some attempts~\cite{lin2019tsm,r2+1d,TDN} introduce additional temporal mining operations to overcome the limited temporal information extraction ability of 2D CNN.
%
3D CNN-based methods~\cite{feichtenhofer2019slowfast, i3d,C3D} inflated 2D kernels for joint spatio-temporal modeling.
%
\cite{bai2020prototype} proposes the prototype similarity learning which pushes the learned representation to the corresponding prototype as close as possible, while our PSL keeps the differences among the same class.

\noindent \textbf{Open-set Action Recognition.} The related work of OSAR is limited~\cite{krishnan2018bar,shu2018odn,yang2019open,bao2021evidential}. Recently, \cite{bao2021evidential} systematically studies the OSAR problem and transfers several open-set image recognition methods to the video domain, including SoftMax~\cite{hendrycks2016baseline}, MC Dropout~\cite{gal2016dropout}, OpenMax~\cite{bendale2016towards}, and RPL~\cite{chen2020learning}. In the benchmark of~\cite{bao2021evidential}, the only two methods designed specifically for the video domain are BNN SVI~\cite{krishnan2018bar} and their proposed DEAR. BNN SVI is a Bayesian NN application in the OSAR, while DEAR adopts the deep evidential learning~\cite{amini2020deep} to calculate the uncertainty, and utilizes two modules to alleviate the over-confidence prediction and appearance bias problem, respectively. Existing methods pursue better uncertainty scores, while the objective of our PSL is to learn more diverse feature representations for better open-set distinguishability.

\noindent \textbf{Information Bottleneck Theory.} Based on the IB theory~\cite{tishby2000information,tishby2015deep}, the NN intends to extract minimum sufficient information of the inputs for the current task. More recent~\cite{tian2020makes,federici2020learning,wang2022rethinking} adopt the IB theory on unsupervised contrastive learning to analyze the representation learning behavior under the corresponding tasks. In this work, we provide a new view to analyze the OSAR problem based on the IB theory.        
\vspace{-0.5cm}\section{Proposed method}
\label{sec:method}
    % \begin{table}[t]
% \centering
% \caption{Details of the datasets used in our work.}
% \label{tab:datasets}
% \begin{tabular}{lccc}
% \hline
% Datasets & \begin{tabular}[c]{@{}c@{}}Exposure\\ Ratios\end{tabular} & \begin{tabular}[c]{@{}c@{}}Training\\ images\end{tabular} & \begin{tabular}[c]{@{}c@{}}Validation\\ images\end{tabular} \\ \hline
% Sony \cite{chen2018learning} & 90,150,300 & 161 &  36\\
% Nikon &  100, 300 & 53 &  24 \\
% Canon \cite{CanonLSID} & 50, 150, 300 & 44 &  21\\ \hline
% \end{tabular}
% \end{table}
%--------------------------------------------------------------------------------------------------------------------
% \begin{figure*}
% \begin{center}
% \includegraphics[width=\textwidth, clip, trim=0cm 15.55cm 0.75cm 0.05cm]{figures/FDA-LSID.png}
% \end{center}
%   \caption{Proposed few-shot domain adaptation model architecture.}
% \label{fig:model}
% \end{figure*}
% % \begin{figure*}
% % \begin{center}
% % \includegraphics[scale=0.42]{figures/convert.png}
% % \end{center}
% %   \caption{Source camera specific 16-to-8-bit converter.}
% % \label{fig:convert}
% % \end{figure*}
\begin{figure}[t]
\begin{tabular}{cc}
\includegraphics[page=1, width=5.8cm]{Images/nikon.png}&\hspace{+1mm}
\includegraphics[page=1, clip, trim=0.6cm 16.75cm 7.5cm 0.15cm, scale=0.9]{Images/block_diagram.pdf}\\
(a)&(b)
\end{tabular}
\caption{(a) Example short-exposure and long-exposure image pairs from the Nikon dataset. The short exposure images are almost entirely dark whereas the long-exposure images have immense scene information. (b) Overview of our few-shot domain adaptation method.}
\label{fig:nikon}
\label{fig:prop_overview}
\end{figure}
With a noisy raw image captured with low-exposure time (i.e., shutter speed) as input, our CNN-based approach is trained to predict a clean long-exposure sRGB output of the same scene. The input is multiplied by an exposure factor calculated by the ratio of output and input exposure times. For example, to generate a 10-second long exposure output, the input 0.1-second low exposure image must be multiplied by 100. As a result of this operation, along with illumination, the noise is also amplified proportionally. Since we multiply the factor in the unprocessed raw domain and expect the output in the sRGB domain, the network must learn camera hardware-specific enhancement as well as its entire ISP pipeline (lens correction, demosaicing, white balancing, color manipulation, tone curve application, color space transform, and Gamma correction). Thus, a model trained on one specific camera data (source domain) does not translate similar performance to a different camera (target domain), hence the domain gap. In this paper, we propose to transfer the enhancement task from large labeled source data and generate output in the target domain using few labeled target data.

\textbf{Problem formulation}: We denote source domain ($\mathbf{S}$) with input short-exposure images as $\{S_n\}$ and corresponding long-exposure ground truth as $\widehat{\mathbf{S}}\!=\!\widehat{S}_n$, $\forall n=1,\cdots,N$. Similarly, the target domain ($\mathbf{T}$) consists of input images $\{T_m\}$ and corresponding ground truth, $\widehat{\mathbf{T}}\!=\!\widehat{T}_m$, $\forall m=1,\cdots,M$. Note that $N$ is much greater than $M$, $N\gg M$. With both $\mathbf{S}$ and $\mathbf{T}$ as input, we train a CNN model ($\mathbb{N}$) to generate enhanced long-exposure output ($\widetilde{\mathbf{S}}$ and $\widetilde{\mathbf{T}}$). Our method is illustrated in Fig. \ref{fig:prop_overview}(b) with the source and target training pipelines. It is an end-to-end trainable deep network that takes the raw sensor arrays as input and performs image enhancement utilizing the source data for few-shot domain adaptation to the target data.

% \begin{figure*}[t]
%     \centering
%     \includegraphics[width=\linewidth]{3_BMVC/Images/Nikon-Results-1.pdf}
%     \caption{Qualitative comparison with methods tested on Nikon target images. (b) HDRCNN and (c) Unprocess are trained on Sony source and fine-tuned on 4-Nikon target images, LSID with (d) 4-Nikon target images and (e) full ($k$=53) Nikon training dataset, (f) Our few-shot domain adaptation approach with 4-Nikon target images and 161 Sony source images.}
%     \label{fig:nikon_eg1}
% \end{figure*}
% \begin{table}[t]
% \centering
% \caption{Quantitative comparison of Sony as source and Nikon as target dataset. The improvement of proposed method over only $k$ shot trained model is shown in brackets. The LSID model trained with full Nikon dataset ($k$=53) achieves 30.74dB PSNR and 0.803 SSIM.}
% \label{tab:nikon}
% \scalebox{0.735}{
% \begin{tabular}{@{}lccc|ccc@{}}
% \hline
%  & \multicolumn{3}{c|}{PSNR} & \multicolumn{3}{c}{SSIM} \\ \hline
% $k$ ($\rightarrow$) & 1 & 2 & 4 & 1 & 2 & 4 \\ \hline
% \begin{tabular}[c]{@{}l@{}}LSID\\ (only $k$ target)\end{tabular} & 23.20 $\pm$ 3.06 & 27.27 $\pm$ 0.384 & 28.05 $\pm$ 1.53 & 0.679 $\pm$ 0.172 & 0.819 $\pm$ 0.031 & 0.864 $\pm$ 0.0111 \\ \hline
% \begin{tabular}[c]{@{}l@{}}Proposed\\ ($k$ target + source)\end{tabular} & \begin{tabular}[c]{@{}c@{}}\textbf{25.27} $\pm$ 0.58\\ (+2.07)\end{tabular} &
% \begin{tabular}[c]{@{}c@{}}\textbf{28.06} $\pm$ 0.671\\ (+0.79)\end{tabular} &
% \begin{tabular}[c]{@{}c@{}}\textbf{30.30} $\pm$ 0.52\\ (+2.25)\end{tabular} & \begin{tabular}[c]{@{}c@{}}\textbf{0.860} $\pm$ 0.010\\ (+0.181)\end{tabular} &
% \begin{tabular}[c]{@{}c@{}}\textbf{0.909} $\pm$ 0.0028\\ (+0.090)\end{tabular} &
% % \begin{tabular}[c]{@{}c@{}}\textbf{0.913} $\pm$ 0.006\\ (+0.049)\end{tabular} \\ \midrule
% % \begin{tabular}[c]{@{}l@{}}LSID\\ (full target, $k$ = 53)\end{tabular} & \multicolumn{3}{c|}{30.74} & \multicolumn{3}{c}{0.803} \\ \bottomrule
% \begin{tabular}[c]{@{}c@{}}\textbf{0.913} $\pm$ 0.006\\ (+0.049)\end{tabular} \\ \hline
% \end{tabular}
% }
% \end{table}

\textbf{Encoders}: \label{sec:pipeline} The significant domain gap between the source and target domains necessitates the extraction of separate and independent features from each domain before processing with a shared enhancement network ($\mathbb{N}$). Hence, we use a source encoder ($\mathcal{E}_S$) and a target encoder ($\mathcal{E}_T$). We first pack the input raw sensor arrays into a four-channel vector (for Bayer arrays from Sony, Nikon, and Canon cameras) and subtract the black level (reference voltage). Then, the packed array is multiplied by the exposure ratio factor and passed as input to the respective domain encoder. It should be noted that the exposure ratio factors need not be the same between the source and the target domain (See Table \ref{tab:datasets}). For the encoder network, we use three convolutional layers with $\{16,32,64\}$ filters and 3$\times$3 kernel size. 

\textbf{Enhancement Network}: The source and target domain encoder features are passed separately to a shared common enhancement network, $\mathbb{N}$. By having a common enhancement network, the large pool of source data helps to improve the enhancement quality of $\mathbb{N}$, while the few target samples ensure that the output is in the target domain. We use U-Net architecture for the enhancement network. Further, the network has a pixel shuffle layer to convert 12-channel prediction to 16-bit three channel sRGB output. The objective of $\mathbb{N}$ is to enhance, denoise, perform other ISP operations (AWB, color manipulation, etc.), and finally demosaicking to generate an sRGB output. $\mathbb{N}$ generates enhanced output $\widetilde{\mathbf{T}}$ for the target domain data as, $\widetilde{\mathbf{T}} = \mathbb{N}\big(\mathcal{E}_T(\mathbf{T}) \big)$. Similarly, $\widetilde{\mathbf{S}}$ for the source domain as, $\widetilde{\mathbf{S}} = \mathbb{N}\big(\mathcal{E}_S(\mathbf{S}) \big)$.
%\footnote{Please refer to the supplementary material for detailed network definition.}

\textbf{Losses}: For the target domain, we compute the $\ell_1$ loss between the prediction ($\widetilde{\mathbf{T}}$) and the ground truth ($\widehat{\mathbf{T}}$) as, $\mathcal{L}_{target} = \ell_1\big(\widetilde{\mathbf{T}},\widehat{\mathbf{T}} \big)$. The source domain loss consists of two components: cosine similarity loss and SSIM loss. We compute cosine similarity between $\widetilde{\mathbf{S}}$ and $\widehat{\mathbf{S}}$ as, 
$
    \mathcal{L}_{CS}(\widetilde{\mathbf{S}},\widehat{\mathbf{S}})= 1 -  \frac{{\widetilde{\mathbf{S}} \cdotp \widehat{\mathbf{S}}}}{\|\widetilde{\mathbf{S}}\|\times\|\widehat{\mathbf{S}}\|}
$.
% \begin{equation}
%     \mathcal{L}_{CS}(\widetilde{\mathbf{S}},\widehat{\mathbf{S}})= 1 -  \frac{{\widetilde{\mathbf{S}} \cdotp \widehat{\mathbf{S}}}}{\|\widetilde{\mathbf{S}}\|\times\|\widehat{\mathbf{S}}\|}
% \end{equation}
Cosine similarity loss is weak supervision for the source domain and is used instead of $\ell_1$ loss since $N\gg M$, and using a strong supervision loss like $\ell_1$ optimizes for pixel values to train $\mathbb{N}$, making the network predict the output in the source domain even for target domain input. Cosine similarity loss ensures that the prediction and the ground truth are in a similar direction. Hence with $\mathcal{L}_{CS}$, $\mathbb{N}$ can still perform enhancement while predicting in target domain even for source domain input. Further, when trained with Sony as source and 4-shot Nikon as target (Table \ref{tab:ablation}) with $L_1$ loss for the source, we obtain only 27.14dB PSNR for target domain validation, whereas using $\mathcal{L}_{CS}$ loss for source achieves 30.30dB PSNR.

% \begin{figure*}[t]
%     \centering
%     \includegraphics[width=\linewidth]{3_BMVC/Images/Canon-Results-1.pdf}
%     \caption{Qualitative comparison with methods tested on Canon target images. (b)HDRCNN and (c) Unprocess are trained on Sony source and fine-tuned on 6-Canon target images, LSID with (d) 6-Canon target images and (e) full ($k$=44) Canon training dataset, (f) Proposed few-shot domain adaptation approach with 6-Canon target images and 161 Sony source images.}
%     \label{fig:canon_eg2}
% \end{figure*}

\begin{figure*}[t]
\centering
\subfigure{\includegraphics[width=\linewidth]{Images/Nikon-Results-1.pdf}}\\ \vspace{-1.65\baselineskip}
\subfigure{\includegraphics[width=\linewidth]{Images/Canon-Results-1.pdf}}
\caption{Qualitative comparison with methods tested on Nikon (top row) and Canon (bottom row) target images. (a) Input after multiplying by exposure factor, results from (b) HDRCNN and (c) Unprocess methods are after training on full Sony source and fine-tuning on $k$-shot target images, LSID with (d) $k$-shot target images and (e) full target training dataset ($k$=53 for Nikon and $k$=44 for Canon), (f) Proposed few-shot domain adaptation method with 161 Sony source images and 4-shot Nikon (top row) and 6-Canon (bottom row) target images.}
\label{fig:nikon_eg1}
\label{fig:canon_eg2}
\end{figure*}

% \begin{table}[t]
% \centering
% \caption{Quantitative comparison of Sony as source and Canon as target dataset. The improvement of proposed method over only $k$ shot trained model is shown in brackets. The LSID model trained with full Canon dataset ($k$=44) attains 32.32dB PSNR and 0.899 SSIM.}
% \label{tab:canon}
% \scalebox{0.73}{
% \begin{tabular}{@{}lccc|ccc@{}}
% \hline
%  & \multicolumn{3}{c|}{PSNR} & \multicolumn{3}{c}{SSIM} \\ \hline
% $k$ ($\rightarrow$) & 1 & 3 & 6 & 1 & 3 & 6 \\ \hline
% \begin{tabular}[c]{@{}l@{}}LSID\\ (only $k$ target)\end{tabular} & 21.54 $\pm$ 2.89 & 26.9 $\pm$ 2.37 & 29.36 $\pm$ 0.763 & 0.588 $\pm$ 0.182 & 0.785 $\pm$ 0.0051 & 0.829 $\pm$ 0.0073 \\ \hline
% % \begin{tabular}[c]{@{}l@{}}Proposed\\ ($k$ target + source)\end{tabular} & \begin{tabular}[c]{@{}c@{}}\textbf{24.29} $\pm$ 3.16\\ (+2.75)\end{tabular} & \begin{tabular}[c]{@{}c@{}}\textbf{28.78} $\pm$ 3.54\\ (+1.8)\end{tabular} & \begin{tabular}[c]{@{}c@{}}\textbf{33.22} $\pm$ 0.45\\ (+3.86)\end{tabular} & \begin{tabular}[c]{@{}c@{}}\textbf{0.623} $\pm$ 0.0074\\ (+0.035)\end{tabular} & \begin{tabular}[c]{@{}c@{}}\textbf{0.841} $\pm$ 0.0335\\ (+0.056)\end{tabular} & \begin{tabular}[c]{@{}c@{}}\textbf{0.896} $\pm$ 0.015\\ (+0.067)\end{tabular} \\ \midrule
% % \begin{tabular}[c]{@{}l@{}}LSID\\ (full target, $k$ = 45)\end{tabular} & \multicolumn{3}{c|}{32.32} & \multicolumn{3}{c}{0.899} \\ \bottomrule
% \begin{tabular}[c]{@{}l@{}}Proposed\\ ($k$ target + source)\end{tabular} & \begin{tabular}[c]{@{}c@{}}\textbf{24.29} $\pm$ 3.16\\ (+2.75)\end{tabular} & \begin{tabular}[c]{@{}c@{}}\textbf{28.78} $\pm$ 3.54\\ (+1.8)\end{tabular} & \begin{tabular}[c]{@{}c@{}}\textbf{33.22} $\pm$ 0.45\\ (+3.86)\end{tabular} & \begin{tabular}[c]{@{}c@{}}\textbf{0.623} $\pm$ 0.0074\\ (+0.035)\end{tabular} & \begin{tabular}[c]{@{}c@{}}\textbf{0.841} $\pm$ 0.0335\\ (+0.056)\end{tabular} & \begin{tabular}[c]{@{}c@{}}\textbf{0.896} $\pm$ 0.015\\ (+0.067)\end{tabular} \\ \hline
% \end{tabular}
% }
% \end{table}

From experiments (in section \ref{sec:exp}), we find better enhancement (in terms of PSNR) using the structural similarity index measure (SSIM) \cite{wang2004image} to compute perceived degradation and preserve the spatial structure in the source output with respect to the ground truth. We do not use SSIM directly on the 16-bit data as that causes the source data to heavily influence the domain adaptation since the source dataset is much larger. Hence, we apply SSIM in JPEG compressed 8-bit domain, where the structural domain difference is less. Since type-casting the 16-bit data to 8-bit will still possess domain-specific details, we train a 16-to-8-bit U-net model ($\mathcal{D}$ in Fig. \ref{fig:prop_overview}) to convert the output from 16-bit to post-processed 8-bit representation. 

The $\mathcal{D}$ network is trained to perform the following non-linear operations: White balancing, Gamma correction, Quantization, and JPEG compression. Even after JPEG compression, the prediction may have traces of source domain specific color information. Further, the SSIM loss is a strong pixel-wise supervision, and in order to avoid the source domain from heavily influencing $\mathbb{N}$, we compute SSIM loss only in grayscale space, not in RGB color space. Also, it follows the intuition that the structure and edge information of a scene will remain the same across images captured with different cameras, while the color space representation may vary. We find that without SSIM loss for the source, we obtain 29.38dB PSNR on target domain validation, whereas using SSIM loss achieves 30.30dB PSNR (Table \ref{tab:ablation}). For computing the SSIM loss, the ground truth ($\widehat{\mathbf{S}}$) is also converted offline to post-processed 8-bit data ($\widehat{\mathbf{S}}_{PP}$) using the rawpy post process function. Hence, the loss is obtained by computing SSIM loss between $\mathcal{D}(\widetilde{\mathbf{S}})$ and $\widehat{\mathbf{S}}_{PP}$,
$
    \mathcal{L}_{SSIM} = 1 - SSIM\Big(\mathcal{D}(\widetilde{\mathbf{S}}), \widehat{\mathbf{S}}_{PP}\Big)
$.
% \begin{equation}
%     \mathcal{L}_{SSIM} = 1 - SSIM\Big(\mathcal{D}(\widetilde{\mathbf{S}}), \widehat{\mathbf{S}}_{PP}\Big)
% \end{equation}
In Fig. \ref{fig:prop_overview}, the top branch guided by the deep red arrows shows the entire source camera training pipeline. It should be noted that $\mathcal{D}$ is used only to compute the loss but not in inference. Finally, we use the sum of cosine similarity loss ($\mathcal{L}_{CS}$) as well as the SSIM loss calculated in the 8-bit domain as the total loss for the source camera pipeline: $\mathcal{L}_{source} = \mathcal{L}_{CS} + \mathcal{L}_{SSIM}$. The total loss is the sum of target and source domain losses: $\mathcal{L}_{total}=\mathcal{L}_{target}+\mathcal{L}_{source}$.
% Source and target models are trained jointly. An
% epoch consists of 161 batches, with one source patch and
% one target domain patch per batch. Both source and target
% patches (of size 512 512, lines 478-480) are cropped at a
% random location from a randomly chosen source and target
% image.

% For the proposed method, we use the respective short-exposure raw sensor data as input to the source ($\mathbf{S}$) and target ($\mathbf{T}$) encoder networks. We first pack the input raw sensor arrays into a four-channel vector (for Bayer arrays from Sony, Nikon, and Canon cameras), subtract the black level (reference voltage), and multiply the input with the exposure ratio. There is one input for the source camera encoder ($\mathcal{E}_S$) and one for the target camera encoder ($\mathcal{E}_T$) in every training step. We pass the output from these camera-specific encoders through a shared $\mathbb{N}$ network to allow the model to learn both camera-specific and camera invariant properties. The output of the $\mathbb{N}$ network is a 12-channel image with half the spatial resolution. % , comprising of a U-net \cite{ronneberger2015u} followed by three CNN layers, 

% \begin{figure}[t]
%     \centering
%     \label{fig:prop_overview}
%     \includegraphics[page=1, clip, trim=0.6cm 16.75cm 7.5cm 0.15cm, scale=0.9]{3_BMVC/Images/block_diagram.pdf}
%     \caption{Overview of our few-shot domain adaptation model. }
% \end{figure}


% \subsection{Source and Target camera pipeline}\label{sec:pipeline}
% The source and target pipelines are trained jointly in an end-to-end manner. An epoch consists of 161 batches, with one source domain patch and one target domain patch per batch 
%given as input to the model. Both source and target patches are $512\times512$ random crops. % are cropped at a random location from a random source and target domain image at each train step.

% \textbf{Source camera pipeline.}
% \label{subsec:source}
% The packed raw input arrays from the source domain are passed to the source encoder, which learns camera-specific parameters to obtain an intermediate representation. We find visible denoising performance from experiments with the encoder heads, and subsequent analysis suggests effective learning of the camera's non-uniform noise model in extreme low-light conditions. The output from the source encoder is passed through the shared $\mathbb{N}$ network to obtain the source output feature maps, $\widetilde{\mathbf{S}}$. Bayer conversion with a sub-pixel layer \cite{shi2016real} unpacks the 12-channel data into a full resolution sRGB image.

% We compute the standard Cosine Similarity loss ($\mathcal{L}_{CS}$) between the source domain output ($\widetilde{\mathbf{S}}$) and the corresponding source domain long-exposure ground-truth image ($\widehat{\mathbf{S}}$),
% % \begin{equation}
% % \mathcal{L}_{CS}(\widetilde{\mathbf{S}},\widehat{\mathbf{S}})= 1 -  \frac{{\widetilde{\mathbf{S}} \cdotp \widehat{\mathbf{S}}}}{\bf \text{max}( \sqrt{({\bf \tilde{S}})^2} \cdotp \sqrt{({\bf \widehat{S}})^2})}
% % \end{equation}
% \begin{align}
%     \widetilde{\mathbf{S}} &= \mathbb{N}\big(\mathcal{E}_S(\mathbf{S}) \big), &    \mathcal{L}_{CS}(\widetilde{\mathbf{S}},\widehat{\mathbf{S}})&= 1 -  \frac{{\widetilde{\mathbf{S}} \cdotp \widehat{\mathbf{S}}}}{\|\widetilde{\mathbf{S}}\|\times\|\widehat{\mathbf{S}}\|}
% \end{align}
% % \begin{equation}
% % \mathcal{L}_{CS}(\widetilde{\mathbf{S}},\widehat{\mathbf{S}})= 1 -  \frac{{\widetilde{\mathbf{S}} \cdotp \widehat{\mathbf{S}}}}{\|\widetilde{\mathbf{S}}\|\times\|\widehat{\mathbf{S}}\|}
% % \end{equation}
  
% Cosine similarity loss is a weak supervision for the source domain and is used instead of $\ell_1$ loss since $N\gg M$, and using a strong supervision loss like $\ell_1$ optimizes for pixel values to train $\mathbb{N}$, making the network predict the output in the source domain even for target domain input. Cosine similarity loss ensures that the prediction and the ground truth are in a similar direction. Hence with $\mathcal{L}_{CS}$, $\mathbb{N}$ can still perform enhancement while predicting in target domain even for source domain input. Further, when trained with Sony as source and 4-shot Nikon as target (Table \ref{tab:ablation}) with $L_1$ loss for the source, we obtain only 27.14dB PSNR for target domain validation, whereas using $\mathcal{L}_{CS}$ loss for source achieves 30.30dB PSNR.

% From experiments (discussed in section \ref{sec:exp}), we find better enhancement (in terms of PSNR) using the structural similarity index measure (SSIM) \cite{wang2004image} to compute perceived degradation and preserve the spatial structure in the source output with respect to the ground truth. We do not use SSIM directly on the 16-bit data as that causes the source data to heavily influence the domain adaptation since the source dataset is much larger. Thus, we apply SSIM in JPEG compressed 8-bit domain, where the structural domain difference is less. Since type-casting the 16-bit data to 8-bit will still possess domain-specific details, we train a 16-to-8-bit U-net model ($\mathcal{D}$ in Fig. \ref{fig:prop_overview}) to convert the output from 16-bit to post-processed 8-bit representation. We find that without SSIM loss for the source, we obtain 29.38dB PSNR on target domain validation, whereas using SSIM loss achieves 30.30dB PSNR (Table \ref{tab:ablation}).

% % (we discuss the converter in section \ref{sec:ablation})

% For computing the SSIM loss, the ground truth ($\widehat{\mathbf{S}}$) is also converted offline to post-processed 8-bit data ($\widehat{\mathbf{S}}_{PP}$) using the rawpy post process function. Hence, the loss is obtained by computing SSIM loss between $\mathcal{D}(\widetilde{\mathbf{S}})$ and $\widehat{\mathbf{S}}_{PP}$.
% \begin{equation}
%     \mathcal{L}_{SSIM} = 1 - SSIM\Big(\mathcal{D}(\widetilde{\mathbf{S}}), \widehat{\mathbf{S}}_{PP}\Big)
% \end{equation}
% In Fig. \ref{fig:prop_overview}, the top branch guided by the deep red arrows shows the entire source camera training pipeline. Finally, we use the sum of cosine similarity loss ($\mathcal{L}_{CS}$) as well as the SSIM loss calculated in the 8-bit domain as the total loss for the source camera pipeline. 
% \begin{equation}
%     \mathcal{L}_{source} = \mathcal{L}_{CS} + \mathcal{L}_{SSIM}
% \end{equation}


% % \begin{figure*}[t]
% %     \centering
% %     \includegraphics[width=17.4cm, height=4.75cm]{ICCV/figures/Nikon-Results-1.pdf}
% %     \caption{Qualitative comparison between different methods tested on images from the Nikon dataset (target). The models are trained on (b) only 4 Nikon images, (c) full Nikon training dataset, and (d) 4 Nikon images and 161 Sony images with our proposed approach.}
% %     \label{fig:nikon_eg1}
% % \end{figure*}

% \textbf{Target camera pipeline.} 
% \label{subsec:target}
% We pass the packed target input raw data to the target encoder ($\mathcal{E}_T$). We then pass the encoded feature maps through the $\mathbb{N}$ network and then through the Bayer converter and calculate the target pipeline's loss in the 16-bit space. From several experiments with various loss functions, we have found the best image enhancement for the target pipeline is achieved with the $\ell_{1}$ loss between the predicted ($\widetilde{\mathbf{T}}$) and ground truth ($\widehat{\mathbf{T}}$),
% \begin{align}
%     \widetilde{\mathbf{T}} &= \mathbb{N}\big(\mathcal{E}_T(\mathbf{T}) \big), &    \mathcal{L}_{target} &= \ell_1\big(\widetilde{\mathbf{T}},\widehat{\mathbf{T}} \big)
% \end{align}

\vspace{-0.5cm}\section{Experiments}
\label{sec:data}
    % \subsection{Source and target datasets} 
\label{section:fsc}
\textbf{Datasets}: We use the Sony camera dataset \cite{chen2018learning} for our source training pipeline. We expect the diverse, high-quality low-light scenes from this dataset to aid few-shot domain adaptation performance in terms of the color spaces and noise model learned by our method. 
\begin{table}[t]
\centering
\caption{Quantitative comparison of Sony as source with Nikon or Canon as target. The improvement of our method over only $k$ shot trained model is in brackets. The LSID model trained with full Nikon dataset ($k$=53) achieves 30.74dB PSNR and 0.803 SSIM and when trained with full Canon dataset ($k$=44) attains 32.32dB PSNR and 0.899 SSIM (see Table \ref{tab:baseline}).}
\label{tab:nikon}
\label{tab:canon}
\scalebox{0.735}{
\begin{tabular}{@{}lccc|ccc@{}}
\thickhline
\textbf{Nikon as target} & \multicolumn{3}{c|}{PSNR} & \multicolumn{3}{c}{SSIM} \\ \hline
$k$ ($\rightarrow$) & 1 & 2 & 4 & 1 & 2 & 4 \\ \hline
\begin{tabular}[c]{@{}l@{}}LSID\\ (only $k$ target)\end{tabular} & 23.20 $\pm$ 3.06 & 27.27 $\pm$ 0.384 & 28.05 $\pm$ 1.53 & 0.679 $\pm$ 0.172 & 0.819 $\pm$ 0.031 & 0.864 $\pm$ 0.011 \\ \hline
\begin{tabular}[c]{@{}l@{}}Proposed\\ ($k$ target + source)\end{tabular} & \begin{tabular}[c]{@{}c@{}}\textbf{25.27} $\pm$ 0.58\\ (+2.07)\end{tabular} &
\begin{tabular}[c]{@{}c@{}}\textbf{28.06} $\pm$ 0.671\\ (+0.79)\end{tabular} &
\begin{tabular}[c]{@{}c@{}}\textbf{30.30} $\pm$ 0.52\\ (+2.25)\end{tabular} & \begin{tabular}[c]{@{}c@{}}\textbf{0.860} $\pm$ 0.010\\ (+0.181)\end{tabular} &
\begin{tabular}[c]{@{}c@{}}\textbf{0.909} $\pm$ 0.003\\ (+0.090)\end{tabular} &
% \begin{tabular}[c]{@{}c@{}}\textbf{0.913} $\pm$ 0.006\\ (+0.049)\end{tabular} \\ \midrule
% \begin{tabular}[c]{@{}l@{}}LSID\\ (full target, $k$ = 53)\end{tabular} & \multicolumn{3}{c|}{30.74} & \multicolumn{3}{c}{0.803} \\ \bottomrule
\begin{tabular}[c]{@{}c@{}}\textbf{0.913} $\pm$ 0.006\\ (+0.049)\end{tabular} \\ \hline \hline
\textbf{Canon as target} &  &  &  &  &  &  \\ \hline
$k$ ($\rightarrow$) & 1 & 3 & 6 & 1 & 3 & 6 \\ \hline
\begin{tabular}[c]{@{}l@{}}LSID\\ (only $k$ target)\end{tabular} & 21.54 $\pm$ 2.89 & 26.9 $\pm$ 2.37 & 29.36 $\pm$ 0.763 & 0.588 $\pm$ 0.182 & 0.785 $\pm$ 0.005 & 0.829 $\pm$ 0.007 \\ \hline
% \begin{tabular}[c]{@{}l@{}}Proposed\\ ($k$ target + source)\end{tabular} & \begin{tabular}[c]{@{}c@{}}\textbf{24.29} $\pm$ 3.16\\ (+2.75)\end{tabular} & \begin{tabular}[c]{@{}c@{}}\textbf{28.78} $\pm$ 3.54\\ (+1.8)\end{tabular} & \begin{tabular}[c]{@{}c@{}}\textbf{33.22} $\pm$ 0.45\\ (+3.86)\end{tabular} & \begin{tabular}[c]{@{}c@{}}\textbf{0.623} $\pm$ 0.0074\\ (+0.035)\end{tabular} & \begin{tabular}[c]{@{}c@{}}\textbf{0.841} $\pm$ 0.0335\\ (+0.056)\end{tabular} & \begin{tabular}[c]{@{}c@{}}\textbf{0.896} $\pm$ 0.015\\ (+0.067)\end{tabular} \\ \midrule
% \begin{tabular}[c]{@{}l@{}}LSID\\ (full target, $k$ = 45)\end{tabular} & \multicolumn{3}{c|}{32.32} & \multicolumn{3}{c}{0.899} \\ \bottomrule
\begin{tabular}[c]{@{}l@{}}Proposed\\ ($k$ target + source)\end{tabular} & \begin{tabular}[c]{@{}c@{}}\textbf{24.29} $\pm$ 3.16\\ (+2.75)\end{tabular} & \begin{tabular}[c]{@{}c@{}}\textbf{28.78} $\pm$ 3.54\\ (+1.8)\end{tabular} & \begin{tabular}[c]{@{}c@{}}\textbf{33.22} $\pm$ 0.45\\ (+3.86)\end{tabular} & \begin{tabular}[c]{@{}c@{}}\textbf{0.623} $\pm$ 0.007\\ (+0.035)\end{tabular} & \begin{tabular}[c]{@{}c@{}}\textbf{0.841} $\pm$ 0.033\\ (+0.056)\end{tabular} & \begin{tabular}[c]{@{}c@{}}\textbf{0.896} $\pm$ 0.015\\ (+0.067)\end{tabular} \\ \thickhline
\end{tabular} }
\end{table}

% \begin{table}
% \begin{center}
% \begin{tabular}{|l|c|c|c|}
% \hline
% Camera & Array & \#Train & \#Test & Ratios \\
% \hline\hline
% Sony & Bayer & 161 & 37 & 100\\
% Sony & Bayer & 161 & 37 & 100, 250, 300\\
% Sony & Bayer & 161 & 37 & 100, 250, 300\\
% \hline
% \end{tabular}
% \end{center}
% \caption{Dataset Info. }
% \end{table}

% \begin{figure*}
% \begin{center}
% \includegraphics[scale=0.4]{figures/fuji_nikon.png}
% \includegraphics[scale=0.4]{figures/fuji_nikon.png}
% \end{center}
%   \caption{Proposed few-shot domain adaptation model architecture.}
% \label{fig:model}
% \end{figure*}

%---------------------------------------------------------------------------------------------------------------
% \begin{figure}[t]
% \begin{center}
% \includegraphics[width=0.7\linewidth]{3_BMVC/Images/nikon.png}
% \end{center}
%   \caption{Examples of short-exposure and long-exposure image pairs from the Nikon dataset. Note that the short exposure images are almost entirely dark whereas the long-exposure images possess immense scene-related information.}
% \label{fig:nikon}
% \end{figure}

For few-shot domain adaptation, we work with very few (< 10) target camera images in every training experiment. We use the open-source Canon camera low-light raw image dataset \cite{CanonLSID} and a new Nikon camera dataset that we have compiled and make available with this work for our target camera training pipeline. We do not investigate burst denoising or the `lucky imaging' phenomenon. Hence, we only take the first short-exposure raw image for each scene from the Sony dataset and use the 161 images for our source camera training pipeline. Note that the Canon dataset has eight different ratios with close ranges such that they can be put into three buckets of ratios: 50, 150, and 300 (Table \ref{tab:datasets}). % We also show qualitative results for low-cost smartphone camera sensors - Google Pixel and OnePlus 5. 
%, to evaluate our approach on severe noise from low-cost smartphone camera sensors. %We also modify the Canon camera dataset to increase the number of training images by transferring a few validation images to the train set. This allows our baselines to learn the camera parameters in a better manner while we stick to few-shot training. 

% We also experiment with raw low-light images taken with smartphone cameras - Google Pixel and OnePlus 5, to investigate the performance of our proposed approach on raw images captured with low-cost smartphone camera sensors, which typically have higher noise severity in low-light conditions.

\textbf{Nikon camera dataset}:
\label{section:nikon}
We have compiled a dataset of raw low-light images captured with a Nikon D5600 camera to train the proposed few-shot domain adaptation architecture. The Nikon dataset consists of short-exposure images captured at $\nicefrac{1}{3}$ or $\nicefrac{1}{10}$ seconds and corresponding ground-truth long-exposure images captured at 10 or 30 seconds in the NEF format. For uniformity, there are two short-exposure images for every long-exposure image such that the exposure ratio (ratio of exposure time between the ground-truth long-exposure image and the input short-exposure image) is 100 and 300, respectively. Similar to \cite{chen2018learning}, we mount the camera on sturdy tripods and use appropriate camera settings to capture the static scenes using a smartphone app. The images captured include 129 short-exposure and 65 long-exposure ground-truth images of indoor and outdoor low-light scenes (sub lux). %Our aim in compiling the Nikon camera raw image dataset is to capture realistic low-light scenes and investigate few-shot domain adaptation with a dataset containing few images. %As discussed in section \ref{sec:related}, several related works compare results with datasets containing as few as only up to 10 images, and we seek to replicate these experimental setups. 

\textbf{Training Setup:} \label{sec:train}
% \textbf{Training $\mathcal{D}$}: The 16-to-8 bit converter used for the source pipeline is a fully-convolutional U-net \cite{ronneberger2015u} network architecture, trained with the ground-truth image pairs of 16-bit and 8-bit representations. We use the $L_{2}$ loss for training the converter and train it for 4000 epochs. We use the trained 16-to-8-bit converter for our model training.
We train the source and target pipelines simultaneously in an end-to-end manner. As discussed in section \ref{sec:pipeline}, we use the respective short-exposure raw images as the input to each of the encoders. We first randomly crop a $512\times512$ image patch and augment it with random-flip and random-rotate. We then subtract the black level and multiply the input raw image with the exposure ratio. We use an initial learning rate of $10^{-4}$ up to 2000 epochs and then reduce it by a factor of 10 for every 1000 epochs thereafter. We use the Adam \cite{kingma2014adam} optimizer for the 8-bit SSIM loss and Cosine Similarity loss ($\mathcal{L}_{source}$) for the source pipeline, and the $\mathcal{L}_{target}$ loss for the target pipeline.

We train the model for 4000 epochs (same as \cite{chen2018learning}) but observe the loss saturating at lower epochs prompting us to employ early stopping. Since the large source domain has 161 images, every epoch has 161 train steps. As we jointly train the source and target pipelines, for every epoch, we use 161 randomly cropped source patches obtained from the 161 source domain raw images and 161 randomly cropped target patches from only $k$-images in the target domain. We find that training our proposed method for up to 2500 epochs is sufficient to obtain the best results and reproduce the results in this paper for few-shot domain adaptation.

The source SSIM loss is calculated in the 8-bit space after passing the output from the shared $\mathbb{N}$ through the 16-to-8-bit converter. The cosine similarity loss for the source domain and $L_{1}$ loss for the target domain are computed in the 16-bit sRGB space. The exposure ratio is computed and provided to the network. At inference time, we use the full-scale raw target image as input to the target camera pipeline and obtain the enhanced target sRGB image. % in the 16-bit sRGB space.
% \begin{table}[t]
%     \centering
%     \caption{Details of the datasets used in our work.}
%     \label{tab:datasets}
%     \scalebox{0.85}{
%     \begin{tabular}{lccc}
%     \hline
%     \multirow{2}{*}{Datasets} & Exposure & Training & Validation \\
%     & Ratios & Images & Images \\
%     \hline
%     Sony \cite{chen2018learning} & 90,15,300 & 161 & 36\\
%     Nikon & 100,300 & 53 & 24\\
%     Canon \cite{CanonLSID} & 50,150,300 & 44 & 21 \\ \hline
%     \end{tabular}
%     }
%     \vspace{-2.5mm}
% \end{table}

% The images in our proposed dataset was captured with ISO 100-200. Other datasets were captured with ISO between 50-200. The range of ISO does not affect the model's performance due to the wide range of amplification factors used to train the model. A single encoder is sufficient to handle all ISO values.

% \begin{table}[t]
% \centering
% \begin{tabular}{lccc}
% \hline
% Datasets & \begin{tabular}[c]{c}Exposure\\ Ratios\end{tabular} & \begin{tabular}[c]{c}Training\\ images\end{tabular} & \begin{tabular}[c]{c}Validation\\ images\end{tabular} \\ \hline
% Sony & 90,150,300 & 161 &  36\\ %\cite{chen2018learning}
% Nikon &  100, 300 & 53 &  24 \\
% Canon & 50, 150, 300 & 44 &  21\\ \hline %\cite{CanonLSID}
% \end{tabular}
% \caption{Details of the datasets used in our work.}
% \label{tab:datasets}
% \end{table}
    
\vspace{-0.5cm}\section{Results}
\label{sec:exp}
    \section{Experiments} 
\label{subsection:baseline_evaluation}
We compare various state-of-the-art (SOTA) VLMs with NSG on the \etv benchmark (see Appendix~\ref{appendix:nsg_training} for NSG's experimental training details).

\subsection{SOTA VLM Baselines}
We investigate 6 VLMs developed for video-language tasks requiring similar reasoning as \etv. Summarized in Table~\ref{table:baseline_results}, \textbf{CLIP4Clip}~\cite{luo2022clip4clip}, \textbf{CLIP Hitchhiker}~\cite{clip_hitchiker}, \textbf{CoCa}~\cite{coca} use image backbones followed by temporal aggregation, while \textbf{VideoCLIP}~\cite{videoclip}, \textbf{MIL-NCE}~\cite{miech2020end}, and \textbf{VIOLIN}~\cite{violin_dataset} use video backbones. With the exception of CoCa, which is trained with contrastive and captioning loss, all other models are trained using contrastive loss~\cite{miech2020end}. Lastly, VideoCLIP and VIOLIN use an explicit fusion of text-vision features. For each model, we freeze all pretrained feature extractors and finetune a fully-connected probe layer, along with the temporal aggregation layers where appropriate (CLIP4Clip-LSTM, VIOLIN), using \etv's train split. 

Finally, to establish upper bounds on \etv, we instantiate: (1) a \textbf{Text2text model}, which constructs video captions using ground-truth labels for objects and actions, encodes the captions and task descriptions using (pretrained) RoBERTa model~\cite{roberta} and measures alignment using the cosine similarity score (see Appendix~\ref{appendix:upper_bound_models}), and (2) an \textbf{Oracle model}, which is trained with full supervision on sub-tasks labels and locations in addition to task verification labels.

%%%%%%%%%%%%%%%%%%%%%%%%%%%%%%%%%%%%%%%%%%%%%%%%%%%%%%%%%%%%%%%%%%%%%%%
\subsection{Results}
 In Table~\ref{table:baseline_results}, we show the performance of NSG vs. SOTA VLMs per split of \etv.~(1)~\textbf{Novel Tasks}: NSG significantly outperforms other baselines due to its ability to decompose and detect sub-tasks while using DP alignment to handle temporal constraints among them. In contrast, other baselines rely on detecting the entire task under temporal constraints, which is more challenging. Further, image-based baselines outperform video-based baselines due to their ability to capture a greater degree of compositional detail through frame-level representations.~(2) \textbf{Novel Steps}: NSG's poor performance in this split could be attributed to its low precision in the \emph{slice} sub-task (which is dominant in this split), as shown in Figure~\ref{figure:complexity-confusion_mat} [Right]. We hypothesize that since NSG only uses the aligned segments while discarding the rest, learning to utilize context from neighboring segments to capture \emph{slice} (like picking up a knife) could be a promising future direction. (3) \textbf{Novel Scenes}: Here, NSG is comparable to the best baseline VIOLIN-ResNet. Since the tasks are identical to the train split, the success of a model is contingent on the vision encoder's ability to accurately detect the same sub-tasks in unseen scenes. Consequently, models with an additional temporal aggregation layer (VIOLIN) finetuned on \etv, tend to outperform image-based models that do not have temporal aggregation (CLIP Hitchhiker) and models with frozen video features (MIL-NCE, VideoCLIP). (4) \textbf{Abstraction}: NSG significantly outperforms the baselines, primarily due to its semantic parser, which captures the underlying structure of the description and encodes the relevant concepts, such as objects and sub-tasks, to generate an (abstract) symbolic output.

%%%%%%%%%%%%%%%%%%%%%%%%%%%%%%%%%%%%%%%%%%%%%%%%%%%%%%%%%

\subsection{Analysis of NSG}
\label{subsec:analysis}

\noindent \textbf{NSG learns to localize task-relevant entities without explicit supervision.} Figure~\ref{figure:complexity-confusion_mat} shows the confusion matrix of \code{StateQuery} \& \code{RelationQuery} outputs, which capture sub-tasks, with their ground truths. The high recall demonstrates NSG's ability to localize task-relevant entities, despite being trained using only task verification labels.

\noindent \textbf{Effect of query types on NSG.} While query types with multiple entity arguments might appear capable of modeling complex dependencies amongst entities and having more expressive power, encoding multiple entities jointly using a single encoder makes the grounding problem more challenging. Hence, in practice, we found that using a combination of \code{StateQuery} \& \code{RelationQuery} types as opposed to \code{ActionQuery} (which encodes multiple entities using a single encoder) enabled better grounding and led to better performance in terms of F1-score (Table~\ref{table:query_comparison}).

% % \input{figures/complexity-ordering}

\begin{figure}[t]
\centering
    \includegraphics[width=\linewidth]{plots/com-ord-comparison.png}
    \caption{\textbf{NSG maintains consistent performance as task complexity and ordering difficulty increases.} F1-score of NSG vs. best-performing baseline for \etv tasks with varying complexity and ordering are shown.}
    \label{figure:complexity-ordering}
%\vspace{-10pt}
\end{figure}

\begin{figure}[t]
\centering
    \includegraphics[width=\linewidth]{plots/complexity-confusion_mat.pdf}
    \caption{[Left] F1-score of NSG vs. best-performing baseline for \etv tasks with varying complexity averaged over all splits (Appendix~\ref{appendix:analysis} shows performance with varying ordering). [Right] Confusion Matrix for NSG Queries on validation split (SQuery: \code{StateQuery}, RQuery: \code{RelationQuery}). See Appendix~\ref{appendix:analysis} for results on all splits.}
    \label{figure:complexity-confusion_mat}
    % \vspace{-0.1cm}
\end{figure}

\noindent \textbf{NSG shows consistent performance with increasing task difficulty.}
In Figure~\ref{figure:complexity-confusion_mat}, NSG's performance is minimally affected by increase in task difficulty characterized by number of sub-tasks (complexity) and ordering constraints (\S~\ref{section:evaluation}) unlike the best-performing baseline (VIOLIN-ResNet). 

\noindent \textbf{NSG is robust to segmentation window size} The effect of $k$ on NSG is minimal (Appendix~\ref{appendix:analysis}).


\noindent \textbf{NSG also enables task verification on real-world data.} NSG outperforms all competitive baselines on CTV significantly with F1-score (NSG: $\mathbf{76.3}$, CoCa: 70.9, VideoCLIP: 49.7, VIOLIN 34.7), demonstrating its causal and compositional reasoning capabilities in real-world applications (see Appendix~\ref{appendix:NSG_crosstask} for details).

\begin{table}[t]
\small
\centering
\begin{tabular}{lcccc}
\hline
\multirow{2}{*}{NSG} & \begin{tabular}[c]{@{}c@{}} Novel  \end{tabular} & \begin{tabular}[c]{@{}c@{}}Novel \end{tabular} & \begin{tabular}[c]{@{}c@{}}Novel\end{tabular} & \multirow{2}{*}{Abstract.} \\ & Tasks & Steps & Scenes & \\
\hline
\code{Action} & 78.2 & 45.6 & 70.6 & 75.5\\
\code{State+Relation} & 90.0 & 64.7 & 84.9 & 80.4\\
\hline
\end{tabular}
\vspace{2pt}
\caption{\code{(State}~+~\code{Relation)Query}~vs.~\code{ActionQuery}}
\label{table:query_comparison}
% \vspace{-5pt}
\end{table}  

\noindent \textbf{Limitations of NSG.} (1) It does not consider multiple simultaneous actions like ``picking an apple while closing the refrigerator door", (2) The assumption of equal-length video segments may be unsuitable for sub-tasks with a highly variable duration. We defer exploration of these limitations to future work, (3) Since NSG aligns the video with the entire task graph, it requires the full task execution video. Without this, alignment is partial, rendering NSG ineffective for online task verification.

%%%%%%%%%%%%%%%%%%%%%%%%%%%%%%%%%%%%%%%%%%%%%%%%%%%%%%%%%%%%%%%%%%%%%%%%%%%%%%%%%%%t
\vspace{-0.5cm}\section{Conclusion}
\label{sec:conclusion}
    %\section{Conclusion}
% Future work
% our finding 




%This study is limited by the number of participants and the lengths of the interactions. In addition, the displayed emotions were not the central focus of the conversations. Instead, they were used spontaneously whenever they matched the context leading to a varying experience for each of the participants.
%Still, it could prove a fantastic potential for using the display of emotion to deepen the connection between humans and robots. Nevertheless, it cannot be disregarded that humanizing robots also carries a significant risk. When the distinctions between humans and robots become blurred, it may soon become impossible for some people to tell the two apart. Therefore, the risks and benefits need to be carefully evaluated, and it might also be reasonable to establish internationally binding guidelines to differentiate robots and humans visually.
%For now, it has to be clearly stated, though, that any of the participants experienced no difficulty in telling that they were not, in fact, interacting with a human, indicating that the robot is not that human-like after all 
%Many participants reported a positive reaction to being smiled at. However, the display of other, potentially more negative emotions, like sadness, fear, or anger, has to be evaluated in further study.
%Future research could solidify our result with quantitative data, directly comparing the interaction with robots who do and do not show emotions.
%It can also be concluded that context awareness and the feeling of being understood by the robot were reported as a more significant benefit than the idea of the robot feeling or communicating joy. It encouraged the participants to talk more, and it could be further investigated whether this effect could be achieved through other visual or auditory cues that are not directly related to a human-typical expression of happiness.
%The study could show that the display of emotion by an Android was generally seen as positive and beneficial but also shed light on the fact that there many ethical questions that need to be investigated. 
    % % use section* for acknowledgment
% \vspace{-0.5cm}\subsubsection*{Acknowledgment}\vspace{-0.3cm}
\noindent\textbf{Acknowledgment}: This work was supported by a project grant from MeitY (No.4(16)/2019-ITEA), Govt. of India.
\bibliography{main}
\end{document}
