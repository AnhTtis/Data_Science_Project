\documentclass[
amsmath,
amssymb,
amsfonts,
aps,
%a4paper,
nofootinbib,
preprintnumbers,
prl,
reprint,
superscriptaddress
]{revtex4-2}

\usepackage{xcolor}
\definecolor{MyDarkBlue}{rgb}{0.15,0.25,0.45} 
\usepackage[
linktocpage=true,
hypertexnames=false,
colorlinks=true,
citecolor=MyDarkBlue,
linkcolor=MyDarkBlue,
urlcolor=MyDarkBlue,
pdfauthor={
    Leron Borsten,
    Branislav Jurco,
    Hyungrok Kim,
    Tommaso Macrelli,
    Christian Saemann,
    Martin Wolf},
pdftitle={Tree-Level Color--Kinematics Duality from Pure Spinor Actions},
pdfsubject={hep-th},
breaklinks=true
]{hyperref}
%\usepackage{mathptmx}

\usepackage{amsthm}
\usepackage{mathtools}
\usepackage{bbm}
\usepackage{bm}
\usepackage{enumerate}
\usepackage{mathrsfs}
\usepackage{multirow}
\usepackage{tikz}
\usetikzlibrary{matrix,cd,arrows,decorations.markings,decorations.pathmorphing,patterns}
%\usetikzlibrary{positioning}
%\usetikzlibrary{shadows.blur}
% Arrow heads
\tikzset{>=stealth}
\tikzcdset{arrow style=tikz}

\tikzset{
    aux/.style={dashed},
    dottedline/.style={dotted},
    gluon/.style={
        decorate, 
        draw=black,
        decoration={
            snake,
            post=lineto,
            post length=0pt,
            segment length=4,
            amplitude=0.9
        }
    },
    matter/.style={
    },
}
\usepackage{stmaryrd}
\usepackage{cleveref}
\usepackage{ifthen}
\usepackage{slashed}
\newcommand{\makecommand}[3]{%
    \foreach \i in #3 {%
        \expandafter\xdef\csname #1\i\endcsname{\noexpand#2{\unexpanded\expandafter{\i}}}%
    }%
}
\newcommand{\latinalphabet}{A,a,B,b,C,c,d,D,E,e,F,f,G,g,H,h,I,i,J,j,K,k,L,l,M,m,N,n,O,o,P,p,Q,q,R,r,S,s,T,t,U,u,V,v,W,w,X,x,Y,y,Z,z}
\makecommand{I}{\mathbbm}{\latinalphabet}
\makecommand{I}{\mathbbm}{{CP}}
\makecommand{bf}{\mathbf}{\latinalphabet}
\makecommand{bm}{\bm}{\latinalphabet}
\makecommand{ca}{\mathcal}{\latinalphabet}
\makecommand{fr}{\mathfrak}{\latinalphabet}
\makecommand{rm}{\mathrm}{\latinalphabet}
\makecommand{sf}{\mathsf}{\latinalphabet}
\makecommand{sf}{\mathsf}{{id,SU,SO,Spin,SL}}
\makecommand{fr}{\mathfrak}{{su,so,spin,sl}}
\makecommand{sc}{\mathscr}{\latinalphabet}
\makecommand{tt}{\mathtt}{\latinalphabet}
\newcommand\CN\caN % used in bibliography
\newcommand{\todo}[1]{{\color{red}{#1}}}
\newcommand{\argue}[1]{{\textcolor{blue}{Argue: #1}}}
\newcommand{\dpar}{\partial}
\newcommand{\wave}{\mathop\square}
\newcommand{\forw}{\uparrow}
\newcommand{\backw}{\downarrow}
\newcommand{\tr}{\mathrm{tr}} % trace
\newcommand{\dbl}[1]{\boldsymbol{#1}}
\newcommand{\parder}[2][]{%
    \ifthenelse{\equal{#1}{}}{%
        \frac{\partial}{\partial #2}%
    }{%
        \frac{\partial #1}{\partial #2}%
    }%
}
\newcommand{\delder}[2][]{%
    \ifthenelse{\equal{#1}{}}{%
        \frac{\delta}{\delta #2}%
    }{%
        \frac{\delta #1}{\delta #2}%
    }%
}
\newcommand{\inner}[2]{\langle#1,#2\rangle}
\newcommand{\eand}{{~~~\mbox{and}~~~}}

\newtheoremstyle{breaknodot}
{\topsep}{\topsep}%
{\itshape}{}%
{\bfseries}{}%
{0pt}{\thmname{#1}\thmnumber{ #2.}~\thmnote{ \normalfont(#3)}}%
\theoremstyle{breaknodot}
\newtheorem{definition}{Definition}
\newtheorem{theorem}{Theorem}
\newcommand{\sstack}[1]{{\large\substack{#1}}}
\newcommand{\BVbox}{\text{BV}^{\wave}}

\let\oldblacksquare\blacksquare
\newcommand{\BBox}{{\textcolor{gray}\oldblacksquare}} % make less noticeable

\begin{document}
    
    \preprint{DMUS--MP--23/05}
    \preprint{EMPG--23--04}
    \title{Tree-Level Color--Kinematics Duality from Pure Spinor Actions}
    %
    \author{Leron Borsten}
    \email[]{l.borsten@herts.ac.uk}
    %\homepage[]{Your web page}
    %\thanks{}
    %\altaffiliation{}
    \affiliation{Department of Physics, Astronomy and Mathematics, University of Hertfordshire\\
Hatfield, Hertfordshire, AL10 9AB, United Kingdom}
    %
    \author{Branislav Jur{\v c}o}
    \email[]{branislav.jurco@gmail.com}
    %\homepage[]{Your web page}
    %\thanks{}
    %\altaffiliation{}
    \affiliation{Charles University Prague\\ Faculty of Mathematics and Physics, Mathematical Institute\\ Prague 186 75, Czech Republic}
    %
    \author{Hyungrok Kim}
    \email[]{hk55@hw.ac.uk}
    %\homepage[]{Your web page}
    %\thanks{}
    %\altaffiliation{}
    \affiliation{Maxwell Institute for Mathematical Sciences\\ Department of Mathematics, Heriot--Watt University\\ Edinburgh EH14 4AS, United Kingdom}
    %
    \author{Tommaso Macrelli}
    \email[]{tmacrelli@phys.ethz.ch}
    %\homepage[]{Your web page}
    %\thanks{}
    %\altaffiliation{}
    \affiliation{Department of Physics, ETH Zürich\\ 8093 Zurich, Switzerland}
    %
    \author{Christian Saemann}
    \email[]{c.saemann@hw.ac.uk}
    %\homepage[]{Your web page}
    %\thanks{}
    %\altaffiliation{}
    \affiliation{Maxwell Institute for Mathematical Sciences\\ Department of Mathematics, Heriot--Watt University\\ Edinburgh EH14 4AS, United Kingdom}
    %
    \author{Martin Wolf}
    \email[]{m.wolf@surrey.ac.uk}
    %\homepage[]{Your web page}
    %\thanks{}
    %\altaffiliation{}
    \affiliation{Department of Mathematics, University of Surrey\\ Guildford GU2 7XH, United Kingdom}
    
    \date{\today}
    
    \begin{abstract}
        Pure spinor actions of various gauge theories are of Chern--Simons form, and they come with natural $\BVbox$-algebra structures that manifest tree-level color--kinematics duality. This has been observed before in~\cite{Ben-Shahar:2021doh} for the currents of 10D supersymmetric Yang--Mills theory. Here, we show that this observation can be extended to the scattering amplitudes as well as to theories with matter. In particular, we show that the BLG, ABJM and ABJ models possess tree-level color--kinematics duality without having to resort to explicit computations. The $\BVbox$-algebra structure explicitly contains the kinematic Lie algebra in the form of a derived bracket. However, regularization issues obstruct the seemingly implied  lift to the loop level, and we give reasons to believe that this is a general feature of manifestly CK-dual actions close to Yang--Mills theory. We also explain the link of our ordinary gauge--matter CK-duality to the 3-Lie algebra form of CK-duality previously discussed in the literature.
    \end{abstract}
    
    \maketitle
    
    \section{Introduction}\label{sec:intro}
    
    Color--kinematics (CK) duality~\cite{Bern:2008qj,Bern:2010ue,Bern:2010yg} is a remarkable hidden feature of certain perturbative quantum field theories that puts the kinematic spacetime structure on the same footing as internal gauge or flavor symmetries. This idea has manifold implications and applications, as reviewed in~\cite{Carrasco:2015iwa,Borsten:2020bgv, Bern:2019prr,Adamo:2022dcm,Bern:2022wqg}.
    In particular, it is key to the famous double copy prescription~\cite{Bern:2008qj,Bern:2010ue,Bern:2010yg}, which allows for the construction of gravitational scattering amplitudes from a particular parameterization of the scattering amplitudes of supersymmetric Yang--Mills (SYM) theory.
    
    CK-duality and the double copy were originally discovered using on-shell amplitudes technology. Having been shown the way by scattering amplitudes, it is natural to ask if one can return to the standard Lagrangian field theory starting point, possibly shedding further light on CK-duality~\cite{Bern:2010yg, Tolotti:2013caa, Borsten:2020zgj,Borsten:2021hua,Borsten:2021rmh,Borsten:2022vtg,Ben-Shahar:2022ixa}. This is all the more natural from the homotopy algebraic perspective, which puts Lagrangians and on-shell amplitudes on the same footing~\cite{Jurco:2018sby,Macrelli:2019afx,Arvanitakis:2019ald,Jurco:2019yfd}. Indeed,  under the right conditions the kinematic Lie algebra and CK-duality are  symmetries of the Lagrangian, just as gauge invariance  is manifest in the Yang--Mills action~\cite{Borsten:2022vtg}.
        
    The main ingredient in our discussion is the notion of $\BVbox$-algebra introduced in~\cite{Reiterer:2019dys} in the context of first-order Yang--Mills theory. In general, any field theory action with an underlying $\BVbox$-algebra has a kinematic Lie algebra, as explained in~\cite{Borsten:2022vtg}, see also~\cite{Reiterer:2019dys}. For perturbative computations with suitable propagators, a kinematic Lie algebra then further implies CK-duality in the conventional sense.
    
    Identifying a theory's underlying $\BVbox$-algebra (if one exists) is non-trivial. A crucial observation here is that the archetypal theory with $\BVbox$-algebra is Chern--Simons (CS) theory, cf.~\cite{Borsten:2022vtg,Ben-Shahar:2021zww}. To establish CK-duality, we thus naturally consider field theories that allow for CS-like reformulations. There are two evident families of candidates to our knowledge\footnote{We do not have anything to say about cubic harmonic superspace actions~\cite{Galperin:2001uw,Schwab:2013hf}, which also take a Chern--Simons form, cf.~\cite{Buchbinder:2008vi}.}: holomorphic CS theory on twistor space (reviewed in~\cite{Wolf:2010av,Adamo:2013cra}) and pure spinor CS actions (reviewed in~\cite{Bedoya:2009np,Cederwall:2010wf,Hoogeveen:2010aa,Cederwall:2013vba,Berkovits:2017ldz,Eager:2021wpi,Cederwall:2022fwu}). We studied the former in~\cite{Borsten:2022vtg} and found an ordinary and a generalized $\BVbox$-algebra for self-dual and full SYM theory, respectively.
    
    Here, we focus on pure spinor actions.\footnote{The pure spinor formalism has been previously applied to the study of CK-duality of SYM in e.g.~\cite{Mafra:2011kj,Mafra:2014gja,Mafra:2015mja,Bridges:2019siz}.} For SYM theory, the pure spinor action is already in CS form, and the propagator is suitable for inducing a conventional form of CK-duality as first observed in~\cite{Ben-Shahar:2021doh}, which shows that the Berends--Giele currents of SYM theory come with a kinematic Lie algebra. However, a problem arises when turning these currents into CK-dual kinematic numerators of scattering amplitudes:  the \emph{tree level} scattering amplitudes require an integral over pure spinor space that generically diverges for individual diagrams even at the  tree level. One may skirt the divergences by regularizing the $\sfb$-operator via heat kernels~\cite{Berkovits:2006vi}, but the regularized $\sfb$-operator fails to be of second order as required for a $\BVbox$-algebra and hence CK-duality.\footnote{While~\cite{Grassi:2009fe} claims a regulator that would be compatible with CK-duality, it has been criticized by~\cite{Aisaka:2009yp}.}
    
    We circumvent this issue by using the $Y$-formalism~\cite{Matone:2002ft,Oda:2005sd,Oda:2007ak} to demonstrate tree-level CK-duality of SYM theory directly from the action, simplifying the available proof and exposing more of the underlying algebraic structure. We then identify the correct notion of $\BVbox$-module that establishes CK-duality for gauge--matter theories, cf.~\cite{Johansson:2014zca}. As examples, we show that the pure spinor actions for the Bagger--Lambert--Gustavsson (BLG), Aharony--Bergman--Jafferis--Maldacena (ABJM), and Aharony--Bergman--Jafferis (ABJ) models of~\cite{Cederwall:2008vd,Cederwall:2008xu} (cf.~\cite{Cederwall:2009ay}) imply suitable $\BVbox$-algebras to establish tree-level CK-duality to all orders. This completes partial results on CK-duality of Chern--Simons--matter (CSM) theories in the literature~\cite{Huang:2012wr,Huang:2013kca,Sivaramakrishnan:2014bpa,Ben-Shahar:2021zww}. The CK-duality we find for these theories uses cubic vertices; we explain how this implies a 3-Lie algebraic CK-duality using quartic vertices~\cite{Huang:2012wr,Huang:2013kca,Sivaramakrishnan:2014bpa}.
    
    \section{Algebras underlying CK-duality}
    
    When approaching CK-duality from the perspective of the action, it is convenient to rewrite the latter into an equivalent form with exclusively cubic interaction terms~\cite{Bern:2010yg,Tolotti:2013caa,Anastasiou:2018rdx,Borsten:2021hua}. One achieves this by introducing auxiliary fields, blowing up non-cubic interaction vertices into cubic ones. Applying the Batalin--Vilkovisky (BV) formalism to the resulting field theory produces a graded vector space graded by ghost number and a BV-differential that encodes gauge transformations and equations of motion. The resulting structure dualizes (cf.\ e.g.~\cite{Jurco:2018sby,Borsten:2021hua}) to a differential graded Lie algebra whose differential encodes the linearized gauge transformations and the linearized action, and whose Lie bracket encodes the non-linear corrections to both. 
    
    This differential graded Lie algebra factorizes into the (ungraded) gauge Lie algebra and a differential graded-commutative algebra (dgca)~\cite{Zeitlin:2008cc,Borsten:2021hua}. This factorization is called ``color-stripping.''\footnote{Here, we restrict ourselves to the case of gauge theories with color Lie algebras for simplicity, although this formalism accommodates various generalizations~\cite{Borsten:2022aa}.} CK-duality can then be regarded as a refinement of the dgca.\footnote{In our previous work~\cite{Borsten:2020zgj,Borsten:2021hua,Borsten:2021rmh}, this appears as compatibility with a twisted tensor product.} Here, we focus on the notion of $\BVbox$-algebra~\cite{Reiterer:2019dys,Borsten:2022vtg}, see also~\cite{Akman:1995tm,Borsten:2022ouu,Borsten:2022aa}. A $\BVbox$-algebra carries, besides the differential $Q$ and the graded-commutative product $\sfm(-,-)$ of the dgca, an operator $\sfb$ of degree $-1$ that squares to zero and satisfies $Q\sfb+\sfb Q=\wave$ (where $\wave$ is the Minkowski space d'Alembertian)\footnote{More general cases may have $Q\sfb+\sfb Q=\BBox\ne\wave$, but this Letter does not need this generalization.} with $\frac{\sfb}{\wave}$ the propagator in Siegel gauge. Moreover, $\sfb$ is a second-order differential operator (in the sense of~\cite{koszul1985crochet,Akman:1995tm}) with respect to the product, which implies that the \emph{derived bracket}
    \begin{equation}\label{eq:derived_bracket}
        \{\phi,\psi\}=\sfb\sfm(\phi,\psi)-\sfm(\sfb\phi,\psi)-(-1)^{|\phi|}\sfm(\phi,\sfb\psi)
    \end{equation}
    for $\phi,\psi$ color-stripped fields of ghost numbers $|\phi|,|\psi|$ is a (grade-shifted) Lie bracket. This is precisely the kinematic Lie algebra that combines with the gauge Lie algebra in cubic interaction vertices. A kinematic Lie algebra then implies conventional CK-duality as long as the resulting numerators do not diverge.
    
    In the case of plain CS theory, for example, the differential graded-commutative algebra consists of differential forms on spacetime; $Q=\rmd$ is the exterior derivative, $\sfm(-,-)=\wedge$ is the wedge product, and $\sfb=-\rmd^\dagger$ is the Hodge codifferential. The resulting kinematic Lie algebra is the Schouten--Nijenhuis algebra of multivector fields~\cite{Borsten:2022vtg}, and the ``scattering amplitudes'' of harmonic one-forms exhibit CK-duality, as first noted in~\cite{Ben-Shahar:2021zww}. 
    
    
    \section{SYM theory with pure spinors}
    
    We first explain how the pure spinor approach for SYM theory manifests CK-duality, building on the observations for currents of~\cite{Ben-Shahar:2021doh}. Our later discussion of CSM theories will use very similar arguments. 
    
    The (non-minimal~\cite{Berkovits:2005bt}) pure spinor space $M_{\text{10D}\,\caN=1}$ of 10D SYM theory enlarges the usual superspace $\IR^{10|16}$ to a supermanifold coordinatized by $(x^M,\theta^A,\lambda^A,\bar\lambda_A,\rmd\bar\lambda_A)$, where indices $A$ belong to the $\mathbf{16}$ or $\overline{\mathbf{16}}$ of $\sfSpin(1,9)$, and only $(\theta^A,\rmd\bar\lambda_A)$ are anticommuting. The coordinates $(\lambda^A,\bar\lambda_A,\rmd\bar\lambda_A)$ carry ghost numbers $(1,-1,0)$ respectively and obey the pure spinor constraints
    \begin{equation}\label{eq:pure_constraint_10D}
        \lambda^A\gamma^M_{AB}\lambda^B=\bar\lambda_A\gamma^{M\,AB}\bar\lambda_B=\bar\lambda_A\gamma^{M\,AB}\rmd\bar\lambda_B=0.
    \end{equation}
    The covariant superderivatives $D_A$ on $\IR^{10|16}$ satisfy the algebra 
    \begin{equation}
        D_AD_B+D_BD_A=-2\gamma^M_{AB}\parder{x^M},
    \end{equation}
    and~\eqref{eq:pure_constraint_10D} implies that the operator 
    \begin{equation}\label{eq:defOfQ}
        Q=\lambda^AD_A+\rmd\bar\lambda_A\parder{\bar\lambda_A}
    \end{equation}
    squares to zero. The natural volume form $\Omega_{M_{\text{10D}\,\caN=1}}$ given in~\cite{Berkovits:2005bt} permits an action principle for a scalar superfield $\Psi$ on $M_{\text{10D}\,\caN=1}$ of ghost number $1$ that takes values in a gauge metric Lie algebra $(\frg,\inner{-}{-}_\frg)$:
    \begin{equation}
        S_{\text{10D}\,\caN=1}=\int\Omega_{M_{\text{10D}\,\caN=1}}\inner{\Psi}{Q\Psi+\tfrac13[\Psi,\Psi]}_\frg.
    \end{equation}
    One can compute perturbative scattering amplitudes using the propagator $\frac{\sfb}{\wave}$ in the Siegel gauge $\sfb\Psi=0$, where the $\sfb$-operator carries ghost number $-1$ and satisfies
    \begin{equation}\label{eq:b-properties}
        Q\sfb+\sfb Q=\wave\eand \sfb^2=0,
    \end{equation}
    cf.~\cite{Bjornsson:2010wm,Bjornsson:2010wu}. The $\sfb$-operator is not unique (cf.~\cite{Berkovits:2013pla,Jusinskas:2013sha,Cederwall:2022qfn}); for tree-level CK-duality it is convenient to work with the non-Lorentz-invariant $\sfb$-operator of the $Y$-formalism~\cite{Matone:2002ft,Oda:2005sd,Oda:2007ak}, which imposes a form of axial gauge along a reference pure spinor $v$ with $v_A \gamma^{M\,AB}v_B=0$ such that
    \begin{equation}\label{eq:def_b_10}
        \sfb=-\frac{v_A\gamma^{M\,AB}D_B}{2\lambda^A v_A}\parder{x^M},
    \end{equation}
    which satisfies~\eqref{eq:b-properties}. Table~\ref{tab:coordinatesOperators} summarizes the properties of all coordinates and operators.
    \begin{table}[h]
        \begin{center}
            \begin{tabular}{c|cccc}
                & \multirow{2}{*}{$\sfSpin(1,9)$} & mass & Grassmann & ghost
                \\[-3pt]
                & & dimension & degree & number
                \\
                \hline
                $x$ & $\mathbf{10}$ & $-1\phantom+$ & $0$ & $\phantom{+}0\phantom+$
                \\
                $\theta$ & $\mathbf{16}$ & $-\frac12\phantom+$ & $1$ & $\phantom{+}0\phantom+$
                \\
                $\lambda$ & $\mathbf{16}$& $-\frac12\phantom+$ & $0$ & $\phantom{+}1\phantom+$
                \\
                $\bar\lambda$ & $\overline{\mathbf{16}}$ & $\phantom{+}\frac12\phantom+$ & $0$ & $-1\phantom+$
                \\
                $\rmd\bar\lambda$ & $\overline{\mathbf{16}}$ & $\phantom{+}\frac12\phantom+$ & $1$ & $\phantom{+}0\phantom+$
                \\[1pt]
                \hline
                $D$ & $\overline{\mathbf{16}}$ & $\phantom{+}\frac12\phantom+$ & $1$ & $\phantom{+}0\phantom+$
                \\
                $Q$ & $\mathbf{1}$ & $\phantom{+}0\phantom+$ & $1$ & $\phantom{+}1\phantom+$
                \\
                $\sfb$ & $\mathbf{1}$ & $\phantom{+}2\phantom+$ & $1$ & $-1\phantom+$
                \\
                \hline
                $\Psi$ & $\mathbf{1}$ & $\phantom{+}0\phantom+$ & $1$ & $\phantom{+}1\phantom+$
            \end{tabular}
            \caption{Properties of 10D coordinates and operators.}
            \label{tab:coordinatesOperators}
        \end{center}
    \end{table}
    
    The $\sfb$-operator~\eqref{eq:def_b_10} induces a $\BVbox$-algebra structure on the color-stripped dgca of this theory, as explained earlier; the currents of this theory feature a kinematic Lie algebra similar to that for the Lorentz-invariant $\sfb$-operator in~\cite{Ben-Shahar:2021doh}. Converting these currents to individual CK-dual tree-level numerators, however, involves an integral over $(\lambda,\bar\lambda)$ that suffers from two kinds of divergences. First, due to the non-compactness of the pure spinor space, there are infrared (large $\lambda,\bar\lambda$) divergences. These are mostly harmless and can be regulated away in a CK-dual fashion, cf.~\cite{Cederwall:2022fwu}. Second, there are ultraviolet divergences: the integrands will contain singularities of the form $\frac{1}{(\lambda^A v_A)^n}$ arising from the propagator $\frac{\sfb}{\wave}$ as well as Siegel gauge $\sfb\Psi=0$. While these singularities cancel in a total scattering amplitude, individual numerators may diverge. We now explain how to extract finite CK-dual numerators for individual Feynman diagrams.
    
    The $Y$-formalism Siegel gauge physical states (unintegrated vertex operators) are obtained by starting with the non-singular representatives $\lambda^A\lambda^A\caA_{AB}$ of the cohomology class of the antifields and applying $\sfb$ to them~\cite{Aisaka:2009yp}. The singularities of the external states are thus of the form $\frac{1}{(\lambda^A v_A)^n}$.
    
    The kinematic Jacobi identities hold order by order in $\frac{1}{\lambda^A v_A}$. In the total scattering amplitude, divergent terms may combine into $Q$-exact terms and thus become discardable. However, since the operator $Q$ does not affect the degree of singularity near $\lambda^A v_A=0$, any such discarding may be restricted to singular terms only.\footnote{In contrast, for the Lorentz-invariant non-minimal $\sfb$~\cite{Berkovits:2005bt}, the relevant ultraviolet divergence occurs near the tip of the pure spinor cone $\lambda\bar\lambda=0$. The numerators are, in general, defined up to $Q$-exact terms, and $Q$ does change the degree of singularity near $\lambda\bar\lambda=0$. So, cancellation of singularities in the total scattering amplitude may involve singular terms combining with non-singular terms to form $Q$-exact terms and being discarded; as this discards some non-singular parts of the numerators, this may ruin kinematic Jacobi identities of the minimally subtracted numerators.} Therefore, we can safely truncate away the singular terms in the numerators without losing the kinematic Jacobi identities, similar to minimal subtraction in dimensional regularization; the minimally subtracted numerators then provide a CK-dual parameterization of the scattering amplitudes with finite numerators. 
    
    Therefore, we have all-order tree-level CK-duality for 10D SYM theory. By dimensional reduction and the usual embedding of non-maximally SYM tree diagrams into maximally ones~(cf.~\cite{Chiodaroli:2013upa}), this establishes tree-level CK-duality for all pure Yang--Mills theories with arbitrary amounts of supersymmetry in any dimension.
    
    \section{General gauge--matter theories}
    
    A CK-dual gauge theory with matter possesses additional algebraic structure as follows. A \emph{metric Lie module} $(\frg,\inner{-}{-}_\frg,V,\inner{-}{-}_V)$ consists of a metric Lie algebra $(\frg,\inner{-}{-}_\frg)$ together with
    a real orthogonal $\frg$-representation $(V,\inner{-}{-}_V)$. Any metric Lie module has a product $\wedge\colon V^2\rightarrow\frg$ defined by
    \begin{equation}
        \inner{X}{u\wedge v}_\frg=\inner{u}{X\cdot v}_V
    \end{equation}
    for all $u,v\in V$ and $X\in\frg$, that is anti-symmetric and $\frg$-equivariant.
    \begin{proof}
        For all $u,v\in V$ and $X,Y\in\frg$, anti-symmetry follows from $\inner{X}{u\wedge v}_\frg=\inner{u}{X\cdot v}_V=-\inner{X\cdot u}{v}_V=-\inner{v}{X\cdot u}_V=-\inner{X}{v\wedge u}_\frg$ and equivariance is due to $\inner{Y}{[X,u\wedge v]}_\frg=-\inner{[X,Y]}{u\wedge v}_\frg=-\inner{u}{[X,Y]\cdot v}_V=-\inner{u}{X\cdot(Y\cdot v)}_V+\inner{u}{Y\cdot(X\cdot v)}_V=\inner{X\cdot u}{Y\cdot v}_V-\inner{Y\cdot u}{X\cdot v}_V=\inner{X\cdot u}{Y\cdot v}_V-\inner{X\cdot v}{Y\cdot u}_V=\inner{Y}{(X\cdot u)\wedge v}_\frg-\inner{Y}{(X\cdot v)\wedge u}_\frg=\inner{Y}{(X\cdot u)\wedge v}_\frg+\inner{Y}{u\wedge(X\cdot v)}_\frg$.
    \end{proof}
    The $\frg$-equivariance
    \begin{equation}
        \inner{X\cdot(u\wedge v)}{Y}_\frg=\inner{(X\cdot u)\wedge v}{Y}_\frg+\inner{u\wedge(X\cdot v)}{Y}_\frg
    \end{equation}
    can be diagrammatically expressed as
    \newcommand\nodearray{
        \matrix (m) [
        matrix of nodes,
        ampersand replacement=\&,
        column sep=0.1cm,
        row sep=0.1cm
        ]{
            $X$ \& {} \& {} \& {} \& $u$
            \\
            {} \& {} \& {} \& {} \& {}
            \\
            {} \& {} \& {} \& {} \& {}
            \\
            {} \& {} \& {} \& {} \& {}
            \\
            $Y$ \& {}\& {} \& {} \& $v$
            \\
        };
    }
    \begin{equation}\label{eq:Jacobi-mixed}
        \begin{tikzpicture}[
            scale=1,
            every node/.style={scale=1},
            baseline={([yshift=-.5ex]current bounding box.center)}
            ]
            \nodearray
            \draw [gluon] (m-1-1) -- (m-3-2.center);
            \draw [gluon] (m-5-1) -- (m-3-2.center);
            \draw [gluon] (m-3-2.center) -- (m-3-4.center);
            \draw [matter] (m-1-5) -- (m-3-4.center);
            \draw [matter] (m-5-5) -- (m-3-4.center);
        \end{tikzpicture}
        =
        \begin{tikzpicture}[
            scale=1,
            every node/.style={scale=1},
            baseline={([yshift=-.5ex]current bounding box.center)}
            ]
            \nodearray
            \draw [gluon] (m-1-1) -- (m-2-3.center);
            \draw [gluon] (m-5-1) -- (m-4-3.center);
            \draw [matter] (m-2-3.center) -- (m-4-3.center);
            \draw [matter] (m-1-5) -- (m-2-3.center);
            \draw [matter] (m-5-5) -- (m-4-3.center);
        \end{tikzpicture}
        +
        \begin{tikzpicture}[
            scale=1,
            every node/.style={scale=1},
            baseline={([yshift=-.5ex]current bounding box.center)}
            ]
            \nodearray
            \draw [gluon] (m-1-1) -- (m-3-4.center);
            \draw [gluon] (m-5-1) -- (m-3-2.center);
            \draw [matter] (m-3-2.center) -- (m-3-4.center);
            \draw [matter] (m-1-5) -- (m-3-2.center);
            \draw [matter] (m-5-5) -- (m-3-4.center);
        \end{tikzpicture}.
    \end{equation}
    The metrics $\langle-,-\rangle_\frg$ and $\langle-,-\rangle_V$ define a 3-bracket $\llbracket-,-,-\rrbracket\colon V^3\rightarrow V$ by
    \begin{equation}\label{eq:3-bracket-antisymmetric-defn}
        \inner{s}{\llbracket u,v,w\rrbracket}_V=\inner{s\wedge u}{v\wedge w}_\frg
    \end{equation}
    for all $s,u,v,w\in V$. If $\llbracket-,-,-\rrbracket$ is totally anti-symmetric, $(V,\llbracket-,-,-\rrbracket)$ is a 3-Lie algebra in the sense of~\cite{Filippov:1985aa}. We call the above data a \emph{3-Lie algebra structure}.\footnote{Operadically, this notion is captured by a two-sorted cyclic quadratic operad $\mathsf{cycLieMod}$, which can be thought of as the cyclic version of the two-sorted quadratic operad $\mathsf{LieMod}$ of Lie algebra modules.}
    
    Color-stripping a metric Lie module from the field content of a theory requires an appropriate replacement for the dgca. To this end, we mimic the above construction for commutative rather than Lie algebras, obtaining \emph{metric Com modules}. These consist of a (possibly non-unital) metric commutative associative algebra $(\frC,\inner{-}{-}_\frC)$ (i.e.~$\inner{X}{Y}_\frC=\inner{Y}{X}_\frC$ and $\inner{XY}{Z}_\frC=\inner{X}{YZ}_\frC$ for all $X,Y,Z\in\frC$) with a symplectic $\frC$-module $(V,\inner{-}{-}_V)$, i.e.~a $\frC$-module $V$ with an $\frC$-invariant symplectic metric $\inner{-}{-}_V$. Then the product $\bullet\colon V^2\rightarrow\frC$ defined by
    \begin{equation}\label{eq:bulletProduct}
        \inner{X}{u\bullet v}_\frC=\inner{u}{X\cdot v}_V
    \end{equation}
    for all $u,v\in V$ and $X\in\frC$ is commutative and $\frC$-bilinear.
    \begin{proof}
        For all $X,Y\in\frC$ and $u,v\in V$, commutativity follows from $\inner{X}{u\bullet v}_\frC=\inner{u}{X\cdot v}_V=-\inner{X\cdot u}{v}_V=\inner{v}{X\cdot u}_V=\inner{X}{v\bullet u}_\frC$ and bilinearity is due to $\inner{Y}{X\cdot(u\bullet v)}_\frC=-\inner{XY}{u\bullet v}_\frC=-\inner{u}{X\cdot(Y\cdot v))}_V=\inner{X\cdot u}{Y\cdot v}_V=\inner{Y}{(X\cdot u)\bullet v}_\frC$.
    \end{proof}
    \noindent
    Analogously to the 3-bracket $\llbracket-,-,-\rrbracket$, we define here a 3-bracket $\llparenthesis-,-,-\rrparenthesis\colon V^3\to V$ by
    \begin{equation}\label{eq:3-bracket-symmetric-defn}
        \inner{s}{\llparenthesis u,v,w\rrparenthesis}_V=\inner{s\bullet u}{v\bullet w}_\frC
    \end{equation}
    for all $s,u,v,w\in V$.
    
    The preceding constructions generalize to the differential graded setting by inserting appropriate sign factors and requiring the evident compatibility of the differential with the binary products. A theory with gauge structure governed by a 3-Lie algebra and cubic interaction vertices is then described by a dg-metric Lie module, which factors into a gauge metric Lie module and a dg-metric Com module. This generalizes color-stripping to color--flavor-stripping. 
    
    For CK-duality with matter, we need, besides the usual $\BVbox$-algebra $\frB$ for the gauge sector, an additional dg-module $(V,Q_V)$ over $\frB$ (in the sense of dg-commutative algebras) endowed with a degree $-1$ map $\sfb_V\colon V\rightarrow V$ that squares to zero, is a second-order differential operator with respect to the module action of $\frB$ on $V$, and such that $Q_V\sfb_V+\sfb_VQ_V=\wave$. Just as a $\BVbox$-algebra $\frB$ comes with an associated kinematic Lie algebra in the form of the derived bracket~\eqref{eq:derived_bracket}, a $\BVbox$-module $(V,Q_V,\sfb_V)$ comes with an associated kinematic Lie module. By a trivial extension of the arguments in~\cite{Borsten:2022vtg}, if a color-stripped theory with matter admits the structure of a $\BVbox$-module, it automatically enjoys gauge--matter CK-duality as long as the resulting numerators do not diverge.
    
    \section{BLG model with pure spinors}
    
    Following~\cite{Cederwall:2008xu}, we consider the dimensional reduction of the pure spinor superspace $M_{\text{10D}\,\caN=1}$ to 3D, obtaining the pure spinor superspace $M_{\text{3D}\,\caN=8}$ coordinatized by $(x^\mu,\theta^{\alpha i},\lambda^{\alpha i},\bar\lambda_{\alpha i},\rmd\bar\lambda_{\alpha i})$, where $\mu,\nu,\ldots=0,1,2$. In the reduction, the $\mathbf{16}$ of $\sfSpin(1,9)$ becomes the $\mathbf{2}\otimes\mathbf{8}$ of $\sfSpin(1,2)\times\sfSpin(7)$, so the spinor index $A$ splits into $(\alpha,i)$ with $\alpha,\beta=1,2$ (raised and lowered using $\varepsilon_{\alpha\beta}$) and $i,j=1,\ldots,8$ (raised and lowered using $\delta_{ij}$). As is well-known, the R-symmetry enlarges from $\sfSpin(7)$ to $\sfSpin(8)$. We use indices $m,n=1,\ldots,8$ for the vector representations $\mathbf{8_v}$ of $\sfSpin(8)$.
    
    The 10D pure spinor constraints~\eqref{eq:pure_constraint_10D} are relaxed in 3D to 
    \begin{equation}\label{eq:pure_constraint_3D}
        \lambda^{\alpha i}\gamma_{\alpha\beta}^\mu\lambda_i^\beta=\bar\lambda^{\alpha i}\gamma_{\alpha\beta}^\mu\bar\lambda^\beta_i=\bar\lambda^{\alpha i}\gamma_{\alpha\beta}^\mu\rmd\bar\lambda^\beta_i=0.
    \end{equation}
    The pure spinor superspace $M_{\text{3D}\,\caN=8}$ has a dimensionless volume form $\Omega_{\text{3D}\,\caN=8}$~\cite{Cederwall:2008xu}.
    
    The color--flavor structure of the BLG model is a metric Lie module $(\frg,\langle-,-\rangle_\frg,V,\langle-,-\rangle_V)$ with $\frg=\frsu(2)\oplus\frsu(2)$ and $V=(\mathbf2,\mathbf1)\otimes(\mathbf1,\mathbf2)$. The metric $\langle-,-\rangle_\frg$ on $\frg$ has signature $(3,3)$, while $\langle-,-\rangle_V$ is positive-definite. The resulting 3-bracket~\eqref{eq:3-bracket-antisymmetric-defn} is totally anti-symmetric.\footnote{Up to direct sums and isomorphisms, this is the only finite-dimensional metric Lie module permitting $\caN=8$ supersymmetry~\cite{Papadopoulos:2008sk,Gauntlett:2008uf,Hosomichi:2008jb,Schnabl:2008wj}.}
    
    The gauge multiplet of the BLG model belongs to a $\frg$-valued superfield $\Psi$ on $M_{\text{3D}\,\caN=8}$ of mass dimension~$0$ and ghost number $1$. The matter superfield $\Phi$ takes values in the tensor product of $V$ with the $\mathbf{8_v}$ of $\sfSpin(8)$ and has mass dimension $\tfrac12$ and ghost number $1$. We quotient the matter field space by the relation
    \begin{equation}
        \Phi^m\sim\Phi^m+\lambda^{\alpha i}\gamma^m_{\alpha\beta}\rho^\beta_i
    \end{equation}
    for arbitrary $\rho^\alpha_i$.\footnote{That is, the matter field $\Phi$ is a section of a certain coherent sheaf over pure spinor space.} The pure spinor action for the BLG model is~\cite{Cederwall:2008xu}
    \begin{equation}\label{eq:BLGAction}
        \begin{aligned}
            S_{\text{3D}\,\caN=8}&=\int\Omega_{\text{3D}\,\caN=8}\Big(\inner{\Psi}{Q\Psi+\tfrac13[\Psi,\Psi]}_\frg
            \\
            &\kern1.5cm+g_{mn}\inner{\Phi^m}{Q_V\Phi^n+\Psi\Phi^n}_V\Big)
        \end{aligned}
    \end{equation}
    with $g_{mn}=\lambda^{\alpha i}\gamma_{mn}{}_\alpha{}^\beta\lambda_{i\beta}$ and $Q_V=Q$.
    
    We can now color--flavor-strip the dg-metric Lie module underlying the action~\eqref{eq:BLGAction}, as explained above, to obtain a dg-metric Com module $V$ over $\frC$. The resulting product~\eqref{eq:bulletProduct} satisfies 
    \begin{equation}
        \inner{s\bullet u}{v\bullet w}_\frC=\inner{s\bullet v}{u\bullet w}_\frC
    \end{equation}
    for all $s,u,v,w\in V$, yielding a totally symmetric 3-bracket~\eqref{eq:3-bracket-symmetric-defn}. 
    
    A $Y$-formalism $\sfb$-operator also works for 3D: choosing a reference pure spinor $v$ with $v_{\alpha i} \gamma^{\mu\,\alpha\beta}\delta^{ij}v_{\beta j}=0$, we define
    \begin{equation}\label{eq:def_b_3}
        \sfb=\sfb_V=-\frac{v_{\alpha i}\gamma^{\mu\,\alpha\beta}\delta^{ij}D_{\beta j}}{2\lambda^{\alpha i} v_{\alpha i}}\parder{x^\mu},
    \end{equation}
    which again satisfies~\eqref{eq:b-properties} and which is also second-order with respect to the module action on the dg-metric Com module. Table~\ref{tab:coordinatesOperators2} summarizes the properties of all objects.
    \begin{table}[h]
        \begin{center}
            \begin{tabular}{c|cccc}
                & \multirow{2}{*}{$\sfSL(2,\IR)\times\sfSpin(8)$} & mass & Grassmann & ghost
                \\[-5pt]
                & & dimension & degree & number
                \\
                \hline
                $x$ & $\mathbf{(3,8_v)}$ & $-1\phantom+$ & $0$ & $\phantom{+}0\phantom+$
                \\
                $\theta$ & $\mathbf{(2,8_s)}$ & $-\frac12\phantom+$ & $1$ & $\phantom+0\phantom+$
                \\
                $\lambda$ & $\mathbf{(2,8_s)}$ & $-\frac12\phantom+$ & $0$ & $\phantom+1\phantom+$
                \\
                $\bar\lambda$ & $\mathbf{(2,8_c)}$ & $\phantom{+}\frac12\phantom+$ & $0$ & $-1\phantom+$
                \\
                $\rmd\bar\lambda$ & $\mathbf{(2,8_c)}$ & $\phantom{+}\frac12\phantom+$ & $1$ & $\phantom{+}0\phantom+$
                %\\
                %$\vartheta$ & $\mathbf{(0,8_v)}$ & $-\frac12\phantom+$ & $1$ & $\phantom+1\phantom+$
                \\[1pt]
                \hline
                $D$ & $\mathbf{(2,8_s)}$ & $\phantom{+}\frac12\phantom+$ & $1$ & $\phantom{+}0\phantom+$
                \\
                $Q$ & $\mathbf{(1,1)}$ & $\phantom{+}0\phantom+$ & $1$ & $\phantom{+}1\phantom+$\\
                $\sfb$ & $\mathbf{(1,1)}$ & $\phantom{+}2\phantom+$ & $1$ & $-1\phantom+$
                \\
                \hline
                $\Psi$ & $\mathbf{(1,1)}$ & $\phantom{+}0\phantom+$ & $1$ & $\phantom+1\phantom+$
                \\
                $\Phi$ & $\mathbf{(1,8_v)}$ & $\phantom{+}\frac12\phantom+$ & $0$ & $\phantom{+}0\phantom+$
            \end{tabular}
            \caption{Properties of 3D coordinates and operators.}
            \label{tab:coordinatesOperators2}
        \end{center}
    \end{table}
    
    As in the case of SYM theory, the operator~\eqref{eq:def_b_3} induces a $\BVbox$-algebra structure for the gauge part, but this extends to a $\BVbox$-module structure on the full dg-metric Com module. This establishes CK-duality for the currents of the BLG model that is based on \emph{cubic} vertices, not the quartic vertices anticipated by 3-Lie algebras.
    
    To turn currents into scattering amplitudes, one integrates expressions with singularities of the form $\frac{1}{\lambda^{\alpha i}v_{\alpha i}}$ over $(\lambda,\bar\lambda)$ space, fully analogously to SYM theory. Our previous arguments regarding a minimal subtraction of singularities still hold, and we obtain all-order tree-level CK-duality for the BLG model.  
    
    
    \section{ABJM/ABJ models with pure spinors}
    
    Some CSM theories with $\caN<8$ supersymmetry may also admit a cubic pure spinor action, in which case they also exhibit tree-level CK-duality using the $Y$-formalism. As examples, consider the $\caN=6$ ABJM~\cite{Aharony:2008ug} and ABJ~\cite{Aharony:2008gk} models in the pure spinor formalism of~\cite{Cederwall:2008xu}. The pure spinor superspace $M_{\text{3D}\,\caN=6}$ for 3D $\caN=6$ theories results from truncating the $\sfSpin(8)$ R-symmetry to $\sfSpin(6)$, so $M_{\text{3D}\,\caN=6}\subset M_{\text{3D}\,\caN=8}$. The index notation remains the same as for the BLG model except that $k,l,m,n,p=1,\ldots,4$ are indices for the $\mathbf{4}$ of $\sfSpin(6)\cong\sfSU(4)$. After the truncation, the pure spinor is of the form $\lambda^{\alpha mn}=-\lambda^{\alpha nm}$. This truncation does not affect the properties of $Q$ and $\sfb$, and the volume form $\Omega_{\text{3D}\,\caN=6}$ remains dimensionless.
    
    The gauge algebra $\frg$ remains a metric Lie algebra, but the representation $V$ is a \emph{complex} $\frg$-representation since the matter fields are in the complex representation $\mathbf{4}$ of $\sfSpin(6)$. The pure spinor actions for ABJM and ABJ models are then of the form~\cite{Cederwall:2008xu} 
    \begin{equation}
        \begin{aligned}
            S_{\text{3D}\,\caN=6}&=\int \Omega_{\text{3D}\,\caN=6}\Big(\inner{\Psi}{Q\Psi+\tfrac13[\Psi,\Psi]}_\frg
            \\
            &\kern1.5cm+g^m{}_n\inner{\bar\Phi_m}{Q\Phi^n+\Psi\Phi^n}_V\Big)
        \end{aligned}
    \end{equation}
    with $g^m{}_n=\tfrac12\varepsilon_{\alpha\beta}\varepsilon_{klpn}\lambda^{\alpha mk}\lambda^{\beta lp}$.
    
    Unfortunately, the kinematic vector space here does not admit a suitable symplectic metric without breaking the pure spinor formalism. We can, however, formally quadruple the matter field space such that the matter fields take values in $(V\oplus V^*)\otimes(\mathbf{4}\oplus \overline{\mathbf{4}})$. This violates the non-linear BRST symmetry~\cite{Cederwall:2008xu} and hence unitarity for arbitrary external states, but the correct tree amplitudes for the ABJM and ABJ models can be produced by restricting to appropriate external states.
    
    After this enlargement, the dg-Lie algebra factorizes into a (gauge) metric Lie module together and a $\BVbox$-module. This shows that also the ABJM and ABJ models are CK-dual at the tree level, in the usual sense and to all orders.
    
    
    \section{Relation to quartic CK-duality}\label{sec:relation-to-literature}
    
    Previous literature~\cite{Huang:2012wr,Huang:2013kca,Sivaramakrishnan:2014bpa} (but not~\cite{Ben-Shahar:2021zww}) considered CK-duality and double copy of CSM theories in terms of 3-Lie algebras and graphs with quartic vertices, rather than the usual cubic notion of CK-duality for gauge--matter theories we recovered above. This section explains the relation between the two notions.
    
    First, consider the BLG model. In the pure spinor formalism, the closure of the BRST symmetry requires total anti-symmetry and total symmetry of the 3-brackets $\llbracket-,-,-\rrbracket$ and $\llparenthesis-,-,-\rrparenthesis$ respectively~\cite{Cederwall:2008vd}. To produce quartic and sextic vertices in $\Phi$, one should integrate out auxiliary modes in the superfield $\Psi$. Schematically,
    \begin{equation}
        \Psi Q\Psi+\Phi\Psi\Phi+\Psi^3\mapsto \Phi^2 Q^{-1}\Phi^2+(Q^{-1}\Phi^2)^3+\cdots.
    \end{equation}
    One can use~\eqref{eq:Jacobi-mixed} and its dg-metric Com module analogue to rewrite sextic vertices in terms of quartic ones:
    \renewcommand\nodearray{
        \matrix (m) [
        matrix of nodes,
        ampersand replacement=\&,
        column sep=0.1cm,
        row sep=0.1cm
        ]{
            {} \& {} \& {} \& {} \& {}
            \\
            {} \& {} \& {} \& {} \& {}
            \\
            {} \& {} \& {} \& {} \& {}
            \\
            {} \& {} \& {} \& {} \& {}
            \\
            {} \& {}\& {} \& {} \& {}
            \\
        };
    }
    \begin{equation}
        \begin{tikzpicture}[
            scale=1,
            every node/.style={scale=1},
            baseline={([yshift=-.5ex]current bounding box.center)}
            ]
            \nodearray
            \draw [matter] (m-1-2) -- (m-2-3.center) -- (m-1-4);
            \draw [gluon] (m-2-3.center) -- (m-3-3.center);
            \draw [gluon] (m-3-3.center) -- (m-4-2.center);
            \draw [gluon] (m-3-3.center) -- (m-4-4.center);
            \draw [matter] (m-5-2) -- (m-4-2.center) -- (m-4-1);
            \draw [matter] (m-5-4) -- (m-4-4.center) -- (m-4-5);
        \end{tikzpicture}
        \to
        \begin{tikzpicture}[
            scale=1,
            every node/.style={scale=1},
            baseline={([yshift=-.5ex]current bounding box.center)}
            ]
            \nodearray
            \draw [matter] (m-1-2) -- (m-3-2.center) -- (m-3-4.center) -- (m-1-4);
            \draw [gluon] (m-3-2.center) -- (m-4-2.center);
            \draw [gluon] (m-3-4.center) -- (m-4-4.center);
            \draw [matter] (m-5-2) -- (m-4-2.center) -- (m-4-1);
            \draw [matter] (m-5-4) -- (m-4-4.center) -- (m-4-5);
        \end{tikzpicture}
        +
        \begin{tikzpicture}[
            scale=1,
            every node/.style={scale=1},
            baseline={([yshift=-.5ex]current bounding box.center)}
            ]
            \nodearray
            \draw [matter] (m-1-2) -- (m-3-4.center) -- (m-3-2.center) -- (m-1-4);
            \draw [gluon] (m-3-2.center) -- (m-4-2.center);
            \draw [gluon] (m-3-4.center) -- (m-4-4.center);
            \draw [matter] (m-5-2) -- (m-4-2.center) -- (m-4-1);
            \draw [matter] (m-5-4) -- (m-4-4.center) -- (m-4-5);
        \end{tikzpicture}.
    \end{equation}
    Note that the right two diagrams are glued together from two quartic $\Phi^4$ vertices. By construction, the coefficients of the resulting total quartic vertex is totally anti-symmetric; equivariance of the cubic vertices induces equivariance and thus CK-duality of the quartic $\Phi^4$ vertices. Therefore, for the BLG model, cubic CK-duality implies quartic CK-duality, in accordance with the observation in~\cite{Huang:2012wr,Huang:2013kca} that the on-shell 3-Lie algebra CK-duality of BLG holds up to $10$ points and that the double copy to 3D $\caN=16$ supergravity works.
    
    Next, consider the ABJM and ABJ models. Here, the 3-brackets~\eqref{eq:3-bracket-antisymmetric-defn} and~\eqref{eq:3-bracket-symmetric-defn} still exist but fail to be totally (anti-)symmetric (although they are cyclic with respect to the metric). Hence, when one translates the cubic graphs into the corresponding quartic graphs, one must remember the cyclic order of the attached edges. Thus, the scattering amplitude is partitioned into terms labeled not by unadorned quartic trees but rather by quartic trees with extra labeling data. This accords with and explains the observation in~\cite{Huang:2013kca,Sivaramakrishnan:2014bpa} that, for the ABJM model, the quartic BCJ identities (formulated with unadorned quartic graphs) fail starting at eight points and that the double copy also fails.
    
    \section{Concluding remarks}
    
    Let us close with a number of remarks on our construction. First of all, the present discussion is rather terse on the mathematical background; we intend to remedy this in~\cite{Borsten:2022aa}. 
    
    An evident consequence of our observations is that one can double-copy~\cite{Borsten:2020zgj,Borsten:2021hua} the pure spinor action of YM theory and the CSM theories we discussed to obtain pure spinor actions of 10D and 3D $\caN=16$ supergravity. Explicitly, one should use the formalism of~\cite{Borsten:2023ab} allowing to double copy field theories with manifest underlying $\BVbox$-algebra.
    
    Next, we note that our claim of cubic CK-duality for $\caN=6$ CSM theories does not conflict with the claim in~\cite{Ben-Shahar:2021zww} that $\caN=4$ is the maximal supersymmetry for CSM theories compatible with CK-duality, since the claims of~\cite{Ben-Shahar:2021zww} only pertain to adjoint matter whereas we allow for general matter.
    
    Finally, we stress that our arguments apply only to the tree level, and there are certain obstructions to reaching the loop level. Suppose a cubic action $S$ of SYM theory, potentially formulated on some auxiliary space (e.g.~pure spinors, twistor space, harmonic or projective superspace) that manifests CK-duality off shell for all fields in some gauge. Further assume that the Kaluza--Klein expansion in the auxiliary coordinates and integrating out all auxiliary fields reproduces the standard SYM action $S_\text{std}=\int\tr(F^2)+\cdots$ in a local, polynomial, Lorentz-invariant gauge. Then, by assumption, the off-shell tree-level correlators of $S_\text{std}$, which equal those of $S$ with external legs one of $(c,A,\phi,\chi,A^+,\phi^+,\chi^+,c^+)$, are CK-dual. Further, $S_\text{std}$ computes SYM loop amplitudes correctly (with the standard path integral measure, i.e.\ defined using dimensional regularization etc.), and it can be truncated to $S_\text{std}^{\caN=0}$ consisting of all terms only containing non-spinorial fields $(c,A,\phi,A^+,\phi^+,c^+)$. The action $S_\text{std}^{\caN=0}$ manifestly suffices for computing $\caN=0$ SYM\footnote{that is, dimensional reduction of 10D $\caN=0$ Yang--Mills theory, which is Yang--Mills theory with adjoint scalars} scattering amplitudes. Moreover, $S_\text{std}^{\caN=0}$ loop amplitudes must be CK-dual, since they can be glued out of off-shell $S_\text{std}^{\caN=0}$ tree correlators, which are a subset of the $S_\text{std}$ off-shell tree correlators. But this contradicts~\cite{Bern:2015ooa},
    which shows that arbitrary-dimensional $\caN=0$ Yang--Mills theory lacks loop-level CK-duality with manifestly Lorentz-invariant  polynomial numerators compatible with some Feynman rules. 
    
    From this perspective, the ambitwistor action in~\cite{Movshev:2004ub,Mason:2005kn,Borsten:2022vtg} with the limitation $Q\sfb+\sfb Q=\BBox\neq\Box$ seems the best one can hope for. It manifests a kinematic Lie algebra at both the tree and loop level, without directly implying CK-duality.\footnote{Again, this does not contradict~\cite{Bern:2015ooa}, because the truncation to $\caN=0$ fails at the loop level due to a gauge anomaly.} 
    
    In the pure spinor picture, failure of loop CK-duality may be seen as  due to an incompatibility between a regulator of an ultraviolet divergence (namely, the regulated $\sfb$-operator) and a tree-level symmetry (the kinematic algebra), which may be interpreted as an anomaly, consistent with the perspective of~\cite{Borsten:2021rmh,Borsten:2022vtg}.
    
    \
        
    \noindent
    {\bf{Data Management.}}
    No additional research data beyond the data presented and cited in this work are needed to validate the research findings in this work. For the purpose of open access, the authors have applied a Creative Commons Attribution (CC BY) license to any Author Accepted Manuscript version arising.
    
    \
    
    \begin{acknowledgments}
        \noindent
        {\bf{Acknowledgments.}}
        We thank Maor Ben-Shahar for discussions and detailed explanations of the results of~\cite{Ben-Shahar:2021doh} as well as Martin Cederwall for helpful comments. H.K. and and C.S.~were supported by the Leverhulme Research Project Grant RPG-2018-329. B.J.~was supported by the GA\v{C}R Grant EXPRO 19-28268X.
    \end{acknowledgments}
    
    \bibliography{ref/bigone}
    
\end{document}
