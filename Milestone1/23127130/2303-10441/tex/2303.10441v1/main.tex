
%%
%% This is file `sample-manuscript.tex',
%% generated with the docstrip utility.
%%
%% The original source files were:
%%
%% samples.dtx  (with options: `manuscript')
%% 
%% IMPORTANT NOTICE:
%% 
%% For the copyright see the source file.
%% 
%% Any modified versions of this file must be renamed
%% with new filenames distinct from sample-manuscript.tex.
%% 
%% For distribution of the original source see the terms
%% for copying and modification in the file samples.dtx.
%% 
%% This generated file may be distributed as long as the
%% original source files, as listed above, are part of the
%% same distribution. (The sources need not necessarily be
%% in the same archive or directory.)
%%
%% The first command in your LaTeX source must be the \documentclass command.
%%%% Small single column format, used for CIE, CSUR, DTRAP, JACM, JDIQ, JEA, JERIC, JETC, PACMCGIT, TAAS, TACCESS, TACO, TALG, TALLIP (formerly TALIP), TCPS, TDSCI, TEAC, TECS, TELO, THRI, TIIS, TIOT, TISSEC, TIST, TKDD, TMIS, TOCE, TOCHI, TOCL, TOCS, TOCT, TODAES, TODS, TOIS, TOIT, TOMACS, TOMM (formerly TOMCCAP), TOMPECS, TOMS, TOPC, TOPLAS, TOPS, TOS, TOSEM, TOSN, TQC, TRETS, TSAS, TSC, TSLP, TWEB.
\documentclass[sigconf]{acmart}

%%%% Large single column format, used for IMWUT, JOCCH, PACMPL, POMACS, TAP, PACMHCI
% \documentclass[acmlarge,screen]{acmart}

%%%% Large double column format, used for TOG
% \documentclass[acmtog, authorversion]{acmart}

%%%% Generic manuscript mode, required for submission
%%%% and peer review
%% Fonts used in the template cannot be substituted; margin 
%% adjustments are not allowed.
%%
%% \BibTeX command to typeset BibTeX logo in the docs
\AtBeginDocument{%
  \providecommand\BibTeX{{%
    \normalfont B\kern-0.5em{\scshape i\kern-0.25em b}\kern-0.8em\TeX}}}




\newcommand \change[1]{{\textcolor{red}{#1}}}
\newcommand \del[1]{\textcolor{red}{\sout{#1}}}
\newcommand \todo[1]{{\textcolor{green}{#1}}}

\newcommand \revision[1]{{\textcolor{black}{#1}}}

%% Rights management information.  This information is sent to you
%% when you complete the rights form.  These commands have SAMPLE
%% values in them; it is your responsibility as an author to replace
%% the commands and values with those provided to you when you
%% complete the rights form.
\setcopyright{acmcopyright}
% \copyrightyear{2023}
% \acmYear{2023}
% \acmDOI{10.1145/1122445.1122456}

%% These commands are for a PROCEEDINGS abstract or paper.
\acmConference[CHI '23]{CHI' 23: The ACM CHI Conference on Human Factors in Computing Systems}{April 23 --- April 28, 2023 }{Hamburg, Germany}
\acmBooktitle{CHI '23: The ACM CHI Conference on Human Factors in Computing Systems, April 23 --- April 28, 2023, Hamburg, Germany}
\acmPrice{15.00}


%%
%% Submission ID.
%% Use this when submitting an article to a sponsored event. You'll
%% receive a unique submission ID from the organizers
%% of the event, and this ID should be used as the parameter to this command.
%%\acmSubmissionID{123-A56-BU3}

%%
%% The majority of ACM publications use numbered citations and
%% references.  The command \citestyle{authoryear} switches to the
%% "author year" style.
%%
%% If you are preparing content for an event
%% sponsored by ACM SIGGRAPH, you must use the "author year" style of
%% citations and references.
%% Uncommenting
%% the next command will enable that style.
%%\citestyle{acmauthoryear}

%%
%% end of the preamble, start of the body of the document source.
\copyrightyear{2023} 
\acmYear{2023} 
\setcopyright{rightsretained} 
\acmConference[CHI '23]{Proceedings of the 2023 CHI Conference on Human Factors in Computing Systems}{April 23--28, 2023}{Hamburg, Germany}
\acmBooktitle{Proceedings of the 2023 CHI Conference on Human Factors in Computing Systems (CHI '23), April 23--28, 2023, Hamburg, Germany}\acmDOI{10.1145/3544548.3581008}
\acmISBN{978-1-4503-9421-5/23/04}


\begin{document}

%%
%% The "title" command has an optional parameter,
%% allowing the author to define a "short title" to be used in page headers.
\title[]{Enabling Voice-Accompanying Hand-to-Face Gesture Recognition with Cross-Device Sensing}

%% Towards Designing and Recognizing 
% Enabling Concomitant Hand-to-Face Gesture Sensing for Voice Interaction

%%
%% The "author" command and its associated commands are used to define
%% the authors and their affiliations.
%% Of note is the shared affiliation of the first two authors, and the
%% "authornote" and "authornotemark" commands
%% used to denote shared contribution to the research.

\author{Zisu Li}
\authornote{indicates equal contribution.}
\authornote{This work was conducted when Zisu Li was a research intern at Tsinghua University.}
\email{zlihe@connect.ust.hk}
\orcid{0000-0001-8825-0191}
\affiliation{
  \institution{Tsinghua University}
  \country{Beijing, China}
}
\affiliation{
  \institution{The Hong Kong University of Science and Technology}
  \country{Hong Kong SAR, China}
}

\author{Chen Liang}
\authornotemark[1]
\email{liang-c19@mails.tsinghua.edu.cn}
\orcid{0000-0003-0579-2716}
\affiliation{
  \institution{Tsinghua University}
  \country{Beijing, China}
}

\author{Yuntao Wang}
\authornote{indicates the corresponding author.}
\email{yuntaowang@tsinghua.edu.cn}
\orcid{0000-0002-4249-8893}
\affiliation{
  \institution{Tsinghua University}
  \country{Beijing, China}
}

\author{Yue Qin}
\email{qiny19@mails.tsinghua.edu.cn}
\orcid{0000-0003-1351-5284}
\affiliation{
  \institution{Tsinghua University}
  \country{Beijing, China}
}

\author{Chun Yu}
\email{chunyu@tsinghua.edu.cn}
\orcid{0000-0003-2591-7993}
\affiliation{
  \institution{Tsinghua University}
  \country{Beijing, China}
}

\author{Yukang Yan}
\email{yukangy@andrew.cmu.edu}
\orcid{0000-0001-7515-3755}
\affiliation{
  \institution{Tsinghua University}
  \country{Beijing, China}
}

\author{Mingming Fan}
\email{mingmingfan@ust.hk}
\orcid{0000-0002-0356-4712}
\affiliation{
 % \institution{Computational Media and Arts Thrust}
  \institution{The Hong Kong University of Science and Technology (Guangzhou)}
  \city{Guangzhou}
  \country{China}
  }
\affiliation{
  \institution{The Hong Kong University of Science and Technology}
  \city{Hong Kong SAR}
  \country{China}
}

\author{Yuanchun Shi}
\email{shiyc@tsinghua.edu.cn}
\orcid{0000-0003-2273-6927}
\affiliation{
  \institution{Tsinghua University}
  \city{Beijing}
  \country{China}
}
\affiliation{
  \institution{Qinghai University}
  \city{Xining}
  \country{China}
}

%%
%% By default, the full list of authors will be used in the page
%% headers. Often, this list is too long, and will overlap
%% other information printed in the page headers. This command allows
%% the author to define a more concise list
%% of authors' names for this purpose.
\renewcommand{\shortauthors}{Li and Liang, et al.}
%%
%% The abstract is a short summary of the work to be presented in the
%% article.
%, to indicate or convey certain intentions,

\begin{abstract}
Gestures performed accompanying the voice are essential for voice interaction to convey complementary semantics for interaction purposes such as wake-up state and input modality. In this paper, we investigated voice-accompanying hand-to-face (VAHF) gestures for voice interaction. We targeted on hand-to-face gestures because such gestures relate closely to speech and yield significant acoustic features (e.g., impeding voice propagation). We conducted a user study to explore the design space of VAHF gestures, where we first gathered candidate gestures and then applied a structural analysis to them in different dimensions (e.g., contact position and type), outputting a total of 8 VAHF gestures with good usability and least confusion. To facilitate VAHF gesture recognition, we proposed a novel cross-device sensing method that leverages heterogeneous channels (vocal, ultrasound, and IMU) of data from commodity devices (earbuds, watches, and rings). Our recognition model achieved an accuracy of 97.3\% for recognizing 3 gestures and 91.5\% for recognizing 8 gestures \revision{(excluding the "empty" gesture)}, proving the high applicability. Quantitative analysis also shed light on the recognition capability of each sensor channel and their different combinations. In the end, we illustrated the feasible use cases and their design principles to demonstrate the applicability of our system in various scenarios. 

%% Old
% Voice interaction with computing devices has been rated as a promising technique which can be used in a wide range of tasks. Unfortunately, the modality control in voice interaction is still a challenging problem due to the implicitness of modality information in speech and the restricted NLP techniques. In this paper, we proposed \projectName{}, a novel sensing technique allowing the user to use hand-to-face gestures to enhance modality control in voice interaction. With \projectName{}, when a user perform a hand-to-face gesture and speak simultaneously, the acoustic feature of his speech influenced by his gesture serves as a channel for gesture recognition and the speech semantics is passed to the application indicated by the recognized gesture as parameters. For example, a user can keep his hand on his specific face region(e.g., cheek) to activate certain function and speaking (e.g., consult the weather to Siri without wake-up, sending the speech to Mom). We explore the design space of \projectName{} gestures and categorized them in different dimensions (face region, touch point, open or closed hands). Then we leverage heterogeneous sensor data from multiple commodity devices to recognize \projectName{} gestures. Our detailed quantitative analysis shed light on the recognition capability of the different sensor combinations including acoustic and movement sensors of earbuds, smart watch and smart ring over gestures with different characteristics. [We achieved xxx data.] In our evaluation and application, we provided guidelines for the design of the recognition system of the combined use of voice and \projectName{} on cross-device interaction.
\end{abstract}

%%
%% The code below is generated by the tool at http://dl.acm.org/ccs.cfm.
%% Please copy and paste the code instead of the example below.
%%

\begin{CCSXML}
<ccs2012>
   <concept>
       <concept_id>10003120.10003138.10003141</concept_id>
       <concept_desc>Human-centered computing~Ubiquitous and mobile devices</concept_desc>
       <concept_significance>500</concept_significance>
       </concept>
   <concept>
       <concept_id>10003120.10003121.10003128.10011755</concept_id>
       <concept_desc>Human-centered computing~Gestural input</concept_desc>
       <concept_significance>500</concept_significance>
       </concept>
 </ccs2012>
\end{CCSXML}

\ccsdesc[500]{Human-centered computing~Ubiquitous and mobile devices}
\ccsdesc[500]{Human-centered computing~Gestural input}

% \begin{CCSXML}
% <ccs2012>
%  <concept>
%   <concept_id>10010520.10010553.10010562</concept_id>
%   <concept_desc>Computer systems organization~Embedded systems</concept_desc>
%   <concept_significance>500</concept_significance>
%  </concept>
%  <concept>
%   <concept_id>10010520.10010575.10010755</concept_id>
%   <concept_desc>Computer systems organization~Redundancy</concept_desc>
%   <concept_significance>300</concept_significance>
%  </concept>
%  <concept>
%   <concept_id>10010520.10010553.10010554</concept_id>
%   <concept_desc>Computer systems organization~Robotics</concept_desc>
%   <concept_significance>100</concept_significance>
%  </concept>
%  <concept>
%   <concept_id>10003033.10003083.10003095</concept_id>
%   <concept_desc>Networks~Network reliability</concept_desc>
%   <concept_significance>100</concept_significance>
%  </concept>
% </ccs2012>
% \end{CCSXML}

% \ccsdesc[500]{Computer systems organization~Embedded systems}
% \ccsdesc[300]{Computer systems organization~Redundancy}
% \ccsdesc{Computer systems organization~Robotics}
% \ccsdesc[100]{Networks~Network reliability}

%%
%% Keywords. The author(s) should pick words that accurately describe
%% the work being presented. Separate the keywords with commas.
\keywords{hand gestures, acoustic sensing, sensor fusion}

%% A "teaser" image appears between the author and affiliation
%% information and the body of the document, and typically spans the
%% page.
\begin{teaserfigure}
  \centering
  \includegraphics[width=0.8\textwidth]{figures/teaser.png}
  \caption{A typical usage scenario enabled by voice-accompanying hand-to-face (VAHF) gestures. (a) The user wants to know how to make egg fried rice. (b) The user can perform different VAHF gestures to redirect their voice input to different targets (e.g., asking Siri, searching on Google with the transcribed text, and sending a voice message to mum). (c) The user performs a "phone call" gesture and speaks simultaneously. (d) The smart devices recognize the user's intention through the performed VAHF gesture and simulate sending a voice message to the user's mum.}
  \Description{This figure contains a typical usage scenario enabled by voice-accompanying hand-to-face (VAHF) gestures. (a) The user wants to know how to make egg-fried rice. (b) The user can perform different VAHF gestures to redirect their voice input to different targets (e.g., asking Siri, searching on Google with the transcribed text, and sending a voice message to mum). (c) The user performs a "phone call" gesture and speaks simultaneously. (d) The smart devices recognize the user's intention through the performed VAHF gesture and simulate sending a voice message to the user's mum. }
  \label{fig:teaser}
\end{teaserfigure}


%%
%% This command processes the author and affiliation and title
%% information and builds the first part of the formatted document.
\maketitle
\section{Introduction}


Recent years have witnessed the rise of human digitization~\cite{habermannDeepCapMonocularHuman2020,alexanderCREATINGPHOTOREALDIGITAL,pengNeuralBodyImplicit2021,alldieckDetailedHumanAvatars2018, rajANRArticulatedNeural2020}. This technology greatly impacts the entertainment, education, design, and engineering industry.
There is a well-developed industry solution for this task.
High-fidelity reconstruction of humans can be achieved either with full-body laser scans~\cite{saitoSCANimateWeaklySupervised2021}, dense synchronized multi-view cameras~\cite{xiangModelingClothingSeparate2021a,xiangDressingAvatarsDeep2022a}, or light stages~\cite{alexanderCREATINGPHOTOREALDIGITAL}.
However, these settings are expensive and tedious to deploy and consist of a complex processing pipeline, preventing the technology's democratization.

Another solution is to view the problem as inverse rendering and learn digital humans directly from custom-collected data.
Traditional approaches directly optimize explicit mesh representation~\cite{loperSMPLSkinnedMultiperson2015, fangRMPERegionalMultiperson2018, pavlakosExpressiveBodyCapture2019} which suffers from the problems of smooth geometry and coarse textures~\cite{prokudinSMPLpixNeuralAvatars2020,alldieckVideoBasedReconstruction2018}. Besides, they require professional artists to design human templates, rigging, and unwrapped UV coordinates.
Recently, with the help of volumetric-based implicit representations~\cite{mildenhallNeRFRepresentingScenes2020, parkDeepSDFLearningContinuous2019, meschederOccupancyNetworksLearning2019} and neural rendering~\cite{laineModularPrimitivesHighPerformance2020, liuSoftRasterizerDifferentiable2019, thiesDeferredNeuralRendering2019}, 
one can easily digitize a quality-plausible human avatar from video footage~\cite{jiangNeuManNeuralHuman2022,wengHumanNeRFFreeviewpointRendering}.
Particularly, volumetric-based implicit representations~\cite{mildenhallNeRFRepresentingScenes2020, pengNeuralBodyImplicit2021} can reconstruct scenes or objects with much higher fidelity against previous neural renderer~\cite{thiesDeferredNeuralRendering2019,prokudinSMPLpixNeuralAvatars2020}, and is more user-friendly as it does not need any human templates, pre-set rigging, or UV coordinates.
Captured visual footage and corresponding skeleton tracking are enough for training.
However, better reconstructions and more friendly usability are at the expense of the following factors.
1) \textbf{Inefficiency:}
They require longer optimization times (typically tens of hours or days) and inference slowly.
Volume rendering~\cite{mildenhallNeRFRepresentingScenes2020,lombardiNeuralVolumesLearning2019} formulates images by querying the densities and colors of millions of spatial coordinates. 
In the training stage, due to memory constraints, only a small fraction of points are sampled which leads to slow convergence speed.
2) \textbf{Entangled representations}:
The geometry, materials, and motion dynamics are entangled in the neural networks. 
Due to the implicit nature of neural nets, one can hardly edit one property without touching the others~\cite{yuanNeRFEditingGeometryEditing2022a,liuEditingConditionalRadiance2021}.
3) \textbf{Graphics incompatibility}:
Volume rendering is incompatible with the current popular graphic pipeline,
which renders triangular/quadrilateral meshes efficiently with the rasterization technique.
Many downstream applications require mesh rasterization in their workflow (\eg, editing~\cite{foundationBlenderOrgHome}, simulation~\cite{benderPositionBasedSimulationMethods2015}, real-time rendering~\cite{akenine2019real}, ray-tracing~\cite{waldRTXRayTracing}).
Although there are approaches~\cite{lorensenMarchingCubesHigh,labelleIsosurfaceStuffingFast2007} can convert volumetric fields into meshes, the gaps from discrete sampling degrade the output quality in terms of both meshes and textures.


To address these issues, we present \textbf{EMA}, a method based on \textbf{E}fficient \textbf{M}eshy neural fields to reconstruct animatable human \textbf{A}vatars.
Our method enjoys flexibility from implicit representations and efficiency from explicit meshes, yet still maintains high-fidelity reconstruction quality.
Given video sequences and the corresponding pose tracking, our method digitizes humans in terms of canonical triangular meshes, physically-based rendering (PBR) materials, and skinning weights \textit{w.r.t.} skeletons.
We jointly learn the above components via inverse rendering~\cite{laineModularPrimitivesHighPerformance2020,chenDIBRLearningPredict2021,chenLearningPredict3D2019} in an end-to-end manner.
Each of them is derived from a separate neural field, which relaxes the requirements of a preset human template, rigging, or UV coordinates.
Specifically, we predict a canonical mesh out of a signed distance field (SDF) by differentiable marching tetrahedra~\cite{shenDeepMarchingTetrahedra2021,gaoGET3DGenerativeModel,gaoLearningDeformableTetrahedral2020,munkbergExtractingTriangular3D2022}, then we extend the marching tetrahedra~\cite{shenDeepMarchingTetrahedra2021} for spatial-varying materials by utilizing a neural field to predict PBR materials \textit{on the mesh surfaces} after rasterization~\cite{munkbergExtractingTriangular3D2022,hasselgrenShapeLightMaterial2022,laineModularPrimitivesHighPerformance2020}.
To make the canonical mesh animatable, we take another neural field to model the forward linear blend skinning for the meshes. 
Given a posed skeleton, the canonical mesh is then transformed into the corresponding poses.
Finally, we shade the mesh with a rasterization-based differentiable renderer~\cite{laineModularPrimitivesHighPerformance2020} and train our models with a photo-metric loss.
After training, we export the mesh with materials and discard the neural fields.

\looseness=-1
There are several merits of our method design.
1) \textbf{Efficiency}:
Powered by efficient mesh rendering, our method can render in real-time.
Besides, the training speed is boosted as well, 
since we compute loss holistically on the whole image and the gradients only flow on the mesh surface. In contrast, volume rendering takes limited pixels for loss computation and back-propagates the gradients in the whole space.
Our method only needs about an hour of training and minutes of optimization are enough for plausible avatar reconstruction.
2) \textbf{Disentangled representations}:
Our shape, materials, and motion modules are disentangled naturally by design, which facilitates editing. 
Besides, Canonical meshes with forward skinning modeling handle the out-of-distribution poses better.
3) \textbf{Graphics compatibility}:
Our derived mesh representation is compatible with 
the prominent graphic pipeline, which leads to instant downstream applications (\eg, the shape and materials can be edited directly in design software~\cite{foundationBlenderOrgHome}).
To further improve reconstruction quality, we additionally optimize image-based environment lights and non-rigid motions.


We conduct extensive experiments on standards benchmarks H36M~\cite{ionescuHuman36MLarge2014b} and ZJU-MoCap~\cite{pengNeuralBodyImplicit2021}.
Our method achieves very competitive performance for novel view synthesis, generalizes better for novel poses, 
and significantly improves both training time and inference speed against previous arts.
Our research-oriented code reaches real-time inference speed (100+ FPS for rendering $512\times512$ images).
We in addition showcase applications including novel pose synthesis, material editing, and relighting.
\section{Related Work} \label{part 2}

\subsection{Deep learning for time series forecasting}

\label{part dl related}

Time series forecasting has been studied for decades. The field has been dominated for a long time by statistical tools such as ARIMA, Exponential Smoothing (ES), or (S)ARIMAX, this last model allowing the use of exogenous variables. It now opens itself to deep learning models \citep{9461796}. These new models recently achieved great performances on many datasets. Three main parts compose typical DNNs: an input layer, several hidden layers and an output layer. In this paper we apply our framework to optimize the hidden layers for a given time series forecasting task (see Figure~\ref{fig:metamodel_monash}). In this part, we introduce usual DNN layers for time series forecasting, which can be used in our search space.

The first layer type from our search space is the fully-connected layer, or Multi-Layer Perceptron (MLP). The input vector is multiplied by a weight matrix. Most architectures use such layers as simple building blocks for dimension matching, input embedding or output modelling. The N-Beats model is a well-known example of a DNN based on fully-connected layers for time series forecasting \citep{NBeats}.

The second layer type \citep{lecun2015deep} is the convolution layer (CNN). Inspired by the human brain's visual cortex, it has mainly been popularised for computer vision. The convolution layer uses a discrete convolution operator between the input data and a small matrix called a filter. The extracted features are local and time-invariant if the considered data are time series. Many architectures designed for time series forecasting are based on convolution layers such as WaveNet \citep{oord2016wavenet} and Temporal Convolution Networks \citep{lea2017temporal}.

The third layer type is the recurrent layer (RNN), specifically designed for sequential data processing, therefore, particularly suitable for time series. These layers scan the sequential data and keep information from the sequence past in memory to predict its future. A popular model based on RNN layers is the Seq2Seq network \citep{seq2seq}. Two RNNs, an encoder and a decoder, are sequentially connected by a fixed-length vector. Various versions of the Seq2Seq model have been introduced in the literature, such as the DeepAR model \citep{salinas2020deepar}, which encompasses an RNN encoder in an autoregressive model. The major weakness of RNN layers is the modelling of long-term dynamics due to the vanishing gradient. Long Short-Term Memory (LSTM) and Gated Recurrent Unit (GRU) layers have been introduced \citep{hochreiter1997long, chung2014empirical} to overcome this problem.

Finally, the layer type from our search space is the attention layer. The attention layer has been popularized within the deep learning community as part of Vaswani's transformer model \citep{vaswani2017attention}. The attention layer is more generic than the convolution. It can model the dependencies of each element from the input sequence with all the others. In the vanilla transformer \citep{vaswani2017attention}, the attention layer does not factor the relative distance between inputs in its modelling but rather the element's absolute position in the sequence. The Transformer-XL \citep{dai2019transformer}, a transformer variant created to tackle long-term dependencies tasks, introduces a self-attention version with relative positions. \citet{cordonnier2019relationship} used this new attention formulation to show that, under a specific configuration of parameters, the attention layers could be trained as convolution layers. Within our search space, we chose this last formulation of attention, with the relative positions.

The three first layers (i.e. MLP, CNN, RNN) were frequently mixed into DNN architectures. Sequential and parallel combinations of convolution, recurrent and fully connected layers often compose state-of-the-art DNN models for time series forecasting. Layer diversity enables the extraction of different and complementary features from input data to allow a better prediction. Some recent DNN models introduce transformers into hybrid DNNs. In \citet{lim2021temporal}, the authors developed the Temporal Fusion Transformer, a hybrid model stacking transformer layers on top of an RNN layer. With this in mind, we built a flexible search space which generalizes hybrid DNN models including MLPs, CNNs, RNNs and transformers.

\subsection{Search spaces for automated deep learning}

Designing an efficient DNN for a given task requires choosing an architecture and tuning its many hyperparameters. It is a difficult, fastidious, and time-consuming optimization task. Moreover, it requires expertise and restricts the discovery of new DNNs to what humans can design. Research related to the automatic design and optimization of DNNs has therefore risen this last decade \citep{talbi2021automated}. The first challenge in automatic deep learning (AutoDL), and more specifically neural architecture search (NAS), is search space design. Typical search spaces for Hyperparameters Optimization (HPO) are a product space of a mixture of continuous and categorical dimensions (e.g. learning rate, number of layers, batch size), while NAS focuses on optimizing the topology of the DNN \citep{white2023neural}. Encoding a DNN topology is a complex task because the encoding should not be too broad and allow too many architectures to keep the search efficient. On the contrary, if the encoding is too restrictive, we may miss promising solutions and novel architectures. This means before creating the search space we need to choose which DNNs or type of DNNs are relevant or not to the problem at hand. Once we have decided on this broad set of DNNs, we define the search space following a set of rules \citep{talbi2021automated}:

\begin{itemize}
\item Completeness: all (or almost all) relevant DNNs from this broad set should be encoded in the search space.
\item Connectedness: a path should always be possible between two encoded DNNs in the search space.
\item Efficiency: the encoding should be easy to manipulate by the search operators (i.e. neighbourhoods, variation operators) of the search strategy.
\item Constraint handling: the encoding should facilitate the handling of the various constraints to generate feasible DNNs.
\end{itemize}

A complete classification of encoding strategies for NAS is presented in \citet{talbi2021automated} and reproduced in Figure \ref{fig:classi_encoding}. We can discriminate between direct and indirect encodings. With direct strategies, the DNNs are completely defined by the encoding, while indirect strategies need a decoder to find the architecture back. Amongst direct strategies, one can discriminate between two categories: flat and hierarchical encodings. In flat encodings, all layers are individually encoded \citep{loni2020deepmaker, sun2018particle, wang2018evolving, wang2019evolving}. The global architecture can be a single chain, with each layer having a single input and a single output, which is called chain structured \citep{assuncao_denser_2018}, but more complex patterns such as multiple outputs, skip connections, have been introduced in the extended flat DNNs encoding \citep{chen_scale-aware_2021}. For hierarchical encodings, they are bundled in blocks \citep{pham2018efficient, shu2019understanding, liu2017hierarchical, zhang2019d}. If the optimization is made on the sequencing of the blocks, with an already chosen content, this is referred to as inner-level fixed \citep{camero2021bayesian, white2021bananas}. If the optimization is made on the blocks' content with a fixed sequencing, it is called outer level fixed. A joint optimization with no level fixed is also an option \citep{liu2019auto}. Regarding the indirect strategies, one popular encoding is the one-shot architecture \citep{bender2018understanding, brock2017smash}. One single large network resuming all candidates from the search space is trained. Then the architectures are found by pruning some branches. Only the best promising architectures are retrained from scratch.

\begin{figure*}[htbp]
\centering
    \begin{tikzpicture}
        \draw node[] (SE) {Solution Encoding};
        \draw node[below left =0.5cm and 1.8cm of SE] (Direct) {Direct};
        \draw node[below right =0.5cm and 1.8cm of SE] (Indirect) {Indirect};
        \draw node[below left =0.5cm and 2cm of Direct] (Flat) {Flat};
        \draw node[below right =0.5cm and 2cm of Direct] (Hierarchical) {Hierarchical};
        \draw node[below left =0.5cm and 0cm of Flat, align=center] (Chain) {Chain\\structured};
        \draw node[below right =0.5cm and 0cm of Flat, align=center] (Extended) {Extended\\Flat DNNs};
        \draw node[below =0.5cm of Hierarchical, align=center] (Outer) {Outer\\Level fixed};
        \draw node[left =0.5cm of Outer, align=center] (Inner) {Inner\\Level Fixed};
        \draw node[right =0.5cm of Outer, align=center] (No) {No level\\fixed};
        \draw node[below =of Indirect, align=center] (One) {One-shot};
        \draw[->] (SE) -- (Direct);
        \draw[->] (SE) -- (Indirect);
        \draw[->] (Direct) -- (Flat);
        \draw[->] (Direct) -- (Hierarchical);
        \draw[->] (Flat) -- (Chain);
        \draw[->] (Flat) -- (Extended);
        \draw[->] (Hierarchical) -- (Inner);
        \draw[->] (Hierarchical) -- (Outer);
        \draw[->] (Hierarchical) -- (No);
        \draw[->] (Indirect) -- (One);
    \end{tikzpicture}
    \caption{Classification of encoding strategies for NAS \citep{talbi2021automated}.}
    \label{fig:classi_encoding}
\end{figure*}

Our search space can be categorized as a direct and extended flat encoding. It is based on the representation of DNNs by DAGs. This representation is very popular among the NAS community and is used by cell-based search spaces such as NAS-Bench-101 inspired by the ResNet architecture \citep{ying2019bench}, as well as one-shot representation such as the DARTS framework (for Differentiable Architecture Search) proposed by \citet{liu2018darts}. In cell-based search spaces, DNNs are represented by repeated cells encoded as DAGs, where each node is an operation belonging to a well-defined list, typically: convolution of size 1, 3, or 5, pooling of size 3, skip connection, or zeroed operation for an image classification task for example. The graphs are then represented either as vectors using path encoding, or as adjacency matrices. In the case of path encoding, different search algorithms can be used, such as Bayesian optimization \citep{white_bananas_2020}, reinforcement learning \citep{zoph2018learning}, particle swarm optimization \citep{wang_evolving_2019}, or evolutionary algorithms \citep{xie2017genetic}, for which classical mutation and crossover operators are usually used and consist in modifying the elements of the path. Adjacency matrices, on the other hand, are more complex objects to optimize. The matrix itself represents the connections within the graph and is usually accompanied by a list representing the nodes content. In the literature, these matrices have been optimized directly with random search algorithms \citep{irwin2019graph} or indirectly with neural predictors based on auto-encoders (see for example \citet{zhang2019d} or \citet{chatzianastasis2021graph}). In the case of one-shot representations, an initial large graph containing all the considered DNN is pruned with a certain search algorithm to only keep the best possible subgraph (and thus the best possible DNN). Various search algorithms can be used to simplify this meta-graph \citep{bender2018understanding} like evolutionary algorithm \citep{guo2020single}. One of the most widely used techniques is DARTS \citep{liu2018darts}, where each edge is associated with a candidate operation, assigned to a probability of being retained in the final subgraph, optimized by gradient descent. The candidate operations can be very traditional, such as for cell-based search spaces, but \citet{chen_scale-aware_2021} proposes to use DARTS with other types of operations, such as inter-variable attention, for multivariate time series prediction. While such search spaces have proven to be efficient for tasks like image classification or language processing, \citet{white2023neural} points out that current NAS search spaces are not very expressive and prevents finding highly novel architectures. This problem is amplified when dealing with tasks for which no known architectures have yet been found.

 Compared to these search spaces, the one we define in this paper is more flexible. We address the optimization of both the architecture and the hyperparameters. We do not fix a list of possible operations with fixed hyperparameters, as is done in these works, but leave the user free to use any operation coded as \textit{PyTorch nn.Module} and to optimize any chosen parameters. Furthermore, we do not fix the generic form of our graph, we do not fix a maximum number of incoming or outgoing edges and we allow to expand or reduce the graphs. DRAGON is capable of generating innovative, original, yet well-performing DNNs. This flexibility may hinder the framework's ability to find good DNNs compared to the NAS state-of-the-art for well-known tasks such as image classification or language processing. However, in cases where DNNs have not been extensively studied and well-performing architectures have not yet been found, such as time series prediction, DRAGON may be more useful and powerful. Finally, we encode our DAGs using their adjacency matrices and provide evolutionary operators to directly modify this representation. To our knowledge, neither such a large search space nor such operators have been used in the literature.

%**************************************************
\subsection{AutoML for time series forecasting}
%**************************************************

The automated design of DNNs called Automated Deep Learning (AutoDL), belongs to a larger field \citep{hutter2019automated} called Automated Machine Learning (AutoML). AutoML aims to automatically design well-performing machine learning pipelines, for a given task. Works on model optimization for time series forecasting mainly focused on AutoML rather than AutoDL \citep{alsharef2022review}. The optimization can be performed at several levels: input features selection, extraction and engineering, model selection and hyperparameters tuning. Initial research works used to focus on one of these subproblems, while more recent works offer complete optimization pipelines.

The first subproblems, input features selection, extraction and engineering, are specific to our learning task: time series forecasting. This tedious task can significantly improve the prediction scores by giving the model relevant information about the data. Methods to select the features are among computing the importance of each feature on the results or using statistical tools on the signals to extract relevant information. Next, the model selection aims at choosing among a set of diverse machine learning models the best-performing one on a given task. Often, the models are trained separately, and the best model is chosen. In general, the selected model has many hyperparameters, such as the number of hidden layers, activation function or learning rate. Their optimization usually allows for improving the performance of the model.

Nowadays, many research works implement complete optimization pipelines combining those subproblems for time series forecasting. The Time Series Pipeline Optimization framework \citep{dahl2020tspo}, is based on an evolutionary algorithm to automatically find the right features thanks to input signal analysis, then the model and its related hyperparameters. AutoAI-TS \citep{shah2021autoai} is also a complete optimization pipeline, with model selection performed among a wide assortment of models: statistical models, machine learning, deep learning models and hybrids models. Closer to our work, the framework Auto-Pytorch-TS \citep{deng2022efficient} is specific to deep learning models optimization for time series forecasting. The framework uses Bayesian optimization with multi-fidelity optimization. Finally, a recent work from Amazon \citep{shchur2023autogluon} introduces a time series version to their AutoML framework, AutoGluon, leveraging ensembles of statistical and machine learning forecasters.

Except for AutoPytorch-TS, cited works covering the entire optimization pipeline for time series do not deepen model optimization and only perform model selection and hyperparameters optimization. However, time series data becomes more complex, and there is a growing need for more sophisticated and data-specific DNNs. Our framework DRAGON, presented in this paper, only tackles the model selection and hyperparameters optimization parts of the pipeline. We made this choice to show the effectiveness of our framework for designing better DNNs. If we had implemented feature selection, it would have been harder to determine whether the superiority of our results came from the input features pool or the model itself.
\section{Designing Voice-Accompanying Hand-to-Face Gestures}
To thoroughly understand and explore the gesture space of voice-accompanying gestures, we conducted a user-centric gesture elicitation study to elicit gesture design from end users. Subsequently, we conducted a hierarchical analysis process to narrow down the gesture space from 15 gestures to 8 gestures which are easy to perform, easy to memorize, and with less ambiguity.  \revision{The design of this study was in line with the previous gesture elicitation work \cite{10.1145/3461778.3462004,Lu-handtohand,earbuddy} consisting the typical phases of gesture proposal, gesture evaluation, and gesture set refinement.}

%Therefore, we generated empirical categories and collected users' subjective perception in different dimensions on the chosen gestures to identify a subset of the most preferable gestures. 

%我们通过brainstorming的方式先选出了一个general hand-to-mouth gesture set,之后通过经验性分类和主观测评最终从15个gesture中筛选出8个gestures。





\begin{table}
  \vspace{-0.3cm}
  \centering
  \caption{Text description and empirical categorization for all the 15 gestures in our gesture set. Each gesture was empirically categorized in three dimensions: contact position, contact type, and occlusion state. Contact position is represented in 5 levels: ear (E), mouth (M), chin(CN), cheek(CK), and none(N). Contact type is represented in 3 levels: finger(F), palm(P), and hand segments(HS). The occlusion state on the sound propagation path for the human voice to ears is represented in 3 levels: hardly(H), partially(P), and completely(C).  }
  \Description{This table contains text description and empirical categorization for all the 15 gestures in our gesture set. Each gesture was empirically categorized in three dimensions: contact position, contact type, and occlusion state. Contact position is represented in 5 levels: ear (E), mouth (M), chin(CN), cheek(CK), and none(N). Contact type is represented in 3 levels: finger(F), palm(P), and hand segments(HS). The occlusion state on the sound propagation path for the human voice to ears is represented in 3 levels: hardly(H), partially(P), and completely(C). }
  ~\label{tab:gestures}
    \vspace{-0.3cm}
    \resizebox{1\columnwidth}{!}{
    \begin{tabular}{l|c|c|c|c|c}
    \toprule
    Index & Gesture & Contact Position & Contact Type & Occlusion State & \revision{Semantics} \\
    \midrule
    1  & pinch ear rim & E &  F & H & \revision{Earphone Manipulation}\\
    \midrule
    2  & thinking face gesture & M, CN & HS & H & \revision{Thinking, Querying}\\
    \midrule
    3  & support cheek with fist  &  M, CK & P & P & \revision{Thinking, Resting}\\
    \midrule
    4 & non-contact cover mouth with palm   &  N &	P &	P & \revision{Directional Speech, Whisper}\\
    \midrule
    5 & support cheek with palm  &  M, CK & P & P & \revision{Thinking, Concentrating}\\
    \midrule
    6  & cover mouth with fist  & M & HS & C & \revision{Interphone, Messaging}\\
    \midrule
    7  & cover ear with arched palm & E & P & C & \revision{Hearing, Phone Call}\\
    \midrule
    8  & hold up palm beside nose and mouth & M, CK& P& C & \revision{Directional Speech, Block}\\
    \midrule
    9  & touch earphone with index finger &  E &  F & H & \revision{Earphone Manipulation}\\
    \midrule
    10  & touch top ear rim  &  E &  F & H & \revision{Earphone Manipulation}\\
    \midrule
    11  & touch vocal cord   &  N&F& H & \revision{Voice Distortion}\\
    \midrule
    12  & cover mouth with palm & M, CK, CN & P & C & \revision{Silence, Whisper}\\
    \midrule
    13  & shushing gesture   & M& F& H & \revision{Silence, Interruption}\\
    \midrule
    14  & touch the back of ear rim &  E &  F & H & \revision{Hearing, Attention}\\
    \midrule
    15  & calling gesture &  M, CK, CN, E& HS & P & \revision{Communication, Phone Call}\\
    \bottomrule
  \end{tabular}
  }
  \vspace{0.2cm}
\end{table}

\subsection{Voice-accompanying Hand-to-Face Gesture Proposal}
We conducted a brainstorming gesture proposal study to understand users' preference on VAHF gestures and derive a gesture set with general agreements. 

\subsubsection{Participants, Brainstorming Design, and Procedure}
%participant
We recruited 10 participants (4 female, all right-handed) from the local campus, with an average age of 21.3 (from 18 to 27, SD=2.4). Their familiarity score of wearable devices and voice interaction was 3.3\/5 (SD=1.2). The whole study took about 1 hour and each participant received 15\$ for compensation. 

%design
The purpose of this study is to encourage participants to brainstorm as many voice-accompanying hand-to-face (VAHF) gestures as they could without considering the sensing feasibility \revision{and together work out a usable VAHF gesture set with common agreements}. To achieve this point, we do not restrict the gestures to specific application scenarios or tasks, which maintained their focus on the physical nature of performing different VAHF gestures. The only constraint we imposed on the design was that the gestures should be static and durable to meet the nature of "voice-accompanying". The whole study consisted of 4 stages: 1) icebreaking and introduction, 2) individual thinking, 3) individual proposal, and 4) group discussion.

%  The only constraint we imposed on the design was that the gestures should imply para-linguistic semantics or have been used in current voice interaction.

%procedure
After a short icebreaking procedure where all the participants introduced themselves and familiarized themselves with each other, the experimenter acknowledged the participants of the purpose and the procedure for the study as well as the definition of VAHF gestures to the participants. Then the participants went through an individual thinking process for 10 minutes where participants worked separately to \revision{come} up with as many gestures as possible and wrote them down on a notebook \revision{as detailed as possible (e.g., encouraging them to write down the motivations, semantics, and potential usages of the gestures other than simply the descriptions).} After that, each participant was asked to verbalize their proposal \revision{(including the gesture descriptions, motivations, semantics, and potential usages)} and perform the proposed gestures by hand orderly. They could also sketch and show their ideas on a public whiteboard. Participants then came up with a group discussion where one could either show the pros and cons of the others' proposal or generate new gestures from the others' inspiration. The discussion ended until all participants worked out a final gesture set with the consistent agreement. \revision{The whole brainstorming process was hosted by two experimenters - one guiding the experiment while the other taking notes of the key points presented by the participants.}

% The brainstorm session, lasting for about an hour, began with a short icebreaking procedure where all the participants introduced themselves and familiarized with each other. The researcher who conducted the brainstorming session also asked about the state of familiarity of wearable devices and voice interaction to analyse whether people with different using experience could have different ideas about para-linguistic gestures. Then the researcher described the purpose and the procedure for the brainstorming session, as well as the definition of para-linguistic gestures to the participants and encouraged them to brainstorm as many para-linguistic gestures as they could, without considering the sensing feasibility; However, we do not describe the specific application scenario or tasks from which to base their ideas. This choice in procedure was to keep the participants' focus on the physical nature of performing different hand-to-face gestures. With more general guidelines, the participants could concentrate more on generating more gestures without evaluation of them at this stage. The only constraint we imposed on the design was that the gestures should imply para-linguistic semantics or have been used in current voice interaction. Then we conducted peer interviews with another set of users to further supplement the gesture set by the same brainstorming  procedure until the gestures that users can think of were already included in our gesture set.

% To record the gestures proposed, the researcher would provide the white board to participants to help them sketch and show their ideas. An individual thinking period was conducted before the group talking session first for 10 minutes where participants work separately to comp up with as many gestures as possible. Then the researcher asked participants to take turns to verbalize their thought and perform the proposed gestures by hands. After group talking session, participants were asked to spend the rest of the time working together to create new ideas, adding to those they thought of individually. The researcher presented questions to spark new lines of thought whenever the group had trouble brainstorming new ideas. We ended our brainstorming session until the gestures thay users can think of were already included in our gesture set.


%We first introduced the definition of para-linguistic gestures to the participants and encouraged them to come up with as many gestures as possible. After giving the participants ten minutes for thinking, we asked them to take turns to clarify their gestures and demonstrate gestures with their hands. We also asked the participants to verbalize their thought process during their speeches. In order to dispel the participants' worries, we did not allow evaluation of gestures at this stage. Then we conducted peer interviews with another set of users to further supplement the gesture set by brainstorming until the gestures that users can think of were already included in our gesture set. (!xxx)

% 语义 - 拟物化,用户倾向于迁移使用智能设备的经验

\subsubsection{Results and Discussion}
Fig. \ref{fig:gestures} and the "Gesture" column of Table \ref{tab:gestures} illustrated the 15 VAHF gestures and their text descriptions proposed by users in the brainstorming study. \revision{The "Semantics" column summarized some of the typical semantics of each gesture from participants' quotes. An interesting finding is that participants tended to design the gestures in a mimetic and semantic-based manner, borrowing the inspirations from their daily activities and usages of smart devices. For example, touching gestures on the ear (gestures 1, 9, and 10 in Table \ref{tab:gestures}, same as below) was regarded as the metaphor of earphone manipulations related to voice interaction (N=9), which came from their experiences of using wireless earbuds (e.g., triggering the voice assistant and controlling the volume and the progress). Participants also presented their potential usages, such as waking up Siri (P2), making a voice memo (P3), and sending a voice message to a specific person (P8). Similarly, participants described gestures 6, 7, and 15 as "the imitation of using certain devices" (N=10). \textit{"Holding up the fist in front of the mouth is a cool gesture. It is just like sending an instant command with an interphone"} (P4). \textit{"Covering the ear with the arched palm is like you are holding the phone while the 'calling' gesture is like you are imitating an old-fashioned telephone. I would prefer the former one because it is easy to perform and seems more natural to others"} (P10).}

\revision{In addition to the above gestures related to device usage, some gestures were proposed for their prevalence in daily communication and social expression. Participants (N=10) showed their will to transfer these gestures to the interaction with voice assistance. For example, the "shushing" gesture (gesture 13) and the "flaring ear" gesture (gesture 14) were proposed because they were frequently used in daily dialog. \textit{"Shushing has the meaning of silence and interruption. We can also use it to interrupt the conversation with the voice assistant" (P3).} \textit{"The 'flaring ear' gesture means 'pardon' or shows attention to the speaker. I guess it would be nice to assign this gesture to functions with similar meanings" (P1).} Gestures 4, 6, 7, and 12 were mouth-related gestures proposed by the participants, with the general meanings of special speech, lowered volume, whisper, and silence. The gestures were distinguished by different ways of covering the mouth. \textit{"When I hold up the palm on one side of the mouth, I probably want to speak to the one on the other side directionally. However, when I cover my mouth, the meaning could be totally different" (P3).} Similarly, participants designed three face-related gestures (gestures 2, 3, and 5), indicating thinking, querying, resting, or concentrating, yet with slightly different implications. \textit{"It would be wonderful the voice assistant could respond to my 'thinking face' gesture by querying my words on the searching engine"} (P2). Exceptionally, P5 proposed a "touch vocal cord" gesture (gesture 11) with a unique position. "The vocal cord affects the timbre, meaning to 'make a different sound' (P5)."}

\revision{Although the semantics of each gesture seemed clear to individuals, we found some conflicts in the group discussion stage. For example, regarding the "cover mouth with fist" gesture (gesture 6), most participants showed approval of the "interphone" metaphor while some participants (P2, P5) thought it should be with the semantics of "silence" and "secrete". Some participants also mentioned the meanings and preferences of certain gestures might vary under different cultural backgrounds, especially for the gestures with social functionalities.}







% 1	pinch the ear rim	E	F	H
% 2	thinking face gesture	M, CN	HS	H
% 3	support cheek with fist	M, CK	HS	P
% 4	non-contact cover mouth with palm	N	P	P
% 5	support cheek with palm	M, CK	P	P
% 6	cover mouth with fist	M	HS	C
% 7	cover ear with palm	E	P	C
% 8	cover nose and mouth with palm	M, CK	P	P
% 9	touch the earphone with index finger	E	F	H
% 10	touch the top ear rim	E	F	H
% 11	touch the vocal cord	N	F	H
% 12	cover mouth with palm	M, CK, CN	P	C
% 13	shushing gesture	M	F	H
% 14	touch the back of ear rim with finger tip	E	F	H
% 15	calling gesture	M, CK, CN, E	HS	P
\begin{figure*}
    \centering
    \includegraphics[width=1\textwidth]{figures/gesture_list2.png}
    \caption{Drafts illustrating each gesture in the gesture set: 1) pinch the ear rim, 2) thinking face gesture, 3) support cheek with fist, 4)non-contact cover mouth with palm, 5)support cheek with palm, 6) cover mouth with fist 7) cover ear with arched palm, 8) hold up palm beside nose and mouth,  9) touch the earphone with index finger, 10) touch the top ear rim, 11) touch the vocal cord, 12) cover mouth with palm, 13) shushing gesture, 14) touch back of ear rim with fingers, 15) calling gesture. The mean±s.d. of users’ subjective scores (1-7, the higher the better) on Usability (U), Social Acceptance (S), Fatigue (F) and Disambiguity (D) is shown on the bottom of each draft. Scores of gestures in our final gesture set are highlighted in orange. }
    \Description{This figure contains each gesture in the gesture set: 1) pinch the ear rim, 2) thinking face gesture, 3) support cheek with fist, 4)non-contact cover mouth with palm, 5)support cheek with palm, 6) cover mouth with fist 7) cover ear with arched palm, 8) hold up palm beside nose and mouth,  9) touch the earphone with index finger, 10) touch the top ear rim, 11) touch the vocal cord, 12) cover mouth with palm, 13) shushing gesture, 14) touch back of ear rim with fingers, 15) calling gesture. The mean±s.d. of users’ subjective scores (1-7, the higher, the better) on Usability (U), Social Acceptance (S), Fatigue (F) and Disambiguity (D) is shown on the bottom of each draft. Scores of gestures in our final gesture set are highlighted in orange. }
    \label{fig:gestures}
\end{figure*}


\subsection{Optimizing VAHF Gesture Set}
To derive the final user-defined gesture set from all gestures proposed by all participants, we collated the gestures and asked participants to perform all the gestures, and conducted subjective ratings from 4 dimensions. We resolved repeatability between gestures by empirical categories, which intuitively characterized the similarity between gestures from 3 dimensions. We chose one gesture from each category to a subset of the most preferable gestures.


\subsubsection{Subjective Evaluation}
After deriving a gesture set with 15 gestures, we sought to find out which gesture is most suited for voice interactions, especially in social acceptance and using fatigue. We recruited 25 participants (10 male and 15 female) for our subjective evaluation, with an average age of 21(from 19 to 32, SD = 2.1). All of the participants were right-handed. Each participant performed all gestures three times using their right hand. The order of the gestures was pre-determined to counterbalance ordering effects. For each gesture, the experimenter would show an example video of this gesture to ensure the participant could perform the gesture correctly. The participant then followed the instructions provided on a laptop screen to perform gestures. After performing the gesture three times, the participant was asked to rate the gesture according to the following four criteria along a 7-point Likert scale (1: strongly disagree to 7: strongly agree), and the results are shown in Fig. \ref{fig:gestures}:
\begin{itemize}
    \item \textbf{Usability} measured ergonomics to reflect the comfort of the gesture. The participants are required to consider the gesture not only in stationary conditions (e.g., sitting) but also under moving conditions (e.g., running). The higher the score, the easier the gesture is to perform.
    \item \textbf{Social Acceptance} measures if the user will feel uncomfortable or embarrassed, or if performing the gesture will disturb others in public settings. The higher the score, the more acceptable the gesture is in social environments.
    \item \textbf{Disambiguity} measures the difficulty of confusing the gesture with daily hand movements or with other gestures. The higher the score, the less ambiguous the gesture is.
    \item \textbf{Fatigue} measures the physiological burden to perform the gesture. The higher the score, the less fatigue the gesture is to perform.
    
\end{itemize}


\subsubsection{Design Principles and Finalized Gesture Set}
In order to eliminate the design consistency and gestures with signal similarity to derive gestures that can be naturally performed and quickly remembered, we categorized all the gestures to propose the most representative one in each category. Considering the propagation path of the human voice around the head, we \revision{identify three structural properties to represent the proposed gesture set}, which are illustrated in Table \ref{tab:gestures}:
\begin{itemize}
    \item \textbf{Contact Position:} Due to the different contact positions of the fingers, the microphone mounted on the ring can receive different sounds. Because the mouth is the source of the sound, the closer the finger touches the sound source, the louder the sound will be picked up by the microphone on the ring. In all gestures, the contact positions of the fingers are the mouth(M), the cheek(CK), the chin(CN), and the ear(E).
    \item \textbf{Contact Type:} Different contact types, specifically divided into fingers(F), palms(P), and hand segments(HS) by which hands used to contact the face region, have clear distinctions in morphology which can be distinguished easily by users without ambiguity. Furthermore, different contact types will form a unique structure on the face to affect the collection of the earphones' feed-forward microphones. 
    \item \textbf{Occlusion State:} The occlusion state, which is separated in 3 levels(hardly(H), partially(P), and Completely(C), will produce different sounds by affecting the propagation path from the human voice to the ears. For example, gesture 7 (cover ear with arched palm) and gesture 12 (cover mouth with palm) shown in Fig. \ref{fig:gestures}, which 'completely' occlude the receiver (ears) and the transmitter (mouth) of the sound the propagation path in the air, will cause a loss of high-frequency sound.
\end{itemize}

% structural property
\revision{We combined the subjective evaluation results shown on Fig. \ref{fig:gestures} (in 4 dimensions: usability, social acceptance, disambiguity, and fatigue) with the gesture set optimization process. The process was based on first grouping gestures with the same structural property combinations from the above three dimensions and chose one gesture according to the subjective scores to represent each category.} From the categorization results, we found that gestures belonging to the E-F-H category include gestures 1, 9, 10, and 14. This type of gesture with fingers to contact the ear region and have similarity in signal. Moreover, E-F-H gestures are also commonly used to interact with earphones. We chose gesture 1 to represent E-F-H gestures in our final subset of VAHF gestures according to subjective ratings; Gestures 3,5 belong to M,CK-P-P gestures. \revision{Considering the operating region of gesture 4 is also near mouth and cheek(M, CK) although it does not actually contact the face and the gesture 4's other two dimensions are the same P(Contact Type)-P(Occulasion State) with gesture 3, 5, we grouped gestures 3,4,5 into one category and chose gesture 5 to represent this category.} Gestures 11 and 13 are omitted due to their lower social acceptance \revision{(e.g., lower than 3.5)}. \revision{And the remaining gestures (2,6,7,8,12,15) are kept in the gesture set due to their specificity in the three dimensions and higher subjective scores.} 
The gesture selection process resulted in 8 gestures. We checked the subjective scores of each dimension of the eight gestures selected again and found they are all above 4.4 and have a comprehensively higher score over others, which proves that our gesture selection is subjectively reasonable and practical for users. 

The above gesture selection process filtered out the following 8 gestures: gesture 1 (E-F-H), gesture 2 (M,CN-HS-H), gesture 5 (M,CK-P-P), gesture 6 (M-HS-C), gesture 7 (E-P-C), gesture 8 (M,CK-P-C), gesture 12 (M,CK,CN-P-C), gesture 15 (M,CK,CN,E-HS-P) where the indexes were consistent with Fig. \ref{fig:gestures}. These gestures constituted our final gesture set. 


























\section{Recognizing Voice-Accompanying Hand-to-Face Gestures with Cross-Device Sensing}

In this section, we introduce the design considerations and the technical details of our cross-device sensing method to recognize VAHF gestures. We explain the implementation considerations regarding device and channel selection. Then we present individual sensing models for vocal, ultrasonic, and IMU channels. Finally, we clarify the sensor combination and fusion strategies for real-world deployment.

\subsection{Considerations and Technical Overview}

We first clarify the considerations of our implementation before going into the technical details. To recognize VAHF gestures, we chose 3 types of commercial wearable devices - wireless ANC earbuds, smartwatches, and smart rings - as the sensor nodes in consideration of real-life deployment. Each wireless ANC earbud consists of an inner microphone and an outer microphone for noise canceling. The smartwatch is equipped with a microphone and a speaker which is capable to play sounds at 22.5 KHz and the ring is equipped with a microphone and an IMU. We chose the microphones and the IMUs as the sensor candidates in consideration of the computation efficiency for the always-availability (e.g., raise-to-speak technique on an Apple watch). These sensors are widely equipped on the aforementioned commercial wearable devices (earbuds, watch, and ring).

An illustration of the entire system is shown in Figure \ref{fig:algorithm}. The sensing system consists of three independent models: vocal model, ultrasonic model, and IMU model. Each channel takes the corresponding preprocessed signal from the devices and outputs the feature vectors, which are fused and fed to the classifier layers to output the prediction logits.

% Each of the above devices is equipped with an inertial measurement unit (IMU) and at least one microphone. each earbud - 2 microphones.  The smart watch 
% is also equipped with a speaker which is capable to play sounds at 22.5 KHz(?). xxx, computational efficient, xx

\begin{figure*}
    \centering
    \includegraphics[width=0.8\textwidth]{figures/algorithms.png}
    \caption{The sensing algorithm pipeline.}
    \Description{This figure contains the sensing system that consists of three independent models: vocal model, ultrasonic model, and IMU model. Each channel takes the corresponding preprocessed signal from the devices and outputs the feature vectors, which are fused and fed to the classifier layers to output the prediction logits.}
    \label{fig:algorithm}
\end{figure*}



% 我们的目标是识别语音交互中的hand-to-face手势,为了部署的可行性(能耗、性能、形态),我们选取了现有几种普及的商用可穿戴硬件设备 - 耳机、智能手表、智能戒指 - 作为可选的传感节点,并选取设备中两种常用的低功耗(computational feasibility)传感器 - 麦克风和IMU - 进行传感。

% 系统流水线如xx图,

\subsection{Recognizing VAHF Gestures with Single-Modality Solutions}

% Vocal, Ultrasonic, IMU

To facilitate efficient recognition of VAHF gestures, we first build three individual sensing models involving three independent channels of features - vocal features, ultrasonic features, and IMU features. Each channel of features serves as individual input of recognition in different dimensions.

\subsubsection{Vocal Model}

Performing hand-to-face gestures while speaking leads to changes in the acoustic property including amplitude, frequency response, and reverberation for the received signal. For example, an "hold up the palm beside nose and mouth" gesture may impede the direct transmission of sound to the left earbud's microphone, resulting in a lower amplitude and decay in high frequency in the corresponding channel. Since we focus on the difference in the acoustic property among the distributed microphones, we first figured out the reference channel. Typically, we chose the inner microphone as the reference channel when inner and outer microphones were simultaneously used and the outer microphone of the right earbud when inner channels were disabled.

%你这段是哪里的?我去看看 对就是内外耳 forward是外 back是内

Similar to prior work, we processes the audio data for classification using mel spectrum \cite{7952132}. Given a set of audio segments from all channels $[a_1,\cdots,a_n;a_{ref}]$ with the sample rate of 16 KHz, we convert each segment into the frequency domain by first applying the short-time Fourier transform then adopting a mel-scale transform with 128 mel filterbanks, after which we pad or trunk each spectrum in the temporal axis with zeros into $128 \times 250$ ($\approx 3$ seconds) and acquire $n+1$ maps $[m_1,\cdots,m_n;m_{ref}]$. Then we subtract the reference 
map $m_{ref}$ from all the monitored map $m_{i}$ to acquire the channel-wise difference in mel spectrum $[m_1 - m_{ref}, \cdots, m_n-m_{ref};m_{ref}]$. Finally, we concatenate all the maps in the first axis into a single input frame that can be fed into a deep-learning classification model. 

A MobileNet V3 Large \cite{Howard_2019_ICCV} model pretrained on ImageNet \cite{russakovsky2015imagenet} is used as the backbone network for feature extraction. The input frame is fed to the feature extractor layers of the pretrained model to generate a 1-D feature series $f_{spec}$. Such a model is chosen in consideration of the balance between computational complexity and performance \cite{Howard_2019_ICCV}. Benefitting from the well-designed network structure and the large parameter space with good initialization, MobileNet V3 Large has the potential in capturing fine-grained textural and geometry features from the concatenated spectrum map. 

Despite the direct use of the neural network on the mel frequency map, we extracted two additional sets of statistics features - transient signal amplitude and pair-wise similarity among Mel-frequency spectrum coefficients (MFCC) series - as classifier input, which is inspired by PrivateTalk's \cite{Yan-UIST-2019} solution in dealing with channel difference and delay between audio segments. For the transient amplitude feature, we use a sliding window with the size and stride of 200 to compute the amplitude series for each segment, after which we pad or trunk each series to a fixed length of 250. Then we concatenate all the amplitude series into a 1-D feature series $f_{amp}$. For pair-wise MFCC similarity, we first compute the MFCC series for each audio channel, then resample each MFCC map in the temporal domain into 20-frame segments with a stride of 10. For each pair of segment series, we compute their similarity using dynamic time warping (DTW) \cite{berndt1994using}. We acquired the pair-wise similarity feature vector $f_{MFCC}$ by concatenating all the above ${1\over2}n(n+1)$ simularity values.

After getting $f_{spec}$, $f_{amp}$, and $f_{MFCC}$, we concatenate them into a 1-dimensional vocal feature $f_{vol}$, which can either be used in an individual recognition model or be combined with other features. For an individual recognition model, $f_{vol}$ is fed into a multi-layer perceptron (MLP) classifier to predict the performed gesture.

% from the original audio segments
% statistics: signal amplitude, pair-wise distance DTW Inspired by PrivateTalk \cite{},

\subsubsection{Ultrasonic Model}

When the user performs a hand-to-face gesture, with his hand reaching different position on the face, the relative positions among the wrist, the finger, and the ears are temporally changing, thus yielding salient positional features. To facilitate such features, we devised an embedded ultrasonic sensing component, where the speaker on the smart watch works as an active source transmitting a 17.5 KHz - 22.5 KHz linear chirp signal which is captured by the microphones on the target devices. Such a design is inspired by the theory of Frequency Modulated Continuous Wave (FMCW) \cite{mao2016cat}, which is widely used in radar and indoor positioning systems to acquire positional tracking information. The sensing principals can be formalized as a typical linear chirp based FMCW. Let $x_0(t) = A_0 cos(2 \pi f_0 t + \pi {B \over T} t^2)$ be the source signal and $x_i(t) = A_1 cos(2 \pi f_0 (t-t_0) + \pi {B \over T} (t-t_0)^2)$ be the signal received by the $i^{th}$ device, where $B=f_1 - f_0$ is the bandwidth and $T$ is the period of the chirp. We first compute the correlation $x_0(t) x_i(t)$ and then pass the result to a low-pass filter to acquire the low-frequency component:

\begin{equation}
\begin{aligned}
    LPF(x_0(t)x_i(t)) = {1\over2} A_0 A_1 cos(2 \pi f_0 t_0 + \pi {B \over T} (2t_0 t - t_0^2))
\end{aligned}
\end{equation}

Note that $LPF(x_0(t)x_i(t))$ is a cosine function form of $t$ with an amplitude of ${1\over2} A_0 A_1$ and a frequency of $2{B \over T}t_0$, where the amplitude indicates the decay of the signal transmission and the frequency is proportional to the delay $t_0$ of the received signal. So we first compute the spectrum of $LPF(x_0(t)x_i(t))$ using short time Fourier transform (STFT). We extracted the following two features based on the spectrum: 1) the image feature $f_{spec}$ of the spectrum using a pretrained MobileNet V3 network and 2) the amplitude and frequency series $f_{stats}$ (which is flattened into a 1-D vector) derived from the spectrum. Finally, we concatenate $f_{spec}$ and $f_{stats}$ into $f_{ultra}$, which can be either fed into a downstream classifier or combined with other features as mentioned above.

\subsubsection{IMU Model}

Devices worn on the user's hand, such as a watch and ring, help to capture the movement and attitude of the user's moving hand, thus beneficial for recognizing hand-to-mouth gestures. In our setting, we choose to mount a 9-axis wireless IMU on the ring as previous work \cite{10.1145/3478114} did. The IMU reports 3-axis acceleration, 3-axis angular velocity, and the quaternion at 200 Hz. For gesture recognition, we adopted a fixed window of 400 frames (or 2 seconds), concatenating the acceleration, angular velocity, and quaternion series into a 4000-length vector. Then we used a 3-layer MLP with the structure of $Dropout(0.5) \rightarrow Linear(4000,512) \rightarrow ReLU \rightarrow Linear(512,512) \rightarrow ReLU \rightarrow Linear(512,9)$
for classification. 

\subsection{Sensor Combination and Fusion Strategies}

In consideration of the real-world deployment, we first figure out the reasonable device and sensor combinations. The devices include 1) left earbud (LE) with inner and outer microphones ($m_{l,i}$, $m_{l,o}$), 2) right earbud (RE) with inner and outer microphones ($m_{r,i}$, $m_{r,o}$), 3) watch with a microphone ($m_w$), and 4) ring with a microphone ($m_r$) and an IMU. Considering earbuds are most commonly used, we chose them as the primary device, which would work in different forms including two-side, one-side (wearing one earbud), and outer-only (for ones without active noise canceling). The introduction of the watch could be beneficial in providing an active ultrasound source as well as a hand-mounted microphone. Last, a ring device with an IMU and a microphone could track the movement of the hand and finger as well as provide a finger-mounted microphone. Based on the above observation, we devised four typical settings as follows for investigation: 1) single earbud (RE); 2) two earbuds (LE+RE); 3) two earbuds + watch (LE+RE+W); and 4) all devices (LE+RE+W+R).

%$m_{l,o}$, $m_{r,o}$, $m_w$, $m_r$

For settings 3) and 4), since the active ultrasound source and the IMU enable the ultrasonic model and the IMU model, a fusion method is required to fuse different recognition models from different channels. We investigated two fusion strategies: 1) logit-level fusion and 2) feature-level fusion. 

Let $F_{v}$, $F_{u}$, and $F_{i}$ be the feature extractor network of vocal, ultrasonic, and IMU models respectively and $C_{v}$, $C_{u}$, and $C_{i}$ be the corresponding multilayer classifier that outputs the logits. For logit-level fusion, the output logits are computed as 
\begin{equation}
    logits = a\cdot C_v(F_v(x_v)) + b\cdot C_u(F_u(x_u)) + c\cdot C_i(F_i(x_i))
\end{equation}
where $a$, $b$, and $c$ are learnable weight parameters ($x_v$, $x_u$, and $x_i$ are the corresponding channels of input). For feature-level fusion, the output logits are computed as 
\begin{equation}
    logits = C_{fuse}([F_v(x_v), F_u(x_u), F_i(x_i)])
\end{equation}
, where [*, *, *] refers to concatenation and $C_{fuse}$ is another MLP classifier that takes the concatenated features as input and outputs the logits.



% two fusion strategies:

% logit-level fusion

% feature-level fusion

% 4 different fusion strategies:

% 1) voting; 2) hierarchical; 3) feature-level fusion; 4) logit-level fusion; 5) + pretrained 

% \subsection{Towards Multi-Objective Optimization for xxxx}

% different targets:

% 1) optimizing recognition method
% 2) optimizing gesture set
% 3) optimizing form factor

%input/output
%不同的channels 
%device groups
%model



\begin{figure*}[htb]
  \centering
  \includegraphics[width=0.99\textwidth]{bars.pdf}
  \caption{Results of ~\framework{} with different encoding tree heights.}
  \label{fig:K-bar}
  \Description{Results with different encoding tree heights.}
\end{figure*}



\section{Results and Analysis}\label{sec:evalu}
In this section, we demonstrate the efficacy of \ \framework{} on semi-supervised node classification (\S ~\ref{sec:exp:overall}, followed by micro-benchmarks that investigate the detailed effect of the submodules on the overall performance and validate the robustness of ~\framework{} when tackling random perturbations (\S ~\ref{sec:exp:micro}). For better interpretation, we visualize the change of structural entropy and graph topology (\S ~\ref{sec:exp:int}).


\subsection{Node Classification}
\label{sec:exp:overall}



\subsubsection{Comparison with baselines}
We compare the node classification performance of ~\framework~ with ten baseline methods on nine benchmark datasets.
Table ~\ref{tab:performance comparison} shows the average accuracy and the standard deviation. 
Note that the results of H$_2$GCN (except PT and TW) and Geom-GCN are from the reported value in original papers ( - for not reported), while the rest are obtained based on the execution of the source code provided by their authors under our experimental settings. Our observations are three-fold: 
% \textbf{(1)} While all GNN methods can achieve satisfactory results on citation networks, graph structural learning frameworks perform significantly better than conventional GNN methods on WebKB, Wiki, and actor co-occurrence networks due to the high heterophily of these networks.
% The reason for this phenomenon is these conventional GNN models agggregate to much classification-irrelevant information from disassortative neighborhoods. 
% In contrast, graph structural learning can optimize the neighborhood topology to achieve better results.

% \textbf{(1)} While all GNN methods can achieve satisfactory results on citation networks, the specialized graph learning frameworks perform significantly better on WebKB, Wiki and actor co-occurence networks due to the heterophily challenge. 


\noindent \textbf{(1)} \framework~ achieves optimal results on 5 datasets, runner-up results on 8 datasets, and advanced results on all datasets. The accuracy can be improved up to 3.41\% on Pubmed, 3.00\% on Cora, and 2.92\% on Citeseer compared to the baselines. This indicates that our design can effectively capture the inherent and deep structure of the graph and hence the classification improvement. 


\noindent \textbf{(2)} \framework~ shows significant improvement on the datasets with heteropily graphs, e.g., up to 37.97\% and 27.13\% improvement against Wisconsin and Texas datasets, respectively. This demonstrates the importance of the graph structure enhancement that can contribute to a more informative and robust node representation.


\noindent \textbf{(3)} While all GNN methods can achieve satisfactory results on citation networks, the graph learning/high-order neighborhood awareness frameworks substantially outperform others on the WebKB datasets and the actor co-occurrence networks, which is highly disassortative. This is because these methods optimize local neighborhoods for better information aggregation. Our method is one of the top performers among them due to the explicit exploitation of the global structure information in the graph hierarchical semantics.




\subsubsection{Comparison base on different backbones}
Table~\ref{tab:backbone comparison} shows the mean classification accuracy of ~\framework{} with different backbone encoders.
Observably, ~\framework{} upon GCN and GAT overwhelmingly outperforms its backbone model, with an accuracy improvement of up to 31.04\% and 27.48\%, respectively. 
This indicates the iterative mechanism in the ~\framework{} pipeline can alternately optimize the node representation and graph structure.
We also notice that despite the lower improvement, ~\framework{} variants based on GraphSAGE and APPNP perform relatively better compared to those on GCN and GAT.
This is most likely due to the backbone model itself being more adapted to handle disassortative settings on graphs.
% We also notice that ~\framework~ based on GraphSAGE has the lowest improvement. This is most likely due to the weak adaptability of the backbone model itself to disassortative settings.

\begin{table}[t]
    \renewcommand{\arraystretch}{1.05}
    \setlength{\abovecaptionskip}{0.15cm}
    \setlength{\belowcaptionskip}{-0.25cm}
    \caption{Classification accuracy(\%) of ~\framework{} and corresponding backbones. Wisc. is short for Wisconsin.}%mean relative 
    \label{tab:backbone comparison}
    \centering
    % \scalebox{0.9}{
    % \setlength{\tabcolsep}{1mm}{
        \begin{tabular}{l|ccccc}
        \hline
        Method & Actor & TW & Texas & Wisc. & Improvement\\
        \hline
        \framework$_{GCN}$   & 35.03 & 66.88 & 75.68 & 79.61 & $\uparrow$ 5.20$\sim$31.04\\
        % Chebnet         &
        % \framework(Chebnet)
        % \midrule
        % SGC             &
        % \framework(SGC)
        % \framework
        % (SAGE)            & 36.34 & 66.92 & \textbf{81.62} & \textbf{86.27} & $\uparrow$ 0.25$\sim$3.79\\
          \framework$_{SAGE}$& 36.20 & 66.92 & \textbf{82.49} & \textbf{86.27} & $\uparrow$ 0.25$\sim$6.79\\
        \framework$_{GAT}$   & 32.46  & 63.57 & 74.59 & 78.82 & $\uparrow$ 4.69$\sim$27.48\\
        % \framework(APPNP) & \textbf{36.62} & 71.45 & 81.28 & 83.14 & 13.34\%\\
        \framework$_{APPNP}$ & \textbf{36.34} & \textbf{66.99} & 81.28 & 83.14 & $\uparrow$ 2.01$\sim$12.16\\
        \hline
        \end{tabular}
    % }
    % }
\end{table}


\begin{table}[t]
    % \renewcommand{\arraystretch}{0.75}
    \setlength{\abovecaptionskip}{0.15cm}
    \setlength{\belowcaptionskip}{-0.25cm}
    \caption{The $k$ selection for each iteration in structural optimization. Bolds represent the $k$ selection when the accuracy reaches maximum.}\label{tab:k-NN comparison}
    \centering
    % \scalebox{0.9}{
    \setlength{\tabcolsep}{2mm}{
        \begin{tabular}{l|ccccccccc}
        \hline
        Iteration & 1 & 2 & 3 & 4 & 5 & 6 & 7 & 8 & 9\\
        \hline
        {Cora}   & 22 & \bf{22} & 19 & 22 & 21 & 22 & 20 & 21 & 20\\
        {Actor}  & 23   & 15 & 15 & 15 & 14 & 15 & 14 & \bf{14} & 15\\
        {TW} & 50 & 16 & 16 & \bf{17} & 15 & 17 & 27 & 16 & 16 \\
        {Wisconsin} & 21 & 16 & \bf{11} & 16 & 14 & 13 & 16 & 13 & 11 \\
        {Texas}  & 21 & 13 & 13 & \bf{13} & 13 & 10 & 14 & 10 & 14 \\
        \hline
        \end{tabular}
    }
\end{table}

\subsection{Micro-benchmarking}
\label{sec:exp:micro}

\subsubsection{Effectiveness of~$k$-selector}
This subsection evaluate how the one-dimensional structural entropy guides the $k$-selector in \S ~\ref{step1}.
Table ~\ref{tab:k-NN comparison} showcases the selected parameter $k$ in each iteration with \framework$_{GCN}$. 
Noticeably, as the iterative optimization proceeds, the optimal parameter $k$ converges to a certain range, indicating the gradual stabilization of the graph structure and node representation. The disparity of parameter $k$ among different datasets also demonstrates the necessity of customizing $k$ in different cases rather than using $k$ as a static hyperparameter.




\subsubsection{Impact of the encoding tree's height $K$}
We evaluate all four variants of ~\framework~ on the website network datasets, and the encoding tree height $K$ involved in \S ~\ref{step2} varies from 2 to 4.
As shown in Fig. ~\ref{fig:K-bar}, there is a huge variation in the optimal tree heights among different datasets. For example, in the variants based on GAT, GCN, and APPNP, the best results can be targeted at $K=3$ in Texas and at $K=4$ in Cornell and Wisconsin. By contrast, in ~\framework$_{SAGE}$,  $K=2$ can enable the best accuracy of 86.27\%. This weak correlation between the best $K$ and the model performance is worth investigating further, which will be left as future work. 


\begin{figure}[tb]
  \centering
  \includegraphics[width=0.48\textwidth]{ptb_exp.pdf}
  \caption{Robustness of ~\framework~ against random noises.}
  \label{fig:pertubation}
  \Description{Results of perturbation experiment.}
\end{figure}


\begin{figure*}[tb]
  \centering
  \includegraphics[width=0.99\textwidth]{sedecrease.pdf}
  \caption{The normalized structural entropy changes during the training of ~\framework$_{GAT}$ with 2-dimensional structural entropy on (a) Texas, (b) Cornell, and (c) Wisconsin. The structure is iterated every 200 epochs. By comparison, (d) shows the entropy changes on Wisconsin without the graph reconstruction strategy.}
  \label{fig:sedecrease}
  \Description{Visualization of structural entropy and acc. variation.}
\end{figure*}


\begin{figure}[tb]
  \centering
  \includegraphics[width=0.49\textwidth]{topovisual.pdf}
  \caption {The visualized evolution of the graph structure on Cora (a,b,c) and Citeseer (d,e,f). The corresponding Structural Entropy (SE) is also shown.}
  \label{fig:topovisual}  
  \Description{Visualization of topology evolution.}
\end{figure}



\subsubsection{Sensitivity to perturbations}
We introduce random edge noises into Cora and Citeseer, with perturbation rates of 0.2, 0.4, 0.6, 0.8, and 1. As shown in Fig.~\ref{fig:pertubation}(a), ~\framework{} outperforms baselines in both GCN and GAT cases under most noise settings. For instance, ~\framework$_{GCN}$ achieves up to 8.09\% improvement against the native GCN when the perturbation rate is 0.8; by contrast, improvements by GCN-Jaccard and GCN-DropEdge are merely 6.99\% and 5.77\%, respectively. A similar phenomenon is observed for most cases in the Citeseer dataset (Fig.~\ref{fig:pertubation}(b)), despite an exception when compared against GCN-Jaccard. Nevertheless, our approach is still competitive and even better than GCN-Jaccard at a high perturbation rate. 

\subsection{Interpretation of Structure Evolution}
\label{sec:exp:int}



\subsubsection{Structural entropy variations analysis}
We evaluate how the structural entropy changes during the training of ~\framework$_{GAT}$ with 2-dimensional structural entropy on WebKB datasets. For comparison, we visualize the entropy changes on Wisconsin without the structure learning. In the experiment setting, both the graph structure and the encoding tree are updated once at each iteration (i.e., 200 GNN epochs), and within one iteration, the structural entropy is only affected by edge weights determined by the similarity matrix. For comparison, we normalize the structural entropy by $ \textstyle{\frac{H^{\mathcal{T}}(G)}{H^1(G)}}$.

As shown in Fig.~\ref{fig:sedecrease}(a)-(c), as the accuracy goes up, the normalized structural entropy constantly decreases during the iterative graph reconstruction, reaching the minimums of 0.7408 in Texas, 0.7245 in Cornell, and 0.7344 in Wisconsin. This means the increasing determinism of the overall graph structure and the reduced amount of information required to determine a vertex. 
Interestingly, if our graph reconstruction mechanism is disabled (as shown in Fig.~\ref{fig:sedecrease}(d)), the normalized structural entropy keeps rising from 0.7878, compared with Fig.~\ref{fig:sedecrease}(c). Accordingly, the final accuracy will even converge to 55.34\%, a much lower level. 

Such a comparison also provides a feasible explanation for the rising trend of the normalized structural entropy within every single iteration. 
This stems from the smoothing effect during the GNN training. 
As the node representation tends to be homogenized, the graph structure will be gradually smoothed, leading to a decrease in the one-dimensional structural entropy thus the normalized structural entropy increases.
% We speculate that the continuous rise in structural entropy may be a feasible explanation for the over-smoothing of graph neural networks, which requires further study.
% In summary, this experiment well explains the robustness of our framework.


\subsubsection{Visualization}
Fig.~\ref{fig:topovisual} visualizes the topology of the original Cora and Citeseer graph and of the 2nd and 5th iterations.
The vertex color indicates the class it belongs to, and the layout denotes connecting relations. Edges are hidden for clarity. As the iteration continues, much clearer clustering manifests -- few outliers and more concentrated clusters.  
Vertices with the same label are more tightly connected due to the iterative graph reconstruction scheme. This improvement hugely facilitates the interpretability of the GSL and the node representation models. 

\section{Application Scenarios}
To demonstrate the applicability of VAHF gestures in voice interaction, we first presented the interaction space created by VAHF gestures along with example real-life scenarios. Then we discussed the design considerations and implications regarding the deployment of VAHF gestures in real practice.

% Then we conducted an informal study to understand users' preference and comments on XXX. Finally,

\subsection{Interaction Space and Scenario Description}
The introduction of VAHF gestures achieves the unique benefit of assigning a multi-class label to speech segments, which brings great potential to broaden the traditional voice interaction space in the following aspects.

\subsubsection{VAHF Gestures as Modality Control Signals.}

\textbf{Wakeup-free interaction.} The most intuitive function for modality control in voice interface is to use hand-to-face gestures (e.g., covering the mouth) to indicate whether the current speech is with interaction intention that should be processed by the voice assistant, which has been achieved and widely researched by previous work \cite{10.1145/3411764.3445687,Yan-UIST-2019,10.1145/3351276}. In our work, VAHF gestures have the inherited capability to support wakeup-free interaction simply by assigning one of the gestures for the wakeup state control.

\textbf{Dynamic modality control in multi-round interaction with voice assistant.} We demonstrate an example scenario using VAHF gestures for dynamic modality control in the multi-round dialog that has never been achieved before. When the user is enrolled in a multi-round dialog with the voice assistant, the complexity of the interaction behavior increases significantly. For example, in a specific dialog round, the user has different options to proceed with the dialog: 1) appending - the user appends a voice command and expects the voice assistant to process the command based on the dialog context in the regular order; 2) interrupting - the user wants to interrupt the current dialog (e.g., the voice assistant stops immediately and waits for new voice commands) and start a new dialog (abandoning the dialog context) with the new commands; and 3) editing - the user wants the voice assistant to edit the commands that they previously asked based on the dialog context and the brief editing command (e.g., the user says "How is the weather today?" When the assistant is answering, the user adds an editing command "No, I mean tomorrow."). Since our technique enables a channel width of up to 9 gestures (including the empty gesture) as modality input, we can assign different VAHF gestures to the three modalities of voice input - appending (e.g., covering the mouth), interrupting (e.g., covering the earphone), and editing (e.g., holding up the palm beside the mouth) - in the multi-round dialog scenario to enable more flexible and intelligent voice interaction flow. 

\subsubsection{Binding Shortcuts to VAHF Gestures}

\textbf{VAHF Gestures as UI shortcuts.} Simulating the execution of certain interaction paths through voice commands is a prevalent form of voice interaction on smartphones and wearable devices. When an interaction path takes a text entry slot or a period of raw speech as the input, it can be replaced with certain VAHF gestures. For example, the user can define the "phone call" VAHF gesture as opening WeChat and sending a voice message of the user's raw speech to Alice. Another example is to define the "thinking face" gesture as opening the Google website and searching for the text transcribed from the raw speech input. Such replacements of complex UI shortcuts with VAHF gestures could potentially reduce the repetition of the interaction path in speech, especially in a multi-round interaction.

\textbf{Registration and reservation of the VAHF-gesture-enabled shortcut session.} Regarding the binding of shortcuts with VAHF gestures, a more exciting design question is how the VAHF gestures are binded in real-world practice. Normally, the binding is fixed and can be set by the GUI (e.g., on a smartphone). On the contrary, we here present a dynamic registration and reservation mechanism for VAHF-gesture-enabled shortcut sessions, which are worthy of extensive exploration. In such a mechanism, for an unbinded VAHF gesture, when the user performs the gesture while narrating the full voice command, the voice assistant would automatically extract the UI shortcut from the command and bind it with the performed gesture. Later when the user wants to access the shortcut for a second time, they could perform the binded gesture while saying the input slot instead of the full command. The session and the dialog context are fully preserved for the gesture until a new command with UI shortcut semantics is input. The voice assistant would ask the user whether to update the binding of the gesture to a new shortcut. We believed such a design of a dynamic registration mechanism for VAHF gestures would benefit memorability, flexibility, and lower cold-start cost.

% that is worthy of extensive exploration The most interesting design is that the XXX can be dynamically registered and preserved 
%  (e.g., "opening Wechat and sending an emoji to Alice")

\subsubsection{VAHF Gestures as Spatial Indicators} VAHF gestures in voice interaction are also capable of indicating the target to interact with from the multiple interactable devices or elements. For example, in an IoT scenario where multiple voice-interactable devices (e.g., a smartphone, a TV, and a smart speaker) are in the same room, the user could perform different VAHF gestures with voice commands to trigger voice interaction with different devices. Similarly, in a complex UI control scenario (e.g., filling in a form with multiple text boxes), a VAHF gesture is displayed beside each text box, and the user could perform the corresponding VAHF gesture to input a particular text box.





 

\subsection{Design Considerations for VAHF Gestures to Enhance Voice Interaction}

The VAHF gestures proposed in our paper open the opportunity to design novel voice interactions for mobile, wearable, and HMD devices that allow users to quickly switch among modalities, accelerate common tasks, and manage multi-device interaction in different scenarios. We discussed two issues regarding the real-world deployment of VAHF gestures. \textbf{1) Combination strategy for better performance.} Although VAHF gestures have shown great potential in applicability, simply adding on all the functions described in the previous section is not feasible due to the channel capacity and the recognition accuracy. For example, as shown in Tables 3 and 4, an accuracy of 91.5\% for 9 classes is not yet highly usable, but a 4-class sub-gesture set achieved an accuracy of 97.3\%, which is considered highly usable. Therefore, a fine-grained design on the selection of gestures (e.g., alleviating using two gestures with higher confusion at the same time) and the switch of gesture sets in different scenarios is key to implementing a highly usable VAHF-gesture-enhanced voice interaction system. \textbf{2) Scalability and extensibility.} Although we only investigated an optimized VAHF gesture set with 8 gestures in our work, our sensing method was open to absorbing other extensive VAHF gestures. Our analysis method in Sections 3.1 and 3.2 provide a practical design guideline to elicit new gestures and analyze their feasibility. Further, our framework of recognizing VAHF gestures by multiple wearable devices has the advantage of appending or cutting down certain sensing channels easily, so the gesture set should be scalable and convertible for the system's flexibility. 












% In this section, we first discuss two example real-life scenarios to demonstrate the applicability of PLHF gestures. Then we present a general design guideline for xxx.

% \subsection{Scenario Description}

% \subsubsection{}

% \subsubsection{}


% \subsection{Design Guideline} 
% increment of gesture or devices




% % 几种模式 
% 1. modality control, binding shortcuts
% 2. passing parameters, e.g., 
% 3. complex control, e.g., 

\section{Limitations and Future Work}
\label{sec:limitations}
The failure analysis conducted over the planning methods in Fig.~\ref{fig:failure} highlights several limitations of our language planning framework.
On Tasks 1-3 (\LH{}, \LG{}), Text2Motion incurs a larger planning and execution failure rate than \hm{} ($\sim$10\%) as a consequence of committing to a single skill sequence in the integrated search algorithm. 
In addition, we observed an undesirable pattern emerge in the planning phase of Text2Motion and the execution phase of \scgs{} and \imgs{}, where \textit{recency bias}~\cite{zhao2021calibrate} in the LLM would induce a cyclic state of repeating feasible skill sequences.

% In Fig.~\ref{fig:failure}, we analyse the several failure modes of Text2Motion: 
% (1) Goal proposition prediction failure (we did not observe this failure mode in our evaluation task suite), 
% (2) No successful plan found,
% (3) Plan found but execution error.

% Aside from the failure modes listed above, we also observed another failure mode that occurred in the planning phase of Text2Motion and execution phase of \scgs{} and \imgs{}: LLM would sometimes enter a \textit{cycle}, repeating sequences of geometrically feasible actions.

As with other methods that use skill libraries, Text2Motion is highly reliant on the fidelity of the low-level skill policies, their value functions, and the ability to accurately predicted future states with learned dynamics models.
This presents non-trivial challenges in scaling Text2Motion to high-dimensional observation spaces, such as images or pointclouds.

% Similar to approaches relying on skill libraries, the overall performance of Text2Motion is highly dependent on the quality of the low-level skill policies, their Q functions and the dynamics models. 
% Additionally, Text2Motion must contend with potentially erroneous latent state representations predicted by a learned dynamics model.

Text2Motion is mainly comprised of learned components, which comes with its associated efficiency benefits.
Nonetheless, runtime complexity was not a core focus of this work, and expensive optimization subroutines~\cite{taps-2022} were frequently called.
LLMs also impose an inference bottleneck as each API query (Sec. \ref{subsec:llm}) requires 2-10 seconds, but we anticipate improvements with advances in LLM inference techniques.

% In this work, we do not focus on runtime optimization and thus run policy sequence optimization from scratch at each planning iteration for Text2Motion and \hm{}.
% We further note that our overall inference time is also bottle-necked by that of the LLM. In this case, each query to the LLM API (see Sec. \ref{subsec:llm}) used in this work can take anywhere between two to ten seconds).
% That said, we expect query times to decrease as LLM inference techniques improve.

% We thus outline several avenues for future work:
% (1) Maintain a more diverse set of task plans through beam search with beam size greater than one. Doing so would allow Text2Motion to handle longer horizon robot manipulation tasks where it helps to maintain diverse plans (along with the proposed iterative integrated planning method). We empirically found that the LLMs we used were not suited to guide search for large beam sizes as the scores generated across skill \textit{sequences} were not calibrated relative to each other. 
% However, as their capabilities improve, LLMs should eventually be able to accurately rank and score skill sequences.
% (2) Decrease inference time by caching policy sequence optimization distributions from earlier planning timesteps.

We outline several avenues for future work based on these observations.
First, we aim to explore algorithmic design choices that unify the strengths of integrated (feasibility) and hierarchical (diversity) planning.
Doing so would enable Text2Motion to solve longer horizon robot manipulation tasks where it helps to maintain diverse plans.
Such strategies may require LLMs to produce calibrated scores of entire skill sequences relative to each other, which we empirically found challenging and warrants further investigation.
Second, there remains an opportunity to increase the plan-time efficiency of our method, for instance, by warm starting policy sequence optimization with distributions cached in earlier planning iterations.
Lastly, we hope to explore the use of multi-modal foundation models~\cite{driess2023palm, openai2023gpt4} that are visually grounded, and as a result, may support scaling our planning system to higher dimensional observation spaces.
Progress on each of these fronts would constitute steps in the direction of reliable and real-time language planning capabilities.


% \section{Discussion}
% \label{sec:discussion}

% \klin{need to format}
% \begin{enumerate}
    % \item One class of failure modes that we observed involved the LLM deciding to repeating certain actions it had previous seen. For example, one of the failure modes in figure \ref{fig:failure} (LH LG PO) requires the robot to use the hook to pick and place two boxes onto the rack, then use the hook to pull an object to be inside the workspace and then pick and place that object onto the rack. However, if the task sequence planed so far has a small deviation from a more `optimal' plan, the LLM tends to place a higher score on actions it had previously executed. ['pick(red-box)', 'place(red-box, rack)', 'pick(cyan-box)', 'place(cyan-box, rack)', 'pick(hook)', 'place(hook, table)', 'pick(hook)', 'pull(yellow-box, hook)', 'place(hook, table)', 'pick(red-box)'].
    % \item LLM goal prediction can sometimes be wrong - if being used as chain of thought, especially for code-davinci, need to specify ``\textit{Predicted} goal predicate set'' rather than ``Goal predicate set''. Text davinci models don't get affected by wording as much.
% \end{enumerate}

\section{Conclusion}

In this paper, we investigated the design space and the recognition method of voice-accompanying hand-to-face (VAHF) gestures to enhance voice interaction with  parallel gesture channels. To design VAHF gestures, we first conducted an elicitation study, resulting in a total proposal of 15 gestures, followed by a hierarchical analysis process to output the most salient 8 gestures with the least ambiguity and physical confusion. Then we proposed a novel cross-device sensing method fusing different sensor channels to recognize para-linguistic hand-to-face gestures, achieving a high recognition accuracy of 97.3\% for 3+1\revision{(empty)} gestures and 91.5\% for 8+1\revision{(empty)} gestures recognition on our cross-device VAHF dataset. The uniqueness of our work is that we explored a broadened and scalable VAHF-gesture-based interaction space, which remains under-researched, to facilitate voice interaction in a more diverse manner (e.g., defining a shortcut or parsing parameters). Compared with prior work \cite{10.1145/3411764.3445687,Yan-UIST-2019} where a specific gesture (e.g., bringing the phone to the mouth\cite{10.1145/3411764.3445687}) was designed and recognized for 1-bit modality control (e.g., activating the voice assistant), our multi-device sensing framework is not only capable for recognizing up to 8 VAHF gestures simultaneously \revision{from the hand-off "empty" gesture}, but also benefits from the scalability (e.g., adding a device or adding a gesture is easy under our framework). As mobile devices and scenarios are becoming prevalent these years, voice input has become an essential modality of pervasive interaction. We envision our work would further enhance the efficiency and capability of current voice interaction and serve an important role in the future voice interaction of various scenarios like AR and IoT.






%% why not .tex




%%
%% The acknowledgments section is defined using the "acks" environment
%% (and NOT an unnumbered section). This ensures the proper
%% identification of the section in the article metadata, and the
%% consistent spelling of the heading.
\begin{acks}
This work is supported by the Natural Science Foundation of China (NSFC) under Grant No. 62132010 and No. 62002198, Tsinghua University Initiative Scientific Research Program, Beijing Key Lab of Networked Multimedia, and Institute for Artificial Intelligence, Tsinghua University.
\end{acks}

\balance{}

%%
%% The next two lines define the bibliography style to be used, and
%% the bibliography file.
\bibliographystyle{ACM-Reference-Format}
\bibliography{sample-base}

%%
%% If your work has an appendix, this is the place to put it.
\appendix

\end{document}
\endinput
%%
%% End of file `sample-authordraft.tex'.
