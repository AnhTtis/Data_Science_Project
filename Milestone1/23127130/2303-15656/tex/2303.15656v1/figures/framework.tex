
\begin{figure*}[t]
    \centering
	\begin{minipage}[t]{0.39\linewidth}
    \includegraphics[width=1.0\textwidth]{images/stl.pdf}
    \subcaption{Single-task learning baseline}
    \label{fig:single}
    \end{minipage}
    \hfill
    \begin{minipage}[t]{0.58\linewidth}
    \hfill
    \includegraphics[width=1.0\textwidth]{images/mtl.pdf}
    \subcaption{Multi-task learning framework}
    \label{fig:multi}
    \end{minipage}
   % \vspace{0.1in}
    \caption{Comparison between (a)  single-task learning baseline and (b) multi-task learning framework. In the multi-task learning framework, input features $\{x^{(d)}\}_{d=1}^{D}$ are first processed by shared hidden layers to produce a latent feature representation. This latent representation is then passed through task-specific hidden layers to generate multiple predictions $\{\hat{\by}^{(j)}\}_{j=1}^{M}$.}
     \vspace{-4mm}
\end{figure*}