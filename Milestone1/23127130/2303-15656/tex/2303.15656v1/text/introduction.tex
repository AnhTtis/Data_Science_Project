\section{Introduction}
\label{sec:intro}

% Background
The neonatal period, encompassing the initial four weeks of an infant's life, marks a crucial stage of development characterized by significant physiological changes and high vulnerability that leads to an elevated neonatal mortality rate~\cite{karimi2019mortality}. Around two-thirds of infant deaths occur during this period~\cite{daemi2019risk}. Predictions indicate that approximately 26 million newborns will die between 2019 and 2030~\cite{rezaeian2020prediction}, making the issue of high neonatal mortality a global concern. Despite the remarkable advancements in neonatal care, infants born before 37 weeks of pregnancy still face a substantial risk of mortality~\cite{richter2019temporal}. In 2021, one-tenth of babies were born too early in the United States.\footnote{\scriptsize\url{https://www.cdc.gov/reproductivehealth/features/premature-birth}} As a result, it is imperative that we prioritize our attention towards Extremely Low Gestational Age Newborns (ELGANs, i.e., less than 28 weeks of gestation) rather than newborns in general.

%% the significance of neonatal death prediction
\textit{Adverse neonatal outcomes} play a significant role in neonatal survival~\cite{workineh2022adverse}. Early diagnosis of adverse neonatal outcomes enables physicians to prepare an early treatment. It also allows researchers to identify the risk factors that lead to a high neonatal mortality rate~\cite{houweling2019prediction}.

% figure 1
\begin{figure*}[t]
    \centering
	\begin{minipage}[t]{0.32\linewidth}
    \includegraphics[width=1.0\textwidth]{images/bpd_stat.pdf}
    \subcaption{Histogram of BPD}
    \label{fig:hist_bpd}
    \end{minipage}
    \hfill
    \begin{minipage}[t]{0.328\linewidth}
    \includegraphics[width=1.0\textwidth]{images/phd_stat.pdf}
    \subcaption{Histogram of PH diagnosis}
    \label{fig:hist_phd}
    \end{minipage}
    \hfill
    \begin{minipage}[t]{0.32\linewidth}
    \includegraphics[width=1.0\textwidth]{images/dw_stat.pdf}
    \subcaption{Histogram of discharge weight z-score}
    \label{fig:hist_dw}
    \end{minipage}
    \caption{Histograms of three adverse neonatal outcomes, including (a) BPD, (b) PH diagnosis, and (c) discharge weight z-score.}\label{fig:hist_total}
\end{figure*}

%% why ML works & related works & problem of single-task
Machine learning (ML) is a branch of artificial intelligence. Given a model and a set of labeled/unlabeled data, ML methods learn to recognize informative patterns from data without explicit programming~\cite{jordan2015machine}. The ML model can make reasonable predictions on unseen test data based on these learned patterns~\cite{al2019survey}. A series of innovations~\cite{mboya2020prediction,sheikhtaheri2021prediction,nguyen2021whom,hsu2021machine,mangold2021machine,turova2020machine,hsu2021machine2} have been proposed to predict adverse neonatal outcomes using machine learning techniques. However, most previous works consider the prediction of various neonatal outcomes as multiple \textit{single-task} learning problems. These methods \textbf{ignore the correlation among different outcomes}, which leads to \textbf{inefficient usage of data}~\cite{standley2020tasks}. Moreover, single-task learning is prone to \textbf{overfitting}~\cite{ruder2017overview,zhang2021survey}, especially when the amount of annotated data is limited, which is often the case with many medical applications. Therefore, this work aims to \textbf{diagnose multiple neonatal outcomes jointly  to improve data efficiency} by leveraging the correlations among these outcomes.


%% data analysis
In this work,  we first collect a dataset including 121 preterm neonates from two medical centers and focus on three adverse neonatal outcomes, including severe bronchopulmonary dysplasia (BPD), pulmonary hypertension (PH) diagnosis, and discharge weight. In particular, BPD~\cite{jobe2001bronchopulmonary} is a form of chronic lung disease, which most often occurs in low-weight neonates born more than two months early. Pulmonary hypertension~\cite{hoeper2013definitions} is when the blood pressure in the lungs' arteries is elevated. The discharge weight is the preterm neonates' weight when discharged from the hospital after birth. Next, we analyze the data distribution and correlations between these outcomes. The results show that the three adverse neonatal outcomes are relevant to each other. For instance, infants with more weight gain might have less risk of BPD and PH. Therefore, we propose to make joint predictions of multiple adverse neonatal outcomes for preterm neonates to improve the accuracy of individual outcome predictions.

%% method
Moreover, by considering the correlations across different neonatal outcomes, we first formulate the diagnosis of multiple neonatal outcomes as a multi-task learning (MTL) problem. We then propose an MTL framework to solve this problem. Technically, our proposed multi-task learning framework is based on a Neural Network (NN) model. It consists of shared hidden layers and multiple task-specific hidden layers followed by predictors for different tasks. In particular, the shared hidden layers aim to capture \textbf{correlated yet often hidden knowledge among all neonatal outcomes}. Different task-specific branches are designed to learn the \textbf{unique features of each neonatal outcome}. By leveraging the correlations and task-specific information of the outcomes, this framework pursues the joint learning of multiple adverse neonatal outcomes. In addition, it enhances data efficiency and mitigates the overfitting problem. 



% Experiment
Adverse neonatal outcome prediction experiments are conducted using the limited data of 121 preterm neonates. Each Electronic Health Record (EHR) consists of 69 input attributes and three primary neonatal outcomes. Specifically, there are two categorical (i.e., severe BPD and PH diagnosis) and one continuous (i.e., discharge weight) adverse neonatal outcome. Ten traditional machine learning algorithms are compared. The F1 score and the Area under the ROC Curve (AUC) are used to evaluate algorithms for the two categorical outcomes. The mean squared error (MSE) is used for the continuous outcome. The empirical results show that each task-specific MTL method outperforms its base single-task model (i.e., NN). To obtain insights into the reasoning behind our model's predictions, we employ  Grad-CAM~\cite{selvaraju2017grad} to estimate the feature importance of 69 input attributes.


Overall, the main contributions of this work are as follows: 1) this work focuses on preterm neonates instead of newborns in general, 2) we propose a novel multi-task learning (MTL) framework for jointly predicting multiple adverse neonatal outcomes using limited annotated data, and 3) we analyze the feature importance for each adverse neonatal outcome to obtain insights into model interpretability.

