\section{Related Work}

{\bf Prediction of neonatal outcomes with machine learning models.} Mboya {\em et al.}~\cite{mboya2020prediction} demonstrate that machine learning models achieve better predictive performance over classical or conventional regression models. Furthermore, Mangold {\em et al.}~\cite{mangold2021machine} conducts a systematic review that collates, critically appraises, and analyzes existing ML models for predicting neonatal outcomes. Unlike Mboya {\em et al.}~\cite{mboya2020prediction} or Mangold {\em et al.}~\cite{mangold2021machine}, Sheikhtaheri {\em et al.}~\cite{sheikhtaheri2021prediction} consider a more practical setting where the experiments are only performed on newborns in neonatal intensive care units (NICUs). Hsu {\em et al.}~\cite{hsu2021machine} leverage ML models (i.e., Random Forest and bagged CART) to estimate the neonatal mortality rate with respiratory failure, which often indicates a higher severity of illness. In contrast to prior research~\cite{mangold2021machine,sheikhtaheri2021prediction,hsu2021machine}, we intend to investigate correlations between different outcomes. Previous works~\cite{hu2022prediction,harutyunyan2019multitask} have shown the effectiveness of a multi-task learning framework for clinical prediction. Therefore, we believe that leveraging these correlations might further improve the ML models. We formulate the prediction of neonatal outcomes as a multi-task learning problem, which considers the potential correlations across multiple outcomes and prevents the ML models from overfitting~\cite{ruder2017overview,zhang2021survey}.


{\bf Multi-task learning.} Multi-task learning (MTL) is a machine learning approach that involves training a single model to perform multiple tasks at the same time~\cite{ruder2017overview, zhang2021survey}. By training a model on multiple tasks simultaneously, it can learn shared features and representations that are useful for all tasks, promoting its generalization ability and making it more robust. MTL has applications in a wide range of fields, including computer vision~\cite{ren2015faster,he2017mask}, natural language processing~\cite{hashimoto2016joint,sogaard2016deep}, and speech synthesis~\cite{wu2015deep}. In the clinical scenario, prior research~\cite{hu2022prediction,harutyunyan2019multitask} has explored the effectiveness of multi-task learning. Hu {\em et al.}~\cite{hu2022prediction} employ an MTL framework to screen commercially available and effective inhibitors against SARS-CoV-2. Harutyunyan {\em et al.}~\cite{harutyunyan2019multitask} build four benchmarks for clinical time series data and propose an MTL framework, which empirically demonstrates that multi-task training acts as a regularizer for almost all tasks. In this work, we first explore the use of MTL in jointly predicting multiple neonatal outcomes for preterm infants.

