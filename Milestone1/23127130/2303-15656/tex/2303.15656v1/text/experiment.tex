\section{Experiment}

\subsection{Experiment Setup}
{\bf Dataset.} Adverse neonatal outcome prediction experiments are conducted using the data of 121 premature infants. Each sample has 69 input features and three adverse neonatal outcomes. In particular, two adverse neonatal outcomes are categorical and the other is numerical. We use $5$-fold cross-validation to obtain a more reliable estimate of the machine learning models' performance.

{\bf Evaluation metric.}  We use the F1 score and Area under the ROC Curve (AUC) of the positive samples to evaluate the performance of the BPD and PH diagnosis classification tasks. We use the mean squared error (MSE)  to estimate the discharged weight prediction, which is a regression task.

\subsection{Baseline Methods}
Extensive experiments have been conducted with various ML techniques on our dataset. We list all ML methods for classification and regression tasks below.

{\bf Classification task.} Six traditional machine learning methods are included in our experiments, including Naive Bayes classifier, Logistic Regression~\cite{cox1958regression}, Random Forest~\cite{ho1995random}, Decision Tree~\cite{wu2008top}, Support Vector Machine (SVM)~\cite{cortes1995support}, and XGBoost~\cite{Chen:2016:XST:2939672.2939785}.

{\bf Regression task.} We implement seven machine learning models to predict the discharged weights of infants. These models are Ridge Regression~\cite{hoerl1970ridge}, Lasso Regression~\cite{tibshirani1996regression}, ElasticNet~\cite{zou2005regularization}, Random Forest~\cite{ho1995random}, Decision Tree~\cite{wu2008top}, Support Vector Regression (SVR)~\cite{cortes1995support}, and XGBoost~\cite{Chen:2016:XST:2939672.2939785}.




\subsection{Implementation Details}
\textbf{Architecture.} Our models are implemented in PyTorch. For each single-task model, we select the optimal number of hidden layers in the range of \{1,2,3,4,5\}. Additionally, we choose the number of neurons per hidden layer from the set of \{128, 256, 512, 1024\}. The hyperparameters of the multi-task model are searched in terms of \{1,2,3,4\}-layer shared hidden layers with \{64, 128, 256, 512\} neurons and \{1,2,3\}-layer task-specific hidden layers with \{64, 128, 256\} neurons. All these hyperparameters are tuned corresponding to the performance on the $5$-fold cross-validation.

\textbf{Optimization.} We optimize the model through an Adam~\cite{kingma2014adam} optimizer with an initial learning rate of $\{5\times10^{-3}, 1\times10^{-2}, 2\times10^{-2}\}$. The batch size is 64, and weight decay is $\{10^{-1}, 10^{-2}, 10^{-3}\}$. The learning rate follows a cosine decay schedule~\cite{loshchilov2016sgdr}. We train all baseline models for \{20, 50, 100\} epochs on a 1080Ti GPU.


\subsection{Comparison with Other Traditional Machine Learning Techniques}
Table~\ref{tab:overall} summarizes the quantitative results on three tasks about the adverse neonatal outcomes. We compare our proposed multi-task learning framework with several traditional machine learning frameworks. 
Overall, the results across different tasks show that our proposed MTL framework outperforms other traditional machine learning methods. Note that Neural Network (NN) is a degraded version of our proposed multi-task learning framework. In other words, comparisons between NN and our MTL methods can clearly ablate the impact of multi-task learning. The comparison between the NN and different task-specific MTL methods suggests that MTL methods can significantly outperform their base model, as shown in Table~\ref{tab:overall}. The improvements indicate that multi-task learning can boost the performance of the primary task by leveraging the other relevant auxiliary tasks, which provide additional information for learning the primary task. Therefore, data efficiency highlights the necessity of exploring MTL techniques, especially for small datasets.  %Nevertheless, the current MTL framework cannot improve the performance of all tasks simultaneously.

\subsection{Feature Importance Analysis}
Feature importance analysis provides insight into the relationships between input features and the predicted outcomes. Gradient-weighted Class Activation Mapping (Grad-CAM)~\cite{selvaraju2017grad} generates a heatmap highlighting the regions of an input image most influential for the network's prediction. Grad-CAM has gained widespread usage in the field of computer vision. We adapt the Grad-CAM method to the neonatal health area. In particular, we employ the Grad-CAM to analyze the feature importance in predicting adverse neonatal outcomes.


{\bf Severe BPD.} We list the top 10 most important features for predicting severe BPD:
\begin{itemize}
    \item Total ventilator days ({\bf the most influential})
    \item Oxygen AUC 28 days
    \item Summary total oxygen exposure
    \item Pulmonary deterioration oxygen adjusted
    \item Total conventional ventilator days
    \item Total high-frequency ventilator days
    \item Pulmonary deterioration AUC
    \item Oxygen severity index 28 days
    \item Maternal Smoking
    \item Delivery room hyperthermia
\end{itemize}
Clinicians believe severe BPD should be most correlated with indicators related to the respiratory system, such as ventilator usage~\cite{gibbs2020ventilation,gupta2009ventilatory}, oxygen exposure~\cite{nesterenko2008exposure}, pulmonary deterioration~\cite{davidson2017bronchopulmonary}, maternal smoking~\cite{aschner2017can,morrow2017antenatal}, and so on. The feature importance computed by Grad-CAM largely agrees with the clinicians. Specifically, \textbf{the usage of the ventilator, exposure to oxygen, and pulmonary deterioration are the most critical factors in predicting severe BPD}, as expected. In addition, it is found that maternal smoking has a significant impact on the severity of BPD. 

{\bf PH Diagnosis.} Similarly, the top 10 most important features of pulmonary hypertension diagnosis are shown as follows:
\begin{itemize}
    \item Birth weight z-score ({\bf the most influential})
    \item respiratory failure reintubations provided
    \item Delivery room - CPAP provided
    \item Birth weight less than 10th percentile (Intrauterine growth restriction)
    \item Delivery room - intubated
    \item Delivery mode 1
    \item Delivery room hyperthermia
    \item Delivery room - resuscitation provided
    \item Delivery room - chest compressions
    \item Summary total protein
\end{itemize}
Clinicians suggest that PH diagnosis is closely associated with birth weight~\cite{bhat2012prospective}, lung conditions~\cite{nathan2019pulmonary}, delivery mode~\cite{wilson2011persistent}, etc. The results of feature importance are \textit{aligned} with the expectations of clinicians, as \textbf{birth weight and respiratory failure reintubations are the primary factors of the PH diagnosis}. Moreover, the results show the conditions of the delivery room are critical factors in PH diagnosis. Furthermore, the nutrition inputs (e.g., protein) influence the PH diagnosis.

{\bf Discharge weight.} The following are the top 10 most important features, in order, for predicting the discharge weight:
\begin{itemize}
    \item Birth Weight z-score ({\bf the most influential})
    \item Infant gestational age
    \item Infant birth weight
    \item Delivery room - intubated
    \item Delivery room hyperthermia
    \item Delivery room - CPAP provided
    \item Delivery room - chest compressions
    \item Delivery mode 1
    \item Maternal smoking
    \item Maternal asthma diagnosis
\end{itemize}
The results show that {\bf discharge weight is most related to infant birth weight and gestational age.} In addition, the delivery mode and conditions of the delivery room have significant impacts on discharge weight.