\section{Methodology}

In this section, we first formulate the problem. We then describe the single-task learning baseline (i.e., Neural Network) for predicting each adverse neonatal outcome. Finally, we present the proposed multi-task learning framework for multiple outcome predictions.


\subsection{Problem Formulation}
We formulate the prediction of adverse neonatal outcomes as a multi-task learning problem. The input feature ${\bx}_i \in \mathbb{R}^{1\times D}$ consists of information about the infant, its parents,  delivery room, nutrition summary, etc. $D$ denotes the number of input features, and $x_i$ denotes the $i\text{-th}$ sample in the dataset. $\by = \{\by^{(j)}\}_{j=1}^{M}$ refers to the ground truth of different adverse neonatal outcomes, where $M$ indicates the total number of adverse neonatal outcomes and $\by^{(j)}$ is the $j\text{-th}$ outcome. For the MTL framework, we define the shared hidden layers as $f(\cdot; \theta)$ transferring input features $\bx$ into latent features $f(\bx; \theta)$, where $\theta$ denotes the learnable parameters. Following the shared hidden layers, $M$ task-specific branches $\{g^{(j)}(\cdot; \theta^{(j)})\}_{j=1}^{M}$ transfer the shared latent features into the final predictions $\hat{\by}_i$ for different neonatal outcomes:
\begin{equation}
    \hat{\by}_i=\{g^{(j)}(f(\bx_i; \theta); \theta^{(j)})\}_{j=1}^{M},
\end{equation}
where $\hat{\by}_i$ denotes the prediction of the $i$-th sample with $M$ outcomes, and $\theta^{(j)}$ represents the parameters of the $j$-th task-specific branch.

\subsection{Single-Task Learning Baseline}
As shown in Fig.~\ref{fig:single}, we employ a Neural Network with multiple hidden layers as our backbone. In particular, the neural network $\text{NN}(\cdot; \theta)$ converts an input sample $\bx_i$ with $D$ input attributes into a prediction $\hat{\by}_i^{(j)}$ for the $j$-th neonatal outcome. 


For the classification task, we obtain the prediction scores $\hat{\by}^{(j)}$ using a softmax function. After that, we use the cross-entropy loss to optimize the classifier:
 \vspace{-1mm}
\begin{equation}
    \label{eq:cls}\mathcal{L}_\mathtt{cls}^{(j)} = -\frac{1}{N} \sum^{N}_{i=1} \by_i^{(j)} \cdot \log(\hat{\by}_i^{(j)}),
\end{equation}
 \vspace{-2mm}

where $\by_i^{(j)}$ denotes the ground truth of the $i\text{-th}$ training sample for the $j\text{-th}$ task, and $N$ denotes the batch size.

For the regression task, we use the mean squared error (MSE) loss to optimize the regression model:
 \vspace{-1mm}
\begin{equation}
    \label{eq:reg}
    \mathcal{L}_\mathtt{reg}^{(j)} = \frac{1}{N}\sum^{N}_{i=1} \Vert \by_i^{(j)} - \hat{\by}_i^{(j)} \Vert^2_2.
\end{equation}
 \vspace{-2mm}



\begin{table*}[t]
\caption{Overall performance on prediction of adverse neonatal outcomes. The BPD and PH Diagnosis refer to bronchopulmonary dysplasia and pulmonary hypertension diagnosis, respectively. The best score for each task is highlighted in bold. The improvements in task-specific MTL methods are shown in red. The results of BPD and PH Diagnosis are in \%.}
\begin{adjustbox}{width=1\textwidth}\setlength\tabcolsep{2pt}
\begin{tabular}{lccccc}
\toprule[1pt]
\multicolumn{1}{c}{\multirow{2}{*}{Methods}} & \multicolumn{2}{c}{\textbf{BPD} (Task 1)}                                 & \multicolumn{2}{c}{\textbf{PH Diagnosis} (Task 2)} & \textbf{Discharge Weight} (Task 3) \\ \cmidrule(r){2-3} \cmidrule(r){4-5} \cmidrule(r){6-6}
\multicolumn{1}{c}{}                         & F1 $\uparrow$    & AUC $\uparrow$ & F1 $\uparrow$              & AUC $\uparrow$             & MSE $\downarrow$                       \\ \midrule[1pt] 
Naive Bayes                       & 22.5 & 60.4 & 33.4           & 59.7           & -                         \\
Logistic Regression~\cite{cox1958regression}                           & 40.2 & 69.0          & 28.0           & 66.4           & -                         \\
Random Forest~\cite{ho1995random}                                 & 42.4 & 81.2 & 44.3           & 75.3  & 0.266            \\
Decision Tree~\cite{wu2008top}                                 & 46.9 & 76.9       & 35.1           & 55.2           & 0.353                     \\
SVM/SVR~\cite{cortes1995support}                                           & 39.2 & 71.0    & 25.9           & 43.2           & 0.293                     \\
XGBoost~\cite{Chen:2016:XST:2939672.2939785}                                       & 31.8          & 72.4  & 39.8           & \textbf{77.2}           & 0.296                     \\
Ridge Regression~\cite{hoerl1970ridge}                              & -              & -              & -               & -               & 0.365                     \\
Lasso Regression~\cite{tibshirani1996regression}                              & -              & -              & -               & -               & 0.270                     \\
ElasticNet~\cite{zou2005regularization}                                    & -              & -              & -               & -               & 0.267                     \\ \midrule[1pt] 
Neural Network~\cite{mcculloch1943logical} (base model)                   &  43.1         &     81.2     & 32.3           & 75.1           & 0.279                     \\
MLT (tuned for BPD)  &    {\hspace{6.8mm} \textbf{48.0} \color{red}{($\uparrow$4.9)}}   &  {\hspace{6.8mm} \textbf{83.2}  \color{red}{($\uparrow$2.0)}}   & 39.7          & 68.2           & 0.288                     \\
MLT (tuned for PH Diagnosis)                                  & 35.8          & 65.7          & {\hspace{6.85mm} \textbf{46.6} \color{red}{($\uparrow$2.3)}} & {\hspace{6.85mm} 76.5 \color{red}{($\uparrow$1.4)}}  & 0.330                     \\
MLT (tuned for Discharge Weight)                                & 43.2          & 79.8         &     42.0       & 71.8           & {\hspace{10mm} \textbf{0.257} \color{red}{($\downarrow$0.022)}}                    \\ \bottomrule[1pt] 
\end{tabular}
\end{adjustbox}\label{tab:overall}
\end{table*}

\subsection{Multi-Task Learning Framework}

Although previous ML-based works have shown the effectiveness of the single-task learning framework, such a framework might ignore the potential correlations between different outcomes, leading to suboptimal results. Moreover, single-task learning models are more likely to suffer from overfitting issues, especially in cases with limited training data. 
As shown in Fig.~\ref{fig:multi}, we propose a novel multi-task learning framework to leverage the potential correlation between various adverse neonatal outcomes and avoid the overfitting problem. Technically, the proposed multi-task learning framework consists of shared hidden layers and multiple task-specific branches. Each task-specific branch contains several hidden layers and a prediction layer. The overall objective $\mathcal{L}_\mathtt{mtl}$ of the MTL framework is computed by combining the weighted losses of multiple tasks:
\begin{equation}
    \mathcal{L}_\mathtt{mtl} = \sum^{M}_{j=1} \lambda^{(j)} \mathcal{L}_\mathtt{stl}^{(j)},
\end{equation}
where $\mathcal{L}_\mathtt{stl}^{(j)}$ denotes the $j$-th task-specific objective with the loss weight $\lambda^{(j)}$, and $M$ is the number of tasks.

Overall, the proposed MTL framework can exploit the correlations among neonatal outcomes since by using multi-task learning, the framework is better equipped to capture the general patterns that are relevant to all tasks. These patterns would promote the generalization ability and prevent ML-based methods from overfitting~\cite{zhang2021survey,ruder2017overview}. Meanwhile, the MTL framework can enhance the performance of a specific primary task by using related tasks as auxiliary tasks~\cite{liebel2018auxiliary}, which provide auxiliary information for the primary task, resulting in a more efficient and effective learning process.
