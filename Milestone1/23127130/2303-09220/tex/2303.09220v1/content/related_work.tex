\section{Related work} \label{sec:related}

%\subsection{Self-adaptation in robotics and ROS}
The UNDERSEA exemplar by Gerasimou \emph{et
al.}~\cite{undersea} provides an AUV simulation in which the robot performs self-adaptation to
deal with uncertainties such as sensor failures and changing goals. 
\acronym is related to UNDERSEA as both address the domain of self-adaptive AUVs. However, a
key difference is the underlying libraries used to develop software
for the robot. UNDERSEA uses MOOS-IvP while \acronym uses
ROS2, a more widely used framework that is considered state of the art in the robotics
research community, which contributes to the reusability and extensibility of
\acronym.

There have been previous exemplars that do use ROS, in particular, the
Body Sensor Network by Gil \emph{et al.}~\cite{bsnexemplarROS}. However,
its application differs significantly from \acronym as it concerns
health monitoring through a series of sensors rather than a robot
vehicle fulfilling a mission autonomously.

Cheng \emph{et al.} proposed AC-ROS~\cite{ACROS}, a framework 
which uses assurance cases to endow a ROS-based system with
self-adaptive capabilities. Specifically, it concerns an `EvoRally'
vehicle, a terrestrial robot tasked with patrolling an environment as
its mission, while meeting requirements such as energy
efficiency. The authors 
do not provide the source code of the proposed system,
which means it does not serve as an exemplar as \acronym
does.

The paper by Bozhinoski \emph{et al.} \cite{bozhinoski22advrob} concerns an earlier iteration of using MROS for runtime adaptation similar to this paper. Their work revolves around two cases, a manipulator robot with a ``pick and place" task and a mobile robot navigating around obstacles on a factory floor. Both of the use cases show a need to deal with uncertainties, e.g., with a safety concern by disabling one of the pick and place arms. When compared to \acronym, the key differences are the migration from ROS to ROS2, as well as the use case being an AUV rather than a manipulator or mobile terrestrial robot. 