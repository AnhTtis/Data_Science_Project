\section{Introduction}

Autonomous robots are an excellent case for applying self-adaptation
techniques
\cite{edwardsArchDrivenRobotics,camara_software_2020,park_task-based_2012,shin_platooning_2021,askarpour_robomax_2021,undersea,ACROS}. These
robots face uncertainty in their operation stemming from both the
system (e.g., sensor failures) and the environment (e.g., different
terrains). They need to complete their missions despite such
uncertainty \cite{Ludvigsen-2021} with minimal or no human supervision
\cite{Huang-2004}.  A subclass of these robots, autonomous underwater
vehicles (AUVs)\cite{Wynn-2014} which are used for, e.g., subsea
observation, are particularly challenging: once they have been
deployed in the real world, they need to take both low-level (e.g.,
increase thruster power) and high-level (e.g., dive deeper) adaptive
decisions without \emph{any} human supervision.

Self-adaptive systems can be implemented as two-layered systems
consisting of a \emph{managed} and a \emph{managing}
subsystem~\cite{weyns2020introduction}. The managed subsystem handles
the domain concerns, while the managing subsystem implements the
adaptation logic and exploits functional alternatives of the managed
subsystem to handle the self-adaptation process.

This paper proposes the exemplar \acronym\footnote{ Self-adaptive
  Underwater Autonomous Vehicle Exemplar.} to facilitate research in
the challenging domain of self-adaptive AUVs and to allow the
comparison of different self-adaptation strategies.  \acronym is based
on ROS2 -- one of the most widely adopted robotics software
frameworks~\cite{doi:10.1126/scirobotics.abm6074}. This ensures that
the system built for \acronym can (i)~run directly on real robots and
not only in simulation environments, (ii)~serve as a basis for other
adaptive robotic missions, and (iii)~be easily extended with new
functionalities and adaptation concerns. The exemplar is publicly
available at \url{https://github.com/kas-lab/suave}.


The exemplar focuses on the scenario of \emph{pipeline inspection} for
a single AUV.  The AUV's mission is to first search for a pipeline on
the seabed, then follow and inspect the pipeline. The functionalities
required to accomplish this mission are implemented in the
\emph{managed subsystem} of \acronym.  During the execution, of the
mission, two types of uncertainties are considered: component failures
in the form of thruster failures (e.g., due to debris getting stuck in
a thruster) and changes in the environmental conditions in the form of
changes in the water visibility (e.g., due to currents disturbing
sediment from the seabed). While the first uncertainty may impact the
robot's motion by making it move unexpectedly, the second impacts the
efficiency of the pipeline search and detection by forcing the robot
to be closer to the seabed to detect the pipeline in case of poor
water visibility, which results in a smaller field of view while
searching.

The exemplar enables the development of a \emph{managing subsystem} to
address the previous uncertainties. The managing subsystem should be
able to monitor the current runtime circumstances, recover the AUV's
thrusters in case of a thruster failure, and adjust the AUV's path
generation algorithm to account for changes in water visibility.

To illustrate the use of adaptation frameworks in \acronym, the
managing subsystem was implemented with
Metacontrol~\cite{corbato2013model,bozhinoski22advrob}, a framework
that enables self-adaptation in robotic systems and promotes the reuse
of the adaptation logic by exploiting a model of the managed subsystem
at runtime. Metacontrol's strength lies in the separation between the
application and adaptation concerns, i.e., in the separation between
the robot's operation and the logic of when and how to adapt.  This
separation of concerns allows the adaptation logic to be reused in a
straightforward way in different applications.  However, it is
important to highlight that even though \acronym is equipped with a
Metacontrol-based adaptation logic, the exemplar can also be used
without Metacontrol, which in addition allows for comparing other
approaches to Metacontrol-based ones.

In summary, the contributions of this paper are:
\begin{itemize}
\item a \emph{self-adaptation exemplar for AUVs using ROS2} that can
  be equipped with different adaptation logics, enables the comparison
  of different self-adaptation strategies, forms a basis for other
  adaptive robotic missions, and can run both on real robots and in
  simulation environments; and
\item a \emph{Metacontol-based adaptation logic formulation} that can
  serve as a baseline for future research and as a benchmark for
  self-adaptation strategies, and is easily reusable for other robotic
  and non-robotic applications.
\end{itemize}

\emph{Paper outline.}  Section~\ref{sec:related} presents related
work, after which Section~\ref{sec:problem_statement} further details
the use case and the overall architecture. The managed subsystem is
described in Section~\ref{sec:managed}, while
Section~\ref{sec:managing_subsystem} discusses the managing subsystem
and how Metacontrol is applied to the use
case. Section~\ref{sec:managing_requirements} briefly explains how the
exemplar can be reused and extended, and Section~\ref{sec:results}
presents and discusses the results of applying Metacontrol. Finally,
Section~\ref{sec:conclusion} concludes the paper.
