\section{Managed Subsystem} \label{sec:managed}

The managed subsystem is implemented as a ROS2-based system and is depicted in Fig.~\ref{fig:system_architecture}. The only non-ROS2 component is \kw{ArduSub}\footnote{\url{https://www.ardusub.com/}}, which is an open-source autopilot for underwater vehicles. In this application it is used to solve the functions \kw{Control Motion} and \kw{Localization}\footnote{It is assumed that the AUV has appropriate sensors for localization}. The \kw{MAVROS}
package works as a bridge between \kw{ArduSub} and the ROS2 components. The \kw{Detect Pipeline} node detects the pipeline and informs \kw{Follow Pipeline} and the \kw{Coordinate Mission} node about its position\footnote{A mock perception system is used.}. The \kw{Coordinate Mission} node coordinates the tasks' execution and sets the adaptation goals. Note that the function \kw{Inspect Pipeline} is not implemented,
since the actual inspection of the pipeline is not the focus of this work. However, the exemplar can easily be extended with this functionality by adding a new node that implements the pipeline inspection.

\kw{Follow Pipeline}, \kw{Generate Search Path},  and \kw{Maintain Motion} are lifecycle nodes, which means that they have internal states, such as \emph{active} and \emph{inactive}, and it is possible to switch between these states at runtime. Furthermore, the System Modes package \cite{nordmann2021system} 
extends the state \emph{active} with additional modes, e.g., \emph{active.low\_altitude}.

To adapt the managed subsystem, the managing subsystem adapts the lifecycle nodes by changing their states. This is done by the \kw{Mode Manager} node, which is used off-the-shelf from the System Modes package. 
The states available for \kw{Generate Search Path} are \kw{deactivated}, \kw{low altitude}, \kw{medium altitude}, and \kw{high altitude}. Subsequently, the states available for \kw{Follow Pipeline} are \kw{deactivated} and \kw{activated}, while the states for \kw{Maintain Motion} are \kw{deactivated}, and \kw{recover thrusters}.

To enable other developers of self-adaptive systems to use this exemplar and compare different approaches, a \kw{Gazebo}-based  \footnote{\url{https://gazebosim.org/home}} simulation of a pipeline inspection environment and a model of the AUV is provided. The BlueROV2\footnote{\url{https://bluerobotics.com/store/rov/bluerov2/}}
robot was selected as the AUV for the exemplar because (i)~it is compatible with \kw{ArduSub}; (ii)~it is easily integrated with \kw{Gazebo} via plugins; and (iii)~the robot has a low price compared to other available AUVs, making it more accessible to researchers to reproduce the exemplar with a real robot.












