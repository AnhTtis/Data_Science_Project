\section{Evaluation}\label{sec:results}
To evaluate the performance of different managing subsystems using this exemplar, the mission described in Section~\ref{sec:problem_statement} was implemented.
The mission consists of the AUV performing $T1$ and $T2$ while subject to $U1$ and $U2$ until a user-provided time limit  is reached. To evaluate the mission, the following metrics were used: the \emph{search time}, the amount of time elapsed from the beginning of the search until the pipeline is found, and the total \emph{distance inspected} of the pipeline.

To provide a baseline for the exemplar, the mission is performed with two different managing subsystems and with no managing subsystem, using a fixed configuration. The managing subsystems are the Metacontrol-based implementation detailed in Section~\ref{sec:managing_subsystem} and a random managing subsystem that selects configurations arbitrarily. 

Since the system is non-deterministic due to characteristics of \kw{Gazebo}, \kw{ArduSub}, and the interaction between them, no run of the simulation is exactly the same. Thus, the mission execution and metrics collection are automated with a runner to allow multiple runs to be easily performed. 

This section briefly describes how to configure the exemplar, and the results of running the exemplar. Further details may be found in the exemplar repository.

\subsection{Configuring the exemplar}

In \acronym, the AUV's mission execution can be varied by changing the parameters of the system. In the \kw{Water Visibility Observer}, the available parameters are the water visibility minimum and maximum values, periodicity, and initial phase shift. In the \kw{Thrusters Monitor}, the available parameter is a list with thruster events indicating which thruster fails and when. In the \kw{Coordinate Mission}, the mission time limit can be set. In the random manager, the adaptation periodicity can be set, and when using no manager, the default states for the lifecycle nodes can be set. In addition, the runner is parametrized with the number of runs to execute, and which managing subsystem to select. All parameters are adjusted using configuration files packaged in the exemplar.

\subsection{Results}
The mission was executed with a time limit of 300 seconds, water visibility  periodicity of 80 seconds, minimum and maximum values of 1.25 and 3.75, no phase shift, and thruster 1 failing after 35 seconds from the start of the mission. The results are shown in Table~\ref{tab:results_m1}. It can be noticed that with the Metacontrol managing subsystem both mean \emph{search time}\footnote{When the pipeline is not found, the time limit is used as the \emph{search time}} is lower, and the \emph{distance inspected} is higher. 
This indicates that, in this exemplar, Metacontrol improves the performance of the system, and outperforms the random managing subsystem and the system without a managing subsystem. 
In addition, the standard deviation (Std) of the \emph{search time} is lower for Metacontrol, indicating that it is more consistent when searching for the pipeline. The Std of the random manager for the \emph{distance inspected} is lower, however, its mean value is also low, indicating that the random manager is consistent in not inspecting the pipeline.
The results shown can be used as a baseline for comparing different managing subsystems.

\begin{table}[t]
\caption{Mission results}
\label{tab:results_m1}
\resizebox{\columnwidth}{!}{%
\begin{tabular}{|l|l|ll|ll|}
\hline
\multirow{2}{*}{\textbf{\begin{tabular}[c]{@{}l@{}}Managing\\ subsystem\end{tabular}}} &
  \multirow{2}{*}{\textbf{\begin{tabular}[c]{@{}l@{}}Number \\ of runs\end{tabular}}} &
  \multicolumn{2}{l|}{\textbf{Search time (s)}} &
  \multicolumn{2}{l|}{\textbf{Distance inspected (m)}} \\ \cline{3-6} 
                     &            & \multicolumn{1}{l|}{\textbf{Mean}} & \textbf{Std} & \multicolumn{1}{l|}{\textbf{Mean}} & \textbf{Std} \\ \hline
None                 & 20          & \multicolumn{1}{l|}{187.85}             & 42.40            & \multicolumn{1}{l|}{31.40}             & 13.00            \\ \hline
Random               & 20          & \multicolumn{1}{l|}{180.15}             & 82.05           & \multicolumn{1}{l|}{3.88}             & 3.19            \\ \hline
\textbf{Metacontrol} & \textbf{20} & \multicolumn{1}{l|}{\textbf{106.95}}    & \textbf{39.46}   & \multicolumn{1}{l|}{\textbf{51.15}}    & \textbf{12.01}   \\ \hline
\end{tabular}%
}
\end{table}



