\section{Conclusion}\label{sec:conclusion}

This work describes \acronym, a ROS2-based exemplar for self-adaptive underwater vehicles used for pipeline inspection. Due to its modular design, \acronym enables different managing subsystems to be applied to the system without the need to modify the managed subsystem, the monitor nodes, and the executing mechanism. In addition, the system can be easily extended with new functionalities and adaptation scenarios by adding new nodes. Furthermore, this paper provides a baseline for comparing the performance of different managing subsystems, and it shows that the addition of a Metacontrol-based managing subsystem increases the performance of the system in comparison to not using any managing subsystem or one that chooses configurations arbitrary.


In future work, \acronym can be extended with: more metrics for a more in-depth evaluation; more tasks (e.g., docking), functionalities (e.g. a de facto perception system), and components (e.g. sonars) for more realistic missions; more adaptation scenarios, e.g., adapting to changes in the water currents, and adapting the thruster configuration matrix when a thruster can not be recovered.