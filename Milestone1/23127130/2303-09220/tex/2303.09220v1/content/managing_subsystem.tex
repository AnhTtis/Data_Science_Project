\section{Managing Subsystem}
\label{sec:managing_subsystem}

The managing subsystem exploits functional alternatives of the managed
subsystem to enable adaptation and thereby increase system reliability.
Metacontrol is used as an example of how a managing
subsystem can be implemented. This section introduces
Metacontrol
and shows how the adaptation problem can be
formulated and implemented with Metacontrol.

\subsection{Metacontrol Background}
\label{subsec:metacontrol}
\emph{Metacontrol} uses the MAPE-K feedback loop \cite{brun09,kephart03computer} to implement self-adaptation. It \emph{Monitors} the managed subsystem during runtime, \emph{Analyzes} whether the system meets its requirements, \emph{Plans} a new configuration if the system does not meet the requirements, and then \emph{Executes} the reconfiguration of the managed subsystem. All this is done using a shared \emph{Knowledge Base} to which each step refers. In Metacontrol, the knowledge base conforms to the TOMASys (Teleological and Ontological Metamodel for Autonomous Systems) metamodel \cite{corbato2013model}. 

\begin{figure}
    \centering
    \includegraphics[width=0.4\textwidth]{figures/TOMASys.pdf}
    \caption{A simplified representation of the TOMASys elements}
    \label{fig:tomasys}
\end{figure}

A simplified version of the TOMASys metamodel is displayed in Fig.~\ref{fig:tomasys}.
TOMASys uses \emph{functions} $F$ to represent the functionalities of the system, e.g., generating a search path for the AUV.
The architectural variants that implement these requirements are captured by \emph{function designs} $FD(F, \mathcal{C}, \qa^{exp})$. To distinguish during runtime which function design is most suited in a given situation, a set $\qa^{exp}$ of \emph{expected quality attributes} is associated with it. An expected quality attribute value reflects how well a function design is supposed to fulfill the function $F$ it solves. Furthermore, a function design requires a set $\mathcal{C}$ of \emph{components} of the managed subsystem to solve $F$. A component $C(S_C)$ is a piece of hardware or software, e.g., a sensor or a path-planning algorithm, respectively. The status $S_C$ of a component indicates its availability, i.e., whether it is functioning or not.

An \emph{objective} $O(F,S_O,\qa^{req})$ is a runtime instantiation of a function $F$, e.g., generating a search path with a minimum required water visibility, whose status $S_O$ reflects whether the objective is currently achieved. Furthermore, the set $\qa^{req}$ of \emph{required quality attributes} specifies which quality attribute value the objective requires in order to work properly.

An objective $O$ is solved by a \emph{function grounding} $FG(O,FD,S_{FG}, \qa^{meas})$, which represents the function design $FD$ that is currently used to solve the objective. 
Its status $S_{FG}$ reflects whether the function grounding is currently able to achieve the objective. The set $\qa^{meas}$ of \emph{measured quality attributes} reflects how well the function grounding currently fulfills $O$ and is computed using sensor data.


\subsection{Metacontrol Formulation}
\label{sec:metacontrol_formulation}
The functions, architectural variants, and quality attributes required to solve the tasks $(T1)$ Search Pipeline and $(T2)$ Inspect Pipeline, described in Section \ref{sec:problem_statement}, are modeled conforming to the TOMASys metamodel. Table~\ref{tbl:functions_exemplar} specifies the functions ($F_1$)~maintain\_motion, ($F_2$)~generate\_search\_path, and ($F_3$)~follow\_pipeline, while Table~\ref{tbl:qas_exemplar} describes the quality attributes ($QA_1$)~water\_visibility, and ($QA_2$)~performance. 
Functions $F_1$ and $F_2$ are required to achieve $T1$, whereas $T2$ is achieved by $F_1$ and $F_3$.
The function designs that solve these functions are specified in Table~\ref{tbl:function_designs_exemplar}. The set of required components is empty for function designs $FD_2 - FD_6$ because they do not require any components that are susceptible to adaptation, or used in the reasoning process.

Since the objectives and function groundings are instantiated during runtime, they are not specified here. An objective for function $F_2$ is for example to generate a search path with no required quality attribute, which is defined as $O_2(F_2, \text{\emph{ok}}, \varnothing)$ in the notation introduced above. A possible function grounding for this objective is $FG_2(O_2, FD_4, \text{\emph{ok}}, \{QA_1^{meas}=1.1\})$.


\begin{table}
    \centering
    \caption{The TOMASys functions used for the exemplar}
    \label{tbl:functions_exemplar}
    \begin{tabular}{|@{\;}l@{\;}|@{\;}l@{\:}|p{4.7cm}@{\:}|}
        \hline
        \textbf{Function} & \textbf{Name} & \textbf{Requirement} \\
        \hline
        \rule{0pt}{2ex} 
        $F_1$ & maintain\_motion & Maintain the motion of the robot\\
        \hline
        \rule{0pt}{2ex} 
        $F_2$ & generate\_search\_path & Generate a path to search for the pipeline\\
        \hline
        \rule{0pt}{2ex} 
        $F_3$ & follow\_pipeline & Follow and inspect the pipeline\\
        \hline
    \end{tabular}
\bigskip
\caption{The TOMASys quality attributes used for the exemplar}
    \label{tbl:qas_exemplar}
    \begin{tabular}{|l|l|l|p{3.9cm}|}
        \hline
        \rule{0pt}{2ex} 
        \textbf{QA} & \textbf{Name} & \textbf{Unit} & \textbf{Description}\\
        \hline
        \rule{0pt}{2ex} 
        $QA_1$ & water\_visibility & [$0,\infty)$ & Reflects the maximum altitude (in meters) from which the AUV can perceive the seabed\\
        \hline
        \rule{0pt}{2ex} 
        $QA_2$ & performance & $[0,1]$ & Reflects how efficient the current search strategy is\\
        \hline
    \end{tabular}
\bigskip
\caption{The TOMASys function designs used for the exemplar}
    \label{tbl:function_designs_exemplar}
    \begin{tabular}{|@{\;}p{3.5cm}@{\;\,}|@{\;}p{2.0cm}@{\,}|p{2.7cm}@{\;}|}
        \hline
        \textbf{Function Design} & \textbf{Name} & \textbf{Description} \\
        \hline
        \rule{-3pt}{2ex} 
        $FD_1(F_1,\{\text{thruster}\_x\mid x=1,\dots,6\}, \{QA_2^{exp}=1\})$ & all\_thrusters & Uses all thrusters \\
        \hline
        \rule{-3pt}{2ex} 
        $FD_2(F_1, \varnothing, \{QA_2^{exp}=0.5)$ & recover\_thrusters & Recovers the thrusters that are in failure\\
        \hline
        \rule{-3pt}{2ex} 
        $FD_3(F_2, \varnothing, \{QA_1^{exp}=0.5,$ $ QA_2^{exp}=0.25\})$ & spiral\_low & Produces a spiral search path with low altitude \\
        \hline
        \rule{-3pt}{2ex} 
        $FD_4(F_2, \varnothing, \{QA_1^{exp}=1,$ $ QA_2^{exp}=0.5\})$ & spiral\_medium & Produces a spiral search path with medium altitude \\
        \hline
        \rule{-3pt}{2ex} 
        $FD_5(F_2, \varnothing, \{QA_1^{exp}=2,$ $ QA_2^{exp}=1\})$ & spiral\_high & Produces a spiral search path with high altitude \\
        \hline
        \rule{-3pt}{2ex} 
        $FD_6(F_3, \varnothing, \varnothing)$ & follow\_pipeline & Follows the pipeline \\
        \hline
    \end{tabular}
\end{table}

The MAPE-K loop steps in this exemplar are formulated as follows. The monitor step is responsible for measuring $QA_1^{meas}$ and for monitoring the state of the six thrusters. 

The analyze step uses Horn rules
to reason about the knowledge base. One example rule that analyzes whether the measured water visibility $QA_1^{meas}$ still satisfies the expected water visibility $QA_1^{exp}$ of the grounded function design is displayed in Fig.~\ref{fig:swrl_rule}. Note that it is written in terms of the notation introduced in Section~\ref{subsec:metacontrol}. Line~\ref{eq:fg} expresses that the rule reasons about a function grounding $FG$ that solves an objective $O$, is of type $FD$, has a status $S_{FG}$ and an associated set of measured quality attributes $\qa^{meas}$. Furthermore, the function design $FD$ solves the function $F$, has a set of required components $\mathcal{C}$ and an associated set of expected quality attributes $\qa^{exp}$. Note that it is implicitly assumed that $FG$ is well-formed, i.e., that the function of which $O$ is a type of is the same as the function that $FD$ solves. Since this rule should analyze the water visibility, Line~\ref{eq:qas_contained} ensures that $QA_1^{meas}$ is an element of the set $\qa^{meas}$ and that $QA_1^{exp}$ is an element of the set $\qa^{exp}$, i.e., that both $FG$ and $FD$ are related to water visibility. Finally, if the measured value of $QA_1$ is less than its expected value associated with the grounded function design, see Line \ref{eq:meas_less_exp}, then the status of the function grounding is set to \emph{error}, see Line~\ref{eq:status_error}.

\begin{figure}[t]
    \centering
    \begin{align}
    &FG\bigl(O, FD, S_{FG}, \qa^{meas}\bigr) \boldsymbol{\wedge} ~ FD(F, \mathcal{C}, \qa^{exp}) \label{eq:fg}\\
    %&\boldsymbol{\wedge} ~ F = G \label{eq:same_function}\\
    & \boldsymbol{\wedge} ~ QA_1^{meas} \in \qa^{meas}
    ~\boldsymbol{\wedge} ~ QA_1^{exp} \in \qa^{exp} \label{eq:qas_contained}\\
    &\boldsymbol{\wedge} ~ QA_1^{meas} < QA_1^{exp} \label{eq:meas_less_exp}\\
    &\boldsymbol{\Rightarrow} ~ S_{FG} = \text{\emph{error}} \label{eq:status_error}
\end{align}
\caption{Rule to analyze whether the measured water visibility $QA_1^{meas}$ still satisfies the expected water visibility $QA_1^{exp}$ of the grounded function design}
\label{fig:swrl_rule}
\end{figure}

In the planning step, the function designs with $QA_1^{exp}$ higher than $QA_1^{meas}$ are filtered out as the visibility they would expect is not measured, afterward the remaining function design with the highest expected search performance ($QA_2^{exp}$) is selected as the desired configuration. The selected configuration is then carried out in the execute step.

\subsection{Metacontrol Implementation}

As depicted in Fig.~\ref{fig:system_architecture}, the monitor step is implemented with the \kw{Water Visibility Observer} and the \kw{Thruster Monitor} nodes. They are used for measuring $QA_1$ and monitoring the status of the six thrusters $\text{thruster}\_x$ where $x\in\{1,\dots,6\}$, respectively. 
To simplify the system and avoid the addition of unnecessary nodes, instead of adding water visibility to the Gazebo simulator, the \kw{Water Visibility Observer} simulates water visibility measurements with a sine function, and instead of probing the managed subsystem to identify thruster failures the \kw{Thruster Monitor} simulates the thruster failures events. Since the monitor step is mocked up, and its probes and intermediary nodes that would be required to provide the probes are not implemented, they are not included in Fig. \ref{fig:system_architecture}. Both nodes publish their data into the \kw{/diagnostics} topic with the ROS2 default \kw{DiagnosticArray} message type. 


The knowledge base (KB), the analyze and plan step are implemented using MROS2~\footnote{\url{https://github.com/meta-control/mc_mros_reasoner}}~\cite{corbatoMROSRuntimeAdaptation2020}, a ROS2-based Metacontrol implementation, as the \kw{MROS Reasoner} node. The KB
is implemented with the Ontology Web Language (OWL)~\cite{Antoniou2004}, the Horn rules used for the analyze step with the Semantic Web Rule Language (SWRL)~\cite{horrocks2004swrl}, and the reasoning is done with Pellet\footnote{\url{https://github.com/stardog-union/pellet}}. The \kw{MROS Reasoner} receives water visibility measurements ($QA_1^{meas}$) and thruster status information from the monitor step, then decides whether adaptation is required, and, in this case, selects a desired configuration which it sends to the execute step (see Section \ref{sec:metacontrol_formulation} for more details). The \kw{MROS Reasoner} initially does not have objectives, so it does not perform adaptation. The adaptation reasoning only starts when the \kw{Coordinate Mission} node sends new objectives, such as $O_2(F_2, null , \varnothing)$, via the \kw{Adaptation Goal Bridge}. New objectives do not have a status yet.

The execute step uses the System Modes' \kw{Mode Manager} to adapt the managed subsystem, and the \kw{System Modes Bridge} bridges the \kw{Mode Manager} with the \kw{MROS Reasoner}. When a reconfiguration is needed, the \kw{MROS Reasoner} requests the new configuration via the \kw{/mros/request\_configuration} service to the \kw{System Modes Bridge}.Then the \kw{System Modes Bridge} forwards the request to the \kw{Mode Manager} using the correct service names, depending on the lifecycle node being adapted. The services used by the \kw{Mode Manager} are listed in Table~\ref{tbl:system_modes_services}, and the available modes are listed in Table~\ref{tbl:system_modes_modes}.


\begin{table}
    \centering
    \caption{Available System Modes' services}
    \label{tbl:system_modes_services}
    \begin{tabular}{|p{3cm}|p{4.4cm}|}
        \hline
        \textbf{Node} & \textbf{Service} \\
        \hline
        f\_generate\_search\_path  & /f\_generate\_search\_path/change\_mode   \\
        \hline
        f\_follow\_pipeline  & /f\_follow\_pipeline/change\_mode   \\
        \hline
        f\_maintain\_motion  & /f\_maintain\_motion/change\_mode   \\
        \hline
    \end{tabular}
    \bigskip
    \caption{Available modes}
    \label{tbl:system_modes_modes}
    \begin{tabular}{|p{3cm}|p{2.5cm}|c|}
        \hline
        \textbf{Node} & \textbf{Mode} & \textbf{Lifecycle state} \\
        \hline
        f\_generate\_search\_path & fd\_spiral\_high   & active               \\
        \hline
        f\_generate\_search\_path & fd\_spiral\_medium   & active             \\
        \hline
        f\_generate\_search\_path & fd\_spiral\_low   & active              \\
        \hline
        f\_generate\_search\_path & fd\_unground   & inactive  \\
        \hline
        f\_follow\_pipeline & fd\_follow\_pipeline   & active  \\
        \hline
        f\_follow\_pipeline & fd\_unground   & inactive  \\
        \hline
        f\_maintain\_motion & fd\_all\_thrusters   & inactive  \\
        \hline
        f\_maintain\_motion & fd\_recover\_thrusters   & active  \\
        \hline
    \end{tabular}
\end{table}


\section{Extending and connecting managing subsystems}\label{sec:managing_requirements}

With the described system implementation, the only Metacontrol-specific nodes of the system are the \kw{MROS Reasoner}, the \kw{System Modes Bridge}, and the \kw{Adaptation Goal Bridge}. All other nodes of the system can be reused with different managing subsystems. The only requirement to connect a managing subsystem to the managed subsystem is to ensure that the managing subsystem adheres to the provided monitor and execute ROS2 interfaces. As described in the previous section, the monitor interface is the ROS2 topic \kw{/diagnostics}, and the execute interfaces are listed in Table~\ref{tbl:system_modes_services}. To show that changing the managing subsystem is possible, a managing subsystem that randomly picks a configuration was also implemented.

Since the system is implemented with a modular design, it can be extended with additional functionalities and adaptation scenarios by adding new lifecycle nodes and updating the system modes' configuration file accordingly. The implemented functionalities can be replaced with different implementations as long as they adhere to the same interfaces; e.g., the \kw{Pipeline Detection} node could be replaced by a node that actually performs perception instead of a mock-up.


