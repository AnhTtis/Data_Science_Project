\section{Pipeline inspection exemplar}
\label{sec:problem_statement}
This section describes the use case and system architecture, the two
system layers are detailed in Sections~\ref{sec:managed} and
\ref{sec:managing_subsystem}.

\subsection{Use case description}
\label{sec:use_case_description}
The use case in this exemplar is about an AUV inspecting pipelines located on a seabed.
Its mission consists of two sequential tasks, $(T1)$~searching for the pipeline, then $(T2)$~simultaneously following and inspecting the pipeline. 


When performing its mission, the AUV is subject to two sources of
uncertainty that could trigger self-adaptation: $(U1)$~thruster
failures and $(U2)$~changes in water visibility. $U1$~arises from the
possibility of the AUV's thrusters failing at runtime, which may cause
the AUV to move unexpectedly. This is relevant for both $T1$ and $T2$. To overcome $U1$, the managed subsystem of the AUV
contains functional alternatives. When
one or more thrusters fail, it is possible to enter a recovery state
in which the thrusters are recovered. 
$U2$
influences the maximum distance at which the AUV can visually perceive
objects. This is relevant for $T1$, higher water visibility
allows the AUV to search for the pipeline at higher altitudes,
resulting in a larger field of view and the possibility of discovering the pipeline faster. On the other hand, if the water
visibility is low, the AUV has to move closer to the seabed to
search for the pipeline, which limits its field of view and therefore
may lead to a longer time to discover the pipeline. Thus, changing the
altitude of the AUV provides functional alternatives for dealing with $U2$.

This exemplar focuses on the problem of overcoming $U1$ and $U2$ using
a self-adaptation logic, implemented by a managing subsystem, that
can be extended and reused for other sources of uncertainty.
The managing subsystem shall overcome $U1$ by recovering the failed
thrusters at runtime, and $U2$ by adapting the maximum altitude for
the path generator algorithm according to the measured water
visibility.  Thus, by reacting to $U1$ and $U2$, the managing
subsystem increases the reliability and performance of the system.

For the feasibility of the exemplar, the use case was simplified while still allowing for a worthwhile application of self-adaptation to an AUV.  
 It is important to highlight that a
realistic operation of an AUV used for pipeline inspection would include
steps that are related to pre-dive, launching and recovery, human
interaction, and intermediary missions that are necessary to enable
the inspection.  Furthermore, there are several sources of uncertainty
not considered here, including ocean dynamics, sensor failures, and
battery duration.

\subsection{System Architecture}

To accomplish the mission described in Section~\ref{sec:use_case_description}, the managed subsystem requires the
functions represented in Fig.~\ref{fig:func_arch}. $T1$ requires the
functions \kw{Control Motion}, \kw{Maintain Motion},
\kw{Localization}, \kw{Detect Pipeline}, \kw{Generate Search
  Path}, and \kw{Coordinate Mission}, while $T2$ requires the functions \kw{Control Motion},
\kw{Maintain Motion}, \kw{Localization}, \kw{Detect Pipeline},
\kw{Follow Pipeline}, \kw{Inspect Pipeline}, and \kw{Coordinate Mission}. During runtime, the
functions must be activated and deactivated according to the task
being performed.


\begin{figure}[h]
    \centering
    \includegraphics[width=0.8\linewidth]{figures/FunctionalHierarchy.pdf}
    \caption{Managed Subsystem's Functional Hierarchy}
    \label{fig:func_arch}
\end{figure}
%

To overcome the uncertainties $U1$ and $U2$, a managing subsystem
requires the functionalities to \emph{monitor} the environment and the
managed subsystem's internal state, \emph{reason} about it, and
\emph{execute} the managed subsystem's reconfiguration.


\begin{figure}
    \centering
    \includegraphics[width=\linewidth]{figures/GeneralArchitecture_small.pdf}
    \caption{System Architecture}
    \label{fig:system_architecture}
\end{figure}

The required functions of the managed and managing subsystems
are realized as depicted in Fig.~\ref{fig:system_architecture}. The
managed subsystem is detailed in Section~\ref{sec:managed} and the
managing subsystem in Section~\ref{sec:managing_subsystem}.
It is important to mention that managed subsystem functions
\kw{Control Motion} and \kw{Localization} are achieved by \kw{ArduSub}, and the function \kw{Inspect Pipeline} is not realized since
the actual inspection of the pipeline is not the focus of this
work. It is also important to highlight that this exemplar implements
the function to \emph{reason} about the managing subsystem with Metacontrol to provide a
baseline for future research. However, it can be replaced with other
solutions, as long as they are compatible with the monitor and execute
interfaces, as described in Section~\ref{sec:managing_requirements}.
