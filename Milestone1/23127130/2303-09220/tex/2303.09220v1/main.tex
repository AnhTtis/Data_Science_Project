\documentclass[10pt,conference]{IEEEtran}
\IEEEoverridecommandlockouts
% \usepackage[latin1]{inputenc}
\usepackage[british]{babel}
\usepackage[all]{xy}
\usepackage{amscd}
\usepackage{amssymb}
\usepackage{amsthm}
\usepackage{enumitem}
\usepackage{mathrsfs,bbm}
\usepackage{xcolor,graphicx}
\usepackage{graphics}
\usepackage{soul}
\usepackage{comment}
\usepackage[all]{xy}
\usepackage{amscd}
\usepackage{amssymb,amsmath,latexsym}
\usepackage{amsthm}
\usepackage{enumitem}
\usepackage{mathrsfs,bbm}
\usepackage{dsfont}
\usepackage{tikz-cd}
\usepackage[T1]{fontenc}
\usepackage[utf8]{inputenc}  
 %
%%%%%%%%%%%%%%%%%%%%%%%%%%%%%%%%%%
%pagestyle
%%%%%%%%%%%%%%%%%%%%%%%%%%%%%%%%%%
%\pagestyle{plain}
\textwidth=430pt
\headsep=.7cm
\evensidemargin=15pt
\oddsidemargin=15pt
\leftmargin=0cm
\rightmargin=0cm
%%
%%%%%%%%%%%%%%%%%%%%%%%
\newcommand*\fixitem {\item[]%
  \refstepcounter{enumi}\hskip-\leftmargin\labelenumi\hskip\labelsep}
\newtheorem*{mainthm}{Main Theorem}
\newtheorem*{mainthm1}{Theorem}
\newtheorem*{maincor}{Corollary}
\usepackage[colorlinks=true]{hyperref}
\DeclareMathOperator{\Forall}{\forall}
\DeclareMathOperator{\Exists}{\exists}
\DeclareMathOperator{\ord}{ord}
\newcommand{\phiD}{\varphi_D}
\newcommand{\phiDI}{\varphi_{\mathbf{D}_I}}
\newcommand{\phiDIj}{\varphi_{\mathbf{D}_I (j)}}
\newcommand{\phiH}{\varphi_H}
\newcommand{\phiTimes}{\phiD \otimes \phiH}
\newcommand{\phiTimesDI}{\varphi_{\mathbf{D}_I} \otimes \phiH}
\newcommand{\R}{\mathscr{A}}
\newcommand{\X}{\mathscr{X}}
\newcommand{\Xf}{\mathscr{X}_{(k_0 ,i)}[r_0]}
\newcommand{\Xfr}{\mathscr{X}_{(k_0,i)}[r]}
\newcommand{\hotimes}{\widehat{\otimes}}
\newcommand{\C}{\mathbb{C}_p}
\newcommand{\V}{\mathscr{V}}
\newcommand{\B}{\mathscr{B}}
\newcommand{\dualD}{\mathfrak{D}}
\newcommand{\Dg}{\mathbf{D}}
\newcommand{\DD}{\mathcal{D}^0}
\newcommand{\DDg}{\mathcal{D}}
\newcommand{\DV}{\mathcal{D}}
\newcommand{\W}{\mathscr{W}_N}
\newcommand{\Ao}{\mathbf{A}^\circ}
\newcommand{\AoK}{\mathbf{A}^\circ_{\K}}
\newcommand{\AK}{\mathbf{A}_{/\K}}
\newcommand{\OOO}{\mathscr{A}^\circ}
\newcommand{\K}{\mathcal{K}} 
\newcommand{\OK}{\mathcal{O}_{\K}}
\newcommand{\varprojlog}[1]{\underleftarrow{\log\!^{#1}}}
\newcommand{\T}{\mathscr{T}}
\newcommand{\TT}{\mathbf{T}}
\newcommand{\VV}{\mathbf{V}}
\newcommand{\HH}{\mathcal{H}}
\newcommand{\hh}{\mathcal{H}^+}
\newcommand{\HG}[2]{\mathcal{H}_{#1}(#2)}
\newcommand{\hhl}{\mathcal{H}^{+,[l]}}
\newcommand{\hhj}{\mathcal{H}^{+,[j]}}
\newcommand{\hhjj}{\mathcal{H}^{+,[l,l']}}
\newcommand{\GS}{G_{\mathbb{Q},S}}
\newcommand{\Rf}{R_{(k_0 ,i)}[r_0]}
\newcommand{\Rfr}{R_{(k_0 ,i)}[r]}
\newcommand{\parT}{\langle T\rangle}
\newcommand{\Zf}{Z_{(k_0 ,i)}[r_0]}
\newcommand{\Zfr}{\mathscr{Z}_{(k_0 ,i)}[r]}
\newcommand{\ZFf}{\mathscr{Z}_{(k_0 ,i)}[r_0]}
\newcommand{\ZFfr}{\mathscr{Z}_{(k_0 ,i)}[r]}
\newcommand{\ZF}{\mathscr{Z}}    % To put all the different packages etc in a separate file

\usepackage{url}
\usepackage{listings}
\usepackage{comment}
\usepackage{xspace}
\usepackage{hyperref}
\usepackage{multirow}
\usepackage{graphicx}

 \lstset{frame=tb,
  language=Java,
  breaklines=true,
  showstringspaces=false,
  columns=flexible,
  numbers=none,
  commentstyle=\color{dkgreen},
  stringstyle=\color{blue},
  tabsize=3
}


\begin{document}

\title{\acronym: An Exemplar for\\ Self-Adaptive Underwater Vehicles
\thanks{This work was supported by the European Union’s Horizon 2020 Framework Programme through the MSCA network REMARO (Grant Agreement No 956200).}
}

\newcommand\red[1]{{\color{red}#1}}
\newcommand\blue[1]{{\color{blue}#1}}
\newcommand\acronym{SUAVE\xspace}
\newcommand\kw[1]{\texttt{{\small{#1}}}}

\author{%
	\IEEEauthorblockN{%
		Gustavo Rezende Silva\IEEEauthorrefmark{1},
        Juliane P{\"a}{\ss}ler\IEEEauthorrefmark{2},
        Jeroen Zwanepol\IEEEauthorrefmark{1},
        Elvin Alberts\IEEEauthorrefmark{3}\IEEEauthorrefmark{1}, 
        S. Lizeth Tapia Tarifa\IEEEauthorrefmark{2},\\
	    Ilias Gerostathopoulos\IEEEauthorrefmark{3},        
		Einar Broch Johnsen\IEEEauthorrefmark{2},
            Carlos Hern{\'a}ndez Corbato\IEEEauthorrefmark{1}
	}%
	\IEEEauthorblockA{\IEEEauthorrefmark{1}\textit{Technical University of Delft, Delft, The Netherlands}\\Email: \{g.rezendesilva,c.h.corbato\}@tudelft.nl}{j.m.zwanepol@student.tudelft.nl}%
    \IEEEauthorblockA{\IEEEauthorrefmark{2}\textit{University of Oslo, Oslo, Norway}\\Email: \{julipas,sltarifa,einarj\}@ifi.uio.no}%
	\IEEEauthorblockA{\IEEEauthorrefmark{3}\textit{Vrije Universiteit Amsterdam, Amsterdam, The Netherlands}\\Email: \{e.g.alberts,i.g.gerostathopoulos\}@vu.nl}%
    
}

\begin{comment}

\end{comment}

\maketitle

\begin{abstract}
  Once deployed in the real world, autonomous underwater vehicles
  (AUVs) are out of reach for human supervision yet need to take
  decisions to adapt to unstable and unpredictable environments. To
  facilitate research on self-adaptive AUVs, this paper presents
  \acronym, an exemplar for two-layered system-level adaptation of
  AUVs, which clearly separates the application and self-adaptation
  concerns. The exemplar focuses on a mission for underwater pipeline
  inspection by a single AUV, implemented as a ROS2-based system. This
  mission must be completed while simultaneously accounting for
  uncertainties such as thruster failures and unfavorable
  environmental conditions.  The paper discusses how \acronym can be
  used with different self-adaptation frameworks, illustrated by an
  experiment using the Metacontrol framework to compare AUV behavior
  with and without self-adaptation.  The experiment shows that the use
  of Metacontrol to adapt the AUV during its mission improves its
  performance when measured by the overall time taken to complete the
  mission or the length of the inspected pipeline.
\end{abstract}

\begin{IEEEkeywords}
exemplar, self-adaptation, robotics, underwater robots, Metacontrol, SUAVE
\end{IEEEkeywords}



\section{Introduction}
\label{sec:introduction}
% \begin{itemize}
%     % Diffusion of FL
%     \item {\st{Diffusion of FL}}
%     % Security threats to FL
%     \item {\st{Security threats to FL with particular focus on model poisoning}}
%     % Limitations of existing countermeasures
%     \item {\st{Current countermeasures (e.g., KRUM) and their limitations}}
%     % Proposed method and its advantages
%     \item {\st{Intuitive description of the proposed method and its difference (i.e., advantages) w.r.t. state of the art}}
%     % Main contributions
%     \item {\st{Summary of the main contributions of this work}}
%     % Paper's structure and organization
%     \item {\st{Paper's structure and organization}}
% \end{itemize}

% Diffusion of FL
Recently, {\em federated learning} (FL) has emerged as the leading paradigm for training distributed, large-scale, and privacy-preserving machine learning (ML) systems~\cite{mcmahan2017googleai,mcmahan2017aistats}. 
The core idea of FL is to allow multiple edge clients to collaboratively train a shared, global model without disclosing their local private training data.
%Specifically, an FL system consists of a central server and many edge clients; 
A typical FL round involves the following steps: {\em(i)} the server randomly picks some clients and sends them the current, global model; {\em(ii)} each selected client locally trains its model with its own private data; then, it sends the resulting local model to the server;\footnote{Whenever we refer to global/local model, we mean global/local model {\em parameters}.} {\em(iii)} the server updates the global model by computing an \emph{aggregation function}, usually the average (FedAvg), on the local models received from clients.
% \begin{enumerate}
%     \item[{\em(i)}] the server sends the current, global model to the clients and appoints some of them for training;
%     \item[{\em(ii)}] each selected client locally trains its copy of the global model with its own private data; then, it sends the resulting local model back to the server;\footnote{Whenever we refer to global/local model, we mean global/local model {\em parameters}.}
%     \item[{\em(iii)}] the server updates the global model by computing an \emph{aggregation function} on the local models received from clients (by default, the average, also referred to as FedAvg~\cite{mcmahan2017aistats}).
% \end{enumerate}
This process goes on until the global model converges. %(e.g., after a certain number of rounds or other similar stopping criteria).
%\\
% The advantages of FL over the traditional, centralized learning paradigm are undoubtedly clear in terms of flexibility/scalability (clients can join/disconnect from the FL network dynamically), network communications (only model weights\footnote{We will use \textit{parameters} and \textit{weights} interchangeably.} are exchanged between clients and server), and privacy (each client's private training data is kept local at the client's end and not uploaded to the server).
\\
% Security threats to FL
%However, the growing adoption of FL also raises security concerns~\cite{costa2022covert}, particularly about its confidentiality, integrity, and availability.
Although its advantages over standard ML, FL also raises security concerns~\cite{costa2022covert}. %, particularly about its confidentiality, integrity, and availability~\cite{costa2022covert}.
% OLD, LONG VERSION
% Indeed, some work deals with privacy leakage that may expose the local data of some clients~\cite{melis2019sp}. 
% A large body of work, instead, investigates attacks that usually aim to detriment the predictive accuracy of the learned global model. For instance, \emph{data poisoning} attacks achieve this goal by letting an adversary pollute the training set of some corrupt FL clients with maliciously crafted examples~\cite{jagielski2018sp}.
% Similarly, in \emph{model poisoning} the attacker attempts to tweak the global model weights~\cite{bhagoji2019pmlr} by directly perturbing the local model's weights of some infected FL clients before these are sent to the central server for aggregation, usually via so-called Byzantine attacks. 
% It turns out that Byzantine model poisoning attacks severely impact standard FedAvg; therefore, more robust aggregation functions must be designed to make FL systems secure.
Here, we focus on \emph{untargeted model poisoning} attacks~\cite{bhagoji2019pmlr}, where an adversary attempts to tweak the global model weights %\footnote{We will use the terms \textit{parameters} and \textit{weights} interchangeably.} 
by directly perturbing the local model's parameters of some infected clients before these are sent to the central server for aggregation.
In doing so, the adversary aims to jeopardize the global model \textit{indiscriminately} at inference time.
Such model poisoning attacks severely impact standard FedAvg; therefore, more robust aggregation functions must be designed to secure FL systems.
\\
% In this paper, we focus on designing a novel robust aggregation scheme at the server's end to contrast the effect of Byzantine model poisoning attacks.
%
% Current countermeasures and their limitations
%Several countermeasures have been proposed in the literature to combat model poisoning attacks on FL systems.
% Some methods use simple statistics more robust than plain average to smooth the impact of malicious updates (e.g., Trimmed Mean and FedMedian~\cite{yin2018icml}). 
% Other defenses implement outlier detection techniques to discard malicious updates from the aggregation performed at the server's end. Those are either based on heuristics (e.g., Krum/Multi-Krum~\cite{blanchard2017nips} and Bulyan~\cite{mhamdi2018pmlr}) or data-driven approaches (e.g., K-means clustering~\cite{shen2016acm} or DnC via spectral analysis~\cite{shejwalkar2021ndss}). 
% Finally, some strategies rely on a centralized ``source of trust'' to spot potential malicious updates (e.g., FLTrust~\cite{cao2020fltrust}).
% Several countermeasures have been proposed in the literature to combat model poisoning attacks on FL systems, i.e., to discard possible malicious local updates from the aggregation performed at the server's end. 
% These techniques range from simple statistics more robust than plain average (e.g., Trimmed Mean and FedMedian~\cite{yin2018icml}) to outlier detection heuristics (e.g., Krum/Multi-Krum~\cite{blanchard2017nips} and Bulyan~\cite{mhamdi2018pmlr}) or data-driven approaches (e.g., spectral analysis via K-means clustering~\cite{shen2016acm} or spectral analysis), or methods based on ``source of trust'' (e.g., FLTrust~\cite{cao2020fltrust}).
% OLD, LONG VERSION
%Several countermeasures have been proposed in the literature to combat Byzantine model poisoning attacks on FL systems.
% Descriptive statistics
% For example, Trimmed Mean and FedMedian aggregate local model updates using more robust statistics than standard average~\cite{yin2018icml}.
%
% % Heuristics for outlier detection
% Many existing Byzantine-resilient strategies implement some outlier detection heuristics to discard the model updates sent by potentially malicious clients from the input of the aggregation function.
% One of the most popular heuristics is Krum~\cite{blanchard2017nips}.
% This strategy tries to mitigate the impact of Byzantine attacks by selecting as a global model the local model with the smallest sum of Euclidean distances to {\em all} the other local models.
% Although powerful, Krum requires the server to know (or, at least, estimate) the number of malicious FL clients upfront, which is generally impossible in a realistic attack scenario. %
% Moreover, Krum may become ineffective for complex, high-dimensional model parameter spaces due to the curse of dimensionality.
% Bulyan~\cite{mhamdi2018pmlr} tries to overcome this issue by combining Krum with a variant of Trimmed Mean.
% % Data-driven outlier detection
% Other strategies use data-driven outlier detection techniques -- e.g., via K-means clustering~\cite{shen2016acm} -- to spot potential malicious local model updates. 
% %For instance, Shen et al. propose to cluster local model updates with K-means and thus identify outliers.
%
% % Other techniques
% As far as the server is concerned, any local model received can be from a potential malicious client. 
% FLTrust~\cite{cao2020fltrust} assumes the server acts as a client, i.e., trains a local model on an additional {\em trustworthy} dataset at the server's end and compares it against all the local models from other clients. 
% This way, the server can rely on some ``source of trust'' when discarding potentially malicious clients.
%\\
% Limitations of existing Byzantine-resilient strategies
Unfortunately, existing defense mechanisms either rely on simple heuristics (e.g., Trimmed Mean and FedMedian by~\cite{yin2018icml}) or need strong and unrealistic assumptions to work effectively (e.g., foreknowledge or estimation of the number of malicious clients in the FL system, as for Krum/Multi-Krum~\cite{blanchard2017nips} and Bulyan~\cite{mhamdi2018pmlr}, which, however, cannot exceed a fixed threshold).
Furthermore, outlier detection methods using K-means clustering~\cite{shen2016acm} or spectral analysis like DnC~\cite{shejwalkar2021ndss} do not directly consider the temporal evolution of local model updates received.
Finally, strategies like FLTrust~\cite{cao2020fltrust} require the server to collect its own dataset and act as a proper client, thereby altering the standard FL protocol.
\\
% OLD, LONG VERSION
% Overall, existing Byzantine-resilient strategies are either simple heuristics (e.g., FedMedian) or, if they are more complex, they rely on strong and unrealistic assumptions to work effectively (e.g., knowing the number of malicious clients in the FL system in advance, as for Krum and alike).
% Furthermore, data-driven outlier detection methods do not consider the temporary evolution of local model updates received (e.g., K-means clustering). 
% Finally, strategies like FLTrust requires the server to collect its own dataset and act as a proper client, thereby altering the standard FL protocol.
%
% Description of the proposed method
This work introduces a novel pre-aggregation \textit{filter} robust to untargeted model poisoning attacks. Notably, this filter $(i)$ operates without requiring prior knowledge or constraints on the number of malicious clients and $(ii)$ inherently integrates temporal dependencies. 
The FL server can employ this filter as a preprocessing step before applying \textit{any} aggregation function, be it standard like FedAvg or robust like Krum or Bulyan.
Specifically, we formulate the problem of identifying corrupted updates as a multidimensional (i.e., matrix-valued) time series anomaly detection task. 
The key idea is that legitimate local updates, resulting from well-calibrated iterative procedures like stochastic gradient descent (SGD) with an appropriate learning rate, show \textit{higher predictability} compared to malicious updates. This hypothesis stems from the fact that the sequence of gradients (thus, model parameters) observed during legitimate training exhibit regular patterns, as validated in Section~\ref{subsec:intuition}. %until convergence. 
%This regularity may be more pronounced for smooth convex loss functions, but it can still be captured within an appropriate time window, even for more complex and convoluted loss surfaces. 
%We provide evidence of this claim in Appendix~B, where we show that the average mutual information (i.e., ``predictability''), calculated over pairs of legitimate model updates sent at different FL rounds, is significantly higher than the corresponding computation for a malicious client.
\\
Inspired by the matrix autoregressive (MAR) framework for multidimensional time series forecasting~\cite{chen2021je}, we propose the FLANDERS ({\em \textbf{F}ederated \textbf{L}earning meets \textbf{AN}omaly \textbf{DE}tection for a \textbf{R}obust and \textbf{S}ecure}) filter.
The main advantages of FLANDERS over existing strategies like FLDetector~\cite{zhao2020multivariate} are its resilience to large-scale attacks, where $50\%$ or more FL participants are hostile, and the capability of working under realistic non-iid scenarios.
We attribute such a capability to two key factors: $(i)$ FLANDERS works without knowing a priori the ratio of corrupted clients, and $(ii)$ it embodies temporal dependencies between intra- and inter-client updates, quickly recognizing local model drifts caused by evil players. Below, we summarize our main contributions:

\begin{itemize}
\item[{\em(i)}]
We provide empirical evidence that the sequence of models sent by legitimate clients is more predictable than those of malicious participants performing untargeted model poisoning attacks.
\\
\item[{\em(ii)}] 
We introduce FLANDERS, the first pre-aggregation filter for FL robust to untargeted model poisoning based on multidimensional time series anomaly detection.
\\
\item[{\em(iii)}] 
We integrate FLANDERS into Flower,\footnote{\scriptsize{\url{https://flower.dev/}}} a popular FL simulation framework for reproducibility.
\\
\item[{\em(iv)}] 
We show that FLANDERS improves the robustness of the existing aggregation methods under multiple settings: different datasets, client's data distribution (non-iid), models, and attack scenarios.
\\
\item[{\em(v)}] 
We publicly release all the implementation code of FLANDERS along with our experiments.\footnote{\scriptsize{\url{https://anonymous.4open.science/r/flanders_exp-7EEB}}}
\end{itemize}

% Paper's structure and organization
The remainder of the paper is structured as follows. %some related work and the current state-of-the-art solutions to security issues that FL entails. 
Section~\ref{sec:background} covers background and preliminaries. 
In Section~\ref{sec:related}, we discuss related work.
Section~\ref{sec:problem} and Section~\ref{sec:method} describe the problem formulation and the method proposed. % to tackle it. 
Section~\ref{sec:experiments} gathers experimental results. %, and Section~\ref{sec:limitations} discusses some limitations of this work.
Finally, we conclude in Section~\ref{sec:conclusion}.
 %discusses the limitations of this work and draws future research directions.
%reports conclusions and draws perspectives for future research directions.

%%%%%%% OLD %%%%%%%
%to overcome the resilience of Byzantine failures in distributed Stochastic Gradient Descent computations. 
% The strength of Krum is its time complexity, which is linear in the gradient dimension. 
% However, the robustness of the approach is guaranteed for gradient-based learning applications only when the majority of the clients are not compromised. 
% Besides, the aggregation mechanism of Krum, as well as that of similar methods, is robust from a coarse-grained perspective and does not provide solutions to errors and perturbations that may occur at inference time.
%A related approach to~\cite{blanchard2017nips} is the work of Su et al.~\cite{su2016dc}. Here, the authors propose an iterated approximate agreement to tackle a multi-layer scenario attacked by Byzantine agents. 
%However, the method works efficiently on the sole discrete context and it is inapplicable to continuous state environments.
%\gabri{Maybe, we should just talk about the main limitations of existing countermeasures without digging into their details (or, we can just mention Krum as this is the most popular one). I will move the description of all these methods to the Related Work section.}
\section{Related work}
% There is extensive recent work on speeding up analytical queries due to the need for consistent execution times in the face of the explosive growth in the volume of available data.
% In this section, we divide existing work into two categories: maintaining data freshness in MVs (\Cref{sec:server_side}) and optimizations for minimizing ad-hoc query latency (\Cref{sec:client_side}).

% \subsection{Maintaining Data Freshness in MVs}
% \label{sec:server_side}
% There exists a variety of data warehousing applications aimed at supporting low-latency analytical queries on fresh data.
% In particular, these applications require efficiency in the propagation of newly ingested data into downstream MVs.
 
\mypara{Efficient MV Refresh}
Incremental view maintenance (IVM) aims to update MVs to reflect newly ingested data, taking advantage of already computed results to perform the update in a manner more efficient than computing from scratch (full refresh)
~\cite{ahmad2012dbtoaster,mcsherry2013differential,armbrust2013generalized,zeng2016iolap, palpanas2002incremental, griffin1995incremental, agiwal2021napa, braun2015analytics}. 
There is an abundance of work in IVM, including incremental updates on duplicate values~\cite{griffin1995incremental}, non-distributive aggregate functions~\cite{palpanas2002incremental}, higher-order views~\cite{ahmad2012dbtoaster}, and sliding windows~\cite{braun2015analytics}. 
More recent works also investigate the scalability aspect of IVM, proposing scale-independent updates~\cite{armbrust2013generalized} and sampled views~\cite{zeng2016iolap}. Since \system is applicable to arbitrary SQL statements, \system is orthogonal to and is fully compatible with existing IVM techniques.

\mypara{MV Refresh Scheduling}
There exist works on scheduling the refresh of a MV set focusing on resolving cyclic dependencies~\cite{folkert2005optimizing}, minimizing weighted average staleness~\cite{golab2009scheduling}, and prioritizing MVs with the highest speedups on predicted future queries~\cite{ahmed2020automated}.
\system's scheduling to speed up the end-to-end refresh of the MV set is not addressed in existing works.

\mypara{DAG Workflow Scheduling}
The execution of workloads consisting of individual jobs with acyclic dependencies is a well-studied topic~\cite{apacheoozie,sparkdag,marchal2018parallel,bathie2020revisiting,baruah2022ilp}; many of these techniques can be applied to MV refresh runs studied in this paper.
Existing workflow scheduling systems such as Apache Oozie~\cite{apacheoozie}, Apache Airflow~\cite{airflow}, and Spark DAG scheduler~\cite{sparkdag} automate the execution of user-defined workflows following a topological order.
There are recent works aimed at finding more optimal execution orders in terms of peak memory usage~\cite{marchal2018parallel, bathie2020revisiting} and execution time on parallel platforms~\cite{baruah2022ilp}.
While \system is designed for use with MV refresh runs/workloads, our technique on joint scheduling and optimization can be reasonably applied to general workloads as a possible future direction.

% \paragraph{Incremental MV indexing}
% Update-optimized indices such as the log-structured merge-trees (LSM)~\cite{o1996log} are used for indexing MVs due to frequent updates induced by data ingestion~\cite{gupta2016mesa,agiwal2021napa}.
% \system is orthogonal to indexing: \system is capable of efficiently performing MV refresh runs regardless of whether the individual MVs are indexed or not.

% \subsection{Ad-hoc Query Latency Reduction}
% \label{sec:client_side}

% The minimization of ad-hoc analytical query response times is a well-studied topic due to latency being negatively correlated with the productivity of a data analyst during a data analysis session~\cite{liu2014effects}.
% Sessions are commonly conducted within visualization systems that contain a variety of optimization techniques to ensure that query response times fall within a certain latency tolerance.

% \mypara{Data prefetching}
% Data is often loaded into memory on a by-need basis in visualization systems to minimize interference with user-issued query computations~\cite{mani2017effective,xin2021enhancing,galakatos2017revisiting, yan2020auto, battle2016dynamic, crotty2016case, jalaparti2018netco}. 
% Query-time data retrieval can be significantly expedited by anticipating the data usage of the user in future queries and pre-loading the data into memory during the downtime between user queries (`think time'). SMART~\cite{mani2017effective} prefetches data for modified versions of current user-issued queries with different filters and dimensions. A-WARE~\cite{crotty2016case} maintains a sample store constantly refined through ingesting data based on speculations of future plots.
% ForeCache~\cite{battle2016dynamic} uses an SVM to predict the user's current analysis phase and accordingly prefetches data tiles partitioned based on different numerical values. NetCo predicts future queries via log analysis, and solves an ILP formulation to prefetch data to maximize the number of SLO-meeting queries~\cite{jalaparti2018netco}.
% In the case of MV refresh workloads, `think time' is nonexistent as individual MVs are refreshed back-to-back, rendering data prefetching techniques non-applicable.

\mypara{Intermediate Data Caching}
Some existing data visualization systems cache user-defined variables to support the typical incremental construction of data visualizations~\cite{zgraggen2016progressive, eichmann2020idebench} during data analysis sessions~\cite{jupyter, rstudio, colab}. 
Recent work proposes a management scheme for these cached variables under a memory constraint that greedily keeps variables with the highest estimated time savings based on predicted future user behavior via neural networks~\cite{xin2021enhancing}.
While useful for data visualization, a greedy approach to memory management fails to achieve satisfactory results compared to \system.

\mypara{Intermediate Result Reuse}

There exist works on storing intermediate results from computations to speedup future computations~\cite{yang2018intermediate, dursun2017revisiting, nagel2013recycling, michiardi2019memory, galakatos2017revisiting}.
Studied topics include the identification of reuse opportunities by finding overlaps in computation graphs of successive jobs~\cite{yang2018intermediate, michiardi2019memory},
selective storage under a space constraint with heuristics such as reuse probability~\cite{dursun2017revisiting}, expected savings~\cite{yang2018intermediate}, and recompute-storage cost difference~\cite{nagel2013recycling},
and rewriting incoming jobs to take advantage of stored intermediates~\cite{galakatos2017revisiting}.
These works share similarity with \system in their selection of items to store under a memory constraint, however, \system's problem setting requires it to uniquely consider the joint (re)ordering of job executions along with the selection of items.

% work that considers both job execution (re)order as well as intermediate result caching with a bounded amount of memory. but notably lack the joint aspect of \system and cannot be used to achieve immediate speedup on an incoming MV refresh run if no intermediates are stored beforehand. 

\mypara{Incremental Query Processing} Incremental processing (IQP) is useful for cases where not all data required for a query is immediately available. Similar to online aggregation~\cite{hellerstein1997online}, initial results of a query are computed on a subset of required data and progressively refined as the rest of the required data arrives in a predictable pattern~\cite{tang2019intermittent,wangtempura}. Tang et al. propose a dynamic programming formulation to pick intermediate states to store in memory given a limited memory budget~\cite{tang2019intermittent}. Tempura rewrites the query plan for more efficient execution based on predicted data arrival patterns~\cite{wangtempura}. While similarities exist between the problem setting of IQP and \system, such as management of bounded memory, \system notably includes additional joint optimization for the order of MV updates.

% \paragraph{Sampling}
% Sampling has seen wide use in visualization systems for reducing the computation time of ad-hoc queries by computing an approximate result over a subset of data as exact results are not always required by the user~\cite{crotty2016case, mani2017effective, zgraggen2014panoramicdata, kraska2021northstar, galakatos2017revisiting, kandula2016quickr}. 
% Commonly studied topics in sampling for ad-hoc queries include complex query sampling~\cite{kandula2016quickr}, rare event aggregation~\cite{kraska2021northstar, galakatos2017revisiting}, and maintaining consistency between related sampled visualizations~\cite{zgraggen2014panoramicdata}.
% Sampling server-side at the MV level compromises the assumptions of downstream applications and is thus not considered in \system.

% \paragraph{Progressive visualization}
% The latency tolerance for time-consuming queries can be circumvented by presenting a partially-computed visualization to the user within the tolerance, which is then incrementally refined until it is fully accurate~\cite{rahman2017ve, zgraggen2016progressive, crotty2015vizdom, kraska2021northstar, kamat2017infiniviz}.
% Example plots which benefit from progressive visualization include bar charts~\cite{kamat2017infiniviz} and heatmaps~\cite{rahman2017ve}.
% Similar to sampling, study on this topic is orthogonal to \system as pushing out partially-updated MVs compromises downstream assumptions.
\section{Pipeline inspection exemplar}
\label{sec:problem_statement}
This section describes the use case and system architecture, the two
system layers are detailed in Sections~\ref{sec:managed} and
\ref{sec:managing_subsystem}.

\subsection{Use case description}
\label{sec:use_case_description}
The use case in this exemplar is about an AUV inspecting pipelines located on a seabed.
Its mission consists of two sequential tasks, $(T1)$~searching for the pipeline, then $(T2)$~simultaneously following and inspecting the pipeline. 


When performing its mission, the AUV is subject to two sources of
uncertainty that could trigger self-adaptation: $(U1)$~thruster
failures and $(U2)$~changes in water visibility. $U1$~arises from the
possibility of the AUV's thrusters failing at runtime, which may cause
the AUV to move unexpectedly. This is relevant for both $T1$ and $T2$. To overcome $U1$, the managed subsystem of the AUV
contains functional alternatives. When
one or more thrusters fail, it is possible to enter a recovery state
in which the thrusters are recovered. 
$U2$
influences the maximum distance at which the AUV can visually perceive
objects. This is relevant for $T1$, higher water visibility
allows the AUV to search for the pipeline at higher altitudes,
resulting in a larger field of view and the possibility of discovering the pipeline faster. On the other hand, if the water
visibility is low, the AUV has to move closer to the seabed to
search for the pipeline, which limits its field of view and therefore
may lead to a longer time to discover the pipeline. Thus, changing the
altitude of the AUV provides functional alternatives for dealing with $U2$.

This exemplar focuses on the problem of overcoming $U1$ and $U2$ using
a self-adaptation logic, implemented by a managing subsystem, that
can be extended and reused for other sources of uncertainty.
The managing subsystem shall overcome $U1$ by recovering the failed
thrusters at runtime, and $U2$ by adapting the maximum altitude for
the path generator algorithm according to the measured water
visibility.  Thus, by reacting to $U1$ and $U2$, the managing
subsystem increases the reliability and performance of the system.

For the feasibility of the exemplar, the use case was simplified while still allowing for a worthwhile application of self-adaptation to an AUV.  
 It is important to highlight that a
realistic operation of an AUV used for pipeline inspection would include
steps that are related to pre-dive, launching and recovery, human
interaction, and intermediary missions that are necessary to enable
the inspection.  Furthermore, there are several sources of uncertainty
not considered here, including ocean dynamics, sensor failures, and
battery duration.

\subsection{System Architecture}

To accomplish the mission described in Section~\ref{sec:use_case_description}, the managed subsystem requires the
functions represented in Fig.~\ref{fig:func_arch}. $T1$ requires the
functions \kw{Control Motion}, \kw{Maintain Motion},
\kw{Localization}, \kw{Detect Pipeline}, \kw{Generate Search
  Path}, and \kw{Coordinate Mission}, while $T2$ requires the functions \kw{Control Motion},
\kw{Maintain Motion}, \kw{Localization}, \kw{Detect Pipeline},
\kw{Follow Pipeline}, \kw{Inspect Pipeline}, and \kw{Coordinate Mission}. During runtime, the
functions must be activated and deactivated according to the task
being performed.


\begin{figure}[h]
    \centering
    \includegraphics[width=0.8\linewidth]{figures/FunctionalHierarchy.pdf}
    \caption{Managed Subsystem's Functional Hierarchy}
    \label{fig:func_arch}
\end{figure}
%

To overcome the uncertainties $U1$ and $U2$, a managing subsystem
requires the functionalities to \emph{monitor} the environment and the
managed subsystem's internal state, \emph{reason} about it, and
\emph{execute} the managed subsystem's reconfiguration.


\begin{figure}
    \centering
    \includegraphics[width=\linewidth]{figures/GeneralArchitecture_small.pdf}
    \caption{System Architecture}
    \label{fig:system_architecture}
\end{figure}

The required functions of the managed and managing subsystems
are realized as depicted in Fig.~\ref{fig:system_architecture}. The
managed subsystem is detailed in Section~\ref{sec:managed} and the
managing subsystem in Section~\ref{sec:managing_subsystem}.
It is important to mention that managed subsystem functions
\kw{Control Motion} and \kw{Localization} are achieved by \kw{ArduSub}, and the function \kw{Inspect Pipeline} is not realized since
the actual inspection of the pipeline is not the focus of this
work. It is also important to highlight that this exemplar implements
the function to \emph{reason} about the managing subsystem with Metacontrol to provide a
baseline for future research. However, it can be replaced with other
solutions, as long as they are compatible with the monitor and execute
interfaces, as described in Section~\ref{sec:managing_requirements}.

\section{Managed Subsystem} \label{sec:managed}

The managed subsystem is implemented as a ROS2-based system and is depicted in Fig.~\ref{fig:system_architecture}. The only non-ROS2 component is \kw{ArduSub}\footnote{\url{https://www.ardusub.com/}}, which is an open-source autopilot for underwater vehicles. In this application it is used to solve the functions \kw{Control Motion} and \kw{Localization}\footnote{It is assumed that the AUV has appropriate sensors for localization}. The \kw{MAVROS}
package works as a bridge between \kw{ArduSub} and the ROS2 components. The \kw{Detect Pipeline} node detects the pipeline and informs \kw{Follow Pipeline} and the \kw{Coordinate Mission} node about its position\footnote{A mock perception system is used.}. The \kw{Coordinate Mission} node coordinates the tasks' execution and sets the adaptation goals. Note that the function \kw{Inspect Pipeline} is not implemented,
since the actual inspection of the pipeline is not the focus of this work. However, the exemplar can easily be extended with this functionality by adding a new node that implements the pipeline inspection.

\kw{Follow Pipeline}, \kw{Generate Search Path},  and \kw{Maintain Motion} are lifecycle nodes, which means that they have internal states, such as \emph{active} and \emph{inactive}, and it is possible to switch between these states at runtime. Furthermore, the System Modes package \cite{nordmann2021system} 
extends the state \emph{active} with additional modes, e.g., \emph{active.low\_altitude}.

To adapt the managed subsystem, the managing subsystem adapts the lifecycle nodes by changing their states. This is done by the \kw{Mode Manager} node, which is used off-the-shelf from the System Modes package. 
The states available for \kw{Generate Search Path} are \kw{deactivated}, \kw{low altitude}, \kw{medium altitude}, and \kw{high altitude}. Subsequently, the states available for \kw{Follow Pipeline} are \kw{deactivated} and \kw{activated}, while the states for \kw{Maintain Motion} are \kw{deactivated}, and \kw{recover thrusters}.

To enable other developers of self-adaptive systems to use this exemplar and compare different approaches, a \kw{Gazebo}-based  \footnote{\url{https://gazebosim.org/home}} simulation of a pipeline inspection environment and a model of the AUV is provided. The BlueROV2\footnote{\url{https://bluerobotics.com/store/rov/bluerov2/}}
robot was selected as the AUV for the exemplar because (i)~it is compatible with \kw{ArduSub}; (ii)~it is easily integrated with \kw{Gazebo} via plugins; and (iii)~the robot has a low price compared to other available AUVs, making it more accessible to researchers to reproduce the exemplar with a real robot.













\section{Managing Subsystem}
\label{sec:managing_subsystem}

The managing subsystem exploits functional alternatives of the managed
subsystem to enable adaptation and thereby increase system reliability.
Metacontrol is used as an example of how a managing
subsystem can be implemented. This section introduces
Metacontrol
and shows how the adaptation problem can be
formulated and implemented with Metacontrol.

\subsection{Metacontrol Background}
\label{subsec:metacontrol}
\emph{Metacontrol} uses the MAPE-K feedback loop \cite{brun09,kephart03computer} to implement self-adaptation. It \emph{Monitors} the managed subsystem during runtime, \emph{Analyzes} whether the system meets its requirements, \emph{Plans} a new configuration if the system does not meet the requirements, and then \emph{Executes} the reconfiguration of the managed subsystem. All this is done using a shared \emph{Knowledge Base} to which each step refers. In Metacontrol, the knowledge base conforms to the TOMASys (Teleological and Ontological Metamodel for Autonomous Systems) metamodel \cite{corbato2013model}. 

\begin{figure}
    \centering
    \includegraphics[width=0.4\textwidth]{figures/TOMASys.pdf}
    \caption{A simplified representation of the TOMASys elements}
    \label{fig:tomasys}
\end{figure}

A simplified version of the TOMASys metamodel is displayed in Fig.~\ref{fig:tomasys}.
TOMASys uses \emph{functions} $F$ to represent the functionalities of the system, e.g., generating a search path for the AUV.
The architectural variants that implement these requirements are captured by \emph{function designs} $FD(F, \mathcal{C}, \qa^{exp})$. To distinguish during runtime which function design is most suited in a given situation, a set $\qa^{exp}$ of \emph{expected quality attributes} is associated with it. An expected quality attribute value reflects how well a function design is supposed to fulfill the function $F$ it solves. Furthermore, a function design requires a set $\mathcal{C}$ of \emph{components} of the managed subsystem to solve $F$. A component $C(S_C)$ is a piece of hardware or software, e.g., a sensor or a path-planning algorithm, respectively. The status $S_C$ of a component indicates its availability, i.e., whether it is functioning or not.

An \emph{objective} $O(F,S_O,\qa^{req})$ is a runtime instantiation of a function $F$, e.g., generating a search path with a minimum required water visibility, whose status $S_O$ reflects whether the objective is currently achieved. Furthermore, the set $\qa^{req}$ of \emph{required quality attributes} specifies which quality attribute value the objective requires in order to work properly.

An objective $O$ is solved by a \emph{function grounding} $FG(O,FD,S_{FG}, \qa^{meas})$, which represents the function design $FD$ that is currently used to solve the objective. 
Its status $S_{FG}$ reflects whether the function grounding is currently able to achieve the objective. The set $\qa^{meas}$ of \emph{measured quality attributes} reflects how well the function grounding currently fulfills $O$ and is computed using sensor data.


\subsection{Metacontrol Formulation}
\label{sec:metacontrol_formulation}
The functions, architectural variants, and quality attributes required to solve the tasks $(T1)$ Search Pipeline and $(T2)$ Inspect Pipeline, described in Section \ref{sec:problem_statement}, are modeled conforming to the TOMASys metamodel. Table~\ref{tbl:functions_exemplar} specifies the functions ($F_1$)~maintain\_motion, ($F_2$)~generate\_search\_path, and ($F_3$)~follow\_pipeline, while Table~\ref{tbl:qas_exemplar} describes the quality attributes ($QA_1$)~water\_visibility, and ($QA_2$)~performance. 
Functions $F_1$ and $F_2$ are required to achieve $T1$, whereas $T2$ is achieved by $F_1$ and $F_3$.
The function designs that solve these functions are specified in Table~\ref{tbl:function_designs_exemplar}. The set of required components is empty for function designs $FD_2 - FD_6$ because they do not require any components that are susceptible to adaptation, or used in the reasoning process.

Since the objectives and function groundings are instantiated during runtime, they are not specified here. An objective for function $F_2$ is for example to generate a search path with no required quality attribute, which is defined as $O_2(F_2, \text{\emph{ok}}, \varnothing)$ in the notation introduced above. A possible function grounding for this objective is $FG_2(O_2, FD_4, \text{\emph{ok}}, \{QA_1^{meas}=1.1\})$.


\begin{table}
    \centering
    \caption{The TOMASys functions used for the exemplar}
    \label{tbl:functions_exemplar}
    \begin{tabular}{|@{\;}l@{\;}|@{\;}l@{\:}|p{4.7cm}@{\:}|}
        \hline
        \textbf{Function} & \textbf{Name} & \textbf{Requirement} \\
        \hline
        \rule{0pt}{2ex} 
        $F_1$ & maintain\_motion & Maintain the motion of the robot\\
        \hline
        \rule{0pt}{2ex} 
        $F_2$ & generate\_search\_path & Generate a path to search for the pipeline\\
        \hline
        \rule{0pt}{2ex} 
        $F_3$ & follow\_pipeline & Follow and inspect the pipeline\\
        \hline
    \end{tabular}
\bigskip
\caption{The TOMASys quality attributes used for the exemplar}
    \label{tbl:qas_exemplar}
    \begin{tabular}{|l|l|l|p{3.9cm}|}
        \hline
        \rule{0pt}{2ex} 
        \textbf{QA} & \textbf{Name} & \textbf{Unit} & \textbf{Description}\\
        \hline
        \rule{0pt}{2ex} 
        $QA_1$ & water\_visibility & [$0,\infty)$ & Reflects the maximum altitude (in meters) from which the AUV can perceive the seabed\\
        \hline
        \rule{0pt}{2ex} 
        $QA_2$ & performance & $[0,1]$ & Reflects how efficient the current search strategy is\\
        \hline
    \end{tabular}
\bigskip
\caption{The TOMASys function designs used for the exemplar}
    \label{tbl:function_designs_exemplar}
    \begin{tabular}{|@{\;}p{3.5cm}@{\;\,}|@{\;}p{2.0cm}@{\,}|p{2.7cm}@{\;}|}
        \hline
        \textbf{Function Design} & \textbf{Name} & \textbf{Description} \\
        \hline
        \rule{-3pt}{2ex} 
        $FD_1(F_1,\{\text{thruster}\_x\mid x=1,\dots,6\}, \{QA_2^{exp}=1\})$ & all\_thrusters & Uses all thrusters \\
        \hline
        \rule{-3pt}{2ex} 
        $FD_2(F_1, \varnothing, \{QA_2^{exp}=0.5)$ & recover\_thrusters & Recovers the thrusters that are in failure\\
        \hline
        \rule{-3pt}{2ex} 
        $FD_3(F_2, \varnothing, \{QA_1^{exp}=0.5,$ $ QA_2^{exp}=0.25\})$ & spiral\_low & Produces a spiral search path with low altitude \\
        \hline
        \rule{-3pt}{2ex} 
        $FD_4(F_2, \varnothing, \{QA_1^{exp}=1,$ $ QA_2^{exp}=0.5\})$ & spiral\_medium & Produces a spiral search path with medium altitude \\
        \hline
        \rule{-3pt}{2ex} 
        $FD_5(F_2, \varnothing, \{QA_1^{exp}=2,$ $ QA_2^{exp}=1\})$ & spiral\_high & Produces a spiral search path with high altitude \\
        \hline
        \rule{-3pt}{2ex} 
        $FD_6(F_3, \varnothing, \varnothing)$ & follow\_pipeline & Follows the pipeline \\
        \hline
    \end{tabular}
\end{table}

The MAPE-K loop steps in this exemplar are formulated as follows. The monitor step is responsible for measuring $QA_1^{meas}$ and for monitoring the state of the six thrusters. 

The analyze step uses Horn rules
to reason about the knowledge base. One example rule that analyzes whether the measured water visibility $QA_1^{meas}$ still satisfies the expected water visibility $QA_1^{exp}$ of the grounded function design is displayed in Fig.~\ref{fig:swrl_rule}. Note that it is written in terms of the notation introduced in Section~\ref{subsec:metacontrol}. Line~\ref{eq:fg} expresses that the rule reasons about a function grounding $FG$ that solves an objective $O$, is of type $FD$, has a status $S_{FG}$ and an associated set of measured quality attributes $\qa^{meas}$. Furthermore, the function design $FD$ solves the function $F$, has a set of required components $\mathcal{C}$ and an associated set of expected quality attributes $\qa^{exp}$. Note that it is implicitly assumed that $FG$ is well-formed, i.e., that the function of which $O$ is a type of is the same as the function that $FD$ solves. Since this rule should analyze the water visibility, Line~\ref{eq:qas_contained} ensures that $QA_1^{meas}$ is an element of the set $\qa^{meas}$ and that $QA_1^{exp}$ is an element of the set $\qa^{exp}$, i.e., that both $FG$ and $FD$ are related to water visibility. Finally, if the measured value of $QA_1$ is less than its expected value associated with the grounded function design, see Line \ref{eq:meas_less_exp}, then the status of the function grounding is set to \emph{error}, see Line~\ref{eq:status_error}.

\begin{figure}[t]
    \centering
    \begin{align}
    &FG\bigl(O, FD, S_{FG}, \qa^{meas}\bigr) \boldsymbol{\wedge} ~ FD(F, \mathcal{C}, \qa^{exp}) \label{eq:fg}\\
    %&\boldsymbol{\wedge} ~ F = G \label{eq:same_function}\\
    & \boldsymbol{\wedge} ~ QA_1^{meas} \in \qa^{meas}
    ~\boldsymbol{\wedge} ~ QA_1^{exp} \in \qa^{exp} \label{eq:qas_contained}\\
    &\boldsymbol{\wedge} ~ QA_1^{meas} < QA_1^{exp} \label{eq:meas_less_exp}\\
    &\boldsymbol{\Rightarrow} ~ S_{FG} = \text{\emph{error}} \label{eq:status_error}
\end{align}
\caption{Rule to analyze whether the measured water visibility $QA_1^{meas}$ still satisfies the expected water visibility $QA_1^{exp}$ of the grounded function design}
\label{fig:swrl_rule}
\end{figure}

In the planning step, the function designs with $QA_1^{exp}$ higher than $QA_1^{meas}$ are filtered out as the visibility they would expect is not measured, afterward the remaining function design with the highest expected search performance ($QA_2^{exp}$) is selected as the desired configuration. The selected configuration is then carried out in the execute step.

\subsection{Metacontrol Implementation}

As depicted in Fig.~\ref{fig:system_architecture}, the monitor step is implemented with the \kw{Water Visibility Observer} and the \kw{Thruster Monitor} nodes. They are used for measuring $QA_1$ and monitoring the status of the six thrusters $\text{thruster}\_x$ where $x\in\{1,\dots,6\}$, respectively. 
To simplify the system and avoid the addition of unnecessary nodes, instead of adding water visibility to the Gazebo simulator, the \kw{Water Visibility Observer} simulates water visibility measurements with a sine function, and instead of probing the managed subsystem to identify thruster failures the \kw{Thruster Monitor} simulates the thruster failures events. Since the monitor step is mocked up, and its probes and intermediary nodes that would be required to provide the probes are not implemented, they are not included in Fig. \ref{fig:system_architecture}. Both nodes publish their data into the \kw{/diagnostics} topic with the ROS2 default \kw{DiagnosticArray} message type. 


The knowledge base (KB), the analyze and plan step are implemented using MROS2~\footnote{\url{https://github.com/meta-control/mc_mros_reasoner}}~\cite{corbatoMROSRuntimeAdaptation2020}, a ROS2-based Metacontrol implementation, as the \kw{MROS Reasoner} node. The KB
is implemented with the Ontology Web Language (OWL)~\cite{Antoniou2004}, the Horn rules used for the analyze step with the Semantic Web Rule Language (SWRL)~\cite{horrocks2004swrl}, and the reasoning is done with Pellet\footnote{\url{https://github.com/stardog-union/pellet}}. The \kw{MROS Reasoner} receives water visibility measurements ($QA_1^{meas}$) and thruster status information from the monitor step, then decides whether adaptation is required, and, in this case, selects a desired configuration which it sends to the execute step (see Section \ref{sec:metacontrol_formulation} for more details). The \kw{MROS Reasoner} initially does not have objectives, so it does not perform adaptation. The adaptation reasoning only starts when the \kw{Coordinate Mission} node sends new objectives, such as $O_2(F_2, null , \varnothing)$, via the \kw{Adaptation Goal Bridge}. New objectives do not have a status yet.

The execute step uses the System Modes' \kw{Mode Manager} to adapt the managed subsystem, and the \kw{System Modes Bridge} bridges the \kw{Mode Manager} with the \kw{MROS Reasoner}. When a reconfiguration is needed, the \kw{MROS Reasoner} requests the new configuration via the \kw{/mros/request\_configuration} service to the \kw{System Modes Bridge}.Then the \kw{System Modes Bridge} forwards the request to the \kw{Mode Manager} using the correct service names, depending on the lifecycle node being adapted. The services used by the \kw{Mode Manager} are listed in Table~\ref{tbl:system_modes_services}, and the available modes are listed in Table~\ref{tbl:system_modes_modes}.


\begin{table}
    \centering
    \caption{Available System Modes' services}
    \label{tbl:system_modes_services}
    \begin{tabular}{|p{3cm}|p{4.4cm}|}
        \hline
        \textbf{Node} & \textbf{Service} \\
        \hline
        f\_generate\_search\_path  & /f\_generate\_search\_path/change\_mode   \\
        \hline
        f\_follow\_pipeline  & /f\_follow\_pipeline/change\_mode   \\
        \hline
        f\_maintain\_motion  & /f\_maintain\_motion/change\_mode   \\
        \hline
    \end{tabular}
    \bigskip
    \caption{Available modes}
    \label{tbl:system_modes_modes}
    \begin{tabular}{|p{3cm}|p{2.5cm}|c|}
        \hline
        \textbf{Node} & \textbf{Mode} & \textbf{Lifecycle state} \\
        \hline
        f\_generate\_search\_path & fd\_spiral\_high   & active               \\
        \hline
        f\_generate\_search\_path & fd\_spiral\_medium   & active             \\
        \hline
        f\_generate\_search\_path & fd\_spiral\_low   & active              \\
        \hline
        f\_generate\_search\_path & fd\_unground   & inactive  \\
        \hline
        f\_follow\_pipeline & fd\_follow\_pipeline   & active  \\
        \hline
        f\_follow\_pipeline & fd\_unground   & inactive  \\
        \hline
        f\_maintain\_motion & fd\_all\_thrusters   & inactive  \\
        \hline
        f\_maintain\_motion & fd\_recover\_thrusters   & active  \\
        \hline
    \end{tabular}
\end{table}


\section{Extending and connecting managing subsystems}\label{sec:managing_requirements}

With the described system implementation, the only Metacontrol-specific nodes of the system are the \kw{MROS Reasoner}, the \kw{System Modes Bridge}, and the \kw{Adaptation Goal Bridge}. All other nodes of the system can be reused with different managing subsystems. The only requirement to connect a managing subsystem to the managed subsystem is to ensure that the managing subsystem adheres to the provided monitor and execute ROS2 interfaces. As described in the previous section, the monitor interface is the ROS2 topic \kw{/diagnostics}, and the execute interfaces are listed in Table~\ref{tbl:system_modes_services}. To show that changing the managing subsystem is possible, a managing subsystem that randomly picks a configuration was also implemented.

Since the system is implemented with a modular design, it can be extended with additional functionalities and adaptation scenarios by adding new lifecycle nodes and updating the system modes' configuration file accordingly. The implemented functionalities can be replaced with different implementations as long as they adhere to the same interfaces; e.g., the \kw{Pipeline Detection} node could be replaced by a node that actually performs perception instead of a mock-up.



\section{Results}
\label{results}

\begin{figure*}[ht]
    \centering
    \includegraphics[scale=0.15,trim={0 2.5cm 0 5cm},clip]{images/aoi-single_burst}
    \caption{The time average peak Age of Information with burst and \gls{soa} loss values against the dynamic reliability logic for different network topologies.}
    \label{fig:aoi_burst}\vspace{-0.4cm}
\end{figure*}


This paper focuses on both transport layer and application layer metrics to determine the feasibility of dynamic reliability. For this, we have selected the session packet volume, as transmitted, retransmitted, lost and backlogged packets as \glspl{kpi} for the transport layer; while focusing on the \gls{aoi} for the application layer. The \gls{aoi} was chosen as a crucial indicator for the freshness of packets in real-time applications. More specifically, this work adopts the time average peak \gls{aoi} equation \cite{aoi_equation} depicted in Eq. \ref{aoi}, where $\Delta(r_{i+1})$ is the $i$th update at the time it was received at the server, for a session time period of $\tau$.

\begin{equation}
    \label{aoi}
    \gls{aoi}_\tau = \frac{1}{n-1}\sum_{i=1}^{n-1} \Delta(r_{i+1})
\end{equation}

We include a comparison between the vanilla QUIC implementation which does not enjoy the dynamic reliability extension, with a number of dynamic reliability policies. The tests were run a number of times for statistical significance, with the mean value of vanilla implementation used as a baseline for comparison. The topology utilised both random loss and bursty loss to explore the bounds of dynamic reliability. The \gls{soa} loss in the figures correspond to the loss values presented in Table. \ref{tab:path_char}, for ease of comparison between bursty and random loss scenarios.

\subsection{Transport-Layer KPIs}

To analyse the performance gain at the transport layer due to dynamic reliability, the volume of transmitted and backlogged packets is examined. The figures are in the form of boxplots, which take the vanilla implementation as a benchmark, depicted as the red dashed line.

As seen in Fig. \ref{fig:sent_burst}, the loss plays a crucial role in the performance of the reliability policies. The policies under random loss did incredibly well for the networks with a larger capacity, namely \gls{mmwave} and Sub-6~GHz, whereas for burst loss, the lower network capacities had a larger packet reduction. With the increase in burst loss, the behaviour of the set split reliable policies became unpredictable, if a reliable assignment happened to coincide with a burst loss, the number of transmitted packets increases, and vice versa. On the other hand, in smarter policies, such as Loss-Aware, the performance lightly matched the vanilla baseline, as the reliable assignment dominated the session to compensate for a higher burst loss. Not only that but, the burst loss also impacted the variance of the transmitted packets for the policies.

Unsurprisingly, the unreliable focused policy, 80-20 split, outperformed other policies for all topologies in random and bursty loss scenarios, with an approximate reduction of 80\%. That being said, the majority of the policies reduced the transmitted packets on the link by approximately 70\% for random loss, while the reduction started at $\approx 15\%$ and decreased as the loss increased for the burst loss scenario.

The retransmitted and lost packets, not shown due to space limitations, followed the same trend as the transmitted packets for the random loss scenarios. However, for the burst loss scenarios, the larger capacity networks had a lower reduction in the retransmitted and lost packets. This can be seen as a favorable outcome since the lower capacity networks are scarce on resources. It is important to note that the Loss-Aware policy mimicked the vanilla approach as the burst loss increased, signifying the overwhelming appointment of reliable packets in adapting to the harsh burst loss conditions.
 
Alternatively, Fig. \ref{fig:backlog_burst} clearly shows a stark comparison between the policies and loss scenario in the reduction of the backlogged packets. The Loss-Aware policy for random loss scenario reduced the backlogged packets by up to 50\%, beating all other policies by approximately 30\%. Furthermore, it is clear that the unreliability focused policies resulted in the lowest backlog for the session. In comparison, we notice that the burst loss and the backlogged frequency have a positive correlation, where the maximum reduction of the backlogged packets for the policies is at most 20\%. Much like the transmitted packets, the probability of a burst loss occurrence plays a vital role in the number of retransmissions sent and by extension the number of backlogged packets. Thus, we can conclude that the stress placed on the buffer is a result of the reliable packets which is tightly coupled with the congestion on the session. Whereas, unreliable focused policies did not encounter such a phenomenon regardless if it was experiencing a burst loss.


\subsection{Application-Layer KPIs}

The feasibility of dynamic reliability for real-time applications can be determined by the \gls{aoi}, with comparison across different topologies and policies. If we take a strict approach and consider anything below $10$~ms is real-time \cite{real-time}, then all the reliability policies passed that requirement, which is attractive for real-time applications, as shown in Fig. \ref{fig:aoi_burst}. Utilising the median as an estimate of the runs, the policies in the WLAN and Sub-6~GHz topology with random loss floated around $4-5$~ms with negligible difference, while the \gls{aoi} for \gls{mmwave} was $\approx 2-3$~ms. It is clear that the \gls{aoi} and the network capacity have a negative correlation, as the network capacity decreases, the \gls{aoi} increases. The same correlation is extended to the bursty loss scenarios, where \gls{mmwave} dominated the other topologies. That being said, it is crucial to note that the \gls{aoi} for the reliability policies is often slightly better than or equal to the \gls{aoi} of the vanilla implementation, proving that dynamic reliability reduces the congestion of the session at no cost to the \gls{aoi}.

\section{Conclusion}\label{sec:conclusion}
In this work, we focus on addressing the fundamental challenge of OOD detection tasks, which is how to fully understand the semantic discrepancy between the ID/OOD samples. We reveal that the key to success in the realistic SCOOD task is to allocate as many ID samples in the unlabeled set correctly as possible. To this end, we propose a novel uncertainty-aware optimal transport scheme that introduces class-specific energy scores as guidance for effective label assignment. Experimental results show that our method achieves better performance than previous state-of-the-art methods on SCOOD benchmarks.

\textbf{Limitations.} In addition to temperature scaling, other techniques such as feature clipping applied in ReAct~\cite{sun2021react} also enhance the performance of energy score, so how to obtain an OOD score that best fits the SCOOD task can be further explored. Moreover, a setting highly related to SCOOD has been proposed in \cite{katz2022training} and formulated as a constrained optimization problem. We will also theoretically analyze these practical OOD settings in our feature work.

% \section*{Acknowledgments}
\textbf{Acknowledgments.} 
This work is supported by National Key R\&D Program of China under Grant 2020AAA0105701, National Natural Science Foundation of China (NSFC) under Grants 61872327, Major Special Science and Technology Project of Anhui, National Natural Science Foundation of China (62033012) and Ant Group through Ant Research Intern Program.


\begin{thebibliography}{10}

\bibitem{edwardsArchDrivenRobotics}
G.~Edwards, J.~Garcia, H.~Tajalli, D.~Popescu, N.~Medvidovic, G.~Sukhatme, and
  B.~Petrus, ``Architecture-driven self-adaptation and self-management in
  robotics systems,'' in {\em 2009 ICSE Workshop on Software Engineering for
  Adaptive and Self-Managing Systems}, pp.~142--151, 2009.

\bibitem{camara_software_2020}
J.~Cámara, B.~Schmerl, and D.~Garlan, ``Software architecture and task plan
  co-adaptation for mobile service robots,'' in {\em Proceedings of the
  {IEEE}/{ACM} 15th {International} {Symposium} on {Software} {Engineering} for
  {Adaptive} and {Self}-{Managing} {Systems}}, (Seoul Republic of Korea),
  pp.~125--136, ACM, June 2020.

\bibitem{park_task-based_2012}
Y.-S. Park, H.-M. Koo, and I.-Y. Ko, ``A task-based and resource-aware approach
  to dynamically generate optimal software architecture for intelligent service
  robots,'' {\em Software: Practice and Experience}, vol.~42, no.~5,
  pp.~519--541, 2012.
\newblock \_eprint: https://onlinelibrary.wiley.com/doi/pdf/10.1002/spe.1074.

\bibitem{shin_platooning_2021}
Y.-J. Shin, L.~Liu, S.~Hyun, and D.-H. Bae, ``Platooning {LEGOs}: {An} {Open}
  {Physical} {Exemplar} for {Engineering} {Self}-{Adaptive} {Cyber}-{Physical}
  {Systems}-of-{Systems},'' in {\em 2021 {International} {Symposium} on
  {Software} {Engineering} for {Adaptive} and {Self}-{Managing} {Systems}
  ({SEAMS})}, pp.~231--237, May 2021.
\newblock ISSN: 2157-2321.

\bibitem{askarpour_robomax_2021}
M.~Askarpour, C.~Tsigkanos, C.~Menghi, R.~Calinescu, P.~Pelliccione,
  S.~García, R.~Caldas, T.~J. von Oertzen, M.~Wimmer, L.~Berardinelli,
  M.~Rossi, M.~M. Bersani, and G.~S. Rodrigues, ``{RoboMAX}: {Robotic}
  {Mission} {Adaptation} {eXemplars},'' in {\em 2021 {International}
  {Symposium} on {Software} {Engineering} for {Adaptive} and {Self}-{Managing}
  {Systems} ({SEAMS})}, pp.~245--251, May 2021.
\newblock ISSN: 2157-2321.

\bibitem{ACROS}
B.~H.~C. Cheng, R.~J. Clark, J.~E. Fleck, M.~A. Langford, and P.~K. McKinley,
  ``Ac-ros: Assurance case driven adaptation for the robot operating system,''
  in {\em Proceedings of the 23rd ACM/IEEE International Conference on Model
  Driven Engineering Languages and Systems}, MODELS '20, (New York, NY, USA),
  p.~102–113, Association for Computing Machinery, 2020.

\bibitem{undersea}
S.~Gerasimou, R.~Calinescu, S.~Shevtsov, and D.~Weyns, ``{UNDERSEA}: An
  exemplar for engineering self-adaptive unmanned underwater vehicles,'' in
  {\em 2017 IEEE/ACM 12th International Symposium on Software Engineering for
  Adaptive and Self-Managing Systems (SEAMS)}, pp.~83--89, 2017.

\bibitem{Ludvigsen-2021}
M.~Ludvigsen, ``Collaborating robots sample the primary production in the
  ocean,'' {\em Science Robotics}, vol.~6, no.~50, p.~eabf4317, 2021.

\bibitem{Huang-2004}
H.~Huang, E.~Messina, R.~Wade, R.~English, B.~Novak, and J.~Albus, ``{Autonomy
  measures for robots},'' in {\em Proceedings of the 2004 ASME International
  Mechanical Engineering Congress \& Exposition, Anaheim, California},
  pp.~1--7, 2004.

\bibitem{Wynn-2014}
R.~B. Wynn, V.~A. Huvenne, T.~P. {Le Bas}, B.~J. Murton, D.~P. Connelly, B.~J.
  Bett, H.~A. Ruhl, K.~J. Morris, J.~Peakall, D.~R. Parsons, E.~J. Sumner,
  S.~E. Darby, R.~M. Dorrell, and J.~E. Hunt, ``Autonomous underwater vehicles
  (auvs): Their past, present and future contributions to the advancement of
  marine geoscience,'' {\em Marine Geology}, vol.~352, pp.~451--468, 2014.
\newblock 50th Anniversary Special Issue.

\vfill
\pagebreak

\bibitem{weyns2020introduction}
D.~Weyns, {\em An Introduction to Self-adaptive Systems: A Contemporary
  Software Engineering Perspective}.
\newblock John Wiley \& Sons, 2020.

\bibitem{doi:10.1126/scirobotics.abm6074}
S.~Macenski, T.~Foote, B.~Gerkey, C.~Lalancette, and W.~Woodall, ``Robot
  operating system 2: Design, architecture, and uses in the wild,'' {\em
  Science Robotics}, vol.~7, no.~66, p.~eabm6074, 2022.

\bibitem{corbato2013model}
C.~H. Corbato, {\em Model-based self-awareness patterns for autonomy}.
\newblock PhD thesis, Universidad Polit{\'e}cnica de Madrid, 2013.

\bibitem{bozhinoski22advrob}
D.~Bozhinoski, M.~G. Oviedo, N.~H. Garcia, H.~Deshpande, G.~van~der Hoorn,
  J.~Tjerngren, A.~Wasowski, and C.~H. Corbato, ``{MROS}: runtime adaptation
  for robot control architectures,'' {\em Advanced Robotics}, vol.~36, no.~11,
  pp.~502--518, 2022.



\bibitem{bsnexemplarROS}
E.~B. Gil, R.~Caldas, A.~Rodrigues, G.~L.~G. da~Silva, G.~N. Rodrigues, and
  P.~Pelliccione, ``Body sensor network: A self-adaptive system exemplar in the
  healthcare domain,'' in {\em 2021 International Symposium on Software
  Engineering for Adaptive and Self-Managing Systems (SEAMS)}, pp.~224--230,
  2021.

\bibitem{nordmann2021system}
A.~Nordmann, R.~Lange, and F.~M. Rico, ``System modes-digestible system (re-)
  configuration for robotics,'' in {\em 2021 IEEE/ACM 3rd International
  Workshop on Robotics Software Engineering (RoSE)}, pp.~19--24, IEEE, 2021.

\bibitem{brun09}
Y.~Brun, G.~D.~M. Serugendo, C.~Gacek, H.~Giese, H.~M. Kienle, M.~Litoiu, H.~A.
  M{\"{u}}ller, M.~Pezz{\`{e}}, and M.~Shaw, ``Engineering self-adaptive
  systems through feedback loops,'' in {\em Software Engineering for
  Self-Adaptive Systems}, vol.~5525 of {\em Lecture Notes in Computer Science},
  pp.~48--70, Springer, 2009.

\bibitem{kephart03computer}
J.~O. Kephart and D.~M. Chess, ``The vision of autonomic computing,'' {\em
  Computer}, vol.~36, no.~1, pp.~41--50, 2003.

\bibitem{corbatoMROSRuntimeAdaptation2020}
D.~Bozhinoski, M.~G. Oviedo, N.~H. Garcia, H.~Deshpande, G.~van~der Hoorn,
  J.~Tjerngren, A.~Wasowski, and C.~H. Corbato, ``{MROS:} runtime adaptation
  for robot control architectures,'' {\em Adv. Robotics}, vol.~36, no.~11,
  pp.~502--518, 2022.

\bibitem{Antoniou2004}
G.~Antoniou and F.~van Harmelen, ``Web ontology language: {OWL},'' in {\em
  Handbook on Ontologies} (S.~Staab and R.~Studer, eds.), International
  Handbooks on Information Systems, pp.~67--92, Springer, 2004.

\bibitem{horrocks2004swrl}
I.~Horrocks, P.~F. Patel-Schneider, H.~Boley, S.~Tabet, B.~Grosof, M.~Dean,
  {\em et~al.}, ``{SWRL}: {A semantic web rule language combining OWL and
  RuleML},'' {\em {W3C} Member submission}, vol.~21, no.~79, pp.~1--31, 2004.

\end{thebibliography}

\end{document}
