%% Beginning of file 'sample631.tex'
%%
%% Modified 2022 May  
%%
%% This is a sample manuscript marked up using the
%% AASTeX v6.31 LaTeX 2e macros.
%%
%% AASTeX is now based on Alexey Vikhlinin's emulateapj.cls 
%% (Copyright 2000-2015).  See the classfile for details.

%% AASTeX requires revtex4-1.cls and other external packages such as
%% latexsym, graphicx, amssymb, longtable, and epsf.  Note that as of 
%% Oct 2020, APS now uses revtex4.2e for its journals but remember that 
%% AASTeX v6+ still uses v4.1. All of these external packages should 
%% already be present in the modern TeX distributions but not always.
%% For example, revtex4.1 seems to be missing in the linux version of
%% TexLive 2020. One should be able to get all packages from www.ctan.org.
%% In particular, revtex v4.1 can be found at 
%% https://www.ctan.org/pkg/revtex4-1.

%% The first piece of markup in an AASTeX v6.x document is the \documentclass
%% command. LaTeX will ignore any data that comes before this command. The 
%% documentclass can take an optional argument to modify the output style.
%% The command below calls the preprint style which will produce a tightly 
%% typeset, one-column, single-spaced document.  It is the default and thus
%% does not need to be explicitly stated.
%%
%% using aastex version 6.3
\documentclass[twocolumn,twocolappendix]{aastex631}

%% The default is a single spaced, 10 point font, single spaced article.
%% There are 5 other style options available via an optional argument. They
%% can be invoked like this:
%%
%% \documentclass[arguments]{aastex631}
%% 
%% where the layout options are:
%%
%%  twocolumn   : two text columns, 10 point font, single spaced article.
%%                This is the most compact and represent the final published
%%                derived PDF copy of the accepted manuscript from the publisher
%%  manuscript  : one text column, 12 point font, double spaced article.
%%  preprint    : one text column, 12 point font, single spaced article.  
%%  preprint2   : two text columns, 12 point font, single spaced article.
%%  modern      : a stylish, single text column, 12 point font, article with
%% 		  wider left and right margins. This uses the Daniel
%% 		  Foreman-Mackey and David Hogg design.
%%  RNAAS       : Supresses an abstract. Originally for RNAAS manuscripts 
%%                but now that abstracts are required this is obsolete for
%%                AAS Journals. Authors might need it for other reasons. DO NOT
%%                use \begin{abstract} and \end{abstract} with this style.
%%
%% Note that you can submit to the AAS Journals in any of these 6 styles.
%%
%% There are other optional arguments one can invoke to allow other stylistic
%% actions. The available options are:
%%
%%   astrosymb    : Loads Astrosymb font and define \astrocommands. 
%%   tighten      : Makes baselineskip slightly smaller, only works with 
%%                  the twocolumn substyle.
%%   times        : uses times font instead of the default
%%   linenumbers  : turn on lineno package.
%%   trackchanges : required to see the revision mark up and print its output
%%   longauthor   : Do not use the more compressed footnote style (default) for 
%%                  the author/collaboration/affiliations. Instead print all
%%                  affiliation information after each name. Creates a much 
%%                  longer author list but may be desirable for short 
%%                  author papers.
%% twocolappendix : make 2 column appendix.
%%   anonymous    : Do not show the authors, affiliations and acknowledgments 
%%                  for dual anonymous review.
%%
%% these can be used in any combination, e.g.
%%
%% \documentclass[twocolumn,linenumbers,trackchanges]{aastex631}
%%
%% AASTeX v6.* now includes \hyperref support. While we have built in specific
%% defaults into the classfile you can manually override them with the
%% \hypersetup command. For example,
%%
%% \hypersetup{linkcolor=red,citecolor=green,filecolor=cyan,urlcolor=magenta}
%%
%% will change the color of the internal links to red, the links to the
%% bibliography to green, the file links to cyan, and the external links to
%% magenta. Additional information on \hyperref options can be found here:
%% https://www.tug.org/applications/hyperref/manual.html#x1-40003
%%
%% Note that in v6.3 "bookmarks" has been changed to "true" in hyperref
%% to improve the accessibility of the compiled pdf file.
%%
%% If you want to create your own macros, you can do so
%% using \newcommand. Your macros should appear before
%% the \begin{document} command.
%%

\usepackage{amssymb,amsmath,amsfonts}

\newcommand{\vdag}{(v)^\dagger}
\newcommand\aastex{AAS\TeX}
\newcommand\latex{La\TeX}

\newcommand{\rcd}{r_{\rm cd}} 
\newcommand{\rid}{r_{\rm id}}
\newcommand{\rt}{r_{\rm t}}
\newcommand{\rvir}{r_{\rm vir}}
\newcommand{\rd}{r_{\rm d}}

\newcommand{\jx}[1]{\textcolor{red}{#1}}
\newcommand{\mf}[1]{\textcolor{cyan}{#1}}
%% Reintroduced the \received and \accepted commands from AASTeX v5.2
%\received{March 1, 2021}
%\revised{April 1, 2021}
%\accepted{\today}

%% Command to document which AAS Journal the manuscript was submitted to.
%% Adds "Submitted to " the argument.
%\submitjournal{PSJ}

%% For manuscript that include authors in collaborations, AASTeX v6.31
%% builds on the \collaboration command to allow greater freedom to 
%% keep the traditional author+affiliation information but only show
%% subsets. The \collaboration command now must appear AFTER the group
%% of authors in the collaboration and it takes TWO arguments. The last
%% is still the collaboration identifier. The text given in this
%% argument is what will be shown in the manuscript. The first argument
%% is the number of author above the \collaboration command to show with
%% the collaboration text. If there are authors that are not part of any
%% collaboration the \nocollaboration command is used. This command takes
%% one argument which is also the number of authors above to show. A
%% dashed line is shown to indicate no collaboration. This example manuscript
%% shows how these commands work to display specific set of authors 
%% on the front page.
%%
%% For manuscript without any need to use \collaboration the 
%% \AuthorCollaborationLimit command from v6.2 can still be used to 
%% show a subset of authors.
%
%\AuthorCollaborationLimit=2
%
%% will only show Schwarz & Muench on the front page of the manuscript
%% (assuming the \collaboration and \nocollaboration commands are
%% commented out).
%%
%% Note that all of the author will be shown in the published article.
%% This feature is meant to be used prior to acceptance to make the
%% front end of a long author article more manageable. Please do not use
%% this functionality for manuscripts with less than 20 authors. Conversely,
%% please do use this when the number of authors exceeds 40.
%%
%% Use \allauthors at the manuscript end to show the full author list.
%% This command should only be used with \AuthorCollaborationLimit is used.

%% The following command can be used to set the latex table counters.  It
%% is needed in this document because it uses a mix of latex tabular and
%% AASTeX deluxetables.  In general it should not be needed.
%\setcounter{table}{1}

%%%%%%%%%%%%%%%%%%%%%%%%%%%%%%%%%%%%%%%%%%%%%%%%%%%%%%%%%%%%%%%%%%%%%%%%%%%%%%%%
%%
%% The following section outlines numerous optional output that
%% can be displayed in the front matter or as running meta-data.
%%
%% If you wish, you may supply running head information, although
%% this information may be modified by the editorial offices.
%\shorttitle{AASTeX v6.3.1 Sample article}
%\shortauthors{Schwarz et al.}
%%
%% You can add a light gray and diagonal water-mark to the first page 
%% with this command:
%% \watermark{text}
%% where "text", e.g. DRAFT, is the text to appear.  If the text is 
%% long you can control the water-mark size with:
%% \setwatermarkfontsize{dimension}
%% where dimension is any recognized LaTeX dimension, e.g. pt, in, etc.
%%
%%%%%%%%%%%%%%%%%%%%%%%%%%%%%%%%%%%%%%%%%%%%%%%%%%%%%%%%%%%%%%%%%%%%%%%%%%%%%%%%
%\graphicspath{{./}{figures/}}
%% This is the end of the preamble.  Indicate the beginning of the
%% manuscript itself with \begin{document}.

\begin{document}
%\received{XXX}
%\revised{YYY}
%\accepted{ZZZ}

%\submitjournal{ApJ}

\shorttitle{Depletion boundary evolution}
\shortauthors{Gao et al.}


\title{Physical evolution of dark matter halo around the depletion boundary}

\author{Hongyu Gao}
\affil{Department of Astronomy, School of Physics and Astronomy, Shanghai Jiao Tong University, Shanghai, 200240, People’s Republic of China}
\affil{Key Laboratory for Particle Astrophysics and Cosmology (MOE), Shanghai 200240, China}
\affil{Shanghai Key Laboratory for Particle Physics and Cosmology, Shanghai Jiao Tong University, Shanghai, 200240, People’s Republic of China}

\author[0000-0002-8010-6715]{Jiaxin Han}
\affil{Department of Astronomy, School of Physics and Astronomy, Shanghai Jiao Tong University, Shanghai, 200240, People’s Republic of China}
\affil{Key Laboratory for Particle Astrophysics and Cosmology (MOE), Shanghai 200240, China}
\affil{Shanghai Key Laboratory for Particle Physics and Cosmology, Shanghai Jiao Tong University, Shanghai, 200240, People’s Republic of China}

\author{Matthew Fong}
\affil{Department of Astronomy, School of Physics and Astronomy, Shanghai Jiao Tong University, Shanghai, 200240, People’s Republic of China}
\affil{Key Laboratory for Particle Astrophysics and Cosmology (MOE), Shanghai 200240, China}
\affil{Shanghai Key Laboratory for Particle Physics and Cosmology, Shanghai Jiao Tong University, Shanghai, 200240, People’s Republic of China}

\author[0000-0002-4534-3125]{Y.P. Jing}
\affil{Department of Astronomy, School of Physics and Astronomy, Shanghai Jiao Tong University, Shanghai, 200240, People’s Republic of China}
\affil{Key Laboratory for Particle Astrophysics and Cosmology (MOE), Shanghai 200240, China}
\affil{Shanghai Key Laboratory for Particle Physics and Cosmology, Shanghai Jiao Tong University, Shanghai, 200240, People’s Republic of China}
\affil{Tsung-Dao Lee Institute, Shanghai Jiao Tong University, Shanghai, 200240, People’s Republic of China}

\author[0000-0001-7890-4964]{Zhaozhou Li}
\affil{Centre for Astrophysics and Planetary Science, Racah Institute of Physics, The Hebrew University, Jerusalem, 91904, Israel}

\correspondingauthor{Jiaxin Han}
\email{jiaxin.han@sjtu.edu.cn}


\begin{abstract}
    We investigate the build-up of the halo profile out to large scale in a cosmological simulation, focusing on the roles played by the recently proposed depletion radii. %By tracing the evolution of various density and kinematic profiles for halo samples with different masses in a cosmological simulation from $z=0$ to $z=5$, 
    We explicitly show that halo growth is accompanied by the depletion of the environment, with the inner depletion radius demarcating the two. This evolution process is also observed via the formation of a trough in the bias profile, with the two depletion radii identifying key scales in the evolution. %The scaled mass profile between the virial radius and the inner depletion radius barely depends on halo mass and evolves only weakly with redshift.%, with residual evolutions within $\sim 10\%$. 
    The ratio between the inner depletion radius and the virial radius is approximately a constant factor of 2 across redshifts and halo masses. The ratio between their enclosed densities is also close to a constant of 0.18. These simple scaling relations reflect the largely universal scaled mass profile on these scales, which only evolves weakly with redshift.
 %We also show that the boundary features provide new physical and quantitative diagnostics to the halo evolution phases.%  
 The overall picture of the boundary evolution can be broadly divided into three stages according to the maturity of the depletion process, with cluster halos lagging behind low mass ones in the evolution. We also show that the traditional slow and fast accretion dichotomy of halo growth can be identified as accelerated and decelerated depletion phases respectively. 
    %These results open a new window for studying the extent and evolution of dark matter halos.
    %\jx{there are some compilation errors to be fixed.}
\end{abstract}


%\keywords{Galaxy dark matter halos (1880)}


\section{Introduction} \label{sec:intro}
With the developments of high-resolution numerical simulations, the dark matter distribution within halos have been extensively investigated. Both the Navarro-Frenk-White (NFW) profile \citep{1996ApJ...462..563N, 1997ApJ...490..493N}, whose slope changes from $-1$ to $-3$ from inside to outside, and the Einasto profile \citep[e.g.,][]{1965TrAlm...5...87E,2004MNRAS.349.1039N,2005ApJ...624L..85M,2006AJ....132.2685M,2008MNRAS.387..536G,2011MNRAS.415.3895L}, with a flatter inner slope, can reasonably describe the halo density profile at small or median scales. On the other hand, as an extended structure embedded in a diffuse large-scale environment, a concise definition of the boundary of a halo is more challenging, and has attracted more attention over recent years in both theories~\citep[e.g.,][]{2005ApJ...631...41T,2008MNRAS.388....2H,2008MNRAS.389..385C,2013MNRAS.430..725V,2014ApJ...789....1D,Zemp14,2021MNRAS.505.1195G} and observations~\citep[e.g.,][]{More2016,Baxter2017,2017ApJ...836..231U,2019MNRAS.485..408C,Deason2020, 2021ApJ...915L..18L,Fong22}.
	
%A precise definition of the halo boundary is demanded both theoretically and observationally. For instance, the outskirts of galaxy clusters, where the equilibrium state is no longer viable, contains adequate hydrodynamic information of the intracluster medium (ICM) and thus is a natural laboratory for understanding the formation of clusters \citep[see][for review]{2013SSRv..177..195R,2019SSRv..215....7W}. Motivated by this reason, ever more observational studies based on weak gravitational lensing measurements \citep[e.g.,][]{2017ApJ...836..231U,2019MNRAS.485..408C,Fong22} and the projected number density of galaxies detected by Sunyaev-Zeldovich (SZ) effect \citep[e.g.,][]{2019ApJ...874..184Z,2020arXiv200811663A} draw their attention from the centre to the outer region of clusters. In addition to the cluster-size halos, the outskirt of our Milky Way is also worth a detailed analysis. Recently, \cite{2021ApJ...915L..18L} explored the outermost boundaries of the Milky Way for the first time by analysing the dynamics of satellites. On the other hand, when modelling the halo two-point correlation function or power spectrum in the halo model framework, it is necessary to consider the halo exclusion effect \citep[e.g.,][]{2005ApJ...631...41T,2008MNRAS.388....2H,2013MNRAS.430..725V}, caused by the overlap of two halos at a given distance, to reduce the redundant contribution of the two-halo term on small-scale clustering. Accordingly, a clear definition of the halo boundary is the key factor to distinguish the contributions from the one-halo term and two-halo term in their transition region \citep[e.g.,][]{2021MNRAS.505.1195G}.
	
%	The definition of the halo boundary or size is constantly being refined and optimised. 
Classical definitions of the halo boundary are established from the spherical collapse model~\citep{1972ApJ...176....1G}. In this model, a uniform spherical region in the Universe first expands till a maximum radius called the turnaround radius, after which it collapses over a freefall time and virializes within the so-called virial radius. The density of the virialized structure is predicted to be a fixed value times that of the background universe~\citep{1996MNRAS.282..263E,1998ApJ...495...80B}. Considering that the spherical collapse model is no longer valid when shell-crossing occurs, \cite{1984ApJ...281....1F} and \cite{1985ApJS...58...39B} derived the self-similar solution by scaling the trajectories of different shells with the characteristic time and length. The density profile predicted by the self-similarity solution shows a clear power-law form, with obvious caustic features that reflect the accumulation of material at the apocentres of different orbits. Since the infalling dark matter particles also have angular momentum and the halo shape deviates significantly from the ideal spherical shape \citep{2002ApJ...574..538J}, the caustic features on the density profile measured in the simulation are almost entirely smoothed out, except for the outermost one that is composed by the latest accretion material. The apogee of this outermost orbit is defined as the well-known splashback radius \citep{2014ApJ...789....1D,2014JCAP...11..019A,2015ApJ...810...36M,2016MNRAS.459.3711S,2017ApJ...841...34M,2017ApJS..231....5D,2017ApJ...843..140D}, which in practice corresponds to the steepest slope of the density profile. This feature can become more visible in the anisotropic density profile~\citep{Wang22}. Within the splashback picture, recent attempts have also been made to define halo boundaries based on the decomposition of material in phasespace into infalling and orbiting components~\citep[][]{Aung21, Diemer22a,Diemer22b,Garcia22}.  Besides, the turnaround radius \citep{Pavlidou14, Tanoglidis15, Korkidis20} is also a meaningful characteristic radius in the spherical collapse model, as it marks the starting point of the infall zone around a halo, namely, where the radial velocity of matter is zero.
	
	Recently, \cite{FH21} (hereafter FH21) proposed the concept of depletion radii to characterise the halo boundary from a new perspective. As a halo consumes material from its surroundings to grow inside, the separation between the growing region and the depleting environment provides a natural boundary definition, which is named the inner depletion radius, $\rid$. FH21 argues that the depletion of the environment outside $\rid$ can be observed as a trough in the bias profile, which is a rescaled version of the halo density profile. The location of the minimum bias is thus defined as the characteristic depletion radius, $\rcd$, reflecting the consequence of depletion. Based on these two interpretations, \citet{Fong22} and \citet{2021ApJ...915L..18L} have measured the characteristic and inner depletion radii in observations respectively, using weak lensing and satellite kinematics. %Consequently, although $\rcd$ and $\rid$ are obtained in different ways, they are highly self-consistent, and their combination can present a meaningful definition of halo boundary, which can clearly describe the connection between the halo growth process and the matter distribution in the outskirts of halos. 
	
	In contrast to the conventional virial radii that suffer from pseudo evolution~\citep{diemer2013pseudo,Zemp14}, the depletion radii characterise the physical evolution of halos on their outskirts by construction. 	Moreover, it is worth mentioning that the $\rid$ is an excellent match to the optimal halo exclusion radius proposed by \cite{2021MNRAS.505.1195G} in a halo model of the correlation function. This further supports that the depletion radius can be used as a physical halo boundary that self-consistently decomposes the cosmic density field into different halos. In contrast to the traditional steepest density slope location definition of the splashback radius, which typically corresponds to a 75\% containment radius of splashback orbits, this inner depletion radius can be interpreted as the radius enclosing a highly complete population of splashback orbits.
	
	The dynamical interpretation of the depletion radii depends crucially on the evolutions of density and bias around a halo. Using the velocity profile at $z=0$, FH21 was able to deduce the evolution trend of the density profile qualitatively. In this work, we extend the study of FH21 by directly examining the evolution of the density, bias, and mass flow profiles in the simulation over a wide range of redshift ($z=0-5$). As we will show, the two radii stand out clearly in the evolution of not only the density profile, but also the bias profile, highly consistent with their depletion interpretation. We will also explore their roles in halo growth as well as their own evolution in detail.

	%This is a critical epoch as the Universe transitions from matter-dominated to dark energy-dominated and has therefore become the main focus of the current galaxy redshift surveys, such as the Dark Energy Spectroscopic Instrument \citep[DESI,][]{2016arXiv161100036D} and the Subaru Prime Focus Spectrograph \citep[PFS,][]{2014PASJ...66R...1T}. 
	
	%We arrange this paper as follows. In Section \ref{data}, we introduce our simulation data, halo sample construction and the method adopted to measure the depletion radius. In Section \ref{evolution_halo}, We present a comprehensive picture of the redshift evolution of halo density and velocity profiles. We further investigate the evolution of depletion radius and its enclosed density in Section \ref{evolution_rd}. Finally, we conclude in Section \ref{conclusion}. Throughout this work, the cosmological parameters at $z=0$ are: $\Omega_{\rm{m},0} = 0.268$, $\Omega_{\Lambda,0} = 0.732$ and $H_0 = 100h \,\rm{km\,s^{-1}\,Mpc^{-1}}$ with $h=0.71$. The halo profiles are all calculated in the physical coordinate system. For convenience, we will refer to the region within the virial radius as the inner halo, and the region outside it up to the depletion scale as the outer halo. The region outside the depletion radius is referred to as the environment of the halo.

	\section{Data}\label{data}
	\subsection{Simulation and the depletion catalog}\label{data_simulation}
	
	We use the same simulation as used in FH21, which is an $N$-body simulation from the CosmicGrowth \citep{2019SCPMA..6219511J} simulation suite run with a $\rm{P^3M}$ code~\citep{2002ApJ...574..538J}, adopting a $\Lambda$CDM cosmology with parameters $\Omega_{\rm{m}} = 0.268$ and $\Omega_{\Lambda} = 0.732$. A total of $3072^3$ dark matter particles are resolved with a box size of $600\,\mathrm{Mpc}\,h^{-1}$ per side, corresponding to a particle mass of $m_{\rm{p}} = 5.54 \times 10^8 M_{\odot}\,h^{-1}$. Halos are identified by the friends-of-friends algorithm (FoF) \citep{1985ApJ...292..371D} with a standard linking parameter $b=0.2$. The subhalos and their merger histories are identified by the \textsc{hbt+} code \citep{2012MNRAS.427.2437H,2018MNRAS.474..604H}\footnote{\url{https://github.com/Kambrian/HBTplus}}.
	
	The FoF halo algorithm with a linking-length of $b=0.2$ is optimized for dissecting halos according to the virial radius. 
	Because the depletion radius is typically a factor of $\sim 2.5$ times the virial radius~\citep{FH21}, our FoF catalog contains halos that overlap in their depletion boundary, distorting the profiles around them on the depletion scale (see Appendix~\ref{app:overlap} for a case study). To avoid such complications, we remove any halo whose distance to a more massive neighbour is smaller than the sum of their estimated $\rcd$'s (i.e., $2.5 r_{\rm vir}$), to produce a depletion-radius based halo catalog. This \emph{depletion} catalog will be used for the majority of this work. To understand the influence of the depletion selection, however, we also show some results for the original FoF catalog in Appendix~\ref{sec:nonclean}.
	
	In addition to the above cleaning, we further limit our analysis to halos with more than 500 bound particles at $z=0$ to ensure sufficient resolution. For each of these $z=0$ halos, we track their evolution from $z=0$ back to $z\simeq 5$ along its main branch as resolved by \textsc{HBT+}, and extract their profiles out to $\sim 10{\mathrm{Mpc}}\,h^{-1}$ in physical radius. Our final sample contains only those halo whose main branches can be identified from $z=0$ to $z\simeq 5$. This tractability requirement removes $\sim 7\%$ and $0.04\%$ of haloes from the galactic and cluster sized halo samples which we study below, and has negligible influences on the profile evolutions. The most bound position of the central subhalo is chosen as the centre of a halo when measuring its profiles. 
	
	
	\subsection{Identifying the depletion radii}\label{data_rd}                
	The depletion radii, along with the classical turnaround radius, can be identified in the bias and mass flow rate (MFR) profiles around halos. In principle, this can be done on a halo-by-halo basis. %However, the profiles around a single halo can be noisy due to the presence of subhalos and neighbouring structures (see Fig.~\ref{fig:merger} for an example). 
	To suppress noises, however, we choose to work with stacked profiles around a sample of halos of similar sizes, and further use second-order local polynomial interpolation to locate the extrema of the profiles.     
	
		\subsubsection{The inner depletion radius and depletion rate}
	The mass flow rate (MFR), $I_{\rm m}$, is defined as
	\begin{equation}
	    I_{\rm m}(r)\equiv 4\pi r^2 \rho(r) v_{\rm r}(r),\label{eq:MFR}
	\end{equation} 
	where $\rho(r)$ is the density profile and $v_{\rm{r}}(r)$ is the radial velocity profile including the Hubble flow. A negative radial velocity corresponds to infalling motion. For convenience, we will also call the negative value of MFR as the Mass Inflow Rate (MIR), and the maximum MIR as the \textit{depletion rate} of a halo.

According to the continuity equation,
% 	\begin{equation}
% 	    \frac{\partial M(r)}{\partial t}+ I_{\rm m}(r,t)=0,
% 	\end{equation} this MFR determines the growth rate of the enclosed mass at $r$.
	\begin{equation}
 	    \frac{\partial \rho(r)}{\partial t}+ \frac{1}{4\pi r^2}\frac{\partial I_{\rm m}(r,t)}{\partial r}=0,
 	\end{equation} 
 	the slope of MFR determines the growth rate of the local density. In regions with a positive slope, $\partial I_{\rm m}/\partial r>0$, matter falls with an increasing rate along the path, causing a net drop in the local density, i.e., $\partial \rho/\partial t<0$. On the other hand, in regions with a negative slope, the infall motion slows towards the center, causing matter to pile up. The minimum of $I_{\rm m}$ (or maximum of MIR) marks the exact transition between the two regions, and is defined as the inner depletion radius, $\rid$. %This evolution pattern was first pointed out in FH21 theoretically, and forms the physical basis for defining the depletion radii. 
	%For a sample of halos, the average MFR is simply computed as the average over the MFR of each halo within a radial bin.
	%To identify the inner depletion radius $\rid$ from the dynamical feature of halos, we first measure the mass flow rate (MFR) profile for each halo and take their average by
% 	\begin{equation}
% 	\mathrm{MFR} (r,z)=4\pi r^2 \times \left \langle\rho(r,z) \times  v_{\rm{r}}(r,z)\right \rangle , \label{mfr}
% 	\end{equation}
% 	where the  %For each halo in our sample, we subtract the average velocity of this halo from the physical velocities of particles to obtain their peculiar velocities, which are then projected to the radial direction. 
	%By averaging the particle radial velocities in each radius bin, we can get its radial peculiar profile $v_{\rm{pec}}\left(r\right)$. 
	%The $\rid$ for a sample is defined at the minimum of the average MFR profile (or the maximum of the MIR). 
% We note that when halos are binned by mass, this location mostly correspond to the minima of the radial velocity profiles (FH21). 
	%Similar to the identification of $\rcd$, we use a second-order polynomial to interpolate the MFR profile around its minimum, and identify the inner depletion radius $\rid$ from the interpolated profile.
	
	The turnaround radius, $r_{\rm{t}}$, can also be identified in the MFR profile at its up-crossing point through $I=0$.%, by interpolating the profile locally and solving for its zero-crossing point, where $v_{\rm r}=0$ and has a positive slope.
	\subsubsection{The characteristic depletion radius}
	
	The bias profile around an individual halo is defined as~\citep{Han19}
	\begin{equation}
	b(r) = \frac{\delta_{\rm{hm}}(r)}{\xi_{\rm{mm}}(r)}, \label{equ:bias_func}
	\end{equation}
	where $\xi_{\rm{mm}}(r)$ denotes the non-linear matter-matter correlation function, and $\delta_{\rm hm}(r)\equiv \rho(r)/\bar{\rho}-1$ is the overdensity profile around the halo.
	
	When averaged over a sample of halos, we recover the commonly used equation of the population bias
	\begin{equation}
	    \langle b(r) \rangle = \frac{\xi_{\rm{hm}}(r)}{\xi_{\rm{mm}}(r)},
	\end{equation} 
	where $\xi_{\rm hm}(r)=\langle \delta_{\rm hm}(r) \rangle$ is the halo-matter correlation function.
	
	%We randomly select $10^5$ particles and average the overdensity profiles of these random particles to estimate the $\xi_{\rm{mm}}(r,z)$ via
% 	\begin{equation}
% 	\xi_{\rm{mm}}(r,z) = \left \langle\rho_{\rm{mm}}(r,z)\right\rangle/\rho_{\rm{m}}(z)-1,
% 	\end{equation}
%     where the mean density of universe $\rho_{\rm{m}}(z)$ at redshift $z$ is computed as
% 	\begin{equation}
% 	\rho_{\rm{m}}(z) = \Omega_{\rm{m}}(z)\times \rho_{\rm{c}}(z)  
% 	=  \frac{\Omega_{\rm{m,0}}(1+z)^33H^2_0 }{8\pi G}. 
% 	\end{equation}
% 	Similarly, given a halo sample with a specific property, the $\xi_{\rm{hm}}(r,z)$ can be calculated by average the
% 	\begin{equation}
% 	\xi_{\rm{hm}}(r,z) = \left \langle\rho_{\rm{hm}}(r,z)\right\rangle/\bar{\rho}_{\rm{m}}(z)-1,
% 	\end{equation}
% 	where $\rho_{\rm{hm}}(r,z)$ is the density profile of each halo in this property bin.
	
	The characteristic depletion radius, $\rcd$, is defined where the bias profile reaches its minimum on the intermediate scale. To precisely identify $\rcd$ in the bias profile, we perform a second-order local polynomial interpolation using the data point with minimum bias as well as its two adjacent points. Then $\rcd$ is determined as the position of the minimum value of this polynomial. 
		

	
	\begin{figure*}
	% To include a figure from a file named example.*
	% Allowable file formats are eps or ps if compiling using latex
	% or pdf, png, jpg if compiling using pdflatex
	\centering
	\includegraphics[scale=0.45]{evo_prof_tracez0_clean_ScaledByz0.pdf}
	\caption{Evolution of density, bias, MFR and radial velocity profiles stacked in galactic-size halo bin (left column) and cluster-size halo bin (right column) are displayed in the four rows from top to bottom. The title of the top panel shows the number of the complete halo sample as well as the range of halo mass at $z=0$. Solid lines with different colors correspond to different redshifts, and the median halo mass at each redshift is also presented. The MFR profiles of the galactic-size halo bin are enlarged to show more clearly. The $\rid$, $\rcd$, and $\rt$ are denoted as different markers. The MFR and velocity profiles are all scaled by the virial quantities at $z=0$.
	%\jx{TODO: clarify the axis label} 
	%\mf{This might end up being too cluttered, but I am wondering if you can try this: I think it would be interesting for the curious reader to have all of the radii shown. However, if the specific radii is not defined in that panel, then make the marker transparent. For example, the bias panel would have $\rid$ and $\rt$ as transparent, the density would have all radii transparent, etc. I suggest making the transparent markers very transparent and barely visible so even though it is cluttered, it is easy to distinguish. not necessary: Also I feel that $\rt$ and $\rid$ markers are a little similar so it is a bit confusing to the eye.}
	}
	\label{fig:1}
    \end{figure*}


 
	\section{Evolution of halo profiles in lights of the depletion radii}\label{evolution_halo}
	
	%The structure and environment of a halo evolve as the halo grows. Such evolutions are reflected in the density, bias, velocity and MFR profiles. In contrast to the conventional virial radii that suffers from pseudo evolution, the depletion radii have been defined to capture these physical evolutions. As we show below, the inner and characteristic depletion radii provide intuitive and natural descriptions of how halos evolve on their outskirts, and help to reveal their build-up process.
	
	We select two samples of halos according to their virial mass at $z=0$, including a galatic-size halo sample with $M_{\rm vir}=10^{12.05}-10^{12.60} M_{\odot} \, h^{-1}$ and a cluster-size sample with $M_{\rm vir}=10^{13.70}-10^{14.25} M_{\odot} \, h^{-1}$. These halos are traced over time back to $z\simeq 5$ according to the \textsc{hbt+} merger tree. Figure~\ref{fig:1} shows the stacked profiles of the density, bias, MFR and radial velocity at a sequence of snapshots, along with various halo radii. %And the two columns in Figure \ref{fig:evo_bias_rho_vel_MFR_M_phy_exclude_extrapolate} correspond the two halo mass bins whose number are labelled in the titles of each column. As described in Section \ref{data_rd}, we adopt the Equation \ref{equ:bias_func} and polynomial to fit the bias and MFR profile respectively. Note that it is more difficult to obtain the characteristic depletion radius $\rcd$ from the bias profiles when the trough has not been well formed at the high redshifts, especially for the massive halos. Therefore, the $\rcd$ of the highest three (six) redshift bins for the galactic-size (cluster-size) halo samples are determined by extrapolating $\rcd - z$ relationship, as we will show in Section \ref{evolution_rd} and Figure \ref{fig:radii_and_delta_and_rho_evo_exclude_extrapolate}. By contrast, as shown the third row in Figure \ref{fig:evo_bias_rho_vel_MFR_M_phy_exclude_extrapolate}, the location of the minimum MFR profile, which is the dynamical definition of the inner depletion radius $\rid$, presents a clearer feature for the galactic-size halos at high redshift and the cluster-size halos in the entire redshift range. For the galactic-size halos at low redshift, the $\rid$ cannot be easily determined as their accretion has mostly stopped. 
	
	%\subsection{Illuminating the roles of the depletion radii}
	The velocity and MFR profiles in Fig.~\ref{fig:1} all show a universal pattern. With an increasing radius from the halo centres, the radial velocity profile starts from a flat inner profile around zero, describing an approximately virialized inner halo, followed by a trough of negative velocity describing the infall of matter, and finally an outflowing outer region dominated by the Hubble expansion. %The separation between the infalling and outflowing regions is the turnaround radius. 
	
 	
 	In the top panels we examine the density profile evolution in our simulation directly. Indeed the density evolves differently across $\rid$, with a growing inner profile and a decaying outer profile. The growth of the inner density also causes $\rid$ to grow over time. Given the finite time separation between the redshift bins, every two consecutive profiles cross each other in between their $\rid$'s.
 	
 	On the largest scale outside of $\rt$, the radial velocity is dominated by the Hubble flow, and the decrease in the density is primarily due to the expansion of the Universe. However, within the turnaround radius, the gravity of the halo becomes important, which can contribute significantly to the depletion of material outside $\rid$. This is best revealed in the evolution of the bias profile, where the average (over)density profile of the universe, $\xi_{\rm mm}$, has been scaled out. In this representation, the large scale matter distribution becomes flat, allowing us to focus on the contribution from the halo itself to the density profile. A clear bias trough can be seen in the bias profile outside $\rid$, which reflects the depletion of material due to the accretion of the halo. The characteristic depletion radius, $\rcd$, defined at the bias minimum, thus characterises the scale with the most prominent depletion signature.
 	
    The evolution of the bias profile clearly reveals the formation of the bias trough as a depletion process, in which the bias profile drops the most around $\rcd$. It is remarkable to see that each bias profile peels off from its progenitor right at $\rid$. This is highly consistent with the expectation that $\rid$ marks the inner edge of depletion, while $\rcd$ is located where the depletion is most significant.
 	
	%As shown in Figure \ref{fig:1}, the halo density profile within $\rid$ is growing with redshift, while the matter in the outskirts is being depleted, relative to the matter density of the Universe. The outer shoulder of the density profile becomes successively lower with time. With the shift of $\rid$, the originally flat slope of the outer profile becomes steeper, and eventually resembles that of the inner profile. 
	
	%The growth of halo is clearly reflected in the evolution of bias, MFR, and radial velocity profile. During the evolution of bias profile, halo depletes the matter around it and weakens the halo-matter to matter-matter correlations. After the material is significantly depleted, a trough at $\rcd$ in the bias profile eventually forms. 
	
    An interesting phenomenon arising from this evolution process is that the bias profile out to $\rid$ is equivalent to the evolution path of $(\rid, b(\rid))$ at least approximately. In other words, the bias profiles at different redshifts are approximately unified within $\rid$, especially for the galactic size halos. This may be used to model the density profile evolution from the final bias profile and the $\xi_{\rm{mm}}$ evolution. We leave such an exploration to a future work.%In section~\ref{sec:prof_recov} we test this explicitly.  \jx{TODO: check how well the density profile evo can be recovered from the bias*corrfunc(z)}
    
    %Moreover, the MFR presents a dynamical interpretation for the growth of halo. At $\rid$ where the MFR is maximum, the mass flow is constant with radius on instantaneous time scales. The amount of matter flowing into $\rid$ from the outskirts is equal to the amount of matter flowing in towards the halo from this region. Thus $\rid$ separates the growing halo from its depleting environment.
    
    As a halo grows, its depletion radii expand, and the corresponding troughs in the bias and MFR profiles move to larger scales. The evolution of the depletion rate, i.e., MIR at $\rid$, is not monotonic. It first increases and then decreases, peaking at $z\sim 2$ and $z\sim 1$ for the galactic and cluster size halos respectively. In particular, the mass accretion of galactic halos has almost completely stopped by $z=0$, with barely a trough in its MFR profile. 
    
  The shape of the radial velocity profile and its evolution are very similar to those of the MFR. Because the density profile is almost proportional to $r^{-2}$ on the $\rid$ scale, the minimum of the MFR is close to the minimum of $v_{\rm{r}}(r)$ according to Equation~\ref{eq:MFR}. Therefore, although the MFR is an intrinsic description of the growth process of halos, the velocity profile can also be used to estimate $\rid$. Especially in observations, the radial velocity profile is more practical for measurements of the inner depletion radius \citep{2021ApJ...915L..18L}.
    
    %In general, the overall picture of the halo evolution is highly consistent with the physical interpretation of the depletion radii in \cite{FH21}. 
    
	

%  \begin{figure*}
% 		\centering
% 	\includegraphics[scale=0.45]{evo_fit_accuracy_enclosed_mass_tracez0_clean.pdf}
% 	\caption{Just have a look. Accuracy of the model defined in Equation \ref{eq:mass_prof}. }
% 	\label{fig:model_accuracy}
% 	\end{figure*}

%  \begin{figure*}
% 		\centering
% 	\includegraphics[scale=0.5]{evo_mass_distribution.pdf}
% 	\caption{Just have a look. Evolution of halo mass distributions of the two halo samples. }
% 	\label{fig: mass_distribution}
% 	\end{figure*}

	\section{Universal evolution of the outer halo}\label{evolution_rd}

	
	\begin{figure*}
	\centering
	\includegraphics[scale=0.6]{evo_radii_and_ratio_3epoch_tracez0_clean.pdf}
	\caption{Evolution of depletion radii. The left and right panels correspond to the galactic-size and cluster-size bins in Figure \ref{fig:1}, respectively. $\rid$, $\rcd$, $\rt$ and $\rvir$ are marked as different symbols. %The best-fitting radius-redshift models defined in Equation \ref{equ:evolution_rd} are also plotted as solid lines with different colours. 
         The three stages of the outer halo are divided by regions with different gray levels. The ratio of each radius to the virial radius is also shown in the bottom panel. We also display $2.0 \times \rvir$ as the blue dotted lines, which agree well with the $\rid$ at each redshift. To compare with the expansion rate of the universe, we plot $r\propto a$ as the orange dotted line. }  
	\label{fig:2}
	\end{figure*}



% 	\begin{figure*}
% 	% To include a figure from a file named example.*
% 	% Allowable file formats are eps or ps if compiling using latex
% 	% or pdf, png, jpg if compiling using pdflatex
% 	\centering
% 	\includegraphics[scale=0.6]{evo_radii_and_ratio_3epoch_tracez0_clean_MedMvirRvir.pdf}
% 	\caption{Similar to Figure \ref{fig:2}, here the $\rvir$ is the median value of the halo sample. The $\rvir$ at each redshift is lower than the previous value, but the slope of the $\rvir - z$ relation is very similar. The $\rvir$ at $z=5$ still leads to a bad fit.  \jx{how different is the result if you use the median $\rvir$ of your sample}}  
% 	\label{fig:check_Rvir}
% 	\end{figure*}
%  \begin{figure*}
% 	% To include a figure from a file named example.*
% 	% Allowable file formats are eps or ps if compiling using latex
% 	% or pdf, png, jpg if compiling using pdflatex
% 	\centering
% 	\includegraphics[scale=0.6]{evo_radii_3epoch_tracez0_clean_fit_log.pdf}
% 	\caption{Similar to Figure \ref{fig:2}, but the fitting is in log-space. The change in slope is small.}  
% 	\label{fig:fit_log}
% 	\end{figure*}

%   \begin{figure*}
% 	% To include a figure from a file named example.*
% 	% Allowable file formats are eps or ps if compiling using latex
% 	% or pdf, png, jpg if compiling using pdflatex
% 	\centering
% 	\includegraphics[scale=0.6]{evo_radii_3epoch_tracez0_clean_fit_log_remove_rvir5.pdf}
% 	\caption{Similar to Figure \ref{fig:fit_log}, but I remove the $\rvir$ of cluster sample at $z=5$ from the fitting. The result is good. Maybe the determination of $\rvir$ at high redshift is also difficult? The determination of $\rvir$ is: I calculate the enclosed mass profile and the $\Delta$ profile. So I get the $\Delta(r) -r$ relationship. Then, I use $\Delta_{\mathrm{vir}}$ to interpolate the $\Delta(r) -r$ relation and get the final $\rvir$. I am not sure if it is reasonable to remove the $\rvir$ at $z=5$. If not, maybe show the ratio of $\rid/ \rvir$ is better ?\jx{how different is the result if you use the median $\rvir$ of your sample}}  
% 	\label{fig:fit_log_removez5}
% 	\end{figure*}

	%\begin{figure*}
	% To include a figure from a file named example.*
	% Allowable file formats are eps or ps if compiling using latex
	% or pdf, png, jpg if compiling using pdflatex
	%\centering
	%\includegraphics[scale=0.6]{evo_delta_3epoch_tracez0_clean.pdf}
	%\caption{Similar to Figure \ref{fig:2}, but the evolution of density contrast $\Delta$ within $\rid$, $\rcd$, $\rt$ and $\rvir$. The density contrast derived by the well-defined radii are represented as thick lines.\jx{remove the extrapolations in Delta}}  
	%\label{fig:3}
	%\end{figure*}
	
	\begin{figure}
	% To include a figure from a file named example.*
	% Allowable file formats are eps or ps if compiling using latex
	% or pdf, png, jpg if compiling using pdflatex
	\centering
	\includegraphics[scale=0.6]{evo_delta_three_tracez0_clean.pdf}
	\caption{The evolution of density contrast of three halo samples with different mass. The solid and dashed lines  correspond to $\Delta \left(< \rid \right)$ and $\Delta \left(< \rcd \right)$, respectively. Different colours represent different mass. The dotted gray line represents $0.18 \times  \Delta \left(< \rvir \right)$.}  
	\label{fig:3}
	\end{figure}	


 
	\subsection{Evolution of various halo radii}\label{trace_rd}
	

	
	Figure~\ref{fig:2} shows the evolution of various halo radii in the two halo samples. At high redshifts, it can be difficult to define $\rcd$ due to the absence of a bias trough. As a result, the evolution history of $\rcd$ is not complete in the figure. We only focus on those $\rcd$s that can be clearly identified. 
	
	%For comparison, the evolution of the turnaround and virial radii are also shown. 
The overall shapes of the evolution histories of different radii are similar, with a slope that gradually flattens over time. %Following the empirical fit to the virial mass growth history of \cite{2002ApJ...568...52W}, we fit each of our measured evolution histories using an exponential form 
	% \begin{equation}
	% r_{\rm{d}}(z) = r_{0}\times e^{-\alpha z}, \label{equ:evolution_rd}
	% \end{equation}
	% where $\alpha$ characterises growth rate of the radius, and $1/\alpha$ can be equivalently interpreted as a formation redshift parameter. %We use this formula to fit all $\rvir$, $r_{\rm t}$, the well-defined $\rcd$ and $\rid$. 
	% Generally, the radius-redshift relations can all be reasonably described by this simple exponential function. Note the flat MFR profile of the cluster sample at $z=5$ makes it difficult to accurately determine its $\rid$, so we do not include this $\rid$ in the fit. %With the best-fitting formula, those $\rcd$ in the feeding stage and $\rid$ in the stagnation stage can be obtained by extrapolation. These extrpolated radii are also plotted in the original profiles in Figure~\ref{fig:1} to indicated the expected locations of these scales.
	%\jx{it seems rid grows faster than the exponential fit in the depleting stage}
	
At late times, $\rid$ grows faster than $\rt$. The depletion region enclosed by $\rid$ and $\rt$ will shrink as a halo evolves, eventually leading to the disappearance of the infall zone. Galactic halos have evolved further in this sequence than cluster halos, consistent with the late formation time of cluster halos in the virialized part. In section~\ref{sec:stages}, we will show that one can study the growth phases of halos both qualitatively and quantitatively according to these features.

We find that $\rid$ grows mostly in proportion to the virial radius across redshifts, following
\begin{equation}
	    \rid \simeq 2.0 \times \rvir. 
\end{equation} \footnote{Here the $\rvir$ is obtained by interpolating the stacked enclosed density profiles with $\Delta_{\rm{vir}}$. If we directly use the median $\rvir$ of the halo sample, this scaling relation is slightly changed to $\rid \simeq 2.1 \times \rvir$.} This scaling is consistent with the findings of FH21 at $z=0$. The good proportionality across redshift means that the growth rate of the outer halo is in pace with that of the virialized region. 

The growth of $\rcd$ appears more distinct. For galactic halos, $\rcd$ is well separated from $\rid$ in contrast to the close proximity of the two in cluster halos. This is different from the results of FH21 who found that the two radii are largely proportional to each other across mass. This difference can be explained by the depletion-radius based cleaning in our work, which primarily affects the low mass halos by excluding those in crowded environments, leading to a larger $\rcd$, while high mass halos are barely affected.
%it grows faster than others and even exceeds $\rt$ at very late time.  
	%\jx{better change to 2 significant digits for all fitting pars; do you get $\rid=2.1\rvir$ when using median rvir?}
	
%\jx{ToDo: shift the $r\propto a$ curve of the second panel to be aligned with the turnaround.}: 
We have also plotted a reference $r\propto a$ curve in Figure~\ref{fig:2}, to compare the growth rate of the radii with the expansion rate of the universe. At early times, all the radii grow faster than the expansion of the universe. At late times, the growth rate of most radii have slowed down to roughly the same rate as the background expansion rate. For galactic halos, however, the turnaround radius has slowed down at an increased rate, leading to a much shrinked depletion region. The extra slowing down of the turnaround radius growth can be interpreted as caused by stronger tidal effects when the low mass halos become more clustered around massive ones, and the turnaround radius is the first to feel this effect as it is the outermost edge. We will carry out a detailed study on the effect of the large scale tidal field on halo growth in future work. The extra slowing down has also led halo growth into a distinct phase, as we will discuss in section~\ref{sec:stages}. Note that the proportionality of most radii to the scale factor at late time does not mean that halos are expanding freely on these scales. Instead, as we will show below, the enclosed densities within the various radii increase significantly relative to the background density at the late times.

   
	\subsection{Evolution of enclosed densities}
	In Figure \ref{fig:3} we show the evolution of density contrast enclosed by $\rcd$ and $\rid$, where $\Delta$ is defined as the ratio between the enclosed density and the mean matter density of the universe. In addition to the two mass bins studied above, we have further included an intermediate halo bin with $10^{12.75}<M_\mathrm{vir}<10^{13.30} M_{\odot} \, h^{-1}$. In general, the density contrasts increase with the expansion of the universe, reflecting that the halo stands out more clearly over time from the background on these scales. Moreover, the evolution of $\Delta(\rid)$ is approximately universal, which is mostly flat up to $a=0.5$ and grows faster afterwards. Intriguingly, such an evolution is well in proportion to the evolution of the virial density contrast, with 
	\begin{equation}
	    \Delta(\rid)\simeq 0.18\Delta(\rvir).
	\end{equation}
	This good proportionality indicates that the inner depletion radius may be modelled following the dynamics of spherical collapse. We will investigate such models in future work.
	
 
    Despite the overall proportionality to the virial density, the detailed evolution for the three mass subsamples still shows some interesting differences from each other especially at low redshifts.  To better understand this, we check whether there are other properties beyond halo mass that affect the depletion radii and densities. Indeed, as shown in Appendix~\ref{multiple}, the $\Delta \left(< \rid \right)$ at $z=0$ is more dependent on the halo formation time than on mass. The early formed halo exhibits a significantly high value of $\Delta \left(< \rid \right)$. Among our three halo samples, the galactic sized sample contains a higher fraction of early-forming halos. This explains its higher $\Delta \left(< \rid \right)$ at $z=0$ than the other two halo samples.
	
	The evolution of the $\Delta(\rcd)$ shows an obvious mass dependence. This mass dependence is not observed in the $z=0$ FoF sample of FH21. This again can be explained by the extra depletion-radius based cleaning applied to our sample, which selects isolated low mass halos that have a lower density environment, while massive halos are barely affected. Without cleaning, the enclosed density within $\rcd$ is also found to be approximately universal, following $\Delta(\rcd)\simeq 44.77\times a^{1.79}$ over the redshift range where $\rcd$ can be identified (see Appendix~\ref{sec:nonclean}). However, as we discussed in section~\ref{data_simulation} and Appendix~\ref{app:overlap}, cleaning is a necessary process to obtain a self-consistent depletion-bound catalog, so we will still focus on results from the clean catalog.
	%The evolution for all three masses reaches $\Delta \left(< \rid \right)\sim 50$ at the lowest redshifts. This is again lower than the value of $\sim 60$ found in FH21, due to the cleaning of the halo catalog described in section~\ref{data_simulation}.
	
	%After the galactic-size halo entering the saturation epoch, its $\Delta \left(< \rid \right)$ increases rapidly and exceeds that of the other two halo samples.
\subsection{The scaled outer mass profile}\label{trace_enclosed_mass}

 \begin{figure*}
		\centering
	\includegraphics[scale=0.45]{evo_enclosed_mass_tracez0_clean_enlarge_r01-10.pdf}
	\caption{Evolution of  the scaled enclosed mass profiles. Each profile has been scaled by the virial radius $\rvir$ and  the virial mass $M_{\mathrm{vir}}$. The inner depletion radii $\rid$ are also marked with a cross on each profile. }
	\label{fig:scaled_enclosed_mass}
	\end{figure*}

%  \begin{figure*}
% 		\centering
% 	\includegraphics[scale=0.65]{evo_ratio_Delta_id_to_vir.pdf}
% 	\caption{Check. The ratio in left panel is calculated by $(M_{\mathrm{id}}/M_{\mathrm{vir}})/(\rid/ \rvir)^3$. The ratio in right panel is calculated by $\Delta(<r_{\mathrm{id}})/\Delta(<r_{\mathrm{vir}}) $ shown in Figure ~\ref{fig:3}. They are consistent.}
% 	\label{fig:check_ratio}
% 	\end{figure*}

  \begin{figure}
		\centering
	\includegraphics[scale=0.65]{evo_enclosed_mass_tracez0_clean_slope_r01-10.pdf}
	\caption{Evolution of  the power-law slopes measured from the scaled enclosed mass profiles (Figure ~\ref{fig:scaled_enclosed_mass}) between $\rvir$ and $\rid$. The blue and red colors correspond to the galactic halo and the cluster halo sample, respectively.}
	\label{fig:scaled_enclosed_mass_slope}
	\end{figure}

 % \begin{figure*}
	% 	\centering
	% \includegraphics[scale=0.45]{evo_enclosed_mass_tracez0_clean_enlarge.pdf}
	% \caption{Evolution of  the scaled enclosed mass profiles. Each profile has been scaled by the virial radius $\rvir$ and  the virial mass $M_{\mathrm{vir}}$. The inner depletion radii $\rid$ are also marked with a cross on each profile. The bottom panels show zoomed-in views to the profiles around $\rid$.}
	% \label{fig:scaled_enclosed_mass}
	% \end{figure*}

 %  \begin{figure*}
	% 	\centering
	% \includegraphics[scale=0.45]{evo_enclosed_mass_tracez0_clean_enlarge_scaled_by_rid.pdf}
	% \caption{Similar to Figure \ref{fig:scaled_enclosed_mass}, but the enclosed mass profiles are scaled by $\rid$ quantities. Here the virial radius $\rvir$ of each profile is marked with a cross.}
	% \label{fig:scaled_enclosed_mass_scaled_by_rid}
	% \end{figure*}
	
    The good proportionality of both the depletion radius and depletion density with the corresponding virial quantities suggests that the mass profile is universal in between the virial and the inner depletion radii, across mass and redshift. This is approximately the case shown in Figure~\ref{fig:scaled_enclosed_mass}, where the mass profiles between these two radii become largely unified when scaled by the virial quantities, for both galactic and cluster halos from $z=5$ to $z=1$. 
    
    In these coordinates, the depletion radii and the corresponding masses are tightly clustered around a single point. The two $z=5.04$ points and the $z=0$ point for the galactic halo appear as outliers, which can be attributed to the difficulty in accurately identifying their inner depletion radii from the corresponding kinematic profiles in Figure~\ref{fig:1}.
%     We can well describe this part of the profile with a power-law as
%     \begin{equation}
%         \frac{M}{M_{\mathrm{vir}}}=(\frac{r}{\rvir})^{0.66}\quad\quad \mathrm{for}\;\;\rvir<r<\rid. \label{eq:mass_prof}
%     \end{equation} 
% To check the accuracy of this model, we also show the residuals between the enclosed density profiles and the model in the bottom panels of Figure ~\ref{fig:scaled_enclosed_mass}.
   
    %The general picture about halo growth is still maintained in these scaled coordinates, with a decaying outer profile outside $\rid$. However, the growth of the inner halo is now manifested as a growing scaled profile within $\rvir$, while the part in between $\rvir$ and $\rid$ remains largely static.
    
    A power-law function of $M\propto r^{0.66}$ can describe the profile in between the two radii to an accuracy of $\lesssim 10\%$ for the redshift and mass ranges investigated. As the virial and inner depletion radii bracket the physically growing part of a halo, such a unified power-law profile can be regarded as a manifestation of the similarity of halo growth~\citep{1984ApJ...281....1F,1985ApJS...58...39B}. 
    
    However, we emphasise that the unification of the profiles is only approximate. A weak but clear evolution of the scaled mass profile in this part is still present, suggesting the break-down of strict self-similarity. Quantitatively, the power-law index of the outer profile between $\rvir$ and $\rid$ decays slowly over time, from $0.8$ at $z=5$ to $\sim 0.6$ at $z=0$, as shown in Figure~\ref{fig:scaled_enclosed_mass_slope}. This deviation from exact universality also means that strict proportionality between $\rid$ and $\rvir$ can not hold simultaneously with strict proportionality between their enclosed densities, and some weak redshift evolution in the scalings is expected. More sophisticated studies of the profiles and of the inner depletion quantities will need to reflect this evolution. %Despite this, Equation~\eqref{eq:mass_prof} is a good first-order approximation which is accurate within $\sim 10\%$ for the mass and redshift range covered here. %\jx{can you also try an alternative plot scaled by the $\rid$ quantities?} 
	
\section{Evolution phases of a halo}

\subsection{Three evolution stages of the outer halo}\label{sec:stages}
    By analysing the evolution of bias and MFR profile, we can phenomenologically divide the evolution of the outer halo into three stages.
	
    We take the evolution of a typical galactic-size halo as an example. The first stage is before the bias profile forms a clear trough (single minimum) on the depletion scale, %The bias decreases monotonically till the turnaround radius and beyond. 
    indicating that there is plenty of material to be fed to the halo. %material around the boundary of halo has not been adequately depleted by the accretion process. 
    At the same time, the MFR profile shows a prominent trough indicating that mass accretion and ongoing depletion is very active. We name this epoch as the ``feeding stage".\footnote{The bias goes to negative value on scales of a few $\rm{Mpc}$ for the two highest redshift bins. This is due to our selection of isolated halos on the depletion scale. The selection barely affects high mass halos, but tends to select low mass halos that are surrounded by under-dense regions at early time.} %\jx{what is trough at around 2Mpc? the trough of the larger parent structure? the final depletion radius based selection have selected halos that are initially in voids? If you do not clean the catalog, is it different?}
    
The feeding stage continues till $z\sim 0.76$, when a clear trough finally appears in the bias profile, representing a relative shortage of material in the halo neighbourhood. We name this stage as the ``depleting stage" during which the depletion is still active with a clear trough in the MFR profile, while the consequence of depletion is also visible as a bias trough. %The trough also indicates a poor environment in contrast to a rich one in the feeding stage. %The halo bias decreases with increasing radius until it approaches a trough, beyond which it begins to increase with the radius and and eventually becomes nearly constant on the large scales. 

	With the depletion of the environment, the halo expands and the depletion radius becomes closer to the turnaround radius, leaving less and less material to be fed to the halo. This is accompanied with a decay in the depletion rate, which eventually approaches zero at $z\sim 0.12$, after which there is no obvious infall region around the halo. The growth of the halo almost halts, and we name this final stage as the ``stagnation stage".
	%The third epoch can be classified from the evolution of MFR profile, which provides a dynamic perspective to study the halo evolution. At redshift $z>0.12$, the MFR is very close to zero in the inner region, and increases with the radius until it reaches its maximum value at $\rid$. Beyond this radius, the MFR quickly decreases to 0 at $r_{\rm{t}}$ where the Hubble flow and halo accretion reach an equilibrium. However, at redshift $z<0.12$, there is no obvious mass inflow in the MFR profile. This implies that the competition between the inflow material accreted by the halo and the outflow material caused by the Hubble expansion has reached a dynamic equilibrium. The growth of halo has almost stopped, and the accretion process is saturated. This epoch can be defined as "saturation epoch".
	
	In the LCDM universe, more massive halos form later. Such a bottom-up structure formation paradigm is also reflected in the evolution phases of the outer halo. As shown in Figure~\ref{fig:1}, cluster halos transit to their depleting stages later ($z\sim 0.53$) than galactic halos ($z\sim 0.76$), and have not yet reached the stagnation stage as galactic halos have. 
	
 	The three evolution stages are distinguished with different levels of shading in Figure~\ref{fig:2}. Intriguingly, the depletion stages for both cluster and galactic halos start when the ratio between the turnaround radius and the inner depletion radius reaches slightly beneath $2$. For galactic halos, the stagnation stage starts when $\rt/\rid$ shrinks below $\sim 1.5$. This ratio measures the width of the active accretion zone, and depicts the richness of the environment that is feeding halo growth.
	
	\subsection{Connection to the inner growth phases}
	\citet{Zhao03} proposed that the growth of the halo structure within the virial radius can be separated into a fast and a slow growth phase. In the fast phase, the halo mass grows faster than the Hubble flow and the inner structure of halo is mainly constructed during this phase. In the slow growth phase, the mass grows nearly in proportion to the Hubble rate and the mass enclosed by scale radius $r_{\rm{s}}$ has almost stopped growing. \citet{Zhao03} suggested that the separation of the two phases can be found at the time where $V_{\rm{vir}}H^{1/4}(z)$ peaks. We find that this transition time coincides with the time of the peak depletion rate, $\mathrm{MIR}(\rid)$. This is true for all the three halo mass samples that we examined, while in Figure~\ref{fig:peak_dep} we only show the galactic mass sample as an example. The transition between the fast and slow growth can also be observed in Figure~\ref{fig:1}, where the density grows relatively less significantly within $\rid$ after $z=2$ for the galactic halo.
	
	According to the evolution of the depletion rate, we can call the two phases alternatively as accelerated and decelerated depletion phases. The coincidence between the two peak times demonstrates that the evolutions of the inner and outer halos are in concert, and detailed studies of the accretion and depletion process in the outer halo could help us to better understand the evolution of the inner halo, and vice versa. In fact, even though the inner halo growth has been separated into two distinct phases, the growth rates of both the virial radius and the $r_{\rm s}$ evolve smoothly over time without a clear cut to divide the two. Thus \citet{Zhao03} have to propose a somewhat indirect proxy of $V_{\rm{vir}}H^{\gamma}(z)$ for separating the two phases, with $\gamma\sim 1/4$ being an empirical parameter. Our results suggest that the depletion rate defined in the outer halo can serve as a more objective and physical proxy for identifying halo growth phases even for the inner halo.
	
	The transitions of the three stages of the outer halo evolution are also shown in Figure~\ref{fig:peak_dep}. These outer halo transitions all happen in the slow growth or decelerated depletion phase. This is consistent with the overall picture that halos grow from inside-out, with the inner part built-up on a smaller timescale than the outer part. 
	
%	Show that the MFR and mass growth rate of the inner halo decays smoothly over time, with no clear separation between fast and slow phase. In this sense, the MIR provides a more objective and unique separation.
	
	%Show that the Rid is well proportional to Rvir. Do not show Rvir in Fig 2. Show Rvir and Rs in plots here.
	
	\begin{figure}
	\includegraphics[width=0.5\textwidth]{MFRid_evo_lowmass.pdf}
	%\includegraphics[width=0.5\textwidth]{}
	\caption{Evolution of the depletion rate, $\mathrm{MIR}(\rid)$, for galactic size halos, scaled by the virial quantities at $z=0$ (left axis and black line). The depletion rate peaks at around $a=0.35$ ($z\sim 2$). This coincides with the transition between fast and slow growth phases of the inner halo, which can be identified at the peak time of the $V_{\rm vir}H(z)^{1/4}$ evolution according to \citet{Zhao03}. The $V_{\rm vir}H(z)^{1/4}$ (right axis and blue line) is normalized by its $z=0$ value. The three evolution stages of the outer halo are also shown in different shadings.}\label{fig:peak_dep}
	\end{figure}
    
	\section{Summary and Conclusions}\label{conclusion}
	
	The depletion radii have been proposed to describe the boundary of a halo according to the expected evolution of halo profiles in \citet{FH21}. By tracking halo evolution in an LCDM simulation, in this work we verify that the evolution of the density and bias profiles are indeed highly consistent with these expectations, with the growth of a halo accompanied by the depletion of its environment. The inner depletion radius, $\rid$, is shown to separate the growing part of the halo from a decaying environment, while also identifying the starting point of the depletion process in the bias profile. The characteristic depletion radius, $\rcd$, identifies the most depleted location in the bias, which is however only visible at late stages of halo growth.
	
	Both depletion radii expand with the growth of a halo. The evolution of $\rid$ closely follows the evolution of the virial radius for halos of a given mass, with $\rid\simeq 2\rvir$. Its enclosed density also evolves in proportion to the virial density, with $\Delta(\rid)\simeq 0.18\Delta(\rvir)$, irrespective of redshift or halo mass. These universal scaling relations are a consequence of the approximately universal mass profile in between these two radii. When scaled by the corresponding virial radii (or equivalently the depletion radii), the mass profiles between the two radii become largely identical across the redshifts and masses covered in this study, following a $M\propto r^{0.66}$ law, with residual evolutions observed at $\lesssim 10\%$ level. As these two radii are expected to bracket the physically growing part of a halo, the similarity of the profiles can be interpreted as a reflection of the approximate self-similarity of halo growth.
	
	For cluster halos, their $\rcd$ form relatively late and are close to the $\rid$s. For galactic halos, their $\rcd$ can be identified up to a higher redshift, which however are closer to their corresponding turnaround radius. Note the turnaround radii of these low mass halos are also intrinsically closer to their $\rid$ at low redshift, reflecting a lack of material to be fed to halo growth and the ceasing depletion process. 
	
	According to whether the $\rcd$ can be clearly identified and whether the depletion process is active, we can broadly divide halos into three evolution stages. These stages show different widths of their active accretion zones quantified by the ratio between the turnaround radius and $\rid$. Moreover, according to the evolution of the depletion rate, we can unambiguously divide halos into accelerated and decelerated depletion phases, which well correspond with the fast and slow growth phases of the inner halo known previously.
	
	These results illustrate the great potential of using the depletion radii as new probes for halo evolution. In a companion work, we will also show that a more concise halo model can be built using these radii. Along with further theoretical understanding of them in analytical models, we expect many more applications of them can be found which can boost our understandings of halo evolution and structure formation in general.

%\iffalse
\section*{Acknowledgments}
	We thank Yifeng Zhou and Xiaokai Chen for useful discussions. This work is supported by NSFC (11973032, 11890691, 11621303, 12133006), National Key Basic Research and Development Program of China (No.\ 2018YFA0404504), 111 project (No.\ B20019), and the science research grants from the China Manned Space Project (No.\ CMS-CSST-2021-A03). We thank the sponsorship from Yangyang Development Fund. We gratefully acknowledge the support of the Key Laboratory for Particle Physics, Astrophysics and Cosmology, Ministry of Education. The computation of this work is partly done on the GRAVITY supercomputer at the Department of Astronomy, Shanghai Jiao Tong University.

\appendix
\section{The profile evolution when halos overlap on depletion scale}\label{app:overlap}

\begin{figure*}
	\centering
	\includegraphics[scale=0.5]{evo_TrackId140802_tracez0_density.png}
	%\label{fig:A1}
        \hspace{0.5in}
	\includegraphics[scale=0.5]{evo_prof_TrackId140802_tracez0.pdf}
	%\label{fig:A2}
	\caption{Left: Evolution and merger history of a pair of halos. The small halo is placed at the origin. The distributions of dark matter near the two halos are projected perpendicular to their alignment with a depth of 6 $\mathrm{Mpc}\,h^{-1}$. The black solid (dotted) inner circles represent the virial radii of the large (small) halo at different redshifts. The outer circle represents 2.5 times the virial radius, which is an approximate estimation of the characteristic depletion radius. The small halo eventually becomes a subhalo (highlighted with a red circle) of the large one at $z=0$. Right: Evolution of the density and radial velocity profiles around the small halo. %Each velocity profile is normalized by the virial velocity at the corresponding redshift.
	} \label{fig:merger}
\end{figure*}


In Figure \ref{fig:merger} we show the evolution history of a pair of halos that merge into a single halo at $z=0$. The density and velocity profiles are plotted from the center of the small halo. The evolution of the density profile is relatively simple, with the large halo appearing as a density peak outside the small one. The two halos are well separated initially, and a velocity profile typical of an isolated halo is observed at $z=3.03$, with a single trough reflecting the infall of matter towards the small halo. This trough becomes shallower and gradually shifts outwards as the halo grows. From $z\sim 0.5$, the velocity profile becomes very different from the expected form around an isolated halo, showing a peak within the trough before reaching the Hubble flow on large scale. This peak is due to the infall towards the large halo. At this time, the large halo is located at $r\approx3\mathrm{Mpc}\,h^{-1}$, forming the outer valley in the velocity profile. The material surrounding it fall with a larger velocity than the small halo, especially around the depletion radius of the large halo. These materials are seen as an outflow relative to the small halo, creating the velocity peak in between the two halos. As shown in the density profile, this process leads to the formation of a relatively low-density region at $r\approx1.5\mathrm{Mpc}\,h^{-1}$.
As the two halos further approach each other, $z=0.25$, the peak also moves to a smaller radius, and the outer trough further carves in reflecting the accretion by the large halo. After the small halo enters the depletion radius of the large one at $z=0.12$, the velocity profile is again dominated by a single trough, which however is due to the infall towards to large halo instead of the small one. %Similar evolution is observed in the MFR profile as well. 
Note at $z=0.12$ the two halos are still well separated outside their virial boundaries, while the infall regions of the two have merged. 

This reflects that the virial radius is not suitable for isolating halos when studying their evolutions out to the depletion scale. In fact, when using the depletion radius to define separate halos (outer circles in the left panel), the merger of the two halos starts soon after $z \sim 0.53$. After that the small halo can be no longer treated as an independent one, leading to complex structures in its velocity and MFR profiles.

This example illustrates the importance of defining halos self-consistently when studying the depletion radius. It could still be possible to study the depletion features for halos overlapping on the depletion scale. However, more careful treatments are needed to separate the overlapping objects and to account for the aspherical shape of the boundary, such as those done in \citet{2021ApJ...915L..18L}, which are not purchased in this work. %According to FH21, for halos of a given mass at $z=0$, a good estimation of the characteristic depletion radius is $\rcd \simeq 2.5 \rvir$. Starting from the original FoF catalog, we remove any halo whose distance to a more massive neighbour is smaller than their summed $\rcd$'s. 

\section{Evolution of the radius and density without cleaning}\label{sec:nonclean}

In Figure~\ref{fig:evolution_radii_noclean} and Figure~\ref{fig:evo_delat_noclean} we show the evolution of the halo radii and densities extracted from the full FoF catalog without the depletion-radius based cleaning. For low mass halos, it becomes difficult to define their inner depletion radii due to the proximity to massive neighbours, as discussed in Appendix~\ref{app:overlap}. As a result, the $z=0$ measurements for $\rid$ are missing, and the low redshift results are generally ill-behaved. The $\rt$ results are affected similarly. Despite this, the $\Delta(<\rcd)$ results are more unified across mass. Note the high mass halos are barely affected by the cleaning.

	\begin{figure*}
	% To include a figure from a file named example.*
	% Allowable file formats are eps or ps if compiling using latex
	% or pdf, png, jpg if compiling using pdflatex
	\centering
	\includegraphics[scale=0.6]{evo_radii_and_ratio_3epoch_tracez0_noclean.pdf}
	\caption{Similar to Figure \ref{fig:2}, but using the original FoF halo catalog before cleaning. Note the determination of $\rid$ can be problematic for galactic halos at low redshifts. }  
	\label{fig:evolution_radii_noclean}
	\end{figure*}
	
	\begin{figure}
	% To include a figure from a file named example.*
	% Allowable file formats are eps or ps if compiling using latex
	% or pdf, png, jpg if compiling using pdflatex
	\centering
	\includegraphics[scale=0.6]{evo_delta_three_tracez0_noclean.pdf}
	\caption{Similar to Figure \ref{fig:3}, but using the original FoF halo catalog before cleaning. Note the determination of $\rid$, hence $\Delta(<\rid)$, can be problematic for galactic halos at low redshifts.}  
	\label{fig:evo_delat_noclean}
	\end{figure}	


\section{Dependence of the depletion radii on multiple halo properties at $z=0$}\label{multiple}

% \begin{figure*}
% \centering
% \includegraphics[scale=0.35]{pixel_Mandproperty_phy_clean.pdf}
% \caption{Two-dimensional joint dependence of the depletion radius and density contrast on halo mass and other properties at $z=0$. Only the bins with more than 50 halos are displayed. The four rows from top to bottom represent the joint dependence of $\rcd$, $\rid$, $\Delta(<\rcd)$ and $\Delta(<\rid)$, respectively. The values of $\rcd$, $\rid$, $\Delta(<\rcd)$ and $\Delta(<\rid)$ are shown on each pixel. The unit of  $\rcd$ and $\rid$ here is $\mathrm{Mpc}/h$.}  
% \label{fig:joint_dependence}
% \end{figure*}

\begin{figure*}
\centering
\includegraphics[scale=0.35]{pixel_Mandproperty_phy_clean_scaled_by_rvir.pdf}
\caption{Two-dimensional joint dependence of the depletion radius and density contrast on halo mass and other properties at $z=0$. Only the bins with more than 50 halos are displayed. The four rows from top to bottom represent the joint dependence of $\rcd$, $\rid$, $\Delta(<\rcd)$ and $\Delta(<\rid)$, respectively. The values of $\rcd$, $\rid$, $\Delta(<\rcd)$ and $\Delta(<\rid)$ are shown on each pixel. Here $\rcd$ and $\rid$ have been scaled by $\rvir$.}  
\label{fig:joint_dependence_scaled_by_rvir}
\end{figure*}

In Figure \ref{fig:joint_dependence_scaled_by_rvir}, we present the two-dimensional joint dependence of depletion radius $\rcd$ and $\rid$ as well as their enclosed density contrast  $\Delta(<\rcd)$ and $\Delta(<\rid)$ on halo mass $M_{\rm{vir}}$ and other halo properties at $z=0$. These physical properties include halo formation time $a_{1/2}$, concentration $V_{\rm{max}}/V_{\rm{vir}}$, shape $e$ and spin $j$. The detailed definition and calculation of these parameters can be found in \cite{Han19} and FH21. Only halos in the depletion catalog are used. 

These results extend the measurements in FH21, and show that the depletion properties are sensitive to many halo properties. Detailed studies on the origin, evolution and interplay of these dependences will be carried out in future works.

% \section{Results from the FoF catalog before cleaning}
% \jx{This section is a memo. will be removed in submitted version.}
% 	\begin{figure*}
% 	% To include a figure from a file named example.*
% 	% Allowable file formats are eps or ps if compiling using latex
% 	% or pdf, png, jpg if compiling using pdflatex
% 	\centering
% 	\includegraphics[scale=0.45]{evo_prof_tracez0_noclean_ScaledByz0.pdf}
% 	\caption{Evolution for the no clean catalog. Just to have a look.
% 	}
% 	\label{fig:noclean}
%     \end{figure*}

% \section{Evolution of the scaled density profile}\label{app:scaled_densit}

% Figure~\ref{fig:scaled_density} shows the evolution of the density profile scaled by the corresponding virial radius and enclosed virial density at each redshift. Evolution in the differential density profiles can be observed in between $\rvir$ and $\rid$. 
%  \begin{figure*}
% 		\centering
% 	\includegraphics[scale=0.45]{evo_density_tracez0_clean_adjust_rcen.pdf}
% 	\caption{Evolution of  the scaled halo density profiles. Each profile has been scaled by the virial radius $\rvir$ and  the enclosed virial density $\rho(<\rvir)$. The inner depletion radii $\rid$ are also marked at each profile.}  
% 	\label{fig:scaled_density}
% 	\end{figure*}


%\fi
\bibliography{depletion_boundary}{}
\bibliographystyle{aasjournal}

\end{document}

% End of file `sample631.tex'.
