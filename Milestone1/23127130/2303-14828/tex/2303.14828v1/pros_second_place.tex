\noindent\textbf{Overview:} HRDA is a context-aware high resolution domain adaptation method for semantic segmentation~\citep{hoyer2022hrda}. The method comprises a multi-resolution training approach for UDA that combines small high-resolution crops and large low-resolution crops to preserve fine segmentation details as well as capture long-range context information. Predictions from both resolution crops are fused using a learned scale attention, which can enable adapting objects at the better-suited scale. As a backbone of their framework, this solution uses the DAFormer~\citep{hoyer2022daformer} architecture that is based on a Transformer network which utilizes self-training. The latter uses pseudo labels generated by a teacher network to iteratively adapt the model to the target domain. Similar to the first place solution, the team concludes after an ablation study in Table \ref{tab:hrda_ab} that rare class sampling (RCS) used to train DAFormer is ineffective for performance.
\\

\noindent\textbf{Results:} Results of this solution are reported in Table \ref{tab:baseline_results_new}, showing that HRDA yields a remarkable improvement in mIoU and mean accuracy compared to the source-only method. In addition, a detailed breakdown of the method's performance is reported in Table \ref{table:zerov2_class_test}. The participating team also provides an ablation study with different source datasets and RCS configurations. In Table \ref{tab:hrda_ab}, the ablation study shows that training on the Zerowaste real-world dataset alone is enough to yield the best performance and that rare class sampling actually deteriorates it.

\begin{table}[h!] \vspace{-10pt}
    \centering
    \begin{tabular}{l c c}
    \toprule
         Source Dataset & RCS & Validation mIoU  \\
         \midrule
         Synthwaste & $\times$ & 20.4\\ 
         Synthwaste + Zerowaste & $\times$ & 45.5 \\
         Synthwaste + Zerowaste, Equal Size & $\times$ & 51.1 \\
         Zerowaste & \checkmark & 47.1 \\
         Zerowaste & $\times$ & \textbf{56.6} \\
         \bottomrule
    \end{tabular}
    \caption{Ablation study by \texttt{HRDA}, the second place solution, examining different source data and the RCS (rare class sampling) configuration of the DAFormer backbone. \vspace{-20pt}}
    \label{tab:hrda_ab}
\end{table}
