Given a local coordinate system with an x-axis in direction of travel and a y-axis from right to left (as depicted in Fig.~\ref{intro_fig}), a consecutive series of locations on a trajectory are classified as $RLC$ if the vehicle at one point crosses the right lane marking and continuously has negative gradient before and after crossing. Similarly, a consecutive series of locations are classified as $LLC$ if the vehicle at one point crosses the left lane marking and continuously has a positive gradient before and after crossing. A waypoint that is neither classified as $LLC$ nor $RLC$ is considered as $LK$. Therefore, lateral displacements that do not lead to or follow a lane crossing are considered as $LK$ manoeuvre.


\begin{table}[]
\centering
\caption{List of selected input features categorised based on their contribution to representation of a driving scene$^a$}
\label{tab:feature_list}
\begin{tabular}{@{}cl@{}}
\toprule
Type                                                                                  & \multicolumn{1}{c}{Features}                                                                              \\ \midrule
\multirow{3}{*}{\begin{tabular}[c]{@{}c@{}}TV \\ Dynamics\end{tabular}}               & (1) Longitudinal acceleration of TV                                                                       \\
                                                                                      & (2) Lateral distance of TV to   the left lane marking                                                     \\
                                                                                      & (3) Lateral acceleration of   the prediction target                                                       \\ \midrule
\multirow{12}{*}{\begin{tabular}[c]{@{}c@{}}Interaction\\  with   SVs$^b$\end{tabular}} & (4) Longitudinal distance of TV   to PV,                                                                  \\
                                                                                      & (5) Longitudinal distance of   TV to RPV                                                                  \\
                                                                                      & (6) Longitudinal distance of TV to FV                                                                     \\
                                                                                      & (7) Lateral distance of TV to RV                                                                          \\
                                                                                      & (8) Lateral distance of TV to RFV                                                                         \\
                                                                                      & (9)   Relative longitudinal velocity of TV w.r.t. PV                                                      \\
                                                                                      & (10)   Relative longitudinal velocity of TV w.r.t. FV                                                     \\
                                                                                      & (11) Relative lateral velocity   of TV w.r.t. PV                                                          \\
                                                                                      & (12)   Relative lateral velocity of TV w.r.t. RPV                                                         \\
                                                                                      & (13)   Relative lateral velocity of TV w.r.t. RV                                                          \\
                                                                                      & (14)   Relative lateral velocity of TV w.r.t. LV                                                          \\
                                                                                      & \begin{tabular}[c]{@{}l@{}}(15)   Relative longitudinal acceleration of TV \\      w.r.t RPV\end{tabular} \\ \midrule
\multirow{3}{*}{\begin{tabular}[c]{@{}c@{}}Environment\\  Context\end{tabular}}       & (16) Existence of left lane                                                                               \\
                                                                                      & (17) Existence of right lane                                                                              \\
                                                                                      & (18) lane width                                                                                           \\ \bottomrule
\end{tabular}
\footnotesize{$^a$TV: Target Vehicle, SV: Surrounding Vehicle, PV: Preceding Vehicle, RPV: Right Preceding Vehicle, RV: Right Vehicle, RFV: Right Following Vehicle, FV: Following Vehicle, LV: Left Vehicle, $^b$ In case of non-existance of a surrounding vehicle, longitudinal distances are set to 400 meters, lateral distances are set to 3 times the lane width, and relative velocities and accelerations are set to 0.
}\\

\end{table}

\begin{figure}[!t]
\centering
\includegraphics[width=3.4in]{Figures/Manoeuvre_Prediction.pdf}
\caption{Multimodal manoeuvre prediction outputs.}
\label{fig:man_outputs}
\end{figure}
