% EA
% GP
% grammar based - problems & solutions
% our work as solution
% biological inspiration
% contribution
% structure



% EVOLUTIONARY ALGORITHMS
%\glspl{ea} are optimization algorithms inspired by the biological processes of natural evolution. A population of individuals (candidate solutions) is evolved over several generations guided by a fitness function. Similar to nature, these individuals suffer similar processes as the selection of individuals, reproduction and genetic variation.
%\glspl{ea} present some issues that affect their performance such as parameter tuning, premature convergence and lack of diversity. To tackle these issues different modifications to \glspl{ea} propose new approaches to the representation of individuals, selection methods, genetic operators and parameter selection. 
% TODO: alguma frase que diga que mutação vai ser o que nos vamos focar agora


%Although most works in the literature continue using a static parameter for mutation and crossover, it has been shown that dynamic parameters may improve the search and introduce more diversity to the population.
%Regarding mutations some of the works in the literature use information from the individuals phenotype or use an adaptive method.
% TODO
% TODO

%\gls{gp} is an \gls{ea} that evolves solutions as programs. % Dizer mais alguma coisa
%Over the years many variants of \gls{gp} have been proposed, namely concerned with how the solutions are represented and evolved. Most approaches use a grammar to set restrictions to the search space \cite{McKay2010}.
%\gls{ge} \cite{handbookge} is one of the most popular grammar-based \gls{gp} methods.
%The individuals are represented as a genotype (a string/vector of integers) which is translated into a phenotype (an executable function) through a grammar. 
% TODO: talvez mencionar que a mutação e o crossover são das principais causas estudadas para a redundancia e localidade \cite{Byrne2009,hugosson2007}
%This approach is relevant, but suffers from high redundancy \cite{Thorhauer2016} and poor locality \cite{Thorhauer2014}, damaging the efficiency of evolution.
%These shortcoming motivated researchers to propose different initialisation methods \cite{Nicolau2017}, representations \cite{Loureno2018,Megane2022}, genetic operators, but also to investigate grammar design \cite{Nicolau2018}.% Grammar design can be used to to incorporate expert knowledge \cite{Nicolau2018}. 

Grammar-based \gls{gp} algorithms have been an important tool for the evolution of computer programs since their inception \cite{whigham,Ryan1998,McKay2010}.
The most popular approach is \gls{ge} which is notable for decoupling the genotype and the phenotype, using a grammar to translate a data structure into an executable program.
The representation and variation operators used by \gls{ge} present some known issues, such as low locality and high redundancy. The first means that small changes in the genotype can cause significant changes in the phenotype, and the second means that most modifications do not affect the phenotype. These characteristics result in a bad trade between exploration and exploitation, which makes the algorithm perform similarly to random search \cite{Whigham2015}.

\gls{sge} \cite{Loureno2018} is a variant of \gls{ge} that uses a different representation for the individuals. The genotype comprises several lists, one for each non-terminal in the grammar, and each list contains the indexes of the rules to be expanded.
\gls{sge} shows better performance when compared to \gls{ge}, and other grammar-based approaches \cite{Loureno2017}, but also improved locality and lower redundancy when compared to standard \gls{ge} \cite{Medvet2017,Loureno2016}, in part due to its operators.
This representation allows the recombination operator to be grammar-aware, preserving the list of each non-terminal. On the other hand, the grammar does not inform the mutation operator.
Mutation in \gls{sge} changes the production rule of the non-terminal selected to mutate.
This operator affects all genes with the same frequency, regardless of grammatical context.
Using a static and equal value for all non-terminals fails to consider that not all mutations are equally destructive.

Specific genes can play an essential role in the solution and, when mutated, may completely ruin the phenotype behavior.
On the other hand, some genes may have a tuning role; in this case, a mutation will only result in a minor adjustment to the solution.
Despite these differences, both types of genes are equal in the eyes of mutation.
Biological processes have evolved to prevent this phenomenon. 
Gerhart et al. \cite{gerhart2007} propose that there are core components vital to the individual which remain unchanged for long periods and regulatory genes that combine existing core components and change frequently.
The result is a system that can quickly adapt to new environments through regulatory changes while preserving the core components that ensure individuals are functional.
We can replicate this behavior in grammar-based algorithms using different adaptive mutation probabilities for each non-terminal. 
This approach enables the system to autonomously regulate mutation rates to match the impact of changes to that non-terminal.
Note that this solution is only as effective as the correlation between non-terminals and mutation impact.
It follows that the effectiveness of this mutation is related to grammar-design \cite{Nicolau2018,nicolau2004,dick2022,Hemberg2008PreIP}, as grammars with more rules enable finer tuning of mutation probabilities.
The grouping of productions within each non-terminal is also relevant; separating low and high impact changes into separate rules should improve performance.

In this work, we propose Adaptive Facilitated Mutation, a biologically inspired grammar-aware self-adaptive mutation operator for \gls{sge} and its variants.
%Our approach automatically adjusts the mutation rate, alleviating the laborious trial-and-error procedure often required to tune evolutionary algorithms.
Furthermore, we propose "Function Grouped Grammars", a method for grammar design that empirically outperforms grammars commonly used for regression in \gls{ge}.
We compare our approach to standard grammar and mutation and find that, when combined, Function Grouped Grammars and Adaptive Facilitated Mutation are statistically superior or similar to the baseline in three relevant \gls{gp} benchmarks.

The remainder of this work is structured as follows: 
First, section \ref{sec:background} presents the background necessary to understand the work presented. 
Section \ref{sec:mutation_levels} presents the proposed mutation and grammar-design method. 
Section \ref{sec:experimental_setup} details the experimentation setup used, and Section \ref{sec:results} the experimental results regarding performance and analysis of probabilities. 
Section \ref{sec:conc} gathers the main conclusions and provides insights regarding future work.

% \gls{ge} is a \gls{gp} variant where the genotype (a string/vector of integers) is translated into a phenotype (an executable function) through a grammar.
% %The grammars used in \gls{ge} are typically Context Free Grammars (CFG).
% %These grammars comprise four sets of symbols: 
% \gls{ge} allows practitioners to restrict the search space and incorporate expert knowledge through grammar design.
% This approach is relevant, but suffers from high redundancy and poor locality, damaging the efficiency of evolution.
% These shortcoming motivated researchers to create variants of \gls{ge} to mitigate these issues.

% In \gls{sge} the genotype is split into several strings, one for each non-terminal in the grammar.
% This representation improves locality and reduces redundancy.
% \gls{sge} uses a grammar-aware recombination operator which preserves the strings of each non-terminal.
% However, the grammar does not inform mutation.
% All genes are equally affected by mutation, regardless of grammatical context.

% In this work, we propose a new grammar-aware mutation operator for \gls{ge} and its variants, inspired by biology.
% Nature employs mechanisms to ensure that low impact mutations are more frequent than high impact ones.
% In grammar-based algorithms, we can reproduce this behavior using different mutation probabilities for each non-terminal.
% The effectiveness of this mutation is related to grammar design, where grammars with more rules enable finer tuning of mutation probabilities and improved performance.
% We find that this approach is statistically superior or similar in all tasks.
 
% TODO: PRESENT STRUCTURE
%The recombination operator used in \gls{sge} and its variations is grammar-aware, while the mutation operator is not. 

%A variant of \gls{sge}, Probabilistic Structured Grammatical Evolution (\gls{psge}), adds %probabilities to the grammar to further improve the efficiency of evolution.
%These probabilities are iteratively updated based on the best individuals and %bias the expansion of non-terminals.
