We use \gls{psge} \cite{Megane2022} in all experiments as it has equal or superior performance to \gls{sge} in the selected tasks.
We compare the algorithm using Standard Mutation (SM) and Facilitated Mutation (FM).
Additionally, we compare the Standard Grammar (SG) and the Function Grouped Grammar (FG) that follows the principles outlined in Section \ref{sec:mutation_levels} (complete grammar shown in Figure \ref{fig:extended_grammar}).

We evaluate the performance of our method in three popular symbolic regression \gls{gp} benchmarks, the Quartic polynomial, the Pagie polynomial, and Boston Housing \cite{White2012,McDermott2012}.
The Quartic polynomial is defined by the mathematical expression shown in Equation \ref{eq:quartic}. The function is sampled in the interval [-1, 1] with a step of 0.1.
\begin{equation}
    x[0]^{4} + x[0]^{3} + x[0]^{2} + x[0]
    \label{eq:quartic}
\end{equation}

The Pagie polynomial is known to be a more difficult symbolic regression benchmark (Equation \ref{eq:pagie}. The outputs are computed in the interval: $-5 \leq x[0], x[1] \leq 5.4$, with step size $0.4$.
In Pagie, both features fall in the same range of values. 

\begin{equation}
    \frac{1}{1 +  x[0]^{-4} } + \frac{1}{1 + x[1]^{-4}}
    \label{eq:pagie}
\end{equation}

\begin{figure}[h!]
    \begin{small}
    \begin{align*}
        {<}\text{start}{>}::= & {<}\text{expr\_var}{>}\\
        {<}\text{expr\_var}{>}::= &  {<}\text{expr}{>} | {<}\text{var}{>}\\ 
        {<}\text{expr}{>}::= &  {<}\text{expr\_op}{>} | \\
        & {<}\text{pre\_op}{>}{(}{<}\text{expr\_var}{>}{)}\\ 
        {<}\text{expr\_op}{>}::= & {<}\text{expr}{>}{<}\text{op}{>}{<}\text{expr}{>}\, |\\ 
        & {(}{<}\text{expr}{>}{<}\text{op}{>}{<}\text{expr}{>}{)}\, \\
        {<}\text{op}{>}::= & {+} | {-} | {*} | {/} \\ 
        {<}\text{pre\_op}{>}::= & {<}\text{trig\_op}{>}\\ 
        & | {<}\text{exp\_log\_op}{>}  | {inv} \\ 
        {<}\text{trig\_op}{>}::= & {sin} | {cos} \\ 
        {<}\text{exp\_log\_op}{>}::= & {exp} | {log} \\ 
        {<}\text{var}{>}::= & {1.0} | {x[n]}
    \end{align*}
    \end{small}
\caption{Function Grouped Grammar used in the experiments. Number $x[n]$ terminals change to match the problem: 1 for Quartic, 2 for Pagie, 12 for Boston Housing} 
\label{fig:extended_grammar}
\end{figure}

The third benchmark is Boston Housing \cite{harrison1978hedonic}. This is a predictive modeling problem, where one needs to build a model to predict the price of Boston houses based on 13 features.
% Results in test data are more relevant as they represent the solutions' ability to solve unknown problem instances.
There are 506 instances split into 90\% for training and 10\% for test. The 13 features that compose the dataset are heterogeneous regarding their intervals, with ranges varying from $0 \leq x[3] \leq 1$ to $0.32 \leq x[11] \leq 396.9$.
This diversity likely reduces the effectiveness of our approach as all features are grouped in the $<var>$ non-terminal, making it difficult for the algorithm to distinguish between different types of features with different mutation rates.
It is possible to adjust the grammar using expert knowledge, separating the variables into non-terminals based on orders of magnitude or function.
This type of grammar design, while possibly effective, is outside the scope of this work.
% Consequently, we assume that Boston Housing is a difficult task for our approach due to the properties of its features.




The fitness functions used to evaluate the individuals consider the minimization of the \gls{rrse} between the individual's solution and the target on a data set.


Table \ref{tab:parameters} summarizes the parameters used in the experiments.
All problems use the same population size, mutation and crossover rates, tournament size, max depth, and number of generations.
In preliminary experimentation, we trialed four parameters for the 
\gls{afm}'s Gaussian perturbation: $\sigma = [0.001, 0.0025, 0.005, 0.01]$
We found that facilitated mutations' $\sigma$ parameter benefited from tuning when moved to different tasks.
Consequently, each experiment uses the best $\sigma$ value for the corresponding problem.
We repeat all experiments 100 times to investigate statistically meaningful differences between the approaches.


\begin{table}[h!]
\centering
\caption{Parameters used in experiments for Quartic, Pagie, and Boston Housing.}
\label{tab:parameters}
% Please add the following required packages to your document preamble:
% \usepackage{graphicx}
\begin{tabular}{|c|ccc|}
\hline
\textbf{Parameters}         & \multicolumn{1}{c|}{Quartic} & \multicolumn{1}{c|}{Pagie}  & \begin{tabular}[c]{@{}c@{}}Boston \\ Housing\end{tabular} \\ \hline
Population Size       & \multicolumn{3}{c|}{1000} \\ \hline
Generations           & \multicolumn{3}{c|}{100}  \\ \hline
Elitism               & \multicolumn{3}{c|}{10\%}  \\ \hline
Mutation            & \multicolumn{3}{c|}{Gaussian N(0,0.5)}  \\ \hline
Mutation Probability  & \multicolumn{3}{c|}{10\%} \\ \hline
Adaptive Facilitated Mutation $\sigma$ & \multicolumn{2}{c|}{0.0025}  & \multicolumn{1}{c|}{0.001}                                                      \\ \hline
Crossover Probability & \multicolumn{3}{c|}{90\%} \\ \hline
Tournament Size       & \multicolumn{3}{c|}{3}    \\ \hline
Max Depth             & \multicolumn{3}{c|}{10}   \\ \hline
\end{tabular}%
\end{table}

