This section presents the results obtained for each problem in terms of the \gls{mbf} of 100 repetitions.
We statistically compare the different approaches using the Mann-Whitney test with Bonferroni correction with a significance level $\alpha = 0.05$.

\begin{figure}[h]
\centering
\includegraphics[width=0.67\textwidth]{images/plot_quartic.png}
\caption{Plot shows the mean best fitness of 100 runs for the Quartic polynomial.} \label{fig:results_quartic}
\end{figure}
Figure \ref{fig:results_quartic} shows the results for the Quartic polynomial. 
At the end of evolution, all approaches achieve a similar result in this problem, except for SM+SG, which performs significantly worse than the others (see Table \ref{tab:stats_quartic} for statistical analysis).
Quartic likely has a local optimum that evolution cannot escape without our proposed mechanism.
While SM+FG and AFM+SG statistically outperform the baseline, AFM+FG's advantages are only empirical.
We hypothesize that AFM+FG would join the other two methods with additional repetitions, creating two tiers of solution quality in Quartic.



\begin{table}[h]
\centering
\caption{P-value for Mann-Whitney Statistical Tests using Bonferroni correction with significance level $\alpha = 0.05$ for Quartic polynomial. Bold indicates that the method in the corresponding row is statistically superior.}
\label{tab:stats_quartic}
\begin{tabular}{l|llll}
\begin{tabular}[c]{@{}l@{}}\textbf{Quartic}\\ \textbf{polynomial}\end{tabular} & SM+SG                    & SM+FG                    & AFM+SG                    & AFM+FG                    \\ \hline
SM+SG                                                  & \cellcolor[HTML]{C0C0C0} & \cellcolor[HTML]{C0C0C0} & \cellcolor[HTML]{C0C0C0} & \cellcolor[HTML]{C0C0C0} \\
SM+FG                                                  &                \textbf{0.048}          & \cellcolor[HTML]{C0C0C0} & \cellcolor[HTML]{C0C0C0} & \cellcolor[HTML]{C0C0C0} \\
AFM+SG                                                  &                \textbf{0.004}          &                  0.356        & \cellcolor[HTML]{C0C0C0} & \cellcolor[HTML]{C0C0C0} \\
AFM+FG                                                  &               0.072       &                     0.760     &           0.332            & \cellcolor[HTML]{C0C0C0}
\end{tabular}
\end{table}


\begin{figure}[h!]
\centering
\includegraphics[width=0.67\textwidth]{images/plot_pagie.png}
\caption{Plot shows the mean best fitness of 100 runs for the Pagie polynomial.} \label{fig:results_pagie}
\end{figure}

Figure \ref{fig:results_pagie} shows the \gls{mbf} for the Pagie polynomial across 100 generations. Looking at the results, one can see that the advantages of the proposed approaches are clear for this problem, particularly the FG grammar.
From generation 25 until termination, both AFM solutions have a better \gls{mbf} than their SM counterparts.
While AFM is empirically superior to SM, FG amplifies the benefits of this approach.

The statistical analysis results (shown in Table \ref{tab:stats_pagie}) reveal that AFM+FG outperforms all SG approaches while SM+FG is similar.
These results suggest that our proposed grammar may be marginally superior, but considerable benefits come from combining it with the appropriate mutation.

\begin{table}[h!]
\centering
\caption{P-value for Mann-Whitney Statistical Tests using Bonferroni correction with significance level $\alpha = 0.05$ for Pagie polynomial. Bold indicates that the method in the corresponding row is statistically superior.}
\label{tab:stats_pagie}
\begin{tabular}{l|llll}
\begin{tabular}[c]{@{}l@{}}\textbf{Pagie}\\\textbf{polynomial}\end{tabular} & SM+SG                    & SM+FG                    & AFM+SG                    & AFM+FG                    \\ \hline
SM+SG                                                  & \cellcolor[HTML]{C0C0C0} & \cellcolor[HTML]{C0C0C0} & \cellcolor[HTML]{C0C0C0} & \cellcolor[HTML]{C0C0C0} \\
SM+FG                                                  &                 0.965         & \cellcolor[HTML]{C0C0C0} & \cellcolor[HTML]{C0C0C0} & \cellcolor[HTML]{C0C0C0} \\
AFM+SG                                                  &         0.426                 &                   0.400       & \cellcolor[HTML]{C0C0C0} & \cellcolor[HTML]{C0C0C0} \\
AFM+FG                                                  &             \textbf{ 0.002}            &            \textbf{ 0.001}             &           \textbf{ 0.001}              & \cellcolor[HTML]{C0C0C0}
\end{tabular}
\end{table}


% TODO: mudar este texto todo, já nao ha diferenças

\begin{figure}[h!]
\centering
    \begin{subfigure}{0.45\textwidth}
        \centering
        \includegraphics[height=4.2cm]{images/plot_bh_train2.png}
        \label{fig:results_bh_train}
        \caption{Training}
    \end{subfigure}
    \begin{subfigure}{0.45\textwidth}
        \centering
        \includegraphics[height=4.2cm]{images/plot_bh_testn2.png}
        \label{fig:results_bh_test}
        \caption{Test}
    \end{subfigure}
\caption{Plot shows the mean best fitness of 100 runs for the Boston Housing dataset.} \label{fig:results_bh}
\end{figure}

Finally, in Figure \ref{fig:results_bh} we show the \gls{mbf} for the Boston Housing Training and Test.
A brief perusal of the results indicates that AFM+FG achieves the best \gls{mbf} in Test.
When comparing the mutations, \gls{afm} generalizes better as it maintains a similar performance between train and test. 



Note that SM+SG appears to be worst at generalizing to the test data.
This is most evident when comparing SM+SG with both AFM approaches, as the differences between these methods are noticeably larger when moved to test data.
Despite the highlighted differences, statistical tests for Boston Housing Training and Test reveal no statistical differences (Tables \ref{tab:stats_bh_train} and \ref{tab:stats_bh_test}).
Given that FG cannot account for the feature variety of Boston Housing, it is remarkable that the approach still achieves competitive results, especially in the test data.
It is possible that a specifically designed FG that uses expert knowledge to group the features based on function could achieve even better and statistically significant results.

\begin{table}[h!]
\centering
\caption{P-value for Mann-Whitney Statistical Tests using Bonferroni correction with significance level $\alpha = 0.05$ for Boston Housing Training. Bold indicates that the method in the corresponding row is statistically superior.}
\label{tab:stats_bh_train}
\begin{tabular}{l|llll}
\begin{tabular}[c]{@{}l@{}}\textbf{Boston Housing}\\\textbf{Training}\end{tabular} & SM+SG                    & SM+FG                    & AFM+SG                    & AFM+FG                    \\ \hline
SM+SG                                                  & \cellcolor[HTML]{C0C0C0} & \cellcolor[HTML]{C0C0C0} & \cellcolor[HTML]{C0C0C0} & \cellcolor[HTML]{C0C0C0} \\
SM+FG                                                  &               0.530           & \cellcolor[HTML]{C0C0C0} & \cellcolor[HTML]{C0C0C0} & \cellcolor[HTML]{C0C0C0} \\
AFM+SG                                                  &                 0.263         &          0.763                & \cellcolor[HTML]{C0C0C0} & \cellcolor[HTML]{C0C0C0} \\
AFM+FG                                                  &                 0.531         &       0.361                   &          0.548             & \cellcolor[HTML]{C0C0C0}
\end{tabular}
\end{table}

\begin{table}[h!]
\centering
\caption{P-value for Mann-Whitney Statistical Tests using Bonferroni correction with significance level $\alpha = 0.05$ for Boston Housing Test. Bold indicates that the method in the corresponding row is statistically superior.}
\label{tab:stats_bh_test}
\begin{tabular}{l|llll}
\begin{tabular}[c]{@{}l@{}}\textbf{Boston Housing}\\\textbf{Test}\end{tabular} & SM+SG                    & SM+FG                    & AFM+SG                    & AFM+FG                    \\ \hline
SM+SG                                                  & \cellcolor[HTML]{C0C0C0} & \cellcolor[HTML]{C0C0C0} & \cellcolor[HTML]{C0C0C0} & \cellcolor[HTML]{C0C0C0} \\
SM+FG                                                  &           0.929               & \cellcolor[HTML]{C0C0C0} & \cellcolor[HTML]{C0C0C0} & \cellcolor[HTML]{C0C0C0} \\
AFM+SG                                                  &              0.377            &        0.333                 & \cellcolor[HTML]{C0C0C0} & \cellcolor[HTML]{C0C0C0} \\
AFM+FG                                                  &             0.364             & 0.403                   &        0.963                  & \cellcolor[HTML]{C0C0C0}
\end{tabular}
\end{table}
