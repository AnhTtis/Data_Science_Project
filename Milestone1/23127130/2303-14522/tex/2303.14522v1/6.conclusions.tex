In this paper, we propose \gls{afm}, a mutation method that leverages grammar-based \gls{gp}'s properties to replicate natural evolutionary phenomena.
This approach divides the single mutation probability commonly found in such approaches into a mutation array, where each grammar non-terminal has a corresponding mutation rate. Each individual has a mutation array that co-evolves with the genetic code. A randomly sampled value from a Gaussian distribution adjusts the mutation rates of all individuals in each generation.

We also propose a grammar-design approach, Function Grouped Grammars, to enhance the effectiveness of the mutation proposed.
Function Grouped Grammars organize non-terminals based on functional similarity rather than the semantic similarity common in the field.
We compare our proposals with a baseline (standard mutation and standard grammar) and find that, when combined, our approaches are statistically superior or similar to the baseline in three relevant \gls{gp} benchmarks.

This approach still requires parameter tuning, one of the problems tackled by the literature by proposing adaptive mutations. However, few of these approaches consider different values for different symbols \cite{Coelho2016}. 
This work shows that an adaptive mutation rate can be beneficial for search and grammar design amplifies these benefits.


\subsection{Future Work}
The results of our experiments are promising, but additional tests are essential to validate our approach further.
The benchmarks addressed are relevant in \gls{gp}, but a more extensive (and varied) set of benchmarks could bring meaningful insights into the general applicability of the FG+FM.

Regarding \gls{afm}, we use a fixed starting mutation rate in all experiments.
Our method leverages adaptability as a tool for improved evolution, but we did not investigate the potential of \gls{afm} as a replacement for mutation rate tuning.
Further development of \gls{afm} may also lead to an implementation that does not rely on Gaussian distributions and $sigma$ tuning for perturbations. 
In the future, we want to explore alternatives where \gls{afm} is a competitive, parameter-less alternative to standard mutation.
Another line of work is to experiment with different inheritance mechanisms during crossover. 
In this work, the offspring inherited the array of probabilities from the most fitted parent.
It would be interesting to explore the random selection of the parent that passes the array or the application of the existing SGE crossover to the mutation arrays of the parents.
Such approaches would better preserve the advantages of co-evolution, possibly improving results. 

Function Grouped Grammars can also be further investigated.
In this work, we still apply this idea conservatively.
Considering the FG used in experiments, it is still possible to separate constants from variables and commutative operators from non-commutative operators.
Barring small details, we use the same grammar for all tasks.
Applying the same approach to problems requiring more complex grammar would be interesting, as such problems yield more opportunities for function grouping.

Finally, the applicability of these ideas to different, compatible grammar-based \gls{gp} must also be investigated.
While this works focuses on \gls{psge}, the same approach is easily applicable in \gls{sge} \cite{Loureno2018}, \gls{copsge} \cite{Megane2022gecco}, and even more different systems like $\pi$GE \cite{ONeill2004}.
Any grammar-based approach where genes are tied to a non-terminal may benefit from FG+FM.





% limitations: the mutation operator only works for grammar-aware representations (SGE and descendents)
% applicable in pi ge, for example

% future work

% different grammars

%