In biology, organisms have evolved to canalize the rate and effect of mutations on the phenotype. 
Gerhart et al. \cite{gerhart2007} propose that organisms adapt to new environments through regulatory changes that enable or disable pre-existing conserved components. This variation increases the probability of viable genetic mutation since core components remain unaffected by these regulatory changes. 
In sum, the modularity, adaptability, and compartmentation of genetic material in organisms allow facilitated variation through regulatory change.

Facilitated Mutation \cite{stefano2023} (FM) is a biologically inspired mutation mechanism that aims to replicate the benefits of facilitated variation.
This mechanism leverages the grammar's inherent compartmentation to regulate the mutation's frequency and destructiveness. With this mutation, each non-terminal has a different mutation probability, rather than the single mutation probability traditionally used by grammar-based \gls{ea}. 


In this work, we propose \gls{afm}, an extension of FM using an adaptive mutation array that removes the need to set a mutation probability for each non-terminal manually.
Each individual carries a \textit{mutation array} containing these mutation probabilities, illustrated in Figure~\ref{fig:mutation_array}.
For each individual in the initial population, the mutation array is initialized using a specified \textit{starting mutation probability}.
\begin{figure}[h]
\centering
\includegraphics[width=0.7\textwidth]{images/facilmut2.png}
\caption{In \gls{sge}, all genes are mutated based on a single probability (pictured left). Adaptive Facilitated Mutation uses a mutation array for each individual containing a probability for each non-terminal (pictured right).} \label{fig:mutation_array}
\end{figure}

In each generation, all individual's probabilities are adjusted using a random value sampled from a Gaussian distribution with $N(0,\sigma)$, where $\sigma$ is a configurable parameter. 
Figure \ref{fig:mutation_evolution} illustrates the evolution of the mutation array.

\begin{figure}[h]
\centering
\includegraphics[width=0.7\textwidth]{images/facilmutop.png}
\caption{Example of the first perturbation to the mutation probability array of an individual. This array is subsequently updated using a value sampled from a Gaussian distribution. This distribution is centered at 0 using a configurable standard deviation $\sigma$.} \label{fig:mutation_evolution}
\end{figure}

\gls{afm} only complements mutation operators by refining the frequency and impact of mutation.
Once the mechanism determines which non-terminals to mutate, other operators should be used to alter the genotype within the defined scope.
During crossover, the offspring individual inherits the mutation array from its fittest parent.
More sophisticated inheritance mechanisms may improve this approach further, but we opted for a simple strategy to validate the approach.
In this work, we use \gls{afm} for \gls{sge} and its variants, but this method is compatible with any grammar-based \gls{gp} algorithms where it is possible to tie each codon to a corresponding non-terminal.

\subsection{Grammar Design For Adaptive Facilitated Mutation}
Since \gls{afm} leverages grammar structure, a purposefully designed grammar may enhance the method's performance.
We hypothesize that \gls{afm} is more effective in grammars with multiple non-terminals containing related symbols.
Additionally, a larger number of non-terminals may improve performance by enabling finer turning of mutation probabilities through a more detailed mutation array.
Non-terminals commonly group symbols based on semantic similarity.
For example, a grammar may use a non-terminal for all operators with a single expansion combining operators related to trigonometry (i.e., $sin$, $cos$) and the power function (i.e., $square$, $sqrt$).
These same symbols can be grouped into several non-terminals based on function rather than semantics.
Following this reasoning, trigonometric operations would be grouped in a specific non-terminal and the power function in a separate one.
In Figure \ref{fig:standard_vs_extended_grammar}, we illustrate how a grammar can be extended into a Function Grouped grammar.


\begin{figure}
\begin{subfigure}{0.5\textwidth}
    \begin{align*}
        {<}\text{start}{>}::= & {<}\text{expr}{>}\\
        {<}\text{expr}{>}::= & {<}\text{expr}{>}{<}\text{op}{>}{<}\text{expr}{>}\, |\\ 
        & {<}\text{pre\_op}{>}{(}{<}\text{expr}{>}{)}\, | \\
        & {<}\text{var}{>} \\
        {<}\text{op}{>}::= & {+} | {-} | {*} | {/} \\ 
        {<}\text{pre\_op}{>}::= & {sin} | {cos} | {sqrt} | {square}  \\ 
        {<}\text{var}{>}::= & {1.0} | {x[n]}
    \end{align*}
    \caption{Initial Grammar.}
    \label{fig:standard_example}
\end{subfigure}
\begin{subfigure}{0.5\textwidth}
\centering
    \begin{small}
    \begin{align*}
        {<}\text{start}{>}::= & {<}\text{expr\_var}{>}\\
        {<}\text{expr\_var}{>}::= &  {<}\text{expr}{>} | {<}\text{var}{>}\\ 
        {<}\text{expr}{>}::= &  {<}\text{expr\_var}{>}{<}\text{op}{>}{<}\text{expr\_var}{>}\, | \\
        & {<}\text{pre\_op}{>}{(}{<}\text{expr\_var}{>}{)}\\ 
        {<}\text{op}{>}::= & {+} | {-} | {*} | {/} \\ 
        {<}\text{pre\_op}{>}::= & {<}\text{trig\_op}{>} | {<}\text{pow\_op}{>} \\ 
        {<}\text{trig\_op}{>}::= & {sin} | {cos} \\ 
        {<}\text{pow\_op}{>}::= & {sqrt} | {square} \\ 
        {<}\text{var}{>}::= & {1.0} | {x[n]}
    \end{align*}
    \end{small}
    \caption{Function Grouped Grammar.}
    \label{fig:extended_example}
\end{subfigure}
\caption{A grammar of semantically grouped non-terminals (\ref{fig:standard_example}) can be re-constructed based on functional groups (\ref{fig:extended_example}). This procedure extends the grammar and possibly improves performance when using Adaptive Facilitated Mutation.} \label{fig:standard_vs_extended_grammar}
\end{figure}

%TODO: Not sure where this fits but it is worth saying.

%Regulatory changes also exist in \gls{sge}.
%When variation operators deactivate parts of the genotype the genes are preserved and may be re-activated at a later date.
%However the genotype in \gls{sge} does not have the properties necessary to yield facilitated variation from regulatory changes.