% ****** Start of file apssamp.tex ******
%
%   This file is part of the APS files in the REVTeX 4.2 distribution.
%   Version 4.2a of REVTeX, December 2014
%
%   Copyright (c) 2014 The American Physical Society.
%
%   See the REVTeX 4 README file for restrictions and more information.
%
% TeX'ing this file requires that you have AMS-LaTeX 2.0 installed
% as well as the rest of the prerequisites for REVTeX 4.2
%
% See the REVTeX 4 README file
% It also requires running BibTeX. The commands are as follows:
%
%  1)  latex apssamp.tex
%  2)  bibtex apssamp
%  3)  latex apssamp.tex
%  4)  latex apssamp.tex
%
\documentclass[
aps,
prfluids,
reprint,
a4,
%superscriptaddress,
%groupedaddress,
%unsortedaddress,
%runinaddress,
%frontmatterverbose, 
%preprint,
%preprintnumbers,
%nofootinbib,
%nobibnotes,
%bibnotes,
%amsmath,
%amssymb,
onecolumn,
%pra,
%rmp,
%prstab,
%prstper,
%floatfix,
]{revtex4-2}

\usepackage{import}
% PA: always, first load graphix, then epstopdf
\usepackage{graphicx}% Include figure files
\usepackage{amsmath,amssymb,epstopdf}
\usepackage{dcolumn}% Align table columns on decimal point
\usepackage{bm}% bold math
\usepackage{hyperref}% add hypertext capabilities
\usepackage{longtable}
%\usepackage{caption}
%\usepackage{subcaption}
\usepackage{booktabs}
\usepackage{multirow}
\usepackage{siunitx}
%\usepackage[mathlines]{lineno}% Enable numbering of text and display math
%\linenumbers\relax % Commence numbering lines

%\usepackage[showframe,%Uncomment any one of the following lines to test 
%%scale=0.7, marginratio={1:1, 2:3}, ignoreall,% default settings
%%text={7in,10in},centering,
%%margin=1.5in,
%%total={6.5in,8.75in}, top=1.2in, left=0.9in, includefoot,
%%height=10in,a5paper,hmargin={3cm,0.8in},
%]{geometry}
%-------------------------------------------
\newcommand{\pd}[2]{\frac{\partial{#1}}{\partial{#2}}}
%
\newcommand{\pdd}[2]{\frac{\partial^2{#1}}{\partial{#2}^2}}

\newcommand{\bsym}[1]{\boldsymbol{#1}}
%
\newcommand{\bu}{\bsym{u}}
\newcommand{\bx}{\bsym{x}}

\newcommand{\bU}{\bsym{U}}
\newcommand{\bex}{\bsym{e}_x}
\newcommand{\Ubulk}{U_{\text{bulk}}}
\newcommand{\utot}{\bsym{u_{\text{tot}}}}
\newcommand{\ptot}{p_{\text{tot}}}
\newcommand{\DT}{\Delta T}
\newcommand{\del}{\bsym{\nabla}}
\newcommand{\grad}{\nabla}
\newcommand{\deldot}{\bsym{\nabla \cdot} }
\newcommand{\delsq}{\nabla^{2} }
\newcommand{\Rei}{Re_{i}}
\newcommand{\Gr}{G}
\newcommand{\Fr}{Fr}

\newcommand{\Ta}{Ta}
\newcommand{\kc}{k_c}

%
\newcommand{\udotgrad}[1]{u\pd {#1}{r} + S \frac{v}{r} \pd{#1}{\phi} + w \pd{#1}{z}  }
\newcommand{\NA}{\text{stable}}
\newcommand{\ub}{u_b}
\newcommand{\vb}{v_b}
\newcommand{\wb}{w_b}
\newcommand{\pb}{p_b}
\newcommand{\Tb}{T_b}
\newcommand{\Lapr} {\delsq u - S\frac{2}{r^{2}} \pd{v}{\phi} - \frac{u}{r^{2}}}
\newcommand{\Lapphi} {S \delsq v - \frac{2}{r^{2}}\pd{u}{\phi} - S\frac{v}{r^{2}}}

\newcommand{\id}{id}

\newcommand{\aghoredit}[1]{\textcolor{cyan}{#1}}
\newcommand{\atifedit}[1]{\textcolor{red}{#1}}

%-------------------------------------------
\begin{document}

\preprint{APS/123-QED}

\title{Effect of outer cylinder rotation on the radially heated Taylor-Couette flow}% Force line breaks with \\
% \thanks{A footnote to the article title}%

\author{Pratik Aghor}
 \email{pratikprashant.aghor@unh.edu}
 \affiliation{Integrated Applied Mathematics Program, University of New Hampshire (US)}%Lines break automatically or can be forced with \\
\author{Mohammad Atif}%
 \email{fmohammad@bnl.gov}
\affiliation{%
 Brookhaven National Laboratory, Upton, New York (US)
}%

% \collaboration{MUSO Collaboration}%\noaffiliation


% \collaboration{CLEO Collaboration}%\noaffiliation

\date{\today}% It is always \today, today,
             %  but any date may be explicitly specified

\begin{abstract}
A radially heated Taylor--Couette setup is considered where an incompressible, Boussinesq fluid is sheared in the annular region between two concentric, independently rotating cylinders maintained at different temperatures. 
The radius ratios chosen for the current study are $0.6$ and $0.9$ corresponding to wide--gap and thin--gap cases. 
% Buoyancy is incorporated via Boussinesq's approximation and 
% the effect of outer cylinder rotation is investigated without centrifugal buoyancy. 
A linear stability analysis assuming infinite axial extent over a range of Grashof numbers with Taylor number as the control parameter is performed. 
The fastest growing modes at the onset of instability are found to be oscillatory for the parameters considered. 
Depending on the ratio of angular velocities and Grashof numbers, both axisymmetric and non-axisymmetric modes are found to be the fastest growing modes. 
Compared to the case where the outer cylinder is held stationary, the rotation of the outer cylinder is shown to have a general stabilizing effect on the linear stability threshold, except for a few cases. 
Additionally, at a Grashof number of 1000, unstable modes are located even in the Rayleigh-stable regime for both wide--gap and thin--gap cases. 

% \begin{description}
% \item[Usage]
% Secondary publications and information retrieval purposes.
% \item[Structure]
% You may use the \texttt{description} environment to structure your abstract;
% use the optional argument of the \verb+\item+ command to give the category of each item. 
% \end{description}
\end{abstract}

%\keywords{Suggested keywords}%Use showkeys class option if keyword
                              %display desired
\maketitle

%\tableofcontents

%----------------------------------%
\import{sections/}{section-1.tex}
\import{sections/}{section-2.tex}
\import{sections/}{section-3.tex}
\import{sections/}{section-4.tex}
%----------------------------------
%----------------------------------
% \clearpage
% \appendix
% \import{./sections/}{supplementary.tex}

%----------------------------------
\clearpage
\bibliography{main}% Produces the bibliography via BibTeX.
%----------------------------------
\end{document}
%
% ****** End of file apssamp.tex ******
