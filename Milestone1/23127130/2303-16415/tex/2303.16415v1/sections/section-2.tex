%----------------------------------%
\section{Problem Formulation and Numerical Method}
\label{sec:2}
%----------------------------------%

The governing equations for the thermally active viscous Newtonian fluid that control the dynamics of the system are the well-known Navier-Stokes equations (continuity and momentum equations), and the energy equation.
Following \citet{ali1990stability} the governing equations are rendered dimensionless using gap width $d = R_{o} - R_{i}$ as the characteristic length, velocity of the inner cylinder $R_{i}\Omega_{i}$ as the characteristic azimuthal velocity, $U_0 = g\beta d^{2} \DT /\nu$ as the characteristic radial and vertical velocity, $d^{2}/\nu$ as the time scale, $\rho U_0^{2}$ as the characteristic pressure, and the temperature contrast $|\DT| = |T_{i} - T_{o}|$ as the temperature scale.

The dimensionless governing equations thus obtained are
\begin{align}
\frac{1}{r} \pd{(ru)}{r} + \frac{1}{r} \pd{v}{\phi} + \pd{w}{z} &= 0, \label{eq:continuity}\\
 \pd{u}{t} + \Gr \left[\udotgrad{u} - S^{2}\frac{v^{2}}{r} \right] &= -\Gr\pd{p}{r} + \left[\left(\delsq - \frac{1}{r^{2}} \right) u  - S\frac{2}{r^{2}}\pd{v}{\phi}\right], \label{eq:r-mom}  \\
 S\left[ \pd{v}{t} + \Gr\left(\udotgrad{v} + \frac{uv}{r} \right) \right]  &= -\Gr\frac{1}{r}\pd{p}{\phi} + \left(\delsq - \frac{1}{r^{2}} \right) v  + \frac{2}{r^{2}}\pd{u}{\phi}, \label{eq:phi-mom}  \\
\pd{w}{t} + \Gr \left[\udotgrad{w}  \right] &= -\Gr \pd{p}{z} + \delsq w  + T, \label{eq:z-mom} \\
\pd{T}{t} + \Gr \left[\udotgrad{T}  \right] &= \frac{1}{Pr} \delsq T, \label{eq:energy}
\end{align}
where $u, v, w$ are the dimensionless velocities in the radial ($r$), azimuthal ($\phi$), and axial ($z$) directions respectively. Here, $p$ and $T$ are dimensionless pressure and temperature respectively, and $\delsq$ is the Laplacian operator defined as  
\begin{eqnarray}\label{eq:delsq}
\delsq \equiv \pdd {}{r} + \frac{1}{r}\pd{}{r} + \frac{1}{r^{2}}\pdd{}{\phi} + \pdd{}{z}.
\end{eqnarray}
The nondimensionalization gives rise to the dimensionless parameters Prandtl ($Pr$), Taylor ($\Ta$) and  Grashof ($\Gr$) numbers given by
\begin{align}
Pr = \frac{\nu}{\kappa}, \quad \Ta = \frac{2 \eta^{2} \Omega_{i}^{2} d^{4}}{\nu^{2}(1-\eta^{2})}, \quad \Gr = \frac{g \beta \Delta T d^3}{\nu^2}.
\end{align}
The Taylor number $\Ta$ is proportional to the rotation rate of the inner cylinder and can also be related to the Reynolds number based on inner cylinder rotation as
\begin{eqnarray}
\Ta = \frac{2 (1-\eta)}{(1+\eta)} Re_{i}^{2},
\end{eqnarray}
where $Re_{i} = \Omega_{i} R_{i} d /\nu$ is the inner Reynolds number. 
The swirl parameter $S = \Omega_{i}R_{i}/U_0$ appears in the governing equations due to the difference of scales between azimuthal ($v$) and meridional velocities ($u$ and $w$) and is not an independent parameter. It can be expressed in terms of other dimensionless parameters as
\begin{eqnarray}\label{eq:S}
S = \frac{\left(Ta(1+\eta)/2(1-\eta) \right)^{1/2}}{\Gr} = \frac{Re_{i}}{\Gr}.
\end{eqnarray}
The classical isothermal TCF with all velocities scaled by $\Omega_i R_i$ is recovered by substituting $T = 0$ and $S = 1$, replacing $\Gr$ with $Re_{i}$.
Infinite cylinders are assumed and Dirichlet boundary conditions in the radial direction are imposed, i.e.,
\begin{eqnarray}\label{eq:bc}
 &&(u, v, w, T) = (0, 1, 0, 1) \textrm{ at } r = r_{i} = \frac{\eta}{1-\eta}, \\
 &&(u, v, w, T) = \left(0, \frac{\mu}{\eta}, 0, 0\right) \textrm{ at } r = r_{o} = \frac{1}{1-\eta}, 
 \end{eqnarray}
 with $r_{i} = R_{i}/d$, $r_{o} = R_{o}/d$ being the dimensionless inner and outer radii respectively. In the next section, we formulate the base state and linearize the system in order to perform a linear stability analysis.

%----------------------------------%
\subsection{Base flow and linearization}
%----------------------------------%

The base state can be obtained analytically for this problem assuming radial dependence of the steady basic fields. For the basic azimuthal flow ($\vb$), the well-known general circular Couette flow profile in the azimuthal direction valid for nonzero $\mu$ is obtained, see for example, \citet{chandrasekhar1961hydrodynamic}. In the case of $\mu = 0$ corresponding to the stationary outer cylinder, the basic azimuthal velocity profile reduces to the one considered by \citet{ali1990stability}. In the classical isothermal Taylor-Couette setup, there is no axial flow in the base state. However, in the presence of a radial temperature gradient an axial flow is induced. Since the axial flow $\wb$ is only dependent on the basic temperature profile, outer cylinder rotation ($\mu \neq 0$) does not alter the basic axial flow and it is the same as reported in Ali and Weidman \cite{ali1990stability}. Thus, the base flow is
\begin{eqnarray}
&& \ub = 0, \\ 
&& \vb = Ar + B/r, \\
&& \Tb = \frac{\ln{\left[(1-\eta) r\right]}}{\ln{\eta}}, \\
&& \wb = \frac{C}{D} \left[ (1-\eta)^{2} r^{2} - 1 + (1-\eta^{2}) T_{b} \right]- \frac{1}{4} \left[(1-\eta)^{2}r^{2} - \eta^{2}\right]  T_{b},    
\end{eqnarray}
with
\begin{eqnarray*}
 A = \frac{\mu - \eta^{2}}{\eta(1+\eta)}, \quad
 B = \frac{\eta(1-\mu)}{(1-\eta)(1-\eta^{2})}, \quad 
 C = (1-\eta^{2})(1-3\eta^{2}) - 4 \eta^{4}\ln{\eta}, \quad
 D = 16\left[(1-\eta^{2})^{2} + (1-\eta^{4})\ln{\eta} \right]. 
\end{eqnarray*}
The above parameters represent the base state of the system used in the subsequent analysis.

%----------------------------------%
% \subsection{Linearized system of equations}
%----------------------------------%

We perturb the primitive variables as
\begin{eqnarray} \label{eq:perturb}
    \left[u, v, w, T, p \right] = [\ub, \vb, \wb, \Tb, \pb] (r) + \left[ u', v', w', T', p' \right],
\end{eqnarray}
and substitute into the governing equations \eqref{eq:continuity} -- \eqref{eq:energy}. Thereafter, imposing the boundary conditions and neglecting higher-order terms in the perturbations we obtain the linearized system of governing equations as (the primes are dropped for brevity)
%----------------------------------%
\begin{align}
 \frac{1}{r} \pd{(ru)}{r} + \frac{S}{r} \pd{v}{\phi} + \pd{w}{z} & = 0, \label{eq:lincontinuity}\\
 \pd{u}{t} + \Gr \left[S \frac{\vb}{r} \pd{u}{\phi} + \wb \pd{u}{z} - S^{2} \frac{2\vb v}{r} \right] &= -\Gr\pd{p}{r} + \Lapr, \label{eq:lin-r-mom}\\
 S\left[ \pd{v}{t} + \Gr\left( u \frac{d \vb}{dr}+ S \frac{\vb}{r} \pd{v}{\phi} + \wb \pd{v}{z} + \frac{u \vb}{r} \right) \right] &= -\Gr\frac{1}{r}\pd{p}{\phi} + \Lapphi, \label{eq:lin-phi-mom} \\
 \pd{w}{t} + \Gr \left[u \frac{d\wb}{dr} + S \frac{\vb}{r} \pd{w}{\phi} + \wb \pd{w}{z}  \right] &= -\Gr \pd{p}{z} + \delsq w  + T, \label{eq:lin-z-mom} \\
 \pd{T}{t} + \Gr \left[u\frac{d\Tb}{dr} + S \frac{\vb}{r} \pd{T}{\phi} + \wb \pd{T}{z} \right] &= \frac{1}{Pr} \delsq T,\label{eq:lin-energy}
\end{align}
and the boundary conditions become
\begin{eqnarray}
 &(u, v, w, T) = (0, 0, 0, 0) \textrm{ at } r = r_{i}, r_{o}.
 % &(u, v, w, T) = (0, 0, 0, 0) \textrm{ at } r = r_{o} = \dfrac{1}{1-\eta}.  
 \label{eq:linbc}
 \end{eqnarray}
%----------------------------------%
Substituting the modal perturbation ansatz 
\begin{eqnarray}\label{eq:modal_ansatz}
  [u, v, w, T, p] = [\hat{u}, \hat{v}, \hat{w}, \hat{T}, \hat{p}]  (r) \exp{\left[i(k z + m \phi) + \sigma t\right]} + c.c.,
\end{eqnarray}
in Eqs. \eqref{eq:lincontinuity}-\eqref{eq:linbc}, a generalized eigenvalue problem of the form $\bsym{A} \bx = \sigma \bsym{B} \bx$ in $r$ is obtained, with $\sigma$ as the eigenvalue and $\bx = [\hat{u}, \hat{v}, \hat{w}, \hat{T}, \hat{p}]  (r)$. Here `c.c.' stands for `complex conjugate'. Since the domain in the azimuthal direction has a natural periodicity of $2\pi$, the azimuthal wavenumber $m$ only takes integer values. On the other hand, since the cylinders are assumed to be infinite in the axial direction, the axial perturbation wavenumber $k$ is a continuous parameter and can take real values. The eigenvalue $\sigma$ is in general a complex number of the form $\sigma = \sigma_{r} + i \sigma_{i}$, with $\sigma_{r}, \sigma_{i} \in \mathcal{R}$. The real part of the eigenvalue $\sigma_{r}$ represents the growth rate and the imaginary part $\sigma_{i}$ represents the frequency of the evolution of perturbations according to Eq. \eqref{eq:modal_ansatz}. 
The linearly stable perturbations are characterized by $\sigma_r < 0$, whereas linearly unstable perturbations have $\sigma_r > 0$. The locus of points satisfying $\sigma_r = 0$ marks the neutral stability boundary which denotes the transition between stable and unstable regimes, see for example \citet{drazin2004hydrodynamic}. 
As mentioned in \citet{ali1990stability}, critical perturbation modes with $\sigma_{r} = 0$ can be completely described by the triplet $(m, k, \sigma_{ic})$. The non-dimensional axial propagation speed of the phase lines $C$, wavelength $\lambda$ normal to lines of constant phase, and inclination of the phase lines $\psi$ with respect to the horizontal are given by
\begin{eqnarray}\label{eq:perturbation_shape}
    C = -\frac{\sigma_{i}}{k}, \quad \lambda = \frac{2\pi}{\left( m^{2}/r^{2} + k^{2} \right)^{1/2}}, \quad \psi = - \tan^{-1} \left( {\frac{m}{rk}} \right).
\end{eqnarray}


As the linearized equations \eqref{eq:lin-r-mom} -- \eqref{eq:lin-energy} each have a second derivative term with respect to $r$, the resulting eigenvalue problem is $8$-dimensional and is closed by $8$ boundary conditions given in Eqs. \eqref{eq:linhatbc}. 

In this study, we solve the resulting eigenvalue problem using Dedalus's (see \citet{burns2020dedalus}) excellent package `eigentools'. Eigentools is equipped with automatic rejection of spurious modes by calculating the drift ratio as given in \citet{boyd2001chebyshev}. 
This is performed by comparing the eigenvalues at a specified resolution and a higher resolution of 1.5 times the original resolution, see \citet{oishi2021eigentools} for details. We use Chebyshev polynomials to discretize in the radial direction. 
Substituting the modal ansatz Eq. \eqref{eq:modal_ansatz} into Eq. \eqref{eq:linbc} appropriate homogeneous Dirichlet boundary conditions are obtained as
\begin{eqnarray} \label{eq:linhatbc}
(\hat{u}, \hat{v}, \hat{w}, \hat{T})  = (0, 0, 0, 0) \textrm{ at } r = r_{i}, r_{o}.   
\end{eqnarray}
%----------------------------------%
\subsection{Symmetries}
%----------------------------------%
We briefly discuss the symmetries of the linearized problem, their consequences and justify our choice of numerical parameters in this section. 
Two symmetries of the linearized perturbation equations for the stationary outer cylinder ($\mu = 0$) were identified previously in by \citet{ali1990stability}. 
We generalize these symmetries to include any nonzero $\mu$ as 
\begin{align}
  S_1 = S^{c/r} (\Omega): [\hat{u}, \hat{v}, \hat{w}, \hat{T}, \hat{p}; Ta, \Gr, k, \sigma, \mu, m] \rightarrow [\hat{u}, -\hat{v}, \hat{w}, \hat{T}, \hat{p}; Ta, \Gr, k, \sigma, \mu, -m], \label{eq:symms1} \\
  S_2 = S^{c/r} (\DT): [\hat{u}, \hat{v}, \hat{w}, \hat{T}, \hat{p}; Ta, \Gr, k, \sigma, \mu, m] \rightarrow [-\hat{u}^{*}, \hat{v}^{*}, \hat{w}^{*}, \hat{T}^{*}, \hat{p}^{*}; Ta, -\Gr, k, \sigma^{*}, \mu, -m]. \label{eq:symms2}
\end{align}

The symmetry $S^{c/r} (\Omega)$ corresponds to a situation where the inner cylinder is rotated in the opposite direction, i.e., $\Omega_i \rightarrow -\Omega_i$.
For the general case considered in our study with $\mu$ allowed to be nonzero, if we consider $\Omega_o \rightarrow -\Omega_o$, keeping $\mu$ invariant, the modified version of $S^{c/r} (\Omega)$ given in Eq. \eqref{eq:symms1} can be shown to hold. 
It is seen that only $\vb \rightarrow -\vb$ is sufficient to quantify the effect of this transformation on the base state whereas boundary conditions for the perturbation fields remain the same as Eq. \eqref{eq:linbc}. 

The symmetry $S^{c/r} (\DT)$ represents the scenario with temperature gradient reversed, i.e., $\Gr \rightarrow -\Gr$. 
We first note that $\Gr \rightarrow -\Gr$ does not alter the basic temperature profile $\Tb$ as it is nondimensionalized using the temperature contrast $|\DT| = |T_{i} - T_{o}|$. 
As a consequence, $\wb$ (which is completely determined by $\Tb$) is also invariant under $\Gr \rightarrow -\Gr$. 
Substituting $S^{c/r} (\DT) [\hat{u}, \hat{v}, \hat{w}, \hat{T}, \hat{p}; Ta, \Gr, k, \sigma, \mu, m]$ in Eq. \eqref{eq:modal_ansatz} and Eqs. \eqref{eq:lincontinuity}-\eqref{eq:lin-energy}, we obtain the complex conjugated version of the original system. 
This is independent of whether $\mu$ is zero or otherwise. 
Hence the $S^{c/r} (\DT)$ symmetry can be extended for nonzero $\mu$ as well, implying that for each $\Gr > 0$ there exists an equivalent case of $\Gr < 0$ with the inner wall being cooler than the outer wall.

Using a combination of these two symmetries, a third symmetry could also be identified, where $S_3 = S_1 \cdot S_2$, given by
\begin{eqnarray}\label{eq:S3}
    && S_3: [\hat{u}, \hat{v}, \hat{w}, \hat{T}, \hat{p}; Ta, \Gr, k, \sigma, \mu, m] \rightarrow [-\hat{u}^{*}, -\hat{v}^{*}, \hat{w}^{*}, \hat{T}^{*}, \hat{p}^{*}; Ta, -\Gr, k, \sigma^{*}, \mu, m].
\end{eqnarray}
The effect of these symmetries can be explained as follows. For each solution of the linearized equations with $(\Omega_i, \DT, \mu)$,, there are 3 equivalent solutions for the cases $(-\Omega_{i}, \DT, \mu)$, $(\Omega_{i}, -\DT, \mu)$ and $(-\Omega_{i}, -\DT, \mu)$. If the first case corresponds to spirals having phase speed and inclination with respect to the horizontal given by $(C, \psi)$, then the $S_1, S_2, S_3$ symmetries imply that there also exist solutions with $(C, -\psi)$, $(-C, -\psi)$ and $(-C, \psi)$. We have therefore generalized the symmetries identified by \citet{ali1990stability} to include nonzero outer-cylinder rotation. 

As noted by \citet{yoshikawa2013instability}, \citet{ kang2015thermal, kang2017radial}, the inclusion of the centrifugal buoyancy breaks these symmetries even in the case where the outer cylinder is held stationary and the effects of inward and outward heating are different. However, since we neglect centrifugal buoyancy, we only consider $\Gr > 0$ in our analysis. Other equivalent cases can be constructed for $\Gr < 0$ using symmetry arguments. 
%------------------------------------------------
%----------------------------------%
\subsection{Validation} \label{sec:validation}
%----------------------------------%
As mentioned earlier, we use eigentools package from Dedalus to solve the resulting eigenvalue problem. 
% We plan to make our routines open on GitHub after the manuscript review process. 
To verify our linear stability routine, We compare neutral stability curves obtained from our code to those reported in \citet{ali1990stability}. 
A sample calculation for $m = -2$ mode is shown in Fig. \ref{fig:compare} at $Pr=15, \Gr=300$ and $\eta=0.6$. It can be seen that the two curves match well. 

%--------------------------------
\begin{figure*}
 \begin{center}
  \includegraphics[scale=0.5]{figs/validation/aw_fig_4a_validation.pdf}
\end{center}
\caption{A sample comparison of the present code with published literature at $Pr = 15, \eta = 0.6, \Gr = 300, m = -2$ in our notation. Dashed green lines with vertical markers represent neutral stability data obtained from the current methodology and black circles represent data extracted from \citet{ali1990stability}.}
 \label{fig:compare}
\end{figure*}
%-------------------------------- 
