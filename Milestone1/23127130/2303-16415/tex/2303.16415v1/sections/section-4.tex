%---------------------------------------
\section{Discussion}
\label{sec:4}
%---------------------------------------

%---------------------------------------
In this section, we discuss observations of our analysis focusing on the fastest growing modes, the effect of outer cylinder rotation, and the impact of imposed temperature gradient and increased radius ratio on the linear stability thresholds. 

We begin with a qualitative description of the fastest growing modes at the onset of instability.
In all cases reported in this article, neutral modes at the onset of instability are found to be oscillatory. 
Depending on the combination of parameters $\{\eta, \Gr, \mu\}$, both axisymmetric toroidal modes ($m = 0$) and non-axisymmetric spiral modes can grow at the fastest rate. 
For all the critical modes reported in Tables \ref{tbl:eta_0.6_neutral} and \ref{tbl:eta_0.9_neutral}, we observe that critical modes with $m < 0$ have a positive oscillation frequency ($\sigma_{ic}$) and $m \geq 0$ have a negative $\sigma_{ic}$. 
Therefore, all the critical non-axisymmetric modes with $m \geq 0$ travel with an upward axial speed, while $m < 0$ travel with a downward axial speed. 
%Prograde or retrograde propagation concerning the direction of rotation of the inner cylinder can be deduced from the sign of $\sigma_{ic}/m$ and was \aghoredit{retrograde ? check} for all the critical modes found. 
Furthermore, we find both positive and negative $m$ to be the fastest growing modes depending on different parameters, indicating negative and positive inclination $\psi$, respectively, of the lines of constant phase with respect to the horizontal direction according to Eq. \eqref{eq:perturbation_shape}.
For both radius ratios considered here, the fastest growing modes at the onset of instability are seen to be confined near the inner cylinder for $\mu = -0.5, -0.1$. 
This effect is more pronounced in the wide--gap case ($\eta = 0.6$) evident from the $2D$ slices of the reconstructed $3D$ critical eigenmodes.
At $\Gr = 1000$ in the wide--gap case ($\eta = 0.6$), we observe closed loops of unstable regions for $\mu = 0.5$ and an unstable mode $m = -4$ for the co-rotation case $\mu = 1$. 
At $\Gr = 1000$ in the thin--gap case ($\eta = 0.9$), we find $m = -20$ to be the fastest growing mode with $m = -19, -21$ having close critical Taylor numbers $\Ta_c$ for the onset of instability. 

We now shift the focus of our discussion from qualitative to quantitative and describe the effects of varying one of $\{\eta, \Gr, \mu \}$ while the other two control parameters are held constant.
First, we fix $\{ \eta, \Gr \}$ and observe the effect of changing $\mu$. 
This amounts to choosing either of the Tables \ref{tbl:eta_0.6_neutral} or \ref{tbl:eta_0.9_neutral}, fixing a value for $\Gr$ and then comparing the highlighted values, moving in the vertical direction to see the effect of changing $\mu$. 
We find that the outer cylinder rotation has a general stabilizing effect in contrast to scenarios where the outer cylinder is held stationary, except for the case $\{ \eta, \Gr \} = \{0.6, 1000\}$. 
This can be seen by comparing $\Ta_c$ of the fastest growing mode for different $\mu$ at a fixed set of $\{\eta, \Gr\}$. In general, $Ta_c$ for $\mu \neq 0$ is larger in comparison to $Ta_c$ for $\mu = 0$, indicating a general stabilizing effect. In the case $\{ \eta, \Gr \} = \{0.6, 1000\}$, counter-rotation stabilizes the flow (as $\mu$ is lowered below $0$, $Ta_c$ increases in comparison to when $\mu = 0$), but co-rotation has a destabilizing effect, shown by the decrease in $Ta_c$ when $\mu$ is increased from $0$ to $0.2$. 

Next, to assess the effect of the increased temperature gradient we fix $\{\eta, \mu\}$ and compare critical $\Ta_c$ as $\Gr$ is varied. This amounts to choosing one of the tables \ref{tbl:eta_0.6_neutral} or \ref{tbl:eta_0.9_neutral}, fixing a value for $\mu$ and then comparing the highlighted values, moving in the horizontal direction to see the effect of changing $\Gr$. 
Here, in the wide--gap case ($\eta=0.6$) increasing $\Gr$ decreases the critical Taylor number $\Ta_c$ required to trigger instability for all $\mu \in [-0.2, 0.2]$. For other cases, the effect of increasing $\Gr$ is not monotonous.
Therefore, increasing $\Gr$ is in general destabilizing for $\mu \in [-0.2, 0.2]$ but has a non--monotonous response for other values of $\mu$ considered here. 
Similarly, for the thin gap case ($\eta=0.9$), increasing $\Gr$ generally destabilizes the flow except for $\mu \in \{-0.5\}$, where its effect is observed to be stabilizing. 

Finally, we compare critical Taylor numbers at $\eta = 0.6$ and $0.9$ for a constant $\{\Gr, \mu\}$ to assess the impact of increasing radius ratio. This amounts to comparing highlighted values across Tables \ref{tbl:eta_0.6_neutral} and \ref{tbl:eta_0.9_neutral} for a pair of values of $\{ \Gr, \mu \}$. 
It is seen that the critical Taylor numbers for the thin--gap case are lower than those for the wide--gap case for $\mu \leq 0$, thus suggesting that increasing $\eta$ has a destabilizing effect in the Rayleigh-unstable counter-rotation regime.
For $\mu = 0.2$, we see a destabilizing effect of increasing $\eta$ (evidenced by decreased $Ta_c$) for $\Gr = \{ 100, 500\}$, but a stabilizing effect of increasing $\eta$ (evidenced by increased $Ta_c$) for $\Gr = 1000$.
The case of $\mu = 0.5$ is qualitatively different for the two radius ratios, i.e., it is in the Rayleigh-stable regime for $\eta = 0.6$, but in the Rayleigh-unstable regime for $\eta = 0.9$. 
Lastly, the co-rotation case ($\mu = 1$) at $\Gr=1000$, although lying in the Rayleigh-stable regime, is found to be unstable at $Ta \sim 6 \times 10^{3}$ at $\eta = 0.6$, while at $\eta = 0.9$ is found to be unstable at $Ta \sim 5 \times 10^{4}$, thus implying a stabilizing effect of increasing $\eta$ for the co-rotation case in the Rayleigh-stable regime. 

%---------------------------------------
\section{Conclusion}
\label{sec:5}
We conducted a systematic linear stability analysis of the radially--heated Taylor--Couette flow and observed the effects of outer--cylinder rotation.
The selected range of parameters includes three Grashof numbers $\Gr \in \{100, 500, 1000\}$ and two radius ratios $\eta \in \{0.6, 0.9\}$, representing wide--gap and thin--gap cases. 
The Prandtl number was fixed to a constant value $Pr = 1$. For each combination of $(\eta, \Gr)$, seven angular velocity ratios were considered $\mu \in [-1, 1]$, ranging from counter-rotation to co-rotation.
We focused on the small Froude number limit $\Fr \rightarrow 0$ where centrifugal buoyancy can be neglected. 
We extended symmetries of the linearized system identified by \citet{ali1990stability} to include nonzero $\mu$ as defined in Eqs. \eqref{eq:symms1}, \eqref{eq:symms2}, and \eqref{eq:S3}.
We also discussed the observations of our analysis focusing on the the description of the fastest --growing modes, the effects of outer cylinder rotation, of imposed temperature gradient and of increasing radius ratio on the linear stability thresholds. 
%For both radius ratios considered, at $\{\Gr, \mu \} = \{1000, 1\}$, we also found unstable modes violating the Rayleigh--stability criterion.
In the future, we plan to extend the analysis to examine the effects of Prandtl numbers and centrifugal buoyancy (non-vanishing Froude numbers).
The linear stability analysis also provides insight into the interesting range of parameters for performing direct numerical simulations of the governing equations which could shed light on the rich nonlinear dynamics of the system.
%---------------------------------------