%----------------------------------%
\section{Introduction}
\label{sec:intro}
%----------------------------------%
Taylor-Couette flow (TCF) is the annular flow between two concentric, independently rotating cylinders.
Since the seminal work of \citet{taylor1923viii}, the problem and its variants have remained of interest even after a century due to experimental viability and a wide range of relevance from industrial applications to geophysical and astrophysical fluid dynamics, see for example, \citet{gollub1975onset}, \citet{aghor2021nonlinear}, \citet{crowley2022turbulence}, \citet{ji2023taylor} and references therein. 

Linear stability analysis is often the first step to understanding the parameter space and demarcating stable and unstable regions under various probing conditions, see \citet{chandrasekhar1961hydrodynamic}, \citet{drazin2004hydrodynamic}. 
Linear stability analysis of an inviscid fluid by \citet{rayleigh1917dynamics} postulated the so-called Rayleigh criterion. 
According to the Rayleigh criterion, the inviscid flow is stable (unstable) when the square of circulation increases (decreases) monotonically with radial distance. 
In the $\Omega_i-\Omega_o$ parameter space with $\Omega_{i},\Omega_{o}$ representing angular velocities of the inner and outer cylinders respectively, the condition for inviscid instability can be written as $\Omega_{i}/\Omega_{o} \leq \eta^{-2}$, where $\eta = R_{i}/R_{o}$ is the radius ratio with $R_{i},R_{o}$ being the inner and outer radii, respectively. 
In terms of the ratio of angular velocities $\mu = \Omega_o/\Omega_i$, the Rayleigh line that marks the boundary for inviscid instability can be expressed as $\mu = \eta^{2}$ (see Fig. \ref{fig:setup}(Right)). 
When other effects such as magnetic fields (see \citet{ji2023taylor}, \citet{nordsiek2015azimuthal}) or density stratification (see for example, \citet{shalybkov2005stability}, \citet{le2007experimental}, \citet{le2010strato}, \citet{robins2020viscous} and references therein) are considered, the classical Rayleigh criterion can be violated.
% For example, astrophysical flows such as accretion disks are known to be unstable below the Rayleigh line, see \citet{ji2023taylor}, \citet{nordsiek2015azimuthal}. 
% It is known that strato-rotational instability can occur in the Rayleigh-stable regime in the presence of stable axial stratification, see for example, \citet{shalybkov2005stability}, \citet{le2007experimental}, \citet{le2010strato}, \citet{robins2020viscous} and references therein.

Another variant of the classical TCF is the radially heated TCF which is the focus of this study. It is important in many problems ranging from rotating machinery in industrial applications (\citet{lee1989heat}) to geophysical and astrophysical flows, see for example \citet{busse1994convection}, \citet{lopez2013boussinesq}, \citet{jiang2020supergravitational}.

Early efforts of studying radially heated TCF such as \citet{roesner1978hydrodynamic}, \citet{soundalgekar1981effects}, \citet{takhar1985effects} considered only axisymmetric disturbances. 
\citet{ali1990stability} considered radially heated Taylor-Couette setup with stationary outer cylinder and rotating inner cylinder along with non-axisymmetric (helical) disturbances as well. 
% They showed the existence of non-axisymmetric oscillating modes at the onset for a variety of Prandtl numbers. 
Other studies such as \citet{kedia1998numerical}, \citet{yoshikawa2013instability}, \citet{guillerm2015flow}, \citet{kang2015thermal,kang2017radial,kang2019numerical} also considered centrifugal buoyancy which was ignored in earlier studies. 
Most of the investigations of radially heated TCF so far have considered the case of rotating inner cylinder and stationary outer cylinder with a few exceptions, see for example  \citet{meyer2021stability} who investigated the effects of centrifugal buoyancy under the assumption of micro-gravity conditions in the Rayleigh-stable regime. 

%--------------------------------
\begin{figure} 
 \begin{center}
  \includegraphics[scale = 0.5]{figs/setup/setup.pdf}
  \includegraphics[scale = 0.5]{figs/setup/tc_rayleigh.pdf}
\end{center}
\caption{(Left) Schematic of the problem setup. Inner and outer cylinders at radii $R_i, R_o$ rotating independently with angular speeds $\Omega_i,  \Omega_o$ are maintained at constant temperatures $T_i, T_o$, respectively. Cylinders are assumed to be infinite in the axial ($z$) direction.(Right) Rayleigh stability criterion.}
 \label{fig:setup}
\end{figure}
%-------------------------------- 
In this article, we focus on the radially heated TCF for a Boussinesq fluid. Specifically, this paper aims to examine the effects of outer cylinder rotation ($\mu \neq 0$) on the linear stability threshold. 
We consider a fluid of density $\rho$, dynamic viscosity $\nu$, thermal diffusivity $\kappa$, and thermal expansion coefficient $\beta$ in a cylindrical annulus of infinite length. The acceleration due to gravity ($g$) points in the negative axial direction. 
The Prandtl number ($Pr$) for this study, defined as the ratio of momentum diffusivity and thermal diffusivity, is fixed at unity unless stated otherwise. 
A schematic of the setup is depicted in Fig. \ref{fig:setup}(left). 
The inner cylinder with radius $R_{i}$ spinning with angular velocity $\Omega_{i}$ is maintained at a temperature $T_{i}$, whereas the outer cylinder with radius $R_{o}$ is spinning with angular velocity $\Omega_{o}$ and is maintained at temperature $T_{o}$. 
To isolate the effect of outer cylinder rotation, we neglect the centrifugal buoyancy term, corresponding to the limit of vanishing Froude number defined by $Fr = \nu/\sqrt{gd^{3}}$, where $d = R_{o} - R_{i}$ is the gap width, see \citet{yoshikawa2013instability}.
Thus, we arrive at a system similar to \citet{ali1990stability}, but with a general basic azimuthal velocity profile that incorporates the effect of outer--cylinder rotation. 

The rest of the paper is organized as follows. 
In Sec. \ref{sec:2} we formulate the problem by non-dimensionalizing the governing equations and obtain the steady base flow. We then linearize the system around the base flow and use modal ansatz to reduce to an eigenvalue problem in the radial direction. 
We also discuss the symmetries of the system and validate our methodology by comparing results with published literature.
In Sec. \ref{sec:3} we list the results of our analysis for a range of parameters. We consider $\mu \in [-1, 1]$ ranging from counter-rotation ($\mu = -1$) to co-rotation ($\mu = 1$) for radius ratios $\eta=0.6$ (wide--gap case) and $\eta=0.9$ (thin--gap case). 
The Rayleigh line for the wide--gap case is given by $\mu = \eta^{2} = 0.36$, whereas it lies at $\mu = \eta^{2} = 0.81$ for the thin--gap case. 
For each of the radius ratios, three typical values of Grashof numbers $\Gr = \{100, 500, 1000\}$ and seven intermediate values of $\mu \in [-1, 1]$ are analyzed.
In Sec. \ref{sec:4} we discuss the findings of our analysis followed by a conclusion in Sec. \ref{sec:5}.
%--------------------------------