To explain why we apply orthogonal constraints for extracting an interpretable basis, it is better to individually consider cases where those contraints are absent. To begin with, let's consider the case where two concepts in a concept pair belong to different groups of mutually exclusive concepts. In a slightly informal way where strict linear relation is not considered, this makes the two concepts either independent from each other, or positively correlated, since mutual-exclusivity implies negative correlation. In the first case, it is apparent that the respective basis vectors should be orthogonal. To give a counter example, let's consider the concept \textit{car} from the group of \textit{objects} = \{\textit{car, tree, person}\} and the concept \textit{red} from the group of \textit{colors} = \{\textit{red, green, blue}\}. In case the angle between the basis vectors of these two concepts is less (greater) than $90^{\circ}$, some feature vectors that are classified as \textit{car} will inevitably also be classified as (not) \textit{red} and vice versa (Fig. \ref{fig:concept-directions} - Middle). This relation implies dependence which is contradictory to our initial assumption that the two concepts are independent. While this bias may be encoded in the CNN's weights, this fact also means that the two concepts are not (linearly) disentangled, eventually harming the interpretability of the feature space. Since our primary goal is to search for an interpretable basis, given the previous discussion, we know a-priory that a non-orthogonal basis cannot satisfy the interpretability criteria for independent concepts. 

For the second case, where the two concepts are positively correlated, the two concepts could possibly be related with a \textit{has-a} relationship. For instance, \textit{car} has a \textit{car-door} and a \textit{car-wheel}. In this case, an image patch of the concept \textit{car-door} or \textit{car-wheel} may also be classified as \textit{car}. Vice versa, a representation of a \textit{car} may have positive components in the direction of \textit{car-door} and \textit{car-wheel}, to justify the \textit{has-a} relationship. This case is not handled by the proposed method. However, the primitive concepts, \textit{car-door} and \textit{car-wheel}, are mutually exclusive. 

Thus, for this last case, considering two concepts coming from the same group of mutually-exclusive concepts, it could be reasonable to expect that this mutual exclusivity, which implies negative correlation, is also encoded in the angle between the respective basis vectors. In that case, the angle between the respective basis vectors could be greater than $90^{\circ}$ Fig. \ref{fig:concept-directions} - Right. To investigate the degree that this is possible, we formulate the problem in a way that is independent from input data. To construct an (ideal) basis for negatively correlated concepts, one might consider embedding $I$ concept vectors in a $D$ dimensional space by maximizing the minimum angle across all pairs of vectors. As it turns out this is linked to spherical coding theory \cite{whyte1952uniqueSphereCode} and the tammes problem \cite{tammes1930origin}. Although more sophisticated approaches exist \cite{kottwitz1991densestSphericalCode, Wang2009SphericalCodes}, we tried to approximately solve the tammes problem via directly maximizing the minimum pairwise vector angle with gradient decent. Experimental results showed that the resulting embedding vectors, in cases where $I \ge 64$, are close to orthogonal. Fig \ref{fig:tammes} depicts distribution statistics for various pairs of $I,D$ with $I \leq D$. Conclusively, we argue that an orthogonal basis can cover (under some approximation) independent and mutually exclusive concepts but not concepts that are positively correlated.

\begin{figure}
    \centering
    \includegraphics[width=3.3in]{figs/tammes.png}
    \caption{Pairwise vector angle distribution when solving the tammes problem. Extremas of the error bars correspond to the minimum and maximum vector pair angle. The horizontal line in the box is equals to the mean of the distribution and box widths are equal to the standard deviation.}
    \label{fig:tammes}
\end{figure}