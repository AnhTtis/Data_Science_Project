\documentclass[journal]{IEEEtai}

\usepackage[colorlinks,urlcolor=blue,linkcolor=blue,citecolor=blue]{hyperref}
\usepackage{color,array}
\usepackage{graphicx}
\usepackage{subfigure}
\usepackage{booktabs} %
\usepackage{amsmath}
\usepackage{amssymb}
\usepackage{relsize}
\usepackage{caption}
\usepackage{bbm}
\usepackage{hyperref}
\usepackage{changes}
\setcounter{page}{1}

%%%%% GENERAL MATH COMMANDS
% Reals
\newcommand{\R}{{\mathbb R}}
% Integers
\newcommand{\Z}{{\mathbb Z}}
% Naturals
\newcommand{\N}{{\mathbb N}}
% Expectation
\DeclareMathOperator*{\E}{\mathbb{E}}
% ^th notation
\newcommand{\tth}{^{\text{th}}}
% Small dots for integer range [a .. b]
\newcommand{\sdots}{\,..\,}
% Vectorized version of matrix
\newcommand{\matvec}{\mbox{vec}}

% := sign
\newcommand{\defeq}{\vcentcolon=}
% Zero function
\newcommand{\zf}{\mathbf{0}}
% Vector of ones
\newcommand{\ones}{\mathbf{1}}

% Argmin and argmax definitions
\DeclareMathOperator*{\argmax}{arg\,max}
\DeclareMathOperator*{\argmin}{arg\,min}


%%%%% PROBLEM STATEMENT NOTATION 
% \newcommandtwoopt{\St}[2][t][]{{S_{#1}^{#2}}} % State
\newcommand{\task}[1][i]{{\mathcal{T}_{#1}}} % Task, optionally takes index
\newcommand{\tasks}{\{ \task \}_{i=1}^N}
\newcommand{\losst}[1][i]{{l_{#1}}}
\newcommand{\lossv}[1][i]{{l_{#1}^{\textrm{val}}}}
\newcommand{\tasktarget}{{\mathcal{T}_{\textrm{target}}}}
\newcommand{\lossttarget}{l_{\textrm{target}}}
\newcommand{\lossvtarget}{l_{\textrm{target}}^{\textrm{val}}}
\newcommand{\lossttargetit}{l_{\textrm{target}}^{(k)}}
\newcommand{\losstotal}{l^{\textrm{total}}}
\newcommand{\lossopt}{l^*}

\newcommand{\thetait}[2]{\theta_{#1}^{(#2)}}
\newcommand{\phit}[1]{\phi^{(#1)}}
\newcommand{\hist}[2]{S_{#1}^{(#2)}}
\newcommand{\grad}[2]{G_{#1}^{(#2)}}

\newcommand{\Alg}{\textup{\textbf{Opt}}}
\newcommand{\MetaAlg}{\textup{\textbf{MetaOpt}}}

%%%%% Theorems
\newtheoremstyle{mytheoremstyle} % name
    {\topsep}                    % Space above
    {\topsep}                    % Space below
    {\itshape}                   % Body font
    {}                           % Indent amount
    {\scshape}                   % Theorem head font
    {.}                          % Punctuation after theorem head
    {.5em}                       % Space after theorem head
    {}  % Theorem head spec (can be left empty, meaning ‘normal’)
\theoremstyle{mytheoremstyle}
\theoremstyle{plain}
\newtheorem{theorem}{Theorem}
\newtheorem{proposition}{Proposition}
\newtheorem{assumption}{Assumption}
\newtheorem{definition}{Definition}
\newtheorem{lemma}{Lemma}
\theoremstyle{remark}
\newtheorem{remark}{Remark}


\begin{document}

\title{Unsupervised Interpretable Basis Extraction for Concept -- Based Visual Explanations} 

\author{Alexandros Doumanoglou, Stylianos Asteriadis, and Dimitrios Zarpalas
\thanks{A. Doumanoglou is with the Information Technologies Institute, Centre for Research and Technology HELLAS, Thessaloniki, Greece and the Department of Advanced Computing Sciences, University of Maastricht, Maastricht, Netherlands (e-mail: aldoum@iti.gr).}
\thanks{S. Asteriadis was with the Department of Advanced Computing Sciences, University of Maastricht, Maastricht, Netherlands (e-mail: stelios.asteriadis@maastrichtuniversity.nl). 
}
\thanks{D. Zarpalas is with the Information Technologies Institute, Centre for Research and Technology HELLAS, Thessaloniki, Greece (e-mail: zarpalas@iti.gr).}
}

\markboth{}
{A. Doumanoglou \MakeLowercase{\textit{et al.}}: UIBE}

\maketitle
\begin{abstract}

An important line of research attempts to explain CNN image classifier predictions and intermediate layer representations in terms of human-understandable concepts. Previous work supports that deep representations are linearly separable with respect to their concept label, implying that the feature space has directions where intermediate representations may be projected onto, to become more understandable. These directions are called interpretable, and when considered as a set, they may form an interpretable feature space basis. Compared to previous top-down probing approaches which use concept annotations to identify the interpretable directions one at a time, in this work, we take a bottom-up approach, identifying the directions from the structure of the feature space, collectively, without relying on supervision from concept labels. Instead, we learn the directions by optimizing for a sparsity property that holds for any interpretable basis. We experiment with existing popular CNNs and demonstrate the effectiveness of our method in extracting an interpretable basis across network architectures and training datasets. We make extensions to existing basis interpretability metrics and show that intermediate layer representations become more interpretable when transformed with the extracted bases. Finally, we compare the bases extracted with our method with the bases derived with supervision and find that, in one aspect, unsupervised basis extraction has a strength that constitutes a limitation of learning the basis with supervision, and we provide potential directions for future research.


%An important line of research attempts to explain CNN image classifier predictions and intermediate layer representations in terms of human-understandable concepts. The success of linear probing followed in previous works indicates that deep representations of image patches are linearly separable based on their concept label. The weight vectors of the linear probes constitute directions which are called interpretable due to their applicability in detecting human-understandable concepts. A set of interpretable directions may form an interpretable feature space basis where feature representations may be projected onto, to become more understandable. Compared to previous top-down probing approaches which use concept annotations to identify the interpretable directions one at a time, in this work, we take a bottom-up approach, identifying the directions collectively as a set from the structure of the feature space, without relying on supervision from concept labels. Instead, our approach learns the directions set by optimizing for a sparsity property that every interpretable basis meets. We experiment with existing popular CNNs and demonstrate the effectiveness of our method in extracting an interpretable basis across network architectures and training datasets. We make extensions to the existing basis interpretability metrics found in the literature and show that, intermediate layer representations become more interpretable when transformed to the bases extracted with our method. Finally, using the basis interpretability metrics, we compare the bases extracted with our method with the bases derived with a supervised approach and find that, in one aspect, the proposed unsupervised approach has a strength that constitutes a limitation of the supervised one and give potential directions for future research.



%An important line of research attempts to explain CNN image classifier predictions and intermediate layer representations in terms of human understandable concepts. In this work, we expand on previous works in the literature that use annotated concept datasets to extract interpretable feature space directions and propose an unsupervised post-hoc method to extract a disentangling interpretable basis by looking for the rotation of the feature space that explains sparse one-hot thresholded transformed representations of pixel activations. We do experimentation with existing popular CNNs and demonstrate the effectiveness of our method in extracting an interpretable basis across network architectures and training datasets. We make extensions to the existing basis interpretability metrics found in the literature and show that, intermediate layer representations become more interpretable when transformed to the bases extracted with our method. Finally, using the basis interpretability metrics, we compare the bases extracted with our method with the bases derived with a supervised approach and find that, in one aspect, the proposed unsupervised approach has a strength that constitutes a limitation of the supervised one and give potential directions for future research.
\end{abstract}

\begin{IEEEImpStatement}
CNN image classifiers have demonstrated outstanding performance in real-world tasks. They can be used in robotics, visual understanding, automatic risk assessment, and more. However, to a human expert, CNNs are often black-boxes and the reasoning behind their predictions can be unclear.  Recent advances in explainable and interpretable artificial intelligence (XAI and IAI) attempt to shed light on this process. In an attempt to understand intermediate layer representations, one can project them onto a feature space basis that quantifies the presence of different concepts in the representation. This basis is called interpretable because it can make representations more understandable. In the typical approach, constructing an interpretable basis requires access to annotations. This work proposes a novel unsupervised method to learn such a basis, without the need for explicit labels. This can ease the process of obtaining explanations, eliminate annotation costs, save time, and eventually help humans debug and trust deep models.

\end{IEEEImpStatement}

\begin{IEEEkeywords}
Explainable Artificial Intelligence (XAI), Interpretable Artificial Intelligence (IAI), Interpretable Basis, Unsupervised Learning.
\end{IEEEkeywords}

\vspace{5cm}
\section{Introduction}
\label{sec:introduction}
\section{Introduction}
\label{sec:introduction}
% \begin{itemize}
%     % Diffusion of FL
%     \item {\st{Diffusion of FL}}
%     % Security threats to FL
%     \item {\st{Security threats to FL with particular focus on model poisoning}}
%     % Limitations of existing countermeasures
%     \item {\st{Current countermeasures (e.g., KRUM) and their limitations}}
%     % Proposed method and its advantages
%     \item {\st{Intuitive description of the proposed method and its difference (i.e., advantages) w.r.t. state of the art}}
%     % Main contributions
%     \item {\st{Summary of the main contributions of this work}}
%     % Paper's structure and organization
%     \item {\st{Paper's structure and organization}}
% \end{itemize}

% Diffusion of FL
Recently, {\em federated learning} (FL) has emerged as the leading paradigm for training distributed, large-scale, and privacy-preserving machine learning (ML) systems~\cite{mcmahan2017googleai,mcmahan2017aistats}. 
The core idea of FL is to allow multiple edge clients to collaboratively train a shared, global model without disclosing their local private training data.
%Specifically, an FL system consists of a central server and many edge clients; 
A typical FL round involves the following steps: {\em(i)} the server randomly picks some clients and sends them the current, global model; {\em(ii)} each selected client locally trains its model with its own private data; then, it sends the resulting local model to the server;\footnote{Whenever we refer to global/local model, we mean global/local model {\em parameters}.} {\em(iii)} the server updates the global model by computing an \emph{aggregation function}, usually the average (FedAvg), on the local models received from clients.
% \begin{enumerate}
%     \item[{\em(i)}] the server sends the current, global model to the clients and appoints some of them for training;
%     \item[{\em(ii)}] each selected client locally trains its copy of the global model with its own private data; then, it sends the resulting local model back to the server;\footnote{Whenever we refer to global/local model, we mean global/local model {\em parameters}.}
%     \item[{\em(iii)}] the server updates the global model by computing an \emph{aggregation function} on the local models received from clients (by default, the average, also referred to as FedAvg~\cite{mcmahan2017aistats}).
% \end{enumerate}
This process goes on until the global model converges. %(e.g., after a certain number of rounds or other similar stopping criteria).
%\\
% The advantages of FL over the traditional, centralized learning paradigm are undoubtedly clear in terms of flexibility/scalability (clients can join/disconnect from the FL network dynamically), network communications (only model weights\footnote{We will use \textit{parameters} and \textit{weights} interchangeably.} are exchanged between clients and server), and privacy (each client's private training data is kept local at the client's end and not uploaded to the server).
\\
% Security threats to FL
%However, the growing adoption of FL also raises security concerns~\cite{costa2022covert}, particularly about its confidentiality, integrity, and availability.
Although its advantages over standard ML, FL also raises security concerns~\cite{costa2022covert}. %, particularly about its confidentiality, integrity, and availability~\cite{costa2022covert}.
% OLD, LONG VERSION
% Indeed, some work deals with privacy leakage that may expose the local data of some clients~\cite{melis2019sp}. 
% A large body of work, instead, investigates attacks that usually aim to detriment the predictive accuracy of the learned global model. For instance, \emph{data poisoning} attacks achieve this goal by letting an adversary pollute the training set of some corrupt FL clients with maliciously crafted examples~\cite{jagielski2018sp}.
% Similarly, in \emph{model poisoning} the attacker attempts to tweak the global model weights~\cite{bhagoji2019pmlr} by directly perturbing the local model's weights of some infected FL clients before these are sent to the central server for aggregation, usually via so-called Byzantine attacks. 
% It turns out that Byzantine model poisoning attacks severely impact standard FedAvg; therefore, more robust aggregation functions must be designed to make FL systems secure.
Here, we focus on \emph{untargeted model poisoning} attacks~\cite{bhagoji2019pmlr}, where an adversary attempts to tweak the global model weights %\footnote{We will use the terms \textit{parameters} and \textit{weights} interchangeably.} 
by directly perturbing the local model's parameters of some infected clients before these are sent to the central server for aggregation.
In doing so, the adversary aims to jeopardize the global model \textit{indiscriminately} at inference time.
Such model poisoning attacks severely impact standard FedAvg; therefore, more robust aggregation functions must be designed to secure FL systems.
\\
% In this paper, we focus on designing a novel robust aggregation scheme at the server's end to contrast the effect of Byzantine model poisoning attacks.
%
% Current countermeasures and their limitations
%Several countermeasures have been proposed in the literature to combat model poisoning attacks on FL systems.
% Some methods use simple statistics more robust than plain average to smooth the impact of malicious updates (e.g., Trimmed Mean and FedMedian~\cite{yin2018icml}). 
% Other defenses implement outlier detection techniques to discard malicious updates from the aggregation performed at the server's end. Those are either based on heuristics (e.g., Krum/Multi-Krum~\cite{blanchard2017nips} and Bulyan~\cite{mhamdi2018pmlr}) or data-driven approaches (e.g., K-means clustering~\cite{shen2016acm} or DnC via spectral analysis~\cite{shejwalkar2021ndss}). 
% Finally, some strategies rely on a centralized ``source of trust'' to spot potential malicious updates (e.g., FLTrust~\cite{cao2020fltrust}).
% Several countermeasures have been proposed in the literature to combat model poisoning attacks on FL systems, i.e., to discard possible malicious local updates from the aggregation performed at the server's end. 
% These techniques range from simple statistics more robust than plain average (e.g., Trimmed Mean and FedMedian~\cite{yin2018icml}) to outlier detection heuristics (e.g., Krum/Multi-Krum~\cite{blanchard2017nips} and Bulyan~\cite{mhamdi2018pmlr}) or data-driven approaches (e.g., spectral analysis via K-means clustering~\cite{shen2016acm} or spectral analysis), or methods based on ``source of trust'' (e.g., FLTrust~\cite{cao2020fltrust}).
% OLD, LONG VERSION
%Several countermeasures have been proposed in the literature to combat Byzantine model poisoning attacks on FL systems.
% Descriptive statistics
% For example, Trimmed Mean and FedMedian aggregate local model updates using more robust statistics than standard average~\cite{yin2018icml}.
%
% % Heuristics for outlier detection
% Many existing Byzantine-resilient strategies implement some outlier detection heuristics to discard the model updates sent by potentially malicious clients from the input of the aggregation function.
% One of the most popular heuristics is Krum~\cite{blanchard2017nips}.
% This strategy tries to mitigate the impact of Byzantine attacks by selecting as a global model the local model with the smallest sum of Euclidean distances to {\em all} the other local models.
% Although powerful, Krum requires the server to know (or, at least, estimate) the number of malicious FL clients upfront, which is generally impossible in a realistic attack scenario. %
% Moreover, Krum may become ineffective for complex, high-dimensional model parameter spaces due to the curse of dimensionality.
% Bulyan~\cite{mhamdi2018pmlr} tries to overcome this issue by combining Krum with a variant of Trimmed Mean.
% % Data-driven outlier detection
% Other strategies use data-driven outlier detection techniques -- e.g., via K-means clustering~\cite{shen2016acm} -- to spot potential malicious local model updates. 
% %For instance, Shen et al. propose to cluster local model updates with K-means and thus identify outliers.
%
% % Other techniques
% As far as the server is concerned, any local model received can be from a potential malicious client. 
% FLTrust~\cite{cao2020fltrust} assumes the server acts as a client, i.e., trains a local model on an additional {\em trustworthy} dataset at the server's end and compares it against all the local models from other clients. 
% This way, the server can rely on some ``source of trust'' when discarding potentially malicious clients.
%\\
% Limitations of existing Byzantine-resilient strategies
Unfortunately, existing defense mechanisms either rely on simple heuristics (e.g., Trimmed Mean and FedMedian by~\cite{yin2018icml}) or need strong and unrealistic assumptions to work effectively (e.g., foreknowledge or estimation of the number of malicious clients in the FL system, as for Krum/Multi-Krum~\cite{blanchard2017nips} and Bulyan~\cite{mhamdi2018pmlr}, which, however, cannot exceed a fixed threshold).
Furthermore, outlier detection methods using K-means clustering~\cite{shen2016acm} or spectral analysis like DnC~\cite{shejwalkar2021ndss} do not directly consider the temporal evolution of local model updates received.
Finally, strategies like FLTrust~\cite{cao2020fltrust} require the server to collect its own dataset and act as a proper client, thereby altering the standard FL protocol.
\\
% OLD, LONG VERSION
% Overall, existing Byzantine-resilient strategies are either simple heuristics (e.g., FedMedian) or, if they are more complex, they rely on strong and unrealistic assumptions to work effectively (e.g., knowing the number of malicious clients in the FL system in advance, as for Krum and alike).
% Furthermore, data-driven outlier detection methods do not consider the temporary evolution of local model updates received (e.g., K-means clustering). 
% Finally, strategies like FLTrust requires the server to collect its own dataset and act as a proper client, thereby altering the standard FL protocol.
%
% Description of the proposed method
This work introduces a novel pre-aggregation \textit{filter} robust to untargeted model poisoning attacks. Notably, this filter $(i)$ operates without requiring prior knowledge or constraints on the number of malicious clients and $(ii)$ inherently integrates temporal dependencies. 
The FL server can employ this filter as a preprocessing step before applying \textit{any} aggregation function, be it standard like FedAvg or robust like Krum or Bulyan.
Specifically, we formulate the problem of identifying corrupted updates as a multidimensional (i.e., matrix-valued) time series anomaly detection task. 
The key idea is that legitimate local updates, resulting from well-calibrated iterative procedures like stochastic gradient descent (SGD) with an appropriate learning rate, show \textit{higher predictability} compared to malicious updates. This hypothesis stems from the fact that the sequence of gradients (thus, model parameters) observed during legitimate training exhibit regular patterns, as validated in Section~\ref{subsec:intuition}. %until convergence. 
%This regularity may be more pronounced for smooth convex loss functions, but it can still be captured within an appropriate time window, even for more complex and convoluted loss surfaces. 
%We provide evidence of this claim in Appendix~B, where we show that the average mutual information (i.e., ``predictability''), calculated over pairs of legitimate model updates sent at different FL rounds, is significantly higher than the corresponding computation for a malicious client.
\\
Inspired by the matrix autoregressive (MAR) framework for multidimensional time series forecasting~\cite{chen2021je}, we propose the FLANDERS ({\em \textbf{F}ederated \textbf{L}earning meets \textbf{AN}omaly \textbf{DE}tection for a \textbf{R}obust and \textbf{S}ecure}) filter.
The main advantages of FLANDERS over existing strategies like FLDetector~\cite{zhao2020multivariate} are its resilience to large-scale attacks, where $50\%$ or more FL participants are hostile, and the capability of working under realistic non-iid scenarios.
We attribute such a capability to two key factors: $(i)$ FLANDERS works without knowing a priori the ratio of corrupted clients, and $(ii)$ it embodies temporal dependencies between intra- and inter-client updates, quickly recognizing local model drifts caused by evil players. Below, we summarize our main contributions:

\begin{itemize}
\item[{\em(i)}]
We provide empirical evidence that the sequence of models sent by legitimate clients is more predictable than those of malicious participants performing untargeted model poisoning attacks.
\\
\item[{\em(ii)}] 
We introduce FLANDERS, the first pre-aggregation filter for FL robust to untargeted model poisoning based on multidimensional time series anomaly detection.
\\
\item[{\em(iii)}] 
We integrate FLANDERS into Flower,\footnote{\scriptsize{\url{https://flower.dev/}}} a popular FL simulation framework for reproducibility.
\\
\item[{\em(iv)}] 
We show that FLANDERS improves the robustness of the existing aggregation methods under multiple settings: different datasets, client's data distribution (non-iid), models, and attack scenarios.
\\
\item[{\em(v)}] 
We publicly release all the implementation code of FLANDERS along with our experiments.\footnote{\scriptsize{\url{https://anonymous.4open.science/r/flanders_exp-7EEB}}}
\end{itemize}

% Paper's structure and organization
The remainder of the paper is structured as follows. %some related work and the current state-of-the-art solutions to security issues that FL entails. 
Section~\ref{sec:background} covers background and preliminaries. 
In Section~\ref{sec:related}, we discuss related work.
Section~\ref{sec:problem} and Section~\ref{sec:method} describe the problem formulation and the method proposed. % to tackle it. 
Section~\ref{sec:experiments} gathers experimental results. %, and Section~\ref{sec:limitations} discusses some limitations of this work.
Finally, we conclude in Section~\ref{sec:conclusion}.
 %discusses the limitations of this work and draws future research directions.
%reports conclusions and draws perspectives for future research directions.

%%%%%%% OLD %%%%%%%
%to overcome the resilience of Byzantine failures in distributed Stochastic Gradient Descent computations. 
% The strength of Krum is its time complexity, which is linear in the gradient dimension. 
% However, the robustness of the approach is guaranteed for gradient-based learning applications only when the majority of the clients are not compromised. 
% Besides, the aggregation mechanism of Krum, as well as that of similar methods, is robust from a coarse-grained perspective and does not provide solutions to errors and perturbations that may occur at inference time.
%A related approach to~\cite{blanchard2017nips} is the work of Su et al.~\cite{su2016dc}. Here, the authors propose an iterated approximate agreement to tackle a multi-layer scenario attacked by Byzantine agents. 
%However, the method works efficiently on the sole discrete context and it is inapplicable to continuous state environments.
%\gabri{Maybe, we should just talk about the main limitations of existing countermeasures without digging into their details (or, we can just mention Krum as this is the most popular one). I will move the description of all these methods to the Related Work section.}

\section{Background \& Related Work}
\label{sec:related_work}
\section{Related work}
% There is extensive recent work on speeding up analytical queries due to the need for consistent execution times in the face of the explosive growth in the volume of available data.
% In this section, we divide existing work into two categories: maintaining data freshness in MVs (\Cref{sec:server_side}) and optimizations for minimizing ad-hoc query latency (\Cref{sec:client_side}).

% \subsection{Maintaining Data Freshness in MVs}
% \label{sec:server_side}
% There exists a variety of data warehousing applications aimed at supporting low-latency analytical queries on fresh data.
% In particular, these applications require efficiency in the propagation of newly ingested data into downstream MVs.
 
\mypara{Efficient MV Refresh}
Incremental view maintenance (IVM) aims to update MVs to reflect newly ingested data, taking advantage of already computed results to perform the update in a manner more efficient than computing from scratch (full refresh)
~\cite{ahmad2012dbtoaster,mcsherry2013differential,armbrust2013generalized,zeng2016iolap, palpanas2002incremental, griffin1995incremental, agiwal2021napa, braun2015analytics}. 
There is an abundance of work in IVM, including incremental updates on duplicate values~\cite{griffin1995incremental}, non-distributive aggregate functions~\cite{palpanas2002incremental}, higher-order views~\cite{ahmad2012dbtoaster}, and sliding windows~\cite{braun2015analytics}. 
More recent works also investigate the scalability aspect of IVM, proposing scale-independent updates~\cite{armbrust2013generalized} and sampled views~\cite{zeng2016iolap}. Since \system is applicable to arbitrary SQL statements, \system is orthogonal to and is fully compatible with existing IVM techniques.

\mypara{MV Refresh Scheduling}
There exist works on scheduling the refresh of a MV set focusing on resolving cyclic dependencies~\cite{folkert2005optimizing}, minimizing weighted average staleness~\cite{golab2009scheduling}, and prioritizing MVs with the highest speedups on predicted future queries~\cite{ahmed2020automated}.
\system's scheduling to speed up the end-to-end refresh of the MV set is not addressed in existing works.

\mypara{DAG Workflow Scheduling}
The execution of workloads consisting of individual jobs with acyclic dependencies is a well-studied topic~\cite{apacheoozie,sparkdag,marchal2018parallel,bathie2020revisiting,baruah2022ilp}; many of these techniques can be applied to MV refresh runs studied in this paper.
Existing workflow scheduling systems such as Apache Oozie~\cite{apacheoozie}, Apache Airflow~\cite{airflow}, and Spark DAG scheduler~\cite{sparkdag} automate the execution of user-defined workflows following a topological order.
There are recent works aimed at finding more optimal execution orders in terms of peak memory usage~\cite{marchal2018parallel, bathie2020revisiting} and execution time on parallel platforms~\cite{baruah2022ilp}.
While \system is designed for use with MV refresh runs/workloads, our technique on joint scheduling and optimization can be reasonably applied to general workloads as a possible future direction.

% \paragraph{Incremental MV indexing}
% Update-optimized indices such as the log-structured merge-trees (LSM)~\cite{o1996log} are used for indexing MVs due to frequent updates induced by data ingestion~\cite{gupta2016mesa,agiwal2021napa}.
% \system is orthogonal to indexing: \system is capable of efficiently performing MV refresh runs regardless of whether the individual MVs are indexed or not.

% \subsection{Ad-hoc Query Latency Reduction}
% \label{sec:client_side}

% The minimization of ad-hoc analytical query response times is a well-studied topic due to latency being negatively correlated with the productivity of a data analyst during a data analysis session~\cite{liu2014effects}.
% Sessions are commonly conducted within visualization systems that contain a variety of optimization techniques to ensure that query response times fall within a certain latency tolerance.

% \mypara{Data prefetching}
% Data is often loaded into memory on a by-need basis in visualization systems to minimize interference with user-issued query computations~\cite{mani2017effective,xin2021enhancing,galakatos2017revisiting, yan2020auto, battle2016dynamic, crotty2016case, jalaparti2018netco}. 
% Query-time data retrieval can be significantly expedited by anticipating the data usage of the user in future queries and pre-loading the data into memory during the downtime between user queries (`think time'). SMART~\cite{mani2017effective} prefetches data for modified versions of current user-issued queries with different filters and dimensions. A-WARE~\cite{crotty2016case} maintains a sample store constantly refined through ingesting data based on speculations of future plots.
% ForeCache~\cite{battle2016dynamic} uses an SVM to predict the user's current analysis phase and accordingly prefetches data tiles partitioned based on different numerical values. NetCo predicts future queries via log analysis, and solves an ILP formulation to prefetch data to maximize the number of SLO-meeting queries~\cite{jalaparti2018netco}.
% In the case of MV refresh workloads, `think time' is nonexistent as individual MVs are refreshed back-to-back, rendering data prefetching techniques non-applicable.

\mypara{Intermediate Data Caching}
Some existing data visualization systems cache user-defined variables to support the typical incremental construction of data visualizations~\cite{zgraggen2016progressive, eichmann2020idebench} during data analysis sessions~\cite{jupyter, rstudio, colab}. 
Recent work proposes a management scheme for these cached variables under a memory constraint that greedily keeps variables with the highest estimated time savings based on predicted future user behavior via neural networks~\cite{xin2021enhancing}.
While useful for data visualization, a greedy approach to memory management fails to achieve satisfactory results compared to \system.

\mypara{Intermediate Result Reuse}

There exist works on storing intermediate results from computations to speedup future computations~\cite{yang2018intermediate, dursun2017revisiting, nagel2013recycling, michiardi2019memory, galakatos2017revisiting}.
Studied topics include the identification of reuse opportunities by finding overlaps in computation graphs of successive jobs~\cite{yang2018intermediate, michiardi2019memory},
selective storage under a space constraint with heuristics such as reuse probability~\cite{dursun2017revisiting}, expected savings~\cite{yang2018intermediate}, and recompute-storage cost difference~\cite{nagel2013recycling},
and rewriting incoming jobs to take advantage of stored intermediates~\cite{galakatos2017revisiting}.
These works share similarity with \system in their selection of items to store under a memory constraint, however, \system's problem setting requires it to uniquely consider the joint (re)ordering of job executions along with the selection of items.

% work that considers both job execution (re)order as well as intermediate result caching with a bounded amount of memory. but notably lack the joint aspect of \system and cannot be used to achieve immediate speedup on an incoming MV refresh run if no intermediates are stored beforehand. 

\mypara{Incremental Query Processing} Incremental processing (IQP) is useful for cases where not all data required for a query is immediately available. Similar to online aggregation~\cite{hellerstein1997online}, initial results of a query are computed on a subset of required data and progressively refined as the rest of the required data arrives in a predictable pattern~\cite{tang2019intermittent,wangtempura}. Tang et al. propose a dynamic programming formulation to pick intermediate states to store in memory given a limited memory budget~\cite{tang2019intermittent}. Tempura rewrites the query plan for more efficient execution based on predicted data arrival patterns~\cite{wangtempura}. While similarities exist between the problem setting of IQP and \system, such as management of bounded memory, \system notably includes additional joint optimization for the order of MV updates.

% \paragraph{Sampling}
% Sampling has seen wide use in visualization systems for reducing the computation time of ad-hoc queries by computing an approximate result over a subset of data as exact results are not always required by the user~\cite{crotty2016case, mani2017effective, zgraggen2014panoramicdata, kraska2021northstar, galakatos2017revisiting, kandula2016quickr}. 
% Commonly studied topics in sampling for ad-hoc queries include complex query sampling~\cite{kandula2016quickr}, rare event aggregation~\cite{kraska2021northstar, galakatos2017revisiting}, and maintaining consistency between related sampled visualizations~\cite{zgraggen2014panoramicdata}.
% Sampling server-side at the MV level compromises the assumptions of downstream applications and is thus not considered in \system.

% \paragraph{Progressive visualization}
% The latency tolerance for time-consuming queries can be circumvented by presenting a partially-computed visualization to the user within the tolerance, which is then incrementally refined until it is fully accurate~\cite{rahman2017ve, zgraggen2016progressive, crotty2015vizdom, kraska2021northstar, kamat2017infiniviz}.
% Example plots which benefit from progressive visualization include bar charts~\cite{kamat2017infiniviz} and heatmaps~\cite{rahman2017ve}.
% Similar to sampling, study on this topic is orthogonal to \system as pushing out partially-updated MVs compromises downstream assumptions.

\section{Motivation}
\label{sec:motivation}
\section{Threat Model and Advantages of Our Hardware-based Adversarial Detector} \label{sec: motivation}
\ry{In this part, I want to highlight the comparison between hardware and software attacks}
%Normally, software-based adversarial detectors are easier to implement, cheaper to develop and more well-studied than those based on hardware computational signals.
% We would like to stress that our goal for investigating hardware-based adversarial detectors is not to achieve better performance in detection than the conventional white-box software based methods.  
\subsection{Threat Model} \label{sec: threat model}
\ry{This section is threat model: attack is `white-box', detector is `black-box'}
The victim is a DNN classifier, which is pre-trained with a public dataset. The testing dataset may be kept private.
We assume the strongest `white-box' attack model, where the attacker has full knowledge of the victim model and training dataset in order to generate adversarial samples with minimum perturbations. 
On the contrary, the detection system assumes the most limited scenario, under a `black-box' view of the victim, without access to the victim's inputs, parameters, and intermediate outputs or execution details. 
The only information available to the detector to distinguish adversarial samples is the EM side-channel measurement and the victim model's prediction class.
For training the adversarial detector with EM traces, a public benign dataset is used. 

\if false 
\ry{In this part, we discuss more settings of the detector especially the data used in two phases.}
In general, the detecting process can be summed up into two phases, training phase and detecting phase.
To begin with, we train an Out-of-Distribution(OOD) detector on a public benign dataset of the same classification task, which should be distinct from the victim's training dataset.
For each query, the detector will obtain the classification result and an EM trace along with the model execution to fit its EM classifiers and anomaly detectors.  
During the detection phase, the victim model is in operation and under attack when the pre-trained detector decides whether the current input is adversarial or not, only based on the victim model output and its EM trace.
\fi 

\subsection{Advantages}
Compared to software-based adversarial detection methods, our hardware-based detector, EMShepherd, has three distinct advantages: privacy-preserving, portability, and robustness.

\begin{itemize}[leftmargin=*]
    \item \ry{Add a new motivation here. The motivation is that using \name can help the user protect their privacy.} 
    \name protects the DNN model user's data privacy as it is agnostic to the model's inputs, which instead are always required by prior reconstruction-based detection methods~\cite{meng2017magnet, yang2022you}. 
    %Most model users are benign whose inputs may be sensitive and should not be shared with \textit{third-party detectors}. 
    The sensitive inputs should not be shared with \textit{third-party detectors}. 
    Our design only requires the output class labels and the EM signals, which are passively leaked to common acquisition equipment. 
    %    Our design is suitable for such cases as it only requires the EM signals and the inference outputs during the model execution. Generally speaking, EM signals and labels have less private information leakage.
    \item \ry{The second motivation is still related to privacy. This time we consider model privacy when the model structure or parameters should be kept private.}
   \name also protects the model confidentiality.  No model information, including %Using hardware-based detectors can prevent the third-party defender from accessing some confidential model information such as  
   hyper-parameters, parameters, and logits, is needed, in stark contrast to the previous software-based detection methods~\cite{ma2019nic,feinman2017detecting}.
    %Our \name only acquires the EM traces during model inference in a passive and noninvasive manner, 
    The EM data processing and the adversarial detector training process are both victim model-agnostic. 
    Therefore, our method has more general usage, applicable to closed-source DNN applications, which are pervasive in edge devices where the user only queries the models for the final prediction output. 
    \item \ry{The third motivation is portability.}  
    Owing to the model-agnostic feature, EMShepherd can be easily ported for wide-range hardware devices with different DNN implementations for diverse applications. It can be used as a `plug and play' (PnP) device, aside from the target system, to work automatically without user intervention or contact with the victim system. 
    \item \ry{The last motivation is about adaptive attacks, we should propose that EM signal is hard to imitate, so it is hard for adaptive attacks to generate sample fraud both detector and victim.} 
    Adaptive attack~\cite{adaptive} is a threat to most software defense methods where the attacker adjusts the adversarial perturbations to mislead both the victim models and defense systems.
   %  The hardware-based detection method can provide a double protection on top of most software defense methods such as adversarial training.
   %  Although the adptive adversarial example fools the robust model, its computation patterns during the DNN model execution are still well kept in the EM traces and our EMShepherd framework still works well for detecting the new type of adversarial examples.  
   %  Meanwhile, due to the high complexity of EM signals and non-explicit dependency of the EM signals on computations, it is extremely hard to have an adaptive attack on our detection method, i.e., adversarial examples whose EM signals are deliberately controlled to evade the EM-based detector.
   However, due to the high complexity and non-explicit dependency of the EM signals on computations and data, 
   it is extremely hard to have an adaptive attack on our detection method, 
   i.e., adversarial examples whose EM signals are deliberately controlled to evade the EM-based detector. 
\end{itemize}






\section{Proposed Method}
\label{sec:proposed_method}
\section{LayoutDM}
Our LayoutDM builds on discrete-state space diffusion models~\cite{austin2021structured,gu2022vector}.
We first briefly review the fundamental of discrete diffusion models in \cref{subsec:discrete_diffusion}.
\cref{subsec:layout_diffusion_unconditional} explains our approach to layout generation within the diffusion framework while discussing features inherent in layout compared with text.
\cref{subsec:layout_diffusion_conditional} discusses how we extend denoising steps to perform various conditional layout generation by imposing conditions in each step of the reverse process.

\subsection{Preliminary: Discrete Diffusion Models}
\label{subsec:discrete_diffusion}


Diffusion models~\cite{sohl2015deep} are generative models characterized by a forward and reverse Markov process.
While many diffusion models are defined on continuous space with Gaussian corruption, D3PM~\cite{austin2021structured} introduces a general diffusion framework for categorical variables designed primarily for texts.
Let $T \in \mathbb{N}$ be a total timestep of the diffusion model, we first explain the forward diffusion process.
For a scalar discrete variable with $K$ categories $z_{t} \in \{1,2,\ldots,K\}$ at timestep $t \in \mathbb{N}$, probabilities that $z_{t-1}$ transits to $z_{t}$ are defined by using a transition matrix $\bm{Q}_{t} \in [0,1]^{K \times K}$, with $[Q_{t}]_{mn} = q(z_{t}\!=\!m | z_{t-1}\!=\!n)$,

\begin{equation}
q(z_{t}|z_{t-1}) = \bm{v}(z_{t})^{\!\top} \mathbf{Q}_{t} \bm{v}(z_{t-1}),
\end{equation}
where $\bm{v}(z_{t}) \in \{0,1\}^{K}$ is a column one-hot vector of $z_{t}$.
The categorical distribution over $z_{t}$ given $z_{t-1}$ is computed by a column vector $\mathbf{Q}_{t} \bm{v}(z_{t-1}) \in [0,1]^{K}$.
Assuming the Markov property,
we can derive $q(z_{t}|z_{0}) = \bm{v}(z_{t})^{\!\top}\overline{\mathbf{Q}}_{t}\bm{v}(z_{0})$ where $\overline{\mathbf{Q}}_{t}\!=\!\mathbf{Q}_{t}\mathbf{Q}_{t-1}\cdots\mathbf{Q}_{1}$ and:
\begin{align}
    &q(z_{t-1}|z_{t}, z_{0}) = \frac{
        q(z_{t}|z_{t-1}, z_{0})\,q(z_{t-1}|z_{0})
    }{
        q(z_{t}|z_{0})
    } \nonumber \\
    &= \frac{
        \left(\bm{v}\!\left(z_{t}\right)^{\!\top}\!\mathbf{Q}_{t}\bm{v}\!\left(z_{t-1}\right)\right)
        \left( \bm{v}\!\left(z_{t-1}\right)^{\!\top}\overline{\mathbf{Q}}_{t-1}\bm{v}\!\left(z_{0}\right) \right)
    }{
        \bm{v}\!\left(z_{t}\right)^{\!\top}\overline{\mathbf{Q}}_{t}\bm{v}\!\left(z_{0}\right)
    }. \label{eq:q_posterior}
\end{align}
Note that due to the Markov property, $q(z_{t}|z_{t-1}, z_{0})=q(z_{t}|z_{t-1})$.
When we consider $N$-dimensional variables $\bm{z}_{t} \in \{1,2,\ldots,K\}^{N}$, the corruption is applied to each variable $z_{t}$ independently.
In the following, we explain with $N$-dimensional variables $\bm{z}_{t}$.


In contrast to the forward process, the reverse denoising process considers a conditional distribution of $\bm{z}_{t-1}$ over $\bm{z}_{t}$ by a neural network $p_{\theta}(\bm{z}_{t-1}|\bm{z}_{t}) \in [0,1]^{N \times K}$.
$\bm{z}_{t-1}$ is sampled according to this distribution.
Note that the typical implementation is to predict unnormalized log probabilities $\log p_{\theta}(\bm{z}_{t-1}|\bm{z}_{t})$ by a stack of bidirectional Transformer encoder blocks.
D3PM uses a neural network $\tilde{p}_{\theta}(\tilde{\bm{z}}_{0}|\bm{z}_{t})$, combines it with the posterior $q(\bm{z}_{t-1}|\bm{z}_{t},\bm{z}_{0})$, and sums over possible $\tilde{\bm{z}_{0}}$ to obtain the following parameterization:
\begin{equation}
p_{\theta}(\bm{z}_{t-1}|\bm{z}_{t}) \propto \sum_{\tilde{\bm{z}}_{0}}q(\bm{z}_{t-1}|\bm{z}_{t},\tilde{\bm{z}}_{0})~\tilde{p}_{\theta}(\tilde{\bm{z}}_{0}|\bm{z}_{t}). \label{eq:single_step_in_inference}
\end{equation}

In addition to the commonly used variational lower bound objective $\mathcal{L}_\mathrm{vb}$, D3PM introduces an auxiliary denoising objective. The overall objective is as follows:
\begin{equation}
    \mathcal{L}_{\lambda} = \mathcal{L}_\mathrm{vb} + \lambda
    \underset{\substack{ \bm{z}_{t} \sim q(\bm{z}_{t}|\bm{z}_{0}) \\ \bm{z}_{0} \sim q(\bm{z}_{0})}}{\mathbb{E}}
    \left[ -\log \tilde{p}_{\theta}\left(\bm{z}_{0}|\bm{z}_{t}\right) \right], \label{eq:total_loss}
\end{equation}
where $\lambda$ is a hyper-parameter to balance the two loss terms.

Although D3PM proposes many variants of $\mathbf{Q}_{t}$, VQDiffusion~\cite{gu2022vector} offers an improved version of $\mathbf{Q}_{t}$ called mask-and-replace strategy.
They introduce an additional special token \texttt{[MASK]}
and three probabilities $\gamma_{t}$ of replacing the current token with the \texttt{[MASK]} token, $\beta_{t}$ of replacing the token with other tokens, and $\alpha_{t}$ of not changing the token.
The \texttt{[MASK]} token never transitions to other states.
The transition matrix $\mathbf{Q}_{t} \in [0,1]^{(K+1)\times(K+1)}$ is defined by:
\begin{equation}
\mathbf{Q}_{t} = \begin{bmatrix}
\alpha_{t}+\beta_{t} & \beta_{t} & \cdots & \beta_{t} & 0  \\
\beta_{t} & \alpha_{t}+\beta_{t} & \cdots & \beta_{t} & 0  \\
\vdots & \vdots & \ddots & \beta_{t} & 0  \\
\beta_{t} & \beta_{t} & \beta_{t} & \alpha_{t}+\beta_{t} &  0  \\
\gamma_{t} & \gamma_{t} & \gamma_{t} & \gamma_{t} & 1 \\
\end{bmatrix}.
\label{eq:Q_mask_and_replace}
\end{equation}
$(\alpha_{t}, \beta_{t}, \gamma_{t})$ is carefully designed so that $z_{t}$ converges to the \texttt{[MASK]} token for sufficiently large $t$.
During testing, we start from $\bm{z}_{T}$ filled with \texttt{[MASK]} tokens and iteratively sample new set of tokens $\bm{z}_{t-1}$ from $p_{\theta}(\bm{z}_{t-1}|\bm{z}_{t})$.

\subsection{Unconditional Layout Generation}
\label{subsec:layout_diffusion_unconditional}

\begin{figure}[t]
    \centering
    \includegraphics[width=\hsize]{images/overview.pdf}
    \caption{
        Overview of the corruption and denoising processes in LayoutDM.
        For simplicity, we use a toy layout consisting of two elements and the model generates three elements at maximum.
    }
    \label{fig:overview}
\end{figure}

A layout $l$ is a set of elements represented by $l = \left\{\left(c_{1}, \bm{b}_{1}\right), \ldots, \left(c_{E}, \bm{b}_{E}\right) \right\}$. $E \in \mathbb{N}$ is the number of elements in the layout. $c_{i} \in \{1, \ldots, C\}$ is categorical information of the $i$-th element in the layout. $\bm{b}_{i} \in [0,1]^4$ is the bounding box of the $i$-th element in normalized coordinates, where the first two values indicate the center location, and the last two indicate the width and height.
Following previous works~\cite{arroyo2021variational,gupta2021layout,kong2022blt} that regard layout generation as generating a sequence of tokens, we
quantize each value in $\bm{b}_{i}$ and obtain $[x_{i}, y_{i}, w_{i}, h_{i}]^\top \in \{1, \ldots, B\}^4$, where $B$ is a number of the bins. The layout $l$ is now represented by $l = \left\{\left(c_{1}, x_{1}, y_{1}, w_{1}, h_{1}\right), \ldots \right\}$.

In this work, we corrupt a layout in a modality-wise manner in the forward process, and we denoise the corrupted layout while considering all elements and modalities in the reverse process, as we illustrate in \cref{fig:overview}.
Similarly to D3PM~\cite{austin2021structured}, we parameterize $p_{\theta}$ by a Transformer encoder~\cite{vaswani2017attention}, which processes an ordered 1D sequence.
To process $l$ by $p_{\theta}$ while avoiding the order dependency issue~\cite{kong2022blt}, we randomly shuffle $l$ in element-wise manner and then flatten it to produce $l_\mathrm{flat} = (c_{1}, x_{1}, y_{1}, w_{1}, h_{1}, c_{2}, x_{2}, \ldots )$.


\paragraph{Variable length generation}
Existing diffusion models generate fixed-dimensional data and are not directly applicable to the layout generation because the number of elements in each layout varies.
To handle this, we introduce a \texttt{[PAD]} token and define a maximum number of elements in the layout as $M \in \mathbb{N}$.
Each layout is fixed-dimensional data composed of $5M$ tokens by appending $5(M-E)$ \texttt{[PAD]} tokens.
\texttt{[PAD]} is treated similarly to the ordinary token in VQDiffusion and $\mathbf{Q}_{t}$'s dimension becomes $(K+2) \times (K+2)$.

\paragraph{Modality-wise diffusion}
Discrete state-space models assume that all the standard tokens are switchable by corruption. However, layout tokens comprise a disjoint set of token groups for each attribute in the element. For example, applying the transition rule \cref{eq:Q_mask_and_replace} may change a token representing an element's category to another token representing the width.
To avoid such invalid switching, we propose to apply disjoint corruption matrices $\mathbf{Q}_{t}^{c},\mathbf{Q}_{t}^{x},\mathbf{Q}_{t}^{y},\mathbf{Q}_{t}^{w},\mathbf{Q}_{t}^{h}$ for tokens representing different attributes $c, x, y, w, h$, as we show in \cref{fig:overview}. The size of each matrix is $(C+2) \times (C+2)$ for $\mathbf{Q}_{t}^{c}$ and otherwise $(B+2) \times (B+2)$, where $+2$ is for \texttt{[{PAD}]} and \texttt{[{MASK}]}.

\paragraph{Adaptive Quantization}
The distribution of the position and size information in layouts is highly imbalanced; e.g., elements tend to be aligned to either left, center, or right.
Applying uniform quantization to those quantities as in existing layout generation models~\cite{arroyo2021variational,gupta2021layout,kong2022blt} results in the loss of information.
As a pre-processing, we propose to apply a classical clustering algorithm, such as KMeans~\cite{macqueen1967classification} on $x$, $y$, $w$, and $h$ independently to obtain balanced position and size tokens for each dataset.
We show in \cref{sec:ablation_study} how quantization strategy affects the resulting quality.

\paragraph{Decoupled Positional Encoding}
Previous works apply standard positional encoding to a flattened sequence of layout tokens $l_\mathrm{flat}$~\cite{arroyo2021variational,gupta2021layout,kong2022blt}.
We argue that this flattening approach could lose the structure information of the layout and lead to inferior generation performance.
In layout, each token has two types of indices: $i$-th element and $j$-th attribute.
We empirically find that independently applying positional encoding to those indices improves final generation performance, which we study in \cref{sec:ablation_study}.




\subsection{Conditional Generation}
\label{subsec:layout_diffusion_conditional}
We elaborate on solving various conditional layout generation tasks using pre-trained frozen LayoutDM.
We inject conditional information in both the initial state $\bm{z}_{T}$ and sampled states $\{\bm{z}_{t}\}_{t=0}^{T-1}$ during inference but do not modify the denoising network $p_{\theta}$.
The actual implementation of the injection differs by the type of conditions.

\paragraph{Strong Constraints}
The most typical condition is partially known layout fields.
Let $\bm{z}^\mathrm{known} \in \mathbb{Z}^{N}$ contain the known fields and $\bm{m} \in \{0, 1\}^{N}$ be a mask vector denoting the known and unknown field as $1$ and $0$, respectively. In each timestep $t$, we sample $\hat{\bm{z}}_{t-1}$ from $p_{\theta}(\bm{z}_{t-1}|\bm{z}_{t})$ in \cref{eq:single_step_in_inference} and then inject the condition by $\bm{z}_{t-1} = \bm{m} \odot \bm{z}^\mathrm{known} + (\bm{1} - \bm{m}) \odot \hat{\bm{z}}_{t-1}$, where $\bm{1}$ denotes a $N$-dimensional all-ones vector and $\odot$ denotes element-wise product.

\paragraph{Weak Constraints}
We may impose a weaker constraint during generation, such as an element in the center. We offer a way to impose such constraints in a unified framework without additional training or external neural network models.
We propose to adjust the logits to inject weak constraints in log probability space by
\begin{equation}
    \log \hat{p}_{\theta}(\bm{z}_{t-1}|\bm{z}_{t}) \propto \log p_{\theta}(\bm{z}_{t-1}|\bm{z}_{t}) + \lambda_{\pi} \bm{\pi}, \\ \label{eq:prior_addition}
\end{equation}
where $\bm{\pi} \in \mathbb{R}^{N \times K}$ is a prior term that weights the desired outputs, and $\lambda_{\pi} \in \mathbb{R}$ is a hyper-parameter.
The prior term can be defined either hard-coded (Refinement in \cref{sec:quantitative_evaluation}) or through differentiable loss functions (Relationship in \cref{sec:quantitative_evaluation}).
Let $\{\mathcal{L}_i\}_{i=1}^L$ be a set of differentiable loss functions given the prediction, the later prior definition can be written by:
\begin{equation}
    \bm{\pi} = -\nabla_{p_{\theta}(\bm{z}_{t-1}|\bm{z}_{t})} \sum_{i=1}^{L} \mathcal{L}_{i}\left(p_{\theta}\left(\bm{z}_{t-1}|\bm{z}_{t}\right)\right). \\ \label{eq:prior_addition_by_gradient}
\end{equation}
Although the formulation of \cref{eq:prior_addition_by_gradient} resembles steering diffusion models by gradients from external models ~\cite{dhariwal2021diffusion,liu2022compositional}, our primal focus is incorporating classical hand-crafted energies for aesthetics principles of layout~\cite{o2014learning} that do not depend on an external model.
In practice, we tune the hyper-parameters for imposing weak constraints, such as $\lambda_{\pi}$.
Note that these hyper-parameters are only for inference and are easier to tune than the other training hyper-parameters.


\section{Basis orthogonality}
\label{sec:orthogonality}
To explain why we apply orthogonal constraints for extracting an interpretable basis, it is better to individually consider cases where those contraints are absent. To begin with, let's consider the case where two concepts in a concept pair belong to different groups of mutually exclusive concepts. In a slightly informal way where strict linear relation is not considered, this makes the two concepts either independent from each other, or positively correlated, since mutual-exclusivity implies negative correlation. In the first case, it is apparent that the respective basis vectors should be orthogonal. To give a counter example, let's consider the concept \textit{car} from the group of \textit{objects} = \{\textit{car, tree, person}\} and the concept \textit{red} from the group of \textit{colors} = \{\textit{red, green, blue}\}. In case the angle between the basis vectors of these two concepts is less (greater) than $90^{\circ}$, some feature vectors that are classified as \textit{car} will inevitably also be classified as (not) \textit{red} and vice versa (Fig. \ref{fig:concept-directions} - Middle). This relation implies dependence which is contradictory to our initial assumption that the two concepts are independent. While this bias may be encoded in the CNN's weights, this fact also means that the two concepts are not (linearly) disentangled, eventually harming the interpretability of the feature space. Since our primary goal is to search for an interpretable basis, given the previous discussion, we know a-priory that a non-orthogonal basis cannot satisfy the interpretability criteria for independent concepts. 

For the second case, where the two concepts are positively correlated, the two concepts could possibly be related with a \textit{has-a} relationship. For instance, \textit{car} has a \textit{car-door} and a \textit{car-wheel}. In this case, an image patch of the concept \textit{car-door} or \textit{car-wheel} may also be classified as \textit{car}. Vice versa, a representation of a \textit{car} may have positive components in the direction of \textit{car-door} and \textit{car-wheel}, to justify the \textit{has-a} relationship. This case is not handled by the proposed method. However, the primitive concepts, \textit{car-door} and \textit{car-wheel}, are mutually exclusive. 

Thus, for this last case, considering two concepts coming from the same group of mutually-exclusive concepts, it could be reasonable to expect that this mutual exclusivity, which implies negative correlation, is also encoded in the angle between the respective basis vectors. In that case, the angle between the respective basis vectors could be greater than $90^{\circ}$ Fig. \ref{fig:concept-directions} - Right. To investigate the degree that this is possible, we formulate the problem in a way that is independent from input data. To construct an (ideal) basis for negatively correlated concepts, one might consider embedding $I$ concept vectors in a $D$ dimensional space by maximizing the minimum angle across all pairs of vectors. As it turns out this is linked to spherical coding theory \cite{whyte1952uniqueSphereCode} and the tammes problem \cite{tammes1930origin}. Although more sophisticated approaches exist \cite{kottwitz1991densestSphericalCode, Wang2009SphericalCodes}, we tried to approximately solve the tammes problem via directly maximizing the minimum pairwise vector angle with gradient decent. Experimental results showed that the resulting embedding vectors, in cases where $I \ge 64$, are close to orthogonal. Fig \ref{fig:tammes} depicts distribution statistics for various pairs of $I,D$ with $I \leq D$. Conclusively, we argue that an orthogonal basis can cover (under some approximation) independent and mutually exclusive concepts but not concepts that are positively correlated.

\begin{figure}
    \centering
    \includegraphics[width=3.3in]{figs/tammes.png}
    \caption{Pairwise vector angle distribution when solving the tammes problem. Extremas of the error bars correspond to the minimum and maximum vector pair angle. The horizontal line in the box is equals to the mean of the distribution and box widths are equal to the standard deviation.}
    \label{fig:tammes}
\end{figure}

\section{Evaluation Metrics}
\label{sec:evaluation_metrics}
\subsection{Basis labeling and classifier validation scores}
To quantitatively evaluate for interpretability the bases extracted with our method, we use a two step process. First, after deriving a basis, we use %
the work of Bau at al. \cite{bau_NetDissection_CVPR, zhou_NetDissection_PAMI}, to assign a concept label to each classifier associated with the basis vectors. Let $\phi(i,c,\K) \in [0,1]$ denote a metric score function that is used to measure the \textit{suitability} of the classifier $i$ ($\{\w_i,b_i\}$) to accurately detect concept $c$ in the annotated concept dataset $\K$. The concept label that is assigned to classifier $i$ is the one that maximizes $\phi(i,c,\Ktr)$ across $c$ over the training split $\Ktr$ of the concept dataset. Subsequently, in the second step, and using the validation split of the concept dataset $\Kvl$, each classifier is assigned a validation score $\phi(i,\cstar_i,\Kvl)$, with $\cstar_i$ denoting the concept label assigned to the classifier during the first step.  For the choice of $\phi$ we use Intersection Over Union (IoU), as originally proposed in \cite{bau_NetDissection_CVPR} and also used in \cite{mu2020compositional,fong_Net2Vec}:
\begin{equation}
    \phi(i,c,\K) = \frac{\sum_{\k \in \K} |\M^i(\k) \cap \L^c(\k)|}{\sum_{\k \in \K} |\M^i(\k)\cup \L^c(\k)|}
    \label{eq:iou}
\end{equation}
In (\ref{eq:iou}), $\M^i(\k)$ denotes the upsampled, hard-thresholded (binarized) map of image $\k$. $\M^i(\k)$ is obtained by applying the rule of the $i$-th classifier ($\wi^T\xp - b_i > 0)$ to each $\xp$ of the upsampled image's representation. Moreover, $\L^c(\k)$ denotes the ground truth segmentation map of image $\k$ for concept $c$ and $|\cdot|$ denotes the cardinality of a set. %
Overall, to label the bases and compute classifier validation scores, we use the exact scheme of \cite{bau_NetDissection_CVPR} with two differences. First, we consider a train/test split of the concept dataset as originally proposed in \cite{fong_Net2Vec} and second, for hard-thresholding in $\M^i(\k)$, we use the biases learned from our method, instead of using the statistical quantile learning of \cite{bau_NetDissection_CVPR}.

\subsection{Overall basis interpretability scores}
Inspired from \cite{bau_NetDissection_CVPR} and \cite{Losch_Fritz_Schiele_2021} we propose two metrics $\Score1$ and $\Score2$ that can be used to measure the interpretability of a basis. Those metrics, essentially aggregate the aforementioned individual classifier validation scores into scalar values that can summarize the interpretability of a basis. 

The first, counts the number of concept detectors in the basis with a validation score better than a threshold $\xi$. In order to make it threshold agnostic, we measure the area under the indicator function ($\mathbbm{1}(x)$) for all $\xi \in [0,1]$:
\begin{equation}
\label{eq:score1}
\Score{1} = \int_{0}^{1} \sum_{i=0}^{I-1} \mathbbm{1}_{x\ge \xi}\big(\phi(i,c^*_i,\Kvl)\big)d\xi
\end{equation}
\noindent This metric is similar to what was proposed in \cite{Losch_Fritz_Schiele_2021} with two differences. First, we use IoU as the choice of $\phi$ in order to comply with our intention to use \cite{bau_NetDissection_CVPR} for labeling the basis. And second, unlike \cite{Losch_Fritz_Schiele_2021},  we do not normalize (\ref{eq:score1}) with the number of vectors in the basis, in order to be able to make absolute comparisons between scores for bases of different sizes. 

The second metric, counts the number of unique concept labels over the set of labels whose respective concept detectors exhibit performance better than $\xi$. This metric is the same as the one proposed in \cite{bau_NetDissection_CVPR}. Inspired by \cite{Losch_Fritz_Schiele_2021}, and with the intention to also make it agnostic to the threshold $\xi$, we use the area under curve:
\begin{equation}
\label{eq:score2}
\Score{2} = \int_{0}^{1} \psi(\xi)d\xi
\end{equation}
with $\psi(\xi) = |\{\cstar_i \,| \, \exists \, i: \phi(i,\cstar,\Kvl) \ge \xi\}|$, i.e. the number of unique concept detectors exhibiting performance better than $\xi$.









\section{Experimental Results}
\label{sec:experimental_results}
\section{Experimental Results}
\label{sec:experimental_results}

This section describes the experimental validations on the effectiveness and reliability of \ourmodel. First, we describe the model setup in Sec.~\ref{sec:experiment_setups}. Sec.~\ref{sec:single_attr_diagnosis} and Sec.~\ref{sec:validation_diagnosis} visualize and validate the model diagnosis results for the single-attribute setting. In Sec.~\ref{sec:multiple_attr_diagnosis}, we show results on synthesized multiple-attribute counterfactual images and apply them to counterfactual training.

\subsection{Model Setup}
\label{sec:experiment_setups}
{\bf Pre-trained models:} We used Stylegan2-ADA \cite{Karras2020ada} pretrained on FFHQ \cite{2019stylegan} and AFHQ \cite{choi2020starganv2} as our base generative networks, and the pre-trained CLIP model \cite{CLIP}  which is parameterized by ViT-B/32. We followed StyleCLIP \cite{2021StyleCLIP} setups to compute the channel relevance matrices $\mathcal{M}$.

{\bf Target models:} Our classifier models are ResNet50 with single fully-connected head initialized by TorchVision\footnote{https://pytorch.org/blog/how-to-train-state-of-the-art-models-using-torchvision-latest-primitives/}. In training the binary classifiers, we use the Adam optimizer with learning rate 0.001 and batch size 128. We train binary classifiers for \textit{Eyeglasses, Perceived Gender, Mustache, Perceived Age} attributes on CelebA and for \textit{cat/dog} classification on AFHQ. For the 98-keypoint detectors, we used the HR-Net architecture~\cite{WangSCJDZLMTWLX19} on WFLW~\cite{wayne2018lab}. %Unless explicitly mentioned, our approach samples 1000 images from StyleGAN for each diagnosis by histogram.

\subsection{Visual Model Diagnosis: Single-Attribute}
\label{sec:single_attr_diagnosis}
Understanding where deep learning model fails is
an essential step towards building trustworthy models. Our proposed work allows us to generate counterfactual images (Sec.~\ref{sec:Counterfactual_Synthesis}) and provide insights on sensitivities of the target model (Sec.~\ref{sec:Attribute_Sensitivity_Analysis}). This section visualizes the counterfactual images in which only one attribute is modified at a time. 

Fig. \ref{fig:age_classifier_single} shows the single-attribute counterfactual images. Interestingly (but not unexpectedly), 
we can see that reducing the hair length or joyfulness causes the age classifier more likely to label the face to an older person. Note that our approach is extendable to multiple domains, as we change the cat-like pupil to dog-like, the dog-cat classification tends towards the dog. 
Using the counterfactual images, we can conduct model diagnosis with the method mentioned in Sec.~\ref{sec:Attribute_Sensitivity_Analysis}, on which attributes the model is sensitive to. In the histogram generated in model diagnosis, a higher bar means the model is more sensitive toward the corresponding attribute.

Fig.~\ref{fig:histograms_vanilla} shows the model diagnosis histograms on regularly-trained classifiers. For instance, the cat/dog classifier histogram shows outstanding sensitivity to green eyes and vertical pupil.
The analysis is intuitive since these are cat-biased traits rarely observed in dog photos. Moreover, the histogram of eyeglasses classifier shows that the mutation on bushy eyebrows is more influential for flipping the model prediction. 
It potentially reveals the positional correlation between eyeglasses and bushy eyebrows. The advantage of single-attribute model diagnosis is that the score of each attribute in the histogram are independent from other attributes, enabling unambiguous understanding of exact semantics that make the model fail. Diagnosis results for additional target models can be found in Appendix B.

\subsection{Validation of Visual Model Diagnosis} 
\label{sec:validation_diagnosis}
Evaluating whether our zero-shot sensitivity histograms (Fig.~\ref{fig:histograms}) explain the true vulnerability is a difficult task, since we do not have access to a sufficiently large and balanced test set fully annotated in an open-vocabulary setting. To verify the performance, we create synthetically imbalanced cases where the model bias is known. We then compare our results with a supervised diagnosis setting~\cite{sia}. In addition, we will validate the decoupling of the attributes by CLIP. 

\vspace{-2mm}
\subsubsection{Creating imbalanced classifiers}
\label{sec:creating_imbalance_classifiers}
\vspace{-1mm}
In order to evaluate whether our sensitivity histogram is correct, we train classifiers that are highly imbalanced towards a known attribute and see whether \ourmodel can detect the sensitivity w.r.t the attribute. For instance, when training the perceived-age classifier (binarized as Young in CelebA), we created a dataset on which the trained classifier is strongly sensitive to Bangs (hair over forehead). The custom dataset is a CelebA training subset that consists of $20,200$ images. More specifically, there are $10,000$ images that have both young people that have bangs, represented as $(1,1)$, 
and $10,000$ images of people that are not young and have no bangs, represented as $(0,0)$. The remaining combinations of $(1,0)$ and $(0,1)$ have only 100 images.
With this imbalanced dataset, bangs is the attribute that dominantly correlates with whether the person is young, and hence the perceived-age classifier would be highly sensitive towards bangs.
% will learn that bangs is the most sensitive attribute to predict age. 
See Fig.~\ref{fig:histogram_attgan} (the right histograms) for an illustration of the sensitivity histogram computed by our method for the case of an age classifier with bangs (top) and lipstick (bottom) being imbalanced. 
\begin{figure}[t]
    \begin{subfigure}[b]{\linewidth}
        \label{fig:histogram_attgan_1}
         \centering
         \includegraphics[width=\linewidth]{images/histograms/attgan_histogram_1.pdf}\\
    \end{subfigure}
    \begin{subfigure}[b]{\linewidth}
    \label{fig:histogram_attgan_2}
         \includegraphics[width=\linewidth]{images/histograms/attgan_histogram_2.pdf}
    \end{subfigure}
        \vspace{-6mm}
         \caption{ The sensitivity of the age classifier is evaluated with \ourmodel (right) and AttGAN (left), achieving comparable results. }
         \label{fig:histogram_attgan}
         \vspace{-1mm}
    %  \end{subfigure}
\end{figure}

 We trained multiple imbalanced classifiers with this methodology,  and visualize the model diagnosis histograms of these imbalanced classifiers in Fig.~\ref{fig:histograms_unbalanced}. We can observe that the \ourmodel histograms successfully detect the synthetically-made bias, which are shown as the highest bars in histograms. See the caption for more information. 

\begin{figure}[t]
    \begin{subfigure}[b]{0.49\linewidth}
        \centering
        \includegraphics[width=\linewidth]{images/matrix/confusion-matrix-Mustache.pdf}
        \caption{Mustache classifier}
        \label{fig:matrix_CLIP_Score_a}
    \end{subfigure}
    \begin{subfigure}[b]{0.49\linewidth}
        \centering
        \includegraphics[width=\linewidth]{images/matrix/confusion-matrix-Young.pdf}
        \caption{Perceived age classifier}
        \label{fig:matrix_CLIP_Score_b}
    \end{subfigure}
    \vspace{-2mm}
    \caption{Confusion matrix of CLIP score variation (vertical axis) when perturbing attributes (horizontal axis). This shows that attributes in \ourmodel are highly decoupled. }
    \label{fig:matrix_CLIP_Score}
    \vspace{-3mm}
\end{figure}

\begin{figure*}[ht]
    \centering
    \includegraphics[width=\linewidth]{images/multi_attr_human.pdf}
    \caption{Multi-attribute counterfactual in faces. The model probability is documented in the upper right corner of each image.}
    \label{fig:human_classifier_multiattr}
    \vspace{-4mm}
\end{figure*}

\vspace{-2mm}
\subsubsection{Comparison with supervised diagnosis}
\vspace{-1mm}
We also validated our histogram by comparing it with the case in which we have access to a generative model that has been explicitly trained to disentangle attributes.  We follow the work on~\cite{sia} and used AttGAN~\cite{attGAN} trained on the CelebA training set over $15$ attributes\footnote{\textit{Bald, Bangs, Black\_Hair, Blond\_Hair, Brown\_Hair, Bushy\_Eyebrows, Eyeglasses, Male, Mouth\_Slightly\_Open, Mustache, No\_Beard, Pale\_Skin, Young, Smiling, Wearing\_Lipstick.}}.
After the training converged, we performed adversarial learning in the attribute space of AttGAN and create a sensitivity histogram using the same approach as Sec.~\ref{sec:Attribute_Sensitivity_Analysis}. Fig.~\ref{fig:histogram_attgan} shows the result of this method on the perceived-age classifier which is made biased towards bangs.  As anticipated, the AttGAN histogram (left) corroborates the histogram derived from our method (right). Interestingly, unlike \ourmodel, AttGAN show less sensitivity to remaining attributes. This is likely because AttGAN has a latent space learned in a supervised manner and hence attributes are better disentangled than with StyleGAN. Note that AttGAN is trained with a fixed set of attributes; if a new attribute of interest is introduced, the dataset needs to be re-labeled and AttGAN retrained. ZOOM, however, merely calls for the addition of a new text prompt.  More results in Appendix B.

\vspace{-2mm}
\subsubsection{Measuring disentanglement of attributes}
\vspace{-1mm}
Previous works demonstrated that the StyleGAN's latent space can be entangled~\cite{interfacegan, EditinginStyle}, adding undesired dependencies when searching single-attribute counterfactuals. This section verifies that our framework can disentangle the attributes and mostly edit the desirable attributes.

We use CLIP as a super annotator to measure attribute changes during single-attribute modifications. For $1,000$ images, we record the attribute change after performing adversarial learning in each attribute, and average the attribute score change. Fig.~\ref{fig:matrix_CLIP_Score} shows the confusion matrix during single-attribute counterfactual synthesis. The horizontal axis is the attribute being edited during the optimization, and the vertical axis represents the CLIP prediction changed by the process. For instance, the first column of Fig.~\ref{fig:matrix_CLIP_Score_a} is generated when we optimize over bangs for the mustache classifier. We record the CLIP prediction variation. It clearly shows that bangs is the dominant attribute changing during the optimization. From the main diagonal of matrices, it is evident that the \ourmodel mostly perturbs the attribute of interest. The results indicate reasonable disentanglement among attributes.



\subsection{Visual Model Diagnosis: Multi-Attributes}
\label{sec:multiple_attr_diagnosis}
In the previous sections, we have visualized and validated single-attribute model diagnosis histograms and counterfactual images. 
In this section, we will assess \ourmodel's ability to produce counterfactual images by concurrently exploring multiple attributes within $\mathcal{A}$, the domain of user-defined attributes.  The approach conducts multi-attribute counterfactual searches across various edit directions, identifying distinct semantic combinations that result in the target model's failure. By doing so, we can effectively create more powerful counterfactuals images (see Fig.~\ref{fig:multiple_attribute_is_more_powerful}).


\begin{figure}[t]
    \centering
    \includegraphics[width=\linewidth]{images/multi_attr_dog_cut.pdf}
    \caption{Multi-attribute counterfactual on Cat/Dog classifier. The number in each image is the predicted probability of being a dog.}
    \label{fig:dog_classifier_multiattr}
    \vspace{-2mm}
\end{figure}

\begin{figure}[t]
    \centering
    \includegraphics[width=\linewidth]{images/multi_attr_is_more_powerful/multi_attr_is_more_powerful_2.pdf}\\
    \vspace{-1mm}
    \includegraphics[width=\linewidth]{images/multi_attr_is_more_powerful/multi_attr_is_more_powerful_1.pdf}
    \vspace{-8mm}
    \caption{ Multiple-Attribute Counterfactual (MAC, Sec.~\ref{sec:multiple_attr_diagnosis}) compared with Single-Attribute Counterfactual (SAC, Sec.~\ref{sec:single_attr_diagnosis}). We can see that optimization along multiple directions enable the generation of more powerful counterfactuals.}
    \label{fig:multiple_attribute_is_more_powerful}
    \vspace{-4mm}
\end{figure}

Fig.~\ref{fig:human_classifier_multiattr} and Fig.~\ref{fig:dog_classifier_multiattr} show examples of multi-attribute counterfactual
images generated by \ourmodel, against human and animal face classifiers. 
It can be observed that multiple face attributes such as lipsticks or hair color are edited in Fig.~\ref{fig:human_classifier_multiattr}, and various cat/dog attributes like nose pinkness, eye shape, and fur patterns are edited in Fig.~\ref{fig:dog_classifier_multiattr}. 
These attribute edits are blended to affect the target model prediction. Appendix B further illustrates \ourmodel counterfactual images for semantic segmentation, multi-class classification, and a church classifier. By mutating semantic representations, \ourmodel reveals semantic combinations as outliers where the target model underfits.


In the following sections, we 
will use the Flip Rate (the percentage of counterfactuals that flipped the model prediction) and Flip Resistance (the percentage of counterfactuals for which the model successfully withheld its prediction) to evaluate the multi-attribute setting. 
\begin{figure}[t]
    \centering
    \begin{subfigure}[b]{\linewidth}
    \includegraphics[width=0.495\linewidth]{images/histograms/multi_attr_eyeglasses.pdf}
    \includegraphics[width=0.495\linewidth]{images/histograms/multi_attr_age_biased_beard.pdf}
    \caption{Sensitivity histograms generated by \ourmodel on attribute combinations.}
    \label{fig:histograms_combination}
    \end{subfigure}\\
    \begin{subfigure}[b]{\linewidth}
    \includegraphics[width=\linewidth]{images/histograms/grand_histogram.pdf}
    \caption{Model diagnosis by \ourmodel over $19$ attributes. Our framework is generalizable to analyze facial attributes of various domains.}
    \label{fig:histograms_grand}
    \end{subfigure}
    \vspace{-6mm}
    \caption{Customizing attribute space for \ourmodel.}
    \label{fig:multiple_attribute_histogram}
    \vspace{-4mm}
\end{figure}
\vspace{-3mm}
\subsubsection{Customizing attribute space}
\vspace{-2mm}
\looseness=-1

In some circumstances,  users may finish one round of model diagnosis and proceed to another round by adding new attributes, or trying a new attribute space.
The linear nature of attribute editing (Eq.~\ref{eq:total_edit}) in \ourmodel makes it possible to easily add or remove attributes. 
Table~\ref{tab:model_flip_rate} shows the flip rates results when adding new attributes into $\mathcal{A}$ for perceived age classifier and big lips classifier.  We can observe that a different attribute space will results in different effectiveness of counterfactual images. Also, increasing the search iteration will make counterfactual more effective (see last row). 
 Note that neither re-training the StyleGAN nor user-collection/labeling of data is required at any point in this procedure.  Moreover, Fig.~\ref{fig:histograms_combination} shows the model diagnosis histograms generated with combinations of two attributes. Fig.~\ref{fig:histograms_grand} demonstrates the capability of \ourmodel in a rich vocabulary setting where we can analyze attributes that are not labeled in existing datasets~\cite{liu2015celeba,MAAD}.
 
\vspace{-4mm}
\subsubsection{Counterfactual training results}
\label{sec:ct_result}
\vspace{-1mm}


This section evaluates regular classifiers trained on CelebA~\cite{liu2015celeba} and counterfactually-trained (CT) classifiers on a mix of CelebA data and counterfactual images as described in Sec.~\ref{sec:ct}. Table \ref{tab:ct_acc_table} presents accuracy and flip resistance (FR) results. CT outperforms the regular classifier. FR is assessed over 10,000 counterfactual images, with FR-25 and FR-100 denoting Flip Resistance after 25 and 100 optimization iterations, respectively. Both use $\eta=0.2$ and $\epsilon=30$. We can observe that the classifiers after CT are way less likely to be flipped by counterfactual images while maintaining a decent accuracy on the CalebA testset. Our approach robustifies the model by increasing the tolerance toward counterfactuals. Note that CT slightly improves the CelebA classifier when trained on a mixture of CelebA images (original images) and the counterfactual images generated with a generative model  trained in the FFHQ~\cite{2019stylegan} images (different domain).  


\begin{table}[t]
  \centering
  \footnotesize
  \begin{tabular}{@{}lccc@{}}
     \toprule
     Method & \makecell{AC Flip Rate (\%)} & \makecell{BC Flip Rate (\%)} \\
     \midrule
     Initialize \ourmodel by $\mathcal{A}$                        & 61.95 &  83.47\\
     + Attribute: Beard                                           &  72.08 & 90.07\\
     + Attribute: Smiling                                        &  87.47 &  \textbf{96.27}\\
     + Attribute: Lipstick                                         &  90.96 &  94.07\\
     + Iterations increased to 200                                &  \textbf{92.91} &  94.87\\
     \bottomrule
  \end{tabular}
  \caption{\label{tab:model_flip_rate} Model flip rate study. The initial attribute space $\mathcal{A} =$ \{Bangs, Blond Hair, Bushy Eyebrows, Pale Skin, Pointy Nose\}. AC is the perceived age classifier and BC is the big lips classifier.} 
  \vspace{-3mm}
\end{table}


\begin{table}[t]
    \centering
    \footnotesize
    \begin{tabular}{ccccc}
        \toprule
         Attribute & \makecell{Metric} & \makecell{Regular (\%)} & \makecell{CT (Ours) (\%)} \\

\midrule
        \multirow{3}{*}{Perceived Age} & CelebA Accuracy   & 86.10 & \textbf{86.29}   \\
        & \ourmodel FR-25  & 19.54 & \textbf{97.36}  \\
        & \ourmodel FR-100  & 9.04 & \textbf{95.65}  \\
        \midrule
        \multirow{3}{*}{Big Lips} & CelebA Accuracy   & 74.36 & \textbf{75.39}    \\
        & \ourmodel FR-25  & 14.12 & \textbf{99.19}  \\
        & \ourmodel FR-100  & 5.93 & \textbf{88.91}  \\
        \bottomrule
    \end{tabular}
    \caption{\label{tab:ct_acc_table} Results of network inference on CelebA original images and \ourmodel-generated counterfactual. The CT classifier is significantly less prone to be flipped by counterfactual images, while test accuracy on CelebA remains performant.}
    \vspace{-6mm}
\end{table}

\vspace{-2mm}

\section{Conclusion and Discussion} \label{conclusion_and_future}
\looseness=-1
\vspace{-2mm}

In this paper, we present \ourmodel, a zero-shot model diagnosis framework that generates sensitivity histograms based on 
user's input of natural language attributes. 
\ourmodel assesses failures and generates corresponding sensitivity histograms for each attribute.  A significant advantage
of our technique is its ability to analyze the failures of a target deep model without the need for laborious collection and annotation of test sets. \ourmodel effectively visualizes the correlation between attributes and model outputs, elucidating model behaviors and intrinsic biases.

Our work has three primary limitations. First, users should possess domain knowledge as their input (text of attributes of interest) should be relevant to the target domain.  Recall that it is a small price to pay for model evaluation without an annotated test set. Second, StyleGAN2-ADA struggles with generating out-of-domain samples. Nevertheless, our adversarial learning framework can be adapted to other generative models (e.g., stable diffusion), and the generator can be improved by training on more images. We have rigorously tested our generator with various user inputs, confirming its effectiveness for regular diagnosis requests. Currently, we are exploring stable diffusion models to generate a broader range of classes while maintaining the core concept. Finally, we rely on a pre-trained model like CLIP which we presume to be free of bias and capable of encompassing all relevant attributes.

{\bf Acknowledgements: }We would like to thank George Cazenavette, Tianyuan Zhang, Yinong Wang, Hanzhe Hu, Bharath Raj for suggestions in the presentation and experiments. We sincerely thank Ken Ziyu Liu, Jiashun Wang, Bowen Li, and Ce Zheng for revisions to improve this work.
\vspace{-7pt}
\section{Conclusion}
\label{sec:conclsion}
\section{Conclusion}\label{sec:conclusion}
In this work, we focus on addressing the fundamental challenge of OOD detection tasks, which is how to fully understand the semantic discrepancy between the ID/OOD samples. We reveal that the key to success in the realistic SCOOD task is to allocate as many ID samples in the unlabeled set correctly as possible. To this end, we propose a novel uncertainty-aware optimal transport scheme that introduces class-specific energy scores as guidance for effective label assignment. Experimental results show that our method achieves better performance than previous state-of-the-art methods on SCOOD benchmarks.

\textbf{Limitations.} In addition to temperature scaling, other techniques such as feature clipping applied in ReAct~\cite{sun2021react} also enhance the performance of energy score, so how to obtain an OOD score that best fits the SCOOD task can be further explored. Moreover, a setting highly related to SCOOD has been proposed in \cite{katz2022training} and formulated as a constrained optimization problem. We will also theoretically analyze these practical OOD settings in our feature work.

% \section*{Acknowledgments}
\textbf{Acknowledgments.} 
This work is supported by National Key R\&D Program of China under Grant 2020AAA0105701, National Natural Science Foundation of China (NSFC) under Grants 61872327, Major Special Science and Technology Project of Anhui, National Natural Science Foundation of China (62033012) and Ant Group through Ant Research Intern Program.


\begin{figure*}[!h]
    \centering
    \includegraphics[width=0.99\linewidth]{assets/qualitative.jpg}
    \caption{
    Qualitative comparisons on Objaverse.
    We compare our textured mesh against CLIPMesh~\cite{mohammad2022clip}, Text2Mesh~\cite{michel2022text2mesh}, Latent-Paint~\cite{metzer2022latent}, and the original textures from Objaverse. In comparison with the baselines, our method produces more consistent and detailed 3D textures with respect to the input geometries. Image best viewed in color.
    }
    \label{fig:qualitative}
\end{figure*}




\bibliographystyle{IEEEtran}
\bibliography{IEEEabrv,refs.bib}

\begin{IEEEbiography}[{\includegraphics[width=1in,height=1.25in,clip,keepaspectratio]{photos/alex.jpg}}]{Alexandros Doumanoglou}{\space} received the Diploma in
electrical and computer engineering from Aristotle University of Thessaloniki, Thessaloniki, Greece, in 2009 and joined the Information Technologies Institute, in 2012. Currently, he is working toward
the Ph.D. degree in explainable artificial intelligence at the Department of Advanced Computing Sciences of Maastricht University, The Netherlands.
His current research focuses on unsupervised learning and explainable and interpretable methods for deep learning models.
\end{IEEEbiography}

\begin{IEEEbiography}
[{\includegraphics[width=1in,height=1.25in,clip,keepaspectratio]{photos/stelios.png}}]{Stylianos Asteriadis}{\space} received the diploma of Electrical and Computer Engineer from Aristotle University of Thessaloniki, Thessaloniki, Greece in 2004, the M.Sc. degree in digital media from the School of Informatics at the same university in
2006, and the Ph.D. in Electrical and Computer Engineering from the National Technical University of Athens, Athens, Greece, in 2011.
He was an Associate Professor at the Department of Advanced Computing Sciences at Maastricht University, Maastricht, The Netherlands, until the
final acceptance of this paper, where he coordinated the Cognitive Systems
Group. He is currently working at the European Commission.\footnote{The information and views set out in this article are those of the authors and do not necessarily reflect the official opinion of the Institution.} 
\end{IEEEbiography}

\begin{IEEEbiography}[{\includegraphics[width=1in,height=1.25in,clip,keepaspectratio]{photos/dimitris.jpg}}]{Dimitrios Zarpalas}{\space} received the Diploma in electrical and computer engineering from Aristotle University of Thessaloniki (A.U.Th), Thessaloniki, Greece in 2003, the M.Sc. degree in electrical engineering from the Pennsylvania State University, Philadelphia, USA, in 2006, and the Ph.D. degree in medical informatics from the Department of Medicine, Health Science School, A.U.Th, in 2014.
He joined the Information Technologies Institute, Thessaloniki, Greece, in 2007, where he is currently a Researcher, grade B. His research interests include tele-immersion applications, 3-D computer vision, 3-D object recognition, and motion capturing.
\end{IEEEbiography}

\end{document}
