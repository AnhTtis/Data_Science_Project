\vspace{-2mm}
\section{Conclusion}
\vspace{-1mm}
In this work, we propose Natural Language-Assisted Sign Language Recognition (NLA-SLR) framework, which leverages semantic information contained in glosses to promote sign language recognition.
Specifically, we first propose language-aware label smoothing to ease model training by generating soft labels whose smoothing
weights are the normalized semantic similarities.
Second, to maximize the separability of signs with distinct semantic meanings, we propose inter-modality mixup which blends vision and gloss features as well as their labels.
Besides, we also present a novel backbone, video-keypoint network, to model both RGB videos and human body keypoints and to absorb knowledge from sign videos of different temporal receptive fields.
% 
Empirically, our approach surpasses previous best methods on three widely-adopted benchmarks.

\small\noindent \textbf{Acknowledgements.}
The work described in this paper was partially supported by a grant from the Research Grants Council of the HKSAR, China (Project No. HKUST16200118).