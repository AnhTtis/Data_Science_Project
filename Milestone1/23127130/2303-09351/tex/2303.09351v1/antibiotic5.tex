\documentclass[12pt]{article}
%\pdfoutput=1
\usepackage{amsthm,amssymb,amsmath,multirow}
\usepackage{longtable}
%\usepackage[round]{natbib} 
%\usepackage[style=authoryear]{biblatex}

\usepackage[utf8]{inputenc} % allow utf-8 input
\usepackage[T1]{fontenc}    % use 8-bit T1 fonts
\usepackage{hyperref}       % hyperlinks
\usepackage{url}            % simple URL typesetting
\usepackage{booktabs}       % professional-quality tables
\usepackage{amsfonts}       % blackboard math symbols
\usepackage{nicefrac}       % compact symbols for 1/2, etc.
\usepackage{microtype}      % microtypography
%\usepackage{lipsum}		% Can be removed after putting your text content
\usepackage{graphicx}
%\usepackage{natbib}
\usepackage{doi}


%\bibliographystyle{unsrtnat}
\bibliographystyle{abbrvnat}

\numberwithin{equation}{section}

\newtheorem{remar}{Remark}
\newtheorem{example}{Example}
%\newtheorem{def}{Definition}
\newtheorem{corol}{Corollary}
\newtheorem{defi}{Definition}
\newtheorem{prob}{Problem}
\newtheorem{prop}{Proposition}
\newtheorem{theorem}{Theorem}
\newtheorem{lemma}{Lemma}
\newtheorem{claim}{Claim}
\newtheorem{remark}{Remark}
\newtheorem{conj}{Conjecture}

\newcommand{\Real}{\mathbb{R}}
\newcommand{\Integer}{\mathbb{Z}}
\newcommand{\Natural}{\mathbb{N}}
\newcommand{\Complex}{\mathbb{C}}
\newcommand{\Field}{\mathbb{K}}

\title{Optimizing antimicrobial treatment schedules: some fundamental analytical results}

%\date{September 9, 1985}	% Here you can change the date presented in the paper title
%\date{} 					
% Or removing it

\author{Guy Katriel\\ Department of Applied Mathematics,\\ Braude College of Engineering,\\ Karmiel, Israel\\}

% Uncomment to remove the date


% Uncomment to override  the `A preprint' in the header
%\renewcommand{\headeright}{Technical Report}
%\renewcommand{\undertitle}{Technical Report}
%\renewcommand{\shorttitle}{\textit{arXiv} Template}

\date{}

\begin{document}

\maketitle



\begin{abstract}
This work studies fundamental questions 
regarding the optimal design 
of antimicrobial treatment protocols,
using standard pharmacodynamic and pharmacokinetic mathematical models.
We consider the problem of designing
an antimicrobial treatment schedule
to achieve eradication of a microbial 
infection, while minimizing the area under the time-concentration curve ($AUC$).
We first solve this problem under the assumption that an arbitrary antimicrobial concentration profile may be chosen, and prove that the {\it{ideal}} concentration profile
consists of a constant concentration over a finite time duration, where explicit expressions for the optimal 
concentration and the time duration are given in terms of the pharmacodynamic parameters. 
Since antimicrobial concentration profiles are induced by a dosing schedule and the antimicrobial pharmacokinetics, the ideal
concentration profile is 
not strictly feasible. We therefore also 
investigate the possibility of achieving 
outcomes which are close to those 
provided by the ideal concentration profile,
using a bolus+continuous dosing schedule, which consists of 
a loading dose followed by  infusion of the antimicrobial at a constant rate.
We explicitly find the optimal bolus+continuous
dosing schedule, and show that, for realistic parameter ranges, this schedule
achieves results which are nearly as 
efficient as those attained by the ideal
concentration profile. The  optimality results obtained here provide a baseline and reference point for comparison and evaluation of antimicrobial treatment plans.
\end{abstract}


% keyword\theta can be removed
%\keyword\theta{First keyword \and Second keyword \and More}

\section{Introduction}

%general on modelling (mechanism based models)
%\cite{macheras,mueller,rao}
%
%fitting models to experimental data: \cite{bhagunde,kesisoglou,regoes,mouton,nielsen}
%
%use of the model
%\cite{austin,bhagunde,bouvier,corvaisier,goranova,hoyle,kesisoglou, mouton,nielsen,nikolaou1,nikolaou2}
%
%optimizing antibiotic treatment using models
%\cite{ali,cicchese,goranova,hoyle,khan,paterson,pena,smith}
%
%short dosing intervals optimal
%\cite{bouvier}
%
%explicit consideration of resistance
%\cite{ali,geli,goranova,khan,lipsitch,marrec,morsky,nielsen,paterson,singh,zilonova}
%
%explicit consideration of immune response
%\cite{geli,goranova,hoyle,tindall}
%
%consideration of density dependence in 
%bacterial growth
%\cite{ali,bhagunde,geli,kesisoglou,khan,levin,nikolaou1,paterson,tindall,zilonova}
%
%periodic dosing
%\cite{khan,marrec,morsky,nikolaou2,singh}
%
%\vspace{1cm}
%

Antimicrobial agents have made an immense contribution to human welfare, and 
their effective and efficient use is an issue of crucial importance \cite{owens,rotschafer}, in particular in view of the global antimicrobial resistance crisis, which is driven, in part, by 
mis-use or over-use \cite{murray,ventola}.
Mathematical modelling plays an important role in exploring the dynamics of microbial 
growth, antimicrobial pharmacokinetics and
pharmacodynamics \cite{nielsen,vinks}.
Coupled with
experimental laboratory work and clinical studies, mathematical modelling helps to design and evaluate treatment protocols 
and guidelines \cite{bulitta,rao,rayner}.

A traditional and widely-employed 
approach to the quantitative design of antimicrobial 
treatment regimens employs several 
PK/PD indices which quantify exposure 
over a time period, and uses experimental studies to determine the index 
which is maximally correlated to measures of 
efficacy for a particular antimicrobial, with respect to a specific microbial species \cite{onufrak,owens,vinks}. A different
methodology, known as `mechanism-based' or `semi-mechanistic' modelling \cite{bouvier,czock,mueller,nielsen,rao} relies on modelling the full time-course of treatment using dynamic
models which describe the time dependence of both the microbial population and the antimicrobial agent's concentration, most often using differential equations. Such models include both pharmacokinetic 
parameters, related to drug distribution and elimination, and pharmacodynamic parameters, 
related to antimicrobial effect on the microbial population, and these parameters are estimated by fitting models to experimental data \cite{bhagunde,czock,kesisoglou,mouton,nielsen,regoes,wen}. Once such a model is calibrated and validated,
it can serve as an {\it{in-silico}} experimental system, allowing to test the outcomes of a variety of treatment schedules. 

The availability of mechanism-based  models raises the prospect of systematic 
determination of {\it{optimal}} treatment plans,
using mathematical and computational approaches, and indeed several researchers have undertaken such investigations. The dynamical models used, as well as the class of candidate treatment schedules considered and the quantities targeted for optimization, vary among different works. Computationally intensive methods are used for the purpose of finding the optimal schedules, including optimal control methods \cite{ali,khan,Leszczynski,pena,zilonova}, genetic algorithms \cite{cicchese,colin,goranova,hoyle,paterson} and machine learning \cite{smith}. 
While such computational work is very valuable
 and has the advantage of enabling the study 
 of relatively elaborate models, it is also important to approach antimicrobial treatment 
 optimization from an analytical point of view, with the aim of obtaining general insights 
 and mathematical results.  Theoretical 
 work of this type is, to the best of our knowledge, lacking in the literature, and the aim of this work
 is to take some steps to close this gap. 
By employing simple, but widely used, mathematical models, and by formulating  natural optimization problems, we can mathematically prove several general
results characterizing optimal treatment plans. An analytic approach provides generic results which are valid for {\it{all}} parameter values of a model, rather than for specific sets of parameters as in numerical studies, and 
enables to obtain useful explicit formulas for determining 
 the quantitites characterizing the optimal treatment regimens.
The results yield fundamental 
understanding of the problem of optimal treatment
with an antimicrobial agent. To the extent that the (standard) mathematical models used here capture the dynamics of microbial growth and the effect of antimicrobials, 
the results offer practical guidelines for the 
design of antimicrobial treatment schedules, as 
will be discussed.

We now provide an overview of the contributions presented 
in this paper, referring to the corresponding 
sections for details.

We formulate and address some 
key issues regarding optimal 
antimicrobial treatment. Stated simply (and to be formulated more precisely below), our question is: 

\begin{itemize}
\item[] Given that we aim to use an antimicrobial agent to eradicate a microbial infection, what is the treatment plan that will do so using a minimal cumulative dosage of antimicrobial? 
\end{itemize}

The essential tradeoff underlying this optimization 
problem is that, while a low concentration of 
antimicrobial will be insufficient to suppress 
microbial growth, a very high concentration will 
be wasteful due to the saturation of the 
antimicrobial effect at high concentrations. To quantitatively illuminate this tradeoff, we use a standard pharmacodynamic model describing the growth of a microbial population and a killing  rate of microbes depending on the antimicrobial concentration, see section \ref{themodel}. In the context of such a model, the microbial population size cannot reach $0$, so 
that `eradication' is defined in terms of reducing the microbial population size by a given factor, which will depend on the initial microbial  population size.

Our work consists of two parts, in which we address the above question on two levels. In our first investigation (Sections \ref{optall}-\ref{hilf}) we allow an arbitrary time-dependence of the antimicrobial concentration at the infection site, and seek to find, among those concentration profiles which lead to eradication of the microbial population, the one for which the area under the time-concentration curve ($AUC$) 
is minimal. The $AUC$ is a standard measure for the overall exposure \cite{nielsen}, and indeed it is proportional to the cumulative antimicrobial dosage (see equation \ref{aucd})).
 In this formulation of the problem, we are focusing on the pharmacodynamics, ignoring the fact that not every concentration profile is {\it{pharamacokinetically feasible}}, in the sense that it can be induced by an appropriate dosing schedule - these pharmacokinetic aspects are addressed in the second part of the paper. In this general context, we obtain several results:  
\begin{itemize}
	\item[(i)] We prove that the optimal antimicrobial concentration profile
	consists of a {\it{constant}} concentration $c_{opt}$ applied for a finite time duration $T_{opt}$.
	\item[(ii)] We find an algebraic equation which allows us to determine the values $c_{opt}$ and $T_{opt}$. In the case that the pharmacodynamics is described by a Hill function (the most commonly employed pharmacodynamic model) we solve this equation
	to obtain explicit expressions for $c_{opt}$ and $T_{opt}$ in terms of the pharmacodynamic parameters and the initial 
	size of the microbial population (see section \ref{hilf}).
	\item[(iii)] We show that the optimal antimicrobial concentration $c_{opt}$ is independent of the initial size of the microbial population, which only affects the duration $T_{opt}$ of the optimal treatment.
\end{itemize}
The above results establish a {\it{baseline}} in the sense that they provide a lower bound for the $AUC$
needed to achieve eradication. However, 
this analysis focuses only on the pharamacodynamics,
that is the drug effect, ignoring the limitations 
on the concentration profile induced by pharmacokinetics - the dynamics of drug absorption and
elimination.
The  `ideal' concentration curve which achieves the lower bound, consisting of a constant concentation value over a
finite time-interval, is not strictly achievable 
in practice, due to the simple fact that drug 
concentration cannot drop to $0$ in a single instant, but rather decays in a gradual way.
Therefore, in section \ref{pharmacokinetic}, we address the question of achieving efficient treatment using 
drug concentration profiles which are {\it{pharmacokinetically feasible}}. 
We would like to achieve results which are close to
the `ideal' baseline determined in the first part 
of this work, but which can be realistically attained by a dosage plan, preferably one that is simple to implement.
In this work we restrict ourselves to a simple one-compartment pharmacokinetic model - leaving 
consideration of more complex pharmacokinetics to 
future work. The only pharmacokinetic parameter
is thus the rate of drug decay.
 In this context, we examine simple dosage 
plans of the {\it{bolus+continuous}} type \cite{derendorf}, in which a single (bolus) dose of the antimicrobial is given at the initiation of treatment, in order to instantaneously raise the drug concentration 
to a level $\bar{c}$, and constant-rate infusion is provided thereafter, for a time duration $T_{bc}$, in order to maintain the same concentration. The 
initial dose and the constant rate of infusion are 
determined by the desired concentration and the
pharmacokinetic parameter (rate of drug decay).
 This choice of dosing schedule mimicks 
the `ideal' concentration profile in that the concentration is constant for a finite duration, but with exponential decay therafter.
Optimizing over all 
such dosing schedules (that is over all choices of $\bar{c}$ and $T_{bc}$) which achieve eradication, with the aim of minimizing the $AUC$ (which is equivalent to minimizing total dosage), we find the following:
\begin{itemize}
\item[(i)] The optimal concentration $\bar{c}$ is (somewhat surprisingly) identical to the value $c_{opt}$ obtained 
for the `ideal' concentration profile in the first part of our work. In particular, it does not depend on the pharamacokinetics, that is on the rate of decay of the antimicrobial.

\item[(ii)] The optimal time duration $T_{bc,opt}$ over which 
the constant-rate infusion of drug should be performed is 
given by an explicit formula, and depends both on the pharmacodynamics and on the rate of decay of the antimicrobial. This duration is always shorter than the duration $T_{opt}$ of the `ideal' concentration profile. 

\item[(iii)] If the antimicrobial decay rate is sufficiently small, then $T_{bc}=0$, that is the optimal bolus+continuous schedule consists only of a bolus dose, and if the 
antimicrobial decay rate is large, then $T_{bc}$ is close to $T_{opt}$, and the 
$AUC$ corresponding to the optimal bolus+continuous treatment is close to (though somewhat higher than) the $AUC$ of the `ideal' concentration profile.
\end{itemize}

Numerical results given in section \ref{bchill}, computed for the case of a Hill-type pharmacodynamic function, with realistic ranges of values of the pharmacokinetic and pharmacodynamic parameters, show that, in most cases, the optimal
bolus+continuous dosing schedule achieves results which are 
nearly as efficient as those attained using the `ideal' concentration profile, in that the $AUC$ valued attained is not significantly higher. We thus conclude that a bolus+continuous schedule, suitably designed,
provides a nearly-optimal solution
under many circumstances.

While the results obtained here provide what we 
believe to be an essential theoretical basis 
for thinking about the optimization of antimicrobial treatments, 
there are various complicating issues that should be taken into account in considering the application of these results in concrete settings. In Section \ref{discussion} we address some of the limitations of the 
standard modelling framework employed in this work, and suggest directions for further investigation.



\section{The pharmacodynamic model}
\label{themodel}

In this section we describe the
modelling framwork which will be employed to 
study antimicrobial treatment schedules, which is standard 
in the field of pharmacodynamics \cite{austin,bhagunde,bouvier,corvaisier,goranova,hoyle,kesisoglou, mouton,nielsen,nikolaou1,nikolaou2}.
The notation to be used is summarized in Table 1.

An antimicrobial treatment schedule will determine a function $C(t)$ ($t\geq 0$) describing the concentration 
of antimicrobial at the infection site as a function of time $t$, which we will call the {\it{concentration profile}}. We allow 
$C(t)$ to be an arbitrary non-negative function in the class  $L^1[0,\infty)$ of integrable functions.

The 
area under the concentration curve
\begin{equation}\label{auc}AUC=\int_0^\infty C(t)dt,\end{equation}
is a standard measure of the intensity of 
the antimicrobial treatment. Indeed it may be seen that the $AUC$ is proportional to 
the cumulative dosage of the antimicrobial
supplied, see equation \eqref{aucd} in section
\ref{pharmacokinetic}.

Denoting by $B(t)$ the size of the microbial population at time $t$, we use the standard constant-rate model 
of microbial growth in the absence of treatment 
$$\frac{dB}{dt}=rB,$$
where $r$ is the difference of the replication rate and the natural death rate, leading to exponential growth, with doubling time
\begin{equation}\label{dtime}T_2=\frac{\ln(2)}{r}.
\end{equation}
The antimicrobial effect is modelled using a function $k(c)$, known as the pharmacodynamic function \cite{regoes}, or the kill curve \cite{mueller}, which describes the kill-rate of the antimicrobial agent at concentration $c$. In the presence of antimicrobial, the microbial population is thus described by
\begin{equation}\label{model}\frac{dB}{dt}=[r-k(C(t))]B,\end{equation}
with solution
\begin{equation}\label{sol}B(t)=B_0\cdot e^{\int_0^t [r-k(C(s))]ds},\end{equation}
where $B_0$ is the initial microbial population size at time $t=0$.

The kill-rate function $k(c)$ will be assumed to have the following properties
\begin{itemize}
	\item[(A1)] $k(0)=0$, and $k(c)$ is continuous and monotone increasing on $[0,\infty)$, and twice differentiable for $c>0$.
	
	\item[(A2)] The kill rate 
	saturates at high 
	concentrations:
	\begin{equation}\label{maxkill}\lim_{c\rightarrow\infty} k(c)=k_{max}<\infty.\end{equation}
\end{itemize}	
We will also make one of the following two assumptions regarding 
the shape of the function $k(c)$: 
\begin{defi}\label{dsig}
(i)	$k(c)$ will be said to be {\bf{concave}} if $k''(c)<0$ for all $c\geq 0$.

(ii) $k(c)$ will be said to be
 {\bf{sigmoidal}} if there exists a value $c_{infl}>0$ (the inflection point) so that 
	$$0\leq c<c_{infl} \;\;\Rightarrow\;\; k''(c)>0\;\;\;{\mbox{and}}\;\;\;c>c_{infl}\;\;\Rightarrow\;\; k''(c)<0.$$
\end{defi}

\begin{example}
The most common functional form used for the pharmacodynamic function is the
Hill function (also known as the 
Sigmoid Emax model \cite{meibohm} or the Zhi model \cite{corvaisier,zhi}) 
\begin{equation}\label{hill}k_H(c)=k_{max}\cdot\frac{c^{\gamma}}{C_{50}^{\gamma}+c^{\gamma}},\end{equation}
where the $\gamma>0$ is called the Hill exponent and
$C_{50}$ is the half-saturation constant, the
concentration at which the kill rate is half 
of the maximal value $k_{max}$.
In the survey \cite{czock} one may find 
tables with estimates of the parameters $C_{50},k_{max},\gamma$ for various combinations 
of antimicrobials and microbial species, obtained through many empirical studies.

The function $k_H(c)$ is concave 
if $\gamma\leq 1$, and sigmoidal if $\gamma>1$, in which case the inflection point is given by
$$c_{infl}=C_{50}\cdot \left(\frac{\gamma-1}{\gamma+1} \right)^{\frac{1}{\gamma}}.$$
Our main results do not depend on this 
specific functional form, but we will 
apply the general results to this specific example, and obtain useful
explicit expressions - see in particular section \ref{hilf}.
\end{example}

It will be useful to introduce the dimensionless parameter 
\begin{equation}\label{dalpha}
\alpha=\frac{k_{max}}{r},
\end{equation}
measuring the maximal kill-rate of the 
antimicrobial relative to the natural 
microbial growth rate, which we will therefore
call the {\it{potency}} of the antimicrobial with 
respect to a microbial species.
We will make the standing assumption that
\begin{equation}\label{kill}
\alpha>1,
\end{equation}
that is $k_{max}>r$, which means that a sufficiently large concentration will lead to a negative net growth rate of the microbial population - if this is 
not the case then the antimicrobial is not effective. Under this assumption, and in view of the assumption (A1) above, there exists a unique value of $c$, denoted by $zMIC$, the pharmacodynamic minimal inhibitory concentration \cite{bouvier,corvaisier}, also referred to as the stationary concentration (SC) \cite{mouton,czock}, such that
\begin{equation}\label{dzMIC}k(zMIC)=r.\end{equation}

\begin{example}
In the case that the kill-rate is given by 
a Hill function \eqref{hill}, we have
\begin{equation}\label{zMIC}zMIC=C_{50}\cdot \left(\alpha-1\right)^{-\frac{1}{\gamma}}.\end{equation}
\end{example}

Note that within the framework of model \eqref{model} the microbial population cannot be reduced to $0$, since $B(t)$
given by \eqref{sol} is always positive, and indeed if $AUC$ is finite then
for large time the microbial population will recover, with $B(t)\rightarrow +\infty$ as $t\rightarrow \infty$. However,
in practice, reaching a sufficiently low value of $B(t)$ at some point in time, {\it{e.g.}} corresponding to less than one organism, implies eradication. 
The appropriate measure for the success of treatment is therefore the {\it{maximal}} reduction in the size of the microbial population achieved at some time. The reduction is standardly expressed on a logarithmic scale, by 
defining the log-reduction at time $T$
$$LR(T)=\log_{10}\left(\frac{B_0}{B(T)} \right)=\frac{1}{\ln(10)}\cdot \ln \left(\frac{B_0}{B(T)} \right),$$
(we use base $10$ in order to be consistent with the literature)
which in view of \eqref{sol} is given by 
\begin{equation}\label{LR}LR(T)=\frac{1}{\ln(10)}\cdot \int_0^T [k(C(t))-r]dt.\end{equation}
The {\it{maximal}} reduction afforded by
the concentration profile $C(t)$ is then
\begin{equation}\label{lrs}LR_{max}=LR_{max}[C]=\max_{T\geq 0} LR(T).\end{equation}
We note that, since $LR(0)=0$ and $\lim_{T\rightarrow\infty}LR(T)=-\infty$, the maximum in \eqref{lrs} certainly exists.

Eradication of the infection thus 
corresponds to achieving $LR_{max}[C]\geq LR_{target}$, where 
the value of $LR_{target}$ is given.


\begin{longtable}{p{0.2\textwidth}p{0.8\textwidth}}
	\label{notation}
%	\begin{tabular}
		Symbol & Description\\
		\hline
		$B(t)$ & Microbial population size at time $t$. \\
		$B_0$ & Initial microbial population size.\\
		$C(t)$ & Antimicrobial concentration at time $t$. \\
		$AUC$ & Area under the time-concentration curve $C(t)$, see \eqref{auc}.\\
		$r$ & Microbial growth rate in absence of antimicrobial. \\
		$T_2=\frac{\ln(2)}{r}$ & Microbial doubling time in absence of antimicrobial.\\
		$k(c)$ & Kill-rate of antimicrobial, in dependence on its concentration.\\
		$k_{max}$ & Maximal kill-rate of antimicrobial, see  \eqref{maxkill}.\\
		$\alpha=\frac{k_{max}}{r}$ & Antimicrobial potency relative to a microbial species.\\
		$c_{infl}$ & Inflection point of $k(c)$, in the sigmoidal case.\\
		$zMIC$ & Pharmacodynamic minimal inhibitory concentration, see  \eqref{dzMIC}.\\
		$k_H(c)$ & Hill model for kill-rate, see  \eqref{hill}\\
		$\gamma$ & Hill exponent, see \eqref{hill}\\
		$C_{50}$ & Half saturation constant for Hill model, see \eqref{hill}\\
		$LR(T)$ & log (base $10$) reduction of microbial population at time $T$, see \eqref{LR}.\\
		$LR_{max}[C]$ & Maximal log reduction of microbial population corresponding to a given concentration profile, see \eqref{lrs}.\\
		$LR_{target}$ & Target log reduction on microbial 
		population for achieving eradication.\\
		$LR_{opt}$ & Maximal log reduction attainable by a concentration profile with given $AUC$, given by \eqref{lropt}.\\
		$C_{opt}(t)$ & Optimal antimicrobial concentration profile, see Theorem \ref{mainr}.\\
		$c_{opt}$ & Optimal antimicrobial concentration, see Theorem \ref{mainr}.\\
		$T_{opt}$ & Duration of optimal concentration profile, see \eqref{topte}.\\
		$AUC_{opt}$ & Minimal $AUC$ attainable by any antimicrobial concetration profile, given by \eqref{aucopt}.\\
		$\kappa$ & Antimicrobial decay rate.\\
		$\tau=\frac{\ln(2)}{\kappa}$ & Antimicrobial half-life.\\
		$d(t)$ & Antimicrobial dosing rate at time $t$.\\
		$d_{bc}(t)$ & bolus+continuous dosing schedule, see \eqref{ds}.\\
		$d_{bc,opt}(t)$ & optimal bolus+continuous dosing schedule, see \eqref{ds1}.\\
		$T_{bc}$ & Duration of dosing for a bolus+continuous schedule.\\
		$T_{bc,opt}$ & Duration of the optimal dosing of bolus+continuous type, see \eqref{bart}.\\
		\hline\\
%	\end{tabular}
	\caption{Notation}\label{tab:parameters}
\end{longtable}


\section{Optimizing treatment: the `ideal' concentration profile}
\label{optall}

Our aim is to choose a concentration profile $C(t)$, among {\it{all}} non-negative integrable functions on $[0,\infty)$, so as to minimize the $AUC$, while achieving a specified  log-reduction of the microbial load. We therefore formulate:

\begin{prob}\label{prmain}
Given a target value $LR_{target}$, find a concentration profile $C(t)$ achieving $LR_{max}[C]=LR_{target}$, where $LR_{max}$ is given by \eqref{lrs}, with
the corresponding 
$AUC$ (given by \eqref{auc}) as {\it{small}} as possible.
\end{prob}

The following theorem provides a complete 
solution to this problem.

\begin{theorem}\label{mainr}
	Assume a value $LR_{target}$ is given. The unique solution of Problem \ref{prmain} is given by the 
	concentration profile 
		\begin{equation}\label{copt1}C_{opt}(t)=\begin{cases}
			c_{opt} & 0\leq t\leq T_{opt}\\
			0 & t>T_{opt}
		\end{cases},\;\;\;\;\end{equation}
where $c_{opt}$ is the unique solution of the equation
\begin{equation}\label{copte}k'(c)=\frac{k(c)-r}{c},\end{equation}
and 
\begin{equation}\label{topte}T_{opt}= \frac{\ln(10)\cdot LR_{target}}{k(c_{opt})-r},\end{equation}
giving the minimal value of $AUC$
\begin{equation}\label{aucopt}AUC_{opt}=T_{opt}\cdot c_{opt}=\ln(10)\cdot LR_{target}\cdot \frac{c_{opt}}{k(c_{opt})-r}.\end{equation}
\end{theorem}

Several notable consequences emerge from  Theorem \ref{mainr}:

\begin{itemize}
	\item[(i)] The `ideal' dosing strategy is to keep 
	the concentration of antimicrobial {\it{constant}} at $c_{opt}$ for the duration of time determined by
	$T_{opt}$. 
	
	\item[(ii)] By \eqref{copte}, $c_{opt}$ depends only on the microbial growth rate $r$ and on the 
	kill-rate function $k(c)$, and not on the target reduction $LR_{target}$. Thus the given target reduction affects the optimal schedule only through its effect on the duration $T_{opt}$, which, by \eqref{topte}, depends linearly on $LR_{target}$.
	
	\item[(iii)] By \eqref{aucopt}, the minimal achievable $AUC$ depends linearly on the target reduction
	$LR_{target}$.
\end{itemize}

We note that equation \eqref{copte} leads to a graphical construction for 
obtaining the optimal concentration $c_{opt}$ - see figure \ref{graphic}.

\begin{figure}
	\begin{center}
		\includegraphics[width=0.8\linewidth]{graphic.eps}
	\end{center}
	\caption{Graphical construction of the optimal concentration $c_{opt}$. Plot the straight line which passes through the point $(0,r)$ and is tangent to the curve $k(c)$. The abscissa at the tangency point is the value $c_{opt}$. In this plot we use the Hill curve with $\gamma=3,C_{50}=1,k_{max}=1,r=0.4$.}
	\label{graphic}
\end{figure}
In section \ref{hilf} the results of Theorem \ref{mainr} will be 
applied to the standard
Hill-type form of the killing curve, leading to explicit expressions for 
$c_{opt}$, $T_{opt}$.

We have formulated our problem and solution in 
terms of minimizing the $AUC$ subject to achieving a given log-reduction $LR_{target}$, but an equivalent problem is to maximize the log-reduction $LR_{max}$ 
subject to a given $AUC$ - since $AUC$ is proportional to the total dosage this problem will arise if we want to maximize the efficacy of a given total dose of the antimicrobial. We can 
re-formulate the result of Theorem \ref{mainr} as follows:
\begin{corol}
Given a value $AUC$, the concentration profile $C(t)$ satisfying
$AUC[C]=AUC$ and 
inducing the maximal log-reduction $LR_{max}[C]$ is given by \eqref{copt1}, where $c_{opt}$ is the 
unique solution of \eqref{copte}, and 
$$T_{opt}=\frac{AUC}{c_{opt}}.$$
The value of $LR_{max}$ attained by this optimal profile is
\begin{equation}\label{lropt}LR_{opt}=\frac{1}{\ln(10)}\cdot \frac{k(c_{opt})-r}{c_{opt}}\cdot AUC.
\end{equation}
\end{corol}

The rest of this section is devoted to the proof of Theorem \ref{mainr}.

As a first step in the analysis leading to Theorem \ref{mainr}, we consider only the specific class of concentration profiles which 
take a constant value $c$ for a 
duration $T_{f}$ (the final time), that is:
\begin{equation}C(t)=\label{cp}\begin{cases}
	c & 0\leq t\leq T_f\\
	0 & t>T_f
\end{cases}\end{equation}
Among these profiles, we will now find the one which achieves
the log-reduction $LR_{target}$ with minimal $AUC$. Later we will prove that the resulting concentration
profile is in fact optimal among
{\it{all}} concentration profiles
satifying $LR_{max}[C]=LR_{target}$.

The log-reduction up to an arbitrary time $T\geq 0$,
corresponding the concentration profile \eqref{cp}, is (see \eqref{LR}) $$LR(T)=\frac{1}{\ln(10)}\cdot \int_0^T [k(C(t))-r]dt=\frac{1}{\ln(10)}\cdot\begin{cases}
	T\cdot (k(c)-r) & T\leq T_{f}\\
	T_{f}\cdot k(c)-T\cdot r & T>T_{f}
\end{cases},$$
which is increasing for $T\leq T_{f}$ and decreasing for $T\geq T_{f}$,
hence
$$LR_{max}=\max_{T\geq 0}LR(T)=LR(T_{f})=\frac{1}{\ln(10)}\cdot T_{f}\cdot (k(c)-r).$$
Therefore to achieve a target 
log-reduction $LR_{target}$ using the constant 
concentration $c$ we need to choose
\begin{equation}\label{tf}
	T_f=\frac{\ln(10)\cdot LR_{target}}{k(c)-r}.\end{equation}
The AUC corresponding to this profile is 
$$AUC=T_f\cdot c=\ln(10)\cdot LR_{target}\cdot \frac{c}{k(c)-r}.$$
Thus to minimize the $AUC$ we need to 
maximize the function $f_r:(0,\infty)\rightarrow \Real$ defined by
\begin{equation}\label{deff}f_r(c)=\frac{k(c)-r}{c}\end{equation}
over all $c>0$. The existence of this maximizer, and 
the fact that it is the unique critical point of $f_r$, is shown in the following Lemma:
\begin{lemma}\label{fmax}
	Assume $0\leq r<k_{max}$.
	The function $f_r$ defined by \eqref{deff}
	has a unique critical point $c_{opt}$,
	that is a value satisfying
	\begin{equation}\label{copt0}f_r'(c)=0\;\;\Leftrightarrow\;\; k'(c)=\frac{k(c)-r}{c},\end{equation}
	which is a global maximizer of $f_r$.
	
	In the sigmoidal case (see definition \ref{dsig}), we always have $c_{opt}>c_{infl}$, where $c_{infl}$ is the inflection point of $k(c)$.
\end{lemma}

\begin{proof}[Proof of Lemma \ref{fmax}]
	We have, for any $r>0$,
	$$c<zMIC\;\;\Rightarrow\;\; f_r(c)<0,\;\;\;c>zMIC\;\;\Rightarrow\;\; f_r(c)>0,$$
	$$\lim_{c\rightarrow+\infty}f_r(c)=0.$$
	These facts imply that 
	$f_r(c)$ has a global maximizer 
	on $(0,\infty)$, which we denote by
	$c_{opt}$. It remains to show that
	$c_{opt}$ is the unique critical point of $f_r(c)$.
	We have
	\begin{equation}\label{df}f_r'(c)=\frac{k'(c)-\frac{k(c)}{c}}{c}+\frac{r}{c^2}
		=\frac{h(c)+r}{c^2}
	\end{equation}
	where 
	\begin{equation}\label{defh}h(c)=ck'(c)-k(c),\end{equation}
	so any critical point of $f_r(c)$ 
	satisfies $h(c)=-r$.
	Note that 
	\begin{equation}\label{hd}h'(c)=ck''(c),\end{equation}
	hence:
	
	(a) if $k$ is concave then 
	$h'(c)<0$, so $h(c)$ is decreasing in $(0,\infty)$, hence the critical point $c_{opt}$ of $f_r(c)$ is unique. 
	
	(b) If 
	$k$ is sigmoidal, then $h'(c)<0$ for $c>c_{infl}$, hence $h(c)$ is 
	decreasing in this range, so that
	$f_r(c)$ has at most one critical point 
	in the interval $[c_{infl},\infty)$.
	To prove uniqueness it therefore suffices to show that $f_r(c)$ has no
	critical point in $[0,c_{infl}]$.
	But note that since
	$k(c)$ is convex on 
	$[0,c_{infl}]$, hence $k'(c)$ is increasing on this interval, we have
	\begin{equation}\label{ii}c\in (0,c_{infl}]\;\;\Rightarrow\;\;k'(c)=\frac{1}{c}\int_0^c k'(c)du>\frac{1}{c}\int_0^c k'(u)du=\frac{k(c)}{c},\end{equation}
	hence, by \eqref{df}, $f_r'(c)>0$ for
	$c\in (0,c_{infl}]$. Note that this also shows that $c_{opt}>c_{infl}$.
\end{proof}

The above considerations show that the optimal
concentration profile, among those of the form 
\eqref{cp} which achieve $LR_{target}$, is given by \eqref{copt1}, with 
$c_{opt}$ given by \eqref{copte}, and $T_{opt}$ given by \eqref{topte}.

We now show that the profile
$C_{opt}(t)$ is in fact optimal among
{\it{all}} concentration profiles which achieve
$LR_{target}$, and is thus the solution to 
Problem \ref{prmain}:

\begin{proof}[Proof of Theorem \ref{mainr}:]
	Let $C(t)$ be {\it{any}} concentration profile with
	$LR_{max}[C]=LR_{target}$, and let $T^*\geq 0$
	be the value for which $LR(T^*)=LR_{max}[C]$. We then have
	\begin{eqnarray}\label{ki1}\ln(10)\cdot LR_{target}&=&\ln(10)\cdot LR(T)= \int_0^{T^*} C(t)\cdot \frac{k(C(t))-r}{C(t)}dt\\&\leq& \left[\max_{c}\frac{k(c)-r}{c}\right]\cdot \int_0^{T^*} C(t)dt\nonumber\\&\leq&  \left[\max_{c}\frac{k(c)-r}{c}\right]\cdot \int_0^\infty C(t)dt= \frac{k(c_{opt})-r}{c_{opt}}\cdot AUC\nonumber\end{eqnarray}
	so that
	\begin{equation}\label{ki2}AUC\geq\ln(10)\cdot\frac{c_{opt}\cdot LR_{target}}{k(c_{opt})-r}= AUC_{opt},\end{equation}
	so we see that $AUC$ cannot be made 
	smaller than the value \eqref{aucopt} obtained by taking the concentration profile $C_{opt}(t)$. Therefore $C_{opt}(t)$ is a minimizer. 
	
	To show uniqueness of the 
	minimizer, note that if we have
	equality in \eqref{ki2}, hence in 
	\eqref{ki1}, then it must be the case that  $$\frac{k(C(t))-r}{C(t)}=\max_{c}\frac{k(c)-r}{c},$$
	implying that $C(t)=c_{opt}$, for almost every $t\in [0,T^*]$, as well as that
	$$\int_0^{T^*} C(t)dt=\int_0^\infty C(t)dt,$$
	implying that $C(t)=0$ for a.e. 
	$t\geq T^*$. We therefore have
	$$LR_{max}[C]=\frac{1}{\ln(10)}\cdot T^*\cdot  (k(c_{opt})-r),$$ implying that
	$$T^*=\frac{\ln(10)LR_{target}}{k(c_{opt})-r}=T_{opt}.$$
	We have thus shown that $C(t)=C_{opt}(t)$ for {\it{a.e.}} $t$, establishing uniqueness.
\end{proof}


\section{Application to Hill-type pharmacodynamic  functions}
\label{hilf}

We now specialize the results to the case that the kill-rate function  is the 
Hill function $k_H(c)$ defined by \eqref{hill}, which 
allows us to obtain explicit expressions for 
the quantities of interest. We use 
these expressions to study the 
dependence of $c_{opt}$ and 
$T_{opt}$ on the relevant parameters.

In the case of a Hill-type pharmacodynamic function \eqref{deff} gives
$$f_r(c)=\frac{1}{c}\left(k_{max}\cdot\frac{c^{\gamma}}{C_{50}^{\gamma}+c^{\gamma}}-r\right)
=\frac{r}{c}\left(\alpha\cdot\frac{c^{\gamma}}{C_{50}^{\gamma}+c^{\gamma}}-1\right),$$
where $\alpha$ is the antimicrobial potency given by \eqref{dalpha}.
Solving the equation 
$f_r'(c)=0$, which is equivalent to
$$(\alpha-1)c^{2\gamma}-((\gamma-1)\alpha+2))c^\gamma -1=0,$$
we find 
that the optimal concentration is
\begin{equation}\label{copth}c_{opt}=c_{opt}(\gamma,\alpha)=C_{50}\cdot \left(\alpha-1\right)^{-\frac{1}{\gamma}}\cdot  \left[\frac{\gamma-1}{2}\cdot \alpha +1 +\sqrt{\left(\frac{\gamma-1}{2}\cdot \alpha \right)^2+\gamma\cdot \alpha }\right]^{\frac{1}{\gamma}},
\end{equation}
giving a kill rate of 
$$k_H(c_{opt}) =k_{max}\cdot\left(1-\frac{\alpha-1}{ \frac{\gamma+1}{2}\cdot \alpha +\sqrt{\left(\frac{\gamma-1}{2}\cdot \alpha \right)^2+\gamma\cdot \alpha } }\right),$$
and from \eqref{topte} we find that the time
$T_{opt}$ for which this concentration
should be maintained is
\begin{equation}\label{topth}
	T_{opt}=T_{opt}(\gamma,\alpha)=T_2\cdot \frac{\ln(10)LR_{target}}{\ln(2)(\alpha-1)}\cdot 
\left(1+\frac{\alpha }{\frac{\gamma-1}{2}\cdot \alpha +\sqrt{\left(\frac{\gamma-1}{2}\cdot \alpha \right)^2+\gamma\cdot \alpha }}\right),
\end{equation}
where $T_2$ is the microbial doubling time in the absence of antimicrobial (see \eqref{dtime}).

Note that:
\begin{itemize}
\item The optimal concentration $c_{opt}$ is 
linearly dependent on the half-saturation
concentration $C_{50}$. Therefore in our presentation of numerical results below we provide the values of the dimensionless ratio $\frac{C_{opt}}{C_{50}}$.

\item The optimal duration
$T_{opt}$ is linearly dependent 
on the target log-reduction
$LR_{target}$, as well as on the microbial doubling time 
$T_2$. Therefore in presentation of numerical results we provide the values of the dimensionless ratio  $\frac{T_{opt}}{T_2}$.
Note that
$T_{opt}$ does {\it{not}} depend on the value of the 
half-saturation constant $C_{50}$.

\item As a consequence of the above and of \eqref{aucopt},
the value $AUC_{opt}$ depends linearly on both 
$T_2$ and $C_{50}$, so that in the presentation of numerical
results we provide the value of the dimensionless ratio
$\frac{AUC}{T_2\cdot C_{50}}$.
\end{itemize}


Table 2 presents the optimal concentration, duration, and $AUC$ for 
parameter values in the range which is typical for most antimicrobials. 
According to studies estimating parameters for 
various antimicrobials, reviewed in \cite{czock},
the Hill coefficient of $\gamma$ is in most cases 
in the range $0.5-5$, and the potency $\alpha=\frac{k_{max}}{r}$ is mostly in the range 
$2-6$. In the calculation of $T_{opt}$ and 
$AUC_{opt}$ we have taken $LR_{target}=7$ -- in view of the linear dependence of these quantities on $LR_{target}$, to obtain $T_{opt},AUC_{opt}$ for any
value of $LR_{target}$ one simply needs to multiply the value in the table by $\frac{LR_{target}}{7}$.

We can use the explicit  expressions
\eqref{copth},\eqref{topth} to study the nature of the 
dependence of $c_{opt}$ and $T_{opt}$ on the  drug potency $\alpha$ and the Hill exponent $\gamma$. The results are 
given in Propositions \ref{proper1},\ref{proper2} - we 
omit the derivation of these results since they are routine applications of elementary calculus arguments.

The following proposition shows that higher antimicrobial potency $\alpha$ 
leads the optimal concentration profile to involve both a lower 
concentration and a shorter duration
(see also figure \ref{res}).

\begin{prop}\label{proper1}
(a) For fixed $\gamma>0$, the function
$c_{opt}(\alpha)=c_{opt}(\gamma,\alpha)$
($\alpha>1$) is monotone decreasing, with
\begin{equation}\label{lx}\lim_{\alpha\rightarrow 1+}c_{opt}(\alpha)=+\infty.\;\;\end{equation}
and
\begin{itemize}
\item[(i)] In the concave case $\gamma\leq 1$:
$\lim_{\alpha\rightarrow \infty}c_{opt}(\alpha)=0$.

\item[(ii)] In the sigmoidal case $\gamma> 1$:
$\lim_{\alpha\rightarrow \infty}c_{opt}(\alpha)=C_{50}\cdot\left(\gamma-1\right)^{\frac{1}{\gamma}}>0$.
\end{itemize}

\noindent
(b) For fixed $\gamma>0$, the function
$T_{opt}(\alpha)=T_{opt}(\gamma,\alpha)$
($\alpha>1$)
is monotone decreasing, with
\begin{equation}\label{llx}\lim_{\alpha\rightarrow 1+}T_{opt}(\alpha)=+\infty,\;\;\;\lim_{\alpha\rightarrow \infty}T_{opt}(\alpha)=0.\end{equation}
\end{prop}
	

\begin{table}[]
	\begin{center}
		Optimal concentration $\frac{c_{opt}}{C_{50}}$\\
		\vspace{0.1cm}
		
\begin{tabular}{ll|lllll|}
	\cline{3-7}
	\multicolumn{2}{l|}{\multirow{2}{*}{}}                & \multicolumn{5}{c|}{$\alpha$}                                                                                        \\ \cline{3-7} 
	\multicolumn{2}{l|}{}                                 & \multicolumn{1}{l|}{2}    & \multicolumn{1}{l|}{3}    & \multicolumn{1}{l|}{4}    & \multicolumn{1}{l|}{5}    & 6    \\ \hline
	\multicolumn{1}{|l|}{\multirow{6}{*}{$\gamma$}} & 0.5 & \multicolumn{1}{l|}{2.62} & \multicolumn{1}{l|}{0.71} & \multicolumn{1}{l|}{0.33} & \multicolumn{1}{l|}{0.19} & 0.13 \\ \cline{2-7} 
	\multicolumn{1}{|l|}{}                          & 1   & \multicolumn{1}{l|}{2.41} & \multicolumn{1}{l|}{1.37} & \multicolumn{1}{l|}{1.00} & \multicolumn{1}{l|}{0.81} & 0.69 \\ \cline{2-7} 
	\multicolumn{1}{|l|}{}                          & 2   & \multicolumn{1}{l|}{2.06} & \multicolumn{1}{l|}{1.64} & \multicolumn{1}{l|}{1.47} & \multicolumn{1}{l|}{1.37} & 1.31 \\ \cline{2-7} 
	\multicolumn{1}{|l|}{}                          & 3   & \multicolumn{1}{l|}{1.83} & \multicolumn{1}{l|}{1.60} & \multicolumn{1}{l|}{1.51} & \multicolumn{1}{l|}{1.46} & 1.42 \\ \cline{2-7} 
	\multicolumn{1}{|l|}{}                          & 4   & \multicolumn{1}{l|}{1.69} & \multicolumn{1}{l|}{1.54} & \multicolumn{1}{l|}{1.48} & \multicolumn{1}{l|}{1.44} & 1.42 \\ \cline{2-7} 
	\multicolumn{1}{|l|}{}                          & 5   & \multicolumn{1}{l|}{1.59} & \multicolumn{1}{l|}{1.48} & \multicolumn{1}{l|}{1.43} & \multicolumn{1}{l|}{1.41} & 1.39 \\ \hline
	\vspace{0.5cm}
\end{tabular}
\hspace{1cm}

Optimal duration $\frac{T_{opt}}{T_2}$\\
\vspace{0.1cm}
\begin{tabular}{ll|lllll|}
	\cline{3-7}
	\multicolumn{2}{l|}{\multirow{2}{*}{}}                & \multicolumn{5}{c|}{$\alpha$}                                                                                        \\ \cline{3-7} 
	\multicolumn{2}{l|}{}                                 & \multicolumn{1}{l|}{2}    & \multicolumn{1}{l|}{3}    & \multicolumn{1}{l|}{4}    & \multicolumn{1}{l|}{5}    & 6    \\ \hline
	\multicolumn{1}{|l|}{\multirow{6}{*}{$\gamma$}} & 0.5 & \multicolumn{1}{l|}{98.5} & \multicolumn{1}{l|}{62.5} & \multicolumn{1}{l|}{50.1} & \multicolumn{1}{l|}{43.8} & 39.9 \\ \cline{2-7} 
	\multicolumn{1}{|l|}{}                          & 1   & \multicolumn{1}{l|}{56.1} & \multicolumn{1}{l|}{31.8} & \multicolumn{1}{l|}{23.3} & \multicolumn{1}{l|}{18.8} & 16.0 \\ \cline{2-7} 
	\multicolumn{1}{|l|}{}                          & 2   & \multicolumn{1}{l|}{37.6} & \multicolumn{1}{l|}{19.6} & \multicolumn{1}{l|}{13.4} & \multicolumn{1}{l|}{10.3} & 8.3  \\ \cline{2-7} 
	\multicolumn{1}{|l|}{}                          & 3   & \multicolumn{1}{l|}{32.3} & \multicolumn{1}{l|}{16.4} & \multicolumn{1}{l|}{11.1} & \multicolumn{1}{l|}{8.4}  & 6.7  \\ \cline{2-7} 
	\multicolumn{1}{|l|}{}                          & 4   & \multicolumn{1}{l|}{29.8} & \multicolumn{1}{l|}{15.1} & \multicolumn{1}{l|}{10.1} & \multicolumn{1}{l|}{7.6}  & 6.1  \\ \cline{2-7} 
	\multicolumn{1}{|l|}{}                          & 5   & \multicolumn{1}{l|}{28.4} & \multicolumn{1}{l|}{14.3} & \multicolumn{1}{l|}{9.6}  & \multicolumn{1}{l|}{7.2}  & 5.8  \\ \hline
	\vspace{0.5cm}
\end{tabular}
\vspace{0.5cm}

 $\frac{AUC_{opt}}{T_2\cdot C_{50}}$\\
\vspace{0.1cm}

\begin{tabular}{ll|lllll|}
	\cline{3-7}
	\multicolumn{2}{l|}{\multirow{2}{*}{}}                & \multicolumn{5}{c|}{$\alpha$}                                                                                         \\ \cline{3-7} 
	\multicolumn{2}{l|}{}                                 & \multicolumn{1}{l|}{2}     & \multicolumn{1}{l|}{3}    & \multicolumn{1}{l|}{4}    & \multicolumn{1}{l|}{5}    & 6    \\ \hline
	\multicolumn{1}{|l|}{\multirow{6}{*}{$\gamma$}} & 0.5 & \multicolumn{1}{l|}{257.9} & \multicolumn{1}{l|}{44.4} & \multicolumn{1}{l|}{16.7} & \multicolumn{1}{l|}{8.5}  & 5.1  \\ \cline{2-7} 
	\multicolumn{1}{|l|}{}                          & 1   & \multicolumn{1}{l|}{135.5} & \multicolumn{1}{l|}{43.4} & \multicolumn{1}{l|}{23.3} & \multicolumn{1}{l|}{15.2} & 11.1 \\ \cline{2-7} 
	\multicolumn{1}{|l|}{}                          & 2   & \multicolumn{1}{l|}{77.4}  & \multicolumn{1}{l|}{32.1} & \multicolumn{1}{l|}{19.7} & \multicolumn{1}{l|}{14.1} & 10.9 \\ \cline{2-7} 
	\multicolumn{1}{|l|}{}                          & 3   & \multicolumn{1}{l|}{59.1}  & \multicolumn{1}{l|}{26.4} & \multicolumn{1}{l|}{16.7} & \multicolumn{1}{l|}{12.2} & 9.6  \\ \cline{2-7} 
	\multicolumn{1}{|l|}{}                          & 4   & \multicolumn{1}{l|}{50.3}  & \multicolumn{1}{l|}{23.1} & \multicolumn{1}{l|}{14.9} & \multicolumn{1}{l|}{11.0} & 8.7  \\ \cline{2-7} 
	\multicolumn{1}{|l|}{}                          & 5   & \multicolumn{1}{l|}{45.0}  & \multicolumn{1}{l|}{21.1} & \multicolumn{1}{l|}{13.7} & \multicolumn{1}{l|}{10.1} & 8.0  \\ \hline
\end{tabular}
\vspace{0.3cm}
\caption{Optimal concentration profiles for 
	different values of $\alpha=\frac{k_{max}}{r}$ and of the Hill exponent $\gamma$.
	Top: Optimal concentration $\frac{c_{opt}}{C_{50}}$, Middle: optimal duration $\frac{T_{opt}}{T_2}$ for achieving log-reduction $LR_{target}=7$.
Bottom: value of $\frac{AUC}{T_2\cdot C_{50}}$ attained by the optimal schedule, assuming $LR_{target}=7$. }
\end{center}
	\label{tablevals}
\end{table}

\begin{figure}
	\begin{center}
		\includegraphics[width=0.45\linewidth]{copt.eps}
		\includegraphics[width=0.45\linewidth]{topt.eps}
	\end{center}
	\caption{The parameters defining the optimal concentration profile as functions of $\alpha=\frac{k_{max}}{r}$, for 
		different values of the Hill exponent.
		Left: the ratio of $c_{opt}$ to the half-saturation value $C_{50}$ of the antimicrobial. Right: the ratio of $T_{opt}$ to the
		microbial doubling time $T_2$. Here it is assumed that $LR_{target}=7$.}
	\label{res}
\end{figure}

\begin{figure}
	\begin{center}
		\includegraphics[width=0.45\linewidth]{copt_gamma.eps}
		\includegraphics[width=0.45\linewidth]{Topt_gamma.eps}
	\end{center}
	\caption{The parameters defining the optimal concentration profile functions of $\gamma$, for 
		different values of the potency $\alpha$.
		Left: the ratio of $c_{opt}$ to the half-saturation value $C_{50}$ of the antimicrobial. Right: the ratio of $T_{opt}$ to the
		microbial doubling time $T_2$. Here it is assumed that $LR_{target}=7$.}
	\label{res2}
\end{figure}


The dependence 
of $c_{opt}$, $T_{opt}$ on the Hill exponent
$\gamma$ is described in the next proposition. Note that the shape of 
the function $c_{opt}(\gamma)$ is different for 
$\alpha<2$, $\alpha=2$, and $\alpha>2$ - see also figure \ref{res2}.

\begin{prop}\label{proper2}
(a) For fixed $\alpha>1$, the function
 $c_{opt}(\gamma)=c_{opt}(\gamma,\alpha)$ satisfies
 \begin{equation}\label{lllx}\lim_{\gamma\rightarrow \infty }c_{opt}(\gamma)=C_{50},\end{equation}
 and
\begin{itemize}
\item[(i)] If $1<\alpha<2$ then $c_{opt}(\gamma)$ is
monotone decreasing, with 
$\lim_{\gamma\rightarrow 0+}c_{opt}(\gamma)=+\infty$.

\item[(ii)] If $\alpha=2$, then $c_{opt}(\gamma)$
is monotone decreasing, with 
$\lim_{\gamma\rightarrow 0+}c_{opt}(\gamma)=e\cdot C_{50}$.

\item[(iii)] If $\alpha>2$, then $c_{opt}(\gamma)$ is increasing for small 
$\gamma$ and decreasing for large $\gamma$, and 
$\lim_{\gamma\rightarrow 0+}c_{opt}(\gamma)=0$.
\end{itemize}

\noindent
(b) For any $\alpha>1$, the function $T_{opt}(\gamma)=T_{opt}(\gamma,\alpha)$ is monotone 
decreasing, with
$$\lim_{\gamma\rightarrow 0+}T_{opt}(\gamma)=+\infty,\;\;\lim_{\gamma\rightarrow \infty}T_{opt}(\gamma)=0.$$
\end{prop}


\section{Pharmacokinetic considerations: the optimal bolus+continuous dosing schedule}
\label{pharmacokinetic}

In the preceding analysis we considered arbitrary antimicrobial concentration profiles $C(t)$. In practice,  however, the concentration profile cannot be chosen at will, since it is the result of a dosage plan and of the pharmacokinetics of the drug. Thus,
while the concentration profile 
given by Theorem \ref{mainr} is the 
optimal one, we show below that it is impossible to achieve this profile precisely, due the fact that the optimal 
profile $C_{opt}(t)$ has a discontinuity at
$t=T_{opt}$, while realistic pharmacokinetics 
precludes such a sharp cutoff.
It then becomes of interest to approximate the 
`ideal' concentration profile, to the extent possible, by a {\it{pharmacokinetically feasible}} one. We will consider one simple and 
natural method of doing so, and determine its
optimal version.

We assume a basic one-compartment pharmacokinetic model with first order degradation
kinetics - a reasonable choice for most commonly prescribed antimicrobials \cite{bouvier}.
The dosing rate - the rate at which 
antimicrobial is added into the compartment, will be denoted by $d(t)$.
The concentration profile is then given by
the solution of the differential equation
\begin{equation}\label{pk}\frac{dC}{dt}=V^{-1}d(t)-\kappa C(t),\;\;\;C(0)=0,\end{equation}
where $V$ is the volume of distribution and $\kappa$ is the degradation/removal rate of the antimicrobial, so that
$$\tau=\frac{\ln(2)}{\kappa}$$
is the antimicrobial half-life. 
Making the natural assumption that the cumulative dose
$$D=\int_0^\infty d(t)dt$$
is finite, it follows that
 $C(\infty)=\lim_{t\rightarrow \infty}C(t)=0$, so that, by integrating
\eqref{pk} over $[0,\infty)$ we obtain
$$0=C(\infty)-C(0)=V^{-1}\int_0^\infty d(t)dt-\kappa\cdot AUC\;\;$$
hence
\begin{equation}\label{aucd}AUC=\kappa^{-1}V^{-1}D.\end{equation}
Note that this shows that the objective of minimizing the $AUC$ is equivalent to that of 
minimizing the cumulative dose. An analogous linear relation between cumulative dosage $D$
and $AUC$ can be derived for more complicated (multi-compartment) pharmacokinetics.

The solution of \eqref{pk} is given by
\begin{equation}\label{csol}C(t)=V^{-1}\int_0^t e^{-\kappa(t-s)}d(s)ds.\end{equation}
We note that \eqref{csol} is meaningful even if
$d(t)$ is not a function, but is rather an
arbitrary non-negative measure - and may therefore include $\delta$-functions which represent 
{\it{bolus doses}}, that is a finite amount
of antimicrobial which is injected instantaneously, as we shall do below.
In any case, $C(t)$ given by \eqref{csol}
 will always be continuous and positive for all $t>0$ sufficiently large, so that the 
function $C_{opt}(t)$ given by \eqref{copt1} cannot be represented in the form \eqref{csol}, that is, it is not pharmacokinetically feasible.
 
We can, however, generate concentration
profiles which take a constant value
$\bar{c}$ for duration $0\leq t\leq T_{bc}$ by
 administering a bolus loading dose of size $V\bar{c}$,
at time $t=0$ to raise the concentration 
to $\bar{c}$, and thereafter supplying the drug as a continuous infusion at rate $\kappa V\bar{c}$ up to time $T_{bc}$, so as to maintain the concentration $\bar{c}$. This is known as a {\it{bolus+continuous}} ($bc$) dosage schedule \cite{derendorf}. Note that in order to achieve reduction in the microbial load
we must take $\bar{c}>zMIC$.
The expression for this dosing schedule is thus
\begin{equation}\label{ds}d_{bc}(t)=V\bar{c}\cdot [\delta(t)+\kappa \cdot H(T_{bc}-t)],\end{equation}
where the $\delta$-function represents 
the bolus dose, and $H$ is the 
Heaviside function: $H(t)=0$ for $t<0$ and
$H(t)=1$ for $t\geq 0$.

The resulting concentration profile, given by the solution of \eqref{pk}, will be 
\begin{equation}\label{BC}C_{bc}(t)=\begin{cases}
	\bar{c} & t\leq T_{bc}\\
	\bar{c}e^{-\kappa (t-T_{bc})}& t>T_{bc}.
\end{cases}.\end{equation}
We now formulate and study the problem of optimizing a bolus+continuous dosing schedule.

\begin{prob}\label{lc}
	Among all dosing schedules of the form \eqref{ds}, parameterized by $\bar{c}$ and 
	$T_{bc}$, for which the corresponding 
	log-reduction is $LR_{max}[C_{bc}]=LR_{target}$, find the 
	one for which the $AUC$	is minimal.
\end{prob}

The solution of this problem is given by
\begin{theorem}\label{main2}
	Define
	\begin{equation}\label{dphi0}
	\rho=\int_{zMIC}^{c_{opt}}
	\frac{k(u)-r}{u}du,\end{equation}
	where $zMIC$ is the solution of \eqref{dzMIC} and $c_{opt}$ is the solution of \eqref{copte}.

	Then:
	
	(i) If $\rho<\ln(10)LR_{target}\cdot \kappa$, then the solution of 
	Problem \ref{lc} is given by
	\begin{equation}\label{ds1}d_{opt}(t)=V\cdot c_{opt}\cdot [\delta(t)+\kappa \cdot H(T_{bc,opt}-t)],\end{equation}
	where
	\begin{equation}\label{barts}T_{bc,opt}=\frac{\ln(10)\cdot LR_{target} -\kappa^{-1}\rho}{k(c_{opt})-r}.\end{equation}
	The resulting concentration profile, given by the solution of \eqref{pk}, is
	\begin{equation}\label{BCopt}C_{bc,opt}(t)=\begin{cases}
			c_{opt} & t\leq T_{bc,opt}\\
			c_{opt}e^{-\kappa (t-T_{bc,opt})}& t>T_{bc,opt}.
		\end{cases}.\end{equation}
The target log-reduction $LR_{target}$ of the microbial population will be achieved at time
\begin{equation}\label{tso}T^*=T_{bc,opt}+\frac{1}{\kappa}\cdot \ln\left(\frac{c_{opt}}{zMIC} \right),\end{equation}
and the corresponding $AUC$ is
	\begin{equation}\label{auc2}AUC_{bc,opt}=\left[\kappa^{-1}+ \frac{\ln(10)\cdot LR_{target}- \kappa^{-1}\rho}{k(c_{opt})-r}\right]\cdot c_{opt}.\end{equation}
	
	(ii) If $\rho\geq \ln(10)LR_{target}\cdot \kappa$, the solution of Problem \ref{lc} is given by
	$$d_{opt}(t)=Vc^*\delta(t),$$
	where $c^*$ is the solution of the equation
	\begin{equation}\label{cstar}\int_{zMIC}^{c^*}
	\frac{k(u)-r}{u}du=\ln(10)LR_{target}\cdot \kappa,\end{equation}
	so that
	$$C_{bc,opt}(t)=c^*e^{-\kappa t}.$$
	The target microbial population 
	will be reached at time
	$$T^*=\frac{1}{\kappa}\cdot \ln\left(\frac{c^*}{zMIC} \right),$$
and
	$$AUC_{bc,opt}=\kappa^{-1}c^*.$$
\end{theorem}

We thus see that:

(i) If $\rho<\ln(10)LR_{target}\cdot \kappa$, corresponding to sufficiently high decay rate of the anitmicrobial, bolus+continuous dosing schedule maintains the {\it{same}} constant concentration $c_{opt}$ as the ideal concentration profile $C_{opt}(t)$ of Theorem \ref{mainr}, but for shorter
time duration. Indeed from \eqref{topte} and \eqref{barts} we have
\begin{equation}\label{dift}T_{opt}-T_{bc,opt}=\frac{\tau}{\ln(2)}\cdot\frac{\rho}{k(c_{opt})-r}>0.\end{equation}
We note also that, since $c_{opt}$ maximizes the function $f_r(c)$ given by \eqref{deff}, we have the inequality
\begin{equation}\label{inrho}\rho=\int_{zMIC}^{c_{opt}}\frac{k(u)-r}{u}du
\leq (c_{opt}-zMIC)\cdot \frac{k(c_{opt})-r}{c_{opt}},\end{equation}
so that \eqref{dift} implies
$$0< T_{opt}-T_{bc,opt}\leq \frac{\tau}{\ln(2)}\cdot \left(1-\frac{zMIC}{c_{opt}}\right),$$
which, in particular, implies that, as the antimicrobial half-life $\tau$ becomes short, $T_{bc,opt}$ 
converges to $T_{opt}$, so that the concentration profile induced by the optimal bolus+continuous schedule approaches the ideal optimal schedule $C_{opt}(t)$.
The $AUC$ achieved by the optimal bolus+continuous schedule will of course be higher than that obtained using the ideal concentration profile attaining the same log-reduction $LR_{target}$.
Indeed from \eqref{aucopt},\eqref{auc2} and \eqref{inrho} we have
\begin{equation}\label{difauc}AUC_{bc,opt}-AUC_{opt}=\frac{\tau}{\ln(2)}\cdot \left(1-\frac{\rho}{k(c_{opt})-r} \right)c_{opt}\geq \frac{\tau}{\ln(2)}\cdot zMIC.\end{equation}
As $\tau$ becomes small, \eqref{difauc} shows that the $AUC_{bc,opt}$ approaches $AUC_{opt}$.

(ii) If $\rho\geq \ln(10)LR_{target}\cdot \kappa$, corresponding to a slow decay rate of the antimicrobial, the optimal dosing schedule consists of a single bolus dose raising the antimicrobial concentration to the value $c^*$
at time $t=0$. From \eqref{cstar} it follows that,
as $\kappa\rightarrow 0$,
$$c^*=zMIC+\frac{\ln(10)LR_{target}}{k'(zMIC)}\cdot \kappa +o(\kappa),$$
so that for small $\kappa$ (long antimicrobial half-life) the optimal bolus dose raises the concentration to slightly above the 
pharmacodynamic minimal inhibitory concentration $zMIC$.

To begin the analysis leading to Theorem \ref{main2}, we calculate the values 
$LR_{max}$ and $AUC$ corresponding to 
a bolus+continuous dosing schedule.

\begin{lemma}
Consider a bolus+continuous schedule $d_{bc}(t)$ (see \eqref{ds}), with $\bar{c}>zMIC$, and the induced 
antimicrobial concentration profile $C_{bc}(t)$ (see \eqref{BC}). Then: 

(i) The maximal log-reduction corresponding to this 
concentration profile is
\begin{equation}\label{bclr}
LR_{max}=\frac{1}{\ln(10)}\left[T_{bc}\cdot (k(\bar{c})-r)+\kappa^{-1}\phi(\bar{c}) \right],
\end{equation}
where the function $\phi(c)$ is defined by:
\begin{equation}\label{dphi}\phi(c)=\int_{zMIC}^{c}
	\frac{k(u)-r}{u}du.\end{equation}
\end{lemma}

(ii) The $AUC$ corresponding to this dosage schedule is
\begin{equation}\label{aucc}AUC=[\kappa^{-1}+T_{bc}]\cdot \bar{c}.\end{equation}

\begin{proof}
(i) By \eqref{LR}, the log-reduction 
of the microbial load corresponding to 
\eqref{BC}, at time $T$, is
$$LR(T)=\frac{1}{\ln(10)}\cdot \int_0^T [k(C_{bc}(t))-r]dt,$$
hence 
$$T<T_{bc}\;\;\Rightarrow\;\; LR'(T)=k(C_{bc}(T))-r=k(\bar{c})-r>0,$$
and
$$\lim_{T\rightarrow \infty}LR'(T)=\lim_{T\rightarrow \infty}(k(C_{bc}(T))-r)=\lim_{T\rightarrow \infty}(k(\bar{c}e^{-\kappa (T-T_{bc})})-r)=-r<0.$$
We thus have that $LR(T)$ is an increasing function for $T<T_{bc}$, and a decreasing function for sufficiently large $T$, so that its maximum attained at some 
$T^*>T_{bc}$ satisfying
$LR'(T^*)=0$, that is 
\begin{eqnarray*}k(C(T^*))-r=0\;\;&\Leftrightarrow&\;\;
C_{bc}(T^*)=zMIC
\;\;\Leftrightarrow\;\;
	\bar{c}e^{-\kappa (T^*-T_{bc})}=zMIC
	\\&\Leftrightarrow&
	T^*=T_{bc}+\frac{1}{\kappa}\cdot \ln\left(\frac{\bar{c}}{zMIC} \right).
\end{eqnarray*}
Using the change of variable 
$u=\bar{c}e^{-\kappa(t-T_{bc})}$ in the integral below, we conclude that
\begin{eqnarray*}\ln(10)\cdot LR_{max}&=&\ln(10)\cdot \max_{T>0}LR(T)=\ln(10)\cdot LR\left(T^* \right)\\
	&=&T_{bc}\cdot (k(\bar{c})-r)+\int_{T_{bc}}^{T^*} \left[k\left(\bar{c}e^{-\kappa(t-T_{bc})}\right)-r\right]dt \\
&=&T_{bc}\cdot (k(\bar{c})-r)+\frac{1}{\kappa}\int_{\bar{c}e^{-\kappa(T^*-T_{bc})}}^{\bar{c}} \frac{k(u)-r}{u}du \\
&=&T_{bc}\cdot (k(\bar{c})-r)+\frac{1}{\kappa}\int_{zMIC}^{\bar{c}} \frac{k(u)-r}{u}du  \\&=&T_{bc}\cdot (k(\bar{c})-r)+\kappa^{-1}\phi(\bar{c}) ,
\end{eqnarray*}
where the function $\phi$ is defined by
\eqref{dphi}.

(ii) Using \eqref{aucd}, the $AUC$ corresponding to the concentration 
profile generated by the dosing schedule 
\eqref{ds} is given by
$$AUC=V^{-1}\kappa^{-1}\int_0^{\infty}d(t)dt=V^{-1}\kappa^{-1}\left[V\bar{c}+\kappa V\bar{c}T_{bc} \right]=[\kappa^{-1}+T_{bc}]\bar{c}.$$
\end{proof}

\begin{proof}[Proof of Theorem \ref{main2}]
	
By \eqref{bclr}, in order to achieve a given 
log-reduction $LR_{target}$ using a dosing schedule of the form \eqref{ds}, the parameters 
$\bar{c},T_{bc}$ defining this schedule 
must satisfy the constraint $LR_{max}[C]=LR_{target}$, or
\begin{equation}\label{co1}T_{bc}\cdot (k(\bar{c})-r)+\kappa^{-1}\phi(\bar{c}) =\ln(10)\cdot LR_{target}.\end{equation}
We need to minimize the expression \eqref{aucc} over $(\bar{c},T_{bc})$, under the constraints 
\eqref{co1} and \begin{equation}\label{ac}\bar{c}\geq zMIC,\;\;\; T_{bc}\geq 0.
\end{equation}

The constraint \eqref{co1} can be written
as 
\begin{equation}\label{bart}T_{bc} =\frac{\ln(10)\cdot LR_{target}- \kappa^{-1}\phi(\bar{c})}{k(\bar{c})-r},\end{equation}
and the inequality constraints \eqref{ac} imply that $\bar{c}$ must 
satisfy 
\begin{equation}\label{ine1}zMIC\leq \bar{c}\leq \phi^{-1}(\ln(10)LR_{target}\cdot \kappa)=c^*,
\end{equation}
where $c^*$ is the solution of \eqref{cstar}.

Substituting \eqref{bart} into \eqref{aucc} we get
\begin{equation}\label{aucn}AUC=AUC(\bar{c})=\left[\kappa^{-1}+ \frac{\ln(10)\cdot LR_{target}-\kappa^{-1} \phi(\bar{c})}{k(\bar{c})-r}\right]\cdot \bar{c},\end{equation}
which must be minimized over $\bar{c}$
satisfying \eqref{ine1}. Noting that 
the expression \eqref{aucn} goes to 
$+\infty$ when $\bar{c}\rightarrow zMIC$,
we see that the minimum is attained either at (a) an interior point of the interval \eqref{ine1}, or (b) at $\bar{c}=c^*$. 

If $c_{opt}<c^*$, then 
since $c_{opt}$ is the maximizer of $f_r(c)$
given by \eqref{deff}, we have

\begin{eqnarray*}&&\int_{c_{opt}}^{c^*}
\frac{k(u)-r}{u}du<  (c^*-c_{opt})\cdot \frac{k(c_{opt})-r}{c_{opt}}\\
&\Leftrightarrow&\;\; \phi(c^*)- \phi(c_{opt})< \frac{c^*-c_{opt}}{c_{opt}}\cdot (k(c_{opt})-r)\\
&\Leftrightarrow&\;\;\left[\kappa^{-1}+ \frac{\ln(10)\cdot LR_{target}- \kappa^{-1}\phi(c_{opt})}{k(c_{opt})-r}\right]\cdot c_{opt}< \kappa^{-1}c^* \\
&\Leftrightarrow& AUC(c_{opt})<AUC(c^*),
\end{eqnarray*}
so that the minimum of $AUC(\bar{c})$ 
in the interval \eqref{ine1} is attained at 
an interior point, at which $AUC'(\bar{c})$
must vanish, leading to
\begin{eqnarray*}AUC'(\bar{c})&=&\kappa^{-1}+ \frac{- \kappa^{-1}\phi'(\bar{c})(k(\bar{c})-r)-[\ln(10)\cdot LR_{target}-\kappa^{-1} \phi(\bar{c})]k'(\bar{c})}{(k(\bar{c})-r)^2}\cdot \bar{c}\\&+& \frac{\ln(10)\cdot LR_{target}- \kappa^{-1}\phi(\bar{c})}{k(\bar{c})-r}=0
\end{eqnarray*}
\begin{eqnarray*}
&\Leftrightarrow&\;\;\frac{[\ln(10)\cdot LR_{target}- \kappa^{-1}\phi(\bar{c})][k(\bar{c})-r-k'(\bar{c})\cdot \bar{c}]}{(k(\bar{c})-r)^2}=0\\&\Leftrightarrow&\;\;k(\bar{c})-r-k'(\bar{c})\cdot \bar{c}=0
\;\;\Leftrightarrow\;\;\bar{c}=c_{opt}.\end{eqnarray*}
Thus, from \eqref{bart} we get \eqref{barts}, and from \eqref{aucc} we get \eqref{auc2}. 

On the other hand, if 
$c_{opt}\geq c^*$, the above calculation shows that the derivative of $AUC(\bar{c})$ does not vanish in the interior of the interval \eqref{ine1}, so that the minimum is attained at $\bar{c}=c^*$, proving part (ii) of the theorem.
\end{proof}

\section{Optimal bolus+continuous dosing in the case of a Hill-type pharmacodynamic function}
\label{bchill}

We now apply the results of Theorem \ref{main2}
to the case in which $k(c)$ is a Hill-type function \eqref{hill}, and provide numerical
examples of the results obtained.

An explicit evaluation of the integral in \eqref{dphi} gives
$$\phi(c)=\frac{r}{\gamma}
\cdot 
\ln\left(\frac{\left(1+\left(\frac{c}{C_{50}}\right)^\gamma\right)^\alpha}{\left(\frac{c}{C_{50}}\right)^{\gamma }}\cdot \frac{(\alpha-1)^{\alpha-1}}{\alpha^\alpha}\right),$$
hence, using \eqref{copth},
$$\rho=\phi(c_{opt})=\frac{r}{\gamma}
\cdot 
\ln\left(\frac{\left(\frac{\gamma+1}{2} +\sqrt{\left(\frac{\gamma-1}{2}\right)^2+\frac{\gamma}{\alpha} }\right)^\alpha}{  \frac{\gamma-1}{2}\cdot \alpha +1 +\sqrt{\left(\frac{\gamma-1}{2}\cdot \alpha \right)^2+\gamma\cdot \alpha }}\right),$$
and $c^*$ is the solution of 
\begin{equation}\label{cse}
\frac{\left(1+\left(\frac{c^*}{C_{50}}\right)^\gamma\right)^\alpha}{\left(\frac{c^*}{C_{50}}\right)^{\gamma }}=10^{LR_{target}\cdot \frac{\kappa\gamma}{r}}\cdot \frac{\alpha^\alpha}{(\alpha-1)^{\alpha-1}}.\end{equation}

The condition
 $\rho<\ln(10)\cdot LR_{target}\cdot \kappa$ holds iff
\begin{equation}\label{ch}\tau<T_2\cdot \gamma\cdot \ln(10)\cdot LR_{target}
\cdot \left[
\ln\left(\frac{\left(\frac{\gamma+1}{2} +\sqrt{\left(\frac{\gamma-1}{2}\right)^2+\frac{\gamma}{\alpha} }\right)^\alpha}{  \frac{\gamma-1}{2}\cdot \alpha +1 +\sqrt{\left(\frac{\gamma-1}{2}\cdot \alpha \right)^2+\gamma\cdot \alpha }}\right)\right]^{-1}.\end{equation}
We thus have:
\begin{itemize}
	\item If \eqref{ch} holds, that is the antimicrobial half-life is sufficiently short, then the solution of Problem \ref{lc}
	is the bolus+continuous schedule \eqref{ds1}, where
	$c_{opt}$ is given by \eqref{copth} and 
	$T_{bc,opt}$ is given by 
	\begin{eqnarray*}&&\frac{T_{bc,opt}}{T_2}=\frac{1}{\ln(2)(\alpha-1)}\cdot 
	\left(1+\frac{\alpha }{\frac{\gamma-1}{2}\cdot \alpha +\sqrt{\left(\frac{\gamma-1}{2}\cdot \alpha \right)^2+\gamma\cdot \alpha }}\right)\\ &\times&\left[\ln(10)LR_{target}-\frac{1}{\gamma}\cdot 
	\ln\left(\frac{\left(\frac{\gamma+1}{2} +\sqrt{\left(\frac{\gamma-1}{2}\right)^2+\frac{\gamma}{\alpha} }\right)^\alpha}{  \frac{\gamma-1}{2}\cdot \alpha +1 +\sqrt{\left(\frac{\gamma-1}{2}\cdot \alpha \right)^2+\gamma\cdot \alpha }}\right)\cdot \frac{\tau}{T_2}\right].\end{eqnarray*}
	Note that the value $T_{bc,opt}$ does not depend on the 
	half-saturation constant $C_{50}$.
	
	\item If the reverse inequality to \eqref{ch} holds, that is if the antimicrobial half-life is sufficiently long, then  the solution of Problem \ref{lc} is a single bolus dose $Vc^*$, where 
	$c^*$ is given by \eqref{cse}.
\end{itemize}



Tables 3,4 present numerical results regarding optimal bolus+continuous schedules, using the above formulae, which can be compared with the 
results concerning the ideal concentration profile
in Table 2, for parameter values which are in a range relevant to applications. In Table 3 
it is assumed that the ratio of the antimicrobial 
half-life to the microbial doubling time is $\frac{\tau}{T_2}=4$, while in Table 4
we take slower antimicrobial decay, $\frac{\tau}{T_2}=2$. For all parameter values considered, the condition \eqref{ch} holds, so that the optimal schedule includes both a bolus and a continuous infusion. Comparing the $AUC$ obtained in Table 3 with the ideal ones in Table
2, we observe that, although, as expected, the value $AUC_{bc,opt}$ values attained by the optimal bolus+continuous schedule are higher than $AUC_{opt}$, in most cases they lie within $25\%$ of that value, with the exception of extreme cases of high Hill coefficient $\gamma$ and antimicrobial potency $\alpha$ (e.g. for $\gamma=5,\alpha=6$, $AUC_{bc,opt}$ is $67.5\%$ than $AUC_{opt}$).
For shorter antimicrobial half-lives, as in Table 4, the $AUC_{bc,opt}$ is even closer to $AUC_{opt}$. In general, we can conclude that 
for realistic parameter values, the optimal bolus+continuous dosing schedule attains outcomes which are quite close to the ideal one.
As an example, in figure \ref{exa} we compare the 
ideal optimal concentration curve and the concentration curve induced by the optimal bolus+continuous schedule, and the corresponding microbial population, using parameters as in the example of \cite{bouvier}, with parameters fitting the antibacterial Tobramycin applied to {\it{Pseudomonas aeruginosa ATCC 27853}} (see figure caption for parameter values). The 
duration of infusion in the optimal bolus+continuous schedule is $T_{bc}=31.3$ hours, slightly shorter than the duration  $T_{opt}=33.0$ of the ideal optimal concentration profile. For the 
ideal concentration profile, the bacterial population reaches the target (eradication) value $10^{-7}B_0$ at $t=T_{opt}$, while 
for the optimal bolus+continuous schedule 
the target value is reached at time $T^*=34.8$ hours (as given by \eqref{tso}), that is $3.5$ hours after 
antimicrobial infusion is ended.
The $AUC$ corresponding to the optimal bolus+continuous schedule is $5.82 \frac{mg\cdot h}{L}$, only $4\%$ higher than the ideal optimal $AUC_{opt}=5.59 \frac{mg\cdot h}{L}$.

\begin{figure}
	\begin{center}
		\includegraphics[width=0.45\linewidth]{example1.eps}
		\includegraphics[width=0.45\linewidth]{example2.eps}
	\end{center}
	\caption{Left: the ideal optimal antimicrobial concentration profile (blue dashed line), and the concentration profile induced by the optimal bolus+continuous schedule (red line). Right: The corresponding microbial populations relative to the initial population. Parameters, corresponding to the example from \cite{bouvier} of Tobramycin applied to {\it{Pseudomonas aeruginosa}}, are: $C_{50}=4.187 \frac{mg}{L}$, $k_{max}=7.115 h^{-1}$, $\gamma=0.416$, $r=0.995 h^{-1}$, $\kappa=0.333 h^{-1}$, $LR_{target}=7$.}
	\label{exa}
\end{figure}


\begin{table}[]
	\begin{center}
		
		Optimal duration $\frac{T_{bc}}{T_2}$\\
		\vspace{0.1cm}
		\begin{tabular}{ll|lllll|}
			\cline{3-7}
			\multicolumn{2}{l|}{\multirow{2}{*}{}}                & \multicolumn{5}{c|}{$\alpha$}                                                                                        \\ \cline{3-7} 
			\multicolumn{2}{l|}{}                                 & \multicolumn{1}{l|}{2}    & \multicolumn{1}{l|}{3}    & \multicolumn{1}{l|}{4}    & \multicolumn{1}{l|}{5}    & 6    \\ \hline
			\multicolumn{1}{|l|}{\multirow{6}{*}{$\gamma$}} & 0.5 & \multicolumn{1}{l|}{95.7} & \multicolumn{1}{l|}{59.5} & \multicolumn{1}{l|}{47.0} & \multicolumn{1}{l|}{40.7} & 36.7 \\ \cline{2-7} 
			\multicolumn{1}{|l|}{}                          & 1   & \multicolumn{1}{l|}{53.5} & \multicolumn{1}{l|}{28.9} & \multicolumn{1}{l|}{20.2} & \multicolumn{1}{l|}{15.7} & 12.8 \\ \cline{2-7} 
			\multicolumn{1}{|l|}{}                          & 2   & \multicolumn{1}{l|}{35.4} & \multicolumn{1}{l|}{17.13} & \multicolumn{1}{l|}{10.8} & \multicolumn{1}{l|}{7.5} & 5.5  \\ \cline{2-7} 
			\multicolumn{1}{|l|}{}                          & 3   & \multicolumn{1}{l|}{30.3} & \multicolumn{1}{l|}{14.3} & \multicolumn{1}{l|}{8.8} & \multicolumn{1}{l|}{6.0}  & 4.3  \\ \cline{2-7} 
			\multicolumn{1}{|l|}{}                          & 4   & \multicolumn{1}{l|}{28.0} & \multicolumn{1}{l|}{13.2} & \multicolumn{1}{l|}{8.1} & \multicolumn{1}{l|}{5.5}  & 4.0  \\ \cline{2-7} 
			\multicolumn{1}{|l|}{}                          & 5   & \multicolumn{1}{l|}{26.8} & \multicolumn{1}{l|}{12.6} & \multicolumn{1}{l|}{7.8}  & \multicolumn{1}{l|}{5.3}  & 3.9  \\ \hline
			\vspace{0.5cm}
		\end{tabular}
		\vspace{0.5cm}
		
		 $\frac{AUC_{bc,opt}}{T_2\cdot C_{50}}$\\
		\vspace{0.1cm}
		
		\begin{tabular}{ll|lllll|}
			\cline{3-7}
			\multicolumn{2}{l|}{\multirow{2}{*}{}}                & \multicolumn{5}{c|}{$\alpha$}                                                                                         \\ \cline{3-7} 
			\multicolumn{2}{l|}{}                                 & \multicolumn{1}{l|}{2}     & \multicolumn{1}{l|}{3}    & \multicolumn{1}{l|}{4}    & \multicolumn{1}{l|}{5}    & 6    \\ \hline
			\multicolumn{1}{|l|}{\multirow{6}{*}{$\gamma$}} & 0.5 & \multicolumn{1}{l|}{265.7} & \multicolumn{1}{l|}{46.4} & \multicolumn{1}{l|}{17.6} & \multicolumn{1}{l|}{9.0}  & 5.5  \\ \cline{2-7} 
			\multicolumn{1}{|l|}{}                          & 1   & \multicolumn{1}{l|}{143.1} & \multicolumn{1}{l|}{47.4} & \multicolumn{1}{l|}{26.0} & \multicolumn{1}{l|}{17.3} & 12.8 \\ \cline{2-7} 
			\multicolumn{1}{|l|}{}                          & 2   & \multicolumn{1}{l|}{84.7}  & \multicolumn{1}{l|}{37.6} & \multicolumn{1}{l|}{24.3} & \multicolumn{1}{l|}{18.2} & 14.7 \\ \cline{2-7} 
			\multicolumn{1}{|l|}{}                          & 3   & \multicolumn{1}{l|}{66.1}  & \multicolumn{1}{l|}{32.2} & \multicolumn{1}{l|}{22.0} & \multicolumn{1}{l|}{17.2} & 14.3  \\ \cline{2-7} 
			\multicolumn{1}{|l|}{}                          & 4   & \multicolumn{1}{l|}{57.1}  & \multicolumn{1}{l|}{29.1} & \multicolumn{1}{l|}{20.5} & \multicolumn{1}{l|}{16.3} & 13.8  \\ \cline{2-7} 
			\multicolumn{1}{|l|}{}                          & 5   & \multicolumn{1}{l|}{51.7}  & \multicolumn{1}{l|}{27.1} & \multicolumn{1}{l|}{19.4} & \multicolumn{1}{l|}{15.6} & 13.4  \\ \hline
		\end{tabular}
		\vspace{0.3cm}
			
		\caption{Optimal bolus+continuous dosage plans for achieving log-reduction $LR_{target}=7$, when the half-life of the antimicrobial satisfies $\frac{\tau}{T_2}=4$, for different values of $\alpha=\frac{k_{max}}{r}$ and of the Hill exponent $\gamma$.
		The concentration $c_{opt}$ to be maintained is the same as is Table 2.
			Top: Optimal duration of dosing, 
			Bottom: value of $AUC$ attained by the optimal schedule}
	\end{center}
\label{tablevals1}
\end{table}



\begin{table}[]
	\begin{center}
		
		Optimal duration $\frac{T_{bc}}{T_2}$\\
		\vspace{0.1cm}
		\begin{tabular}{ll|lllll|}
			\cline{3-7}
			\multicolumn{2}{l|}{\multirow{2}{*}{}}                & \multicolumn{5}{c|}{$\alpha$}                                                                                        \\ \cline{3-7} 
			\multicolumn{2}{l|}{}                                 & \multicolumn{1}{l|}{2}    & \multicolumn{1}{l|}{3}    & \multicolumn{1}{l|}{4}    & \multicolumn{1}{l|}{5}    & 6    \\ \hline
			\multicolumn{1}{|l|}{\multirow{6}{*}{$\gamma$}} & 0.5 & \multicolumn{1}{l|}{97.1} & \multicolumn{1}{l|}{61.0} & \multicolumn{1}{l|}{48.6} & \multicolumn{1}{l|}{42.2} & 38.3 \\ \cline{2-7} 
			\multicolumn{1}{|l|}{}                          & 1   & \multicolumn{1}{l|}{54.8} & \multicolumn{1}{l|}{30.3} & \multicolumn{1}{l|}{24.6} & \multicolumn{1}{l|}{17.2} & 14.4 \\ \cline{2-7} 
			\multicolumn{1}{|l|}{}                          & 2   & \multicolumn{1}{l|}{36.5} & \multicolumn{1}{l|}{18.4} & \multicolumn{1}{l|}{12.1} & \multicolumn{1}{l|}{8.9} & 6.9  \\ \cline{2-7} 
			\multicolumn{1}{|l|}{}                          & 3   & \multicolumn{1}{l|}{31.3} & \multicolumn{1}{l|}{15.4} & \multicolumn{1}{l|}{10.0} & \multicolumn{1}{l|}{7.2}  & 5.5  \\ \cline{2-7} 
			\multicolumn{1}{|l|}{}                          & 4   & \multicolumn{1}{l|}{28.9} & \multicolumn{1}{l|}{14.1} & \multicolumn{1}{l|}{9.1} & \multicolumn{1}{l|}{6.6}  & 5.0  \\ \cline{2-7} 
			\multicolumn{1}{|l|}{}                          & 5   & \multicolumn{1}{l|}{27.6} & \multicolumn{1}{l|}{24.1} & \multicolumn{1}{l|}{8.7}  & \multicolumn{1}{l|}{6.3}  & 4.8  \\ \hline
			\vspace{0.5cm}
		\end{tabular}
		\vspace{0.5cm}
		
		 $\frac{AUC_{bc,opt}}{T_2\cdot C_{50}}$\\
		\vspace{0.1cm}
		
		\begin{tabular}{ll|lllll|}
			\cline{3-7}
			\multicolumn{2}{l|}{\multirow{2}{*}{}}                & \multicolumn{5}{c|}{$\alpha$}                                                                                         \\ \cline{3-7} 
			\multicolumn{2}{l|}{}                                 & \multicolumn{1}{l|}{2}     & \multicolumn{1}{l|}{3}    & \multicolumn{1}{l|}{4}    & \multicolumn{1}{l|}{5}    & 6    \\ \hline
			\multicolumn{1}{|l|}{\multirow{6}{*}{$\gamma$}} & 0.5 & \multicolumn{1}{l|}{261.8} & \multicolumn{1}{l|}{45.4} & \multicolumn{1}{l|}{17.2} & \multicolumn{1}{l|}{8.8}  & 5.3  \\ \cline{2-7} 
			\multicolumn{1}{|l|}{}                          & 1   & \multicolumn{1}{l|}{139.3} & \multicolumn{1}{l|}{45.4} & \multicolumn{1}{l|}{24.6} & \multicolumn{1}{l|}{16.3} & 11.9 \\ \cline{2-7} 
			\multicolumn{1}{|l|}{}                          & 2   & \multicolumn{1}{l|}{81.1}  & \multicolumn{1}{l|}{34.8} & \multicolumn{1}{l|}{22.0} & \multicolumn{1}{l|}{16.2} & 12.8 \\ \cline{2-7} 
			\multicolumn{1}{|l|}{}                          & 3   & \multicolumn{1}{l|}{62.6}  & \multicolumn{1}{l|}{29.3} & \multicolumn{1}{l|}{19.4} & \multicolumn{1}{l|}{14.7} & 12.0  \\ \cline{2-7} 
			\multicolumn{1}{|l|}{}                          & 4   & \multicolumn{1}{l|}{53.7}  & \multicolumn{1}{l|}{26.1} & \multicolumn{1}{l|}{17.7} & \multicolumn{1}{l|}{13.6} & 11.2  \\ \cline{2-7} 
			\multicolumn{1}{|l|}{}                          & 5   & \multicolumn{1}{l|}{48.4}  & \multicolumn{1}{l|}{24.1} & \multicolumn{1}{l|}{16.6} & \multicolumn{1}{l|}{12.9} & 10.7  \\ \hline
		\end{tabular}
		\vspace{0.3cm}
			\caption{Optimal bolus+continuous dosage plans for achieving log-reduction $LR_{target}=7$, when the half-life of the antimicrobial satisfies $\frac{\tau}{T_2}=2$, for different values of $\alpha=\frac{k_{max}}{r}$ and of the Hill exponent $\gamma$.
			The concentration $c_{opt}$ to be maintained is the same as is Table 2.
			Top: Optimal duration of dosing, 
			Bottom: value of $AUC$ attained by the optimal schedule}
		
	\end{center}
	\label{tablevals2}
\end{table}


\section{Discussion}
\label{discussion}

The results obtained in this work provide a 
a baseline and a reference point for evaluating the efficient use of antimicrobials. Theorem \ref{mainr} 
describes the ideal concentration profile
leading to eradication of the microbial population, with a minimal $AUC$ - which consists of a constant concentration 
$c_{opt}$ applied over a period of duration $T_{opt}$.
We provided simple equations allowing to compute the key quantities 
$c_{opt}$ and $T_{opt}$ for an arbitrary
pharmacodynamic function $k(c)$, and explicit expressions for these quantities in the case of the widely-used Hill-type function (see \eqref{copth},\eqref{topth}).

Since the `ideal' concentration profile
is not strictly feasible due to 
pharmacokinetic contraints, we have also considered the question of determining
an optimal bolus+continuous dosing schedule, assuming first order pharmacokinetics. 
Our results show that the optimal 
dosing leads to the {\it{same}} constant concentration $c_{opt}$ as for the ideal concentration profile during a dosing period 
$T_{bc,opt}<T_{opt}$. Our numerical comparisons show that the results obtained 
using this optimal bolus+continuous dosing plan are in most cases only slightly 
inferior to those obtained using the `ideal' 
concentration profile. 

We note that while the `ideal' concentration profile was proved to be optimal among {\it{all}} concentration profiles, in the investigation of dosing plans under 
pharmacokinetic constraints we restricted ourselves in advance to bolus+continuous plans with 
constant dosing rate following the loading dose. In fact we conjecture that no 
dosing schedule with a non-constant dosing rate can improve upon the performance of the dosing plans considered, but leave a full treatment of this question to future work. 

In practice, the administration of a continuous dosing schedule requires the use of intravenous infusion, infusion pumps, or sustained/controlled release formulations. An intermittent dosing schedule, involving a series of bolus doses, is often opted for \cite{derendorf}. It is 
intuitively obvious, however, and can be formally proved, that sufficiently frequent intermittent infusions, with appropriate doses, can be used to approximate a bolus+continuous schedule, so that our results concerning the optimal bolus+continuous schedule are also relevant to the design of intermittent schedules, and in particular can be used to assess the degree to which a proposed intermittent schedule can be improved upon by increasing the frequency of dosing or by shifting to continuous infusion.

The models which we employed in the study are 
standard ones, which are widely applied in 
the quantitative literature on antimicrobial pharmacology. 
However, as always with mathematical modelling, it
is important to take into account the limitations of the models employed, and their possible implications regarding the conclusions drawn using the model. We now discuss several such limitations, which suggest issues which
should be addressed in future work.

Our investigation of optimal dosing under pharmacokinetic constraints was restricted to a basic one-compartment model with first order 
kinetics. It is also important to consider optimal dosing plans in the context of
more elaborate models. In particular, if the 
antimicrobial enters a compartment distinct from the site of action, the resulting delay implies that an initial bolus dose cannot immediately raise the concentration at the site of action to 
an optimal level. Implications of this fact for 
the optimization of dosing schedules merit  systematic study.

Some of the works on dynamical modelling 
of microbial dynamics under the effect of antimicrobials, in particular the more theoretical studies, include mechanisms 
which are absent from the basic model 
considered here. 
One such mechanism is density-dependence of the microbial population growth, which will be relevant if this population reaches a size at which resource limitation 
reduces its growth rate \cite{ali,bhagunde,geli,kesisoglou,khan,levin,nikolaou1,paterson,tindall,zilonova}.
Another element which influences microbial dynamics 
is immune response, and if the strength of immune supression is comparable to that of the antimicrobial, it might be important to include
an immune system component in the model
\cite{geli,goranova,hoyle,tindall}.
It is important to examine whether and how 
the incorporation of these additional mechanisms 
into the models modifies the conclusions 
obtained here. Since more elaborate models include nonlinearities and/or are higher dimensional, such extension of our investigation is non-trivial. In particular general analytical results may not always be feasible, so that it might be necessary to resort to numerical simulation and optimization \cite{ali,cicchese,colin,goranova,hoyle,khan,paterson,pena,smith}. The analytical results presented here should form a useful point of reference for comparison with results obtained using extended models.

The effort to prevent antimicrobial resistance is an important motivation for the efficient use of antimicrobial agents, which has been studied here. However the model we employed does not directly address this issue, as do models which include a 
separate compartment for resistent strains which may arise due to mutations \cite{ali,geli,goranova,khan,lipsitch,marrec,morsky,nielsen,paterson,singh,zilonova}. This can lead to additional considerations regarding optimization
of antimicrobial use, which are beyond our scope here.

Finally, we note that our analyses assume 
known pharmacodynamic and pharmacokinetic parameters, but these parameters themselves will vary among individuals, and the design of treatment plans must also take 
into account this heterogeneity, a fact which has led to the development of population PK/PD modelling \cite{develde,vinks}. Incorporating the insights obtained from the results in the present work into the population perspective remains a task for future research.


\begin{thebibliography}{1}

\bibitem{ali}
Ali, A., Imran, M., Sial, S., \& Khan, A. (2022). Effective antimicrobial dosing in the presence of resistant strains. PLoS ONE, 17(10), e0275762.

%\bibitem{Anastassiou}
%G.A. Anastassiou, General Moment Optimization Problems, in 
%Floudas, C. A., Pardalos, P. M. (Eds.), (2008) Encyclopedia of optimization. Springer.

\bibitem{austin}
Austin, D. J., White, N. J., \& Anderson, R. M. (1998). The dynamics of drug action on the within-host population growth of infectious agents: melding pharmacokinetics with pathogen population dynamics. Journal of Theoretical Biology, 194(3), 313-339.

\bibitem{bhagunde}
Bhagunde, P. R., Nikolaou, M., \& Tam, V. H. (2015). Modeling heterogeneous bacterial populations exposed to antibiotics: The logistic dynamics case. AIChE Journal, 61(8), 2385-2393.

\bibitem{bouvier}
Bouvier d’Yvoire, M. J., \& Maire, P. H. (1996). Dosage regimens of antibacterials. Clinical Drug Investigation, 11(4), 229-239.

\bibitem{bulitta}
Bulitta, J. B., Hope, W. W., Eakin, A. E., Guina, T., Tam, V. H., Louie, A., Drusano, G.L., \& Hoover, J. L. (2019). Generating robust and informative nonclinical in vitro and in vivo bacterial infection model efficacy data to support translation to humans. Antimicrobial Agents and Chemotherapy, 63(5), e02307-18.

\bibitem{cicchese}
Cicchese, J. M., Pienaar, E., Kirschner, D. E., \& Linderman, J. J. (2017). Applying optimization algorithms to tuberculosis antibiotic treatment regimens. Cellular and Molecular Bioengineering, 10, 523-535.

\bibitem{colin}
Colin, P. J., Eleveld, D. J., \& Thomson, A. H. (2020). Genetic Algorithms as a Tool for Dosing Guideline Optimization: Application to Intermittent Infusion Dosing for Vancomycin in Adults. CPT: Pharmacometrics \& Systems Pharmacology, 9(5), 294-302.

\bibitem{corvaisier}
Corvaisier, S., Maire, P. H., Bouvier d’Yvoire, M. Y., Barbaut, X., Bleyzac, N., Jelliffe, R. W. (1998). Comparisons between antimicrobial pharmacodynamic indices and bacterial killing as described by using the Zhi model. Antimicrobial Agents and Chemotherapy, 42(7), 1731-1737.

\bibitem{czock}
Czock, D., \& Keller, F. (2007). Mechanism-based pharmacokinetic-pharmacodynamic modeling of antimicrobial drug effects. Journal of Pharmacokinetics and Pharmacodynamics, 34(6), 727-751.

\bibitem{derendorf}
Derendorf, H., \& Schmidt, S. (2019). Rowland and Tozer's clinical pharmacokinetics and pharmacodynamics: concepts and applications, Wolters Kluwer.

\bibitem{develde}
de Velde, F., Mouton, J. W., de Winter, B. C., van Gelder, T., \& Koch, B. C. (2018). Clinical applications of population pharmacokinetic models of antibiotics: Challenges and perspectives. Pharmacological Research, 134, 280-288.

\bibitem{geli}
Geli, P., Laxminarayan, R., Dunne, M., \& Smith, D. L. (2012). “One-size-fits-all”? Optimizing treatment duration for bacterial infections. PloS ONE, 7(1), e29838.

\bibitem{goranova}
Goranova, M., Ochoa, G., Maier, P., \& Hoyle, A. (2022). Evolutionary optimisation of antimicrobial dosing regimens for bacteria with different levels of resistance. Artificial Intelligence in Medicine, 102405.

\bibitem{hoyle}
Hoyle, A., Cairns, D., Paterson, I., McMillan, S., Ochoa, G., \& Desbois, A. P. (2020). Optimising efficacy of antimicrobials against systemic infection by varying dosage quantities and times. PLoS computational biology, 16(8), e1008037.

\bibitem{kesisoglou}
Kesisoglou, I., Tam, V. H., Tomaras, A. P., \& Nikolaou, M. (2022). Discerning in vitro pharmacodynamics from OD measurements: A model-based approach. Computers \& chemical engineering, 158, 107617.

\bibitem{khan}
Khan, A., \& Imran, M. (2018). Optimal dosing strategies against susceptible and resistant bacteria. Journal of Biological Systems, 26(01), 41-58.

\bibitem{Leszczynski}
Leszczy\'nski, M., Ledzewicz, U., \& Schättler, H. (2020). Optimal control for a mathematical model for chemotherapy with pharmacometrics. Mathematical Modelling of Natural Phenomena, 15, 69.

\bibitem{levin}
Levin, B. R., \& Udekwu, K. I. (2010). Population dynamics of antibiotic treatment: a mathematical model and hypotheses for time-kill and continuous-culture experiments. Antimicrobial Agents and Chemotherapy, 54(8), 3414-3426.

\bibitem{lipsitch}
Lipsitch, M., \& Levin, B. R. (1997). The population dynamics of antimicrobial chemotherapy. Antimicrobial Agents and Chemotherapy, 41(2), 363-373.

%\bibitem{kemperman}
%Kemperman, J. H. (1968). The general moment problem, a geometric approach. The Annals of Mathematical Statistics, 39(1), 93-122.

\bibitem{luterbach}
Luterbach, C. L., \& Rao, G. G. (2022). Use of pharmacokinetic/pharmacodynamic approaches for dose optimization: a case study of plazomicin. Current Opinion in Microbiology, 70, 102204.

\bibitem{macheras}
Macheras, P., \& Iliadis, A. (2016). Modeling in biopharmaceutics, pharmacokinetics and pharmacodynamics: homogeneous and heterogeneous approaches, Springer.

\bibitem{marrec}
Marrec, L., \& Bitbol, A. F. (2020). Resist or perish: fate of a microbial population subjected to a periodic presence of antimicrobial. PLoS Computational Biology, 16(4), e1007798.

\bibitem{meibohm}
Meibohm, B., \& Derendorf, H. (1997). Basic concepts of pharmacokinetic/pharmacodynamic (PK/PD) modelling. International Journal of Clinical Pharmacology and Therapeutics, 35(10), 401-413.

\bibitem{morsky}
Morsky, B., \& Vural, D. C. (2022). Suppressing evolution of antibiotic resistance through environmental switching. Theoretical Ecology, 15(2), 115-127.

\bibitem{mouton}
Mouton, J. W., \& Vinks, A. A. (2005). PK-PD modelling of antibiotics in vitro and in vivo using bacterial growth and kill kinetics: the zMIC vs stationary concentrations. Clinical Pharmacokinetics, 44, 201-10.

\bibitem{mueller}
Mueller, M., de la Pena, A., \& Derendorf, H. (2004). Issues in pharmacokinetics and pharmacodynamics of anti-infective agents: kill curves versus MIC. Antimicrobial Agents and Chemotherapy, 48(2), 369-377.

\bibitem{murray}
Murray, C. J., Ikuta, K. S., Sharara, F., et al (2022). Global burden of bacterial antimicrobial resistance in 2019: a systematic analysis. The Lancet, 399(10325), 629-655.

\bibitem{nielsen}
Nielsen, E. I., \& Friberg, L. E. (2013). Pharmacokinetic-pharmacodynamic modeling of antibacterial drugs. Pharmacological Reviews, 65(3), 1053-1090.

\bibitem{nikolaou1}
Nikolaou, M., \& Tam, V. H. (2006). A new modeling approach to the effect of antimicrobial agents on heterogeneous microbial populations. Journal of Mathematical Biology, 52(2), 154-182.

\bibitem{nikolaou2}
Nikolaou, M., Schilling, A. N., Vo, G., Chang, K. T., \& Tam, V. H. (2007). Modeling of microbial population responses to time-periodic concentrations of antimicrobial agents. Annals of Biomedical Engineering, 35(8), 1458-1470.

\bibitem{onufrak}
Onufrak, N. J., Forrest, A., \& Gonzalez, D. (2016). Pharmacokinetic and pharmacodynamic principles of anti-infective dosing. Clinical Therapeutics, 38(9), 1930-1947.

\bibitem{owens}
Owens, R. C., Nightingale, C. H., \& Ambrose, P. G. (Eds.). (2004). Antibiotic optimization: concepts and strategies in clinical practice. CRC Press.

\bibitem{paterson}
Paterson, I. K., Hoyle, A., Ochoa, G., Baker-Austin, C., \& Taylor, N. G. (2016). Optimising antimicrobial usage to treat bacterial infections. Scientific Reports, 6(1), 1-10.

\bibitem{pena}
Pe\~na-Miller, R., Lähnemann, D., Schulenburg, H., Ackermann, M., \& Beardmore, R. (2012). Selecting against antibiotic-resistant pathogens: optimal treatments in the presence of commensal bacteria. Bulletin of Mathematical Biology, 74(4), 908-934.

\bibitem{rayner}
Rayner, C. R., Smith, P. F. et. al. (2021). Model informed drug development for anti‐infectives: state of the art and future. Clinical Pharmacology \& Therapeutics, 109(4), 867-891.

\bibitem{rao}
Rao, G. G., \& Landersdorfer, C. B. (2021). Antibiotic pharmacokinetic/pharmacodynamic modelling: zMIC, pharmacodynamic indices and beyond. International Journal of Antimicrobial Agents, 58(2), 106368.

\bibitem{regoes}
Regoes, R. R., Wiuff, C., Zappala, R. M., Garner, K. N., Baquero, F., \& Levin, B. R. (2004). Pharmacodynamic functions: a multiparameter approach to the design of antimicrobial treatment regimens. Antimicrobial Agents and Chemotherapy, 48(10), 3670-3676.

%\bibitem{Rogosinski}
%Rogosinski, W. W. (1958). Moments of non-negative mass. Proceedings of the Royal Society of London. Series A. Mathematical and Physical Sciences, 245(1240), 1-27.

\bibitem{rotschafer}
Rotschafer, J. C., Andes, D. R., \& Rodvold, K. A. (Eds.). (2016). Antibiotic Pharmacodynamics. Humana press.

\bibitem{singh}
Singh, G., Orman, M. A., Conrad, J. C., \& Nikolaou, M. (2023). Systematic design of pulse dosing to eradicate persister bacteria. PLoS Computational Biology, 19(1), e1010243.

\bibitem{smith}
Smith, N. M., Lenhard, J. R. et. al. (2020). Using machine learning to optimize antimicrobial combinations: dosing strategies for meropenem and polymyxin B against carbapenem-resistant Acinetobacter baumannii. Clinical Microbiology and Infection, 26(9), 1207-1213.

\bibitem{tindall}
Tindall, M., Chappell, M. J., \& Yates, J. W. (2022). The ingredients for an antimicrobial mathematical modelling broth. International Journal of Antimicrobial Agents, 106641.

\bibitem{ventola}
Ventola, C. L. (2015). The antibiotic resistance crisis: part 1: causes and threats. Pharmacy and Therapeutics, 40(4), 277.

\bibitem{vinks}
Vinks, A. A., Derendorf, H., \& Mouton, J. W. (Eds.). (2014). Fundamentals of antimicrobial pharmacokinetics and pharmacodynamics. Springer.

\bibitem{wen}
Wen, X., Gehring, R., Stallbaumer, A., Riviere, J. E., \& Volkova, V. V. (2016). Limitations of zMIC as sole metric of pharmacodynamic response across the range of antimicrobial susceptibilities within a single bacterial species. Scientific Reports, 6(1), 1-8.

%\bibitem{zhang}
%Zhang, L., Xie, H., Wang, Y., Wang, H., Hu, J., \& Zhang, G. (2022). Pharmacodynamic Parameters of Pharmacokinetic/Pharmacodynamic (PK/PD) Integration Models. Frontiers in Veterinary Science, 306.

\bibitem{zhi}
Zhi, J., Nightingale, C. H., \& Quintiliani, R. (1988). Microbial pharmacodynamics of piperacillin in neutropenic mice of systematic infection due to Pseudomonas aeruginosa. Journal of pharmacokinetics and biopharmaceutics, 16, 355-375.

\bibitem{zilonova}
Zilonova, E. M., \& Bratus, A. S. (2016). Optimal strategies in antibiotic treatment of microbial populations. Applicable Analysis, 95(7), 1534-1547.


\end{thebibliography}{}

\end{document}