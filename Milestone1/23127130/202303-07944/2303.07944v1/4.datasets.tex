\section{Datasets}
We use PURE~\cite{Stricker2014}, UBFC-rPPG~\cite{Bobbia2019}, and DDPM~\cite{Speth_IJCB_2021} as benchmark rPPG datasets for training and testing, and CelebV-HQ dataset~\cite{zhu2022celebvhq} and HKBU-MARs~\cite{Liu_2016_CVPRW} for unsupervised training only.

\noindent
\textbf{ Deception Detection and Physiological Monitoring (DDPM)}~\cite{Speth_IJCB_2021,Vance2022} consists of 86 subjects in an interview setting, where subjects attempted to answer questions deceptively. Interviews were recorded at 90 frames-per-second for more than 10 minutes on average. Natural conversation and frequent head pose changes make it a difficult and less-constrained rPPG dataset.

%\noindent
\textbf{PURE}~\cite{Stricker2014} is a benchmark rPPG dataset consisting of 10 subjects recorded over 6 sessions. Each session lasted approximately 1 minute, and raw video was recorded at 30 fps. The 6 sessions for each subject consisted of: (1) steady, (2) talking, (3) slow head translation, (4) fast head translation, (5) small and (6) medium head rotations. Pulse rates are at or close to the subject's resting rate.

%\noindent
\textbf{UBFC-rPPG}~\cite{Bobbia2019} contains 1-minute long videos from 42 subjects recorded at 30 fps. Subjects played a time-sensitive mathematical game to raise their heart rates, but head motion is limited during the recording.

%\noindent
\textbf{HKBU 3D Mask Attack with Real World Variations (HKBU-MARs)}~\cite{Liu_2016_CVPRW} consists of 12 subjects captured over 6 different lighting configurations with 7 different cameras each, resulting in 504 videos lasting 10 seconds each. The diverse lighting and camera sensors make it a valuable dataset for unsupervised training. We use version 2 of HKBU-MARs, which contains videos with both realistic 3D masks and unmasked subjects.

%\noindent
\textbf{High-Quality Celebrity Video Dataset (CelebV-HQ)}~\cite{zhu2022celebvhq} is a set of processed YouTube videos containing 35,666 face videos from over 15,000 identities. The videos vary dramatically in length, lighting, emotion, motion, skin tones, and camera sensors. The greatest challenge in harnessing online videos is their reduced quality due to compression before upload and by the video provider. Compression is a known challenge for rPPG, since the blood volume pulse is so subtle optically~\cite{McDuff2017,Nowara_ICCVW_2019,Rapczynski2019,Nowara_BOE_2021}.