\section{Discussion}

\subsection{Improvements over Supervised Learning}
It is initially surprising that unsupervised training leads to similar or improved rPPG estimation models compared to those trained in a supervised manner. However, there are several potential benefits to unsupervised training. From a hardware perspective, one of the difficulties in supervised training is aligning the contact pulse waveform with the video frames~\cite{Zhan2020}. The pulse sensor and camera may have a time lag, effectively giving the model an out-of-phase target at training time. Unsupervised training gives the model freedom to learn the phase directly from the video. The contact-PPG signal is also sensitive to motion and may be noisy. Since motion may co-occur at the face and fingertip, the contact signal may misguide the model towards visual features for which they should be invariant.

From a physiological perspective, the pulse observed optically at the fingertip with a contact sensor has a different phase than that of the face, since blood propagates along a different path before reaching the peripheral microvasculature, making alignment nearly impossible without shifting the targets to rPPG estimates from existing methods~\cite{Speth_CVIU_2021}. Additionally, the morphological shape of the contact-PPG waveform depends on numerous factors such as the wavelength of light (and corresponding tissue penetration depth), external pressure from the oximeter clip, and vasodilation at the measurement site~\cite{Moco2018,Abraham2013}. This indicates that the morphology and phase of the target PPG waveform is likely different from the observed rPPG waveform.

\subsection{Why Does It Work?}
The success of the proposed non-contrastive approach depends on specific properties of the data, model, and how the two interact.
Limited model capacity is actually a strength, since it forces discovering features to generalize across inputs.
An infinite capacity network could discover spurious signals in the training data and fail to generalize.
By constraining the model's predictions to have specific periodic properties the limited-capacity model must find a general set of features to produce a signal that exists in all of the training samples, which happens to be the blood volume pulse in our datasets.

As a beneficial side-effect, the model intrinsically learns to ignore common noise factors such as illumination, rigid motion, non-rigid motion (\eg talking, smiling, etc.), and sensor noise, since they may preside outside the predefined bandlimits or with uniform power spectra.
Even if noise exhibits periodic tendencies within the bandlimits for some samples, those features would produce poor signals on other samples.
Therefore, end-to-end unsupervised approaches are particularly well-suited for periodic problems.

