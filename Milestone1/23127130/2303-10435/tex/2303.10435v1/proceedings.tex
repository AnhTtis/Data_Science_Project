%%
%% This is file `sample-authordraft.tex',
%% generated with the docstrip utility.
%%
%% The original source files were:
%%
%% samples.dtx  (with options: `authordraft')
%% 
%% IMPORTANT NOTICE:
%% 
%% For the copyright see the source file.
%% 
%% Any modified versions of this file must be renamed
%% with new filenames distinct from sample-authordraft.tex.
%% 
%% For distribution of the original source see the terms
%% for copying and modification in the file samples.dtx.
%% 
%% This generated file may be distributed as long as the
%% original source files, as listed above, are part of the
%% same distribution. (The sources need not necessarily be
%% in the same archive or directory.)
%%
%% The first command in your LaTeX source must be the \documentclass command.
% \documentclass[sigconf,authordraft]{acmart}

%%%% As of March 2017, [siggraph] is no longer used. Please use sigconf (above) for SIGGRAPH conferences.

%%%% As of May 2020, [sigchi] and [sigchi-a] are no longer used. Please use sigconf (above) for SIGCHI conferences.

%%%% Proceedings format for SIGPLAN conferences 
% \documentclass[sigplan, anonymous, authordraft]{acmart}

%%%% Proceedings format for conferences using one-column small layout
%\documentclass[acmlarge,review,anonymous]{acmart}
% \documentclass[manuscript,review,anonymous]{acmart}
\documentclass[sigconf]{acmart}
% NOTE that a single column version is required for submission and peer review. This can be done by changing the \doucmentclass[...]{acmart} in this template to 
% \documentclass[manuscript,screen]{acmart}
\usepackage{multirow}
\usepackage{stfloats}
\usepackage{subcaption}
\usepackage{array}
\newcolumntype{M}[1]{>{\centering\arraybackslash}m{#1}}
\newcommand{\todo}[1]{\textcolor{teal}{\emph{{#1}}}}
\newcommand{\yuntao}[1]{\textcolor{green}{\emph{{#1}}}}
\newcommand{\zirui}[1]{\textcolor{orange}{\emph{{#1}}}}
\newcommand{\incomplete}[1]{\textcolor{red}{\emph{{#1}}}}

%%
%% \BibTeX command to typeset BibTeX logo in the docs
\AtBeginDocument{%
  \providecommand\BibTeX{{%
    \normalfont B\kern-0.5em{\scshape i\kern-0.25em b}\kern-0.8em\TeX}}}

%% Rights management information.  This information is sent to you
%% when you complete the rights form.  These commands have SAMPLE
%% values in them; it is your responsibility as an author to replace
%% the commands and values with those provided to you when you
%% complete the rights form.
% \setcopyright{acmcopyright}
% \copyrightyear{2021}
% \acmYear{2021}
% \acmDOI{10.1145/1122445.1122456}

% \acmJournal{IMWUT}
% \acmVolume{0}
% \acmNumber{0}
% \acmArticle{0}
% \acmYear{0000}
% \acmMonth{0}
% \acmArticleSeq{00}

%% These commands are for a PROCEEDINGS abstract or paper.
\copyrightyear{2023}
\acmYear{2023}
\setcopyright{rightsretained}
\acmConference[CHI '23]{Proceedings of the 2023 CHI Conference on Human Factors in Computing Systems}{April 23--28, 2023}{Hamburg, Germany}
\acmBooktitle{Proceedings of the 2023 CHI Conference on Human Factors in Computing Systems (CHI '23), April 23--28, 2023, Hamburg, Germany}
\acmDOI{10.1145/3544548.3581425}
\acmISBN{978-1-4503-9421-5/23/04}
%\acmISBN{}

%%
%% Submission ID.
%% Use this when submitting an article to a sponsored event. You'll
%% receive a unique submission ID from the organizers
%% of the event, and this ID should be used as the parameter to this command.
%%\acmSubmissionID{123-A56-BU3}

%%
%% The majority of ACM publications use numbered citations and
%% references.  The command \citestyle{authoryear} switches to the
%% "author year" style.
%%
%% If you are preparing content for an event
%% sponsored by ACM SIGGRAPH, you must use the "author year" style of
%% citations and references.
%% Uncommenting
%% the next command will enable that style.
%%\citestyle{acmauthoryear}

%%
%% end of the preamble, start of the body of the document source.
\begin{document}

%%
%% The "title" command has an optional parameter,
%% allowing the author to define a "short title" to be used in page headers.
%\title[Counterbalancing Visual Privacy Awareness...]{Counterbalancing Visual Privacy Awareness and Activities of Daily Living (ADLs) on Low-resolution Imagers}

%Optimizing Visual Privacy Preserving Machine Recognition on Low-resolution Image Sensors

\title[Modeling the Trade-off of Privacy ...]{Modeling the Trade-off of Privacy Preservation and Activity Recognition on Low-Resolution Images}

%\title[Exploring the Effect of Image Resolution ...]{Exploring the Effect of Image Resolution on Privacy-Preserving Machine Recognition}%of Activities of Daily Living (ADLs)}

% \title[Exploring Low Resolution Image Privacy Preserving  ...]{Exploring the Effect of Image Resolution on Visual Privacy Preserving Machine  Recognition of Activities of Daily Living (ADLs)}

%\title[Finding the Sweet Spot]{Understanding the Effect of Image Resolution on Visual Privacy Preserving Activities of Daily Living (ADLs) Recognition}

%Finding the Sweet Spot: Modeling the Effect of Image Resolution for Visual Privacy Preserving Machine Recognition in Daily Living Environments

%%
%% The "author" command and its associated commands are used to define
%% the authors and their affiliations.
%% Of note is the shared affiliation of the first two authors, and the
%% "authornote" and "authornotemark" commands
%% used to denote shared contribution to the research.

\author{Yuntao Wang}
\authornote{The authors contribute equally to this paper.}
\email{yuntaowang@tsinghua.edu.cn}
\affiliation{
  \institution{Key Laboratory of Pervasive Computing, Ministry of Education, \\ Department of Computer Science and Technology, \\ Tsinghua University}
  \city{Beijing}
  \country{China}}

\author{Zirui Cheng}\authornotemark[1]
\email{chengzr19@mails.tsinghua.edu.cn}
\affiliation{
  \institution{Department of Computer Science and Technology, \\ Tsinghua University}
  \city{Beijing}
  \country{China}}

\author{Xin Yi}\authornote{denotes as the corresponding author.}
\email{yixin@tsinghua.edu.cn}
\affiliation{
  \institution{Institute for Network Sciences and Cyberspace, \\ Tsinghua University}
  \city{Beijing}
  \country{China}}
\affiliation{
    \institution{Zhongguancun Laboratory}
    \city{Beijing}
    \country{China}
}

\author{Yan Kong}
\email{ky21@mails.tsinghua.edu.cn}
\affiliation{
  \institution{Institute for Network Sciences and Cyberspace, \\ Tsinghua University}
  \city{Beijing}
  \country{China}}

\author{Xueyang Wang}
\email{wang-xy22@mails.tsinghua.edu.cn}
\affiliation{
  \institution{Institute for Network Sciences and Cyberspace,\\ Tsinghua University}
  \city{Beijing}
  \country{China}}

\author{Xuhai Xu}
\email{xuhaixu@cs.washington.edu}
\affiliation{
  \institution{Information School, \\ University of Washington}
  \city{Seattle, WA}
  \country{USA}}

\author{Yukang Yan}
\email{yanyukanglwy@gmail.com}
\affiliation{
  \institution{Department of Computer Science and Technology, \\ Tsinghua University}
  \city{Beijing}
  \country{China}}

\author{Chun Yu}
\email{chunyu@mail.tsinghua.edu.cn}
\affiliation{
  \institution{Department of Computer Science and Technology, \\ Tsinghua University}
  \city{Beijing}
  \country{China}}

\author{Shwetak Patel}
\email{shwetak@cs.washington.edu}
\affiliation{
  \institution{Paul G. Allen School of Computer Science and Engineering, \\ University of Washington}
  \city{Seattle, WA}
  \country{USA}}

\author{Yuanchun Shi}
\email{shiyc@tsinghua.edu.cn}
\affiliation{
  \institution{Department of Computer Science and Technology, \\ Tsinghua University}
  \city{Beijing}
  \country{China}}
\affiliation{
  \institution{Qinghai University}
  \city{Xining, Qinghai}
  \country{China}}

%%
%% By default, the full list of authors will be used in the page
%% headers. Often, this list is too long, and will overlap
%% other information printed in the page headers. This command allows
%% the author to define a more concise list
%% of authors' names for this purpose.
\renewcommand{\shortauthors}{Yuntao Wang, et al.}

%%
%% The abstract is a short summary of the work to be presented in the
%% article.
\begin{abstract}

A computer vision system using low-resolution image sensors can provide intelligent services (e.g., activity recognition) but preserve unnecessary visual privacy information from the hardware level. However, preserving visual privacy and enabling accurate machine recognition have adversarial needs on image resolution. Modeling the trade-off of privacy preservation and machine recognition performance can guide future privacy-preserving computer vision systems using low-resolution image sensors. In this paper, using the at-home activity of daily livings (ADLs) as the scenario, we first obtained the most important visual privacy features through a user survey. Then we quantified and analyzed the effects of image resolution on human and machine recognition performance in activity recognition and privacy awareness tasks. We also investigated how modern image super-resolution techniques influence these effects. Based on the results, we proposed a method for modeling the trade-off of privacy preservation and activity recognition on low-resolution images. 

%A computer vision system using low-resolution image sensors can provide smart services (e.g., activity recognition) but preserve unnecessary visual privacy information from the hardware level.
%However, preserving visual privacy and enabling accurate machine recognition have adversarial needs on image resolution.
%It is unclear how to find the balanced image resolution that can optimize the two goals simultaneously.
%In this paper, using the at-home activity of daily livings (ADLs) as the scenario, we explored the effect of image resolution on privacy-preserving machine recognition. 
%We first obtained the most user-concerned visual privacy features through a user survey. Then we quantified how image resolution affects both the human's perception ability and the machine's recognition performance using cutting-edge machine learning methods in both activity recognition and privacy awareness tasks. 
%We also investigated how modern super-resolution techniques influence the results. 
%Based on the results, we proposed a method for modeling privacy-preserving machine recognition over image resolution. 
%We expect our work can inspire future privacy-preserving intelligent computer vision systems. 

%A low-resolution image sensor can achieve this purpose from the hardware level.
 %\todo{Need to describe the method.}
 %Understanding the effect of image resolution on the main service task and the awareness of privacy is essential for privacy preserving  applications. 
%There is a trade-off between the main visual recognition task and the privacy preserving need. 
% \changed{chose recognition of activities of daily living (ADLs), one crucial assessment method for behavior tracking and health monitoring, to build the resolution model for visual privacy preserving intelligent applications.}
% Based on the results, we proposed a method for calculating a proper resolution range that can maximize the machine's ADLs recognition performance while preserving visual privacy. 
%\changed{However, no existing research explored an optimal image resolution range to balance both machine recognition performance and visual privacy preservation capability. }

% Low-resolution image sensor makes it possible for visual privacy preserving machine recognition. However, current researches are insufficient on  determining an optimal image resolution, on which we can balance both the machine recognition performance and the visual privacy preserving capability. In this paper, we chose Activities of Daily Living (ADLs) in a home environment as an example to explore the image resolution model for visual privacy preserving intelligent applications. We firstly obtained the most user-concerned visual privacy features through a survey. Then we modeled the effect of image resolution on both human's ADLs recognition performance and visual privacy awareness through another user study. Also, we empirically modeled the effect of image resolution on cutting-edge machine learning approaches for ADLs recognition. Based on the results, we proposed the  optimal range of image resolution that can minimize the user's visual privacy awareness, while preserving the machine's ADLs recognition performance..
% A low-resolution image sensor makes it possible for visual-privacy-preserving machine recognition. However, current research is insufficient to determine an optimal image resolution, on which we can balance both the machine recognition performance and the visual-privacy-preserving capability. In this paper, we chose Activities of Daily Living (ADLs) in a home environment as an example to explore an effective modeling method for visual privacy, preserving intelligent applications. Therefore, we firstly obtained the most user-concerned visual privacy features through a user study. Then we modeled the effects of image resolution on human’s ADLs recognition performance and visual privacy awareness through another user study. As much, we modeled the impact of image resolution on cutting-edge machine learning approaches for ADLs recognition. Finally, we investigated the optimal resolution calculation method by solving the built model.

% version - Aug. 30, 2021
%A low-resolution image sensor makes it possible for visual-privacy-preserving machine recognition. However, current research is insufficient to determine an optimal image resolution, on which we can balance the machine recognition performance and the visual privacy preserving capability. In this paper, we chose recognition of activities of daily living (ADLs), which is important for health monitoring and assessment, to explore the image resolution model for visual privacy preserving intelligent applications. We first obtained the most user-concerned visual privacy features through a user survey. Then we modeled the effect of image resolution on ADLs recognition performance and visual privacy awareness from both the human and the machine with cutting-edge machine learning methods. %Also, we empirically modeled the impact of image resolution on cutting-edge machine learning approaches for ADLs recognition. 
%On this basis, we proposed the modeling method for calculating the optimal resolution range that can maximize the machine's ADLs recognition performance while preserving the visual privacy. We envision that our method can be easily applied or adapted to other vision-based smart applications that require to balance privacy awareness and machine recognition performance. 

%A low-resolution image sensor makes it possible for visual privacy preserving machine recognition. 

%A low-resolution image sensor makes it possible for visual privacy preserving machine recognition. However, current research is insufficient to determine an optimal image resolution, on which we can balance both the machine recognition performance and the visual-privacy-preserving capability. In this paper, we chose Activities of Daily Living (ADLs) in a home environment as an example to explore the image resolution model for visual-privacy-preserving intelligent applications. We firstly obtained the most user-concerned visual privacy features through a survey. Then we modeled the effect of image resolution on both human's ADLs recognition performance and visual privacy awareness through another user study. Also, we empirically modeled the impact of image resolution on cutting-edge machine learning approaches for ADLs recognition. Based on the results, we proposed the modeling method for calculating the optimal resolution that can minimize the user's visual privacy awareness while preserving the machine's ADLs recognition performance.
\end{abstract}



%%
%% The code below is generated by the tool at http://dl.acm.org/ccs.cfm.
%% Please copy and paste the code instead of the example below.
%%
\begin{CCSXML}
<ccs2012>
   <concept>
       <concept_id>10003120.10003121.10003122.10003334</concept_id>
       <concept_desc>Human-centered computing~User studies</concept_desc>
       <concept_significance>500</concept_significance>
       </concept>
   <concept>
       <concept_id>10002978.10003029.10011150</concept_id>
       <concept_desc>Security and privacy~Privacy protections</concept_desc>
       <concept_significance>500</concept_significance>
       </concept>
   <concept>
       <concept_id>10010147.10010178.10010224</concept_id>
       <concept_desc>Computing methodologies~Computer vision</concept_desc>
       <concept_significance>500</concept_significance>
       </concept>
 </ccs2012>
\end{CCSXML}

\ccsdesc[500]{Human-centered computing~User studies}
\ccsdesc[500]{Security and privacy~Privacy protections}
\ccsdesc[500]{Computing methodologies~Computer vision}

%%
%% Keywords. The author(s) should pick words that accurately describe
%% the work being presented. Separate the keywords with commas.
\keywords{Privacy, visual privacy, privacy preserving, activities of daily living, ADLs, low-resolution image.}

%% A "teaser" image appears between the author and affiliation
%% information and the body of the document, and typically spans the
%% page.

%%
%% This command processes the author and affiliation and title
%% information and builds the first part of the formatted document.
\maketitle

\section{Introduction}
\IEEEPARstart{T}{he} method Neural Radiance Fields (NeRF)~\cite{mildenhall2020nerf} is proposed for photorealistic novel view synthesis. Given many views of the scene, it creates implicit multi-view geometry and learns for view synthesis. However, it has poor generalizations to new scenes and requires retraining or fine-tuning on each scene. 
 
 Recent work~\cite{Yu_2021_CVPR,Trevithick_2021_ICCV} has explored the ways of using a single image to train NeRF. They introduce a convolutional feature encoder to learn the image representation which gives it some limited generalization abilities to unseen scenes.  But, without fine-tuning, these methods produce many floats and artifacts in rendering novel views. 
 
  Multi-Plane Images (MPI) representation that learns multiple RGB images from a single image is also used in \cite{Wu_2021_ICCV,Tucker_2020_CVPR,wu2022remote} for  novel view synthesis. However, MPI heavily relies on the qualities of the planar images and needs plenty of image planes to avoid blurs. There is no strong 3D geometry constraint and it fails in many complex scenes.
  
  MINE~\cite{Li_2021_ICCV2} introduces the volume rendering of NeRF into the MPI. It runs faster and produces better depth rendering quality compared with single-view NeRFs~\cite{Yu_2021_CVPR,Trevithick_2021_ICCV}. However, the rendering quality heavily relies on the number of image planes. It needs high-resolution 4D volumes to store the 4-channel  (RGB and volume density) image planes that cost a large amount of GPU memory in both training and 
 prediction.  
 

 
 \begin{figure}[t]
\setlength{\abovecaptionskip}{7pt}
\setlength{\belowcaptionskip}{0pt}
	\centering
% 	\subfigure[MINE (PSNR:14.9)]{  % for AAAI
	\subfloat[MINE (PSNR:14.9)]{
%			\centering
			\includegraphics[width=0.23\textwidth]{figure/intro/DJI_20200223_163206_598_0_MINE.png}
%			\label{subfig:pixelnerf}
	}\subfloat[MINE (depth)]{
%			\centering
			\includegraphics[width=0.23\textwidth]{figure/intro/MINE_disp.png}
%			\label{subfig:mpi}
	}
	\\[-3mm]
	\subfloat[Ours (PSNR:17.0)]{
%			\centering
			\includegraphics[width=0.23\textwidth]{figure/intro/DJI_20200223_163206_598_0_ours.png} 
	}\subfloat[Ours (depth)]{
%			\centering
			\includegraphics[width=0.23\textwidth]{figure/intro/ours_disp.png}
	}
	\caption{Comparison with state-of-the-art methods. (a-b) RGB and depth rendering results of  \cite{Li_2021_ICCV2}. It produces many blurs and floats in the occluded regions and at the object/depth edges. 
	(c-d) Our method employs a joint rendering mechanism that preserves more image details and predicts sharp depth edges.}
	\label{fig:performance_illustration}
\end{figure}
 
 In this paper, we propose a joint rendering mechanism that takes the MPI strategy for coarse sampling proposals and the MLP\&volume-based rendering~\cite{mildenhall2020nerf} for fine sampling and rendering. Then, both the coarse point samples and the fine samples are combined according to their geometry distribution to realize a more accurate joint rendering. More importantly, we introduce a depth teacher net that serves as the guidance for the joint rendering. The monocular depth teacher predicts dense pseudo depth maps that assist the consistent 3D geometry learning between the MPI, the fine volume, and the joint rendering. It also boosts the multi-view geometry consistency between the source view and the target novel views that 
helps handle the occlusions, reduce the blurs and floats, and render accurate depths. 
 
In the experiments,  we verify the effectiveness of our method on three challenging real-scene datasets (RealEstate10K~\cite{zhou2018stereo}, NYU~\cite{silberman2012indoor} and  NeRF-LLFF~\cite{mildenhall2020nerf}) for novel view synthesis or depth estimation. Given a single image as input, our method is shown able to produce higher qualities in both the RGB image rendering and depth map prediction. It far outperforms state-of-the-art methods~\cite{Li_2021_ICCV2,Yu_2021_CVPR} with improvements of 5$\sim$20\% in PSNR and SSIM for the RGB rendering and reduces 20$\sim$50\% of the errors for the depth prediction. 
\section{Related Work}

\subsection{Video Summarization}
 Video summarization datasets (\eg,  SumMe~\cite{gygli2014creating}, TVSum~\cite{song2015tvsum}, and YouTube~\cite{de2011vsumm}) have enabled the development of state-of-the-art video summarization methods~\cite{narasimhan2021clip,zhang2016video,zhou2018deep,park2020sumgraph,saquil2021multiple}. Among these models, vsLSTM \cite{zhang2016video} first attempted to learn frame importance by modeling the temporal dependency among frames using LSTM \cite{graves2012long} units. The model can be combined with a determinantal point process (DPP) to improve the diversity of generated video summary. Following vsLSTM, several other approaches were proposed to model the temporal dependency, e.g., H-RNN~\cite{zhao2017hierarchical}, HSA-RNN\cite{zhao2018hsa}, DASP~\cite{ji2020deep}. Another solution models the spatiotemporal structure of the video to learn frame importance, such as MerryGoRoundNet~\cite{lal2019online}, and CRSum \cite{yuan2019spatiotemporal}. Adversarial learning-based methods~\cite{fu2019attentive,zhang2019dtr} can also perform well. Recently, multimodal-based video summarization method~\cite{narasimhan2021clip} leverages generated text summaries to promote predictions of frame-level scores for video summaries. Different from multimodal-based video summarization, the cross-modal video summarization task requires simultaneously producing both visual and textual summaries from a source video, which goes beyond generic video summarization. Moreover, it ensures semantic coherence between these two modalities.


\vspace{-2mm}
\subsection{Video Captioning}
% Video Captioning aims to automatically generate short descriptions for a video by understanding the action and event in a video, which can help retrieve videos efficiently through text queries.
Video Captioning aims to describe a video with text, which requires the capability of understanding actions and events.
Existing benchmarks (e.g., MSVD~\cite{chen2011collecting}, YouCook~\cite{zhou2018towards}, MSR-VTT~\cite{xu2016msr}, and ActivityNet Captions \cite{krishna2017dense}) have helped to promote the ability of language models to generate reasonable captions for video. Benefiting from these human-annotated datasets, many novel approaches are proposed. 
Attention-based methods~\cite{yao2015describing,yan2019stat} employ attention mechanisms to help the model in associating relevant frames since not every frame in a video is equally important.
%Graph-based video captioning~\cite{zhang2019object,zhang2020video} leverage inter-frame and intra-frame object interactions in both time and space degree. 
DENSE~\cite{krishna2017dense} is an early attempt at dense video captioning, which detects events with an event proposal module and associates them with LSTM. Wang et al.~\cite{wang2018bidirectional} develop a bidirectional process to encode context for detecting event proposals. Moreover, Masked Transformer~\cite{zhou2018end} proposes a differentiable masking scheme to ensure consistency between event proposal and caption generation modules.




\vspace{-2mm}
\subsection{Multimodal Pretraining}
Large language models (LLMs) \cite{brown2020language,devlin2018bert,lewis2019bart,raffel2020exploring} have revolutionized NLP research in recent years. Following the large-scale pretraining models in the field of NLP, numerous works \cite{ju2022prompting,kim2021vilt,wang2020minilm,xue2021probing,zhang2021vinvl,hu2022promptcap} on exploring the combination of vision and language (VL) pretraining have achieved great success. Since then, image-text pretraining has become
a default approach to tackling VL tasks \cite{biten2019scene,lin2014microsoft,regneri2013grounding,singh2019towards}. In addition, the introduction of Vision Transformers \cite{dosovitskiy2020image} enables vision and language modalities to be jointly modeled by Transformers in a more scalable fashion \cite{alayrac2022flamingo,wang2022git,yu2022coca,yuan2021florence}. According to the encoding strategies for image and language modalities, VL models can be categorized into fusion encoder \cite{li2019visualbert,lu2019vilbert,su2019vl,tan2019lxmert}, dual encoder \cite{radford2021learning}, and a combination of both \cite{bao2021vlmo,du2022survey,singh2022flava}. Several video-language pretrained models have also shown strong performance on video captioning and other video tasks, such as HERO \cite{li2020hero}, VideoBERT \cite{sun2019videobert}, and UniVL \cite{luo2020univl}.
In this work, cross-modal video summarization requires models with strong video understanding and language modeling capabilities. Therefore, this new task provides a practical scenario to assess the superiority of multimodal pretrained models.


The ARMBench dataset presents: 1) a collection of sensor data acquired by a robotic manipulation workcell performing pick-and-place operation, 2) metadata and reference images for objects in containers, 3) a set of annotations acquired either automatically, by virtue of the system design, or via manual labeling, and 4) tasks and metrics to benchmark perception algorithms for robotic manipulation. Fig.\ \ref{fig:contributions} illustrates the benchmark tasks and variety of objects captured in the dataset. The dataset captures diversity in objects with respect to Amazon product categories as well as physical characteristics such as size, shape, material, deformability, appearance, fragility, etc. 

The data collection platform is a robotic manipulation workcell performing pick-and-place operation in a warehouse \cite{Sparrow2022}. The workcell contains a robotic arm mounted with a vacuum-based end-effector. It is presented with a heterogeneous collection of objects placed in unstructured configurations within a container (storage tote). The robotic arm is tasked with picking one object at a time (singulation) and place it on moving trays until the container is empty. The empty container ejects the workcell and is replaced by a new container. While the operation is completely autonomous, it includes a human-in-the-loop to monitor the status of each pick-and-place activity, annotate, and resolve any defects during manipulation. Multiple imaging sensors are placed in the workcell to facilitate and validate the pick-and-place operation. Following is a list of sensor data (Fig.\ \ref{fig:intro}) associated with each pick activity:
\begin{itemize}
\item Pick-image: A 5\,MP camera is used to capture a top-down image of the container.
% \item Pick-3D: Two Ensenso sensors capture the 3D point cloud of the source container.
\item Transfer-images: Multiple 5\,MP cameras are placed on different sides in the workcell to capture the moving object from different viewpoints.
% \item Transfer-Barcode: Multiple Cognex barcode sensors are used to scan the barcode of the object during transfer.
\item Place-image: A top-down view of the object is captured once it is placed on the tray.
\item Video: A camera is mounted to capture 720p videos of pick-and-place manipulation processes at 30\,FPS
\end{itemize}
Additionally, the following metadata (Fig.\ \ref{fig:contributions} (b)) is available by virtue of a warehouse tracking system:
\begin{itemize}
\item Container-manifest: A list of objects present in the container along with data such as product description, coarse dimensions, and weight.
\item Reference images: One or more images of objects from previous operations within the warehouse.
\end{itemize}
The sensor data and metadata were consumed by perception algorithms required to autonomously operate the robotic workcell. Benchmarking against these algorithms would not only optimize a manipulation task such as the one used for data collection but also enable more complex and intentional manipulation. This work considers a subset of such perception tasks namely object segmentation, object identification, and defect detection. These are critical not only to make informed grasping and motion decisions but also to track the state of the objects and containers within the warehouse. The following sections will describe these tasks and present the challenges using annotations, baseline algorithms, and evaluation metrics.

\vspace{-0.3em}
\section{Method}
\vspace{-0.3em}

Our sensitivity-aware visual parameter-efficient fine-tuning consists of two stages. In the first stage, SPT measures the task-specific sensitivity for the pre-trained parameters (Section~\ref{subsec:sensitivity}). Based on the parameter sensitivity and a given parameter budget, SPT then adaptively allocates trainable parameters to task-specific important positions (Section~\ref{subsec:SPT}).

\vspace{-0.3em}
\subsection{Task-specific Parameter Sensitivity}
\label{subsec:sensitivity}
\vspace{-0.3em}

Recent research has observed that pre-trained backbone parameters exhibit varying feature patterns~\cite{raghu2021vision,naseer2021intriguing} and criticality~\cite{zhang2019all,chatterji2019intriguing} at distinct positions. 
Moreover, when transferred to downstream tasks, their efficacy varies depending on how much pre-trained features are reused and how well they adapt to the specific domain gap~\cite{yosinski2014transferable,kumar2022finetuning,neyshabur2020being}. Motivated by these observations, we argue that not all parameters contribute equally to the performance across different tasks in PEFT and propose a new criterion to measure the sensitivity of the parameters in the pre-trained backbone for a given task.

Specifically, given the training dataset $\gD_t$ for the $t$-th task and the pre-trained model weights $\vw=\left\{w_1, w_2, \ldots, w_N\right\}\in \sR^N$ where $N$ is the total number of parameters, the objective for the task is to minimize the empirical risk: $\min_{\vw} E(\gD_t, \vw)$.
We denote the parameter sensitivity \bohan{set} as $\gS=\{s_1, \ldots, s_N\}$ and the sensitivity $s_n$ for parameter $w_n$ is measured by the empirical risk difference when tuning it:
\begin{equation}
\vspace{-0.3em}
    \begin{aligned}
        s_n = E(\gD_t, \vw)-E(\gD_t, \vw\mid w_n=w_n^*),
    \end{aligned}
\label{eq:sensitivity}
\end{equation}
where $w_n^*=\underset{w_n}{\rm argmin}(E(\gD_t, \vw))$. We can reparameterize the tuned parameters as  $w_n^*=w_n+\Delta_{w_n}$, where $\Delta_{w_n}$ denotes the update for $w_n$ after tuning. Here we individually measure the sensitivity of each parameter, which is reasonable given that most of the parameters are frozen during fine-tuning in PEFT. However, it is still computationally intensive to compute Eq.~(\ref{eq:sensitivity}) for two reasons. Firstly, getting the empirical risk for $N$ parameters requires forwarding the entire network $N$ times, which is time-consuming. Secondly, it is challenging to derive $\Delta_{w_n}$, as we have to tune each individual $w_n$ until convergence.

{\begin{algorithm}[t!]
\caption{\label{alg:tps} Computing task-specific parameter sensitivities}
\begin{algorithmic}
    \STATE \textbf{Input:} Pre-trained model with network parameters $\vw$, training set $\gD_t$ for the $t$-th task, and number of training samples $C$ used to calculate the parameter sensitivities
    \STATE \textbf{Output:} Sensitivity set $\gS=\{s_1, \ldots, s_N\}$
    \STATE Initialize $\gS=\{0\}^N$
    \FOR{$i\in\{1,\ldots,C\}$}
        \STATE Get the $i$-th training sample of $\gD_t$
	    \STATE Compute loss $E$
		\STATE Compute gradients $\vg$
		\FOR{$n\in\{1,\ldots,N\}$}
                \STATE Update sensitivity for the $n$-th parameter: $s_{n} = s_{n} + g_n^2$
		    \ENDFOR
    \ENDFOR
\end{algorithmic}
\end{algorithm}}


\begin{figure*}[t]
\begin{center}
    \includegraphics[width=\linewidth]{main_figure.pdf}
\end{center}\vspace{-2em}
\caption{Overview of our trainable parameter allocation strategy. With the parameter sensitivity \bohan{set} $\gS$, we first get the top-$\tau$ sensitive parameters. Instead of directly tuning these sensitive parameters, we also boost the representational capability by replacing unstructured tuning with structured tuning at sensitive weight matrices that have a large number of sensitive parameters, which can be implemented by an existing structured tuning method, \eg, LoRA~\cite{hu2022lora} and Adapter~\cite{houlsby2019parameter}. Red lines and blocks represent trainable parameters and modules, while blue lines represent frozen parameters.}
\label{fig:main}
\vspace{-1.5em}
\end{figure*}


To overcome the first barrier, we simplify the empirical loss by approximating $s_n$ in the vicinity of $\vw$ by its first-order Taylor expansion
\vspace{-0.3em}
\begin{equation}
\vspace{-0.5em}
    \begin{aligned}
        s_n^{(1)} = -g_n\Delta_{w_n},
    \end{aligned}
\label{eq:first-order}
\end{equation}
where the gradients $\vg=\partial E/\partial\vw$, and $g_n$ is the gradient of the $n$-th element of $\vg$. 
To address the second barrier, following~\cite{liu2018darts,cai2018proxylessnas}, we take the one-step unrolled weight as the surrogate for $w_n^*$ and approximate $\Delta_{w_n}$ in Eq.~(\ref{eq:first-order}) with a single step of gradient descent. We can accordingly get $s_n^{(1)} \approx g_n^2\epsilon$,
where $\epsilon$ is the learning rate. Since $\epsilon$ is the same for all parameters, we can eliminate it when comparing the sensitivity with the other parameters and finally get 
\vspace{-0.5em}
\begin{equation}
\vspace{-0.3em}
    \begin{aligned}
        s_n^{(1)} \approx g_n^2.
    \end{aligned}
\label{eq:first-order-simp}
\end{equation}
Therefore, the sensitivity of a parameter can be efficiently measured by its potential to reduce the loss on the target domain. Note that although our criterion draws inspiration from pruning work~\cite{molchanov2019importance}, it is distinct from it. \cite{molchanov2019importance} measures the parameter importance by the squared change in loss when removing them, \ie, $\left( E(\gD_t, \vw)-E(\gD_t, \vw\mid w_n=0) \right)^2$ and finally derives the parameter importance by $\left( g_n w_n \right)^2$, which is different from our formulations in Eqs.~(\ref{eq:sensitivity}) and~(\ref{eq:first-order-simp}).

In practice, we accumulate $\gS$ from a total number of $C$ training samples ahead of fine-tuning to generate accurate sensitivity as shown in Algorithm~\ref{alg:tps}, where $C$ is a pre-defined hyper-parameter. In Section~\ref{subsec:abl}, we show that employing only 400 training samples is sufficient for getting reasonable parameter sensitivity, which requires only 5.5 seconds with a single GPU for any VTAB-1k dataset with ViT-B/16 backbone~\cite{vit}.

\vspace{-0.3em}
\subsection{Adaptive Trainable Parameters Allocation}
\label{subsec:SPT}
\vspace{-0.2em}

Our next step is to allocate trainable parameters based on the obtained parameter sensitivity set $\gS$ and a desired parameter budget $\tau$. A straightforward solution is to directly tune the top-$\tau$ most sensitive unstructured connections (parameters) \rev{while keeping the rest frozen}, which we name unstructured tuning. Specifically, we select the top-$\tau$ most sensitive weight connections in $\gS$ to form the sensitive weight connection set $\gT$. Then, for \rev{a} weight matrix $\mW\in \sR^{d_{\rm in}\times d_{\rm out}}$, we can get a binary mask $\mM\in \sR^{d_{\rm in}\times d_{\rm out}}$ computed by
\vspace{-0.5em}
\begin{equation}
\vspace{-0.5em}
    {\begin{array}{ll}
    \small
    \begin{aligned}
    \mM^j =
    \left\{\begin{array}{ll} 
    1 ~~~~~ \mW^j \in \gT \\
    0 ~~~~~ \mW^j \notin \gT
    \end{array}\right.
    \end{aligned},
    \small
    \end{array}}
\label{eq:mask}
\end{equation}
where $\mW^j$ and $\mM^j$ are the $j$-th element in $\mW$ and $\mM$, respectively. Accordingly, we can train the sensitive parameters by gradient descent and the updated weight matrix can be formulated as $\mW'\leftarrow \mW - \epsilon\vg_{\mW}\odot\mM$, where $\vg_{\mW}$ is the gradient for $\mW$.

However, considering PEFT approaches generally limit the proportion of trainable parameters to less than 1\%, tuning only a small number of unstructured weight connections might not have enough representational capability to handle the downstream datasets with large domain gaps from the source pre-training data. Therefore, to improve the representational capability, we propose to replace unstructured tuning with structured tuning at the sensitive weight matrices that have a high number of sensitive parameters. To preserve the parameter budget, we can implement structured tuning with an existing efficient structured tuning PEFT method~\cite{hu2022lora,chen2022adaptformer,houlsby2019parameter,jie2022convolutional} that learns to directly adjust \rev{all hidden dimensions at once}. We depict an overview of our trainable parameter allocation strategy in Figure~\ref{fig:main}. For example, we can employ the low-rank reparameterization trick LoRA~\cite{hu2022lora} to the sensitive weight matrices \rev{and the one-step update for $\mW$ can be formulated as}
\vspace{-0.4em}
\begin{equation}
\vspace{-0.4em}
    {\begin{array}{ll}
    \small
    \begin{aligned}
    \mW' = \left\{\begin{array}{ll} 
    \mW + \mW_{\rm down}\mW_{\rm up} & ~~ \text { if } ~~ \sum_{j=0}^{d_{\rm in}\times d_{\rm out}} \mM^j \geq \sigma_{\rm opt} \\
    \mW - \epsilon\vg_{\mW}\odot\mM & ~~ {\rm otherwise}
    \end{array}\right.
    \end{aligned},
    \small
    \end{array}}
\label{eq:weight_updat}
\end{equation}
where $\mW_{\rm down}\in \sR^{d_{\rm in}\times r}$ and $\mW_{\rm up}\in \sR^{r\times d_{\rm out}}$ are two learnable low-rank matrices to approximate the update of $\mW$ and rank $r$ is a hyper-parameter where $r \ll {\rm min}(d_{\rm in},d_{\rm out})$. In this way, we perform structured tuning on $\mW$ when its number of sensitive parameters exceeds $\sigma_{\rm opt}$, whose value depends on the pre-defined type of structured tuning method. For example, since implementing structured tuning with LoRA requires $2\times d_{\rm in} \times d_{\rm out} \times r$ trainable parameters for each sensitive weight matrix, we set $\sigma_{\rm LoRA} \leftarrow 2\times d_{\rm in} \times d_{\rm out} \times r$ to ensure that the number of trainable parameters introduced by structured tuning is always equal to or lower than the number of sensitive parameters.

In this way, our SPT adaptively incorporates both structured and unstructured tuning granularities to enable higher flexibility and stronger representational power, simultaneously. In Section~\ref{subsec:abl}, we show that structured tuning is important for the downstream tasks with larger domain gaps and both unstructured and structured tuning contribute clearly to the superior performance of our SPT.
\section{Quantifying the Importance of Visual Privacy Features}
\label{sec:study1}
Our first user study aims to understand what visual privacy features users value the most and quantify the importance of those visual privacy features, thus simplifying the to-be-built model (Equation~\ref{eq:problem}). Inspired by related works~\cite{Orekondy-privacy-advisor,categories-of-privacy,Li-Human-Perception}, we obtained 25 visual privacy features that exist in a home environment. Here we divide them into 5 categories as below. 

\begin{itemize}
    \item \textbf{Biometric Identification}: identifiable face, gender, skin color, age group, weight group, hair color, eye color, and height group.
    \item \textbf{Personal Marker/Information}: nudity, home address, number/code, medical treatment, physical disability, handwriting, birthday, clothing, and tattoo.
    \item \textbf{Ethnicity}: religion, race, and nationality.
    \item \textbf{Society}: relationship, employment and pet.
    \item \textbf{Safety}: valuable property and living schedule.
\end{itemize}

%We conducted a user survey to explore user perceived importance of these visual privacy features and categories. Therefore, we can focus on the most user-concerned visual privacy features, 

% \input{tables/privacy_category.tex}

% \subsection{Pilot Study to Identify the Rating Method}
% We first conducted a pilot study to explore the most suitable rating method for the quantitative user survey. We tested three methods: 1) 5-point Likert scale rating; 2) 100-point rating in a slider design with a range from 1 to 100; and 3) participants sort a list of tags in the order of visual privacy. We also applied attention check questions to filter out invalid answers. 

% We recruited 40 participants (23 males and 17 female) with an average age of 31.2 (s.d. = 10.4) from Amazon Mechanical Turk (MTurk)~\footnote{https://www.mturk.com/}. Each participants rated the 25 visual privacy features with three rating methods. The whole user survey lasted around 20 minutes. Each participant who passed the attention check received a 8 USD Amazon gift card. We gathered and analyzed their answers and feedback. Results show that participants preferred scoring strategy considering the difficulty of sorting 25 visual privacy features. Specifically, 32 participants chose 100-point scale and 24 participants chose 5-point Likert scale as their most preferred rating method. Further, results show that there is no statistic significance among the top 15 rated visual privacy features. This indicates that 5-point Likert scale is not sufficient to differentiate the priority of 25 visual privacy features when compared with 100-point scale. Therefore, we utilized 100-point scale for the user survey below. 

\subsection{User Survey on Importance of Visual Privacy Features} 
We recruited 125 participants (66 females, 59 males) from MTurk. They had an average age of 32.7 (s.d. = 14.7). The whole survey lasted around 15 minutes. Each participant who passed the attention check received a 6 USD Amazon gift card. 

In the user survey hosted by Qualtrics~\footnote{https://www.qualtrics.com/}, we first introduced the smart home scenario where cameras are installed for ADLs recognition. Then we asked the participants to assume that they were living in the demonstrated house/apartment. 
% \input{tikz/rank_interface.tex}
% The user interface is illustrated in Figure~\ref{fig:rank_interface}. 
% On the one hand, each participant was required to finish the question of rating the importance of 5 privacy categories using 5-point Likert scale. The participant could click the options from the most important to the least important one by one with 1 for the most important and 5 for the least important. This is to explore how user value each category of visual privacy shown in the left column of Table~\ref{tab:privacy_category}. 
Then we evaluated the importance of each privacy feature with or without low-resolution to find out what privacy features users value the most and explore the effect of low-resolution on users' perceived importance of privacy features. 
For the high resolution test, we showed participants five high-resolution ($300 \times 300$) images. Each image captured one of the five basic daily activities: functional mobility, feeding, intimacy, entertainment, and personal hygiene. We did not control the participants' backgrounds regarding their culture, age, gender, and technical knowledge. In the instruction, we explicitly stated the scenario of visual privacy leakage as their similar pictures were posted on the Internet and thus can be accessed by everyone.
%\changed{We pointed to them explicitly that the images they saw in the questionnaire which contained certain types of privacy information could have been leaked to the public and possibly go everywhere on the Internet. Despite the difference in technical background of the participants which may have an impact on their perception of privacy as suggested by~\cite{kang2015everywhere}, we have tried out best to ensure that the way and content of privacy leakage understood by the participants are consistent in our study.} 
For the low-resolution test, we just showed participants the same five images in low-resolution ($50 \times 50$).
Under each resolution, the participant was asked how he/she values the importance of the different visual privacy information listed in the questionnaire. Then, the participants were required to rate the importance of each visual privacy feature using a 100-point slider where 0 stands for not important at all and 100 stands for extremely important. The score of each privacy feature shown on the slider updates along with the participant's choice. 

We designed two attention check questions under each condition. Each attention check question requires the participant to slide to a certain score that was generated randomly before each survey. All the questions were provided to the participant in random order. 
\begin{table*}[!ht] 
    \small
    \centering
    \caption{The statistic of the user rated importance scores of the 25 visual privacy features in 5 categories with and without the low-resolution conditions. $p < 0.05$ indicates significant difference between high and low resolution conditions. }
    \begin{tabular}
    {|M{0.2\textwidth}|M{0.17\textwidth}|M{0.07\textwidth}|M{0.07\textwidth}|M{0.07\textwidth}|M{0.07\textwidth}|c|}
    % {|c|c|c|c|c|c|c|}
    \hline
    
    \multirow{2}{*}{\textbf{Category}} & \multirow{2}{*}{\textbf{Feature}} & \multicolumn{2}{c|}{\textbf{High Resolution}} & \multicolumn{2}{c|}{\textbf{Low Resolution}} & \multirow{2}{*}{\textbf{Significance}} \\
    
    \cline{3-6} & & avg. & std. & avg. & std. & \\
    
    \hline
    \multirow{8}{*}{Biometric Identification} & Identifiable Face & 60.2 & 24.3 & \textbf{57.5} & 26.0 & $p = 0.13$\\
    \cline{2-7} & Gender & 43.5 & 29.2 & 43.4 & 29.4 & $p = 0.81$\\
    \cline{2-7} & Skin Color & 42.0 & 28.6 & 43.1 & 27.3 & $p = 0.94$\\
    \cline{2-7} & Age Group & 42.9 & 25.1 & 41.2 & 25.8 & $p = 0.35$\\
    \cline{2-7} & Weight Group & 43.9 & 27.2 & 40.9 & 27.2 & $p = 0.16$\\
    \cline{2-7} & Hair Color & 36.2 & 27.4 & 40.9 & 28.1 & $p = 0.05$\\
    \cline{2-7} & Eye Color & 40.4 & 28.9 & 40.3 & 28.4 & $p = 0.90$\\
    \cline{2-7} & Height Group & 37.3 & 25.8 & 40.0 & 27.7 & $p = 0.30$\\
      % \toprule
    \hline
    \multirow{9}{*}{\parbox{0.15\textwidth}{ \centering Personal Marker / Information}} & Nudity & 61.6 & 30.9 & \textbf{62.9} & 29.4 & $p = 0.71$\\
    \cline{2-7} & Home Address & 62.8 & 23.1 & 55.6 & 26.1 & $p = 0.01$\\
    \cline{2-7} & Number/code & 57.5 & 25.5 & 55.6 & 26.6 & $p = 0.79$\\
    \cline{2-7} & Medical Treatment & 60.4 & 23.2 & 51.7 & 25.9 & $p < 0.001$\\
    \cline{2-7} & Physical Disability & 52.1 & 25.1 & 49.4 & 26.0 & $p = 0.25$\\
    \cline{2-7} & Hand Writing & 52.6 & 26.4 & 44.9 & 27.7 & $p < 0.01$\\
    \cline{2-7} & Birthday & 54.2 & 26.8 & 44.7 & 28.5 & $p < 0.01$\\
    \cline{2-7} & Clothing & 40.5 & 27.9 & 41.5 & 27.5  & $p = 0.94$\\
    \cline{2-7} & Tattoo & 42.2 & 28.7 & 39.2 & 28.6 & $p = 0.34$\\
    \hline
    
    \multirow{3}{*}{Ethnicity} & Religion & 41.8 & 27.7 & 44.6 & 26.6 & $p = 0.29$\\ 
    \cline{2-7} & Race & 40.1 & 26.5 & 42.2 & 27.7 & $p = 0.64$\\
    \cline{2-7} & Nationality & 42.1 & 28.3 &  41.3 & 27.5 & $p = 0.46$\\
    \hline
    
    \multirow{3}{*}{Society} & Relationship & 60.3 & 24.8 & \textbf{52.9} & 25.7 & $p < 0.001$\\
    \cline{2-7} & Employment & 58.2 & 22.8 & 52.1 & 25.8 & $p = 0.05$\\
    \cline{2-7} & Pet & 37.3 &  24.4 & 39.1 & 27.8 & $p = 0.46$\\
    \hline
    
    \multirow{2}{*}{Safety} & Valuable Property & 64.0 & 25.0 & \textbf{59.6} & 26.1 & $p = 0.34$\\
    \cline{2-7} & Living Schedule & 59.3 & 24.4 & 59.1 & 26.3  & $p = 0.10$\\
    \hline
      % \bottomrule
    \end{tabular}
    \label{tab:privacy_importance}
  \end{table*}
\subsection{Result}
In total, we received 115 valid responses out of 120 total responses, in which respondents successfully completed the survey and passed all attention check questions. We utilized the Wilcoxon signed-rank test ($p < 0.05$) and Friedman test ($p < 0.05$) for statistical analysis since the rating scores are ordinal.

% Results reveal significance for the human perceived importance on different visual privacy categories ($\chi^2(4, N=115) = 37.6, p < 0.001$) using Friedman test. We used the Nemenyi test as a post-hoc test to illustrate the pairwise comparison between visual privacy categories. Results shown in table~\ref{tab:catergory_sig} indicate that people value bio-metric identification ($avg. = 3.53, s.d. = 1.33$) and personal marker/information ($avg. = 3.67, s.d. = 1.36$) more than other categories of visual privacy, including safety ($avg. = 4.00, s.d. = 1.34$), society ($avg. = 4.12, s.d. = 1.35$) or ethnicity ($avg. = 4.68, s.d. = 1.43$). Participants rated safety to be significantly more important than ethnicity. Further, society is significantly more important than ethnicity. \input{tables/category_sig.tex}

The analysis results are listed in Table~\ref{tab:privacy_importance}. We concluded with the following findings.

1. \textbf{Lowering the image sensor's resolution can significantly decrease users' concerns about visual privacy}. Table~\ref{tab:privacy_importance} shows the average and the standard deviation of the rating score of each visual privacy feature under two different resolution conditions. 
% Results show that users rated a significant lower score on the importance of visual privacy on low-resolution images ($Z=-4.02, p < 0.001$). 
On average, people rated visual privacy features with significantly lower importance scores ($Z=-4.02, p < 0.001$) under the low-resolution condition ($avg. = 45.1$) than the high-resolution condition ($avg. = 49.3$). 


2. \textbf{Identifiable face, nudity, home address, number/code, medical treatment, relationship, employment, valuable property, and living schedule are considered to be more important than other visual privacy features.} Statistic analysis indicates that visual privacy features have significant effects on the human perceived important scores under either the high-resolution condition ($\chi^2(25, N=115) = 298.5, p < 0.001$) or low-resolution condition ($\chi^2(25, N=115) = 169.9, p < 0.001$). When we ran the pairwise statistical analysis using Wilcoxon signed-rank test among visual privacy features, we concluded with the following major results. On both high-resolution and low-resolution images, identifiable face, nudity, home address, number/code, medical treatment, relationship, employment, valuable property, and living schedule were considered the most important visual privacy features, since users rated them with significantly higher scores than other features ($p < 0.05$). Among these important privacy features, medical treatment and employment were considered less important ($p < 0.05$). 
%nudity has a significantly higher importance score ($avg. = 62.9, s.d. = 29.4$) than other visual features ($p < 0.05$) except identifiable face ($p = 0.25$), valuable property ($p = 0.24$) and living schedule ($p = 0.33$). 
%Participants rated identifiable face a higher importance score than other visual privacy features ($p < 0.05$) except home address ($p = 0.25$), nudity ($p = 0.25$), number/code ($p = 0.31$), valuable property ($p = 0.72$) and living schedule ($p = 0.74$). 
%Number/code is considered to be more important than other visual features ($p < 0.05$) except identifiable face ($p = 0.31$), relationship ($p = 0.15$), employment ($p = 0.20$), valuable property ($p = 0.24$) and living schedule ($p = 0.33$). 
%Home address is considered to be more important than other visual features ($p < 0.05$) except identifiable face ($p = 0.25$), number/code ($p = 0.77$), medical treatment ($p = 0.18$), physical disability ($p = 0.12$), relationship ($p = 0.18$), employment ($p = 0.26$), valuable property ($p = 0.16$). 
%Valuable property is considered to be more important than other visual features ($p < 0.05$) except identifiable face ($p = 0.72$), home address ($p = 0.16$), and living schedule ($p = 0.87$). 
%Living schedule is considered to be more important than other visual features ($p < 0.05$) except identifiable face ($p = 0.74$), nudity ($p = 0.33$), number/code ($p = 0.12$), and valuable property ($p = 0.87$). 

3. \textbf{Nudity, identifiable face, valuable property, and living schedule are the most important privacy features despite the image resolution.} When compared with the high-resolution condition, we observed significantly lower importance scores on features including home address ($p = 0.01$), medical treatment ($p < 0.001$), and relationship ($p < 0.01$) under the low-resolution condition. This finding is reasonable since these privacy features require high-resolution details to interpret. For instance, people were less concerned about the readable texts on low-resolution images. However, nudity, identifiable face, valuable property, and living schedule still lead to the most concerned visual privacy features in the low-resolution setting, with an average score above 57.

Instead of considering all the visual privacy features, we want to explore the most concerned ones that have the highest importance score and are potentially still vulnerable to low-resolution images. Therefore, we chose the most important visual privacy features in each category under the low-resolution condition with a minimum importance score threshold of 50.0. As a result, four visual privacy features including \textbf{\textit{nudity}}, \textbf{\textit{identifiable face}}, \textbf{\textit{valuable property}} and \textbf{\textit{relationship}} were chosen for later user studies and analysis. 
% We acknowledge that these four visual privacy features are limited to the scenario of daily activity recognition in the home environment. However, we envision that the research methods we introduced in this section can be applied to other scenarios. \xueyang{I didn't understand the last two sentences.}

 
\section{ADLs Dataset with Visual Privacy Features}
\label{sec:dataset}

This section describes the dataset we used to explore the effect of image resolution on humans' and machines' performance on activity recognition and visual privacy awareness tasks.

\subsection{Constructing the ADLs Dataset}

In order to evaluate the model in realistic environments, we used the publicly-available PA-HMDB51 dataset for privacy-preserving activity recognition~\cite{wu2019framework}. This dataset consists of about 355 minutes and 51 types of human activity videos collected from realistic environments with various visual privacy features annotated. 

In this paper, we mainly focus on activities of daily living (ADLs) in a smart home scenario. Therefore, three of our authors selected the qualified videos from the PA-HMDB51 dataset together with the following requirements. 1) The video represents a home environment. 2) All authors agreed that the main character conducted the same kind of activities. 3) All authors felt comfortable to publish the video online. For instance, due to the internet policy, we only chose men's or kids' topless videos in this study. Then, we divided the human activities in the PA-HMDB51 dataset into five basic kinds of activities of daily living (ADLs) including \textit{functional mobility}, \textit{feeding}, \textit{intimacy}, \textit{entertainment}, and \textit{personal hygiene}. Finally, we obtained 46, 30, 22, 37, and 16 minutes of videos for functional mobility, feeding, intimacy, entertainment, and personal hygiene, respectively.

We randomly split the PA-HMDB51 dataset into a training dataset, a validation dataset, and an evaluation dataset, which accounts for 90\%, 5\%, and 5\%, respectively. Considering the difference of the video duration in the PA-HMDB51 dataset, we divided all the videos into 2-second clips for later training and evaluation without affecting the judgment of the video content. Therefore, there are 226 clips of the videos in the evaluation dataset, with 69, 45, 33, 55, and 24 clips for functional mobility, feeding, intimacy, entertainment, and personal hygiene, respectively.

\subsection{Labeling the Privacy Features}

Based on the user study results presented in section~\ref{sec:study1}, we annotated each frame and each clip in our dataset with privacy features including \textit{nudity}, \textit{identifiable face}, \textit{valuable property}, and \textit{relationship}. 
Since privacy features may vary during the video clip, for example, even in the same video clip, the visibility of a person's face may be different, we provided both \textit{frame-level} and \textit{clip-level} labels of for each video in our dataset. First of all, we annotated all of the privacy attributes on each frame of different clips. Then, we annotated each clip according to the frames in the clip for later user studies and machine experiments. The detailed description of both frame-level and clip-level labels are listed below.
\begin{figure*}[!ht]
    \centering
    \begin{subfigure}{0.475\textwidth}
        \centering
        \includegraphics[width=1\textwidth]{figures/source/annotation_1.pdf}
    \end{subfigure}
    \begin{subfigure}{0.475\textwidth}
        \centering
        \includegraphics[width=1\textwidth]{figures/source/annotation_2.pdf}
    \end{subfigure} 
    \begin{subfigure}{0.475\textwidth}
        \centering
        \includegraphics[width=1\textwidth]{figures/source/annotation_3.pdf}
    \end{subfigure}
    \begin{subfigure}{0.475\textwidth}
        \centering
        \includegraphics[width=1\textwidth]{figures/source/annotation_4.pdf}
    \end{subfigure} 
    \caption{Examples of the annotated frames in our dataset.}
    \label{fig:annotation_example}
    \Description{Examples of the annotated frames in our dataset. The four samples are shown in the upper left, upper right, lower left, and lower right of the figure, respectively. Each example is shown with a frame on the left, and labels of types of ADLs, nudity, identifiable face, valuable property, and relationship on the right.}
\end{figure*}
\begin{itemize}
    \item \textbf{Nudity}. The nudity label of each frame included three types that are \textit{naked or semi-naked (topless or bottomless)}, \textit{fully clothed}, and \textit{no person}. A clip is labeled as \textit{naked or semi-naked (topless or bottomless)} if at least one frame of the clip is labeled as \textit{naked or semi-naked (topless or bottomless)}. Otherwise, the clip is labeled as \textit{fully clothed} in a similar way. If every frame is labeled as \textit{no person}, we will finally label the clip as \textit{no person}.
    \item \textbf{Identifiable face}. If more than 70\% of a human face is visible, we consider the frame to contain an identifiable face. Therefore, each frame is labeled as \textit{yes}, \textit{no}, and \textit{no person}. A clip with more than one frame labeled as \textit{yes} is labeled as \textit{yes}, otherwise \textit{no}. A clip with every frame labeled as \textit{no person} is then labeled as \textit{no person}. 
    \item \textbf{Valuable property}. We only consider safe box, jewelry, watch, ring, and cash as valuable properties. Each frame is labeled as \textit{yes}, \textit{no}, and \textit{no person}. We label clips with at least one frame labeled \textit{yes} as \textit{yes}, otherwise \textit{no}. Clips with no person on any frame are labeled as \textit{no person}.
    \item \textbf{Relationship}. We consider the relationship of all the people presented in the video. There are four types of labels for each frame: \textit{intimate relationship}, \textit{non-intimate relationship}, \textit{only one person}, and \textit{no person}. A video clip is labeled as \textit{intimate relationship} if at least one frame of the clip is labeled as \textit{intimate relationship} and the frames labeled as \textit{intimate relationship} are no less than those labeled as \textit{non-intimate relationship}. Otherwise, a clip is labeled as \textit{non-intimate relationship} in a similar way. A clip with only one person presented is labeled as \textit{only one person} and labeled as \textit{no person} if there is no person existing in the clip.
\end{itemize}

Examples of the annotated frames in the dataset are demonstrated in Figure~\ref{fig:annotation_example}. Each frame was annotated by at least three of our authors and then cross-checked.
\section{Effect of Resolution on Human's Recognition Performance}
\label{sec:study2}
\begin{figure*}[!ht]
    \centering
    \includegraphics[width=1\textwidth]{figures/source/interface.pdf}
  \caption{Example of the web-based user interface. Video clips of different resolutions is displayed on the left side. All tasks are listed on the right side of the web page.}
  \label{fig:us2_interface}
  \Description{Example of the web-based user interface. Video clips of different resolutions is displayed on the left side. Users are required to watch a video before answering the questions. All tasks including ADLs recognition, facial identification, nudity recognition, property detection, relationship classification, and attention checks are listed on the right side of the web page.}
\end{figure*}

After identifying the most important visual privacy features in the first study, we model the effect of image resolution on human performance in recognizing activities of daily living and visual privacy features. We describe the procedure and results in this section. 

\subsection{User Interface}
\label{sub:us2_user_interface_and_alg}

We developed a web-based user interface as shown in Figure~\ref{fig:us2_interface}. Each problem set in the test for the participants includes one ADLs recognition task and four privacy feature recognition tasks including face, nudity, valuable property, and relationship. The user interface also includes attention-check questions in each test. Responses with incorrect answers to the attention check questions were treated as invalid. A starting page, shown before the testing procedure, introduces the purpose of the user study and requires the participant's demographic information.

We sampled the image resolutions into seven values including $15 \times 15$, $20 \times 20$, $30 \times 30$, $50 \times 50$, $100 \times 100$, $160 \times 160$ and $240 \times 240$. We utilized a randomization strategy on the back-end server so that each participant could view 4 randomly chosen videos, with each video in a random resolution among these seven values. The same video did not appear twice to each participant. In addition, different clips from the same video did not appear to the same participant.

\subsection{Participant and Procedure}

We recruited 240 participants (105 females, 135 males) with an average age of 22.23 (s.d. = 5.25, ranging from 18 to 30). All participants were required to have healthy eye conditions without any historical disease (e.g., color blindness) and use their laptop or desktop web browser to finish the whole test. 
The starting page of the web-based user interface introduced the purpose of the study. 
Participants were required to fill in their demographic information, including gender, age, and historical eye diseases. Following were two practice tests using two $240 \times 240$ resolution example videos excluded from the evaluation dataset. Finally, each participant finished the 28 rounds of the test. The user study lasted around 10 minutes. Each participant was offered a 5 USD gift card for compensation. 

\subsection{Results and Findings}

In total, we obtained 6, 720 answer records, with 457 (6.80\%) invalid due to the failure of the attention check questions. We utilized One-way ANOVA for the statistic analysis ($p < 0.05$) with independent-samples t-test ($p < 0.05$) as post-hoc analysis. We present our major results and findings below.

\begin{figure}[!bp]
  \centering
  \begin{subfigure}{0.475\textwidth}
    \centering
    \includegraphics[width=1\textwidth]{figures/source/questionnaire.pdf}
    % \includegraphics[width=1\textwidth]{figures/source/questionnaire_labeled.pdf}
  \end{subfigure}
  \caption{Humans' recognition performance on main activity and privacy feature recognition tasks.}
  \label{fig:us2_result}
  \Description{Humans' recognition performance on main activity and privacy feature recognition tasks. In the picture, the x-axis represents resolution and the y-axis represents accuracy. We plotted humans' recognition performance on tasks as curves including activity classification, facial identification, nudity recognition, property detection, and relationship classification.}
\end{figure}
% \begin{table*}[ht] 
    \centering
    \caption{Humans' recognition performance on main activity and privacy feature recognition tasks.}
    \begin{tabular}{|M{0.275\textwidth}|M{0.075\textwidth}|M{0.075\textwidth}|M{0.075\textwidth}|M{0.075\textwidth}|M{0.075\textwidth}|M{0.075\textwidth}|M{0.075\textwidth}|}
    % {|c|c|c|c|c|c|c|c|}
    \hline
    \textbf{Resolution}  & \textbf{$15 \times 15$} & \textbf{$20 \times 20$} & \textbf{$30 \times 30$}  & \textbf{$50 \times 50$} & \textbf{$100 \times 100$} & \textbf{$160 \times 160$} & \textbf{$240 \times 240$}\\
    \hline
   \textbf{ADLs Recognition} & 37.5\% &  52.5\% & 75.8\% & 88.4\% & 89.6\% & 89.9\% & 90.6\% \\
    \hline
    \textbf{Facial Identification} & 24.5\% & 29.5\% & 42.8\% & 62.9\% & 79.2\% & 84.8\% & 88.0\% \\
    \hline
    \textbf{Property Detection} & 78.7\% & 80.1\% & 80.1\% & 80.4\% & 81.2\% & 81.4\% & 81.8\% \\
    \hline
    \textbf{Nudity Detection} & 53.7\% &  66.4\% & 83.7\% & 89.4\% & 90.9\% & 91.5\% & 91.6\% \\
    \hline
    \textbf{Relationship Classification} & 51.7\% & 62.4\% & 80.7\% & 91.1\% & 93.0\% & 93.3\% & 93.5\% \\
    \hline
    \end{tabular}
    \label{tab:us2_result}
  \end{table*}

\textbf{Low-resolution images are effective in preserving visual privacy but the effects are highly dependent on privacy features.} Figure~\ref{fig:us2_result} shows the effect of image resolution on human recognition performance of ADLs, face, valuable property, nudity, and relationship. We observed the significant effect of image resolution on all visual privacy recognition tasks ($p < 0.001$). 
Further, there is no significant difference between resolutions of $160 \times 160$ and $240\times 240$, indicating that resolutions above $160 \times 160$ pixels do not further contribute to visual privacy awareness statistically.
However, the effect of the image resolution is highly task-dependent. Statistical analysis indicates that the type of privacy features has significant effects on the perception performance ($F_{3,25048}=427.2$, $p < 0.001$). Specifically, pair-wise comparisons show that human eyes are more sensitive to nudity ($p < 0.001$) when the image resolution is below $50 \times 50$ pixels, followed by the relationship task. However, tasks including face identification and valuable property recognition require higher resolution images ($\geq 100 \times 100$ pixels) to achieve higher performance. For example, participants can only identify human faces with an accuracy of 79.2\% when the resolution is $100\times 100$ pixels. This is because both face identification and valuable property rely on detailed visual information. Therefore, a low-resolution image sensor can preserve but not fully protect visual privacy from the perspective of a human recognizer. 

\textbf{Lowering the image resolution has a significant negative impact on human recognition performance on ADLs.} Results show that there is a statistically significant effect of changed resolution on human ADLs recognition performance ($F_{6,6256}=278.0$, $p < 0.001$). With resolutions lower than $30\times 30$ pixels, human eyes can only recognize the ADLs with an accuracy below 75.8\%. When the image resolution increases to $50 \times 50$ pixels, participants can recognize the activity with a fair accuracy --- 88.4\%. However, participants are aware of some privacy features at the resolution of $50 \times 50$. For example, they can recognize the relationship and nudity with an accuracy of 91.1\% and 89.4\%, respectively.
\section{Effect of Resolution on Machine's Recognition Performance}
\label{sec:study3}

In this section, we explore the effect of image resolution on machine's recognition performance of ADLs and visual privacy features. We adopted the open-access cutting-edge deep learning methods as the machine recognizer. 
% We describe the machine learning approaches and the major results. 

\subsection{ADLs Recognition}
\label{sub:adl_machine}

\subsubsection{Training and Evaluation Dataset}
We applied data augmentation approaches to the training dataset in section~\ref{sec:dataset}, including horizontal flip, and Gaussian Noise, enlarging the dataset by four times. To fairly compare the recognition performance of the machine and the human, we utilized the same evaluation dataset in section~\ref{sec:study2}. 

\subsubsection{Training and Evaluation Procedure}
We utilized both convolutional neural networks and transformer-based models as our ADLs classifiers, including \textbf{ResNet50}~\cite{he2015deep}, \textbf{Efficient Net}~\cite{Mingxing2019EfficientNet} and \textbf{Vision Transformer (ViT)}~\cite{dosovitskiy2020vit}. All the models used here were pretrained with ImageNet dataset~\cite{imagenet} that output 1000 probabilistic values. In this experiment, we took every fame of the video clips in our dataset as the model input during our training, validating, and testing procedure. We first scaled the image of low resolution to $512\times 512$ pixels to standardize the input of the model. Then, we fine-tuned the pretrained network using the training dataset with a certain resolution ($r$) in which the images were all at the resolution of $r\times r$. To transfer the pretrained network model to our application, we added an additional five-node fully connected layer at the end of the network. We used sigmoid as the activation function. Once we finished the training procedure, we evaluated the fine-tuned model using the evaluation dataset under the same image resolution ($r$).
As we have described in section~\ref{sec:dataset}, we use the randomly chosen 5\% of the total dataset as the validation dataset in our implementation. In order to avoid the over-fitting problem, we used the early stopping method. In other words, we will stop our training procedure when the accuracy on the validation dataset does not rise anymore for 5 successive epochs. 

\subsubsection{Result}
Table~\ref{tab:us3_result} shows the effect of image resolution on machines' performance of the ADLs recognition task. Results indicate that \textbf{the machine outperforms the human regarding the ADLs recognition task on low-resolution images}. Vision Transformer can maintain an accuracy of 84.4\% even when the image resolution is as low as $20\times 20$. However, such a resolution is far from enough for humans to recognize ADLs at an ideal accuracy level. For resolutions above $100 \times 100$, both humans and machines can achieve a high accuracy above 90\%. Such results show the possibility of constructing a range of image resolutions to preserve visual privacy without bearing great loss in ADLs recognition simultaneously.

\begin{table*}[!ht] 
    \centering
    \caption{The ADLs recognition performance of ViT, ResNet50, and EfficientNet, compared with human.}
    \begin{tabular}{|M{0.2\textwidth}|M{0.15\textwidth}|M{0.15\textwidth}|M{0.15\textwidth}|M{0.15\textwidth}|}
    \hline
    \multirow{2}{*}{\textbf{Resolution}} & \multirow{2}{*}{\textbf{Human}} & \multicolumn{3}{c|}{\textbf{Machine}} \\ \cline{3-5} & & \multicolumn{1}{c|}{\textbf{ViT}} & \multicolumn{1}{c|}{\textbf{ResNet50}} & \textbf{EfficientNet} \\ \hline
    $15\times 15$  & 37.5\% & 81.0\% & 63.9\% & 52.9\% \\ \hline
    $20\times 20$  & 52.5\% & 84.4\% & 66.3\% & 63.5\% \\ \hline
    $30\times 30$  & 75.8\% & 89.8\% & 75.1\% & 68.0\% \\ \hline
    $50\times 50$  & 88.4\% & 90.7\% & 80.5\% & 74.6\% \\ \hline
    $100\times 100$ & 89.6\% & 92.2\% & 81.5\% & 75.1\% \\ \hline
    $160\times 160$ & 89.9\% & 93.2\% & 82.0\% & 80.0\% \\ \hline
    $240\times 240$ & 90.6\% & 94.6\% & 88.8\% & 83.9\% \\ \hline
    \end{tabular}
    \label{tab:us3_result}
\end{table*}
\begin{figure*}[htbp]
    \centering
    \begin{subfigure}{0.475\textwidth}
      \centering
      \includegraphics[width=1\textwidth]{figures/source/machine_1.pdf}
    \end{subfigure}
    \begin{subfigure}{0.475\textwidth}
      \centering
      \includegraphics[width=1\textwidth]{figures/source/machine_2.pdf}
    \end{subfigure}
    \begin{subfigure}{0.475\textwidth}
      \centering
      \includegraphics[width=1\textwidth]{figures/source/machine_3.pdf}
    \end{subfigure}
    \begin{subfigure}{0.475\textwidth}
      \centering
      \includegraphics[width=1\textwidth]{figures/source/machine_4.pdf}
    \end{subfigure}
    \caption{Machines' recognition performance of privacy features.}
    \label{fig:machine_result}
    \Description{Machines' recognition performance of privacy features. In the picture, the x-axis represents resolution and the y-axis represents accuracy. Shown in the upper left of the figure are the recognition results of InsightFace. Shown in the upper right of the figure are the recognition results of NudeNet. Shown in the lower left of the figure are the recognition results of DETR. Shown in the lower right of the figure are the recognition results of GRM.}
  \end{figure*}

\subsection{Privacy Features Recognition}

\subsubsection{Facial Identification}
\label{subsec:machine_face}
We adopted \textbf{InsightFace}~\cite{deng2018arcface} for facial identification by testing whether the model can recognize human faces in certain areas of the frames. We used the pretrained \textbf{ArcFace} model for facial identification provided by InsightFace. Also, we checked every frame of the video clips in this experiment. The result is shown in Figure~\ref{fig:machine_result} as the teal line. Results from the ArcFace model indicate that even the state-of-art models cannot detect any human faces below $50\times 50$ pixels. However, as the resolution increases from $100 \times 100$ to $240\times 240$ pixels, machine's facial identification performance significantly increases from 71.0\% to 100.0\%. Such results imply that identifiable faces can be preserved well against the machine attacker when the image resolution is below $50 \times 50$ pixels.

\subsubsection{Nudity Recognition}
\label{subsec:machine_nudity}
We adopted the pretrained \textbf{NudeNet}~\footnote{Software DOI: 10.5281/zenodo.3584720} for binary nudity recognition. This model was trained to detect nude parts of the human body in images. Here we utilized the classifier model to help us make a distinction between safe and unsafe images. We report the result of NudeNet as the orange line in Figure~\ref{fig:machine_result}. The precision and recall of NudeNet also reveal that it cannot identify any nude parts below the resolution of $30 \times 30$ pixels. Under the resolution of $100 \times 100$ pixels, NudeNet can recognize frames containing nude parts with an accuracy of 88.0\%. Therefore, we conclude that resolutions below $30 \times 30$ pixels can effectively preserve the nudity privacy feature.

\subsubsection{Property and Object Detection}
\label{subsec:machine_value}
We adopted \textbf{DETR}~\cite{Nicolas2020DETR} pretrained on the COCO dataset~\footnote{https://cocodataset.org/} for property and object detection. 
Considering the availability of pretrained object detection models, we used the detection performance of DETR on COCO objects as an estimation of machine's recognition performance on valuable properties.
We manually annotated the objects which belong to the COCO classes in each frame as ground truth. 
In our implementation, we first resized videos of different resolutions up to $240\times 240$ pixels. Then we kept bounding boxes with a confidence level above a pre-set threshold (e.g., 0.75) as a result of the model. 
To evaluate the model performance under different resolutions, we compared the objects detected by the model and the ground truth of each frame one by one to calculate the recognition accuracy. 
The purple line in Figure~\ref{fig:machine_result} shows the recognition accuracy of DETR. Results show that below the resolution of $50 \times 50$, DETR fails to detect any object. On images with a resolution of $100 \times 100$, DETR can achieve an accuracy of 72.0\%. Under the resolution of $160 \times 160$, large objects such as the main character can be detected precisely with an overall accuracy of 77.0\%. Under the resolution of $240\times 240$, the object detection results are more accurate, and small targets such as bottles and cups can be detected, too. Therefore, DETR can finally achieve an accuracy of 82.0\%.

\subsubsection{Relationship Classification}
\label{subsec:machine_relationship}

We adopted a pretrained cutting-edge social relationship classification model \textbf{GRM}~\cite{Wang2018Deep} on the evaluation dataset with different image resolutions. This model utilized a node message propagation mechanism and a graph attention mechanism to explore the interaction between the person pair of interest and contextual objects. The prerequisite to inferring the relationship between people is to obtain the context information using the object detection model. In our implementation, we resized the raw video of different resolutions to $240 \times 240$ pixels. Then, we annotated the bounding boxes and classes of different objects using DETR and labeled the bounding boxes of the person pair whose social relationship we wanted to examine. The model took every frame of the raw videos and objects list as input and generated the classification result as output.

The accuracy result we reported as the magenta line in Figure~\ref{fig:machine_result} describes the performance of the GRM model on the four-classes social relationship recognition task including \textit{intimate relationship}, \textit{non-intimate relationship}, \textit{no relationship}, and \textit{no person}. As is shown, the GRM model can detect nothing and will classify any input image as the \textit{no person} type under resolutions below $30 \times 30$ pixels. Our results here also proved that a low resolution below $30 \times 30$ is sufficient to preserve the privacy of social relationships against the cutting-edge machine recognition method. When the resolution is $100\times 100$ pixels, GRM can recognize social relationships in the video with an accuracy of 34.1\%. For resolutions of $160\times 160$ pixels and $240 \times 240$ pixels, GRM can achieve an accuracy of 60.9\% and 80.5\%, respectively.
\section{Justify the Influence of Image Super-Resolution}
\label{sec:super}

Image super-resolution techniques were proposed by researchers to reconstruct a high-resolution image from a low-resolution image~\cite{Wang2020, Liu_SpliteSR}. 
% Empirically, we expect super-resolution methods may have an impact on the recognition performance of humans and machines. 
In this section, we justice whether cutting-edge super-resolution techniques influence our results and findings regarding the effects of low resolution on activity recognition and privacy awareness through a user study.
%we conducted an additional user study to quantitatively investigate the effect of super-resolution. 
%the robustness of the recognition results of both humans and machines against several super-resolution techniques so as to provide a solid foundation for calculating the model in section~\ref{sec:discussion}. %So far, we have studied the recognition performance of humans and machines in section~\ref{sec:study2} and section~\ref{sec:study3}, respectively. %To our best knowledge, no existing work has thoroughly investigated the effect of super-resolution on humans' recognition performance. 

\subsection{User Study Procedure and Participant}
We adopted one of the cutting-edge image super-resolution methods SwinIR~\cite{Liang2021SwinIR} based on Transformer architectures as well as the traditional bicubic method to upscale the videos in our evaluation dataset by four times. Three examples of super-resolution processed videos are shown in Figure~\ref{fig:super_resolution}. 

We adopted a similar web-based interface as Figure~\ref{fig:us2_interface} shows except for changing the attention check question to addition and subtraction test. In this study, we first introduced the purpose and the procedure of our study. Then each participant took 8 trials with each trial having one test on the raw video and one test with videos after super-resolution. In each trial, we first presented each participant with a randomly-chosen raw video in the evaluation dataset and asked them to answer questions of ADLs and privacy features recognition as illustrated in Figure~\ref{fig:us2_interface}. The raw video's resolution was set to a random value among $15 \times 15$, $20 \times 20$, $30 \times 30$, $50 \times 50$, $100 \times 100$, $160 \times 160$ and $240\times 240$. Then, we presented them with the super-resolution videos together with raw videos simultaneously and asked them to answer the same questions. To avoid cross effects between videos under different resolutions, the same raw video did not appear twice to each participant. Further, we also ensured that the participants in this study were different from those who participated in the previous studies. 

We recruited 306 participants (123 females, 183 males) with an average age of 21.76 (s.d. = 4.56). The user study lasted around 10 minutes. Each participant was offered a 5 USD gift card for compensation. 

\begin{figure}[!ht]
    \centering
    \includegraphics[width=0.475\textwidth]{figures/source/super_resolution.pdf}
    \caption{Examples of the effect of super resolution on videos of low resolutions including $15\times 15$, $20\times 20$, and $30\times 30$.}
    \label{fig:super_resolution}
    \Description{Examples of the effect of super resolution on videos of low resolutions including $15\times 15$, $20\times 20$, and $30\times 30$. Shown on the left side of the figure are frames not processed by super-resolution techniques. Shown on the right side of the figure are the same frames processed by super-resolution techniques.}
  \end{figure}

\subsection{Results and Findings}
In total, we received 4,896 test records with 273 (5.57\%) of them failed the attention check. Table~\ref{tab:super_resolution_adl} and Table~\ref{tab:super_resolution_privacy} show the comparison of participants' overall recognition accuracy with or without super-resolution. Results indicate that participants performed better on super-resolution videos than on raw videos. Statistical analysis suggests that when image resolution is below $20\times 20$ pixels, super-resolution techniques can significantly improve human recognition performance on both activity recognition and privacy recognition tasks. But it is worth noting that the improvement in recognition performance brought about by super-resolution technology is still less than that brought about by increasing the resolution itself. Such a finding reveals that super-resolution techniques do not provide enough additional information for humans to enhance their perception ability in both activity recognition and visual privacy awareness tasks. 

In terms of the impact of the super-resolution technique on the machine's recognition performance, researchers have proved that super-resolution can slightly facilitate vision-based recognition task such as activity recognition~\cite{demir2021tinyvirat, hou2021extreme},  object and text recognition~\cite{xi2020see, Liu_SpliteSR}. However, the influence of the super-resolution technique is very limited. The results are still significantly inferior to that with the original high-resolution images~\cite{Dai2015super}.

%scene recognition~\cite{Dai2015super}. Their experiment results illustrated that although super-resolution methods are helpful in general for other vision tasks when the resolution of input images are low, the performance with the super-resolved images are still significantly inferior to that with the original, high-resolution images.

In conclusion, the additional visual information introduced by the image super-resolution technique is insufficient to overcome the effect of resolution on the recognition performance of humans and machines. Therefore, we believe that the effects of image resolution on human (section~\ref{sec:study2}) and the machine's (section~\ref{sec:study3}) ADLs and visual privacy recognition performance are robust against image super-resolution techniques.

\begin{table}[htbp] 
    \centering
    \caption{The statistic of the overall accuracy on main activity recognition with or without super resolution conditions. $p < 0.05$ indicates a significant difference between with or without super resolution conditions. }
    \begin{tabular}
    {|M{0.1\textwidth}|M{0.0375\textwidth}|M{0.0375\textwidth}|M{0.0375\textwidth}|M{0.0375\textwidth}|M{0.1\textwidth}|}
    % {|c|c|c|c|c|c|}
    \hline
    \multirow{2}{*}{\textbf{Resolution}} & \multicolumn{2}{c|}{\textbf{Before}} & \multicolumn{2}{c|}{\textbf{After}} & \multirow{2}{*}{\textbf{Significance}} \\
    \cline{2-5} & avg. & std. & avg. & std. & \\ \hline
$15\times 15$                          & \multicolumn{1}{c|}{0.386} & 0.487 & \multicolumn{1}{c|}{0.452} & 0.498 & $p<0.001$                         \\ \hline
$20\times 20$                          & \multicolumn{1}{c|}{0.593} & 0.491 & \multicolumn{1}{c|}{0.706} & 0.456 & $p=0.002$                         \\ \hline
$30\times 30$                          & \multicolumn{1}{c|}{0.803} & 0.397 & \multicolumn{1}{c|}{0.845} & 0.362 & $p=0.149$                         \\ \hline
$50\times 50$                         & \multicolumn{1}{c|}{0.891} & 0.310 & \multicolumn{1}{c|}{0.893} & 0.308 & $p=0.932$                         \\ \hline
$100\times 100$                         & \multicolumn{1}{c|}{0.846} & 0.360 & \multicolumn{1}{c|}{0.898} & 0.302 & $p=0.046$                         \\ \hline
$160\times 160$                          & \multicolumn{1}{c|}{0.899} & 0.301 & \multicolumn{1}{c|}{0.908} & 0.289 & $p=0.701$                         \\ \hline
$240\times 240$                         & \multicolumn{1}{c|}{0.908} & 0.289 & \multicolumn{1}{c|}{0.927} & 0.260 & $p=0.386$                         \\ \hline

    \end{tabular}
    \label{tab:super_resolution_adl}
\end{table}
\begin{table}[htbp] 
    \centering
    \caption{The statistic of the overall accuracy on privacy features recognition with or without super resolution conditions. $p < 0.05$ indicates a significant difference between with or without super resolution conditions. }
    \begin{tabular}
    {|M{0.1\textwidth}|M{0.0375\textwidth}|M{0.0375\textwidth}|M{0.0375\textwidth}|M{0.0375\textwidth}|M{0.1\textwidth}|}
    % {|c|c|c|c|c|c|}
    \hline
    \multirow{2}{*}{\textbf{Resolution}} & \multicolumn{2}{c|}{\textbf{Before}} & \multicolumn{2}{c|}{\textbf{After}} & \multirow{2}{*}{\textbf{Significance}} \\
    \cline{2-5} & avg. & std. & avg. & std. & \\ \hline
    $15\times 15$ & \multicolumn{1}{c|}{0.558} & 0.497 & \multicolumn{1}{c|}{0.602} & 0.476 & $p<0.001$ \\ \hline
    $20\times 20$ & \multicolumn{1}{c|}{0.673} & 0.469 & \multicolumn{1}{c|}{0.736} & 0.440 & $p<0.001$ \\ \hline
    $30\times 30$ & \multicolumn{1}{c|}{0.793} & 0.404 & \multicolumn{1}{c|}{0.823} & 0.381 & $p=0.038$ \\ \hline
    $50\times 50$ & \multicolumn{1}{c|}{0.851} & 0.356 & \multicolumn{1}{c|}{0.866} & 0.340 & $p=0.276$ \\ \hline
    $100\times 100$ & \multicolumn{1}{c|}{0.895} & 0.305 & \multicolumn{1}{c|}{0.906} & 0.291 & $p=0.359$ \\ \hline
    $160\times 160$ & \multicolumn{1}{c|}{0.905} & 0.292 & \multicolumn{1}{c|}{0.913} & 0.280 & $p=0.488$ \\ \hline
    $240\times 240$ & \multicolumn{1}{c|}{0.921} & 0.268 & \multicolumn{1}{c|}{0.925} & 0.263 & $p=0.766$ \\ \hline
    \end{tabular}
    \label{tab:super_resolution_privacy}
  \end{table}
\section{Modeling the Trade-off of Privacy Preservation and Activity Recognition}
\label{sec:discussion}

In this paper, our goal is to present a method to model the trade-off between privacy preservation and machine recognition. We have obtained the estimation results of the main components in Equation~\ref{eq:problem}. In this section, we take all these results into consideration and explain how we can utilize them to model the trade-off between privacy preservation and machine recognition. Based on our modeling results, we further present how to apply our model to applications.

\subsection{Build the Model Using the Parameters from the Studies}

To summarize, we have investigated users' perceived importance of different privacy features under high or low image resolutions in section~\ref{sec:study1}. We chose users' rating of these privacy features under high-resolution image condition as the importance weight $\omega_i$ in the model, which was shown in Table~\ref{tab:privacy_importance}. 
Next, we examined both human's and the machine's recognition performance under varying resolutions in order to obtain an approximation of the evaluation function $L_T$ and $L_P$ in our formulation.
% Section~\ref{sec:dataset} described the dataset we used to explore the effect of image resolution on human's and the machine's performance. 
% In section~\ref{sec:study2}, we conducted a user study to understand human performance in recognizing main activities and privacy features. 
% In section~\ref{sec:study3}, we utilized open-access cutting-edge deep learning methods to explore the machine's recognition abilities on the same tasks. 
In realistic environments, intelligent applications may rely on either humans or machines to obtain private information from raw images. Therefore, we take both human and machine recognizers into consideration to preserve privacy features in a comprehensive way. For the main recognition task $T$, which is activity recognition in our implementation, the Vision Transformer outperforms all other models including humans even on extremely low-resolution videos from the dataset. Therefore, we choose the Vision Transformer as our final recognition function $f_T$ and the evaluation results of the Vision Transformer $L_T$ have been demonstrated in Table~\ref{tab:us3_result}. For each privacy feature $P_i$ including nudity, identifiable face, valuable property, and relationship, we found that humans are generally more effective recognizers compared with machines, especially on ultra-low-resolution videos from the dataset. Therefore, we consider humans as the final $f_{P_i}$ in our calculation. The evaluation results of each  $L_{P_i}$ we are going to use has been depicted in Figure~\ref{fig:us2_result}.

\subsection{Calculating the Objective Function}

Based on the results of $L_T$, $L_{P_i}$, and $\omega_i$ we have discussed above, we can calculate the objective function $S(r)$ in Equation~\ref{eq:problem} for each resolution we have sampled. Figure~\ref{fig:calculation_result} illustrates how the values of our objective function $S(r)$ change with resolutions $r$. The scaling factor $\lambda$ in our formulation indicates the sensitivity ratio of privacy preservation over activity recognition which can be flexibly adjusted according to the deployment environment or user experience. Here we have only shown the cases for three different lambda values, including $0.75$, $1.00$, and $1.25$. 

As is demonstrated in Figure~\ref{fig:calculation_result}, the value of the objective function $S(r)$ shows a trend of first increasing and then decreasing with the increase of resolution $r$. For the case where lambda is $1.00$, the objective function takes its maximum value at a resolution between $20\times 20$ and $30\times 30$, which indicates a proper resolution for balancing privacy preservation and activity recognition. Such an image resolution value can be easily extended to a certain image resolution range where the trade-off result is also acceptable. However, the objective function takes a low value when the resolution is too low (e.g., $15\times 15$) or too high (e.g., $240\times 240$). The reason behind this is also consistent with our expectations. When the image resolution is too low, although the privacy features can be better preserved, the machine's ADLs recognition performance is far from satisfactory. On the contrary, high image resolution may greatly increase the risk of privacy feature leakage except for improving ADLs recognition performance. 

Here we also noticed that as the scaling factor $\lambda$ increases, the maximum point of the objective function is also shifted to the left in Figure~\ref{fig:calculation_result}. Such a finding shows that a lower resolution of the image sensor is required if users are more concerned with privacy preservation compared with activity recognition performance. 

\subsection{Applying the Model and the Modeling Method to Applications}
In this section, we present how to apply our method and model to privacy-preserving machine recognition applications.

\subsubsection{Deployment to a Real Scenario Application}
When deploying a real scenario application based on our method, one can install an ultra-low-resolution (e.g., $20 \times 20$ pixels) image sensor with an edge computer running a machine learning method for ADLs recognition at home. To apply our framework for quantifying the trade-off between privacy preservation and activity recognition, one first needs to determine the sensitivity indicator $\lambda$ in Equation~\ref{eq:problem}, which is closely related to deployment environment and user experience. In our ADLs recognition example, the bathroom is a more visual privacy-sensitive location than the kitchen. Therefore, we would expect the image sensor in the bathroom having a lower resolution to preserve more visual privacy. 
Second, with the development of computer vision technologies, the performance of machine recognition on both activities and privacy features will exceed the current results stated in this paper. Future designers just need to fine-tune the results of the evaluation function $L_T$ and $L_P$ by selecting better recognizers $f_T$ and $f_P$ to consider the results of these technological advances.  
Third, one can leverage activities' probability distribution regarding the different environments in a home environment which may have an effect on the results of the evaluation function $L_T$ and $L_P$ in our formulation. For instance, personal and toilet hygiene is highly possible to happen in the restroom, while feeding is highly possible to occur in the kitchen. Future designers need to modify their training and evaluation data set according to the probability distribution of activities of daily living (ADLs) in different scenarios.

\subsubsection{Generalization to Other Applications}
For other computer vision based applications in a real scenario, we believe that our pipeline and method can be easily adapted. For instance, using an always-on low-resolution camera on AR glasses for activity recognition, or using a low-resolution smartphone camera for hand gesture recognition, etc. Even though different applications have their own usage scenarios with different visual privacy features, our method's key idea and basic framework can still be used efficiently. 
Although low-resolution image sensors can preserve visual privacy from the hardware level, deploying the hardware itself costs a large of human labor and money. Instead of purchasing the low-resolution image sensor, we can simply update the firmware to limit the camera's resolution, turning them to low-resolution image sensors. Further, we can attach an additional layer or lens on top of available commercial RGB cameras. For instance, we can add a piece of frosted glass, or a lens built for the Passive Infrared (PIR) motion sensor to the commodity cameras ~\footnote{https://en.wikipedia.org/wiki/Passive\_infrared\_sensor}. Most of these camera system parameter selection issues can be discussed and solved in a more generalized way of our methods.

\begin{figure*}[!ht]
    \centering
    \begin{subfigure}{0.325\textwidth}
        \centering
        \includegraphics[width=1\textwidth]{figures/source/score_1.pdf}
    \end{subfigure}
    \begin{subfigure}{0.325\textwidth}
        \centering
        \includegraphics[width=1\textwidth]{figures/source/score_2.pdf}
    \end{subfigure}
    \begin{subfigure}{0.325\textwidth}
        \centering
        \includegraphics[width=1\textwidth]{figures/source/score_3.pdf}
    \end{subfigure}
    \caption{Depicting the objective function based on the results of both humans' and machines' recognition performance.}
    \label{fig:calculation_result}
    \Description{Depicting the objective function based on the results of both humans' and machines' recognition performance. Presented are three cases of the objective function where $\lambda$ equals 0.75, 1, and 1.25, respectively.}
\end{figure*}

\section{Limitations and Future Work}

Our work is targeted at modeling the trade-off between visual privacy and the core machine recognition task, e.g., activity recognition in our case. Our purpose is to inspire future work to explore more quantitative methods for privacy-preserving applications. Therefore, future designers can apply or adapt these models according to their applications to preserve users' privacy as much as possible. However, there do exist several limitations of our work and we describe them below.

\paragraph{Utilizing Multimodal Information}

We acknowledge that an image sensor deployed at home can only collect images at a fixed position, distance, and field of view after installation. 
Only with the single modality data captured by images sensors, both machine's and human's recognition performance can be easily affected by the aforementioned factors.
We also acknowledge that we didn't take multimodal data, for example, audio data into consideration. Prior works have proved the effectiveness of leveraging multimodal data in activity recognition. With multimodal information, we can alleviate existing algorithms' dependence on images, thus allowing for a lower resolution of image sensors.
We expect future research can investigate how the modeling results of the trade-off between privacy preservation and activity recognition can be changed by multimodal information.

\paragraph{Privacy Preserving Methods}

In this work, we only use pixelization filters as the privacy-preserving method for the main task. The advantages of using low-resolution images have already been discussed in prior works. Nevertheless, we have to admit that researchers have shown that low resolution alone does not provide enough privacy guarantees. McPherson et al. found that obfuscated images contain enough information correlated with the obfuscated content to enable accurate reconstruction of the latter~\cite{McPherson2016defeating}. Although we have compared the privacy recognition performance of state-of-art machine learning algorithms on low-resolution images, we believe that our evaluation results on low-resolution images leave much room for discussion. We expect future research can explore the effect of more privacy-preserving methods on the trade-off between privacy preservation and activity recognition.

\paragraph{User Survey on Importance of Visual Privacy Features}

We acknowledge that our user study in section~\ref{sec:study1} aims to assess users' perceived importance of visual privacy features. We didn't limit participants' culture, age, gender, or technical backgrounds. However, there are many other factors that may affect participants' perception of privacy. For example, researchers have found that users on Amazon Mechanical Turk, where our participants were from, tend to be more privacy conscious~\cite{kang2014mturk,ross2010demographics}, thus are not representative of the general population all over the world. 
It is also undeniable that the perception of privacy varies substantially across cultures, societies, times, and locations~\cite{albayaydh2022jordan, ahmed2017digital-privacy, ahmed2017sim, crabtree2017repacking, palen2003unpacking, sambasivan2018rich, kang2015everywhere}. Therefore, our estimation of the perceived importance ($\omega$ in our formulation) of privacy features obtained through our user studies is possibly not applicable to populations in different cultural contexts across the world. However, the framework proposed in this paper is meant to inspire future researchers to consider humans' assessments of the importance of different visual privacy features. We expect that there will be more independent works to explore the influence of other factors on humans' perception of privacy.

\section{Conclusion}
\label{sec:conclusion}
Using the at-home activity of daily livings (ADLs) as the scenario, this paper models the trade-off of visual privacy preservation and activity recognition over image resolution. To achieve this purpose, we first conducted a user survey to obtain the most important visual privacy features, including nudity, identifiable face, valuable property, and relationship. Then, using the PA-HMDB51 dataset, which contains videos from realistic environments, we quantified the effect of image resolution on the human's performance on ADLs recognition and visual privacy awareness tasks through a user study. We further modeled the impact of image resolution on the machine's ability to recognize ADLs and visual privacy features using cutting-edge machine learning methods. Finally, we proposed a method with adjustable parameters to model the trade-off of privacy-preserving ADLs recognition using low-resolution images. Using this method, we can calculate an optimal range of image resolution for visual privacy preserving ADLs recognition. We envision that our method can inspire other vision-based systems that require balancing privacy awareness and machine recognition performance.

\begin{acks}
This work is supported by the Natural Science Foundation of China (NSFC) under Grant No. 62002198 and No. 62132010, Tsinghua University Initiative Scientific Research Program, Beijing Key Lab of Networked Multimedia, Institute for Artificial Intelligence, Tsinghua University, and Beijing National Research Center for Information Science and Technology (BNRist).
\end{acks}

% Bibliography
\bibliographystyle{ACM-Reference-Format}
\bibliography{proceedings}

%%
%% If your work has an appendix, this is the place to put it.
%% \appendix

\end{document}
\endinput
%%
