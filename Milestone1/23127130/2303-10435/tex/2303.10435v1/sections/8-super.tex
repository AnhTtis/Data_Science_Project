\section{Justify the Influence of Image Super-Resolution}
\label{sec:super}

Image super-resolution techniques were proposed by researchers to reconstruct a high-resolution image from a low-resolution image~\cite{Wang2020, Liu_SpliteSR}. 
% Empirically, we expect super-resolution methods may have an impact on the recognition performance of humans and machines. 
In this section, we justice whether cutting-edge super-resolution techniques influence our results and findings regarding the effects of low resolution on activity recognition and privacy awareness through a user study.
%we conducted an additional user study to quantitatively investigate the effect of super-resolution. 
%the robustness of the recognition results of both humans and machines against several super-resolution techniques so as to provide a solid foundation for calculating the model in section~\ref{sec:discussion}. %So far, we have studied the recognition performance of humans and machines in section~\ref{sec:study2} and section~\ref{sec:study3}, respectively. %To our best knowledge, no existing work has thoroughly investigated the effect of super-resolution on humans' recognition performance. 

\subsection{User Study Procedure and Participant}
We adopted one of the cutting-edge image super-resolution methods SwinIR~\cite{Liang2021SwinIR} based on Transformer architectures as well as the traditional bicubic method to upscale the videos in our evaluation dataset by four times. Three examples of super-resolution processed videos are shown in Figure~\ref{fig:super_resolution}. 

We adopted a similar web-based interface as Figure~\ref{fig:us2_interface} shows except for changing the attention check question to addition and subtraction test. In this study, we first introduced the purpose and the procedure of our study. Then each participant took 8 trials with each trial having one test on the raw video and one test with videos after super-resolution. In each trial, we first presented each participant with a randomly-chosen raw video in the evaluation dataset and asked them to answer questions of ADLs and privacy features recognition as illustrated in Figure~\ref{fig:us2_interface}. The raw video's resolution was set to a random value among $15 \times 15$, $20 \times 20$, $30 \times 30$, $50 \times 50$, $100 \times 100$, $160 \times 160$ and $240\times 240$. Then, we presented them with the super-resolution videos together with raw videos simultaneously and asked them to answer the same questions. To avoid cross effects between videos under different resolutions, the same raw video did not appear twice to each participant. Further, we also ensured that the participants in this study were different from those who participated in the previous studies. 

We recruited 306 participants (123 females, 183 males) with an average age of 21.76 (s.d. = 4.56). The user study lasted around 10 minutes. Each participant was offered a 5 USD gift card for compensation. 

\begin{figure}[!ht]
    \centering
    \includegraphics[width=0.475\textwidth]{figures/source/super_resolution.pdf}
    \caption{Examples of the effect of super resolution on videos of low resolutions including $15\times 15$, $20\times 20$, and $30\times 30$.}
    \label{fig:super_resolution}
    \Description{Examples of the effect of super resolution on videos of low resolutions including $15\times 15$, $20\times 20$, and $30\times 30$. Shown on the left side of the figure are frames not processed by super-resolution techniques. Shown on the right side of the figure are the same frames processed by super-resolution techniques.}
  \end{figure}

\subsection{Results and Findings}
In total, we received 4,896 test records with 273 (5.57\%) of them failed the attention check. Table~\ref{tab:super_resolution_adl} and Table~\ref{tab:super_resolution_privacy} show the comparison of participants' overall recognition accuracy with or without super-resolution. Results indicate that participants performed better on super-resolution videos than on raw videos. Statistical analysis suggests that when image resolution is below $20\times 20$ pixels, super-resolution techniques can significantly improve human recognition performance on both activity recognition and privacy recognition tasks. But it is worth noting that the improvement in recognition performance brought about by super-resolution technology is still less than that brought about by increasing the resolution itself. Such a finding reveals that super-resolution techniques do not provide enough additional information for humans to enhance their perception ability in both activity recognition and visual privacy awareness tasks. 

In terms of the impact of the super-resolution technique on the machine's recognition performance, researchers have proved that super-resolution can slightly facilitate vision-based recognition task such as activity recognition~\cite{demir2021tinyvirat, hou2021extreme},  object and text recognition~\cite{xi2020see, Liu_SpliteSR}. However, the influence of the super-resolution technique is very limited. The results are still significantly inferior to that with the original high-resolution images~\cite{Dai2015super}.

%scene recognition~\cite{Dai2015super}. Their experiment results illustrated that although super-resolution methods are helpful in general for other vision tasks when the resolution of input images are low, the performance with the super-resolved images are still significantly inferior to that with the original, high-resolution images.

In conclusion, the additional visual information introduced by the image super-resolution technique is insufficient to overcome the effect of resolution on the recognition performance of humans and machines. Therefore, we believe that the effects of image resolution on human (section~\ref{sec:study2}) and the machine's (section~\ref{sec:study3}) ADLs and visual privacy recognition performance are robust against image super-resolution techniques.

\begin{table}[htbp] 
    \centering
    \caption{The statistic of the overall accuracy on main activity recognition with or without super resolution conditions. $p < 0.05$ indicates a significant difference between with or without super resolution conditions. }
    \begin{tabular}
    {|M{0.1\textwidth}|M{0.0375\textwidth}|M{0.0375\textwidth}|M{0.0375\textwidth}|M{0.0375\textwidth}|M{0.1\textwidth}|}
    % {|c|c|c|c|c|c|}
    \hline
    \multirow{2}{*}{\textbf{Resolution}} & \multicolumn{2}{c|}{\textbf{Before}} & \multicolumn{2}{c|}{\textbf{After}} & \multirow{2}{*}{\textbf{Significance}} \\
    \cline{2-5} & avg. & std. & avg. & std. & \\ \hline
$15\times 15$                          & \multicolumn{1}{c|}{0.386} & 0.487 & \multicolumn{1}{c|}{0.452} & 0.498 & $p<0.001$                         \\ \hline
$20\times 20$                          & \multicolumn{1}{c|}{0.593} & 0.491 & \multicolumn{1}{c|}{0.706} & 0.456 & $p=0.002$                         \\ \hline
$30\times 30$                          & \multicolumn{1}{c|}{0.803} & 0.397 & \multicolumn{1}{c|}{0.845} & 0.362 & $p=0.149$                         \\ \hline
$50\times 50$                         & \multicolumn{1}{c|}{0.891} & 0.310 & \multicolumn{1}{c|}{0.893} & 0.308 & $p=0.932$                         \\ \hline
$100\times 100$                         & \multicolumn{1}{c|}{0.846} & 0.360 & \multicolumn{1}{c|}{0.898} & 0.302 & $p=0.046$                         \\ \hline
$160\times 160$                          & \multicolumn{1}{c|}{0.899} & 0.301 & \multicolumn{1}{c|}{0.908} & 0.289 & $p=0.701$                         \\ \hline
$240\times 240$                         & \multicolumn{1}{c|}{0.908} & 0.289 & \multicolumn{1}{c|}{0.927} & 0.260 & $p=0.386$                         \\ \hline

    \end{tabular}
    \label{tab:super_resolution_adl}
\end{table}
\begin{table}[htbp] 
    \centering
    \caption{The statistic of the overall accuracy on privacy features recognition with or without super resolution conditions. $p < 0.05$ indicates a significant difference between with or without super resolution conditions. }
    \begin{tabular}
    {|M{0.1\textwidth}|M{0.0375\textwidth}|M{0.0375\textwidth}|M{0.0375\textwidth}|M{0.0375\textwidth}|M{0.1\textwidth}|}
    % {|c|c|c|c|c|c|}
    \hline
    \multirow{2}{*}{\textbf{Resolution}} & \multicolumn{2}{c|}{\textbf{Before}} & \multicolumn{2}{c|}{\textbf{After}} & \multirow{2}{*}{\textbf{Significance}} \\
    \cline{2-5} & avg. & std. & avg. & std. & \\ \hline
    $15\times 15$ & \multicolumn{1}{c|}{0.558} & 0.497 & \multicolumn{1}{c|}{0.602} & 0.476 & $p<0.001$ \\ \hline
    $20\times 20$ & \multicolumn{1}{c|}{0.673} & 0.469 & \multicolumn{1}{c|}{0.736} & 0.440 & $p<0.001$ \\ \hline
    $30\times 30$ & \multicolumn{1}{c|}{0.793} & 0.404 & \multicolumn{1}{c|}{0.823} & 0.381 & $p=0.038$ \\ \hline
    $50\times 50$ & \multicolumn{1}{c|}{0.851} & 0.356 & \multicolumn{1}{c|}{0.866} & 0.340 & $p=0.276$ \\ \hline
    $100\times 100$ & \multicolumn{1}{c|}{0.895} & 0.305 & \multicolumn{1}{c|}{0.906} & 0.291 & $p=0.359$ \\ \hline
    $160\times 160$ & \multicolumn{1}{c|}{0.905} & 0.292 & \multicolumn{1}{c|}{0.913} & 0.280 & $p=0.488$ \\ \hline
    $240\times 240$ & \multicolumn{1}{c|}{0.921} & 0.268 & \multicolumn{1}{c|}{0.925} & 0.263 & $p=0.766$ \\ \hline
    \end{tabular}
    \label{tab:super_resolution_privacy}
  \end{table}