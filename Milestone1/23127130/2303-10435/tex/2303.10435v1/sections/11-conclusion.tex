\section{Conclusion}
\label{sec:conclusion}
Using the at-home activity of daily livings (ADLs) as the scenario, this paper models the trade-off of visual privacy preservation and activity recognition over image resolution. To achieve this purpose, we first conducted a user survey to obtain the most important visual privacy features, including nudity, identifiable face, valuable property, and relationship. Then, using the PA-HMDB51 dataset, which contains videos from realistic environments, we quantified the effect of image resolution on the human's performance on ADLs recognition and visual privacy awareness tasks through a user study. We further modeled the impact of image resolution on the machine's ability to recognize ADLs and visual privacy features using cutting-edge machine learning methods. Finally, we proposed a method with adjustable parameters to model the trade-off of privacy-preserving ADLs recognition using low-resolution images. Using this method, we can calculate an optimal range of image resolution for visual privacy preserving ADLs recognition. We envision that our method can inspire other vision-based systems that require balancing privacy awareness and machine recognition performance.
