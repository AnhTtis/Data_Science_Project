\section{Related work}
\label{sec:backgr-relat-work}
We describe the related work in this section, including visual privacy features and taxonomy, privacy-preserving machine recognition, and balancing the trade-off between privacy preservation and machine recognition.

\subsection{Visual Privacy Features and Taxonomy}
Privacy is described as "the right to select what personal information about me is known to what people" \cite{Westin-Privacy-and-freedom}. 
Pictures or videos convey a broad spectrum of privacy information, namely visual privacy. 
While legal and government entities legislated laws and policies on privacy protection \cite{privacy-1974, wiki-privacy-law}, their guidance leaves room for intruding visual privacy. 
Recently, researchers have explored the visual privacy exposure degree, visual privacy taxonomies/features, visual privacy importance, and visual privacy risk assessment using social media image databases~\cite{Li-Human-Perception, Orekondy_2018_CVPR, Orekondy-privacy-advisor}.
Orekondy et al. summarized 68 kinds of visual privacy features on social media images and then explored the feasibility of evaluating visual privacy exposure degree through machine learning approaches~\cite{Orekondy-privacy-advisor, Orekondy_2018_CVPR}. Li et al. summarized 7 categories, including 22 visual privacy features by crowdsourcing users' descriptions in their photo album~\cite{Li-Human-Perception}.
% Gurari et al. proposed 23 visual privacy features by summarizing images on the crowdsourcing platform for visually impaired people~\cite{Gurari_2019_CVPR}. 
These researches provide fundamental guidelines on taxonomy and the importance of visual privacy, which inspired us to design our user survey to explore the perceived importance of visual privacy in a home environment under varying image resolutions.  

\subsection{Privacy-Preserving Machine Recognition}

% We locate our work within the field of daily activity recognition in a home environment, specifically, activities of daily living (ADLs). ADLs are commonly used for health monitoring~\cite{Debes-ADL,Lawton-ADL}, summarizing activities and daily routines on which the ability of a person to live independently is assessed. Monitoring ADLs automatically is one key element in ambient assisted living (AAL) technologies, especially for elders. There are six basic ADLs: bathing/showering, dressing, feeding, functional mobility, personal hygiene, and continence. However, the home environment is a private space that contains lots of visual privacy information. Therefore, there is a strong demand for visual privacy preserving activity recognition for AAL techniques.
A growing number of privacy preserving computation technologies have emerged in recent years, which share the common promise of preserving privacy while also obtaining the benefits of computational analysis~\cite{agrawal2021computation}. 
% Generally speaking, within the field of activities of daily living (ADLs) recognition, there are two major categories of existing solutions: (1) wearable-based solutions~\cite{Pirsiavash2019, Debes-ADL, Pirsiavash2012, Nguyen-Egocentric-review} and (2) ambient sensing-based solutions~\cite{Debes-ADL, Eldib2016, visual-privacy-survey, Luo2017, Daher2017, Mashiyama2015}. Wearable solutions mainly rely on motion sensors~\cite{Pirsiavash2019} or egocentric cameras~\cite{Debes-ADL, Pirsiavash2012, Nguyen-Egocentric-review} to recognize activities accurately. However, they require devices to be attached to the user's body and suffer from limited battery life. Ambient sensing solutions, on the other hand, install devices in the home environment, such as RGB cameras~\cite{Eldib2016,visual-privacy-survey}, passive infrared (PIR) motion sensors~\cite{Luo2017}, depth cameras~\cite{Daher2017}, or low-resolution thermal image sensors~\cite{Mashiyama2015}. Among all these sensors, the optical image sensor attracted researchers due to its advantages of cost efficiency and ubiquitousness~\cite{Eldib2016,visual-privacy-survey}. 
To preserve visual privacy, existing solutions mainly adopted post-processing techniques such as image blurring and encryption techniques for images containing visual privacy information, e.g., human faces~\cite{visual-privacy-survey, Ilia-Face-Off, Lang-privacy-design,Ryoo2017, Gross2009, Boult2005, Saini2014, Frome2009}. However, these solutions are insufficient to protect all privacy information, including readable addresses, phone numbers, etc.~\cite{visual-privacy-survey, Ryoo2017, Ryoo-2016}. 
% Currently, in dynamic scenarios, post-processing methods to preserve all visual privacy cues are nearly possible~\cite{Orekondy-privacy-advisor}. 

Recently, researchers proposed a fundamental solution for a privacy-preserving vision-based system --- to lower the image sensor's resolution from the hardware level~\cite{Miyazaki2015, Ryoo2017, Ryoo2018, Xu-fully-coupled, Chen2016, Xu-pose-low-res}. 
Specifically, Miyazaki et al. developed a technology that can accurately detect the flow of people on low-resolution videos in which the faces cannot be distinguished~\cite{Miyazaki2015}. 
Dai et al. simulated a privacy-protected smart room prototype and then studied the performance impact of the image resolution from a single pixel to 10 $\times$ 10 pixels~\cite{Dai-privacy-activity}. They evaluated that five 10 $\times$ 10 resolution cameras can achieve a fairly high accuracy of 89.6\% on recognizing 9 human poses. 
Ryoo et al. proposed the inverse super-resolution (ISR) method for activity recognition on ultra-low-resolution videos, which also achieved state-of-art recognition accuracy while preserving identifiable personal information~\cite{Ryoo2017, Ryoo2018}.
% Xu et al. published a fully-coupled two-stream spatio-temporal network for human behavior recognition using extremely low-resolution videos~\cite{Xu-fully-coupled}.
%There are four important of the framework: 1) a fully-coupled network trained with high-resolution images to learn cross-domain transformation between high and low-resolution feature spaces; 2) a 3D convolutional components that extract compact and efficient spatio-temporal features for short video units; 3) a Recurrent Neural Network (RNN) for long-range temporal motion information; 4) two network streams for detailed motion features between adjacent two frames.
%As for the latest research, there is an on-going study conducted by Department of Computer Science and School of Medicine in Stanford University[28], which pays special attention on the privacy intrusiveness and violation issue caused by healthcare visual surveillance in hospitals. To solve this problem, they are building a privately-trained DCSCN super-resolution model that can enhance the utility of downsampled low-resolution depth images. Hopefully, this framework can guarantee a high degree of privacy while retaining enough utilities to perform healthcare-related human behavior detection.

These solutions showed the feasibility of activity recognition on low-resolution images. However, they only assumed an image resolution threshold (e.g., 10 $\times$ 10~\cite{Dai-privacy-activity}) to be able to preserve visual privacy without evidence. 
% Further, they failed to answer the question on "what's a proper resolution range for visual privacy preserving activity recognition?"
% only evaluated the machine's recognition performance with a preset image resolution without proper guidance (e.g., 10 $\times$ 10~\cite{Dai-privacy-activity}).
Obviously, the lower the resolution is, the better the visual privacy can be preserved. However, a lower resolution will inevitably decrease the amount of information for activity interpretation. It remains unknown how to balance the two adversarial demands on image resolution for recognizing the activity and safeguarding visual privacy. Our work is to answer this question by proposing a mathematical trade-off model and a method to calculate the optimal resolution range.

% \subsection{Effect of Image Resolution on Visual Privacy Awareness}

\subsection{Balancing Privacy Preservation and Machine Recognition}
%To evaluate the effect of image resolution on privacy awareness, a fine gain \xueyang{fine gain or fine-grained?} model is highly demanded. 
Researchers have discussed the trade-off between privacy preservation and activity recognition by quantifying humans' perceptions of privacy features. 
% An essential step in evaluating privacy preservation effects is to estimate people's ability to recognize privacy features with different privacy-preserving technologies. 
Some existing works have explored the effect of image resolution on human ability in facial recognition~\cite{Harmon-Masking, Yip-face-recognition}. Harmon and Julesz found that humans are good at facial recognition even when the portrait's resolution is down to 16 $\times$ 16 pixels~\cite{Harmon-Masking}. Yip and Sinha found that humans can still recognize celebrities' faces on portraits with a resolution of merely 7 $\times$ 10 pixels~\cite{Yip-face-recognition}. Some researchers also explored the impact of blur or pixelize filters at various levels on visual privacy awareness and activity recognition in the context of common workplace activities~\cite{Boyle-effects-of-filtered-video} or crowdsourced behavioral video coding~\cite{Lasecki-trade-offs}. They concluded the feasibility of achieving activity awareness while preserving visual privacy when tested on human eyes.

Taking human or the machine recognition performances into account, some researchers tried to understand how to balance privacy preservation and recognition performance. Alharbi et al. evaluated the effect of varying degrees of obfuscation on bystander privacy and visual confirmation utility~\cite{alharbi2019mask}. Hasan et al. studied the relative trade-offs between privacy (revealing and concealing selective attributes of objects) and utility (the visual aesthetics and user satisfaction of the image) of different image transforms~\cite{hasan2019experience}. Wu et al. formulated a novel adversarial training framework to learn anonymization transform for input videos such that the trade-off between target utility task performance and the associated privacy budgets is explicitly optimized on the anonymized videos~\cite{wu2019framework}.

However, these existing works have three limitations. First, They only tested humans' interpretation ability. An intelligent application relies on the machine for the main recognition task rather than the human. A more comprehensive study is highly demanded to explore how resolution affects both the human and the machine's recognition performance. Second, they mainly regarded the character's face as a privacy feature, which is insufficient to quantify a fine-grained model for privacy-preserving applications. Third, applying post-processing filters to high-resolution images differs from lowering the image sensor's resolution, which can preserve the visual privacy information from the hardware level with fewer on-device computing resources required.

Our work fills the gap mentioned above. We targeted enabling visual privacy-preserving machine recognition applications on low-resolution image sensors. We modeled the effects of image resolution on both the human and machine's ability in activity recognition and visual privacy awareness. Further, we proposed a quantitative survey method to model the importance of comprehensive visual privacy features.

%Closest to our work, some researchers explored the impact of blur or pixelize filter at various levels on both visual privacy awareness and activity recognition in the context of common workplace activities~\cite{Boyle-effects-of-filtered-video} or crowdsourced behavioral video coding~\cite{Lasecki-trade-offs}. However, they only consider face identification as the visual privacy, which is not sufficient to quantify a fine gain model for calculating the optimal resolution for visual privacy preserving applications. Besides, they only tested the performance of human recognition performance on activities. However, an intelligent application relies on the machine for the main recognition task rather than the human. 
