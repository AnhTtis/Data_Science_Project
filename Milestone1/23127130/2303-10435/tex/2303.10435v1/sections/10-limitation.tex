\section{Limitations and Future Work}

Our work is targeted at modeling the trade-off between visual privacy and the core machine recognition task, e.g., activity recognition in our case. Our purpose is to inspire future work to explore more quantitative methods for privacy-preserving applications. Therefore, future designers can apply or adapt these models according to their applications to preserve users' privacy as much as possible. However, there do exist several limitations of our work and we describe them below.

\paragraph{Utilizing Multimodal Information}

We acknowledge that an image sensor deployed at home can only collect images at a fixed position, distance, and field of view after installation. 
Only with the single modality data captured by images sensors, both machine's and human's recognition performance can be easily affected by the aforementioned factors.
We also acknowledge that we didn't take multimodal data, for example, audio data into consideration. Prior works have proved the effectiveness of leveraging multimodal data in activity recognition. With multimodal information, we can alleviate existing algorithms' dependence on images, thus allowing for a lower resolution of image sensors.
We expect future research can investigate how the modeling results of the trade-off between privacy preservation and activity recognition can be changed by multimodal information.

\paragraph{Privacy Preserving Methods}

In this work, we only use pixelization filters as the privacy-preserving method for the main task. The advantages of using low-resolution images have already been discussed in prior works. Nevertheless, we have to admit that researchers have shown that low resolution alone does not provide enough privacy guarantees. McPherson et al. found that obfuscated images contain enough information correlated with the obfuscated content to enable accurate reconstruction of the latter~\cite{McPherson2016defeating}. Although we have compared the privacy recognition performance of state-of-art machine learning algorithms on low-resolution images, we believe that our evaluation results on low-resolution images leave much room for discussion. We expect future research can explore the effect of more privacy-preserving methods on the trade-off between privacy preservation and activity recognition.

\paragraph{User Survey on Importance of Visual Privacy Features}

We acknowledge that our user study in section~\ref{sec:study1} aims to assess users' perceived importance of visual privacy features. We didn't limit participants' culture, age, gender, or technical backgrounds. However, there are many other factors that may affect participants' perception of privacy. For example, researchers have found that users on Amazon Mechanical Turk, where our participants were from, tend to be more privacy conscious~\cite{kang2014mturk,ross2010demographics}, thus are not representative of the general population all over the world. 
It is also undeniable that the perception of privacy varies substantially across cultures, societies, times, and locations~\cite{albayaydh2022jordan, ahmed2017digital-privacy, ahmed2017sim, crabtree2017repacking, palen2003unpacking, sambasivan2018rich, kang2015everywhere}. Therefore, our estimation of the perceived importance ($\omega$ in our formulation) of privacy features obtained through our user studies is possibly not applicable to populations in different cultural contexts across the world. However, the framework proposed in this paper is meant to inspire future researchers to consider humans' assessments of the importance of different visual privacy features. We expect that there will be more independent works to explore the influence of other factors on humans' perception of privacy.
