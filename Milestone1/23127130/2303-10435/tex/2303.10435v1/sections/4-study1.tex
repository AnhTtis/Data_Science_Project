\section{Quantifying the Importance of Visual Privacy Features}
\label{sec:study1}
Our first user study aims to understand what visual privacy features users value the most and quantify the importance of those visual privacy features, thus simplifying the to-be-built model (Equation~\ref{eq:problem}). Inspired by related works~\cite{Orekondy-privacy-advisor,categories-of-privacy,Li-Human-Perception}, we obtained 25 visual privacy features that exist in a home environment. Here we divide them into 5 categories as below. 

\begin{itemize}
    \item \textbf{Biometric Identification}: identifiable face, gender, skin color, age group, weight group, hair color, eye color, and height group.
    \item \textbf{Personal Marker/Information}: nudity, home address, number/code, medical treatment, physical disability, handwriting, birthday, clothing, and tattoo.
    \item \textbf{Ethnicity}: religion, race, and nationality.
    \item \textbf{Society}: relationship, employment and pet.
    \item \textbf{Safety}: valuable property and living schedule.
\end{itemize}

%We conducted a user survey to explore user perceived importance of these visual privacy features and categories. Therefore, we can focus on the most user-concerned visual privacy features, 

% \input{tables/privacy_category.tex}

% \subsection{Pilot Study to Identify the Rating Method}
% We first conducted a pilot study to explore the most suitable rating method for the quantitative user survey. We tested three methods: 1) 5-point Likert scale rating; 2) 100-point rating in a slider design with a range from 1 to 100; and 3) participants sort a list of tags in the order of visual privacy. We also applied attention check questions to filter out invalid answers. 

% We recruited 40 participants (23 males and 17 female) with an average age of 31.2 (s.d. = 10.4) from Amazon Mechanical Turk (MTurk)~\footnote{https://www.mturk.com/}. Each participants rated the 25 visual privacy features with three rating methods. The whole user survey lasted around 20 minutes. Each participant who passed the attention check received a 8 USD Amazon gift card. We gathered and analyzed their answers and feedback. Results show that participants preferred scoring strategy considering the difficulty of sorting 25 visual privacy features. Specifically, 32 participants chose 100-point scale and 24 participants chose 5-point Likert scale as their most preferred rating method. Further, results show that there is no statistic significance among the top 15 rated visual privacy features. This indicates that 5-point Likert scale is not sufficient to differentiate the priority of 25 visual privacy features when compared with 100-point scale. Therefore, we utilized 100-point scale for the user survey below. 

\subsection{User Survey on Importance of Visual Privacy Features} 
We recruited 125 participants (66 females, 59 males) from MTurk. They had an average age of 32.7 (s.d. = 14.7). The whole survey lasted around 15 minutes. Each participant who passed the attention check received a 6 USD Amazon gift card. 

In the user survey hosted by Qualtrics~\footnote{https://www.qualtrics.com/}, we first introduced the smart home scenario where cameras are installed for ADLs recognition. Then we asked the participants to assume that they were living in the demonstrated house/apartment. 
% \input{tikz/rank_interface.tex}
% The user interface is illustrated in Figure~\ref{fig:rank_interface}. 
% On the one hand, each participant was required to finish the question of rating the importance of 5 privacy categories using 5-point Likert scale. The participant could click the options from the most important to the least important one by one with 1 for the most important and 5 for the least important. This is to explore how user value each category of visual privacy shown in the left column of Table~\ref{tab:privacy_category}. 
Then we evaluated the importance of each privacy feature with or without low-resolution to find out what privacy features users value the most and explore the effect of low-resolution on users' perceived importance of privacy features. 
For the high resolution test, we showed participants five high-resolution ($300 \times 300$) images. Each image captured one of the five basic daily activities: functional mobility, feeding, intimacy, entertainment, and personal hygiene. We did not control the participants' backgrounds regarding their culture, age, gender, and technical knowledge. In the instruction, we explicitly stated the scenario of visual privacy leakage as their similar pictures were posted on the Internet and thus can be accessed by everyone.
%\changed{We pointed to them explicitly that the images they saw in the questionnaire which contained certain types of privacy information could have been leaked to the public and possibly go everywhere on the Internet. Despite the difference in technical background of the participants which may have an impact on their perception of privacy as suggested by~\cite{kang2015everywhere}, we have tried out best to ensure that the way and content of privacy leakage understood by the participants are consistent in our study.} 
For the low-resolution test, we just showed participants the same five images in low-resolution ($50 \times 50$).
Under each resolution, the participant was asked how he/she values the importance of the different visual privacy information listed in the questionnaire. Then, the participants were required to rate the importance of each visual privacy feature using a 100-point slider where 0 stands for not important at all and 100 stands for extremely important. The score of each privacy feature shown on the slider updates along with the participant's choice. 

We designed two attention check questions under each condition. Each attention check question requires the participant to slide to a certain score that was generated randomly before each survey. All the questions were provided to the participant in random order. 
\begin{table*}[!ht] 
    \small
    \centering
    \caption{The statistic of the user rated importance scores of the 25 visual privacy features in 5 categories with and without the low-resolution conditions. $p < 0.05$ indicates significant difference between high and low resolution conditions. }
    \begin{tabular}
    {|M{0.2\textwidth}|M{0.17\textwidth}|M{0.07\textwidth}|M{0.07\textwidth}|M{0.07\textwidth}|M{0.07\textwidth}|c|}
    % {|c|c|c|c|c|c|c|}
    \hline
    
    \multirow{2}{*}{\textbf{Category}} & \multirow{2}{*}{\textbf{Feature}} & \multicolumn{2}{c|}{\textbf{High Resolution}} & \multicolumn{2}{c|}{\textbf{Low Resolution}} & \multirow{2}{*}{\textbf{Significance}} \\
    
    \cline{3-6} & & avg. & std. & avg. & std. & \\
    
    \hline
    \multirow{8}{*}{Biometric Identification} & Identifiable Face & 60.2 & 24.3 & \textbf{57.5} & 26.0 & $p = 0.13$\\
    \cline{2-7} & Gender & 43.5 & 29.2 & 43.4 & 29.4 & $p = 0.81$\\
    \cline{2-7} & Skin Color & 42.0 & 28.6 & 43.1 & 27.3 & $p = 0.94$\\
    \cline{2-7} & Age Group & 42.9 & 25.1 & 41.2 & 25.8 & $p = 0.35$\\
    \cline{2-7} & Weight Group & 43.9 & 27.2 & 40.9 & 27.2 & $p = 0.16$\\
    \cline{2-7} & Hair Color & 36.2 & 27.4 & 40.9 & 28.1 & $p = 0.05$\\
    \cline{2-7} & Eye Color & 40.4 & 28.9 & 40.3 & 28.4 & $p = 0.90$\\
    \cline{2-7} & Height Group & 37.3 & 25.8 & 40.0 & 27.7 & $p = 0.30$\\
      % \toprule
    \hline
    \multirow{9}{*}{\parbox{0.15\textwidth}{ \centering Personal Marker / Information}} & Nudity & 61.6 & 30.9 & \textbf{62.9} & 29.4 & $p = 0.71$\\
    \cline{2-7} & Home Address & 62.8 & 23.1 & 55.6 & 26.1 & $p = 0.01$\\
    \cline{2-7} & Number/code & 57.5 & 25.5 & 55.6 & 26.6 & $p = 0.79$\\
    \cline{2-7} & Medical Treatment & 60.4 & 23.2 & 51.7 & 25.9 & $p < 0.001$\\
    \cline{2-7} & Physical Disability & 52.1 & 25.1 & 49.4 & 26.0 & $p = 0.25$\\
    \cline{2-7} & Hand Writing & 52.6 & 26.4 & 44.9 & 27.7 & $p < 0.01$\\
    \cline{2-7} & Birthday & 54.2 & 26.8 & 44.7 & 28.5 & $p < 0.01$\\
    \cline{2-7} & Clothing & 40.5 & 27.9 & 41.5 & 27.5  & $p = 0.94$\\
    \cline{2-7} & Tattoo & 42.2 & 28.7 & 39.2 & 28.6 & $p = 0.34$\\
    \hline
    
    \multirow{3}{*}{Ethnicity} & Religion & 41.8 & 27.7 & 44.6 & 26.6 & $p = 0.29$\\ 
    \cline{2-7} & Race & 40.1 & 26.5 & 42.2 & 27.7 & $p = 0.64$\\
    \cline{2-7} & Nationality & 42.1 & 28.3 &  41.3 & 27.5 & $p = 0.46$\\
    \hline
    
    \multirow{3}{*}{Society} & Relationship & 60.3 & 24.8 & \textbf{52.9} & 25.7 & $p < 0.001$\\
    \cline{2-7} & Employment & 58.2 & 22.8 & 52.1 & 25.8 & $p = 0.05$\\
    \cline{2-7} & Pet & 37.3 &  24.4 & 39.1 & 27.8 & $p = 0.46$\\
    \hline
    
    \multirow{2}{*}{Safety} & Valuable Property & 64.0 & 25.0 & \textbf{59.6} & 26.1 & $p = 0.34$\\
    \cline{2-7} & Living Schedule & 59.3 & 24.4 & 59.1 & 26.3  & $p = 0.10$\\
    \hline
      % \bottomrule
    \end{tabular}
    \label{tab:privacy_importance}
  \end{table*}
\subsection{Result}
In total, we received 115 valid responses out of 120 total responses, in which respondents successfully completed the survey and passed all attention check questions. We utilized the Wilcoxon signed-rank test ($p < 0.05$) and Friedman test ($p < 0.05$) for statistical analysis since the rating scores are ordinal.

% Results reveal significance for the human perceived importance on different visual privacy categories ($\chi^2(4, N=115) = 37.6, p < 0.001$) using Friedman test. We used the Nemenyi test as a post-hoc test to illustrate the pairwise comparison between visual privacy categories. Results shown in table~\ref{tab:catergory_sig} indicate that people value bio-metric identification ($avg. = 3.53, s.d. = 1.33$) and personal marker/information ($avg. = 3.67, s.d. = 1.36$) more than other categories of visual privacy, including safety ($avg. = 4.00, s.d. = 1.34$), society ($avg. = 4.12, s.d. = 1.35$) or ethnicity ($avg. = 4.68, s.d. = 1.43$). Participants rated safety to be significantly more important than ethnicity. Further, society is significantly more important than ethnicity. \input{tables/category_sig.tex}

The analysis results are listed in Table~\ref{tab:privacy_importance}. We concluded with the following findings.

1. \textbf{Lowering the image sensor's resolution can significantly decrease users' concerns about visual privacy}. Table~\ref{tab:privacy_importance} shows the average and the standard deviation of the rating score of each visual privacy feature under two different resolution conditions. 
% Results show that users rated a significant lower score on the importance of visual privacy on low-resolution images ($Z=-4.02, p < 0.001$). 
On average, people rated visual privacy features with significantly lower importance scores ($Z=-4.02, p < 0.001$) under the low-resolution condition ($avg. = 45.1$) than the high-resolution condition ($avg. = 49.3$). 


2. \textbf{Identifiable face, nudity, home address, number/code, medical treatment, relationship, employment, valuable property, and living schedule are considered to be more important than other visual privacy features.} Statistic analysis indicates that visual privacy features have significant effects on the human perceived important scores under either the high-resolution condition ($\chi^2(25, N=115) = 298.5, p < 0.001$) or low-resolution condition ($\chi^2(25, N=115) = 169.9, p < 0.001$). When we ran the pairwise statistical analysis using Wilcoxon signed-rank test among visual privacy features, we concluded with the following major results. On both high-resolution and low-resolution images, identifiable face, nudity, home address, number/code, medical treatment, relationship, employment, valuable property, and living schedule were considered the most important visual privacy features, since users rated them with significantly higher scores than other features ($p < 0.05$). Among these important privacy features, medical treatment and employment were considered less important ($p < 0.05$). 
%nudity has a significantly higher importance score ($avg. = 62.9, s.d. = 29.4$) than other visual features ($p < 0.05$) except identifiable face ($p = 0.25$), valuable property ($p = 0.24$) and living schedule ($p = 0.33$). 
%Participants rated identifiable face a higher importance score than other visual privacy features ($p < 0.05$) except home address ($p = 0.25$), nudity ($p = 0.25$), number/code ($p = 0.31$), valuable property ($p = 0.72$) and living schedule ($p = 0.74$). 
%Number/code is considered to be more important than other visual features ($p < 0.05$) except identifiable face ($p = 0.31$), relationship ($p = 0.15$), employment ($p = 0.20$), valuable property ($p = 0.24$) and living schedule ($p = 0.33$). 
%Home address is considered to be more important than other visual features ($p < 0.05$) except identifiable face ($p = 0.25$), number/code ($p = 0.77$), medical treatment ($p = 0.18$), physical disability ($p = 0.12$), relationship ($p = 0.18$), employment ($p = 0.26$), valuable property ($p = 0.16$). 
%Valuable property is considered to be more important than other visual features ($p < 0.05$) except identifiable face ($p = 0.72$), home address ($p = 0.16$), and living schedule ($p = 0.87$). 
%Living schedule is considered to be more important than other visual features ($p < 0.05$) except identifiable face ($p = 0.74$), nudity ($p = 0.33$), number/code ($p = 0.12$), and valuable property ($p = 0.87$). 

3. \textbf{Nudity, identifiable face, valuable property, and living schedule are the most important privacy features despite the image resolution.} When compared with the high-resolution condition, we observed significantly lower importance scores on features including home address ($p = 0.01$), medical treatment ($p < 0.001$), and relationship ($p < 0.01$) under the low-resolution condition. This finding is reasonable since these privacy features require high-resolution details to interpret. For instance, people were less concerned about the readable texts on low-resolution images. However, nudity, identifiable face, valuable property, and living schedule still lead to the most concerned visual privacy features in the low-resolution setting, with an average score above 57.

Instead of considering all the visual privacy features, we want to explore the most concerned ones that have the highest importance score and are potentially still vulnerable to low-resolution images. Therefore, we chose the most important visual privacy features in each category under the low-resolution condition with a minimum importance score threshold of 50.0. As a result, four visual privacy features including \textbf{\textit{nudity}}, \textbf{\textit{identifiable face}}, \textbf{\textit{valuable property}} and \textbf{\textit{relationship}} were chosen for later user studies and analysis. 
% We acknowledge that these four visual privacy features are limited to the scenario of daily activity recognition in the home environment. However, we envision that the research methods we introduced in this section can be applied to other scenarios. \xueyang{I didn't understand the last two sentences.}

