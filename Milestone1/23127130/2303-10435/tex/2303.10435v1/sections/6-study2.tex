\section{Effect of Resolution on Human's Recognition Performance}
\label{sec:study2}
\begin{figure*}[!ht]
    \centering
    \includegraphics[width=1\textwidth]{figures/source/interface.pdf}
  \caption{Example of the web-based user interface. Video clips of different resolutions is displayed on the left side. All tasks are listed on the right side of the web page.}
  \label{fig:us2_interface}
  \Description{Example of the web-based user interface. Video clips of different resolutions is displayed on the left side. Users are required to watch a video before answering the questions. All tasks including ADLs recognition, facial identification, nudity recognition, property detection, relationship classification, and attention checks are listed on the right side of the web page.}
\end{figure*}

After identifying the most important visual privacy features in the first study, we model the effect of image resolution on human performance in recognizing activities of daily living and visual privacy features. We describe the procedure and results in this section. 

\subsection{User Interface}
\label{sub:us2_user_interface_and_alg}

We developed a web-based user interface as shown in Figure~\ref{fig:us2_interface}. Each problem set in the test for the participants includes one ADLs recognition task and four privacy feature recognition tasks including face, nudity, valuable property, and relationship. The user interface also includes attention-check questions in each test. Responses with incorrect answers to the attention check questions were treated as invalid. A starting page, shown before the testing procedure, introduces the purpose of the user study and requires the participant's demographic information.

We sampled the image resolutions into seven values including $15 \times 15$, $20 \times 20$, $30 \times 30$, $50 \times 50$, $100 \times 100$, $160 \times 160$ and $240 \times 240$. We utilized a randomization strategy on the back-end server so that each participant could view 4 randomly chosen videos, with each video in a random resolution among these seven values. The same video did not appear twice to each participant. In addition, different clips from the same video did not appear to the same participant.

\subsection{Participant and Procedure}

We recruited 240 participants (105 females, 135 males) with an average age of 22.23 (s.d. = 5.25, ranging from 18 to 30). All participants were required to have healthy eye conditions without any historical disease (e.g., color blindness) and use their laptop or desktop web browser to finish the whole test. 
The starting page of the web-based user interface introduced the purpose of the study. 
Participants were required to fill in their demographic information, including gender, age, and historical eye diseases. Following were two practice tests using two $240 \times 240$ resolution example videos excluded from the evaluation dataset. Finally, each participant finished the 28 rounds of the test. The user study lasted around 10 minutes. Each participant was offered a 5 USD gift card for compensation. 

\subsection{Results and Findings}

In total, we obtained 6, 720 answer records, with 457 (6.80\%) invalid due to the failure of the attention check questions. We utilized One-way ANOVA for the statistic analysis ($p < 0.05$) with independent-samples t-test ($p < 0.05$) as post-hoc analysis. We present our major results and findings below.

\begin{figure}[!bp]
  \centering
  \begin{subfigure}{0.475\textwidth}
    \centering
    \includegraphics[width=1\textwidth]{figures/source/questionnaire.pdf}
    % \includegraphics[width=1\textwidth]{figures/source/questionnaire_labeled.pdf}
  \end{subfigure}
  \caption{Humans' recognition performance on main activity and privacy feature recognition tasks.}
  \label{fig:us2_result}
  \Description{Humans' recognition performance on main activity and privacy feature recognition tasks. In the picture, the x-axis represents resolution and the y-axis represents accuracy. We plotted humans' recognition performance on tasks as curves including activity classification, facial identification, nudity recognition, property detection, and relationship classification.}
\end{figure}
% \begin{table*}[ht] 
    \centering
    \caption{Humans' recognition performance on main activity and privacy feature recognition tasks.}
    \begin{tabular}{|M{0.275\textwidth}|M{0.075\textwidth}|M{0.075\textwidth}|M{0.075\textwidth}|M{0.075\textwidth}|M{0.075\textwidth}|M{0.075\textwidth}|M{0.075\textwidth}|}
    % {|c|c|c|c|c|c|c|c|}
    \hline
    \textbf{Resolution}  & \textbf{$15 \times 15$} & \textbf{$20 \times 20$} & \textbf{$30 \times 30$}  & \textbf{$50 \times 50$} & \textbf{$100 \times 100$} & \textbf{$160 \times 160$} & \textbf{$240 \times 240$}\\
    \hline
   \textbf{ADLs Recognition} & 37.5\% &  52.5\% & 75.8\% & 88.4\% & 89.6\% & 89.9\% & 90.6\% \\
    \hline
    \textbf{Facial Identification} & 24.5\% & 29.5\% & 42.8\% & 62.9\% & 79.2\% & 84.8\% & 88.0\% \\
    \hline
    \textbf{Property Detection} & 78.7\% & 80.1\% & 80.1\% & 80.4\% & 81.2\% & 81.4\% & 81.8\% \\
    \hline
    \textbf{Nudity Detection} & 53.7\% &  66.4\% & 83.7\% & 89.4\% & 90.9\% & 91.5\% & 91.6\% \\
    \hline
    \textbf{Relationship Classification} & 51.7\% & 62.4\% & 80.7\% & 91.1\% & 93.0\% & 93.3\% & 93.5\% \\
    \hline
    \end{tabular}
    \label{tab:us2_result}
  \end{table*}

\textbf{Low-resolution images are effective in preserving visual privacy but the effects are highly dependent on privacy features.} Figure~\ref{fig:us2_result} shows the effect of image resolution on human recognition performance of ADLs, face, valuable property, nudity, and relationship. We observed the significant effect of image resolution on all visual privacy recognition tasks ($p < 0.001$). 
Further, there is no significant difference between resolutions of $160 \times 160$ and $240\times 240$, indicating that resolutions above $160 \times 160$ pixels do not further contribute to visual privacy awareness statistically.
However, the effect of the image resolution is highly task-dependent. Statistical analysis indicates that the type of privacy features has significant effects on the perception performance ($F_{3,25048}=427.2$, $p < 0.001$). Specifically, pair-wise comparisons show that human eyes are more sensitive to nudity ($p < 0.001$) when the image resolution is below $50 \times 50$ pixels, followed by the relationship task. However, tasks including face identification and valuable property recognition require higher resolution images ($\geq 100 \times 100$ pixels) to achieve higher performance. For example, participants can only identify human faces with an accuracy of 79.2\% when the resolution is $100\times 100$ pixels. This is because both face identification and valuable property rely on detailed visual information. Therefore, a low-resolution image sensor can preserve but not fully protect visual privacy from the perspective of a human recognizer. 

\textbf{Lowering the image resolution has a significant negative impact on human recognition performance on ADLs.} Results show that there is a statistically significant effect of changed resolution on human ADLs recognition performance ($F_{6,6256}=278.0$, $p < 0.001$). With resolutions lower than $30\times 30$ pixels, human eyes can only recognize the ADLs with an accuracy below 75.8\%. When the image resolution increases to $50 \times 50$ pixels, participants can recognize the activity with a fair accuracy --- 88.4\%. However, participants are aware of some privacy features at the resolution of $50 \times 50$. For example, they can recognize the relationship and nudity with an accuracy of 91.1\% and 89.4\%, respectively.