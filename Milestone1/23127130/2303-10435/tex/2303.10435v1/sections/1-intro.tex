\section{Introduction}
\label{sec:introduction}
The advances in technological engineering have enabled cameras to be increasingly ubiquitous.
Nowadays, many cameras can be manufactured at a low cost in a power-efficient manner, and with small sizes.
With the help of artificial intelligence, these cameras are enabled with automatic recognition abilities, providing smart services and applications publicly or privately~\cite{Applin-Amazon}. 
%. For instance, one prevalent application is to recognize people's daily behaviors and activities
However, in realistic scenarios, this brings up a major concern --- visual privacy exposure. 
We expect a vision-based system that can bring intelligent applications while preserving visual privacy. 

To achieve this purpose, researchers have explored many post-processing methods, which were often accomplished by decoupling the personally identifiable information (e.g., face)~\cite{visual-privacy-survey,Lang-privacy-design, Gross2009,Boult2005,Saini2014,Frome2009}.
However, these solutions are not sufficient to process all visual privacy cues~\cite{Orekondy-privacy-advisor, visual-privacy-survey,Ryoo2017, Ryoo-2016}. 

\begin{figure*}[!bp]
    \centering
    \begin{subfigure}{0.625\textwidth}
      \centering
      \includegraphics[width=1\textwidth]{figures/source/demo_effect.pdf}
    \end{subfigure}
    \hfill
    \begin{subfigure}{0.325\textwidth}
      \begin{subfigure}{0.475\textwidth}
        \includegraphics[height=1\textwidth]{figures/source/demo_20x20.jpg}
        \caption{$20\times 20$}
      \end{subfigure}
      \begin{subfigure}{0.475\textwidth}
        \includegraphics[height=1\textwidth]{figures/source/demo_30x30.jpg}
        \caption{$30\times 30$}
      \end{subfigure}\\
      \begin{subfigure}{0.475\textwidth}
        \includegraphics[height=1\textwidth]{figures/source/demo_50x50.jpg}
        \caption{$50\times 50$}
      \end{subfigure}
      \begin{subfigure}{0.475\textwidth}
        \includegraphics[height=1\textwidth]{figures/source/demo_100x100.jpg}
        \caption{$100\times 100$}
      \end{subfigure}
    \end{subfigure}
    \caption{Demonstration of the effects of image resolution on the performance of the main vision-based recognition task and visual privacy awareness.}
    \label{fig:intro}
    \Description{Demonstration of the effects of image resolution on the performance of the main vision-based recognition task and visual privacy awareness. On the left figure, two curves indicate the trend of the main recognition task performance and visual privacy awareness performance over image resolution. On the right side, four images are displayed with resolutions of 20 * 20, 30 * 30, 50 * 50, and 100 * 100. In the image, an elder female is trying to stand up from a coach.}
\end{figure*}

As suggested by related works~\cite{Ryoo-2016, Ryoo2017, Ryoo2018, Xu-fully-coupled, Chen2016}, a fundamental solution toward the construction of a privacy-preserving vision-based system is to lower the image sensor's resolution from the hardware level. Thus, machines can achieve applicable performance in the main recognition task (e.g., activity recognition), while preserving visual privacy as much as possible.
Related works have proved that a low-resolution image (e.g., $16 \times 12$ pixels) possesses sufficient visual features for the main recognition task but not for visual privacy awareness. However, a high-resolution image can provide enough visual features for both of these two tasks. Thus, there is a trade-off regarding the effect of the image resolution on the main recognition task and visual privacy awareness as Figure~\ref{fig:intro} illustrates. Understanding and modeling such a trade-off will provide guidance for the privacy-preserving vision-based system with low-resolution visual sensors. 
%Thus, achieving good vision-based recognition performance and preserving the visual privacy have adversarial needs on the image resolution.
%Understanding how image resolution 
%The low-resolution image sensor can feed sufficient low-frequency information for the main recognition task, while preserve unnecessary high-frequency visual privacy information. 
%Its goal is to perform a specific vision-based recognition task while preserving visual privacy information as much as possible.
%The working principle is that visual privacy awareness requires high frequency (resolution) information, while the main recognition task can be inferred using low frequency (resolution) information. 
%For instance, researchers have proved the feasibility of conducting activity recognition on extreme low-resolution images (e.g., $16 \times 12$ pixels)~\cite{Ryoo-2017, Ryoo-2016, Xu-fully-coupled}. 

%as Figure~\ref{fig:intro} illustrates, there is a trade-off between machine recognition ability and visual privacy awareness regarding the image sensor's resolution: as image resolution goes higher and reaches some level, the visual privacy exposure increases significantly, while the performance of the main recognition tasks only has marginal increase, approaching its asymptotic-bound.
%Therefore, there is an optimal image resolution range where the machine can achieve good performance on the main recognition tasks while preserving visual privacy against attackers (humans or machines). 
%To our best knowledge, no existing work thoroughly investigated the method to identify the aforementioned image resolution range to best balance machine recognition performance with visual privacy protection. 

% \begin{figure}[ht]
%     \centering
%         \includegraphics[width=0.75\columnwidth]{figures/intro.jpg}
%     \caption{Demonstration of the effect of image resolution on the recognition of activities and visual privacy features. A: 10 $\times$ 10 pixels, B: 30 $\times$ 30 pixels, C: 50 $\times$ 50 pixels, D: 224 $\times$ 224 pixels.}
%     \label{fig:intro}
% \end{figure}

In this paper, we focus on a smart home scenario where low-resolution image sensors automatically recognize activities of daily living (ADLs), such as feeding, entertainment, personal hygiene, intimacy, and functional mobility. ADLs recognition system can summarize activities and daily routines on which the ability of a person living independently is assessed; thus is widely used for health monitoring, especially for elderly care~\cite{Debes-ADL,Lawton-ADL}. 
% In a privacy-sensitive home environment, we expect the vision-based system accurately recognize the activity while preserving as much visual privacy as possible. 
% These two demands post adversarial needs over image resolution. 
% For instance, on images with high resolution (Figure~\ref{fig:intro}D), it is easy to recognize the activity as functional mobility. However, we can also easily identify the character's face and valuable properties (e.g., necklace). On images with ultra-low-resolution (Figure~\ref{fig:intro}A), it is hard for interpreters to recognize the privacy features but the activity either. Figure~\ref{fig:intro}B has a proper resolution, on which we can recognize the main activity but not the character's face or the property. Therefore, our goal is to propose a method to calculate an optimal resolution range for the visual privacy preserving ADLs recognition application. 
In realistic home environments, the data captured by an image sensor may be single-frame pictures~\cite{Ryoo2018, Chen2016} or multi-frame videos~\cite{Gao_MMTSA, Miyazaki2015}. We regard both of them as \textit{images} to model the trade-off between privacy preservation and activity recognition.

We considered such a trade-off as an optimization problem over image resolution. We conducted an online user survey with 115 participants to obtain the most important visual privacy features including nudity, identifiable face, valuable property, and relationship. 
In this paper, we regarded both the human and the machine as recognizers. 
Thus, we explored the effect of image resolution on both human and machines' ability in activity recognition and visual privacy awareness on the PA-HMDB51 dataset, which consists of over 500 videos from realistic environments~\cite{wu2019framework}. Specifically, we conducted a user study with 240 participants to investigate the effect of image resolution on human recognition performance. We evaluated the machine's performance on ADLs and visual privacy recognition tasks with cutting-edge machine learning approaches. Finally, we built a modeling method for calculating the trade-off of visual privacy preserving ADLs recognition using low-resolution images. We envision that our method can inspire other vision-based systems that require balancing privacy awareness and machine recognition performance.
Overall, the contributions of our paper are two-fold.

1) Using the at-home ADLs recognition as a scenario, we proposed a pipeline to investigate the effect of the image resolution on both human and machine performance on the main activity recognition task and visual privacy awareness. 

2) We presented a model for calculating the trade-off of visual privacy preserving activity recognition using low-resolution images. Using the proposed model, we can calculate an optimal resolution range of the image sensor for privacy-preserving activity recognition applications. 

%3) We proposed a quantitative survey method to model the effect of image resolution on user-perceived importance of visual privacy. By applying this method to the ADLs recognition on low-resolution images, we obtained the most user-concerned privacy features: nudity, identifiable face, valuable property, and relationship.

% 1) We proposed an optimization modeling method and evaluated its feasibility to calculate the optimal image resolution range for visual privacy preserving machine recognition on activities of daily living. 
 
% 2) We proposed a quantitative survey method to model the effect of image resolution on user-perceived importance of visual privacy. By applying this method to the ADLs recognition on low-resolution images, we obtained the most user-concerned privacy features: nudity, identifiable face, valuable property, and relationship.
% %2) We applied our method in the domain of ADLs recognition, and our results showed the effectiveness of our approach and the feasibility for machine-based visual privacy preserving ADLs recognition. 

% 3) We investigated and modeled the effects of image resolution on ADLs recognition and visual privacy awareness from both the human and the machine with cutting-edge machine learning methods. 



% Orson: maybe can be replaced with the previous paragraph
% Cameras are becoming increasingly ubiquitous and pervasive recognizing people's activities or motions at public or even private places~\cite{Applin-Amazon}. Such abundance of cameras is causing a big concern --- visual privacy disclosure. 
%In order to help increase the recognition performance, researchers tended to use cameras with higher and higher resolution and FOV (e.g., \cite{nest-camera}). 
%As the percentage of the elderly population grows rapidly, there is a huge demand developing health monitoring technologies for elderly people living independently in their aging places. One of the promising solutions is to look for changes in the \textit{activities of daily living (ADLs)} \cite{Lawton-ADL} using image sensors with the advancement of computer vision technologies. However, there exists a big social challenge: visual privacy exposure from video recordings. We want a camera system to recognize important events and assist elders by understanding its videos, but not intruding their visual privacy, which is visual privacy preserving ADLs recognition. 
 
%cameras are becoming increasingly ubiquitous and pervasive recording people's everyday behavior at public or even private places. Such abundance of cameras is causing 
%a higher resolution would contribute to a higher recognition performance but lower the visual privacy preserving capability~\cite{Boyle-effects-of-filtered-video}. As a result,%\xueyang{What is the meaning of this sentence? Does it mean:
%Further, due to the increasing availability of dynamic scene computing resources, post-processing methods which can recover all visual privacy are likely on the horizon.} %Typical solutions to accomplish such a goal is de-identification by removing/blurring/covering people's partial (e.g., face) or the whole body. 


%To address this gap, we propose a method to identify an effective image resolution range that maximizes machine performance while preserving as much visual privacy information as possible. 
