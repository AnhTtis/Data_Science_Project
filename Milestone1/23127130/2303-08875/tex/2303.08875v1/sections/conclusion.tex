\section{Conclusions and further direction of work} \label{sec5}

As a result of the research, several databases competing with goleveldb in terms of performance and functionality were selected and embedded in HLF as StateDBs. Furthermore, performance benchmarks were created and conducted for each of the possible alternative StateDBs. In the end, BadgerDB was identified as a favorite of the potential StateDBs.

BadgerDB showed a decent advantage in the StateDB write performance benchmarks. BadgerDB had the strongest advantage over goleveldb for write values of 64KB (TPS is almost 1.5 times higher). In addition, BadgerDB provides StateDB with features that were not implemented with goleveldb: it guarantees ACID properties and makes it possible to use custom queries in HLF by using an additional tool \cite{l29}. 


The HLF source code with linked databases is available in the repositories: \cite{l32} (RocksDB), \cite{l33} (bbolt), \cite{l34} (BadgerDB).


To run the performance benchmarks, the HLF fabric-samples/test-network was modified to provide the minimum required set of network components in order to test the performance of StateDB peer with the new databases \url{https://github.com/fubss/fabric-samples/tree/dbs_selector}.


Based on the Hyperledger Caliper \cite{l36} demo, a custom repository was created for the HLF load, adapted for OS Linux and OS X (in different branches) \url{https://github.com/fubss/caliper-workspace-3}.


In further research, it is suggested that:
\begin{itemize}
    \item Compare other performance metrics of embedded databases;
    \item Conduct a read workload performance benchmark;
    \item Identify and eliminate other HLF components that slow it down.
\end{itemize}

\section*{Acknowledgment}

We thank Vladimir Chechetkin for embedding BadgerDB and other contributions to the work on this paper.

