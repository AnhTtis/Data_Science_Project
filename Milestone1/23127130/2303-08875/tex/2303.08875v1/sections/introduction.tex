\section{Introduction}


The increasing popularity of blockchain technology~\cite{l66, l49,l47,l48} is paving the way for businesses to start using blockchain-based solutions particularly those implemented with Hyperledger Fabric~\cite{l50,l46,l67}. Hyperledger Fabric (HLF) is an open-source initiative devoted to creating an enterprise-grade, permissioned blockchain platform~\cite{l62,l63}.


The State Database is one of HLF's most important elements (StateDB). StateDB stores a snapshot of the last blockchain network state. HLF peer relies on the StateDB to read the current values with their version to form the Read-Set for transactions. Then during the commit and validation phase, while executing the multi-value concurrency control~\cite{l65} check, peers read from StateDB to compare Read-Set versions. Finally, once validation is over, the peer commits validated transactions into StateDB to reflect recent changes.~\cite{l64}
 

According to the~\cite{l43}, the commit phase in HLF dominates the transaction processing time and, as a result, constitutes a bottleneck in terms of performance.
In addition, the study demonstrates that the read operation plays an important role when peers replicate transactions.
Clearly, the interaction with the StateDB directly affects the overall performance of the HLF.
Currently, HLF presents two potential StateDB implementations, one based on GoLevelDB and the other on CouchDB.
Therefore, the Fabric community acknowledged the necessity to develop a superior alternative to StateDB~\cite{l2}.
In this paper, we will analyze various options for StateDB and examine the various implementations of the embedded databases based on LSM-trees or B+ trees~\cite{l1, l44}. 

