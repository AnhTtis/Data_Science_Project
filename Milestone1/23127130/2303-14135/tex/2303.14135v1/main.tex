%% Beginning of file 'sample631.tex'
%%
%% Modified 2021 March
%%
%% This is a sample manuscript marked up using the
%% AASTeX v6.31 LaTeX 2e macros.
%%
%% AASTeX is now based on Alexey Vikhlinin's emulateapj.cls 
%% (Copyright 2000-2015).  See the classfile for details.

%% AASTeX requires revtex4-1.cls and other external packages such as
%% latexsym, graphicx, amssymb, longtable, and epsf.  Note that as of 
%% Oct 2020, APS now uses revtex4.2e for its journals but remember that 
%% AASTeX v6+ still uses v4.1. All of these external packages should 
%% already be present in the modern TeX distributions but not always.
%% For example, revtex4.1 seems to be missing in the linux version of
%% TexLive 2020. One should be able to get all packages from www.ctan.org.
%% In particular, revtex v4.1 can be found at 
%% https://www.ctan.org/pkg/revtex4-1.

%% The first piece of markup in an AASTeX v6.x document is the \documentclass
%% command. LaTeX will ignore any data that comes before this command. The 
%% documentclass can take an optional argument to modify the output style.
%% The command below calls the preprint style which will produce a tightly 
%% typeset, one-column, single-spaced document.  It is the default and thus
%% does not need to be explicitly stated.
%%
%% using aastex version 6.3
\documentclass[]{aastex631}
\newcommand{\vdag}{(v)^\dagger}
\newcommand\aastex{AAS\TeX}
\newcommand\latex{La\TeX}
\usepackage{natbib}

%% Reintroduced the \received and \accepted commands from AASTeX v5.2
\received{23 April, 2022}
\revised{27 April, 2022}
\accepted{6 May, 2022}


%\setcounter{table}{1}

\graphicspath{{./}{figures/}}
%% This is the end of the preamble.  Indicate the beginning of the
%% manuscript itself with \begin{document}.

\begin{document}

\title{Spectroscopic Signature of a Re-established Accretion Disk in  Symbiotic -like Recurrent Nova RS Ophiuchi}
 
\correspondingauthor{S. N. Shore}
\email{steven.neil.shore@unipi.it}

\author[0000-0002-0786-7307]{Alessandra Azzollini}
\affiliation{ Lehrstuhl f\"ur Astronomie, 
 Universit\"at W\"urzburg\\ 
 Emil-Fischer-St. 31, W\"urzburg, 97074, Germany\\
and\\ Dipartimwnto  di Fisica. Universit\'a di Pisa \\
largo B. Pontecorvo 3 \\
Pisa 56127, Italy }

\author[0000-0003-1677-8004]{Steven N. Shore }
\affiliation{Dipartimento di Fisica,Uniersit\'a di Pisa and INFN - Sezione di Pisa \\
largo B. Pontecorvo 3 \\
Pisa 56127 Italy}

\author[0000-0003-4650-4186]{N. Paul Kuin }
\affiliation{ Mullard Space Science Laboratory\\     
 University College London\\
Holmbury St. Mary, Dorking, United Kingdom}
 
 
%% Mark off the abstract in the ``abstract'' environment. 
\begin{abstract}

A novel method is presented which can pin down the time the accretion disk re-established 
itself in the RS Oph system after it experienced a nova disruption. The method is based on 
the re-ionisation of the ejecta by photoionisation from the radiation released in the boundary 
layer from accretion.  


 
\end{abstract}

%% Keywords should appear after the \end{abstract} command. 
%% The AAS Journals now uses Unified Astronomy Thesaurus concepts:
%% https://astrothesaurus.org
%% You will be asked to selected these concepts during the submission process
%% but this old "keyword" functionality is maintained in case authors want
%% to include these concepts in their preprints.
\keywords{Classical Novae (251) --- Ultraviolet astronomy(1736) }
 
\section{Introduction} \label{sec:intro}

 Among the recurrent novae, one small subgroup erupts within the wind of a close red giant companion (see, e.g., 
 \citet{Bode10}, \citet{anup13}, \citet{Darn20}).  These systems are very different than the classical, compact novae for which the ejecta are expanding freely once the explosion terminates.  In these symbiotic-like binaries, the accretion is, at least in part, driven by capture of the imbedding wind by the companion white dwarf.  A disk forms and the accretion proceeds similarly to the compact, cataclysmic type systems.  The main difference is in the post-eruption behavior of the expelled mass.  The ejecta plot into a dense, stratified environment in which a strong leading shock is formed that also modifies the ionization and density structure of the wind.  An important question, as yet unresolved, is how the system relaxes back to the pre-outburst accretion state and when the disk -- that disrupts during the explosion, reforms.  

\section{Disks and stochastic variability: flickering}

\citet{Zam220} and \citet{Rom22} have recently reported that flickering -- taken as the signature of vigorous accretion in cataclysmic binaries -- has reappeared in the post-outburst symbiotic-like recurrent nova RS Ophiuchi.  The current outburst has been followed across the spectrum.  The previous outburst, in 2006, was also extensively covered, especially well with the UVOT grism on the {\it Neil Gehrels Swift Observatory} from day 30 for about one year in the range 1700 - 7000 \AA with a resolution of about 150 \citep{Azz21}.  The 2021 outburst of RS Oph began on Aug 8 (MJD 59674), corresponding to orbital phase 0.$^p$72, with zero being inferior conjunction of the red giant \citep{Brandi09}.  The 2006 outburst began on Feb. 12, at orbital phase 0.$^p$26.  Thus, views through the wind of the giant of the white dwarf and its environs were completely different, in the 2006 event the system was seen without the intervening obscuration, around 0.$^p$65 and in this latest the orbital phase was about 0.$^p$11.  Thus, we concentrate on that spectra sequence. 

The spectrum was composite, consisting of contributions from the shock and associated ejecta from the nova, emission lines from the wind of the red giant produced by the UV precursor generated by the shock, and the additional ionization following the appearance of a soft X-ray source below 1 keV due to emission from the pointlike white dwarf.  The latter appeared only after the combined neutral column density of the ejecta and wind  was sufficiently reduced by the advance of the ionization front.  The He II 4686\AA\ and Balmer lines were especially well observed throughout the interval and we show the variations of H$\delta$ and He II in figure 1.    Overplotted we display the latest flux measurements from the 2021 eruption that follow precisely the same pattern.


\begin{figure}[ht!]
\plotone{RSOph_HeIIHd_r.pdf}
\caption{{\it Swift} UVot grism development of H$\delta$ and He II 4686\AA\ emission lines in RS Oph comparing the 2006 and 2021 outbursts (uncorrected for extinction).  The scaling between the two events is a constant factor of 22.   See text for discussion.  \label{fig:general}}
\end{figure}

The variations can be phenomenologically separated into four developmental states.  The first is from the shock during its free expansion stage and its subsequent Sedov-Taylor expansion.  There was also an accumulation of mass from the environment, a peculiarity of these systems in contrast to classical nova explosions that expand ballistically from the start.  This snowplow increased the mass of the ejecta and slowed them rapidly.  The second stage, from about day $\sim 75$, 
is breakout, when the further mass loading of the ejecta is negligible but the gas is continually ionized by  the residual nuclear burning on the white dwarf (the so-called supersoft source).  The third stage, from around day 100,  is when following the turnoff of the central source, the ballistic expansion rate competes with the recombinations \citep{Sho96}.  The final, fourth stage is when the accretion disk has reformed and re-establishes the boundary layer and again photoionizes the ejecta.  
  
It is this last stage that is signalled by a renewed increase in the strength of the higher Balmer and ionized permitted lines, especially He II 4686\AA.  In the 2006 sequence, this last stage began with minimum strength for the emission lines at after around day 150 for the higher Balmer sequence and about day 160 for He II.  In the current outburst that corresponds to after about 2022 Feb. 15, when \citet{Mar22} did not detect optical flickering.  The formation of a disk is assured in these systems by the continued accretion from the companion's wind but the photometric flickering requires the formation of a steady state disk and boundary layer.  Therefore, we propose that the beginning stages of disk formation were consistent with the last stage of emission line decline, thus providing an independent means for establishing how and when  the white dwarf recommences its buildup to the next explosion.  The delay between the photometric and spectroscopic indicators can be attributed to the time required for viscosity to bring the accretion disk to a steady state. 
 
\begin{acknowledgments}
We thank Kim Page, Jordi Jos\'e,and Marco Bellomo for discussions.   NPMK acknowledges support by the UKSA.
.\end{acknowledgments} 
%% To help institutions obtain information on the effectiveness of their 
%% telescopes the AAS Journals has created a group of keywords for telescope 
%% facilities.
%
%% Following the acknowledgments section, use the following syntax and the
%% \facility{} or \facilities{} macros to list the keywords of facilities used 
%% in the research for the paper.  Each keyword is check against the master 
%% list during copy editing.  Individual instruments can be provided in 
%% parentheses, after the keyword, but they are not verified.

\vspace{5mm}
\facilities{Swift(XRT and UVOT), AAVSO}
 %% For this sample we use BibTeX plus aasjournals.bst to generate the
%% the bibliography. The sample631.bib file was populated from ADS. To
%% get the citations to show in the compiled file do the following:
%%
%% pdflatex sample631.tex
%% bibtext sample631
%% pdflatex sample631.tex
%% pdflatex sample631.tex

%\bibliography{sample631}{}
\bibliographystyle{aasjournal}

\begin{thebibliography}{}
\bibitem[Azzollini (2021)]{Azz21} Azzollini, A. 2021, MSc thesis, Physics, Univ. di Pisa (https://etd.adm.unipi.it/t/etd-07012021-023228/)
\bibitem[Anupama (2013)]{anup13} Anupama, G. 2013, IAUS, 281, 154
\bibitem[Bode (2010)]{Bode10} Bode, M. 2010, AN, 331, 160
\bibitem[Brandi et al. (2009)]{Brandi09} Brandi, E., Quiroga, C, Mikolajewska, J, et al.  2009, A\&A, 497, 815
\bibitem[Darnley \& Henze.(1990)]{Darn20} Darnley, M. \& Henze, M. 2020, AdSpR, 66, 1147
\bibitem[Marchev et al. (2022)]{Mar22} Marchev, O. et al. 2022, ATel 15296
\bibitem[Romanov (2022)]{Rom22} Romanov, F. 2022, ATel 15338
\bibitem[Shore et al. (1996)]{Sho96} Shore, S. N., Sonneborn, G., 1996, ApJ, 463, L21
\bibitem[Zamanov et al. (2022)]{Zam220} Zamanov, R. et al. 2022, ATel 15330

\end{thebibliography}%% This command is needed to show the entire author+affiliation list when
%% the collaboration and author truncation commands are used.  It has to
%% go at the end of the manuscript.
%\allauthors

%% Include this line if you are using the \added, \replaced, \deleted
%% commands to see a summary list of all changes at the end of the article.
%\listofchanges

\end{document}
 


