\section{Modelling Interactive Behaviour with an NLP Method}%
\label{sec:method}

\begin{figure}[t]
    \centering
    \includegraphics[width=\textwidth]{pipeline.pdf}
    \caption{Overview of our pipeline of exploring modelling interactive behaviour from an NLP perspective.
    }
    \label{fig:pipeline}
\end{figure}
As Fig.~\ref{fig:pipeline} shows, the raw data (mouse and keyboard action sequences) are first segmented into subsequences.
Core to our approach is BPE %
learning a vocabulary of subwords, i.e. a set of meaningful mouse and keyboard activities, and then encoding the behaviour based on the vocabulary.
As BPE requires discrete inputs, mouse data are preprocessed additionally using
the dispersion-threshold identification (I-DT) algorithm, that converts continuous-valued mouse coordinates into discrete tokens.
The encodings generated by BPE are then evaluated in two ways to explore if a natural language-like structure exists in mouse and keyboard behaviour that can be captured by this widely used NLP method:
(1) analyse the semantic meaning of the vocabulary, i.e., interaction goals underlying learnt activities, and (2) as input to train a Transformer-based classifier for task recognition.
The two evaluations are demonstrated in Section~\ref{sec:evaluation}.

\subsection{Data Preprocessing}
\label{sec:mousepreprocessing}
Different from natural language where words and sentences are separated by spaces and punctuations, modelling interactive behaviour first requires %
splitting data into smaller units. 
Thus, a sliding non-overlapping window was used to segment the long raw data.
On the keyboard actions, the window lengths $L_{win}$ were empirically set to 10, 50, and 100.
The window lengths $L_{win}$ for the mouse actions were set to 20, 100 and 200, as we observed on both datasets that, the number of generated mouse actions for a fixed time window is roughly twice as many as the keyboard actions.
When using both modalities jointly, the window lengths were set to the mean value of those for single modalities, i.e. $L_{win}=$ 15, 75 and 150.
For keyboard actions, the action type and the key value were concatenated as a token, e.g., \textit{KeyDown\_a} (a${\downarrow}$) or \textit{KeyUp\_Shift} (Shift${\uparrow}$).
Buffalo recorded 91 key values, while EMAKI had 137 values, yielding 182 and 274 atomic actions forming the starting vocabulary, respectively.
With more types of keys, EMAKI can potentially reflect more behaviour varieties.

Participants completed our study on their own computers with different screen resolutions, so we first re-scaled the mouse coordinates to $[0,1]$.
For consistency, we re-scaled Buffalo mouse data to the same range.
We observed two categories of mouse behaviour: \textit{pinpoint}, i.e. interacting with the target UI element in a small area, where moves are shorter, slower and more concentrated, resembling gaze fixations;
and \textit{re-direction} between targets, resembling fast saccadic eye movements between fixations~\cite{salvucci2000identifying}.
Inspired by gaze fixation detection, we used I-DT~\cite{salvucci2000identifying} to preprocess mouse data (see Appendix).
Then we divided the screen equally into four areas  (0: top-left, 1: top-right, 2: bottom-left, 3: bottom-right).
The action type (move or click), mouse behaviour category (pinpoint or re-direction), and the screen area were concatenated as a token, e.g., \textit{Move\_Redirection\_Area0} or \textit{Click\_Pinpoint\_Area3}.
When representing clicks, Buffalo only recorded a \textit{Click}, while we recorded both \textit{Down} (press) and \textit{Up} (release) events.
Therefore, Buffalo has $2{\times}2{\times}4{=}16$ atomic actions and EMAKI has $3{\times}2{\times}4{=}24$.

\subsection{Encoding Mouse and Keyboard Behaviour with BPE}
\label{sec:BPE-method}

We employed BPE (see Appendix for its algorithm) to learn a vocabulary of subwords, i.e., activities that consist of various numbers of consecutive actions.
Starting from the action sequence set $D$, the vocabulary $V$ is built after $k$ iterations.
In each iteration, the most frequent pair of actions or activities form a new activity, which is added into $V$ and used to update $D$.
We consider each action as a character, given it is an inseparable, atomic unit.
The initial vocabulary is composed of actions and one extra token representing the end of the action sequence from one task trial.
Thus, the initial vocabulary sizes are $|V|_\text{mouse}=17$ and $|V|_\text{key}=183$ in Buffalo, and $|V|_\text{mouse}=25$, $|V|_\text{key}=275$ in EMAKI.
We set $k$ to 300, 600 and 900 empirically.%
