\section{Conclusion}
We explored the similarity between interactive behaviour and natural language, given that both of them have a sequential and hierarchical structure.
Towards the goal, we applied a widely used NLP method, BPE, to encode mouse and keyboard behaviour by learning its subwords, i.e., activities.
Results on an existing controlled dataset and a novel out-of-the-lab dataset showed that the method can capture meaningful activities.
Moreover, encoding with BPE %
significantly improved interactive task recognition, which is commonly required in intelligent interactive systems. %
Taken together, our exploratory work links interactive behaviour with natural language and provides a promising NLP perspective for modelling interactive behaviour, %
which has the potential to improve the generalisability of computational interactive behaviour models (Section~\ref{sec:discussion-language}) and also performances of interactive behaviour-based HCI tasks.