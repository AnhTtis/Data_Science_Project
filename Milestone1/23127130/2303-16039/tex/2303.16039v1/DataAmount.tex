\section{Analysis of EMAKI Data Amount for Interactive Task Recognition}
As written in Section 6.3 in the main paper, we evaluated if the size of EMAKI allows our data-driven method to recognise interactive tasks.
We used different percentages of the training set to train the method and examined their performances.
According to Figure 6c in the main paper, the best results were achieved by BPE-300 when windows have 150 actions.
Therefore, we followed the above setting.
Fig.~\ref{fig:dataamount} shows the results of interactive task recognition by training the model with 1\%, 5\%, 15\%, 25\%, 50\%, 75\% of randomly selected training instances, as well as with the entire training set (100\%).
It can be seen that as the percentage increases, the F1 score first increases fast (before 25\%) but then slowly (25\% to 75\%).
The increase in F1 score from using 75\% of training data and the entire training set was subtle (only 0.004).
Taken together, the amount of data in our dataset is sufficient to perform interactive task recognition.
\begin{figure}[t]
    \centering
    \includegraphics[width=.5\textwidth]{plotPercent.pdf}
    \caption{F1 scores of interactive task recognition achieved by BPE, trained on different percentages of the training set.}
    \label{fig:dataamount}
\end{figure}