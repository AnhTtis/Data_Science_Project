\section{The Algorithm of Byte Pair Encoding}
\begin{algorithm}[t]
    \caption{Byte pair encoding (BPE)~\cite{zhan2019effective,heinzerling2017bpemb}
    }
    \label{alg:BPE}
    \begin{algorithmic}
        \State \textbf{Input:} action sequence set $D$, the number of iterations $k$
        \Procedure{BPE}{$D$, $k$}
        \State $V \gets \text{all unique actions in } {D} $
        \For{$i \gets 1$ to $k$}  %
        \State $t_L, t_R \gets \text{Most frequent two consecutive units (actions or activities) in } {D} $
        \State $t_\text{new} \gets t_L + t_R$\Comment{Merge to form a new activity}
        \State $V \gets V + [t_\text{new}]$
        \State $\text{Replace each occurrence of } t_L, t_R \text{ with } t_{\text{new}} \text{ in } {D}$
        \EndFor
        \State \textbf{return} $V$ 
        \EndProcedure
    \end{algorithmic}
\end{algorithm}
Algorithm \autoref{alg:BPE} shows how byte pair encoding (BPE) constructs the vocabulary $V$, as introduced in Section 4.2 in the main paper.
