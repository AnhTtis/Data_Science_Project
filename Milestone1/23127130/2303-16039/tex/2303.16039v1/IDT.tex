\section{Preprocessing Mouse Data with I-DT}
As written in Section 4.1 in the main paper, the Dispersion-threshold identification (I-DT) algorithm~\cite{salvucci2000identifying} was used to categorise mouse behaviour to \textit{pinpointing} a target (resembling gaze fixations) and \textit{re-direction} between targets.
I-DT operates on a window of duration-threshold consecutive samples.
On this window, it calculates the dispersion value as 
$Dispersion=[max(x)-min(x)]+[max(y)-min(y)]$.
If the dispersion value exceeds the dispersion threshold, samples inside the window are not considered to belong to a pinpoint and the window is slid forward by one sample.
If the value is below the threshold, the samples within the window are considered to belong to a pinpoint.
The window then expands to incorporate new samples until the dispersion value is above the threshold again.
We empirically set the duration threshold to 100\,ms and the dispersion threshold to 0.1.