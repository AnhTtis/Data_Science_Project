\begin{figure}[t]
    \centering
    \includegraphics[scale=0.245]{images/pg_revamped.PNG}
    \vspace{-6mm}
    \caption{\textbf{Measuring attribute sensitivity in contextualized object grounding}. We find limited sensitivity to attribute meaning in default contextualized grounding, but enhanced sensitivity with (plausible) adjective negatives added. This observation is supported by  
    AP differences \textcolor{black}{(in black)} with incorrect adjectives used for the adjective negative vs. default contextualized models. The drops are discernibly larger with adjective negatives: \textcolor{Greenish}{-1.36\% } vs. \textcolor{red}{$<$0.2\% }from baseline captions to changing plausibly and \textcolor{Greenish}{-1.57\% } vs. \textcolor{red}{$<$0.2\% } from baseline captions to changing randomly. Values are avgs. over 3 training runs. Bars show std. error.}

    \label{attribute_pg_fig2}
\end{figure}  

% Through unsupervised grounding, we show that default contextualization does \textit{not} result in substantial AP@IoU=30:10:50 drops with adjectives changed (\textcolor{red}{red}), illustrating a lack of sensitivity to attribute meaning. When we add adjective negatives (plausible in this case), contextualization gains enhanced sensitivity to attribute meaning, shown in the decreases from baseline to removal to changing (\textcolor{green}{green}). Note that such trends hold over values of $th_{sim}$. The presented values are averages over 3 pretraining trials, and error bars show standard error.