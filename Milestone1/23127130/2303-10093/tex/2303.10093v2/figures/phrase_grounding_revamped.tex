\begin{figure*}[t]
    \centering
    \includegraphics[scale=0.225]{images/pg_double_revamped.png}
    \vspace{-3mm}
    \caption{\textbf{Measuring attribute sensitivity in contextualized object grounding}. In an attribute-sensitive model, grounding performance should drop if an incorrect attribute is used, which occurs when changing adjectives. Through unsupervised grounding, we show that default contextualization does \textit{not} result in substantial AP@IoU=30:10:50 drops vs. the baseline with adjectives changed (\textcolor{red}{red}), illustrating a lack of sensitivity to attribute meaning. When we add adjective negatives (plausible in this case), contextualization gains enhanced sensitivity to attribute meaning, shown in the decreases from baseline to changing (\textcolor{Greenish}{green}). \textbf{Note that such trends hold over values of $th_{sim}$}. The presented values are averages over 3 pretraining trials, and error bars show standard error.}
    \label{pg_supp_thresh}
\end{figure*} 