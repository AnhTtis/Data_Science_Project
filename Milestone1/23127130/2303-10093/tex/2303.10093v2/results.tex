 
\section{Experimental results and analysis} 
    \label{sec:results}

    In Section \ref{insights}, we analyze the attribute sensitivity of VL alignment. For OVR-CNN region-word grounding, we test removing context in the captions used for pretraining detection (Fig. \ref{remove_fig}) and perturbing captions in unsupervised phrase grounding (Fig. \ref{attribute_pg_fig2}). For CLIP image-text alignment, we test perturbing the text prompts for classification via description (Fig. \ref{classviadesc_fig}). In Section \ref{evaluate}, we further evaluate how attribute context sensitivity impacts practical downstream tasks. We evaluate the impact of attribute sensitivity on an \textit{object-focused} task, in particular open-vocabulary detection with OVR-CNN (Table \ref{negative_strategies}/\ref{soa}). We also evaluate models on two \textit{fine-grained} tasks that require attribute knowledge, namely, text-region retrieval and object attribution (Table \ref{clipretrieval}/\ref{retrieval}).


    \subsection{Gauging the role of attribute context}
        \label{insights}
        \begin{figure}[t]
    \centering
    \includegraphics[scale=0.31]{images/updated_remove.PNG}
    \caption{\textbf{The impact of language context in pretraining for open-vocabulary detection (MiniCOCO)}. 
    We pretrain with adjective modifiers, prepositional phrases, or verb phrases \emph{removed}. We conclude the context-free method largely ignores this context.}     
    \label{remove_fig}
\end{figure} 

        \begin{figure}[t]
    \centering
    \includegraphics[scale=0.245]{images/pg_revamped.PNG}
    \vspace{-6mm}
    \caption{\textbf{Measuring attribute sensitivity in contextualized object grounding}. We find limited sensitivity to attribute meaning in default contextualized grounding, but enhanced sensitivity with (plausible) adjective negatives added. This observation is supported by  
    AP differences \textcolor{black}{(in black)} with incorrect adjectives used for the adjective negative vs. default contextualized models. The drops are discernibly larger with adjective negatives: \textcolor{Greenish}{-1.36\% } vs. \textcolor{red}{$<$0.2\% }from baseline captions to changing plausibly and \textcolor{Greenish}{-1.57\% } vs. \textcolor{red}{$<$0.2\% } from baseline captions to changing randomly. Values are avgs. over 3 training runs. Bars show std. error.}

    \label{attribute_pg_fig2}
\end{figure}  

% Through unsupervised grounding, we show that default contextualization does \textit{not} result in substantial AP@IoU=30:10:50 drops with adjectives changed (\textcolor{red}{red}), illustrating a lack of sensitivity to attribute meaning. When we add adjective negatives (plausible in this case), contextualization gains enhanced sensitivity to attribute meaning, shown in the decreases from baseline to removal to changing (\textcolor{green}{green}). Note that such trends hold over values of $th_{sim}$. The presented values are averages over 3 pretraining trials, and error bars show standard error.


         \setlength{\tabcolsep}{3.0pt}
\begin{table*}[t]
    \begin{center}
    \begin{tabular}{c|c|c|c|cg|cg|cgcgcg}
    
    \hline
    
    \scriptsize Adjective & \scriptsize Noun & \scriptsize Grounding & \scriptsize LE/PL  & \multicolumn{2}{c|}{\scriptsize Base-Only} & \multicolumn{2}{c|}{\scriptsize Target-Only} & \multicolumn{6}{c}{\scriptsize Generalized} \\
       \scriptsize Negative & \scriptsize Negative & \scriptsize Type & \scriptsize Trained  & \scriptsize AP$_{50}$ &  \multicolumn{1}{c|}{\scriptsize $\Delta$}  & \scriptsize \scriptsize AP$_{50}$ & \multicolumn{1}{c|}{\scriptsize $\Delta$}  & \scriptsize All AP$_{50}$ & \multicolumn{1}{c}{\scriptsize $\Delta$}  & \scriptsize Base AP$_{50}$ & \multicolumn{1}{c}{\scriptsize $\Delta$} & \scriptsize Target \scriptsize AP$_{50}$ & \multicolumn{1}{c}{\scriptsize $\Delta$}  \\
    \hline 
    \rowcolor{LLGray}
     - & - & \scriptsize Context-Free & - &  \small 32.8  \scriptsize  $\pm$ 0.08 & \scriptsize $-$ & \small 15.8  \scriptsize  $\pm$ 0.11 & \scriptsize $-$ & \small 26.3  \scriptsize  $\pm$ 0.04 & \scriptsize $-$  & \small 31.4  \scriptsize  $\pm$ 0.15 & \scriptsize $-$ & \small 11.8  \scriptsize $\pm$ 0.28 & \scriptsize $-$   \\
     \hline
    \hline
        \scriptsize Plausible &  \scriptsize \checkmark & \scriptsize Contextualized & \scriptsize \checkmark  &  \small \textbf{35.8} \scriptsize $\pm$ 0.09 & \textbf{\color{Greenish}{\small +3.0}} & \small 17.7  \scriptsize $\pm$ 0.38 & \color{Greenish}{\small +1.9} & \small \textbf{28.8}  \scriptsize  $\pm$ 0.17 & \textbf{\color{Greenish}{\small +2.5}} & \small \textbf{33.9}  \scriptsize $\pm$ 0.24 & \textbf{\color{Greenish}{\small +2.5}} & \small 14.2  \scriptsize  $\pm$ 0.34 & \color{Greenish}{\small +2.4}  \\
    \hline
        \scriptsize Random &  \scriptsize \checkmark & \scriptsize Contextualized & \scriptsize \checkmark  &    \small 35.7 \scriptsize $\pm$ 0.31 & \color{Greenish}{\small +2.9} & \small 18.0  \scriptsize $\pm$ 0.25 & \color{Greenish}{\small +2.2} & \small 28.6 \scriptsize $\pm$ 0.33 & \color{Greenish}{\small +2.3} & \small 33.5 \scriptsize $\pm$ 0.30 & \color{Greenish}{\small +2.1} & \small \textbf{14.5} \scriptsize $\pm$ 0.41& \color{Greenish}{\small \textbf{+2.7}}  \\
    \hline
            - &\scriptsize \checkmark & \scriptsize Contextualized & \scriptsize \checkmark  & \small 35.3  \scriptsize $\pm$ 0.19 & \color{Greenish}{\small +2.5} & \small 17.8  \scriptsize  $\pm$ 0.18 & \color{Greenish}{\small +2.0} & \small 28.3  \scriptsize  $\pm$ 0.12 & \color{Greenish}{\small +2.0} & \small 33.3  \scriptsize $\pm$ 0.13 & \color{Greenish}{\small +1.9} & \small 14.2  \scriptsize $\pm$ 0.13 & \color{Greenish}{\small +2.4}  \\
    \hline
            - & - &\scriptsize Contextualized &\scriptsize \checkmark & \small 35.2   \scriptsize $\pm$ 0.13  & \color{Greenish}{\small +2.4} & \small 16.7   \scriptsize $\pm$ 0.26 & \color{Greenish}{\small +0.9} & \small 28.3  \scriptsize $\pm$ 0.20 & \color{Greenish}{\small +2.0}  & \small 33.6  \scriptsize  $\pm$ 0.16 & \color{Greenish}{\small +2.2} & \small 13.1  \scriptsize  $\pm$ 0.30 & \color{Greenish}{\small +1.3} \\
     \hline
        - & - & \scriptsize Contextualized & -  & \small 31.8  \scriptsize $\pm$ 0.14  & \color{red}{\small -1.0} & \small 10.5 \scriptsize  $\pm$ 0.28 & \color{red}{\small -5.3} & \small 22.7 \scriptsize  $\pm$ 0.83 & \color{red}{\small -3.6}  & \small 28.0  \scriptsize  $\pm$ 0.98 & \color{red}{\small -3.4} & \small 7.5 \scriptsize  $\pm$ 0.47 & \color{red}{\small -4.3}\\
    \hline
    \hline
        \scriptsize Plausible & \scriptsize \checkmark & \scriptsize Context-Free &\scriptsize \checkmark  & \small 34.1  \scriptsize $\pm$ 0.21  & \color{Greenish}{\small +1.3} & \small \textbf{19.3} \scriptsize  $\pm$ 0.29 & \textbf{\color{Greenish}{\small +3.5}} & \small 28.4  \scriptsize  $\pm$ 0.17 & \color{Greenish}{\small +2.1}  & \small 33.4  \scriptsize  $\pm$ 0.19 & \color{Greenish}{\small +2.0} & \small 14.3 \scriptsize  $\pm$ 0.38 & \color{Greenish}{\small +2.5}\\
    \hline
          - & - & \scriptsize Context-Free &\scriptsize \checkmark  & \small 34.1  \scriptsize $\pm$ 0.01  & \color{Greenish}{\small +1.3} & \small 19.1 \scriptsize  $\pm$ 0.72 & \color{Greenish}{\small +3.3} & \small 28.3  \scriptsize  $\pm$ 0.27 & \color{Greenish}{\small +2.0}  & \small 33.2  \scriptsize  $\pm$ 0.12 & \color{Greenish}{\small +1.8} & \small 14.4 \scriptsize  $\pm$ 0.70 & \color{Greenish}{\small +2.6}\\
    \hline
    \end{tabular}
    \end{center}
    \vspace{-5mm}
    \caption{\textbf{Adapting OVR-CNN \cite{zareian2021open} with attribute context enhancement strategies (Sec. \ref{rwgroundcase}/\ref{enhance}): adjective/noun negative caption sampling, contextualized grounding, language encoder/projection layer training (LE/PL)}, AP$_{50}$ mean over 3 trials $\pm$ std error, $\Delta$=change vs. default OVR-CNN \cite{zareian2021open} (top row). Using adjective negatives \textit{with} contextualization yields base/generalized AP$_{50}$ increases, and top base/generalized AP$_{50}$ overall, as the model is able to take into account attribute meaning in object embeddings.}

    % AP$_{50}$ is shown for open-vocabulary detection on COCO in base-only, target-only, and generalized (w.r.t. base, target, and all classes) settings

    \label{negative_strategies}
\end{table*}


        \noindent \textbf{Attribute context has limited impact in region-word pretraining for object detection.} We first examine the role of attributes through \textit{removing all ``amod" from captions} during VL pretraining with OVR-CNN. Open-vocabulary detection results for baseline OVR-CNN \cite{zareian2021open} are shown in Fig. \ref{remove_fig}. Note that the max. drop from training with to without adjectives is -0.36 AP$_{50}$ (base), and there are not discernible drops in target/generalized settings. \textit{These results point to attribute context being wasted and not helpful when learning object grounding}, and thus serve as inspiration for our investigation of ways to boost use of attribute context. 
        


        \noindent \textbf{Contextualizing object grounding does not result in embeddings with high sensitivity to attribute meaning.} As outlined in Sec. \ref{rwgroundcase}, we contextualize grounding in OVR-CNN as one strategy to integrate attribute context. Then through unsupervised phrase grounding, we gauge sensitivity to attribute meaning and analyze whether the attributes contextualizing an object noun (\eg ``a \textit{red} \underline{car}") impact performance. Fig. \ref{attribute_pg_fig2} shows AP@IoU=30:10:50 for (1) OVR-CNN with contextualization and (2) OVR-CNN with contextualization \textit{and} plausible adjective/noun negatives, on the four region-word grounding scenarios of interest (baseline grounding, removing adjectives, changing adjectives plausibly, and changing adjectives randomly). On the left of Fig. \ref{attribute_pg_fig2}, we find that with default contextualization, changing adjectives plausibly/randomly yields similar AP to using baseline captions or captions with removed adjectives (a max. difference of 0.13 AP@IoU=30:10:50). \textit{These observations are counterintuitive}, as embeddings can be contextualized by incorrect adjectives, yet ground similarly to when there are correct adjectives. We posit that the model may be sensitive to caption structure, where object embeddings with different adjectives are close together, and the model does not have an incentive to differentiate them. Such lack of sensitivity to attribute meaning motivates our exploration of \textit{adjective negatives}; we show the effects on the right in Fig. \ref{attribute_pg_fig2}. Contextualization aptly becomes less aligned with incorrect adjectives, 
        reaching notable drops when changing plausible/randomly with respect to the baseline
        (-1.36\%/-1.57\% respectively). In Sec. \ref{evaluate}, we show the importance of sensitivity in detection and retrieval.

        
        \begin{figure}[t]
    \centering
    \includegraphics[scale=0.235]{images/revamped_classvia.png}
    \vspace{-2mm}
    \caption{\textbf{Perturbing attributes and object names in CLIP descriptions used for ImageNetV2 classification.} Removing and changing adjectives have small effects on accuracy. When classes are described without object names, accuracy significantly drops.
     }
    \label{classviadesc_fig}
\end{figure}



        \noindent \textbf{Describing classes in terms of attributes alone is ineffective.} We measure CLIP's sensitivity to attributes with classification via description. As outlined in Sec. \ref{sens_method}, zero-shot inference is performed on ImageNetV2 using CLIP default prompting, LLM-based sets of object feature descriptions \cite{menon2022visual}, and LLM-based single-sentence descriptions of objects \cite{pratt2022does}. In Fig.~\ref{classviadesc_fig}, we show the results of removing/changing adjectives and removing class names in terms of top1/5 accuracy. Removing/changing adjectives results in \emph{insignificant drops} vs. the baseline with \cite{menon2022visual}, and slightly bigger drops with the \cite{pratt2022does}-like method (-4.1\% drop baseline to changing),  
        potentially as a result of more adjective-dense descriptions (supported in the supp.).
        However, \textit{removing class names} results in close to \textit{ten times more substantial} drops (max -40.5\% top1 accuracy). These results bring into question the model's ability to leverage attribute descriptions since \emph{class names drive performance}. Such results also limit the appeal of using attribute descriptions for new/custom objects with names not in the pretraining set.

            
    \subsection{Evaluating context enhancement strategies}
        \label{evaluate}
           
        \noindent \textbf{Enhancing context sensitivity helps open-vocabulary detection in base and generalized settings.} We evaluate OVR-CNN with four strategies to boost attribute context in region-word pretraining: (1) contextualized grounding, (2) adjective negative sampling (plausible/random), (3) noun negative sampling (random), and (4) language encoder/projection layer training, with results shown at an experimental scale in Table \ref{negative_strategies}. Compared to the baseline \cite{zareian2021open}, combining all strategies, in both plausible and random adjective negative cases, provides the largest gains in base-only and generalized (all-class) settings (\eg +3.0 and +2.5 AP$_{50}$ respectively with plausible). In Table \ref{soa}, we also present a proof-of-concept showing that enhancing attribute context improves the results reported in \cite{zareian2021open} in 4/5 settings (+0.9-1.0 AP$_{50}$ in base-only and all generalized settings). \textit{Such results highlight value in better using context, especially attributes, when learning grounding for detection}.
         \setlength{\tabcolsep}{3pt}
\begin{table}[t]
    \begin{center}
    \begin{tabular}{c|c|c|c|c|c}
    \hline
    
    \small Method & \small Base & \small Target & \multicolumn{3}{c}{\small Generalized}     \\
    & &     & \multicolumn{1}{c}{\small Base} & \multicolumn{1}{c}{\small Target} & \multicolumn{1}{c}{\small All}  \\
    \hline
    \hline
   %     \small \textbf{Cap2Det} \cite{ye2019cap2det} & \small WSD & \small - & \small - & \small 20.1 & \small \small 20.3 & \small 20.1 \\

    %    \small \textbf{MIL+RPN} \cite{uijlings2018revisiting} & \small MSD & \small - & \small - & \small 27.8 & \small \small 22.6 & \small 26.4 \\
    
     %   \small \textbf{PL} \cite{rahman2020improved} & \small ZSD & \small 36.8 & \small 10.0 & \small 35.9 & \small \small 4.12 & \small 27.9 \\
    \rowcolor{LLGray}
        \small \textbf{OVR-CNN} \cite{zareian2021open} & \small 46.8 & \small \textbf{27.5} & \small 46.0 & \small \small 22.8 & \small 39.9  \\
    \hline
        \small + Context Enhancement & \textbf{\small 47.7} & \small 26.5 & \textbf{\small 46.9} & \textbf{\small 23.8} & \textbf{\small 40.8}  \\
    \hline
    \end{tabular}
    \end{center}
    \vspace{-5mm}
    \caption{\textbf{OVR-CNN at full scale with various context enhancement strategies} (plausible adjective/noun negatives, contextualized grounding, language encoder/projection layer training), compared to baseline reported in \cite{zareian2021open}.  AP$_{50}$ reported on COCO.}
    \label{soa}
\end{table}

        Breaking down Table \ref{negative_strategies}, a key observation is that plausible/random adjective negatives, when used with contextualized grounding, result in (comparable) base and generalized gains over all other baselines (+0.5 and +0.4 AP$_{50}$ with plausible). These results can be ascribed to increased attention to attribute meaning that is obtainable with contextualized grounding, but not with context-free grounding since embeddings do not vary with context. \textit{There is marked benefit to learning to ground objects with attribute signals for detection.} Still, there is a tradeoff between contextualized and context-free grounding. Contextualized models result in top AP$_{50}$ in base-only and all generalized settings, but context-free results in top AP$_{50}$ in target-only. These results can be attributed to using contextualized embeddings and \textit{not} adjective negatives, since all contextualized methods obtain worse target performance than the best context-free method. We reason that the drop is due to the need for a prompt: we use a simple ``A/an $<$objName$>$." (see Sec. \ref{rwgroundcase}), but this prompt may be suboptimal to represent the large variance of contextualized embeddings for an object. Training with box annotations in base may allow visual embeddings to adjust to prompts, explaining base gains, but with no target training, adjustment cannot occur. 
        We surmise that recent work in context optimization \cite{du2022learning} can overcome this challenge. The noun negatives notably improve target-only vs. contextualized (+1.1 AP$_{50}$), showing that differentiating nouns in the same context may also help.
        
        We further inspect the plausible case by comparing class-by-class results using models in row 2/4 of Table \ref{negative_strategies}. Notably, the classes with top AP$_{50}$ gains are \textit{oven} (+4.6), \textit{bear} (+4.3), \textit{horse} (+3.6), and \textit{frisbee} (+3.4). Upon inspection of the corpus, these are commonly described in captions with visually distinctive adjectives  that may help grounding such as colors (\eg ``yellow frisbee"). Overall, we observe that 32/48 classes improve in AP$_{50}$ with adjective negatives.


        \noindent \textbf{Adjective negatives increase CLIP's fine-grained utility in multiple tasks.} We use text-region retrieval and attribution as fine-grained tasks to evaluate attribute-object understanding. Table \ref{clipretrieval} shows these results comparing strategies for finetuning CLIP on COCO: (1) choosing a random negative caption, (2) order-perturbing adjectives/nouns \cite{yuksekgonul2022and}, (3) random adjective sampling, and (4)  plausible adjective sampling. On retrieval,
        random adjective sampling is generally most effective across values of $k$, plausible is second, and both strategies outperform a random caption baseline and the order-perturbing captions of \cite{yuksekgonul2022and}. \textit{The fine-grained differentiation needed for retrieval is aided best by adjective negatives}. On the attribution task, the order-perturbing negatives perform best, which makes sense given that attribution involves determining the correct order of adjectives and nouns. It is notable that adjective negatives improve on this task \textit{and} retrieval vs. a random caption baseline, \emph{unlike the order-perturbing captions}. This shows adjective negatives achieve more generalizable attribute-object understanding across tasks. Adjective negatives similarly improve in retrieval for OVR-CNN (Table \ref{retrieval}). Plausible and random adjective sampling are more competitive in this scenario, though random sampling has highest R@1/P@1 and plausible sampling P/R@5/10. We surmise that random adjective sampling may solidify easier retrievals 
        by comparing to a wide array of adjectives, while plausible sampling may help the model differentiate between tougher cases as plausible adjectives serve as more realistic, \textit{harder} negatives.
        
        %In general, the results show that training with adjective negatives is simple and effective to improve fine-grained utility.


         \setlength{\tabcolsep}{1.4pt}
\begin{table}
    \begin{center}
    \begin{tabular}{c|ccc|ccc||c}
    \hline 
    \small Method & \small R@1 & \small R@5 & \small R@10  & \small P@1 & \small P@5 & \small P@10 & \small VGA
    \\
    \hline
    \hline
        \scriptsize Default CLIP & \small 48.92 & \small 82.97 & \small 90.40 & \small 48.92 & \small 42.66 & \small 37.62 & \small 62.82  \\
    \hline
    \hline
        \scriptsize Random Neg. & \small 57.59 & \small \underline{87.62} & \small \textbf{94.12} & \small 57.59 & \small \underline{50.96} & \small 44.37 & \small 64.64 \\
    \hline\hline
        \scriptsize Order-Based Neg. \cite{yuksekgonul2022and}  & \small 56.97 & \small 85.76 & \small 92.88 & \small 56.97 & \small 48.73 & \small 42.79 & \color{Greenish}{\small \textbf{73.87}} \\
    \hline
    \hline
        \scriptsize Plausible Adj. Neg.  & \color{Greenish}{\small \underline{58.82}} & \small 86.69 & \small \underline{93.81} & \small \color{Greenish}{\underline{58.82}} & \small \underline{50.96} & \small \color{Greenish}{\underline{44.77}} & \color{Greenish}{\small \underline{67.94}} \\
    \hline
        \scriptsize Random Adj. Neg.  & \small \color{Greenish}{\textbf{60.06}} & \color{Greenish}{\small \textbf{88.24}} & \small 92.26 & \small \color{Greenish}{\textbf{60.06}} & \color{Greenish}{\small \textbf{51.76}} & \color{Greenish}{\small \textbf{44.98}} & \color{Greenish}{\small 67.93} \\
    \hline

    \end{tabular}
    \end{center}
    \vspace{-5mm}
    \caption{\textbf{Fine-grained utility of CLIP finetuned with negative sampling strategies, on T2R retrieval and Visual Genome Attribution (VGA) \cite{yuksekgonul2022and}.} Recall/precision@$k$=1,5,10 are reported for T2R retrieval and accuracy for VGA. Best=\textbf{bold}, second=\underline{underlined}, results $>$ random baseline (row 2) in \color{Greenish}{green}\color{black}. Note that adjective sampling offers improvements across \textit{both} attribute tasks, while order only helps on the order-based VGA task.}
    \label{clipretrieval}
\end{table}

        \begin{table*}%[htpb]
    \centering
    \small
    \setlength{\tabcolsep}{4pt}
    \resizebox{\textwidth}{!}{
    \begin{tabu}{lr|ccccccccc|ccccccccc}
        \toprule
        \multirow{2}{*}{\bf Method} & \multirow{2}{*}{\bf \#Pairs} & \multicolumn{9}{c|}{\bf FT Retrieval \ \ R@1 / R@5 / R@10} & \multicolumn{9}{c}{\bf ZS Retrieval \ \ R@1 / R@5 / R@10} \\
        & & \multicolumn{3}{c}{MSRVTT} & \multicolumn{3}{c}{DiDeMo} & \multicolumn{3}{c|}{ActivityNet} & \multicolumn{3}{c}{MSRVTT} & \multicolumn{3}{c}{DiDeMo} & \multicolumn{3}{c}{ActivityNet}  \\
        \midrule
        ClipBERT~\cite{lei2021less}  & 5.4M & 22.0 & 46.8 & 59.9 & 20.4 & 48.0 & 60.8 & 21.3 & 49.0 & 63.5 & -& -& -& -& -& -& -& -& -\\
        VideoCLIP~\cite{xu2021videoclip}  & 136M & 30.9 & 55.4 & 66.8 & -& -& -& -& -& -& 10.4 & 22.2 & 30.0 & 16.6 & 46.9 & -& -& -& -\\
        Frozen~\cite{bain2021frozen}  & 5M & 31.0 & 59.5 & 70.5 & 34.6 & 65.0 & 74.7  & -& -& -& 18.7 & 39.5 & 51.6 & 20.2 & 46.4 & 58.5 & -& -& -\\
        ALPRO~\cite{li2022align}  & 5M & 33.9 & 60.7 & 73.2 & 35.9 & 67.5 & 78.8 & -& -& -& 24.1 & 44.7 & 55.4 & 23.8 & 47.3 & 57.9 & -& -& -\\
        VIOLET~\cite{fu2021violet}  & 138M & 34.5 & 63.0 & 73.4 & 32.6 & 62.8 & 74.7 & -& -& -& 25.9 & 49.5 & 59.7 & 23.5 & 49.8 & 59.8 & -& -& - \\
        All-in-one~\cite{wang2022all} & 138M & 37.9 & 68.1 & 77.1 & 32.7 & 61.4 & 73.5 & 22.4 & 53.7 & 67.7 & -& -& -& -& -& -& -& -& -\\
        LAVENDER~\cite{li2022lavender} & 30M & 40.7 & 66.9 & 77.6 & 53.4 & 78.6 & 85.3 &  - & -& -& -& -& -& -& -& -& -& -& -\\
        Singularity~\cite{lei2022revealing} & 17M & 42.7 & 69.5 & 78.1 & 53.1 & 79.9 & 88.1 & 48.9 & 77.0 & 86.3 & 34.0 & 56.7 & 66.7 & 37.1 & 61.7 & 69.9 & 30.6 & 55.6 & 66.9 \\
        OmniVL~\cite{wang2022omnivl} & 17M & 47.8 & 74.2 & 83.8 & 52.4 & 79.5 & 85.4 & -& -& -& 34.6 & 58.4 & 66.6 & 33.3 & 58.7 & 68.5 & -& -& -\\ 
        VINDLU~\cite{Cheng2022VindLUAR} & 25M & 46.5 & 71.5 & 80.4 & 61.2 & 85.8 & 91.0 & 55.0 & 81.4 & 89.7 & 32.0 & 54.6 & 62.0 & 36.9 & 61.7 & 70.5 & 30.9 & 57.0 & 68.2 \\
        \rowfont{\color{Gray}}
        CLIP4Clip~\cite{luo2022clip4clip} & 400M & 44.5 & 71.4 & 81.6 & 42.8 & 68.5 & 79.2 & 40.5 & 72.4 & 83.4 & 31.2 & 53.7 & 64.2 & -& -& -& -& -& -\\
        % \rowfont{\color{Gray}}
        % CLIP-Hhiker~\cite{bain2022clip} & 400M & 47.7 & 74.1 & 82.9 & -& -& -& 44.0 & 74.9 & 86.1 & -& -& -& -& -& -& -& -& -\\
        \rowfont{\color{Gray}}
        CLIP-ViP~\cite{xue2022clip} & 500M & 54.2 & 77.2 & 84.8 & 50.5 & 78.4 & 87.1 & 53.4 & 81.4 & 90.0 & -& -& -& -& -& -& -& -& -\\
        \rowfont{\color{Gray}}
        InternVideo~\cite{Wang2022InternVideoGV} & 646M & 55.2 & 79.6 & 87.5 & 57.9 & 82.4 & 88.9 & 62.2  & 85.9 & 93.2 & 40.7 & 65.3 & 74.1 & 31.5 & 57.6 & 68.2 & 30.7 & 57.4 & 70.2 \\
        \midrule
        \multirow{3}{*}{\Modelname-Base} & 5M & 46.3 & 72.7 & 82.0 & 54.8 & 83.0 & 89.0 & 52.1 & 80.5 & 89.6 & 29.6 & 52.8 & 61.9 & 33.4 & 58.3 & 67.0 & 28.3 & 53.0 & 64.2 \\
        & 17M & 50.6 & 75.4 & 83.5 & 60.8 & 85.1 & 91.0 & 56.1 & 82.5 & 91.2 & 35.5 & 59.3 & 68.6 & 41.9 & 66.7 & 75.0 & 33.8 & 59.1 & 70.4 \\
        & 25M & 51.0 & 76.5 & 84.2 & 61.6 & 86.8 & 91.5 & 58.3 & 83.9 & 91.5 & 35.2 & 57.8 & 66.0 & 41.2 & 65.4 & 74.9 & 35.5 & 60.6 & 71.8 \\
        \hline
        \multirow{3}{*}{\Modelname-Large} & 5M & 53.3 & 76.6 & 83.9 & 59.7 & 84.9 & 90.8 & 58.1 & 85.5 & 92.9 & 33.3 & 58.1 & 66.7 & 34.0 & 60.4 & 68.7 & 31.9 & 60.2 & 72.0 \\
        & 17M & \underline{56.5} & \underline{80.1} & \underline{87.4} & \underline{66.6} & \underline{89.9} & \textbf{93.7} & \underline{66.6} & \underline{88.6} & \underline{94.7} & \textbf{42.6} & \textbf{64.4} & \textbf{73.1} & \underline{46.4} & \underline{70.0} & \underline{78.8} & \textbf{42.8} & \textbf{69.6} & \textbf{79.8} \\
        & 25M & \textbf{58.8} & \textbf{81.0} & \textbf{87.1} & \textbf{70.4} & \textbf{90.1} & \underline{93.5} & \textbf{66.8} & \textbf{89.1} & \textbf{94.9} & \underline{40.7} & \underline{63.4} & \underline{71.8} & \textbf{48.6} & \textbf{72.9} & \textbf{80.0} & \underline{41.9} & \underline{68.9} & \underline{80.3} \\
        \bottomrule
    \end{tabu}
    }
    \vspace{-0.3cm}
    \caption{Comparison to the state-of-the-art text-to-video retrieval methods on MSRVTT, DiDeMo and AcitivityNet.
    \#Pairs denotes the number of pre-training pairs.
    ``FT'' and ``ZS'' refer to the fine-tuning and zero-shot results.
    }
    \label{tab:retrieval}
\end{table*}


    

    %\noindent \textbf{Further analysis and discussion}

    % Possible
    % Attribute type breakdown - PG
    % Distance in language embedding space
    % Visualizations
    % Maybe something with usneen attr
    % Maybe test on UnRel (domain shift) 
    % PT on other datasets Concaps
    % Test on other datsets (LVIS, Flickr, VAW) 
 
