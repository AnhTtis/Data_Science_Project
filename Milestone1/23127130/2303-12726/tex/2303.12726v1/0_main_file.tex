% ---------------------------------------------------------------------
% EG author guidelines plus sample file for EG publication using LaTeX2e input
% D.Fellner, v2.03, Dec 14, 2018


\title[Greedy Shape Curriculum]%
      {Learning to Transfer In-Hand Manipulations Using a Greedy Shape Curriculum}

% for anonymous conference submission please enter your SUBMISSION ID
% instead of the author's name (and leave the affiliation blank) !!
% for final version: please provide your *own* ORCID in the brackets following \orcid; see https://orcid.org/ for more details.
\author[Yunbo Zhang \& Alexander Clegg  \& Sehoon Ha  \& Greg Turk \& Yuting Ye]
{\parbox{\textwidth}{\centering Yunbo Zhang$^{1}$, Alexander Clegg$^{3}$, Sehoon Ha$^{1}$, Greg Turk$^{1}$,Yuting Ye$^{2}$
%        S. Spencer$^2$\thanks{Chairman Siggraph Publications Board}
    } 
    \\
{\parbox{\textwidth}{\centering $^{1}$Georgia Institute of Technology , $^{2}$Meta Reality Lab, $^{3}$Meta AI Research}
%        S. Spencer$^2$\thanks{Chairman Siggraph Publications Board}
}
}

% For Computer Graphics Forum: Please use the abbreviation of your first name.
% {\parbox{\textwidth}{\centering $^1$TU Darmstadt \& Fraunhofer IGD, Germany\\
%          $^2$Graz University of Technology, Institute of Computer Graphics and Knowledge Visualization, Austria
% %        $^2$ Another Department to illustrate the use in papers from authors
% %             with different affiliations
%       }
% }

% ------------------------------------------------------------------------

% if the Editors-in-Chief have given you the data, you may uncomment
% the following five lines and insert it here
%
% \volume{36}   % the volume in which the issue will be published;
% \issue{1}     % the issue number of the publication
% \pStartPage{1}      % set starting page

%-------------------------------------------------------------------------
\begin{document}

% uncomment for using teaser
% \teaser{
%  \includegraphics[width=\linewidth]{eg_new}
%  \centering
%   \caption{New EG Logo}
% \label{fig:teaser}
%}

\maketitle
%-------------------------------------------------------------------------
\begin{abstract}
In-hand object manipulation is challenging to simulate due to complex contact dynamics, non-repetitive finger gaits, and the need to indirectly control unactuated objects. Further adapting a successful manipulation skill to new objects with different shapes and physical properties is a similarly challenging problem. In this work, we show that natural and robust in-hand manipulation of simple objects in a dynamic simulation can be learned from a high quality motion capture example via deep reinforcement learning with careful designs of the imitation learning problem. We apply our approach on both single-handed and two-handed dexterous manipulations of diverse object shapes and motions. We then demonstrate further adaptation of the example motion to a more complex shape through curriculum learning on intermediate shapes morphed between the source and target object. While a naive curriculum of progressive morphs often falls short, we propose a simple greedy curriculum search algorithm that can successfully apply to a range of objects such as a teapot, bunny, bottle, train, and elephant.
%-------------------------------------------------------------------------
%  ACM CCS 1998
%  (see https://www.acm.org/publications/computing-classification-system/1998)
% \begin{classification} % according to https://www.acm.org/publications/computing-classification-system/1998
% \CCScat{Computer Graphics}{I.3.3}{Picture/Image Generation}{Line and curve generation}
% \end{classification}
%-------------------------------------------------------------------------
%  ACM CCS 2012
%   (see https://www.acm.org/publications/class-2012)
%The tool at \url{http://dl.acm.org/ccs.cfm} can be used to generate
% CCS codes.
%Example:
% \begin{CCSXML}
% <ccs2012>
% <concept>
% <concept_id>10010147.10010371.10010352.10010381</concept_id>
% <concept_desc>Computing methodologies~Collision detection</concept_desc>
% <concept_significance>300</concept_significance>
% </concept>
% <concept>
% <concept_id>10010583.10010588.10010559</concept_id>
% <concept_desc>Hardware~Sensors and actuators</concept_desc>
% <concept_significance>300</concept_significance>
% </concept>
% <concept>
% <concept_id>10010583.10010584.10010587</concept_id>
% <concept_desc>Hardware~PCB design and layout</concept_desc>
% <concept_significance>100</concept_significance>
% </concept>
% </ccs2012>
% \end{CCSXML}
\begin{CCSXML}
<ccs2012>
   <concept>
       <concept_id>10010147.10010371.10010352.10010379</concept_id>
       <concept_desc>Computing methodologies~Physical simulation</concept_desc>
       <concept_significance>500</concept_significance>
       </concept>
   <concept>
       <concept_id>10010147.10010371.10010352.10010238</concept_id>
       <concept_desc>Computing methodologies~Motion capture</concept_desc>
       <concept_significance>500</concept_significance>
       </concept>
   <concept>
       <concept_id>10010147.10010257.10010258.10010261</concept_id>
       <concept_desc>Computing methodologies~Reinforcement learning</concept_desc>
       <concept_significance>300</concept_significance>
       </concept>
   <concept>
       <concept_id>10010147.10010257.10010282.10010290</concept_id>
       <concept_desc>Computing methodologies~Learning from demonstrations</concept_desc>
       <concept_significance>300</concept_significance>
       </concept>
 </ccs2012>
\end{CCSXML}

\ccsdesc[500]{Computing methodologies~Physical simulation}
\ccsdesc[500]{Computing methodologies~Motion capture}
\ccsdesc[300]{Computing methodologies~Reinforcement learning}
\ccsdesc[300]{Computing methodologies~Learning from demonstrations}

% \ccsdesc[300]{Computing methodologies~Collision detection}
% \ccsdesc[300]{Hardware~Sensors and actuators}
% \ccsdesc[100]{Hardware~PCB design and layout}

\printccsdesc   
\end{abstract}  
%-------------------------------------------------------------------------
%%%%% GENERAL MATH COMMANDS
% Reals
\newcommand{\R}{{\mathbb R}}
% Integers
\newcommand{\Z}{{\mathbb Z}}
% Naturals
\newcommand{\N}{{\mathbb N}}
% Expectation
\DeclareMathOperator*{\E}{\mathbb{E}}
% ^th notation
\newcommand{\tth}{^{\text{th}}}
% Small dots for integer range [a .. b]
\newcommand{\sdots}{\,..\,}
% Vectorized version of matrix
\newcommand{\matvec}{\mbox{vec}}

% := sign
\newcommand{\defeq}{\vcentcolon=}
% Zero function
\newcommand{\zf}{\mathbf{0}}
% Vector of ones
\newcommand{\ones}{\mathbf{1}}

% Argmin and argmax definitions
\DeclareMathOperator*{\argmax}{arg\,max}
\DeclareMathOperator*{\argmin}{arg\,min}


%%%%% PROBLEM STATEMENT NOTATION 
% \newcommandtwoopt{\St}[2][t][]{{S_{#1}^{#2}}} % State
\newcommand{\task}[1][i]{{\mathcal{T}_{#1}}} % Task, optionally takes index
\newcommand{\tasks}{\{ \task \}_{i=1}^N}
\newcommand{\losst}[1][i]{{l_{#1}}}
\newcommand{\lossv}[1][i]{{l_{#1}^{\textrm{val}}}}
\newcommand{\tasktarget}{{\mathcal{T}_{\textrm{target}}}}
\newcommand{\lossttarget}{l_{\textrm{target}}}
\newcommand{\lossvtarget}{l_{\textrm{target}}^{\textrm{val}}}
\newcommand{\lossttargetit}{l_{\textrm{target}}^{(k)}}
\newcommand{\losstotal}{l^{\textrm{total}}}
\newcommand{\lossopt}{l^*}

\newcommand{\thetait}[2]{\theta_{#1}^{(#2)}}
\newcommand{\phit}[1]{\phi^{(#1)}}
\newcommand{\hist}[2]{S_{#1}^{(#2)}}
\newcommand{\grad}[2]{G_{#1}^{(#2)}}

\newcommand{\Alg}{\textup{\textbf{Opt}}}
\newcommand{\MetaAlg}{\textup{\textbf{MetaOpt}}}

%%%%% Theorems
\newtheoremstyle{mytheoremstyle} % name
    {\topsep}                    % Space above
    {\topsep}                    % Space below
    {\itshape}                   % Body font
    {}                           % Indent amount
    {\scshape}                   % Theorem head font
    {.}                          % Punctuation after theorem head
    {.5em}                       % Space after theorem head
    {}  % Theorem head spec (can be left empty, meaning ‘normal’)
\theoremstyle{mytheoremstyle}
\theoremstyle{plain}
\newtheorem{theorem}{Theorem}
\newtheorem{proposition}{Proposition}
\newtheorem{assumption}{Assumption}
\newtheorem{definition}{Definition}
\newtheorem{lemma}{Lemma}
\theoremstyle{remark}
\newtheorem{remark}{Remark}


% \begin{figure}[t]
%     % \begin{subfigure}{1\linewidth}
%     %   \centering
%     % %   \includegraphics[width=1\linewidth]{figs/fig_1_moti_textattn.pdf}  
%     % %   \includegraphics[width=1\linewidth]{figs/fig_1_moti_textattn_v2.pdf}  
%     %   \includegraphics[width=1\linewidth]{figs/fig_1_moti_textattn_v5.pdf}  
%     %   \vspace{-0.5cm}
%     %     \caption{Amount of attention added to each video clip from the source video and query text in the self-attention layers of Moment-DETR encoder.}
%     %     % \caption{Distribution of attention for source and query in Moment-DETR encoder}
%     %     % Visualization of video clip's self-attention score in Moment-DETR encoder.
%     %   \label{fig:fig1_text_attn_ex}
%     % \end{subfigure}%\hfill% or  or \hspace{0.3\textwidth}
%     \vspace{0.2cm}
%     % \begin{subfigure}{1\linewidth}
%       \centering
%     %   \includegraphics[width=1\linewidth]{figs/fig1_moti_negattn.pdf}  
%       \includegraphics[width=1\linewidth]{figs/fig1_moti_negattn_v3.pdf}  
%       \vspace{-0.4cm}
%     %   \caption{Correspondence of saliency scores on the relevance between video clips and the text query.}
%     % \caption{Predicted saliency scores against the video relevant positive query and video irrelevant negative query}
%       \label{fig:fig1_neg_attn_ex}
%     % \end{subfigure}%\hfill% or  or \hspace{0.3\textwidth}
%     \caption{
%     % 원준 원본
%     % (a) Comparison between attention scores of source and query for each video clip~(We sum the attention scores from video and text). 
%     % We observe that the attention scores are dominated by other clips in the source video. 
%     % Text queries do not account for much attention regardless of the relevance to the video clips.
%     % \textbf{(a)} Inspection of the query dependency in Moment-DETR encoder.
%     % % We visualize the attention score of video tokens in the transformer encoder and observe that text query accounts for only a low portion of attention.
%     % % This tendency occurs regardless of the relevance between the text query and video clips. 
%     % We visualize the attention score of video tokens in the transformer encoder and observe 1) text query only accounts for a low portion of attention, and 2) relevance between video-query pair does not affect the attention scores ratio of text.
%     \textbf{(b)} Comparison of highlight-ness when relevant and non-relevant queries are input.
%     As observed in , existing work only uses queries to play an insignificant role, thereby may not be capable of detecting false queries and considering the video-query relevance even when the problem in (a) is resolved. 
%     % \SE{} % 이 부분이 "not capable of" 란 용어가 세다는 피드백이 있는 듯 합니다. 이러한 능력이 없다는 것은 굉장히 강한 어조인거 같기는 하고, 이러한 경우들이 종종 있다거나 좀 약화시킬 필요가 있어보이긴 하네요.
%     On the other hand, our QD-DETR yields a query-dependent representation that the relevance between the source video and query text is updated in the saliency scores.
%     There is a large gap between positive and negative saliency scores, and scores are consistent since the clips are all highly correlated to others.
%     }
%     \label{fig:motivation_ex}
%     % \captionsetup{belowskip=13pt}
%     % \setlength{\belowcaptionskip}{-10pt}
% \end{figure}
\begin{figure}
    \centering
    \includegraphics[width=1\linewidth]{figs/fig1_moti_negattn_1111.pdf}
    % \includegraphics[width=1\linewidth]{figs/fig1_moti_negattn_1109.pdf}
    % \includegraphics[width=1\linewidth]{figs/fig1_moti_negattn_stat.pdf}
    \vspace{-0.6cm}
    \caption{
        % \SE{} % 수정 필요
        Comparison of highlight-ness~(saliency score) when relevant and non-relevant queries are given.
        We found that the existing work only uses queries to play an insignificant role, thereby may not be capable of detecting negative queries and video-query relevance; saliency scores for clips in ground-truth~(GT) moments are low and equivalent for positive and negative queries.
        % This also results in mispredicted moments when ground-truth~(GT) moment is dominated by clips unrelated to GT since their prediction is highly focused on the video.
        % \SE{} % 여기 한번 더 보면 좋을 듯 합니다. GT moment에 unrelated한 clip이 많으면? label이 틀렷을 경우를 말씀하시는건지?
        % As observed in saliency graph, existing work only uses queries to play an insignificant role, thereby may not be capable of detecting false queries and considering the video-query relevance.
        On the other hand, query-dependent representations of QD-DETR result in corresponding saliency scores to the video-query relevance and precisely localized moments.
        % On the other hand, our QD-DETR yields a query-dependent representation that the
        % saliency scores are in accordance with the relevance between the video and query.
        % text is in accordance with the saliency scores.
        % There is a large gap between positive and negative saliency scores, and scores are consistent since the clips are all highly correlated to others.
}
    \label{fig:motivation_ex}
\end{figure}


\section{Introduction}
% 원준 원본
% Along with the advance of digital devices and platforms, video is now one of the most desired data type for consumers. However, although the large information capacity of videos may be beneficial in many aspects, e.g., informative and entertaining, on the contrary perspective, videos are time-consuming, and hard to search for desirable moments. 
% This has led many creators to use extra manpower to crop and edit the video to generate highlight clips to gain the consumer’s attention.
Along with the advance of digital devices and platforms, video is now one of the most desired data types for consumers~\cite{apostolidis2021video,wu2017deep}.
% SE: Video aware deep learning application & survey papers?
Although the large information capacity of videos might be beneficial in many aspects, e.g., informative and entertaining, inspecting the videos is time-consuming, so that it is hard to capture the desired moments~\cite{anne2017localizing,apostolidis2021video}. 
% This has led many creators to use extra manpower to crop and edit the video to generate highlight clips to gain the consumer’s attention.


% On the other side, 
Indeed, the need to retrieve user-requested or highlight moments within videos is greatly raised.
Numerous research efforts were put into the search for the requested moments in the video~\cite{anne2017localizing, gao2017tall, liu2015multi, escorcia2019temporal} and summarizing the video highlights~\cite{zhang2016video, mahasseni2017unsupervised, badamdorj2022contrastive, wei2022learning}.
% Numerous research efforts were put into the search for the requested moments in the video~\cite{anne2017localizing, gao2017tall, liu2015multi, escorcia2019temporal}, summarizing the video to generate highlights was another popular topic~\cite{zhang2016video, mahasseni2017unsupervised, badamdorj2022contrastive, wei2022learning}.
Recently, Moment-DETR~\cite{momentdetr} further spotlighted the topic by proposing a QVHighlights dataset that enables the model to perform both tasks, retrieving the moments with their highlight-ness, simultaneously.

% 원준 원본
% To detect the desired moments, previous works employed transformer encoder-decoder architectural designs to fuse the text query into the video representations. Moment-DETR~\cite{mDETR} modified detection transformer to process capture the moment as a set, and UMT~\cite{umt} implemented transformer decoder as to output clip-wise saliency. 
% Yet to their outstanding breakthroughs in the literature of moment retrieval with the seminal architectures, their limitation is that the role of the given text query is insignificant in representing the query-conditioned video representation; the attention mechanism of moment DETR is not explicitly conditioned on the text query, and the text query is conditioned on multi-modal clips where the differences between the clips are smoothed after encoding process in UMT.



% \begin{figure}[t]
% \centering
%     \begin{subfigure}[l]{0.37\linewidth}
%       \centering
%       \vspace{0.20cm}
%     %   \includegraphics[width=1\linewidth]{figs/fig_1_moti_textattn.pdf}  
%     %   \includegraphics[width=1\linewidth]{figs/fig_1_moti_textattn_v2.pdf}  
%       \includegraphics[width=1\linewidth]{figs/fig1_moti_violin_a.pdf}  
%       \vspace{-0.60cm}
%     %   \caption{text attention}
%         \caption{Importance of queries in video representation}
%       \label{fig:fig1_text_attn}
%     \end{subfigure}%\hfill% or  or \hspace{0.3\textwidth}
%     \vspace{0.2cm}
%     \begin{subfigure}[r]{0.61\linewidth}
%       \centering
%     %   \includegraphics[width=1\linewidth]{figs/fig1_moti_negattn.pdf}  
%       \includegraphics[width=1\linewidth]{figs/fig1_moti_violin_b.pdf}  
%     %   \caption{neg attention}
%         % \caption{Relation between the highlight-ness and the relevance between videos and query texts.}
%         \caption{Highlight-ness~(saliency) histogram of positive and negative video-query pairs\SE{}}
%       \label{fig:fig1_neg_attn}
%     \end{subfigure}%\hfill% or  or \hspace{0.3\textwidth}
%     % \vspace{-0.2cm}
%     \caption{Overall statistics for attention scores in Fig.~\ref{fig:motivation_ex} in QVHighlights dataset. 
%     (a) For the attention scores that measure how much the text query is generally involved in video representation, we use violin plots to show the probability density. We plot the score for each layer in the encoder.
%     % (b) Using the histogram, we compare how the baseline and QD-DETR yield different salient scores given the positive and negative video-text pairs.
%     (b) Saliency histogram shows the distributional gap between positive and negative video-text query pairs of baseline~(Moment-DETR) and proposed QD-DETR.\SE{}
%     }
%     \label{fig:motivation}
%     % \captionsetup{belowskip=13pt}
%     % \setlength{\belowcaptionskip}{-10pt}
% \end{figure}

% \begin{figure}[t]
% \centering

%     \begin{subfigure}[r]{1\linewidth}
%       \centering
%       \hspace{-0.2cm}
%     %   \includegraphics[width=1\linewidth]{figs/fig1_moti_negattn.pdf}  
%       \includegraphics[width=1.1\linewidth]{figs/fig1_moti_violin_a_v2.pdf}  
%     %   \caption{neg attention}
%         % \caption{Relation between the highlight-ness and the relevance between videos and query texts.}
%         \vspace{-0.5cm}
%         % \caption{Saliency histogram of positive and negative video-query pairs}
%         \caption{We plot the histograms and its average value~(dotted line) to compare saliency scores when true and false text queries are given for each method. (left) Since the video representations do not include much textual information, both the true and false queries yield similar saliency scores. (Middle) Even when the video representation is enforced to be updated with the textual information, the issue is not much resolved. (Right) By extracting discriminative features in the text query, distributions are differentiated.
%         % \SE{} % R1@0.5 설명
%         Also, R1@0.5 indicates evaluation metric, Recall at 1 with IoU 0.5 threshold on QVhighlight \textit{val} set.
%         }
%       \label{fig:fig1_neg_attn}
%     \end{subfigure}%\hfill% or  or \hspace{0.3\textwidth}
%     \\
%     \begin{tabular}{cc}
%     \hspace{-0.2cm}
%         \begin{minipage}{.4\linewidth}
%             \begin{subfigure}[l]{1\linewidth}
%               \centering
%             %   \vspace{0.20cm}
%             %   \includegraphics[width=1\linewidth]{figs/fig_1_moti_textattn.pdf}  
%             %   \includegraphics[width=1\linewidth]{figs/fig_1_moti_textattn_v2.pdf}  
%               \includegraphics[width=1\linewidth]{figs/fig1_moti_violin_a.pdf}  
%               \vspace{-0.60cm}
%             %   \caption{text attention}
%                 \caption{Importance of queries in video representation}
%               \label{fig:fig1_text_attn}
%             \end{subfigure}%\hfill% or  or \hspace{0.3\textwidth}
%         \end{minipage}
        
%         \begin{minipage}{.6\linewidth}
%             \vspace{-0.2cm}
%             \caption{Overall statistics of Fig.~\ref{fig:motivation_ex} in QVHighlights dataset. 
%             (a) Saliency histogram shows the distributional gap between positive and negative video-text query pairs.
%             % (a) For the attention scores that measure how much the text query is generally involved in video representation, we use violin plots to show the probability density. We plot the score for each layer in the encoder.
%             % (b) Using the histogram, we compare how the baseline and QD-DETR yield different salient scores given the positive and negative video-text pairs.
%             % (b) Text ratio in self-attention layer to  of Moment-DETR
%             % (b) Ratio of text when representing video tokens in self-attention of Moment-DETR.
%             % (b) Magnitude of attention text query involved.
%             % (b) Attention score of video tokens
%             % (b) Magnitude of text query to refine the video tokens in self-attention layer of Moment-DETR.
%             (b) Probability density depicting the weight of the text query in attention score for video clips. Scores are from the self-attention layers in Moment-DETR encoder.
%             % (b) The text query ratio in attention score of video clips (Self-attention layer in Moment-DETR encoder). We use violin plots to show probability density.
%             % 텍스트 쿼리가, 비디오 피쳐에 얼만큼 attend 하는지
%             }
%         \end{minipage}
    
%     \end{tabular}
%     \vspace{-0.5cm}
%     \label{fig:moti}
%     % \captionsetup{belowskip=13pt}
%     % \setlength{\belowcaptionskip}{-10pt}
% \end{figure}


% \begin{figure}
%     \centering
%     % \includegraphics[width=1\linewidth]{figs/fig1_moti_negattn_1109.pdf}
%     \includegraphics[width=1\linewidth]{figs/fig1_moti_negattn_stat_v2.pdf}
%     \vspace{-0.8cm}
%     \caption{
%         Histogram of saliency when the positive and negative queries are given. We plot the histograms and its average value~(dotted line) to compare saliency scores when relevant~(positive) and irrelevant~(negative) text queries are given for each method. (Left) Since the video representations do not properly reflect textual information, both the positive and negative queries yield similar saliency scores. 
%         % (Middle) Even when the video representation is enforced to be updated with the textual information, the issue is not much resolved. 
%         (Right) By representing video clips in query-dependent manner, distributions are differentiated.
%     }
%     \vspace{-0.6cm}
%     \label{fig:motivation}
% \end{figure}


% One of the demanding task is moment retrieval task, which is detecting the desired moments from the given query, typically the text query.
When describing the moment, one of the most favored types of query is the natural language sentence~(text)\cite{anne2017localizing}. 
While early methods utilized convolution networks~\cite{zhang2020learning, gao2021fast, wang2020temporally}, recent approaches have shown that deploying the attention mechanism of transformer architecture is more effective to fuse the text query into the video representation.
% To handle these modalities, previous works simply employed the attention mechanism of transformer architecture to fuse the text query into the video representation.
For example, Moment-DETR~\cite{momentdetr} introduced the transformer architecture which processes both text and video tokens as input by modifying the detection transformer~(DETR), and UMT~\cite{umt} proposed transformer architectures to take multi-modal sources, e.g., video and audio. 
Also, they utilized the text queries in the transformer decoder.
Although they brought breakthroughs in the field of MR/HD with seminal architectures, they overlooked the role of the text query.
To validate our claim, we investigate the Moment-DETR~\cite{momentdetr} in terms of the impact of text query in MR/HD~(Fig.\ref{fig:motivation_ex}).
Given the video clips with a relevant positive query and an irrelevant negative query, we observe that the baseline often neglects the given text query when estimating the query-relevance scores, i.e., saliency scores, for each video clip.
% the output saliency score, i.e. query-relevance scores.
% Based on the observation, we traced the actual saliency prediction of the model against both the video-relevant query and the irrelevant dummy one where we find that the baseline often neglects the given text query when estimating the query-relevance scores of video clips.
% For example, in Fig.~\ref{fig:motivation_ex}, saliency scores are not affected even when the query is substituted with the dummy.
% % General statistics for Fig.~\ref{fig:motivation_ex} is shown in Fig.~\ref{fig:motivation}. 
% General statistics corresponding to Fig.~\ref{fig:motivation_ex} are also shown in Fig.~\ref{fig:motivation}.



% The limitation of the concrete baseline~\cite{momentdetr} is inspected in two different aspects; 1) Utilization of text-query in the encoding process and 2) the output saliency score, i.e. query-relevance scores.
% Firstly, we visualize the attention score when video clips are given as a query in self-attention. 
% We observe that the text queries have relatively small impacts compared to other video features, as shown in Fig.~\ref{fig:fig1_text_attn_ex}.
% That is, the text does not account for much in representing every video clip, although the goal of MR/HD is to detect query-relevant moments.
% Based on the observation, we traced the actual saliency prediction of the model against both the video-relevant query and the irrelevant dummy one where we find that the baseline often neglects the given text query when estimating the query-relevance scores of video clips.
% For example, in Fig.~\ref{fig:motivation_ex}, saliency scores are not affected even when the query is substituted with the dummy.
% % General statistics for Fig.~\ref{fig:motivation_ex} is shown in Fig.~\ref{fig:motivation}. 
% General statistics are also shown in Fig.~\ref{fig:motivation}.

% Consequently, in Fig.~\ref{fig:fig1_neg_attn_ex}~(b), we found that the baseline often neglects the given text query when estimating the query-relevance scores of video clips; 
% For example, 


% We validate the previous work sometimes neglects the given query when estimating the saliency of video clips.
% For example, there is an example that the saliency scores from positive and negative queries cannot be distinguishable, as shown in Fig.~\ref{fig:fig1_neg_attn_ex}.
% % 우리는 추가로 text attention을 추가도 해봤지만, 효과가 있긴 했으나, still 이슈가 있는 것을 확인하였다?
% % Still, we observe that assuring the high attendance of text queries does not resolve the overlap which motivates us to question the quality of the naive use of task-agnostic text representation~\cite{momentdetr, umt}.
% We found that introducing the text-attention for ensuring the high attendance of text queries relieve the overlap, but there still be a severe overlap.


% To validate their limitations, we inspect the impacts of text queries in the concrete baseline~\cite{momentdetr} with the two different aspects, 1) tendency of attention in self-attention layer and 2) saliency score, i.e. query-relevance scores. \SE{} % attention 이 갑자기 등장하는가?
% Firstly, we visualize the attention score when video clips are given as a query in self-attention. We observe the text queries have relatively low attention scores compared to the video features, as shown in Fig.~\ref{fig:fig1_text_attn_ex}.
% That is, the text does not account for much in representing every video clip, although the goal of MR/HD is to detect query-relevant moments.
% Based on this observation, we trace the actual saliency prediction of the model against both positive and negative text queries.
% We validate the previous work sometimes neglects the given query when estimating the saliency of video clips.
% For example, there is an example that the saliency scores from positive and negative queries cannot be distinguishable, as shown in Fig.~\ref{fig:fig1_neg_attn_ex}.
% % 우리는 추가로 text attention을 추가도 해봤지만, 효과가 있긴 했으나, still 이슈가 있는 것을 확인하였다?
% % Still, we observe that assuring the high attendance of text queries does not resolve the overlap which motivates us to question the quality of the naive use of task-agnostic text representation~\cite{momentdetr, umt}.
% We found that introducing the text-attention for ensuring the high attendance of text queries relieve the overlap, but there still be a severe overlap.



% Thus, we 
% query dependency를 높이기 위해 
% Cross-attention? text-attention? detailed explanation on text-attention should be needed?
% By handling these two issues, we find that more precise retrieval can be achieved.
% 
% 
%
% By projecting video-discriminative text features with high text attendance to source video, we f 
% We also find the need to improve the quality of query features since assuring high text attendance also results in...
% pairs are not finetuned to be discriminative that even the similarity within the pairs does not reflect the relevance between the query and the video clips.
% General statistics for Fig.~\ref{fig:motivation_ex} is shown in Fig.~\ref{fig:motivation}. 
% \SE{} % 이거 ??로 뜨는데, 위처럼 figure 그리면 label이 안되는걸까요
% \SE{}
% 형님 아래 사항 생각 좀 해보는게 좋을 거 같아요.
% fig 1. (a) 그림만 봤을 때 모든 clip에 대해 text attention이 일정이상 존재하긴 하니까, 뭔가 not assured to be conditioned가 와닿지 않는거 같아요.
% + 왜 text가 항상 attend 해야하나?
% not assured to be conditioned --> text shows relatively low affects compared to video 같이 실제 나타난 현상까지 같이 적으면 어떨까 싶어요.
% fig 1. (b) 덜 반영한다?

% \SU{}
% 일단 text가 attend 잘 되어야 한다는 것에 좀 궁금점이 생깁니다. 결국에는 text와 관련있는 frame들을 attend해서 higlight를 찾아야 하는게 아닐까요? 그리고, 현제 저희의 모델 구조상 text query가 Key와 Value로 거의 활용되고 있는데 그렇다면 결국에는 해당 모델은 text에 대한 attention이 전혀 없다고 봐도 무방하지 않을까요? 그런 면에서 text attention을 강조하는게 좀 걸리긴 합니다.

% Specifically, the text query is not assured to be explicitly conditioned on every clip of the video, and as the query texts are evenly treated, discriminative keywords may not be spotlighted.
% attention mechanism of Moment-DETR is not explicitly conditioned on the text query as shown in Fig~\ref{}(d), and in UMT, the text are only used for conditioning the queries while the video representation are refined itself by self-attention.

% \begin{figure}[t]
%     \begin{subfigure}{1\linewidth}
%       \centering
%     %   \includegraphics[width=1\linewidth]{figs/fig_1_moti_textattn.pdf}  
%     %   \includegraphics[width=1\linewidth]{figs/fig_1_moti_textattn_v2.pdf}  
%       \includegraphics[width=1\linewidth]{figs/fig_1_moti_textattn_v4.pdf}  
%       \vspace{-0.5cm}
%     %   \caption{text attention}
%         \caption{Distribution of attention scores in Moment-DETR encoder}
%       \label{fig:fig1_text_attn}
%     \end{subfigure}%\hfill% or  or \hspace{0.3\textwidth}
%     \vspace{0.2cm}
%     \begin{subfigure}{1\linewidth}
%       \centering
%     %   \includegraphics[width=1\linewidth]{figs/fig1_moti_negattn.pdf}  
%       \includegraphics[width=1\linewidth]{figs/fig1_moti_negattn_v2.pdf}  
%       \vspace{-0.5cm}
%     %   \caption{neg attention}
%         \caption{Saliency score against positive and negative text queries}
%       \label{fig:fig1_neg_attn}
%     \end{subfigure}%\hfill% or  or \hspace{0.3\textwidth}
%     \vspace{0.2cm}
%     \begin{subfigure}{1\linewidth}
%       \centering
%     %   \includegraphics[width=1\linewidth]{figs/fig1_moti_violin.pdf}  
%       \includegraphics[width=1\linewidth]{figs/fig1_moti_violin_v2.pdf}  
%       \vspace{-0.5cm}
%       \caption{violin}
%       \label{fig:fig1_violin}
%     \end{subfigure}%\hfill% or  or \hspace{0.3\textwidth}
%     \vspace{-0.2cm}
%     \caption{(a) 1. portion of text attention vs. video attention 2. relation with text query and content (e.g. fg, bg) of clip seems not to affect the attention score
%     (b) 1. high variability even though entire clips are highly correlated with the given text query 2. positive and negative query makes overlaps on saliency score distribution
%     (3) actual distribution on validation dataset.}
%     \label{fig:motivation}
%     % \captionsetup{belowskip=13pt}
%     % \setlength{\belowcaptionskip}{-10pt}
% \end{figure}

To this end, we propose Query-Dependent DETR~(QD-DETR) that produces query-dependent video representation.
% Our key focus is to ensure each clip in predicted moments is explicitly conditioned by the query, particularly on the video-descriptive portion of the text query.
% Our key focus is to ensure that query-relevant clips are predicted by enforcing each clip to be explicitly conditioned by the query.
%Our key focus is to ensure that the model prediction for each clip is highly relevant to the query.
Our key focus is to ensure that the model's prediction for each clip is highly dependent on the query.
% by enforcing each clip to be explicitly conditioned by the query. :)
% hmm...
% \SE {} % "query-relevant clips are predicted" 이 문장이 좀 애매한거 같습니다. relevant 클립을 놓지지 않고 찾는 것을 보장한다? 이런 느낌인지 아니면 높은 saliency 를 주는게 목적이다? model prediction이 query-relevance를 반영하는 것을 보장한다?
% Our key focus is to ensure that the model prediction reflects query-relevance of clips by enforcing each clip to be explicitly conditioned by the query.
First, to fully utilize the contextual information in the query, we revise the transformer encoder to be equipped with cross-attention layers at the very first layers.
% 상익's thought :  single video - query간의 관계만 고려 - 같은 word가 더 많이 쓰이는 것을 보고 
% 교수님's thought : neg pair 를 쓰면 쿼리를 보지 않고서는 video clip간만 고려하는 것이 사라짐. 왜냐면 0으로 내보내야 하기 때문. --> SE: relative difference 만 고려하다가, 
By inserting a video as the query and a text as the key and value of the cross-attention layers, our encoder enforces the engagement of the text query in extracting video representation.
% 원준 교수님 코멘트 반영해서 다시
Then, in order to not only inject a lot of textual information into the video feature but also make it fully exploited, we leverage the negative video-query pairs generated by mixing the original pairs.
Specifically, the model is learned to suppress the saliency scores of such  negative~(irrelevant) pairs.
Our expectation is the increased contribution of the text query in prediction since the videos will be sometimes required to yield high saliency scores and sometimes low ones depending on whether the text query is relevant or not.
% \SE{}
% learns to?
% By suppressing the saliency scores of the irrelevant video-query pairs, the model learns to spotlight only the video-specific discriminative words in the query.
% % \SE{} % ====================== 상익 수정 ========================
% However, this architectural design still lacks the capability of identifying the video-descriptive keywords in the query.
% % However, this architectural design still lacks in identifying proper query relevance.
% This is because the current training scheme only focuses on the interactions of video and clips within a single video while neglecting information shared throughout the entire video.
% % We argue the problem of the current training scheme that only focuses on distinguishing the clips in a single video while neglecting information shared throughout the entire video.
% Therefore, we leverage the negative video-query relationships to enhance the capability of identifying the contextual similarity of query and video clips.
% 
% 원준 원본 
% However, this architectural design heavily relies on the quality of the text query.
% Therefore, we leverage the negative video-query relationships to enable the model to emphasize key corresponding query features.
% By suppressing the saliency scores of the irrelevant video-query pairs, the model learns to spotlight only the video-specific discriminative words in the query.
% =========================================================
Lastly, to apply the dynamic criterion to mark highlights for each instance, we deploy a saliency token to represent the entire video and utilize it as an input-adaptive saliency criterion. 
With all components combined, our QD-DETR produces query-dependent video representation by integrating source and query modalities.
This further allows the use of positional queries~\cite{dabdetr} in the transformer decoder.
% Furthermore, we can exploit the advanced DETR decoder architectures using the positional information, e.g., DAB-DETR, since our encoded tokens consist of identical position representations from a single modality.
% \SE{} % ====================== 상익 수정 ========================
% Furthermore, we can exploit the advanced DETR decoder architectures using the positional information, e.g., DAB-DETR, since our video clip tokens consist of identical position representations from a single modality.
% 원준 원본
% It also enables the use of advanced DETR decoder architectures, e.g., DAB-DETR, for the first time, as these works exploit the position information within a single modality.
% =========================================================
Overall, our superior performances over the existing approaches validate the significance of the role of text query for MR/HD.
% Our extensive experiments on QVHighlights, TVSum, and Charades-STA datasets validate the significance of considering the role and the quality of text query.

% All components combined with dynamic anchor moments for the query of decoder, our FOQUE fosters the query-dependent video representation, thereby making the 
% All components combined, our modified transformer encoding process fosters the query-dependent video representation thereby achieving the state-of-the-art results on various benchmarks of moment-retrieval and highlight detection.
	
% -	Video Platform & Streamer & Consumer의 증가. 
% Video는 다른 데이터 타입보다 정보가 많아 유용하지만, 이는 다른 말로 해석하면 video를 보는 것은 time-consuming 하고, 원하는 것을 찾아보기에는 힘들 수 있음.
% 따라서, 많은 매체에서는 사람들의 더 많은 이목을 끌기 위해 highlight 비디오라는 것을 편집하여 공유도 함.
% 하지만, highlight video를 만들기 위해 사람의 노력이 필요한 현 시점에서, This spotlights the need to retrieve the user-requested / Highlight moments in the video.

% -	이전에도 이러한 문제를 해결하기 위해 (asdfasdf) for moment retrieval, (asdfasdf) for highlight detection 등이 제안 되었지만, 이들은 비디오의 특정 영역을 찾는다는 공통된 목적을 가지고 있으면서도, 데이터 셋의 한계로 인해 따로 연구되었음. 이를 문제 삼으며, 최근에는 두 task를 동시에 학습할 수 있는 dataset이 소개 되었는데, 컴퓨터비전에서 최근 각광을 받고 있는 Transformer 모델 도입과 함께 큰 발전을 거듭하고 있음.

% -	구체적으로, 이 두가지 task를 수행하기 위해서는 transformer를 두가지 방법으로 이용할 수 있는데, moment-DETR 처럼 moment 를 clip의 set 단위로 예측할 수 있고, UMT 처럼 clip-wise prediction을 할 수 있음. 하지만, 이들은 query를 condition이 아닌 video와 동등한 레벨로 취급하거나 [mDETR], 매 클립이 self-attention으로 mixing 된 후에 condition을 걸어주어 clip간의 차이를 확실하지 이용하지 못하였고, 또한, 확실하게 condition으로 주지 못하였고, video와 query 사이의 관계를 한정적으로만 이용하였다.

% -	따라서, we explore three different ways to fully exploit query information. First, we design one-way cross-attention layer to condition every clip with the query features. Then, we utilized the negative video-text pairs to better model the relationships between the video and the text embeddings. Lastly, we define the saliency token to be the video-query dependent saliency estimator.


















% ===================== neg pair 부분 ===========================
% Nevertheless, the current training scheme, only considering the given video-query pair, still disturbs the model from identifying proper query-relevance prediction.
% In detail, the model focus on learning the fine-grained discrepancy between video clips, while neglecting the information they share, which contains significant clues to understand the context of video.
% Therefore, we leverage the negative video-query relationships to enhance the capability of identifying the contextual similarity of query and video clips.
% Therefore, we leverage the negative video-query relationships by suppressing those pairs, so that enhance the capability of identifying the contextual similarity of query and video clips.
% We hypothsize the diversity in query-video pairs are insufficient to learn the general relationship between text query and video.
% Therefore, we leverage the negative video-query relationships by suppressing the saliency scores of the irrelevant video-query pairs.
% However, this architectural design still lacks in identifying proper query relevance.
% We argue that the current training scheme only focuses on learning the fine-grained discrepancy between clips in a single video, while neglecting the information they share, which contains significant clues to understand the context of the video.
% Therefore, we leverage the negative video-query relationships to enhance the capability of identifying the contextual similarity of query and video clips.
% However, this architectural design still lacks in identifying proper query relevance.
% We argue the problem of the current training scheme that only focuses on learning the fine-grained discrepancy between clips in a single video.
% That is, the current design neglects the information shared throughout the video, although it contains significant clues to understand the context of the video.
\section{Related Work}
\label{sec:related_work}
\subsection{Co-Speech Gesture Synthesis}
The early approaches for generating co-speech gestures often involve creating linguistic rules to translate speech input into a sequence of pre-collected gesture segments, which are typically referred to as rule-based methods \cite{cassell1994rulefullbody,cassell2001beat,kipp2004gesture,kopp2006bml}. \citet{wagner2014rulereview} provide a comprehensive review of these methods. Rule-based methods produce interpretable and controllable results, but creating gesture datasets and rules requires significant effort. To alleviate the manual effort of designing rules in rule-based methods, data-driven approaches have gradually become predominant in this field. \citet{nyatsanga2023data_driven_gesture_survey} offer a thorough survey of these methods. Early data-driven approaches aim to directly learn mapping rules from data through statistical models \cite{neff2008videogesture,levine2009prosodygesture,levine2010gesturecontroller} and combine them with predefined gesture units for gesture generation. Later, the powerful modeling capability of deep neural networks makes it possible to train complex end-to-end models using raw speech-gesture data directly. One option is deterministic models, such as MLP \cite{kucherenko2020gesticulator}, CNN \cite{habibie2021videogesture}, RNN \cite{yoon2019robot,yoon2020trimodalgesture,bhattacharya2021affectivegesture,liu2022hierarchicalgesture}, and Transformer \cite{bhattacharya2021text2gestures}. Another choice is generative models, including flow-based models \cite{alexanderson2020stylegesture,ye2022styleflowgesture}, VAEs \cite{li2021audio2gesture,ghorbani2022zeroeggs}, and VQ-VAE \cite{yi2022talkshow,yazdian2022gesture2vec,liu2022vqgesturevideo}. Due to the inherent many-to-many relationship between speech and gesture, end-to-end models can generate natural-looking gestures but face challenges in ensuring content matching between speech and generated gestures \cite{yoon2022genea}. To address this issue, some neural systems aim to explicitly model both rhythm and semantics from the perspective of model structure \cite{kucherenko2021speech2properties2gestures,ao2022rhythmicgesticulator,liu2022disco} or training supervision strategy \cite{liang2022seeg}. Furthermore, hybrid systems, such as the combination of deep features and motion graphs \cite{zhou2022gesturemaster}, have been proposed to harness the advantages of different approaches. Recently, diffusion models \cite{sohldickstein2015diffusion,song2020improvedscore,ho2020ddpm} have demonstrated impressive results in image synthesis \cite{ramesh2022dalle2} and human motion generation \cite{tevet2022humanmotiondiffusion, zhang2022motiondiffuse}. Inspired by these works, our system adapts the latent diffusion model \cite{rombach2022latentdiffusion} for the co-speech gesture generation task and achieves appealing results.

\subsection{Style Control for Human Motion}
A typical approach to style control for human motion involves specifying a motion clip as a reference and transferring the reference clip's style to the source motion. This task is also known as \emph{style transfer}. Early works in motion style transfer integrate traditional machine learning techniques with manually defined features to infer motion styles \cite{hsu2005motion_style_translation,ma2010motion_style_transfer,xia2015realtime_motion_style_transfer,yumer2016spectral_motion_style_transfer}. Recently, deep learning-based methods have significantly enhanced motion quality. \citet{holden2016deepmotion} first propose a learning framework enabling motion style control through optimization in the motion manifold space. \citet{du2019stylemotioncvae} improve transfer efficiency by training a conditional VAE. \citet{mason2018few-shot_motion_style_transfer} use few-shot learning to generate stylized locomotion. \citet{aberman2020adain} employ a temporally invariant adaptive instance normalization (AdaIN) layer for target style injection, eliminating the need for paired data during training. \citet{wen2021stylemotionflow} achieve unsupervised style transfer using a flow model. \citet{jang2022motionpuzzle} introduce a method capable of controlling styles for individual body parts.

Previous co-speech gesture synthesis systems with style control can be categorized based on whether or not they require style labels. For methods needing labeled data, early works can only learn an individual style for one generator \cite{levine2010gesturecontroller,neff2008videogesture,ginosar2019stylegesture}. \citet{ahuja2022lowresource} propose a strategy that efficiently adapts the source generator to another speaker style using low-resource data. Some works learn a speaker style embedding space with labeled speaker-motion data, enabling gesture style control by sampling from this space \cite{ahuja2020stylegesture,yoon2020trimodalgesture,bhattacharya2021affectivegesture}. \citet{alexanderson2020stylegesture} aimat controlling fine-grained styles, such as gesturing speed and spatial scope, using preprocessed control signal-motion data. Their later work \cite{alexanderson2022diffusiongesture} utilizes a diffusion model for audio-driven motion synthesis, achieving label-based style control by training the model on labeled data. For methods not requiring style labels, \citet{habibie2022motionmatching} propose a motion matching framework to achieve flexible style control. Other studies achieve arbitrary style control by imitating an example given as a video \cite{liu2022hierarchicalgesture} or a motion clip \cite{ghorbani2022zeroeggs,ye2022styleflowgesture,kuriyama2022tokenizedgestures}.  In this work, we utilize a CLIP-based encoder to extract a style embedding from an arbitrary text prompt and incorporate it into the generator via an AdaIN layer, guiding the synthesis of stylized gestures. Our system supports fine-grained multimodal style prompts as opposed to label-based style control. It employs a self-supervised learning scheme and eliminates the need for labeled data. Additionally, we use an autoregressive model rather than a parallel model, making it potentially suitable for real-time applications.
\section{Simulation Setup}
We perform physics simulation using Mujoco~\cite{todorov2012mujoco} inspired by successful prior work in hand manipulation~\cite{garcia2020physics,andrychowicz2020learning}. 
% Considering related work \cite{garcia2020physics,andrychowicz2020learning} applying DRL to contact-rich motor control tasks, we choose to leverage the Mujoco physics simulator for our experimental environment~\cite{todorov2012mujoco}. %\alex{Should we explcitly reference mimic here?} \yuting{A sentence of overview}

\subsection{Motion Capture Sequences}

% I think we should omit this reference, since the tracking method is fully described in the DeepLabels paper:
%yuting: I added it back because it's not only how the data is captured, but also what the content is; it also signals that we didn't capture the data ourselves for this project, it's an open sourced dataset.

\begin{wrapfigure}{r}{0.40\columnwidth}
    \centering
    \vspace{-0.25cm}
    \includegraphics[width=0.40\columnwidth]{figures/mocap_paper_figure.png}
    \caption{A still frame from an input motion capture sequence.}
    \label{fig:mocap_example} 
    \vspace{-0.15in}
\end{wrapfigure}

We use the object manipulation motion capture dataset from Zhang et al.~\shortcite{zhang2021manipnet}. The data was captured at 120 frames per second with the Optitrack \cite{optitrack} system using reflective markers.  A total of 19 markers are placed on each hand: four on the thumb, three on each finger, and three more on the back of the hand. The Optitrack system outputs object motions as rigid bodies, and the finger motions are reconstructed using DeepLabels \cite{han2018online}.

% We apply the control policy at the same frequency as the motion, but run the physics simulation at 600 frames per second for more stable results.

\subsection{Hand and Object Models}

\begin{wrapfigure}{r}{0.40\columnwidth}
    \centering
    \vspace{-0.25cm}
    \includegraphics[width=0.40\columnwidth]{figures/hand_model.png}
    \caption{Hand model with $20$ rotational axes.}
    \label{fig:hand_model} 
    \vspace{-0.15in}
\end{wrapfigure}

The dataset comes with a hand model and object models. The hand model consists of $20$ degrees of freedom, and we model each finger segment as capsules (see \revised{Figure}~\ref{fig:hand_model}).  For efficient collision detection in the simulation, objects are modeled either as convex primitive shapes such as a cube or a cylinder, or approximated as a combination of convex primitives if they are nonconvex (e.g. torus and wine glass). \revised{For simplicity purpose, we only enable the collision between the hand and the object. Self collision within the hand are disabled during the manipulation training.}

%On each capsule, we also place a contact sensor of the same shape and slightly bigger size, which accumulates contact forces exerting on it at each control step.

%The motion capture data contains sequences of hand and object poses. Figure ~\ref{fig:mocap_stillframe} shows a still frame of the mocap of grasping a cube using the right hand. As illustrated in Figure \ref{fig:hand_model}, the skeleton of each hand is consist of $19$  capsule segments and $20$ revolute joints on its fingers. On each capsule segment, we place a capsule contact sensor covers that segment and will record the cumulative contact forces exerting on the segment.

% \begin{figure}[h]
%     \centering
%     \includegraphics[width=1.7in]{figures/hand_model.png}
%     \caption{Our hand model with $20$ rotational axes.}
%     \label{fig:hand_model} 
% \end{figure}


%     \caption{Example of simulated object.}
%     \label{fig:object_example} 
% \end{figure}
%\begin{figure}[h]
%    \centering
%    \includegraphics[width=2in]{figures/mocap_stillframe.png}
%    \caption{Motion Capture}
%    \label{fig:mocap_stillframe}
%\end{figure}

\subsection{Contact Parameters}
We find that the contact-rich nature of in-hand manipulation tasks demands greater care when choosing simulation parameters related to collision resolution. Mujoco formulates contacts as a constrained optimization problem rather than a linear complementary problem to control the compliance of contact surfaces. In all of our experiments, we choose \revised{Mujoco built-in contact} parameters that correspond to highly stiff surfaces. Specifically, we set the fixed constraint impedance to $0.95$. We also reduce the solver's time constant from $0.02$ to $0.005$, which is the inverse of the natural frequency times the damping ratio. Unfortunately, such stiff contacts increase the instability of the simulator. As a mitigation, we choose elliptic friction cones and the RK4 integrator, and run the simulation at $600$Hz. We additionally found slipping could be a common failure. To increase friction, we set the ratio of friction-to-normal contact impedance to $5$, and strengthen the friction forces with higher normal forces. For more details of the contact model, please refer to the official documentation~\cite{todorov2012mujoco}. We found the above settings empirically works over all the motion sequences we tried, allowing us to achieve effective dexterous manipulation without the fingers ``sinking'' into the object's surface or sliding over sharp edges.

% \subsection{Upper Body Animation}
% For the preview of the hand animation on a human character, we generate upper body motions using a kinematic approach. For the given hand motion trajectory, we solve an inverse kinematic problem to match the position at the end of the lower arms to the position of the roots of the hands at every frame. 
% \sehoon{Still not sure where is the best place for this paragraph.}
% We present a straight forward way to generate animation for a humanoid character. For a hand motion trajectory, we are solving for an inverse kinematic problem for an upper body humanoid character so that the position at the end of the lower arms are matching the position of the roots of the hands at every frame. 
\section{Motion Imitation}
The foundation of our algorithm is an imitation learning framework. In the following sections, we describe the observations, actions, reward functions, and the training procedure. We explain these for a single hand, but our method can trivially incorporate both hands by increasing the state and observation dimensions.

\subsection{Problem Formulation}
Given motion capture trajectories for a hand (or both hands) and an object, we aim to learn an effective policy that can manipulate the object by following a reference trajectory. 
% We apply deep reinforcement learning (DRL) to learn such a control policy.

We formulate the control problem as a partially observable Markov Decision Process (PoMDP) with tuple $(\mathcal{S}, \mathcal{O},o, \mathcal{A},\mathcal{T},r, o, \gamma)$, where $\mathcal{S}$ is the state space, $\mathcal{O}$ is the observation space for the hand and the object, $\mathcal{A}$ is the action space for actuating the hand, $\mathcal{T}$ is the transition dynamics (physics simulation), $r$ is the reward function, $o$ is the observation emission function, and $\gamma$ is the discount factor. At a high level, we want to find a parameterized control policy $\pi_{\theta}$ so that it will maximize the expected sum of rewards over a distribution of trajectories
\begin{equation}
    \pi_{\theta^*} = \argmax_{\theta} \mathbb{E}_{(s_0,s_1,\dots,s_T)}\left[\sum_{t=0}^{T}\gamma^{t}r(s_t, \pi_{\theta}(s_t))\right].
\end{equation}

\subsection{Observation representation}
%\sloppy The observation of the agent is encoded in the following vector form $\mathbf{o} = (\mathbf{x}_{hand},\mathbf{x}_{obj},\mathbf{v}_{hand},\mathbf{v}_{obj},\bar{\mathbf{x}}_{hand},\bar{\mathbf{x}}_{obj},\bar{\mathbf{x}}_{hand}\ominus \mathbf{x}_{hand},\bar{\mathbf{x}}_{obj}\ominus \mathbf{x}_{obj},\mathcal{C})$. 
The observation space can be divided into four components: the states of the simulated hand and the object $\{\mathbf{x}_{hand},\mathbf{x}_{obj}\}$, reference states of hand and object from mocap $\{\bar{\mathbf{x}}_{hand},\bar{\mathbf{x}}_{obj}\}$, their differences from simulation $\{\bar{x}_{hand}\ominus x_{hand} ,\bar{x}_{obj}\ominus x_{obj}\}$, and contact information $\{\mathcal{C}\}$. As studied in Bergamin et al.~\shortcite{bergamin2019drecon}, using the difference between simulation and reference as observation improves both training speed and quality. 

\emph{Simulated states:} The positional state of the hand $\mathbf{x}_{hand} =\linebreak[1]  (\mathbf{x}_{h\_root},\linebreak[1]  \mathbf{R}_{h\_root},\linebreak[1]  \cos(q),\linebreak[1]  \sin(q))$ is a $46D$ vector containing the pose of the root and all joint angles of the hand, where $\mathbf{x}_{h\_root} \in \mathbb{R}^3$ is the $3D$ position of the wrist, $\mathbf{R}_{h\_root} \in \mathbb{R}^3$ orientation of the wrist represented as axis-angle, and $q \in \mathbb{R}^{20}$ are all the joint angles in a hand. $\mathbf{x}_{obj} = (\mathbf{x}_{o\_root},\mathbf{R}_{o\_root})$ is a $6D$ vector containing the position and axis-angle orientation of the simulated object. $\mathbf{v}_{hand}$ and $\mathbf{v}_{obj}$ are $26D$ and $6D$ vectors containing the velocities of the hand and the object. Note that we actuate the hand's orientation directly, and we found the use of axis-angle for hand orientation to be especially important for success.

\emph{Reference states:} $\bar{\mathbf{x}}_{hand}$ and $\bar{\mathbf{x}}_{obj}$ are the reference poses of the hand and object at the current frame expressed in the same format as the pose of simulated hand and object. 

\emph{State differences:} $\bar{x}_{hand}\ominus x_{hand}$ and $\bar{x}_{obj}\ominus x_{obj}$ are the differences between the simulated pose and the reference pose of both the hand and object. For all the rotational information, we evaluate the rotation differences in $SO(3)$ and express the difference into the corresponding format \revised{of axis angle or sine and cosine of Euler angles} used in the observation.

\emph{Contact information:} We include an additional $19D$ vector $\mathcal{C}$ to capture the contact forces between the hand and the object. We surround each finger capsule with a contact sensor that is a slightly larger capsule with a 10\% larger radius. These contact sensors register contact forces from the object at each control step, and we record the sum of all contact forces exerted by the object on each rigid segments of the hand through the contact sensors. Because the sensors correspond to finger segments, their readings indicate how much contact forces are being exerted, and implicitly inform where the contacts are. 
%In the last part of the observation, we want to implicitly inform the policy about the geometry of the object, so that the policy will be able to adjust it's behaviour based on the shape changes of the object. To do so, 
% \yuting{not clear how contact force magnitude encodes geometry info}

Combining all the described observation components, we have a $207D$ vector that describes the state of the simulation when we are considering a manipulating task involving a single hand. When we train the policy to track a two-hand manipulation sequence, we double the observations for the hand and end up with a $390D$ vector.

\subsection{Action Representation}
Similar to the approach in \cite{bergamin2019drecon}, at each time step $t$, the policy outputs an action $\mathbf{a}_t = \{\Delta{\mathbf{x}},\Delta{\mathbf{R}},\Delta{\mathbf{q}}\}$, a $26D$ vector specifying the spatial displacement to the hand's reference pose, where $\Delta{\mathbf{x}} \in \mathbb{R}^3$ is the hand root linear displacement, $\Delta{\mathbf{R}} \in \mathbb{R}^3$ is the root orientation displacement expressed in axis-angle, and $\Delta{\mathbf{q}} \in \mathbb{R}^{20}$ is the joint angle displacement for each joint. We apply an exponential action filter with $\alpha=0.3$ to generate smoother motions. Once the PD target is computed, we compute joint torques using the stable-PD controller~\cite{tan2011stable}. \revised{The full control loop is shown in Figure \ref{fig:control_loop}}
\begin{figure}
\centering
\includegraphics[width=0.48\textwidth]{figures/control_loop.png}
\caption{Overview of the control loop}
\label{fig:control_loop}
\end{figure}
%For more details, please refer to the original Stable PD paper.

% After an action is generated, we compute the target $q_{targ}=(1-\alpha) q_{prev} + \alpha (q_{ref}+a_{t})$ as blending between hand pose from previous frame and new pose generated by the action so that the generated target across simluation is smooth.

% Instead of directly using proportional derivative (PD) controllers, we adapt the stable-PD controller proposed by Tan et al.~\shortcite{tan2011stable} in simulation. This change allows us to more easily set the proportional and damping coefficients in the controller to make sure the simulated hand can always accurately reach the desired pose in the next few simulation steps without overshooting. %\sehoon{Provide a one-line justification of Stable PD}
%  Since our hands have full translation and rotation freedom, the learned policy needs to directly control both the roots as well as all the finger joints. A torque for each degree of freedom is computed using the stable-PD formulation describe as:
% \begin{equation}
%     \tau = -K_p (q_t + \dot{q}_t\Delta{t}-q_{targ})-K_d(\dot{q}_t+\ddot{q}_t\Delta{t}).
% \end{equation}
% For more details, please refer to the original Stable PD paper.

\subsection{Reward Function}
Our goal is to track the reference motions of both the hands and the object as closely as possible. Inspired by the original DeepMimic paper~\cite{peng2018deepmimic}, we design our reward function as follows:
\begin{equation}
    r = w_{od}r_{od}+w_{or}r_{or}+w_{hd}r_{hd}+w_{hr}r_{hr}+w_{hj}r_{hj}
\end{equation}
which consists of the object position term $r_{od}$, the object rotation term $r_{or}$, the hand position term $r_{hd}$, the hand orientation term $r_{hr}$, and the hand joint term $r_{hj}$.
To enforce a match between the simulated object and the reference object's position and orientation, we define the terms $r_{od}$ and $r_{or}$:
\begin{equation}
    r_{od} = \exp\left(-k_{od}\|\hat{x}_{obj}-x_{obj}\|^2\right),
\end{equation} and
\begin{equation}
    r_{or}=\exp\left(-k_{or}\|\hat{q}_{obj}^{-1}q_{obj}\|^2\right),
\end{equation}
which compares the object's position $x_{obj}$ and orientation $q_{obj}$ to their desired values.
For all $N$ rigid segment of the hand with the index i, we define the reward terms $r_{hd}$ and $r_{hr}$:
\begin{equation}
    r_{hd}=\exp\left(-k_{hd}\sum_{i =1}^N\|\hat{x}_{i}-x_{i}\|^2\right),
\end{equation} and 
\begin{equation}
    r_{hr}=\exp\left(-k_{hr}\sum_{i =1}^N\|\hat{q}_{i}^{-1}q_{i}\|^2\right),
\end{equation}
where $x_{i}$ and $q_{i}$ represent the position and orientation of the $i$th body segment.
In addition enforcing the hand rigid segment's tracking, we also define the reward term $r_{hj}$
\begin{equation}
    r_{hj}=\exp\left(-k_{hj}\sum_{i =1}^{M}\|\hat{\theta}_{i}-\theta_{i}\|^2\right),
\end{equation}
by comparing all the current and desired joint angles, $\theta$ and $\hat{\theta}$. For all experiments, we set the weights as $w_{od}=4$, $w_{or}=4$, $w_{hd}=0.05$, $w_{hr}=0.05$, and $w_{hj}=0.1$. 
% These terms minimize positional differences of each rigid finger segment between simulation and the reference as well as joint angle differences between simulated and reference hands.

\subsection{Terminal Condition}
As studied in DeepMimic~\cite{peng2018deepmimic}, early termination of a rollout when the simulation enters an unrecoverable state can save computation on low value trajectories. We design the early termination criteria to restrict how much the object's state is allowed to deviate from the reference: either $d_{thr}=10cm$ in translation or $\phi_{thr}=60^\circ$ in rotation. We choose these thresholds to allow the hands to explore its action space more freely, but this still eliminates irredeemable failures.  

% \subsection{Policy Training}
% We use Proximal Policy Optimization (PPO)~\cite{schulman2017proximal}, a common on-policy reinforcement learning algorithm in all our training. \sehoon{Yunbo, please say a sentence or two about the policy architecture}
% PPO aims to minimized two optimization losses $L_{surrogate}$ and $L_{KL}$ described as:
% \begin{align}
%     L_{PPO}(\theta) &=L_{surrogate}+L_{KL}\nonumber\\
%     &=-\mathbb{E}_{t}\left[min(r(\theta)A_t,clip(r(\theta),1-\epsilon,1+\epsilon)\right]\\
%     &\ \ \ - \beta\mathbb{E}_{t}\left[KL\left[\pi_{\theta}(s_t)|\pi_{\theta_{old}}(s_t)\right]\right]\nonumber .
% \end{align}
% A dynamically adjusted coefficient $\beta$ is used to make sure policies from two consecutive iterations do not deviate too much in their KL divergence. 



%We terminate the rollout when the simulated object is drifted away more than distance $d_{thr}=10cm$ from the reference or its orientation is off by $\phi_{thr}=60^\circ$ from the reference. This criteria gives some tolerance for the hand to adjust its motion to capture the object back in place, and it will be triggered most likely when an object is out of control and about to fall out of the hand. 

 


\section{Greedy Shape Curriculum for Novel Objects}


% Motivation for reusing the mocap
It would be undesirable to record a new motion capture sequence for each new object that we want to manipulate. Instead, we would like to generalize an existing motion example to different objects in simulation. For example, we may want to manipulate a teapot or a toy train using the same reference motion for a cube. However, it is not straightforward to learn an effective policy for a new shape because it often requires significant changes in the control strategy.

% Intuition. Why would a naive curriculum not work?
Our key intuition is that we can co-train policies on a set of intermediate shapes morphing between the original object and the target object as a curriculum. A naive method would be to use a training curriculum that starts the learning from the source object and gradually morph the shape to the target in a linear progression. In practice, however, this linear curriculum is often unsuccessful because the morphing progression may not exactly correlate to the task's difficulty. \revised{A better tuned morphing algorithm might be able to give stronger correlation between morphing progression and training difficulty, but such algorithm requires additional human effort, and may not be able to generalize across different source target pairs.}


\begin{figure}
\centering
\includegraphics[width=0.48\textwidth]{figures/system_diagram.png}
\caption{Illustration of our greedy shape curriculum. Each iteration of the algorithm (1) selects and trains the most promising (policy, shape) pair, (2) evaluates the updated policy on all shapes, and (3) overwrites a shape’s policy pairing if the new policy is better than the cached policy. Example policy performance metrics are displayed numerically below each policy shape.}
\label{fig:morph_training}
\end{figure}
% Our approach: insight and summary.
Instead, we design a novel training schedule that allows greedily switching between any intermediate shape morphs for a more flexible curriculum. Our algorithm maintains a collection of best policies for each shape. For every $K=20$ policy iteration, it selects the best performing \emph{unsuccessful} morph and its paired policy for the next round of training. Once the policy is further trained, the newly updated policy's performance is evaluated across the entire collection of shapes, and overrides existing policies if it performs better (Figure \ref{fig:morph_training}). Despite its greedy nature, we found this automated curriculum more effective in policy transfer than naive fining-tuning or a fix curriculum. The full procedure is described in Algorithm~\ref{alg:shape_morph_training}.
\begin{figure*}[h!]
    \centering
    \includegraphics[height=\low]{figures/sequence_still_frames/cube_bunny_0.0.png}
    \hfill
    \includegraphics[height=\low]{figures/sequence_still_frames/cube_bunny_0.2.png}
    \hfill
    \includegraphics[height=\low]{figures/sequence_still_frames/cube_bunny_0.4.png}
    \hfill
    \includegraphics[height=\low]{figures/sequence_still_frames/cube_bunny_0.6.png}
    \hfill
    \includegraphics[height=\low]{figures/sequence_still_frames/cube_bunny_0.8.png}
    \hfill
    \includegraphics[height=\low]{figures/sequence_still_frames/cube_bunny_1.0.png}
    \hfill
    \caption{Morph stages of the collision shapes for transferring the cube motion to a bunny after applying V-HACD.}
    \label{fig:collision_shape_morphs}
\end{figure*} 
\begin{figure*}[h!]
    \centering
    \includegraphics[height=\high]{figures/sequence_still_frames/wineglass_1.png}
    \hfill
    \includegraphics[height=\high]{figures/sequence_still_frames/wineglass_2.png}
    \hfill
    \includegraphics[height=\high]{figures/sequence_still_frames/wineglass_3.png}
    \hfill
    \includegraphics[height=\high]{figures/sequence_still_frames/wineglass_4.png}
    \hfill
    \includegraphics[height=\high]{figures/sequence_still_frames/wineglass_5.png}
    \hfill
    \label{fig:torus_large1_single}
\end{figure*}

\begin{figure*}[h!]
    \centering
    \includegraphics[height=\high]{figures/sequence_still_frames/hemi_0.png}
    \hfill
    \includegraphics[height=\high]{figures/sequence_still_frames/hemi_1.png}
    \hfill
    \includegraphics[height=\high]{figures/sequence_still_frames/hemi_2.png}
    \hfill
    \includegraphics[height=\high]{figures/sequence_still_frames/hemi_3.png}
    \hfill
    \includegraphics[height=\high]{figures/sequence_still_frames/hemi_4.png}
    \hfill
    \caption{Still frames from manipulation sequences involving a wineglass (\textbf{top}) and a hemisphere (\textbf{bottom}).}
    \label{fig:hemisphere_large1_single}
    \vspace{-0.1in}
\end{figure*}
\begin{algorithm}[tb]
\caption{Greedy Shape Curriculum}
\label{alg:shape_morph_training}
\begin{algorithmic}[1]
    \STATE Initialize score list $S$
    \STATE Initialize policy list $\Pi$
    \STATE Initialize current shape $s=0$ and current policy $\pi = \Pi[s]$
    \FOR{$i = 0, 1, 2 ,\dots$}
    \IF{$i \mod k ==0$}
    \FOR{Every shape $j$}
    \STATE score = rollout on $j$ using policy $\pi$
    \IF{score > $S[j]$}
    \STATE $S[j]$ = score
    \STATE $\Pi[j] = \pi$
    \ENDIF
    \ENDFOR
    \STATE $s$ = Get best unsuccessful shape
    \STATE $\pi = \Pi[s]$
    \ENDIF
    \STATE PPO using $s$ and $\pi$
    \ENDFOR
\end{algorithmic}
\end{algorithm}

A key component of our greedy shape curriculum is a \emph{goodness score} that describes how likely a policy will succeed on a given shape. This metric will be used to update the best policy for a shape, and for selecting the next (shape, policy) pair for training. An obvious choice would be the average episodic reward. However, the consideration here is slightly different. A high episodic reward imposes a more strict constraint to the quality of object pose matching, making it hard to achieve when the target shape is too different from the source shape. A low episodic reward, on the other hand, cannot guarantee the completion of a rollout. On the other hand, the rollout length alone is too simple and fails to reflect the quality of the motion. We want a criteria with high tolerance to object deviation but with low tolerance to failure of completion. To this end, we design our criteria as a combination of the rollout duration and tracking accuracy, and this works robustly in practice. As described in \revised{Equation} \ref{equ:eval_score}, we use the product between the normalized episode length and the sum of hand joint reward of the rollout as the goodness score of a policy for a given shape. This encourages the resulting policy to use a similar manipulation strategy to the input. We consider a score higher than $d=0.55$ as \emph{successful}, and we only pick from the unsuccessful shape morphs for policy training to make progress.
%we compare this evaluation score against a threshold $d$, and mark all shapes with score below that as unsuccessful. This makes sure that we are always training policies greedily from all the unsuccessful candidates.
\begin{equation}
    \label{equ:eval_score}
    f = \frac{L}{T} \cdot \frac{\sum_{0}^{L}{r_{joint}}}{T}
\end{equation}

Our greedy schedule is effectively an exploitation strategy, and we still need to balance it with some exploration to avoid local minima. If a particular shape is repeatedly picked for training and starving other shapes, we instead randomly select another shape and its paired policy in the next iteration. This is especially helpful when a policy gets ``stuck'' on a challenging frame towards the end of a sequence while other shapes have not had much training yet. Training progress on other shapes can then help improve such challenging cases later. Similarly, if we are successful on all shapes before the compute budget has been reached, we randomly pick a policy to continue training for further improvement. 
%This would make sure all shapes can continue to improve. In the meantime, if a particular morph is being too challenging to be solved, this design gives the learning a chance to attend to some other morphs and potentially helping solve the challenging morph. 



% It would be undesirable to need to record a new motion capture sequence for each new object that we want to manipulate.  Instead, we would like to be able to learn how to manipulate a new object using the mocap from another object. As an example, we may want to manipulate a teapot following the reference motion from a cube. \sehoon{challenge?} 

% To train such a policy, we start by creating a set of intermediate shape morphs between the original object and target object.  We then use a learning process to train a collection of policies, one policy for each individual morph, including the original and target object. The intermediate morphs act as stepping stones between the original shape from the mocap sequence and the new object that we want to manipulate. 

% The most straightforward way to train the collection of morphs would be to use a training curriculum.  We would start by training a policy on the source object. Once this policy is successful, this would be used as a starting policy that is then further trained using the next of the morph shapes. The policy from the second shape would then be use for the third shape, and so on. Unfortunately, we found that this strict curriculum over object shapes is often unsuccessful. \sehoon{provide more insights} We hypothesized that it may be beneficial to jump between shapes out of sequence, and indeed this turned out to be the case.

% \sehoon{Two key aspects: how to maintain a list, and when to transfer.}

% \sehoon{Little more focus on greedy aspects..}
% Instead of training a policy based on a sequential curriculum, we maintain a list of the best performing policies for each individual shape in the list as well as their performances. For every $k=20$ training iteration, the system selects the best performing unsuccessful morph and its paired policy for the next round of training. Once the policy is further trained, the new policy's performance is evaluated for each of the shapes. When it is better than the shape's current policy, it becomes the new policy for that shape. This full procedure is described in Algorithm \ref{alg:shape_morph_training}.


%We have found our training approach to be more successful than a naive training curriculum across shapes. Figure~\ref{fig:policy_tree} shows our approach for the source and target shapes of a cube and an elephant. Each arrow indicates when an updated policy performs better than the current policy for another shape.

% \greg{Yunbo, please add a description of how you calculate a given policy's score.}
% To properly measure how well a policy performs on a shape morph, we take consideration of both the duration of a rollout.
% \begin{algorithm}[tb]
% \caption{Shape Morphing Training}
% \label{alg:shape_morph_training}
% \begin{algorithmic}[1]
%     \STATE Initialize score list $S$
%     \STATE Initialize policy list $\Pi$
%     \STATE Initialize current shape $s=0$ and current policy $\pi = \Pi[s]$
%     \FOR{$i = 0, 1, 2 ,\dots$}
%     \IF{$i \mod k ==0$}
%     \FOR{Every shape $j$}
%     \STATE score = rollout on $j$ using policy $\pi$
%     \IF{score > $S[j]$}
%     \STATE $S[j]$ = score
%     \STATE $\Pi[j] = \pi$
%     \ENDIF
%     \ENDFOR
%     \STATE $s$ = Get best unsuccessful shape
%     \STATE $\pi = \Pi[s]$
%     \ENDIF
%     \STATE PPO using $s$ and $\pi$
%     \ENDFOR
% \end{algorithmic}
% \end{algorithm}


\section{Results}

% 这个部分展示了我们结果的性能优势,首先,我们通过多组对照试验确定了最优的架构参数,然后我们基于UNet与SwinUNETR预训练了通用模型与任务特定模型,在结果表中,*表示通用模型,否则为任务特定模型。同时,我们使用了多种来源的数据,涉及了不同模态,不同器官和不同的分割目标来验证HybridMIM的鲁棒性。此外,我们还验证了不同有标签数据比例下,HybridMIM依然能够有较高的性能优势。最后,我们还进行了消融实验,验证了HybridMIM中不同模块的有效性。
This section demonstrates the significance of our proposed HybridMIM method. 
%%
First, we make comparison with the current state-of-the-art approaches from four aspects: downstream segmentation performance (quantitative and qualitative), annotation cost reduction, and pre-training speed. 
%%
We then conduct ablation experiments to explain how to determine the optimal architectural parameters, and illustrate the contribution of each component to the performance of HybridMIM.

%This section demonstrates the performance advantage of our results. First, we determine the optimal architectural parameters by multiple controlled trials, and then we pre-train the generic and task-specific models based on UNet with SwinUNETR. In the result table, * indicates the generic model, otherwise the task-specific model. Also, we use data from multiple sources involving different modalities, organs, and segmentation targets to validate the robustness of HybridMIM. In addition, we verify that HybridMIM can still have high-performance advantages with different scales of labeled data. Finally, we also conduct ablation experiments to validate the effectiveness of different modules in HybridMIM.

\begin{table}[th]
    %\centering
    % 其中MSD Liver数据集需要分割肝脏和对应的肿瘤。MSD Spleen数据集需要分割脾脏。我们使用Dice和HD95来评估不同对比方法的性能。无论基于UNet架构还是SwinTransformer架构,MP-SSL方法都对其有很高的性能提升,并实现了state-of-the-art的结果。
    %\vspace{-2mm}
    \caption{The MSD Liver dataset requires segmentation of the liver and the corresponding tumor. and the MSD Spleen dataset requires segmentation of the spleen.}
    \label{tab:msd_segmentation}
    \renewcommand\arraystretch{1.3}
    \setlength\tabcolsep{3pt}%调列距
    \resizebox{\columnwidth}{!}{
    \begin{tabular}{c | c c c c c c | c c c}

    \hline
    Organ & \multicolumn{6}{c}{Liver} & \multicolumn{2}{c}{Spleen} \\
    \hline
    Metrics & Dice & Dice & Dice & HD & HD & HD & Dice & HD \\
     & liver & tumor & Avg & liver & tumor & Avg &  &  \\
    \hline
    SegresNet & 95.53 & 48.26 & 71.90 & 0.81 & {15.31} & 25.31 & 94.10 & 0.5\\
    UNETR & 93.07 & 33.59 & 63.33& 1.26 & 30.50 & 15.88 & 94.04 & 0.58\\
    SwinUNETR & 95.14 & 45.11 & 70.13 & 0.89 & 21.31 & 11.11 & 94.61 & 0.25\\
    \hline
    ModelGen & 95.22 & {52.53} & 73.87 & 0.67 & 18.83 & 9.75 & 94.43 & 0.63 \\
    TransVW & 95.67 & 52.10 & 73.88 & 0.60 & 21.36 & 10.98 & 95.55 & 0.41 \\
    UNetFormer* & 95.50 & 49.81 & 72.65 & {0.52} & 21.72 & 11.12 & 95.36 & 0.25 \\
    UNetFormer & 95.83 & 50.25 & 73.04 & 0.43 & 18.66 & 9.55 & 95.59 & 0.30 \\
    
    \hline
    HybridMIM*(Swin) & 95.45 & 50.19 & 72.82 & 0.69 & \textbf{15.21} & \textbf{7.95} & 95.87 & 0.25\\
    HybridMIM*(UNet) & \textbf{96.35} & 52.38 & \textbf{74.36} & 0.59 & 19.98 & 10.28 & {95.94} & \textbf{0.20} \\
    \hline
    HybridMIM(Swin) & 95.86 & 50.45 & 73.16 & 0.42 & 17.36 & 8.89 & 95.97 & 0.20 \\
    HybridMIM(UNet) & 95.70 & \textbf{52.81} & 74.26 & \textbf{0.27} & 18.25 & 9.26 & \textbf{96.05} & \textbf{0.20} \\
    \hline 
    \end{tabular}
    }
    \vspace{-2mm}
\end{table}




\subsection{Quantitative Comparison to Previous Methods} 
%
\textbf{BTCV multi-organ segmentation.} The multi-organ segmentation results are listed in Table \ref{tab:btcv_segmentation}, in which
the first, second, and third best dice scores are marked in red, blue, and green colors, respectively. 
%%
Among the comparative methods, we can see that those with self-supervised pre-training generally achieve averagely better results than those fully supervised methods. 
%%
TransVW obtains the best average Dice of 82.27\%,  
%%
while for UNetFormer, its generic pre-trained model presents an average Dice of 82.44\%, outperforming the task-specific pre-trained model UNetFormer* by 0.26\%. 

% 与其他对比方法相比,我们的基于UNet和SwinTransformer架构的方法均取得了有竞争力的结果。红色,蓝色,绿色分别代表最高的dice得分,第二高的dice得分与第三高的dice得分。可以清楚的发现,基于SwinUNETR架构的任务特定模型Swin(HybridMIM)在7项指标中均位于前三名,实现了82.41%的Dice平均值。而基于UNet架构的通用预训练模型UNet*(HybridMIM),在4项指标中位于前两名,相比于其他方法实现了最高的平均Dice,83.00。在BTCV多器官分割任务中,通用预训练模型的性能均高于任务特定预训练模型。
In comparison, our methods on both UNet and SwinTransformer architectures outperform most SOTA methods, and the generic pre-trained models get better performance than their task-specific pre-trained counterparts.  
%%
Specifically, the generic pre-trained model HybridMIM(UNet) presents the highest average Dice of 83.00\%,
%We can find that the task-specific model Swin (HybridMIM) based on SwinUNETR architecture is in the top three in all seven metrics, achieving an 82.41\% Dice average. 
% 拿性能最好的UNet*(HybridMIM)来说,它实现了最高的83.0%的平均Dice,比表现较好的同样在通用数据集上预训练的UNetFormer*模型提升了0.56%。并且UNet*(HybridMIM)在13个分割目标中有9个目标的分割结果均优于UNetFormer*。
which is 0.56\% better than the best SOTA model UNetFormer, and outperforms it in 9 out of 13 segmentation targets.
% 并且基于SwinUNETR架构的任务特定预训练模型在Lag器官上分割效果明显优于其他对比方法,达到了68.47%的dice值,比第二名UNETR高出1.82%。而基于UNet架构的任务特定预训练模型在Gall器官上分割效果显著,达到了 the dice of 78.67%,而第二名UNetFormer与第三名Segresnet方法的dice均没有超过76%。
%
Furthermore, the task-specific pre-trained model HybridMIM*(Swin) segmented significantly better than the other methods on the Lag organ, reaching the Dice of 68.47\%, which is 1.82\% higher than the second place UNETR, while HybridMIM*(UNet) reports a significantly better result on the Gall organ, reaching a Dice of 78.67\%. 
%In comparison, neither the second-place UNetFormer nor the third-place Segresnet method had more than 76\% Dice.

% 肝脏与肝脏肿瘤分割结果被展示在表3的左侧。加粗字体表示最优的指标。可以清晰的看到,我们提出的基于UNet架构的任务特定预训练模型UNet(HybridMIM)在肝脏的分割上有最好的Dice of 96.35%,比第二名TransVW提升了0.68%。同时其在肝脏肿瘤的分割中达到了Dice of 52.38%,仅次于ModelGen方法的52.53%。此外,UNet(HybridMIM)也实现了两个分割指标的最好的平均Dice,为74.36,比第二名TransVW方法提升了0.48%。
\textbf{Liver and liver tmuor segmentation.} As shown in Table \ref{tab:msd_segmentation}, 
%The bolded font indicates the best metrics.
our task-specific pre-trained model HybridMIM*(UNet) achieves the best average Dice of 74.36\%, with an improvement of 0.48\% over the second-place TransVW method.
Furthermore, it reports the best Dice of 96.35\% for the segmentation of the liver, which is 0.68\% better than the second place TransVW; and obtains a Dice of 52.38\% in the segmentation of liver tumors, only slightly lower than the second place ModelGen method with 52.53\%. 
% 对于HD95分割指标,基于UNet(HybridMIM)在肝脏的分割中位于第二名,HD95结果为0.59,略高于UNetFormer方法的0.52。在肝脏肿瘤的分割中为第三名,HD95为19.98。
For the HD95 segmentation metric, the HybridMIM*(UNet) gets an average HD95 of 10.28, ranked in the third place.
%is in second place in the segmentation of the liver with an HD95 result of 0.59, slightly higher than the UNetFormer method of 0.52. It was in third place in the segmentation of liver tumors with an HD95 result of 19.98, and the average HD95 was also in third place.
% 同时,Swin(HybridMIM)总体来说在HD95指标上表现更好。其在肝脏肿瘤的分割上拥有最好的HD95,为15.21,并且其在肝脏与肝脏肿瘤两个分割目标上实现了最好的的平均HD95,为7.95,比第二名ModelGen方法降低了1.8。相比于没有经过预训练SwinUNETR方法,Swin(MP-SSL)有更加明显的提升。其在肝脏与肝脏肿瘤的平均Dice得分达到了72.82%,比SwinUNETR方法提升了2.17%。
%Meanwhile, the Swin(HybridMIM) performed better overall on HD95 metrics. 
%It achieves the best HD95 of 15.21 for liver tumor segmentation and the best average HD95 of 7.95 for liver and liver tumor segmentation targets, which is 1.8 lower than the ModelGen method in second place. 
In addition, compared to the SwinUNETR method without pre-training, both HybridMIM*(Swin) and HybridMIM(Swin) which employ SwinUNETR as the underlying architecture, have more significant improvements in all the metrics. 
%%
%HybridMIM*(Swin) and HybridMIM(Swin) get an average Dice score of 72.82\% and 73.16\%, 2.69\% and 3.03\% higher than the SwinUNETR method, respectively.



% 脾脏的分割结果被展示在表3的右侧。可以看到,基于UNet与SwinUNETR架构的HybridMIM均表现出了优秀的性能,无论是在Dice还是在HD95上。基于UNet*(HybridMIM)获得了 state-of-the-art 的Dice与HD95,分别为96.05与0.20,在Dice得分上相比于同样表现较好的对比方法TransVW提升了0.50%,比基于Transformer架构的UNETR提升了2.1%。此外,Swin*(HybridMIM)实现了95.97%的Dice与0.20的HD95,仅次于UNet(HybridMIM)。
\textbf{Spleen segmentation.} The spleen segmentation results are listed on the right side of Table~\ref{tab:msd_segmentation}.
%%
The HybridMIM based on both UNet and SwinUNETR architectures presented improved performance, both on Dice and HD95. 
%%
HybridMIM(UNet) obtains Dice and HD95 with 96.05 and 0.20, respectively, improving the Dice score by 0.50\% compared to TransVW, and by 2.1\% compared to UNETR. 
%%
%In addition, Swin*(HybridMIM) achieves 95.97\% Dice and 0.20 HD95, second only to UNet (HybridMIM).
% 值得注意的是,SwinUNETR方法的Dice得分为94.61,而我们提出的通用预训练模型Swin* (HybridMIM)方法则达到了95.97的Dice得分,实现了1.36%的提升。通过我们提出的Hybrid的多层次自监督学习方式首先学习丰富的3D脾脏数据的空间解剖学特征,然后通过迁移学习在下游分割任务中训练,可以明显的提升原模型的效果。
Among the fully supervised methods, SwinUNETR gets the best Dice score of 94.61, and HD 0.25.
%%
Our generic pre-trained model HybridMIM(Swin) further improves SwinUNETR to achieve a Dice score of 95.97, realizing an increase of 1.36\%.
%%
%The original model can significantly improve by learning the spatial anatomical features of the rich 3D spleen data through our proposed Hybrid's multi-level self-supervised learning approach and then training it in the downstream segmentation task through transfer learning.

\begin{figure*}[tbp] %H为当前位置,!htb为忽略美学标准,htbp为浮动图形
\vspace{-4mm}
\centering %图片居中
\includegraphics[width=\textwidth]{figures/visual_1.pdf} %插入图片,[]中设置图片大小,{}中是图片文件名
% Ours为Swin*(HybridMIM)方法,三行视觉比较结果分别为BraTS2020,Liver和BTCV。我们提出的方法更够更好的分割细微的病灶(第一行),并且分割的完整度更高(第二行,第三行)。
\vspace{-3mm}
\caption{Qualitative visualizations of the proposed HybridMIM and baseline methods. "Ours" is the HybridMIM(Swin) method. The three rows of visual comparison results are from BraTS2020, Liver, and BTCV datasets. Our proposed method is better for segmenting tiny lesions (first row) and has higher segmentation integrity (second row, third row).} %最终文档中希望显示的图片标题
\label{fig:visual} %用于文内引用的标签
\end{figure*}

% 基于BraTS2020数据的脑胶质瘤的分割结果被展示在表4中。我们使用Dice来评测不同方法的性能。其中WT,TC,ET分别代表了全部肿瘤,肿瘤核心,增强肿瘤,Avg代表3个分割目标的Dice均值。
\textbf{Brain tumor segmentation.} The segmentation results of gliomas for BraTS2020 dataset are summarized in Table \ref{tab:brats_segmentation}. 
%We use Dice to evaluate the performance of different methods. 
WT, TC, ET represent whole tumor region, tumor core, and enhanced tumor region, respectively, and Avg is the Dice mean of the three segmentation targets.
% 我们提出的Swin(MP-SSL)方法实现了一个state-of-the-art的分割结果并且在WT,TC,ET三个分割目标中均达到了最优,分别为91.48%,86.88%,80.81%。相比于没有加入预训练的SwinUNETR方法,Swin(MP-SSL)在三个分割目标中均有较大幅度的提升,分别提升了1.4%,1.69%,0.8%,且三个分割目标的平均Dice得分比第二名TransVW方法提升了0.59%。
Our task-specific pre-trained model HybridMIM*(Swin) reports the best in WT, ET, and Avg with 91.48\%, 80.81\%, and 86.39\% respectively.
% 对比没有预训练的SwinUNETR方法,Swin(HybridMIM)与Swin* (HybridMIM)在三个分割目标上均有较大的提升,相比SwinUNETR,平均的Dice分别提升了1.3%, 1.24%。
%Compared with the SwinUNETR method without pre-training, Swin(HybridMIM) and Swin* (HybridMIM) show a considerable improvement in all three segmentation objectives, with an average Dice improvement of 1.3\%, 1.24\%, respectively, compared to SwinUNETR.
% 此外,UNet方法经过预训练后,也有了非常明显的提升,像表中最后一行展示的那样,UNet* (HybridMIM)方法在三个分割目标7分别实现了90.41%, 86.49%, 80.61%的Dice得分,相比于同样为UNet架构的ModelGen,三个分割指标的平均Dice提升了0.12%。以上的结果充分证明了MP-SSL方法良好的迁移学习和模型泛化能力。
%%In addition, the UNet method shows a significant improvement after pre-training, as shown in the last row of the table. 
As for UNet as the underlying architecture, the generic pre-trained model HybridMIM(UNet) achieves Dice scores of 90.41\%, 86.49\%, and 80.61\% for the three segmentation targets, respectively. Compared with ModelGen which is also built on UNet, we has the average Dice improved by 0.12\%. 
%%
%The above results fully demonstrate the good transfer learning and model generalization ability of the HybridMIM method.
It is also noted that on BraTS2020 dataset, the task-specific pre-trained mode gets better performance than the generic pre-trained mode. 

\begin{table}[t]
    \centering
    % BraTS2020数据集包含四个模态,三个分割目标。我们选择UNet和SwinTransformer作为backbone,分别于有监督学习方法跟自监督学习方法对比,结果展示了UniLearn对不同架构的有效性。
    \caption{Quantitative comparison on BraTS 2020 dataset, which contains four modalities and three segmentation targets. }
    % \vspace{-3mm}
    \label{tab:brats_segmentation}
    \renewcommand\arraystretch{1.3}
    \setlength\tabcolsep{10pt}%调列距
    \resizebox{0.48\textwidth}{!}{
    \begin{tabular}{c | c c c c}
    \hline
    Methods & WT & TC & ET & Avg\\
    \hline
    SegresNet & 90.04 & 85.08 & 78.81 & 84.64 \\
    
    UNETR & 89.92 & 84.79 & 79.51 & 84.74\\
    SwinUNETR & 90.08 & 85.19 & 80.01 & 85.09\\
    \hline
    ModelGen & 90.60 & 86.59 & 79.95 & 85.71\\
    TransVW & 90.96 & 86.26 & 80.20 & 85.80 \\
    UNetFormer* & 90.93 & 86.17 & 79.97 & 85.69\\
    UNetFormer & 90.71 & 86.22 & 80.19 & 85.71\\
    \hline
    HybridMIM*(Swin) & \textbf{91.48} & {86.88} & \textbf{80.81} & \textbf{86.39} \\
    HybridMIM*(UNet) & 90.62 & 86.28 & 80.17 & 85.69\\
    \hline
    HybridMIM(Swin) & 90.95 & \textbf{87.34} & 80.71 & 86.33\\
    HybridMIM(UNet) & 90.41 & 86.49 & 80.61 & 85.83 \\
    \hline
    \end{tabular}
    }
    \vspace{-2mm}
\end{table}



\begin{figure}[htbp] %H为当前位置,!htb为忽略美学标准,htbp为浮动图形
\centering %图片居中
\vspace{-2mm}
\includegraphics[width=0.8\columnwidth]{figures/data_proportion.pdf} %插入图片,[]中设置图片大小,{}中是图片文件名
% 不同有标签数据规模对迁移学习结果的影响。我们分别选择了BraTS2020数据集中训练数据的10%,20%,40%,60%,80%,100%,验证在不同自监督学习方法的迁移学习能力。
\caption{Effect of different labeled data sizes on migration learning results. We selected 10\%, 20\%, 40\%, 60\%, 80\%, and 100\% of the training data in the BraTS2020 dataset to verify the transfer learning ability in different self-supervised learning methods.} %最终文档中希望显示的图片标题
\label{fig:data_proportion}
\vspace{-2mm}
%用于文内引用的标签
\end{figure}

\vspace{-2mm}
\subsection{Qualitative Comparison to Previous Methods}

% 为了更加直观的对比不同方法的分割结果,我们选择Swin*(HybirdMIM)和其他六个性能较好的对比方法在BraTS2020,Liver和BTCV数据集上进行视觉比较。
To compare the segmentation results of different methods more intuitively, we choose HybridMIM(Swin) and four comparative methods with better performance on the BraTS2020, Liver, and BTCV datasets for visual comparison.
% 像Fig. 6. 所展示的,Swin*(HybridMIM)能够提升病灶识别的准确度和完整度,并且针对细微的病灶依然可以高效的识别出来。模型经过HybridMIM方法预训练后,对局部区域的感知能力更强。
As shown in Figure~\ref{fig:visual}, HybridMIM(Swin) can improve the accuracy and completeness of lesion identification,  and still perceive subtle lesions. 
%The model is pre-trained by the HybridMIM method and better perceives localized regions.
%在Fig. 6. 的第一行,可以明显看出我们的方法相比于其他对比方法可以更加精准的分割微小的病灶。在Liver数据集中(Fig. 6.第二行),Swin*(HybridMIM)分割的完整性更高,没有出现像其他对比方法中的分割区域不连续的情况。同时,在BTCV数据集中的可视化结果中,我们的方法的分割结果包含的空洞更少,与其他对比方法相比,有较高的完整度。 
To be specific, for brain tumor in BraTS2020 (the first row of Figure~\ref{fig:visual}), our method segments the whole tumor with more accurate boundary, while the comparative methods all enlarge the tumor region. 
%%
In the liver segmentation task (the second row), we can clear see that the comparative methods generate obvious discontinuity in the segmented areas. Especially UNETR and SegResNet fail to detect the lower part of the liver, while the detected liver region from our method exhibits a clearly higher integrity. 
%%
For the BTCV dataset, TransVW, UNetFormer, SiwnUNETR generates small holes in stomach; ModelGen even is subjected to a much large missing detected part. In contrast, our segmentation result is more close to the ground truth.

\vspace{-2mm}
\subsection{Reduce Manual Labeling Efforts}
% 为了验证随着有标签数据比例逐渐降低,HybridMIM方法相比于其他自监督学习方法依然能保持良好的迁移学习能力,我们选择UNetFormer与TransVW作为对比方法,BraTS2020作为下游分割任务数据集,采用10%,20%,40%,60%,80%,100%的数据比例进行对比实验。
To evaluate the transfer learning ability with annotation scarcity challenge in medical imaging, we conduct the experiment of finetuning using a subset of BraTS2020 data.  
%%
Figure~\ref{fig:data_proportion} demonstrates the comparison results between HybridMIM(Swin), TransVW and UNetFormer. 
%%
%In order to verify that as the proportion of labeled data gradually decreases, the HybridMIM method still maintains good transfer learning ability. We choose UNetFormer and TransVW as the comparison methods and BraTS2020 as the downstream segmentation task dataset and use 10\%, 20\%, 40\%, 60\%, 80\%, and 100\% data proportions for comparison experiments.
% Fig. 4. 展示了减少有标签数据比例的实验结果。实验结果表明,当有标签数据比例降低至60%时,UNetFormer与TransVW方法在BraTS2020分割数据集上的迁移学习能力明显降低。而通过HybridMIM方法预训练的通用模型SwinUNETR在有标签数据比例为20%时依然能够实现0.825的平均Dice。
%Fig. \ref{data_proportion} shows the experimental results of reducing the proportion of labeled data.
It is clear that the generic pre-trained model HybridMIM(Swin) presents the best performance when using the same portion of labelled data.
%%
On employing 20\% labelled data, HybridMIM(Swin) already achieves an average Dice of 82.55\%, with 1.42\% and 3.17\% higher than UNetFormer and TransVW, respectively.  
%%
The Dice 85.24\% can be achieved by using HybridMIM(Swin) with 60\% labelled data, while UNetFormer requires about 80\% data and TransVW requires nearly 90\% data.
%%
%%On employing 40\% labelled data, HybridMIM(Swin) obtains an average Dice of ??, even higher than UNetFormer and TransVW employing 60\% labelled data. 
 
%The experimental results show that the transfer learning ability of UNetFormer and TransVW methods declined significantly on the BraTS2020 segmented dataset when reducing the proportion of labeled data to 60\%. In contrast, Swin, a generic model pre-trained by the HybridMIM method, still achieves an average Dice of 0.825 when the proportion of labeled data is 20\%.
% 此外,当有标签数据的比例相同时,Swin*(HybridMIM)较其他对比方法均有明显的性能优势。并且Swin*(HybridMIM)需要更少的数据便可以实现其他对比方法需要更多数据才能实现的性能,例如Swin*(HybridMIM)利用60%的有标签数据达到的迁移学习的性能,UNetFormer需要80%的数据,TransVW需要90%的数据。
%In addition, the HybridMIM(Swin) has a significant performance advantage over other comparison methods when the proportion of labeled data is the same. For example, the HybridMIM(Swin) achieves transfer learning performance with 60\% of labeled data, while UNetFormer requires 80\% of data and TransVW requires 90\% of data.

\vspace{-2mm}
\subsection{Pre-training Speed Comparison}
% 在自监督学习的过程中,由于无标签数据的数据量通常较大,因此训练速度是一个影响自监督学习方法的非常重要的因素。MP-SSL通过灵活的选择局部的一级区域重建来提升预训练速度。我们与其他的自监督学习方法进行对比,像图3(d)中展示的那样,我们分别列举了基于UNet与SwinTransformer架构的MP-SSL方法与其他自监督方法的时间消耗。
In self-supervised learning, the training speed is a notable factor to consider, because the unlabeled data scale is usually large especially in the generic training mode. 
%%
Figure~\ref{fig:pretraining_time} demonstrates the time consumption of those self-supervised methods in the pre-training stage on BraTS2020 dataset.
%%
%The HybridMIM enhances the pre-training speed by flexibly selecting local first-level region reconstruction. We compare with other self-supervised learning methods, as shown in Fig. \ref{pretraining_time}, and we enumerate the time consumption of the HybridMIM method based on UNet and SWinUNETR architectures, respectively, with other self-supervised methods.
% 值得注意的是,为了更加公平的进行对比,我们对比了每个自监督学习方法运行一步的平均时间消耗。其中一步内包含了前向传播,反向传播,更新参数,而不包含数据读取,数据预处理等时间消耗不确定的操作。
For a fair comparison, we count the average time of running one step for each method, which contains forward prediction, backward propagation, and updating network parameters, but does not include data reading and preprocessing operations.
%

\begin{figure}[htbp] %H为当前位置,!htb为忽略美学标准,htbp为浮动图形
\centering %图片居中
\vspace{-2mm}
\includegraphics[width=0.8\columnwidth]{figures/time-1.pdf} %插入图片,[]中设置图片大小,{}中是图片文件名
% 不同自监督学习方法预训练时间消耗对比。横坐标为不同自监督学习方法和不同重建大小的HybridMIM方法,128是全局重建大小,96是我们提出的局部重建方式。纵坐标表示预训练时每步的时间消耗。
\caption{Comparison of pre-training time consumption for different SSL methods. 
%The horizontal coordinates are different self-supervised learning methods. 
``Not partial'' denotes that the partial region prediction scheme is not used.
%, which spend more time in pre-training. The vertical coordinate indicates the time consumption of each step during pre-training.
} %最终文档中希望显示的图片标题
\label{fig:pretraining_time} %用于文内引用的标签
\vspace{-2mm}
\end{figure}

% 因此,由图3(d)可以看出,TransVW与ModelGenesis方法时间消耗最多。Swin(HybridMIM)当使用(128,128,128)作为重构区域时,由于其包含更多的损失函数,因此时间消耗高于类似架构的UNetFormer方法。但是随着我们将需要重构的局部区域降低为(96,96,96),预训练时间大幅度降低,相比于TransVW与ModelGen方法,预训练速度提升48%,相比于UNetFormer方法,预训练速度提升36%。
As Figure \ref{fig:pretraining_time} shows, the TransVW and ModelGenesis methods with the same underlying architecture have the highest time consumption, both of which are 1.42s per step. 
%%
HybridMIM(Swin), when predicting all the masked sub-volumes (denoted as ``Not partial''; see the fourth bar), has a higher time consumption than the UNetFormer method. 
%%
It is because that although they have the similar underlying architecture, HybridMIM(Swin) involves  more loss functions. 
%%
On the other hand, when we apply the partial region prediction, the pre-training time of HybridMIM(Swin) decreases dramatically, in which the speedup is 48\% with respect to TransVW and ModelGen, and 36\% against UNetFormer.


% 类似的,当使用UNet(HybridMIM)方法时,此时虽然由于所使用的UNet本身的结构特殊性,有更低时间消耗,但通过选择局部区域重建,训练速度依然有显著的提升。像表3d中展示的那样,当使用(128,128,128)大小作为重构尺寸时,每步时间消耗为1.03s,而当使用(96,96,96)大小时,每步时间消耗降低0.35s,相比TransVW和ModelGen方法,预训练速度快52%,相比UNetFormer方法,预训练速度加快40%。
When using the HybridMIM(UNet) method, there is a lower time consumption due to the structural simplicity of the UNet (see the rightmost two bars). 
%%
The partial region prediction enables it to get a significant improvement in the pre-training speed, with the time consumption per step reduced by 0.35s.
%%
HybridMIM(UNet) achieves a pre-training  speed of 0.68s, 52\% faster than the TransVW and ModelsGenesis methods, and 40\% faster than the UNetFormer method.
%%
% It is worthy noting that the pre-training speed is close to the training speed in the finetuning, despite that the later has fewer losses to compute.
% %%
% Therefore, our method can also have faster time performance in the finetuning stage.



\vspace{-2mm}
\subsection{Ablation Study}
\subsubsection{Selection of the optimal architecture settings}
%\vspace{-4m}
\begin{figure}[htbp] %H为当前位置,!htb为忽略美学标准,htbp为浮动图形
\vspace{-4mm}
\centering %图片居中
\includegraphics[width=0.48\textwidth]{figures/architecture_3.pdf} %插入图片,[]中设置图片大小,{}中是图片文件名
% 不同架构参数对迁移学习性能与预训练时间的影响。(a)中横坐标中a-b-c分别代表一级区域大小,二级区域大小,重建区域大小。纵坐标表示在BraTS2020数据集迁移学习能力(三个分割目标的Dice平均值)。(b)中右侧纵坐标表示预训练时每个step消耗的时间。我们首先通过(a)确定最优的一级区域与二级区域,32-16-128迁移学习效果最好。之后,我们通过(b)改变重建区域的大小,兼顾性能与时间选择最优的架构参数设置。
\caption{Effect of different architecture parameters on transfer learning performance and pre-training time. The a-b-c in the horizontal coordinates in (a) represent the first-level, second-level, and reconstructed region sizes, respectively. The vertical coordinates represent the transfer learning capability in the BraTS2020 dataset (average Dice for the three segmentation targets). The right vertical coordinate in (b) indicates the time consumed per step during pre-training. The two red dashed boxes indicate the optimal architectural parameters we choose in (a) and (b), respectively. } %最终文档中希望显示的图片标题
%% 两个红色虚线框分别表示了我们在(a)和(b)中选择的最优架构参数。
%% We determine the optimal first-level and second-level regions by (a), and 32-16-128 migration learning works best. After that, we change the size of the reconstructed region by (b) choosing the optimal architecture parameter settings considering the performance and time.
\label{fig:pretraining_setting}
\vspace{-2mm}
%用于文内引用的标签
\end{figure}
%\label{pretraining_settings}

% 为了选择一个更好的架构参数,我们进行了多组对照实验。我们选择UNet架构预训练多组通用模型,see Fi. 3. 横坐标架构设置a-b-c中,a表示一级区域的大小,b表示二级区域的大小,c表示重建大小。纵坐标为通用模型在BraTS2020数据集中finetuning的Dice指标。

%%
In order to choose an optimal architecture setting, we conduct a multigroup control experiment. 
%%
We choose the UNet architecture to pre-train the possible settings (see Figure~\ref{fig:pretraining_setting}), where the three numbers under each bar represent the first-level sub-volume size, the second-level patch size, and the region size for partial region prediction.  
%%
The left vertical coordinates are the Dice metrics of finetuning the generic pre-trained model on the BraTS2020 dataset.
% Fig. 3. (b)中右侧纵坐标为每个step的时间消耗。
The right vertical coordinate in Figure~\ref{fig:pretraining_setting} (b) is the time consumption of each pre-training step.
% 像Fig. 3.(a)中所展示的那样,我们固定预训练的重构大小为128,选取了64-32,64-16,32-16,32-8四组参数预训练通用模型,之后在BraTS2020分割任务中进行finetuning,结果显示,32-16-128的参数设置表现最好,实现了最好的Dice。

As Figure~\ref{fig:pretraining_setting} (a) shows, we first fix the region size for partial region prediction to be 128, select four sets of parameters (64-32, 64-16, 32-16, and 32-8) for sub-volume and patch sizes.
%, and later perform finetuning in the BraTS2020 segmentation task. 
The results show that the parameter setting of 32-16-128 performs the best and achieves the best Dice of 85.79\%.

% 之后,我们选择32-16参数设置,逐步减小重构大小,see Fig. 3. (b),实验结果展示,重建大小由128降低到96时,每个step的时间由1.03s降低至0.68s。下游分割任务的Dice指标由0.875降低至0.860。当重建大小继续降低至64时,每个step的时间为0.50s,Dice指标为85.38。为了实现更快的预训练速度并使性能影响降低,我们选择32-16-96作为我们的架构参数设置。^^
Afterwards, we fix the optimal sub-volume and patch sizes (32-16), and gradually decrease the reconstruction region size; see Figure~\ref{fig:pretraining_setting} (b). 
%%
We can see that with a smaller reconstruction region size, the Dice score decreases a little bit, while the time performance reduces greatly. 
%%
For instance, when reducing the reconstruction size from 128 to 96, the Dice score for the downstream segmentation task decreases from 85.79\% to 85.57\%, and the time per step decreases from 1.03s to 0.68s. 
%when the reconstruction size decreases to 64, the time per step is 0.50s, and the Dice metric is 85.38. The Dice metric is 85.38 for 0.50s. 
Considering the trade-off between the segmentation accuracy and pre-training speed, we choose 32-16-96 as our architecture parameters for the case that the input sample has a size of $128\times128\times128$ (BraTS2020 dataset).
%%
Taking this experiments as guidance, we use an architectural parameter setting of 32-16-64 for the case that the input sample has a size of $96\times96\times96$ (BTCV, MSD Liver and MSD Spleen).

% 我们分别使用了UNet与SwinTransformer作为backbone,在BraTS2020数据集上通过消融实验充分的验证了我们提出的每个模块的有效性。实验结果被展示在表5中。Loss单元格包含五个不同的损失函数,分别为LR(local reconstruction), Num(number), Loc(location), Consis(consistency), CL(contrastive learning),其中LR代表了像素层次的3D医学图像表征的学习,Num,Loc,Consis代表了区域层次的表征学习,而CL代表了样本层次的学习。我们验证了MP-SSL在不同层次上的自监督学习对下游分割任务的性能提升。
\subsubsection{Efficiency of Self-Supervised Objectives}

We comprehensively validate the effectiveness of our modules through ablation experiments on the BraTS2020 dataset. 
%%
The experimental results using the generic pre-training mode are presented in Table~\ref{tab:ablation}. 
%%
We have five loss functions, namely $\mathcal{L}_{\mathrm{PR}}$ (partial region prediction), $\mathcal{L}_{\mathrm{Num}}$ (number prediction), $\mathcal{L}_{\mathrm{Loc}}$ (location prediction), $\mathcal{L}_{\mathrm{Con}}$ (consistency between number and location prediction), and $\mathcal{L}_{\mathrm{CL}}$ (contrastive learning).
%%
$\mathcal{L}_{\mathrm{PR}}$ facilitates the learning of 3D medical image latent representations at the pixel level; the combination of $\mathcal{L}_{\mathrm{Num}}$, $\mathcal{L}_{\mathrm{Loc}}$, and $\mathcal{L}_{\mathrm{Con}}$ facilitates the learning at the region level; and $\mathcal{L}_{\mathrm{CL}}$ facilitates the learning at the sample level. 
%We validate the performance improvement of the HybridMIM method at different levels of self-supervised learning for downstream segmentation tasks.
% Segmentation Target表示BraTS2020数据集不同的分割目标,Avg代表三个分割目标的平均指标。
%Segmentation Target represents the different segmentation targets of the BraTS2020 dataset, and Avg represents the average metric of the three segmentation targets.
% 表格中每个backbone的第一行结果为基线,不进行预训练,而是直接在下游分割任务上进行训练。之后,我们在预训练过程中逐渐添加不同的损失函数,来验证我们提出的不同模块对不同网络架构的性能提升能力。
We make comparison to the baseline with supervised training from scratch on the BraTS2020 dataset (see the first row for each backbone). 
%After that, we gradually add different loss functions during the pre-training process to verify the performance improvement capability of our proposed different modules for different network architectures.

\begin{table}[th]
    \centering
    \vspace{-3mm}
    % 在BraTS2020数据集上进行消融实验。我们选择UNet与SwinTransformer作为backbone,逐个添加我们提出的不同层次的损失函数。其中LR为局部重建损失,Num为数量分布预测损失,Loc为位置分布预测损失,Consis为一致性损失,CL为对比学习损失。下游任务的分割结果展示了我们提出的每个损失函数对于不同架构的有效性。
    \caption{Ablation experiments are performed on the BraTS2020 dataset. $\mathcal{L}_{\mathrm{LR}}$: the local reconstruction loss, $\mathcal{L}_{\mathrm{Num}}$: the number distribution prediction loss, $\mathcal{L}_{\mathrm{Loc}}$: the location distribution prediction loss, $\mathcal{L}_{\mathrm{Con}}$: the consistency loss, $\mathcal{L}_{\mathrm{CL}}$: the contrastive learning loss. }
    %The segmentation results of the downstream task demonstrate the effectiveness of each of our proposed loss functions for different architectures.
    % \vspace{-3mm}
    \label{tab:ablation}
    \renewcommand\arraystretch{1.2}
    \setlength\tabcolsep{5pt}%调列距
    \resizebox{\columnwidth}{!}{
    \begin{tabular}{c| l | c c c c c c}

    \hline
    \multirow{2}*{\makecell{Backbone}} & \multirow{2}*{Loss} & \multicolumn{4}{c}{Segmentation Target} & \\
    % \cline{3-7} \cline{10-13}
     & &  WT & TC & ET & Avg & \\
    \hline
    %% LR & Num & Loc & Consis & CL
    \multirow{6}{*}{UNet} & Supervised learning & 89.75 & 84.65 & 78.83 & 84.41 &\\
     & $\mathcal{L}_{\mathrm{PR}}$ & 90.19 & 85.50 & 79.48 & 85.06 & \\
     & $\mathcal{L}_{\mathrm{PR}} + \mathcal{L}_{\mathrm{Num}}$ & 90.05 & 85.48 & 79.97 & 85.17 & \\
     & $\mathcal{L}_{\mathrm{PR}} + \mathcal{L}_{\mathrm{Num}} + \mathcal{L}_{\mathrm{Loc}}$ & 90.15 & 85.65 & 80.10 & 85.30 & \\
     & $\mathcal{L}_{\mathrm{PR}} + \mathcal{L}_{\mathrm{Num}} + \mathcal{L}_{\mathrm{Loc}} + \mathcal{L}_{\mathrm{Con}}$ & 90.30 & 85.36 & \textbf{80.56} & 85.40 & \\
     & $\mathcal{L}_{\mathrm{PR}} + \mathcal{L}_{\mathrm{Num}} + \mathcal{L}_{\mathrm{Loc}} + \mathcal{L}_{\mathrm{Con}} + \mathcal{L}_{\mathrm{CL}}$ & \textbf{90.62} & \textbf{86.28} & {80.17} & \textbf{85.69} & \\
     \hline
     
     \multirow{6}{*}{Swin} & Supervised learning & 90.08 & 85.19 & 80.01 & 85.09 &\\
     & $\mathcal{L}_{\mathrm{PR}}$ & 90.95 & 86.17 & 80.22 & 85.78 & \\
     & $\mathcal{L}_{\mathrm{PR}} + \mathcal{L}_{\mathrm{Num}}$ & 90.93 & 86.94 & 80.48 & 86.12 & \\
     & $\mathcal{L}_{\mathrm{PR}} + \mathcal{L}_{\mathrm{Num}} + \mathcal{L}_{\mathrm{Loc}}$ & 91.18 & 86.33 & \textbf{81.10} & 86.20 & \\
     & $\mathcal{L}_{\mathrm{PR}} + \mathcal{L}_{\mathrm{Num}} + \mathcal{L}_{\mathrm{Loc}}  + \mathcal{L}_{\mathrm{Con}}$ & 90.98 & \textbf{87.06} & 80.71 & 86.24 & \\
     & $\mathcal{L}_{\mathrm{PR}} + \mathcal{L}_{\mathrm{Num}} + \mathcal{L}_{\mathrm{Loc}}  + \mathcal{L}_{\mathrm{Con}} + \mathcal{L}_{\mathrm{CL}}$ & \textbf{91.48} & {86.88} & {80.81} & \textbf{86.39} & \\
    %  \hline

     
    % \multirow{4}{*}{Swin} & & &  & & & & & & 90.08 & 85.19 & 80.01 & 85.09 & \\
    %  & & \checkmark & & & &  & & & 90.95 & 86.17 & 80.22 & 85.78 & \\
    %  & & \checkmark & \checkmark & & &  & & & 90.93 & 86.94 & 80.48 & 86.12 & \\
    %  & & \checkmark & \checkmark & \checkmark & &  & & & 91.18 & 86.33 & 81.10 & 86.20 & \\
    %  & & \checkmark & \checkmark & \checkmark & \checkmark &  & & & 90.98 & 87.06 & 80.71 & 86.24 & \\
    %  & & \checkmark & \checkmark &\checkmark & \checkmark & \checkmark & & & {91.48} & {86.88} & {80.81} & {86.39} & \\
    \hline
    \end{tabular}
    }
    \vspace{-2mm}
\end{table}




% 从表5中可以清晰的看出,当使用UNet架构在BraTS2020数据集上从零开始训练时,三个分割目标的Dice得分分别为89.75%, 84.65%, 78.83%, 平均值为84.41%。
\textbf{UNet architecture.} The baseline that is trained from scratch reports the Dice scores 89.75\%, 84.65\%, and 78.83\%, for the three segmentation targets respectively, with an average number of 84.41\%. 
% 此时加入第一个自监督学习损失LR(local reconstruction),该损失从像素层次来重建原图像被掩蔽区域的分布。在下游分割任务上加载由LR损失预训练得到的模型权重,使得每项分割目标均有不同程度的提升,平均值达到85.06%,较从零开始训练提升了0.65%。
At this point, we add the first self-supervised learning loss $\mathcal{L}_{\mathrm{PR}}$, which reconstructs the masked regions of the original image at the pixel level. The model weights fine-tuned onto the downstream segmentation task, result in a Dice average of 85.06\%, with an improvement of 0.65\% over the baseline.
% 之后,添加区域层次的自监督损失Num(number),Loc(location),Consis(consistency),提升模型表征空间区域分布的能力,分割目标的均值由85.06%提升至85.40%。
The addition of region-perception losses, i.e. $\mathcal{L}_{\mathrm{Num}}$, $\mathcal{L}_{\mathrm{Loc}}$, $\mathcal{L}_{\mathrm{Con}}$, improves the model's ability to characterize the distribution of spatial regions, and the mean Dice value is increased from 85.06\% to 85.40\%, getting an improvement of 0.34\%.
% 最后,添加样本层次的自监督损失CL(contrastive learning),提升模型对于不同样本表征的区分能力。通过CL损失,在下游分割任务中,三个分割目标的Dice得分均值达到了85.69%,并且在WT与TC上的Dice得分也达到了最高,分别为90.62%和86.28%。
Finally, we add the sample-level self-supervised loss $\mathcal{L}_{\mathrm{CL}}$ to enhance the model's ability to distinguish between different sample representations. With $\mathcal{L}_{\mathrm{CL}}$, the mean Dice score reaches 85.69\% in the downstream segmentation task, and the highest Dice scores of 90.62\% and 86.28\% on WT and TC, respectively. 
%%
In the end, the average Dice score with pre-training was 1.29\% higher than that without pre-training.

% 类似的,MP-SSL方法对于SwinTransformer架构也有较大程度的提升。三个分割目标的平均Dice得分由没有预训练时候的85.09%最终提升到了86.39%,在BraTS2020数据集上实现了SOTA的分割结果。
\textbf{SwinUNETR architecture.} Similarly, the HybridMIM method also achieves obvious improvements for the SwinUNETR architecture. The average Dice score of the three segmentation targets was finally improved from 85.09\% without pre-training to 86.39\%, achieving SOTA segmentation results on the BraTS2020 dataset. 

% \textbf{Analysis of self-supervised loss enhancement effects.} 对于UNet跟SwinTransformer架构,从表中可以看出,LR损失发挥了比较大的作用。UNet架构加入LR损失后,三个分割指标的平均Dice得分提升了0.65%,而SwinTransformer架构加入LR损失后,三个指标的平均Dice得分提升了0.69%。
\textbf{Analysis of self-supervised loss enhancement effects.} 
For the UNet and SwinTransformer architectures, Table~\ref{tab:ablation} shows that the $\mathcal{L}_{\mathrm{PR}}$ plays a larger role. The average Dice score of the three segmentation targets increases by 0.65\% with the aid of $\mathcal{L}_{\mathrm{PR}}$ upon the UNet architecture, while the average Dice score of the three metrics increased by 0.69\% upon the SwinTransformer architecture.
% 此外Consis损失由于具有保持预测的数量与位置信息一致的作用,提升自监督学习的可解释性,因此其对于下游分割任务的提升较小。对于UNet架构,平均Dice得分提升了0.1%,而对于SwinTransformer结构,平均Dice提升了0.04%。
%%
The region perception losses ($\mathcal{L}_{\mathrm{Num}}$, $\mathcal{L}_{\mathrm{Loc}}$, $\mathcal{L}_{\mathrm{Con}}$ together) are the second important. 
%%
Also note that although the $\mathcal{L}_{\mathrm{Con}}$ has a relatively small improvement for the downstream segmentation task, it has a role in keeping the predicted quantity consistent with the location information, improving the interpretability of the self-supervised learning. 
%For the UNet architecture, the average Dice score increased by 0.1\%, while for the SwinTransformer structure, the average Dice increased by 0.04\%.


\section{Limitations}
While we present a study of multiple decoding settings, we generate all machine responses using the same transformers based model architecture. Thus, the presented work does not yet explore individual differences between different model architectures. Additionally, due to limited resources we were not able to collect large-scale human annotations across multiple corpora and acknowledge the same as part of future efforts.  

\section{Conclusion}
Throughout the paper, we first analyze the current evaluation methods for diffusion-based adversarial purification and then propose a recommendation for the reliable evaluation of the robustness of adversarial purification. We further investigate the influence of hyperparameters of the diffusion model on the robustness of the purification. Based on our analysis, we propose a new strategy to maximize the benefit of the purification methods.
\section{Acknowledgement}
We acknowledge Noah Maestre for his help on preparing the rendering setup of the animations. This work is partially supported by Meta Reality Lab.
\bibliographystyle{eg-alpha-doi} 
\bibliography{references}       

% biblatex with biber
% \printbibliography                

%-------------------------------------------------------------------------
% \newpage
\end{document}
