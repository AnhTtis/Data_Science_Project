\section{Simulation Setup}
We perform physics simulation using Mujoco~\cite{todorov2012mujoco} inspired by successful prior work in hand manipulation~\cite{garcia2020physics,andrychowicz2020learning}. 
% Considering related work \cite{garcia2020physics,andrychowicz2020learning} applying DRL to contact-rich motor control tasks, we choose to leverage the Mujoco physics simulator for our experimental environment~\cite{todorov2012mujoco}. %\alex{Should we explcitly reference mimic here?} \yuting{A sentence of overview}

\subsection{Motion Capture Sequences}

% I think we should omit this reference, since the tracking method is fully described in the DeepLabels paper:
%yuting: I added it back because it's not only how the data is captured, but also what the content is; it also signals that we didn't capture the data ourselves for this project, it's an open sourced dataset.

\begin{wrapfigure}{r}{0.40\columnwidth}
    \centering
    \vspace{-0.25cm}
    \includegraphics[width=0.40\columnwidth]{figures/mocap_paper_figure.png}
    \caption{A still frame from an input motion capture sequence.}
    \label{fig:mocap_example} 
    \vspace{-0.15in}
\end{wrapfigure}

We use the object manipulation motion capture dataset from Zhang et al.~\shortcite{zhang2021manipnet}. The data was captured at 120 frames per second with the Optitrack \cite{optitrack} system using reflective markers.  A total of 19 markers are placed on each hand: four on the thumb, three on each finger, and three more on the back of the hand. The Optitrack system outputs object motions as rigid bodies, and the finger motions are reconstructed using DeepLabels \cite{han2018online}.

% We apply the control policy at the same frequency as the motion, but run the physics simulation at 600 frames per second for more stable results.

\subsection{Hand and Object Models}

\begin{wrapfigure}{r}{0.40\columnwidth}
    \centering
    \vspace{-0.25cm}
    \includegraphics[width=0.40\columnwidth]{figures/hand_model.png}
    \caption{Hand model with $20$ rotational axes.}
    \label{fig:hand_model} 
    \vspace{-0.15in}
\end{wrapfigure}

The dataset comes with a hand model and object models. The hand model consists of $20$ degrees of freedom, and we model each finger segment as capsules (see \revised{Figure}~\ref{fig:hand_model}).  For efficient collision detection in the simulation, objects are modeled either as convex primitive shapes such as a cube or a cylinder, or approximated as a combination of convex primitives if they are nonconvex (e.g. torus and wine glass). \revised{For simplicity purpose, we only enable the collision between the hand and the object. Self collision within the hand are disabled during the manipulation training.}

%On each capsule, we also place a contact sensor of the same shape and slightly bigger size, which accumulates contact forces exerting on it at each control step.

%The motion capture data contains sequences of hand and object poses. Figure ~\ref{fig:mocap_stillframe} shows a still frame of the mocap of grasping a cube using the right hand. As illustrated in Figure \ref{fig:hand_model}, the skeleton of each hand is consist of $19$  capsule segments and $20$ revolute joints on its fingers. On each capsule segment, we place a capsule contact sensor covers that segment and will record the cumulative contact forces exerting on the segment.

% \begin{figure}[h]
%     \centering
%     \includegraphics[width=1.7in]{figures/hand_model.png}
%     \caption{Our hand model with $20$ rotational axes.}
%     \label{fig:hand_model} 
% \end{figure}


%     \caption{Example of simulated object.}
%     \label{fig:object_example} 
% \end{figure}
%\begin{figure}[h]
%    \centering
%    \includegraphics[width=2in]{figures/mocap_stillframe.png}
%    \caption{Motion Capture}
%    \label{fig:mocap_stillframe}
%\end{figure}

\subsection{Contact Parameters}
We find that the contact-rich nature of in-hand manipulation tasks demands greater care when choosing simulation parameters related to collision resolution. Mujoco formulates contacts as a constrained optimization problem rather than a linear complementary problem to control the compliance of contact surfaces. In all of our experiments, we choose \revised{Mujoco built-in contact} parameters that correspond to highly stiff surfaces. Specifically, we set the fixed constraint impedance to $0.95$. We also reduce the solver's time constant from $0.02$ to $0.005$, which is the inverse of the natural frequency times the damping ratio. Unfortunately, such stiff contacts increase the instability of the simulator. As a mitigation, we choose elliptic friction cones and the RK4 integrator, and run the simulation at $600$Hz. We additionally found slipping could be a common failure. To increase friction, we set the ratio of friction-to-normal contact impedance to $5$, and strengthen the friction forces with higher normal forces. For more details of the contact model, please refer to the official documentation~\cite{todorov2012mujoco}. We found the above settings empirically works over all the motion sequences we tried, allowing us to achieve effective dexterous manipulation without the fingers ``sinking'' into the object's surface or sliding over sharp edges.

% \subsection{Upper Body Animation}
% For the preview of the hand animation on a human character, we generate upper body motions using a kinematic approach. For the given hand motion trajectory, we solve an inverse kinematic problem to match the position at the end of the lower arms to the position of the roots of the hands at every frame. 
% \sehoon{Still not sure where is the best place for this paragraph.}
% We present a straight forward way to generate animation for a humanoid character. For a hand motion trajectory, we are solving for an inverse kinematic problem for an upper body humanoid character so that the position at the end of the lower arms are matching the position of the roots of the hands at every frame. 