% intro 


% Starter
% ideas quantify quantum chaos but! integrable systems can be 
% 
\section{Introduction}
The scrambling of quantum correlations is an ubiquitous phenomenon  across the physics of interacting many-body systems \cite{SwingleTutorial,Richter2022}. Its connection to quantum chaos  has been established in systems ranging from models for black holes  \cite{Maldacena_2016,Hayden_2007,Sekino:2008he,Kitaev2019} to realistic many-body systems such as the SYK-model \cite{standfordOTOC2021}, even comprising systems without a classical limit \cite{OTOCfermions}.
Due to the appealing connection with the powerful concepts of quantum chaos, Out-of-Time-Ordered Correlators (OTOCs) \cite{classic} 
% with their short-time exponential growth 
represent a major probe of scrambling and thus have received a swiftly increasing huge theoretical interest \cite{Maldacena_2016,SwingleTutorial} that has driven efforts for experimental proposals \cite{proposalExperimentOTOC} and realizations \cite{Garttner2017,NMR_OTOC}.

In systems with a semiclassical regime, fast scrambling is considered an unambiguous indicator of classical (mean-field) instabilities \cite{Richter2022}. As such, Bose-Hubbard systems, with their well understood and controlled classical (mean-field) limit and a large semiclassical region of state space, are prime models to study imprints of scrambling~\cite{dimerBH,Shen2017,Bohrdt2017}.
%\cite{Maldacena_2016,SwingleTutorial, Hummel2019, argentinians} 
% transition to the actual topic
Recently it was shown~\cite{Hummel2019,Papparlardi2018} that an initial exponential growth of OTOCs does not necessarily imply chaotic dynamics of the system's classical counterpart, {\em i.e.} such OTOC behavior alone cannot serve as clear-cut probe of quantum chaos. These works~\cite{Hummel2019,Papparlardi2018} and further ones picking up the same idea \cite{Scaffidi2020,Santos1,Santos2} show that for quantum (many-body) systems with a classical limit and a semiclassical regime it is sufficient to have local instabilities in a (possibly integrable) phase space to generate exponentially growing OTOCs. Several examples of this situation have been numerical studied \cite{Pilatowsky2020OTOCregularSystem,Kidd2021Dicke}, including basic models such as an inverted harmonic oscillator \cite{jhep11(2020)068}.

Generically, the prime example of the mechanism for an exponential OTOC growth in an integrable system in the existence of an unstable (hyperbolic) fixed point. Although, by definition, all Lyapunov exponents $\lambda_{\rm L}$ are zero in integrable systems, the classical dynamics around fixed points is locally hyperbolic if they have at least one positive stability exponent $\ls>0$. This type of instability will be considered here.
In the early time regime, defined up to a time scale depending logarithmically on the effective Planck constant $\hbarE$, the OTOCs involving dynamics around unstable points of many-body integrable systems display two markedly different behaviors. In Refs.~\cite{Hummel2019} and \cite{Scaffidi2020}, the quantum Lyapunov exponents $\lq$ quantifying the OTOC growth rate are compared to the stability exponents of dominant unstable fixed points of the corresponding classical mean-field dynamics yielding good agreement with~$2\ls$ \cite{Hummel2019} or $\ls$ \cite{Scaffidi2020}, respectively.

In this paper we resolve this apparent discrepancy with regard to the operator growth and provide a unified dynamical mechanism explaining the two results in a comprehensive way. We demonstrate that there is a universal $2\ls$ to $\ls$ transition for the OTOC growth rate $\lq$  for dynamics around unstable fixed points in the pre-Ehrenfest time regime which interpolates between these two limits.
Furthermore, the crossover develops a kink in the strictly classical limit $\hbarE \to 0$. 
Moreover, we show that this $2\ls$ to $\ls$ transition is related to an underlying dynamical transition of the initial quantum state in a phase-space representation and argue that the $2\ls$ to $\ls$ transition is a hallmark of integrable systems. 
We propose that this effect can thus be used to distinguish chaotic and integrable systems by properly analyzing the growth behavior of OTOCs at pre-Ehrenfest time scales. 

% contents
The paper is structured into three further sections. In Sec.~\ref{sec:OTOC_theory}, we present a heuristic argument for OTOCs to exhibit different exponential regimes and the corresponding $2\ls$ to $\ls$ transition. 
In order to verify the picture put forward, we perform an extensive study of the Bose-Hubbard dimer in Sec.~\ref{sec:boseHubbard}. There, we start with the definition of the dimer and a study of the classical mean-field system including an analytical study of the classical OTOC. 
Then, numerical results for the OTOCs follow obtained from extensive simulations that display excellent agreement with the classical results for the OTOCs. Furthermore, we show how one can tune the $2\ls$ to $\ls$-transition, and study its robustness with regard to changing the system parameters. 
In Sec.~\ref{sec:conclusion} we summarize our findings and discuss their possible extension to non-integrable systems.

