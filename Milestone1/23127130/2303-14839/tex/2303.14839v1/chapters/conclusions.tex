\section{Conclusion \& Outlook}

\label{sec:conclusion}
% Conclusion
\subsection{Conclusion -- integrable systems}
A localized wavepacket around a hyperbolic fixed point in a quantum system with integrable classical (mean-field) limit undergoes a transition between different dynamical regimes driven by a leaking mechanism of phase space volume along classical separatrices. If located within the pre-Ehrenfest time scale, this dynamical transition imprints a characteristic kink structure to the scrambling as measured by the exponential form of out-of-time-order correlators. Specifically, the exponential growth changes from $2\ls $ to $\ls$ and the kink develops for $\hbarE\to 0$ where $\ls$ is the stability exponent of the fixed point. We have derived an analytical theory and shown how  this behavior is actually directly related to the classical limit of the out-of-time-order correlators when their time dependence is governed by the separatrix dynamics emerging around an unstable FP.  

Following this picture, we have shown that squeezing the initial coherent state allows us to engineer the leaking time and thus the dynamical transition itself exactly as predicted by our analytical considerations. 

If the phase-space localization scale of the initial state is strong enough, the leaking time is beyond the Ehrenfest time and we obtain the standard $2\ls$ exponent. In contrast, an uniform state starts to leak immediately and even before the ergodic time. Therefore, the infinite temperature OTOC growths only with the reduced exponent~$\ls$.

We have supported our picture of the dynamical transition by means of extensive simulations on the experimentally accessible, and integrable,  Bose-Hubbard dimer. The extremely clean fixed-point and separatrix structure of this systems allows us for a detailed study of the mechanism and the   analytical expectations of Sec.~\ref{sec:OTOC_theory} are verified to an excellent degree.  
%For the sake of completeness, the infinite temperature OTOC is in the appendix \ref{sec:appendixInfTemp} following the result from \cite{PhysRevLett.124.140602}. 

\subsection{Outlook -- beyond integrability}
In order to focus on the sepratrix effects like the leaking mechanism, a very well controlled classical phase space is required, so our numerical findings and the corresponding analytical theory has been restricted so far to an integrable system where the theory in Sec.~\ref{sec:OTOC_theory} assumes a bounded linearized regime around the fixed point. In order to transfer this concepts to the realm of chaotic systems, a straightforward generalization is to assume a bounded Lyapunov zone inside a chaotic region, i.e., a finite deviation $\delta x_{0}$ grows only for a finite time like~$\delta x(t) \sim e^{\lL t}\delta x_{0}$.  Following this idea, the role of the stability exponent~$\ls$ from the fixed point is now translated to the Lyapunov exponent~$\lambda_{\rm L}$ of the chaotic sea. 
An open question is if a wave-packet is actually sensitive to leaking out of the Lyapunov zone. Unlike the presented integrable case, the source of instability is the non-local whole mixing chaotic sea conflicting the idea of a leaking wave-packet, a question that is now under investigation. 

Lacking of a full-fledged analytical approach for chaotic systems, one option is to turn to an effective type of description. A first numerical exploration of this matter can be found in \cite{meier2023signatures}, where the authors treat  the combination of unstable fixed points and chaotic layers investigating the role of the stability exponent $\ls$ versus the Lyapunov exponent $\lambda_{\rm L}$ for pre-Ehrenfest time scales. 
% As a first step, we calculate OTOCs for a number-projected coherent state at the hom. FP in the trimer $L=3$ and scan the parameter range for the first and second window in Fig.~\ref{fig:parameterScanTrimer}.% %Trimer example
% \begin{figure}[h!]
%     \centering
%     \includegraphics[width=\linewidth]{figures/L3/homogeneousL3_homogeneous_fitShort.pdf}\\
%         \includegraphics[width=\linewidth]{figures/L3/homogeneousL3_homogeneous_fitLong.pdf}
%     %\includegraphics[width=0.8\linewidth]{figures/L3/homogeneousL3_N[10, 100, 200]_firstWindow.pdf}\\
%     %\includegraphics[width=0.8\linewidth]{figures/L3/homogeneousL3_N[10, 100, 200]_secondWindow.pdf}
%     \caption{Parameter scan for the trimer: $\hbarE$ is order of magnitudes smaller compared to dimer; $2\ls -\ls$ tendency is recognizable similar to dimer; the Lyapunov exponent $\lambda_{\rm L}$ is small compared to the stability exponent $\ls$}
%     \label{fig:parameterScanTrimer}
% \end{figure}
% Even at much lower $\hbarE$, we recognize a trend to towards to the $2\ls -\ls$ transition, where we compared the fitting to the stability exponents of the FP. 
% %The Lyapunov exponent is not included, but since it is by roughly a magnitude of order smaller, we can not recognize a contribution to the OTOC. 

% A possible explanation is that the trimer is still 'too integrable', i.e. , $\lambda_{\rm L} \ll\ls$, or this dynamical transition appears also in the chaotic case.
In such a situation, it is an open and thrilling question whether the $2\ls-\ls$ transition could distinguish a chaotic from an integrable system. In the same vein, we may ask the immediate question, whether this transition can be generalized in the mixed dynamics, as a series of different exponential growth regions given by several different local Ehrenfest times.

%Another direction staying within the dimer is to introduce a time-dependency and tune  the dimer to be chaotic \cite{dimerTimeDepOTOC}