


\section{Bose-Hubbard Dimer}
\label{sec:boseHubbard}
The Bose-Hubbard dimer describes bosonic degrees of freedom occupying on two discrete levels or sites. Prime physical setups are individual Josephson-junctions~\cite{superconductorJosephson2001} or cold atoms within a small two-sited optical lattice~\cite{Albiez2005,F_lling_2007,Witthaut2008,cheinet2008,Kierig2008,Tomkovi2017}. In all these cases, one ends with an effective description in terms of the following Hamiltonian
\begin{align}
    \label{eq:DefQuantumDimer}
\hat{H} = -2 J \big( \had{2}\ha{1}+\ha{2}\had{1} \big) + \frac{g}{2} \big( \had{1}{}^{2} \ha{1}^{2}  +\had{2}{}^{2} \ha{2}^{2}   \big),
\end{align}

where the parameter $J$ is the hopping and $g$ is the (local) interaction strength between particles given in units of energy. Our Hamiltonian differs from the usual dimer Hamiltonian, the hopping coefficient $2J$ (instead of $J$) is motivated to be consistent with a ring topology for higher number of wells. Consequently, the two-site ring has a doubled counted hopping term. We also introduce a new  dimensionless parameter~$\Theta$ with $J=\epsilon_{0}\cos \Theta$ and $g = \epsilon_{0}\frac{2}{N} \sin \Theta$. Using $\Theta$ fixes the scale of the parameters to $\epsilon_{0} = \sqrt{J^{2} + \big(\frac{gN}{2}\big)^{2}}$ in unites of an energy scale $\epsilon_{0}$. We set $\epsilon_{0}=[1]$ for a convenient unit system, that also renders the time unit $\hbar/\epsilon_{0}=1$, thus compactifying the parameter space to $\Theta \in [-\frac{\pi}{2},\frac{\pi}{2}]$.
Furthermore, this parametrization of $J$ and $g$ makes the spectrum linearly in the particle number $N$ in leading order. %\rc{Sorry man, you must introduce here physical units, as now J and g appear dimensionless}. 

\subsection{Classical mean-field limit}
We follow the standard approach \cite{negele1995quantum} to derive the classical limit for bosons and replace the operators by complex numbers 
\begin{align*}
    \ha{j}, \had{j} ~\longmapsto&~ \psi_{j},\psi^{\ast}_{j}
\end{align*}
inside the normal-ordered quantum Hamiltonian Eq.~\eqref{eq:DefQuantumDimer} to get a classical mean-field system. 
Hamilton's equations of motion $i \dot{\psi}_{j} = \frac{\partial H}{\partial \psi_{j}^{\ast}}$ define the classical dynamics.
Due to the conserved total particle number $N$, we define new set of conjugated classical variables
\begin{align*}
    \begin{matrix*}[l]
        N  =  n_1 + n_2 \,, &\phi  =  \frac{1}{2}( \varphi_1 + \varphi_2 ) \,, \\
        n  =  \frac{1}{2}( n_1 - n_2) \,, &\varphi  =  \varphi_1 - \varphi_2 - \pi \,,
    \end{matrix*}
\end{align*}
where the two mean fields $\psi_{j}=\sqrt{n_{j}}e^{i\varphi_{j}}$ are written in phase~$\varphi_{j}$ and occupations~$n_{j}$. Hence, the Hamiltonian takes the form
\begin{align}
    \begin{split}
        H(N&,\phi, n,\varphi) =
        \\
        =&\, 2\cos{\Theta} \sqrt{N^2 - 4 n^2} \cos \varphi +\sin\Theta \Big( \frac{2n^2 }{N} +  \frac{N}{2} \Big)\,.
    \end{split}
    \label{eq:classicalHamiltonian}
\end{align}
We can reduce the dynamics to an 1d-system with a single conjugated pair $(z=\frac{2n}{N},\varphi= \varphi_{1}-\varphi_{2}-\pi)$ given by the population inversion and relative phase~\cite{Campbell2020Dimer}. 
With these coordinates, we can reduce the equations of motion to two coupled real-valued ODEs
\begin{align}
    \begin{split}
        \dot{z} &= -4\cos\Theta \sqrt{1-z^{2}} \sin \varphi,
        \\
        \dot{\varphi} &=  4\cos\Theta \frac{z\cos\varphi}{\sqrt{1-z^{2}}} - 2\sin \Theta \, z,
    \end{split}
    \label{eq:EquationOfMotion}
\end{align}
%{\color{red} forgot hopping $J$ I think, multiply everything by $J= \cos(\Theta)$, to check!}
which is an 1-degree of freedom system. 
Conveniently, the mean-field system Eq. \eqref{eq:EquationOfMotion} is exactly solvable and  independent of the  particle number $N$. 


\subsection{Fixed points}

A straightforward calculation shows that there are two fixed points $(z=1$, $\varphi=\pi)$ and $(z=1$, $\varphi=0)$ which are independent from the system parameter $\Theta$. 
We call these two FPs the hom. and antihom. FP, since both have homogeneous occupations and a zero or $\pi$ phase difference between site one and two. We set the zero point for the relative phase $\varphi$ to the unstable antihom FP. 


Two bifurcations appear: at $\Theta = - \arctan 2$ for the hom. FP and at $\Theta =  \arctan 2$ for antihom. FP. 
We restrict ourselves only to the antihom. FP, since there is a symmetry between them under change of the sign of the parameter~$\Theta$.
%%%%%%%%%%%%5
 % need to check if stability analysis in the reduced form give the same diagram!!!!!
%%%%%%%%

\begin{figure}[h!]
    \centering
    \includegraphics[width=1\linewidth]{figures/L2/classical/antihomogeneousL2_antihomogeneous_stabilityExp.pdf}
    \caption{Stability exponent of the antihom. FP over the whole parameter space: change from unstable to stable at the bifurcation point $\Theta=\arctan 2$.}
    \label{fig:StablHomFP}
\end{figure}
The stability diagram of the antihom. FP in Fig. \ref{fig:StablHomFP} shows the bifurcation at $\arctan 2$, where the stability exponent becomes positive. Its maximum $\ls=0.97$ is reached at $\Theta_{\ast}\approx1.35$.

At the maximal unstable parameter $\Theta_{\ast}$, Fig.~\ref{fig:PhaseSpaceFP} shows the reduced phase space structure of the system Eq.~\eqref{eq:EquationOfMotion}. Note in particular 
the (red) separatrix defined by the unstable and stable manifold originating from the antihom. FP. The merging of stable and unstable manifolds indicates that the linearized regime is bounded, namely, any classical trajectory on the unstable manifolds converges to the stable manifold and (in infinite time) to the hyperbolic FP.
The exact size of the linearized regime (modeled by the constant $c$ in the previous Sec.~\ref{sec:OTOC_theory}) plays a neglectable role for $\hbarE\to 0$, since it is additive and $\hbarE$-independent constant in the Ehrenfest time $\tEhr$. Therefore we leave $c$ undefined in the subsequent discussion.

\begin{figure}[h!]
    \centering
    \includegraphics[width=1\linewidth]{figures/L2/classical/phase_space-dimer_piPhase_Theta1.351.pdf}
    \caption{Reduced phase space structure ($z,\varphi)$ for $\Theta_{\ast}$: contour lines of the Hamiltonian correspond to the classical trajectories, there are three stable fixed and one hyperbolic fixed point with its red dashed separatrix. Arrows on the separatrix indicate the stable and unstable manifolds.}
    \label{fig:PhaseSpaceFP}
\end{figure}


We verified that, as shown in  Fig.~\ref{fig:PhaseSpaceFP} for $\Theta_{\ast}$, we verify that this is the only unstable FP in the classical mean-field limit. This is also true for the whole range $\Theta\in (\arctan 2,\pi/2)$, where the antihom. FP is unstable.

Armed with this very specific phase-space structure, we carry an in-depth analytical study of the OTOC~$\mathbf{C}(t)$ for the dimer in the next section. 


\subsection{Microscopic approach: separatrix dynamics} 
\label{sec:peter}
\ifthenelse{1=1}
{}
{
In this section we provide a microscopic understanding of the dynamical transition by means of the exact separatrix dynamics of the mean field limit.

In this section, we work out an analytical expression for the classical OTOC
associated with a quantum state that is launched on the separatrix point of the
Bose-Hubbard dimer.
The latter is described by the quantum Hamiltonian
\begin{equation}
  \hat{H} = - J \left( \hat{a}_1^\dagger \hat{a}_2 + \hat{a}_2^\dagger \hat{a}_1
  \right) + \frac{U}{2} \sum_{l=1,2} \hat{a}_l^\dagger\hat{a}_l^\dagger
  \hat{a}_l\hat{a}_l
\end{equation}
with $\hat{a}_l^\dagger,\hat{a}_l$ the bosonic particle creation and
annihilation operator on the site $l=1,2$,
$J$ the hopping parameter, and $U$ the interaction parameter.

This quantum system has a classical counterpart which is given in terms of
the discrete nonlinear Schr\"odinger equation (setting $\hbar = 1$)
\begin{eqnarray}
  i \frac{d \psi_1}{d t} & = & - J \psi_2 + U |\psi_1|^2 \psi_1 \,, \\
  i \frac{d \psi_2}{d t} & = & - J \psi_1 + U |\psi_2|^2 \psi_2 \,,
\end{eqnarray}
with $\psi_l$ the classical field amplitude that is associated with the site
$l=1,2$.
Expressing $\psi_l = \sqrt{n_l}e^{i \varphi_l}$ in terms of the (real)
classical site occupancies $n_l$ and phases $\varphi_l$, and performing
the canonical transformation
$(n1,n2,\varphi_1,\varphi_2) \mapsto (N,n,\phi,\varphi)$
to new classical variables defined through
\begin{eqnarray}
  N & = & n_1 + n_2 \,, \\
  n & = & \frac{1}{2}( n_1 - n_2) \,, \\
  \phi & = & \frac{1}{2}( \varphi_1 + \varphi_2 ) \,, \\
  \varphi & = & \varphi_1 - \varphi_2 - \pi \,,
\end{eqnarray}
we end up with a classical Hamiltonian system described the inter-site
population exchange dynamics via the Hamiltonian
\begin{equation}
  H(n,\varphi) = U n^2 + J \sqrt{N^2 - 4 n^2} \cos \varphi \,,
  \label{eq:BH.Hc}
\end{equation}
which parametrically depends on the constant of motion $N$ corresponding
to the total population of the system.
The classical time evolution of the system is given in terms of the
Hamiltonian equations of motion
\begin{eqnarray}
  \frac{d n}{d t} & = & \frac{\partial H}{\partial \varphi}(n,\varphi) =
  - J \sqrt{N^2 - 4 n^2} \sin \varphi \,, \label{eq:BH.n} \\
  \frac{d \varphi}{d t} & = & - \frac{\partial H}{\partial n}(n,\varphi) =
  - 2 U n + \frac{4 J n \cos \varphi}{\sqrt{N^2 - 4 n^2}} \,, \label{eq:BH.ph}
\end{eqnarray}
Expressed in terms of the relative population imbalance $z = 2 n / N$,
the rescaled dimensionless time $\tau = J t$, and the dimensionless
nonlinearity parameter
\begin{equation}
  \gamma = \frac{N U}{2 J} \,,
\end{equation}
Eqs.~\eqref{eq:BH.n} and \eqref{eq:BH.ph} are rewritten as
\begin{eqnarray}
  \dot{z} \equiv \frac{d z}{d \tau} & = & - 2 \sqrt{1 - z^2} \sin\varphi \,,
  \label{eq:BH.z} \\
  \dot{\varphi} \equiv \frac{d \varphi}{d \tau} & = & - 2 \gamma z
  + \frac{2 z \cos\varphi}{\sqrt{1 - z^2}} \label{eq:BH.p} \,.
\end{eqnarray}
}
We set for the dimer the operators $\hat{A}=\hat{B}=\hat{n}_{1}$ to be the number operator $\hat{n}_{1}= \hat{a}^{\dagger}_{1} \hat{a}_{1}$ at the first site. Therefore, we get
\begin{align}
    \mathbf{C}(t) =\langle || [ \hat{n}_{1}(t), \hat{n}_{1} ] ||^2 \rangle = \langle || [ \hat{n}(t), \hat{n} ] ||^2 \rangle\, ,
    \label{eq:OTOC_defDimer}
\end{align} 
where $\hat{n} = \frac{1}{2}(\hat{n}_{1}-\hat{n}_{2})$. For this OTOC, we are interested in evaluating the classical expression which is
given by
\begin{equation}
  O(t) = \iint dn d\varphi\, W(n,\varphi)
  \left(\frac{\partial n_t}{\partial \varphi_0}\right)^2 \,, \label{eq:OTOCc}
\end{equation}
with $W(n,\varphi)$ the Wigner function associated with the initial state.
The latter is, for the sake of simplicity, modeled as a coherent quantum state
$e^{\sqrt{N_0}(\hat{a}_-^\dagger - \hat{a}_-)} \ket{0}$ with $\hat{a}_{-}=\hat{a}_{1} -\hat{a}_{2}$, keeping in mind that for large
$N_0$ this coherent state features very similar properties as a
number-projected coherent state with total particle number $N_0$ as far as the
site population exchange dynamics is concerned.
The Wigner function associated with this initial state would be given by
\begin{align}
    \begin{split}
        W(N&,\phi,n,\varphi) \simeq 
        \\
        \frac{1}{\pi^2} &\exp\left(-
        \frac{(N - N_0)^2}{2 N_0} - 2 N_0 \phi^2 - \frac{2 n^2}{N_0} -
        \frac{N_0 \varphi^2}{2} \right)
    \end{split}
\end{align}
  in the framework of a quadratic expansion valid for $N_0 \gg 1$
(using again $\hbar = 1$).
Since the Wigner function $W$ describes a tight localization of $N$ about $N_0$, we set
$N_0 = N$ henceforth and model the initial quantum state concerning the
inter-site population exchange dynamics by the Wigner function
\begin{equation}
  W(n,\varphi) = \frac{1}{\pi} \exp\left(- \frac{2 n^2}{\omega N}
  -  \frac{N \omega \varphi^2}{2} \right) \,,
\end{equation}
where the squeezing parameter $\omega$ allows for some flexibility
in the definition of the initial quantum state.

Let us first discuss the linearized dynamics in the near vicinity of
the FP~$(n,\varphi) = (0,0)$.
Linearizing Eq.~\eqref{eq:EquationOfMotion}, we obtain
the system of equations 
\begin{align}
\begin{split}
     \dot{z} & =  - 4\cos \Theta \varphi \,,  \\
  \dot{\varphi} & = - 2 (\sin \Theta -2\cos{\Theta}) z \,, 
\end{split}
\label{eq:BH.peterlinearized}
\end{align}
which is readily solved as
\begin{align}
    \begin{split}
      z_t & = z_0 \cosh \ls t - \frac{4\cos{\Theta} \varphi_0}{\ls}
      \sinh \ls t\,,  \\
      \varphi_t & = \varphi_0\cosh \ls t - \frac{\ls z_0}{4\cos{\Theta}}
      \sinh \ls t 
    \end{split}
\label{eq:BH.peterlinearizTime}
\end{align}
in terms of the stability exponent
\begin{equation}
  \ls = 4\cos{\Theta}\sqrt{\frac{\gamma}{2} - 1} \, , \label{eq:BH.la}
\end{equation}
where we defined the so called nonlinearity parameter $\gamma=\tan \Theta = \frac{gN}{LJ}$.
The latter becomes purely imaginary for $\gamma < 2$, which implies that
$(n,\varphi) = (0,0)$ turns into a stable fixed point if the nonlinearity
parameter $\gamma$ is decreased below two, as it is plotted in Fig.~\ref{fig:StablHomFP}.

Considering $\gamma > 2$ henceforth, and assuming that the point
$(z_0,\varphi_0)$ is located very closely to the origin in this phase space,
we can, as in the previous section, identify a time scale
$\tLoc \gg \ls^{-1}$ for which we still have $|z_{\tLoc}| \ll 1$ and
$|\varphi_{\tLoc}|\ll 1$, such that the above linearization Eq.~\eqref{eq:BH.peterlinearized} of the classical
equations of motion remains valid until $t = \tLoc$.
Since at the same time we have $\ls \tLoc \gg 1$ by assumption,
the solution Eq.~\eqref{eq:BH.peterlinearizTime} of the linearized
Eq.~\eqref{eq:BH.peterlinearized} for $t= \tLoc$
simplifies as
\begin{eqnarray}
  z_{\tLoc} & = & \left(\frac{z_0}{2} -  \frac{2\cos \Theta\varphi_0}{\ls}\right)
  e^{\ls \tLoc} \,, \label{eq:BH.za} \\
  \varphi_{\tLoc} & = & \left(\frac{\varphi_0}{2} - \frac{\ls z_0}{8 \cos \Theta}
  \right) e^{\ls \tLoc} \,.
\end{eqnarray}

From the time $\tLoc$ on, we can safely assume that the trajectory
under consideration very closely follows the separatrix structure emanating
from the unstable antihom. fixed point $(z,\varphi) =(0,0)$. 
This separatrix structure is obtained through the identification of the
energy
\begin{equation}
  H(n,\varphi) = 2\cos \Theta N + \sin{\Theta} \frac{N}{2}
\end{equation}
of the classical Hamiltonian Eq. \eqref{eq:classicalHamiltonian}, from which follows the identity
\begin{equation*}
  \cos\varphi = \frac{1 - \frac{\gamma}{4} z^2 }{\sqrt{1 - z^2}} \,.
\end{equation*}
Inserting this expression into Eq.~\eqref{eq:EquationOfMotion} yields the differential
equation
\begin{equation*}
  \dot{z} = z \sqrt{\ls^2 - \sin^2\Theta  z^2}
\end{equation*}
describing the motion along the upper or lower separatrix branch.
This equation is straightforwardly integrated yielding
\begin{align*}
  t - \tLoc =& \int_{z_{\tLoc}}^{z_t}
  \frac{\d z}{z \sqrt{\ls^2 - \sin^2\Theta z^2}}
  \\
   =& - \frac{1}{\ls}\left[ \mathrm{arcosh}
    \left(\frac{\ls}{\sin \Theta |z_{t}|} \right) - \mathrm{arcosh}
    \left(\frac{\ls}{\sin \Theta |z_{\tLoc}|} \right) \right] \,,
\end{align*}
from which we obtain
\begin{equation}
  z_t= \frac{\mathrm{sgn}(z_{\tLoc}) \ls / \sin\Theta}
  {\cosh\left[\mathrm{arcosh} \left(\frac{\ls}{\sin\Theta |z_{\tLoc}|}
      \right) - \ls (t - \tLoc) \right]} \,.
      \label{eq:zt}
\end{equation}
Using $|z_{\tLoc}| \ll 1$ and hence also $\sin\Theta |z_{\tLoc}|/\ls \ll 1$
for finite values of $\sin\Theta$ and $\ls$, we define
\begin{align}
  \begin{split}
    x_t & = \mathrm{sgn}(z_{\tLoc})\exp\left[-\mathrm{arcosh}
    \left(\frac{\ls}{\gamma |z_{\tLoc}|}\right) + \ls (t - \tLoc)
    \right]  
    \\
  & \simeq  \frac{\sin \Theta}{2 \ls} \left(\frac{z_0}{2} -
  2\cos \Theta\frac{\varphi_0}{\ls}\right)e^{\ls t} 
  \\
  &\simeq
  \frac{\sin \Theta}{\ls} \left(\frac{n_0}{N} - 2\cos \Theta
  \frac{\varphi_0}{\ls}\right)\sinh(\ls t) \,, 
  \end{split}
  \label{eq:xt}
\end{align}
where we make use of the asymptotic expression
\begin{equation}
  \mathrm{arcosh}(u) = \ln\left(u + \sqrt{u^2 - 1}\right) \simeq \ln(2u)
   + O(u^{-2})
   \label{eq:arcosh}
\end{equation}
for large $u$, in combination with Eq.~\eqref{eq:BH.za}.
With $(\cosh u)^{-1} = 2 e^u / ( 1 + e^{2u})$ together with Eqs.~\eqref{eq:arcosh} and \eqref{eq:xt}, Eq.~\eqref{eq:zt} yields
\begin{equation}
  z_\tau = \frac{2 \ls}{\sin \Theta} \frac{x_t}{1 + x_t^2}
\end{equation}
and thus
\begin{equation}
  n_t = \frac{N \ls}{\sin \Theta} \frac{x_{t}}{1 + x_{t}^2}
  \label{eq:BH.nt} \,.
\end{equation}
Replacing $e^{\ls t}$ with $2 \sinh(\ls t)$ in Eq.~\eqref{eq:xt}
is clearly valid for large $\ls t\gg 1$ and has the additional advantage
that the short-time regime in the time evolution of $n_t$ will thereby be
correctly captured as well within Eq.~\eqref{eq:BH.nt}.

The classical limit of the quantum OTOC, Eq.~\eqref{eq:OTOCc},
is then evaluated as
\begin{align}
    \begin{split}
    O(t) = \frac{2\cos^2\Theta N^2}{ \sqrt{\pi} a \ls^2}& \sinh(\ls t)
    \\
  \int \frac{(1 - x^2)^2}{(1 + x^2)^4}&
  \exp\left[-\left(\frac{ x}{2a \sinh(\ls t)}\right)^2\right] \d x 
    \end{split}
    \label{eq:exact.clOTOC}
\end{align}
where the dimensionless scale is defined as
\begin{equation}
  a = \frac{\sin{\Theta}/\ls}{\sqrt{8 \omega N}}
  \sqrt{\omega^2 + \frac{16 \cos^2 \Theta }{\ls^2}} \,.
  \label{eq:aParameter}
\end{equation}
% Using Eq.~\eqref{eq:BH.la}, this expression simplifies in the special
% case $\omega = 1$ yielding
% \begin{equation}
%   a = \frac{1}{\gamma - 1}\sqrt{\frac{\gamma^3}{32 N}}
%   = \frac{N \sqrt{U^3/J}}{8(N U - 2 J)} \,.
% \end{equation}
The short-time behavior of the OTOC, for $t \ll \tLeak = - \ls^{-1} \ln a$,
is yielded as
\begin{equation}
  O(t) \simeq 4\cos^{2} \Theta \frac{N^2}{\ls^2} \sinh^2(\ls t) \,,
  \label{eq:shortClassical}
\end{equation}
while for $t \gg \tLeak$ we obtain
\begin{equation}
  O(t) \simeq \cos^{2} \Theta \frac{\sqrt{\pi} N^2}{4 a \ls^2} e^{\ls t} \,.
  \label{eq:longClassical}
\end{equation}
These two limits correspond to the heuristic derived $2\ls-\ls$ transition Eq.~\eqref{eq:OverviewFP} in the previous Sec.~\ref{sec:OTOC_theory}. Note the here defined $\tLeak$ agrees with the case ii) in Sec.~\ref{sec:OTOC_theory}.
The dimensionless constant $\ln a$ encodes the linear width of the wave-packet along the unstable direction. 

With this extensive classical calculation at hand, we analyze the OTOC centered around this local hyperbolic antihom. FP using $\Theta_{\ast}$ at the maximal value of the stability exponent $\ls = 0.97$. 

\subsection{Numerical results for the Out-of-Time-Order Correlator}


We proceed now with the numerical study and calculate the OTOC via Eq.~\eqref{eq:OTOC_defDimer} by means of extensive numerically exact simulations for the operators $\hat{A}=\hat{B}=\hat{n}_{1}$. We will consider the state 
\begin{align*}
    |{\vec{\xi}}\,\rangle = \frac{1}{\mathcal{N}} \big( \Vec{\xi}\cdot \hat{\vec{a}}^{\dagger} \big)^{N} \ket{0},
\end{align*}
which is a number-projected coherent state centered at the antihom. FP $\vec{\xi} = (\sqrt{N/2},-\sqrt{N/2})$, with $\mathcal{N} = \sqrt{N^N N!}$ a normalization constant.

For large total particle number $N$, the projected coherent state inherits properties from the coherent state, in particular the linear width of $\hbar_{{\rm eff}}^{1/2}$ in each phase space direction \cite{gardiner2004quantum}, including the unstable direction in Fig.~\ref{fig:PhaseSpaceFP}. Furthermore it sets the squeezing parameter $\omega=1$ in the classical analysis in Eq.~\eqref{eq:aParameter}. 
Following the discussion in Sec.~\ref{sec:OTOC_theory}, case ii), the leaking time $\tLeak$ is therefore half the Ehrenfest time $\tEhr$.
\begin{figure}[h!]
    \centering
    \includegraphics[width=\linewidth]{figures/L2/OTOC-plots/OTOC_antihomogeneous_N1000_L2_Theta1.351.pdf}\\
    \includegraphics[width=\linewidth]{figures/L2/OTOC-plots/OTOC_antihomogeneous_N10000_L2_Theta1.351.pdf}\\
    \includegraphics[width=\linewidth]{figures/L2/OTOC-plots/OTOC_antihomogeneous_N50000_L2_Theta1.351.pdf}
    %\includegraphics[width=0.75\linewidth,page=1]{figures/L2/OTOC-plots/comb_Theta=-1.35_N=1000_L=2.pdf}\\
    %\includegraphics[width=0.75\linewidth,page=1]{figures/L2/OTOC-plots/comb_Theta=-1.35_N=10000_L=2.pdf}\\
    %\includegraphics[width=0.75\linewidth,page=1]{figures/L2/OTOC-plots/comb_Theta=-1.35_N=50000_L=2.pdf}
    \caption{Top to bottom: OTOC $\mathbf{C}(t)$ for $N=10^3, ~10^4, ~5\cdot 10^4$ and $\Theta=1.35$; shaping kink from the $2\ls-\ls$ transition at $\tLeak=\tEhr/2$;  the classical expressions Eq.~\eqref{eq:shortClassical} and Eq.~\eqref{eq:longClassical} fit tightly the OTOC in each region showing $e^{2\ls t}$ and $e^{\ls t}$ exponential growth rates.}
    \label{fig:OTOC_FP_numeric}
\end{figure}
 We display our numerical OTOCs for increasing particle $N=10^3, ~10^4, ~5\cdot 10^4$ in Fig.~\ref{fig:OTOC_FP_numeric}, where we observe the predicted $2\ls-\ls$ transition, precisely following the heuristic arguments of Sec.~\ref{sec:OTOC_theory} and the exact analysis of Sec.~\ref{sec:peter}. In particular, the analytical result for the classical OTOC Eq.~\eqref{eq:exact.clOTOC} follows perfectly the quantum OTOC $\mathbf{C}(t)$, i.e., it captures both regimes and the transition.
 The kink at the transition gets sharper for $N\to \infty$.  Here, the insets showing the time-derivative of $\log (\mathbf{C}(t))$ confirm a more and more pronounced $2\ls$ and $\ls$ regions of exponential growth.


\begin{figure*}[!ht]
    \centering
    %\includegraphics[width=0.35\linewidth,page=21,trim ={ 1.2cm 1.9cm 7.5cm 1.75cm},clip]{figures/L2/Husimi-phase-space/Husimi_log_hom_Theta=-1.35_N=1000_L=2-1.pdf}%
    %\includegraphics[width=0.35\linewidth,page=26,trim ={ 1.2cm 1.9cm 7.5cm 1.75cm},clip]{figures/L2/Husimi-phase-space/Husimi_log_hom_Theta=-1.35_N=1000_L=2-1.pdf}
    %\\
    %\includegraphics[width=0.35\linewidth,page=31,trim ={ 1.2cm 1.9cm 7.5cm 1.75cm},clip]{figures/L2/Husimi-phase-space/Husimi_log_hom_Theta=-1.35_N=1000_L=2-1.pdf}%
    %\includegraphics[width=0.35\linewidth,page=36,trim ={ 1.2cm 1.9cm 7.5cm 1.75cm},clip]{figures/L2/Husimi-phase-space/Husimi_log_hom_Theta=-1.35_N=1000_L=2-1.pdf}
    \includegraphics[width=0.5\linewidth,page=21,clip]{figures/L2/Husimi-phase-space/antihom_comb_Theta=1.35_N=1000_L=2.pdf}%
    \includegraphics[width=0.5\linewidth,page=26,clip]{figures/L2/Husimi-phase-space/antihom_comb_Theta=1.35_N=1000_L=2.pdf}%
    \\
    \includegraphics[width=0.5\linewidth,page=31,clip]{figures/L2/Husimi-phase-space/antihom_comb_Theta=1.35_N=1000_L=2.pdf}%
    \includegraphics[width=0.5\linewidth,page=36,clip]{figures/L2/Husimi-phase-space/antihom_comb_Theta=1.35_N=1000_L=2.pdf}%
    
    %\includegraphics[width=0.35\linewidth,page=31,trim ={ 1.2cm 1.9cm 7.5cm 1.75cm},clip]{figures/L2/Husimi-phase-space/Husimi_log_hom_Theta=-1.35_N=1000_L=2-1.pdf}%
    %\includegraphics[width=0.35\linewidth,page=36,trim ={ 1.2cm 1.9cm 7.5cm 1.75cm},clip]{figures/L2/Husimi-phase-space/Husimi_log_hom_Theta=-1.35_N=1000_L=2-1.pdf}
    
    \caption{Time-evolution of the Husimi-distribution for the state $|\vec{\xi}\, \rangle$ centered at the hyperbolic antihom. FP with $N=10^3$ particles, we observe scrambling along the unstable manifold on the separatrix until $t\approx \tLeak$.}
    \label{fig:husimiFP}
\end{figure*}
In order to justify the leaking from the linearized region around the FP, we visualize the time dynamical evolution of the Husimi distribution for $N=10^{3}$ in Fig.~\ref{fig:husimiFP}. 
With time, the linear width of the wave-packet increases and evolves to the upper right and lower left corner of the phase space. 
We recognize that at time $\tLeak = \tEhr/2$ the wave-packet folds back from the unstable to the stable manifold. 
This back-folding corresponds to the dynamical transition of leaking from the linearized regime.

The excellent agreement between our physical picture based on the leaking mechanism and the numerical simulations opens the possibility of manipulating the leaking time $\tLeak$, and therefore the different scrambling regimes, via squeezing the initial state $|{\vec{\xi}}\,\rangle$, as we discuss next.


\subsection{Squeezing -- engineering the leaking time $\tLeak$}

%\rc{Jesus Christ, what the hell means "A further manifest of our theory"?}
An important consequence of the leaking mechanism is that the linear width of the initial state along the unstable manifolds is the key ingredient for the exact position of the $2\ls-\ls$ transition. 

In order to check this dependence, we proceed to squeeze the coherent state on the antihom. FP and subsequently calculate the OTOC. Interestingly, squeezed states in optical lattices can be archived experimentally \cite{squeezingBEC2008} to an excellent degree. In our theoretical setup, squeezing protocol can be effectively (and unitarily) realized by the reversing the time evolution itself. This means, we replace $\ket{\xi }$ by 
\begin{align*}
    \ket{\xi (t_{0})} = \hat{U}(t_{0})\ket{\xi}
\end{align*}
with $t_{0} = -\tEhr/2$, where $\hat{U}(t_{0})$ is the time-evolution operator, and then  calculate the OTOC Eq.~\eqref{eq:OTOC_def} for the initial state $\hat{\rho}(t_{0}) = \ket{\xi (t_{0}) }\bra{\xi (t_{0})} $. The corresponding scrambling dynamics is shown in then the left panel of Fig. \ref{fig:OTOC_FP_numeric_squeezed} for $N=10^3$, while the right panel depicts the initial Husimi distribution of the squeezed state.
\begin{figure*}[th!]
    \centering
    \includegraphics[width=0.49\linewidth,page=11]{figures/L2/OTOC-plots/timeShift_antihom_Theta=1.35_N=1000_L=2.pdf}%
    %\includegraphics[width=0.54\linewidth,page=16]{figures/L2/OTOC-plots/timeShift_t0_Theta=-1.35_N=1000_L=2.pdf}%
    %\includegraphics[width=0.46\linewidth,page=7]{figures/L2/Husimi-phase-space/Husimi_log_hom_Theta=-1.35_N=1000_L=2-1.pdf}
    \includegraphics[width=0.51\linewidth,page=11,clip]{figures/L2/Husimi-phase-space/antihom_comb_Theta=1.35_N=1000_L=2.pdf}%
    \caption{Left panel: OTOC for a squeezed coherent state for the system parameter $\Theta=1.35$ and particle number $N=10^{3}$; the transition to $\ls$ vanishes. Right panel: Husimi distribution of the squeezed coherent state. The state is distributed along the stable and localized $\sim \hbarE$ along the unstable manifold.}
    \label{fig:OTOC_FP_numeric_squeezed}
\end{figure*}
This backward-time evaluated coherent state has a reduced linear width along the unstable manifold by taking $t_{0}$ to $-\tEhr/2$ (for the non-squeezed coherent state), i.e., we transform $\Delta u \sim \hbar_{{\rm eff}}^{1/2}$ to $\Delta u \sim \hbarE$. Thus, the new leaking time $\tLeak^{\ast}$ is at the Ehrenfest time and no $2\ls-\ls$~transition is expected to exist, as fully confirmed by the numerical simulations.

\subsection{OTOC -- parameter scan }
So far, we handpicked one of the most unstable configuration for the antihom. FP and did not perform a fitting with an exponential function beyond the visual agreement in Figs.~\ref{fig:OTOC_FP_numeric} and ~\ref{fig:OTOC_FP_numeric_squeezed}. 
We now show the results of a proper fitting procedure in Fig. \ref{fig:parameterScan}  for the number projected state $|{\vec{\xi}}\,\rangle$. Here, we show the fitted quantum Lyapunov exponents $\lq$ of the OTOC in both regions $[\tLoc,\tLeak]$ and $[\tLeak,\tEhr]$ along the whole parameter range of $\Theta$ and for several values of the total particle number $N$.  
We see an increasing agreement with the $2\ls$- and $\ls$-regions: the fitted exponents show the same $\Theta$-dependency as the stability exponent $\ls$. Further, their magnitudes approach the classical value with $\hbarE\to 0$. The discrepancy originates from the fitting procedure, as it ignores the transition between the two exponential regimes. 
\begin{figure*}[]
    \centering
    \includegraphics[width=0.5\linewidth]{figures/L2/parameter-Fits/antihomogeneousL2_antihomogeneous_fitShort.pdf}%
    \includegraphics[width=0.5\linewidth]{figures/L2/parameter-Fits/antihomogeneousL2_antihomogeneous_fitLong.pdf}
    %\includegraphics[width=0.8\linewidth]{figures/L2/parameter-Fits/homogeneousL2_N[10, 100, 1000, 10000, 40000]_firstWindow.pdf}\\
    %\includegraphics[width=0.8\linewidth]{figures/L2/parameter-Fits/homogeneousL2_N[10, 100, 1000, 10000, 40000]_secondWindow.pdf}
    \caption{Fitted exponents for the $2\ls$ (upper panel) and the~$\ls$ (lower panel) region; for $N\to \infty$, we see an increasing agreement with the classical predictions from Eq.~\eqref{eq:OverviewFP}.}
    \label{fig:parameterScan}
\end{figure*}
We conclude that the~$2\ls-\ls$ transition is independent of the specific value of~$\Theta$, i.e., it is a robust signal of a dynamical transition.
%\FloatBarrier