
\subsection{Microscopic approach: separatrix dynamics} 
\label{sec:peter}
\ifthenelse{1=1}
{}
{
In this section we provide a microscopic understanding of the dynamical transition by means of the exact separatrix dynamics of the mean field limit.

In this section, we work out an analytical expression for the classical OTOC
associated with a quantum state that is launched on the separatrix point of the
Bose-Hubbard dimer.
The latter is described by the quantum Hamiltonian
\begin{equation}
  \hat{H} = - J \left( \hat{a}_1^\dagger \hat{a}_2 + \hat{a}_2^\dagger \hat{a}_1
  \right) + \frac{U}{2} \sum_{l=1,2} \hat{a}_l^\dagger\hat{a}_l^\dagger
  \hat{a}_l\hat{a}_l
\end{equation}
with $\hat{a}_l^\dagger,\hat{a}_l$ the bosonic particle creation and
annihilation operator on the site $l=1,2$,
$J$ the hopping parameter, and $U$ the interaction parameter.

This quantum system has a classical counterpart which is given in terms of
the discrete nonlinear Schr\"odinger equation (setting $\hbar = 1$)
\begin{eqnarray}
  i \frac{d \psi_1}{d t} & = & - J \psi_2 + U |\psi_1|^2 \psi_1 \,, \\
  i \frac{d \psi_2}{d t} & = & - J \psi_1 + U |\psi_2|^2 \psi_2 \,,
\end{eqnarray}
with $\psi_l$ the classical field amplitude that is associated with the site
$l=1,2$.
Expressing $\psi_l = \sqrt{n_l}e^{i \varphi_l}$ in terms of the (real)
classical site occupancies $n_l$ and phases $\varphi_l$, and performing
the canonical transformation
$(n1,n2,\varphi_1,\varphi_2) \mapsto (N,n,\phi,\varphi)$
to new classical variables defined through
\begin{eqnarray}
  N & = & n_1 + n_2 \,, \\
  n & = & \frac{1}{2}( n_1 - n_2) \,, \\
  \phi & = & \frac{1}{2}( \varphi_1 + \varphi_2 ) \,, \\
  \varphi & = & \varphi_1 - \varphi_2 - \pi \,,
\end{eqnarray}
we end up with a classical Hamiltonian system described the inter-site
population exchange dynamics via the Hamiltonian
\begin{equation}
  H(n,\varphi) = U n^2 + J \sqrt{N^2 - 4 n^2} \cos \varphi \,,
  \label{eq:BH.Hc}
\end{equation}
which parametrically depends on the constant of motion $N$ corresponding
to the total population of the system.
The classical time evolution of the system is given in terms of the
Hamiltonian equations of motion
\begin{eqnarray}
  \frac{d n}{d t} & = & \frac{\partial H}{\partial \varphi}(n,\varphi) =
  - J \sqrt{N^2 - 4 n^2} \sin \varphi \,, \label{eq:BH.n} \\
  \frac{d \varphi}{d t} & = & - \frac{\partial H}{\partial n}(n,\varphi) =
  - 2 U n + \frac{4 J n \cos \varphi}{\sqrt{N^2 - 4 n^2}} \,, \label{eq:BH.ph}
\end{eqnarray}
Expressed in terms of the relative population imbalance $z = 2 n / N$,
the rescaled dimensionless time $\tau = J t$, and the dimensionless
nonlinearity parameter
\begin{equation}
  \gamma = \frac{N U}{2 J} \,,
\end{equation}
Eqs.~\eqref{eq:BH.n} and \eqref{eq:BH.ph} are rewritten as
\begin{eqnarray}
  \dot{z} \equiv \frac{d z}{d \tau} & = & - 2 \sqrt{1 - z^2} \sin\varphi \,,
  \label{eq:BH.z} \\
  \dot{\varphi} \equiv \frac{d \varphi}{d \tau} & = & - 2 \gamma z
  + \frac{2 z \cos\varphi}{\sqrt{1 - z^2}} \label{eq:BH.p} \,.
\end{eqnarray}
}
We set for the dimer the operators $\hat{A}=\hat{B}=\hat{n}_{1}$ to be the number operator $\hat{n}_{1}= \hat{a}^{\dagger}_{1} \hat{a}_{1}$ at the first site. Therefore, we get
\begin{align}
    \mathbf{C}(t) =\langle || [ \hat{n}_{1}(t), \hat{n}_{1} ] ||^2 \rangle = \langle || [ \hat{n}(t), \hat{n} ] ||^2 \rangle\, ,
    \label{eq:OTOC_defDimer}
\end{align} 
where $\hat{n} = \frac{1}{2}(\hat{n}_{1}-\hat{n}_{2})$. For this OTOC, we are interested in evaluating the classical expression which is
given by
\begin{equation}
  O(t) = \iint dn d\varphi\, W(n,\varphi)
  \left(\frac{\partial n_t}{\partial \varphi_0}\right)^2 \,, \label{eq:OTOCc}
\end{equation}
with $W(n,\varphi)$ the Wigner function associated with the initial state.
The latter is, for the sake of simplicity, modeled as a coherent quantum state
$e^{\sqrt{N_0}(\hat{a}_-^\dagger - \hat{a}_-)} \ket{0}$ with $\hat{a}_{-}=\hat{a}_{1} -\hat{a}_{2}$, keeping in mind that for large
$N_0$ this coherent state features very similar properties as a
number-projected coherent state with total particle number $N_0$ as far as the
site population exchange dynamics is concerned.
The Wigner function associated with this initial state would be given by
\begin{align}
    \begin{split}
        W(N&,\phi,n,\varphi) \simeq 
        \\
        \frac{1}{\pi^2} &\exp\left(-
        \frac{(N - N_0)^2}{2 N_0} - 2 N_0 \phi^2 - \frac{2 n^2}{N_0} -
        \frac{N_0 \varphi^2}{2} \right)
    \end{split}
\end{align}
  in the framework of a quadratic expansion valid for $N_0 \gg 1$
(using again $\hbar = 1$).
Since the Wigner function $W$ describes a tight localization of $N$ about $N_0$, we set
$N_0 = N$ henceforth and model the initial quantum state concerning the
inter-site population exchange dynamics by the Wigner function
\begin{equation}
  W(n,\varphi) = \frac{1}{\pi} \exp\left(- \frac{2 n^2}{\omega N}
  -  \frac{N \omega \varphi^2}{2} \right) \,,
\end{equation}
where the squeezing parameter $\omega$ allows for some flexibility
in the definition of the initial quantum state.

Let us first discuss the linearized dynamics in the near vicinity of
the FP~$(n,\varphi) = (0,0)$.
Linearizing Eq.~\eqref{eq:EquationOfMotion}, we obtain
the system of equations 
\begin{align}
\begin{split}
     \dot{z} & =  - 4\cos \Theta \varphi \,,  \\
  \dot{\varphi} & = - 2 (\sin \Theta -2\cos{\Theta}) z \,, 
\end{split}
\label{eq:BH.peterlinearized}
\end{align}
which is readily solved as
\begin{align}
    \begin{split}
      z_t & = z_0 \cosh \ls t - \frac{4\cos{\Theta} \varphi_0}{\ls}
      \sinh \ls t\,,  \\
      \varphi_t & = \varphi_0\cosh \ls t - \frac{\ls z_0}{4\cos{\Theta}}
      \sinh \ls t 
    \end{split}
\label{eq:BH.peterlinearizTime}
\end{align}
in terms of the stability exponent
\begin{equation}
  \ls = 4\cos{\Theta}\sqrt{\frac{\gamma}{2} - 1} \, , \label{eq:BH.la}
\end{equation}
where we defined the so called nonlinearity parameter $\gamma=\tan \Theta = \frac{gN}{LJ}$.
The latter becomes purely imaginary for $\gamma < 2$, which implies that
$(n,\varphi) = (0,0)$ turns into a stable fixed point if the nonlinearity
parameter $\gamma$ is decreased below two, as it is plotted in Fig.~\ref{fig:StablHomFP}.

Considering $\gamma > 2$ henceforth, and assuming that the point
$(z_0,\varphi_0)$ is located very closely to the origin in this phase space,
we can, as in the previous section, identify a time scale
$\tLoc \gg \ls^{-1}$ for which we still have $|z_{\tLoc}| \ll 1$ and
$|\varphi_{\tLoc}|\ll 1$, such that the above linearization Eq.~\eqref{eq:BH.peterlinearized} of the classical
equations of motion remains valid until $t = \tLoc$.
Since at the same time we have $\ls \tLoc \gg 1$ by assumption,
the solution Eq.~\eqref{eq:BH.peterlinearizTime} of the linearized
Eq.~\eqref{eq:BH.peterlinearized} for $t= \tLoc$
simplifies as
\begin{eqnarray}
  z_{\tLoc} & = & \left(\frac{z_0}{2} -  \frac{2\cos \Theta\varphi_0}{\ls}\right)
  e^{\ls \tLoc} \,, \label{eq:BH.za} \\
  \varphi_{\tLoc} & = & \left(\frac{\varphi_0}{2} - \frac{\ls z_0}{8 \cos \Theta}
  \right) e^{\ls \tLoc} \,.
\end{eqnarray}

From the time $\tLoc$ on, we can safely assume that the trajectory
under consideration very closely follows the separatrix structure emanating
from the unstable antihom. fixed point $(z,\varphi) =(0,0)$. 
This separatrix structure is obtained through the identification of the
energy
\begin{equation}
  H(n,\varphi) = 2\cos \Theta N + \sin{\Theta} \frac{N}{2}
\end{equation}
of the classical Hamiltonian Eq. \eqref{eq:classicalHamiltonian}, from which follows the identity
\begin{equation*}
  \cos\varphi = \frac{1 - \frac{\gamma}{4} z^2 }{\sqrt{1 - z^2}} \,.
\end{equation*}
Inserting this expression into Eq.~\eqref{eq:EquationOfMotion} yields the differential
equation
\begin{equation*}
  \dot{z} = z \sqrt{\ls^2 - \sin^2\Theta  z^2}
\end{equation*}
describing the motion along the upper or lower separatrix branch.
This equation is straightforwardly integrated yielding
\begin{align*}
  t - \tLoc =& \int_{z_{\tLoc}}^{z_t}
  \frac{\d z}{z \sqrt{\ls^2 - \sin^2\Theta z^2}}
  \\
   =& - \frac{1}{\ls}\left[ \mathrm{arcosh}
    \left(\frac{\ls}{\sin \Theta |z_{t}|} \right) - \mathrm{arcosh}
    \left(\frac{\ls}{\sin \Theta |z_{\tLoc}|} \right) \right] \,,
\end{align*}
from which we obtain
\begin{equation}
  z_t= \frac{\mathrm{sgn}(z_{\tLoc}) \ls / \sin\Theta}
  {\cosh\left[\mathrm{arcosh} \left(\frac{\ls}{\sin\Theta |z_{\tLoc}|}
      \right) - \ls (t - \tLoc) \right]} \,.
      \label{eq:zt}
\end{equation}
Using $|z_{\tLoc}| \ll 1$ and hence also $\sin\Theta |z_{\tLoc}|/\ls \ll 1$
for finite values of $\sin\Theta$ and $\ls$, we define
\begin{align}
  \begin{split}
    x_t & = \mathrm{sgn}(z_{\tLoc})\exp\left[-\mathrm{arcosh}
    \left(\frac{\ls}{\gamma |z_{\tLoc}|}\right) + \ls (t - \tLoc)
    \right]  
    \\
  & \simeq  \frac{\sin \Theta}{2 \ls} \left(\frac{z_0}{2} -
  2\cos \Theta\frac{\varphi_0}{\ls}\right)e^{\ls t} 
  \\
  &\simeq
  \frac{\sin \Theta}{\ls} \left(\frac{n_0}{N} - 2\cos \Theta
  \frac{\varphi_0}{\ls}\right)\sinh(\ls t) \,, 
  \end{split}
  \label{eq:xt}
\end{align}
where we make use of the asymptotic expression
\begin{equation}
  \mathrm{arcosh}(u) = \ln\left(u + \sqrt{u^2 - 1}\right) \simeq \ln(2u)
   + O(u^{-2})
   \label{eq:arcosh}
\end{equation}
for large $u$, in combination with Eq.~\eqref{eq:BH.za}.
With $(\cosh u)^{-1} = 2 e^u / ( 1 + e^{2u})$ together with Eqs.~\eqref{eq:arcosh} and \eqref{eq:xt}, Eq.~\eqref{eq:zt} yields
\begin{equation}
  z_\tau = \frac{2 \ls}{\sin \Theta} \frac{x_t}{1 + x_t^2}
\end{equation}
and thus
\begin{equation}
  n_t = \frac{N \ls}{\sin \Theta} \frac{x_{t}}{1 + x_{t}^2}
  \label{eq:BH.nt} \,.
\end{equation}
Replacing $e^{\ls t}$ with $2 \sinh(\ls t)$ in Eq.~\eqref{eq:xt}
is clearly valid for large $\ls t\gg 1$ and has the additional advantage
that the short-time regime in the time evolution of $n_t$ will thereby be
correctly captured as well within Eq.~\eqref{eq:BH.nt}.

The classical limit of the quantum OTOC, Eq.~\eqref{eq:OTOCc},
is then evaluated as
\begin{align}
    \begin{split}
    O(t) = \frac{2\cos^2\Theta N^2}{ \sqrt{\pi} a \ls^2}& \sinh(\ls t)
    \\
  \int \frac{(1 - x^2)^2}{(1 + x^2)^4}&
  \exp\left[-\left(\frac{ x}{2a \sinh(\ls t)}\right)^2\right] \d x 
    \end{split}
    \label{eq:exact.clOTOC}
\end{align}
where the dimensionless scale is defined as
\begin{equation}
  a = \frac{\sin{\Theta}/\ls}{\sqrt{8 \omega N}}
  \sqrt{\omega^2 + \frac{16 \cos^2 \Theta }{\ls^2}} \,.
  \label{eq:aParameter}
\end{equation}
% Using Eq.~\eqref{eq:BH.la}, this expression simplifies in the special
% case $\omega = 1$ yielding
% \begin{equation}
%   a = \frac{1}{\gamma - 1}\sqrt{\frac{\gamma^3}{32 N}}
%   = \frac{N \sqrt{U^3/J}}{8(N U - 2 J)} \,.
% \end{equation}
The short-time behavior of the OTOC, for $t \ll \tLeak = - \ls^{-1} \ln a$,
is yielded as
\begin{equation}
  O(t) \simeq 4\cos^{2} \Theta \frac{N^2}{\ls^2} \sinh^2(\ls t) \,,
  \label{eq:shortClassical}
\end{equation}
while for $t \gg \tLeak$ we obtain
\begin{equation}
  O(t) \simeq \cos^{2} \Theta \frac{\sqrt{\pi} N^2}{4 a \ls^2} e^{\ls t} \,.
  \label{eq:longClassical}
\end{equation}
These two limits correspond to the heuristic derived $2\ls-\ls$ transition Eq.~\eqref{eq:OverviewFP} in the previous Sec.~\ref{sec:OTOC_theory}. Note the here defined $\tLeak$ agrees with the case ii) in Sec.~\ref{sec:OTOC_theory}.
The dimensionless constant $\ln a$ encodes the linear width of the wave-packet along the unstable direction. 