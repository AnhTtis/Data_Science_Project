\section*{Appendix}

\subsection{Wigner-Moyal expansion}
\label{sec:appendixWignerMoyal}
We start with Eq.~\eqref{eq:OTOC_def} and apply the phase-space formulation of quantum mechanics \cite{Books}
\begin{align*}
    sdfa
\end{align*}

% \subsection{Infinite temperature Out-of-Time-Order Correlator}
% \label{sec:appendixInfTemp}
% We take the infinite temperature state and have a diagonal density matrix
% \begin{align*}
%     \hat{\rho} = \frac{1}{\dim \mathcal{H}} \mathds{1}.
% \end{align*}
% Its phase-space distribution is homogeneous, therefore we have a linear width of order $1$, i.e., $\Delta u\sim O(1)$.
% \begin{figure}[h!]
%     \centering
%     \includegraphics[width=0.75\linewidth]{figures/L2/OTOC-plots/inf_temp_Theta=-1.35_N=20000_L=2.pdf}
%     \caption{OTOC for the infinite temperature state: $\lambda_{s}$-exponential growth dominates for the hole pre-Ehrenfest time window; this is supported by the plateau at $\lambda_{s}$ in the inset}
%     \label{fig:OTOC_infTemp}
% \end{figure}
% As predicted by \cite{PhysRevLett.124.140602} and in Eq. \ref{eq:OverviewFP}, the numerical calculated OTOC in Fig. \ref{fig:OTOC_infTemp} shows only the $\lambda_{s}$-exponential regime and no transition or a $2\lambda_{s}$-regime is visible.