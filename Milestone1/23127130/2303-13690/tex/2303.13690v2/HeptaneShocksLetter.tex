% ****** Start of file apssamp.tex ******
%
%   This file is part of the APS files in the REVTeX 4.2 distribution.
%   Version 4.2a of REVTeX, December 2014
%
%   Copyright (c) 2014 The American Physical Society.
%
%   See the REVTeX 4 README file for restrictions and more information.
%
% TeX'ing this file requires that you have AMS-LaTeX 2.0 installed
% as well as the rest of the prerequisites for REVTeX 4.2
%
% See the REVTeX 4 README file
% It also requires running BibTeX. The commands are as follows:
%
%  1)  latex apssamp.tex
%  2)  bibtex apssamp
%  3)  latex apssamp.tex
%  4)  latex apssamp.tex
%


\documentclass[%
 reprint,
superscriptaddress,
%groupedaddress,
%unsortedaddress,
%runinaddress,
%frontmatterverbose, 
%preprint,
%preprintnumbers,
%nofootinbib,
%nobibnotes,
%bibnotes,
 amsmath,amssymb,
 aps,
%pra,
%prb,
%rmp,
%prstab,
%prstper,
floatfix,
]{revtex4-2}

\usepackage{siunitx}


%for use of \FloatBarrier
\usepackage{placeins}

%make references and citations clickable.
\usepackage[hidelinks]{hyperref} 


\usepackage{graphicx}% Include figure files
\usepackage{dcolumn}% Align table columns on decimal point
\usepackage{bm}% bold math
%\usepackage{hyperref}% add hypertext capabilities
%\usepackage[mathlines]{lineno}% Enable numbering of text and display math
%\linenumbers\relax % Commence numbering lines

%\usepackage[showframe,%Uncomment any one of the following lines to test 
%%scale=0.7, marginratio={1:1, 2:3}, ignoreall,% default settings
%%text={7in,10in},centering,
%%margin=1.5in,
%%total={6.5in,8.75in}, top=1.2in, left=0.9in, includefoot,
%%height=10in,a5paper,hmargin={3cm,0.8in},
%]{geometry}

\begin{document}


\preprint{APS/123-QED}

%\title{Ultrafast X-ray Phase Contrast Imaging of Low-energy Shockwaves and Dynamic Processes Using a High-Repetition Rate Pulsed Plasma Target}
\title{Ultrafast X-ray Phase Contrast Imaging of High Repetition Rate Shockwaves}

\author{Christopher S. Campbell}
\affiliation{Department of Mechanical Engineering, Texas A\&M University, College Station, Texas 77843, USA}
\affiliation{Los Alamos National Laboratory, Los Alamos, New Mexico 87545, USA}

\author{Mirza Akhter}%
\affiliation{Department of Mechanical Engineering, Texas A\&M University, College Station, Texas 77843, USA}

\author{Samuel Clark}
\author{Kamel Fezzaa}
\affiliation{X-ray Science Division, Advanced Photon Source, Argonne National Laboratory, Argonne, Illinois 60439, USA}

\author{Zhehui Wang}
\affiliation{Los Alamos National Laboratory, Los Alamos, New Mexico 87545, USA}
 \email{zwang@lanl.gov}

\author{David Staack}
\affiliation{Department of Mechanical Engineering, Texas A\&M University, College Station, Texas 77843, USA}
 \email{dstaack@tamu.edu}
%\altaffiliation[Also at ]{Physics Department, XYZ University.}%Lines break automatically or can be forced with \\



\date{\today}% It is always \today, today,
             %  but any date may be explicitly specified

\begin{abstract}
High-repetition-rate plasma-induced shockwaves in liquid have been observed using ultrafast X-ray phase contrast imaging (PCI) for the first time.
Using a laser-triggered nanosecond-pulsed plasma device in heptane at ambient conditions, it is demonstrated that these well-timed weak shocks can be generated at an unprecedented repetition rate ($>$3 per minute), significantly faster than that of more commonly-used dynamic targets (exploding wire, gas gun).
This simple portable target can easily be adapted to study discharges in different media (water, oils, solids) at comparably high repetition rates and over a wide range of possible input energies.
Compared to previously PCI-imaged shocks, these shocks are relatively weak ($1 < \text{Mach number} < 1.4$), which advances the resolution and sensitivity limits of this high-speed imaging diagnostic.
Numeric solutions of a Fresnel-Kirchhoff diffraction model are used to estimate post-shock thermodynamic conditions, the results of which show good agreement with expectations based on Rankine-Hugoniot normal shock thermodynamic relations.
%A comparison in shock imaging sensitivity between LYSO and LuAG scintillators is also discussed, showing that the short decay tail of LYSO brings the shock profile above the detectability limit for this implementation of PCI.



\end{abstract}

%\keywords{Suggested keywords}%Use showkeys class option if keyword
                              %display desired
\maketitle


%\begin{figure}[b!]
 %   \centering
 %   \includegraphics[width=\linewidth]{event014frame54_annotated.png} %0014 from sunday, Sept '22
 %   \caption{    
 %   Expanding cylindrical shock front induced by pulsed plasma event in heptane.
 %   The two main features visible in the frame are the relatively fast low-contrast weak shock (A), followed by the relatively slow plasma-generated expanding gas bubble %(B).
%    This particular frame is delayed by 134ns relative to the initiation of the high-current spark discharge, with an exposure time is 33ps.
%    }
%    \label{fig:setup}
%\end{figure}

%INTRODUCTION
In the fields of high-speed X-ray science and synchrotron radiation, the maximum achievable repetition rate of a dynamic target of interest is an important figure of merit when pursuing efficient use of limited beamtime.
However, many of the common dynamic processes of most interest feature destructible devices, requiring complete or partial reassembly of the target after each imaging event \cite{Ram2012,Dattelbaum2020ShockwaveStructures,Sechrest2020} which severely limits repetition rate.
It would therefore be beneficial to develop a target which requires minimal to no maintenance between events, without compromising phenomena of interest such as high instantaneous power density, high mass density gradients, high pressure and temperature gradients, and supersonic behavior/shockwaves.
In this letter we present such a target, a pulsed power device submerged in ambient liquid heptane which can produce well-timed nanosecond-pulsed spark discharges.
This target was taken to the Advanced Photon Source (APS) for ultrafast phase-contrast imaging (PCI) experiments, the results of which are presented herein.

Of particular interest in this subset of imaging results is the presence of a visible expanding shock front generated by the spark discharge event, which to the best of our knowledge represents one of the weakest shock fronts ever imaged using PCI (Ma $\approx$ 1.2).
While a sufficiently strong shock would be easily visible using less sensitive imaging techniques due to its relatively high mass density ratio, the fact that such a weak shock front is still observable in this work highlights the superior sensitivity of this implementation of PCI to very subtle dynamic phenomena, while still revealing the limits of current techniques and the path forward for the next generation of ultrafast imaging.
The high repetition rate ($>$3 events/minute), low cost ($<$US\$100k), and portability of this imaging target makes it quite attractive to those fields interested in events of similar timescales and power densities (e.g. ICF, dynamic compression, shock physics), but which rely on apparatuses which are either immovable or have a prohibitively slow event repetition rate.
This target has the potential to open new opportunities for such fields to benefit from the superior imaging capabilities of the APS and similar user facilities.


%%%%%%%%%%%%%%%%%%%%%%%%%%%%%%%%%%%%%%%%%%%%%%%%%%%%%%%%%%%%%%%%%%%%%%%%%%%%%%%%%%%%%%%%%%%%%%%%

%PCI results feature a localized ($<2 \si{\micro\meter}$) cylindrically expanding shock front when using LYSO, however the same phenomena was not detectable using LuAG.
%The short decay tail of LYSO over LuAG allows for minimal ghosting, which in this case facilitated superior imaging of the expanding shockwave generated by the spark discharge.


The pulsed power device and high-voltage circuit used in this work to generate the submerged spark discharge event is similar to those used in our prior work \cite{Campbell2021,Akhter2021}, with this implementation consisting of two electrodes between which a well-timed submerged spark discharge occurs (Figure \ref{fig:setup}).
The event of interest dissipates approximately 100mJ of nanosecond-timescale plasma processes (light, sound, chemistry, shockwaves) in the target over a pulse duration of 100ns, implying an instantaneous power of roughly 1MW.
Assuming an approximate discharge cross section of $5 \si{\micro\meter}$ during peak current across a gap of 0.5mm, we estimate a peak energy density of $15 \si{\giga\joule/\kg}$, within two orders of magnitude (albeit at a lower instantaneous power) of the $1 \si{\tera\joule/kg}$ implied by recent hotspot energy and mass results from the National Ignition Facility \cite{Zylstra2022}.
The X-ray imaging method consists of a 128-frame Shimadzu HPV-X2 camera ($3 \si{\micro\meter}$/pixel), used to image a scintillator placed 46cm from the imaging target (see Figure \ref{fig:setup}).
This setup is capable of a 6.5MHz X-ray framerate, made possible by the APS's 24-singlet standard operating mode \cite{APSparams}.

\begin{figure}[!t]
    \centering
    \includegraphics[width=\linewidth]{fig_imagingsetup.png} %0014 from sunday, Sept '22
    \caption{Simplified schematic of the target and field of view for X-ray imaging.
    %See prior work \cite{Campbell2021,Akhter2021} and supplementary material for more detail of the high-voltage circuit used to generate this plasma.
    }
    \label{fig:setup}
\end{figure}

\begin{figure}[!t] %selected event
    \centering
    \includegraphics[width=\linewidth]{fig_singleevent_annotate.png} %0014 from sunday, Sept '22
    \caption{    
    Selected frames from a single spark discharge event in heptane for which the plasma-induced shock is visible, with timestamps measured relative to spark initiation.
    Left and right columns show contrast-enhanced raw images and corresponding background-subtracted frames respectively; the average of all 128 frames from the event was used as the background.
    Note the location of the shock front visible at t=79ns and t=232ns, denoted by red arrows.
    }
    \label{fig:singleevent}
\end{figure}

\begin{figure}[!t]
    \centering
    \includegraphics[width=\linewidth]{fig_sortedframes_annotate.png} 
    \caption{    
    Frames from selected PCI heptane spark events in which a shock front was visible, sorted by frame time relative to spark initiation.
    Each frame is duplicated across both columns, the right column includes annotations which indicate the contour of the shock using red splines. 
    %The effective shock positions from the events like the ones shown here are plotted as a function of time in Figure \ref{fig:shockspeed}.
    \vspace{0.25in}
    }
    \label{fig:sortedframes}
\end{figure}


\begin{figure}[!t]
    \centering
    \includegraphics[width=\linewidth]{fig1_shockspeed_combined_v2.png}
    \caption{
    Plot of cylindrical shock positions/radii, relative to the axis of the plasma, compiling data from eighty-five different PCI frames for which the shock was visible (Figure \ref{fig:sortedframes}), some of which came from more than one frame of a single event.
    Shock position measurement was performed manually for each of the 85 frames in which a shock was visible.
    The twenty measurements for each frame were then used to determine uncertainty (two standard deviations away from the average).
    The solid line (red) shows the least-squares linear fit to these points (1.45 km/s), with the two red dotted lines assume a shock speed 90\% and 110\% of that linear fit to roughly illustrate inherent shock speed uncertainty. 
    Blue circles indicate shock speed subestimates calculated by fitting to subsamples of the full set (within 100ns of that subestimate's position on the horizontal axis).
    The green curve shows a quadratic fit to the position vs. time data, used for later analysis.
    %Considering the scope of discussion in this work, we have decided that an uncertainty of 10\% is appropriate: $v_\text{shock} = 1.45 \pm 0.15 \si{km/s} $.
    }
    \label{fig:shockspeed}
\end{figure}


\begin{figure}[!t]
    \centering
        \includegraphics[width=\linewidth]{fig_comparetomodel.png}
    \caption{
    Illustration of a cutline extraction algorithm which uses a spline fit to convert the two-dimensional X-ray frame of the shock front (a) into a one-dimensional plot (b). 
    The image analyzed here corresponds with the fifth image from Figure \ref{fig:sortedframes} (t=234ns).
    The black solid line in represents the average cutline (dotted lines show upper and lower quartiles), and the red line shows the best simulated shock front PCI profile with a post-shock density of $\rho_2 = 0.799^{+0.116}_{-0.057} \si{g/cc}$ ($\rho_2/\rho_1 = 1.176^{+0.171}_{-0.084}$).
    %Note that while the cutline illustration in (a) shows the background-subtracted version of the X-ray frame, the cutline was generated using raw image data.
    %, denoted in Figure \ref{fig:densityVSspeed} using a blue datapoint.
    }
    \label{fig:comparetomodel}
\end{figure}





See Figure \ref{fig:singleevent} for selected sequential PCI frames from a single spark discharge event.
Additionally, see Figure \ref{fig:sortedframes} for a compilation of frames from multiple similar events in which the shock is visible in frame, sorted by frame time relative to the instant of plasma initiation.
The estimated speed of this shock is shown in Figure \ref{fig:shockspeed} to be $1.45 \pm 0.13 \si{km/s}$ for this dataset, corresponding to a Mach number in ambient heptane ($v_\text{sound} = 1.129 \si{km/s}$) of $1.28 \pm 0.13$.
The transverse profile of these shock images is consistent with expected PCI for a step discontinuity in density.
%By fitting a line to the position of these shock images over time, we can estimate the speed of the shock.

Also apparent via comparison to this linear trend is the slight negative concavity of the data. 
This suggests a shock speed which decreases with time, which is consistent which the time-dependent shock speed found by linearly fitting to subsamples of the full dataset (within 100ns of a given instant in time).
While at first glance the time dependency from Taylor–von Neumann–Sedov blast wave theory (proportional to $t^{0.4}$ for spherically expanding shocks \cite{Taylor1950} and $t^{0.5}$ for cylindrically expanding shocks \cite{Lin1954}) would presumably serve as a physically-grounded model to fit to this data, the plasma-induced shock front imaged here violates two of the main assumptions required by the Taylor–von Neumann–Sedov theory: instantaneous energy input with the shock originating from a zero-radius point or line (compare to Section SM.II), and negligible ambient pressure ($p_\text{post-shock} \gg 
p_\text{ambient}$).
By the time that the shock becomes visible to this diagnostic (earliest measured shock image at 45ns after initiation), the post-shock pressure has decreased drastically, resulting in a near-linear position vs. time data.
For this analysis it was decided that a phenomenological quadratic fit (green curve on Figure \ref{fig:shockspeed}) would be appropriate, since it requires the least amount of assumptions but still captures this apparent negative concavity.
%This linearity may also be affected by the fact that this event's energy input is not instantaneous (see Section SM.II).
%The Taylor–von Neumann–Sedov self-similar solution best matches very early times in the plasma-induced shock event, during which the ambient fluid pressure is insignificant compared with the post-shock pressure, or ram pressure.
%for the earlier-time datapoints the dotted red line which shows a 10\% higher shock speed appears to be a better fit, and the same is the case for later-time datapoints and the 10\% slower shock speed line.

The radiographic attenuation contrast for such a weak shock is quite low; in our imaging target, the ambient X-ray path consists of a 6mm-thick layer of heptane, the shock has a characteristic size of about $100\si{\micro\meter}$ within the field of view, and the expected density within the post-shock region is approximately 1.2 times that of ambient, implying a maximum possible attenuation contrast of 0.02\% in this case, which is well below the the level of detectability in this experiment.
However, the diffraction-induced contrast enhancement and edge detection features of PCI cause this shock to be visible above background noise as localized maxima and minima in brightness, with the maxima occurring on the side of the discontinuity with lower density, and the minima on the side with higher density.
This type of diffraction is the essence of PCI and is governed by the Fresnel–Kirchhoff integral \cite{MITocw_fresnel}, modified for cylindrically symmetric geometries in Equation \ref{eq:fresnel} to ease computation:
\begin{flalign}
    g_\text{out}(x',y') &= \frac{e^{2\pi i z/\lambda}}{i \lambda z} \iint g_\text{in}(x,y) e^{\frac{i \pi}{\lambda z} ((x'-x)^2 + (y'-y)^2)   }dx dy \\
    \label{eq:fresnel} &= \frac{e^{2\pi i z/\lambda}}{\sqrt{i \lambda z}} \int g_\text{in} (x) e^{\frac{i \pi}{\lambda z} (x'-x)^2} dx
\end{flalign} 
where $g_\text{in}$ and $g_\text{out}$ represent the complex-valued electric field at the target and the imaging plane respectively.
In this work, $z = 46 \si{\centi\meter}$, and all complex index of refraction data is from \cite{Henke_1993}. 
This model can now be fit to experiment (Figure \ref{fig:comparetomodel}, analyzing the fifth frame from Figure \ref{fig:sortedframes}), constituting a measurement technique to estimate post-shock density. 
See Section SM.III for a more complete derivation of Equation \ref{eq:fresnel} and further explanation of how the computational model was implemented, and also refer to the similar model from our prior work \cite{Campbell2021}.


%We can now compare PCI shock front profiles from experiment to these diffraction model solutions, as long as the portion of the shock front image being considered has sufficient symmetry along a particular axis, which need not be linear.

%Regarding the appropriate value for $\lambda = hc/E_\text{photon}$, the APS's X-ray source was operating in a polychromatic mode consisting of nine peaks ranging from 10keV to 60keV, see Section SM.IV for a plot of the spectrum.
%For this reason it was appropriate to proceed by taking a weighed average of nine monochromatic solutions to Equation \ref{eq:fresnel}, found by using $\lambda = \lambda_\text{peak}$ for each of the nine peaks.

%Although the shock front's position in the frame gives us a cylindrical shock radial dimension that would otherwise need to be taken into account as an additional optimization parameter, by simple inspection of the image itself we know this dimension to high precision, therefore we consider it reasonable in this case to fix this dimension at the value measured from the image, which in this case was $301 \si{\micro\meter}$.

%Results from such a routine are shown in Figure \ref{fig:comparetomodel}; in this example, the fifth frame from Figure \ref{fig:sortedframes} (t=234ns) was analyzed, and was found to have an estimated post-shock density of $\rho_2 = 0.795^{+0.085}_{-0.036} \si{g/cc}$ ($\rho_2/\rho_1 = 1.170^{+0.124}_{-0.053}$).


\begin{figure}[!t]
    \centering
    \includegraphics[width=0.9\linewidth]{fig_thermomodel_annotated.png}
    \caption{
    Hugoniot states in the $\rho_2$--$v_\text{shock}$ space, showing how values estimated from this work's PCI data and diffraction model (black) compare to normal shock thermodynamic relations in heptane both with (green) and without (red) the cavitation bubble compression effect.
    The dashed portion of the red curve indicates extrapolation of the heptane equation of state.
    The blue datapoint corresponds with the particular X-ray diffraction model fit result from Figure \ref{fig:comparetomodel}.
    %The shock velocity and shock speed both decrease with time, illustrated by the arrow annotation. 
    %Note that the experimental data points suggest a higher post-shock density than what is predicted from the normal shock model; this is expected, since the post-shock density estimated by the diffraction model does not take into account the compression caused by the cavitation bubble; the data plotted in green asterisks accounts for this the compression effect, leading to higher densities. 
    %A more complete multiphysics model that couples the cavitation dynamics, fluid dynamics, and shock conditions would presumably better match experiment.
    %for this reason, this plot also includes a second $\rho_2/\rho_1$ vs. $v_\text{shock}$ curve which assumes the ``compression'' ratio implied by the $t=134\si{ns}$ frame in Figure \ref{fig:singleevent}. 
    %Plot of cylindrical shock positions/radii, relative to the axis of the plasma, compiling data from the seven different events for which the relative delay of each frame was known (Figure \ref{fig:sortedframes}). The resulting trend is strongly linear, consistent with the expected near-constant speed of a cylindrically expanding shock.
    %The shock speed presented here in each experimental data point was determined via the time derivative of a quadratic fit of the shock positions in Figure \ref{fig:shockspeed}, and each density ratio was determined via a X-ray diffraction model fit.
    }
    \label{fig:densityVSspeed}
\end{figure}



Alongside this X-ray diffraction method for estimating the density change across the shock from experimental data, it is also possible to relate $\rho_2$ and $v_\text{shock}$ by solving the system of Rankine–Hugoniot thermodynamic shock relations in heptane: $\rho_1 v_1 = \rho_2 v_2$ (mass), $p_1 + \rho_1 v_1^2 = p_2 + \rho_2 v_2^2$ (momentum), $h_1 + \frac{1}{2} v_1^2 = h_2 + \frac{1}{2}v_2^2$ (energy), and a tabulated equation of state for heptane \cite{Span2003}.
See Section SM.I for a derivation of this system of equations.
Implicitly solving this system results in a unique relationship between $\rho_2$ and $v_\text{shock}$, shown in Figure \ref{fig:densityVSspeed} with a red curve.
However, measured values for $\rho_2$ and $v_\text{shock}$ (from the diffraction model and quadratic fit to Figure \ref{fig:shockspeed}) tend to reside above this red curve, suggesting a higher-density post-shock region than what is implied by this thermodynamic model.
We attribute this to the fact that the thermodynamic model ignores the cavitation bubble which follows the expanding shock, clearly visible in Figures \ref{fig:singleevent} and \ref{fig:sortedframes}.
This bubble interface is generated at a relatively small radius ($<10 \si{\micro\meter}$) and high energy density, and expands outward with a large amount of inertia, compressing the post-shock region to a higher density.
This is shown with green asterisks in Figure \ref{fig:densityVSspeed}, generated by assuming that this compression has no effect on the speed of the shock; although this still does not result in a model that sufficiently matches experiment, it represents the maximum post-shock density $\rho_2$ which could be justified by the data.
The original thermodynamic model (red curve) represents the minimum bound on $\rho_2$, since it completely neglects this compression effect.
In reality, the higher $\rho_2$ caused by this compression effect should increase the shock speed, though the full nature of this effect is not explored here; to fully investigate this effect, a multiphysics model would be necessary which couples together the dynamics of the cavitation bubble (Rayleigh-Plesset), the liquid post-shock region (Navier–Stokes), and the jump conditions which govern the shock (Rankine-Hugoniot).


%Despite this model limitation, Figure \ref{fig:densityVSspeed} shows that the $\rho_2/\rho_1$ and $v_\text{shock}$ result from the X-ray diffraction model is consistent with the thermodynamic model result, validating the accuracy of both models.



%This highlights the limitations of this simplified model; this result should therefore be treated as an upper bound on the post-shock density (see dashed red curve in Figure \ref{fig:densityVSspeed}).

%, which we first determined to have the volume of a simple cylinder $\pi R_\text{shock}^2L$, into a smaller volume ($\pi R_\text{shock}^2 - \pi R_\text{bubble}^2 )L$) shaped like a thick-walled cylindrical shell, expressible as the ``compression'' ratio $1/\big(1 - (R_\text{bubble}/R_\text{shock})^2\big)$; for the frame being analyzed in Figure \ref{fig:comparetomodel}, this ratio is 1.13.
%However, the remaining assumption in this scenario is that the cavitation bubble is initially generated as a one-dimensional line and grows exclusively by compressing the post-shock region, while in reality the bubble has some nonzero initial radius formed via chemistry and phase change of liquid heptane.
%The actual ``compression'' ratio would therefore be somewhere between 1 and 1.13.




%When this is done for the t=134ns frame from Figure \ref{fig:sortedframes}, the resulting post-shock thermodynamic conditions report a shock velocity of roughly 1.4km/s, which is consistent with experimental error of the shock speed calculated from the fit line if Figure \ref{fig:shockspeed} (1.38 km/s), validating both the X-ray diffraction model and the thermodynamic model.


%See Figure \ref{fig:comparetomodel} for an example of this comparison, with emphasis on the qualitative similarity between simulated results and experiment.




Similarity of this work to the submerged exploding wire PCI experiments by Yanuka \cite{Yanuka2018} at the European Synchrotron Radiation Facility (ESRF) allows us to directly compare figures of merit for the two different implementations of PCI and of cylindrically expanding shock events, of which the most dramatic difference is peak instantaneous power and total event energy; for Yanuka, these values were approximately 1GW and 300J, respectively (recall 1MW and 100mJ for this work).
The fact that the shock front images presented in this work are still visible with such a small event energy demonstrates the superior sensitivity and utility of PCI as a diagnostic for observing and analyzing propagating shocks.

%, specifically in terms of frame rate (6.5MHz), resolution ($2 \si{\micro\meter}$), and exposure time (33ps).


%It is also worth noting that the manufacturing quality of the LYSO and LuAG scintillators is noticeable in this set of results; by averaging all 128 frames of a single event together (Figure \ref{fig:LuAGvsLYSO}c and \ref{fig:LuAGvsLYSO}d), the resulting reduction in random noise reveals several scratches in the LYSO crystal.
%This is generally a common difference between LYSO and LuAG, due to the higher demand and more widespread use of LuAG for PCI.

In summary, we present successful observation of weak shocks in liquid heptane; this work constitutes a feat in imaging sensitivity and resolution and serves as strong evidence that further utilization of PCI for shock imaging in different media and phases of matter may likely prove fruitful.
The plasma-based method of shock generation used here exhibits an order of magnitude increase in repetition rate (multiple events per minute) over conventional dynamic targets in similar experiments (e.g. exploding wire), and can easily be increased to well over 1Hz given a sufficient data acquisition scheme.
The ability to quickly generate large datasets is potentially useful for machine learning applications; the eighty-five shock fronts cataloged in Figure \ref{fig:shockspeed} could conceivably be used to train a deep learning model which could then rapidly compute shock parameters such as position and density.
A wide range of possible parameter sweeps is achievable using this target, either by changing the input energy via choice of charging capacitor or simply taking advantage of the stochasticity of the phenomena of interest to automatically vary parameters such as imaging delay time, shock shape, or breakdown voltage.
By exchanging the heptane for alternate discharge media (e.g. water, mineral oil, ice, plastics, rock), shock propagation can be studied in these materials without compromising repetition rate.
Future work will push the PCI sensitivity limit further, while at the same time continuing to develop a quantitative analysis toolkit for shock imaging as well as PCI in general.
%; this work represents the next step on this research avenue (refer to prior work on PCI and quantitative analysis of nanosecond plasma initiation phenomena in water \cite{Campbell2021}, from which the X-ray diffraction model in this work was developed).

\begin{acknowledgements}
The authors would like to acknowledge the support of the U.S. DOE NNSA. Los Alamos National Laboratory is managed by Triad National Security, LLC for the U.S. Department of Energy's NNSA. 
This document has been approved for release under No. LA-UR-23-22175.
This research used resources of the Advanced Photon Source, a U.S. Department of Energy (DOE) Office of Science user facility operated for the DOE Office of Science by Argonne National Laboratory under Contract No. DE-AC02-06CH11357.
\end{acknowledgements}

\bibliography{mybib}
\bibliographystyle{ieeetr}


\end{document}

