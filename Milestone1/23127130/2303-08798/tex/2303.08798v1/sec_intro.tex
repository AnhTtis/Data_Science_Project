\section{Introduction}\label{sec:intro}
A \emph{graph} is an order pair $G=(V,E)$, where $V$ is called the set of vertices and $E$ is called the set of edges. The set $E$ is consist of $2$-element subsets of $V$. A subject topological combinatorics is consist of the study of homotopy invariants
of certain cell complexes constructed using the graph to obtain combinatorial information about the graph $G$. The first example of this is the Lov{\'a}sz \cite{Lovaz} celebrated proof of the Kneser conjecture. The neighborhood complex $\mathcal{N}(G)$ of a graph $G$ is an abstract simplicial complex whose simplices are subsets of the vertex set $V$ having common neighbors.
Lov{\'a}sz uses the connectivity of neighborhood complex $\mathcal{N}(G)$ to compute a lower bound on the chromatic number of the corresponding graph. 
Lov{\'a}sz stated the similar conjecture that the lower bound on the chromatic number of a graph $G$ can be given in terms of the connectivity of certain cell complexes associated with cycle graph and $G$. These complexes are known as \emph{hom complexes}. Babson and Kozlov \cite{Lovazconj} proved Lov{\'a}sz conjecture, where they relate these hom complexes to \emph{independence complexes}. 


 %A subset $S \subseteq V$ of the vertex set of $G$, is called an \emph{independent set} if any two vertices of $S$ are non-adjacent. 
 An abstract simplcial complex consist of all independent subsets of $V$ is called the \emph{independence complex} of $G$. It is denoted by $\mathcal{I}(G)$.  
 In the last two decades, the general problem of determining all the possible homotopy type of independence complexes for various classes of graphs has received considerable attention. For example, Kozlov \cite{Koz99} determined the homotopy type of independence complexes for path and cycle graphs, Kawamura \cite{indchordal, indforest} for chordal graphs and forests, Bousquet-M\'{e}lou, Mireille and Linusson, Svante and Nevo, Eran \cite{indsquaregrid} for square-grid graphs, Engstr{\"o}m \cite{Engstrclawfree} investigated this problem for claw-free graphs, and Raun \cite{Braunindstbknes} for stable Kneser graphs  etc. 

The purpose of this paper is to introduce the notion of a wedge of graphs and compute the all possible homotopy type of independence complexes of wedges of some classes of graphs. In particular, we consider the class of wedges of path graphs and cycle graphs. 
We also describe the relationships between combinatorial and topological invariants associated with the wedge of graphs with the corresponding combinatorial and topological invariants of the components. We hope these computations will be helpful in other parts of mathematics.

The paper is organized as follows: We begin \Cref{sec:prelim} by defining some combinatorial and topological objects associated with a graph. Then we introduce the notion of a wedge of graphs. We end this section by recalling some results about the homotopy type of independence complexes of path graphs and cycle graphs. 

Finally in \Cref{sec:res}, we prove the main results of this paper. Here, we compute all possible homotopy types of the wedge of finitely many path graphs and wedge of finitely many different cycle graphs. Next, we consider the wedge of cycle and path graphs with respect to different base points and comupute all possible homotopy types of corresponding independence complexes.

%The \Cref{sec: ind blow} is devoted to the computations of the independence complexes of blow-up graphs.   