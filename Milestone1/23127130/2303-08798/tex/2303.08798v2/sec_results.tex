\section{Main results}\label{sec:res}
We begin this section by proving a sufficient condition for the link of a vertex to be contractible inside deletion. This is crucial in using the \Cref{lem: del link}. Then we compute the homotopy type of wedge of path graphs and a wedge of cycle graphs.  
\begin{lemma}\label{lemma:contractibleReduction}
    Let $G$ be a graph, and $v\in \mathcal{I}(G)$ be a vertex of $G$. Let $\sigma\in \del{v}$ be a maximal simplex such that $\sigma\notin \link{v,\I(G)}$ and $\del{v,\I(G)}\setminus \{\sigma\}$ is contractible. Then
    \begin{displaymath}
        \mathcal{I}(G) \simeq \del{v,\I(G)}\vee \Sigma \left( \link{v,\I(G)}\right).
    \end{displaymath}   
\end{lemma}
\begin{proof}
    Since $\sigma\notin \link{v},~\link{v}\subseteq \del{v}\setminus\{\sigma\}$, and $\del{v}\setminus\{\sigma\}$ is contractible hence we have $\link{v}$ is contractible in $\del{v}$. Then the result follows from \Cref{lem: del link}.
\end{proof}
We start with the case when the wedge is taken to be of two path graphs. We give a complete description of all the cases.
In the later part of the section, this will be generalized in the case of a wedge of finitely many path graphs, using the concept of the terminal wedge.
\begin{theorem}\label{thm:pathWedgePath}
    Let $P_l$ be the path graph on $l$ vertices. Then 
    \begin{equation}\label{eq:pathWedgePath}
        \mathcal{I}\left(P_m \wpoint{a} P_n \right)
    \end{equation}
    is either a point or a sphere.
\end{theorem}
\begin{proof}
    If the wedge point is the terminal point for both of the graphs, then the wedge graph is $P_{m+n-1}$ hence the independence complex can be found using \autoref{thm:ICofPm}. Now suppose that $a=a_m=b_l,~m\ge 4$. Then Using \autoref{lemma:foldLemma} (with $v=a_1$ and $w=a_3$), we have that 
    \begin{align*}
        \mathcal{I}(P_m\wpoint{a}P_n) \simeq \mathbb{S}^0 \star \mathcal{I}(P_{m-3} \wpoint{a} P_n).
    \end{align*} 
    Thus we only need to compute the independence complexes for the cases $m=1,2$ and $3$ (Referring to the diagram \Cref{fig:PmWedgePn_type-1}, we assume $n=l+k$).
    \begin{figure}[H]
        \centering
        \incfig[0.5]{PmWedgePn_type_1}
        \caption{Wedge of path graphs.}
        \label{fig:PmWedgePn_type-1}
    \end{figure}\
    \paragraph{\textcolor{black}{\textbf{Case 1} $m=1$}} In this case the wedge is same as $P_n$.

    \paragraph{\textcolor{black}{\textbf{Case 2} $m=2$}} Take $v=a_1$ and $w=b_{l-1}$. Then using \autoref{lemma:foldLemma}
    \begin{displaymath}
        \mathcal{I}(P_2\wpoint{a}P_n) \simeq \mathcal{I}(P_{l-2}) \star \mathcal{I}(P_{n-l+2}).
    \end{displaymath}  

    \paragraph{\textcolor{black}{\textbf{Case 3} $m=3$}} Take $v=a_1$ and $w=a$. Then using \autoref{lemma:foldLemma},
    \begin{displaymath}
        \mathcal{I}(P_3\wpoint{a}P_n) \simeq \mathbb{S}^0 \star \mathcal{I}(P_{l-1}) \star \mathcal{I}(P_{n-l}).
    \end{displaymath}
    Now, the result follows from \cref{eq:ICofPm}.
\end{proof}
The case of a wedge of the cycle graphs is not straightforward and hence we need to do a more careful analysis. This is an almost replica of Kozlov's work. 
But here we have more than one cycle, so we need to consider the maximal simplex appropriately inside the deletion which doesn't belong to the link. Here is the statement of the main result. 
\begin{lemma}\label{lemma: Gmn as delvslk}
 Let $C_m \wpoint{a} C_n$ be the wedge of cycle graphs and $G_{m,n}=\mathcal{I}\left(C_m \wpoint{a} C_n \right)$. Then \[G_{m,n}\simeq \del{a, G_{m,n})}\vee \sum \link{a,G_{m,n}}.\]
\end{lemma}
\begin{proof}
We denote $\I\left(C_m \wpoint{a} C_n \right)$ by $G_{m,n}$.
 We show that    $\link{a,G_{m,n}}$ is contractible in $\del{a, G_{m,n}}$. 
 Then the result follows from \Cref{lem: del link}. 
 Observe that, in general we have 
\[\link{a,G_{m,n}}\simeq \I(P_{m-2})*\I(P_{n-2}) ~~\text{ and }~~ \del{a,G_{m,n}}\simeq \I(P_{m-1})*\I(P_{n-1}).\]
 We consider the following cases:
 
\noindent{}\textbf{Case 1}\emph{ $n=3k$ or $m=3k'$.}
\vspace{.2mm}

In this case, either $\I(P_{m-2})\sim \I(P_{3(k-1)+1})$ or $\I(P_{n-2})\simeq \I(P_{3(k'-1)+1})$. From \Cref{thm:ICofPm}, we get that either $\I(P_{m-2})$ or $\I(P_{n-2})$ is contractible. Therefore, $\link{a,G_{m,n}}$ is contractible. This, from \Cref{lem: del link} result follows. 

\noindent{}\textbf{Case 2}\emph{ $n=3k+1$ or $m=3k'+1$.}
\vspace{.2mm}

Now consider the maximal simplex  \[\sigma=\{a_1,a_3,\dots,a_{3k-3},a_{3k-1}\}\cup\{b_1,b_3,\dots,b_{3k'-3},b_{3k'-1}\}.\] Then $\sigma\in\del{a,G_{m,n}}\setminus \link{a,G_{m,n}}$. We also have $\del{a,G_{m,n}}\simeq \I(P_{3k})*\I(P_{3k'})\simeq \mathbb{S}^{k+k'+1}$ using \Cref{thm:ICofPm}. Therefore, $\link{a,G_{m,n}}\setminus {\sigma}$ is contractible. Now  the result follows from \Cref{lemma:contractibleReduction}. 

\noindent{}\textbf{Case 2}\emph{ $n=3k+1$ and $m=3k'+2$.}
\vspace{.2mm}

In this case, $\del{a,G_{m,n}}$ is contractible. Therefore, the result follows from \Cref{lem: del link}.

In the remaining cases 

\noindent{}\textbf{Case 3}\emph{ $n=3k+2$ and $m=3k'+1$}
 and 

\noindent{}\textbf{Case 4}\emph{ $n=3k+2$ and $m=3k'+2$}
\vspace{.2mm} we get that $\del{a,G_{m,n}}$  is contractible. Therefore, the result follows from \Cref{lem: del link}.
Finally, in any case we have $G_{m,n}\simeq \del{a, G_{m,n})}\vee \Sigma \link{a,G_{m,n}}$.
\end{proof}
Now we consider the general case, where we have a wedge of $k$-many cycle graphs, with the wedge point to be $a$. Consider the cycle graphs $C_{m_i}$ with vertices $\{a_1^i,a_2^i,\ldots,a_{m_i}^i\}$. Let $G=\vee_{i=1}^kC_{m_i}$. Then observe that \[\link {a,\I(\bigvee_{i=1}^k C_{m_i})}\simeq *_{i=1}^k\I(P_{m_i-2}) ~~\text{ and }~~ \del{v,\I(\bigvee_{i=1}^k C_{m_i})}\simeq *_{i=1}^k\I(P_{m_i-1}).\] Now one can see that if one of the $m_i$ is of the form $3k$ or $3k+2$, then either $\link{a,}$  or $\del{a,}$ is contractible. Now if all $m_i$'s are of the form $3l_{i}+1$, then consider the maximal simplex  \[\sigma=\{a^1_1,a^1_3,\dots,a^1_{3k-3},a^1_{3l_1-1}\}\cup\dots\cup\{a^k_1,a^k_3,\dots,a^k_{3k'-3},a^k_{3l_k-1}\}.\] 
Note that $\sigma\in \del{a, G_{m_1,\dots,m_k}}\setminus \link{a,G_{m_1,\dots,m_k}}$. Observe that $\del{a,G_{m_1,\dots,m_k}}\simeq \mathbb{S}^{\Sigma_{i}^{k}l_i-1}$. Therefore, $\del{a,G_{m_1,\dots,m_k}}\setminus\{\sigma\}$ is contractible.
Therefore, using \Cref{lemma:contractibleReduction} we have the following.

\begin{lemma}\label{lem:gen-cyc-wedge}
Let $G=\vee_{i=1}^{k}C_{m_i}$ be the wedge of $k$-many cycle graphs and $G_{m_1,\dots,m_k}$ be its independence complex. Then $G_{m_1,\dots,m_k}\simeq \del{a,G_{m_1,\dots,m_k}}\vee \Sigma \link{a,G_{m_1,\dots,m_k}}$.    
\end{lemma}
We are now ready to compute the homotopy type of the independence complex of a wedge of cycle graphs. Although this will be further generalized in the later part, we present the complete computation to ease the reader's mind.
\begin{theorem}\label{thm:cycleWedgeCycle}
Let $C_m \wpoint{a} C_n$ be the wedge of cycle graphs and $G_{m,n}$ be the independence complex of $C_m \wpoint{a} C_n$. Then $G_{m,n}$ is contractible or homotopy equivalent to  wedge of spheres.
%if one of the following holds:
%\begin{enumerate}\item $m=3k$ and $n=3k'+1$\item $m=3k+1$ and $n=3k'$\item $m=3k+2$ and $n=3k'+1$\item  $m=3k+2$ and $n=3k'+2$\item $m=3k+1$ and $n=3k'+2$\end{enumerate}.Then\[\mathcal{I}\left(C_m \wpoint{a} C_n \right)=\begin{cases}
 %   \mathbb{S} ^{k+k^\prime -1} \vee \mathbb{S} ^{k+k^\prime } & \text{ if } m=3k+1 \text{ and } n=3k'+1\\
%    * & \text{ if }
%   \end{cases}\] \textcolor{red}{complete the statement}
\end{theorem}
\begin{proof}
%Let us denote $\mathcal{I}\left(C_m \wpoint{a} C_n \right)$ by $G_{m,n}$. Consider the following two sets
% \begin{align*}
%        A=\St_{G_{m,n}}(a)\text{ and }B&=G_{m,n}\setminus\{a\}\\
%        &=P_m \star P_n.
  %  \end{align*}
 %   If $n-1$ or $m-1$ is of the form $3k+1$, then $B$ is contractible, hence we get that
 %   \begin{align*}
 %       G_{m,n}&\cong \Sigma(\link{a})\\
 %       &=\Sigma(A\cap B).
 %   \end{align*}
 %   \begin{table}[!h]
 %       \centering
%        \begin{tabular}{|c|c|c|c|}
%            \hline
%            \backslashbox{$n-2$ }{$m-2$ }  & $3k$  & $3k+1$  & $3k+2$    \\
  %          \hline
   %         $3k^\prime $   & $\mathbb{S} ^{k+k^\prime -1}$  & \text{pt} & $\mathbb{S} ^{k+k^\prime }$   \\
%                \hline
%                $3k^\prime +1$  & \text{pt} & \text{pt} &  \text{pt} \\
 %               \hline
 %               $3k^\prime +2$  & $\mathbb{S} ^{k+k^\prime }$  & \text{pt} & $\mathbb{S} ^{k+k^\prime +1}$   \\
 %           \hline
 %       \end{tabular}
 %       \caption{$\mathcal{I} \left( P_{m-2} \right)  ~{\star}~ \mathcal{I} \left( P _{n-2} \right) $ }
        \label{tab:label}
 %   \end{table}

We consider the following cases: 

\noindent{}\textbf{Case 1} \emph{either  $n=3k$ or $m=3k'$}
\vspace{.2mm}

In this case $\link{a,G_{m,n}}$ is contractible as we saw in the first case of \Cref{lemma: Gmn as delvslk}. Therefore, by \Cref{lemma: Gmn as delvslk}, we have $G_{m,n}\simeq \del{a,G_{m,n}}\simeq \I(P_{m-1})*\I(P_{n-1})$. Now we have following subcases using \Cref{thm:ICofPm}: 
\begin{enumerate}
\item \emph{If $m=3k$ and $n=3k$}. In this case, $G_{m,n}\simeq\del{a, G_{m,n}}\simeq \mathbb{S}^{k-1}*\mathbb{S}^{k'-1}=\mathbb{S}^{k+k'-1}$.
\item \emph{If $m=3k$ and $n=3k'+1$.} In this case, $G_{m,n}\simeq\del{a, G_{m,n}}\simeq \mathbb{S}^{k-1}*\mathbb{S}^{k'}=\mathbb{S}^{k+k'}$.
\item \emph{If $m=3k$ and $n=3k'+2$.} In this case, $G_{m,n}\simeq\del{a, G_{m,n}}\simeq \mathbb{S}^{k-1}* \{pt\}=\{pt\}$.
\item \emph{If $m=3k+1$ and $n=3k'$.} In this case, $G_{m,n}\simeq\del{a, G_{m,n}}\simeq \mathbb{S}^{k}*\mathbb{S}^{k'-1}=\mathbb{S}^{k+k'}$.
\item \emph{If $m=3k+2$ and $n=3k'$.} In this case, $G_{m,n}\simeq\del{a, G_{m,n}}\simeq \{pt\}* \mathbb{S}^{k'-1} =\{pt\}$.
\end{enumerate}

\noindent{}\textbf{Case 2} \emph{$m=3k+1$, $n=3k^\prime +1$.}
\vspace{.2mm}

Consider the following diagram \Cref{fig: Gmn} of wedge of two cycles where the wedge point $a$ is taken to be $a_{3k+1}=b_{3k^\prime +1}$. 
    %\textcolor{red}{Insert the diagram from page 82}.

\begin{figure}[h]
    \centering
    \includegraphics[scale=1]{Gmn.pdf}
    \caption{Wedge of $C_{3k+1}$ and $C_{3k'+1}$}
    \label{fig: Gmn}
\end{figure}

    \noindent Consider the simplex
    \[\sigma = \left\{ a_1,a_3,a_6,a_9,\cdots,a_{3k} \right\} \cup \left\{ b_1,b_3,b_6,\cdots,b_{3k^\prime } \right\} .
    \] 
    Since $\sigma \in \del{a}\setminus \link{a}$ and $\del{a}\setminus \{\sigma \}$ is contractible in $\del{a}$, we get using \Cref{lemma:contractibleReduction} that 
    \begin{align*}
        G_{m,n} & = \del{a,G_{m,n}}\vee \Sigma \link{a,G_{m,n}}\\
        & = \mathbb{S} ^{k+k^\prime -1} \vee \mathbb{S} ^{k+k^\prime},
    \end{align*}
    as $\del{a,G_{m,n}}\simeq \I(P_{m-1})*\I(P_{n-1})\simeq \I(P_{3k})*\I(P_{3k'})\simeq \mathbb{S}^{k+k'-1}$ and $\link{a, G_{m,n}}\simeq \I(P_{m-2})*\I(P_{n-2})\simeq \I(P_{3k-1})*\I(P_{3k'-1})\simeq \I(P_{3(k-1)+2})*\I(P_{3(k'-1)+2})\simeq \mathbb{S}^{k+k'-1}$.

\noindent{}\textbf{Case 3} \emph{$m=3k+2$, $n=3k^\prime +1$.}
\vspace{.2mm}  

In this case, we have $\del{a,G_{m,n}}$
is contractible as shown in \Cref{lemma: Gmn as delvslk}. Therefore, using \Cref{lemma: Gmn as delvslk}, we have $G_{m,n}\simeq \Sigma \link{a,G_{m,n}}$. Therefore,
\begin{align*}
    G_{m,n} & \simeq \Sigma \bigg(\I(P_{m-2})*\I(P_{n-2})\bigg)\\
    &\simeq \Sigma \bigg(\I(P_{3k})*\I(P_{3(k'-1)+2})\bigg)\\
    &\simeq \Sigma \bigg(\mathbb{S}^{k-1}*\mathbb{S}^{k'-1} \bigg) \simeq \mathbb{S}^{k+k'}.
\end{align*}

\noindent{}\textbf{Case 4} \emph{$m=3k+1$, $n=3k^\prime +2$.}
\vspace{.2mm}  

In this case, we have $\del{a,G_{m,n}}$
is contractible as shown in \Cref{lemma: Gmn as delvslk}. Therefore, using \Cref{lemma: Gmn as delvslk}, we have $G_{m,n}\simeq \Sigma \link{a,G_{m,n}}$. Therefore,
\begin{align*}
    G_{m,n} & \simeq \Sigma \bigg(\I(P_{m-2})*\I(P_{n-2})\bigg)\\
    &\simeq \Sigma \bigg(\I(P_{3(k-1)+2})*\I(P_{3k'})\bigg)\\
    &\simeq \Sigma \bigg(\mathbb{S}^{k-1}*\mathbb{S}^{k'-1} \bigg) \simeq \mathbb{S}^{k+k'}.
\end{align*}


\noindent{}\textbf{Case 5} \emph{$m=3k+2$, $n=3k^\prime +2$.}
\vspace{.2mm}  

In this case, we have $\del{a,G_{m,n}}$
is contractible as shown in \Cref{lemma: Gmn as delvslk}. Therefore, using \Cref{lemma: Gmn as delvslk}, we have $G_{m,n}\simeq \Sigma \link{a,G_{m,n}}$.
Therefore,
\begin{align*}
    G_{m,n} & \simeq \Sigma \bigg(\I(P_{m-2})*\I(P_{n-2})\bigg)\\
    &\simeq \Sigma \bigg(\I(P_{3k})*\I(P_{3k'})\bigg)\\
    &\simeq \Sigma \bigg(\mathbb{S}^{k-1}*\mathbb{S}^{k'-1} \bigg) \simeq \mathbb{S}^{k+k'}.
\end{align*}
This proves the theorem.
\end{proof}
Now we present the generalization of the previous theorem. We hope the path to generalization will be clear to the reader. We present the theorem without a detailed proof here, to avoid cumbersomeness.
\begin{theorem}
Let $\bigvee_{i=1}^k C_{m_i}$ be the wedge of $k$-many cycle graphs. Then $\I(\bigvee_{i=1}^k C_{m_i})$ is contractible or wedge of spheres.   
\end{theorem}
\begin{proof}
The proof follows from using  \Cref{lem:gen-cyc-wedge}  and induction on $k$.
\end{proof}

The next class of graphs is a wedge of a path and a cycle graph. Assume $C_n$ to be the cycle graph whose vertex set is $\{a_1,\dots, a_n\}$ oriented counterclockwise and $P_m$ be the path graph with vertex set is $\{b_1\dots, b_n\}$. For $1\leq k\leq \lceil n/2 \rceil$, consider the wedge graph $G_k=C_n\wpoint{c_k=a_1\sim b_k}P_m$. Note that the function $\lceil~\rceil$ appears to avoid obvious isomorphic classes of graphs. See \cref{fig:my_label} for a visual. Next we compute the independence complex of this graph.
\begin{figure}[h]
    \centering
    \includegraphics[scale=.9]{CnVPm.pdf}
    \caption{A graph $G_k$}
    \label{fig:my_label}
\end{figure}
\begin{theorem}\label{thm:pathWedgeCycle}
 The independence complex  $\mathcal{I}(G_k)$ is homotopy equivalent to either a point or wedge of spheres.
 %one of the following: $*$, $S^{\alpha-2}$, $S^{\alpha-1}$, $S^{\alpha}$, $S^{\alpha}\vee S^{\alpha+1}$ and $S^{\alpha+1}$. 
\end{theorem}
\begin{proof}
Let $c_k$ be the wedge point of the graph $G_k$. 
Let $C_n(n-3)$ be the subgraph of $G_k$ on vertices $\{a_3,\dots,a_{n-1}\}$. One can see that $C_n(n-3)$ is isomorphic to the path graph $P_{n-3}$ on $n-3$ vertices. Let $P_m(k-2)$ be the subgraph of $P_m$ on vertices $\{b_1,\dots,b_{k-2}\}$ and $P_m(m-k-1)$ be the subgraph of $P_m$ on $\{b_{k+2},\dots, b_{m}\}$. One can see that $P_{m}(k-2)$ is isomorphic to the path graph $P_{k-2}$ and $P_{m}(m-k-1)$ is isomorphic to $P_{m-k-1}$.
Note that the link $\Lk(c_k,\I(G_k))$ can be described as 
\[\bigg \{ A\sqcup B\sqcup C \mid A\in \I(C_n(n-3)), B\in P_m(k-2) \text{ and } C\in  P_{m}(m-k-1)  \bigg\}.\]

Therefore, we get the following \[\Lk(c_k,\I(G_k))\simeq \I(P_{n-3})*\I(P_{k-2})*\I(P_{m-k-1})\] and \[\Dl(c_k,\I(G_k))\simeq \I(P_{n-1})*\I(P_{k-1})*\I(P_{m-k}).\] It is easy to see that $\Lk(c_k,\I(G_k))$ is contractible in $\Dl(c_k,\I(G_k))$. Then by \Cref{lem: del link}, we get $\I(G_k)\simeq
\Dl(c_k,\I(G_k))\vee \Sigma \Lk(c_k,\I(G_k))$. \textcolor{black}{Let $\alpha =a+b+c$. }

\begin{table}[H]
    \centering
    \begin{tabular}{|c|c|c|c|c|c|c|c|}
        \hline
        & & \multicolumn{3}{c|}{$\Dl$} & \multicolumn{3}{c|}{$\Lk$} \\
        \hline
        & \backslashbox{$m-k$ }{$k$ } & $3b$ & $3b+1$ & $3b+2$ & $3b$ & $3b+1$ & $3b+2$  \\ 
        \hline 
        \multirow[c]{3}{*}{$n=3a$} & $3c$ & $\mathbb{S} ^{\alpha-1}$ & $\mathbb{S} ^{\alpha-1}$ & \text{pt} & \text{pt} & \text{pt} & \text{pt} \\ 
        \cline{2-8}
        & $3c+1$ & \text{pt} & \text{pt} & \text{pt} & \text{pt} & $\mathbb{S} ^{\alpha}$ & $\mathbb{S} ^{\alpha}$ \\ 
        \cline{2-8}
        & $3c+2$ & $\mathbb{S} ^{\alpha}$ & $\mathbb{S} ^{\alpha}$ & \text{pt} & \text{pt} & $\mathbb{S} ^{\alpha}$ & $\mathbb{S} ^{\alpha}$ \\ 
        \hline
        \multirow[c]{3}{*}{$n=3a+1$} & $3c$ & $\mathbb{S} ^{\alpha-2}$ & $\mathbb{S} ^{\alpha-2}$ & \text{pt} & \text{pt} & \text{pt} & \text{pt} \\ 
        \cline{2-8}
        & $3c+1$ & \text{pt} & \text{pt} & \text{pt} & \text{pt} & \text{pt} & \text{pt} \\ 
        \cline{2-8}
        & $3c+2$ & $\mathbb{S} ^{\alpha-1}$ & $\mathbb{S} ^{\alpha-1}$ & \text{pt} & \text{pt} & \text{pt} & \text{pt} \\ 
        \hline
        \multirow[c]{3}{*}{$n=3a+2$} & $3c$ & \text{pt} & \text{pt} & \text{pt} & \text{pt} & \text{pt} & \text{pt} \\ 
        \cline{2-8}
        & $3c+1$ & \text{pt} & \text{pt} & \text{pt} & \text{pt} & $\mathbb{S} ^{\alpha+1}$ & $\mathbb{S} ^{\alpha+1}$ \\ 
        \cline{2-8}
        & $3c+2$ & \text{pt} & \text{pt} & \text{pt} & \text{pt} & $\mathbb{S} ^{\alpha+1}$ & $\mathbb{S} ^{\alpha+1}$ \\ 
        \hline
    \end{tabular}
    \caption{Computation for $\Dl$  and $\Lk$ }
    \label{tab:label1}
\end{table}

\begin{table}[H]
    \centering
    \begin{tabular}{|c|c|c|c|c|}
        \hline
        & \backslashbox{$m-k$ }{$k$ } & $3b$ & $3b+1$ & $3b+2$\\ 
        \hline 
        \multirow[c]{3}{*}{$n=3a$} & $3c$ & $\mathbb{S} ^{\alpha-1}$ & $\mathbb{S} ^{\alpha-1}$ & \text{pt}\\ 
        \cline{2-5}
        & $3c+1$ & \text{pt} & $\mathbb{S} ^{\alpha+1}$  & $\mathbb{S} ^{\alpha+1}$ \\ 
        \cline{2-5}
        & $3c+2$ & $\mathbb{S} ^{\alpha}$ & $\mathbb{S} ^{\alpha}\vee \mathbb{S} ^{\alpha+1} $ & $\mathbb{S} ^{\alpha+1}$ \\ 
        \hline
        \multirow[c]{3}{*}{$n=3a+1$} & $3c$ & $\mathbb{S} ^{\alpha-2}$ & $\mathbb{S} ^{\alpha-2}$ & \text{pt}\\ 
        \cline{2-5}
        & $3c+1$ & \text{pt} & \text{pt} & \text{pt} \\ 
        \cline{2-5}
        & $3c+2$ & $\mathbb{S} ^{\alpha-1}$ & $\mathbb{S} ^{\alpha-1}$ & \text{pt}\\ 
        \hline
        \multirow[c]{3}{*}{$n=3a+2$} & $3c$ & \text{pt} & \text{pt} & \text{pt}\\ 
        \cline{2-5}
        & $3c+1$ & \text{pt} & $\mathbb{S} ^{\alpha+2}$  & $\mathbb{S} ^{\alpha+2}$ \\ 
        \cline{2-5}
        & $3c+2$ & \text{pt} & $\mathbb{S} ^{\alpha+2}$  & $\mathbb{S} ^{\alpha+2}$ \\ 
        \hline
    \end{tabular}
    \caption{Independence complex}
    \label{tab:label2}
\end{table}
\end{proof}

Finally, we will compute the independence complex of a wedge of $k$ many path graphs. Note that the wedge graphs vary with the choice of wedge point. Furthermore, the graph obtained after choosing one wedge point, it can be viewed as \emph{terminal wedge of path graphs}.

\begin{figure}[!htb]
    \centering
    \incfig[0.5]{path-wedge-path-1}
    \caption{$P_7\vee P_7 \vee P_5 = P_4\vee P_4\vee P_4\vee P_4\vee P_3\vee P_3$\label{fig:PathWedgePathMany}}
\end{figure}

\begin{theorem}\label{thm:pathWedgePathMany}
    The independence complex $\I \left(\bigvee_{i=1}^k P_{m_i}\right)$ of terminal wedge $k$ many path graphs is either a point or a sphere.
\end{theorem}

\begin{proof}
    \begin{figure}[!htb]
        \centering
        \incfig[0.8]{path-wedge-path-2}
        \caption{Terminal wedge of path graphs\label{fig:sample}}
    \end{figure}
    First we focus on the case if one of them is of the form $3m$. Without loss of generality we will assume that $m_1=3m$. 
    \begin{figure}[h]
        \centering
        \includegraphics[scale=.9]{reduction-fold-lemma.pdf}
        \caption{Reduction step}
        \label{fig:reduction-step}
    \end{figure}
    Applying \autoref{lemma:foldLemma} (see \Cref{fig:reduction-step}) to $P_{m_1}$, we get that 
    \begin{equation}\label{eq:PathWedge3m}
        \I \left(P_{m_1}\vee P_{m_2}\vee\cdots\vee P_{m_k}\right) = \left(\I(P_2)\right)^{\star(m)} \star \I(P_{m_2-1}) \star \cdots \star \I(P_{m_k-1}).
    \end{equation} 
    Next, we will compute the independence complex of a wedge of path graphs when all of them are either of the form $3l_i+1$ or $3l_i+2$. Now we consider the case if all of them are of the form $3l_i+1$. Then using the \autoref{lemma:foldLemma} (See \cref{fig:reduction-step}), we get that independence complexes 
    \begin{align*}
         \I\left(\bigvee_{i=1}^k P_{m_i}\right) & = \I(\{a\}) \star \I(P_2)^{\star (l_1)} \star \I(P_2)^{\star(l_2)} \star \cdots \star \I(P_2)^{\star(l_k)} \\
         & = \textup{pt}.
    \end{align*}   
    Next, assume that all $m_i$'s are of the form $3l_i+2$. In this case, using \autoref{lemma:foldLemma}  (See \cref{fig:reduction-step}), we get that independence complexes 
    \begin{align*}
        \I\left(\bigvee_{i=1}^k P_{m_i}\right) & = \I(P_2) \star \I(P_2)^{\star (l_1)} \star \I(P_2)^{\star(l_2)} \star \cdots \star \I(P_2)^{\star(l_k)} \\
        & = \mathbb{S}^{l_1 + \cdots + l_k}.
   \end{align*} 
   Finally, we will compute if some of them are of the form $3l_i+1$ and some are of the form $3l_i+2$. We assume that $k=t+r$, where $m_i$'s are of the form $3l_i+1$ for $i=1,2,\cdots,t$ and 
   the rest of them are of the form $3l_i+2$ for $i=t+1,\cdots,t+r$. Applying \autoref{lemma:foldLemma} repeatedly (See \cref{fig:reduction-step}), we obtain 
   \begin{align*}
        I \left(\bigvee_{i=1}^k P_{m_i}\right) & = \mathbb{S}^{l_1+\cdots+l_k-1}.
   \end{align*}
\end{proof}
