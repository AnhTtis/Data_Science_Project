\section{Preliminaries}\label{sec:prelim}
In this section, we will define the main object of our study.
A couple of examples will follow this. 
Also, we will mention the \emph{fold lemma}, which will be a main ingredient to some of the proofs. Furthermore, we will mention some of the previously proven results, which will be 
used to prove our main theorems. 

\begin{definition}
An abstract simplicial complex $K$ is a collection of subsets of $\{v_1,\dots, v_n\}$, such that 
\begin{enumerate}
\item $\{ v_i \} \in K$ for all $1\leq i\leq n$,
\item if $\sigma \in K$ and $\tau\subseteq \sigma$, then $\tau\in K$.
\end{enumerate}
\end{definition}
The elements of $K$ are called faces. The \emph{dimension} of a face $\sigma$ is defined as $|\sigma|-1$. 
In this paper, by simplicial complex we mean a geometric realization of an abstract simplicial complex. Without loss of generality we use simplicial complex for an abstract simplicial complex.

Now we define an important simplicial complex associated with a graph.
\begin{definition}[Independence complex]
For a finite simple graph $G$, with the vertex set $V$, the independence complex $\mathcal{I}(G)$ is the simplicial complex consisting of
all independent (i.e. no two vertices are adjacent) subsets of $V$ as its simplices.
\end{definition}
The independence complex of a graph $G$ is denoted by $\mathcal{I}(G)$.

\begin{example}
We now see some examples of the independence complex of graphs.
\begin{enumerate}
\item Let $P_3$ be a path graph on theree vertices $\{1,2,3\}$. Then  
\[\I(P_3)=\bigg\{\{1\},\{2\},\{3\},\{1,3\}\bigg\}.\]
Hence the independence complex is homotopy equivalent to $\mathbb{S}^0$, which will be written as $\mathcal{I}(P_3)\cong \mathbb{S}^0$ (see \Cref{fig:indp3}).
\begin{figure}[H]
    \centering
    \includegraphics[scale=0.9]{Indp3.pdf}
    \caption{Path graph $P_3$ and $\I(P_3)$.}
    \label{fig:indp3}
\end{figure}

\item Let $C_4$ the cycle graph on $4$ vertices $\{1,2,3,4\}$. Then the independence complex is 
\[\I(C_4)=\bigg\{\{1\},\{2\},\{3\},\{4\},\{1,3\}, \{2,4\}\bigg\}.\]
Hence we get that $\mathcal{I}(C_4)\cong \mathbb{S}^0$ (see \Cref{fig:indc4}).
\begin{figure}[H]
    \centering
    \includegraphics[scale=0.9]{indc4.pdf}
    \caption{Cycle graph $C_4$ and $\I(C_4)$.}
    \label{fig:indc4}
\end{figure}
\end{enumerate}

\end{example}





A \emph{subcomlex} of a simplicial complex is a simplicial complex whose faces are contained in $K$.
There are two important subcomplexes associated with any simplicial complex.

\begin{definition}
Let $K$ be a (abstract) simplicial complex. The link of a vertex $v\in K$ is defined as 
\[\Lk (v,K) :=\{\sigma\in K \mid v\notin \sigma \text{ and } \sigma\cup {v}\in K \}.\]    
\end{definition}

\begin{definition}
Let $K$ be a (abstract) simplicial complex. The deletion of a vertex $v\in K$ is defined as 
\[\Dl(v,K) :=\{\sigma\in K \mid v\notin \sigma \} .\]    
\end{definition}
Observe that for any vertex in $K$, the subcomplex $\Lk(v,K)$ is a subcomplex of $\Dl(v,K)$.
The following is an important result which describes the homotopy type of a simplicial complex in terms of the link and deletion.
\begin{lemma}[{\cite[Lemma 2]{singhgrid}\label{lem: del link}}]
Let $K$ be a simplicial complex and $v\in K$ be a vertex such that $\Lk(v,K)$ is contractible in $\Dl(v,K)$. Then $K\simeq \Dl(v,K)\vee \sum \Lk(v,K)$.   
\end{lemma}
Now we define the central object of study in this paper. This will be followed by a few examples. Note that the definition depends on the choice of the so-called `wedge point'.
\begin{definition}[Wedge of graphs]
Given a finite family of graphs $G_i=(V_i,E_i)$, and $a_i\in V_i$ for all $i$, an \emph{wedge of graphs} is defined to be a graph $G=(V,E)$ such that 
\begin{align*}
        V&=\left(\bigcup_i V_i\setminus\{a_i\}\right)\cup \{a\},\\
        E&=\left(\bigcup_iE_i\setminus\{e_j\in E_i|a_i\in\partial(e_j)\}\right)\cup\left\{ab_k|a_ib_k\in E_i\right\}.
\end{align*}
    The common point $a$ will be called \emph{wedge point}.
\end{definition}

\begin{remark}
Let $\chi(G)$ be the chromatic number of a graph. Then we can observe that \[\chi(G_1\vee G_2)=\text{max}\{\chi(G_1),\chi(G_2)\}.\]    
\end{remark}

\begin{example}
    Consider two path graphs $P_3$ and $P_4$  on $3,4$ vertices respectively. Then choosing different wedge points, we can obtain different wedge graphs. We describe some of them here (see \Cref{fig:p3p4wedge}).

    \begin{figure}[H]
        \centering
        \includegraphics[scale=.7]{p3p4wedge.pdf}
        \caption{Different wedges graphs of $P_3$ and $P_4$}
        \label{fig:p3p4wedge}
    \end{figure}
\end{example}
\begin{remark}
Note that if we vary the wedge points, the obtained wedge graphs \textcolor{black}{need not be} isomorphic. Furthermore, they do not have a homotopic independence complex as well.
Consider the first and the last wedge graph of $P_3$ and $P_4$ described in \Cref{fig:p3p4wedge}. Let us denote them by $G$ and $H$, respectively. Then one can see that $G$ and $H$ are not isomorphic. Moreover,  $\I(G)\simeq \star$ and $\I(H)\simeq S^1$.
\end{remark}
Note that we have introduced the main two terms of the title, we mention one of the key ingredients for the proof of our theorems. \begin{theorem}[Fold lemma]\label{lemma:foldLemma}\cite[Lemma 3.4]{Engstrclawfree}
    Let $G$ be a graph, and $v\neq w$ vertices of $G$. If $N(v)\subseteq N(w)$ then the inclusion $\mathcal{I}(G\setminus w )\hookrightarrow \mathcal{I}(G)$ is a homotopy equivalence.  
\end{theorem}

We will end this section by mentioning two important results, due to Kozlov. This will be required in the proof of our theorem, as we are going to consider a few classes of wedges of these graphs. 

\begin{theorem}[{\cite[Proposition 4.6]{Koz99}\label{thm:ICofPm}}]
    Let $P_m$ be the path graph on $m$ vertices. Then
    \begin{equation}\label{eq:ICofPm}
        \mathcal{I}(P_m) \simeq 
        \begin{cases}
            \mathbb{S}^{k-1},& \text{ if } m=3k \\
            \text{pt},       & \text{ if } m=3k+1 \\
            \mathbb{S}^{k},  & \text{ if } m=3k+2.
        \end{cases}
    \end{equation}
\end{theorem}

\begin{theorem}[{\cite[Proposition 5.2]{Koz99}\label{thm:ICofCn}}]
    Let $C_n$ be the cycle graph on $n$ vertices. Then
    \begin{equation}\label{eq:ICofCn}
        \mathcal{I}(C_n) \simeq 
        \begin{cases}
            \mathbb{S}^{k-1}\vee \mathbb{S}^{k-1},& \text{ if } n=3k \\
            \mathbb{S}^{k-1},       & \text{ if } n=3k+1 \\
            \mathbb{S}^{k},  & \text{ if } n=3k+2.
        \end{cases}
    \end{equation}
\end{theorem}
