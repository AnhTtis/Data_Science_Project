\usepackage{
    amsmath,  amsfonts,  amssymb,   amsthm,    amscd,
    gensymb,   comment,   etoolbox,  url,
    booktabs, stackrel,  mathtools, enumitem,  mathdots,  microtype, lmodern,  mathrsfs,   tikz,  
    longtable, tabularx, float,     tikz,      pst-node, tikz-cd,   multirow, tabularx,  amscd,     bm, 
    array,     makecell, diagbox,   booktabs,  ragged2e, caption, subcaption}
\usepackage{slashbox}
\usepackage{graphicx,wrapfig}
\graphicspath{{./figures/}}
\usepackage{xcolor}
\usepackage[utf8]{inputenc}
\usepackage{microtype, fullpage, wrapfig,textcomp,csquotes,fbb}
\usepackage[colorlinks=true, linkcolor=blue, citecolor=blue, urlcolor=blue, breaklinks=true]{hyperref}
\usepackage[capitalise,nameinlink]{cleveref}
\setlength{\marginparwidth}{2cm}
\usepackage{todonotes}
\usetikzlibrary{positioning}
\usetikzlibrary{shapes,arrows.meta,calc}
% \tikzset{
% 	block/.style={rectangle, draw,  text width=2em,
% 		text centered, rounded corners, minimum height=1.5em},
% 	arrow/.style={-{Stealth[]}}
% }
% \tikzset{
% 	every node/.style={font=\sffamily\small},
% 	main node/.style={thick,circle,draw,font=\sffamily\Large}
% }

%-------------------------------------------------------------------------
% Theorem and lemma style 
%------------------------------------------------------------------
\usetikzlibrary{arrows}
\newcommand{\bigzero}{\mbox{\normalfont\Large\bfseries 0}}
\newcommand{\rvline}{\hspace*{-\arraycolsep}\vline\hspace*{-\arraycolsep}}
\newtheorem{theorem}{Theorem}[section]
\newtheorem{acknowledgement}[theorem]{Acknowledgement}
\newtheorem{conjecture}[theorem]{Conjecture}
\newtheorem{corollary}[theorem] {Corollary}
\newtheorem{definition}[theorem]{Definition}
\newtheorem{example}[theorem]{Example}
\newtheorem{lemma}[theorem]{Lemma}
\newtheorem{notation}[theorem]{Notation}
\newtheorem{observation}[theorem]{Observation}
\newtheorem{problem}[theorem]{Problem}
\newtheorem{proposition}[theorem]{Proposition}
\newtheorem{remark}[theorem]{Remark}
\newtheorem{question}[]{Question}

%---------------------
% Math definitions 
%--------------------
\def\S{\mathbb{S}}
\newcommand\Q{\mathbb{Q}}
\newcommand\R{\mathbb{R}}
\newcommand\Z{\mathbb{Z}}
\newcommand{\cs}{\mathcal{S}}
\newcommand{\TC}{\mathrm{TC}}
\newcommand{\Lk}{\mathrm{lk}}
\newcommand{\Dl}{\mathrm{del}}
\newcommand{\St}{\mathrm{St}}
\newcommand{\A}{\mathcal{A}}
\newcommand{\I}{\mathcal{I}}
\newcommand{\co}{\mathrm{coind}}
\newcommand{\ind}{\mathrm{ind}}
\newcommand{\hts}{\mathrm{ht}}
\def\mbar{\overline{M}}
\newcommand{\malpha}{\mathrm{M}_{\alpha}}
\newcommand{\mbalpha}{\overline{\mathrm{M}}_{\alpha}}
\newcommand{\colim}{\mathrm{colim}}
\newcolumntype{x}[1]{>{\centering\arraybackslash}p{#1}}
\newcommand\diag[4]{%
  \multicolumn{1}{p{#2}|}{\hskip-\tabcolsep
  $\vcenter{\begin{tikzpicture}[baseline=0,anchor=south west,inner sep=#1]
  \path[use as bounding box] (0,0) rectangle (#2+2\tabcolsep,\baselineskip);
  \node[minimum width={#2+2\tabcolsep},minimum height=\baselineskip+\extrarowheight] (box) {};
  \draw (box.north west) -- (box.south east);
  \node[anchor=south west] at (box.south west) {#3};
  \node[anchor=north east] at (box.north east) {#4};
 \end{tikzpicture}}$\hskip-\tabcolsep}}

\def\sgnote#1{\textcolor{magenta}{#1}}


\hbadness=99999
\hfuzz=999pt

% = = Newcommand added = = = =  = = 
\newcommand{\del}[1]{\textup{del}{\left(#1\right)}}
\newcommand{\link}[1]{\textup{lk}{\left(#1\right)}}
\newcommand{\wpoint}[1]{\underset{#1}{\vee}}

\usepackage{import}
\usepackage{xifthen}
% \pdfminorversion=7
\usepackage{pdfpages}
\usepackage{transparent}
\newcommand{\incfig}[2][1]{%
    \def\svgwidth{#1\columnwidth}
    \import{./figures/}{#2.pdf_tex}
}


% Table rowspan and colspan
\usepackage{multirow}
\renewcommand{\arraystretch}{1.5}

\renewcommand{\star}{*}