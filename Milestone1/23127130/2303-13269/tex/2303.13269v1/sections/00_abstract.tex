With the increasing ubiquity of cameras and smart sensors, humanity is generating data at an exponential rate. 
Access to this trove of information, often covering yet-underrepresented use-cases (\eg, AI in medical settings) could fuel a new generation of deep-learning tools. 
However, eager data scientists should first provide satisfying guarantees \wrt the privacy of individuals present in these untapped datasets. 
This is especially important for images or videos depicting faces, as their biometric information is the target of most identification methods.
While a variety of solutions have been proposed to de-identify such images, they often corrupt other non-identifying facial attributes that would be relevant for downstream tasks.

In this paper, we propose \textit{Disguise}, a novel algorithm to seamlessly de-identify facial images while ensuring the usability of the altered data. Unlike prior arts, we ground our solution in both differential privacy and ensemble-learning research domains. Our method extracts and swaps depicted identities with fake ones, synthesized via variational mechanisms to maximize obfuscation and non-invertibility; while leveraging the supervision from a mixture-of-experts to disentangle and preserve other utility attributes.
We extensively evaluate our method on multiple datasets, demonstrating higher de-identification rate and superior consistency than prior art \wrt various downstream tasks.