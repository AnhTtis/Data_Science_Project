% \begin{figure*}[t]
%     \centering
%     \includegraphics[width=0.8\linewidth]{figures/teaser.png}
%     \caption{Visual effects of different face anonymization methods.}
%     \label{fig:teaser}
% \end{figure*}

% \twocolumn[{%
% \renewcommand\twocolumn[1][]{#1}%
% \maketitle
% \begin{center}
%     \centering
%     \captionsetup{type=figure}
%     \includegraphics[width=.8\textwidth]{figures/teaser.png}
%     \captionof{figure}{Visual effects of different face anonymization methods.}
% \end{center}%
% }]
\begin{figure}[t]
    \centering
    \includegraphics[width=0.45\textwidth]{figures/teaser_v3.pdf}
    \caption{\textit{Disguise} anonymizes face images while preserving their utility (i.e., attributes relevant to downstream tasks). 
    For instance, facial landmarks and gaze direction are better preserved compared to existing methods, as shown in the figure that the red dots for landmarks and red arrows for gazes in the new images are more aligned with the blue ones in the original images. We outperform prior art by a large margin along various axes, including privacy, utility, and image quality. For image quality, small radius indicates higher FIQ \cite{terhorst2020ser} score and better image quality.}
    \label{fig:teaser}\vspace{-10pt}
\end{figure}
