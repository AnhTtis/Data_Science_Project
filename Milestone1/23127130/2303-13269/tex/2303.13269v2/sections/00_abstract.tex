% With the increasing ubiquity of cameras and smart sensors, humanity is generating data at an exponential rate. 
% Access to this trove of information, often covering yet-underrepresented use-cases (\eg, AI in medical settings) could fuel a new generation of deep-learning tools. 
% However, eager data scientists should first provide satisfying guarantees \wrt the privacy of individuals present in these untapped datasets. 
% This is especially important for images or videos depicting faces, as their biometric information is the target of most identification methods.
% While a variety of solutions have been proposed to de-identify such images, they often corrupt other non-identifying facial attributes that would be relevant for downstream tasks.
% In this paper, we propose \textit{Disguise}, a novel algorithm to seamlessly de-identify facial images while ensuring the usability of the altered data. Unlike prior arts, we ground our solution in both differential privacy and ensemble-learning research domains. Our method extracts and swaps depicted identities with fake ones, synthesized via variational mechanisms to maximize obfuscation and non-invertibility; while leveraging the supervision from a mixture-of-experts to disentangle and preserve other utility attributes.
% We extensively evaluate our method on multiple datasets, demonstrating higher de-identification rate and superior consistency than prior art \wrt various downstream tasks.

% With the increasing prevalence of cameras and smart sensors, an exponential amount of data is being generated by humanity. Access to this valuable information, which often covers underrepresented use-cases (e.g., AI in medical settings), has the potential to drive the development of a new generation of deep-learning tools. However, it is crucial for data scientists to prioritize providing satisfactory guarantees regarding the privacy of individuals featured in these untapped datasets. This is particularly significant for images or videos containing faces, as their biometric data is the primary target of most identification methods. While numerous solutions have been proposed to de-identify such images, they often compromise other non-identifying facial attributes that are pertinent to downstream tasks.
With the rise of cameras and smart sensors, humanity generates an exponential amount of data. This valuable information, including underrepresented cases like AI in medical settings, can fuel new deep-learning tools. However, data scientists must prioritize ensuring privacy for individuals in these untapped datasets, especially for images or videos with faces, which are prime targets for identification methods. Proposed solutions to de-identify such images often compromise non-identifying facial attributes relevant to downstream tasks.
%
In this paper, we introduce \textit{Disguise}, a novel algorithm that seamlessly de-identifies facial images while ensuring the usability of the modified data. Unlike previous approaches, our solution is firmly grounded in the domains of differential privacy and ensemble-learning research. Our method involves extracting and substituting depicted identities with synthetic ones, generated using variational mechanisms to maximize obfuscation and non-invertibility. Additionally, we leverage supervision from a mixture-of-experts to disentangle and preserve other utility attributes. We extensively evaluate our method using multiple datasets, demonstrating a higher de-identification rate and superior consistency compared to prior approaches in various downstream tasks.