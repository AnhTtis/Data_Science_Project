\section{Conclusion and Discussion}
% We proposed \textit{Disguise}, a model for face de-identification that ensures both the privacy of depicted people and the usability of altered images. 
% Through experiments, we showed how it can pre-process sensitive data before its use in inference or training contexts.
% Grounded in privacy and mixture-of-experts theory, it excels in terms of re-identification robustness and utility preservation compared to prior work.

% \noindent\textbf{Limitations.} 
% One should note that our current model is tailored for face obfuscation and does not consider other visual attributes that may leak identity (\eg, conspicuous glasses or haircut, background, \etc). While this could be tackled by considering broader ID-extracting methods $H_\mathcal{Z}$ \cite{bhanu2017deep}.
% We also acknowledge that \textit{Disguise} could benefit from research in multi-objective learning \cite{desideri2012multiple,momma2022multi} to find better optima for cases when identity and utility features overlap.
We introduced \textit{Disguise}, a privacy-enhancing face de-identification model that ensures both depicted people's privacy and image usability. Our experiments demonstrate its effectiveness in pre-processing sensitive data for inference or training. Rooted in privacy and mixture-of-experts theory, it outperforms prior methods in re-identification robustness and utility preservation.

\noindent\textbf{Limitations.}
Note that our model is tailored for face obfuscation and does not address other identity-revealing visual attributes (\eg, distinctive glasses, haircuts, backgrounds). Broader ID-extracting methods like $H_\mathcal{Z}$ \cite{bhanu2017deep} could potentially handle this. Additionally, \textit{Disguise} might benefit from multi-objective learning research \cite{desideri2012multiple,momma2022multi} to optimize cases where identity and utility features overlap.

%\noindent\textbf{Societal Impact.} 
%The above limitations may impact the adoption of our work to specific use-cases. In some medical settings, other biometric attributes may need obfuscation (\eg, scars, outlying body types, \etc).
% Moreover, like all generative solutions, our method does not offer any safeguard against the (though-unlikely) possibility that a synthesized identity actually matches a real person. The concepts of originality and authorship \wrt deep-learning creations have yet to be clearly defined and regulated \cite{deltorn2017deep}, though we expect the community to catch up given the rising popularity of AI-based generative tools. 
% That being said, even though we rely on face-swapping technology, we do not have to deal with the same moral conundrum (\ie, \textit{deep fake} concerns), as our method does not allow user inputs \wrt replacing identity. 
% All in all, we believe that our work can positively impact the community and society as a whole, by enabling the safe usage of precious yet untapped data while providing privacy guarantees to people depicted. 