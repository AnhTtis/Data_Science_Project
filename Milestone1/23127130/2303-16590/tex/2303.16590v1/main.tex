\documentclass[twocolumn,showkeys,showpacs,preprintnumbers,prd,superscriptaddress,nofootinbib]{revtex4-1}
%\usepackage{newtxtext}
\usepackage{graphicx}	
\usepackage{amssymb}
\usepackage{dcolumn}
\usepackage{mathtools}
\usepackage{amsmath}
\usepackage{xcolor}
\usepackage{color}
\usepackage{subfigure, rotating, bm, array}
\usepackage[pagebackref=false, colorlinks=true]{hyperref}
\hypersetup{linkcolor=blue, citecolor=blue,urlcolor=blue} 
\renewcommand{\baselinestretch}{1.25}

\begin{document}

\title{Naked Singularity as a Possible Source of Ultra-High Energy Cosmic Rays}


\author{Kauntey Acharya}
\email{kaunteyacharya2000@gmail.com}
\affiliation{International Centre for Space and Cosmology, School of Arts and Sciences, Ahmedabad University, Ahmedabad-380009 (Gujarat), India}

\author{Kshitij Pandey}
\email{pkshitij45@gmail.com}
\affiliation{PDPIAS, Charusat University, Anand-388421 (Gujarat), India}
\affiliation{International Center for Cosmology, Charusat University, Anand-388421 (Gujarat), India}

\author{Parth Bambhaniya}
\email{grcollapse@gmail.com}
\affiliation{PDPIAS, Charusat University, Anand-388421 (Gujarat), India}
\affiliation{International Center for Cosmology, Charusat University, Anand-388421 (Gujarat), India}


\author{Pankaj S. Joshi}
\email{psjcosmos@gmail.com}
\affiliation{International Centre for Space and Cosmology, School of Arts and Sciences, Ahmedabad University, Ahmedabad-380009 (Gujarat), India}
\affiliation{International Center for Cosmology, Charusat University, Anand-388421 (Gujarat), India}

\author{Vishva Patel}
\email{vishwapatel2550@gmail.com}
\affiliation{International Center for Cosmology, Charusat University, Anand-388421 (Gujarat), India}

\date{\today}

\begin{abstract}
The source of Ultra-High Energy Cosmic Rays (UHECRs) remains one of the greatest mysteries in astrophysics. Their possible source can be the galactic nuclei, where the ultra-high gravity region plays a crucial role. Cosmic rays are extremely energetic particles that travel through space with energies exceeding $10^{20}eV$, but their origin is still a mystery despite years of studies and observations. In view of this, in this work, we studied the Joshi-Malafarina-Narayan (JMN-1) naked singularity as a natural particle accelerator. We derived the necessary expressions to find center of mass energy when two particles collide. We have obtained results showing that center of mass energy of the two particles will reach to Planck energy scale. This will form a microscopic black hole which will decay in Hawking radiation, having energy on the order of $10^{26} eV$ from the ultra-high gravity region of Sgr A*. These outgoing highly energetic particles from the naked singularity could be the possible sources of UHECRs.

\bigskip
Key words: Ultra-compact objects, Naked singularity, Black holes, Particle acceleration, Cosmic rays, Hawking radiation.
\end{abstract}
\maketitle


\section{Introduction}
\label{sec_intro}

Recent investigations have shown that a significant number of events have been observed so far that exceed $10^{19} eV$ energy scale \cite{CTAConsortium:2013tmf}. However, there are no evidences of the predicted Greisen-Zatsepin-Kuzmin (GZK) cutoff \cite{Aloisio:2009sj}. The GZK cutoff is a feature of high-energy cosmic rays, which are charged particles that originate from far-off astrophysical sources and are propelled to exceptionally high energies. As these cosmic rays travel through the universe, they interact with photons of the cosmic microwave background radiation, which is a relic radiation left over from the Big Bang \cite{Bergman:2006vt}. These interactions can cause the cosmic rays to lose energy, particularly at very high energies. The GZK cutoff is the energy threshold above which cosmic rays are expected to be strongly attenuated due to these interactions \cite{Haungs:2003jv}.\\

The absence of the GZK cutoff is surprising as it should be present if the ultra-high energy particles are from extra-galactic sources. Cosmic ray protons with energies above a few $10^{19} eV$ lose energy through photopion production off the cosmic microwave background (CMB) and cannot originate beyond 50 $Mpc$ from the Earth \cite{Aloisio:2009sj}. This motivates us to study the UHECRs production phenomena within 50 $Mpc$. \\
The first feature, known as the pair-production dip, is a relatively weak signal from tesulting from the pair production process and can be expressed as 
\begin{eqnarray}
    p + \gamma_{CMB} \longrightarrow e^{+} + e^{-} + p.
\end{eqnarray}
The second feature is a pronounced steepening of the spectrum, known as the GZK cutoff, which is caused by pion photo-production and is expressed as:
\begin{eqnarray}
    p + \gamma_{CMB} \longrightarrow \pi + p.
\end{eqnarray}

The existence of ultra-compact objects such as black holes, naked singularities, gravastars and many others are yet to be proven observationally. On the other hand, it is known about the visible and hidden singularities that might form from the continuous gravitational collapse \cite{joshi,goswami,mosani1,mosani2,mosani3,mosani4,Deshingkar:1998ge,Jhingan:2014gpa,Joshi:2011zm}. Regarding the observational aspects, the recent findings of the Event Horizon Telescope (EHT) suggest that JMN-1 naked singularity is the best Schwarzschild black hole mimicker \cite{EventHorizonTelescope:2022xqj}. The study of accretion discs, shadow properties, tidal forces, gravitational lensing, energy extractions, timelike and null geodesics around such compact objects are eisxplored in various literature \cite{Page:1974he,Liu:2020vkh,Liu:2021yev,Joshi:2013dva,Bambhaniya:2021ugr,Tahelyani:2022uxw,Rahaman:2021kge,Harko:2008vy,Harko:2009xf,Kovacs:2010xm,Guo:2020tgv,Chowdhury:2011aa,Lattimer:1976kbf,Chicone:2004pv,Madan:2022spd,Shaikh:2019jfr,Shaikh:2018oul,Paul:2020ufc,Virbhadra:2007kw,Gyulchev:2008ff,Kala:2020prt,Sahu:2012er,Martinez,Madan:2022spd,Eva1,Eva2,tsirulev,Joshi:2019rdo,Bambhaniya:2019pbr,Dey:2019fpv,Bam2020,Bambhaniya:2022xbz,Gralla:2019xty,Vagnozzi:2019apd,Chen:2022nbb,Dey:2020haf,atamurotov_2015,abdujabbarov_2015b,Li:2021,Hu:2020usx,Patel:2022vlu,Kaur:2021wgy,Saurabh:2020zqg,Pugliese:2022oes,Vagnozzi:2022moj,Saurabh:2022jjv,Patel:2022jbk,Patel:2023efv}.\\

The collision of Ultra-high energy particles near the ultra-compact object through particle acceleration can be one of the most prominent sources of these cosmic rays. Therefore, in this paper, we aim to explore the mechanisms behind particle acceleration in the vicinity of ultra-compact objects. In particular, we focus on timelike geodesics to understand the acceleration process. Our approach does not consider the role of radiation or fields in the acceleration process, allowing us to isolate and analyze the effect of geodesic motions on particle acceleration.\\

Several papers have shown the particle acceleration phenomena around ultra-compact objects. In \cite{Patil:2011yb}, authors conclude that for Sgr A*, the collisional energy of  two particles can reach up to $10^{3}\,TeV$. Furthermore, particle acceleration in Schwarzschild and Kerr black holes has been shown \cite{Banados:2009pr}. The study of Janis-Newman-Winicour, Resinner-Nordstrom and Kerr naked singularities as particle accelerators are discussed in \cite{Patil:2011aa,Patil:2011uf,Patil:2011ya}. Due to the high gravity region, ultra-compact objects can accelerate the particles up to the Planck scale physics as shown in \cite{Patil:2011aw}.\\

Initially, the phenomenon of particle collisions around black holes was known as the Banados-Silk-West effect. However, this effect required many finely-tuned scenarios for the collision of particles to release energy, which makes it less appealing as a mechanism for explaining high-energy astrophysical phenomena.
Following this, authors in \cite{Patil:2011aa,Patil:2011uf,Patil:2011ya} studied the particle acceleration process in the vicinity of the naked singularity which requires less fine tuning for the possibility of high energy collisions of particles.\\

Thus, in this work, we consider particle acceleration as a potential mechanism to explain the high energy phenomena by considering the collision of particles in the vicinity of JMN-1 naked singularity. We investigate the behaviour of energy of colliding particles in the centre of mass frame $(E_{CM})$ with respect to radial distance as well as for the angular momenta of two particles. Further, we also derive the numerical value of the centre of mass energy $(E_{CM})$ by considering JMN-1 naked singularity at the milky-way galactic center.\\

This paper is organised as follows: The JMN-1 geometry is discussed in section (\ref{sec_jmn1}). The geodesic motion of a particle and the center of mass energy around the JMN-1 naked singularity are derived in section (\ref{sec_particle}). In the section (\ref{sec_microbh}), we try to explain the possibility of formation of microscopic black holes. This possibility is described with a detailed discussion of Planck scale energy and Hawking radiation for the high center of mass energy coming from the particle collision. After that, the energy of outgoing particles is discussed in section (\ref{sec_ecmgc}). Finally, we conclude and discuss the results of this work in section (\ref{sec_conclusion}). Only when necessary, we defined the gravitational constant (G) and speed of light (c) values, otherwise defined as the geometrized units. The signature of the metric is considered as (-,+,+,+).
%In the section (\ref{sec_microbh}), we explain the formation of microscopic black holes with a detailed discussion of Plank scale energy and Hawking radiation. 
\section{Joshi–Malafarina–Narayan (JMN-1) spacetime}
\label{sec_jmn1}

The very first model of gravitational collapse of a spherically symmetric and homogeneous dust cloud was proposed by Oppenheimer, Snyder, and Dutt, (OSD) which indeed indicates the formation of a black hole as an end state \cite{Oppenheimer:1939ue}. In this OSD model of gravitational collapse, trapped surfaces form around the centre before the formation of the central space-like singularity. As a result, the central singularity is causally disconnected from other points of spacetime, indicating the presence of a horizon in the spacetime structure. However, the OSD model assumes homogeneous density and zero pressure within the massive collapsing star. Thus it is considered to be an idealistic scenario of gravitational collapse.
Despite the fact that of Cosmic Censorship Conjecture (CCC) by Roger Penrose, several investigations have been carried out to understand the process of gravitational collapse in more realistic physical scenarios, including the inhomogeneous distribution of matter in the collapsing cloud and non-zero pressure profiles \cite{Joshi:2011zm}.\\

When an inhomogeneous matter cloud was considered i.e. higher density at the centre which represents a more realistic physical condition, studies suggested that an end state can be locally or globally visible strong curvature singularity following certain equilibrium conditions \cite{Mosani:2022nsp,mosani3}. In \cite{Joshi:2011zm}, authors have extensively studied equilibrium configurations of the collapsing cloud under the influence of gravity with zero radial pressure and non-zero tangential pressure i.e. presence of an-isotropic fluid. Non-zero tangential pressure can prevent the formation of trapped surfaces around the central region. An end state of a massive star under such conditions can lead to the formation of JMN-1 naked singularity. The line element of the JMN-1 naked singularity spacetime is,
\begin{equation}
     ds^2 = - (1- M_{0}) \left( \frac{r}{Rb}\right)^{\frac{M_{0}}{1 - M_{0}}} dt^2 + \frac{1}{(1- M_{0})}  dr^2 +\, r^2  d\Omega^2,
     \label{JNWst}
\end{equation}
where, $d\Omega^2= d\theta^2+\,sin^2\,\theta\,d\phi^2$. $M_{0}$ and $R_{b}$ are positive constants. $M_{0}$ can have any value within the range $0<M_{0}\,<\,1$. $R_{b}$ is the radius of matter distributed around the central singularity.\\

The spacetime metric is modelled by considering a high-density compact region in a vacuum, which means that the spacetime configuration should be asymptotically flat. For this purpose, the JMN-1 spacetime is matched with an exterior Schwarzschild spacetime
at $r=R_{b}$ radius. The line element of this exterior spacetime can be written as,
\begin{equation}
    ds^2=-\left(1-\frac{M_{0}R_{b}}{r}\right)dt^2+\frac{dr^2}{\left(1-\frac{M_{0}R_{b}}{r}\right)}+r^2d\Omega^2,
    \label{extsch}
\end{equation}
where, Schwarzschild radius is $R_{s}=M_{0}R_{b}$ and the Schwarzschild mass is given by, $M=\frac{M_{0}R_{b}}{2}.$
Now for matching of two spacetimes, there are two junction conditions which suggest that (i) extrinsic curvatures of both spacetimes should smoothly match at a null hyper-surface and (ii) induced metrics of exterior and interior geometries be equivalent on the hyper-surface where matching is considered. Since the radial pressure is zero in JMN-1 naked singularity spacetime, the extrinsic curvatures of the interior JMN-1 spacetime and the exterior Schwarzschild spacetime are smoothly match at $r=R_{b}$. It is also important to note that all the energy conditions are satisfied in the above spacetime structure. In the next section (\ref{sec_particle}), we consider collisions of particles in the vicinity of the above discussed compact object with a central high-density region of JMN-1 spacetime and outer Schwarzschild metric.

\section{Particle acceleration near a JMN-1 naked singularity}
\label{sec_particle}

In this section, we describe the mechanism of particle acceleration by investigating collisions of particles near the central singularity of JMN-1 spacetime. We first consider the geodesic motion of colliding particles with turning points and derive the expression of center of mass energy. For this, we also study the allowed values of metric parameter $M_{0}$ for which  energy collisions are possible because of the existence of turning points. 

\subsection{Geodesic motions}
\label{sec_geodesiceqn}

We first describe the motion of the massive particles of mass $m$ moving with the four velocities $U^{\mu}$ in time-like geodesics in the JMN-1 spacetime. For simplicity, we consider that particles move in an equatorial plane ($\theta\,=\,\pi/2$), which suggests that $U^{\theta}=0$. The equations of motion of particles can be written by using constants of motion, i.e. conserved energy and conserved angular momentum per unit mass of the particle. From these constants of motion, the expressions of four velocity are written as,

\begin{eqnarray}
    && u^{t}=\frac{e}{(1- M_{0}) \left( \frac{r}{Rb}\right)^{\frac{M_{0}}{1 - M_{0}}}},\\
    && u^{\phi}=\frac{L}{r^{2}}.
\end{eqnarray}
The radial component of the four velocity can be written as follows using the normalization condition $u^{\mu}u_{\mu}=-1$,

\begin{equation}
    u^{r}=\pm \bigg[e^{2}-\frac{1}{2}\bigg((1\;-\;M_{o}) \bigg(\frac{r}{R_{b}}\bigg)^{\frac{\;M_{o}}{1\;-\;M_{o}}}\bigg(1 + \frac{L^2}{r^2}\bigg) - 1\bigg)\bigg]^\frac{1}{2},
\end{equation}
Here $\pm$ represents the motion of particle in radial outward and inward directions, respectively. from which one can write the effective potential and total energy as,
\begin{equation}
    (u^{r})^2+V_{eff}(r)=E,
\end{equation}
where $E=\frac{e^2-1}{2}$ is the total energy and $V_{eff}(r)$ is the effective potential as a function of the radial coordinate for a massive particle which can be expressed as,
 \begin{equation}
 V_{eff} (r) =  \frac{1}{2}\bigg[(1\;-\;M_{o}) \bigg(\frac{r}{R_{b}}\bigg)^{\frac{\;M_{o}}{1\;-\;M_{o}}}\bigg(1 + \frac{L^2}{r^2}\bigg) - 1\bigg]. 
 \end{equation} 
Now we consider a scenario where non-relativistic particles at infinity fall inward under the influence of gravity near the JMN-1 naked singularity. However, the exterior region of the JMN-1 naked singularity is represented by the Schwarzschild metric. 
Thus it is important to consider equations of motion for the exterior Schwarzschild metric as it is considered that colliding particles are coming from infinity. The expressions of four velocity of a test particle using constants of motion can be written as,
 \begin{eqnarray}
     &&  u^{t}=\frac{e}{\left(1-\frac{M_{0}R_{b}}{r}\right)},\\
     &&  u^{\phi}=\frac{L}{r^{2}}.
 \end{eqnarray}
 Using the normalization condition $u^{\mu}u_{\mu}=-1$ for timelike geodesics, the radial component of the four velocity in exterior Schwarzschild geometry can be written as,
\begin{equation}
    (u^{r})^2=\left[e^{2}-\frac{1}{2}\bigg[\left(1-\frac{M_{0}R_{b}}{r}\right)\bigg(1 + \frac{L^2}{r^2}\bigg) - 1\bigg]\right],
\end{equation}
and the expression of an effective potential for Schwarzschild spacetime is,
\begin{equation}
     V_{eff} (r) =  \frac{1}{2}\bigg[\left(1-\frac{M_{0}R_{b}}{r}\right)\bigg(1 + \frac{L^2}{r^2}\bigg) - 1\bigg].
\end{equation}
Now for particles at infinity, as $r \rightarrow \infty$, $ U^{r} \rightarrow 0$, the conserved energy per unit rest mass of the particles should be $e=1$. Since total energy of the particle is $E=\frac{e^{2}-1}{2}$, $E=0$ and hence the radial component of the four velocity can be written as,

\begin{equation}
    (U^{r})^2=\left[1-\left(1-\frac{M_{0}R_{b}}{r}\right)\left(1+\frac{L^{2}}{r^{2}}\right)\right],
\end{equation}

In general for high energy collisions, the process must happen near the central singularity so high gravitational effects can be considered. For this, one of the particles must turn back before it reaches near the singularity. For a particle to turn back at any given radial distance $r$, $U^{r}=0$ and $E=1$. From which one can write the expression of required angular momentum $L$ for the particle to turn back at $r$ in the exterior Schwarzschild geometry,

\begin{equation}
    L=\pm \left(\frac{r M_{0}R_{b}}{r-M_{0}R_{b}}\right)^\frac{1}{2}.
\end{equation}
However to consider the maximum energy extraction from the collision of particles, one needs to consider this phenomenon of collisions near singularity. Since the central high-density region is represented by JMN-1 spacetime metric, we consider the collision of particles within this spacetime. Here again, it is considered that particles must turn back before it reaches the singularity as shown in Fig. (\ref{fig:my_label}). In this geometry, for a particle to turn back at any given radial distance, $U^{r}=0$ and $E=1$. The expression of required angular momentum $L$ for the particle to turn back at any radial distance $r$ in the JMN-1 spacetime can be written as,


%\begin{figure*}[]
%\includegraphics[width=7cm]{f00.pdf}
%\hspace{0.1cm}
%\caption{The schematic diagram of particle collision around naked singularity.}
%\label{fig:schematic}
%\end{figure*}


\begin{figure*}[ht!]
\centering
\subfigure[$M_{0}$ = 0.55, $R_{b}$ = 3.64, $L_{1} $= 4.05, $L_{2} $= -4.05.]
{\includegraphics[width=8cm]{f1.pdf}\label{fig:radiusandecmjmn1}}
\hspace{0.8cm}
\subfigure[ The bar on right side of the figure represents the different values of parameter $M_{0}$.]
{\includegraphics[width=8cm]{f2.pdf}\label{fig:jmn1angular}}
\hspace{0.8cm}
 \caption{The change in $E_{cm}$ for the radial distance is shown in Fig.\,\ref{fig:radiusandecmjmn1}. Fig.\,\ref{fig:jmn1angular} shows the change in angular momentum ($L(r)$) of a particle with respect to the radial distance ($r$).}
 \label{fig:1}
\end{figure*}



\begin{equation}
    L=\pm \left[\frac{R_{0}^\frac{M_{0}}{1-M_{0}}}{\left(1-M_{0}\right)r^\frac{M_{0}}{1-M_{0}}}-1\right]^{-\frac{1}{2}}.\label{angmomjmn}
\end{equation}
%\begin{figure*}[]
%\includegraphics[width=10cm]{f2.pdf}
%\hspace{0.1cm}
%\caption{
%The change in angular momentum ($L(r)$) of a particle with respect to the radial distance ($r$) is shown in above figure. The bar on right side of the figure represents the different values of parameter $M_{0}$.}
%\label{fig:jmn1angular}
%\end{figure*}
%\begin{figure*}[]
%\includegraphics[width=10cm]{f1.pdf}
%\hspace{0.1cm}
%\caption{$M_{0}$ = 0.55, $R_{b}$ = 3.64, $L_{1} $= 4.05, $L_{2} $= -4.05.}
%\label{fig:radiusandecmjmn1}
%\end{figure*}

The plot of the above expression is shown in Fig.\,\ref{fig:jmn1angular}. It represents the variation in angular momentum for particle to turn back at any radial distance with respect to radius in JMN-1 geometry for different values of $M_{0}$ parameter between $1/2\,<\,M_{0}\,<\,3/4$. Other values of $M_{0}$ are not considered because there does not exist any turning points for the corresponding values. It should be noted that for $M_{0}\,<\,1/2$, the radius at which the collision would take place is greater than the boundary radius $R_{b}$. Since the minima in the graph of $L\rightarrow r$ is outside the boundary radius for JMN-1 metric which is not possible. While for $M_{0}\,>\,2/3$, the radius at which we get the minimum angular momentum  which is imaginary. This can also be seen from the corresponding lines representing the angular momentum of a particle with respect to distance $r$ that there are no minima in the angular momentum plot for $M_{0}=0.7, 0.75$, which suggest that the particle does not turn back at any value of radial distance and hence collision is not possible. Thus the particle collisions need to be considered only between the range $1/2\,\leq\,M_{0}\,\leq\,2/3$. Corresponding to these values of $M_{0}$, the radial distance at which the collisions would occur can be found using eq.(\ref{angmomjmn}), which is $r=4.00M$ for $M_{0}=0.50$, $r=3.23M$ for $M_{0}=0.55$, $r=2.43M$ for $M_{0}=0.60$, and $r= 1.31M$ for $M_{0}=0.65$. Here, $M$ is the mass of the central compact object. For the case of $M_{0}<\frac{1}{2}$, the radius at which turning occurs outside the boundary radius $R_{b}$ and for $M_{0}>\frac{2}{3}$ the particle will plunge inside the singularity.

\begin{figure}[ht]
    
    \includegraphics[width=\columnwidth]{f05.png}
    \caption {The figure shows that particle 1 is turned at a turning point and collides head-on with particle 2 at a radial distance of $r=3.2M$ from the singularity, corresponding to $M_{0}=0.55$ case. Here $r_{max}$ and $r_{min}$ are the maximum and minimum possible radius for the turning point.}
    \label{fig:my_label}
\end{figure}
%In the Fig.\,(\ref{fig:schematic}) one can see the schematic diagram that particle\,(1) is turned at a turning point and collide with the particle\,(2). This collision is also known as head-on collision. 

Now the range of possible values of angular momentum for the collision to occur is known. Using this information, we can find out  energy particles after the collision. However, since the collision is taking place in curved spacetime, there is no fixed coordinate system or reference frame for which the energy of the particles can be defined. Thus we consider a centre of mass frame of two particles as a reference frame for this purpose.


\subsection{Center of mass energy in JMN-1 naked singularity}
\label{sec_ecm}

To define the center of mass energy of colliding particles, we first consider a general curved spacetime defined by $g_{\mu \nu}$. In this spacetime, we can consider any arbitrary point as a lab frame where the spacetime can be considered to be locally flat and has a form of Minkowski metric. The local lab frame can be described by basis vectors $\hat{e_{a}}$ for which $\hat{e_{a}} \hat{e_{b}}=\eta_{ab}$. Here for any given basis, the coordinate basis can be written as $\Vec{e_{\mu}}=e^{a}_{\mu} \hat{e_{a}}$. Now if the curved spacetime metric $g_{\mu \nu}$ is given, then the matrix $e^{a}_{\mu}$ can be uniquely defined and thus, from the information about the world-line history $x^{\mu}(\tau)$ and four velocity $u^{\mu}$, three velocity of any particle in that local lab frame can be written as,

\begin{figure*}[ht!]
\centering
\subfigure[$M_{0}$ = 0.50, $R_{b}$ = 4.00.]
{\includegraphics[width=7cm]{f01.pdf}\label{fig:31}}
\hspace{1cm}
\subfigure[$M_{0}$ = 0.55, $R_{b}$ = 3.64.]
{\includegraphics[width=7cm]{f02.pdf}\label{fig:32}}
\subfigure[$M_{0}$ = 0.60, $R_{b}$ = 3.33. ]
{\includegraphics[width=7cm]{f03.pdf}\label{fig:33}}
\hspace{1cm}
\subfigure[$M_{0}$ = 0.65, $R_{b}$ = 3.08. ]
{\includegraphics[width=7cm]{f04.pdf}\label{fig:34}}
\hspace{1cm}
 \caption{The above figures shows the change in center of mass energy ($E_{cm}$) per unit rest mass with respect to the different angular momentum of two colliding particle. The bar on right side of the figures shows the numerical value of $E_{cm}$. }\label{fig:3}
\end{figure*}
\begin{equation}
    v^{(i)}=\frac{e^{(i)}_{\mu} u^{\mu}}{e^{(0)}_{\mu} u^{\mu}},
\end{equation}
The energy of center of mass in this frame can be expressed as,

\begin{equation}
    E_{cm}^{2}=2m^{2}\left(1-\eta_{\mu \nu}u^{\mu}_{(1)}u^{\nu}_{(2)}\right)^{\frac{1}{2}}.
\end{equation}
The above equation can be written for a generally curved spacetime metric from $g_{\mu \nu}=e^{a}_{\mu} e^{b}_{\nu} \eta_{ab}$ as,

\begin{equation}
     E_{cm}^{2}=2m^{2}\left(1-g_{\mu \nu}u^{\mu}_{(1)}u^{\nu}_{(2)}\right)^{\frac{1}{2}}.
\end{equation}

Now we can use the above equation for exterior Schwarzschild metric and get the following expression of center of mass energy,
\begin{widetext}
\begin{align}
 \left(\frac{ E_{cm}^2}{2m^{2}}\right)_{SCH} &=  1+ \frac{1}{\left(1- \frac{M_{0}R_{b}}{r}\right)}- 
   \frac{1}{\left(1- \frac{M_{0}R_{b}}{r}\right)} \times \nonumber \\ &\left(1-\left(1-\frac{M_{0}R_{b}}{r}\right) \left(1+\frac{L_{1}^{2}}{r^{2}}\right)\right)^{\frac{1}{2}} \left(1-\left(1-\frac{M_{0}R_{b}}{r}\right)\left(1+\frac{L_{2}^{2}}{r^{2}}\right)\right)^{\frac{1}{2}}-\frac{L_{1}L_{2}}{r^{2}}.
\end{align}
\end{widetext}
However, as we consider that the collision could also occur at $r < R_{b}$, we need to find the expression of $E_{cm}$ for JMN-1 naked singularity spacetime which can be written as,

\begin{widetext}
\begin{align}
     \left(\frac{ E_{cm}^2}{2m^{2}}\right)_{JMN-1} &= 1+\frac{1}{(1- M_{0}) \left( \frac{r}{Rb}\right)^{\frac{M_{0}}{1 - M_{0}}}}-\frac{1}{1-M_{0}} \times \nonumber \\ &\left(1-(1- M_{0}) \left( \frac{r}{Rb}\right)^{\frac{M_{0}}{1 - M_{0}}}\left(1+\frac{L_{1}^{2}}{r^{2}}\right)\right)^{\frac{1}{2}}\left(1-(1- M_{0}) \left( \frac{r}{Rb}\right)^{\frac{M_{0}}{1 - M_{0}}}\left(1+\frac{L_{2}^{2}}{r^{2}}\right)\right)^{\frac{1}{2}} -\frac{L_{1}L_{2}}{r^{2}}
     \label{jmn1ecm}
\end{align}
\end{widetext}
From this expression, the energy of both particles after collision in the JMN-1 geometry from the center of mass frame can be evaluated.

At first, to understand the behavior of energy of particle per unit rest mass from center of mass frame of reference $E_{cm}$ with radial distance coordinate, we consider particle collision for $M_{0}=0.55$ (which is in the range of values of $M_{0}$ for which particle collision can be considered as the turning point exists at $r<R_{b}$) and get the Fig.\,\ref{fig:radiusandecmjmn1}. The graph shows that as we move away from the singularity, the energy of particle per unit rest mass from the center of mass frame of reference decreases.

However, as we have considered earlier, collision does just take place where the angular momentum of the particles possess values in a certain range, thus it is important to see the behavior of $E_{cm}$ with angular momenta $L_{1}$ and $L_{2}$ of two particles. As stated previously, we can consider the collision of particles only when $1/2\,\leq\,M_{0}\,\leq\,2/3$ for the corresponding limiting values of angular momentum. Considering those values, we get plots of energy of two particles per unit rest mass from the center of mass frame with angular momenta of two particles  $L_{1}$ and $L_{2}$ which can be seen in Fig.\,(\ref{fig:3}). From this Fig.\,(\ref{fig:3}), one can see that the energy from center of mass frame is maximum when both particles possess opposite values of angular momenta. This is because the head-on collision is taking place between the particles as they are approaching each other. The scenario refers to a situation where a particle is moving towards a singularity and another particle is moving away from the same singularity. The two particles collide at the turning point.

Also, for the similar values of angular momenta, which suggest the collision of two particles moving in the same direction, $E_{cm}$ is very small. Furthermore, $E_{cm}$ is zero when the magnitude of $L_{1}$ and $L_{2}$ is maximum from the theoretically obtained possible range of angular momenta. Note that as the value of $M_{0}$ parameter increases, the maximal $E_{cm}$ also increases between $\frac{1}{2} \leq M_{0} \leq \frac{2}{3}$. We evaluate the $E_{cm}$ using Eq.\ref{jmn1ecm} and Eq.\ref{angmomjmn} by considering mass of a neutron as mass $m = 1.67 \times 10^{-27}$ kg of colliding particles and the mass of Sgr A $M=4.1 \times 10^{6} M_{\odot}$ as mass of JMN-1 naked singularity. For the value of parameter $M_{0}=0.5$, the corresponding  value of centre of mass energy $E_{CM}$ of colliding particles is in the order of $10^{28} eV$, which is Planck energy scale.


\section{Formation of Microscopic Black Holes}
\label{sec_microbh}

According to the theory of general relativity, all forms of energy, including momentum, generate gravity. As a result, it's believed that gravity will become a significant factor at very high energies, specifically when the center of mass energies approach the Planck scale. At Planck energy, the collision of two particles can even create a microscopic black hole \cite{Choptuik:2009ww}.\\

The four dimensional Planck scale energy is $10^{19} GeV$ or $10^{28} eV$. According to hoop conjecture given by Kip Thorne in 1972, if a large amount of matter and energy, represented by $E$, are compressed into a region, and if a hoop with proper circumference $2\pi R$ can fully encircle this matter in all directions. Then a black hole will form if the resulting Schwarzschild radius (calculated as $R_{s} = 2GE/c^{4}$) is larger than the value of $R$. The parameters used in this calculation include Newton's constant ($G$) and the speed of light ($c$) \cite{Choptuik:2009ww}. Since the energy of the collision reaches a Planck scale, it can form a microscopic black hole. However, these microscopic black holes are not stable as they decay very rapidly through Hawking radiation \cite{CMS:2010oej}, \cite{Alok:2022xiy}.

The particles that makeup Hawking radiation are generated by the intense gravitational forces that exist around a black hole. These forces are strong enough to separate virtual particles, which are pairs of particles constantly popping in and out of existence in empty space. When one of these virtual particles is separated from its partner and falls into the black hole, the other particle is free to escape and create Hawking radiation. Thus through Hawking Radiation, these micro black holes will decay instantaneously \cite{Kovacik:2021qms}.\\

It is generally believed that miniature black holes decay by releasing elementary particles in the form of a spectrum of energy consistent with that of a black body \cite{Anchordoqui:2002cp}. The energy of these particles can be calculated through Hawking temperature. 
The energy $E$ of Hawking radiation is given by
\begin{eqnarray}
    E = \frac{\hbar c^{3}}{16 \pi GM},
\end{eqnarray}
where $\hbar$ is the reduced Planck constant, $c$ is the speed of light, and $M$ is the mass of the black hole.

\subsection{Energy of the outgoing particles}
\label{sec_ecmgc}

The particles that will be coming from the decay of microscopic black holes are of great interest. The energy of primary particles coming out of this microscopic black hole can be obtained using Hawking Temperature, as mentioned above. Here the mass of the black hole can be obtained using $M = \frac{E_{CM}}{c^{2}}$.
Since the $E_{CM}$ is equal to the Planck energy, the mass of the microscopic black hole that we obtain will be Planck mass which is in the order of $10^{19} GeV$.
Considering all the above-given parameters, the energy of these primary particles in the local frame will be in the order of $10^{26} eV$. However, the end results of Hawking emission are not the elementary particles only released by BHs. Others only exist in composite states, while some of them are unstable and decay (hadrons) \cite{Arbey:2019mbc}. After all the decay, only the stable particle with a very large half-life will be visible to an asymptotic observer.
The energy that we are getting near JMN-1 is significantly higher than the UHECRs that we detect on Earth. However, here we have not considered any other interaction such as coulomb interaction, Brehmstrahlung radiation (a type of electromagnetic radiation that is emitted when charged particles, such as electrons, are accelerated or decelerated by a strong electric field), and many other types of interactions. Furthermore, for an asymptotic observer like on Earth, the gravitational redshift will also play a significant role in downscaling the energy. In a physical scenario, these particles' interactions and redshift combine will down-scale the energy scale of particles, which leads us to the possibility of forming UHECRs. The detailed investigation of the down-scaling of the energy due to various interactions is left for future work, which will consider the hadronization of Hawking radiation.

\section{Discussion and Conclusions}
\label{sec_conclusion}

 In this paper, we have considered the phenomena of collision of particles in the vicinity of the JMN-1 naked singularity. We have also derived a numerical value of energy of particles after collision  corresponding to  $M_{0} = 0.5$. Our investigation leads to the following conclusion:
\begin{itemize}
   
    \item We considered collision of particles with turning points in this study. For this, we define a range of metric parameter $M_{0}$ for which particles turn back in the vicinity of the singularity. We find turning of particle is possible within the range  $\frac{1}{2} \leq M_{0} \leq \frac{2}{3}$. For the case of  $ M_{0} >\frac{2}{3}$, the radius at which particle turns back is imaginary, which suggests that particle plunges inside the singularity. For $M_{0}\,<\frac{1}{2}$, the radius of turning point is outside the boundary radius $R_{b}$.

    \item We find values of radial distance coordinate at which particle turns for different values of $M_{0}$. This radial distance is $r=4.00M$ for $M_{0}=0.50$, $r=3.23M$ for $M_{0}=0.55$, $r=2.43M$ for $M_{0}=0.60$, and $r= 1.31M$ for $M_{0}=0.65$. Here, $r_{min}$ would be the smallest value of radial distance coordinate at which particle can turn back. For $r<r_{min}$, there does not exist turning points as particles plunge inside the singularity. We considered collision of particles for these values of radial distance. It is important to note here that, in case of Schwarzschild black hole, the collisions can be defined for values of radial distance $r>2M$ as there is event horizon at $r_{s}=2M$. However in JMN-1 naked singularity case, the collisions can be defined at distances closer to singularity compared to black hole case.

\item From the range of JMN-1 metric parameter $\frac{1}{2} \leq M_{0} \leq \frac{2}{3}$ and values of radial distance coordinates, we define corresponding values of angular momentum $L$ of two particles approaching the central high density region moving along the geodesic.
Finally we studied collision of two particles between the range $\frac{1}{2} \leq M_{0} \leq \frac{2}{3}$ for corresponding values of angular momentum $L$ and radial distance coordinate $r$. In Fig.\,\ref{fig:radiusandecmjmn1} the behavior of centre of mass energy $E_{CM}$ of particles after collision with respect to the radial distance is shown. It can be seen that centre of mass energy decreases as the radial distance increases which is because the gravitational influence of singularity is weak at large distances. 

\item Another point which should be noted is, in case of black holes, the presence of event horizon leads to requirement of comparatively more fine tuning conditions for the phenomena of particle collision in the vicinity of the high curvature region. However in this case, it is easier to theoretically show the possibility of collision of particles in the vicinity of the singularity because of absence of horizon.

\item We derive the numerical value of centre of mass energy by considering the mass of colliding particles as a neutron. The mass of JMN-1 naked singularity is considered to equivalent to Sgr A. Here, for $M_{0}=0.5$, the boundary radius is $R_{b}=2.419 \times \, 10^{10} m$. For this, we get the centre of mass energy $E_{CM}$ of particles after collision in the order of $10^{28} eV$, which is in Planck scale. According to Hoop conjecture, collision at these energy scales will form a microscopic black hole, whose mass is given by $m = E_{CM}/c^{2}$. The microscopic black hole will decay instantaneously through Hawking radiation.

\item The energy of the particles coming out of the Hawking radiation near JMN-1 naked singularity will be given by $E = \frac{\hbar c^{3}}{16 \pi GM}$. For the given parameters, the energy of the initial particles will be $10^{26} eV$. However, most of the particles coming out of the Hawking radiation will decay either due to a lesser half-life or interaction with other particles. Moreover, for an asymptotic observer like on earth, the effect of gravitational redshift will also play a role. The combination of interactions among particles and their redshift can cause a reduction in the energy scale of particles, which in turn increases the likelihood of producing UHECRs.\\
 
\end{itemize}
Massive compact objects can act as natural particle accelerators; not only can they accelerate the particles at an energy that is next to impossible on earth, but also these accelerators can answer some of the most underlying questions about the ultra-high energy phenomena. For future work, we will deal with the phenomenology of these particles and how their energy will change with various interactions and processes.
\section{ACKNOWLEDGMENTS}
The authors would like to express their gratitude towards Prof. P C Vinodkumar, Dr. Tapobroto Bhanja and Mr. Saurabh for their valuable suggestions
and comments. V. P. would like to acknowledge the support of the SHODH fellowship (ScHeme Of Developing High quality research - MYSY). 

\begin{thebibliography}{999}

%\cite{CTAConsortium:2013tmf}
\bibitem{CTAConsortium:2013tmf}
H.~Sol \textit{et al.} [CTA Consortium],
%``Active Galactic Nuclei under the scrutiny of CTA,''
\href{https://www.sciencedirect.com/science/article/abs/pii/S0927650512002228?via%3Dihub}{Astropart. Phys. \textbf{43} (2013), 215-240.}
%doi:10.1016/j.astropartphys.2012.12.005
%[arXiv:1304.3024 [astro-ph.HE]].
%75 citations counted in INSPIRE as of 05 Feb 2023

%\cite{Aloisio:2009sj}
\bibitem{Aloisio:2009sj}
R.~Aloisio, V.~Berezinsky and A.~Gazizov,
%``Ultra High Energy Cosmic Rays: The disappointing model,''
\href{https://www.sciencedirect.com/science/article/abs/pii/S0927650510002434?via%3Dihub}{Astropart. Phys. \textbf{34} (2011), 620-626.}
%doi:10.1016/j.astropartphys.2010.12.008
%[arXiv:0907.5194 [astro-ph.HE]].
%139 citations counted in INSPIRE as of 05 Feb 2023

%\cite{Bergman:2006vt}
\bibitem{Bergman:2006vt}
D.~R.~Bergman [HiRes],
%``Observation of the GZK Cutoff Using the HiRes Detector,''
\href{https://www.sciencedirect.com/science/article/abs/pii/S0920563206008371}{Nucl. Phys. B Proc. Suppl. \textbf{165}, 19-26 (2007).}
%doi:10.1016/j.nuclphysbps.2006.11.004
%[arXiv:astro-ph/0609453 [astro-ph]].
%29 citations counted in INSPIRE as of 05 Mar 2023

%\cite{Haungs:2003jv}
\bibitem{Haungs:2003jv}
A.~Haungs, H.~Rebel and M.~Roth,
%``Energy spectrum and mass composition of high-energy cosmic rays,''
\href{https://iopscience.iop.org/article/10.1088/0034-4885/66/7/202}{Rept. Prog. Phys. \textbf{66}, 1145-1206 (2003).}
%doi:10.1088/0034-4885/66/7/202
%114 citations counted in INSPIRE as of 05 Mar 2023

%nsbhgravitationalcollapse

 \bibitem{joshi} 
 P. S. Joshi and I. H. Dwivedi,
 \href{https://journals.aps.org/prd/abstract/10.1103/PhysRevD.47.5357}{Phys. Rev. D \textbf{47}, 5357 (1993). } 

\bibitem{goswami}
R. Goswami and P. S. Joshi, 
\href{https://iopscience.iop.org/article/10.1088/0264-9381/21/15/002/meta}{Classical Quantum Gravity \textbf{21}, 3645 (2004).} 

\bibitem{mosani1}
K. Mosani, D. Dey, and P. S. Joshi, 
\href{https://journals.aps.org/prd/abstract/10.1103/P

hysRevD.101.044052}{Phys. Rev. D \textbf{101}, 044052 (2020).} 

\bibitem{mosani2}
K.~Mosani, D.~Dey and P.~S.~Joshi,
%``Global visibility of a strong curvature singularity in nonmarginally bound dust collapse,''
\href{https://journals.aps.org/prd/abstract/10.1103/PhysRevD.102.044037}{Phys. Rev. D \textbf{102}, no.4, 044037 (2020).}

\bibitem{mosani3}
K.~Mosani, D.~Dey and P.~S.~Joshi,
%``Globally visible singularity in an astrophysical setup,''
\href{https://academic.oup.com/mnras/article-abstract/504/4/4743/6253197?redirectedFrom=fulltext}{Monthly Notices of the Royal Astronomical Society, \textbf{ 504}, 4, 4743–4750 (2021).}

\bibitem{mosani4}
D.~Dey, P.~S.~Joshi, K.~Mosani and V.~Vertogradov,
%``Causal structure of singularity in non-spherical gravitational collapse,''
\href{https://link.springer.com/article/10.1140/epjc/s10052-022-10401-1}{Eur. Phys. J. C \textbf{82} (2022) no.5, 431.}
%doi:10.1140/epjc/s10052-022-10401-1
%[arXiv:2103.07190 [gr-qc]].
%9 citations counted in INSPIRE as of 28 Dec 2022

%\cite{Deshingkar:1998ge}
\bibitem{Deshingkar:1998ge}
S.~S.~Deshingkar, S.~Jhingan and P.~S.~Joshi,
%``On the global visibility of singularity in quasispherical collapse,''
\href{https://link.springer.com/article/10.1023/A:1018813108516}{Gen. Rel. Grav. \textbf{30} (1998), 1477-1499.}
%doi:10.1023/A:1018813108516
%[arXiv:gr-qc/9806055 [gr-qc]].
%34 citations counted in INSPIRE as of 28 Dec 2022

%\cite{Jhingan:2014gpa}
\bibitem{Jhingan:2014gpa}
S.~Jhingan and S.~Kaushik,
%``Global visibility of a singularity in spherically symmetric gravitational collapse,''
\href{https://journals.aps.org/prd/abstract/10.1103/PhysRevD.90.024009}{Phys. Rev. D \textbf{90} (2014) no.2, 024009.}
%doi:10.1103/PhysRevD.90.024009
%[arXiv:1406.3087 [gr-qc]].
%9 citations counted in INSPIRE as of 28 Dec 2022

%\cite{Joshi:2011zm}
\bibitem{Joshi:2011zm}
P.~S.~Joshi, D.~Malafarina and R.~Narayan,
%``Equilibrium configurations from gravitational collapse,''
\href{https://iopscience.iop.org/article/10.1088/0264-9381/28/23/235018}{Class. Quant. Grav. \textbf{28} (2011), 235018.}
%doi:10.1088/0264-9381/28/23/235018
%[arXiv:1106.5438 [gr-qc]].
%86 citations counted in INSPIRE as of 28 Dec 2022

%\cite{EventHorizonTelescope:2022xqj}
\bibitem{EventHorizonTelescope:2022xqj}
K.~Akiyama \textit{et al.} [Event Horizon Telescope],
%``First Sagittarius A* Event Horizon Telescope Results. VI. Testing the Black Hole Metric,''
\href{doi:10.3847/2041-8213/ac6756}{Astrophys. J. Lett. \textbf{930} (2022) no.2, L17.}
%12 citations counted in INSPIRE as of 18 May 2022

%accreationdisk

%\cite{Page:1974he}
\bibitem{Page:1974he}
D.~N.~Page and K.~S.~Thorne,
%``Disk-Accretion onto a Black Hole. Time-Averaged Structure of Accretion Disk,''
\href{https://ui.adsabs.harvard.edu/abs/1974ApJ...191..499P/abstract}{Astrophys. J. \textbf{191} (1974), 499-506.}
%doi:10.1086/152990
%525 citations counted in INSPIRE as of 27 Dec 2022

%\cite{Liu:2020vkh}
\bibitem{Liu:2020vkh}
C.~Liu, T.~Zhu and Q.~Wu,
%``Thin Accretion Disk around a four-dimensional Einstein-Gauss-Bonnet Black Hole,''
\href{https://iopscience.iop.org/article/10.1088/1674-1137/abc16c}{Chin. Phys. C \textbf{45} (2021) no.1, 015105.}
%doi:10.1088/1674-1137/abc16c
%[arXiv:2004.01662 [gr-qc]].
%76 citations counted in INSPIRE as of 27 Dec 2022

%\cite{Liu:2021yev}
\bibitem{Liu:2021yev}
C.~Liu, S.~Yang, Q.~Wu and T.~Zhu,
%``Thin accretion disk onto slowly rotating black holes in Einstein-\AE{}ther theory,''
\href{https://iopscience.iop.org/article/10.1088/1475-7516/2022/02/034}{JCAP \textbf{02} (2022) no.02, 034.}
%doi:10.1088/1475-7516/2022/02/034
%[arXiv:2107.04811 [gr-qc]].
%15 citations counted in INSPIRE as of 27 Dec 2022

%\cite{Joshi:2013dva}
\bibitem{Joshi:2013dva}
P.~S.~Joshi, D.~Malafarina and R.~Narayan,
%``Distinguishing black holes from naked singularities through their accretion disc properties,''
\href{https://iopscience.iop.org/article/10.1088/0264-9381/31/1/015002}{Class. Quant. Grav. \textbf{31} (2014), 015002.}
%doi:10.1088/0264-9381/31/1/015002
%[arXiv:1304.7331 [gr-qc]].
%117 citations counted in INSPIRE as of 27 Dec 2022

%\cite{Bambhaniya:2021ugr}
\bibitem{Bambhaniya:2021ugr}
P.~Bambhaniya, S.~K, K.~Jusufi and P.~S.~Joshi,
%``Thin accretion disk in the Simpson-Visser black-bounce and wormhole spacetimes,''
\href{https://journals.aps.org/prd/abstract/10.1103/PhysRevD.105.023021}{Phys. Rev. D \textbf{105} (2022) no.2, 023021.}
%doi:10.1103/PhysRevD.105.023021
%[arXiv:2109.15054 [gr-qc]].
%18 citations counted in INSPIRE as of 27 Dec 2022

%\cite{Tahelyani:2022uxw}
\bibitem{Tahelyani:2022uxw}
D.~Tahelyani, A.~B.~Joshi, D.~Dey and P.~S.~Joshi,
%``Comparing thin accretion disk properties of naked singularities and black holes,''
\href{https://journals.aps.org/prd/abstract/10.1103/PhysRevD.106.044036}{Phys. Rev. D \textbf{106} (2022) no.4, 044036.}
%doi:10.1103/PhysRevD.106.044036
%[arXiv:2205.04055 [gr-qc]].
%5 citations counted in INSPIRE as of 28 Dec 2022

%\cite{Rahaman:2021kge}
\bibitem{Rahaman:2021kge}
F.~Rahaman, T.~Manna, R.~Shaikh, S.~Aktar, M.~Mondal and B.~Samanta,
%``Thin accretion disks around traversable wormholes,''
\href{https://www.sciencedirect.com/science/article/pii/S0550321321002455?via%3Dihub}{Nucl. Phys. B \textbf{972} (2021), 115548.}
%doi:10.1016/j.nuclphysb.2021.115548
%[arXiv:2110.09820 [gr-qc]].
%10 citations counted in INSPIRE as of 27 Dec 2022

%\cite{Harko:2008vy}
\bibitem{Harko:2008vy}
T.~Harko, Z.~Kovacs and F.~S.~N.~Lobo,
%``Electromagnetic signatures of thin accretion disks in wormhole geometries,''
\href{https://journals.aps.org/prd/abstract/10.1103/PhysRevD.78.084005}{Phys. Rev. D \textbf{78} (2008), 084005.}
%doi:10.1103/PhysRevD.78.084005
%[arXiv:0808.3306 [gr-qc]].
%88 citations counted in INSPIRE as of 27 Dec 2022

%\cite{Harko:2009xf}
\bibitem{Harko:2009xf}
T.~Harko, Z.~Kovacs and F.~S.~N.~Lobo,
%``Thin accretion disks in stationary axisymmetric wormhole spacetimes,''
\href{https://journals.aps.org/prd/abstract/10.1103/PhysRevD.79.064001}{Phys. Rev. D \textbf{79} (2009), 064001.}
%doi:10.1103/PhysRevD.79.064001
%[arXiv:0901.3926 [gr-qc]].
%126 citations counted in INSPIRE as of 27 Dec 2022

%\cite{Kovacs:2010xm}
\bibitem{Kovacs:2010xm}
Z.~Kovacs and T.~Harko,
%``Can accretion disk properties observationally distinguish black holes from naked singularities?,''
\href{https://journals.aps.org/prd/abstract/10.1103/PhysRevD.82.124047}{Phys. Rev. D \textbf{82} (2010), 124047.}
%doi:10.1103/PhysRevD.82.124047
%[arXiv:1011.4127 [gr-qc]].
%77 citations counted in INSPIRE as of 27 Dec 2022

%\cite{Guo:2020tgv}
\bibitem{Guo:2020tgv}
J.~Q.~Guo, P.~S.~Joshi, R.~Narayan and L.~Zhang,
%``Accretion disks around naked singularities,''
\href{https://iopscience.iop.org/article/10.1088/1361-6382/abce44}{Class. Quant. Grav. \textbf{38} (2021) no.3, 035012.}
%doi:10.1088/1361-6382/abce44
%[arXiv:2011.06154 [gr-qc]].
%11 citations counted in INSPIRE as of 27 Dec 2022

%\cite{Chowdhury:2011aa}
\bibitem{Chowdhury:2011aa}
A.~N.~Chowdhury, M.~Patil, D.~Malafarina and P.~S.~Joshi,
%``Circular geodesics and accretion disks in Janis-Newman-Winicour and Gamma metric,''
\href{https://journals.aps.org/prd/abstract/10.1103/PhysRevD.85.104031}{Phys. Rev. D \textbf{85} (2012), 104031.}
%doi:10.1103/PhysRevD.85.104031
%[arXiv:1112.2522 [gr-qc]].
%97 citations counted in INSPIRE as of 27 Dec 2022

%tidalforce

%\cite{Lattimer:1976kbf}
\bibitem{Lattimer:1976kbf}
J.~M.~Lattimer and D.~N.~Schramm,
%``The tidal disruption of neutron stars by black holes in close binaries,''
\href{https://ui.adsabs.harvard.edu/abs/1976ApJ...210..549L/abstract}{Astrophys. J. \textbf{210} (1976), 549.}
%doi:10.1086/154860
%282 citations counted in INSPIRE as of 28 Dec 2022

%\cite{Chicone:2004pv}
\bibitem{Chicone:2004pv}
C.~Chicone and B.~Mashhoon,
%``Tidal dynamics of relativistic flows near black holes,''
\href{https://onlinelibrary.wiley.com/doi/10.1002/andp.20055170502}{Annalen Phys. \textbf{14} (2005), 290-308.}
%doi:10.1002/andp.200410126
%[arXiv:astro-ph/0404170 [astro-ph]].
%18 citations counted in INSPIRE as of 28 Dec 2022

%\cite{Madan:2022spd}
\bibitem{Madan:2022spd}
S.~Madan and P.~Bambhaniya,
%``Tidal force effects and periodic orbits in null naked singularity spacetime,''
\href{https://arxiv.org/pdf/2201.13163.pdf}{[arXiv:2201.13163 [gr-qc]].}
%2 citations counted in INSPIRE as of 28 Dec 2022

%gravitaionallensing

%\cite{Shaikh:2019jfr}
\bibitem{Shaikh:2019jfr}
R.~Shaikh, P.~Banerjee, S.~Paul and T.~Sarkar,
%``Strong gravitational lensing by wormholes,''
\href{doi:10.1088/1475-7516/2019/07/028}{JCAP \textbf{07} (2019), 028.}
%[arXiv:1905.06932 [gr-qc]].
%41 citations counted in INSPIRE as of 18 May 2022

%\cite{Shaikh:2018oul}
\bibitem{Shaikh:2018oul}
R.~Shaikh, P.~Banerjee, S.~Paul and T.~Sarkar,
%``A novel gravitational lensing feature by wormholes,''
\href{doi:10.1016/j.physletb.2018.12.030}{Phys. Lett. B \textbf{789} (2019), 270-275
[erratum: Phys. Lett. B \textbf{791} (2019), 422-423]}
%[arXiv:1811.08245 [gr-qc]].
%48 citations counted in INSPIRE as of 18 May 2022

%\cite{Paul:2020ufc}
\bibitem{Paul:2020ufc}
S.~Paul,
%``Strong gravitational lensing by a strongly naked null singularity,''
\href{doi:10.1103/PhysRevD.102.064045}{Phys. Rev. D \textbf{102} (2020) no.6, 064045.}
%[arXiv:2007.05509 [gr-qc]].
%14 citations counted in INSPIRE as of 18 May 2022

%\cite{Virbhadra:2007kw}
\bibitem{Virbhadra:2007kw}
K.~S.~Virbhadra and C.~R.~Keeton,
%``Time delay and magnification centroid due to gravitational lensing by black holes and naked singularities,''
\href{doi:10.1103/PhysRevD.77.124014}{Phys. Rev. D \textbf{77} (2008), 124014.}
%[arXiv:0710.2333 [gr-qc]].
%311 citations counted in INSPIRE as of 18 May 2022

%\cite{Gyulchev:2008ff}
\bibitem{Gyulchev:2008ff}
G.~N.~Gyulchev and S.~S.~Yazadjiev,
%``Gravitational Lensing by Rotating Naked Singularities,''
\href{doi:10.1103/PhysRevD.78.083004}{Phys. Rev. D \textbf{78} (2008), 083004.}
%[arXiv:0806.3289 [gr-qc]].
%82 citations counted in INSPIRE as of 18 May 2022

%\cite{Sahu:2012er}

%\cite{Kala:2020prt}
\bibitem{Kala:2020prt}
S.~Kala, Saurabh, H.~Nandan and P.~Sharma,
%``Deflection of light and shadow cast by a dual-charged stringy black hole,''
\href{doi:10.1142/S0217751X20501778}{Int. J. Mod. Phys. A \textbf{35}  (2020), no.28, 2050177.}
%arXiv:2010.03615 [gr-qc]].
%6 citations counted in INSPIRE as of 01 Jun 2022

\bibitem{Sahu:2012er}
S.~Sahu, M.~Patil, D.~Narasimha and P.~S.~Joshi,
%``Can strong gravitational lensing distinguish naked singularities from black holes?,''
\href{doi:10.1103/PhysRevD.86.063010}{Phys. Rev. D \textbf{86} (2012), 063010.}
%[arXiv:1206.3077 [gr-qc]].
%60 citations counted in INSPIRE as of 18 May 2022

%precession

\bibitem{Martinez}
C.~Mart\'\i{}nez, N.~Parra, N.~Vald\'es and J.~Zanelli,
%``Geodesic structure of naked singularities in AdS$_3$ spacetime,''
\href{https://journals.aps.org/prd/abstract/10.1103/PhysRevD.100.024026}{Phys. Rev. D \textbf{100}, no.2, 024026 (2019).}

\bibitem{Eva1}
 %Analytic solutions of the geodesic equation in axially symmetric space-times
E. Hackmann, V. Kagramanova2, J. Kunz2 and C. Lämmerzahl1
\href{https://iopscience.iop.org/article/10.1209/0295-5075/88/30008}{Published 13 November 2009 • Europhysics} 

\bibitem{Eva2}
%Motion of spinning test bodies in Kerr spacetime
Eva Hackmann, Claus Lämmerzahl, Yuri N. Obukhov, Dirk Puetzfeld, and Isabell Schaffer
\href{https://journals.aps.org/prd/abstract/10.1103/PhysRevD.90.064035}{Phys. Rev. D 90, 064035 – Published 22 September 2014}

\bibitem{tsirulev}
%Bound orbits near scalar field naked singularities
I. M. PotashovJu. V. TchemarinaA. N. Tsirulev
\href{https://doi.org/10.1140/epjc/s10052-019-7192-7}{The European Physical Journal C, August 2019, 79:709}

\bibitem{Joshi:2019rdo} 
  A.~B.~Joshi, P.~Bambhaniya, D.~Dey and P.~S.~Joshi,
  %``Timelike Geodesics in Naked Singularity and Black Hole Spacetimes II,''
  \href{https://arxiv.org/abs/1909.08873}{arXiv:1909.08873 [gr-qc].} 

\bibitem{Bambhaniya:2019pbr}
P.~Bambhaniya, A.~B.~Joshi, D.~Dey and P.~S.~Joshi,
%``Timelike geodesics in Naked Singularity and Black Hole Spacetimes,''
\href{https://journals.aps.org/prd/abstract/10.1103/PhysRevD.100.124020}{Phys. Rev. D \textbf{100}, no.12, 124020 (2019).}

\bibitem{Dey:2019fpv} 
  D.~Dey, P.~S.~Joshi, A.~Joshi and P.~Bambhaniya,
  %``Towards an observational test of black hole versus naked singularity at the galactic center,''
\href{https://www.worldscientific.com/doi/abs/10.1142/S0218271819300246}{Int.\ J.\ Mod.\ Phys.\ D {\bf 28}, no. 14, 1930024 (2019).}

\bibitem{Bam2020}
P.~Bambhaniya, D.~N.~Solanki, D.~Dey, A.~B.~Joshi, P.~S.~Joshi and V.~Patel,
%``Precession of timelike bound orbits in Kerr spacetime,''
\href{https://link.springer.com/article/10.1140/epjc/s10052-021-08997-x}{Eur. Phys. J. C \textbf{81}, no.3, 205 (2021).}
%doi:10.1140/epjc/s10052-021-08997-x
%[arXiv:2007.12086 [gr-qc]].
%19 citations counted in INSPIRE as of 26 Jan 2023

%\cite{Bambhaniya:2022xbz}
\bibitem{Bambhaniya:2022xbz}
P.~Bambhaniya, A.~B.~Joshi, D.~Dey, P.~S.~Joshi, A.~Mazumdar, T.~Harada and K.~i.~Nakao,
%``Relativistic orbits of S2 star in the presence of scalar field,''
\href{https://inspirehep.net/literature/2156721}{[arXiv:2209.12610 [gr-qc]].}
%0 citations counted in INSPIRE as of 26 Jan 2023

%shadow

\bibitem{Gralla:2019xty} 
S.~E.~Gralla, D.~E.~Holz and R.~M.~Wald,
%``Black Hole Shadows, Photon Rings, and Lensing Rings,''
\href{https://journals.aps.org/prd/abstract/10.1103/PhysRevD.100.024018}{Phys.\ Rev.\ D {\bf 100}, no. 2, 024018 (2019).}

\bibitem{Vagnozzi:2019apd} 
S.~Vagnozzi and L.~Visinelli,
%``Hunting for extra dimensions in the shadow of M87*,''
\href{https://journals.aps.org/prd/abstract/10.1103/PhysRevD.100.024020}{Phys.\ Rev.\ D {\bf 100}, no. 2, 024020 (2019).}

%\cite{Chen:2022nbb}
\bibitem{Chen:2022nbb}
Y.~Chen, R.~Roy, S.~Vagnozzi and L.~Visinelli,
%``Superradiant evolution of the shadow and photon ring of Sgr A$^\star$,''
\href{https://arxiv.org/pdf/2205.06238.pdf}{arXiv:2205.06238 [astro-ph.HE] (2022).}

\bibitem{Dey:2020haf}
D.~Dey, R.~Shaikh and P.~S.~Joshi,
%``Perihelion Precession and Shadows near Blackholes and Naked Singularities,''
\href{https://journals.aps.org/prd/abstract/10.1103/PhysRevD.102.044042}{Phys. Rev. D \textbf{102}, no.4, 044042 (2020).}

\bibitem{atamurotov_2015}
F. Atamurotov, B. Ahmedov, and A. Abdujabbarov, 
%``Optical properties of black holes in the presence of a plasma: The shadow," 
\href{https://journals.aps.org/prd/abstract/10.1103/PhysRevD.92.084005}{Phys. Rev. D {\bf 92}, 084005 (2015).}

\bibitem{abdujabbarov_2015b} 
A. A. Abdujabbarov, L. Rezzolla, and B. J. Ahmedov, 
%``A coordinate-independent characterization of a black hole shadow," 
\href{https://academic.oup.com/mnras/article/454/3/2423/1196174}{Mon. Not. R. Astron. Soc. {\bf 454}, 2423 (2015).}

\bibitem{Li:2021}
G.~P.~Li and K.~J.~He,
%``Shadows and rings of the Kehagias-Sfetsos black hole surrounded by thin disk accretion,''
\href{https://iopscience.iop.org/article/10.1088/1475-7516/2021/06/037}{JCAP \textbf{06}, 037 (2021).}

\bibitem{Hu:2020usx}
Z.~Hu, Z.~Zhong, P.~C.~Li, M.~Guo and B.~Chen,
%``QED effect on a black hole shadow,''
\href{https://journals.aps.org/prd/abstract/10.1103/PhysRevD.103.044057}{Phys. Rev. D \textbf{103}, no.4, 044057 (2021).}

%\cite{Patel:2022vlu}
\bibitem{Patel:2022vlu}
V.~Patel, D.~Tahelyani, A.~B.~Joshi, D.~Dey and P.~S.~Joshi,
%``Light trajectory and shadow shape in the rotating naked singularity,''
\href{https://link.springer.com/article/10.1140/epjc/s10052-022-10638-w}{Eur. Phys. J. C \textbf{82} (2022) no.9, 798.}
%doi:10.1140/epjc/s10052-022-10638-w
%[arXiv:2206.06750 [gr-qc]].
%1 citations counted in INSPIRE as of 28 Dec 2022

\bibitem{Kaur:2021wgy}
K.~P.~Kaur, P.~S.~Joshi, D.~Dey, A.~B.~Joshi and R.~P.~Desai,
%``Comparing Shadows of Blackhole and Naked Singularity,''
\href{https://arxiv.org/pdf/2106.13175.pdf}{arXiv:2106.13175 [gr-qc], (2021).}

%\cite{Saurabh:2020zqg}
\bibitem{Saurabh:2020zqg}
K.~Saurabh and K.~Jusufi,
%``Imprints of dark matter on black hole shadows using spherical accretions,''
\href{https://link.springer.com/article/10.1140/epjc/s10052-021-09280-9}{Eur. Phys. J. C \textbf{81}, no.6, 490 (2021).}
%doi:10.1140/epjc/s10052-021-09280-9
%[arXiv:2009.10599 [gr-qc]].
%40 citations counted in INSPIRE as of 19 Jan 2023

%\cite{Pugliese:2022oes}
\bibitem{Pugliese:2022oes}
D.~Pugliese and Z.~Stuchl\'\i{}k,
%``Dark matter effect on black hole accretion disks,''
\href{https://journals.aps.org/prd/abstract/10.1103/PhysRevD.106.124034}{Phys. Rev. D \textbf{106}, no.12, 124034 (2022).}
%doi:10.1103/PhysRevD.106.124034
%0 citations counted in INSPIRE as of 19 Jan 2023

%\cite{Vagnozzi:2022moj}
\bibitem{Vagnozzi:2022moj}
S.~Vagnozzi, R.~Roy, Y.~D.~Tsai, L.~Visinelli, M.~Afrin, A.~Allahyari, P.~Bambhaniya, D.~Dey, S.~G.~Ghosh and P.~S.~Joshi, \textit{et al.}
%``Horizon-scale tests of gravity theories and fundamental physics from the Event Horizon Telescope image of Sagittarius A$^*$,''
\href{https://inspirehep.net/literature/2082606}{[arXiv:2205.07787 [gr-qc]].}
%75 citations counted in INSPIRE as of 26 Jan 2023

%\cite{Saurabh:2022jjv}
\bibitem{Saurabh:2022jjv}
Saurabh, P.~Bambhaniya and P.~S.~Joshi,
%``Probing the Shadow Image of the Sagittarius A* with Event Horizon Telescope,''
\href{https://inspirehep.net/literature/2023956}{[arXiv:2202.00588 [gr-qc]].}
%4 citations counted in INSPIRE as of 26 Jan 2023

%energyextraction

%\cite{Patel:2022jbk}
\bibitem{Patel:2022jbk}
V.~Patel, K.~Acharya, P.~Bambhaniya and P.~S.~Joshi,
%``Rotational energy extraction from the Kerr black hole's mimickers,''
\href{https://www.mdpi.com/2218-1997/8/11/571}{Universe \textbf{8} (2022) no.11, 571.}
%doi:10.3390/universe8110571
%[arXiv:2206.00428 [gr-qc]].
%3 citations counted in INSPIRE as of 28 Dec 2022

%\cite{Patel:2023efv}
\bibitem{Patel:2023efv}
V.~Patel, K.~Acharya, P.~Bambhaniya and P.~S.~Joshi,
%``Energy extraction from Janis-Newman-Winicour naked singularity,''
\href{https://inspirehep.net/literature/2626726}{[arXiv:2301.11052 [gr-qc]].}
%0 citations counted in INSPIRE as of 05 Mar 2023

%\cite{Patil:2011yb}
\bibitem{Patil:2011yb}
M.~Patil and P.~S.~Joshi,
%``High energy particle collisions in superspinning Kerr geometry,''
\href{https://journals.aps.org/prd/abstract/10.1103/PhysRevD.84.104001}{Phys. Rev. D \textbf{84} (2011), 104001.}
%doi:10.1103/PhysRevD.84.104001
%[arXiv:1103.1083 [gr-qc]].
%46 citations counted in INSPIRE as of 02 Mar 2023

%\cite{Banados:2009pr}
\bibitem{Banados:2009pr}
M.~Banados, J.~Silk and S.~M.~West,
%``Kerr Black Holes as Particle Accelerators to Arbitrarily High Energy,''
\href{https://journals.aps.org/prl/abstract/10.1103/PhysRevLett.103.111102}{Phys. Rev. Lett. \textbf{103} (2009), 111102.}
%doi:10.1103/PhysRevLett.103.111102
%[arXiv:0909.0169 [hep-ph]].
%397 citations counted in INSPIRE as of 02 Mar 2023

%\cite{Patil:2011aa}
\bibitem{Patil:2011aa}
M.~Patil and P.~S.~Joshi,
%``Acceleration of particles in Janis-Newman-Winicour singularities,''
\href{https://journals.aps.org/prd/abstract/10.1103/PhysRevD.85.104014}{Phys. Rev. D \textbf{85} (2012), 104014}
%doi:10.1103/PhysRevD.85.104014
%[arXiv:1112.2525 [gr-qc]].
%39 citations counted in INSPIRE as of 02 Mar 2023

%\cite{Patil:2011uf}
\bibitem{Patil:2011uf}
M.~Patil, P.~S.~Joshi, M.~Kimura and K.~i.~Nakao,
%``Acceleration of particles and shells by Reissner-Nordstr\"om naked singularities,''
\href{https://journals.aps.org/prd/abstract/10.1103/PhysRevD.86.084023}{Phys. Rev. D \textbf{86} (2012), 084023.}
%doi:10.1103/PhysRevD.86.084023
%[arXiv:1108.0288 [gr-qc]].
%61 citations counted in INSPIRE as of 02 Mar 2023

%\cite{Patil:2011ya}
\bibitem{Patil:2011ya}
M.~Patil and P.~S.~Joshi,
%``Kerr Naked Singularities as Particle Accelerators,''
\href{https://iopscience.iop.org/article/10.1088/0264-9381/28/23/235012}{Class. Quant. Grav. \textbf{28} (2011), 235012.}
%doi:10.1088/0264-9381/28/23/235012
%[arXiv:1103.1082 [gr-qc]].
%83 citations counted in INSPIRE as of 02 Mar 2023

%\cite{Patil:2011aw}
\bibitem{Patil:2011aw}
M.~Patil, P.~S.~Joshi and D.~Malafarina,
%``Naked Singularities as Particle Accelerators II,''
\href{https://journals.aps.org/prd/abstract/10.1103/PhysRevD.83.064007}{Phys. Rev. D \textbf{83} (2011), 064007.}

%\cite{Oppenheimer:1939ue}
\bibitem{Oppenheimer:1939ue}
J.~R.~Oppenheimer and H.~Snyder,
%``On Continued gravitational contraction,''
\href{https://journals.aps.org/pr/abstract/10.1103/PhysRev.56.455}{Phys. Rev. \textbf{56}, 455-459 (1939).}
%doi:10.1103/PhysRev.56.455
%1100 citations counted in INSPIRE as of 05 Mar 2023

%\cite{Mosani:2022nsp}
\bibitem{Mosani:2022nsp}
K.~Mosani,
%``Aspects of visible singularities in gravitational collapse,''
\href{https://inspirehep.net/literature/2181892}{[arXiv:2211.06604 [gr-qc]].}
%0 citations counted in INSPIRE as of 28 Mar 2023

%\cite{Choptuik:2009ww}
\bibitem{Choptuik:2009ww}
M.~W.~Choptuik and F.~Pretorius,
%``Ultra Relativistic Particle Collisions,''
\href{https://journals.aps.org/prl/abstract/10.1103/PhysRevLett.104.111101}{Phys. Rev. Lett. \textbf{104} (2010), 111101.}
%doi:10.1103/PhysRevLett.104.111101
%[arXiv:0908.1780 [gr-qc]].
%130 citations counted in INSPIRE as of 05 Feb 2023


%\cite{Choptuik:2009ww}
\bibitem{Choptuik:2009ww}
M.~W.~Choptuik and F.~Pretorius,
%``Ultra Relativistic Particle Collisions,''
\href{https://journals.aps.org/prl/abstract/10.1103/PhysRevLett.104.111101}{Phys. Rev. Lett. \textbf{104} (2010), 111101}
%doi:10.1103/PhysRevLett.104.111101
%[arXiv:0908.1780 [gr-qc]].
%131 citations counted in INSPIRE as of 29 Mar 2023


%\cite{CMS:2010oej}
\bibitem{CMS:2010oej}
V.~Khachatryan \textit{et al.} [CMS],
\href{https://www.sciencedirect.com/science/article/pii/S0370269311001778?via%3Dihub}{Phys. Lett. B \textbf{697} (2011), 434-453.}
%``Search for Microscopic Black Hole Signatures at the Large Hadron Collider,''
%doi:10.1016/j.physletb.2011.02.032
%[arXiv:1012.3375 [hep-ex]].
%124 citations counted in INSPIRE as of 05 Mar 2023

%\cite{Alok:2022xiy}
\bibitem{Alok:2022xiy}
A.~K.~Alok, T.~Sarkar and S.~Yadav,
%``Effects of non-standard interaction on microscopic black holes from ultra-high energy neutrinos,''
\href{https://link.springer.com/article/10.1140/epjc/s10052-022-10674-6}{Eur. Phys. J. C \textbf{82} (2022) no.8, 711.}
%doi:10.1140/epjc/s10052-022-10674-6
%[arXiv:2202.02775 [hep-ph]].
%1 citations counted in INSPIRE as of 09 Mar 2023

%\cite{Kovacik:2021qms}
\bibitem{Kovacik:2021qms}
S.~Kov\'a\v{c}ik,
%``Hawking-radiation recoil of microscopic black hole,''
\href{https://www.sciencedirect.com/science/article/abs/pii/S2212686421001333?via%3Dihub}{Phys. Dark Univ. \textbf{34} (2021), 100906.}
%doi:10.1016/j.dark.2021.100906
%[arXiv:2102.06517 [gr-qc]].
%8 citations counted in INSPIRE as of 05 Feb 2023

%\cite{Anchordoqui:2002cp}
\bibitem{Anchordoqui:2002cp}
L.~Anchordoqui and H.~Goldberg,
%``Black hole chromosphere at the CERN LHC,''
\href{https://journals.aps.org/prd/abstract/10.1103/PhysRevD.67.064010}{Phys. Rev. D \textbf{67} (2003), 064010.}
%doi:10.1103/PhysRevD.67.064010
%[arXiv:hep-ph/0209337 [hep-ph]].
%78 citations counted in INSPIRE as of 09 Mar 2023

%\cite{Arbey:2019mbc}
\bibitem{Arbey:2019mbc}
A.~Arbey and J.~Auffinger,
%``BlackHawk: A public code for calculating the Hawking evaporation spectra of any black hole distribution,''
\href{https://link.springer.com/article/10.1140/epjc/s10052-019-7161-1}{Eur. Phys. J. C \textbf{79} (2019) no.8, 693.}
%doi:10.1140/epjc/s10052-019-7161-1
%[arXiv:1905.04268 [gr-qc]].
%94 citations counted in INSPIRE as of 27 Feb 2023

\end{thebibliography}
\end{document}
