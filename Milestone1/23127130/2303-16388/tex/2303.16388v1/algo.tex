\section{PTAS}
\label{sec:ptas}

We present a PTAS for computing an $\eps$-approximate BNE in an FPA under the uniform tie-breaking rule.
%, and a quasi-polynomial time algorithm for ultimately uniform tie-breaking rule.


\PTAS*


Our approach proceeds in the following four steps. 
Given an FPA $(\mN, \mB, \D, \Gamma)$ where $\Gamma$ is the uniform tie-breaking rule, we first round the bidding space $\mB$ and reduce its size to $O(1/\eps)$. 
We then prune the valuation distribution $\D$ and work on a weak notion of $(\eps, \delta)$-approximate BNE that relaxes the no-overbidding requirement.
In the third step, we argue the existence of an $(\eps, \delta)$-approximate BNE profile over a discretized space and in the final step, we develop a suitable searching algorithm for $(\eps, \delta)$-approximate BNE in the discretized space.

\vspace{+2mm}
{\noindent \bf Step 1: Rounding bids. \ \ }
Given a first-price auction with bidding space $\mB = \{b_0, b_1, \ldots, b_m\}$ with $0 = b_0 < b_1 < \cdots < b_m \leq 1$, 
we define $\mB_{\eps} = \{b_{0, \max}, b_{1, \max}, \ldots, b_{10\eps^{-1}, \max}\}$ as follows.
First we take $b_{0, \max} = b_0=0$.
Then for each $t \in [10\eps^{-1}]$, let %$\mB_t$ be the intersection of $\mB$ and the interval $(\frac{(t-1)\eps}{10}, \frac{t\eps}{10}]$, i.e., $\mB_t = \mB \cap (\frac{(t-1)\eps}{10}, \frac{t\eps}{10}]$.
%For each grid $t \in [10\eps^{-1}]$, let 
$b_{t, \max}$ be the maximum bid in
$$
\mB_t := \mB \cap \left(\frac{(t-1)\eps}{10}, \frac{t\eps}{10}\right]
$$
if $\mB_t$ is not empty; and set
$b_{t,\max}=\nil$ (meaning that we don't add an element to $\mB_\eps$) if $\mB_t$ is empty. %$$$
%$\mB_t$ and it equals $\nil$ if %$\mB_t = \emptyset$, i.e.
%\begin{align*}
%b_{t, \max} = \left\{
%\begin{matrix}
%\max_{b \in \mB_t} b & |\mB_t| \geq 1\\
%\nil & \mB_t = \emptyset
%\end{matrix}
%\right.
%\end{align*}
%and \footnote{\color{red}Xi: I added $=0$. We assume $b_0=0$, right?}. , 
We prove that it suffices to find an $({\eps}/{2})$-approximate BNE over bidding space $\mB_{\eps}$.
\begin{lemma}
\label{lem:ptas-step1}
Given any first-price auction $(\mN, \mB, \D, \Gamma)$, let $B_{\eps}$ be the rounded bidding space defined above. Then any $ ({\eps}/{2})$-approximate BNE  of  $(\mN, \mB_{\eps}, \D, \Gamma)$ is also an $\eps$-approximate BNE of $(\mN, \mB, \D, \Gamma)$.
\end{lemma}
\begin{proof}
Let $s=(s_1,\ldots,s_n)$ be an $(\eps/2)$-approximate BNE of $(\mN, \mB_{\eps}, \D, \Gamma)$. By definition, each $s_i$ satisfies
%$\frac{\eps}{2}$-approximate BNE of FPA $(\mN, \mB_{\eps}, \D, \Gamma)$, the equilibrium strategy $s_i$ of any player $i \in \mN$ satisfies
\begin{align*}
\E_{v\sim \D}\big[u_i(v_i; s_i(v_i), s_{-i}(v_{-i})) - u_i(v_i; \bs_{\eps}(v_i, s_{-i}), s_{-i}(v_{-i}))\big] \geq -\frac{\eps}{2},
\end{align*}
where 
\[
\bs_{\eps}(v_i, s_{-i}) :=  \arg\max_{b\in \mB_{\eps}} \E_{v_{-i}\sim \D_{-i}}u_i(v_i; b, s_{-i}(v_{-i}))
\]
is the best response in $\mB_{\eps}$.
For each value $v_i \in [0,1]$, let 
\[
\eps_i(v_i) := \E_{v_{-i}\sim \D_{-i}}\big[u_i(v_i; s_i(v_i), s_{-i}(v_{-i})) - u_i(v_i; \bs_{\eps}(v_i, s_{-i}), s_{-i}(v_{-i}))\big]
\]
be the deficiency of utility at value point $v_i$ and we know that $\E_{v_i}[\eps_i(v_i)] \geq -{\eps}/{2}$. 

Let $\bs(v_i, s_{-i}) \in \mB$ be the best response in the original bidding space $\mB$.
Given that the case of $\bs(v_i, s_{-i})  = 0$ is trivial, we may assume without loss of generality that
  $$\bs(v_i, s_{-i}) \in \left(\frac{(t-1)\eps}{10}, \frac{t\eps}{10}\right],\quad\text{for some $t \in [10\eps^{-1}]$;}$$ hence $\mB_{t}\neq \emptyset$. 
Our goal is to prove that the revenue of bidding $b_{t,\max}$ or $0$ (both of which are in $\mB_\eps$) is at most $ {\eps}/{10}$ worse comparing to $\bs(v_{i}, s_{-i})$.

{\bf Case 1.}  Suppose $v_i \geq  {t\eps}/{10}$. Then it is valid to bid $b_{t, \max}$ and we have
\begin{align}
     \E_{v_{-i}\sim \D_{-i}}&\big[u_i(v_i; \bs_{\eps}(v_i, s_{-i}), s_{-i}(v_{-i})) - u_i(v_i; \bs(v_i, s_{-i}), s_{-i}(v_{-i}))\big]\notag \\
    \geq &~ \E_{v_{-i}\sim \D_{-i}}\big[u_i(v_i; b_{t, \max}, s_{-i}(v_{-i})) - u_i(v_i; \bs(v_i, s_{-i}), s_{-i}(v_{-i}))\big]\notag \\
    = &~ \E_{v_{-i}\sim \D_{-i}}[(v_i - b_{t, \max})\Gamma_i(b_{t, \max}, s_{-i}(v_{-i})) - (v_i - \bs(v_i, s_{-i}))\Gamma_i(\bs(v_i, s_{-i}), s_{-i}(v_{-i}))] \notag \\
    \geq &~ \bs(v_i, s_{-i}) - b_{t, \max}   \geq -\eps/10 \label{eq:step1-1}
\end{align}%\footnote{\color{red}I think on (14) it should be $b_{t,\max}-bs(v_i,s_{-i})$?}
where the first step follows from $b_{t, \max} \in \mB_{\eps}$, the third step follows from $v_i \geq b_{t, \max}$ and $$\Gamma_i(b_{t, \max}, s_{-i}(v_{-i})) \geq \Gamma_i(\bs(v_i, s_{-i}), s_{-i}(v_{-i})).$$ 
%\footnote{\color{red}I don't know why we used $v_i\ge b_{t,\max}$? We used that $bs(v_i,s_{-i})$ is in the same box as $b_{t,\max}$ though in the last step.\Binghui{The current writing is correct, I will explain later}}

{\bf Case 2.}  Suppose $v_i \in (\frac{(t-1)\eps}{10}, \frac{t\eps}{10}]$ (note than we cannot have $v_i<(t-1)\eps/10$ due to the no overbidding assumption). In this case bidder $i$ may not be able to bid $b_{t, \max}$ due to the no-overbidding assumption. However, in this case the utility is small anyway so we can let bidder bid
  $b_0=0$ as follows:
\begin{align}
    \E_{v_{-i}\sim \D_{-i}}&\big[u_i(v_i; \bs_{\eps}(v_i, s_{-i}), s_{-i}(v_{-i})) - u_i(v_i; \bs(v_i, s_{-i}), s_{-i}(v_{-i}))\big]\notag \\
    \geq &~ 0 - \E_{v_{-i}\sim \D_{-i}}\big[u_i(v_i; \bs(v_i, s_{-i}), s_{-i}(v_{-i}))\big]\notag \\[0.8ex]
    = &~ \E_{v_{-i}\sim \D_{-i}}\big[(-v_i + \bs(v_i, s_{-i}))\Gamma_i(\bs(v_i, s_{-i}), s_{-i}(v_i))\big] \geq -\eps/10. \label{eq:step1-2}
\end{align}
Combining Eq.~\eqref{eq:step1-1} and Eq.~\eqref{eq:step1-2}, we have proved that
\[
\E_{v_{-i}\sim \D_{-i}}\big[u_i(v_i; s_i(v_i), s_{-i}(v_{-i})) - u_i(v_i; \bs(v_i, s_{-i}), s_{-i}(v_{-i}))\big] \geq \eps_i(v_i) - \frac{\eps}{10}.
\]
Taking an expectation over $v_i$, we complete the proof of the lemma.
%{\color{red}We conclude that $s$ is an $(3\eps/5)$-approximate BNE of the original FPA
%$(\mN, \mB, \D, \Gamma)$. This finishes the proof of the lemma.}
\end{proof}


\vspace{+2mm}
{\noindent \bf Step 2: Rounding distribution. \ \ } 
Given a first-price auction $(\mN, \mB_{\eps}, \D, \Gamma)$, we would like to round the value distribution $\D_i$ such that it is supported over discrete values, and we truncate off the probability mass if it is too small. 
Formally, letting $\delta \in (0, \frac{\eps}{20})$ be a parameter to be specified later\footnote{Looking head, the parameter $\delta$ shall be much smaller than $\eps$}, we define $\D_{i}^{\eps, \delta}$ for each $i\in [n]$ as follows:
\begin{align*}
p_i^{\eps, \delta}(t) = \Pr_{v_i \sim \D_i^{\eps, \delta}}\left[v_i = \frac{t\eps}{10}\right] = 
\max\left\{0,\hspace{0.06cm}\int_{\frac{t\eps}{10}}^{\frac{(t+1)\eps}{10}}  \Pr_{v_i \sim \D_i}[v_i = v]\,\mathsf{d} v - \delta \right\} % \quad  \text{ for any }t \in [10\eps^{-1} - 1]
\end{align*}
for each $t\in [10\eps^{-1}-1]$,
and define
\begin{align*}
p_i^{\eps, \delta}(0) = \Pr_{v_i \sim \D_i^{\eps, \delta}}\big[v_i = 0\big] = 1- \sum_{t=1}^{10\eps^{-1}-1} \Pr_{v_i \sim \D_i^{\eps, \delta}}\left[v_i = \frac{t\eps}{10}\right].
\end{align*}
That is, the valuation distribution is rounded (down) to discrete values $V_{\eps} := \{0,  {\eps}/{10}, \ldots, 1 -  {\eps}/{10} \}$ and truncated at $\delta$; the extra probability mass is put on $0$.
From now on, we would consider both continuous and discrete valuation distribution of bidders.



Let $\D^{\eps, \delta} = \D_{1}^{\eps, \delta} \times \cdots \times \D_{n}^{\eps, \delta}$.
Next we define the notion of 
  $(\eps, \delta)$-approximate BNE --- a weaker notation of equilibrium with relaxed constraint on overbidding,
and show that it suffices to 
find an $(\eps,\delta)$-approximate BNE of the rounded FPA $(\mN, \mB_{\eps}, \D^{\eps, \delta}, \Gamma)$.

\begin{definition}[$(\eps, \delta)$-approximate Bayesian Nash equilibrium]
Let $n, m \geq 2$. Given a Bayesian FPA ($\mN, \mB_\eps, \D^{\eps, \delta},\Gamma$), a strategy profile $s = (s_1, \ldots, s_{n})$ is said to be an $(\eps, \delta)$-approximate Bayesian Nash equilibrium ($(\eps, \delta)$-approximate BNE), if for any player $i\in [n]$, its strategy $s_i$ is monotone and at most $\eps$-worse than the best response:  
%\footnote{\color{red}Should we remove the $\forall b\in \mB$ below?Is the best response below over $\mB$ or over $\mB_\eps$? I think it is over $\mB_\eps$? It might be helpful to define it in terms of a generic game.\Binghui{We can define over general space $\mB$ and $\D$, the only subtle things here is that $v_{i,\max}$ may not exist consider a continuous distribution, but we can properly define it as the limit.}}
\begin{align*}
    \E_{v \sim \D}\big[u_i(v_i; s_i(v_i), s_{-i}(v_{-i}))\big] \geq \E_{v\sim \D}\big[u_i(v_i; \bs(v_i, s_{-i}), s_{-i}(v_{-i}))\big] - \eps, 
\end{align*} 
Moreover, letting $v_{i, \max} = \max_{v_i \in V_{\eps}, p_i(v_{i, \max}) > 0} v_i$, the player $i$ never bids higher than $v_{i, \max}$ and its total overbidding probability is at most $\delta$, i.e.,
\begin{align*}
\Pr_{v_i\sim \calD_i^{\eps,\delta}}\big[s_i(v_i) > v_{i, \max}\big] = 0\quad\text{and}\quad\Pr_{v_i\sim \calD_i^{\eps,\delta}}\big[s_i(v_i) > v_i\big] < \delta.
\end{align*}
\end{definition}


We prove that it suffices to find an $(\eps, \delta)$-approximate BNE in the rounded FPA $(\mN, \mB_{\eps}, \D^{\eps, \delta}, \Gamma)$. 
%Given an $(\eps,\delta)$-approximate BNE
In the proof we will perform the operation of converting a bidding strategy $s_i$ over bidding space $\mB_\eps$ and value distribution $\calD_i$ to a (unique) monotone bidding strategy $s_i'$ over the same bidding space $\mB_\eps$ but a different value distribution $\calD_i'$ such that their induced distributions over $\mB_\eps$ are the same:

%We map an $(\eps, \delta)$-approximate BNE strategy $s$ (defined over $\D^{\eps, \delta}$) to its {\em monotone analogue} on $\D$.
\begin{definition}[Monotone analogue]
Let $s=(s_1,\ldots,s_n)$ be a strategy profile over bidding space $\mB_\eps$ and  value distribution $\D=\D_1\times\cdots\times \D_n$, and let $\D'=\D_1'\times \cdots\times \D_n'$ be another value distribution. The \emph{monotone analogue} of $s$  with respect to  $\D'$ (denoted as $\ms$) is defined as the unique monotone strategy profile $\ms=(\ms_1,\ldots,\ms_n)$ such that
\begin{align*}
\Pr_{v_i\sim \D_i'}\big[\ms_i(v_{i}) = b\big] = \Pr_{v_i\sim \D_i}\big[s_i(v_{i}) = b\big], \quad \text{for all }i\in [n]\ \text{and}\  b\in \mB_\eps.
\end{align*}
\end{definition}

 
\begin{lemma}
\label{lem:ptas-step2}
Given any FPA $(\mN, \mB_{\eps}, \D, \Gamma)$, let $\D^{\eps, \delta}$ be the rounded distribution defined above. Then the monotone analogue of any $(\eps, \delta)$-approximate BNE in $(\mN, \mB_{\eps}, \D^{\eps, \delta}, \Gamma)$ is an $(2\eps + 22\delta/\eps)$-approximate BNE in $(\mN, \mB_{\eps}, \D, \Gamma)$.
\end{lemma}
\begin{proof}
For clarity of the presentation, we divide the proof into two parts. 

{\bf First part. \ } Define the distribution $\D_{i}^{\eps}$ for each $i\in [n]$ as follow:
\begin{align*}
p_i^{\eps}(t) = \Pr_{v_i \sim \D_i}\left[v_i = \frac{t\eps}{10}\right] = 
\int_{\frac{t\eps}{10}}^{\frac{(t+1)\eps}{10}}  \Pr_{v_i \sim \D_i}[v_i = v]\,\mathsf{d} v,  \quad  \text{ for each }t \in [10\eps^{-1} - 1]
\end{align*}
and 
\begin{align*}
p_i^{\eps}(0) = \Pr_{v_i \sim \D_i}\big[v_i = 0\big] = 1- \sum_{t=1}^{10\eps^{-1}-1} \Pr_{v_i \sim \D_i}\left[v_i = \frac{t\eps}{10}\right].
\end{align*}
The distribution $\D^{\eps}_i$ rounds down $\D_i$ to discrete values $V_{\eps}$.
Given any $\eps$-approximate BNE strategy $s$ of FPA $(\mN, \mB_{\eps}, \D^{\eps}, \Gamma)$, we  show that its monotone analogue $\ms$ (with respect to $\D$) is an $\frac{11}{10}\eps$-approximate BNE of FPA $(\mN, \mB_{\eps}, \D, \Gamma)$. For any $v\in [0,1]$, let $v^{\eps} = \lfloor\frac{10v}{\eps}\rfloor\cdot \frac{\eps}{10}$ be the closest while smaller discrete value in $V_{\eps}$.  For any player $i\in [n]$, we have
\begin{align*}
 \E_{v_{-i}\sim \D_{-i}} &\big[u_i(v_i; \ms_i(v_i), \ms_{-i}(v_{-i}))\big] - \E_{v_{-i}\sim \D_{-i}}\big[u_i(v_i; \bs(v_i, \ms_{-i}), \ms_{-i}(v_{-i}))\big] \\
\geq &~\E_{v_{-i}\sim \D_{-i}}\big[u_i(v_i^{\eps}; \ms_i(v_{i}), \ms_{-i}(v_{-i}))\big] - \E_{v_{-i}\sim \D_{-i}}\big[u_i(v_i^{\eps}; \bs(v_i, \ms_{-i}), \ms_{-i}(v_{-i}))\big] - \frac{\eps}{10}\\
= &~\E_{v_{-i}\sim \D_{-i}^{\eps}}\big[u_i(v_i^{\eps}; \ms_i(v_{i}), s_{-i}(v_{-i}))\big] - \E_{v_{-i}\sim \D_{-i}^{\eps}}\big[u_i(v_i^{\eps}; \bs(v_i, s_{-i}), s_{-i}(v_{-i}))\big] - \frac{\eps}{10}\\
\geq &~ \E_{v_{-i}\sim \D_{-i}^{\eps}}\big[u_i(v_i^{\eps}; \ms_i(v_{i}), s_{-i}(v_{-i}))\big] - \E_{v_{-i}\sim \D_{-i}^{\eps}}\big[u_i(v_i^{\eps}; \bs(v_i^{\eps}, s_{-i}), s_{-i}(v_{-i}))\big] - \frac{\eps}{10}.
\end{align*}
The first step holds since 
$u_i(v_i; \ms_i(v_i), \ms_{-i}(v_{-i})) \geq u_i(v_i^{\eps}; \ms_i(v_i), \ms_{-i}(v_{-i}))$ ($v_i \geq v_i^{\eps}$)
and 
\[
u_i(v_i; \bs(v_i, \ms_{-i}), \ms_{-i}(v_{-i})) - u_i(v_i^{\eps}; \bs(v_i, \ms_{-i}), \ms_{-i}(v_{-i})) \leq (v_i - v_i^{\eps}) \leq \frac{\eps}{10}.
\]
The second step holds because the bidding profile $s_{-i}(v_{-i})$ when $v_{-i}\sim \D_{-i}^{\eps}$ is identical to  $\ms_{-i}(v_{-i})$ when $v_{-i}\sim \D_{-i}$.


Taking an expectation over $v_i \sim \D_i$, we obtain
\begin{align*}
 \E_{v\sim \D} &[u_i(v_i; \ms_i(v_i), \ms_{-i}(v_{-i}))] - \E_{v\sim \D}\big[u_i(v_i; \bs(v_i, \ms_{-i}), \ms_{-i}(v_{-i}))\big] \\[0.6ex]
\geq &~\E_{v_i \sim \D_i}\E_{v_{-i}\sim \D_{-i}^{\eps}}\big[u_i(v_i^{\eps}; \ms_i(v_{i}), s_{-i}(v_{-i}))\big] - \E_{v_i \sim \D_i} \E_{v_{-i}\sim \D_{-i}^{\eps}}\big[u_i(v_i^{\eps}; \bs(v_i^{\eps}, s_{-i}), s_{-i}(v_{-i}))\big] - \frac{\eps}{10}\\
= &~ \E_{v_i \sim \D_i^{\eps}}\E_{v_{-i}\sim \D_{-i}^{\eps}}\big[u_i(v_i; \ms_i(v_{i}), s_{-i}(v_{-i}))\big] - \E_{v_i \sim \D_i^{\eps}} \E_{v_{-i}\sim \D_{-i}^{\eps}}\big[u_i(v_i; \bs(v_i, s_{-i}), s_{-i}(v_{-i}))\big] - \frac{\eps}{10}\\
\geq &~ -\eps - \frac{\eps}{10} = -\frac{11}{10}\eps.
\end{align*}
where the second step holds due to $v_{i}^{\eps} (v_i \sim \D_i)$ has identical distribution as $v_i (v_i \sim \D_{i}^{\eps})$.
It is easy to verify that $\ms$ does not overbid given that $s$ does not overbid as an $\eps$-approximate BNE. We conclude that $\ms$ is an $\frac{11}{10}\eps$-approximate BNE of FPA $(\mN, \mB_{\eps}, \D^{\eps}, \Gamma)$.

{\bf Second part. \ } We prove that, given any $(\eps, \delta)$-approximate BNE $s$ of FPA $(\mN, \mB_{\eps}, \D^{\eps, \delta}, \Gamma)$, its monotone analogue $\ms$ with respect to the distribution $\calD^\eps$ must be  an $(\eps + 20\delta/\eps)$-approximate BNE of $(\mN, \mB_{\eps}, \D^{\eps}, \Gamma)$.
%by considering the monotone analogue. Given an $(\eps, \delta)$-approximate BNE strategy $s$ of FPA $(\mN, \mB_{\eps}, \D^{\eps, \delta}, \Gamma)$, let $\ms$ be the monotone analogue with respect to the distribution $\D^{\eps}$.

To prove the correctness, one needs to verify for each player $i\in [n]$,
\begin{enumerate}
    \item $\ms_i$ is monotone,
    \item $\ms_i$ is at most $(\eps + 20\delta/\eps)$-worse than the best response; and
    \item $\ms_i$ does not overbid.
\end{enumerate} 

The first claim of monotonicity follows directly from the definition of $\ms$ as the monotone~analogue of $s$. 
To prove the second claim, a critical observation is that the bidding profile is always preserved, i.e., $s(v)$ ($v\sim \D^{\eps, \delta}$) and $\ms(v)$ ($v\sim \D^{\eps}$) are identical. 
As a consequence, we have
\begin{align}
\E_{v\sim \D^{\eps}}\big[u_i(v_i; \ms_i(v_i), \ms_{-i}(v_{-i}))\big] = &~ \E_{v \sim \D^{\eps}}\big[(v_i - \ms_i(v_i))\Gamma_{i}(\ms_i(v_i), \ms_{-i}(v_{-i}))\big]\notag\\
\geq &~ \E_{v \sim \D^{\eps,\delta}}\big[(v_i - s_i(v_i))\Gamma_{i}(s_i(v_i), s_{-i}(v_{-i}))\big]\notag\\
%\geq &~\E_{v \sim \D^{\eps, \delta}}(v_i - s_i(v_i))\Gamma_{i}(s_i(v_i), s_{-i}(v_{-i}))\notag \\
= &~\E_{v \sim \D^{\eps, \delta}}\big[u_i(v_i; s_i(v_i), s_{-i}(v_{-i}))\big]. \label{eq:step2-1}
\end{align}
The second step holds since (i) the expected payment is the same under $\ms$ and $s$, i.e.,
\begin{align*}
 \E_{v \sim \D^{\eps}}\big[\ms_i(v_i)\Gamma_{i}(\ms_i(v_i), \ms_{-i}(v_{-i}))\big] = \E_{v \sim \D^{\eps,\delta}}\big[s_i(v_i)\Gamma_{i}(s_i(v_i), s_{-i}(v_{-i}))\big] 
\end{align*}
and (ii) the distribution $\D_i^{\eps}$ stochastically dominates the distribution $\D_i^{\eps,\delta}$.
\iffalse
The third step requires some explanation. Let 
\[
\Gamma_i(v_i) := \E_{v_{-i}\sim \D^{\eps, \delta}_{-i}}\Gamma_i(s_i(v_i), s_{-i}(v_{-i})) \quad \text{and} \quad \hat{\Gamma}_{i}(v_i):=\E_{v_{-i}\sim \D^{\eps, \delta}_{-i}}\Gamma_i(\hat{\ms_i}(v_i), \hat{\ms_{-i}}(v_{-i})).
\]
The last step holds since the expected payment is the same under $\hat{\ms}$ and $s$, i.e.,
\begin{align*}
  \E_{v \sim \D^{\eps,\delta}}[\hat{\ms_i}(v_i)\Gamma_{i}(\hat{\ms_i}(v_i), \hat{\ms_{-i}}(v_{-i}))] = \E_{v \sim \D^{\eps,\delta}}[s_i(v_i)\Gamma_{i}(s_i(v_i), s_{-i}(v_{-i}))] 
\end{align*}
and 
\begin{align*}
\E_{v \sim \D^{\eps,\delta}} [v_i \cdot \Gamma_{i}(\hat{\ms_i}(v_i), \hat{\ms_{-i}}(v_{-i}))] = &~ \int_{0}^{1} \nu\cdot p_{i}^{\eps}(\nu) \cdot \hat{\Gamma}_{i}(\nu) \mathsf{d} \nu \\
= &~ \int_{0}^{1}p_{i}^{\eps}(\nu) \cdot \hat{\Gamma}_{i}(\nu) \mathsf{d}\nu - \int_{0}^{1} \int_{0}^{\nu'}p_{i}^{\eps}(\nu') \cdot \hat{\Gamma}_{i}(\nu') \mathsf{d}\nu'\mathsf{d}\nu\\
\geq &~\int_{0}^{1}p_{i}^{\eps}(\nu) \cdot \Gamma_{i}(\nu) \mathsf{d}\nu - \int_{0}^{1} \int_{0}^{\nu'}p_{i}^{\eps}(\nu') \cdot \Gamma_{i}(\nu') \mathsf{d}\nu'\mathsf{d}\nu\\
=&~ \E_{v \sim \D^{\eps,\delta}} [v_i \cdot \Gamma_{i}(s_i(v_i), s_i(v_{-i}))].
\end{align*}
We expand the definition of expectation in the first step and we integral by parts in the second step. The third step follows from the property of monotone analogue and the monotonicity of $\Gamma_{i}(\cdot)$ and $\hat{\Gamma}_{i}(\cdot)$.
\fi

Similarly, we have
\begin{align}
\E_{v\sim \D^{\eps}}\big[u_i(v_i; \bs(v_i, \ms_{-i}), \ms_{-i}(v_{-i}))\big] = &~ \E_{v \sim \D^{\eps}}\big[(v_i - \bs(v_i, \ms_{-i}))\Gamma_{i}(\bs(v_i, \ms_{-i}), \ms_{-i}(v_{-i}))\big]\notag\\
\leq &~ \E_{v \sim \D^{\eps, \delta}}\big[(v_i - \bs(v_i, s_{-i}))\Gamma_{i}(\bs(v_i, s_{-i}), s_{-i}(v_{-i}))\big] + 20\delta/\eps \notag \\
= &~ \E_{v\sim \D^{\eps, \delta}}\big[u_i(v_i; \bs(v_i, s_{-i}), s_{-i}(v_{-i}))\big]+ 20\delta/\eps \label{eq:step2-2}.
\end{align}
where the second step holds since the TV distance between $\D^{\eps}$ and $\D^{\eps, \delta}$ is bounded by $20\delta/\eps$ and $\ms_{-i}(v_{-i})$ ($v_{-i}\sim \D_{-i}^{\eps}$) has the same distribution as $s_{-i}(v_{-i})$ ($v_{-i}\sim \D_{-i}^{\eps, \delta}$).
Combining Eq.~\eqref{eq:step2-1} and Eq.~\eqref{eq:step2-2}, we conclude the $\ms$ is at most $(\eps +20\delta/\eps)$-worse than the best response.

Finally we prove the last item that $\ms_i$ does not overbid. 
We prove by contradiction and suppose that at some bid $b \in \mB_{\eps}$, %\footnote{\color{red}$\mB_\eps$?} 
we have $\Pr_{v_i \sim \D^{\eps}_i}[\ms_i(v_i) = b > v_i] > 0$. Let $\nu_i \in [0, 1]$ be the smallest value such that $\Pr[\ms_i(\nu_i) = b] > 0$, and let  $v_{i, \max}^{\eps, \delta} = \max_{v_i \in V_{\eps}, p_i^{\eps, \delta}(v_{i, \max}^{\eps, \delta}) > 0} v_i$.
We note that $\nu_i < b \leq v_{i, \max}^{\eps,\delta}$ due to the definition of $(\eps, \delta)$-approximate BNE, i.e., there is no overbidding over $v_{i,\max}^{\eps,\delta}$. Hence, we have
\begin{align}
\Pr_{v_i \sim \D^{\eps, \delta}}\big[s_i(v_i) \geq b\big] = &~ \Pr_{v_i \sim \D^{\eps}}\big[\ms_i(v_i) \geq b\big] \geq \Pr_{v_i \sim \D^{\eps}}\big[v_i \geq \nu_{i}\big] \geq \Pr_{v_i \sim \D^{\eps, \delta}}\big[v_i \geq \nu_{i}\big] + \delta\notag\\
\geq &~ \Pr_{v_i \sim \D^{\eps, \delta}}\big[v_i \geq b\big] + \delta \label{eq:step2-3}
\end{align}
The first step follows from the definition of monotone analogue, the second step
%\footnote{\color{red}Is this an equation?\Binghui{Not exactly, since there are point mass..}} 
holds since $\ms_i$ is monotone. The third step holds since $\nu_i < v_{i, \max}^{\eps, \delta}$ and $\D^{\eps,\delta}$ truncates at least $\delta$ probability mass at $v_{i, \max}^{\eps, \delta}$.
Eq.~\eqref{eq:step2-3} implies 
\[
\Pr_{v_i \sim \D^{\eps, \delta}}[v_i < b \wedge s_i(v_i) \geq b] \geq\delta,
\]
i.e., the overbidding probability is at least $\delta$. This contradicts with the fact that $s$ is an $(\eps, \delta)$-approximate BNE of FPA $(\mN, \mB_{\eps}, \D^{\eps}, \Gamma)$.
%\footnote{\color{red}The statement is correct but the reasoning needs some work.} 
We complete the proof here.
\end{proof}







\vspace{+2mm}
{\noindent \bf Step 3: Existence of discretized $(\eps,\delta)$-approximate BNE. \ \ }
Given a first-price auction $(\mN, \mB, \D, \Gamma)$, we prove the existence of a (suitably) discretized $(\eps, \delta)$-approximate BNE.
We describe this step using a generic FPA $(\mN, \mB, \D, \Gamma)$ but will apply it on the rounded FPA $(\mN, \mB_{\eps}, \D^{\eps, \delta}, \Gamma)$ later.
For any $j \in [0:m]$, $i \in [n]$, let $p_{i, j} = \Pr[s_i(v_i) = b_j]$ be the probability of player $i$ bidding $b_j$ in a strategy profile $s$.

\begin{lemma}[Discretization]
\label{lem:ptas-step3} Let $m, n \geq 2$, given any first-price auction $(\mN, \mB, \D, \Gamma)$ and $\mB = \{b_0, b_1, \ldots, b_m\}$ with $0 = b_0 < b_1 < \cdots < b_m \leq 1$, 
there exists an $(\eps, \delta)$-approximate BNE strategy profile $s$ such that $p_{i, j}$ is a integer multiple of $$\ell(m)\cdot \frac{1}{ \eps^{6}\delta  }$$ for any $i \in [n]$ and $ j \in [0:m]$, where $\ell(m)$ is some exponential function of $m$.
\end{lemma}


We make use of the following result from \cite{daskalakis2015approximate}. We note this the major part that we need a uniform tie-breaking rule.
\begin{theorem}[Theorem 3 of \cite{daskalakis2015approximate}]
\label{thm:dis}
Let $p_i \in \Delta_{m+1}$ for $i\in [n]$, and let $\{X_i \in \R^{m+1}\}_{i\in[n]}$ be a set of independent $(m+1)$-dimensional random unit vectors such that, for all $i \in [n]$, $j \in [0:m]$, $\Pr[X_i = \mathsf{1}_{j}] = p_{i,j}$.  Let $z > 0$ be an integer. Then there exists another set of probability vectors $\{\hat{p}_{i}\in \Delta_{m+1}\}_{i \in [n]}$ such that the following conditions hold:
\begin{itemize}
\item $|\hat{p}_{i, j} - p_{i, j}| = O( {1}/{z})$, for all $i \in [n]$ and $ j \in [0:m]$;
\item $\hat{p}_{i,j}$ is an integer multiple of $\frac{1}{2^{m}} \cdot \frac{1}{z}$ for all $i \in [n]$ and $ j \in [0:m]$;
\item If $p_{i, j} = 0$ then $\hat{p}_{i, j} = 0$;
\item Let $\{\hat{X}_i \in \R^{m+1}\}_{i\in[n]}$ be a set of independent $(m+1)$-dimensional random unit vectors\\ such that $\Pr[X_i = \mathsf{1}_{j}] = \hat{p}_{i,j}$ for all $i \in [n]$, $j \in [0:m]$.Then
\begin{align}
\left\|\sum_{i\in [n]}X_i - \sum_{i\in [n]}\hat{X}_i\right\|_{\mathsf{TV}} = O\left(h(m)\cdot \frac{\log z}{z^{1/5}}\right). \label{eq:pro1}
\end{align}
Moreover, for all $i' \in [n]$, we have 
\begin{align}
\left\|\sum_{i \in [n]\setminus \{i'\}}X_{i} - \sum_{i \in [n]\setminus \{i'\}}\hat{X}_{i}\right\|_{\mathsf{TV}} = O\left(h(m)\cdot \frac{\log z}{z^{1/5}}\right). \label{eq:pro2}
\end{align}
where $h(m)$ is some exponential function
%\footnote{\color{red}I don't know what exactly do you (or [DP15]) mean by exponential functions?} of $m$.
\end{itemize}
\end{theorem}



\begin{proof}[Proof of Lemma \ref{lem:ptas-step3}]
Given a BNE strategy profile $s$ of FPA $(\mN, \mB, \D, \Gamma)$, we take 
\[
z = \Omega\Big(\max\big\{2h(m)^6 \eps^{-6}, 2m^2\delta^{-1}\big\}\Big) \quad \text{and} \quad \gamma =  h(m)\cdot \frac{\log z}{z^{1/5}}.
\]
Let $\{p_{i} \in \Delta^{m+1}\}_{i \in [n]}$ be the probability vectors that correspond to $s$, i.e., $p_{i,j}=\Pr[s_i(v_i)=b_j]$.
Using Theorem \ref{thm:dis}, let $\{\hat{p}_{i} \in \Delta^{m+1}\}_{i \in [n]}$ be the set of discretized probability vectors, and let $\hat{s}$ be the unique monotone strategy determined by $\hat{p}$ (with respect to the same value distribution $\calD$).
We prove that $\hat{s}$ forms an $(\eps, \delta)$-approximate BNE of the auction. We need to verify that for every player $i\in [n]$, (1) $\hat{s}_{i}$ is at most $\eps$-worse than the best response; and (2) the overbidding probability is small.

By Eq.~\eqref{eq:pro2}, for any $j \in [0:m]$, one has 
\begin{align}
\E_{v_{-i}\sim \D_{-i}}\big[\Gamma_i(b_j, s_{-i}(v_{-i})) \big]= \E_{v_{-i}\sim \D_{-i}}\big[\Gamma_i(b_j, \hat{s}_{-i}(v_{-i}))\big] \pm \gamma.\label{eq:step3-1}
\end{align}
since the allocation probability (under the uniform tie-breaking) is determined by the histogram of other players' bidding histogram, which shifts by at most $\gamma$ between $s_{-i}$ and $\hat{s}_{-i}$ in total variance distance. 

For the utility, we have
\begin{align}
\E_{v \sim \D} \big[u_{i}(v_i; \hat{s}_i(v_i), \hat{s}_{-i}(v_{-i}))\big] = &~ \E_{v \sim \D} \big[(v_{i} - \hat{s}_i(v_i)) \cdot \Gamma_i(\hat{s}(v_i), \hat{s}_{-i}(v_{-i}))\big]\notag\\
= &~ \E_{v \sim \D} \big[(v_{i} - \hat{s}_i(v_i)) \cdot \Gamma_i(\hat{s}_i(v_i), s_{-i}(v_i))\big] \pm \gamma \notag\\
= &~ \E_{v \sim \D} \big[(v_{i} - s_i(v_i)) \cdot \Gamma_i(s_i(v_i), s_{-i}(v_i))\big] \pm O\left(m^2\cdot \frac{1}{z}+\gamma\right) \notag \\
= &~ \E_{v \sim \D} \big[u_{i}(v_i; s(v_i), s_{-i}(v_{-i}))\big] \pm \frac{\eps}{2}.\label{eq:step3-2}
\end{align}
The second step follows from Eq.~\eqref{eq:step3-1}, the third step holds since the TV distance between $(v_i, s_i(v_i))$ and $(v_i, \hat{s}_i(v_i))$ is at most $O(m^2 \cdot \frac{1}{z})$.
The last step follows from the choice of $z$.

At the same time, we have
\begin{align}
\E_{v \sim \D} \big[u_{i}(v_i; \bs(v_i, \hat{s}_{-i}), \hat{s}_{-i}(v_{-i}))\big] \leq &~ \E_{v \sim \D} \big[u_{i}(v_i, \bs(v_i, \hat{s}_{-i}), s_{-i}(v_{-i}))\big] +\gamma \notag \\
\leq &~ \E_{v \sim \D} \big[u_{i}(v_i, \bs(v_i, s_{-i}), s_{-i}(v_{-i}))\big] +\gamma \notag \\
\leq &~ \E_{v \sim \D} \big[u_{i}(v_i; \bs(v_i, s_{-i}), s_{-i}(v_{-i})) \big]+\frac{\eps}{2}. \label{eq:step3-3}
\end{align}
Here the first step follows from the bidding histogram shifts by at most $\gamma$ between $s_{-i}$ and $\hat{s}_{-i}$ in total variance distance. The last step follows from the choice of parameters.
%\footnote{\color{red}The last inequality is a bit confusing; it should be $\le E[] + \gamma$ and then continue with $\gamma\le \eps/2$, right? If so maybe split it into two inequalities.}

Combining Eq.~\eqref{eq:step3-2} and Eq.~\eqref{eq:step3-3}, we have that $\hat{s}_i$ is at most $\eps$-worse than the best response.

To bound the probability of overbidding, let $j(i) \in [0:m]$ be the maximum bid that receives non-zero probability under $\hat{s}_i$, then it is easy to verify that  $b_{j(i)} \leq v_{i,\max}$: otherwise $b_{j(i)}>v_{i,\max}$ and $p_{i, j(i)}>0$ would contradict with the no overbidding assumption of BNE on $s$.  

Finally we have
\begin{align*}
\Pr_{v_i\sim\calD_i}\big[\hat{s}_i(v_i) > v_i\big] \leq \Pr_{v_i\sim\calD_i}\big[ {s}_i(v_i) > v_i\big] + O\left(m^2 \cdot \frac{1}{z}\right) = 0 + O\left(m^2 \cdot \frac{1}{z}\right) \leq \delta
\end{align*}
since the total variation distance between $(v_i, s_i(v_i))$ and $(v_i, \hat{s}_i(v_i))$ is at most $O(m^2 \cdot \frac{1}{z})$.
\end{proof}


%While it might be impossible to prove a similar dimension-independent bound (like Theorem \ref{thm:dis}) for the general tie-breaking rule, we have
%\begin{lemma}
%Given a first price auction $(\mN, \mB, \D, \Gamma)$ and a monotone strategy $s$. Let $\{p_{i} \in \Delta^{m+1}\}_{i \in [n]}$ be the set of probability vector associated with $s$. Let $z = O(\eps^{-1}\log(mn))$ be a integer. There exists another set of probability vector $\{\hat{p}_{i} \in \Delta^{m+1}\}_{i \in [n]}$ and the corresponding monotone strategy $\hat{s}$, such that
%\begin{itemize}
%\item $|\hat{p}_{i, j} - p_{i, j}| = O(\frac{1}{z})$, for all $i \in [n], j \in [0:m]$;
%\item $p_{i,j}$ is an integer multiple of $\frac{1}{z}$ for all $i \in [n], j \in [0:m]$;
%\item If $p_{i, j} = 0$ then $\hat{p}_{i, j} = 0$;
%\item $\Gamma_{i}(b_j, \hat{s}_{-i}) = \Gamma_{i}(b_j, s_{-i}) \pm \frac{\eps}{4}$ for all $i \in [n], j \in [0:m]$;
%\end{itemize}
%\end{lemma}
%\begin{proof}
%\Binghui{TBD}
%This is a simple application of the probabilistic method. Given the vector $\{p_{i} \in \Delta^{m+1}\}_{i \in [n]}$, we round it to the grid point.
%t%he random variable $p_{i, j}$
%\end{proof}



%\begin{lemma}[Discretization for general tie-breaking]
%\end{lemma}






\vspace{+2mm}
{\noindent \bf Step 4: Searching for an $(\eps,\delta)$-approximate BNE. \ \ } 
Finally we provide a simple searching algorithm for $(\eps,\delta)$-approximate BNE over the discretized space. The key observation comes from the allocation rule of a first-price auction, i.e., only players with the highest bid could win the item. 
We say a strategy profile $s$ lies in the grid $S_{\omega}$ for some $\omega \in (0,1)$ if $p_{i, j} = \Pr_{v_i \sim \D_i}[s_i(v_i) = b_j]$ is a multiple of $\omega$ for every $i \in [n]$ and $j \in [0:m]$.

\begin{lemma}
\label{lem:ptas-step4}
Given a first-price auction $(\mN, \mB, \D, \Gamma)$, and suppose there exists at least one $(\eps, \delta)$-approximate BNE over the grid $S_{\omega}$, then 
there is an algorithm that runs in $n^4m \cdot 2^{\tilde{O}(m/\eps\omega)}$ time and returns an $(2\eps, \delta)$-approximate BNE under the uniform tie-breaking rule.
%\item There is an algorithm that runs in $n^{2^{O(\eps/\omega)}}$ time and returns an $(3\eps, \delta)$-approximate BNE under the ultimately uniform tie-breaking rule.\Binghui{TBD}
%\end{itemize}
\end{lemma}
\begin{proof}
Let $R = 100(m+1)/\eps\omega$. The discretized strategy profiles $\sE_{\omega} \subseteq [\Delta_{m+1}]^{R}$ are defined as follows. A strategy profile $(p_1, \ldots, p_{R}) \in \sE_{\omega}$ is parameterized by a bid level $j^{*} \in [0:m-1]$ and  $k_0,k_1, \ldots, k_{j^{*}} \in [0:10\eps^{-1} ]$ such that $p_{i, j}$ is a multiple of $\omega$ for all $i \in [n], j \in [0:m]$, and
\begin{align*}
\sum_{r=1}^{R} p_{r, m - j} = k_j,\, \forall j \in [0: j^{*}] \quad \text{and} \quad p_{r, m - j} = 0, \, \forall r\in [R],j \in [j^{*}+ 1: m-1].
\end{align*}


The size of $\sE_{\omega}$ satisfies
\[
|\sE_{\omega}| \leq (m+1) \cdot (10/\eps\omega + 1/\omega)^{m+1} \cdot \binom{R}{10/\eps\omega + 1/\omega} \leq 2^{\tilde{O}(m/\eps\omega)}.
\]

For any strategy profile $(p_1, \ldots, p_R) \in \sE_{\omega}$, one can augment with $(n-R)$ default strategies $(1,0, \ldots, 0)$ (that is, bidding $b_0=0$ with probability $1$). 
Slightly abuse of notation, we also use $\sE_{\omega} \subseteq [\Delta_{m+1}]^{n}$ to denote the augmented strategy profiles. We shall prove 
\begin{itemize}
\item There is an $(2\eps, \delta)$-approximate BNE in $\sE_{\omega}$ (up to a matching with players), and
\item One can identify the matching efficiently.
\end{itemize}

\paragraph{Existence of $(2\eps, \delta)$-approximate BNE} Recall that there exists an $(\eps, \delta)$-approximate BNE strategy $s$ over the grid $S_{\omega}$. Let $j^{*}$ be the first index over $[0:m]$ such that $\sum_{i\in [n]} p_{i,m-j} \leq 10/\eps$ ($\forall j < j^{*}$) and $\sum_{i\in [n]} p_{i,m- j^{*}} \geq 10/\eps$.
If $j^{*} = m$, then we have $s \in \sE_{\omega}$ (up to a matching between players).
If $j^{*} \leq m-1$, then let $n^{*} \in [n]$ be the smallest player such that $\sum_{i\in [n^{*}]} p_{n,m-j^{*}} \geq 10/\eps$ (w.l.o.g. we assume it takes the equality). 
Truncate the strategy profile to $s'$ such that
\begin{align*}
p'_{i, j} = \Pr_{v_i\sim \D_i}[s_i'(v_i) = b_j] = \left\{
\begin{matrix}
p_{i, j} & j > m - j^{*} \vee (j = m - j^{*} \wedge i \leq n^{*})\\
0 & j \in [1: m - j^{*} - 1] \vee (j = m - j^{*} \wedge i > n^{*})\\
1 - \sum_{j' =1}^{m}p'_{i, j'} & j = 0.
\end{matrix}
\right.
\end{align*}

It is clear the new strategy $s' \in \sE_{\omega}$ (up to a matching between players), and for each player $i\in [n]$, the new strategy $s_i'$ is monotone and the probability of overbidding is no more than $\delta$ (because we only move bidding probability to $b_0 = 0$). It suffices to prove it is at most $2\eps$-worse than the best response. 
The key observation is that the allocation probability of bidding $b_{j}$ with $j > m- j^{*}$ remains the same, i.e., 
\begin{align*}
\E_{v_{-i}\sim \D_{-i}}[\Gamma_{i}(b_j, s_{-i}'(v_{-i}))] = \E_{v_{-i}\sim \D_{-i}} [\Gamma_{i}(b_j, s_{-i}(v_{-i}))] \quad \forall j \in [m- j^{*} + 1: m]
\end{align*}
and moreover, the allocation probability of bidding no more than $b_{m-j^{*}}$ is small
\begin{align*}
\E_{v_{-i}\sim \D_{-i}}[\Gamma_{i}(b_j; s_{-i}'(v_{-i}))] \leq \eps/4 \quad \forall j \in [0: m - j^{*}].
\end{align*}
This holds since with probability at least $1-\exp(-5/3\eps)$, there are at least $5/\eps$ players bid no less than $b_{m-j^{*}}$ by Chernoff bound.

Therefore, at any value point $v_i$, if $s_i'(v_i) > b_{m- j^{*}}$, then $s_i(v_i) = s_i'(v_i)$ and
\begin{align}
\E_{v_{-i}\sim \D_{-i}}[u_{i}(v_i; s_i'(v_i); s_{-i}'(v_{-i}))] = \E_{v_{-i}\sim \D_{-i}}[u_{i}(v_i; s_i(v_i); s_{-i}(v_{-i}))] \label{eq:step4-1}
\end{align}
If $s_i'(v_i) \leq b_{m- j^{*}}$, then $s_i(v_i) \leq b_{m- j^{*}}$, and 
\begin{align}
\E_{v_{-i}\sim \D_{-i}}[u_{i}(v_i; s_i'(v_i); s_{-i}'(v_{-i}))] \geq -\eps/4 \quad \text{and}\quad \E_{v_{-i}\sim \D_{-i}}[u_{i}(v_i; s_i(v_i); s_{-i}(v_{-i}))] \leq \eps/4. \label{eq:step4-2}
\end{align}
Combining Eq.~\eqref{eq:step4-1} and Eq.~\eqref{eq:step4-2}, for any $v_i \in [0, 1]$, one has
\begin{align}
\E_{v_{-i}\sim \D_{-i}}[u_{i}(v_i; s_i'(v_i); s_{-i}'(v_{-i})) - u_{i}(v_i; s_i(v_i); s_{-i}(v_{-i}))] \geq -\eps/2. \label{eq:step4-4}
\end{align}
Similarly, one can prove 
\begin{align}
\E_{v_{-i}\sim \D_{-i}}[u_i(v_i;\bs(v_i, s_{-i}'), s_{-i}'(v_{-i}))] \leq \E_{v_{-i}\sim \D_{-i}}[u_i(v_i;\bs(v_i, s_{-i}), s_{-i}(v_{-i}))]  + \eps/2. \label{eq:step4-3}
\end{align}
Combining Eq.~\eqref{eq:step4-4} \eqref{eq:step4-3}, we have proved $s_i'$ is at most $2\eps$-worse than the best response.


\paragraph{Find a matching}Given a strategy profile $(p_1, \ldots, p_n)\in \sE_{\omega}$, we show how to define a bipartite matching problem, such that the $(2\eps, \delta)$-approximate BNE are one-to-one correspondence to the perfect bipartite matching of the graph.
%Under the uniform tie-breaking rule, this can be done via finding a perfect matching over a bipartite graph.
The bipartite matching problem is defined between players $[n]$ and the strategy $\{p_i\}_{i \in [n]}$. 
We draw an edge between player $i_1$ and the strategy $p_{i_2}$, if $p_{i_2}$ is at most $2\eps$-worse than the best response (note the histogram of other players' bidding profile is determined) and the overbidding probability is small at most $\delta$. 
We note the best response can be computed in $O(mn^2)$ time and one can estimate the utility of a bid in $O(n^2)$ time using dynamic programming (similar as \cite{filos2021complexity}). Hence, it takes $n^2 \cdot O(mn^2)$ time to construct the bipartite graph, and a perfect matching can be found in time $O(n^3)$.
%\Binghui{A few words about general tie-breaking rule}
\end{proof}


Combining the above four steps (Lemma~\ref{lem:ptas-step1}, Lemma~\ref{lem:ptas-step2}, Lemma~\ref{lem:ptas-step3} and Lemma~\ref{lem:ptas-step4}), we conclude the proof of Theorem \ref{thm:ptas}.

%\Binghui{Add conclusion remark by combining the above Lemmas.}