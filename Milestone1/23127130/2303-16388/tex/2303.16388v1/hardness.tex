\section{PPAD-hardness}
\label{sec:ppad}

%\Binghui{We need to switch to the new definition, but it should be easy given they are almost equivalent under polynomial approximation.}
%{\color{red}We need to write down the definition and show that they equivalent.}

Recall our main hardness result

\Hardness*


In the rest of section, we construct the hard instances of FPA in Section \ref{sec:construction} and provide some basic properties in Section \ref{sec:basic}. 
We reduce from the $\eps$-generalized-circuit problem in Section \ref{sec:reduce}.



\subsection{Construction of first price auctions}
\label{sec:construction}

It suffices to prove finding $\eps$-BNE is hard for some $\eps = 1/\poly(n)$ due to Lemma \ref{lem:notion}.
We will use the following three parameters in the construction:
\begin{align*}
\eps = \frac{1}{n^{40}}, \quad \delta = \frac{1}{n^{10}} \quad \text{and} \quad \beta = \frac{1}{n^4}.
\end{align*}
%where $\eps$ is the precision of BNE and one can boost it to $\frac{1}{n}$ via a padding argument. 



We describe the bidding space $\mB$, the valuation distribution $\D$ and the tie-breaking rule $\Gamma$.

\paragraph{Bidding space.} The bidding space $\mB = \{b_0,b_1, b_2\}$ contains $3$ bids in total, where $b_0 = 0$, 
$$b_1 = \frac{\delta^2}{n^4}\quad\text{and}\quad b_2 = \frac{\delta}{n^2}.$$
\paragraph{Valuation distribution.} 
There are $ n+1$ players --- $n$ standard players indexed by $[n]$ and one pivot player $n+1$. We will describe  the value distribution $\D_i$ of player $i$ by specifying its density function $p_i: [0,1]\rightarrow \R^{+}$. 
The density function $p_{n+1}$ of the pivot player is set as follows:
\begin{align*}
    p_{n+1}(v) = 
    \begin{cases}
    1/(2\eps) & v \in [0,\eps]\\
    1/(2\eps) & v \in [1-\eps, 1]
    \end{cases}
   .
\end{align*}
In another word, $\D_{n+1}$ has $0.5$ probability mass around $0$ and $0.5$ probability mass around $1$. 

The density function $p_i$ of each standard player $i \in [n]$ is set as follows:
\begin{align*}
    p_i(v) =  
    \begin{cases}
    (1 - (2 + \beta)\delta)/\eps & v \in [0,\eps]\\
    \delta /\eps & v \in [b_2 - \eps, b_2]\\
    \tilde{p}_i(v) & v \in (b_2, 1 - \eps)\\
    \delta/\eps & v \in [1- \eps, 1] 
    \end{cases}
\end{align*} 
where $\tilde{p_i}(v)$ is defined over $(b_2,1-\eps)$, satisfies $\int_{b_2}^{1-\eps}\tilde{p}_i(v) \mathsf{d} v = \beta \delta$, but will be specified later in the reduction in Section \ref{sec:reduce}.
In short, a standard player $i$ has most its probability mass around $0$, $\delta$ mass around $b_2$, $\delta$ mass around $1$ and $\beta\delta$ mass in $(b_{2}, 1-\eps)$
  to be specified later. 






\paragraph{Tie-breaking rule.}
We describe the trilateral tie-breaking rule $\Gamma$ as follows. For any bidding profile $\beta$ with $2\le |W(\beta)|\le 3$, the tie-breaking rule depends on the presence of $n+1$ in $W(\beta)$:
\begin{itemize}
\item Suppose $n+1 \notin W(\beta)$. Then
\begin{flushleft}\begin{itemize}
\item If $|W(\beta)| = 2$, i.e., $W(b) = \{i, j\}$, the tie-breaking rule is given by a matrix $\Sigma^{A} \in [0,1]^{n\times n}$ such that player $i$ obtains $\Sigma^{A}_{i, j}$ unit of the item and player $j$ obtains 
$\Sigma^{A}_{j,i}$ unit. 
So the matrix $\Sigma^A$ needs to satisfy $\Sigma^A + (\Sigma^A)^{\top} = (J_n - I_n)$. 
We will specify $\Sigma^A$ in the reduction later but will  guarantee that all of its off-diagonal entries lie in $[1/4,3/4]$.
\item If $|W(\beta)| = 3$, then we use the uniform allocation.
\end{itemize}\end{flushleft}
\item Suppose  $n+1 \in W(\beta)$. Then
\begin{flushleft}\begin{itemize}
\item If $|W(\beta)| = 2$, then the item is fully allocated to the pivot player $n+1$.
\item If $|W(\beta)| = 3$, i.e., $W(b) = \{i, j, n+1\}$, then the tie breaking is given by a matrix $\Sigma_B \in [0,1]^{n\times n}$  such that player $i$ obtains $\Sigma^{B}_{ i, j}$ unit of the item, player $j$ obtains $\Sigma^B_{j,i}$ unit and player $n+1$ obtains $1-\Sigma^B_{i,j}-\Sigma^B_{j,i}$ unit. So the matrix $\Sigma^B$ needs to satisfy $\Sigma^B + (\Sigma^B)^{\top} \leq J_n - I_n$, i.e., $\Sigma^B + (\Sigma^B)^{\top}$ is entrywise dominated by $J_n - I_n$.
\end{itemize}\end{flushleft}
\end{itemize}




\subsection{Basic properties}\label{sec:basic}

Let $s=(s_1,\ldots,s_{n+1})$ be an $\eps$-BNE of the instance.
We prove a few properties of $s$ in this subsection.
Given  $s$, for each player $i$ we define $f_i: \mB\rightarrow [0,1]$ and $F_i: \mB\rightarrow [0,1]$ as follows:
\[
f_i(b ) = \Pr_{v_i\sim \D_i}[s_i(v_i) = b ] \quad \text{and} \quad F_i(b ) = \Pr_{v_i\sim \D_i}[s_i(v_i) \leq b ].
\]
In the rest part of section, we abbreviate 
\[
\Gamma_i(b, s_{-i}) := \E_{v_{-i}\sim \D_{-i}}[\Gamma_i(b, s_{-i}(v_{-i}))] \quad \text{and}\quad u_i(v_i; b, s_{-i}) := \E_{v_{-i}\sim \D_{-i}}[u_i(v_i; b; s_{-i}(v_{-i}))] 
\]
when there is no confusion.

We start with the following lemma.
\begin{lemma}[Separable bid]
\label{lem:separable}
In any $\eps$-BNE, the equilibrium strategy $s$ satisfies
\begin{itemize}
\item For a standard player $i \in [n]$, its equilibrium strategy satisfies 
\begin{itemize}
\item when $v_i \in [0, \eps]$, $s_i(v_i) = b_0$; 
\item when $v_i \in [b_2 - \eps, b_2]$, $s_i(v_i) = b_1$; and 
\item when $v_i \in [1-\eps, 1]$, $s_i(v_i)= b_2$. 
\end{itemize}
\item For the pivot player, its equilibrium strategy satisfies 
\begin{itemize}
\item when $v_{n+1}\in [0, \eps]$, $s_{n+1}(v_{n+1}) = b_0$; and 
\item when $v_{n+1}\in [1-\eps, 1]$, $s_{n+1}(v_{n+1}) = b_2$.
\end{itemize}
\end{itemize}
\end{lemma}
\begin{proof}
The claim of $s_i(\eps) = 0$ holds trivially for all $i \in [n+1]$ due to the no-overbidding assumption.
A standard player $i$ chooses between $b_0$ and $b_1$ for $v_i \in [b_2 - \eps, b_2]$.
The allocation probability $\Gamma_i(b_0, s_{-i})$ of bidding $b_0$ satisfies $\Gamma_i(b_0, s_{-i}) \leq \frac{1}{n+1}$, hence the utility of bidding $b_0 = 0$ is at most 
\[
u_i(v_i; b_0, s_{-i}) = (v_i - b_0) \cdot \Gamma_i(b_0, s_{-i}) \leq \frac{1}{n} b_2.
\]
The allocation probability of bidding $b_1$ is at least 
\[
\Gamma_i(b_1, s_{-i}) \geq \prod_{i\in [n+1]}f_i(b_0) \geq (1 - (2+\beta)\delta)^{n} \cdot \frac{1}{2} \geq \frac{1}{3}
\] 
hence the utility of bidding $b_1$ is at least  
\[
u_i(v_i; b_1, s_{-i}) = (v_i - b_1) \cdot \Gamma_i(b_1, s_{-i}) \geq  (b_2 -\eps - b_1) \cdot \frac{1}{3} > u_i(v; b_0, s_{-i})  + \eps.
\]

Finally, we analyse the equilibrium strategy around $v \in [1-\eps, 1]$ for all $n+1$ players. Via an analysis similar to the above argument, it is clear that both standard players and the pivot player would choose between $b_1$ and $b_2$.
For a standard player $i \in [n]$,  
%given that $s_{i}(1 - \eps) \geq s_i(b_2) = b_{1}$
%  due to the monotonicity assumption, we have that player $i$ would choose between $b_{1}$ and $b_{2}$ on $v\in [1-\eps,1]$. 
the allocation probability of bidding $b_{2}$ satisfies
\begin{align}
    \Gamma_i(b_{2}, s_{-i}) \geq &~ F_{n+1}(b_1) \cdot \prod_{j \in [n]\setminus [i]}F_{j}(b_1) \notag\\
    \geq &~ F_{n+1}(b_1) \cdot \left(\prod_{j \in [n]\setminus [i]}f_{j}(b_0) + \sum_{j \in [n]\setminus [i]}f_j(b_1) \prod_{r\in [n]\setminus \{i, j\}}f_r(b_0) \right)\notag \\
    \geq &~ \frac{1}{2}  \left(\prod_{j \in [n]\setminus [i]}f_j(b_0) + \sum_{j \in [n]\setminus [i]}f_j(b_1) \prod_{r\in [n]\setminus \{i, j\}}f_r(b_0) \right) + \frac{1}{2}f_{n+1}(b_1) ,\label{eq:separable-4}
\end{align}
where the last step holds as $f_{n+1}(b_0) = \frac{1}{2}$ and
\begin{align}
\prod_{j \in [n]\setminus [i]}f_j(b_0)  \geq (1 -(2+\beta)\delta)^{n-1} \geq \frac{1}{2} . \label{eq:separable-5}
\end{align}
The allocation probability of bidding $b_1$ satisfies
\begin{align}
    \Gamma_i(b_{1}, s_{-i}) \leq &~  f_{n+1}(b_0) \cdot \left(\prod_{j \in [n]\setminus [i]}f_j(b_0) + \sum_{j \in [n]\setminus [i]}f_j(b_1) \cdot \Sigma^A_{i, j} \prod_{r\in [n]\setminus \{i, j\}}f_r(b_0) + 9n^2\delta^2\right) \notag\\
    +&~  f_{n+1}(b_1) \cdot \sum_{j \in [n]\setminus\{i\}}f_j(b_1) \notag \\
     \le &~ \frac{1}{2} \left(\prod_{j \in [n]\setminus [i]}f_j(b_0) + \sum_{j \in [n]\setminus [i]}f_j(b_1) \cdot \Sigma^{A}_{i, j} \prod_{r\in [n]\setminus \{i, j\}}f_r(b_0)\right) + f_{n+1}(b_1)\cdot  3n\delta + 9n^2\delta^2\label{eq:separable-3}.
\end{align}
Here the first step holds since (1) when the pivot player bids $b_0$, player $i$ obtains $\Sigma^{A}_{i, j}$ unit of item when (only) player $j$ bids $b_1$, and the probability of at least two players bidding $b_1$ is bounded as
\begin{align*}
\sum_{j, r \in [n]\backslash \{i\}} \Pr[s_r(v_r) = b_1 \wedge s_j(v_j) = b_1] \leq n^2\cdot  9\delta^2,
\end{align*}
(2) when the pivot player bids $b_1$, the player $i$ obtains the item only if there exists at least one other standard player $j$ bids $b_1$ as the tie breaking rule assigns the item fully to player $n+1$ when there are only two winners. The second step follows from $f_j(b_1)\leq 3\delta$
and that the pivot player does not bid $b_0$ in $[1-\eps,1]$ so $f_{n+1}(b_0)=1/2$.

Subtracting Eq.~\eqref{eq:separable-3} and Eq.~\eqref{eq:separable-4}, one obtains
\begin{align}
\Gamma_i(b_{2}, s{-i})  - \Gamma_i(b_{1}, s_{-i})\notag \\
\geq &~   \frac{1}{2}\sum_{j \in [n]\setminus [i]}f_j(b_1) \cdot (1 - \Sigma^{A}_{i, j}) \prod_{r\in [n]\setminus \{i, j\}}f_r(b_0) +\frac{1}{2}f_{n+1}(b_1) -3n\delta f_{n+1}(b_1)-9n^2\delta^2 \notag \\
\geq &~ \frac{1}{2} \cdot (n-1) \cdot \delta \cdot \frac{1}{4} \cdot \frac{1}{2} + \frac{1}{2}f_{n+1}(b_1) -3n\delta f_{n+1}(b_1)-9n^2\delta^2\geq \frac{n\delta}{32}\label{eq:diff}.
\end{align} 
The second step holds due to $f_j(b_1) \geq \delta$, $\Sigma^{A}_{i, j} \in [1/4, 3/4]$ and Eq~\eqref{eq:separable-5}.
Hence we claim player $i$ prefers $b_2$ than $b_1$ at value $v_{i} \in [1-\eps, 1]$, since
\begin{align*}
u_{i}(v_{i}; b_2, s_{-i}) - u_{i}(v_i; b_1, s_{-i}) =&~  (v_{i} - b_2) \cdot \Gamma_{i}(b_2, s_{-i}) - (v_{i} - b_1) \cdot \Gamma_{i}(b_1, s_{-i}) \\
\geq &~ v_{i} \cdot (\Gamma_{i}(b_2, s_{-i}) - \Gamma_{i}(b_1, s_{-i})) - b_2\\
\geq &~ (1-\eps) \cdot \frac{n\delta}{32} - b_2 > \eps.
\end{align*}

Finally, for the pivot player $n+1$, the allocation probability of bidding $b_1$ satisfies
\begin{align}
\Gamma_{n+1}(b_1, s_{-i}) \leq &~ \prod_{i \in [n]}F_i(b_1)
\end{align}
and the allocation probability of $b_2$ satisfies
\begin{align}
\Gamma_{n+1}(b_2, s_{-i}) \geq &~ \prod_{i \in [n]}F_i(b_1) + \sum_{i \in [n]}f_i(b_2) \prod_{j \in [n]\backslash \{i\}}F_i(b_1)\notag \\
\geq &~\prod_{i \in [n]}F_i(b_1) + \frac{n\delta}{2}.\label{eq:separable2}
\end{align}
The first step holds since the tie-breaking rule favors player $n+1$ when at most one player in $[n]$ bids $b_2$, the second step holds due to $f_i(b_2) \geq \delta$ and Eq.~\eqref{eq:separable-5}.
Hence, at any $v_{n+1} \in [1-\eps, 1]$ we have
\begin{align*}
u_{n+1}(v_{n+1}; b_2, s_{-i}) - u_{n+1}(v; b_1, s_{-i})  = &~ (v_{n+1} - b_2) \cdot \Gamma_{n+1}(b_2, s_{-i}) - (v_{n+1} - b_1) \cdot \Gamma_{n+1}(b_1, s_{-i}) \\
\geq &~ v_{n+1} \cdot (\Gamma_{n+1}(b_2, s_{-i}) - \Gamma_{n+1}(b_1, s_{-i})) - b_2\\
\geq &~ (1-\eps) \cdot \frac{n\delta}{2} - b_2 > \eps.
\end{align*}
We conclude the proof of the lemma here.
\end{proof}


Lemma \ref{lem:separable} confirms that $b_0, b_1, b_2$ would appear in an $\eps$-BNE profile for every player $i\in [n]$. It still remains to determine at which value point a standard player $i \in [n]$ jumps from $b_1$ to $b_2$ in $s_i$.
Let $\tau_{i} \in (b_2, 1-\eps)$ be the jumping point from $b_{1}$ to $b_{2}$ of a standard player $i$. The following formula is convenient to use.


\begin{lemma}[Jumping point formula]
\label{lem:formula}
The jumping point $\tau_i$ of a standard player $i \in [n]$ satisfies
\begin{align*}
    \tau_{i} = b_{2} + \frac{\Gamma_i(b_{1}, s_{-i})\cdot (b_{2} - b_1)}{\Gamma_i(b_{2}, s_{-i}) - \Gamma_i(b_{1}, s_{-i})} \pm O\left(\frac{\eps}{\delta}\right).
\end{align*}
\end{lemma}
\begin{proof}
At any value point $v \in [0, 1]$, recall the utility of bidding $b_1$ equals
\begin{align*}
    u_{i}(v_i; b_1,s_{-i}) = (v_i - b_1)\Gamma_i(b_1, s_{-i})
\end{align*}
and the utility of bidding $b_{2}$ equals 
\[
u_{i}(v_i; b_{2},s_{-i}) = (v_i - b_{2})\Gamma_i(b_{2}, s_{-i}).
\]
Solving for $u_{i}(\tau_i, b_1,s_{-i}) = u_{i}(\tau_i, b_{2},s_{-i}) \pm \eps$, one obtains
\begin{align*}
    \tau_{i} = &~  \frac{\Gamma_i(b_{2}, s_{-i}) b_{2} - \Gamma_i(b_{1}, s_{-i})b_1 \pm \eps }{\Gamma_i(b_{2}, s_{-i}) - \Gamma_i(b_{1}, s_{-i})} \\
    = &~ b_{2} + \frac{\Gamma_i(b_{1}, s_{-i})\cdot (b_{2} - b_1) \pm \eps}{\Gamma_i(b_{2}, s_{-i}) - \Gamma_i(b_{1}, s_{-i})}\\ 
    = &~ b_{2} + \frac{\Gamma_i(b_{1}, s_{-i})\cdot (b_{2} - b_1)}{\Gamma_i(b_{2}, s_{-i}) - \Gamma_i(b_{1}, s_{-i})} \pm O\left(\frac{\eps}{\delta}\right) \label{eq:diff}.
\end{align*}
The last step follows from Eq.~\eqref{eq:diff}, and this finishes the proof of the lemma.%\footnote{\color{red}Add $n$ if the red $n$ in (7) is correct.}
\end{proof}

Let $x_i \in [0, \beta\delta]$ be the probability mass over the interval $(b_2, \tau_i)$, i.e., $x_i = \int_{b_2}^{\tau_i}p_i(v)\mathsf{d}v$, which we will refer to the jumping probability of $s_i$. We state a few facts that will be used repeatedly.
\begin{lemma}[Basic facts]
\label{lem:basic-fact}\ 
\begin{itemize}
\item For any standard player $i \in [n]$, we have
%\footnote{\color{red}Is the probability just $f_i(b_e)\cdot f_j(b_e)$?}
\[
\prod_{j \in [n]\setminus \{i\}}f_j(b_0) = 1 -(n-1)(2+\beta)\delta \pm O(n^2\delta^2)
\]
and
\[
\prod_{j \in [n]\setminus \{i\}}F_j(b_1) = 1 -(n-1)(\beta+1)\delta + \sum_{j \in [n]\setminus \{i\}}x_j \pm O(n^2\delta^2).
\]
\item For any $e \in \{1,2\}$, we have 
$$\sum_{i\neq j \in [n]} \Pr\big[s_i(v_i) = b_e \wedge s_j(v_j) = b_e\big] = O(n^2\delta^2).$$
\end{itemize}
\end{lemma}
\begin{proof}
For the first claim, we have
\begin{align*}
\prod_{j \in [n]\setminus \{i\}}f_j(b_0) = (1 - (2+\beta)\delta)^{n -1} = 1 - (n-1)(2+\beta)\delta  \pm O(n^2\delta^2)
\end{align*}
due to the choice of $\delta$. Similarly we have (using $x_i\in [0,\beta\delta]$
\begin{align*}
\prod_{j \in [n]\setminus \{i\}}F_j(b_1)  =  &~ \prod_{j \in [n]\setminus \{i\}}(1 - (1+\beta)\delta + x_i) \\
= &~ 1 - (n-1)(1+\beta)\delta + \sum_{j \in [n]\backslash \{i\}}x_j \pm O(n^2\delta^2).
\end{align*}
For the second claim, for any $e\in \{1,2\}$, we have
\begin{align*}
\sum_{i \neq j \in [n]} \Pr\big[s_i(v_i) = b_e \wedge s_j(v_j) = b_e\big] \leq \sum_{i\neq j \in [n]}(1+\beta)\delta^2= O(n^2\delta^2).
\end{align*}
We conclude the proof here.
\end{proof}

The key step is to determine the jumping point, where we use approximation.
\begin{lemma}[Jumping point]
\label{lem:jump}
The jumping point $\tau_i$ of a standard player $i\in [n]$ satisfies
\begin{align*}
\tau_i = \left(\frac{\Delta_{i,1} - \Delta_{i,2}}{\Delta_{i,1}^2} \pm O\left(\frac{\beta^2}{\Delta_{i,1}}\right) \right)\cdot b_2 
\end{align*}
where
\begin{align*}
\Delta_{i, 1} := &~ (n-1)\delta + \sum_{j\in [n]\setminus \{i\}}\Big(\beta\delta\cdot \Sigma^{A}_{i,j} + (1+\beta)\delta\cdot \Sigma^{B}_{i,j}\Big) \in \big[(n-1)\delta, 2n\delta\big]\quad\text{and}\\
\Delta_{i, 2} := &~ \sum_{j \in [n]\backslash \{i\}} \Big(1 - 2\Sigma^{A}_{ i, j} - \Sigma^{B}_{i, j}\Big)x_{j}  \in \big[-2n\beta\delta, n\beta\delta\big].
\end{align*}
\end{lemma}
\begin{proof}
For any standard player $i \in [n]$, we compute when it jumps from $b_{1}$ to $b_2$ using the formula in Lemma \ref{lem:formula}.
To do so, we first compute $\Gamma_i(b_{1}, s_{-i})$ and $\Gamma_i(b_{2}, s_{-i})$.
\begin{align}
\Gamma_i(b_{1}, s_{-i}) =&~ f_{n+1}(b_0) \cdot \left( \prod_{j \in [n]\setminus \{i\}}f_j(b_0) + \sum_{j\in [n]\setminus\{i\}}f_j(b_1)\prod_{j \in [n]\setminus \{i, j\}}f_j(b_0)\cdot \Sigma^{A}_{ i, j} \pm O(n^2\delta^2)  \right)\notag\\
= &~ \frac{1}{2}  \left( 1 - (n-1)(2+\beta)\delta + \sum_{j\in [n]\setminus\{i\}} (\delta + x_j)\Sigma^{A}_{i,j}  \right)\pm O(n^2\delta^2). \label{eq:prob1}
\end{align}
Here the first step follows from the tie-breaking rule and the second claim of Lemma \ref{lem:basic-fact}, the second step follows from $f_j(b_1) = \delta + x_j$ and the first claim of Lemma \ref{lem:basic-fact}.

The allocation probability of bidding $b_2$ obeys
\begin{align*}
    \Gamma_i(b_{2}, s_{-i}) = &~ f_{n+1}(b_0) \cdot \left(\prod_{j \in [n]\setminus \{i\}}F_j(b_1) +  \sum_{j\in [n]\setminus\{i\}}f_j(b_2) \prod_{j \in [n]\setminus \{i, j\}}F_j(b_1) \cdot \Sigma^{A}_{i, j} \pm O(n^2\delta^2) \right)\\
    &~ + f_{n+1}(b_2) \cdot \left(\sum_{j\in [n]\setminus\{i\}}f_j(b_2)\prod_{k \in [n]\setminus \{i, j\}}F_k(b_1) \cdot \Sigma^{B}_{i, j} \pm O(n^2\delta^2) \right)\\
    = &~ \frac{1}{2} \left(1 - (n-1)(1+\beta)\delta + \sum_{j \in [n]\setminus \{i\}}x_j + \sum_{j \in [n]\setminus \{i\}}(\delta + \beta\delta - x_j)\Sigma^{A}_{i,j}  \right)\\
    &~ + \frac{1}{2}\sum_{j \in [n]\setminus \{i\}}(\delta + \beta\delta - x_j)\Sigma^{B}_{i, j} \pm O(n^2 \delta^2).
\end{align*}
The first step uses the tie breaking rule and requires some explanations. In particular, (1) when the pivot player $n+1$ bids $b_0$, the player $i$ obtains $1$ unit of item when other players bid less than $b_2$, $\Sigma_{A, i, j}$ unit of item when only player $j$ bids $b_2$; we also make use of the second claim of Lemma \ref{lem:basic-fact} to omit the other case; (2) when the player $n+1$ bids $b_2$, the player $i$ obtains $0$ unit of good when no other players bid $b_0$ and obtains $\Sigma_{B,i, j}$ unit of goods when one other player $j$ bids $b_2$, and we omit other cases using Lemma \ref{lem:basic-fact}.
The second step follows from Lemma \ref{lem:basic-fact} and $f_j(b_2) = \delta + \beta\delta - x_j$.

Combining the above expression, we have
\begin{align}
\Gamma_i(b_{2}, s_{-i}) - \Gamma_i(b_{1}, s_{-i}) = &~ \frac{1}{2}(n-1)\delta + \frac{1}{2}\sum_{j\in [n]\setminus \{i\}}\Big(\beta\delta\cdot \Sigma^{A}_{i,j} + (1+\beta)\delta\cdot \Sigma^{B}_{i,j}\Big) \notag \\
&~ + \frac{1}{2}\sum_{j \in [n]\backslash \{i\}} \Big(1 - 2\Sigma^{A}_{i, j} - \Sigma^{B}_{i, j}\Big)x_{j} \pm O(n^2\delta^2)\label{eq:diff2}.
%(J - I - 2\Sigma_{A} - \Sigma_{B})_i^{\top} x 
\end{align}
Let $\Delta_{i,1}$ and $\Delta_{i,2}$ be defined as in the statement of the lemma.
%We abbreviate the above expression as 
%\begin{align*}
%\Delta_{i, 1} := &~ (n-1)\delta + \sum_{j\in [n]\setminus \{i\}}\Big(\beta\delta\cdot\Sigma^{A}_{i,j} + (1+\beta)\delta\cdot \Sigma^{B}_{i,j}\Big)\\
%\Delta_{i, 2} := &~ \sum_{j \in [n]\backslash \{i\}} \Big(1 - 2\Sigma^{A}_{i, j} - \Sigma^{B}_{i, j}\Big)x_{j} 
%\end{align*}
Note that $\Delta_{i, 1}$ does not depend on $\{x_j\}_{j\ne i}$ while $\Delta_{i, 2}$ depends on $\{x_j\}_{j \neq i}$.
It is easy to see that 
\begin{align}
\Delta_{i,1} \in \big[(n-1)\delta, 2n\delta\big] \quad \text{and} \quad  \Delta_{i, 2} = \big[-2(n-1)\beta\delta, (n-1)\beta\delta\big]\label{eq:range}.
\end{align}

Finally we can compute the jumping point $\tau_i$ using Lemma \ref{lem:formula} as follows:
\begin{align*}
\tau_i = &~  b_2 + \frac{\Gamma_i(b_{1}, s_{-i})\cdot (b_{2} - b_1)}{\Gamma_i(b_{2}, s_{-i}) - \Gamma_i(b_{1}, s_{-i})} \pm O\left(\frac{\eps}{\delta}\right) \\
= &~ b_2 + \frac{1 \pm O(n\delta)}{\Delta_{i,1}+\Delta_{i,2}} \cdot b_2 \pm O\left(\frac{\eps}{\delta}\right)\\
= &~ \left(\frac{\Delta_{i,1} - \Delta_{i,2}}{\Delta_{i,1}^2} \pm O\left(\frac{\beta^2}{\Delta_{i,1}}\right) \right)\cdot b_2 
\end{align*}
The second step follows from Eq.~\eqref{eq:diff2}, $\Gamma_i(b_1, \beta_i) = (1/2) \pm O(n\delta)$ (see Eq.~\eqref{eq:prob1}) and the choice of $b_1,b_2$.
The last step follows from Eq.~\eqref{eq:range}.
%\footnote{\color{red}Let's double check the last step.}
\end{proof}








\subsection{Reduction from generalized circuit}
\label{sec:reduce}

Given $\alpha<\beta$,
  we write $\mathsf{T}_{[\alpha, \beta]}: \R \rightarrow [\alpha, \beta]$ to denote the  truncation function with  $$\mathsf{T}_{[\alpha, \beta]}(x) = \min\big\{\max\{x, \alpha\}, \beta\big\}.$$ We recall the  generalized circuit problem  \cite{chen2009settling} and  present a simplified version from \cite{filos2021complexity}. 
\begin{definition}[(Simplified) generalized circuit]
	\label{def:generelized-circuit}
	A generalized circuit is a tuple $(V, G)$, where $V$ is a set of nodes and $G$ is a collection of gates.
    Each node $v \in V$ is associated with a gate $G_v$ that falls into one of two types $\{G_{1-}, G_{+}\}$: If $G_v$ is a $G_+$ gate, then it has two input nodes $v_1,v_2\in V\setminus \{v\}$; if it is a $G_{1-}$ gate then it takes one input node $v_1\in V\setminus \{v\}$.
Given $\kappa>0$,
%The generalized circuit problem asks to find 
a $\kappa$-approximation solution to $(V,G)$ is an assignment $x\in [0, 1]^{V}$such that for every node $v$:
     \begin{itemize}
    \item If $G_v$ is a $G_{+}$ gate and  takes input nodes $v_1, v_2 \in V\backslash \{v\}$, then $x_v = \mathsf{T}_{[0, 1]}(x_{v_1} + x_{v_2}\pm \kappa) $
    \item If $G_v$ is a  $G_{1-}$ gate and takes  an input node $v_1 \in V\backslash \{v\}$, then   $x_{v} = \mathsf{T}_{[0,1]}(1 - x_{v_1}\pm \kappa) $.
    \end{itemize}   

\end{definition}

The generalized circuit problem is known to be PPAD-hard for constant $\kappa$.
\begin{theorem}[\cite{rubinstein2015inapproximability,deligkas2022pure}]
There is a constant $\kappa>0$ such that it is PPAD-hard to find an $\kappa$-approximate solution of a generalized circuit.
\end{theorem}



We prove Theorem \ref{thm:hardness} via a reduction from the generalized circuit problem.

%\begin{proof}[Proof of Theorem \ref{thm:hardness}]
Given an instance of generalized circuit defined over nodes set $V$ ($|V| = m$), we let $V_1 = [m_1]$ be the set of nodes with gate $G_{+}$  and $V_2 = [m_1+1: m]$  be the set of nodes with gate $G_{1-}$.
We construct an instance of first price auction with $n = m_1 + 2(m - m_1) = 2m - m_1$ standard players and one pivot player. 



Let $\mN = \mN_1 \cup \mN_{2} \cup \mN_{3}$ be the set of standard players, where $ \mN_1 = [m_1]$, $ \mN_{2} = [m_1+1: m]$ and $\mN_{3} = [m+1: 2m - m_1]$.
From a high level, we use players in $\mN_1$ to represent the set of nodes with $G_{+}$ gates, players in $\mN_{2}$ to represent the set of nodes with $G_{1-}$ gates. 
Players in $\mN_{3}$ are used in constructing $G_{1-}$. 
We first specify the probability density $\tilde{p}$ over interval $(b_2, 1-\eps)$ and the tie-breaking matrices $\Sigma^A$ and $\Sigma^B$ to complete the description of the FPA instance.



\begin{itemize}
\item For player $i$ in $\mN_1$ (i.e., $i \in [m_1]$), its valuation distribution $\tilde{p}_i$ is uniform over the interval
\[
\left[\frac{1}{\Delta_{i, 1}} \cdot b_2, \left(\frac{1}{\Delta_{i, 1}} + \frac{1}{10}\cdot\frac{\beta\delta}{\Delta_{i, 1}^2}\right)\cdot b_2\right]
\]
with a total probability mass of $\beta\delta$. Let $i(1), i(2) \in [m]=\mN_1\cup \mN_2$ be the input nodes of the $G_{+}$ gates, we set $\Sigma^{B}_{i, i(1)} = \Sigma^{B}_{i, i(2)}= 1/10$.
\item For player $m_1 + j \in \mN_{2}$ (i.e., $j \in [m - m_1]$), its valuation distribution $\tilde{p}_{m_1+j}$ is uniform over 
\[
\left[\left(\frac{1}{\Delta_{m_1 + j, 1}} - \frac{1}{10}\cdot\frac{\beta\delta}{\Delta_{m_1 +j, 1}^2}\right)\cdot b_2, \frac{1}{\Delta_{m_1+j, 1}}\cdot b_2\right]
\] 
with a total probability mass of $\beta\delta$.
We set $\Sigma^{A}_{m_1 + j, m+ j} = 9/20.$
%$\frac{1}{2} - \frac{1}{20}$.
\item For player $m + j \in \mN_{3}$ (i.e., $j \in [m - m_1]$), its valuation distribution $\tilde{p}_{m+j}$ is uniform over  
\[
\left[\left(\frac{1}{\Delta_{m+j, 1}} + \frac{1}{10}\cdot\frac{\beta\delta}{\Delta_{m+j, 1}^2}\right)\cdot b_2, \left(\frac{1}{\Delta_{m+j, 1}} + \frac{1}{5}\cdot\frac{\beta\delta}{\Delta_{m+j, 1}^2}\right)\cdot b_2\right]
\]
with a total probability mass of $\beta\delta$. We set $\Sigma^{A}_{m + j, m_1+ j} = 11/20$. Let $j(1) \in [m]$ be the input node of $G_{1-}$, then set $\Sigma^{B}_{m+j, j(1)} =  {1}/{5}$.
\item For any entry of $\Sigma^{A}$ that has not been determined above, we set it to be $1/2$, and for any entry of $\Sigma^{B}$ that has not been determined, we set it to be $0$.
\end{itemize}


It is easy to verify that $\Sigma^{A}$ and $\Sigma^B$ satisfy the following properties as promised earlier: (1) the off-diagonal entries of $\Sigma^{A}$ lie in $[1/4, 3/4]$;  (2) $\Sigma^{A} + (\Sigma^{A})^{\top} = J_n - I_n$; and (3) the off-diagonal entries of $\Sigma^{B}$ belong to $[0,1/2]$.

Letting $\kappa: = n\beta = {1}/{n^3}$, we prove that any $\eps$-BNE of the first price auction gives an $O(\kappa)$-approximate solution to the generalized circuit.
Indeed the following lemma shows that by taking $x_{i}' =  {x_{i}}/{(\beta\delta)}$, where $x_i$ is the jumping probability of $s_i$, we obtain an $O(\kappa)$-approximate solution $(x_1',\ldots,x_m')$ to the input generalized circuit. This finishes the proof of Theorem \ref{thm:hardness}.
\begin{lemma}
Given an $\eps$-BNE of the first price auction and let $(x_1, \ldots, x_n)\in [0, \beta\delta]^{n}$ be the tuple of jumping probabilities, then we have
\begin{itemize}
\item For any $i \in [m_1]$, $x_{i} = \mathsf{T}_{[0, \beta\delta]}(x_{i(1)} + x_{i(2)} \pm O(\kappa\beta\delta))$
\item For any $j \in [m- m_1]$, $x_{m_1 + j} = \mathsf{T}_{[0, \beta\delta]}(\beta\delta - x_{j(1)} \pm O(\kappa\beta\delta))$
\end{itemize}
\end{lemma}
\begin{proof}
For the first claim, for any $i \in [m_1]$, one has
\begin{align*}
\Delta_{i, 2} = \sum_{r \in [n]\backslash \{i\}} \Big(1 - 2\Sigma^{A}_{i, r} - \Sigma^{B}_{ i, r}\Big)x_{r} = -\frac{1}{10}x_{i(1)}  - \frac{1}{10}x_{i(2)},
\end{align*}
where the second step follows from $\Sigma^{A}_{i, r} = 1/2$ for all $r\in [n]\backslash \{i\}$, $\Sigma^{B}_{i, i(1)} = \Sigma^{B}_{i, i(2)} = 1/10$ and $\Sigma^{B}_{i, r} = 0$ for all other $r \in [n]\backslash \{i, i_1, i_2\}$.
By Lemma \ref{lem:jump}, one has
\begin{align*}
\tau_i = \left(\frac{\Delta_{i,1} - \Delta_{i,2}}{\Delta_{i,1}^2} \pm O\left(\frac{\beta^2}{\Delta_{i,1}}\right)\right) \cdot b_2 = \left(\frac{1}{\Delta_{i,1}} + \frac{1}{10}\cdot \frac{x_{i(1)} + x_{i(2)}}{\Delta_{i,1}^2}  \pm O\left(\frac{\beta^2}{\Delta_{i,1}}\right)\right) \cdot b_2.
\end{align*}
Since $\D_{i}$ is uniform over $[\frac{1}{\Delta_{i, 1}} \cdot b_2, (\frac{1}{\Delta_{i, 1}} + \frac{1}{10}\cdot\frac{\beta\delta}{\Delta_{i, 1}^2})\cdot b_2]$ with probability mass $\beta\delta$, we have 
\[
x_i = \mathsf{T}_{[0, \beta\delta]}(x_{i(1)} + x_{i(2)} \pm O(\kappa\beta\delta)).
\]


For the second claim, we first analyse the jumping probability of player $m_1 + j$. We have
\begin{align*}
\Delta_{m_1 + j, 2} = \sum_{r \in [n]\backslash \{m_1+j\}} \Big(1 - 2\Sigma^{A}_{m_1+j, r} - \Sigma^{B}_{m_1 +j, r}\Big)x_{r} = \frac{1}{10}x_{m+ j}  ,
\end{align*}
where the second follows from $\Sigma^{A}_{m_1 + j, m+j} = 9/{20}$, $\Sigma^{A}_{m_1+j, r} = 1/2$ for all $r\in [n]\backslash \{m_1 + j,m+j \}$, and $\Sigma^{B}_{m_1+j, r} = 0$ for all $r \in [n]\backslash \{m_1 +j \}$. Hence, by Lemma \ref{lem:jump}, we have
\begin{align*}
\tau_{m_1 + j} = &~ \left(\frac{\Delta_{m_1 +j,1} - \Delta_{m_1 + j,2}}{\Delta_{m_1 + j,1}^2} \pm O\left(\frac{\beta^2}{\Delta_{m_1 + j, 1}}\right)\right) \cdot b_2\\
=&~ \left(\frac{1}{\Delta_{m_1 +j,1}} - \frac{1}{10}\cdot \frac{x_{m+ j}}{\Delta_{m_1+j,1}^2}  \pm O\left(\frac{\beta^2}{\Delta_{m_1+j,1}}\right)\right) \cdot b_2
\end{align*}
Given that $\D_{m_1+j}$ is uniform over $[(\frac{1}{\Delta_{m_1 + j, 1}} - \frac{1}{10}\cdot\frac{\beta\delta}{\Delta_{m_1 +j, 1}^2})\cdot b_2, \frac{1}{\Delta_{m_1+j, 1}}\cdot b_2]
$ with mass $\beta\delta$, we have 
\begin{align}
\label{eq:fix1}
x_{m_1+j} = \mathsf{T}_{[0, \beta\delta]}(\beta\delta - x_{m+ j} \pm O(\kappa\beta\delta)) .
\end{align}
It remains to analyse the jumping probability of player $m+j$, and we have 
\begin{align*}
\Delta_{m + j, 2} = &~ \sum_{r \in [n]\backslash \{m+j\}} \Big(1 - 2\Sigma^{A}_{m+j, r} - \Sigma^{B}_{m+j, r}\Big)x_{r}\\
= &~ -\frac{1}{10}x_{m_1+ j}   - \frac{1}{5}x_{j(1)} = -\frac{1}{10}\left(\beta\delta - x_{m + j} + 2x_{j(1)}\right) \pm O(\kappa\beta\delta).
\end{align*}
Here the second step follows from $\Sigma^{A}_{m + j, m_1+ j} = 11/20$, $\Sigma^{A}_{m + j, r} = 1/2$ for $ r\in [n]\backslash \{m_1+j, m+j\}$, $\Sigma^{B}_{m+j, j(1)} = 1/5$ and $\Sigma^{B}_{m + j, r}= 0$ for any $r\in [n]\backslash \{m+j, j(1)\}$. The last step follows from Eq.~\eqref{eq:fix1}. 

Now, by Lemma \ref{lem:jump}, one has
\begin{align*}
\tau_{m + j} = &~ \left(\frac{\Delta_{m+j,1} - \Delta_{m+j,2}}{\Delta_{m+j,1}^2} \pm O\left(\frac{\beta^2}{\Delta_{m+j,1}}\right)\right) \cdot b_2 \\
= &~ \left(\frac{1}{\Delta_{m+j,1}} + \frac{1}{10}\cdot \frac{\beta\delta - x_{m+ j} + 2x_{j(1)} \pm O(\kappa\beta\delta)}{\Delta_{m+j,1}^2}  \pm O\left(\frac{\beta^2}{\Delta_{m+j,1}}\right)\right) \cdot b_2
\end{align*}
Given that $\D_{m+j}$ is uniform over $[(\frac{1}{\Delta_{m+j, 1}} + \frac{1}{10}\cdot\frac{\beta\delta}{\Delta_{m+j, 1}^2})\cdot b_2,(\frac{1}{\Delta_{m+j, 1}} + \frac{1}{5}\cdot\frac{\beta\delta}{\Delta_{m+j, 1}^2})\cdot b_2]$ with mass $\beta\delta$, we conclude that
%\footnote{\color{red}Correct but may need a bit more explanation.}
\begin{align*}
x_{m+j} = \mathsf{T}_{[0, \beta\delta]}(x_{j(1)} \pm O(\kappa \beta\delta)).
\end{align*}
Plugging into Eq.~\eqref{eq:fix1}, we obtain
\begin{align*}
x_{m_1+j} = \mathsf{T}_{[0, \beta\delta]}(\beta\delta - x_{j(1)} \pm O(\kappa \beta\delta)).
\end{align*}
This completes the proof of the second claim.
\end{proof}


 



 





