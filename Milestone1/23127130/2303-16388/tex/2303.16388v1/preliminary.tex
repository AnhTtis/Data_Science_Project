\section{Preliminary}
\label{sec:pre}


\paragraph{Notation.} We write $[n]$ to denote $\{1,2,  \ldots, n\}$ and $[n_1:n_2]$ to denote $ \{n_1, n_1+1, \ldots, n_2\}$. Let $\mathsf{1}_{i}$ be an indicator vector -- it equals the all $0$ vector, except the $i$-th coordinate which equals $1$.
Let $\Delta_n$ contains all probability distribution over $[n]$. Given a vector $v \in \R^n$, and an index $i \in [n]$, $v_i$ denotes the $i$-th entry of $v$ while $v_{-i}$ denotes $(v_1, v_2, \ldots, v_{i-1}, v_{i+1}, \ldots, v_{n})$, i.e., all entries except the $i$-th coordinate. We write $x = y \pm \eps$ if $x \in [y-\eps, y+\eps]$. 
Let $J_n \in \R^{n\times n}$ be the $n\times n$ all-$1$ matrix and $I_n$ be the $n\times n$ identity matrix.

\subsection{Model}
In a Bayesian first-price auction (FPA), there is one single item to sell and it is specified by a tuple $(\mN, \mB, \D, \Gamma)$, where $\mN = [n]$ is the set of players, $\mB$ is the bid space, $\D$ is the value distribution and $\Gamma$ is the tie-breaking rule.
For each play $i \in \mN$, it has a private value $v_i$ of the item that is drawn from a 
(continuous) distribution $\D_{i}$ supported over $[0, 1]$ (written as $v_i \sim \D_{i}$).
We consider the standard {\em independent common prior} setting --- the joint value distribution $\D = \D_1 \times \cdots \times \D_{n}$ is the product distribution of $\{\D_i\}_{i \in [n]}$ and we assume the value profile $v = (v_1, \ldots, v_{n}) \in [0,1]^{n}$ is drawn from $\D$.
Let $\mB = \{b_0, b_1, \ldots, b_m\} \subset  [0, 1]$ be the bid space, where $0 = b_0 < b_1 < \cdots < b_m \leq 1$. 

In a first-price (sealed-bid) auction, each bidder $i$ submits a bid $\beta_i \in \mB$ simultaneously to the seller. 
The seller assigns the item to the winning player $i^{*}$ which submits the highest bid, and charges $i^{*}$ a payment equals to its bid $\beta_{i^{*}}$.


\paragraph{Allocation and tie-breaking rule.} When there are multiple players submitting the same highest bid, the seller assigns and charges the item to one of those winning players, following a pre-described {\em tie-breaking} rule $\Gamma$. 
A tie breaking rule $\Gamma: \{0, 1\}^n \rightarrow \Delta_n$ maps a set of winning players $W \subseteq [n]$ to an allocation profile $\Gamma(W) \in \Delta_n$ supported on $W$ that specifies the winning probability of each player $i \in W$ as $\Gamma_i(W)$.
Formally, given a bidding profile $\beta \in \mB^n$, the set of winning players $W(\beta)$ are those who submit the highest bids \[
W(\beta) = \left\{i  \in [n]: \beta_{i } = \max_{j\in [n]}\beta_{j}\right\}.
\]
The tie breaking rule $\Gamma(W(\beta)) \in \Delta_n$ specifies the winning probability of each player in $W(\beta)$ and $\Gamma_i(W(\beta))$ is the probability that the bidder $i$ obtains the item under the bidding profile $\beta$.
The tie-breaking rule needs to satisfy (1) $\Gamma_i(W(b)) > 0$ only if $i \in W(\beta)$, i.e., the item is assigned only to  players with the highest bid; and (2) $\sum_{i\in [n]}\Gamma_i(W(\beta)) = 1$, i.e., the total allocation is $1$.
When there is no confusion, we also abbreviate $\Gamma(\beta) = \Gamma(W(\beta))$.





It is known that the tie-breaking rule plays a subtle yet critical rule on the equilibrium of Bayesian FPA. 
Our hardness result is built upon the {\em trilateral} tie-breaking rule, a simple generalization of the commonly used uniform tie-breaking method.
\begin{definition}[Trilateral tie-breaking]
A \emph{trilateral} tie-breaking rule $\Gamma$ is 
  specified by the following tuples of nonnegative numbers 
$$\big(w_{i,j}:1\le i<j\le n\big)\quad \text{and}\quad
\left(\sigma_{i,j,k}^{(1)},\sigma_{i,j,k}^{(2)}:1\le i<j<k\le n\right)
  $$
such that $w_{i,j}\le 1$ and $\sigma_{i,j,k}^{(1)}+\sigma_{i,j,k}^{(2)}\le 1$.
%Let $n \geq 3$, $w_{i, j} \in [0, 1]$ for any $1 \leq i < j \leq n$, $\sigma_{i, j, k}^{(1)}, \sigma_{i, j, k}^{(2)} \in [0,1]$ for any $1\leq i < j < k \leq n$. 
Given a bidding profile $\beta\in \mB^n$ and the winning set $W(\beta)$, the item is distributed according to $\Gamma$ as follows
\begin{flushleft}\begin{enumerate}
    \item If $W(\beta)=\{i\}$ for some $i\in [n]$, then $\Gamma_i(\beta)=1$;
    \item If $W(\beta)=\{i,j\}$ for some $1\le i<j\le n$, then $\Gamma_i(\beta)=w_{i,j}$ and $\Gamma_j(\beta)=1-w_{i,j}$;
    \item If $W(\beta)=\{i,j,k\}$ for some $1\le i<j<k\le n$, then $\Gamma_i(\beta)=\sigma_{i,j,k}^{(1)}$,
    $\Gamma_j(\beta)=\sigma_{i,j,k}^{(2)}$ and 
    $\Gamma_k(\beta)=1-\sigma_{i,j,k}^{(1)}-\sigma_{i,j,k}^{(2)}$; and 
    \item When $| W(\beta)|\ge 4$, the item is evenly distributed among players in $W(\beta)$. \footnote{We note our hardness result actually holds regardless of the tie-breaking rule among more than $3$ players (i.e., not necessarily uniform).}
\end{enumerate}\end{flushleft}

%trilateral tie breaking rule is specified
%\begin{align*}
%\Gamma(\beta) = \left\{
%\begin{matrix}
%w_{i, j} \cdot \mathsf{1}_{i} + (1 - w_{i, j})\mathsf{1}_{j}  & W(\beta) = \{i, j\}\\
%\sigma_{i, j, k}^{(1)}  \cdot \mathsf{1}_{i}  + \sigma_{i, j, k}^{(2)} \cdot \mathsf{1}_{j} +  \cdot (1 - \sigma_{i, j, k}^{(1)} - \sigma_{i, j, k}^{(2)}) \cdot \mathsf{1}_{k} & W(\beta) = \{i, j, k\}  \\
%\frac{1}{|W(\beta)|} \cdot \mathsf{1}_{W(\beta)} & |W(\beta)| \geq 4.
%\end{matrix}
%\right.
%\end{align*}
%In another word, a trilateral tie-breaking rule splits evenly over the winning players except when there are only two/three winning bidders. In the later case, the tie-breaking is specified by the parameters $\{w_{i, j}\}_{1 \leq i<j\leq n}$, $\{\sigma_{i, j, k}^{(1)}, \sigma_{i, j, k}^{(2)}\}_{1\leq i < j < k \leq n}$.
\end{definition}


%\Binghui{Maybe we do not need this}
%Our algorithmic result aims to handle as general tie-breaking rules as possible.
%\begin{definition}[Ultimately uniform tie-breaking]
%Let $n \geq 2$. Given any bidding profile $\beta \in \mB^{n}$, a tie-breaking rule $\Gamma$ satisfies the {\em ultimately uniform property}, iff $\|\Gamma(\beta) - \frac{1}{|W(\beta)|}\cdot \mathsf{1}_{W(\beta)}\|_\infty \leq \frac{C}{|W(\beta)|}$ holds for some universal constant $C > 0$.
%\end{definition}



\paragraph{Equilibrium and strategy}
Given a tie-breaking rule $\Gamma$ and a bidding profile $\beta = (\beta_1, \ldots, \beta_n)$, the {\em ex-post} utility of a bidder $i$ is given by 
\begin{align*}
    u_i(v_i; \beta_i, \beta_{-i}) = (v_i - \beta_i) \cdot \Gamma_i(\beta).
\end{align*}
A strategy $s_i: [0, 1] \rightarrow \mB$ of player $i$ is a map from her (private) value $v_i$ to a bid $s(v_i) \in \mB$, with the following two properties:
\begin{itemize}
    \item {\bf No overbidding}. A player never submits a bid larger than her private value, i.e., $s_i(v_i) \leq v_i$ for all $v_i \in [0,1]$.
    \item {\bf Monotonicity.}  $s_i$ is a non-decreasing function.
\end{itemize}
These are common assumptions in the literature of first-price auction \cite{maskin2003uniqueness,lebrun2006uniqueness,filos2021complexity} and they rule out spurious equilibria in Bayesian auctions \cite{cai2014simultaneous}. 
Due to the monotonicity assumption, one can write a strategy $s_i$ as $m$ thresholds $0 \leq \tau_{i,1} \leq \cdots \leq \tau_{i, m} \leq 1$, where the player $i$  bids $b_j$ in the interval $(\tau_{i, j}, \tau_{i, j+ 1}]$\footnote{If the valuation distribution contains a point mass, then the strategy might be randomized at the point mass.}. Here we set by default $\tau_{i, 0}=0$ and $\tau_{i, m+1} = 1$.


The $\eps$-approximate Bayesian Nash equilibrium ($\eps$-approximate BNE) of FPA is defined as follow.
\begin{definition}[$\eps$-approximate Bayesian Nash equilibrium]
Let $n, m \geq 2$. Given a first-price auction ($\mN, \mB, \D,\Gamma$), a strategy profile $s = (s_1, \ldots, s_{n})$ is an $\eps$-approximate Bayesian Nash equilibrium ($\eps$-approximate BNE) if for any player $i\in [n]$, we have
\begin{align*}
    \E_{v \sim \D}\big[u_i(v_i; s_i(v_i), s_{-i}(v_{-i}))\big] \geq \E_{v \sim \D}\big[u_i(v_i; \bs(v_i, s_{-i}), s_{-i}(v_{-i}))\big] - \eps, 
\end{align*}
where $\bs(v_i, s_{-i}) \in \mB$ is the best response of player $i$ given other players' strategy $s_{-i}$, i.e.
\begin{align*}
\bs(v_i, s_{-i})\in \arg\max_{b\in \mB} \E_{v_{-i}\sim \D_{-i}}\big[u_i(v_i; b, s_{-i}(v_{-i}))\big].
\end{align*}
\end{definition}

The existence and the PPAD membership of finding a $1/\poly(n)$-approximate BNE can be established via a similar approach of \cite{filos2021complexity} (In particular, Theorem 4.1 and Theorem 4.4 of \cite{filos2021complexity}), and we omit the standard proof here.


We shall also use another notion of equilibrium which is more convenient in our hardness reduction. The $\eps$-approximately well-supported Bayesian Nash equilibrium ($\eps$-BNE) is defined as\footnote{We note the $\eps$-approximate BNE is known also {\em ex-ante} approximate BNE, and the $\eps$-BNE is known as {\em ex-interim} approximate BNE in some of the literature.}
\begin{definition}[$\eps$-approximately well-supported Bayesian Nash equilibrium]
Let $n, m \geq 2$. Given a first-price auction ($\mN, \mB, \D,\Gamma$), a strategy profile $s = (s_1, \ldots, s_{n})$ is an $\eps$-approximately well-supported Bayesian Nash equilibrium ($\eps$-BNE) if for any player $i\in [n]$ and $v_i \in [0, 1]$, we have
\begin{align*}
    \E_{v_{-i} \sim \D_{-i}}\big[u_i(v_i; s_i(v_i), s_{-i}(v_{-i}))\big] \geq \E_{v_{-i}\sim \D_{-i}}\big[u_i(v_i; \bs(v_i, s_{-i}), s_{-i}(v_{-i}))\big] - \eps. 
\end{align*}
\end{definition}


The notion of $\eps$-BNE and $\eps$-approximate BNE can be reduced to each other in polynomial time, losing at most a polynomial factor of precision. 
It is clear that an $\eps$-BNE is also an $\eps$-approximate BNE. Lemma \ref{lem:notion} states the other direction and the proof is deferred to the appendix.
\begin{lemma}\label{lem:notion}
Given a first-price auction ($\mN, \mB, \D,\Gamma$) and an $\eps$-approximate BNE $s$, there is a polynomial time algorithm that maps $s$ to an $\eps'$-BNE, where $\eps' = (2n+10)\sqrt{\eps}$. 
\end{lemma}


%{\color{red} Need to state the existence of Nash equilibria. We should also say something about how the distributions are given. For hardness we can even require them to be piecewise linear? but our algorithm works for more general distributions.}
%\Binghui{The existence and PPAD-membership should be true given the EC paper.}

















