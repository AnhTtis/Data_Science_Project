\section{Introduction}

\def\calD{\mathcal{D}}

First-price auction is arguably the most commonly used auction format in practice \cite{vickrey1961counterspeculation,roughgarden2010algorithmic,conitzer2022pacing}, in which the highest bidder wins the item and pays her bid.
First-price auction and its variants have been widely used in online ad auctions:
when a user visits a platform, an auction is run among interested advertisers to determine the ad
to be displayed to the user.
Despite of its simplicity,
first-price auctions are {\em not} incentive compatible ---
it is the most well-known example in auction theory that does not admit a truthful strategy.
%vulnerable to strategic manipulations.
%\footnote{\Binghui{Just want to double check: I am not sure strategic manipulations means? Does it mean non-truthful?}}
This has led to significant effort in economics \cite{lebrun1996existence,lebrun1999first,maskin2000equilibrium,lizzeri2000uniqueness,athey2001single,maskin2003uniqueness,reny2004existence,lebrun2006uniqueness,bergemann2017first} and more recently, in computer science \cite{chawla2013auctions,wang2020bayesian,filos2021complexity}, to understand equilibria of first-price auctions.

In this paper we study the computational complexity of finding a Bayesian Nash equilibrium in a first-price auction.
We consider the following {\em independent common prior} setting.
There is one single item to sell, and $n$ bidders are interested in it. Each bidder has a continuous value distribution $\calD_i$ supported over $[0,1]$. 
The joint value distribution $\calD$ is the product   of $\calD_i$'s. 
While $\calD_i$'s are public, 
each bidder $i$ has a private value $v_i$ for the item drawn from $\calD_i$. 
Each bidder chooses a bidding strategy, which maps her private value to a bid from a {\em discrete} bid space $\mB = \{b_0,b_1 \ldots, b_m\}$. A Bayesian Nash equilibrium is a tuple of bidding strategies, one for each bidder, such that every bidder gets a best response to other bidders' strategies
(see formal definition in Section \ref{sec:pre}, including the two conditions that bidding strategies need to satisfy: no overbidding and monotonicity).

This game between bidders, however, is not fully specified without a tie-breaking rule: how the item is allocated when more than one bidder have the highest bid. 
%A tie-breaking rule plays a subtle yet critical role on the equilibra of a first-price auction and a versatile form have been considered in the literature: 
A variety of tie-breaking rules have been considered in the literature.
The uniform tie-breaking rule, where the item is allocated to one of the winners uniformly at random, has been the most common offset. The other commonly used
tie-breaking rule is to  perform an additional round of Vickrey auction to ensure the existence of equilibria when the bidding space is continuous \cite{lebrun1996existence, deng2022efficiency}.
A recent line of works \cite{syrgkanis2014efficiency,hartline2014price,jin2022first,jin2023price} used monopoly tie-breaking rules that always give the item to one player  when establishing worst-case price-of-anarchy (POA) bounds.
%(or more generally, favors the lowest index players %\cite{cai2014simultaneous,}).
%Other tie-breaking rule has been considered when existence is of consideration, e.g., \cite{ lebrun1996existence} consider performing a second round of Vickrey auction if ties exist.
%When social welfare is of consideration, the line of work consider a tie breaking favors a monopoly \cite{hartline2014price}.
%Many other tie-breaking, e.g., favor the lowest index player

To accommodate 
%the versatile formats of 
tie-breaking rules in the problem, we consider the setting where the auctioneer specifies a tie-breaking rule $\Gamma$  to be used in the auction (as part of the input). $\Gamma$  maps each $W\subseteq [n]$ as the set of winners to a distribution $\Gamma(W)$ over $W$ as the allocation of the item to bidders in $W$. While a general tie-breaking rule takes exponentially many entries to describe, our PPAD-hardness result is built upon the succinct family of so-called \emph{trilateral} tie-breaking rules: such a tie-breaking rule $\Gamma$  specifies item allocations when no more than three bidders are in tie, and follows the uniform tie-breaking rule otherwise (when more than three bidders are in tie).
The hardness result rules out the  possibility of an efficient algorithm for finding a Bayesian Nash equilibrium  in a first-price auction when the tie-breaking rule is given as part of the input (unless PPAD is in P). 
We compliment our hardness result with a polynomial time approximation scheme (PTAS) for finding a constant-approximate Bayesian Nash equilibrium under the uniform tie-breaking rule.



%\footnote{\Binghui{Probably mention general tie-breaking rule has been considered, e.g \cite{lebrun1996existence}, the tie favors a monopoly \cite{hartline2014price} (essentially all those POA paper uses a non-uniform ties for their hard instance)}}


\subsection{Our results}



Our main hardness result shows that the problem of finding an $\eps$-approximate Bayesian Nash equilibrium in a first-price auction is PPAD-complete with trilateral tie-breaking.

\begin{restatable}[Computational hardness]{theorem}{Hardness}
\label{thm:hardness}
It is PPAD-complete to find an $\eps$-approximate Bayesian Nash equilibrium in a first-price auction under a trilateral tie-breaking rule for $\eps = 1/\poly(n)$.
\end{restatable}

It is worth pointing out that the hardness above holds even when (1) the bid space $\mathcal{B}$ has size $3$; and (2) the density function of each $\calD_i$ is a piecewise-constant function with no more than four nonzero pieces.

On the positive side, we obtain a PTAS for finding a Bayesian Nash equilibrium in a first-price auction under the uniform tie-breaking rule:

\begin{restatable}[PTAS under uniform tie-breaking]{theorem}{PTAS}
\label{thm:ptas}
For any $\eps > 0$, $n, m \geq 2$, there is an algorithm that finds an $\eps$-approximate Bayesian Nash equilibrium using $O(n^4 \cdot g(1/\eps))$ time under the uniform tie-breaking rule. 
\end{restatable}

Our algorithm works as long as it has oracle access to the CDF of each value distribution $\calD_i$.



\subsection{Related work}


\paragraph{First-price auction}
The study of first-price auction dates back to the seminal work of Vickrey \cite{vickrey1961counterspeculation} in 1960s.
Despite its extremely simple form and a wide range of applications, the incentive has been a central issue and it is perhaps the most well known mechanism that does not admit a truthful strategy. 
A long line of works in the economic literature \cite{riley1981optimal,plum1992characterization,marshall1994numerical,lebrun1996existence,lebrun1999first,maskin2000equilibrium,lizzeri2000uniqueness,athey2001single,maskin2003uniqueness,reny2004existence,lebrun2006uniqueness,chawla2013auctions,bergemann2017first} devote to characterizing the existence, uniqueness and closed-form expression of a pure Bayesian Nash equilibrium (or BNE).
However, the BNE of first-price auction is only well-understood in a few special cases, including when the players have symmetric valuation distributions \cite{chawla2013auctions}, when all players have probability density function bounded above $0$ and atomic probability mass at the lowest points \cite{lebrun2006uniqueness},  when there are only two bidders with uniform valuation distributions \cite{kaplan2012asymmetric} or when the players have discrete value and continuous bidding space and the tie-breaking is performed with an extra round of Vickrey (second-price) auction \cite{wang2020bayesian}. 

A formal study on the computational complexity of equilibria in a first-price auction has been raised by the recent work of \cite{filos2021complexity}, which is most closest to us. \cite{filos2021complexity} examines the computation complexity under a {\em subjective prior}, that is, each bidder has a different belief of other's valuation distribution. They prove the PPAD-completeness and the FIXP-completeness of finding an $\eps$-BNE (for some constant $\eps > 0$) and an exact BNE, under the uniform tie-breaking rule. As we shall explain soon, the techniques to obtain their results are quite different from us. 
It is worth noting that most aforementioned literature are on the common prior setup, and \cite{filos2021complexity} also leaves an open question of characterizing the computational complexity of $\eps$-BNE under the standard setting of independent common prior.
\cite{filos2021complexity} also provides a polynomial time algorithm for finding a high precision BNE for {\em constant} number of players and bids, when the input distribution are piecewise polynomial. Their approach is based on polynomial system solvers and thus different from us. 
The work of \cite{cai2014simultaneous} studies the Bayesian combinatorial auctions, where there are multi-items to sell for multiple bidders. They prove the complexity of Bayesian Nash equilibrium is at least PP-hard (a complexity class between the polynomial hierarchy and PSPACE), the model is quite different, because the agents' valuation could be much more complex, defining over subsets of items.

Other aspects of first-price auction have also been studied in the literature, including the price of anarchy/stability \cite{syrgkanis2013composable,syrgkanis2014efficiency,feldman2013simultaneous,hoy2018tighter,jin2022first,jin2023price} and parameter estimation \cite{guerre2000optimal,cherapanamjeri2022estimation}.








%When the goal is to design revenue maximization auctions, complexity results are known for 







\paragraph{Equilibrium computation}

The complexity class of PPAD (Polynomial Parity Arguments on Directed graphs) was first introduced by Papadimitriou~\cite{papadimitriou1994complexity} to capture one particular genre of total search functions. 
The seminal work \cite{daskalakis2009complexity,chen2009settling} established the PPAD-hardness of normal-form games. The hardness of approximation was settled by subsequent work~\cite{rubinstein2018inapproximability,rubinstein2016settling, deligaks2022pure} in the past few years. 
A broad range of problems have been proved to be PPAD-hard, and notable examples including  equilibrium computation in special but important class of games (win-or-lose game \cite{abbott2005complexity, chen2007approximation}, anonymous game \cite{chen2015complexity}, constant rank game \cite {mehta2014constant}, graphical game \cite{papadimitriou2021public}),
market equilibrium (Arrow-Debreu market~\cite{chen2009spending, chen2009settling-market, vazirani2011market}, non-monotone market \cite{chen2013complexity,rubinstein2019hardness}, Hylland-Zeckhauser scheme \cite{chen2022computational}),
fair division \cite{othman2016complexity, chaudhury2021competitive}, min-max optimization~\cite{daskalakis2021complexity} 
and reinforcement learning \cite{daskalakis2022complexity,jin2022complexity}. 


The PTAS is known for anonymous game \cite{daskalakis2015approximate}, which is closely related to our work. 
The \cite{daskalakis2015approximate} presented a $n^{g(m, 1/\eps)} \cdot U$ algorithm for $m$-action $n$-player anonymous games for some exponential function $g$. Here $U$ denotes the number of bits to represent a payoff value in the game. 
Instead, our algorithm finds an $\eps$-BNE of first-price auction with running time $n^4 \cdot g(1/\eps)$, which does not depend on the size of bidding space and the bit-size of the representation of the distributions.
It crucially utilizes the structure of first-price auction in the rounding and searching step, and could have a broader application in auction theory.


\subsection{Technical overview}
The challenge of obtaining the PPAD-hardness arises from two folds.
%\begin{enumerate}
%\item 
%\item The utility does not admit a close form expression and it is computed via a dynamic programming approach.
%\end{enumerate}
First, the utility function does not admit a closed-form expression, in terms of other player's strategy. It depends on an exponential number of possible bidding profiles and is computed only via a dynamic programming approach.
Second, the game structure is highly symmetric under the (independent) common prior.
In a first-price auction, the allocation is determined by the entire bidding profile, and each player faces ``almost'' the same set of profile. 
From this perspective, it is more like an anonymous game. 
Perhaps even worse, in an anonymous game, the utility function of each player is different and could be designed for the sake of reduction.
While in a first-price auction, the utility function of each player is the same, and depends only on the allocation probability.
Of course, the general (non-uniform) tie-breaking rule as well as the different valuation distributions could be used for breaking the symmetry.
We note the above challenges are unique to the common prior setting.
In a sharp contrast, in the subjective prior setting \cite{filos2021complexity}, the players' subjective belief could be different. A player could presume most other players
have zero value and submit zero bid, hence, the game is {\em local} and non-symmetric.

To resolve the above challenges, our key ideas are (1) linearizing the allocation probability and expanding a first order approximation of the utility function; and (2) carefully incorporating a (simple) general tie-breaking rule to break the symmetry.


\paragraph{Technical highlight: Linearizing ``everything''}
Given a strategy profile $s$, the distribution over the entire bidding profile (and therefore the allocation probability, the utility, the best response) could be complicated to compute, especially when multiple players submit the highest bid.
To circumvent this issue, we assign a large probability $(1-\delta)$ around value $0$ for all players, for some polynomially small $\delta > 0$.\footnote{This is the reason that our hardness result only applies for (inverse) polynomially small $\eps$.}
By doing this, the probability that a player bids nonzero is small, so one can ignore higher order term.
Concretely, let $p_{i, j}$ be the probability that player $i$ gets the item when bidding $b_j$ given that the other players have strategy $s_{-i}$, and let $\Gamma_i(b_j, s_{-i})$ be the allocation for player $i$ of bidding $b_j$ given other player's strategy $s_{-i}$. 
The immediate advantage is that the allocation probability can be approximated as 
\begin{align}
\Gamma_i(b_j,s_{-i}) \approx (1 - \sum_{i'\in [n]\setminus \{i\}}\sum_{j' > j}p_{i', j'}) + \sum_{i'\in [n]\setminus \{i\}} \Sigma_{i, i'} \cdot  p_{i', j}\label{eq:tie2}
\end{align}
under a {\em bilateral} tie-breaking rule. 
Here $\Sigma \in [0,1]^{n\times n}$ specifies the allocation when there is a tie between a pair of players $(i_1, i_2)$ and satisfies $\Sigma + \Sigma^{\top} = (J - I)$.
%As common for PPAD-reduction, 
At this stage,
it is tempting to use $p_{i, j}$ to encode variables of a generalized circuit problem and the choice of best response to encode constraints. 
%Our construction is as follow. The bid grows polynomial large, with $b_0 = 0, b_1 = 1/N^2, b_2 = 1/N$ and the probability between $b_{1}, b_{2}$ sums to one and we wish to use $p_{i,1}$ to represent the variable. 
In our final construction, we only need three bids $0 = b_{0} < b_1 < b_2$ and the variables are encoded by the jump point $\tau_{i}$ between $b_{1}, b_{2}$ (i.e., when player $i$ bids $b_2$ instead of $b_1$), which has the closed-form expression of
\begin{align}
    \tau_{i} = b_{2} + \frac{\Gamma_i(b_{1}, s_{-i})\cdot (b_{2} - b_1)}{\Gamma_i(b_{2}, s_{-i}) - \Gamma_i(b_{1}, s_{-i})}.\label{eq:tie-4}
\end{align}
Even after the linearization step of Eq.~\eqref{eq:tie2}, the above expression is still quite formidable to handle. 
Our next idea is to restrict the jumping point in a small interval between $(b_2,1)$, and assign only a small total probability mass of $\beta\delta$ over the interval, here $\beta$ is another polynomially small value. There is a (fixed) probability mass of $\delta$ around $b_2$ and $1$. 
One can further perform a first order approximation to Eq.~\eqref{eq:tie-4}, and again linearize the jumping point expression.


\paragraph{Incorporating tie-breaking rule}
Abstracting away some construction details, the above construction reduces the first-price auction from a fix point problem, obeys the following form
\begin{align}
\vec{p} = f(G \vec{p}) \quad \text{where} \quad G = 2\Sigma^{A} - J + I,\label{eq:fix8}
\end{align}
where $\vec{p}$ is the probability of bidding $b_1$ (inside the small interval), $f = (f_1, \ldots, f_n)$ is operated coordinate-wise over $G\vec{p}$, $f_i$ is a monotone function maps from $[a_i,b_i]$ (some fixed interval) to $[0, 1]$, $J$ is the all $1$ matrix and $I$ is the identity matrix.
The fixed point problem is fairly general and subsumes the generalized circuit problem, {\em if} $\Sigma^A$ is an arbitrary matrix in $[0,1]^{n\times n}$. 
Unfortunately, it is not true due to the constraint of $\Sigma^A + (\Sigma^{A})^{\top} = (J- I)$. 
We resolve the issue by adding an extra pivot player. 
The pivot player is guaranteed to bid $b_0 = 0$ and $b_2$ with equal probability of $1/2$. 
From a high level, the pivot player splits the equilibrium computation into two cases,  the case when it bids $b_0$ is similar,  while the case of  bidding $b_2$ introduces another tie-breaking matrix $\Sigma^{B} \in [0,1]^{n \times n}$ among the original players in $[n]$ (hence it becomes a trilateral rule). 
It transforms the fix point problem (i.e., Eq.~\eqref{eq:fix8}) to a more convenient form 
\begin{align}
\vec{p} = f(G' \vec{p}) \quad \text{where} \quad G' = 2\Sigma^{A} + \Sigma^{B} - J + I,\label{eq:fix9}
\end{align}
and one can construct gadgets to reduce from the generalized circuit problem. 
The last step is fairly common and details can be found in Section \ref{sec:ppad}.



%\subsubsection{The PTAS}
%Our PTAS algorithm follows from a few steps of rounding, discretization and search, which follows the general idea in the literature.
%A few new observations are required to obtain our results for first-price auction, these observations could be fairly general for auctions, and maybe of independent interests.






