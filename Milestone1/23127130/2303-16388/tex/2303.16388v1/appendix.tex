\section{Missing proof from Section \ref{sec:pre}}
We provide the proof of Lemma \ref{lem:notion}.
\begin{proof}[Proof of Lemma \ref{lem:notion}]
Given an $\eps$-approximate BNE strategy $s$, we transform each individual strategy $s_i$ into another strategy $s_i'$, such that
\begin{align}
\E_{v_{-i}\sim \D_{-i}}[u_i(v; s_i'(v_i), s_{-i}(v_{-i})) - u_i(v; \bs(v_i; s_{-i}),  s_{-i}(v_{-i}))] \leq 10\sqrt{\eps} \quad  \forall v\in [0,1], i \in [n], \label{eq:transform1}
\end{align}
and
\begin{align}
\Pr_{v_i \sim \D_i}[s_i'(v_i) \neq s_{i}(v_i)] \leq \sqrt{\eps}\quad \forall i \in [n].\label{eq:transform2}
\end{align}

Given the above properties, one can easily conclude that $s' = (s_1', \ldots, s_n')$ is an $\eps'$-BNE with $\eps' = 10\sqrt{\eps} + 2n\sqrt{\eps}$, since the bidding histogram changes at most $n\sqrt{\eps}$ in total variational distance.


Let $\bs_{\delta}(v_i, s_{-i})\subseteq \mB$ be the set of $\delta$-best response (note we slightly abuse notation and its meaning is different from Section \ref{sec:ptas}), i.e., 
\begin{align*}
\bs_{\delta}(v, s_{-i}):= \big\{b\in \mB: \E_{v_{-i}\sim \D_{-i}}\big[u_i(v; \bs(v, s_{-i}), s_{-i}(v_{-i})) - u_i(v; b, s_{-i}(v_{-i}))\big] \leq \delta, b\leq v \big\}.
\end{align*}


To obtain the desired strategy $s_i'$, we modify the strategy $s_i$ in a descending order, from $s_i(1)$ to $s_i(0)$.
It would be convenient to describe it in a continuous way, though we note it could be easily implemented in polynomial time.
For each value $v \in [0, 1]$, let $b^{+}(v)$ be the smallest bid given the private value is above $v$, i.e., $b^{+}(v):= \min_{b \in \mB, v' > v}s_i'(v')$.
We divide into cases.
\begin{enumerate}
    \item If $s_i(v) \in \bs_{4\sqrt{\eps}}(v; s_{-i})$ and $s_i(v) \leq b^{+}(v)$, then we keep the strategy unchanged, i.e., $s_{i}'(v) = s_{i}(v)$;
    \item Otherwise, let $B_{\delta}(v) := \bs_{\delta}(v_i, s_{-i}) \cap [0, b^{+}(v)]$ be the set of available $\delta$-best response at $v$, then
    \begin{enumerate}
    \item If $B_{4\sqrt{\eps}}(v) \neq \emptyset$, then take $s_i'(v) = \max_{b\in B_{5\sqrt{\eps}}(v)}b$; otherwise
    \item Take $s_i'(v) = b^{+}(v)$ to be the maximum possible bid.
    \end{enumerate}
\end{enumerate}
It is easy to verify that the strategy is monotone. There is no overbidding since in case (2-b), $b^{+}(v) \leq \bs(v, s_{-i}) \leq v$. 
%For proof convenience, we 
We next verify the strategy $s'$ satisfies the aforementioned properties.

To prove Eq.~\eqref{eq:transform1}, it suffices to consider the value $v\in [0,1]$ such that $s_i'(v) = b^{+}(v)$ and  $b^{+}(v) < b_{\min}(v)$ (i.e., case 2-b), where $b_{\min}(v) =\min_{b \in \bs_{4\sqrt{\eps}}(v, s_{-i})}b$ .
Let $v^{+} > v$ be the largest value such that $s_i'(v^{+}) = b^{+}(v)$, the maximum exists because it is wlog to assume the strategy is right closed. 
Then we know that $b^{+}(v) \in \bs_{5\sqrt{\eps}}(v^{+}, s_{-i})$ (i.e., it must fall into case 1 or case 2-a), therefore, 
\begin{align*}
5\sqrt{\eps} \geq &~ \E[u_{i}(v^{+};b_{\min}(v), s_{-i}(v_{-i})) - u_{i}(v^{+}; s_i'(v^{+}), s_{-i}(v_{-i}))] \\
\geq &~ \E[u_{i}(v;b_{\min}(v), s_{-i}(v_{-i})) - u_{i}(v; s_i'(v^{+}), s_{-i}(v_{-i}))].
\end{align*}
Here the second step holds due to $v^{+} \geq v$, $b_{\min}(v) > s_i'(v^{+}) = b^{+}(v)$ and the monotonicity of the utility difference.

Therefore, we have
\begin{align*}
&~ \E[u_{i}(v;\bs(v, s_{-i}), s_{-i}(v_{-i})) - u_{i}(v; s_{i}'(v), s_{-i}(v_{-i}))] \\
= &~ \E[u_{i}(v;\bs(v, s_{-i}), s_{-i}(v_{-i})) -u_i(v;b_{\min}(v), s_{-i}(v_{-i})) +  u_i(v;b_{\min}(v), s_{-i}(v_{-i})) - u_{i}(v; s_i'(v^{+}), s_{-i}(v_{-i}))] \\
\leq &~ 4\sqrt{\eps} + 5\sqrt{\eps} = 9\sqrt{\eps}.
\end{align*}
The second step holds since $b_{\min}(v) \in \bs_{4\sqrt{\eps}}(v, s_{-i})$. This proves Eq.~\eqref{eq:transform1}.

%We know $b^{+}(v) \in B_{10\sqrt{\eps}}(v^{+})$ by the maximum assumption.
%At value $v$, we have 
%\begin{align*}
%10\sqrt{\eps} < &~ \E[u_{i}(v;\bs(v, s_{-i}), s_{-i}(v_{-i})) - u_{i}(v; b^{+}(v), s_{-i}(v_{-i}))] \\
%\leq &~ \E[u_{i}(v^{+};\bs(v, s_{-i}), s_{-i}(v_{-i})) - u_{i}(v^{+}; b^{+}(v), s_{-i}(v_{-i}))] \leq 10\sqrt{\eps}.
%\end{align*}

To prove  Eq.~\eqref{eq:transform2}, let $V_i = \{v: s_i(v) \notin \bs_{\sqrt{\eps}}(v; s_{-i})\}$. By Markov inequality, the probability on $V_i$ is small, i.e., 
\[
\Pr[v\in V_i] \leq \sqrt{\eps}.
\]

We next prove $s_i'(v) = s_i(v)$ holds for any $v \in [0,1]\backslash V_i$.
We prove by contradiction and suppose Eq.~\eqref{eq:transform2} is violated at value $v \in [0,1]\backslash V_i$. 
Then we have $s_i(v) > b^{+}(v)$ (i.e., not case 1).
Let $v^{+} > v$ be the maximum value such that $s_i'(v^{+}) < s_{i}(v)$, again the maximum exists because it is wlog to assume the strategy is right closed.
At value $v^{+}$, the bid $s_i(v)$ is feasible (i.e., $s_i(v) \leq b^{+}(v^{+})$) and we are not at case 2-b, due to the maximum assumption. 
It we are at case 1, then we have $s_i'(v^{+}) = s_{i}(v^{+}) \geq  s_{i}(v)$ due to monotonicity of $s$, this contradicts with the assumption that $s_i'(v^{+}) < s_{i}(v)$.
Therefore, the only possibility is that we are at Case 2-a, and it suffices to prove $s_i(v) \in \bs_{5\sqrt{\eps}}(v^{+}, s_{-i})$ to establish a contradiction. 
%We divide into cases. If $\bs(v^{+}; s_{-i}) \leq s_i(v)$, then we have
%\begin{align*}
%\sqrt{\eps} \geq &~ \E[u_{i}(v; \bs(v^{+}; s_{-i}), s_{-i}(v_{-i})) - u_{i}(v; s_i(v), s_{-i}(v_{-i}))]\\
%\geq &~ \E[u_{i}(v^{+}; \bs(v^{+}; s_{-i}), s_{-i}(v_{-i})) - u_{i}(v^{+}; s_i(v), s_{-i}(v_{-i}))] .
%\end{align*}
%The first step follows from $v\notin V_1$, the second step follows from the $v < v^{+}$, $\bs(v^{+}; s_{-i}) \leq s_i(v)$ and the monotonicity of the utility difference.
%On the other hand, if $\bs(v_1; s_{-i}) > s_i(v)$. 

At case 2-a, let $b' \in B_{4\sqrt{\eps}}(v^{+})$ be any feasible $4\sqrt{\eps}$-best response. We have $b' \leq s_i'(v_1)$, and
\begin{align*}
&~\E[u_{i}(v^{+}; \bs(v^{+}; s_{-i}), s_{-i}(v_{-i})) - u_{i}(v^{+}; s_i(v), s_{-i}(v_{-i}))]\\
=&~\E[u_{i}(v^{+}; \bs(v^{+}; s_{-i}), s_{-i}(v_{-i})) - u_{i}(v^{+}; b', s_{-i}(v_{-i})) + u_{i}(v^{+}; b', s_{-i}(v_{-i})) - u_{i}(v^{+}; s_i(v), s_{-i}(v_{-i}))]\\
\leq &~4\sqrt{\eps} + \E[u_{i}(v^{+}; b', s_{-i}(v_{-i})) - u_{i}(v^{+}; s_i(v), s_{-i}(v_{-i}))]\\
\leq &~4\sqrt{\eps} + \E[u_{i}(v; b', s_{-i}(v_{-i})) - u_{i}(v; s_i(v), s_{-i}(v_{-i}))]
\leq 5\sqrt{\eps}.
\end{align*}
Here, the second step follows from $b' \in B_{4\sqrt{\eps}}(v^{+}) \subseteq \bs_{4\sqrt{\eps}}(v^{+}, s_{-i})$, the third step holds due to $v^{+} > v$, $b' \leq s_{i}(v)$ and the monotonicity of utility difference, 
and the last step follows from $v \notin V_1$. We finish the proof here.
\end{proof}