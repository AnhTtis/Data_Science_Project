\documentclass[11pt]{article}
%%%%
\usepackage{amsmath}
\usepackage{amsthm}
\usepackage{amssymb}
\usepackage{algorithm}
\usepackage{color}
\usepackage[english]{babel}
\usepackage{graphicx}
\usepackage{grffile}
%\usepackage{natbib}
\usepackage{wrapfig,epsfig}
\usepackage{epstopdf}
\usepackage{url}
\usepackage{color}
\usepackage{epstopdf}
\usepackage{algpseudocode}
\usepackage[T1]{fontenc}
\usepackage{bbm}
\usepackage{comment}
\usepackage{dsfont}
\usepackage{thm-restate}
\usepackage{bm}
\usepackage{tcolorbox}
\usepackage{subcaption}



\usepackage{enumitem}





\let\C\relax
\usepackage{tikz}
%\usepackage{hyperref}  %%% arxiv don't allow this.
%\hypersetup{colorlinks=true,citecolor=blue,linkcolor=blue}
\usetikzlibrary{arrows}

\usepackage[margin=1in]{geometry}


\graphicspath{{./figs/}}
\usepackage{mathtools}

\newtheorem{theorem}{Theorem}[section]
\newtheorem*{thm}{Theorem} 
\newtheorem{lemma}[theorem]{Lemma}
\newtheorem{definition}[theorem]{Definition}
\newtheorem{notation}[theorem]{Notation}
%\newtheorem{proof}[theorem]{Proof}
\newtheorem{proposition}[theorem]{Proposition}
\newtheorem{corollary}[theorem]{Corollary}
\newtheorem{conjecture}[theorem]{Conjecture}
\newtheorem{assumption}[theorem]{Assumption}
\newtheorem{observation}[theorem]{Observation}
\newtheorem{fact}[theorem]{Fact}
\newtheorem{remark}[theorem]{Remark}
\newtheorem{claim}[theorem]{Claim}
\newtheorem{example}[theorem]{Example}
\newtheorem{problem}[theorem]{Problem}
\newtheorem{open}[theorem]{Open Problem}
\newtheorem{hypothesis}[theorem]{Hypothesis}
\newtheorem{question}[theorem]{Question}


\newcommand{\Tmat}{\mathcal{T}_{\mathrm{mat}}}


\newcommand{\wh}{\widehat}
\newcommand{\wt}{\widetilde}
\newcommand{\ov}{\overline}
\newcommand{\eps}{\epsilon}
\newcommand{\N}{\mathcal{N}}
\newcommand{\R}{\mathbb{R}}
\newcommand{\F}{\mathbb{F}}
\newcommand{\vol}{\mathrm{vol}}
\newcommand{\RHS}{\mathrm{RHS}}
\newcommand{\LHS}{\mathrm{LHS}}
\renewcommand{\i}{\mathbf{i}}
\renewcommand{\varepsilon}{\epsilon}
\renewcommand{\tilde}{\wt}
\renewcommand{\hat}{\wh}
\renewcommand{\bar}{\overline}
\renewcommand{\eps}{\epsilon}
\renewcommand{\d}{\mathrm{d}}
\newcommand{\norm}[1]{\left\|#1\right\|}
\newcommand{\tnabla}{\tilde{\nabla}}
\newcommand{\bx}{\mathbf{x}}
\newcommand{\ba}{\mathbf{a}}
\newcommand{\by}{\mathbf{y}}
\newcommand{\bz}{\mathbf{z}}
\newcommand{\bb}{\mathbf{b}}
\newcommand{\prox}{\mathsf{prox}}
\newcommand{\bg}{\mathbf{g}}
\newcommand{\bu}{\mathbf{u}}
\newcommand{\br}{\mathbf{r}}
\newcommand{\bc}{\mathbf{c}}
\newcommand{\bv}{\mathbf{v}}
\newcommand{\bl}{\bm{\ell}}
\newcommand{\bh}{\mathbf{h}}
\newcommand{\blambda}{\bm{\lambda}}
\newcommand{\sign}{\mathsf{sign}}
\newcommand{\relu}{\mathsf{ReLU}}
\newcommand{\CC}{\mathsf{CC}}
\newcommand{\TV}{\mathsf{TV}}
\newcommand{\Cv}{\mathsf{U}}
\newcommand{\trun}{\mathsf{trun}}
\newcommand{\bad}{\mathsf{bad}}
\newcommand{\bs}{\mathsf{bs}}




\newcommand{\Binghui}[1]{{\color{blue}[Binghui: #1]}}




\newcommand{\loss}{\mathbf{Loss}}
\newcommand{\reg}{\mathbf{Reg}}
\newcommand{\Obj}{\mathbf{Obj}}




\newcommand{\bX}{\mathbf{X}}
\newcommand{\bY}{\mathbf{Y}}
\newcommand{\bZ}{\mathbf{Z}}
\newcommand{\bR}{\mathbf{R}}
\newcommand{\bS}{\mathbf{S}}
\newcommand{\bH}{\mathbf{H}}
\newcommand{\bM}{\mathbf{M}}
\newcommand{\bA}{\mathbf{A}}
\newcommand{\bB}{\mathbf{B}}
\newcommand{\bD}{\mathbf{D}}
\newcommand{\KL}{\mathsf{KL}}
\newcommand{\Img}{\mathsf{Img}}
\newcommand{\lrg}{\mathsf{large}}
\newcommand{\nil}{\mathsf{nil}}
\newcommand{\ms}{s^{\mathsf{m}}}







\newcommand{\D}{\mathcal{D}}
\newcommand{\mH}{\mathcal{H}}
\newcommand{\X}{\mathcal{X}}
\newcommand{\A}{\mathcal{A}}
\newcommand{\mP}{\mathcal{P}}
\newcommand{\mF}{\mathcal{F}}
\newcommand{\mL}{\mathcal{L}}
\newcommand{\mQ}{\mathcal{Q}}
\newcommand{\mB}{\mathcal{B}}
\newcommand{\mI}{\mathcal{I}}
\newcommand{\mN}{\mathcal{N}}
\newcommand{\sE}{\mathsf{E}}




\newcommand{\tj}{t\rightarrow t+1}






\DeclareMathOperator*{\E}{{\mathbb{E}}}
\DeclareMathOperator*{\Var}{{\bf {Var}}}
%\DeclareMathOperator*{\var}{\mathrm{Var}}
\DeclareMathOperator*{\Z}{\mathbb{Z}}
%\DeclareMathOperator*{\R}{\mathbb{R}}
\DeclareMathOperator*{\C}{\mathbb{C}}
%\DeclareMathOperator*{\AND}{\mathrm{AND}}
\DeclareMathOperator*{\OR}{\mathrm{OR}}
\DeclareMathOperator*{\median}{median}
\DeclareMathOperator*{\mean}{mean}
\DeclareMathOperator{\OPT}{OPT}
\DeclareMathOperator{\ALG}{ALG}
\DeclareMathOperator{\supp}{supp}
%\DeclareMathOperator{\head}{head}
%\DeclareMathOperator{\tail}{tail}
\DeclareMathOperator{\sparse}{sparse}
\DeclareMathOperator{\poly}{poly}
\DeclareMathOperator{\Tr}{Tr}
\DeclareMathOperator{\nnz}{nnz}
\DeclareMathOperator{\emp}{emp}
\DeclareMathOperator{\est}{est}
\DeclareMathOperator{\sparsity}{sparsity}
\DeclareMathOperator{\rank}{rank}
\DeclareMathOperator{\Diag}{diag}
\DeclareMathOperator{\dist}{dist}
\DeclareMathOperator{\dis}{dis}
\DeclareMathOperator{\signal}{signal}
\DeclareMathOperator{\cost}{cost}
\DeclareMathOperator{\vect}{vec}
\DeclareMathOperator{\tr}{tr}
\DeclareMathOperator{\RAM}{RAM}
\DeclareMathOperator{\diag}{diag}
\DeclareMathOperator{\Err}{Err}
\DeclareMathOperator{\head}{head}
\DeclareMathOperator{\tail}{tail}
\DeclareMathOperator{\cts}{cts}
\DeclareMathOperator{\jl}{jl}
\DeclareMathOperator{\new}{new}
\DeclareMathOperator{\old}{old}
\DeclareMathOperator{\refs}{ref}

\DeclareMathOperator{\obj}{obj}
\DeclareMathOperator{\init}{init}
\DeclareMathOperator{\unif}{unif}
\DeclareMathOperator{\regret}{regret}
\DeclareMathOperator{\im}{Im}
\DeclareMathOperator{\adv}{adv}





% \DeclareMathOperator{\vec}{vec}
\renewcommand{\algorithmicrequire}{\textbf{Input:}}
\renewcommand{\algorithmicensure}{\textbf{Output:}}
\algnewcommand\algorithmicforeach{\textbf{for each}}
\algdef{S}[FOR]{ForEach}[1]{\algorithmicforeach\ #1\ \algorithmicdo}



\makeatletter
\newcommand*{\RN}[1]{\expandafter\@slowromancap\romannumeral #1@}
\makeatother

\title{Complexity of Equilibria in First-Price Auctions\\ under General Tie-Breaking Rules }
\author{}
\author{  
Xi Chen \\ Columbia University \\ \texttt{xichen@cs.columbia.edu}
\and Binghui Peng \\ Columbia University \\ \texttt{bp2601@columbia.edu}
}
\date{}






\begin{document}
\maketitle



\begin{abstract}
We study the complexity of finding an approximate (pure) Bayesian Nash equilibrium in a first-price auction with common priors when the tie-breaking rule is part of the input.
We show that the problem is PPAD-complete even when the tie-breaking rule is trilateral (i.e., it specifies item allocations when no more than three bidders are in tie, and adopts the uniform tie-breaking rule otherwise). This is the first hardness result for equilibrium computation in first-price auctions with common priors.
On the positive side, we give a PTAS for the problem under the uniform tie-breaking rule.
\end{abstract}


\setcounter{page}{0}
\thispagestyle{empty}


\newpage
\section{Introduction}

The increasing complexity of source code poses a key challenge to the reliability of large-scale software systems. Software bugs in these systems can lead to safety issues~\cite{bug_safety} for users around the world as well as cause non-negligible financial losses~\cite{bug_loss}. As such, developers have to spend a large amount of time and effort on bug fixing. Consequently, \aprfull (\apr), designed to automatically generate patches to fix software bugs, has attracted wide attention from both academia and industry~\cite{long2016prophet, legoues2012genprog, long2015spr, lou2020can, tufano2018empstudy}. 


To achieve \apr, one popular approach is known as Generate-and-Validate (G\&V)~\cite{qi2015gv, ghanbari2019prapr, lou2020can, le2016hdrepair, legoues2012genprog, wen2018capgen, hua2018sketchfix, martinez2016astor, koyuncu2020fixminder, liu2019tbar, liu2019avatar}, which is typically based on the following pipeline: First, fault localization techniques~\cite{wong2016fl, abreu2007ochiai, zhang2013injecting, papadakis2015metallaxis, li2019deepfl, li2017transforming} are applied to determine the suspicious locations in programs where bugs are likely to exist. Then, the buggy locations are used by the \apr tools to generate a list of patches that replace buggy lines with correct lines. Afterward, each patch is validated against the original test suite to identify any \emph{plausible patches} (i.e., passing all tests in the test suite). Finally, to determine the \emph{correct patches}, developers examine the list of plausible patches to see if any of them can correctly fix the bug. 

Traditional \apr tools can mainly be categorized into heuristic-based~\cite{legoues2012genprog, le2016hdrepair, wen2018capgen}, constraint-based~\cite{mechtaev2016angelix, le2017s3, demacro2014nopol, long2015spr} and \template~\cite{ghanbari2019prapr, hua2018sketchfix, martinez2016astor, liu2019tbar, liu2019avatar}. Among these traditional tools, \template \apr tools~\cite{ghanbari2019prapr, liu2019tbar, benton2020effectiveness} have been able to achieve state-of-the-art results. \Template \apr tools typically leverage pre-defined templates (e.g., adding a nullness check) for bug fixing. However, since these fix templates are typically handcrafted, the number and types of bugs they are able to fix can be limited. 



To address the limitations of traditional \apr, researchers have proposed various \learning \apr tools~\cite{li2020dlfix, chen2018sequencer, jiang2021cure, lutellier2020coconut, zhu2021recoder, ye2022rewardrepair} based on the \nmtfull (\nmt) architecture~\cite{sutskever2014mt} where the input is the buggy code snippets and the goal is to translate the buggy code snippets into a fixed version. To accomplish this, \learning \apr tools require supervised training datasets with pairs of both buggy and fixed code snippets in order to learn how to perform this translation step. These training data are usually obtained by mining historical bug fixes using heuristics/keywords~\cite{dallmeier2007benchmark}, which can be imprecise for identifying bug-fixing commits; even the actual bug-fixing commits can include irrelevant code changes, leading to further pollution in the dataset~\cite{xia2022alpharepair}.
% 
Moreover, it can be hard for such \apr tools to generalize and fix bug types unseen during training. 



To better leverage recent advances in \plmfull{s} (\plm{s}), researchers~\cite{xia2022alpharepair, xia2023repairstudy, kolak2022patch, prenner2021codexws} have directly applied \plm{s} to generate patches without bug-fixing datasets. These \llm-based \apr tools work by either directly generating a complete code function~\cite{prenner2021codexws, xia2023repairstudy} or predict/infill the correct code snippet given its surrounding context~\cite{xia2022alpharepair, xia2023repairstudy}. By directly using \llm{s} that are pre-trained on billions of open-source code snippets, \llm-based \apr tools can achieve state-of-the-art performance on many repair datasets~\cite{xia2022alpharepair}. 


% 
%
%

Traditional \apr tools have long used the insight of the \emph{plastic surgery hypothesis}~\cite{barr2014plastic} where it states that the code ingredients to fix a bug already exist within the same project. Traditional \apr tools have manually designed pattern-~\cite{ghanbari2019prapr, saha2017elixir} or heuristic-based~\cite{jiang2018simfix, legoues2012genprog} approaches to finding and using such relevant code ingredients to generate fixes for bugs. However, the plastic surgery hypothesis has been largely ignored in \llm-based \apr. In fact, \llm provides a unique opportunity to fully automate the plastic surgery hypothesis idea via fine-tuning (learning project-specific information via model updates from the buggy project) and prompting (directly providing relevant code ingredients to the model), and make it directly applicable to different languages (since the \llm{s} are typically multi-lingual).%
Moreover, despite the intensive manual efforts involved, traditional \apr tools still cannot fully leverage project-specific information due to large search space for leveraging/composing existing code ingredients. In contrast, the project-specific information can effectively leveraged by \llm{s} due to their power in code understanding/vectorization, e.g., even partial/imprecise information may still guide \llm{s} in correct patch generation!
 To this end, we ask the question: \emph{How useful is the plastic surgery hypothesis in the era of \plm{s}}?








\mypara{Our Work.} To answer the question, we present \ourtech{\xspace} -- a \llm-based approach that automatically utilizes the plastic surgery hypothesis by systematically combining multiple fine-tuning and prompting strategies for \apr. \ourtech fine-tunes \plm{s} using two novel domain-specific training strategies: \textbf{\epfinetune} -- we fine-tune using the original buggy project by aggressively masking out a high percentage of tokens, which allows \plm to learn project-specific code tokens and programming styles; and \textbf{\rofinetune} -- which only masks out a single continuous code sequence per training sample, allowing the model to get used to the final \csapr task of predicting a single continuous code sequence. Furthermore, we directly leverage the ability for \plm{s} to understand natural language instructions and introduce a novel prompting strategy, \textbf{\idprompting}, which uses information retrieval and static analysis to obtain a list of relevant identifiers for the buggy lines. While such relevant identifiers are critical for fixing some difficult bugs, they may not be seen by the \llm during inference due to limited context window size. Through the use of prompting, we directly tell the model to use these extracted identifiers (relevant code ingredients) to generate the correct code. Finally, to perform repair, we combine all four model variants (including the base model, both fine-tuned models and the base model with prompting) for the final repair.





While our insight of leveraging the plastic surgery hypothesis for \llm-based \apr is generalizable across different types of \plm{s}, to implement \ourtech, we choose a recent \plm{\xspace}, \ctfive~\cite{wang2021codet5}, which is pre-trained on millions of open-source code snippets. \ctfive is an encoder-decoder model trained using \mspfull (\msp) objective where a percentage of tokens are masked out and each continuous masked token sequence is referred to as a masked span. Also, although we only extract relevant identifiers from the current buggy project (since this paper focuses on the plastic surgery hypothesis), our work can be easily extended to obtain other code information (such as relevant statements or functions) from other sources, such as  the massive pre-training corpora~\cite{husain2020codesearchnet} or historical bug-fixing datasets~\cite{jiang2019infer}, which can provide more coding knowledge for \llm{s}. Besides, although we mainly focus on using traditional string comparison algorithms for information retrieval in this paper, these techniques can be easily replaced by other frequency-based retrieval~\cite{robertson2009probabilistic} and neural search (or embedding-based search)~\cite{reimers2019sentence}.
  In summary, this paper makes the following contributions:


%


\begin{itemize}[noitemsep, leftmargin=*, topsep=0pt]
    \item \textbf{Dimension.} This paper is the first to revisit the important plastic surgery hypothesis in the era of \llm{s}. It opens up a new dimension for \llm-based \apr to incorporate previously neglected information from the buggy project itself to boost \apr performance. Furthermore, it demonstrates the promising future of retrieval-based prompting for modern \llm-based \apr.
    \item \textbf{Implementation.} We implement \ourtech based on the recent \ctfive model. We augment the model using two novel fine-tuning strategies: \epfinetune and \rofinetune, along with a novel prompting strategy based on information retrieval and static analysis: \idprompting. We combine the patches generated by all four models together and perform patch ranking to speed up \apr.% 
    \item \textbf{Evaluation Study.} We conduct an extensive evaluation against state-of-the-art \apr tools. On the widely studied \dfj 1.2 and 2.0 datasets~\cite{just2014dfj}, \ourtech is able to achieve the new state-of-the-art results of 89 and 44 correct bug fixes (15 and 8 more than best baseline) respectively.  Furthermore, we perform a broad ablation study to justify our design. \ourtech demonstrates for the first time that the plastic surgery hypothesis can substantially boost \llm-based \apr and advance state-of-the-art \apr, while being fully automated and general. Moreover, even partial/imprecise code ingredients may still effectively guide \llm{s} for \apr!
\end{itemize}


%!TEX root = ../main.tex

\section{Inductive Conformal Prediction}
\label{sec:pre:icp}
Given a set $\{ z_i = ( x_i, y_i ) \}_{i=1}^l$ with observation $x_i \in \calX$ and label $y_i \in \calY$ such that each $z_i \in \calZ := \calX \times \calY $ is drawn i.i.d. from an \emph{unknown} distribution on $\calZ$, inductive conformal prediction (ICP) provides 
% a simple yet powerful framework to learn 
a \emph{set prediction} $\Feps(x) \subseteq \calY$, parameterized by an error rate $0 < \epsilon <1$, such that given a new sample $z_{l+1} = (x_{l+1},y_{l+1})$ satisfying an \emph{exchangeability} condition (elaborated in Theorem~\ref{thm:icp-validity}), we have
\bea\label{eq:icpmiscoverage}
\probof{ y_{l+1} \in \Feps(x_{l+1}) } \geq 1-\epsilon, 
\eea
\ie, the prediction set $\Feps$ guarantees to contain the true label $y_{l+1}$ with probability at least $1-\epsilon$. 

% In order to achieve the probabilistic coverage in~\eqref{eq:icpmiscoverage}, ICP performs the following three steps.

{\bf Training}. We start by dividing the dataset into a \emph{proper training set} $\{ z_1,\dots,z_m \}$ and a \emph{calibration set} $\{ z_{m+1},\dots,z_{l} \}$. We shorthand $n = l - m$ as the size of the calibration set.
We learn a prediction function $f: \calX \rightarrow \tcalY$ from the proper training set using \emph{any} architecture, which allows us to fully exploit the power of modern deep learning. The prediction space $\tcalY$ can be the same as the label space $\calY$, or can contain auxiliary information such as a heuristic notion of uncertainty (\eg, softmax scores in classification or a heatmap in the case of keypoint detection). 

{\bf Conformal calibration}. 
% Leveraging the learned $f$, 
We define a \emph{nonconformity} function $S: \calZ^{m} \times \calZ \rightarrow \Real{}$ to measure how well a given sample $z = (x,y)$ \emph{conforms} to the proper training set. A popular instance of $S$ leverages the learned prediction $f$:
\bea \label{eq:nonconformity}
S\parentheses{\cbrace{z_1,\dots,z_m},(x,y)} \stackrel{\eg}{=} r(y,f(x)),
\eea
where $r: \calY \times \tcalY \rightarrow \Real{}$ is a measure of disagreement between the label $y$ and the prediction $f(x)$. For example, consider $\calY = \tcalY = \Real{}$, one can design $r(y,f(x)) = \abs{y - f(x)}$: if $(x,y)$ poorly conforms to the training set, $f$ will incur large errors.   
While the function $S$ can be arbitrary (\eg, a learnable neural network~\cite{stutz22iclr-learnconformal}), \eqref{eq:nonconformity} is a convenient definition since $f$ is implicitly dependent on $\{z_i\}_{i=1}^m$ and $r$ can incorporate domain-specific knowledge.
We then compute the nonconformity scores on the calibration set as $\alpha_i = r(y_i,f(x_i)), i = m+1,\dots,l$,
and sort them in \emph{nonincreasing} order $\alpha_{\pi(1)}\geq\dots \geq \alpha_{\pi(n)}$, where $\pi(i) \in \{m+1,\dots,l\}$ is an index permutation.
 % (offset by $m$).

{\bf Conformal prediction}. Given a new observation $x_{l+1}$ (with an unknown $y_{l+1}$) and a user-specified $\epsilon \in (0,1)$, we compute the inductive conformal prediction (ICP) set as
\bea\label{eq:icpcompute}
\Feps \parentheses{x_{l+1}} = \cbrace{y \in \calY \mid \alpha^y \leq \alpha_{\pi(\floor{(n+1)\epsilon})}},
\eea
where $\alpha^y = r(y,f(x_{l+1}))$
is the nonconformity score of the new sample when fixing the true label to be $y$. In other words, the ICP set~\eqref{eq:icpcompute} outputs the set of all labels that make the nonconformity score of the new sample no greater than $\alpha_{\pi(\floor{(n+1)\epsilon})}$ -- the $\floor{(n+1)\epsilon}$-th largest nonconformity score in the calibration set. 
% By doing so, ICP ensures that there are at least $\floor{(n+1)\epsilon}$ samples in the calibration set that are less conforming than the new sample. 
We have the following result stating the probabilistic coverage of the ICP set~\eqref{eq:icpcompute}.
% provides a valid statistical coverage of the true label $y_{l+1}$.

\begin{theorem}[Validity of ICP Coverage {\cite{vovk05book-conformal,lei18jasa-conformal,vovk12acml-icpconditional}}] \label{thm:icp-validity}
If $z_{m+1},\dots,z_l$, $z_{l+1} = (x_{l+1},y_{l+1})$ are exchangeable, \ie, their distribution is invariant under permutation, then
\bea\label{eq:icpvalidity}
1 - \epsilon \leq \probof{y_{l+1} \in \Feps(x_{l+1})} \leq 1 - \epsilon + 1/(n+1)
\eea
for any $\epsilon \in (0,1)$. Furthermore, when conditioned on the calibration set, calling $h = \floor{(n+1)\epsilon}$, we have
\begin{equation}\label{eq:beta}
\hspace{-4mm}\probof{y_{l+1}\!\in\!\Feps(x_{l+1})\!\mid\!\{z_{m+1},\dots,z_l\}}\!\sim\!\mathrm{Beta}(n+1\!-\!h,h).
\end{equation}
\end{theorem}
A few remarks are in order about Theorem~\ref{thm:icp-validity}.
First, asking $z_{m+1},\dots,z_l,z_{l+1}$ to be exchangeable is weaker than asking them to be independent. However, this assumption typically fails when the calibration set is a single video sequence, where the image frames $\{z_{m+1},\dots,z_l\}$ are temporally correlated~\cite{luo21arxiv-conformalsafety}. Fortunately, as we detail in Section~\ref{sec:experiments}, the way the LineMOD Occlusion dataset~\cite{brachmann14eccv-linemodocc} was collected makes the exchangeability condition easily satisfied, which also suggests best practices to make the exchangeability condition hold in computer vision. 
Second, the lower bound in~\eqref{eq:icpvalidity} can be intuitively proved because under exchangeability, $\alpha_{l+1} := r(y_{l+1},f(x_{l+1}))$ --the nonconformity score of the new sample with the true label-- is \emph{exchangeable} with the nonconformity scores of the calibration samples, and hence \emph{equally likely} to fall in anywhere between the scores $\{ \alpha_{\pi(i)}\}_{i=1}^n$. Consequently, $\probof{y_{l+1} \in \Feps(x_{l+1})} = \probof{\alpha_{l+1} \leq \alpha_{\pi(\floor{(n+1)\epsilon})}} = 1 - \floor{(n+1)\epsilon}/(n+1) \geq 1 - \epsilon$. The upper bound in \eqref{eq:icpvalidity} states that $1-\epsilon$ is not overly conservative (indeed tight if $n$ is large). 
Lastly, the probabilistic guarantee in \eqref{eq:icpvalidity} is \emph{marginal} over the randomness of the calibration set, meaning if one chooses an infinite number of calibration sets,  the \emph{average} empirical coverage will converge to $1-\epsilon$. This, however, implies that the empirical coverage given one calibration set is a random variable that fluctuates as the Beta distribution~\eqref{eq:beta}. Fig.~\ref{fig:beta-distribution} plots the Beta distribution at $\epsilon=0.1$ with different sizes of the calibration set. We observe that as $n$ increases the empirical coverage becomes more concentrated at $1-\epsilon$. Our experiments show that even with a small ($n=200$) calibration set, the empirical coverage is close to, and mostly higher than, $1-\epsilon$.

% \begin{proposition}[Conditional Validity of ICP {\cite{vovk12acml-icpconditional}}] \label{prop:icp-conditional-validity}
% \red{To be filled out}
% \end{proposition}


% Proposition~\ref{prop:icp-validity} states that, if the new observation $z_{l+1}$ is exchangeable with the calibration set (which is a weaker condition than requiring $z_{l+1}$ is jointly i.i.d. with the calibration set), then no matter which prediction function $f$ has been learned from the proper training set, and which function $A$ has been chosen to compute the nonconformity score, we have at least $1-\epsilon$ confidence that the ICP $\Feps$ defined in \eqref{eq:icp} contains the true label. Of course, the caveat here is that the quality of the learned prediction function $f$ and the nonconformity function $A$ will decide the conservativeness of the ICP $\Feps$. For example, if $f$ has poor predictive power, then the set $\Feps$ may be arbitrarily large so that it tells little information about the true label $y$. \red{Fortunately, as we will show in experiments, with modern deep learning architectures for learning $f$, we can obtain ICPs that are both confident and tight.}

%!TEX root = ../main.tex
% \begin{figure}
% \hspace{-4mm}\includegraphics[width=1.1\columnwidth]{icp-overview-half.pdf}
% \caption{Given a learned prediction function and a calibration set of $n$ samples, conformal calibration uses a nonconformity function~\eqref{eq:nonconformity} to compute and sort nonconformity scores $\{ \alpha_{\pi(i)}\}_{i=1}^n$. Given a new observation and an error rate $\epsilon$, conformal prediction~\eqref{eq:icpcompute} outputs a prediction set of all labels under which the nonconformity score of the new sample is no larger than $\alpha_{\pi(\floor{(n+1)\epsilon})}$.
% \label{fig:icp-overview}}
% \end{figure}
\begin{figure}
\vspace{-4mm}
\begin{center}
\includegraphics[width=0.6\columnwidth]{beta.pdf}
\end{center}
\vspace{-6mm}
\caption{Beta distribution of the conditional coverage in~\eqref{eq:beta} with $\epsilon=0.1$ and different $n$. Notice how the conditional probability becomes more concentrated around $1-\epsilon$ when $n$ increases.
\label{fig:beta-distribution}}
\vspace{-7mm}
\end{figure}
\section{PPAD-hardness}
\label{sec:ppad}

%\Binghui{We need to switch to the new definition, but it should be easy given they are almost equivalent under polynomial approximation.}
%{\color{red}We need to write down the definition and show that they equivalent.}

Recall our main hardness result

\Hardness*


In the rest of section, we construct the hard instances of FPA in Section \ref{sec:construction} and provide some basic properties in Section \ref{sec:basic}. 
We reduce from the $\eps$-generalized-circuit problem in Section \ref{sec:reduce}.



\subsection{Construction of first price auctions}
\label{sec:construction}

It suffices to prove finding $\eps$-BNE is hard for some $\eps = 1/\poly(n)$ due to Lemma \ref{lem:notion}.
We will use the following three parameters in the construction:
\begin{align*}
\eps = \frac{1}{n^{40}}, \quad \delta = \frac{1}{n^{10}} \quad \text{and} \quad \beta = \frac{1}{n^4}.
\end{align*}
%where $\eps$ is the precision of BNE and one can boost it to $\frac{1}{n}$ via a padding argument. 



We describe the bidding space $\mB$, the valuation distribution $\D$ and the tie-breaking rule $\Gamma$.

\paragraph{Bidding space.} The bidding space $\mB = \{b_0,b_1, b_2\}$ contains $3$ bids in total, where $b_0 = 0$, 
$$b_1 = \frac{\delta^2}{n^4}\quad\text{and}\quad b_2 = \frac{\delta}{n^2}.$$
\paragraph{Valuation distribution.} 
There are $ n+1$ players --- $n$ standard players indexed by $[n]$ and one pivot player $n+1$. We will describe  the value distribution $\D_i$ of player $i$ by specifying its density function $p_i: [0,1]\rightarrow \R^{+}$. 
The density function $p_{n+1}$ of the pivot player is set as follows:
\begin{align*}
    p_{n+1}(v) = 
    \begin{cases}
    1/(2\eps) & v \in [0,\eps]\\
    1/(2\eps) & v \in [1-\eps, 1]
    \end{cases}
   .
\end{align*}
In another word, $\D_{n+1}$ has $0.5$ probability mass around $0$ and $0.5$ probability mass around $1$. 

The density function $p_i$ of each standard player $i \in [n]$ is set as follows:
\begin{align*}
    p_i(v) =  
    \begin{cases}
    (1 - (2 + \beta)\delta)/\eps & v \in [0,\eps]\\
    \delta /\eps & v \in [b_2 - \eps, b_2]\\
    \tilde{p}_i(v) & v \in (b_2, 1 - \eps)\\
    \delta/\eps & v \in [1- \eps, 1] 
    \end{cases}
\end{align*} 
where $\tilde{p_i}(v)$ is defined over $(b_2,1-\eps)$, satisfies $\int_{b_2}^{1-\eps}\tilde{p}_i(v) \mathsf{d} v = \beta \delta$, but will be specified later in the reduction in Section \ref{sec:reduce}.
In short, a standard player $i$ has most its probability mass around $0$, $\delta$ mass around $b_2$, $\delta$ mass around $1$ and $\beta\delta$ mass in $(b_{2}, 1-\eps)$
  to be specified later. 






\paragraph{Tie-breaking rule.}
We describe the trilateral tie-breaking rule $\Gamma$ as follows. For any bidding profile $\beta$ with $2\le |W(\beta)|\le 3$, the tie-breaking rule depends on the presence of $n+1$ in $W(\beta)$:
\begin{itemize}
\item Suppose $n+1 \notin W(\beta)$. Then
\begin{flushleft}\begin{itemize}
\item If $|W(\beta)| = 2$, i.e., $W(b) = \{i, j\}$, the tie-breaking rule is given by a matrix $\Sigma^{A} \in [0,1]^{n\times n}$ such that player $i$ obtains $\Sigma^{A}_{i, j}$ unit of the item and player $j$ obtains 
$\Sigma^{A}_{j,i}$ unit. 
So the matrix $\Sigma^A$ needs to satisfy $\Sigma^A + (\Sigma^A)^{\top} = (J_n - I_n)$. 
We will specify $\Sigma^A$ in the reduction later but will  guarantee that all of its off-diagonal entries lie in $[1/4,3/4]$.
\item If $|W(\beta)| = 3$, then we use the uniform allocation.
\end{itemize}\end{flushleft}
\item Suppose  $n+1 \in W(\beta)$. Then
\begin{flushleft}\begin{itemize}
\item If $|W(\beta)| = 2$, then the item is fully allocated to the pivot player $n+1$.
\item If $|W(\beta)| = 3$, i.e., $W(b) = \{i, j, n+1\}$, then the tie breaking is given by a matrix $\Sigma_B \in [0,1]^{n\times n}$  such that player $i$ obtains $\Sigma^{B}_{ i, j}$ unit of the item, player $j$ obtains $\Sigma^B_{j,i}$ unit and player $n+1$ obtains $1-\Sigma^B_{i,j}-\Sigma^B_{j,i}$ unit. So the matrix $\Sigma^B$ needs to satisfy $\Sigma^B + (\Sigma^B)^{\top} \leq J_n - I_n$, i.e., $\Sigma^B + (\Sigma^B)^{\top}$ is entrywise dominated by $J_n - I_n$.
\end{itemize}\end{flushleft}
\end{itemize}




\subsection{Basic properties}\label{sec:basic}

Let $s=(s_1,\ldots,s_{n+1})$ be an $\eps$-BNE of the instance.
We prove a few properties of $s$ in this subsection.
Given  $s$, for each player $i$ we define $f_i: \mB\rightarrow [0,1]$ and $F_i: \mB\rightarrow [0,1]$ as follows:
\[
f_i(b ) = \Pr_{v_i\sim \D_i}[s_i(v_i) = b ] \quad \text{and} \quad F_i(b ) = \Pr_{v_i\sim \D_i}[s_i(v_i) \leq b ].
\]
In the rest part of section, we abbreviate 
\[
\Gamma_i(b, s_{-i}) := \E_{v_{-i}\sim \D_{-i}}[\Gamma_i(b, s_{-i}(v_{-i}))] \quad \text{and}\quad u_i(v_i; b, s_{-i}) := \E_{v_{-i}\sim \D_{-i}}[u_i(v_i; b; s_{-i}(v_{-i}))] 
\]
when there is no confusion.

We start with the following lemma.
\begin{lemma}[Separable bid]
\label{lem:separable}
In any $\eps$-BNE, the equilibrium strategy $s$ satisfies
\begin{itemize}
\item For a standard player $i \in [n]$, its equilibrium strategy satisfies 
\begin{itemize}
\item when $v_i \in [0, \eps]$, $s_i(v_i) = b_0$; 
\item when $v_i \in [b_2 - \eps, b_2]$, $s_i(v_i) = b_1$; and 
\item when $v_i \in [1-\eps, 1]$, $s_i(v_i)= b_2$. 
\end{itemize}
\item For the pivot player, its equilibrium strategy satisfies 
\begin{itemize}
\item when $v_{n+1}\in [0, \eps]$, $s_{n+1}(v_{n+1}) = b_0$; and 
\item when $v_{n+1}\in [1-\eps, 1]$, $s_{n+1}(v_{n+1}) = b_2$.
\end{itemize}
\end{itemize}
\end{lemma}
\begin{proof}
The claim of $s_i(\eps) = 0$ holds trivially for all $i \in [n+1]$ due to the no-overbidding assumption.
A standard player $i$ chooses between $b_0$ and $b_1$ for $v_i \in [b_2 - \eps, b_2]$.
The allocation probability $\Gamma_i(b_0, s_{-i})$ of bidding $b_0$ satisfies $\Gamma_i(b_0, s_{-i}) \leq \frac{1}{n+1}$, hence the utility of bidding $b_0 = 0$ is at most 
\[
u_i(v_i; b_0, s_{-i}) = (v_i - b_0) \cdot \Gamma_i(b_0, s_{-i}) \leq \frac{1}{n} b_2.
\]
The allocation probability of bidding $b_1$ is at least 
\[
\Gamma_i(b_1, s_{-i}) \geq \prod_{i\in [n+1]}f_i(b_0) \geq (1 - (2+\beta)\delta)^{n} \cdot \frac{1}{2} \geq \frac{1}{3}
\] 
hence the utility of bidding $b_1$ is at least  
\[
u_i(v_i; b_1, s_{-i}) = (v_i - b_1) \cdot \Gamma_i(b_1, s_{-i}) \geq  (b_2 -\eps - b_1) \cdot \frac{1}{3} > u_i(v; b_0, s_{-i})  + \eps.
\]

Finally, we analyse the equilibrium strategy around $v \in [1-\eps, 1]$ for all $n+1$ players. Via an analysis similar to the above argument, it is clear that both standard players and the pivot player would choose between $b_1$ and $b_2$.
For a standard player $i \in [n]$,  
%given that $s_{i}(1 - \eps) \geq s_i(b_2) = b_{1}$
%  due to the monotonicity assumption, we have that player $i$ would choose between $b_{1}$ and $b_{2}$ on $v\in [1-\eps,1]$. 
the allocation probability of bidding $b_{2}$ satisfies
\begin{align}
    \Gamma_i(b_{2}, s_{-i}) \geq &~ F_{n+1}(b_1) \cdot \prod_{j \in [n]\setminus [i]}F_{j}(b_1) \notag\\
    \geq &~ F_{n+1}(b_1) \cdot \left(\prod_{j \in [n]\setminus [i]}f_{j}(b_0) + \sum_{j \in [n]\setminus [i]}f_j(b_1) \prod_{r\in [n]\setminus \{i, j\}}f_r(b_0) \right)\notag \\
    \geq &~ \frac{1}{2}  \left(\prod_{j \in [n]\setminus [i]}f_j(b_0) + \sum_{j \in [n]\setminus [i]}f_j(b_1) \prod_{r\in [n]\setminus \{i, j\}}f_r(b_0) \right) + \frac{1}{2}f_{n+1}(b_1) ,\label{eq:separable-4}
\end{align}
where the last step holds as $f_{n+1}(b_0) = \frac{1}{2}$ and
\begin{align}
\prod_{j \in [n]\setminus [i]}f_j(b_0)  \geq (1 -(2+\beta)\delta)^{n-1} \geq \frac{1}{2} . \label{eq:separable-5}
\end{align}
The allocation probability of bidding $b_1$ satisfies
\begin{align}
    \Gamma_i(b_{1}, s_{-i}) \leq &~  f_{n+1}(b_0) \cdot \left(\prod_{j \in [n]\setminus [i]}f_j(b_0) + \sum_{j \in [n]\setminus [i]}f_j(b_1) \cdot \Sigma^A_{i, j} \prod_{r\in [n]\setminus \{i, j\}}f_r(b_0) + 9n^2\delta^2\right) \notag\\
    +&~  f_{n+1}(b_1) \cdot \sum_{j \in [n]\setminus\{i\}}f_j(b_1) \notag \\
     \le &~ \frac{1}{2} \left(\prod_{j \in [n]\setminus [i]}f_j(b_0) + \sum_{j \in [n]\setminus [i]}f_j(b_1) \cdot \Sigma^{A}_{i, j} \prod_{r\in [n]\setminus \{i, j\}}f_r(b_0)\right) + f_{n+1}(b_1)\cdot  3n\delta + 9n^2\delta^2\label{eq:separable-3}.
\end{align}
Here the first step holds since (1) when the pivot player bids $b_0$, player $i$ obtains $\Sigma^{A}_{i, j}$ unit of item when (only) player $j$ bids $b_1$, and the probability of at least two players bidding $b_1$ is bounded as
\begin{align*}
\sum_{j, r \in [n]\backslash \{i\}} \Pr[s_r(v_r) = b_1 \wedge s_j(v_j) = b_1] \leq n^2\cdot  9\delta^2,
\end{align*}
(2) when the pivot player bids $b_1$, the player $i$ obtains the item only if there exists at least one other standard player $j$ bids $b_1$ as the tie breaking rule assigns the item fully to player $n+1$ when there are only two winners. The second step follows from $f_j(b_1)\leq 3\delta$
and that the pivot player does not bid $b_0$ in $[1-\eps,1]$ so $f_{n+1}(b_0)=1/2$.

Subtracting Eq.~\eqref{eq:separable-3} and Eq.~\eqref{eq:separable-4}, one obtains
\begin{align}
\Gamma_i(b_{2}, s{-i})  - \Gamma_i(b_{1}, s_{-i})\notag \\
\geq &~   \frac{1}{2}\sum_{j \in [n]\setminus [i]}f_j(b_1) \cdot (1 - \Sigma^{A}_{i, j}) \prod_{r\in [n]\setminus \{i, j\}}f_r(b_0) +\frac{1}{2}f_{n+1}(b_1) -3n\delta f_{n+1}(b_1)-9n^2\delta^2 \notag \\
\geq &~ \frac{1}{2} \cdot (n-1) \cdot \delta \cdot \frac{1}{4} \cdot \frac{1}{2} + \frac{1}{2}f_{n+1}(b_1) -3n\delta f_{n+1}(b_1)-9n^2\delta^2\geq \frac{n\delta}{32}\label{eq:diff}.
\end{align} 
The second step holds due to $f_j(b_1) \geq \delta$, $\Sigma^{A}_{i, j} \in [1/4, 3/4]$ and Eq~\eqref{eq:separable-5}.
Hence we claim player $i$ prefers $b_2$ than $b_1$ at value $v_{i} \in [1-\eps, 1]$, since
\begin{align*}
u_{i}(v_{i}; b_2, s_{-i}) - u_{i}(v_i; b_1, s_{-i}) =&~  (v_{i} - b_2) \cdot \Gamma_{i}(b_2, s_{-i}) - (v_{i} - b_1) \cdot \Gamma_{i}(b_1, s_{-i}) \\
\geq &~ v_{i} \cdot (\Gamma_{i}(b_2, s_{-i}) - \Gamma_{i}(b_1, s_{-i})) - b_2\\
\geq &~ (1-\eps) \cdot \frac{n\delta}{32} - b_2 > \eps.
\end{align*}

Finally, for the pivot player $n+1$, the allocation probability of bidding $b_1$ satisfies
\begin{align}
\Gamma_{n+1}(b_1, s_{-i}) \leq &~ \prod_{i \in [n]}F_i(b_1)
\end{align}
and the allocation probability of $b_2$ satisfies
\begin{align}
\Gamma_{n+1}(b_2, s_{-i}) \geq &~ \prod_{i \in [n]}F_i(b_1) + \sum_{i \in [n]}f_i(b_2) \prod_{j \in [n]\backslash \{i\}}F_i(b_1)\notag \\
\geq &~\prod_{i \in [n]}F_i(b_1) + \frac{n\delta}{2}.\label{eq:separable2}
\end{align}
The first step holds since the tie-breaking rule favors player $n+1$ when at most one player in $[n]$ bids $b_2$, the second step holds due to $f_i(b_2) \geq \delta$ and Eq.~\eqref{eq:separable-5}.
Hence, at any $v_{n+1} \in [1-\eps, 1]$ we have
\begin{align*}
u_{n+1}(v_{n+1}; b_2, s_{-i}) - u_{n+1}(v; b_1, s_{-i})  = &~ (v_{n+1} - b_2) \cdot \Gamma_{n+1}(b_2, s_{-i}) - (v_{n+1} - b_1) \cdot \Gamma_{n+1}(b_1, s_{-i}) \\
\geq &~ v_{n+1} \cdot (\Gamma_{n+1}(b_2, s_{-i}) - \Gamma_{n+1}(b_1, s_{-i})) - b_2\\
\geq &~ (1-\eps) \cdot \frac{n\delta}{2} - b_2 > \eps.
\end{align*}
We conclude the proof of the lemma here.
\end{proof}


Lemma \ref{lem:separable} confirms that $b_0, b_1, b_2$ would appear in an $\eps$-BNE profile for every player $i\in [n]$. It still remains to determine at which value point a standard player $i \in [n]$ jumps from $b_1$ to $b_2$ in $s_i$.
Let $\tau_{i} \in (b_2, 1-\eps)$ be the jumping point from $b_{1}$ to $b_{2}$ of a standard player $i$. The following formula is convenient to use.


\begin{lemma}[Jumping point formula]
\label{lem:formula}
The jumping point $\tau_i$ of a standard player $i \in [n]$ satisfies
\begin{align*}
    \tau_{i} = b_{2} + \frac{\Gamma_i(b_{1}, s_{-i})\cdot (b_{2} - b_1)}{\Gamma_i(b_{2}, s_{-i}) - \Gamma_i(b_{1}, s_{-i})} \pm O\left(\frac{\eps}{\delta}\right).
\end{align*}
\end{lemma}
\begin{proof}
At any value point $v \in [0, 1]$, recall the utility of bidding $b_1$ equals
\begin{align*}
    u_{i}(v_i; b_1,s_{-i}) = (v_i - b_1)\Gamma_i(b_1, s_{-i})
\end{align*}
and the utility of bidding $b_{2}$ equals 
\[
u_{i}(v_i; b_{2},s_{-i}) = (v_i - b_{2})\Gamma_i(b_{2}, s_{-i}).
\]
Solving for $u_{i}(\tau_i, b_1,s_{-i}) = u_{i}(\tau_i, b_{2},s_{-i}) \pm \eps$, one obtains
\begin{align*}
    \tau_{i} = &~  \frac{\Gamma_i(b_{2}, s_{-i}) b_{2} - \Gamma_i(b_{1}, s_{-i})b_1 \pm \eps }{\Gamma_i(b_{2}, s_{-i}) - \Gamma_i(b_{1}, s_{-i})} \\
    = &~ b_{2} + \frac{\Gamma_i(b_{1}, s_{-i})\cdot (b_{2} - b_1) \pm \eps}{\Gamma_i(b_{2}, s_{-i}) - \Gamma_i(b_{1}, s_{-i})}\\ 
    = &~ b_{2} + \frac{\Gamma_i(b_{1}, s_{-i})\cdot (b_{2} - b_1)}{\Gamma_i(b_{2}, s_{-i}) - \Gamma_i(b_{1}, s_{-i})} \pm O\left(\frac{\eps}{\delta}\right) \label{eq:diff}.
\end{align*}
The last step follows from Eq.~\eqref{eq:diff}, and this finishes the proof of the lemma.%\footnote{\color{red}Add $n$ if the red $n$ in (7) is correct.}
\end{proof}

Let $x_i \in [0, \beta\delta]$ be the probability mass over the interval $(b_2, \tau_i)$, i.e., $x_i = \int_{b_2}^{\tau_i}p_i(v)\mathsf{d}v$, which we will refer to the jumping probability of $s_i$. We state a few facts that will be used repeatedly.
\begin{lemma}[Basic facts]
\label{lem:basic-fact}\ 
\begin{itemize}
\item For any standard player $i \in [n]$, we have
%\footnote{\color{red}Is the probability just $f_i(b_e)\cdot f_j(b_e)$?}
\[
\prod_{j \in [n]\setminus \{i\}}f_j(b_0) = 1 -(n-1)(2+\beta)\delta \pm O(n^2\delta^2)
\]
and
\[
\prod_{j \in [n]\setminus \{i\}}F_j(b_1) = 1 -(n-1)(\beta+1)\delta + \sum_{j \in [n]\setminus \{i\}}x_j \pm O(n^2\delta^2).
\]
\item For any $e \in \{1,2\}$, we have 
$$\sum_{i\neq j \in [n]} \Pr\big[s_i(v_i) = b_e \wedge s_j(v_j) = b_e\big] = O(n^2\delta^2).$$
\end{itemize}
\end{lemma}
\begin{proof}
For the first claim, we have
\begin{align*}
\prod_{j \in [n]\setminus \{i\}}f_j(b_0) = (1 - (2+\beta)\delta)^{n -1} = 1 - (n-1)(2+\beta)\delta  \pm O(n^2\delta^2)
\end{align*}
due to the choice of $\delta$. Similarly we have (using $x_i\in [0,\beta\delta]$
\begin{align*}
\prod_{j \in [n]\setminus \{i\}}F_j(b_1)  =  &~ \prod_{j \in [n]\setminus \{i\}}(1 - (1+\beta)\delta + x_i) \\
= &~ 1 - (n-1)(1+\beta)\delta + \sum_{j \in [n]\backslash \{i\}}x_j \pm O(n^2\delta^2).
\end{align*}
For the second claim, for any $e\in \{1,2\}$, we have
\begin{align*}
\sum_{i \neq j \in [n]} \Pr\big[s_i(v_i) = b_e \wedge s_j(v_j) = b_e\big] \leq \sum_{i\neq j \in [n]}(1+\beta)\delta^2= O(n^2\delta^2).
\end{align*}
We conclude the proof here.
\end{proof}

The key step is to determine the jumping point, where we use approximation.
\begin{lemma}[Jumping point]
\label{lem:jump}
The jumping point $\tau_i$ of a standard player $i\in [n]$ satisfies
\begin{align*}
\tau_i = \left(\frac{\Delta_{i,1} - \Delta_{i,2}}{\Delta_{i,1}^2} \pm O\left(\frac{\beta^2}{\Delta_{i,1}}\right) \right)\cdot b_2 
\end{align*}
where
\begin{align*}
\Delta_{i, 1} := &~ (n-1)\delta + \sum_{j\in [n]\setminus \{i\}}\Big(\beta\delta\cdot \Sigma^{A}_{i,j} + (1+\beta)\delta\cdot \Sigma^{B}_{i,j}\Big) \in \big[(n-1)\delta, 2n\delta\big]\quad\text{and}\\
\Delta_{i, 2} := &~ \sum_{j \in [n]\backslash \{i\}} \Big(1 - 2\Sigma^{A}_{ i, j} - \Sigma^{B}_{i, j}\Big)x_{j}  \in \big[-2n\beta\delta, n\beta\delta\big].
\end{align*}
\end{lemma}
\begin{proof}
For any standard player $i \in [n]$, we compute when it jumps from $b_{1}$ to $b_2$ using the formula in Lemma \ref{lem:formula}.
To do so, we first compute $\Gamma_i(b_{1}, s_{-i})$ and $\Gamma_i(b_{2}, s_{-i})$.
\begin{align}
\Gamma_i(b_{1}, s_{-i}) =&~ f_{n+1}(b_0) \cdot \left( \prod_{j \in [n]\setminus \{i\}}f_j(b_0) + \sum_{j\in [n]\setminus\{i\}}f_j(b_1)\prod_{j \in [n]\setminus \{i, j\}}f_j(b_0)\cdot \Sigma^{A}_{ i, j} \pm O(n^2\delta^2)  \right)\notag\\
= &~ \frac{1}{2}  \left( 1 - (n-1)(2+\beta)\delta + \sum_{j\in [n]\setminus\{i\}} (\delta + x_j)\Sigma^{A}_{i,j}  \right)\pm O(n^2\delta^2). \label{eq:prob1}
\end{align}
Here the first step follows from the tie-breaking rule and the second claim of Lemma \ref{lem:basic-fact}, the second step follows from $f_j(b_1) = \delta + x_j$ and the first claim of Lemma \ref{lem:basic-fact}.

The allocation probability of bidding $b_2$ obeys
\begin{align*}
    \Gamma_i(b_{2}, s_{-i}) = &~ f_{n+1}(b_0) \cdot \left(\prod_{j \in [n]\setminus \{i\}}F_j(b_1) +  \sum_{j\in [n]\setminus\{i\}}f_j(b_2) \prod_{j \in [n]\setminus \{i, j\}}F_j(b_1) \cdot \Sigma^{A}_{i, j} \pm O(n^2\delta^2) \right)\\
    &~ + f_{n+1}(b_2) \cdot \left(\sum_{j\in [n]\setminus\{i\}}f_j(b_2)\prod_{k \in [n]\setminus \{i, j\}}F_k(b_1) \cdot \Sigma^{B}_{i, j} \pm O(n^2\delta^2) \right)\\
    = &~ \frac{1}{2} \left(1 - (n-1)(1+\beta)\delta + \sum_{j \in [n]\setminus \{i\}}x_j + \sum_{j \in [n]\setminus \{i\}}(\delta + \beta\delta - x_j)\Sigma^{A}_{i,j}  \right)\\
    &~ + \frac{1}{2}\sum_{j \in [n]\setminus \{i\}}(\delta + \beta\delta - x_j)\Sigma^{B}_{i, j} \pm O(n^2 \delta^2).
\end{align*}
The first step uses the tie breaking rule and requires some explanations. In particular, (1) when the pivot player $n+1$ bids $b_0$, the player $i$ obtains $1$ unit of item when other players bid less than $b_2$, $\Sigma_{A, i, j}$ unit of item when only player $j$ bids $b_2$; we also make use of the second claim of Lemma \ref{lem:basic-fact} to omit the other case; (2) when the player $n+1$ bids $b_2$, the player $i$ obtains $0$ unit of good when no other players bid $b_0$ and obtains $\Sigma_{B,i, j}$ unit of goods when one other player $j$ bids $b_2$, and we omit other cases using Lemma \ref{lem:basic-fact}.
The second step follows from Lemma \ref{lem:basic-fact} and $f_j(b_2) = \delta + \beta\delta - x_j$.

Combining the above expression, we have
\begin{align}
\Gamma_i(b_{2}, s_{-i}) - \Gamma_i(b_{1}, s_{-i}) = &~ \frac{1}{2}(n-1)\delta + \frac{1}{2}\sum_{j\in [n]\setminus \{i\}}\Big(\beta\delta\cdot \Sigma^{A}_{i,j} + (1+\beta)\delta\cdot \Sigma^{B}_{i,j}\Big) \notag \\
&~ + \frac{1}{2}\sum_{j \in [n]\backslash \{i\}} \Big(1 - 2\Sigma^{A}_{i, j} - \Sigma^{B}_{i, j}\Big)x_{j} \pm O(n^2\delta^2)\label{eq:diff2}.
%(J - I - 2\Sigma_{A} - \Sigma_{B})_i^{\top} x 
\end{align}
Let $\Delta_{i,1}$ and $\Delta_{i,2}$ be defined as in the statement of the lemma.
%We abbreviate the above expression as 
%\begin{align*}
%\Delta_{i, 1} := &~ (n-1)\delta + \sum_{j\in [n]\setminus \{i\}}\Big(\beta\delta\cdot\Sigma^{A}_{i,j} + (1+\beta)\delta\cdot \Sigma^{B}_{i,j}\Big)\\
%\Delta_{i, 2} := &~ \sum_{j \in [n]\backslash \{i\}} \Big(1 - 2\Sigma^{A}_{i, j} - \Sigma^{B}_{i, j}\Big)x_{j} 
%\end{align*}
Note that $\Delta_{i, 1}$ does not depend on $\{x_j\}_{j\ne i}$ while $\Delta_{i, 2}$ depends on $\{x_j\}_{j \neq i}$.
It is easy to see that 
\begin{align}
\Delta_{i,1} \in \big[(n-1)\delta, 2n\delta\big] \quad \text{and} \quad  \Delta_{i, 2} = \big[-2(n-1)\beta\delta, (n-1)\beta\delta\big]\label{eq:range}.
\end{align}

Finally we can compute the jumping point $\tau_i$ using Lemma \ref{lem:formula} as follows:
\begin{align*}
\tau_i = &~  b_2 + \frac{\Gamma_i(b_{1}, s_{-i})\cdot (b_{2} - b_1)}{\Gamma_i(b_{2}, s_{-i}) - \Gamma_i(b_{1}, s_{-i})} \pm O\left(\frac{\eps}{\delta}\right) \\
= &~ b_2 + \frac{1 \pm O(n\delta)}{\Delta_{i,1}+\Delta_{i,2}} \cdot b_2 \pm O\left(\frac{\eps}{\delta}\right)\\
= &~ \left(\frac{\Delta_{i,1} - \Delta_{i,2}}{\Delta_{i,1}^2} \pm O\left(\frac{\beta^2}{\Delta_{i,1}}\right) \right)\cdot b_2 
\end{align*}
The second step follows from Eq.~\eqref{eq:diff2}, $\Gamma_i(b_1, \beta_i) = (1/2) \pm O(n\delta)$ (see Eq.~\eqref{eq:prob1}) and the choice of $b_1,b_2$.
The last step follows from Eq.~\eqref{eq:range}.
%\footnote{\color{red}Let's double check the last step.}
\end{proof}








\subsection{Reduction from generalized circuit}
\label{sec:reduce}

Given $\alpha<\beta$,
  we write $\mathsf{T}_{[\alpha, \beta]}: \R \rightarrow [\alpha, \beta]$ to denote the  truncation function with  $$\mathsf{T}_{[\alpha, \beta]}(x) = \min\big\{\max\{x, \alpha\}, \beta\big\}.$$ We recall the  generalized circuit problem  \cite{chen2009settling} and  present a simplified version from \cite{filos2021complexity}. 
\begin{definition}[(Simplified) generalized circuit]
	\label{def:generelized-circuit}
	A generalized circuit is a tuple $(V, G)$, where $V$ is a set of nodes and $G$ is a collection of gates.
    Each node $v \in V$ is associated with a gate $G_v$ that falls into one of two types $\{G_{1-}, G_{+}\}$: If $G_v$ is a $G_+$ gate, then it has two input nodes $v_1,v_2\in V\setminus \{v\}$; if it is a $G_{1-}$ gate then it takes one input node $v_1\in V\setminus \{v\}$.
Given $\kappa>0$,
%The generalized circuit problem asks to find 
a $\kappa$-approximation solution to $(V,G)$ is an assignment $x\in [0, 1]^{V}$such that for every node $v$:
     \begin{itemize}
    \item If $G_v$ is a $G_{+}$ gate and  takes input nodes $v_1, v_2 \in V\backslash \{v\}$, then $x_v = \mathsf{T}_{[0, 1]}(x_{v_1} + x_{v_2}\pm \kappa) $
    \item If $G_v$ is a  $G_{1-}$ gate and takes  an input node $v_1 \in V\backslash \{v\}$, then   $x_{v} = \mathsf{T}_{[0,1]}(1 - x_{v_1}\pm \kappa) $.
    \end{itemize}   

\end{definition}

The generalized circuit problem is known to be PPAD-hard for constant $\kappa$.
\begin{theorem}[\cite{rubinstein2015inapproximability,deligkas2022pure}]
There is a constant $\kappa>0$ such that it is PPAD-hard to find an $\kappa$-approximate solution of a generalized circuit.
\end{theorem}



We prove Theorem \ref{thm:hardness} via a reduction from the generalized circuit problem.

%\begin{proof}[Proof of Theorem \ref{thm:hardness}]
Given an instance of generalized circuit defined over nodes set $V$ ($|V| = m$), we let $V_1 = [m_1]$ be the set of nodes with gate $G_{+}$  and $V_2 = [m_1+1: m]$  be the set of nodes with gate $G_{1-}$.
We construct an instance of first price auction with $n = m_1 + 2(m - m_1) = 2m - m_1$ standard players and one pivot player. 



Let $\mN = \mN_1 \cup \mN_{2} \cup \mN_{3}$ be the set of standard players, where $ \mN_1 = [m_1]$, $ \mN_{2} = [m_1+1: m]$ and $\mN_{3} = [m+1: 2m - m_1]$.
From a high level, we use players in $\mN_1$ to represent the set of nodes with $G_{+}$ gates, players in $\mN_{2}$ to represent the set of nodes with $G_{1-}$ gates. 
Players in $\mN_{3}$ are used in constructing $G_{1-}$. 
We first specify the probability density $\tilde{p}$ over interval $(b_2, 1-\eps)$ and the tie-breaking matrices $\Sigma^A$ and $\Sigma^B$ to complete the description of the FPA instance.



\begin{itemize}
\item For player $i$ in $\mN_1$ (i.e., $i \in [m_1]$), its valuation distribution $\tilde{p}_i$ is uniform over the interval
\[
\left[\frac{1}{\Delta_{i, 1}} \cdot b_2, \left(\frac{1}{\Delta_{i, 1}} + \frac{1}{10}\cdot\frac{\beta\delta}{\Delta_{i, 1}^2}\right)\cdot b_2\right]
\]
with a total probability mass of $\beta\delta$. Let $i(1), i(2) \in [m]=\mN_1\cup \mN_2$ be the input nodes of the $G_{+}$ gates, we set $\Sigma^{B}_{i, i(1)} = \Sigma^{B}_{i, i(2)}= 1/10$.
\item For player $m_1 + j \in \mN_{2}$ (i.e., $j \in [m - m_1]$), its valuation distribution $\tilde{p}_{m_1+j}$ is uniform over 
\[
\left[\left(\frac{1}{\Delta_{m_1 + j, 1}} - \frac{1}{10}\cdot\frac{\beta\delta}{\Delta_{m_1 +j, 1}^2}\right)\cdot b_2, \frac{1}{\Delta_{m_1+j, 1}}\cdot b_2\right]
\] 
with a total probability mass of $\beta\delta$.
We set $\Sigma^{A}_{m_1 + j, m+ j} = 9/20.$
%$\frac{1}{2} - \frac{1}{20}$.
\item For player $m + j \in \mN_{3}$ (i.e., $j \in [m - m_1]$), its valuation distribution $\tilde{p}_{m+j}$ is uniform over  
\[
\left[\left(\frac{1}{\Delta_{m+j, 1}} + \frac{1}{10}\cdot\frac{\beta\delta}{\Delta_{m+j, 1}^2}\right)\cdot b_2, \left(\frac{1}{\Delta_{m+j, 1}} + \frac{1}{5}\cdot\frac{\beta\delta}{\Delta_{m+j, 1}^2}\right)\cdot b_2\right]
\]
with a total probability mass of $\beta\delta$. We set $\Sigma^{A}_{m + j, m_1+ j} = 11/20$. Let $j(1) \in [m]$ be the input node of $G_{1-}$, then set $\Sigma^{B}_{m+j, j(1)} =  {1}/{5}$.
\item For any entry of $\Sigma^{A}$ that has not been determined above, we set it to be $1/2$, and for any entry of $\Sigma^{B}$ that has not been determined, we set it to be $0$.
\end{itemize}


It is easy to verify that $\Sigma^{A}$ and $\Sigma^B$ satisfy the following properties as promised earlier: (1) the off-diagonal entries of $\Sigma^{A}$ lie in $[1/4, 3/4]$;  (2) $\Sigma^{A} + (\Sigma^{A})^{\top} = J_n - I_n$; and (3) the off-diagonal entries of $\Sigma^{B}$ belong to $[0,1/2]$.

Letting $\kappa: = n\beta = {1}/{n^3}$, we prove that any $\eps$-BNE of the first price auction gives an $O(\kappa)$-approximate solution to the generalized circuit.
Indeed the following lemma shows that by taking $x_{i}' =  {x_{i}}/{(\beta\delta)}$, where $x_i$ is the jumping probability of $s_i$, we obtain an $O(\kappa)$-approximate solution $(x_1',\ldots,x_m')$ to the input generalized circuit. This finishes the proof of Theorem \ref{thm:hardness}.
\begin{lemma}
Given an $\eps$-BNE of the first price auction and let $(x_1, \ldots, x_n)\in [0, \beta\delta]^{n}$ be the tuple of jumping probabilities, then we have
\begin{itemize}
\item For any $i \in [m_1]$, $x_{i} = \mathsf{T}_{[0, \beta\delta]}(x_{i(1)} + x_{i(2)} \pm O(\kappa\beta\delta))$
\item For any $j \in [m- m_1]$, $x_{m_1 + j} = \mathsf{T}_{[0, \beta\delta]}(\beta\delta - x_{j(1)} \pm O(\kappa\beta\delta))$
\end{itemize}
\end{lemma}
\begin{proof}
For the first claim, for any $i \in [m_1]$, one has
\begin{align*}
\Delta_{i, 2} = \sum_{r \in [n]\backslash \{i\}} \Big(1 - 2\Sigma^{A}_{i, r} - \Sigma^{B}_{ i, r}\Big)x_{r} = -\frac{1}{10}x_{i(1)}  - \frac{1}{10}x_{i(2)},
\end{align*}
where the second step follows from $\Sigma^{A}_{i, r} = 1/2$ for all $r\in [n]\backslash \{i\}$, $\Sigma^{B}_{i, i(1)} = \Sigma^{B}_{i, i(2)} = 1/10$ and $\Sigma^{B}_{i, r} = 0$ for all other $r \in [n]\backslash \{i, i_1, i_2\}$.
By Lemma \ref{lem:jump}, one has
\begin{align*}
\tau_i = \left(\frac{\Delta_{i,1} - \Delta_{i,2}}{\Delta_{i,1}^2} \pm O\left(\frac{\beta^2}{\Delta_{i,1}}\right)\right) \cdot b_2 = \left(\frac{1}{\Delta_{i,1}} + \frac{1}{10}\cdot \frac{x_{i(1)} + x_{i(2)}}{\Delta_{i,1}^2}  \pm O\left(\frac{\beta^2}{\Delta_{i,1}}\right)\right) \cdot b_2.
\end{align*}
Since $\D_{i}$ is uniform over $[\frac{1}{\Delta_{i, 1}} \cdot b_2, (\frac{1}{\Delta_{i, 1}} + \frac{1}{10}\cdot\frac{\beta\delta}{\Delta_{i, 1}^2})\cdot b_2]$ with probability mass $\beta\delta$, we have 
\[
x_i = \mathsf{T}_{[0, \beta\delta]}(x_{i(1)} + x_{i(2)} \pm O(\kappa\beta\delta)).
\]


For the second claim, we first analyse the jumping probability of player $m_1 + j$. We have
\begin{align*}
\Delta_{m_1 + j, 2} = \sum_{r \in [n]\backslash \{m_1+j\}} \Big(1 - 2\Sigma^{A}_{m_1+j, r} - \Sigma^{B}_{m_1 +j, r}\Big)x_{r} = \frac{1}{10}x_{m+ j}  ,
\end{align*}
where the second follows from $\Sigma^{A}_{m_1 + j, m+j} = 9/{20}$, $\Sigma^{A}_{m_1+j, r} = 1/2$ for all $r\in [n]\backslash \{m_1 + j,m+j \}$, and $\Sigma^{B}_{m_1+j, r} = 0$ for all $r \in [n]\backslash \{m_1 +j \}$. Hence, by Lemma \ref{lem:jump}, we have
\begin{align*}
\tau_{m_1 + j} = &~ \left(\frac{\Delta_{m_1 +j,1} - \Delta_{m_1 + j,2}}{\Delta_{m_1 + j,1}^2} \pm O\left(\frac{\beta^2}{\Delta_{m_1 + j, 1}}\right)\right) \cdot b_2\\
=&~ \left(\frac{1}{\Delta_{m_1 +j,1}} - \frac{1}{10}\cdot \frac{x_{m+ j}}{\Delta_{m_1+j,1}^2}  \pm O\left(\frac{\beta^2}{\Delta_{m_1+j,1}}\right)\right) \cdot b_2
\end{align*}
Given that $\D_{m_1+j}$ is uniform over $[(\frac{1}{\Delta_{m_1 + j, 1}} - \frac{1}{10}\cdot\frac{\beta\delta}{\Delta_{m_1 +j, 1}^2})\cdot b_2, \frac{1}{\Delta_{m_1+j, 1}}\cdot b_2]
$ with mass $\beta\delta$, we have 
\begin{align}
\label{eq:fix1}
x_{m_1+j} = \mathsf{T}_{[0, \beta\delta]}(\beta\delta - x_{m+ j} \pm O(\kappa\beta\delta)) .
\end{align}
It remains to analyse the jumping probability of player $m+j$, and we have 
\begin{align*}
\Delta_{m + j, 2} = &~ \sum_{r \in [n]\backslash \{m+j\}} \Big(1 - 2\Sigma^{A}_{m+j, r} - \Sigma^{B}_{m+j, r}\Big)x_{r}\\
= &~ -\frac{1}{10}x_{m_1+ j}   - \frac{1}{5}x_{j(1)} = -\frac{1}{10}\left(\beta\delta - x_{m + j} + 2x_{j(1)}\right) \pm O(\kappa\beta\delta).
\end{align*}
Here the second step follows from $\Sigma^{A}_{m + j, m_1+ j} = 11/20$, $\Sigma^{A}_{m + j, r} = 1/2$ for $ r\in [n]\backslash \{m_1+j, m+j\}$, $\Sigma^{B}_{m+j, j(1)} = 1/5$ and $\Sigma^{B}_{m + j, r}= 0$ for any $r\in [n]\backslash \{m+j, j(1)\}$. The last step follows from Eq.~\eqref{eq:fix1}. 

Now, by Lemma \ref{lem:jump}, one has
\begin{align*}
\tau_{m + j} = &~ \left(\frac{\Delta_{m+j,1} - \Delta_{m+j,2}}{\Delta_{m+j,1}^2} \pm O\left(\frac{\beta^2}{\Delta_{m+j,1}}\right)\right) \cdot b_2 \\
= &~ \left(\frac{1}{\Delta_{m+j,1}} + \frac{1}{10}\cdot \frac{\beta\delta - x_{m+ j} + 2x_{j(1)} \pm O(\kappa\beta\delta)}{\Delta_{m+j,1}^2}  \pm O\left(\frac{\beta^2}{\Delta_{m+j,1}}\right)\right) \cdot b_2
\end{align*}
Given that $\D_{m+j}$ is uniform over $[(\frac{1}{\Delta_{m+j, 1}} + \frac{1}{10}\cdot\frac{\beta\delta}{\Delta_{m+j, 1}^2})\cdot b_2,(\frac{1}{\Delta_{m+j, 1}} + \frac{1}{5}\cdot\frac{\beta\delta}{\Delta_{m+j, 1}^2})\cdot b_2]$ with mass $\beta\delta$, we conclude that
%\footnote{\color{red}Correct but may need a bit more explanation.}
\begin{align*}
x_{m+j} = \mathsf{T}_{[0, \beta\delta]}(x_{j(1)} \pm O(\kappa \beta\delta)).
\end{align*}
Plugging into Eq.~\eqref{eq:fix1}, we obtain
\begin{align*}
x_{m_1+j} = \mathsf{T}_{[0, \beta\delta]}(\beta\delta - x_{j(1)} \pm O(\kappa \beta\delta)).
\end{align*}
This completes the proof of the second claim.
\end{proof}


 



 







% \normalem
\begin{algorithm}[t!]
\small{
\SetAlgoLined
\SetKwInOut{Input}{Input}
\SetKwInOut{Output}{Output}
\Indmm \Indmm
\Input{Backbone network $f$, projection head $g$, two prediction heads $h_1$, $h_2$, sample generator $q_\phi$ and $p_\theta$, and the training set $\mathcal{D}$}
\Output{Trained backbone network $f$}
\Indpp \Indpp
\ForEach{batch $\mathcal{B}^s \subset \mathcal{D}$}
{
    $L_{\text{fair-CL}}, L_{\text{self-KD}} \gets 0, 0 $ \\
    $\mathcal{B}^{s}_{\text{pert}} \gets \text{TabMix}(\mathcal{B}^s, \text{Sample}(\mathcal{D} \setminus \mathcal{B}^s))$  \tcp{(Eq.~\ref{eq:tabmix})}
    $\mathcal{B}^{s'}_{\text{cnt}} \gets \text{Convert}(\mathcal{B}^s, q_\phi, p_\theta)$ \tcp{(Sec. 3.2)}
    \ForEach{$\mathbf{x}^s \in \mathcal{B}_s$, $\mathbf{x}^s_{\text{pert}} \in \mathcal{B}^{s}_{\text{pert}}$, and $\mathbf{x}^{s'}_{\text{cnt}} \in \mathcal{B}^{s'}_{\text{cnt}}$}
    {   
        $\mathcal{X}^s_{-} = \{ \mathbf{x} | \mathbf{x} \in \mathcal{B}_s\setminus\{\mathbf{x}^s\} \}$ \\
         \tcp{Fairness-aware contrastive loss (Eq.~\ref{eq:contrast_loss})}
         $L_{\text{fair-CL}} \gets L_{\text{fair-CL}} + L_{\text{gen-c}} (\mathbf{x}^s, \mathbf{x}^{s'}_{\text{cnt}}, \mathcal{X}^s_{-})$ \\ \vspace{2mm}
         \tcp{Self-knowledge distillation loss (Eq.~\ref{eq:self-kd})}
         $\mathbf{p}_{\text{student}} \gets h_2 \circ g \circ f(\mathbf{x}^s_{\text{pert}})$ \\
         $\mathbf{z}_{\text{teacher}} \gets (g \circ f(\mathbf{x}^s)).\text{detach()}$ \\
         $L_{\text{self-KD}} \gets L_{\text{self-KD}} - ({\mathbf{p}_{\text{student}} \over || \mathbf{p}_{\text{student}} ||_2} \cdot {\mathbf{z}_{\text{teacher}}  \over || \mathbf{z}_{\text{teacher}} ||_2})$  \\
    }
    $L_{\text{total}} = {1 \over |\mathcal{B}_s|}(L_{\text{fair-CL}} +  L_{\text{self-KD}})$  \\
    Update weights via back-propagation
}
\caption{Overall training procedure (for one epoch)}}
\label{algo:algorithm}
\end{algorithm}
% \ULforem
%We provide some comments on the growth conditions which constituted the majority of our analysis in sections \ref{sec:Hmixing} and \ref{sec:Hsigma}. In the simplest cases of Lemma \ref{lemma:unstableGrowth}, growth was established in an analogous fashion to the old one-step expansion condition (\ref{eq:oldOneStepExpansion}), finding the relevant Jacobians $M_j$ and checking that their expansion factors $K(M_j)$ satisfy
\begin{equation}
    \label{eq:discussionOneStep}
    \sum_j \frac{1}{K(M_j)} <1.
\end{equation}
For the more complicated cases, the inductive method used to establish growth near the accumulation points in Lemma \ref{lemma:unstableGrowth} and the weakened one-step expansion condition (\ref{eq:oneStep}) both address the same fundamental issue: the splitting of unstable curves by singularities into an unbounded number of small components. They circumvent this obstacle in rather different ways, however. While (\ref{eq:oneStep}) generalises (\ref{eq:discussionOneStep}) to ensure an growth of unstable curves `on average' (see \cite{chernov_statistical_2009} for a precise statement), our inductive method is a more direct adaptation of (\ref{eq:discussionOneStep}), using it to generate contradictory geometric conditions which a hypothetical non-growing unstable curve must satisfy. It may be possible to prove Theorem \ref{sec:Hmixing} using (\ref{eq:oneStep}) as the basis for growth. Since we required (\ref{eq:oneStep}) anyway for proving Theorem \ref{thm:HsigmaExp}, this could potentially condense our analysis, but only to a minor extent. A convenience of the method used in section \ref{sec:Hmixing} is that, by way of the `simple intersection' property, it naturally gives geometric information on the images of manifolds, useful for proving the property \textbf{(M)} of Theorem \ref{thm:katok-strelcyn}.

We expect that essentially analogous analysis can be applied to establish mixing properties in a wide class of piecewise linear non-uniformly hyperbolic maps, including those (like the OTM) which sit on the boundary of ergodicity and beyond. While we have relied on the precise partition structure of $H_\sigma$, its fundamental feature (self-similar sequences of elements $A^k$, sharing boundaries with its neighbours $A^{k-1},A^{k+1}$ and accumulating onto some point $p$) is quite typical to return map systems. See, for example, those of various stadium billiards \cite{chernov_chaotic_2006,chernov_improved_2008,chernov_statistical_2009} and LTMs \cite{springham_polynomial_2014}. Indeed, the same method can be used to prove the Bernoulli property for non-monotonic LTMs \cite{myers_hill_mixing_2022}, where monotonicity of the manifold images cannot be assumed and the classical argument \cite{sturman_mathematical_2006} fails. The OTM is the pointwise limit of these maps as the boundary shrinks to null measure. It further has utility in proving growth conditions for maps which are uniformly hyperbolic but possess regions $A_j$ where the hyperbolicity is very weak, signified by $K(M_j) \approx 1$, so that (\ref{eq:discussionOneStep}) fails. Typically this leads to suboptimal bounds on mixing windows, see e.g. \cite{wojtkowski_model_1981,przytycki_ergodicity_1983,myers_hill_family_2022}. The map $H_{(\eta,\eta)}$ for $\eta \approx 1/2$ is another example, possessing weak hyperbolicity over $A_2, A_3$. Letting $\varepsilon = |\eta-1/2|>0$, there is an upper bound $N = N(\varepsilon)$ on escape times from the intersections $A_2\cap \sigma, A_3 \cap \sigma$. The growth lemma then follows by applying the inductive step roughly $N$ times and can be established for arbitrarily small $\varepsilon$, opening the door to establishing optimal mixing windows.

The above gives two examples of piecewise linear perturbations to $H$ where mixing with respect to Lebesgue is preserved and our methods can be applied. Nonlinear perturbations to the shear profiles complicate the analysis in several ways. Firstly as the map's Jacobians takes on a broader range of values, cone invariance becomes an increasingly harder condition to establish. Cones must be widened, giving looser bounds on expansion factors, which may already be weak due to new regions of weaker stretching. This, together with the change from polygonal to curvilinear return time partition elements and nonlinear local manifolds, adds some complexity to showing growth conditions. This does not rule out certain (small) nonlinear perturbations however. There is some leeway in the inequalities which govern cone invariance and growth of local manifolds, the latter of which is not too dissimilar from the piecewise linear setting (see Lemmas \ref{lemma:piecewiseApprox}, \ref{lemma:componentLength}). Certain small perturbations would not alter the \emph{topological} structure of the return time partition, i.e. which elements share boundaries, the key information needed for setting up the induction. Finally while the partition elements would no longer be polygonal, only coarse geometric information is required for verifying each inductive step. Following the above, a potential perturbation could be to replace the linear portions of each shear by a cubic, perturbing the tent profile
\[  f(t) = \begin{cases} 2t & 0 \leq t \leq 1/2, \\ 2(1-t) & 1/2 \leq t \leq 1 ,\end{cases} \]
of the OTM shears to
\[  f_a(t) = \begin{cases} \frac{1}{8} t \left(16 - a + 6at - 8at^{2} \right) & 0 \leq t \leq 1/2, \\ \frac{1}{8}\left(1-t\right)\left( 16 - a + 6a\left(1-t\right) - 8a\left(1-t\right)^{2}\right)  & 1/2 \leq t \leq 1, \end{cases}   \]
for $a>0$. For small enough $a$ the gradient range $f'(t)$ is restricted to small neighbourhoods of $\{ 2, -2\}$ and the escape time partition retains a similar structure. We illustrate this in Figure \ref{fig:perturbations}, showing escapes from the square $S_3$ under the map $G \circ F$, equivalent to escapes from the perturbed $A_3$ under the $G \circ F$, but with a cleaner geometry for comparison. When $a$ is too large the analogy to the OTM breaks down. At $a=16$ the map is twice differentiable everywhere and features a new source of slowed mixing, the Jacobian is the identity at the corner points $x,y \in \{  0, 1/2 \}$ giving locally parabolic behaviour (visible in the escape time partition). 

\begin{figure}
    \centering
    \includegraphics[width=0.24 \linewidth]{0.png}
    \includegraphics[width=0.24 \linewidth]{4.png}
    \includegraphics[width=0.24 \linewidth]{8.png}
    \includegraphics[width=0.24 \linewidth]{16.png}
    \caption{Partition of escape times from $S_3$ under the mapping $F \circ G$ for $a= 0,4,8,16$. }
    \label{fig:perturbations}
\end{figure}


\section*{Acknowledgement}
X.C. and B.P. would like to thank Aviad Rubinstein and anonymous STOC reviewers for helpful suggestions on the paper. The research of X.C. and B.P. is supported by NSF grants CCF-1703925, IIS-1838154, CCF-2106429 and CCF-2107187, CCF-1763970, CCF-2212233, COLL2134095, COLL2212745.


\newpage
\bibliographystyle{alpha}
\bibliography{ref}

\newpage
\appendix
\section{Appendix for Proofs}

\paragraph{Proof of Theorem \ref{thm:main}.}

\begin{proof}
\label{proof:main}
Our proof has two steps. In Step 1, we will show that SimCLR is equivalent to minimizing the cross entropy loss defined in Eqn.~(\ref{eqn:cross-entropy}). 
In Step 2, we will show  that minimizing the cross-entropy loss 
is equivalent to spectral clustering on $\bfpi$. 
Combining the two steps together, we have proved our theorem. 

\textbf{Step 1: } SimCLR is equivalent to minimizing the cross entropy loss.

The cross-entropy loss takes expectation over 
$\bfW_\bfX\sim \mathbb{P}(\cdot ; \bfpi)$, 
which means $\bfW_\bfX$ has exactly one non-zero entry in each row $i$. By Lemma~\ref{lem:multinomial}, we know every row $i$ of $\bfW_\bfX$ is independent of other rows. Moreover, 
$\bfW_{\bfX,i}\sim \mathcal{M}(1, \bfpi_i/\sum_j \bfpi_{i,j})=\mathcal{M}(1, \bfpi_i)$, because $\bfpi_i$ itself is a probability distribution.
Similarly, we know $\bfW_\bfZ$ also has the row-independent property by sampling over $\mathbb{P}(\cdot;\bfK_\bfZ)$.
Therefore, by Lemma~\ref{lem:cross_split}, we know Eqn.~(\ref{eqn:cross-entropy}) is equivalent to:
\[
 -\sum_{i=1}^n \mathbb{E}_{\bfW_{\bfX,i}}[\log \mathbb{P}(\bfW_{\bfZ,i}=\bfW_{\bfX,i};\bfK_\bfZ)],
\]

This expression takes expectation over $\bfW_{\bfX,i}$ for the given row $i$. Notice that 
$\bfW_{\bfX,i}$ has exactly one non-zero entry, which equals $1$ (same for $\bfW_{\bfZ,i}$). 
As a result
we expand the above expression to be:
\begin{equation}
 -\sum_{i=1}^n \sum_{j\neq i} \Pr(\bfW_{\bfX,i,j}=1)\log \Pr(\bfW_{\bfZ,i,j}=1).
\label{eqn:detailed-expansion}    
\end{equation}


By Lemma~\ref{lem:multinomial}, $\Pr(\bfW_{\bfZ,i,j}=1)=\bfK_{\bfZ,i,j}/\|\bfK_{\bfZ,i}\|_1$ for $j\neq i$. Recall that $\bfK_\bfZ=(k(\bfZ_i-\bfZ_j))_{(i,j)\in[n]^2}$, which means 
$\bfK_{\bfZ,i,j}/\|\bfK_{\bfZ,i}\|_1=\frac{\exp(-\|\bfZ_i-\bfZ_j\|^2/{2\tau})}{\sum_{k\neq i}
\exp(-\|\bfZ_i-\bfZ_k\|^2/{2\tau})
}$ for $j\neq i$, when $k$ is the Gaussian kernel with variance $\tau$. 

Notice that $\bfZ_i=f(\bfX_i)$, so we know
\begin{equation}
-\log \Pr(\bfW_{\bfZ,i,j}=1)=
-\log \frac{\exp(-\|f(\bfX_i)-f(\bfX_j)\|^2/{2\tau})}{\sum_{k\neq i}
\exp(-\|f(\bfX_i)-f(\bfX_k)\|^2/{2\tau}),
}
\label{eqn:infonce-equivalence}    
\end{equation}


The right hand side is exactly the InfoNCE loss defined in Eqn.~(\ref{eqn:infonce}).
Inserting Eqn.~(\ref{eqn:infonce-equivalence}) into Eqn.~(\ref{eqn:detailed-expansion}), we get the SimCLR algorithm, which first samples augmentation pairs $(i,j)$ with $\Pr(\bfW_{\bfX,i,j}=1)$ for each row $i$, and then optimize the InfoNCE loss. 

\textbf{Step 2: } minimizing the cross entropy loss 
is equivalent to spectral clustering on $\bfpi$.


By Lemma~\ref{lem:convert_to_spectral}, we may further convert the loss to 
\begin{equation}
\label{eqn:main-theorem-repul-attr}
\min_{\bfZ}
-\sum_{(i,j)\in [n]^2} \mathbf{P}_{i,j}
\log k (\bfZ_i-\bfZ_j)+\log \mathbf{R}(\bfZ).
\end{equation}
Since $k$ is the Gaussian kernel, this reduces to \[
\min_\bfZ \mathrm{tr}(\bfZ^\top \mathbf{L}(\bfpi) \bfZ)
+\log \mathbf{R}(\bfZ),
\]

where we use the fact that $\mathbb{E}_{\bfW_\bfX\sim \mathbb{P}(\cdot; \bfpi)}[\mathbf{L}(\bfW_\bfX)]
=\mathbf{L}(\bfpi)
$, because the Laplacian operator is linear and $
\mathbb{E}_{\bfW_\bfX\sim \mathbb{P}(\cdot; \bfpi)}(\bfW_\bfX)=\bfpi
$.
\end{proof}

\paragraph{Proof of Theorem \ref{thm:clip}.}
\begin{proof}
Since $\bfW_\bfX\sim \mathbb{P}(\cdot;\bfpi_{\mathbf{A}, \mathbf{B}})$, we know 
$\bfW_\bfX$ has exactly one non-zero entry in each row, denoting the pair that got sampled. 
A notable difference compared to the previous proof is we now have $n_\mathcal{A}+n_\mathcal{B}$ objects in our graph. CLIP deals with this by taking a mini-batch of size $2N$, 
such that $n_\mathcal{A}=n_\mathcal{B}=N$, and adding the $2N$ InfoNCE losses together. We label the objects in $\mathcal{A}$ as $[n_\mathcal{A}]$, and the objects in $\mathcal{B}$ as $\{n_\mathcal{A}+1, \cdots, n_\mathcal{A}+n_\mathcal{B}\}$. 

Notice that $\bfpi_{\mathbf{A}, \mathbf{B}}$ is a bipartite graph, so the edges of objects in $\mathcal{A}$ will only connect to object in $\mathcal{B}$ and vice versa. We can define the similarity matrix in $\cZ$ as $\bfK_\bfZ$, 
where $\bfK_\bfZ(i, j+n_\mathcal{A})=\bfK_\bfZ(j+n_\mathcal{A},i)= k(\bfZ_i-\bfZ_j)$ for $i\in [n_\mathcal{A}], j\in [n_\mathcal{B}]$, and otherwise we set $\bfK_\bfZ(i,j)=0$. 
The rest is same as the previous proof. 
\end{proof}

\paragraph{Proof of Theorem \ref{thm:exponential}.}

\begin{proof}
\label{proof:exponential}
Since the objective function consists of a linear term combined with an entropy regularization, which is a strongly concave function, the maximization problem is a convex optimization problem. Owing to the implicit constraints provided by the entropy function, the problem is equivalent to having only the equality constraint. We then introduce the Lagrangian multiplier $\lambda$ and obtain the following relaxed problem:

$$
\widetilde{E}(\boldsymbol{\alpha})=\psi_{1}-\sum_{i=1}^n \alpha_{i} \psi_{i}+\tau \sum_{i=1}^n \alpha_{i}\log \alpha_{i}+\lambda\left(\boldsymbol{\alpha}^{\top} \mathbf{1}_n-1\right).
$$

As the relaxed problem is unconstrained, taking the derivative with respect to $\alpha_{i}$ yields

$$
\frac{\partial \widetilde{E}(\boldsymbol{\alpha})}{\partial \alpha_{i}}=-\psi_{i}+\tau\left(\log \alpha_{i}+\alpha_{i} \frac{1}{\alpha_{i}}\right)+\lambda=0.
$$

Solving the above equation implies that $\alpha_{i}$ takes the form
$
\alpha_{i}=\exp \left(\frac{1}{\tau} \psi_{i}\right) \exp \left(\frac{-\lambda}{\tau}-1\right).
$ Since $\alpha_{i}$ lies on the probability simplex, the optimal $\alpha_{i}$ is explicitly given by
$
\alpha^{*}_{i}=\frac{\exp \left(\frac{1}{\tau} \psi_{i}\right)}{\sum_{i^{\prime}=1}^n \exp \left(\frac{1}{\tau} \psi_{i^{\prime}}\right)} .
$ Substituting the optimal point into the objective function, we obtain
$$
\begin{aligned}
E\left(\boldsymbol{\alpha}^*\right)  &=\psi_1-\sum_{i=1}^n \frac{\exp \left(\frac{1}{\tau} \psi_{i}\right)}{\sum_{i^{\prime}=1}^n \exp \left(\frac{1}{\tau} \psi_{i^{\prime}}\right)} \psi_{i}+\tau \sum_{i=1}^n \frac{\exp \left(\frac{1}{\tau} \psi_{i}\right)}{\sum_{i^{\prime}=1}^n \exp \left(\frac{1}{\tau} \psi_{i^{\prime}}\right)}\log \frac{\exp \left(\frac{1}{\tau} \psi_{i}\right)}{\sum_{i^{\prime}=1}^n \exp \left(\frac{1}{\tau} \psi_{i^{\prime}}\right)} \\
& =\psi_1 - \tau \log \left(\sum_{i=1}^n \exp \left(\frac{1}{\tau} \psi_{i}\right)\right).
\end{aligned}
$$
Thus, the Lagrangian dual function is given by
\begin{equation*}
-E\left(\boldsymbol{\alpha}^*\right)= -\tau \log \frac{\exp \left(\frac{1}{\tau} \psi_{1}\right)}{\sum_{i=1}^n \exp \left(\frac{1}{\tau} \psi_{i}\right)}.\qedhere
\end{equation*}
\end{proof}



\section{More on Experiments} \label{section: experiment_details}

\paragraph{CIFAR-10 and CIFAR-100} CIFAR-10 ~\citep{krizhevsky2009learning} and CIFAR-100 ~\citep{krizhevsky2009learning} are well-known classic image classification datasets. Both CIFAR-10 and CIFAR-100 contain a total of 60k $32 \times 32$ labeled images of different classes, with 50k for training and 10k for testing. CIFAR-10 is similar to CIFAR-100, except there are 10 different classes in CIFAR-10 and 100 classes in CIFAR-100.

\paragraph{TinyImageNet} TinyImageNet ~\citep{le2015tiny} is a subset of ImageNet ~\citep{deng2009imagenet}. There are 200 different object classes in TinyImageNet, with 500 training images, 50 validation images, and 50 test images for each class. All the images in TinyImageNet are colored and labeled with a size of $64 \times 64$.

\textbf{Pseudo-code.} Algorithm \ref{alg:Training Procedure} presents the pseudo-code for our empirical training procedure.

\begin{algorithm}[!htbp]
\caption{Training Procedure}
\label{alg:Training Procedure}
\begin{algorithmic}[1]
\REQUIRE trainable encoder network $f$, batch size $N$, augmentation strategy \textit{aug}, loss function $L$ with hyperparameters \textit{args}
\FOR {sampled minibatch ${x_i}_{i=1}^N$}
\FORALL{$i \in { 1, ..., N }$}
\STATE draw two augmentations $t_i = \textit{aug}\left(x_i\right) $, $t_i' = \textit{aug}\left(x_i\right) $
\STATE $z_i = f\left(t_i\right)$, $z_i' = f\left(t_i'\right)$
\ENDFOR
\STATE compute loss $\mathcal{L} = L(N, z, z', \textit{args})$
\STATE update encoder network $f$ to minimize $\mathcal{L}$
\ENDFOR
\STATE \textbf{Return} encoder network $f$
\end{algorithmic}
\end{algorithm}

We also provide the pseudo-code for our core loss function used in the training procedure in Algorithm \ref{alg:Core loss}. The pseudo-code is almost identical to SimCLR's loss function, with the exception of an extra parameter $\gamma$.

\begin{algorithm}[!htbp]
\caption{Core loss function $\mathcal{C}$}
\label{alg:Core loss}
\begin{algorithmic}[1]
\REQUIRE batch size $N$, two encoded minibatches $z_1, z_2$, $\gamma$, temperature $\tau$
\STATE $z = \textit{concat}\left(z_1, z_2\right)$
\FOR {$i \in {1, ..., 2N }, j \in {1, ..., 2N}$ }
\STATE $s_{i,j} = \Vert z_i - z_j \Vert_2^{\gamma}$
\ENDFOR
\STATE \textbf{define} $l(i, j)$ \textbf{as} $l(i, j) = - \log \frac{exp\left(s_{i,j}/\tau \right)}{\sum_{k=1}^{2N} \mathbf{1}{[k \ne i]} exp\left(s{i, j} / \tau \right)} $
\STATE \textbf{Return} $\frac{1}{2N} \sum_{k=1}^N\left[l(i, i+N) + l(i+N, i)\right]$
\end{algorithmic}
\end{algorithm}

Utilizing the core loss function $\mathcal{C}$, we can define all kernel loss functions used in our experiments in Table \ref{table: loss definition}. For all $z_i \in z$ with even dimensions $n$, we define $z_{L_i} = z_i\left[0:n/2\right]$ and $z_{R_i} = z_i\left[n/2:n\right]$.

\begin{table}[ht]
\centering
\begin{tabular}{{@{}l|l@{}}}
Kernel  &  Loss function \\ \midrule
Laplacian & $\mathcal{C}\left(N, z, z', \gamma=1, \tau\right)$\\ \midrule
Sum       & $\lambda * \mathcal{C}\left(N, z, z', \gamma=1, \tau_1\right) + (1-\lambda) * \mathcal{C}\left(N, z, z', \gamma=2, \tau_2\right)$  \\ \midrule
Concatenation Sum&$\lambda * \mathcal{C}\left(N, z_L, z'_L, \gamma=1, \tau_1\right) + (1-\lambda) * \mathcal{C}\left(N, z_R, z'_R, \gamma=2, \tau_2\right)$\\ \midrule
$\gamma = 0.5$ & $\mathcal{C}\left(N, z, z', \gamma=0.5, \tau\right)$          \\ 

\end{tabular}

\caption{Definition of kernel loss functions in our experiments}
\label {table: loss definition}
\end{table}

\textbf{Baselines.} We reproduce the SimCLR algorithm using PyTorch Lightning~\citep{PytorchLightning}.

\textbf{Encoder details.}
The encoder $f$ consists of a backbone network and a projection network. We employ ResNet50~\citep{ResNet} as the backbone and a 2-layer MLP (connected by a batch normalization~\citep{ioffe2015batch} layer and a ReLU \cite{nair2010rectified} layer) with hidden dimensions 2048 and output dimensions 128 (or 256 in the concatenation kernel case).

\textbf{Encoder hyperparameter tuning.}
For each encoder training case, we randomly sample 500 hyperparameter groups (sample details are shown in Table \ref{table: Hyperparameter sample}) and train these samples simultaneously using Ray Tune ~\citep{RayTune}, with the ASHA scheduler~\citep{li2018massively}. Ultimately, the hyperparameter group that maximizes the online validation accuracy (integrated in PyTorch Lightning) within 5000 validation steps is chosen for the given encoder training case.

\begin{table}[ht]
\centering

\begin{tabular}{@{}l|l|l@{}}
\midrule
Hyperparameter  & Sample Range & Sample Strategy \\ \midrule
start learning rate & $\left[10^{-2}, 10\right]$ & log uniform \\ \midrule
$\lambda$       & $\left[0, 1\right]$ & uniform \\ \midrule
$\tau$, $\tau_1$, $\tau_2$ & $\left[0, 1\right]$ & log uniform \\ \midrule
\end{tabular}

\caption{Hyperparameters sample strategy}
\label {table: Hyperparameter sample}
\end{table}

\textbf{Encoder training.} 
We train each encoder using the LARS optimizer~\citep{LARSOptimizer}, LambdaLR Scheduler in PyTorch, momentum 0.9, weight decay $10^{-6}$, batch size 256, and the aforementioned hyperparameters for 400 epochs on a single A-100 GPU.

\textbf{Image transformation.} The image transformation strategy, including augmentation, is identical to the default transformation strategy provided by PyTorch Lightning.

\textbf{Linear evaluation.}
The linear head is trained using the SGD optimizer with a cosine learning rate scheduler, batch size 64, and weight decay $10^{-6}$ for 100 epochs. The learning rate starts at $0.3$ and ends at $0$.

\textbf{Moco Experiments.} We also tested our method based on MoCo~\citep{he2019moco}. The results are summarized in Table \ref{tab:results-moco}. Here we choose ResNet18~\citep{ResNet} as the backbone and set a temperature of $0.1$ as default. For our simple sum kernel, we set $\lambda=0.8$. The results show that our method outperforms the original MoCo method.

\begin{table}[thb]
\centering
\caption{MoCo Experiment Results on CIFAR-10 and CIFAR-100.}
\label{tab:results-moco}
\resizebox{\textwidth}{!}{%
\begin{tabular}{@{}c|ccc|ccc@{}}
\toprule
\multirow{3}{*}{Method} & \multicolumn{3}{c|}{CIFAR-10} & \multicolumn{3}{c}{CIFAR-100} \\ \cmidrule(lr){2-4} \cmidrule(lr){5-7} 
                        & 200 epochs & 400 epochs    & 1000 epochs   & 200 epochs & 400 epochs & 1000 epochs         \\ \midrule
MoCo (repro.)         & $76.41 \pm 0.12$    & $80.01 \pm 0.15$          & $84.45 \pm 0.08$    & $\mathbf{47.02 \pm 0.11}$ & $52.50 \pm 0.07$ & $57.62 \pm 0.15$            \\
\midrule
Laplacian Kernel        & ${78.09 \pm 0.10}$    & $\mathbf{83.85 \pm 0.09}$          & $\mathbf{88.34 \pm 0.16}$    & $46.12 \pm 0.22$   & $53.44 \pm 0.17$ & $59.10 \pm 0.14$        \\
Simple Sum Kernel & $\mathbf{78.12 \pm 0.15}$   & $83.23 \pm 0.18$ & $87.50 \pm 0.20$ & $46.65 \pm 0.06$ & $\mathbf{53.62 \pm 0.19}$ & $\mathbf{59.83 \pm 0.12}$\\
\bottomrule
\end{tabular}
}
\end{table}



\section{More Experiments on Synthetic Data}


Consider a scenario with $n$ clusters, each containing $k$ vertices. Let the probability of vertices $u$ and $v$ from the same cluster belonging to $\bfpi$ be $p$. Conversely, for vertices $u$ and $v$ from different clusters, let the probability of belonging to $\pi$ be $q$. We generate the graph $\bfpi$ randomly, based on $p$ and $q$. We experiment with values of $k=100$ and $n=6$ for ease of visualization, embedding all points in a two-dimensional space. Each vertex's initial position originates from a normal distribution. In each iteration, we sample a subgraph of $\bfpi$ uniformly, ensuring each vertex has an out-degree of $1$. We then optimize the corresponding vectors using InfoNCE loss with an SGD optimizer and iterate until convergence. Our experimental setup consists of an SGD learning rate of $1$, an InfoNCE loss temperature of $0.5$, and a batch size of $50$. We evaluate two scenarios with different $p$ and $q$ values: $p=1$, $q=0$, and $p=0.75$, $q=0.2$. The results of these experiments are visualized in Figure \ref{fig:vis-spectral-cluster}. The obtained embeddings exhibit the hallmark pattern of spectral clustering of graph $\bfpi$.

\begin{figure}[!tb]
\centering
\subfigure{
\includegraphics[width=1\textwidth]{Figures/cluster_pi.png}
\label{fig:vis-cluster}
}
\subfigure{
\includegraphics[width=1\textwidth]{Figures/noised_cluster_pi.png}
\label{fig:vis-noised-cluster}
}
\caption{Visualizations of the optimization process using InfoNCE Loss on the vectors corresponding to $\bfpi$. Points of identical color belong to the same cluster within $\bfpi$. To showcase the internal structure of $\bfpi$, we randomly select 10 vertices from each cluster to display the edge distribution of $\bfpi$.}
\label{fig:vis-spectral-cluster}
\end{figure}









\end{document}

