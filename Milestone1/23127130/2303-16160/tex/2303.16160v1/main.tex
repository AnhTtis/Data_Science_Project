\documentclass[10pt,twocolumn,letterpaper]{article}
\usepackage[final]{cvpr}

% Include other packages here, before hyperref.
\usepackage{graphicx}
\usepackage{amsmath}
\usepackage{amssymb}
\usepackage{booktabs}

\usepackage[accsupp]{axessibility}  % Improves PDF readability for those with disabilities.

% If you comment hyperref and then uncomment it, you should delete
% egpaper.aux before re-running latex.  (Or just hit 'q' on the first latex
% run, let it finish, and you should be clear).
\usepackage[pagebackref,breaklinks,colorlinks,bookmarks=false]{hyperref}

\usepackage{multirow}
\usepackage{enumitem}
\usepackage{rotating}
\usepackage{makecell}
\usepackage{marvosym}

\usepackage{ragged2e}
\usepackage{caption}
\usepackage{subcaption}
\usepackage[normalem]{ulem}
%\usepackage[breaklinks=true,colorlinks,bookmarks=false]{hyperref}

% Support for easy cross-referencing
\usepackage[capitalize]{cleveref}
\crefname{section}{Sec.}{Secs.}
\Crefname{section}{Section}{Sections}
\Crefname{table}{Table}{Tables}
\crefname{table}{Tab.}{Tabs.}
\usepackage{color}
\usepackage{xcolor}
\usepackage{color, colortbl}
\usepackage{booktabs}
% \usepackage{subfigure}
\usepackage{float}
\usepackage{wrapfig}
\definecolor{Gray}{gray}{0.9}
\definecolor{Red}{RGB}{230, 57, 70}
\definecolor{Blue}{RGB}{0, 100, 148}
\usepackage{mmstyle}
\usepackage{xspace}
\newcommand{\TODO}[1]{\textcolor{red}{(TODO: #1)}}
\newcommand{\modelname}{\emph{OSX}\xspace}
\newcommand{\dataname}{\emph{UBody}\xspace}

\newcommand\blfootnote[1]{%
  \begingroup
  \renewcommand\thefootnote{}\footnote{#1}%
  \addtocounter{footnote}{-1}%
  \endgroup
}


%%%%%%%%% PAPER ID  - PLEASE UPDATE



\def\cvprPaperID{3184} % *** Enter the CVPR Paper ID here
\def\confName{CVPR}
\def\confYear{2023}
\begin{document}
%%%%%%%%% TITLE - PLEASE UPDATE
\title{One-Stage 3D Whole-Body Mesh Recovery with Component Aware Transformer}  % **** Enter the paper title here

\author{%
	Jing Lin$^{1,2\S}$, Ailing Zeng$^{1\P}$, Haoqian Wang$^{2}$, Lei Zhang$^{1}$, Yu Li$^{1}$ \\
		$^{1}$ International Digital Economy Academy (IDEA), \\ $^2$  Shenzhen International Graduate School, Tsinghua University \\
		\url{https://osx-ubody.github.io}
}
%\maketitle 


\vspace{-0.75cm}
\twocolumn[{%
    \renewcommand\twocolumn[1][]{#1}%
    \maketitle
    \centering
    \captionsetup{type=figure}
    \includegraphics[width=0.9\linewidth]{fig/teaser.pdf}
    % \vspace{-0.2cm}
    \caption{A comparison of existing whole-body mesh recovery methods and ours. Most existing methods leverage a multi-stage pipeline which uses separate expert models to process body component (\eg, \textbf{E1}: HeadNet, \textbf{E2}: HandNet, \textbf{E3}: BodyNet) and fuse them to get the whole-body prediction in a copy-paste manner. The result (from~\cite{PavlakosGeorgios2020expose}) produces unnatural wrist poses. In contrast, our pipeline is a neat one-stage framework with a single encoder-decoder and can predict more accurately with natural meshes.}
    \vspace{0.7cm}
    \label{fig:teaser}
}]
\vspace{0.7cm}


%%%%%%%%% ABSTRACT
\begin{abstract}

%  final
Whole-body mesh recovery aims to estimate the 3D human body, face, and hands parameters from a single image. It is challenging to perform this task with a single network due to resolution issues, i.e., the face and hands are usually located in extremely small regions. Existing works usually detect hands and faces, enlarge their resolution to feed in a specific network to predict the parameter, and finally fuse the results. While this copy-paste pipeline can capture the fine-grained details of the face and hands, the connections between different parts cannot be easily recovered in late fusion, leading to implausible 3D rotation and unnatural pose. In this work, we propose a one-stage pipeline for expressive whole-body mesh recovery, named OSX, without separate networks for each part. Specifically, we design a Component Aware Transformer (CAT) composed of a global body encoder and a local face/hand decoder. The encoder predicts the body parameters and provides a high-quality feature map for the decoder, which performs a feature-level upsample-crop scheme to extract high-resolution part-specific features and adopt keypoint-guided deformable attention to estimate hand and face precisely. The whole pipeline is simple yet effective without any manual post-processing and naturally avoids implausible prediction. Comprehensive experiments demonstrate the effectiveness of OSX. Lastly, we build a large-scale Upper-Body dataset (UBody) with high-quality 2D and 3D whole-body annotations. It contains persons with partially visible bodies in diverse real-life scenarios to bridge the gap between the basic task and downstream applications. 
%\footnote{The code and dataset will be released upon acceptance.\\
\vspace{0pt}
\blfootnote{$\S$ Work done during an internship at IDEA; ${\P}$~Corresponding author.}
\end{abstract}

\section{Introduction}

The increasing complexity of source code poses a key challenge to the reliability of large-scale software systems. Software bugs in these systems can lead to safety issues~\cite{bug_safety} for users around the world as well as cause non-negligible financial losses~\cite{bug_loss}. As such, developers have to spend a large amount of time and effort on bug fixing. Consequently, \aprfull (\apr), designed to automatically generate patches to fix software bugs, has attracted wide attention from both academia and industry~\cite{long2016prophet, legoues2012genprog, long2015spr, lou2020can, tufano2018empstudy}. 


To achieve \apr, one popular approach is known as Generate-and-Validate (G\&V)~\cite{qi2015gv, ghanbari2019prapr, lou2020can, le2016hdrepair, legoues2012genprog, wen2018capgen, hua2018sketchfix, martinez2016astor, koyuncu2020fixminder, liu2019tbar, liu2019avatar}, which is typically based on the following pipeline: First, fault localization techniques~\cite{wong2016fl, abreu2007ochiai, zhang2013injecting, papadakis2015metallaxis, li2019deepfl, li2017transforming} are applied to determine the suspicious locations in programs where bugs are likely to exist. Then, the buggy locations are used by the \apr tools to generate a list of patches that replace buggy lines with correct lines. Afterward, each patch is validated against the original test suite to identify any \emph{plausible patches} (i.e., passing all tests in the test suite). Finally, to determine the \emph{correct patches}, developers examine the list of plausible patches to see if any of them can correctly fix the bug. 

Traditional \apr tools can mainly be categorized into heuristic-based~\cite{legoues2012genprog, le2016hdrepair, wen2018capgen}, constraint-based~\cite{mechtaev2016angelix, le2017s3, demacro2014nopol, long2015spr} and \template~\cite{ghanbari2019prapr, hua2018sketchfix, martinez2016astor, liu2019tbar, liu2019avatar}. Among these traditional tools, \template \apr tools~\cite{ghanbari2019prapr, liu2019tbar, benton2020effectiveness} have been able to achieve state-of-the-art results. \Template \apr tools typically leverage pre-defined templates (e.g., adding a nullness check) for bug fixing. However, since these fix templates are typically handcrafted, the number and types of bugs they are able to fix can be limited. 



To address the limitations of traditional \apr, researchers have proposed various \learning \apr tools~\cite{li2020dlfix, chen2018sequencer, jiang2021cure, lutellier2020coconut, zhu2021recoder, ye2022rewardrepair} based on the \nmtfull (\nmt) architecture~\cite{sutskever2014mt} where the input is the buggy code snippets and the goal is to translate the buggy code snippets into a fixed version. To accomplish this, \learning \apr tools require supervised training datasets with pairs of both buggy and fixed code snippets in order to learn how to perform this translation step. These training data are usually obtained by mining historical bug fixes using heuristics/keywords~\cite{dallmeier2007benchmark}, which can be imprecise for identifying bug-fixing commits; even the actual bug-fixing commits can include irrelevant code changes, leading to further pollution in the dataset~\cite{xia2022alpharepair}.
% 
Moreover, it can be hard for such \apr tools to generalize and fix bug types unseen during training. 



To better leverage recent advances in \plmfull{s} (\plm{s}), researchers~\cite{xia2022alpharepair, xia2023repairstudy, kolak2022patch, prenner2021codexws} have directly applied \plm{s} to generate patches without bug-fixing datasets. These \llm-based \apr tools work by either directly generating a complete code function~\cite{prenner2021codexws, xia2023repairstudy} or predict/infill the correct code snippet given its surrounding context~\cite{xia2022alpharepair, xia2023repairstudy}. By directly using \llm{s} that are pre-trained on billions of open-source code snippets, \llm-based \apr tools can achieve state-of-the-art performance on many repair datasets~\cite{xia2022alpharepair}. 


% 
%
%

Traditional \apr tools have long used the insight of the \emph{plastic surgery hypothesis}~\cite{barr2014plastic} where it states that the code ingredients to fix a bug already exist within the same project. Traditional \apr tools have manually designed pattern-~\cite{ghanbari2019prapr, saha2017elixir} or heuristic-based~\cite{jiang2018simfix, legoues2012genprog} approaches to finding and using such relevant code ingredients to generate fixes for bugs. However, the plastic surgery hypothesis has been largely ignored in \llm-based \apr. In fact, \llm provides a unique opportunity to fully automate the plastic surgery hypothesis idea via fine-tuning (learning project-specific information via model updates from the buggy project) and prompting (directly providing relevant code ingredients to the model), and make it directly applicable to different languages (since the \llm{s} are typically multi-lingual).%
Moreover, despite the intensive manual efforts involved, traditional \apr tools still cannot fully leverage project-specific information due to large search space for leveraging/composing existing code ingredients. In contrast, the project-specific information can effectively leveraged by \llm{s} due to their power in code understanding/vectorization, e.g., even partial/imprecise information may still guide \llm{s} in correct patch generation!
 To this end, we ask the question: \emph{How useful is the plastic surgery hypothesis in the era of \plm{s}}?








\mypara{Our Work.} To answer the question, we present \ourtech{\xspace} -- a \llm-based approach that automatically utilizes the plastic surgery hypothesis by systematically combining multiple fine-tuning and prompting strategies for \apr. \ourtech fine-tunes \plm{s} using two novel domain-specific training strategies: \textbf{\epfinetune} -- we fine-tune using the original buggy project by aggressively masking out a high percentage of tokens, which allows \plm to learn project-specific code tokens and programming styles; and \textbf{\rofinetune} -- which only masks out a single continuous code sequence per training sample, allowing the model to get used to the final \csapr task of predicting a single continuous code sequence. Furthermore, we directly leverage the ability for \plm{s} to understand natural language instructions and introduce a novel prompting strategy, \textbf{\idprompting}, which uses information retrieval and static analysis to obtain a list of relevant identifiers for the buggy lines. While such relevant identifiers are critical for fixing some difficult bugs, they may not be seen by the \llm during inference due to limited context window size. Through the use of prompting, we directly tell the model to use these extracted identifiers (relevant code ingredients) to generate the correct code. Finally, to perform repair, we combine all four model variants (including the base model, both fine-tuned models and the base model with prompting) for the final repair.





While our insight of leveraging the plastic surgery hypothesis for \llm-based \apr is generalizable across different types of \plm{s}, to implement \ourtech, we choose a recent \plm{\xspace}, \ctfive~\cite{wang2021codet5}, which is pre-trained on millions of open-source code snippets. \ctfive is an encoder-decoder model trained using \mspfull (\msp) objective where a percentage of tokens are masked out and each continuous masked token sequence is referred to as a masked span. Also, although we only extract relevant identifiers from the current buggy project (since this paper focuses on the plastic surgery hypothesis), our work can be easily extended to obtain other code information (such as relevant statements or functions) from other sources, such as  the massive pre-training corpora~\cite{husain2020codesearchnet} or historical bug-fixing datasets~\cite{jiang2019infer}, which can provide more coding knowledge for \llm{s}. Besides, although we mainly focus on using traditional string comparison algorithms for information retrieval in this paper, these techniques can be easily replaced by other frequency-based retrieval~\cite{robertson2009probabilistic} and neural search (or embedding-based search)~\cite{reimers2019sentence}.
  In summary, this paper makes the following contributions:


%


\begin{itemize}[noitemsep, leftmargin=*, topsep=0pt]
    \item \textbf{Dimension.} This paper is the first to revisit the important plastic surgery hypothesis in the era of \llm{s}. It opens up a new dimension for \llm-based \apr to incorporate previously neglected information from the buggy project itself to boost \apr performance. Furthermore, it demonstrates the promising future of retrieval-based prompting for modern \llm-based \apr.
    \item \textbf{Implementation.} We implement \ourtech based on the recent \ctfive model. We augment the model using two novel fine-tuning strategies: \epfinetune and \rofinetune, along with a novel prompting strategy based on information retrieval and static analysis: \idprompting. We combine the patches generated by all four models together and perform patch ranking to speed up \apr.% 
    \item \textbf{Evaluation Study.} We conduct an extensive evaluation against state-of-the-art \apr tools. On the widely studied \dfj 1.2 and 2.0 datasets~\cite{just2014dfj}, \ourtech is able to achieve the new state-of-the-art results of 89 and 44 correct bug fixes (15 and 8 more than best baseline) respectively.  Furthermore, we perform a broad ablation study to justify our design. \ourtech demonstrates for the first time that the plastic surgery hypothesis can substantially boost \llm-based \apr and advance state-of-the-art \apr, while being fully automated and general. Moreover, even partial/imprecise code ingredients may still effectively guide \llm{s} for \apr!
\end{itemize}


\section{Related work}
% There is extensive recent work on speeding up analytical queries due to the need for consistent execution times in the face of the explosive growth in the volume of available data.
% In this section, we divide existing work into two categories: maintaining data freshness in MVs (\Cref{sec:server_side}) and optimizations for minimizing ad-hoc query latency (\Cref{sec:client_side}).

% \subsection{Maintaining Data Freshness in MVs}
% \label{sec:server_side}
% There exists a variety of data warehousing applications aimed at supporting low-latency analytical queries on fresh data.
% In particular, these applications require efficiency in the propagation of newly ingested data into downstream MVs.
 
\mypara{Efficient MV Refresh}
Incremental view maintenance (IVM) aims to update MVs to reflect newly ingested data, taking advantage of already computed results to perform the update in a manner more efficient than computing from scratch (full refresh)
~\cite{ahmad2012dbtoaster,mcsherry2013differential,armbrust2013generalized,zeng2016iolap, palpanas2002incremental, griffin1995incremental, agiwal2021napa, braun2015analytics}. 
There is an abundance of work in IVM, including incremental updates on duplicate values~\cite{griffin1995incremental}, non-distributive aggregate functions~\cite{palpanas2002incremental}, higher-order views~\cite{ahmad2012dbtoaster}, and sliding windows~\cite{braun2015analytics}. 
More recent works also investigate the scalability aspect of IVM, proposing scale-independent updates~\cite{armbrust2013generalized} and sampled views~\cite{zeng2016iolap}. Since \system is applicable to arbitrary SQL statements, \system is orthogonal to and is fully compatible with existing IVM techniques.

\mypara{MV Refresh Scheduling}
There exist works on scheduling the refresh of a MV set focusing on resolving cyclic dependencies~\cite{folkert2005optimizing}, minimizing weighted average staleness~\cite{golab2009scheduling}, and prioritizing MVs with the highest speedups on predicted future queries~\cite{ahmed2020automated}.
\system's scheduling to speed up the end-to-end refresh of the MV set is not addressed in existing works.

\mypara{DAG Workflow Scheduling}
The execution of workloads consisting of individual jobs with acyclic dependencies is a well-studied topic~\cite{apacheoozie,sparkdag,marchal2018parallel,bathie2020revisiting,baruah2022ilp}; many of these techniques can be applied to MV refresh runs studied in this paper.
Existing workflow scheduling systems such as Apache Oozie~\cite{apacheoozie}, Apache Airflow~\cite{airflow}, and Spark DAG scheduler~\cite{sparkdag} automate the execution of user-defined workflows following a topological order.
There are recent works aimed at finding more optimal execution orders in terms of peak memory usage~\cite{marchal2018parallel, bathie2020revisiting} and execution time on parallel platforms~\cite{baruah2022ilp}.
While \system is designed for use with MV refresh runs/workloads, our technique on joint scheduling and optimization can be reasonably applied to general workloads as a possible future direction.

% \paragraph{Incremental MV indexing}
% Update-optimized indices such as the log-structured merge-trees (LSM)~\cite{o1996log} are used for indexing MVs due to frequent updates induced by data ingestion~\cite{gupta2016mesa,agiwal2021napa}.
% \system is orthogonal to indexing: \system is capable of efficiently performing MV refresh runs regardless of whether the individual MVs are indexed or not.

% \subsection{Ad-hoc Query Latency Reduction}
% \label{sec:client_side}

% The minimization of ad-hoc analytical query response times is a well-studied topic due to latency being negatively correlated with the productivity of a data analyst during a data analysis session~\cite{liu2014effects}.
% Sessions are commonly conducted within visualization systems that contain a variety of optimization techniques to ensure that query response times fall within a certain latency tolerance.

% \mypara{Data prefetching}
% Data is often loaded into memory on a by-need basis in visualization systems to minimize interference with user-issued query computations~\cite{mani2017effective,xin2021enhancing,galakatos2017revisiting, yan2020auto, battle2016dynamic, crotty2016case, jalaparti2018netco}. 
% Query-time data retrieval can be significantly expedited by anticipating the data usage of the user in future queries and pre-loading the data into memory during the downtime between user queries (`think time'). SMART~\cite{mani2017effective} prefetches data for modified versions of current user-issued queries with different filters and dimensions. A-WARE~\cite{crotty2016case} maintains a sample store constantly refined through ingesting data based on speculations of future plots.
% ForeCache~\cite{battle2016dynamic} uses an SVM to predict the user's current analysis phase and accordingly prefetches data tiles partitioned based on different numerical values. NetCo predicts future queries via log analysis, and solves an ILP formulation to prefetch data to maximize the number of SLO-meeting queries~\cite{jalaparti2018netco}.
% In the case of MV refresh workloads, `think time' is nonexistent as individual MVs are refreshed back-to-back, rendering data prefetching techniques non-applicable.

\mypara{Intermediate Data Caching}
Some existing data visualization systems cache user-defined variables to support the typical incremental construction of data visualizations~\cite{zgraggen2016progressive, eichmann2020idebench} during data analysis sessions~\cite{jupyter, rstudio, colab}. 
Recent work proposes a management scheme for these cached variables under a memory constraint that greedily keeps variables with the highest estimated time savings based on predicted future user behavior via neural networks~\cite{xin2021enhancing}.
While useful for data visualization, a greedy approach to memory management fails to achieve satisfactory results compared to \system.

\mypara{Intermediate Result Reuse}

There exist works on storing intermediate results from computations to speedup future computations~\cite{yang2018intermediate, dursun2017revisiting, nagel2013recycling, michiardi2019memory, galakatos2017revisiting}.
Studied topics include the identification of reuse opportunities by finding overlaps in computation graphs of successive jobs~\cite{yang2018intermediate, michiardi2019memory},
selective storage under a space constraint with heuristics such as reuse probability~\cite{dursun2017revisiting}, expected savings~\cite{yang2018intermediate}, and recompute-storage cost difference~\cite{nagel2013recycling},
and rewriting incoming jobs to take advantage of stored intermediates~\cite{galakatos2017revisiting}.
These works share similarity with \system in their selection of items to store under a memory constraint, however, \system's problem setting requires it to uniquely consider the joint (re)ordering of job executions along with the selection of items.

% work that considers both job execution (re)order as well as intermediate result caching with a bounded amount of memory. but notably lack the joint aspect of \system and cannot be used to achieve immediate speedup on an incoming MV refresh run if no intermediates are stored beforehand. 

\mypara{Incremental Query Processing} Incremental processing (IQP) is useful for cases where not all data required for a query is immediately available. Similar to online aggregation~\cite{hellerstein1997online}, initial results of a query are computed on a subset of required data and progressively refined as the rest of the required data arrives in a predictable pattern~\cite{tang2019intermittent,wangtempura}. Tang et al. propose a dynamic programming formulation to pick intermediate states to store in memory given a limited memory budget~\cite{tang2019intermittent}. Tempura rewrites the query plan for more efficient execution based on predicted data arrival patterns~\cite{wangtempura}. While similarities exist between the problem setting of IQP and \system, such as management of bounded memory, \system notably includes additional joint optimization for the order of MV updates.

% \paragraph{Sampling}
% Sampling has seen wide use in visualization systems for reducing the computation time of ad-hoc queries by computing an approximate result over a subset of data as exact results are not always required by the user~\cite{crotty2016case, mani2017effective, zgraggen2014panoramicdata, kraska2021northstar, galakatos2017revisiting, kandula2016quickr}. 
% Commonly studied topics in sampling for ad-hoc queries include complex query sampling~\cite{kandula2016quickr}, rare event aggregation~\cite{kraska2021northstar, galakatos2017revisiting}, and maintaining consistency between related sampled visualizations~\cite{zgraggen2014panoramicdata}.
% Sampling server-side at the MV level compromises the assumptions of downstream applications and is thus not considered in \system.

% \paragraph{Progressive visualization}
% The latency tolerance for time-consuming queries can be circumvented by presenting a partially-computed visualization to the user within the tolerance, which is then incrementally refined until it is fully accurate~\cite{rahman2017ve, zgraggen2016progressive, crotty2015vizdom, kraska2021northstar, kamat2017infiniviz}.
% Example plots which benefit from progressive visualization include bar charts~\cite{kamat2017infiniviz} and heatmaps~\cite{rahman2017ve}.
% Similar to sampling, study on this topic is orthogonal to \system as pushing out partially-updated MVs compromises downstream assumptions.
\section{Method}
\label{sec:method}

% \ml{``Inconsistent'' to ``large variation''}

% In this section, we propose our methods based on the observations in Section \ref{sec:motivation}.
In this section, we propose two techniques to further enhance the strong baseline to capture the variation of activation distributions better.
We first introduce spatial re-scaling to adapt the network to pixel-to-pixel variation.
We then propose channel-wise shifting and re-scaling to better capture the channel-to-channel variation.
Meanwhile, as both of the two methods are image-dependent, the image-to-image variation can be captured naturally.
By combining the two methods with our strong baseline, we build our enhanced BNN for SR, named EBSR.

% Because the activation distributions among pixels, channels and images have large variations \red{**are highly inconsistent} in SR networks, we introduce spatial re-scaling to adapt to pixel-wise variations and channel shift and re-scaling to adapt to channel-wise variations. And both of them are image-dependent to adapt to image-wise variations, which means during inference our network re-scales and shifts the distributions of activations flexibly for different input images. Based on these methods, we build an enhanced binary neural network for image super-resolution (EBSR).

% According to [3], the difference of activation magnitudes indicates different scaling factors are needed for each pixel.

\subsection{Spatial Re-scaling}
% It is better to use different scaling factors for different pixels to reduce the quantization error and retain more detailed information for image super-resolution. 

% \ml{In the main method, we do not need to introduce the previous works but can focus on introducing our own method. Channel rescaling in Real-to-binary Net is not relevant in this context.}

% Re-scaling the output of binary convolutions was proposed at the birth of BNN in XNOR-Net \cite{rastegari2016xnor} to reduce quantization error and improve accuracy for image classification tasks.
% It is computed as below:
% \begin{equation}
% \mathcal{A} * \mathcal{W} \approx(\operatorname{sign}(\mathcal{A}) \circledast \operatorname{sign}(\mathcal{W})) \odot \mathcal{K} \alpha
% \label{eq:xnor-net rescale}
% \end{equation}
% where $\circledast$ denotes the binary convolution and $\odot$ denotes the element-wise multiplication.
% $\mathcal{A}$, $\mathcal{W}$, $\alpha$, and $\mathcal{K}$ denote the activation, weight, weight scaling factor, and activation scaling factor, respectively.
%  Later in XNOR-Net++ \cite{bulat2019xnor}, Bulat et al. fuse the activation and weight scaling factors into a single one that is learned end-to-end based on gradients and this improves the classification accuracy on ImageNet dataset.

% % It is computed as Eq.~\ref{eq:xnor-net rescale}, where $\circledast$ denotes 
% %  the binary convolution and $\odot$ denotes the element-wise multiplication. The binary convolution of $\mathcal{A}$ and $\mathcal{W}$ is rescaled by the weight scaling factor $\alpha$ and the activation scaling factor $\mathcal{K}$, both of which are calculated analytically.


% \zc{Similarly, you should explain the meaning of A, W and the operators $\circledast$ in the formula}
% Then in Real-to-binary Net \cite{martinez2020training}, Martinez et al. used a data-driven channel re-scaling module that takes the pre-convolution activations as input to predict the activation scaling factor. Unlike that in XNOR-Net++ \cite{bulat2019xnor}, these scaling factors are not fixed during inference but rather inferred from data. By doing this, they further improved the classification accuracy on ImageNet over XNOR-Net++. 
As is shown in Figure \ref{fig:pixel}, activation distributions have large pixel-to-pixel variation in SR networks
and the difference of activation magnitudes indicates different scaling factors are preferred for different pixels.
Inspired by \cite{martinez2020training}, we propose spatial re-scaling to better adapt the network to the spatial variation
of activation distributions in SR networks.
% fit the various pixel-wise distributions in SR networks.
We take the real-valued activations $A$ before convolution as input and predict pixel-wise scaling factors $S(A)$, which re-scale the binary convolution output. Spatial re-scaling process can be formulated as follows:
\begin{equation}
A * W \approx(\operatorname{sign}(A) \circledast \operatorname{sign}(W)) \odot \alpha \odot S(A)
\label{eq:spatial rescale}
\end{equation}
where $\circledast$ denotes 
the binary convolution and $\odot$ denotes the element-wise multiplication. $A$, $W$, $\alpha$, and $S\left(A\right)$ denote real-valued activations, weights, the scaling factor of weights, and the spatial-wise scaling factor of activations respectively. $S\left(A\right) \in \mathbb{R}^{1\times H\times W}$ can be calculated with a convolution and a sigmoid function.
% as $\sigma\left( CONV\left(A\right)\right)$. 
As shown in Figure \ref{fig:method}(a), real-valued activations first go through a convolution layer,
which has an input channel of $C$ and an output channel of 1, 
and then pass through a sigmoid function to produce the scaling factors $S(A)$ along the spatial dimension.
During inference, the scaling factor will change dynamically according to different input feature maps.
By re-scaling binary convolution output using $S(A)$, we can reduce the quantization error and the original pixel-wise information in FP activation
will be preserved much better.
Spatial re-scaling leads to a large PSNR improvement of 0.24 dB (from 30.30 dB to 31.54 dB) on Set5 and 0.22 dB (from 25.09 dB to 25.31 dB)
on Urban100 compared with our strong baseline. 

\subsection{Channel-wise Shifting and Re-scaling}

\begin{table}[!tb]
\centering
\caption{Comparison between whether to fuse channel-wise shifting and re-scaling or not based on our baseline with spatial re-scaling. }
\label{tab:fusing}

\scalebox{0.65}{
\begin{tabular}{c|cc|cc|cc}
\hline
\multirow{2}{*}{Method}     & \multirow{2}{*}{OPs} & \multirow{2}{*}{Params} & \multicolumn{2}{c|}{Set5} & \multicolumn{2}{c}{Urban100} \\ \cline{4-7} 
                            &                      &                         & PSNR        & SSIM        & PSNR          & SSIM         \\ \hline
Baseline + spatial re-scale & 2.16G                & 0.05M                   & 31.54       & 0.883       & 25.31         & 0.759        \\
+ channel-wise shift and re-scale             & 2.34G                & 0.09M                   & 31.61       & 0.885       & 25.35         & 0.761        \\
+ w/ fusing                   & 2.27G                & 0.08M                   & \textbf{31.64}       & \textbf{0.885}       & \textbf{25.36}         & \textbf{0.761}        \\ \hline
\end{tabular}
}
\end{table}

In SR networks, activation distributions exhibit larger channel-to-channel variation (Figure \ref{fig:chl}).
Both the mean and magnitude of the activation distributions vary significantly across channels.
% Thus we use channel-wise shifting and re-scaling to adapt to various channel-wise distributions. 
\cite{martinez2020training} has proposed the data-driven channel re-scaling, 
but our method differs from them in further introducing data-driven thresholds to handle the channel-wise variation of both mean and magnitude.
Since the blocks to generate the scaling factors and thresholds are very similar, we further propose to fuse them into one module.
% and fusing channel-wise shifting and re-scaling into one module.
We evaluate the effect of fusing the two blocks in Table \ref{tab:fusing}.
With channel-wise shifting and re-scaling fused, our models have fewer operations and parameters overhead and slightly higher performance.

For the specific process, we take the real-valued activations as input and predict different thresholds and scaling factors for each channel. They are also image dependent, e.g., $\beta_{i}$ in Eq.\ref{eq:act_binarize} is no longer fixed during inference but generated according to different input feature maps. Channel-wise shifting and re-scaling can be formulated as follows:
\begin{equation}
A * W \approx(\operatorname{sign}(A-C_s(A)) \circledast \operatorname{sign}(W)) \odot \alpha \odot C_r(A)
\label{eq:channel-wise_shift_and_rescale}
\end{equation}
where $\circledast$ denotes 
the binary convolution and $\odot$ denotes the element-wise multiplication. $C_s(A), C_r(A) \in \mathbb{R}^{C\times1\times1}$ denote the channel-wise threshold and scaling factor, respectively. 
We show the block diagram in Figure \ref{fig:method}(b).
The real-valued input feature map is first squeezed to a ${C\times1\times1}$ vector by a global average pooling (GAP) layer.
The subsequent fully connected layers and ReLU learn the channel-wise information and output a ${2C\times1\times1}$ vector.
Then the ${2C\times1\times1}$ vector is split into two ${C\times1\times1}$ vectors.
We use the first $C$ channels as the channel-wise bias and pass the last $C$ channels through a sigmoid layer 
as the channel-wise scaling factor, which are used to shift the real-valued activations and re-scale the binary convolution output, respectively. 


% \ml{We can mention previously, channel-wise re-scale has been proposed. We propose to fuse them. Add the comparison between fuse v.s. no fuse.}

\begin{figure}[!tbp]%
  \centering
    \includegraphics[width=0.4\textwidth]{fig/methods.png}
  
% \subfloat[channel-wise shifting\&re-scale]{
%     \label{subfig:channel-wise shifting and re-scale}
%     \includegraphics[width=0.2\textwidth]{fig/chl shift and rescale.png}
%   }

  \caption{Block diagram for spatial re-scaling, and channel-wise shifting and re-scaling.} 
  % Input A is the real-valued activation tensor and C, H, and W denote its dimension. GAP stands for global average pooling. The reduction ratio r is set to 16 for a better trade-off between the performance and the number of operations and parameters.}
  \label{fig:method}
\end{figure}


\subsection{Network Structure}

Combining the spatial re-scaling and the channel-wise shifting and re-scaling methods, we construct the enhanced convolution layer (E-Conv).
Then we build our EBSR model based on E-Conv.
In Figure \ref{fig:E-conv}, we compare the binary convolution layer used in the baseline network and our proposed E-Conv.
We use spatial and channel-wise scaling factors to re-scale the binary convolution output,
and use channel-wise shifting to learn appropriate thresholds for each channel before binarization.
The scaling factors and threshold used in E-Conv are learnable and depend on the real-valued input activations.
In this way, our proposed EBSR can adapt to pixel-to-pixel, channel-to-channel, and image-to-image variations
to reduce the large binarization error and preserve more details.
% In this way, our proposed E-Conv reduces the large quantization error caused by binarization and keeps the original information of input feature maps to a large extent.


\begin{figure}[!tb]%
  \centering

    \includegraphics[width=0.5\textwidth]{fig/E-conv.png}

  \caption{Comparison of (a) the binary convolution layer with a skip connection used in our baseline network and (b) the proposed E-Conv.}
  \label{fig:E-conv}
\end{figure}


Figure \ref{fig:network} shows the basic block based on the E-Conv and our EBSR composed of the basic blocks. Following existing works, the convolution layers in the head and tail modules are not binarized. We choose the lightweight EDSR which has 16 basic blocks and 64 channels, and EDSR which has 32 basic blocks and 256 channels as our backbones, which correspond to EBSR-light and EBSR, respectively.

\begin{figure}[!tb]%
  \centering
  {
    \includegraphics[width=0.35\textwidth]{fig/network.png}
  }
  
  \caption{The structure of our proposed EBSR.  Convolution layers in purple are real-valued vanilla 3x3 convolutions.}
  \label{fig:network}
\end{figure}
% PTMTorrrent
\newcommand{\numberOfModelHub}{5\xspace}

\newcommand{\TotalNumberOfPackages}{{15,913}\xspace}
% 12401 from Hugging Face
% 185 from ONNX
% 33 from Model Hub
% 3245 from Model Zoo
% 49 from PyTorch Hub
% SUM (by Nick): 15,913

\newcommand{\HFNumberOfPackages}{{12,401}\xspace}
\newcommand{\HFNumberOfPackagesMetadata}{{124,427}\xspace}
\newcommand{\MZNumberOfPackages}{3,245\xspace}
\newcommand{\PHNumberOfPackages}{{49}\xspace}
\newcommand{\MHNumberOfPackages}{{33}\xspace}
\newcommand{\ONNXNumberOfPackages}{{185}\xspace}

\newcommand{\TotalDataSize}{\textasciitilde{61TB}\xspace}
\newcommand{\HFDataSize}{{61TB}\xspace}
\newcommand{\MZDataSize}{{115GB}\xspace}
\newcommand{\PHDataSize}{{1.5GB}\xspace}
\newcommand{\MHDataSize}{{721MB}\xspace}
\newcommand{\ONNXDataSize}{{441MB}\xspace}
%%%



% ICSE submission - HFTorrent v1

\newcommand{\PTMDatasetNPackages}{63,182\xspace}
\newcommand{\PTMDatasetPercentage}{{99.7\%}\xspace}
\newcommand{\PTMDatasetFailedPackages}{{186}\xspace}
\newcommand{\PTMDatasetFailedPercentage}{{0.3\%}\xspace}

\newcommand{\PTMDatasetNReposWithSignedCommits}{{132}\xspace}
\newcommand{\PTMDatasetPercentOfSignedCommits}{{0.208\%}\xspace}


\newcommand{\PercentOfVerifiedOrgs}{{3.188\%}\xspace}
\newcommand{\NOrganizations}{{6,243}\xspace}
\newcommand{\NVerifedOrgs}{{199}\xspace}

\newcommand{\NOfRepositoriesWithMalware}{{1}\xspace}
\newcommand{\PercentageOfRepositoriesWithMalware}{{0.002\%}\xspace}
\newcommand{\TotalRepositoriesForMalwareScanning}{{63,366}\xspace}
\section{Experiments}
\label{sec:exp}

In this section, we demonstrate the wide range of applications and the high capabilities of Uni-Fusion. 
First, we evaluate Uni-Fusion in application 1) Incremental surface and color reconstruction, comparing its performance with SOTAs.
%
For applications 2) and 5), which are new topics, no specific benchmarks are available. 
Therefore, we showcase the performance on existing results.
%
Next, we implement application 3) and compare it with SOTA zero-shot semantic segmentation models.
%
Finally, for application 4), since infrared data is not commonly used, we collect our own dataset containing infrared values and show all applications on this data.

\subsection{Implementation Details}
\label{sec:exp:details}

In the experiments, we use our sample-based GPIS for local geometry encoding.
For each point, two additional points are sampled along normal direction, one positive and one negative, with distance $d_s=0.1$ in the local voxel's normalized space. 
Compared to derivative-based GPIS, our sample-based GPIS is more efficient in both space and time. 
For the encoder, we randomly sample $256$ anchor points from the range $[-0.5,0.5]^3$.
We utilize the first $20$ eigenpairs, resulting in a feature dimension of $20$.
The model selection process is discussed in the ablation study.

Different latent maps use different granularities.
For the surface LIM, we use a voxel size of $5\si{\centi\meter}$. 
For color which requires later comparison to NeRF, we use a voxel size of $2\si{\centi\meter}$.
For other property LIM and feature LIM, we use a voxel size of $10\si{\centi\meter}$.

For smooth reconstruction, the encoded voxel is designed overlapped following~\cite{huang2021di}.
The encoded voxel uses twice the voxel size, resulting in a half-space overlap with each neighboring voxel.
During meshing, SDFs are retrieved and interpolated from the overlapped voxels~\cite{huang2021di}.
While for the remaining properties, we sample only from its own voxel part.

The implementation runs on a PC with AMD Ryzen 9 5950X 16-core CPU and an Nvidia Geforce RTX 3090 GPU (24 GB).

\subsection{Datasets}

We evaluate incremental reconstruction on the ScanNet dataset~\cite{dai2017scannet}, TUM RGB-D dataset~\cite{sturm2012benchmark}, and Replica dataset~\cite{sucar2021imap}.
Using MSG-Net~\cite{zhang2018multi}'s material set, we transfer styles to the 3D canvas.
For open-vocabulary scene understanding, we evaluate on ScanNet segmentation data~\cite{qi2017pointnet++} and S3DIS dataset~\cite{armeni20163d}.

\subsubsection{ScanNet~\cite{dai2017scannet}}

ScanNet is a densely annotated RGB-D video dataset.
It is captured with the structure sensor~\cite{occipital} and contains 1513 scenes for training and validation.
For each scene, both images and a 3D mesh is provided, along with their 2D and 3D semantic annotations. 

ScanNet provides 312 scenes for validation, which contains a wide range of different room structures.
It has now been widely used in the thorough evaluation of the performance of reconstruction and semantic segmentation.

\subsubsection{TUM RGB-D~\cite{sturm2012benchmark}}

TUM RGB-D is a benchmark to mainly evaluate the tracking performance.
It is captured with Microsoft Kinect sensor together with ground-truth trajectory from the sensor.

\subsubsection{Replica~\cite{sucar2021imap}}

The Replica dataset refers to iMAP's pre-processed dataset~\cite{sucar2021imap}.
It is a synthetic rendered RGB-D dataset from given 3D models.
The advantage of including this dataset is that Replica does not have motion blur. 
This is better to evaluate the capability of the algorithms on reconstructing surface color.

\subsubsection{MSG-Net Style~\cite{zhang2018multi}}

MSG-Net provides material images for transfering the styles.
We select 21style fold for demonstration.
These images are given in \cref{fig:style} together with our result.

\subsubsection{ScanNet Point Cloud Segmentation Data~\cite{qi2017pointnet++}}

For point cloud semantic segmentation benchmarking, PointNet++~\cite{qi2017pointnet++} preprocesses the original ScanNet~\cite{dai2017scannet} and generates subsampled point clouds and corresponding annotations for each scene.

\subsubsection{S3DIS~\cite{armeni20163d} and 2D-3D-S~\cite{armeni2017joint}}

S3DIS is a semantic segmentation dataset for 3D point clouds.
Which is also a subset of the 2D-3D-S dataset.
The 2D-3D-S dataset is a multi-modality dataset containing 2D, 2.5D and 3D domains. 
This dataset is densely annotated with semantic labels.

Note that 2D-3D-S's 2D captures is not a RGB-D video as ScanNet.
2D-3D-S's images only have small overlap. 
Therefore, it is only suitable for semantic segmentation and not for incremental reconstruction.

\subsection{Baselines}

For online surface mapping evaluation, we select TSDF-Fusion~\cite{curless1996volumetric}, iMAP~\cite{sucar2021imap}, SOTA DI-Fusion~\cite{huang2021di} and BNV-Fusion~\cite{li2022bnv} as four baseline methods.

For the color field, we choose TSDF-Fusion~\cite{curless1996volumetric}, $\sigma$-Fusion~\cite{rosinol2023probabilistic}, iMAP~\cite{sucar2021imap}, NICE-SLAM~\cite{zhu2022nice} and even the recent hot Neural Radiance Fields model NeRF-SLAM~\cite{rosinol2022nerf} as five baselines.
While including NeRF in the comparison may not be entirely fair, we want to show how Uni-Fusion narrows the performance gap.

For the scene understanding application, we evaluate generalized zero-shot point cloud semantic segmentation with ZSLPC~\cite{cheraghian2019zero}, DeViSe~\cite{frome2013devise} and SOTA 3DGenZ~\cite{michele2021generative} for comparison.

\subsection{Metrics}

For incremental reconstruction, we evaluate the geometric reconstruction using \textbf{Accuracy}, \textbf{Completeness}, and \textbf{F1 score} according to SOTA BNV-Fusion. It firstly uniformly samples $100,000$ points from the reconstruction and ground truth meshes respectively.
Then \textbf{Accuracy} (\textbf{Completeness}) measures the percentage of reconstruction-to-groundtruth (groundtruth-to-reconstruction) distances that are lower than $2.5\si{\centi\meter}$ threshold. \textbf{F1 score} is the harmonic mean of accuracy and completeness.
For tracking performance, we use \textbf{ATE RMSE}.

To evaluate color reconstruction, we follow SOTA on this topic, NeRF to render both depth and RGB images to compare the image level \textbf{Depth L1} and \textbf{RGB PSNR}.

To compare scene understanding, we follow zero-shot point cloud semantic segmentation SOTA 3DGenZ to evaluate the \textbf{Intersection-of-Union (IoU)} and \textbf{Accuracy}.


\subsection{Reconstruction Results}

For evaluation, we first use the ScanNet validation set with 312 sequences to thoroughly test the geometric reconstruction on a large variant of scenes.
%
Then, we use TUM RGB-D to compare our modified tracking model with related works.
Because this part is not the main contribution, we give a rough overview of the tracking results.
%
To fairly evaluate the color reconstruction, we use the high quality rendered Replica dataset to compare with related works, including NeRF.

%\subsubsection{Object}
% on instance-gp
% Objective data usually has more fine detail
% 1. for detail precision
% A: no, object reconstruction is not as good as instance-ngp, so cancelled.
\begin{table*}[!]
	\centering
	\caption{Comparison to ScanNet~\cite{dai2017scannet}.
       Our method generalizes better to various scenes.
       $^*$ indicates the result from our runs of the official BNV-Fusion code.}
	\small
	%\setlength{\tabcolsep}{5mm}
	\setlength{\tabcolsep}{0.9em}
		%\resizebox{\textwidth}{!}{
		\begin{tabular}{l  c c c| c c c }
			\toprule
			Method & \begin{tabular}{@{}c@{}}Pre-Train\\ with extra dataset\end{tabular} & \begin{tabular}{@{}c@{}}Train \\ with sequences\end{tabular} & Real-time & Accuracy (\%) $\uparrow$ & Completeness (\%) $\uparrow$ & F1 score $\uparrow$ \\
			\midrule
			TSDF Fusion~\cite{zhou2018open3d} & None & None & $\checkmark$ &73.83 & 85.85 & 78.84 \\
			iMAP~\cite{sucar2021imap} & None & Online train& &68.96 & 82.12 & 74.96 \\
			DI-Fusion~\cite{huang2021di} &Object Pretrain & None & $\checkmark$&66.34 & 79.65 & 72.97 \\
			BNV-Fusion~\cite{li2022bnv} &Object Pretrain &  Post Optimization& &{74.90} & \textbf{88.12} & {80.56} \\
			BNV-Fusion$^{*}$~\cite{li2022bnv} &Object Pretrain & Post Optimization &&{73.42} & {81.75} & {77.18} \\
			\textbf{Uni-Fusion (Ours)} &None &None &$\checkmark$&\textbf{80.43} & {84.91} & \textbf{82.44} \\
			\bottomrule
		\end{tabular}
	  %}
	\label{tab:scannet}
	\vspace{-.6cm}
\end{table*}
\begin{figure*}[t]
	\subfloat[width=.33\textwidth][Accuracy]{
		\centering
		\includegraphics[width=.22\linewidth]{im/exp/recons/scannet/scannet_acc.png}
		\includegraphics[width=.1\linewidth]{im/exp/recons/scannet/scannet_acc_box.png}
	}
	\subfloat[width=.33\textwidth][Completeness]{
		\centering
		\includegraphics[width=.22\linewidth]{im/exp/recons/scannet/scannet_comp.png}
		\includegraphics[width=.1\linewidth]{im/exp/recons/scannet/scannet_comp_box.png}
	}
	\subfloat[width=.33\textwidth][F1 score]{
		\centering
		\includegraphics[width=.22\linewidth]{im/exp/recons/scannet/scannet_F1.png}
		\includegraphics[width=.1\linewidth]{im/exp/recons/scannet/scannet_F1_box.png}
	}
	\label{fig:recon:scannet:elementwise}
	\caption{Quantitative comparison on 312 scenes of the ScanNet validation set.
       We demonstrate the performance of SOTA BNV-Fusion and our Uni-Fusion.
       We sort our evaluation value and reordered all of the scores.
       The zigzag pink is the BNV-Fusion result;
       we also plot a deep-pink smoothed curve for better visualization.}
\end{figure*}

\subsubsection{Evaluation on ScanNet Dataset~\cite{dai2017scannet}}
\label{sec:exp:scannet}

We use the 312 diversified scenes from the ScanNet validation set to evaluate the performance of surface reconstruction. 
We follow the pure mapping SOTA BNV-Fusion to take every 10th posed frame as input. 
%
Without using any learning (in contrast iMAP, DI-Fusion, and BNV-Fusion do) or any post optimization (as BNV-Fusion does), our Uni-Fusion is capable to achieve precise continuous mapping performance. 

As shown in~\cref{tab:scannet}, our Uni-Fusion achieves \textbf{$+6$ higher accuracy} than the incremental surface reconstruction SOTA BNV-Fusion.
Our model does not exceed on completeness comparing to BNV-Fusion that support completion in post-optimization.
Though, Uni-Fusion's completion is still much higher than one other optimization based model iMAP.
%We consider it because our model does not support hole-completion as the optimization based models iMap and BNV-Fusion.
Overall, our Uni-Fusion model achieves higher F1-scores that quantifies the overall quality.

Please note that, SOTA BNV-Fusion is not real-time capable, since it requires post optimization of all fed frames.
Without the post-optimization, the real-time model Di-Fusion shows much worse results.
However, our \textbf{real-time} model \textbf{Uni-Fusion} is able to achieves \textbf{much better} reconstruction quality than these approaches even without post-optimization. 

\newcommand{\scannetImSize}{.16}
\begin{figure*}[t!]
	\centering
	\setlength{\tabcolsep}{0.1em}
	\renewcommand{\arraystretch}{.1}
	\begin{tabular}{|c | c |c |||c |c | c|}
		\hline
		{\Large{BNV-Fusion}} & {\Large{Uni-Fusion}} &{\Large{Ground Truth}} & {\Large{BNV-Fusion}} &{\Large{Uni-Fusion}} & {\Large{Ground Truth}} \\ \hline \hline
		
\includegraphics[width=\scannetImSize\linewidth]{im/exp/recons/scannet_qualifi/scene0568_00_bnv.png}
		&\includegraphics[width=\scannetImSize\linewidth]{im/exp/recons/scannet_qualifi/scene0568_00_mine.png}
		&\includegraphics[width=\scannetImSize\linewidth]{im/exp/recons/scannet_qualifi/scene0568_00_gt.png}
		&		\includegraphics[width=\scannetImSize\linewidth]{im/exp/recons/scannet_qualifi/scene0164_00_bnv.png}
		&\includegraphics[width=\scannetImSize\linewidth]{im/exp/recons/scannet_qualifi/scene0164_00_mine.png}
		&\includegraphics[width=\scannetImSize\linewidth]{im/exp/recons/scannet_qualifi/scene0164_00_gt.png}\\
		
\includegraphics[width=\scannetImSize\linewidth]{im/exp/recons/scannet_qualifi/scene0249_00_bnv.png}
		&\includegraphics[width=\scannetImSize\linewidth]{im/exp/recons/scannet_qualifi/scene0249_00_mine.png}
		&\includegraphics[width=\scannetImSize\linewidth]{im/exp/recons/scannet_qualifi/scene0249_00_gt.png}
		&		\includegraphics[width=\scannetImSize\linewidth]{im/exp/recons/scannet_qualifi/scene0435_00_bnv.png}
		&\includegraphics[width=\scannetImSize\linewidth]{im/exp/recons/scannet_qualifi/scene0435_00_mine.png}
		&\includegraphics[width=\scannetImSize\linewidth]{im/exp/recons/scannet_qualifi/scene0435_00_gt.png}\\
		
\includegraphics[width=\scannetImSize\linewidth]{im/exp/recons/scannet_qualifi/scene0046_00_bnv.png}
		&\includegraphics[width=\scannetImSize\linewidth]{im/exp/recons/scannet_qualifi/scene0046_00_mine.png}
		&\includegraphics[width=\scannetImSize\linewidth]{im/exp/recons/scannet_qualifi/scene0046_00_gt.png}
		&		\includegraphics[width=\scannetImSize\linewidth]{im/exp/recons/scannet_qualifi/scene0050_00_bnv.png}
		&\includegraphics[width=\scannetImSize\linewidth]{im/exp/recons/scannet_qualifi/scene0050_00_mine.png}
		&\includegraphics[width=\scannetImSize\linewidth]{im/exp/recons/scannet_qualifi/scene0050_00_gt.png}\\
		\hline
	\end{tabular}
	%\captionof{figure}
	\caption{Surface reconstruction on ScanNet dataset.}
	\label{fig:recons:scannet_demo}
	\vspace{-.5cm}
\end{figure*}

We additionally run BNV-Fusion's official implementation (emphasized with $^*$) on the 312 videos of ScanNet and do a post element-wise comparison in \cref{fig:recon:scannet:elementwise}. 
Our result is the {\color{Cyan}light blue} curve, BNV-Fusion's result is colored with {\color{Lavender}pink}.
Scene index is sorted corresponding to the score value of Uni-Fusion.
For better visualization, we smooth BNV-Fusion's curve and plot it with dark pink.
It is obvious that the score of Uni-Fusion is overall higher than BNV-Fusion's. 
Moreover, we use box-plot to conclude the statistics besides the curve plot. Uni-Fusion's scores are distributed in a higher region. For completeness which is less obvious better, Uni-Fusion's box is smaller while in a relative higher position. This means that Uni-Fusion has more stable completeness result while BNV-Fusion is more likely to get low completeness in some cases.

To summarize, our model is almost better on all 312 scenes on all accuracy, completeness and F1-score.
Which is also revealed in \cref{tab:scannet} with BNV-Fusion$^*$, that the BNV-Fusion's official implementation does not exceed Uni-Fusion on all metrics.

We plot reconstruction on selected scenes from ScanNet in~\cref{fig:recons:scannet_demo}. 
Both BNV-Fusion and our Uni-Fusion are able to produce high quality reconstruction.
We see that BNV-Fusion gives lots of small meshes on walls, which are shown as small particles in the reconstruction. 
We consider it is because BNV-Fusion use very small voxel size ($0.02\si{\meter}$) to get a high score.
This is also revealed by their \textbf{\SI{247}{MB}} mesh in average, while ours is only \textbf{\SI{54}{Mb}} in average.
Furthermore, our Uni-Fusion's mesh is more smooth and
%Both BNV-Fusion and Uni-Fusion demonstrate high quality result.
also provides high-precise color to the mesh which is not available for the Surface SOTA.

%In this test, we purely evaluate the surface reconstruction capacity with SOTAs. 
%While reconstruction is not merely surface.  
%Thus in the following, we find benchmarks for both surface and color.


\subsubsection{Tracking Evaluation on TUM RGB-D Dataset~\cite{sturm2012benchmark}}
% follow nice-slam

In the above test, we compare the performance of pure mapping.
While tracking is not the contribution focus in our paper, it is part of the reconstruction model. We follow the novel reconstruction model NICE-SLAM~\cite{zhu2022nice} to evaluate the camera tracking on the small-scale TUM RGB-D dataset.
Our Uni-Fusion uses a coarse-to-fine strategy for 3D reconstruction tracking.
From~\cref{tab:tum_rmse}, it demonstrates overall better ATE RMSE than other implicit representation models.

\begin{table}[]
		\caption{Tracking on TUM RGB-D~\cite{sturm2012benchmark}.
		ATE RMSE [$\si{\centi\meter}$] ($\downarrow$) is used as the evaluation metric.
	}
	\centering
	\footnotesize
	\setlength{\tabcolsep}{0.7em}
	\resizebox{\linewidth}{!}{
		\begin{tabular}{l|ccc}
			\hline
			& \tt{fr1/desk} &  \tt{fr2/xyz} &  \tt{fr3/office} \\
			
			\hline
			{iMAP}~\cite{sucar2021imap}      & 4.9 & 2.0 & 5.8  \\
			{iMAP$^*$}~\cite{sucar2021imap} & 7.2 & 2.1  & 9.0 \\
			{DI-Fusion~\cite{huang2021di}} & 4.4 & 2.3 & 15.6 \\
			NICE-SLAM~\cite{zhu2022nice}           & 2.7 & 1.8 & 3.0 \\
			Ours& 1.8& 0.5& 2.1 \\
			\hline
			{BAD-SLAM}\cite{schops2019bad} & 1.7  & 1.1  & 1.7 \\
			{Kintinuous}\cite{whelan2012kintinuous} & 3.7  &  2.9  & 3.0 \\
			{ORB-SLAM2}\cite{mur2017orb} & \bf 1.6  & \bf 0.4  & \bf 1.0 \\
			\hline
	\end{tabular}}
	\vspace{2pt}

	\label{tab:tum_rmse}
\end{table}

On the other hand, there also exist high accuracy algorithms from SLAM. 
By additional using Bundle Adjustment and Loop-closing techniques, their tracking quality is much better than all of the implicit based models.

%But it is dangerous to directly apply SLAM result on reconstruction. Please find our demonstration in Fig [?]. Which explains the more widely used frame-to-model strategy in 3D reconstruction.
Even though, our coarse-to-fine strategy firstly ensure not easy to lose track. Secondly, it is more suitable for surface fitting.

Which further support our test in Replica dataset.



\begin{table*}[t!]
	\centering
	\caption{Geometric (L1) and Photometric (PSNR) evaluation on the Replica dataset~\cite{sucar2021imap}.}
	\footnotesize
	\setlength{\tabcolsep}{0.36em}
	\renewcommand{\arraystretch}{1.2}
	\begin{tabular}{clcccccccccccccccccc}
		\toprule
		& & \multicolumn{1}{c}{\makecell{\tt{office-0}}} & \multicolumn{1}{c}{\makecell{\tt{office-1}}} & \multicolumn{1}{c}{\makecell{\tt{office-2}}}& \multicolumn{1}{c}{\makecell{\tt{office-3}}} & \multicolumn{1}{c}{\makecell{\tt{office-4}}} & \multicolumn{1}{c}{\makecell{\tt{room-0}}} & \multicolumn{1}{c}{\makecell{\tt{room-1}}} &  \multicolumn{1}{c}{\makecell{\tt{room-2}}} & Avg. \\
		\midrule
		\multicolumn{5}{l}{\textit{Non-continuous mapping method}}\\
		\multirow{2}{*}{\makecell{\textbf{TSDF-Fusion}~\cite{curless1996volumetric}}}
		& {\bf Depth L1} [$\si{\centi\meter}$] $\downarrow$
	 & 14.11 & 10.50 & 30.89 & 28.92 & 22.83	& 23.51 & 20.94 & 23.34 & 21.88 \\
		& {\bf PSNR } [$\si{\dB}$] $\uparrow$
		& 11.16 & 15.92 & 4.86 & 5.68 & 5.46 & 3.43 & 4.51 & 5.57 & 7.07 \\
		
		\midrule
		\multirow{2}{*}{\makecell{\textbf{$\sigma$-Fusion}\cite{rosinol2023probabilistic} }}
		& {\bf Depth L1} [$\si{\centi\meter}$] $\downarrow$
		 & 13.80 & 10.21 & 22.27 & 28.70 & 22.21& 21.92 & 19.28 & 22.40 & 20.10 \\
		& {\bf PSNR } [$\si{\dB}$] $\uparrow$
		 & 11.16 & 15.92 & 4.86 & 5.69 & 5.46& 3.45  & 4.51 & 5.57 & 7.08 \\
		
		
		
		
		
		\midrule
		\midrule
		\multicolumn{5}{l}{\textit{Continuous mapping method}}\\
		\multirow{2}{*}{\makecell{\textbf{iMAP$^*$}~\cite{sucar2021imap}}}
		& {\bf Depth L1} [$\si{\centi\meter}$] $\downarrow$
		 & 6.43 & 7.41 & 14.23 & 8.68 & 6.80& 5.70 & 4.93 & 6.94 & 7.64\\
		& {\bf PSNR } [$\si{\dB}$] $\uparrow$
		& 7.39 & 11.89 & 8.12 & 5.62 & 5.98& 5.66 & 5.31 & 5.64  & 6.95\\
		\midrule
		\multirow{2}{*}{{\makecell{\textbf{Nice-SLAM}~\cite{zhu2022nice} }}}
		& {\bf Depth L1} [$\si{\centi\meter}$] $\downarrow$
		& { 1.51 } & { 0.93 } & { 8.41 } & { 10.48 } & {2.43} & { 2.53 } & { 3.45 } & { 2.93 }  & { 4.08 } \\
		& {\bf PSNR } [$\si{\dB}$] $\uparrow$
		 & { 22.44 } & { 25.22 } & { 22.79 } & { 22.94 } & { 24.72 } & \textbf{ 29.90 } & \textbf{ 29.12 } & { 19.80 }& { 24.61 } \\
		
		
		
		
		\midrule	
		\multirow{2}{*}{{\makecell{\textbf{Uni-Fusion} (Ours) }}}
		% using abs(diff)
		%	& {\bf Depth L1} [$\si{\centi\meter}$] $\downarrow$ &\textbf{1.98}&\textbf{1.18}&\textbf{1.64}&\textbf{1.23}&\textbf{0.84}&\textbf{1.61}&\textbf{3.01}&\textbf{1.60} &\textbf{1.64}
		% follow nerf-slam to remove outlier gt first
		& {\bf Depth L1} [$\si{\centi\meter}$] $\downarrow$
		& \textbf{0.79}&\textbf{0.56}&\textbf{1.59}&\textbf{2.71}&\textbf{1.66}&\textbf{1.94}&\textbf{0.69}&\textbf{1.80}& \textbf{1.47}
		\\
		& {\bf PSNR } [$\si{\dB}$] $\uparrow$ &\textbf{33.88}&\textbf{33.31}&\textbf{25.84}&\textbf{26.01}&\textbf{28.14}&24.02&26.20&\textbf{27.17} &\textbf{28.07}
		\\
		
		\midrule
		\midrule
		\multicolumn{5}{l}{\textit{Neural radiance field method}}\\
		\multirow{2}{*}{{\makecell{\textbf{NeRF-SLAM}~\cite{rosinol2022nerf} }}}
		& {\bf Depth L1} [$\si{\centi\meter}$] $\downarrow$
	 & {2.49}   & {1.98}  & {9.13}  & {10.58} & {3.59}	& {2.97}  & {2.63}  & {2.58}  & {4.49} \\
		& {\bf PSNR } [$\si{\dB}$] $\uparrow$
	 & \textbf{48.07}  & \textbf{53.44} & \textbf{39.30} & \textbf{38.63} & \textbf{39.21} 	& \textbf{34.90} & \textbf{36.95} & \textbf{40.75}& \textbf{41.40} \\
		
		\bottomrule
	\end{tabular}%
	
	\label{tab:replica_per_scene}
\end{table*}


\begin{table*}[t!]
	\centering
	\caption{Differences among different Surface \& Color reconstruction models.}
	\small
	\setlength{\tabcolsep}{.6em}
	%{
		%\resizebox{\textwidth}{!}{
			\begin{tabular}{l | c c c c c c }
				\toprule
				Method & 
				\begin{tabular}{@{}c@{}}Pre-Train\\ with extra dataset\end{tabular}
				& \begin{tabular}{@{}c@{}}Train\\ with sequences\end{tabular}
				& Real-time	
				& Direct Output &  \begin{tabular}{@{}c@{}}Light\\ direction\end{tabular} 
				&Render\\
				\hline 
				TSDF-Fusion & None & None & $\checkmark$& Discrete TSDF &  &Ray Rasterization\\\hline
				$\sigma$-Fusion & None & None &$\checkmark$&Discrete TSDF  && Ray Rasterization\\\hline
				iMAP & None & Online Train && MLPs  & &Volumetric Rendering\\\hline
				NICE-SLAM & \begin{tabular}{@{}c@{}}Pretrain\\ with indoor dataset\end{tabular} & Online Train&& Neural Implicit Grid&  & Volumetric Rendering\\\hline
				
				NeRF-SLAM & None & Train hundred epochs &-&NeRF & $\checkmark$ &Volumetric Rendering \\\hline
				
				\textbf{Uni-Fusion} & None & None&$\checkmark$& Latent Implicit Map && Ray Rasterization\\				
				\hline
			\end{tabular}
		%}
	%}
	\label{tab:replica_diff}
\end{table*}
\subsubsection{Evaluation on Replica RGB-D Dataset~\cite{sucar2021imap}}
In this evaluation, we compare with implicit reconstruction (TSDF-Fusion, $\sigma$-Fusion) and latent implicit reconstruction models (iMAP, NICE-SLAM) that support colors. 
We also add a large-scale NeRF model, NeRF-SLAM in to the table.  
NeRF is SOTA in view-synthesis task, which is unfair to direct compare with the rest. As the rest model does not even model light directions.
We add NeRF in this part to demonstrate that Uni-Fusion strongly reduce the gap.
Note that, NeRF-SLAM embeds external tracking model ~\cite{teed2021droid} to provide poses while using SOTA NeRF implementation Instance-ngp~\cite{muller2022instant} for NeRF construction.
%Therefore it is considered the SOTA to model the colors.

Uni-Fusion track and follow our previous setting in ScanNet test to take every 10 frames for mapping.
NICE-SLAM and NeRF-SLAM produce depth and color by rendering,
To obtain result from Uni-Fusion, we cast rays from virtual camera to our result surface mesh for depth image. 
Then Uni-Fusion infer the cast points in Uni-Fusion's color LIM for color result.

From~\cref{tab:replica_per_scene}, Uni-Fusion demonstrate
best Depth L1 on all scenes with an average of \textbf{$\pmb{1.47}$$\si{\centi\meter}$ depth L1}. Which is \textbf{$\pmb{177\%}$ boost} comparing to the second best.

Moreover, excluding NeRF, our Uni-Fusion also shows the best performance to model the colors with an average of $28.07$$\si{\dB}$ PSNR.

However, it is strange that NICE-SLAM lost details while in two cases, it shows better PSNR than Uni-Fusion. 
To highlight the true result,
we plot the rendered image in \cref{fig:replica_render}.
It is obvious that our Uni-Fusion models the details of painting, carpet and quilt well. 
While NICE-SLAM just roughly models the average color.

Moreover, from the  \cref{fig:replica_render}, our Uni-Fusion's rendering quality is as precise as NeRF. 
Please also find the painting, carpet and quilt, Uni-Fusion recovered the original appearance well.
Please find the {\color{green} green window} for the emphasized region.
Uni-Fusion reproduce the high-quality appearance which is very close to NeRF on qualitative evaluation.
%It can hardly find difference between the results from NeRF-SLAM, Uni-Fusion and Ground Truth.
%
But, Uni-Fusion still has a quantitative score gap to the color rending of NeRF ($41.4$$\si{\dB}$).
Though the Uni-Fusion's rendering result is highly close to NeRF and ground truth.
%
We consider the main reasons are that \textbf{1.} Uni-Fusion does not model the light directions to points, which is essential to NeRF.
\textbf{2.} NeRF optimizes on the rendering image quality by focussing mainly on color while less on depth.
It can be revealed by the higher color rendering score with much worse depth rendering than our Uni-Fusion.
\textbf{3.} our Uni-Fusion does not support hole filling.
This directly leads to black holes in our rendered images.

We summarize the differences in \cref{tab:replica_diff}.
Similar to TSDF-Fusion and $\sigma$-Fusion, our Uni-Fusion is a forward method which, does not need any training, i.e., pre- or online training. 
Uni-Fusion also produces similar to NICE-SLAM and NeRF-SLAM an implicit map with set of latent that outputs results at arbitrary resolution.
However, we differ on the extracting of the signed distance field.
%FIXME: I do not understand the next sentence.
Uni-Fusion's latent feature rule its own region independently.
Each query value is directly inferred with the corresponding ruling latent.
While NICE-SLAM and NeRF-SLAM use a much denser grid to interpolate query features. This requires volumetric rendering for inference.

Similar to TSDF-Fusion, $\sigma$-Fusion, our Uni-Fusion is also a real-time algorithm.
iMAP, NICE-SLAM and NeRF-SLAM run hardly in real-time.
NeRF-SLAM is claiming to be real-time, which is questionable as they still need hundreds of epochs training after feeding the data.

Nevertheless, optimization with backpropagation learns pixel-to-pixel well.
It is theoretically advanced for a regression-and-fusion strategy. 
Though Uni-Fusion demonstrates its high capability to model the color, NeRF-like post-optimization would still be a good direction for further improvements of Uni-Fusion.

\newcommand{\replicaImSize}{.24}
\begin{figure*}[t]
	\centering
	\setlength{\tabcolsep}{0.1em}
	\renewcommand{\arraystretch}{.1}
	\begin{tabular}{|c | c |c |c| }
		 \hline
		{\Large{NICE-SLAM}} &{\Large{NeRF-SLAM}}&\textbf{\Large{Uni-Fusion}}&\Large{Ground Truth}\\
		%		\hline
		%		\includegraphics[width=\replicaImSize\linewidth]{im/exp/recons/replica/nice-slam/of2_1286.png} &
		%		\includegraphics[width=\replicaImSize\linewidth]{im/exp/recons/replica/mine/of2_1286.jpg} &
		%		\includegraphics[width=\replicaImSize\linewidth]{im/exp/recons/replica/mine/of2_1286.jpg} &
		%		\includegraphics[width=\replicaImSize\linewidth]{im/exp/recons/replica/gt/of2_1286.jpg} \\
		
		\hline
		\includegraphics[width=\replicaImSize\linewidth]{im/exp/recons/replica/nice-slam/rm0_769_window.png} &
		\includegraphics[width=\replicaImSize\linewidth]{im/exp/recons/replica/nerf-slam/rm0_769_window.jpg} &
		\includegraphics[width=\replicaImSize\linewidth]{im/exp/recons/replica/mine/rm0_769_window.jpg} &
		\includegraphics[width=\replicaImSize\linewidth]{im/exp/recons/replica/gt/rm0_769_window.jpg} \\
		\hline
		\includegraphics[width=\replicaImSize\linewidth]{im/exp/recons/replica/nice-slam/of3_575_window.png} &
		\includegraphics[width=\replicaImSize\linewidth]{im/exp/recons/replica/nerf-slam/of3_575_window.jpg} &
		\includegraphics[width=\replicaImSize\linewidth]{im/exp/recons/replica/mine/of3_575_window.jpg} &
		\includegraphics[width=\replicaImSize\linewidth]{im/exp/recons/replica/gt/of3_575_window.jpg} \\
		\hline
		\includegraphics[width=\replicaImSize\linewidth]{im/exp/recons/replica/nice-slam/rm1_425_window.png} &
		\includegraphics[width=\replicaImSize\linewidth]{im/exp/recons/replica/nerf-slam/rm1_425_window.jpg} &
		\includegraphics[width=\replicaImSize\linewidth]{im/exp/recons/replica/mine/rm1_425_window.jpg} &
		\includegraphics[width=\replicaImSize\linewidth]{im/exp/recons/replica/gt/rm1_425_window.jpg} \\
		\hline
		\includegraphics[width=\replicaImSize\linewidth]{im/exp/recons/replica/nice-slam/rm2_1085_window.png} &
		\includegraphics[width=\replicaImSize\linewidth]{im/exp/recons/replica/nerf-slam/rm2_1085_window.jpg} &
		\includegraphics[width=\replicaImSize\linewidth]{im/exp/recons/replica/mine/rm2_1085_window.jpg} &
		\includegraphics[width=\replicaImSize\linewidth]{im/exp/recons/replica/gt/rm2_1085_window.jpg} \\		
		
	\end{tabular}
	%\captionof{figure}
	\caption{Demonstration of color rendering on the Replica dataset. Fine appearances are highlighted in {\color{green}green window}. Small flaws are in a {\color{red}red} box.}
	\label{fig:replica_render}
	\vspace{-.5cm}
\end{figure*}

%(2) NeRF model learning radiance field that model the light on different direction on surface. While Uni-Fusion naturally treat different directional light the same color.

%\begin{table*}[t!]
%	\centering
%	\setlength{\tabcolsep}{0.1em}
%	\renewcommand{\arraystretch}{.1}
%	\begin{tabular}{c | c |c |c |c }
%		\hline 
%		\rotatebox{90}{\large{NICE-SLAM}} &
%		\includegraphics[width=\replicaImSize\linewidth]{im/exp/recons/replica/nice-slam/of3_575.png} &
%		\includegraphics[width=\replicaImSize\linewidth]{im/exp/recons/replica/nice-slam/rm0_769.png} &
%		\includegraphics[width=\replicaImSize\linewidth]{im/exp/recons/replica/nice-slam/rm1_425.png} &
%		\includegraphics[width=\replicaImSize\linewidth]{im/exp/recons/replica/nice-slam/rm2_1085.png} \\
%		\hline
%		\rotatebox{90}{\large{NeRF-SLAM}} &
%		\includegraphics[width=\replicaImSize\linewidth]{im/exp/recons/replica/mine/of3_575.jpg} &
%		\includegraphics[width=\replicaImSize\linewidth]{im/exp/recons/replica/mine/rm0_769.jpg} &
%		\includegraphics[width=\replicaImSize\linewidth]{im/exp/recons/replica/mine/rm1_425.jpg} &
%		\includegraphics[width=\replicaImSize\linewidth]{im/exp/recons/replica/mine/rm2_1085.jpg} \\	
%		\hline
%		\rotatebox{90}{\textbf{\Large{Uni-Fusion}}} &
%		\includegraphics[width=\replicaImSize\linewidth]{im/exp/recons/replica/mine/of3_575.jpg} &
%		\includegraphics[width=\replicaImSize\linewidth]{im/exp/recons/replica/mine/rm0_769.jpg} &
%		\includegraphics[width=\replicaImSize\linewidth]{im/exp/recons/replica/mine/rm1_425.jpg} &
%		\includegraphics[width=\replicaImSize\linewidth]{im/exp/recons/replica/mine/rm2_1085.jpg} \\
%		\hline
%		\rotatebox{90}{\large{Ground Truth}} &
%		\includegraphics[width=\replicaImSize\linewidth]{im/exp/recons/replica/gt/of3_575.jpg} &
%		\includegraphics[width=\replicaImSize\linewidth]{im/exp/recons/replica/gt/rm0_769.jpg} &
%		\includegraphics[width=\replicaImSize\linewidth]{im/exp/recons/replica/gt/rm1_425.jpg} &
%		\includegraphics[width=\replicaImSize\linewidth]{im/exp/recons/replica/gt/rm2_1085.jpg} \\
%		\hline		
%		
%	\end{tabular}
%	\captionof{figure}{Demonstration of color rendering on Replica dataset.}
%\end{table*}

\subsection{Ablation study}
\label{exp:surface:ablation}

\begin{figure}[]
	\centering
%		\subfloat[width=\textwidth][Sample based]{
%		\centering
%		\includegraphics[width=.7\linewidth]{im/exp/ablation/GPIS/seq3_sample_color.png}
%	}\\
%	\subfloat[width=\textwidth][Derivative based]{
%		\centering
%		\includegraphics[width=.7\linewidth]{im/exp/ablation/GPIS/seq3_derivative_color.png}
%	}
		\includegraphics[width=.7\linewidth]{im/exp/ablation/GPIS/seq3_sample_color_a.png}
		\includegraphics[width=.7\linewidth]{im/exp/ablation/GPIS/seq3_derivative_color_b.png}
	\caption{Ablation study on surface construction basis. (a) Sample based. (b) Derivative based.}
	\label{fig:ablation:GPIS}
\end{figure}


\begin{table}[]
	\caption{Ablation study on tracking.
	}
	\centering
	\footnotesize
	\setlength{\tabcolsep}{0.7em}
	\resizebox{\linewidth}{!}{
		\begin{tabular}{l|ccc}
			\hline
			& \tt{fr1/desk} &  \tt{fr2/xyz} &  \tt{fr3/office} \\
			\hline
			External& 2.1& 0.5& 2.5 \\
			External+Internal&1.8& 0.5& 2.1 \\
			\hline
	\end{tabular}}
	\vspace{-2pt}
	%\vspace{-1cm}
	\label{tab:tum_rmse2}
\end{table}

\begin{figure}
	\centering
	\includegraphics[width=.7\linewidth]{im/exp/ablation/voxel_size/seq_voxel_size.png}
	%	\subfloat[width=.33\textwidth][0.1]{
		%		\centering
		%		\includegraphics[width=.33\linewidth]{im/exp/ablation/voxel_size/seq3_0_1_color.png}
		%	}
	%	\subfloat[width=.3\textwidth][0.05]{
		%		\centering
		%		\includegraphics[width=.33\linewidth]{im/exp/ablation/GPIS/seq3_sample_color.png}
		%	}
	%	\subfloat[width=.33\textwidth][0.02]{
		%	\centering
		%	\includegraphics[width=.33\linewidth]{im/exp/ablation/voxel_size/seq3_0_02_color.png}
		%	}
	\caption{Ablation study on voxel size.}
	\label{fig:ablation:voxel_size}
\end{figure}




 \newcommand{\styleImSize}{.2}
\begin{figure*}[b!]
	\vspace{-.5cm}
	\centering
	\setlength{\tabcolsep}{0.1em}
	\renewcommand{\arraystretch}{.1}
	\resizebox{\textwidth}{!}{\begin{tabular}{ccccc}
			%		\includegraphics[width=\styleImSize\line]{im/exp/style/style/0} &
			%		\includegraphics[width=\styleImSize\linewidth]{im/exp/style/style/1} &
			%		\includegraphics[width=\styleImSize\linewidth]{im/exp/style/style/2} &
			%		\includegraphics[width=\styleImSize\linewidth]{im/exp/style/style/3} &
			%		\includegraphics[width=\styleImSize\linewidth]{im/exp/style/style/4} &
			%		\includegraphics[width=\styleImSize\linewidth]{im/exp/style/style/5} &
			%		\includegraphics[width=\styleImSize\linewidth]{im/exp/style/style/6} \\
			\hline\hline
			\includegraphics[width=\styleImSize\linewidth]{im/exp/style/processed/office0_0.png} &
			\includegraphics[width=\styleImSize\linewidth]{im/exp/style/processed/office0_1.png} &
			\includegraphics[width=\styleImSize\linewidth]{im/exp/style/processed/office0_2.png} &
			\includegraphics[width=\styleImSize\linewidth]{im/exp/style/processed/office0_3.png} &
			\includegraphics[width=\styleImSize\linewidth]{im/exp/style/processed/office0_4.png} \\
			\includegraphics[width=\styleImSize\linewidth]{im/exp/style/processed/office0_5.png} &
			\includegraphics[width=\styleImSize\linewidth]{im/exp/style/processed/office0_6.png} &
			\includegraphics[width=\styleImSize\linewidth]{im/exp/style/processed/office0_7.png} &
			\includegraphics[width=\styleImSize\linewidth]{im/exp/style/processed/office0_8.png} &
			\includegraphics[width=\styleImSize\linewidth]{im/exp/style/processed/office0_9.png} \\
			\includegraphics[width=\styleImSize\linewidth]{im/exp/style/processed/office0_10.png} &
			\includegraphics[width=\styleImSize\linewidth]{im/exp/style/processed/office0_11.png} &
			\includegraphics[width=\styleImSize\linewidth]{im/exp/style/processed/office0_12.png} &
			\includegraphics[width=\styleImSize\linewidth]{im/exp/style/processed/office0_13.png} &
			\includegraphics[width=\styleImSize\linewidth]{im/exp/style/processed/office0_14.png} \\
			\includegraphics[width=\styleImSize\linewidth]{im/exp/style/processed/office0_15.png} &
			%\includegraphics[width=\styleImSize\linewidth]{im/exp/style/processed/office0_16.png} &
			\includegraphics[width=\styleImSize\linewidth]{im/exp/style/processed/office0_17.png} &
			\includegraphics[width=\styleImSize\linewidth]{im/exp/style/processed/office0_18.png} &
			\includegraphics[width=\styleImSize\linewidth]{im/exp/style/processed/office0_19.png} &
			\includegraphics[width=\styleImSize\linewidth]{im/exp/style/processed/office0_20.png} 
			\\	\hline
		\end{tabular}
	}
	%\captionof{figure}
	\caption{Style transfer on 3D canvas.}
	\label{fig:style}
\end{figure*}


\subsubsection{ Sample-based or Derivative-based}

We select the surface model with our own captured sequences. 
All settings are detailed in \cref{sec:exp:details}.
As shown in~\cref{fig:ablation:GPIS}, reconstruction of Yijun's office is demonstrated. 
Both models are able to construct, but the derivative-based model produces a lot of noise on the surface.
This is because for smoothness purpose, we build voxels that are overlapped to its neighbor, which causes redundant voxels near the surface.
For those redundant voxels, no center sample is provided and thus the derivative based surface construction builds bad SDFs on unknow region of the voxels.


Instead, sample-based surface construction does not have this problem as it adds more points in voxels and is able to construct highly-smooth surfaces.
From which, we find well constructed and colored white board, chair, school bag and even the oranges.

\subsubsection{Tracking}


Our Uni-Fusion use a coarse-to-fine strategy for tracking. 
An external tracking model is running in one thread aside from the mapping thread.
In the mapping thread, it takes pose result from the external tracking and applies the internal tracking for colored point cloud.

The result is demonstrated in~\cref{tab:tum_rmse2}. 
The coarse-to-fine is relatively better on trajectory estimation.

\subsubsection{Voxel size}

Testing the office scene, we vary the voxel size from low to high. 
From~\cref{fig:ablation:voxel_size}, when low voxel size $0.02$m is used, the surface is rough.
Then when voxel size goes larger, the smoothness is improved.
However, when we use $0.1$m voxel size, the surface color is blur. 
Considering Uni-Fusion produces a surface color field, the quality of surface directly affect the coloring.
Thus, continuing enlarging the voxel size also results in worse surface results.

Therefore, in the above experiments, $0.05$m voxel size is utilized for surface construction.
In addition, each voxel for encoding are actually with size $0.1$m, since we use overlapped voxel.

%\subsubsection{Anchor number and feature dimension}


\subsection{Application: 2D-to-3D Transfer}
\label{sec:fabircated_prop}

Applications such as 2) and 4) can be easily integrated with application 1) incremental reconstruction (\cref{sec:incremental_reconstruction}) by incorporating the fabricated result together with the point cloud.
%
For instance, given RGB-D frames, we detect saliency or transfer image styles to generate a fabricated $X$ image. Here, $X$ represents saliency, style, or other properties. 
By combining $X$ with depth information through unprojection,
we assign
the fabricated values to corresponding points, resulting in point pairs ($\V X$, $\V Q_{X}$).

Similar to the reconstruction pipeline in~\cref{fig:recons_and_scene_understanding}, we employ encoding (\cref{sec:encoder}) and fusion (\cref{eq:fuse}) to construct a global LIM for the fabricated properties $X$.
This global LIM represents a surface $X$ fields that is utilized for subsequent inference.

While it is possible to similarly transfer a 2D semantic image to 3D,
it may not be feasible in practice due to the need for multiple passes of different categories of semantic information 
 on the same dataset (such as object, usability, etc.).
Therefore, in the following section, we demonstrate the construction of a surface feature field for scene understanding application that satisfies various 
requirements through a single mapping pass.

\begin{table*}[b!]
	%\renewcommand{\arraystretch}{0.9}
	%\setlength{\tabcolsep}{3pt}
	\caption{GZSL semantic segmentation results. Scores are in \%.
	  $^\dagger$ indicate 3DGenZ's adaption of the method.
       Note that, Uni-Fusion-SU does not even train with the seen classes.}
	\centering
	\begin{tabular}{l|c|c |c ||ccc|ccc}
		\toprule
		\multicolumn{1}{c}{}& \multicolumn{2}{c|}{Training set} & Inference input &\multicolumn{3}{c|}{ScanNet } & \multicolumn{3}{c}{S3DIS}\\
		& Backbone & Classifier & &$Seen$& $Unseen$ & $All$&$Seen$& $Unseen$ & $All$
%		\multicolumn{3}{c|}{mIoU} & 
%		\multicolumn{3}{c|}{mIoU} \\ 
%		&&&& $Seen$& $Unseen$ & $All$&$Seen$& $Unseen$ & $All$
		%\cellcolor{white}{}  
		%\cellcolor{white}{}
		\\
		\midrule
		
		\multicolumn{5}{l}{\textit{Supervised methods with different levels of supervision}}\\
		
		Full supervision & $seen \cup unseen$ & $seen \cup unseen$ & Point Cloud &43.3&51.9 &45.1&74.0&50.0&66.6 \\
		
		ZSL backbone & $seen$ & $seen \cup unseen$  &Point Cloud&41.5&39.2 & 40.3&60.9& 21.5&  48.7 \\
		
		ZSL-trivial & $seen$ & $seen$ &Point Cloud&39.2&0.0&31.3&70.2 &0.0&48.6  \\
		\midrule
		\multicolumn{5}{l}{\textit{Generalized zero-shot-learning methods}}\\
		
		ZSLPC-Seg~\cite{cheraghian2019zero}$^\dagger$ & $seen$ & $unseen$  &Point Cloud&28.2&0.0& 22.6&65.6 &0.0& 45.3\\
		
		DeViSe-3DSeg~\cite{frome2013devise}$^\dagger$ & $seen$ & $unseen$   &Point Cloud &20.0&0.0&16.0&70.2&0.0& 48.6\\ 
		%ZSLPC-Seg~\cite{cheraghian2019zero}$^\dagger$ & $seen$ & $unseen$  &  4.0&13.9\\
		%DeViSe-3DSeg~\cite{frome2013devise}$^\dagger$ & $seen$ & $unseen$   &  3.0&10.9\\
		3DGenZ~\cite{michele2021generative} & $seen$ & $seen \cup \hat{unseen}$  &Point Cloud &32.8&7.7& {27.8}&53.1&7.3&   \textbf{39.0} \\
		\midrule
		\multicolumn{5}{l}{\textit{Zero-shot learning + map fusion}}\\
		Uni-Fusion-SU (Ours) &None&None&Sparse Frames&31.0&\textbf{41.9}&\textbf{32.9} &31.3&\textbf{24.0}&29.0\\
		\bottomrule
		\multicolumn{1}{l}{}\\[-7pt]
	\end{tabular}

	\label{tab:sem_seg_overview}
\end{table*}

\begin{figure*}[t!]
	\centering
	\setlength{\tabcolsep}{0.1em}
	\renewcommand{\arraystretch}{.1}
	\begin{tabular}{|c | c |c |||c |c | c|}
		\toprule
		{\Large{3DGenZ}} & {\Large{Uni-Fusion}} &{\Large{Ground Truth}} & {\Large{3DGenZ}} &{\Large{Uni-Fusion-SU}} & {\Large{Ground Truth}} \\ \midrule
		
		\includegraphics[width=\scannetImSize\linewidth]{im/exp//ss/gen3dz_0568.png}
		&\includegraphics[width=\scannetImSize\linewidth]{im/exp//ss/mine_0568.png}
		&\includegraphics[width=\scannetImSize\linewidth]{im/exp//ss/gt_0568.png}
		&		\includegraphics[width=\scannetImSize\linewidth]{im/exp//ss/gen3dz_0164.png}
		&\includegraphics[width=\scannetImSize\linewidth]{im/exp//ss/mine_0164.png}
		&\includegraphics[width=\scannetImSize\linewidth]{im/exp//ss/gt_0164.png}\\
		
		
		\includegraphics[width=\scannetImSize\linewidth]{im/exp//ss/gen3dz_0249.png}
		&\includegraphics[width=\scannetImSize\linewidth]{im/exp//ss/mine_0249.png}
		&\includegraphics[width=\scannetImSize\linewidth]{im/exp//ss/gt_0249.png}
		&		\includegraphics[width=\scannetImSize\linewidth]{im/exp//ss/gen3dz_0435.png}
		&\includegraphics[width=\scannetImSize\linewidth]{im/exp//ss/mine_0435.png}
		&\includegraphics[width=\scannetImSize\linewidth]{im/exp//ss/gt_0435.png}\\
		
		
		\includegraphics[width=\scannetImSize\linewidth]{im/exp//ss/gen3dz_0046.png}
		&\includegraphics[width=\scannetImSize\linewidth]{im/exp//ss/mine_0046.png}
		&\includegraphics[width=\scannetImSize\linewidth]{im/exp//ss/gt_0046.png}
		&		\includegraphics[width=\scannetImSize\linewidth]{im/exp//ss/gen3dz_0050.png}
		&\includegraphics[width=\scannetImSize\linewidth]{im/exp//ss/mine_0050.png}
		&\includegraphics[width=\scannetImSize\linewidth]{im/exp//ss/gt_0050.png}\\
		\bottomrule
		
	\end{tabular}
	\includegraphics[width=\linewidth]{im/ss_colorbar}
	%\captionof{figure}
	\caption{Demonstration of semantic segmentation on the ScanNet dataset.
       Selected scenes are consistent with~\cref{fig:recons:scannet_demo}}
	\label{fig:segmentation_demo}
	
\end{figure*}

\subsection{Scene Understanding Results}

Saliency detection effectively highlights the objects of interest.
This is also considered part of 3D semantic understanding.
However, as the semantics categories vary, fusing different categories of semantics into multiple LIMs can be inefficient.
%
Therefore, in this section, we utilize Uni-Fusion to fuse and construct a surface field for high-dimensional CLIP embeddings.
With a single LIM, we can generate different semantic results based on corresponding commands.
%
Since now our Uni-Fusion works with OpenSeg for scene understanding purposes, we call it Uni-Fusion-SU.

\subsubsection{Semantic Segmentation}
\label{sec:exp:semantic}

We first evaluate our model on generalized zero-shot point cloud semantic segmentation application.
Generalized Zero-Shot Learning (GZSL) differs from Zero-Shot Learning (ZSL) in that ZSL only predicts classes unseen during training, while GZSL predicts both unseen and seen classes~\cite{michele2021generative}.
Therefore, comparing our results with GZSL SOTAs provides a better understanding of the potential of Uni-Fusion-SU, as it does not train on both seen and unseen. 

This test uses ScanNet and S3DIS datasets for benchmarking. 
It is important to note that the \textbf{compared baselines are trained on the corresponding datasets}.
Our Uni-Fusion-SU uses OpenSeg to provide the 2D image level feature ebmedding.
Although \textbf{Uni-Fusion-SU} is also zero-shot, \textbf{it does not touch any ScanNet or S3DIS annotations}.

We demonstrate the mIoU scores in~\cref{tab:sem_seg_overview}.
In particular, our model achieves best results among the zero-shot learning methods on the ScanNet dataset and remains competitive with fully supervised methods.

Furthermore, we provide results specifically for the unseen classes in~\cref{sup:tab:sn_acc_miou}.
Although not as good as the fully supervised approach, Uni-Fusion-SU performs much better than 3DGenZ.
In addition, our Uni-Fusion-SU demonstrates high precision in classes such as sofa and Toilet, even when compared to the fully supervised model.

\begin{table}[htbp]
		\caption{Classwise GZSL semantic segmentation performance (\%) on the ScanNet unseen split.}
	\centering
	\newcommand*\rotext{\multicolumn{1}{R{45}{1em}}}
	\setlength{\tabcolsep}{1.7pt}
	\begin{tabular}{@{}l@{~}c|rrrr|r@{}}
		\toprule		
		& &
		{\textbf{Bookshelf}} & {\textbf{Desk}} & {\textbf{Sofa}} & {\textbf{Toilet}} & \stackbox{mean} \\
		
		\midrule
		FSL (Fully supervise) & IoU & 	56.9&	30.0&	57.4&	63.4 & 51.9
		\\ 
		3DGenZ (Zero-shot) & IoU & 	6.3&	3.3&	13.1&	8.1 & 7.7
		\\
		Uni-Fusion-SU (Ours) & IoU &38.3&16.8&51.7&60.9&41.9
	\\ \midrule 
	3DGenZ (Zero-shot)& Acc. & 	13.4&	5.9&	49.6&	26.3 &23.8
	\\
	Uni-Fusion-SU (Ours) & Acc. &61.9&29.6&67.4&91.6& 62.6
		\\
		\bottomrule
	\end{tabular}

	\label{sup:tab:sn_acc_miou}
\end{table}

However, in the S3DIS dataset, our model does not outperform 3DGenZ and other methods as shown in~\cref{tab:sem_seg_overview}.

Even in the result of unsceened data, as presented in \cref{sup:tab:s3dis_acc_miou}, we observe that Uni-Fusion-SU hardly finds some classed, e.g. Beam and Column, which are not commonly annotated objects. 
However, for common objects like sofa and window, our model performs much better.

\begin{table}[htbp]
		\caption{Classwise GZSL semantic segmentation performance (\%) on the S3DIS unseen split.}
	\centering
	\newcommand*\rotext{\multicolumn{1}{R{45}{1em}}}
	\setlength{\tabcolsep}{1.7pt}
	\begin{tabular}{@{}l@{~}c|rrrr|r@{}}
		\toprule		
		& &
		{\textbf{Beam}} & {\textbf{Column}} & {\textbf{Sofa}} & {\textbf{Window}} & \stackbox{mean} \\
		
		\midrule
		FSL (Fully supervise) & IoU & 	63.1&	10.2&	54.1&	72.4 & 50.0
		\\ 
		3DGenZ (Zero-shot) & IoU & 	13.9&	2.4&4.9&	8.1 &7.3
		\\
		Uni-Fusion-SU (Ours) & IoU &5.5&0.02&57.4&32.9&	24.0
		\\ \midrule 
		3DGenZ (Zero-shot) & Acc. & 	20.0&	9.1&	62.4&	23.7 &28.8
		\\
		Uni-Fusion-SU (Ours) & Acc. &41.5&0.02&78.3&42.1& 40.5
		\\	
		\bottomrule
	\end{tabular}

	\label{sup:tab:s3dis_acc_miou}
\end{table}

We present the results of the semantic segmentation in~\cref{fig:segmentation_demo}. 
It is evident that, 3DGenZ's result contains more noise, as seen in the spotted sofa, bed and other objects.
Conversely, Uni-Fusion-SU's results are generally smoother and more precise.

%
%\begin{figure*}[htbp]
%	\centering
%	\includegraphics[width=.3\linewidth]{example-image-golden}
%	\includegraphics[width=.3\linewidth]{example-image-golden}
%	\includegraphics[width=.3\linewidth]{example-image-golden}
%	\\
%	\includegraphics[width=.3\linewidth]{example-image-golden}
%	\includegraphics[width=.3\linewidth]{example-image-golden}
%	\includegraphics[width=.3\linewidth]{example-image-golden}
%	
%	\caption{Semantic segmentation result on ScanNet.}
%\end{figure*}
%
%\begin{figure*}[htbp]
%	\centering
%	\includegraphics[width=.3\linewidth]{example-image-golden}
%	\includegraphics[width=.3\linewidth]{example-image-golden}
%	\includegraphics[width=.3\linewidth]{example-image-golden}
%	\\
%	\includegraphics[width=.3\linewidth]{example-image-golden}
%	\includegraphics[width=.3\linewidth]{example-image-golden}
%	\includegraphics[width=.3\linewidth]{example-image-golden}
%	
%	\caption{Semantic segmentation result on S3DIS.}
%\end{figure*}

\subsubsection{Scene Understanding with Different Properties}

\begin{figure*}[t!]
	\centering
	\setlength{\tabcolsep}{0.1em}
	\renewcommand{\arraystretch}{.1}
	\resizebox{\textwidth}{!}{\begin{tabular}{|c | c | c | c | c | c|}
			\toprule 
			& \textbf{scene0568\_00} & \textbf{scene0249\_00} & \textbf{scene0435\_00} & \textbf{office3} & \textbf{room0}\\
			\midrule
			{} &
			\raisebox{-.5\height}{\includegraphics[width=\fabImSize\linewidth]{im/exp/fab/scannet/0568_color.png}} & %\raisebox{-.5\height}{\includegraphics[width=\fabImSize\linewidth]{im/exp/fab/scannet/0164_color.png}} &
			\raisebox{-.5\height}{\includegraphics[width=\fabImSize\linewidth]{im/exp/fab/scannet/0249_color.png}} & \raisebox{-.5\height}{\includegraphics[width=\fabImSize\linewidth]{im/exp/fab/scannet/0435_color.png}}
			&
			\raisebox{-.5\height}{\includegraphics[width=\fabImSize\linewidth]{im/exp/fab/replica/office3_color.png}}
			&
			\raisebox{-.5\height}{\includegraphics[width=\fabImSize\linewidth]{im/exp/fab/replica/room0_color.png}}\\ %\raisebox{-.5\height}{\includegraphics[width=\fabImSize\linewidth]{im/exp/fab/scannet/0050_color.png}} %\includegraphics[width=\fabImSize\linewidth]{im/exp/fab/replica/office3_color.png}
			\\
			\textbf{Desk}  &
			\raisebox{-.5\height}{\includegraphics[width=\fabImSize\linewidth]{im/exp/fab/scannet/0568_lt_desk.png}}&
			%\raisebox{-.5\height}{\includegraphics[width=\fabImSize\linewidth]{im/exp/fab/scannet/0164_lt_desk.png}}&
			\raisebox{-.5\height}{\includegraphics[width=\fabImSize\linewidth]{im/exp/fab/scannet/0249_lt_desk.png}}&
			\raisebox{-.5\height}{\includegraphics[width=\fabImSize\linewidth]{im/exp/fab/scannet/0435_lt_desk.png}}
			&
			\raisebox{-.5\height}{\includegraphics[width=\fabImSize\linewidth]{im/exp/fab/replica/office3_lt_desk.png}}
			&
			\raisebox{-.5\height}{\includegraphics[width=\fabImSize\linewidth]{im/exp/fab/replica/room0_lt_desk.png}}\\
			%\raisebox{-.5\height}{\includegraphics[width=\fabImSize\linewidth]{im/exp/fab/scannet/0050_lt_desk.png}}
			%\includegraphics[width=\fabImSize\linewidth]{im/exp/fab/replica/office3_saliency.png}
			\\
			
			\textbf{Sofa} &
			\raisebox{-.5\height}{\includegraphics[width=\fabImSize\linewidth]{im/exp/fab/scannet/0568_lt_sofa.png}} &
			%\raisebox{-.5\height}{\includegraphics[width=\fabImSize\linewidth]{im/exp/fab/scannet/0164_lt_sofa.png}} &
			\raisebox{-.5\height}{\includegraphics[width=\fabImSize\linewidth]{im/exp/fab/scannet/0249_lt_sofa.png}} &
			\raisebox{-.5\height}{\includegraphics[width=\fabImSize\linewidth]{im/exp/fab/scannet/0435_lt_sofa.png}}&
			\raisebox{-.5\height}{\includegraphics[width=\fabImSize\linewidth]{im/exp/fab/replica/office3_lt_sofa.png}}
			&
			\raisebox{-.5\height}{\includegraphics[width=\fabImSize\linewidth]{im/exp/fab/replica/room0_lt_sofa.png}}\\
			%\raisebox{-.5\height}{\includegraphics[width=\fabImSize\linewidth]{im/exp/fab/scannet/0050_lt_sofa.png}}
			%\includegraphics[width=\fabImSize\linewidth]{im/exp/fab/replica/office3_style.png}
			\\
			\textbf{Work} &
			\raisebox{-.5\height}{\includegraphics[width=\fabImSize\linewidth]{im/exp/fab/scannet/0568_lt_work.png}} &
			%\raisebox{-.5\height}{\includegraphics[width=\fabImSize\linewidth]{im/exp/fab/scannet/0164_lt_work.png}} &
			\raisebox{-.5\height}{\includegraphics[width=\fabImSize\linewidth]{im/exp/fab/scannet/0249_lt_work.png}} &
			\raisebox{-.5\height}{\includegraphics[width=\fabImSize\linewidth]{im/exp/fab/scannet/0435_lt_work.png}}&
			\raisebox{-.5\height}{\includegraphics[width=\fabImSize\linewidth]{im/exp/fab/replica/office3_lt_work.png}}
			&
			\raisebox{-.5\height}{\includegraphics[width=\fabImSize\linewidth]{im/exp/fab/replica/room0_lt_work.png}}\\
			%\raisebox{-.5\height}{\includegraphics[width=\fabImSize\linewidth]{im/exp/fab/scannet/0050_lt_work.png}}
			%\includegraphics[width=\fabImSize\linewidth]{im/exp/fab/replica/office3_style.png}
			\\
			\textbf{Sittable} &
			\raisebox{-.5\height}{\includegraphics[width=\fabImSize\linewidth]{im/exp/fab/scannet/0568_lt_sit.png}} &
			%\raisebox{-.5\height}{\includegraphics[width=\fabImSize\linewidth]{im/exp/fab/scannet/0164_lt_sit.png}} &
			\raisebox{-.5\height}{\includegraphics[width=\fabImSize\linewidth]{im/exp/fab/scannet/0249_lt_sit.png}} &
			\raisebox{-.5\height}{\includegraphics[width=\fabImSize\linewidth]{im/exp/fab/scannet/0435_lt_sit.png}}&
			\raisebox{-.5\height}{\includegraphics[width=\fabImSize\linewidth]{im/exp/fab/replica/office3_lt_sit.png}}
			&
			\raisebox{-.5\height}{\includegraphics[width=\fabImSize\linewidth]{im/exp/fab/replica/room0_lt_sit.png}}\\
			%\raisebox{-.5\height}{\includegraphics[width=\fabImSize\linewidth]{im/exp/fab/scannet/0050_lt_sit.png}}
			%\includegraphics[width=\fabImSize\linewidth]{im/exp/fab/replica/office3_style.png}
			\\
			\textbf{Wood} &
			\raisebox{-.5\height}{\includegraphics[width=\fabImSize\linewidth]{im/exp/fab/scannet/0568_lt_wood.png}} &
			%\raisebox{-.5\height}{\includegraphics[width=\fabImSize\linewidth]{im/exp/fab/scannet/0164_lt_wood.png}} &
			\raisebox{-.5\height}{\includegraphics[width=\fabImSize\linewidth]{im/exp/fab/scannet/0249_lt_wood.png}} &
			\raisebox{-.5\height}{\includegraphics[width=\fabImSize\linewidth]{im/exp/fab/scannet/0435_lt_wood.png}}&
			\raisebox{-.5\height}{\includegraphics[width=\fabImSize\linewidth]{im/exp/fab/replica/office3_lt_wood.png}}
			&
			\raisebox{-.5\height}{\includegraphics[width=\fabImSize\linewidth]{im/exp/fab/replica/room0_lt_wood.png}}\\
			%\raisebox{-.5\height}{\includegraphics[width=\fabImSize\linewidth]{im/exp/fab/scannet/0050_lt_wood.png}}
			%\includegraphics[width=\fabImSize\linewidth]{im/exp/fab/replica/office3_style.png}
			\\
			
			
			\bottomrule
		\end{tabular}
	}
	%\captionof{figure}
	\caption{Demonstration of the original mesh, highlighted semantic mesh given various queries.}
	\label{fig:fab_lt}
	\vspace{-.5cm}
\end{figure*}

The main contribution of this application is that, Uni-Fusion is the first model to construct a continuous mapping of high-dimensional embeddings onto the surface without the need for any training of the map representation.
%
In the previous experiment (\cref{sec:exp:semantic}), we evaluate the performance of generalized zero-shot semantic segmentation.
However, the potential of Uni-Fusion goes beyond semantic segmentation.
%
By constructing a LIM, we obtain a surface CLIP feature field.
This enables us to query various semantic categories such as 
%without the need of multiple LIMs or rerun for other properties, we query 
\textbf{Object, Room Type, Material, Affordance and Activity} without requiring multiple LIMs or re-running the model.

We present the results in \cref{fig:fab_lt}, 
where we query object (desk, sofa), activity (work), affordance (sittable), and material (wood).
Uni-Fusion-SU accurately identifies and highlights the object and material regions.
However, for less specific commands such as work or sittable, the model provides a wider range of results with less confidence (indicated by dull yellow).
Nevertheless, the suggested options are also roughly correct.









\subsection{Time}

We run all of the applications in a single pass using our captured office sequences and evaluate the time cost of construction and fusion of each LIM. 
The average time cost across frames is shown in~\cref{tab:time}.

\begin{table}[htbp]
	\caption{Time required for each frame.
	}
	\centering
	\footnotesize
	\setlength{\tabcolsep}{0.7em}
	\resizebox{\linewidth}{!}{
		\begin{tabular}{l|ccccccc}
			\toprule
			&Surface & Color & Infrared & Style & Saliency & Latent&Internal Track \\ \midrule
			Time ($\si{\second}$)&0.100 & 0.038 & 0.045 & 0.048 & 0.045 &0.011 &0.225 \\ \bottomrule
	\end{tabular}}
	
	\label{tab:time}
\end{table}

\newcommand{\mineImSize}{.32}
%\begin{table*}[t!]
%	\centering
%	\setlength{\tabcolsep}{0.1em}
%	\renewcommand{\arraystretch}{.1}
%	\begin{tabular}{|c | c |c |}
%		\hline 
%		{Color} &{Infrared} & {Saliency} \\
%		\includegraphics[width=\mineImSize\linewidth]{im/exp/fab/mine/office/seq3_color.png} &
%		\includegraphics[width=\mineImSize\linewidth]{im/exp/fab/mine/office/seq3_color.png} &
%		\includegraphics[width=\mineImSize\linewidth]{im/exp/fab/mine/office/seq3_saliency.png} \\
%		{Style 1}&{Style 1}&{Style 1}\\
%		\includegraphics[width=\mineImSize\linewidth]{im/exp/fab/mine/office/seq3_style.png}&
%		\includegraphics[width=\mineImSize\linewidth]{im/exp/fab/mine/office/seq3_style.png}&
%		\includegraphics[width=\mineImSize\linewidth]{im/exp/fab/mine/office/seq3_style.png}\\
%		{Sofa}&{Desk}&{Soft}\\
%		\includegraphics[width=\mineImSize\linewidth]{im/exp/fab/mine/office/seq3_lt_sofa.png}&
%		\includegraphics[width=\mineImSize\linewidth]{im/exp/fab/mine/office/seq3_lt_desk.png}&
%		\includegraphics[width=\mineImSize\linewidth]{im/exp/fab/mine/office/seq3_lt_soft.png}\\		
%		\hline
%	\end{tabular}
%	\captionof{figure}{Demonstration on captured office data.}
%	\label{fig:mine_demo}
%\end{table*}
%\begin{figure*}[t!]
%	\centering
%	\setlength{\tabcolsep}{0.1em}
%	\renewcommand{\arraystretch}{.1}
%	\begin{tabular}{|c | c |c |}
%		\hline \hline
%		\includegraphics[width=\mineImSize\linewidth]{im/exp/fab/mine/office/seq3_w_slam_color.png}&	\includegraphics[width=\mineImSize\linewidth]{im/exp/fab/mine/office/seq3_w_slam_ir.png}&	\includegraphics[width=\mineImSize\linewidth]{im/exp/fab/mine/office/seq3_w_slam_saliency.png}\\
%		{Color} &{Infrared} & {Saliency}\\
%<<<<<<< HEAD


%=======
%		
%		\includegraphics[width=\mineImSize\linewidth]{im/exp/fab/mine/office/seq3_w_slam_style.png}
%		&\includegraphics[width=\mineImSize\linewidth]{im/exp/fab/mine/office/seq3_w_slam_lt_desk.png}
%		&\includegraphics[width=\mineImSize\linewidth]{im/exp/fab/mine/office/seq3_w_slam_lt_wood.png}\\
%		{Style} & {Object-desk} & {Material-wood} \\\hline
%>>>>>>> e014bc950c14dec9ffa1d2d7a6de9b7abfefabdd
%	\end{tabular}
%	%\captionof{figure}
%	\caption{Demonstration on captured Office data.}
%	\label{fig:office}
%\end{figure*}

\begin{figure*}[]
	\centering
	\setlength{\tabcolsep}{0.1em}
	\renewcommand{\arraystretch}{.1}
	\begin{tabular}{|c | c |c |}
	\hline \hline
	\includegraphics[width=\mineImSize\linewidth]{im/exp/fab/mine/appartment2/appartment2_color.png}&	\includegraphics[width=\mineImSize\linewidth]{im/exp/fab/mine/appartment2/appartment2_ir.png}&	\includegraphics[width=\mineImSize\linewidth]{im/exp/fab/mine/appartment2/appartment2_saliency.png}\\
		{Color} &{Infrared} & {Saliency}\\
	\includegraphics[width=\mineImSize\linewidth]{im/exp/fab/mine/appartment2/appartment2_style.png}
&\includegraphics[width=\mineImSize\linewidth]{im/exp/fab/mine/appartment2/appartment2_lt_sofa.png}
&\includegraphics[width=\mineImSize\linewidth]{im/exp/fab/mine/appartment2/appartment2_lt_desk.png}
\\
{Style} & {Object-sofa} & {Object-desk}\\
\includegraphics[width=\mineImSize\linewidth]{im/exp/fab/mine/appartment2/appartment2_lt_coat.png}
&\includegraphics[width=\mineImSize\linewidth]{im/exp/fab/mine/appartment2/appartment2_lt_sit.png}
&\includegraphics[width=\mineImSize\linewidth]{im/exp/fab/mine/appartment2/appartment2_lt_wood.png}\\
{Object-coat} & {Affordance-sit} & {Material-wood} \\\hline
	\end{tabular}
%\captionof{figure}
\caption{Demonstration on the captured apartment data.}
\label{fig:appartment}
%\vspace{-.5cm}
\end{figure*}


Using depth and property images of size $720\times1280$ as input, it is evident from the table, that our model operates at a frequency of $\sim10\si{\hertz}$ for  surface (sample mode) LIM construction and integration. 
It alse achieves a frequency of over $20\si{\hertz}$ for color, infrared, style, and saliency.
These results demonstrate the suitability of Uni-Fusion for real-time applications.

However, our internal tracking process takes around $0.225\si{\second}$ per frame, which is relatively slower compared to the mapping module. 
Nevertheless, Uni-Fusion uses external tracking to prevent tracking loss, enabling our internal tracking and mapping to operate at a lower frequency.
As a result, the entire model can be effectively applied in real-time in various scenarios.

\section{Extensive experiment on our own dataset}

In previous experiments, we evaluate the capabilities of Uni-Fusion in different applications. 
To further demonstrate its effectiveness in robotic environmental understanding, we capture our own dataset to show all applications together.

We capture two scenes: The office and apartment of the first author using a Microsoft Kinect Azure. 
%
RGB-D and infrared video are captured. After calibration, RGB, depth, infrared inputs have resolution of $720\times1280$.
Uni-Fusion tracks and reconstructs all applications in one pass.
%
While office data has been involved in ablation study (\cref{exp:surface:ablation}), we showcase all applications using the apartment dataset, as depicted in~\cref{fig:appartment}.

For better visualization, the ceiling of reconstruction is removed.
The top row of images presents the colored mesh with room details, the infrared mesh revealing the lighting effect, and the saliency reconstruction highlighting objects crucial for navigation.
Additionally, we select the second style from~\cref{fig:style} for style transfer to the apartment canvas.
%
As a result, the wooden floor in the room is colored with dark green.
The whole apartment is in a warm style.

The remaining results are generated from the surface field of the CLIP embeddings. 
We issue commands to locate objects, e.g., where is the sofa, desk and coat.
In addition, it easily identifies affordances such as being sittable.
For material, it successfully detects the wooden floor in each room.

\section{Conclusion}\label{sec:conclusion}
In this work, we focus on addressing the fundamental challenge of OOD detection tasks, which is how to fully understand the semantic discrepancy between the ID/OOD samples. We reveal that the key to success in the realistic SCOOD task is to allocate as many ID samples in the unlabeled set correctly as possible. To this end, we propose a novel uncertainty-aware optimal transport scheme that introduces class-specific energy scores as guidance for effective label assignment. Experimental results show that our method achieves better performance than previous state-of-the-art methods on SCOOD benchmarks.

\textbf{Limitations.} In addition to temperature scaling, other techniques such as feature clipping applied in ReAct~\cite{sun2021react} also enhance the performance of energy score, so how to obtain an OOD score that best fits the SCOOD task can be further explored. Moreover, a setting highly related to SCOOD has been proposed in \cite{katz2022training} and formulated as a constrained optimization problem. We will also theoretically analyze these practical OOD settings in our feature work.

% \section*{Acknowledgments}
\textbf{Acknowledgments.} 
This work is supported by National Key R\&D Program of China under Grant 2020AAA0105701, National Natural Science Foundation of China (NSFC) under Grants 61872327, Major Special Science and Technology Project of Anhui, National Natural Science Foundation of China (62033012) and Ant Group through Ant Research Intern Program.

% {\small
% \bibliographystyle{ieee_fullname}
% \bibliography{egbib}
% }
% This must be in the first 5 lines to tell arXiv to use pdfLaTeX, which is strongly recommended.
\pdfoutput=1
% In particular, the hyperref package requires pdfLaTeX in order to break URLs across lines.

\documentclass[11pt]{article}

% Remove the "review" option to generate the final version.
%\usepackage[review]{ACL2023}
\usepackage{ACL2023}

% Standard package includes
\usepackage{times}
\usepackage{latexsym}

% For proper rendering and hyphenation of words containing Latin characters (including in bib files)
\usepackage[T1]{fontenc}
% For Vietnamese characters
% \usepackage[T5]{fontenc}
% See https://www.latex-project.org/help/documentation/encguide.pdf for other character sets

% This assumes your files are encoded as UTF8
\usepackage[utf8]{inputenc}

% This is not strictly necessary, and may be commented out.
% However, it will improve the layout of the manuscript,
% and will typically save some space.
\usepackage{microtype}

% This is also not strictly necessary, and may be commented out.
% However, it will improve the aesthetics of text in
% the typewriter font.
\usepackage{inconsolata}


% If the title and author information does not fit in the area allocated, uncomment the following
%
%\setlength\titlebox{10cm}
%
% and set <dim> to something 5cm or larger.

%%%%%%%%%%%%%%%%%%%%%%%%%%%%%%%%%%
\usepackage{graphicx}
\usepackage{amsfonts}
\usepackage{amsmath}
\usepackage{bigdelim}
\usepackage{diagbox}
\usepackage{amsthm}
\usepackage{makecell}
\usepackage{mathtools}
\usepackage{booktabs}
\usepackage[shortlabels]{enumitem}
\graphicspath{ {figs/} }

\theoremstyle{remark}
\newtheorem*{question}{Question}

\newcommand{\tk}[1]{\textcolor{blue}{{#1}}}
\newcommand{\sy}[1]{\textcolor{red}{{#1}}}
\newcommand{\mg}[1]{\textcolor{purple}{{#1}}}
\newcommand{\lh}[1]{\textcolor{green}{{#1}}}
\newcommand{\lc}[1]{\textcolor{green}{{#1}}}

% Rounded color box
\definecolor{light_blue}{HTML}{cfdfff}
\usepackage[most]{tcolorbox}
\tcbset{on line, 
        boxsep=1pt, left=0pt,right=0pt,top=0pt,bottom=0pt,
        colframe=white,colback=light_blue,  
        highlight math style={enhanced}
        }

\newcommand{\quash}[1]{}  %Anything in \quash is ignored
\newcommand{\gpt}{\textsc{GPT-2}}
\newcommand{\bert}{\textsc{BERT}}
\newcommand{\bertlarge}{\textsc{BERT-large}}
\newcommand{\mask}{\texttt{[MASK]}}
\newcommand{\cls}{\texttt{[CLS]}}
\newcommand{\sep}{\texttt{[SEP]}}
\newcommand{\mat}{\texttt{mat}}
\newcommand{\id}{\texttt{id}}
\newcommand{\matl}{\texttt{mat}_{\ell \rightarrow \ell'}}
\newcommand{\matattnl}{\texttt{mat\_attn}_{\ell \rightarrow \ell'}}
\newcommand{\matffl}{\texttt{mat\_ffn}_{\ell \rightarrow \ell'}}
\newcommand{\matlnl}{\texttt{mat\_ln1\_ln2}_{\ell \rightarrow \ell'}}
\newcommand{\idl}{\texttt{id}_{\ell \rightarrow \ell'}}
\newcommand{\matlL}{\texttt{mat}_{\ell \rightarrow L}}
\newcommand{\matattnlL}{\texttt{mat\_attn}_{\ell \rightarrow L}}
\newcommand{\matfflL}{\texttt{mat\_ffn}_{\ell \rightarrow L}}
\newcommand{\matlnlL}{\texttt{mat\_ln1\_ln2}_{\ell \rightarrow L}}
\newcommand{\idlL}{\texttt{id}_{\ell \rightarrow L}}

\definecolor{blue(munsell)}{rgb}{0.0, 0.5, 0.69}
%%%%%%%%%%%%%%%%%%%%%%%%%%%%%%%%%%

\title{Jump to Conclusions: Short-Cutting Transformers\\With Linear Transformations}

% Author information can be set in various styles:
% For several authors from the same institution:
% \author{Author 1 \and ... \and Author n \\
%         Address line \\ ... \\ Address line}
% if the names do not fit well on one line use
%         Author 1 \\ {\bf Author 2} \\ ... \\ {\bf Author n} \\
% For authors from different institutions:
% \author{Author 1 \\ Address line \\  ... \\ Address line
%         \And  ... \And
%         Author n \\ Address line \\ ... \\ Address line}
% To start a seperate ``row'' of authors use \AND, as in
% \author{Author 1 \\ Address line \\  ... \\ Address line
%         \AND
%         Author 2 \\ Address line \\ ... \\ Address line \And
%         Author 3 \\ Address line \\ ... \\ Address line}

\author{Alexander Yom Din$^{1}$ ~~~~~ Taelin Karidi$^{1}$ ~~~~~ Leshem Choshen$^{1}$ ~~~~~
Mor Geva$^{2}$ 
\vspace{0.2cm} \\
$^1$Hebrew University of Jerusalem ~~~ $^2$Google Research \\
\small{\texttt{\{alexander.yomdin, taelin.karidi, leshem.choshen\}@mail.huji.ac.il}}, \small{\texttt{pipek@google.com}}}

\quash{
\author{Alexander Yom Din \\
  Hebrew University of Jerusalem \\ \texttt{alexander.yomdin@mail.huji.ac.il} \\\And
  Taelin Karidi \\
  Hebrew University of Jerusalem \\
  \texttt{taelin.karidi@mail.huji.ac.il} \\\And
  Leshem Choshen \\
  Hebrew University of Jerusalem \\ \texttt{leshem.choshen@mail.huji.ac.il} \\\And
  Mor Geva \\
  Google Research \\
  \texttt{pipek@google.com} \\}
}

\begin{document}
\maketitle



\begin{abstract}
% \vspace{-1em}
The diffusion-based generative models have achieved remarkable success in text-based image generation. However, since it contains enormous randomness in generation progress, it is still challenging to apply such models for real-world visual content editing, especially in videos. 
In this paper, we propose \texttt{FateZero}, a zero-shot text-based editing method on real-world videos without per-prompt training or use-specific mask. 
\RM{Specifically, different from a pipeline of two independent inversion and then generation stages, we find the intermediate attention maps during inversions store better structure and motion information. We thus reform them to temporally casual attention and replace them in the generation progress. To further reduce the unnecessary semantic leakage of source video and enhance the editing quality, we then remix the temporally casual attentions via the cross-attention features of the source prompt as the mask.}
To edit videos consistently, we propose several techniques based on the pre-trained models. Firstly, in contrast to the straightforward DDIM inversion technique, our approach captures intermediate attention maps during inversion, which effectively retain both structural and motion information. These maps are directly fused in the editing process rather than generated during denoising. To further minimize semantic leakage of the source video, we then fuse self-attentions with a blending mask obtained by cross-attention features from the source prompt. Furthermore, we have implemented a reform of the self-attention mechanism in denoising UNet by introducing spatial-temporal attention to ensure frame consistency.
Yet succinct, our method is the first one to show the ability of zero-shot text-driven video style and local attribute editing from the trained text-to-image model. We also have a better zero-shot shape-aware editing ability based on the text-to-video model~\cite{tuneavideo}. \RM{Besides video, our unified method also achieves state-of-the-art performance in zero-shot image editing.\chenyang{Need exp or remove the zero-shot image}} Extensive experiments demonstrate our superior temporal consistency and editing capability than previous works.
% The code will be released.
% \chenyang{emphasize: our observation at inversion time} \xiaodong{replacing the bold part to the actual pipeline: \textbf{Specifically, we work on replacing and mixing the attention maps between the inversion and generation since the self-attention map keeps the structure of the original natural image and the cross-attention is semantic-related, after remixing, we replace them in the corresponding generation steps for denoising.}}
% \footnote{Since there is no general video diffusion model is publicly available, we use one-shot video generation method~(Tune-A-Video~\cite{tuneavideo}) as the pretrained video diffusion model for zero-shot video editing\xiaodong{can be removed if we actually zero-shot on video}.}.
\end{abstract}
\section{Introduction}

The ability to reason about plans is critical for performing long-horizon tasks \citep{erol1996hierarchical, sohn2018hierarchical, sharma-etal-2022-skill}, compositional generalization \citep{corona-etal-2021-modular} and generalization to unseen tasks and environments \citep{shridhar2020alfred}.
Consider a simple long-horizon planning scenario where a robot is tasked with preparing a meal and serving it on the table. 
This presents a non-trivial planning problem since the agent needs to understand the sequence of operations required to perform the task and search for the relevant objects in the unfamiliar environment by interacting with various objects. %



Large language models have been recently shown to possess commonsense knowledge about the world such as object affordances and physical dynamics \citep{ouyang2022training,chowdhery2022palm}.
Early approaches considered text based environments and fine-tuned PLMs to predict actions given the history of past observations and actions \citep{jansen-2020-visually,micheli-fleuret-2021-language,yao-etal-2020-keep}.
Recent work has used this ability to reason about plans from text instructions in simulated household environments with simplifying assumptions such as text-only environment observations or feedback \citep{huang2022language,ahn2022can,li2022pre,logeswaran-etal-2022-shot}.


We focus on \emph{visually grounded planning} with PLMs --- the ability to adapt plans based on interaction and visual feedback from the environment.
While PLMs have strong planning commonsense priors, predictions from a PLM may not be directly realizable in the environment since the observation and action spaces are unknown.
This requires \emph{grounding} the PLM in the environment and adapting it to observe visual feedback, which is highly non-trivial.
Some prior works assume the availability of a pre-trained affordance function \citep{ahn2022can} or a success detector \citep{mirchandani2021ella}.
Notably, SayCan \citep{ahn2022can} completely decouples the PLM from observation information by selecting actions that have both high affordability (through a pre-trained affordance model) and high PLM likelihood.
Although this partially addresses the grounding problem, the use of visual feedback for action affordance alone is limited.
Often an agent must choose one of many affordable actions using information from observations.
For example, a driving agent should re-navigate and possibly turn around when encountering a ``road closed'' sign, but both turning around and driving forward are indistinguishable to SayCan because they are both affordable and the PLM is blind to observations.

Another workaround explored in prior work is translating the information in the visual observations to text using a pre-trained captioning system \citep{shridhar2021alfworld,huang2022language}.
However, it can be difficult to faithfully describe an image in words and information is lost in this inherently noisy process, which limits the information available to the planner.



Recent work shows that PLMs can be adapted for various natural language tasks by inserting tunable embeddings or soft prompts at the input of the PLM (also called prompt tuning or prefix tuning)~\citep{li-liang-2021-prefix,lester-etal-2021-power}.
This approach also extends to multi-modal understanding tasks such as image captioning \citep{mokady2021clipcap} and VQA \citep{tsimpoukelli2021multimodal} where images are encoded as soft prompts and finetuned for the target task.
Transformer based architectures have also been successfully applied to offline Reinforcement Learning in recent work \citep{chen2021decision,janner2021offline,li2022pre,reid2022can}.

Taking inspiration from these works, we propose the simple approach of embedding visual observations (`visual prompts') and \textit{directly inserting them as PLM input embeddings}.
The visual encoder and PLM are jointly trained for the target task, an approach we call \textbf{\oursfull}~(\ours).
By teaching the PLM to use observations for planning in an end to end manner, we remove the dependency on external data such as captions and affordability information that was used in prior work.
We show that this simple approach performs better than prior PLM-based planning approaches on two embodied planning benchmarks based on ALFWorld~\citep{shridhar2021alfworld} and Virtualhome~\cite{puig2018virtualhome}.



\section{Related Work}

%Here we summarize prior work on transfer learning and property inference.

%\shortsection{Transfer Learning}
%%Transfer learning reuses features learned by pre-trained models for new tasks, with the pretext that inherent similarities in the generic features will be useful for the downstream tasks and hence reducing their cost of downstream training. Specifically, the downstream model trainer will use a pre-trained upstream model as the starting point for the downstream training, with inclusion of (or replacement with) the task-specific classification layer/module. The downstream model is then trained by either updating all layers of the model (including ones reused from upstream model) or freezing some earlier layers of the reused parts as the ``feature extractor'' and only updating the rest. The latter approach is more popular as the reused feature extractors can already learn useful feature representations and the training cost is also much lower and affordable for individuals with limited computational resources. We study the vulnerability of the latter transfer learning approach in this paper. 


%\shortsection{Transfer Learning} 
Several works have demonstrated risks associated with transfer learning across a variety of attack goals. Wang et al.~\cite{wang2018great} and Yao et al.~\cite{yao2019latent} consider manipulating the upstream model such that the fine-tuned downstream models contain backdoors, misclassifying test inputs that contain predefined backdoor triggers. These transfer manipulations are tailored to their particular attack goals and cannot be applied for the property inference goal considered in this paper. Zou et al.~\cite{zou2020privacy} study the threat of membership inference attacks on transfer learning, but with normally trained upstream models.  
%\dnote{its clear that the goals are different for these attacks, but how similar are the methods?} \ynote{similarity of the methods? more details about the methods? do not know what is expected here}
%In contrast, we investigate the possibility of boosting the effectiveness of property inference by manipulating the upstream model training. % Schuster et al.~\cite{schuster2020humpty} show that the attacker can modify the corpus on which the word embedding is trained such that the downstream NLP models which use that embedding will behave abnormally.

%\shortsection{Property Inference}
The risk of property inference was introduced by Ateniese et al.~\cite{ateniese2015hacking}, % introduces the threat of inferring properties of the training data from pre-trained models, 
and several subsequent works have developed property inference (also known as distribution inference) attacks~\cite{Wang2022GroupPI, suri2022formalizing, Jurez2022BlackBoxAF, Hartmann2022DistributionIR}.
% Ganju et al.~\cite{ganju2018property} and Suri and Evans~\cite{suri2022formalizing} 
These works study property inference against normally trained models, and they launch attacks using a variety of black-box and white-box attacks. All the white-box attacks use meta-classifiers, which take the permutation-invariant representation~\cite{ganju2018property} of the model parameters as the features. We use the state-of-the-art white-box attack~\cite{suri2022formalizing} in our experiments.
%We will use the state-of-the-art white-box method proposed by Ganju et al.~\cite{ganju2018property} and later extended by suri et al.~\cite{suri2022formalizing} in this paper.
%\dnote{do we use these attacks?} 
Melis et al.~\cite{melis2019exploiting} and Zhang et al.~\cite{zhang2021leakage} focus on property inference in distributed training scenarios. In their settings, the attacker is a participant in the global model training and conducts property inference using meta-classifiers that are trained on model outputs or gradients. Similarly, Suri et al.~\cite{suri2022subject} focus on federated learning settings where the attacker is a participant (or the central server) that utilizes black-box attacks for inferring membership of data from particular subjects. %\dnote{if we use black-box attacks, explain which ones, or how ours are related to previous ones} 
For our experiments, We improve the black-box meta-classifier proposed by Zhang et al.~\cite{zhang2021leakage} using the ``query tuning'' technique in Xu et al.~\cite{xu2019detecting}. 

The closest works to ours are Chase et al.~\cite{saeed} and Chaudhari et al.~\cite{Chaudhari2022SNAPEE}, which both consider a scenario where the attacker can manipulate some of the training data of the model to induce a model that significantly increases property inference risk.
% \dnote{it enables precise property inference attacks?}.
These works assume an adversary with the ability to poison the victim's training data, while the adversary in our scenario has no access to the victim's training data, and therefore, their methods are not applicable.
% \dnote{example how different from ours, and why the methods are not applicable}
%Thus, their methods are not applicable to our transfer learning scenario.
%Their methods rely on inducing certain behavior correlated with the properties to be inferred, and thus are not applicable to our transfer learning scenario. \anote{Still a bit unclear why that is the case.}
%
There are also works similar to ours that leverage ``adversarial initializations'' for attack purposes.
% \cite{grosse2019adversarial, boenisch2021curious, wen2022fishing, fowl2021robbing}.
Grosse et al.~\cite{grosse2019adversarial} focus on scenarios where the attacker can control the parameter initialization of a model, and demonstrate that the attacker can use special initializations to damage the performance of the trained model. %This attack is orthogonal to ours.
Other works \cite{boenisch2021curious, wen2022fishing, fowl2021robbing} show that the malicious central server in a federated learning protocol can reconstruct some training samples via falsifying the global model in some training rounds and then analyzing the submitted gradients. These kinds of attacks do not apply to our transfer-learning scenario since the attacker cannot access the downstream gradients, and can only manipulate the upstream training.

\iffalse %%%%%%%%%%%%%%%%%%%%%%%%%%%%%%%%

In this section, we provide the background and also the summary of prior attacks on transfer learning (Section~\ref{sec:transfer_learning}) and property inference (Section~\ref{sec:property_inference}). Then, we introduce the closely related manipulation attacks against machine learning models to boost different privacy risks in Section~\ref{sec:active_inference_attacks}.

%\anote{Do we really need a dedicated section for this? It's barely 2 paragraphs right now.}

%\dnote{the most closely related work to ours are works that attempt to amplify inference attacks by poisoning models, the two most relevant I know of are \url{https://www.computer.org/csdl/proceedings-article/sp/2022/131600b569/1CIO8nmuota} and \url{https://arxiv.org/abs/2204.00032}, but need to look thoroughly for others. We should definitely be describing this and relating it to our work, probably in the introduction. Most of what is here is Background, but should be clear what this section is for (not muddling background and related work)}

\subsection{Transfer Learning} \label{sec:transfer_learning}
Transfer learning reuses features learned by pre-trained models for new tasks, with the pretext that inherent similarities in generic features can be useful for downstream tasks, thus reducing the cost of downstream training. Specifically, the downstream model trainer uses a pre-trained upstream model as the starting point for downstream training, with the inclusion (or replacement) of task-specific classification layers/modules. The downstream model is then trained by either updating all layers of the model (including ones reused from the upstream model) or freezing some earlier layers of the reused parts as the ``feature extractor'' and only updating the rest. The latter approach is more popular as the reused feature extractors can already learn useful feature representations and the training cost is also much lower and affordable for individuals with limited computational resources. We study the vulnerability of the latter transfer learning approach in this paper. 
%mainly in two ways:  1) all the layers (including ones reused from ) and tune the full model; the other one is to freeze some earlier layers of the model as the feature extractor and only tune the rest later layers. The second update strategy could achieve better efficiency since the frozen layers can already produce meaningful feature representations~\cite{wang2018great,yao2019latent}, and we will study the transfer learning using this strategy. 

Recently, various attacks have been proposed for the transfer learning setting, but with different attack goals from ours. Wang et al.~\cite{wang2018great} generate adversarial examples against black-box student models that transfer knowledge from publicly available teacher models without repeated queries. Yao et al.~\cite{yao2019latent} propose to manipulate the upstream model such that the downstream models derived from the upstream model contain backdoors, which would misclassify test inputs that contain some predefined backdoor triggers. Zou et al.~\cite{zou2020privacy} study the threat of membership inference attacks on transfer learning and the upstream models are trained normally. In contrast, we investigate the possibility of boosting the effectiveness of property inference by manipulating the upstream model training. Schuster et al.~\cite{schuster2020humpty} show that the attacker can modify the corpus on which the word embedding is trained such that the downstream NLP models which use that embedding will behave abnormally.

%This additionally allows model trainers to achieve satisfactory performance with limited training samples, leading to reduced computational costs. The most common approach reuses parameters in the earlier layers of the pre-trained model, either by fixing them as the feature extractor or just using them for initialization, to conduct downstream training.

\subsection{Property Inference} \label{sec:property_inference}

\shortsection{Property Inference Attacks} In property inference attacks, the adversary aims to infer some sensitive properties of some data, given a model trained on it. For example, the adversary may be interested in sensitive properties like the presence of people of a specific race in the dataset~\cite{ateniese2015hacking, melis2019exploiting}), or even be curious about the 
the statistics of the training set (e.g, the ratio of people with a specific gender~\cite{saeed, ganju2018property, suri2022formalizing, zhang2021leakage}).


Ateniese et al.~\cite{ateniese2015hacking} were the first to identify the threat of inferring properties of the training data from pre-trained models. Ganju et al.~\cite{ganju2018property} and Suri and Evans~\cite{suri2022formalizing} 
study property inference against normally trained models, and they launch attacks using white-box meta-classifiers, which utilize the permutation-invariance representation~\cite{ganju2018property} of the model parameters, while other works focus on distributed training~\cite{zhang2021leakage} where the attacker is a participant in the global model training and conducts property inference using meta-classifiers trained on model outputs. Similarly, Suri et al.~\cite{suri2022subject} focus on federated learning, where the attacker is a participant (or the central server) that utilizes black-box attacks for inferring membership of data from particular subjects. Chase et al.~\cite{saeed} propose an active property inference attack for data poisoning scenarios, which we will cover and compare to in Section~\ref{sec:active_inference_attacks}.

%The closest work to ours are by Chase et al.~\cite{saeed} and Tramer et al.~\cite{tramer2022truth}. In their work, the attacker can manipulate some of the training data of the model such that a model trained (from scratch) on the poisoned data has an increased inference risk. However, their methods are not applicable to the transfer learning scenario. 
%In this work, we will focus on the property inference in transfer learning scenarios in which the attacker releases the upstream model and infer sensitive properties of the downstream models tuned from that upstream model.
% 

\shortsection{Defenses}
Defending against property inference attacks is an open problem. There are no studies in the current literature on active adversaries, and only a couple on passive ones. Ma et. al.~\cite{ma2021nosnoop} propose a defense against property inference attacks on data batches in the  collaborative learning setting. However, adversaries in the transfer-learning setting do not have access to batch-wise gradients of the downstream trainer. Chen and Ohrimenko~\cite{chen2022protecting} utilize mechanisms that add carefully-crafted noise to features to provide theoretical guarantees against inference adversaries, but focus on query-based access to the underlying dataset, not a machine learning model trained on it. These existing defenses thus do not apply to our threat model.

%propose a framework that reduces property inference to Boolean functions of individual members, posing the ratio of members satisfying the given function in a dataset as the property. These property inference attacks have since then been proposed as distribution inference attacks~\cite{suri2022formalizing}, presenting such attacks as inferring properties of the distributions used to sample datasets, differentiating them from exact inference attacks like dataset inference~\cite{maini2021dataset}. Nearly all property inference attacks use meta-classifiers to perform inference: training models on versions of datasets with and without the target property, followed by training a meta-classifier on top of these classifiers's model representations. These representations can take several forms: using model weights themselves with permutation-invariance~\cite{ganju2018property}, or model activations or logits for a generated set of query points~\cite{xu2019detecting}. However, the capability of such approaches is limited: the most that these attacks have been shown to work is medium-sized convolutional networks on the CelebA dataset~\cite{suri2022formalizing}.


\subsection{Active Privacy Attacks} \label{sec:active_inference_attacks}
% Perhaps the closely related works to ours as ones that proactively enhance the effectiveness of privacy attacks by manipulating the model training process in certain ways~\cite{saeed, melis2019exploiting, nasr2019comprehensive, tramer2022truth}. 
%shown that the adversary can, by using proactive ways, achieve stronger attacks that infer private information from deep learning systems~\cite{nasr2019comprehensive, melis2019exploiting, tramer2022truth, saeed}. In this section, we introduce the ones that are close to ours.

In the decentralized federated learning training, by submitting specially crafted gradients to the central server, malicious agents can increase membership inference risk~\cite{nasr2019comprehensive} and property inference risks~\cite{melis2019exploiting} of other benign agents' training data. However, these attacks do not apply to transfer learning scenario, as the attacker cannot control model gradients of downstream training. In the centralized setting, researchers propose attacks to poison the victim's training data such that the impacts of attribute inference and membership inference~\cite{tramer2022truth} and property inference~\cite{saeed} attacks are amplified on the poisoned model.
The ability to poison the victim's data is a threat model orthogonal to ours, since we have no access to the victim's downstream data. While there is scope to combine such approaches for stronger attacks (albeit with stronger access assumptions), we choose to focus on the scenario with no read/write access to the victim's data.

\fi %%%%%%%%%%%%%%%%%%%%%%%%%%%%%%%%

\section{Linear Shortcut Across Blocks}
\label{sec:layer_jump}

To use a hidden representation from layer $\ell<L$ as a final representation, we propose to cast it using linear regression, while skipping the computation in-between these layers. More generally, this approach can be applied to cast any $\ell$-th hidden representation to any subsequent layer $\ell'>\ell$.


\subsection{Method}
\label{subsec:methodology_linear_shortcut}

Given a source layer $\ell$ and a target layer $\ell'$ such that $0 \leq \ell < \ell' \leq L$, our goal is to learn a mapping
%$A_{\ell', \ell} \in \mathbb{R}^{d_h \times d_h}$
from hidden representations at layer $\ell$ to those at layer $\ell'$. To this end, we first collect a set of corresponding hidden representation pairs $(h^\ell, h^{\ell'})$. Concretely, we run a set $\mathcal{T}$ of input sequences through the model, and for each input $s$, we extract the hidden representations $h_{i_s}^{\ell}, h_{i_s}^{\ell'}$, where $i_s$ is a random position in $s$.
Next, we learn a matrix $A_{\ell', \ell} \in \mathbb{R}^{d_h \times d_h}$ by fitting linear regression over $\mathcal{T}$, i.e., $A_{\ell', \ell}$ is a numerical minimizer for:
$$ A \mapsto \sum_{s \in \mathcal{T}} || A \cdot h_{i_s}^\ell - h_{i_s}^{\ell'} ||^2,$$ 
and define the mapping of a representation $h$ from layer $\ell$ to layer $\ell'$ as:
\begin{equation}
\label{eq:linear_jump}
    \matl{} (h) \coloneqq A_{\ell', \ell} \cdot h.
\end{equation}


\subsection{Baseline}
\label{subsec:baseline}

We evaluate 
% our method against 
the prevalent approach of ``reading'' hidden representations directly, without any transformation. 
Namely, the propagation of a hidden representation from layer $\ell$ to layer $\ell'$ is given by the identity function, dubbed \id{}:

$$ \idl{} (h) \coloneqq h.$$

% Notably, 
This baseline 
assumes that representations at different layers operate in the same linear space.

\subsection{Quality of Fit}
\label{subsec:experiments_r2}

We first evaluate our method by measuring how well the learned linear mappings approximate the representations at the target layer. To this end, we calculate the (coordinate-averaged) $r^2$-score of our mapping's outputs with respect to the representations obtained from a full inference pass, and compare to the same for the \id{} baseline.


\paragraph{Models.}

We use \gpt{} \cite{radford2019language}, a decoder-only auto-regressive LM, with $L = 48$, $d_h = 1600$, and \bert{} \cite{devlin-etal-2019-bert}, an encoder-only model trained with masked language modeling, with $L=24$, $d_h=1024$.
% \footnote{\label{footnote:hf}We use models and data from Huggingface \cite{wolf-etal-2020-transformers,lhoest-etal-2021-datasets}.}
%For masked token prediction, we use a masked LM head pre-trained for our \bert{} model.

% \footnote{Specifically, we use the Huggingface Transformers \cite{wolf-etal-2020-transformers} implementations of all these models.}

%\sy{We use \gpt{} \cite{radford2019language}, a decoder-only auto-regressive LM, coming in four scales; $\texttt{gpt2}$ ($L = 12$, $d_h = 768$), $\texttt{gpt2-medium}$ ($L = 24$, $d_h = 1024$), $\texttt{gpt2-large}$ ($L = 36$, $d_h = 1280$) and $\texttt{gpt2-xl}$ ($L = 48$, $d_h = 1600$). Also, we use \bert{} \cite{devlin-etal-2019-bert}, an encoder-only model trained with masked language modeling, coming in two scales;  \texttt{bert-base-uncased} ($L=12$, $d_h=768$) and \texttt{bert-large-uncased} ($L=24$, $d_h=1024$). For masked token prediction, we use masked LM heads pre-trained for our models. Specifically, we use the Huggingface Transformers \cite{wolf-etal-2020-transformers} implementations of all these models. The plots presented in this section are for $48$-layered \gpt{} and $24$-layered \bert{}.}

%\sy{We use \gpt{} \cite{radford2019language}, a decoder-only auto-regressive LM, in the Huggingface \cite{wolf-etal-2020-transformers} implementation\footnote{\url{https://huggingface.co/gpt2}}, coming in four scales; $\texttt{gpt2}$ ($L = 12$, $d_h = 768$), $\texttt{gpt2-medium}$ ($L = 24$, $d_h = 1024$), $\texttt{gpt2-large}$ ($L = 36$, $d_h = 1280$) and $\texttt{gpt2-xl}$ ($L = 48$, $d_h = 1600$). Also, we use \bert{} \cite{devlin-etal-2019-bert}, an encoder-only model trained with masked language modeling, in the Hugginface implementation, coming in two scales;  \texttt{bert-base-uncased}\footnote{\url{https://huggingface.co/bert-base-uncased}} ($L=12$, $d_h=768$) and \texttt{bert-large-uncased}\footnote{\url{https://huggingface.co/bert-large-uncased}} ($L=24$, $d_h=1024$). For masked token prediction, we use the \texttt{BertForMaskedLM} heads from Huggingface, pretrained for these models. The plots presented in this section are for $48$-layered \gpt{} and $24$-layered \bert{}.}

\paragraph{Data.}
We sample random sentences from Wikipedia,
% \footref{footnote:hf} 
collecting 9,000 (resp. 3,000) sentences for the training set $\mathcal{T}$ (resp. validation set $\mathcal{V}$).\footnote{We use sentences rather than full documents to simplify the analysis.}
%\sy{We use two data sources to evaluate our method. One is Wikiepdia \cite{lhoest-etal-2021-datasets}\footnote{\url{https://huggingface.co/datasets/wikipedia}}; we use \texttt{spaCy}\footnote{\url{https://spacy.io/}} to divide documents into sentences\footnote{We use sentences rather than full documents to simplify the analysis.}\footnote{We pick randomly a Wikipedia document and then pick randomly a sentence ending in a newline character in it. \sy{[maybe this footnote is not needed?]}}, collecting 9,000 (resp. 3,000) random sentences for the training set $\mathcal{T}$ (resp. validation set $\mathcal{V}$). The second is a news article sentences dataset, the 10K English 2020 news sentences corpus
% \footnote{\url{https://downloads.wortschatz-leipzig.de/corpora/eng_news_2020_10K.tar.gz}} from the Leipzig Corpora Collection \cite{goldhahn-etal-2012-building}, which we randomly divide into a training set $\mathcal{T}$ consisting of 9,000 examples and a validation set $\mathcal{V}$ consisting of 1,000 examples.
% We truncate sentences to the maximal token length allowed by the model \mg{do we ever need to truncate? a sentence has about 10 words and the max. input len is thousands} \sy{[I surely did not need to in Leipzig, but discovered (via a transformers runtime warning) that I do need to for some (probably a minority) of the Wikipedia sentences. This probably has to do with that it is not really ``sentences" necessarily, for example, I noticed that it has some listings or something like that (bulleted items)... So some minority might get very long I guess...]}.
For each example $s$, we select a random position $i_s$ and extract the hidden representations $h_{i_s}^{\ell}$ at that position from all the layers.
For \bert{}, we first replace the input token at position $i_s$ with a \mask{} token, as our motivation is interpreting predictions, which are obtained via masked tokens in \bert{} (see \S\ref{subsec:BERT}).
Thus, in this case, the hidden representations we consider
%in the case of \bert{}
are of \mask{} tokens only.
%As we observed highly similar results for the two data sources across all our experiments, throughout the paper we will mainly report results for Wikipedia (except for \S\ref{sec:robustness}, where we cross-validate).


\begin{figure}[t]
\includegraphics[scale=0.2]{figs/r2_scores_48.pdf}
% \includegraphics[width=\columnwidth]{figs/r2_scores_48.pdf}
\caption{The coordinate-averaged $r^2$-score of $\matl{}$ (left) and $\idl{}$ (right) (\gpt{}).}
\label{fig:r2_scores}
\end{figure}


\begin{figure}[t]
\setlength{\belowcaptionskip}{-10pt}
\includegraphics[scale=0.2]{figs/bertmask_r2_scores_24.pdf}
% \includegraphics[width=\columnwidth]{figs/bertmask_r2_scores_24.pdf}
\caption{The coordinate-averaged $r^2$-score of $\matl{}$ (left) and $\idl{}$ (right) (\bert{}).}
\label{fig:bertmask_r2_scores}
\end{figure}



\paragraph{Evaluation.}
For every pair of layers $\ell, \ell'$, such that $0 \leq \ell < \ell' \leq L$, we use the training set $\mathcal{T}$ to fit linear regression as described in \S\ref{subsec:methodology_linear_shortcut}, and obtain a mapping $\matl{}$. 
Next, we evaluate the quality of $\matl{}$ as well as of $\idl{}$ using the $r^2$-coefficient, uniformly averaged over all coordinates. Concretely, we compute the $r^2$-coefficient of each of the predicted representations $\matl{} (h_{i_s}^{\ell})$ and $\idl{} (h_{i_s}^{\ell})$ versus the true representations $h_{i_s}^{\ell'}$
over all $s \in \mathcal{V}$.
%as we vary $s \in \mathcal{V}$.
%for every $s \in \mathcal{V}$.



\paragraph{Results.}
Results for \gpt{} and \bert{} are presented in Figs.~\ref{fig:r2_scores} and~\ref{fig:bertmask_r2_scores}, respectively.
In both models, \mat{} consistently yields better approximations than \id{}, as it obtains higher $r^2$-scores (in blue) across the network. 
This gap between \mat{} and \id{} is especially evident in \bert{}, where \id{} completely fails to map the representations between most layers, suggesting that hidden representations are modified  substantially by every transformer block.
Overall, this highlights the shortcoming of existing practices to inspect representations in the same linear space, and the gains from using our method to approximate future layers.
% in the network.
\section{Linear Shortcut for Language Modeling}
\label{sec:prediction}

We saw that our method approximates future hidden representations substantially better than a naive propagation. 
In this section, we will show that this improvement also translates to better predictive abilities from earlier layers. Specifically, we will use our method to estimate how often intermediate representations encode the final prediction, in the context of two fundamental LM tasks; next token prediction and masked token prediction.

\paragraph{Evaluation Metrics.}
Let $h, h' \in \mathbb{R}^{d_h}$ be a final representation and a substitute final representation obtained by some mapping, and denote by $\delta (h), \delta (h') \in \mathbb{R}^{d_v}$ their corresponding output probability distributions (obtained through projection to the output vocabulary -- see details below). 
We measure the prediction quality of $h'$ with respect to $h$ using two metrics:
\begin{itemize}
[leftmargin=*,topsep=1pt,parsep=1pt]
    \item \textbf{Precision@$k$} ($\uparrow$ is better): This checks whether the token with the highest probability according to $\delta(h')$ appears in the top-$k$ tokens according to $\delta(h)$. Namely, we sort $\delta(h)$ and assign a score of $1$ if $\arg\max(\delta(h'))$ appears in the top-$k$ tokens by $\delta(h)$, and $0$ otherwise.
    
    \item \textbf{Surprisal} ($\downarrow$ is better): We measure the minus log-probability according to $\delta(h)$, of the highest-probability token according to $\delta(h')$. Intuitively, low values mean that the model sees the substitute result as probable and hence not surprising.
\end{itemize}

\noindent We report the average Precision@$k$ and Surprisal over the validation set $\mathcal{V}$.



\subsection{Next Token Prediction}
\label{subsec:next_token_prediction_task}

Auto-regressive LMs output for every position a probability distribution over the vocabulary for the next token. Specifically, the output distribution for every position $i$ is given by $\delta (h_i^L)$, where:
\begin{equation}\label{eq:output_distribution}
    \delta (h) = \texttt{softmax} ( E^\top \cdot h) \in \mathbb{R}^{d_v}
\end{equation}
For some LMs, including \gpt{}, a layer normalization $\texttt{ln\_f}$ is applied to the final layer representation before this conversion (i.e., computing $\delta (\texttt{ln\_f}(h))$ rather than $\delta (h)$).

Recall that our goal is to measure how well this distribution can be estimated from intermediate representations, i.e. estimating $\delta (h_i^L)$ from $\delta (h_i^\ell)$ where $\ell<L$. To this end, we first run examples from the validation set through the model, while extracting for each example $s$ the hidden representation of a random position $i_s$ at every layer. Next, we apply our mappings $\matlL{}$ and the $\idlL{}$ baseline to cast the hidden representations of every layer $\ell$ to final layer substitutes (see \S\ref{sec:layer_jump}). Last, for each layer, we convert its corresponding final-layer substitute to an output distribution (Eq.~\ref{eq:output_distribution}) and compute the average Precision@$k$ (for $k=1,5,10$) and Surprisal scores with respect to the final output distribution, over the validation set.

\paragraph{Results.}
Figs.~\ref{fig:pre} and~\ref{fig:surp} show the average Precision@$k$ and Surprisal scores per layer in $48$-layered \gpt{}, respectively (the plots for the other \gpt{} models are presented in \S\ref{sec:app_scale}). Across all layers, \mat{} outperforms \id{} in terms of both scores, often by a large margin (e.g. till layer $44$ the Precision@$1$ achieved by \mat{} is bigger than that of $\id{}$ by more than $0.2$). 
This shows that linear mappings enable not just better estimation of final layer representations, but also of the predictions they induce. Moreover, the relatively high Precision@$k$ scores of \mat{} in early layers ($0.62$-$0.82$ for $k=10$, $0.52$-$0.74$ for $k=5$, and $0.28$-$0.45$ for $k=1$) suggest that early representations already encode a good estimation of the final prediction. Also, the substantially lower Surprisal scores of \mat{} compared to \id{} imply that our method allows for a more representative reading into the layer-wise prediction-formation of the model than allowed through direct projection to the vocabulary.

\begin{figure}[t]
\centering
\includegraphics[scale=0.4]{figs/pre_48.pdf}
\caption{Precision@$k$ ($k = 1,5, 10$) of $\matlL{}$ and $\idlL{}$ for next token prediction in $48$-layered \gpt{}.}
\label{fig:pre}
\end{figure}

\begin{figure}[t]
\centering
\includegraphics[scale=0.35]{figs/surp_48.pdf}
\caption{Surprisal for $\matlL$ and the baseline $\idlL{}$ ($48$-layered \gpt{} next token prediction task). A 95\% confidence interval surrounds the lines.}
\label{fig:surp}
\end{figure}

\subsection{Masked Token Prediction}
\label{subsec:BERT}

We now conduct the same experiment for the task of masked language modeling, where the model predicts a probability distribution of a masked token in the input rather than the token that follows the input. Unlike next token prediction, where the output distribution is computed from representations of varying input tokens, in masked token prediction the output is always obtained from representations of the same input token (i.e. \texttt{[MASK]}).

For this experiment, we use \bert{}, on top of which we use a pretrained masked language model head $\delta$; given a token sequence $s$, a \mask{} token inside it and its final representation $h$, $\delta (h) \in \mathbb{R}^{d_v}$
 is a probability distribution over tokens giving the model's assessment
 of the likelihood of tokens to be fitting in place of the \mask{} token in $s$.


\begin{figure}[t]
\centering
\includegraphics[scale=0.4]{figs/bertmask_pre_24.pdf}
\caption{Precision@$k$ ($k = 1,5, 10$) for  $\matlL{}$ and the baseline $\idlL{}$ ($24$-layered \bert{} masked token prediction task).}
\label{fig:bertmask_pre}
\end{figure}

\begin{figure}[t]
\centering
\includegraphics[scale=0.35]{figs/bertmask_surp_24.pdf}
\caption{Surprisal for $\matlL{}$ and the baseline $\idlL{}$ ($24$-layered \bert{} masked token prediction task). A 95\% confidence interval surrounds the lines.}
\label{fig:bertmask_surp}
\end{figure}

\paragraph{Results.}
Figs.~\ref{fig:bertmask_pre} and~\ref{fig:bertmask_surp} present the average Precision@$k$ and Surprisal scores per layer in $24$-layered \bert{} (the plots for the $12$-layered \bert{} model are presented in \S\ref{sec:app_scale}), overall showing trends similar to those observed for next token prediction in \gpt{} (\S\ref{subsec:next_token_prediction_task}). This is despite the differences between the two tasks and the considerable architectural differences between \bert{} and \gpt{}.
Notably, the superiority of \mat{} over \id{} in this setting is even more prominent; 
while \mat{}'s precision is between $0.2-0.6$ in the first ten layers (Fig.~\ref{fig:bertmask_pre}), \id{}'s precision for all values of $k$ is close to zero, again strongly indicating that our method allows for better reading into early layer hidden representations. 
More generally, \mat{} improves the Precision@$1$ of \id{} by more than $17\%$ at most layers, and unveils that a substantial amount of predictions ($>25\%$ starting from layer $3$) appear already in the very first layers.
Interestingly, the (rough) divide between the first half of layers and last half of layers for $\id{}$ in Figs.~\ref{fig:bertmask_pre},~\ref{fig:bertmask_surp} seems to align with the two-hump shape of the blue region for $\mat{}$ in Fig.~\ref{fig:bertmask_r2_scores}.

\paragraph{Analysis.}
We manually compare the predictions of our mapping $\matlL{}$ with $\idlL{}$, for a $24$-layered \bert{} model.  Concretely, we select 50 random sentences from the Leipzig dataset. Next, for each layer $\ell$, we manually analyze how many of the top-$5$ tokens according to $\matlL{}$ and $\idlL{}$ fit into context. We consider a token to fit into context if it is grammatically plausible within the sentence (see Tab.~\ref{tab:manual} for concrete examples).
In the resulting $1250$ instances (i.e. $50$ sentences $\times$ $25$ representations), we observe a substantially higher plausibility rate of $85.36\%$ for \mat{} compared to $52.8\%$ for \id{}. In fact, only in less than $4.3\%$ of the instances there are more plausible tokens among the top-$5$ tokens according to \id{} than among the top-$5$ tokens according to \mat{}, further supporting the Surprisal results above.

\begin{table*}
\footnotesize
\setlength{\belowcaptionskip}{-15pt}
\begin{tabular}{p{0.3\linewidth}ccccc}
& $\texttt{id}_{4 \rightarrow 24}$ & $\texttt{mat}_{4 \rightarrow 24}$ & $\texttt{id}_{12 \rightarrow 24}$ & $\texttt{mat}_{12 \rightarrow 24}$ & $\texttt{id}_{24 \rightarrow 24}$ \\ \midrule
\multirow{5}{=}{aldridge had shoulder surgery in \mask{}.} & fellowship & \tcbox{time} & cyclist & \tcbox{2009} & \tcbox{september} \\
& employment & \tcbox{it} & emergencies & \tcbox{2008} & \tcbox{november} \\
& agreement & her & seniors & \tcbox{2010} & \tcbox{december} \\
& \#\#ostal & them & cycling & \tcbox{2006} & \tcbox{august} \\
& \#\#com & work & \tcbox{pennsylvania} & \tcbox{2007} & \tcbox{july} \\ \midrule
\multirow{5}{=}{on your next view you will be asked to \mask{} continue reading.} & \#\#com & be & be & be & \tcbox{please} \\
& accreditation & get & undergo & \tcbox{please} & \tcbox{simply} \\ 
& $	\copyright$ & go & spartans & help & \tcbox{also} \\ 
& fellowship & \tcbox{help} & seniors & \tcbox{simply} & \tcbox{again} \\ 
& summer & have & * & say & \tcbox{immediately} \\ \bottomrule
\end{tabular}
\caption{Examples of top-$5$ predictions at layers $4$, $12$ and $24$, under the mappings $\matlL{}$ and $\idlL{}$, for a $24$-layered \bert{} model. Grammatically plausible predictions (according to a human annotator) are marked in \tcbox{blue}. Note that at layer $24$ the predictions of $\matlL{}$ and $\idlL{}$ are the same (by definition).} 
\label{tab:manual}
\end{table*}

\section{Implication to Early Exiting}
\label{sec:applications}

%The fact that it is often possible to approximate
The possibility of approximating
the final prediction already in the early layers has important implications for efficiency; applying our linear mapping instead of executing transformer blocks of quadratic time complexity, could save a substantial portion of the computation. In this section, we demonstrate this in the context of early exiting.

When 
% performing transformer model inference under 
using an early exit strategy \cite{schwartz-etal-2020-right, xin-etal-2020-deebert, schuster2022confident}, one aims at deciding dynamically at which layer to stop the computation and ``read'' the prediction from the hidden representation of that layer.
More precisely, under a confidence measure paradigm, one decides to stop the computation for a position $i$ at layer $\ell$ based on a confidence criterion, that is derived from casting the hidden representation $h_i^\ell$ as a final-layer representation and converting it to an output probability distribution. Specifically, following \citet{schuster2022confident}, a decision to exit is made if the difference between the highest and the second highest probabilities is bigger than $$ 0.9 \cdot \lambda + 0.1 \cdot {\rm exp} (-4 i / N),$$
where $N$ is the average length of the input until position $i_s$ for $s \in \mathcal{V}$, and $\lambda$ is a hyper-parameter.

\begin{figure}[t]
\setlength{\belowcaptionskip}{-10pt}
\centering
\includegraphics[width=\columnwidth]{figs/ee_gpt2bert.pdf}
\caption{Precision@$1$ with early exit and ``fixed exit'', applied to the $24$-layer \gpt{} for next token prediction (left) and the $24$-layer \bert{} for masked token prediction (right). Varying the confidence parameter $\lambda$, the $x$-coordinate is the average number of layers processed before an early exit decision is reached.}
\label{fig:ee_gpt2bert}
\end{figure}

\quash{
\begin{figure}[t]
\setlength{\belowcaptionskip}{-10pt}
\centering
\includegraphics[scale=0.35]{figs/ee_pre1_24.pdf}
\caption{Precision@$1$ for the various early exit methods, and previous ``fixed exit'' methods for comparison ($24$-layer \gpt{} next token prediction task). Varying the confidence parameter $\lambda$, the $x$-coordinate is the average number of layers processed before an early exit decision is reached.}
\label{fig:ee_pre1}
\end{figure}
}

\paragraph{Experiment.}
We assess the utility of our mapping $\matlL{}$ for early exit as a plug-and-play replacement for $\idlL{}$, through which intermediate representations are cast into final-layer representations.
We use \gpt{} for the next token prediction and \bert{} for masked token prediction (both with 24 layers).
We run each of the models over the validation set examples, while varying the confidence parameter $\lambda$ and using either $\idlL{}$ or $\matlL{}$ for casting intermediate representations.
Furthermore, we compare these early exit variants to the ``fixed exit'' strategy from \S\ref{sec:prediction}, where the computation is stopped after a pre-defined number of layers rather than relying on a dynamic decision.
We evaluate each variant in terms of both prediction's accuracy, using the Precision@$1$ metric (see \S\ref{sec:prediction}), and efficiency, measured as the average number of transformer layers processed during inference.


\paragraph{Results.}
%Figs.~\ref{fig:ee_pre1} and~\ref{fig:bertmask_ee_pre1}
Fig.~\ref{fig:ee_gpt2bert}
plots the average Precision@$1$ score against the average number of layers processed, for $24$-layer \gpt{} and $24$-layer \bert{}. For both models, under an early exit strategy our mapping \mat{} again provides a substantial improvement over \id{}.
For example, aiming at $95\%$ average precision, \mat{} saves $\sim3.3$ ($13.8$\%) layers in \gpt{} compared to only $\sim1.4$ ($5.9$\%) layers by \id{}, and $\sim4.8$ ($20$\%) layers in \bert{} versus $\sim3.5$ ($14.6$\%) layers by \id{}.
These results highlight the potential gains prominent early exit methods can obtain by using our method.
Notably, in both models and for each of the mapping methods, early exit obtains better results than fixed layer exit, as expected. 

\quash{
\begin{figure}[t]
\setlength{\belowcaptionskip}{-10pt}
\centering
\includegraphics[scale=0.35]{figs/bertmask_ee_pre1_24.pdf}
\caption{Precision@$1$ for the various early exit methods, and previous ``fixed exit'' methods for comparison ($24$-layer \bert{} masked token prediction task). Varying the confidence parameter $\lambda$, the $x$-coordinate is the average number of layers processed before an early exit decision is reached.}
\label{fig:bertmask_ee_pre1}
\end{figure}
}
\section{Linear Shortcut Across Sub-Modules}
\label{sec:submodules}

% Our experiments show that
% , despite the commonly-applied simplification by interpretability works, transformer layers do not operate in the same linear space and 
% there is a major gap in approximating future representations using an identity mapping (\S\ref{sec:layer_jump}, \S\ref{sec:prediction}).
% Here, 
In this section, we investigate whether discrepancies across layers result from specific sub-modules or are a general behaviour of all sub-modules in the network.  
This is done by extending our approach to test how well particular components in transformer blocks can be linearly approximated. 


\paragraph{Method.}

Consider \gpt{} for definiteness, then:
% we have 
$$ \texttt{b}_{\ell} = \texttt{b}_{\ell}^{\texttt{ffn}} \circ \texttt{b}_{\ell}^{\texttt{attn}}$$ 
% with
\begin{equation}\label{eq:attn} \texttt{b}^{\texttt{attn}}_{\ell} (H) = \texttt{attn}_{\ell} (\texttt{ln1}_{\ell} (H)) + H,\end{equation} 
where $\texttt{attn}_{\ell}$ is
%a multi-head self-attention
a MHSA
layer and \texttt{ln1} is a layer normalization (LN), and 
$$ \texttt{b}^{\texttt{ffn}}_{\ell} (H) = \texttt{ffn}_{\ell} (\texttt{ln2}_{\ell} (H)) + H,$$  
where $\texttt{ffn}_{\ell}$ is
%a feed-forward network
an FFN
layer and $\texttt{ln2}$ is a LN.
\quash{
Given a block $\texttt{b}_\ell$ and one of its sub-modules $\texttt{ln1}_\ell, \ \texttt{attn}_\ell, \ \texttt{ln2}_\ell$, or $\texttt{ffn}_\ell$, we fit linear regression approximating the output of the sub-module given its input and then use it in order to define mappings, as we now describe.
}
Given a block $\texttt{b}_\ell$ and one of its sub-modules $\texttt{ln1}_\ell, \ \texttt{attn}_\ell, \ \texttt{ln2}_\ell$, or $\texttt{ffn}_\ell$, we fit linear regression approximating the output of the sub-module given its input, and then use it to define mappings $\matattnl{}$, $\matlnl{}$ and $\matffl{}$.
%We provide the definition of $\matattnl{}$ below, and that of the other two in App. \ref{sec:app_submodule_skip_description}.
We provide the formal definitions of these mappings in App. \ref{sec:app_submodule_skip_description}.
\iffalse
\paragraph{$\matattnl{}$.}
%Illustrating this on $\texttt{attn}_\ell$ for definiteness,
For an input $s$, let $v^\ell_{i_s}$ be the vector at position $i_s$ in the output of $\texttt{attn}_\ell (\texttt{ln1}_\ell (H^{\ell - 1}))$. We denote by $A_\ell^{\texttt{attn}} \in \mathbb{R}^{d_h \times d_h}$ the matrix numerically minimizing 
$$ A \mapsto \sum_{s \in \mathcal{T}} || A \cdot \texttt{ln1}_\ell (h^{\ell-1}_{i_s}) - v^\ell_{i_s}||^2,$$
and define an attention sub-module replacement (Eq.~\ref{eq:attn}) by $$
\texttt{b}^{\overline{\texttt{attn}}}_\ell (h) \coloneqq A_{\ell}^{\texttt{attn}} \cdot \texttt{ln1}_\ell (h) + h. $$
We then define a mapping between two layers ${\ell \rightarrow \ell'}$ by:
$$ \matattnl{} (h) \coloneqq $$
$$ \texttt{b}^{\texttt{ffn}}_{\ell'} ( \texttt{b}^{\overline{\texttt{attn}}}_{\ell'} ( \ldots (\texttt{b}^{\texttt{ffn}}_{\ell+1} ( \texttt{b}^{\overline{\texttt{attn}}}_{\ell+1} (h)))\ldots)).$$ 
Namely, when applying each $\ell''$-th block, $\ell < \ell'' \leq \ell'$, we replace its attention sub-module $\texttt{attn}_{\ell''}$ by its linear approximation.
%In an analogous way, we consider the mappings $\matffl{}$ and $\matlnl{}$, where in the latter we perform the linear shortcut both for \texttt{ln1} and for \texttt{ln2} (see~\S\ref{sec:app_submodule_skip_description} for precise descriptions).
Importantly, unlike the original attention module, the approximation $\texttt{b}^{\overline{\texttt{attn}}}_\ell$ operates on each position independently, and therefore applying $\matattnl{}$ disables any contextualization between the layers $\ell$ and $\ell'$. Note that this is not the case for $\matffl{}$ and $\matlnl{}$, which retain the self-attention sub-modules and operate contextually.
\fi

\paragraph{Evaluation.}


We analyze the $24$-layered \gpt{}, and proceed completely analogously to \S\ref{subsec:next_token_prediction_task}, evaluating the Precision@$1$ and Surprisal metrics for the mappings $\matattnlL{}$, $\matfflL{}$ and $\matlnlL{}$.

\begin{figure}[t]
\setlength{\belowcaptionskip}{-0pt}
\centering
%\includegraphics[scale=0.2]
\includegraphics[width=\columnwidth]{figs/parts_presurp_24.pdf}
\caption{Precision@$1$ and Surprisal for the various sub-module linear mappings, and $\matlL{}$ for comparison ($24$-layer \gpt{} next token prediction task). A 95\% confidence interval surrounds the Surprisal lines.}
\label{fig:parts_presurp}
\end{figure}

\quash{
\begin{figure}[t]
\centering
\includegraphics[scale=0.4]{figs/parts_pre1_24.pdf}
\caption{Precision@$1$ for the various sub-module linear shortcut mappings, and the mapping $\matlL{}$ for comparison (\gpt{} next token prediction task).}
\label{fig:parts_pre1}
\end{figure}

\begin{figure}[t]
\centering
\includegraphics[scale=0.35]{figs/parts_surp_24.pdf}
\caption{Surprisal for the various sub-module linear shortcut mappings, and the mapping $\matlL{}$ for comparison (\gpt{} next token prediction task). A 95\% confidence interval surrounds the lines.}
\label{fig:parts_surp}
\end{figure}
}

\paragraph{Results.}
Fig.~\ref{fig:parts_presurp} shows the average Precision@$1$ and Surprisal scores per layer.
From a certain layer (\textasciitilde$7$), all sub-module mappings achieve better results than the full-block mapping $\matlL{}$. Thus, it is not just the cumulative effect of all the sub-modules in the transformer block that is amenable to linear approximation, but also individual sub-modules can be linearly approximated. 
Furthermore, the linear approximation of attention sub-modules is less harmful than that of the FFN or LN sub-modules. 
% Hypothetically, 
A possible reason is that the linear replacement of FFN or LN ``erodes'' the self-attention computation after a few layers. 
Moreover, the good performance of $\matattnlL{}$ suggests that contextualization often exhausts itself in early layers; speculatively, it is only in more delicate cases that the self-attention of late layers adds important information. Last, remark the sharp ascent of the scores for layer normalization in layers $5$-$8$, for which we do not currently see a particular reason. To conclude, we see that the possibility of linear approximation permeates
%the various
transformer components.


\section{Related Work}

Recently, there was a lot of interest in utilizing intermediate representations in transformer-based LMs, both for interpretability and for efficiency.

In the direction of interpretability, one seeks to understand the prediction construction process of the model \cite{tenney-etal-2019-bert, voita-etal-2019-bottom}.

More recent works use mechanistic interpretability and view the inference pass as a residual stream of information \cite{dar2022analyzing,geva-etal-2022-transformer}. Additionally, there are works on probing, attempting to understand what features are stored in the hidden representations \cite{adi2017finegrained, conneau-etal-2018-cram,liu-etal-2019-linguistic}. Our work is different in that it attempts to convert intermediate representations into a final-layer form, which is interpretable by design.

In the direction of efficiency, there is the thread of work on early exit, where computation is cut at a dynamically-decided earlier stage \cite{schwartz-etal-2020-right,xin-etal-2020-deebert,schuster2022confident}. Other works utilize a fixed early stage network to parallelize inference \citep{leviathan2022fast, chen2023accelerating}. However, intermediate representations are directly propagated in these works, which we show is substantially worse than our approach. Moreover, our method requires training considerably less parameters than methods such as \citet{schuster-etal-2021-consistent}, that learn a different output softmax for each intermediate layer.  

More broadly, skipping transformer layers and analyzing the linearity properties of transformer components have been discussed in prior works \cite{Zhao2021of,mickus-etal-2022-dissect,wang-etal-2022-skipbert,lamparth2023analyzing}.


\section{Conclusion and Future Work}

We present a simple and effective method for enhancing utilization of hidden representations in transformer-based LMs, that uses 
pre-fitted context-free and token-uniform linear mappings.
Through a series of experiments on different data sources, model architectures and scales, we show that our method consistently outperforms the prevalent practice of interpreting representations in the final-layer space of the model, yielding better approximations of succeeding representations and the predictions they induce, thus allowing a more faithful interpretation of the model's prediction-formation.
We demonstrate the practicality of our method for improving computation efficiency, saving a substantial amount of compute on top of prominent early exiting approaches. 
Also, by extending our method to sub-modules, 
% more specifically the attention sub-modules, 
we observe that replacing a part of the transformer inference by a non-contextual linear computation often results in a small deterioration of the prediction.
This opens new research directions for improving model efficiency,
% and parallelizability.
% including breaking the computation into several parallelizable tasks.
including breaking the computation into parallel tasks.

\section*{Limitations}

Although we see in this work that there is more linear structure to transformer inference than could be explained solely by the residual connection, we do not elucidate a reason for that. We also do not try to formulate formal criteria according to which to judge, in principle, the quality of ways of short-cutting transformer inference in-between layers. In addition, our experiments cover only English data.


%\section*{Ethics Statement}
%Scientific work published at ACL 2023 must comply with the ACL Ethics Policy.\footnote{\url{https://www.aclweb.org/portal/content/acl-code-ethics}} We encourage all authors to include an explicit ethics statement on the broader impact of the work, or other ethical considerations after the conclusion but before the references. The ethics statement will not count toward the page limit (8 pages for long, 4 pages for short papers).

\section*{Acknowledgements}

We thank Tal Schuster for constructive comments.

% Entries for the entire Anthology, followed by custom entries
\bibliography{anthology,custom}
\bibliographystyle{acl_natbib}

\appendix

\section{Descriptions of $\matattn{}$, $\matff{}$ and $\matln{}$}
\label{sec:app_submodule_skip_description}

Here we detail the definitions of the mappings $\matattnl{}$, $\matffl{}$ and $\matlnl{}$ utilized in \S\ref{sec:submodules}.

\paragraph{Description of $\matattnl{}$.}
%Illustrating this on $\texttt{attn}_\ell$ for definiteness,
For an input $s$, let $v^\ell_{i_s}$ be the vector at position $i_s$ in the output of $\texttt{attn}_\ell (\texttt{ln1}_\ell (H^{\ell - 1}))$. We denote by $A_\ell^{\texttt{attn}} \in \mathbb{R}^{d_h \times d_h}$ the matrix numerically minimizing 
$$ A \mapsto \sum_{s \in \mathcal{T}} || A \cdot \texttt{ln1}_\ell (h^{\ell-1}_{i_s}) - v^\ell_{i_s}||^2,$$
and define an attention sub-module replacement (Eq.~\ref{eq:attn}) by $$
\texttt{b}^{\overline{\texttt{attn}}}_\ell (h) \coloneqq A_{\ell}^{\texttt{attn}} \cdot \texttt{ln1}_\ell (h) + h. $$
We then define a mapping between two layers ${\ell \rightarrow \ell'}$ by:
$$ \matattnl{} (h) \coloneqq $$
$$ \texttt{b}^{\texttt{ffn}}_{\ell'} ( \texttt{b}^{\overline{\texttt{attn}}}_{\ell'} ( \ldots (\texttt{b}^{\texttt{ffn}}_{\ell+1} ( \texttt{b}^{\overline{\texttt{attn}}}_{\ell+1} (h)))\ldots)).$$ 
Namely, when applying each $\ell''$-th block, $\ell < \ell'' \leq \ell'$, we replace its attention sub-module $\texttt{attn}_{\ell''}$ by its linear approximation.
%In an analogous way, we consider the mappings $\matffl{}$ and $\matlnl{}$, where in the latter we perform the linear shortcut both for \texttt{ln1} and for \texttt{ln2} (see~\S\ref{sec:app_submodule_skip_description} for precise descriptions).
Importantly, unlike the original attention module, the approximation $\texttt{b}^{\overline{\texttt{attn}}}_\ell$ operates on each position independently, and therefore applying $\matattnl{}$ disables any contextualization between the layers $\ell$ and $\ell'$. Note that this is not the case for $\matffl{}$ and $\matlnl{}$, which retain the self-attention sub-modules and operate contextually.

\paragraph{Description of $\matffl{}$.}
Let $v^\ell_{i_s}$ be the vector at position $i_s$ in the output of $\texttt{ln2}_{\ell} (\texttt{b}_\ell^{\texttt{attn}} (H^{\ell - 1}))$, for a given input $s$. We denote by $A_\ell^{\texttt{ffn}} \in \mathbb{R}^{d_h \times d_h}$ the matrix numerically minimizing 
$$ A \mapsto \sum_{s \in \mathcal{T}} || A \cdot v^{\ell}_{i_s} - \texttt{ffn}_{\ell} (v^\ell_{i_s})||^2,$$
and define a replacement of the feed-forward sub-module $\texttt{b}_{\ell}^{\texttt{ffn}}$ by $$ \texttt{b}^{\overline{\texttt{ffn}}}_\ell (H) \coloneqq A_{\ell}^{\texttt{ffn}} \cdot \texttt{ln2}_\ell (H) + H.$$
We then define a mapping between two layers ${\ell \rightarrow \ell'}$ by:
$$ \matffl{} (H) \coloneqq $$
$$ \texttt{b}^{\overline{\texttt{ffn}}}_{\ell'} ( \texttt{b}^{\texttt{attn}}_{\ell'} ( \ldots (\texttt{b}^{\overline{\texttt{ffn}}}_{\ell+1} ( \texttt{b}^{\texttt{attn}}_{\ell+1} (H))\ldots)).$$

\paragraph{Description of $\matlnl{}$.}
Let $v^\ell_{i_s}$ be the vector at position $i_s$ in the output of $\texttt{b}^{\texttt{attn}}_{\ell} (H^{\ell - 1})$, for a given input $s$. We denote by $A_\ell^{\texttt{ln1}} \in \mathbb{R}^{d_h \times d_h}$ the matrix numerically minimizing 
$$ A \mapsto \sum_{s \in \mathcal{T}} || A \cdot h^{\ell}_{i_s} - \texttt{ln1}_{\ell} (h^\ell_{i_s})||^2$$ and we denote by $A_\ell^{\texttt{ln2}} \in \mathbb{R}^{d_h \times d_h}$ the matrix numerically minimizing $$ A \mapsto \sum_{s \in \mathcal{T}} || A \cdot v^{\ell}_{i_s} - \texttt{ln2}_{\ell} (v^\ell_{i_s})||^2.$$ We define a replacement of the block $\texttt{b}^{\texttt{attn}}_{\ell}$ by \begin{equation} \texttt{b}^{\overline{\texttt{ln1}}}_\ell (H) \coloneqq \texttt{attn}_{\ell} (A_{\ell}^{\texttt{ln1}} \cdot H) + H\end{equation} and we define a replacement of the block $\texttt{b}^{\texttt{ffn}}_{\ell}$ by \begin{equation} \texttt{b}^{\overline{\texttt{ln2}}}_\ell (H) \coloneqq \texttt{ffn}_{\ell} (A_{\ell}^{\texttt{ln2}} \cdot H) + H.\end{equation}
We then define a mapping between two layers ${\ell \rightarrow \ell'}$ by:
$$ \matlnl{} (H) \coloneqq $$
$$ \texttt{b}^{\overline{\texttt{ln2}}}_{\ell'} ( \texttt{b}^{\overline{\texttt{ln1}}}_{\ell'} ( \ldots (\texttt{b}^{\overline{\texttt{ln2}}}_{\ell+1} ( \texttt{b}^{\overline{\texttt{ln1}}}_{\ell+1} (H))\ldots)).$$


\end{document}


{\small
\bibliographystyle{ieee_fullname}
}
Below we first briefly describe the selected models and then their implementation details during pre-training.

% Traditional convolutional action recognition networks before 2017 are mostly built to process single frame or multiple consecutive frames; however, such simple structures overlook the importance of long-range temporal context in action recognition, which somehow underestimates the intrinsic temporal information within videos. 
Temporal segment networks (TSN) proposes segment-based sampling to learn temporal information across frames. 
Specifically, in TSN, a video is evenly divided into several temporal segments, which one random frame is sampled from. 
Then the output from each segment will be aggregated via pooling to obtain the final prediction. 
Temporal Shift Module (TSM) shifts feature channels along the temporal axis, which facilitates information exchanged among neighboring frames. 
It can be plug-and-played in 2D networks to enable stronger temporal modeling at zero computation and zero parameters.
Thus, TSM can achieve the performance of heavy 3D CNNs while maintaining the efficiency of 2D CNNs.
% TSM introduces stronger temporal learning capacity to 2D networks while maintaining light-weight. 

Inflated 3D ConvNet (I3D) is designed to bootstrap from the corresponding 2D network since (1) the architecture of 2D network is well designed and (2) the  weights of 2D network is well pre-trained, e.g., Inception~\cite{inception} $\rightarrow$ Inception-I3D~\cite{carreira2017quo}. 
% utilize pre-trained weights from the corresponding 2D network since these 2D weights have been well-designed and trained to perceive visual concepts.
I3D initializes its 3D kernels by duplicating the 2D ones along the temporal dimension, which helps the convergence of 3D CNNs. 
Inspired by~\cite{vaswani2017attention}, non-local networks (NL) adapts the non-local operation (i.e., self-attention~\cite{vaswani2017attention}) in its building block to model long-range dependency.
For video action recognition, its goal is to relate the same object, or person-object interaction within a distant time interval in videos.
Similar to TSM, non-local block is compatible to most convolutional networks.


TimeSformer is a pure transformer-based model, which is an extension of ViT~\cite{dosovitskiy2020image} to the spatiotemporal space. 
Given the quadratic complexity of self-attention, TimeSformer compares several attention strategies when considering temporal dimention in videos.
Finally, TimeSformer introduces the divided space-time attention to greatly reduce the computation burden but achieves promising results.
% on most video action recognition datasets. 
% This structure shows both effectiveness and efficiency in their reported results. 
Continuing this modeling shift from CNNs to Transformers, VideoSwin extends Swin Transformer~\cite{liu2021swin} by adding the inductive bias of locality in video transformers. 
Simply speaking, it adapts the idea of 2D shifted window self-attention to 3D space, which results in better speed-accuracy trade-off compared to previous approaches~\cite{bertasius2021space,arnab2021vivit}.
% Similarly, VideoSwin is an extension of Swin Transformer~\cite{liu2021swin}, by adapting the 2D shifted window self-attention to 3D.
% And shifted window ensure the connection across distant regions in the spatiotemporal tensors.


\begin{figure}[t]
\centering
    \includegraphics[width=8cm]{figures/radar_new.pdf}
    \caption{The rank of the averaged performance within different data domains for the 6 models in different settings. The most outside in these radar images means the highest performance. For each domain, we average the top-1 accuracy as the scores in finetuning and average the top-1 accuracy of 16-shot results in few-shot learning. Complete results are shown in Table~\ref{tab:finetune} and Figure~\ref{fewshot}.}
    \label{radar}
\end{figure}

%%%%%%%%% REFERENCES


\end{document}
