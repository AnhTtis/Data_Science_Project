\vspace{-0.1cm}
\section{Conclusion}
\vspace{-0.1cm}
In this work, we propose the first one-stage pipeline for 3D whole-body mesh recovery that achieves SOTA performance on three benchmarks in a simple yet effective manner. Moreover, to bridge the gap between the basic task of full-body pose and shape estimation and their downstream tasks, we develop a large-scale dataset with comprehensive scenes covering our daily life. With our proposed annotation method, we show that training on \emph{UBody} can effectively improve the performance of mesh recovery in upper-body scenes. We hope this work can contribute new insights to this area, both in terms of methodology and dataset.

\noindent \textbf{Limitation and future work.}
Currently, our training does not use additional hand and face-specific datasets. It is worth studying how to make the best use of them in our pipeline to further improve performance. Also, we can validate the effectiveness of \emph{UBody} on some downstream applications, \eg, gesture recognition, driving avatar.

\textbf{Acknowledgements:} This work was partially funded through the National Key Research and Development Program of China (Project No.2022YFB36066), in part by the Shenzhen Science and Technology Project under Grant (CJGJZD20200617102601004, JCYJ20220818101001004).