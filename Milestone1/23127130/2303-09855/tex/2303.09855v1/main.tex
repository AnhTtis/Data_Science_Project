\pdfoutput=1
%%
%% This is file `sample-authordraft.tex',
%% generated with the docstrip utility.
%%
%% The original source files were:
%%
%% samples.dtx  (with options: `authordraft')
%% 
%% IMPORTANT NOTICE:
%% 
%% For the copyright see the source file.
%% 
%% Any modified versions of this file must be renamed
%% with new filenames distinct from sample-authordraft.tex.
%% 
%% For distribution of the original source see the terms
%% for copying and modification in the file samples.dtx.
%% 
%% This generated file may be distributed as long as the
%% original source files, as listed above, are part of the
%% same distribution. (The sources need not necessarily be
%% in the same archive or directory.)
%%
%% Commands for TeXCount
%TC:macro \cite [option:text,text]
%TC:macro \citep [option:text,text]
%TC:macro \citet [option:text,text]
%TC:envir table 0 1
%TC:envir table* 0 1
%TC:envir tabular [ignore] word
%TC:envir displaymath 0 word
%TC:envir math 0 word
%TC:envir comment 0 0
%%
%%
%% The first command in your LaTeX source must be the \documentclass command.
% \documentclass[sigconf, anonymous]{acmart}
\documentclass[sigconf]{acmart}
%\usepackage{algorithm} 
%\usepackage{algpseudocode} 
%\usepackage[ruled,linesnumbered]{algorithm2e}
\usepackage[linesnumbered,ruled,vlined]{algorithm2e}
% \usepackage{amssymb}% http://ctan.org/pkg/amssymb
\usepackage{pifont}% http://ctan.org/pkg/pifont
\usepackage{svg} %
\usepackage{float}%exact location 
\usepackage{graphicx} 
% \usepackage{subfigure}
\usepackage{caption}
\usepackage{subcaption} % for subfigures
% \usepackage[pdftex]{graphicx}

% \newcommand{\CHENG}{\color{blue}}
% \newcommand{\JIANYANG}{\color{red}}
% \newcommand{\CHENGB}{\color{green}}
% \newcommand{\JIANYANGB}{\color{orange}}

\newcommand{\CHENG}{}
\newcommand{\JIANYANG}{}
\newcommand{\CHENGB}{}
\newcommand{\JIANYANGB}{}
\newcommand{\JIANYANGLAST}{}
\newcommand{\CHENGC}{}
\newcommand{\JIANYANGREVISION}{}
\newcommand{\chengr}{}
\newcommand{\JIANYANGCAMERA}{}
\newcommand{\chengf}{}

% \newlength{\commentWidth}
% \setlength{\commentWidth}{7cm}
% \newcommand{\atcp}[1]{\tcp*[r]{\makebox[\commentWidth]{#1\hfill}}}

% \newcommand{\cmark}{\ding{51}}%
% \newcommand{\xmark}{\ding{55}}%
%% NOTE that a single column version may required for 
%% submission and peer review. This can be done by changing
%% the \doucmentclass[...]{acmart} in this template to 
%% \documentclass[manuscript,screen]{acmart}
%% 
%% To ensure 100% compatibility, please check the white list of
%% approved LaTeX packages to be used with the Master Article Template at
%% https://www.acm.org/publications/taps/whitelist-of-latex-packages 
%% before creating your document. The white list page provides 
%% information on how to submit additional LaTeX packages for 
%% review and adoption.
%% Fonts used in the template cannot be substituted; margin 
%% adjustments are not allowed.

%%
%% \BibTeX command to typeset BibTeX logo in the docs
\AtBeginDocument{%
  \providecommand\BibTeX{{%
    \normalfont B\kern-0.5em{\scshape i\kern-0.25em b}\kern-0.8em\TeX}}}

%% Rights management information.  This information is sent to you
%% when you complete the rights form.  These commands have SAMPLE
%% values in them; it is your responsibility as an author to replace
%% the commands and values with those provided to you when you
%% complete the rights form.
\setcopyright{acmcopyright}
\copyrightyear{2023}
\acmYear{2023}
\acmDOI{XXXXXXX.XXXXXXX}

%% These commands are for a PROCEEDINGS abstract or paper.
\acmConference[SIGMOD '23]{the 2023
International Conference on Management of Data}{June 18--23,
  2023}{Seattle, WA, USA}
%
%  Uncomment \acmBooktitle if th title of the proceedings is different
%  from ``Proceedings of ...''!
%
%\acmBooktitle{Woodstock '18: ACM Symposium on Neural Gaze Detection,
%  June 03--05, 2018, Woodstock, NY} 
\acmPrice{15.00}
\acmISBN{978-1-4503-XXXX-X/18/06}


%%
%% Submission ID.
%% Use this when submitting an article to a sponsored event. You'll
%% receive a unique submission ID from the organizers
%% of the event, and this ID should be used as the parameter to this command.
%%\acmSubmissionID{123-A56-BU3}

%%
%% The majority of ACM publications use numbered citations and
%% references.  The command \citestyle{authoryear} switches to the
%% "author year" style.
%%
%% If you are preparing content for an event
%% sponsored by ACM SIGGRAPH, you must use the "author year" style of
%% citations and references.
%% Uncommenting
%% the next command will enable that style.
%%\citestyle{acmauthoryear}

%%
%% end of the preamble, start of the body of the document source.
\begin{document}
\sloppy

% \input{content/0-revision.tex}

%%
%% The "title" command has an optional parameter,
%% allowing the author to define a "short title" to be used in page headers.
% \title{Adaptive Dimension Sampling: Based on Random Orthogonal Transformation for Nearest Neighbor Search}
\title{High-Dimensional Approximate Nearest Neighbor Search: with Reliable and Efficient Distance Comparison Operations}
% \title{Approximate Nearest Neighbor Search: Towards More Cost-Effective Distance Comparison Operations}
% Adaptive Dimension Sampling in a Random Orthogonal Transformed Space}

%%
%% The "author" command and its associated commands are used to define
%% the authors and their affiliations.
%% Of note is the shared affiliation of the first two authors, and the
%% "authornote" and "authornotemark" commands
%% used to denote shared contribution to the research.
% \if 0
\author{Jianyang Gao}
\affiliation{%
  \institution{Nanyang Technological University}
  \country{Singapore}}
\email{jianyang.gao@ntu.edu.sg}

\author{Cheng Long}
\affiliation{%
  \institution{Nanyang Technological University}
  \country{Singapore}}
\email{c.long@ntu.edu.sg}
% \fi
%%
%% By default, the full list of authors will be used in the page
%% headers. Often, this list is too long, and will overlap
%% other information printed in the page headers. This command allows
%% the author to define a more concise list
%% of authors' names for this purpose.
\renewcommand{\shortauthors}{Jianyang Gao and Cheng Long}

%%
%% The abstract is a short summary of the work to be presented in the
%% article.
\begin{abstract}
%   A clear and well-documented \LaTeX\ document is presented as an
%   article formatted for publication by ACM in a conference proceedings
%   or journal publication. Based on the ``acmart'' document class, this
%   article presents and explains many of the common variations, as well
%   as many of the formatting elements an author may use in the
%   preparation of the documentation of their work.
Approximate K nearest neighbor (AKNN) search {\JIANYANGREVISION in the high-dimensional Euclidean vector space} is a fundamental and challenging problem. {\CHENGB We observe that} in high-dimensional space, the time consumption of {\CHENGB nearly all} AKNN algorithms is dominated by {\CHENGB that of} the distance comparison operations (DCOs). For each operation, it scans full dimensions of an object and thus, runs in linear time wrt the dimensionality.
To speed it up, we propose a randomized algorithm named \texttt{ADSampling} which runs in logarithmic time {\CHENGC wrt} the dimensionality {\CHENGB for the majority of DCOs} and succeeds with 
% theoretically guaranteed 
high probability. 
In addition, based on \texttt{ADSampling} we develop one {\chengf generic} and two algorithm-specific techniques as plugins to enhance existing AKNN algorithms.
Both theoretical and empirical studies confirm that: (1) our techniques introduce nearly no accuracy loss and (2) they consistently improve the efficiency.
\end{abstract}

%%
%% The code below is generated by the tool at http://dl.acm.org/ccs.cfm.
%% Please copy and paste the code instead of the example below.
%%
% \begin{CCSXML}
% <ccs2012>
%  <concept>
%   <concept_id>10010520.10010553.10010562</concept_id>
%   <concept_desc>Computer systems organization~Embedded systems</concept_desc>
%   <concept_significance>500</concept_significance>
%  </concept>
%  <concept>
%   <concept_id>10010520.10010575.10010755</concept_id>
%   <concept_desc>Computer systems organization~Redundancy</concept_desc>
%   <concept_significance>300</concept_significance>
%  </concept>
%  <concept>
%   <concept_id>10010520.10010553.10010554</concept_id>
%   <concept_desc>Computer systems organization~Robotics</concept_desc>
%   <concept_significance>100</concept_significance>
%  </concept>
%  <concept>
%   <concept_id>10003033.10003083.10003095</concept_id>
%   <concept_desc>Networks~Network reliability</concept_desc>
%   <concept_significance>100</concept_significance>
%  </concept>
% </ccs2012>
% \end{CCSXML}

% \ccsdesc[500]{Computer systems organization~Embedded systems}
% \ccsdesc[300]{Computer systems organization~Redundancy}
% \ccsdesc{Computer systems organization~Robotics}
% \ccsdesc[100]{Networks~Network reliability}

%%
%% Keywords. The author(s) should pick words that accurately describe
%% the work being presented. Separate the keywords with commas.
% \keywords{datasets, neural networks, gaze detection, text tagging}

%% A "teaser" image appears between the author and affiliation
%% information and the body of the document, and typically spans the
%% page.
 
%%
%% This command processes the author and affiliation and title
%% information and builds the first part of the formatted document.
\maketitle






\section{Introduction}


Recent years have witnessed the rise of human digitization~\cite{habermannDeepCapMonocularHuman2020,alexanderCREATINGPHOTOREALDIGITAL,pengNeuralBodyImplicit2021,alldieckDetailedHumanAvatars2018, rajANRArticulatedNeural2020}. This technology greatly impacts the entertainment, education, design, and engineering industry.
There is a well-developed industry solution for this task.
High-fidelity reconstruction of humans can be achieved either with full-body laser scans~\cite{saitoSCANimateWeaklySupervised2021}, dense synchronized multi-view cameras~\cite{xiangModelingClothingSeparate2021a,xiangDressingAvatarsDeep2022a}, or light stages~\cite{alexanderCREATINGPHOTOREALDIGITAL}.
However, these settings are expensive and tedious to deploy and consist of a complex processing pipeline, preventing the technology's democratization.

Another solution is to view the problem as inverse rendering and learn digital humans directly from custom-collected data.
Traditional approaches directly optimize explicit mesh representation~\cite{loperSMPLSkinnedMultiperson2015, fangRMPERegionalMultiperson2018, pavlakosExpressiveBodyCapture2019} which suffers from the problems of smooth geometry and coarse textures~\cite{prokudinSMPLpixNeuralAvatars2020,alldieckVideoBasedReconstruction2018}. Besides, they require professional artists to design human templates, rigging, and unwrapped UV coordinates.
Recently, with the help of volumetric-based implicit representations~\cite{mildenhallNeRFRepresentingScenes2020, parkDeepSDFLearningContinuous2019, meschederOccupancyNetworksLearning2019} and neural rendering~\cite{laineModularPrimitivesHighPerformance2020, liuSoftRasterizerDifferentiable2019, thiesDeferredNeuralRendering2019}, 
one can easily digitize a quality-plausible human avatar from video footage~\cite{jiangNeuManNeuralHuman2022,wengHumanNeRFFreeviewpointRendering}.
Particularly, volumetric-based implicit representations~\cite{mildenhallNeRFRepresentingScenes2020, pengNeuralBodyImplicit2021} can reconstruct scenes or objects with much higher fidelity against previous neural renderer~\cite{thiesDeferredNeuralRendering2019,prokudinSMPLpixNeuralAvatars2020}, and is more user-friendly as it does not need any human templates, pre-set rigging, or UV coordinates.
Captured visual footage and corresponding skeleton tracking are enough for training.
However, better reconstructions and more friendly usability are at the expense of the following factors.
1) \textbf{Inefficiency:}
They require longer optimization times (typically tens of hours or days) and inference slowly.
Volume rendering~\cite{mildenhallNeRFRepresentingScenes2020,lombardiNeuralVolumesLearning2019} formulates images by querying the densities and colors of millions of spatial coordinates. 
In the training stage, due to memory constraints, only a small fraction of points are sampled which leads to slow convergence speed.
2) \textbf{Entangled representations}:
The geometry, materials, and motion dynamics are entangled in the neural networks. 
Due to the implicit nature of neural nets, one can hardly edit one property without touching the others~\cite{yuanNeRFEditingGeometryEditing2022a,liuEditingConditionalRadiance2021}.
3) \textbf{Graphics incompatibility}:
Volume rendering is incompatible with the current popular graphic pipeline,
which renders triangular/quadrilateral meshes efficiently with the rasterization technique.
Many downstream applications require mesh rasterization in their workflow (\eg, editing~\cite{foundationBlenderOrgHome}, simulation~\cite{benderPositionBasedSimulationMethods2015}, real-time rendering~\cite{akenine2019real}, ray-tracing~\cite{waldRTXRayTracing}).
Although there are approaches~\cite{lorensenMarchingCubesHigh,labelleIsosurfaceStuffingFast2007} can convert volumetric fields into meshes, the gaps from discrete sampling degrade the output quality in terms of both meshes and textures.


To address these issues, we present \textbf{EMA}, a method based on \textbf{E}fficient \textbf{M}eshy neural fields to reconstruct animatable human \textbf{A}vatars.
Our method enjoys flexibility from implicit representations and efficiency from explicit meshes, yet still maintains high-fidelity reconstruction quality.
Given video sequences and the corresponding pose tracking, our method digitizes humans in terms of canonical triangular meshes, physically-based rendering (PBR) materials, and skinning weights \textit{w.r.t.} skeletons.
We jointly learn the above components via inverse rendering~\cite{laineModularPrimitivesHighPerformance2020,chenDIBRLearningPredict2021,chenLearningPredict3D2019} in an end-to-end manner.
Each of them is derived from a separate neural field, which relaxes the requirements of a preset human template, rigging, or UV coordinates.
Specifically, we predict a canonical mesh out of a signed distance field (SDF) by differentiable marching tetrahedra~\cite{shenDeepMarchingTetrahedra2021,gaoGET3DGenerativeModel,gaoLearningDeformableTetrahedral2020,munkbergExtractingTriangular3D2022}, then we extend the marching tetrahedra~\cite{shenDeepMarchingTetrahedra2021} for spatial-varying materials by utilizing a neural field to predict PBR materials \textit{on the mesh surfaces} after rasterization~\cite{munkbergExtractingTriangular3D2022,hasselgrenShapeLightMaterial2022,laineModularPrimitivesHighPerformance2020}.
To make the canonical mesh animatable, we take another neural field to model the forward linear blend skinning for the meshes. 
Given a posed skeleton, the canonical mesh is then transformed into the corresponding poses.
Finally, we shade the mesh with a rasterization-based differentiable renderer~\cite{laineModularPrimitivesHighPerformance2020} and train our models with a photo-metric loss.
After training, we export the mesh with materials and discard the neural fields.

\looseness=-1
There are several merits of our method design.
1) \textbf{Efficiency}:
Powered by efficient mesh rendering, our method can render in real-time.
Besides, the training speed is boosted as well, 
since we compute loss holistically on the whole image and the gradients only flow on the mesh surface. In contrast, volume rendering takes limited pixels for loss computation and back-propagates the gradients in the whole space.
Our method only needs about an hour of training and minutes of optimization are enough for plausible avatar reconstruction.
2) \textbf{Disentangled representations}:
Our shape, materials, and motion modules are disentangled naturally by design, which facilitates editing. 
Besides, Canonical meshes with forward skinning modeling handle the out-of-distribution poses better.
3) \textbf{Graphics compatibility}:
Our derived mesh representation is compatible with 
the prominent graphic pipeline, which leads to instant downstream applications (\eg, the shape and materials can be edited directly in design software~\cite{foundationBlenderOrgHome}).
To further improve reconstruction quality, we additionally optimize image-based environment lights and non-rigid motions.


We conduct extensive experiments on standards benchmarks H36M~\cite{ionescuHuman36MLarge2014b} and ZJU-MoCap~\cite{pengNeuralBodyImplicit2021}.
Our method achieves very competitive performance for novel view synthesis, generalizes better for novel poses, 
and significantly improves both training time and inference speed against previous arts.
Our research-oriented code reaches real-time inference speed (100+ FPS for rendering $512\times512$ images).
We in addition showcase applications including novel pose synthesis, material editing, and relighting.


\section{The Distance Comparison Operation}
\label{sec:dco}
% In this section, we give problem settings of KNN query, introduce random projection and discuss two classes of algorithms. 

\subsection{KNN Query and Distance Comparison Operation}
\label{subsec:definition}

% Let $\mathcal O = \{\mathbf{o}_1,\mathbf{o}_2,...,\mathbf{o}_N\}$ 
Let $\mathcal O$ be a database of $N$ objects in a $D$-dimensional Euclidean space $\mathbb{R}^D$ and $\mathbf{q}$ be a query. {\CHENG In this paper, we use ``object'' (resp. ``query'') and ``data vector'' (resp. ``query vector'') interchangeably}. 
% We denote $\mathbf{x}_i:=\mathbf{o}_i - \mathbf{q}$ and $\mathcal X := \{\mathbf{x}_1,\mathbf{x}_2,...,\mathbf{x}_N\}$ for simplicity. Then the Euclidean norm $\|\mathbf{x}_i\| $ is the to-query-distance of object $i$.
{\CHENG We note that operations on $\mathcal{O}$ can be conducted before the query $\mathbf{q}$ comes (i.e., the index phase) while those on the query $\mathbf{q}$ can only be conducted after it comes (i.e., the query phase). 
For an object $\mathbf{o}$, we define its difference from the query $\mathbf{q}$ as $\mathbf{x}$, i.e., $\mathbf{x} = \mathbf{o} - \mathbf{q}$. We refer to an object $\mathbf{o}$ by its corresponding vector $\mathbf{x}$ when the context is clear.}
Without ambiguity, by ``the distance of an object $\mathbf{o}$'', we refer to its distance {\CHENGC from} the query vector $\mathbf{q}$, {\CHENG which we denote by $dis$}. 
% For smooth reading, we also use $dis_i$ to represent the distance of object $i$ in algorithm descriptions.
%
%Considering a database of $N$ data objects $\mathcal O = \{\mathbf{o}_1,\mathbf{o}_2,...,\mathbf{o}_N   \}$ in $D$-dimensional Euclidean space $\mathbb{R}^D $, for a given query $\mathbf{q} \in \mathbf{R}^D $, we denote the residual vectors as $\mathcal X = \{\mathbf{x}_i=\mathbf{o}_i - \mathbf{q}| \forall \mathbf{o}_i \in \mathcal O\}$. In order to keep smooth reading, we denote the distance between $\mathbf{o}_i $ and $\mathbf{q} $(distance of $\mathbf{o}_i$ for short) as $\|\mathbf{x} _i\|$ for theoretical analysis, while $dis_i$ for algorithm description.
%
The \textbf{K nearest neighbor (KNN)} query is to find the top-K objects with the minimum distance {\CHENGC from} the query $\mathbf{q}$. 
% In particular, we denote the ground-truth KNNs {\CHENG by their indices} as $i^*_1, i^*_2, ..., i^*_K$ ordered by exact distance increasingly. 
% During search, we maintain currently searched KNNs in a KNN set $\mathcal K$, whose elements (KNNs) are similarly denoted as $i_1, i_2, ..., i_K$.
%Similarly, the currently maintained KNNs in $\mathcal{K}$ is denoted as $i_1, i_2, ..., i_K$. 

%The $c$-approximate K nearest neighbor ($c$-AKNN) search query only requires the returned object $i_j$ to be $c$-approximate: $\|\mathbf{x}_{i_j} \| \le c \| \mathbf{x}_{i_j^*}\|$. 
%The $r$-near neighbor ($r$-NN) search is to find one object $i$ whose distance is no greater than $r$: $\| \mathbf{x}_i\| \le r $. The $c$-approximate $r$-near neighbor ($cr$-NN) is to return one object $i$ whose distance is no greater than $cr$: $\| \mathbf{x}_i \| \le cr $ if there exists an object $i^*$ with $\| \mathbf{x}_{i^*}\| \le r$. 
% In the present work, we focus on an important component in KNN query, i.e. \textit{distance comparison}. 

{\CHENG In this paper, we study the \emph{distance comparison operation} (DCO), which is defined as follows.}
%
\begin{definition}[Distance Comparison Operation]
% Let positive number $r$ be a distance threshold and object $i$ be a data vector. The problem of distance comparison is to 
{\CHENG Given an object $\mathbf{o}$ and a distance threshold $r$, the \textbf{distance comparison operation} (DCO) is to decide whether object $\mathbf{o}$ has its distance $dis$ {\JIANYANGLAST no greater than $r$}
% at most $r$ 
and if so, return $dis$.
% return the order between object $i$'s exact distance $dis_i$ and the distance threshold $r$. 
% {\CHENG decide whether object $i$'s exact distance $dis_i$ is at most the distance threshold $r$.}
% In particular, we use $1$ (positive) and $0$ (negative) to represent the comparison results of $dis_i\le r$ and $dis_i > r$, respectively.
In particular, we say object $\mathbf{o}$ is a \emph{positive} object if {\CHENGC $dis\le r$} and a \emph{negative} object otherwise.}
\end{definition}
%Formally, let $r$ be a threshold (in practice, usually $dis_{i_K}$). For a new candidate data vector $\mathbf{o}_i$, we want to conclude the order between its distance $dis_i$ and $r$ without evaluating all $D$ dimensions.
%
% The distance comparison operation plays a vital role in most (if not all) existing AKNN algorithms, which we elaborate on next.

{\CHENG As mentioned in Section~\ref{sec:introduction}, DCOs are heavily involved in many AKNN algorithms. These algorithms conduct the DCO for an object $\mathbf{o}$ and a distance $r$ naturally by computing $\mathbf{o}$'s distance and comparing the distance against $r$. We call this conventional method \texttt{FDScanning} since it uses \emph{all} dimensions of $\mathbf{o}$ {\CHENGC for computing} the distance. Clearly, \texttt{FDScanning} has the time complexity of $O(D)$. 
% With \texttt{FDScanning} adopted for DCOs, nearly all existing AKNN algorithms have their time costs dominated by those of DCOs. 

Next, we review the existing AKNN algorithms and validate both theoretically and empirically the critical role of DCOs in these algorithms.}
% With a conventional method for the distance comparison operation on object $i$ is to compute its distance $dis_i$ and compare $dis_i$ against $r$. and dominate their time costs.

% {\CHENG 14-07-Cheng: Once this problem is presented, it is better to repeat the explanations on how this problem/operation is involved in many existing AKNN algorithms.}

% {\JIANYANG 15-07-Jianyang: I thought it was the same problem as the one in Section 1, which might need further discussion.}




% {\CHENG As mentioned in Section~\ref{sec:introduction}, distance comparison operations are heavily involved in many AKNN algorithms, including\texttt{HNSW}and IVF, and dominate their time costs. A conventional method for the distance comparison operation on object $i$ is to compute its distance $dis_i$ and compare $dis_i$ against $r$. This method has its time complexity of $O(D)$. Considering that $D$ is usually hundreds and even thousands for high-dimensional objects and 
% % the distance comparison operation is frequently invoked by an AKNN algorithm
% the \texttt{FDScanning} method is an overkill for negative objects which take a large proportion of generated candidates
% , in this paper, we aim to 
% % conduct distance comparison operations \emph{approximately} with correctness guarantees for better efficiency.
% {\JIANYANG
% optimize distance comparison operations of negative objects for better time-accuracy tradeoff.
% }
% }

% {\JIANYANG
% By providing both theoretical and empirical evidence, we first justify our statements: (1) distance comparison operation dominates the time cost of most (if not all) AKNN algorithms (2) negative objects further take dominant proportion among generated candidates of AKNN algorithms.
% \underline{First}, we study the time complexity of two state-of-the-art AKNN algorithms, HNSW~\cite{malkov2018efficient} and IVF~\cite{jegou2010product}. Let $N_s$ be the number of generated candidates of an AKNN algorithm. The total time complexity of\texttt{HNSW}(resp. IVF) is $O(N_s D + N_s  \log N_s$ (resp. $O(N_s D + N_s \log K)$), where $O(N_s D)$ is the cost of distance comparison and $O(N_s \log N_s)$ (resp. $O(N_s \log K)$) is that of other operations. Since $D$ can be hundreds or even thousands of dimensions while $\log N_s$ (resp. $\log K$) is only a few dozens, the cost of AKNN algorithms is dominated by the distance comparison operations. 
% \underline{Second}, assume that we evaluated all the objects in a random order (i.e., a random permutation). Then the expected number of positive objects is $O(K \log N)$ ($O(N_{ef} \log N)$ for HNSW) (due to the limit of space, we'll provide detailed proof in technical report). Note that in practice we evaluate candidates in an AKNN-determined order, which is expected to be better than the random one. Thus, the number of positive objects in an AKNN algorithm is supposed to be bounded by $O(K \log N)$ ($O(N_{ef} \log N)$ for HNSW).  
% We verify the above statements empirically. Figure~\ref{fig:cost statistics} profiles the time consumption of AKNN algorithms on three real-world datasets when targeting 95\% recall with $K = 100$. We observe that (1) on datasets with various dimensions from 256 to 960, distance comparison takes up 77.2\% to 87.6\% of total running time of\texttt{HNSW}and 85.0\% to 95.3\% of that of IVF;  (2) For HNSW, the number of negative objects is 4.3x to 5.2x times more than the positive, and for IVF, the number of negative objects is 60x to 869x more than the positive. Thus, in the following sections, we focus on optimizing the distance comparison operation, in particular for negative objects.
% }

% \subsection{Random Projection}
% \label{subsec:rp}
% Random projection is a method that projects high-dimensional vectors into a low-dimensional space with random matrices and at the same time, largely preserves distance information. There are a large class of random matrices used in random projection~\cite{DBfriendly, blockjlt, datar2004locality, johnson1984extensions, fftjlt}. The most popular kind in database community is \textit{random Gaussian projection}~\cite{datar2004locality, c2lsh, huang2015query, tao2010efficient,sun2014srs, zheng2020pm} whose entries of the random matrix are independent standard Gaussian variables. 

% Another famous kind
% % , the one we adopted in the present work, 
% is \textit{random orthogonal projection}~\cite{johnson1984extensions, choromanski2017unreasonable,yu2016orthogonal}\footnote{{\JIANYANG Random orthogonal projection~\cite{johnson1984extensions} for low distortion metric embedding was studied prior to random Gaussian projection~\cite{indyk1998approximate}, and they show similar empirical effectiveness. 
% %At the same time, these two methods show very similar empirical effectiveness. Since generating random Gaussian projection is simpler, it becomes the most popular in database community.
% }}. 
% Its corresponding random matrix $P \in \mathbb{R}^{d\times D}$ is composed of $d$ mutually orthogonal unit row vectors, which aims at projecting vectors to a $d$-dimensional subspace drawn "uniformly" from all $d$-dimensional subspaces. A common way to generate a random orthogonal projection is to first generate a square random Gaussian projection matrix $G\in \mathbb{R}^{D\times D} $, and then orthogonalize it with QR decomposition $G=QR$, where $Q$ is an orthogonal matrix and $R$ is an upper triangular matrix. By taking arbitrary $d$ rows of $Q$, a random orthogonal projection matrix in $\mathbb{R}^{d \times D}$ is obtained~\cite{choromanski2017unreasonable}. In particular, the random orthogonal projection with a square matrix in $\mathbb{R}^{D\times D} $ is usually termed as \textit{random orthogonal transformation}~\cite{ITQ, randomortho} (or random rotation). Without ambiguity, in the following sections, we refer to random orthogonal projection as \textit{random projection} and random orthogonal transformation as \textit{random transformation} for short. 
% The process in which we take $d$ rows of a matrix or a column vector is termed \textit{row sampling}. 

% \begin{figure}
%     \centering
%     \includegraphics[height=0.3\linewidth]{figure/random projection.pdf}
%     \caption{Random Projection}
%     \label{fig:random projection}
% \end{figure}

% \begin{figure}
%     \centering
%     \includegraphics[height=0.3\linewidth]{figure/dimension sampling.pdf}
%     \caption{Dimension Sampling}
%     \label{fig:dimension sampling}
% \end{figure}

% {\CHENG 14-07-Cheng: Would it be better to use figures to illustrate two kinds of random projection?}

% {\JIANYANG 14-07-Jianyang: Geometric interpretation of a particular random projection is hard to give since random projection is an abstract mathematical concept.}

% {\JIANYANG 23-07-Jianyang: Modified}

%For NN query, random projection provides a low-distortion metric embedding which largely reduces dimensionality while just introduces acceptable error. The most popular kind in database community is \textit{random Gaussian projection}~\cite{datar2004locality, c2lsh, huang2015query, tao2010efficient,sun2014srs, zheng2020pm}. We also introduce another one, i.e. \textit{random orthogonal projection}, which was also widely researched and utilized~\cite{johnson1984extensions,choromanski2017unreasonable, yu2016orthogonal}.

%Let $P \in \mathbb{R}^{d\times D} $ be a random projection matrix to a $d$-dimensional space. Random Gaussian projection generates $P$ by making entries independent standard Gaussian variables $\mathcal N(0,1)$\footnote{It should be scaled by a factor of $1/\sqrt {D}$ to make Lemma~\ref{eq:concen} hold.}. For random orthogonal projection, it first generates a square random Gaussian projection matrix $G \in \mathbb{R}^{D\times D} $, and then orthogonalize it with QR decomposition $G=QR$, where $Q$ is an orthogonal matrix and $R$ is an upper triangular matrix, then by taking arbitrary $d$ rows of $Q$, a random orthogonal projection matrix  in $\mathbb{R}^{d \times D}$ is obtained\footnote{It's shown by Bartlett decomposition theorem~\cite{muirhead2009aspects} that $Q$ is uniformly distributed on the space of all orthogonal matrices (Stiefel manifold).}. In the following sections we refer to random orthogonal(Gaussian) projections of square matrices $\mathbb{R}^{D\times D} $ as \textit{random orthogonal(Gaussian) transformation}. The process in which we takes $d$ rows of a matrix or column vector is termed \textit{row sampling}.

%Another way {\color{red} that shows great performance in machine learning community} is random orthogonal projection~\cite{yu2016orthogonal, choromanski2017unreasonable}. It first generates a $D \times D$ random Gaussian matrix $G$ and does QR decomposition, where the $Q$ matrix is an orthogonal matrix and $R$ is an upper triangular matrix. It's shown by Bartlett decomposition theorem~\cite{muirhead2009aspects} that $Q$ is uniformly distributed on the space of all orthogonal matrices(Stiefel manifold). Then by selecting the first $d$ rows, we get a random orthogonal projection matrix in $\mathbb{R}^{d\times D} $. 

% \begin{figure}[t]
%   \centering 
%   \includegraphics[width=\linewidth]{figure/concentration.pdf}
%   %\includesvg[width=\linewidth]{figure/concentration.svg}
%   \caption{Concentration of Random Projection.
%   %: probability density function of $\sqrt {D/d} \frac{\| P \mathbf{x}\|}{\| \mathbf{x} \| }  $ concentrates around 1.
%   }
%   \label{fig:concen}
% \end{figure}

% The performance of random projection on metric embedding is theoretically guaranteed by the well-known concentration inequality of random projection~\cite{vershynin_2018}:
% %Random orthogonal projection exhibit good performance on low-distortion metric embedding, which is theoretically guaranteed by the well-known concentration inequality of random projection~\cite{vershynin_2018}:
% \begin{lemma}
%     For a fixed point $\mathbf{x} \in \mathbb{R}^D  $, a random projection $P \in \mathbb{R}^{d\times D} $ preserves its Euclidean norm with $\epsilon $ multiplicative error bound with the probability of
% \begin{align}
%     \mathbb{P} \left\{ \left| \sqrt {\frac{D}{d} } \| P \mathbf{x}\| -\| \mathbf{x} \| \right| \le \epsilon \| \mathbf{x} \|  \right\}   \ge 1 - 2e^{-c_0 d \epsilon ^2} 
%     \label{equ:concentration}
% \end{align}
% %    The corresponding failure probability is bounded by
% %\begin{align}
% %    \mathbb{P} \left\{ \left| \sqrt {\frac{D}{d} } \| P \mathbf{x}\| -\| \mathbf{x} \| \right| > \epsilon \| \mathbf{x} \|  \right\}   \le 2e^{-c_0 d \epsilon ^2} \notag 
% %\end{align}
%     where $c_0$ is a constant factor. \label{eq:fail_concen}\label{eq:concen}
%     \label{lemma:concen}
% \end{lemma}

% {\CHENG 14-07-Cheng: Perhaps, we should provide the details of the constant $c_0$ here. Note that the significance of the hypothesis to be dicussed later on relies on the value of $c_0$.}

% {\JIANYANG 14-07-Jianyang: The problem is that this expression is only tight in order but not in constant, meaning that if we set the parameter $\epsilon_1$ according to the theoretically given $c_0$, it would not be the optimal choice in practice. If we provide concrete $c_0$, we'll need to explain the gap between theory and practice.}

% {\CHENG 22-07-Cheng: I don't quite understand the explanation above. Let's discuss later on.}

%{\color{red} An intuitive explanation is that random projection approximately uniformly assigns the squared norm $\| \mathbf{x} \|^2 =\sum_{i=1}^{D} x_i^2 $ to each single dimension, so the square of each single dimension is approximately $\| \mathbf{x}\|^2 / D$. Consequently, the average of after-projection squared norm $\|P \mathbf{x}\|^2 / d$ should be a good estimation for the true average of squared norm $\| \mathbf{x}\|^2 / D$. } Correspondingly, we have the upper bound of failure probability on the right hand side: 

% Note that there are three parameters in Lemma~\ref{eq:fail_concen}: (1) multiplicative error bound $\epsilon$ ; (2) dimensionality $d$; (3) failure probability {\JIANYANG (i.e., the probability that $\sqrt {D/d} \| P \mathbf{x}\| $ falls outside the $\epsilon$ error bound)} $2\exp (-c_0 d \epsilon ^2)$. Fixing any one of them, the relationship between the other two is determined, which derives a certain interpretation. Here we investigate two cases:

% \textbf{\textit{Property 1.} Concentration of Distance.} Fixing dimensionality, the failure probability $2\exp (-c_0 d \epsilon ^2)$ decays 
% %super-exponentially 
% {\JIANYANG quadratic-exponentially}
% with respect to $\epsilon$. It intuitively indicates that the distribution of the approximate distance $\sqrt {D/d}\|P \mathbf{x}\| $ highly concentrates around the exact distance $\|\mathbf{x}\|$.
% %as shown in Figure~\ref{fig:concen}.

% \textbf{\textit{Property 2.} Decrease of Uncertainty.} Fixing failure probability, as the dimensionality $d$ increases, the multiplicative error bound $\epsilon$ decays at the order of $\epsilon \propto 1/\sqrt {d}$.
% %as shown by the width between the dash lines in Figure~\ref{fig:concen}, indicating that it becomes more concentrated.

% {\JIANYANG Such concentration inequality directly derives a powerful theorem as well as a Monte Carlo randomized algorithm for low-distortion metric embedding for $\ell_2$ Euclidean space, i.e., Johnson-Lindenstrauss Lemma~\cite{johnson1984extensions}. 

% \begin{theorem}[Johnson-Lindenstrauss Lemma (Restate)]
% Given multiplicative error bound $\epsilon$, and a finite set $X \subset \mathbb{R}^D $ of $N$ points. A random projection $P \in \mathbb{R}^{d \times D} $, $d = \Theta (\epsilon^{-2}\log \frac{N}{\delta} )$ guarantees to preserve the mutual distance of points in $X$ within multiplicative error $1 \pm \epsilon $ with probability at least $1- \delta$.
% \end{theorem}

% Note that it provides a very non-trivial result for dimension reduction by claiming that $\Theta(\epsilon^{-2} \log \frac{N}{\delta} )$ dimensions are enough to describe the distance information of a finite set with arbitrary structure and regardless of its original dimension $D$. We aim to incorporate its power into the problem of distance comparison.
% }
% {\CHENG 22-07-Cheng: Comments on Figure~\ref{fig:concen}: (1) Are the distributions real? They look to be Gaussian like distributions.  (2) To me, it does not help that much for illustrating Properties 1 and 2. Perhaps, we drop this figure.}

% {\JIANYANG 22-07-Jianyang: (1) It's actually sub-Gaussian (a large class of distribution with the same order of decay speed as Gaussian.) This figure is simply for illustration. It's drawn with square root of a $\chi^2(d)$ distribution, which should be the result of random Gaussian projection. (2) Sure. }

%where $c_0$ is a constant factor. The failure probability decays at the rate of $\exp (-c_0d \epsilon ^2)$ which is termed subgaussian tail bound, indicating that it decays by the same order of speed as Gaussian distribution. Intuitively, as the failure probability decays exponentially, the distribution of the observed distance in the $d$-dimensional space $\|P \mathbf{x}\| $ highly concentrates around the true distance $\|\mathbf{x}\| $ after scaling by $\sqrt {d/D } $.  {\color{red} I may need some figures here.}

%\subsection{Hypothesis Testing}
%Hypothesis testing is a famous method for statistical inference. It answers the question that whether sampled data are sufficient enough to support a hypothesis about a distribution or its parameters. 

%The procedure can be breifly summarized as follows:
%\begin{itemize}
%    \item Determine the unknown parameter we want to test. 
%    \item Set up a null hypothesis $H_0$ and its corresponding alternative hypothesis $H_1$ for the parameter. 
%    \item Construct statistics based on raw samples.
%    \item Set up a significance level to control the failure probability.
%    \item Derive rejection region for the constructed statistics.
%\end{itemize}

% {\JIANYANG
% \subsection{Hypothesis Testing}
% Hypothesis testing is a classic technique of statistical inference. It aims at drawing some conclusions about an \textbf{unknown parameter} with some samples drawn from a distribution, where the distribution is dependent on the unknown parameter. Usual procedure of hypothesis testing is summarized as below:
% \begin{enumerate}
%     % \item Specify a question about an unknown parameter (e.g., let exact distance $dis_i$ be the unknown parameter. We are interested in the result of a distance comparison, i.e., whether $dis_i \le r$).  
%     \item Propose a null hypothesis $H_0$ and its corresponding alternative hypothesis $H_1$ about the question.
%     % (e.g., $H_0:dis_i \le r, H_1: dis_i > r$ ).
%     \item Use a random variable as an estimator of the unknown parameter and specify their relationship.
%     % (e.g., let approximate distance obtained by random projection $dis_i'$ be the estimator. Its distribution is bounded by Lemma~\ref{eq:concen}, which is dependent on $dis_i$.).
%     % \item Consider that if the hypothesis $H_0$ holds, what the distribution of the estimator would be like (e.g., note that when $H_0: dis_i \le r$ holds, since $dis_i'$ should concentrate around $dis_i$, $dis_i'$ cannot be much larger than $r$.). 
%     % \item Select a significance level $\delta$, i.e., we'll reject the null hypothesis if the probability of the observed event is below $\delta$. Note that we do not directly calculate the probability of an event.
%     \item Compute the rejection region of the estimator based on a significance level $\delta$: the event that the observed value falls inside the rejection region indicates that the probability of the observed event is below $\delta$.
%     % {\CHENG (e.g., the rejection region of $dis_i'$ corresponds to a range in the form of $(\gamma\cdot r, +\infty)$, where 
%     %$\gamma$ is a positive value
%     % {\JIANYANG $\gamma > 1$}. The intuition is that the event that the observed $dis_i'$ is much larger than $r$ (i.e., $dis_i' > \gamma\cdot r$) implies that $H_0$ is probably not true).} 
%     \item Check whether an observation of the estimator falls inside the rejection region. If so, reject $H_0$. Otherwise, do not reject it. Note that ``not reject'' does not mean ``accept''. 
% \end{enumerate}

% }

\subsection{AKNN Algorithms and Their DCOs}
\label{subsec:aknn}


\begin{figure}[thb]
    \vspace{-4mm}
    \centering
    \begin{subfigure}[b]{0.5\linewidth}
        \centering
        \includegraphics[width=0.8\textwidth]{figure/HNSW.pdf}
        \caption{\texttt{HNSW}}
        \label{fig:routing}
    \end{subfigure} 
    \begin{subfigure}[b]{0.45\linewidth}
        \centering
        \includegraphics[width=0.8\textwidth]{figure/IVF.pdf}
        \caption{\texttt{IVF}}
	\label{fig:ivf}
    \end{subfigure} 
    % \subfigure[\texttt{HNSW}]{
    %     \includegraphics[height=0.3\linewidth]{figure/HNSW.pdf}
    %     \label{fig:routing}
    % }
    % \subfigure[\texttt{IVF}]{
	   %  \includegraphics[height=0.3\linewidth]{figure/IVF.pdf}
	   %  \label{fig:ivf}
    % }
    \vspace{-4mm}
    \caption{Illustrations of AKNN algorithms.}
    \vspace{-4mm}
    \label{fig:aknn}
\end{figure}

\subsubsection{Graph-Based Methods}
%Similarity graph is one family of state-of-the-art methods that exhibits dominant performance on time-accuracy tradeoff for in-memory AKNN query. It builds a graph upon a database, where each vertex corresponds to a data vector. One famous graph-based method is the hierarchical navigable small world graphs (HNSW)~\cite{malkov2018efficient}.
{\JIANYANG Graph-based methods are one family of state-of-the-art AKNN algorithms that exhibit dominant performance on the time-accuracy tradeoff for 
% in-memory AKNN query~\cite{malkov2018efficient, li2019approximate, FuNSG17, fu2019fast, diskann, NSW}.
{\JIANYANGREVISION in-memory AKNN query~\cite{malkov2018efficient, NSW, li2019approximate, fu2019fast, fu2021high, SISAP_graph}}.
{\CHENGC These methods construct graphs based on the data vectors, where} a vertex corresponds to a data vector. }
One famous graph-based method is the hierarchical navigable small world graphs (\texttt{HNSW})~\cite{malkov2018efficient}. It's composed of several layers. Layer 0 (base layer) contains all data vectors and layer $i+1$ only keeps a subset of the vectors in layer $i$ randomly. 
The size of each layer decays exponentially as it goes up.
% and the subset to keep is chosen randomly. 
In particular, the top layer contains only one vertex. Within each layer, a vertex is connected to its several approximate nearest neighbors,
%\footnote{Two parameters are preset to control its construction: $M$ to control the number of neighbors and $efConstruction$ to control the quality of approximate nearest neighbors.}
{\JIANYANG while between adjacent layers, two vertexes are connected only if they represent the same vector. }
% only the vertexes representing the same vector are connected.
{\CHENG An illustration of the \texttt{HNSW} graph is provided in Figure~\ref{fig:routing}.}

During the query phase, greedy search is first performed on upper layers to find a good entry at layer 0 (the base layer). 
Specifically, the search starts from the only vertex of the top layer. 
%Within each layer, it does greedy search until it falls into a minimum. 
{\JIANYANG Within each layer, it does greedy search iteratively. At each iteration, it accesses all the neighbors of its currently located vertex and goes to the one with the minimum distance. It terminates the search when {\CHENG none of the neighbors has a smaller distance than the currently located vertex.}} Then it goes to the next layer and repeats the process until {\CHENG it arrives at} layer 0. 
% As for the search algorithm in layer 0, it's a general strategy applied in most graph-based methods~\cite{malkov2018efficient, li2019approximate, FuNSG17, fu2019fast, diskann, NSW} as summarized in \cite{graphbenchmark}, i.e. \textit{greedy beam search} (\textit{best first search} in \cite{graphbenchmark}).
{\CHENG At layer 0, it conducts \textit{greedy beam search}~\cite{graphbenchmark} (\textit{best first search}), which is adopted by 
% most graph-based methods~\cite{malkov2018efficient, li2019approximate, FuNSG17, fu2019fast, diskann, NSW}.
most graph-based methods~\cite{malkov2018efficient, li2019approximate, fu2019fast, diskann, NSW}.
}
%As summarized in \cite{graphbenchmark}, most existing methods~\cite{malkov2018efficient, li2019approximate, FuNSG17, fu2019fast, diskann, NSW} adopt the strategy of greedy beam search (also termed best first search in \cite{graphbenchmark}).
%\textbf{Best First Search} (also refer to as greedy beam search in \cite{adaptive2020ml, learning2route2019ml})}. 
To be specific, {\CHENG greedy beam search maintains two sets}: a search set $\mathcal S$ (a min-heap by exact distances) and a result set $\mathcal R$ (a max-heap by exact distances). The search set $\mathcal S$ {\CHENG has its size unbounded and} maintains candidates yet to be searched. The result set $\mathcal R$ {\CHENG has its size bounded by $N_{ef}$ and} maintains $N_{ef}$ nearest neighbors {\CHENG visited so far}, where the size $N_{ef}$ is the parameter to control time-accuracy trade-off. 
At the beginning, a start point {\CHENG at layer 0} is inserted into both $\mathcal S$ and $\mathcal R$. Then {\CHENG it proceeds in iterations. At each iteration, it pops 
% an object from set $\mathcal{S}$
{\JIANYANG the object with the smallest distance in set $\mathcal{S}$}
and enumerates the neighbors of the object. 
For each neighbor, it \textbf{checks whether its distance from the query object is 
% at most
{\JIANYANGLAST no greater than}
the maximum distance in set $\mathcal{R}$ and if so, it computes the distance} (i.e., it conducts a DCO).
% (we call this a \underline{distance comparison} operation)
% evaluates its distance from query object, and 
In addition, 
% if the object is a positive one,
{\CHENGB if the distance is smaller than the maximum distance in $\mathcal{R}$,}
it (1) pushes the object {\JIANYANGLAST into} both set $\mathcal{S}$ and set $\mathcal{R}$ (using the computed distance as the key)
% if the distance is at most the maximum one in $\mathcal{R}$. 
and (2) pops {\CHENGC the object with the maximum distance} from set $\mathcal{R}$ whenever $\mathcal{R}$ involves more than $N_{ef}$ objects so that the size of $\mathcal{R}$ is bounded by $N_{ef}$.
It returns $K$ objects in $\mathcal{R}$ with the smallest distances when the minimum distance in $\mathcal S$ becomes larger than the maximum distance in $\mathcal R$ and stops.
% {\JIANYANG It's worth noting that the greedy beam search follows the framework of iteratively generating candidates and maintaining KNN in an implicit way. The KNN set is implicitly maintained inside the result set $\mathcal R$.}
% {\JIANYANG Finally, it returns the $K$ nearest objects in set $\mathcal{R}$ as the results. }
%When visiting a new candidate, greedy beam search compares its distance with the maximum in $\mathcal R$ (the $N_{ef}$th NN) instead of the current $K$th NN because it needs exact distance to guide graph routing, while at the same time, its KNNs are also successfully maintained in $\mathcal R$.}
%
We note that the greedy search at upper layers corresponds to a greedy beam search process with $N_{ef}=1$.

 
% at each iteration, it greedily fetches the object with the minimum distance in $\mathcal S$ {\CHENG via a pop operation}, and evaluates the distance of all its neighbors to update $\mathcal S$ and $\mathcal R$, i.e. if the distance of a newly visited object is at most the maximum of $\mathcal R$, then it's inserted into $\mathcal S$ and replace the maximum in $\mathcal R$. 
% Finally, when the minimum in $\mathcal S$ is larger than the maximum in $\mathcal R$, indicating that the search gets into a local minimum, then the algorithm terminates and returns the best $K$ out of $N_{ef}$ objects in $\mathcal R$. 
% In this algorithm, distance comparison lies in the process of evaluating distance and dynamically updating $\mathcal S$ and $\mathcal R$, which can be enhanced by our framework. 
%it maintains a search set $\mathcal S$ and a result set $\mathcal R$ with the restriction of $|\mathcal R| \le N_r$. The search starts by initializing $\mathcal {S,R}$ with some entry points. Then iteratively, each time, it greedily accesses the vertex  with the minimum distance in $\mathcal S$ and visits all its neighbors. For a neighbor, if its distance is at most that of the maximum of $\mathcal R$, then it's inserted into $\mathcal R$ to update the result set. To keep the size of $\mathcal{R}$, the vertices with the maximum distance are deleted if $|\mathcal R| > N_r$. The algorithm terminates when the minimum in $\mathcal S$ is larger than the maximum of $\mathcal R$, i.e. $\mathcal R$ is not updated anymore. Obviously, a larger $N_r$ leads to more visited candidates, and as a result, higher accuracy and longer response time.

% {\CHENG 21-07-Cheng: (1) We need discuss the importance of the distance comparison operation; (2) Is the time complexity of a graph-based method is dominated by the cost of greedy beam search process?

% \begin{figure}[thb]
%     \centering
%     \vspace{-4mm}
%     \subfigure[\texttt{HNSW}]{
%         \includegraphics[width=0.45\linewidth]{experimental result/HNSW.pdf}
%        \label{fig:cost HNSW}
%     }
%     \subfigure[\texttt{IVF}]{
% 	    \includegraphics[width=0.45\linewidth]{experimental result/IVF.pdf}
% 	    \label{fig:cost IVF}
%     }
%     \vspace{-4mm}
%     \caption{{\CHENG Breakdown of Running Times of} AKNN Algorithms.}
%     % \vspace{-4mm}
%     \label{fig:cost statistics}
% \end{figure}

\begin{figure}[thb]
    \captionsetup[subfigure]{aboveskip=-3pt}
    \centering
    \vspace{-4mm}
    \begin{subfigure}[b]{0.32\linewidth}
        \includegraphics[width=\textwidth]{revision experimental result/HNSW.pdf}
        \caption{\texttt{HNSW}}
        \label{fig:cost HNSW}
    \end{subfigure} 
    \begin{subfigure}[b]{0.32\linewidth}
        \includegraphics[width=\textwidth]{revision experimental result/IVF.pdf}
        \caption{\texttt{IVF}}
	  \label{fig:cost IVF}
    \end{subfigure} 
    \begin{subfigure}[b]{0.32\linewidth}
        \includegraphics[width=\textwidth]{revision experimental result/IVFPQ.pdf}
        \caption{\texttt{IVFPQ}}
	  \label{fig:cost IVFPQ}
    \end{subfigure} 
    % \subfigure[\texttt{HNSW}]{
    %     \includegraphics[width=0.3\linewidth]{revision experimental result/HNSW.pdf}
    %    \label{fig:cost HNSW}
    % }
    % \subfigure[\texttt{IVF}]{
	   %  \includegraphics[width=0.3\linewidth]{revision experimental result/IVF.pdf}
	   %  \label{fig:cost IVF}
    % }
    % \subfigure[\texttt{IVFPQ}]{
	   %  \includegraphics[width=0.3\linewidth]{revision experimental result/IVFPQ.pdf}
	   %  \label{fig:cost IVFPQ}
    % }
    \vspace{-4mm}
    \caption{{\JIANYANGREVISION {\CHENG Breakdown of Running Times of} AKNN Algorithms.}}
    \vspace{-4mm}
    \label{fig:cost statistics}
\end{figure}


\smallskip
\noindent\textbf{DCO v.s. Overall Time Costs.} We review the time complexity of \texttt{HNSW} 
% {\JIANYANGLAST with} 
assuming {\JIANYANGLAST that} it adopts \texttt{FDScanning} for DCOs. Let $N_s$ be the number of {\JIANYANGLAST the} candidates of KNN objects, which are visited by \texttt{HNSW}.
% and $D$ be the number of dimensions of an object. 
Then, the total cost of the DCOs is $O(N_s D)$ and that of updating the sets $\mathcal{S} $ and $\mathcal{R} $ is $O(N_s \log N_s)$. Therefore, the time complexity of \texttt{HNSW} is $O(N_s D + N_s \log N_s)$. In practice, the total cost of DCOs should be the dominating part since $D$ can be 
% hundreds or thousands
{\JIANYANGREVISION hundreds}
while $\log N_s$ is a few dozens only for a big dataset involving millions of objects.
%
We verify this empirically as well. 
Figure~\ref{fig:cost HNSW} profiles the time consumption of \texttt{HNSW} on three real-world datasets when targeting 95\% recall with $K = 100$. 
According to the results, on datasets with various dimensions from 256 to 960, DCOs take from 77.2\% to 87.6\% of the total running time of \texttt{HNSW} (as indicated by the blue and orange portions of the bars). 
%
% With techniques developed in this paper, the time cost of each distance comparison would be reduced from
% $O(D)$ to $O(\alpha^{-2}\log D)$ in most cases
% being linear wrt $D$ to being logarithmic wrt $D$ in most cases with accuracy guarantees.
% and correspondingly, the time cost of \texttt{HNSW} would be reduced to 
% % $O(N_s \cdot %\log DK\log D+ N_s \cdot \log N_s)$.
% $O(D \log N + c_{\alpha} \cdot N_s \log D + N_s \log N_s)$,
% where $c_{\alpha}$...

\smallskip
\noindent\textbf{Positive v.s. Negative Objects.} We verify empirically that for \texttt{HNSW}, the number of DCOs on negative objects is significantly larger than that of DCOs on positive objects. 
The results are shown in Figure~\ref{fig:cost HNSW}. We note that in the figure, the ratio between the cost of DCOs (on  negative objects) and that (on positive objects) reflects the ratio between the numbers of negative and positive objects since a DCO on a negative object and that on a positive object have the same cost. 
According to the results, the number of negative objects is 4.3x to 5.2x times more than that of the positive ones.
% ; for IVF, the number of negative objects is 60x to 869x more than that of the positive ones.
}
% {\JIANYANG Moreover, the greedy search at upper layers is a special case of greedy beam search with $N_{ef}=1$. Thus, overall, distance comparison is the dominant part all through the searching algorithm.}
% {\JIANYANG Our techniques improve the time complexity of each distance comparison to be \textbf{logarithmically} dependent on $D$.}

% {\CHENG ***(1) Please give it some thoughts whether we can provide some running time statistics of the search process above layer 0, distance comparison operations (at layer 0) and updating the sets (at layer 0). (2) We can also make a comment on how much we have improved the time complexity of time comparison operations in this paper here.***}

% {\JIANYANG 22-07-Jianyang: (1) We may reduce the greedy search at upper layers to greedy beam search.(modified above) *** (2) I think without the background of Johnson-Lindenstrauss Lemma, readers might not feel the improvement. ***}

% {\JIANYANG 23-07-Jianyang: modified}



\subsubsection{Inverted File Index}

Inverted file~\cite{jegou2010product} index is another popular index method for AKNN query. {\CHENG According to~\cite{adaptive2020ml}, \texttt{IVF} is one of the state-of-the-art approaches for AKNN. Indeed, according to our experimental results in Section~\ref{section:time-accuracy}, it outperforms \texttt{HNSW} on some datasets. 
% \texttt{IVF} has two phases, namely index phase and query phase.
} 
During the index phase, the algorithm clusters data vectors with the K-means algorithm, builds a bucket for each cluster and assigns each data vector to its corresponding bucket. Then during the query phase, for a given query, the algorithm first selects the $N_{probe}$ nearest clusters based on their centroids, retrieves all vectors in these corresponding buckets as candidates, and then 
% selects final 
finds out KNNs {\CHENG among the retrieved vectors.}
% In the \texttt{IVF} algorithm, 
Here,
$N_{probe}$ is a user parameter which controls 
% the number of visited buckets so as to control 
the time-accuracy trade-off. 
%{\JIANYANG
%When selecting final KNNs, it first pre-computes exact distance for all retrieved vectors.
% for upcoming distance comparisons. 
%Then to find out KNNs, it applies quick select~\cite{quickselect}, where distance comparisons are heavily utilized. 
%*** By the way, I thought it's already been very obvious that selecting K nearest neighbor from a candidate set needs distance comparison. ***
%In particular, the time complexity of exact distance evaluation is $O(N_s \cdot D)$ and that of quick select is $O(N_s)$. 
%As a result, distance evaluation is still the dominant part of this algorithm.
%}
% {\JIANYANG
% When selecting final KNNs, it maintains a KNN set $\mathcal K$ as scanning candidates. Similar to HNSW, let $N_s$ be the number of candidate objects. The total cost of the DCOs is $O(N_s \cdot D)$ and that of updating $\mathcal K$ is $O(N_s \log K)$, in which the cost of DCO is also the dominant part.}
{\CHENG When {\JIANYANGLAST finding out} KNNs, a commonly used method is to maintain a KNN set $\mathcal K$ with a max-heap of size $K$. It then scans all candidates, and for each one, it \textbf{checks whether its distance is 
% at most
{\JIANYANGLAST no greater than}
the maximum of $\mathcal{K}$ and if so, it computes the distance} (i.e., it conducts a DCO). Here, the maximum distance is defined to be $+\infty$ if $\mathcal{K}$ is not full.
% If the result of the DCO is yes, 
{\CHENGB If the distance is smaller than the maximum distance in $\mathcal{K}$,}
it 
% inserts the candidate to $\mathcal{K}$
{\JIANYANG updates $\mathcal{K}$ with the candidate}
(by using the computed distance as the key). It returns the objects in $\mathcal{K}$ at the end. An illustration of the \texttt{IVF} structure is provided in Figure~\ref{fig:ivf}.

\smallskip
\noindent\textbf{DCO v.s. Overall Time Costs.} We review the time complexity of \texttt{IVF}
% {\JIANYANGLAST with} 
assuming {\JIANYANGLAST that} it adopts \texttt{FDScanning} for DCOs.
Let $N_s$ be the number of candidate objects. The total cost of \texttt{IVF} is $O(N_s D + N_s\log K)$, where the first term is the cost of the DCOs and the second term is that of updating $\mathcal K$. As can be noticed, the cost of DCOs is the dominating part. }
%
We verify this empirically as we did for \texttt{HNSW}. 
Figure~\ref{fig:cost IVF} shows the results.
% profiles the time consumption of \texttt{IVF} on three real-world datasets when targeting 95\% recall with $K = 100$. 
According to the results, on datasets with various dimensions from 256 to 960, DCOs take from 85.0\% to 95.3\% of the total running time of \texttt{IVF}. 
%
% With the techniques developed in this paper, the time time complexity of \texttt{IVF} would be reduced to $O(N_s\cdot \log D + N_s\log K)$.}
% {\CHENG ***Again, it would be better to talk about how the algorithm selects the final KNNs and show that distance comparisons are involved. In addition, we can provide some discussions on the importance of the DCOs (e.g., with time complexity analysis as we do for the \texttt{HNSW} algorithm).***}
% with evaluating exact distance. 
% In this algorithm, distance comparison happens during evaluating candidates and selecting final KNNs, which can be accelerated by our framework. 
% It's worth noting that \texttt{IVF} is also usually used in combination with product quantization~\cite{jegou2010product,annbenchmark, ge2013optimized, ITQ} {\CHENG for cases with limited memory.}

\smallskip
\noindent\textbf{Positive v.s. Negative Objects.} We verify empirically that for \texttt{IVF}, the number of DCOs on negative objects is significantly larger than that of DCOs on positive objects. 
The results are shown in Figure~\ref{fig:cost IVF}. 
According to the results, the number of negative objects is 60x to 869x more than that of the positive ones.
% In these cases, the algorithms would also involve distance comparisons, which can be improved with the techniques developed in this paper.
% for the concern of limited memory, which is not the focus of our work.

%Filter-and-verification is a popular framework in NN query, especially for quantization-based and hashing-based algorithms. In particular, quantization-based methods~\cite{jegou2010product, imi} suggest a three-stage strategy: 1) generate candidate lists with coarse quantization code; 2) shrink the list with finer code; 3) re-rank the list with exact distance. {\color{red} Hashing-based methods also follow filter-and-verification framework~\cite{datar2004locality, c2lsh}.} During indexing phase, hashing-based methods map all data vectors into hash codes and put vectors with the same code into the same bucket. Then during query phase, they first map query vectors to hash codes and then generate candidates by collecting objects in the corresponding buckets or adjacent buckets. The generated candidates are then re-ranked with exact distance to find out NNs. {\color{red} bad paragraph}

%Though the re-ranking stage is seldom emphasized and sometimes even dropped due to memory constraint~\cite{johnson2019billion, jegou2011searching}, {\color{red} it's widely and independently reported that without the re-ranking stage~\cite{ge2013optimized, adaptive2020ml, sun2014srs}}, the accuracy(measured by recall at K for KNN query) is usually unacceptablely low, especially for queries with large K(e.g. K=100), which reveals the importance of it. Our methods target to accelerate re-ranking while disregard the specific methods of generating candidates. 

% {\CHENG 14-07-Cheng: (1) After each AKNN is reviewed, make some comment on how the distance comparison plays a role in the algorithm; (2) Better to include a paragraph discussing other AKNN algorithms (somehow in a bit more brief way) and and also comment on how distance comparison plays a role in them.}

% {\JIANYANG 15-07-Jianyang: (1) (2) I thought it was the same problem as the one in Section 1, which might need further discussion.}


\subsubsection{Other AKNN Algorithms}
% {\JIANYANGREVISION (1) graph-based~\cite{malkov2018efficient, NSW, li2019approximate, fu2019fast, fu2021high, SISAP_graph}}, (2) quantization-based~\cite{jegou2010product, ge2013optimized, guo2020accelerating, ITQ, additivePQ, imi}, (3) tree-based ~\cite{muja2014scalable, dasgupta2008random, ram2019revisiting, beygelzimer2006cover, reviewer_M_tree} and (4) hashing-based~\cite{indyk1998approximate, datar2004locality, c2lsh, tao2010efficient, huang2015query, sun2014srs, lu2020vhp, zheng2020pm}.

{\JIANYANG
In other AKNN algorithms {\CHENGB including tree-based, hashing-based, and quantization-based methods}, DCOs are also ubiquitous. 
% The main difference among AKNN algorithms lies in how they generate candidate vectors. 
{\JIANYANGREVISION Tree-based methods~\cite{muja2014scalable, dasgupta2008random, ram2019revisiting, beygelzimer2006cover, reviewer_M_tree}} generate candidate vectors through tree routing and {\JIANYANGLAST find out} KNNs with DCOs {\JIANYANGB (similarly as \texttt{IVF} does).}
% \texttt{HNSW} (resp. \texttt{IVF}) does when the candidates are generated dynamically (resp. in a batch)).
% \footnote{Some might also generate candidates and update KNNs alternatively like graph-based methods}.
Hashing-based methods~\cite{indyk1998approximate, datar2004locality, c2lsh, tao2010efficient, huang2015query, sun2014srs, lu2020vhp, zheng2020pm} generate candidate vectors via hashing {\CHENGC codes} and {\JIANYANGLAST find out} KNNs with DCOs (similarly as \texttt{IVF} does). 
Quantization-based methods~\cite{jegou2010product, learningtohash,ge2013optimized, ITQ, imi, additivePQ} generate candidates with short quantization codes, and conduct re-ranking (for {\JIANYANGLAST finding out} KNNs) with DCOs (similarly as \texttt{IVF} does).
%
% For nearly all of the AKNN algorithms above, 
{\CHENG For tree-based and hashing-based methods,}
the cost of DCOs is dominant because (1) one time tree routing or hashing bucket probing generates multiple candidates ({\CHENG which entail} multiple DCOs) and (2) tree routing and hashing bucket probing are much faster than a DCO (which has the time complexity of $O(D)$). 
{\CHENG For product quantization-based methods~\cite{jegou2010product, ge2013optimized, ITQ, imi}, DCOs are involved in its re-ranking stage, which is less dominant because the main cost lies in evaluating quantization codes.} {\JIANYANGREVISION {\chengr For comparison, we show the time decomposition results of \texttt{IVFPQ}, which is a quantization-based method~\cite{jegou2010product},}
% We also show its results of time decomposition 
in Figure~\ref{fig:cost IVFPQ} under the typical setting of \cite{learningtohash, jegou2010product}.}
% where DCOs are not involved. 
% {\CHENG Project quantization involves DCOs in its re-ranking stage.}
% However, there are also some DCOs in its subsequent re-ranking stage, in which our methods can bring some improvements.
}
% {\CHENG 22-07-Cheng: I think we'd better review briefly those other types of AKNN algorithms and point out that they also involve DCOs for which our algorithm can help with though we do not conduct the experiments for them.}


%\section{Threat Model and Advantages of Our Hardware-based Adversarial Detector} \label{sec: motivation}
\ry{In this part, I want to highlight the comparison between hardware and software attacks}
%Normally, software-based adversarial detectors are easier to implement, cheaper to develop and more well-studied than those based on hardware computational signals.
% We would like to stress that our goal for investigating hardware-based adversarial detectors is not to achieve better performance in detection than the conventional white-box software based methods.  
\subsection{Threat Model} \label{sec: threat model}
\ry{This section is threat model: attack is `white-box', detector is `black-box'}
The victim is a DNN classifier, which is pre-trained with a public dataset. The testing dataset may be kept private.
We assume the strongest `white-box' attack model, where the attacker has full knowledge of the victim model and training dataset in order to generate adversarial samples with minimum perturbations. 
On the contrary, the detection system assumes the most limited scenario, under a `black-box' view of the victim, without access to the victim's inputs, parameters, and intermediate outputs or execution details. 
The only information available to the detector to distinguish adversarial samples is the EM side-channel measurement and the victim model's prediction class.
For training the adversarial detector with EM traces, a public benign dataset is used. 

\if false 
\ry{In this part, we discuss more settings of the detector especially the data used in two phases.}
In general, the detecting process can be summed up into two phases, training phase and detecting phase.
To begin with, we train an Out-of-Distribution(OOD) detector on a public benign dataset of the same classification task, which should be distinct from the victim's training dataset.
For each query, the detector will obtain the classification result and an EM trace along with the model execution to fit its EM classifiers and anomaly detectors.  
During the detection phase, the victim model is in operation and under attack when the pre-trained detector decides whether the current input is adversarial or not, only based on the victim model output and its EM trace.
\fi 

\subsection{Advantages}
Compared to software-based adversarial detection methods, our hardware-based detector, EMShepherd, has three distinct advantages: privacy-preserving, portability, and robustness.

\begin{itemize}[leftmargin=*]
    \item \ry{Add a new motivation here. The motivation is that using \name can help the user protect their privacy.} 
    \name protects the DNN model user's data privacy as it is agnostic to the model's inputs, which instead are always required by prior reconstruction-based detection methods~\cite{meng2017magnet, yang2022you}. 
    %Most model users are benign whose inputs may be sensitive and should not be shared with \textit{third-party detectors}. 
    The sensitive inputs should not be shared with \textit{third-party detectors}. 
    Our design only requires the output class labels and the EM signals, which are passively leaked to common acquisition equipment. 
    %    Our design is suitable for such cases as it only requires the EM signals and the inference outputs during the model execution. Generally speaking, EM signals and labels have less private information leakage.
    \item \ry{The second motivation is still related to privacy. This time we consider model privacy when the model structure or parameters should be kept private.}
   \name also protects the model confidentiality.  No model information, including %Using hardware-based detectors can prevent the third-party defender from accessing some confidential model information such as  
   hyper-parameters, parameters, and logits, is needed, in stark contrast to the previous software-based detection methods~\cite{ma2019nic,feinman2017detecting}.
    %Our \name only acquires the EM traces during model inference in a passive and noninvasive manner, 
    The EM data processing and the adversarial detector training process are both victim model-agnostic. 
    Therefore, our method has more general usage, applicable to closed-source DNN applications, which are pervasive in edge devices where the user only queries the models for the final prediction output. 
    \item \ry{The third motivation is portability.}  
    Owing to the model-agnostic feature, EMShepherd can be easily ported for wide-range hardware devices with different DNN implementations for diverse applications. It can be used as a `plug and play' (PnP) device, aside from the target system, to work automatically without user intervention or contact with the victim system. 
    \item \ry{The last motivation is about adaptive attacks, we should propose that EM signal is hard to imitate, so it is hard for adaptive attacks to generate sample fraud both detector and victim.} 
    Adaptive attack~\cite{adaptive} is a threat to most software defense methods where the attacker adjusts the adversarial perturbations to mislead both the victim models and defense systems.
   %  The hardware-based detection method can provide a double protection on top of most software defense methods such as adversarial training.
   %  Although the adptive adversarial example fools the robust model, its computation patterns during the DNN model execution are still well kept in the EM traces and our EMShepherd framework still works well for detecting the new type of adversarial examples.  
   %  Meanwhile, due to the high complexity of EM signals and non-explicit dependency of the EM signals on computations, it is extremely hard to have an adaptive attack on our detection method, i.e., adversarial examples whose EM signals are deliberately controlled to evade the EM-based detector.
   However, due to the high complexity and non-explicit dependency of the EM signals on computations and data, 
   it is extremely hard to have an adaptive attack on our detection method, 
   i.e., adversarial examples whose EM signals are deliberately controlled to evade the EM-based detector. 
\end{itemize}





%\section{PoseRAC Model}
\label{sec4}

\begin{figure*}[t]
\centering
\includegraphics[width=1.0\textwidth]{figure5.pdf}
\caption{Overview of our proposed PoseRAC. For a input video, the repetitive count can be obtained through Pose Estimation, Transformer Encoder, Pose Mapping and Action-trigger, where only the Encoder and the Pose Mapping need to be trained. We use Triplet Margin Loss to train the Encoder while Binary Cross Entropy Loss to train both the Encoder and the Pose Mapping. In addition to achieving the state-of-the-art performance so far, the biggest highlight of our PoseRAC is that it is lightweight enough to be easily trained on a CPU.}
\label{fig5}
\end{figure*}

Given a video $V={\{x_i\}}^{T}_{1}\in \mathbb{R}^{C\times H\times W\times T}$ with $T$ RGB frames, repetitive action counting model aims to predict a certain value $Y$, which is the number of repetitive actions. In this section, we will introduce our PoseRAC in detail.

\subsection{Model Overview}

As shown in Figure \ref{fig5}, PoseRAC consists of four parts. 

\begin{itemize}

\item The first is a state-of-the-art and lightweight Pose Estimation Network~($\S\ref{first}$), which is used to estimate the poses represented by lots of human pose key points from each frame of the original video sequence. 

\item The second is a simple Transformer Encoder~($\S\ref{second}$) to embed the key points of poses into high-level feature space, where the same class have similar distances, while the distances of different classes are far apart.

\item The third is a Pose Mapping Module~($\S\ref{third}$), where the unique mapping relationship between the salient poses and the action classes can be learned. Each pose can be mapped to the action class with the highest probability after the previous encoding.

\item The fourth part is a lightweight Action-trigger Module~($\S\ref{fourth}$). When we get the salient action classification results of all frames of the entire video sequence, we can use this module to calculate the repetition count in a short time.

\end{itemize}

\subsection{Pose Estimation Network}
\label{first}
Our model first converts the video sequence into a sequence of human pose key points, which can be defined as: 
\begin{equation}
\begin{split}
&V={\{x_i\}}^{T}_{1}\in \mathbb{R}^{C\times H\times W\times T}\\
&V\xrightarrow{\mathrm{Pose Estimation}} P={\{p_i\}}^{T}_{1}\in \mathbb{R}^{D\times K\times T}
\end{split}
\label{eq1}
\end{equation}
where each $x_i$ represents a single RGB frame, and each $p_i$ represents the key points of each frame. To express the key points of each frame, we use $D\times K$ sequence, which includes two parts, one ($K$) is the number of key points to fully represent the current pose, the other ($D$) is the dimension of each key point, generally three, which are the two coordinates of the planes and the depth estimation.

Here we use state-of-the-art pose estimation models such as Vitpose\cite{xu2022vitpose} and BlazePose\cite{bazarevsky2020blazepose}. The pose estimation algorithms themselves are not designed by us, but we introduce pose information into the action counting task, which is a novel design not explored by previous work.

Moreover, our pose-level poses estimation processes the primitive information of video, which is similar to the feature extraction network in all video-level algorithms such as I3D\cite{carreira2017quo}, VideoSwinTransformer\cite{liu2022video}, and TSN\cite{wang2016temporal}. But the difference is that the result of video-level incorporates all information, while pose-level only produces core information, which greatly improves the performance. Additionally, using pose information can contribute to the lightweight of model. For instance, for a 1024-frame video, video-level feature extraction with an output dimension of 512 would produce a data volume of $1024\times 512=524288$, while using pose information with 33 key points produces a data volume of only $1024\times 33 \times 3=101376$.

\subsection{Encoding Poses with Transformer}
\label{second}
Here we specify our data representation for the Transformer Encoder, which requires input batch size, sequence length, and embedding dimensions. In our pose-level approach, each frame is a batch, the number of key points in each frame is the sequence length, and the feature dimension of each key point is the embedding dimension.

First we get the pose of each frame ${p_i}\in \mathbb{R}^{D\times K}$ through the Pose Estimation Network, where $i\in {1, 2, \dots, T}$ is the frame index, $K$ is the number of key points, and $D$ is the dimension of each key point. We further define $p_i = {\{k_j\}}^{K}_{1}$ to represent each key point, where $k_j\in \mathbb{R}^D$, and we embed it to obtain richer information. Our embedding projection $\mathrm{\bf{E}}$ is a simple MLP network with ReLU as the activation function. These calculations can be defined as:
\begin{equation}
\begin{split}
\mathrm{\bf{Z}}^0 = [\mathrm{\bf{E}}(k_1), \mathrm{\bf{E}}(k_2), \dots, \mathrm{\bf{E}}(k_K)]^T
\end{split}
\end{equation}
where $\mathrm{\bf{E}}(k_j)\in \mathbb{R}^{D^{\prime}}$ is the embedding feature. Then the next Transformer takes $\mathrm{\bf{Z}}^0$ as input and encodes it with self-attention. Given $\mathrm{\bf{Z}}^0\in \mathbb{R}^{K\times D^{\prime}}$ with $K$ key point features, each of which is $D^{\prime}$-dimensional, $\mathrm{\bf{Z}}^0$ is projected using $\mathrm{\bf{W}}_Q\in \mathbb{R}^{D^{\prime}\times D_q}$, $\mathrm{\bf{W}}_K\in \mathbb{R}^{D^{\prime}\times D_k}$, $\mathrm{\bf{W}}_V\in \mathbb{R}^{D^{\prime}\times D_v}$, where $D_k=D_q$, to extract feature representations query($\mathrm{\bf{Q}}$), key($\mathrm{\bf{K}}$) and value($\mathrm{\bf{V}}$), which can be defined as:
\begin{equation}
\begin{split}
&\mathrm{\bf{Q}}=\mathrm{\bf{Z}}^0\times \mathrm{\bf{W}}_Q\\
&\mathrm{\bf{K}}=\mathrm{\bf{Z}}^0\times \mathrm{\bf{W}}_K\\
&\mathrm{\bf{V}}=\mathrm{\bf{Z}}^0\times \mathrm{\bf{W}}_V
\end{split}
\end{equation}
and the output of self-attention can be computed as:
\begin{equation}
\begin{split}
\mathrm{\bf{Attn}}=\mathrm{Softmax}(\frac{\mathrm{\bf{Q}}\mathrm{\bf{K}}^T}{\sqrt{D_q}})\mathrm{\bf{V}}
\end{split}
\end{equation}
where $\mathrm{\bf{Attn}}\in \mathbb{R}^{K\times D^{\prime}}$. Also, we use common multi-head self-attention (MHSA) to make several self-attention operations calculate in parallel.

Now we introduce the overall architecture of Transformer Encoder, which has $L$ layers with each layer consisting of MHSA and MLP blocks. Also, LayerNorm and Residual Connection are applied before and after every MHSA or MLP block, respectively. Because the number of key points of each frame is  a bit less, so our encoder does not include the downsampling module that other models may have. The overall process can be defined as:
\begin{equation}
\begin{split}
&\mathrm{\bf{\hat{Z}}}^l = \mathrm{MHSA}(\mathrm{LN}(\mathrm{\bf{Z}}^{l-1})) + \mathrm{\bf{Z}}^{l-1}\\
&\mathrm{\bf{Z}}^l = \mathrm{MLP}(\mathrm{LN}(\mathrm{\bf{\hat{Z}}}^l)) + \mathrm{\bf{\hat{Z}}}^l
\end{split}
\end{equation}
where $\mathrm{\bf{Z}}^{l-1}$, $\mathrm{\bf{\hat{Z}}}^l$, $\mathrm{\bf{Z}}^l\in \mathbb{R}^{K\times D^{\prime}}$.


\subsection{Pose Mapping}
\label{third}
Taking the Encoder output $\mathrm{\bf{Z}}^L\in \mathbb{R}^{K\times D^{\prime}}$ as input, Pose Mapping module outputs probability scores $\mathrm{\bf{S}}\in \mathbb{R}^{C}$ of the current frame over all action classes. We perform binary classification after Sigmoid activation for each class, with the two salient poses of each class represented by the same bit data. To realize such a module, we use a very lightweight MLP network, which avoids the complexity. First, the two dimensions $K$ and $D^{\prime}$ of $\mathrm{\bf{Z}}^L$ are flattened into $\mathbb{R}^{KD^{\prime}}$, and then it passes through an MLP module, where the output channels is set to $C$, which can be defined as:
\begin{equation}
\begin{split}
\mathrm{\bf{S}} = \sigma(\mathrm{MLP}(\mathrm{Flatten}(\mathrm{\bf{Z}}^L)))
\end{split}
\end{equation}
where $\sigma$ represents the Sigmoid activation function.

With such Pose Mapping, we can obtain the scores of single frame. It should be noted that we extract the poses of all frames, and use the convenience of matrix operations to obtain scores in parallel, which is actually consistent with the idea of mini batch. So at last, we combine the scores of all frames to get the video score matrix $\mathrm{\bf{\hat{S}}}\in \mathbb{R}^{C\times T}$, where $T$ represents the number of frames in the current video. 


\subsection{Action-trigger Module}
\label{fourth}
We use the lightweight Action-trigger Module to obtain the final output $Y$, the repetitive action count, which has a time complexity of $\mathcal{O}(n)$. First, we get the scores $S_c\in \mathbb{R}^T$ of a given action class from $\mathrm{\bf{\hat{S}}}$. Then, we scan all frames and use the action-trigger mechanism to count when the two salient poses of the action class occur sequentially. We set upper and lower bounds to distinguish the scores of the two salient poses, which cluster non-salient poses in the middle and easily classify the salient poses to the two ends.

\subsection{Losses and Metric Learning}

The modules need to be trained are Embedding, Transformer Encoder and Pose Mapping, and because we perform binary classification for each class, so we use the Binary Cross Entropy Loss, which can be defined as follows:
\begin{gather}
\mathcal{L}_{bce} = -\frac{1}{N}\sum\limits_{i=1}^{N}(\frac{1}{C}\sum\limits_{j=1}^{C}loss(i,j))  \\
 loss(i,j)=y_{ij}\log p_{ij} + (1-y_{ij})\log(1-p_{ij})
\end{gather}
where $N$ represents the batch size (in our method, each frame is a batch), $C$ represents the number of classes, $y$ and $p$ are the labels and our predictions, respectively.

Moreover, we use Metric Learning to improve our Encoder and introduce the Pose Triplet Loss. Given a pose, Encoder produces higher-level features $\mathrm{\bf{Z}}^L$, which should be more representative. As shown in Figure \ref{fig5}, we achieve this with Triplet Margin Loss function, which selects anchors, same class positive samples, and different classes negative samples in a batch. It can be expressed as:
\begin{equation}
\begin{split}
\mathcal{L}_{tri} = \mathrm{max}(\mathrm{CS}(a,p)-\mathrm{CS}(a,n)+\mathrm{margin},0)
\end{split}
\end{equation}
where $a$, $p$, $d$ are anchors, positive and negative samples, and $\mathrm{CS}$ represents the Cosine Similarity to measure the distance between features. We pay more attention to hard samples, where the distances between anchors and negative samples are even smaller than those of positive samples. After Metric Learning, the poses of each action can be distinguishable, which cluster in the high-level space.

At last, our overall training combines these two losses:
\begin{equation}
\begin{split}
\mathcal{L} = \mathcal{L}_{bce} + \alpha\mathcal{L}_{tri}
\end{split}
\end{equation}
where $\alpha$ is the weight factor to control the two losses in the same numeric scale.
\subsection{Implementation Details}

\noindent{\bf Training.} We use the \emph{RepCount-pose} and \emph{UCFRep-pose} dataset we created to train our model. Only the frames with salient poses are inputted into the network instead of the entire video to speed up the fitting.

\noindent{\bf Inference.} During inference, the entire video sequence is inputted into the model. The poses of all frames pass through the Encoder and Pose Mapping, and then enter the Action-trigger Module to output the repetitive count.

% In this section, we begin with a novel framework for DCO, adaptive dimension sampling. This framework serves as the foundation of our algorithms ARSearch, ARRoute and ARSelect proposed and analyzed as follows.



\section{The \texttt{ADSampling} Method}
\label{sec:adsampling}
% As discussed in previous sections, the inherent needed resolution varies among different candidates. To ensure preciseness and conciseness, we suggest to adaptively set the resolution. However, there are three main challenges: 1) How can we make resolution flexible? 2) We have no prior knowledge about the difficulty of a query. How do we find the minimum needed resolution? 3) Some queries are extremely fragile. How do we recover full-accuracy distance when needed? 

% \subsection{Motivations and Overview}
% \label{subsec:overview}

% {\CHENG 
% % We propose to conduct \emph{nearly-exact} DCOs based on \emph{approximate distances} with probabilistic guarantees for better efficiency. 
% For better efficiency, it is a natural idea is to conduct DCOs based on \emph{approximate distances}.
% Below, we explain (1) the desiderata an approximate distance based method, (2) insufficiency of existing methods (wrt the desiderata), and (3) the overview of our new method \texttt{ADSampling}.
% % Specifically, we aim to develop a method of computing approximate distances of data vectors and then conduct DCOs based on the approximate distances.
% % Specifically, we aim to compute approximate distances and then conduct DCOs based on the approximate distances. 
% % Apart from the ability of providing an error guarantee for the approximate distance, 

% \smallskip 
% \noindent\textbf{(1) Desiderata of distance approximation methods for DCOs.}
% % \subsubsection{Desiderata of distance approximation methods for DCOs.}
% % We identify four desiderata of the method to fully unleash the power of using approximate distances for reliable DCOs.
% \underline{First}, it should be able to provide an error guarantee of an approximate distance since 
% % otherwise, the accuracy of DCOs would not be guaranteed.
% with the error guarantees of approximate distances, the results of DCOs can be accurate with guarantees and even exact.
% For example, suppose the distance threshold $r$ is 1 and a data vector $\mathbf{o}_1$ has an approximate distance $1.05$ (with a relative error bound $5\%$). Then, we know that the exact distance of $\mathbf{o}_1$ is at least $1.05 / (1+0.05)=1$ (i.e., $r$).
% % , is not smaller than the maximum in $\mathcal{K}$ (i.e., 1).
% \underline{Second}, it should have the \emph{flexibility} of achieving different ``resolutions'' of approximate distances for different datan objects, and correspondingly, it would provide different error bounds for different objects.
% % One problem is that fixed resolution cannot provide exact comparison results for all candidates. 
% Back to the aforementioned example, we consider another candidate $\mathbf{o}_2$. Suppose $\mathbf{o}_2$'s approximate distance to the query, when measured at the same resolution as that of $\mathbf{o}_1$'s approximate distance, is $1.01$, and the corresponding error bound is $5\%$. In this case, we are not able to conclude on whether $\mathbf{o}_2$ has a smaller distance than $r$ or not.
% % the maximum one in $\mathcal{K}$. 
% Instead, we need to increase the resolution so that it would provide a better error bound to be able to make a conclusion, e.g., at a resolution with its corresponding error bound of $1\%$, we know that the exact distance of $\mathbf{o}_2$ is at least $1.01 / (1+0.01)=1$ (i.e., $r$).
% %
% In general, for objects that are farther away from the query vector, approximate distances with lower resolutions would be sufficient for accurate DCOs; whereas for objects that are closer from the query vector, for objects closer from the query vector, approximate distances with higher resolutions (or even full resolutions) are necessary for accurate DCOs.
% % , is not smaller than the maximum in $\mathcal{K}$ (i.e., 1).
% % $dis_{2,A} \ge dis'_{2,A} / (1+1\%) = 1 = dis_{i_K}$.
% }
% % can have exact DCO results with the approximate distance.}
% % {\CHENG For object 2, if the resolution of the approximate distance an error bound of $5\%$ is not sufficient for deciding whether object 2 has a smaller distance from the query object than the current Kth NN in $\mathcal{K}$ or not, but an error bound of $1\%$ is.}
% % Unlike object 1, with the error bound of $5\%$, the order between object 2 and the current Kth NN in $\mathcal{K}$ is undecidable. It becomes decidable once the error bound becomes $1\%$, but then Once we increase the resolution to $1\%$, however, it would be redundant for candidate 1 so as to harm efficiency. Thus, it's of vital importance to realize \textit{flexibility} of resolution, i.e. different resolution for different candidates. 
% % Another problem comes from the way we set the resolution for a certain candidate. For a newly visited object, we have no prior knowledge about it. To find its minimum needed resolution, we need a mechanism to determine it \textit{adaptively}. 
% {\CHENG \underline{Third}, it should have the ability to \emph{adaptively} determine an appropriate resolution of the approximate distance for an object such that the it is neither more than enough (which means some computation of distance evaluation could be saved otherwise) nor lower than necessary (which means a firmed result of DCO cannot be obtained).}
% % Finally, some candidates are extremely fragile to approximation, which really requires exact distance to obtain comparison results. Conventional methods evaluate exact distance with raw vectors from scratch when fragility is detected. However, it gives rise to re-evaluation overhead. We claim that a method should better progressively \textit{recover} exact distance from known approximate distance to guarantee that it's no worse than the plain distance evaluation at any case.
% %
% {\CHENG\underline{Fourth}, it should have the ability to \textit{recover} the exact distance by reaching the highest resolution with the error bound of 
% %1 
% {\JIANYANG 0\%}.
% % from known approximate distance to guarantee that it's no worse than the plain distance evaluation at any case. 
% This is because some candidates have their distances extremely fragile to approximation, and in these cases, exact distances are not avoidable to obtain firmed results of DCO.}
% %
% In summary, to efficiently produce reliable DCO results with approximate distances, a method should have (1) guaranteed error bound, (2) flexibility of multiple resolutions, (3) adaptivity of reaching an appropriate resolution and (4) recoverability of the exact distance. 

% \begin{table}[h]\small
%   \caption{Methods of Distance Approximation}
%   \label{tab:freq}
%   \begin{tabular}{c|cccc}
%     \toprule
%     &DR &QT &RP &\textbf{ADSampling}\\
%     \midrule
%   Error Bound &NO &NO & \textbf{Probabilistic} &\textbf{Probabilistic}\\
%   Flexibility &NO &\textbf{Limited} & NO &\textbf{YES}\\
%   Adaptivity &NO &NO & NO &\textbf{YES}\\
%   Recoverability &NO &NO  & NO &\textbf{YES}\\
% %   \midrule
% %   DCO &NO &NO &NO &\textbf{Probabilistic}\\
%   \bottomrule
% \end{tabular}
% \end{table}

% \smallskip 
% \noindent\textbf{(2) Insufficiency of existing distance approximation methods for DCOs.}
% % \subsubsection{Insufficiency of existing distance approximation methods for DCOs.}
% % Unfortunately, no existing distance approximation methods satisfy all of the four aforementioned desiderata, to the best of our knowledge. 
% % While there there quite a few distance approximation methods that have been developed for AKNN, none of them satisfy all of the four aforementioned desiderata, to the best of our knowledge.
% There are two main forms of distance approximation that have been used for AKNN: 
% % nearest neighbor search: 
% 1) quantization (QT)~\cite{ge2013optimized, jegou2010product,ITQ, additivePQ, guo2020accelerating}, and 2) dimension reduction, in which dimension reduction can be further categorized into optimization-based dimension reduction (DR)~\cite{wold1987principal, kruskal1964multidimensional} and random projection (RP)~\cite{johnson1984extensions, blockjlt, fftjlt}.
% {\CHENG Unfortunately, none of these existing methods has all of the aforementioned four desiderata} (a summary is presented in Table~\ref{tab:freq}). 
% First, QT and DR (e.g., PCA~\cite{wold1987principal}) optimize a compressed representation to minimize the \emph{total} approximation error instead of the \emph{maximum} one, which fails to guarantee an error bound. {\CHENG Only RP provides some probabilistic error bounds.
% Second, DR and RP do not support flexible resolutions of approximate distances.
% % Flexible resolution and is a long but implicitly adopted strategy in QT. 
% Only QT supports flexible resolutions to some extent with a three-stage strategy \cite{jegou2010product, imi, surveyl2hash}:}
% % to apply different resolution on different objects: 
% 1) generate candidate lists with coarse code; 2) shrink the list with finer code; 3) re-rank the list with exact distance~\footnote{Some stages could be skipped according to specific requirements, e.g., memory constraint~\cite{johnson2019billion, jegou2011searching}.}. 
% {\CHENG Nevertheless, the list size and code size at each stage are preset hyper-parameters and fixed for all queries, and thus QT provides very limited flexibility only.}
% % so it's very limited in flexibility and hard to tune. 
% Third, to the best of our knowledge, no existing methods achieve adaptivity and recoverability on resolution of distance for high-dimensional nearest neighbor search. It's worth noting that hashing-based methods~\cite{indyk1998approximate, datar2004locality, c2lsh, tao2010efficient, huang2015query, lu2020vhp} are also popular for data compression. They target to map close vectors to similar hash codes and {\CHENG use code comparison as a proxy of DCO}.
% % DCO with hash code comparison. 
% However, hashing 
% %doesn't 
% does not explicitly approximate distances, and thus they're not within the scope of our discussion.  \footnote{Also, since different vectors may be mapped to the same code, hashing cannot 
% % identify their order.
% {\CHENG help with exact DCO}.
% }
% {\JIANYANG We also emphasize that hashing and quantization cannot provide guarantee for DCO, and thus can be used only in the first stage of AKNN query, i.e., generating candidates but not the second.}

{\CHENGB Recall that our goal is to achieve \emph{reliable} DCOs with better efficiency than \texttt{FDScanning}. To this end,}
we develop a new method called \texttt{ADSampling}. 
At its core, \texttt{ADSampling} projects the objects to vectors with \emph{fewer} dimensions and conduct DCOs based on the projected vectors for better efficiency.
% \emph{random projection} process~\cite{johnson1984extensions, datar2004locality, indyk1998approximate, sun2014srs, c2lsh}, 
Different from the conventional and widely-adopted random projection technique~\cite{johnson1984extensions, datar2004locality, sun2014srs, c2lsh}, which projects \emph{all} objects to vectors with \emph{equal} dimensions, \texttt{ADSampling} is novel in the following aspects. 
First, it projects \emph{different} objects to vectors with \emph{different} numbers of dimensions during the query phase \emph{flexibly}.
We will elaborate on details of how this idea is implemented in Section~\ref{subsec:idea-1}. 
Second, it decides the number of dimensions to be sampled for each object \emph{adaptively} based on 
% the query 
% {\JIANYANGLAST its corresponding DCO}
{\CHENGC the DCO on the object}
during the query phase, but not pre-sets it to a certain number during the index phase (which is knowledge demanding and difficult to set in practice).
We will elaborate on details of how this idea is implemented in Section~\ref{subsec:idea2}. 
In addition, we summarize \texttt{ADSampling} and 
% show
{\JIANYANG prove}
that it has its time \emph{logarithmic} wrt $D$ for negative objects (which is significantly better than the time complexity $O(D)$ of \texttt{FDScanning}) in Section~\ref{subsection:theoretical analysis of ADSampling}.
% {\CHENG
% Recall a DCO operation is to decide for a given object $\mathbf{x}$, whether its distance $dis$ is 
% % larger than a 
% smaller than or equal to a threshold $r$ (i.e., $dis \le r$). As explained in Section~\ref{sec:introduction}, almost all existing AKNN algorithms (such as the \texttt{HNSW} algorithm and the \texttt{IVF} algorithm) have their time costs largely dominated by that of DCO operations. Therefore, the efficiency of this operation is critical and largely determines that of an AKNN algorithm. 

% In an AKNN algorithm, this operation is typically conducted by evaluating the distance and comparing the distance with $r$, which has the time complexity of $O(D)$. Since $D$ is usually hundreds and even thousands for high-dimensional objects in practice and this operation is frequently conducted by an AKNN algorithm, the efficiency of the AKNN algorithm is largely affected. To achieve better efficiency of an AKNN algorithm, we aim to conduct the DCO operation \emph{near-exactly} based on \emph{approximate distances}. 
%
% \smallskip 
% \noindent\textbf{(3) Overview of a new method \texttt{ADSampling} for DCOs.}
% % \subsubsection{Overview of a new method \texttt{ADSampling} for DCOs.}
% In this paper, we develop a method called \texttt{ADSampling}, which has all the aforementioned four desiderata with two core ideas. 
% In this paper, we develop a method called \texttt{ADSampling}, 
% % which projects different objects to vectors with different numbers of dimensions \emph{flexibly} and \emph{adaptively} and then computes approximate distances based on the sampled dimensions for reliable DCOs.
% {\JIANYANG 
% which \emph{flexibly} and \emph{adaptively} samples different number of dimensions for reliably conducting the DCOs of different objects. 
% }
% Specifically, it has two core 
% % ideas
% {\JIANYANG steps.}
% % which has all the aforementioned four desiderata with two core ideas. 
% \underline{First}, 
% {\JIANYANG during the index phase, it preprocesses data vectors with \emph{random orthogonal transformation}~\cite{johnson1984extensions, choromanski2017unreasonable, randomortho, vershynin_2018}.}
% % it randomly transforms an object with \emph{random orthogonal transformation}~\cite{johnson1984extensions, choromanski2017unreasonable, randomortho, vershynin_2018} and computes an approximate distance based on some sampled dimensions of the transformed vector. 
% % {\JIANYANG \underline{First}, we consider incorporating the power of random projection into our method. Specifically, we target to make row sampling on vectors equivalent to random projection so as to utilize Lemma~\ref{lemma:concen}, and hopefully, sampling all $D$ dimensions equivalent to evaluating exact distance.} 
% % We will show in Section~\ref{subsec:idea-1} that this can be achieved by pre-processing data vectors (during index phase) and query vectors (during query phase) with random orthogonal transformation.
% % This idea enables \texttt{ADSampling} to achieve \emph{guaranteed error bound} on the approximate distances, \emph{flexible resolutions} of approximate distances, and \emph{recoverability} of exact distance.
% This idea enables \texttt{ADSampling} to achieve \emph{flexible resolutions} of the approximate distances since it can sample different numbers of dimensions of the transformed vectors of different objects.
% %
% In addition, 
% {\JIANYANG the approximate distances computed from sampled dimensions has some equivalence to those from the low-dimensional vectors produced by random projection~\cite{johnson1984extensions, vershynin_2018} and thus, share all its advantages including low error rate and rigorous theoretical guarantee.
% }
% % the errors of the computed approximate distances are bounded, which we verify by establishing an equivalence between the vector of sampled dimensions and that produced by random projection~\cite{johnson1984extensions, vershynin_2018}.
% %
% We will elaborate on more details of this idea in Section~\ref{subsec:idea-1}. 
% \underline{Second}, {\JIANYANG during the query phase, when handling a DCO,} it samples the dimensions of a transformed vector in an \emph{aggressive} and \emph{incremental} manner (i.e., it starts with a few dimensions and samples more and more dimensions) and leverages hypothesis testing for deciding whether a sufficiently confident conclusion for a DCO can be made with the approximate distance computed based on the dimensions sampled so far. This idea enables \texttt{ADSampling} to achieve the \emph{adaptivity} of sampling the minimum number of dimensions for reliable DCOs. 
% We will elaborate on more details of this idea in Section~\ref{subsec:idea-2}. 
% Finally, we summarize \texttt{ADSampling} and 
% % show
% {\JIANYANG prove}
% that it has its time \emph{logarithmic} wrt $D$ for negative objects (which is significantly better than the time complexity $O(D)$ of \texttt{FDScanning}) in Section~\ref{subsection:theoretical analysis of ADSampling}.
% }

%To solve these issues, we propose adaptive dimension sampling (AdaSampling) framework. AdaSampling is a probabilistic DCO framework for high-dimensional Euclidean space. It automatically adapts to the inherent difficulty of each comparison. The idea is as follows: for two vectors $\mathbf{o}_i$ and $\mathbf{o}_j$, to compare their distance to query $\mathbf{q} $, supposing that we have the true distance $dis_i$ of object $i$, we first sample a small number of dimensions and calculate the observed distance $dis'_j$ for object $j$ on the sampled space. Then with the observed distance, if we don't have enough confidence to conclude their order, then we "level it up" by sampling more dimensions. Repeatedly, we keep leveling up object $j$ until their order can be confidently determined. 

%{\color{red} I need some figures here.}
\subsection{\textbf{Dimension Sampling over {\JIANYANG Randomly} Transformed Vectors}}
\label{subsec:idea-1}

{\CHENG 
% To achieve flexible resolutions of approximate distances, a natural idea is to take different numbers of dimensions of an object flexibly. However, simply taking several dimensions of an object can hardly preserve the distance of the object in practice, let alone guaranteed error bound. To achieve error bounds of approximate distances, 
{\CHENGB For better efficiency of a DCO, a natural idea is} to conduct a random projection~\cite{johnson1984extensions, vershynin_2018} on an object (i.e., to multiply the object (specifically its vector) with a $\mathbb{R}^{d\times D} $ random matrix $P$ where $d < D$~\footnote{{\JIANYANG There are multiple types of random matrices used for random projection~\cite{blockjlt, fftjlt, datar2004locality, johnson1984extensions}. 
% The most popular type in the database community is random Gaussian matrix whose entries are independent standard Gaussian random variable~\cite{datar2004locality, c2lsh, sun2014srs, huang2015query}. 
In the present work, by random projection, we refer to the random projection based on random orthogonal matrix, which can be generated through orthonormalizing a random Gaussian matrix, {\JIANYANGB whose entries are independent standard Gaussian random variables}~\cite{choromanski2017unreasonable, randomortho, johnson1984extensions, vershynin_2018}.}}),
{\CHENGB and then conduct the DCO using the approximate distance that can be computed based on the projected vector, namely $\sqrt {D/d } \| P \mathbf{x}\|$.}
%
It is well-known that there exists a concentration inequality on the approximate distance as presented in the following lemma~\cite{vershynin_2018}.
%
\begin{lemma}%~\cite{vershynin_2018}
    For a given object $\mathbf{x} \in \mathbb{R}^D  $, a random projection $P \in \mathbb{R}^{d\times D} $ preserves its Euclidean norm with $\epsilon $ multiplicative error bound with the probability of
\begin{align}
    \mathbb{P} \left\{ \left| \sqrt {\frac{D}{d} } \| P \mathbf{x}\| -\| \mathbf{x} \| \right| \le \epsilon \| \mathbf{x} \|  \right\}   \ge 1 - 2e^{-c_0 d \epsilon ^2} 
    \label{equ:concentration}
\end{align}
%    The corresponding failure probability is bounded by
%\begin{align}
%    \mathbb{P} \left\{ \left| \sqrt {\frac{D}{d} } \| P \mathbf{x}\| -\| \mathbf{x} \| \right| > \epsilon \| \mathbf{x} \|  \right\}   \le 2e^{-c_0 d \epsilon ^2} \notag 
%\end{align}
    where $c_0$ is a constant factor and {\JIANYANGREVISION $\epsilon \in (0, +\infty)$}. \label{eq:fail_concen}\label{eq:concen}
    \label{lemma:concen}
\end{lemma}
%
Nevertheless, once an object is projected, the corresponding approximate distance would have a certain resolution {\CHENGC that would be fixed}.
% the reduced dimensionality $d$ is usually fixed and uniform for all objects and thus 
Therefore, it lacks of flexibility of achieving different reduced dimensionalities for different objects (correspondingly different resolutions of approximate distances) during the query phase.

% We aim to combine merits of both the dimension sampling (i.e., its flexibility) and the random projection (i.e., its guaranteed error bounds) while avoiding their shortcomings. 
{\CHENGB We aim to project \emph{different} objects to vectors with \emph{different} numbers of dimensions during the query phase \emph{flexibly}.}
To this end, we propose to \emph{randomly transform} an object (with \emph{random orthogonal transformation}~\cite{johnson1984extensions, vershynin_2018, ITQ}, geometrically, to randomly rotate it) and then flexibly sample dimensions of the transformed vector for computing an approximate distance.
% sample conduct \emph{dimension sampling} on the \emph{random transformed} of data vector via random orthogonal transformation. 
Formally, given an object $\mathbf{x}$, we first apply a \emph{random orthogonal matrix} $P' \in \mathbb{R}^{D \times D} $
% ~\footnote{Following \cite{choromanski2017unreasonable}, we generate random orthogonal matrix $P'$ by (1) generating a square random Gaussian matrix $G \in \mathbb{R}^{D\times D}$, (2) orthogonalizing $G$ with QR decomposition $G=QR$, where $Q$ is an orthogonal matrix and $R$ is an upper triangular matrix, and (3) using $Q$ as $P'$.} 
to $\mathbf{x}$ and then sample $d$ rows on it (for simplicity, the first $d$ rows).
The result is denoted by $(P' \mathbf{x})|_{[1,2,...,d]}$. 
%
% Here, the random orthogonal matrix $P'$ can be obtained by (1) generating a square random Gaussian projection matrix $G\in \mathbb{R}^{D\times D} $, (2) orthogonalizing $G$ with QR decomposition $G=QR$, where $Q$ is an orthogonal matrix and $R$ is an upper triangular matrix, and (3) using $Q$ as $P'$~\cite{choromanski2017unreasonable, yu2016orthogonal}.
%
{\CHENGB This method entails two benefits.} 
First, we achieve the flexibility since we can sample $d$ dimensions of a rotated vector for different $d$'s during the query phase. Second, we achieve a guaranteed error bound since sampling $d$ dimensions on a transformed vector is equivalent to obtaining a $d$-dimensional vector via random projection, which we explain as follows. 

% Suppose that we conduct a random projection operation on $\mathbf{x}$. It would apply 
Recall that a random projection on $\mathbf{x}$ is to apply a random projection matrix $P \in \mathbb{R}^{d\times D}$ to $\mathbf{x}$, and the result is denoted by $P \mathbf{x} $.}
% To achieve 
% % the flexibility of resolution
% {\CHENG a flexible resolution of approximate distances}, a straight-forward idea is to {\CHENG take different numbers of dimensions of a data vector flexibly}. However, 
% % for distance evaluation, each vector is a whole. 
% simply taking several dimensions of a data vector can hardly 
% % produces reasonable results, 
% {\CHENG preserve the distance of the data vector}, let alone guaranteed error bound. 
% %Though optimization-based methods can align subspaces according to their importance, for unseen queries, their performances are totally unpredictable.
% To resolve this issue, we propose to conduct \emph{dimension sampling} on randomly transformed vectors, for which we next show its equivalence to random projection. Such equivalence enables us to ``set different dimensionality of random projection for different candidates'' during query time, which brings flexibility. 
%
% %. To provide clear interpretation and rigorous guarantee for it, we first show the equivalence between random projection and row sampling on randomly transformed vectors.
%
% Consider that we target to reduce the dimensionality of vector $\mathbf{x} $ from $D$ to $d$. A conventional random projection operation is to apply a random projection matrix $P \in \mathbb{R}^{d\times D} $ to $\mathbf{x}$ and get $P \mathbf{x} $. However, in terms of our proposal, we first apply a random orthogonal matrix $P' \in \mathbb{R}^{D \times D} $ to $\mathbf{x} $, sample $d$ rows on it (for simplicity, the first $d$ rows) and get $(P' \mathbf{x})|_{1,2,...,d}$. 
% We claim the above two operations are equivalent in the sense that 
We claim that $(P' \mathbf{x})|_{[1,2,...,d]}$ (the result of our proposed method) and $P \mathbf{x}$ (the result of a random projection) are identically distributed. This is based on an elementary property of matrix multiplication 
% as intuitively illustrated in Figure~\ref{fig:equivalence}, 
that row samplings before and after a matrix multiplication are identical:
%Random projection is to apply a random projection matrix $P \in \mathbb{R}^{d\times D} $ to $\mathbf{x} $. In terms of row sampling on randomly transformed vectors, it first applies a random orthogonal matrix $P' \in \mathbb{R}^{D\times D} $ to $\mathbf{x}$ , and samples $d$ rows on it (for simplicity, the first $d$ rows) to get $(P' \mathbf{x})|_{1,2,...,d}$. We claim the above two methods are equivalent in the sense that $(P' \mathbf{x})|_{1,2,...,d} $ and $P \mathbf{x} $ are identically distributed. This conclusion comes from an elementary property of matrix multiplication, {\color{red} i.e. row sampling before and after a matrix multiplication are identical}:
\begin{align}
    (P'\mathbf{x} )|_{[1,2,...,d]} = P'|_{[1,2,...,d]} \mathbf{x} 
\end{align}
%
% Recall that as introduced in Section~\ref{subsec:rp}, 
We note that {\CHENGB $P'|_{[1,2,...,d]}$ corresponds to a random matrix $P$ for random projection since} one conventional way to generate a random projection matrix $P$ is to sample rows of a $D\times D$ random orthogonal matrix~\cite{choromanski2017unreasonable}.
% , and thus $P'|_{1,2,...,d}$ and $P$ are identically distributed. 
% As a result, sampling $d$ dimensions on a randomly transformed vector is equivalent to % evaluating a $d$-dimensional embedding from {\CHENG obtaining a $d$-dimensional vector via} random projection. 
%
% Note that mathematically, we simply shift row sampling from projection matrices to transformed vectors, while algorithmically, such shift enables us to set dimensionality at any time we want, which brings flexibility. What's more, 
%
Therefore, the concentration inequality for random projection over raw objects (as given in Equation~(\ref{equ:concentration})) can be applied to dimension sampling over {\JIANYANG randomly} transformed vectors, which provides solid foundation for our following discussion. 
% For the sake of convenience, we denote the transformed vectors as $\mathbf{y} := P' \mathbf{x}$.
%

{\CHENGB We denote the transformed vector as $\mathbf{y} := P' \mathbf{x}$.}
{\CHENG Based on the sampled dimensions, we can compute an approximate distance of $\mathbf{x}$, denoted by $dis'$, as follows,
\begin{align}
    dis' := \sqrt { \frac{D}{d}} \left\| \mathbf{y}|_{[1,2,...,d]} \right\|  
    % = \sqrt { \frac{D}{d} \sum_{j=1}^d y_{j}^2} 
\end{align}
where $d$ is the number of sampled dimensions. We note that the time complexity of computing an approximate distance based on $d$ sampled dimensions is $O(d)$.
%
Furthermore, when all $D$ dimensions are sampled, the distance $dis'$ computed based on the sampled dimensions would be equal to the true distance $dis$, which is due to the fact
{\JIANYANG that random orthogonal transformation preserves the norm of any vector (since it simply rotates the space without distorting the distances). 
% This is the reason we adopt random orthogonal matrix instead of some random Gaussian matrices that are commonly used for random projection.
}
}
%a randomly transformed vector has its norm the same as the original vector. This is the reason why we randomly transform each data vector before we sample its dimensions.}

% {\CHENG 14-07-Cheng: If possible, we can include some remarks/discussions on any new ideas proposed, e.g., we can discuss why we adopt random orthogonal project instead of conventional random projection {\JIANYANG it's now discussed in section 3.1.3}, we can explain how we use random orthogonal projection different from other existing studies that use random orthogonal projection, etc.} {\JIANYANG it's now discussed in section 3.1.4}

% {\JIANYANG 15-07-Jianyang: I modified the figures and illustrated Figure.3(c) in the next page. }

% {\CHENG 22-07-Cheng: Some discussions you included can be too late. We can discuss further on these issues.}

%A standard way, Gaussian random projection, is to generate a random projection matrix $G \in \mathbb{R}^{d\times D} $ with i.i.d. standard Gaussian entries scaled by a factor of $1/\sqrt {d} $. The second way we consider here is to first generate a square transformation matrix $G' \in \mathbb{R}^{D \times D} $ in the same way. Then by sampling $d$ rows of $G'$ (for simplicity we take the first $d$ rows), we obtain a submatrix $G'|_{[1,2,...,d]} \in \mathbb{R}^{d\times D} $. 

%We aim to show that the distribution of $G'|_{[1,2,...,d]}$ is identical to that of $G$, i.e. :
%\begin{align}
%    \forall A \in \mathbb{R}^{d\times D}, p_{G}(A) = \int_{\mathbb{R}^{(D-d) \times D} } p_{G'}(A, B) \mathrm{d} B  \label{eq:eq1}
    %\mathbb{P}\left[ G = A \right]  = \int \mathbb{P} \left[ G'|_{[1,2,...,d]}=A, G'|_{[d+1,...,D]} = B \right]  \mathrm{d} B
%\end{align}
%where $A \in \mathbb{R}^{d\times D} ,B \in \mathbb{R}^{(D-d) \times D} $ are submatrices with $d$ and $D-d$ rows. $p_G (A)$ and  $p_{G'} (A, B)$ are the probabilistic density functions of $G$ and $G'$ correspondingly. We marginalize the probability over $B$ because only the first $d$ rows are taken. For random Gaussian projection, Equation (\ref{eq:eq1}) obviously holds because any $d$ rows are generated exactly in the same way of $G$. Thus, we claim that random projection and row sampling with random transformation are identical in distribution for the scaled i.i.d. Gaussian case. {\color{red} would it be distracting if I repeatedly emphasize Gaussian? These properties don't trivially hold for any random projections. I'll state that they hold for random orthogonal transformation later. }

%Then an another elementary fact is that sampling before and after projection are equivalent. It is a basic property of matrix multiplication, shown as 
%\begin{align}
%    G'|_{[1,2,...,d]} \mathbf{x}  = (G'\mathbf{x} )|_{[1,2,...,d]}
%\end{align}

%Consequently, we note that $G \mathbf{x} $ and $(G'\mathbf{x})|_{[1,2,...,d]}$ are identically distributed. During indexing phase, we perform a $G'$ random transformation on all database vectors. Then during query phase, we first transform the query vector and next we can sample a certain amount of dimensions as needed. With its equivalence to random projection, we realize the flexibility of dimensionality. In the following sections, for simplicity, we denote the transformed residual vectors as $\mathbf{y}_i=G' \mathbf{o}_i - G' \mathbf{q} $. {\color{red} I may need figures here.}

%{\color{red} To achieve the flexibility of dimensions, we'll need to ensure that a subspace spanned by canonical basis is meaningful. bad sentence, need refine}.  We consider two ways to generate random projection matrix $G \in \mathbb{R}^{d \times D} $. In the first way, we simply generate each entry with standard Gaussian distribution $g_{ij} \sim \mathcal N (0,1)$. In the second way, we generate a larger matrix $G \in \mathbb{R} ^{D\times D} $ and uniformly sample $d$ rows from it (for simplicity, take the first $d$ rows).{\color{red} it's intuitive that they're equivalent.}

%{\color{red} It's also obvious that the sampling before and after the projection is equivalent.}

%{\color{red} Do I need permutation invariance here? NO I don't, it might confuse readers, but if I use fast orthogonal Johnson-Lindenstrauss Transformation, I may need to do so.}

% \begin{figure}[htb]
%     \centering
%     \subfigure[Equivalence]{
%         \includegraphics[height=0.3\linewidth]{figure/equivalence.pdf}
%     \label{fig:equivalence}
%     }
%     \subfigure[Level Up]{
% 	\includegraphics[height=0.3\linewidth]{figure/levelup.pdf}
% 	\label{fig:levelup}
%     }
%     \caption{Dimension Sampling}
% \end{figure}

% \subsection{\textbf{Sequential Hypothesis Testing}}
% \subsection{\textbf{Aggressive and Incremental Sampling with Hypothesis Testing}} 
\subsection{{\JIANYANGREVISION \textbf{Incremental Sampling with Hypothesis Testing}}}
\label{subsec:idea-2}
\label{subsec:idea2}

{\CHENG 
% Recall the problem is to decide whether the distance of an object $\mathbf{o}$ is larger than the threshold $r$ (i.e., $dis > r$) or not based on some sampled dimensions of the randomly rotated vector of $\mathbf{x}$ (i.e., $\mathbf{y}$). 
% We have proposed to sample some dimensions of the randomly rotated vector of $\mathbf{x}$ (i.e., $\mathbf{y}$)
One remaining issue is how to determine the number of dimensions of $\mathbf{y}$ we need to sample in order to make a sufficiently confident conclusion for the DCO (i.e., to decide whether $dis \le r$). Intuitively, with more sampled dimensions, the approximate distance $dis'$ would be 
% closer to the true distance $dis$ with guaranteed probability,
{\JIANYANGB more accurate, }
% (according to the aforementioned Property 2), 
and we would be able to make a more confident conclusion. On the other hand, sampling more dimensions would result in higher cost of computing the approximate distance (since the cost is linear wrt the number of sampled dimensions). We aim to sample the minimum possible number of dimensions, which are sufficient to make a confident conclusion. 

Specifically, we propose to sample the dimensions of $\mathbf{y}$ in an 
% \emph{aggressive} and 
{\JIANYANGREVISION \emph{incremental}}
manner, i.e., we start with a few dimensions. If with the current sampled dimensions, we cannot make a confident conclusion, we continue to sample 
% more dimensions
{\JIANYANGLAST some more}
until we can make a confident conclusion or we have sampled all dimensions. 
% {\JIANYANG (Similar techniques have also been used in sampling binary hashing codes for shortlisting candidates ~\cite{sequentialLSH, satuluri2011bayesian}.)}
% In the case that we sample all dimensions, the distance computed based on the sampled dimensions would be equal to the \emph{true} distance, as explained in Section~\ref{subsec:idea-1}. With the true distance, we can make an \emph{exact} conclusion.
%
As a result, the problem reduces to the one of deciding whether we can make a sufficiently confident conclusion with a certain, say $d$, sampled dimensions? In a statistics language, the observed distance $dis'$ (computed based on the sampled dimensions) is an estimator of the true distance $dis$ and its distribution depends only on the true value $dis$ and the number of sampled dimensions $d$. The task is to draw a conclusion about a true value $dis$ (i.e., whether $dis \le r$) with an observed value $dis'$. It's exactly what \emph{hypothesis testing} typically does. Motivated by this, we propose to leverage hypothesis testing to solve the problem.
% of deciding whether we can make a sufficiently confident conclusion on whether $dis \le r$ with $dis'$. 
Specifically, we conduct the hypothesis testing as follows.

\begin{enumerate}
    \item We define a null hypothesis $H_0:dis \le r$ and its alternative $H_1: dis > r$.
    % \begin{equation}
    %     H_0:dis \le r~~and~~H_1: dis > r
    % \end{equation}
    \item We use $dis'$ as the estimator of $dis$. The relationship between $dis'$ and $dis$ is provided in Lemma~\ref{eq:concen} (i.e., the difference between $dis'$ and $dis$ is bounded by $\epsilon\cdot dis$ with the failure probability at most $2\exp (-c_0 d  \epsilon^2)$).
    \item We set the significance level $p$ to be $2 \exp(-c_0  \epsilon_0^2)$, where $\epsilon_0$ is a parameter to be tuned empirically. 
    % Then, the rejection region of $dis'$ corresponds to a range in the form of $( (1 + \epsilon_0/\sqrt{d}) \cdot r, +\infty)$.
    % % (which can be derived based on Property 2). 
    % The intuition is that 
    % We derive that (1) 
    {\CHENGB With this,} the event that the observed $dis'$ is much larger than $r$ (i.e., $dis' > (1 + \epsilon_0/\sqrt{d})\cdot r$) has its probability below the significance level $p$ (which can be verified based on Lemma~\ref{eq:concen} with $\epsilon = \epsilon_0/\sqrt{d}$ and $H_0:dis \le r$).
    % and (2) if this event happens (i.e., $dis'$ falls in the rejection region), it implies that $H_0$ is not true with sufficient confidence.
    \item We check whether 
    % $dis'$ falls in the rejection region, i.e., 
    {\CHENGB the event happens} ($dis' > (1 + \epsilon_0/\sqrt{d})\cdot r$). If so, {\CHENGB we can reject $H_0$ and conclude} $H_1: dis > r$ {\CHENGB with sufficient confidence}; otherwise, we cannot.
\end{enumerate}

There are three cases for the outcome of the hypothesis testing. \underline{Case 1}: we reject the hypothesis (i.e., we conclude $dis > r$) and $d < D$. In this case, the time cost (which is mainly for evaluating the approximate distance) is $O(d)$, which is smaller than that of computing the true distance in $O(D)$ time. \underline{Case 2}: we cannot reject the hypothesis and $d < D$. In this case, we would continue to sample some more dimensions of $\mathbf{y}$ \emph{incrementally} and conduct another hypothesis testing. \underline{Case 3}: $d = D$. In this case, we have sampled all dimensions of $\mathbf{y}$ and the approximate distance based on the sampled vector is equal to the true distance. Therefore, we can conduct an \emph{exact} DCO. We note that the 
% aggressive and 
{\JIANYANGREVISION incremental} dimension sampling process with (potentially sequential) hypothesis testing would have its time cost strictly smaller than $O(D)$ (when it terminates in Case 1) and equal to $O(D)$ (when it terminates in Case 3). 
}
% The next question is about the way we find the minimum needed dimensionality for a particular candidate. In general, we design an adaptation-based mechanism to ensure that it keeps sampling until we have enough information. 

{\CHENGB We note that hypothesis testing has also been used for deciding a certain number of hashes for LSH in the context of similarity search~\cite{sequentialLSH, satuluri2011bayesian}. The differences between our technique and \cite{sequentialLSH, satuluri2011bayesian} include: (1) ours is based on a random process of sampling dimensions of a transformed vector while \cite{sequentialLSH, satuluri2011bayesian} are on one of sampling hash functions, which entail significantly different hypothesis testings and (2) ours targets the Euclidean distance function while \cite{sequentialLSH, satuluri2011bayesian} target similarity functions such as Jaccard and Cosine similarity measures (it remains non-trivial to adapt the latter to the Euclidean space),}
{\JIANYANGB and (3) ours guarantees to be no worse than the method of evaluating exact distances (in our case, i.e., \texttt{FDScanning}) because it obtains exact distances when it has sampled all the dimensions while \cite{sequentialLSH, satuluri2011bayesian} have no such guarantee (when they have sampled all the hash functions and still cannot produce a firmed result, they would have to re-evaluate exact similarities from scratch).}
 
%For two given objects $i$ and $j$, the goal is to find the one with smaller distance. Assume that we already have object $i$'s true distance $dis_i$ and object $j$'s $d$-dimensional observed distance $dis'_j$, which is formally defined by

% Consider that we are dealing with a new candidate $i$, i.e. compare its distance with the current maintained $dis_{i_K}$ in $\mathcal K$. In particular, we want to check whether its distance is smaller than $dis_{i_K}$ so as to be able to update $\mathcal K$. 
% Assume that we have already sampled $d$ dimensions and have the corresponding $d$-dimensional observed distance, which is formally defined by: 
% %Consider that we are comparing distance for two given objects $i$ and $j$. Under the context of NN query, in paticular, we want to find out the one with smaller distance. Assume that we already have object $i$'s true distance $dis_i$ and object $j$'s $d$-dimensional observed distance, which is formally defined by
% \begin{align}
%     dis'_i := \sqrt { \frac{D}{d}} \left\| \mathbf{y}_i|_{[1,2,...,d]} \right\|   = \sqrt { \frac{D}{d} \sum_{j=1}^d y_{i,j}^2} 
% \end{align}
% Our questions are: 1) Is $dis_i$ smaller than $dis_{i_K}$? 2) Just with the observed distance $dis'_i$, how much confidence do we have? 

% From the perspective of statistics, the observed distance $dis_i'$ is an estimator of the true distance $dis_i$. Its distribution depends only on the true value $dis_i$ and the number of samples $d$, and luckily, is explicitly and sharply bounded by concentration inequality. Now our task is draw conclusions about a true value $dis_i$ (whether it's smaller than $dis_{i_K}$) with an observed value $dis_i'$.
% %and the number of samples $d$. 
% It's exactly what hypothesis testing does. 

% %Note that based on Lemma~\ref{eq:concen}, the distribution of $dis_j'$ concentrates around $dis_j$ as shown in Figure~\ref{fig:concen}.  Meanwhile, as the sampled dimensionality $d$ increases, it becomes more concentrated. This phenomenon matches our intuition that the more samples we have, the less uncertain we are. Now we want to draw conclusions about the true distance $dis_j$ with observed data $dis_j'$. It's exactly what statistical hypothesis testing does. 



% %Statistical hypothesis testing is base on the philosophy of falsificationism: with finite observed data, we cannot fully accept a statement, but with one counterexample, we can reject it. We hypothesize that 
% Specifically, we propose a null hypothesis $H_0$ and the corresponding alternative hypothesis $H_1$ that
% \begin{align}
%     H_0: dis_i \le dis_{i_K}, H_1: {\CHENG dis_i > dis_{i_K} } \notag
% \end{align}
% If hypothesis $H_0$ holds, then the observed distance $dis'_i$ is almost impossible to be much larger than $dis_{i_K}$ because it should fall around the true distance $dis_i$ with high probability. On the contrary, if we observe that $dis'_i \gg dis_{i_K}$ 
% %(i.e. falling in rejection region)
% , either an extremely-low-probability event happens, or the hypothesis $H_0$ is wrong. In this case, we can confidently reject it, terminate sampling and claim $dis_i > dis_{i_K}$.
% %, as well as returning object $i$ as the nearer object. 
% However, when we cannot reject it, unlike conventional hypothesis testing, we don't accept $H_0$ immediately. According to \textit{Property 2} of Lemma~\ref{eq:concen}, as the sampled dimensionality $d$ increases, the distribution of $dis_i'$ becomes more concentrated. This phenomenon matches our intuition that the more samples we have, the less uncertain we are. Thus, we continue sampling to collect more evidence, and then do another hypothesis testing. The process is repeated until $H_0$ is rejected or all $D$ dimensions are evaluated. \footnote{Rigorously, our hypothesis testing is based on multiple statistics. Only if none of them can reject $H_0$, do we accept $H_0$.}



% \begin{figure*}[htb]
%     \centering
% \subfigure[$dis_i \le r$]{
% 	\includesvg[width=0.47
% 	%0.47
% 	\linewidth]{figure/ADSamplingPositive.svg}
% 	\label{fig:ADSamplingpos}
% }
% \subfigure[$dis_i > r$]{
% 	\includesvg[width=0.47
% 	%0.47
% 	\linewidth]{figure/ADSamplingNegative.svg}
% 	\label{fig:ADSamplingneg}
% }
% \caption{ADSampling}
% \label{fig:ADSampling}
% \end{figure*}

% % {\CHENG 14-07-Cheng: Comments on the above figure: (1) Can better put all the matrices and vectors upside down so that the level up operation looks more like going one level up; (2) in Figure (a), it looks like one box represents one entry (and one dimension in a vector) while in Figure (b) and (c), one box can represent a few dimensions. Some revision is needed for better consistency; (3) Figure 3 needs further elaborations (e.g., what does it mean by positive, lines with arrows, and distributions, etc.). Currently, this figure is simply referred to without much explanation (you may give it some more thoughts whether this figure really illustrates the ADSampling framework. In addition, the font size in this figure is too small.}

% % {\JIANYANG The figures are modified. May we have a discussion? I thought there might not be much space for us to illustrate it in this way?}

% More specifically, we consider progressively doing hypothesis testing for $L$ times, each at dimensionality $d_1,d_2,...,d_L$ respectively, where $0 = d_0 < d_1 <... <d_L = D$. For an integer $l\in[0,L]$, we say an object is currently at level $l$ if we have evaluated $d_{l}$ dimensions of it. The current level of object $i$ is denoted as $l_i$. In case that $H_0$ cannot be rejected at level $l_i$, we next sample another $d_{l_i+1}-d_{l_i}$ dimensions (for simplicity, we just take the consecutive ones). For simplicity, we name this operation as "level up", which is described in Algorithm~\ref{code:levelup} and depicted in Figure~\ref{fig:levelup}, where line 1 recovers the sum of squared difference of evaluated dimensions, line 2-3 evaluate new dimensions and line 4-5 update level and observed distance. In particular, $y_{i,j}$ is computed with the transformed database vector $P' \mathbf{o}_i $ and query vector $P' \mathbf{q}$ on the fly, formally $y_{i,j}=(P' \mathbf{o}_i )_j -(P' \mathbf{q} )_j$.
% \input{pseudocode/levelup}
% % as described in Algorithm~\ref{code:levelup}. 
% %\input{pseudocode/levelup}
% %It recovers the squared distance in line 1, evaluates new dimensions in line 2-4 and updates the level and observed distance in line 5-6. 
% %it doesn't mean $dis_j \le dis_i$. It means that we don't have enough evidence yet, so more dimensions should be sampled. \footnote{Unlike conventional hypothesis testing, we don't accept $H_1$ immediately when $H_0$ cannot be rejected. Rigorously our hypothesis testing is based on multiple statistics. Only if we fail to reject $H_0$ with all of them, do we accept $H_1$.}

% 
\begin{comment}

                        
\begin{algorithm}
	\caption{ADSampling ($i$, $dis_{comp}$)} 
	\begin{algorithmic}[1]
	    \State Initialize $l \leftarrow 0, dis' \leftarrow 0$.
        \While{$l<L$ \textbf{and} $dis' \le \gamma(d_{l}) \cdot dis_{comp}$ }
            \State $(i, l, dis') \leftarrow \mathrm{level\ up} (i, l, dis')$.
        \EndWhile
	    \State Return $(l, dis')$. 
	\end{algorithmic} 
    
\end{algorithm}
\end{comment}

\IncMargin{1em}
\begin{algorithm}
\DontPrintSemicolon
\SetKwFunction{LevelUp}{LevelUp}
\SetKwData{and}{and}
\SetKwInOut{Input}{Input}\SetKwInOut{Output}{Output}
\Input{The identity of an object and the distance comparison threshold $(i, r)$}
\Output{The result of distance comparison ($1$ for $dis_i<r$ and $0$ for $dis_i \ge r$) and the observed distance when sampling is terminated $(result, dis'_i)$}
\BlankLine
\emph{Initialize $l \leftarrow 0, dis'_i \leftarrow 0$}\;
\While{$l < L$ \textbf{and} $dis'_i \le \gamma(d_l) \cdot r$}{
    \emph{\LevelUp{$i, l, dis'_i$}}\;
}
\If{$l=L$ \textbf{and} $dis'_i < r$}{
    \emph{$result \leftarrow 1$}\;
}
\Else{
    \emph{$result \leftarrow 0$}\;
}
\KwRet{$(result, dis'_i)$}
\caption{ADSampling}\label{code:adasampling}
\end{algorithm}\DecMargin{1em}


% Quantitatively, we investigate \textit{Property 2} again to obtain the rejection region of hypothesis testing. We preset the significance (i.e., the false negative failure probability) of a single hypothesis testing to be $2\exp(-c_0\epsilon_0^2)$ ({\CHENG ***explain (1) we can do hypothesis without knowing the exact value of $c_0$; (2) we set $\epsilon_0$ empirically***}), where $\epsilon_0$ is a parameter we actually tune to control the failure probability. 
% {\JIANYANG Note that the physical interpretation of $\epsilon_0$ is the multiplicative error bound of $d=1$. Based on our discussion in \textit{Property 2}, when the significance is fixed, the error bound for $d < D$ can be given by $\epsilon_0/\sqrt {d} $, which can be computed with only $\epsilon_0$ and $d$. Thus, we can do hypothesis testing without knowing the exact value of $c_0$. In terms of the actual significance, we refer readers to our empirical study in Figure~\ref{figure:verification}.
% ***This significance is for a single hypothesis testing, while the empirical result is for a DCO. ***}
% %{\JIANYANG Note that $\epsilon_0$ is also the error bound of $d=1$. In this case, the random projection matrix consists of only one row. Note that for one row, there is no orthogonalization operation, so it is indeed generated with normalized i.i.d standard Gaussian entries. Thus, the result of random projection is also a Gaussian random variable and the error bound $(1+\epsilon_0)$ corresponds to the failure probability of $(1+\epsilon_0)\sigma$-region of Gaussian distribution (e.g., $\epsilon_0=1, 2, 3$ corresponds to $2\sigma, 3\sigma, 4\sigma$-region, whose failure probability is around $5\%, 0.3\%, 0.006\%$). }
% %Based on \textit{Property 2}, for a given dimensionality $d_{l_i} < D$, to reach the preset significance, the corresponding multiplicative error bound should be $\epsilon _1/\sqrt {d_{l_i}}$. 
% Then the rejection region is given as:
% \begin{align}
%     dis_i' > \gamma (d_{l_i}) \cdot dis_{i_K}  
%     \label{eq:testing}
% \end{align}
% where $\gamma(d)$ is defined to be $\gamma(d):=1 + \epsilon _1 / \sqrt {d} $ for $d<D$. Intuitively, it implies that we can confidently claim $dis_i > dis_{i_K}$ when the observed distance $dis_i'$ is $\gamma(d_{l_i})$ times larger than $dis_{i_K}$. In discussion of the parameter $\epsilon_0$, we consider two extreme cases. When $\epsilon_0 = 0$, it means that we fully trust the comparison result after random projection: only if $dis_i' \le dis_{i_K}$, do we consider the possibility that $dis_i \le dis_{i_K}$. When $\epsilon_0 \to \infty$, it means that we are extremely risk-averse, always doing comparison with exact distances. We'll show in Section 5 that the failure probability decays super-exponentially with respect to $\epsilon_0$ (Equation~\ref{eq:superexp}). Gathering all discussed components, we get the algorithm ADSampling described in Algorithm~\ref{code:adasampling}. Note that this algorithm can be applied to anywhere DCO is needed, so to emphasize its universality we use $r$ instead of $dis_{i_K}$ to denote the comparison threshold.

% {\JIANYANG 

% We depict the workflow of our ADSampling algorithm in Figure~\ref{fig:ADSampling}. 
% The panels from top to bottom describe four levels from 1 to 4. The empty dot and the dashed curve represent the exact distance $dis_i$ and density of approximate distance $dis_i'$. Note that they only exist conceptually and are unknown by us. The colored dot represents the observed $dis_i'$. It follows the distribution described by the colored dash curve. The black dot and vertical line corresponds to the distance threshold $r$ and $\gamma(d) \cdot r$. The right hand side of $\gamma(d) \cdot r$ is the rejection region.
% Figure~\ref{fig:ADSamplingneg} depicts the case of \underline{$dis_i > r$ (negative)}. At level 1, we observe $dis_i' > r$. However, it's less than $\gamma(d_1) \cdot r$, indicating that it's still possible to be positive. Similarly, at level 2, $dis_i' < \gamma(d_2) \cdot r$ suggests that we should continue sampling. At level 3, it's worth noting that $dis_i > \gamma(d_3) \cdot r$ and more than a half of the density is on the right side of $\gamma(d_3) \cdot r$, indicating that there is at least $\frac{1}{2} $ probability to correctly terminate sampling, though the observed $dis_i'$ might also be less than $\gamma(d_3)\cdot r$. Finally at level 4, we find that $dis_i'$ falls into the rejection region, so we can terminate sampling and claim $dis_i > r$.
% Figure~\ref{fig:ADSamplingpos} corresponds to the case of \underline{$dis_i \le r$ (positive)}. The process is similar. It's worth noting that: (1) it's possible that $dis_i' > r$ (level 2) (2) it's almost impossible that $dis_i' > \gamma(d) \cdot r$ (most of the density is always on the left side of $\gamma(d) \cdot r$). 

% % \begin{comment}
% % In Figure~\ref{fig:ADSampling}, we depict the process of our ADSampling algorithm for a comparison between object $i$ and distance threshold $r$. The four panels from top to bottom describe the status of our algorithm at four different levels. Specifically, in each panel, the solid vertical line represents the distance threshold $r$. The dash vertical line and the colored dash curve represent the exact distance $dis_i$ and the density function of the approximate distance $dis_i'$ respectively. Note that we \textbf{do not} explicitly know them all through the algorithm. The only known information to inference $dis_i$ is the observed $dis_i'$ (dot) at each level. In the beginning, we first level up object $i$ to level 1 and obtain the observed distance $dis_i'$ (red dot). Though $dis'_i$ in this case is larger than $r$, according to our hypothesis testing, only if $dis_i' > r \cdot \gamma(d_1)$ (equivalently $dis_i' / \gamma(d_1) > r$), can we conclude $dis_i > r$. However, $dis_i' / \gamma(d_1)$ (red left-arrow) is smaller than $r$, so there is still unignorable possibility that $dis_i \le r$. Thus, we should continue sampling. Then at level 2, with more sampled dimensions, the observed distance becomes more accurate (closer to $dis_i$). At the same time, since $\gamma(d_2) < \gamma(d_1)$, the error bar (between left-arrow to dot) also shrinks. However, since $dis_i'/\gamma(d_2)$ is still less than $r$, we should keep sampling. Finally, at level 4, we find currently $dis_i'/\gamma(d_4)$ (green left-arrow) to be larger than $r$. We can confidently claim that the exact $dis_i$ is larger than $r$ and terminate sampling. 
% % \end{comment}
% }

%{\JIANYANG Note that the above hypothesis testing terminates sampling only when $dis_i > r$ can be concluded. We can also apply the same technique for the cases where $dis'_i \ll r$. However, we don't suggest to do so because (1) it leads to false positive failure and (2) exact distance is needed for following DCO. Specifically, \underline{first}, concluding $dis_i \le r$ with approximate distance may wrongly replace a KNN object with an non-KNN object, which is unacceptable. \underline{Second}, using approximate distance for following comparison cannot guarantee correctness. Thus, for those objects that cannot be rejected, we keep leveling them up even their observed approximate distance is much less than $r$.}

%$dis_i \le r$ indicates that object $i$ should be inserted into the KNN set $\mathcal K$. Terminating sampling in this case means that we are maintaining $\mathcal K$ with approximate distance and using it for following DCOs. However, directly using approximate distance for following DCOs cannot guarantee correctness, i.e., one distance larger than the approximate distance is not necessarily larger than the corresponding exact distance. Another option is to inference a safe upper bound of exact distance with approximate distance 
%(e.g., $dis_i'/(1-\frac{\epsilon_0}{\sqrt {d_l} })$, right-arrow in Figure~\ref{fig:ADSampling})
%. However, this operation definitely overestimates it (or it is not a safe upper bound)
%as is also illustrated in Figure~\ref{fig:ADSampling}, i.e., right-arrow is on the right hand side of $dis_i$
%. As a result, it increases distance threshold $r$ for \textbf{ALL} the following comparisons and consequently increases the needed dimensions, which is not worthy. 

%In Figure~\ref{fig:ADSampling}, the vertical solid line represents the unknown exact distance and the colored dash curve depicts the corresponding distribution of the observed distance. When running ADSampling, we only know the observed distance (colored dot) and the range of possible true distance $dis_i$ (colored arrow) according its current level. At level 1, with the observed distance and the error bar, we cannot make sure that.

% {\CHENG 14-07-Cheng: As we discussed before, some discussions on cases where the approximate distances are smaller than the threshold by a certain extent should be better provided.}


% \begin{comment}
% {\color{red} I commented out the definition of safe distance because we may drop adaptive priority queue.}

% Equivalent to Equation (\ref{eq:testing}), we have
% \begin{align}
%      \frac{dis'_j}{\gamma(d_{l_j})} > dis_i \label{eq:safe_distance}
% \end{align}
% Note that the left hand side depends only on $j$. We define it to be the \textit{safe distance} $dis^*_j$ of object $j$. Formally, 
% \begin{align}
%     dis^*_j := \frac{dis_j'}{\gamma(d_{l_j})} 
% \end{align}
% It's named safe distance because when $dis^*_j > dis_i$, it induces that $dis'_j > \gamma(d_{l_j}) \cdot dis_i$, so it's safe to claim that $dis_j > dis_i$ based on hypothesis testing. In other word, safe distance $dis^*_i$ is a lower bound of true distance $dis_i$ with high probability. Note that safe distance $dis^*_j$ and observed distance $dis'_j$ can be easily converted to each other by multiplying or dividing by $\gamma(d_{l_j})$, so for conciseness, we skip the level up operation of safe distance.

% \end{comment}

%The next question is about the timing of hypothesis testing. The ideally optimal strategy is to do hypothesis tesing every time after sampling one new dimension: once we have enough information, we stop immediately. However, frequent hypothesis testing, on the one hand, increases the overall failure probability, and on the other hand, introduces testing overhead. {\color{red} Thus, in practice, to anchor hypothesis testing, we preset a few discrete dimensionalities which we term "levels" in the following sections.} Formally, assuming that we have $L+1$ levels, each level $i$ corresponds to a dimensionality $d_i$ where $0 = d_0 < d_1 <... <d_L = D$. Then, for an object $i$, we grant it with two states: its current level $l_i$ and the corresponding $d_{l_i}$-dimensional observed distance $dis_i'$. By "level up" operation, we evaluate the dimensions in $(d_{l_i}, d_{l_i+1}]$, modify the observed distance and increase the level by $1$. With hypothesis testing, leveling up stops when we have enough confidence. Thus, the dimensionality automatically adapts to its inherent difficulty. {\color{red} extremely simple, need pseudo code?}

%The next question is about the timing of hypothesis testing. The ideally optimal strategy is to do hypothesis tesing every time after sampling one new dimension: once we have enough information, we stop immediately. However, frequent hypothesis testing, on the one hand, increases the overall failure probability, and on the other hand, introduces testing overhead. Thus, in practice we preset a few discrete levels to anchor hypothesis testing. Formally, assuming that we have $L+1$ levels, each level $i$ corresponds to a dimensionality $d_i$ where $0 = d_0 < d_1 <... <d_L = D$. Then, for an object $i$, we grant it with two states: its current level $l_i$ and the corresponding $d_{l_i}$-dimensional observed distance $dis_i'$. By "level up" operation, we evaluate the dimensions in $(d_{l_i}, d_{l_i+1}]$, modify the observed distance and increase the level by $1$. With hypothesis testing, leveling up stops when we have enough confidence. Thus, the dimensionality automatically adapts to its inherent difficulty. {\color{red} extremely simple, need pseudo code?}

% \subsection{\textbf{Random Orthogonal Transformation}}

% Finally we come to the problem of dealing with extremely fragile comparisons.
% %According to the study in Section 3, some queries are extremely fragile to random projection. Thus, it would be of vital importance to recover full accuracy. 
% In high-dimensional space, some DCOs are extremely fragile to approximation, i.e. the ratio between distances are very close to 1, which disables dimension reduction. A naive idea is to re-evaluate distance with raw vectors from scratch.
% %However, it might double the time consumption. 
% However, re-evaluation leads to twice cost, i.e. $D$ dimensions on transformed vector + $D$ dimensions on raw vector. 
% %We utilize random orthogonal transformation here to resolve this issue.
% Random orthogonal transformation resolves this issue naturally. To be specific, random orthogonal transformation exhibits both stochastic and deterministic properties. Like other types of random projections (e.g. Gaussian), random orthogonal projection demonstrates concentration properties that approximately preserves distance in $d$-dimensional subspace. However, unlike other types of projections, it fully preserves distance when all $D$ dimensions are evaluated. Thus, it naturally recovers exact distance when all $D$ dimensions are sampled. That's exactly the reason we adopt random orthogonal transformation instead of others.

% In summary, our adaptive dimension sampling makes resolution flexible by doing dimension sampling on randomly transformed vectors, determines the needed dimensionality by progressive sequential hypothesis testing and recovers exact distance naturally with the distance-preserving property of random orthogonal transformation. 



\subsection{Summary and Theoretical Analysis}
\label{subsection:theoretical analysis of ADSampling}

{\CHENG \noindent\textbf{Summary.} 
% \input{content/pseudo_ADSampling_original.tex}
We summarize the process of \texttt{ADSampling} in Algorithm~\ref{code:adasampling}.
{\JIANYANGREVISION It takes a transformed data vector $\mathbf{o}'$, a transformed query vector $\mathbf {q}'$ and a distance threshold $r$ as inputs and outputs the result of the DCO of whether $dis \le r$: 1 for yes {\CHENGC (in this case, it returns $dis$ as well)} and 0 for no. 
We note that the transformation of the data vectors is conducted in the index phase and its cost can be amortized by all the subsequent queries on the same database. The transformation of the query vector is conducted in the query phase when a query comes and its cost can be amortized by all the DCOs involved for answering the same query.} 
Specifically, the algorithm maintains the number of sampled dimensions with a variable $d$ with $d = 0$ initially (line 1). 
It then performs an iterative process if $d < D$ (line 2). 
At each iteration, it samples some more dimensions incrementally and updates $d$ {\JIANYANGB and the approximate distance $dis'$} accordingly (line 3-4)
and conducts a hypothesis testing with the null hypothesis as $dis \le r$ based on the approximate distance $dis'$ (line 5). 
It then returns the result in three cases as explained in Section~\ref{subsec:idea-2} (line 6 - 11). 

\begin{algorithm}[tbh]
\DontPrintSemicolon
\SetKwFunction{LevelUp}{LevelUp}
\SetKwData{and}{and}
\SetKwInOut{Input}{Input}\SetKwInOut{Output}{Output}
\Input{{\JIANYANGREVISION A transformed data vector $\mathbf{o}'$, a transformed query vector $\mathbf q'$ and a distance threshold $r$}}
\Output{The result of DCO (i.e., whether $dis \le r$): 1 means yes and 0 means no; {\CHENG In case of the result of 1, an {\CHENGB exact} distance is also returned}}
\BlankLine
% \tcc{ 1. $\mathbf o' \leftarrow P' \mathbf o$ in the index phase.
    % \\2. $\mathbf q' \leftarrow P' \mathbf q$ at the start of the query phase.}
{\CHENGB Initialize the number of sampled dimensions $d$ to be 0\;}
\While{$d < D$}{
        {\JIANYANGREVISION Sample some more dimensions $y_i$ {\chengr incrementally with} $y_i=\mathbf{o}_i'-\mathbf{q}_i'$\;}
        Update $d$ and the approximate distance $dis'$ accordingly\;
	Conduct a hypothesis testing with the null hypothesis {\JIANYANGB $H_0$} as $dis \le r$ based on the approximate distance $dis'$\;
	\If(\tcp*[f]{Case 1}){{\JIANYANGB $H_0$} is rejected and $d < D$}{
			\textbf{return} 0  \;
		}
	\ElseIf(\tcp*[f]{Case 2}){{\JIANYANGB $H_0$} is not rejected and $d < D$}{
		\textbf{continue}\;
		}
	\Else(\tcp*[f]{Case 3}){
	    \textbf{return} 1 {\CHENG (and $dis'$)} if $dis' \le r$ and 0 otherwise\;
	}
}
\caption{\texttt{ADSampling}}
\label{code:adasampling}
\end{algorithm}

\vspace{-4mm}
\smallskip\noindent\textbf{Failure Probability Analysis.}
%
Note that \texttt{ADSampling} terminates in either Case 1 (with the hypothesis being rejected and $d < D$) or Case 3 (with $d = D$). When it terminates in Case 3, there would be no failure since in this case, the approximate distance $dis'$ is equal to the true distance $dis$ and the DCO result is exact. When it terminates in Case 1, a failure would happen if $dis \le r$ holds since in this case, it concludes that $dis > r$ (by rejecting the null hypothesis). We analyze the probability of the failure. {\JIANYANG As discussed in Section~\ref{subsec:idea-2}, we can control the failure probability with $\epsilon_0$. The following lemma presents the relationship between $\epsilon_0$ and the failure probability of a DCO with \texttt{ADSampling}. } 
}

\begin{lemma}
For a DCO in $D$-dimensional space, the failure probability of \texttt{ADSampling} is given by
\begin{align}
    &\mathbb{P}\left\{ {\CHENG failure}  \right\}  =0 {\CHENG \text{~~if $dis > r$}}
    % {\CHENG \text{~~for a negative object}}
    \\&
    \mathbb{P}\left\{ {\CHENG failure} \right\}  \le \exp \left( -c_0 \epsilon_0^2 + \log D \right) 
    {\CHENG \text{~~if $dis \le r$}}
    % {\CHENG \text{~~for a positive object}}
\end{align}
\label{theorem:ADSampling accuracy}
\label{lemma:ADSampling accuracy}
\end{lemma}
\begin{proof}
% The case of false positive is as specified above. In terms of the false negative, with our discussion above, 
{\CHENG The correctness for the case of $dis > r$ is obvious and that of the other case ($dis \le r$) can be verified as follows.}
\begin{align}
    &\mathbb{P}\left\{ {\CHENG failure} \right\} = \mathbb{P} \left\{ \exists d < D, dis' > (1 + \epsilon_0 / \sqrt {d} ) \cdot r \right\} \label{eq:rejection}
    % \\\le &\mathbb{P} \left\{ \exists d < D, dis' > (1 + \epsilon_0 / \sqrt {d} ) \cdot dis \right\} \label{eq:dis<=r}
    \\\le &\sum_{d=1}^{D-1} \mathbb{P} \left\{ dis' > (1 + \epsilon_0 / \sqrt {d} ) \cdot dis \right\}  \label{eq:union bound}
    \\\le &\sum_{d=1}^{D-1} \exp \left( -c_0 \epsilon_0^{2}  \right)\le \exp \left( -c_0 \epsilon_0^2 + \log D \right) \label{eq:concentrationlemma3.1}
    % \\\le &\sum_{d=1}^{D-1} \exp \left( -c_0 \epsilon_0^{2}  \right)= \exp \left( -c_0 \epsilon_0^2 + \log D \right) \label{eq:concentrationlemma3.1}
    % \\&= \mathbb{P} \left\{ \exists d < D, \sqrt {\frac{D}{d} } \| \mathbf{y}|_{[1,2,...,d]}\|  > (1 + \frac{\epsilon_0}{\sqrt {d} } ) \cdot \|\mathbf{y}\|  \right\}
    % \\&\le \sum_{d=1}^{D-1} \mathbb{P} \left\{  \sqrt {\frac{D}{d} } \| \mathbf{y}|_{[1,2,...,d]}\|  > (1 + \frac{\epsilon_0}{\sqrt {d} } ) \cdot \|\mathbf{y}\| \right\}  
    % \\&= \sum_{d=1}^{D-1} \mathbb{P} \left\{ \sqrt {\frac{D}{d} } \| P|_{[1,2,...,d]} \mathbf{x} \|  > (1 + \frac{\epsilon_0}{\sqrt {d} } ) \cdot \|\mathbf{x}\|  \right\}
    % \\&\le \sum_{d=1}^{D-1} \exp \left( -c_0 \epsilon_0^{2}  \right) \le D \cdot \exp \left( -c_0 \epsilon_0^{2}  \right)
    % \\&= \exp \left( -c_0 \epsilon_0^{2} + \log D \right) 
\end{align}
{\JIANYANG 
where (\ref{eq:rejection}) is because a failure happens if and only if we accidentally reject the hypothesis for some $d<D$; 
% (\ref{eq:dis<=r}) is because $dis \le r$; (\ref{eq:union bound}) applies union bound; 
(\ref{eq:union bound}) applies union bound and the fact that $dis \le r$; and (\ref{eq:concentrationlemma3.1}) is due to Lemma~\ref{eq:concen}.
}
\end{proof}


% {\CHENG 27-07-Cheng: The first equation in the above does not look clear to me and needs some explanation.}

% {\JIANYANG 27-07-Jianyang: It corresponds to the sentence "In terms of false negative, it happens if and only if at some dimension $d < D$, the approximate distance of a positive object ($dis_i \le r$) accidentally falls into the rejection region."}

% {\CHENG 27-07-Cheng: I still don't understand. I believe the event that ``$\exists d < D, dis' > \gamma(d) \cdot r$'' happens is a necessary but not sufficient condition, and thus the ``equation'' should be ``less than'' in deduction. Perhaps, I got some misunderstanding here. If necessary, we can arrange a chat soon to discuss this issue and some other issues.}

{\CHENG

\smallskip\noindent\textbf{Time Complexity Analysis.}
Let $\hat{D}$ be the number of sampled dimensions by \texttt{ADSampling}. Clearly, the time complexity \texttt{ADSampling} is $O(\hat{D})$. Given the stochastic nature of the method, $\hat{D}$ is a random variable. Next, we analyze the expectation of $\hat{D}$, denoted by $\mathbb{E}[\hat{D}]$. 
% We discuss in two cases. \underline{Case 1:} $dis \le r$. In this case, \texttt{ADSampling} would most likely terminate in Case 3 by sampling all $D$ dimensions and very unlikely terminate in Case 1 by concluding $dis > r$ (which corresponds to a failure). Therefore, we have $\mathbb{E}[\hat{D}] \le D$. \underline{Case 2:} $dis > r$. In this case, 
First of all, since $\hat{D}$ is always at most $D$, we know $\mathbb{E}[\hat{D}] \le D$. Furthermore, for the DCO on a negative object with $dis > r$, we can derive that $\mathbb{E}[\hat{D}]$ relies on $\epsilon_0$ and $\alpha = (dis-r)/r$ {\JIANYANG (which we call the \emph{distance gap} between $dis$ and $r$)}, as presented below (detailed proof can be found in 
% Section~\ref{section:theory}
{\JIANYANGREVISION Appendix~\ref{section:theory}).}
% {\JIANYANGREVISION a technical report~\cite{technical_report}}).
%
\begin{lemma}
% For a DCO with threshold $r$ and a negative object, let $(1+\alpha)$ be the ratio between $dis$ and $r$, $\alpha > 0$. The expected terminate dimensionality is 
When \texttt{ADSampling} is used for the DCO on an object and a threshold $r$ with $dis > r$, letting $\alpha =(dis-r)/r$, we have
\begin{align}
    \mathbb{E} \left[ \hat D  \right]  = O \left[ \min \left( D, \alpha _{}^{-2} \cdot \epsilon _{0}^{2}  \right)  \right] 
\end{align}
\label{theorem:ADSampling efficiency}
\end{lemma}
%
The above result is well aligned with the intuitions that (1) when the distance gap between $dis$ and $r$, i.e., $(dis-r)/r$, is larger, fewer dimensions would be sampled for making a sufficiently confident conclusion and (2) when $\epsilon_0$ is larger (i.e., the significance value of the hypothesis testings is smaller, which means a higher requirement on the confidence), more dimensions would be sampled.

% Based on Lemma~\ref{theorem:ADSampling accuracy} and Lemma \ref{theorem:ADSampling efficiency}, 
We further derive the time-accuracy trade-off of \texttt{ADSampling}.
% for DCOs.
% on a negative object.
% and let the false negative failure probability be $\delta$. Then we have the following theorem:
\begin{theorem}
% Let $\delta$ be an upper bound of the false negative failure probability of ADSampling. 
% For a DCO with threshold $r$ and a negative object. Let $(1+\alpha)$ be the ratio between $dis$ and $r$, $\alpha >0$. The expected terminate dimensionality is
When \texttt{ADSampling} is used for the DCO on an object and a threshold $r$ with $dis > r$, letting $\alpha =(dis-r)/r$, we have
\begin{align}
    \mathbb{E} \left[ \hat D \right]  = O \left[ \min \left( D, \frac{1}{\alpha ^2} \log \frac{D}{\delta}  \right)  \right] 
\end{align}
for achieving its failure probability {\JIANYANG (of positive objects)} at most $\delta$.
% \footnote{We note that since random orthogonal projection exactly preserves the distance between any two points, when all $D$ dimensions are sampled }.
\label{theorem:time-accuracy of ADSampling}
\end{theorem}
}

{\JIANYANG 
\begin{proof}
Making the failure probability in Lemma~\ref{theorem:ADSampling accuracy} be equal to $\delta$, we obtain the corresponding $\epsilon_0$. Then by substituting $\epsilon_0$ in Lemma~\ref{theorem:ADSampling efficiency}, we have the theorem. 
\end{proof}
% We next quantitatively investigate the time-accuracy tradeoff of ADSampling, i.e., the failure probability and expected time complexity. 

% First, the failure of a DCO can be categorized into two types: \emph{false positive} (an object is negative $dis_i > r$, but we wrongly claim that it's positive $dis_i \le r$) and \emph{false negative} (an object is positive $dis_i \le r$, but we wrongly claim that it's negative $dis_i > r$).
% Let's consider during ADSampling in which cases these two types of failure can happen. Recall that during ADSampling, when we sampled \underline{$d<D$ dimensions} of a data vector, based on the observed approximate distance $dis_i'$, there are two circumstances: (1) we terminate sampling and claim $dis_i > r$ (return a negative result) when $dis_i' > \gamma(d) \cdot r$, i.e., falls into the rejection region and (2) we cannot reject it and keep sampling when $dis_i' \le \gamma(d) \cdot r$. When we sampled \underline{all $D$ dimensions}, we obtained exact distance and can return the exact comparison results. Note that our algorithm returns a positive result only when it sampled all $D$ dimensions and obtained exact distance. Thus, it will never produce false positive failure. In terms of false negative, it happens if and only if at some dimension $d < D$, the approximate distance of a positive object ($dis_i \le r$) accidentally falls into the rejection region. Based on these thoughts, we have the following theorem: 
% \begin{theorem}
% For a DCO in $D$-dimensional space, the failure probability of ADSampling is given by
% \begin{align}
%     &\mathbb{P}\left\{ {false\ positive}  \right\}  =0 
%     \\&\mathbb{P}\left\{ {false\ negative} \right\}  \le \exp \left( -c_0 \epsilon_0^2 + \log D \right) 
% \end{align}
% \label{theorem:ADSampling dimensionality}
% \end{theorem}
% \begin{proof}
% The case of false positive is as specified above. In terms of the false negative, with our discussion above, 
% \begin{align}
%     &\mathbb{P}\left\{ {false\ negative} \right\} = \mathbb{P} \left\{ \exists d < D, dis' > \gamma(d) \cdot r \right\} 
%     \\\le &\mathbb{P} \left\{ \exists d < D, dis' > \gamma(d) \cdot dis \right\}
%     \\\le &\sum_{d=1}^{D-1} \mathbb{P} \left\{ dis' > \gamma(d) \cdot dis \right\}  \le \sum_{d=1}^{D-1} \exp \left( -c_0 \epsilon_0^{2}  \right) 
%     \\\le &D\exp \left( -c_0 \epsilon_0^{2}  \right)= \exp \left( -c_0 \epsilon_0^2 + \log D \right) 
%     % \\&= \mathbb{P} \left\{ \exists d < D, \sqrt {\frac{D}{d} } \| \mathbf{y}|_{[1,2,...,d]}\|  > (1 + \frac{\epsilon_0}{\sqrt {d} } ) \cdot \|\mathbf{y}\|  \right\}
%     % \\&\le \sum_{d=1}^{D-1} \mathbb{P} \left\{  \sqrt {\frac{D}{d} } \| \mathbf{y}|_{[1,2,...,d]}\|  > (1 + \frac{\epsilon_0}{\sqrt {d} } ) \cdot \|\mathbf{y}\| \right\}  
%     % \\&= \sum_{d=1}^{D-1} \mathbb{P} \left\{ \sqrt {\frac{D}{d} } \| P|_{[1,2,...,d]} \mathbf{x} \|  > (1 + \frac{\epsilon_0}{\sqrt {d} } ) \cdot \|\mathbf{x}\|  \right\}
%     % \\&\le \sum_{d=1}^{D-1} \exp \left( -c_0 \epsilon_0^{2}  \right) \le D \cdot \exp \left( -c_0 \epsilon_0^{2}  \right)
%     % \\&= \exp \left( -c_0 \epsilon_0^{2} + \log D \right) 
% \end{align}
% \end{proof}
%Note that the above hypothesis testing terminates sampling only when $dis_i > r$ can be concluded. We can also apply the same technique for the cases where $dis'_i \ll r$. However, we don't suggest to do so because (1) it leads to false positive failure and (2) exact distance is needed for following DCO. Specifically, \underline{first}, concluding $dis_i \le r$ with approximate distance may wrongly replace a KNN object with an non-KNN object, which is unacceptable. \underline{Second}, using approximate distance for following comparison cannot guarantee correctness. Thus, for those objects that cannot be rejected, we keep leveling them up even their observed approximate distance is much less than $r$.
% Then we analyze the time complexity of ADSampling. Note that in ADSampling, the terminate dimension depends on the random $dis_i'$. Thus, the terminate dimension is also a random variable, which we denoted as $\hat D$. In terms of the time complexity of ADSampling, we analyzes the expected terminate dimension $\mathbb{E} \left[ \hat D \right]$. It's also worth noting that since ADSampling always requires exact DCO for positive objects, it brings acceleration only for negative objects. 
% For a negative object, an obvious fact about $\mathbb{E} \left[ \hat D \right]  $ is that the more different $dis$ and $r$ are, the fewer dimensions are needed, where the difference can be measured by their ratio, i.e., $dis / r$. Similar to the failure probability analysis, $\epsilon_0$ also controls time complexity because intuitively, to achieve better accuracy, we need to evaluate more dimensions. Quantitatively, the following lemma provides the relationship between $\mathbb{E}\left[ \hat D \right]  $ and $\alpha, \epsilon_0$, whose detailed proof is given in Section~\ref{section:theory}: 
%Then we analyze the expected time complexity, i.e., the expected terminate dimensionality of negative objects. 
% \begin{lemma}
% For a DCO with threshold $r$ and a negative object, let $(1+\alpha)$ be the ratio between $dis$ and $r$, $\alpha > 0$. The expected terminate dimensionality is 
% \begin{align}
%     \mathbb{E} \left[ \hat D  \right]  = O \left[ \min \left( D, \alpha _{}^{-2} \cdot \epsilon _{0}^{2}  \right)  \right] 
% \end{align}
% \label{theorem:ADSampling efficiency}
% \end{lemma}
% Note that due to the recoverability of our method, when the comparison is extremely fragile ($\alpha$ is extremely small), ADSampling guarantees to produce exact DCO results with $D$ dimensions, which makes sure that it's at least no worse than the plain comparison. 
% {\JIANYANGB
% \smallskip
% \noindent\textbf{Remarks.}
% We compare Theorem~\ref{theorem:time-accuracy of ADSampling} with Johnson-Lindenstrauss Lemma (JL Lemma)~\cite{johnson1984extensions}, which states that a random projection to the dimensionality $\Theta( \frac{1}{\epsilon^2} \log \frac{1}{\delta} )$ can preserve the norm of a vector with $\epsilon$ multiplicative error with at most $\delta$ failure probability. We note that our method has very similar result as JL Lemma despite that (1) applying JL Lemma needs to set the dimensionality according to the needed multiplicative error bound while ours automatically adapts to the needed dimensionality according to the distance gap $\alpha$ and (2) Theorem~\ref{theorem:time-accuracy of ADSampling} is weaker by a factor of $\log D$. This is because Lemma~\ref{lemma:ADSampling accuracy} has not been tightly analyzed yet. Since the tightness of the theoretical analysis is not the focus of this paper, we leave the careful analysis in future works. 

% % We also emphasize that a series of studies~\cite{larsen2017optimality,optimalJL, optimalJL2} prove the theoreticaly optimality of JL Lemma, i.e., there exists a set of size $N$, if the dimensionality is reduced to $o(\frac{1}{\epsilon^2} \log \frac{N}{\delta})$, then its distance information would be largely distorted. Note that for a DCO, its maximum allowed multiplicative error bound should be the distance gap $\alpha$. These two points imply that our method achieves the theoretically instance-optimality.
% }
% (1) our method has very simi (1) applying Johnson-Lindenstrauss Lemma requires to 
% with failure probability at most $\delta$, to guarantee a multiplicative error of at most $\epsilon$ for a single object, .

\smallskip
\noindent\textbf{\texttt{ADSampling v.s. FDScanning.}}
Compared with \texttt{FDScanning}, \texttt{ADSampling} improves the complexity for negative objects from being linear to being logarithmic wrt $D$ at the cost of the accuracy for positive objects (Theorem~\ref{theorem:time-accuracy of ADSampling}).
%
% According to Lemma~\ref{theorem:ADSampling accuracy} and \ref{theorem:ADSampling efficiency}, we notice that 
We emphasize that the failure probability (of positive objects) decays \textbf{quadratic-exponentially} (Lemma~\ref{theorem:ADSampling accuracy}) while the time complexity (of negative objects) grows \textbf{quadratically} (Lemma~\ref{theorem:ADSampling efficiency}), both with respect to $\epsilon_0$. It indicates that to achieve \emph{nearly-exact} DCOs, we only need sample a few dimensions.  
{\JIANYANG 
We empirically verify these results in Section~\ref{subsubsec:theoretical results}. It shows that with \texttt{ADSampling} as a plugin, an exact KNN algorithm, namely linear scan, needs only on average 55 dimensions per vector on GIST (originally 960 dimensions) to achieve >99.9\% recall. }

% To better illustrate the time-accuracy tradeoff, we combine Lemma~\ref{theorem:ADSampling accuracy} and \ref{theorem:ADSampling efficiency} and let the false negative failure probability be $\delta$. Then we have the following theorem:
% \begin{theorem}
% Let $\delta$ be an upper bound of the false negative failure probability of ADSampling. For a DCO with threshold $r$ and a negative object. Let $(1+\alpha)$ be the ratio between $dis$ and $r$, $\alpha >0$. The expected terminate dimensionality is
% \begin{align}
%     \mathbb{E} \left[ \hat D \right]  = O \left[ \min \left( D, \frac{1}{\alpha ^2} \log \frac{D}{\delta}  \right)  \right] 
% \end{align}
% \label{theorem:time-accuracy of ADSampling}
% \end{theorem}

% For further discussion, note that we can also trade off between the complexity for the positive and the accuracy of the negative by adding another hypothesis testing with $H_0: dis_i > r, H_1: dis_i \le r$. However, under the context of KNN, we don't suggest to do so because (1) exact distance of positive objects is needed for following DCO (2) in KNN query, the comparisons for negative objects are dominant. In terms of the second point, specifically, providing ground truth $dis_{i_K^*}$ as the distance threshold, we have $K$ positive objects (at most hundreds) and $N-K$ negative objects (millions or billions). In our setting, we reduce the time complexity of the majority (all negative objects) and at the same time, prevent them from false positive failure. While in terms of the positive, though they need full $D$ dimensions, there are only $K$ of them. At the same time, though they might cause false negative failure, in our setting we only need to suppress the failure probability (by increasing $\epsilon_0$) of $K$ objects instead of the remaining $N-K$.
} 

%Now we introduce adaptive dimension sampling for a single comparison between two objects. We next specify how it adapts to KNN query.

\begin{comment}
\begin{figure}[htb]
  \centering 
  \includesvg[width=0.5\linewidth]{figure/KNN3.svg}
  \caption{Draft - Adaptive Dimension Sampling for KNN Query.}
  \label{fig:KNNcomp}
\end{figure}

\subsubsection{\textbf{Adaptive Dimension Sampling for KNN Query}}
As is already discussed in Section 3, KNN is a query to distinguish apart positive objects and negative objects. The result of a comparison matters only when a positive object and a negative one are compared. In condition of producing the correct result for it, the total quota of approximation has already been determined by its inherent difficulty, i.e. distance ratio. However, as we can determine the dimensionality of each object individually, the quota of approximation is still to be assigned. In other word, once we increase the resolution of one object, there would be more space of approximation left for the other. At the same time, in KNN query, positive objects are much less than the negative, which implies that increasing the resolution of one positive object can benefits many comparisons with the negative as shown in figure~\ref{fig:KNNcomp}. Thus, we make it exact for all potential positive objects during KNN search. {\color{red} (That's why we consider the case where one is exact and the other is approximate in hypothesis testing.) } 



Then we equip two classes of ANN algorithms, i.e. graph and filter-and-verification framework with our adaptive dimension sampling framework.
\end{comment}



%Random orthogonal transformation exhibits both deterministic and stochastic properties. On the one hand, it fully preserves the distance between any two vectors when all $D$ dimensions are evaluated (as a result, $\gamma(D)=1,dis_i^*=dis_i$ when $d=D$). On the other hand, properties needed in the previous sections also hold: 1)its $d$-dimensional row sampling is equivalent to $d$-dimensional random orthogonal projection, which is shown by the process of its generation. 2) concentration inequality also holds for random orthogonal projection \footnote{Its proof relies on the concentration of Lipschitz function over $D$-dimensional unit sphere $\mathbb{S}^{D-1}$~\cite{vershynin_2018}.}. Thus, by applying random orthogonal transformation instead of Gaussian, we naturally recover full-precision distance when all $D$ dimensions are sampled, which ensures the number of evaluated dimensions to be no greater than $D$. 

%Random orthogonal transformation is a technique widely used in KNN query. {\color{red} cite } It exhibits both deterministic and stochastic properties. First, it fully preserves the distance between any two vectors when all its $D$ dimensions are evaluated. Then, its $d$-row sampling is equivalent to $d$-dimensional random orthogonal projection, for which concentration inequality still holds. {\color{red} (cite: it's a conclusion in a textbook, maybe I'll need to prove the row-sampling thing.)} Thus, by replacing random Gaussian transformation with random orthogonal transformation, we naturally recovers full accuracy when all $D$ dimensions are sampled, which ensures the number of dimensions we evaluate to be strictly no greater than the full-key evaluation. {\color{red} I didn't talk about its generation.}

%In terms of random projection, empirically, Gaussian and orthogonal transformation show comparable performance for low target dimensions. In terms of random Gaussian transformation, the distribution of the observed distance (or its square) can be explicitly given by $\chi^2(d)$ distribution or approximately by Chernoff bound {\color{red} cite}, both of which are tighter than the given subgaussian tail bound, while for random orthogonal transformation, we cannot provide such a concrete distribution as $\chi^2(d)$. Also, we drop some high ordered terms $O(\epsilon ^3)$ for simplicity {\color{red} cite}. {\color{red} However, {\textbf{we argue that }} the high ordered terms have little effect on the performance when $\epsilon $ is small, and also the tightness of the bound is not the focus of our work. Thus, we just initiate the framework with the subgaussian tail bound.}

%In terms of random projection, empirically, Gaussian and orthogonal transformation show comparable performance for low target dimensions. For random Gaussian transformation, the distribution of the observed distance (or its square) can be explicitly given by $\chi^2(d)$ distribution or approximately by Chernoff bound {\color{red} cite}, both of which are tighter than the given subgaussian tail bound, while for random orthogonal transformation, we cannot provide such a concrete distribution as $\chi^2(d)$. {\color{red} However, the tightness of the bound is not the focus of our work. Thus, we just initiate the framework with subgaussian tail bound.}

%the high ordered terms have little effect on the performance when $\epsilon $ is small, and also


% We then specify how ADSampling is applied to KNN query with three proposed algorithms. 

%We emphasize that AdaSampling is a highly coupled framework. Components in it cannot be replaced trivially. For example, PCA, a famous optimization-based dimensionality reduction method, cannot trivially substitute random projection because it fails to provide a probabilistic maximum error bound for hypothesis testing. We also note that each component in this framework was individually used in KNN query. {\color{red} cite: xxx, xxx, \cite{lu2020vhp}} in LSH family samples some sub-codes from a large codebook for each individual query. \cite{lu2020vhp} evaluates distance under different subspaces.  \cite{ram2019revisiting} also shares the same spirit when discussing random partition tree and KD-Tree. {\color{red} (I'll need to carefully reread the paper to see what was talked about.  I'll still need to cite another paper from the same author.)} However, none of these methods make it progressive and design adaptation mechanisms with theoretical guarantees. SRS~\cite{sun2014srs} utilized cumulative density function of $\chi^2(d)$ distribution for early termination, which shares the same spirit of hypothesis testing. However, they only consider the $c$-approximation case. {\color{red} (I should be careful about it. It may need discussion.)} Random orthogonal transformation is used in quantization methods {\color{red} cite} with the heuristic motivation of uniformly distributing distance information. {\color{red} I should be careful.} Thus, we emphasize that our contribution is more about the whole adaptive dimension sampling framework but not each individual component.

%Plus, note that our method targets to produce correct results for all DCOs, which means with high probability the framework itself does not introduce any error. Thus, the al . However, the most common setting of random projection, the one given by LSH family, targets the relaxed version of NN query. exact NN query with high successful probability instead of approximate NN query. As shown in the experiment using Adaptive Priority Queue for linear scan, it reaches full accuracy. Thus, the Adaptive Priority Queue itself doesn't lose accuracy with high confidence. Though we target exact NN case, for approximate NN case, our method is still applicable in the sense that we can further loosen the ratio by dividing a factor of $1+c$, which we don't discuss in detail. 

% \input{content/ADSampling-backup}


{\CHENG
% \section{\texttt{ADSampling} for Improving AKNN Algorithms}
% \label{sec:adsampling-aknn}
\section{\texttt{AKNN+}: Improving AKNN Algorithms with \texttt{ADSampling} as a Plug-in Component}
\label{sec:aknn+}



% \subsection{\texttt{ADSampling} for a General AKNN Algorithm: As a Plug-in Component}
% \subsection{\texttt{AKNN+}: AKNN Algorithms with \texttt{ADSampling} as a Plug-in Component}
\label{subsec:adsampling-any-aknn}
Recall that an AKNN algorithm, which we denote by \texttt{AKNN} and could be any one among many existing algorithms~\cite{malkov2018efficient, jegou2010product, muja2014scalable, fu2019fast, datar2004locality}, involves many DCOs. 
% Each operation is to decide whether an object $\mathbf{o}$ has its distance $dis$ from a query $\mathbf{q}$ smaller than a threshold $r$, and if so, returns $dis$. 
% The AKNN algorithm typically conducts each operation by computing $dis$ exactly, which runs in $O(D)$ time. 
In the literature, \texttt{FDScanning} is typically adopted for DCOs and runs in $O(D)$ time.
Given that \texttt{ADSampling} can conduct 
% DCOs with lower time complexities
% than $O(D)$ 
% and high {\JIANYANG success} probabilities, 
% DCOs with better cost-effectiveness,
{\CHENGB reliable DCOs with better efficiency,}
a natural idea is to improve the AKNN algorithms by adopting \texttt{ADSampling} for the DCOs.
% in a \emph{plug-in} manner. 
{\JIANYANG Specifically,} since \texttt{ADSampling} is based on randomly transformed data vectors and query vectors, before any query comes, we randomly transform all data vectors, and when a query comes, we randomly transform the query vector. Then, we run the AKNN algorithm based on the transformed data and query vectors. Recall that the time cost of transforming the data vectors can be amortized across different queries and the time cost of transforming the query vector can be amortized across many different DCOs involved for answering the query. During the running process of the AKNN algorithm, whenever it conducts a DCO, we use the \texttt{ADSampling} method. For example, for graph-based methods such as \texttt{HNSW}, we use the \texttt{ADSampling} method when comparing {\CHENGC the distance of} a newly visited object with the maximum in the result set $\mathcal R$. 
% For quantization-based methods such as 
{\chengf For other AKNN algorithms such as} \texttt{IVF}, we apply \texttt{ADSampling} when {\CHENGB comparing the distance of a candidate and the maximum in the currently maintained KNN set $\mathcal K$} for selecting the final KNNs from the generated candidates.
% we compare it with the maximum in the currently maintained KNN set $\mathcal K$.

For an AKNN algorithm \texttt{AKNN}, which adopts \texttt{ADSampling} for DCOs, we call it \texttt{AKNN+}. For example, we call \texttt{HNSW} and \texttt{IVF} with \texttt{ADSampling} adopted for DCOs \texttt{HNSW+} and \texttt{IVF+}, respectively. 

{\JIANYANG
\smallskip\noindent\textbf{{\CHENGB Theoretical Analysis}.} 
Recall that \texttt{ADSampling} improves the efficiency of DCOs on negative objects at the cost of the accuracy of those on positive objects. 
% Therefore, there exists a tradeoff between the time complexity {\CHENG (of negative objects)} and the failure probability {\CHENG (of positive objects)}. 
% Thus, it is necessary to analyze the them together instead of individually. 
% Recall that as analyzed in Section~\ref{sec:problem setting}, the time cost of an AKNN algorithm is dominated by DCOs, in particular of negative objects. 
% With replacing \texttt{FDScanning} with \texttt{ADSampling}, we reduce the time consumption of DCOs of negative objects while keep that of the remaining computation. Thus, here we focus on analyzing the cost and gains of DCOs. 
% We first investigate when an \texttt{AKNN+} algorithm would fail to return the same results as its corresponding \texttt{AKNN} algorithm does. 
% Note that we only change the DCO, so if all \texttt{ADSampling} produce correct results, the final results would be preserved. 
{\CHENGB We 
% of the time-accuracy tradeoff of \texttt{AKNN+}.
show the relationship between the probability that \texttt{AKNN+} fails to return the same results as \texttt{AKNN} and the time complexity of the DCO on a negative object involved in \texttt{AKNN+} below.} 
{\JIANYANGB Basically, to preserve the returned results of \texttt{AKNN}, it suffices to produce correct results for all DCOs, whose number is at most $N$. Then with union bound, the failure probability of \texttt{AKNN+} is upper {\CHENG bounded} by the sum of the failure probability of each single DCO. Thus, making the failure probability of \texttt{ADSampling} be $\delta = \delta' / N$ yields the following corollary.}
\begin{corollary}
Let $\delta'$ be the probability that \texttt{AKNN+} fails to return the same results as \texttt{AKNN}. The expected time complexity of the DCO on a negative object with distance gap $\alpha$ is reduced to
\begin{align}
    \mathbb{E} \left[ \hat D \right]  = O \left[ \min \left( D, \frac{1}{\alpha ^2} \log \frac{DN}{\delta'}  \right)  \right] 
\end{align}
and the remaining time cost ({\CHENGC for} DCOs on positive objects and other {\CHENGC computations}) is unchanged.
\end{corollary}
% \begin{proof}
% % As analyzed above, i
% In order to preserve the returned results of \texttt{AKNN}, it's sufficient to produce correct results for all DCOs, whose number is at most $N$.
% % \footnote{It's worth noting that as analyzed in Lemma~\ref{lemma:ADSampling accuracy}, the failure probability of negative objects is 0, so it's sufficient to apply union bound on positive objects only, whose number is supposed to be $O(K \log N)$ as analyzed in Section~\ref{sec:problem setting}. Because the time complexity is logarithmically dependent on it, which makes very minor difference, for conciseness, we present the corollary with union bound on $N$ rather than $O(K \log N)$.}. 
% Then with union bound, the failure probability of \texttt{AKNN+} is upper {\CHENG bounded} by the sum of failure probability of each single DCO. Thus, it suffices to make the failure probability of \texttt{ADSampling} be $\delta = \delta' / N$. Then with Theorem~\ref{theorem:time-accuracy of ADSampling}, we have this corollary. 
% \end{proof}

Furthermore, for those \texttt{AKNN+} algorithms which generate candidates all at once (e.g., {\CHENGC{\texttt{IVF}}}), 
% we apply \texttt{ADSampling} when finding out KNNs among the generated candidates. 
producing correct DCO results for KNN objects ($K$ objects) rather than for all objects (at most $N$ objects) is sufficient to return the same results as its corresponding \texttt{AKNN} algorithm. This is because once we produce correct results for {\CHENGB the DCOs on} the true KNN objects, we also obtain their exact distances {\CHENGC (note that all of them would be positive objects)}. It ensures to return them as the final answers. Thus, we have the following corollary.

\begin{corollary}
Let $\delta'$ be the probability that \texttt{AKNN+} fails to return the KNNs of the generated candidates. The expected time complexity of DCO on a negative object with distance gap $\alpha$ is reduced to
\begin{align}
    \mathbb{E} \left[ \hat D \right]  = O \left[ \min \left( D, \frac{1}{\alpha ^2} \log \frac{DK}{\delta'}  \right)  \right] 
\end{align}
and the remaining time cost ({\CHENGC for} DCOs on positive objects and other {\CHENGC computations}) is unchanged.
\label{corollary: find out KNNs}
\end{corollary}

}
% \smallskip\noindent\textbf{Failure Probability Analysis.} We investigate the probability that an \texttt{AKNN+} algorithm would fail to return the same results as \texttt{AKNN} does. We bound this probability with \emph{union bound}. That is, the probability that \texttt{AKNN+} fails is at most the sum of the failure probabilities of the DCOs for positive objects (recall that those for negative objects are equal to 0). Specifically, we have the following result of the failure probability of \texttt{AKNN+}.

% \begin{corollary}
% The probability that \texttt{AKNN+} fails to return the same results as \texttt{AKNN} is at most $\exp \left( -c_0 \epsilon_0^2 + \log (DK\log N) \right)$. %where $N_{pos}$ is the number of DCOs for positive objects.
% \label{theorem: AKNN+ failure probability}
% \end{corollary}
% \begin{proof}
% With the union bound over all $O(K \log N)$ positive objects and Lemma~\ref{theorem:ADSampling accuracy}, we have this corollary.
% \end{proof}
% \smallskip\noindent\textbf{Time Complexity Analysis.} We analyze the time complexity of \texttt{AKNN+}. Its time cost consists of the cost of conducting DCOs with \texttt{ADSampling}, which we denote by $C_1$ and that of the rest computations, which we denote by $C_2$. We first analyze $C_1$. Let $N_s$ be the number of candidates considered by the \texttt{AKNN+} algorithm and we have $N_s \le N$. We know that there would be $N_s$ DCOs. Recall that the time complexity of \texttt{ADsampling} for a DCO for an object with a distance $dis$ and a threshold $r$ is $O(D)$ if the object is a positive one and $O(\min(D, \frac{1}{\alpha^2}\log \frac{D}{\delta}))$ if the object is a negative one, where $\alpha = (dis-r)/r$ and $\delta$ is the failure probability of {\JIANYANG a single} \texttt{ADSampling}. Furthermore, we verify that among the $N_s$ DCOs, $O(K\log N)$ operations are for positive objects and $O(N_s)$ operations are for negative objects (for which the proof is provided in an appendix). We then analyze $C_2$. $C_2$ is different for different \texttt{AKNN+} algorithms. For example, for \texttt{HNSW+}, $C_2$ is $O(N_s \log N_s)$, and for \texttt{IVF+}, $C_2$ is $O(N_s \log K)$. In summary, we provide the time complexity results of \texttt{AKNN+} below.

% \begin{corollary}
% Let $\delta'$ be the probability that \texttt{AKNN+} fails to return the same results as \texttt{AKNN}. The time complexity of \texttt{AKNN+} is $O(C_1 + C_2)$. 
% % The expected time complexity of \texttt{AKNN+} is 
% For $C_1$, its expectation is as follows.
% \begin{align}
%     \mathbb{E} \left[ C_1 \right] = O \left( K \log N \cdot D +  N_s \cdot c_\alpha \log \frac{DK\log N}{\delta'} \right) 
%     % \mathbb{E} \left[ C_1 \right] = O \left( DK \log N + \min\{N_s D, N_s \cdot \bar{\alpha}^{-2} \log \frac{DK\log N}{\delta'}\} \right) 
% \end{align}
% where $c_\alpha  =  \frac{1}{N_{s}} \sum_{i=1}^{N_{s}} \min \left( \alpha_i^{-2}, D / \log \frac{DK \log N}{\delta'}  \right)$ and $\alpha_i$ is the $alpha$ value of the $i^{th}$ candidate object. 
% % where $\bar{alpha}$ is the average of the $\alpha$'s of the negative objects.
% For $C_2$, it is dependent on the specific \texttt{AKNN+} algorithm. 
% For \texttt{HNSW+}, $C_2$ is $O(N_s \log N_s)$, and for \texttt{IVF+}, $C_2$ is $O(N_s \log K)$. 
% %
% In a simplified form, the total time complexity of \texttt{HNSW+} is $O( \log N\cdot D + N_s \cdot c_{\alpha} \log D + N_s \log N_s)$, where we treat $K$ and $\delta'$ as constants.
% % ***In this theorem, we can define $\delta'$ as the failure probability for \texttt{AKNN+}, which should be based on $\delta$.***
% \label{theorem:AKNN+ time-accuracy}
% \end{corollary}
% \begin{proof}
% {\JIANYANG To make the failure probability of \texttt{AKNN+} provided in Theorem~\ref{theorem: AKNN+ failure probability} be $\delta'$, it suffices to make the failure probability of a single \texttt{ADSampling} be no greater than $\delta = \frac{\delta'}{K \log N}$.
% }
% % Let the failure probability of \texttt{AKNN+} provided in Theorem~\ref{theorem: AKNN+ failure probability} be $\delta'$. The failure probability of a single DCO should be no greater than $\delta = \delta' / (K\log N)$. 
% Then with Lemma~\ref{theorem:ADSampling efficiency}, we obtain the complexity for each DCO on a negative object: $\mathbb{E} \left[ \hat D \right] = 
% % \min\left( D, \alpha^{-2} \log (DK\log N)/\delta' \right) $
% O \left( \min\left( D, \alpha^{-2} \log \frac{DK\log N}{\delta'} \right) \right)  $. Gathering the cost of positive objects and negative objects, we prove the theorem. 
% \end{proof}

% \smallskip\noindent\textbf{\texttt{AKNN+} v.s. \texttt{AKNN}.}
% % Recall that the time complexity of \texttt{AKNN} is 
% Recall that \texttt{AKNN} has $C_1$ as $O(N_s D)$, which is normally significantly larger $C_2$.
% %
% \texttt{AKNN+} improves \texttt{AKNN} by reducing the most costly part $C_1$ to $O \left( K \log N \cdot D +  N_s \cdot c_\alpha \log \frac{DK\log N}{\delta'} \right)$ (in expectation). 

% Compared with \texttt{AKNN}, \texttt{AKNN+} has the time complexity of $C_1$ (which is significantly larger than $C_2$) 
% Our algorithm improves each DCO of negative objects from $O(D)$ to $O(c_\alpha \log \frac{DK\log N}{\delta'} )$. 

% ***Better to make some comments on the comparison between \texttt{AKNN}'s time complexity and \texttt{AKNN+}'s and highlight the improvement here.***
}



{\CHENG
\section{\texttt{AKNN++}: Improving \texttt{AKNN+} Algorithms with Algorithm Specific Optimizations}
\label{sec:aknn++}

\subsection{\texttt{HNSW++}: Towards More Approximation}
\label{subsec:hnsw++}
% \subsection{\texttt{ADSampling} for HNSW: Towards More Approximation}
% \label{subsec:adsampling-hnsw}

Recall that \texttt{HNSW+} maintains a result set $\mathcal{R}$ with a max-heap of size $N_{ef}$ and distances as keys, where $N_{ef} > K$. For each newly generated candidate object, it checks whether its distance is 
% at most 
{\JIANYANGLAST no greater than}
the largest distance (of an object) in $\mathcal{R}$ and if so, it inserts the object in the set $\mathcal{R}$. Specifically, it uses \texttt{ADSampling} to conduct the DCO for each candidate object with the largest distance in $\mathcal{R}$ as the threshold distance. We identify two roles played by the set $\mathcal{R}$. 
\underline{First}, it maintains the KNNs with the smallest distances among those candidates generated so far. These KNNs would be returned as the outputs of the algorithm at the end. 
\underline{Second}, it maintains the $N_{ef}^{th}$ largest distance among the candidates generated so far. This distance is used as the threshold distance of the DCOs through the course of the algorithm, whose results would affect how the candidates are generated. 
% (e.g., if a neighbor of a candidate in the graph involved in \texttt{HNSW+} has its distance at most the $N_{ef}^{th}$ distance in $\mathcal{R}$, it would be generated as a candidate). 
{\JIANYANG Specifically, if a candidate {\CHENGC generated by} \texttt{HNSW+} has its distance at most the $N_{ef}^{th}$ distance in $\mathcal{R}$, it would be {\CHENGB added to} the search set $\mathcal S$ for further candidate generation.}

This dual-role design is attributed to the fact that in \texttt{HNSW+}, \emph{exact} distances are used for fulfilling both roles. 
{\JIANYANG As shown in 
% Figure~\ref{fig:illustration HNSW+ v.s. HNSW++}(a)
Figure~\ref{fig:illustration HNSW+}, \texttt{HNSW+} always maintains $\mathcal{R} $ and $\mathcal{S} $ with exact distances (dark green), and the first $K$ objects in $\mathcal{R}$ are the KNN objects.}
Using the exact distances is desirable for the first role (of maintaining the KNNs) 
since the outputs of the algorithms are defined based on the exact distances. Yet we argue that it may not be cost-effective for the second role (of maintaining the $N_{ef}^{th}$ largest distance) since the procedure that uses this distance for generating candidates is a heuristic one (i.e., greedy beam search) and may still work well with an approximate distance. 

Therefore, we propose to decouple the two roles of $\mathcal{R}$ by maintaining two sets $\mathcal{R}_1$ and $\mathcal{R}_2$, one for each role {\JIANYANG (as illustrated in 
% Figure~\ref{fig:illustration HNSW+ v.s. HNSW++}(b)
Figure~\ref{fig:illustration HNSW++})}. 
%Set $\mathcal{R}_1$ serves the first role and set $\mathcal{R}_2$ serves the second one. 
Set $\mathcal{R}_1$ has a size of $K$ and is based on exact distances {\JIANYANG (dark green)}. Set $\mathcal{R}_2$ has its size of $N_{ef}$ and is based on distances, {\CHENGC which could be either exact or approximate}. Specifically, for each newly generated candidate, it checks whether its distance is 
% at most
{\JIANYANGLAST no greater than}
the maximum distance in set $\mathcal{R}_1$, and if so, it inserts the candidate in set $\mathcal{R}_1$. 
{\JIANYANG Furthermore, this DCO produces a by-product, namely the observed distance $dis'$ (light green) when {\CHENG \texttt{ADSampling} terminates}, which could be exact (if all $D$ dimensions are sampled) or approximate (if it terminates with $d<D$). }
% Furthermore, a by-product of this DCO is a distance of the candidate, which is exact if \texttt{ADSampling} for the operation samples all $D$ dimensions or approximate if it samples less than $D$ dimensions.
It then maintains the set $\mathcal{R}_2$ and the set $\mathcal{S}$ based on the 
% computed 
{\JIANYANG observed} distances similarly as \texttt{HNSW+} maintains $\mathcal{R}$ and $\mathcal{S}$, respectively. We call the resulting algorithm that is based on this decoupled-role design \texttt{HNSW++}. 

\begin{figure}[thb]
    \centering
    \vspace{-4mm}
    \captionsetup[subfigure]{aboveskip=-1pt}
   %  \begin{subfigure}[b]{0.32\linewidth}
   %      \includegraphics[width=\textwidth]{revision experimental result/IVFPQ.pdf}
   %      \caption{\texttt{IVFPQ}}
	  % \label{fig:cost IVFPQ}
   %  \end{subfigure} 
    \begin{subfigure}[b]{0.45\linewidth}
        \centering
        \includegraphics[width=0.8\textwidth]{figure/illu_HNSW+.pdf}
        \caption{\texttt{HNSW+}}
        \label{fig:illustration HNSW+}
    \end{subfigure}   
    \begin{subfigure}[b]{0.45\linewidth}
        \centering
        \includegraphics[width=0.8\textwidth]{figure/illu_HNSW++.pdf}
        \caption{\texttt{HNSW++}}
        \label{fig:illustration HNSW++}
    \end{subfigure}   
 %    \subfigure[\texttt{HNSW+}]{
 %        \includegraphics[width=0.45\linewidth]{figure/[illu]HNSW+.pdf}
 %    \label{fig:illustration HNSW+}
 %    }
 %    \subfigure[\texttt{HNSW++}]{
	% \includegraphics[width=0.45\linewidth]{figure/[illu]HNSW++.pdf}
	% \label{fig:illustration HNSW++}
 %    }
    \vspace{-4mm}
    \caption{\texttt{HNSW+} v.s. \texttt{HNSW++}}
    \vspace{-4mm}
    \label{fig:illustration HNSW+ v.s. HNSW++}
\end{figure}
% \smallskip\noindent\textbf{Failure Probability Analysis.} We note that different from \texttt{HNSW+}, which would return the same results as \texttt{HNSW} with high probabilities, \texttt{HNSW++} does not aim to return the same results as \texttt{HNSW} {\JIANYANG (though in practice, it returns nearly the same results). 
% \smallskip\noindent\textbf{Time-Accuracy Tradeoff.} 
\smallskip\noindent\textbf{{\CHENGB Theoretical Analysis}.}
We note that different from \texttt{HNSW+}, which would return the same results as \texttt{HNSW} with high probability, \texttt{HNSW++} does not aim to return the same results as \texttt{HNSW} {\JIANYANG (though in practice, it returns nearly the same results as verified in Section~\ref{subsub:dimensions and recall}). 
% Therefore, \texttt{HNSW++} does not provide the guarantee as Theorem~\ref{theorem: AKNN+ failure probability}. However, since it maintains KNN set $\mathcal R_1$ with the \emph{nearly-exact} DCO algorithm \texttt{ADSampling}, it guarantees to return the KNNs of generated candidates. 
% We note that in Theorem~\ref{theorem: AKNN+ failure probability} when aiming at preserving the results, the algorithm is supposed to produce correct results for all DCOs. Thus, the overall failure probability should be derived from the union bound over all objects with $dis\le r$ (because the failure probability of $dis > r$ is $0$), whose number is $O(K\log N)$. However, in order to return the KNNs of generated candidates, producing correct results for the KNNs is sufficient because in this case, we obtain exact distance of the KNNs, which would be sufficient to keep them in the KNN set $\mathcal R_1$. Thus, applying union bound over these $K$ candidates, we have the following theorem.
{\CHENG Specifically, \texttt{HNSW++} would generate a set of candidates, which might be different from that of \texttt{HNSW+} or \texttt{HNSW}. 
{
% \JIANYANG However, it's worth noting that \texttt{HNSW++} still maintains KNNs with \texttt{ADSampling}, which guarantees the high probability of finding out KNNs among generated candidates (though the candidates can be different from those of \texttt{HNSW}). Then under this type of guarantee, its time-accuracy tradeoff is provided in Corollary~\ref{corollary: find out KNNs}.
Among the generated candidates, \texttt{HNSW++} guarantees to return their KNNs with high probability because it still maintains KNNs with \texttt{ADSampling}, {\CHENG and its guarantee is the same as the one in Corollary~\ref{corollary: find out KNNs}}. 
% Then under such type of guarantee, i.e., returning KNNs of generated candidates (though the candidates can be different from those of \texttt{HNSW} and \texttt{HNSW+}), its time-accuracy tradeoff is provided in Corollary~\ref{corollary: find out KNNs}.
}
}
% Among the generated candidates, \texttt{HNSW++} would return the KNNs with a high probability as shown in the following theorem.
}

% \begin{corollary}
% The probability that \texttt{HNSW++} fails to return the KNNs of generated candidates is at most $\exp \left( -c_0 \epsilon_0^2 + \log (DK) \right)$. %where $N_{pos}$ is the number of DCOs for positive objects.
% \label{theorem: HNSW++ failure probability}
% \end{corollary}
% \begin{proof}
% Different from the analysis of Corollary~\ref{theorem: AKNN+ failure probability}, to return KNNs of generated candidates only requires the success of the DCOs of KNN objects rather than all positive objects. This is because with the success of KNN objects, we can obtain their exact distance, which makes sure we return them as final answers. As a result, in this case, we apply union bound of all KNN objects instead of all $O(K \log N)$ positive objects.
% \end{proof}
% }
% %Therefore, we would not define a failure probability for \texttt{HNSW+}.

% % {\JIANYANG
% %In terms of theoretical guarantee, \texttt{HNSW++} does not guarantee to preserve the full process of greedy beam search. 

% %However, since it does KNN checking with ADSampling for all candidates, it provides the guarantee of returning KNNs of generated candidates.
% % %another type of guarantee , i.e., guarantee to return KNNs of the generated candidates. 
% % Under the context of such type of failure, we provide the following theorem to show the time-accuracy tradeoff of ARRoute. 

% % \begin{corollary}
% % Let $\delta$ be an upper bound of the probability of failing to find out KNNs of generated candidates. For a DCO with threshold $r$ and a negative object. Let $(1+\alpha)$ be the ratio between $dis$ and $r$, $\alpha >0$. The expected terminate dimensionality is 
% % \begin{align}
% %     \mathbb{E} \left[ \hat D \right]  = O \left[ \min \left( D, \frac{1}{\alpha ^2} \log \frac{D\cdot K}{\delta}  \right)  \right] 
% % \end{align}

% % }

% \smallskip\noindent\textbf{Time Complexity Analysis.} 
% % Similar to \texttt{AKNN+}, \texttt{AKNN++}'s time cost consists of $C_1$ for DCO of \texttt{ADSampling} and $C_2$ for the rest computations. 
% % With the same analysis of Theorem~\ref{theorem:AKNN+ time-accuracy}, we provide the time complexity results of \texttt{HNSW++} in the following theorem. 
% Similarly as we analyze the time complexity of \texttt{AKNN+}, we analyze the time complexity of \texttt{AKNN++} and present the results as follows.

% \begin{corollary}
% Let $\delta'$ be the probability that \texttt{HNSW++} fails to return the KNNs of the generated candidates. The expected time complexity of \texttt{HNSW++} is $C_1 + C_2$.
% For $C_1$, we have
% \begin{align}
%     C_1 = O \left( DK \log N + c_\alpha N_s \log \frac{DK}{\delta'} \right) 
% \end{align}
% where $c_\alpha  =  \frac{1}{N_{s}} \sum_{i=1}^{N_{s}} \min \left( \alpha_i^{-2}, D / \log \frac{DK}{\delta'}  \right)$ and $\alpha_i$ is the $alpha$ value of the $i^{th}$ candidate object.
% For $C_2$, we have $C_2 = O(N_s \log N_s)$. 
% % ***In this theorem, we can define $\delta'$ as the failure probability for \texttt{AKNN+}, which should be based on $\delta$.***
% \label{theorem:HNSW++ time-accuracy}
% \end{corollary}
% \begin{proof}
% It follows directly from Corollary~\ref{theorem: HNSW++ failure probability} and Lemma~\ref{theorem:ADSampling efficiency}.
% \end{proof}

% ***Jianyang, please help to include the time complexity of \texttt{HNSW++} here. ***

\smallskip\noindent\textbf{\texttt{HNSW++} v.s. \texttt{HNSW+}.}
Compared with \texttt{HNSW+}, \texttt{HNSW++} is expected to have a better time-accuracy trade-off, which we explain as follows. \underline{First}, consider the time cost. In \texttt{HNSW++}, for each DCO, the threshold distance is the $K^{th}$ largest distance, which is smaller than that used in \texttt{HNSW+} (i.e., the $N_{ef}^{th}$ largest distance). Correspondingly, in \texttt{HNSW++}, the $\alpha$ value, which is defined as $(dis - r)/r$, is larger than that in \texttt{HNSW+}. Therefore, the time cost for this DCO would be smaller than that in \texttt{HNSW+} according to the time complexity analysis of \texttt{ADSampling} in Section~\ref{subsection:theoretical analysis of ADSampling}. \underline{Second}, consider the effectiveness. While \texttt{HNSW++} and \texttt{HNSW+} use different distances for generating the candidates, we expect that they would generate candidates with similar qualities given that (1) the distances used by the two algorithms should be close (or the same in some cases) and (2) the method used for generating candidates, i.e., greedy beam-search, has a heuristic nature and there is no strong clue that it favors exact distances over approximate ones.

{ \JIANYANG
\noindent\textbf{Remarks.} We note that the technique of \texttt{HNSW++} can also be used in other 
% graph-based methods~\cite{malkov2018efficient, li2019approximate, FuNSG17, fu2019fast, diskann, NSW}.
graph-based methods~\cite{malkov2018efficient, li2019approximate, fu2019fast, diskann, NSW}.
This is because these algorithms also apply the greedy beam search {\CHENGC based on a set $\mathcal{R}$} in the query phase.
}

\subsection{\texttt{IVF++}: Towards Cache Friendliness}
\label{subsec:ivf++}
% \subsection{\texttt{ADSampling} for IVF: Towards Cache Friendliness}
% \label{subsec:adsampling-ivf}
}

% \section{Adaptive Resolution Algorithms for AKNN}
% As discussed in Section 1, DCO is ubiquitous and vital in existing AKNN algorithms. Thus, a straight-forward idea is to simply replace the plain DCO with our ADSampling framework, which forms our first basic algorithm ARSearch. Then by exploiting the properties of existing AKNN algorithms, we further propose adaptive resolution route (ARRoute) and adaptive resolution select (ARSelect), which bring more significant improvement.


% \subsection{Adaptive Resolution Search}
% \label{subsection:ARSearch}

% % \begin{figure}[thb]
% %     \centering
% %     \includegraphics[width=\linewidth]{figure/workflow.pdf}
% %     \caption{Workflow of Adaptive Algorithms}
% %     \label{fig:workflow}
% % \end{figure}

% % {\CHENG 14-07-Cheng: It looks that the above figure is for all three adaptive resolution algorithms but not for ARSearch only. Not sure if we can figures to illustrate different adaptive resolution algorithms.}

% % {\JIANYANG 15-07-Jianyang: Yes, it is for all three adaptive resolution algorithms. The workflows of all of them are identical except for the "Querying with Adaptive Algorithms" (red) part. The figure wants to emphasize that our algorithms just need a simple rotation.}

% ARSearch is a randomized algorithm which with high probability fully preserves the semantics of the original AKNN algorithms while be able to skip unnecessary dimension evaluation. Its workflow is extremely simple. During index phase, we simply apply random orthogonal transformation $P'$ on all database vectors and feed them into an existing AKNN index algorithm. During query phase, we first transform the query with $P'$, and feed it into the corresponding AKNN search algorithm while replacing DCO with ADSampling. Specifically, for graph-based methods which represent the dynamic cases, we apply ADSampling when comparing a newly visited object $i$ with the maximum in the result set $\mathcal R$. For IVF which represents the static cases, ADSampling is applied during selecting final KNNs from generated candidates, i.e. when scanning a new candidate, we compare it with the maximum in the currently maintained KNN set $\mathcal K$.

% {\JIANYANG We then investigate \emph{the failure probability of ARSearch}, where its failure is formally defined to be \emph{failing to preserve the full search process of an AKNN algorithm.} Let's consider when ARSearch would fail. \underline{First}, for graph-based methods, we are dynamically visiting new candidates and comparing their distance with the maximum in the result set $\mathcal R$ to determine whether to update the result set $\mathcal R$ and the search set $\mathcal S$. \underline{Second}, for IVF, we are scanning generated candidates and comparing their distance with the maximum in the KNN set $\mathcal K$. Suppose that none of DCOs fail in ARSearch, then for negative objects, we successfully prevent them from updating $\mathcal R$ or $\mathcal K$. In terms of the positive objects, we successfully evaluate their exact distance and use them to update $\mathcal R$ or $\mathcal K$, which also correctly preserve the search process. As a result, the success of all DCOs in ARSearch is a \emph{sufficient condition} of the success of ARSearch. Note that the ADSampling has no false negative failure. Thus, we have the following corollary:

% \begin{corollary}
% Let $\delta$ be an upper bound of failure probability of ARSearch. For a DCO with threshold $r$ and a negative object. Let $(1+\alpha)$ be the ratio between $dis$ and $r$, $\alpha >0$, and $N_{pos}$ be the number of positive objects, $N_{pos} \le N$. The expected terminate dimensionality is 
% \begin{align}
%     \mathbb{E} \left[ \hat D \right]  = O \left[ \min \left( D, \frac{1}{\alpha ^2} \log \frac{D\cdot N_{pos}}{\delta}  \right)  \right]  \label{equation: ARSearch time complexity}
% \end{align}
% The overall expected time complexity of ARSearch is
% %of DCOs in ARSearch is 
% \begin{align}
%     %O \left( DK \log N + c_\alpha  N_{s}\log \frac{DK}{\delta}  + c_\alpha  N_{s} \log\log N_s \right) \label{equation: overall ARSearch time complexity}
%     O \left( c_\alpha  N_{s}\log \frac{D}{\delta}  + c_\alpha N_s \log K  + c_\alpha  N_{s} \log\log N_s \right) \label{equation: overall ARSearch time complexity}
% \end{align}
% where $c_\alpha  =  \frac{1}{N_{s}} \sum_{i=1}^{N_{s}} \min \left( \alpha_i^{-2}, D / \log \frac{DK \log N_s}{\delta}  \right) $.
% \label{corollary:ARSearch}
% \end{corollary}

% \begin{proof}
% With Lemma~\ref{theorem:ADSampling accuracy} and union bound, we have 
% \begin{align}
%     \mathbb{P} \left\{ \emph{failure of ARSearch} \right\} \le  \exp \left( -c_0 \epsilon_0^2 + \log D \cdot N_{pos} \right) 
% \end{align}
% Then let the failure probability be $\delta$, we can obtain the corresponding $\epsilon_0$. Then by substituting $\epsilon_0$ in Lemma~\ref{theorem:ADSampling efficiency}, we have the Equation~\ref{equation: ARSearch time complexity}.
% % Like the proof of Theorem~\ref{theorem:time-accuracy of ADSampling}, we first show how $\epsilon_0$ controls failure probability and time complexity with Lemma~\ref{theorem:ADSampling accuracy} and \ref{theorem:ADSampling efficiency}, and them combine them to get the overall time-accuracy tradeoff.
% % First, with Lemma~\ref{theorem:ADSampling accuracy} and union bound, we have 
% % \begin{align}
% %     \mathbb{P} \left\{ \emph{failure of ARSearch} \right\} \le N_{pos}  \cdot \exp \left( -c_0 \epsilon_0^2 + \log D \right) 
% % \end{align}
% % Then, let the failure probability be $\delta$, we obtain 
% % \begin{align}
% %     \epsilon_0 = O \left( \sqrt {\log \frac{D\cdot N_{pos}}{\delta} }  \right) 
% % \end{align}
% % Substitute the $\epsilon_0$ in Lemma~\ref{theorem:ADSampling efficiency}, we proved Equation~\ref{equation: ARSearch time complexity}.

% In terms of $N_{pos}$, we analyze the $\hat N_{pos}$ of a random permutation as its upper bound because an AKNN algorithm is expected to generate candidates in a better order than random permutation. Note that this is a classic problem of "the expected number of updates of $K$ maxima", whose result is given by $\mathbb{E} \left[ \hat N_{pos} \right] \approx K\log N $. Due to the limit of space, we omit its proof here.
% % We then analyze the order of $N_{pos}$. Instead of the $N_{pos}$ of an AKNN algorithm, we analyze the $\hat N_{pos}$ of a random permutation as an upper bound because an AKNN algorithm is expected to generate candidates in a better order than random permutation. Let $I_i$ be the indicator random variable of rank-$i$ object. In particular, $I_i=1$ ($I_i=0$) indicates the rank-$i$ object would have positive (negative) comparison results. Then
% % \begin{align}
% %     \mathbb{E} \left[ \hat N_{pos} \right]  = \mathbb{E} \left[ \sum_{i=1}^{N} I_i \right]  = \sum_{i=1}^{N} \mathbb{E} \left[ I_i \right] 
% % \end{align}
% % Then looking into $I_i$, we notice that the rank-$i$ is positive if and only if there are no more than $K-1$ smaller objects on its left side, which means that within the top $i$ objects, the rank-$i$ object should be placed at $1-K$th position in the permutation, whose probability is $K / i$. Thus, $\mathbb{E} \left[ I_i \right] = K / i  $. As a result, 
% % \begin{align}
% %     \mathbb{E} \left[ \hat N_{pos} \right] \le \sum_{i=1}^{N} K / i = K \sum_{i=1}^{N} 1 / i
% % \end{align}
% % %where the last equation is due to harmonic series. 
% % Note that it's a harmonic series, we have
% % \begin{align}
% %     N_{pos} = O(K \log N)
% % \end{align}
% Then note that in ARSearch for DCOs of $N_{pos}$ positive objects, the time complexity of each single is $O(D)$ and for those of negative objects, that of each single is given in Equation~\ref{equation: ARSearch time complexity}. We have the overall complexity given in Equation~\ref{equation: overall ARSearch time complexity}.
% \end{proof}
% } 

% %We then investigate its failure probability, i.e. the probability of failing to preserve the semantics of the original AKNN algorithms. Let's first consider when our algorithm would fail. For graph-based methods, we're consistently visiting new objects and using them to update the result set $\mathcal R$ and the search set $\mathcal S$. However, only part of the visited objects could affect the search process, while others never become the minimum of the search set $\mathcal S$. We refer to those that affect the search as \textit{the positive} and those not as \textit{the negative}. Let $N_{pos}$ be the number of positive objects. Then once we ensure these $N_{pos}$ objects are correctly leveled up to $L$ and used to update $\mathcal R$ and $\mathcal S$, the result of search will be preserved. 
% %For IVF, similarly, once we ensure the KNNs among the candidates are correctly leveled up to $L$ and used to update $\mathcal K$ (as a result, $N_{pos}=K$ for IVF), the result of search will be preserved. 
% %{\JIANYANG In terms of IVF, similarly, a failure happens only when we wrongly rejected some KNN objects (positive) at level $l<L$. We emphasize that non-KNN objects (negative) do not affect the failure probability. Specifically, if a non-KNN object is rejected at level $l<L$, then it's correctly rejected. Otherwise, if it's leveled up to $L$, then its exact distance is obtained and an exact comparison can be made. As a result, in terms of the problem of DCO, our ADSampling framework will never produce negative positive results. } 
% %Recall that as introduced in Section 3.1, the failure probability is controlled by parameter $\epsilon_1$. It should be large enough to ensure that $N_{pos}$ objects would not be wrongly rejected at some level $l$. Following these thoughts, we have the following theorem whose detailed proof is later given in Section 5.

% % \begin{theorem}
% % For a KNN query in $D$-dimensional space, given an AKNN algorithm, let $N_{pos}$ be its number of positive objects. Replacing plain DCO with ADSampling, the error bound parameter
% % \begin{align}
% %     \epsilon_1 = O \left( \sqrt { \log D + \log \frac{N_{pos}}{\delta} }  \right) 
% % \end{align}
% % guarantees the probability of failing to fully preserve the results of the algorithm to be no greater than $\delta$. 
% % \label{theorem:eps}
% % \end{theorem}

% % {\JIANYANG
% % alternative expression

% % \begin{theorem}
% % For a KNN query in $D$-dimensional space, given an AKNN algorithm, let $N_{pos}$ be its number of positive objects and $L$ be the number of levels of ADSampling. Replacing plain DCO with ADSampling, the probability of failing to fully preserve the results of the algorithm decays super-exponentially with respect to $\epsilon_1$: 
% % \begin{align}
% %     \mathbb{P} \left\{ fail \right\}  \le \exp \left( -c_0 \cdot \epsilon_1^2 + \log L + \log N_{pos} \right) 
% % \end{align}
% % \label{theorem:eps_alter}
% % \end{theorem}
% % }

% % {\CHENG 14-07-Cheng: (1) Better to comment on the above theoretical result (e.g., what we can interpret from this result); (2) Would it possible to collect some empirical results for this theoretical result, e.g., we vary $\epsilon_1$ and show the failure probability results?}

% % {\JIANYANG 15-07-Jianyang: (1) I think this theorem is very counter-intuitive and it would be hard to provide intuitive interpretation. Actually, the format of this theorem is quite similar to Johnson-Lindenstrauss Lemma, which is seen by all to be quite counter-intuitive... (2) Yes. I think this experiment might need to be conducted by combing ARSearch with linear scan to eliminate the error introduced by AKNN algorithms. What do you think of it?}

% %Under the condition of Theorem~\ref{theorem:eps}, we then talk about the benefits ARSearch brings. As discussed in Section 3.1, our methods bring acceleration by avoiding unnecessary dimension evaluation when the distance of object $i$ is greater than its corresponding threshold $r$ of DCO. We're curious about how many dimensions are needed to reject object $i$. Formally, assuming that we do hypothesis testing every time after sampling one dimension, let random variable $\hat D_i$ be the terminate dimension of object $i$. We want to figure out its expected value $\mathbb{E} \left[ \hat D_i \right] $.

% %An obvious fact about $\mathbb{E} \left[ \hat D_i \right]  $ is that the more different $dis_i$ and $r$ are, the fewer dimensions are needed. Let $(1+\alpha_i)$ be the ratio between $dis_i$ and $r$, $\alpha_i>1$ (though we don't know $dis_i$ and $\alpha_i$ in prior). 
% %We claim that $\alpha_i$ is the maximum allowed multiplicative error (minimum resolution) to correctly compare object $i$ and threshold $r$, because with an error larger than $\alpha_i$, even if we have the precise estimation of $dis_i$, we cannot conclude the comparison results. 
% %We show that our methods reach the theoretically optimal result, i.e. rejecting an object with its minimum needed resolution, with the following theorem whose detailed proof is also given in Section 5.

% % \begin{theorem}
% % Under the condition of Theorem~\ref{theorem:eps}, let $(1+\alpha_i)$ be the ratio between a negative object $i$ and its corresponding distance threshold $r$. The expected terminate dimension of $i$ is 
% % \begin{align}
% %     \mathbb{E} \left[ \hat D_i \right]  = O \left[   \min \left( D, \frac{1}{\alpha_i^2}\log D + \frac{1}{\alpha_i^2}\log \frac{N_{pos}}{\delta} \right)  \right]  
% % \end{align}\label{theorem:efficiency}
% % \end{theorem}


% % {\JIANYANG
% % alternative expression
% % \begin{theorem}
% % Let $(1+\alpha_i)$ be the ratio between the exact distance of a negative object $i$ and its corresponding distance threshold $r$. The expected terminate dimensionality of $i$ is: 
% % \begin{align}
% %     \mathbb{E} \left[ \hat D_i \right]  = O \left[   \min \left( D,  \frac{1}{\alpha_i^2} \epsilon_1^2 \right)  \right]  
% % \end{align}
% % \label{theorem:efficiency_alter}
% % \end{theorem}

% % }


% % {\JIANYANG

% % Combing Theorem~\ref{theorem:efficiency_alter} and \ref{theorem:eps_alter}, we have the following corollary:

% % \begin{corollary}
% % For a KNN query in $D$-dimensional space, given an AKNN algorithm, let $N_{pos}$ be its number of positive objects and $(1+\alpha_i)$ be the ratio between a negative object $i$ and its corresponding distance threshold $r$. To guarantee the failure probability no greater than $\delta$, the expected terminate dimensionality is: 
% % \begin{align}
% %     \mathbb{E} \left[ \hat D_i \right]  = O \left[   \min \left( D, \frac{1}{\alpha_i^2}\log D + \frac{1}{\alpha_i^2}\log \frac{N_{pos}}{\delta} \right)  \right]  
% % \end{align}
% % \label{corollary:Npos}
% % \end{corollary}

% % }

% {\JIANYANG
% TODO: discussion on JL Lemma is commented first
% \begin{comment}

% We compare the corollary with the seminal work of random projection, Johnson-Lindenstrauss Lemma~\cite{johnson1984extensions} (JL Lemma), which states that with the failure probability of at most $\delta$, random projection preserves mutual distance among a finite set $\mathcal X$ of $N$ objects with at most $\epsilon$ multiplicative error when reducing dimensionality to $\Theta(\epsilon^{-2} \log \frac{N}{\delta})$, which is very similar to Corollary~\ref{corollary:Npos}. However, our algorithm achieves adaptivity by reducing the dimensionality according to the unknown $\alpha_i$. It's also worth noting that $N_{pos} < N$ (usually $N_{pos}$ is at the order of $K$, which is independent of the size of the databse). 

% Also according to JL Lemma, we conjecture that the tight complexity should be 
% \begin{conjecture}
% For a KNN query in Euclidean space, given an AKNN algorithm, let $N_{pos}$ be its number of positive objects, object $i$ be a negative object and $(1+\alpha_i)$ be the ratio between $dis_i$ and distance threshold $r$. To guarantee the failure probability no greater than $\delta$, the expected terminate dimensionality is
% \begin{align}
%     \mathbb{E} \left[ \hat D_i \right]  =O \left[ \min \left( D, \frac{1}{\alpha _{i}^{2} } \log \frac{N_{pos}}{\delta}  \right)  \right] 
% \end{align}
% \end{conjecture}
% It requires more careful theoretical analysis, which we leave as future work. 
% \end{comment}
% }

%Note that due to the recoverability of our method, when the comparison is extremely fragile ($\alpha_i$ is extremely small), ADSampling guarantees to produce exact DCO results with $D$ dimensions, which makes sure that it's at least no worse than the plain comparison. Plus, the theorem also shows that the complexity largely depends on the ratio $\alpha_i$, i.e. increasing the ratio can reduce the complexity, which motivates our following proposed algorithms.

%In ARSearch, the definitions of positive and negative are slightly different from that of KNN query. Note that in graph routing, some vertices are visited but never become the minimum object in the search set $\mathcal S$. Thus, they have no effect on the following search path. We refer to those that affect the following search as \textit{vertices in the path} and those not as \textit{vertices out of the path}. Now our task is to distinguish apart those in the path and those out of the path.

%According to our discussion, leveling up positive objects to level $L$ releases more space for approximation. However, in practice, we have no prior knowledge on the positive objects, i.e. when visiting a vertex, we don't know whether it'll affect the search path or not. Thus, we can only make an inference based on current searching results. Specifically, in standard greedy beam search, a necessary condition that an object becomes positive is that it should be smaller than $\max_{j\in \mathcal R} dis_j$. Thus, we compare a newly visited object with $\max_{j\in \mathcal R} dis_j$ to determine its dimensionality, which leads to our algorithm ARSearch:

%\input{pseudocode/ARSearch (algorithm package)}

%The main difference between ARSearch and standard greedy beam search lies line 13 to line 21. When visiting a vertex, standard greedy beam search always evaluates exact distance, while our method applies adaptive dimension sampling so that evaluates fewer dimensions when the visited object is negative. When it's positive, it'll always be leveled up to $L$, so that like the standard search, $\mathcal{R}$ and $\mathcal S$ only contain objects with exact distance.

% {\CHENG 14-07-Cheng: As we have discussed, (1) we can think about use shorter names for the three adaptive resolution algorithms; (2) re-organize the algorithms by whether they are general or specific for one scenario.}

% {\JIANYANG modified}

% {\JIANYANG
% Besides the general ARSearch for all AKNN algorithms, we also propose two specific methods: ARRoute and ARSelect corresponding to graph-based AKNN algorithms (represented by \texttt{HNSW}) and those AKNN algorithms generating candidates in batch (represented by \texttt{IVF}).
% }

% \subsection{Adaptive Resolution Route}
% \label{subsection: adaptive resolution route}
% {\JIANYANG
% We first optimize our algorithms for graph routing. Based on Corollary~\ref{corollary:ARSearch}, the complexity of DCO in ARSearch is largely determined by $\alpha = \frac{dis}{r} - 1 $ (When $dis < r$, we let $\alpha =0^+$, i.e., an extremely small positive number). To increase $\alpha$, since the exact distance of a candidate is always fixed, we target to decrease the distance threshold $r$.

% Recall that in ARSearch for graph-based methods, we compare candidates with the current $N_{ef}$th NN so as to preserve the full search process and consequently preserve the returned AKNNs. However, rather than preserving the full search process, our actual goal is to retrieve KNNs. It suggests that fully preserving the search process might not be necessary. We may allow some approximation on the search process while do nearly-exact DCOs for maintaining KNNs. In this case, we can do DCO with a smaller threshold, i.e. the $K$th NN instead of the $N_{ef}$th NN.

% Following these thoughts, we propose our adaptive resolution route (ARRoute) algorithm. In ARRoute, we consider that besides $\mathcal {R,S}$ we maintain an extra KNN set $\mathcal K$. For a new candidate, we first compare its distance with the maximum in $\mathcal K$ to maintain KNNs. Note that such comparison will produce an observed distance as a by-product (it can be exact or approximate) when terminating sampling. Then we use the observed distance to maintain $\mathcal R$ and $\mathcal S$ for greedy beam search. We notice that such observed distance has a promising property for graph routing, i.e., the closer one candidate is to the query ($dis$ is small and as a result $\alpha$ is small), the finer its observed distance would be (terminate dimension $\hat D$ would be large), which matches the importance of distance information for graph routing. Thus, potentially, it only introduces slight random perturbation on the heuristic greedy beam search, which might not harm the performance. We empirically verify this statement in Section~\ref{section:experiment}. 

% %we compare its distance with the maximum in $\mathcal K$. Then for two possible cases, we have the following discussion. (1) It's smaller and we obtain its exact distance. In this case, we first update the KNN set.

% %Specifically, besides result set $\mathcal R$ and search set $\mathcal S$, we explicitly maintain a KNN set $\mathcal K$. 




% %In ARRoute, to explicitly maintain KNNs, we propose to.

% %which motivates us to explicitly maintain a KNN set $\mathcal K$ besides the result set $\mathcal R$ and the search set $\mathcal S$ during graph routing.


% %As dicussed in Section~\ref{subsec:aknn}, DCOs with the $N_{ef}$th NN have two functions. (1) Implicitly maintaining KNN set. (2) Guiding the search process of greedy beam search. A natural question is that can we do KNN checking with nearly-exact DCOs while allows some approximation in generating search process?  

% %Thus, a natural question is that with the purpose of returning high quality AKNNs, can we do DCO with a smaller distance threshold, e.g., the distance of the current $K$th NN? 


% %We then investigate the obstacles when we do DCO with the $K$th NN. 

% %Furthermore, note that the returned AKNNs are the first $K$th objects in the result set $\mathcal R$. Thus, the result set $\mathcal R$ actually plays a dual role. (1) It \emph{implicitly maintains currently searched KNNs}. (2) It maintains greedy beam search. 

% %Combined with the discussion above, we propose to do DCO with $K$th NN. 

% %maintain currently searched KNNs with exact.


% %to preserve the returned AKNNs, the aforementioned condition is not \emph{necessary}. Suppose that we have obtained all the candidates generated from graph routing (we'll specify later how we actually generate them). We claim that for each candidate, doing DCO with the currently searched $K$th NN (denoted as $dis_{i_K}$) is enough (note that $K < N_{ef}$, so the distance threshold is decreased). Moreover, the success of all DCOs between KNN objects and the currently searched $K$th NN (technically, we maintain a max-heap $\mathcal K$ by exact distance as the KNN set to realize so) is \emph{sufficient and necessary} to preserve the returned AKNNs. 

% %In terms of \underline{sufficiency}, when evaluating a candidate, there are two cases. (1) The candidate is a KNN object. With the assumption of its success in DCO (it returns positive results because it's a KNN object), we can obtain its exact distance and use it to update $\mathcal K$. (2) The candidate a non-KNN object. When the current $\mathcal K$ is filled with KNN objects, since ADSampling does not produce false positive, it must correctly return negative results. When the current $\mathcal K$ still contains some non-KNN objects. We don't care about the comparison results because later when we meet KNN objects, all non-KNN objects will be successfully replaced. Thus, the success of the DCOs of all KNN objects is sufficient to preserve the returned AKNNs. In terms of \underline{necessity}, suppose that there is a failure of DCO of a KNN object. Then it cannot correctly update $\mathcal K$. As a result, it fails to preserve the returned AKNNs. Thus, to preserve the returned AKNN, doing DCO with the currently searched $K$th NN is safe and efficient.

% %Let's get back to an aforementioned by yet to be solved problem, i.e., how to generate candidates for graph routing when we only compare distance with the $K$th NN. Note that in both original graph routing and ARSearch, the exact distance of all objects in the search process will be obtained and used for heuristic greedy beam search. However, in our new setting of comparing distance with $dis_{i_K}$, when the distance a new candidate is larger than $dis_{i_K}$, we cannot obtain its exact distance. We resolve this issue with a by-product of \texttt{ADSampling}, i.e., the observed distance obtained when terminating sampling (termed terminate distance). 

% %In \texttt{ADSampling}, recall that when an observed distance falls into the rejection region, we terminate sampling. Such terminate distance is an estimator of the true distance and has good properties for graph routing, i.e., the smaller its distance is (corresponds to a small $dis$ and as a result, a small $\alpha$), the more accurate the estimator would be (corresponds to a larger $\mathbb{E} \left[ \hat D \right]  $). It implies that it would be coarse when a candidate is far away from the query and fine when it's close, which matches the importance of distance information for graph routing. We


% %We also note that greedy beam search is highly heuristic. Though using terminate distance for graph routing introduces slight random perturbation on the search process, we expect it to do little harm to its performance on AKNN query. We empirically verify this statement in Section~\ref{figure:evaluated_dimension}. 

% %Gathering all the components above, we propose our adaptive resolution route (ARRoute) algorithm. As described in Algorithm~\ref{code:ARRoute}, technically, instead of maintaining two sets in ARSearch: a result set $\mathcal R$ ({\JIANYANG technically, a max-heap by } exact distance) and a search set $\mathcal S$ ({\JIANYANG a min-heap by } exact distance), here we maintain three sets: a result set $\mathcal R$ ({\JIANYANG a max-heap by terminate distance}), a search set $\mathcal S$ ({\JIANYANG a max-heap by terminate distance}) and a KNN set $\mathcal K$ ({\JIANYANG a max-heap by } exact distance). During search, we compare a newly visited object with the current maximum of $\mathcal K$. When it returns positive results $dis_i \le r$, then it's used to update $\mathcal K$ (line 12-18) as well as updating $\mathcal R$ and $\mathcal K$ (line 19-22). If it returns negative $dis_i > r$ and only obtain approximate distance $dis_i'$ unlike what we did in ARSearch, we don't drop it directly. Instead, we still use its $dis_i'$ to update $\mathcal R$ and $\mathcal S$ (line 19-22). 


% }

%We next further optimize our adaptive resolution algorithm for graph routing. Note that for preserving the full greedy search path, ARSearch does ADSampling to compare a new object $i$ and the maximum of the result set $\mathcal R$, whose size $N_{ef}$ is usually larger than $K$, meaning that we are comparing $i$ with the current $N_{ef}$NN. However, to check whether a newly visited object can update the answers, we only need to compare it with the current KNN. Since the distance of $N_{ef}$NN is larger than that of KNN, it actually decreases the distance ratio $\alpha_i$ so as to increase complexity. 
%, which decreases the ratio $\alpha_i$ as well as increasing complexity. In a word, fully preserving search path does harm to efficiency in the context ADSampling. 
%At the same time, fully preserving search path is not necessary. Graph-based methods are highly heuristic and have no theoretical guarantee. There's no evidence to show the necessity of strictly following the path of greedy search. Slight perturbation might do little harm to its performance. Motivated by these two points, we propose ARRoute.
%which allows a larger extent of approximation on graph routing.

% \input{pseudocode/ARRoute}

%The core idea of ARRoute is to do exact comparison for KNN checking and  "approximate comparison" for graph routing. 
%{\JIANYANG To achieve this, we'll need to access the maximum of the current KNNs efficiently, which necessitates maintaining an extra KNN set $\mathcal K$. } 
%As described in Algorithm~\ref{code:ARRoute}, technically, instead of maintaining two sets in ARSearch: a result set $\mathcal R$ ({\JIANYANG technically, a max-heap by } exact distance) and a search set $\mathcal S$ ({\JIANYANG a min-heap by } exact distance), here we maintain three sets: a result set $\mathcal R$ ({\JIANYANG a max-heap by } approximate distance), a search set $\mathcal S$ ({\JIANYANG a max-heap by } approximate distance) and a KNN set $\mathcal K$ ({\JIANYANG a max-heap by } exact distance). During search, we compare a newly visited object with the current maximum of $\mathcal K$ instead of $\mathcal R$ with ADSampling (line 11). If it's leveled up to $L$ and $dis_i < r$, then it's used to update $\mathcal K$ (line 12-18). However, if it's rejected at some level $l$ and only obtain approximate distance $dis_i'$, unlike what we did in ARSearch, we don't drop it directly. Instead, we still use its approximate distance to update $\mathcal R$ and $\mathcal S$ (line 19-22). 

% {\CHENG 14-07-Cheng: For the above new design of using a third queue, we'd better explain the intuitions behind the design. Currently, we simply tell what we do but do not explain why we do this. We talked about some intuitions in an earlier paragraph, but there is a gap here, e.g., can we directly compare the Kth distance in set S, but not to introduce a third set K?}

% {\JIANYANG modified}

%We emphasize that though we don't expect ARRoute to produce exact search path, it would still be desirable if the approximate distance is of high accuracy when an object is close to the current KNNs, while is allowed to be coarse when it's far away. Luckily, such adaptivity for graph routing can also be naturally realized by ADSampling. Suppose that we're comparing a newly visited non-KNN object $i$ (so it's only for graph routing) with the distance of the current KNN. Then its terminate dimension (at the same time, accuracy) is determined by its distance ratio $\alpha_i$. The closer it is to the query, the higher the accuracy would be. Thus, it naturally achieve the adaptivity of resolution for graph routing.

%During standard greedy beam search, exact distance is evaluated for two purposes: 1) to maintain KNNs and 2) to route graph search. In ARRoute, for a newly visited object, we consider preserving the result of KNN checking while allows approximation for graph routing.
%We present our ARRoute in Algorithm~\ref{code:ARRoute}. Technically, instead of maintaining two sets in ARSearch: a result set $\mathcal R$ (exact distance) and a search set $\mathcal S$ (exact distance), here we maintain three sets: a result set $\mathcal R$ (approximate distance), a search set $\mathcal S$ (approximate distance) and a KNN set $\mathcal K$ (exact distance). During search, we compare a newly visited object with the current maximum of $\mathcal K$ instead of $\mathcal R$ with ADSampling (line 11). If it's leveled up to $L$ and $dis_i < r$, then it's used to update $\mathcal K$ (line 12-18). However, if it's rejected at some level $l$ and only obtain approximate distance $dis_i'$, unlike what we did in ARSearch, we don't drop it directly. Instead, we still use its approximate distance to update $\mathcal R$ and $\mathcal S$ (line 19-22). 

% {\JIANYANG
% In terms of theoretical guarantee, ARRoute does not guarantee to exactly preserve the full search process of greedy beam search. However, since it does KNN checking with ADSampling for all candidates, it provides the guarantee of returning KNNs of generated candidates.
% %another type of guarantee , i.e., guarantee to return KNNs of the generated candidates. 
% Under the context of such type of failure, we provide the following theorem to show the time-accuracy tradeoff of ARRoute. 

% \begin{corollary}
% Let $\delta$ be an upper bound of the probability of failing to find out KNNs of generated candidates. For a DCO with threshold $r$ and a negative object. Let $(1+\alpha)$ be the ratio between $dis$ and $r$, $\alpha >0$. The expected terminate dimensionality is 
% \begin{align}
%     \mathbb{E} \left[ \hat D \right]  = O \left[ \min \left( D, \frac{1}{\alpha ^2} \log \frac{D\cdot K}{\delta}  \right)  \right] 
% \end{align}
% The overall expected time complexity is
% %of DCOs in ARSearch is 
% \begin{align}
%     O \left( c_\alpha  N_{s}\log \frac{D}{\delta}  + c_{\alpha} N_s \log K \right) \label{equation: overall ARSearch time complexity}
% \end{align}
% where $c_\alpha  =  \frac{1}{N_{s}} \sum_{i=1}^{N_{s}} \min \left( \alpha_i^{-2}, D / \log \frac{D\cdot K}{\delta}  \right) $.
% \label{corollary:ARRoute}
% \end{corollary}
% \begin{proof}
% Note that Corollary~\ref{corollary:ARRoute} replaces $N_{pos}$ in Corollary~\ref{corollary:ARSearch} with $K$. This is because within generated candidates, if the DCOs of KNN objects are successful (as a result, we obtain their exact distance), then they can be successfully retrieved. Then the proof is the same as Corollary~\ref{corollary:ARSearch}. 
% \end{proof}
% }

%the failure of ARRoute is defined to be failing to find out KNNs among generated candidates.
%However, it still guarantees to find out KNNs from visited objects with high probability, which is a straight-forward corollary of the aforementioned theorems.

% \begin{corollary}
% For a KNN query in $D$-dimensional space, applying adaptive resolution algorithms, the error bound parameter
% \begin{align}
%     \epsilon_1 = O \left( \sqrt { \log D + \log \frac{K}{\delta} }  \right) 
% \end{align}
% guarantees the probability of failing to find out the KNNs of visited objects to be no greater than $\delta$. 
% \label{corollary:eps}
% \end{corollary}

% \begin{corollary}
% Under the condition of Corollary~\ref{corollary:eps}, let $(1+\alpha_i)$ be the ratio between a negative object $i$ and its corresponding distance threshold $r$. The expected terminate dimension of $i$ is 
% \begin{align}
%     \mathbb{E} \left[ \hat D_i \right]  = O \left[   \min \left( D, \frac{1}{\alpha_i^2}\log D + \frac{1}{\alpha_i^2}\log \frac{K}{\delta} \right)  \right]  
% \end{align}
% \label{corollary:efficiency}
% \end{corollary}
%We maintain one more set $\mathcal{K}$ in graph routing, whose size is constrained within $K$ to maintain KNNs among currently visited objects. During search, a newly visited is compared with the current maximum of $\mathcal {K}$ instead of $\mathcal R$ to determine its dimensionality. If it reaches level $L$, then it's used to update $\mathcal{K}$. In terms of $\mathcal R$ and $\mathcal{S}$, unlike ARSearch, we always update them even when the distance is approximate. 
%Then if it reaches level $L$ and can update $\mathcal{K}$, it's also used to update $\mathcal R$ and $\mathcal S$. Otherwise, when it cannot update $\mathcal {K}$, it might not be leveled up to $L$, meaning that its distance is approximate. Then we also use the approximate distance to update $\mathcal R$ and $\mathcal S$. As a result, unlike ARSearch, search* not always uses exact distance for graph routing. 
% It's worth noting that Corollary~\ref{corollary:ARRoute} is independent of the size of the database, indicating that there is no need to tune the parameters as a database scales up. 

% \subsection{Adaptive Resolution Scan}
{\JIANYANG

% We then further optimize AKNN algorithms which generate candidates in batch (represented by \texttt{IVF}). Recall that in ARSearch for \texttt{IVF}, we first generate all candidates in batch and sequentially conduct DCO for each candidate with the currently searched $K$th NN. According to our discussion in Section~\ref{subsection: adaptive resolution route}, such comparison is sufficient and necessary to preserve returned AKNNs. Thus, in this direction there is no space of further optimization unless we can predict the ground truth $dis_{i_K^*}$ in the very beginning. 

% Jianyang: Would it be helpful to introduce cache if we prove that the algorithm cannot be improved from the perspective of dimensionality? 

% {\CHENG 04-08-Cheng: I think we can drop the above discussion since we cannot theoretically justify that this idea would not work.}

% We turn to another direction, i.e., optimizing data layout. 
In the original \texttt{IVF} algorithm, the vectors in the same cluster are stored sequentially. When evaluating their distances, the algorithm scans all the dimensions of these vectors \emph{sequentially}, which exhibits strong locality of reference, {\CHENG and thus it is} cache-friendly. {\CHENG Figure~\ref{fig:data layout plain} illustrates the corresponding data layout {\JIANYANGB (as indicated by the arrow)} and 
% the sequence of data accesses.
{\CHENGB the data needed (as indicated by the colored background).}
% , which is indicated by the arrow.
} In {\CHENG \texttt{IVF+}}, 
% with the same data layout, 
though {\CHENG it scans} fewer dimensions than {\texttt{IVF}}, 
% its locality of reference is not fully utilized. 
% In Figure~\ref{fig:data layout}, we depict the data layout and the needed dimensions during scanning where the arrow line indicates the sequential store of dimensions. Note that in the plain case, since no dimension can be avoided, naturally storing data vectors one by one is cache-friendly. However, under the context of \texttt{ADSampling}, as shown in Figure~\ref{fig:data layout ARSearch}, many dimensions can be avoided. In this case, such storing strategy 
it would not be cache-friendly with the same data layout. 
Specifically, when {\CHENG \texttt{IVF+}} terminates {\CHENG the dimension sampling process} for a data vector, 
% due to the principle of locality, 
the subsequent dimensions {\CHENG would probably have been} loaded into cache from main memory though they are not needed. {\CHENG Figure~\ref{fig:data layout ARSearch} illustrates the corresponding data layout {\CHENGB and data needed.}
% {\JIANYANGB (as indicated by the arrow)} 
% and the sequence of data accesses of \texttt{IVF+} 
% {\JIANYANGB (as indicated by the colored background)}.
}
% , which brings overhead. 
}

{\CHENG
% Thus, the main issue is how we design an order to access the needed dimensions to make it as sequential as possible. 
We propose to re-organize the data layout of the candidates and adjust the order of the dimensions of the candidates to be fed to \texttt{ADSampling} accordingly so as to achieve more cache-friendly data accesses. Recall that for each candidate, \texttt{ADSampling} would definitely sample a few, say $d_1$, dimensions of the candidate first and then 
% aggressively 
{\JIANYANGREVISION incrementally}
sample more dimensions depending on the hypothesis testing outcomes. That is, the first $d_1$ dimensions of each candidate would be accessed for sure. Motivated by this, we store the first $d_1$ dimensions of all candidates sequentially in an array $A_1$ and the remaining $D - d_1$ dimensions of all candidates sequentially in another array $A_2$. 
% Figure~\ref{fig:data layout ARScan} shows this data layout. 
We note that the process of re-organizing the data layout can be conducted during the index phase. During the query phase, when using \texttt{ADSampling} for DCOs on the candidates, we follow the following order of the dimensions of the candidates: the first $d_1$ dimensions of the first candidate, the first $d_1$ dimensions of the second candidate, ..., the first $d_1$ dimensions of the last candidate, the $D-d_1$ dimensions of the first candidate, ..., the $D-d_1$ dimensions of the last candidate. 
Figure~\ref{fig:data layout ARScan} 
% {\JIANYANGB shows this order of the storage of the dimensions}
illustrates the corresponding data layout {\CHENGB and data needed.}
% dimensions to be fed to \texttt{ADSampling} 
% (as indicated by the arrow),
% {\JIANYANGB and the colored part represents the dimensions needed by \texttt{ADSampling}.}
% The resulting algorithm, which we call \texttt{IVF++}, differs from \texttt{IVF+} in its data layout and the order of dimensions of candidates to be fed to \texttt{ADSampling}. 
We call the resulting algorithm \texttt{IVF++}. \texttt{IVF++} and \texttt{IVF+} would produce exactly the same results, but the former is more cache friendly since it utilizes the locality of reference for the first $d_1$ dimensions of all candidates.
}
% Recall that in \texttt{ADSampling}, for a candidate, we start from sampling a few dimensions. It indicates that at any case, these dimensions are unavoidable. Thus, our strategy is to separately store the first few dimensions and the remaining. Specifically, during index phase we pick the first $d_1$ dimensions of all data vectors in a cluster and store them sequentially in an array $A_1$. For the rest $D - d_1$ dimensions of these vectors, we store them sequentially in another array $A_2$ as shown in Figure~\ref{fig:data layout ARScan}. {\CHENG ***Did we count the overhead of re-organizing the data layout? I believe we should do so.***} Then during query phase, when scanning data vectors in a cluster, we first scan the first $d_1$ dimensions of all data vectors (i.e., scanning $A_1$ in the way that the plain algorithm scans all dimensions in Figure~\ref{fig:data layout plain}) to obtain an approximate distance of $d_1$ dimensions. This process could be viewed as the suspension of \texttt{ADSampling} operations at dimension $d_1$. Then we enumerate the data vectors one by one. With the approximate distance of $d_1$ dimensions, if we can firmly claim that $dis > r$ (reject $H_0$ in hypothesis testing), then we skip the whole rest dimensions of the vector in $A_2$. Otherwise, we restart \texttt{ADSampling} from dimension $d_1$ by sampling more dimensions in the rest dimensions of $A_2$. Note that the algorithm differ from \texttt{IVF+} only in data layout, i.e., the order of dimension evaluation. Therefore, their number of evaluated dimensions and accuracy must be the same.

% {\CHENG ***The description in this paragraph is too brief. I cannot figure out how exactly the algorithm works with the description here.***}
% {\JIANYANG It has been modified.}

%\input{pseudocode/ARKSelect}

\begin{figure}[thb]
    \centering
    \vspace{-2mm}
    % \captionsetup[subfigure]{aboveskip=-1pt}
   %  \begin{subfigure}[b]{0.32\linewidth}
   %      \includegraphics[width=\textwidth]{revision experimental result/IVFPQ.pdf}
   %      \caption{\texttt{IVFPQ}}
	  % \label{fig:cost IVFPQ}
   %  \end{subfigure} 
    \begin{subfigure}[b]{0.3\linewidth}
        \centering
        \includegraphics[height=0.9\textwidth]{figure/ARScanPlain.pdf}
        \caption{{\CHENG \texttt{IVF}}}
        \label{fig:data layout plain}
    \end{subfigure}       
    \begin{subfigure}[b]{0.3\linewidth}
        \centering
        \includegraphics[height=0.9\textwidth]{figure/ARScanARSearch.pdf}
        \caption{{\CHENG \texttt{IVF+}}}
        \label{fig:data layout ARSearch}
    \end{subfigure}       
    \begin{subfigure}[b]{0.3\linewidth}
        \centering
        \includegraphics[height=0.9\textwidth]{figure/ARScanARScan.pdf}
        \caption{{\CHENG \texttt{IVF++}}}
        \label{fig:data layout ARScan}
    \end{subfigure}       
 %    \subfigure[{\CHENG \texttt{IVF}}]{
 %        \includegraphics[height=0.25\linewidth]{figure/ARScanPlain.pdf}
 %        \label{fig:data layout plain}
 %    }
 %    \subfigure[{\CHENG \texttt{IVF+}}]{
	% \includegraphics[height=0.25\linewidth]{figure/ARScanARSearch.pdf}
 %        \label{fig:data layout ARSearch}
 %    }
 %    \subfigure[{\CHENG \texttt{IVF++}}]{
	% \includegraphics[height=0.25\linewidth]{figure/ARScanARScan.pdf}
 %        \label{fig:data layout ARScan}
 %    }
    \vspace{-4mm}
    \caption{Data Layout {\CHENGC and Data Needed}}
    \vspace{-4mm}
    \label{fig:data layout}
\end{figure}


%In ARRoute, we further accelerate the adaptive algorithm by increasing the distance ratio $\alpha_i$ for graph routing (the dynamic case). Now following the same thread, we propose adaptive resolution select to further optimize the static case, where all candidates are given all at once. Different from the dynamic case, whose order of candidates is largely determined by graph routing, in the static case, we could manipulate the evaluation order for further acceleration. 

%Suppose that we evaluate the candidates sequentially. Let's first investigate two extreme examples: 1) Candidates are given in the decreasing order. Then every time, the scanned object is supposed to be leveled up to $L$ and update the KNN set $\mathcal K$. Thus, the algorithm degrades to the plain exact distance evaluation.  2) Candidates are given in the increasing order. In this case, we can get the ground truth $dis_{i_K}$ very soon, meaning that for the following candidates, we are comparing their distance with the minimum possible threshold, which reaches the optimal complexity. Thus, the core of the problem is find a good estimation of $dis_{i_K}$ as soon as possible.

%We just considered the scenario in graph routing, where new objects are dynamically visited and evaluated. In this case, since we have no prior knowledge about the ground truth distance of Kth NN $dis_{i_k^*}$, we can only proxy it with that of the currently visited Kth NN $\max_{j\in \mathcal K} dis_j$, which obviously overestimates it and reduces the space for approximation. Also, since the order of objects is largely determined by graph routing, we cannot manipulate it on our own to obtain $dis_{i_k^*}$ earlier. However, in the framework of filter-and-verification, we're facing a static case, where candidates are given all at once. Thus, we may determine the evaluation order to obtain $dis_{i_k^*}$ as soon as possible, which motivates us to design adaptive resolution re-ranking: 

%We next propose AdaSort. Compared with the dynamic case handled by priority queue, partial sort is the optimal algorithm for the static case, i.e. given a set of vectors all at once without dynamic insertion. A conventional solution is to first evaluate full-precision distance for all the objects, and then apply partial sort to find the smallest Ks. {\color{red} cite? textbook algorithm.} It's commonly used in filter-and-verification-based KNN algorithms: after $N$ candidates are generated, partial sort is applied to find the true KNNs.

%Unlike the dynamic case where we cannot determine the insertion and pop-out order, in static case, we seek for a good order to mitigate the overhead introduced by AdaSampling. The overhead of AdaSampling mainly comes from the unknown true distance of the Kth NN $dis_K$. If we know it a prior, we can deal with each object independently by leveling it up until $dis_i^* > dis_K$  or $l_i = L$ without dynamically maintaining a partial heap. Thus, for the static case, our strategy is to find $dis_K$ as soon as possible and do AdaSampling based on $dis_K$.

%\input{pseudocode/ARKSelect}

% {\CHENG 14-07-Cheng: (1) As we discussed before, we explain on more benefits of this design (e.g., data layout). (2) Does it make sense to first level up to a ith level and tune i as a hyperparameter. When i=1, it reduces the algorithm we consider now.}

% {\JIANYANG 15-07-Jianyang: (1) TODO (2) It might not bring significant benefits because tuning i is equivalent to predict the minimum needed resolution, which is achieved by the ADSampling framework.}

%To obtain a good estimation of $dis_{i_K}$, our strategy is to select K objects with coarse code as initial KNNs $\mathcal K$. Specifically, as shown in Algorithm~\ref{code:ARKSelect} we first level up all the candidates to level $1$ (line 1-2) and select the first $K$ objects according their approximate $dis'_i$ with quick select~\cite{quickselect} (line 3). The first $K$ objects with the minimum approximate distance are likely to provide good estimation of $dis_{i_K}$, so we start with evaluating them (line 4-6) and then sequentially the others (line 8-13). Note that during evaluating other objects, since we have leveled up all objects to 1, we can simply continue ADSampling from it. In practice, for the concern of cache, we separately store the dimensions of the first level and the rest. %Note that our method is different from the multi-stage filter-and-verification because after selecting the first $K$ we still scan the remaining. The step of selection is simply for acquiring better estimation of $dis_{i_k^*}$.
%In terms of theoretical guarantee, since for the static case, the only concern is to correctly find out the KNNs of the generated candidates. Thus, it has the same guarantee as Corollary~\ref{corollary:eps} and \ref{corollary:efficiency}.

{\CHENGB
\smallskip
\noindent\textbf{Theoretical Analysis.}}
{\JIANYANG 
% In terms of theoretical analysis, note that 
% Since \texttt{IVF++} and \texttt{IVF+} differ only in data layout, they have the same theoretical guarantee, which is provided in Corollary~\ref{corollary: find out KNNs}.
Since \texttt{IVF++} and \texttt{IVF+} differ only in data layout, they have the same theoretical guarantee (Corollary~\ref{corollary: find out KNNs}).
}

% {\JIANYANG In terms of theoretical guarantee, though Theorem~\ref{theorem:AKNN+ time-accuracy} provides a general result for all \texttt{AKNN+} algorithms (including \texttt{IVF+}, at the same time \texttt{IVF++} because they differ only in data layout), for methods which generate candidates all at once like IVF, we can provide a stronger guarantee. {\CHENG ***I don't quite understand why we have stronger guarantees for IVF than for HNSW. Because it uses a smaller distance threshold in general?***} We recall that as discussed in Theorem~\ref{theorem:HNSW++ time-accuracy}, returning KNNs of generated candidates only needs to produce correct comparison results for KNN objects. Thus, we have the following theorem. 

% \begin{corollary}
% The probability that \texttt{IVF+}/\texttt{IVF++} fails to return the same results as \texttt{IVF} is at most $\exp \left( -c_0 \epsilon_0^2 + \log (DK) \right)$. %where $N_{pos}$ is the number of DCOs for positive objects.
% \label{theorem: HNSW++ failure probability}
% \end{corollary}

% And correspondingly, its time complexity is given as follows.

% \begin{corollary}
% Let $\delta'$ be the failure probability of \texttt{IVF+/IVF++}. The expected time complexity of \texttt{IVF+/IVF++} is 
% \begin{align}
%     C_1 = O \left( DK \log N + c_\alpha N_s \log \frac{DK}{\delta'} \right) 
% \end{align}
% where $c_\alpha  =  \frac{1}{N_{s}} \sum_{i=1}^{N_{s}} \min \left( \alpha_i^{-2}, D / \log \frac{DK}{\delta'}  \right)$. $C_2 = O(N_s \log K)$ for \texttt{IVF}. 
% % ***In this theorem, we can define $\delta'$ as the failure probability for \texttt{AKNN+}, which should be based on $\delta$.***
% \label{theorem:HNSW++ time-accuracy}
% \end{corollary}

% %same as the discussion in Section~\label{subsection: adaptive resolution route}, returning correct results for KNN objects within generated candidates is sufficient and necessary to preserve returned AKNNs. Thus, it has the same guarantee of Corollary~\ref{corollary:ARRoute}.
% }

%recall that our ADSampling framework has no false positive failure. Thus, a query succeeds if and only if the positive are correctly leveled up to $L$ and used to update $\mathcal K$. In the static case, the positive is the KNNs objects of generated candidates. As a result, it has the same guarantee as Corollary~\ref{corollary:eps} and \ref{corollary:efficiency}.}
%Since for the static case, a query succeeds if and only if the KNNs objects are successfully leveled up to $L$ and used to update $\mathcal K$.}
%To obtain better estimation of $dis_{i^*_k}$, the strategy is to select K objects with coarse code. Specifically, we first level up all the objects to level $1$ and select the first $K$ objects with minimum $dis'_i$ with quick select~\cite{quickselect}. The first $K$ objects with the minimum approximate distance are likely to provide good estimation of $dis_{i_k^*}$, so we start with evaluating them and then sequentially the others. Note that our method is different from the multi-stage filter-and-verification because after selecting the first $K$ we still scan the remaining. The step of selection is simply for acquiring better estimation of $dis_{i_k^*}$.

%do conventional partial sort by the key of safe distance $dis^*_i$. The objects with small safe distance are likely to small true distance. Then we evaluate the full-precision distance of the first K elements and use a max-heap $H$ to maintain $dis_K$. Note that $H$ only contains objects of full precision. Then we scan and evaluate the remaining objects with AdaSampling independently. Once one object is leveled up to $L$, we insert it into $H$ and pop out the maximum to maintain $dis_K$. {\color{red} The complexity of AdaSort is $O(N \log K)$ instead of $O(L\cdot N \log N)$.  } 

%{\color{red} And the failure probability is different from AdaQ. }

%{\color{red} Some questions when describing time complexity: should I include the complexity of time evaluation? }

%{\color{red} Adaptive Partial Sort is a little bit trivial?}




%\subsection{Applications and Implementations}
%{\color{red} Is it necessary to write this part? I thought previous subsections have already given the way of applications and implementations.}

%\subsubsection{\textbf{Tree and Graph}}

%\subsubsection{\textbf{Filter-and-Verification Framework}}

%\subsubsection{\textbf{More High-Dimensional Data Management Problems}}

{\CHENG
\smallskip
\noindent\textbf{Remarks.} We note that the technique used for improving \texttt{IVF+} with cache friendliness can also be used for improving some other \texttt{AKNN+} algorithms, including those of tree-based methods~\cite{muja2014scalable, dasgupta2008random, ram2019revisiting, beygelzimer2006cover}, quantization-based methods~\cite{jegou2010product, imi} and hashing-based methods~\cite{datar2004locality, c2lsh, Dong2020Learning}.
% FLANN~\cite{muja2014scalable},  LSH~\cite{datar2004locality} and Neural LSH~\cite{Dong2020Learning}.
% XXX, XXX, and XXX
This is because all these algorithms generate the candidates in a batch and then re-rank the candidates for finding out KNNs.
}


% {\color{red} Let me first comment out remarks here to avoid distraction. It discusses our advantages.}

\begin{comment}

\subsubsection{\textbf{Remarks}}
{\color{red} may need discussion, emphasize the advantages}
\begin{itemize}
    %\item Note that AdaQ and AdaSort target to make improvements on real world datasets. It cannot decrease the worst case complexity because for extremely fragile queries, their dimensionality is inherently irreducible. {\color{red} It's also worth noting that the value of extremely fragile queries might need to be questioned. }
    %\item We emphasize that our methods target to produce correct results instead of $c$-approximate ones, which is different from the most popular setting of random-projection-based methods adopted by LSH family {\color{red} cite} and random partition tree {\color{red} cite}. Our methods themselves don't introduce any errors with high probability. 
    %\item Our methods also support the relaxed and decisive versions of KNN, i.e. $c$-KNN, $r$-NN. They adapt to $c$-approximate version by simply dividing $\gamma(d)$ by a factor of $c$, $\gamma_c(d) := \gamma(d)/c$. While for radius-based queries, it could be regarded as KNN queries providing the ground truth $dis_K$ in the beginning. Thus, it enables independent ADSampling just as we did in AdaSort, which improves the performance.
    %\item Our methods are friendly to practitioners: 1) Their implementations and applications are extremely easy. To apply them, one only needs to randomly transform datasets and queries, and substitutes classic algorithms with our adaptive ones.
    {\color{red} 2) They support dynamic update. Since they are data-oblivious, so unlike machine-learning-based adaptive methods, dynamic update doesn't corrupt their performance. }
    %{3) Our adaptive algorithms suit nearest neighbor search libraries consisted of multiple methods. Unlike algorithm-specific adaptive methods {\color{red} cite}, our methods suit all algorithms with one preprocessing. }
\end{itemize}

\end{comment}
\section{Experiment}

\captionsetup[subfigure]{aboveskip=10pt}

\label{section:experiment}


% In this section, we conduct experimental study to compare the end-to-end performance of two algorithms (\texttt{IVF}, \texttt{HNSW}) combined with or without our plugin adaptive resolution algorithms. 

\subsection{Experimental Setup}
\label{subsec: experimental setup}
\noindent
\textbf{Datasets.} We {\CHENG use} six public datasets with varying {\CHENG sizes and dimensionalities}~\footnote{{\JIANYANG Note that our techniques introduce nearly no extra space consumption (the only extra space consumption is brought by a $D\times D$ random orthogonal matrix, which is ignorable compared with the huge $D$-dimensional database of size $N$). Thus, {\CHENGC they do not affect} the scalability of the AKNN algorithms. We thus focus on million-scale datasets to verify {\CHENGC their} effectiveness in speeding up the AKNN algorithms.}}, whose details are shown in Table~\ref{tab:data}. These datasets {\CHENG have been} widely used to benchmark AKNN algorithms ~\cite{lu2021hvs,adaptive2020ml, li2019approximate}.
%{\CHENG ***Better provide some references here.***}
% {\CHENG We note that these {\JIANYANGREVISION public} datasets 
% % involve 
% {\JIANYANGREVISION provide} both data {\JIANYANGB and query} vectors}.
{\JIANYANGREVISION We note that these  public datasets 
% involve 
provide both data and query vectors}.
% , which is independent with database vectors.

% \begin{comment}
% \begin{table}[h]
%     \caption{Dataset Statistics}
%     \label{tab:data}
% \begin{tabular}{c|cc}
% \hline
% Dataset & Size      & Dimensionality \\ \hline
% NUS     & 268,643   & 500            \\
% MSong   & 992,272   & 420            \\
% DEEP    & 1,000,000 & 256            \\
% GIST    & 1,000,000 & 960            \\
% GloVe   & 2,196,017 & 300            \\
% Tiny5M  & 5,000,000 & 384            \\ \hline
% \end{tabular}
% \end{table}
% \end{comment}
\begin{table}[h]
% \vspace{-4mm}
\caption{Dataset Statistics}
\vspace{-4mm}
\label{tab:data}
\begin{tabular}{c|cccc}
\hline
Dataset & Size      & $D$ & {\JIANYANGREVISION Query Size} & Data Type \\ \hline
Msong   & 992,272   & 420       & {\JIANYANGREVISION 200}     & Audio     \\
DEEP    & 1,000,000 & 256       & {\JIANYANGREVISION 1,000}     & Image     \\
Word2Vec & 1,000,000 & 300      & {\JIANYANGREVISION 1,000}      & Text      \\
GIST    & 1,000,000 & 960       & {\JIANYANGREVISION 1,000}     & Image     \\
GloVe   & 2,196,017 & 300       & {\JIANYANGREVISION 1,000}     & Text      \\
Tiny5M  & 5,000,000 & 384       & {\JIANYANGREVISION 1,000}     & Image     \\ \hline
\end{tabular}
% \vspace{-4mm}
\end{table}

\smallskip
\noindent
% \textbf{{\CHENGB Algorithms and} Performance Metric.} 
{\JIANYANGREVISION \textbf{{\CHENGB Algorithms}.} }
{\CHENGB For reliable DCOs, we compare our proposed method \texttt{ADSampling} with the conventional \texttt{FDScanning} {\JIANYANGREVISION and \texttt{PDScanning} (Partial Dimension Scanning), which we explain below. \texttt{PDScanning} 
% is a method which 
incrementally scans the dimensions of {\chengr a} raw vector 
% as \texttt{ADSampling} does on the randomized vector, while different from \texttt{ADSampling}, it
{\chengr and} terminates {\chengr the process} when the distance based on the partially scanned $d$ dimensions, i.e., $\sqrt {\sum_{i=1}^d x_i^2}$, is greater than the distance threshold $r$. {\chengr We note that \texttt{PDScanning} starts with zero dimensions but not a pre-set number of dimensions since (1) it is hard to set the number and (2) starting from a certain number of dimensions or zero dimensions have very similar performance given the fact that the dimensions are scanned incrementally.} We also note that \texttt{PDScanning} is an exact algorithm for {\chengr DCOs} and has the worst-case time complexity of $O(D)$. We name the AKNN algorithms {\chengr with} \texttt{PDScanning} {\chengr for DCOs} as 
% \texttt{AKNN^*}
\texttt{AKNN}*
and the one with a further optimized data layout as 
% \texttt{AKNN^{**}}
\texttt{AKNN}**
(for \texttt{IVF} only).} We exclude those distance approximation methods such as product quantization and random projection from comparison since as explained in Section~\ref{sec:introduction} and further verified in Section~\ref{subsubsec:reliable-dco}, they can hardly achieve reliable DCOs. 
{\JIANYANGREVISION For AKNN algorithms, we  mainly focus on \texttt{HNSW}~\cite{malkov2018efficient} and \texttt{IVF}~\cite{jegou2010product} for providing the contexts of DCOs since they correspond to two state-of-the-art AKNN algorithms as benchmarked in~\cite{annbenchmark, li2019approximate}. We note that these methods are widely adopted in industry (including 
% Faiss~\cite{johnson2019billion}~\footnote{\url{https://github.com/facebookresearch/faiss}}, 
Faiss~\cite{johnson2019billion}, 
Milvus~\cite{milvus} and PASE~\cite{PASE}).}}
% {\JIANYANGREVISION For better comprehensiveness, we also consider one of the best tree-based methods \texttt{Annoy}~\footnote{\url{https://github.com/spotify/annoy}} (as benchmarked in \cite{annbenchmark, li2019approximate}) and a hashing-based method \texttt{PMLSH}~\cite{zheng2020pm}. We note that their performance of time-accuracy tradeoff is suboptimal compared with \texttt{HNSW} and \texttt{IVF}. 
{\JIANYANGREVISION For better comprehensiveness, we also consider one of the best tree-based methods \texttt{Annoy}~\cite{annoy} (as benchmarked in \cite{annbenchmark, li2019approximate}) and a hashing-based method \texttt{PMLSH}~\cite{zheng2020pm}. We note that their performance of time-accuracy tradeoff is suboptimal compared with \texttt{HNSW} and \texttt{IVF}. 
% To avoid potential distraction, 
Due to the limit of space, 
we include their results in 
Appendix~\ref{appendix:section tree and hashing}.}
% the technical report~\cite{technical_report}.}
% We use recall, i.e., the ratio between the number of successfully retrieved ground truth KNNs and $K$, to measure accuracy and {\JIANYANG query-per-second (QPS), i.e., the number of handled queries per second, to measure efficiency. 

\smallskip
\noindent
{\JIANYANGREVISION \textbf{{\CHENGB Performance Metrics}.} }
{\JIANYANGREVISION We use two metrics to measure the accuracy: (1) 
recall~\cite{annbenchmark, li2019approximate, malkov2018efficient, jegou2010product}, i.e., the ratio between the number of successfully retrieved ground truth KNNs and $K$ and (2) {\chengr average distance} ratio~\cite{reviewer_paper, c2lsh, huang2015query, sun2014srs, reviewer_SISAP_metric}, i.e., the average of the {\chengr distance ratios 
% (i.e., relative distance errors 
% %  $+1$
% )
(which equals to the average relative error on distance {\chengr plus one})
of}
% corresponding distance ratio between 
the retrieved $K$ objects {\chengr wrt} the ground truth KNNs.
% {\chengr where each distance is incremented by 1 for avoiding undefined ratios}. 
We adopt the query-per-second (QPS), i.e., the number of handled queries per second, to measure efficiency. }
% Note that the query time is measured \textit{end-to-end}, meaning that the time cost of random transformation on query vector is included. 
{\JIANYANGB Note that the query time is measured \textit{end-to-end} (i.e., including the time of random transformation on query vectors). {\JIANYANGREVISION We decompose the time cost in Section~\ref{subsec: time decompostion}.}}
% We also measure the number of {\CHENG sampled} dimensions {\CHENG by \texttt{ADSampling} for a DCO operation} to verify our {\CHENG theoretical results}.
{\CHENG We also measure the total number of dimensions evaluated by an algorithm. For \texttt{AKNN} algorithms, it means the total number of dimensions of the candidates 
% considered by the algorithm 
(since for each candidate, all of its dimensions are used for computing its distance). 
% For \texttt{AKNN+} {\JIANYANGREVISION (\texttt{AKNN}*)} and \texttt{AKNN++} {\JIANYANGREVISION (\texttt{AKNN}**)} algorithms, it means the total number of {\chengr sampled (scanned)} dimensions of the candidates (since for a candidate, only those sampled {\chengr (scanned)} dimensions are used for computing its distance approximately). {\JIANYANGREVISION All the mentioned metrics are averaged over the whole query set.}
{\JIANYANGREVISION For \texttt{AKNN+} (\texttt{AKNN}*) and \texttt{AKNN++} (\texttt{AKNN}**) algorithms, it means the total number of {\chengr sampled (scanned)} dimensions of the candidates (since for a candidate, only those sampled {\chengr (scanned)} dimensions are used for computing its distance approximately). All the mentioned metrics are averaged over the whole query set.}
}

\smallskip
\noindent
\textbf{Implementation.} 
% {\JIANYANG We verify our techniques on two state-of-the-art AKNN algorithms \texttt{HNSW}~\cite{malkov2018efficient} and \texttt{IVF}~\cite{jegou2010product}.} 
The implementation of an AKNN algorithm consists of two phases. During {\CHENG the} \underline{index phase}, we first generate a random orthogonal transformation matrix with the NumPy library, store it and apply the transformation to all {\CHENG data} vectors. 
% Then we feed the transformed vectors {\JIANYANGREVISION (the raw vectors for \texttt{AKNN}, \texttt{AKNN}* and \texttt{AKNN}**)} into existing AKNN algorithms. In particular, 
% % for \texttt{HNSW}, \texttt{HNSW+} and \texttt{HNSW++} 
% {\JIANYANGREVISION for \texttt{HNSW}, \texttt{HNSW+},  \texttt{HNSW++} and \texttt{HNSW}*}
% (note that they have the same graph structure), our implementation is based on hnswlib~\cite{malkov2018efficient}, while 
% % for \texttt{IVF}, \texttt{IVF+}, and \texttt{IVF++} 
% {\JIANYANGREVISION for \texttt{IVF}, \texttt{IVF+}, \texttt{IVF++}, \texttt{IVF}* and \texttt{IVF}**}
% (note that they have the same cluster structure), our implementation of {\CHENGB K-means} clustering is based on the Faiss library~\cite{johnson2019billion}. 
% Then during {\CHENG the} \underline{query phase}, all {\CHENG algorithms} are implemented in C++.
% For a new query, we first transform the query vector with the Eigen library~\cite{eigenweb} for fast matrix multiplication {\JIANYANGREVISION when running \texttt{AKNN+} and \texttt{AKNN}++ algorithms (For \texttt{AKNN}, \texttt{AKNN}* and \texttt{AKNN}**, they involve no transformation). } Then we feed the vector into 
% % the \texttt{AKNN}, \texttt{AKNN+} and \texttt{AKNN++} 
% {\JIANYANGREVISION the \texttt{AKNN}, \texttt{AKNN+},  \texttt{AKNN++}, \texttt{AKNN}* and \texttt{AKNN}**}
% algorithms. 
{\JIANYANGREVISION Then we feed the transformed vectors (the raw vectors for \texttt{AKNN}, \texttt{AKNN}* and \texttt{AKNN}**) into existing AKNN algorithms. In particular, 
% for \texttt{HNSW}, \texttt{HNSW+} and \texttt{HNSW++} 
for \texttt{HNSW}, \texttt{HNSW+},  \texttt{HNSW++} and \texttt{HNSW}*
(note that they have the same graph structure), our implementation is based on hnswlib~\cite{malkov2018efficient}, while 
% for \texttt{IVF}, \texttt{IVF+}, and \texttt{IVF++} 
for \texttt{IVF}, \texttt{IVF+}, \texttt{IVF++}, \texttt{IVF}* and \texttt{IVF}**
(note that they have the same cluster structure), our implementation of {\CHENGB K-means} clustering is based on the Faiss library~\cite{johnson2019billion}.} 
Then during {\CHENG the} \underline{query phase}, all {\CHENG algorithms} are implemented in C++.
For a new query, we first transform the query vector with the Eigen library~\cite{eigenweb} for fast matrix multiplication when running \texttt{AKNN+} and \texttt{AKNN}++ algorithms 
{\JIANYANGREVISION (For \texttt{AKNN}, \texttt{AKNN}* and \texttt{AKNN}**, they involve no transformation). Then we feed the vector into 
% the \texttt{AKNN}, \texttt{AKNN+} and \texttt{AKNN++} 
the \texttt{AKNN}, \texttt{AKNN+},  \texttt{AKNN++}, \texttt{AKNN}* and \texttt{AKNN}**
algorithms.}
Following~\cite{graphbenchmark, li2019approximate}, we disable all hardware-specific optimizations including SIMD, memory prefetching and multi-threading ({\JIANYANG including those in the Eigen library}) {\CHENG so as to focus on the comparison among algorithms themselves}. 

%We {\CHENG conduct} random orthogonal transformation with the NumPy library and KMeans {\CHENG clustering (for \texttt{IVF}, \texttt{IVF+}, and \texttt{IVF++})} with the Faiss library during index phase. {\CHENG ***The reference to the Faiss library can be better provided.***}
%During query phase, all benchmarks were implemented in C++. {\CHENG ***We need conduct the random transformation on a query (i.e., during the query phase), for which NumPy library is used, and this is not in C++? I feel this part needs to be revised.***}
%Our implementation of \texttt{HNSW} is based on hnswlib~\cite{malkov2018efficient} and that of matrix multiplication (for random orthogonal transformation) is based on the Eigen library. {\CHENG ***The description is not clear enough. It is mentioned earlier that random orthogonal transformation is based NumPy.***} 
%Following~\cite{graphbenchmark, li2019approximate}, we disable all hardware-specific optimizations including SIMD, memory prefetching and multi-threading ({\JIANYANG including those in the Eigen library}) {\CHENG so as to focus on the comparison among algorithms themselves}. 

\smallskip
\noindent
\textbf{Parameter Setting.} 
{\JIANYANG For \texttt{HNSW}, two parameters are preset to control {\CHENG the construction of the graph}, namely $M$ to control the number of connected neighbors and $efConstruction$ to control the quality of approximate nearest neighbors. We follow the parameter settings of its original work~\cite{malkov2018efficient} where the parameters are set as $M=16$ and $efConstruction=500$.} 
For \texttt{IVF}, as suggested in the Faiss library~\footnote{\url{https://github.com/facebookresearch/faiss/wiki/Guidelines-to-choose-an-index}}, the number of clusters should be around the square root of the cardinality of the database. Since we focus on million-scale datasets, it's set to be {\CHENG 4,096}. 
For \texttt{ADSampling}, 
% though Theorem~\ref{theorem:eps} provides the asymptotic order of $\epsilon_0$, empirically here we suggest to 
we vary $\epsilon_0$ {\CHENG with values of 
% 1, 2, and 3 
1.5, 1.8, 2.1, 2.4, 2.7 and 3.0
and study its effects in Section~\ref{subsub:parameter}. Based on the results, we adopt the setting of 
% $\epsilon_0 = 2$
$\epsilon_0 = 2.1$
as the default one}.
% from 1.0 to 3.0, within which, we set $\epsilon_0=2.0$ by default. 
% {\CHENG ***Note that $\epsilon_0$ has been changed to $\epsilon_0$. This should be done throughout this section for consistency.***}
{\CHENG 
% In \texttt{ADSampling}, it aggressively and incrementally sample dimensions of a data vector in iterations and at each iteration, it performs a hypothesis testing. 
{\JIANYANG Recall that in \texttt{ADSampling}, it 
% aggressively and incrementally 
{\JIANYANGREVISION incrementally samples} some dimensions of a data vector and performs a hypothesis testing in iterations. }
To avoid the overhead of 
frequent hypothesis testings, {\JIANYANG we sample $\Delta_d$ dimensions at each iteration. By default, we set $\Delta_d=32$.}
{\JIANYANG Its parameter study is also provided in Section~\ref{subsub:parameter}.}
%double the number of dimensions sampled at each iteration. Specifically, it samples $d_1=32$, $d_2=64$, $d_3=128$, ..., $D$ dimensions at the 1st, 2nd, 3rd, ..., last iterations, respectively.
}
% To avoid the overhead of frequent hypothesis testings, we set the testing dimensions to be the power of 2 starting from 32, i.e. $d_1=32, d_2=64, d_3=128, ..., d_L = D$. 
% {\color{red} It makes sure that the actual evaluated dimension is at most twice as the theoretically expected. (possibly not true)} 

% \textbf{Environment.} 
All C++ source codes are complied by g++ 9.4.0 with 
% the standard 
\texttt{-O3} optimization under
% the environment of 
Ubuntu 20.04LTS. The Python source codes (which are used during the index phase) are run on Python 3.8. All experiments are {\CHENG conducted} on a machine with AMD Threadripper PRO 3955WX 3.9GHz 16C/32T processor and 64GB RAM. The code and datasets are available at \url{https://github.com/gaoj0017/ADSampling}.
%{\CHENG ***We normally specify the OS as well.***}

% \begin{figure*}[ht]
% % \vspace*{-4mm}
%   \centering 
%   % \includesvg[width=17cm]{experimental result/time-accuracy.svg}
%   % \includesvg[width=17cm]{revision experimental result/time-accuracy.svg}
%   \includesvg[width=17cm]{revision experimental result/time-accuracy-main.svg}
%   \vspace*{-4mm}
%   \caption{{\JIANYANGREVISION Time-Accuracy Tradeoff (\texttt{HNSW} and \texttt{IVF}).}}
%   % \vspace*{-4mm}
%   \label{figure:time-accuracy}
% \end{figure*}

\begin{figure*}[ht]
% \vspace*{-4mm}
  \centering 
    \includegraphics[width=17cm]{revision experimental result/time-accuracy-main.pdf}
  % \includesvg[width=17cm]{revision experimental result/time-accuracy-main.svg}
  \vspace*{-4mm}
  \caption{{\JIANYANGREVISION Time-Accuracy Tradeoff (\texttt{HNSW} and \texttt{IVF}).}}
  % \vspace*{-4mm}
  \label{figure:time-accuracy}
\end{figure*}

\begin{figure*}[ht]

% \captionsetup[subfigure]{aboveskip=-1pt,belowskip=-1pt}
% \vspace*{-4mm}
% \setlength{\abovecaptionskip}{-4.mm}
   %  \begin{subfigure}[b]{0.32\linewidth}
   %      \includegraphics[width=\textwidth]{revision experimental result/IVFPQ.pdf}
   %      \caption{\texttt{IVFPQ}}
	  % \label{fig:cost IVFPQ}
   %  \end{subfigure} 
   \captionsetup[subfigure]{aboveskip=-5pt}
\begin{subfigure}[b]{0.48\linewidth}
    % \includesvg[width=8.5cm]{revision experimental result/GIST-HNSW.svg}
    \includegraphics[width=\textwidth]{revision experimental result/GIST-HNSW.pdf}
    \caption{\texttt{HNSW}}
    \label{figure:evaluated_dimension HNSW}
\end{subfigure}  
\begin{subfigure}[b]{0.48\linewidth}
    \includegraphics[width=\textwidth]{revision experimental result/GIST-IVF.pdf}
    % \includesvg[width=8.5cm]{revision experimental result/GIST-IVF.svg}
    \caption{\texttt{IVF}}
    \label{figure:evaluated_dimension IVF}
\end{subfigure} 
% \subfigure[\texttt{HNSW}]{
%     % \captionsetup{skip=10pt}
% % \subfigure[]{
%     % \vspace*{-4mm}
%     \includesvg[width=8.5cm]{revision experimental result/GIST-HNSW.svg}
%     \label{figure:evaluated_dimension HNSW}
% }
% \subfigure[\texttt{IVF}]{
% % \subfigure[]{
%     % \vspace*{-4mm}
% 	\includesvg[width=8.5cm]{revision experimental result/GIST-IVF.svg}
% 	\label{figure:evaluated_dimension IVF}
% }
\vspace*{-4mm}
\caption{\JIANYANGREVISION Evaluated Dimensionality and Accuracy.}
\vspace*{-4mm}
\label{figure:evaluated_dimension}
\end{figure*}

\subsection{{\CHENG Experimental Results}}
\label{section:time-accuracy}
\subsubsection{\textbf{{\CHENG Overall Results (Time-Accuracy Trade-Off)}}}
\label{subsec:main result}
% We first compare the end-to-end search performance of the \texttt{IVF} and \texttt{HNSW} algorithms, each with and without our adpative resolution plugins. 
{\JIANYANGREVISION We {\CHENG plot} the QPS-recall curve (upper panels, upper-right is better) and the QPS-ratio curve (lower panels, upper-left is better) by varying $N_{ef}$ for \texttt{HNSW}/\texttt{HNSW}*/\texttt{HNSW+}/\texttt{HNSW++} and $N_{probe}$ for \texttt{IVF}/\texttt{IVF}*/\texttt{IVF}**/\texttt{IVF+}/\texttt{IVF++} in Figure~\ref{figure:time-accuracy}.} 
We focus only on the region with the recall at least 80\% {\CHENG based on} practical needs.
%
Overall, with the results in Figure~\ref{figure:time-accuracy}, we can observe clearly 
%except for $K=20$ on NUS 
% that the AKNN algorithms equipped with our adaptive plugins outperform the plain baseline significantly. 
% {\CHENG that (1) the \texttt{AKNN+} algorithms (represented by yellow curves) outperform the plain \texttt{AKNN} algorihtms (represented by greeen curves) and (2) the \texttt{AKNN++} algorithms (represented by red curves) further outperform the \texttt{AKNN+} algorithms.}
% {\CHENG ***We may elaborate on the results more with discussions (e.g., when the gap is larger (in terms of recall, datasets, etc.), how large is the gap in general, and any other findings, any comments on how \texttt{ADSampling} improves \texttt{HNSW} and \texttt{IVF} differently, comments on the comparison between \texttt{HNSW} and \texttt{IVF} (e.g., it can be mentioned that no algorithm dominates the other one on all datasets), etc. These results are the major ones in this paper, and it is fine that we discuss these results in more detail.***}
%Even for $K=20$ on NUS, our algorithms only fail in high accuracy region of \texttt{HNSW} algorithm, where it's outperformed by \texttt{IVF}, which proves the effectiveness of our techniques.
% Besides, within our adaptive resolution algorithms, it shows that the ARSelect and ARRoute (green) stably outperform the adaptive resolution search (yellow), which verifies the effectiveness of our algorithm-specific optimization. 
%
{\JIANYANG that (1) the \texttt{AKNN+} algorithms (represented by the orange curves) outperform the plain \texttt{AKNN} algorithms (represented by the red curves), (2) the \texttt{AKNN++} algorithms (represented by the green curves) further outperform the \texttt{AKNN+} algorithms}, {\JIANYANGREVISION (3) the baseline method \texttt{HNSW}* (represented by the blue curves) brings very minor improvements on \texttt{HNSW} for all the tested datasets and (4) the baseline methods \texttt{IVF}* {\chengr (represented by the blue curves)} and \texttt{IVF}** (represented by the violet curves) are outperformed by \texttt{IVF+} consistently and significantly ({\chengr and by} \texttt{IVF++} {\chengr with an} even larger margin).}

{\JIANYANG
Besides, we have the following observations. (1) Our techniques bring more improvements on \texttt{IVF} than on \texttt{HNSW} (even when \texttt{IVF} performs better than \texttt{HNSW}, e.g., on Word2Vec). We ascribe it to the fact that other computations than DCOs of \texttt{HNSW} are heavier than {\CHENG those} of \texttt{IVF} (as shown in Figure~\ref{fig:cost statistics}). (2) Our techniques in general bring more improvements on high accuracy region than on low accuracy region (e.g., GIST 95\% v.s. 85\%). This is because {\CHENGB when} an AKNN algorithm targets higher accuracy, it unavoidably generates more low-quality candidates with larger distance gap $\alpha$, 
% which needs fewer dimensions.
{\CHENGB for which it needs fewer dimensions for reliable DCOs.} {\JIANYANGREVISION (3) The data layout optimization brings more improvements on \texttt{IVF+} (i.e., \texttt{IVF++} v.s. \texttt{IVF+}) than on \texttt{IVF}* (i.e., \texttt{IVF}** v.s. \texttt{IVF}*). This is because \texttt{ADSampling} has the logarithmic complexity while the baseline \texttt{PDScanning} has the linear complexity. Specifically, the first $\Delta_d$ dimensions are sufficient for many DCOs when using \texttt{ADSampling} and thus, many accesses to the second array $A_2$ in Figure~\ref{fig:data layout ARScan} can be avoided. When using \texttt{PDScanning} for the DCOs, it will still access the second array frequently because it needs {\chengr more than $\Delta_d$} dimensions.}
}

% \textbf{\texttt{ADSampling} on datasets with different dimensionality.} We study the improvement \texttt{ADSampling} brings on datasets with different dimensionality. 

% \textbf{\texttt{HNSW}/\texttt{HNSW+}/\texttt{HNSW++}.} 

% \textbf{\texttt{IVF}/\texttt{IVF+}/\texttt{IVF++}.} 

\subsubsection{\textbf{{\CHENG Results of Evaluated Dimensions and Recall}}}
\label{subsub:dimensions and recall}
% {\CHENG ***From this point on, the descriptions are out-dated - they are based on the previous presentation flow and notions. Please revise them to be betwee aligned with the latest presentation and notions.***}

{\JIANYANG 
We then study the number of evaluated dimensions and the recall of 
% \texttt{AKNN}/\texttt{AKNN+}/\texttt{AKNN++}
{\JIANYANGREVISION \texttt{AKNN}/\texttt{AKNN+}/\texttt{AKNN++}/\texttt{AKNN}* (\texttt{AKNN}** has exactly the same curve as \texttt{AKNN}* and thus, is omitted)}
under the same search parameter setting ($N_{ef}$ for \texttt{HNSW} and $N_{probe}$ for \texttt{IVF}). {\CHENG For the number of evaluated dimensions, we measure its ratio over that of 
% the plain
\texttt{AKNN} in percentage for the ease of comparison. The results are shown in Figure~\ref{figure:evaluated_dimension}.}
% Specifically, for \texttt{HNSW} (resp. IVF) we plot the total evaluated dimension\footnote{For conciseness, we show its ratio over that of the plain \texttt{AKNN} in percentage.}-$N_{ef}$ (resp. $N_{probe}$) and recall-${N_{ef}}$ (resp. recall-${N_{probe}}$) curves in Figure~\ref{figure:evaluated_dimension}.
}

%We then verify our statement that our adaptive algorithms can almost preserve the search results while avoid a large number of unnecessary dimension evaluation. We study the total evaluated dimensions and accuracy of \texttt{HNSW} and \texttt{IVF} with and without our adaptive resolution algorithms under the same search parameter ($N_{probe}$ for \texttt{IVF} and $N_{ef}$ for \texttt{HNSW}) setting. In Figure~\ref{figure:evaluated_dimension}, the upper panel shows the curve of the ratio of evaluated dimension between adaptive resolution algorithms (yellow and green) and the plain baseline (red) and the lower shows the recall curve, both with respect to the search parameters $N_{ef}$ and $N_{probe}$. 

% \begin{comment}
% \begin{figure*}[ht]
%   \centering 
%   \includesvg[width=18cm]{experimental result/evaluated_dimension.svg}
%   \caption{Evaluated Dimensionality and Accuracy.}
%   \label{figure:evaluated_dimension}
% \end{figure*}
% \end{comment}

% \begin{figure*}[ht]
% % \vspace*{-4mm}
% \subfigure[\texttt{HNSW}]{
%     % \vspace*{-4mm}
%     \includesvg[width=8.5cm]{experimental result/HNSW.svg}
%     \label{figure:evaluated_dimension HNSW}
% }
% \subfigure[\texttt{IVF}]{
%     % \vspace*{-4mm}
% 	\includesvg[width=8.5cm]{experimental result/IVF.svg}
% 	\label{figure:evaluated_dimension IVF}
% }
% \vspace*{-4mm}
% \caption{Evaluated Dimensionality and Accuracy.}
% \vspace*{-4mm}
% \label{figure:evaluated_dimension}
% \end{figure*}

% \begin{figure*}[ht]
% % \captionsetup[subfigure]{aboveskip=-1pt,belowskip=-1pt}
% % \vspace*{-4mm}
% % \setlength{\abovecaptionskip}{-4.mm}
%    %  \begin{subfigure}[b]{0.32\linewidth}
%    %      \includegraphics[width=\textwidth]{revision experimental result/IVFPQ.pdf}
%    %      \caption{\texttt{IVFPQ}}
% 	  % \label{fig:cost IVFPQ}
%    %  \end{subfigure} 
%    \captionsetup[subfigure]{aboveskip=-5pt}
% \begin{subfigure}[b]{0.48\linewidth}
%     % \includesvg[width=8.5cm]{revision experimental result/GIST-HNSW.svg}
%     \includegraphics[width=\textwidth]{revision experimental result/GIST-HNSW.pdf}
%     \caption{\texttt{HNSW}}
%     \label{figure:evaluated_dimension HNSW}
% \end{subfigure}  
% \begin{subfigure}[b]{0.48\linewidth}
%     \includegraphics[width=\textwidth]{revision experimental result/GIST-IVF.pdf}
%     % \includesvg[width=8.5cm]{revision experimental result/GIST-IVF.svg}
%     \caption{\texttt{IVF}}
%     \label{figure:evaluated_dimension IVF}
% \end{subfigure} 
% % \subfigure[\texttt{HNSW}]{
% %     % \captionsetup{skip=10pt}
% % % \subfigure[]{
% %     % \vspace*{-4mm}
% %     \includesvg[width=8.5cm]{revision experimental result/GIST-HNSW.svg}
% %     \label{figure:evaluated_dimension HNSW}
% % }
% % \subfigure[\texttt{IVF}]{
% % % \subfigure[]{
% %     % \vspace*{-4mm}
% % 	\includesvg[width=8.5cm]{revision experimental result/GIST-IVF.svg}
% % 	\label{figure:evaluated_dimension IVF}
% % }
% \vspace*{-4mm}
% \caption{\JIANYANGREVISION Evaluated Dimensionality and Accuracy.}
% \vspace*{-4mm}
% \label{figure:evaluated_dimension}
% \end{figure*}

% \begin{figure}[thb]
%   \centering 
% %   \vspace{-4mm}
%   \includesvg[width=\linewidth]{revision experimental result/dimension-tradeoff.svg}
%   \vspace{-8mm}
%   \caption{Evaluated Dimensionality and Accuracy.}
%   \vspace{-4mm}
%   \label{figure:evaluated_dimension}
%   \label{figure:evaluated_dimension IVF}
%   \label{figure:evaluated_dimension HNSW}
% \end{figure}

% \begin{figure}[thb]
%   \centering 
% %   \vspace{-4mm}
%   % \include[width=\linewidth]{revision experimental result/dimension-tradeoff.svg}
%   \includegraphics[width=\linewidth]{revision experimental result/decomposition.pdf}
%   \vspace{-8mm}
%   \caption{Decomposition of Time Cost.}
%   \vspace{-4mm}
%   \label{fig:decomposition}
% \end{figure}



% \begin{figure}[ht]
% % \vspace*{-4mm}
% \subfigure[\texttt{HNSW}]{
%     \vspace{-4mm}
%     \includesvg[width=\linewidth]{experimental result/HNSW.svg}
%     \label{figure:evaluated_dimension HNSW}
% }
% \subfigure[\texttt{IVF}]{
%     \vspace{-4mm}
% 	\includesvg[width=\linewidth]{experimental result/IVF.svg}
% 	\label{figure:evaluated_dimension IVF}
% }
% \vspace{-8mm}
% \caption{Evaluated Dimensionality and Accuracy.}
% \vspace{-4mm}
% \label{figure:evaluated_dimension}
% \end{figure}

\smallskip
\noindent
\textbf{{\CHENG Overall Results.}} 
%As shown in the lower panel, the recall curves of the plain baseline, adaptive resolution search and ARSelect (ARRoute) largely overlap, indicating that our methods themselves indeed preserve the search results well. At the same time, as shown in the upper panel, under the same parameter setting, all our adaptive algorithms skip lots of dimensions (on \texttt{IVF}, GIST, K=20, it even saves up to 89.95\% total dimension evaluation).
{\JIANYANG
In Figure~\ref{figure:evaluated_dimension}, we can observe clearly that \texttt{AKNN+} and \texttt{AKNN++} {\CHENG evaluate} much fewer dimensions than \texttt{AKNN} while {\CHENG reaching} nearly the same recall. Specifically, on GIST, for all tested values of $N_{ef}$, the accuracy loss of \texttt{HNSW+} and \texttt{HNSW++} (compared with \texttt{HNSW}) is no more than 0.14\% and that of \texttt{IVF+} and \texttt{IVF++} is no more than 0.1\%. At the same time,
\texttt{HNSW++} saves from 39.4\% to 75.3\% of the total dimensions, 
\texttt{HNSW+} saves from 34.5\% to 39.4\% and 
\texttt{IVF+}/\texttt{IVF++} save from 76.5\% to 89.2\%. {\JIANYANGREVISION The baseline method \texttt{HNSW}* saves from 10.9\% to 15.7\%, which explains its minor improvement on \texttt{HNSW}. \texttt{IVF}*/\texttt{IVF}** saves from 28.9\% to 40.4\%.} 
}

\begin{figure}[thb]
  \centering 
  % \vspace{-4mm}
  % \includesvg[width=\linewidth]{revision experimental result/decomposition_merge.svg}
  \includegraphics[width=\linewidth]{revision experimental result/decomposition_merge.pdf}
  \vspace{-8mm}
  \caption{{\JIANYANGREVISION Decomposition of Time Cost (At a particular {\chengr recall}, the three horizontal bars from top to bottom, represent the \texttt{AKNN++}, \texttt{AKNN+} and \texttt{AKNN} algorithms, {\chengr respectively}. The time cost is normalized by the cost of the original \texttt{AKNN} algorithms).}}
  \vspace{-4mm}
  \label{fig:decomposition_simple}
\end{figure}

\smallskip
\noindent
% \textbf{{\CHENG Results of \texttt{HNSW} Algorithms}.}
\textbf{{\JIANYANG \texttt{HNSW+} v.s. \texttt{HNSW++}}.}
{\JIANYANG
We further compare \texttt{HNSW+} and \texttt{HNSW++}. According to Figure~\ref{figure:evaluated_dimension HNSW}, we have the following observations.
(1) \texttt{HNSW++} evaluates \emph{fewer dimensions} than \texttt{HNSW+}, 
% {\CHENG which is well aligned with our theoretical results in Theorem~\ref{theorem:time-accuracy of ADSampling} (recall that the ratio $\alpha$ in \texttt{HNSW++} is larger than that in \texttt{HNSW+}.}
% which verifies our claim that \texttt{HNSW++} improves \texttt{HNSW+} by reducing more dimensions. 
{\CHENG which largely explains the result that \texttt{HNSW++} runs faster than \texttt{HNSW+}.}
(2) \texttt{HNSW++} reaches \emph{nearly the same recall} as \texttt{HNSW+}, which empirically shows that using approximate distances for graph routing has nearly the same effectiveness as using exact distances. 
% (3) \texttt{HNSW++} brings more improvement in the case of $K=20$ than $K=100$. This is because in \texttt{HNSW++}, a DCO is conducted between a candidate and the current $K^{th}$ NN. A smaller $K$ implies a larger distance gap $\alpha$ and consequently {\CHENG fewer} needed dimensions {\CHENG (according to Theorem~\ref{theorem:time-accuracy of ADSampling})}.
% (4) For \texttt{HNSW+}, there is no difference between $K=20$ and $K=100$. This is because in \texttt{HNSW+}, DCO is conducted between a candidate and the current $N_{ef}^{th}$ NN, which is independent of $K$.
(3) The evaluated dimensions (its ratio over those of \texttt{HNSW} in percentage) of \texttt{HNSW+} increases wrt $N_{ef}$ while those of \texttt{HNSW++} decreases wrt $N_{ef}$. This is because \texttt{HNSW+} conducts DCOs with the $N_{ef}^{th}$ NN's distance as the threshold, whose distance increases wrt $N_{ef}$, and thus a larger $N_{ef}$ leads to smaller distance gap $\alpha$, which entails more dimensions for a reliable DCO. 
For \texttt{HNSW++},
% conducts DCOs with the $K^{th}$ NN's distance as the threshold, which is unchanged wrt $N_{ef}$. At the same time, 
as mentioned in \ref{subsec:main result}, when targeting high recall, an AKNN algorithm inevitably generates many low-quality candidates with larger $\alpha$'s, and thus, it needs fewer dimensions for DCOs.
}
%{\CHENG ***The description here does not look sorted to me. Needs to be revised.***}Let's focus on ARSearch and ARRoute on \texttt{HNSW}. In terms of ARRoute (green), by increasing $N_{ef}$, we're searching for the very few remaining KNNs with visiting a lot of candidates, most of which can be far away from the query, and as a result, have large space for dimension reduction. The phenomenon that the greed curve decreases monotonically with $N_{ef}$ matches our expectation perfectly. Also, since DCO is conducted between a new candidate and $dis_{i_K}$, a smaller $K$ implies more space for dimension reduction. As a result, when K=20, it brings more benefits than the case of K=100. However, in terms of adaptive resolution search (yellow), as it aims to preserve the full search path of greedy beam search, its performance depends only on $N_{ef}$ rather than $K$, which causes the yellow lines of K=20 and K=100 are exactly identical. Plus, as $N_{ef}$ increases, the radius of DCO, i.e. the distance of $N_{ef}$NN, grows respectively. Thus, even though for large $N_{ef}$ we're visiting a large number of low-quality candidates, the space for dimension reduction decreases.

\smallskip
\noindent
% \textbf{{\CHENG Results of \texttt{IVF} Algorithms}.}
\textbf{{\JIANYANG \texttt{IVF+} v.s. \texttt{IVF++}}.} \texttt{IVF+} and \texttt{IVF++} differ only in data layout, and thus they have exactly the same accuracy and evaluated dimensions. 
%We exhibit the results of ARSearch and ARScan (green) on \texttt{IVF}\footnote{Note that these two methods differ only in data layout, so their evaluated dimensions and recall should be exactly the same}. We also show another curve, the one where we provide ground truth $dis_{i_K^*}$ as the distance threshold. Note that in KNN query, such threshold is obviously the minimum and safe threshold for maintaining KNN set (black). Based on Figure~\ref{figure:evaluated_dimension IVF}, we verified two statements. \underline{First}, ARSearch and ARScan indeed reduce a lot of dimensions (up to 89.95\% on GIST) without harming accuracy for \texttt{IVF}. \underline{Second}, there is little space for reducing more dimensions because our ARSearch and ARScan reduce almost as many dimensions as the case of providing the minimum ground truth radius.

%Figure~\ref{figure:evaluated_dimension IVF} shows that (1) .

%we also compare the results of ARSearch/ARScan on \texttt{IVF}. We can see their curves highly overlap with each other, indicating that the evaluated dimensions are very close, {\color{red} it's currently contradictory to our expectation.}

{\JIANYANGREVISION
\subsubsection{\textbf{Results of Time Cost Decomposition}}
\label{subsec: time decompostion}
We note that applying \texttt{ADSampling} entails the extra cost of randomly transforming the data and query vectors. In particular, the cost of transforming the data vectors lies in the index phase and can be amortized by all the subsequent queries on the same database. The transformation of the query vectors is conducted during the query phase when a query comes and its cost can be amortized by all the DCOs involved for answering the same query. 
% This step takes $O(D^2)$ 
{\chengr We implement this step {\JIANYANGREVISION (a.k.a, Johnson-Lindenstrauss Transformation~\cite{johnson1984extensions, JL_survey})} as a matrix multiplication operation for simplicity, which takes $O(D^2)$ time. We note that this step can be performed in less time with advanced algorithms~\cite{JL_survey}, e.g., it takes $O(D\log D)$ time with Kac's Walk~\cite{kac_walk}.}
% it takes  with a straightforward method (i.e., matrix multiplication) and }. 
We show the results of time cost decomposition on the dataset GIST. It has the highest dimensionality and correspondingly the largest overhead for {\chengr random transformation}. 
% (TODO, discuss on the time decomposition figure and describe it)
We decompose the time cost in 
% Version 1
% Figure~\ref{fig:decomposition}.
% Version 2
Figure~\ref{fig:decomposition_simple}.
% At a particular accuracy, the three horizontal bars from top to bottom, represent the \texttt{AKNN++}, \texttt{AKNN+} and \texttt{AKNN} algorithms correspondingly. The time cost is normalized by the cost of the original \texttt{AKNN} algorithms.
% Version 1
% We note that the cost of randomization for \texttt{HNSW+}/\texttt{HNSW++} takes at most 5.42\% of the total cost of the original \texttt{HNSW}. As the requirement of accuracy increases, its cost takes smaller proportion (e.g., 95\% recall, it reduces to 1.83\%). For \texttt{IVF+}, its cost is no greater than 1.00\%.
% Version 2
We note that the cost of {\chengr random transformation} for \texttt{HNSW+}/\texttt{HNSW++} takes at most 6.18\% of the total cost of the original \texttt{HNSW}. As the accuracy increases, the percentage decreases (e.g., for 95\% recall, it reduces to 2.02\%). For \texttt{IVF+}, the percentage is no greater than 1.11\%.

% We note that our study initiates \texttt{ADSampling} with the very basic algorithm for randomization, and its performance is good enough. There are also plentiful of advanced algorithms for fast randomization (e.g., Kac's Walk which runs in $O(D\log D)$~\cite{kac_walk}), which can further reduce the overhead. We refer readers to a recent comprehensive survey~\cite{JL_survey}. 
}

% {\JIANYANGB

\subsubsection{\textbf{Results for {\CHENGB Evaluating} the Feasibility {\CHENGB of Distance Approximation Techniques} for Reliable DCOs}}
\label{subsubsec:reliable-dco}
\begin{figure}[thb]
  \centering 
  % \vspace{-4mm}
%   \includesvg[width=\linewidth]{revision experimental result/acc_clear.svg}
   % \includesvg[width=\linewidth]{revision experimental result/acc.svg}
   \includegraphics[width=\linewidth]{revision experimental result/acc.pdf}
  % \includesvg[width=\linewidth]{revision experimental result/feasibility.svg}
  \vspace{-8mm}
  \caption{{\JIANYANGREVISION Feasibility for Reliable DCOs (Recall).}}
  \vspace{-4mm}
  \label{figure:feasibility}
\end{figure}

\begin{figure}[thb]
    \centering
    % \vspace{-1mm}
    % \includesvg[width=\linewidth]{revision experimental result/qps_clear.svg}
    % \includesvg[width=\linewidth]{revision experimental result/qps.svg}
    \includegraphics[width=\linewidth]{revision experimental result/qps.pdf}
    % \includesvg[width=\linewidth]{revision experimental result/qps_linear.svg}
    \vspace{-8mm}
    \caption{{\JIANYANGREVISION Feasibility for Reliable DCOs (QPS).}}
    \vspace{-4mm}
    \label{fig:qps linear}
\end{figure}


We next study the feasibility of 
% {\CHENGB the distance approximation} methods mentioned in Section~\ref{subsec: experimental setup}, 
{\CHENGB two distance approximation methods, including random projection and product quantization~\cite{jegou2010product} {\JIANYANGREVISION (with the typical setting of 256 centroids per partition~\cite{jegou2010product, learningtohashsurvey})},} for reliable DCOs. {\CHENGB We include the results of \texttt{ADSampling}, {\JIANYANGREVISION \texttt{PDScanning}} and \texttt{FDScanning} for comparison.} To test the best possible recall a method can reach, we conduct this experiment with an exact KNN algorithm, namely linear scan. Specifically, for random projection and product quantization, we scan all the data objects and return the K objects with the minimum approximate distances. For \texttt{ADSampling} and {\JIANYANGREVISION \texttt{PDScanning}}, like \texttt{IVF}, we maintain a KNN set and conduct DCOs for each object sequentially. We plot the recall-number of dimensions/lookups
~\footnote{
For product quantization, it refers to the quantization code size,
% {\JIANYANGREVISION (the number of partitions)} 
where evaluating each code would look up a table in memory (i.e., access memory randomly). For other methods, it refers to the number of dimensions, where evaluating each dimension applies some arithmetic computations. Note that they are not directly comparable because in modern CPUs, the former is much slower than the latter. 

% {\JIANYANGREVISION This issue is also acknowledged by the authors of the classic PQ~\cite{jegou2010product} in the Faiss open-source community~\url{https://github.com/facebookresearch/faiss/issues/148}.}
% We emphasize that the number of lookups of product quantization is not directly comparable with the dimensionality of other methods because evaluating a quantization code entails looking up a table in memory (i.e., access memory randomly) while evaluating a dimension applies arithmetic computations. In modern CPUs, the former is much slower than the latter. 
% In the extreme case that the code size is 50\% of the dimensionality, it runs even slower than \texttt{FDScanning}.
} 
curves in Figure~\ref{figure:feasibility}. For \emph{random projection}, we vary the dimensionality of the projected vectors and observe that (1) it introduces 3.71\% accuracy loss {\CHENGB while} reducing only 1.04\% dimensions (2) {\CHENGB when} reducing half of the dimensionality, its recall is no more than 70\%. For \emph{product quantization}, we vary its quantization code size {(i.e., the number of partitions)} and observe that 
in the best possible case
{\JIANYANGREVISION (i.e., the case of encoding every two dimensions with one code)},
% ~\footnote{That is the case when the code size is 50\% of the full dimensionality (encoding two dimensions with one code). We emphasize that in this case, it runs even slower than \texttt{FDScanning} {\JIANYANGREVISION as shown in Figure~\ref{fig:qps linear}}.}
it still introduces 6.2\% accuracy loss. 
{\CHENGC Therefore, neither product quantization nor random projection can achieve reliable DCOs with remarkably better efficiency than \texttt{FDScanning}.}
% its best possible recall is 93.842\% {\JIANYANG (i.e., one code for two dimensions, with this size, it runs even slower than \texttt{FDScanning} because evaluating quantization codes is slow)}, which introduces unignorable loss. Therefore, the distance approximation methods can hardly achieve reliable DCOs. 

For \texttt{ADSampling}, we test two settings $\Delta_d=1$ and $\Delta_d=32$. The former represents the best possible recall-dimension tradeoff of our method and the latter represents a practical setting with less frequent hypothesis testing (i.e., our default setting). We plot their curves by varying $\epsilon_0$ from 0.0 to 4.0. We observe that for $\Delta_d=1$, it samples 6.61\% of the total dimensions {\CHENGB while reaching} >99.9\% recall and for $\Delta_d=32$, it samples 7.11\% of the total dimensions {\CHENGB while reaching} > 99.9\% recall. Thus, \texttt{ADSampling} achieves much better recall-dimension tradeoff than \texttt{FDScanning}. 

\begin{figure}[thb]
  \centering 
  % \vspace{-4mm}
  % \includesvg[width=\linewidth]{experimental result/epsilon.svg}
  \includegraphics[width=\linewidth]{experimental result/epsilon.pdf}
  \vspace{-8mm}
  \caption{Parameter Study on $\epsilon_0$ of \texttt{AKNN++} Algorithms.}
  \vspace{-4mm}
  \label{figure:parameter}
\end{figure}


\begin{figure}[thb]
  \centering 
  % \vspace{-1mm}
  % \includesvg[width=\linewidth]{experimental result/gap.svg}
  \includegraphics[width=\linewidth]{experimental result/gap.pdf}
  \vspace{-8mm}
  \caption{Parameter Study on the $\Delta_d$ of \texttt{AKNN++} Algorithms.}
  \vspace{-4mm}
  \label{figure:parameter gap}
\end{figure}

\begin{figure}[thb]
  \centering 
  % \vspace{-1mm}
  % \includesvg[width=\linewidth]{experimental result/verification.svg}
  \includegraphics[width=\linewidth]{experimental result/verification.pdf}
  \vspace{-8mm}
  \caption{Verification for Theoretical Analysis.}
  \vspace{-4mm}
  \label{figure:verification}
\end{figure}

{\JIANYANGREVISION 
In Figure~\ref{fig:qps linear}, we plot the QPS-dimensions/lookups curves. We have the following observations. 
% (1) When {\chengf product quantization} achieves its best possible accuracy (still 6.2\% recall loss), its efficiency is worse than \texttt{FDScanning}.
% We can further improve the cache-friendliness of \texttt{ADSampling} by re-organizing the data layout like \texttt{IVF++} (the result is not included). 
(1) {\chengr \texttt{ADSampling} (with default setting), which is marked with a green cross within a green circle, has the QPS significantly higher than \texttt{FDScanning} and \texttt{PDScanning}. This is because it exploits only 7.11\% of the total dimensions while achieving a recall over $99.9\%$.} 
% has the recall over $99.9\%$. We note that it exploits only 7.11\% of the total dimensions of \texttt{FDScanning}, while its improvement on the QPS is less than 10x. This is also due to the issue of cache. 
(2) At the same dimensionality, random projection has its efficiency better than \texttt{ADSampling}. This is because random projection has fixed dimensionality and can organize the projected vectors sequentially in an array to achieve better cache-friendliness. However, we note that when random projection has the same QPS as \texttt{ADSampling} (default), its recall does not exceed 40\%. 
% At the same dimensionality, \texttt{ADSampling} has its efficiency worse than random projection. This is because random projection sequentially organizes the projected vectors in an array and thus, is more cache-friendly. {\chengr We emphasize that random projection cannot perform reliable DCOs as indicated by the results in Figure~\ref{figure:feasibility}.}

% In Figure~\ref{fig:qps linear}, we plot the QPS-dimensions/lookups curve for the methods which achieve reliable DCOs including \texttt{FDScanning}, \texttt{PDScanning} and \texttt{ADSampling} ($\Delta_d=32,\epsilon_0=2.1$). We also include the results of product quantization~\cite{jegou2010product}. It shows that (1) when PQ achieves its best possible accuracy (still 6.2\% accuracy loss), its efficiency is worse than \texttt{FDScanning} and (2) though \texttt{ADSampling} exploits only 7.11\% of the total dimensions of \texttt{FDScanning}, its improvement on the QPS is less than 10x.
% We observe in Figure~\ref{figure:feasibility} that 
% We verify the fact that distance approximation techniques including random projection~\cite{johnson1984extensions} and product quantization~\cite{jegou2010product} cannot achieve 
}

\subsubsection{\textbf{{\CHENG Results of} Parameter Study}}
\label{subsub:parameter}

% \begin{figure}[ht]
%   \centering 
%   \includesvg[width=\linewidth]{experimental result/verification_merge.svg}
%   \caption{Verification for Time-Accuracy Tradeoff.}
%   \label{figure:verification}
% \end{figure}

% \begin{table}[h]
\small
\centering
\caption{
Statistics of \NAME v1.0.
}
% \begin{tabular}{P{1.7cm}P{1.2cm}P{1.2cm}P{1.0cm}P{1.2cm}P{1.2cm}P{1.2cm}P{1.1cm}P{1.1cm}P{1.1cm}}
\begin{tabular}{lr}
\toprule[1.2pt]
Statistics \\
\midrule
\# of prompts & 4,081 \\
$\ $ - \# of COCO captions & 2,000 \\
$\ $ - \# of DrawBench, PartiPrompt, PaintSkill prompts & 2,081 \\
\midrule
\# of questions & 25,829 \\
$\ $ - \# of binary questions & 17,226 \\
$\ $ - \# of multiple-choice questions & 8,603 \\
\midrule
avg. \# of questions per prompt & 6.3 \\
avg. \# of words per prompt & 10.5 \\
avg. \# of elements per prompt & 4.3 \\
\bottomrule[1.2pt]
\end{tabular}
\label{tab:statistics}
% \bottomrule
\vspace{-3mm}
\end{table}

{\JIANYANG
% We then conduct parameter study on \texttt{AKNN+} and \texttt{AKNN++} algorithms. Note that the targets of \texttt{AKNN+} and \texttt{AKNN++} are different. \texttt{AKNN+} aims to preserve the results of {\CHENG a general} AKNN algorithm while \texttt{AKNN++} targets better efficiency. Thus, we design two different types of experiments for them. In particular, for \texttt{AKNN+}, we focus on verifying our theoretical analysis, which is discussed later in Section~\ref{subsubsec:theoretical results}, while for \texttt{AKNN++} we provide empirical advice on parameter tuning. 
%

{\CHENG Parameter $\epsilon_0$ is a critical parameter for the \texttt{ADSampling} algorithm since it directly controls the trade-off between the accuracy and the efficiency (recall that a larger $\epsilon_0$ means a smaller significance value for the hypothesis testing, which further implies a more accurate result of the hypothesis testing).}
Figure~\ref{figure:parameter} plots the QPS-recall curves of \texttt{HNSW++} (left panel) and \texttt{IVF++} (right panel) with different $\epsilon_0$. 
%
In general, we {\CHENGC observe} from the figures that with {\CHENG a larger} $\epsilon_0$, the QPS-recall curves moves lower right. This is because {\CHENG a larger} $\epsilon_0$ leads to better accuracy at the cost of efficiency.
% When $\epsilon_0$ is small 
% (e.g., $\epsilon_0=1.5, 1.8$), 
% decreasing $\epsilon_0$ would introduce unignorable accuracy loss
% and increase the efficiency. However, when it is large 
% (e.g., $\epsilon_0=2.7, 3.0$), increasing $\epsilon_0$ does not improve accuracy {\CHENG further}
% ~\footnote{{\JIANYANGB Note that $\epsilon_0$ can only control the probability that \texttt{ADSampling} successfully finds out KNN objects from the generated candidates.}}
% {\CHENG but would} decrease efficiency. 
% Thus, we suggest to set $\epsilon_0$ small when targeting low-accuracy region and to tune it around $2.1$ when targeting improving efficiency without losing much accuracy.
{\JIANYANGB We observe that when $\epsilon_0=2.1$, it introduces little accuracy loss while further increasing $\epsilon_0$ would decrease the efficiency. Thus, in order to improve the efficiency without losing much accuracy, we suggest to set $\epsilon_0$ around $2.1$.}
% nearly no extra accuracy loss. 
%
The results for \texttt{HNSW+} and \texttt{IVF+} are similar and omitted due to the page limit.

% Note that the recall 
% %(the proportion of retrieved ground truth KNNs) 
% is affected by two factors: \texttt{AKNN} algorithms and \texttt{ADSampling}, where $\epsilon_0$ controls only the second part.
% Another interesting phenomenon is about the movement of the curve when increasing $\epsilon_0$. Basically, we see from the figures that with increasing $\epsilon_0$, the QPS-recall curves in general moves lower right. This is because increasing $\epsilon_0$ leads to better accuracy at the cost of efficiency. What is interesting is that 
% {\CHENG We also observe that} when $\epsilon_0$ is small, increasing it mainly causes horizontal movement (e.g., from 1.5 (red) to 1.8 (orange)) while when it's large, the curve almost moves vertically (e.g., from 2.7 (violet) to 3.0 (purple)). Note that in both cases the increment is 0.5. This phenomenon can be clearly explained by our theoretical analysis of Lemma~\ref{theorem:ADSampling accuracy} and \ref{theorem:ADSampling efficiency}. Specifically, on one hand, with respect to $\epsilon_0$, the failure probability decays quadratic-exponentially (Lemma~\ref{theorem:ADSampling accuracy}), implying that it decays fast for small $\epsilon_0$ and slow for large $\epsilon_0$. On the other hand, the expected time complexity grows quadratically (Lemma~\ref{theorem:ADSampling efficiency}), implying that it grows slow for small $\epsilon_0$ and fast for large $\epsilon_0$. 
% As a result, when $\epsilon_0$ is small, the curve moves mainly horizontally while when it's large, it mainly moves vertically, which matches the experimental results well. 

Figure~\ref{figure:parameter gap} plots the QPS-recall curves of \texttt{HNSW++} and \texttt{IVF++} with different $\Delta_d$. 
We observe that too frequent hypothesis testing (e.g., when $\Delta_d=1$) would do harm to the performance. 
It's worth noting that a small $\Delta_d$ implies that it can terminate sampling immediately when enough information is collected, {\CHENGB but it would} require more arithmetic operations for hypothesis testing. Our empirical study shows that when $\Delta_d=16, 32, 64$, it achieves the best {\CHENGC trade-off.}
}
%Moreover, though it's discussed that increasing $\epsilon_0$ leads to better accuracy at the cost of efficiency (as a result, the curve moves towards lower right), the extent of movement on each axis is unclear. 
%An interesting phenomenon in Figure~\ref{figure:parameter} is that when $\epsilon_0$ is small, increasing it mainly causes horizontal movement, e.g. from 1.0 (red) to 1.5 (orange), while when it's large, the curve almost moves vertically, e.g. from 2.5 (blue) to 3.0 (purple). Note that in both cases the increments are 0.5. This phenomenon can be perfectly explained by our theoretical analysis in Section 5. Specifically, on the one hand, with respect to $\epsilon_0$, the failure probability decays super-exponentially (Equation~\ref{eq:superexp}), implying that it decays fast for for small $\epsilon_0$ and slow for large $\epsilon_0$. On the other hand, the expected time complexity grows quadratically (Equation~\ref{eq:quadratic}), implying that it grows slow for small $\epsilon_0$ and fast for large $\epsilon_0$. As a result, when $\epsilon_0$ is small, horizontal movement dominates the behavior of the curve while when it's large, it mainly moves vertically, which matches the experimental results well. 

%when $\epsilon_0$ is small, the \texttt{AKNN++} algorithms themselves introduce unignorable accuracy loss. However, at the same time, they also increase the efficiency at low-accuracy region, suggesting that when targeting low-accuracy region, $\epsilon_0$ 

%from the figures that when $\epsilon_0$ is small, the adaptive algorithms themselves introduce unignorable accuracy loss (about 5\% for $\epsilon_0=1.0$ and $1\%$ for $\epsilon_0=1.5$). However, they also increase the efficiency at low-accuracy region, suggesting that when targeting low-accuracy region, $\epsilon_0$ should be tuned smaller, while targeting nearly no extra accuracy loss, $\epsilon_0$ is suggested to be set around $2.0$. 

%: (1) whether an AKNN algorithm generates 

%{\JIANYANG Note that the targets of our adaptive algorithms are different. The generic ARSearch aims to preserve the semantics of an arbitrary AKNN algorithm, while the specific ARRoute and ARScan are proposed for better efficiency. We next do parameter study on ARRoute (on \texttt{HNSW}) and ARScan (on \texttt{IVF}) to provide empirical advice on parameter tuning and ARSearch (on linear scan 
%with providing ground truth $dis_{i_K^*}$ to eliminate the error caused by AKNN algorithms
%) to verify our theoretical analysis on DCO (Lemma~\ref{theorem:ADSampling accuracy} and \ref{theorem:ADSampling efficiency}). } 
%Figure~\ref{figure:parameter} plots the QPS-recall curves with different $\epsilon_0$ of \texttt{HNSW} + ARRoute (left panel) and \texttt{IVF} + ARScan (right panel). Note that the recall is controlled by two factors: 1) whether AKNN algorithms visit the ground truth KNNs 2) whether it's successfully leveled up to level $L$ and updates the KNN set $\mathcal{K}$, where $\epsilon_0$ controls only the second part. 
%We see from the figures that when $\epsilon_0$ is small, the adaptive algorithms themselves introduce unignorable accuracy loss (about 5\% for $\epsilon_0=1.0$ and $1\%$ for $\epsilon_0=1.5$). However, they also increase the efficiency at low-accuracy region, suggesting that when targeting low-accuracy region, $\epsilon_0$ should be tuned smaller, while targeting nearly no extra accuracy loss, $\epsilon_0$ is suggested to be set around $2.0$. 


{\JIANYANG
\smallskip
\noindent
\subsubsection{\textbf{{\CHENG Results for Verifying Theoretical Results}}}
\label{subsubsec:theoretical results}

We further empirically verify Lemma~\ref{theorem:ADSampling accuracy} and ~\ref{theorem:ADSampling efficiency}. 
% {\JIANYANGB Because this study is conducted for verifying theoretical analysis, we .}
As a verification study, the experimental setting is different. 
To eliminate the accuracy loss caused by AKNN algorithms, we conduct the verification study based on linear scan, which itself is an exact KNN algorithm.
{\JIANYANGB  Note that in KNN query processing, the result of the former DCOs can affect the distance thresholds of the latter DCOs, which introduces some bias into a verification study. To eliminate it, we provide a fixed distance threshold (the exact distance of the ground truth Kth NN) to make the DCOs independent with each other.}
% To focus on the problem of DCO, we also provide a fixed distance threshold (the exact distance of the ground truth Kth NN). 
To test the actual needed dimensionality, we set $\Delta_d=1$.
% do hypothesis testing after sampling every single dimension
{\JIANYANGB With this setting, the recall represents the proportion of successful DCOs for positive objects (recall that those for negative objects will never fail), and thus, it empirically reflects the success probability of a single {\CHENGC DCO with} \texttt{ADSampling}.}
The left panel of Figure~\ref{figure:verification} shows that the failure probability indeed decays following a quadratic-exponential trend and reaches near 100\% accuracy around $\epsilon_0=2$. The right panel shows that {\CHENG the number of evaluated dimensions} increases following a quadratic trend, which is slow when $\epsilon_0$ is small (when $\epsilon_0=2$, the total {\CHENG number of} evaluated dimensions {\CHENG is} around $3\%$ of {\CHENG that of} the plain \texttt{FDScanning}). 
% Note that this result is not directly comparable with Figure~\ref{figure:evaluated_dimension} due to their different experimental settings. 

%Also, note that in terms of KNN query, in our framework, the order of evaluation also affect time complexity. Let's investigate two extreme cases: (1) Candidates are given in decreasing order with respect to distance. Then all the comparisons will produce positive results. Thus, it brings no speedup. (2) Candidates are given in increasing order with respect to distance. Then we can get the ground truth $dis_{i_K}$ very soon, meaning that for the following comparisons, we are comparing their distance with the minimum possible threshold, which reaches the optimal complexity. 
}
%Recall that $\epsilon_0$ is the parameter that controls the failure probability and expected time complexity of our adaptive algorithms. Thus, increasing $\epsilon_0$ leads to higher accuracy and lower QPS (moving towards bottom right), {\color{red} which is verified by the experimental results}. 
%When $\epsilon_0$ is small ($\epsilon_0=1.0, 1.5$), the figures show that .
%Note that according to the proof of Theorem~\ref{theorem:eps} and \ref{theorem:efficiency} in Section 4, with respect to $\epsilon_0$, the failure probability decays super-exponentially and the time complexity grows quadratically, indicating that when $\epsilon_0$ is small, slightly increasing it largely reduces the error while 

\section{Related Work}
\label{sec:related work}

\noindent
\textbf{Approximate K Nearest Neighbor Search.} 
Existing AKNN algorithms can be categorized into four types: 
% {\JIANYANGREVISION tree-based ~\cite{muja2014scalable, dasgupta2008random, ram2019revisiting, beygelzimer2006cover, reviewer_M_tree}, graph-based~\cite{malkov2018efficient, NSW, li2019approximate, fu2019fast, fu2021high, SISAP_graph}}, quantization-based~\cite{jegou2010product, ge2013optimized, guo2020accelerating, ITQ, additivePQ, imi} and hashing-based~\cite{indyk1998approximate, datar2004locality, c2lsh, tao2010efficient, huang2015query, sun2014srs, lu2020vhp, zheng2020pm} algorithms. 
{\JIANYANGREVISION (1) graph-based~\cite{malkov2018efficient, NSW, li2019approximate, fu2019fast, fu2021high, SISAP_graph}}, (2) quantization-based~\cite{jegou2010product, ge2013optimized, guo2020accelerating, ITQ, additivePQ, imi}, (3) {\JIANYANGREVISION tree-based ~\cite{muja2014scalable, dasgupta2008random, ram2019revisiting, beygelzimer2006cover, reviewer_M_tree}} and (4) hashing-based~\cite{indyk1998approximate, datar2004locality, c2lsh, tao2010efficient, huang2015query, sun2014srs, lu2020vhp, zheng2020pm, james_cheng}.
In particular, graph-based methods show superior performance for in-memory AKNN query. 
Quantization-based methods are powerful when memory is limited. Hashing-based methods provide rigorous theoretical guarantee. 
We refer readers to recent tutorials~\cite{tutorialThemis, tutorialXiao}, {\JIANYANGREVISION reviews and benchmarks~\cite{li2019approximate, annbenchmark, SISAP_benchmark, reviewer_paper, graphbenchmark}} for details. 
{\JIANYANG
There are also plentiful {\CHENG studies, which apply}
% works `
machine learning (ML) to accelerate AKNN~\cite{learning2route2019ml, reinforcement2route, adaptive2020ml, Dong2020Learning}. \cite{learning2route2019ml, reinforcement2route} apply reinforcement learning in graph routing which substitutes greedy beam search. \cite{adaptive2020ml} learns to early terminate searching. \cite{Dong2020Learning} uses ML to construct an index structure. Note that all above methods apply ML for candidate generation. 
{\CHENG These ML-based methods are orthogonal to our techniques} and our techniques can {\JIANYANGB help them with} 
% enhance these methods {\CHENG in their re-ranking phase for}
finding KNNs among the generated candidates.
}

% \smallskip
% \noindent
% \textbf{Distance Approximation.} There are two main forms of distance approximation that have been used for AKNN: (1) product quantization~\cite{ge2013optimized, jegou2010product,ITQ, additivePQ, guo2020accelerating} and (2) dimension reduction, in which dimension reduction can be further categorized into optimization-based dimension reduction~\cite{wold1987principal, kruskal1964multidimensional} and random projection~\cite{johnson1984extensions, blockjlt, fftjlt}. The above methods can produce approximate distances, but they cannot help with DCO. Specifically, product quantization and optimization-based dimension reduction optimize a compressed representation to minimize the total approximation error instead of the maximum one, which fails to guarantee an error bound. Though random projection provides probabilistic guarantee on the error bound, existing methods only support a fixed resolution. Thus, for the DCOs with large distance gap, it can be redundant, while for those with small distance gap, it might not be enough. It's worth noting that hashing-based methods~\cite{indyk1998approximate, datar2004locality, c2lsh, tao2010efficient, huang2015query, lu2020vhp} are also popular in AKNN for data compression. They target to map close vectors into similar hashing code and use code comparison as a proxy of DCO. However, it does not explicitly approximate distances and thus cannot help with DCO. Consequently, since 
% the methods above cannot help with DCO, in AKNN query, they can be used only in the first stage (i.e., generating candidates), but not the second (finding out KNNs among generated candidates).

\smallskip
\noindent
\textbf{Random Projection {\CHENGB for AKNN}.} 
% The seminal work of random projection is the Johnson-Lindenstrauss Lemma~\cite{johnson1984extensions}.
% % , {\CHENGB which shows that the difference between the distance random projection provides probabilistic guarantee on the error bound}. 
% % In a recent work~\cite{larsen2017optimality}, it was proved to be theoretically optimal. 
% A series of studies~\cite{larsen2017optimality,optimalJL, optimalJL2} prove its optimality in theory. 
% For AKNN query, besides dimension reduction, 
{\CHENGB While random projection can hardly be used for reliable DCOs during the phase of re-ranking candidates of KNNs as explained and verified earlier, it has been} widely applied in LSH~\cite{indyk1998approximate, datar2004locality, c2lsh, tao2010efficient, sun2014srs, huang2015query, lu2020vhp} and random partition/projection tree~\cite{ram2019revisiting, dasgupta2008random} {\CHENGB during  the phase of generating candidates of} KNNs. 
{\CHENGB Our study differs from these studies in (1) we project different objects to vectors with different dimensions flexibly while these studies project all objects to vectors with equal dimensions; (2) we set the number of dimensions of a projected vector for an object automatically based on its DCO via hypothesis testing while these studies need to set the number with manual efforts; and (3) we use the projected vectors (in DCOs) during the phase of 
% re-ranking 
{\JIANYANGB finding out KNNs from the generated}
candidates while these studies use the projected vectors during the phase of generating candidates. Therefore, these studies are orthogonal to our study.}
% Most of these methods guarantee to return $(\delta,c)$-approximate AKNNs (i.e., with at most $\delta$ failure probability, the distances of the returned objects are at most $c$ times larger than those of the ground-truth KNNs), where $c>1$ {\CHENGB holds} strictly. Note that since $c > 1$ {\CHENGB holds} strictly, these methods have no guarantee to return the ground-truth KNNs. The only exception is the SRS-2 algorithm in \cite{sun2014srs}, {\CHENGB which} can make $c=1$, but does not guarantee to have better time complexity than the brute force linear scan. {\CHENGB In contrast, our \texttt{ADSampling} technique}, when combined with an exact KNN algorithm, would return $(\delta,1)$-approximate algorithm,~\footnote{Our methods also support $(\delta,c)$-approximate query by trivially replacing $(1+\epsilon_0/\sqrt {d} )$ with $(1+\epsilon_0/\sqrt {d} )/c$ in hypothesis testing. However, when recall is the main concern, we don't suggest to do so because such $c$-approximation brings slight acceleration but disastrously decreases recall.} which has its time complexity significantly better than the brute force linear scan (Corollary~\ref{corollary: find out KNNs}). 

{\JIANYANGREVISION 
\smallskip
\noindent
\textbf{Dimension Sampling {\CHENGB for AKNN}.} 
We notice that a MAB (multi-armed bandit)-oriented approach~\cite{dimension_sampling_aistat} also applies dimension sampling and claims logarithmic complexity. Our study is different from \cite{dimension_sampling_aistat} in the following aspects. Problem-wise, \cite{dimension_sampling_aistat} targets the AKNN problem itself and aims to find a superset of the set containing the KNNs. It is non-trivial to adapt \cite{dimension_sampling_aistat} to DCOs (the focus of our paper). Theory-wise, \cite{dimension_sampling_aistat}'s logarithmic complexity relies on some strong assumptions on the data (which may not hold in practice) while ours relies on no assumptions. Technique-wise, (1) \cite{dimension_sampling_aistat} samples the original vectors directly while ours first randomly transforms the vectors and then samples the transformed vectors. Our way has the advantage that the error bound of an approximate distance is based on the concentration inequality of random projection and does not rely on any assumptions as \cite{dimension_sampling_aistat} does; and (2) \cite{dimension_sampling_aistat} uses some lower/upper bounds to determine the number of sampled dimensions while ours uses sequential hypothesis testing. Our way has no false positives while \cite{dimension_sampling_aistat} has both false positives and false negatives. In summary, \cite{dimension_sampling_aistat} and our work only share a high-level idea of dimension sampling and differ in many aspects including problem, theory and technique.
% , which we shall discuss in a revision.
}
% Many other works~\cite{fftjlt, blockjlt} target to construct structured random matrices to accelerate Johnson-Lindenstrauss transformation. 
% Random projection was also utilized in nearest neighbor search in the scheme of dimension reduction~\cite{fftjlt}, Locality Sensitive Hashing(LSH)~\cite{datar2004locality, c2lsh} and random partition/projection tree~\cite{ram2019revisiting, dasgupta2008random}. 

%With the necessity of guaranteed error bound, we revisit random projection due to its plentiful theoretical results~\cite{johnson1984extensions,  larsen2017optimality, fftjlt, datar2004locality, blockjlt}. Random projection is a famous technique for low-distortion metric embedding. It provides theoretical guarantee on error bound in a probablistic manner. To be specific, the seminal work of random projection, Johnson-Lindenstrauss Lemma~\cite{johnson1984extensions} (JL Lemma), states that with the probability of $1-\delta$, random projection preserves mutual distance among a finite set $\mathcal X$ of $N$ objects with at most $\epsilon$ multiplicative error when reducing dimensionality to $\Theta(\epsilon^{-2} \log \frac{N}{\delta})$. Though random projection was utilized in nearest neighbor search in the scheme of dimension reduction~\cite{fftjlt, blockjlt}, Locality Sensitive Hashing(LSH)~\cite{datar2004locality, c2lsh} and random partition/projection tree~\cite{ram2019revisiting, dasgupta2008random}, existing works always evaluate a fixed-sized code during query. Also, though random projection guarantees the error bound, it has no guarantee on the result of an arbitrary distance comparison. We propose a novel scheme originated from the core lemma of JL Lemma, i.e. concentration inequality, but make it flexible, adaptive and recoverable. 
%\textbf{Random Projection}Extensive studies have been conducted on both applications and theories of random projection. Since the seminal work of Johnson-Lindenstrauss Lemma~\cite{johnson1984extensions} in 1984, 

% \textbf{Flexibility and Adaptivity} In terms of flexibility of resolution, \cite{lu2021hvs} proposed a method comprised of multi-granular quantization, which is motivated by the heuristics that initial search steps require coarser distance. It achieves flexibility of resolution but not adaptivity to a particular query. \cite{adaptive2020ml} proposed to do early termination with machine learning models, i.e. adaptively determining the number of objects to visit, which is orthogonal to our work of adaptive resolution. Sequential hypothesis testing~\cite{wald1945sequential, Berger2017sequential} is a historical topic in statistics, with which practitioners keep sampling until a significant conclusion can be obtained, making the number of samples adaptive. It is especially meaningful for the scenarios where sampling is expensive, e.g. clinical analysis~\cite{korosteleva2008clinical}. This technique was also applied in problems in computer science including network intrusion detection~\cite{network}, pattern matching~\cite{computervision} and LSH for all-pairs similarity search~\cite{satuluri2011bayesian, sequentialLSH}. Note that the main difference of sequential hypothesis testing under different scenarios is how a statistical problem is formulated. Our formulation is based on concentration inequality of random projection for Euclidean space, which is totally different from that of \cite{satuluri2011bayesian, sequentialLSH}.

%Recently, a hybrid method of graph and quantization, HVS~\cite{lu2021hvs} utilized multi-granular quantization, which is motivated by the heuristics that initial search steps require coarser distance. 

%Flexibility and adaptivity of distance resolution are seldom researched under the context of high-dimensional nearest neighbor search.  However, the list size and code size at each stage are preset hyper-parameters, so it's very limited in flexibility and cannot adapt to a particular query. \cite{lu2021hvs} proposed a method comprised of multi-granular quantization, which is motivated by the heuristics that initial search steps require coarser distance. It achieves flexibility of resolution but not adaptivity to a particular query. Last but not least, to the best of our knowledge, no existing methods can recover exact distance without evaluating raw vectors from scratch.  %\cite{adaptive2020ml} proposed to adaptively early terminate ANN search with machine learning. The resolution of its representation is fixed, so it's orthogonal to our work. 

%topology to navigate ANN search, while compression-based ones accelerate distance computation with short code generated by quantization~\cite{jegou2010product, ge2013optimized, guo2020accelerating}, hashing and dimensionality reduction. 

%Random projection is a technique widely used in nearest neighbor search~\cite{achlioptas2003database} {\color{red} cite}, kernel approximation~\cite{choromanski2017unreasonable, yu2016orthogonal} {\color{red} cite} and machine learning{\color{red} cite}. 

%Adaptive methods for KNN query - machine learning for database - SIGMOD 2020, Tao arxiv 2022, unlike these two works, our work depends on adaptation instead of prediction.

% {\JIANYANG
% \subsection{\textbf{Remarks}}
% %\subsubsection{ {\color{red} \textbf{Remarks}} } 
% %{\color{red} I think we'll need to clarify our contribution here? need discussion}

% We emphasize that ADSampling is a highly coupled framework. Some of its components were individually applied in many other AKNN algorithms, but only our method linked them up associatedly:

% \underline{Guaranteed error bound of random projection} was widely applied in LSH~\cite{datar2004locality, c2lsh, huang2015query} and random projection/partition tree~\cite{dasgupta2008random, ram2019revisiting} to guarantee the quality of returned AKNNs. Specifically, they guarantee that their returned AKNNs are $(\delta,c)$-approximate, i.e., with at most $\delta$ failure probability, the distances of the returned KNN objects are at most $c$ times larger than those of the ground-truth KNNs, where $c$ must be strictly greater than $1$ and empirically not so small~\cite{datar2004locality}. Our method, when combined with an exact KNN algorithm, instead provides the guarantee of $(\delta,1)$-approximation which cannot be achieved by the aforementioned algorithms~\footnote{Our method also support $(\delta,c)$-approximate query by trivially replacing $\gamma(d)$ with $\gamma_c(d):=\gamma(d)/c$. However, when recall is the main concern, we don't suggest to do so because such $c$-approximation brings slight acceleration while disastrously decrease recall (data not shown). }. \cite{sun2014srs} uses the guarantee for early termination at the re-ranking stage, which is similar to our hypothesis testing. Its SRS-2 algorithm also provides the $(\delta, 1)$-approximation guarantee. However, it lacks of flexibility, adaptivity and recoverability. Also it was not made a plugin for other AKNN algorithms.

% \underline{Sequential hypothesis testing} is a historic topic in statistics~\cite{wald1945sequential}. In AKNN-related problems, it was used in LSH for all-pairs similarity search under non-Euclidean metrics~\cite{sequentialLSH, satuluri2011bayesian}. Note that the main difference of sequential hypothesis testing in different scenarios is how a sampling problem is formulated. Our formulation is based on concentration inequality of random orthogonal projection of 
%$\ell_2$ norm 
% Euclidean space, where samples (dimensions) are \textbf{not} independent with each other (because the rows of random orthogonal transformation are correlated to each other), which is totally different from the independent sampling of binary hashing code in \cite{sequentialLSH, satuluri2011bayesian}.

% \underline{Random orthongoal transformation} was widely used in quantization~\cite{jegou2010product, ITQ} to balance the variance among all dimensions heuristically. However, none of these works investigate its properties quantitatively (e.g., concentration inequality), which is, however, essential in our framework.
% }

% \smallskip 
% \noindent\textbf{(2) Insufficiency of existing distance approximation methods for DCOs.}
% \subsubsection{Insufficiency of existing distance approximation methods for DCOs.}
% Unfortunately, no existing distance approximation methods satisfy all of the four aforementioned desiderata, to the best of our knowledge. 
% While there there quite a few distance approximation methods that have been developed for AKNN, none of them satisfy all of the four aforementioned desiderata, to the best of our knowledge.
% There are two main forms of distance approximation that have been used for AKNN: 
% % nearest neighbor search: 
% 1) quantization (QT)~\cite{ge2013optimized, jegou2010product,ITQ, additivePQ, guo2020accelerating}, and 2) dimension reduction, in which dimension reduction can be further categorized into optimization-based dimension reduction (DR)~\cite{wold1987principal, kruskal1964multidimensional} and random projection (RP)~\cite{johnson1984extensions, blockjlt, fftjlt}.
% {\CHENG Unfortunately, none of these existing methods has all of the aforementioned four desiderata} (a summary is presented in Table~\ref{tab:freq}). 
% First, QT and DR (e.g., PCA~\cite{wold1987principal}) optimize a compressed representation to minimize the \emph{total} approximation error instead of the \emph{maximum} one, which fails to guarantee an error bound. {\CHENG Only RP provides some probabilistic error bounds.
% Second, DR and RP do not support flexible resolutions of approximate distances.
% % Flexible resolution and is a long but implicitly adopted strategy in QT. 
% Only QT supports flexible resolutions to some extent with a three-stage strategy \cite{jegou2010product, imi, surveyl2hash}:}
% % to apply different resolution on different objects: 
% 1) generate candidate lists with coarse code; 2) shrink the list with finer code; 3) re-rank the list with exact distance~\footnote{Some stages could be skipped according to specific requirements, e.g., memory constraint~\cite{johnson2019billion, jegou2011searching}.}. 
% {\CHENG Nevertheless, the list size and code size at each stage are preset hyper-parameters and fixed for all queries, and thus QT provides very limited flexibility only.}
% % so it's very limited in flexibility and hard to tune. 
% Third, to the best of our knowledge, no existing methods achieve adaptivity and recoverability on resolution of distance for high-dimensional nearest neighbor search. It's worth noting that hashing-based methods~\cite{indyk1998approximate, datar2004locality, c2lsh, tao2010efficient, huang2015query, lu2020vhp} are also popular for data compression. They target to map close vectors to similar hash codes and {\CHENG use code comparison as a proxy of DCO}.
% % DCO with hash code comparison. 
% However, hashing 
% %doesn't 
% does not explicitly approximate distances, and thus they're not within the scope of our discussion.  \footnote{Also, since different vectors may be mapped to the same code, hashing cannot 
% % identify their order.
% {\CHENG help with exact DCO}.
% }
% {\JIANYANG We also emphasize that hashing and quantization cannot provide guarantee for DCO, and thus can be used only in the first stage of AKNN query, i.e., generating candidates but not the second.}
% \section{Theoretical Analysis}
\label{section:theory}

%study the accuracy and efficiency of our randomized algorithms from theoretical perspective. Specifically, we analyze the failure probability of a query and expected evaluated dimensionality of a non-KNN object. 

% \subsection{Failure Probability}
% Let's continue with our analysis in Section 3.2. ARSearch algorithm fails only when some positive objects of a search are wrongly rejected at some level $l$. Let $\mathcal O_+$ be the set of positive objects and $r $ be the radius of distance comparison of positive object $i$. We have
% \begin{align}
%     \mathbb{P} \left\{ fail \right\}  = \mathbb{P} \left\{ \exists l \le L, \exists i \in \mathcal O_{+}, dis ' > r  \cdot \gamma (d_l) \right\}  
% \end{align}
% Note that $r  > dis $ because object $i$ is positive.
% \begin{align}
%     &\le \mathbb{P} \left\{ \exists l \le L, \exists i\in \mathcal O_{+}, dis ' > dis  \cdot \gamma(d_l) \right\}  
% \end{align}
% We then plugin the definition of $dis , dis '$ and $\gamma(d_l)$:
% \begin{align}
%     =\mathbb{P} \left\{ \exists l \le L, \exists i \in \mathcal O_{+},  \sqrt {\frac{D}{d_l} } \left\|\mathbf{y} |_{[1,2,...,d_l]} \right\|
%     %\frac{\left\| \mathbf{y} |_{[1,2,...,d_l]} \right\|}{\gamma(d_l)}   
%     > \left( 1 + \frac{\epsilon _1}{\sqrt {d_l} }  \right) \| \mathbf{y}  \|  \right\} \label{eq:plugin} 
% \end{align}
% Applying union bound, i.e. the probability of the union of events is no greater than the sum of their individual probability, we have:
% \begin{align}
%     \le \sum_{l=1}^{L} \sum_{i\in \mathcal O_+} \mathbb{P} \left\{ \sqrt {\frac{D}{d_l} } \left\|\mathbf{y} |_{[1,2,...,d_l]} \right\|
%     > \left( 1 + \frac{\epsilon _1}{\sqrt {d_l} }  \right)  \| \mathbf{y}  \|  \right\} 
% \end{align}
% With the equivalence between random projection and row sampling with random transformation and the distance-preserving property of random orthogonal transformation, we have:
% \begin{align}
%     &= \sum_{l=1}^{L} \sum_{i\in \mathcal O_+} \mathbb{P} \left\{ \sqrt {\frac{D}{d_{l}} } \left\| P'|_{[1,2,...,d_l]}\mathbf{x}  \right\|   > \left( 1 + \frac{\epsilon _1}{\sqrt {d_l} }  \right)\| \mathbf{x}  \| \right\} 
% \end{align}
% Note that $P'|_{[1,2,...,d_l]}$ is a $d_l \times D$ random projection matrix. Applying Lemma~\ref{eq:concen}, we finally have
% \begin{align}
%     \mathbb{P} \left\{ fail \right\}  &\le \sum_{l=1}^{L} \sum_{i\in \mathcal O_+} \exp(-c_0 \epsilon_1^2) = L \cdot N_{pos} \cdot \exp(-c_0 \epsilon_1^2)  \\
%     &\le D \cdot N_{pos} \cdot \exp (-c_0 \epsilon _{1}^{2})  \label{eq:superexp}
%     %\\ &= \exp (-c_0 \epsilon_1^2 + \log |\mathcal O_+| + \log L)  
% \end{align}
% Letting the failure probability to be no greater than $\delta$ to solve $\epsilon_1$, then we finally have Theorem~\ref{theorem:eps}.
% In terms of Corollary~\ref{corollary:eps}, it simply shifts our concern from all positive objects to KNN objects only. %Note that the corollary is not at all related with the cardinality of a database $N$. %As a result, as a database scales up, there's no need to tune the parameters of our adaptive algorithms. 
% %when $L$ and $|\mathcal{O}_+|$ are not changed.

% \begin{comment}

% For a fixed query, suppose that $N_{vis}$ vectors are visited. For object $i$, our algorithm depends on the fact that $dis^*  \le dis $ with high probability. A failure happens if there exists an object $i$ fails at a level $l$. Thus, the overall failure probability is given as:
% \begin{align}
%     \mathbb{P}\left\{ \exists i \le N_{vis}, l < L, dis ^* > dis  \right\}
% \end{align}
% We plug-in the definiton of $dis $ and $dis^* $: 
% \begin{align}
%     =\mathbb{P} \left\{ \exists i\le N_{vis}, l < L, \sqrt {\frac{D}{d_l} } \frac{\left\| \mathbf{y} |_{[1,2,...,d_l]} \right\|}{\gamma(d_l)}   > \| \mathbf{y}  \|  \right\} \label{eq:plugin}  
% \end{align}
% Applying union bound, i.e. the probability of the union of events is no greater than the sum of their individual probability, we have:
% \begin{align}
%     \le \sum_{l=1}^{L-1} \sum_{i=1}^{N_{vis}} \mathbb{P} \left\{ \sqrt {\frac{D}{d_l} } \frac{\left\| \mathbf{y} |_{[1,2,...,d_l]} \right\|}{\gamma(d_l)}   > \| \mathbf{y}  \|  \right\}  
% \end{align}
% With the equivalence between random projection and row sampling with random transformation and preset constant significance, we finally obtain: 
% \begin{align}
%     &= \sum_{l=1}^{L-1} \sum_{i=1}^{N_{vis}} \mathbb{P} \left\{ \sqrt {\frac{D}{d_{l}} } \left\| P\mathbf{x}  \right\|   > (1 + \frac{\epsilon _1}{\sqrt {d_l} } )\| \mathbf{x}  \| \right\}  
%     \\ &\le \sum_{l=1}^{L-1} \sum_{i=1}^{N_{vis}} \exp(-c_0 \epsilon_1^2) < L \cdot N_{vis} \cdot \exp(-c_0 \epsilon_1^2)  
%     \\ &= \exp (-c_0 \epsilon_1^2 + \log N_{vis} + \log L)  
% \end{align}
% Letting the failure probability be no greater than $\delta$, then $\epsilon_1$ should be $\Theta \left( \sqrt {\log \frac{L\cdot N_{vis}}{\delta}}  \right)$.
% \end{comment}

%{\color{red} To avoid distration, I commented out the discussion about the tightness of the theoretical results.}
%{\color{red} There is a gap between theory and practice. Note that the theory only provides the correct order of $\epsilon_1$. The gap is due to 1) The tail bound of random projection is not tight in constant (but tight in order). 2) Union bound guarantees the failure probability for the worst case among all the datasets, so it highly overestimates the failure probability for real world datasets which have some good properties (they are embeddings produced by models, so they cannot be some arbitrary things.). 3) Failures not necessarily affect the correctness of result. For example, failures happened at negative objects (Non-KNN objects) even benefit the algorithm for it stops unnecessary sampling earlier. //I think it might need discussion.} 


%{\color{red} Note that improving the number of testing number and also overhead from extra priority queue operations. We should be careful about the settings of layers.}

%\subsection{Expected Terminate Dimensionality}
%We next investigate the expected dimensionality of negative objects. For a negative object $i$, supposing that it's compared with radius $r $, let $(1+\alpha )$ be the ratio between $dis $ and $r $ (though we don't know $dis $ and $\alpha $ in prior). Let random variable $\hat D $ be the terminate dimensionality of object $i$. We assume that we do hypothesis testing every time after sampling a new dimension. 
In this section, we prove Lemma~\ref{theorem:ADSampling efficiency} in detail. 
% For a negative object, let $\alpha = (dis - r) / r$ be the gap between $dis$ and $r$. Let random variable $\hat D$ be the terminate dimension (corresponds to time complexity). 
We assume that {\CHENGB we sample one additional dimension of $\mathbf{y}$ each time.}
% we do hypothesis testing each time after sampling one dimension. 
Let $\gamma(d) = (1 + \epsilon_0 / \sqrt {d} )$. 
% \begin{proof}
% Note that $\hat D$ is a non-negative random variable. 
We have
\begingroup
\allowdisplaybreaks
\begin{align}
    \mathbb{E} \left[ \hat D \right]  
    % =& \sum_{d=1}^{+\infty} \mathbb{P} \left\{ \hat D \ge d \right\}
    =& \sum_{d=1}^{D} d\cdot \mathbb{P} \left\{ \hat D = d \right\}
    = \sum_{d=1}^{D} \mathbb{P} \left\{ \hat D \ge d \right\}
% \end{align}
% {\CHENGB Here,} $\hat D \ge d$ {\CHENGB represents the event that} all the previous hypothesis testing cannot reject the hypothesis, i.e.,
% \begin{align}
    \\=& \sum_{d=1}^{D} \mathbb{P} \left\{  \forall p<d,  \sqrt {\frac{D}{p}}  \| \mathbf{y}|_{1,2,...,p}\| \le \gamma(p) \cdot r  \right\}      \label{reduction:interpret}
    \\=& \sum_{d=1}^{D} \mathbb{P} \left\{  \forall p<d, \sqrt {\frac{D}{p}} \| \mathbf{y}|_{1,2,...,p}\| \le \gamma(p) \cdot \frac{\| \mathbf{y}\| }{1+\alpha} \right\} 
% \end{align}
% We relax the condition of all previous hypothesis testing to the last testing:
% %because the filter-out probability decays exponentially. The last testing dominates the filter-out probability. 
% \begin{align}
    \\\le& 1 + \sum_{d=1}^{D-1} \mathbb{P} \left\{ \sqrt {\frac{D}{d}} \| \mathbf{y}|_{1,2,...,d}\| \le \gamma(d) \cdot \frac{\| \mathbf{y}\| }{1+\alpha}  \right\}    \label{reduction: relax hypothesis testing}  
    % \\\le&  1 + d_0   + \sum_{d=d_0 +1 }^{D} \mathbb{P} \left\{ \sqrt {\frac{D}{d}} \| \mathbf{y}|_{1,2,...,d}\| \le \frac{\gamma (d)}{1+\alpha } \| \mathbf{y} \| \right\}  \label{reduction:separate analyze}
\end{align}
\endgroup
where (\ref{reduction:interpret}) is because $\hat D \ge d$ {\CHENGB represents the event that} all the previous hypothesis testings cannot reject the hypothesis and (\ref{reduction: relax hypothesis testing}) relaxes the event corresponding to all the testings (i.e., $\forall p < d$) to that corresponding to the last testing (i.e., $p=d-1$). 
% Let $\tilde d= \epsilon _{0}^{2} /\alpha^2$ and $d_0 = \mathrm{ceil} (\tilde d)$. (\ref{reduction:separate analyze}) relaxes the probability for $d \le d_0$ to 1. 

We denote $\tilde d:= \epsilon _{0}^{2} / \alpha^2, d_0 := \mathrm{ceil} (\tilde d) $ and relax the probability for $d \le d_0$ to $1$:
\begin{align}
    \mathbb{E} \left[ \hat D \right]  \le  1 + d_0   + \sum_{d=d_0 +1 }^{D} \mathbb{P} \left\{ \sqrt {\frac{D}{d}} \| \mathbf{y}|_{1,2,...,d}\| \le \frac{\gamma (d)}{1+\alpha } \| \mathbf{y} \| \right\}  \label{reduction:separate analyze}
\end{align}
% Then we separately analyze the cases when 1) $\gamma(d) \ge 1+\alpha$ and 2) when $\gamma(d) < 1+\alpha $ for different $d$.
% %, because for $d$ whose $\gamma(d) \ge \alpha + 1$, the bound provided by Lemma~\ref{eq:concen} is trivial. 
% We denote
% \begin{align}
%     \tilde d = \lceil \frac{1}{\alpha^2} \cdot \epsilon_0^2  \rceil 
%     %= \Theta \left( \frac{1}{\alpha ^2} \cdot \log \frac{D \cdot K}{\delta} \right)
%     %\max \left( \log \frac{D}{\delta} ,\log \frac{K}{\delta}  \right) 
% \end{align}
% Note that we have $\gamma(d) \ge \alpha + 1$ for $d \le \tilde d$ and $\gamma(d) < \alpha+1$ for $d > \tilde{d}$. 
%Without loss of generality we assume that $l_0$ is an integer (otherwise, we let $l_0$ be $\lfloor \epsilon_1^2/\alpha ^2 \rfloor $). 
% With relaxing the probability for $d \le \tilde d$ to $1$, we have
% \begin{align}
%     \mathbb{E} \left[ \hat D  \right]  \le  1 + d_0   + \sum_{d=d_0 +1 }^{D} \mathbb{P} \left\{ \sqrt {\frac{D}{d}} \| \mathbf{y}|_{1,2,...,d}\| \le \frac{\gamma (d)}{1+\alpha } \| \mathbf{y} \| \right\}  \label{reduction:separate analyze}
% \end{align}
% Note again that now for $d > \tilde{d}$, $\gamma(d) /(1+\alpha) < 1$. 
% Note that for $d>d_0$, $\gamma(d) < 1 + \alpha$. 
Let's focus on the last term of (\ref{reduction:separate analyze}) and deduce from it as follows,
\begingroup
\allowdisplaybreaks
\begin{align}
    &\sum_{d=d_0 +1 }^{D} \mathbb{P} \left\{ \sqrt {\frac{D}{d}} \| \mathbf{y}|_{1,2,...,d}\| \le \frac{\gamma (d)}{1+\alpha } \| \mathbf{y} \| \right\}   \\
    =&\sum_{d=d_0 +1}^{D} \mathbb{P} \left\{ \sqrt {\frac{D}{d}} \| \mathbf{y}|_{1,2,...,d}\| \le \left[ 1- \left( 1-\frac{\gamma(d)}{1+\alpha}  \right)  \right]   \| \mathbf{y} \| \right\}    \label{reduction:rewrite}
% \end{align}
% Then with Lemma~\ref{eq:concen}, we have 
% \begin{align}
    \\\le& \sum_{d=d_0+1}^{D} \exp \left[ -c_0 d \left( 1 - \frac{\gamma(d)}{1+\alpha}  \right)^2 \right]  
    % \qquad \text{(Lemma~\ref{eq:concen})}  
    \label{reduction:lemma3.1}
    \\=&\sum_{d=d_0+1}^{D} \exp \left[ -\frac{c_0 \alpha^2}{(1+\alpha)^2}\left( \sqrt {d} - \sqrt{\tilde{d}}  \right) ^2  \right]  \label{reduction:expand and sort out}
% \end{align}
% Note that this expression is monotonically decreasing with respect to $d$, so the summation is bounded by the integration:
% \begin{align}
    \\\le& \int_{d_0}^{D} \exp \left[ -\frac{c_0 \alpha^2}{(1+\alpha)^2}\left( \sqrt {x} - \sqrt {\tilde{d}}  \right) ^2  \right] \mathrm{d} x  \label{reduction:integration}
    \\=& \int_{d_0}^{D} \exp \left[ -\frac{c_0 \epsilon_0^2}{(1+\alpha)^2}\left( \sqrt {\frac{x}{\tilde{d}} } - 1 \right) ^2  \right] \mathrm{d} x  
% \end{align}
% Let $u = x / \tilde{d}$, then we have: 
% \begin{align}
    \\=&\tilde{d} \int_{d_0 /\tilde d}^{D/\tilde{d}} \exp \left[ - \frac{c_0 \epsilon_0^2}{(1+\alpha)^2} \left( \sqrt {u} -1 \right)^2  \right]  \mathrm{d} u  
    % \quad \text{(Let}~u= x/\tilde d) 
    \label{reduction:substitute u}
    \\\le& \tilde{d} \int_{1}^{+\infty} \exp \left[ - \frac{c_0 \epsilon_0^2}{(1+\alpha)^2} \left( \sqrt {u} -1 \right)^2  \right]  \mathrm{d} u \label{reduction:relaxing to inf}
\end{align}
\endgroup
{\CHENGC where
(\ref{reduction:rewrite}) rewrites it to fit the format of Lemma~\ref{eq:concen}, 
(\ref{reduction:lemma3.1}) applies Lemma~\ref{lemma:concen}, 
(\ref{reduction:expand and sort out}) plugs in $\gamma(d)$, 
(\ref{reduction:integration}) relaxes (\ref{reduction:expand and sort out}) to an integration,
(\ref{reduction:substitute u}) substitutes $u = x / \tilde d$, and
(\ref{reduction:relaxing to inf}) relaxes the integration to $[1,+\infty)$.}
{\JIANYANG 
Next we first analyze the case of $\alpha \le \epsilon_0$ as follows.
% Under the region of reasonable accuracy, we can assume $\epsilon_0 \ge 1$ and yield (\ref{reduction:eps >= 1}). 
}
% Without loss of generality, we assume that $\tilde{d} \ge 1$ because when $\tilde{d} < 1$ the distance gap $\alpha$ is larger than the error bound $\epsilon_0$, in which constant dimensions are enough to provide firmed comparison results. Then we have $\epsilon_0 \ge \alpha $. 
\begin{align}
    \text{(\ref{reduction:relaxing to inf})}\le& \tilde{d} \int_1^{+\infty} \exp \left[ - \frac{c_0 \epsilon_0^2}{(1+\epsilon_0)^2} \left( \sqrt {u} -1 \right)^2  \right]  \mathrm{d} u  \label{reduction:alpha <= eps}
% \end{align}
% Under the region of reasonable accuracy, we can assume that $\epsilon_0\ge 1$. 
% %With {\color{red} Lévy's Concentration Inequality~\cite{wainwright_2019}}, the constant $c_0$ is no smaller than $1/2$. 
% Then we have:
% \begin{align}
    %&\le \tilde{d} \int_1^{+\infty} \exp \left[ - \frac{1}{8} \left( \sqrt {u} -1 \right)^2  \right]  \mathrm{d} u
    \\\le& \tilde{d} \int_1^{+\infty} \exp \left[ - \frac{c_0}{4} \left( \sqrt {u} -1 \right)^2  \right]  \mathrm{d} u \label{reduction:eps >= 1}
\end{align}
{\CHENGC where (\ref{reduction:alpha <= eps}) is because $\alpha \le \epsilon_0$ and (\ref{reduction:eps >= 1}) is yielded when setting $\epsilon_0 \ge 1$ for reasonable accuracy.}
Note that the integration is convergent so as to be bounded by a constant. For the case of $\alpha \le \epsilon_0$, we have 
\begin{align}
    \mathbb{E} \left[ \hat D  \right]  = 1 + d_0 + O(\tilde{d}) = O(\tilde{d}) = O \left( \alpha^{-2} \cdot \epsilon_0^2 \right)  \label{eq:quadratic}
\end{align}
{\JIANYANG
{\CHENGB For the case of} $\alpha > \epsilon_0$, its expected dimensionality must be no greater than the case of $\alpha =\epsilon_0$ because its distance gap $\alpha$ is larger. Thus, its expected dimensionality is upper bounded by $O(1)$.
% , i.e., a constant.
}
%Applying the $\epsilon_0$ given in Theorem~\ref{theorem:eps}, we have Theorem~\ref{theorem:efficiency}.
% \end{proof}

% \begin{comment}

% \begin{theorem}
% For a KNN query in $D$-dimensional space, let $(1+\alpha )$ be the ratio between negative object $i$ and its corresponding largest positive object. Then the expected evaluated dimensionality of $i$ is 
% \begin{align}
%     %\mathbb{E} \left[ \hat d  \right]  = O \left( \frac{1}{\alpha ^2} \cdot \log \frac{L \cdot |\mathcal O_+|}{\delta} \right)  
%     %\mathbb{E} \left[ \hat d  \right]  = O \left( \min ( D,  \frac{1}{\alpha ^2}\log D + \frac{1}{\alpha ^2}\log \frac{K}{\delta} )  \right)   
%     \mathbb{E} \left[ \hat d  \right]  = O \left(   \frac{1}{\alpha ^2}\log D + \frac{1}{\alpha ^2}\log \frac{K}{\delta} \right)   
% \end{align}
% \end{theorem}
% \end{comment}


\section{Conclusion and Discussion}
\label{sec:conclusion}

% {\color{red} It's also trivial to give a space-efficient LSH like index structure by simply sampling from the rotated space. Based on the experiment of Section 3, a reasonable dimensionality is as low as around 30-40d. We also want to point out that even if we do KNN query with the space-efficient LSH like indexes, we'll need to evaluate a large amount of full-precision distance to select the true KNN due to the large KNN expansion rate. Thus, we highly suggest not to do so.}

We identify the distance comparison operation which dominates the time cost of {\CHENGB nearly all} AKNN algorithms and demonstrate opportunities to improve its efficiency. 
We propose a new randomized algorithm for the DCO which runs in logarithmic time wrt $D$ {\CHENGB in most cases} and succeeds with high probability. Based on it, we further develop one {\chengf generic} and two algorithm-specific techniques for AKNN algorithms. 
Our experiments show that the enhanced AKNN algorithms outperform the original ones consistently. 
We also provide rigorous theoretical analysis for all our techniques. 

We would like to highlight the following extensions and applications of our techniques.
(1) Our techniques can be trivially extended to two other {\chengf widely}-adopted similarity metrics, {\chengf namely} cosine similarity and inner product, via simple transformation. 
Specifically, the cosine-based similarity search on some given data and query vectors is equivalent to the Euclidean nearest neighbor search on their normalized data and query vectors where \texttt{ADSampling} is applicable.  
The inner product comparison of whether $\left< \mathbf{o}, \mathbf{q}  \right> \ge r$ can be reduced to the DCO of whether $\left\| \mathbf{o} / \|\mathbf{o} \|- \mathbf{q} / \|\mathbf{q} \| \right\|  \le \sqrt {2 - 2r / (\| \mathbf{o}\| \cdot \|\mathbf{q}\| )} $, where the distance threshold equals to $\sqrt {2 - 2r / (\| \mathbf{o}\| \cdot \|\mathbf{q}\| )} $~\footnote{It can be verified as follows: $\left< \mathbf{o}, \mathbf{q}  \right> \ge r \iff  \left< \mathbf{o} / \|\mathbf{o} \|, \mathbf{q} / \|\mathbf{q} \|\right> \ge r / (\| \mathbf{o}\| \cdot \|\mathbf{q}\| ) $
$\iff \| \mathbf{o} / \|\mathbf{o} \| \|^2 - 2\left< \mathbf{o} / \|\mathbf{o} \|, \mathbf{q} / \|\mathbf{q} \|\right> + \| \mathbf{q} / \|\mathbf{q} \|\|^2 \le 2 - 2r / (\| \mathbf{o}\| \cdot \|\mathbf{q}\| ) \\ \iff \left\| \mathbf{o} / \|\mathbf{o} \|- \mathbf{q} / \|\mathbf{q} \| \right\| ^2 \le 2 - 2r / (\| \mathbf{o}\| \cdot \|\mathbf{q}\| )$.}.
% The inner product comparison of whether $\left< \mathbf{o}, \mathbf{q}  \right> \ge r$ can be reduced to the DCO of vectors $\left\| \mathbf{o} / \|\mathbf{o} \|- \mathbf{q} / \|\mathbf{q} \| \right\| ^2 \le 2 - 2r / (\| \mathbf{o}\| \cdot \|\mathbf{q}\| )$.
% \begin{equation}
%     \left< \mathbf{o}, \mathbf{q}  \right> \ge r \iff \left\| \frac{\mathbf{o} }{ \|\mathbf{o} \|}- \frac{\mathbf{q} }{ \|\mathbf{q} \|} \right\| ^2 \le 2 - \frac{2r}{\| \mathbf{o}\| \cdot \|\mathbf{q}\| } 
% \end{equation}
(2) DCOs are also ubiquitous in many other {\chengf tasks} of high-dimensional data management and analysis such as clustering~\cite{kmeans} and outlier detection~\cite{outlier_detection}. Our techniques have the potential to accelerate existing methods for those {\chengf tasks} by reducing the cost of DCOs while keeping the accuracy. 

% We demonstrate that the distance comparison operation dominates the time cost of most AKNN algorithms and there 



% Distance comparison operations dominate the time cost of most AKNN algorithms. We show 

\section{Acknowledgements}
This research is supported by the Ministry of Education, Singapore, under its Academic Research Fund (Tier 2 Awards MOE-T2EP20220-0011 and MOE-T2EP20221-0013). Any opinions, findings and conclusions or recommendations expressed in this material are those of the author(s) and do not reflect the views of the Ministry of Education, Singapore. We would like to thank Yi Li (SPMS, NTU) for answering many questions about high-dimensional probability and the anonymous reviewers for providing constructive feedback and valuable suggestions.







\bibliographystyle{ACM-Reference-Format}
\bibliography{sample-base}


\appendix
\section*{appendix}

\section{Theoretical Analysis}
\label{section:theory}

%study the accuracy and efficiency of our randomized algorithms from theoretical perspective. Specifically, we analyze the failure probability of a query and expected evaluated dimensionality of a non-KNN object. 

% \subsection{Failure Probability}
% Let's continue with our analysis in Section 3.2. ARSearch algorithm fails only when some positive objects of a search are wrongly rejected at some level $l$. Let $\mathcal O_+$ be the set of positive objects and $r $ be the radius of distance comparison of positive object $i$. We have
% \begin{align}
%     \mathbb{P} \left\{ fail \right\}  = \mathbb{P} \left\{ \exists l \le L, \exists i \in \mathcal O_{+}, dis ' > r  \cdot \gamma (d_l) \right\}  
% \end{align}
% Note that $r  > dis $ because object $i$ is positive.
% \begin{align}
%     &\le \mathbb{P} \left\{ \exists l \le L, \exists i\in \mathcal O_{+}, dis ' > dis  \cdot \gamma(d_l) \right\}  
% \end{align}
% We then plugin the definition of $dis , dis '$ and $\gamma(d_l)$:
% \begin{align}
%     =\mathbb{P} \left\{ \exists l \le L, \exists i \in \mathcal O_{+},  \sqrt {\frac{D}{d_l} } \left\|\mathbf{y} |_{[1,2,...,d_l]} \right\|
%     %\frac{\left\| \mathbf{y} |_{[1,2,...,d_l]} \right\|}{\gamma(d_l)}   
%     > \left( 1 + \frac{\epsilon _1}{\sqrt {d_l} }  \right) \| \mathbf{y}  \|  \right\} \label{eq:plugin} 
% \end{align}
% Applying union bound, i.e. the probability of the union of events is no greater than the sum of their individual probability, we have:
% \begin{align}
%     \le \sum_{l=1}^{L} \sum_{i\in \mathcal O_+} \mathbb{P} \left\{ \sqrt {\frac{D}{d_l} } \left\|\mathbf{y} |_{[1,2,...,d_l]} \right\|
%     > \left( 1 + \frac{\epsilon _1}{\sqrt {d_l} }  \right)  \| \mathbf{y}  \|  \right\} 
% \end{align}
% With the equivalence between random projection and row sampling with random transformation and the distance-preserving property of random orthogonal transformation, we have:
% \begin{align}
%     &= \sum_{l=1}^{L} \sum_{i\in \mathcal O_+} \mathbb{P} \left\{ \sqrt {\frac{D}{d_{l}} } \left\| P'|_{[1,2,...,d_l]}\mathbf{x}  \right\|   > \left( 1 + \frac{\epsilon _1}{\sqrt {d_l} }  \right)\| \mathbf{x}  \| \right\} 
% \end{align}
% Note that $P'|_{[1,2,...,d_l]}$ is a $d_l \times D$ random projection matrix. Applying Lemma~\ref{eq:concen}, we finally have
% \begin{align}
%     \mathbb{P} \left\{ fail \right\}  &\le \sum_{l=1}^{L} \sum_{i\in \mathcal O_+} \exp(-c_0 \epsilon_1^2) = L \cdot N_{pos} \cdot \exp(-c_0 \epsilon_1^2)  \\
%     &\le D \cdot N_{pos} \cdot \exp (-c_0 \epsilon _{1}^{2})  \label{eq:superexp}
%     %\\ &= \exp (-c_0 \epsilon_1^2 + \log |\mathcal O_+| + \log L)  
% \end{align}
% Letting the failure probability to be no greater than $\delta$ to solve $\epsilon_1$, then we finally have Theorem~\ref{theorem:eps}.
% In terms of Corollary~\ref{corollary:eps}, it simply shifts our concern from all positive objects to KNN objects only. %Note that the corollary is not at all related with the cardinality of a database $N$. %As a result, as a database scales up, there's no need to tune the parameters of our adaptive algorithms. 
% %when $L$ and $|\mathcal{O}_+|$ are not changed.

% \begin{comment}

% For a fixed query, suppose that $N_{vis}$ vectors are visited. For object $i$, our algorithm depends on the fact that $dis^*  \le dis $ with high probability. A failure happens if there exists an object $i$ fails at a level $l$. Thus, the overall failure probability is given as:
% \begin{align}
%     \mathbb{P}\left\{ \exists i \le N_{vis}, l < L, dis ^* > dis  \right\}
% \end{align}
% We plug-in the definiton of $dis $ and $dis^* $: 
% \begin{align}
%     =\mathbb{P} \left\{ \exists i\le N_{vis}, l < L, \sqrt {\frac{D}{d_l} } \frac{\left\| \mathbf{y} |_{[1,2,...,d_l]} \right\|}{\gamma(d_l)}   > \| \mathbf{y}  \|  \right\} \label{eq:plugin}  
% \end{align}
% Applying union bound, i.e. the probability of the union of events is no greater than the sum of their individual probability, we have:
% \begin{align}
%     \le \sum_{l=1}^{L-1} \sum_{i=1}^{N_{vis}} \mathbb{P} \left\{ \sqrt {\frac{D}{d_l} } \frac{\left\| \mathbf{y} |_{[1,2,...,d_l]} \right\|}{\gamma(d_l)}   > \| \mathbf{y}  \|  \right\}  
% \end{align}
% With the equivalence between random projection and row sampling with random transformation and preset constant significance, we finally obtain: 
% \begin{align}
%     &= \sum_{l=1}^{L-1} \sum_{i=1}^{N_{vis}} \mathbb{P} \left\{ \sqrt {\frac{D}{d_{l}} } \left\| P\mathbf{x}  \right\|   > (1 + \frac{\epsilon _1}{\sqrt {d_l} } )\| \mathbf{x}  \| \right\}  
%     \\ &\le \sum_{l=1}^{L-1} \sum_{i=1}^{N_{vis}} \exp(-c_0 \epsilon_1^2) < L \cdot N_{vis} \cdot \exp(-c_0 \epsilon_1^2)  
%     \\ &= \exp (-c_0 \epsilon_1^2 + \log N_{vis} + \log L)  
% \end{align}
% Letting the failure probability be no greater than $\delta$, then $\epsilon_1$ should be $\Theta \left( \sqrt {\log \frac{L\cdot N_{vis}}{\delta}}  \right)$.
% \end{comment}

%{\color{red} To avoid distration, I commented out the discussion about the tightness of the theoretical results.}
%{\color{red} There is a gap between theory and practice. Note that the theory only provides the correct order of $\epsilon_1$. The gap is due to 1) The tail bound of random projection is not tight in constant (but tight in order). 2) Union bound guarantees the failure probability for the worst case among all the datasets, so it highly overestimates the failure probability for real world datasets which have some good properties (they are embeddings produced by models, so they cannot be some arbitrary things.). 3) Failures not necessarily affect the correctness of result. For example, failures happened at negative objects (Non-KNN objects) even benefit the algorithm for it stops unnecessary sampling earlier. //I think it might need discussion.} 


%{\color{red} Note that improving the number of testing number and also overhead from extra priority queue operations. We should be careful about the settings of layers.}

%\subsection{Expected Terminate Dimensionality}
%We next investigate the expected dimensionality of negative objects. For a negative object $i$, supposing that it's compared with radius $r $, let $(1+\alpha )$ be the ratio between $dis $ and $r $ (though we don't know $dis $ and $\alpha $ in prior). Let random variable $\hat D $ be the terminate dimensionality of object $i$. We assume that we do hypothesis testing every time after sampling a new dimension. 
In this section, we prove Lemma~\ref{theorem:ADSampling efficiency} in detail. 
% For a negative object, let $\alpha = (dis - r) / r$ be the gap between $dis$ and $r$. Let random variable $\hat D$ be the terminate dimension (corresponds to time complexity). 
We assume that {\CHENGB we sample one additional dimension of $\mathbf{y}$ each time.}
% we do hypothesis testing each time after sampling one dimension. 
Let $\gamma(d) = (1 + \epsilon_0 / \sqrt {d} )$. 
% \begin{proof}
% Note that $\hat D$ is a non-negative random variable. 
We have
\begingroup
\allowdisplaybreaks
\begin{align}
    \mathbb{E} \left[ \hat D \right]  
    % =& \sum_{d=1}^{+\infty} \mathbb{P} \left\{ \hat D \ge d \right\}
    =& \sum_{d=1}^{D} d\cdot \mathbb{P} \left\{ \hat D = d \right\}
    = \sum_{d=1}^{D} \mathbb{P} \left\{ \hat D \ge d \right\}
% \end{align}
% {\CHENGB Here,} $\hat D \ge d$ {\CHENGB represents the event that} all the previous hypothesis testing cannot reject the hypothesis, i.e.,
% \begin{align}
    \\=& \sum_{d=1}^{D} \mathbb{P} \left\{  \forall p<d,  \sqrt {\frac{D}{p}}  \| \mathbf{y}|_{1,2,...,p}\| \le \gamma(p) \cdot r  \right\}      \label{reduction:interpret}
    \\=& \sum_{d=1}^{D} \mathbb{P} \left\{  \forall p<d, \sqrt {\frac{D}{p}} \| \mathbf{y}|_{1,2,...,p}\| \le \gamma(p) \cdot \frac{\| \mathbf{y}\| }{1+\alpha} \right\} 
% \end{align}
% We relax the condition of all previous hypothesis testing to the last testing:
% %because the filter-out probability decays exponentially. The last testing dominates the filter-out probability. 
% \begin{align}
    \\\le& 1 + \sum_{d=1}^{D-1} \mathbb{P} \left\{ \sqrt {\frac{D}{d}} \| \mathbf{y}|_{1,2,...,d}\| \le \gamma(d) \cdot \frac{\| \mathbf{y}\| }{1+\alpha}  \right\}    \label{reduction: relax hypothesis testing}  
    % \\\le&  1 + d_0   + \sum_{d=d_0 +1 }^{D} \mathbb{P} \left\{ \sqrt {\frac{D}{d}} \| \mathbf{y}|_{1,2,...,d}\| \le \frac{\gamma (d)}{1+\alpha } \| \mathbf{y} \| \right\}  \label{reduction:separate analyze}
\end{align}
\endgroup
where (\ref{reduction:interpret}) is because $\hat D \ge d$ {\CHENGB represents the event that} all the previous hypothesis testings cannot reject the hypothesis and (\ref{reduction: relax hypothesis testing}) relaxes the event corresponding to all the testings (i.e., $\forall p < d$) to that corresponding to the last testing (i.e., $p=d-1$). 
% Let $\tilde d= \epsilon _{0}^{2} /\alpha^2$ and $d_0 = \mathrm{ceil} (\tilde d)$. (\ref{reduction:separate analyze}) relaxes the probability for $d \le d_0$ to 1. 

We denote $\tilde d:= \epsilon _{0}^{2} / \alpha^2, d_0 := \mathrm{ceil} (\tilde d) $ and relax the probability for $d \le d_0$ to $1$:
\begin{align}
    \mathbb{E} \left[ \hat D \right]  \le  1 + d_0   + \sum_{d=d_0 +1 }^{D} \mathbb{P} \left\{ \sqrt {\frac{D}{d}} \| \mathbf{y}|_{1,2,...,d}\| \le \frac{\gamma (d)}{1+\alpha } \| \mathbf{y} \| \right\}  \label{reduction:separate analyze}
\end{align}
% Then we separately analyze the cases when 1) $\gamma(d) \ge 1+\alpha$ and 2) when $\gamma(d) < 1+\alpha $ for different $d$.
% %, because for $d$ whose $\gamma(d) \ge \alpha + 1$, the bound provided by Lemma~\ref{eq:concen} is trivial. 
% We denote
% \begin{align}
%     \tilde d = \lceil \frac{1}{\alpha^2} \cdot \epsilon_0^2  \rceil 
%     %= \Theta \left( \frac{1}{\alpha ^2} \cdot \log \frac{D \cdot K}{\delta} \right)
%     %\max \left( \log \frac{D}{\delta} ,\log \frac{K}{\delta}  \right) 
% \end{align}
% Note that we have $\gamma(d) \ge \alpha + 1$ for $d \le \tilde d$ and $\gamma(d) < \alpha+1$ for $d > \tilde{d}$. 
%Without loss of generality we assume that $l_0$ is an integer (otherwise, we let $l_0$ be $\lfloor \epsilon_1^2/\alpha ^2 \rfloor $). 
% With relaxing the probability for $d \le \tilde d$ to $1$, we have
% \begin{align}
%     \mathbb{E} \left[ \hat D  \right]  \le  1 + d_0   + \sum_{d=d_0 +1 }^{D} \mathbb{P} \left\{ \sqrt {\frac{D}{d}} \| \mathbf{y}|_{1,2,...,d}\| \le \frac{\gamma (d)}{1+\alpha } \| \mathbf{y} \| \right\}  \label{reduction:separate analyze}
% \end{align}
% Note again that now for $d > \tilde{d}$, $\gamma(d) /(1+\alpha) < 1$. 
% Note that for $d>d_0$, $\gamma(d) < 1 + \alpha$. 
Let's focus on the last term of (\ref{reduction:separate analyze}) and deduce from it as follows,
\begingroup
\allowdisplaybreaks
\begin{align}
    &\sum_{d=d_0 +1 }^{D} \mathbb{P} \left\{ \sqrt {\frac{D}{d}} \| \mathbf{y}|_{1,2,...,d}\| \le \frac{\gamma (d)}{1+\alpha } \| \mathbf{y} \| \right\}   \\
    =&\sum_{d=d_0 +1}^{D} \mathbb{P} \left\{ \sqrt {\frac{D}{d}} \| \mathbf{y}|_{1,2,...,d}\| \le \left[ 1- \left( 1-\frac{\gamma(d)}{1+\alpha}  \right)  \right]   \| \mathbf{y} \| \right\}    \label{reduction:rewrite}
% \end{align}
% Then with Lemma~\ref{eq:concen}, we have 
% \begin{align}
    \\\le& \sum_{d=d_0+1}^{D} \exp \left[ -c_0 d \left( 1 - \frac{\gamma(d)}{1+\alpha}  \right)^2 \right]  
    % \qquad \text{(Lemma~\ref{eq:concen})}  
    \label{reduction:lemma3.1}
    \\=&\sum_{d=d_0+1}^{D} \exp \left[ -\frac{c_0 \alpha^2}{(1+\alpha)^2}\left( \sqrt {d} - \sqrt{\tilde{d}}  \right) ^2  \right]  \label{reduction:expand and sort out}
% \end{align}
% Note that this expression is monotonically decreasing with respect to $d$, so the summation is bounded by the integration:
% \begin{align}
    \\\le& \int_{d_0}^{D} \exp \left[ -\frac{c_0 \alpha^2}{(1+\alpha)^2}\left( \sqrt {x} - \sqrt {\tilde{d}}  \right) ^2  \right] \mathrm{d} x  \label{reduction:integration}
    \\=& \int_{d_0}^{D} \exp \left[ -\frac{c_0 \epsilon_0^2}{(1+\alpha)^2}\left( \sqrt {\frac{x}{\tilde{d}} } - 1 \right) ^2  \right] \mathrm{d} x  
% \end{align}
% Let $u = x / \tilde{d}$, then we have: 
% \begin{align}
    \\=&\tilde{d} \int_{d_0 /\tilde d}^{D/\tilde{d}} \exp \left[ - \frac{c_0 \epsilon_0^2}{(1+\alpha)^2} \left( \sqrt {u} -1 \right)^2  \right]  \mathrm{d} u  
    % \quad \text{(Let}~u= x/\tilde d) 
    \label{reduction:substitute u}
    \\\le& \tilde{d} \int_{1}^{+\infty} \exp \left[ - \frac{c_0 \epsilon_0^2}{(1+\alpha)^2} \left( \sqrt {u} -1 \right)^2  \right]  \mathrm{d} u \label{reduction:relaxing to inf}
\end{align}
\endgroup
{\CHENGC where
(\ref{reduction:rewrite}) rewrites it to fit the format of Lemma~\ref{eq:concen}, 
(\ref{reduction:lemma3.1}) applies Lemma~\ref{lemma:concen}, 
(\ref{reduction:expand and sort out}) plugs in $\gamma(d)$, 
(\ref{reduction:integration}) relaxes (\ref{reduction:expand and sort out}) to an integration,
(\ref{reduction:substitute u}) substitutes $u = x / \tilde d$, and
(\ref{reduction:relaxing to inf}) relaxes the integration to $[1,+\infty)$.}
{\JIANYANG 
Next we first analyze the case of $\alpha \le \epsilon_0$ as follows.
% Under the region of reasonable accuracy, we can assume $\epsilon_0 \ge 1$ and yield (\ref{reduction:eps >= 1}). 
}
% Without loss of generality, we assume that $\tilde{d} \ge 1$ because when $\tilde{d} < 1$ the distance gap $\alpha$ is larger than the error bound $\epsilon_0$, in which constant dimensions are enough to provide firmed comparison results. Then we have $\epsilon_0 \ge \alpha $. 
\begin{align}
    \text{(\ref{reduction:relaxing to inf})}\le& \tilde{d} \int_1^{+\infty} \exp \left[ - \frac{c_0 \epsilon_0^2}{(1+\epsilon_0)^2} \left( \sqrt {u} -1 \right)^2  \right]  \mathrm{d} u  \label{reduction:alpha <= eps}
% \end{align}
% Under the region of reasonable accuracy, we can assume that $\epsilon_0\ge 1$. 
% %With {\color{red} Lévy's Concentration Inequality~\cite{wainwright_2019}}, the constant $c_0$ is no smaller than $1/2$. 
% Then we have:
% \begin{align}
    %&\le \tilde{d} \int_1^{+\infty} \exp \left[ - \frac{1}{8} \left( \sqrt {u} -1 \right)^2  \right]  \mathrm{d} u
    \\\le& \tilde{d} \int_1^{+\infty} \exp \left[ - \frac{c_0}{4} \left( \sqrt {u} -1 \right)^2  \right]  \mathrm{d} u \label{reduction:eps >= 1}
\end{align}
{\CHENGC where (\ref{reduction:alpha <= eps}) is because $\alpha \le \epsilon_0$ and (\ref{reduction:eps >= 1}) is yielded when setting $\epsilon_0 \ge 1$ for reasonable accuracy.}
Note that the integration is convergent so as to be bounded by a constant. For the case of $\alpha \le \epsilon_0$, we have 
\begin{align}
    \mathbb{E} \left[ \hat D  \right]  = 1 + d_0 + O(\tilde{d}) = O(\tilde{d}) = O \left( \alpha^{-2} \cdot \epsilon_0^2 \right)  \label{eq:quadratic}
\end{align}
{\JIANYANG
{\CHENGB For the case of} $\alpha > \epsilon_0$, its expected dimensionality must be no greater than the case of $\alpha =\epsilon_0$ because its distance gap $\alpha$ is larger. Thus, its expected dimensionality is upper bounded by $O(1)$.
% , i.e., a constant.
}
%Applying the $\epsilon_0$ given in Theorem~\ref{theorem:eps}, we have Theorem~\ref{theorem:efficiency}.
% \end{proof}

% \begin{comment}

% \begin{theorem}
% For a KNN query in $D$-dimensional space, let $(1+\alpha )$ be the ratio between negative object $i$ and its corresponding largest positive object. Then the expected evaluated dimensionality of $i$ is 
% \begin{align}
%     %\mathbb{E} \left[ \hat d  \right]  = O \left( \frac{1}{\alpha ^2} \cdot \log \frac{L \cdot |\mathcal O_+|}{\delta} \right)  
%     %\mathbb{E} \left[ \hat d  \right]  = O \left( \min ( D,  \frac{1}{\alpha ^2}\log D + \frac{1}{\alpha ^2}\log \frac{K}{\delta} )  \right)   
%     \mathbb{E} \left[ \hat d  \right]  = O \left(   \frac{1}{\alpha ^2}\log D + \frac{1}{\alpha ^2}\log \frac{K}{\delta} \right)   
% \end{align}
% \end{theorem}
% \end{comment}



% \section{Product Quantization}

% \begin{figure}[hbt]
%     \centering
%     \vspace{-4mm}
%     \subfigure[\texttt{IVFPQ 64}]{
%         \includegraphics[width=0.45\linewidth]{revision experimental result/IVFPQ_64.pdf}
%        \label{fig:cost IVFPQ64}
%     }
%     \subfigure[\texttt{IVFPQ 128}]{
% 	    \includegraphics[width=0.45\linewidth]{revision experimental result/IVFPQ_128.pdf}
% 	    \label{fig:cost IVFPQ128}
%     }
%     \vspace{-4mm}
%     \caption{{\CHENG Breakdown of Running Times of} IVFPQ.}
%     % \vspace{-4mm}
%     \label{fig:cost statistics-IVF}
% \end{figure}

% \begin{figure}
%     \centering
%     \vspace{-4mm}
%     \includegraphics[width=\linewidth]{revision experimental result/QPS_linear.png}
%     \vspace{-4mm}
%     \caption{Efficiency of DCOs.}
%     \label{fig:qps linear}
% \end{figure}
{\JIANYANGREVISION
\section{Results of Tree-based and Hashing-based Methods}
\label{appendix:section tree and hashing}
\begin{figure*}[thb]
% \vspace*{-4mm}
  \centering 
  % \includesvg[width=17cm]{experimental result/time-accuracy.svg}
  % \includesvg[width=17cm]{revision experimental result/time-accuracy-hash.svg}
    \includegraphics[width=17cm]{revision experimental result/time-accuracy-hash.pdf}
  \vspace*{-4mm}
  \caption{{\JIANYANGREVISION Time-Accuracy Tradeoff (\texttt{PMLSH} and \texttt{Annoy}).}}
  \vspace*{-4mm}
  \label{figure:time-accuracy-tree-and-hashing}
\end{figure*}
For \texttt{Annoy}, following \cite{li2019approximate}, we set the number of trees $N_{tree} = 50$. 
{\JIANYANGREVISION During the \underline{index phase}, we feed the raw data vectors into the indexing algorithm of \texttt{Annoy} (note that \texttt{Annoy}, \texttt{Annoy+} and \texttt{Annoy}* have the same index structure). Then during the \underline{query phase}, for \texttt{Annoy}/\texttt{Annoy}*, we load the index and the raw data vectors into main memory, generate candidates by feeding the raw query vector into the the query algorithm of \texttt{Annoy} and re-rank the candidates with \texttt{FDScanning}/\texttt{PDScanning}. For \texttt{Annoy+}, we load the index and the transformed data vectors into main memory, generate candidates by feeding the raw query vector into the the query algorithm of \texttt{Annoy} and re-rank the candidates with \texttt{ADSampling}. }
For \texttt{PMLSH}, following \cite{zheng2020pm}, we set the dimensionality of random projection as 15, the size of internal and leaf nodes of the PM-Tree as 16. Similar to \texttt{Annoy}, during the \underline{index phase}, we build the indexes based on the raw data vectors. During the \underline{query phase}, we generate candidates by feeding the raw query vectors to the search algorithm of \texttt{PMLSH} and re-rank them with \texttt{FDScanning}, \texttt{PDScanning} and \texttt{ADSampling} based on raw vectors, raw vectors and transformed vectors respectively. 
For both methods, we vary the number of accessed candidates to control the time-accuracy tradeoff.  
{\chengr We exclude the optimization of data layout for this experiment since it is not applicable for index ensembles (e.g., tree ensembles of \texttt{Annoy}).}
% 2) according to Section~\ref{subsec:main result}, the \texttt{ADSampling} method contributes the most improvement for \texttt{IVF}, (3) as shown in Figure~\ref{figure:time-accuracy-tree-and-hashing}, \texttt{Annoy+} and \texttt{PMLSH+} have already brought consistent and significant improvement 
% and (4) \texttt{Annoy} and \texttt{PMLSH} are not the focus of our paper due to their suboptimal performance,
% {\chengr we } we exclude the optimization of data layout.
We plot the QPS-recall and QPS-average distance ratio curves {\chengr of the compared algorithms} in Figure~\ref{figure:time-accuracy-tree-and-hashing}. It shows that \texttt{AKNN+} outperforms the \texttt{AKNN}* and \texttt{AKNN} algorithms consistently and significantly. 



}

\end{document}
\endinput
%%
%% End of file `sample-authordraft.tex'.
