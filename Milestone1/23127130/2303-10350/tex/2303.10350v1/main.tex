\documentclass{article}

\usepackage{arxiv}

\usepackage[utf8]{inputenc} % allow utf-8 input
\usepackage[T1]{fontenc}    % use 8-bit T1 fonts
\usepackage[english]{babel}
\usepackage{hyphenat}
\usepackage{lmodern}
\usepackage{dsfont}
\usepackage{bold-extra}
% \usepackage{pdfsync}
\usepackage{mathtools}
\mathtoolsset{showonlyrefs=true,showmanualtags=true} % equations would be numbered which are referred to in the text %
\usepackage{latexsym}
\usepackage[%
    pdftex,
    bookmarks=true,
    pdfstartview={FitH}
]{hyperref}                 % hyperlinks
\usepackage{url}            % simple URL typesetting
\usepackage{booktabs}       % professional-quality tables
\usepackage{amsmath,amsfonts,amssymb,amsthm,amstext}       % AMS packages
\usepackage[nice]{nicefrac} % compact symbols for 1/2, etc.
\usepackage{microtype}      % microtypography
\usepackage{lipsum}		    % Can be removed after putting your text content
\usepackage{mathscinet}
\usepackage[inline]{enumitem}
\usepackage{graphicx}
\usepackage{float}
\usepackage{caption}
\usepackage{subcaption}
\usepackage{color}
\usepackage[dvipsnames]{xcolor}
\usepackage{tikz}
\usepackage[nottoc]{tocbibind}
\usepackage[square,sort,comma,numbers,sectionbib]{natbib}
\graphicspath{ {Figures/} }
\usepackage{doi}
\usepackage{indentfirst}
\usepackage{orcidlink}       % ORCiD linking package
% \usepackage{showkeys}      % shows labels in the output file

%%% definitions %%%
\setlength{\parindent}{20pt}

\numberwithin{equation}{section}

\theoremstyle{plain}
\newtheorem{theorem}{Theorem}[section]
\newtheorem*{theorem*}{Theorem}
\newtheorem{corollary}[theorem]{Corollary}
\newtheorem{lemma}[theorem]{Lemma}
\newtheorem*{lemma*}{Lemma}
\newtheorem{remark}[theorem]{Remark}
\newtheorem*{remark*}{Remark}
\newtheorem{definition}[theorem]{Definition}
\newtheorem*{definition*}{Definition}
\newtheorem{test}{Test}[section]
\newtheorem*{test*}{Test}

\makeatletter
\renewenvironment{proof}[1][\proofname]{%
	\par\pushQED{\qed}\normalfont%
	\topsep6\p@\@plus6\p@\relax
	\trivlist\item[\hskip\labelsep\bfseries#1\@addpunct{.}]%
	\ignorespaces
}{%
	\popQED\endtrivlist\@endpefalse
}
\makeatother

%%% Modified Letters %%%
\long\def\c{\mathrm{c}}
\long\def\M{\mathrm{M}}
\long\def\d{\mathrm{d}}
\long\def\e{\mathrm{e}}
\long\def\bigO{\mathcal{O}}
\long\def\L{\mathcal{L}}
\long\def\ie{\textit{i.e.}}
\long\def\eg{\textit{e.g.}}
\long\def\cf{\textit{cf.}}
\long\def\et{\textit{et al.}}

%\renewcommand\qedsymbol{$\blacksquare$}

% Title and Author %
\def\articletitle{On Convergence of a Three-layer Semi-discrete scheme for the Nonlinear Dynamic String Equation of Kirchhoff-type with Time-dependent Coefficients}
\def\articleauthorR{Jemal~Rogava}
\def\articleauthorV{Zurab~Vashakidze}

\title{\articletitle}

\date{} 					% Or removing it

\author{
        {\articleauthorR}\,\orcidlink{0000-0001-9460-4283}\\
	Faculty of Exact and Natural Sciences,\\
	Ivane Javakhishvili Tbilisi State University (TSU),\\
	Ilia Vekua Institute of Applied Mathematics (VIAM),\\
	2 University St., Tbilisi 0186, Georgia\\
	\href{mailto:jemal.rogava@tsu.ge}{\texttt{jemal.rogava@tsu.ge}}\\
	%% examples of more authors
	\And
	{\articleauthorV}\,\orcidlink{0000-0001-8736-6213}\\
	Institute of Mathematics, School of Science and Technology,\\
	The University of Georgia (UG),\\
	77a M. Kostava St., Tbilisi 0171, Georgia\\[6pt]
	Department of Numerical Mathematics and Modelling,\\
	Ilia Vekua Institute of Applied Mathematics (VIAM),\\
	Ivane Javakhishvili Tbilisi State University (TSU),\\
	2 University St., Tbilisi 0186, Georgia\\
	\href{mailto:zurab.vashakidze@gmail.com}{\texttt{zurab.vashakidze@gmail.com}}, \href{mailto:z.vashakidze@ug.edu.ge}{\texttt{z.vashakidze@ug.edu.ge}}
}

% Uncomment to remove the date
%\date{}

% Uncomment to override  the `A preprint' in the header
%\renewcommand{\headeright}{Technical Report}
%\renewcommand{\undertitle}{Technical Report}
\renewcommand{\shorttitle}{Convergence of a Semi-discrete Scheme for a Nonlinear Dynamic String Equation}

%%% Add PDF metadata to help others organize their library
%%% Once the PDF is generated, you can check the metadata with
%%% $ pdfinfo template.pdf
\hypersetup{
    pdftitle={\articletitle},
    pdfsubject={Numerical Analysis (math.NA); Analysis of PDEs (math.AP)},
    pdfauthor={\articleauthorR,~\articleauthorV},
    pdfkeywords={Non-linear Kirchhoff string equation, Cauchy problem, Three-layer semi-discrete scheme, Galerkin method, Cholesky decomposition},
    hidelinks,
    colorlinks=true,
    linkcolor=blue,
    citecolor=red,
    filecolor=cyan,
    urlcolor=teal,
}

\begin{document}
\maketitle
\vfill
\begin{abstract}\label{abstract}
    This paper considers the Cauchy problem for the nonlinear dynamic string equation of Kirchhoff-type with time-varying coefficients. The objective of this work is to develop a temporal discretization algorithm capable of approximating a solution to this initial-boundary value problem. To this end, a symmetric three-layer semi-discrete scheme is employed with respect to the temporal variable, wherein the value of a nonlinear term is evaluated at the middle node point. This approach enables the numerical solutions per temporal step to be obtained by inverting the linear operators, yielding a system of second-order linear ordinary differential equations. Local convergence of the proposed scheme is established, and it achieves quadratic convergence concerning the step size of the discretization of time on the local temporal interval.
    % This paper considers the Cauchy problem for the nonlinear dynamic string equation of Kirchhoff-type with time-varying coefficients. The purpose of this work is to develop a temporal discretization algorithm that can approximate a solution for this initial-boundary value problem. To accomplish this, a symmetric three-layer semi-discrete scheme is employed with respect to the temporal variable. In this scheme, the value of a nonlinear term is evaluated at the middle node point, enabling the numerical solutions per temporal steps to be found by inverting the linear operators, yielding a system of second-order linear ordinary differential equations. Local convergence of the proposed scheme is established, and it achieves quadratic convergence concerning the step size of the discretization of time on the local temporal interval.
\end{abstract}
\vfill
% keywords
\keywords{{Non-linear Kirchhoff string equation} \and {Cauchy problem} \and {Three-layer semi-discrete scheme}.}

% MSC 2010
\msc{{65M06} \and {65N12} \and {65N22} \and {65Q30}.}

%\newpage
\section{Introduction}\label{sec:intro}
\subsection{Formulation of the problem}\label{subsec:statement}
Consider the Cauchy problem for the nonlinear dynamic Kirchhoff string equation
\begin{subequations}
    \begin{gather}
        \frac{{\partial}^{2}u\left( x,t \right)}{\partial{t}^{2}} - \left( \alpha\left( t \right) + \beta\left( t \right)\int\limits_{0}^{\ell}{\left[ \frac{\partial u\left( x,t \right)}{\partial x} \right]}^{2}\d x \right)\frac{{\partial}^{2}u\left( x,t \right)}{\partial{x}^{2}} = f\left( x,t \right)\,,\quad \left( x,t \right)\in \left( 0,\ell \right)\times \left( 0,T \right]\,,\label{eq:main_eqt} \\
        u\left( x,0 \right) = {\psi}_{0}\left( x \right)\,,\quad {u}_{t}^{\prime}\left( x,0 \right) = {\psi}_{1}\left( x \right)\,,\label{eq:initial_conds} \\
        u\left( 0,t \right) = 0\,,\quad u\left( \ell,t \right) = 0\,.\label{eq:boundary_conds}
    \end{gather}
\end{subequations}
Given that $\alpha\left( t \right) \geq {\c}_{0} > 0$ and $\beta\left( t \right) \geq {\c}_{1} > 0$, where $\alpha\left( t \right)$ and $\beta\left( t \right)$ are continuously differentiable functions. Further, let $f\left( x,t \right)$ be a continuous function, $u\left( x,t \right)$ an unknown function, and ${\psi}_{0}\left( x \right)$ and ${\psi}_{1}\left( x \right)$ continuous functions. It is also assumed that compatibility conditions are provided for the function ${\psi}_{0}\left( x \right)$, {\ie} ${\psi}_{0}\left( 0 \right) = {\psi}_{0}\left( \ell \right) = 0$.

\subsection{Historical background and framework of the problem}\label{subsec:frameofprob}

The nonlinear dynamic Kirchhoff string equation is a hyperbolic partial differential equation that accurately models the behaviour of a vibrating string,
\begin{equation}\label{eq:intro_kirchhoff_equation}
    {u}_{tt}\left( x,t \right) = \left( {\alpha} + {\beta}\int\limits_{0}^{\ell}{u}_{x}^{2}\left( x,t \right)\,\d{x} \right){u}_{xx}\left( x,t \right)\,,\quad {\alpha} > 0\,,\quad {\beta} > 0\,.\tag{K}
\end{equation}
This equation \eqref{eq:intro_kirchhoff_equation} was first derived by the prominent German physicist, Gustav Kirchhoff, in the mid-nineteenth century (see \cite{kirchhoff1876vorlesungen}) as an extension of D'Alembert's wave equation for free vibrations of elastic strings, and takes into account the tension present in the string and various forces acting upon it, such as gravity and air resistance. The solution of the equation provides the displacement of the string at any given point in time, and its application extends to a multitude of physical systems, including musical instruments, speech production, and seismic waves produced during earthquakes.

Despite its antiquity, the Kirchhoff dynamic string equation remains a valuable tool in the study of wave motion and is extensively utilized in the fields of physics, engineering, and related sciences. Numerous distinguished scientists have contributed to the study of the equation and its various modifications. However, obtaining an exact solution to the equation remains a challenging problem, and numerical methods are commonly employed to approximate the solution. The equation has been extensively investigated using both numerical and analytical methods.

The conditions governing the existence of solutions for equation \eqref{eq:intro_kirchhoff_equation} were first examined by S. Bernstein in 1940 \cite{bernstein1940}. Bernstein's work focused on the scenario where the initial data are $2\pi$-periodic analytic functions on $\mathbb{R}$. Poho\v{z}aev extended Bernstein's results in 1975 \cite{pohozaev1975} to encompass the multidimensional case concerning spatial variables. In 1978, Lions presented an abstract framework for the Kirchhoff equation \eqref{eq:intro_kirchhoff_equation} in a publication that significantly contributed to its popularity \cite{lions1978}. Since then, many researchers have studied the solvability issues of the classical and modified Kirchhoff static and dynamic equations in general function spaces and developed a qualitative approach. These scholars include Alves, Corr\^{e}a, and Ma \cite{alves2005}; Arosio and Panizzi \cite{arosio1996}; Berselli and Manfrin \cite{bereselli2000}; D'Ancona and Spagnolo \cite{dancona1993,dancona1994,dancona1995}; Greenberg and Hu \cite{greenberg198081}; Kajitani \cite{kajitani2009}; Liu and Rincon \cite{liu2003}; Ma \cite{MA2005e1967}; Manfrin \cite{manfrin2002}; Matos \cite{matos1991}; Medeiros \cite{medeiros1979}; Nishihara \cite{nishihara1984}; and Rzymowski \cite{rzymowski2002}.

Over the years, many numerical methods have been developed to address the boundary and initial-boundary value problems arising from both the classical and modified Kirchhoff's string equations. These methods encompass a variety of techniques, including but not limited to finite difference, spectral, and finite element methods. Moreover, several of these methods have been combined with other numerical schemes, resulting in even more sophisticated approaches. Christie and Sanz-Serna \cite{christie1984} applied the finite element method to discretize the Kirchhoff equation \eqref{eq:intro_kirchhoff_equation}. They utilized predictor-corrector methods based on Crank-Nicolson-type schemes for time integration. Gudi \cite{gudi2012} investigated a finite element method for a second-order nonlinear elliptic boundary value problem of the stationary version of the generalized Kirchhoff equation. Here, the finite element system is replaced with an equivalent sparse system for which the Jacobian of the Newton-Raphson iterative method is sparse. Liu and Rincon \cite{liu2003} solved the Kirchhoff equation for nonlinear elastic strings with moving boundaries using the explicit Lax-Friedrichs difference scheme. Peradze \cite{peradze2005} developed an algorithm for the numerical solution of the dynamic Kirchhoff string equation, which utilizes the Galerkin method, the modified Crank-Nicolson difference scheme, and a Picard-type iteration process. In \cite{peradze2009}, Peradze proposed a numerical algorithm based on the projection method and the finite difference method for approximating the Kirchhoff wave equation with respect to spatial and time variables. In \cite{ren2017}, Q. Ren and H. Tian developed a numerical scheme for obtaining an approximate solution to a nonlocal stationary analogue of the Kirchhoff equation using the Legendre-Galerkin spectral method, which reduces the problem to a nonlinear finite-dimensional system, followed by an iterative solution. In all of the works considered, numerical algorithms are designed to combine the reduction of the original problem to a finite-dimensional system of nonlinear equations with an iterative process for finding an approximate solution to the obtained system.

The works authored by J. Rogava and M. Tsiklauri \cite{RogavaTsiklauri_LocConvg2012,RogavaTsiklauri_EvolEqt2014} are primarily concerned with the development and analysis of a symmetric three-layer semi-discrete scheme for solving the Cauchy problem associated with the abstract generalization of the dynamic Kirchhoff equation. In the proposed scheme, the value of a nonlinear term is evaluated at the middle node point, resulting in the transformation of the original problem into a linear one for each temporal layer. In \cite{vashakidze2020,vashakidze2022}, the same approach is extended through the combination of the Legendre-Galerkin spectral method for numerically solving the spatial one-dimensional nonlinear dynamic Kirchhoff string equation. This choice leads to the reduction of the stated nonlinear problem to a system of linear equations per temporal layer, with the corresponding coefficient matrix being sparse and featuring nonzero elements exclusively on the main diagonal, the second diagonal below it, and the second diagonal above it.

The present study addresses the initial-boundary value problem \eqref{eq:main_eqt}-\eqref{eq:boundary_conds} pertaining to the nonlinear dynamic Kirchhoff-type string equation with varying coefficients over time. Our objective is to obtain an approximate solution for this problem by employing a symmetric three-layer semi-discrete scheme with respect to the temporal variable. Notably, in this scheme, the nonlinear term is taken at the middle node point ({\cf} \cite{RogavaTsiklauri_LocConvg2012,RogavaTsiklauri_EvolEqt2014,vashakidze2020,vashakidze2022}). By applying this approach, the stated hyperbolic nonlinear partial differential equation can be reduced to a system of linear ordinary differential equations of second order. We complete an investigation into the convergence of the proposed semi-discrete scheme for approximating the solution to the problem. Our research demonstrates that the considered scheme attains quadratic convergence regarding the step size of the discretization of time on the local temporal interval.

\section{Description of the temporal discretization algorithm}\label{subsec:descr_algor}

Let the temporal interval $\left[ 0,T \right]$ be divided into equal sub-intervals ${\left[ {t}_{i - 1},{t}_{i} \right]}_{i = 1}^{n}$ using a uniform grid with step size $\tau$, \ie
\begin{equation*}
    0 = {t}_{0} < {t}_{1} < \cdots < {t}_{n - 1} < {t}_{n} = T\,,\quad {t}_{k} = {k}{\tau}\,,\quad {k} = 0, 1, \ldots, n\,,\quad {\tau} = \frac{T}{n}\,.
\end{equation*}
The equation given in \eqref{eq:main_eqt} is transformed into the following form at discrete time points $t = {t}_{k}$ ($k = 1, 2, \ldots, {n - 1}$):
\begin{equation}\label{eq:discrete_main_eqt}
    \frac{{\Delta}^{2}u\left( x,{t}_{k - 1} \right)}{{\tau}^{2}} - \frac{1}{2}q\left( {t}_{k} \right)\left( \frac{\d ^{2}u\left( x,{t}_{k + 1} \right)}{\d {x}^{2}} + \frac{\d ^{2}u\left( x,{t}_{k - 1} \right)}{\d {x}^{2}} \right) = f\left( x, {t}_{k} \right) + {R}_{1,k}\left( x,\tau \right) + {R}_{2,k}\left( x,\tau \right)\,,
\end{equation}
where
\begin{gather*}
    {\Delta}u\left( x,{t}_{k} \right) = u\left( x,{t}_{k + 1} \right) - u\left( x,{t}_{k} \right)\,,\\
    q\left( t \right) = \alpha\left( t \right) + \beta\left( t \right)\int\limits_{0}^{\ell}{\left[ \frac{\partial u\left( x,t \right)}{\partial x} \right]}^{2}\d x\,,\\
    {R}_{1,k}\left( x,\tau \right) = \frac{{\Delta}^{2}u\left( x,{t}_{k - 1} \right)}{{\tau}^{2}} - \frac{{\partial}^{2}u\left( x,{t}_{k} \right)}{\partial{t}^{2}}\,,\\
    {R}_{2,k}\left( x,\tau \right) = -\frac{1}{2}q\left( {t}_{k} \right)\frac{\d ^{2}}{\d {x}^{2}}{\Delta}^{2}u\left( x,{t}_{k - 1} \right)\,.
\end{gather*}
It is shown that ${R}_{1,k}\left( x,\tau \right) = \bigO\left( {\tau}^{2} \right)$ and ${R}_{2,k}\left( x,\tau \right) = \bigO\left( {\tau}^{2} \right)$ (see the \hyperref[theorem:theorem1]{\bf Theorem \ref*{theorem:theorem1}}). By neglecting the remainder terms ${R}_{1,k}\left( x,\tau \right)$ and ${R}_{2,k}\left( x,\tau \right)$ in equation \eqref{eq:discrete_main_eqt}, the following symmetric three-layer semi-discrete scheme is obtained
\begin{equation}\label{eq:semidiscrete_scheme}
    \frac{{\Delta}^{2}{u}_{k - 1}\left( x \right)}{{\tau}^{2}} - \frac{1}{2}{q}_{k}\left( \frac{\d ^{2}{u}_{k + 1}\left( x \right)}{\d {x}^{2}} + \frac{\d ^{2}{u}_{k - 1}\left( x \right)}{\d {x}^{2}} \right) = {f}_{k}\left( x \right)\,,\quad k = 1, 2, \ldots, {n - 1}\,,
\end{equation}
where ${f}_{k}\left( x \right) = f\left( x, {t}_{k} \right)$ and
\begin{equation*}
    {q}_{k} = {\alpha}_{k} + {\beta}_{k}\int\limits_{0}^{\ell}{\left( \frac{\d {u}_{k}\left( x \right)}{\d x} \right)}^{2}\d x\,,\quad {\alpha}_{k} = \alpha\left( {t}_{k} \right)~\text{and}~{\beta}_{k} = \beta\left( {t}_{k} \right)\,.
\end{equation*}
In the following, the solution to the obtained differential-difference equations \eqref{eq:semidiscrete_scheme} are denoted by ${u}_{k}\left( x \right)$ and are considered as approximate solutions to the problem \eqref{eq:main_eqt}-\eqref{eq:boundary_conds} at the time instants $t = {t}_{k}$, thus $u\left( x,{t}_{k} \right) \approx {u}_{k}\left( x \right)$.

We now express equation \eqref{eq:semidiscrete_scheme} in an alternative form
\begin{equation}\label{eq:discrt_operator_eqt}
    \left( 2{I} - {\tau}^{2}{q}_{k}\frac{\d ^{2}}{\d {x}^{2}} \right){u}_{k + 1}\left( x \right) = {g}_{k}\left( x \right)\,,\quad k = 1, 2, \ldots, {n - 1}\,,
\end{equation}
where
\begin{equation*}
    {g}_{k}\left( x \right) = {2}\left( {\tau}^{2}{f}_{k}\left( x \right) + {2}{u}_{k}\left( x \right) \right) - \left( 2{I} - {\tau}^{2}{q}_{k}\frac{\d ^{2}}{\d {x}^{2}} \right){u}_{k - 1}\left( x \right)\,.
\end{equation*}

The values of the unknown functions at the zeroth and first layers are determined by the initial conditions given in \eqref{eq:initial_conds} and the equation \eqref{eq:main_eqt},
\begin{align}
    {u}_{0}\left( x \right) &= {\psi}_{0}\left( x \right)\,,\nonumber \\
    {u}_{1}\left( x \right) &= {\psi}_{0}\left( x \right) + {\tau}{\psi}_{1}\left( x \right) + \frac{{\tau}^{2}}{2}{\psi}_{2}\left( x \right)\,,\quad {\psi}_{2}\left( x \right) = {f}_{0}\left( x \right) + {q}_{0}\frac{\d ^{2}{\psi}_{0}\left( x \right)}{\d {x}^{2}}\,.\label{eq:semidscrete_scheme_first_layer}
\end{align}
For each discrete time layer, the boundary conditions are rewritten as follows, with $k = 1, 2, \ldots, {n - 1}$:
\begin{equation*}
    {u}_{k + 1}\left( 0 \right) = 0\,,\quad {u}_{k + 1}\left( \ell \right) = 0\,.
\end{equation*}

Let us define the notation for a second-order differential operator
\begin{equation}\label{eq:second_order_diff_operator}
    {\L}_{0} = -\frac{\d ^{2}}{\d {x}^{2}}\,,\quad D\left( {\L}_{0} \right) = \left\{ u\left( x \right) \in {C}^{2}\left( \left[ 0,\ell \right] \right) \mid u\left( 0 \right) = u\left( \ell \right) = 0 \right\}\,.
\end{equation}
With the use of the notation \eqref{eq:second_order_diff_operator}, equation \eqref{eq:semidiscrete_scheme} can be expressed as follows
\begin{equation}\label{eq:semidiscrete_scheme_operator_form}
    \frac{{\Delta}^{2}{u}_{k - 1}\left( x \right)}{{\tau}^{2}} + \frac{1}{2}{q}_{k}\left( {\L}_{0}{u}_{k + 1}\left( x \right) + {\L}_{0}{u}_{k - 1}\left( x \right) \right) = {f}_{k}\left( x \right)\,,\quad k = 1, 2, \ldots, {n - 1}\,,
\end{equation}
it can be easily deduced that through the integration by parts ${q}_{k}$ can be expressed in the following form using the notation \eqref{eq:second_order_diff_operator}, \ie
\begin{equation*}
    {q}_{k} = {\alpha}_{k} + {\beta}_{k}\left( {\L}_{0}{u}_{k}, {u}_{k} \right)\,,\quad {\alpha}_{k} = \alpha\left( {t}_{k} \right)~\text{and}~{\beta}_{k} = \beta\left( {t}_{k} \right)\,.
\end{equation*}
Throughout the text and in the subsequent discussions, $\left( {\cdot},{\cdot} \right)$ represents the inner product in ${L}_{2}\left( 0,\ell \right)$ and the associated norm is denoted by $\left\| {\cdot} \right\|$.

The extension of the operator ${\L}_{0}$ to a self-adjoint one is denoted by ${\L}$ (${\L}_{0} \subset {\L}$). By using this operator ${\L}$, the equation \eqref{eq:semidiscrete_scheme_operator_form} can be rewritten in the following manner, {\ie}
\begin{equation}\label{eq:semidiscrete_scheme_extension_operator_form}
    \frac{{\Delta}^{2}{u}_{k - 1}\left( x \right)}{{\tau}^{2}} + \frac{1}{2}{{\tilde q}_{k}}\left( {\L}{u}_{k + 1}\left( x \right) + {\L}{u}_{k - 1}\left( x \right) \right) = {f}_{k}\left( x \right)\,,\quad k = 1, 2, \ldots, {n - 1}\,,
\end{equation}
where
\begin{equation*}
    {\tilde q}_{k} = {\alpha}_{k} + {\beta}_{k}{\left\| {\L}^{\nicefrac{1}{2}} {u}_{k} \right\|}^{2}\,,\quad {\alpha}_{k} = \alpha\left( {t}_{k} \right)~\text{and}~{\beta}_{k} = \beta\left( {t}_{k} \right)\,.
\end{equation*}

\section{Error estimation of the approximate solution resulting from the temporal discretization}\label{sec:error_estimate}
\subsection{Auxiliary lemmata}\label{subsec:auxiliary_lemmata}
Prior to evaluating the error of the approximate solution of problem \eqref{eq:main_eqt}-\eqref{eq:boundary_conds}, it is necessary to consider a simple but important lemma that is crucial to fulfilling our objective. This lemma serves as a discrete analogue of the well-known {Gr\"{o}nwall}-type inequality. Numerous authors have considered various variations of this lemma ({\cf} \cite{BookDiffElaydi2005,emmrich1999}), and we present one such version.
\begin{lemma}[\textbf{Discrete {Gr\"{o}nwall}-type inequality}]\label{lemma:gronwall-inequality}
    Let $\left\{ {\varepsilon}_{k} \right\}$, $\left\{ {a}_{k} \right\}$ and $\left\{ {h}_{k} \right\}$ be sequences of nonnegative real numbers that satisfy the following inequality
    \begin{equation}\label{eq:aux_lemma_gronwall_main_inqt}
        {\varepsilon}_{k + 1} \leq \sum\limits_{i = 1}^{k}{{a}_{i}{\varepsilon}_{i}} + \sum\limits_{i = 0}^{k}{{h}_{i}}\,,
    \end{equation}
    then
    \begin{equation}\label{eq:aux_lemma_gronwall_inqt_to_proof}
        {\varepsilon}_{k + 1} \leq \sum\limits_{i = 1}^{k}{{\alpha}_{i,k}{h}_{i - 1}} + {h}_{k}\,,\quad {\alpha}_{i,k} = {\left( 1 + {a}_{i} \right)}{\cdots}{\left( 1 + {a}_{k} \right)}\,,\quad {i}\leq{k}\,.
    \end{equation}
\end{lemma}
\begin{proof}
    In order to prove this lemma, mathematical induction is applied. Before proceeding to the proof, it is advisable to note that ${\alpha}_{i,k} = {\alpha}_{i, i}\cdots{\alpha}_{k,k} = {\alpha}_{i,j}{\alpha}_{j + 1,k}$, ${1} \leq {i} \leq {j} \leq {k - 1}$. The validity of the statement \eqref{eq:aux_lemma_gronwall_inqt_to_proof} for ${k} = {0}$ is self-evident. This follows from \eqref{eq:aux_lemma_gronwall_main_inqt}, which provides evidence for the assertion that ${\varepsilon}_{1} \leq {h}_{0}$.
    
    Using \eqref{eq:aux_lemma_gronwall_main_inqt} for ${k} = {1}$, the following inequality can be derived through the application of the previously established inequality
    \begin{equation*}
        {\varepsilon}_{2} \leq {a}_{1}{\varepsilon}_{1} + {h}_{0} + {h}_{1} \leq \left( 1 + {a}_{1} \right){h}_{0} + {h}_{1} = {\alpha}_{1,1}{h}_{0} + {h}_{1}\,.
    \end{equation*}
    % In accordance with the preceding cases, it is possible to derive the following conclusion for ${k} = {2}$,
    % \begin{align*}
    %     {\varepsilon}_{3} &\leq {a}_{1}{\varepsilon}_{1} + {a}_{2}{\varepsilon}_{2} + {h}_{0} + {h}_{1} + {h}_{2} \leq {a}_{1}{h}_{0} + {a}_{2}\left( {\alpha}_{1,1}{h}_{0} + {h}_{1} \right) + {h}_{0} + {h}_{1} + {h}_{2} \\
    %     &= {a}_{2}{\alpha}_{1,1}{h}_{0} + {a}_{1}{h}_{0} + {a}_{2}{h}_{1} + {h}_{0} + {h}_{1} + {h}_{2} \\
    %     &= {a}_{2}{\alpha}_{1,1}{h}_{0} + \left( 1 + {a}_{1} \right){h}_{0} + \left( 1 + {a}_{2} \right){h}_{1} + {h}_{2} \\
    %     &= {a}_{2}{\alpha}_{1,1}{h}_{0} + {\alpha}_{1,1}{h}_{0} + {\alpha}_{2,2}{h}_{1} + {h}_{2} = \left( {\alpha}_{1,1} + {a}_{2}{\alpha}_{1,1} \right){h}_{0} + {\alpha}_{2,2}{h}_{1} + {h}_{2} \\
    %     &= {\alpha}_{1,2}{h}_{0} + {\alpha}_{2,2}{h}_{1} + {h}_{2}\,.
    % \end{align*}
    Let us consider the assumption that the inequality \eqref{eq:aux_lemma_gronwall_inqt_to_proof} holds true for any ${k} \leq {m - 1}$ (${m} \geq {1}$). Our objective is to demonstrate that \eqref{eq:aux_lemma_gronwall_inqt_to_proof} is also valid for ${k} = {m}$. As per \eqref{eq:aux_lemma_gronwall_main_inqt}, it follows that
    \begin{align*}
        {\varepsilon}_{m + 1} &\leq \sum\limits_{i = 1}^{m}{{a}_{i}{\varepsilon}_{i}} + \sum\limits_{i = 0}^{m}{{h}_{i}} = {a}_{1}{\varepsilon}_{1} + \sum\limits_{i = 2}^{m}{{a}_{i}{\varepsilon}_{i}} + \sum\limits_{i = 1}^{m}{{h}_{i - 1}} + {h}_{m} \\
        &\leq {a}_{1}{h}_{0} + \sum\limits_{i = 2}^{m}{{a}_{i}\left( \sum\limits_{j = 1}^{i - 1}{{\alpha}_{j,i - 1}{h}_{j - 1}} + {h}_{i - 1} \right)} + \sum\limits_{i = 1}^{m}{{h}_{i - 1}} + {h}_{m} \\
        &= \sum\limits_{i = 2}^{m}{\sum\limits_{j = 1}^{i - 1}{{a}_{i}{\alpha}_{j,i - 1}{h}_{j - 1}}} + \sum\limits_{i = 1}^{m}{{\alpha}_{i,i}{h}_{i - 1}} + {h}_{m} \\
        &= \sum\limits_{j = 1}^{m - 1}{\sum\limits_{i = j + 1}^{m}{{a}_{i}{\alpha}_{j,i - 1}{h}_{j - 1}}} + \sum\limits_{j = 1}^{m - 1}{{\alpha}_{j,j}{h}_{j - 1}} + {\alpha}_{m,m}{h}_{m - 1} + {h}_{m} \\
        &= \sum\limits_{j = 1}^{m - 1}{\left( {\alpha}_{j,j} + \sum\limits_{i = j + 1}^{m}{{a}_{i}{\alpha}_{j,i - 1}} \right){h}_{j - 1}} + {\alpha}_{m,m}{h}_{m - 1} + {h}_{m} \\
        &= \sum\limits_{j = 1}^{m - 1}{\left( 1 + {a}_{m} \right){\alpha}_{j,m - 1}{h}_{j - 1}} + {\alpha}_{m,m}{h}_{m - 1} + {h}_{m} \\
        &= \sum\limits_{j = 1}^{m}{{\alpha}_{j,m}{h}_{j - 1}} + {h}_{m}\,.
    \end{align*}
    % \begin{align*}
    %     \sum\limits_{i = 2}^{m}{\sum\limits_{j = 1}^{i - 1}}{{x}_{i,j}} = &{x}_{2,1} \\
    %     + &{x}_{3,1} + {x}_{3,2} \\
    %     + &{x}_{4,1} + {x}_{4,2} + {x}_{4,3} \\
    %     + &{x}_{5,1} + {x}_{5,2} + {x}_{5,3} + {x}_{5,4} \\
    %     + &\ldots \\
    %     + &{x}_{m,1} + {x}_{m,2} + {x}_{m,3} + {x}_{m,4} + {x}_{m,4} + \ldots + {x}_{m,m - 1} \\
    %     = &\sum\limits_{j = 1}^{m - 1}{\sum\limits_{i = j + 1}^{m}}{{x}_{i,j}}\,.
    % \end{align*}
    Ultimately, the following conclusion can be drawn
    \begin{equation*}
        {\varepsilon}_{m + 1} \leq \sum\limits_{i = 1}^{m}{{\alpha}_{i,m}{h}_{i - 1}} + {h}_{m}\,.
    \end{equation*}
    Hence, through the application of mathematical induction, it can be established that the inequality \eqref{eq:aux_lemma_gronwall_inqt_to_proof} holds for all values of ${k} \geq {0}$.
\end{proof}
\begin{remark}\label{rmk:remark1}
    The application of the well-established inequality between the arithmetic and geometric means (the \uppercase{am}-\uppercase{gm} inequality) allows us to arrive at the following conclusion
    \begin{align*}
        {\alpha}_{i,k} &= \prod\limits_{j = i}^{k}{\left( 1 + {a}_{j} \right)} \leq {\left( {1} + \frac{1}{k - i + 1}\sum\limits_{j = i}^{k}{{a}_{j}} \right)}^{k - i + 1} \leq \exp{\left( \sum\limits_{j = i}^{k}{{a}_{j}} \right)} \leq {\e}^{{\nu}_{k}}\,,\quad {\nu}_{k} = \sum\limits_{j = 1}^{k}{{a}_{j}}\,.
    \end{align*}
    It is worth mentioning that if the series $\displaystyle \sum\limits_{j = 1}^{\infty}{{a}_{j}}$ converges, then the following estimate holds true
    \begin{equation*}
        {\alpha}_{i,k} \leq {\e}^{{\nu}}\,,\quad {\nu} = \sum\limits_{j = 1}^{\infty}{{a}_{j}}\,.
    \end{equation*}
\end{remark}

We formulate the following lemma (a detailed proof of this lemma can be found in \cite[\textbf{Lemma 3.2}]{RogavaTsiklauri_LocConvg2012}) while maintaining this paper self-contained.
\begin{lemma}[see the \textbf{Lemma 3.2} in \cite{RogavaTsiklauri_LocConvg2012}]\label{lemma:rogava-tsiklauri}
    Let the sequences of nonnegative numbers, ${\left\{ {\alpha}_{k} \right\}}_{k = 0}^{n}$ and ${\left\{ {c}_{k} \right\}}_{k = 0}^{n}$, be such that they fulfill the following inequality:
    \begin{equation*}
        {\alpha}_{k + 1} \leq {\alpha}_{k}\left( 1 + {\tau}{\alpha}_{k}^{s} \right) + {\tau}{c}_{k}\,,
    \end{equation*}
    where ${s} > 0$ and ${\tau} > 0$.

    Therefore, the estimate is valid
    \begin{equation*}
        {\alpha}_{k} \leq \frac{\alpha}{{\left( 1 - {s}{\alpha}^{s}{t}_{k}{a}_{k} \right)}^{\nicefrac{1}{s}}}\,,\quad {t}_{k} = {k}{\tau} < \frac{1}{{s}{\alpha}^{s}{a}_{k}}\,,\quad {\alpha} = \max\left( 1,{\alpha}_{0} \right)\,,\quad {a}_{k} = 1 + \max\limits_{{0} \leq {i} \leq {k}}{\left( {c}_{i} \right)}\,.
    \end{equation*}
\end{lemma}

\subsection{Main lemmata}\label{subsec:main_lemmata}
It is noteworthy that, throughout the text, the following letters ${\c}$ and ${\M}$ enumerated with lower indices represent positive constants.
\begin{lemma}\label{lemma:lemma1}
    The sequences of functions $\left( {u}_{k}\left( x \right) - {u}_{k - 1}\left( x \right) \right) {/} {\tau}$ and ${\L}^{\nicefrac{1}{2}}{u}_{k}\left( x \right)$ are uniformly bounded with respect to the ${L}_{2}$-norm, {\ie}, there exist constants ${\M}_{1}$ and ${\M}_{2}$, which are independent of $\tau$, such that the following inequalities hold:
    \begin{equation*}
        \left\| \frac{{u}_{k} - {u}_{k - 1}}{\tau} \right\| \leq {\M}_{1}\,,\quad \left\| {\L}^{\nicefrac{1}{2}}{u}_{k} \right\| \leq {\M}_{2}\,,\quad {k} = 1, 2, \ldots, n\,.
    \end{equation*}
\end{lemma}
%
\begin{proof}
    By taking the inner product of both sides of equation \eqref{eq:semidiscrete_scheme_extension_operator_form} with ${u}_{k + 1} - {u}_{k - 1} = \Delta{u}_{k} + \Delta{u}_{k - 1}$ and applying integration by parts, we can obtain
    \begin{equation}\label{eq:lemma1_inner_product_eqt}
        {\left\| \frac{\Delta{u}_{k}}{\tau} \right\|}^{2} + \frac{1}{2}{{\tilde q}_{k}}{\left\| {\L}^{\nicefrac{1}{2}}{u}_{k + 1} \right\|}^{2} = {\left\| \frac{\Delta{u}_{k - 1}}{\tau} \right\|}^{2} + \frac{1}{2}{{\tilde q}_{k}}{\left\| {\L}^{\nicefrac{1}{2}}{u}_{k - 1} \right\|}^{2} + \left( {f}_{k},\Delta{u}_{k} \right) + \left( {f}_{k},\Delta{u}_{k - 1} \right)\,,
    \end{equation}
    recall that,
    \begin{gather*}
        \Delta{u}_{k} = {u}_{k + 1} - {u}_{k}\,,\\
        {\tilde q}_{k} = {\alpha}_{k} + {\beta}_{k}{\left\| {\L}^{\nicefrac{1}{2}} {u}_{k} \right\|}^{2}\,,\quad {\alpha}_{k} = \alpha\left( {t}_{k} \right)~\text{and}~{\beta}_{k} = \beta\left( {t}_{k} \right)\,.
    \end{gather*}
    Let us denote
    \begin{equation*}
        {\mu}_{k} = {\left\| \frac{\Delta{u}_{k - 1}}{\tau} \right\|}^{2}\,,\quad {\gamma}_{k} = {\left\| {\L}^{\nicefrac{1}{2}}{u}_{k} \right\|}^{2}\,,
    \end{equation*}
    and
    \begin{equation*}
        {\tilde q}_{k} = {\alpha}_{k} + {\beta}_{k}{\gamma}_{k}\,.
    \end{equation*}
    Using these notations, the equality \eqref{eq:lemma1_inner_product_eqt} should be written in that way
    \begin{equation*}
        {\mu}_{k + 1} + \frac{1}{2}\left( {\alpha}_{k} + {\beta}_{k}{\gamma}_{k} \right){\gamma}_{k + 1} = {\mu}_{k} + \frac{1}{2}\left( {\alpha}_{k} + {\beta}_{k}{\gamma}_{k} \right){\gamma}_{k - 1} + \left( {f}_{k},\Delta{u}_{k} \right) + \left( {f}_{k},\Delta{u}_{k - 1} \right)\,.
    \end{equation*}
    By utilizing the Cauchy-Schwarz inequality on the right-hand side of the aforementioned equation, it can be deduced that
    \begin{equation*}
        {\mu}_{k + 1} + \frac{1}{2}\left( {\alpha}_{k} + {\beta}_{k}{\gamma}_{k} \right){\gamma}_{k + 1} \leq {\mu}_{k} + \frac{1}{2}\left( {\alpha}_{k} + {\beta}_{k}{\gamma}_{k} \right){\gamma}_{k - 1} + {\tau}\left( \sqrt{{\mu}_{k + 1}} + \sqrt{{\mu}_{k}} \right)\left\| {f}_{k} \right\|\,.
    \end{equation*}
    The following notations are introduced:
    \begin{align*}
        {\lambda}_{k} &= {\mu}_{k} + \frac{1}{2}\left( {\alpha}_{k - 1} + {\beta}_{k - 1}{\gamma}_{k - 1} \right){\gamma}_{k}\,,\\
        {\varepsilon}_{k} &= \left( {\xi}_{k} - {\xi}_{k + 1} \right) + {\eta}_{k} + {\tau}\left( \sqrt{{\mu}_{k}} + \sqrt{{\mu}_{k + 1}} \right)\left\| {f}_{k} \right\|\,,\\
        {\xi}_{k} &= \frac{1}{2}{\alpha}_{k}{\gamma}_{k - 1}\,,\\
        {\eta}_{k} &= \frac{1}{2}\left( {\alpha}_{k + 1} - {\alpha}_{k - 1} \right){\gamma}_{k} + \frac{1}{2}\left( {\beta}_{k} - {\beta}_{k - 1} \right){\gamma}_{k - 1}{\gamma}_{k}\,.
    \end{align*}
    We shall rewrite the inequality mentioned above by using the introduced notations, indeed we derive
    \begin{equation}\label{eq:lemma1_main_inequality}
        {\lambda}_{k + 1} \leq {\lambda}_{k} + {\varepsilon}_{k}\,.
    \end{equation}
    
    It is assumed that ${\alpha}\left( t \right)$ and ${\beta}\left( t \right)$ are continuous and continuously differentiable functions over the interval $t \in \left[ 0,T \right]$, with ${\alpha}\left( t \right) \geq {\c}_{0} > 0$ and ${\beta}\left( t \right) \geq {\c}_{1} > 0$. Based on these conditions, the following estimations hold:
    \begin{equation}\label{eq:lemma1_alpha_ineqt}
        \left| {\alpha}_{k + 1} - {\alpha}_{k - 1} \right| \leq \int\limits_{{t}_{k - 1}}^{{t}_{k + 1}}{\left| {\alpha}^{\prime}\left( t \right) \right|}{\d{t}} \leq 2{{\c}_{2}}{\tau}\,,\quad {\c}_{2} = \max\limits_{0 \leq t \leq T}{\left| {\alpha}^{\prime}\left( t \right) \right|}\,,
    \end{equation}
    analogously,
    \begin{equation}\label{eq:lemma1_beta_ineqt}
        \left| {\beta}_{k} - {\beta}_{k - 1} \right| \leq {{\c}_{3}}{\tau}\,,\quad {\c}_{3} = \max\limits_{0 \leq t \leq T}{\left| {\beta}^{\prime}\left( t \right) \right|}\,.
    \end{equation}
    By considering \eqref{eq:lemma1_alpha_ineqt} and \eqref{eq:lemma1_beta_ineqt}, it is possible to reach the following conclusion
    \begin{align}\label{eq:lemma1_eta_ineqt}
        \left| {\eta}_{k} \right| &\leq \frac{1}{2}\left| {\alpha}_{k + 1} - {\alpha}_{k - 1} \right|{\gamma}_{k} + \frac{1}{2}\left| {\beta}_{k} - {\beta}_{k - 1} \right|{\gamma}_{k - 1}{\gamma}_{k}\nonumber \\
        &\leq {\tau}\left( {\c}_{2} + \frac{1}{2}{\c}_{3}{\gamma}_{k - 1} \right){\gamma}_{k} \leq \max\left( {\c}_{2},\frac{1}{2}{\c}_{3} \right){\tau}\left( 1 + {\gamma}_{k - 1} \right){\gamma}_{k}\nonumber \\
        &= \max\left( {\c}_{2},\frac{1}{2}{\c}_{3} \right){\tau}\left( \frac{{\alpha}_{k - 1}}{{\alpha}_{k - 1}} + \frac{{\beta}_{k - 1}}{{\beta}_{k - 1}}{\gamma}_{k - 1} \right){\gamma}_{k}\nonumber \\
        &\leq \max\left( {\c}_{2},\frac{1}{2}{\c}_{3} \right){\tau}\left( \frac{{\alpha}_{k - 1}}{{\c}_{0}} + \frac{{\beta}_{k - 1}}{{\c}_{1}}{\gamma}_{k - 1} \right){\gamma}_{k}\nonumber \\
        &\leq {2}\max\left( {\c}_{2},\frac{1}{2}{\c}_{3} \right)\max\left( \frac{1}{{\c}_{0}},\frac{1}{{\c}_{1}} \right){\tau}\left[ \frac{1}{2}\left( {\alpha}_{k - 1} + {\beta}_{k - 1}{\gamma}_{k - 1} \right){\gamma}_{k} \right]\nonumber \\
        &= {\c}_{4}{\tau} \left[ \frac{1}{2}\left( {\alpha}_{k - 1} + {\beta}_{k - 1}{\gamma}_{k - 1} \right){\gamma}_{k} \right] \leq {\c}_{4}{\tau}\left[ {\mu}_{k} + \frac{1}{2}\left( {\alpha}_{k - 1} + {\beta}_{k - 1}{\gamma}_{k - 1} \right){\gamma}_{k} \right] = {\c}_{4}{\tau}{\lambda}_{k}\,,
    \end{align}
    where
    \begin{equation*}
        {\c}_{4} = {2}\max\left( {\c}_{2},\frac{1}{2}{\c}_{3} \right)\max\left( \frac{1}{{\c}_{0}},\frac{1}{{\c}_{1}} \right)\,.
    \end{equation*}
    In accordance with equation \eqref{eq:lemma1_eta_ineqt}, it can be deduced that ${\varepsilon}_{k} \leq {\tilde \varepsilon}_{k}$, where
    \begin{equation*}
        {\tilde \varepsilon}_{k} = \left( {\xi}_{k} - {\xi}_{k + 1} \right) + {\c}_{4}{\tau}{\lambda}_{k} + {\tau}\left( \sqrt{{\mu}_{k}} + \sqrt{{\mu}_{k + 1}} \right)\left\| {f}_{k} \right\|\,.
    \end{equation*}
    As a result, we arrive at the following inequality
    \begin{equation}\label{eq:lemma1_lambda_varepsilon_inqt}
        {\lambda}_{k + 1} \leq {\lambda}_{k} + {\varepsilon}_{k} \leq {\lambda}_{k} + {\tilde \varepsilon}_{k}\,.
    \end{equation}
    Through the application of a telescoping series cancellation technique on inequality \eqref{eq:lemma1_lambda_varepsilon_inqt}, it follows that
    \begin{align*}
        {\lambda}_{k + 1} \leq {\lambda}_{1} + \sum\limits_{i = 1}^{k}{\tilde \varepsilon}_{i} &= {\lambda}_{1} + \sum\limits_{i = 1}^{k}\left( {\xi}_{i} - {\xi}_{i + 1} \right) + {\c}_{4}{\tau}\sum\limits_{i = 1}^{k}{\lambda}_{i} + {\tau}\sum\limits_{i = 1}^{k}\left( \sqrt{{\mu}_{i}} + \sqrt{{\mu}_{i + 1}} \right)\left\| {f}_{i} \right\| \\
        &= {\lambda}_{1} + \left( {\xi}_{1} - {\xi}_{k + 1} \right) + {\c}_{4}{\tau}\sum\limits_{i = 1}^{k}{\lambda}_{i} + {\tau}\sum\limits_{i = 1}^{k}\left( \sqrt{{\mu}_{i}} + \sqrt{{\mu}_{i + 1}} \right)\left\| {f}_{i} \right\|\,.
    \end{align*}
    Subsequently, by rearranging the terms, it is obtained that
    \begin{equation*}
        {\lambda}_{k + 1} + {\xi}_{k + 1} \leq {\lambda}_{1} + {\xi}_{1} + {\c}_{4}{\tau}\sum\limits_{i = 1}^{k}{\lambda}_{i} + {\tau}\sum\limits_{i = 1}^{k}\left( \sqrt{{\mu}_{i}} + \sqrt{{\mu}_{i + 1}} \right)\left\| {f}_{i} \right\|\,.
    \end{equation*}
    Let us denote ${\delta}_{k} = \sqrt{{\lambda}_{k} + {\xi}_{k}}$, it is clear that this yields
    \begin{equation*}
        {\delta}_{k + 1}^{2} \leq {\delta}_{1}^{2} + {\c}_{4}{\tau}\sum\limits_{i = 1}^{k}{\delta}_{i}^{2} + {\tau}\sum\limits_{i = 1}^{k}\left( {\delta}_{i} + {\delta}_{i + 1} \right)\left\| {f}_{i} \right\|\,.
    \end{equation*}
    Assuming that ${\delta}_{j} = \max\limits_{1 \leq i \leq k + 1}{\delta}_{i}$. it follows (as demonstrated in \cite{RogavaTsiklauri_EvolEqt2014}) that
    \begin{align*}
        {\delta}_{j}^{2} &\leq {\delta}_{1}^{2} + {\c}_{4}{\tau}\sum\limits_{i = 1}^{j - 1}{\delta}_{i}^{2} + {\tau}\sum\limits_{i = 1}^{j - 1}\left( {\delta}_{i} + {\delta}_{i + 1} \right)\left\| {f}_{i} \right\|\,.
    \end{align*}
    Dividing both sides of the above inequality by ${\delta}_{j}$ gives us
    \begin{align*}
        {\delta}_{j} &\leq \frac{{\delta}_{1}}{{\delta}_{j}}{\delta}_{1} + {\c}_{4}{\tau}\sum\limits_{i = 1}^{j - 1}\frac{{\delta}_{i}}{{\delta}_{j}}{\delta}_{i} + {\tau}\sum\limits_{i = 1}^{j - 1}\left( \frac{{\delta}_{i}}{{\delta}_{j}} + \frac{{\delta}_{i + 1}}{{\delta}_{j}} \right)\left\| {f}_{i} \right\| \\
        &\leq {\delta}_{1} + {\c}_{4}{\tau}\sum\limits_{i = 1}^{j - 1}{\delta}_{i} + {2}{\tau}\sum\limits_{i = 1}^{j - 1}\left\| {f}_{i} \right\| \\
        &\leq {\delta}_{1} + {\c}_{4}{\tau}\sum\limits_{i = 1}^{k}{\delta}_{i} + {2}{\tau}\sum\limits_{i = 1}^{k}\left\| {f}_{i} \right\|\,.
    \end{align*}
    Due to the fact that ${\delta}_{k + 1} \leq {\delta}_{j}$, it can be concluded that
    \begin{equation*}
        {\delta}_{k + 1} \leq {\delta}_{1} + {\c}_{4}{\tau}\sum\limits_{i = 1}^{k}{\delta}_{i} + {2}{\tau}\sum\limits_{i = 1}^{k}\left\| {f}_{i} \right\|\,.
    \end{equation*}
    Thus, on account of the application of \hyperref[lemma:gronwall-inequality]{\bf Lemma \ref*{lemma:gronwall-inequality} (Discrete Gr\"{o}nwall-type inequality)} together with \hyperref[rmk:remark1]{\bf Remark \ref*{rmk:remark1}}, one can establish that
    \begin{equation}\label{eq:lemma1_gronwall_inequality}
        {\delta}_{k + 1} \leq {\e}^{{\c}_{4}{t}_{k}}\left( {\delta}_{1} + {2}{\tau}\sum\limits_{i = 1}^{k}{\left\| {f}_{i} \right\|} \right)\,.
    \end{equation}
    Considering the estimate
    \begin{equation*}
        \sum\limits_{i = 1}^{k}\left\| {f}_{i} \right\| \leq {k}\max\limits_{1 \leq i \leq k}{\left\| {f}_{i} \right\|}\,,
    \end{equation*}
    we observe that the inequality \eqref{eq:lemma1_gronwall_inequality} should be expressed as
    \begin{equation*}
        {\delta}_{k + 1} \leq {\e}^{{\c}_{4}{t}_{k}}\left( {\delta}_{1} + {2}{t}_{k}\max\limits_{1 \leq i \leq k}{\left\| {f}_{i} \right\|} \right)\,.
    \end{equation*}
    It then follows that ${\mu}_{k}$ and ${\gamma}_{k}$ are uniformly bounded.
\end{proof}
%
\begin{lemma}\label{lemma:lemma2}
    The sequences of functions ${\L}^{\nicefrac{1}{2}}\left( {u}_{k}\left( x \right) - {u}_{k - 1}\left( x \right) \right) / {\tau}$ and ${\L}{u}_{k}\left( x \right)$ are locally uniformly bounded with respect to the ${L}_{2}$-norm, {\ie}, there exists $\overline{T} > 0$ such that
    \begin{equation*}
        \left\| {\L}^{\nicefrac{1}{2}}\frac{{u}_{k} - {u}_{k - 1}}{\tau} \right\| \leq {\M}_{3}\,,\quad \left\| {\L}{u}_{k} \right\| \leq {\M}_{4}\,,\quad {k} = 1, 2, \ldots, \left[ \frac{\overline{T}}{\tau} \right]\,.
    \end{equation*}
    Here, ${\M}_{3}$ and ${\M}_{4}$ are positive constants that depend on the value of $\overline{T}$.
\end{lemma}
%
\begin{proof}
    Let us consider taking the inner product of both sides of the equation \eqref{eq:semidiscrete_scheme_extension_operator_form} with ${\L}\left(  {u}_{k + 1} - {u}_{k - 1} \right) = {\L}\left( \Delta{u}_{k} \right) + {\L}\left( \Delta{u}_{k - 1} \right)$. Upon integrating by parts, we obtain the following identity:
    \begin{align}\label{eq:lemma2_inner_product_eqt}
        {\left\| \frac{1}{\tau}{\L}^{\nicefrac{1}{2}}\left( \Delta{u}_{k} \right) \right\|}^{2} + \frac{1}{2}{{\tilde q}_{k}}{\left\| {\L}{u}_{k + 1} \right\|}^{2} &= {\left\| \frac{1}{\tau}{\L}^{\nicefrac{1}{2}}\left( \Delta{u}_{k - 1} \right) \right\|}^{2} + \frac{1}{2}{{\tilde q}_{k}}{\left\| {\L}{u}_{k - 1} \right\|}^{2}\nonumber \\
        &+ \left( {\L}^{\nicefrac{1}{2}}{f}_{k},{\L}^{\nicefrac{1}{2}}\left( \Delta{u}_{k} \right) \right) + \left( {\L}^{\nicefrac{1}{2}}{f}_{k},{\L}^{\nicefrac{1}{2}}\left( \Delta{u}_{k - 1} \right) \right)\,,
    \end{align}
    here, we assume that ${f}_{k} \in D\left( {\L}^{\nicefrac{1}{2}} \right)$.

    \noindent By applying the Cauchy-Schwarz inequality, we can derive
    \begin{align}\label{eq:lemma2_f_k_inequality}
        &\left| \left( {\L}^{\nicefrac{1}{2}}{f}_{k},{\L}^{\nicefrac{1}{2}}\left( \Delta{u}_{k} \right) \right) + \left( {\L}^{\nicefrac{1}{2}}{f}_{k},{\L}^{\nicefrac{1}{2}}\left( \Delta{u}_{k - 1} \right) \right) \right| \leq\nonumber \\
        &{\tau}\left\| {\L}^{\nicefrac{1}{2}}{f}_{k} \right\|\left( \left\| \frac{1}{\tau}{\L}^{\nicefrac{1}{2}}\left( \Delta{u}_{k} \right) \right\| + \left\| \frac{1}{\tau}{\L}^{\nicefrac{1}{2}}\left( \Delta{u}_{k - 1} \right) \right\| \right)\,.
    \end{align}
    Let us introduce the denotations
    \begin{equation*}
        {\tilde \mu}_{k} = {\left\| \frac{1}{\tau}{\L}^{\nicefrac{1}{2}}\left( \Delta{u}_{k - 1} \right) \right\|}^{2}\,,\quad {\nu}_{k} = {\left\| {\L}{u}_{k} \right\|}^{2}\,,\quad {\gamma}_{k} = {\left\| {\L}^{\nicefrac{1}{2}}{u}_{k} \right\|}^{2}\,,\quad {\sigma}_{k} = \left\| {\L}^{\nicefrac{1}{2}}{f}_{k} \right\|\,.
    \end{equation*}
    Substituting the notations introduced in the previous step and using the inequality \eqref{eq:lemma2_f_k_inequality}, we can rewrite the equality \eqref{eq:lemma2_inner_product_eqt} as follows
    \begin{equation*}
        {\tilde \mu}_{k + 1} + \frac{1}{2}\left( {\alpha}_{k} + {\beta}_{k}{\gamma}_{k} \right){\nu}_{k + 1} \leq {\tilde \mu}_{k} + \frac{1}{2}\left( {\alpha}_{k} + {\beta}_{k}{\gamma}_{k} \right){\nu}_{k - 1} + {\tau}{\sigma}_{k}\left( \sqrt{{\tilde \mu}_{k}} + \sqrt{{\tilde \mu}_{k + 1}} \right)\,.
    \end{equation*}
    Continuing from the previous step, using the notation $\displaystyle {\tilde \nu}_{k} = \frac{1}{2}\left( {\nu}_{k - 1} + {\nu}_{k} \right)$, we have
    \begin{equation*}
        {\tilde \mu}_{k + 1} + \left( {\alpha}_{k} + {\beta}_{k}{\gamma}_{k} \right){\tilde \nu}_{k + 1} \leq {\tilde \mu}_{k} + \left( {\alpha}_{k} + {\beta}_{k}{\gamma}_{k} \right){\tilde \nu}_{k} + {\tau}{\sigma}_{k}\left( \sqrt{{\tilde \mu}_{k}} + \sqrt{{\tilde \mu}_{k + 1}} \right)\,.
    \end{equation*}
    Expanding the inequality in the previous step, we obtain
    \begin{align}\label{eq:lemma2_lambda_inequality}
        {\tilde \mu}_{k + 1} + \left( {\alpha}_{k} + {\beta}_{k}{\gamma}_{k} \right){\tilde \nu}_{k + 1} &\leq {\tilde \mu}_{k} + \left( {\alpha}_{k - 1} + {\beta}_{k - 1}{\gamma}_{k - 1} \right){\tilde \nu}_{k}\nonumber\\
        &+ \left( {\alpha}_{k} - {\alpha}_{k - 1} \right){\tilde \nu}_{k} + \left( {\beta}_{k}{\gamma}_{k} - {\beta}_{k - 1}{\gamma}_{k - 1} \right){\tilde \nu}_{k} + {\tau}{\sigma}_{k}\left( \sqrt{{\tilde \mu}_{k}} + \sqrt{{\tilde \mu}_{k + 1}} \right)\,.
    \end{align}
    Evaluating the absolute value of the following difference ${\gamma}_{k} - {\gamma}_{k - 1}$ by virtue of the \hyperref[lemma:lemma1]{\bf Lemma \ref*{lemma:lemma1}}
    \begin{equation}\label{eq:lemma2_difference_of_gamma}
        \left| {\gamma}_{k} - {\gamma}_{k - 1} \right| \leq \left( \sqrt{{\gamma}_{k - 1}} + \sqrt{{\gamma}_{k}} \right){\tau}\sqrt{{\tilde \mu}_{k}} \leq {2}{\M}_{2}{\tau}\sqrt{{\tilde \mu}_{k}}\,.
    \end{equation}
    Similar to the estimation presented in the inequality \eqref{eq:lemma1_alpha_ineqt}, we can derive the following
    \begin{equation}\label{eq:lemma2_alpha_ineqt}
        \left| {\alpha}_{k} - {\alpha}_{k - 1} \right| \leq {\c}_{2}{\tau}\,.
    \end{equation}
    By applying the inequalities \eqref{eq:lemma1_beta_ineqt} and \eqref{eq:lemma2_difference_of_gamma} along with \hyperref[lemma:lemma1]{\bf Lemma \ref*{lemma:lemma1}}, one can obtain the following estimates:
    \begin{align}\label{eq:lemma2_beta_gamma}
        \left| {\beta}_{k}{\gamma}_{k} - {\beta}_{k - 1}{\gamma}_{k - 1} \right| &\leq \left| {\beta}_{k} - {\beta}_{k - 1} \right|{\gamma}_{k} + {\beta}_{k - 1}\left| {\gamma}_{k} - {\gamma}_{k - 1} \right|\nonumber \\
        &\leq {\c}_{3}{\tau}{\gamma}_{k} + {2}{\beta}_{k - 1}{\M}_{2}{\tau}\sqrt{{\tilde \mu}_{k}}\nonumber \\
        &\leq {\c}_{3}{\M}_{2}^{2}{\tau} + {2}{\beta}_{k - 1}{\M}_{2}{\tau}\sqrt{{\tilde \mu}_{k}}\nonumber \\
        &\leq \max\left( {\c}_{3}{\M}_{2}^{2},{2}{\beta}_{k - 1}{\M}_{2} \right){\tau}\left( 1 + \sqrt{{\tilde \mu}_{k}} \right)\nonumber \\
        &={\c}_{5}{\tau}\left( 1 + \sqrt{{\tilde \mu}_{k}} \right)\,.
    \end{align}
    Through the utilization of inequalities \eqref{eq:lemma2_alpha_ineqt} and \eqref{eq:lemma2_beta_gamma}, we are able to reformulate inequality \eqref{eq:lemma2_lambda_inequality} in the following manner
    \begin{align*}
        {\tilde \mu}_{k + 1} + \left( {\alpha}_{k} + {\beta}_{k}{\gamma}_{k} \right){\tilde \nu}_{k + 1} &\leq {\tilde \mu}_{k} + \left( {\alpha}_{k - 1} + {\beta}_{k - 1}{\gamma}_{k - 1} \right){\tilde \nu}_{k} \\
        &+ {\c}_{2}{\tau}{\tilde \nu}_{k} + {\c}_{5}{\tau}\left( 1 + \sqrt{{\tilde \mu}_{k}} \right){\tilde \nu}_{k} + {\tau}{\sigma}_{k}\left( \sqrt{{\tilde \mu}_{k}} + \sqrt{{\tilde \mu}_{k + 1}} \right) \\
        &= {\tilde \mu}_{k} + \left( {\alpha}_{k - 1} + {\beta}_{k - 1}{\gamma}_{k - 1} \right){\tilde \nu}_{k} + {\c}_{6}{\tau}{\tilde \nu}_{k} \\
        &+ {\c}_{5}{\tau}\sqrt{{\tilde \mu}_{k}}{\tilde \nu}_{k} + {\tau}{\sigma}_{k}\left( \sqrt{{\tilde \mu}_{k}} + \sqrt{{\tilde \mu}_{k + 1}} \right)\,,\quad {\c}_{6} = {\c}_{2} + {\c}_{5}\,.
    \end{align*}
    If we introduce the notation given by $\displaystyle {\tilde \lambda}_{k} = {\tilde \mu}_{k} + \left( {\alpha}_{k - 1} + {\beta}_{k - 1}{\gamma}_{k - 1} \right){\tilde \nu}_{k}$, we shall obtain
    \begin{equation}\label{eq:lemma2_tilde_lambda_eqt}
        {\tilde \lambda}_{k + 1} \leq {\tilde \lambda}_{k} + {\c}_{6}{\tau}{\tilde \nu}_{k} + {\c}_{5}{\tau}\sqrt{{\tilde \mu}_{k}}{\tilde \nu}_{k} + {\tau}{\sigma}_{k}\left( \sqrt{{\tilde \mu}_{k}} + \sqrt{{\tilde \mu}_{k + 1}} \right)\,.
    \end{equation}
    Let us now consider the following set of inequalities:
    \begin{gather*}
        {\tilde \mu}_{k} \leq {\tilde \lambda}_{k}\,,\quad {\tilde \nu}_{k} \leq \frac{1}{{\c}_{0}}{\tilde \lambda}_{k} = {\c}_{7}{\tilde \lambda}_{k}\,,\quad {\sigma}_{k} \leq {\c}_{8} = \max\limits_{0 \leq t \leq {T}}\left\| {\L}^{\nicefrac{1}{2}} f\left( \cdot,{t} \right) \right\|\,, \\
        \sqrt{{\tilde \mu}_{k}} \leq \frac{1}{2}\left( 1 + {\tilde \mu}_{k} \right) \leq \frac{1}{2}\left( 1 + {\tilde \lambda}_{k} \right)\,,
    \end{gather*}
    By utilizing equation \eqref{eq:lemma2_tilde_lambda_eqt}, we are able to draw the conclusion that
    \begin{align*}
        {\tilde \lambda}_{k + 1} &\leq {\tilde \lambda}_{k} + {\c}_{9}{\tau}{\tilde \lambda}_{k} + {\c}_{10}{\tau}\sqrt{{\tilde \lambda}_{k}}{\tilde \lambda}_{k} + {\c}_{8}{\tau}\sqrt{{\tilde \lambda}_{k + 1}} + \frac{1}{2}{\c}_{8}{\tau}\left( 1 + {\tilde \lambda}_{k} \right) \\
        &=\left( 1 + {\c}_{11}{\tau} + {\c}_{10}{\tau}\sqrt{{\tilde \lambda}_{k}} \right){\tilde \lambda}_{k} + {\c}_{8}{\tau}\sqrt{{\tilde \lambda}_{k + 1}} + {\c}_{12}{\tau}\,, \\
        & {\c}_{9} = {\c}_{6}{\c}_{7}\,,\quad {\c}_{10} = {\c}_{5}{\c}_{7}\,,\quad {\c}_{11} = {\c}_{9} + {\c}_{12}\,,\quad {\c}_{12} = \frac{1}{2}{\c}_{8}\,.
    \end{align*}
    Hence, we are able to conclude that
    \begin{equation*}
        {\tilde \lambda}_{k + 1} \leq {\tilde \lambda}_{k}\left( 1 + {\c}_{11}{\tau} \right)\left( 1 + \frac{{\c}_{10}}{1 + {\c}_{11}{\tau}}{\tau}\sqrt{{\tilde \lambda}_{k}} \right) + {\c}_{8}{\tau}\sqrt{{\tilde \lambda}_{k + 1}} + {\c}_{12}{\tau}
    \end{equation*}
    Let us divide both sides of the aforementioned inequality by ${\left( 1 + {\c}_{11}{\tau} \right)}^{k + 1}$; as a result, we obtain
    \begin{align*}
        \frac{{\tilde \lambda}_{k + 1}}{{\left( 1 + {\c}_{11}{\tau} \right)}^{k + 1}} &\leq \frac{{\tilde \lambda}_{k}}{{\left( 1 + {\c}_{11}{\tau} \right)}^{k}}\left( 1 + \frac{{\c}_{10}}{1 + {\c}_{11}{\tau}}{\tau}\sqrt{{\tilde \lambda}_{k}} \right) + \frac{{\c}_{8}\sqrt{{\tilde \lambda}_{k + 1}}}{{\left( 1 + {\c}_{11}{\tau} \right)}^{k + 1}}{\tau} + \frac{{\c}_{12}}{{\left( 1 + {\c}_{11}{\tau} \right)}^{k + 1}}{\tau} \\
        &= \frac{{\tilde \lambda}_{k}}{{\left( 1 + {\c}_{11}{\tau} \right)}^{k}}\left( 1 + \frac{{\c}_{10}\sqrt{{\left( 1 + {\c}_{11}{\tau} \right)}^{k}}}{1 + {\c}_{11}{\tau}}{\tau}\sqrt{\frac{{\tilde \lambda}_{k}}{{\left( 1 + {\c}_{11}{\tau} \right)}^{k}}} \right) \\
        &+\frac{{\c}_{8}}{\sqrt{{\left( 1 + {\c}_{11}{\tau} \right)}^{k + 1}}}{\tau}\sqrt{\frac{{\tilde \lambda}_{k + 1}}{{\left( 1 + {\c}_{11}{\tau} \right)}^{k + 1}}} + \frac{{\c}_{12}}{{\left( 1 + {\c}_{11}{\tau} \right)}^{k + 1}}{\tau}\,.
    \end{align*}
    By utilizing the notation $\displaystyle {\tilde \xi}_{k} = \frac{{\tilde \lambda}_{k}}{{\left( 1 + {\c}_{11}{\tau} \right)}^{k}}$, and considering the following set of plain inequalities
    \begin{gather*}
        \frac{{\c}_{10}\sqrt{{\left( 1 + {\c}_{11}{\tau} \right)}^{k}}}{1 + {\c}_{11}{\tau}} \leq {\c}_{10}\sqrt{{\e}^{{\c}_{11}{T}}} = {\c}_{13}\,, \\
        \frac{{\c}_{8}}{\sqrt{{\left( 1 + {\c}_{11}{\tau} \right)}^{k + 1}}} \leq {\c}_{8}\,,\quad\frac{{\c}_{12}}{{\left( 1 + {\c}_{11}{\tau} \right)}^{k + 1}} \leq {\c}_{12}\,,
    \end{gather*}
    As a result of the aforementioned inequalities, we are able to determine that
    \begin{equation}\label{eq:lemma2_tilde_xi_inqt}
        {\tilde \xi}_{k + 1} \leq {\tilde \xi}_{k}\left( 1 + {\c}_{13}{\tau}\sqrt{{\tilde \xi}_{k}} \right) + {\c}_{8}{\tau}\sqrt{{\tilde \xi}_{k + 1}} + {\c}_{12}{\tau}\,.
    \end{equation}
    In order to simplify inequality \eqref{eq:lemma2_tilde_xi_inqt} and express it in a more straightforward manner, let us introduce new denotations
    \begin{equation*}
        {y}_{k + 1} = \sqrt{{\tilde \xi}_{k + 1}},\quad {w}_{k} = {\tilde \xi}_{k}\left( 1 + {\c}_{13}{\tau}\sqrt{{\tilde \xi}_{k}} \right) + {\c}_{12}{\tau}\,,
    \end{equation*}
    correspondingly, we derive the following quadratic inequality for ${y}_{k + 1}$,
    \begin{equation*}
        {y}_{k + 1}^{2} - {\c}_{8}{\tau}{y}_{k + 1} - {w}_{k} \leq 0\,.
    \end{equation*}
    It should be noted that the discriminant of this quadratic polynomial is given by $\displaystyle {\c}_{8}^{2}{\tau}^{2} + {4}{w}_{k}$, which is non-negative. Therefore, we can write the following inequality
    \begin{equation*}
        {y}_{k + 1} \leq \frac{{\c}_{8}}{2}{\tau} + \sqrt{\frac{{\c}_{8}^{2}}{4}{\tau}^{2} + {w}_{k}}\,.
    \end{equation*}
    By squaring both sides of the aforementioned inequality, we can readily draw a conclusion
    \begin{equation}\label{eq:lemma2_quad_poly_y_k+1}
        {y}_{k + 1}^{2} \leq \frac{{\c}_{8}^{2}}{2}{\tau}^{2} + {\c}_{8}{\tau}\sqrt{\frac{{\c}_{8}^{2}}{4}{\tau}^{2} + {w}_{k}} + {w}_{k}\,.
    \end{equation}
    Let us evaluate the second term in the right-hand side of inequality \eqref{eq:lemma2_quad_poly_y_k+1}. We have
    \begin{align*}
        {\c}_{8}{\tau}\sqrt{\frac{{\c}_{8}^{2}}{4}{\tau}^{2} + {w}_{k}} &= \sqrt{{\c}_{8}{\tau}}\sqrt{\frac{{\c}_{8}^{3}}{4}{\tau}^{3} + {\c}_{8}{w}_{k}{\tau}} \\
        &\leq \frac{1}{2}\left( {\c}_{8}{\tau} + \frac{{\c}_{8}^{3}}{4}{\tau}^{3} + {\c}_{8}{w}_{k}{\tau} \right)\,.
    \end{align*}
    Taking the last inequality into account in \eqref{eq:lemma2_quad_poly_y_k+1}, we eventually conclude that
    \begin{align}\label{eq:lemma2_y_k+1_ineqt}
        {y}_{k + 1}^{2} &\leq \frac{{\c}_{8}}{2}{\tau} + \frac{{\c}_{8}^{2}}{2}{\tau}^{2} + \frac{{\c}_{8}^{3}}{8}{\tau}^{3} + {w}_{k} + \frac{{\c}_{8}}{2}{w}_{k}{\tau}\nonumber \\
        &= \frac{{\c}_{8}}{2}{\tau}\left( 1 + {\c}_{8}{\tau} + \frac{{\c}_{8}^{2}}{4}{\tau}^{2} \right) + \left( 1 + \frac{{\c}_{8}}{2}{\tau} \right){w}_{k}\,.
    \end{align}
    Upon reversing the substitution of $\displaystyle {y}_{k + 1} = \sqrt{{\tilde \xi}_{k + 1}}$ and $\displaystyle {w}_{k} = {\tilde \xi}_{k}\left( 1 + {\c}_{13}{\tau}\sqrt{{\tilde \xi}_{k}} \right) + {\c}_{12}{\tau}$ in inequality \eqref{eq:lemma2_y_k+1_ineqt}, we obtain the following expression
    \begin{align*}
        {\tilde \xi}_{k + 1} \leq \left( 1 + \frac{{\c}_{8}}{2}{\tau} \right){\tilde \xi}_{k}\left( 1 + {\c}_{13}{\tau}\sqrt{{\tilde \xi}_{k}} \right) &+ \left( 1 + \frac{{\c}_{8}}{2}{\tau} \right){\c}_{12}{\tau} \\
        &+ \frac{{\c}_{8}}{2}{\tau}\left( 1 + {\c}_{8}{\tau} + \frac{{\c}_{8}^{2}}{4}{\tau}^{2} \right)\,.
    \end{align*}
    By reintroducing the notations $\displaystyle {\c}_{14} = \frac{{\c}_{8}}{2}$ and $\displaystyle {\c}_{15} = \left( 1 + {\c}_{14}{\tau} \right){\c}_{12} + {\c}_{14}\left( 1 + {2}{\c}_{14}{\tau} + {\c}_{14}^{2}{\tau}^{2} \right)$, we can conclude that
    \begin{equation*}
        {\tilde \xi}_{k + 1} \leq \left( 1 + {\c}_{14}{\tau} \right){\tilde \xi}_{k}\left( 1 + {\c}_{13}{\tau}\sqrt{{\tilde \xi}_{k}} \right) + {\c}_{15}{\tau}\,.
    \end{equation*}
    Upon dividing both sides of the aforementioned inequality by ${\left( 1 + {\c}_{14}{\tau} \right)}^{k + 1}$, we obtain the following expression
    \begin{align}\label{eq:lemma2_frac_xi_k_inequality}
        \frac{{\tilde \xi}_{k + 1}}{{\left( 1 + {\c}_{14}{\tau} \right)}^{k + 1}} &\leq \frac{{\tilde \xi}_{k}}{{\left( 1 + {\c}_{14}{\tau} \right)}^{k}}\left( 1 + {\c}_{13}{\tau}\sqrt{{\tilde \xi}_{k}} \right) + \frac{{\c}_{15}}{{\left( 1 + {\c}_{14}{\tau} \right)}^{k + 1}}{\tau}\nonumber \\
        &= \frac{{\tilde \xi}_{k}}{{\left( 1 + {\c}_{14}{\tau} \right)}^{k}}\left( 1 + {\c}_{13}\sqrt{{\left( 1 + {\c}_{14}{\tau} \right)}^{k}}{\tau}\sqrt{\frac{{\tilde \xi}_{k}}{{\left( 1 + {\c}_{14}{\tau} \right)}^{k}}}\,\, \right) + \frac{{\c}_{15}}{{\left( 1 + {\c}_{14}{\tau} \right)}^{k + 1}}{\tau}\,.
    \end{align}
    The following estimates are applicable
    \begin{gather*}
        {\c}_{13}\sqrt{{\left( 1 + {\c}_{14}{\tau} \right)}^{k}} \leq {\c}_{13}\sqrt{{\e}^{{\c}_{14}{T}}} = {\c}_{16}\,,\quad \frac{{\c}_{15}}{{\left( 1 + {\c}_{14}{\tau} \right)}^{k + 1}} \leq {\c}_{15}\,.
    \end{gather*}
    By introducing the notation $\displaystyle {\zeta}_{k} = \frac{{\tilde \xi}_{k}}{{\left( 1 + {\c}_{14}{\tau} \right)}^{k}}$ and applying it along with the aforementioned inequalities in \eqref{eq:lemma2_frac_xi_k_inequality}, we can deduce the following conclusion
    \begin{equation*}
        {\zeta}_{k + 1} \leq {\zeta}_{k}\left( 1 + {\c}_{16}{\tau}\sqrt{{\zeta}_{k}} \right) + {\c}_{15}{\tau}\,.
    \end{equation*}
    Let us make a certain transformation, specifically by introducing $\displaystyle {\overline{\tau}} = {\c}_{16}{\tau}$, we have
    \begin{equation*}
        {\zeta}_{k + 1} \leq {\zeta}_{k}\left( 1 + {\overline{\tau}}\sqrt{{\zeta}_{k}} \right) + {\c}_{17}{\overline{\tau}}\,,\quad {\c}_{17} = \frac{{\c}_{15}}{{\c}_{16}}\,.
    \end{equation*}
    As a result of \hyperref[lemma:rogava-tsiklauri]{\bf Lemma \ref*{lemma:rogava-tsiklauri}}, it can be deduced that
    \begin{equation}\label{eq:lemma2_result_of_lemma_rogavatsiklauri}
        {\zeta}_{k} \leq \frac{\zeta}{{\left( 1 - \dfrac{\left( 1 + {\c}_{17} \right)\sqrt{\zeta}}{2}{\overline{t}}_{k} \right)}^{2}} \leq \frac{{\tilde \lambda}}{{\left( 1 - \dfrac{{\c}_{16}\left( 1 + {\c}_{17} \right)\sqrt{{\tilde \lambda}}}{2}{{t}_{k}} \right)}^{2}}\,,\quad {k} = 1, 2, \ldots, {m}\,,
    \end{equation}
    where
    \begin{equation*}
        {\zeta} = \max\left( 1,{\zeta}_{1} \right) \leq \max\left( 1,{\tilde \lambda}_{1} \right) = {\tilde \lambda}\,,\quad {\overline{t}}_{k} = {k}{\overline{\tau}} = {\c}_{16}{{t}_{k}} < \frac{2}{\left( 1 + {\c}_{17} \right)\sqrt{{\tilde \lambda}}} \leq \frac{2}{\left( 1 + {\c}_{17} \right)\sqrt{{\zeta}}}\,.
    \end{equation*}
    According to the introduced denotation, we have
    \begin{equation}\label{eq:eq:lemma2_conn_btwn_zeta_k_lambda_k_inqt}
        {\zeta}_{k} = \dfrac{{\tilde \lambda}_{k}}{{\left( 1 + {\c}_{11}{\tau} \right)}^{k}{\left( 1 + {\c}_{14}{\tau} \right)}^{k}} \geq \dfrac{{\tilde \lambda}_{k}}{{\e}^{\left( {\c}_{11} + {\c}_{14} \right){t}_{k}}}\,.
    \end{equation}
    The inequality \eqref{eq:lemma2_result_of_lemma_rogavatsiklauri} with the help of \eqref{eq:eq:lemma2_conn_btwn_zeta_k_lambda_k_inqt} gives us the following estimation
    \begin{equation}\label{eq:lemma2_result_of_lemma_rogavatsiklauri_tilde_lambda}
        {\tilde \lambda}_{k} \leq \frac{{\tilde \lambda}}{{\left( 1 - {\c}_{18}\sqrt{{\tilde \lambda}}{{t}_{k}} \right)}^{2}}{\e}^{{\c}_{19}{t}_{k}}\,,\quad {k} = 1, 2, \ldots, {m}\,,
    \end{equation}
    
    It is worth mentioning that the value of ${m}$ in inequality \eqref{eq:lemma2_result_of_lemma_rogavatsiklauri_tilde_lambda} is dependent on both the coefficient of ${t}_{k}$ (in the denominator of the fraction) and the number of time interval divisions, ${n}$. The coefficient of ${t}_{k}$ can be explicitly estimated through the use of data from the problem \eqref{eq:main_eqt}-\eqref{eq:boundary_conds}, taking into account $T$ as well. The inequality ${\tilde \lambda} \leq {\widetilde{\M}}$ holds, with ${\widetilde{\M}}$ being a positive constant that is dependent on $\left\| {\L}^{2}{\psi}_{0} \right\|$, $\left\| {\L}{\psi}_{1} \right\|$, $\left\| {\L}{f}_{0} \right\|$, and $T$.
    % \begin{align*}
    %     {\tilde \lambda}_{1} &= {\tilde \mu}_{1} + \left( {\alpha}_{0} + {\beta}_{0}{\gamma}_{0} \right){\tilde \nu}_{0} \\
    %     &= {\left\| \frac{1}{\tau}{\L}^{\nicefrac{1}{2}}\left( {u}_{1} - {u}_{0} \right) \right\|}^{2} + \frac{1}{2}\left( {\alpha}_{0} + {\beta}_{0}{\left\| {\L}^{\nicefrac{1}{2}}{u}_{0} \right\|}^{2} \right)\left( {\left\| {\L}{u}_{0} \right\|}^{2} + {\left\| {\L}{u}_{1} \right\|}^{2} \right)\,.
    % \end{align*}
    % \begin{equation*}
    %     {\tilde \mu}_{k} = {\left\| \frac{1}{\tau}{\L}^{\nicefrac{1}{2}}\left( \Delta{u}_{k - 1} \right) \right\|}^{2}\,,\quad {\nu}_{k} = {\left\| {\L}{u}_{k} \right\|}^{2}\,,\quad {\gamma}_{k} = {\left\| {\L}^{\nicefrac{1}{2}}{u}_{k} \right\|}^{2}\,,\quad {\tilde \nu}_{k} = \frac{1}{2}\left( {\nu}_{k - 1} + {\nu}_{k} \right)\,.
    % \end{equation*}
    
    From \eqref{eq:lemma2_result_of_lemma_rogavatsiklauri_tilde_lambda}, the following inequality follows:
    \begin{equation}\label{eq:lemma2_final_result_loc_bound}
        {\tilde \lambda}_{k} \leq \dfrac{\widetilde{\M}}{{\left( 1 - \overline{\M}{\,}\overline{T} \right)}^{2}}{\e}^{{\c}_{19}\overline{T}}\,,\quad {k} = 1, 2, \ldots, \left[ \frac{\overline{T}}{{\tau}} \right]\,,
    \end{equation}
    where $\overline{\M} = {\c}_{18}\sqrt{\widetilde{\M}}$, $\overline{T} = \dfrac{q}{\overline{\M}}$, ${0} < {q} < {1}$.

    The inequality \eqref{eq:lemma2_final_result_loc_bound} implies the uniform boundedness of the vectors ${\L}^{\nicefrac{1}{2}}\left( {u}_{k}\left( x \right) - {u}_{k - 1}\left( x \right) \right) / {\tau}$ and ${\L}{u}_{k}\left( x \right)$ over the local interval $\left[ {0}, \overline{T} \right]$.
\end{proof}
%
\begin{remark}\label{rmk:remark-lemma1}
    It is evident that when the sequences of functions ${u}_{k}\left( x \right) \in D\left( {\L}_{0} \right)$, from the second estimate of \hyperref[lemma:lemma1]{\bf Lemma \ref*{lemma:lemma1}} follows that the norm of the derivative of ${u}_{k}$ with respect to $x$ is bounded above by a constant ${\M}_{2}$, that is
    \begin{equation*}
        \left\| \frac{\d {u}_{k}}{\d x} \right\| \leq {\M}_{2}\,.
    \end{equation*}
\end{remark}
%
\subsection{Error estimate of the approximate solution}\label{subsec:error_estimate}
It should be noted that throughout the text, the index $t$ is occasionally omitted in expressing the derivative of the function ${u}\left( x,t \right)$ with respect to the temporal variable $t$. Specifically, the notation ${u}^{\prime}\left( x,t \right)$ is used to represent the first-order derivative of the function ${u}\left( x,t \right)$ with respect to its second argument, rather than ${u}_{{t}}^{\prime}\left( x,t \right)$ or ${u}_{{t}}\left( x,t \right)$. When mixed partial derivatives of the function ${u}\left( x,t \right)$ with respect to the spatial and the temporal variables are encountered, lower indices ${x}$ and ${t}$ are used to denote the appropriate partial derivatives. For instance, ${u}_{{x}{x}{t}}\left( x,t \right)$ represents the second-order and the first-order partial derivatives of the function ${u}\left( x,t \right)$ with respect to the spatial and the temporal variables, respectively.

Before presenting the theorem on the convergence of the scheme \eqref{eq:semidiscrete_scheme}, we first provide a remark on the smoothness of the solution to problem \eqref{eq:main_eqt}-\eqref{eq:boundary_conds} to ascertain the order of convergence of the proposed symmetric three-layer semi-discrete scheme \eqref{eq:semidiscrete_scheme}. To ensure the well-posedness of the problem, a minimum degree of smoothness of the solution is required, which guarantees convergence but is insufficient to determine the order of convergence. By raising the smoothness of the solution by one degree, the order of convergence becomes equal to one (specifically, in both the previous and current scenarios, where ${u}_{1}\left( x \right) = {\psi}_{0}\left( x \right) + {\tau}{\psi}_{1}\left( x \right)$ is sufficient). Furthermore, if the smoothness degree is risen by two and the initial function is specified using formula \eqref{eq:semidscrete_scheme_first_layer}, the order of convergence is improved by an additional degree, resulting in a total order of two. However, further increasing the smoothness degree would be superfluous, as the approximation order of scheme \eqref{eq:semidiscrete_scheme} does not exceed two.

The subsequent theorem is established regarding the convergence of the scheme \eqref{eq:semidiscrete_scheme}.
%
\begin{theorem}\label{theorem:theorem1}
    Let the problem \eqref{eq:main_eqt}-\eqref{eq:boundary_conds} be well-posed. Besides, the following conditions are fulfilled:
    \begin{enumerate}[label=(\alph*)]
        \item\label{itm_theorem_a} ${\psi}_{0}\left( x \right) \in D\left( {\L}_{0} \right)$, ${\psi}_{1}\left( x \right) \in {C}^{1}\left( \left[ {0},{\ell} \right] \right)$ and the function ${u}_{{x}}\left( x,t \right)$ has a second-order continuous derivative with respect to the temporal variable.
        \item\label{itm_theorem_b} The solution ${u}\left( x,t \right)$ of the problem \eqref{eq:main_eqt}-\eqref{eq:boundary_conds} is a continuously differentiable function up to and including the third order with respect to the temporal variable, moreover, ${u}^{{\prime}{\prime}{\prime}}\left( x,t \right)$ is a Lipschitz continuous function with respect to the temporal variable.
        \item\label{itm_theorem_c} The function ${u}_{{x}{x}}\left( x,t \right)$ is continuously differentiable with respect to the temporal variable, as well ${u}_{{x}{x}{t}}\left( x,t \right)$ satisfies the Lipschitz condition with respect to the temporal variable.
    \end{enumerate}
    Then there exists $\overline{T}$ $\left( 0 < \overline{T} \leq {T} \right)$ such that for the error of an approximate solution denoted by ${z}_{k}\left( x \right)$ and defined as follows ${z}_{k}\left( x \right) = {u}\left( x,{t}_{k} \right) - {u}_{k}\left( x \right)$ the following estimates hold:
    \begin{equation*}
        \max\limits_{1 \leq k \leq m}\left\| \frac{{\d}{z}_{k}}{{\d}{x}} \right\| \leq {\M}_{5}{\tau}^{2}\,,\quad \max\limits_{0 \leq k \leq m - 1}\left\| \frac{{z}_{k + 1} - {z}_{k}}{\tau} \right\| \leq {\M}_{6}{\tau}^{2}\,,
    \end{equation*}
    where $\displaystyle {m} = \left[ \frac{\overline{T}}{{\tau}} \right]$.
\end{theorem}
%
\begin{proof}
    If we evaluate the exact representation of equation \eqref{eq:main_eqt} at ${t} = {t}_{k}$, which is expressed as equality \eqref{eq:discrete_main_eqt}, and subtract it from equation \eqref{eq:semidiscrete_scheme}, evidently for ${z}_{k}\left( x \right)$ we obtain
    \begin{equation}\label{eq:theorem_semidiscrete_scheme_for_z_k}
        \frac{{\Delta}^{2}{z}_{k - 1}\left( x \right)}{{\tau}^{2}} - \frac{1}{2}{q}_{k}\left( \frac{{\d}^{2}{z}_{k + 1}\left( x \right)}{{\d}{x}^{2}} + \frac{{\d}^{2}{z}_{k - 1}\left( x \right)}{{\d}{x}^{2}} \right) = {g}_{k}\left( x \right)\,,\quad {k} = 1, 2, \ldots, {n - 1}\,,
    \end{equation}
    here
    \begin{align*}
        {\Delta}{z}_{k}\left( x \right) &= {z}_{k + 1}\left( x \right) - {z}_{k}\left( x \right)\,, \\
        {g}_{k}\left( x \right) &= \frac{1}{2}\left( {q}\left( {t}_{k} \right) - {q}_{k} \right)\left( \frac{{\d}^{2}{u}\left( x,{t}_{k + 1} \right)}{{\d}{x}^{2}} + \frac{{\d}^{2}{u}\left( x,{t}_{k - 1} \right)}{{\d}{x}^{2}} \right) + {R}_{k}\left( x,{\tau} \right)\,, \\
        {R}_{k}\left( x,{\tau} \right) &= {R}_{1,k}\left( x,\tau \right) + {R}_{2,k}\left( x,\tau \right) = \underbrace{\left( \frac{{\Delta}^{2}u\left( x,{t}_{k - 1} \right)}{{\tau}^{2}} - \frac{{\partial}^{2}u\left( x,{t}_{k} \right)}{\partial{t}^{2}} \right)}_{{R}_{1,k}\left( x,\tau \right)} + \underbrace{\left( - \frac{1}{2}{q}\left( {t}_{k} \right)\frac{\d ^{2}}{\d {x}^{2}}{\Delta}^{2}u\left( x,{t}_{k - 1} \right) \right)}_{{R}_{2,k}\left( x,\tau \right)}\,.
    \end{align*}
    We shall proceed to estimate the remainder term ${R}_{k}\left( x,{\tau} \right)$ in equation \eqref{eq:theorem_semidiscrete_scheme_for_z_k}. To do so, we consider the Taylor expansion of the function ${u}\left( x,t \right)$ around the point $t = {t}_{k}$. Specifically, we have
    \begin{equation}\label{eq:theorem_taylor_series_four_terms}
        {u}\left( x,t \right) = {u}\left( x,{t}_{k} \right) + {\left( t - {t}_{k} \right)}{u}^{{\prime}}\left( x,{t}_{k} \right) + \frac{{\left( t - {t}_{k} \right)}^{2}}{2}{u}^{{\prime}{\prime}}\left( x,{t}_{k} \right) + \frac{{\left( t - {t}_{k} \right)}^{3}}{6}{u}^{{\prime}{\prime}{\prime}}\left( x,{t}_{k} \right) + {\widetilde R}_{3}\left( x,t \right)\,.
    \end{equation}
    In this context, the remainder term ${\widetilde R}_{3}\left( x,t \right)$, also known as the Lagrange remainder, is given by
    \begin{equation*}
        {\widetilde R}_{3}\left( x,t \right) = \int\limits_{{t}_{k}}^{t}\int\limits_{{t}_{k}}^{{s}_{1}}\int\limits_{{t}_{k}}^{{s}_{2}}{\left( {u}^{{\prime}{\prime}{\prime}}\left( x,{s}_{3} \right) - {u}^{{\prime}{\prime}{\prime}}\left( x,{t}_{k} \right) \right)}{\d {s}_{3}}{\d {s}_{2}}{\d {s}_{1}}\,.
    \end{equation*}
    By virtue of condition \ref{itm_theorem_b} in \hyperref[theorem:theorem1]{\bf Theorem \ref*{theorem:theorem1}}, it can be simply conclude that
    \begin{equation}\label{eq:theorem_est_lagrange_rem_triple_integ}
        \max\limits_{0 \leq x \leq \ell}\left| {\widetilde R}_{3}\left( x,t \right) \right| \leq \frac{{\c}_{20}}{24}{\left( {t} - {t}_{k} \right)}^{4}\,.
    \end{equation}
    Upon substituting $t = {t}_{k - 1}$ and $t = {t}_{k + 1}$ in equality \eqref{eq:theorem_taylor_series_four_terms}, we obtain the following result
    \begin{equation*}
        \frac{{\Delta}^{2}u\left( x,{t}_{k - 1} \right)}{{\tau}^{2}} - \frac{{\partial}^{2}u\left( x,{t}_{k} \right)}{\partial{t}^{2}} = \frac{1}{{\tau}^{2}}\left( {\widetilde R}_{3}\left( x,{t}_{k - 1} \right) + {\widetilde R}_{3}\left( x,{t}_{k + 1} \right) \right)\,,
    \end{equation*}
    and from here by taking into account inequality \eqref{eq:theorem_est_lagrange_rem_triple_integ}, we obtain the following estimate
    \begin{align}\label{eq:theorem_abs_r1k}
        \max\limits_{0 \leq x \leq \ell}\left| {R}_{1,k}\left( x,\tau \right) \right| &= \max\limits_{0 \leq x \leq \ell}\left| \frac{{\Delta}^{2}u\left( x,{t}_{k - 1} \right)}{{\tau}^{2}} - \frac{{\partial}^{2}u\left( x,{t}_{k} \right)}{\partial{t}^{2}} \right| = \frac{1}{{\tau}^{2}}\max\limits_{0 \leq x \leq \ell}\left| {\widetilde R}_{3}\left( x,{t}_{k - 1} \right) + {\widetilde R}_{3}\left( x,{t}_{k + 1} \right) \right|\nonumber \\
        &\leq \frac{1}{{\tau}^{2}}\left( \max\limits_{0 \leq x \leq \ell}\left| {\widetilde R}_{3}\left( x,{t}_{k - 1} \right) \right| + \max\limits_{0 \leq x \leq \ell}\left| {\widetilde R}_{3}\left( x,{t}_{k + 1} \right) \right| \right) \leq \frac{{\c}_{20}}{12}{\tau}^{2}\,.
    \end{align}
    To obtain an estimation for ${R}_{2,k}\left( x,\tau \right)$, it is crucial to consider the Taylor series for the function ${u}\left( x,t \right)$ around the point $t = {t}_{k}$ while retaining the first two terms, {\ie}
    \begin{equation}\label{eq:theorem_taylor_series_two_terms}
        {u}\left( x,t \right) = {u}\left( x,{t}_{k} \right) + {\left( t - {t}_{k} \right)}{u}^{{\prime}}\left( x,{t}_{k} \right) + {\widetilde R}_{1}\left( x,t \right)\,,
    \end{equation}
    the remainder term ${\widetilde R}_{1}\left( x,t \right)$ can be expressed in integral form as follows
    \begin{equation*}
        {\widetilde R}_{1}\left( x,t \right) = \int\limits_{{t}_{k}}^{{t}}{\left( {u}^{{\prime}}\left( x,s \right) - {u}^{{\prime}}\left( x,{t}_{k} \right) \right)}{\d s}\,.
    \end{equation*}
    By virtue of condition \ref{itm_theorem_c} in \hyperref[theorem:theorem1]{\bf Theorem \ref*{theorem:theorem1}} the following inequality can be readily obtained
    \begin{equation}\label{eq:theorem_est_lagrange_rem_single_integ}
        \max\limits_{0 \leq x \leq \ell}\left| \frac{\d ^{2}}{\d {x}^{2}}{\widetilde R}_{1}\left( x,t \right) \right| \leq \frac{{\c}_{21}}{2}{\left( t - {t}_{k} \right)}^{2}\,.
    \end{equation}
    We substitute $t = {t}_{k - 1}$ and $t = {t}_{k + 1}$ in equality \eqref{eq:theorem_taylor_series_two_terms}, sum and arrange them, which yields
    \begin{equation*}
        {\Delta}^{2}u\left( x,{t}_{k - 1} \right) = {\widetilde R}_{1}\left( x,{t}_{k - 1} \right) + {\widetilde R}_{1}\left( x,{t}_{k + 1} \right)\,.
    \end{equation*}
    Through \eqref{eq:theorem_est_lagrange_rem_single_integ}, the last equality can be further estimated as follows
    \begin{align*}
        \max\limits_{0 \leq x \leq \ell}\left| \frac{\d ^{2}}{\d {x}^{2}}{\Delta}^{2}u\left( x,{t}_{k - 1} \right) \right| \leq \max\limits_{0 \leq x \leq \ell}\left| \frac{\d ^{2}}{\d {x}^{2}}{\widetilde R}_{1}\left( x,{t}_{k - 1} \right) \right| + \max\limits_{0 \leq x \leq \ell}\left| \frac{\d ^{2}}{\d {x}^{2}}{\widetilde R}_{1}\left( x,{t}_{k + 1} \right) \right| \leq {\c}_{21}{\tau}^{2}\,,
    \end{align*}
    from here
    \begin{equation}\label{eq:theorem_abs_r2k}
        \max\limits_{0 \leq x \leq \ell}\left| {R}_{2,k}\left( x,\tau \right) \right| = \frac{1}{2}{q}\left( {t}_{k} \right) \max\limits_{0 \leq x \leq \ell}\left| \frac{\d ^{2}}{\d {x}^{2}}{\Delta}^{2}u\left( x,{t}_{k - 1} \right) \right| \leq \frac{1}{2}{q}\left( {t}_{k} \right){\c}_{21}{\tau}^{2}\,.
    \end{equation}
    Thus, by combining \eqref{eq:theorem_abs_r1k} and \eqref{eq:theorem_abs_r2k}, we arrive at the following result
    \begin{align}\label{eq:theorem_main_remainder_term}
        \left\| {R}_{k}\left( x,{\tau} \right) \right\| &\leq \left\| {R}_{1,k}\left( x,{\tau} \right) \right\| + \left\| {R}_{2,k}\left( x,{\tau} \right) \right\| \nonumber \\
        &\leq \frac{\sqrt{\ell}}{2}\left( \frac{{\c}_{20}}{6} + {\max\limits_{0 \leq t \leq {T}}}{q}\left( t \right){\c}_{21} \right){\tau}^{2} = {\c}_{22}{\tau}^{2}\,.
    \end{align}

    Considering the inner product for both sides of the equality \eqref{eq:theorem_semidiscrete_scheme_for_z_k} with ${z}_{k + 1} - {z}_{k - 1} = {\Delta}{z}_{k} + {\Delta}{z}_{k - 1}$, we have
    \begin{equation}\label{eq:theorem_inner_product_z_k_eqt}
        {\left\| \frac{{\Delta}{z}_{k}}{{\tau}} \right\|}^{2} + \frac{1}{2}{q}_{k}{\left\| \frac{{\d}{z}_{k + 1}}{{\d}{x}} \right\|}^{2} = {\left\| \frac{{\Delta}{z}_{k - 1}}{{\tau}} \right\|}^{2} + \frac{1}{2}{q}_{k}{\left\| \frac{{\d}{z}_{k - 1}}{{\d}{x}} \right\|}^{2} + \left( {g}_{k}, {\Delta}{z}_{k} + {\Delta}{z}_{k - 1} \right)\,.
    \end{equation}
    Introducing the denotations
    \begin{gather*}
        {\overline{\mu}}_{k} = \left\| \frac{{\Delta}{z}_{k - 1}}{{\tau}} \right\|\,,\quad {\overline{\vartheta}}_{k} = \left\| \frac{{\d}{z}_{k}}{{\d}{x}} \right\|\,,\quad {\vartheta}_{k} = \left\| \frac{{\d}{u}_{k}}{{\d}{x}} \right\|\,,\quad {\overline{\delta}}_{k} = \left( {g}_{k}, {\Delta}{z}_{k} + {\Delta}{z}_{k - 1} \right)\,, \\
        {q}_{k} = {\alpha}_{k} + {\beta}_{k}{\vartheta}_{k}^{2}\,.
    \end{gather*}
    Thus, the equality expressed in \eqref{eq:theorem_inner_product_z_k_eqt} should be rewritten in the following manner
    \begin{equation*}
        {\overline{\mu}}_{k + 1}^{2} + \frac{1}{2}\left( {\alpha}_{k} + {\beta}_{k}{\vartheta}_{k}^{2} \right){\overline{\vartheta}}_{k + 1}^{2} = {\overline{\mu}}_{k}^{2} + \frac{1}{2}\left( {\alpha}_{k} + {\beta}_{k}{\vartheta}_{k}^{2} \right){\overline{\vartheta}}_{k - 1}^{2} + {\overline{\delta}}_{k}\,.
    \end{equation*}
    Finally, by using the last equality, we obtain
    \begin{equation}\label{eq:theorem_lambda_eqt}
        {\overline{\lambda}}_{k + 1} = {\overline{\lambda}}_{k} + \left( {\overline{\varepsilon}}_{k} + {\overline{\delta}}_{k} \right)\,,
    \end{equation}
    where
    \begin{align*}
        {\overline{\lambda}}_{k} &= {\overline{\mu}}_{k}^{2} + \frac{1}{2}\left( {\alpha}_{k - 1} + {\beta}_{k - 1}{\vartheta}_{k - 1}^{2} \right){\overline{\vartheta}}_{k}^{2}\,, \\
        {\overline{\varepsilon}}_{k} &= \frac{1}{2}\left( {\alpha}_{k} + {\beta}_{k}{\vartheta}_{k}^{2} \right){\overline{\vartheta}}_{k - 1}^{2} - \frac{1}{2}\left( {\alpha}_{k - 1} + {\beta}_{k - 1}{\vartheta}_{k - 1}^{2} \right){\overline{\vartheta}}_{k}^{2} \\
        &= \frac{1}{2}\left( {\alpha}_{k}{\overline{\vartheta}}_{k - 1}^{2} - {\alpha}_{k - 1}{\overline{\vartheta}}_{k}^{2} \right) + \frac{1}{2}\left( {\beta}_{k}{\vartheta}_{k}^{2}{\overline{\vartheta}}_{k - 1}^{2} - {\beta}_{k - 1}{\vartheta}_{k - 1}^{2}{\overline{\vartheta}}_{k}^{2} \right)\,.
    \end{align*}
    From \eqref{eq:theorem_lambda_eqt}, we have
    \begin{align}\label{eq:theorem_lambda_sum_eqt}
        {\overline{\lambda}}_{k + 1} &= {\overline{\lambda}}_{1} + \sum\limits_{i = 1}^{k}{\left( {\overline{\varepsilon}}_{i} + {\overline{\delta}}_{i} \right)}\nonumber \\
        &= {\overline{\lambda}}_{1} + \frac{1}{2}\sum\limits_{i = 1}^{k}{\left( {\alpha}_{i}{\overline{\vartheta}}_{i - 1}^{2} - {\alpha}_{i - 1}{\overline{\vartheta}}_{i}^{2} \right)} + \frac{1}{2}\sum\limits_{i = 1}^{k}{\left( {\beta}_{i}{\vartheta}_{i}^{2}{\overline{\vartheta}}_{i - 1}^{2} - {\beta}_{i - 1}{\vartheta}_{i - 1}^{2}{\overline{\vartheta}}_{i}^{2} \right)} + \sum\limits_{i = 1}^{k}{{\overline{\delta}}_{i}}\,.
    \end{align}
    The following representations are faithful
    \begin{align}
        &\sum\limits_{i = 1}^{k}{\left( {\alpha}_{i}{\overline{\vartheta}}_{i - 1}^{2} - {\alpha}_{i - 1}{\overline{\vartheta}}_{i}^{2} \right)} = \sum\limits_{i = 1}^{k}{\left( {\alpha}_{i}{\overline{\vartheta}}_{i - 1}^{2} - {\alpha}_{i + 1}{\overline{\vartheta}}_{i}^{2} \right)} + \sum\limits_{i = 1}^{k}{\left( {\alpha}_{i + 1} - {\alpha}_{i - 1} \right){\overline{\vartheta}}_{i}^{2}}\nonumber \\
        &= {\alpha}_{1}{\overline{\vartheta}}_{0}^{2} - {\alpha}_{k + 1}{\overline{\vartheta}}_{k}^{2} + \sum\limits_{i = 1}^{k}{\left( {\alpha}_{i + 1} - {\alpha}_{i - 1} \right){\overline{\vartheta}}_{i}^{2}}\,.\label{eq:theorem_sum_alpha_varrho} \\
        &\sum\limits_{i = 1}^{k}{\left( {\beta}_{i}{\vartheta}_{i}^{2}{\overline{\vartheta}}_{i - 1}^{2} - {\beta}_{i - 1}{\vartheta}_{i - 1}^{2}{\overline{\vartheta}}_{i}^{2} \right)} = {\beta}_{1}{\vartheta}_{1}^{2}{\overline{\vartheta}}_{0}^{2} + \sum\limits_{i = 1}^{k - 1}{\left( {\beta}_{i + 1}{\vartheta}_{i + 1}^{2} - {\beta}_{i - 1}{\vartheta}_{i - 1}^{2} \right){\overline{\vartheta}}_{i}^{2}} - {\beta}_{k - 1}{\vartheta}_{k - 1}^{2}{\overline{\vartheta}}_{k}^{2}\nonumber \\
        &= {\beta}_{1}{\vartheta}_{1}^{2}{\overline{\vartheta}}_{0}^{2} + \sum\limits_{i = 1}^{k - 1}{\left( {\beta}_{i + 1} - {\beta}_{i - 1} \right){\vartheta}_{i + 1}^{2}{\overline{\vartheta}}_{i}^{2}} + \sum\limits_{i = 1}^{k - 1}{{\beta}_{i - 1}\left( {\vartheta}_{i + 1}^{2} - {\vartheta}_{i - 1}^{2} \right){\overline{\vartheta}}_{i}^{2}} - {\beta}_{k - 1}{\vartheta}_{k - 1}^{2}{\overline{\vartheta}}_{k}^{2}\,.\label{eq:theorem_sum_sum_beta_varrho}
    \end{align}
    Through the insertion of equalities \eqref{eq:theorem_sum_alpha_varrho} and \eqref{eq:theorem_sum_sum_beta_varrho} into \eqref{eq:theorem_lambda_sum_eqt}, we derive that
    \begin{align}\label{eq:theorem_expand_sum_lambda_eqt}
        &{\overline{\lambda}}_{k + 1} + \frac{1}{2}\left( {\alpha}_{k + 1} + {\beta}_{k - 1}{\vartheta}_{k - 1}^{2} \right){\overline{\vartheta}}_{k}^{2} = {\overline{\lambda}}_{1} + \frac{1}{2}\left( {\alpha}_{1} + {\beta}_{1}{\vartheta}_{1}^{2} \right){\overline{\vartheta}}_{0}^{2}\nonumber \\
        &+ \frac{1}{2}\sum\limits_{i = 1}^{k}{\left( {\alpha}_{i + 1} - {\alpha}_{i - 1} \right){\overline{\vartheta}}_{i}^{2}} + \frac{1}{2}\sum\limits_{i = 1}^{k - 1}{\left( {\beta}_{i + 1} - {\beta}_{i - 1} \right){\vartheta}_{i + 1}^{2}{\overline{\vartheta}}_{i}^{2}} + \frac{1}{2}\sum\limits_{i = 1}^{k - 1}{{\beta}_{i - 1}\left( {\vartheta}_{i + 1}^{2} - {\vartheta}_{i - 1}^{2} \right){\overline{\vartheta}}_{i}^{2}} + \sum\limits_{i = 1}^{k}{{\overline{\delta}}_{i}}\,.
    \end{align}
    By virtue of equation \eqref{eq:lemma1_alpha_ineqt}, we arrive at the straightforward conclusion that
    \begin{equation}\label{eq:theorem_sum_alpha_ineqt}
        \frac{1}{2}\sum\limits_{i = 1}^{k}{\left( {\alpha}_{i + 1} - {\alpha}_{i - 1} \right){\overline{\vartheta}}_{i}^{2}} \leq {\c}_{2}{\tau}\sum\limits_{i = 1}^{k}{{\overline{\vartheta}}_{i}^{2}}\,,\quad {\c}_{2} = \max\limits_{0 \leq t \leq {T}}{\left| {\alpha}^{\prime}\left( t \right) \right|}\,.
    \end{equation}
    As a consequence of \eqref{eq:lemma1_beta_ineqt} and the \hyperref[lemma:lemma1]{\bf Lemma \ref*{lemma:lemma1}} (see also the \hyperref[rmk:remark-lemma1]{\bf Remark \ref*{rmk:remark-lemma1}}), we find that
    \begin{equation}\label{eq:theorem_sum_beta_ineqt}
        \frac{1}{2}\sum\limits_{i = 1}^{k - 1}{\left( {\beta}_{i + 1} - {\beta}_{i - 1} \right){\vartheta}_{i + 1}^{2}{\overline{\vartheta}}_{i}^{2}} \leq {\c}_{3}{\M}_{2}^{2}{\tau}\sum\limits_{i = 1}^{k - 1}{{\overline{\vartheta}}_{i}^{2}}\,,\quad {\c}_{3} = \max\limits_{0 \leq t \leq {T}}{\left| {\beta}^{\prime}\left( t \right) \right|}\,.
    \end{equation}
    According to \hyperref[lemma:lemma1]{\bf Lemma \ref*{lemma:lemma1}} and the \hyperref[lemma:lemma2]{\bf Lemma \ref*{lemma:lemma2}}, an estimate for the difference $\displaystyle {\vartheta}_{i + 1}^{2} - {\vartheta}_{i - 1}^{2}$ can be derived
    \begin{align}\label{eq:theorem_varrho_square_inqt}
        \left| {\vartheta}_{i + 1}^{2} - {\vartheta}_{i - 1}^{2} \right| &\leq \left| {\vartheta}_{i + 1} - {\vartheta}_{i - 1} \right|\left( {\vartheta}_{i + 1} + {\vartheta}_{i - 1} \right)\nonumber \\
        &\leq \left( \left| {\vartheta}_{i + 1} - {\vartheta}_{i} \right| + \left| {\vartheta}_{i} - {\vartheta}_{i - 1} \right| \right)\left( {\vartheta}_{i + 1} + {\vartheta}_{i - 1} \right)\nonumber \\
        &\leq {\tau}\left( \left\| {\frac{\d}{{\d}{x}}} \left( \frac{\Delta{u}_{i}}{\tau} \right) \right\| + \left\| {\frac{\d}{{\d}{x}}} \left( \frac{\Delta{u}_{i - 1}}{\tau} \right) \right\| \right)\left( \left\| \frac{{\d}{u}_{i + 1}}{{\d}{x}} \right\| + \left\| \frac{{\d}{u}_{i - 1}}{{\d}{x}} \right\| \right)\nonumber \\
        &\leq {4}{\M}_{2}{\M}_{3}{\tau}\,.
    \end{align}
    By virtue of equality \eqref{eq:theorem_varrho_square_inqt}, it can be deduced that
    \begin{equation}\label{eq:theorem_beta_varrho_square_inqt}
        \frac{1}{2}\sum\limits_{i = 1}^{k - 1}{{\beta}_{i - 1}\left( {\vartheta}_{i + 1}^{2} - {\vartheta}_{i - 1}^{2} \right){\overline{\vartheta}}_{i}^{2}} \leq {2}{\M}_{2}{\M}_{3}{\c}_{23}{\tau}\sum\limits_{i = 1}^{k - 1}{{\overline{\vartheta}}_{i}^{2}}\,,\quad {\c}_{23} = \max\limits_{0 \leq t \leq {T}}{\beta\left( t \right)}\,.
    \end{equation}
    By employing the Cauchy-Schwarz inequality, we can find that
    \begin{align*}
        \left| {\overline{\delta}}_{i} \right| &\leq \left\| {g}_{i} \right\|\left( \left\| {\Delta}{z}_{i} \right\| + \left\| {\Delta}{z}_{i - 1} \right\| \right) = {\tau}\left\| {g}_{i} \right\|\left( {\overline{\mu}}_{i + 1} + {\overline{\mu}}_{i} \right) \\
        &\leq {\tau}\left( \frac{1}{2}\left| {q}\left( {t}_{i} \right) - {q}_{i} \right|\left( \left\| \frac{{\d}^{2}{u}\left( \cdot,{t}_{i + 1} \right)}{{\d}{x}^{2}} \right\| + \left\| \frac{{\d}^{2}{u}\left( \cdot,{t}_{i - 1} \right)}{{\d}{x}^{2}} \right\| \right) + \left\| {R}_{i}\left( \cdot,{\tau} \right) \right\| \right)\left( {\overline{\mu}}_{i + 1} + {\overline{\mu}}_{i} \right)\,.
    \end{align*}
    We shall proceed to estimate the given difference while taking into consideration the implications of \hyperref[lemma:lemma1]{\bf Lemma \ref*{lemma:lemma1}}
    \begin{align*}
        \frac{1}{2}\left| {q}\left( {t}_{i} \right) - {q}_{i} \right| &= \frac{1}{2}{\beta}_{i}\left| {\left\| \frac{{\d}u\left( \cdot,{t}_{i} \right)}{{\d}{x}} \right\|}^{2} - {\left\| \frac{{\d}{u}_{i}}{{\d}{x}} \right\|}^{2} \right| = \frac{1}{2}{\beta}_{i}\left| \left\| \frac{{\d}u\left( \cdot,{t}_{i} \right)}{{\d}{x}} \right\| - \left\| \frac{{\d}{u}_{i}}{{\d}{x}} \right\| \right|\left( \left\| \frac{{\d}u\left( \cdot,{t}_{i} \right)}{{\d}{x}} \right\| + \left\| \frac{{\d}{u}_{i}}{{\d}{x}} \right\| \right) \\
        &\leq \frac{1}{2}{\beta}_{i}\left\| \frac{{\d}{z}_{i}}{{\d}{x}} \right\|\left( \left\| \frac{{\d}u\left( \cdot,{t}_{i} \right)}{{\d}{x}} \right\| + \left\| \frac{{\d}{u}_{i}}{{\d}{x}} \right\| \right) \leq \frac{1}{2}{\c}_{23}\left( {\c}_{24} + {\M}_{2} \right){{\overline{\vartheta}}_{i}} = {\c}_{25}{{\overline{\vartheta}}_{i}}\,.
    \end{align*}
    Utilizing the last inequality and combining it with \eqref{eq:theorem_main_remainder_term} for ${\overline{\delta}}_{i}$, we arrive at the following result
    \begin{align}\label{eq:theorem_overline_delta_i}
        \left| {\overline{\delta}}_{i} \right| &\leq {\tau}\left( {\c}_{25}{\c}_{26}{{\overline{\vartheta}}_{i}} + {\c}_{22}{\tau}^{2} \right)\left( {\overline{\mu}}_{i + 1} + {\overline{\mu}}_{i} \right)\nonumber \\
        &\leq \max\left( {\c}_{25}{\c}_{26},{\c}_{22} \right){\tau}\left( {\overline{\vartheta}}_{i} + {\tau}^{2} \right)\left( {\overline{\mu}}_{i + 1} + {\overline{\mu}}_{i} \right)\nonumber \\
        &={\c}_{27}{\tau}\left( {\overline{\vartheta}}_{i} + {\tau}^{2} \right)\left( {\overline{\mu}}_{i + 1} + {\overline{\mu}}_{i} \right)\nonumber \\
        &= {\c}_{27}{\tau}\left[ \left( {\overline{\vartheta}}_{i}{\overline{\mu}}_{i + 1} + {\overline{\vartheta}}_{i}{\overline{\mu}}_{i} \right) + {\tau}^{2}\left( {\overline{\mu}}_{i + 1} + {\overline{\mu}}_{i} \right) \right]\nonumber \\
        &\leq {\c}_{27}{\tau}\left[ \frac{1}{2}\left( {\overline{\vartheta}}_{i}^{2} + {\overline{\mu}}_{i + 1}^{2} \right) + \frac{1}{2}\left( {\overline{\vartheta}}_{i}^{2} + {\overline{\mu}}_{i}^{2} \right) + \frac{1}{2}\left( {\tau}^{4} + {\left( {\overline{\mu}}_{i + 1} + {\overline{\mu}}_{i} \right)}^{2} \right) \right]\nonumber \\
        &= {\c}_{27}{\tau}\left[ {\overline{\vartheta}}_{i}^{2} + \left( {\overline{\mu}}_{i + 1}^{2} + {\overline{\mu}}_{i}^{2} \right) + \frac{1}{2}{\tau}^{4} + {\overline{\mu}}_{i + 1}{\overline{\mu}}_{i} \right]\nonumber \\
        &\leq {\c}_{27}{\tau}\left[ {\overline{\vartheta}}_{i}^{2} + \left( {\overline{\mu}}_{i + 1}^{2} + {\overline{\mu}}_{i}^{2} \right) + \frac{1}{2}{\tau}^{4} + \frac{1}{2}\left( {\overline{\mu}}_{i + 1}^{2} + {\overline{\mu}}_{i}^{2} \right) \right]\nonumber \\
        &= {\c}_{27}{\tau}\left( {\overline{\vartheta}}_{i}^{2} + \frac{3}{2}{\overline{\mu}}_{i + 1}^{2} + \frac{3}{2}{\overline{\mu}}_{i}^{2} + \frac{1}{2}{\tau}^{4} \right) \leq {\c}_{28}{\tau}\left( {\overline{\vartheta}}_{i}^{2} + {\overline{\mu}}_{i}^{2} + {\overline{\mu}}_{i + 1}^{2} \right) + {\c}_{28}{\tau}^{5}\,.
    \end{align}
    In order to obtain the desired estimate for ${\overline{\delta}}_{i}$, it is necessary to use the following inequalities. It is important to note that ${\alpha}\left( t \right) \geq {\c}_{0} > 0$. Therefore, we have:
    \begin{align*}
        {\overline{\lambda}}_{i} &= {\overline{\mu}}_{i}^{2} + \frac{1}{2}\left( {\alpha}_{i - 1} + {\beta}_{i - 1}{\vartheta}_{i - 1}^{2} \right){\overline{\vartheta}}_{i}^{2} \geq {\overline{\mu}}_{i}^{2} + \frac{{\c}_{0}}{2}{\overline{\vartheta}}_{i}^{2} \geq {\c}_{29}\left( {\overline{\mu}}_{i}^{2} + {\overline{\vartheta}}_{i}^{2} \right)\,,\quad {\c}_{29} = \min\left( 1,\frac{{\c}_{0}}{2} \right)\,,
    \end{align*}
    consequently, it can be inferred that
    \begin{equation}\label{eq:theorem_sum_mu_square_and_varrho_square}
        {\overline{\mu}}_{i}^{2} + {\overline{\vartheta}}_{i}^{2} \leq \frac{1}{{\c}_{29}}{\overline{\lambda}}_{i}\,.
    \end{equation}
    It is evident that the following inequality holds
    \begin{equation}\label{eq:theorem_overl_mu_leq_overl_lambda}
        {\overline{\mu}}_{i}^{2} \leq {\overline{\lambda}}_{i} = {\overline{\mu}}_{i}^{2} + \frac{1}{2}\left( {\alpha}_{i - 1} + {\beta}_{i - 1}{\vartheta}_{i - 1}^{2} \right){\overline{\vartheta}}_{i}^{2}\,.
    \end{equation}
    Taking into consideration the estimates \eqref{eq:theorem_sum_mu_square_and_varrho_square} and \eqref{eq:theorem_overl_mu_leq_overl_lambda}, we can obtain from \eqref{eq:theorem_overline_delta_i}:
    \begin{align}\label{eq:theorem_overal_detla_final_ineqt}
        \left| {\overline{\delta}}_{i} \right| &\leq {\c}_{28}{\tau}\left( \left( {\overline{\mu}}_{i}^{2} + {\overline{\vartheta}}_{i}^{2} \right) + {\overline{\mu}}_{i + 1}^{2} \right) + {\c}_{28}{\tau}^{5} \leq {\c}_{28}{\tau}\left( \frac{1}{{\c}_{29}}{\overline{\lambda}}_{i} + {\overline{\lambda}}_{i + 1} \right) + {\c}_{28}{\tau}^{5}\nonumber \\
        &\leq {\c}_{28}\max\left( 1,\frac{1}{{\c}_{29}} \right){\tau}\left( {\overline{\lambda}}_{i} + {\overline{\lambda}}_{i + 1} \right) + {\c}_{28}{\tau}^{5}\nonumber \\
        &= {\c}_{28}\max\left( 1,\frac{2}{{\c}_{0}} \right){\tau}\left( {\overline{\lambda}}_{i} + {\overline{\lambda}}_{i + 1} \right) + {\c}_{28}{\tau}^{5}\nonumber \\
        &= {\c}_{30}{\tau}\left( {\overline{\lambda}}_{i} + {\overline{\lambda}}_{i + 1} \right) + {\c}_{28}{\tau}^{5}\,.
    \end{align}
    Incorporating the inequalities \eqref{eq:theorem_sum_alpha_ineqt}, \eqref{eq:theorem_sum_beta_ineqt}, \eqref{eq:theorem_beta_varrho_square_inqt} and \eqref{eq:theorem_overal_detla_final_ineqt} into \eqref{eq:theorem_expand_sum_lambda_eqt}, we arrive at
    \begin{align*}
        &{\overline{\lambda}}_{k + 1} + \frac{1}{2}\left( {\alpha}_{k + 1} + {\beta}_{k - 1}{\vartheta}_{k - 1}^{2} \right){\overline{\vartheta}}_{k}^{2} \leq {\overline{\lambda}}_{1} + \frac{1}{2}\left( {\alpha}_{1} + {\beta}_{1}{\vartheta}_{1}^{2} \right){\overline{\vartheta}}_{0}^{2} \\
        &+ {\c}_{2}{\tau}\sum\limits_{i = 1}^{k}{{\overline{\vartheta}}_{i}^{2}} + \left( {\c}_{3}{\M}_{2}^{2} + {2}{\M}_{2}{\M}_{3}{\c}_{23} \right){\tau}\sum\limits_{i = 1}^{k - 1}{{\overline{\vartheta}}_{i}^{2}} + {\c}_{30}{\tau}\sum\limits_{i = 1}^{k}{\left( {\overline{\lambda}}_{i} + {\overline{\lambda}}_{i + 1} \right)} + {\c}_{28}{k}{\tau}^{5} \\
        &\leq {\overline{\lambda}}_{1} + \frac{1}{2}\left( {\alpha}_{1} + {\beta}_{1}{\vartheta}_{1}^{2} \right){\overline{\vartheta}}_{0}^{2} + \left( {\c}_{2} + {\c}_{3}{\M}_{2}^{2} + {2}{\M}_{2}{\M}_{3}{\c}_{23} \right){\tau}\sum\limits_{i = 1}^{k}{{\overline{\vartheta}}_{i}^{2}} + {\c}_{30}{\tau}\sum\limits_{i = 1}^{k}{\left( {\overline{\lambda}}_{i} + {\overline{\lambda}}_{i + 1} \right)} + {\c}_{28}{k}{\tau}^{5} \\
        &= {\overline{\lambda}}_{1} + \frac{1}{2}\left( {\alpha}_{1} + {\beta}_{1}{\vartheta}_{1}^{2} \right){\overline{\vartheta}}_{0}^{2} + {\c}_{31}{\tau}\sum\limits_{i = 1}^{k}{{\overline{\vartheta}}_{i}^{2}} + {\c}_{30}{\tau}\sum\limits_{i = 1}^{k}{\left( {\overline{\lambda}}_{i} + {\overline{\lambda}}_{i + 1} \right)} + {\c}_{28}{k}{\tau}^{5}\,.
    \end{align*}
    The following three simple estimates can be derived:
    \begin{gather*}
        {\overline{\vartheta}}_{i}^{2} \leq \frac{2}{{\c}_{0}}{\overline{\lambda}}_{i} = {2}{\c}_{7}{\overline{\lambda}}_{i}\,, \\
        {\c}_{30}{\tau}\sum\limits_{i = 1}^{k}{\left( {\overline{\lambda}}_{i} + {\overline{\lambda}}_{i + 1} \right)} = {\c}_{30}{\tau}\left( {\overline{\lambda}}_{1} + {2}\sum\limits_{i = 2}^{k}{{\overline{\lambda}}_{i}} + {\overline{\lambda}}_{k + 1} \right) \leq {2}{\c}_{30}{\tau}\sum\limits_{i = 1}^{k + 1}{{\overline{\lambda}}_{i}}\,, \\
        {\c}_{28}{k}{\tau}^{5} = {\c}_{28}\left( {k}{\tau} \right){\tau}^{4} \leq {\c}_{28}{\overline{T}}{\tau}^{4}\,.
    \end{gather*}
    Taking into account the above-mentioned inequalities, it can be concluded that
    \begin{align*}
        {\overline{\lambda}}_{k + 1} + \frac{1}{2}\left( {\alpha}_{k + 1} + {\beta}_{k - 1}{\vartheta}_{k - 1}^{2} \right){\overline{\vartheta}}_{k}^{2} &\leq {\overline{\lambda}}_{1} + \frac{1}{2}\left( {\alpha}_{1} + {\beta}_{1}{\vartheta}_{1}^{2} \right){\overline{\vartheta}}_{0}^{2} \\
        &+ {2}\left( {\c}_{7}{\c}_{31} + {\c}_{30} \right){\tau}\sum\limits_{i = 1}^{k}{{\overline{\lambda}}_{i}} + {2}{\c}_{30}{\tau}{\overline{\lambda}}_{k + 1} + {\c}_{28}{\overline{T}}{\tau}^{4} \\
        &= {\c}_{32}{\tau}\sum\limits_{i = 1}^{k}{{\overline{\lambda}}_{i}} + {\c}_{33}{\tau}{\overline{\lambda}}_{k + 1} + {\c}_{28}{\overline{T}}{\tau}^{4}\,.
    \end{align*}
    Therefore, we have
    \begin{equation}\label{eq:theorem_final_overl_lambda_inqt}
        \left( 1 - {\c}_{33}{\tau} \right){\overline{\lambda}}_{k + 1} \leq \overline{\alpha} + {\c}_{32}{\tau}\sum\limits_{i = 1}^{k}{{\overline{\lambda}}_{i}}\,,
    \end{equation}
    where
    \begin{equation*}
        \overline{\alpha} = {\overline{\lambda}}_{1} + \frac{1}{2}\left( {\alpha}_{1} + {\beta}_{1}{\vartheta}_{1}^{2} \right){\overline{\vartheta}}_{0}^{2} + {\c}_{28}{\overline{T}}{\tau}^{4}\,.
    \end{equation*}
    Assuming that ${\c}_{33}{\tau} < 1$, we can derive from \eqref{eq:theorem_final_overl_lambda_inqt} that
    \begin{equation}\label{eq:theorem_the_previous_inqt_gronwall_lemma}
        {\overline{\lambda}}_{k + 1} \leq {\widetilde \alpha} + {\widetilde c}{\tau}\sum\limits_{i = 1}^{k}{{\overline{\lambda}}_{i}}\,,
    \end{equation}
    here
    \begin{equation*}
        {\widetilde \alpha} = \frac{\overline{\alpha}}{1 - {\c}_{33}{\tau}}\,,\quad {\widetilde c} = \frac{{\c}_{32}}{1 - {\c}_{33}{\tau}}\,.
    \end{equation*}
    From \eqref{eq:theorem_the_previous_inqt_gronwall_lemma} by applying \hyperref[lemma:gronwall-inequality]{\bf Lemma \ref*{lemma:gronwall-inequality} (Discrete Gr\"{o}nwall-type inequality)} along with \hyperref[rmk:remark1]{\bf Remark \ref*{rmk:remark1}} we get
    \begin{equation}\label{eq:theorem_gronwall_lemma}
        {\overline{\lambda}}_{k + 1} \leq {\e}^{{\tilde c}{t}_{k}}{\widetilde \alpha} \leq {\e}^{{\tilde c}\overline{T}}{\widetilde \alpha}\,.
    \end{equation}
    To estimate ${\widetilde \alpha}$, we need to first estimate ${\overline{\lambda}}_{1}$. For this purpose, we use conditions \ref{itm_theorem_a} and \ref{itm_theorem_b} of \hyperref[theorem:theorem1]{\bf Theorem \ref*{theorem:theorem1}}, and consider the Taylor expansion of the function ${u}\left( x,{t}_{1} \right)$ around the point ${t} = {0}$ with respect to the temporal variable, keeping the first three terms. We then apply equation \eqref{eq:main_eqt} and the initial conditions \eqref{eq:initial_conds} for the functions ${u}\left( x,0 \right)$, ${u}^{{\prime}}\left( x,0 \right)$ and ${u}^{{\prime}{\prime}}\left( x,0 \right)$, which gives us
    \begin{align*}
        {u}\left( x,{t}_{1} \right) &= {u}\left( x,0 \right) + {{t}_{1}}{u}^{{\prime}}\left( x,0 \right) + \frac{{t}_{1}^{2}}{2}{u}^{{\prime}{\prime}}\left( x,0 \right) + {\widetilde R}_{2}\left( x,{t}_{1} \right) \\
        &= {\psi}_{0}\left( x \right) + {\tau}{\psi}_{1}\left( x \right) + \frac{{\tau}^{2}}{2}{\psi}_{2}\left( x \right) + {\widetilde R}_{2}\left( x,{\tau} \right)\,,
    \end{align*}
    where
    \begin{equation*}
        {\psi}_{2}\left( x \right) = {f}_{0}\left( x \right) + {q}_{0}\frac{\d ^{2}{\psi}_{0}\left( x \right)}{\d {x}^{2}}\,,\quad {\widetilde R}_{2}\left( x,{\tau} \right) = \frac{1}{2}\int\limits_{0}^{{\tau}}{{\left( {\tau} - t \right)}^{2}{u}^{{\prime}{\prime}{\prime}}\left( x,t \right)}{\d t}\,,
    \end{equation*}
    on the other hand, recall that
    \begin{equation*}
        {u}_{1}\left( x \right) = {\psi}_{0}\left( x \right) + {\tau}{\psi}_{1}\left( x \right) + \frac{{\tau}^{2}}{2}{\psi}_{2}\left( x \right)\,.
    \end{equation*}
    We obtain the following estimation for ${z}_{1}\left( x \right)$
    \begin{align*}
        \max\limits_{0 \leq x \leq \ell}\left| {z}_{1}\left( x \right) \right| &= \max\limits_{0 \leq x \leq \ell}\left| {u}\left( x,{t}_{1} \right) - {u}_{1}\left( x \right) \right| = \max\limits_{0 \leq x \leq \ell}\left| {\widetilde R}_{2}\left( x,{\tau} \right) \right| \\
        &= \frac{1}{2}\max\limits_{0 \leq x \leq \ell}\left| \int\limits_{0}^{{\tau}}{{\left( {\tau} - t \right)}^{2}{u}^{{\prime}{\prime}{\prime}}\left( x,t \right)}{\d t} \right| \leq \frac{1}{2}\max\limits_{\left( x,t \right)}\left| {u}^{{\prime}{\prime}{\prime}}\left( x,t \right) \right|\int\limits_{0}^{{\tau}}{{\left( {\tau} - t \right)}^{2}}{\d t} \\
        &= \frac{1}{6}\max\limits_{\left( x,t \right)}\left| {u}^{{\prime}{\prime}{\prime}}\left( x,t \right) \right|{\tau}^{3} = {\c}_{34}{\tau}^{3}\,,
    \end{align*}
    from here, we find that
    \begin{equation}\label{eq:theorem_norm_z1}
        {\left\| {z}_{1} \right\|}^{2} = \int\limits_{0}^{{\ell}}{{\left( {u}\left( x,{t}_{1} \right) - {u}_{1}\left( x \right) \right)}^{2}}{\d x} \leq {\ell}{\c}_{34}^{2}{\tau}^{6}\,.
    \end{equation}
    In order to obtain an estimation for ${\overline{\vartheta}}_{1}^{2}$, we consider the Taylor series expansion of the function ${u}\left( x,{t}_{1} \right)$ about the point ${t} = {0}$ with respect to the temporal variable, but this time we keep only the first two terms. Similar to before, we apply the initial conditions \eqref{eq:initial_conds} for the functions ${u}\left( x,0 \right)$, ${u}^{{\prime}}\left( x,0 \right)$ to obtain the following expression, {\ie}
    \begin{align*}
        {u}\left( x,{t}_{1} \right) &= {u}\left( x,0 \right) + {{t}_{1}}{u}^{{\prime}}\left( x,0 \right) + {\widetilde R}_{1}\left( x,{t}_{1} \right) \nonumber \\
        &= {\psi}_{0}\left( x \right) + {\tau}{\psi}_{1}\left( x \right) + {\widetilde R}_{1}\left( x,{\tau} \right)\,,
    \end{align*}
    here
    \begin{equation*}
        {\widetilde R}_{1}\left( x,{\tau} \right) = \int\limits_{0}^{{\tau}}{{\left( {\tau} - t \right)}{u}^{{\prime}{\prime}}\left( x,t \right)}{\d t}\,.
    \end{equation*}
    It should be noted that ${u}_{1}\left( x \right)$ is an approximation of ${u}\left( x,{t}_{1} \right)$, {\ie} ${u}\left( x,{t}_{1} \right) \approx {u}_{1}\left( x \right) = {\psi}_{0}\left( x \right) + {\tau}{\psi}_{1}\left( x \right)$.
    \begin{align*}
        \max\limits_{0 \leq x \leq \ell}\left| \frac{{\d}{z}_{1}\left( x \right)}{{\d}{x}} \right| &= \max\limits_{0 \leq x \leq \ell}\left| \frac{{\d}}{{\d}{x}}{\left( {u}\left( x,{t}_{1} \right) - {u}_{1}\left( x \right) \right)} \right| = \max\limits_{0 \leq x \leq \ell}\left| \frac{{\d}}{{\d}{x}}{{\widetilde R}_{1}\left( x,{\tau} \right)} \right| \\
        &= \max\limits_{0 \leq x \leq \ell}\left| \frac{{\d}}{{\d}{x}}{\int\limits_{0}^{{\tau}}{{\left( {\tau} - t \right)}{u}^{{\prime}{\prime}}\left( x,t \right)}{\d t}} \right| \leq \max\limits_{\left( x,t \right)}{\left| {u}_{{x}{t}{t}}\left( x,t \right) \right|}{\int\limits_{0}^{{\tau}}{{\left( {\tau} - t \right)}}{\d t}} \\
        &\leq \frac{1}{2}\max\limits_{\left( x,t \right)}{\left| {u}_{{x}{t}{t}}\left( x,t \right) \right|}{\tau}^{2} = {\c}_{35}{\tau}^{2}\,,
    \end{align*}
    from here, we obtain
    \begin{equation}\label{eq:theorem_norm_of_second_deriv_of_z1}
        {\left\| \frac{{\d}{z}_{1}}{{\d}{x}} \right\|}^{2} = \int\limits_{0}^{{\ell}}{{\left[ \frac{{\d}}{{\d}{x}}\left( {u}\left( x,{t}_{1} \right) - {u}_{1}\left( x \right) \right) \right]}^{2}}{\d x} \leq {\ell}{{\c}_{35}^{2}{\tau}^{4}}\,.
    \end{equation}
    Based on \eqref{eq:theorem_norm_z1} and \eqref{eq:theorem_norm_of_second_deriv_of_z1}, we can derive the following estimations
    \begin{align}\label{eq:theorem_norm_of_overline_lambda_1}
        {\overline{\lambda}}_{1} &= {\overline{\mu}}_{1}^{2} + \frac{1}{2}\left( {\alpha}_{0} + {\beta}_{0}{\vartheta}_{0}^{2} \right){\overline{\vartheta}}_{1}^{2}\nonumber \\
        &= {\left\| \frac{{z}_{1} - {z}_{0}}{\tau} \right\|}^{2} + \frac{1}{2}\left( {\alpha}_{0} + {\beta}_{0}{\left\| \frac{{\d}{u}_{0}}{{\d}{x}} \right\|}^{2} \right){\left\| \frac{{\d}{z}_{1}}{{\d}{x}} \right\|}^{2}\nonumber \\
        &= \frac{1}{{\tau}^{2}}{\left\| {z}_{1} \right\|}^{2} + \frac{1}{2}\left( {\alpha}_{0} + {\beta}_{0}{\left\| \frac{{\d}{\psi}_{0}}{{\d}{x}} \right\|}^{2} \right){\left\| \frac{{\d}{z}_{1}}{{\d}{x}} \right\|}^{2}\nonumber \\
        &\leq {\ell}\left( {\c}_{34}^{2} + \frac{1}{2}\left( {\alpha}_{0} + {\beta}_{0}{{\c}_{36}^{2}} \right){\c}_{35}^{2} \right){\tau}^{4} = {\c}_{37}{\tau}^{4}\,,\quad {z}_{0}\left( x \right) = {u}\left( x,0 \right) - {u}_{0}\left( x \right) \equiv 0\,,
    \end{align}
    furthermore, it can be concluded from equation \eqref{eq:theorem_norm_of_overline_lambda_1} that
    \begin{align}\label{eq:theorem_norm_of_widetilde_alpha}
        {\widetilde \alpha} &= \frac{\overline{\alpha}}{1 - {\c}_{33}{\tau}} = \frac{1}{1 - {\c}_{33}{\tau}}\left( {\overline{\lambda}}_{1} + \frac{1}{2}\left( {\alpha}_{1} + {\beta}_{1}{\vartheta}_{1}^{2} \right){\overline{\vartheta}}_{0}^{2} + {\c}_{28}{\overline{T}}{\tau}^{4} \right)\nonumber \\
        &= \frac{1}{1 - {\c}_{33}{\tau}}\left( {\overline{\lambda}}_{1} + {\c}_{28}{\overline{T}}{\tau}^{4} \right) \leq \frac{{\c}_{37} + {\c}_{28}{\overline{T}}}{1 - {\c}_{33}{\tau}}{\tau}^{4} = {\c}_{38}{\tau}^{4}\,.
    \end{align}
    From \eqref{eq:theorem_gronwall_lemma}, taking the inequalities \eqref{eq:theorem_norm_of_overline_lambda_1} and \eqref{eq:theorem_norm_of_widetilde_alpha} into consideration, the estimates for \hyperref[theorem:theorem1]{\bf Theorem \ref*{theorem:theorem1}} are obtained.
\end{proof}
%
%According to the \hyperref[theorem:theorem1]{\bf Theorem \ref*{theorem:theorem1}}, the following statement is valid.
\begin{remark}\label{rmk:remark2}
    The error ${z}_{k}\left( x \right) = {u}\left( x,{t}_{k} \right) - {u}_{k}\left( x \right)$ of an approximate solution to the problem \eqref{eq:main_eqt}-\eqref{eq:boundary_conds} is bounded by the following inequality:
    \begin{equation}\label{eq:remark_estimate}
        \max\limits_{1 \leq {k} \leq m}{\left\| {z}_{k} \right\|} \leq {\M}_{7}{\tau}^{2}\,,
    \end{equation}
    where $\displaystyle {m} = \left[ \frac{\overline{T}}{{\tau}} \right]$, and $\overline{T}$ satisfies $0 < \overline{T} \leq {T}$.

    The validity of inequality \eqref{eq:remark_estimate} follows immediately from the identity
    \begin{equation*}
        {z}_{k}\left( x \right) = {z}_{0}\left( x \right) + {\tau}\sum\limits_{i = 0}^{k - 1}{\frac{{\Delta}{z}_{i}\left( x \right)}{\tau}}\,,\quad {z}_{0}\left( x \right) = {u}\left( x,0 \right) - {u}_{0}\left( x \right) \equiv 0\,,
    \end{equation*}
    by virtue of the second bound of \hyperref[theorem:theorem1]{\bf Theorem \ref*{theorem:theorem1}}.
\end{remark}


\section*{Acknowledgement}\label{sec:acknowledgement}
% The second author of this article received support from the Shota Rustaveli National Science Foundation of Georgia (SRNSFG) through grant number FR-21-301 for the project entitled ``Metamaterials with Cracks and Wave Diffraction Problems''.
\noindent The second author of this article was supported by the Shota Rustaveli National Science Foundation of Georgia (SRNSFG) [grant number: FR-21-301, project title: ``Metamaterials with Cracks and Wave Diffraction Problems''].

\def\printchapternonum{}
% \bibliographystyle{unsrtnat}
\bibliographystyle{plainnat}
% \bibliographystyle{plain}
\bibliography{bibsource}

\end{document}