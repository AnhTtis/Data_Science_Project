\documentclass[12pt]{article}
%\documentclass[11pt]{article}
%\usepackage[left=2cm, right=2cm,
%            top=2cm
%           ]{geometry} % draft mode
\usepackage{geometry}

\usepackage{amsmath,amsthm}
\usepackage{amssymb}
\usepackage[font=footnotesize,
    %width=.75\textwidth,
   ]{caption} 
\usepackage{easy-todo}
\newcommand{\td}[1]{\todo{\normalsize \textsf{TODO: {#1}}}}
\newcommand{\ns}[1]{\todo{\normalsize \textsf{UNSURE: {#1}}}}

% \usepackage{mlmodern}
% \usepackage[T1]{fontenc}
%%%%%%%%%%%%%%%%%%%% 
%%%%%%%%  charter  fonts
\let\circledS\relax % amssymb clashed with mathdesign
\usepackage[charter,cal=cmcal]{mathdesign}\linespread{1.2}
\usepackage{XCharter}



%\usepackage[mathletters]{ucs}
%\usepackage[utf8x]{inputenc}
\usepackage[english]{babel}  
\usepackage[spacing=true,kerning=true,babel=true,tracking=true]{microtype}
%\usepackage[utopia,cal=cmcal]{mathdesign}\linespread{1.2}
%\usepackage{bera}\linespread{1.2}
%\usepackage{fourier}\linespread{1.2}
%\usepackage[light,nomath]{kpfonts}\linespread{1.2}
%\usepackage[nomath]{kpfonts}
%\usepackage[]{charter}\linespread{1.6}
%\usepackage{bm}


%\usepackage{fontspec}
%\setmainfont{PT Sans}\linespread{1.1}
%\renewcommand{\familydefault}{\sfdefault}
%\usepackage[document]{ragged2e}
\usepackage[usenames,dvipsnames,svgnames,table]{xcolor}
\usepackage[colorlinks=true,
            linkcolor=blue,
            urlcolor=black,
            citecolor=black,backref=page]{hyperref}

\usepackage{cleveref}

%\usepackage{euler}
%%%% draft
% \usepackage{euler}
% \usepackage{fontspec}
% \setmainfont[
%   AutoFakeSlant,
%   BoldItalicFeatures={FakeSlant},
% ]{Inconsolatazi4}
% \linespread{1.05}
% \usepackage{parskip}

% \usepackage{luacode}
% \begin{luacode}
% function dot2newline ( s )
%   s = s:gsub ( "([%.%?!])%s+" , "%1\\newline " ) 
%   s = s:gsub ( "([%.%?!])$"   , "%1\\newline " )
%   return s
% end
% \end{luacode}
% \AtBeginDocument{\directlua{ luatexbase.add_to_callback (
%   "process_input_buffer" , dot2newline , "dot2newline" )}}
%%%%%%%%%%


\usepackage{color}
\usepackage{braket}

\usepackage{natbib}
\usepackage{graphicx} % add pdf as image
\usepackage{multicol} % two-columns
\usepackage{comment,cancel}
\usepackage[shortlabels,inline]{enumitem} % enumerate style
\usepackage{graphicx}
\usepackage{subcaption}


%\theoremstyle{theorem}
\newtheorem{claim}{Claim}
\newtheorem{lmm}{Lemma}
\newtheorem{thm}{Theorem}
\newtheorem{prp}{Proposition}
\newtheorem{corollary}{Corollary}

\theoremstyle{definition}
\newtheorem{dfn}{Definition}

\usepackage{float}
\usepackage{tikz,xcolor}
\usetikzlibrary{patterns, fit, decorations.pathreplacing, shadings}
\usepackage{tikz-3dplot}



%%%%%%%%%%%%%%%%
% \usepackage{fancyhdr} % to change header and footers
% % Redefine plain style, which is used for titlepage and chapter beginnings
% % From https://tex.stackexchange.com/a/30230/828
% \fancypagestyle{plain}{%
%     \renewcommand{\headrulewidth}{0pt}%
%     \fancyhf{}%
% %    \fancyhead[R]{Submitted to \textit{Theoretical Economics}}
% }
%%%%%%%%%%%%%%%%%%%%%%%%%%%%%%%%


%%%%%%%%%%%%%%%%%%%%%%%%%%%%%%%%
%%%%%% SECTION TITLES
%%%%%%%%%%%%%%%%%%%%%%%%%%%%%%%%
\usepackage[explicit]{titlesec}
%
% \titleformat{name=\section}[block]{\bfseries \filcenter \thesection}{}{1em}{#1}[]
%
% \titleformat{name=\section,numberless}[block]{\bfseries\filcenter}{}{1em}{#1}[]
%
\titleformat{name=\subsection}[block]{\filcenter  \thesubsection}{}{1em}{#1}[]
\usepackage[explicit]{titlesec}

\titleformat{name=\section}[block]{\bfseries\filcenter\thesection}{}{1em}{#1}[]

\titleformat{name=\section,numberless}[block]{\bfseries\filcenter}{}{1em}{#1}[]

\titleformat{name=\subsection}[block]{\filcenter \thesubsection}{}{1em}{#1}[]
%%%%%%%%%%%%%%%%%%%%%%%%%%%%%%%%



\usepackage[hyperpageref]{backref}

\newcommand*{\ps}{p_2}

\renewcommand*{\pm}{p_1}
\newcommand*{\ls}{l_2}
\newcommand*{\lm}{l_1}
\newcommand*{\cs}{c_2}
\newcommand*{\cm}{c_1}
\newcommand*{\pro}{\mu}
\newcommand*{\an}{a_0}
\newcommand*{\cn}{c_0}
\newcommand*{\am}{a_1}
\newcommand*{\as}{a_2}
\newcommand*{\pim}{\pi_1}
\newcommand*{\pis}{\pi_2}
\newcommand*{\h}{\theta}
\newcommand*{\qcpi}{\tilde{\pi}}
\newcommand*{\qcv}{\tilde{v}}
\newcommand*{\bv}{\bar{v}}
\newcommand*{\tv}{\tilde{v}}
\newcommand*{\bpi}{\bar{\pi}}
\newcommand*{\bp}{\bar{p}}



\newcommand*{\hpi}{\hat{\pi}}
\newcommand*{\cq}{\hat{q}}
\newcommand*{\us}{\underline{s}}


\newcommand*{\m}{\mathsf{m}}
\newcommand*{\E}{\mathbb{E}}
\newcommand*{\prior}{\mu}
\newcommand*{\post}{\mu}
\newcommand*{\Post}{\mu}
\newcommand*{\belief}{\mu}
\renewcommand*{\E}{\mathbb{E}}


\newcommand*{\quasi}{quasiconcavification}
\newcommand*{\Quasi}{Quasiconcavification}


\newcommand{\R}{\mathbb{R}}

\newcommand{\co}{\text{co}\,}
\newcommand{\Prior}{\mu^0}
\newcommand{\I}{\mathcal{I}}
\newcommand{\ip}{\iota}
\newcommand{\lr}{LR}
\newcommand{\hv}{\hat{v}}
\renewcommand{\t}{t}
\newcommand{\hq}{\hat{q}}



\newcommand{\lippaper}{Lipnowski and Ravid~(2020)}
\newcommand{\seq}[2]{\{\, {#1}, \dots, {#2} \,\}}
\newcommand{\GF}{\mathcal{G}^\text{FLW}}
\newcommand{\GA}{\mathcal{G}^\text{A}}
\newcommand{\GH}{\mathcal{G}^\text{HL}}

\newcommand{\bi}{\bar{\imath}}
\newcommand{\ui}{\underline{i}}
\newcommand{\tp}{\tilde{p}}

\newcommand{\FLW}{\text{FLW}}


\DeclareMathOperator{\qc}{\mathsf{qcav}}
\DeclareMathOperator{\supp}{\mathsf{supp}}


\let\arg\relax
\DeclareMathOperator{\arg}{\mathsf{arg}} % Redifine \arg

\let\max\relax
\DeclareMathOperator{\max}{\mathsf{max}} % Redifine \max

\geometry{vmargin=2cm}
%\geometry{b5paper,margin=1.1cm}
\title{\Large Information transmission in monopolistic \\ credence goods markets%
  \thanks{We thank participants at various seminars and conferences for helpful comments.
  All errors are ours.}
}
\author{\large
  Xiaoxiao Hu%
    \thanks{Economics and Management School, Wuhan University. Email: \url{xhuah@connect.ust.hk}.} 
  \and \large
  Haoran Lei%
    \thanks{School of Economics and Trade, Hunan University. Email: \url{hleiaa@connect.ust.hk}.}
}
\date{March, 2023}
\begin{document}

\maketitle

\begin{abstract}
We study a general credence goods model with $N$ problem types and $N$ treatments.
Communication between the expert seller and the client
is modeled as cheap talk.
We find that the expert's equilibrium payoffs admit a geometric characterization, described by the quasiconcave envelope of his belief-based profits function under discriminatory pricing.
We establish the existence of client-worst equilibria, 
apply the geometric characterization to previous research on credence goods, and provide a necessary and
sufficient condition for when
communication benefits the expert.
For the binary case, we solve for all equilibria and analyze their welfare properties.



% Somewhat surprisingly, the expert's equilibrium payoffs do not increase even if he is allowed to charge differentiated prices targeted at clients with different posteriors.

\bigskip

\textbf{Keywords:} Credence Goods, Cheap Talk, Quasiconcave Envelope, Belief-based Approach

\textbf{JEL Code:} D82, D83, L12
% D82 - Asymmetric and Private Information; Mechanism Design 
% D83 - Search; Learning; Information and Knowledge; Communication; Belief; Unawareness
% L12 - Monopoly; Monopolization Strategies
\end{abstract}
\thispagestyle{plain}
\clearpage

%\renewcommand{\baselinestretch}{1.2}\normalsize
\tableofcontents

%\listoftodos

\thispagestyle{plain}
\clearpage

\renewcommand{\baselinestretch}{1.4}\normalsize

\section{Introduction}

A credence good is a product or service whose qualities or appropriateness are hard to evaluate for clients even after consumption.
Typical examples of credence goods include medical services, automobile repairs and financial advice.
Patients who undergo an expensive medical treatment 
may find it difficult to evaluate its worth after they have been cured,
clients may be skeptical about the necessity of replacing an entire engine for car repairs, and investors with limited financial literacy are not able to determine the appropriateness of a financial product even after the returns have been realized. 
In contrast, expert sellers often have superior knowledge about the appropriateness of credence goods, 
and may exploit this information advantage for their own benefits.

A common assumption in the credence goods literature is that clients do not have \textit{full choice rights}; that is,
a client can only accept or reject the expert's recommended option, or may even commit ex ante to accepting the expert's recommendation when they visit the expert.%
%
    \footnote{See \cite{dulleck2006survey} 
    for further discussions on the commitment assumption in the credence goods literature.}
%    
While this assumption applies to many important credence goods markets, such as medical services,%
    \footnote{In many countries, patients are not able to self-prescribe treatment and can only obey or reject the doctor's order.}
it does not apply to other credence goods markets such as car repair, financial advice, and taxi rides.
For example, an investor may choose a more conservative asset allocation after being advised to take on higher risks for more potential rewards, and a passenger can often insist on choosing their preferred route despite the taxi driver's suggestion of another route. To address this discrepancy, this study examines a credence goods model in which clients have the full choice right, allowing them to self-select their most desired options after receiving expert advice.
Accordingly, an expert's recommendation is modeled as a cheap-talk message and does not restrict a client's choice set.

In this paper, we consider a general $N$-by-$N$ model of credence goods, in which the client's problem falls into one of $N$ possible categories and the expert offers $N$ treatment options for the client to choose from.
The expert sets prices before the client's visit.
After privately observing the client's problem type, the expert sends a cheap-talk message to the client.
Then the client self-selects her most desired option.
We characterize the expert's payoffs following the price setting stage using the quasiconcave envelope approach introduced by \cite{lipnowski2020}, and further show that the expert's equilibrium payoffs of the whole game are the quasiconcave envelope of the belief-based monopoly profits function under discriminatory pricing.
This geometric characterization allows for a novel interpretation: one can compute the expert's equilibrium payoffs as if the client and the expert are playing an \textit{alternative game} in which
the expert sets prices after sending the message.
Somewhat surprisingly, the expert's equilibrium payoffs
do not increase
even if he is allowed to charge discriminatory prices targeted at clients with different posteriors.
Furthermore,
based on the alternative game, we show that our characterization of the expert's equilibrium payoffs also applies to previous research on credence goods, in which the client either accepts or rejects the recommended treatment.
This further strengthens the connection between our work and the credence goods literature. 

\subsection{Literature}\label{sec:lit}

This paper is mostly related to the literature of credence goods.
The concept of credence goods was first introduced by \cite{darbykarni}, and
subsequent works have examined the incentives for the expert to defraud clients.
Due to the significant information advantages that experts possess, the market for credence goods may break down if experts' behaviors are not constrained.
To address this issue, existing works have either assumed that (i) an expert is liable to fix a client's problem \citep*{pitchik1987,wolinsky1993,fong2005,liu2011,hu2020} or that (ii) the type of expert-provided good or service is observable and verifiable by clients \citep{emons1997,dulleck2009,fong2014,bester2018}.
In a comprehensive review by Dulleck and Kerschbamer~(2006), they refer to these two assumptions as the ``liability assumption'' and the ``verifiability assumption'' respectively.
In this paper, we impose the verifiability assumption.
Furthermore, as mentioned in the introduction, it is assumed that clients are free to choose their preferred option after receiving the expert's recommendation. However, in previous research on credence goods, the expert's recommendation serves not only to provide information but also to restrict the client's choice set. 
Specifically, some previous studies assume that clients must commit to accepting the recommended treatment before consulting the expert
\citep[e.g.,][]{emons1997,bester2018},
which is known as the ``commitment assumption'' in Dulleck and Kerschbamer~(2006).
Other studies assume that clients can only accept or reject the recommended treatment from the expert \citep[e.g.,][]{pitchik1987, wolinsky1993, fong2005, dulleck2009, liu2011, fong2014, hu2020}.

Our paper is closest to \cite{fong2014}
who study a specific case of our setup
with two treatments and two problem types.
In \cite{fong2014}, the expert recommends a treatment 
and the client can only accept or reject the recommended treatment.
However, in our model, the expert sends a cheap-talk message and then the client self-selects their most desired option.
Furthermore, the belief-based approach we adopt is distinct from the methodology used in \cite{fong2014}, which focuses on players' strategies in equilibria.
Our characterization of the expert's equilibrium payoffs can also be applied to the setting of \cite{fong2014},
which will be discussed in \Cref{sec:game_A}.
%\todo{Final: check the cross-refer.}



Second, our paper is also related to the literature on cheap talk, particularly studies that focus on cheap talk models with state-independent sender preferences
\citep{chakraborty2010,lipnowski2020,diehl2021}.
In this paper, we model the expert's recommendation as a cheap-talk message. Furthermore,
the expert's payoffs are determined solely by the client's choice after prices are set.
For this class of cheap talk models,
\cite{lipnowski2020} develop a quasiconcave envelope approach that can be used to
characterize the expert's equilibrium payoffs in a general setting.
In our model, the subgames following the expert's price-setting stage
are specialized case of \cite{lipnowski2020}.
However, with the additional price-setting stage before the communication, 
it is difficult to calculate the expert's equilibrium payoffs as they become the upper envelope of those quasiconcave envelopes.
One of our main contributions is to further characterize the expert's equilibrium payoffs as the 
quasiconcave envelope of his monopoly profits without communication (\Cref{thm:main}).
Based on that finding,
we demonstrate that the quasiconcave envelope approach
for cheap talk games can be applied to many other games without cheap talk communication. 
%(See \td{Add cross ref for ``new games!''})



Finally, our paper contributes to the growing field of research
at the interplay of information transmission
(or information acquisition) and optimal pricing. 
A recurring theme in this literature is 
the strategic intersection between
the seller's (buyer's) information revelation (acquisition)
and the seller's price setting.
\cite{lewis1994} consider a scenario in which a seller both sets prices and designs the information structure of the buyers' valuations for the product. 
\cite{roesler2017} examine a bilateral trade model in which the buyer is free to choose any signal structure concerning her valuation. %and then the seller sets a take-it-or-leave-it price.
They find that there exists a buyer-optimal signal structure which results in efficient trade and a unit-elastic demand.
In a subsequent work, \cite{ravid2022} investigate a similar model in which learning is costly for the buyer and the
buyer's signal structure is not observed by the seller.
They find that as the learning cost approaches zero,
the equilibria converge to the worst free-learning equilibrium. % when learning incurs no cost.
\cite{chen2020} study a bilateral trade model in which %the seller is privately informed of whether the product is of high or low quality, and 
the seller utilizes both Bayesian persuasion and pricing to signal the product quality.
They identify a unique separating equilibrium that satisfies the intuitive criterion. %, in which the seller of high-quality product discloses inefficiently more information and charges a higher price.
In our paper, the setup involves multiple products instead of a single product.
The expert (i.e., seller) employs a combination of strategic pricing and cheap-talk communication to maximize profits. 
Following the credence goods literature,
the expert sets a uniform price list and then gives recommendations to clients. 
We find that the expert's equilibrium payoffs
admit a geometric characterization and,
somewhat surprisingly,  
his equilibrium payoffs
do not increase even if he is allowed to charge differentiated prices targeted at buyers with different posteriors.
See \Cref{sec:game_A} for further discussions on this point. 


%\todo{Final: review/rewrite the structure of the paper before submitting.}
The rest of the paper is as follows.
\Cref{sec:model} presents the $N$-by-$N$ model and our main findings.
\Cref{sec:game_A} applies the geometric 
characterization to previous credence goods research where talk is not cheap.
\Cref{sec:binary} characterizes all the equilibria in the binary case.
\Cref{sec:value} gives a necessary and sufficient condition for when communication benefits the expert.
\Cref{sec:conclude} concludes.
Omitted proofs are relegated to Appendix.


\section{Model} \label{sec:model}

\subsection{Environment}

There are two players: Client (she) and Expert (he).
Both players are risk-neutral.
Client has a problem but is uncertain about the problem type $t$, whose distribution follows
$\Prior$ with support $T \equiv \seq{t_1}{t_N}$.
If a problem of type $t_i$ remains unresolved, Client will suffer a disutility of $l_{i}$ for $i \in \seq{1}{N}$.
Client consults Expert who privately observes
the problem type.
%Assume that Expert incurs no cost in diagnosing the problem type.
Expert is capable of providing $N$ different treatments, denoted by
$\{a_j\}_{j=1}^N$, that vary in their efficacy.
Specifically,
treatment $a_j$ 
is effective in fully resolving a problem of type $t_i$ for all $i \le j$, but 
cannot resolve a problem of type $t_{i'}$ with $i' > j$;
treatment $a_N$ is a panacea that resolves all problems.\footnote{%
    Our $N$-problem, $N$-treatment setting nests the two-problem,  two-treatment setting that is commonly used in the credence goods literature (e.g., \cite{fong2005}, \cite{liu2011}, \cite{fong2014} and \cite{bester2018}). \citet{pesendorfer2003} and \cite{liu2022} are two exceptions as they study models with a continuum of problem types.
} 
Denote by $c_j$ the cost of treatment $a_j$ for Expert.
Expert's treatments are credence goods: Once a problem has been resolved through the use of treatment $a_j$ with $j\ge 2$, Client remains uncertain about whether an alternative treatment, such as $a_{i}$ with $i < j$, could have resolved
her problem as well.

As previously mentioned in the introduction, we deviate from previous research by (i) focusing on the information transmission between the players and (ii) granting Client the full choice right.
In our model, Expert sends a cheap-talk message, interpreted as a treatment recommendation, to Client after the diagnosis. Upon receiving this message, Client chooses $a \in A \equiv \{\an, a_1, \dots, a_N\}$, where $\an$ indicates the decision to not purchase any treatment. 
It is assumed that Client is able to observe and verify the type of performed treatment (i.e., the verifiability assumption).

Timing of the game is as follows:
%
\begin{enumerate}
\item 
  Expert posts a price list $p = (p_1,\dots, p_N) \in \R_+^N$, where $p_j$ is the price for treatment $a_j$ for $j \in \set{1,\dots,N}$. 
\item
  Expert privately observes the problem type $t$, which is drawn  according to the distribution $\Prior$ with support $T = \set{t_1, \dots, t_N}$.
\item 
  Expert sends a message $m$, from the message set $M$,
  to Client. 
  Suppose $|M| \ge N + 1$.   
\item 
  After receiving $m$, Client chooses $a \in A = \set{a_0, a_1, \dots, a_N}$.
\item 
  Client and Expert obtain payoffs $u^C(t, a, p)$ and $ u^E ( a, p)$ respectively.
  When $a = \an$, their payoffs are $u^C(\t_i, \an, p) = -l_i$
  and $ u^E ( \an, p) = 0$ for all $i \in \seq{1}{N}$ and $p \in \R_+^N$; when $a \ne \an$,
  \begin{equation}
    u^C(\t_i, a_j, p) = -p_j -\mathbf{1}(j < i) \cdot l_i,
  \quad
   u^E ( a_j, p) = p_j - c_j
  \label{eq:payoff}  
  \end{equation}
where $\mathbf{1}(j < i)$ is the indicator function that takes the value of $1$ when $j < i$ and $0$ otherwise.
%For convenience of expression, let $\cn = 0$ and $p_0 = 0$ 
%so that the payoff functions \eqref{eq:payoff} apply to $a_j = \an$ as well.
\end{enumerate} 

Throughout this paper, we impose the following two restrictions.
First, the cost of a treatment is increasing in its index:
$c_{j} > c_{i}$ whenever $j > i$.
This assumption is realistic as a treatment of a higher index is always more likely to resolve the problem.
Second, all treatments are efficient in the sense that $s_j \equiv l_j - c_j > 0$ for all $j \in \seq{1}{N}$.  
We do \textit{not} require that the loss caused by a problem is increasing in its type index (i.e., $l_j > l_i$ whenever $j > i$). The implication is that in our setup, a problem of a higher type is more difficult to resolve, but may not cause a higher loss.

\subsection{Solution concept}

Our solution concept is the \textit{Expert-optimal perfect Bayesian equilibrium} (henceforth, \textit{equilibrium}).
To elaborate, we first define the notion of $p$-\textit{equilibrium}
for the subgame following Expert posting some price list $p$.

\begin{dfn}[$p$-equilibrium]
Given a price list  $p \in \R_+^N$,
a \emph{$p$-equilibrium} consists three maps:
Expert's signalling strategy $\sigma: T \to \Delta (M)$;
Client's strategy $\rho : M \to \Delta (A)$;
and Client's belief updating
$\beta: M \to \Delta T$; such that 
\begin{enumerate}[(a)]
\item 
  Given Expert's signalling strategy $\sigma$, Client's posterior belief after receiving $m$ is
  obtained from the prior $\Prior$ using Bayes's rule:
  $$\beta(t_i \mid m) = \frac
    {\Prior(t_i) \sigma (m \mid t_i) }
    {\sum_{k=1}^N \Prior(t_k) \sigma (m \mid t_k)}
  $$
  for $t_i \in T$;   
\item
  Given the belief updating $\beta$, Client's strategy
  $\rho (\cdot \mid m)$ after receiving message $m$ is supported on 
  $$
  \arg\max_{a \in A} \sum_{t \in T} u^C(a, t , p) \beta(t \mid m)
  $$
  for $m \in M$;
\item 
  Given Client's strategy $\rho$, Expert's signalling strategy
  $\sigma (\cdot \mid t)$ is supported on 
  $$
  \arg\max_{m   \in M} \sum_{a \in A} \rho(a \mid m)  u^E ( a, p)
  $$
  for $t \in T$. 
\end{enumerate}
\label{dfn:p-eq}
\end{dfn}

When there exist multiple $p$-equilibria under the same price list,
assume (one of) those yielding the highest payoff for Expert, referred as \textit{Expert-optimal $p$-equilibria},  
will be played. Specifically,
given a price list $\tilde{p}$, 
Expert's expected payoff in any Expert-optimal
$\tilde{p}$-equilibrium%  
  \footnote{The existence of Expert-optimal $p$-equilibrium for any $p \in \R^N_+$
  is guaranteed by \Cref{lmm:p-eq-payoffs}.}
is:
\begin{equation}
\begin{split}
  &U^E(\tilde{p}) = \sum_{i=1}^N \Prior(\t_i) \left( \sum_{a \in A} \rho (a \mid {m_i})  u^E ( a, \tilde{p}) \right)
\end{split}    \label{eq:p-eq-payoff}
\end{equation}
where $(\beta, \rho, \sigma)$ is \textit{some} Expert-optimal $\tilde{p}$-equilibrium and $m_i \in \supp\sigma(\cdot \mid \t_i)$.
We refer to $U^E(\tilde{p})$ as Expert's maximal
$\tilde{p}$-\textit{equilibrium payoff}, i.e., his 
ex ante payoff in an Expert-optimal 
$\tilde{p}$-equilibrium.
Note that one can use any Expert-optimal $\tilde{p}$-equilibrium to compute $U^E(\tilde{p})$
because, given the same price list $\tilde{p}$, all Expert-optimal $\tilde{p}$-equilibria yield the same payoff for Expert.
Also, in \Cref{eq:p-eq-payoff},   
$m_i$  can be any message from the support of $\sigma(\cdot \mid \t_i)$. The reason is that 
in a $\tilde{p}$-equilibrium, Expert is indifferent among sending any message from the support of $\sigma (\cdot \mid t_i)$ given the problem type $t_i$.

Finally, Expert chooses the price list to maximize $U^E(p)$: 
\begin{equation}
\max_{p \in \R^N_+} U^E(p). \label{eq:eq-p}  
\end{equation}
To sum up, an equilibrium of our model is specified by
a price list $p^*$ that solves \eqref{eq:eq-p}
and the Expert-optimal $p$-equilibria for all $p \in \R^N_+$.
While we consider only pure pricing strategies in problem~\eqref{eq:eq-p},
this is not a restriction of generality.
The reason is that an equilibrium in which Expert takes
a mixed price-setting strategy exists if and only if there are multiple price lists that yield the highest payoffs for Expert; when such an equilibrium exists, it can be expressed as a convex combination of the corresponding equilibria with pure pricing strategies that also yield the highest payoff for Expert.

\paragraph{Roadmap}
Methodologically, we first use the quasiconcave envelope approach invented by \citet{lipnowski2020}
to describe Expert's maximal $p$-equilibrium payoffs.
Based on that result,  
we further give a geometric characterization of
Expert's equilibrium payoffs.
This characterization also implies the existence of some specific kind of equilibria in which Client does not benefit from Expert's services. 
%In this equilibrium, Expert essentially sets a list of monopoly prices based on Client's possible posteriors and Client is indifferent between $\an$ and choosing her most desired treatment at any possible posterior.

\subsection{Expert's $p$-equilibrium payoffs}

We calculate Expert's $p$-equilibrium payoffs using the belief-based approach,
which focuses on the equilibrium outcomes 
and abstracts away Expert's signalling strategy $\sigma$ and Client's strategy $\rho$. 
Let $\set{\mu^1, \dots , \mu^K}$ be a set of beliefs 
where $\mu^k \in \Delta(T)$ for each $k$, and
we call this set a \textit{splitting of $\mu$}
if there exist $K$ non-negative numbers, $\set{{\lambda_1}, \dots, {\lambda_K}}$
with $\sum_{k=1}^K \lambda_k =1$, 
such that $\sum_{k=1}^K \lambda_k \mu^k = \mu$.
We use a splitting of the prior $\Prior$ to 
describe Client's possible posteriors in a game outcome. 
An \textit{(Expert-optimal) $p$-equilibrium outcome} is a pair
$(\seq{\mu^1}{\mu^K}, v)$ if there exists some (Expert-optimal)
$p$-equilibrium in which the ex ante distribution of
Client's posteriors has $\seq{\mu^1 }{\mu^K}$ as its support
and Expert's $p$-equilibrium payoff is $v \in \R_+$.
An \textit{equilibrium outcome} is a triple
$(\tilde{p}, \seq{\mu^1}{\mu^K}, \tilde{v})$ if the price list $\tilde{p}$ solves problem \eqref{eq:eq-p} and
$(\seq{\mu^1}{\mu^K}, \tilde{v})$ is a 
$\tilde{p}$-equilibrium outcome.

Let $\bar{a} (\belief, p)$ be the set of Client's (possibly mixed)
optimal choices given belief $\belief$ and price list $p$.
Correspondingly,
the set of Expert's possible payoffs is:
$$
V(\belief ; p) \equiv u^E (\bar{a} (\belief , p ) , p).
$$
Here $V (\,\cdot \, ;  p) : \Delta T \rightrightarrows \mathbb R$ is a correspondence parameterized by $p \in \R_+^N$.
Let $\bv ( \mu \, ; p) \equiv \max V(\mu \,; p)$.
For any $p \in \R^N_+$, it follows from
Berge's theorem that
$V (\,\cdot \, ;  p)$ is a Kakutani correspondence on $\Delta(T)$ 
and the function $\bv (\, \cdot \, ; p)$ is upper semicontinuous on $\Delta(T)$.
Fixing any $p \in \R_+^N$,
Expert's maximal $p$-equilibrium payoff $U^E(p)$,
as defined in \eqref{eq:p-eq-payoff},
is equal to the value of the following maximization problem:%
    \footnote{The expression of $U^E(p)$ by \eqref{eq:optimization-def} is essentially the same with Lemma~1 of \cite{lipnowski2020}.}
%    
\begin{equation}
\begin{split}
U^E(p) = & \max_{\{\mu^1, \dots, \mu^K \}} {v}  \\
\text{subject to }
& \text{(i) $ v \in V (\mu^k ; p)$ for all }k \in \seq{1}{K};  \\ 
& \text{(ii) $\{\belief^{1}$, \dots, $\belief^{K} \}$ is a belief--splitting of $\mu^0$}.  
\end{split} \tag{$\mathcal{M}$} 
\label{eq:optimization-def}
\end{equation}
Condition (i) combines restrictions (b) and (c) of \Cref{dfn:p-eq} and
condition (ii) captures restriction (a) of \Cref{dfn:p-eq}.
In a seminal paper,
\lippaper{} show that
the value function for problem~\eqref{eq:optimization-def} allows for
a geometric representation.
Specifically, call $\qc \bv (\,\cdot\, ; p)$ the \textit{quasiconcave envelope}
of $\bv ( \,\cdot\, ; p)$ if, fixing $p$, function
$\qc \bv ( \,\cdot\, ; p)$ is the pointwise lowest quasiconcave function that is everywhere above $\bv (\,\cdot\, ; p)$.
Expert's maximal $p$-equilibrium payoff as described by \eqref{eq:optimization-def} is exactly
$\qc \bv (\Prior ; p)$ where $\Prior$ is the prior.

\begin{lmm}\label{lmm:p-eq-payoffs}
For all $p \in \R^N_+$,
an Expert-optimal $p$-equilibrium exists, in which
Expert's expected payoff is $U^E (p) = \qc \bv (\Prior ; p)$.
\end{lmm}

\begin{proof}[Proof of \Cref{lmm:p-eq-payoffs}]
Theorem~2 of \cite{lipnowski2020}
has this
\namecref{lmm:p-eq-payoffs} 
as a special case. 
\end{proof}

In \Cref{fig:qcav}, we illustrate Expert's value function $\bv(\mu ; \tp)$ and its
quasiconcave envelope $\qc \bv (\cdot; \tp)$ when $N=2$.
Given the price list $\tp$,
Client chooses $a_1$ ($a_2$) when she is sufficiently confident that the problem type is $t_1$ ($t_2$), and chooses $\an$ if she is not sufficiently certain of either type. 
At the threshold belief $\tilde{\mu}$ ($\hat\mu$), 
Client is indifferent between choosing $\an$ and $a_1$ ($a_2$).
Therefore, the correspondence $V(\cdot; \tp)$, denoted by the solid lines, is multi-valued at $\tilde{\mu}$
and $\hat\mu$.
The value function $\bv (\mu; \tp)$ is denoted by
the dashed lines in
\Cref{fig:2d-v-bar}, and the quasiconcave envelope
denoted by the dashed lines in 
\Cref{fig:2d-qcav-v}.
By \Cref{lmm:p-eq-payoffs}, 
communication helps Expert achieve a higher $\tp$-equilibrium payoff
when the prior $\Prior$ lies between $\hat{\mu}$ and $\tilde{\mu}$.

\begin{figure}[H]
     \centering
     \begin{subfigure}[b]{0.48\textwidth}
         \centering
         \includegraphics[width=\textwidth]{xiong-tu/2d-v-bar.pdf}
        \subcaption{$\bv (\mu ; 
        \tp)$}
		\label{fig:2d-v-bar}
     \end{subfigure}
     \hfill
     \begin{subfigure}[b]{0.48\textwidth}
         \centering
         \includegraphics[width=\textwidth]{xiong-tu/2d-quasi-v.pdf}
         \subcaption{$\qc \bv (\mu ; \tp) $}
	\label{fig:2d-qcav-v}
     \end{subfigure}
    \caption{The value function $\bv(\cdot, \tp)$ and its quasiconcave envelope $\qc \bv(\cdot, \tp)$, as denoted by the dashed lines in Figure~1a and Figure~1b respectively.}
\label{fig:qcav}
\end{figure}

\subsection{Expert's equilibrium payoffs}

Our solution concept implies that
Expert's equilibrium payoff should be his \textit{highest} Expert-optimal $p$-equilibrium payoff
across all price lists. By \Cref{lmm:p-eq-payoffs}, Expert's equilibrium payoff $\max_{p \in \R^N_+ } U^E(p)$ can be written as:
\begin{equation}
\max_{p \in \R^N_+ } \qc \bv (\Prior ; p).
\label{eq:get-eq-profits}  
\end{equation}
However, using \eqref{eq:get-eq-profits} to
calculate Expert's equilibrium payoffs
seems intractable as it requires
calculating the quasiconcave envelopes for all $p$. 
One of our main results is that
in computing Expert's equilibrium payoffs,
one can switch the order of the two operators, $\max_p$ and $\qc$, in \eqref{eq:get-eq-profits}.
By doing this, the computation of the quasiconcave envelope only needs to be performed once, thus greatly simplifying the computations.
We formally state this result in \Cref{thm:main}.
To introduce some notations, let $\mu_i \equiv \mu(t_i)$,
$\bp_i(\mu) \equiv \sum_{k=1}^i \mu_k l_k$ be the belief-based monopoly price for treatment $a_i$,
and $\pi_i(\mu) \equiv \bp_i(\mu) - c_i$ be the
corresponding profits.
Note that Client is indifferent between $a_0$ and purchasing treatment $a_i$  given its price $\bp_i (\mu)$ under belief $\mu$. 
Furthermore,
let ${\bpi}(\mu)  \equiv \max_{p \in \R_+^N} \bv (\mu; p)$ be the belief-based monopoly profits for Expert under discriminatory pricing.
Since Client has unit demand, profits $\bpi (\belief)$ can be achieved
when Expert sets the (belief-based) monopoly price list $(\bp_1(\mu), \dots, \bp_N(\mu))$ and Client chooses some Expert-preferred option:
\begin{equation}\label{eq:unit-demand}
\bpi (\belief) = \max \{0, \pi_1(\mu), \dots, \pi_N(\mu) \}.  
\end{equation}
These belief-based profits functions allow straightforward geometrical interpretations. 
A \textit{point} is a belief plus a value, $(\mu, v) \in \R^{N+1}$.
For each $i \in \seq{1}{N}$, the set $\set{(\mu, v) : v = \pi_i(\mu)}$ is a hyperplane in $\R^{N+1}$, and
we use $\pi_i(\mu)$ to refer to this hyperplane.
\Cref{eq:unit-demand} implies that
$\bpi(\mu)$ is the upper envelope of the $N+1$ hyperplanes.
It can be verified that $\bpi(\mu)$ is continuous and convex;
furthermore, if $\seq{\mu^1}{\mu^K}$ is a splitting of $\mu$ with $\mu = \sum_{k=1}^K \lambda_k \mu^k$, 
then 
\begin{equation}
\bpi(\mu) \le  \sum_{k=1}^K \lambda_k \bpi (\mu^k) 
\label{eq:convex}  
\end{equation}
and the equality of \eqref{eq:convex}
holds if and only if the $K + 1$ points, $(\mu, \bpi(\mu))$ and $\left(\mu^k, \bpi(\mu^k)\right)$ for $k \in \seq{1}{K}$, are all on the same hyperplane 
$\pi_i (\mu)$ for some $i \in \seq{1}{N}$.

\begin{thm}\label{thm:main}
Expert's equilibrium payoff is $\qc {\bpi}(\mu^0)$.
\end{thm}

\begin{proof}[Proof of \Cref{thm:main}]
  See \Cref{app:a2}.
\end{proof}

Below we provide a proof sketch of \Cref{thm:main}. First,
\Cref{thm:main} can be stated equivalently as below:
\begin{equation}
  \max_{p \in \R^N_+ }  \qc \bv (\belief ; p) =
  \qc \max_{p \in \R^N_+ } \bv (\belief ; p) \text{ for all $\belief \in \Delta(T)$.}
  \label{eq:switch}  
\end{equation}
Equation~\eqref{eq:switch} is nontrivial.
As Expert posts the price list before observing the problem type, the price list cannot be contingent on
the sent message or Client's posterior belief.
However, the right-hand side of Equation~\eqref{eq:switch},
$\qc \max_{p \in \R^N_{+}} \bv (\belief ; p)$, 
allows the price list to be contingent on Client's belief.
It is not hard to verify that
the right-hand side of Equation~\eqref{eq:switch} is
weakly greater than the left-hand side:
$\max_{p \in \R^N_+ }  \qc \bv (\belief ; p) \le 
 \qc \max_{p \in \R^N_+ } \bv (\belief ; p)$ for all $\mu$.

To prove the other direction, it suffices
to find a special 
price list $p'$ such that $\qc \bv (\belief ; p') \ge \qc \max_{p \in \R_+^N} \bv (\belief ; p)$.
The key ingredient of our proof is the construction of
the desired price list $p'$, which essentially
consists of monopoly prices targeted at Client's possible posteriors.
Specifically, fix any $\mu' \in \Delta(T)$ and
let $v' \equiv \qc \bpi (\mu')$.
The \quasi{} implies that $v'$ can be attained by some splitting of $\mu'$, say $\set{\mu^1, \dots, \mu^K}$,
such that $v' = \bpi (\mu^k)$ for $k \in \seq{1}{K}$.
Moreover, Client purchases some treatment \textit{with probability one} under each posterior.
\Cref{lmm:intermediate} formally states this result.

\begin{lmm} \label{lmm:intermediate}
For all $\mu \in \Delta (T)$,
there exists a splitting of $\mu$,
$\seq{\belief^{1}}{\belief^{K}}$ with 
$K \le N$, and $K$ different indices $\{i_1, i_2, \dots, i_K\}$ such that
$\qc \bpi (\mu)  = \bpi (\mu^k) = \bp _{i_k} (\mu^k) - c_{i_k}$ for $k \in \seq{1}{K}$.
\end{lmm}

\begin{proof}[Proof of \Cref{lmm:intermediate}]
  See the proofs for Claims~\ref{clm:bebu} and \ref{clm:intermediate} in \Cref{app:a2}.
\end{proof}

Under the splitting described in \Cref{lmm:intermediate},
Client's choices at each posterior are non-stochastic and never identical.
Therefore, there exists some splitting $\{\mu^1, \dots, \mu^K\}$ of the fixed $\mu'$ and an \textit{into} function
$\varphi : \seq{1}{K} \to \seq{1}{N}$
such that
\begin{equation}
    \qc \bpi (\mu') =  
    \bpi(\mu^{k}) = 
    \bp_{\varphi(k)} (\mu^{k}) - c_{\varphi(k)}, 
    \text{ $ \forall  k \in \seq{1}{K}$.}
\end{equation}
Set $p' = (p'_1, \dots, p'_N)$ with
$p_i' = \bp _ {\varphi(k)} (\mu^{{ k}})$
if $i \in \seq{\varphi(1)}{\varphi(K)}$
and otherwise $p_i' > l_N$.%
    \footnote{The exact values of $p_i'$ for $i \not\in \seq{\varphi(1), \varphi(2)}{\varphi(K)}$ are not essential. Any price $p_i'$ suffices as long as it is prohibitively high such that Client prefers $a_0$ to $a_i$ under any belief.
    }
Given the constructed price list $p'$, it can be verified that
choosing the treatment $a_{\varphi(k)}$ is optimal for Client under belief $\mu^k$, and
$u^E (a_{{\varphi(k)}} , p') =  \bpi(\mu^{k}) = \qc \bpi(\mu')$ for all $k \in \seq{1}{K}$.%
%
    \footnote{For details, see \Cref{clm:final} in \Cref{app:a2}.}
%    
Therefore, the splitting 
$\seq{\mu^1}{\mu^K}$ and the payoff $\qc \bpi(\mu')$ 
constitute a $p'$-equilibrium outcome.
The proof sketch concludes.
  
Indeed, the constructed price list $p'$ and  $\left(\seq{\mu^1}{\mu^K}, \qc \bpi(\mu')\right)$ constitute an equilibrium outcome at prior $\mu'$.
Our proof sketch also implies the existence of some specific kind of equilibria for any prior.
An equilibrium is called a \textit{Client-worst equilibrium} if 
Client does not benefit from Expert's services in that equilibrium.
When $\qc \bpi(\Prior) = \bpi (\Prior)$, there exists a \textit{silent} Client-worst equilibrium in which Expert sets the monopoly-price list $(\bp_1(\mu^0) , \dots , \bp_N(\mu^0) )$
and reveals \textit{no information} to Client.
When $\qc \bpi(\Prior) > \bpi (\Prior)$,
the splitting $\seq{\mu^1}{\mu^K}$ and the
constructed $p'$ in the proof sketch
lead to a non-silent Client-worst equilibrium as stated in \Cref{cor:client-worst}.  

\begin{corollary}
  Suppose $\qc \bpi(\Prior) > \bpi (\Prior)$.
  Then there exists a splitting $\seq{\mu^1}{\mu^K}$ of $\Prior$ for some $K \in [2 , N]$,
  an into function $\varphi : \{1,\dots, K \} \to \{1, \dots ,N\}$ 
  and a Client-worst equilibrium in which:
\begin{enumerate}[(a)]
\item
  The splitting is
  $\seq{\belief^{1}}{\belief^{K}}$ and
  $\qc \bpi(\Prior) = \bpi (\belief^{k})$ for each $k \in\seq{1}{K}$;
\item 
  Expert posts the price list $p' = (p'_1, \dots, p'_N)$ with $p_i' = \bp_{\varphi(k)} (\mu^k)$ if $i \in \seq{\varphi(1)}{\varphi(K)}$ and otherwise $p_i' > l_N$;
\item 
  At each posterior $\mu^k$, Client adopts the pure strategy of choosing treatment $a_{\varphi(k)}$.
\end{enumerate}
\label{cor:client-worst}  
\end{corollary}

The geometric characterization presented in \Cref{thm:main}
suggests that we can calculate Expert's equilibrium payoffs
as if he can set discriminatory prices targeted at Clients with different posteriors.
In \Cref{sec:game_A},
we formalize this intuition by proposing an \textit{alternative game} in which discriminatory pricing is allowed.
Specifically, we show that Expert's (maximal) equilibrium value remains the same in the alternative game,
the proof of which relies heavily on
the Client-worst equilibria described in \Cref{cor:client-worst}.
Additionally,
there generally exist many other equilibria
besides the Client-worst equilibria
in \Cref{cor:client-worst}, including both non-Client-worst equilibria in which
Client benefits from Expert's services and 
other Client-worst equilibria.
In \Cref{sec:binary},
we explicitly solve for all equilibria and characterize
Client's possible welfare when $N=2$.

\section{Application to previous research\label{sec:game_A}}

Our characterization of Expert's equilibrium payoffs 
%can be interpreted in a game theory context.
%Specifically,
$\qc \bpi(\Prior)$ can be viewed as Expert's
equilibrium payoffs in an alternative game $\GA$,
where Expert is allowed to use discriminatory pricing instead of posting a uniform price list.
This interpretation enables us to apply the geometric characterization to previous works 
in which Client does not have the full choice right,
thereby strengthening the connection between our paper and the  literature on credence goods.
From a methodological perspective, this application also demonstrates that
the existing methods and insights in cheap talk and information design can be leveraged to understand and solve the problems associated with credence goods.

\subsection{Alternative game $\GA$}

The alternative game $\GA$ shares the same $N$-by-$N$ credence goods environment as that of our model, so we do not repeat its description here.
Timing of $\GA$ is as follows:
\begin{enumerate*}[(1)]
\item
  Expert privately observes the problem type $t \in T$;
\item 
  Expert sends a cheap-talk message $m \in M$ to Client;
\item  
  Expert proposes a price list $p \in \R_+^N$;   
\item 
  Client chooses $a \in A$;
\item 
  Players' payoffs are realized.
\end{enumerate*} 
A \textit{perfect bayesian equilibrium} (and henceforth \textit{equilibrium}) of $\GA$ consists four maps:
(i) Expert's signalling strategy
$\sigma^A: T \to \Delta(M)$,
(ii) Expert's pricing strategy
$p^A: T \times M \to \R_+^N$,
(iii) Client's choice strategy $\rho^A: M \times  \R_+^N \to \Delta(A)$, and
(iv) Client's belief updating
$\beta^A: M \times \R_+^N \to \Delta(T)$; such that
\begin{enumerate}[(a)]
\item 
  Given $\sigma^A$ and $p^A$, 
  Client's posterior belief after observing $m$ and $p$ is
  obtained from the prior $\Prior$ using Bayes's rule:
  $$
    \beta ^A(t_i \mid m,p) = \frac
    {\Prior(t_i) \sigma^A (m \mid t_i) \Pr(p \mid m, t_i) }
    {\sum_{k=1}^N \Prior(t_k) \sigma^A (m \mid t_k) \Pr (p \mid m, t_k)}
  $$
  for $t_i \in T$, where $\Pr(p \mid m,t_i)$ takes the value of $1$ if $p^A (t_i, m) = p$ or otherwise $0$.     
  As for belief updating on the off-equilibrium paths, we restrict that
  if the sent message ${m} \not\in \supp \sigma^A (\cdot \mid t)$ for all $t \in T$  or the proposed price list
  $p \neq p^A (t, {m})$ for all $t \in T$, then Client's posterior should put probability one on (one of) the problem type with the least potential surplus.\footnote{Under this restriction, Expert's maximal off-equilibrium-path payoffs will be $\min \set{l_1-c_1, \dots ,l_N-c_N}$, which is lower than $\qc \bpi (\Prior)$. As discussed later, Expert's maximal equilibrium payoffs in $\GA$ is still $\qc \bpi (\Prior)$, and he will not deviate by posting some off-equilibrium-path price list or sending some off-equilibrium-path message.
  }
  
\item
  Given the belief updating $\beta^A$, Client's strategy
  $\rho^A (\cdot \mid m,p)$ after observing $m$ and $p$ is supported on 
  $$
  \arg\max_{a \in A} \sum_{t \in T} u^C(a, t , p) \beta^A(t \mid m,p)
  $$
  for $m \in M$ and $p \in \R_+^N$.
  
\item 
  Given Client's $\rho^A$ and Expert's $p^A$, Expert's signalling strategy
  $\sigma^A (\cdot \mid t)$ is supported on 
  $$
  \arg\max_{m   \in M} \sum_{a \in A} \rho^A \left(a \mid m , p^A(t, m) \right)  u^E \left( a, p^A(t, m)\right)
  $$
  for $t \in T$;
\item 
  Given Client's $\rho^A$ and sent message $m$, Expert's pricing strategy satisfies:
  $$
  p^A(t,m) \in \arg\max_p
    \sum_{a \in A} \rho^A(a \mid m,p)  u^E ( a, p).
  $$  
\end{enumerate}
Similar to the notions of the main model,
$(\seq{\mu^1}{\mu^K}, \tilde{v}, \tilde{p})$ is called an \textit{outcome of}  $\GA$
where $\seq{\mu^1}{\mu^K}$ is the \textit{ex ante} distribution of Client's posteriors, $\tilde{v}$ is Expert's \textit{ex ante} payoff and $\tilde{p}$ is the proposed price list, and
an outcome of $\GA$ is called an \textit{equilibrium outcome} of $\GA$ if it corresponds to some equilibrium of $\GA$.


Say that Expert's pricing strategy does \textit{not signal the problem type} if
$p^A (t, m) = p^A (t', m)$ for all $m \in M$ and all $t,t' \in T$.
We argue that as long as the message space is sufficiently rich,
it is without loss of generality to focus on equilibria
in which Expert's pricing strategy does not signal the problem type. 
The reason is that any equilibrium $(\sigma^A, p^A, \beta^A, \rho^A)$ can be transformed into another equilibrium
$(\tilde\sigma^A, \tilde{p}^A, \tilde\beta^A, \tilde\rho^A)$
that 
leads to the same outcome, but with a pricing strategy not signalling the problem type.
The trick for the transformation is that for any pair $(\sigma^A , p^A)$, we can find a more Blackwell-informative signalling strategy $\tilde{\sigma}^A$
such that
observing both the posted price list and sent message
under  $({\sigma}^A , {p}^A)$ is informationally equivalent for Client to observing the message alone under
$\tilde{\sigma}^A$.
\Cref{fig:signal} illustrates the transformation process. 
\Cref{fig:signal}(a) represents the original strategy pair $(\sigma^A, p^A)$, and
\Cref{fig:signal}(b) represents the transformed strategy pair $(\tilde \sigma^A, \tilde p^A)$.
Each node represents a problem type, a sent message or a proposed price list, and the numbers on the paths indicate the transition probabilities. 
In \Cref{fig:signal}(a), the pricing strategy signals the type information in that $p^A(t_1, m_1) \ne p^A(t_2, m_1)$, and Client's  belief will be dependent on the proposed price list after Client observes $m_1$. 
In \Cref{fig:signal}(b), the final nodes and transition probabilities remain unchanged; however,
there are four new distinct messages indexed by $\zeta(1,1)$,
$\zeta(2,3)$, $\zeta(1,2)$ and $\zeta(3,3)$.
When observing the message indexed by $\zeta(i,j)$, Client updates her belief as if she observed $m_i$ and $p^j$
in the situation described by \Cref{fig:signal}(a).
Therefore, in the situation described by \Cref{fig:signal}(b), Client's posterior is independent of the proposed price list, and her posterior and the price list at each final node are the same as in
the situation described by \Cref{fig:signal}(a).

Formally, for any equilibrium $(\sigma^A, p^A, \beta^A, \rho^A)$, let 
$M' \equiv \bigcup_{t \in T} \supp \sigma^A (\cdot \mid t ) $ be the set of messages that may be sent under $\sigma^A$,
and let $P' \equiv \set{p^A(t, m) : t \in T, m \in M'} $ be the set of price lists
that may be proposed under $(\sigma^A, p^A)$.
We index the set $P'$ by $\seq{p^1}{p^K}$
and let
$p^{k(i,j)} \equiv p^A(t_i, m_j)$ be the proposed price list under $p^A$ given $t_i$ and $m_j$ where $k(i,j)$ is the index of the price list.
Consider an into function
$\zeta: M' \times P' \to M \setminus M'$
and a new strategy profile
$( \tilde{\sigma}^A , \tilde{p}^A, \tilde\beta^A, \tilde\rho^A)$ with 
\begin{equation*}
\begin{split}
  \tilde{\sigma}^A ( m_{\zeta(j, k(i,j))}  \mid t_i) = \sigma^A (m_j \mid t_i), 
  \quad 
  \tilde{p}^A (t_i  ,  m_{\zeta(j, k(i,j))}) =
  p^{k(i,j)} = p^A(t_i, m_j),  
  \\
  \tilde\beta^A (m_{\zeta(j, k(i,j))} , 
  p^{k(i,j)})  = \beta^A( m_j , p^{k(i,j)}), 
  \quad 
  \rho^A(a \mid t_i, m_{\zeta(j, k(i,j))}) =  \tilde\rho^A(a \mid t_i, m_j).
\end{split}  
\end{equation*}
for all $t_i \in T$ and all $m_j \in M'$.
Since observing both $m_j$ and $p^{k(i,j)}$ under 
$({\sigma}^A , {p}^A)$ is informationally equivalent for Client to observing $m_{\zeta(j, k(i,j))}$ under
$( \tilde{\sigma}^A , \tilde{p}^A)$,
the profile $( \tilde{\sigma}^A , \tilde{p}^A, \tilde\beta^A, \tilde\rho^A)$ is an equilibrium
in which the pricing strategy does not signal the problem type.


\begin{figure}[H]
\begin{subfigure}[b]{0.48\textwidth}
         \centering
         \includegraphics{xiong-tu/tree1.pdf}
         \caption{Expert's strategy pair $({\sigma}^A , {p}^A)$}
\end{subfigure}
\hfill
\begin{subfigure}[b]{0.48\textwidth}
         \centering
         \includegraphics{xiong-tu/tree2.pdf}
         \caption{Expert's strategy pair $(\tilde{\sigma}^A , \tilde {p}^A)$}
\end{subfigure}
\caption{\label{fig:signal}%
  Transforming $(\sigma^A , p^A)$ into $(\tilde{\sigma}^A , \tilde {p}^A)$
  so that the pricing strategy does not signal the problem type 
  }  
\end{figure}

When Expert's pricing strategy has no signalling effect,
Expert's possible payoffs given Customer's posterior $\mu$
constitute the interval
$[0 , \bpi(\mu)]$ and his value function 
is exactly $\bpi (\mu)$.
Therefore, Expert's maximal equilibrium payoffs in $\GA$
are $\qc \bpi(\Prior)$.
Moreover, the Client-worst equilibrium described in
\Cref{cor:client-worst}
can be reinterpreted as an equilibrium of game $\GA$.
We summarize these results in \Cref{prp:new-client-worst}.


\begin{prp}
\begin{enumerate}[(a)]
\item 
  Expert's maximal equilibrium payoffs of $\GA$ are
  $\qc \bpi(\Prior)$.
\item
  Suppose $\qc \bpi(\Prior) > \bpi (\Prior)$. Then there exists an into function $\varphi : \{1,\dots, K \} \to \{1,... ,N\}$ for some integer $K \in [2 , N]$
  and an equilibrium of $\GA$ in which
  (i) the belief splitting is
  $\seq{\belief^{1}}{\belief^{K}}$ satisfying
  $\qc \bpi(\Prior) = \bpi (\belief^{1}) = \dots = \bpi (\belief^{K})$;
  (ii) at posterior $\mu^k \in \seq{\mu^1}{\mu^K}$,
  Expert proposes the price list $p^*$ with
  $p^*_i = \bp_{\varphi(k)} (\mu^k)$ if $i = \varphi(k)$
  and otherwise $p_i^* > l_N$;
  (iii) Client purchases treatment $a_{\varphi(k)}$ at  posterior $\mu^k$ for $\mu^k \in \seq{\mu^1}{\mu^K}$.
\end{enumerate}
\label{prp:new-client-worst}  
\end{prp}

Note that in the Client-worst equilibria described in
\Cref{prp:new-client-worst}, for each posterior~$\mu^k$ Expert 
essentially restricts Client's choice set by
setting prohibitively high prices for all treatments except
$a_{\varphi(k)}$.
This is equivalent to the non-full-choice-right feature in previous works, in which Client either accepts or rejects Expert's recommended treatment.
Based on this observation, we proceed to show that our geometric characterization can be applied to previous works.

\subsection{Fong, Liu, and Wright (2014)}

Consider the generalized game of
\cite{fong2014} with $N$ types and $N$ available treatments, 
which is referred to as $\GF$ from now on.
Games $\GA$ and $\GF$ differ in (i) whether Expert can charge discriminatory prices or not and (ii) whether Expert sends a cheap talk message or makes a treatment recommendation.

Timing of $\GF$ is as follows:
\begin{enumerate*}[(1)]
\item
  Expert posts a price list $p$;
\item 
  Expert privately observes the problem type $t \in T$;
\item
  Expert recommends some treatment $a_i \in A$, where 
  Expert refuses to offer a treatment for Client by recommending $a_i = \an$;
\item 
  Client chooses between the recommended treatment $a_i$ and $\an$.
\item 
  Players' payoffs are realized.
\end{enumerate*} 
Following Expert posting some price list $p \in \R^N_+$,
a \textit{$p$-equilibrium of $\GF$} consists of:
(i) Expert's recommendation strategy
$\gamma^{\FLW}: T \to \Delta(A)$;
(ii) Client's belief updating
$\beta^{\FLW}: A \to \Delta(T)$;
(iii) Client's choice strategy $\rho^{\FLW}: A  \to \Delta(A)$; such that
\begin{enumerate}[(a)]
\item 
  Given Expert's recommendation strategy $\gamma^\FLW$, Client's posterior belief after receiving recommendation $a \in A$ is
  obtained from the prior $\Prior$ using Bayes's rule:
  $$
  \beta^\FLW(t_i \mid a) = \frac
    {\Prior(t_i) \gamma^\FLW (a \mid t_i) }
    {\sum_{k=1}^N \Prior(t_k) \gamma^\FLW (a \mid t_k)}
  $$
  for $t_i \in T$;   
\item
  Given the belief updating $\beta^\FLW$, Client's strategy
  $\rho^\FLW (\cdot \mid a_i)$ after receiving recommendation $a_i$ is supported on 
  $$
  \arg\max_{a \in \set{a_i, \an}} \sum_{t \in T} u^C(a, t , p) \beta^\FLW(t \mid a_i)
  $$
  for $a_i \in A$;
\item 
  Given Client's strategy $\rho^\FLW$, Expert's recommendation strategy
  $\gamma^\FLW (\cdot \mid t)$ is supported on 
  $$
  \arg\max_{a_i \in A} \sum_{a \in \set{\an, a_i}} \rho^\FLW (a \mid a_i)  u^E (a, p)
  $$
  for $t \in T$. 
\end{enumerate}
We focus on the Expert-optimal perfect bayesian equilibria of $\GF$.
It follows that
Expert's equilibrium payoffs in $\GF$ are his highest $p$-equilibrium payoffs among all $p \in \R^N_+$.

We argue that Expert's equilibrium payoffs 
in $\GF$ are also $\qc \bpi (\mu^0)$.
On the one hand, any $p$-equilibrium of $\GF$
corresponds to an equilibrium of $\GA$ in which
Expert's ex ante payoffs are unchanged.
Specifically, whenever Expert recommends $a_i$ in $\GF$,
we can replace it with Expert
sending message $m_i$ and proposing $\tilde{p}'$ in $\GA$,
with $\tilde{p}'_i = p'_i$ and
$\tilde{p}'_j > l_N + 1$ for all $j \ne i$.
This implies that Expert's equilibrium payoffs
in $\GF$ must be weakly lower than Expert's maximal equilibrium payoffs in $\GA$.
On the other hand,
any Client-worst equilibrium of $\GA$ described in
\Cref{prp:new-client-worst}(b)
also corresponds to some $\tilde{p}^*$-equilibrium of $\GF$
in which Expert's ex ante payoffs remain unchanged.
Specifically, whenever Expert sends message
$m_{\varphi(k)}$ and proposes $p^*$ with 
$p^*_{\varphi(k)} = \bp_{\varphi(k)} (\mu^k)$
in some Client-worst equilibrium of $\GA$,
we can replace it with Expert recommending treatment $a_{\varphi(k)}$  in $\GA$,
after posting the uniform price list $\tilde{p}^*$
with $\tilde{p}^*_i = \bp_{\varphi(k)} (\mu^k)$ 
if $i = \varphi(k)$ for some $k \in \seq{1}{K}$
or otherwise $\tilde{p}^*_i = l_N + 1$.

\begin{prp}
In game $\GF$, Expert's equilibrium payoffs 
are $\qc \bpi (\mu^0)$.
\label{prp:fong}
\end{prp}

We note that \Cref{prp:fong} cannot be derived directly from a comparison between $\GF$ and our main model.
While the Client-worst equilibria of our main model described in \Cref{cor:client-worst} can also be reinterpreted
as the $p$-equilibria of $\GF$, we can only conclude that 
Expert's equilibrium payoffs in $\GF$ are weakly higher than
$\qc \bpi (\Prior)$.
The reason is that, in contrast to our main model,
Expert in $\GF$ has the advantage of restricting Client's choice set by making a treatment recommendation.
Therefore, without the aid of game $\GA$,
it remains uncertain whether Expert's equilibrium payoffs in $\GF$ can be strictly higher than $\qc \bpi (\Prior)$.

\section{Binary case}\label{sec:binary}

In this section, we focus on the binary case, where there are two problem types and two available treatments, and explicitly solve for all equilibria.
We assume that problems of type~$t_2$ are more cost-effective to resolve (i.e., $l_2 - c_2 > l_1 - c_1$).
The equilibria for the opposite scenario, where resolving problems of type~$t_1$ is more cost-effective, can be derived in a similar manner.

To begin, we calculate Expert's equilibrium payoffs in the binary case.
Let $\Prior = (1-q, q)$, where
$q \equiv \mu^0 (t_2)$ is the probability weight put on $t_2$ under the prior belief. In \Cref{fig:binary},
we plot the belief-based profits function $\bpi(\Prior)$, represented by the dashed lines, and its quasiconcave envelope $\qc \bpi(\Prior)$, represented by the solid lines.
Specifically, there are two possible cases of $\bpi (\Prior)$.
If $l_2 \ge \frac{c_2 - c_1}{l_1 - c_1} l_1$,
Client will always purchase one of the two treatments and $\bpi(\Prior) > 0$ for all $\mu$.
Otherwise, when $q$ is intermediate,
Client will not purchase either treatment.
Both cases result in the same quasiconcave envelope: 
$\qc \bpi (\Prior)$ is flat for $0 < q < \hat{q} \equiv \frac{c_2 - c_1}{l_2 - l_1}$ and then increases linearly to $l_2 - c_2$ as $q$ approaches $1$.

\begin{figure}[H]
     \centering
     \begin{subfigure}[b]{0.48\textwidth}
         \centering
         \includegraphics[width=\textwidth]{xiong-tu/2d-quasi-1.pdf}
        \subcaption{Case I: $\ls \geq \frac{\cs - \cm}{\lm - \cm} \lm$}
		\label{fig:no-0}
     \end{subfigure}
     \hfill
     \begin{subfigure}[b]{0.48\textwidth}
         \centering
         \includegraphics[width=\textwidth]{xiong-tu/2d-quasi-2.pdf}
         \subcaption{Case II: $\ls < \frac{\cs - \cm}{\lm - \cm} \lm$}
	\label{fig:with-0}
     \end{subfigure}
    \caption{Possible cases of $\bpi (q)$ and $\qc \bpi(q)$, represented by the dashed lines and the solid lines respectively.}
\label{fig:binary}
\end{figure}

We say communication \textit{benefits} Expert 
if $\qc \bpi (\Prior) > \bpi (\Prior)$. 
\Cref{fig:binary} implies that in the binary case, 
communication benefits Expert
if and only if $q \in (0, \hq)$.
It follows that when $\mu^0(t_2) \ge \hq$,
there exists a \textit{silent} Client-worst equilibrium in which
(i) Expert charges the monopoly price for treatment $a_2$ and reveals no information to Client, and (ii) Client chooses $a_2$.
Indeed, this silent equilibrium is unique when $q > \hq$.

\begin{prp}
When $q \in  (\hq , 1]$, there exists a unique equilibrium in which (a) Expert sets the prices
$p_2^* = q l_2 + (1-q) l_1$ and $p_1^* > l_1$ and
does not reveal any information to Client,
and (b) Client chooses $a_2$.
\end{prp}

\begin{proof}
It suffices to show that there exists no outcome with communication that can achieve the payoff $\bpi (\Prior) = q (l_2 - l_1) + (l_1 - c_2)$ for Expert.
With communication, the belief splitting must contain at least one belief $\mu'$ with $\mu'(t_2) < q$.
However, the highest payoff that Expert can achieve at $\mu'$ 
is strictly lower than $\bpi (\Prior)$.
Therefore, Expert's ex ante payoffs in any outcome with communication
must be strictly less than $\bpi (\Prior)$.
\end{proof}

Now we solve for the equilibria when $q$ is in the interval $(0, \hq)$.
As shown in \Cref{fig:binary},
Expert's equilibrium payoffs will always be $l_1 - c_1$ for any $q \in (0, \hq)$.
Let $\set{\mu^1, \dots, \mu^{K^*}}$
be the splitting of $\Prior$ in some equilibrium outcome 
with $\mu^1(t_2) < \dots < \mu^{K^*}(t_2)$.
Since $\qc \bpi(\mu^0) > \bpi(\mu^0)$,
all equilibria must be non-silent and 
$K^* \ge 2$.
Since $\mu^1(t_2) < \hq$,
Expert must set $\mu^1(t_2) = 0$ and $p_1 = l_1$
to obtain the payoff of $l_1 - c_1$ at posterior $\mu^1$. 
Additionally, there does not exist any other posterior whose probability weight on $t_2$ is strictly fewer than $\hq$: 
$\mu^k(t_2) \ge  \hq$ for all $k \ge 2$.
Otherwise, if there exists some $k$ such that $\mu^k(t_2) \in (0, \hq)$, then
Expert's payoffs at posterior $\mu^k$ must be 
strictly less than $l_1 - c_1$. 

\begin{lmm} \label{lmm:binary-splitting}
Suppose $q \in (0, \hq)$ and let $\set{\mu^1, \dots, \mu^{K^*}}$
be the belief splitting in some equilibrium outcome
with $\mu^1(t_2) < \dots < \mu^{K^*}(t_2)$.
Then $K^* \ge 2$ and
in all equilibria, Expert sets $p_1^* = l_1$, $\mu^1(t_2) = 0$,
and $\mu^k(t_2) \ge \hq$ for all $k \in \seq{2}{K^*}$.
\end{lmm}

We characterize all equilibria
based on \Cref{lmm:binary-splitting}.
In contrast to the case when $q > \hq$,
multiple equilibria always exist when $q \in (0, \hq)$.
Furthermore, any belief splitting
$\seq{\mu^1}{\mu^{K^*}}$ can appear in some equilibrium outcome as long as $\mu^1(t_2) = 0$
and $\mu^k(t_2) \ge \hq$ for all $k \in \seq{2}{K^*}$.

\begin{prp}\label{prp:all-eq}
Suppose $q \in (0, \hq)$.
Fix a splitting of $\Prior$,
$\seq{\mu^1}{\mu^{K^*}}$, that satisfies $K^* \ge 2$,
$\mu^1(t_2) = 0$ and $\mu^k(t_2) \ge \hq$ for $k \in \seq{2}{K^*}$.
Then $\seq{\mu^1}{\mu^{K^*}}$ appears in some equilibrium outcome.
Specifically,
\begin{enumerate}[(a)]
\item 
  When ${K^*} > 2$, there exists a unique equilibrium with
  the splitting being $\seq{\mu^1}{\mu^{K^*}}$. In that equilibrium,
  Expert sets $p_1^* = l_1$ and $p_2^* = c_2 + l_1 - c_1$,
  and Client chooses 
  $a_1$ at posterior $\mu^1$ and chooses $a_2$ at posterior $\mu^k$ for $k \in \seq{2}{K^*}$.
\item 
  When $K^* =2$ and $\mu^2(t_2) > \hq$, there exist two equilibria with
  the splitting being $\set{\mu^1, \mu^2}$. One is a Client-worst equilibrium, in which Expert sets
  $p_1^* = l_1$ and $p_2^* = l_1 + \mu^2(t_2) (l_2 - l_1)$,
  and Client chooses $a_1$ at posterior $\mu^1$ and, at posterior~$\mu^2$, Client chooses $a_2$ with probability 
  $\frac{l_1 - c_1}{l_1 + \mu^2(t_2) (l_2 - l_1) -c_2}$
  and chooses $\an$ with the complementary probability.
  The other is a non-Client-worst equilibrium, in which Expert sets $p^*_1 = l_1$ and $p^*_2 = c_2 + l_1 - c_1$, and Client chooses $a_1$ at posterior $\mu^1$ and chooses $a_2$ at posterior $\mu^2$.
\item
  When $K^* =2$ and $\mu^2(t_2) = \hq$, there exists a unique equilibrium with
  the splitting being $\set{\mu^1, \mu^2}$.
  In that equilibrium,
  Expert sets $p_1^* = l_1$ and $p_2^* = c_2 + l_1 - c_1$,
  and Client chooses 
  $a_1$ at posterior $\mu^1$ and chooses $a_2$ at posterior $\mu^2$.
\end{enumerate}
\end{prp}

\begin{proof}
  See \Cref{app:prp-nibary}.
\end{proof}



When the belief splitting contains more than two posteriors,
\Cref{prp:all-eq}(a) implies that there is a unique corresponding equilibrium. In that equilibrium,
Expert charges \textit{equal margins} for both treatments 
and Client adopts a pure strategy, choosing $a_1$ at posterior $\mu^1$ or otherwise choosing $a_2$.
To demonstrate this, consider Client's optimal response at posterior $\mu^k$ for some $k \in \seq{2}{K}$. 
First, Client will never choose $a_1$ at $\mu^k$ since 
$\mu^k(t_1) l_1 < l_1 = p_1^*$. 
Client will adopt a pure strategy of choosing $\an$ when $p_2 > \mu^k(t_1) l_1 + \mu^k(t_2) l_2$ or of choosing  $a_2$ when $p_2 <  \mu^k(t_1) l_1 + \mu^k(t_2) l_2$. 
Furthermore, Client may mix between $\an$ and $\as$ when these two options yield the same payoff,
$p_2 = \mu^k(t_1) l_1 + \mu^k(t_2) l_2$.
It follows that once Client mixes between $\an$ and $\as$
at $\mu^k$, she must adopt a pure strategy of choosing 
$\an$ or $\as$ at any other posterior $\mu^{k'}$ with
$k' \ge 2$.
However, in that case, Expert will not be indifferent between sending message $m_{k}$ and $m_{k'}$.
Therefore, in any equilibrium Client chooses $\as$ at $\mu^k$ for $k \ge 2$, and the equal-margin condition must hold, $p_2 - c_2 = l_1 - c_1$.

When the belief splitting contains two posteriors and $\mu^2(t_2) > \hq$,
\Cref{prp:all-eq}(b) implies that there are precisely two corresponding equilibria. 
One equilibrium requires Client to mix between $\an$ and $\as$ at posterior $\mu^2$, while in the other equilibrium 
Expert set equal margins over both treatments
and Client chooses $\as$ at $\mu^2$.
The analysis mirrors that of the case when $K^* > 2$, 
and the existence of the additional equilibrium where Client 
adopts a mixed strategy arises from the fact that 
only one posterior exists such that $\mu^k(t_2) \ge \hq$.
When $\mu^2(t_2) = \hq$,
$\frac{l_1 - c_1}{l_1 + \mu^2(t_2) (l_2 - l_1) -c_2} = 1$
and the two equilibria coincide as in \Cref{prp:all-eq}(c).

When discussing equilibria featuring equal margins, previous works in the credence goods literature mostly focus on the \textit{honest equilibria} \citep[e.g.,][]{fong2014,dulleck2006survey}
in which Expert always discloses the true problem type 
(i.e., $K^* = 2$ and $\mu^2 = (0,1)$).
However, we also identify many other equilibria
featuring equal margins,
in which Expert does not always disclose the true problem type.
Specifically,
the identified equilibria with $K^* = 2$ can all be ``translated'' into the equilibria of $\GF$ as below.
Expert employs a \textit{partial-overtreatment} strategy:
when faced with type $t_2$, Expert always recommends treatment $a_2$;
when faced with type $t_1$, Expert recommends treatment $a_2$ with probability $\frac{l_1 - c_1}{l_1 + \mu^2(t_2) (l_2 - l_1) -c_2}$
and recommends $a_2$ with the complementary probability.
Client's posterior is $\mu^1$ ($\mu^2$) given the recommendation $a_1$ ($a_2$), and
always accepts Expert's recommendation.


\begin{comment}

\textbf{Old, to be merged...}

\paragraph{Client-worst equilibria}
In a Client-worst equilibrium, Client is indifferent between $\an$
and purchasing her most desired treatment at any posterior.
For any $q \in (0, \hq)$,
there exist a family of Client-worst equilibria, indexed by $q' \in [\hq, 1]$, in which the belief splitting contains two posteriors.
In these equilibria, Expert splits Client's prior to
$\mu^1 = (1 , 0)$ and $\mu^2 = (1 - q', q')$ and
sets $p^*_2 = l_1 + q' (l_2 - l_1)$;
Client always chooses $a_1$ given the posterior $\mu^1$ and, given the posterior $\mu^2$, mixes between $a_0$ and $a_2$ with certain probabilities%
    \footnote{Specifically, Client chooses $a_2$ with probability  $\frac{l_1 - c_1}{l_1 + q' (l_2 - l_1) -c_2}$   
    and chooses $\an$ with the complementary probability.}
such that Expert's expected payoffs remain $l_1 - c_1$.

There also exist Client-worst equilibria in which the belief splitting contains more than two posteriors.
Denote by $\seq{\mu^1}{\mu^K}$ the belief splitting in equilibrium with
$\mu^1(t_2) < \dots \mu^K(t_2)$ and $K \ge 3$.
To achieve the payoff of $\lm - \cm$ at posteriors $\mu^i$ for $i \in \seq{2}{K}$, Expert must set $\ps = l_1 - c_1 + c_2$.
Additionally, in equilibrium, Client adopts a pure strategy, choosing $a_1$ at the posterior $\mu^1$ and choosing $a_2$ at all other posteriors.

By \Cref{lmm:binary-splitting}, 
Expert still chooses $ \mu^2(t_2), \dots, \mu^N (t_2) \in [\hq, 1]$ and set $\ps$.
To achieve a payoff of $\lm - \cm$ at $\mu^i$ for $i = 2,...,N$, Expert must set $\ps = l_1 - c_1 + c_2$.
Moreover, in equilibrium, Client adopts a pure strategy, choosing $a_1$ and $a_2$ at belief $\mu^1$ and $\mu^i$ for $i = 2,...,N$ respectively.

\paragraph{Non-Client-worst equilibria}
For any $q \in (0, \hq)$,
there exist a family of equilibria indexed by $q' \in (\hq, 1]$ in which Client benefits from Expert's services.
In these equilibria,
Expert sets $p^* = (l_1, l_1 - c_1 + c_2)$ and
splits Client's prior into $\set{\mu^1, \mu^2}$, where $\mu^1 = (1 , 0)$ and $\mu^2 = (1 - q', q')$;
Client chooses $a_1$ ($a_2$) at the posterior $\mu^1$ ($\mu^2$).
In these equilibria,
Expert always charges \textit{equal margins} for both treatments. 

When discussing equilibria featuring equal margins,
previous works in the credence goods literature mostly focus on the \textit{honest equilibrium} \citep[e.g.,][]{fong2014,dulleck2006survey}
in which Expert always discloses the true problem type.
However, we also identify many other equal-margins
equilibria in which Expert does not always disclose the true problem type.
Indeed, these two types of equilibria are the only equilibria with two posterior beliefs.
By \Cref{lmm:binary-splitting}, 
the freedom is to choose $q' \in [\hq, 1]$ and set $\ps$.
For a fixed value of $q' \in [\hq, 1]$, Expert's pricing strategy and Client's action strategy outlined in the Client-worst equilibria and non-Client-worst equilibria are the only methods by which Expert can achieve a payoff of $\lm - \cm$ at $\mu^2$.

\end{comment}

\subsection{Value of Expert's services}

All equilibria yield the same payoff for Expert given a fixed prior.
However, they may have different implications for welfare since Client's equilibrium payoffs can vary.
We use the value of Expert's services for Client as proxy for
Client's welfare.
Since the potential social surplus is $S (\Prior) \equiv (1 - q) (l_1 - c_1) +  q (l_2 - c_2)$ where $q = \Prior(t_2)$, 
the value of Expert's services 
cannot exceed $S(\Prior) - \qc \bpi  (\Prior)$, where
$\qc \bpi  (\Prior)$ is Expert's equilibrium payoff.
From now on, we focus on Client's possible welfare in equilibria. We demonstrate that the value of Expert's services can take any value in
the interval
$\big[0, S (\Prior) - \qc \bpi(\Prior) \big]$ %\equiv \big[ 0, q [(l_2 - l_1) - (c_2 - c_1)] \big]
when $q \in (0, \hq)$.%
    \footnote{When $q > \hq$, there exists a unique equilibrium in which
    the value of Expert's services is zero.}

Since a Client-worst equilibrium always exists
(\Cref{cor:client-worst}), 
the lower bound for the value of Expert's services
is attainable for any $q \in (0, \hq)$.
On the other hand,
consider the non-Client-worst equilibria
when the splitting contains two posteriors, $\set{\mu^1, \mu^2}$, as described in \Cref{prp:all-eq}(b). 
Among these equilibria,
% Client's welfare is increasing in $\mu^2(t_2)$.
% Specifically, fix some $q'  \in (\frac{c_2 - c_1}{l_2 - l_1}, 1]$,
the value of Expert's services is given by%
    \footnote{To obtain this payoff, note that at the posterior  $\mu^1 = (0,1)$, Client obtains no payoff; and at the posterior $\mu^2$, Client obtains a payoff of $(1 - \mu^2(t_2)) l_1 + \mu^2(t_2) l_2 - (l_1 - c_1 + c_2)$.}
$$
q (l_2 - l_1) - \frac{q}{\mu^2(t_2)} (c_2 - c_1).
$$
As $\mu^2(t_2)$ goes from $\hq$ to $1$, 
the value of Expert's services changes monotonically from $0$ to
$q [(l_2 - l_1) - (c_2 - c_1)] = S (\Prior) - \qc \bpi(\Prior)$.

\begin{prp}
	When $q \in (0, \hq)$, the possible values of Expert's services among all equilibria
	constitute the interval $[0, S (\Prior) - \qc \bpi(\Prior)]$.
\end{prp}

%\todo{xiong: should we discuss other equilibrium properties?
%(1) overtreatment, undertreatment, honesty 
%(2) sources of welfare loss (i) some problems are left untreated and (ii) some $t_1$ problem are treated using treatment $a_2$.}

\section{When does communication benefit Expert?}\label{sec:value}

Recall that when $N=2$, Expert benefits from communication if and only if $\mu^0(t_2) \in (0, \hq)$.
In this section, we generalize this observation and
provide a necessary and sufficient condition for when 
communication benefits Expert.
Denote by $s_{(k)}$ the $k$-th highest element of the set $\seq{s_1}{s_N}$,
where $s_i = l_i - c_i$ is the social surplus of resolving a problem of type $t_i$.
As demonstrated in \Cref{thm:smart-split},
the desired condition concerns the second-highest surplus, $s_{(2)}$.

\begin{thm}
\label{thm:smart-split}
For all $\Prior$ in the interior of $\Delta(T)$,
Expert benefits from communication (i.e., $\qc \bpi(\Prior) > \bpi(\Prior)$)
if and only if
\begin{equation}
\bpi (\Prior) < s_{(2)}.
\label{eq:necessary-sufficient}
\end{equation}
\end{thm}

\begin{proof}
  See \Cref{app:thm-splitting}.
\end{proof}

Note that the prerequisite in \Cref{thm:smart-split}
of the prior being interior is not a restriction.
Whenever $\Prior$ is not in the interior of $\Delta(T)$, the $N$ by $N$ credence goods setup
can be reduced to an $N'$ by $N'$ setup with a smaller type space $T' \subset T$, where the prior will be in the interior of $\Delta(T')$.
We can compute the corresponding second-highest potential surplus in the reduced setup and then apply \Cref{thm:smart-split}.


We illustrate \Cref{thm:smart-split} when $N=3$ and $s_1 < s_2 < s_3$
in \Cref{fig:3d}, where
\Cref{fig:3d}(a) and \Cref{fig:3d}(b) illustrate $\bpi(\mu)$ and $\qc\bpi(\mu)$ respectively.
In \Cref{fig:3d}(b),
the area filled with line patterns, together with the bold lines,
indicates the intersection of the graphs of $\bpi(\mu)$ and $\qc\bpi(\mu)$.
According to \Cref{thm:smart-split}, communication does not benefit Expert for those interior priors sufficiently close to the extreme belief $e^3$ such that
$\bpi(\Prior) \ge s_2$.
The area with line patterns in \Cref{fig:3d}(b) confirms this prediction.

\begin{figure}
     \centering
     \begin{subfigure}[b]{0.48\textwidth}
         \centering
         \includegraphics[width=\textwidth]{xiong-tu/3d-21.pdf}
        \subcaption{$\bpi (\Prior)$}
     \end{subfigure}
     \hfill
     \begin{subfigure}[b]{0.48\textwidth}
         \centering
         \includegraphics[width=\textwidth]{xiong-tu/3d-22.pdf}
         \subcaption{$\qc \bpi(\Prior)$}
     \end{subfigure}
        \caption{Surfaces of $\bpi (\Prior)$ and $\qc \bpi(\Prior)$ when $N=3$ and $s_2 < s_3$.}
\label{fig:3d}
\end{figure}

A direct corollary of \Cref{thm:smart-split} is that when 
$s_{(1)} = s_{(2)}$, Expert benefits from communication for \textit{all interior} priors.
\Cref{fig:3d2} illustrates this situation when $N=3$ and
$l_1 - c_1 < l_2 - c_2 = l_3 - c_3$,
where
\Cref{fig:3d2}(a) and \Cref{fig:3d2}(b) illustrate $\bpi(\mu)$ and $\qc\bpi(\mu)$ respectively.
In this case, the equality of $\bpi(\mu)$ and $\qc\bpi(\mu)$
only occurs at the boundary of $\Delta(T)$, denoted by
the bold lines in \Cref{fig:3d2}(b).


\begin{figure}
     \centering
     \begin{subfigure}[b]{0.48\textwidth}
         \centering
         \includegraphics[width=\textwidth]{xiong-tu/3d-11.pdf}
        \subcaption{$\bpi (\Prior)$}
     \end{subfigure}
     \hfill
     \begin{subfigure}[b]{0.48\textwidth}
         \centering
         \includegraphics[width=\textwidth]{xiong-tu/3d-12.pdf}
         \subcaption{$\qc \bpi (\Prior)$}
    %\label{fig:3d-22}
     \end{subfigure}
        \caption{Surfaces of $\bpi (\Prior)$ and $\qc \bpi(\Prior)$ when $N=3$ and $s_2 = s_3$.}
\label{fig:3d2}
\end{figure}




\section{Concluding remarks} \label{sec:conclude}

This paper is motivated by the observation that in many credence goods markets, the clients possess the full choice right. 
With this motivation, we study
a credence goods model in which the expert sends a cheap-talk message to the client, and then the client self-selects her most desired treatment.
We adopt the belief-based approach and provide a geometric characterization of the expert's equilibrium payoffs.
Based on that characterization,
we explicitly solve for all the equilibria in the binary case.
One implication of our study is that, by considering a general setting, the existing methods and insights in information design and cheap talk can be used to understand and solve the credence goods problems.

As the first paper on credence goods that allows
strategic information transmission in the client-expert interaction, 
this paper has identified a number of areas for future research.
For instance, while 
we have proved the existence of Client-worst equilibrium, we only characterize all the equilibria in the binary case. It would be beneficial for future research to further explore the possible equilibrium outcomes in a different setting. 
On the other hand, it would also be intriguing to study an
alternative communication protocol apart from cheap talk 
in the credence goods setting.
For example, one may consider some
communication protocol between cheap talk and bayesian persuasion.
Such a framework has the potential to provide insights to 
many stylized facts in credence-goods markets.

\newpage
\setcitestyle{numbers} % numeric list

\bibliographystyle{jpe}
\bibliography{cg.bib}
\newpage
%%%%%%%%%%%%% References
\appendix
\renewcommand{\thesection}{\Alph{section}}
\renewcommand{\thesubsection}{\Alph{section}\arabic{subsection}}


\section{Appendix: Proofs} \label{app:proof} 

\subsection{Proof of \Cref{thm:main}} 
\label{app:a2}

Throughout the proof, we use
$\max_p$ as an abbreviation for $\max_{p \in \R_+^N}$
and let $\mu_i \equiv \mu(t_i)$ for all $i \in \seq{1}{N}$.

\paragraph{Step 1} $\max_{p} \qc \bv (\belief ; p) \le
\qc \max_{p} \bv (\belief ; p)$ for all $\belief \in \Delta(T)$.

For all ${p \in \R^N_+ }$ and $\belief \in \Delta(T)$, 
it follows from
$\bv (\mu; p) \le \max_p  \bv (\mu ; p)$
that
$\qc \bv (\mu; p) \le \qc \max_p  \bv (\mu; p)$.
Therefore, $\max_p \qc \pi (\belief; p) \le \qc \max_p  \pi (\belief; p)$.

\paragraph{Step 2} $\max_p \qc \bv (\mu ; p) \ge \qc \max_p  \bv (\mu ; p)$
for all $\belief \in \Delta(T)$. 

To introduce some notations,
let $p_i^\m(\mu) \equiv \sum_{k=1}^i \mu_k l_k$ be
the ``monopoly price'' for treatment $a_i$ given belief $\mu$.
When the price for treatment $a_i$ is $\bp_i (\mu)$, Client is indifferent between $a_i$ and $a_0$ given $\mu$.
Let ${\bpi}(\mu) \equiv \max_p  \bv (\mu; p)$
be Expert's monopoly profits given Client's belief $\mu$.
It suffices to show that for any $\mu' \in \Delta(T)$,
there exists some price list $p' \in \mathbb{R}^N_+$ such that 
\begin{equation}
\qc \bv (\mu' ; p') \ge \qc \bpi (\mu'). 
\label{eq:step2}   
\end{equation}
%
%
Since Client has unit demand,
\begin{equation}\label{eq:unit-demand-proof}
\bpi (\belief) = \max \{0, \bp_1(\mu) - c_1, \dots, \bp_N(\mu) - c_N \}.  
\end{equation}
%
%
Since the maximum of $N+1$ convex (continuous) functions is also convex (continuous),
\Cref{eq:unit-demand-proof} implies that
$\bpi(\cdot)$ is a convex and continuous function over $\Delta(T)$.
Let $v' \equiv \qc \bpi (\mu')$.
When $\bpi(\mu') = v'$, inequality~\eqref{eq:step2} holds
by setting $p' \in \arg\max_p  \pi (\mu', p)$.
Below we focus on the case when $v' > \bpi (\mu')$.

%
%We call $(\mu^1, \dots, \mu^K)$ a \textit{belief--splitting} of $\mu$ if 
%$(\mu^1, \dots, \mu^K)$ is a convex combination of
%$\mu$; that is,
%there exist $K$ numbers,
%$\{\lambda_i\}_{i=1}^K$ satisfying 
%$\lambda_i \ge 0$ for all $i$ and $\sum_{i=1}^K \lambda_i =1$, 
%such that $\sum _{i=1}^K \lambda_i \mu^i = \mu$.
By Corollary 1 and Theorem 2 of \lippaper{},
\begin{equation}
\begin{split}
  \qc \bpi (\mu') = &\max_{\mu^1, \dots, \mu^K} \min \{ \bpi (\mu^1), \dots \bpi (\mu^K)\} \\
\text{subject to } \quad & \text{ ($\belief^{1}$, \dots, $\belief^{K}$) is a belief--splitting of $\mu'$}. 
\end{split}
\label{eq:qcav}  
\end{equation}



%\Cref{clm:bpi} implies that $\bpi(\cdot)$ is a convex and continuous function.\footnote{Here we use the fact that 
%    the maximum of $N+1$ convex (continuous) functions is also convex (continuous).} 
%Let $v' = \qc \bpi (\mu')$.
%When $\bpi(\mu') = v'$, inequality~\eqref{eq:step2} holds
%by setting $p' \in \arg\max_p  \pi (\mu', p)$.
%Below we focus on the case when $v' > \bpi (\mu')$.
%We call $(\mu^1, \dots, \mu^K)$ a \textit{belief--splitting} of
%$\mu$ if there exist $K$ numbers,
%$\{\lambda_i\}_{i=1}^K$ satisfying 
%$\lambda_i \ge 0$ for each $i$ and $\sum_i \lambda_i =1$ 
%such that $\sum _ i \lambda_i \mu^i = \mu$.

\begin{claim}
If $v' > \bpi (\mu')$, then
there exists a belief--splitting 
$(\belief^{1}$, \dots, $\belief^{K})$ of $\mu'$
such that 
$\bpi (\belief^{k}) = v'$ for all $\belief^{k} \in \{ \belief^{1}, \dots, \belief^{K}\}$.
\label{clm:bebu}
\end{claim}

\begin{proof}
Let $(\mu^1, \dots, \mu^K)$ be a maximizer to program~\eqref{eq:qcav} and then $\bpi (\mu^k) \ge v'$.
Suppose $\bpi (\mu^k) > v'$.
It suffices to show that there exists some belief $\tilde{\mu}^k$
satisfying (i) $\bpi (\tilde\mu^k) = v'$ and that (ii)
$(\mu^1, \dots, \mu^{k-1}, \tilde\mu^{k}, \mu^{k+1}, \dots, \mu^{K})$ is a belief splitting of $\mu'$.

Note that $\bpi (\mu^k) > v' > \bpi(\mu')$.
Since $\bpi(\cdot)$ is continuous, by intermediate value theorem
there exists $\tilde\mu^k = \lambda \mu' + (1-\lambda) \mu^k$ for some $\lambda \in (0,1)$ such that $\bpi(\tilde\mu^k) = v'$.
Suppose $\sum_{i=1}^K\lambda_i \mu^i = \mu'$.
That $(\mu^1, \dots, \mu^{k-1}, \tilde\mu^{k}, \mu^{k+1}, \dots, \mu^{K})$ is a belief splitting of $\mu'$ follows from
$\mu' = \sum_{i \ne k} \frac{\lambda_i (1-\lambda )}{ \lambda \lambda_k +1-\lambda} \mu^i  + \frac{\lambda_k}{\lambda \lambda_k +1 -\lambda} \tilde\mu^k$.
\end{proof}

Based on \Cref{clm:bebu}, we further show that
there exist a belief--splitting, $\seq{\mu^1}{\mu^K}$
and $K$ different numbers, $\seq{i_1}{i_K}$
such that
$\bpi (\belief^k) = \bp _{i_k} (\belief^{k}) - c_{i_k}
= v'$
for each $k \in \seq{1}{K}$.
We formally state this finding in \Cref{clm:intermediate} below.

\begin{claim}
If $v' > \bpi (\mu')$, then
there exists a belief splitting of $\mu'$, 
$\seq{\belief^{1}}{\belief^{K}}$ with $K \le N$, such that
$\bpi (\mu^k) = \bp_{i_k} (\mu^k) - c_{i_k}= v'$ for $K$ different $\{i_1, \dots, i_K\}$.
\label{clm:intermediate}
\end{claim}

\begin{proof}

By \Cref{clm:bebu}, there exists a belief--splitting
$(\mu^1, \dots, \mu^L)$ 
with $\sum_{i=1}^L\lambda_i \mu^i = \mu'$
satisfying
$\bpi(\mu^\ell) = v'$ for each $\ell \in \seq{1}{L}$.
By \Cref{eq:unit-demand-proof}, for each $\ell$ there exists some $i_\ell$
such that $\bpi (\mu^\ell) = \bp_{i_\ell} (\mu^\ell) - c_{i_\ell}$.
The mapping 
\begin{equation*}
\begin{split}
\eta : &\seq{1}{L} \to \seq{1}{N}    \\
        &\ell \mapsto i_\ell  
\end{split}  
\end{equation*}
induces a partition of 
the set $\seq{1}{L}$.
%
Denote by $K (\le N)$ the number of partitions.
For the $k$-th partition $\seq{k_1}{k_J}$,
define $\tilde{\mu}^{i_k} \equiv 
           \frac{\sum_{j=1}^J \lambda_{k_j} \mu ^{k_j}}
                {\sum_{j=1}^J \lambda_{k_j}}$.
It follows that $(\tilde{\mu}^{i_1}, \dots, \tilde{\mu}^{i_K})$ is also a belief--splitting of $\mu'$. 
Since $\bp_{i_k}(\mu) - c_{i_k}$ is an affine function of $\mu$, we have
$\bp_{i_k} (\tilde{\mu}^{i_k})  - c_{i_k}  = v'$.
\end{proof}

By \Cref{clm:intermediate}, there exist an
into function
$\varphi : \seq{1}{K} \to \seq{1}{N}$
and a belief--splitting
$(\belief^{1}, ..., \belief^{K})$ of 
$\belief'$
such that
\begin{equation} \label{eq:niu-xiong}
  \bpi(\mu^{k}) = \bp_{\varphi(k)} (\mu^{k}) - c_{\varphi(k)} = v',
  \text{ $ \forall  k \in \seq{1}{K}$.}
\end{equation}
We show that inequality~\eqref{eq:step2} holds by setting $p' = (p'_1, \dots, p'_N)$ with
$$
p'_i = \begin{cases}
  \bp _ {\varphi(k)} (\mu^{k}), 
  &\text{if } i \in \seq{\varphi(1)}{\varphi(K)};
  \\
  p_N + 1, &\text{otherwise}.
\end{cases}
$$
%
Let $U^C(a; \mu, p) \equiv \E_{\belief} u^C(a, t, p)$
be Client's expected payoff from $a \in A$
given $\belief$ and $p$.
It suffices to check that under $\mu^{k}$ and $p'$,
choosing $a_{{\varphi(k)}}$ is Client's optimal response
for $k \in \seq{1}{K}$.
If that were true, $\bv (\belief^{k} ; p') \ge u^E (a_{{\varphi(k)}} , p') = v'$ for each $k$ and the LHS of inequality~\eqref{eq:step2} would be weakly greater than $v'$.

We finish the proof of
\Cref{thm:main} by verifying \Cref{clm:final} below.
Note that we only need to check that Client prefers
$a_{\varphi(k)}$ to $a_{\varphi(\ell)}$ for $\varphi(\ell) \in \seq{\varphi(1)}{\varphi(K)}$.
The reason is that $p'_j$ is prohibitively high
for $j \not \in  \seq{\varphi(1)}{\varphi(K)}$.

\begin{claim}
Given the constructed price list $p'$, 
belief--splitting $(\belief^{1}, ..., \belief^{K})$ and into function $\varphi(\cdot)$,
\begin{equation}
  U(a_{\varphi(k)}; \mu^{k}, p')  \ge 
  U(a_{\varphi(\ell)}; \mu^{k}, p')
\label{eq:fake-ic}  
\end{equation}
for all $k \in \seq{1}{K}$ and all $\ell \in \seq{1}{K}$. 
\label{clm:final}
\end{claim}

\begin{proof}
Note that $U(a_{\varphi(k)}; \mu^{k}, p') 
= U(\an; \mu^{k}, p')  
= - \bp _{N} (\mu^{k})$ 
for all $k$.
On the other hand,
\begin{equation*}
  U(a_{\varphi(\ell)}; \mu^{k}, p') = 
  \begin{cases}
    - \sum_{h= \varphi(\ell) + 1}^N \mu^{k} (t_h) l_h - \bp_{\varphi(\ell)}(\mu^{\ell})  & \text{if $\varphi(\ell) < N$} \\
    - \bp_N (\mu ^{N}) & \text{if $\varphi(\ell) = N$ } 
  \end{cases} 
\end{equation*}
When $\varphi(\ell) \ne N$,
\begin{align}
  &\qquad\quad
  U(a_{\varphi(k)}; \mu^{k}, p')  \ge 
  U(a_{\varphi(\ell)}; \mu^{k}, p')
  \notag 
\\   
 &\iff 
 -\sum_{h=1}^N  \mu^{k}(t_{h}) l_h 
 \ge 
 - \sum_{h= \varphi(\ell) + 1}^N \mu^{k} (t_h) l_h - \sum_{h= 1}^{\varphi(\ell)} \mu^{\ell} (t_h) l_h 
  \notag
\\
  &\iff
  \sum_{h=1}^{\varphi(\ell)} \mu^{\ell} (t_h) l_h  \ge \sum_{h=1}^{\varphi(\ell)}
  \mu^{k} (t_h) l_h 
  \notag
\\  
  &\iff
  \bp_{\varphi(\ell)} (\mu^{\ell})  
  \ge 
  \bp_{\varphi(\ell)} (\mu^{k}) \label{eq:xiong}
\end{align}
%
By \Cref{eq:niu-xiong},  
$\bpi _{\varphi(\ell)} (\mu^{\ell}) = \bp_{\varphi(k)} (\mu^{k}) - c_{\varphi(k)} = v' $.
By \Cref{eq:unit-demand-proof},
$\bpi(\mu^{k}) \ge
\bp_{\varphi(\ell)} (\mu^{k}) - c_{\varphi(\ell)}$.
We have 
$\bp_{\varphi(\ell)} (\mu^{k}) - c_{\varphi(\ell)} 
\le 
\bpi_{\varphi(\ell)} (\mu^{\ell})
= \bp_{\varphi(\ell)} (\mu^{\ell})  - c _{\varphi(\ell)}$.
Therefore, inequality~\eqref{eq:xiong} holds.

When $\varphi(\ell) = N$,
it's easy to verify that inequality~\eqref{eq:xiong}
is still equivalent to \eqref{eq:fake-ic}, 
which follows from the same argument as above. 
\end{proof}


\subsection{Proof of \Cref{prp:all-eq}\label{app:prp-nibary}} 

  Let $\seq{\mu^1}{\mu^{K^*}}$ be the splitting in equilibrium. By \Cref{lmm:binary-splitting}, it holds that $K^* \ge 2$,
  $\mu^1(t_2) = 0$ and $\mu^k(t_2) \ge \hq$ for $k \in \seq{2}{K^*}$.
  We first argue that it cannot happen in equilibrium where $\mu^i = \mu^j$ and Client adopts different strategies at posteriors $\mu^i$ and $\mu^j$.
  If that is the case, then by \Cref{lmm:binary-splitting} Client must be indifferent between $\an$ and $\as$ at $\mu^i$. Furthermore, since Client adopts different strategies at posteriors $\mu^i$ and $\mu^j$,
  Expert cannot be indifferent between sending message $m_i$ and $m_j$. Contradiction.

  From now on, assume $\mu^i(t_2) < \mu^j(t_2)$ whenever $i<j$. For part (a), since $p_1^* = l_1$, Client will never choose $a_1$ at posterior $\mu^k$ for $k\ge 2$.
  It follows that Client either chooses $\as$ or mixes between $\an$ and $\as$ at $\mu^k$ for $k\ge 2$.
  Suppose Client mixes between $\an$ and $\as$ at some $\mu^{i}$ with $i \ge 2$. Since $K^* >2$, Client must adopt a pure strategy of choosing $\an$ or of choosing $\as$ at another $\mu^j$ with $j \ge 2$. In either case, Expert will not be indifferent between sending $m_i$ and $m_j$. Therefore, Client adopts the pure strategy of choosing $\as$ at $\mu^k$ for all $k \ge 2$.
  Given Client's equilibrium strategy, Expert must set $p^*_2 = c_2 + l_1 - c_1$ so that he is indifferent between sending $m_1$ and $m_2$.
  
For part (b), suppose $K^* = 2$ and $\mu^2(t_2) > \hq$.
It follows that Client either chooses $\as$ or mixes between $\an$ and $\as$ at $\mu^2$.
When Client chooses $\as$  at $\mu^2$, it follows from the
same argument as in part (a) that
Expert must set $p^*_2 = c_2 + l_1 - c_1$, which leads to the non-Client-worst equilibrium.
When Client mixes between $\an$ and $\as$  at $\mu^2$,
she must be indifferent choosing 
$\an$ and $\as$  at $\mu^2$ and thus
$p_2^* = l_1 + \mu^2(t_2) (l_2 - l_1)$.
Furthermore,
Client chooses $a_2$ with probability 
$\frac{l_1 - c_1}{l_1 + \mu^2(t_2) (l_2 - l_1) -c_2}$
so that Expert is indifferent between sending $m_1$ and $m_2$.

For part (c), suppose $K^* = 2$ and $\mu^2(t_2) > \hq$.
Similar arguments in part (b) apply. However, the two equilibria identified in part (b) coincide as 
$\frac{l_1 - c_1}{l_1 + \mu^2(t_2) (l_2 - l_1) -c_2} = 1$.

\subsection{Proof of \Cref{thm:smart-split}}
\label{app:thm-splitting}

Denote by $\iota(i)$ the index of the problem type with potential surplus $s_{(i)}$. Then $s_{(i)} = s_{\iota(i)} = l_{\iota(i)} - c_{\iota(i)}$.

%The proof of the necessity of condition~\eqref{eq:necessary-sufficient} is as follows.
\paragraph{``Only if'' part}
Assume by contradiction that
$\qc \bpi (\Prior) > \bpi (\Prior)$ and $\bpi(\Prior) \ge s_{(2)}$.
By \Cref{cor:client-worst},
there exists a Client-worst equilibrium in which (i) the belief-splitting is
$\seq{\mu^1}{\mu^K}$ with $K \ge 2$ satisfying
$\bpi (\mu^k) > \bpi (\Prior)$ for $k \in \seq{1}{K}$ and (ii)
Client chooses treatments
$a_{\varphi(k)}$ under posterior $\mu^k$.
This implies that there exist at least two different posteriors,
say $\mu^1$ and $\mu^2$, 
such that $\bp_{\varphi(1)} (\mu^1) - c_{\varphi(1)} > \bpi(\Prior)$ and 
$\bp_{\varphi(2)} (\mu^2) - c_{\varphi(2)} > \bpi(\Prior)$.
Since $\qc \bpi (\Prior) > s_{(2)}$,
both $\bp_{\varphi(1)} (\mu^1) - c_{\varphi(1)}$ and
$\bp_{\varphi(2)} (\mu^2)- c_{\varphi(2)}$ are strictly higher than $s_{(2)}$. 
On the other hand, at least one of
$a_{\varphi(1)}$ and $a_{\varphi(2)}$ is distinct from
$a_{\iota(1)}$.
Suppose $a_{\varphi(1)} \ne a_{\iota(1)}$.
Then $\bp_{\varphi(1)} (\mu^1) - c_{\varphi(1)} \le l_{\varphi(1)} - c_{\varphi(1)} \le s_{(2)}$.
Contradiction.

\paragraph{``If'' part}
Note that Expert can always secure the payoffs of $s_{(N)}$
by setting the price list $p^*$
with $p^*_i = c_i + s_{(N)}$ and revealing all information to Client.
It follows that Expert benefits from communication whenever
$\bpi(\Prior) < s_{(N)}$.
From now on,
focus on the priors satisfying $\bpi(\Prior) \ge s_{(N)}$.
The proof relies on
\Cref{lmm:smart-splitting-2}.

\begin{lmm}
\label{lmm:smart-splitting-2}
Suppose the prior $\Prior$ is in the interior of $\Delta(T)$ and
let $v^0 \equiv \bpi(\Prior)$.
If there exists some $\hat{\mu} \in \Delta(T)$ such that (i) $\bpi(\hat{\mu}) = v^0$ and (ii) $(\Prior, \bpi(\Prior))$ and $(\hat{\mu}, \bpi(\hat{\mu}))$ are not on the same hyperplane $\pi_i(\mu)$ for all $i \in \seq{1}{N}$, then communication benefits Expert, i.e., $\qc \bpi(\Prior) > \bpi(\Prior)$.   
\end{lmm}

\begin{proof}[Proof of \Cref{lmm:smart-splitting-2}]
Since $\Prior$ is in the interior of $\Delta(T)$ and $\hat{\mu} \in \Delta(T)$,
there exists some $\tilde{\mu} \in \Delta(T)$
such that
$\lambda \hat{\mu} + (1-\lambda) \tilde{\mu} = \Prior$ for some
$\lambda \in (0,1)$.
Since $\bpi(\Prior) = v^0 = \bpi(\hat{\mu})$,
by convexity of $\bpi$ we have
$\bpi(\tilde{\mu}) \ge v^0$.
Moreover, $\bpi(\tilde{\mu}) > v^0$ because
$(\Prior, \bpi(\Prior))$ and $(\hat{\mu}, \bpi(\hat{\mu}))$ are not on the same hyperplane $\pi_i(\mu)$ 
for all $i \in \seq{1}{N}$.
By continuity of $\bpi$, there exists some $\varepsilon > 0$ such that $\bpi (\mu) > v^0$ for all $\mu$ satisfying $||\mu - \tilde{\mu}|| < \varepsilon$.
On the other hand, for arbitrarily small $\varepsilon' > 0$,
there exists some $\hat{\mu}'$ such that $||\hat{\mu}' - \hat{\mu}|| < \varepsilon'$ and 
$\bpi(\hat{\mu}') >  \bpi(\hat{\mu}) = v^0$.
It follows that there exists a splitting
$\set{\hat{\mu}', \tilde{\mu}'}$
with 
$||\tilde{\mu}' - \tilde{\mu}|| < \varepsilon$ such that
$\bpi(\hat{\mu}') > v^0$ and
$\bpi(\tilde{\mu}') > v^0$.
Therefore, 
$\qc \bpi(\Prior) > \bpi(\Prior)$
by Theorem 1 (the Securability Theorem) of \lippaper{}. 
\end{proof}

Let $v^0 \equiv \bpi(\Prior) \in [s_{(N)},  s_{(2)})$.
Consider those beliefs $\hat{\mu}$ with $\hat{\mu}(t_{\iota(N)}) + \hat{\mu}(t_{\iota(1)}) = 1$
(i.e., $\hat{\mu}(t_i) = 0$ for all $i \not \in \{\iota(1), \iota(N)\}$).
Our analysis of the binary model in \Cref{sec:binary} 
guarantees that for any 
$v^0 \in [s_{(N)},  s_{(2)})$, there exists some
$\hat{\mu}$ with $\hat{\mu}(t_{\iota(N)}) + \hat{\mu}(t_{\iota(1)}) = 1$
such that $v^0 = \bpi(\hat{\mu})$.
Moreover, $(\hat{\mu}, \bpi(\hat{\mu}))$ is on the hyperplane $\pi_{\iota(1)}(\mu)$.
Similarly, there exists another belief~$\tilde{\mu}$
with $\hat{\mu}(t_{\iota(N)}) + \hat{\mu}(t_{\iota(2)}) = 1$
such that $v^0 = \bpi(\tilde{\mu})$,
and $(\tilde{\mu}, \bpi(\tilde{\mu}))$ is on the hyperplane $\pi_{\iota(2)}(\mu)$.

When point $(\hat{\mu}, \bpi(\hat{\mu}))$  (or $(\tilde{\mu}, \bpi(\tilde{\mu}))$) and point
$(\Prior, \bpi(\Prior)$
are not on the same hyperplane $\pi_i(\mu)$ for all $i \in \seq{1}{N}$,
then $\qc \bpi(\Prior) > \bpi(\Prior)$ by   
\Cref{lmm:smart-splitting-2}.
When 
$(\Prior, \bpi(\Prior)$ is at the intersection of
hyperplanes $\pi_{\iota(1)}(\mu)$ and  $\pi_{\iota(2)}(\mu)$,
consider \Cref{lmm:smart-splitting} that extends
\Cref{lmm:smart-splitting-2} to the case of three-point splittings.

\begin{lmm}
\label{lmm:smart-splitting}
Suppose the prior $\Prior$ is in the interior of $\Delta(T)$ and
let $v^0 \equiv \bpi(\Prior)$.
If there exists $\hat{\mu} \in \Delta(T)$ and $\tilde{\mu} \in \Delta(T)$ such that (i) $\bpi(\hat{\mu}) = \bpi(\tilde{\mu}) = v^0$ and (ii)  $(\hat{\mu}, \bpi(\hat{\mu}))$ and $(\tilde{\mu}, \bpi(\tilde{\mu}))$ are not on the same hyperplane $\pi_i(\mu)$ for all $i \in \seq{1}{N}$, then communication benefits Expert, i.e., $\qc \bpi(\Prior) > \bpi(\Prior)$.   
\end{lmm}

\begin{proof}[Proof of \Cref{lmm:smart-splitting}]

Since $\Prior$ is in the interior of $\Delta(T)$,
there exists some $\bar{\mu} \in \Delta(T)$
and $\lambda_i \in (0,1)$ for $i \in \set{1,2,3}$ with
$\sum_{i = 1}^3 \lambda_i = 1$ such that
$\lambda_1 \hat{\mu} + \lambda_2 \tilde{\mu} + \lambda_3 \bar{\mu} = \Prior$.
Since $\bpi(\Prior) = v^0 = \bpi(\hat{\mu}) = \bpi(\tilde{\mu}) $,
by convexity of $\bpi$ we have
$\bpi(\bar{\mu}) \ge v^0$.
Moreover, $\bpi(\bar{\mu}) > v^0$ because
$(\hat{\mu}, \bpi(\hat{\mu}))$ and $(\tilde{\mu}, \bpi(\tilde{\mu}))$ are not on the same hyperplane $\pi_i(\mu)$ 
for all $i \in \seq{1}{N}$.
By continuity of $\bpi$, there exists some $\varepsilon > 0$ such that $\bpi (\mu) > v^0$ for all $\mu$ satisfying $||\mu - \bar{\mu}|| < \varepsilon$.
On the other hand, for arbitrarily small $\varepsilon' > 0$,
there exists some $\hat{\mu}'$ such that $||\hat{\mu}' - \hat{\mu}|| < \varepsilon'$ and 
$\bpi(\hat{\mu}') >  \bpi(\hat{\mu}) = v^0$.
Similarly, for arbitrarily small $\varepsilon'' > 0$,
there exists some $\tilde{\mu}'$ such that $||\tilde{\mu}' - \tilde{\mu}|| < \varepsilon''$ and 
$\bpi(\tilde{\mu}') >  \bpi(\tilde{\mu}) = v^0$.
It follows that there exists a splitting of $\Prior$,
$\set{\hat{\mu}', \tilde{\mu}', \bar{\mu}'}$
with 
$||\bar{\mu}' - \bar{\mu}|| < \varepsilon$,
such that $\bpi(\hat{\mu}') > v^0$, 
$\bpi(\tilde{\mu}') > v^0$
and $\bpi(\bar{\mu}') > v^0$.
Therefore, 
$\qc \bpi(\Prior) > \bpi(\Prior)$
by Theorem 1 (the Securability Theorem) of \lippaper{}. 
\end{proof}

Since $\Prior$ is in the interior of $\Delta(T)$, $\hat{\mu} \in \Delta(T)$, $\tilde{\mu} \in \Delta(T)$, and 
(i) $\bpi(\hat{\mu}) = \bpi(\tilde{\mu}) = v^0$ and 
(ii)  $(\hat{\mu}, \bpi(\hat{\mu}))$ and $(\tilde{\mu}, \bpi(\tilde{\mu}))$ are on the hyperplanes $\pi_{\iota(1)}(\mu)$ and $\pi_{\iota(2)}(\mu)$ respectively,
the proof concludes by \Cref{lmm:smart-splitting}.
\end{document}