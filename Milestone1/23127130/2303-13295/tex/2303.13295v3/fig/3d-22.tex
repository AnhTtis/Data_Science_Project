\documentclass[tikz,margin=2mm]{standalone}
\usepackage{tikz,xcolor}
\usetikzlibrary{patterns, fit, decorations.pathreplacing, shadings}
\usepackage{tikz-3dplot}

\begin{document}
\tdplotsetmaincoords{70}{20}
\begin{tikzpicture}[
  domain=0.001:1, scale=1, yscale=0.7,
  tdplot_main_coords,
  ]
%%% parameters l3=18, l2=12, l1=6, c3=16, c2=10, c1=5

%%% edges
%\draw[dotted]
%(0,6,0) -- (0,6,6) ;

%% shades--quasiconcave
\fill[gray, opacity=0.4] 
(0,0,2) -- (5,0,2) -- (0, 14/3,2);
\shade [top color = gray!80, bottom color = gray!40] (6,0,4) -- (5,0,2) -- (0, 14/3,2)  -- (0,16/3,4);
\fill[gray, opacity=0.8] 
(6,0,4) -- (0,16/3,4) -- (1,5,4) -- (6,0,4) ;
\fill [pattern = north east lines,  pattern color = gray] (0,16/3,4) -- (0,6,6) -- (1,5,4);

%% pi^m
\draw[thick, gray, dotted] 
(0,0,2) node [left] {\color{black} $l(\omega_1) - c(a_1)$} -- (1,0,0) -- (0,1,0) -- (0,0,2);
\draw[thick, gray, dotted] 
(6,0,4) node [right] {\color{black} $l(\omega_2) - c(a_2)$} -- (4,0,0) -- (5,1,0);
\draw[thick, gray, dotted]
(5,1,0) -- (6,0,4);
\draw[thick, gray, dotted] 
(0,6,6) node [above right] {\color{black}  $l(\omega_3) - c(a_3)$} -- (0,4,0) -- (3,3,0);
\draw[thick, gray, dotted]
(3,3,0) -- (0,6,6);
\draw[thick, gray, dotted] 
(1,0,0) -- (4,0,0) ;
\draw[thick, gray, dotted] 
(0,4,0) -- (0,1,0);
\draw[thick, gray, dotted]
(5,1,0) -- (3,3,0);

%% quasiconcave envelope
%% flat 1
\draw[] 
(0,0,2) -- (5,0,2) ;
\draw[gray] 
(5,0,2) -- (0, 14/3,2);
\draw[] 
(0, 14/3,2) -- (0,0,2);

%% flat 2
\draw[gray] 
(6,0,4) -- (0,16/3,4);
\draw[gray] 
(0,16/3,4) -- (1,5,4) ;
\draw[] 
(1,5,4) -- (6,0,4) ;

%% 
\draw[
  % ultra thick,
  ] 
(6,0,4) -- (5,0,2)
(0,16/3,4) -- (0, 14/3,2);

\draw[
  % ,ultra thick
  ]
(0,16/3,4) -- (0,6,6) -- (1,5,4);


%% equation for the quasiconcave envelope 
%% z = 1.5 [ (2y + x -14) + sqrt ((14-2y-x)^2 - 8 (20-4y-5x))]

\draw[gray] 
  (0,0,0) -- (6,0,0);
% \draw[gray, dashed]   
%   (0,0,0) -- (0,15,0);
\draw[gray]   
  (0,0,0) -- (0,0,6);
  
\draw[->, gray] 
  (6,0,0) -- ++ (0.5,0,0) node[anchor=west]{\color{black}  $\mu_0(\omega_2)$};
\draw[->, dashed,gray]
  (0,0,0) -- ++ (0,14,0);
\draw[]
  (0,14.9,0) node[xshift=0.3cm]{  {$\mu_0(\omega_3)$}};


  \draw[->, gray] 
(0,0,6) -- ++(0,0,2) node[anchor=south]{\color{black}  qcav$\pi^*$};

%%% edges
%\draw[thick, dotted]
%(6,0,0) -- (0,6,0) ;
%\draw[thick, dotted]
%(6,0,0) -- (6,0,6) ;
%\draw[thick, dotted]
%(0,6,6) -- (0,0,6) ;
%\draw[thick, dotted]
%(0,0,6) -- (6,0,6) ;
%\draw[thick, dotted]
%(0,6,6) -- (6,0,6) ;

%%% 1 & o 
%\draw[] (6,0,0) node [below]{$1$}-- (6,0,0.1);
%\draw[] (0,6,0) node [above, xshift=-0.1cm]{$1$}-- (0,6,0.1);
%\draw[] (0,0,0) node [below left]{ $ o $} ;

%% 1 & o 
\draw[] (0,0,0) node [below left]{ $ o $} ;
\draw[dotted, gray] (0,6,0) node [below, xshift=0.1cm, yshift=0.1cm]{\color{black}  $1$} -- (0,6,6);
\draw[dotted, gray] (6,0,0) node [below]{\color{black}  $1$} -- (6,0,4);

%% l1-c1 
\draw[fill=black] (0,0, 2) circle[radius= 0.1 em]; 

\end{tikzpicture}

\end{document}