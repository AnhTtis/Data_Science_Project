%%%%%%%%%%%%%%%%%%%%%%%%%%%%%%
%       Interpretation       %
%%%%%%%%%%%%%%%%%%%%%%%%%%%%%%

With the discovery of GRB\,211211A, a long GRB ($\textrm{T}_{\rm 90}=34.2\pm0.6$\,s; \citealt{Veres2023}) being more consistent with a compact binary merger origin rather than a collapsar origin \citep{Gompertz2023, Rastinejad2022, Troja2022, Yang2022}, the class of short GRBs with extended emission is now being reconsidered. This class of GRBs was first introduced after GRB\,060614 was detected by \swift-BAT \citep{DellaValle2006, Gal-Yam2006, Gehrels2006}. This means \grb, being a long GRB due to its $\textrm{T}_{90}$ duration of $289\pm1$\,s (Section\,\ref{subsec:duration}), is no longer sufficient evidence for associating its progenitor to a massive star.

%-------------------------------%
%----- central engine -----%
%-------------------------------%

\subsection{Central Engine} \label{subsec:central_engine}

Comparison of the triggering pulse (Figure\,\ref{fig:main}, region I) against the orbital-averaged background (Figure\,\ref{fig:background_osv}) suggests a near full return to background around \t0+100\,s, followed by weak signal at \t0+121\,s (Figure\,\ref{fig:main}, region II) before the onset of the main pulse. The triggering pulse of \grb is unique due to its isolation in time, outlier spectral parameter values, no detectable emission prior to its onset, and \lle photons leading the \gbm data, possibly pointing to a distinct physical origin.

The best fit model for the triggering pulse is a mBB which is suggestive of photospheric thermal emission arriving at the observer from different locations on the equal arrival time surface \citep{Peer2008, Deng2014}. The COMP model, which fits the triggering pulse equally well, can be generated via the mBB model when $m=0$. Although the value of $m$ in Table\,\ref{tab:time-res_spec_fits} is not zero, it is still reasonably close to reproducing the COMP spectral shape.

For a spherically emitting shell, the MVT relates the radius of the shell to the Lorentz factor via,
\begin{equation} \label{eq:min_radius}
    \textrm{R} \sim \Delta t_{\textrm{min}} \p{\frac{\Gamma^{2} c}{\p{1+z}}}
\end{equation}
\noindent
where R is the radius of the spherically emitting shell, $\Gamma$ is the Lorentz factor, $c$ is the speed of light, and $z$ is the measured redshift. Along these lines, the MVT and the bulk Lorentz factor can be used to place limits the on emitting shell radius, assuming a singular emitting region. In interval Ia, the average value of MVT is \abt0.1\,s. Using the afterglow-derived bulk Lorentz factor for the ISM ($\Gamma^{\rm ag, \rm ISM}_{min} \gtrsim 260$) and the wind-type external medium ($\Gamma^{\rm ag, \rm wind}_{min} \gtrsim 282$) as a proxy for the bulk Lorentz factor in this region gives us an estimate on the radius of the emitting star. We find $R^{\rm ISM}_{\star} \gtrsim 1.7 \times 10^{14}$\,cm and $R^{\rm wind}_{\star} \gtrsim 2.0 \times 10^{14}$\,cm for the ISM and wind-type media respectively. If we instead use the Lorentz factor derived for this region ($\Gamma^{\rm trig, \rm ot}_{min}\gtrsim 188$), we achieve an estimated radius of $R^{\rm trig}_{\star} \gtrsim 9.0 \times 10^{13}$\,cm. All of which are roughly consistent with the radius of the outer wind from the Wolf-Rayet progenitor star ($R_{\star} > 1 \times 10^{13}$\,cm) \citep{Crowther2007}.

With the triggering pulse having thermally dominant spectral properties, a clear and distinct start time, a derived emitting shell radius consistent with that of a Wolf-Rayet progenitor star, along with the discovery of the associated supernova SN 2022xiw \citep{gcn32800, gcn32850, Fulton2023, Srinivasaragavan2023}; \grb is likely the result of a massive core-collapse supernovae progenitor. Although the beam-corrected $E_{\gamma}$ value of $6.1\times10^{51} \left(\theta_j/2\right)^2$\,erg from Section\,\ref{sec:energetics} is within the maximum energy release limit for a magnetar progenitor ($\sim3\times10^{52}$\,erg; \citealt{Usov1992}), this progenitor source is unlikely.

%--------------------------------%
%----- shock breakout -----%
%--------------------------------%

\subsection{Shock Breakout} \label{subsec:shock_breakout}

Shock breakout occurs when the radiation transport velocity is faster than the shock velocity and the radiation is no longer trapped within the shock \citep{Fryer2023}. This is not strictly at the stellar photosphere, but is often near it for a supernova shock and will be further out for a relativistic shock \citep{Fryer2020}. Just as with shock breakout and shock interaction for core-collapse supernovae, if the early emission is thermal, it can be used to probe characteristics of the progenitor and its immediate surroundings as well as the structure of the outflow. In the context of gamma-ray bursts, thermal emission can probe the progenitor-star photosphere (which is set by the mass loss and stellar radius), inhomogeneities of the mass-loss and structure of the jet, and its cocoon. For massive Wolf-Rayet stars, the likely progenitors of gamma-ray bursts, the high wind mass-loss rate often places the photosphere in the stellar wind and, for relativistic shocks, the shock breakout radius will be in the stellar wind. For this paper, we focus on more fundamental aspects behind a thermal component to determine whether it is a reasonable explanation of the observed emission, deferring a detailed comparison of the data to models for a later paper. For a more detailed discussion on the physics of shock breakout see \citealt{Fryer2020}.

If we assume the observed triggering emission is produced by the Lorentz-boosted thermal emission of shock breakout, the limits and shape of the emission can be used to constrain the properties of the shock as it emerges from the star. In the strong shock limit for a highly-relativistic gas, the pressure of the shock ($P_{\rm shock}$) is,

\begin{equation} \label{eq:P_shock}
    P_{\rm shock} \approx \Gamma^2 \rho_{\rm CSM} c^2
\end{equation}

\noindent
where $\Gamma$ is the Lorentz factor, $\rho_{\rm CSM}$ is the density in the region of shock breakout (in the wind of the massive star), and $c$ is the speed of light. Assuming the pressure is radiation-dominated, the corresponding temperature ($T_{\rm shock}$) of the emitting gas in the gas comoving frame is,

\begin{equation} \label{eq:T_shock}
    T_{\rm shock} \approx 1.9 (\Gamma/100)^{0.5} (\rho_{\rm CSM}/10^{-10}\,g\,cm^{-3})^{0.25} \; \textrm{keV}
\end{equation}

\noindent
and the corresponding peak energy of the emitted photons ($\nu_{\rm peak}$) in the observer frame is,

\begin{equation} \label{eq:nu_shock}
    \nu_{\rm peak} \approx 1 (\Gamma/100)^{1.5} (\rho_{\rm CSM}/10^{-10}\,g\,cm^{-3})^{0.25} \; \textrm{MeV}
\end{equation}

In the thermal shock breakout paradigm, the broad range and relatively flat spectra for the prompt emission requires a distribution of Lorentz factors. The observed peak emission around 15\,MeV places strong constraints on the upper limit of the Lorentz factor, corresponding to peak Lorentz factors lying between 300--1000, with corresponding densities of $70$--$0.05 \times 10^{-10} \, {\rm g \, cm^{-3}}$.

%---------------------------------%
%----- prompt emission ----%
%---------------------------------%

\subsection{Prompt Emission} \label{subsec:prompt_emission}

Unlike the triggering pulse, the spectra for the bulk of the prompt emission (regions III through VII) are all best fit with a Band function. Unlike the mBB and COMP spectral models, the Band function can not be generated through the superposition of Planck-like spectra, pointing to the bulk of the prompt emission having a non-thermal origin. Using Equation \ref{eq:min_radius}, the lower limit of the shock breakout derived triggering pulse Lorentz factor ($\Gamma=300$), and a delay time of \abt220\,s between regions I and III, we get an internal dissipation radius of \abt$6\times10^{17}$\,cm. This value is larger than is typically discussed but could be consistent with the ICMART model of $10^{16}$\,cm, which is consistent with the non-detection of neutrinos \citep{gcn32665, gcn32741} and a Poynting flux dominted jet \citep{Zhang2010}. The transition from thermal to non-thermal emission is not unique amongst GRBs. But this transition occurring in two isolated emission episodes separated by long quiet period has only ever been seen in one other burst, GRB\,160625B \citep{Zhang2018}. However, what is unique about \grb is we can, for the first time, directly say that from onset to afterglow the central engine never appears to shut off throughout the duration of the GRB.

%-----------------------------%
%-------- afterglow -------%
%-----------------------------%

\subsection{Afterglow} \label{subsec:afterglow}

As discussed in Section\,\ref{sec:afterglow}, we estimate the afterglow onset time to be $\sim t_{0}+280 $\,s. This time is at the end of the first \gbm BTI region, but it is possible that the afterglow began within the first \gbm BTI region itself. This aligns both with the -2 power-law spectral component fitted in region IVc and with the detection of the 99\,GeV \lat photon, further reinforcing that the afterglow begins during the prompt emission phase and the two are seen in a superposition with one another. Furthermore, We were also able to isolate a time at which the emission is comprised of solely-afterglow which provides us with a distinct time at which the central engine ceased emitting.

Comparing the highest and lowest Lorentz factors derived for the prompt emission ($\Gamma^{\rm prompt, \rm pp}\sim1560$ and $\Gamma^{\rm prompt, \rm pp}\sim780$ after correcting for a factor of 2) to those of the afterglow ($\Gamma^{\rm ag, \rm wind}\sim282$ and $\Gamma^{\rm ag, \rm ISM}\sim260$) yields a deceleration of $\Delta\Gamma^{\rm ISM}\sim1300$ and $\Delta\Gamma_{\rm wind}\sim1278$ (or $\Delta\Gamma_{\rm ISM}\sim520$ and $\Delta\Gamma_{\rm wind}\sim498$) over \abt380\,s. This rapid deceleration is suggestive of a relativistic reverse-shock (i.e., a \enquote{thick shell}) and is consistent with the deceleration time not exceeding the prompt GRB duration.

%---------------------------------%
%----- GRB properties -----%
%---------------------------------%

\subsection{Uniqueness} \label{subsec:uniqueness}

We therefore consider if a triggering pulse like this would be identified in other GRBs. Some bright long GRBs have weak trigger intervals followed by quiescence before the main emission episode, but most do not. The 1.024\,s peak-flux interval of this pulse would trigger \gbm out to z\,$\approx$\,1.3. Approximately half of \gbm GRBs with measured redshifts are within this range. Thus, a similar pulse should have been recovered in other bursts. The particularly high $\Gamma$ for this burst would produce a higher luminosity shock breakout, which may explain the lack of identification in other collapsars.

Figure\,\ref{fig:energetics} places the isotropic energy and luminosity into context with a broad sample of \gbm and \lat bursts with known redshift through 2017 \citep{Abbott2017a}, with updated spectral measures from \citet{Poolakkil2021}. \grb stands alone. It is the highest $E_{iso}$ in the \gbm sample, and with no higher claim existing in the literature, it is the brightest $E_{iso}$ ever recorded for a GRB. It is four orders of magnitude higher than GRBs seen at comparable redshifts and is an order of magnitude greater than the \enquote{Nearby Ordinary Monster} GRB\,130427A\, making \grb a nearerby extraordinary monster. In $L_{\rm iso}$, only GRB\,160625B is marginally higher; however, effects due to PPU and calibration issues caused by the geometry of incident photons with respect to \fermi make this our most limited measure.