%%%%%%%%%%%%%%%%%%%%%%%%%%%%%%
%      Prompt Emission       %
%%%%%%%%%%%%%%%%%%%%%%%%%%%%%%

%-----------------------------Figure Start--------------------------
\begin{figure*}[h!tbp]
    \centering
    \includegraphics[width=\textwidth]{figures/figure_3.pdf}
    \caption{Top: The gray lightcurves are of \grb as seen in \gbm detectors N4, N8, and B1. The black lightcurves are the averaged background from 30 orbits before and 30 orbits after the \gbm trigger time, when \fermi was in the same orbital location and orientation. The red shaded regions mark the background selection regions used for polynomial fitting. The red lines are the 4\texp{th} order polynomial fits to the \grb lightcurves that best match the averaged orbital background. The gray shaded area on the right marks the time when the GRB was occulted by the Earth. The discrepancy at $t=0$ between the gray GRB lightcurve and the black averaged background lightcurve in panel N4 is due to the propagation of the Earth occultation steps in the orbital fits (as seen around 1100\,s and 1200\,s). The polynomial order and background range can be found in Section\,\ref{sec:prompt_emission_phases}. Bottom: The $\chi^{2}_{\nu}$ fits of the polynomial fit to the \grb lightcurve in each of the 128 \cspec energy bins for each detector.}
    \label{fig:background_osv}
\end{figure*}
%-----------------------------Figure End----------------------------

Throughout our spectral analyses, we test a variety of standard spectral models including a power-law (PL), a power-law with an exponential cutoff (COMP), a Band function (Band; \citealt{Band1993}), a blackbody (BB), and combinations thereof \citep{Goldstein2012, Gruber2014, Poolakkil2021}. We additionally investigate two more complex spectral models, a double smoothly broken power-law (DSBPL; \citealt{Ravasio2018}) and a multicolor blackbody (mBB; \citealt{Hou2018}). When performing these spectral analyses, we used the Rmfit version 4.3.2\textsuperscript{\ref{ss:gbmsoftware}} spectral fitting package and the \gbm Data Tools\textsuperscript{\ref{ss:gbmsoftware}} Python software package \citep{GbmDataTools}. For analyses that could be performed by both programs, results were compared and shown to be consistent. 

The desire to investigate more complex spectral models (e.g., DSBPL and mBB) for \grb resulted in the extension of the \gbm Data Tools functionality to include a new Bayesian Markov Chain fitting technique, similar to the \textit{McSpecFit} package described in \cite{Zhang2016}. The validity of our Bayesian Markov Chain fitting technique was tested against the traditional \gbm Data Tools and Rmfit fitting routines using the standard spectral models mentioned previously and was shown to produce consistent results.

Figure\,\ref{fig:background_osv} shows that the prompt emission of \grb remained above background for most of its duration in \gbm. The \abt1200\,s of continuous emission means the standard polynomial background estimation method has a higher chance of misrepresenting the true background rate. To mitigate this, we used the orbital background estimation method\footnote[6]{\url{https://fermi.gsfc.nasa.gov/ssc/data/analysis/user/}\label{ss:gbmsoftware_osv}} for \gbm data presented in \citet{Fitzpatrick2011} which allowed us to better understand the background trend throughout the prompt emission phase. We performed a standard polynomial fitting technique to the lightcurve over a known background region on either side of the emission episode while adjusting the polynomial fit to best match the orbital background estimate (see Figure\,\ref{fig:background_osv}). The $\chi^{2}_{\nu}$ fit statistic for our polynomial fitting technique across all 128 \cspec energy channels is shown at the bottom of Figure\,\ref{fig:background_osv}. For the \lle data we used the standard polynomial fitting method with polynomial order 2 and background selection regions from \t0-200\,s to \t0-2\,s and from \t0+550\,s to \t0+560\,s. Due to the exceptionally long duration of this event, multiple detector response matrices DRMs for intervals spanning the burst duration were generated via the \gbm Response Generator\footnote[7]{\url{https://fermi.gsfc.nasa.gov/ssc/data/analysis/gbm/}\label{ss:gbmsoftware}}. Both the polynomial background estimate and these DRMs were used throughout our spectral analyses.

All spectral fits discussed in this section correspond to the lightcurve intervals and sub-intervals shown in Figure\,\ref{fig:main} and their fit parameters can be found in Table\,\ref{tab:time-res_spec_fits}. The fit statistics quoted in Table\,\ref{tab:time-res_spec_fits} are higher than expected. We attribute this to our coarse time resolution of our spectral fits. The time intervals used are likely not well fit because the spectral evolution evolves on a very short time scale and produces time-integrated shapes that significantly deviate from the time-resolved shape.

%-----------------------------%
%------ pre-trigger -------%
%-----------------------------%

\subsection{Pre-Trigger Interval} \label{subsec:pre-trigger_interval}

Using the \swift-XRT localization of \grb and \fermi's orbital location, we find the GRB was first visible to \gbm \abt2111\,s before \t0. We used the \gbm \targeted, an offline search algorithm designed to find subthreshold short GRBs in \gbm data, to search for emission prior to the \gbm trigger time \citep{Blackburn2015, Goldstein2016}. No such emission was found. Using the \targeted spectral templates representative of spectrally-hard and \enquote{normal} GRBs, we set 3$\sigma$ flux upper limits over this pre-trigger interval at $5.1 \times 10^{-8}$ erg cm$^{-2}$ s$^{-1}$ and $8.0 \times 10^{-8}$ erg cm$^{-2}$ s$^{-1}$, respectively, for the 1-s timescale in the 10\,keV to 1\,MeV energy range. Using the redshift of \grb, we limit the isotropic luminosity of pre-trigger emission to $L_{iso} < 7.1 \times 10^{48}$ erg s$^{-1}$ and $L_{iso} < 7.5 \times 10^{48}$ erg s$^{-1}$, respectively.

%-----------------------------Figure Start--------------------------
\begin{figure*}[h!tbp]
    \centering
    \includegraphics[width=\textwidth]{figures/figure_4_top.pdf}
    \begin{minipage}[b]{0.32\linewidth}
        \includegraphics[width=\textwidth]{figures/figure_4_bottom_left.pdf}
    \end{minipage}
	\begin{minipage}[b]{0.32\linewidth}
        \includegraphics[width=\textwidth]{figures/figure_4_bottom_middle.pdf}
    \end{minipage}
	\begin{minipage}[b]{0.33\linewidth}
        \includegraphics[width=\textwidth]{figures/figure_4_bottom_right.pdf}
    \end{minipage}
    \caption{Top: Background subtracted and normalized lightcurves of 0.512\,s binned \gbm \tte data and 1.0\,s binned \lle data for the triggering pulse with the higher-energy photons arriving prior to the lower-energy photons. Bottom: The fitted counts spectrum (left), the model spectrum (center), and the posteriors of the model parameters (right) for the multicolor blackbody function fitted to the triggering pulse.}
    \label{fig:precursor}
\end{figure*}
%-----------------------------Figure End----------------------------

%------------------------------%
%---- triggering pulse ----%
%------------------------------%

\subsection{Triggering Pulse} \label{subsec:triggering_pulse}

The triggering pulse (Figure\,\ref{fig:main}, region I) was analyzed using a combination of \gbm \cspec and \lat \lle data. Figure\,\ref{fig:precursor} shows high-energy \lle photons arriving \abt1.5\,s earlier than lower-energy BGO and NaI photons, with a lag that increases as energy decreases. Although distinct, this behavior is not unique as it has been previously observed in the initial pulse of GRB\,130427A as well \citep{Ackermann2014, Preece1998}. 

The first interval, referred to as sub-region Ia in Table\,\ref{tab:time-res_spec_fits}, spans the first 8 seconds of region I and was best fit by a COMP function with low-energy photon index $\alpha=-1.69$ and $\textrm{E}_{peak}=4$\,MeV. We compared these values to the corresponding time-integrated parameter distributions in the \gbm 10-year Spectral Catalog \citep{Poolakkil2021} and find that only 0.26\% of bursts (6/2295) have $\alpha<-1.69$ and 0.23\% have $\textrm{E}_{peak}$ above 4\,MeV (3/1311). While we note that this is not a direct comparison as this fit is neither time-integrated for a whole burst nor a 1.024\,s peak-flux interval, the $\alpha$ and $\textrm{E}_{peak}$ values are both outliers at $\sim$3$\sigma$ significance.

The full triggering pulse (region I), from \t0-1.3\,s to \t0+42.9\,s, was best fit with a COMP model (see Table\,\ref{tab:time-res_spec_fits}). The mBB model, comprised of a superposition of blackbodies with different temperatures, can produce the COMP model when the power-law index (m) equals zero \citep{Hou2018}. We tested whether the triggering pulse of \grb is consistent with a quasi-thermal origin by fitting region I with this function. Although the best fit does not reproduce the COMP spectra, these two models both produce adequate fits to the data. As shown in Table\,\ref{tab:time-res_spec_fits}, this fit yields a \kTmin of \abt1.6\,keV, and a \kTmax of \abt5\,MeV. Both counts and model spectra for this fit are shown in Figure\,\ref{fig:precursor}. \kTmin and \kTmax can be converted into peak (or \enquote{break}) energies with the relation $E_{b,1}\sim3kT_{min}$ (or $E_{b,2}\sim3kT_{max}$) \citep{Hou2018}. This yields $E_{b,1}\approx4.8$\,keV and $E_{b,2}\approx15$\,MeV. The \gbm+\lle energy range has lower and upper limits of \abt20\,keV and \abt300\,MeV, respectively for our analysis of \grb. This means we were only able to confidently constrain $E_{b,2}$ due to the addition of \lle data. The $E_{b,1}$ value from our analysis only tells us that the power-law index in the mBB fit extends below the \gbm energy range. The mBB model from \citealt{Hou2018} also gives a direct measure of luminosity. The fit value of $14.5_{-0.6}^{+0.5}$ with the scale $L_{39}/D^2_{L,10 \rm{kpc}}$ gives a corresponding value of $7.5_{-0.3}^{+0.3} \times 10^{49}$\,erg\,s$^{-1}$ at z=0.151 ($D_L = 724$\,Mpc).

%------------------------------%
%------ quiet period ------%
%------------------------------%

\subsection{Quiet Period} \label{subsec:quiet}

We define the quiet period (Figure\,\ref{fig:main}, interval II) as the time between the triggering pulse and the onset of the main prompt emission phase. This region appears to have no emission, but careful analysis reveals detectable low-level emission during most of this time (see NaI panels of Figure\,\ref{fig:background_osv}). We associate the flux to \grb through consistent localizations of the low-level flux to the location of the GRB. The \t0+121\,s to \t0+164\,s interval, just before the onset of the main prompt emission, is best fit by a COMP function with an additional BB 

%----- time-res spec table -----%
\begin{longtable*}{cc|ccccc|cc}
\label{tab:time-res_spec_fits}\\
\hline
\hline
\textbf{Model} & \textbf{Time} &  &  & \textbf{Model} &  &  & \multirow{2}{*}{\textbf{$\textrm{C}_{stat}$/DoF}} & \lle \\
\textbf{(region)} & \textbf{Range} &  &  & \textbf{Components} &  &  &  & \textbf{Used?} \\
\endfirsthead
\endhead
\hline
\hline
%-----------------------------------------------------
% precursor - first 8 sec (COMP)
%-----------------------------------------------------
COMP & -0.003 &  &  & $\alpha$ & $\textrm{E}_{peak}$ &  & \multirow{2}{*}{485/343} & \multirow{2}{*}{Y} \\
\cline{3-7}
(Ia) & 8.576 &  &  & $-1.69\pm0.01$ & $3980\pm366$ &  &  &  \\
\hline
\hline
%-----------------------------------------------------
% precursor (COMP)
%-----------------------------------------------------
COMP & -1.343 &  &  & $\alpha$ & $\textrm{E}_{peak}$ &  & \multirow{2}{*}{497/340} & \multirow{2}{*}{Y} \\
\cline{3-7}
(I) & 42.881 &  &  & $-1.73\pm0.02$ & $10440\pm1900$ &  &  &  \\
\hline
%-----------------------------------------------------
% precursor (BB+COMP)
%-----------------------------------------------------
BB+COMP & -1.343 &  & kT & $\alpha$ & $\textrm{E}_{peak}$ &  & \multirow{2}{*}{494/338} & \multirow{2}{*}{Y} \\
\cline{3-7}
(I) & 42.881 &  & $5.23\pm2.37$ & $-1.69\pm0.03$ & $9767\pm1570$ &  &  &  \\
\hline
%-----------------------------------------------------
% precursor (mBB)
%-----------------------------------------------------
mBB & -1.343 & K & $\textrm{kT}_{min}$ & m & $\textrm{kT}_{max}$ &  & \multirow{2}{*}{487/379} & \multirow{2}{*}{Y} \\
% mBB & -1.343 & K & $\textrm{kT}_{min}$ & m & $\textrm{kT}_{max}$ &  & \multirow{2}{*}{487/338} & \multirow{2}{*}{Y} \\
\cline{3-7}
(I) & 42.881 & $14.5^{+0.5}_{-0.6}$ & $1.62^{+0.20}_{-0.34}$ & $-0.854^{+0.009}_{-0.011}$ & $5000^{+600}_{-400}$ &  &  &  \\
\hline
\hline
%-----------------------------------------------------
% quiescent (BB+COMP)
%-----------------------------------------------------
BB+COMP & 121.219 &  & kT & $\alpha$ & $\textrm{E}_{peak}$ &  & \multirow{2}{*}{2169/334} & \multirow{2}{*}{Y} \\
\cline{3-7}
(II) & 164.228 &  & $19.27\pm1.66$ & $-0.4\pm0.2$ & $5722\pm564$ &  &  &  \\
\hline
%-----------------------------------------------------
% pre-main (Band)
%-----------------------------------------------------
Band & 176.516 &  &  & $\alpha$ & $\textrm{E}_{peak}$ & $\beta$ & \multirow{2}{*}{2130/338} & \multirow{2}{*}{Y} \\
\cline{3-7}
(III) & 210.309 &  &  & $-1.158\pm0.003$ & $1023\pm12$ & $-3.28\pm0.04$ &  &  \\
\hline
%-----------------------------------------------------
% pre-BTI 1 (Band)
%-----------------------------------------------------
Band & 210.309 &  &  & $\alpha$ & $\textrm{E}_{peak}$ & $\beta$ & \multirow{2}{*}{1741/352} & \multirow{2}{*}{Y$^{*}$} \\
\cline{3-7}
(IVa) & 219.525 &  &  & $-1.159\pm0.003$ & $3664\pm47$ & $-2.70\pm0.03$ &  &  \\
\hline
%-----------------------------------------------------
% post-BTI 1 (Band+PL)
%-----------------------------------------------------
Band+PL & 277.894 &  & Index & $\alpha$ & $\textrm{E}_{peak}$ & $\beta$ & \multirow{2}{*}{5707/336} & \multirow{2}{*}{Y$^{*}$} \\
\cline{3-7}
(IVc) & 323.975 &  & $-1.916\pm0.009$ & $-1.583\pm0.001$ & $1387\pm9$ & $-3.77\pm0.01$ &  &  \\
\hline
%-----------------------------------------------------
% IPP 1 (Band)
%-----------------------------------------------------
Band & 326.023 &  &  & $\alpha$ & $\textrm{E}_{peak}$ & $\beta$ & \multirow{2}{*}{3801/335} & \multirow{2}{*}{Y} \\
\cline{3-7}
(Va) & 381.023 &  &  & $-1.80\pm0.01$ & $69\pm1$ & $-2.24\pm0.01$ &  &  \\
\hline
%-----------------------------------------------------
% IPP 2 (Band)
%-----------------------------------------------------
Band & 381.023 &  &  & $\alpha$ & $\textrm{E}_{peak}$ & $\beta$ & \multirow{2}{*}{4217/335} & \multirow{2}{*}{Y} \\
\cline{3-7}
(Vb) & 435.546 &  &  & $-1.658\pm0.003$ & $520\pm11$ & $-2.98\pm0.02$ &  &  \\
\hline
%-----------------------------------------------------
% IPP 3 (Band)
%-----------------------------------------------------
Band & 433.546 &  &  & $\alpha$ & $\textrm{E}_{peak}$ & $\beta$ & \multirow{2}{*}{3605/318} &  \\
\cline{3-7}
(Vc) & 482.699 &  &  & $-1.610\pm0.003$ & $527\pm9$ & $-2.46\pm0.02$ &  &  \\
\hline
%-----------------------------------------------------
% pre-BTI 2 (Band)
%-----------------------------------------------------
Band & 482.699 &  &  & $\alpha$ & $\textrm{E}_{peak}$ & $\beta$ & \multirow{2}{*}{1992/317} &  \\
\cline{3-7}
(VIa) & 508.299 &  &  & $-1.512\pm0.003$ & $564\pm8$ & $-2.55\pm0.02$ &  &  \\
\hline
%-----------------------------------------------------
% post-BTI 2 (Band)
%-----------------------------------------------------
Band & 515.467 &  &  & $\alpha$ & $\textrm{E}_{peak}$ & $\beta$ & \multirow{2}{*}{3677/335} & \multirow{2}{*}{Y} \\
\cline{3-7}
(VIc) & 546.188 &  &  & $-1.484\pm0.002$ & $1133\pm11$ & $-3.53\pm0.04$ &  &  \\
\hline
%-----------------------------------------------------
% final (Band)
%-----------------------------------------------------
Band & 546.188 &  &  & $\alpha$ & $\textrm{E}_{peak}$ & $\beta$ & \multirow{2}{*}{3111/318} &  \\
\cline{3-7}
(VII) & 597.389 &  &  & $-1.648\pm0.004$ & $280\pm6$ & $-2.23\pm0.01$ &  &  \\
\hline
%-----------------------------------------------------

\hline
\hline
% \begin{minipage}[b]{0.42\linewidth}
    \caption{The best fitting spectral functions for the lightcurve intervals and sub-intervals described in Section\,\ref{sec:temporal_analysis} and presented in Section\,\ref{sec:prompt_emission_phases}. \enquote{Y$^{*}$} is used to denote regions where the \lat \lle data spans a shorter time interval then the \gbm \cspec data due to the differing \lat and \gbm BTI regions. $\alpha$ is the low-energy photon index, $\beta$ is the high-energy photon index, and Index refers to the additional PL component index. For the mBB model, $m$ is the shape parameter described in \citealt{Hou2018} and $K$ is defined as $L_{39}/D^2_{L,10 \rm{kpc}}$, where $L$ is the luminosity in the rest frame in units of $10^{39}$\,erg s\texp{-1} and $D_{L}$ is the luminosity distance in units of 10\,kpc. All values of $\textrm{E}_{peak}$ and kT are in keV. Interval VIII (the afterglow) is not included in this table.}
% \end{minipage}
\end{longtable*}



%-------------------------------% 

\noindent
component peaking at \abt19\,keV. The model spectrum for this fit is shown in the top left panel of Figure\,\ref{fig:ppu}.

%--------------------------------%
%----- pre-main pulse -----%
%--------------------------------%

\subsection{Pre-Main Pulse} \label{subsec:pre-main_pulse}

The bulk of the main emission episode begins at \t0+176\,s with a subdominant pulse from \t0+176\,s to \t0+210\,s (Figure \ref{fig:main}, interval III). Taken by itself, this would be one of the brightest GRBs in the \gbm sample in terms of peak flux. Although part of the main pulse, we analyzed this pulse separately because it is clearly distinct from the following bright main emission and is the last contiguous emission interval before the onset of \gbm data issues. This interval is best fit by a Band function, the parameters of this fit can be found in Table\,\ref{tab:time-res_spec_fits}.

%-----------------------------Figure Start--------------------------
\begin{figure*}[h!tbp]
    \centering
    \begin{minipage}[b]{0.32\linewidth}
    	\includegraphics[width=\textwidth]{figures/figure_5_top_left.pdf}
    \end{minipage}
	\begin{minipage}[b]{0.32\linewidth}
        \includegraphics[width=\textwidth]{figures/figure_5_top_middle.pdf}
    \end{minipage}
	\begin{minipage}[b]{0.32\linewidth}
        \includegraphics[width=\textwidth]{figures/figure_5_top_right.pdf}
    \end{minipage}
    \begin{minipage}[b]{0.32\linewidth}
        \includegraphics[width=\textwidth]{figures/figure_5_bottom_left.png}
    \end{minipage}
    \begin{minipage}[b]{0.32\linewidth}
        \includegraphics[width=\textwidth]{figures/figure_5_bottom_middle.png}
    \end{minipage}
	\begin{minipage}[b]{0.32\linewidth}
        \includegraphics[width=\textwidth]{figures/figure_5_bottom_right.png}
    \end{minipage}
    \caption{Top: The $\nu$F($\nu$) spectra for region I best fit by a COMP (left), the $\nu$F($\nu$) spectra for region II best fit by a COMP+BB (middle) and the $\nu$F($\nu$) spectra for sub-region IVc best fit by a Band+PL (right). Bottom: Average examples of counts spectrum plots during the \gbm BTI region of sub-interval IVb (Left: \t0+243.1\,s, Middle: \t0+244.1\,s, Right: \t0+265.6\,s) achieved by using the PPU-correction method described in Section\,\ref{subsec:BTIs}. The spike at 511\,keV is ignored during the fitting process and does not affect the fit statistic.}
    \label{fig:ppu}
\end{figure*}
%-----------------------------Figure End----------------------------

%-------------------------------%
%------ primary pulse -----%
%-------------------------------%

\subsection{Primary Pulse} \label{subsec:primary_pulse}

We define the main pulse (Figure\,\ref{fig:main}, interval IV) as the region between \t0+210\,s to \t0+324\,s. The bulk of the main emission occurred during the first \gbm BTI, during which both \gbm and \lat experienced data issues\textsuperscript{\ref{ss:btis}}. Due to these data issues we divide this interval into three sub-intervals, before (IVa), during (IVb), and after (IVc) the first \gbm BTI (sub-intervals shown in Figure\,\ref{fig:main}). Although the \lat BTI falls into sub-intervals IVa and IVc, we do not attempt to correct or use any \lat \lle data within the published \lat BTIs.

Region IVa begins before the onset of \gbm data issues and is best fit by a Band function that peaks in energy (\Epk) around 3.7\,MeV. Region IVc begins after the \gbm data issues have subsided and is best fit by a Band function with an additional PL component, extending the fit out to higher energies. The Band function in this region peaks in energy (\Epk) at \abt1.4\,MeV. The additional PL of this fit has photon index of \abt-1.9, which is consistent with the canonical PL value of $\Gamma$= -2 expected from the high-energy component of the electron synchrotron spectrum for both the slow- and fast- cooling regime, for an assumed power-law electron energy distribution of $p=2$ \citep{Granot2002}. This spectral component is consistent with the emergence of the early afterglow, over which the rest of the prompt emission is superimposed, similar to behaviour observed in GRB\,190114C \citep{Ajello2020}. The model spectrum for this fit can be found in Figure\,\ref{fig:ppu}.

Region IVb is the time of the first \gbm BTI. As mentioned previously, the \tte data within the \gbm BTIs is irrecoverable. Although the \cspec data experienced PPU and deadtime effects, the data are corrected via the method described in Section\,\ref{subsec:BTIs}. Within this BTI the lightcurve consists of two distinct peaks. Our assumption of a Band function for the underlying spectral shape produces adequate fits throughout. However, we observe varying goodness-of-fit measures that coincide with the times of these two peaks. Examples of PPU-corrected counts spectra during these peaks are shown in Figure\,\ref{fig:ppu}. Due to the current limitations of our PPU-correction technique described in Section\,\ref{subsec:BTIs} we are unable to discern whether these variations are due to spectrotemporal evolution in the data or uncorrected PPU effects. Spectral fit parameters are not reported for fits within the \gbm BTI regions because the purpose of these fits is to determine the energetics of \grb. Further analysis is needed to determine the reliability of our PPU-correction technique for spectral modeling of this GRB with additional spectral models being considered.

%-------------------------------%
%--- intra-pulse period ---%
%-------------------------------%

\subsection{Intra-Pulse Period} \label{subsec:intra-pulse_period}

The intra-pulse period (region V) consists of smoother emission with three distinct pulses in sub-regions Va, Vb, and Vc as shown in Figure\,\ref{fig:main}. We fit each pulse independently, using \lle data when available. All three pulses were best fit with Band functions. Although the pulses in sub-intervals Va and Vb have similar temporal structures and MVTs (see Figure\,\ref{fig:mvt}), the pulse in sub-interval Va peaks at a much lower energy (\Epk=\abt69\,keV and \Epk=\abt520\,keV, respectively). The pulse in sub-interval Vc occurs right before the onset of the second high intensity region and has a MVT and temporal structure that clearly differs from the first two pulses. Despite these differences, this pulse peaks at an energy similar to sub-interval Vb (\Epk=\abt527\,keV). Although the intensity and variability of the emission in this period is lower than the surrounding regions, at no point does the emission truly become quiescent, showing non-thermal activity throughout the interval. 

%---------------------------------%
%---- secondary pulse -----%
%---------------------------------%

\subsection{Secondary Pulse} \label{subsec:secondary_pulse}

Due to the secondary pulse (Figure\,\ref{fig:main}, region VI) containing the second \gbm BTI within it, we separate this region into three sub-regions (VIa, VIb, and VIc). As with the primary pulse, sub-regions VIa and VIc were not affected by \gbm data issues so standard spectral analyses could be performed.

Although the first (VIa) and last (VIc) sub-regions were both best fit by a Band function with comparable low-energy photon indices $\alpha$ of \abt-1.5 and high-energy photon indices $\beta$ values of \abt-2.5 and \abt-3.5 respectively, they differed largely in peak energy measurements. Sub-region VIa peaked in energy at \abt564\,keV while sub-regions VIc had a measured \Epk\,of \abt1.1\,MeV. Such a difference is expected as sub-regions VIc directly followed a period of reenergization.

Sub-region VIb is bound by the second \gbm BTI interval and contains only 7 time bins of \gbm \cspec data. Within this time the data contains only a single bright spike. We follow the same procedure as Section\,\ref{subsec:primary_pulse} and as described in Section\,\ref{subsec:BTIs}. The resulting spectral fits, residuals, and fit statistics adequately constrain the data in every time bin, but produce more residuals near the peak, as expected (Figure\,\ref{fig:ppu}). As discussed in Section\,\ref{subsec:primary_pulse}, we do not report any spectral fit parameters within this region in Table\,\ref{tab:time-res_spec_fits}.

%-----------------------------%
%------ final pulse -------%
%-----------------------------%

\subsection{Final Pulse} \label{subsec:final_pulse}

The final pulse (Figure\,\ref{fig:main}, region VII) occurs from \t0+560\,s to \t0+597\,s. We define the end time of this region arbitrarily as it inevitably merges with the long, smooth decay phase marked as region VIII in Figure\,\ref{fig:main}. This pulse was best fit by a Band function, the parameters of which can be found in Table\,\ref{tab:time-res_spec_fits}.
