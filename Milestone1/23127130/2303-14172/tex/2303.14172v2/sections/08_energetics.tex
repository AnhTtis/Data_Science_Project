%%%%%%%%%%%%%%%%%%%%%%%%%%%%%%
%         Energetics         %
%%%%%%%%%%%%%%%%%%%%%%%%%%%%%%

%-----------------------------Figure Start--------------------------
\begin{figure*}[h!tbp]
    \centering
    \includegraphics[width=\textwidth]{figures/figure_7_top.pdf}
    \begin{minipage}[b]{0.49\linewidth}
        \includegraphics[width=\textwidth]{figures/figure_7_bottom_left.pdf}
    \end{minipage}
	\begin{minipage}[b]{0.49\linewidth}
        \includegraphics[width=\textwidth]{figures/figure_7_bottom_right.pdf}
    \end{minipage}
    \caption{Top: The lightcurve of \grb from 8\,keV to 40\,MeV before and after PPU-correcting in the \gbm BTI region. Bottom: Short and long GRB $E_{\rm iso}$ (left) and $L_{\rm iso}$ (right) measures of \gbm-detected GRBs with known redshift through 2017 \citep{Abbott2017a}, with updated spectral measures from \citet{Poolakkil2021}. Key GRBs are highlighted. The downward arrows indicate GRBs best fit by a PL spectral model, which must turnover somewhere, so they are shown as upper limits. The $E_{\rm iso}$ plot is supplemented with low or intermediate luminosity GRBs with associated supernova from \citet{Cano2017}. We do not utilize their luminosity measures as they are averaged (not peak).}
    \label{fig:energetics}
\end{figure*}
%%-----------------------------Figure End----------------------------

In order to calculate burst energetics the PPU-corrected data within the two \gbm BTI regions must be used. As was done with the $\textrm{T}_{90}$ analysis, all non-BTI regions shown in Figure\,\ref{fig:main} were also fit with a  Band function. For intervals where a Band function is not the preferred spectral form, we find it still produces an adequate fit and therefore introduces negligible errors in our results.

We derive the total isotropic equivalent energy ($E_{\gamma,\textrm{iso}}$) in the 1-10,000\,keV range from the fluences in the individual time intervals from \t0-2.7\,s to \t0+1449.5\,s and perform k-corrections in all intervals. We find the total Fluence = $(9.47\pm0.07)\times10^{-2}$\,erg\,cm$^{-2}$ and $E_{\gamma,\textrm{iso}}=(1.01\pm0.007)\times10^{55}\,\erg$. Assuming an opening angle of 2 degrees \citep{Negro2023}, the beaming corrected energy is $E_{\gamma}=6.1\times10^{51} \left(\theta_j/2\right)^2$\,erg. We obtain the isotropic-equivalent luminosity, by integrating the spectrum in the \t0+230.8\,s to \t0+231.8\,s interval. The energy flux here is $F=(8.48 \pm 0.06)\times10^{-2}$\,erg s$^{-1}$\,cm$^{-2}$. After k-correction, we find the 1\,s peak luminosity to be $L_{\gamma,\textrm{iso}}=(9.91 \pm 0.06)\times 10^{53}$\,erg s$^{-1}$. These values are consistent with those independently produced by \citet{Frederiks2023}, \citet{Ripa2023}, and \citet{An2023} which demonstrate the validity of the \gbm PPU-correction technique for this GRB.
