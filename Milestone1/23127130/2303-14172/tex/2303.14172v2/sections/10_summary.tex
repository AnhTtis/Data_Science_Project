%%%%%%%%%%%%%%%%%%%%%%%%%%%%%%
%           Summary          %
%%%%%%%%%%%%%%%%%%%%%%%%%%%%%%

Our dedicated search suggests that there is no emission from the central engine prior to the \gbm trigger time. Weak emission from the triggering pulse (region I) can be localized to \grb for up to 100\,s before fully returning to background. The first \abt8\,s of the triggering pulse has a peak energy (\Epk) of 4\,MeV with the highest energy photons arriving first. The entire triggering pulse has spectrotemporal properties best characterized by the Lorentz-boosted thermal emission of shock breakout. Using the peak energy of the assumed shock breakout emission we are able to place the Lorentz factor ($\Gamma$) of this pulse between 300 and 1000 with corresponding shock densities between 70 and 0.05$\times 10^{-10}\, {\rm g \, cm^{-3}}$. These properties along with our derived emitting shell radius and the discovery of the associated supernova SN 2022xiw  point to a core-collapse Wolf-Rayet star being the most likely central engine progenitor.

We find another weakly emitting pulse (region II) which we were able to localize to \grb just before the bulk of the main emission began. The central engine is then continuously active, showing non-thermal activity throughout the bulk of the prompt emission. With two periods of severe re-brightening, at no point does the prompt emission become quiescent. After correcting for PPU effects in the \gbm data we were able to characterize the total energetics of this burst. With a total isotropic-equivalent energy of $\textrm{E}_{\gamma,\textrm{iso}}=1\times10^{55}$\,erg and an isotropic-equivalent luminosity of $\textrm{L}_{\gamma,\textrm{iso}}=9.9\times10^{53}$\,erg s$^{-1}$, \grb is the most intrinsically energetic and second most intrinsically luminous in the \gbm sample. Assuming a MVT of 0.05\,s we place constraints on the Lorentz factor during the brightest interval via pair-creation using the \lat reported 99.3\,GeV photon and by assuming an optically thin environment. These values are $\Gamma^{\rm prompt, \rm pp}_{min}\gtrsim 1560$ and $\Gamma^{\rm prompt, \rm ot}_{min}\gtrsim 1470$ respectively (without the factor of 2 correction).

By comparing our measurements with those observed by \swift-XRT we are able to characterize the onset of the early-afterglow period. We place lower limits on the afterglow peak time assuming both ISM and wind-type external media of $t^{\rm ag, \rm ISM}_{peak}\gtrsim t_{0}+(140\pm1.5)$\,s and $t^{\rm ag, \rm wind}_{peak}\gtrsim t_{0}+(120\pm6.5)$\,s respectively. We additionally find an early plateau region with a slope $\alpha^{ag}_{decay}=-0.82\pm0.03$ which gradually steepens to the observed \swift-XRT slope of $\alpha_{s}=-1.5$ at \t0$>$1400\,s. We are also able to project this afterglow back into the prompt emission, where the two are in a superposition with one another, and estimate the start time of the afterglow ($t^{\rm ag, \rm start, \rm ISM}_{peak}\gtrsim t_{0}+280$\,s). Again assuming both an ISM and wind-type external media we calculate the Lorentz factor of the external shock as $\Gamma^{\rm ag, \rm ISM}_{min} \gtrsim 260$ and $\Gamma^{\rm ag, \rm wind}_{min} \gtrsim 282$ respectively. The change in Lorentz factor from prompt emission to afterglow is suggestive of a relativistic reverse-shock.

The observations presented here provide an unrivaled probe into the continuously active central engine. Without the abundance of photons provided by this GRB, the lower flux intervals (e.g., the triggering pulse, between the \gbm BTIs, and the afterglow) would have been lost amongst the background, creating a \enquote{tip of the iceberg} effect with only the high flux intervals being visible. While we could infer the continuity of the central engine in other GRBs, we have never had the abundant photons to observe the central engine as well as was done for \grb. While the spectroscopy of the triggering pulse is strongly indicative of a thermal, photospheric origin, as is expected to occur early in GRBs, its significantly lower intensity compared to the rest of the emission makes its detection in other bursts particularly challenging. Furthermore, the abundance of photons in \grb allows us to track the evolution of the bulk Lorentz factor through to the afterglow phase, providing a stronger indication that the reverse shock is encountered as the emission enters the afterglow phase.

%------------------------------% 
%------- dedication -------%
%------------------------------%

\vspace{5mm}

\noindent We dedicate this paper to the memory of William \enquote{Bill} Paciesas who passed away in June 2022. Bill was a co-investigator of BATSE and, for a time, the Principal Investigator of \gbm, but he was arguably more well known by the high-energy astrophysics community for his punny humor and beer connoisseurship. With \grb being the Brightest Event Ever Recorded (BEER) and Bill not here to share it with us; this BEER's for you.

\vspace{5mm}
