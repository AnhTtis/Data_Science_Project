%--------------------------% 
%----- lorentz factor -----%
%--------------------------% 

With the requirement that the emission site be optically thin, we can derive lower limits on the bulk Lorentz factor ($\Gamma^{\rm ot}_{min}$; \citealt{Lithwick2001}). For this method we usually use the MVT in conjunction with the spectral shape. Based on our COMP function spectral fit in region Ia of the triggering pulse ($\alpha=-1.69\pm0.01$, $E_{\rm peak}=3.98\pm0.37$\,MeV, and energy flux=$(1.98\pm0.03)\times 10^{-6}$ erg cm$^{-2}$ s$^{-1}$ in the 10-1000 keV range), we find a bulk Lorentz factor of $\Gamma^{\rm trig, \rm ot}_{min}\gtrsim 188$. 

Deriving the bulk Lorentz factor in the main prompt emission in the same way is more difficult because we can only derive the MVT for time intervals without PPU. However, the spectrum within the \gbm BTIs yields much stronger constraints on $\Gamma^{\rm ot}_{min}$ even with conservative assumptions for the MVT. Based on Figure\,\ref{fig:mvt} and the discussions in Section\,\ref{subsec:mvt}, we assume a conservative MVT estimate of $\Delta t_{\rm var}=0.1 $\,s. Again requiring that the prompt emission be optically thin and utilizing the PPU-corrected spectrum in the brightest time interval we set a limit of $\Gamma^{\rm prompt, \rm ot}_{min}\gtrsim 1040$. If we instead assume $\Delta t_{\rm var}=0.05 $\,s, which is still reasonable given the MVT values surrounding region IV, we achieve a Lorentz factor limit of $\Gamma^{\rm prompt, \rm ot}_{min}\gtrsim 1470$. 

We can also set single zone Lorentz factor limit through pair-production for the highest energy photons ($\Gamma^{\rm pp}_{min}$). \lat observed a 99.3 GeV photon at \t0+240\,s \citep{gcn32658}. Assuming the \gbm emission emanates from the same volume and the requirement that photons can escape without producing $e^{\pm}$ pairs yields a Lorentz factor constraint of $\Gamma^{\rm prompt, \rm pp}_{min}\gtrsim 1560$. We note that the calculation of $\Gamma^{\rm pp}_{min}$ by \citealt{Lithwick2001} is done under the assumption of a single zone. If we instead consider a more realistic situation, taking into account the angular, temporal, and spatial dependence of the radiation field, our values of $\Gamma^{\rm pp}_{min}$ could be lower by a factor of \abt2 (i.e., $\Gamma^{\rm prompt, \rm pp}_{min}\gtrsim 780$) \citep{Hascoet2012, Gill2018, Vianello2018, Arimoto2020}. Although both MVT-derived optically thin Lorentz factor lower limits fall below the single zone pair-production Lorentz factor lower limit, the two methods independently produce consistent results.

The peak of the afterglow emission denotes the beginning of the external shock deceleration time. By identifying this peak for both ISM and wind-type external media as $t^{\rm ag, \rm ISM}_{peak}\gtrsim t_{0}+(140\pm1.5)$\,s and $t^{\rm ag, \rm wind}_{peak}\gtrsim t_{0}+(120\pm6.5)$\,s respectively, we can derive upper limits for the Lorentz factor of the afterglow via,

\begin{equation} \label{eq:Lorentz_factor_general}
    \Gamma^{\rm ag}_{min} = \pp{\p{\frac{(17-4s)}{16\pi(4-s)}}\p{\frac{E_{k}}{n_0 m_{\rm p}c^{5-s}}}}^{\frac{1}{8-2s}}\p{\frac{t_{\rm peak}}{(1+z)}}^{-\frac{3-s}{8-2s}}
\end{equation}

\noindent
where $s=0$ assumes an ISM external density profile and $s=2$ assumes a wind-type external medium \citep{Nappo2014, Ghirlanda2018}. For both cases we use the VLT reported redshift of z = 0.151 \citep{gcn32648} and assume the kinetic energy, $E_k$ is approximately equal to the isotropic-equivalent gamma-ray energy ($E_{k}\approx E_{iso}$).

Assuming an ISM external density profile ($s=0$) we have:

\begin{equation} \label{eq:Lorentz_factor_ISM}
    \Gamma^{\rm ag, \rm ISM}_{min} \gtrsim 260 \left(\frac{E_{k,55}}{n/1~ {\rm cm}^{-3}}\right)^{1/8} \left(\frac{t_{\rm peak}}{120 ~{\rm s} \times 1.151}\right)^{-3/8}
\end{equation}
Using the $t^{\rm ag, \rm start, \rm ISM}_{peak} = 105 $\,s value, the Lorentz factor increases slightly to $\Gamma^{\rm ag, \rm start, \rm ISM}_{min}\gtrsim 270$ with all scaling parameters being the same.

For the wind-type external medium ($s=2$) the density parameter, $n= 3\times 10^{35}~A_\star~ {\rm cm}^{-1}$ \citep{grb221009a_fermi_lat_collaboration_2023}, gives us:

\begin{equation} \label{eq:Lorentz_factor_wind}
    \Gamma^{\rm ag, \rm wind}_{min}\gtrsim 282 \left(\frac{E_{k,55}}{A_{\star,-1}}\right)^{1/4} \left(\frac{t_{\rm peak}}{ 140 ~{\rm s} \times 1.151}\right)^{-1/4}
\end{equation}