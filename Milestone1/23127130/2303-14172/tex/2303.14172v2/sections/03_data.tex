%%%%%%%%%%%%%%%%%%%%%%%%%%%%%%
%       Data Selection       %
%%%%%%%%%%%%%%%%%%%%%%%%%%%%%%

\fermi includes two scientific instruments, \gbm and \lat. \gbm is a wide-field ($>$8 sr) survey instrument comprised of twelve sodium iodide (NaI) detectors and two bismuth germanate (BGO) detectors \citep{Meegan2009}. The NaI detectors cover the energy range from 8\,keV to 1000\,keV and are oriented in different directions around the spacecraft as to observe the entire unocculted sky. The two BGO detectors are on opposite sides of the spacecraft and cover an energy range from 200\,keV to 40\,MeV. The \lat is a pair-conversion telescope at the zenith of the spacecraft and is sensitive to gamma-ray energies from 20\,MeV to more than 300\,GeV \citep{Atwood2009}.

The highest resolution \gbm data product is the Time-Tagged Event (\tte) data, tagging individual photons to 2 microsecond temporal precision at the full 128 spectral channel resolution of \gbm. Continuous Spectroscopy (\cspec) data has the same full spectral resolution as \tte data but is a binned dataset with temporal resolution as fine as 1.024\,s in the interval following the trigger. Continuous Time (\ctime) data has 8 spectral channels with a temporal resolution of 64 milliseconds following a trigger. 

In this work we also use the \lat Low Energy (\lle) data in the 30\,MeV to 10\,GeV energy range \citep{Pelassa2010}. All \lat \lle data are used within time ranges specified in \citet{grb221009a_fermi_lat_collaboration_2023}. Both \gbm and \lle data are available for download via the public archive at the Fermi Science Support Center (FSSC) website\footnote[3]{\url{https://fermi.gsfc.nasa.gov/ssc/data/}}\texp{,}\footnote[4]{\url{https://heasarc.gsfc.nasa.gov/FTP/fermi/}}. For details about how to properly analyze \gbm data products, see Appendix\,\ref{app:fermi_intervals}. A complementary analysis including proper treatment of the \lat intervals of intense photon fluxes will be reported in \citet{grb221009a_fermi_lat_collaboration_2023}. 

\grb was detected by \gbm beginning at the trigger time (\t0) and lasting until \t0+1467\,s when it was occulted by the Earth. Prior to detection, the position of the GRB was within the field of view of \fermi beginning at \t0$-$2111\,s. \gbm detectors N3, N4, N6, N7, and N8 all had viewing angles within $60\deg$ of the burst at trigger time, but only detectors N4 and N8 stayed within $60\deg$ of the burst throughout the emission episode. These detectors, along with B1, are used throughout this analysis unless otherwise stated.

From the time of detection until Earth occultation, the incident photons hit \fermi at zenith angles ranging from 62$\deg$ to 110$\deg$ with respect to the \lat boresight and azimuth angles ranging from 258$\deg$ to 263$\deg$ (from 12$\deg$ to 7$\deg$ with respect to the BGO endcaps on the BGO detector plane). At such geometries, the incident photons travel through the BGO photomultiplier tubes and their housings, which are not adequately modeled in the detector response at low energies. We therefore omit the BGO data below 400\,keV throughout our analysis. The NaI spectra show deviations between the model and the data below 20\,keV that needs further analysis; these NaI data are therefore omitted from the spectral analysis.

Due to the extraordinarily high photon flux produced by \grb, both \gbm and \lat experienced periods with data issues\footnote[5]{\url{https://fermi.gsfc.nasa.gov/ssc/data/analysis/grb221009a.html}\label{ss:btis}} \citep{gcn32642, gcn32760, gcn32916}. We identify these periods as bad time intervals (BTIs). In \gbm, during these BTIs, the majority of \tte data are unrecoverable due to the summed count rate of all detectors exceeding the 375\,kHz data rate limit of the \gbm high speed science data bus, causing loss of \tte telemetry packets \citep{Meegan2009}. We therefore limit our analysis to the pre-binned 1.024\,s \cspec data within these regions, which are available without data loss, and use additional \lat \lle data, when available.

High counting rates create deadtime which is automatically corrected for in the exposure value in \gbm FITS files using Equation 4.24 in \citet{Knoll2010}. At input count rates above \abt50k counts per second (cps) more complex deadtime effects such as pulse pile-up (PPU) can occur \citep{Meegan2009}. PPU occurs when the overlap between electronic pulses in \gbm causes distortions in both the observed spectral shape and intensity \citep{Chaplin2013, Bhat2014}. This rate was significantly exceeded during the reported \gbm BTIs and cannot be corrected for automatically via the standard method. In order to obtain a full picture of \grb, we perform PPU-correction within the \gbm BTIs using the method described in Section\,\ref{subsec:BTIs}.

%-------------------------------%
%--- bad time intervals ---%
%-------------------------------%

\subsection{Pulse Pile-up Correction} \label{subsec:BTIs}

A number of instruments have successfully applied PPU-correction techniques to a wide range of gamma-ray transients \citep{Mazets1999, Lysenko2019, Mailyan2016, Mailyan2019}. The effects of PPU on \gbm data and its correction have been analytically studied in the work of \cite{Chaplin2013} and correction techniques have been verified through Monte Carlo simulations and lab experiments with radioactive sources \citep{Bhat2014}. This analytical PPU method successfully achieved spectral fits of \gbm terrestrial gamma-ray flashes that could not have been satisfactorily fit without PPU corrections \citep{Mailyan2016, Mailyan2019}. In our work we apply the \gbm PPU-correction technique to the \gbm \cspec data for \grb.

The technique parallels that of a normal spectral analysis with a few additional steps to account for the distorted data. We first assume a parameterized photon model and forward fold it through the \gbm detector response matrix (DRM) to obtain a detector count spectrum. The count spectrum is then adjusted for PPU effects using the analytic method of \citet{Chaplin2013} and an assumed count rate. Since the analytical PPU-correction model includes all deadtime effects, the exposure time in the \gbm FITS files needs to be replaced with the observing time to avoid double counting deadtime. The PPU-adjusted count spectrum is then compared to the observed count spectrum. The assumed photon model parameters and count rate are then iteratively adjusted to maximize the likelihood between the PPU-adjusted counts spectrum and the observed counts spectrum. 

Unlike for the Cs$^{137}$ and Co$^{60}$ sources used when testing a BGO detector in a lab \citep{Bhat2014}, we do not know the true count rate or spectrum of \grb. Additionally, there are some residual uncertainties concerning the reliability of the PPU-correction technique at such extreme counting rates. Since our primary goal is to determine the energetics of this GRB we assume a simple Band function throughout the PPU reconstruction process and do not attempt to improve these fits by adding spectral breaks or additional spectral components. We do not take into account uncertainties associated with our non-optimal choice of spectral shape. For reasons which will be discussed in Section\,\ref{subsec:primary_pulse} and Section\,\ref{sec:energetics}, we consider this technique reliable for determining the total energetics of \grb and note that our results are consistent with those reported in \cite{Frederiks2023}, \cite{Ripa2023}, and \cite{An2023}.
