%%%%%%%%%%%%%%%%%%%%%%%%%%%%%%
%     Temporal Analysis      %
%%%%%%%%%%%%%%%%%%%%%%%%%%%%%%

%-----------------------------Figure Start--------------------------
\begin{figure*}[h!tbp]
    \centering
    \includegraphics[width=\textwidth]{figures/figure_1_top.pdf}
    \includegraphics[width=\textwidth]{figures/figure_1_bottom.pdf}
    \caption{The top plot shows the uncorrected lightcurve of \grb in the 20\,keV to 40\,MeV energy range as seen by the three \gbm detectors with the lowest continuous viewing angles. The uncorrected lightcurve is divided into eight time intervals (I-VIII) differentiated by vertical dashed lines. Beyond region VIII (right of the last vertical dashed line at $t_{0}+1464$\,s) marks the time when \grb was occulted by the Earth. Intervals IV, V, and VI are further subdivided into three sub-intervals shown in the bottom three panels in the same energy range (20\,keV to 40\,MeV). The NaIs, BGOs, and \lle plots show the uncorrected lightcurve of \grb in different energy bands. The two gray vertical shaded regions in the \gbm plots denote the BTIs of \gbm (\t0+219.0-\t0+277.0\,s \& \t0+508.0-\t0+514.0\,s). The red vertical shaded region in the \lle plot denotes the revised BTI of \lat (\t0+217 to \t0+280\,s).}
    \label{fig:main}
\end{figure*}
%-----------------------------Figure End----------------------------

Figure\,\ref{fig:main} shows the lightcurve of \grb as seen by \gbm and \lat with the \gbm BTIs highlighted with gray vertical shaded regions and the \lat BTI highlighted with a red vertical shaded region. Here we use the revised \lat BTI region, the details of which can be found in \citealt{grb221009a_fermi_lat_collaboration_2023}. Pulses are the basic unit of measure for GRB emission \citep{Hakkila2014}. Pulses are simple structures underlying the more complex GRB emission and are often superposed with one another. With this motivation, we separated the lightcurve of \grb into eight distinct emission intervals, and some sub-intervals, based on its morphology:
\begin{enumerate}[label=\Roman*)]
    \item The \enquote{Triggering Pulse} (\t0-1\,s -- \t0+43\,s)
    \item The \enquote{Quiet Period} (\t0+121\,s -- \t0+164\,s)
    \item The beginning of the primary emission which we refer to as the\enquote{Pre-Main Pulse} (\t0+177\,s -- \t0+210\,s)
    \item The \enquote{Primary Pulse} which contains the first \gbm BTI (\t0+210\,s -- \t0+324\,s)
    \begin{enumerate}[label=IV\alph*)]
        \item The interval before the first \gbm BTI
        \item The first \gbm BTI region
        \item The interval after the first \gbm BTI
    \end{enumerate}
    \item The smooth, then variable \enquote{Intra-Pulse Period} (\t0+326\,s -- \t0+483\,s)
    \item The \enquote{Secondary Pulse} which contains the second \gbm BTI (\t0+483\,s -- \t0+546\,s)
    \begin{enumerate}[label=VI\alph*)]
        \item The interval before the second \gbm BTI
        \item The second \gbm BTI region
        \item The interval after the second \gbm BTI
    \end{enumerate}
    \item The \enquote{Final Pulse} (\t0+546\,s -- \t0+597\,s)
    \item The \enquote{Afterglow} (\t0+597\,s -- \t0+1467\,s)
\end{enumerate}
Another motivation for these many intervals and sub-intervals is track the spectrotemporal evolution of this extremely bright GRB which is known to exist from previous GRB observations (e.g., \citealt{Liang1996}). These lightcurve intervals and their corresponding sub-intervals will be referenced throughout this work.

%----------------------------%
%----------- qpo ----------%
%----------------------------%

\subsection{Periodicity Searches} \label{subsec:qpo}

In light of the recent quasi-periodic oscillation (QPO) study of GRB\,211211A reported in \citet{Xiao2022} and of GRB\,910711 and GRB\,931101B reported in \citet{Chirenti2023}, we performed a QPO search on \grb using the high time-resolution \gbm \tte data across a wide range of frequencies using two different methods. 

We separately searched interval I (the triggering pulse), intervals III+IVa (the beginning of the main emission period before the first \gbm BTI), intervals IVc+V+VIa (the emission between the two \gbm BTIs), and intervals VIc+VII+VIII (the end of the main emission period after the second \gbm BTI). Since the loss of \tte packets during the \gbm BTIs creates artificial structure in the lightcurves, we do not consider these regions in our analysis. Visual inspection of sub-interval Vc suggests potentially interesting variability, thus we also search that segment separately.

We search for QPOs using Fourier-based methods (e.g., \citealt{vanderklis1989}) at frequencies $\gtrsim 20 \,\mathrm{Hz}$, where the overall variability of the GRB is relatively unimportant and photon counting noise dominates. We generate periodograms up to $5000\,\mathrm{Hz}$, and look at both linearly and logarithmically binned periodograms. As a threshold for significance, we choose $p < 0.001$ ($\sim3\sigma$), corrected for the number of frequencies searched. We find no credible QPO detection in any of the segments in the white noise-dominated regime above $20\,\mathrm{Hz}$. In all cases the significance remains far below the threshold corrected for the number of trials.

For frequencies $<20\,\mathrm{Hz}$, we follow the formalism in \citet{Huebner2022} and use a Bayes factor comparing a Gaussian process with a covariance function describing aperiodic red noise variability to a Gaussian process with a covariance function describing a combination of aperiodic red noise variability and a QPO. We modelled all segments individually with the Gaussian process, and refer the reader to \citet{Huebner2022} for more details on the analysis, including prior choices. We find no credible detections in intervals I, III+IVa, or VIc+VII+VIII, and focus our attention on region IVc+V+VIa and its sub-interval Vc, where the variability suggests the possible presence of a low-frequency signal. Indeed, in the model including both aperiodic variability and a QPO signal the posterior probability density for the QPO period is well-constrained to $12.34^{+2.12}_{-1.67}\,\mathrm{s}$. The Bayes factor for the two models is,

\begin{equation} \label{eq:bayes_factor}
    \log\p{\mathcal{B}} = \log\p{\frac{\mathcal{L}\p{D | M_1}}{\mathcal{L}\p{D | M_2}}} = 2.21
\end{equation}

\noindent
where $M_1$ is the model with QPO, $M_2$ is the model without QPO, and $D$ is a placeholder for the data, in this case the lightcurve. A Bayes factor of $2.21$ is considered moderate evidence for a QPO, but we caution the reader that Bayes factors are extremely sensitive to prior choices. A more detailed QPO analysis will be deferred to a future study.

%----------------------------%
%----------- mvt ----------%
%----------------------------%

\subsection{Minimum Variability Timescale} \label{subsec:mvt}

GRB lightcurves exhibit variability on various time scales. Internal shocks can produce the observed complex temporal structure provided that the source itself is variable. The minimum variability timescale (MVT; $\Delta t_{\textrm{min}}$) is a measure of the shortest coherent variations of the lightcurve \citep{Bhat2012} and, in turn, can be related to the minimum Lorentz factor of the relativistic shells emitted by the central engine. This measure carries information about the variability timescale of the central engine itself. 

Here we use the statistical method described in \citet{Bhat2013} to estimate the MVTs of a GRB using \gbm \tte data. We first identify the prompt emission region and an equal duration of background region. We then derive a differential of each lightcurve and compute the ratio of the variances of the GRB prompt region to that of the background region. This is repeated by varying the bin widths of the lightcurve starting at sub-millisecond values. This ratio, divided by the bin width, is plotted as a function of bin width. At very fine bin widths this ratio falls monotonically with increasing bin width (1/bin width variation) signifying that at fine bin widths the variations in the background and burst regions are statistically identical. In other words, the signal in the burst lightcurve is indistinguishable from Poissonian fluctuations. At some point, as the bin widths increase, the ratio starts increasing with bin width. The point where the correlation between the ratio and bin width change is defined as the minimum variability timescale. We measure the bin width at this valley by fitting it with a quadratic function with the $\Delta t_{\textrm{min}}$ being the minimum of the quadratic \citep{Bhat2013}.

Usually, a single MVT value is quoted for each burst. However, in the case of \grb, it is possible to derive $\Delta t_{\textrm{min}}$ as a function of time. Figure\,\ref{fig:mvt} shows the evolution of $\Delta t_{\textrm{min}}$ together with the \gbm NaI lightcurve (8 - 1000\,keV). The $\Delta t_{\textrm{min}}$ has been shown to be equal to the shortest rise of a lognormal pulse if one tries to fit the entire lightcurve as a superposition of multiple lognormal pulses \citep{Bhat2013}. With low counting statistics, one would expect the number of detected photons within a certain time bin to anti-correlate with $\Delta t_{\textrm{min}}$, as has been seen in previous GRB observations. However, the MVT is not a measure of the count rate, but rather a measure of how fast the count rate is changing. Therefore, Figure\,\ref{fig:mvt} does not show an anti-correlation between the the count rate and $\Delta t_{\textrm{min}}$, but rather between the variability of the count rate and $\Delta t_{\textrm{min}}$. Meaning $\Delta t_{\textrm{min}}$ is not determined by Poissonian noise, but instead by the intrinsic time variability thanks to the exceptional brightness of this GRB.

During the period affected by PPU, we use the \gbm \ctime data with 64\,ms resolution to visually inspect the lightcurve. We clearly identify pulses that have a rise time of one temporal bin. Based on this observation, and the fact that there is an anti-correlation between the change in count rate and the MVT, we conservatively assume a variability timescale of 0.1\,s during the PPU period.

%-----------------------------Figure Start--------------------------
\begin{figure*}[h!tbp]
    \centering
    \includegraphics[width=\textwidth]{figures/figure_2.pdf}
    \caption{Top: Uncorrected \gbm \cspec data (1.024\,s resolution) in the five NaI detectors with viewing angles to \grb below 60$^{\deg}$ at trigger time and the evolution of the minimum variabtility timescale (MVT; $\Delta t_{\textrm{min}}$) obtained from the \gbm non-BTI regions. The MVT ranges from $\sim$0.01\,s to $\sim$1\,s. The decreasing MVT occurs in regions where the count rate is more variable. Bottom: Uncorrected \gbm \ctime data (0.064\,s resolution) for the same two NaI detectors during the first \gbm BTI region.}
    \label{fig:mvt}
\end{figure*}
%-----------------------------Figure End----------------------------

%-----------------------------%
%-------- duration --------%
%-----------------------------%

\subsection{Duration} \label{subsec:duration}

The standard duration measure of a GRB ($\textrm{T}_{90}$) is defined as the temporal interval containing 90\% of the total time-integrated photon flux in the 50–300\,keV energy range. The standard $\textrm{T}_{90}$ analysis using the \gbm \ctime files without PPU-correction overestimates the duration of \grb, as the count rate of the brightest portions of the lightcurve are decreased in intensity due to PPU and deadtime effects. To mitigate these effects, we used the PPU-corrected \cspec data within the two \gbm BTIs. Because a Band function was assumed in our PPU-correction method (see Section\,\ref{subsec:BTIs}), we performed time-integrated fits with a Band function to each of the non-BTI sub-interval shown in Figure\,\ref{fig:main}. We then integrated each fit from 50-300\,keV to obtain a PPU-corrected $\textrm{T}_{50}$ and $\textrm{T}_{90}$ measurement. For \grb, we estimate $\textrm{T}_{50}=25.6\pm1.0$\,s and $\textrm{T}_{90}=289\pm1$\,s. The duration $\textrm{T}_{90}=289$\,s is longer than 97.5\,\% of all \gbm GRBs and longer than 96.4\,\% of all long \gbm GRBs based on the classification of \citet{vonKienlin2020}.
