%%%%%%%%%%%%%%%%%%%%%%%%%%%%%%
%          Abstract          %
%%%%%%%%%%%%%%%%%%%%%%%%%%%%%%

\begin{abstract}

We report the discovery of \grb, the highest flux gamma-ray burst ever observed by the \fermi Gamma-ray Burst Monitor (\gbm). This GRB has continuous prompt emission lasting more than 600 seconds, afterglow visible in the \gbm energy range (8\,keV--40\,MeV), and total energetics higher than any other burst in the \gbm sample. By using a variety of new and existing analysis techniques we probe the spectral and temporal evolution of \grb. We find no emission prior to the \gbm trigger time (\t0; 2022 October 9 at 13:16:59.99 UTC), indicating that this is the time of prompt emission onset. The triggering pulse exhibits distinct spectral and temporal properties suggestive of shock-breakout with significant emission up to $\sim$15\,MeV. We characterize the onset of external shock at \t0+600\,s and find evidence of a plateau region in the early-afterglow phase which transitions to a slope consistent with \swift-XRT afterglow measurements. We place the total energetics of \grb in context with the rest of the \gbm sample and find that this GRB has the highest total isotropic-equivalent energy ($\textrm{E}_{\gamma,\textrm{iso}}=1.0\times10^{55}$\,erg) and second highest isotropic-equivalent luminosity ($\textrm{L}_{\gamma,\textrm{iso}}=9.9\times10^{53}$\,erg/s) based on redshift of z = 0.151. These extreme energetics are what allowed \gbm to observe the continuously emitting central engine from the beginning of the prompt emission phase through the onset of early afterglow.

\end{abstract}