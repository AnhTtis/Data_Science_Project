%%%%%%%%%%%%%%%%%%%%%%%%%%%%%%
%        Introduction        %
%%%%%%%%%%%%%%%%%%%%%%%%%%%%%%

Gamma-ray bursts (GRBs) are the brightest signatures of stellar deaths in the Universe. They are produced by ultra-relativistic, collimated jets formed via two main progenitor channels: the merging of binary neutron star systems (BNSs; \citealt{Eichler1989, Narayan1992, Abbott2017a}) and the core collapse of rapidly-rotating massive, stripped-envelope stars \citep{Woosley1993,MacFadyen1999}. Prompt GRB emission is highly variable, lasting from tens of milliseconds up to ten thousand seconds, and is thought to arise from either internal shocks or magnetic reconnections within a relativistically expanding jet \citep{Goodman1986, Paczynski1986, Rees1994}. The prompt emission is followed by a longer-lived external shock \enquote{afterglow} component that arises from the interaction of the jet with the surrounding medium.  This afterglow component can be observed across the electromagnetic spectrum and is well-described as non-thermal synchrotron radiation arising from shock-accelerated electrons leading the jet \citep{Meszaros1993,Meszaros+97ag, Sari1998}. 

The Gamma-ray Burst Monitor onboard the \textit{Fermi Gamma-ray Space Telescope} (\gbm) has detected over 3500 GRBs, with approximately 8\% of these bursts also detected by the \fermi Large Area Telescope (\lat) at energies $>$100\,MeV \citep{Ajello2019}. The resulting observations have dramatically expanded our current understanding of broadband emission from GRBs, yet fundamental questions regarding the physical processes and radiation mechanisms that produce the GRB prompt emission still remain.

Emitting the bulk of their energy between 1\,keV and 1\,MeV, GRBs exhibit a range of spectral shapes and can include multiple components, posing significant challenges to attributing the prompt emission to any single physical process or radiation mechanism. The broadband prompt emission is characterized as a featureless non-thermal spectrum that is most commonly modeled with a phenomenological smoothly broken power-law model known as the Band function \citep{Band1993}. This emission is typically attributed to optically thin synchrotron emission, despite long-standing challenges in matching model predictions to the observed spectral shape of the prompt emission \citep{Preece1998}. Detailed time-resolved analysis of \gbm observations have also revealed additional low-energy components superimposed on the broadband spectra, including a quasi-thermal component thought to be due to emission from an optically thick photosphere \citep{Ryde2011, Guiriec2011, Axelsson2012, Guiriec2015}. Studies have shown that some of the difficulties in attributing the shape of the GRB prompt emission to synchrotron emission can be alleviated by the inclusion of these additional low-energy components \citep{Burgess2011, Guiriec2011, Beniamini2013, Oganesyan2017}. 

Deciphering GRB spectra is further complicated by the presence (or lack thereof) of high-energy emission ($\gtrsim$100\,MeV). For GRBs observed by both the \gbm and \lat, an additional high-energy power-law is often required to fit emission extending above $\sim$100\,MeV, and evidence for such a component can often be seen extending down into X-ray energies ($< 20$\,keV). The emergence of this component is typically delayed with respect to the initial keV emission and has been attributed to afterglow emission arising during the prompt phase \citep{Abdo2009, Ackermann_2010, ackermann_2011, Ackermann2014}. A higher energy component has also been observed at TeV energies, with GRB\,190114C representing the first published detection of such photons from a GRB \citep{Acciari2019, MAGIC+21ssc, Ajello2020}. These observations provided the first evidence of possible synchrotron self-Compton (SSC) radiation long predicted to play a role in the high-energy GRB afterglow emission. Subsequent reports of TeV emission with various levels of significance in GRB afterglows, GRB\,160821B \citep{Acciari2021}, GRB\,180720B \citep{Abdalla2019}, GRB\,190829A \citep{Abdalla2021}, GRB\,201015A \citep{gcn28659}, and GRB\,201216C \citep{Fukami2021} have shown that this Very High Energy (VHE) component may common in GRB emission.

 Here we present \gbm observations of \grb, by far the brightest GRB ever detected by \gbm, and for which there is an abundance of photons to accommodate the most thorough and exhaustive investigation of its spectral and temporal properties. In Section\,\ref{sec:discovery} we discuss the timeline of discovery for \grb and Section\,\ref{sec:data} explains how to properly use the available \gbm data products, given the unique challenges of analyzing this event. Section\,\ref{sec:temporal_analysis} provides an overview of the temporal structure within the prompt emission followed by an analysis of the spectral evolution in Section\,\ref{sec:prompt_emission_phases}. The analysis results of the afterglow period are presented separately in Section\,\ref{sec:afterglow}. We then discuss the energetics associated with \grb in Section\,\ref{sec:energetics} before concluding in Section\,\ref{sec:discussion} where we put this event in context with other extraordinarily bright GRBs. Finally, we provide a summary of our results in Section\,\ref{sec:summary}.

