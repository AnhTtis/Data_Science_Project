%%%%%%%%%%%%%%%%%%%%%%%%%%%%%%
%          Discovery         %
%%%%%%%%%%%%%%%%%%%%%%%%%%%%%%

On 2022 October 9 at 13:16:59.99 UTC (\t0), the \gbm flight software triggered on \grb. The \gbm gamma-ray burst coordinates network (GCN)\footnote[1]{\url{https://gcn.gsfc.nasa.gov/gcn3_archive.html}\label{ss:gcn}} \enquote{Trigger Notice} was automatically distributed, but due to issues in the ground segment, further notices containing classification and localization information were not disseminated. At the time of the \gbm detection, the GRB was occulted by the Earth for the Neil Gehrels Swift Observatory (\swift). When the source position first became visible, \swift was transiting the South Atlantic Anomaly (SAA) and was unable to collect data, as reported by \cite{Williams2023}. At 14:10:17 UTC \grb was detected by the \swift Burst Alert Telescope (BAT) and observations were performed by the \swift X-ray Telescope (XRT) and Ultra-violet Optical Telescope (UVOT) instruments, yielding a best localization at RA(J2000) = 19\texp{h} 13\texp{m} 3.48\texp{s} ($288.26452\deg$), DEC(J2000) = +19$^{\deg}$ 46' 24.6" ($19.77350\deg$) with a 90\%-confidence error radius of 0.61" \citep{gcn32632}. Due to the particularly bright signal and a localization within the Galactic plane, the \swift team initially classified the event as having a Galactic origin.

The \swift GCN notices initiated the automated \lat analysis pipeline which searched and identified a bright high-energy signal within the \swift interval. This in turn initiated a deeper inspection of the \gbm data which appeared as one of the brightest transients ever discovered by \gbm. Private communication between the \gbm team, the \lat Collaboration, and the \swift team lead to the conclusion that the location of all of these transients were consistent within the respective positional uncertainties of each instrument. An initial picture of identifying the same, incredibly bright GRB, with the prompt emission being detected by \fermi and the afterglow being detected by \swift an hour later, became the working hypothesis. GCN circulars reporting the brightest GRB ever seen were promptly sent by all parties involved \citep{gcn32635, gcn32636, gcn32637} encouraging follow-up observations from the community. Prompt signal association was later confirmed by the InterPlanetary Network \citep{gcn32641}.

These initial GCN circulars were followed by more than 100 additional circulars, and some Astronomer's Telegram notices (ATels)\footnote[2]{\url{https://astronomerstelegram.org/}\label{ss:atel}}, reporting detections across the electromagnetic spectrum and upper limits from non-electromagnetic messengers. Some highlights from the prompt emission include:
\bullets{
    The first detection of TeV energy photons during the prompt emission by LHAASO (up to 18\,TeV$\textrm{;}$ \citealt{gcn32677});
    A redshift of z = 0.151 reported by the VLT \citep{gcn32648};
    The identification of rings from dust echoes from the \swift-XRT \citep{atel15661};
    Polarization observations from IXPE \citep{gcn32690, gcn32754};
    Unsaturated observations from a Solar instrument (STIX) \citep{gcn32661};
    Observations from GRB CubeSats like GRBAlpha and SIRI-2 \citep{gcn32685, gcn32746};
    Non-detections of neutrinos in both IceCube and KM3NeT \citep{gcn32665, gcn32741};
    Measurable disturbances in Earth's ionosphere \citep{gcn32744, gcn32745}
}
The Hubble Space Telescope performed photometry \citep{gcn32921} and the James Webb Space Telescope performed spectroscopy \citep{gcn32821, JWST_221009A, fulton_2023} of the potential supernova component and emission was still detectable at radio frequencies up to 4 months later \citep{gcn33305}.

