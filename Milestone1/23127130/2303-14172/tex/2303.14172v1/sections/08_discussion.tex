%%%%%%%%%%%%%%%%%%%%%%%%%%%%%%
%         Discussion         %
%%%%%%%%%%%%%%%%%%%%%%%%%%%%%%

%--------------------------% 
%----- shock breakout -----%
%--------------------------% 
\subsection{Shock Breakout} \label{subsec:shock_breakout}

Just as with shock breakout and shock interaction for core-collapse supernovae, if the early emission is thermal, it can be used to probe characteristics of the progenitor and its immediate surroundings as well as the structure of the outflow. In the context of gamma-ray bursts, thermal emission can probe the progenitor-star photosphere (which is set by the mass loss and stellar radius), inhomogeneities of the mass-loss and structure of the jet, and its cocoon. For this paper, we focus on more fundamental aspects behind a thermal component to determine whether it is a reasonable explanation of the observed emission, deferring a detailed comparison of the data to models for a later paper.

If we assume the observed triggering emission is produced by the Lorentz-boosted thermal emission of shock breakout, the limits and shape of the emission can be used to constrain the properties of the shock as it emerges from the star. In the strong shock limit for a highly-relativistic gas, the pressure of the shock ($P_{\rm shock}$) is,

\begin{equation} \label{eq:P_shock}
    P_{\rm shock} \approx \Gamma^2 \rho_{\rm CSM} c^2
\end{equation}

\noindent
where $\Gamma$ is the Lorentz factor, $\rho_{\rm CSM}$ is the density in the region of shock breakout (in the wind of the massive star), and $c$ is the speed of light. Assuming the pressure is radiation-dominated, the corresponding temperature ($T_{\rm shock}$) of the emitting gas in the gas comoving frame is,

\begin{equation} \label{eq:T_shock}
    T_{\rm shock} \approx 1.9 (\Gamma/100)^{0.5} (\rho_{\rm CSM}/10^{-10}\,g\,cm^{-3})^{0.25} \; \textrm{keV}
\end{equation}

\noindent
and the corresponding peak energy of the emitted photons ($\nu_{\rm peak}$) in the observer frame is,

\begin{equation} \label{eq:nu_shock}
    \nu_{\rm peak} \approx 1 (\Gamma/100)^{1.5} (\rho_{\rm CSM}/10^{-10}\,g\,cm^{-3})^{0.25} \; \textrm{MeV}
\end{equation}

In the thermal shock breakout paradigm, the broad range and relatively flat spectra for the prompt emission requires a distribution of Lorentz factors. The observed peak emission around 15\,MeV places strong constraints on the upper limit of the Lorentz factor, corresponding to peak Lorentz factors lying between 300--1000, with corresponding densities of $70$--$0.05 \times 10^{-10} \, {\rm g \, cm^{-3}}$.

We consider if this triggering pulse would be identified in other GRBs. Some bright long GRBs have weak trigger intervals followed by quiescence before the main emission episode, but most do not. The 1.024\,s peak-flux interval of this pulse would trigger \gbm out to z$\approx$1.3. Approximately half of \gbm GRBs with measured redshifts are within this range. Thus, a similar pulse should have been recovered in other bursts. The particularly high $\Gamma$ for this burst would produce a higher luminosity shock breakout, which may explain the lack of identification in other collapsars.

%--------------------------% 
%----- lorentz factor -----%
%--------------------------% 
\subsection{Lorentz Factor} \label{subsec:lorentz_factor}

With the requirement that the prompt emission site be optically thin, we can derive lower limits on the Lorentz factor. For this method we usually use the minimum variability timescale in conjunction with the spectral shape. Here we can only derive the MVT for the time intervals without PPU. The spectrum within the \gbm BTIs however yields stronger constraints on $\Gamma$ even with conservative assumptions on the minimum variability timescale. Based on Figure\,\ref{fig:mvt} and the discussions in Section\,\ref{subsec:mvt}, we assume a conservative variability timescale estimate of $\Delta t_{\rm var}=0.1 $\,s.

The Lorentz factor limits involve pair-production ($\Gamma_{\rm pp}$) for the highest energy photons interacting with the \gbm prompt spectrum and the overall requirement that the \gbm spectrum be optically thin ($\Gamma_{\rm ot}$; \citealt{lithwick01}). \lat observed a 99.3 GeV photon at \t0+240\,s \citep{gcn32658}. Assuming the \gbm emission emanates from the same volume, the requirement that such a photon can escape without producing $e^{\pm}$ pairs yields a Lorentz factor constraint of $\Gamma_{\rm pp, min}\gtrsim 1560$. The calculation of $\Gamma_{\rm min}$ by \citealt{Lithwick2001} is done under the assumption of a single zone. If we consider a more realistic situation, taking into account the angular, temporal and spatial dependence of the radiation field, our values of $\Gamma_{\rm min}$ could be lower by a factor of \abt2 (i.e., $\Gamma_{\rm pp, min}\gtrsim 780$) \citep{Hascoet+12gamgam,Gill_2018, Vianello_2018, Arimoto_2020}.

The requirement that the \gbm emission is optically thin, given the spectrum in the brightest interval, yields a limit of $\Gamma_{\rm ot, min}\gtrsim 1040$. If we instead use $\Delta t_{\rm var}=0.05 $\,s, which is still reasonable given the MVT values surrounding region IV, we find a Lorentz factor limit of $\Gamma_{\rm ot, min}\gtrsim 1470$. Although both MVT-derived Lorentz factor lower limits fall below the single zone pair-production Lorentz factor lower limit, the two methods independently produce consistent results.

We can also place constraints on the Lorentz factor from the triggering pulse. Based on our COMP function spectral fits in region Ia ($\alpha=-1.69\pm0.01$, $E_{\rm peak}=3.98\pm0.37$\,MeV, and energy flux=$(1.98\pm0.03)\times 10^{-6}$ erg cm$^{-2}$ s$^{-1}$ in the 10-1000 keV range) and requiring the source be optically thin, we find $\Gamma_{\rm trig, min}\gtrsim 188$. Although less constraining, this value is consistent with the value of 300-1000 derived via a shock-breakout assumption. There are no photons in the GeV range that are unambiguously associated with the GRB during this time interval, so the pair-creation Lorentz factor limit is not constraining in this case.

The peak of the afterglow emission denotes the beginning of the external shock deceleration time. By identifying this peak for ISM and wind-type external media as $t^{\rm ag, ISM}_{peak}\gtrsim t_{0}+(140\pm1.5)$\,s and $t^{\rm ag, wind}_{peak}\gtrsim t_{0}+(120\pm6.5)$\,s respectively we can derive upper limits for the Lorentz factor of the afterglow via,

\begin{equation} \label{eq:Lorentz_factor_general}
    \Gamma_{\rm ag} = \pp{\p{\frac{(17-4s)}{16\pi(4-s)}}\p{\frac{E_{k}}{n_0 m_{\rm p}c^{5-s}}}}^{\frac{1}{8-2s}}\p{\frac{t_{\rm peak}}{(1+z)}}^{-\frac{3-s}{8-2s}}
\end{equation}

\noindent
where $s=0$ assumes an ISM external density profile and $s=2$ assumes a wind-type external medium \citep{GammaNappo,Ghirlanda+18Lorentz}. For both cases we use the VLT reported redshift of z = 0.151 \citep{gcn32648} and assume the kinetic energy, $E_k$ is approximately the isotropic-equivalent gamma-ray energy ($E_{k}\approx E_{iso}$).

Assuming ISM ($s=0$) we have:

\begin{equation} \label{eq:Lorentz_factor_ISM}
    \Gamma_{\rm ag, ISM} \gtrsim 260 \left(\frac{E_{k,55}}{n/1~ {\rm cm}^{-3}}\right)^{1/8} \left(\frac{t_{\rm peak}}{120 ~{\rm s} \times 1.151}\right)^{-3/8}
\end{equation}
Using the $t_{\rm peak} = 105 $ s value, increases slightly to $\Gamma_{\rm ag, ISM}\gtrsim 270$ (all scaling parameters are the same)

For the wind-type external medium ($s=2$) the density parameter $n= 3\times 10^{35}~A_\star~ {\rm cm}^{-1}$ \citep{grb221009a_fermi_lat_collaboration_2023} which gives us,

\begin{equation} \label{eq:Lorentz_factor_wind}
    \Gamma_{\rm ag, wind} \gtrsim 282 \left(\frac{E_{k,55}}{A_{\star,-1}}\right)^{1/4} \left(\frac{t_{\rm peak}}{ 140 ~{\rm s} \times 1.151}\right)^{-1/4}
\end{equation}

For a spherically emitting shell, the MVT relates the radius of the shell to the Lorentz factor via,

\begin{equation} \label{eq:min_radius}
    \textrm{R} \sim \Delta t_{\textrm{min}} \p{\frac{\Gamma^{2} c}{\p{1+z}}}
\end{equation}

\noindent
where R is the radius of the spherically emitting shell, $\Gamma$ is the Lorentz factor, $c$ is the speed of light, and $z$ is the measured redshift. Along these lines, the MVT and the bulk Lorentz factor can be used to place limits on emitting shell radius, assuming a singular emitting region. In interval Ia, the average value of MVT is \abt0.1\,s. Using the afterglow-derived bulk Lorentz factor for the ISM ($\Gamma_{\rm ag, ISM} \gtrsim 260$) and the wind-type external medium ($\Gamma_{\rm ag, wind} \gtrsim 282$) as a proxy for the bulk Lorentz factor in this region gives us an estimate on the radius of the emitting star. We find $R_{\star, ISM} \gtrsim 1.7 \times 10^{14}$\,cm and $R_{\star, wind} \gtrsim 2.0 \times 10^{14}$\,cm for the ISM and wind-type media respectively. Both of these values are roughly consistent with the radius of the outer wind from the Wolf-Rayet progenitor star ($R_{\star} > 1 \times 10^{13}$\,cm) \citep{Crowther2007}.

Using this same equation, the lower limit of the shock breakout derived triggering pulse Lorentz factor ($\Gamma=300$), and a delay time of \abt220\,s between regions I and III, we get an internal dissipation radius of \abt$6\times10^{17}$\,cm. This value is larger than is typically discussed but could be consistent with the ICMART model of $10^{16}$\,cm, which is consistent with the non-detection of neutrinos \citep{gcn32665, gcn32741} and a Poynting flux dominted jet \citep{Zhang2010}.

%--------------------------% 
%------- neutrinos --------%
%--------------------------% 
% \subsection{Neutrinos} \label{subsec:neutrinos}


%--------------------------% 
%---------- SSC -----------%
%--------------------------% 
% \subsection{Synchrotron Self-Compton} \label{subsec:ssc}

