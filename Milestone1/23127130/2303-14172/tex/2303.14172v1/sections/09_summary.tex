%%%%%%%%%%%%%%%%%%%%%%%%%%%%%%
%           Summary          %
%%%%%%%%%%%%%%%%%%%%%%%%%%%%%%

Our dedicated search suggests that there is no emission from the central engine prior to the \gbm trigger time. Weak emission from the triggering pulse (region I) can be localized to \grb for up to 100\,s before returning to background. We find another weakly emitting pulse (region II) which we were able to localize to \grb just before the bulk of the main emission began. The central engine is then continuously active for \abt600\,s (region III through region VII) when the early onset of GRB afterglow is first observed (region VIII).
 
The first \abt8\,s of the triggering pulse has a peak energy (\Epk) of 4\,MeV with the highest energy photons arriving first. The entire triggering pulse has spectrotemporal properties best characterized by the Lorentz-boosted thermal emission of shock breakout. Using the peak energy of the assumed shock breakout emission we are able to place the Lorentz factor ($\Gamma$) of this pulse between 300 and 1000 with corresponding shock densities between 70 and 0.05$\times 10^{-10}\, {\rm g \, cm^{-3}}$.

After correcting for the effects of PPU in the \gbm data we were able to characterize the total energetics of this burst. With a total isotropic-equivalent energy of $\textrm{E}_{\gamma,\textrm{iso}}=1\times10^{55}$\,erg and an isotropic-equivalent luminosity of $\textrm{L}_{\gamma,\textrm{iso}}=9.9\times10^{53}$\,erg s$^{-1}$, \grb is the most energetic and second most luminous in the \gbm sample. Assuming a MVT of 0.05\,s we place constraints on the Lorentz factor during the brightest interval via pair-creation using the \lat reported 99.3\,GeV photon and by assuming an optically thin environment. These values are $\Gamma_{\rm min, A}\gtrsim 1560$ and $\Gamma_{\rm min, A}\gtrsim 1470$ respectively (without the factor of 2 correction).

By comparing measurements with those observed by \swift-XRT we are able to characterize the onset of the early-afterglow period. We place lower limits on the afterglow peak time assuming both ISM and wind-type external media of $t^{\rm ag, ISM}_{peak}\gtrsim t_{0}+(140\pm1.5)$\,s and $t^{\rm ag, wind}_{peak}\gtrsim t_{0}+(120\pm6.5)$\,s respectively. We additionally find an early plateau region with a slope $\alpha^{ag}_{decay}=-0.82\pm0.03$ which gradually steepens to the observed \swift-XRT slope of $\alpha_s=-1.5$ at \t0$>$1400\,s. Again assuming both an ISM external medium and a wind-type external medium we calculate the Lorentz factor of the external shock as $\Gamma_{\rm ag, ISM} \gtrsim 260$ and $\Gamma_{\rm ag, wind} \gtrsim 282$ respectively.

Comparing the Lorentz factor of the prompt emission ($\Gamma_{\rm pr}\sim1560$; or $\Gamma_{\rm pr}\sim780$ after correcting for a factor of 2) to that of the afterglow ($\Gamma_{\rm ag, ISM}\sim260$ or $\Gamma_{\rm ag, wind}\sim282$) yields a deceleration of $\Delta\Gamma_{\rm ISM}\sim1300$ and $\Delta\Gamma_{\rm wind}\sim1278$ (or $\Delta\Gamma_{\rm ISM}\sim520$ and $\Delta\Gamma_{\rm wind}\sim498$) over \abt380\,s. This rapid deceleration is suggestive of a relativistic reverse-shock (i.e., a \enquote{thick shell}) and is consistent with the deceleration time not exceeding the prompt GRB duration.

Taken altogether, the set of measurements presented here provide an unrivaled probe into the entirety of \grb's central engine emission process. And while extremely energetic GRBs like \grb may be rare in their occurrence, this detection joins an ever-growing list of groundbreaking GRB observations stretching all the way back to their discovery in 1967.

%--------------------------% 
%------- dedication -------%
%--------------------------%

\vspace{5mm}
\noindent We dedicate this paper to the memory of William \enquote{Bill} Paciesas who passed away in June 2022. Bill was a co-investigator of BATSE and, for a time, the Principal Investigator of \gbm, but he was arguably more well known by the high-energy astrophysics community for his punny humor and beer connoisseurship. With \grb being the Brightest Event Ever Recorded (BEER) and Bill not here to share it with us; this BEER's for you.