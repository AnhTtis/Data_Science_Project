%%%%%%%%%%%%%%%%%%%%%%%%%%%%%%
%         Afterglow          %
%%%%%%%%%%%%%%%%%%%%%%%%%%%%%%

In the case of some GRBs the afterglow is so bright it can contribute detectable flux in the \gbm band pass \citep{Giblin+99afterglow, connaughton02afterglow}. As seen in GRB\,190114C and discussed in Section\,\ref{subsec:primary_pulse}, this afterglow flux can also overlap with the highly variable prompt emission. For \grb, the last discernible pulse (region VII) peaks around \t0+575\,s and is followed by a long, smooth decay period (region VIII). This decay period having no appreciable variability (see Figure\,\ref{fig:mvt}) is consistent with an afterglow origin. We define the end of this period to be \t0+1460\,s, the time when \grb is occulted by Earth for \fermi. 

We temporally bin the lightcurve from \t0+510\,s (small bump in region VIc) to the end of region VIII, requiring a signal to noise ratio of 80 to ensure an adequate amount of signal. We then fit the spectrum in each resulting interval with a Band function. Next, we extrapolated the spectrum down to 10\,keV (Figure\,\ref{fig:afterglow}, blue) for comparison with \swift-XRT \citep{Williams2023}. 

The first possibility constrain the peak of the afterglow, we fit the lightcurve after the last bright pulse (VI) using two prompt emission lightcurve pulses with \citet{Norris2005} models (Figure\,\ref{fig:afterglow}, green lines) and the afterglow region with a broken power-law (Figure\,\ref{fig:afterglow}, yellow line).  The gamma-ray afterglow lightcurve will rise like a power law of index 3 (ISM) or 1/2 (wind), then decay as a power law (different indices for rising are possible depending on the ordering of the characteristic frequencies). We only detect the late decay and the peak of the afterglow is hidden under prompt emission components. The broken power-law  (Figure\,\ref{fig:afterglow}, yellow dots) peaks at $t^{\rm ag, ISM}_{peak}\gtrsim t_{\rm ref}+(140\pm2)$\,s and $t^{\rm ag, wind}_{peak}\gtrsim t_{\rm ref}+(120\pm6)$\,s with a decay slope of $\alpha^{\rm ag}_{decay}=-0.82\pm0.03$ ($t_{ref}=510 ~s$). The slope is relatively shallow compared to normal afterglow decay. We tentatively identify $t_{\rm peak}$ as the peak of onset of the solely-afterglow emission and assume the true afterglow peak is overwhelmed by the prompt emission. 

Additionally, the slope ($\alpha^{\rm ag}_{s}$) of the following afterglow region resembles a plateau phase with mean index $\overline{\alpha^{\rm ag}_{s}}=-0.6$ and standard deviation $\sigma_{\alpha^{\rm ag}_{s}}=0.4$ \citep{Grupe+13swiftcat}. The later emission ($t>t_{0}+1400$\,s) \swift-XRT slope of $\alpha_s=-1.514 \pm0.003$\footnote[6]{\url{https://www.swift.ac.uk/xrt_live_cat/01126853/}} agrees with the expected value of the afterglow ($\overline{\alpha_s}=-1.5$, $\sigma_{\alpha_s} =0.6$; \citealt{Grupe+13swiftcat}).


We obtain different constrain for the peak of the afterglow by extrapolating the smooth emission of phase VIII back in time. We fixed the afterglow lightcurve decay index to our measured $\alpha^{\rm ag}_{decay}$ value, set our reference time to the start of the bright phase at $t_{\rm ref} = t_{0}+175$\,s. We use $\alpha^{\rm ag}_{rise} = 3$ because it gives the tightest constraints on the peak time. Requiring that the extrapolated afterglow flux not exceed the prompt emission flux, we place a limit of $t^{\rm ag, 2}_{peak}\gtrsim t_{\rm ref}+105 $\,s. 

\gbm can also measure source fluxes using an Earth occultation technique, modeling the change in count rate when a source of interest goes behind the Earth or emerges from behind the Earth \citep{WilsonHodge2012}. Using the known \swift-XRT sky localization for \grb, Earth occultation times were first estimated for the time of 50\% transmission at 100\,keV. A 240\,s data window surrounding each occultation step was fitted using a model comprised of a quadratic background and source terms; in this case for \grb and Cygnus X-1. Independent fits were performed for each energy channel and each detector viewing \grb within 60 degrees from the detector normal. The source terms consist of an energy dependent model of the atmospheric transmission convolved with the detector response and an assumed spectral model. Two fixed spectral models were tested, a single power-law with a photon index of -2, and a Band function with $\alpha=-0.85$, $\beta=-2.0$, and $E_{\textrm{peak}}=21.0$\,keV based on the afterglow analysis. Because each energy channel is fitted independently, these models only apply across a single \gbm \ctime or combined set of \gbm \cspec channels, so results between the two models were consistent. The four Earth occultation steps with excesses above background starting 1470\,s after the trigger are shown in Figure\,\ref{fig:afterglow} as green stars. The afterglow fluxes estimated from the Earth occultation steps are consistent with the temporal decay observed by \swift-XRT.

The Earth occultation analysis for \grb differs slightly from that described in \citet{WilsonHodge2012} because the usual focus of the \gbm occultation technique is on the study of longer-term variations (days to years) rather than individual steps. Data filtering, normally used to reject highly variable backgrounds, was omitted for these fits because the time-interval of interest was getting rejected due to the brightness of \grb. Only Cygnus X-1, the brightest potentially interfering source, was fitted within the time windows rather than the full catalog of flaring sources in \citet{WilsonHodge2012} to avoid rejecting the data of interest. 
