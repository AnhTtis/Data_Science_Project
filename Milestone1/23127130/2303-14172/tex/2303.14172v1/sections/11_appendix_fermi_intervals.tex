%%%%%%%%%%%%%%%%%%%%%%%%%%%%%%
%  Appendix Fermi Intervals  %
%%%%%%%%%%%%%%%%%%%%%%%%%%%%%%

\fermi provides spectral coverage over seven orders of magnitude in energy for short duration transients, beginning around 8\,keV in \gbm and reaching into the hundreds of GeV with the \lat. Time-resolved analyses using \fermi data can map the evolution of distinct spectral components in time. For most GRBs, the \gbm \tte data are binned to match the intervals of interest determined by burst structure or the counts in a given data type. Due to the various data issues caused by \grb, this burst requires special care when determining the appropriate temporal intervals to use. 

The selections must first account for the burst duration exceeding the pointing stability timescale of \fermi. Visibility by \gbm to the source position begins 2111\,s before trigger time (\t0) and lasts until \t0+1550\,s when the region is occulted by the Earth. The visibility for the \lat is limited by the \lat field of view. The source leaves the \lat primary field of view at \t0+380\,s and exits the wider \lle data field of view at \t0+600\,s. 

Data selections must also account for the distinct intervals of \gbm and \lat with and without pulse pile-up, and the capability to correct for pulse pile-up effects. In \gbm these intervals are from approximately \t0+219\,s to \t0+277\,s and from approximately \t0+508\,s to \t0+514\,s. Within these time intervals \tte data were lost due to the limited bandwidth of the \gbm electronics and are irrecoverable. \ctime and \cspec data within these intervals exist but are severely effected by pulse pile-up and are unreliable in their current, uncorrected state. For regions surrounding the \gbm intervals with pulse pile-up, the temporal boundaries to switch between \tte and binned \cspec (or \ctime) data are based on the temporal binning and phase of the binned data. This ensures the transition occurs on the edges of the binned data. Using the bin edges that fully cover the intervals with pulse pile-up gives selections times from \t0+218.501\,s to \t0+277.894\,s and from \t0+507.275\,s to \t0+514.443\,s.

Throughout the duration of \grb both the standard \lat data product and the \lat \lle data experienced data issues within certain time regions. Although we leave the details of these effected time regions to the \lat paper (CITE LAT PAPER) we would like to mention how to properly handle using both data products simultaneously. \lle data are generated at 0.1\,s and 1.0\,s temporal resolution with their phase set to the bin boundary at \gbm trigger time. For \gbm intervals without pulse pile-up, the \gbm \tte data can be binned to match the edges of the \lle data. This is not possible during \gbm intervals with pulse pile-up because \tte data are irrecoverable during these times. Therefore, pulse pile-up corrected \cspec or \ctime data must be used. For regions when the \gbm \cspec or \ctime data are out of phase with the \lle data, the \lle data must be re-binned to be in-phase with the \gbm binned data. A similar re-binning procedure must also be done over the edges of these time intervals with either \tte or \lle data to handle the transitions both in and out of the \gbm intervals with pulse pile-up.

\begin{table}[h!tbp]
\begin{tabular}{cccc}
\hline
\hline
$t_{\rm start}$ & $t_{\rm stop}$ & \tte & \cspec (or \ctime) \\
\hline
-10.000 & 210.000  & Y & Y       \\
210.000 & 218.501  & Y & Y       \\
218.501 & 277.894  &   & Y$^{*}$ \\
277.894 & 290.000  & Y & Y       \\
290.000 & 380.000  & Y & Y       \\
380.000 & 507.275  & Y & Y       \\
507.275 & 514.443  &   & Y$^{*}$ \\
514.443 & 600.000  & Y & Y       \\
600.000 & 1500.000 & Y & Y       \\
\hline
\hline
\end{tabular}
\caption{The \gbm data products valid within a given time interval. All time range values are relative to the \gbm trigger. The data types with an asterisk represent \gbm data types with pulse pile-up and should only be used for scientific interpretation after correcting for pulse pile-up effects.}
\end{table}












% \begin{table}[h!tbp]
% \begin{tabular}{cccccc}
% \hline
% \hline
% Time Range (s) &  & Valid Data &  &  \\
% $t_{\rm start}$ to $t_{\rm stop}$ & \tte & \cspec (or \ctime) & \lle & \lat \\
% \hline
% -10.000 to 210.000  & Y & Y  & Y & Y \\
% 210.000 to 218.501  & Y & Y  &   &   \\
% 218.501 to 277.894  &   & Y$^{*}$ &   &   \\
% 277.894 to 290.000  & Y & Y  &   &   \\
% 290.000 to 380.000  & Y & Y  & Y & Y \\
% 380.000 to 507.275  & Y & Y  & Y &   \\
% 507.275 to 514.443  &   & Y$^{*}$ & Y &   \\
% 514.443 to 600.000  & Y & Y  & Y &   \\
% 600.000 to 1500.000 & Y & Y  &   &   \\
% \hline
% \hline
% \end{tabular}
% \caption{The \fermi data products valid within a given time interval. All time range values are relative to the \gbm trigger. The data types with an asterisk represent \gbm data types with pulse pile-up and should only be used for scientific interpretation after correcting for pulse pile-up effects.}
% \end{table}
