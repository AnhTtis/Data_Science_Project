age intensity. A more physically correct model should be
bp = f(c
′
p ∗ h), where c
′
indicates the scene irradiance
and f(·) is the camera response function (CRF) that maps
the scene irradiance to image intensity. A nonlinear CRF
will increase the complexity of the blur kernel and make
the learning of DSK difficult if the linear model in Eq. (4) is
used, especially in high contrast regions [46]. To compensate for the nonlinear CRF, we assume that our sharp NeRF
predicts colors in linear space and adopt a simple gamma
correction function in the final output:
 \mathbf {b}_\mathbf {p} = g(\sum _{\mathbf {q}\in \mathcal {N}(\mathbf {p})} w_\mathbf {q} \mathbf {c}'_\mathbf {q}), \label {eq:gamma+spasekernel} (7)
where g(c
′
) = c
′
1
2.2 is the gamma correction function.
More complex CRFs could be used to model real world
cameras, such as pre-calibrated CRFs, or jointly optimizing
the CRFs during training. But we find this simple scheme is
enough to compensate the nonlinearity in the imaging process and improve the quality. More discussions about modeling the CRF can be found in supplementary material.
