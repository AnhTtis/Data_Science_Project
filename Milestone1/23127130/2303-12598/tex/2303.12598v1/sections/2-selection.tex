\section{Candidate selection}
\label{sec:method}

\subsection{Sample description} 
\label{subsec:sample}

LoTSS DR2 comprises 4,396,228 radio sources spanning over 5600\,deg$^2$ of the northern sky.
LoTSS operates at a central frequency of 144\,MHz with 48\,MHz of bandwidth ($120-168$\,MHz).
The survey has a  $\sim6\arcsec\times({\small \frac{144\,\rm{MHz}}{v}})$ angular resolution and a median root mean square (rms) sensitivity of about $80\,\rm{\upmu Jy/beam}$.
Furthermore, with a $0\,\farcs2$ astrometric uncertainty\footnote{Namely, the uncertainty tying the LoTSS radio frame to the PanSTARRS~\citep[PS1][]{Chambers2016arXiv161205560C} frame, along with the formal error to evaluate the source centroid, given by $\frac{6\arcsec}{\rm{S/N}}$, where S/N is the signal-to-noise ratio. For $\gtrsim$mJy level sources detected at $\gtrsim10\sigma$, the latter is dominant.} for sources brighter than 20\,mJy, comparable to optical surveys, and a 90\% point source completeness for sources $\geq0.8$\,mJy/beam, LoTSS DR2 represents an excellent catalogue for our study.

We cross-match LoTSS DR2 to the \clu catalog---a compilation of all known galaxies out to 200\,Mpc~\citep{Cook2019ApJ...880....7C}.
CLU is a photometric survey carried out with four narrow-band H$\upalpha$ filters corresponding to redshifts up to $z=0.0471$.
The first H$\upalpha$ filter is centered on rest-frame H$\upalpha$, while the last filter's full width at half maximum extends to $200\,{\rm Mpc}$.
\clu is an extension to CLU that comprises 271,867 sources\footnote{\clu is compilation from existing galaxy databases [NASA/IPAC Extragalactic Database (NED, \url{https://ned.ipac.caltech.edu}), Hyperleda, \url{http://leda.univ-lyon1.fr}, Extragalactic Distance Database \url{http://edd.ifa.hawaii.edu}, the Sloan Digital Sky Survey DR12, the 2dF Galaxy Redshift Survey, and The Arecibo Legacy Fast ALFA (ALFALFA)] aimed at providing the most complete list of galaxies with measured distances in the LIGO sensitivity volume.} of which 95,047 fall within the LoTSS DR2 footprint. 
For brevity, we use the terms CLU and \clu interchangeably throughout the paper. 
Distances based on Tully–Fischer methods were favored over kinematic (i.e., redshift) distances; however, the majority of the distances are based upon redshift information. 
For cases where neither Tully-Fischer nor redshift was available, a distance based on the H$\upalpha$ filters is provided. 

In addition to distances, \clu also contains compiled photometric information. 
In particular, sources have been cross-matched to within 4\arcsec~with : 
i. Sloan Digital Sky Survey ({\em SDSS}) data release 12~\citep{Alam2015ApJS..219...12A} for optical fluxes; 
ii. {\em GALEX} all-sky~\citep{Martin2005ApJ...619L...1M, Bianchi2014AdSpR..53..900B} for far- and near-ultraviolet (FUV, NUV) Kron fluxes~\citep{Kron1980ApJS...43..305K}; and
iii. Wide-field Infrared Survey Explorer~\citep[WISE;][]{Wright2010AJ....140.1868W} for mid-infrared fluxes.
Additionally, \clu sources these ancillary data to cull contaminants (i.e. bright stars, high-redshift sources with emission lines shifted to within CLU's H$\upalpha$ bands) and measure several physical properties of galaxies such as the stellar mass, and star formation rate (SFR).

We select sources in the mass range corresponding to dwarf galaxies ($10^7 \leq M_*/M_\odot \leq 3\times10^9$), leaving 31,190 sources.
We finally keep only galaxies with a valid SFR measurement, leaving 18,159 galaxies that form our parent sample.
With regards to the LoTSS DR2 catalogue, we select sources with a peak brightness $\ge0.8$\,mJy/beam, for a total of 2,622,903 sources.
Furthermore, we constrain the matched sample to reject extended sources using the $R_{99.9}$ compactness criterion~\citep[see][Equation~2]{Shimwell2022A&A...659A...1S}, where the ratio of the natural logarithm of the integrated flux density ($S_I$) to peak brightness ($S_P$) is less or equal than the envelope that encompasses the 99.9 percentile of the $S_I/S_P$ distribution---leaving 2,275,400 sources.
Finally, we constrain sources to those fitted with a single Gaussian component by the source finder {\texttt pyBDSF}~\citep{Mohan2015ascl.soft02007M} that was used to produce the LoTSS source catalogue---leaving 2,051,534 sources.


\subsection{Cross-matching \& filtering} 
\label{subsec:crossmatch}

We aim to identify OCRs dwelling in dwarf galaxies with luminosity exceeding the contribution expected from star formation alone. 
To find these, we cross-match our subset of LoTSS and CLU sources (\S~\ref{subsec:sample}) using a radius of $(6+\epsilon)\arcsec$ (the angular resolution of LoTSS, with $\epsilon$ being the astrometric uncertainty for a given source), yielding 708 matches.
We then reject all sources that lie within three standard deviations of the radio luminosity vs. star-formation rate (L-SFR) relationship presented by \citet{Gurkan2018MNRAS.475.3010G}\footnote{For consistency with \citet{Gurkan2018MNRAS.475.3010G}, throughout this paper, we use a concordance cosmology with $H_0=70\,{\rm km\,s^{-1}\,Mpc^{-1}}$, $\Omega_m=0.3$, and $\Omega_{\Lambda}=0.7$.}. 
This yields just 32 sources.

We verify that redshift values listed in CLU matches that of other measurements listed in the NASA/IPAC Extragalactic Database (NED) since redshifted emission lines other than H$\upalpha$ could fall within CLU’s filter and masquerade as H$\upalpha$. 
Indeed, objects with wrongly attributed redshifts will be over-represented in our sample that is selected based on non-adherence to the L-SFR relationship.
Among the 32 sources, we find that three objects have redshifts inconsistent with the median redshift from NED. 
We thus remove these from our sample, leaving 29 candidates.
From these, 11 out of the 29 sources have spectra available on NED.
Next, we check infrared magnitudes for any obvious active galactic nuclei (AGNe) by evaluating sources whose WISE colours fall within the AGN region prescribed by \citet[][Eq. 1]{Jarrett2011ApJ...735..112J}, represented in Figure~\ref{fig:wise}.
Two candidates fall within this AGN region, leaving 27 sources to be further scrutinized.

We depict the candidate selection on the L-SFR plane in Figure~\ref{fig:selection}.
We discuss the distribution of matched galaxies (including all galaxy masses) along the L-SFR plane in Appendix \ref{app:l_sfr}.
Candidates are summarized in Table~\ref{table:candidates}, which will be investigated further in the next sections.
Astrometric uncertainties provided by LoTSS were taken into account during cross-matching. 
Hence, while a projected offset of $7\arcsec$ is listed for ILT~J125944.53+275800.9, the lower limit on its offset taking into account the radio astrometric uncertainties on right ascension and declination is $5\,\farcs4$. 

\begin{sidewaystable*}
\caption{Properties of the selected candidates.
        Columns: 
        source name in LoTSS DR2; 
        host galaxy name in CLU; 
        projected offset between the radio source and H$\upalpha$ source coordinates in arcsec, and corresponding transverse distance in parsec; 
        redshift ($z$), with distance method indicated in parentheses: (k) kinematic; (m) median (redshift-independent), (n) narrowband (H$\upalpha$); 
        star formation rate ($\log {\rm SFR}$); 
        radio luminosity at 144\,MHz ($\log L$); 
        standard deviation ($\sigma$) above the L-SFR relation; 
        radio spectral index ($\alpha$; \S\ref{subsec:spectral}); radio loudness $R_g$ (\S\ref{sec:discussion}).
        Sources with spectral line measurements in SDSS table galSpecLine~\citep[\S\ref{subsec:bpt};][]{Tremonti2004ApJ...613..898T, Brinchmann2004MNRAS.351.1151B} have their redshifts marked in bold. 
        Candidates below and above an offset value of $(2+\epsilon)\arcsec$ are demarcated with a solid line, with $\epsilon$ being LoTSS' astrometric uncertainty.
        The third (bottom-most) section includes candidates in the AGN region of Figure \ref{fig:wise}.}
\centering
\begin{tabular}{llrrrrrrrrr}
    \toprule
        Source name & Host name & \multicolumn{2}{r}{Projected offset} & $z$ & $S$ & $\log L$ & $\log {\rm SFR}$ & $\sigma$ & $\alpha$ & $R_g$ \\
        (ILT~J) & ~ & (arcsec) & (pc) & ~  & mJy & (${\rm W/Hz}$) & (${\rm M_\odot/yr}$) & ~ & ~ & ~  \\
    \midrule
	003532.36+303008.0 & CLU J003532.3+303008 & $0.7\pm0.1$ & $165$ & $0.0121\pm0.01196$ (n) & $11.89\pm0.35$ & $21.6$ & $-1.02\pm-2.16$ & $3.85$ & $-0.56$ & $1.93$ \\
	125915.34+274604.2 & SSTSL2 J125915.27+274604.1 & $0.8\pm0.4$ & $321$ & $0.0211$ (k) & $2.57\pm0.28$ & $21.4$ & $-1.29\pm-1.63$ & $4.56$ & -- & -- \\
	231715.38+184339.0 & 2MASX J23171540+1843385 & $0.8\pm0.7$ & $355$ & $0.0409\pm0.00034$ (m) & $2.57\pm0.50$ & $22.0$ & $-0.98\pm-1.45$ & $6.09$ & -- & $0.13$ \\
	021835.45+262040.9 & LAMOST J021835.51+262040.7 & $0.9\pm0.3$ & $57$ & $0.0033$ (k) & $6.55\pm0.44$ & $20.2$ & $-2.32\pm-3.35$ & $3.79$ & $-0.53$ & $0.63$ \\
	075257.15+401026.3 & UGC 04068 & $0.9\pm0.2$ & $133$ & {\bf 0.0412$\pm$0.00010} (m) & $7.37\pm0.35$ & $22.5$ & $-1.64\pm-2.40$ & $13.37$ & $-0.54$ & -- \\
	165252.24+391151.7 & CLU J165252.32+391152.0 & $1.0\pm0.8$ & $540$ & $0.0272\pm0.00700$ (k) & $1.83\pm0.38$ & $21.5$ & $-1.00\pm-1.48$ & $3.08$ & -- & $0.53$ \\
	162244.56+321259.3 & 2MASS J16224461+3213007 & $1.0\pm0.1$ & $451$ & {\bf 0.0221$\pm$0.00009} (k) & $8.06\pm0.18$ & $22.0$ & $-0.75\pm-1.56$ & $4.29$ & $-0.63$ & $0.50$ \\
	153943.52+592730.7 & CLU J153943.44+592729.8 & $1.1\pm0.2$ & $883$ & $0.0411\pm0.00695$ (k) & $3.06\pm0.17$ & $22.1$ & $-0.55\pm-1.23$ & $3.72$ & -- & $0.52$ \\
	015915.79+242500.6 & KUG 0156+241 & $1.2\pm0.5$ & $128$ & $0.0130\pm0.00003$ (m) & $4.30\pm0.43$ & $21.2$ & $-1.35\pm-2.73$ & $3.69$ & $-0.01$ & $0.04$ \\
	131858.22+332859.9 & CLU J131858.32+332859.8 & $1.3\pm0.4$ & $74$ & $0.0029\pm0.00285$ (k) & $1.58\pm0.19$ & $19.5$ & $-3.01\pm-3.56$ & $3.77$ & -- & $0.51$ \\
	161439.00+545334.8 & CLU J161439.12+545334.0 & $1.3\pm1.1$ & $76$ & $0.0029\pm0.00285$ (k) & $1.03\pm0.25$ & $19.3$ & $-3.32\pm-3.74$ & $4.69$ & -- & $0.47$ \\
	142859.42+331005.2 & 2MASX J14285953+3310067 & $1.5\pm0.8$ & $863$ & {\bf 0.0291$\pm$0.00009} (k) & $1.97\pm0.29$ & $21.6$ & $-1.00\pm-1.42$ & $3.68$ & -- & $0.08$ \\
	140549.55+365943.9 & CLU J140549.44+365944.5 & $1.5\pm0.3$ & $86$ & $0.0029\pm0.00285$ (k) & $2.66\pm0.21$ & $19.7$ & $-2.74\pm-3.50$ & $3.40$ & -- & $0.73$ \\
	121407.57+423829.2 & KISSR 1246 & $1.8\pm0.1$ & $675$ & $0.0181\pm0.00280$ (k) & $2.86\pm0.11$ & $21.3$ & $-1.28\pm-1.84$ & $3.90$ & -- & $1.06$ \\
	163850.81+352901.0 & CLU J163850.64+352900.9 & $2.0\pm0.7$ & $118$ & {\bf 0.0029$\pm$0.00285} (k) & $2.15\pm0.34$ & $19.6$ & $-2.99\pm-3.47$ & $4.48$ & -- & $0.30$ \\
	110704.14+391812.3 & CLU J110704.32+391811.8 & $2.2\pm0.4$ & $127$ & $0.0029\pm0.00285$ (k) & $2.51\pm0.25$ & $19.7$ & $-2.79\pm-3.46$ & $3.61$ & -- & $0.34$ \\
	\hline
    143050.99+410642.6 & SDSS J143051.12+410640.8 & $2.4\pm0.2$ & $1632$ & {\bf 0.0342$\pm$0.00011} (k) & $2.71\pm0.18$ & $21.9$ & $-0.66\pm-1.41$ & $3.12$ & -- & $0.41$ \\
	220737.01+231516.0 & CLU J220737.20+231515.8 & $2.6\pm0.1$ & $154$ & $0.0029\pm0.00285$ (k) & $12.19\pm0.33$ & $20.3$ & $-2.30\pm-3.27$ & $4.56$ & -- & $1.39$ \\
	090406.54+530314.6 & SDSS J090406.38+530311.8 & $3.1\pm0.0$ & $2410$ & {\bf 0.0386$\pm$0.00007} (k) & $6.69\pm0.13$ & $22.4$ & $-0.43\pm-1.23$ & $4.74$ & $0.18$ & -- \\
	023058.18+232412.6 & CLU J023058.15+232409.3 & $3.3\pm0.6$ & $191$ & $0.0029\pm0.00285$ (k) & $11.49\pm1.31$ & $20.3$ & $-2.35\pm-3.39$ & $4.73$ & $-0.66$ & $0.97$ \\
	140524.35+613358.7 & 2MASS J14052457+6134020 & $3.3\pm0.0$ & $386$ & {\bf 0.0057$\pm$0.00001} (k) & $22.37\pm0.13$ & $21.2$ & $-1.88\pm-2.80$ & $7.23$ & $-0.88$ & $0.55$ \\
	130022.42+281451.7 & 2MASS J13002220+2814499 & $3.7\pm0.8$ & $2002$ & {\bf 0.0266} (k) & $1.47\pm0.33$ & $21.4$ & $-1.12\pm-1.69$ & $3.17$ & -- & $-0.04$ \\
	122250.31+681434.2 & SDSS J122249.71+681431.8 & $3.9\pm0.1$ & $2630$ & {\bf 0.0332$\pm$0.00005} (k) & $2.79\pm0.15$ & $21.8$ & $-0.71\pm-1.34$ & $3.37$ & -- & $0.43$ \\
	091333.83+300056.9 & 2MASX J09133387+3000514 & $5.5\pm0.4$ & $2281$ & {\bf 0.0204$\pm$0.00002} (k) & $5.89\pm0.49$ & $21.7$ & $-0.88\pm-1.66$ & $3.88$ & $-0.97$ & $-0.10$ \\
	113634.77+592533.3 & SBS 1133+597 & $5.6\pm0.0$ & $1182$ & {\bf 0.0104$\pm$0.00011} (k) & $12.82\pm0.12$ & $21.5$ & $-1.30\pm-2.15$ & $5.08$ & $-0.96$ & $0.15$ \\
	125940.18+275123.5 & 2MASX J12594007+2751177 & $5.8\pm0.0$ & $1491$ & {\bf 0.0127$\pm$0.00004} (k) & $22.34\pm0.20$ & $21.9$ & $-1.69\pm-2.27$ & $10.28$ & $-1.00$ & $0.39$ \\
	125944.53+275800.9 & SDSS J125944.76+275807.1 & $7.0\pm1.6$ & $4670$ & $0.0329\pm0.00025$ (k) & $2.38\pm0.56$ & $21.8$ & $-0.87\pm-1.44$ & $4.02$ & -- & -- \\
    \midrule
    \midrule
	161111.24+360401.0 & CLU J161111.2+360400 & $0.3\pm0.1$ & $75$ & $0.0121\pm0.01196$ (n) & $8.45\pm0.17$ & $21.4$ & $-1.10\pm-2.30$ & $3.49$ & $-0.79$ & $1.15$ \\
    143037.30+352053.3 & SDSS J143037.09+352052.8 & $2.6\pm0.6$ & $2036$ & $0.0390\pm0.00000$ (k) & $1.18\pm0.21$ & $21.6$ & $-0.92\pm-1.29$ & $3.35$ & -- & $1.64$ \\
    \bottomrule
    \end{tabular}
    \label{table:candidates}
\end{sidewaystable*}

\begin{figure*}
\includegraphics[width=17cm]{figures/wise.pdf}
\caption{WISE color-color plot. 
Photometric ratios in magnitude (mag) between three infrared bands (W1${\,\rm_{3.4\upmu\\m}}$, W2${\,\rm_{4.6\upmu\\m}}$, W3${\,\rm_{12\upmu\\m}}$, with W1-W2, W2-W3).
Black markers correspond to dwarf galaxies matched to a compact radio source with luminosity exceeding $3\sigma$ in the L-SFR relation. 
Grey-filled markers indicate dwarf galaxies matched to a radio source. 
We demarcate the two sources falling within the AGN region~\citep{Jarrett2011ApJ...735..112J} marked by dashed lines within our sample as being classified.}

\label{fig:wise}
\end{figure*}


\begin{figure*}
\includegraphics[width=17cm]{figures/selection_paper_mass_flux_validSFR_const__6_arcsec_filtered.pdf}
\caption{Candidate selection via the L-SFR relation. 
Grey-filled markers indicate compact radio sources matched to dwarf galaxies matched. 
Black filled circles correspond to our final selection of OCRs matched to a dwarf galaxy with luminosity exceeding $3\sigma$ (dashed line) on the L-SFR relation by \citet[][solid line]{Gurkan2018MNRAS.475.3010G} with validated redshift.
The uncertainty on luminosity is below the markers size.
Orange markers are sources within the AGN region in Figure \ref{fig:wise}.
As reference, we show PRSs luminosity and hosts SFR for both \RI~\citep{Tendulkar2021ApJ...908L..12T, Law2022ApJ...927...55L} and \RItwin~\citep{Niu2022Natur.606..873N}, with luminosity measurements scaled to 144\,MHz using $\alpha$ from \citet[][evaluated between 2\,MHz--10\,GHz]{Resmi2021A&A...655A.102R} and \citet[][evaluated between 1.5--5.5\,GHz]{Niu2022Natur.606..873N}, respectively.
Finally, we show \citet{Reines2020ApJ...888...36R} galaxies (J0909+5655, J1136+2643, J1220+3020) matched in CLU for which SFR information is available, with luminosity scaled to 144\,MHz using $\alpha$ values fitted between 1.4\,GHz and 9\,GHz by \citet{Eftekhari2020ApJ...895...98E}. 
The large scatter in luminosity is discussed in \S\ref{subsec:chance} and Appendix \ref{app:l_sfr}.
}
\label{fig:selection}
\end{figure*}


\subsection{Chance alignment probability}
\label{subsec:chance}

We begin by evaluating the likelihood of matching a galaxy and a background radio source by coincidence due to chance alignment following the two methods described by \citet{Reines2020ApJ...888...36R}.

Firstly, we estimate the cumulative number of compact radio source counts per steradian $N(S_{min})$ with 144\,MHz flux densities greater than $S_{min}=0.8\,\rm{mJy}$, taking only into account compact sources based on {\tt pyBDSF} single component cases and the $R_{99.9}$ compactness criterion.
Multiplying $N(S_{min})\approx2,051,534\,{\rm steradian}^{-1}$
by the area confined in a $6\arcsec$ radius circle gives $N_{bk,gal} = 0.0032$, the expected number of background sources for a given galaxy.
Across our entire parent sample of 18,159 %15,972 
dwarf galaxies, we expect $N_{bk,samp} \approx 58 \pm \sqrt{58} \approx 58 \pm 8$ background sources (where the error is computed for Poisson statistics).
We therefore expect $N_{bk,samp}$ to be present in our original 708 matches, or about 0.8\%.
Performing a similar analysis using a cross-matching radius of $2\arcsec$ leads to $N_{bk,samp} \approx 6 \pm 2$ background sources (with an original number of matches of 573 using a $2\arcsec$ cut), or about $1\%$.
Considering that outliers caused by a false association should be evenly distributed above and below the L-SFR relation (Figure \ref{fig:selection}), it is likely that nearly all candidates below the $(2+\epsilon)\arcsec$ mark are true associations, and about half of the candidates above this threshold.

Secondly, we perform an empirical estimate of the expected number of false associations using the data.
We cross-match each source in the CLU subsample to its nearest neighbour in LoTSS, up to a maximal matching radius of $40\arcsec$.
Figure~\ref{fig:chance_hist} shows the observed offset distribution.
The offset probability histogram for chance associations should equal zero at an offset of zero and rise linearly for small offsets (blue solid line in Figure~\ref{fig:chance_hist}).
The observed distribution is minimal at a matching radius of $6\arcsec$, beyond which the number of sources per offset, $N(d_{\rm{off}})$, increases linearly as $N(d_{\rm{off}})=3.1d_{\rm{off}}$, with offset $d_{\rm{off}}$ in unit of arcsec.
The total estimated number of background sources with offsets less than $6\arcsec$ is found by integrating $N(d_{\rm{off}})$ from $N(d_{\rm{off}}=0\arcsec$ to $5\arcsec$, which gives $N_{bk,samp} \approx 64 \pm 8$ sources.
This is consistent with our calculation above using known radio source counts.
Here again, putting a cut at $2\arcsec$ (pink vertical line) leads to $N_{bk,samp} \approx 9 \pm 3$, which is also consistent within errors with the analytical method.

\begin{figure}
\resizebox{\hsize}{!}{\includegraphics{figures/chance_alignment_hist.pdf}}
% \includegraphics[width=8.5cm]{figures/chance_alignment_hist.pdf}
\caption{Observed offset distribution after cross-matching CLU and LoTSS surveys, using a match radius of $40\arcsec$.
We select our target galaxies to have offsets $2\arcsec$ (pink vertical dashed line) and $6\arcsec$ (blue vertical dashed line).
The blue solid line shows the expected number of chance alignments with background sources as a function of offset.}
\label{fig:chance_hist}
\end{figure}

From these two analyses, it is unclear whether outliers on the L-SFR relation found in \S\ref{sec:method} are caused by chance alignment where two unrelated sources would lead to an incorrect luminosity calculation (e.g., a radio source is assigned an incorrect redshift).
To resolve this issue, we performed Monte Carlo simulations.
We repeated all cross-matching and outlier selection steps (as described in \S\ref{subsec:sample} and \S\ref{subsec:crossmatch}) after randomly shifting spatial coordinates for all sources within the LoTSS set in each Monte-Carlo run.
For both cases using $2\arcsec$ and $6\arcsec$ as the cross-matching radius limit, we performed 1000 realisations of the process, selecting spatial shifts from a uniform distribution of $[-10, 10]$ arcmin, keeping track of the number of outliers at each realisation.

Cumulative distributions summarizing each Monte Carlo simulation are shown in Figure~\ref{fig:montecarlo}.
At $6\arcsec$, on average for the entire sample of CLU, it is common to find 10 matches (median) above the $3\sigma$ mark of the L-SFR relation by chance, and is unlikely to find 27 by chance ($p \ll 10^{-4}$). 
At a $2\arcsec$ cutoff, it is common to get two matches by chance, and highly unlikely to find 16.
Given our 16 matches selected below $(2+\epsilon)\arcsec$, we expect a false positive fraction of 0.125. 
The remaining 11 sources with offsets between $(2+\epsilon)-(6+\epsilon)\arcsec$ may be chance alignments.

\begin{figure}
\resizebox{\hsize}{!}{\includegraphics{figures/monte_carlo_1000_uniform_shift_10_arcmin.pdf}}
% \includegraphics[width=8.5cm]{figures/monte_carlo_1000_uniform_shift_10_arcmin.pdf}
\caption{Cumulative distributions from 1000 Monte Carlo realisations of matches exceeding $3\sigma$ on the L-SFR relation from \citet{Gurkan2018MNRAS.475.3010G}. Top panel: $6\arcsec$ matching radius; bottom panel: $2\arcsec$ matching radius.}
\label{fig:montecarlo}
\end{figure}

While it does not mean that candidates with a projected offset greater than $2\arcsec$ need to be chance alignments---e.g. a small transverse distance may indicate a true association---with the data we currently have at hand, assessing whether they are real associations remains difficult.
For this reason, we demarcate candidates below and above $(2+\epsilon)\arcsec$ in Table~\ref{table:candidates} and the following sections with a black line.
Nevertheless, we will show in the next Section that nearly all candidates---both below and above this demarcation---fall within the optical footprint of their respective matched galaxies.

Finally, the redshift distribution of selected candidates listed in Table \ref{table:candidates} differs wildly from that of all dwarf galaxies from CLU in the LoTSS field (Figure \ref{fig:ks-test}).
A two-sample Kolmogorov-Smirnov (KS) test between these two distributions lends a $p$-value of $1.08 \times 10^{-5}$, indicating the selected candidates do not simply track the redshift distribution of the whole CLU dwarf galaxy sample. 

\begin{figure}
\resizebox{\hsize}{!}{\includegraphics{figures/redshift_dist.pdf}}
% \includegraphics[width=8.5cm]{figures/redshift_dist.pdf}
\caption{Redshift distributions of selected candidates listed in Table \ref{table:candidates} (grey) and for all dwarf galaxies from CLU in the LoTSS field (black). 
A two-sample KS test between these two distributions lends a $p$-value of $1.08 \times 10^{-5}$. }
\label{fig:ks-test}
\end{figure}
