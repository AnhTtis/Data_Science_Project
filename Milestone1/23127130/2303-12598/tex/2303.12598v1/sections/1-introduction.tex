\section{Introduction} \label{sec:intro}

\RI~was the first fast radio burst (FRB) source found to repeat~\citep{Spitler2016Natur.531..202S}. 
Its repetitive nature rules out progenitor models related to cataclysmic explosions for at least a fraction of all FRBs.
\RI~was also the first FRB source to be precisely localized to a host galaxy~\citep{Chatterjee2017Natur.541...58C} from a coordinated observing campaign between the Karl G. Jansky Very Large Array (VLA) and the 305-m William E. Gordon Telescope at the Arecibo Observatory. 
The host galaxy is a low-mass, low-metallicity dwarf at redshift $z=0.19273\pm0.0008$~\citep{Tendulkar2017ApJ...834L...7T}, which is a typical host for long gamma-ray bursts and super-luminous supernovae.
These simple facts provide a hint that the FRB engine may have been born in such an explosion.
Moreover, \RI~was the first FRB found to be co-located with a persistent radio source~\citep[PRS;][]{Chatterjee2017Natur.541...58C}, with a luminosity $L_\mathrm{radio}\sim10^{39}\,\mathrm{erg\,s^{-1}}$ that is $>50\times$ what would be expected from star formation activity alone.

Very long baseline interferometry (VLBI) observations with the European VLBI Network (EVN) showed that the FRB and the PRS are located within $\lesssim 40\,{\rm pc}$ (transverse distance) of one another~\citep{Marcote2017ApJ...834L...8M}, strongly connecting the two.
Optical and far-infrared observations using the {\it Hubble Space Telescope} revealed that the FRB/PRS location is slightly offset from the centroid of a star formation knot within the host~\citep{Bassa2017ApJ...843L...8B}.
The PRS has a flat spectral index
$S_\nu\propto\nu^\alpha$, with $\alpha\sim-0.07\pm0.03$ below 10\,GHz~\citep{Resmi2021A&A...655A.102R}, with a possible turnover at lower frequencies, estimated as $\alpha\sim0.3$ between 433\,MHz and 1.4\,GHz by \citet{Mondal2020MNRAS.498.3863M}. 
The spectral energy distribution---derived from multi-wavelength measurements and upper limits---matches that of the Crab nebula, though with orders of magnitude higher luminosity.
Finally, bursts from \RI~have high and variable Faraday rotation measure~\citep[RM;][]{Michilli2018Natur.553..182M}, and the dispersion measure (DM) also shows secular changes~\citep{Hessels2019ApJ...876L..23H, Platts2021MNRAS.505.3041P}. 

The combination of these measurements make a plausible case for an FRB engine that is a young, highly magnetized neutron star embedded in an expanding supernova remnant and powering a pulsar wind nebula\footnote{Magnetar wind nebulae have also been discussed in the literature as the source of the persistent radio emission~\citep[e.g.][]{Zhao2021ApJ...923L..17Z}.}~\citep[PWN;][]{Murase2016MNRAS.461.1498M, Margalit2018ApJ...868L...4M}.
Another plausible explanation is that the FRB engine is within the vicinity of a massive black hole, which in turn creates the PRS.
In this context, the FRB source may be a neutron star near a black hole, or even be produced by a black hole jet.
There is only one other repeating FRB co-localised to a PRS currently known~\citep[][]{Niu2022Natur.606..873N}.
Like \RI, \RItwin~is hosted in a star-forming dwarf galaxy, with a large $\mathrm{DM}_\mathrm{host}$ contribution, a high repetition rate, and its associated PRS has a shallow spectral index ($\alpha=-0.41\pm0.04$).

Given these two cases of FRB/PRS connection, it seems that PRSs represent an important aspect of some FRBs, even if their nature remains mysterious.
If PRSs are wind nebulae, a limited lifespan during which they could be detected~\citep[$\sim$few centuries;][]{Gaensler2006ARA&A..44...17G} could explain why only a subset of FRBs have a PRS counterpart.
Considering a sample of 15 localised FRBs with radio sensitivity limits that could allow to detect a PRS (including six repeating FRBs), \citet{Law2022ApJ...927...55L} estimate that PRS occurrence could be as high as 20\% for repeating FRBs given that 2 out of the 6 repeating FRBs in their sample are associated to a PRS. 
Furthermore, these authors estimate that PRS detectability should not be strongly biased by distance.

Separately, \citet{Reines2020ApJ...888...36R} have identified a sample of compact radio sources associated with dwarf galaxies and suggested that they may be the long-sought population of intermediate-mass black holes (IMBHs, ${\rm \sim10^2-10^5\,M_\odot}$) that were predicted to reside in dwarf galaxies~\citep[e.g.][]{McConnell2013ApJ...764..184M, Reines2015ApJ...813...82R}. 
Furthermore, the compact radio sample presented by \citet{Reines2020ApJ...888...36R} has been shown to share many similarities to the PRS associated with \RI, with radio luminosities, spectral energy distributions, light curves, ratios of radio-to-optical flux, and spatial offsets between the radio source and the host optical center being consistent as arising from the same population~\citep{Eftekhari2020ApJ...895...98E}.
Although the connection between IMBHs, PRSs and FRBs remains unclear---most FRB models prefer a magnetar progenitor~\citep[e.g., see discussion in][]{Eftekhari2020ApJ...895...98E}---overluminous compact radio sources (OCRs hereafter) in dwarf galaxies are an interesting radio source population in their own right. 

To improve our understanding of PRSs and their potential connection to the FRB and/or IMBH phenomena (e.g. constrain FRB progenitor models), it is imperative to increase the known sample size.
Here, we present a targeted search for compact radio sources coincident with dwarf galaxies, using the LOFAR Two-Meter Sky Survey (LoTSS) second data release~\citep[DR2;][]{Shimwell2022A&A...659A...1S}---the most sensitive large-area survey for optically thin synchrotron emission---as our radio reference catalogue, and the Palomar Observatory’s 48-inch Samuel Oschin Telescope's `Census of the Local Universe'~\citep[CLU;][]{Cook2019ApJ...880....7C} as our optical reference catalogue. 

The article is organized as follows: \S\ref{sec:method} describes our candidate selection methodology and discusses chance alignment probability; \S\ref{sec:candidates} discusses the selected candidates, and augments information about our sample with ancillary survey data at various wavelengths; \S\ref{sec:discussion} discusses the potential nature of these sources and plans for future work; and we close with a summary in \S\ref{sec:summary}.