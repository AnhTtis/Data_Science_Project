\section{Candidates}
\label{sec:candidates}

The final candidate list includes 27 sources, with 16 having a projected offset lower than $(2+\epsilon)\arcsec$. 
Composite images for each candidate in ascending order of offset are presented in Figure~\ref{fig:family_plot} (offset $\lesssim2\arcsec$) and Figure~\ref{fig:family_plot_continued} ($\gtrsim2\arcsec$ and OCRs within the AGN region of Figure \ref{fig:wise}), and candidates are detailed in Table~\ref{table:candidates}. 
Given that all H$\upalpha$ filters used by CLU fall within the frequency range covered by the PanSTARRS~\citep[PS1;][]{Chambers2016arXiv161205560C, Flewelling2020ApJS..251....7F} r filter, we use PS1 to generate image cutouts around our candidates' coordinates.
The central coordinates of matched objects from LoTSS and CLU are respectively indicated by black `$\times$' and `+' markers. 

All candidates shown in Figure~\ref{fig:family_plot} and most candidates in Figure~\ref{fig:family_plot_continued} fall well within the galaxy contours. 
ILT~J125940.18+275123.5 fall within low surface brightness regions of its host galaxy, while the associated host galaxy of ILT~J125940.18+275123.5 (candidate with largest offset) is of extremely low surface density, and likely a chance alignment (further discussed in Appendix \ref{sec:appendix1}).
SDSS~J143037.09+352052.8 (matched to the AGN candidate ILT~143037.30+352053.3) is also extremely faint in the PS1's r filter image and may also be chance alignment. 
However, data from the JVO Subaru/Suprime-Cam~\citep{Aihara2019PASJ...71..114A} show the radio source to be within the galaxy (Appendix \ref{sec:appendix1}).

We note that galaxies
2MASS J13002220+2814499 (ILT~J130022.42+281451.7), 
SDSS J125944.76+275807.1 (ILT~J125944.53+275800.9), 
2MASX J12594007+2751177 (ILT~J125940.18+275123.5), and 
SSTSL2 J125915.27+274604.1 (ILT~J125915.34+274604.2) are part of the cluster of galaxies ACO~1656 (Appendix \ref{sec:appendix2}).
Similarly, 2MASX~J23171540+1843385 (ILT~J231715.38+184339.0) is part of the compact group of galaxies HCG~94. 
Finally, the \href{http://simbad.cds.unistra.fr/simbad/sim-id?Ident=HCG++37&NbIdent=query_hlinks&Coord=09+13+35.6%2B30+00+51&children=5&submit=children&hlinksdisplay=h_all}{Simbad} service\footnote{\url{http://simbad.cds.unistra.fr/simbad}} indicates 2MASX~J09133387+3000514 (ILT~J091333.83+300056.9) as having a 75\% probability of being an hierarchical member of the galaxy cluster HCG~37.

In the following sub-sections, we investigate the possible nature of the candidates by considering complementary information in ancillary surveys --- e.g., optical spectra and flux density measurements at other radio wavelengths. 

\begin{figure*}
\centering
\begin{tabular}{cccc}
\includegraphics[width=41mm]{figures/ILTJ003532.36+303008.0.pdf} &
\includegraphics[width=41mm]{figures/ILTJ125915.34+274604.2.pdf} &
\includegraphics[width=41mm]{figures/ILTJ231715.38+184339.0.pdf} &
\includegraphics[width=41mm]{figures/ILTJ021835.45+262040.9.pdf} \\[0.2cm]
\includegraphics[width=41mm]{figures/ILTJ075257.15+401026.3.pdf} &
\includegraphics[width=41mm]{figures/ILTJ165252.24+391151.7.pdf} &
\includegraphics[width=41mm]{figures/ILTJ162244.56+321259.3.pdf} &
\includegraphics[width=41mm]{figures/ILTJ153943.52+592730.7.pdf} \\[0.2cm]
\includegraphics[width=41mm]{figures/ILTJ015915.79+242500.6.pdf} &
\includegraphics[width=41mm]{figures/ILTJ131858.22+332859.9.pdf} &
\includegraphics[width=41mm]{figures/ILTJ161439.00+545334.8.pdf} &
\includegraphics[width=41mm]{figures/ILTJ142859.42+331005.2.pdf} \\[0.2cm]
\includegraphics[width=41mm]{figures/ILTJ140549.55+365943.9.pdf} &
\includegraphics[width=41mm]{figures/ILTJ121407.57+423829.2.pdf} &
\includegraphics[width=41mm]{figures/ILTJ163850.81+352901.0.pdf} &
\includegraphics[width=41mm]{figures/ILTJ110704.14+391812.3.pdf} 
\end{tabular}
\caption{
    OCRs with projected offset less than $(2+\epsilon)\arcsec$. Each panel indicates the source name from LoTSS and host galaxy name above a PS1 R filter image in logarithmic grey scale. 
    Contours indicates optical data from PS1~\citep[][]{Chambers2016arXiv161205560C, Flewelling2020ApJS..251....7F} at $\mu+[1, 2, ..., 10]\sigma$ levels, with $\mu$ and $\sigma$ being the median and standard deviation, respectively. 
    The black ($\times$, +) markers indicate the central coordinates of matched objects from LoTSS and CLU, respectively. 
    LoTSS astrometric uncertainty is marked as a yellow box.
    Where available in the CLU catalogue, we indicate a yellow ellipse corresponding to the H$\upalpha$ detection isophote (D25).
    For each panel, top left: we indicate the spatial offset ($\arcsec$) between the $\times$ and + markers as a white bar, along with corresponding transverse distance (pc); 
    top right: LoTSS radio flux at 144\,MHz in mJy, with uncertainty on the last digit in parenthesis, host galaxy stellar mass (${\rm M^{\rm host}_*}$) in ${\rm M_\odot}$, and redshift ($z$), with distance method indicated in parentheses: (k) kinematic; (m) median (redshift-independent), (n) narrowband (H$\upalpha$). 
    Where available, we indicate the spectral index ($\alpha$; Figure~\ref{fig:spectral_indices}; \S\ref{subsec:spectral}]) and the location of the SDSS spectroscopic fiber as a filled yellow circle (Figure \ref{fig:bpt}; \S\ref{subsec:bpt}, \S\ref{fig:spectral_indices}).
    Bottom left: white circles indicate the LoTSS $6\arcsec$ beam, noting that the restoring beam used in DDFacet~\citep{Tasse2018A&A...611A..87T} for each image product type is kept constant over the data release region and that all image products are made with a $uv$-minimum of 100\,m with the $uv$-maximum varied to provide images at different resolutions -- the highest resolution $6\arcsec$ images use baselines up to 120\,km (i.e. all LOFAR stations within the Netherlands).
}
\label{fig:family_plot}
\end{figure*}

\begin{figure*}
\centering
\begin{tabular}{cccc}
\includegraphics[width=41mm]{figures/ILTJ143050.99+410642.6.pdf} &
\includegraphics[width=41mm]{figures/ILTJ220737.01+231516.0.pdf} &
\includegraphics[width=41mm]{figures/ILTJ090406.54+530314.6.pdf} &
\includegraphics[width=41mm]{figures/ILTJ023058.18+232412.6.pdf} \\[0.2cm]
\includegraphics[width=41mm]{figures/ILTJ140524.35+613358.7.pdf} &
\includegraphics[width=41mm]{figures/ILTJ130022.42+281451.7.pdf} &
\includegraphics[width=41mm]{figures/ILTJ122250.31+681434.2.pdf} &
\includegraphics[width=41mm]{figures/ILTJ091333.83+300056.9.pdf} \\[0.2cm]
\end{tabular}
\begin{tabular}{ccc}
\includegraphics[width=41mm]{figures/ILTJ113634.77+592533.3.pdf} &
\includegraphics[width=41mm]{figures/ILTJ125940.18+275123.5.pdf} &
\includegraphics[width=41mm]{figures/ILTJ125944.53+275800.9.pdf} \\[0.2cm]
\end{tabular}
\begin{tabular}{cc}
\hline
\hline
\includegraphics[width=41mm]{figures/ILTJ161111.24+360401.0.pdf} &
\includegraphics[width=41mm]{figures/ILTJ143037.30+352053.3.pdf} 
\end{tabular}
\caption{
    OCRs with projected offset greater than $(2+\epsilon)\arcsec$. 
    OCRs within the AGN region in Figure \ref{fig:wise} are shown below demarcation lines.
    See Figure~\ref{fig:family_plot} for components description.
}
\label{fig:family_plot_continued}
\end{figure*}



\subsection{Main source of ionisation in host galaxies}
\label{subsec:bpt}

A useful tool for distinguishing between galaxies with different prevailing photo-ionization sources is the family of emission line ratio diagnostic diagrams introduced by \citet*[hereafter, BPT]{Baldwin1981PASP...93....5B} in which the source location is determined by a pair of low-ionization, emission-line intensity ratios.
We searched archival data\footnote{Data from: SDSS~\citep{Abdurrouf2022ApJS..259...35A} DR 17, CALIFA~\citep[Calar Alto Legacy Integral Field Area;][]{Gonzalez2015A&A...581A.103G} survey, and LAMOST~\citep[Large Sky Area Multi-Object Fibre Spectroscopic Telescope;][]{Guo2022yCat..36670044G} survey.} for optical spectra to within $6\arcsec$ of CLU’s coordinates. 
In SDSS, we found eleven matches within our candidates, and 483 out of the total 708 matched dwarf galaxies. 
It is worth noting, however, that $\sim40$\% of nearby radio-loud AGN are too gas poor and optically inactive to be detected via optical line ratio~\citep{Gereb2015A&A...580A..43G}.

Using the normalized emission line measurements from the MPA-JHU\footnote{A collaboration of researchers from the Max Planck Institute for Astrophysics (MPA) and the Johns Hopkins University (JHU).} spectroscopic reanalysis\footnote{Values are taken from the table galSpecLine. A table description is available at \href{http://skyserver.sdss.org/dr17/MoreTools/browser/}{this url} (last visited 7 September 2022).}~\citep{Tremonti2004ApJ...613..898T, Brinchmann2004MNRAS.351.1151B}, we evaluate the ratios of measured line fluxes (Table~\ref{table:BPT}) for $\rm {[O{\sc III}]/H\upbeta}$ against ${\rm [N{\sc II}]/H\upalpha}$ and ${\rm [S{\sc II}]/H\upalpha}$. 
We show the results in Figure~\ref{fig:bpt}. 

\begin{figure*}
\includegraphics[width=17cm]{figures/bpt_sdss_with_agns.pdf}
\caption{BPT diagrams for sources with SDSS spectra showing the main source of ionisation. 
Green and black markers correspond to sources with projected separation less than or greater than $(2+\epsilon)\arcsec$, respectively. 
Grey circles are all `compact radio source-dwarf galaxy' matches with spectral line information in SDSS.
Markers with an inner circle color other than black occupy interesting parts of parameter space and are discussed in \S\ref{subsec:bpt}.
The galaxy marked with a white inner circles lies above the grey line in the ${\rm S{\sc II}/H\upalpha}$ panel, while being well within the star formation range in the ${\rm [N{\sc II}]/H\upalpha}$ panel. 
Blue and pink inner circles fall within the composite region in the ${\rm [N{\sc II}]/H\upalpha}$ panel, where we can expect a contribution from both star formation ($H{\sc II}$ regions) and AGN activity. 
Finally, the galaxy marked by cyan inner circles has been classified as an AGN candidate by \citet{Truebenbach2017MNRAS.468..196T} by selecting sources detected in the AllWISE and FIRST catalogues, while not being detected in 2MASS or SDSS DR7 and DR9. 
}
\label{fig:bpt}
\end{figure*}

\begin{table*}
\centering
\caption{Emission line ratio measurements for host galaxies observed by SDSS (also shown in Figure~\ref{fig:bpt}). \\ {\em Markers:} $\times$ Negative flux; * With photometric match but not found in table galSpecLine.}
\begin{tabular}{llrrrrr}
\toprule
           Source name & Host name & $\log~[{\rm N{\sc II}]/H}\upalpha$ & $\log~[{\rm O{\sc II}I]/H}\upbeta$  & $\log~[{\rm S{\sc II}]/H}\upalpha$ & $\alpha$ & $R_g$ \\
\midrule
	ILTJ075257.15+401026.3 & UGC~04068 & $-0.143(129)$ & $-0.017(052)$ & $-0.371(115)$ & $-0.54$ & -- \\
	ILTJ162244.56+321259.3 & 2MASS~J16224461+3213007 & $-0.777(311)$ & $0.073(050)$ & $-0.366(111)$ & $-0.63$ & $0.50$ \\
	ILTJ142859.42+331005.2 & 2MASX~J14285953+3310067 & $-0.500(179)$ & $-0.316(005)$ & $-0.260(170)$ & -- & $0.08$ \\
    \hline
	ILTJ143050.99+410642.6 & SDSS~J143051.12+410640.8 & $-0.946(158)$ & $0.127(044)$ & $-0.331(210)$ & -- & $0.41$ \\
	ILTJ090406.54+530314.6 & SDSS~J090406.38+530311.8 & $-1.120(148)$ & $0.469(147)$ & $-0.483(159)$ & $0.18$ & -- \\
	ILTJ140524.35+613358.7 & 2MASS~J14052457+6134020 & $-1.254(837)$ & $0.538(426)$ & $-0.895(491)$ & $-0.88$ & $0.55$ \\
	ILTJ122250.31+681434.2 & SDSS~J122249.71+681431.8 & $-0.960(283)$ & $0.360(123)$ & $-0.592(091)$ & -- & $0.43$ \\
	ILTJ113634.77+592533.3 & SBS~1133+597 & $-0.651(474)$ & $-0.024(005)$ & $-0.565(235)$ & $-0.96$ & $0.15$ \\
	ILTJ125940.18+275123.5 & 2MASX~J12594007+2751177 & $-0.190(234)$ & $0.049(211)$ & $-1.510(505)$ & $-1.00$ & $0.39$ \\
\bottomrule
\end{tabular}
\label{table:BPT}
\end{table*}


\subsection{Spectral indices in the radio band}
\label{subsec:spectral}

A radio continuum spectrum dominated by non-thermal synchrotron emission has a characteristic power-law slope, $S_v \propto v^\alpha$. 
To evaluate the spectral index $\alpha$ of our candidates, we cross-matched other radio surveys to within a $6\arcsec$ radius from our source center.
In particular, we searched the Rapid ASKAP Continuum Survey~\citep[RACS;][1.25\,GHz]{McConnell2020PASA...37...48M}, the Faint Images of the Radio Sky at Twenty centimeters survey~\citep[FIRST;][1.4\,GHz]{Becker1995ApJ...450..559B}, the NRAO VLA Sky Survey~\citep[NVSS;][1.4\,GHz]{Condon1998AJ....115.1693C}, and the Very Large Array Sky Survey~\citep[VLASS;][3\,GHz]{Lacy2020PASP..132c5001L}. 
The coverage of each survey can be found in Appendix \ref{app:coverage}.

We find a total of 12 matches among the 29 source candidates.
Spatial coverage by the various surveys is discussed in Appendix \ref{app:coverage}. 
In Table~\ref{table:fluxes} we list the multi-frequency flux measurements used to evaluate spectral indices ($\alpha$), following the order of sources in Table~\ref{table:candidates}. 
Figure~\ref{fig:spectral_indices} presents for each matched source a radio spectrum and best model from fitting a single power law to flux measurements including uncertainties (dashed line), with corresponding $\alpha$. 
The resulting spectral indices range between $0.2\pm0.01$ and $-1.0\pm0.10$.

\begin{table*}
    \centering
    \caption{Fluxes for candidates matched in at least one ancillary radio survey. Measurements from LoTSS, VLA FIRST, NVSS, and VLASS at 1.4\,GHz, 1.4\,GHz, and 3\,GHz, respectively. Values in brackets are estimated based on cutout image from \href{http://cutouts.cirada.ca}{CIRADA Image Cutout Web Service}. Only spectral indices in brackets include estimates from CIRADA Cutouts. 
    }
    \begin{tabular}{lrrrrrrr}
    \toprule
        Source name & $S_{\rm LoTSS}$ & $S_{\rm RACS}$ & $S_{\rm FIRST}$ & $S_{\rm NVSS}$ & $S_{\rm VLASS}$ & $\alpha$ & $\alpha_{\rm HF}$   \\
        (ILT~J) & (mJy) & (mJy) & (mJy) & (mJy) & (mJy) & ~ & ~ \\
    \midrule
	003532.36+303008.0 & $11.89\pm0.35$ & -- & -- & $3.90\pm0.50$ & $1.99\pm0.28$ & $-0.6\pm0.04$ & $-0.9\pm0.25$ \\
	021835.45+262040.9 & $6.55\pm0.44$ & $2.07\pm0.86$ & -- & -- & -- & $-0.5\pm0.19$ & -- \\
	075257.15+401026.3 & $7.37\pm0.35$ & -- & $2.14\pm0.15$ & -- & [$0.50\pm0.05$] & $-0.5\pm0.04$ & [$-1.9\pm0.13$] \\
	162244.56+321259.3 & $8.06\pm0.18$ & -- & $1.81\pm0.15$ & $2.70\pm0.40$ & [$1.20\pm0.05$] & $-0.6\pm0.03$ & [$-0.8\pm0.05$] \\
    015915.79+242500.6 & $4.30\pm0.43$ & $4.18\pm1.13$ & -- & -- & -- & $-0.0\pm0.13$ & -- \\
    \hline
	090406.54+530314.6 & $6.69\pm0.13$ & -- & $12.65\pm0.15$ & $15.30\pm0.90$ & $9.52\pm0.23$ & $0.2\pm0.01$ & $-0.4\pm0.03$ \\
	023058.18+232412.6 & $11.49\pm1.31$ & $3.51\pm0.89$ & -- & $2.40\pm0.40$ & -- & $-0.7\pm0.08$ & -- \\
	140524.35+613358.7 & $22.37\pm0.13$ & -- & $3.11\pm0.14$ & -- & $1.08\pm0.24$ & $-0.9\pm0.02$ & $-1.4\pm0.30$ \\
	091333.83+300056.9 & $5.89\pm0.49$ & -- & $0.65\pm0.13$ & -- & -- & $-1.0\pm0.10$ & -- \\
	113634.77+592533.3 & $12.82\pm0.12$ & -- & $1.44\pm0.20$ & -- & -- & $-1.0\pm0.06$ & -- \\
	125940.18+275123.5 & $22.34\pm0.20$ & $5.83\pm1.38$ & $2.09\pm0.18$ & $3.00\pm0.40$ & -- & $-1.0\pm0.03$ & -- \\
    \midrule
    \midrule
    161111.24+360401.0 & $8.45\pm0.17$ & -- & $1.41\pm0.14$ & -- & [$0.50\pm0.05$] & $-0.8\pm0.04$ & [$-1.4\pm0.18$] \\
    \bottomrule
    \end{tabular}
    \label{table:fluxes}
\end{table*}

\begin{figure*}
    \centering
    \includegraphics[width=17cm]{figures/spectral_indices_candidates.pdf} 
    \caption{Radio spectra of candidates matched in one or more of the RACS, FIRST, NVSS, and VLASS surveys, with central frequencies of 1.4\,GHz, 1.4\,GHz, and 3\,GHz, respectively. LoTSS observes at a central frequency of 144\,MHz. The source name in LoTSS is indicated for each candidate. Black markers show flux measurements with uncertainties (small enough not to be visible), and the dashed-blue lines show the best power-law fit over all frequencies, with power law index $\alpha$ also indicated. Where applicable, a grey-dashed line shows the best power-law fit between $1.4-3$\,GHz. }%
    \label{fig:spectral_indices}%
\end{figure*}
