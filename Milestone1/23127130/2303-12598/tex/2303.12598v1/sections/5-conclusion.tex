
\section{Summary}
\label{sec:summary}

In this paper, we presented a targeted search for OCRs coincident with dwarf galaxies up to $z\lesssim 0.05$. 
\begin{enumerate}
\item We identified candidate compact sources with luminosity exceeding $3\sigma$ relative to the L-SFR relation~\citep{Gurkan2018MNRAS.475.3010G}.
\item Through ancillary surveys, we investigated the possible nature of the candidates.
\item Emission line ratios from SDSS spectra shows the main source of ionisation in the host galaxies where the candidates are located is likely star formation, and not AGN activity. 
\item Spectral indices suggest that our candidates could be SNRs or AGNe---although, combining $\alpha$ to emission line ratios and radio loudness hint that we may be observing sources other than typical AGNe. 
\item A comparison to the luminosity functions for star formation and AGNe of \citet{Condon2019ApJ...872..148C} indicates that radio emission from our candidates to be more likely attributed to star formation.
\item We evaluated a 95\% confidence-level upper limit on the number density $\rho \lesssim 4 \times 10^{-6}\,{\rm Mpc^{-3}}$ consistent that found by \citet{Ofek2017ApJ...846...44O}.
\end{enumerate}

Follow-up high angular resolution imaging should allow to further describe these outlying radio sources.
Furthermore, searching these sources for high time resolution bursts may inform us about FRB progenitor.
If only a subsample of our candidates turn out to be active cases of persistent radio source associated to FRBs as is the case those presented by \citet[][]{Chatterjee2017Natur.541...58C} and \citet[][]{Niu2022Natur.606..873N}, it would possible to evaluate if these are indeed calorimeter allowing to estimate the energy output of the central FRB engine. 

We end by noting that due to LoTSS' unprecedented sensitivity to optically thin synchrotron sources in a wide-angle survey, we have been able to select interesting radio sources in dwarf galaxies. 
The proposed VLBI observations are a crucial step towards the discovery of a new population of either wind nebulae or black-holes in nearby dwarf galaxies---both outcomes being scientifically interesting.  

% \subsection{Reproducible pipeline on European Cloud infrastructure}

