\section{Discussion}
\label{sec:discussion}

\subsection{Cause of radio emission}
Given emission line ratios and spectral index measurements presented in \S\ref{sec:candidates}, what can be ascertained about the potential progenitors of our selected candidates? 

Markers below and to the left of the solid and dashed grey lines in Figure \ref{fig:bpt} indicate that the emission lines are due to star formation and not due to AGN activity~\citep{Kewley2001ApJ...556..121K, Kauffmann2003MNRAS.346.1055K}. 
Measurement uncertainties cannot definitively rule out an AGN contribution in three cases. 
For galaxies falling within the star formation region of this parameter space, ionizing flux is primarily provided by hot, massive, young stars and associated supernovae that are surrounded by H{\sc II} regions~\citep{Zajacek2019A&A...630A..83Z}.

A few cases occupy interesting regions of parameter space. 
ILT~J090406.54+530314.6 (white inner circles) sits above the grey line in the ${\rm S{\sc II}/H\upalpha}$ panel, while sitting well within the star formation range in the ${\rm [N{\sc II}]/H\upalpha}$ panel. 
ILT~J075257.15+401026.3 (blue inner circles) and ILT~J125940.18+275123.5 (pink inner circles) fall within the composite region between models from \citet{Kewley2001ApJ...556..121K} and \citet{Kauffmann2003MNRAS.346.1055K} in the ${\rm [N{\sc II}]/H\upalpha}$ panel. 
For these, we can expect a contribution from both star formation (H{\sc II} regions) and AGN activity. 

Similarly, galaxy UGC~04068 hosting ILT~J075257.15+401026.3 has been classified as an AGN by \citet{Veron2010A&A...518A..10V}, while the SDSS spectrum is simply being classified as a galaxy (rather than other considered classes in SDSS nomenclature, such as QSO). 
There is a bright star located near the centroid of the galaxy (slightly leftward in Figure \ref{fig:family_plot}) that may impact the overall flux observed in the spectrum, especially given the location of SDSS' spectrograph fiber (pink circle) almost exactly between the galaxy centroid and that of the bright star. 
It is also the only case within the candidates where the CLU catalogue contains a fitted H$\upalpha$ D25 measurement. 
CLU~J163850.64+352900.9 and 2MASX~J09133387+3000514 are unclassified in SDSS, while all other matches are classified as galaxies.

Finally, we note that ILT~J113634.77+592533.3 (cyan inner circles) has been classified as an AGN candidate by \citet{Truebenbach2017MNRAS.468..196T} by selecting sources detected in the AllWISE and FIRST catalogues, but not detected in 2MASS or SDSS DR7 and DR9. 
However, we note that the source may be matched to the galaxy SBS~1133+597, which has been observed by SDSS and for which the BPT diagram rather indicates that the driving source of ionization in the galaxy can be attributed to star formation. 

The value of $\alpha$ helps distinguish between optically thin and optically thick emission mechanisms.
We note that flux measurements from the archival surveys we used span several decades of observations, with FIRST and NVSS being the oldest, and RACS, VLASS and LoTSS being contemporaneous, but not simultaneous. 
Using archival data spanning many decades comes with the caveat that measurements may be affected by time-dependent phenomena like scintillation or source evolution.

The range of spectral indices covered by various source types is known to differ.
Pulsars have spectral indices $\lesssim-1.2$~\citep{Bates2013MNRAS.431.1352B}. 
Given the range of values of our candidates, they are unlikely pulsars. 
Moreover, we can assume that if an OCR (detected by LoTSS/FIRST/NVSS) is a radio pulsar, it would have to be galactic, and hence be an unrelated foreground object in a chance alignment with the background galaxy as LoTSS should not be sensitive to extra galactic pulsars. 
Supernova remnants tend to have spectral indices ranging between $-0.1$ to $-0.8$~\citep{Kothes2006A&A...457.1081K, Alvarez2001A&A...372..636A}. 
Six of our candidates fall within this range, including five cases with offset below $(2+\epsilon)\arcsec$. 
\citet{Planck2011A&A...536A..15P} showed that spectral indices of AGNs at low frequencies ($1.1-\leq70$\,GHz) are fairly flat, with an average of $-0.06$. 
Their distribution is narrow, with 91\% of the indices being in the range $\alpha\in[$-0.5$, 0.5]$. 
However, a few sources have remarkably steep spectra $\leq-0.8$,  while others have inverted spectra ($\alpha=0.86$).
Although our spectral indices are calculated at lower frequencies than these (we also computed where possible the spectral index at higher frequency between $1.4-3$\,GHz), they all fall within this broad range. 

Comparing spectral indices between 4.85 and 10.45\,GHz\footnote{These spectral indices taken at high frequencies cannot directly be compared to our results between $\geq$54\,MHz to $\leq$3\,GHz given the potential for flattening and/or turnover at lower frequencies that can be caused by synchrotron self-absorption or free-free absorption.} from a distribution of radio sources with optical counterparts, \citet{Zajacek2019A&A...630A..83Z} showed that the ionization potential of sources with an inverted radio spectrum ($\alpha>-0.4$) is weaker than that of sources with a steep radio spectrum ($\alpha<-0.7$). 
In particular, simultaneous two-point $\alpha$ measurement at 4.85 and 10.45\,GHz at Effelsberg highlighted that decreasing spectral indices from steep to flat ($-0.7<\alpha<-0.4$) to inverted leads to a decrease in typical line ratios (BPT diagram), particularly ${\rm [O{\sc III}]/H\upbeta}$.

\citet{Zajacek2019A&A...630A..83Z} considered radio loudness $R_g$ in addition to $\alpha$ and ionization ratio to highlight three distinct classes of radio emitters resulting from the recurrent nuclear jet activity, distributed along the transition from Seyfert to LINER sources in the optical diagnostic, namely sources with: 
    (class 1) steep $\alpha$, high ionization ratio, and high radio loudness; 
    (class 2) flat $\alpha$, lower ionization ratio, and intermediate radio loudness; 
    (class 3) inverted $\alpha$, low ionization ratio, and low radio loudness. 

To compare our results to those of \citet{Zajacek2019A&A...630A..83Z}, we computed $R_g$ using the flux density from LoTSS, $F_{144}$. 
We converted $F_{144}$ into the $AB_\nu$ radio magnitude system of \citet{Oke1983ApJ...266..713O}, according to \citet{Ivezic2002AJ....124.2364I}: $m_{1.4} = -2.5\log{F_{1.4} / 3631\,{\rm Jy}}$, in which the zero point 3631\,Jy does not depend on the wavelength, scaling fluxes from 144\,MHz to 1.4\,GHz using either our fitted spectral indices or $\alpha=-0.7$ otherwise. 
Subsequently, the radio loudness can be calculated as the ratio of the radio flux density to the optical flux density, $R_g \equiv \log{F_{\rm radio}/F_{\rm optical}} = 0.4 (g-m_{144})$, with $g$ being the magnitude in at $g$ band in the optical.
We use SDSS magnitudes in the g band where available, and g-band Kron magnitudes from PS1 where available otherwise.

We list values of $R_g$ in Table~\ref{table:candidates}. 
For our whole set of candidates, $R_g$ ranges between 0.05 and 2.48, with mean, median, and standard deviation of 1.21, 1.16 and 0.48, respectively.
Sources for which we can evaluate $R_g$ and emission line ratio between ${\rm [O{\sc III}]}$ and ${\rm H\upbeta}$ are shown in the upper panel of Figure~\ref{fig:loudness}, displaying $\alpha$ where possible using the colour map\footnote{We note that the point with $\alpha \sim -0.6$ shown in pink is calculated using detections at all available frequencies, and that its equivalent at higher frequencies corresponds to $-0.88$, as shown in Table~\ref{table:fluxes}.} and in grey otherwise. 
In addition, we show distributions of $R_g$ (central panel) and $\alpha$ (lower panel) for candidates with and without available optical spectra in grey and black, respectively.
We mark in the lower panel the split between steep ($\alpha<-0.7$, class 1, blue), flat ($-0.7 \leq \alpha \leq -0.4$, class 2, green), and inverted ($\alpha>-0.4$, class 3, pink) spectrum sources, as defined by \citet{Eckart1986A&A...168...17E} to reflect the distributions of the spectral index for samples of radio-loud galaxies between ${\rm 1.6-5\,GHz}$), also used to classify sources by \citet{Zajacek2019A&A...630A..83Z}.

The region occupied in the emission line ratio---loudness plane by our matched sources corresponds to that of sources classified as class 3 by \citet[][e.g., their Figure~12]{Zajacek2019A&A...630A..83Z} though with much lower $R_g$ (\citeauthor{Zajacek2019A&A...630A..83Z}'s lower bound on $R_g$ being $\sim0.7$). 
Their sample covered supermassive black holes ($\rm >10^5\,M_\odot$), explaining the loudness discrepancy where dwarf galaxies could be hosting IMBHs instead.
Their spectral index distribution (black, lower panel) rather point towards class 1 or 2, even when considering $\alpha$ evaluated at higher frequencies ($1.4-3$\,GHz; orange dotted border: sources with associated spectrum; blue border: all cases matched at high frequencies).
Unfortunately, given that only four sources shown in the top panel of Figure~\ref{fig:loudness} were matched in other surveys to evaluate a spectral index, these results provide only small number statistics.
Nevertheless, the initial information they carry points towards a mismatch between the primarily AGN-related sources studies by \citet{Zajacek2019A&A...630A..83Z}, which mainly fall along the demarcation line between Seyfert and LINER, and our candidates primarily located well within the star formation region of the BPT diagram---strengthening our hypothesis that the OCRs in our sample are not AGN. 


\begin{figure}
\resizebox{\hsize}{!}{\includegraphics{figures/loudness.pdf}}
% \includegraphics[width=8.5cm]{figures/loudness.pdf}
\caption{Comparing radio loudness $R_g$, emission line ratio ${\rm [O{\sc III}]/H\upbeta}$ and spectral index, where available. 
From the sample of 26 $R_g$ values (Table \ref{table:candidates}), 7 also have ${\rm [O{\sc III}]/H\upbeta}$ values available (Table \ref{table:BPT}, alias T2; black bars in central panel), and 9 have fitted $\alpha$ values (Table \ref{table:fluxes}, alias T3). 
HF indicates spectral indices fitted between 1.4 and 3\,GHz.
Class 1, 2, and 3 (c1, c2, c3) as defined by \citet{Eckart1986A&A...168...17E}.
Unfilled histograms in middle and lower panels represent all available `compact radio source-dwarf galaxy' matches below 3$\sigma$ on the L-SFR relation. 
}
\label{fig:loudness}
\end{figure}

\subsection{Volumetric density and rate of overluminous compact radio sources}
\label{subsec:volumetric}

The CLU catalogue consists of galaxies selected to be within 200\,Mpc and is complete to a flux limit of $10^{-14}\,$erg\,s$^{-1}$\,cm$^{-2}$. 
This limit corresponds to a dust-unobscured star formation rate of $\approx 1$\,M$_\odot$\,yr$^{-1}$ \citep{Cook2019ApJ...880....7C}.
The 200\,Mpc distance is small enough for us to assume a Euclidean geometry in our volume density calculations. 
Based on the 29 sources in Table 1, we can summarily compute a lower limit of $856\pm150$ sources\,Gpc$^{-3}$ ($1\sigma$ Poisson bounds in parentheses) for compact radio sources (on arcsecond-scales) above 0.8\,mJy at 144\, MHz that deviate by more than $3\sigma$ from the radio-SFR relationship. 
The limit is preliminary because VLBI observations are necessary to conclusively rule out a star-formation origin for our sources.

We can compare this rate to that suggested by \citet{Law2022ApJ...927...55L} for persistent radio sources associated with FRB progenitors. 
They compute a volume density of $50-10,000$\,Gpc$^{-3}$ for sources with a 1.4\,GHz radio luminosity greater than $10^{29}\,$\,ergs\,s$^{-1}$\,Hz$^{-1}$. 
A separate volume density computed by \citet{Ofek2017ApJ...846...44O} is close to the upper end of the density computed by \citet{Law2022ApJ...927...55L}. 
A source of $10^{29}$\,erg\,s$^{-1}$\,Hz$^{-1}$  luminosity placed at our survey horizon of 200\,Mpc would have a flux density of 2.2\,mJy. 
Given our survey completeness of 0.8\,mJy, such a source would be detected in our survey if were optically thin or if it had an inverted spectrum  with a spectral index shallower than $\approx 0.4$ (i.e a relatively flat spectrum). 
However, our survey is also sensitivity to nearer sources with much lower flux densities. 

To account for this, we assumed that the FRB-related PRS sources follow a Schechter luminosity function \citep{1976ApJ...203..297S} with an exponent of $-1$ and a cut-off luminosity that is ten times the \citet{Law2022ApJ...927...55L} normalisation point of $10^{29}$\,erg\,s$^{-1}$\,Hz$^{-1}$. 
We also assumed that the sources have a flat spectrum and computed the number of sources that will exceed the 0.9\,mJy completeness limit of our survey.
We then numerically computed the expected number of detectable sources within a 200\,Mpc.
For the $50-10,000$\,Gpc$^{-3}$ range specified in \citet{Law2022ApJ...927...55L}, we expect to detect $0.3-58$ sources which is consistent with our yield of 28 candidates. 
Values of cut-off luminosity in excess of $10^{30.5}\,$erg\,s$^{-1}$\,Hz$^{-1}$ are necessary to create a tension between our yield and the \citet{Law2022ApJ...927...55L} rates. 
While the consistency is heartening, we caution against drawing strict conclusions because of the unknown FRB beaming fraction on which the \citet{Law2022ApJ...927...55L} estimate is based, along with the disparate selection filters that their and our analysis have necessarily had to apply.  



\subsection{Star formation vs. active (intermediate mass) black holes}

\citet{Condon2019ApJ...872..148C} evaluated the luminosity functions for sources whose radio emission is dominated by star formation and AGNe respectively. 
After scaling the luminosity of our candidates to 1.4\,GHz using either the spectral indices evaluated in \S\ref{subsec:spectral} or $\alpha\approx-0.7$ for typical synchrotron spectra of optically thin radio sources for the remaining sources, we find luminosity values ranging between $10^{17.9}$ to $10^{22.7}$\,$\mathrm{W\,Hz^{-1}}$, with a median of $10^{19.9}$\,$\mathrm{W\,Hz^{-1}}$. 
Compared to the luminosity functions for star formation and AGNe of \citet{Condon2019ApJ...872..148C}, the radio emission from our candidates can be more likely attributed to star formation. 

A critical step towards establishing the candidates presented in previous sections as potential FRB hosts is to conclusively determine the compactness of these sources. 
For this purpose, we have obtained time on the European VLBI network (EVN) and e-MERLIN to observe the most likely candidates.
Compactness in LoTSS images only ensures a brightness temperature of $\gtrsim 10^4\,{\rm K}$, which is insufficient to rule out unusually radio-bright star formation as the cause of the radio emission. 
Moreover, our sources may have a significant component of their radio flux attributed to star formation with the rest in a compact source component. 
VLBI at $\lesssim 10\,$mas resolution is therefore the best observational technique to totally eliminate (i.e. resolve out) the star-formation component and establish the presence of a compact source. 
Furthermore, including the e-MERLIN array should allow to disentangle between compact and star-formation components, if present.
In addition, we are in the process of re-imaging the archival LoTSS radio data on these sources while including the international stations from LOFAR. 
The resulting images should have a resolution of about $0\farcs25$ \citep[LOFAR-VLBI;][]{Morabito2022A&A...658A...1M}.
These higher angular resolution images should inform about the following possible outcomes.

If a target is not detected at very high resolution then it will confirm the star-formation hypothesis. 
This will be a rather unusual conclusion as the selected targets all violate the radio-SFR relationship, which would cast doubts on the canonical radio-AGN selection technique that is widely used \citep{Davis2022MNRAS.511.4109D}.
It is possible that AGN-related flux is present on intermediate scales of a few 100\,mas that are inaccessible to the EVN, but should be accessible by the intermediate scale of e-MERLIN. 
Moreover, in such a case, it should also be possible to be detected with LOFAR long baseline data.

If a core-jet structure is detected, it will confirm the AGN-like IMBH hypothesis. 
Although the radio detection of black-hole jet candidates in dwarf galaxies based on the procedure mentioned in \S\ref{sec:method} is now becoming feasible \citep{Davis2022MNRAS.511.4109D}, confirmatory VLBI detection of the jet (or structure thereof) is rare \citep{Paragi2014ApJ...791....2P, Yang2020MNRAS.495L..71Y, Eftekhari2020ApJ...895...98E}. 
As such a confirmation of the AGN hypothesis will have interesting scientific impact on studies of feedback in dwarf galaxies.

If an unresolved point is detected, although it would rule out the star-formation hypothesis, both the PWN and unresolved AGN would remain plausible---even if the source proves to be slightly ($\ll 1\arcsec$) offset from the optical stellar light centroid. 
Based on the known properties of starburst galaxies, a detection on EVN long baselines should exclude star formation as the cause of the bulk of the radio emission~\citep{Condon1991ApJ...378...65C}.
Here, a path forward would be to follow-up such sources to model their broad-band spectral energy distribution (e.g., with optical spectroscopy directly on-source to search for canonical AGN signatures, and with radio observations at C, X and K bands) to decipher between the PWN and AGN hypotheses.

\subsection{Future search for FRBs}

Finally, we also plan to search these targets for millisecond-duration bursts with the 25-meter Westerbork Synthesis Radio Telescope. 
Starting with the hypothesis that some are similar in nature to currently known PRSs, we can expect these to be repeating FRB sources.
Furthermore, given the periodic activity of some FRBs like \RI~\citep{Cruces2021MNRAS.500..448C, Rajwade2020MNRAS.495.3551R} and FRB\,20180916B~\citep{Chime2020Natur.582..351C}, it is plausible that a subset of our candidates can also display on/off phases of FRB emission.
