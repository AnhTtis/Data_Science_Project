\begin{appendix}

\section{On the obtained L-SFR distribution}
\label{app:l_sfr}

The zero point on the L-SFR relation from \citet{Gurkan2018MNRAS.475.3010G} was obtained on more massive ($\sim10^{8.5}$--$10^11\,M_\odot$) galaxies than our dwarf sample. 
In figure \ref{fig:l_sfr_plane}, we show that the zero point corresponds to the peak of the distribution when matching compact radio sources to all galaxy masses available within CLU using a $6\arcsec$ matching radius. 
The peak of the distribution of matched dwarf galaxies falls just below the zero point.
Objects below $-10\sigma$ (14 dwarf galaxies) have redshifts below 0.002, except two extra cases at higher masses having redshifts of 0.003 and 0.027 respectively. 


\begin{figure}
    \centering
    \includegraphics[width=17cm]{figures/l_sfr_plane.pdf} 
    \caption{
    Matched compact radio objects along the L-SFR plane. 
    Grey-filled bins indicate cases matched to dwarf galaxies. 
    Grey border bins indicate cases matched to galaxies at all masses available in CLU.
    Blue bins indicate OCRs.
    Very low luminosity cases encountered in Figure \ref{fig:selection} are further investigated in Figures \ref{fig:high_sigmas_vs_redshift_offset_dwarfs}. 
    }%
    \label{fig:l_sfr_plane}%
\end{figure}

\begin{figure}
    \centering
    \includegraphics[width=17cm]{figures/high_sigmas_vs_redshift_offset.pdf}
    \caption{Compact radio objects matched to dwarf galaxies (left) and all galaxies (right) along the L-SFR plane as a function of redshift and offset. 
    Very low luminosity cases encountered in Figure \ref{fig:selection} all have redshifts below 0.002.}%
    \label{fig:high_sigmas_vs_redshift_offset_dwarfs}%
\end{figure}


\section{2MASX~J12594007+2751177 and SDSS~J143037.09+352052.8}
\label{sec:appendix1}
The dwarf galaxy 2MASX~J12594007+2751177 was shown to be of very low surface density in PS1 (Figure \ref{fig:family_plot_continued}). To better highlight the environment in which this galaxy resides, Figure \ref{fig:HSC} shows the same field as observed by PS1 in r filter (right) as shown previouly, and JVO Subaru/Suprime-Cam~\citep{Aihara2019PASJ...71..114A} composite image (left) using all filters---where 2MASX~J12594007+2751177 is apparent. 
Green markers (x, +) show the location of the LoTSS and CLU detections, respectively.
The Subaru Suprime-Cam image strengthens the hyposthesis that 2MASX~J12594007+2751177 and ILTJ125944.53+275800.9 are unrelated.

\begin{figure}
    \centering
    \includegraphics[width=17cm]{figures/2MASXJ12594007+2751177_aladin.png} 
    \caption{Environment around 2MASX~J12594007+2751177 as observed by PS1 in r filter (right, Figure \ref{fig:family_plot_continued}), and JVO Subaru/Suprime-Cam composite image (left) using all filters---where 2MASX~J12594007+2751177 (pink markers) is apparent.
    Green markers ($\times$, +) show the location of the LoTSS and CLU detections, respectively.
    Figure generated with Aladin Desktop~\citep{Bonnarel2000A&AS..143...33B}.
    }%
    \label{fig:HSC}%
\end{figure}

Finally, the AGN candidate ILT~143037.30+352053.3 was matched to the host galaxy SDSS~J143037.09+352052.8. 
SDSS~J143037.09+352052.8 is not resolved in PS1 r filter shown in Figure \ref{fig:family_plot_continued}. 
We found a JVO Subaru/Suprime-Cam image for this field which indicates the radio source to be within the optical footprint of the host, which we show in Figure \ref{fig:agn}.

\begin{figure}
    \centering
    \includegraphics[width=17cm]{figures/agn_sdss.png} 
    \caption{Environment around SDSS~J143037.09+352052.8 as observed by PS1 in r filter (right, Figure \ref{fig:family_plot_continued}), and JVO Subaru/Suprime-Cam composite image (left) using all filters---where SDSS~J143037.09+352052.8 (pink markers) is apparent.
    Green markers ($\times$, +) show the location of the LoTSS and CLU detections, respectively.
    }%
    \label{fig:agn}%
\end{figure}

\section{ACO~1656}
\label{sec:appendix2}

We noted in \S\ref{sec:candidates} that four of our selected compact radio sources fall within galaxies that are members of the cluster of galaxies ACO~1656. These are galaxies and respective matched radio sources (Table~\ref{table:candidates}) are 2MASS J13002220+2814499 (ILTJ130022.42+281451.7), 
SDSS J125944.76+275807.1 (ILTJ125944.53+275800.9), 
2MASX J12594007+2751177 (ILTJ125940.18 +275123.5), and 
SSTSL2 J125915.27+274604.1 (ILTJ125915.34+274604.2). 
We show the position of these galaxies along with the central coordinate of ACO~1656 in Figure \ref{fig:HSC}, which displays a PS1 composite image of the z and g filters, as plotted in Aladin Desktop~\citep{Bonnarel2000A&AS..143...33B}.

\begin{figure}
    \centering
    \includegraphics[width=17cm]{figures/ACO1656_aladin.png} 
    \caption{The cluster of galaxies ACO~1656 (pink markers), and four dwarf galaxies matched to compact radio sources (pink squares).}%
    \label{fig:HSC}%
\end{figure}

\section{Host properties}
\label{app:host_properties}

In Table \ref{table:candidates}, a number of physical properties are listed, including luminosity of the compact radio source, as well as redshift and star formation rate of the host galaxy. 
Here, we also present the distribution relative to stellar mass and specific star formation rate (sSFR, the star formation rate normalized by the stellar mass) in Figure \ref{fig:mstar_vs_sfr_ssfr}. 
Similarly, Figure \ref{fig:mstar_vs_sfr_ssfr} shows SFR and sSFR in relation to stellar mass. 
Here, we note that none of our candidates reach quite the same level of SFR as the host galaxies from \RI~\citep{Tendulkar2017ApJ...834L...7T} and \RItwin~\citep{Niu2022Natur.606..873N}, with \RItwin being within range.
\RI's host galaxy stands out with respect of its sSFR compared to the rest of the data points. 
Host galaxies of \citeauthor{Reines2020ApJ...888...36R}'s IMBH candidates fall within the SFR and sSFR ranges of our candidates. 
Two-sample KS tests between the respective candidates distributions to that of the background sample (`compact radio sources-dwarf galaxy' matches below 3$\sigma$ on the L-SFR relation) for stellar mass, SFR, and sSFR yield $p$-values of 0.2, 0.8 and 0.4 respectively, and hence are consistent with being drawn from the same population.


\begin{figure}
    \centering
    \includegraphics[width=17cm]{figures/mstar_sfr_ssfr.pdf} 
    \caption{Distribution of host stellar mass (left panel), star formation rate (middle panel), and specific star formation rate (right panel) for candidates listed in Table \ref{table:candidates}.
    Grey bars indicate values for all `compact radio source-dwarf galaxy' matches below 3$\sigma$ on the L-SFR relation.
    }%
    \label{fig:mstar_sfr_ssfr}%
\end{figure}

\begin{figure}
    \centering
    \includegraphics[width=17cm]{figures/mstar_vs_sfr_ssfr.pdf} 
    \caption{Host stellar mass as a function of star formation and specific star formation rate for our candidates. We also indicate values for \RI, \RItwin, and \citet{Reines2020ApJ...888...36R} sources from Figure \ref{fig:selection}.
    Grey markers indicate all `compact radio source-dwarf galaxy' matches below 3$\sigma$ on the L-SFR relation.
    }%
    \label{fig:mstar_vs_sfr_ssfr}%
\end{figure}

\section{Radio luminosity vs ${\sc OI}/{\rm H\upalpha}$}

\citet{Reines2020ApJ...888...36R} showed that the relation between the emission line ratio $[{\sc OI}]_{\lambda3600}/{\rm H\upalpha}$ and the luminosity at 9\,GHz could be utilise to separate radio emission from radio AGN and star formation. 
We used the spectral indices fitted in \S\ref{subsec:spectral} (and the canonical $\alpha= -0.7$ otherwise) to scale the luminosity of our sources to 9\,GHz for sources with spectral line measurements available in SDSS. 
We show the results in Figure \ref{fig:L_OI_Halpha}. 
Most of our sources fill the gap region found in Figure 10 of \citet{Reines2020ApJ...888...36R} dividing between radio emission consistent with star formation and radio AGN. 
We note that, despite the fact that scaling flux from 144\,MHz to 9\,GHz remains an approximation, a typical 10\% of the flux being resolved out at higher frequencies would only marginally affect the luminosity of sources plotted here.

\begin{figure}
    \centering
    \resizebox{\hsize}{!}{\includegraphics{figures/L_OI_Halpha.pdf}}
    \caption{
    Black markers indicate values where the luminosity was scaled to 9\,GHz with spectral indices obtained in \S\ref{subsec:spectral}, while grey markers were scaled using the canonical $\alpha=-0.7$ for synchrotron spectra of optically thin radio sources. 
    }%
    \label{fig:L_OI_Halpha}%
\end{figure}


\section{Spatial coverage of ancillary surveys}
\label{app:coverage}

In this section, we evaluate the spatial coverage of radio surveys utilized in \S\ref{subsec:spectral}. Figure \ref{fig:mocs} shows multi-order coverage (MOC) maps for each of these surveys, seen as subdivided cells. 
Each survey is color coded and described in the figure caption. 
In addition, we overplot our candidates using yellow `+' markers. 
We note that 7 sources have not been observed by FIRST and 7 have been observed by RACS (Figure \ref{fig:spectral_indices}). 
All other sources could have been observed by the remaining surveys.
Finally, we also show coverage for SDSS DR12 (\S\ref{subsec:bpt}). 

\begin{figure}
    \centering
    \includegraphics[trim={0 0.5in 0 0.5in},clip,width=17cm]{figures/lotss_ocr.pdf} 
    \caption{Coverage maps of surveys searched in \S\ref{sec:candidates}.
    Source candidates discussed in this paper are shown as white circles.
    }%
    \label{fig:mocs}%
\end{figure}

\end{appendix}