\documentclass[aps,prl,twocolumn,nopacs,superscriptaddress,longbibliography]{revtex4-1}

\usepackage[breaklinks=true,colorlinks,citecolor=blue,linkcolor=blue,urlcolor=blue]{hyperref}
\usepackage{epsfig,mathrsfs,color,latexsym,subfigure,marginnote,graphicx,verbatim,relsize,mathrsfs,color,array,amsmath,amsfonts,amssymb,graphicx}
\pagestyle{headings}

\newcommand{\bs}[1]{\textcolor{blue}{[BS]\,[#1]}}
\newcommand{\rpl}[1]{\textcolor{blue}{ #1}}

%%%%%%%%%%%%%%%%%%%%%%%%%%%%%%%%%%%%%%%%%%%%%%%%%%%%%

\begin{document}


\title{High-$T_c$ superconductors as a New Playground for High-order Van Hove singularities and Flat-band Physics}


\author{Robert S. Markiewicz}
\email{r.markiewicz@northeastern.edu}
\affiliation{Department of Physics, Northeastern University, Boston, Massachusetts 02115, USA}

\author{Bahadur Singh}
\email{bahadur.singh@tifr.res.in}
\affiliation{Department of Condensed Matter Physics and Materials Science, Tata Institute of Fundamental Research, Colaba, Mumbai 400005, India}

\author{Christopher Lane}
\affiliation{Theoretical Division, Los Alamos National Laboratory, Los Alamos, New Mexico 87545, USA}
\affiliation{Center for Integrated Nanotechnologies, Los Alamos National Laboratory, Los Alamos, New Mexico 87545, USA}

\author{Arun Bansil}
\affiliation{Department of Physics, Northeastern University, Boston, Massachusetts 02115, USA}

\begin{abstract}

Beyond the two-dimensional (2D) saddle-point Van Hove singularities (VHSs) with logarithmic divergences in the density of states (DOS), recent studies have identified higher-order VHSs with faster-than-logarithmic divergences that can amplify electron correlation effects. Here we show that the cuprate high-Tc superconductors harbor high-order VHSs in their electronic spectra and unveil a new correlation that the cuprates with high-order VHSs display higher T$_c$’s. Our analysis indicates that the normal and higher-order VHSs can provide a straightforward new marker for identifying propensity of a material toward the occurrence of correlated phases such as excitonic insulators and supermetals. Our study opens up a new materials playground for exploring the interplay between high-order VHSs, superconducting transition temperatures and electron correlation effects in the cuprates and related high-Tc superconductors.
\end{abstract} 

\maketitle

{\it Introduction.$-$} Many properties of materials are driven by their one-particle electronic density of states (DOS). A large DOS at the Fermi level implies that many electrons contribute to the low-energy phenomena so that many-body interactions are enhanced. In particular, extrema and saddle-points in band dispersions induce Van Hove singularities (VHSs) in the DOS \cite{VHS} and the associated logarithmic divergences in two-dimensions (2D) have been a focus of interest for many years. Recent studies, however, show that 2D and 3D VHSs can be anomalously strong in some cases to yield higher-order VHSs with power-law divergences \cite{Liang, Liang3, MBMB, Marsig} that can further amplify many-body interaction and drive exotic correlated phenomena such as supermetals \cite{Liang2}. Such high-order VHSs have been reported in Moire heterostructures, slow-graphene, magic-angle twisted bilayer and trilayer graphene, Sr$_3$Ru$_2$O$_7$, and other systems that host flat bands and reduced bandwidths \cite{Liang,Tgraphene,Ref1b,Ref2a,Ref2b,SRO327}. Beyond classifying the DOS anomalies of higher-order VHSs, it is important to understand their complex effects to fully appreciate their general promise of driving correlated physics in materials. 

The role of normal VHSs in the cuprates and the related high-$T_c$ superconductors has been debated for decades. One of the earliest theories of cuprate superconductivity is that it is driven by a large DOS associated with the VHSs \cite{HirSc}. However, the connections between the cuprate superconductivity, doping, and VHSs are complex and material dependent. Thus, while in the lanthanum-based cuprates, the VHS is at the Fermi level near optimal doping, in many cuprates the VHS crosses the Fermi level well into the overdoped regime. 

Over the past 35 years, evidence for what is now known as high-order VHSs has arisen in cuprates in a number of contexts, starting with an experimental observation\cite{Ref1c} in YBa$_2$Cu$_3$O$_7$ (YBCO) of a high-order VHS in what may be an unusual surface state wherein the surface remains overdoped and nonsuperconducting independent of the bulk doping\cite{Damasc1}. Further observations include Andersen's extended VHS\cite{Andex} with power-law diverence $p_V=-\partial N/\partial E$ =0.25\cite{Zehyer} (where $N(E)$ is the DOS) and a high-order VHS which may be associated with pseudogap collapse \cite{Michon}, consistent with an earlier prediction \cite{RM70}. We hope that this systematic analysis of the many ways high-order VHSs affect the cuprate phase diagram will provide guidance for their study in other materials, while shedding light on some of the many puzzles remaining in cuprates.  

Here, we explore the existence of high-order VHSs in the cuprates and discuss how such VHSs can drive competing orders and other complex effects in materials. We find that VHS singularities can exist not only in the DOS or the $Q=(0,0)$ susceptibility but also in the susceptibility at finite  momenta (e.g. $Q\sim(\pi,\pi)$). These two VHSs compete with each other and undergo independent evolutions when the material is doped or its dispersion is tuned. By tuning the dispersions, one can generate flat bands with high-order VHSs that lead to frustration rather than instabilities. We also show the existence of high-order VHSs in bosonic bands and discuss that if the susceptibility is considered as a dispersion of electron-hole pair bosons, the resulting high-order bosonic VHSs would allow a straightforward identification of excitonic phases in materials.  Importantly, we demonstrate a correlation between the superconducting transition temperature $T_c$ and the strength of the associated VHS.

Beyond the cuprates, our study has important implications in several areas of high current interest as follows.  (1) We show how the definition of higher-order VHSs can be extended to reveal its connection with the broad area of {\it flat-band} physics. (2) We demonstrate a connection between higher-order VHSs and the electronic dimensionality. (3) We show that the connection with superconductivity is indirect, implying that additional factors are involved, most likely associated with competing instabilities.


{\it High-order VHSs$-$} We begin by recalling that the energy dispersion in the cuprates can be approximated by a $t-t'-t''$ model \cite{PavOK,MBMB}

\begin{equation}
E=-2t(c_{x}+c_{y})-4t'c_{x}c_{y}-2t''(c_{2x}+c_{2y}),
\end{equation}
where $c_{nr} = cos(nk_ra)$, $a$ is the lattice constant, $r = \{x, y\}$, and the hopping parameters are defined in the inset to Fig.~\ref{fig:1vhs}(a).  In this model, $t$ sets the energy scale and thus, the evolution of energy dispersions depends on the two parameters, $t'/t$ and $t''/t$ which constitute the material dependence. We examine this energy dispersion considering $t'/t$ and $t''/t$ values relevant to cuprates and delineate high-order VHSs. Our choices of $t'$ values for different cuprates is discussed in the Supplementary Material (SM) Section S-I.  We consider an average DFT value of $t=-0.5$ eV for all the calculations.  It should be noted that nearly localized $d$- and $f$-electrons may be sensitive to just a few hopping parameters so that similar calculations should determine the characteristic properties of high-order VHSs in many correlated materials beyond cuprates. 


Figure \ref{fig:1vhs}(b) presents the evolution of VHS lineshapes with $t'$ for the special value $t''=-t'/2$ that best describes most families of cuprates \cite{PavOK,MBMB}.  The corresponding VHS peak is considered as a marker of $t'$ which locates various cuprates along the $x$-axis. Such a correlation of $t'$ with the VHS allows us to compare the strength of the VHS divergence with $T_c$ for several families of cuprates. The vertical black dashed and dotted lines in Fig.~\ref{fig:1vhs}(b) delineate the range
over which cuprates with $T_c> 80K$ are found, while cuprates to the right of the dotted line have $T_c \le 80K$.  Strikingly, the high-$T_c$ cuprates
are all clustered to one side of the hoVHS, with $T_c$ increasing as the hoVHS is approached, with one exception.  The Bi-cuprates are the only cuprate family in which $T_c$ changes significantly with number of CuO$_2$-layers per unit cell, even though all have similar $t'$-values. The present results suggest a possible explanation for this anomaly.  Thus Bi2201, with maximum $T_c$ of 40K, has the highest magnitude of $t'$, placing it very close to the hoVHS -- or perhaps beyond it, given some uncertainty in $t'$.  The low $T_c$ could then follow from $|t'|$ being too large.  In Bi2212, bilayer splitting causes the effective $t'$ values to split (SM S-I), pushing the antibonding band into the range for its $T_c=$90K.  

Note further that for bilayer cuprates, the correlation holds for the antibonding band that is closer to the Fermi level, whereas the bonding bands all have $t'$s that correspond to super VHSs.  These results clearly suggest that high-order VHSs play a significant role in the superconductivity of the cuprates.
 
To further understand the evolution of VHSs, it is convenient to measure energy $E$ from the energy of the $X =(\pi,0)$ point, $i.e.$, $E_X=4 (t'-t'')$. As seen in Fig.~\ref{fig:1vhs}(c), there is always a VHS at $E_X$, which evolves from logarithmic (saddle-point) at small $t'$ to a step at larger $t'$, where the associated single Fermi surface changes into a region of Fermi surface with three pockets. The step is the point at which two pockets first appear for a given $t'$.   The crossover occurs at a critical value $t'_c$ where a pocket forms with the strongest VHS divergence (for that $t''/t'$). It has a step on the low-energy side and a power-law divergence on the high energy side.  This evolution is further illustrated by replotting the data for $E>E_X$ on logarithmic scales in Fig.~\ref{fig:1vhs}(d).  There are two types of behavior, separated by the turquoise line.  For small $|t'|$ (red to turquoise curve), the divergent peak stays at $E_X$, evolving from logarithmic to power law.  The turquoise curve has the largest, pure power-law divergence.  For larger $|t'|$, all curves (black to turquoise) start off with the same power law growth at large $\delta E=E_f-E_X$, but as $\delta E$ decreases, the curves gradually split off on realizing a power-law to  logarithmic crossover at an energy away from $E_X$. Finally, we note from Fig.~\ref{fig:1vhs}(a) that the $t'$ of strongest VHS can be approximately determined from the dispersion, as corresponding to the flattest band near $(\pi,0)$.  Such a strong VHS corresponds to Andersen's extended VHS \cite{Andex}.  The critical value of $t'$ has a simple analytic interpretation, of maximizing the degree to which the fermi surface at the VHS is tangent to the $x$- and $y$- axes -- i.e., maximizing the one-dimensionality.  Thus, for the VHS at $(0,\pi)$, $\partial E/\partial (ak_x) = 2s_x(t+2t'c_y+4t''c_x)\rightarrow 2s_x(t+2t'-4t'')$, where $s_x=sin(k_xa)$.  This vanishes when 
\begin{equation}
t'_c/t = - 1/(2-4t''/t')=-1/4,
\label{eq:2}
\end{equation}
close to the Bi cuprates.

\begin{figure}[h!]
\centering
\includegraphics[width=0.99\columnwidth]{fig1_update_v2.png}
\caption{{\bf higher order VHSs in the cuprates ($t''=-t'/2$).} (a) Dispersions for $t'$ = -0.12 (dark red), -0.17 (red dashed), -0.258 (light green -- the hoVHS), -0.32 (violet dashed), and -0.33 (blue dashed).  Inset: Definition of the hopping parameters $t$, $t'$, and $t''$.  (b) DOS $N(E)$ for several values of $t'$.   As VHS moves from right to left, the white-background curves correspond to $t'/t$ = 0 (red curve), -0.1 (orange), -0.2 (yellow-green), -0.25 (green), -0.258 (light blue), -0.3 (blue), -0.4 (violet),  and -0.5 (black), while the colored-background curves correspond to the monolayer cuprates as indicated in the legends with $t'$ values from SM S-I.  Horizontal bars indicate range of VHS peak positions for 10 bilayer or trilayer cuprates\cite{PavOK}, sorted by optimal $T_c$s, with red ($70K\ge T_c\ge 50K$), green ($100K\ge T_c\ge 90K$), and blue ($135K\ge T_c\ge 125K$) colors. The antibonding bands are above and the bonding bands below. The length of the bars indicates the range of energies over which the corresponding VHS peaks would occur.  A clear correlation of high-order VHSs with higher superconducting $T_c$s is seen.  (c,d) White-background data from frame (b) reploted as $E_F-E_X$, on linear (c) or logarithmic (d) scales.  } 
\label{fig:1vhs} 
\end{figure}




{\it VHS Dichotomy $-$} However, our problem is far from solved.  The instabilities associated with the VHSs form a Lie group which is SO(8) for cuprates.\cite{Mikefest} For our purpose, the most important subgroup is whether the instability involves intra-VHS coupling that corresponds to $q=0$ ({\it i.e.}, a peak in the DOS) or inter-VHS coupling that yields a peak in the $Q=(\pi,\pi)$ susceptibility. These two unstable modes compete such that in the original Hubbard model the $(\pi,\pi)$ instability at half-filling is a (well-known) high-order VHS with a $ln^2$ instability which dominates the $ln$ DOS. Only with finite doping does the $(\pi,\pi)$ peak weaken as the DOS peak increases and begins to play an important role.  Thus, focusing solely on the DOS would miss the strong antiferromagnetism of cuprates. The existence of this $ln^2$ effect has been questioned since the Hubbard model requires extreme fine-tuning with all hopping parameters set to zero except nearest-neighbor $t$. However, in SM S-IV, we display a large family of dispersions with $ln^2$ susceptibility divergence.


Remarkably, the $(\pi,\pi)$ VHS is completely insensitive to the Fermi surface nesting that produces structure in the DOS, only gradually crossing over from $ln^2$ to $ln$ as the dispersion is tuned away from the Hubbard limit by either doping or tuning hopping parameters.  Hence, in general, at some hopping $t'_{cross}$ the DOS instability will become dominant.  This is accompanied by a doping $x_{cross}$ where the dominant near-$(\pi,\pi)$ instability crosses over to a near-$\Gamma$ instability (SM S-II.B). We find that optimal superconductivity falls close to $t'_{cross}$ (SM Fig.~S2) where superconductivity can tilt the balance between the two competing phases, and suggest similar behavior at $x_{cross}$.


{\it Secondary VHSs.$-$} The above analyses do not exhaust the possibilities for high-order VHSs. When a phase transition opens a gap in the electronic spectrum, the resulting subbands each develop secondary saddle-point VHSs, with corresponding new families of higher order VHSs having properties that can be quite different from the primary VHSs discussed above. Here we provide two examples of such secondary VHSs.  
Firstly, let us consider the DOS in the mean-field antiferromagnetic (AFM) phase in a pure Hubbard model ($t'=t''=0$), where the secondary VHS displays strong frustration.  Figures~\ref{fig:3b}(a) and (b) show that the DOS has a strong power-law divergence associated with quasi-1D nesting along the Brillouin zone diagonals, with enhanced flatness near $(\pi,0)$ and $(0,\pi)$.  Notably, when $t'$ is non-zero this evolves into a Mexican hat dispersion, with a local maximum at $(\pi/2,\pi/2)$, SM S-III.A.  With increased doping, the AFM gap closes near the point where the high-order VHS of the lower magnetic band crosses the fermi level, leading to a discontinuous loss of AFM order (see Ref.~\onlinecite{RSMstripe} and Fig.~17 of Ref.~\onlinecite{RM70}).  Such a scenario was recently seen experimentally\cite{Michon}, but the VHS interpretation was discarded because the feature was too intense to be a conventional logarithmic VHS \cite{Michon,Horio}.


Secondly, we investigate the case of an excitonic insulator where electron and hole pockets bind together to form avoided crossings at the Fermi level. While the electron and hole pockets can have different geometries or be shifted by a fixed $q$, for simplicity, we consider pockets of the same area and shape, with $q=0$. Adding a hybridization term to such a model leads to an avoided crossing with a `Mexican-hat' dispersion (Fig.~\ref{fig:3b}(c)). From Fig.~\ref{fig:3b}(d) it is seen that this dispersion leads to a highly characteristic DOS in either hybridized band. The logarithmic VHSs of the original bands at $\pm 1.5$eV remain unchanged by hybridization whereas the band-edge VHSs split into two VHSs- a step and a power-law VHS as one moves towards the band edge. The power-law saddle points are associated with an unusual Higgs-like one-dimensionality in the bands where they are flat along the brim of the Mexican hat but form an extremum along the radial $k$-direction away from $(\pi,\pi)$. Similar avoided crossings are found in TiSe$_2$\cite{singh2017} and often seen in topological insulators with a band inversion \cite{Bansil2016}.    

\begin{figure}[h!]
\centering
\includegraphics[width=0.99\columnwidth]{./fig3_update.png}
\caption{{\bf Emergence of secondary VHSs} (a) $(\pi,\pi)$ Antiferromagnetic Hubbard model $(t'=t''=0)$ at half-filling with bare dispersion (red line) and gapped dispersion with mean-field gap parameter $\Delta= 0.5$~eV (blue curves). The width of the blue curve indicates spectral intensity. (b) The associated DOS. (c) Excitonic insulator model with bare dispersion (red line) and gapped dispersion with mean-field gap parameter $\Delta= 0.4$~eV (green curves) and (d) the resulting DOS. (e) Panels (b) and (d) are replotted on a {\it ln-ln} plot to show the high-order VHSs. The dashed black and red lines provide the reference slopes of -0.54 and -1/2, respectively.}
\label{fig:3b}
\end{figure}

We emphasize that these VHSs can be dubbed as Overhauser VHSs \cite{Over} since his model of charge density waves involves singular interactions on a 3D spherical Fermi surface. The circular VHSs would therefore provide a realistic 2D version of this effect. In this example, a flat band leads to strong frustration which can greatly lower the transition temperature. We find in Fig.~\ref{fig:3c}(e) that the secondary VHSs have distinct power laws ($p_V = 0.5, 0.54$) from the primary VHS ($p_V = 0.29$).  These exceptionally strong divergences satisfy the criteria required in high-order VHSs. The secondary high-order VHSs thus can be used as a signature of excitonic instabilities in materials. This is reasonable since the optical spectra (i.e. the joint DOS) of many semiconductors and insulators are dominated by prominent VHSs and an analysis of their associated dispersion geometries would ease the identification of excitonic states \cite{RSMdd, Liang4,JCP}.  Further details on both these secondary VHSs are found in SM Section S-III.

 
\begin{figure}[h!]
\centering
\includegraphics[width=0.99\columnwidth]{flsup1mg6a2_tp_MCcor_L2.pdf}
\caption{{\bf Cuprate bosonic dispersion $\omega_q$.}    Dispersion at four different temperatures: T = 300 (blue line), 200 (pale blue), 100 (pink), and 10K (red). Inset: Blowup near band minimum.}
\label{fig:3c}
\end{figure}

 
{\it High-order VHSs in Bosonic systems.$-$} Since VHSs occur in bosonic systems, one can ask the question if those VHSs can be of high-order. Here we demonstrate that bosonic VHSs carry features similar to the secondary VHSs discussed above. Reference~\onlinecite{MBMB} introduced the idea of a susceptibility density of states (SDOS) and showed its usefulness in mode-coupling theory and as a map of Fermi surface nesting. There has been recent interest in interpreting this susceptibility as a bosonic Green's function for electron-hole pairs\cite{Bosemet,spinon}.  We therefore consider $\chi_0(q,\omega)=\frac{1}{(\omega +\omega_{q0}-i\gamma_q)}-\frac{1}{(\omega-\omega_{q0}+i\gamma_q)}$, where $\omega_{q0}$ is a bosonic (electron-hole) frequency and $\gamma_q$ a damping rate. For $\omega\rightarrow 0$, $\chi_0$ becomes real and $\gamma_q\rightarrow 0$, so $\chi_0^{-1}=\omega_{q0}/2$ which gives the bosonic DOS up to a factor of 2. Coulomb interaction renormalizes $\omega_{q0}$ to $\omega_q=\omega_{q0}-2U$, where $U$ is the Hubbard interaction, Fig.~\ref{fig:3c} (see SM S-III.D). The dispersion in Fig.~\ref{fig:3c} looks unusual because the electronic susceptibility contains nonanalytic features at $T=0$ due to Fermi surface nesting, which also show up in Kohn anomalies of phonons \cite{RSMch}.  Interestingly, we find that both the Mexican hat and drumhead (flat-band) dispersions exist in Bosonic systems and give rise to high-order VHSs similar to the electronic case shown in Fig.~\ref{fig:3b} or (Fig.~2 of Ref.~\onlinecite{MBMB}). 

The bosonic high-order VHSs are particularly appealing in cuprates since they make the transition from commensurate AFM to incommensurate spin-density wave (SDW) highly anomalous. At the crossover point, any signs of the electronic order are lost, leading to an emergent spin-liquid phase. Since the AFM corresponds to what one expects from a Hubbard model (insensitive to shape of the Fermi surface), while the SDW is driven by Fermi surface nesting, it is appropriate to call this a Mott-Slater transition \cite{MBMB}, and the emergent spin-liquid phase suggests why it is so hard to explain cuprate superconductivity starting from the undoped insulator. A similar commensurate-incommensurate transition with hints of emergent spin-liquid behavior at the crossover has been observed for the 3D Hubbard model which has finite-temperature phase transitions \cite{KohnAn}.

Insight into the bosonic ring and drumhead dispersions can be gained from phononic dispersions. The electrons can be considered as moving in a quasi-static potential generated by the phonons, and a phonon soft-mode introduces a new component to the potential. The ring dispersion thus causes the electrons to move in a Mexican-hat dispersion which is a signature of the Jahn-Teller effect. For phonons, the Mexican hat dispersion is typically connected to a point of conical intersection, where several phonon modes are degenerate. This point typically lies above the brim of the Mexican hat, signaling that the high-symmetry point is unstable.  Notably, the resulting strong electron-phonon coupling leads to highly exotic physics, including breakdown of the Born-Oppenheimer approximation and, possibly, time crystals \cite{Bob1993}.

           
{\it Discussion.$-$}
While high-order VHSs are likely to play a prominent role in strongly-correlated materials and exotic superconductivity, a quick review of the various ways high-order VHSs arise in cuprates illustrates how complicated their role can be.  There is inter-VHS nesting responsible for the strong AFM effects and Mott physics; a bosonic high-order VHS that controls the transition from commensurate $(\pi,\pi)$ AFM to incommensurate SDW order, with an emergent spin-liquid phase \cite{MBMB}; and  crossovers, both at a critical $t'_c$ and at doping $x_{cross}$, from $(\pi,\pi)$ VHSs to $\Gamma$-centered VHSs that may be associated with the peak of the superconducting dome. 
\begin{figure}[h!]
\centering 
\includegraphics[width=0.5\textwidth]{lscobba_ttt_vhs_G14c}
\caption{ Competition between near-$(\pi,\pi)$ (blue line) and $\Gamma$-VHSs (red line), for $t''/t'$ = -0.5.   Black dotted line indicates analytic approximation for the former.  Black vertical line marks $t'_{cross}$.  Green line plots $T_c/10$ for monolayer cuprates of Fig.~2(b), using same color code for the dots.}
\label{fig:3} 
\end{figure}

To further illustrate the last point, in SM Section S-II.A we present susceptibility calculations of the VHS competition.  Figure~\ref{fig:3} shows the close connection between the crossover at $t'_c$ and optimal $T_c$ for the cuprates.  We see that for small $|t'|$ cuprate physics is dominated by AFM order associated with the $(\pi,\pi)$ VHS, but the hoVHS at $t'/t=-0.258$ drives a crossover at $t'_c/t=-0.22$ to the predominantly $\Gamma$ dominated VHS peak, while the highest superconducting $T_c$ of the Hg cuprates falls just at the crossover.  A similar crossover in each cuprate should occur as a function of doping at $x_{cross}<x_{VHS}$. 

This observation suggests possible explanations as to why $T_c$ in cuprates is so high.  For example, if high -$T_c$ superconductivity arises  by tipping the balance between the competing orders, favoring the $(\pi,\pi)$ instability, then superconductivity should be most important near $t'_C4,$ when the dominant instability crosses over from $(\pi,\pi)$ to $\Gamma$. In this case, $T_c$ should maximize near $x_{cross}$ and decrease with larger doping. Alternatively, superconductivity could actually benefit from the proximity of the strongly correlated and frustrated Mott phase with the weakly conducting Slater phase\cite{ABB}.  These scenarios provide a good description of the cuprates. For example, in Bi2212 $T_c$ is maximal near $x_{cross}$, and $\rightarrow 0$ near $x_{VHS}$\cite{Zhou}, where the pairing strength also vanishes \cite{TallStorey}.  
  
Additional issues include understanding the dual role of a VHS, in both increasing correlations via the peak in the DOS/susceptibility, and decreasing correlations by enhancing dielectric screening, and the role of secondary VHSs in driving/suppressing further instabilities. The bosonic high-order VHSs may have important relevance to Bose metals \cite{Bosemet} and spinon bands \cite{spinon}. There is an ongoing search for exotic phase transitions that do not fit into the conventional Moriya-Hertz-Millis model of quantum criticality, particularly when nontrivial emergent excitations arise near the quantum critical point \cite{Sachdev}. Thus, our finding that such an emergent phase can be driven by bosonic high-order VHSs may constitute the most direct evidence of the importance of high-order VHSs in correlated materials. 


After years of debate on the significance of the VHSs in the cuprates, it is exciting to realize that they can be significantly more singular than previously imagined. The correlation between higher VHSs and higher superconducting $T_c$ we delineate here shows that high-order VHSs play a crucial role in high-$T_c$ cuprate superconductors.  However, the complicated nature of this correlation can be appreciated by noting that the superconducting $T_c$ does not usually maximize at the VHS doping $x_{VHS}$ but at a considerably lower doping. It will be interesting to see if similar complex intertwined orders are associated with high-order VHSs in other materials.

We now show that high-order VHSs exist if the electronic dimensions are smaller than lattice dimensions. During the early days of many-body perturbation theory, it was postulated that the effective electron dimensionality could be smaller than the crystal lattice dimension {\it i.e.} if a Fermi surface has flat parallel sections there would be good nesting, leading to quasi-1D behavior.  For example, the early high-T$_c$ superconductors such as the A15 compounds were assumed to be composed of three orthogonal interpenetrating chains of electrons \cite{Labbe}.  The three VHSs we have discussed in the cuprates illustrate this point.  The high-order VHSs of the bare dispersion remain point-like, but with power-law divergence, $p_V=0.29$, while the dispersion of the secondary AFM VHS is flat along an extended, curved line -- i.e., quasi-one dimensional, with $p_V=0.54$, and the bosonic VHS is flat over a finite area.  

The above results signal a substantial extension to the definition of a higher-order VHS\cite{Liang3}. Recall that a 2D saddle-point VHS corresponds to a point in $k$ space where the local dispersion has the form $ak_x^2-bk_y^2$, leading to a logarithmic peak in the DOS while a high-order VHS was defined as a point-like object where, for example, $b\rightarrow 0$. In contrast, here we consider the possibility that $b=0$ over an extended line segment.  This definition of high-order VHS is necessary, since the 2D materials must have a diverging VHS, and this divergence we identified is the only candidate. Such a definition can help to classify high-order VHSs in a wide range of flat-band materials, which are also characterized by having dispersions with flat line- or area-segments.

One final note on the connection between the higher-order VHSs and flat bands.  We have demonstrated that higher order VHSs are surrounded by a broad wake of 'enhanced-logarithmic' VHSs, that start out with a power-law divergence before crossing over to logarithmic, as seen in Fig.~\ref{fig:1vhs}(d).  On the other hand, in the study of flat bands, there is no consensus on how flat is flat: does a material count as a flat band if there is still a small dispersion?  We hypothesize that many flat band materials are associated with these enhanced-logarithmic VHSs, becoming flatter as they approach more closely to the higher-order VHSs.






\begin{thebibliography}{33}%
\makeatletter
\providecommand \@ifxundefined [1]{%
 \@ifx{#1\undefined}
}%
\providecommand \@ifnum [1]{%
 \ifnum #1\expandafter \@firstoftwo
 \else \expandafter \@secondoftwo
 \fi
}%
\providecommand \@ifx [1]{%
 \ifx #1\expandafter \@firstoftwo
 \else \expandafter \@secondoftwo
 \fi
}%
\providecommand \natexlab [1]{#1}%
\providecommand \enquote  [1]{``#1''}%
\providecommand \bibnamefont  [1]{#1}%
\providecommand \bibfnamefont [1]{#1}%
\providecommand \citenamefont [1]{#1}%
\providecommand \href@noop [0]{\@secondoftwo}%
\providecommand \href [0]{\begingroup \@sanitize@url \@href}%
\providecommand \@href[1]{\@@startlink{#1}\@@href}%
\providecommand \@@href[1]{\endgroup#1\@@endlink}%
\providecommand \@sanitize@url [0]{\catcode `\\12\catcode `\$12\catcode
  `\&12\catcode `\#12\catcode `\^12\catcode `\_12\catcode `\%12\relax}%
\providecommand \@@startlink[1]{}%
\providecommand \@@endlink[0]{}%
\providecommand \url  [0]{\begingroup\@sanitize@url \@url }%
\providecommand \@url [1]{\endgroup\@href {#1}{\urlprefix }}%
\providecommand \urlprefix  [0]{URL }%
\providecommand \Eprint [0]{\href }%
\providecommand \doibase [0]{http://dx.doi.org/}%
\providecommand \selectlanguage [0]{\@gobble}%
\providecommand \bibinfo  [0]{\@secondoftwo}%
\providecommand \bibfield  [0]{\@secondoftwo}%
\providecommand \translation [1]{[#1]}%
\providecommand \BibitemOpen [0]{}%
\providecommand \bibitemStop [0]{}%
\providecommand \bibitemNoStop [0]{.\EOS\space}%
\providecommand \EOS [0]{\spacefactor3000\relax}%
\providecommand \BibitemShut  [1]{\csname bibitem#1\endcsname}%
\let\auto@bib@innerbib\@empty
%</preamble>

\bibitem [{\citenamefont {Van~Hove}(1953)}]{VHS}%
  \BibitemOpen
  \bibfield  {author} {\bibinfo {author} {\bibfnamefont {L\'eon}\ \bibnamefont
  {Van~Hove}},\ }\bibfield  {title} {\enquote {\bibinfo {title} {The occurrence
  of singularities in the elastic frequency distribution of a crystal},}\
  }\href {\doibase 10.1103/PhysRev.89.1189} {\bibfield  {journal} {\bibinfo
  {journal} {Phys. Rev.}\ }\textbf {\bibinfo {volume} {89}},\ \bibinfo {pages}
  {1189--1193} (\bibinfo {year} {1953})}\BibitemShut {NoStop}%
\bibitem [{\citenamefont {Yuan}\ \emph {et~al.}(2019)\citenamefont {Yuan},
  \citenamefont {Isobe},\ and\ \citenamefont {Fu}}]{Liang}%
  \BibitemOpen
  \bibfield  {author} {\bibinfo {author} {\bibfnamefont {Noah F.~Q.}\
  \bibnamefont {Yuan}}, \bibinfo {author} {\bibfnamefont {Hiroki}\ \bibnamefont
  {Isobe}}, \ and\ \bibinfo {author} {\bibfnamefont {Liang}\ \bibnamefont
  {Fu}},\ }\bibfield  {title} {\enquote {\bibinfo {title} {Magic of high-order
  van hove singularity},}\ }\href {\doibase 10.1038/s41467-019-13670-9}
  {\bibfield  {journal} {\bibinfo  {journal} {Nature Communications}\ }\textbf
  {\bibinfo {volume} {10}},\ \bibinfo {pages} {5769} (\bibinfo {year}
  {2019})}\BibitemShut {NoStop}%
\bibitem [{\citenamefont {Yuan}\ and\ \citenamefont {Fu}(2020)}]{Liang3}%
  \BibitemOpen
  \bibfield  {author} {\bibinfo {author} {\bibfnamefont {Noah F.~Q.}\
  \bibnamefont {Yuan}}\ and\ \bibinfo {author} {\bibfnamefont {Liang}\
  \bibnamefont {Fu}},\ }\bibfield  {title} {\enquote {\bibinfo {title}
  {Classification of critical points in energy bands based on topology,
  scaling, and symmetry},}\ }\href {\doibase 10.1103/PhysRevB.101.125120}
  {\bibfield  {journal} {\bibinfo  {journal} {Phys. Rev. B}\ }\textbf {\bibinfo
  {volume} {101}},\ \bibinfo {pages} {125120} (\bibinfo {year}
  {2020})}\BibitemShut {NoStop}%  
\bibitem [{\citenamefont {Markiewicz}\ \emph {et~al.}(2017)\citenamefont
  {Markiewicz}, \citenamefont {Buda}, \citenamefont {Mistark}, \citenamefont
  {Lane},\ and\ \citenamefont {Bansil}}]{MBMB}%
  \BibitemOpen
\bibfield  {author} {\bibinfo {author} {\bibfnamefont {R.~S.}\ \bibnamefont
  {Markiewicz}}, \bibinfo {author} {\bibfnamefont {I.~G.}\ \bibnamefont
  {Buda}}, \bibinfo {author} {\bibfnamefont {P.}~\bibnamefont {Mistark}},
  \bibinfo {author} {\bibfnamefont {C.}~\bibnamefont {Lane}}, \ and\ \bibinfo
  {author} {\bibfnamefont {A.}~\bibnamefont {Bansil}},\ }\bibfield  {title}
  {\enquote {\bibinfo {title} {Entropic origin of pseudogap physics and a
  mott-slater transition in cuprates},}\ }\href {\doibase 10.1038/srep44008}
  {\bibfield  {journal} {\bibinfo  {journal} {Scientific Reports}\ }\textbf
  {\bibinfo {volume} {7}},\ \bibinfo {pages} {44008} (\bibinfo {year}
  {2017})}\BibitemShut {NoStop}%  
\bibitem [{\citenamefont {Souza}\ and\ \citenamefont
  {Marsiglio}(2017)}]{Marsig}%
  \BibitemOpen
  \bibfield  {author} {\bibinfo {author} {\bibfnamefont {Thiago X.~R.}\
  \bibnamefont {Souza}}\ and\ \bibinfo {author} {\bibfnamefont
  {F.}~\bibnamefont {Marsiglio}},\ }\bibfield  {title} {\enquote {\bibinfo
  {title} {The possible role of van hove singularities in the high $t_c$ of
  superconducting H$_3$S},}\ }\href {\doibase 10.1142/S0217979217450035}
  {\bibfield  {journal} {\bibinfo  {journal} {International Journal of Modern
  Physics B}\ }\textbf {\bibinfo {volume} {31}},\ \bibinfo {pages} {1745003}
  (\bibinfo {year} {2017})}\BibitemShut {NoStop}%  
\bibitem [{\citenamefont {Isobe}\ and\ \citenamefont {Fu}(2019)}]{Liang2}%
  \BibitemOpen
  \bibfield  {author} {\bibinfo {author} {\bibfnamefont {Hiroki}\ \bibnamefont
  {Isobe}}\ and\ \bibinfo {author} {\bibfnamefont {Liang}\ \bibnamefont {Fu}},\
  }\bibfield  {title} {\enquote {\bibinfo {title} {Supermetal},}\ }\href
  {\doibase 10.1103/PhysRevResearch.1.033206} {\bibfield  {journal} {\bibinfo
  {journal} {Phys. Rev. Research}\ }\textbf {\bibinfo {volume} {1}},\ \bibinfo
  {pages} {033206} (\bibinfo {year} {2019})}\BibitemShut {NoStop}%  
  \bibitem{Ref1b}Daniele Guerci, Pascal Simon, and Christophe Mora,
Higher-order van Hove singularity in magic-angle twisted trilayer graphene, 
arXiv:2106.14911.
\bibitem{Ref2a}Noah F. Q. Yuan and Liang Fu,
  Classification of critical points in energy bands based on topology, scaling, and symmetry,
Phys. Rev. B 101, 125120 (2020).
\bibitem{Ref2b}Anirudh Chandrasekaran, Alex Shtyk, Joseph J. Betouras, and Claudio Chamon,
Catastrophe theory classification of Fermi surface topological transitions in two dimensions,
Phys. Rev. Research 2, 013355 (2020).



\bibitem [{\citenamefont {Kerelsky}\ \emph {et~al.}(2019)\citenamefont
  {Kerelsky}, \citenamefont {McGilly}, \citenamefont {Kennes}, \citenamefont
  {Xian}, \citenamefont {Yankowitz}, \citenamefont {Chen}, \citenamefont
  {Watanabe}, \citenamefont {Taniguchi}, \citenamefont {Hone}, \citenamefont
  {Dean}, \citenamefont {Rubio},\ and\ \citenamefont {Pasupathy}}]{Tgraphene}%
  \BibitemOpen
  \bibfield  {author} {\bibinfo {author} {\bibfnamefont {Alexander}\
  \bibnamefont {Kerelsky}}, \bibinfo {author} {\bibfnamefont {Leo~J.}\
  \bibnamefont {McGilly}}, \bibinfo {author} {\bibfnamefont {Dante~M.}\
  \bibnamefont {Kennes}}, \bibinfo {author} {\bibfnamefont {Lede}\ \bibnamefont
  {Xian}}, \bibinfo {author} {\bibfnamefont {Matthew}\ \bibnamefont
  {Yankowitz}}, \bibinfo {author} {\bibfnamefont {Shaowen}\ \bibnamefont
  {Chen}}, \bibinfo {author} {\bibfnamefont {K.}~\bibnamefont {Watanabe}},
  \bibinfo {author} {\bibfnamefont {T.}~\bibnamefont {Taniguchi}}, \bibinfo
  {author} {\bibfnamefont {James}\ \bibnamefont {Hone}}, \bibinfo {author}
  {\bibfnamefont {Cory}\ \bibnamefont {Dean}}, \bibinfo {author} {\bibfnamefont
  {Angel}\ \bibnamefont {Rubio}}, \ and\ \bibinfo {author} {\bibfnamefont
  {Abhay~N.}\ \bibnamefont {Pasupathy}},\ }\bibfield  {title} {\enquote
  {\bibinfo {title} {Maximized electron interactions at the magic angle in
  twisted bilayer graphene},}\ }\href {\doibase 10.1038/s41586-019-1431-9}
  {\bibfield  {journal} {\bibinfo  {journal} {Nature}\ }\textbf {\bibinfo
  {volume} {572}},\ \bibinfo {pages} {95--100} (\bibinfo {year}
  {2019})}\BibitemShut {NoStop}%
\bibitem [{\citenamefont {Efremov}\ \emph {et~al.}(2019)\citenamefont
  {Efremov}, \citenamefont {Shtyk}, \citenamefont {Rost}, \citenamefont
  {Chamon}, \citenamefont {Mackenzie},\ and\ \citenamefont
  {Betouras}}]{SRO327}%
  \BibitemOpen
  \bibfield  {author} {\bibinfo {author} {\bibfnamefont {Dmitry~V.}\
  \bibnamefont {Efremov}}, \bibinfo {author} {\bibfnamefont {Alex}\
  \bibnamefont {Shtyk}}, \bibinfo {author} {\bibfnamefont {Andreas~W.}\
  \bibnamefont {Rost}}, \bibinfo {author} {\bibfnamefont {Claudio}\
  \bibnamefont {Chamon}}, \bibinfo {author} {\bibfnamefont {Andrew~P.}\
  \bibnamefont {Mackenzie}}, \ and\ \bibinfo {author} {\bibfnamefont
  {Joseph~J.}\ \bibnamefont {Betouras}},\ }\bibfield  {title} {\enquote
  {\bibinfo {title} {Multicritical fermi surface topological transitions},}\
  }\href {\doibase 10.1103/PhysRevLett.123.207202} {\bibfield  {journal}
  {\bibinfo  {journal} {Phys. Rev. Lett.}\ }\textbf {\bibinfo {volume} {123}},\
  \bibinfo {pages} {207202} (\bibinfo {year} {2019})}\BibitemShut {NoStop}%
\bibitem [{\citenamefont {Hirsch}\ and\ \citenamefont
  {Scalapino}(1986)}]{HirSc}%
  \BibitemOpen
  \bibfield  {author} {\bibinfo {author} {\bibfnamefont {J.~E.}\ \bibnamefont
  {Hirsch}}\ and\ \bibinfo {author} {\bibfnamefont {D.~J.}\ \bibnamefont
  {Scalapino}},\ }\bibfield  {title} {\enquote {\bibinfo {title} {Enhanced
  superconductivity in quasi two-dimensional systems},}\ }\href {\doibase
  10.1103/PhysRevLett.56.2732} {\bibfield  {journal} {\bibinfo  {journal}
  {Phys. Rev. Lett.}\ }\textbf {\bibinfo {volume} {56}},\ \bibinfo {pages}
  {2732--2735} (\bibinfo {year} {1986})}\BibitemShut {NoStop}%
\bibitem{Ref1c}K. Gofron, J. C. Campuzano, A. A. Abrikosov, M. Lindroos, A. Bansil, H. Ding, D. Koelling, and B. Dabrowski,
Observation of an "Extended" Van Hove Singularity in YBa$_2$Cu$_4$O$_8$ by Ultrahigh Energy Resolution Angle-Resolved Photoemission,
Phys. Rev. Lett. 73, 3302 (1994).
\bibitem{Damasc1}M. A. Hossain, J. D. F. Mottershead, D. Fournier, A. Bostwick, J. L. McChesney, E. Rotenberg, R. Liang, W. N. Hardy, G. A. Sawatzky, I. S. Elfimov, D. A. Bonn, and A. Damascelli,
{\it In situ} doping control of the surface of high-temperature superconductors,
Nature Physics 4, 527 (2008).
\bibitem{Andex}O.K. Andersen, O. Jepsen, A.I. Liechtenstein, and I.I. Mazin,
Plane dimpling and saddle-point bifurcation in the band structures of optimally doped high-temperature superconductors: A tight-binding model,
Phys. Rev. B 49, 4145 (1994).
\bibitem{Zehyer}R. Zehyer,
Transition temperature and isotope effect near an extended van Hove singularity,
Z. Phys. B97, 3 (1995).


\bibitem [{\citenamefont {Michon}\ \emph {et~al.}(2019)\citenamefont {Michon},
  \citenamefont {Girod}, \citenamefont {Badoux}, \citenamefont {Ka{\v c}mar{\v
  c}{\'\i}k}, \citenamefont {Ma}, \citenamefont {Dragomir}, \citenamefont
  {Dabkowska}, \citenamefont {Gaulin}, \citenamefont {Zhou}, \citenamefont
  {Pyon}, \citenamefont {Takayama}, \citenamefont {Takagi}, \citenamefont
  {Verret}, \citenamefont {Doiron-Leyraud}, \citenamefont {Marcenat},
  \citenamefont {Taillefer},\ and\ \citenamefont {Klein}}]{Michon}%
  \BibitemOpen
  \bibfield  {author} {\bibinfo {author} {\bibfnamefont {B.}~\bibnamefont
  {Michon}}, \bibinfo {author} {\bibfnamefont {C.}~\bibnamefont {Girod}},
  \bibinfo {author} {\bibfnamefont {S.}~\bibnamefont {Badoux}}, \bibinfo
  {author} {\bibfnamefont {J.}~\bibnamefont {Ka{\v c}mar{\v c}{\'\i}k}},
  \bibinfo {author} {\bibfnamefont {Q.}~\bibnamefont {Ma}}, \bibinfo {author}
  {\bibfnamefont {M.}~\bibnamefont {Dragomir}}, \bibinfo {author}
  {\bibfnamefont {H.~A.}\ \bibnamefont {Dabkowska}}, \bibinfo {author}
  {\bibfnamefont {B.~D.}\ \bibnamefont {Gaulin}}, \bibinfo {author}
  {\bibfnamefont {J.~S.}\ \bibnamefont {Zhou}}, \bibinfo {author}
  {\bibfnamefont {S.}~\bibnamefont {Pyon}}, \bibinfo {author} {\bibfnamefont
  {T.}~\bibnamefont {Takayama}}, \bibinfo {author} {\bibfnamefont
  {H.}~\bibnamefont {Takagi}}, \bibinfo {author} {\bibfnamefont
  {S.}~\bibnamefont {Verret}}, \bibinfo {author} {\bibfnamefont
  {N.}~\bibnamefont {Doiron-Leyraud}}, \bibinfo {author} {\bibfnamefont
  {C.}~\bibnamefont {Marcenat}}, \bibinfo {author} {\bibfnamefont
  {L.}~\bibnamefont {Taillefer}}, \ and\ \bibinfo {author} {\bibfnamefont
  {T.}~\bibnamefont {Klein}},\ }\bibfield  {title} {\enquote {\bibinfo {title}
  {Thermodynamic signatures of quantum criticality in cuprate
  superconductors},}\ }\href {\doibase 10.1038/s41586-019-0932-x} {\bibfield
  {journal} {\bibinfo  {journal} {Nature}\ }\textbf {\bibinfo {volume} {567}},\
  \bibinfo {pages} {218--222} (\bibinfo {year} {2019})}\BibitemShut {NoStop}%
\bibitem [{\citenamefont {Markiewicz}(2004)}]{RM70}%
  \BibitemOpen
  \bibfield  {author} {\bibinfo {author} {\bibfnamefont {R.~S.}\ \bibnamefont
  {Markiewicz}},\ }\bibfield  {title} {\enquote {\bibinfo {title}
  {Mode-coupling model of mott gap collapse in the cuprates: Natural phase
  boundary for quantum critical points},}\ }\href {\doibase
  10.1103/PhysRevB.70.174518} {\bibfield  {journal} {\bibinfo  {journal} {Phys.
  Rev. B}\ }\textbf {\bibinfo {volume} {70}},\ \bibinfo {pages} {174518}
  (\bibinfo {year} {2004})}\BibitemShut {NoStop}%

\bibitem [{\citenamefont {Pavarini}\ \emph {et~al.}(2001)\citenamefont
  {Pavarini}, \citenamefont {Dasgupta}, \citenamefont {Saha-Dasgupta},
  \citenamefont {Jepsen},\ and\ \citenamefont {Andersen}}]{PavOK}%
  \BibitemOpen
  \bibfield  {author} {\bibinfo {author} {\bibfnamefont {E.}~\bibnamefont
  {Pavarini}}, \bibinfo {author} {\bibfnamefont {I.}~\bibnamefont {Dasgupta}},
  \bibinfo {author} {\bibfnamefont {T.}~\bibnamefont {Saha-Dasgupta}}, \bibinfo
  {author} {\bibfnamefont {O.}~\bibnamefont {Jepsen}}, \ and\ \bibinfo {author}
  {\bibfnamefont {O.~K.}\ \bibnamefont {Andersen}},\ }\bibfield  {title}
  {\enquote {\bibinfo {title} {Band-structure trend in hole-doped cuprates and
  correlation with ${\mathit{t}}_{\mathit{c}\mathrm{max}}$},}\ }\href {\doibase
  10.1103/PhysRevLett.87.047003} {\bibfield  {journal} {\bibinfo  {journal}
  {Phys. Rev. Lett.}\ }\textbf {\bibinfo {volume} {87}},\ \bibinfo {pages}
  {047003} (\bibinfo {year} {2001})}\BibitemShut {NoStop}%

\bibitem [{\citenamefont {Markiewicz}\ and\ \citenamefont
  {Vaughn}(1998)}]{Mikefest}%
  \BibitemOpen
  \bibfield  {author} {\bibinfo {author} {\bibfnamefont {R.~S.}\ \bibnamefont
  {Markiewicz}}\ and\ \bibinfo {author} {\bibfnamefont {M.~T.}\ \bibnamefont
  {Vaughn}},\ }\bibfield  {title} {\enquote {\bibinfo {title} {Classification
  of the van hove scenario as an SO(8) spectrum-generating algebra},}\ }\href
  {\doibase 10.1103/PhysRevB.57.R14052} {\bibfield  {journal} {\bibinfo
  {journal} {Phys. Rev. B}\ }\textbf {\bibinfo {volume} {57}},\ \bibinfo
  {pages} {R14052} (\bibinfo {year} {1998})}\BibitemShut {NoStop}%
\bibitem [{\citenamefont {Markiewicz}(2000)}]{RSMstripe}%
  \BibitemOpen
  \bibfield  {author} {\bibinfo {author} {\bibfnamefont {R.~S.}\ \bibnamefont
  {Markiewicz}},\ }\bibfield  {title} {\enquote {\bibinfo {title} {Dispersion
  of ordered stripe phases in the cuprates},}\ }\href {\doibase
  10.1103/PhysRevB.62.1252} {\bibfield  {journal} {\bibinfo  {journal} {Phys.
  Rev. B}\ }\textbf {\bibinfo {volume} {62}},\ \bibinfo {pages} {1252--1269}
  (\bibinfo {year} {2000})}\BibitemShut {NoStop}%
\bibitem [{\citenamefont {Horio}\ \emph {et~al.}(2018)\citenamefont {Horio},
  \citenamefont {Hauser}, \citenamefont {Sassa}, \citenamefont {Mingazheva},
  \citenamefont {Sutter}, \citenamefont {Kramer}, \citenamefont {Cook},
  \citenamefont {Nocerino}, \citenamefont {Forslund}, \citenamefont
  {Tjernberg}, \citenamefont {Kobayashi}, \citenamefont {Chikina},
  \citenamefont {Schr\"oter}, \citenamefont {Krieger}, \citenamefont {Schmitt},
  \citenamefont {Strocov}, \citenamefont {Pyon}, \citenamefont {Takayama},
  \citenamefont {Takagi}, \citenamefont {Lipscombe}, \citenamefont {Hayden},
  \citenamefont {Ishikado}, \citenamefont {Eisaki}, \citenamefont {Neupert},
  \citenamefont {M\aa{}nsson}, \citenamefont {Matt},\ and\ \citenamefont
  {Chang}}]{Horio}%
  \BibitemOpen
  \bibfield  {author} {\bibinfo {author} {\bibfnamefont {M.}~\bibnamefont
  {Horio}}, \bibinfo {author} {\bibfnamefont {K.}~\bibnamefont {Hauser}},
  \bibinfo {author} {\bibfnamefont {Y.}~\bibnamefont {Sassa}}, \bibinfo
  {author} {\bibfnamefont {Z.}~\bibnamefont {Mingazheva}}, \bibinfo {author}
  {\bibfnamefont {D.}~\bibnamefont {Sutter}}, \bibinfo {author} {\bibfnamefont
  {K.}~\bibnamefont {Kramer}}, \bibinfo {author} {\bibfnamefont
  {A.}~\bibnamefont {Cook}}, \bibinfo {author} {\bibfnamefont {E.}~\bibnamefont
  {Nocerino}}, \bibinfo {author} {\bibfnamefont {O.~K.}\ \bibnamefont
  {Forslund}}, \bibinfo {author} {\bibfnamefont {O.}~\bibnamefont {Tjernberg}},
  \bibinfo {author} {\bibfnamefont {M.}~\bibnamefont {Kobayashi}}, \bibinfo
  {author} {\bibfnamefont {A.}~\bibnamefont {Chikina}}, \bibinfo {author}
  {\bibfnamefont {N.~B.~M.}\ \bibnamefont {Schr\"oter}}, \bibinfo {author}
  {\bibfnamefont {J.~A.}\ \bibnamefont {Krieger}}, \bibinfo {author}
  {\bibfnamefont {T.}~\bibnamefont {Schmitt}}, \bibinfo {author} {\bibfnamefont
  {V.~N.}\ \bibnamefont {Strocov}}, \bibinfo {author} {\bibfnamefont
  {S.}~\bibnamefont {Pyon}}, \bibinfo {author} {\bibfnamefont {T.}~\bibnamefont
  {Takayama}}, \bibinfo {author} {\bibfnamefont {H.}~\bibnamefont {Takagi}},
  \bibinfo {author} {\bibfnamefont {O.~J.}\ \bibnamefont {Lipscombe}}, \bibinfo
  {author} {\bibfnamefont {S.~M.}\ \bibnamefont {Hayden}}, \bibinfo {author}
  {\bibfnamefont {M.}~\bibnamefont {Ishikado}}, \bibinfo {author}
  {\bibfnamefont {H.}~\bibnamefont {Eisaki}}, \bibinfo {author} {\bibfnamefont
  {T.}~\bibnamefont {Neupert}}, \bibinfo {author} {\bibfnamefont
  {M.}~\bibnamefont {M\aa{}nsson}}, \bibinfo {author} {\bibfnamefont {C.~E.}\
  \bibnamefont {Matt}}, \ and\ \bibinfo {author} {\bibfnamefont
  {J.}~\bibnamefont {Chang}},\ }\bibfield  {title} {\enquote {\bibinfo {title}
  {Three-dimensional fermi surface of overdoped la-based cuprates},}\ }\href
  {\doibase 10.1103/PhysRevLett.121.077004} {\bibfield  {journal} {\bibinfo
  {journal} {Phys. Rev. Lett.}\ }\textbf {\bibinfo {volume} {121}},\ \bibinfo
  {pages} {077004} (\bibinfo {year} {2018})}\BibitemShut {NoStop}%
\bibitem [{\citenamefont {Singh}\ \emph {et~al.}(2017)\citenamefont {Singh},
  \citenamefont {Hsu}, \citenamefont {Tsai}, \citenamefont {Pereira},\ and\
  \citenamefont {Lin}}]{singh2017}%
  \BibitemOpen
  \bibfield  {author} {\bibinfo {author} {\bibfnamefont {Bahadur}\ \bibnamefont
  {Singh}}, \bibinfo {author} {\bibfnamefont {Chuang-Han}\ \bibnamefont {Hsu}},
  \bibinfo {author} {\bibfnamefont {Wei-Feng}\ \bibnamefont {Tsai}}, \bibinfo
  {author} {\bibfnamefont {Vitor~M.}\ \bibnamefont {Pereira}}, \ and\ \bibinfo
  {author} {\bibfnamefont {Hsin}\ \bibnamefont {Lin}},\ }\bibfield  {title}
  {\enquote {\bibinfo {title} {Stable charge density wave phase in a
  $1t--{\mathrm{TiSe}}_{2}$ monolayer},}\ }\href {\doibase
  10.1103/PhysRevB.95.245136} {\bibfield  {journal} {\bibinfo  {journal} {Phys.
  Rev. B}\ }\textbf {\bibinfo {volume} {95}},\ \bibinfo {pages} {245136}
  (\bibinfo {year} {2017})}\BibitemShut {NoStop}%
\bibitem [{\citenamefont {Bansil}\ \emph {et~al.}(2016)\citenamefont {Bansil},
  \citenamefont {Lin},\ and\ \citenamefont {Das}}]{Bansil2016}%
  \BibitemOpen
  \bibfield  {author} {\bibinfo {author} {\bibfnamefont {A.}~\bibnamefont
  {Bansil}}, \bibinfo {author} {\bibfnamefont {Hsin}\ \bibnamefont {Lin}}, \
  and\ \bibinfo {author} {\bibfnamefont {Tanmoy}\ \bibnamefont {Das}},\
  }\bibfield  {title} {\enquote {\bibinfo {title} {Colloquium: Topological band
  theory},}\ }\href {\doibase 10.1103/RevModPhys.88.021004} {\bibfield
  {journal} {\bibinfo  {journal} {Rev. Mod. Phys.}\ }\textbf {\bibinfo {volume}
  {88}},\ \bibinfo {pages} {021004} (\bibinfo {year} {2016})}\BibitemShut
  {NoStop}%
\bibitem [{\citenamefont {Overhauser}(1968)}]{Over}%
  \BibitemOpen
  \bibfield  {author} {\bibinfo {author} {\bibfnamefont {A.~W.}\ \bibnamefont
  {Overhauser}},\ }\bibfield  {title} {\enquote {\bibinfo {title} {Exchange and
  correlation instabilities of simple metals},}\ }\href {\doibase
  10.1103/PhysRev.167.691} {\bibfield  {journal} {\bibinfo  {journal} {Phys.
  Rev.}\ }\textbf {\bibinfo {volume} {167}},\ \bibinfo {pages} {691--698}
  (\bibinfo {year} {1968})}\BibitemShut {NoStop}%
\bibitem [{\citenamefont {Markiewicz}\ and\ \citenamefont
  {Bansil}(2018)}]{RSMdd}%
  \BibitemOpen
  \bibfield  {author} {\bibinfo {author} {\bibfnamefont {R.~S.}\ \bibnamefont
  {Markiewicz}}\ and\ \bibinfo {author} {\bibfnamefont {A.}~\bibnamefont
  {Bansil}},\ }\bibfield  {title} {\enquote {\bibinfo {title} {Excitonic
  insulators as a model of $d-d$ and mott transitions in strongly correlated
  materials},}\ }\href {https://arxiv.org/abs/1708.02270} {\bibfield  {journal}
  {\bibinfo  {journal} {ArXiv:1708.02270}\ } (\bibinfo {year}
  {2018})}\BibitemShut {NoStop}%
\bibitem [{\citenamefont {Fu}\ and\ \citenamefont {Bi}(2019)}]{Liang4}%
  \BibitemOpen
  \bibfield  {author} {\bibinfo {author} {\bibfnamefont {Liang}\ \bibnamefont
  {Fu}}\ and\ \bibinfo {author} {\bibfnamefont {Zhen}\ \bibnamefont {Bi}},\
  }\bibfield  {title} {\enquote {\bibinfo {title} {Excitonic density wave and
  spin-valley superfluid in bilayer transition metal dichalcogenide},}\ }\href
  {https://arxiv.org/abs/1911.04493} {\bibfield  {journal} {\bibinfo  {journal}
  {ArXiv:1911.04493}\ } (\bibinfo {year} {2019})}\BibitemShut {NoStop}%
\bibitem [{\citenamefont {Phillips}(1964)}]{JCP}%
  \BibitemOpen
  \bibfield  {author} {\bibinfo {author} {\bibfnamefont {J.~C.}\ \bibnamefont
  {Phillips}},\ }\bibfield  {title} {\enquote {\bibinfo {title} {Ultraviolet
  absorption of insulators. iii. fcc alkali halides},}\ }\href {\doibase
  10.1103/PhysRev.136.A1705} {\bibfield  {journal} {\bibinfo  {journal} {Phys.
  Rev.}\ }\textbf {\bibinfo {volume} {136}},\ \bibinfo {pages} {A1705}
  (\bibinfo {year} {1964})}\BibitemShut {NoStop}%
\bibitem [{\citenamefont {Sheng}\ \emph {et~al.}(2009)\citenamefont {Sheng},
  \citenamefont {Motrunich},\ and\ \citenamefont {Fisher}}]{Bosemet}%
  \BibitemOpen
  \bibfield  {author} {\bibinfo {author} {\bibfnamefont {D.~N.}\ \bibnamefont
  {Sheng}}, \bibinfo {author} {\bibfnamefont {Olexei~I.}\ \bibnamefont
  {Motrunich}}, \ and\ \bibinfo {author} {\bibfnamefont {Matthew P.~A.}\
  \bibnamefont {Fisher}},\ }\bibfield  {title} {\enquote {\bibinfo {title}
  {Spin bose-metal phase in a spin-$\frac{1}{2}$ model with ring exchange on a
  two-leg triangular strip},}\ }\href {\doibase 10.1103/PhysRevB.79.205112}
  {\bibfield  {journal} {\bibinfo  {journal} {Phys. Rev. B}\ }\textbf {\bibinfo
  {volume} {79}},\ \bibinfo {pages} {205112} (\bibinfo {year}
  {2009})}\BibitemShut {NoStop}%
\bibitem [{\citenamefont {Hu}\ \emph {et~al.}(2020)\citenamefont {Hu},
  \citenamefont {Zhang}, \citenamefont {Nevidomskyy}, \citenamefont {Dagotto},
  \citenamefont {Si},\ and\ \citenamefont {Lai}}]{spinon}%
  \BibitemOpen
  \bibfield  {author} {\bibinfo {author} {\bibfnamefont {Wen-Jun}\ \bibnamefont
  {Hu}}, \bibinfo {author} {\bibfnamefont {Yi}~\bibnamefont {Zhang}}, \bibinfo
  {author} {\bibfnamefont {Andriy~H.}\ \bibnamefont {Nevidomskyy}}, \bibinfo
  {author} {\bibfnamefont {Elbio}\ \bibnamefont {Dagotto}}, \bibinfo {author}
  {\bibfnamefont {Qimiao}\ \bibnamefont {Si}}, \ and\ \bibinfo {author}
  {\bibfnamefont {Hsin-Hua}\ \bibnamefont {Lai}},\ }\bibfield  {title}
  {\enquote {\bibinfo {title} {Fractionalized excitations revealed by
  entanglement entropy},}\ }\href {\doibase 10.1103/PhysRevLett.124.237201}
  {\bibfield  {journal} {\bibinfo  {journal} {Phys. Rev. Lett.}\ }\textbf
  {\bibinfo {volume} {124}},\ \bibinfo {pages} {237201} (\bibinfo {year}
  {2020})}\BibitemShut {NoStop}%
\bibitem [{\citenamefont {Markiewicz}\ \emph {et~al.}(2015)\citenamefont
  {Markiewicz}, \citenamefont {Seibold}, \citenamefont {Lorenzana},\ and\
  \citenamefont {Bansil}}]{RSMch}%
  \BibitemOpen
  \bibfield  {author} {\bibinfo {author} {\bibfnamefont {R.~S.}\ \bibnamefont
  {Markiewicz}}, \bibinfo {author} {\bibfnamefont {G.}~\bibnamefont {Seibold}},
  \bibinfo {author} {\bibfnamefont {J.}~\bibnamefont {Lorenzana}}, \ and\
  \bibinfo {author} {\bibfnamefont {A.}~\bibnamefont {Bansil}},\ }\bibfield
  {title} {\enquote {\bibinfo {title} {Gutzwiller charge phase diagram of
  cuprates, including electron{\textendash}phonon coupling effects},}\ }\href
  {\doibase 10.1088/1367-2630/17/2/023074} {\bibfield  {journal} {\bibinfo
  {journal} {New Journal of Physics}\ }\textbf {\bibinfo {volume} {17}},\
  \bibinfo {pages} {023074} (\bibinfo {year} {2015})}\BibitemShut {NoStop}%
\bibitem [{\citenamefont {Sch\"afer}\ \emph {et~al.}(2017)\citenamefont
  {Sch\"afer}, \citenamefont {Katanin}, \citenamefont {Held},\ and\
  \citenamefont {Toschi}}]{KohnAn}%
  \BibitemOpen
  \bibfield  {author} {\bibinfo {author} {\bibfnamefont {T.}~\bibnamefont
  {Sch\"afer}}, \bibinfo {author} {\bibfnamefont {A.~A.}\ \bibnamefont
  {Katanin}}, \bibinfo {author} {\bibfnamefont {K.}~\bibnamefont {Held}}, \
  and\ \bibinfo {author} {\bibfnamefont {A.}~\bibnamefont {Toschi}},\
  }\bibfield  {title} {\enquote {\bibinfo {title} {Interplay of correlations
  and kohn anomalies in three dimensions: Quantum criticality with a twist},}\
  }\href {\doibase 10.1103/PhysRevLett.119.046402} {\bibfield  {journal}
  {\bibinfo  {journal} {Phys. Rev. Lett.}\ }\textbf {\bibinfo {volume} {119}},\
  \bibinfo {pages} {046402} (\bibinfo {year} {2017})}\BibitemShut {NoStop}%
\bibitem [{\citenamefont {Markiewicz}(1993)}]{Bob1993}%
  \BibitemOpen
  \bibfield  {author} {\bibinfo {author} {\bibfnamefont {R.~S.}\ \bibnamefont
  {Markiewicz}},\ }\bibfield  {title} {\enquote {\bibinfo {title} {Van hove
  exciton-cageons and high-tc superconductivity: Viiid. solitons and nonlinear
  dynamics},}\ }\href {\doibase https://doi.org/10.1016/0921-4534(93)90030-T}
  {\bibfield  {journal} {\bibinfo  {journal} {Physica C: Superconductivity}\
  }\textbf {\bibinfo {volume} {210}},\ \bibinfo {pages} {264} (\bibinfo {year}
  {1993})}\BibitemShut {NoStop}%
\bibitem [{\citenamefont {Allender}\ \emph {et~al.}(1974)\citenamefont
  {Allender}, \citenamefont {Bray},\ and\ \citenamefont {Bardeen}}]{ABB}%
  \BibitemOpen
  \bibfield  {author} {\bibinfo {author} {\bibfnamefont {David}\ \bibnamefont
  {Allender}}, \bibinfo {author} {\bibfnamefont {J.~W.}\ \bibnamefont {Bray}},
  \ and\ \bibinfo {author} {\bibfnamefont {John}\ \bibnamefont {Bardeen}},\
  }\bibfield  {title} {\enquote {\bibinfo {title} {Theory of fluctuation
  superconductivity from electron-phonon interactions in pseudo-one-dimensional
  systems},}\ }\href {\doibase 10.1103/PhysRevB.9.119} {\bibfield  {journal}
  {\bibinfo  {journal} {Phys. Rev. B}\ }\textbf {\bibinfo {volume} {9}},\
  \bibinfo {pages} {119--129} (\bibinfo {year} {1974})}\BibitemShut {NoStop}%

\bibitem [{\citenamefont {Ding}\ \emph {et~al.}(2019)\citenamefont {Ding},
  \citenamefont {Zhao}, \citenamefont {Yan}, \citenamefont {Gao}, \citenamefont
  {Liu}, \citenamefont {Hu}, \citenamefont {Huang}, \citenamefont {Li},
  \citenamefont {Xu}, \citenamefont {Cai}, \citenamefont {Rong}, \citenamefont
  {Wu}, \citenamefont {Song}, \citenamefont {Zhou}, \citenamefont {Dong},
  \citenamefont {Liu}, \citenamefont {Wang}, \citenamefont {Zhang},
  \citenamefont {Wang}, \citenamefont {Zhang}, \citenamefont {Yang},
  \citenamefont {Peng}, \citenamefont {Xu}, \citenamefont {Chen},\ and\
  \citenamefont {Zhou}}]{Zhou}%
  \BibitemOpen
  \bibfield  {author} {\bibinfo {author} {\bibfnamefont {Ying}\ \bibnamefont
  {Ding}}, \bibinfo {author} {\bibfnamefont {Lin}\ \bibnamefont {Zhao}},
  \bibinfo {author} {\bibfnamefont {Hongtao}\ \bibnamefont {Yan}}, \bibinfo
  {author} {\bibfnamefont {Qiang}\ \bibnamefont {Gao}}, \bibinfo {author}
  {\bibfnamefont {Jing}\ \bibnamefont {Liu}}, \bibinfo {author} {\bibfnamefont
  {Cheng}\ \bibnamefont {Hu}}, \bibinfo {author} {\bibfnamefont {Jianwei}\
  \bibnamefont {Huang}}, \bibinfo {author} {\bibfnamefont {Cong}\ \bibnamefont
  {Li}}, \bibinfo {author} {\bibfnamefont {Yu}~\bibnamefont {Xu}}, \bibinfo
  {author} {\bibfnamefont {Yongqing}\ \bibnamefont {Cai}}, \bibinfo {author}
  {\bibfnamefont {Hongtao}\ \bibnamefont {Rong}}, \bibinfo {author}
  {\bibfnamefont {Dingsong}\ \bibnamefont {Wu}}, \bibinfo {author}
  {\bibfnamefont {Chunyao}\ \bibnamefont {Song}}, \bibinfo {author}
  {\bibfnamefont {Huaxue}\ \bibnamefont {Zhou}}, \bibinfo {author}
  {\bibfnamefont {Xiaoli}\ \bibnamefont {Dong}}, \bibinfo {author}
  {\bibfnamefont {Guodong}\ \bibnamefont {Liu}}, \bibinfo {author}
  {\bibfnamefont {Qingyan}\ \bibnamefont {Wang}}, \bibinfo {author}
  {\bibfnamefont {Shenjin}\ \bibnamefont {Zhang}}, \bibinfo {author}
  {\bibfnamefont {Zhimin}\ \bibnamefont {Wang}}, \bibinfo {author}
  {\bibfnamefont {Fengfeng}\ \bibnamefont {Zhang}}, \bibinfo {author}
  {\bibfnamefont {Feng}\ \bibnamefont {Yang}}, \bibinfo {author} {\bibfnamefont
  {Qinjun}\ \bibnamefont {Peng}}, \bibinfo {author} {\bibfnamefont {Zuyan}\
  \bibnamefont {Xu}}, \bibinfo {author} {\bibfnamefont {Chuangtian}\
  \bibnamefont {Chen}}, \ and\ \bibinfo {author} {\bibfnamefont {X.~J.}\
  \bibnamefont {Zhou}},\ }\bibfield  {title} {\enquote {\bibinfo {title}
  {Disappearance of superconductivity and a concomitant lifshitz transition in
  heavily overdoped bi$_2$sr$_2$cuo$_6$ superconductor revealed by
  angle-resolved photoemission spectroscopy},}\ }\href {\doibase
  10.1088/0256-307X/36/1/017402} {\bibfield  {journal} {\bibinfo  {journal}
  {Chin. Phys. Lett.}\ }\textbf {\bibinfo {volume} {36}},\ \bibinfo {pages}
  {017402} (\bibinfo {year} {2019})}\BibitemShut {NoStop}%
\bibitem [{\citenamefont {Storey}\ \emph {et~al.}(2007)\citenamefont {Storey},
  \citenamefont {Tallon},\ and\ \citenamefont {Williams}}]{TallStorey}%
  \BibitemOpen
  \bibfield  {author} {\bibinfo {author} {\bibfnamefont {J.~G.}\ \bibnamefont
  {Storey}}, \bibinfo {author} {\bibfnamefont {J.~L.}\ \bibnamefont {Tallon}},
  \ and\ \bibinfo {author} {\bibfnamefont {G.~V.~M.}\ \bibnamefont
  {Williams}},\ }\bibfield  {title} {\enquote {\bibinfo {title} {Saddle-point
  van hove singularity and the phase diagram of high-${T}_{c}$ cuprates},}\
  }\href {\doibase 10.1103/PhysRevB.76.174522} {\bibfield  {journal} {\bibinfo
  {journal} {Phys. Rev. B}\ }\textbf {\bibinfo {volume} {76}},\ \bibinfo
  {pages} {174522} (\bibinfo {year} {2007})}\BibitemShut {NoStop}%

\bibitem [{\citenamefont {Sachdev}(2000)}]{Sachdev}%
  \BibitemOpen
  \bibfield  {author} {\bibinfo {author} {\bibfnamefont {Subir}\ \bibnamefont
  {Sachdev}},\ }\bibfield  {title} {\enquote {\bibinfo {title} {Quantum
  criticality: Competing ground states in low dimensions},}\ }\href {\doibase
  10.1126/science.288.5465.475} {\bibfield  {journal} {\bibinfo  {journal}
  {Science}\ }\textbf {\bibinfo {volume} {288}},\ \bibinfo {pages} {475}
  (\bibinfo {year} {2000})}\BibitemShut {NoStop}%
\bibitem [{\citenamefont {Labbe}\ and\ \citenamefont {Friedel}(1966)}]{Labbe}%
  \BibitemOpen
  \bibfield  {author} {\bibinfo {author} {\bibfnamefont {J.}~\bibnamefont
  {Labbe}}\ and\ \bibinfo {author} {\bibfnamefont {J.}~\bibnamefont
  {Friedel}},\ }\bibfield  {title} {\enquote {\bibinfo {title} {Classification
  of critical points in energy bands based on topology, scaling, and
  symmetry},}\ }\href {\doibase
  https://doi.org/10.1051/jphys:01966002703-4015300} {\bibfield  {journal}
  {\bibinfo  {journal} {J. Phys. France}\ }\textbf {\bibinfo {volume} {27}},\
  \bibinfo {pages} {153 -- 165} (\bibinfo {year} {1966})}\BibitemShut {NoStop}%

\end{thebibliography}%






\section*{Acknowledgements}     
We thank Adrian Feiguin for stimulating discussions. This work is supported by the US Department of Energy, Office of Science, Basic Energy Sciences grant number DE-FG02-07ER46352, and benefited from Northeastern University's Advanced Scientific Computation Center (ASCC) and the allocation of supercomputer time at NERSC through grant number DE-AC02-05CH11231.  The work at LANL was supported by the U.S. DOE NNSA under Cont. No. 89233218CNA000001 through the LANL LDRD Program and the CINT, a DOE BES user facility.  

\section*{Author contributions}
R.S.M., B.S., C.L., and A.B. all contributed to the research reported in this study and the writing of the manuscript.

\section*{Data Availability}
All data supporting the findings of this study are available from the corresponding authors upon request.

\section*{Additional information}
The authors declare no competing financial interests. 


\end{document}.
