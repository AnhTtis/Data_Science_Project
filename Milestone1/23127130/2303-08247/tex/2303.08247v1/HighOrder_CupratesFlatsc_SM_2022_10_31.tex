\documentclass[aps,prb,twocolumn,nopacs,superscriptaddress,longbibliography]{revtex4-1}
\usepackage{amsmath}
\usepackage[breaklinks=true,colorlinks,citecolor=blue,linkcolor=blue,urlcolor=blue]{hyperref}
\usepackage{epsfig,mathrsfs,color,latexsym,subfigure,marginnote,graphicx,verbatim,relsize,mathrsfs,color,array,amsmath,amsfonts,amssymb,graphicx}
\pagestyle{headings}

% Search path for includegraphics+graphicx
\graphicspath{ {./Figs/} } 
\newcommand{\figmaxwidth}{\textwidth}

% Editing notes
\RequirePackage[usenames,dvipsnames]{xcolor}
\newcommand{\bs}[1]{\textcolor{red}{[B]\,[\small\sffamily #1]}}


% Macros
\newcommand{\Fref}[1]{Fig.~\ref{#1}}
\newcommand{\Frefs}[1]{Figs.~\ref{#1}}
\newcommand{\SFref}[1]{supplementary Fig.~\ref{#1}}
\newcommand{\SFrefs}[1]{supplementary Figs.~\ref{#1}}
\newcommand{\Eqref}[1]{Eq.~(\ref{#1})}
\newcommand{\Eqsref}[1]{Eqs.~(\ref{#1})}


% Supplementery material
\newcommand{\SI}{Supplementary Material}
\newcommand{\beginsupplement}{%
  \setcounter{equation}{0}
  \renewcommand{\theequation}{S\arabic{equation}}%
  \setcounter{table}{0}
  \renewcommand{\thetable}{S\arabic{table}}%
  \setcounter{figure}{0}
  \renewcommand{\thefigure}{S\arabic{figure}}%
  \setcounter{section}{0}
  \renewcommand{\thesection}{S-\Roman{section}}%
  \setcounter{subsection}{0}
  \renewcommand{\thesubsection}{S-\Roman{section}.\Alph{subsection}}%
  \renewcommand{\refname}{Supplementary References}%
}
%%%%%%%%%%%%%%%%%%%%%%%%%%%%%%%%%%%%%%%%%%%%%%%%%%%%%

\begin{document}
\title{
--- Supplementary Information ---\\*[1em]
High-$T_c$ superconductors as a New Playground for High-order Van Hove singularities and Flat-band Physics }


\author{Robert S. Markiewicz}
\email{r.markiewicz@northeastern.edu}
\affiliation{Department of Physics, Northeastern University, Boston, Massachusetts 02115, USA}

\author{Bahadur Singh}
\email{bahadur.singh@tifr.res.in}
\affiliation{Department of Condensed Matter Physics and Materials Science, Tata Institute of Fundamental Research, Colaba, Mumbai 400005, India}

\author{Christopher Lane}
\affiliation{Theoretical Division, Los Alamos National Laboratory, Los Alamos, New Mexico 87545, USA}
\affiliation{Center for Integrated Nanotechnologies, Los Alamos National Laboratory, Los Alamos, New Mexico 87545, USA}

\author{Arun Bansil}
\affiliation{Department of Physics, Northeastern University, Boston, Massachusetts 02115, USA}

\maketitle

\beginsupplement
\section{Optimizing $t'$ values}
\begin{figure}
\leavevmode
\rotatebox{0}{\scalebox{0.50}{\includegraphics{lscobbaAF_L6}}}
\vskip0.5cm
\caption{
{\bf Proposed Fermi surface models of Tl- (a,b) and Bi-cuprates (c,d)}, for $t'/t$ = -.32 (a), -0.22 (b), -0.26 (c), and -0.17 (d).}
\label{fig:6}
\end{figure}

The fermi surfaces of the antiferromagnetic, electron-doped cuprates\cite{nparm} have become an iconic image in physics -- hole pockets around $(0,\pi)$ and $(\pi,0)$, electron pocket near $(\pi/2,\pi/2)$, separated by hot-spots. Yet modeling that structure initially proved difficult.  Indeed, in the commonly used $t-t'$ model the hole pocket never appears.  However, Andersen\cite{OKA} provided chemical arguments that cuprates are better described by a $t-t'-t''$ model with the special value $t''=-t'/2$.  By adding precisely this value of $t''$, the hole pocket was readily produced\cite{Kusko}.  Similarly, for hole doped cuprates, modeling them with a $t-t'$ model with different $t'$ for different families of cuprates is unsatisfactory, leading to too strongly nested fermi surfaces for larger $t'$, whereas adding $t''=-t'/2$ produces reasonable results\cite{MBMB}.

While we adopt Andersen's ratio $t''/t'=-1/2$ in this work, our philosophy of using the $t-t'-t''$ model is different from that of Pavarini {\it et al}\cite{PavOK} (PA). PA downfold the DFT dispersions of several bands to a single band representing the cuprates, then fit the band to a single parameter $r$, which is approximately equal to $t'/t$.  We find that improved fits to the full dispersion require more than three hopping parameters,\cite{RSM1b} which can be used to calculate the susceptibility, and thence the self-energies and phase diagrams, either at random-phase approximation\cite{Das} or with mode-coupling corrections\cite{MBMB}. Then, we introduce a {\it reference family} -- the $t-t'-t''$ model for a given cuprate whose susceptibility most closely matches that of the many-$t$ model as a functon of doping and temperature.  In this way, we find that the family with $t''=-0.5t'$ provides a reasonable model for cuprates, where each cuprate is associated with a particular value of $t'$, and one can easily explore parameter space where no cuprates are found.\cite{MBMB}

However, by comparison of individual materials to experiment, we find that the PA values $t'/t=\alpha r$, $\alpha = 1$, tend to overestimate $x_{VHS}$.  Here, we focus on a comparison of Tl2201 and Bi2201, Fig.~\ref{fig:6}, where experimental data on the fermi surfaces are available\cite{Damasc,deHuss,Valla}. In Fig.~\ref{fig:6}, we compare fermi surfaces assuming the PA values of $t'$ ($\alpha =1$) for Tl2201 (frame (a), $t'/t$ = -0.32) and Bi2201 (frame (c), $t'/t$ = -0.26) with smaller values of $\alpha =2/3$ for Tl2201 (frame (b), $t'/t$ = -0.22) and Bi2201 (frame (d), $t'/t$ = -0.17).  Clearly, frame (a) bears little resemblance to the known fermi surface of Ti2201\cite{Damasc,deHuss}, whereas frame (b) provides a much better match.  In contrast, for Bi2201, the $\alpha=1$ result comes much closer to the fermi surface seen in ARPES\cite{Valla}, while the theoretical $x_{VHS}=0.50$ is close to the experimental $0.45$ (contrast $x_{VHS}=0.29$ for frame (d)).  Moreover, the area of the experimental Tl2201 fermi surfaces, not at the VHS, corresponds to a doping $x$ = 0.26\cite{Damasc} or 0.24\cite{deHuss}, to be compared to the dark green curves in frames (a) and (b), both calculated at $x$ = 0.25.  It can be seen that the $t'=-0.22$ fermi surface (for convenience reproduced as a light green line in frame (a)) approaches much closer to $(\pi,0)$, in better agreement with experiment\cite{Damasc,deHuss}.  The corresponding theoretical values of $x_{VHS}$ are 0.563 (frame (a)) and 0.398 (frame (b)).  This suggests that the PA $t'$ values are misordered, and that Bi-cuprates have the largest value of $t'$.  The simplest correction is that $\alpha$ = 1 for Bi2201 and = 2/3 for the other cuprates.  This corrects the Tl2201 value and brings LSCO into good agreement with Ref.~\onlinecite{MBMB}, while keeping the Tl and Hg values close to each other.

The finding that Bi2201 has the largest $t'$ value could explain several puzzles of the Bi cuprates.  First, the small $T_c$ in Bi2201 could be due to too large a value of $t'$, causing the VHS to fall either at the hoVHS or perhaps just off the $(\pi,\pi)$ plateau.  Moreover, the large enhancement of $T_c$ in Bi2212 follows since bilayer splitting reduces the $t'$ of the antibonding band into a more appropriate range of $t'$, with $x_{VHS}\sim 0.32$.\cite{RSM1b} [See discussion below, near Eq.~\ref{eq:4b}.]

We note that with the $\alpha\sim$~2/3 correction, superconductivity in cuprates is optimized for a band structure close to that of the hoVHS.  This also means that high $T_c$ superconductivity is confined on the $(\pi,\pi)$-plateau.  [For comparison purposes, we refer the reader to our original version, ArXiv:2105.04546.v1, Fig. 1, to see the comparison of VHS and $T_c$ for the original PA parameters.]

 


\section{Van Hove dichotomy}
\subsection{At the VHS, vs $t'$}

In the main text, we have considered VHSs with diverging $q=0$ susceptibility. However, the VHS also has a diverging susceptibility at a second wave momentum $Q$ near $(\pi,\pi)$ as well. This is resolved in Fig.~4 of the main text, where the variation of $t'$ shifts the dominant susceptibility peak from $(\pi,\pi)$ to $\Gamma=(0,0)$ as $-t'$ increases. Similar effects arise as doping $x$ increases from 0.  This dichotomy is very relevant for cuprate physics since only the VHS near $(\pi,\pi)$ is correlated with the pseudogap crossover temperature, $T_Q\simeq T_{pg}$. We now demonstrate how these two VHSs compete and evolve with $t'$ and $x$.


\begin{figure}[h!]
\centering 
\includegraphics[width=0.5\textwidth]{flsup1mg6a4_mpi_tpmX_L1c.jpg}
\caption{Susceptibility maps at saddle-point VHS for $t'/t$ = (a) 0.0, (b) -0.05, (c) -0.10, (d) -0.15, (e) -0.20, (f) -0.25, (g) -0.258 [Bi2201], (h) -0.30, (i) -0.35, (j) -0.40, (k) -0.45, and (l) -0.50.   Blue and white identify minimum and maximum intensities, respectively.}
\label{fig:2} 
\end{figure}



To capture the dichotomy, we calculate the full DFT-Lindhard susceptibility, 
\begin{equation}
\chi_0(q,\omega)=-\sum_k\frac{f(\epsilon_k)-f(\epsilon_{k+q})}{\omega -\epsilon_{k+q}+\epsilon_k}
\label{eq:1}
\end{equation}  
The real part of $\chi_0(q,\omega =0)$ contains two contributions: the diverging peaks associated with VHS nesting\cite{RiSc} plus a folded ($q=2k_F$) map of the Fermi surface, whose intensity quantifies the strength of Fermi surface nesting.  The susceptibility is not confined to the Fermi surface, but contains a substantial bulk contribution which can favor certain $k_F$ values, or even shift the peak to a near-by commensurate value, necessitating a careful numerical evaluation of $\chi_0$.\cite{MLSB}
   
Figure~\ref{fig:2} shows the evolution of the susceptibility map at the VHS doping as $t'$ is varied, keeping $t''=-0.5t'$. In \Fref{fig:2}(g), we show the Fermi surface nesting curve $q=2k_F$ (red line) over part of the first Brillouin zone. This clearly identifies that the nonanalytic features of the susceptibility are associated with Fermi-surface nesting.  It can be seen that for each $|t'|$, there are susceptibility divergences at $\Gamma$ and  $(\pi,\pi)$, due to intra- and inter-VHS scattering, respectively.   However for small $|t'|$, the more intense VHS feature is at $(\pi,\pi)$ whereas for larger $|t'|$, the peak at $\Gamma$ becomes most intense.  From Fig.~4, we note that the crossover occurs near $t'=-0.22t$, pinned by the high-order VHS peak.  For intermediate values of $|t'|$, there is an additional peak at $(\pi-\delta,\pi-\delta)$ (see \Fref{fig:2}(h) for example). 
 
To understand the $(\pi,\pi)$ VHS evolution, we recall that for the pure Hubbard model ($t'=0$), the $(\pi,\pi)$-susceptibility has a $log^2T$ divergence that for finite $t'$ evolves to 
$$\chi((\pi,\pi),0)\sim log(T)log(T_X),$$ where $T_X=max\{T,T_{eh}\}$ and $k_BT_{eh}=|t'|$.\cite{VHS2} The dotted black line in Fig.~4 is proportional to $ln(|t'|)$ (with an effective $t'$ at $t'=0$, to account for the finite $k$-mesh), showing that this is a good approximation over a broad range of $t'$, even in the presence of $t''\ne 0$.  



 
Near high symmetry points the susceptibility peak can deviate from the nesting map.  This can be understood as follows, taking the symmetry point $(\pi,\pi)$ as an example.  While the susceptibility at $\Gamma$ is constrained to be a Fermi surface effect, for general $q$ there can be a strong contribution to $\chi_0$ from states far from $E_F$, leading to peaks with very broad tails.  These tails play an important role near high symmetry points, when multiple Fermi surface sections approach the symmetry point.  Figure~\ref{fig:2} illustrates this effect for the $(\pi,\pi)$ plateau.   For larger $|t'|$, Figs. \ref{fig:2}(f)-(l), the weight is concentrated near the nesting map, $q=2k_F$.  However, as $|t'|$ decreases, overlap of the bulk spectral weight rapidly shifts the peak to exactly $(\pi,\pi)$, as the plateau shrinks down to a point for $t'=0$ (Fig. \ref{fig:2}(a)).  Similar overlap effects explain why the susceptibility peaks near $\Gamma$ can shift away from the nesting map.

The pileup of states at $(\pi,\pi)$ for small $|t'|$ has two consequences- first, a prominent commensurate-incommensurate transition\cite{MLSB}, and second, the rapid decrease of intensity at $(\pi,\pi)$ with increasing $|t'|$ seen in Fig.~4.  We note that the Fermi surfaces responsible for the $q\sim (\pi,\pi)$ plateau come from $k_F\sim (\pi/2,\pi/2)$ -- i.e., far from the VHS and hence not sensitive to $t''$.  Indeed, the $t'$-dependence of the $(\pi,\pi)$ plateau area, \Fref{fig:2}, is qualitatively similar to the doping dependence of the plateau area at fixed $t'$, \Fref{fig:1}.


\subsection{Tuning away from the VHS at fixed $t'$}
While the $t'$ dependence of the VHS susceptibility is easy to calculate, since only the doping $x_{VHS}$ is involved, the doping dependence at fixed $t'$ (i.e., for a particular cuprate) is often more important.  This is more complex since, for $x \ne x_{VHS}$, each VHS evolves into a finite susceptibility peak with different doping dependence.  We illustrate this evolution  in Figure~\ref{fig:1} for a typical $t'/t = -0.3$.  The susceptibility has a number of nonanalytic peaks, associated with Fermi surface nesting.  To illustrate this, we include superposed maps of the main barrel Fermi surfaces, doubled and folded back to the first Brillouin zone (BZ), to highlight the $q=2k_F$ nesting features.\cite{MLSB}  Additional features in frames (f-h) are associated with non-$2k_F$ nesting,\cite{MLSB,MGu}, e.g., inter-Fermi surface pocket nesting as illustrated in frame (b) for the doping of frame (g).  

To better understand the structure in these susceptibility maps, we briefly review the evolution of the Fermi surface, and its two topological transitions\cite{MLSB,MGu}.  Figure~\ref{fig:1}(a) shows the corresponding DOS, with points at which the susceptibility maps were sampled.  The DOS has two VHSs -- a logarithmic peak where a pocket pinches off near $(\pi,0)$, frame (f), and a step down when the pocket disappears with further hole doping, frame (h).  Due to the folding, the pockets appear near $\Gamma$ in the susceptibility maps.  


\begin{figure}[h!]
\centering 
\includegraphics[width=0.5\textwidth]{flsup1mg6a4_mpi_tpm3_L2d}
\caption{{\bf Susceptibility maps showing evolution of nesting lines with doping for $t'/t=-0.3$, $t''/t'=-0.5.$} 
(a) DOS, showing doping of each susceptibility map. (b) Fermi surface for frame (g) showing three inter-surface nesting vectors.  (c-h) Susceptibility maps, with $q=2k_F$ nesting surface superposed as a red line; in frames (f-h), one branch is omitted to emphasize details of the nesting maps. Color scheme same as \Fref{fig:2}.  In frames (c-h), the $(\pi,\pi)$-plateau is defined as the area enclosed by the folded Fermi surface nearest to $(\pi,\pi).$  Arrows in (g) match those in (b).}
\label{fig:1}
\end{figure}

 

For present purposes, our main result is to see how the VHS competition plays out as the Fermi level is shifted away from $E_{VHS}$.  As doping $x$ increases, the dominant nesting vector shifts away from $(\pi,\pi-\delta)$ at low doping (violet triangles in frame (a)) to $(\delta',\delta')$, or antinodal nesting (ANN), at higher doping (blue dots), near $x_{cross}\sim 0.12$.   Experimentally, the peak at $(\pi,\pi-\delta)$ is associated with an incommensurate SDW, while the ANN peak drives a CDW instability.  As the VHS peak is approached, the ANN peak flows to $\Gamma$, $\delta'\rightarrow 0$ at the VHS, while the peak at $(\pi,\pi-\delta)$ evolves to a weak peak at $(\pi-\delta'',\pi-\delta'')$ at the VHS, green circle in \Fref{fig:1}(f). Thus we have demonstrated that this competition is controlled by the competition between intra-VHS scattering near $\Gamma$ and the inter-VHS scattering near $(\pi,\pi)$.
  That is, the VHSs are important not only at $x_{VHS}$, but influence the dominant susceptibility over a wide doping range.  

The peak crossover seen at $x_{cross}=0.12$ in \Fref{fig:1} is reflected in a similar crossover of the VHS peaks, Fig.~4, at $t'_{cross}\sim -0.22$, for $t''=-0.5t'$.  For all $t'>t'_{cross}$ there is no $x_{cross}$ and the $(\pi,\pi)$-VHS is always dominant, while for $t'<t'_{cross}$ there is always an $x_{cross}$ where the influence of the $(\pi,\pi)$-VHS is lost.  Note that in the example of \Fref{fig:1}, $x_{cross}=0.12 << x_{VHS}=0.54$.

\section{More on secondary and bosonic VHSs}
\subsection{Many-body perturbation theory background}
The calculations of this paper are based on a many-body perturbation theory (MBPT) of the cuprates\cite{Das,MBMB}, using tight-binding approximations to DFT dispersions.  At the time these were developed, it was unclear whether DFT could capture the gapped physics of strongly correlated materials like cuprates, and it remains a useful tool for understanding the newer DFT results.  The theory implements Hedin's scheme\cite{Hedin}, wherein MBPT can be formally solved exactly by taking the lowest order perturbation theory results for the electron Green's function $G_0$, self energy $\Sigma_0$, susceptibility $\chi_0$, and vertex correction $\Gamma_0$, and replacing them by the fully dressed functions, $G_0\rightarrow G$, etc., wherein the dressed functions must be simultaneously found self-consistently.  

In Ref.~\onlinecite{Das}, an approximate $G_ZW_Z$ scheme was developed for solving these equations in the limit where vertex corrections are ignored, $\Gamma\rightarrow 1$ -- a form of GW approximation\cite{Hedin}.  The $G_ZW_Z$ scheme successfully captured the splitting of the electronic dispersion into a low energy branch of coherently dressed quasiparticles and a high energy incoherent background, separated by an almost discontinuous step in the dispersion at an intermediate energy -- the waterfall effect.  For present purposes, the key result is that the incoherent part of the spectrum can be ignored for the low energy physics, and the coherent part can be approximated as the DFT dispersion renormalized by a constant factor $Z<1$.  The resulting gap equation can then be approximated by either the random-phase approximation (RPA) or the Hartree-Fock gap equation for the renormalized quasiparticles in terms of a renormalized Hubbard interaction $U$.  Since DFT calculations do not capture the renormalization, we further simplify our calculation by setting $Z\rightarrow 1$, which does not change the fermi surface or the shape of the DOS, except for an overall factor of $Z$.  These results are discussed in Section S-III.B.

Reference~\onlinecite{MBMB} extended the quasiparticle part of Ref.~\onlinecite{Das} by including a vertex correction to account for mode coupling.  This is important to account for the dominance of short-range order in cuprates, produced by a combination of Mermin-Wagner\cite{MW} physics and frustration from competition between different electronic instabilities -- an electronic analog of McMillan's bosonic entropy in srongly-correlated charge density waves\cite{McMill}.  In defining the mode coupling, a natural parameter arose -- the inverse-susceptibility DOS\cite{MBMB}.  Here we extend the analogy between electronic spin-density waves and phonon-induced charge density waves by rewriting the electronic susceptibility in terms of the Green's function of an electronic boson, which represents an electron-hole pair or parexciton.  This idea is developed in Section. S-III.D.

\subsection{AFM VHSs}
To our knowledge, the self-consistent Hartree-Foch calculation was the first to demonstrate the important role of the VHS of the lower AFM band in causing the first-order collapse of the AFM order.\cite{RSMstripe}  For arbitrary hopping parameters, AFM order splits the band into lower and upper subbands, with dispersion
\begin{equation}
\epsilon_{\pm}=\epsilon_{0+}\pm\sqrt{\epsilon_{0-}^2+\Delta^2},
\label{eq:14}
\end{equation}
where $\epsilon_{0-}=-2t(c_x+c_y)$, $\epsilon_{0+}=-4t'c_xc_y-2t''(c_{2x}+c_{2y})$, and the gap $\Delta$ must be found self-consistently as a function of doping.  Physically, the DOS of the lower band $\epsilon_-$ is more important, although for the Hubbard model ($t'=t''=0$) discussed in the text both bands have the same dispersion.  Since the AFM order is sensitive to the exact value of $U$, we use the quaiparticle-GW parameters\cite{Das} $U=3t$ and $Z=0.5$, following Ref.~\onlinecite{gapsend}.

We here discuss Eq.~\ref{eq:14} for more general dispersions, recalling that in general $\epsilon(k)_{0\pm}=(\epsilon(k)\pm\epsilon(k+Q)/2$, where $Q$ is the AFM wave number, in our case $Q=(\pi,\pi)/a$, and we explicitly show the lattice constant $a$.  We find that the AFM VHS evolution is controlled by the dispersion along the AFM zone boundary (ZB), which corresponds to $k_x+k_y=\pi/a$, plus symmetry-related lines.  In Fig.~\ref{fig:A9}, we show how the VHS evolves as a function of $t'$ for the self-consistent gap parameter, $\Delta=US$,\cite{gapsend}, where the value of the dimensionless magnetization $S=<n_{\uparrow}- n_{\downarrow}>/2$ are listed in the figure caption.  Along the AFM ZB $\epsilon(k)_{0}$ is identically zero, for arbitrary dispersion, and Eq.~\ref{eq:14} reduces to $\epsilon_{\pm}=\epsilon_{0+}\pm\Delta$, while along the AFM ZB $\epsilon_{0+}$ simplifies, becoming $\epsilon_{0+}=4t'c_x^2-4t''c_{2x}=4(t'-2t'')c_x^2+4t''$ for the $t-t'-t''$ model.  Thus for all $t-t'-t''$ models, the lower AFM band has an identical (4-sided) Mexican-hat minimum, 
\begin{equation}
\epsilon_-=C+Ac_x^2,
\label{eq:15}
\end{equation} 
where $C=4t''-\Delta$ and $A=4(t'-2t'')$ are constants. This behavior is clearly illustrated in Fig.~\ref{fig:A9}(b), where we plot half of the AFM ZB dispersion, $X\rightarrow M/2 (\rightarrow Y)$.  

This simple evolution along the AFM ZB leads to a universal behavior for the VHSs, Fig.~\ref{fig:A9}.  In an ordered phase, the dispersion in the superlattice Brillouin zone must be unfolded into the primitive Brillouin zone for comparison to experimental angle-resolved photoemission (ARPES) dispersions.  This process leads to a modulation of the spectral weight by a coherence factor, as discussed in conjunction with Fig.~2(b). Figure~\ref{fig:A9}(a) shows the corresponding dispersions for the AFM bands for 6 different values of $t'/t$, covering the range of the cuprates, with the width of the dispersion line being proportional to the spectral intensity.  This plot is repeated for the lower magnetic band in Fig.~\ref{fig:A9}(b) without the spectral weight for ease in identifying the VHS features.  For each curve, the fermi energy is adjusted so the VHS is at the fermi level, $E=0$.  From these figures, it is clear that the VHS is associated with the top of the dispersion along the lines $\Gamma\rightarrow X$ and $X\rightarrow M$.  [Parenthetically, we note that this is a good method for finding the energy of the VHS peak. Fig.~\ref{fig:A9}(h) shows how a small offset in $E_f$ distorts the corresponding fermi surface.]  

Having found the fermi energy of the VHS peak, Figs.~\ref{fig:A9}(d-i) show the six corresponding fermi surfaces.  In each frame, we plot the spectral weight for all data points within 5~meV of the fermi level, with dark blue corresponding to zero weight, and white to weight = 1.  As a check, the red dots correspond to the points where the dispersions cross the fermi level along the line $M\rightarrow \Gamma$ in frame (b).  The resulting pattern is remarkably simple: for $t'=0$, there is no dispersion along the AFM ZB, so the fermi surface lies along this line, spreading out near $X$ and $Y$.  When $t'\ne 0$, the ZB develops a dispersion, and the VHS fermi surface evolves into a box with the long sides split away symmetrically from the ZB, with the splitting increasing with $|t'|$.  While for $t'=0$, the DOS has a power-law divergence at the VHS, for $t'\ne 0$ one finds an enhanced logarithmic divergence\cite{gapsend}.




\begin{figure}[h!]
\centering 
\includegraphics[width=0.5\textwidth]{lscobba_flatbott_L1}
\caption{{\bf (a-c) AFM dispersion maps showing evolution of VHS energy (at $E=0$) with $t'$} for $t''/t'=-0.5,$ with (a) and without (b,c) spectral weight correction.  Colors in (a,b) and VHS energy surface in (d-i) correspond to $t'/t = $ (d) 0 (blue, $\Delta = 0.283$ eV), (e) -0.08 (violet, 0.2828), (f) -0.12 (red, 0.2829), (g) -0.17 (orange, 0.3044), (h) -0.26 (gold, 0.3482), and (i) -0.32 (light green, 0.3674).  Frame (b): corresponding dispersions along AFM zone boundary, $X\rightarrow M/2$,  showing characteristic $C+Ac_x^2$ form, with $A\propto t'$.  (c): contrast of hoVHS dispersions for NM (light green, from Fig.~1(b)) and AFM (blue, from (a)).}
\label{fig:A9}
\end{figure}

Since $\epsilon_{0-}$ universally vanishes on the AFM zone boundary, can we say something similar about $\epsilon_{0+}$ beyond the $t-t'-t''$ model?  Remarkably, we can.  In Section IV.B, we show that the tight-binding expansion including arbitrary higher order hopping in a 1-band model of cuprates can be rewritten as a Taylor series with arbitrary powers of $X=(c_x+c_y)$ and $Y=(c_x-c_y)$.  Along the AFM zone boundary, $X=0$ and $Y=2c_x$.  Thus, including more distant hopping can change Eq.~\ref{eq:15} to an arbitrary polynomial in $c_x^2$, where only even powers are allowed to preserve the $x-y$ symmetry of the lattice. It is likely that these extra terms will lead to more complicated shapes of VHS fermi surfaces.

\subsection{Excitonic Insulators}
While an exciton is usually considered to be a photoinduced excited state, an interesting situation arises if the exciton binding energy $E_x$ exceeds the gap between the conduction and valence bands from which it is formed.\cite{Kohn}  Then, the situation resembles a Cooper pair in a superconductor -- the assumed ground state is unstable and the true ground state is a condensate of excitons, or an excitonic insulator.  While the resulting formalism is similar to that of a superconductor, there is an important difference.  In a superconductor, the hole band is just the electron band reflected about the fermi energy, so the bands are always perfectly nested.  In the excitonic insulator the conduction and valence bands are independent, could fall at different $k$-points, and need not be well nested, all of which can affect the insulator's stability.  

In the text, we explored only the simplest case of perfect nesting at the same $k$ point.  To relate this problem to the present results, we assumed that the valence band $E_v$ is represented by the same $t-t'-t''$ model, and to ensure perfect nesting, we assumed that the shifted (by $E_x$) conduction band $E_c$ is just the mirror image of the valence band.  This leads to excitonic insulator bands of the same form as Eq.~\ref{eq:14}, but with $\epsilon_{0\pm}=(E_v\pm E_c)/2$, or for our choice of bands $\epsilon_{0+}=0$, $\epsilon_{0-}=E_v-\epsilon_f$, where $\epsilon_f$ is the fermi energy.

The same model would also describe the bands in a superconductor, except that in the cuprates the gap has d-wave symmetry rather than being constant.  We also note that similar phenomena may be found in topological nodal line semimetals\cite{NLSM1,NLSM2}, which have a closed nodal line of band degeneracy in the bulk and a sheet of drumhead surface states on the sample surface.  However, in general both the loop and the surface states can be strongly dispersing and it is not clear if a recipe for producing flat bands can be found.

\subsection{bosonic DOS}
In this Subsection, we discuss the inverse susceptibility DOS (ISDOS) introduced in Ref.~\onlinecite{MBMB}.  However, to put it in more familiar terms, we formally redefine the susceptibility in terms of a bosonic Green's function, in which case the ISDOS is simply the resulting bosonic DOS, and the resulting hoVHS peaks quantify McMillan's ideas of bosonic entropy driving strongly coupled CDW physics\cite{McMill}.  For the RPA, a magnetic instability arises from Dyson's equation, $\chi^{-1} =\chi_0^{-1}-U$, when the dressed susceptibility $\chi$ diverges.  In terms of the bosonic model, this should be a soft-mode instability, $\omega_q=\omega_{q0}-2U\rightarrow 0$.  However, it was found\cite{MBMB} that the RPA breaks down in the presence of strong mode coupling.  Specifically, the bosonic DOS quantifies the number of competing modes, adding a correction to the denominator of $\chi$ which slows its divergence.  

By reviewing the susceptibility maps in Fig.~\ref{fig:1}, we can understand what is happening in Fig.~3.  We focus on Fig.~\ref{fig:1}(c), where the effect is most clearly seen.  The susceptibility contains a nesting map -- a folded image of the fermi surface (red line) where the intensity represents the strength of fermi surface nesting for each point of the fermi surface.  In addition there is a $(\pi,\pi)$ plateau, the roughly diamond-shaped region of the susceptibility map centered on $(\pi,\pi)$ and extending to the fermi surface map on all sides (bright white area in Fig.~\ref{fig:1}(c).   Due to the fermi function cutoff, the susceptibility is much stronger on the plateau than off of it.  Moreover, as the plateau area decreases (lower doping or smaller $|t'|$), the tails of the VHS peaks overlap and push the suscepibility peak to exactly $(\pi,\pi)$ -- away from conventional fermi surface nesting.     

Thus, for a given material ($t'$), at low doping the susceptibility peaks at the commensurate value $(\pi,\pi)$ independent of doping.  As doping increases, the $(\pi,\pi)$ plateau expands, the peak height decreases, and at some point $(\pi,\pi)$ turns into a local minimum with the maximum susceptibility moved to an incommensurate $q$ on the ring of fermi surface nesting.  At the crossover the susceptibility becomes very flat on the plateau -- the coefficient of the $k^2$ term in a $k\cdot p$ expansion must vanish -- and the area of the flat band in Fig.~3 is approximately the area of the $(\pi,\pi)$-plateau.  Moreover, we have adjusted $t'=-0.267t$ (with $t''=0$) in Fig.~3 so that $\omega_q$ (as defined at RPA level) softens almost to zero as $T\rightarrow 0.$  However, due to the large bosonic DOS, the mode coupling correction overwhelms the mode softening, so that the correlation length remains very small even as $T\rightarrow 0$ leaving a spin liquid phase surrounded by commensurate or incommensurate phases with much larger correlation lengths.\cite{MBMB}  Reference~\onlinecite{MBMB} noted an interesting connection between this commensurate-incommensurate transition and the excited-state quantum phase transitions in the spectrum-generating algebras of nuclear theory.\cite{Iach}

\section{Improved tight-binding model}
\subsection{New $ln^2T$ VHSs}
In this Section, we propose a further generalization of higher-order VHSs, in which the faster-than-logarithmic divergence is not in the DOS, but in a competing, finite-$q$ susceptibility.  At present, the only known example is the original Hubbard model, $t'=t''=0$, with a $ln^2(T)$ $q=(\pi,\pi)$ susceptibility divergence.  Here, we demonstrate that the Hubbard model is not unique by generating a family of dispersions with this property.  As a byproduct, we demonstrate an improved technique for generating tight-binding models, in which the hopping parameters are approximately orthogonal.

Here, we generalize the analysis of Ref.~\onlinecite{VHS2} for finite $t''$.  The susceptibility divergence arises from the energy integral in Eq.~\ref{eq:1}, with $\epsilon_k$ near, e.g., $(\pi,0)$, while $\epsilon_{k+Q}$ is near $(0,\pi)$.  For the $t-t'-t''$ model, the $t'$ and $t''$ terms cancel, and the denominator becomes $\epsilon_k-\epsilon_{k+Q}=-4t(c_x+c_y)$.  Near $(0,\pi)$, $c_x+c_y\sim k_x^2-k_y^{'2} \sim k^2cos(2\phi)$, with $k_y'=\pi-k_y$.  Then the integral $\int kdk/k^2$ produces a logarithmic divergence, which can be cut off by, e.g., the temperature $T$. The angle integral $\int d\phi/cos(2\phi)$ also can diverge, if $\phi\rightarrow\pi/4$, leading to a second $ln(T)$ factor.   However, the fermi functions in Eq.~\ref{eq:1} can cut off this divergence, since at low $T$ they only allow scattering from full to empty states, and hence cut the $\phi$ integral off at the Fermi surface.  To test this possibility, we rewrite Eq.~1 by using $cos(2\theta)=2cos^2(\theta)-1$, as
\begin{equation}
\epsilon_k=-2t(c_x+c_y)-(t'+2t'')(c_x+c_y)^2+(t'-2t'')(c_x-c_y)^2
+4t''.
\label{eq:4}
\end{equation}
Thus, when $t''\ne 0$, the cutoff of Eq.~\ref{eq:4} becomes $k_BT_{eh}=|t'-2t''|$, and when $2t''=t'$, the susceptibility has a $ln^2(T)$-divergence independent of $t'+t''$, leading to a new reference family of anomalous VHSs, \Fref{fig:A8}.

\begin{figure}[h!]
\centering 
\includegraphics[width=0.5\textwidth]{lscobba_diag_L0}
\caption{{\bf Dispersion maps $E-E_X$ showing evolution of constant energy surfaces with $t'$ for $t''/t'=0.5.$}  $t'/t = $ (a) 0, (b) -0.5, (c) -1, and (d) -0.2.  Constant energy contours are labeled in eV.}
\label{fig:A8}
\end{figure}


Figure~\ref{fig:A8} shows that the dispersion changes strongly with $t'$, whereas the contours of constant energy are independent of $t'$, since they represent $c_x+c_y=R=constant$.  Consequently, for all values of $t'$, the zone diagonal $(0,\pi)\rightarrow (\pi,0)$ lies at the VHS, $E_F=E_X$, ensuring a square Fermi surface and a $ln^2(T)$-divergence of the $(\pi,\pi)$ susceptibility.  However, there is still a competition of $(\pi,\pi)$ and $\Gamma$ susceptibilities, except now in terms of high-order VHSs.  Note that for $t'=0$ (the conventional Hubbard model) the dispersion is electron-hole symmetric.  For $t'<0$, the dispersion becomes asymmetric, with a local peak developing at $\Gamma$, leading to a Mexican hat potential, which moves toward the $(0,\pi)\rightarrow (\pi,0)$ line as $|t'|$ increases.  The minimum of the Mexican hat produces a flat band quite similar to that found in secondary high-order VHSs.  

This similarity is not accidental.  For a mean-field model of $Q=(\pi,\pi)$ AFM order, the dispersion in the AFM phase becomes $E_{\pm}= \epsilon_+\pm\sqrt{\epsilon_-^2+U^2/4}$, where $\epsilon_{\pm}=(\epsilon_k\pm\epsilon_{k+Q})/2$.  When $U$ is large, the square-root term becomes $U/2 +\epsilon_-^2/U$.  In the $t-t'-t''$ families, $\epsilon_-=-2t(c_x+c_y)$, and the square-root term becomes $U/2 +J(c_x+c_y)^2$, $J=4t^2/ U$, having the same nonlinear term as Eq.~\ref{eq:4}.

Thus, remarkably, whereas the DOS is highly sensitive to the Fermi surface, the $q=(\pi,\pi)$ susceptibility obeys the simpler result of Eq.~\ref{eq:4}, and can only have $ln$ or $ln^2$ divergences in the $(\pi,\pi)$ susceptibility.  We note that the original Hubbard model required extreme fine tuning (all hopping parameters except $t$ equal to zero), calling into question the relevance of a square Fermi surface and resultant $ln^2$ divergence.  By showing that a square Fermi surface can be found along an entire cut in parameter space, we have greatly enhanced the likelihood that such a feature, with accompanying high-order VHSs, might be experimentally observed.  

Finally, we have defined our parameter space in terms of just the three nearest-neighbor hopping parameters, and for this to be of value, it is necessary that higher-order hopping parameters have negligible effects in most cases.  This new, `Hubbard' family offers a simple test: can a line of square-Fermi-surface states persist as higher order hopping parameters are included?  We explore this issue by looking at the most general TB model with terms of order $c_i^3$ ($E_3$) and $c_i^4$ ($E_4$), $\epsilon_{k,4} = \epsilon_k +E_3 +E_4$.  Here $\epsilon_k$ is given by Eq~1 and
\begin{multline}
E_3=-2t_3(c_{3x}+c_{3y})-4t_4(c_{2x}c_y+c_{2y}c_x)\\
=2a_3(c_x+c_y)-4A_3(c_x+c_y)^3-4B_3(c_x-c_y)^3,
\label{eq:4a}
\end{multline}
with $a_3=t_3+6t_4$, $A_3,B_3=t_3\pm t_4/3$,
\begin{multline}E_4=-2t_5(c_{4x}+c_{4y})-4t_6(c_{3x}c_y+c_{3y}c_x)-4t_7c_{2x}c_{2y}\\
=e_{40}-a_4(c_x+c_y)^2-b_4(c_x-c_y)^2-2A_4(c_x+c_y)^4\\
-2B_4(c_x-c_y)^4
-4C_4(c_xc_y)^2,
\label{eq:4b}
\end{multline}
with $e_{40}=-4(t_5+t_7)$, $a_4,b_4=4(t_5+t_7)\mp 3t_6$, $A_4,B_4=4t_5\pm t_6$, $C_4=4(t_7-6t_5)$.  Finally, since $c_xc_y=[(c_x+c_y)^2-(c_x-c_y)^2]/4$, $\epsilon_{k,4}$ can be written as a polynomial in $(c_x+c_y)$ and $(c_x-c_y)$.  Eliminating all the terms in $(c_x-c_y)$ leaves behind a polynomial in $(c_x+c_y)$, a 4th order generalization of the Hubbard reference family.

\subsection{Double Taylor series expansion}
We note in passing that our new series expansion, a Taylor series in the two variables $X=c_x+c_y$ and $Y=c_x-c_y$, should be a significant improvement over the conventional neighbor-by-neighbor tight-binding model.  We see from Eqs.~\ref{eq:4a} and~\ref{eq:4b} that each additional neighbor introduces two kinds of correction: new terms in $X$ and $Y$ (terms whose coefficients are capital letters on right-hand side of equations) and renormalizations of lower-order terms (terms with lower-case coefficients).  In contrast, the new series is essentially a symmetry-corrected Fourier series, ensuring that the new terms added at each order are orthogonal to the lower-order terms, and will be accepted only if they genuinely improve the fit to the dispersion.  Moreover, the series in X is electron-hole symmetric while the series in Y is not.  Only the latter contributes to shifting the VHS from half-filling.

For the monolayer cuprates, Eq.~\ref{eq:4} becomes
\begin{equation}
\epsilon_k=-2t(c_x+c_y)+2t'[(c_x-c_y)^2-1].
\label{eq:4b}
\end{equation}
For bilayer cuprates, hopping between the two layers making up the bilayer can lead to significant modifications of Eq.~\ref{eq:4b}, while inter-bilayer hopping remains weak.  In this case, the resulting dispersions depend on the layer stacking.  For YBCO, the CuO$_2$ layers stack Cu above Cu, and the bilayer hopping leads to a constant splitting of the bonding and antibonding bands, with no change of $t'$.  In contrast, many cuprates, including Bi2212, have a body-centered tetragonal stacking, leading to an approximate dispersive interlayer hopping\cite{PWA}
\begin{equation}
\epsilon_{bi}=\pm 2t_{bi}(c_x-c_y)^2.
\label{eq:4c}
\end{equation}
Combining with Eq.~\ref{eq:4b}, this leads to a splitting of the bonding and antibonding bands of $\pm 2t_{bi}$, and a change of dispersion, $t'\rightarrow t'\pm t_{bi}$.  More detailed analyses for particular cuprates are found in Ref.~\onlinecite{RSM1b}.

For Bi2212, this shifts the antibonding VHS into a parameter range more suitable for high $T_c$, while shifting the bonding band off of the $(\pi,\pi)$ plateau, leading to its distinctive square-cross fermi surface, as in Fig.~\ref{fig:6}(a).

\begin{thebibliography}{99}
\bibitem{nparm}N. P. Armitage, D. H. Lu, C. Kim, A. Damascelli, K. M. Shen, F. Ronning, D. L. Feng, P. Bogdanov, Z.-X. Shen, Y. Onose, Y. Taguchi, Y. Tokura, P. K. Mang, N. Kaneko, and M. Greven,
Anomalous Electronic Structure and Pseudogap Effects in Nd$_{1.85}$Ce$0.15$CuO$_{4}$,
Phys. Rev. Lett. 87, 147003 (2001).
\bibitem{OKA}O.K. Andersen,
LDA energy bands, low-energy hamiltonians, t', t'', $t_{\perp}$ (k), and $J_{\perp}$, 
J. Phys. Chem. Solids 56, 1573 (1995).
\bibitem{Kusko}C. Kusko, R. S. Markiewicz, M. Lindroos, and A. Bansil,
Fermi surface evolution and collapse of the Mott pseudogap in Nd$_{2-x}$Ce$_x$CuO$_{4\pm\delta}$,
Phys. Rev. B 66, 140513(R) (2002).
\bibitem{MBMB}R.S. Markiewicz, I.G. Buda, P. Mistark, an A. Bansil, 
Entropic origin of pseudogap physics and a Mott-Slater transition in cuprates
{\it Nature Scientific Reports} {\bf 7,} 44008 (2017).
\bibitem{PavOK}E. Pavarini, I. Dasgupta,  T. Saha-Dasgupta, O. Jepsen, and O.K. Andersen,   
Band-structure trend in hole-doped cuprates and correlation with T$_{c max}$.
{\it Phys. Rev. Lett.} {\bf 87,} 047003 (2001).
\bibitem{RSM1b}R. S. Markiewicz, S. Sahrakorpi, M. Lindroos, Hsin Lin, and A. Bansil,
One-band tight-binding model parametrization of the high-T$_c$  cuprates including the effect of k$_z$  dispersion,
Phys. Rev. B 72, 054519 (2005).
\bibitem{Das}T. Das, R.S. Markiewicz, and A. Bansil,
Intermediate coupling model of the cuprates,
Adv. in Phys. 63, 151 (2014).
\bibitem{Damasc}M. Plat\'e, J. D. F. Mottershead, I. S. Elfimov, D. C. Peets, Ruixing Liang, D. A. Bonn, W. N. Hardy, S. Chiuzbaian, M. Falub, M. Shi, L. Patthey, and A. Damascelli,
Fermi Surface and Quasiparticle Excitations of Overdoped 
Tl$_2$Ba$_2$CuO$_{6+\delta}$,
Phys. Rev. Lett. 95, 077001 (2005).
\bibitem{deHuss}P.M.C. Rourke, A.F. Bangura, T.M. Benseman, M. Matusiak, J.R. Cooper, A. Carrington, and N.E. Hussey,
A detailed de Haas–van Alphen effect study of the
overdoped cuprate Tl$_2$Ba$_2$CuO$_{6+\delta}$,
New Journal of Physics 12, 105009 (2010).
\bibitem{Valla}T. Valla, P. Pervan, I. Pletikosi\'c, I. K. Drozdov, Asish K. Kundu, Zebin Wu, and G. D. Gu,
Hole-like Fermi surface in the overdoped non-superconducting
Bi$_{1.8}$Pb$_{0.4}$Sr$_2$CuO$_{6+\delta}$,
EPL, 134, 17002 (2021).
\bibitem{RiSc}T. M. Rice and G. K. Scott,
New Mechanism for a Charge-Density-Wave Instability,
Phys. Rev. Lett. 35, 120 (1975). 

\bibitem{MLSB}R.S. Markiewicz, J.  Lorenzana, G. Seibold, and A. Bansil,  
Gutzwiller magnetic phase diagram of the cuprates.
{\it Phys. Rev. B}{\bf 81,} 014509 (2010).
\bibitem{MGu}R.S. Markiewicz, J.  Lorenzana, and G. Seibold, 
Gutzwiller magnetic phase diagram of the undoped $t$--$t'$--$U$ Hubbard model
{\it Phys. Rev. B}{\bf 81,} 014510 (2010).
\bibitem{Hedin}L. Hedin, 
On correlation effects in electron spectroscopies and the GW approximation,
J. Phys. Condens. Matter 11, R489 (1999).
\bibitem{MW}N.D. Mermin and H. Wagner,  
Absence of ferromagnetism or antiferromagnetism in one- or two-dimensional isotropic Heisenberg models,
Phys. Rev. Lett., 17, 1133 (1966).
\bibitem{McMill}W.L. McMillan,  
Microscopic model of charge-density waves in 2H-TaSe$_2$.
Phys. Rev. B16, 643 (1977).
\bibitem{RSMstripe}R.S. Markiewicz, Dispersion of ordered stripe phases in the cuprates, Phys. Rev. B 62, 1252 (2000).
\bibitem{gapsend}R.S. Markiewicz and A. Bansil,
Theory of Cuprate Pseudogap as Antiferromagnetic Order with Domain Walls,
arXiv:2206.00077.
\bibitem{Kohn}W. Kohn, ``Metals and insulators'' in {\it Many Body Physics}, edited by C. DeWitt and R. Balian (Gordon \& Breach, New York, 1968), pp. 351-411.
\bibitem{NLSM1}A. A. Burkov, M. D. Hook, and L. Balents, Topological nodal
semimetals, Phys. Rev. B 84, 235126 (2011).
\bibitem{NLSM2}C. Fang, Y. Chen, H.-Y. Kee, and L. Fu, Topological nodal line
semimetals with and without spin-orbital coupling, Phys. Rev. B 92, 081201 (2015).
\bibitem{Iach}M. Caprio, P. Cejnar, and F. Iachello, Excited state quantum phase transitions in many-body systems 
Ann. Phys. 323, 1106 (2008).
\bibitem{VHS2}R.S.Markiewicz, 
A survey of the Van Hove scenario for high-$T_c$ superconductivity with special emphasis on pseudogaps and striped phases
{\it J. Phys. Chem. Sol.} {\bf 58,} 1179 (1997).
\bibitem{PWA}S. Chakravarty, A. Sudb\/o, P.W. Anderson, and S. Strong,
Interlayer tunneling and gap anisotropy in high-temperature superconductors,
Science 261, 337 (1993).



\end{thebibliography}

\end{document}
