%%
%% This is file `sample-acmsmall.tex',
%% generated with the docstrip utility.
%%
%% The original source files were:
%%
%% samples.dtx  (with options: `acmsmall')
%% 
%% IMPORTANT NOTICE:
%% 
%% For the copyright see the source file.
%% 
%% Any modified versions of this file must be renamed
%% with new filenames distinct from sample-acmsmall.tex.
%% 
%% For distribution of the original source see the terms
%% for copying and modification in the file samples.dtx.
%% 
%% This generated file may be distributed as long as the
%% original source files, as listed above, are part of the
%% same distribution. (The sources need not necessarily be
%% in the same archive or directory.)
%%
%% Commands for TeXCount
%TC:macro \cite [option:text,text]
%TC:macro \citep [option:text,text]
%TC:macro \citet [option:text,text]
%TC:envir table 0 1
%TC:envir table* 0 1
%TC:envir tabular [ignore] word
%TC:envir displaymath 0 word
%TC:envir math 0 word
%TC:envir comment 0 0
%%
%%
%% The first command in your LaTeX source must be the \documentclass command.
\documentclass[acmsmall]{acmart}
%% NOTE that a single column version is required for 
%% submission and peer review. This can be done by changing
%% the \doucmentclass[...]{acmart} in this template to 
%% \documentclass[manuscript,screen]{acmart}
%% 
%% To ensure 100% compatibility, please check the white list of
%% approved LaTeX packages to be used with the Master Article Template at
%% https://www.acm.org/publications/taps/whitelist-of-latex-packages 
%% before creating your document. The white list page provides 
%% information on how to submit additional LaTeX packages for 
%% review and adoption.
%% Fonts used in the template cannot be substituted; margin 
%% adjustments are not allowed.
%%
%% \BibTeX command to typeset BibTeX logo in the docs
\AtBeginDocument{%
  \providecommand\BibTeX{{%
    \normalfont B\kern-0.5em{\scshape i\kern-0.25em b}\kern-0.8em\TeX}}}

%% Rights management information.  This information is sent to you
%% when you complete the rights form.  These commands have SAMPLE
%% values in them; it is your responsibility as an author to replace
%% the commands and values with those provided to you when you
%% complete the rights form.
\setcopyright{acmcopyright}
\copyrightyear{2018}
\acmYear{2018}
\acmDOI{XXXXXXX.XXXXXXX}


%%
%% These commands are for a JOURNAL article.
\acmJournal{JACM}
\acmVolume{37}
\acmNumber{4}
\acmArticle{111}
\acmMonth{8}
\usepackage{multirow}
\usepackage{bbding}
\usepackage{graphicx}  %插入图片的宏包
\usepackage{float}  %设置图片浮动位置的宏包
\usepackage{subfigure}  %插入多图时用子图显示的宏包
% \usepackage[numbers]{natbib}

%%
%% Submission ID.
%% Use this when submitting an article to a sponsored event. You'll
%% receive a unique submission ID from the organizers
%% of the event, and this ID should be used as the parameter to this command.
%%\acmSubmissionID{123-A56-BU3}

%%
%% For managing citations, it is recommended to use bibliography
%% files in BibTeX format.
%%
%% You can then either use BibTeX with the ACM-Reference-Format style,
%% or BibLaTeX with the acmnumeric or acmauthoryear sytles, that include
%% support for advanced citation of software artefact from the
%% biblatex-software package, also separately available on CTAN.
%%
%% Look at the sample-*-biblatex.tex files for templates showcasing
%% the biblatex styles.
%%

%%
%% The majority of ACM publications use numbered citations and
%% references.  The command \citestyle{authoryear} switches to the
%% "author year" style.
%%
%% If you are preparing content for an event
%% sponsored by ACM SIGGRAPH, you must use the "author year" style of
%% citations and references.
%% Uncommenting
%% the next command will enable that style.
%%\citestyle{acmauthoryear}

%%
%% end of the preamble, start of the body of the document source.
\begin{document}

%%
%% The "title" command has an optional parameter,
%% allowing the author to define a "short title" to be used in page headers.
\title{Contrastive Self-supervised Learning in Recommender Systems: A Survey}

%%
%% The "author" command and its associated commands are used to define
%% the authors and their affiliations.
%% Of note is the shared affiliation of the first two authors, and the
%% "authornote" and "authornotemark" commands
%% used to denote shared contribution to the research.
\author{Mengyuan Jing}
\email{jingmy@sjtu.edu.cn}
\affiliation{%
  \institution{Shanghai Jiao Tong University}
  \streetaddress{No.800 Dongchuan Road, Minhang District}
  \city{Shanghai}
  \state{Shanghai}
  \country{China}
  \postcode{200240}
}
\author{Yanmin Zhu}\authornote{corresponding author}
\email{yzhu@sjtu.edu.cn}
\affiliation{%
  \institution{Shanghai Jiao Tong University}
  \streetaddress{No.800 Dongchuan Road, Minhang District}
  \city{Shanghai}
  \state{Shanghai}
  \country{China}
  \postcode{200240}
}
\author{Tianzi Zang}
\email{zangtianzi@sjtu.edu.cn}
\affiliation{%
  \institution{Shanghai Jiao Tong University}
  \streetaddress{No.800 Dongchuan Road, Minhang District}
  \city{Shanghai}
  \state{Shanghai}
  \country{China}
  \postcode{200240}
}
\author{Ke Wang}
\email{onecall@sjtu.edu.cn}
\affiliation{%
  \institution{Shanghai Jiao Tong University}
  \streetaddress{No.800 Dongchuan Road, Minhang District}
  \city{Shanghai}
  \state{Shanghai}
  \country{China}
  \postcode{200240}
}




%%
%% By default, the full list of authors will be used in the page
%% headers. Often, this list is too long, and will overlap
%% other information printed in the page headers. This command allows
%% the author to define a more concise list
%% of authors' names for this purpose.
\renewcommand{\shortauthors}{Mengyuan Jing, et al.}

%%
%% The abstract is a short summary of the work to be presented in the
%% article.
\begin{abstract}
 Deep learning-based recommender systems have achieved remarkable success in recent years. However, these methods usually heavily rely on labeled data (i.e., user-item interactions), suffering from problems such as data sparsity and cold-start. Self-supervised learning, an emerging paradigm that extracts information from unlabeled data, provides insights into addressing these problems. Specifically, contrastive self-supervised learning, due to its flexibility and promising performance, has attracted considerable interest and recently become a dominant branch in self-supervised learning-based recommendation methods. In this survey, we provide an up-to-date and comprehensive review of current contrastive self-supervised learning-based recommendation methods. Firstly, we propose a unified framework for these methods. We then introduce a taxonomy based on the key components of the framework, including view generation strategy, contrastive task, and contrastive objective. For each component, we provide detailed descriptions and discussions to guide the choice of the appropriate method. Finally, we outline open issues and promising directions for future research.
 % Using contrastive self-supervised learning in recommender systems has attracted increasing interests in recent years. 
\end{abstract}


%%
%% The code below is generated by the tool at http://dl.acm.org/ccs.cfm.
%% Please copy and paste the code instead of the example below.
%%
\begin{CCSXML}
<ccs2012>
<concept>
<concept_id>10002951.10003317.10003347.10003350</concept_id>
<concept_desc>Information systems~Recommender systems</concept_desc>
<concept_significance>500</concept_significance>
</concept>
</ccs2012>
\end{CCSXML}

\ccsdesc[500]{Information systems~Recommender systems}
%%
%% Keywords. The author(s) should pick words that accurately describe
%% the work being presented. Separate the keywords with commas.
\keywords{contrastive learning, self-supervised learning, unsupervised learning, survey, deep learning}


% \received{20 February 2007}
% \received[revised]{12 March 2009}
% \received[accepted]{5 June 2009}

%%
%% This command processes the author and affiliation and title
%% information and builds the first part of the formatted document.
\maketitle

\section{Introduction}

% \textbf{Background and Motivation of SSL.}
Recommender systems, as the most effective way to alleviate information overloading, have been an indispensable tool in daily life~\cite{10.1145/2959100.2959190, zhang2019deep}.
They are intensively employed in a broad range of online services such as e-commerce platforms, social media, and music platforms. Owing to the ability to effectively capture the user-item relationships, deep learning techniques have been widely used in recommender systems~\cite{cheng2016wide, he2017neural}. Despite their effectiveness, most deep learning-based methods focus on supervised learning settings. The recommendation model is trained with abundant labeled data (i.e., user-item interactions). However, user-item interaction records are very sparse compared to the interaction space~\cite{bayer2017generic, he2016ups}. 
Hence, these methods usually suffer from the problem of data sparsity~\cite{wuSelfsupervisedGraphLearning2021}. Meanwhile, these methods are prone to the problem of over-fitting and generalization error~\cite{liu2021self}.

Self-supervised learning (SSL)~\cite{liu2021self}, as a novel learning paradigm, provides new insights to overcome aforementioned problems. The basic idea of SSL is to acquire transferable knowledge from the data itself without the need for manual annotated labels. This is achieved by solving auxiliary tasks (named pretext tasks). The acquired knowledge is then used in downstream tasks. Due to its efficiency, SSL has been widely used in many fields such as computer vision (CV)~\cite{chen2020simclr, hjelm2018learning, oord2018representation}, natural language processing (NLP)~\cite{gao2021simcse, logeswaran2018efficient} and graph learning~\cite{wu2022discovering, DGI}. Inspired by the success of SSL in other fields, there is growing interest in applying SSL to the area of recommendation.

Particularly, Contrastive Self-supervised Learning, also known as Contrastive Learning (CL), has attracted considerable attention. 
% CL aims to learn representations by contrasting positive and negative data pairs.
CL aims to maximize the agreement between positive data pairs and minimize the agreement of negative pairs. 
Due to its lightweight and promising performance, CL has been applied to various recommendation tasks, like sequential recommendation~\cite{CL4SRec, zhouS3RecSelfSupervisedLearning2020,DuoRec,EC4SRec}, general collaborative filtering~\cite{wuSelfsupervisedGraphLearning2021, SimGCL, XSimGCL, RocSE,linImprovingGraphCollaborative2022a}, etc. Moreover, it has recently become the main branch of SSL-based recommendation~\cite{SSLSurvey}. Specifically, the number of publications related to CL-based recommendation exceeds 50\% of the number of publications related to SSL-based recommendation in the ACM Digital Library\footnote{https://dl.acm.org/}. Considering the increasing trend of CL-based recommendation methods, we provide a timely and comprehensive review to summarize these methods in this paper.

Although there are several reviews\cite{jaiswal2020survey, khan2022contrastive} on contrastive learning, they mainly focus on methods in CV and NLP without reviewing CL-based recommendation methods. However, due to the uniqueness of the recommendation, it is difficult to apply existing CL-based methods from other fields to recommendation. Specifically, in CV/NLP, the models usually deal with dense input data and treat each data instance as isolated. However, in recommendation, the input data are extremely sparse (e.g., one-hot ID and categorical features of users/items) and there is homophily between users or items. Furthermore, there are various recommendation tasks that are unique to recommender systems, such as bundle recommendation, and multi-behavior recommendation. Therefore, a comprehensive survey is necessary to review CL-based recommendation methods, considering the differences between the recommendation and other fields.
% Furthermore, in addition to accuracy, there are several objectives unique to recommender systems such as the diversity and fairness of recommendations. 

In the field of recommendation, the most relevant survey is \cite{SSLSurvey}. This survey reviews SSL-based recommendation methods which contain CL-based methods. In contrast to \cite{SSLSurvey}, our survey has the following differences. Firstly, we present a more comprehensive taxonomy. For instance, in addition to the data-based augmentation methods introduced by \cite{SSLSurvey}, we also introduce model-based augmentation methods and methods without augmentation when presenting view generation strategies. Secondly, we discuss the pros and cons of different options for key components of CL-based recommendation methods, which can provide guidance for the selection of these components. This critical discussion is not present in \cite{SSLSurvey}. Finally, because of the increasing popularity of CL-based recommendation methods, we provide a more timely review that summarizes recently published studies that were not included in \cite{SSLSurvey}.

To sum up, the key contributions of this paper are summarized as follows:

\begin{itemize}
  \item We propose a general framework to unify the CL-based methods for recommendation. Based on the framework, we review existing research according to three key components: view generation, pretext task, and contrastive objective. 
  \item We provide an up-to-date and comprehensive review of CL-based recommendation methods. We provide detailed descriptions and discussions for each key component to guide the choice of the appropriate method. We also introduce the relevant background knowledge to help readers easily understand CL-based recommendation. 
  \item We identify the limitations of existing research and propose promising future directions for CL-based recommendation to inspire new research.
\end{itemize}

\textbf{Paper Collection.}
We first adopt Google Scholar as the main search engine to collect related papers.
Then, we search for related work from top-tier conferences and journals, such as SIGIR, KDD, WWW, AAAI, IJCAI, WSDM, CIKM, NuerIPS, ICML, TKDE, TOIS, etc. Specifically, we search with keywords including  "self-supervised", "contrastive" in combination with  "recommend", "collaborative filtering".
To prevent omissions of relevant work, we further look through the references of each paper.

\textbf{Survey Organization.}
The remainder of the survey is organized as follows. In Section~\ref{sec:background}, we introduce background knowledge.
We then introduce the unified framework and taxonomy in Section~\ref{sec:taxonomy}.
% introduces the preliminaries of RS and SSL. 
Section~\ref{sec:view_gen}, Section~\ref{sec:pretext_task} and Section~\ref{sec:obj} are the main contents, which review contrastive learning in recommender systems. 
In Section~\ref{sec:future}, we discuss  the open problems and future directions. Finally, we conclude the survey in Section~\ref{sec:conclusion}.


\section{Background}\label{sec:background}
In this section, we introduce essential background knowledge about CL-based recommendation.
First, we provide the definitions of relevant concepts.
Then, we give a brief introduction to contrastive learning.
At last, we introduce training strategies used in CL-based recommendation methods.
In addition, we summarize the notations used in this survey in Table.~\ref{tab:notations}.
\begin{table*}[t]
  \caption{Key notations.}
  \label{tab:notations}
  \begin{tabular}{c|c}
    \toprule
    Notations&Discriptions\\
    \midrule
    $\mathcal{U}$& The set of users\\
    $\mathcal{I}$& The set of items\\
    $\mathcal{B}$& The mini-batch\\
    % $Y$& The user-item interaction matrix\\
    $\mathbf{h}, \mathbf{c}, \mathbf{g}, \mathbf{z}$ & The learned representations\\
    % $\mathbf{h}_i$& Representation of item $i$\\
    % $\hat{y}_{u,i}$& The preference score for user $u$ on item $i$\\
    % $s$& The session data of session $s$\\
    $f_\theta$& The encoder to learn representations\\
    $p_\omega$& The pretext decoder\\
    $q_\phi$& The downstream decoder\\
    $\theta$, $\omega$, $\phi$, $\xi$, $\psi$& Learnable parameters\\
    $\lambda, \rho, \epsilon$& The hyperparameter\\
    $\mathcal{T}$& View generation strategy\\
    $\mathbf{L}$& The location matrix\\
    $\mathbf{A}$ & The adjacent matrix of graph\\
    $\mathbf{X}$ & The feature matrix\\
    $\mathbf{H}$ & The representation matrix\\
    $\mathcal{G}$ & The graph\\
    $\mathcal{V}$/$\mathcal{E}$ & The set of the graph nodes/edges\\
    $s_u$& The interaction sequence of user $u$\\
    $\mathcal{M I}$& Mutual information function\\
    $||$& Concatenation operation\\
    $\circ$& The Hadamard product\\
    $|\cdot|$&The length of a set\\
  \bottomrule
\end{tabular}
\end{table*}

\subsection{Term Definitions}
\subsubsection{Supervised Learning, Unsupervised Learning, and Self-supervised Learning}
Supervised learning refers to a learning paradigm that trains models with manual annotated labels.
In contrast, unsupervised learning indicates the learning paradigm that trains models without using manual annotated labels.
Self-supervised learning can be viewed as a subset of unsupervised learning as it requires no manual annotated labels.
However, unlike other unsupervised learning methods (e.g., clustering) that concentrate on mining data patterns, self-supervised learning aims to generate supervision signals from data itself, and models are still trained in supervised settings.

\subsubsection{Pretext Tasks Versus Downstream Tasks}
Pretext tasks are pre-designed tasks to be solved by models (e.g., node self-discrimination~\cite{wuSelfsupervisedGraphLearning2021}). By learning the objective functions of the pretext tasks, models learn more generalized representations from unlabeled data, thus benefiting downstream tasks.
Downstream tasks refer to tasks used to evaluate the quality of representations learned by models.
Specifically, in recommender systems, downstream tasks are the recommendation tasks such as sequential recommendation and social recommendation.
In general, solving downstream tasks requires manual annotated labels.
\begin{figure}
\centering\includegraphics[width=0.9\linewidth]{figs/contrastive_learning2.pdf}
    \caption{Pipeline of contrastive learning.}
    \label{fig:cl}
\end{figure}

\subsection{Contrastive Learning}
The core idea of contrasting learning (CL) is to maximize agreement between different views, where the agreement is usually measured by Mutual Information (MI). 
The general pipeline of CL is shown in Fig.\ref{fig:cl}. 
In specific, two different views are generated using view generation strategies. Then, representations in different views are generated by an encoder, which is usually shared by the two views. Finally, the model is optimized by contrastive loss to maximize the agreement between positive pairs and minimize the agreement between negative pairs. 
In general, positive pairs are the same instance from different views, while negative pairs are different instances from different views.
Formally, contrastive self-supervised learning can be formulated as:
\begin{equation} \label{eq:contrastive}
  \theta^{*}, \omega^{*}=\underset{\theta, \omega}{\arg \min } \mathcal{L}_{con}\left(p_\omega\left(f_\theta\left(\tilde{\mathcal{D}}^{(1)}\right), f_\theta\left(\tilde{\mathcal{D}}^{(2)}\right)\right)\right)
\end{equation}
where $\tilde{\mathcal{D}}^{(1)}$ and $\tilde{\mathcal{D}}^{(2)}$ are two generated data views. $f_\theta(\cdot)$ is the (shared) encoder to learn representations of instances in different views. $p_\omega(\cdot)$ is the pretext decoder that estimates the agreement between two instances. 
$\mathcal{L}_{con}$ denotes the contrastive loss.

\begin{figure}[t]
    \centering
    \subfigure[Joint Learning.]{ 
        \includegraphics[width=0.75\linewidth]{figs/jl2.pdf}\label{fig:jt}}
    \subfigure[Pre-training and Fine-tuning.]{        
        \includegraphics[width=0.75\linewidth]{figs/p_and_f2.pdf}\label{fig:pf}}
    \caption{Two types of training strategies for CL-based recommendation.} 
    \label{fig:training_strategies}
    %\vspace{-0.1in}
\end{figure}

\subsection{Training Strategy}
Currently, CL-based recommendation methods employ two typical training strategies: Pre-training and Fine-tuning, and Joint Learning. The detailed workflow of them is shown in Fig.~\ref{fig:training_strategies}. 

\subsubsection{Pre-training and Fine-tuning (P\&F)}
In this strategy, the model is trained in two stages. In the pre-training stage, the encoder $f_\theta(\cdot)$ is first pre-trained with contrastive tasks. In addition, the pre-trained parameter $\theta_{init}$ is then used as the initialization parameter for the encoder $f_{\theta_{init}}(\cdot)$. 
In the fine-tuning stage, the pre-trained encoder $f_{\theta_{init}}(\cdot)$ is fine-tuned with the downstream decoder $q_\phi(\cdot)$ supervised by the recommendation task.
The formulation of this strategy can be defined as: 
\begin{equation}
\begin{aligned}
    \theta_{init}, \omega^* &= \arg \min _{\theta, \omega} \mathcal{L}_{con}\left(f_{\theta}, p_\omega\right)\\
    \theta^*, \phi^* &=\arg \min _{\theta_{init}, \phi} \mathcal{L}_{rec}\left(f_{\theta_{init}}, q_\phi\right)
\end{aligned}
\end{equation}
where $\mathcal{L}_{con}$ is the contrastive loss and $\mathcal{L}_{r e c}$ is the recommendation loss. $q_\phi$ is the downstream decoder.

\subsubsection{Joint Learning (JL)}
In this strategy, the encoder $f_\theta(\cdot)$ is jointly trained with the pretext tasks and downstream tasks (i.e., recommendation tasks). Moreover, the encoder is usually shared by pretext and recommendation tasks. This strategy can be considered a type of multi-tasking learning strategy, in which the contrastive pretext task is the auxiliary task to regularize the recommendation task.
The loss function consists of both contrastive loss and recommendation loss. The learning objective can be formalized as:
\begin{equation}
  \theta^{*}, \omega^{*}, \phi^{*} =\arg \min _{\theta, \omega, \phi} \left[\mathcal{L}_{r e c}\left(f_{\theta}, q_\phi\right)+\lambda \mathcal{L}_{con}\left(f_{\theta}, p_{\omega}\right)\right]
\end{equation}
where $\lambda$ is a trade-off hyperparameter that controls the contribution of $\mathcal{L}_{con}$. 

% todo:加过渡
% with initialization $\theta_{init}=\arg \min _\theta \mathcal{L}_{ssl}\left(f_\theta\right)$, where $\mathcal{L}_{task}$ and $\mathcal{L}_{ssl}$ is the loss function of downstream tasks and self-supervised pretext tasks, respectively.
% todo:确定位置
% Note that other reviews introduce another training strategy, called unsupervised representation learning, which has a first stage similar to pre-training and fine-tuning, but it freezes $\theta_{init}$ and only trains $q_\phi$ in the second stage.
% Since it is not used in current CL-based recommendation methods, we will not introduce it in detail in this paper.

% \subsubsection{Unsupervised Representation Learning}
% This strategy can also be considered as a two-stage paradigm, with the first stage similar to Pre-training. However, at the second stage, the pre-trained parameters $\theta_{init}$ are frozen and the model is trained on the frozen representations with downstream tasks only. The learning objective is formulated as
% \begin{equation}
%   \omega^*=\arg \min _\omega \mathcal{L}_{task}\left(f_{\theta_{init}}, g_\omega\right)
% \end{equation}
% with initialization $\theta_{\text {init }}=\arg \min _\theta \mathcal{L}_{s s l}\left(f_\theta\right)$. Compared to other schemes, unsupervised representation learning is more challenging since the learning of the encoder $f_\theta(\cdot)$ depends only on the pretext task and is frozen in the second stage. In contrast, in the P\&F strategy, the encoder $f_\theta(\cdot)$ can be further optimized under the supervision of the downstream task during the fine-tuning stage.


% \multirow{2}{*}{SG}&  	\multirow{2}{*}{SIGIR}&	\multirow{2}{*}{2021}	&Node Dropout/Edge Dropout&	\multirow{2}{*}{InfoNCE}&	\multirow{2}{*}{L-L}&	\multirow{2}{*}{JL}&\multirow{2}{*}{General}\\ 
%       & &		&/Random Walk&	&	&	&	\\

\section{Taxonomy}\label{sec:taxonomy}
% We introduce a new breakdown along three independent
% axes. For each axis we provide a taxonomy that reflects the
% current meta-learning landscape.
% Meta-Representation (“What?”) The first axis is the
% choice of meta-knowledge ω to meta-learn. This could be
% anything from initial model parameters [16] to readable
% code in the case of program induction [93].
% Meta-Optimizer (“How?”) The second axis is the choice
% of optimizer to use for the outer level during meta-training
% (see Eq. 5). The outer-level optimizer for ω can take a variety of forms from gradient-descent [16], to reinforcement
% learning [93] and evolutionary search [23].
% Meta-Objective (“Why?”) The third axis is the goal of
% meta-learning which is determined by choice of metaobjective L
% meta (Eq. 5), task distribution p(T ), and dataflow between the two levels. Together these can customize
% meta-learning for different purposes such as sample efficient
% few-shot learning [16], [38], fast many-shot optimization
% [93], [94], robustness to domain-shift [42], [95], label noise
% [96], and adversarial attack [97].
% Together these axes provide a design-space for metalearning methods that can orient the development of new
% algorithms and customization for particular applications.
% Note that the base model representation is not included
% in this taxonomy, since it is determined and optimized in a
% way that is specific to the application at hand.

% As introduced in Section.~\ref{sec:background}, the general framework of CL-based methods is first to perform view generation to obtain two views and then maximize the mutual information (MI) of positive pairs in these views by conducting the contrastive pretext task. 

% Therefore, we review contrastive methods from three perspectives: (1) view generations which generate different views; (2) pretext tasks which are conducted to obtain supervision signals; (3) \textbf{mutual information estimation which measures the MI between instances and forms the contrastive learning objective together.}

In this section, we first propose a unified framework of CL-based recommendation methods. Then we introduce our proposed taxonomy with three perspectives.
% as shown in Fig.~\ref{fig:taxonomy}.

\subsection{Unified Framework}\label{sec:unified_framework}
As introduced in Section~\ref{sec:background}, the general framework of CL-based methods is first to perform view generation strategies to obtain multiple views and then maximize the agreement of positive pairs in these views by conducting the contrastive pretext task. Specifically, given the data $\mathcal{D}$, $K$ data views $\{\tilde{\mathcal{D}}^{(k)}\}_{k=1}^{K}$ are obtained through $K$ view generation strategies $\{\mathcal{T}_k(\cdot)\}_{k=1}^{K}$, which can be formulated as: 
\begin{equation}
    \tilde{\mathcal{D}}^{(k)} = \mathcal{T}_k(\mathcal{D}), k = 1,\cdots,K
\end{equation}
Then, encoders $\{f_{\theta_k}(\cdot)\}_{k=1}^K$ are applied to generate representations $\{\mathbf{h}_k\}_{k=1}^K$ for each data view. Formally, we have 
\begin{equation}
    \mathbf{h}_k = f_{\theta_k} (\tilde{\mathcal{D}}^{(k)}), k = 1,\cdots,K
\end{equation}
In addition, $\{\mathbf{h}_k\}_{k=1}^K$ may have different scales depending on the type of pretext tasks. For example, it can be a representation of an item or a representation of a sequence that consists of multiple items. 

During training, contrastive learning is to maximize the agreement between representations of positive pairs $(\mathbf{h}_i, \mathbf{h}_j)$ in two views. Moreover, the mutual information $\mathcal{M I}\left(\mathbf{h}_i, \mathbf{h}_j\right)$ is usually applied to measure the agreement. The contrastive objective can be defined as:
% Moreover, projection heads $\{p_{\omega}\}_{k=1}^K$ are optionally applied before calculating the loss and optimized simultaneously with encoders, defined as
% \begin{equation}
%     \mathbf{z}_k = p_{\omega_k}(\mathbf{h}_k), k=1,\cdots,K
% \end{equation}
% \begin{equation}
% \max _{\left\{f_i, p_i\right\}_{i=1}^K} \sum_i \left[\sum_{i \neq j} \lambda{i j} \mathcal{M}\mathcal{I}_{p_i, p_j}\left(\mathbf{h}_i, \mathbf{h}_j\right)\right]
% \end{equation}
\begin{equation}\label{eq:objective}
\max _{\left\{\theta\right\}_{i=1}^K} \sum_i \sum_{i \neq j} \lambda_{i j} \mathcal{M I}\left(\mathbf{h}_i, \mathbf{h}_j\right)
\end{equation}
where $\lambda \in \{0, 1\}$, if the mutual information between $\mathbf{h}_i$ and $\mathbf{h}_j$ is calculated then $\lambda=1$, otherwise $\lambda=0$.

Since it is difficult to directly calculate mutual information, mutual information estimators 
% $\widehat{\mathcal{M}\mathcal{I}}$ 
are usually used instead. The estimation is calculated based on the discriminator $p_\omega(\cdot)$ (i.e., the pretext decoder). Moreover, projection heads~\cite{CLUE} can be optionally applied to $\{\mathbf{h}_k\}_{k=1}^K$, defined as: 
% and optimized simultaneously with encoders, defined as
\begin{equation}
    \mathbf{z}_k = g_{\xi_k}(\mathbf{h}_k), k=1,\cdots,K
\end{equation}
where $g_{\xi_i}(\cdot)$ is a projection head, which can be the Multi-Layer Perceptron (MLP) or linear projection. For the sake of convenience, we treat the projection head as part of the pretext decoder $p_\omega(\cdot)$. 
% Encoders and pretext decoders (with optional projection heads) are optimized by minimizing Eq.(\ref{eq:contrastive}) and obtain optimal $f_{\theta^*}(\cdot)$ and $p_{\omega^*}(\cdot)$.
Then $f_{\theta^*}(\cdot)$ and $p_{\omega^*}(\cdot)$ can be obtained by learning Eq.(\ref{eq:objective}).
Furthermore, by utilizing $f_{\theta^*}(\cdot)$, the generated representations can be used for recommendation tasks. The recommendation task can be formulated as: 
\begin{equation}
    \theta^{* *}, \phi^*=\underset{\theta^*, \phi}{\arg \min } \mathcal{L}_{rec}\left(q_\phi(f_{\theta^*}(\mathcal{D})), y\right)
\end{equation}
where $y$ denotes the labels. $\mathcal{L}_{{rec}}$ is the supervised loss for recommendation tasks such as the cross-entropy (CE) loss.

% \begin{equation}
% \max _{\left\{f_i, p_i\right\}_{i=1}^K} \sum_i \left[\sum_{i \neq j} \lambda{i j} \widehat{\mathcal{M}\mathcal{I}}_{p_i, p_j}\left(\mathbf{h}_i, \mathbf{h}_j\right)\right]
% \end{equation}
% \max _{\left\{\theta_i, \omega_i\right\}_{i=1}^K} \sum_i \left[\sum_{i \neq j} \alpha_{i j} \mathcal{M}\mathcal{I}_{p_{\omega_i}, p_{\omega_j}}\left(\mathbf{h}_i, \bmathbf{h}_j\right)\right]
% The overall process of CL-based recommendation can be formulated as:
% \begin{equation}
% \begin{aligned}
%     \theta^*, \omega^*&=\underset{\theta, \omega}{\arg \min } \mathcal{L}_{con}\left(f_\theta, p_\omega, \mathcal{D}\right)\\
%     \theta^{* *}, \phi^*&=\underset{\theta^*, \phi}{\arg \min } \mathcal{L}_{\text {rec}}\left(f_{\theta^*}, q_\phi, \mathcal{D}, y\right),
% \end{aligned}
% \end{equation}
% In addition, $\{\mathbf{h}_k\}_{k=1}^K$ may have different scale depending on the type of pretext tasks. For example, it can be a representation of an item or a representation of a sequence which consists of multiple items. 
% Pretext decoder $p_{\omega}$ can be a linear projection, an Multilayer Perceptron (MLP) or identical mapping.
\subsection{Proposed Taxonomy}
The differences among contrastive learning methods lie in three key components: view generation strategies, pretext tasks, and contrastive objectives. A CL-based recommendation method can be determined by specifying these components. Note that the encoder $f_\theta(\cdot)$ is not included in our taxonomy, as it is not the focus of CL-based recommendation and is determined by the specific recommendation tasks. Therefore, we propose a taxonomy based on these components as shown in Fig.~\ref{fig:taxonomy}. Table.\ref{tab:overview} shows representative researches of CL-based recommendation.

% todo: 没写完,说法需要再改一下
\textbf{View Generation} is the design of how to generate data views. 
Depending on whether the augmentation is needed, we classify the view generation strategies into view generation with augmentation and without augmentation.

% todo: view scale的说明
\textbf{Pretext Task} is the design of how to obtain supervision signals.
Depending on the scale of the instances being contrasted, we classify the pretext tasks into same-scale contrasting and cross-scale contrasting.

\textbf{Contrastive Objective} is the design of how to measure mutual information.
Depending on whether an estimation of lower-bound of mutual information is provided, we classify the contrastive objectives into bound objective and non-bound objective.

% Note that the choice of these components generally depends on the characteristics of input data and downstream tasks. For example, sequence-based augmentations are not suitable to graph-based recommendation. As sequential recommendation aims to learn good sequence representations, the corresponding pretext tasks usually contrast sequence representations. 
% In general, pretext tasks are firstly designed according to the specific recommendation tasks. Then, view generation strategies and contrastive objectives are selected according to the pretext tasks.
% Moreover, the selection of components are not completely independent. 
% Although the same pretext task can utilize view generation methods,  
% some may not very effective. 
% For example, if the pretext task aims to model the item dependency, feature-based augmentation may not effective. 
% However, how to choose the optimal method is still unclear.
% Therefore, currently, the choice of these components are based on their impact on the performance of downstream tasks.
% In general, the choice of these components are based on their impact on the performance of downstream tasks and not solely on its compatibility with each other.
% Note that the selection of these components generally depends on the characteristics of the input data and the downstream tasks. For example, sequence-based augmentation methods are not applicable to graph data. Since the goal of sequence recommendation is to learn good sequence representations, the corresponding pretext tasks usually contrast sequence representations. Typically, CL-based recommendation methods first design the pretext task based on the specific recommendation task, and then select view generation methods and contrastive objectives. It is worth noting that the selection of components is not entirely independent. 
% While the same pretext task can be conducted with different view generation methods or contrastive objectives, some of them may not be very effective. For example, if the pretext task aims to model the sequential relationships of items, feature-based augmentation methods may not be effective. However, it is not clear how to choose the optimal components, and therefore, the selection of components usually relies on empirical evaluation.
Note that the selection of these components depends on the characteristics of the input data and downstream tasks. For instance, sequence-based augmentation methods may not be suitable for graph data. For sequential recommendation, the pretext tasks generally contrast sequence representations as the primary objective is to learn good sequence representations. It is worth noting that the selection of components is not entirely independent. Although the same pretext task can be performed with different view generation strategies or contrastive objectives, some may not be effective. For instance, feature-based augmentation methods may not be effective when the pretext task aims to model the sequential relationships of items. Therefore, when design a CL-based recommendation methods, we can first design the pretext task based on the specific recommendation task and then select view generation strategies and contrastive objectives accordingly. 
% However, it is currently unclear how to choose the optimal components, and thus, the selection of components typically relies on empirical evaluation.

% Therefore, we review contrastive methods from three perspectives: (1) view generations which generate different views; (2) pretext tasks which are conducted to obtain supervision signals; (3) \textbf{mutual information estimation which measures the MI between instances and forms the contrastive learning objective together.}
\begin{figure}
    \centering
    \includegraphics[width=0.9\linewidth]{figs/taxonomy3.pdf}
    \caption{Taxonomy of contrastive learning-based recommendation.}
    \label{fig:taxonomy}
\end{figure}
\begin{table*}[htbp]
  \caption{A summary of CL-based recommendation methods. "L-L", "C-C", and "G-G" mean Local-Local, Contextual-Contextual, and Global-Global, respectively. "CF" means Collaborative Filtering. "KG" means Knowledge Graph.}
  \label{tab:overview}
  \resizebox{.93\textwidth}{!}{
  \begin{tabular}{c|c|c|c|c|c|c|c}
    \toprule
    Model&Venue&Year&View Generation&Pretext Task&Objective& Training Strategy&Recommendation Task\\
    \midrule
    \multirow{2}{*}{SGL~\cite{wuSelfsupervisedGraphLearning2021}}&  	\multirow{2}{*}{SIGIR}&	\multirow{2}{*}{2021}	&Node/Edge Dropout&		\multirow{2}{*}{L-L}&\multirow{2}{*}{InfoNCE}&	\multirow{2}{*}{JL}&\multirow{2}{*}{Graph-based CF}\\ 
      	& &		&/Random Walk&	&	&	&	\\
    %   	\hline
    % SGL~\cite{wuSelfsupervisedGraphLearning2021}&	SIGIR&	2021	&Node Dropout/Edge Dropout/Random Walk&	InfoNCE&	L-L&	JL&	General\\
    DCL~\cite{DCL}&	Arxiv&	2021&	Edge Dropout&		L-L&InfoNCE&	JL&	Graph-based CF\\
    GDCL~\cite{GDCL}&	DASFAA&	2022&	Graph Diffusion&		L-L&InfoNCE&	JL&	Graph-based CF\\
    SimGCL~\cite{SimGCL}&	SIGIR&	2022&	Embedding Noise&	L-L&InfoNCE	&	JL&	Graph-based CF\\
    XSimGCL~\cite{XSimGCL}&	Arxiv&	2022&	Embedding Noise&	L-L&InfoNCE	&	JL&	Graph-based CF\\
    RocSE~\cite{RocSE}&	TOIS&	2023&	Embedding Noise&		L-L&InfoNCE&	JL&	Graph-based CF\\
    LightGCL~\cite{LightGCL}&	ICLR&	2023&	SVD-based Augmentation	&	L-L&InfoNCE&	JL&	Graph-based CF\\
    RGCL~\cite{RGCL}&	SIGIR&	2022&	Node Dropout&		L-L&InfoNCE&	JL&	Review-based\\
    MCLSR~\cite{MCLSR}&	CIKM&	2022&	Without Augmentation&		L-L&InfoNCE	&JL	&Sequential\\
    HCCF~\cite{HCCF}&	SIGIR&	2022&	Edge Dropout&		L-L&InfoNCE	&JL	&Graph-based CF\\
    SGCCL~\cite{li2023sgccl}&	WSDM&	2023&	Edge/Feature Dropout&	L-L&	InfoNCE&	JL&	Graph-based CF\\
    MPT~\cite{MPT}&	TOIS&	2023&	Node Masking/Substituting/Deleting		&L-L&InfoNCE	&P\&F	&Cold-start\\
    MCCLK~\cite{MCCLK}&	SIGIR&	2022&	Without Augmentation&	L-L&	InfoNCE	&JL	&KG-based\\
    KACL~\cite{KACL}&	WSDM&	2023&	Edge Dropout&	L-L&	InfoNCE	&JL&	KG-based\\
    KGCL~\cite{KGCL}&	SIGIR&	2022&	Edge Dropout&	L-L&	InfoNCE	&JL	&KG-based\\
    S-MBRec~\cite{S-MBRec}&	IJCAI&	2022&	Without Augmentation	&	L-L&InfoNCE	&JL&	Multi-behavior\\
    MMCLR~\cite{MMCLR}&	DASFAA&	2022	&Without Augmentation&		L-L	&BPR&JL	&Multi-behavior\\
    KMCLR~\cite{KMCLR}&	WSDM&	2023&	Edge Dropout&	L-L&InfoNCE	&	JL&	Multi-behavior\\
    SEPT~\cite{SEPT}&	KDD&	2022&	Edge Dropout/Predicted amples&		L-L&InfoNCE&	JL&	Social\\
    HGCL\_S~\cite{HGCL_s}&	WSDM&	2023&	Without Augmentation&		L-L	&InfoNCE&JL	&Social\\
    CCDR~\cite{CCDR}&	KDD	&2022&	Subgraph Sampling	&L-L&	InfoNCE&	JL&	Cross-domain\\
    ML-SAT~\cite{ML-SAT}&	CIKM&	2022&	Without Augmentation&	L-L&	InfoNCE	&P\&F&	Cross-domain\\
    DR-MTCDR~\cite{DR_MTCDR}&	TOIS&	2022&	Edge/Node Dropout	&L-L&	InfoNCE&	JL&	Cross-domain\\
    COTREC~\cite{COTREC}&	CIKM&	2022&	Predicted Samples&	L-L&	InfoNCE	&JL&	Session-based\\
    S$^2$-HHGR~\cite{zhangDoubleScaleSelfSupervisedHypergraph2021}&	CIKM&	2021&	Node Dropout&	L-L&	JS&	JL&	Group\\
    SGGCF~\cite{SGGCF}&	WSDM&	2023&	Edge/Node Dropout&	L-L&	InfoNCE	&JL&	Group\\
    CrossCBR~\cite{CrossCBR}&	KDD	&2022&	Edge/Message Dropout &	L-L&	InfoNCE	&JL&	Bundle\\
    HMG-CR~\cite{yangHyperMetaPathContrastive2021}&	Arxiv&	2021&	Without Augmentation&	C-C	&	InfoNCE&JL&	Multi-behavior\\
    % CHEST~\cite{CHEST}	&TOIS	&2023&	Subgraph Sampling/Path removal/insertion/substitution (edge perturbation)&	InfoNCE&	C-C	&P\&F&	Graph-based CF\\
    \multirow{2}{*}{CHEST~\cite{CHEST}}&  	\multirow{2}{*}{TOIS}&	\multirow{2}{*}{2023}	&Subgraph Sampling&		\multirow{2}{*}{C-C}&\multirow{2}{*}{InfoNCE}&	\multirow{2}{*}{P\&F}&\multirow{2}{*}{HIN-based}\\ 
      	& &		&/Path Dropout/Inserting/Substituting&	&	&	&	\\
    % CHEST~\cite{CHEST}	&TOIS	&2023&	Subgraph Sampling/Edge perturbation&	C-C	&	InfoNCE&P\&F&	Graph-based CF\\
    KGIC~\cite{KGIC}&	CIKM&	2022&	Subgraph Sampling&	C-C&	InfoNCE	&JL	&KG-based\\
    GCL4SR~\cite{GCL4SR}&	IJCAI&	2022&	Subgraph Sampling&C-C&	InfoNCE	&	JL&	Sequential\\
    MISS~\cite{MISS}&	ICDE&	2022&	Feature Extractor	&C-C&	InfoNCE&	JL&	Sequential\\
%     \multirow{2}{*}{GCL4SR~\cite{GCL4SR}}&  	\multirow{2}{*}{ICDE}&	\multirow{2}{*}{2022}	&Item Shuffling/Cropping &	\multirow{2}{*}{InfoNCE}&	\multirow{2}{*}{G-G}&	\multirow{2}{*}{JL}&\multirow{2}{*}{Sequential}\\ 
%   & &		&/Masking&	&	&	&	\\
    CL4SRec~\cite{CL4SRec}&	ICDE&	2022&	Item Masking/Shuffling/Cropping	&G-G&	InfoNCE&	JL&	Sequential\\
    H$^2$SeqRec~\cite{H2SeqRec}	&CIKM&	2021&	Item Masking/Cropping	&	G-G&InfoNCE&	P\&F&	Sequential\\
    \multirow{2}{*}{CoSeRec~\cite{CoSeRec}}&  	\multirow{2}{*}{Arxiv}&	\multirow{2}{*}{2021}	&Item Shuffling/Cropping &		\multirow{2}{*}{G-G}&\multirow{2}{*}{InfoNCE}&	\multirow{2}{*}{JL}&\multirow{2}{*}{Sequential}\\ 
  & &		&/Masking/Substituting/Inserting&	&	&	&	\\
    % CoSeRec~\cite{CoSeRec}&	Arxiv&	2021&	Item Cropping/Masking/Shuffling/Substituting/Inserting&	InfoNCE&	G-G	&JL&	Sequential\\
    ContraRec~\cite{ContraRec}&	TOIS&	2022&	Item Masking/Shuffling/Overlapping&		G-G&InfoNCE&	JL&	Sequential\\
    \multirow{2}{*}{TiCoseRec~\cite{TiCoseRec}}&  	\multirow{2}{*}{AAAI}&	\multirow{2}{*}{2023}	&Ti-crop/Ti-mask/Ti-reorder &		\multirow{2}{*}{G-G}&\multirow{2}{*}{InfoNCE}&	\multirow{2}{*}{JL}&\multirow{2}{*}{Sequential}\\ 
  & &		&/Ti-substitute/Ti-insert&	&	&	&	\\
    % TiCoseRec~\cite{TiCoseRec}&	AAAI&	2023&	Ti-crop/Ti-mask/Ti-reorder/Ti-substitute/Ti-insert&	InfoNCE&	G-G	&JL	&Sequential\\
    \multirow{2}{*}{IOCRec~\cite{IOCRec}}&  	\multirow{2}{*}{WSDM}&	\multirow{2}{*}{2023}	&Item Shuffling/Cropping &	\multirow{2}{*}{G-G}&	\multirow{2}{*}{InfoNCE}&	\multirow{2}{*}{JL}&\multirow{2}{*}{Sequential}\\ 
  & &		&/Masking/Substituting/Inserting&	&	&	&	\\
    % IOCRec~\cite{IOCRec}&	WSDM&	2023&	Item Cropping/Masking/Shuffling/Substituting/Inserting&	InfoNCE&	G-G	&JL&	Sequential\\
    \multirow{2}{*}{MCCM~\cite{MCCM}}&  	\multirow{2}{*}{WSDM}&	\multirow{2}{*}{2023}	&Feature Extractor&		\multirow{2}{*}{L-L/G-G}&\multirow{2}{*}{InfoNCE}&	\multirow{2}{*}{JL}&\multirow{2}{*}{News}\\ 
  & &		&/FItem Masking/Instituting&	&	&	&	\\
    CCL~\cite{CCL}&	CIKM&	2021&	Item Masking/Sequence Generator	&G-G&	InfoNCE&	P\&F&	Sequential\\
    MIC~\cite{MIC}	&CIKM&	2022&	Feature/Message Dropout	&	G-G&InfoNCE	&JL	&Sequential\\
    EC4SRec~\cite{EC4SRec}	&CIKM&	2022&	Item Cropping/Masking/Shuffling&G-G&	InfoNCE	&	JL	&Sequential\\
    DuoRec~\cite{DuoRec}&	WSDM&	2022&	Message Dropout&G-G&	InfoNCE	&	JL	&Sequential\\
    CBiT~\cite{CBiT}&	CIKM&	2022&	Item Masking/Message Dropout&G-G	&InfoNCE	&	JL	&Sequential\\
    \multirow{2}{*}{ContrastVAE~\cite{ContrastVAE}}&  	\multirow{2}{*}{CIKM}&	\multirow{2}{*}{2022}	&Item Masking/Cropping/Shuffling&		\multirow{2}{*}{G-G}&\multirow{2}{*}{InfoNCE}&	\multirow{2}{*}{JL}&\multirow{2}{*}{Sequential}\\ 
      & &		&/Message Dropout/Variational Dropout&	&	&	&	\\
    % ContrastVAE~\cite{ContrastVAE}&	CIKM	&2022&	Item Masking/Cropping/Shuffling/Message Dropout/Variational Dropout	&InfoNCE&	G-G	&JL&	Sequential\\
    \multirow{2}{*}{CLUE~\cite{CLUE}}&  	\multirow{2}{*}{CIKM}&	\multirow{2}{*}{2022}	&Item Masking/Cropping/Shuffling&		\multirow{2}{*}{G-G}&\multirow{2}{*}{BYOL}&	\multirow{2}{*}{P\&F}&\multirow{2}{*}{Sequential}\\ 
      & &		&/Message Dropout&	&	&	&	\\
      FDSA\_CL~\cite{FDSA_CL} &TKDE&2023&Message Dropout&G-G&InfoNCE&JL&Sequential\\
    DHCN~\cite{DHCN}&	AAAI&	2021&	Feature Shuffling&	G-G	&	JS&JL	&Session-based\\
    OD-Rec\cite{on_device}	&SIGIR&	2022&	Without Augmentation	&	L-C	&InfoNCE&JL&	Session-based\\
    CGL~\cite{CGL_tois22}&	TOIS&	2022&	Without Augmentation&		G-G	&JS&JL	&Session-based\\
    CFM~\cite{CFM}&	CIKM	&2021&	Feature Dropout	&	G-G&InfoNCE&	JL&	Feature-based\\
    \multirow{2}{*}{CL4CTR~\cite{CL4CTR}}&  	\multirow{2}{*}{WSDM}&	\multirow{2}{*}{2023}	&Message/feature Dropout&	\multirow{2}{*}{G-G}&	\multirow{2}{*}{BYOL}&	\multirow{2}{*}{JL}&\multirow{2}{*}{CTR Prediction}\\ 
  & &		&/Dimension Masking&	&	&	&	\\
    % CL4CTR~\cite{CL4CTR}&	WSDM&	2023&	Message/feature Dropout/Dimension Masking&	BYOL&	G-G	&JL&	CTR prediction\\
    CLCRec~\cite{CLCRec}&	ACM MM&	2021&	Without Augmentation&	G-G&	InfoNCE&	JL&	Cold-start\\
    NCL~\cite{linImprovingGraphCollaborative2022a}&	WWW&	2022	&Clustering&L-C&	InfoNCE		&JL	&Graph-based CF\\
    ICL~\cite{chenIntentContrastiveLearning2022}&	WWW&	2022	&Clustering	&	L-C&InfoNCE&	JL&	Sequential\\
    SITN~\cite{SITN}& AAAI&2023&Clustering&L-C&InfoNCE&P\&F&Cross-domain\\
    MHCN~\cite{MHCN}&	WWW	&2021&	Subgraph Sampling/Feature Shuffling	&	L-C&Triplet&	JL&	Social\\
    SMIN~\cite{longSocialRecommendationSelfSupervised2021a}&	CIKM	&2021&	Graph Diffusion&	L-C&	JS&	JL&	Social\\
    % EGLN~\cite{yang2021egln}&	SIGIR&	2021&	Edge Dropout/Edge adding/Feature shuffling&	JS&	L-G	&JL&	Graph-based CF\\
    \multirow{2}{*}{EGLN~\cite{yang2021egln}}&  	\multirow{2}{*}{SIGIR}&	\multirow{2}{*}{2021}	&Edge Dropout/Adding&	\multirow{2}{*}{G-G}&	\multirow{2}{*}{JS}&	\multirow{2}{*}{JL}&\multirow{2}{*}{Graph-based CF}\\ 
      & &		&/Feature Shuffling&	&	&	&	\\
    HGCL~\cite{cai2022hgcl}	&TMM&	2022&	Feature Shuffling&		L-G	&JS&JL	&Micro-video\\
    BiGI~\cite{cao2021bigi}	&WSDM&	2021&	Subgraph Sampling&C-G&	JS&	JL&Graph-based CF\\
    MMSSL~\cite{MMSSL}&	WWW	&2023&	Without Augmentation&	C-G	&	InfoNCE&JL&	Graph-based CF\\
    C$^{2}$DSR~\cite{C2DSR}&	CIKM&	2022&	Item Substituting&	C-G	&	JS&JL&	Cross-domain\\
    SSI~\cite{SSI}& IJCAI&2021& Item Masking&C-G&InfoNCE&P\&F&Sequential\\
    S$^3$-Rec~\cite{zhouS3RecSelfSupervisedLearning2020}&	SIGIR&	2020&	Item Masking/Cropping&	L-C/C-G/L-G&	InfoNCE	&P\&F&	Sequential\\
    TCPSRec~\cite{TCPSRec}&	CIKM&	2022&	Sequence Dividing	&L-C/L-G/C-C&	InfoNCE&	P\&F&	Sequential\\
    % MCCM~\cite{MCCM}&	WSDM&	2023	&Feature Extractor/Item Masking/Instituting&	InfoNCE	&L-L/G-G	&JL&	News\\
  \bottomrule
\end{tabular}
}
\end{table*}

\section{View Generation}\label{sec:view_gen}
Recent works~\cite{wang2022strongerAug, tian2020makesview, li2022nlpAug} in other fields have shown that contrastive learning relies heavily on view generation, as generating multiple views facilitates models to explore richer underlying semantic information. In practice, if multiple data views naturally exist, such as interaction views and social networks in social recommendations, pretext tasks can be performed directly on these views. 
In addition, multiple views are not available in many scenarios, so augmentations are needed to generate data views from the original data~\cite{gao2021simcse, chen2020simclr, gutmann2010nce}.
Therefore, we divide existing view generation strategies into view generation with augmentation and without augmentation.

\subsection{With Augmentation}
Augmentation strategies can be categorized into data-based augmentation and model-based augmentation. 
The former generates views based on the data, while the latter is based on the model (i.e., the encoder).

\subsubsection{Data-based Augmentation}
Based on the type of data to be augmented, we classify data-based augmentation into graph-based augmentation, sequence-based augmentation, and feature-based augmentation.
\begin{figure}
    \centering
    \includegraphics[width=0.6\linewidth]{figs/graph_aug1.pdf}
    \caption{Graph-based augmentation.}
    \label{fig:graph_aug}
\end{figure}

\textbf{Graph-based Augmentation.}
% todo: cite crosscbr [30]
This strategy (shown in Fig.\ref{fig:graph_aug}) performs augmentations on the graph (e.g., interaction graph and social graph) to generate multiple views. Note that since the augmentations of node attributes in graphs are similar to feature-based augmentation, under this subcategory we only present the augmentations of the graph structure (shown in Fig.~\ref{fig:graph_aug}).
Formally, given a graph $\mathcal{G} = (\mathcal{V}, \mathcal{E})$, graph-based augmentation transforms the adjacent matrix $\mathbf{A}$ of $\mathcal{G}$, i.e., $\mathcal{T} = \mathcal{T}(\mathbf{A})$.

\textit{Edge perturbation.}
% todo: cite GCARec
This strategy~\cite{wuSelfsupervisedGraphLearning2021, DCL, HCCF, li2023sgccl, KGCL, KACL, SGGCF, CrossCBR,DR_MTCDR,yang2021egln,CHEST} generates graph views through randomly adding or dropping edges. It can be defined as:
\begin{equation}
    \mathcal{T}(\mathbf{A}) = \mathbf{A} \circ (1- \mathbf{L}) + (1 - \mathbf{A}) \circ \mathbf{L}
\end{equation}
where $\mathbf{L}$ is the location matrix. If $\mathbf{L}_{ij}=1$, the edge between $i$ and $j$ will be perturbed. Specifically, if $\mathbf{A}_{ij}=1, \mathbf{L}_{ij}=1$, the edge between $i$ and $j$ will be dropped. If $\mathbf{A}_{ij}=0, \mathbf{L}_{ij}=1$, an edge will be added between $i$ and $j$.
$\mathbf{L}$ can be randomly sampled~\cite{wuSelfsupervisedGraphLearning2021, KGCL} or manually set.
% todo: grace引入
Furthermore, $\mathbf{L}$ can also be calculated adaptively~\cite{LDA_GCL,GCARec} to keep important edges while perturbing possibly unimportant ones.

\textit{Graph Diffusion.} The graph diffusion~\cite{GDCL,longSocialRecommendationSelfSupervised2021a} incorporates the global information to the original graph by creating new edges between nodes. It can be formulated as:
\begin{equation}
\mathcal{T}(\mathbf{A})=\sum_{k=0}^{\infty} \Theta_k \mathbf{T}^k
\end{equation}
where $\Theta_k$ is the weighting coefficient. $\mathbf{T}$ denotes the generalized transition matrix. 
For example, SMIN~\cite{longSocialRecommendationSelfSupervised2021a} generates substructure-aware adjacent matrix and injects it into the user-item interaction graph.

\textit{Subgraph Sampling.}
This strategy samples a node subset and corresponding edges to generate a subgraph as the data view. Existing methods usually obtain the node subset $\mathcal{V}'$ by uniform sampling and ego-net sampling and knowledge-based sampling.
\textit{Uniform sampling}~\cite{wuSelfsupervisedGraphLearning2021,RGCL,MPT,zhangDoubleScaleSelfSupervisedHypergraph2021,DR_MTCDR} uniformly samples a certain portion of nodes and corresponding edges to augment the views. Node dropout belongs to uniform sampling. For example, SGL~\cite{wuSelfsupervisedGraphLearning2021} randomly drops a portion of nodes, which is denoted as $\mathcal{V}_d$. Therefore, the sampled node subset can be obtained by $\mathcal{V}' = \mathcal{V} - \mathcal{V}_d$.
\textit{Ego-net sampling}~\cite{cao2021bigi} samples the $L$-hop neighbors of each node in a graph, also known as the $L$-ego net.
Therefore, the node subset can be represented as
   $\mathcal{V}' = \left\{j| d(i, j) \leq L \right\}$
, where $d(v_i, v_j)$ is the shortest distance between node $i$ and $j$.
\textit{Knowledge-based sampling}~\cite{MHCN, CHEST} incorporates domain knowledge when sampling subgraph. For example, MHCN~\cite{MHCN} designs three types of triangular motifs based on underlying semantics. Motifs specify high-order relations like "having a mutual friend".

\begin{figure}
    \centering
    \includegraphics[width=0.8\linewidth]{figs/seq_aug1.pdf}
    \caption{Sequence-based augmentation.}
    \label{fig:seq_aug}
\end{figure}
\textbf{Sequence-based Augmentation.}
This strategy (shown in Fig.\ref{fig:seq_aug}) performs augmentations on the user interaction sequences. Formally, give the interaction sequence $s_u$ of user $u$, it can be formulated as $\tilde{s}_u = \mathcal{T}(s_u)$.

\textit{Item Shuffling.}
The item shuffling~\cite{CL4SRec,CoSeRec,ContraRec,IOCRec,ContrastVAE,EC4SRec} randomly shuffle a continuous subsequence of the interaction sequence to generate the augmented sequence:
\begin{equation}
    \mathcal{T}(s_u) = [i_{u,1}, i_{u,2}, \cdots, \tilde{i}_{u,k}, \cdots, \tilde{i}_{u, k+l_s-1}, \cdots, i_{u,|s_u|}]
\end{equation}
where $[{i}_{u,k}, \cdots, {i}_{u, k+l_s-1}]$ is shuffled as $[\tilde{i}_{u,k}, \cdots, \tilde{i}_{u, k+l_s-1}]$ . $l_c = \lceil \rho_s |s_u| \rceil$  is the length of the subsequence and $\rho_s \in [0,1]$.

\textit{Item Cropping.}
The item cropping~\cite{CL4SRec,H2SeqRec,CoSeRec,IOCRec,ContrastVAE,EC4SRec} randomly chooses a continuous sub-sequence of the interaction sequence and can be represented as:
\begin{equation}
    \mathcal{T}(s_u) = \left[i_{u,k}, i_{u,k+1}, \cdots, i_{u,k+l_c-1}\right] 
\end{equation}
where $l_c = \lceil \rho_c |s_u| \rceil$ is the length of the subsequence and $\rho_c \in [0, 1]$ is the hyperparameter.

\textit{Item Masking.}
This strategy~\cite{CL4SRec,H2SeqRec,CoSeRec,ContraRec,CBiT,IOCRec,ContrastVAE,EC4SRec} randomly chooses a portion of items in the interaction sequence and replaces them with a [mask] token, which can be formulated as:
\begin{equation}
    \mathcal{T}(s_u) = \left[\tilde{i}_{u,1}, \tilde{i}_{u,2}, \cdots, \tilde{i}_{u,|s_u|}\right]
\end{equation}
where $\tilde{i}_{u,k} = [\text{mask}]$ if $i_{u,k}$ is masked, otherwise $\tilde{i}_{u,k}=i_{u,k}$. 

\textit{Item Substituting.}~\cite{CoSeRec,IOCRec} 
As dropout-based augmentation methods such as item masking may exacerbate the problem of data sparsity and cold-start, item substituting and item inserting are proposed. The item substituting randomly replaces a portion of items in the sequence with other items, which can be formulated as:
\begin{equation}
    \mathcal{T} = \left[i_{u,1}, i_{u,2}, \cdots, \tilde{i}_{u,k},\cdots,i_{u,|s_u|}\right]
\end{equation}
where $\tilde{i}_{u,k}$ replaces $i_{u,k}$. Moreover, CoSeRec~\cite{CoSeRec} substitutes items with highly correlated items to maintain the item correlations in the sequences.

\textit{Item Inserting.}~\cite{CoSeRec,IOCRec}
Fewer interactions are recorded in the interaction sequence than the complete behavior of the user, as interaction data from other sources may be missing. Therefore, the comprehensive user preferences and item correlations cannot be captured.
To complete the sequence, CoSeRec~\cite{CoSeRec} proposes the item inserting to generate the augmented sequence.
Firstly, it randomly samples a portion of items in the sequence. Then, items that correlated to sampled items are inserted around them:
\begin{equation}
    \mathcal{T}(s_u) = [i_{u,1}, i_{u,2}, \cdots, \tilde{i}_{u,k}, {i}_{u,k},\cdots, i_{u,|s_u|}]
\end{equation}
where ${i}_{u,k}$ is the sampled item and $\tilde{i}_{u,k}$ is the item related to it.

\begin{figure}
    \centering
    \includegraphics[width=0.6\linewidth]{figs/feature_aug3.pdf}
    \caption{Feature-based Augmentation.}
    \label{fig:feat_aug}
\end{figure}

\textbf{Feature-based Augmentation.}
Feature-based augmentation (shown in Fig.\ref{fig:feat_aug}) performs augmentations on the feature vectors, which can be categorical features or feature representations (e.g., embeddings). Given feature matrix $\mathbf{X}$, the augmented view is represented as $\tilde{\mathbf{X}} = \mathcal{T}(\mathbf{X})$.

\textit{Feature Dropout.}
The feature dropout (masking)~\cite{CFM,MIC,CL4CTR} masks/drops a portion of the features and is formulated as:
\begin{equation}
    % \mathcal{T}(\mathbf{X}) = \mathbf{X} \circ (1-\mathbf{L}) + \mathbf{M} \circ \mathbf{L}
    \mathcal{T}(\mathbf{X}) = \mathbf{X} \circ (1-\mathbf{L})
\end{equation}
where $\mathbf{L}$ is the masking matrix that indicates the masking locations. If the $j$-th feature of $i$ is masked/dropped, then $\mathbf{L}_{ij}=1$, otherwise $\mathbf{L}_{ij}=0$. Similar to edge perturbation, $\mathbf{L}$ can be uniformly sampled or manually assigned. $\circ$ is the Hadamard product. 

\textit{Feature Shuffling.} The feature shuffling~\cite{yang2021egln, cai2022hgcl, MHCN, DHCN} perturbs the feature matrix by row or column. 
It can be formulated as:
\begin{equation}
    \mathcal{T}(\mathbf{X}) = \mathbf{X}[idx_r, idx_c]
    % \mathcal{T}(\mathbf{X}) = \mathbf{X}[idx_r, :]
\end{equation}
where $idx_r$ and $idx_c$ are the shuffled row index and the shuffled column index, respectively.

\begin{figure}
    \centering
    \includegraphics[width=\linewidth]{figs/model_aug2.pdf}
    \caption{Model-based augmentation.}
    \label{fig:model_aug}
\end{figure}
\subsubsection{Model-based Augmentation}
Model-based augmentation (shown in Fig.\ref{fig:model_aug}) generates views by perturbing the model (i.e., encoder) and can be formulated as:
\begin{equation}
    \mathcal{T}(\mathcal{D}) = (f_{\theta}(\mathcal{D}), f'_{\theta '}(\mathcal{D}))
\end{equation}
where $f_\theta$ and $f'_{\theta '}$ are the encoder and perturbed encoder, respectively. $\mathcal{D}$ is the original data.

\textbf{Message Dropout.}
This strategy\cite{DuoRec,CrossCBR,CBiT,CL4CTR,SRMA} randomly masks the neurons in the layers for a certain dropout ratio~\cite{gao2021simcse}. Then, by applying different dropout masks, multiple views can be obtained with the same input data.
For example, DuoRec~\cite{DuoRec} applies two different dropout masks on the Transformer-based model to generate two different views. 

\textbf{Embedding Noise.}
This strategy~\cite{SimGCL,XSimGCL,RocSE} generates different views by adding different noises to original embeddings. 
Unlike feature-based augmentations that only perturb input embeddings or the final representations, this strategy adds noise to the embeddings at different layers of the encoder.
It can be formulated as:
\begin{equation}
    \tilde{\mathbf{E}_l} = \mathbf{E}_l + \Delta_l
\end{equation}
where $\mathbf{E}_l$ is the original embedding and $\Delta_l$ is the perturbation noise at the $l$-th layer. In SimGCL~\cite{SimGCL}, the $\Delta_l \sim U(0,1)$ is the random uniform noise. In RocSE~\cite{RocSE}, $\Delta_l = \epsilon \cdot f_{\text{norm}}(f_{\text{shuffle}}(\mathbf{E}_l))$, where $f_{\text{shuffle}}$ and $f_{\text{norm}}$ are the random shuffling and normalization operations, respectively. $\epsilon$ is a hyper-parameter.

\textbf{Parameter Noise.} This strategy~\cite{xia2022simgrace} adds noises to the parameters of the encoder, which is formulated as:
\begin{equation}
    \theta'_l = \theta_l + \epsilon \Delta_l
\end{equation}
where $\theta_l$ and $\theta'_l$ are the original parameters and perturbed parameters of $l$-th layer, respectively. The $\Delta_l$ is the random noise, that can be sampled from the Gaussian distribution. $\epsilon$ is a hyper-parameter.

\textbf{Architecture Perturbation.}
Unlike the above strategies that perturb learnable parameters in the model, some works generate different views by changing the model architecture. For example, SRMA~\cite{SRMA} proposes \textit{Layer Dropout} and \textit{Encoder Complementing}. Specifically, the \textit{Layer Dropout} randomly drops a portion of layers in the model during training to enable contrastive learning between shallow features and deep features. The \textit{Encoder Complementing} uses a pre-trained encoder to generate representations. These representations are combined with the representations generated by the original encoder for contrastive learning. 
% todo:图学习里的
MA-GCL~\cite{MA-GCL} proposes to perturb the architecture of graph neural network (GNN) encoders by varying the number and permutations of propagation and transformation operators.

\subsection{Without Augmentation}
The key idea of contrastive learning is to maximize the agreement between different views. Thus, if multiple views naturally exist, these views can be contrasted directly without additional augmentations. 
% In other words, methods under this category are to construct views existing in the data.
For example, in cross-domain recommendation, the two domains can be considered as two views. Therefore, some methods such as CCDR~\cite{CCDR} and ML-SAT~\cite{ML-SAT}, directly perform contrastive learning between these domains.
For knowledge graph-based recommendation, Some methods~\cite{MCCLK,KACL} use the knowledge graph as a contrastive view.
% todo: multi-behavior, knowledge graph
For multi-behavior recommendation, views can be constructed based on the auxiliary behavior data. For example, S-MBRec~\cite{S-MBRec} treats each type of behavior as a view. 
Specifically, HMG-CR~\cite{yangHyperMetaPathContrastive2021} build different hyper meta-graphs based on the hyper meta-paths constructed using the distance between auxiliary behavior and target behavior \textit{buy}. 
In bundle recommendation, user-item interaction and user-bundle interaction can also be contrasted~\cite{CrossCBR}. 

\begin{table*}
\caption{Comparison between different view generation strategies.}
\label{tab:aug_discussion}
% \resizebox{\linewidth}{!}{
\begin{tabular}{cccc}
\toprule
\multicolumn{1}{c|}{\multirow{2}{*}{}} & \multicolumn{2}{c|}{With Augmentation}                                 & \multirow{2}{*}{Without Augmentation} \\ \cline{2-3}
\multicolumn{1}{c|}{}                  & \multicolumn{1}{c|}{Data-based} & \multicolumn{1}{c|}{Model-based} & \\ 
\midrule
Trial-and-errors Free& \XSolidBrush& \CheckmarkBold& \XSolidBrush\\ 
% \midrule
Domain Knowledge Free& \CheckmarkBold& \CheckmarkBold& \XSolidBrush\\ 
% \midrule
Generalizability& \XSolidBrush& \CheckmarkBold& \XSolidBrush\\
% \midrule
Semantic Preservation& \XSolidBrush& \CheckmarkBold& \CheckmarkBold\\
% \midrule
% Randomness& \CheckmarkBold& \CheckmarkBold& \XSolidBrush\\ 
\bottomrule
\end{tabular}
% }
\end{table*}

\subsection{Discussion}
Table.~\ref{tab:aug_discussion} shows the comparison between different view generation strategies. 
In specific, most existing CL-based recommendation methods adopt data-based augmentation strategies as they are easy to implement. However, data-based augmentations are usually selected by manual trial-and-errors, which significantly limits the generalizability of these methods. 
In addition, some data-based augmentations destroy the semantic information of the original data, potentially harming recommendation performance ~\cite{SimGCL}.

Strategies without augmentation do not require trial-and-error. Moreover, these strategies typically use domain knowledge to build auxiliary views, which preserves the semantics of the data. However, domain knowledge is expensive and cannot be applied to other domains. Furthermore, since the views are fixed during model training, strategies without augmentation lack the introduction of randomness that helps to learn noise-invariant representations.

Compared to other strategies, model-based augmentations have better generalizability because they vary the learned representations without considering the original data. Although model-based augmentations require no trial-and-error and domain knowledge, settings such as the dropout ratio of messages/layers still require manual tuning. This limits their generalizability to some extent. Additionally, designing architecture perturbations is challenging.
% methods that add noises may also destroy the semantic information.
% The design of architecture perturbation is difficult.

% todo: cite
Furthermore, many works~\cite{CBiT, ContrastVAE, CrossCBR, CL4CTR} adopt hybrid methods by combining multiple view generation strategies.
By doing so, the advantages of different strategies can be combined. However, some disadvantages may still exist. For instance, combining strategies without augmentation with data-based augmentations can be helpful in introducing randomness but data-based augmentation still requires manual trial-and-errors.
% todo:增加理论分析
In addition, how to choose the optimal view generation strategy for a specific recommendation task is still unclear.


% There have been neither rigorous understanding of how and why they work nor rules or guidelines clearly telling what good augmentations are for recommendation.
% Researchers have proved that data augmentation can boost contrastive learning’s performance, but the theory for why and how it helps is still quite ambiguous. This hinders its application.

% Despite the prosperous development of CL methods, data
% augmentation schemes, proved to be a critical component for visual
% representation learning [48], remain rarely explored.
% There have been neither rigorous understanding of how and why they work nor rules or guidelines clearly telling what good augmentations are for recommendation. Besides, some common augmentations, which were considered to be useful, recently even have been proved to has a negative impact on the recommendation performance [101]. As such, without knowing what augmentations are informative, the contrastive task may fall short of expectation.

% Currently, there is no theoretical analysis guiding the generation of the view for graphs. However, Tian et al. [80] theoretically and empirically analyze the problem in a general view and image domain, considering the generation of the view from the aspect of mutual information. In particular, a good view generation should minimize the MI between two views I(v1, v2), subject to I(v1, y) = I(v2, y) = I(x, y). Intuitively, to guarantee that contrastive learning works, the generated views vi should not affect the information that determines the prediction for the downstream task, under which restriction, stronger disagreement between views leads to better learning results. Following the above idea, AD-GCL [81] proposes to generate views of graphs that achieve the above minimum under constraints by parameterizing the above types of transformations and propose learnable transformations. In particular, the transformations are learned in an adversarial manner - the transformation (views generator) is trained to minimize I(v1, v2) subject to I(v1, y) = I(v2, y) = I(x, y), whereas the encoder is trained to maximize I(v1, v2). Following a similar principle, InfoGCL [82] proposes to discretely select optimal views from a list of candidate views based on the mutual information with downstream tasks.

% From the manifold point of view, a recent analytic study [83] proposes the expansion assumption and explains the data augmentation as to prompt the continuity in the neighborhood for each instance. It indicates similar requirements for the view generating by augmentation, i.e., an ideal augmentation should satisfy the following two conditions, 1) the (augmentation) neighborhood of an instance does not intersect the neighborhood of instances that belong to the other class in the downstream task, 2) the neighborhood of an instance should be as large as possible, subject to 1. 

% To this end, the learning on datasets with different downstream tasks may benefit from different types of transformations. For example, the property of a social network to be predicted in a downstream task may be more tolerant of minor changes in node attributes, for which the feature transformations can be more suitable. On the other hand, the property of a molecule usually depends on bonds in some functional groups, for which the edge perturbation may harm the learning while the subgraph sampling could help. Empirically, You et al. [49] observes similar results. For example, edge perturbation is found contributory to the performance on social network datasets but harmful to some molecule data.
\section{Pretext Task}~\label{sec:pretext_task}
The goal of contrastive learning is to maximize the mutual information (MI) between positive pairs (i.e., instances with the same semantic information) and minimize the MI between negative pairs (i.e., instances with unrelated semantic information). According to the scale of instances, we classify existing contrastive pretext tasks into two categories: same-scale contrasting and cross-scale contrasting. 

Specifically, there are three contrastive scales: local, contextual, and global. The local scale usually represents the minimum granularity of the input data, while the global scale represents the maximum. For instance, in graph (sequence) data, the local scale represents the node (item/feature), and the global scale represents the whole graph (sequence). The contextual scale is between the local and global scales and represents the subgraph (subsequence).

% For graph data, local scale represents the node and global scale represents the whole graph.
% For sequence data, local scale represents the item and global scale represents the sequence.
% For feature data, local scale represents the feature and global scale represents the feature vector (all features of a user or an item).
% todo: c
% Fig. 8 and Table 3 provide the pipelines and summaries of contrastive learning-based recommendation methods, respectively.
% note:pretext task figures
% \begin{figure}[t]
%     \centering
%     \subfigure[Same-Scale Contrasting.]{ 
%         \includegraphics[width=0.38\linewidth]{figs/same_scale.pdf}\label{fig:same_scale}}
%     \subfigure[Cross-Scale Contrasting.]{
%         \includegraphics[width=0.38\linewidth]{figs/cross_scale.pdf}\label{fig:cross_scale}}
%     \caption{Two categories of Pretext Task.} 
%     \label{fig:ssls}
%     %\vspace{-0.1in}
% \end{figure}

\begin{figure}
    \centering
    \includegraphics[width=0.4\linewidth]{figs/same_scale4.pdf}
    \caption{Illustration of same-scale contrasting.}
    \label{fig:same_scale_whole}
\end{figure}
\subsection{Same-Scale Contrasting}
Depending on the different scales being contrasted, same-scale contrasting(shown in Fig.\ref{fig:same_scale_whole}) can be further divided into three sub-types: local-local (L-L) contrasting, contextual-contextual (C-C) contrasting, and global-global (G-G) contrasting. Considering the unique characteristics of the recommendation tasks, we present existing methods based on their recommendation tasks.

\subsubsection{Local-Local Contrasting}
% todo:没提f_theta
Methods under this category mainly discriminate the local representations (i.e., representation of users/items) and can be formulated as
\begin{equation}
   \theta^{*}, \omega^{*}=\underset{\theta, \omega}{\arg \min } \mathcal{L}_{con}\left(p_\omega\left(\mathbf{h}_i, \mathbf{h}_j\right)\right)
\end{equation}
where $\mathbf{h}_i$ and $\mathbf{h}_j$ are the representation of instance $i$ and $j$ in different views respectively. Furthermore, these representations are generated by encoder $f_\theta(\cdot)$, which is usually shared by different views.

% \textbf{General User-Item Collaborative Filtering.}
\textbf{Graph-based Collaborative Filtering.}
% Existing CL-based methods for collaborative filtering are typically graph-based methods. 
Depending on the types of graphs being contrasted, methods can be categorized into \textit{contrasting on user-item graph} and \textit{contrasting on different graphs}.

(i) \textit{Contrasting on User-Item Graph.}
As only one graph exists, methods under this category should perform augmentations on the user-item interaction graph to generate different views.
% Note that graph-based methods usually adopt same-scale contrasting by performing node discrimination.
% in which $i$ and $j$ are of same type, i.e., they are both users or both items.

\textbf{SGL}~\cite{wuSelfsupervisedGraphLearning2021} first applies contrastive learning to graph-based recommendation. Given a user-item interaction graph $\mathcal{G}$. It first generates two different graph views $\tilde{\mathcal{G}}^{(1)}=\mathcal{T}(\mathcal{G})$ and $\tilde{\mathcal{G}}^{(2)}=\mathcal{T}(\mathcal{G})$. $\mathcal{T}$ is the view generation strategy. Moreover, it utilizes three data-based augmentations including node dropout, edge dropout, and random walk (apply edge dropout at each layer). 

Then, it utilizes LightGCN~\cite{he2020lightgcn} as graph encoder $f_\theta(\cdot)$ to generate node representations  $\mathbf{H}^{(1)}=f_\theta(\tilde{\mathcal{G}}^{(1)})$ and $\mathbf{H}^{(2)}=f_\theta(\tilde{\mathcal{G}}^{(2)})$. 
% todo: 添加GCARec DCL(survey)引用,确定SimGCL的对比loss
Afterward, it performs the node self-discrimination task. Specifically, it makes the representations of the same node (i.e., the positive pair) in different views similar while making representations of different nodes (i.e., the negative pairs) in different views dissimilar. The contrastive loss of the user side can be formulated as: 
\begin{equation}\label{eq:sgl_loss}
    \mathcal{L}^{user}_{con} = - \log \frac{\exp(p_\omega(\mathbf{h}_u^{(1)}, \mathbf{h}_u^{(2)}))}{\exp(p_\omega(\mathbf{h}_u^{(1)}, \mathbf{h}_u^{(2)})) + Neg}
\end{equation}
where $\mathbf{h}_u^{(1)} \in \mathbf{H}^{(1)}$ and $\mathbf{h}_u^{(2)} \in \mathbf{H}^{(2)}$ are representations of user $u$. $p_\omega(\cdot)$ is the cosine similarity with a temperature parameter $\tau$. $p_\omega(\mathbf{z}_u^{(1)}, \mathbf{z}_u^{(2)}) = (\mathbf{z}_u^{(1)})^T\mathbf{z}_u^{(2)}/\tau$ and $\mathbf{z}_u^{(1)} = \mathbf{h}_u^{(1)}/||\mathbf{h}_u^{(1)}||$. 
In addition, for efficiency, SGL adopts in-batch negative sampling, considering only different nodes of the same batch $\mathcal{B}$ instead of using all other nodes as negative samples. Therefore, $Neg$ is defined as
\begin{equation}\label{eq:neg}
    Neg = \sum_{v\in\mathcal{B}, v \neq u} \exp(p_\omega(\mathbf{h}_u^{(1)}, \mathbf{h}_v^{(2)}))
\end{equation}
Note that $(\mathbf{h}_u^{(1)}, \mathbf{h}_v^{(2)})$ is the \textit{inter-view} negative pairs. The loss of the item side $\mathcal{L}^{item}_{con}$ can be obtained in the same way. The contrastive loss is $ \mathcal{L}_{con} = \mathcal{L}^{user}_{con} + \mathcal{L}^{item}_{con}$.
Finally, SGL adopts a jointly learning strategy to optimize the contrastive loss and recommendation loss.
% \begin{equation}
%     \mathcal{L} = \mathcal{L}_{con} + \lambda \mathcal{L}_{rec}
% \end{equation}
% where $\lambda$ is a hyperparameter that controls strength of contrastive loss. 
% todo:GCARec
% Based on the framework of SGL, GCARec [11] introduces an adaptive augmentation scheme to adaptively drop edges.
% ~\cite{DCL, SimGCL, XSimGCL, RocSE, RGCL, LightGCL}

Based on the framework of SGL, several works are proposed.
The main difference with SGL is in the view generation strategies.
\textbf{DCL}~\cite{DCL} perturbs the edges in $L$-ego net of each node to obtain views. 
\textbf{GDCL}~\cite{GDCL} generate new graph view using graph diffusion. Moreover, it constructs the \textit{intra-view} negative pairs and the Eq.(\ref{eq:neg}) can be rewritten as
\begin{equation}
    Neg = \sum_{v\in\mathcal{U}, v \neq u} \exp(p_\omega(\mathbf{h}_u^{(1)}, \mathbf{h}_v^{(2)}) + p_\omega(\mathbf{h}_u^{(1)}, \mathbf{h}_v^{(1)}))
\end{equation}
\textbf{SimGCL}~\cite{SimGCL}, \textbf{XSimGCL}~\cite{XSimGCL}, and \textbf{RocSE}\cite{RocSE}
generate views by adding uniform noises to node representations. Moreover, to reduce the computational complexity, XSimGCL~\cite{XSimGCL} replaces the final-layer contrast with cross-layer contrasting. It only utilizes one GNN-encoder and contrasts embeddings of different layers.
\textbf{LightGCL}~\cite{LightGCL} proposes a singular value decomposition (SVD)-based graph augmentation strategy to effectively distill global collaborative signals. 
In specific, SVD is first performed on the adjacency matrix. Then, the list of singular values is truncated to retain the largest $K$ values and truncated matrices are used to reconstruct the adjacency matrix. The node contrastive learning is performed between the reconstructed graph and the original graph. 
\textbf{RGCL}~\cite{RGCL} also performs edge contrastive learning. It maximizes the MI between the review representation and the corresponding interaction representation. Specifically, the interaction representation is obtained by feeding the user and item representations into an MLP. 
% \begin{figure}[t]
%     \centering
%     \includegraphics[width=0.5\linewidth]{figs/self_dis.pdf}
%     \caption{Node Self-Discrimination.}
%     \label{fig:my_label}
% \end{figure}


% For local level contrast, dropout-based augmentations are the most preferred methods to create perturbed local views. SGL [30], as a representative graph CL-based recommendation model, applies three types of stochastic graph augmentations: node dropout, edge dropout, and random walk (multi-layer edge dropout) on the user-item bipartite graph. It first generates two augmented graphs with the same type of augmentation operator. Then it applies a shared graph LightGCN encoder fθ [87] to learn node embeddings from the augmented graphs. The node level contrast is conducted by optimizing the InfoNCE loss [13] with in-batch negative sampling. Finally, SGL jointly optimizes the above InfoNCE loss and the Bayesian personalized ranking (BPR) loss [88] for recommendation.
% Opposite to the neuron masking which discards some information in the hidden representations, SimGCL [101] directly adds random uniform noises to the hidden representations for augmentations. It experimentally proves that optimizing the InfoNCE loss in fact learns more uniform node representations and adjusting the noise magnitude can provide a finer-grained regulation of representation uniformity, which mitigates the popularity bias issue [39]. Benefiting from the noised-based augmentations, SimGCL shows distinct advantages over SGL on both recommendation accuracy and model training efficiency.

(ii) \textit{Contrasting on Different Graphs.}
In addition to the user-item interaction graph, some works construct other graphs using interaction data. 
That is, views are usually generated without augmentation.
% Therefore, methods under this category generally perform augmentations on these views to incorporate randomness.
% note:sequential

\textbf{MCLSR}~\cite{MCLSR} constructs three graphs based on the interaction sequences, including a user-item relation graph, an item-item relation graph, and a user-user relation graph. 
% After generating the representations in these graphs, it maximizes the MI between the representations of same nodes and minimizes the MI between the representations of different nodes in corresponding graphs.
% op
\textbf{HCCF}~\cite{HCCF} constructs two views, including a user-item interaction graph and a learnable hypergraph. The node discrimination task same as SGL is performed in MCLSR and HCCF.
% To be more specific, it treats representations of the same user/item in different views as a positive pair while treats representations different users/items in different views as negative pairs. 
\textbf{SGCCL}~\cite{li2023sgccl} constructs a user-user graph $\mathcal{G}_{uu}$ and an item-item graph $\mathcal{G}_{ii}$ and performs edge/feature dropout to augment them. Node self-discrimination is conducted on the $\tilde{\mathcal{G}}_{uu}$ and $\tilde{\mathcal{G}}_{ii}$.  
% todo: 统一edge dropout的说法
% In addition, it applies the edge dropout over these two views to further alleviate the over-fitting problem.
\textbf{MPT}~\cite{MPT} extends PT-GNN~\cite{PT_GNN} which performs reconstruction tasks and can only model the intra-correlations. It leverages contrastive tasks to capture the inter-correlations within the data.
Specifically, it samples subgraphs/paths for each user. 
% The path is obtained by random walk strategy from interaction data. 
Node dropout/substitution are applied to augment subgraphs/paths. 
% Then the representations are generated by GNN and Transformer-based encoder, respectively. 
The MI between representations of the same user in augmented subgraphs/paths is maximized. 


% todo:提一下knowledge graph
\textbf{Knowledge Graph-based Recommendation.}
Apart from interaction data, the knowledge graph (KG) is also utilized for CL-based recommendation, as it can bring rich semantic information.

% todo:kGIC的类别
% It is worth noting that most existing efforts using knowledge graphs adopt original preservation, that is, the views are generated by manual design.
Generally, CL-based recommendation methods using KG generate views by manual design.
% op
\textbf{MCCLK}~\cite{MCCLK} constructs three graph views, including user-item graph $\mathcal{G}_{ui}$, item-entity graph $\mathcal{G}_{ie}$ and user-item-entity graph $\mathcal{G}_{uie}$. 
% A multi-level contrastive learning is conducted. Specifically, 
It maximizes the MI between the user representations $\mathbf{h}_u^{(ui)}$, $\mathbf{h}_u^{(uie)}$ in $\mathcal{G}_{ui}$ and $\mathcal{G}_{uie}$ and between the item representations $\mathbf{h}_i^{(ui)}$, $\mathbf{h}_i^{(ie)}$ in $\mathcal{G}_{ui}$ and $\mathcal{G}_{ie}$. Furthermore, it generates a new representation for each item
\begin{equation}
\mathbf{h}_i^\prime = \mathbf{h}_i^{(ui)} || \mathbf{h}_i^{(ie)}    
\end{equation}
where $||$ is the concatenation operation. 
Then the MI between $\mathbf{h}_i^\prime$ and $\mathbf{h}_i^{(uie)}$ is maximized by performing node-self discrimination.
\textbf{KACL}~\cite{KACL} maximizes the MI between representations of the same item in the augmented user-item graph and the augmented knowledge graph. Moreover, the augmented graphs are generated by automatically dropping unimportant edges. 

KG can also be used to guide the generation of different user-item graph views. 
For example, \textbf{KGCL}~\cite{KGCL} performs stochastic augmentation on the knowledge graph to generate two different views $\tilde{\mathcal{G}}_k^{(1)}$ and $\tilde{\mathcal{G}}_k^{(2)}$. 
The the item consistency is $c_i= sim(\mathbf{h}^{(1)}_i, \mathbf{h}^{(2)}_i)$. 
% \begin{equation}
%     c_i= sim(\mathbf{h}^{(1)}_i\mathbf{h}^{(2)}_i)
% % \frac{\mathbf{h}^{(1)}_i\mathbf{h}^{(2)}_i}{||\mathbf{h}^{(1)}_i|| ||\mathbf{h}^{(2)}_i||}
% \end{equation}
$sim(\cdot)$ is the cosine similarity fucntion. $\mathbf{h}^{(1)}_i$ and $\mathbf{h}^{(2)}_i$ are item representations in $\tilde{\mathcal{G}}_k^{(1)}$ and $\tilde{\mathcal{G}}_k^{(2)}$, respectively.
Then, the user-item interaction graph $\mathcal{G}_{ui}$ is augmented using knowledge-guided data augmentation. Specifically, it performs edge dropout based on the probability obtained on the consistency.

\begin{equation}
\begin{gathered}
w_{ui}  =\exp \left(c_i\right)\\
p_{ui}^{\prime} =\max \left(\frac{w_{ui}-w^{\max }}{w^{\max }-w^{\min }}, p_\tau\right) \\
p_{ui}  =p_a \cdot \mu_{p^{\prime}} \cdot p_{ui}^{\prime}
\end{gathered}
\end{equation}
where $p_{ui}$ is the dropout probability of edge $(u,i)$ in $\mathcal{G}_{ui}$. $p_\tau$ is the threshold. 
$p_{ui}^{\prime}$ is an intermediate variable that is integrated with the mean value $\mu_{p^{\prime}}$.
$p_a$ is a strength controller.
With the $p_{ui}$, masking vectors $\mathcal{M}_i \in \{0,1\}$ are generated based on the Bernoulli distribution~\cite{marshall1985family}. The augmented graphs is $\tilde{\mathcal{G}}_{ui}^{(i)} = (\mathcal{V}, \mathcal{E} \circ \mathcal{M}_i)$. 
Moreover, KGCL performs both intra-view contrasting and inter-view contrasting.
% It treats the representations of the same node in different views as the positive pair and representations of different nodes in both views as negative pairs.
% After integrating  the augmentation on the knowledge graph and the interaction graph 

\textbf{Multi-behavior Recommendation.}
% In some recommendation scenarios, views can also be obtained based on other data. 
For enhancing user intention modeling, multi-behavior recommendation methods incorporate multiple types of user behavior data, which can be used to build contrastive views.
% note: multi-behavior
% todo: cml是否放入
% CML~\cite{weiContrastiveMetaLearning2022} obtains contrastive loss by contrasting one behavior and other behaviors. Moreover, it incorporates contrastive learning in meta-learning framework, in which the meta-knowledge is extracted based on the contrastive loss. 
\textbf{S-MBRec}~\cite{S-MBRec} adopts a star-style contrastive task, i.e., it only performs contrastive learning between the target behavior (usually the buy) and each auxiliary behavior. It samples positive samples based on the similarity under target behavior. The similarity is calculated by point-wise mutual information~\cite{PMI}. 
\begin{equation}
\begin{gathered}
P M I\left(u, u^{\prime}\right)  =\log \frac{p\left(u, u^{\prime}\right)}{p(u) p\left(u^{\prime}\right)}, \\
p(u) =\frac{|\mathcal{I}(u)|}{|\mathcal{I}|}, \\
p\left(u, u^{\prime}\right) =\frac{\left|\mathcal{I}(u) \cap \mathcal{I}\left(u^{\prime}\right)\right|}{|\mathcal{I}|},
\end{gathered}
\end{equation}
where $\mathcal{I}(u)$ is items that user $u$ has interacted. $\mathcal{I}$ is the item set and $|\mathcal{I}|$ is the number of items.
If similarity $P M I\left(u, u^{\prime}\right) > t$, $(u, u^{\prime})$ are considered as positive pairs. $t$ is the threshold. The positive samples of the items are selected in a similar way. Moreover, negative samples are selected randomly. 
\textbf{MMCLR}~\cite{MMCLR} constructs a graph view (user-item graph) $\mathcal{G}$ and a sequence view (multi-behavior sequence) $\mathcal{S}$. For each view, different behavior representations of the same users (e.g., $\mathbf{h}_{u,b_1}^{g}$ and $\mathbf{h}_{u,b_2}^{g}$) are treated as positive pairs. It also maximizes the MI between overall representations of the same user (e.g., $\mathbf{h}_{u}^g$ and $\mathbf{h}_{u}^s$) in different views. 
% todo: 具体怎么knowledge-aware没讲
\textbf{KMCLR}~\cite{KMCLR} maximizes the MI between different behaviors of the same user. In addition, it performs knowledge-aware contrastive learning. It leverages a knowledge graph to guide the augmentation of the user-item graph under the target behavior $\mathcal{G}_{ui}^{b_t}$. 
It first calculates the consistency $c_i$ of each item $i$ like KGCL.
Then the edge dropout probability is obtained by
\begin{equation}
\begin{gathered}
\hat{p}_{ui}=\sigma\left(\mathbf{h}_u^T \mathbf{h}_i\right) \circ c_i, \\
p_{ui}=\left(1-\operatorname{Min} \_\operatorname{Max}\left(\hat{p}_{u i}\right)\right) a+\operatorname{Min\_ Max}\left(\hat{p}_{ui}\right) b
\end{gathered}
\end{equation}
where $\mathbf{h}_u$ and $\mathbf{h}_i$ are the user representation and item representation in $\mathcal{G}_{ui}^{b_t}$, respectively. $\operatorname{Min} \_\operatorname{Max}(\cdot)$ is the min-max normalization
function. $a$ and $b$ are hyperparameters that control the value interval of $p_{ui}$.
Then it performs edge dropout in the way similar to KGCL. Moreover, KMCLR adopts the node self-discrimination task.


% todo:BCD
% BCD [] constructs a graph for each type of behavior. It uses two different dropout ratios to randomly drop the edges of the graph to obtain two views.
% For the intra-behavior relation, it maximize the mutual information between the same user in the different views of the same auxiliary behavior while minimize the mutual information between different users in these views.
% For the inter-behavior relation, it maximize the mutual information between the same user in auxiliary behavior an target behavior and minimize the mutual information between users in these views.
% Moreover, it distills knowledge from target behavior by applying contrastive learning on the target behavior.

\textbf{Social Recommendation.}
For social recommendation, social networks are utilized to improve recommendation performance. 
Similar to KG-based recommendation, views can be generated based on manual design.
\textbf{SEPT}~\cite{SEPT} constructs three graph views based on the interaction data and social network, including preference view $\mathcal{G}_r$, friend view $\mathcal{G}_f$, and sharing view $\mathcal{G}_s$. 
$\mathcal{G}_r$ is the user-item interaction graph. Other views are constructed based on two types of triangle motifs. 
Moreover, it leverages tri-training~\cite{zhou2005tri} to predict positive samples for each view. The Eq.(\ref{eq:sgl_loss}) is changed as follows
\begin{equation}
\mathcal{L}_{con}=- \sum_{v \in\{r, s, f\}}\log \frac{\sum_{p \in \mathcal{P}_{u+}^v} p_\omega\left(\mathbf{h}_u^v, \tilde{\mathbf{h}}_p\right)}{\sum_{p \in \mathcal{P}_{u+}^v} p_\omega\left(\mathbf{h}_u^v, \tilde{\mathbf{h}}_p\right)+\sum_{j \in \mathcal{U} / \mathcal{P}_{u+}^v} p_\omega\left(\mathbf{h}_u^v, \tilde{\mathbf{h}}_j\right)}
\end{equation}
where $\mathcal{P}_{u+}^v$ is the set of predict positive samples.
$\mathbf{h}_u^{v}$ is the user representation in view $v$. 
$p_\omega$ is the cosine similarity with temperature parameter $\tau$. $\tilde{\mathbf{h}}_p$ is the representation of user $p$ in $\tilde{\mathcal{G}}$, which is obtained by performing random edge dropout on the joint graph of the user-item interaction graph and the social network.
% \begin{equation}
% \tilde{\mathcal{G}}=\left(\mathcal{V}_r \cup \mathcal{V}_s, \mathcal{M} \odot\left(\mathcal{E}_r \cup \mathcal{E}_s\right)\right),
% \end{equation}
% where $\mathcal{M}$
\textbf{HGCL\_S}~\cite{HGCL_s} constructs three types of graphs: user-item graph, user-user graph, and item-item graph. It performs the node self-discrimination task on the corresponding graphs. Moreover, when generating node representations in the user-user graph and item-item graph, HGCL\_S utilizes a meta network.

% note: cross-domain
\textbf{Cross-domain Recommendation.}
Different domains in cross-domain recommendation can be considered as different views.
For cross-domain recommendation, two types of contrastive tasks can be conducted, i.e., \textit{single-domain} contrasting and \textit{cross-domain} contrasting. 

\textbf{CCDR}~\cite{CCDR} performs both  single-domain contrasting and cross-domain contrasting. For the single-domain contrasting, it samples two subgraphs of each node to generate two node representations. Then the MI between these two representations is maximized. 
For the cross-domain contrasting,
It maximizes the MI between the representations of the same node in the source domain and the target domain. Besides, to extract more cross-domain knowledge between unaligned nodes, CCDR maximizes the MI between the representation of an aligned node in the source domain and the representations of its target-domain neighbors.
% todo:要不要提scenario

\textbf{ML-SAT}~\cite{ML-SAT} studies the multi-scenario problem, which can be viewed as a multi-domain recommendation problem. Moreover, it only performs the cross-domain contrasting between two different scenarios. It treats the representations of the same users/items in different domains as positive pairs and representations of other users/items in both scenarios as negative pairs.
\textbf{DR-MTCDR}~\cite{DR_MTCDR} only performs sing-domain contrasting. It augments the user-item graph in each domain by edge/node dropout. For each domain, it generates $K$ channel node representations. For each channel, MI between representations of the same node is maximized. 

% note:位置 hhgr group recommendation
\textbf{Group Recommendation.}
To improve group recommendation performance, \textbf{S$^2$-HHGR}~\cite{zhangDoubleScaleSelfSupervisedHypergraph2021} builds a hierarchical hypergraph based on the user-item, group-item, and user-group interactions. It applies double-scale node dropout strategies including coarse- and fine-grained dropout on the hypergraph. The former drops users in all groups. The latter only drops some nodes in a specific group while other groups still contain these users. It performs node self-discrimination on the coarse- and fine-grained user representations as follows: 
\begin{equation}
\begin{gathered}
\mathcal{L}_{con}=-\log \sigma\left(p_\omega\left(\boldsymbol{h}_u^{\prime}, \boldsymbol{h}_u^{\prime \prime}\right)\right)-\sum_{j=1}^n\left[\log \sigma\left(1-p_\omega\left(\boldsymbol{h}_j^{\prime}, \boldsymbol{h}_u^{\prime \prime}\right)\right)\right], \\
p_\omega\left(\boldsymbol{h}_u^{\prime}, \boldsymbol{h}_u^{\prime \prime}\right)=\sigma\left(\boldsymbol{h}_u^{\prime} \mathbf{W}\boldsymbol{h}_u^{\prime \prime T}\right)
\end{gathered}
\end{equation}
where $\boldsymbol{h}_u^{\prime}$ and $\boldsymbol{h}_u^{\prime \prime}$ are the coarse-grained user representation and fine-grained user representation, respectively.  $n$ is the number of negative samples. \textbf{SGGCF}~\cite{SGGCF} performs user node dropout and edge dropout on the user-item-group graph. Moreover, representations of the same nodes in original and augmented graphs are viewed as positive pairs. Besides, it performs cross-layer contrasting. The MI between initial node embedding and $l$-th ($l$ is an even number) layer embedding is maximized.

\textbf{Bunndle Recommendation.} Based on user-item interaction, user-bundle interaction, and item-bundle 
affiliation, \textbf{CrossCBR}~\cite{CrossCBR} constructs a bundle view (user-bundle graph) and an item view (user-item graph and bundle-item graph). It performs edge dropout and message dropout on these views and generates user representation and bundle representations. Moreover, MI between representations of the same user/bundle in corresponding views is maximized.

\textbf{Session-based Recommendation.}
Based on the session data, \textbf{COTREC}~\cite{COTREC} constructs two graph views, including item view (item-item graph) and session view (session-session graph). Inspired by SEPT, it utilizes co-training~\cite{co_training} to predict the positive and negative samples for each session. Note that the positive and negative samples are items. Furthermore, it maximizes the MI between the representation of the last item in the session and representations of the predicted positive samples. The MI between the representation of the last item and representations of the negative samples is minimized.
% Moreover, to avoid the mode collapse, COTREC applies a divergence constraint on encoders.

\subsubsection{Contextual-Contextual Contrasting}
% todo: user representation or graph representation (c-c or l-l)
For contextual-contextual contrasting, discrimination is performed on the contextual representations. It can be formulated as:
\begin{equation}
    \theta^{*}, \omega^{*}=\underset{\theta, \omega}{\arg \min } \mathcal{L}_{con}\left(p_\omega\left(\mathbf{c}_i, \mathbf{c}_j\right)\right)
\end{equation}
where $\mathbf{c}_i$ and $\mathbf{c}_j$ are contextual representations denoting data with similar contextual information.
% The main branch under this category is based on subgraph sampling. 
% For example,

\textbf{Multi-behavior Recommendation.}
% note: multi-behavior
 To capture different user behavior patterns, \textbf{HMG-CR}~\cite{yangHyperMetaPathContrastive2021} performs the hyper meta-graph discrimination task.
It first constructs different hyper meta-graphs $\{\mathcal{G}_u^{(i)}\}_{i=1}^{K}$ for each user based on hyper meta-paths. In specific, the hyper meta-path is constructed based on the distance to the target behavior.
Then representations of these hyper meta-graphs are generated through different encoders, i.e., $\mathbf{c}_u^{(i)} = f_\theta^i(\mathcal{G}_u^{(i)})$.
Furthermore, it treats the representations of adjacent hyper meta-graphs ($\mathbf{c}_u^{(i-1)}, \mathbf{c}_u^{(i)}$) as negative pairs. The positive samples are generated by feeding the current hyper meta-graph into the encoder of the adjacent hyper meta-graph. For example, the positive sample of $\mathbf{c}_u^{(i)}$ is $\mathbf{c}_p = f_\theta^{i-1}(\mathcal{G}_u^{(i)})$. 

\textbf{HIN-based Recommendation.}
\textbf{CHEST}~\cite{CHEST} obtain subgraphs based on the relevance of paths in the heterogeneous information networks (HIN). It also performs data-based augmentations on these subgraphs. The augmented subgraphs generated from the same subgraphs are viewed as positive samples. Subgraphs that connect the same user to other items are considered as negative samples.
Moreover, it performs generative pretext tasks such as masked node/edge prediction to capture local information. It leverages curriculum learning~\cite{bengio2009curriculum} to pre-train the model in an elementary-to-advanced process. The contrastive pretext tasks serve as the advanced course while generative pretext tasks serve as the elementary course.
\textbf{KGIC}~\cite{KGIC} constructs local and non-local graphs by combining user-item interactions and the knowledge graph. Moreover, \textit{intra-view} contrasting and \textit{inter-view} contrasting are performed among these graphs.

% note:没有提表示
% In addition to constructing views based on the meta-path, 
\textbf{Sequential Recommendation.} 
\textbf{GCL4SR}~\cite{GCL4SR} obtains subgraphs based on uniform sampling. Specifically, it first constructs a transition graph based on all user interaction sequences. For each sequence node, it randomly samples two different subgraphs. The subgraphs of the same sequence are positive pairs, and those of different sequences are viewed as negative pairs.
Unlike other works that perform contrasting on the graph, \textbf{MISS}~\cite{MISS} performs it on the feature vectors, which consist of categorical and sequential features.
It leverages two CNN-based models to extract multiple interests contained in each feature vector. 
Moreover, it makes representations of the same interest similar and representations of different interests dissimilar.

% Similar to CCL [63], CHEST [70] also connects curriculum learning with SSL to pre-train a Transformer-based recommendation model on heterogeneous information networks. However, its pretext tasks are unlinked. Given the user-item graph and associated attributes, CHEST conducts meta-path-based random walks starting from a user node and ending with an item node to form interaction-specific subgraphs, each of which consists of multiple meta-path instances. The generative task in CHEST predicts the masked node/edge in a subgraph with the rest, which exploits the local context information and is considered the elementary course. The contrastive task learns subgraph-level semantics for user-item interaction by pulling the original subgraph and augmented subgraphs closer, which exploits the global correlations and is considered the advanced course.


\subsubsection{Global-Global Contrasting}
Methods under this category discriminate the global representations, which can be represented as:
\begin{equation}
   \theta^{*}, \omega^{*}=\underset{\theta, \omega}{\arg \min } \mathcal{L}_{con}\left(p_\omega\left(\mathbf{g}_i, \mathbf{g}_j\right)\right)
\end{equation}
where $\mathbf{g}_i$ and $\mathbf{g}_j$ are the global representation. 
Moreover, global-global contrasting is typically used in sequential recommendation and session-based recommendation, where $\mathbf{g}$ represents a sequence or a session.

%note: sequential
\textbf{Sequential Recommendation.}
We introduce the methods for sequential recommendation according to the view generation strategies. Generally, existing methods adopt \textit{data-based} augmentation and \textit{model-based} augmentation.

(i) \textit{Methods using Data-based Augmentation.}
\textbf{CL4SRec}~\cite{CL4SRec} utilizes three types of sequence-based augmentation: sequence cropping, sequence shuffling, and item masking. 
Given an interaction sequence of user $u$, 
it applies augmentation $\mathcal{T}$ that is randomly sampled from three augmentations to generate different sequence views. $\tilde{s}_u^{(1)} = \mathcal{T}(s_u)$ and $\tilde{s}_u^{(2)} = \mathcal{T}(s_u)$. 
SASRec~\cite{SASRec} is used as sequence encoder $f_\theta(\cdot)$ to generate sequence (global-level) representations $\mathbf{g}_u^{(1)} = f_\theta(\tilde{s}_u^{(1)})$ and $\mathbf{g}_u^{(2)} = f_\theta(\tilde{s}_u^{(2)})$. 
It performs the sequence self-discrimination task. Specifically, it makes the representations of augmented sequences from the same sequence (i.e., positive pairs) to be similar and those from different sequences (i.e., negative pairs) to be dissimilar. 
\begin{equation}
\mathcal{L}_{con}=- \log \frac{\exp (p_\omega(\mathbf{g}_u^{(1)}, \mathbf{g}_u^{(2)}))}{\exp (p_\omega(\mathbf{g}_u^{(1)}, \mathbf{g}_u^{(2)}))+Neg}.
\end{equation}
where $p_\omega(\cdot)$ is the cosine similarity with temperature parameter $\tau$. 
CL4SRec adopts in-batch negative sampling and the $Neg$ is defined as
\begin{equation}
    Neg = \sum_{v \in \mathcal{B}, v \neq u} \exp (p_\omega((\mathbf{g}_u^{(1)}, \mathbf{g}_v^{(1)})) + p_\omega((\mathbf{g}_u^{(1)}, \mathbf{g}_v^{(2)})))
\end{equation}
Similar to CL4SRec, \textbf{H$^2$SeqRec}~\cite{H2SeqRec} contrasts sequence representations. Moreover, the model is pre-trained with contrastive tasks.
\textbf{CoSeRec}~\cite{CoSeRec} and \textbf{ContraRec}~\cite{ContraRec} adopt the same framework and objective as CL4SRec. Moreover, CoSeRec proposes two robust augmentation strategies, i.e., \textit{item substituting} and \textit{item inserting}. In addition to the augmented sequence from the same sequence, ContraRec also treats the sequences that have the same target item as positive samples. Based on CoseRec, \textbf{TiCoseRec}~\cite{TiCoseRec} proposes five data augmentation based on time intervals to generate uniform sequences. Moreover, a uniform (un-uniform) sequence is one in which the standard deviation value of its time interval series is relatively small (large).
\textbf{IOCRec}~\cite{IOCRec} generates $K$ intention representations for each augmented sequence. It maximizes MI between representations of the same sequence with the same intent, which can be formulated as:
\begin{equation}
\mathcal{L}^{k}_{con}=-\log \frac{\exp (p_\omega(\mathbf{g}_{u,k}^{(1)},\mathbf{g}_{u,k}^{(2)}))}{\sum_{g \in \mathcal{N}} \exp (p_\omega(\mathbf{g}_{u,k}^{(1)}, \mathbf{g}))} .
\end{equation}
where $\{\mathbf{g}_{u,k}^{(1)}\}_{k=1}^K$ and $\{\mathbf{g}_{u,k}^{(2)}\}_{k=1}^K$ denote the intention representations of augmentation sequences. $p_\omega(\cdot)$ is the dot product. 
% Given $|\mathcal{B}|$ users in a minibatch, 
$\mathcal{N}$ is the set of negative samples, including different intention representations of the same user and all intention representations of different users.

% todo: copy
% model-level augmentation is proposed in this paper. In the computation of a sequence vector, there are Dropout modules in both the embedding layer and the Transformer encoder. Forward-passing an input sequence twice with different Dropout masks will generate two different vectors,
\textbf{MCCM}~\cite{MCCM} incorporates contrastive learning into news recommendation. It augments sequences by performing item masking/substituting. Moreover, the masking/substituting probability is calculated based on the frequency of news. 
\begin{equation}
p_i=\frac{\log (\operatorname{count}(i))}{\max _{j \in \mathcal{I}} \log (\operatorname{count}(j))}(p_{\max }-p_{\text {min }})+p_{\text {min }}
\end{equation}
where $p_{\max}$ and $p_{\min}$ are the predefined boundaries of the probability. $\operatorname{count}(i)$ is the frequency of news $i$ in the dataset. $\mathcal{I}$ is the news set. It performs sequence self-discrimination similar to CL4SRec. 
% The MI between the representations of the same sequence is maximized. 
% todo:这里没看懂
% Moreover, it utilizes feature extractor to generates different aspects of
% the news and maximizes the MI between different aspects of the same news.
\textbf{CCL}~\cite{CCL} augments sequences by leveraging a data generator based on the mask-and-fill operation. In specific, it first masks a portion of items in each sequence. Then the data generator recovers the original
sequence. The original sequence and recovered sequences from the same user are treated as positive pairs and other recovered sequences from different users are negative samples. Moreover, CCL utilizes curriculum learning to conduct contrastive learning via an easy-to-difficult process.

In addition to interaction data, \textbf{MIC}~\cite{MIC} utilizes attributes of user/items.
It constructs user/item sequences, which consist of user/item attributes and interaction records. Then, feature dropout is applied to generate different user/item sequences. Moreover, it treats k-nearest neighbors of the user/item as positive samples. It also clusters the users/items using k-means++ and treats users/items that are from different clusters as negative samples. 

\textbf{EC4SRec}~\cite{EC4SRec} further incorporates explanation methods into data-based augmentation. It obtains the importance scores of items to guide the augmentation based on explanation methods. The importance scores of items in sequence $s_u$ are calculated as
\begin{equation}
{score}(s_u)= F_e(y_u, s_u, f_\theta), 
\end{equation}
where $F_e$ is any explanation method. $f_\theta$ is the sequence encoder. $y_u$ is the prediction probability for the next item.
${score}(s_u) = [{score}(i_{{u},{1}}),\cdots,{score}(i_{u,{|s_u|}})]$ and ${score}(i_{{u},{1}})$ is the important score of item $i_{u, 1}$. 
Then, it crops/masks/shuffles the items with the lowest scores to generate positive samples. 
It also adopts supervised positive sampling to sample sequences like ContraRec and takes the sequences with high important scores among them as the positive samples. 
To generate negative samples, it masks the items with the highest scores. The cropped items also form negative samples. In addition to making positive samples from the same users similar, it also makes negative samples from different users more similar than positive samples from all users.
% EC4SRec [45] extends CoSeRec [25] with contrastive signals selected by explanation methods

(ii) \textit{Methods using Model-based Augmentation.}
% Besides data-based augmentation, some works generate views using model-based augmentation. 
\textbf{DuoRec}~\cite{DuoRec} obtains different sequence representations based on message dropout. Specifically, the sequence is fed twice with different dropout masks in the Transformer-based encoder. Then it performs sequence self-discrimination on these representations. Similar to ContraRec, DuoRec also uses supervised positive sampling, where an interaction sequence with the same target item is randomly selected as a positive sample.
% todo: 确定contrarec positive是augmented还是original (augmented √)

There are several methods that also apply employ dropout to generate different sequence representations. Besides message dropout, \textbf{CBiT}~\cite{CBiT} also uses item masking. Specifically, it first generates $K$ different sequences by item masking. Then, these sequences are fed into bidirectional Transformers with different dropout masks to generate representations. In addition, CBiT proposes multi-pair contrastive learning. All $K$ augmented sequences are treated as positive samples. It defined the loss as
\begin{equation}
\mathcal{L}_{\text {con }}=- \sum_{x=1}^{K} \sum_{y=1}^{K} \mathbf{1}_{[x \neq y]}   \log \frac{\exp (p_\omega(\mathbf{g}_u^{(x)}, \mathbf{g}_u^{(y)}))}{\exp (p_\omega(\mathbf{g}_u^{(x)}, \mathbf{g}_u^{(y)}))+ \sum_{\mathbf{g} \in \mathcal{N}}\exp (\mathbf{g}_u^{(x)}, \mathbf{g})}
\end{equation}
where $\mathcal{N}$ is the set of negative samples, which are the sequence representations of different users.
% note:variational aug是否写入view generation, 接下来的方法又加了增广方式

\textbf{ContrastVAE}~\cite{ContrastVAE} incorporates contrastive learning into the Variational
AutoEncoder. Besides message dropout and data-based augmentations, it utilizes the variational dropout that introduces a learnable Gaussian dropout rate during the sampling step. The selection of negative and positive pairs is similar to CL4SRec. \textbf{CLUE}~\cite{CLUE} also adopts both message dropout and sequence-based augmentations. The main difference is that CLUE dose not use negative samples. 
\textbf{FDSA\_CL}~\cite{FDSA_CL} constructs feature sequences consisting of item features. It maximizes the MI between the representations of feature sequences of and interaction sequences. In addition, it applies message dropout to generate $K$ representations for each feature sequence. 
% todo:不知道怎么改,要不要去掉
% \textbf{CLSR}~\cite{CLSR} uses two different encoders to generate long- and short-term interest representations. To supervise the encoders, it obtains proxies for long- and short-term interest from the sequence. Specifically, the long-term proxy is the average of all interactions in the sequence, while the short-term proxy is the average of the most recent $K$ interactions. The contrastive learning is performed between representations and proxies to make the representations more similar to corresponding proxies than to other proxies. 
% \textbf{DisenPOI}~\cite{DisenPOI} constructs a geographical graph and a sequential for given visiting sequence and generates geographic and sequential representations of the sequence. In addition, the mean pooling is applied to the geographic neighbors and the POI previously visited by the user to generate two proxies for both representations. The proxies are viewed as the positive sample for corresponding representation.

% note: session-based
\textbf{Session-based Recommendation.}
\textbf{DHCN}~\cite{DHCN} constructs two hypergraphs (i.e., $\mathcal{G}_h$ and $\mathcal{G}_l$) to represent intra- and inter-session information, respectively. 
In DHCN, representations of the same session (e.g., $\mathbf{g}_s^{h}$ and $\mathbf{g}_s^{l}$) is the positive pair. 
Moreover, it perturbs the representations matrix $\mathbf{H}^h$ with row- and column-wise shuffling.  
The negative samples are the perturbed representation $\tilde{\mathbf{g}}_s^{h} \in \tilde{\mathbf{H}}$ of the same session. $\tilde{\mathbf{H}}$ is the perturbed representation matrix. It maximizes MI between positive pairs and minimizes MI between negative pairs by
\begin{equation}
\mathcal{L}_{con}=-\log \sigma(p_\omega(\mathbf{g}_s^{h}, \mathbf{g}_s^{l}))-\log \sigma(1-p_\omega(\tilde{\mathbf{g}}_s^{h}, \mathbf{g}_s^{l}))
\end{equation}
where $p_\omega(\cdot)$ is the dot product and $\sigma(\cdot)$ is the sigmoid function.
% Moreover, the perturbed representations are obtained by shuffling the original representations.

% todo: ref
% note:  On-device SSKD
\textbf{OD-Rec}\cite{on_device} incorporates contrastive learning into knowledge distillation for session-based recommendation. Specifically, it maximizes the mutual information between the representations of the same session $s$ that learned from the teacher model and student model (i.e., $\mathbf{g}_s^{tea}$, and $\mathbf{g}_s^{stu}$). The negative pairs are the $(\mathbf{g}_s^{tea}, \mathbf{g}_{s^\prime}^{tea})$. 
\textbf{CGL}~\cite{CGL_tois22} constructs a global graph based on the similarity of sessions. In the graph, each session node is connected with their $M$ most similar sessions. It maximizes the MI between a session node and its neighbors and minimizes the MI of unconnected sessions. Moreover, the session representation is obtained by aggregating the representations of items in it. 
% \textbf{DCAN}~\cite{DCAN} incorporates social network in session-based recommendation. It adopts message dropout to obtain different item probability distributions. Then the KL divergence and JS divergence are applied for minimizing the divergence between this two distributions.
% In addition, DCAN does not require negative sampling.

\textbf{Feature-based Recommendation.}
\textbf{CFM}~\cite{CFM} adopts feature dropout to augment feature vectors. The vectors generated from the same feature vector are treated as positive pairs, while those generated from different feature vectors are treated as negative pairs.
Moreover, to make the pretext task more difficult, it masks/drops the related features. 
\textbf{CL4CTR}~\cite{CL4CTR} 
performs feature-based augmentation strategies to generate two different feature vectors of each samples.
It minimizes the expected distance between representations of the same samples and does not use negative sample. 
Besides using feature representations, \textbf{CLCRec}~\cite{CLCRec} also generates item collaborative representations based on the interaction data. Therefore, it can be viewed as \textit{hybrid recommendation}~\cite{rec_survey}.
The feature representation and collaborative representation of the same item are treated as positive pairs and those of different items are treated as negative pairs. 


\begin{figure}
    \centering
    \includegraphics[width=0.4\linewidth]{figs/cross_scale4.pdf}
    \caption{Illustration of cross-scale contrasting.}
    \label{fig:cross_scale_whole}
\end{figure}
\subsection{Cross-Scale Contrasting}
% todo: 说法修改
% In the same-scale contrasting, the positive pairs and negative pairs are contrasted in the same scale. 
In the cross-scale contrasting (shown in Fig.\ref{fig:cross_scale_whole}), the contrasting is conducted across different scales.
Furthermore, according to the scale, we further divide this branch of methods into three sub-types: local-contextual (L-C) contrasting, local-global (L-G) contrasting, and contextual-global (C-G) contrasting.
\subsubsection{Local-Contextual Contrasting}
% note: sequential
The local-contextual contrasting can be formulated as:
\begin{equation}
   \theta^{*}, \omega^{*}=\underset{\theta, \omega}{\arg \min } \mathcal{L}_{con}\left(p_\omega\left(\mathbf{h}_i, \mathbf{c}_j\right)\right)
\end{equation}
where $\mathbf{h}_i$ is the local representation and $\mathbf{c}_j$ is the contextual representation.


\textbf{Graph-based Collaborative Filtering.}
To capture contextual information, \textbf{NCL}~\cite{linImprovingGraphCollaborative2022a} proposes a prototype-contrastive objective. In specific, for each item/user, the positive sample is the prototype of the cluster it belongs to, and the negative sample is the prototype of other clusters. The prototype is the representation of the cluster center. Moreover, the prototype-contrastive objective is learned with Expectation-Maximization (EM) algorithm. NCL also performs the cross-layer contrasting. For each user/item, corresponding representations output from the even-numbered layer GNN are treated as positive samples.
% NCL [83] follows [84] to design a prototypical contrastive objective to capture the correlations between a user/item and its prototype. The prototype can be seen as the context of each user/item which represents a group of semantic neighbors even that they are not structurally connected in the user-item graph. Regarding the prototype learning as a type of feature clustering for data augmentation, it obtains prototypes by clustering over all the user or item embeddings with K-means algorithm. Then the EM algorithm is used to recursively adjust the prototypes.

\textbf{Sequential Recommendation.}
\textbf{ICL}~\cite{chenIntentContrastiveLearning2022} relies on a framework similar to that of NCL~\cite{linImprovingGraphCollaborative2022a}. 
The difference is that NCL clusters the representations of users or items, while ICL clusters the representations of sequences. Specifically, in ICL, the representation of a sequence can be viewed as a local view of the cluster it belongs to. In addition, each prototype represents the user intents.
% ICL treats the sequences and the prototype the sequences are in as positive pairs, and the sequences and other prototypes as negative pairs. Besides, it also applies sequence cropping, sequence shuffling and item masking to generate augmented sequences and performs global-global contrasting (i.e., contrasts sequence representations).
Besides, ICL also places contrasting between sequences like sequential recommendation methods in global-global contrasting.
% and the only difference is
% that ICL is designed for sequential recommendation where
% the semantic prototypes in NCL are modeled as user intents
% in ICL, and the belonged sequence here is a local view of the
% prototype.

\textbf{Cross-domain Recommendation.} \textbf{SITN}~\cite{SITN} only performs inter-domain contrasting. Like ICL, it clusters user sequence representations to represent the user interests and maximizes the MI between the representations of users and clusters. Specifically, the positive pairs are representations of a user and its corresponding cluster in different domains. The negative samples are new clusters. Additionally, SITN also maximizes the MI of user representations in different domains.

% we treat users’ sequential behavioral representations as instances and the interests encoded by the
% instances as clusters, which can be trained within batches
% in an end-to-end manner.

% The left part shows the contrastive learning module at
% a high level. The right part shows the instance-to-cluster encoder. The role of the first part of instance-to-cluster encoder is to
% generate new clusters from instances and original clusters. The lower right part shows a example of clusters and four instances.
% The role of the second part of instance-to-cluster encoder is to generate MV clusters from instances and new clusters.

\textbf{Social Recommendation.}
% note: social
In addition to using clustering algorithms, contextual information is also modeled based on human prior knowledge.
\textbf{MHCN}~\cite{MHCN} designs three types of triangle motifs based on social relations and models them with a multi-channel hypergraph encoder. Moreover, a multi-channel hypergraph is proposed to model the information. 
In each channel, MHCN hierarchically maximizes the mutual information between the user representation, the user-centered sub-hypergraph representation, and the hypergraph representation.
\textbf{SMIN}~\cite{longSocialRecommendationSelfSupervised2021a} constructs the context by generating a substructure-aware adjacent matrix based on the addition operations of different order adjacent matrices.

\subsubsection{Local-Global Contrasting}
This branch of methods conducts contrastive tasks between the local representation and global representation, which can be presented as
\begin{equation}
    \theta^*, \omega^* = \underset{\theta, \omega}{\arg \min } \mathcal{L}_{con}\left(p_\omega\left(\mathbf{h}_i, \mathbf{g}_j\right)\right)
\end{equation}
where $\mathbf{h}_i$ is the local representation and $\mathbf{g}_j$ is the global representation. 
% Moreover, $\mathbf{g}_j$ is typically obtained through a readout function (so-called summary function), i.e., $\mathbf{g}_j = f_{\text{readout}}(\{\mathbf{h}_i| i \in \mathcal{G}^{(j)}\})$, where $\mathcal{G}^{(j)}$ is the global view $j$.

% todo:为什么要对比全图

\textbf{Graph-based Collaborative Filtering.} \textbf{EGLN}~\cite{yang2021egln} places the contrasting across the edge representation (i.e., the concatenation of representations of its connected nodes) and the global graph representations (i.e., the average of all edge representations). Specifically, the positive samples of $\mathbf{g}_1$ (representation of $\mathcal{G}^{(1)}$) are the edge representations in the graph $\mathcal{G}^{(1)}$ and the negative samples are the edge representations in the augmented graph $\mathcal{G}^{(2)}$. \textbf{HGCL}~\cite{cai2022hgcl} constructs node-type specific homogeneous graphs to preserve the heterogeneity. Following DGI~\cite{DGI}, it maximizes the MI between a node representation and corresponding graph presentation. Moreover, to incorporate the relationship between different node types, HGCL also designs a cross-type contrasting object. For each node type pair $(t_1, t_2)$, given the node-type specific homogeneous graph $\mathcal{G}^{(t_2)}$, the positive samples of it are the node representations in $\mathcal{G}^{(t_2)}$, and the negative samples are the node representation in the augmented graph of $\mathcal{G}^{(t_1)}$.

\subsubsection{Contextual-Global Contrasting} These methods contrast the contextual representation with global representation, which can be defined as: 
\begin{equation}
    \theta^*, \omega^* = \underset{\theta, \omega}{\arg \min } \mathcal{L}_{con}\left(p_\omega\left(\mathbf{c}_i, \mathbf{g}_j\right)\right)
\end{equation}
% where $\mathbf{h}_s$ is the contextual representation and $\mathbf{h}_g$ is the global representation.

\textbf{Graph-based Collaborative Filtering.}
For each edge $(i, j)$, \textbf{BiGI}~\cite{cao2021bigi} performs ego-net sampling to get two subgraphs centered at $i$ and $j$, respectively. Then, it adopts attention mechanism to obtain two contextual representations. The contextual representation of this edge $\mathbf{s}_{ij}$ is the concatenation of contextual representations of $i$ and $j$. Specifically, positive samples of $\mathbf{g}_1$ are the contextual representations of edges in $\mathcal{G}^{(1)}$ and negative samples are the contextual representation of edges in the augmented graph $\mathcal{G}^{(2)}$. 
% multi-modal
\textbf{MMSSL}~\cite{MMSSL} generates multiple modal-specific representations of users. Moreover, it maximizes MI between the modality-specific representation and the overall representation of the same user. The negative samples of a modality-specific representation $\mathbf{c}_u^m$ are both the modality-specific representation $\mathbf{c}_{u^\prime}^m$ and overall representation $\mathbf{g}^m_{u^\prime}$ of different users.

\textbf{Cross-domain Recommendation.}
\textbf{C$^2$DSR}~\cite{C2DSR} obtains cross-domain sequences through merging single-domain sequences in chronological order.
Then, it generates two augmented cross-domain sequences based on item substituting. Based on the self-attention mechanism, it generates item representations in the sequences. The representations of cross-domain sequences are obtained by aggregating the representations of items in each domain. Similarly, the representations of single-domain sequences are obtained. It maximizes the MI between the single-domain sequences and the original cross-domain sequences and minimizes the MI between the single-domain sequences and the augmented cross-domain sequences.

% todo: 两个混合多种的放在哪
\textbf{Sequential Recommendation.}
\textbf{SSI}~\cite{SSI} leverages contrastive learning to capture global consistency in sequential recommendation. 
% Specifically, it masks several items in the interaction sequences. It then maximizes the MI between the representations of masked items and that of the entire sequence. Negative samples are subsequences sampled from other sequences.
Specifically, it samples a subsequence from the interaction sequence and masks the corresponding items in that sequence. It then maximizes the MI between the representation of the subsequence and that of the entire sequence. Negative samples are subsequences sampled from other sequences.

Moreover, based on the belonging relationships in interaction sequences, \textbf{S$^3$-Rec}~\cite{zhouS3RecSelfSupervisedLearning2020} and \textbf{TCPSRec}~\cite{TCPSRec} conduct multiple contrastive tasks. Specifically, S$^3$-Rec devises four objectives, including sequence-item, sequence-attribute, item-attribute, and sequence-subsequence mutual information maximization.
TCPSRec performs item-sequence and item-subsequence contrasting as well as subsequence-subsequence contrasting at both coarse- and fine-grained periodicity levels. In TCPSRec, the subsequences are generated by dividing the interaction sequence when the time interval is greater than a threshold.

\begin{table*}
\caption{Comparison between different pretext tasks.}
\label{tab:pre_discussion}
% \resizebox{\textwidth}{!}
% {
\begin{tabular}{ccccccc}
\toprule
\multicolumn{1}{c|}{\multirow{2}{*}{}} & \multicolumn{3}{c|}{Same-Scale} & \multicolumn{3}{c}{Cross-Scale} \\ \cline{2-7}
\multicolumn{1}{c|}{}                  & \multicolumn{1}{c|}{L-L} & \multicolumn{1}{c|}{C-C} & \multicolumn{1}{c|}{G-G} & \multicolumn{1}{c|}{L-C}& \multicolumn{1}{c|}{L-G} & \multicolumn{1}{c}{C-G}\\ 
\midrule
Context Extraction Free& \CheckmarkBold& \XSolidBrush& \CheckmarkBold& \XSolidBrush & \CheckmarkBold & \XSolidBrush\\ 
Summary(Readout) Function Free& \CheckmarkBold& \XSolidBrush& \XSolidBrush& \XSolidBrush & \XSolidBrush & \XSolidBrush\\ 
% More Noise& \multicolumn{3}{c}{\XSolidBrush} & \multicolumn{3}{c}{\CheckmarkBold}\\
Noise Level& \multicolumn{3}{c}{Low} & \multicolumn{3}{c}{High}\\
% More Information& \multicolumn{3}{c}{\XSolidBrush} & \multicolumn{3}{c}{\CheckmarkBold}\\
% Self-supervision& \multicolumn{3}{c}{Identity} & \multicolumn{3}{c}{Belonging}\\
% Task & \multicolumn{3}{c}{Self-Discrimination} & \multicolumn{3}{c}{Belonging Relationship Modeling}\\
% Negative Samples& L& C& G& C& G& G\\
\bottomrule
\end{tabular}
% }
\end{table*}

\subsection{Discussion}
Table.\ref{tab:pre_discussion} shows the comparison between different pretext tasks. Most existing methods utilize the same-scale contrasting, as it is only necessary to generate representations of the same scale. In contrast, cross-scale contrasting requires generating the corresponding representations for all the different scales. Furthermore, since existing CL-based methods usually use the shared encoder to generate representations, methods that adopt cross-scale contrasting usually require an additional module (i.e., summary function) to generate large-scale representations after generating small-scale representations. 
Take local-global contrasting in the graph-based recommendation as an example, it needs to learn the representation of each node first and then aggregate these representations to generate the graph representation using a readout function.
The contextual contrasting also tends to have high complexity, because it needs to design the corresponding strategy (i.e., context extraction) to decide which part of the data to generate the contextual representation. 

In addition, compared to same-scale contrasting, which usually aims to identify different instances, cross-scale contrasting focuses on modeling the belonging relationship between small and large scales. Considering the complexity, in the cross-scale contrasting, the negative samples are usually selected from the small-scale representations.
% the positive pairs and negative pairs are the graph and the node that not belong to it.
% Therefore, in the cross-scale contrasting, the negative samples are usually selected from the larger scale representations.
Moreover, cross-scale contrasting can introduce more information into the small-scale representations, but this may also introduce more noises, i.e., irrelevant information.

\section{Contrastive Objective}\label{sec:obj}
As introduced in Section.~\ref{sec:unified_framework}, the contrastive objective is to maximize the mutual information (MI) between different views. Specifically, given representations $(\mathbf{h}_i, \mathbf{h}_j)$ of instances $(i,j)$, the MI between them can be represented as:
\begin{equation}
    \mathcal{M I}\left(\mathbf{h}_i, \mathbf{h}_j\right)=KL\left(P\left(\mathbf{h}_i, \mathbf{h}_j\right) \| P\left(\mathbf{h}_i\right) P\left(\mathbf{h}_j\right)\right)=\mathbb{E}_{P\left(\mathbf{h}_i, \mathbf{h}_j\right)}\left[\log \frac{P\left(\mathbf{h}_i, \mathbf{h}_j\right)}{P\left(\mathbf{h}_i\right) P\left(\mathbf{h}_j\right)}\right]
\end{equation}
% todo: estimator cite paper
where $KL(\cdot)$ is the Kullback-Leibler (KL) divergence. Contrastive learning aims to maximize the MI between positive pairs and minimize the MI between negative pairs. Moreover, the positive pair comes from the joint distribution $P(\mathbf{h}_i, \mathbf{h}_j)$ and the negative pair comes from the product of marginal distributions $P(\mathbf{h}_i)P(\mathbf{h}_j)$. 

Depending on whether an estimation of lower-bound of mutual information is provided, we classify the contrastive objective into bound objective and non-bound objective.

% Moreover, of these three estimators, only $\mathcal{M I}_{J S}$ and $\mathcal{M I}_{N C E}$ are currently used for CL-based recommendation.


% In the following subsections, we introduce these two lower bound MI estimators and two non-bound MI estimators.

% \subsection{Donsker-Varadhan Estimator.}
% The Donsker-Varadhan (DV) estimator can be defined as
% \begin{equation}
% \mathcal{M I}_{D V}\left(\mathbf{h}_i, \mathbf{h}_j\right) = \mathbb{E}_{P\left(\mathbf{h}_i, \mathbf{h}_j\right)}\left[p_\omega\left(\mathbf{h}_i, \mathbf{h}_j\right)\right] -\log \mathbb{E}_{P\left(\mathbf{h}_i\right) P\left(\mathbf{h}_j\right)}\left[e^{p_\omega\left(\mathbf{h}_i, \mathbf{h}_j\right)}\right]
% \end{equation}
% % todo:统一p_omega说法
% where $p_\omega(\cdot)$ is the discriminator (i.e., pretext decoder), which generates the agreement score of $\mathbf{h}_i$ and $\mathbf{h}_j$. Moreover, there may be a projection head $g_\phi(\cdot)$ in $p_\omega(\cdot)$, which map representation $\mathbf{h}_i$ to $\amthbf{z}_i$. Specifically, $g_\phi$ can be a linear mapping, MLP, or identical mapping. The discriminator $p_\omega$ can be inner product $p_\omega(\mathbf{z}_i, \mathbf{z}_j) = \mathbf{z}_i^T \mathbf{z}_j$, the cosine similarity $p_\omega(\mathbf{z}_i, \mathbf{z}_j) = \mathbf{z}_i^T \mathbf{z}_j / (||\mathbf{z}_i|| ||\mathbf{z}_j||)$.

\subsection{Bound Objective}
As calculating MI directly is difficult,
lower-bounds are derived to estimate it~\cite{hjelm2018learning}, such as the Donsker-Varadhan estimator $\mathcal{M I}_{D V}$~\cite{donsker1983asymptotic, belghazi2018mutual}, the Jensen-Shannon estimator$\mathcal{M I}_{J S}$~\cite{nowozin2016f}, and the noise-contrastive estimator (InfoNCE) $\mathcal{M I}_{N C E}$~\cite{oord2018representation, gutmann2010noise}. Therefore, MI can be maximized by maximizing the lower-bound.
Moreover, of these three estimators, only
$\mathcal{M I}_{J S}$ and $\mathcal{M I}_{N C E}$ are currently used for CL-based recommendation.

\subsubsection{Jensen-Shannon Estimator.}
% cite more efficidnt
Compared to the DV estimator, the Jensen-Shannon (JS) estimator enables more efficient estimation of MI. It replaces the Kullback-Leibler divergence with the Jensen-Shannon divergence. The contrastive loss based on it can be defined as 

% \begin{equation}
% \begin{aligned}
%   \mathcal{L}_{con} (p_\omega(\mathbf{h}_i, \mathbf{h}_j)) &= - \mathcal{M  I } _ { J S }\left(\mathbf{h}_i, \mathbf{h}_j\right)\\
%   &= - \mathbb{E}_{P}\left[\log \left(p_\omega\left(\mathbf{h}_i, \mathbf{h}_j\right)\right)\right] 
% +\log \mathbb{E}_{P \times \tilde{P}}\left[\log \left(1-p_\omega\left(\mathbf{h}_i, \mathbf{h}^{\prime}_j\right)\right)\right]  \\
% \end{aligned}
% \end{equation}
\begin{equation}
   \mathcal{L}_{JS} = -\mathcal{M  I } _ { J S }\left(\mathbf{h}_i, \mathbf{h}_j\right)\\
  = -\mathbb{E}_{P}\left[\log \left(p_\omega\left(\mathbf{h}_i, \mathbf{h}_j\right)\right)\right] 
- \mathbb{E}_{P \times \tilde{P}}\left[\log \left(1-p_\omega\left(\mathbf{h}_i, \mathbf{h}^{\prime}_j\right)\right)\right]
\end{equation}
% todo
$\mathbf{h}_i$ and $\mathbf{h}_j$ are sampled from distribution $P$, and $\mathbf{h}^{\prime}_j$ is sampled from distribution $\tilde{P}$.
$p_\omega(\cdot)$ is the discriminator (i.e., pretext decoder), which generates the agreement score of $\mathbf{h}_i$ and $\mathbf{h}_j$. Moreover, there may be a projection head $g_\xi(\cdot)$ in $p_\omega(\cdot)$, which map representation $\mathbf{h}_i$ to $\mathbf{z}_i$. Specifically, $g_\xi(\cdot)$ can be a linear mapping, MLP, or identical mapping. The $p_\omega(\cdot)$ can be inner product $\mathbf{z}_i^T \mathbf{z}_j$, the cosine similarity $\mathbf{z}_i^T \mathbf{z}_j / (||\mathbf{z}_i|| ||\mathbf{z}_j||)$, or bi-linear transformation $\mathbf{z}_i^T \mathbf{W} \mathbf{z}_j$.

% There are various forms of $p_\omega(\cdot)$, besides the aforementioned ones there are also some methods \cite{yang2021egln, cao2021bigi} that employ bi-linear transformations (i.e., $p(\mathbf{h}_i, \mathbf{h}_j) = \mathbf{h}_i^T \mathbf{W} \mathbf{h}_j$). 
% Moreover, there is a softplus version of the JS estimator~\cite{hassani2020contrastive, sun2019infograph} by letting $p_\omega\left(\mathbf{h}_i, \mathbf{h}_j\right) = \text{sigmoid} \left(p_{\omega}^{\prime}\left(\mathbf{h}_i, \mathbf{h}_j\right) \right)$,
% \begin{equation}
% \mathcal{MI}_{(J S-S P)}\left(\mathbf{h}_i, \mathbf{h}_j\right)=\mathbb{E}_{P}\left[-s p\left(-p_{\omega}^{\prime}\left(\mathbf{h}_i, \mathbf{h}_j\right)\right)\right]-\mathbb{E}_{P \times \tilde{P}}\left[s p\left(p_{\omega}^{\prime}\left(\mathbf{h}_i, \mathbf{h}_j\right)\right)\right]
% \end{equation}
% where $sp(x) = \log (1+e^{(x)})$. 

\subsubsection{InfoNCE  Estimator.}
% todo:符号
InfoNCE is the most popular MI lower-bound adopted in CL-based methods for recommendation. The contrastive loss based on it can be formulated as
\begin{equation}
\mathcal{L}_{NCE} = -\mathcal{M I}_{N C E} \left(\mathbf{h}_i, \mathbf{h}_j\right)=-\mathbb{E}_{P}\left[p_\omega\left(\mathbf{h}_i, \mathbf{h}_j\right)\right.\left.-\mathbb{E}_{K \sim \tilde{P}^N}\left[\log \frac{1}{N} \sum_{\mathbf{h}_j^{\prime} \in K} e^{p_\omega\left(\mathbf{h}_i, \mathbf{h}_j^{\prime}\right)}\right]\right]
\end{equation}
% todo: dot product 
where $K$ is the set of samples that consists of $N$ random variables identically and independently distributed from $\tilde{P}$. Generally, $p_\omega(\cdot)$ is the cosine similarity with a temperature parameter $\tau$, i.e., $p_\omega(\mathbf{z}_i, \mathbf{z}_j) = \mathbf{z}_i\mathbf{z}_j/\tau$ and $\mathbf{z}_i = \mathbf{h}_i/||\mathbf{h}_i||$. This also knows as the NT-Xent~\cite{sohn2016improved} loss.

In practice, InfoNCE is calculated on a mini-batch $\mathcal{B}$ whose size is $N+1$. Specifically, for each instance $i$ in $\mathcal{B}$, the rest $N$ instances are considered as negative samples. The loss based on InfoNCE can be
\begin{equation}
\mathcal{L}_{NCE}=-\frac{1}{N+1} \sum_{i \in \mathcal{B}}\left[\log \frac{e^{p_\omega\left(\boldsymbol{h}_i, \boldsymbol{h}_j\right)}}{\sum_{j \in \mathcal{B}} e^{p_\omega\left(\boldsymbol{h}_i, \boldsymbol{h}_j^{\prime}\right)}}\right] .
\end{equation}

\subsection{Non-Bound Objective}
In addition to the lower-bound MI estimators mentioned above, some other objectives are used to optimize contrastive learning, i.e., triple loss and BYOL loss. However, they have not proven to be the lower-bound of MI and thus minimizing it does not guarantee to maximize mutual information.
\subsubsection{Triplet Loss.}
The triplet margin does not minimize the agreement of the negative pairs, but only makes the agreement of the positive pairs greater than that of the negative pairs.
It is defined as:
\begin{equation}
\mathcal{L}_{Triplet}\left(\mathbf{h}_i, \mathbf{h}_j\right)=\mathbb{E}_{P \times \tilde{P}}\left[
\max \left\{p_\omega\left(\mathbf{h}_i, \mathbf{h}_j^{\prime}\right)-p_\omega\left(\mathbf{h}_i, \mathbf{h}_j\right)+\epsilon, 0\right\}\right]
\end{equation}
where $\epsilon$ is the margin value. $\mathbf{h}_i$ and $\mathbf{h}_j$ are sampled from distribution $P$, and $\mathbf{h}_j^{\prime}$ is sampled from $\tilde{P}$.
% todo: 超出页面
The discriminator $p_\omega$ can calculated the agreement by $p_\omega{(\mathbf{h}_i, \mathbf{h}_j) = \text{sigmoid}(\mathbf{h}_i, \mathbf{h}_j)}$ or $p_\omega(\mathbf{h}_i, \mathbf{h}_j) = ||(\mathbf{h}_i - \mathbf{h}_j)||$.
% Moreover, the triplet margin does not minimize the agreement of the negative pairs $(\mathbf{h}_i, \mathbf{h}_j^{\prime})$, but only makes the agreement of the positive pairs $(\mathbf{h}_i, \mathbf{h}_j)$ greater than that of the negative pairs.

\subsubsection{BYOL Loss.}
This objective is used by BYOL~\cite{BYOL}. It only maximizes the agreement of positive pairs and does not use negative samples. It is defined as:

\begin{equation}
\mathcal{L}_{BYOL}\left(\mathbf{h}_i, \mathbf{h}_j\right)=\mathbb{E}_{{P} \times {P}}\left[2-2 \cdot \frac{\left[p_\psi\left(\mathbf{h}_i\right)\right]^T \mathbf{h}_j}{\left\|p_\psi\left(\mathbf{h}_i\right)\right\|\left\|\mathbf{h}_j\right\|}\right]
\end{equation}
where $\mathbf{h}_i$ and $\mathbf{h}_j$ are sampled from $P$.
$p_\psi(\cdot)$ is an online predictor. 
As it does not use negative samples to prevent collapse, other designs are needed. For example, BYOL~\cite{BYOL} utilizes momentum encoders, stop gradient, etc. 
% In particular, the pretext decoder in this case denotes the mean square error between two instances, which has been expanded in the above equation.

\begin{table*}
\caption{Comparison between different contrastive objectives.}
\label{tab:ob_discussion}
\resizebox{\textwidth}{!}
{
\begin{tabular}{cccccc}
\toprule
\multicolumn{1}{c|}{\multirow{2}{*}{}} & \multicolumn{2}{c|}{Bound} & \multicolumn{2}{c}{Non-bound} \\ \cline{2-5}
\multicolumn{1}{c|}{}                  & \multicolumn{1}{c|}{Jensen-Shannon Estimator} & \multicolumn{1}{c|}{InfoNCE Estimator} & \multicolumn{1}{c|}{Triplet Loss} & \multicolumn{1}{c}{BYOL Loss}\\ 
\midrule
% Theoretical Foundation& \CheckmarkBold& \CheckmarkBold& \XSolidBrush& \XSolidBrush\\ 
% \midrule
Lower-bound MI estimation& \CheckmarkBold& \CheckmarkBold& \XSolidBrush& \XSolidBrush\\ 
% \midrule
% Negative Samples& \CheckmarkBold& \CheckmarkBold & \CheckmarkBold& \XSolidBrush\\
% \midrule
Batch-Size Independence& \CheckmarkBold& \XSolidBrush& \CheckmarkBold& \CheckmarkBold\\
Uniformity& \CheckmarkBold& \CheckmarkBold& \CheckmarkBold& \XSolidBrush\\
Low Variance& \XSolidBrush& \CheckmarkBold& N.A.& N.A.\\
\bottomrule
\end{tabular}
}
\end{table*}

\subsection{Discussion}
Table.\ref{tab:ob_discussion} shows the comparison between different contrastive objectives.
Among all the contrastive objectives, InfoNCE is the most widely used due to its good performance. Moreover, both InfoNCE and JS estimate MI based on lower-bound, and \citet{low_variance} demonstrates that InfoNCE has lower variance of the estimated MI than JS.
However, InfoNCE requires a large number of negative samples and thus a large batch size during training. This leads to high computational and time complexity. 
In contrast, JS can achieve better performance when the batch size is small. 

Triplet loss and BYOL loss are also independent of the large batch size. However, they lack theoretical support, i.e. there is no theory that proves that maximizing them will achieve the goal of maximizing mutual information. Moreover, triplet loss just makes the agreement of negative pairs smaller than that of positive pairs. 
Therefore, selecting informative positive/negative samples that are difficult to discriminate can lead to better performance, whereas using random or easy samples leads to poor performance. 
% Therefore, it is more suitable for cases where positive and negative pairs should not be absolutely discriminated.
Additionally, triplet loss can be sensitive to the choice of margin value, hence it requires careful adjustment.

BYOL loss is the most efficient since it does not require negative samples. 
However, BYOL loss does not contain the uniformity proposed by~\citet{uniform_alignment}, which suggests that normalized representations should be uniformly distributed over the unit hypersphere.
% Moreover, ~\cite{uniform_alignment} proposes two key properties of contrastive learning: alignment and uniformity. Alignment 
% refers to that similar samples (positive pairs) should have similar. 
% Uniformity refers to that normalized representations should be uniformly distributed over the unit hypersphere.
Hence, it easily encounters the problem of collapse. Therefore, if BYOL loss is used, additional design is usually required to prevent it.
% todo: c
% Moreover, as many semantically similar instances are mixed with the unrelated instances and are fed to the loss as false negative samples, which will impair recommendation performance.

\section{Open Issues and Future Directions}\label{sec:future}
While contrastive learning-based recommendation methods have achieved great success, there are still some open issues. In this section, we discuss these issues and outline some potential future research directions.

\subsection{View Generation}
View generation is a key component of CL-based methods. However, unlike in computer vision, where various data augmentation methods (e.g., resize, rotation, color distortion, etc.) are available, the way of generating views for CL-based recommendation is still not well explored. Specifically, most existing CL-based recommendation methods are limited to randomly removing some interactions or disrupting the order of the interaction sequence.
Moreover, these methods are often based on intuitive designs and may not be applicable to downstream recommendation tasks~\cite{SimGCL}. 
Therefore, designing more effective view generation strategies is a promising future direction.
Generally, the view generation strategies need to have the following properties: (1) Adaptability, the generated views should be adaptive to different tasks, as different tasks may use different types of data and require different information. (2) Efficiency, view generation strategies should not have high computational or time complexity. 
Moreover, dynamically updating the augmentation strategy during training is also a promising direction.
% Moreover, how to obtain the optimal views for a specific recommendation downstream task is a promising but still unexplored direction.

\subsection{Pretext Task}
By solving pretext tasks, the model acquires the knowledge from data for downstream tasks. Therefore, extracting useful knowledge is an important issue. For example, CGI~\cite{CGI_ib} proposes an information bottleneck-based method that enables representations to capture the minimum sufficient information for the recommended task. However, it is designed for graph-based recommendation and is difficult to apply to other recommendation tasks. Moreover, as different tasks can capture different information, learning with multiple different pretext tasks can further improve recommendation performance. It is also worthwhile to further investigate how to adaptively combine different pretext tasks for the specific recommended tasks.

In contrastive pretext tasks, negative samples are essential, but obtaining informative negative samples is challenging. The commonly used uniform sampling strategy, which obtains negative samples by random sampling, suffers from false negatives. Besides, easy negative samples may degrade the performance of contrastive learning as they provide little information. Therefore, effective negative sampling strategies deserve further investigation. Some works~\cite{chuang2020debiased, hard_neg_mixing} explore this problem in computer vision. However, these methods are specifically designed for image data and are difficult to apply to recommendation methods. Moreover, since current methods require a large number of negative samples, efficient negative sample strategies also need to be explored.


\subsection{Contrastive Objective}
Most CL-based recommendation methods use InfoNCE as their objective function due to its simplicity and effectiveness. Although great success has been achieved, two issues need further exploration.
% First, better mutual information measurement. 
First, the measurement of mutual information in InfoNCE is based on KL divergence. Therefore, it suffers from problems stemming from KL divergence (e.g., asymmetrical estimation and unstable training). Hence, the better mutual information measurement is required but a few works~\cite{w_mutual} investigate this issue. 
% MStein
~\citet{w_mutual} propose the Wasserstein discrepancy measurement based on the 2-Wasserstein distance to measure mutual information. In addition, it has only been applied to sequential recommendation, and its applicability to other recommendation tasks needs to be further explored.
% Second, alternative measures of the agreement. 
Second, current methods use mutual information to measure the agreement. However, mutual information has several shortcomings. Besides hard to estimate, mutual information can also lead to suboptimal representations~\cite{other_information}. Therefore, exploring alternative measures of the agreement, such as $\mathcal{V}$-information proposed by \citet{v_information}, is a promising direction.
% that extend classic mutual information.

\subsection{Misc.}
\subsubsection{Meeting Real-World Recommendation}
% online conversation multi-modal 
% Although CL-based recommendation  have
% achieved good performance, the training process is timeconsuming. Thus such models can be regarded as static preference recommendation. 
Most existing CL-based recommendation models are trained offline. However, in real-world recommendation scenarios, such as online shopping and news recommendations, large-scale interaction data are continuously generated and user preferences are dynamic. Offline trained models may suffer from the problem of information asymmetry as they rely only on historical user interaction data to make recommendations. Hence, exploring online learning strategies in CL-based recommendation to quickly capture dynamic preference trends would be a potential direction. Moreover, conversational recommendation~\cite{sun2018conversational,zhou2022c2} is also proposed to address the information asymmetry. Specifically, the model makes recommendations based on the multi-turn interaction (e.g., dialogues) with users. By leveraging real-time user feedback, the users' current preferences can be modeled. Combining contrastive learning with conversational recommendation methods would be an interesting direction to explore.

Additionally, in recommender systems, data is often multi-modal, including video, text, images, etc. These modalities contain rich information and are useful for improving recommendation performance, particularly when interaction data is sparse. Therefore, deriving useful knowledge from multi-modal data through contrastive learning is a promising direction. However, only a limited number of studies~\cite{PAMD,MMSSL,SLMRec} have explored this area.


% In existing recommender systems, there may exist the issue of information asymmetry that the system can only estimate users’ preferences based on their historically collected behavior data. To address it, recently, conversational (interactive) recommendation researches, propose the new paradigm the user can interact with the system in conversations, and then new data can be dynamically collected. Specifically, users can chat with the system to explicitly convey their consumption demands or offer positive/negative feedback on the recommended items. As future work, the advances of contrastive learning can be combined with preference learning in the conversational recommendation.
% todo:加几个multi-modal的论文
\subsubsection{Learning with Advanced Techniques}
% meta-learning, semi-supervised learning curriculum AutoML
% todo:cite SEPT COTREC co- tri-traning.
With the rapid development of deep learning, many advanced techniques can be used to improve the performance of CL-based recommendation. For example, semi-supervised learning can be utilized to obtain more supervision signals. In specific, SEPT\cite{SEPT} and COTREC~\cite{COTREC} use tri-training and co-training to acquire more informative samples, respectively. CML~\cite{weiContrastiveMetaLearning2022} unifies contrastive learning and meta-learning, by capturing meta-knowledge through contrastive learning. CCL~\cite{CCL} incorporate the curriculum learning~\cite{bengio2009curriculum} into contrastive learning. Despite the success achieved by these works, these are only preliminary explorations, and there are still many new technologies that can be utilized for CL-based recommendation. For example, there are various choices for view generation strategies, contrastive tasks, encoders, etc. in CL-based methods. Therefore, it is a promising direction to use Automated Machine Learning (AutoML)~\cite{automl} to automatically select the appropriate method to reduce human effort.


\section{Conclusion}~\label{sec:conclusion}
% Using contrastive self-supervised learning in recommender systems has attracted increasing interest in recent years. 
In this survey, we present a comprehensive and systematic review of recent works in contrastive self-supervised learning-based recommendation. We first propose a unified framework and then introduce a taxonomy based on its key components, which include view generation strategy, pretext task, and contrastive objective. For each component, we provide detailed descriptions and discussions to guide the choice of the appropriate method. Finally, we discuss open issues and promising research directions for contrastive self-supervised learning-based recommendation in the future. We hope that this survey can provide both junior and experienced researchers with a comprehensive understanding of contrastive self-supervised learning-based recommendation and inspire future research in this area.

% Finally, we discuss some promising potential research directions for further research on cross-domain recommendation. We hope that this survey can provide both newcomers and experts of cross-domain recommendation with a comprehensive understanding of the problem definition of this field, clarify existing works clearly, and shed some light on future studies

% Finally, we suggest open challenges and promising research directions of graph SSL in the future

% For each category, we briefly clarify the main issues, and detail the corresponding strategies adopted by the representative models. We also discuss the advantages and limitations of the existing strategies. Furthermore, we suggest several promising directions for future researches.


% We demonstrate that any existing method can be realized by determinating its view generation, pretext task and contrastive objective. For each 

% comprehensively reviews the existing selfsupervised representation learning approaches in natural language processing (NLP), computer vision (CV), graph learning, and beyond

% timely and systematical review of the research efforts on SSR
% e present a unified framework and further provide a systematic taxonomy that groups graph SSL into four categories: generation-based, auxiliary property-based, contrast-based, and hybrid approaches.

% For each category, its concept and formulation, the involved methods, and the pros and cons are introduced in turn.
%%
%% The acknowledgments section is defined using the "acks" environment
%% (and NOT an unnumbered section). This ensures the proper
%% identification of the section in the article metadata, and the
%% consistent spelling of the heading.
% \begin{acks}
% To Robert, for the bagels and explaining CMYK and color spaces.
% \end{acks}

%%
%% The next two lines define the bibliography style to be used, and
%% the bibliography file.
\bibliographystyle{ACM-Reference-Format}
\bibliography{sample-base, contrastive, generative, predictive}

\end{document}
\endinput
%%
%% End of file `sample-acmsmall.tex'.
