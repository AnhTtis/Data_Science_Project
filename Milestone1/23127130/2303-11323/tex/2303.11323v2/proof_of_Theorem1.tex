 Therefore, we can focus on each difference term $D_l^n$ and omit the feature and layer indices to simplify the notations. 
 We can write the convolution operation in the spectral domain as
 \begin{align}
    & \nonumber\|\bbh(\Delta_n)\samp_n^{\mathcal{X}} \F - \samp_n^{\mathcal{X}}\bbh(\Delta) \F\| \nonumber \\
   & = \Bigg\| \sum_{i=1}^n \hat{h}(\lambda_i^n) \langle \samp_n^{\mathcal{X}}\F,\bphi_i^n \rangle_{\tm_n}\bphi_i^n \nonumber \\
    & \qquad \qquad \qquad \quad- \sum_{i=1}^\infty \hat{h}(\lambda_i)\langle \F,\bphi_i\rangle_{\tm} \samp_n^{\mathcal{X}} \bphi_i  \Bigg\| 
   \label{eqn:conv-1}
 \end{align}
By adding and subtracting $\sum_{i=1}^{n}\hat{h}(\lambda_i) \langle \samp_n^{\mathcal{X}}\F,\bphi_i^n \rangle_{\tm_n}\bphi_i^n$, by coupling the terms with the same index and using the triangle inequality, we can then write
\begin{align}
        & \nonumber {\Bigg\| \sum_{i=1}^n \hat{h}(\lambda_i^n) \langle \samp_n^{\mathcal{X}}\F,\bphi_i^n \rangle_{\tm_n}\bphi_i^n \nonumber - \sum_{i=1}^\infty \hat{h}(\lambda_i)\langle \F,\bphi_i\rangle_{\tm} \samp_n^{\mathcal{X}} \bphi_i  \Bigg\| }
        %\left\| \sum_{i=1}^n \Big(\hat{h}(\lambda_i)\langle  \F, \bphi_i\rangle_{\ccalT\ccalM}\samp_n^\ccalX \bphi_i  -\hat{h}(\lambda_i^n)\langle \samp_n^\ccalX\F, \bphi_i^n \rangle_{\ccalT\ccalM_n} \bphi_i^n\Big) \right\|
     \\ 
     &\nonumber \leq  \left\| \sum_{i=1}^n \left(\hat{h}(\lambda_i^n)-\hat {h}(\lambda_i) \right) \langle \samp_n^{\mathcal{X}}\F,\bphi_i^n \rangle_{\tm_n}\bphi_i^n \right\| [T1]\\
     &  +\left\| \sum_{i=1}^n \hat{h}(\lambda_i)\left( \langle \samp_n^{\mathcal{X}}\F,\bphi_i^n \rangle_{\tm_n}\bphi_i^n - \langle \F,\bphi_i\rangle_{\tm} \samp_n^{\mathcal{X}} \bphi_i  \right)  \right\| [T2] \nonumber\\
     & + \Bigg\| \sum_{p=n+1}^\infty \hat{h}(\lambda_p)\langle  \F, \bphi_p\rangle_{\ccalT\ccalM}\samp_n^\ccalX \bphi_p \Bigg\| [T3] \label{eqn:decompose}
\end{align}
We now proceed to prove that $[T1]$ converges to zero in probability as $n$ increases. Fixed a $M_{[T1]} \in \mathbb{N}$, we can always rewrite [T1] as
\begin{align}\label{eqn:t1_M}
  [T1] &= \Bigg\| \sum_{i=1}^{\min\{n,M_{[T1]}\}} \left(\hat{h}(\lambda_i^n)-\hat {h}(\lambda_i) \right) \langle \samp_n^{\mathcal{X}}\F,\bphi_i^n \rangle_{\tm_n}\bphi_i^n \nonumber \\
  & + \sum_{i=M_{[T1]}+1}^{n} \left(\hat{h}(\lambda_i^n)-\hat {h}(\lambda_i) \right) \langle \samp_n^{\mathcal{X}}\F,\bphi_i^n \rangle_{\tm_n}\bphi_i^n \Bigg\|
\end{align}
Please notice that,  when $n<M_{[T1]}$, the last sum is an empty sum. By using the triangle inequality, the orthonormality of the $\bphi_i^n$, the Cauchy-Schwartz inequality $|\langle \samp_n^{\mathcal{X}}\F,\bphi_i^n \rangle_{\tm_n}| \leq \|\samp_n^{\mathcal{X}}\F\|$, and the finiteness of $\|\samp_n^{\mathcal{X}}\F\|$, we can further bound the RHS of \eqref{eqn:t1_M}, obtaining
\begin{align}\label{eqn:t1_M_2}
  [T1] &\leq C_{[T1]}\sum_{i=1}^{\min\{n,M_{[T1]}\}} |\hat{h}(\lambda_i^n)-\hat {h}(\lambda_i)| \nonumber \\
  &+ C_{[T1]}\sum_{i=M_{[T1]}+1}^{n} |\hat{h}(\lambda_i^n)-\hat {h}(\lambda_i)|,
\end{align}
for some constant $C_{[T1]}>0$. At this point, by using the fact that $|a-b| \leq |b|$ and the Lipschitz continuity of $\hat{h}(\cdot)$ (\textbf{A1}), we can further bound the RHS of \eqref{eqn:t1_M_2} as
\begin{align}\label{eqn:t1_M_3}
  [T1] &\leq \underbrace{C_{[T1]}\sum_{i=1}^{\min\{n,M_{[T1]}\}} |\lambda_i^n-\lambda_i|}_{[T1.1]} + \underbrace{C_{[T1]}\sum_{i=M_{[T1]}+1}^{\infty} |\hat{h}(\lambda_i)|}_{[T1.2]}
  \end{align}
 It is clear that we can make  $[T1.2]$ in \eqref{eqn:t1_M_3} arbitrarily small by increasing $M_{[T1]}$ since it is the reminder of a convergent series with positive summands ($\textbf{A2}$). Therefore, for all $\gamma_{[T1]}>0$, we can always choose an $M_{[T1]}$ such that $[T1.2]$ is smaller than $\gamma_{[T1]}/2C$. Fixed $M_{[T1]}$, we can further bound $[T1.1]$ using the spectral convergence result in \eqref{eqn:convergence_spectrum}. In particular, using the definition of limit in probability, letting $0<\gamma_i \leq \gamma_{[T1]}/2CM$, for all $\delta_i>0$, there exist $N_i$ such that for all $n \geq N_i$, it holds
 \begin{gather}
 \label{eqn:eigenvalue}   \mathbb{P}(|\lambda_i^n-\lambda_i|\leq \gamma_i)\geq 1-\delta_i.
 \end{gather}
Therefore, for all $\gamma_{[T1]}>0$ and for all $n \geq \max_i N_i$, it holds
 \begin{align}\label{T11_1}
     [T1.1] \leq C_{[T1]}\sum_{i=1}^{\min\{n,M_{[T1]}\}}\gamma_i\leq \gamma_{[T1]}/2
 \end{align}
 with probability at least $\prod_{i =1}^{\min\{n,M_{[T1]}\}}(1-\delta_i) := 1-\delta_{[T1]}$. This allows us to state that for all $\gamma_{[T1]}>0$, for all $\delta_{[T1]} > 0$, there exist an $N_{[T1]}$ such that, for all $n>N_{[T1]}$, we have
  \begin{gather}
 \label{eqn:conv_T1}   \mathbb{P}([T1]\leq \gamma_{[T1]})\geq 1-\delta_{[T1]},
 \end{gather}
i.e. $[T1]$ converges in probability to zero.
We now proceed to show that $[T2]$ in \eqref{eqn:decompose} converges to zero in probability as $n$ increases. By adding and subtracting $\sum_{i =1}^n\hat{h}(\lambda_i) \langle \samp_n^{\mathcal{X}}\F,\bphi_i^n \rangle_{\tm_n}\samp_n^{\mathcal{X}}\bphi_i$, and by using the triangle inequality, we can write \begin{align}\label{eqn:T2_decompose}
   &\nonumber \left\| \sum_{i=1}^n  \hat{h}(\lambda_i)( \langle \samp_n^{\mathcal{X}} \F,\bphi_i^n \rangle_{\tm_n}\bphi_i^n -  \langle \F,\bphi_i \rangle_{\tm} \samp_n^{\mathcal{X}} \bphi_i )\right\|\\
   &\leq \nonumber \Bigg\|  \sum_{i=1}^n \hat{h}(\lambda_i)\Big(\langle \samp_n^{\mathcal{X}} \F,\bphi_i^n\rangle_{\tm_n}\bphi_i^n  \nonumber \\
   &\qquad \qquad \qquad \qquad - \langle \samp_n^{\mathcal{X}} \F,\bphi_i^n \rangle_{\tm_n} \samp_n^{\mathcal{X}}\bphi_i \Big)\Bigg\| [T2.1]\nonumber \\
   &+ \Bigg\| \sum_{i=1}^n  \hat{h} (\lambda_i)\Big(\langle \samp_n^{\mathcal{X}} \F,\bphi_i^n\rangle_{\tm_n} \samp_n^{\mathcal{X}}\bphi_i \nonumber \\
   &\qquad \qquad\qquad \qquad -\langle \F,\bphi_i\rangle_\tm \samp_n^{\mathcal{X}}\bphi_i \Big) \Bigg\| [T2.2]
\end{align}
We can use now the same approach of $[T1]$. In particular, fixed a $M_{[T2.1]} \in \mathbb{N}$, we can always rewrite $[T2.1]$ and then bound it using the triangle inequality as
\begin{align}\label{eqn:t21_M}
  [T2.1] &= \Bigg\| \sum_{i=1}^{\min\{n,M_{[T2.1]}\}} \hat{h}(\lambda_i)\Big(\langle \samp_n^{\mathcal{X}} \F,\bphi_i^n\rangle_{\tm_n}\bphi_i^n  \nonumber \\
   &\qquad \qquad \qquad \qquad - \langle \samp_n^{\mathcal{X}} \F,\bphi_i^n \rangle_{\tm_n} \samp_n^{\mathcal{X}}\bphi_i \Big) \nonumber \\
  & + \sum_{i=M_{[T2.1]}+1}^{n} \hat{h}(\lambda_i)\Big(\langle \samp_n^{\mathcal{X}} \F,\bphi_i^n\rangle_{\tm_n}\bphi_i^n  \nonumber \\
   &\qquad \qquad \qquad \qquad - \langle \samp_n^{\mathcal{X}} \F,\bphi_i^n \rangle_{\tm_n} \samp_n^{\mathcal{X}}\bphi_i \Big)\Bigg\| \nonumber \\
   &\leq \Bigg\| \sum_{i=1}^{\min\{n,M_{[T2.1]}\}} \hat{h}(\lambda_i)\Big(\langle \samp_n^{\mathcal{X}} \F,\bphi_i^n\rangle_{\tm_n}\bphi_i^n  \nonumber \\
   &\qquad \qquad \qquad \qquad - \langle \samp_n^{\mathcal{X}} \F,\bphi_i^n \rangle_{\tm_n} \samp_n^{\mathcal{X}}\bphi_i \Big)\Bigg\| \nonumber \\
   &+ \Bigg \|\sum_{i=M_{[T2.1]}+1}^{n} \hat{h}(\lambda_i)\Big(\langle \samp_n^{\mathcal{X}} \F,\bphi_i^n\rangle_{\tm_n}\bphi_i^n  \nonumber \\
   &\qquad \qquad \qquad \qquad - \langle \samp_n^{\mathcal{X}} \F,\bphi_i^n \rangle_{\tm_n} \samp_n^{\mathcal{X}}\bphi_i \Big)\Bigg\| 
\end{align}
We can now further bound the RHS of \eqref{eqn:t21_M} by using the triangle inequality, the Cauchy-Schwarz inequality $|\langle \samp_n^{\mathcal{X}}\F,\bphi_i^n \rangle_{\tm_n}| \leq \|\samp_n^{\mathcal{X}}\F\|$ , the non-amplifying frequency response (\textbf{A1}, for the first term), the finiteness of $\|\samp_n^{\mathcal{X}}\F\|$, and the finiteness of $\|\bphi_i^n-\samp_n^{\mathcal{X}}\bphi_i\|$ (for the second term) as
\begin{align}\label{eqn:t21_decom}
  [T2.1] &\leq \underbrace{C_{[T2.1]} \sum_{i=1}^{\min\{n,M_{[T2.1]}\}} \|\bphi_i^n-\samp_n^{\mathcal{X}}\bphi_i\|}_{[T2.1.1]} \nonumber \\
  & + \underbrace{C_{[T2.1]} \sum_{i=M_{[T2.1]}+1}^\infty |\hat{h}(\lambda_i)|}_{[T2.1.2]},
\end{align}
for some constant $C_{[T2.1]}>0$. Leveraging the same arguments we used for $[T1.1]$ and $[T1.2]$ in 
\eqref{eqn:t1_M_3} to bound $[T2.1.1]$ and $[T2.1.2]$ in 
\eqref{eqn:t21_decom}, respectively, but using the convergence of the eigenvectors and not of the eigenvalues from \eqref{eqn:convergence_spectrum}, we can state that for all $\gamma_{[T2.1]}>0$, for all $\delta_{[T2.1]} > 0$, there exist an $N_{[T2.1]}$ such that, for all $n>N_{[T2.1]}$, we have
  \begin{gather}
 \label{eqn:conv_T21}   \mathbb{P}([T2.1]\leq \gamma_{[T2.1]})\geq 1-\delta_{[T2.1]},
 \end{gather}
i.e. $[T2.1]$ converges in probability to zero. Following the same procedure we used to obtain the bound in \eqref{eqn:t21_M} for $[T2.1]$, we can obtain the following bound for $[T2.2]$:
\begin{align}\label{eqn:t22_M}
  [T2.2] &\leq \Bigg\| \sum_{i=1}^{\min\{n,M_{[T2.2]}\}}\hat{h} (\lambda_i)\Big(\langle \samp_n^{\mathcal{X}} \F,\bphi_i^n\rangle_{\tm_n} \samp_n^{\mathcal{X}}\bphi_i \nonumber \\
   &\qquad \qquad\qquad \qquad -\langle \F,\bphi_i\rangle_\tm \samp_n^{\mathcal{X}}\bphi_i \Big)\Bigg\| \nonumber \nonumber \\
   &+ \Bigg \|\sum_{i=M_{[T2.2]}+1}^{n} \hat{h} (\lambda_i)\Big(\langle \samp_n^{\mathcal{X}} \F,\bphi_i^n\rangle_{\tm_n} \samp_n^{\mathcal{X}}\bphi_i \nonumber \\
   &\qquad \qquad\qquad \qquad -\langle \F,\bphi_i\rangle_\tm \samp_n^{\mathcal{X}}\bphi_i \Big)\Bigg\| 
\end{align}
We further bound the RHS of \eqref{eqn:t22_M} by using the triangle and Cauchy-Schwarz inequalities, the non-amplifying frequency response (for the first term),  the finiteness of $\|\samp_n^{\mathcal{X}}\F\|$ and $\|\F\|$, and the finiteness of $\|\bphi_i^n-\samp_n^{\mathcal{X}}\bphi_i\|$ (for the second term), as
\begin{align}\label{eqn:t22_decom}
  [T2.2] &\leq \underbrace{C_{[T2.2]} \sum_{i=1}^{\min\{n,M_{[T2.2]}\}} |\langle \samp_n^{\mathcal{X}} \F,\bphi_i^n\rangle_{\tm_n} -\langle \F,\bphi_i\rangle_\tm|}_{[T2.2.1]} \nonumber\\
  & + \underbrace{C_{[T2.2]} \sum_{i=M_{[T2.2]}+1}^\infty |\hat{h}(\lambda_i)|}_{[T2.2.2]},
\end{align}
for some constant $C_{[T2.2]}>0$.  It is trivial, from the weak law of large numbers and from \eqref{emp_metr}-\eqref{emp_metr_versions}, that 
 \begin{equation}\label{eqn:conv_inn_1}
\lim_{n\rightarrow \infty}\left|\langle \samp_n^{\mathcal{X}} \F,\samp_n^{\mathcal{X}}\bphi_i\rangle_{\tm_n}  -\langle \F,\bphi_i \rangle_\tm\right| = 0,
\end{equation}
with the limit taken in probability. By direct substitution and using the distributive law of the dot product, we can write
\begin{align}
    &\left|\langle \samp_n^{\mathcal{X}} \F,\samp_n^{\mathcal{X}}\bphi_i\rangle_{\tm_n} - \langle \samp_n^{\mathcal{X}} \F,\bphi_i^n\rangle_{\tm_n}\right| \nonumber \\
    & = \left| \frac{1}{n}\sum_{i=1}^n \left(\samp_n^{\mathcal{X}} \F(x_i)\dotp\samp_n^{\mathcal{X}}\bphi_i(x_i)-\samp_n^{\mathcal{X}} \F(x_i)\dotp\bphi_i^n(x_i)\right)\right| \nonumber \\
    & = \left|\langle \samp_n^{\mathcal{X}} \F,\samp_n^{\mathcal{X}}\bphi_i-\bphi_i^n\rangle_{\tm_n}\right|  \leq \|\samp_n^{\mathcal{X}} \F\| \|\samp_n^{\mathcal{X}}\bphi_i-\bphi_i^n\|,
\end{align}
where the last inequality is obtained using the Cauchy-Schwartz inequality. Therefore, using again the spectral convergence of eigenvectors from \eqref{eqn:convergence_spectrum}, we can write
\begin{equation}\label{eqn:conv_inn2}
\lim_{n \rightarrow \infty}\left|\langle \samp_n^{\mathcal{X}} \F,\samp_n^{\mathcal{X}}\bphi_i\rangle_{\tm_n} - \langle \samp_n^{\mathcal{X}} \F,\bphi_i^n\rangle_{\tm_n}\right| = 0,
\end{equation}
where the limit is taken in probability.
As a direct consequence of the \eqref{eqn:conv_inn_1} and \eqref{eqn:conv_inn2}, we can directly state that
\begin{equation}\label{eqn:conv_inn_fin}
\lim_{n \rightarrow \infty}\left|\langle \samp_n^{\mathcal{X}} \F,\bphi_i^n\rangle_{\tm_n}-\langle  \F,\bphi_i\rangle_\tm\right| = 0,
\end{equation}
again with the limit in probability. At this point, leveraging the same arguments we used for $[T1.1]$ and $[T1.2]$ in 
\eqref{eqn:t1_M_3} (and for $[T2.1.1]$ and $[T2.1.2]$ in \eqref{eqn:t21_decom})  to bound $[T2.2.1]$ and $[T2.2.2]$ in 
\eqref{eqn:t22_decom}, respectively, but using the convergence of the inner products in \eqref{eqn:conv_inn_fin}, we can state that for all $\gamma_{[T2.2]}>0$, for all $\delta_{[T2.2]} > 0$, there exist an $N_{[T2.2]}$ such that, for all $n>N_{[T2.2]}$:
  \begin{gather}
 \label{eqn:conv_T22}   \mathbb{P}([T2.2]\leq \gamma_{[T2.2]})\geq 1-\delta_{[T2.2]},
 \end{gather}
i.e. $[T2.2]$ converges in probability to zero. As a consequence, we can state that for all $\gamma_{[T2]}>0$, for all $\delta_{[T2]} > 0$, there exist an $N_{[T2]}$ such that, for all $n>N_{[T2]}$, we have
  \begin{gather}
 \label{eqn:conv_T2}   \mathbb{P}([T2]\leq \gamma_{[T2]})\geq 1-\delta_{[T2]},
 \end{gather}
i.e. $[T2]$ converges in probability to zero. We are now missing only the convergence in probability of $[T3]$ from \eqref{eqn:decompose}. However, $[T3]$ is again the reminder of a convergent series with positive summands ($\textbf{A2}$), implying that it deterministically goes to zero as $n$ increases. Therefore, for all $\gamma_{[T3]}>0$, there exist an $N_{[T3]}$ such that, for all $n>N_{[T3]}$, we have
  \begin{gather}
 \label{eqn:conv_T3}   [T3]\leq \gamma_{[T3]}
 \end{gather}

As a direct consequence of
\eqref{eqn:conv_T1}-\eqref{eqn:conv_T2}-\eqref{eqn:conv_T3}, we can state that for all $\gamma > 0$, for all $\delta>0$, there exist a $N$ such that, for all $n>N$, we have
  \begin{gather}
 \label{eqn:conv_T}   \mathbb{P}([T1]+[T2]+[T3]\leq \gamma)\geq 1-\delta,
 \end{gather}
Combining \eqref{eqn:conv_T} with \eqref{eqn:decompose}, we can finally state that
\begin{equation}
    \lim_{n \rightarrow \infty }D_l^n=\lim_{n \rightarrow \infty }\|\bbh(\Delta_n)\samp_n^{\mathcal{X}} \F - \samp_n^{\mathcal{X}}\bbh(\Delta) \F\| = 0,
\label{eqn:final_conv}
\end{equation}
where the limit is taken in probability. The proof is concluded by combining \eqref{eqn:final_conv} and \eqref{eqn:separate_layers}.



