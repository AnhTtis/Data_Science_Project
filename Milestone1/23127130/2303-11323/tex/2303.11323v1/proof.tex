% Template for ICIP-2022 paper; to be used with:
%          spconf.sty  - ICASSP/ICIP LaTeX style file, and
%          IEEEbib.bst - IEEE bibliography style file.
% --------------------------------------------------------------------------
\documentclass[journal]{IEEEtran}
\usepackage[utf8]{inputenc}
\usepackage[pdftex]{graphicx}
%\usepackage{graphicx}
\usepackage{epstopdf}
% \usepackage{titling}
\usepackage[utf8]{inputenc} % allow utf-8 input
\usepackage[T1]{fontenc}    % use 8-bit T1 fonts
\usepackage{hyperref}       % hyperlinks
\usepackage{url}            % simple URL typesetting
\usepackage{booktabs}       % professional-quality tables
\usepackage{amsfonts}       % blackboard math symbols
\usepackage{nicefrac}       % compact symbols for 1/2, etc.
\usepackage{microtype}      % microtypography
\usepackage{amsmath,graphicx}
\usepackage{amsbsy}
\usepackage{epsfig}
\usepackage{float}
\usepackage{graphicx}
\usepackage{color}
\usepackage{url}
\usepackage{cite}
\usepackage{amssymb}
\usepackage{pgf, tikz, pgfplots, epstopdf}
\usepackage{graphicx}
\usepackage{lipsum}
\usepackage{float}
\usepackage{cuted}
\usepackage{balance}
\usepackage{booktabs}
\usepackage{verbatim}
\usepackage{hyperref}
\usepackage{calc}  
\usepackage{mathtools}
\usepackage{enumitem}  
\usepackage{multirow}
%\usepackage{algorithmic}
\newfloat{algorithm}{tbp}{loa}
\providecommand{\algorithmname}{Algorithm}
\usepackage{algorithm}
\usepackage[noend]{algpseudocode}
% #1 = text to left, #2 = text to right
%\usepackage{algorithm2e}
\usepackage{subcaption}
\input{mysymbol.sty}
% Example definitions.
% --------------------

\makeatletter
\newcommand*\dotp{\mathpalette\dotp@{.5}}
\newcommand*\dotp@[2]{\mathbin{\vcenter{\hbox{\scalebox{#2}{$\m@th#1\bullet$}}}}}
\newcommand\svdeq{\stackrel{\mathclap{\scriptsize\mbox{ SVD }}}{=}}

\makeatother
\def\x{{\mathbf x}}
\def\f{{\mathbf f}}
\def\g{{\mathbf g}}
\def\z{{\mathbf z}}
\def\F{{\mathbf F}}
\def\U{{\mathbf U}}
\def\G{{\mathbf G}}
\def\D{{\mathbf D}}
\def\S{{\mathbf S}}
\def\B{{\mathbf B}}
\def\R{{\mathbf R}}
\def\h{{\mathbf h}}
\def\L{{\cal L}}
\def\tmx{{\mathcal{T}_x\mathcal{M}}}
\def\tmxrp{{\mathcal{T}_x\mathbb{R}^p}}
\def\tmxrd{{\mathcal{T}_x\mathbb{R}^d}}
\def\tmy{{\mathcal{T}_y\mathcal{M}}}
\def\tm{{\mathcal{T}\mathcal{M}}}
\def\M{{\mathcal{M}}}
\def\i{\iota}
\def\rp{{\mathbb{R}^p}}
% \def\inc{{\underline{\triangleleft}}}
\makeatletter
\newcommand{\inc}{%
  \mathrel{\mathpalette\inc@\relax}%
}
\newcommand{\inc@}[2]{%
  \sbox\z@{$#1\lhd$}%
  \sbox\tw@{$#1\leqslant$}%
  \dimen@=\ht\tw@
  \advance\dimen@-\ht\z@
  \ifx#1\displaystyle
    \advance\dimen@ .2pt
  \else
    \ifx#1\textstyle
      \advance\dimen@ .2pt
    \fi
  \fi
  \ooalign{\raisebox{\dimen@}{$\m@th#1\lhd$}\cr$\m@th#1\leqslant$\cr}%
}
\makeatother
\makeatletter
\newcommand{\coinc}{%
  \mathrel{\mathpalette\coinc@\relax}%
}
\newcommand{\coinc@}[2]{%
  \sbox\z@{$#1\rhd$}%
  \sbox\tw@{$#1\geqslant$}%
  \dimen@=\ht\tw@
  \advance\dimen@-\ht\z@
  \ifx#1\displaystyle
    \advance\dimen@ .2pt
  \else
    \ifx#1\textstyle
      \advance\dimen@ .2pt
    \fi
  \fi
  \ooalign{\raisebox{\dimen@}{$\m@th#1\rhd$}\cr$\m@th#1\geqslant$\cr}%
}
\makeatother

\def\ltm{{\mathcal{L}^2(\tm)}}
\def\ltmn{{\mathcal{L}^2(\tm_n)}}
\def\lmn{{\mathcal{L}^2(\mathcal{M}_n)}}
\def\lx{{\mathcal{L}^2(\mathcal{X})}}
\def\pxy{{\mathcal{P}_x^y}}
\def \Phii{{\boldsymbol{\phi}_i}}
\def \th{{\widetilde{h}}}
\def \bPsi{{\boldsymbol{\Psi}}}
\def\Oi{{\mathbf{O}_i}}
\def\Oj{{\mathbf{O}_j}}
\def\Oij{{\mathbf{O}_{i,j}}}
\def\hd{{\hat{d}}}
\def \samp{\boldsymbol{\Omega}}
\def \tGamma{\widetilde{\Gamma}}
\def \eigGammai{\widetilde{\boldsymbol{\phi}}^n_{i}}
\def \eivGammai{\widetilde{\lambda}^n_{i}}
\floatstyle{ruled}
\newfloat{algorithm}{tbp}{loa}
\providecommand{\algorithmname}{Algorithm}
\floatname{algorithm}{\protect\algorithmname}

%THEOREM ENVIRONMENTS

\usepackage{amsthm}
\newtheoremstyle{claudio}% name of the style to be used
  {-0.1em}% measure of space to leave above the theorem. E.g.: 3pt
  {-0.1em}% measure of space to leave below the theorem. E.g.: 3pt
  {}% name of font to use in the body of the theorem
  {}% measure of space to indent
  {\itshape\bfseries}% name of head font
  {.}% punctuation between head and body
  { }% space after theorem head; " " = normal interword space
  {\thmname{#1}\thmnumber{ #2 }\thmnote{(#3)}}% Manually specify head

\theoremstyle{claudio}
\newtheorem{theorem}{Theorem}
\newtheorem{definition}{Definition}
\newtheorem{proposition}{Proposition}
\newtheorem{remark}{Remark}
\begin{document}

\twocolumn[%
%  Title and authors
   \begin{center}
     {\huge \textit{Supplemental Materials for the paper:} \\ "Tangent Bundle Convolutional Learning: \\ from Manifolds to Cellular Sheaves and Back" }\\
     \end{center}\vspace{0.5cm}
]
% \centering
% \textbf{\large Supplemental Materials}
\vspace{1cm}
% \section*{Supplementary Material}
\setcounter{subsection}{0}
\subsection{Proof of Theorem 1}
\noindent\textbf{\textit{Theorem 1.}} Let $\mathcal{X}=\{x_1,\dots,x_n\}\subset \mathbb{R}^p$ be a set of $n$ i.i.d. sampled points from measure $\mu$ over $\M \subset \mathbb{R}^p$ and $\F$ a tangent bundle signal. Let $\tm_n$ be a cellular sheaf built from $\mathcal{X}$ as explained above, with $\epsilon = n^{-2/(\hd+4)}$. Let  $\bPsi_u\big(\mathcal{H}, \cdot, \cdot \big)$ be the $u-th$ output of a neural network with $L$ layers parameterized
by the operator $\Delta$ of  $\tm$  or by the discrete operator $\Delta_n$ of  $\tm_n$. If:
\begin{itemize}
    \item $\Delta$ has an accumulation point at $-\infty$;
    \item  the filters in $\mathcal{H}$ are $\alpha-$FDT filters
    \item the frequency response of filters in $\mathcal{H}$ are non-amplifying Lipschitz continuous;
    \item $\widetilde{\sigma}$ from Definition 4 is point-wise normalized Lipschitz continuous,
\end{itemize}
then it holds for each $u = 1, 2, \dots, F_L$ that:
\begin{equation}
\label{convergence}
\lim_{n \rightarrow \infty} ||\bPsi_u\big(\mathcal{H}, \Delta_n, \samp_n^{\mathcal{X}}\F\big) - \samp_n^{\mathcal{X}}\bPsi_u\big(\mathcal{H}, \Delta,\F\big)||_{\tm_n} = 0,
\end{equation}
with the limit taken in probability.

\noindent\textbf{\textit{Proof.}}  We define an inner product for sheaf signals $\mathbf{f}$ and $\mathbf{u}$ on a general cellular sheaf $\tm_n$ as
\begin{align}
\label{emp_metr_sheaf}
\langle \mathbf{f}, \mathbf{u} \rangle_{\tm_n}   & = \frac{1}{n}\sum_{i = 1}^n  \f_{n}(x_i) \dotp \mathbf{u}_{n}(x_i) ,
\end{align}
and the induced norm $||\mathbf{f}||^2_{\tm_n} = \langle \mathbf{f}, \mathbf{f} \rangle_{\tm_n}$.
Under the assumption that the points in $\mathcal{X}$ are sampled i.i.d. from the uniform probability measure $\mu$ given by the induced metric on $\M$ and that $\tm_n$ is built as in Section 5, the inner product in \eqref{emp_metr_sheaf} is equivalent to the following inner product for tangent bundle signals $\F$ and $\U$:
\begin{align}
\label{emp_metr}
\langle \F, \U \rangle_{\tm_n}   &= \int_{\M} \langle \F(x), \U(x) \rangle_{\tmx} \textrm{d}\mu_n(x) \nonumber \\
&= \frac{1}{n}\sum_{i = 1}^n \langle \F(x_i), \mathbf{U}(x_i) \rangle_{\mathcal{T}_{x_i}\M},
\end{align}
and the induced norm $||\F||^2_{\tm_n} = \langle \F, \F \rangle_{\tm_n}$, where $\mu_n = \frac{1}{n}\sum_{i=1}^n\delta_{x_i}$ is the empirical measure corresponding to $\mu$. Indeed, from Definition 2 and due to the orthogonality of the transformations $\Oi$ in Section 5, \eqref{emp_metr} can be  rewritten as
\begin{align}
\label{emp_metr_versions}
\langle \F, \U \rangle_{\tm_n}&=\frac{1}{n}\sum_{i = 1}^n \langle \F(x_i), \mathbf{U}(x_i) \rangle_{\mathcal{T}_{x_i}\M} \nonumber \\
&= \frac{1}{n}\sum_{i = 1}^n  d\i\F(x_i) \dotp   d\i\mathbf{U}(x_i) \nonumber \\
&= \frac{1}{n}\sum_{i = 1}^n \Oi^T d\i\F(x_i) \dotp  \Oi^T d\i\mathbf{U}(x_i) \nonumber \\
&= \frac{1}{n}\sum_{i = 1}^n \f_{n}(x_i) \dotp \mathbf{u}_{n}(x_i)  = \langle \mathbf{f}_n, \mathbf{u}_n \rangle_{\tm_n}
\end{align}
where $\f_n = \samp_n^{\mathcal{X}}\F$ and $\mathbf{u}_n = \samp_n^{\mathcal{X}}\U$, respectively. We denote with $\ltmn$ the Hilbert Space of finite energy tangent bundle signals w.r.t. the empirical measure $\mu_n$ (or, equivalently, the Hilbert Space of finite energy sheaf signals w.r.t the norm induced by \eqref{emp_metr_sheaf}). In the following, we will denote the norm $||\cdot||_{\tm_n}$ with $||\cdot||$  when there is no risk of confusion. In \cite{singer2013spectral}, the spectral convergence of the constructed Sheaf Laplacian in (20) based on the discretized manifold to the Connection Laplacian of the underlying manifold has been proved, and we will exploit that result for proving the following proposition.

\noindent\textbf{\textit{Proposition 3. (Consequence of Theorem 6.3 \cite{singer2013spectral})}}
 Let $\mathcal{X}=\{x_1,\dots,x_n\}\subset \mathbb{R}^p$ be a set of $n$ i.i.d. sampled points from measure $\mu$ over $\M \subset \mathbb{R}^p$. Let $\tm_n$ be a cellular sheaf built from $\mathcal{X}$ as explained in Section 5, with $\epsilon = n^{-2/(\hd+4)}$. Let $\Delta_n$ be the Sheaf Laplacian of $\tm_n$ and $\Delta$ be the Connection Laplacian operator of $\M$. Let $\lambda_{i}^n$ be the $i$-th eigenvalue of $\Delta_n$ and $\Phii^n$ the corresponding eigenvector. Let $\lambda_i$ be the $i$-th eigenvalue of $\Delta$ and $\Phii$  the corresponding eigenvector field of $\Delta$, respectively. Then it holds:
\begin{equation}
\label{eqn:convergence_spectrum}
    \lim_{n\rightarrow \infty } \lambda_i^n = \lambda_i, \quad\lim_{n\rightarrow \infty} \|\Phii^{n} -  \samp_n^{\mathcal{X}}\Phii\|_{\tm_n}=0,
\end{equation}
where the limits are taken in probability.

\noindent\textbf{\textit{Proof.}} These proposition is a consequence of Theorem 6.3 in \cite{chung1997spectral}. Indeed, we  rely on the operator introduced in Definition 6.1 in \cite{singer2013spectral} with $\alpha=1$ and $h_n = n^{-2/(\hd+4)}$ (our $\epsilon$), here denoted as $\Gamma:\ltm \rightarrow \ltm$, and on the operator $\widetilde{\Gamma} = \epsilon^{-1}\big(\Gamma - \textrm{id}\big)$, where $\textrm{id}$ is the identity mapping. It is straightforward to check that:
\begin{equation}
    \label{omega_delta_equiv}
    \widetilde{\Gamma}\F(x_j) =  d\i^{-1}\Oj\big(\Delta_n\samp_n^{\mathcal{X}}\F\big)(x_j),
\end{equation}
for $j = 1,\dots,n$. We now show that the eigenvectors sampled on $\mathcal{X}$ and eigenvalues of $\widetilde{\Gamma}$  correspond to the eigenvectors and eigenvalues of $\Delta_n$. Let us denote the the $i-th$ eigenvector and eigenvalue of $\widetilde{\Gamma}$ with $\widetilde{\boldsymbol{\phi}}^n_{i}$ and $-\widetilde{\lambda}^n_{i}$, respectively. We have:
\begin{align}
    \label{eigen_conv}
    \tGamma\eigGammai(x_j) = -\eivGammai\eigGammai(x_j)=d\i^{-1}\Oj\big(\Delta_n\samp_n^{\mathcal{X}}\eigGammai\big)(x_j)
\end{align}
If we apply the mapping $i$ to the last two equalities of \eqref{eigen_conv} and we exploit the orthoghonality of $\Oj$, we obtain:
\begin{align}
\label{eig_gamma_delt}
    \big(\Delta_n\samp_n^{\mathcal{X}}\eigGammai\big)(x_j) = -\eivGammai\Oj^T d\i\eigGammai &= -\eivGammai\samp_n^{\mathcal{X}}\eigGammai(x_j)
\end{align}
where the second equality applies the definition of $\samp_n^{\mathcal{X}}$ in (22). Therefore, we have:
\begin{align}
\label{eig_gamma_delta}
    \lambda_i^n = \eivGammai, \quad \Phii^n(x_j) = \samp_n^{\mathcal{X}}\eigGammai(x_j),
\end{align}
$j= 1,\dots,n$. At this point, we can recall Theorem 6.3 in \cite{singer2013spectral}, that, in the setting of our Theorem 1, states:
\begin{equation}
    \label{spect_VDM}
    \lim_{n\rightarrow \infty } \widetilde{\lambda}_i^n = \lambda_i, \quad\lim_{n\rightarrow \infty} \|\eigGammai -  \Phii\|_{\tm}=0,
\end{equation}
with the limit taken in probability, $j= 1,\dots,n$. 
Injecting the empirical measure in \eqref{spect_VDM} and exploiting the results in \eqref{emp_metr_versions} and \eqref{eig_gamma_delta}, we obtain:
\begin{align}
\label{norm_tm}
    &\|\eigGammai-  \Phii\|_{\tm_n} = \|\Phii^n- \samp_n^{\mathcal{X}}\Phii\|_{\tm_n}
\end{align}
The results in \eqref{spect_VDM} and \eqref{norm_tm}
combined with the a.s. convergence of the empirical measure $\mu_n$ to the measure $\mu$ conclude the proof.

For the sake of clarity, in the following we will drop the dependence on the NNs output index $u$; from the definitions of TNNs in (15) and D-TNNS in (24), we can thus write:
 \begin{align}
    \nonumber  \|\bPsi\big(\mathcal{H}, \Delta_n, \samp_n^{\mathcal{X}}\F\big) - \samp_n^{\mathcal{X}}\bPsi\big(\mathcal{H}, \Delta,\F\big)\|&= \left\| \bbx_{n,L}-\samp_n^{\mathcal{X}}\F_L \right\|.
 \end{align}
 Further explicating the layers definitions, at layer $l$ we have: 
 \begin{align}
   \nonumber  &\left\| \bbx_{n,l}- \samp_n^{\mathcal{X}} \F_l \right\|\\
     &=\left\| \sigma\left(\sum_{q=1}^{F_{l-1}} \bbh_l^{q}(\Delta_n) \bbx_{n,l-1}^q \right) -\samp_n^{\mathcal{X}} \sigma\left(\sum_{q=1}^{F_{l-1}} \bbh_l^{q}(\Delta) \F_{l-1}^q\right) \right\|
 \end{align}
 with $\bbx_{n,0}^q=\samp_n^{\mathcal{X}} \F^q$ for $q=1,\dots,F_0$. Exploiting the normalized point-wise Lipschitz continuity of the non-linearities and the linearity of the sampling operator $\samp_n^{\mathcal{X}}$, we have:
  \begin{align}
  \label{proof_1}
    \| \bbx_{n,l} - \samp_n^{\mathcal{X}} \F_l  \| &\leq \Bigg\|  \sum_{q=1}^{F_{l-1}} \bbh_l^{q}(\Delta_n) \bbx_{n,l-1}^q    \nonumber \\
    &- \samp_n^{\mathcal{X}} \sum_{q=1}^{F_{l-1}} \bbh_l^{q}(\Delta)  \F_{l-1}^q\Bigg\| \nonumber\\
    & \leq \sum_{q=1}^{F_{l-1}} \left\|    \bbh_l^{q}(\Delta_n) \bbx_{n,l-1}^q    - \samp_n^{\mathcal{X}}   \bbh_l^{q}(\Delta)  \F_{l-1}^q\right\|
 \end{align}
 The difference term in the last LHS of \eqref{proof_1} can be further decomposed for every $q=1,\dots,F_{l-1}$ as
\begin{align}\label{proof_2}
   \nonumber   \|    \bbh_l^{q}(\Delta_n) & \bbx_{n,l-1}^q    - \samp_n^{\mathcal{X}}   \bbh_l^{q}(\Delta)  \F_{l-1}^q \| 
   \\ \nonumber&\leq \|
\bbh_l^{q}(\Delta_n) \bbx_{n,l-1}^q  - \bbh_l^{q}(\Delta_n) \samp_n^{\mathcal{X}} \F_{l-1}^q \\ &\qquad +\bbh_l^{q}(\Delta_n) \samp_n^{\mathcal{X}} \F_{l-1}^q  - \samp_n^{\mathcal{X}}   \bbh_l^{q}(\Delta)  \F_{l-1}^q
    \|\nonumber \\ \nonumber
   & \leq \left\|
    \bbh_l^{q}(\Delta_n) \bbx_{n,l-1}^q  - \bbh_l^{q}(\Delta_n) \samp_n^{\mathcal{X}} \F_{l-1}^q
    \right\|
  \\ &\qquad +
    \left\|
    \bbh_l^{q}(\Delta_n) \samp_n^{\mathcal{X}} \F_{l-1}^q  - \samp_n^{\mathcal{X}}   \bbh_l^{q}(\Delta)  \F_{l-1}^q
    \right\|
\end{align}
The first term of the last inequality in \eqref{proof_2} can be bounded as $\| \bbx_{n,l-1}^q - \samp_n^{\mathcal{X}}\F_{l-1}^q\|$ with the initial condition $\|\bbx_{n,0}^q - \samp_n^{\mathcal{X}} \F_0^q\|=0$ for $q = 1,\dots,F_0$. Denoting the second term with $D_{l-1}^n$,  and iterating the bounds derived above through layers and features, we obtain:
\begin{align}
 \nonumber \|\bPsi(\mathcal{H},\Delta_n,\samp_n^{\mathcal{X}} \F) - \samp_n^{\mathcal{X}} \bPsi(\mathcal{H},\Delta,\F)\|
 \leq
 \sum_{l=0}^L \prod\limits_{l'=l}^L F_{l'} D_l^n.
 \end{align}
 Therefore, we can focus on each difference term $D_l^n$ and omit the feature and layer indices to simplify the notations. 
 We can write the convolution operation in the spectral domain as
 \begin{align}
    & \nonumber\|\bbh(\Delta_n)\samp_n^{\mathcal{X}} \F - \samp_n^{\mathcal{X}}\bbh(\Delta) \F\| \nonumber \\
   & = \Bigg\| \sum_{i=1}^n \hat{h}(\lambda_i^n) \langle \samp_n^{\mathcal{X}}\F,\Phii^n \rangle_{\tm_n}\Phii^n \nonumber \\
    & \qquad \qquad \qquad \quad- \sum_{i=1}^\infty \hat{h}(\lambda_i)\langle \F,\Phii\rangle_{\tm} \samp_n^{\mathcal{X}} \Phii  \Bigg\| 
   %   \\ 
   %   &\nonumber \leq  \Bigg\| \sum_{i=1}^M \hat{h}(\lambda_i^n) \langle \samp_n^{\mathcal{X}}\F,\Phii^n \rangle_{\tm_n}\Phii^n \nonumber \\
   %   & \qquad \quad - \sum_{i=1}^M \hat{h}(\lambda_i) \langle \samp_n^{\mathcal{X}}\F,\Phii^n \rangle_{\tm_n}\Phii^n\Bigg\| \nonumber \\
   %   & \quad +\Bigg\| \sum_{i=1}^M \hat{h}(\lambda_i) \langle \samp_n^{\mathcal{X}} \F,\Phii^n \rangle_{\tm_n} \Phii^n \nonumber\\
   %   & \qquad \quad - \sum_{i=1}^M \hat{h}(\lambda_i) \langle \F,\Phii \rangle_{\ccalM} \samp_n^{\mathcal{X}} \Phii \Bigg\|,
   \label{eqn:conv-1}
 \end{align}
We can separately bound the difference in \eqref{eqn:conv-1}; indeed, due to the fact that the filter involved is $\alpha-$FDT by assumption, we can decompose its frequency response as follows:
\begin{align}
\label{eqn:h0-gamma}& h^{(0)}(\lambda) = \left\{ 
\begin{array}{cc} 
                \hat{h}(\lambda)-\sum\limits_{l\in\ccalK_m}\hat{h}(C_l)  &  \lambda\in[\Lambda_k(\gamma)]_{k\in\ccalK_s} \\
                0& \text{otherwise}  \\
                \end{array} \right.  \\
\label{eqn:hl-gamma}& h^{(l)}(\lambda) = \left\{ 
\begin{array}{cc} 
                \hat{h}(C_l) &  \lambda\in[\Lambda_k(\gamma)]_{k\in\ccalK_s} \\
                \hat{h}(\lambda) & 
                \lambda\in\Lambda_l(\gamma)\\
                0 &
                \text{otherwise}  \\
                \end{array} \right.             
\end{align}
where now $\hat{h}(\lambda)=h^{(0)}(\lambda)+\sum_{l\in\ccalK_m}h^{(l)}(\lambda)$ with $\ccalK_s$ defined as the group index set of singletons and $\ccalK_m$ the set of partitions that contain multiple eigenvalues. With the triangle inequality and $n > N_\alpha=\max_{i}\{\lambda_i\in[\Lambda_k(\alpha)]_{k\in\ccalK_s}\}$, we start by analyzing the output difference of $h^{(0)}(\lambda)$ as

\begin{align}
    & \nonumber \Bigg\| \sum_{i=1}^{N_\alpha} {h}^{(0)}(\lambda_i^n) \langle \samp_n^{\mathcal{X}}\F,\Phii^n \rangle_{\tm_n}\Phii^n \nonumber \\
    &- \sum_{i=1}^{N_\alpha} {h}^{(0)}(\lambda_i)\langle \F,\Phii\rangle_{\tm} \samp_n^{\mathcal{X}} \Phii   \Bigg\| \nonumber
     \\ 
     &\nonumber \leq  \left\| \sum_{i=1}^{N_\alpha} \left({h}^{(0)}(\lambda_i^n)- {h}^{(0)}(\lambda_i) \right) \langle \samp_n^{\mathcal{X}}\F,\Phii^n \rangle_{\tm_n}\Phii^n \right\| \\
     &  +\left\| \sum_{i=1}^{N_\alpha} {h}^{(0)}(\lambda_i)\left( \langle \samp_n^{\mathcal{X}}\F,\Phii^n \rangle_{\tm_n}\Phii^n - \langle \F,\Phii\rangle_{\tm} \samp_n^{\mathcal{X}} \Phii  \right)  \right\|.\label{eqn:conv-2}
 \end{align}

The first term of the last bound in \eqref{eqn:conv-2} can be further bounded by exploiting the $C$-Lipschitz continuity of the frequency response function and the convergence in probability stated in \eqref{eqn:convergence_spectrum}: indeed, we can claim that for each eigenvalue $\lambda_i \leq \lambda_{N_\alpha}$, for all $\epsilon_i>0$ and all $\delta_i>0$, there exists some $N_i$ such that for all $n>N_i$, we have:
\begin{gather}
 \label{eqn:eigenvalue}   \mathbb{P}(|\lambda_i^n-\lambda_i|\leq \epsilon_i)\geq 1-\delta_i,
 \end{gather}
Letting $\epsilon_i < \epsilon$ with $\epsilon > 0$, with probability at least $\prod_{i=1}^M(1-\delta_i) := 1-\delta$, we obtain:
\begin{align}
   &\left\| \sum_{i=1}^{N_\alpha} ({h}^{(0)}(\lambda_i^n) - {h}^{(0)}(\lambda_i))\langle \samp_n^{\mathcal{X}} \F,\Phii^n \rangle_{\tm_n} \Phii^n  \right\| \nonumber
   \\
   &\qquad \leq \sum_{i=1}^{N_\alpha} |{h}^{(0)}(\lambda_i^n)-{h}^{(0)}(\lambda_i)| |\langle \samp_n^{\mathcal{X}} \F,\Phii^n \rangle_{\tm_n}| \|\Phii^n\| \nonumber\\
   &\qquad \leq \sum_{i=1}^{N_\alpha}  C |\lambda_i^n-\lambda_i| \|\samp_n^{\mathcal{X}} \F\| \|\Phii^n \|^2\leq N_s C\epsilon,
\end{align} 
for all $n>\max\{\max_i N_i, N_\alpha \}:= N$, where the first inequality is obtained applying the triangle inequality, the second inequality exploits the $C-$Lipschitz continuity of the frequency response, and the last inequality exploits \eqref{eqn:eigenvalue}.
The second term of the last bound in \eqref{eqn:conv-1} can be bounded eploiting the convergence of eigenvectors in \eqref{eqn:convergence_spectrum}. We start with
\begin{align}\label{eqn:term1}
   &\nonumber \left\|\sum_{i=1}^{N_\alpha}{h}^{(0)}(\lambda_i)( \langle \samp_n^{\mathcal{X}} \F,\Phii^n \rangle_{\tm_n}\Phii^n -  \langle f,\Phii \rangle_{\tm} \samp_n^{\mathcal{X}} \Phii )\right\|\\
   &\leq \nonumber \Bigg\|  \sum_{i=1}^{N_\alpha} {h}^{(0)}(\lambda_i)\Big(\langle \samp_n^{\mathcal{X}} \F,\Phii^n\rangle_{\tm_n}\Phii^n  \nonumber \\
   &- \langle \samp_n^{\mathcal{X}} \F,\Phii^n \rangle_{\tm_n} \samp_n^{\mathcal{X}}\Phii\Big)\Bigg\| \nonumber \\
   &+ \Bigg\| \sum_{i=1}^{N_\alpha} {h}^{(0)}(\lambda_i)\Big(\langle \samp_n^{\mathcal{X}} \F,\Phii^n\rangle_{\tm_n} \samp_n^{\mathcal{X}}\Phii \nonumber \\
   &-\langle \F,\Phii\rangle_\tm \samp_n^{\mathcal{X}}\Phii \Big) \Bigg\|
\end{align}
From the convergence in probability stated in \eqref{eqn:convergence_spectrum}, we can claim that for some fixed eigenvector field $\Phii$,  for all $\epsilon_i>0$ and all $\delta_i>0$, there exists some $N_i$ such that for all $n>N_i$, we have
\begin{gather}
 \label{eqn:eigenfunction}    \mathbb{P}(\|\Phii^n - \samp_n^{\mathcal{X}}\Phii\|\leq \epsilon_i)\geq 1-\delta_i.
 \end{gather}
 Therefore, letting $\epsilon_i < \epsilon$ with $\epsilon > 0$, with probability at least $\prod_{i=1}^M(1-\delta_i) := 1-\delta$, for all $n>\max\{ \max_i N_i, N_\alpha\} := N$, the first term in \eqref{eqn:term1} can be bounded as
\begin{align}
&\nonumber \Bigg\| \sum_{i=1}^{N_\alpha} {h}^{(0)}(\lambda_i) \Big(\langle \samp_n^{\mathcal{X}} f,\Phii^n\rangle_{\tm_n}\Phii^n  \nonumber \\
&\qquad \qquad \qquad- \langle \samp_n^{\mathcal{X}} \F,\Phii^n \rangle_{\tm_n} \samp_n^{\mathcal{X}}\Phii\Big)\Bigg\|\\
&\qquad \qquad \qquad \leq \sum_{i=1}^{N_\alpha} \|\samp_n^{\mathcal{X}} \F\|\|\Phii^n - \samp_n^{\mathcal{X}}\Phii\|\leq N_s \epsilon,
\end{align}
considering the boundedness of frequency response function.
The second term in \eqref{eqn:term1} can be written as
\begin{align}
  \nonumber &\left\| \sum_{i=1}^{N_\alpha} {h}^{(0)}(\lambda_i^n) \left(\langle \samp_n^{\mathcal{X}} \F,\Phii^n\rangle_{\tm_n} \samp_n^{\mathcal{X}}\Phii -\langle \F,\Phii\rangle_\ccalM \samp_n^{\mathcal{X}}\Phii \right) \right\| \\
   &\leq \sum_{i=1}^{N_\alpha}|{h}^{(0)}(\lambda_i^n)| \left|\langle \samp_n^{\mathcal{X}} \F,\Phii^n\rangle_{\tm_n}  -\langle \F,\Phii\rangle_\ccalM\right|\|\samp_n^{\mathcal{X}}\Phii\|.
\end{align}
Because $\{x_1, x_2,\cdots,x_n\}$ is a set of uniform sampled points from $\ccalM$, based on Proposition 11 in \cite{von2008consistency}, we can claim that 
\begin{equation}
    \lim_{n\to \infty} \mathbb{P}\left(\left|\langle \samp_n^{\mathcal{X}} \F,\Phii^n\rangle_{\tm_n}  -\langle \F,\Phii\rangle_\tm\right|\leq\epsilon \right)\geq 1-\delta,
\end{equation}
for all $\epsilon>0$ and $\delta>0$. Consider the boundedness of frequency response $|{h}^{(0)}(\lambda)|\leq 1$ and the bounded energy of $\|\samp_n^{\mathcal{X}}\Phii\|$, we have for all $\epsilon>0$ and $\delta>0$:
\begin{align}
\lim_{n\to \infty}\mathbb{P}\Bigg(\Bigg\| & \sum_{i=1}^{N_\alpha}|{h}^{(0)}(\lambda_i^n) |\bigg(\langle \samp_n^{\mathcal{X}} \F,\Phii^n\rangle_{\tm_n}  \nonumber \\
 &-\langle \F,\Phii\rangle_\tm\bigg)\samp_n^{\mathcal{X}}\Phii  \Bigg\|\leq N_s \epsilon\Bigg) \geq 1-\delta.
\end{align}

Combining the above results, we can bound the output difference of $h^{(0)}(\lambda)$. Then we need to analyze the output difference of $h^{(l)}(\lambda)$ and bound this as
\begin{align}
    \nonumber &\left\| \samp_n^{\mathcal{X}}\bbh^{(l)}(\Delta) \F -\bbh^{(l)} (\Delta_n)\samp_n^{\mathcal{X}} \F \right\| 
    \\& \leq \left\| (\hat{h}(C_l)+\delta) \samp_n^{\mathcal{X}} \F - (\hat{h}(C_l)-\delta)\samp_n^{\mathcal{X}} \F \right\| \leq 2\delta\|\samp_n^{\mathcal{X}} \F\|,
\end{align}
where $\bbh^{(l)}(\Delta)$ and $\bbh^{(l)}(\Delta_n)$ are filters with filter function $h^{(l)}(\lambda)$ on the connection Laplacian $\Delta$ and Sheaf Laplacian $\Delta_n$ respectively.
Combining the filter functions, we can write
\begin{align}
   \nonumber &\|\samp_n^{\mathcal{X}} \bbh(\Delta)\F-\bbh(\Delta_n)\samp_n^{\mathcal{X}} \F\|\\\nonumber &=
    \Bigg\|\samp_n^{\mathcal{X}}\bbh^{(0)}(\Delta)\F +\samp_n^{\mathcal{X}}\sum_{l\in\ccalK_m}\bbh^{(l)}(\Delta)\F -\\& \qquad \qquad \qquad \bbh^{(0)}(\Delta_n)\samp_n^{\mathcal{X}} \F - \sum_{l\in\ccalK_m} \bbh^{(l)}(\Delta_n)\samp_n^{\mathcal{X}} \F  \Bigg\|\\
    &\nonumber \leq \|\samp_n^{\mathcal{X}} \bbh^{(0)}(\Delta)\F-\bbh^{(0)}(\Delta_n)\samp_n^{\mathcal{X}} \F\|+\\
    &\qquad \qquad \qquad \sum_{l\in\ccalK_m}\|\samp_n^{\mathcal{X}} \bbh^{(l)}(\Delta) \F-\bbh^{(l)}\samp_n^{\mathcal{X}}\F\|.
\end{align}


Combining all these results, we can claim that for all $\epsilon'>0$ and $\delta>0$, there exists some $N$, such that for all $n>N$ we have
\begin{equation}
    \mathbb{P}(\|\bbh(\Delta_n)\samp_n^{\mathcal{X}} \F - \samp_n^{\mathcal{X}}\bbh(\Delta) \F\|\leq \epsilon')\geq 1-\delta.
\end{equation}

With $\lim\limits_{n\rightarrow \infty}D_l^n=0$ in high probability, this concludes the proof.

\bibliographystyle{IEEEtran}
\bibliography{refs}
\end{document}