\documentclass[a4paper,11pt]{article}
\usepackage{jheppub} 
\usepackage[T1]{fontenc} 
\RequirePackage{graphicx}
\usepackage{enumerate}
\usepackage{bm}
\usepackage{braket}
\usepackage{slashed}
\usepackage{comment}
\usepackage{color}
\title{Non-perturbative particle production and differential geometry}
\author[a]{Tomohiro Matsuda}
\affiliation[b]{Laboratory of Physics, Saitama Institute of Technology,
Fukaya, Saitama 369-0293, Japan}
\emailAdd{matsuda@sit.ac.jp}

\abstract{
This paper proposes a basic method for understanding stationary
 radiation on manifolds by means of the Stokes phenomenon, for which
the most famous examples are the Schwinger effect and Hawking
 radiation. 
The former uses gauge symmetry and the latter uses general
 relativity.
The local Stokes phenomenon of the Schwinger effect is already known,
 but it cannot be used naively as an explanation for stationary
 radiation.
The main difficulty of this problem lies in the convention of using
 asymptotic states to define the vacuum.
More precisely, the ``vacuum definition'' here refers to the distinction
 between $\pm$ pairs of vacuum solutions that define the creation and
 annihilation operators.
To address the problem of the vacuum definition and stationary
 radiation on manifolds, we show how to understand the ``vacuum'' of the
Schwinger effect.
Then we show that Hawking radiation should be
 understood by defining the vacuum for a local inertial frame, rather
 than for a local Lorentz frame. 
We show how this difference plays a crucial role in Hawking radiation.
Unravelling the local nature of the Stokes phenomenon and discriminating
similar phenomena will enable us to clearly distinguish between black
 holes and their analogues.
Finally, we show that the Schwinger and Unruh effects can be
 observed simultaneously under a strong electric field, which can be
 verified experimentally.}
\begin{document}

\maketitle
\section{Introduction}
The creation of particles from a vacuum has long been studied as a
fundamental problem in quantum mechanics\cite{Birrell:1982ix}.
Among them, the most fundamental scenario is the case when 
the mass of the particle varies with time\cite{Kofman:1997yn}.
The equation of motion of the particle is reduced to a second-order
ordinary differential equation, and particle production can be explained
by the appearance of the Stokes phenomenon, which mixes the two
solutions (a pair of plus and minus
sign solutions)\cite{Enomoto:2020xlf,Enomoto:2021hfv}.
In this case, the ``vacua'' are defined by two pairs of asymptotic
solutions, one in the past and one in the future, and particles
are created from the vacuum due to the different definitions of the
creation and the annihilation operators in the two limits\cite{Berry:1972na}.

Although the definition of the vacuum using asymptotic solutions looks
good, it is sometimes difficult to know how to define the vacuum.
For example, if one wants to see the Schwinger
effect\cite{Schwinger:1951nm} in a constant
electric field, one has
to assume states where the electric field artificially disappears in the
past and in the future, and furthermore the vacuum solutions have to be
adiabatically connected to the asymptotic states except for the region
where particles are created\cite{Haro:2010mx}.
If the constant electric field exists for a
long enough time in quantum terms, this approach seems quite artificial
 and leaves the simple question of why local analysis is not possible.

The situation is even more difficult in Hawking's original
paper\cite{Hawking:1975vcx}.
Hawking calculates the Bogoliubov coefficient to estimate the radiation
from the black hole, using the natural vacuum before the gravitational
collapse as the in-vacuum, and the vacuum of the distant observer after
the black hole is created as the out-vacuum.
If Hawking radiation is produced by local physics near the horizon,
this analysis seems very artificial.

Since these models are describing stationary radiations defined on
manifolds, it would be natural to assume that they can be explained in
 terms of the basic properties of manifolds.
This is the main motivation for this paper.


The Schwinger effect and Hawking radiation have also been analyzed using
path integrals\cite{Schwinger:1951nm, Hartle:1976tp},
 in addition to those based on field equations.
In the original paper\cite{Schwinger:1951nm}, Schwinger used a one-loop 
calculation to determine the ``vacuum decay rate'', noting that the
equation of motion does not result in a covariant formulation.
As will be explained later, the term ``vacuum'' used here should also be
treated with caution. 
The Schwinger effect is not the transition of the state $|0\rangle\rightarrow
|0'\rangle$, which must be accompanied by the domain walls.


About the Schwinger effect, it is worth noting that when Bogoliubov
coefficients of a charged scalar field are calculated 
using the Stokes phenomena, exact solutions are given by parabolic
cylinder functions\cite{Birrell:1982ix,Haro:2010mx,Dumlu:2010ua}.
These exact solutions are solutions in the presence of an electric field.
If a constant electric field of the Schwinger effect 
exists almost forever and the radiation is stationary, a very simple
question arises: where exactly is the ``vacuum'' on the manifold?

Our answer to this question is as follows.
Given the differential geometry of the manifold, a tangent vector
space can be defined at any point.
Then, we define a vacuum on this tangent vector space.
It is worth elaborating on this definition.

First, we describe the manifolds with local Lorentz transformation.
Mathematically, due to Lorentz symmetry, there are an infinite number of
ways to define the vacuum, so one might think that the phrase
``defining a vacuum in a tangent vector space'' leaves an ambiguity.
This can be resolved by considering the difference between the treatment
of manifolds as mathematics and the treatment of manifolds in the
description of physical phenomena.
Mathematical manifolds are constructed so that they are consistent for all observers.
On the other hand, when discussing physical phenomena, there are
observers, so the observer's frame of reference is naturally chosen.
So, who is the 'observer' in the Schwinger effect?
Opinions may differ as to whether the person performing the experiment
should be the ``observer'' or whether the generated particles are the
``observers'' of the vacua, but to understand the mechanism of the
Schwinger effect one 
needs to understand the vacuum for the generated particles we are
looking at.
Each particle produced has a different momentum, and the ``vacuum'' for
the particle is defined by its own unique (rest) frame. 
Although it may seem that the gauge symmetry is important for the
Schwinger effect and the Lorentz transformation is irrelevant, actual
calculations of the Stokes phenomenon show that the Lorentz
transformation plays an important role in understanding stationary
radiation.



If we consider gauge transformations on manifolds, what is the
definition of a vacuum?
In the case of Lorentz symmetry, the observer's frame has to be 
chosen for the vacuum.
In this way, there was no speed gap between the observer and the
frame.
Similarly, we choose the vacuum on $A_\mu=0$ section,\footnote{ 
The incompatibility between the gauge $A_\mu=0$ and the path integrals is
explained on page 295 of the textbook\cite{Peskin-textbook}
by Peskin and Schroeder.} 
taking into account the continuity of the tangent vectors and the
covariant derivatives.
The above vacuum choices may seem trivial, but they are crucial to
understanding the steady-state radiation on manifolds.


Defining the vacuum in this way avoids the need to artificially
introduce asymptotic states.
Such asymptotic states might have been essential in analyses using the
usual WKB approximation, but thanks to the development of the Exact
WKB\cite{RPN:2017,Delabaere:1993,Silverstone:2008, Voros:1983,
Virtual:2015HKT,
WKB-recent:2022a,WKB-recent:2022b,WKB-recent:2022c,WKB-recent:2022d,WKB-recent:2022e,WKB-recent:2022f,WKB-recent:2022g,WKB-recent:2022h,WKB-recent:2022i}, 
 the local structure of the Stokes phenomenon is now sufficiently well
 understood in terms of resurgence.
In this paper, we will discuss what happens if particle production 
is solved by means of the local Stokes phenomena.

With regard to the definition of a vacuum, it is also shown that 
the vacuum of Hawking radiation must be defined on
a local inertial frame, not on the local Lorentz frame.
The crucial difference between the two frames is not very clear
in many textbooks, so this paper uses a vierbein to illustrate the
difference.

The idea of defining a vacuum in an inertial system and calculating
Bogoliubov coefficients for an accelerating observer is beautifully
summarized in Ref.\cite{Birrell:1982ix} for the Unruh
effect\cite{Unruh:1976db}.
However, their analysis uses the global nature of the Rindler
coordinates, and this hides the local Stokes phenomenon on the manifold.


First, we discuss the Schwinger effect in a constant electric field as an
obvious example in gauge theory.
We will show why the vacuum locally defined on the manifold can
explain stationary radiation in terms of the local Stokes phenomena.

Then we discuss Hawking radiation and the Unruh effect\cite{Unruh:1976db}.
We show that the vacuum must be defined in terms
of a local inertial frame rather than in a local Lorentz frame.
In contrast to the Schwinger effect, it is not the connection that is
important for the local Stokes phenomenon:
the Stokes phenomenon cannot be seen in the ``same'' way as the
Schwinger effect.

Although the Schwinger effect and Hawking radiation share many
similarities, these comparisons using differential geometry and the
Stokes phenomenon highlight crucial differences between them.


\section{The Schwinger effect}
Here we consider the case where the electric field is
spatially homogeneous and has a constant value in the z-direction.
For a complex scalar field $\phi$ of mass $m$ in a four-dimensional
Minkowski spacetime, we consider the action $S_0$ on a tangent space given
by 
\begin{eqnarray}
S_0&=&\int d^4x \left(\partial_\mu \phi \partial^\mu\phi^*-m^2
	       \phi\phi^*\right).
\end{eqnarray}
Introducing a gauge field $A_\mu$, the partial derivatives are replaced
by covariant derivatives of the differential geometry:
\begin{eqnarray}
\partial_\mu &\rightarrow& \nabla_\mu \equiv\partial_\mu+i q A_\mu.
\end{eqnarray}
We define the vacuum on the tangent space attached to $A_\mu=0$.
Assuming the limit where dynamics of the gauge field is negligible (the
gauge field does not propagate in this limit), the 
gauge field is external and given by
\begin{eqnarray}
A^\mu&=&(0,0,0, -E (t-t_0))
\end{eqnarray}
with the electric field strength $\vec{E}=(0,0,E)$ for an arbitrary
$t_0$.
For the scalar field $\phi$, the equation of motion after Fourier
transformation is
\begin{eqnarray}
\label{eq-field-Sch}
\ddot{\phi}_k+\omega_k^2(t)\phi_k&=&0,
\end{eqnarray}
where $\omega_k^2(t)=m^2+k_\perp^2+(k_z-qE(t-t_0))^2$.
We solve the equation with $\bf{\it k}$ set to 0 (or $\bf{\it k}^2\ll \omega^2$), 
since we are choosing the rest frame of the particle.
We omit explicit Lorentz factors of $E$ and $\omega$, since
they will be canceled in the final result.
Eq.(\ref{eq-field-Sch}) looks like the Schr\"odinger equation and can be
solved as a scattering problem for the Schr\"odinger equation with the 
potential given by
\begin{eqnarray}
V(t)&=&-q^2A_z^2=-q^2E^2t^2,
\end{eqnarray}
where we set $t_0=0$ for simplicity.
The potential is shown in Fig. 1 together with the Stokes line on the
complex $t$-plane.
\begin{figure}[h]
\centering
\includegraphics[width=0.8\columnwidth]{fig1.eps}
 \caption{The potential $V(t)=-q^2E^2t^2$ and the Stokes lines (on the
 complex $t$ plane) are
 shown. Particle production occurs when the real $t$-axis crosses the
 Stokes line (indicated by the arrow at the origin).}
\label{fig-1}
\end{figure}
The exact solutions are described by using the parabolic cylinder
functions\cite{Birrell:1982ix,Kofman:1997yn} or the Weber
functions \cite{Enomoto:2020xlf}.
Using the Exact WKB, one can find a Merged pair of Turning Points
(MTP)\cite{Enomoto:2020xlf} whose Stokes line crosses on the real axis at
the origin ($t=0$).

In this simplest situation, one can ``always'' find a local vacuum
attached to $A_\mu=0$ on which the vacuum and the Stokes line coincide.
This is the very reason why radiation of the Schwinger effect can be
stationary.
The situation is illustrated in Fig\ref{fig-2}.
\begin{figure}[h]
\centering
\includegraphics[width=0.5\columnwidth]{fig2.eps}
 \caption{The potential ($V(t)=-q^2E^2t^2$) is illustrated on
 the manifold taking into  account the gauge degrees of freedom.
Since the Stokes phenomenon occurs on the vacuum attached to 
the red line (the line connecting vertices of the potential), the Stokes
 phenomenon is always found on this manifold.}
\label{fig-2}
\end{figure}
The local Stokes phenomena are therefore properly defined without
relying on artificial asymptotic states, and the
definition of the ``vacuum'' is consistent with stationary radiation.
The importance of the frame should be mentioned here.
If the frame had not been chosen to be the rest frame of the particle,
the Stokes line would not appear on the vacuum at $A_\mu=0$ due 
to the fact that $k_z\ne0$. 
(See the definition of $\omega^2_k(t)$.)
The importance of the choice of frame is illustrated by the fact that
the Stokes line must be just above the vacuum to explain that the
radiation is stationary.


We have seen that on manifolds one can always find the the vacuum of particle creation 
which is defined locally on tangent vector space by choosing the section
$A_\mu=0$ and the rest frame of the particle.
One does not have to choose fictitious asymptotic states at a distance. 
The definition of the vacuum is obvious in the light of manifolds.
As for the Schwinger effect, we have seen that the local Stokes
phenomenon is induced simply by the covariant derivatives.
Let us now show that this is not the case with Hawking radiation
and the Unruh effect.

\section{Hawking radiation}
Following the Schwinger effect, one can define a local vacuum
to find stationary radiation due to the Stokes phenomenon.
The vacuum of the particle production is defined by choosing a frame for which the
particle is at rest.
In general relativity, there are two different definitions of coordinate
systems in which the above condition is satisfied: the local Lorentz
frame and the local inertial frame.
Although the two frames are often considered to be physically almost
identical because they give the same metric, the difference between the
two frames is crucial when discussing stationary radiation on manifolds.


In order to understand the difference with gauge theory, it is first
necessary to understand what happens if the derivatives are naively
replaced by covariant derivatives.
This manipulation gives the Klein-Gordon equation on curved
spacetime, but unlike the Schwinger effect, it cannot explain the
local Stokes phenomenon\cite{Enomoto:2022mti}.\footnote{Stokes
events around black holes can occur outside the horizon.
We discuss Stokes events that are directly related to local Hawking
radiation at the event horizon.
See Ref.\cite{Dumlu:2020wvd} and \cite{Enomoto:2022mti} for the Stokes
phenomena appearing outside the horizon.} 

To understand the essence of the story, it is better to consider the
Unruh effect before Hawking radiation.
The vacuum seen by observers in the Unruh effect was a local inertial
system, not the local Lorentz system. 
The metric is exactly the same whether it is a local inertial
system or a local Lorentz system, so we rarely distinguish between the
two, but in the present case there must be some crucial difference.
In fact, when calculating the vierbein of the Rindler spacetime given by
\begin{eqnarray}
\label{eq-rindler-t}
t&=&\frac{1+\alpha x_r}{\alpha}\sinh (\alpha t_r)\nonumber\\
x&=&\frac{1+\alpha x_r}{\alpha}\cosh (\alpha t_r),
\end{eqnarray}
which describes the coordinate system of an object
moving at constant acceleration $\alpha$ through a flat space-time
represented by $(t,x)$, one will find
\begin{eqnarray}
dt&=&\left(1+\alpha x_r\right)\cosh (\alpha t_r) dt_r
+\sinh(\alpha t)dx_r\nonumber\\
dx&=&\cosh(\alpha t)dx_r +\left(1+\alpha x_r\right)\sinh (\alpha t_r) dt_r
\end{eqnarray}
for local inertial frame.
The metric is calculated as
\begin{eqnarray}
g_{\mu\nu}&=&\eta_{mn}e^m_\mu e^n_\nu,
\end{eqnarray}
where $\eta_{mn}$ is for the local Minkowski space.
This gives the Rindler metric given by 
\begin{eqnarray}
\label{eq-metric-R}
ds^2&=&-(1+ \alpha x_r)^2dt_r^2+dx_r^2.
\end{eqnarray}

Note that the metric is identical for both frames but the explicit
$t_r$-dependence of the vierbein disappears for local Lorentz frame.
(The vierbein of the local Lorentz frame is diagonal.)
As we will see, this is where the Stokes phenomenon of stationary
radiation on curved space-time comes from.

As is summarized in Ref.\cite{Birrell:1982ix}, it is possible to
calculate the Bogoliubov coefficients by considering carefully 
the global structure of the Rindler coordinates and the relationship
between the vacuum solutions written in the two coordinate systems.
However, it may seem unnatural that global information about a distant
pole is essential when its motion can be 
viewed as constant acceleration only over a certain period of time.
Furthermore, what we want to consider is a local system defined
in the neighborhood of a point: a global system is excessive for us.
On the other hand, their global calculation revealed a surprising
correlation of particle production between two causally disconnected
wedges, which will later explain the factor two discrepancy between
 our local calculation.
More details about the Bogoliubov transformation on the Rindler
coordinates can be found in Ref.\cite{Birrell:1982ix}.


Let us see how the Stokes phenomenon of the Unruh effect appears
locally.
To keep the integral of the solutions intact, we use
$dt=\cosh (\alpha t_r) dt_r$
to write the vacuum solutions after the Fourier transform
into
\begin{eqnarray}
\label{eq-rindler-sol}
\phi_k^\pm(t)&=&A_k e^{\pm i \int \omega dt}\nonumber\\
&=&A_k e^{\pm i \int \omega_k \cosh(\alpha t_r) dt_r}.
\end{eqnarray}
We are choosing the particle's rest frame, for which ${\bf \it k}^2 \ll \omega^2$. 
It is normally difficult to recognize the Stokes phenomenon from these
solutions, but the problem can be solved using the basic properties of
the Exact WKB.
First define $Q(t)_0\equiv -\omega_k^2\cosh^2(\alpha t_r)$ and consider
 the following ``Schr\"odinger'' equation
\begin{eqnarray}
\label{eq-rindler-EoM}
\left(-\frac{d^2}{dt^2}+\eta^2 Q(t,\eta)\right)\psi(t,\eta)&=&0,
\end{eqnarray}
where $\eta\gg 1$ and $Q(t,\eta)$ is expanded as
\begin{eqnarray}
Q(t,\eta)&=&Q_0(t)+\eta^{-1}Q_1(t)+\eta^{-2}Q(t)+\cdots.
\end{eqnarray}
The solution of this equation can be written as $\psi(t,\eta)\equiv
e^{\int S(t,\eta) dt}$, where $S(t,\eta)$ can be expanded as 
\begin{eqnarray}
S&=&S_{-1}(t)\eta +S_0(t)+S_1(t)\eta^{-1}+\cdots.
\end{eqnarray}
The point of this argument is that after introducing $\eta$ properly
in Eq.(\ref{eq-rindler-sol}), one can choose $Q_i(t), i\ge 1$ to
find the ``Schr\"odinger equation'' which gives the solutions 
Eq.(\ref{eq-rindler-sol}).\footnote{See also Section 5 of
Ref.\cite{Costin:2008}.}
Here $\hbar$ of quantum mechanics has been replaced by $\eta$ according to
mathematical convention.

This procedure allows us to make use of a powerful analysis of the
Exact WKB: one can calculate the Stokes lines only by using 
$Q_0(t)$.\footnote{The Stokes lines of $Q_0(t)$ are explicitly shown in
Fig.3 of Ref.\cite{Enomoto:2022mti}.}
After drawing the Stokes lines, one can see that a Stokes line crosses on the
real axis at the origin\cite{Enomoto:2022mti}.
This allowed us to expand $Q(t)_0$ near the origin and finally we have
\begin{eqnarray}
Q(t)_0&=& -\omega_k^2\cosh(\alpha t_r)\nonumber\\
&\simeq&-\omega_k^2-\alpha^2 \omega_k^2 t_r^2,
\end{eqnarray}
which gives a familiar Schr\"odinger equation of scattering by an
inverted quadratic potential.
The equation can be solved using the parabolic cylinder functions or the
Weber functions, giving the MTP structure of the Stokes
lines\cite{Enomoto:2020xlf,Enomoto:2021hfv,
Enomoto:2022mti}.\footnote{See also Fig.\ref{fig-1}.}
The Stokes phenomenon occurs at $t=0$, which can be regarded as the
point of contact ($v=0$) between the observer in the accelerating system and the
vacuum in the inertial system. 
The vacuum is defined in different frames depending on the particles
generated.
Due to the nature of the manifold, it can be understood that 
one can always find particle production of any momentum.

The Bogoliubov coefficients of this calculation have the
non-perturbative factor $\sim e^{-\pi \omega_k/\alpha}$.
As we have already mentioned, the global analysis of Ref.\cite{Birrell:1982ix} 
gives $\sim e^{-2\pi \omega_k/\alpha}=\left(e^{-\pi
\omega_k/\alpha}\right)^2$, since they are calculating
the probability of strongly correlated particle production in the two
distinct regions.
We can see that two particles are produced simultaneously in global
while we are calculating local and single particle production.
The same applies to Hawking radiation, because Hawking radiation
requires creation of a pair\footnote{Unlike the Schwinger effect, real
scalar and Majorana fermions can be generated not as the
matter-antimatter pair} of negative and positive energy
particles across the horizon, but only the positive energy
particle outside the horizon can be observed as radiation.

\section{Simultaneous Schwinger-Unruh-Hawking effect}
Finally, it should be mentioned that the Schwinger and the Unruh effects
(Hawking radiation without the event horizon) occur simultaneously under strong electric fields.
As noted above, the way to choose the vacuum is ``local inertial system,
rest frame for the particle and $A_\mu=0$'', but considering that the
 generated particles are
accelerated by the electric field, the particles are generated from 
``the vacuum seen by an accelerating observer (particle)''.
This is the same situation as in the case of the Unruh effect.
Furthermore, unlike the Unruh effect, pair production is possible
without violating the law of conservation of energy if the electric
field is strong enough.
Therefore, radiation is observed even in the absence of the event
horizon. 
Furthermore, since one of the two particles of a pair 
does not disappear into the horizon, both are observed as radiation.


Perhaps the most interesting aspect of this story is whether it is
possible to verify experimentally that the Schwinger and Unruh effects
occur simultaneously.
As shown in this paper, the two effects arise from separate physical
phenomena, but contribute in the same form to the same equation.
When written in the Exact WKB notation, it appears that there 
are two contributions $Q_0^{Schwinger}(t)$ and $Q_0^{Unruh}(t)$ to
$Q_0(t)=Q_0^{Schwinger}(t)+Q_0^{Unruh}(t)$, 
so that the coefficient of the inverted quadratic potential
causing the Stokes phenomenon is the sum of the two contributions. 
One issue arises here.
The acceleration of a charged particle in a strong electromagnetic field
is $\alpha=qE/m$, which, when used in the equation for the Unruh effect,
gives an inverted quadratic potential with exactly the same coefficient as
the Schwinger effect.
If this consideration is correct, the actual probability of particle
production should be increased, and the quantitative property is
significantly different from Schwinger's calculation.
In this way, the simultaneous occurrence of the Unruh and Schwinger
effects is a phenomenon that can be demonstrated experimentally.
The previously mentioned problem with factor 2 of the Unruh effect 
can now be confirmed by experimenting.
If, on the other hand, the amplification noted here did not occur, then
it can be concluded that the Schwinger and Unruh effects are in fact
identical physical phenomena that are linked even more deeply in the
theory.




\section{Conclusions and discussions}

Particle production from a vacuum, including the Schwinger effect, Hawking
radiation and the reheating of the universe, is a very important area of
research in theoretical physics.
We believe that local analysis is essential to understand these phenomena.
However, conventional WKB approximation does not allow local
analysis at locations where the adiabatic condition is violated.
The development of the Exact WKB largely solved this problem, but the
practice persisted for a long time, as important papers of the time set
up asymptotic vacuum at a far distance.
Serious problem does not arise in simple models because in such models
only the physical quantity (e.g., the mass) changes with time.
However, in cases where the physical quantity does not change, such as
the Schwinger effect in a constant electric field, the Stokes phenomenon
of the solutions raises major questions about the definition of the
vacuum.
This paper aims to establish the definition of the vacuum in
the Schwinger effect and Hawking radiation using the notion of
manifolds.

The benefits of our local analysis are significant:
without defining a vacuum at a distance or creating a collapse gap,
local stationary radiation can be analyzed using manifolds.

It is the development of the Exact WKB that makes it possible to deal with the
seemingly complex equations that result from such analysis.
By understanding the local nature of the Stokes phenomenon, one can
clearly distinguish between black holes and their
analogs\cite{Enomoto:2022mti,Giovanazzi:2004zv}. 
If black hole chaos\cite{Maldacena:2015waa} is due to the Stokes
phenomenon\cite{Morita:2019bfr}, we hope we can get closer to the nature.
 
\section{Acknowledgments}
The author would like to express his gratitude to Seishi Enomoto for his
cooperation in the early stages of this work.
In writing the revised version, we express our gratitude to
Satoshi Ohya for inviting us to the seminar at Nihon University and to the
participants of the seminar for clarifying points that needed
to be explained.


\begin{thebibliography}{1}
\bibitem{Birrell:1982ix}
N.~D.~Birrell and P.~C.~W.~Davies,
``Quantum Fields in Curved Space,''
\bibitem{Kofman:1997yn}
  L.~Kofman, A.~D.~Linde and A.~A.~Starobinsky,
  ``Towards the theory of reheating after inflation,''
  Phys.\ Rev.\ D {\bf 56} (1997) 3258
  [hep-ph/9704452].
\bibitem{Enomoto:2020xlf}
S.~Enomoto and T.~Matsuda,
``The exact WKB for cosmological particle production,''
JHEP \textbf{03} (2021), 090
[arXiv:2010.14835 [hep-ph]].
\bibitem{Enomoto:2021hfv}
S.~Enomoto and T.~Matsuda,
``The exact WKB and the Landau-Zener transition for asymmetry in cosmological particle production,''
JHEP \textbf{02} (2022), 131
[arXiv:2104.02312 [hep-th]].
\bibitem{Berry:1972na}
M.~V.~Berry and K.~Mount,
``Semiclassical approximations in wave mechanics,''
Rept.\ Prog.\ Phys.\  \textbf{35} (1972), 315
\bibitem{Schwinger:1951nm}
J.~S.~Schwinger,
``On gauge invariance and vacuum polarization,''
Phys. Rev. \textbf{82} (1951), 664-679
\bibitem{Haro:2010mx}
J.~Haro,
``Topics in Quantum Field Theory in Curved Space,''
[arXiv:1011.4772 [gr-qc]].
\bibitem{Hawking:1975vcx}
S.~W.~Hawking,
``Particle Creation by Black Holes,''
Commun. Math. Phys. \textbf{43} (1975), 199-220
[erratum: Commun. Math. Phys. \textbf{46} (1976), 206]
\bibitem{Hartle:1976tp}
J.~B.~Hartle and S.~W.~Hawking,
``Path Integral Derivation of Black Hole Radiance,''
Phys. Rev. D \textbf{13} (1976), 2188-2203
\bibitem{Dumlu:2010ua}
C.~K.~Dumlu and G.~V.~Dunne,
``The Stokes Phenomenon and Schwinger Vacuum Pair Production in Time-Dependent Laser Pulses,''
Phys. Rev. Lett. \textbf{104} (2010), 250402
[arXiv:1004.2509 [hep-th]].
\bibitem{Peskin-textbook}
M.~E.~Peskin and D.~V.~Schroeder,
``An Introduction to Quantum Field Theory''
Westview Press (1995)
\bibitem{RPN:2017}
``Resurgence, Physics and Numbers'' edited by F.~Fauvet, D.~Manchon,
	S.~Marmi and  D.~Sauzin, Publications of the Scuola Normale
	Superiore, 978-88-7642-613-1
\bibitem{Voros:1983}
A.~Voros, ``The return of the quartic oscillator -- The complex WKB
method'', Ann. Inst. Henri Poincare, 39 (1983), 211-338.
\bibitem{Delabaere:1993}
E. Delabaere, H. Dillinger and F. Pham: Resurgence de Voros et peeriodes
des courves hyperelliptique. Annales de l'Institut Fourier, 43 (1993), 163-
199.
\bibitem{Silverstone:2008}
H.~Shen and H.~J.~Silverstone, ``Observations on the JWKB treatment of
the quadratic barrier, Algebraic analysis of differential equations from
	microlocal analysis to exponential asymptotics'', Springer,
	2008, pp. 237 - 250.
\bibitem{Virtual:2015HKT}
N.~Honda, T.~Kawai and Y.~Takei,
``Virtual Turning Points'', Springer (2015),
ISBN 978-4-431-55702-9.
\bibitem{WKB-recent:2022a}
H.~Taya, T.~Fujimori, T.~Misumi, M.~Nitta and N.~Sakai,
``Exact WKB analysis of the vacuum pair production by time-dependent
	electric fields,''
JHEP \textbf{03} (2021), 082
[arXiv:2010.16080 [hep-th]]
\bibitem{WKB-recent:2022b}
S.~K.~Ashok, D.~P.~Jatkar, R.~R.~John, M.~Raman and J.~Troost,
``Exact WKB analysis of $ \mathcal{N} $ = 2 gauge theories,''
JHEP \textbf{07} (2016), 115
[arXiv:1604.05520 [hep-th]]
\bibitem{WKB-recent:2022c}
F.~Yan,
``Exact WKB and the quantum Seiberg-Witten curve for 4d N = 2 pure SU(3) Yang-Mills. Abelianization,''
JHEP \textbf{03} (2022), 164
[arXiv:2012.15658 [hep-th]]
\bibitem{WKB-recent:2022d}
A.~Grassi, Q.~Hao and A.~Neitzke,
``Exact WKB methods in SU(2) N$_{f}$ = 1,''
JHEP \textbf{01} (2022), 046
[arXiv:2105.03777 [hep-th]]
\bibitem{WKB-recent:2022e}
S.~Kamata, T.~Misumi, N.~Sueishi and M.~\"Unsal,
``Exact-WKB analysis for SUSY and quantum deformed potentials: Quantum mechanics with Grassmann fields and Wess-Zumino terms,''
[arXiv:2111.05922 [hep-th]]
\bibitem{WKB-recent:2022f}
M.~Alim, L.~Hollands and I.~Tulli,
``Quantum curves, resurgence and exact WKB,''
[arXiv:2203.08249 [hep-th]]
\bibitem{WKB-recent:2022g}
M.~F.~Girard,
``Exact WKB-like Formulae for the Energies by means of the Quantum Hamilton-Jacobi Equation,''
[arXiv:2204.02708 [quant-ph]]
\bibitem{WKB-recent:2022h}
A.~van Spaendonck and M.~Vonk,
``Painlev\'e I and exact WKB: Stokes phenomenon for two-parameter transseries,''
[arXiv:2204.09062 [hep-th]]
\bibitem{WKB-recent:2022i}
K.~Imaizumi,
``Quasi-normal modes for the D3-branes and Exact WKB analysis,''
Phys. Lett. B \textbf{834} (2022), 137450
[arXiv:2207.09961 [hep-th]].
\bibitem{Unruh:1976db}
W.~G.~Unruh,
``Notes on black hole evaporation,''
Phys. Rev. D \textbf{14} (1976), 870
\bibitem{Enomoto:2022mti}
S.~Enomoto and T.~Matsuda,
``The Exact WKB analysis and the Stokes phenomena of the Unruh effect and Hawking radiation,''
JHEP \textbf{12} (2022), 037
[arXiv:2203.04501 [hep-th]].
\bibitem{Dumlu:2020wvd}
C.~K.~Dumlu,
``Stokes phenomenon and Hawking radiation,''
Phys. Rev. D \textbf{102} (2020) no.12, 125006
[arXiv:2009.09851 [hep-th]].
\bibitem{Costin:2008}
O.~Costin, ``Asymptotics and Borel Summability'' (2008),
Chapman and Hall/CRC. https://doi.org/10.1201/9781420070323
\bibitem{Giovanazzi:2004zv}
S.~Giovanazzi,
``Hawking radiation in sonic black holes,''
Phys. Rev. Lett. \textbf{94} (2005), 061302
[arXiv:physics/0411064 [physics]].
\bibitem{Maldacena:2015waa}
J.~Maldacena, S.~H.~Shenker and D.~Stanford,
``A bound on chaos,''
JHEP \textbf{08} (2016), 106
[arXiv:1503.01409 [hep-th]].
\bibitem{Morita:2019bfr}
T.~Morita,
``Thermal Emission from Semi-classical Dynamical Systems,''
Phys. Rev. Lett. \textbf{122} (2019) no.10, 101603
[arXiv:1902.06940 [hep-th]].
\end{thebibliography}
\end{document}
