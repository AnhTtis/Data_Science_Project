\documentclass[12pt]{article}

\usepackage{amstext,amsmath,amssymb,amsfonts,bbm,slashed}
\usepackage[latin1]{inputenc}
\usepackage{epsfig}
\usepackage{hyperref}
\usepackage{amsthm}
\usepackage{subfigure}
\usepackage{color}
\usepackage{multirow}
\usepackage{calc}
\usepackage{comment}

%\usepackage[sort,nocompress]{cite}

\usepackage{stmaryrd}


%\usepackage{showkeys}

\usepackage{cancel}
\usepackage{graphicx}

\usepackage{float}

\usepackage{soul}

\usepackage[normalem]{ulem}
\usepackage{amsmath}
\newcommand{\stkout}[1]{\ifmmode\text{\sout{\ensuremath{#1}}}\else\sout{#1}\fi}


%\usepackage{showkeys}


\usepackage[T1]{fontenc}
\usepackage{array}
\usepackage{makecell}
\newcolumntype{x}[1]{>{\centering\arraybackslash}p{#1}}
\usepackage{tikz}
\newcommand\diag[4]{%
  \multicolumn{1}{p{#2}|}{\hskip-\tabcolsep
  $\vcenter{\begin{tikzpicture}[baseline=0,anchor=south west,inner sep=#1]
  \path[use as bounding box] (0,0) rectangle (#2+2\tabcolsep,\baselineskip);
  \node[minimum width={#2+2\tabcolsep},minimum height=\baselineskip+\extrarowheight] (box) {};
  \draw (box.north west) -- (box.south east);
  \node[anchor=south west] at (box.south west) {#3};
  \node[anchor=north east] at (box.north east) {#4};
 \end{tikzpicture}}$\hskip-\tabcolsep}}
%
\usepackage{pbox}





\newcommand{\I}{\mathbb{I}}
\newcommand{\kin}{{\rm{kin\,}}}
\newcommand{\ext}{{\rm{ext\,}}}
\newcommand{\inter}{{\rm{int\,}}}


\newcommand{\lam}{\lambda}
\newcommand{\laren}{\lambda_{\rm ren} }
%\newcommand{\lacren}{\lambda^{c}_{\rm ren}}
\newcommand{\lapren}{ \lambda_{+, {\rm ren}} }
%\newcommand{\lapcren}{\lambda^{c}_{\rm +;ren}}
\newcommand{\lac}{\lambda^{(c)}}
\newcommand{\lacren}{\lambda^{(c)}_{\rm ren}}
\newcommand{\lapc}{ \lambda^{(c)}_+ }
\newcommand{\lapcren}{\lambda^{(c)}_{\rm +;ren}}

\newcommand{\pet}{\lambda_{+}}

\newcommand{\bet}{\eta_{\bullet}}
\newcommand{\tet}{\eta_{\times}}
\newcommand{\rhop}{\rho_{+}}
\newcommand{\rhot}{\rho_{\times}}

\newcommand{\omp}{\omega_{{\rm d};+}}
\newcommand{\omt}{\omega_{{\rm d};\times}}
\newcommand{\omb}{\omega_{{\rm d};\bullet}}


\newcommand{\lin}{l_{\rm in}}
\newcommand{\lex}{ {l_{\rm ext}} }

\newcommand{\od}{{\rm{od}}}
\newcommand{\ev}{{\rm{ev}}}

\newcommand{\sset}{\subset}


\def\C{{\mathbbm C}}
\def\N{{\mathbbm N}}
\def\Z{{\mathbbm Z}}
\def\R{{\mathbbm R}}



\newcommand{\cG}{{\mathcal G}}
\newcommand{\bG}{{\partial\mathcal G}}
\newcommand{\tJ}{{\widetilde{J}}}


\newcommand{\cD}{{\mathfrak{d}}}

\newcommand{\cexG}{\mathcal G_{\text{color}}}
\newcommand{\bJ}{ J_{\partial} }

\newcommand{\cO}{{\mathcal O}}
\newcommand{\cR}{{\mathcal R}}
\newcommand{\cL}{{\mathcal L}}
\newcommand{\cV}{{\mathcal V}}
\newcommand{\cF}{{\mathcal F}}
\newcommand{\bcF}{{\overline{\mathcal F}}}
\newcommand{\cP}{{\mathcal P}}

\newcommand{\cU}{{\mathcal U}}
\newcommand{\mU}{{\mathbb{U}}}

 \newcommand{\cE}{\mathcal{E}}
\newcommand{\cK}{\mathcal{K}}


 \newcommand{\HU}{  {\rm{HU}} }

\newcommand{\cH}{\mathcal{HR}}
\newcommand{\cHE}{\mathcal{HE}}

\newcommand{\cFL}{\mathcal{FL}}


\newcommand{\bdel}{ {\boldsymbol{\delta}} }

\newcommand{\Om}{ {\boldsymbol{\Omega}} }

\newcommand{\eps}{\epsilon}
\newcommand{\beps}{\bar\epsilon}

\newcommand{\Fev}{\cF^{\ev}}
\newcommand{\Fod}{\cF^{\od}}


\newcommand{\bmu}{\boldsymbol{\mu}}
\newcommand{\col}{ \rm{color} } 
\newcommand{\uncol}{ \rm{uncolor} } 

\newcommand{\ma}{{\rm m}_0}

\newcommand{\cred}{ \color{red} } 

\newcommand{\cbu}{ \color{blue} } 

\newcommand{\dee}{{\mathbf d}}
\newcommand{\bee}{{\mathbf b}}

\newcommand{\ren}{\text{ren\,}}

\newcommand{\Tr}{{\rm Tr}}

\newcommand{\Sym}{ {\rm Sym} } 


\newtheorem{lemma}{Lemma}
\newtheorem{conj}{Conjecture}
\newtheorem{remark}{Remark}
\newtheorem{definition}{Definition}
\newtheorem{theorem}{Theorem}
\newtheorem{proposition}{Proposition}
\newtheorem{corollary}{Corollary}


\newcommand{\bea}{\begin{eqnarray}}
\newcommand{\eea}{\end{eqnarray}}
\newcommand{\beq}{\begin{equation}}
\newcommand{\eeq}{\end{equation}}

\newcommand{\be}{\begin{equation}}
\newcommand{\ee}{\end{equation}}

%%%%%%%%%%%%%%

\baselineskip 16pt \oddsidemargin 0pt \evensidemargin 0pt \topmargin
0pt \headheight 0pt \headsep 0pt \footskip 32pt \textheight
40\baselineskip \advance \textheight by \topskip \textwidth 470pt
\makeatletter

%%%%%%%%%%%%%%


\begin{document}




\begin{titlepage}
\begin{flushright}
\end{flushright}

\vspace{20pt}

\begin{center}

{\Large\bf Beta-functions of enhanced quartic tensor field theories \\

\medskip

}
\vspace{15pt}

{\large Joseph Ben Geloun$^{a,c,\dag}$ and Reiko Toriumi$^{b,\ddag} $}

\vspace{15pt}

$^{a}${\sl Laboratoire d'Informatique de Paris Nord UMR CNRS 7030}\\
{\sl Universit\'e Paris 13, 99, avenue J.-B. Clement, 93430 Villetaneuse, France} \\

\vspace{5pt}

$^{b}${\sl Okinawa Institute of Science and Technology }\\
{\sl 1919-1 Tancha, Onna-son, Kunigami-gun, Okinawa, Japan 904-0495 }

\vspace{5pt}

$^{c}${\sl International Chair in Mathematical Physics 
and Applications\\ (ICMPA-UNESCO Chair), University of Abomey-Calavi,\\
072B.P.50, Cotonou, Rep. of Benin}\\

\vspace{5pt}

E-mails:  {\sl $^{\dag}$bengeloun@lipn.univ-paris13.fr,
$^\ddag$reiko.toriumi@oist.jp}


\vspace{10pt}


\begin{abstract}
Enhanced tensor field theories (eTFT) have dominant graphs 
that do not correspond to  melonic diagrams or ordinary tensor field theories
(TFT). They therefore describe pertinent candidates to escape the 
so-called branched polymer phase, the universality geometry found for  tensor models. For generic rank $d$ of the tensor field,  we compute the perturbative $\beta$-functions at one-loop of two just-renormalizable quartic eTFT coined by $+$ or $\times$,  depending on the combinatorics of their vertex weights. 
Both models, $+$  and $\times$, have two quartic coupling constants
$(\lambda, \lambda_{+})$ and $(\lambda, \lambda_{\times})$,  
respectively, other 2-point couplings, (mass, $Z_a$) for $+$ and 
(mass, $Z_a$, $Z_{2a}$) for $\times$. 
At all loops,
both models have a constant wave function renormalization: 
$Z=1$ 
and therefore no anomalous dimension. This signals the
fact that these quantum models could behave like classical ones, 
however, we find that there are nontrivial renormalizations to couplings in our models.
The RG flow of the first model coined by $+$ exhibits neither asymptotic
freedom nor the ordinary Landau ghost of $\phi^4_4$ model: $\lambda_{+}$ is a fixed point and $\lambda$ has linear behavior in time scale $t =  \log(k/k_0)$. 
The mass behaves likewise (linear in $t$) in the UV, 
and diverge in the IR, whereas $Z_a$ decreases exponentially in the UV and diverge in the IR, exactly like any relevant operator. 
For the second model $\times$, both $\lambda$ and $\lambda_{\times}$ do not flow, whereas all remaining 2-point coupling constants
are linear function of the time scale. 
\end{abstract}

 
\today


\end{center}




\noindent  Pacs numbers:  11.10.Gh, 04.60.-m, 02.10.Ox
\\
\noindent  Key words: Renormalization group, beta-function, tensor models, tensor field theories, 
quantum gravity

\bigskip 


\setcounter{footnote}{0}


\end{titlepage}

%%%%%%%%%%%%%%%%%%%%%%%


\tableofcontents




\section{Introduction}
\label{intro}


In the search for a gravitational theory resulting from the random generation of geometries, the most notable success to date is restricted in two dimensions \cite{Di Francesco:1993nw}. The models at the base of this success are those of the celebrated random matrices. These statistical models generate simplicial complexes of dimension 2 and, therefore, discretized random surfaces. An impressive number of results exist on these models (KPZ equation, integrable models, string theory, etc) 
\cite{Knizhnik:1988ak} 
\cite{David:1988hj}
\cite{kawai} 
\cite{Morozov:1995pb}
\cite{thooft}.
More recently, it has been proven \cite{legallmiermont} \cite{Miller2015LiouvilleQG} that the continuous limit of such random surfaces provide a spherical geometry (2D) endowed with a conformal field called Liouville (the Brownian sphere).
Building on the success of statistical matrix models, models of random tensors  
\cite{razvanbook} \cite{adrianbook}
generate discrete random geometries in higher dimensions and
aim at  extending these results  to higher dimensions \cite{ambj3dqg} \cite{mmgravity} \cite{sasa1} 
\cite{boul}.  
However, their continuous limit is singular and of (Hausdorff) dimension less than 2 
\cite{Bonzom:2011zz}.
The generated continuous space is called the branched polymers. These cannot be used for the geometric description of our 4-dimensional space-time. 

Let us take a closer look at these tensor models and their universality classes. By now it is known that the continuous limit of tensor statistical models is dominated by a class of graphs called melons. 
They topologically define a subclass of spheres and form a tree structure that characterizes a so-called branched polymer phase 
\cite{Gurau:2013cbh}. 
It is a striking feature of tensor patterns that such melons dominate so universal and can be studied analytically. Nevertheless, in the context of quantum gravity, this class of universal branched polymers is undesirable, therefore, we aspire to escape from this branched polymer phase.

In order to improve its critical behavior, an interesting idea then emerged: one could adjust the scale $N$, or the size of the indices of the tensors, considering constants of couplings of non-melonic interactions of particular scales in such a way that a broader class of graphs, including non-melonic ones, contributes to the critical behavior of tensor models \cite{Bonzom:2015axa}. 
Thus, making the graphs that were previously suppressed contribute to the large scale limit $N$ defines the working hypothesis now put forward. We call the models of such a  program {\it enhanced tensor models}, and a model exhibiting a phase transition to a two-dimensional geometry was found \cite{Bonzom:2015axa}. 
Therefore, it is possible that enhanced tensor models help to produce a different phase than branched polymers.  
However, notably, the counterpart of these results in quantum field theory (QFT) still remains a vast territory to explore.

QFT is one of the most successful languages of modern physics. It describes with incredible precision complex systems with an infinity of degrees of freedom via the renormalization group. 
The most illustrious prediction of the QFT is that of the measurement of the fine structure constant of electromagnetism. 
In a different perspective from statistical models whose degrees of freedom are fixed, the QFT, characterized by a nontrivial propagator, unfolds a flow of a model from microscopic (ultraviolet) scales to macroscopic (infrared) scales. 
We consider this point of view to be relevant for the generation of random geometries: if the degrees of freedom, or modes of the tensor fields, are geometric (like spacetime atoms) 
and the presence of the nontrivial kinetic term induces the flow from UV to IR moved by dynamics in the ultraviolet, we consider that going towards the infrared, these degrees could agglomerate and form new degrees of freedom such as during a phenomenon of condensation 
\cite{oriti}
\cite{Oriti:2006ar} 
\cite{Marchetti:2022igl} \cite{Eichhorn:2017xhy}
\cite{BenGeloun:2011rc} .
The fact that some theories of tensor fields, or QFT of tensor fields, are asymptotically free, and therefore act like Quantum Chromodynamics, brings one more stone to this building 
\cite{Geloun:2013saa} \cite{BenGeloun:2013vwi}
\cite{Rivasseau:2015ova}.






Indeed, enhanced tensor field theories are proven renormalizable 
\cite{BenGeloun:2017xbd} 
\cite{Geloun:2015lta} and non-melonic graphs contribute to the flow of the renormalization group. 
Therefore, they are crucial candidates for generating a geometric phase different from that of branched polymers and those of 2D geometries.

Studying tensor field theories requires adopting powerful renormalization methods of QFT to extract their critical behavior. The renormalization of a tensor field theories is complex because it is a nonlocal field theory 
\cite{BenGeloun:2011rc} \cite{Geloun:2014ema} 
\cite{Carrozza:2013mna} 
\cite{Geloun:2012fq}
\cite{BenGeloun:2012pu}
\cite{BenGeloun:2012yk}.
Additionally, one needs to carefully treat the dimensions of coupling constants in order to analyze their flows
\cite{Carrozza:2014rba} 
\cite{Carrozza:2014rya}
\cite{Benedetti:2014qsa}.



In this work, we take further the analysis of \cite{BenGeloun:2017xbd} and compute $\beta$-functions of two eTFT models. One model is called $+$ and
the second model is labeled by $\times$. They were shown to be perturbatively just-renormalizable, at all orders, for given sets of parameters $(d,D,a,b)$, where $d$ is the tensor field rank, 
$D$ is the background group dimension, 
$b$ parametrizes the kinetic term $\sim p^{2b}(\phi_{p})^2$, 
and $a$ determines the vertex weight $\sim p^{2a}(\phi_{p})^4$. 
The $\beta$-function of model $+$ is studied for generic rank 
$d\ge 3$ of the tensor field and $(D=1, a=(d-2)/2, b=(d-3/2)/2)$. 
The second model $\times$ has fixed parameters
 $(D=1,d=3, a=1/2, b=1)$. 
Both models show exotic features compared to the ordinary TFT. 

Our results are the following: 

(1) In both models,  the wave function renormalization is constant $Z=1$ and therefore the anomalous dimension is vanishing. Although this hints at a classical behavior,  there is still a RG flow; 
the system of $\beta$-functions is remarkably simple 
and explicitly integrable 
at one loop;
we give the solutions of all 
equations in terms of the so-called  time scale ($\log k$). 
Based on the power counting theorem of each model and given the type of corrections that appear, we  conjecture that the system of $\beta$-functions
of each model can be explicitly 
solved at all orders of perturbation theory. This is a
genuine feature of these quantum models.  


(2) In the model $+$, one quartic coupling is still running and integrates
as a linear function of the time scale. There is therefore running of the coupling and an evolution. 
The second coupling does not flow: this was expected
as there were no divergent terms associated with that coupling. 
For generic non vanishing couplings, 
this behavior is neither comparable to asymptotic freedom, 
 asymptotic safety,  nor Landau ghost. 
The model $+$ has a mass and another 2-point coupling $Z_a$
for a quadratic term $p^{2a}(\phi_{p})^2$ (which is not the wave function 
renormalization, and was introduced to stabilize the renormalization). 
They behave just like relevant operators and tend to blow up in the IR; 
however, at the contrary of relevant operators of ordinary QFT, the mass 
 grows linearly in the UV and therefore is also relevant in that 
direction. $Z_a$ on the other hand, is suppressed in the UV, so behaves just like a relevant operator in the UV as well as IR;  

(3) Concerning the model $\times$, the  quartic couplings do not flow for the same reason above: no divergences are associated with these couplings. Inspecting the flow of the mass and the 2-point couplings $Z_a$ and $Z_{2a}$ introduced
to make sense of a renormalization analysis, they all behave linearly
in terms of the time scale. 



The paper is organized as follows.
In section \ref{rev}, we review the essense of \cite{BenGeloun:2017xbd} and present the two models named the model $+$ and the model $\times$.
In section \ref{betaf}, we compute the $\beta$-functions of the model 
$+$ at 1-loop and interpret the results. 
The next section \ref{betat} presents the same analysis for the 
the model $\times$. 
The conclusion in section \ref{conc} provides a summary of our work 
and present some perspectives. 
In appendix \ref{app:SumtoInt}, we  detail the integral approximation techniques 
that we will use to tackle the spectral sums which appear in the calculation of the $\beta$-functions. In appendices \ref{app:2ndO+} and \ref{app:2ndOx}, we illustrate the divergent graphs at second order in perturbation theory.





\section{Review of the renormalizable enhanced quartic TFT}
\label{rev}


We review here the main 
results of  \cite{BenGeloun:2017xbd}
introducing a class of enhanced quartic TFTs parametrized by 
$(D,d,a,b)\in \N\times \N\times\R_+\times\R_+$. 
In the following sections, we will specify the 4-tuple dealing only 
with specific models. 

\subsection{Enhanced TFT models}
\label{sec:etftmodels}

Consider a field theory defined by a complex function $\phi: (U(1)^{D})^{\times d} \to \C$. 
The Fourier transform of this complex field yields 
 a rank $d$ complex  tensor $\phi_{\bf P}$, with ${\bf P}=(p_1,p_2,\dots,p_d)$ a multi-index. 
$\bar\phi_{\bf P}$ denotes its complex conjugate. 
 Note that in the notation $\phi_{\bf P}$ the indices $p_s$ define themselves multi-indices: 
\be
p_{s} =(p_{s,1},p_{s,2},\dots,p_{s,D}) \,,\; \, p_{s,i}  \in \Z\,. 
\ee

A tensor field theory action  $S$ is built by a sum of convolutions of the tensors 
$\phi_{\bf P}$ and $\bar\phi_{\bf P}$:  
\bea\label{eq:actiond}
&&
S[\bar\phi,\phi]=\Tr_2 (\bar\phi \cdot 
{\bf K}
\cdot \phi) 
+ \mu
\, \Tr_2 (\phi^2) + S^{\inter}[\bar\phi,\phi]\,, 
\cr\cr
&&
\Tr_2 (\bar\phi \cdot 
{\bf K}
\cdot \phi) =
\sum_{{\bf P}, \, {\bf P}'} \bar\phi_{{\bf P}} \, 
{\bf K}({\bf P};{\bf P}')
\, \phi_{{\bf P}' } \,, 
\qquad 
\Tr_{2}(\phi^2) = \sum_{{\bf P}} \bar\phi_{{\bf P}}\phi_{{\bf P}}\,, 
\label{eq:generalaction}
\eea
and where $S^{\inter}[\bar\phi,\phi]$ is another convolution involving, 
in the case we are focusing on, 4 tensors. 

 
 Hence, giving  ${\bf K}$ and   $S^{\inter}[\bar\phi,\phi]$ entirely determines 
 the models. For a real parameter $b\ge 0$, we  introduce the class of kernels  
 ${\bf K} ={\bf {K}}_b({ \bf P}; {\bf P'} ) 
=\bdel_{ { \bf P}; {\bf P'} } {\bf P}^{2b}$ where 
\bea
\label{delmom}
&&
\bdel_{ { \bf P}; {\bf P'} } =    \prod_{s=1}^d\prod_{i=1}^D  \delta_{ p_{s,i}, p'_{s,i}}  \,,\qquad
{\bf P}^{2b} = \sum_{s=1}^d |p_s|^{2b}\,,  \qquad  |p_{s}|^{2b} = \sum_{i=1}^D |p_{s,i}|^{2b} \,,
\eea
where $ \delta_{ p,q}$ denotes the usual Kronecker symbol on $\Z$. 
It is clear that ${\bf K}_b$ represents 
a power sum of eigenvalues of  $d$ Laplacian operators over the $d$ copies of $U(1)^D$
 ($b=1$ precisely  corresponds  to Laplacian 
eigenvalues on the torus). 
Seeking renormalizable theories \cite{BenGeloun:2017xbd}, more freedom
for $b$ values, allowing even values different from integers, leads to interesting models. 
In ordinary quantum field theory (QFT),   the restriction  
$b\leq 1$ ensures the Osterwalder-Schrader (OS)
positivity axiom \cite{Rivasseau:1991ub,Rivasseau:2011hm}.   
In any case,  $b$ will be set as a strictly positive real parameter
with no other restriction. 

We concentrate on the interaction part. Given a parameter $a\ge 0$,  
we distinguish the following quartic interactions: 
\bea
&& 
\Tr_{4;1}(\phi^4) = \sum_{p_{s},  p'_{s} \in  \Z^D}
\phi_{12\dots d} \,\bar\phi_{1'23\dots d} \,\phi_{1'2'3'\dots d'} \,\bar\phi_{12'3'\dots d'} 
\,, 
\label{phi4sim}\\
&&
\Tr_{4;1}([p^{2a}+p'^{2a}]\,\phi^4) = \sum_{p_{s}, p'_{s} \in \Z^D} 
\Big( |p_{1}|^{2a} + |{p'}_{1}|^{2a}\Big)\phi_{12\dots d} \,\bar\phi_{1'23\dots d} \,\phi_{1'2'3'\dots d'} \,\bar\phi_{12'3'\dots d'}
\,, \cr\cr
&& = 
2\sum_{p_{s}, p'_{s} \in \Z^D} 
|p_{1}|^{2a}\,\phi_{12\dots d} \,\bar\phi_{1'23\dots d} \,\phi_{1'2'3'\dots d'} \,\bar\phi_{12'3'\dots d'} = 2\,\Tr_{4;1}(p^{2a}\,\phi^4) 
 \,,
\label{intplus}  \\ 
&&
\Tr_{4;1}([p^{2a}p'^{2a}]\,\phi^4) = \sum_{p_{s}, p'_{s} \in \Z^D} 
\Big( |p_{1}|^{2a}  |{p'}_{1}|^{2a}\Big)\phi_{12\dots d} \,\bar\phi_{1'23\dots d} \,\phi_{1'2'3'\dots d'} \,\bar\phi_{12'3'\dots d'}
\,.
\label{intprod} 
 \eea
 where the notation $\phi_{12\dots d} $ stands for $\phi_{p_1,p_2,\dots, p_d} = \phi_{\bf P}$.
In the above equations \eqref{phi4sim}, \eqref{intplus} and \eqref{intprod}, the color index 1 plays
a special role.  The sum over all possible color configurations delivers a colored symmetric  interactions: 
\bea
&& 
 \Tr_{4}(\phi^4)
 :=  \Tr_{4;1} (\phi^4) + \Sym (1 \to 2 \to \dots \to d) 
\,, \label{eq:3dinter} \\
&& 
\Tr_{4}(p^{2a}\,\phi^4)
 :=  
 \Tr_{4;1} (p^{2a}\,\phi^4)
 + \Sym (1 \to 2 \to \dots \to d) \,, 
 \label{eq:ourinteraction+}\\
&& 
\Tr_{4}([p^{2a}p'^{2a}]\,\phi^4)
 :=  
 \Tr_{4;1} ([p^{2a}p'^{2a}]\,\phi^4)+ \Sym (1 \to 2 \to \dots \to d) 
 \,. 
\label{eq:ourinteractionx}
\eea
One might consider 
the momentum weights  in the above interactions as  derivative couplings
for particular choices of $a$. 
Seeking just renormalizable models in our theory space,  we will constrain $a$ 
{\it a-posteriori} to some particular 
values. 
 It turns out that  the two interactions \eqref{eq:ourinteraction+}
and \eqref{eq:ourinteractionx} generate two new 2-point diverging graphs
that neither the mass nor the wave function renormalization can absorb. 
They are of the form: 
\bea
 \Tr_2 (p^{2\xi}\phi^2) = 
\Tr_2 (\bar\phi \cdot 
{\bf K}_{\xi}
\cdot \phi)\,, \qquad \xi = a,2a\,,
\eea
In \cite{BenGeloun:2017xbd}, the authors handled these
by adding them in kinetic term, in addition 
to the former term $\Tr_2 (p^{2b}\phi^2)$. 
The models that proves renormalizable have the following kinetic terms and  interactions: 
\bea
\text{model }+: 
&&
S^{\inter}_+[\bar\phi,\phi] =  \frac{ \lambda}{2}\,\Tr_{4}(\phi^4)  
 +\frac{\pet}{2}\, \Tr_{4}(p^{2a}\,\phi^4)
  + CT_{2}[\bar\phi,\phi] +\sum_{\xi=a,b}CT_{2;\xi}[\bar\phi,\phi] 
 \cr\cr
&& 
S^{\kin}_+[\bar\phi,\phi] =  
\sum_{\xi=a,b}  
Z_\xi   \Tr_2 (p^{2 \xi} \phi^2) 
 + \mu \Tr_2 (\phi^2) \,,
\label{model1} \\ 
\text{model }\times: &&
S^{\inter}_\times[\bar\phi,\phi] =   \frac{ \lambda}{2}\,\Tr_{4}(\phi^4)  
 +\frac{\lambda_\times}{2}\, \Tr_{4}([p^{2a}p'^{2a}]\,\phi^4)
 + CT_{2}[\bar\phi,\phi] +\sum_{\xi=a,2a,b}CT_{2;\xi}[\bar\phi,\phi]  \,,
 \cr\cr
&& 
S^{\kin}_\times[\bar\phi,\phi] =  
\sum_{\xi=a,2a,b} Z_\xi  \Tr_2 (p^{2 \xi} \phi^2) 
 + \mu \Tr_2 (\phi^2)
\label{model2}
\eea
where  $\lambda$, $\pet$, $\lambda_\times$ are coupling constants, 
$CT_{\cdot}$ are counterterms, $\mu$ is the mass coupling, $Z_a$ and $Z_{2a}$
are other kinetic term couplings,  and $Z_b$ will be called the wave function
renormalization. 

Note that we could be also interested 
in a model $\lam=\pet$, in which case these
couplings merge in a single one. 
We will address this after the extraction
of the $\beta$-function. 

Another important issue is the following: the choice of modifying
the covariance of the theory using a kinetic
term as $\sum_{\xi}   Z_\xi   \Tr_2 (p^{2 \xi} \phi^2) 
 + \mu \Tr_2 (\phi^2)$ make computations  lengthier. 
Indeed, the analysis of the amplitude could be made easier
by keeping a kinetic term as  $  Z_b   \Tr_2 (p^{2 b} \phi^2) 
 + \mu \Tr_2 (\phi^2)$ and let the extra term $ Z_a  \Tr_2 (p^{2 a} \phi^2) $ in the interaction
 part. This makes the covariance much simpler. We will show that
both model share the same power counting and the renormalization analysis
applies equally well on both. The $\beta$-function will be computed with 
the second kind of models (with simpler covariance). 
 




\subsection{Amplitudes and renormalizability}
\label{sect:perturb}
The models $+$ and $\times$ associated with the actions given 
by  \eqref{model1} and \eqref{model2}, respectively, give
the quantum models determined by the partition function
\be
Z_\bullet  = \int d\nu_{C_\bullet}(\bar\phi,\phi) \; e^{-S^{\inter}_{\bullet}[\bar\phi,\phi]}\,, 
\ee
where $\bullet =+,\times$, and $d\nu_{C_\bullet}(\bar\phi,\phi)$ is a field Gaussian measure with covariance $C_\bullet$
given by the inverse of the kinetic term:
\be
C_\bullet({\bf P};{\bf P'}) =\tilde{C}_\bullet({\bf P})\, \bdel_{{\bf P},{\bf P}'}\,,\qquad
\tilde{C}_\bullet({\bf P})\,= \frac{1}{\sum_\xi {\bf P}^{2 \xi}+ \mu }\,. 
\label{eq:cov}
\ee 
where, if $\bullet =+$, $\xi=a,b$ and if $\bullet =\times$,  $\xi=a,2a,b$.

\

\noindent 
{\bf Feynman graphs  in TFTs.}
Feynman graphs of TFTs and enhanced TFTs  have 
 two equivalent  representations. One is called ``stranded graph'' representation 
 and  incorporates more details of the structure
of the Feynman graph. The other representation of a Feynman
graph in this theory is  a bipartite colored graph \cite{Bonzom:2012hw,Gurau:2011xp}.
 
The propagator is drawn as a set of $d$ non-intersecting segments
called strands, or as a dotted line (see Figure \ref{fig:propacol}). 
\begin{figure}[H]
\centering
     \begin{minipage}[t]{0.7\textwidth}
\begin{tikzpicture}
\node (r1) at (-0.3,0.0)  {$p_{1}$};  
\node (r2) at (-0.3,0.3)  {$p_{2}$};
\node (r1) at (-0.3,0.8)  {$\vdots$};  
\node (r3) at (-0.3,1)  {$p_{d}$};  
\draw (0,0.3) -- (4,0.3) ;
\draw (0,1) -- (4,1) ;
\draw (0,0) -- (4,0) ;
%%
\draw[dashed] (6,0.5) -- (10,0.5);
\end{tikzpicture}
\caption{\small  Two ways of representing the propagator of the theory.}
\label{fig:propacol}
\end{minipage}
\end{figure}
 In Figure \ref{fig:4vertex}, an interaction is pictured as a stranded vertex 
 (pictures above) or by a $d$-regular colored bipartite graph (pictures below).
We list therein all possible quadratic and quartic vertices
and a bold line represents a momentum weight of the term. 
\begin{figure}[H]
 \centering
     \begin{minipage}{1\textwidth}
     \centering
\includegraphics[angle=0, width=14cm, height=6cm]{vertex_total.pdf}
\caption{ {\small Rank $d$  vertices of the mass, $\phi^2$- and $\phi^4$-terms.   
}} 
\label{fig:4vertex}
\end{minipage}
\end{figure}
In Figure \ref{fig:graphs}, we give some examples of two 4-pt graphs. 
\begin{figure}[H]
 \centering
     \begin{minipage}{1\textwidth}
     \centering
\includegraphics[angle=0, width=12cm, height=5cm]{graph2_total.pdf} 
%
\caption{ {\small Rank $d=3$  Feynman graphs.   }} 
\label{fig:graphs}
\end{minipage}
\end{figure}



At the perturbative level, we compute the amplitude of a Feynman 
graph $\cG(\cV,\cL)$ with set of vertices $\cV$ and set of propagator 
line $\cL$,  in the standard way: 
\bea
A_{\cG}(\{p_{\ext}\}) = \sum_{{\bf P}_v} \prod_{l \in \cL} C_{\bullet, l} ({\bf P}_{v},{\bf P}'_{v'} )
\prod_{v \in \cV} (-{\bf V}_v({\bf P}_v) ) 
\eea
where  $C_{\bullet, l}$ is a propagator with line index $l$, ${\bf V}_v({\bf P}_v)$
is a given vertex weight that contains a coupling constant but also a momentum 
weight if the vertex $v$ is enhanced.  The sum is performed over internal momenta 
and will follow the ordinary momentum routine whereas the set $\{p_{\ext}\}$ defines external momenta that are not summed over. 


Some notation is needed to distinguish the different vertices
and their weight that we shall deal with: 

 -  the set $\cV_{4;s}$ of vertices 
 $\Tr_{4;s}(\phi^4)  $ with color $s$ and with kernel  ${\bf V}_{4;s}$, 
$\cV_4 = \sqcup_{s=1}^d \cV_{4;s}$ (disjoint union notation); 

-  the set $\cV_{+; 4;s}$ of vertices 
$\Tr_{4;s}(p^{2a}\,\phi^4)$ with color $s$ and 
with vertex kernel  ${\bf V}_{+;4;s}$,
$\cV_{+;4}= \sqcup_{s =1}^d\cV_{+; 4;s}$; 

-  the set $\cV_{\times; 4;s}$ of vertices 
$\Tr_{4;s}(([p^{2a}p'^{2a}]\,\phi^4)$
with vertex kernel  ${\bf V}_{+;4;s}$,
$\cV_{\times;4}= \sqcup_{s =1}^d\cV_{\times; 4;s}$; 

- the set $\cV_2$  of mass vertices with kernel 
${\bf V}_{2}$,
 the set  $\cV_{2;\xi;s}$ of vertices 
$ Z_\xi   \Tr_2 (p^{2 \xi} \phi^2) $ each kind corresponding to 
a kernel ${\bf V}_{2;\xi;s}$,
 $\cV_{2;s} = \cup_{\xi} \cV_{2; \xi;s}$. 
  
We denote
the cardinalities
$|\cV_{4;s}|=V_{4;s}$,
$|\cV_{4}|=V_4$,
$|\cV_{\bullet;4;s}|=V_{\bullet;4;s}$, 
$|\cV_{\bullet;4}|=V_{\bullet;4}$, 
$\bullet=+,\times$; 
$|\cV_{2;\xi;s}|=V_{2;\xi;s}$,  $V_{2;\xi}= \sum_s V_{2;\xi;s}$. 
Then, 
$\cV = \sqcup_{s =1}^d (\cV_{4;s} \cup \cV_{\bullet;4; s} \cup \cV_{2;s})$,
$|\cV|=V$. 



\ 


\noindent{\bf Power counting theorems
and list of divergent graphs.}
We restrict now to primitively divergent graphs. 
Given $a\le b$, the degree of divergence of 
a graph amplitude of the model $+$ is given by 
\bea
&&
\omp(\cG) =
\crcr
&& - {2 D \over (d-1)!} ( \omega(\cG_{\rm color}) - \omega (\partial \cG))  - D (C_{\partial \cG} - 1) 
-{1\over 2} \left[ ( D \,  (d-1) - 2 b) N_{\ext} - 2 D \,  (d-1) \right] 
\crcr
&&+ {1 \over 2} \left[ -2 D\,  (d-1) + (D \,  (d-1) - 2 b) n \right] \cdot V  + 2 a \rhop
+2a\rho_{2;a}+ 2b \rho_{2;b}  \,.
\eea
where $ \omega(\cG_{\rm color})$ is the Gurau degree \cite{Gur3,Gur4} of the extended 
colored graph $\cG_{\rm color}$ of $\cG$, $\partial \cG$ is the boundary graph 
of $\cG$ \cite{BenGeloun:2011rc}, $C_{\partial \cG}$ is the number of connected component of $\partial \cG$, $N_{\ext}  $ is the number of external legs of the diagram $\cG$, 
$ n \cdot V = \sum_{k} k V_k$ where $k$ is the valence of the vertex $V_k$, 
$\rhop$ is the number of times that a momentum is enhanced passing 
through a strand of an enhanced vertex $\cV_{\bullet;4}$;
likewise $\rho_{2;\xi}$, $\xi=a,b$ is the number of times that a momentum gets an enhancement
passing through a vertex $\cV_{2;\xi;s}$. 


Concerning the model $\times$, 
one obtains the degree of divergence in this model valid for $3a\leq 2b$,
\bea
\omt(\cG)
 &=&  - {2 D \over (d^-)!} ( \omega(\cG_{\rm color}) - \omega (\partial \cG))  - D (C_{\partial \cG} - 1) 
-{1\over 2} \left[ ( D \, d^- - 2 b) N_{\ext} - 2 D \, d^- \right] 
\crcr
&&+ {1 \over 2} \left[ -2 D\, d^- + (D \, d^- - 2 b) n \right] \cdot V  + 2 a \rhot
+\sum_{\xi=a,2a,b}2\xi\rho_{2;\xi} \,.
\eea

The following statement has been proved in \cite{BenGeloun:2017xbd}. 

\begin{proposition}[List of primitively divergent graphs for the model $+$]
\label{prop:list+}
The $p^{2a}\phi^4$-model $+$ with parameters $a=D(d^--1)/2, b=D(d^--\frac12)/2$
for two integers $d>2$ and $D>0$, has primitively  divergent graphs  with $(\Omega(\cG)=\omega(\cexG) - \omega(\bG))$:


\begin{table}[H]
\centering
\begin{tabular}{lcccccccccccccccc}
\hline\hline
$\cG$ && $N_{\ext}$ && $V_{2}$ &&  $V_{2;a}$  && $V_{4}$ && $\rhop$   && $C_{\bG}-1$ && $\Omega(\cG)$ && $\omp(\cG)$  \\
\hline\hline
 && 4 && 0 && 0 && 0 && $V_{+;4}$ && 0 && 1&& 0\\
I && 2 && 0  && 0  && 0 && $V_{+;4}$ && 0 && 1 && ${D \over 2}$  \\
II && 2 && 0  && 0 && 0 && $V_{+;4}-1$  && 0 && 0 && ${D \over 2}$ \\
III && 2 && 0  && 0 && 1 && $V_{+;4}$  && 0 && 0 && ${D \over 2}$ \\
IV && 2 && 0  && 1  && 0 && $V_{+;4}$ && 0 && 1 && $0$  \\
V && 2 && 0  && 1 && 0 && $V_{+;4}-1$  && 0 && 0 && $0$ \\
VI && 2 && 0  && 1 && 1 && $V_{+;4}$  && 0 && 0 && $0$ \\
\hline\hline
\end{tabular}
\caption{List of primitively divergent graphs of the $p^{2a}\phi^4$-model $+$.} 
\label{tab:listprim1}
\end{table}

\end{proposition}

Having a look at Table \ref{tab:listprim1} and the row labeled by $\Omega(\cG)$, 
we see that the dominant graphs are not those labeled by $\Omega(\cG)=0$ which are the melonic diagrams, but are those that are $\Omega(\cG)=1$ and thus non-melonic graphs. 
This shows that the model $+$ delivers the expected output. 

\begin{theorem}
\label{theorem+}
The $p^{2a}\phi^4$ model $+$  with parameters $a=D(d^--1)/2, b=D(d^--\frac12)/2$ 
for arbitrary rank $d\ge 3$ and dimension $D>0$ with action defined by \eqref{eq:actiond} is just-renormalizable at all orders of perturbation
theory. 
\end{theorem}



\begin{proposition}[List of primitively divergent graphs for  the model $\times$]
\label{prop:listx}
The $p^{2a}\phi^4$-model $\times$ with parameters $D = 1, d=3,a={1 \over2}, b=1$, 
has the following primitively divergent  graphs which  obey
$(\Omega(\cG)= \omega(\cexG) - \omega(\bG))$
\begin{table}[H]
\begin{center}
\begin{tabular}{lcccccccccccccccc}
\hline\hline
$\cG$ && $N_{\ext}$ && $V_{2}$  && $V_{2;a}$ && $ V_4$ && $\rhot$   && $C_{\bG}-1$ && $\Omega(\cG)$ && $\omt(\cG)$  \\
\hline\hline
I && 2 && 0  && 0 && 0 &&  $2 V_{\times;4} -1$ && 0 && 1 && ${0}$  \\
II && 2 && 0  && 0 && 0 && $2 V_{\times;4}-2$  && 0 && 0 && ${0}$ \\
III && 2 && 0  && 0 && 1 && $2 V_{\times;4}$  && 0 && 0 && ${0}$ \\
\hline\hline
\end{tabular}
{\caption{List of primitively divergent graphs of the $p^{2a}\phi^4$-model  $\times$. \label{tab:listprim2}}}
\end{center}
\end{table}
\end{proposition}

\begin{theorem}
\label{theoremx}
The $p^{2a}\phi^4$ model $\times$  with parameters $D = 1, d=3,a={1 \over2}, b=1$,  
with the action defined by \eqref{eq:actiond}  is renormalizable at all orders of perturbation.
\label{theorenx}
\end{theorem}

This model $\times$ has an unexpected behavior: it does not have  divergent 4pt-graphs,
only log-divergent 2pt-graphs. One may ask if this is a super-renormalizable 
model with a finite number of divergent graphs. The answer is no because the model has  infinite  terms participating in the mass flow. The property can be 
regarded as a particularity entailed by both the nonlocality and the presence of enhanced
vertices  which make the mass behave like a marginal coupling. 



\subsection{An alternative enhanced TFT model}

As stated previously, the presence of the enhanced vertices
generate new 2pt-terms with external data following the pattern 
of $ \Tr_2 (p^{2 a} \phi^2) $ for the model $+$, and 
the patterns of
 $ \Tr_2 (p^{2a} \phi^2) $ and
$ \Tr_2 (p^{4a} \phi^2) $ for the model $\times$. 
Consequently, in \cite{BenGeloun:2017xbd}, the authors have modified 
the  
covariance and performed 
the
renormalization analysis. 
We present here an alternative way of dealing with 
these terms that make the whole simpler. 
We simply demand that the new terms are interactions and 
therefore propose the following models: 
(in the following, we use the same notation as in the previous sections
as no confusion may arise)
\bea
\text{model }+: 
&&
S^{\inter}_+[\bar\phi,\phi] =  \frac{ \lambda}{2}\,\Tr_{4}(\phi^4)  
 +\frac{\pet}{2}\, \Tr_{4}(p^{2a}\,\phi^4)
 + Z_a   \Tr_2 (p^{2} \phi^2)  \crcr
 && 
  + CT_{2}[\bar\phi,\phi] +CT_{2;b}[\bar\phi,\phi] 
 \cr\cr
&& 
S^{\kin}_+[\bar\phi,\phi] =  
Z_b   \Tr_2 (p^{2b } \phi^2) 
 + \mu \Tr_2 (\phi^2) \,,
\label{qmodel1} \\ \cr
\text{model }\times: &&
S^{\inter}_\times[\bar\phi,\phi] =   \frac{ \lambda}{2}\,\Tr_{4}(\phi^4)  
 +\frac{\lambda_\times}{2}\, \Tr_{4}([p^{2a}p'^{2a}]\,\phi^4)
 + \sum_{\xi=a,2a} Z_\xi  \Tr_2 (p^{2 \xi} \phi^2) 
 \crcr
 &&
 + CT_{2}[\bar\phi,\phi] 
 +CT_{2;b}[\bar\phi,\phi]  \,,
 \cr\cr
&& 
S^{\kin}_\times[\bar\phi,\phi] =  
Z_b \Tr_2 (p^{2 b} \phi^2) 
 + \mu \Tr_2 (\phi^2)
\label{qmodel2}
\eea

Considering this proposal, the covariance of these models 
is unique and given by 
\be
C({\bf P};{\bf P'}) =\tilde{C}({\bf P})\, \bdel_{{\bf P},{\bf P}'}\,,\qquad
\tilde{C}({\bf P})\,= \frac{1}{
{\bf P}^{2 b}+ \mu }\,. 
\label{eq:newcov}
\ee 
The difference between this propagator \eqref{eq:newcov} and the former \eqref{eq:cov}
is that, for a given strand, the momentum which was previously
  $\sum_{\xi} p_s^{2\xi}$ becomes $p_s^{2b}$. 
The multiscale analysis of  \cite{BenGeloun:2017xbd} 
with appropriate parameter $M^{2b}$ ($M>1$)  can be mimicked 
with no difficulty to obtain the sliced propagator as
\bea
C_0 \le K \,, \qquad \quad 
C_i ({\bf P}) \le K M^{2bi } e^{\delta M^{-bi} ({\bf P}^{b}+ \mu)} 
\eea
for $i>0$ a high slice  index, and for some constant $K$ and $\delta$. 

To write the multiscale amplitude where each propagator
lives in an arbitrary slice is identical to the previous analysis 
\cite{BenGeloun:2017xbd} for both the model $+$ and the model $\times$. We need a convenient
bound on that quantity. One should arrive at the expression
of a sliced amplitude according to 
a momentum attribution $\bmu = (i_1, \dots i_{| \cL | })$: 
\bea
|A_{\cG; \bmu}| \leq  \tilde K  \prod_{l \in \cL} M^{-2bi_l }
\sum_{p_{f_s}} \prod_{f_s \in \ F_{\inter}} 
e^{-\delta (\sum_{l\in f_s} M^{-bi_l})  |p_{f_s}^{b}|}
\prod_{i=1}^d \prod_{v_s\in \cV_{\bullet;4;s} }
{\rm weight} (v_s)
\eea
where $\tilde K$ is a constant, 	and weight$(v_s)$ is self-explanatory and is identical 
to the former case. 

The rest of the analysis consists in performing the spectral sums
$\sum_{p_{f_s}} (\cdot)$. A close inspection of the method introduced
in \cite{BenGeloun:2017xbd} shows that one condition is imposed  
and gets rid of $a$ at leading order. The parameter $a$ does not contribute
to the sum. We have at $a\ge 0$:
\bea
\sum_{p_1; \dots p_D=1}^{\infty}
(\sum_{l=1}^{D}p_{l}^c )^n e^{-B (p^b + p^a)}
= k B ^{-\frac{cn + D}{b}} e^{-B^{1- \frac{a}{b}}} (1+ \cO(B^{\frac1b}))
\eea
Requesting $a\le b$ for the model $+$ and $3a\le 2b$ for the model $\times$
removes the parameter $a$ from the expansion at leading order in $B$. 
Then, the remaining part
of the momentum integration exactly performs in the same way. We  therefore reach the same power counting, the same list of divergent graphs, Propositions \ref{tab:listprim1} 
and \ref{tab:listprim2} are valid for the models $+$ and $\times$, respectively.
We perform the same subtraction procedure
making the models  \eqref{qmodel1} and \eqref{qmodel2} renormalizable 
at all orders of perturbation theory. Theorems \ref{theorem+} and \ref{theoremx}
hold for the models $+$ and $\times$, respectively. 

From this point onwards, we consider the models  \eqref{qmodel1}
and \eqref{qmodel2}. To simplify the  determination of the graph  combinatorial
factors, we will  distinguish all couplings by providing them with  colors. For instance, 
for the model $+$, we write:  
\bea
S'^{\inter}_+[\bar\phi,\phi] 
 := \sum_{c=1}^d \frac{\lac}{2} \; \Tr_{4;c}(\phi^4)
 + 
\sum_{c=1}^d \frac{\lapc}{2}\; \Tr_{4;c}(p^{2a}\,\phi^4) 
+\sum_{c=1}^d  Z_a^{(c)} \Tr_{2;c}(p^{2 a } \phi^2) \,.
\label{eq:ourinteraction}
\eea
In the end, the $\beta$-functions of 
$\lam$, $\pet$, and $Z_a$ will be directly inferred
by letting $\lac \to \lam$, $\lapc \to \pet$
and $ Z_a^{(c)} \to Z_a$. 
The same will be done for the model $\times$. 



\section{
One-loop beta-functions  of the model $+$ }
\label{betaf}

This section now addresses a first set of new results in this contribution: we determine the RG flow at 1-loop of the model $+$. We compute the $\beta$-function of the coupling $\lambda$, the coupling $Z_a$ and the mass. 
Since, the action contains a vertex weight and an extra quadratic coupling, we carefully carry out the formalism of finding the effective action and write the resulting RG flow equations.  

\subsection{Effective action}

The presence of several coupling constants in our theory urges us to handle the effective action with care. 
We will compute the renormalization of coupling constants via multiscale analysis.
We start with performing a slice decomposition of the covariance.
We let $M >1$ a positive real number and we define the (sharp) cutoff functions as
\begin{equation}
    \chi^0(\alpha) = 
    \begin{cases}
      1 & \text{if  $1 \le \alpha $}\\
      0 & \text{if $1 > \alpha$}
    \end{cases}  \,,
\label{eq:chi0}
\end{equation}

$\forall i >0$,
\begin{equation}
\chi^i(\alpha) = 
    \begin{cases}
      0 & \text{if $\alpha \le M^{-2 bi}$}\\
      1 & \text{if  $M^{-2 bi} < \alpha \le M^{-2 b(i-1)} $}\\
      0 & \text{if $ M^{-2 b(i-1)} < \alpha $}
    \end{cases}  
\label{eq:chii}
\end{equation}
The presence of $2b$ in 
the cutoff function is
related to the propagator momentum power and
has a normalization 
effect for the degree of divergence $\omp(\cG)$. 
Then we write the covariance in \eqref{eq:cov} as expressed in Schwinger parametrization,
\be
C({\bf P};{\bf P'}) =\tilde{C}({\bf P})\, \bdel_{{\bf P},{\bf P}'}\,,\qquad
\tilde{C}({\bf P})\,= \frac{1}{{\bf P}^{2 b}+ \mu }
=
\sum_{i =0}^{\infty} \tilde{C}_{i} ({\bf P})
\,,
\label{eq:C}
\ee
with
\begin{equation}
\tilde{C}_i ({\bf P})
=
\int_{0}^{\infty} d \alpha \, 
e^{- \alpha ({\bf P}^{2 b} + \mu)} \chi^i(\alpha)\,.
\label{eq:Ctilde}
\end{equation}
We integrate out the fields at high scales greater than $i$ and include their effects in the effective action $W^i$. 
As written below,
\begin{equation}
Z = \int d \nu_{C_{\le i}} (\bar\phi_{\le i}, { {\phi}}_{\le i})
\, 
e^{- W^i (\bar\phi_{\le i},\, { {\phi}}_{\le i})}\,,
\end{equation}
where
\begin{equation}
C_{\le i} ({\bf P}; {\bf P}') = \bdel_{{\bf P}, {\bf P}'}\sum_{j \le i} {\tilde C}_j({\bf P})\,.
\end{equation}
%
Following Wilsonian renormalization group idea, after integrating up to the scale $i$, we continue integrating the slice $i$ in order to obtain an effective action at scale $i-1$.
We can do this by using the property of Gaussian measure; we can readily decompose $C_{\le i} =  C_i + C_{\le i-1} $ and the corresponding fields
$\phi_{\le i} = \psi_i + \phi_{\le i-1}$ (${\bar \phi}_{\le i} = {\bar \psi}_i + {\bar \phi}_{\le i-1}$). Then the partition function becomes
\begin{equation}
Z = \int 
d \nu_{C_{\le i-1}}(\bar\phi_{\le i-1}, {\phi}_{\le i-1})
e^{-W^{i-1}( \bar\phi_{\le i-1}, \, { \phi}_{\le i-1})}\,,
\end{equation}
where the effective action at  scale $i-1$ is given by
\begin{equation}
e^{-W^{i-1}( \bar\phi_{\le i-1}, \, { \phi}_{\le i-1})}
=
\int 
d \nu_{C_i}(\bar\psi_{i}, { \psi}_{i}) e^{-W^i(\bar \psi_i + \bar \phi_{\le i-1}, \, {\psi}_i + { \phi}_{\le i-1})}\,,
\end{equation}
%
Note that here, in the case that the theory is renormalizable, one can assert the effective action at any scale $i$ takes the same form as the interaction action,
\begin{equation}
W^{i}(\bar\phi_{\le i},\phi_{\le i}) = S^{\inter}
(\bar\phi_{\le i}, \phi_{\le i})\,,
\end{equation}
We can, then,  write formally: 
\bea
&& 
- W^{i-1}(\bar\phi_{\le i-1}, { \phi}_{\le i-1})
=
 \Tr_2 (\bar \phi_{\le i-1} \cdot \Sigma
 \cdot \phi_{\le i-1} )  
+
\frac{1}{2} \Tr_4({\phi^4_{\le i-1}} \cdot \Gamma_4)
+ R(\phi_{\le i-1})
\,,
\label{eq:effectiveW}
\eea
where $\Sigma(\{p\})$ is the sum over all amputated 1PI 2-pt graphs, 
$\Gamma_4(\{p\})$ is the sum of 1PI 4-pt graphs following the pattern of $\Tr_4(\phi^4)$, and 
$R(\phi_{\le i-1})$ is the rest of the terms containing 1PR graphs (they do not contribute to the iteration process) and the finite terms.  In the above equation, 
$\Sigma$ and $\Gamma_4$ are kernels
that are convoluted with the tensors
fields. 


We separate the graph amplitudes into local and nonlocal parts, therefore,
\begin{equation}
\Sigma(\{p\}) 
= 
\Sigma (\{0\}) 
+ 
\sum_{c}
\vert p_{c}\vert^{2 b} \partial_{\vert p_{c}\vert^{2 b}} \Sigma \big\vert_{\{p\}=0}
+
\sum_{c}
\vert p_{c}\vert^{2 a} \partial_{\vert p_{c}\vert^{2 a}} \Sigma \big\vert_{\{p\}=0} 
+
\cdots
\label{eq:sigma} 
\end{equation} 
As a result of \cite{BenGeloun:2017xbd}, the renormalization analysis of the model dictated by the rows of I, III, IV, and VI in  Table \ref{tab:listprim1}  proved that 
$\partial_{\vert p_{c}\vert^{2 b}} \Sigma \big \vert_{\{p\}=0} =0 $ and the rest of the terms in $\cdots$ are finite, 
whereas the mass renormalization
$\Sigma (\{0\})  $
and
$\partial_{\vert p_{c}\vert^{2 a}} \Sigma\vert_{\{p\}=0} \equiv \Gamma^{(c)}_2 (\{0\})$ are divergent.
$\vert p_{c}\vert^{2 a}\Gamma^{(c)}_2(\{p\})$
is the sum of all  amputated 1PI (one-particle-irreducible) 2pt-functions following the pattern of 
${\rm Tr}_2 ( p_{c}^{2a} \phi^2)$ 
on their boundary graphs as dictated by II and V rows of Table \ref{tab:listprim1}.
The boundary of the graphs contributing to this function are all melonic without the $\vert p\vert^{2a}$-enhancement (see Proposition \ref{prop:list+} and lines with $\Omega(\cG) = 0$). 


Similarly, one can expand the contribution coming from $4$-point function
\begin{equation}
\Gamma_4(\{p\}) 
=
\sum_c 
\Big\{
\Gamma_4^{(c)}(\{0\})
+
\vert p_{c} \vert^{2 a}
\partial_{\vert p_{c} \vert^{2 a}} \Gamma_4^{(c)}\Big\vert_{\{p\} =0}
+
\vert p_{c} \vert^{2 b}
\partial_{\vert p_{c} \vert^{2 b}} \Gamma_4^{(c)}\Big\vert_{\{p\} =0}
\Big \}
+\cdots\,,
\label{eq:4ptfcn} 
\end{equation}
where $ \sum_c \Gamma_4^{(c)}(\{0\})
\equiv \Gamma_4(\{0\}) 
$ is the sum of all amputated 1PI  4pt-functions 
following the pattern of ${\rm Tr}_4 (\phi^4)$ on their boundary graphs  as dictated by the first row of Table \ref{tab:listprim1}.
We define
$
\partial_{\vert p_{c} \vert^{2 a}} \Gamma^{(c)}_{4}\big\vert_{\{p\} =0}
\equiv
\Gamma_{4;+}^{(c)}(\{0\})
$ which are all amputated 1PI 4pt-functions following the pattern of $ \Tr_{4;c}( p^{2a} \phi^4)$, with characteristics given by the first row of Table \ref{tab:listprim1} and having a boundary with external $\vert p\vert^{2a}$-enhancement (see Proposition \ref{prop:list+}). 
In fact, from Proposition \ref{prop:list+}, there is only the leading order ${\mathcal O}(\lambda_+)$ contribution in $\Gamma_{4;+}^{(c)}(\{0\})$ and there are no contributions from higher orders in perturbation theory
in $\lambda_+$.
Also, from the Proposition \ref{prop:list+}, the rest denoted by $\cdots$ and
 $\partial_{\vert p_{c} \vert^{2 b}} \Gamma_4^{(c)} \big \vert_{\{p\} =0}$ are finite.

Now, reorganizing and absorbing all the finite parts into ${\tilde R}({\phi_{\le i-1}})$,
however intentionally leaving the term with
$\partial_{\vert p_{c}\vert^{2 b}} \Sigma \big \vert_{\{p\}=0} $
even though it is zero,
\bea
- W^{i-1}(\phi_{\le i-1}, {\bar \phi}_{\le i-1})
&=&
\Sigma_{i-1} (\{0\})
\Tr_2 (\phi^2_{\le i-1})  
+
(\partial_{\vert{p_{c}\vert}^{2 b}} \Sigma \big \vert_{\{p\}=0})_{i-1}
\Tr_{2} (p^{2 b}\phi^2_{\le i-1}) 
\nonumber
\\
&&+
\sum_c
\Gamma^{(c)}_{2, \; i-1} (\{0\})
\Tr_{2;c}(p^{2 a}\phi^2_{\le i-1})
+
\sum_c
\frac{\Gamma^{(c)}_{4, \; i-1} (\{0\})}{ 2} \Tr_{4;c}({\phi^4_{\le i-1}})
\nonumber
\\
&&+ 
\sum_c
\frac{\Gamma^{(c)}_{4;+, \; i-1} (\{0\})}{ 2} \Tr_{4;c}( p^{2a}\phi^4_{\le i-1})
+
{\tilde R}(\phi_{\le i-1})\,.
\label{eq:effectiveW2}
\eea


The effective theory is then defined by a new measure given by
\bea
d \nu_{\tilde{C}^{i-1}(\phi_{\le i-1})}
\exp \Big[
{\Sigma_{i-1}(\{0\}) {\rm Tr}_2 (\phi_{\le i-1}^2) 
+ 
\sum_c 
(\partial_{\vert p_c\vert^{2 b}} \Sigma  \vert_{\{p\}=0})_{i-1}
{\rm Tr}_2( p_c^{2b} \phi_{\le i-1}^{2})}
\Big]
\,,
\eea
which is still a Gaussian measure.
Let us compute the new covariance for the above Gaussian measure, 
\bea 
\frac{1}
{Z_{b,\, i-1}}
\int_0^\infty
d \alpha \,
e^{- \alpha (\vert p \vert^{2b} + \mu_{{\rm ren}, i-1})}
\chi^{i-1}(\alpha)
=
\frac{1}
{Z_{b,\, i-1}}
{\tilde{C}}^{i-1}(p)
\,,
\eea
where we defined the renormalized mass $\mu_{{\rm ren}, i-1}$ to be
\beq
\mu_{{\rm ren}, i-1} = \frac{\mu_{i-1} - \Sigma_{i-1}(\{0\})}
{Z_{b,\, i-1}}
\,,
\eeq
and the wave function renormalization to be
\beq
Z_{b,\, i-1} \equiv 1 + 
(\partial_{\vert p_c\vert^{2 b}} \Sigma \big \vert_{\{p\}=0})_{i-1}
\,.
\eeq
Note that the color dependence on   $(\partial_{\vert p_{c}\vert^{2 b}} \Sigma \big \vert_{\{p\}=0})$
should be actually absent.  
Then, the effective theory for $\phi_{\le i-1}$ can be written as
\bea
d \nu_{\frac{1}{Z_{b,\, i-1}}{\tilde{C}^{i-1}}}(\phi_{\le i-1})
&\exp&
\Big[
\sum_c \Gamma^{(c)}_{2, \, i-1}(\{0\}) \Tr_2 ( p_c^{2a} \phi_{\le i-1}^2) 
+
\sum_c 
\frac{\Gamma^{(c)}_{4, \, i-1} (\{0\})}{2} \Tr_4 (\phi_{\le i-1}^4) 
\crcr
&&
+
\sum_c 
\frac{\Gamma^{(c)}_{4;+, \, i-1} (\{0\})}{2} \Tr_4 ( p_c^{2a} \phi_{\le i-1}^4) 
+
{\tilde R} (\phi_{\le i-1})
\Big]
\,.
\eea
With a field rescaling $\phi_{\le i-1} \rightarrow \sqrt{Z_{b,\, i-1}} \phi_{\le i-1}$ (which in our specific case, there is no actual rescaling because $Z_{b,\, i-1} = 1$ and trivial), the effective theory 
for $\phi_{\le i-1}$ can be recast:
\bea
d \nu_{{\tilde C}_{i-1}}(\phi_{\le i-1})
&{\rm exp}&
\Big[
\sum_c\frac{\Gamma^{(c)}_{2, i-1}(\{0\})}{Z_{b, i-1}}
\Tr_{2;c}( p^{2 a}
\phi_{\le i-1}^2)
+
\sum_c
\frac{\Gamma^{(c)}_{4, i-1}(\{0\})}{{2}Z_{b, i-1}^2}
\Tr_{4;c}(\phi_{\le i-1}^4)
\nonumber 
\\
&&
+
\sum_c
\frac{\Gamma^{(c)}_{4;+, i-1}(\{0\})}{{ 2} Z_{b, i-1}^2}
\Tr_{4;c}( p^{2a} \phi_{\le i-1}^4)
+
{\tilde R} (\sqrt{Z_{b,\, i-1}}\phi_{\le i-1})
\Big]
\,.
\eea
Now we can identify the effective couplings at scale $i-1$, 
\bea
Z_{a, i-1} 
&=&
-\frac{\Gamma^{(c)}_{2, i-1}(\{0\})}{Z_{b, i-1}} 
\,,
\nonumber
\\
\lambda^{(c)}_{i-1}
&=&
-\frac{\Gamma^{(c)}_{4, i-1}(\{0\})}{Z_{b, i-1}^2}
\,,
\nonumber 
\\
\lambda^{(c)}_{+; \, i-1}
&=&
-\frac{\Gamma^{(c)}_{4;+, i-1}(\{0\})}{Z_{b, i-1}^2}
\,.
\eea
Note that in our case, 
$(\partial_{\vert p_{c}\vert^{2 b}} \Sigma \big \vert_{\{p\}=0})_{i-1} = 0$
therefore, throughout, we actually had 
\bea
Z_{b,\, i-1} &=& 1\,,
\crcr
\mu_{{\rm ren}, i-1} &=& \mu_{i-1} - 
\Sigma_{i-1}(\{0\})
\,,
\crcr
Z_{a, i-1} 
&=&
-\Gamma^{(c)}_{2, i-1}(\{0\})\,,
\crcr
\lambda^{(c)}_{i-1}
&=&
-\Gamma^{(c)}_{4, i-1}(\{0\})\,,
\crcr
\lambda^{(c)}_{+; \, i-1}
&=&
-\Gamma^{(c)}_{4;+, i-1}(\{0\})
\,.
\label{eq:rgeqns}
\eea


In the following sections, we will compute these perturbative renormalization group
flow equations restricting to 1-loop 
corrections.  
For the model $+$, 
the $\beta$-functions can be computed for generic parameters
$a=(d-2)/2$, and $b=(d-3/2)/2$ and $d>2$ but with 
fixed group dimension $D=1$. 
For numerics, we will concentrate on a certain model $+$, specified by  $d=3$, $a=1$, and $b = 3/4$. 
More general equations for $D>1$ can be also computed 
with a bit more work \ref{theorem+}. 





\subsection{4-point function and its beta-function}
\label{sect:}


We will restrict the analysis by only considering up to one-loop corrections to the $\beta$-function in the perturbation theory.
Note that, from Proposition \ref{prop:list+}, 
$\Gamma^{(c)}_{4;+}(\{p\})$
 only contains the leading order zero-loop graph with one $\lambda_+$ coupling with four external legs. 
Next, we focus on $\Gamma^{(c)}_{4}(\{p\})$
and write 
\bea
\Gamma^{(c)}_{4}(\{p\})
 = \sum_{\cG_{4,\iota}^{(c)}} K_{\cG_{4,\iota}^{(c)}} \;  S_{\cG_{4,\iota}^{(c)}} (\{p\})\,,
\eea
where the sum over $\cG_{4,\iota}^{(c)}$
runs over a list of 4pt-graphs obeying the first row of Table \ref{tab:listprim1}, 
$K_{\cG_{4,\iota}^{(c)}}$ is a combinatorial factor and $  S_{\cG_{4,\iota}^{(c)}} (\{p\})$ is a formal amplitude sum. 
At zero-loop, the leading order graph is made of one $\lambda$ interaction bubble with four external legs.
At one-loop 
a single graph that we call $n_4^{(c)}$ shown in Fig. \ref{fig:V4singleloop} contributes. 
If one is further interested in two-loop contribution, see Appendix \ref{app:4pt2+}.


\begin{figure}[H]
\centering
     \begin{minipage}[t]{0.7\textwidth}
      \centering
\def\svgwidth{1\columnwidth}
\tiny{
\input{V4singleloop.pdf_tex}
}
\caption{ {\small  
The only one-loop primitively divergent graph, $n_4^{(c)}$ that contributes to the renormalization of $4$-pt function in $\Gamma_4^{(c)}(\{p\})$.
Here the rank $d=3$.  
}} 
\label{fig:V4singleloop}
\end{minipage}
\end{figure}


Explicitly, 
at one-loop,
\bea
&&
 K_{n_4^{(c)} } = 2 \,,
 \crcr
 &&
  S_{n_4^{(c)}} (\{\mathbf{p}, \mathbf{p}'\}) = \frac{1}{2!}
  \Big(\frac{-\lapc}{2}\Big)^2 
  \sum_{q_{c}} 
  \frac{\vert q_{c}\vert^{2a}}{(|\mathbf{p}_{\check c}|^{2b} + \vert q_{c}\vert^{2b} + \mu)}
    \frac{\vert q_{c}\vert^{2a}}{( |\mathbf{p}'_{\check c}|^{2b} + \vert q_{c}\vert^{2b} + \mu)}
    \,,
\eea
where $\mathbf{p}_{\check c} = (p_{1},\dots, p_{c-1},p_{c+1}, \dots, p_{d})$, 
and 
$|\mathbf{p}_{\check c}|^{2b} = 
\sum_{l=1| l \ne c}^d p_{l}^{2b}$. 



We compile these equations and deliver the $\beta$-functions of the couplings up to one-loop as 
\bea
&&
\lacren =  \lam^{(c)}  -  \frac{1}{4} (\lambda^{(c)}_+)^2 S_0 
\,,
\qquad \quad 
S_0 = \sum_{q}\frac{\vert q \vert^{4a}}{(\vert q\vert^{2b} + \mu_i)^2}
\,,
  \\
  &&
\lapcren  =  \lapc
\,,
\eea 
where $\lam_{\rm ren} $ and $\lam_{+, {\rm ren}} $ are the obvious corresponding renormalized coupling constant. 
We set all couplings to 
$\lac = \lam$, and $\lapc= \lam_{+} $ to simplify 
these 
\bea
&&
\laren =  \lam   -  \frac{1}{4} (\lambda_+)^2 S_0 
\,,
  \\
  &&
\lapren  =  \lam_{+} \; . 
\eea 
These RG flow equations carry already a lot of information. The second equation displays the fact
that the coupling $ \lam_{+}$ does
not run and defines a fixed point  at all orders of perturbation. 
 A second fact is that the two couplings
 $ \lam$ and $\lam_{+} $ never coincide
 and  could not be set at equal value. At the perturbative level, the model $\lam = \lam_{+}$
 is ill-defined (nonrenormalizable). 
Another piece of information conveyed by 
 the first equation
is that $\laren < \lam $ ($S_0$ may be formal but is positive) which means that $\lam$ increases in the UV.
One may ask if this leads to the ordinary Landau ghost (like the $\phi^4_4$ theory for instance), the 
answer is no. Indeed, note that the second term is not a correction in $\lambda^2$, but rather a correction in $\lambda_{+}^2$. 
This is a new effect that we must 
investigate better. 

We want to understand the
qualitative feature of RG flow
given by the above coupled system. 
In the multiscale analysis with discrete scale $i$, the system can be written as
\bea
&&
\lam_{i-1} =  \lam_i-  \frac{1}{4} \lam_{+,i}^2 S_{0,i} 
\,,
\label{eq:lambda+S0i}
  \\
  &&
\lam_{+,i-1} =  \lam_{+,i}  \,,
\eea
where $S_{0,i}$ stands for the cut-off
amplitude of  the formal sum $S_0$
and  formulates as
\bea
S_{0,i} &=& \sum_{q}\vert q\vert^{4a}
\int_0^\infty 
d\alpha \chi^{i} (\alpha) e^{-\alpha 
(\vert q\vert^{2b} + \mu_i)}
\int_0^\infty 
d\alpha' \chi^{i} (\alpha') e^{-\alpha' 
(\vert q\vert^{2b} + \mu_i)}
\crcr
&= &
\int_0^\infty 
d\alpha \,\chi^{i} (\alpha) 
\int_0^\infty 
d\alpha' \chi^{i} (\alpha')
e^{-(\alpha+  \alpha')
\mu_i}
\sum_{q}\vert q\vert^{4a}
e^{-(\alpha+  \alpha')
\vert q\vert^{2b} }\,,
\label{s0idebut}
\eea
where the sharp cutoff function $\chi^i(\alpha)$ was defined in \eqref{eq:chii}.
The sum over $q$ 
will be handled
using an integral via Euler-Maclaurin approximation (Appendix \ref{app:SumtoInt} details the
following result, and we have $S_{0,i} = \widetilde S_{0,i}$ and therefore, $\lambda_i = \widetilde \lambda_i$ and $\lambda_{+,i} = \widetilde \lambda_{+,i}$, where $\; \widetilde{}\; $ indicates a dimensionless quantity): 
\bea
S_{0,i} =
 \frac{1}{b}
 \log 
  \frac{(M^{2b} +1)^2}{
  4 M^{2b }}
 + {\mathcal O}(  M^{-2b i}
 \log(M^{-2b i})) 
 \,.
 \label{s0iapprox}
\eea
We write the $\beta$-function for a given coupling $g$ as $\beta_g (k) = k\partial_k g (k)$, where $k$ is a momentum scale. In our present setting dealing 
with discrete slices (multiscale analysis), 
we have finite difference equations that we will turn into differential equation. 
The momentum scale must be compared
to the slice range as
 $k/k_0 \sim M^{i}$, given in terms of a given momentum unit $k_0 $. 
Using the so-called time scale
 $t= \log (k/k_0) \sim i \log M$,
given that the coupling 
$\lambda_{+} = \lambda_{+, i}$ does
not run and that, at leading order
 $S_{0, i}= c \log M $,
where $c>0$,  
the difference $\lambda_{i-1} - \lambda_i$ takes the form: 
\bea
-(\lambda_{i-1} - \lambda_i) = 
\frac{\partial \lambda_{i}}{   \partial i}
&=&
\frac{1}{4} \lambda_{+,i}^2 \, S_{0, i}
\,,
\eea
which can be translated into 
the following first order ODE
\bea
\frac{\partial \lambda_{i}}{\partial ( (\log M)  i) } = \partial_t \lambda (t)
&=&
- \beta_\lambda  \lambda_{+}^2 
\crcr
\beta_\lambda 
&=& 
- 
\frac{1}{4b}
\frac{
\log 
  \frac{(M^{2b} +1)^2}{
  4 M^{2b }} 
  }
 {\log(M)} <0 
\,.
\eea
One may wonder why $M$, the propagator slice parameter,  appears
in the perturbative expansion and if this the present
equation does not depend on the slicing scheme. 
The multi-scale analysis justifies this
entirely: we are computing a flow
between two scales $\sim M^{-2bi}$, therefore
$M$ becomes an input of our equation. 

This easily integrates to give a solution: 
\bea
&&
\lambda(t) 
= 
{ (- \beta_\lambda)}
\, \lambda_{+}^2\,  t + C 
\,, \qquad C = cste
\cr\cr
&& 
\lambda(t) 
= { (- \beta_\lambda)} \, \lambda_{+}^2\,  (t-t_0) + \lambda(t_0)
\,,
\label{eq:lambdarunlinear+}
\eea
where at some IR scale 
$ e^{t_0} \ll \Lambda/k_0 = e^{t}$. As opposed to the usual $\phi^4_4$ model, 
there is no pole in the solution at first order. This proves that 
this is not a Landau ghost and therefore this model 
is not similar to the ordinary $\phi^4_4$ model. 
Furthermore, 
at large $t\ge t_0$,
and for non vanishing $\lambda_{+}\ne 0$, 
since $\lam > \lambda(t_0)$, 
the bare coupling 
is suppposedly not vanishing.
We conclude that the model is not asymptotically free. 
This hints at
an asymptotically safe model that only non perturbative calculation can make rigorous. 
If $\lam_+=0$,   then 
$\lam(t) = C$ and
we have a fixed point. However, 
the enhancement disappears, 
both couplings $\lam$ 
and $\lam_+$ do not flow,
all amplitudes become finite. 
Note that the resulting model is not the usual 
$T^4$-TFT model
with only $\lam$ coupling
and a different class of dominant graphs (melonic ones). 






\subsection{Computing $\Gamma_2$
and $Z_{a}$ renormalization}
\label{compGamma2+}

Now we compute the renormalization of the $2$-point coupling, $\Gamma_2^{(c)}(\{0 \})$, at fixed color $c$: 
\bea
 \vert p_{c}\vert^{2 a}
\Gamma_2^{(c)}(\{p \})
& =& \sum_{{\cal G}_{2;a;\iota}^{(c)} }
K_{{\cal G}_{2;a;\iota}^{(c)}} S_{{\cal G}_{2;a;\iota}^{(c)}} (\{p\})\,,
\label{Gamma2Za}
\eea
where the sum is over all amputated $1$PI $2$-pt graphs at $1$-loop whose boundaries to be in the form of ${\rm Tr}_{2;(c)} ({ p}^{2 a} \phi^2)$.

Up to the first order in perturbation theory, we have 
${\cal G}_{2;a; \iota}^{(c)} \in \{z_a^{(c)},  m_{e}^{(c)} \}$, where $z_a^{(c)}$ is the leading order (zero-loop) graph with one $Z_a^{(c)}$ interaction with two external legs. (Appendix \ref{app:2ndO+} shows additional graphs up to the second order in perturbation theory.) 
In  \eqref{eq:sigma}, we have identified $\partial_{\vert p_{c}\vert^{2 a}} \Sigma\vert_{\{p\}=0} \equiv \Gamma^{(c)}_2 (\{0\})$ as divergent.
The other  contributing graphs, namely $m_{e}^{(c)}$, $c=1,2,3$,  are divergent at first order in perturbation theory  (row II of Table \ref{tab:listprim1}) 
 with  $\omega_{m_{e}^{(c)}} = \frac{D}{2}$. 

\begin{figure}[H]
\centering
     \begin{minipage}[t]{0.7\textwidth}
      \centering
\def\svgwidth{0.7\columnwidth}
\tiny{
\input{m1e.pdf_tex}
}
\caption{ {\small  $m_{e}^{(c=1)}$ for $d=3$ is illustrated in colored (left) and stranded (right) representations. $\{q_{\check c}\} = \{q_{2}, \; q_{3}\}$, ${|{\bf q}_{\check c}|}^{2 b} = {|q_{2}|}^{2 b} + {|q_{3}|}^{2 b}$.
}} 
\label{fig:m1e}
\end{minipage}
\end{figure}
\bea
K_{m_{e}^{(c)}} S_{m_{ e}^{(c)}} (\{p\}) 
&=&
K_{m_{e}^{(c)}}
\Big( - \frac{{\lambda_{+}}^{(c)}}{2}\Big)
\sum_{\{q_{\check c}\}} \frac{\vert p_{c}\vert^{2 a}}{( {|{\bf q}_{\check c}|}^{2 b} + \vert p_{c}\vert ^{2 b} + \mu )}
\,,
\eea
where
$K_{m_{e}^{(c)}} = 2$.
Putting all together, up to first order in perturbation theory, 
\bea
 \vert p_{c}\vert^{2 a}
\Gamma_2^{(c)}(\{p\})
=
 - \vert p_{c}\vert^{2 a}
Z_a^{(c)} 
 - \lambda_+^{(c)} \vert p_{c}\vert^{2 a}
\sum_{\{q_{\check c}\}} \frac{1}{( {|{\bf q}_{\check c}|}^{2 b} + {\vert p_{c}\vert}^{2 b} + \mu )}
\,.
\eea
$\Gamma_2^{(c)}(\{p\})$ is therefore
identified easily. 
For general $d$, recalling the renormalization group equations \eqref{eq:rgeqns} and setting all the external momenta $\{p\} =0$, we obtain: 
\bea
Z^{(c)}_{a,{\rm ren}} &=&
- \Gamma^{(c)}_2(\{0\})
=
Z^{(c)}_{a}
+
\lambda_+^{(c)} 
\sum_{\{q_{\check c}\}} \frac{1}{( {|{\bf q}_{\check c}|}^{2 b} + \mu )}
\,.
\eea
Note that we only keep the first order in Taylor expansion in couplings.
Setting the couplings independent of colors, 
\bea
Z_{a,{\rm ren}} 
&=&
Z_{a}
+
\lambda_+ 
\sum_{\{q_1,\dots, q_{d-1}\}} 
\frac{1}{({|{\bf q}|}^{2 b}  + \mu )}
\,.
\eea
In the multiscale analysis language,
\bea
Z_{a,i-1} 
=
Z_{a,i}
+
\lambda_{+,i} S_{1,i}
\,,
\eea
where $S_{1,i}$ is given in \eqref{s1idebut} with explicit expression given in \eqref{eq:S1itildeS1i}, where we fixed $D=1$.
We recall that $\lambda_+ = \lambda_{+,i}$ does not run.
Making explicit the dimensions, we obtain the renormalization group equation for $Z_a$ as
\bea
Z_{a,i-1} 
=
Z_{a,i} + 
k^{1/2}
\lambda_{+,i} 
\,
\widetilde S_{1,i}
\,,
\eea
which,  
using $\widetilde S_{1,i}$ given in \eqref{eq:runningmass2}, 
gives  the $\beta$-function for $Z_a$: 
\bea
- ( Z_{a, i-1} -  Z_{a,i}) 
&=&
\frac{\partial Z_{a,i}}{\partial i}
=
- k^{1/2} \lambda_{+,i} \, \widetilde S_{1,i}
\,,
\crcr
\frac{\partial  Z_{a,i}}{\partial ( (\log M)  i) } 
&=&
\partial_t  Z_{a} (t)
= 
- k^{1/2}  \beta_{Z_a} \, \lambda_{+}
\,,
\crcr
\beta_{Z_a} 
&=&
\frac{\widetilde S_{1,i}}{\log(M)} > 0 
\,,
\eea
Introducing dimensionless quantities, 
$Z_a(t)= k^{1/2}\widetilde Z_a(t)$, 
the dimensionless RG equation can be written 
\bea
\partial_t  \widetilde Z_a(t)
&=& - \frac{1}{2} \widetilde Z_a(t)+ k^{-1/2} \partial_t Z_a(t) \crcr
& = &
- \frac{1}{2} \widetilde Z_a(t)
-  \beta_{Z_a} \, \lambda_{+}
\,.
\eea
This integrates easily with respect to $t$ and
gives
\bea
 \widetilde Z_a(t) = 
 c_1 \, e^{- t/2}   - 2 \beta_{Z_a}  \, \lambda_{+} 
 \,,
\eea
where $c_1$ is again an integration constant.
This equation just expresses the fact that
 $\widetilde Z_a(t)$ is a relevant coupling and
decreases exponentially in the UV ($t \rightarrow \infty$) and suppressed, whereas in the IR ($t \rightarrow -\infty$),  it blows up.


\

\subsection{Self energy and mass renormalization}
\label{sect:selfenergmass}

Our following task is to compute the so-called self energy, which we denote by $\Sigma_b(\{ p \})$: 
\bea
\Sigma_b(\{p \})
=
\sum_{c=1}^d 
\sum_{{\cal G}^{(c)}_{2,\iota}} {K_{{\cal G}^{(c)}_{2,\iota}}} {S_{{\cal G}^{(c)}_{2,\iota}}} (\{p\})\,,
\label{selfnrg}
\eea
where the first sum 
is broken by color $c$
and the second sum 
is performed over 
all  amputated 1PI 2pt-graphs 
${\cal G}^{(c)}_{2,\iota}$
(with color label $c$) at 1-loop 
 with  boundary to be in  the form of ${\rm Tr}_2 (p^{2 b} \phi^2)$
 (hence the index $b$
 in subscript in $\Sigma_b(\{p \})$). 
 
It is noteworthy that 
$\Sigma_b(\{p \})$ corresponds to the part 
$ \Sigma (\{0\}) 
+  \sum_{c}
\vert p_{c}\vert^{2 b} \partial_{\vert p_{c}\vert^{2 b}} \Sigma \big\vert_{\{p\}=0}$
of total self-energy   function $\Sigma(\{ p \})$ in \eqref{eq:sigma}. 
Noting that the second term $\partial_{\vert p_{c}\vert^{2 b}} \Sigma \big\vert_{\{p\}=0}=0$, we only focus on the contribution $\Sigma (\{0\}) 
$, namely the contribution to the mass renormalization. 

The graphs we are interested in are denoted
${{\cal G}^{(c)}_{2,\iota}} \in \{ m^{(c)}, n^{(c)}\}_{c=1,2,\dots, d}$, see Figures \ref{figu:m1} and
\ref{fig:n1}. 

\begin{itemize}
\item
For the graph $m^{(c)}$, $c=1,\dots, d$,
the degree of divergence
$\omega_{d;+}(m^{(c)})$ reaches $\frac{D}{2}$,
as shown in the class III of Table \ref{tab:listprim1}. 
In Figure \ref{figu:m1}, 
we display $m^{(c)}$, when the vertical line is color $c$. 

\begin{figure}[H]
\centering
     \begin{minipage}[t]{0.5\textwidth}
      \centering
\def\svgwidth{1\columnwidth}
\tiny{
\input{m1.pdf_tex}
}
\caption{ {\small {The graph $m^{(c)}$ in the case $d=3$ in colored (left)
and stranded (right) representations. 
}}} 
\label{figu:m1}
\end{minipage}
\end{figure}

The contribution  to the amplitude brought by the graphs $m^{(c)}$ is 
\bea
\sum_{c=1}^{d}
K_{m^{(c)}} S_{m^{(c)}}  ( \{p\})
&=&
\sum_{c=1}^{d} 2 \big(- \frac{\lambda^{(c)}}{2} \big) \sum_{\{q_{\check c}\}} 
\frac{1}{({|{\bf q}_{\check c}|}^{2 b} + \vert p_{c}\vert^{2 b} + \mu )} \crcr
&=& 
\sum_{c=1}^{d} \sum_{\{q_{\check c}\}} 
(- \lambda^{(c)})
\frac{1}{({|{\bf q}_{\check c}|}^{2 b} + \vert p_{c}\vert^{2 b} + \mu )}
\,,
\eea
where $K_{m^{(c)}} = 2$,
for any $c$. 


\item
In the second class of graphs denoted each $n^{(c)}$, the amplitude has divergence degree $\omega_{d;+}(n^{(c)}) = \frac{D}{2}$.
This graph belongs to the class I in Table \ref{tab:listprim1}.
Figure \ref{fig:n1} shows $n^{(c)}$
at rank $d=3$ with horizontal line colored by index $c$. 

\begin{figure}[H]
\centering
     \begin{minipage}[t]{0.5\textwidth}
      \centering
\def\svgwidth{1\columnwidth}
\tiny{
\input{n1.pdf_tex}
}
\caption{ {\small  For $d=3$, $n^{(c)}$ is shown above in colored (left) and stranded (right) representations. 
}} 
\label{fig:n1}
\end{minipage}
\end{figure}

The sum of Feynman amplitude associated with the graph $n{(c)}$ yields
\bea
\sum_{c=1}^{d}
K_{n^{(c)}} S_{n^{(c)}} ( \{p\})
&=&
\frac12
\sum_{c=1}^{d}
(-\lambda_{+}^{(c)} )
\sum_{q_{c}}
\frac{\vert q_{c}\vert^{2 a}}{(\vert q_{c}\vert ^{2 b} + {|{\bf p}_{\check c}|}^{2 b} + \mu)}\,,
\eea
where $K_{n^{(c)}} = 1$ for any $c$. 
\end{itemize}

Up to the 
first order in perturbation theory and 
evaluating at $0$ external momenta, we obtain 
\bea
&&
\Sigma_b(\{0\})
=
\sum_{c=1}^d  
\Big( K_{m^{(c)}} S_{m^{(c)}} (\{p\}) 
+  K_{n^{(c)}} S_{n^{(c)}} (\{p\}) 
\Big)  \Big|_{p=0}
\crcr
&& = 
\sum_{c=1}^{d}
\Big[(- \lambda^{(c)}) \sum_{\{q_{\check c}\}} 
\frac{1}{({|{\bf q}_{\check c}|}^{2 b} + \mu )}
+
\frac{1}{2}(- 
{\lambda_{+}}^{(c)})
\sum_{q_{c}}
\frac{\vert q_{c}\vert^{2 a}}{(\vert q_{c}\vert^{2 b} + \mu)} \Big]
\crcr
&&
\eea
Therefore, for general $d$, the renormalized mass equation up to the first order in perturbation is given by:
\bea
{\mu_{\rm ren}} 
= 
\mu +
\sum_{c=1}^{d}
\Bigg\{
\lambda^{(c)} \sum_{\{q_{\check c}\}} 
\frac{1}{({|{\bf q}_{\check c}|}^{2 b}  + \mu )}
+
\frac{1}{2}
{\lambda_{+}}^{(c)} 
\sum_{q_{c}}
\frac{\vert q_{c}\vert^{2 a}}{(\vert q_{c}\vert^{2 b}  + \mu)}
\Bigg\}
\,.
\eea
Now assert color independence, namely $\lambda^{(c)} = \lambda$, ${\lambda_{+}}^{(c)} = \lambda_{+}$, one gets 
\bea
&&
\mu_{\rm ren}
=
\mu +
d \Big(
\lambda S_1
\: + \;   \frac{1}{2}
{\lambda_{+}}
S_2 
\Big) 
\crcr
&&
S_1 = \sum_{\{q_1,\dots, q_{d-1}\}} 
\frac{1}{({|{\bf q}|}^{2 b}  + \mu )}
\qquad 
S_2 = 
\sum_{q}
\frac{\vert q \vert^{2 a}}{(\vert q \vert^{2 b}  + \mu)}
\,,
\label{eq:S1S2}
\eea
where we restrict ourselves to $D=1$.

To compute the perturbative flow, we
must reach dimensionless coupling
equations and therefore
must set up the scaling dimension of the mass coupling (note that 
$\lam$ and $\lam_+$ were marginal 
and therefore dimensionless). 
Following \cite{Geloun:2016qyb}, 
the mass scaling dimension
is determined by the maximal degree of divergence of the 2pt amplitudes. 


We cut off the propagators, switch to dimensionful quantities, and write the sums
\bea
&&
S_{1,i} = 
\sum_{{\bf q} \in k \Z^{d-1}}
\int_{0}^{\infty}
d\alpha \; \chi^i(\alpha)\, 
e^{ - \alpha( |{\bf q}|^{2 b} + \mu_i )}
\,,
\label{s1idebut}
\\
&&
S_{2,i} = 
\sum_{q\in  k \Z} 
\int_{0}^{\infty}
d\alpha \; \chi^i(\alpha)\, 
\vert q \vert^{2 a}
e^{ - \alpha( |q|^{2 b}  + \mu_i )}
\,.
\label{s2idebut}
\eea
In writing explicitly the dimensions of $q$ and $\alpha$ in terms of  momentum  scale $k$, \eqref{eq:S1S2dimfuldimless} in Appendix \ref{app:SumtoInt} proves that
we can approximate those
as
\bea
&&
S_{1,i} = k^{1/2} 
\Bigg(
\Big( \frac{1}{b} \Gamma\left(\frac{1}{2b} \right) \Big)^{d-1} 
(2d-3)
\Big(  1 - M^{-1/2} \Big)
M^{i/2}
+{\mathcal O}(M^{-i/2}) 
\Bigg)
\,,
\label{eq:S1itildeS1i}
\\
&&
S_{2,i} = k^{1/2} 
\Bigg(4 \, \Gamma\Big(\frac{2(d-1)}{2d-3}\Big)
(1 - M^{-1/2} ) M^{i/2}
 + 
 {\mathcal O}\big(  M^{-i (d -2)}  \big) 
 \Bigg)
 \label{eq:S2itildeS2i}
 \,,
\eea
where we have set $D=1$ at any rank $d$  and   $b = \frac{1}{2} (d^{-} - \frac{1}{2})$ given in Proposition \ref{prop:list+}.
Compiling this, 
we obtain the following
equation at leading order,
\bea
%\mu_{i-1} = M^{\frac{1}{2}(i-1)}\widetilde\mu_{i-1}
\mu_{i-1} 
&=&
%M^{\frac{1}{2}i}\widetilde \mu_i 
 \mu_i 
+
 k^{1/2} 
d\Big(
\lambda_i \widetilde S_{1,i}
+
\frac{1}{2}
\lambda_{+, i}
\widetilde S_{2,i}
\Big) 
\,,
\crcr
\widetilde S_{1,i} & = & k^{-1/2}S_{1,i}
\,,
\crcr
\widetilde S_{2,i} &=&  k^{-1/2}S_{2,i}
\,,
\label{eq:murenorm}
\eea
where we recall that $\lambda$ and $\lambda_+$ are marginal, therefore dimensionless in scales, and where $\widetilde S_{1,i}$ and $\widetilde S_{2,i}$ are given in \eqref{eq:runningmass2} and \eqref{eq:runningmass1}.




We obtain the $\beta$-function for the
mass 
\bea
- ( \mu_{i-1}- \mu_i)
&=&
\frac{\partial  \mu_i}{\partial i}
= - k^{1/2} d \Big( \widetilde S_{1,i} \, \lambda_i 
+ \frac{1}{2} \widetilde S_{2,i} \, \lambda_{+,i}\Big) 
\,,
\crcr
\frac{\partial \mu_i}{\partial ((\log M) i)}
&=&
\partial_t  \mu 
= 
- k^{1/2} (\beta_{\mu, 1}\,  \lambda
+  \beta_{\mu, 2}\, \lambda_{+})
\,,
\crcr
\beta_{\mu, 1} 
&=&
\frac{d}{\log M}  \, \widetilde S_{1,i} >0
\,,
\crcr
\beta_{\mu, 2}
&=&
\frac{d}{2 \, \log M} \,  \widetilde S_{2,i} >0
\,.
\eea
We switch to dimensionless quantities, fixing $\mu = k^{1/2} \widetilde \mu$, where $\widetilde \mu$ is dimensionless in scale, therefore, $\partial_t  \mu= k \partial_k  \mu = k^{1/2} ( \frac{1}{2} \widetilde \mu  + \partial_t  \widetilde \mu)$.
The previous equation becomes
\bea
\partial_t  \widetilde \mu(t)=- \frac{1}{2} \widetilde \mu(t)+ k^{-1/2} \partial_t \mu (t)
\,.
\eea
Given that the coupling 
$\lambda_{+} =  \lambda_{+, i}$ does
not run and that,
 $ \lambda$ runs according to \eqref{eq:lambdarunlinear+}, we make the $t$ dependence explicit and 
\bea
\partial_t \widetilde \mu (t)
&=& 
- \frac{1}{2} \widetilde \mu (t)
- \beta_{\mu, 1}\, \lambda(t)
- \beta_{\mu, 2}\,  \lambda_{+}
\crcr
&=& 
- \frac{1}{2} \widetilde \mu (t)
+
\beta_{\mu, 1} \,  \beta_\lambda \, \lambda_{+}^2\,  t 
- 
\Big( 
\beta_{\mu, 2}\,  \lambda_{+}
+ \beta_{\mu, 1} \,  \beta_\lambda \,   \lambda_{+}^2\,  t_0 
+\beta_{\mu, 1} \, \lambda(t_0)
\Big)\,,
\eea
where we recall $\beta_\lambda  < 0 $\,.
We can solve this differential equation and obtain
\bea
\widetilde \mu (t) 
&=&
 c_1 e^{-t/2}
 +4 \beta \, t - 4 \beta + 2 \gamma
 \,,
\crcr
\beta &=& \beta_{\mu, 1} \,  \beta_\lambda \, \lambda_{+}^2 < 0 
\,,
\crcr
\gamma &=& 
- 
\Big( 
\beta_{\mu, 2}\,  \lambda_{+}
+ \beta_{\mu, 1} \,  \beta_\lambda \,   \lambda_{+}^2\,  t_0 
+\beta_{\mu, 1} \, \lambda(t_0)
\Big)
\,.
\eea
where $c_1$ is an integration constant
that should be set by an initial condition of the flow. 

In the UV, the first exponential term vanishes and the second linear term dominates. As $\beta = \beta_{\mu, 1} \,  \beta_\lambda \, \lambda_{+}^2 < 0$, the mass becomes negative and grows linearly. This 
is not the ordinary behavior of 
scalar field theory where relevant operators in the UV are suppressed. 
In the IR, the exponential term dominates and this is the common behavior of relevant operators in ordinary $\phi^4$-theory. 

\ 


\noindent{\bf Summary --} 
We list our dimensionless 1-loop RG flow equations for the
model $+$ and their solutions below.
\begin{center}
\begin{tabular}{l|l}
\hline 
\hline 
$\partial_t \,\lambda (t) = - \beta_\lambda  \lambda_{+}^2 $\,,
 \qquad $\beta_{\lambda} <0$  &  $\lambda(t) 
= - \beta_\lambda \, \lambda_{+}^2\,  (t-t_0) + \lambda(t_0)$ \\ 
\\
$\partial_t\,  \lambda_{+} = 0 $  & $\lambda_{+} =cste$\\
\\
$\partial_t \,\widetilde \mu (t)
=
- \frac{1}{2} \widetilde \mu (t)
- \beta_{\mu, 1}\, \lambda(t)
- \beta_{\mu, 2}\,  \lambda_{+}$ 
 & $\widetilde \mu (t)= c_1 e^{-t/2}
 +4 \beta \, t - 4 \beta + 2 \gamma$ \\ 
$\beta_{\mu, 1} > 0\,, \;\, \beta_{\mu, 2} > 0 $
& $\beta = \beta_{\mu, 1} \,  \beta_\lambda \, \lambda_{+}^2 < 0 $\,, \\ 
& $\gamma =  - 
\Big( 
\beta_{\mu, 2}\,  \lambda_{+}
+ \beta_{\mu, 1} \,  \beta_\lambda \,   \lambda_{+}^2\,  t_0 
+\beta_{\mu, 1} \, \lambda(t_0)
\Big)$ \\
 \\
$\partial_t \,  \widetilde Z_a(t)
 = 
- \frac{1}{2} \widetilde Z_a(t)
-  \beta_{Z_a} \, \lambda_{+} $\,,
\qquad 
  $\beta_{Z_a} > 0$   &
$\widetilde Z_a(t) = 
 c_2 \, e^{- t/2}   - 2 \beta_{Z_a}  \, \lambda_{+} $\\ 
 \hline 
 \hline 
\end{tabular}
\label{tab:RG+}
\end{center}


\subsection{Integration at arbitrary loops}
\label{sec:allloops+}

The power counting of the model $+$, provided by Proposition \ref{prop:list+},  gives us a lot of information about the $\beta$-functions of the couplings even in an arbitrarily high order in perturbation theory. 
In this section, we investigate general forms of $\beta$-functions of the couplings in the model $+$ at all orders in perturbation theory.

\

\noindent{\bf 4-pt couplings  $\lambda$ and $\lambda_+$  RG equation --}
 Proposition \ref{prop:list+} dictates that there are no diverging amplitudes contributing to the renormalization of $\lambda_+$ at all orders in perturbation, therefore, {\it $\lambda_+$ is constant at all orders}.
Furthermore, from the first row of
Table \ref{tab:listprim1} of Proposition \ref{prop:list+}, which governs the renormalization of the coupling $\lambda$, we know that in order for the amplitude to be divergent, Feynman graph must only contain $\lambda_+$ couplings, and no other couplings. Then, we can readily conclude that,
the subleading corrections to $\lambda$ at  $n$-loops, $n$ being arbitrary,
can be written as a  polynomial $P_n(\lambda_+)$ in the variable $\lambda_+$ that is fixed to a constant. 
At arbitrary $n$-th order in perturbation, the $\beta$-function of the coupling $\lambda$ assumes the form 
\bea
\partial_t \lambda (t)
&=&
 P_n(\lambda_{+}) 
 \,,
 \crcr
 \lambda(t) 
&=& 
P_n(\lambda_{+})  (t-t_0) + \lambda(t_0)
\,.
\label{eq:lambda+allorders}
\eea
The particular form of $P_n(\lambda_{+}) = -\beta_{\lambda}\lambda_{+}^2 + \dots $ is left for future  investigation.  
Note that we may find an asymptotic safe theory we must solve  $P_n(\lambda_+)=0$ which may have several solutions
$\lambda_+^*$.

\




\noindent{\bf 2-pt coupling $Z_a$ RG equation --}
The classes II and V of Proposition \ref{prop:list+} contribute to the renormalization of the $2$-pt coupling $Z_a$.
All class II amplitudes involve only the coupling $\lambda_+$ 
yielding a similar result as to the one-loop computation above.  
The class V contains  contributions with exactly one $Z_a$ and 
several $\lambda_+$ (an example is given in Fig. \ref{fig:m2e} Appendix \ref{app:2ndO+}).
Therefore, at an arbitrary $n$-th order in perturbation theory, 
we expect
\bea
\partial_t Z_a(t) 
&=&
k^{1/2} Q_{1;n} (\lambda_+) + \log (k/k_0) \, Z_a(t) Q_{2;n} (\lambda_+)
\,,
\eea
where $Q_{i;n}  (\lambda_+) $, $i=1,2$ are polynomials in $\lambda_+$. 
Then, in terms of dimensionless quantities, we obtain
\bea
\partial_t \widetilde Z_a(t) 
=
 \Big( t \, Q_{2;n} (\lambda_+) - \frac{1}{2}\Big) \widetilde Z_a(t) 
+ Q_{1;n}  (\lambda_+) 
\,. 
\eea
which will integrate to yield
\bea
\widetilde Z_a(t) = 
 e^{\frac{Q_{2;n}  t^2}{2}-\frac{t}{2}}\Big[
c_1+
   \sqrt{2}\, 
  \frac{ Q_{1;n} }{ \sqrt{Q_{2;n}}} \, 
   e^{\frac{1}{8 Q_{2;n} }} \, 
   \text{Erf'}\left(\frac{2
  Q_{2;n}  t-1}{2 \sqrt{2}
   \sqrt{Q_{2;n}}}\right)
   \Big]
   \label{eq:Za+allorders}
\eea
where Erf' is the non normalized incomplete error
function Erf'$(z) = \int_{0}^{z} e^{-s^2} ds$,
and $c_1$ an integration constant. 
Understanding the flow of $\widetilde Z_a(t) $ rests upon the 
the understanding of the polynomials $ Q_{1;n}$ and  $ Q_{2;n}$. 
Focusing on the asymptotic of the Erf function, 
Erf'$(z) \sim e^{-z^2}/z^2$, 
we see that 
$\widetilde Z_a(t) 
\sim 
 e^{\frac{Q_{2;n}  t^2}{2}-\frac{t}{2}}$. Then, 
 $\widetilde Z_a(t) $ behaves the same 
 in the UV ($t \rightarrow \infty$) and the IR ($t \rightarrow - \infty$), either can be suppressed or blows up depending on the sign of $  Q_{2;n}$. 



\ 


\noindent{\bf Mass RG equation --}
Looking at Proposition \ref{prop:list+}, the classes that contribute to the renormalization of the mass are I, III, IV, and VI.
The class I only contains $\lambda_+$, which is held constant.
The class III contains only exactly one $\lambda$ and the rests are all $\lambda_+$.
The class IV contains only exactly one $Z_a$ coupling and the rests are all $\lambda_+$.
Finally, the class VI contains exactly only one $Z_a$, and exactly only one $\lambda$, and the rests are $\lambda_+$.
Noting that $\lambda_+$ does not run at all orders, the most complicated one could have is the class VI where $\lambda$ and $Z_a$ are coupled (whose example is given in Fig.{\ref{fig:m2}}).
Therefore, we expect 
\bea
\partial_t  \widetilde \mu(t)
&=&
- \frac{1}{2} \widetilde \mu(t)
+ R_{1;n}(\lambda_+) \, \lambda (t)
+ R_{2;n}(\lambda_+) 
\cr
&&
+ 
R_{3;n}(\lambda_+) \, t \,\widetilde Z_a (t)
+ 
 R_{4;n} (\lambda_+) \, t \, \widetilde Z_a (t) \, \lambda (t)
\,,
\label{eq:mass+allorders}
\eea
to an arbitrary $n$-th order in perturbation theory, and $R_{i;n}$, with $i = 1, 2, 3, 4$ are polynomials in $\lambda_+$ and some constants.

\

In summary, we 
expect that the coupled system of RG equations of the model $+$ to an arbitrary $n$-th order is given by 
\bea
\partial_t \lambda_+ 
&=&
0
\,,
\crcr
\partial_t \lambda (t)
&=&
 P_n(\lambda_{+}) 
 \,,
 \crcr
\partial_t \widetilde Z_a(t) 
&=&
 \Big( t \, Q_{2;n} (\lambda_+) - \frac{1}{2}\Big) \widetilde Z_a(t) 
+ Q_{1;n}  (\lambda_+) 
\cr
\partial_t  \widetilde \mu(t)
&=&
- \frac{1}{2} \widetilde \mu(t)
+ R_{1;n}(\lambda_+) \, \lambda (t)
+ R_{2;n}(\lambda_+) 
\cr
&&
+ 
t \,\widetilde Z_a (t) \, R_{3;n}(\lambda_+)
+ 
t \, \widetilde Z_a (t) \, \lambda (t) \, R_{4;n} (\lambda_+)
\,.
\eea


\ 



\section{One-loop beta-functions of
the model $\times$}
\label{betat}

Just as performed for the model $+$, we will compute the 
1-loop RG flow equations of coupling constants via multiscale analysis
for the model $\times$. 
Because the scheme and proofs are nearly identical, the derivations and explanations are given in a streamlined analysis. 
We stress the following fact: although the notation and expressions 
are similar to previous section, they actually
refer to different quantities.  As the model is different, there is no confusion 
and no need to introduce new notation. 

\subsection{Effective   coupling equations}

The integration of high modes yields a formal effective action of the form \eqref{eq:effectiveW}
where notation keeps its meaning but adapts to the present model. 
Therein, the 2-point  amplitudes expand in local and nonlocal parts, obtaining,
a self-energy of the same form as
given in  \eqref{eq:sigma} in addition with the following term: 
\bea
 \sum_{c}
\vert p_{c}\vert^{4 a} \partial_{\vert p_{c}\vert^{4 a}} \Sigma \big\vert_{\{p\}=0} 
\,.
\label{eq:sigmax}
\eea
According to \cite{BenGeloun:2017xbd},  Table \ref{tab:listprim2} dictates  the renormalization analysis of the model. One shows that 
\begin{align}
 \partial_{\vert p_{c}\vert^{2 b}} \Sigma \big \vert_{\{p\}=0} =0& 
  \qquad \text{implies }\;    Z_b = 1 \quad  \text{wave function renormalization} \crcr
 \Sigma (\{0\})  \sim   \log-{\rm divergent}&   \qquad    \text{(row III) \quad mass renormalization} \crcr
\partial_{\vert p_{c}\vert^{2 a}} \Sigma\vert_{ \{p\}=0 } \equiv \Gamma^{(c)}_{2;a} (\{0\})  \sim   \log-{\rm divergent} &    \qquad    \text{(row I)  \quad}  Z_a \text{\; renormalization} \crcr
\partial_{\vert p_{c}\vert^{4 a}} \Sigma\vert_{\{p\}=0} \equiv \Gamma^{(c)}_{2;2a} (\{0\})  \sim   \log-{\rm divergent}&    \qquad     \text{(row II)  \quad}  Z_{2a} \text{\; renormalization}
\end{align}
where $\vert p_{c}\vert^{2 a}\Gamma^{(c)}_{2;a}(\{p\})$
and $\vert p_{c}\vert^{4 a}\Gamma^{(c)}_{2;2a}(\{p\})$
are the sum of all  amputated 1PI 2pt-functions following the patterns of 
${\rm Tr}_{2;c} ({ p}^{2a} \phi^2)$ and ${\rm Tr}_{2;c} ( {p}^{4a} \phi^2)$, respectively, on their boundary graphs. 
Because $4$-point functions converge we do not need to report them. 

We reorganize the effective action as 
\bea
- W^{i-1}(\phi_{\le i-1}, {\bar \phi}_{\le i-1})
&=&
\Sigma_{i-1} (\{0\})
\Tr_2 (\phi^2_{\le i-1})  
+
\sum_c
\Gamma^{(c)}_{2;a, \; i-1} (\{0\})
\Tr_{2;c}({p}^{2 a}\phi^2_{\le i-1})
\crcr
&&+
\sum_c
\Gamma^{(c)}_{2;2a, \; i-1} (\{0\})
\Tr_{2;c}( {p}^{4 a}\phi^2_{\le i-1})
+
{\tilde R}(\phi_{\le i-1})\,,
\eea
where ${\tilde R}(\phi_{\le i-1})$ contains all finite contributions. 

Following step by step, the previous analysis, we deliver the 
the effective couplings at scale $i-1$: 
\bea
Z_{b,\, i-1} &=& 1\,,
\\
\mu_{{\rm ren}, i-1} &=& \mu_{i-1} - 
\Sigma_{i-1} (\{0\})
\,,
\label{eq:renormmassxi}
\\
Z_{a, i-1} 
&=&
-\Gamma^{(c)}_{2;a, i-1}(\{0\})\,,
\\
Z_{2a, i-1} 
&=&
-\Gamma^{(c)}_{2;2a, i-1}(\{0\})\,,
\label{eq:rgeqnsx}
\eea


Following Proposition {\ref{prop:listx}}, we work with the set of parameters
$d=3$, $D=1$, $a = \frac{1}{2}$, and $b= 1$ so that the model is just-renormalizable.


\subsection{Self energy and mass renormalization}
\label{sect:}

 For the model $\times$, we calculate the self energy which we denote again by 
$\Sigma_{b}(\{ p \})$ and which has an expression similar to \eqref{selfnrg}, 
where the graphs ${\cal G}^{(c)}_{2,\iota}$ should be choosen 
among all  amputated 1PI 2pt-graphs 
with  boundary  of the form ${\rm Tr}_{2; c} (p^{2 b} \phi^2)$. 

Up to the first order in perturbation theory, 
${\cal G}^{(c)}_{2,\iota} \in \{ m^{(1)}, m^{(2)}, m^{(3)}\}$. 
(At second order in perturbation theory, the interested reader 
may look at Fig. \ref{fig:mmee} in Appendix \ref{app:2ndOx}.)

Recall that $\Sigma_{b}(\{p \})$ corresponds to
$  \Sigma (\{0\}) 
+  \sum_{c}
\vert p_{c}\vert^{2 b} \partial_{\vert p_{c}\vert^{2 b}} \Sigma \big\vert_{\{p\}=0}$. 
Considering  that the second term vanishes, i.e. $\partial_{\vert p_{c}\vert^{2 b}} \Sigma \big\vert_{\{p\}=0}=0$, only the contribution $\Sigma (\{0\}) $,
 namely the mass renormalization, survives.


For $m^{(c)}$,  $c=1,2,3$, 
$\omega_{d; \times}({m^{(c)}}) = 0$.
This graph belongs to the class III in Table \ref{tab:listprim2}, also, it appears at first order. 
\begin{figure}[H]
\centering
     \begin{minipage}[t]{0.7\textwidth}
      \centering
\def\svgwidth{0.8\columnwidth}
\tiny{ \input{m1.pdf_tex} }
\caption{{\small {The graph $m^{(c)}$. In the case $d=3$.  
${|{\bf q}_{\check c}|}^{2 b} = |q_{1}|^{2 b} + |q_{2}|^{ 2b}$. Here the parameters should follow that of 
Proposition {\ref{prop:listx}}.
}}}
\label{fig:m1}
\end{minipage}
\end{figure}
We sum the contributions of the graphs $m^{(c)}$ 
and write: 
\bea
\Sigma_{b}(\{p \}) = 
\sum_{c=1}^{d}
K_{m^{(c)}} S_{m^{(c)}} ( \{p\})
&=&
\sum_{c=1}^{d}
2 \big(- \frac{\lambda^{(c)}}{2} \big) \sum_{\{q_{\check c}\}} 
\frac{1}{({|{\bf q}_{\check c}|}^{2 b} + p^{2 b}_{c} + \mu )}
\,,
\eea
where $K_{m^{(c)}} = 2$.

The renormalized mass  \eqref{eq:renormmassxi} finds the expression: 
\beq
{\mu_{\rm ren}} 
=  \mu - \Sigma_b(\{p\})\Big|_{ \{p\}=\{0\} } 
=  \mu + \sum_{c=1}^{d} \lambda^{(c)}  \sum_{\{q_{\check c}\}} 
\frac{1}{({|{\bf q}_{\check c}|}^{2 b}  + \mu )}
\,.
\eeq
Now assert color independence, namely $\lambda^{(c)} = \lambda$, and ${\lambda_{\times}}^{(c)} = \lambda_{\times}$, 
the renormalized mass can be expressed as 
\bea
\mu_{\rm ren}
&=&
\mu +
d  \lambda \sum_{\{q_{\check c}\}} 
\frac{1}{({|{\bf q}_{\check c}|}^{2 b}  + \mu )}
=
\mu + d \, \lambda \, S_{1}
\,,
\eea
where $S_{1}$ is defined earlier  \eqref{eq:S1S2}.
We perform a similar analysis as done in section \ref{sect:selfenergmass}, 
use \eqref{eq:runningmass2x} for $\widetilde S_{1,i}$ and  \eqref{eq:S1S2dimfuldimless}
to obtain 
 the running of the mass at leading order,
\bea
\mu_{i-1} = \mu_{i} + d \, \widetilde S_{1,i} \, \lambda_i 
\,.
\eea
Note that the mass does not have a scaling dimension, and
 that $\lambda$ does not run  in this model (see \eqref{eq:rgeqnsx}), then 
we write the RG flow equation for $\mu$: 
\bea
- (\mu_{i-1}- \mu_i)
&=&
\frac{\partial \mu_i}{\partial i}
= - d \, \widetilde S_{1,i} \, \lambda_i 
\,,
\crcr
\frac{ \partial\mu_i}{\partial ((\log M) i)}
&=&
\partial_t   \mu (t) 
= 
- \beta_{\mu, 1} \, \lambda
\,,
\crcr
\beta_{\mu, 1} 
&=&
\frac{d}{\log M}  \, \widetilde S_{1,i} = 2d \pi >0
\,.
\label{eq:massRGfirstorderx}
\eea
Fixing an initial condition  at $t_0$, 
this integrates to give
\bea
\mu (t) = - (t-t_0) \beta_{\mu,1} \,\lambda  + \mu(t_0)
\,.
\label{eq:massfirstorderx}
\eea




Therefore, the mass in this theory, the model $\times$ grows linearly in $t$ in its magnitude.








\subsection{Computing $\Gamma_{2;a}$ and  $Z_a$ renormalization}

We address here the flow of the 2-point coupling $Z_a$. 
The 2-pt diagram sum $\Gamma^{(c)}_{2;a} (\{p\})$ follows again
an equation similar to \eqref{Gamma2Za} of the previous section \ref{compGamma2+}. 
The sum performs over all amputated 1PI 2-pt graphs at 1-loop 
of the form ${\rm Tr}_{2;c} ({ p}^{2 a} \phi^2)$.



At first order, the diagrams $n_e^{(c)}$, $c=1,2,3$, see Fig. \ref{fig:nee}, contribute to the flow. 
(The next order in perturbation theory will have the additional graphs 
of Appendix \ref{app:2ndOx}.) 
In  \eqref{eq:sigma}, we have defined $\partial_{\vert p_{c}\vert^{2 a}} \Sigma\vert_{\{p\}=0} \equiv \Gamma^{(c)}_{2;a} (\{0\})$ which is divergent.

The graphs $n_{e}^{(c)}$ satisfy  $\omega_{d;\times}(n_{e}^{(c)}) = 0$
and   belong  to the class I in Table \ref{tab:listprim2}. 

\begin{figure}[H]
\centering
     \begin{minipage}[t]{0.7\textwidth}
      \centering
\def\svgwidth{0.7\columnwidth}
\tiny{
\input{nee.pdf_tex}
}
\caption{ {\small  For $d=3$, $n_{e}^{(c=1)}$ is shown above in colored (left) and stranded (right) representations. $\{p\} = \{p_{1}, p_{2}, p_{3}\}$ and ${|{\bf p}_{\check c}|}^{2 b}={|p_{}|}^{2 b}+{|p_{2}|}^{2 b}$.
}} 
\label{fig:nee}
\end{minipage}
\end{figure}
The Feynman amplitude associated with the graph $n_{e}^{(c)}$ shown in Fig. {\ref{fig:nee}} is
\bea
K_{n_{e}^{(c)}} S_{n_{e}^{(c)}} ( \{p\})
&=&
2 
\Big(- \frac{{\lambda_{\times}}^{(c)}}{2} \Big)
\vert p_{c}\vert^{2 a}
\sum_{q_{c}}
\frac{\vert q_{c}\vert^{2 a}}{(\vert q_{c}\vert ^{2 b} + {|{\bf p}_{\check c}|}^{2 b} + \mu)}\,,
\eea
where  $K_{n_{e}^{(c)}} = 2$.

Let us now compute the renormalization of $Z_a$, up to first order in perturbation theory 
\bea
 \vert p_{c}\vert^{2 a} 
\Gamma_{2;a}(\{p\})
&=&
- \vert p_{c}\vert^{2 a}  Z_{a}^{(c)} +
K_{n_{e}^{(c)}} S_{n_{e}^{(c)}} ( \{p\})
\nonumber
\\
&=&
- \vert p_{c}\vert^{2 a}  Z_{a}^{(c)} 
- \lambda_{\times}^{(c)}
\vert p_{c}\vert^{2 a}
\sum_{q_{c}}
\frac{\vert q_{c}\vert^{2 a}}{(\vert q_{c}\vert ^{2 b} + {|{\bf p}_{\check c}|}^{2 b} + \mu)}
\,.
\eea
Therefore, to the first order in perturbation, 
\bea
Z_{a, {\rm ren}}^{(c)} 
&=& 
- \Gamma_{2;a}^{(c)} (\{p\}) \big \vert_{\{p\}=0}
=
Z_{ a}^{(c)} 
+ \lambda_{\times}^{(c)}
\sum_{q }
\frac{\vert q \vert^{2 a}}{\vert q \vert ^{2 b} 
+ \mu}
\,.
\eea
By imposing the color independence,  we write
\bea
Z_{a, {\rm ren}}
=
Z_{ a}
+ \lambda_{\times}\,
S_2
\,,
\eea
where $S_2$ is given in \eqref{eq:S1S2}.
In multiscale analysis, we write
\bea
Z_{a, i-1}
=
Z_{ a, i}
+ \lambda_{\times, i}\,
S_{2, i}
\,,
\eea
where $S_{2, i}$ is given in \eqref{s2idebut}. 

We approximate $S_{2, i}$ in \eqref{eq:S2iexacttimes}. 
Now, we use the fact that  $Z_a$  has no scaling dimension \eqref{eq:S1S2dimfuldimless},  
and that $\lambda_{\times}$  does not run as presented in \eqref{eq:rgeqnsx}, 
and express the $\beta$-function of the coupling $Z_a$ as: 
\bea
- (Z_{a,i-1}- Z_{a,i})
&=&
\frac{\partial Z_{a,i}}{\partial i}
= - \, \widetilde S_{2,i} \, \lambda_{\times,i} 
\,,
\crcr
\frac{\partial  Z_{a,i}}{\partial ((\log M) i)}
&=&
\partial_t Z_a(t)  =  - \beta_{Z_a} \, \lambda_\times
\,,
\crcr
\beta_{Z_a} 
&=&
\frac{\widetilde S_{2,i}}{\log M} = 2 >0
\,,
\label{eq:ZaRGfirstorderx}
\eea
where $\widetilde S_{2,i}$ is given in \eqref{eq:S2iexacttimes}.
The solution is straightforward, after fixing an initial condition at some $t_0$
\bea
Z_{a}(t) 
= - (t-t_0) \, \beta_{Z_a} \, \lambda_\times + Z_{a}(t_0)
\,.
\label{eq:Zafirstorderx}
\eea
Therefore, in exactly the same manner as the mass in this theory, the $2$-pt coupling $Z_a$ in the model $\times$, grows linearly in $t$ in its magnitude.















\subsection{Computing of $\Gamma_{2;2a}$ and 
$Z_{2a}$ renormalization}

We focus on the RG flow for $Z_{2a}$. 
We denote by $\vert p_{c}\vert^{4 a}  \Gamma_{2;2 a}^{(c)}(\{p \}) $
the sum is over all amputated 1PI 2-pt graphs at 1-loop 
whose boundaries  follow the pattern  of ${\rm Tr}_{2;c} ({ p}^{4 a} \phi^2)$.


The diagrams that will contribute at 1-loop are denoted $m_{e e}^{(c)}$
and  depicted in Fig.  \ref{fig:mee}. 
(Higher order diagrams are listed in Appendix \ref{app:2ndOx})
For $m_{e e}^{(c)}$, $\omega_{d; \times}(m_{e e}^{(c)}) = 0$ 
and belongs to the class II in Table \ref{tab:listprim2}. 
\begin{figure}[H]
\centering
     \begin{minipage}[t]{0.5\textwidth}
      \centering
\def\svgwidth{1\columnwidth}
\tiny{
\input{mee.pdf_tex}
}
\caption{{\small  $m_{e e}^{(c=1)}$ for $d=3$ is illustrated in colored (left) and stranded (right) representations. $\{q_{\check c}\} = \{q_{2}, \; q_{3}\}$, ${|{\bf q}_{\check c}|}^{2 b} = {|q_{2}|}^{2 b} + {|q_{3}|}^{2 b}$.}}
\label{fig:mee}
\end{minipage}
\end{figure}
\bea
K_{m_{ee}^{(c)}} S_{m_{e e}^{(c)}} (\{p\}) 
&=&
2
\Big( - \frac{{\lambda_{\times}}^{(c)}}{2}\Big) 
\vert p_{c}\vert^{4 a}
\sum_{\{q_{\check c}\}} \frac{1}{( {|{\bf q}_{\check c}|}^{2 b} + \vert p_{c}\vert ^{2 b} + \mu )}
\,,
\eea
where $K_{m_{e e}^{(c)}} = 2$.
Given this,  we compute the renormalization of $Z_{2 a}$: 
%
where
\bea
\vert p_{c}\vert^{4 a}
\Gamma_{2; 2 a}^{(c)}\big (\{p\}) 
&=&
- \vert p_{c}\vert^{4 a}   Z_{2 a}^{(c)} + 
K_{m_{ee}^{(c)}} S_{m_{e e}^{(c)}} (\{p\}) 
\nonumber
\\
&=&
-\vert p_{c}\vert^{4 a} Z_{2 a}^{(c)} 
- {\lambda_{\times}}^{(c)}
\vert p_{c}\vert^{4 a}
\sum_{\{q_{\check c}\}} \frac{1}{( {|{\bf q}_{\check c}|}^{2 b} + \vert p_{c}\vert ^{2 b} + \mu )}
\,.
\eea
After a small manipulation,  we obtain
\bea
Z_{2 a, {\rm ren}}^{(c)} 
&=&
Z_{2 a}^{(c)} 
+ {\lambda_{\times}}^{(c)}
\sum_{\{q_{\check c}\}} \frac{1}{( {|{\bf q}_{\check c}|}^{2 b} 
+ \mu )}
\,.
\eea
The equation for $Z_{ 2 a}$ is reached after requiring color 
independence: 
\bea
Z_{2 a, {\rm ren}}
=
Z_{ 2 a}
+ \lambda_{\times}\,
S_1
\,,
\eea
where we recognise  $S_1$  \eqref{eq:S1S2}.
Passing through the same steps via multiscale
regularization, the coupling equation can be written  as: 
\bea
Z_{2 a, i-1}
&=&
Z_{2 a, i}
+ \lambda_{\times, i}\,
S_{1, i}
\,, \crcr
- (Z_{2a,i-1}- Z_{2a,i})
&=&
\frac{\partial Z_{2a,i}}{\partial i}
= - \, \widetilde S_{1,i} \, \lambda_{\times,i} 
\,,
\crcr
\frac{\partial  Z_{2 a,i}}{\partial ((\log M) i)}
&=&
\partial _t Z_{2 a}(t) 
= 
- \beta_{Z_{2a}} \, \lambda_\times
\,,
\crcr
\beta_{Z_{2a}} 
&=&
\frac{\widetilde S_{1,i}}{\log M} = 2 \pi >0
\,,
\label{eq:Z2aRGfirstorderx}
\eea
where $\widetilde S_{1,i}$ is computed in \eqref{eq:runningmass2x}.
Then, at this order of perturbation, 
$Z_{2a}$ yields also a linear function in the time scale $t$.  
\bea
Z_{2 a}(t) 
&=& - (t-t_0) \, \beta_{Z_{2a}} \, \lambda_\times + Z_{a}(t_0)
\,.
\label{eq:Z2afirstorderx}
\eea
The argument goes the same as the mass and the other $2$-point coupling $Z_a$, i.e., the 2-pt coupling $Z_{2a}$ in the model $\times$ grows linearly in $t$ in its magnitude.


\


\noindent{\bf Summary --} 
We  give a summary of the 1-loop RG flow equations for the
model $\times$ and their solutions: 
\begin{center}
\begin{tabular}{l|l}
\hline 
\hline 
$\partial_t \,\lambda (t) =0 $  &   $\lambda(t)  = cste$ \\ 
%%%
$\partial_t\,  \lambda_{\times} (t) = 0 $  &  $\lambda_{\times} (t)=cste$\\
%%
$\partial_t \, \mu (t)= - \beta_{\mu,1} \lambda  $\,,
\qquad   $\beta_{\mu,1}= 2d\pi >0$
 & $ \mu (t)= 
 -2d\pi   \lambda (t-t_0)+ \mu (t_0)$ \\ 
%%% 
$\partial_t \,   Z_a(t)
 = 
-  \beta_{Z_a} \, \lambda_{\times} $\,,
\qquad 
  $\beta_{Z_a} = 2 > 0$   &
$Z_{2 a}(t) 
= -2  \lambda_\times \,  (t-t_0)+ Z_{a}(t_0)$\\ 
 %%% 
 $\partial_t \,   Z_a(t)
 = 
- \beta_{Z_{2a}} \, \lambda_\times $ 
\,,
\qquad 
  $\beta_{Z_a} = 2\pi > 0$   &
$Z_{2 a}(t) 
= - 2 \pi \lambda_\times \,(t-t_0)  \,  + Z_{a}(t_0)$\\ 
 \hline 
 \hline 
\end{tabular}
\label{tab:RGx}
\end{center}






\subsection{Integration at arbitrary loops}
\label{sec:allloopsx}



The power counting of the model $\times$, provided by Proposition \ref{prop:listx},  gives us a remarkable amount of information on the $\beta$-functions of the couplings in an arbitrarily high order in perturbation theory. 
The information is stringent enough to let us present explicit general forms of $\beta$-functions of the couplings in the model $\times$ at all orders in perturbation theory.


\

\noindent{\bf 4-pt couplings  $\lambda$ and $\lambda_\times$  RG equation --}
The power counting theorem of the model $\times$ with Proposition \ref{prop:listx} determines that at all orders in pertubation theory, there are no amplitudes which are divergent contributing to the renormalization of 4-pt couplings $\lambda$ and $\lambda_\times$. Hence, $\lambda$ and $\lambda_\times$ of the model $\times$ are constant and do not flow with scales, therefore 
\bea
\partial_t \lambda &=& 0
\,,
\crcr
\partial_t \lambda_{\times} &=& 0
\,, 
\eea
which trivially yield
\bea
\lambda (t) &=& cste
\,,
\crcr
\lambda_{\times}(t) &=& cste
\,.
\label{eq:4ptallordersx}
\eea

\



\noindent{\bf Mass,  2-pt couplings $Z_a$ and $Z_{2a}$ RG equations --}
Observation of Proposition \ref{prop:listx} tells us that the mass renormalization is decided by the class III, where only exactly one $\lambda$ and a number of $\lambda_\times$ contribute.

The $Z_a$ renormalization is decided by the class I, where only $\lambda_\times$ contributes.

We also notice that only $\lambda_\times$ contributes to the renormalization of $Z_{2a}$, as class II dictates.

With \eqref{eq:4ptallordersx}, one immediately conclude that the coupled differential RG equations for the couplings of the model $\times$, which generalize trivially the equations for the first order RG equations \eqref{eq:massRGfirstorderx}, \eqref{eq:ZaRGfirstorderx}, and \eqref{eq:Z2aRGfirstorderx}, to arbitrary $n$-th orders, by introducing polynomials $P_n (\lambda_\times)$, $Q_n ( \lambda_\times)$, and $R_n ( \lambda_\times)$,  are
\bea
\partial _t \mu (t) 
&=& 
\lambda \, 
P_n( \lambda_\times)
\,,
\crcr
\partial _t Z_{a}(t) 
&=& 
Q_n( \lambda_\times)
\,,
\crcr
\partial _t Z_{2 a}(t) 
&=& 
R_n( \lambda_\times)
\,.
\eea

One can immediately solve such system of equations and obtain too all orders in perturbation theory,
\bea
\mu (t) &=& (t- t_0) \lambda \, P_n ( \lambda_\times) + \mu(t_0)
\,,
\crcr
Z_a (t) &=& (t- t_0) Q_n ( \lambda_\times) + Z_a(t_0)
\,,
\crcr
Z_{2 a}(t) &=& (t-t_0) R_n (\lambda_+) + Z_{2a}(t_0)
\,,
\eea
where we notice that all couplings above grow linearly in $t$ in their respective  magnitudes.










\section{Conclusion}
\label{conc}


We have computed explicitly 
the one-loop $\beta$-functions of the couplings of enhanced tensor field theories, the model $+$ and the model $\times$ in the first order in perturbation theory, summarized in Table \ref{tab:RG+} and Table \ref{tab:RGx}. 
The system RG flow equations can be explicitly computed. 

Both models $+$ and $\times$ do have a constant wavefunction renormalization ($Z_b=1$). Nevertheless, we have obtained some nontrivial renormalization group flows in some of the couplings. 

For the model $+$, the enhanced $4$-pt coupling $\lambda_+$ does not run and therefore is constant, whereas the ordinary $4$-pt coupling $\lambda$ grows linearly in $t= \log k/k_0$, where $k$ is momentum scale, in its magnitude. This statement is true for all orders in perturbation theory. 
Once we include all orders of perturbation theory, we may find asymptotic safety for the coupling $\lambda$. 
The $2$-pt coupling $Z_a$ has an exponential behavior in $t$. 
To the first order in perturbation theory, we found that the mass behaves different from relevant operators and grows linearly (but negative) in $t$ in the UV. In the IR, the mass behaves like a relevant operator and blows up exponentially.
Additionally, the renormalization equation for the mass to all orders in perturbation theory is also predicted in \eqref{eq:mass+allorders}.



For the model $\times$, the $4$-pt couplings $\lambda$ and $\lambda_\times$ do not run and therefore constant at all orders. On the other hand, all the $2$-pt couplings (mass, $Z_a$  and $Z_{2a}$) grow in their magnitude linearly in the scale $t = \log k/k_0$.


Furthermore, with the explicit forms of the running of the couplings at first order, together with the observation from the power counting theorems for each model that we obtained earlier in \cite{BenGeloun:2017xbd}, we are able to deduce the behaviors of $\beta$-functions of couplings to all orders in perturbation theory for both the models $+$ and $\times$, see Section \ref{sec:allloops+} and Section \ref{sec:allloopsx}.
We will leave it in the future, to compute precisely the coefficients of all-loop order. The $\beta$-functions of these models seem simple enough that we should be able to resum at all orders in perturbation theory.


Another interesting study in the future is to interpret the propagator differently and compute the renormalization group flows in perturbation theory based on such a covariance measure.
In our current study, we chose the kinetic term to be $\Tr(p^{2b} \phi^2) + \mu \phi^2$. However, one could have chosen the kinetic term for the model $+$ to be composed of several terms
$Z_b \, \Tr_2(p^{2b} \phi^2) + Z_a \, \Tr_2(p^{2a} \phi^2) + \mu\,  \Tr_2 (\phi^2)$\,.
This will entail that we redefine the field as:
\beq
\phi \to 
Z_b^{1/4} Z_a^{1/4} \phi
\,,
\eeq
and lead us to an action in the form of
\bea
&&
\Tr_2 
\Bigg(
\big( Z_b^{1/4} Z_a^{1/4} {\bar \phi}\big)
\cdot 
\Big( 
\frac{\sqrt{Z_b}}{\sqrt{Z_a}} \vert p\vert^{2b} + \frac{\sqrt{Z_a}}{\sqrt{Z_b}} \vert p\vert^{2a}
+ \frac{\mu}{\sqrt{Z_a Z_b}} 
\Big)
\cdot
\big( Z_b^{1/4} Z_a^{1/4}  \phi\big)
\Bigg)
\crcr
&+&
\frac{\lambda_+}{2 \, Z_a Z_b} 
\Tr_4 
\big( 
p^{2a} \phi^4 
\big)
+
\frac{\lambda}{2 \, Z_a \, Z_b} 
\Tr_4 
(
\phi^4
)
\,.
\eea
In the case that $Z_b$ is still a constant and $Z_a$  divergent, 
this will send ${\sqrt{Z_b}}/{\sqrt{Z_a}} \rightarrow 0$. 
Therefore, one may interpret that even though we started out with a theory with a kinetic term
$ \Tr_2(p^{2b} \phi^2) +  \Tr_2(p^{2a} \phi^2) + \mu \Tr_2 (\phi^2)$, we may flow toward and end up with a theory with a kinetic term with only
$ \Tr_2(p^{2a} \phi^2) + \mu \Tr_2 (\phi^2)$\,.














\section{Acknowledgements}
We would like to thank Dario Benedetti, Sylvain Carrozza, Riccardo Martini, and Fabien Vignes-Tourneret for the insightful discussions.
The authors would also like to thank the thematic program "Quantum Gravity, Random Geometry, and Holography" 9 January - 17 February 2023 at Institut Henri Poincar\'e, Paris, France
for the platform for discussions and the collaboration and letting us progress further on this project.
The authors acknowledge support of the Institut Henri Poincar\'e (UAR 839 CNRS-Sorbonne Universit\'e) and LabEx CARMIN (ANR-10-LABX-59-01).

















\section*{ Appendix}
\label{app}



\appendix

\renewcommand{\theequation}{\Alph{section}.\arabic{equation}}
\setcounter{equation}{0}





\section{Euler-Maclaurin formula 
and Feynman amplitude approximations}
\label{app:SumtoInt}

We briefly review 
the approximation of 
a discrete spectral sum by an integral
using Euler-Maclaurin formula \cite{BenGeloun:2013vwi}. In perturbation theory, this approximation is well controlled and sufficient to achieve our calculations. 

Consider $h_n(x) = x^n e^{-A x^a}$, with $x \ge 0$, $n \in \N$, and the sum $\sum_{p=0}^{\infty} h_n(p)$.
We use Euler-Maclaurin formula to obtain for a finite integer $q \le 1$,
\beq
\sum_{p=1}^q h_n(p) = \int_1^q h_n(p) dp + R(q)
\,, 
\eeq
where
\beq
R(q) = - B_1\big(h_n(1) + h_n(q) \big)
+
\sum_{k=1}^\infty 
\frac{B_{2k}}{2 k!} 
\big(
h_n^{(2 k -1)} (q)
-
h_n^{(2k-)} (1)
\big)\,,
\eeq
where $B_k$ are Bernoulli numbers and $h^{(2k-1)}$ denotes $2k-1^{\rm th} $ derivative of $h_n(p)$ with respect to $p$.

One can show that 
\beq
{\rm lim}_{q \rightarrow \infty} R(q) 
=
- B_1 
- \sum_{k=1}^{\infty} \frac{B_{2 k}}{2 k}
\binom{n}{2k-1} 
+ \cO(A)
=
\cO(1) 
+ \cO(A)
\,.
\eeq

For any $A$, the integral below is exact,
\bea
{\rm lim}_{q \rightarrow \infty}\int_1^q h_n(p) dp 
=
\frac{1}{a}
A^{- \frac{1+n}{a}}
\Gamma\Big[ \frac{1 + n}{a}, A \Big] 
=
\frac{1}{a}
A^{- \frac{1+n}{a}}
\Gamma\Big[ \frac{1 + n}{a}\Big] 
-
\frac{1}{1+n}
+
\cO(A)
\,,
\eea
with the incomplete Gamma function $\Gamma [ \cdot, \cdot]$, and Euler Gamma function $\Gamma[\cdot]$.
This then gives us
\bea
\sum_{p=1}^{\infty} h_n(p)
=
{\rm lim}_{q \rightarrow \infty}
\sum_{p=1}^q h_n(p)
=
\frac{1}{a}
A^{- \frac{1+n}{a}}
\Gamma\Big[ \frac{1 + n}{a}\Big] 
-
\frac{1}{1+n}
+\cO(1)
+
\cO(A)\,.
\label{sumh}
\eea


\noindent{\bf Expanding $\widetilde{S}_{0,i}$.}  
The goal is to approximate $S_{0,i}$ \eqref{s0idebut}
using the above developments. 
Use Euler-Maclaurin expansion, we compute the sum 
\bea
\tilde \tau 
&=&
\sum_{q\in \Z}\vert q\vert^{4a}
e^{-(\alpha+  \alpha') 
\vert q\vert^{2b} }
= 2\sum_{q= 1}^\infty   q^{4a} 
e^{-(\alpha+  \alpha') \vert q\vert^{2b} }
\crcr 
&& 
= 2\int_{1}^{\infty}
dq\,
 q^{4a}
e^{-(\alpha+  \alpha') 
\vert q\vert^{2b} }
+
R
 = 2 \times 
\frac{1}{2b (\alpha+  \alpha')^{\frac{4a+1}{2b}}}
\Gamma\left(\frac{4a+1}{2b},\alpha+  \alpha'\right)
+R
\crcr
&&
= 
\frac{(\alpha+  \alpha')^{\frac{-(4a+1)}{2b}}}{b}
\Gamma\left(\frac{4a+1}{2b},\alpha+  \alpha'\right)
+R \,,
\label{sumoverq}
\eea
where $\Gamma(s,x)= \int_{x}^\infty t^{s-1} e^{-t} dt$ 
denotes the upper incomplete Gamma function. $R$ is the Euler-Maclaurin remainder which behaves, according to \eqref{sumh}, as $R = {\mathcal O}(1) + {\mathcal O}(\alpha + \alpha')$.  $\Gamma(s,x)$ admits a known asymptotic expansion when $x\to 0^+$ given by 
\bea
\Gamma(s,x)= 
\Gamma(s) - \frac{x^s}{s} + {\mathcal O}(x^{s+1})
\label{eq:incompletegammaexpand0}
\eea
with the obstruction $s\notin \{0,-1,-2,\dots\}$. 
Now, for simplicity, 
we constrain our 
model and use $D=1$, 
but keep $a$ and $b$
accordingly as 
given by Theorem \ref{theorem+}. 
We use the above expansion \eqref{eq:incompletegammaexpand0}
in \eqref{sumoverq} as $ \alpha+  \alpha'$ is a small parameter (recalling that $\alpha$ and $\alpha'$ are small in the UV)
\bea
&&
\sum_{q\in \Z}\vert q\vert^{4a}
e^{-(\alpha+  \alpha') 
\vert q\vert^{2b} } 
\crcr
&&
=
\frac{(\alpha+  \alpha')^{\frac{-(4a+1)}{2b}}}{b}
\left( 
\Gamma\left(\frac{4a+1}{2b}\right) 
- \frac{(\alpha+  \alpha')^{\frac{4a+1}{2b}} }{\frac{4a+1}{2b}} + \cO\big((\alpha+  \alpha')^{\frac{4a+1}{2b}+1}\big) \right)  
+ R
\crcr
&&
= 
\frac{1}{b}
\Gamma\left(\frac{4a+1}{2b}\right) (\alpha+  \alpha')^{\frac{-(4a+1)}{2b}} 
 + {\mathcal O}(1) + {\mathcal O}(\alpha+  \alpha') \,.
\eea
Having a look at \eqref{s0idebut}, 
using  $e^{-(\alpha+  \alpha')\mu} = 1  
+ \cO(\alpha+  \alpha')$,
we integrate this expression above over $\alpha$ and $\alpha'$, 
with $\frac{1}{b}
\Gamma\left(\frac{4a+1}{2b}\right)= \frac1b \Gamma(2) = \frac1b$, and 
write: 
\bea
\widetilde S_{0,i}&=&
\int_{M^{-2 b i}}^{M^{-2 b(i-1)}}
d\alpha 
\int_{M^{-2b i}}^{M^{-2 b(i-1)}}
d\alpha' 
\Big[ 
 \frac{1}{b}(\alpha+  \alpha')^{-2} 
+ 
 {\mathcal O}\big( (\alpha+  \alpha')^{-1}
 \big)
 \Big] 
 \crcr
 &=&
\int_{M^{-2 bi}}^{M^{-2 b(i-1)}}
d\alpha 
\Big[ -\frac{1}{b} (\alpha +\alpha ')^{ -1}
 +
 {\mathcal O}\big( 
\log(\alpha+  \alpha')\big)
\Big]_{\alpha'=M^{-2 bi}}^{\alpha'=M^{-2 b(i-1)}}
 \crcr
 &=&
\frac{1}{b}
 \Big[
 \log(\alpha +M^{-2b i})
  - 
  \log(\alpha +M^{-2b(i-1)})
 \crcr
 &&
+ 
\Big(
  \;
  {\mathcal O}\big( \alpha)
 +
 {\mathcal O}\big( \alpha \log(\alpha+  M^{-2b i})\big)
+ 
  {\mathcal O}\big( M^{-2b i} \log(\alpha+  M^{-2b i})\big)
\Big)
 \Big]_{\alpha = M^{-2b i}}^{\alpha =M^{-2b (i-1)}}
 \crcr
 &=&
 \frac{1}{b}
 \Big[
 -
 \log(2 M^{-2b i})
+ 2
 \log(M^{-2b (i-1)} +M^{-2b i})
  - \log(2 M^{-2b (i-1)} )
 \crcr
 &&
 + \;
  {\mathcal O}\big( M^{-2b i})
 +
 {\mathcal O}\big( M^{-2b i} \log( M^{-2b i})\big)
 \Big]
 \crcr 
&=&
 \frac{1}{b}
 \Big[
2 \log [M^{-2b i}(M^{2b} +1)]
 - \log(2 M^{-2b i})
  - 
  \log(2M^{-2b i+2b })
 \Big]
 \crcr
 &&
 + {\mathcal O}(M^{-2b i})
+  {\mathcal O}\big( M^{-2b i} \log (M^{-2b i})\big)
 \cr\cr 
&=&
 \frac{1}{b}
 \Big[
2 \log (M^{2b} +1)
 - \log(4)
  - 
  \log(M^{2b })
 \Big]
 + {\mathcal O}( M^{-2b i})
+  {\mathcal O}\big(  M^{-2b i} \log (M^{-2b i})\big) \crcr
& = & 
\frac{1}{b}
 \Big[
  \log 
  \frac{(M^{2b} +1)^2}{
  4 M^{2b }}
 \Big]
 + {\mathcal O}( M^{-2b i})
+  {\mathcal O}\big(  M^{-2b i} \log (M^{-2b i})\big)
\crcr
&&
\eea
Thus $\widetilde  S_{0,i}$ is approximated by 
\bea
\widetilde  S_{0,i} =
\frac{1}{b}
 \log 
  \frac{(M^{2b} +1)^2}{
  4 M^{2b }}
 + {\mathcal O}( M^{-2b i}
 \log(M^{-2b i})) 
\eea
As $M\ge 1$, 
$\frac{(M^{2b} +1)^2}{
  (4 M^{2b })}>1$, 
  then the leading coefficient in the above expanding is positive. 


\


We can put the above in 
a standard $\log M$ approximation by expanding at large enough
$M\gg 1$: 
\bea
&&
\frac{1}{b}
 \log 
  \frac{(M^{2b} +1)^2}{
  (4 M^{2b })}
 = 
\frac{1}{b}
 \Big[
\log  (M^{2b})  +  2 \log (1+1/ M^{2b} )
 - \log(4)
 \Big] 
 \crcr
 &&
 = \frac{1}{b}\log M^{2b}
 + \frac{2}{b}\left (\frac{1}{M^{2b}}
 - \log(2) \right) 
 + \cO(M^{-4b})
\eea 
At large $M$, the first term will correspond to 
the standard $\log \Lambda $ divergence for marginal coupling. 




  \ 


\noindent{\bf Dimensionful computations for $S_{0,i}$.}
We address now the crucial question of the dimension 
in our computation. The previous calculations
of were performed without taking care of that aspect
that we now restore. 
The couplings and fields have scaling dimensions.
Expressing the propagator in Schwinger parameterization as in  \eqref{eq:Ctilde}, we notice that $\alpha$ should have 
a dimension of $-2b$ in units of momentum scale $k$. In other words, we can write
$\alpha = k^{-2b} \, {\tilde {\alpha}}$
where ${\tilde {\alpha}}$ is dimensionless.
There is a dimensionful quantity 
$\tau$ associated with $\tilde \tau$ \eqref{sumoverq},
in which $q$ acquires dimension of $1$ in the units of the momentum scale. 
The integral approximates the discrete sum  over $q$ as in the second line of \eqref{sumoverq}, but gains one more dimension
in the computation of $\tau$. 
A similar fact concerning discrete sums having scaling dimensions 
was advocated and used in \cite{Benedetti:2014qsa}. 
This also can be understood by the power counting theorem: 
in order to make the couplings marginal, so $\log$-divergent in 
the cut-off, discrete sums must carry scale dimension. 
We perform the following change of variables to let the dimensions be explicit in terms of a momentum scale $k$:
\bea
&&
q  = k \tilde q \, , \qquad  \tilde q \in  \Z\crcr
&& \alpha = k^{-2b} \, {\tilde {\alpha}} 
\label{eq:dimfuldimless}
\eea
We obtain in terms of dimensionless $\tilde \tau$ 
and $\widetilde S_{0,i}$: 
\bea
&&
\tau  =   k^{4 a +1} \tilde \tau 
\crcr 
&&
S_{0,i} =  k^{4 a +1} 
 k^{-4b}
\widetilde S_{0,i}  = \widetilde S_{0,i}
\,,
\eea
where in the last equality, we notice $4 a +1 - 4b = 0$, and therefore
we get $S_{0,i}$ \eqref{s0iapprox}. 



\ 


\noindent{\bf Expanding $\widetilde S_{1,i}$ and  $\widetilde S_{2,i}$ for model $+$.} 
We use a similar technique to  provide an approximation of  $S_{1,i}$ 
 \eqref{s1idebut} and  $S_{2,i}$  \eqref{s2idebut}. 

Express the propagators in Schwinger representation and carefully converting the above sums into integrals and get 
\bea
\widetilde
S_{2,i} 
= 
\int_0^\infty 
d\alpha \chi^{i} (\alpha) 
e^{-\alpha
\mu_i}
\sum_{q \in \Z}\vert q\vert^{2 a}
e^{-\alpha
\vert q\vert^{2 b} }\,.
\label{eq:masslambda+}
\eea
This integral is similar to \eqref{sumoverq}, changing only  $\alpha+\alpha'$ to $\alpha$ 
and for a particular choice of $2a$. 
We perform an approximation in an  analogous
way as before and get: 
\bea
\sum_{q\in \Z}\vert q\vert^{2a}
e^{-\alpha 
\vert q\vert^{2b} }
= 
\frac{\alpha^{\frac{-(2a+1)}{2b}}}{b}
\Gamma\left(\frac{2a+1}{2b},\alpha\right)
+R 
 = \frac{1}{b}
\Gamma\left(\frac{2a+1}{2b}\right) \alpha^{\frac{-(2a+1)}{2b}} 
 + {\mathcal O}(1) 
\eea
We insert this expression in \eqref{eq:masslambda+} and obtain 
\bea
\widetilde
S_{2,i}
&=&
\int_{M^{-2 bi}}^{M^{-2b(i-1)}}
d\alpha 
e^{-\alpha 
\mu_i}
\Big[ c_{1;a,b} \, \alpha^{\frac{-(2a+1)}{2b}} 
 +{\mathcal O}(1)
 \Big]
\crcr
&=&
\int_{M^{-2bi}}^{M^{-2b(i-1)}}
d\alpha   
(1   + {\mathcal O}(  \alpha) ) 
\Big[ c_{1;a,b} \, \alpha^{\frac{-(2a+1)}{2b}} 
 +{\mathcal O}(1) 
 \Big]
 \label{eq:S2icommon}
\crcr
&=&
\int_{M^{-2bi}}^{M^{-2b(i-1)}}
d\alpha 
\Big[ c_{1;a,b} \, \alpha^{\frac{-(2a+1)}{2b}} 
 + 
 {\mathcal O}\big(  \alpha^{1- \frac{(2a+1)}{2b} }\big)
 \Big] \crcr
&=&
\Big[ 
\frac{1}{ 1 - \frac{(2a+1)}{2b}} \, c_{1;a,b}\, \alpha^{1-\frac{(2a+1)}{2b}} 
+ 
{\mathcal O} \big(  \alpha^{2-\frac{(2a+1)}{2b}}\big)
 \Big]_{M^{-2bi}}^{M^{-2b(i-1)}}
 \,,
\eea
where $c_{1; a,b}= \frac{1}{b}
\Gamma\left(\frac{2a+1}{2b}\right)$.

For the model $+$, we use 
$ a = \frac{1}{2} D (d^- - 1)$ and
$ b = \frac{1}{2} D (d^- - \frac{1}{2})$
for just-renormalizability (see Proposition \ref{prop:list+}) with $D = 1$, then,
$\frac{2a+1}{2b} = \frac{2(d-1)}{2d-3}= 1 + \frac{1}{2d-3}$.  
Thus 
\bea
\widetilde
S_{2,i}
&=&
\Big[ 
- \frac{(2d-3)}{(2d-3)/4} \, \Gamma\left(\frac{2(d-1)}{2d-3}\right) \alpha^{-1/(2d-3)} 
 + 
{\mathcal O}\big(  \alpha ^{1- 1/(2d-3)} \big) 
 \Big]_{M^{-2 bi}}^{M^{-2b(i-1)}} 
 \crcr
 &=&
 4 \, \Gamma\Big(\frac{2(d-1)}{2d-3}\Big)
 (M^{-2bi(-1/4b) }
 - M^{-2b(i-1)(-1/4b) } )
 + 
  {\mathcal O}\big(  M^{-i (d -2)}  \big) 
 \crcr
 &=&
 4 \, \Gamma\Big(\frac{2(d-1)}{2d-3}\Big)
 M^{i/2}(
1 - M^{-1/2} )
 + 
 {\mathcal O}\big(  M^{-i (d -2)}  \big) 
\,.
 \label{eq:runningmass1}
\eea





Now we compute $\widetilde S_{1,i}$
 namely,
\bea
\widetilde
S_{1,i}
 = 
\sum_{{\bf q} \in \Z^{d-1}}
\int_{0}^{\infty}
d\alpha \; \chi^i(\alpha)\, 
e^{ - \alpha( |{\bf q}|^{2 b}  + \mu_i )}
 = 
\int_0^\infty 
d\alpha \chi^{i} (\alpha) 
e^{-\alpha \mu_i}
\left(\sum_{q\in \Z}
e^{-\alpha
\vert q\vert^{2b} }
\right)^{d-1}\,.
\label{eq:masslambdaintegral}
\eea
Let us focus on 
a single sum over $q$: 
\bea
&&
\sum_{q\in \Z}
e^{-\alpha 
\vert q\vert^{2b}}
= 2\sum_{q= 1}^\infty
e^{-\alpha q^{2b}}
+
1
= 
2
\int_{1}^{\infty} dq
e^{-\alpha q^{2b} }
+
R
+
1
\crcr
&&
 = 2 
\Bigg(\frac{1}{2b} \alpha^{-\frac{1}{2b}}
\Gamma\left(\frac{1}{2b},\alpha \right)\Bigg)
+R+1\,,
\label{eq:sumtoint}
\eea
where $R = {\mathcal O}(1) + {\mathcal O}(\alpha)$. 
We use \eqref{eq:incompletegammaexpand0}
in \eqref{eq:sumtoint} as $ \alpha$ is a small parameter in the UV, 
\beq
\sum_{q\in \Z}
e^{-\alpha 
\vert q\vert^{2b}}
=
\frac{1}{b} \alpha^{-\frac{1}{2b}}
\Big[
\Gamma\left(\frac{1}{2b}\right)
- 2b \alpha^{\frac{1}{2b}}
+ \cO(\alpha^{\frac{1}{2b}+1})
\Big] +R+1
 = 
\frac{1}{b} 
\Gamma\left(\frac{1}{2b}\right)
\alpha^{-\frac{1}{2b}} +{\mathcal O}(1) 
\,.
\eeq
We insert this expression in  \eqref{eq:masslambdaintegral},
\bea
&&
\widetilde
S_{1,i} = 
\int_0^\infty 
d\alpha \chi^{i} (\alpha) 
e^{-\alpha \mu_i}
\left(\frac{1}{b} 
\Gamma\left(\frac{1}{2b}\right)
\alpha^{-\frac{1}{2b}} +{\mathcal O}(1) \right)^{d-1}
\crcr
&=&
\int_{M^{-2 b i}}^{M^{-2 b (i-1)}}
d\alpha  \,
e^{-\alpha \mu_i}
\Big[
\Bigg(\frac{1}{b} \Gamma\left(\frac{1}{2b} \right) \Bigg)^{d-1} \alpha^{-\frac{d-1}{2b}}
+{\mathcal O}(\alpha^{-\frac{d-2}{2b}}) 
 \Big]
\crcr
&=&
\int_{M^{-2 b i}}^{M^{-2 b (i-1)}}
d\alpha 
(1   + {\mathcal O}(\alpha) 
)
\Big[
\Bigg(\frac{1}{b} \Gamma\left(\frac{1}{2b} \right) \Bigg)^{d-1} \alpha^{-\frac{d-1}{2b}}
+{\mathcal O}(\alpha^{-\frac{d-2}{2b}}) 
 \Big]
\crcr
&=&
\Big[
\Bigg(\frac{1}{b} \Gamma\left(\frac{1}{2b} \right) \Bigg)^{d-1} 
\frac{1}{1-\frac{d-1}{2b}}\alpha^{1-\frac{d-1}{2b}}
+{\mathcal O}(\alpha^{1-\frac{d-2}{2b}}) 
 \Big]_{M^{-2 b i}}^{M^{-2 b (i-1)}}
 \crcr
&=&
\Bigg(\frac{1}{b} \Gamma\left(\frac{1}{2b} \right) \Bigg)^{d-1} 
\frac{2b}{2b-d+1}
M^{-i(2b-d+1)}
\Big( 
M^{(2b-d+1)} - 1
\Big)
+{\mathcal O}(M^{-i(2b-d+2)})
 \,,
 \qquad
 \label{eq:interm}
\eea
For the model $+$, we use 
$ b = \frac{1}{2} D (d^- - \frac{1}{2})$
for just-renormalizability (see Proposition \ref{prop:list+}) with $D = 1$, then:
\beq
\widetilde
S_{1,i}
=
\Bigg( \frac{1}{b} \Gamma\left(\frac{1}{2b} \right) \Bigg)^{d-1} 
(2d-3)
M^{i/2}
\Big(  1 - M^{-1/2} \Big)
+{\mathcal O}(M^{-i/2}) 
\,.
\label{eq:runningmass2}
\eeq


\noindent{\bf  $\widetilde S_{1,i}$ and  $\widetilde S_{2,i}$ for the model $\times$.} 
For the model $\times$, from the Proposition \ref{prop:listx}, we have a specific set of parameters, i.e., $D=1$, $d=3$, $a=\frac{1}{2}$, $b=1$, and $\frac{2a+1}{2b}=1$. 
We then specialize the previous computation.  Starting from \eqref{eq:S2icommon},
we have 
\bea
\widetilde S_{2,i} &=&
\int_{M^{-2i}}^{M^{-2(i-1)}}
d\alpha  
(1   + {\mathcal O}(  \alpha) ) 
\Big[ \alpha^{-1}   +{\mathcal O}(1)   \Big]
 \crcr
 &=&
 \Big[\log \alpha + { \cO (\alpha) } 
 \Big]_{M^{-2i}}^{M^{-2(i-1)}}
 \crcr
 &=&
2 \log M  +   \cO(M^{-2(i-1)})
\,.
\label{eq:S2iexacttimes}
\eea

For the model $\times$, from the Proposition \ref{prop:listx}, we have a specific set of parameters, i.e., $D=1$, $d=3$, $a=\frac{1}{2}$, $b=1$, and $\frac{d-1}{2b}=1$.
So recomputing \eqref{eq:interm}, we get: 
\bea
\widetilde
S_{1,i}
&=&
\int_{M^{-2  i}}^{M^{-2  (i-1)}}
d\alpha 
(1   + {\mathcal O}( \alpha) 
)
\Big[
\Big( \Gamma\left(\frac{1}{2} \right) \Big)^{2}   \alpha^{-1}
+{\mathcal O}(\alpha^{-\frac{1}{2}}) 
 \Big]
\crcr
&=&
\Big[ \pi  \log \alpha + \cO (\alpha^{\frac{1}{2}})
\Big]^{M^{-2  (i-1)}}_{M^{-2  i}}
\crcr
&=&
2\pi \log M 
+ \cO \big( {M^{-  i}}\big)
\,,
\label{eq:runningmass2x}
\eea
where we substitute $ \Gamma\left(\frac{1}{2} \right) = \sqrt {\pi}$.

\ 

\noindent{\bf
Dimensionful computations of $S_{1,i}$ \eqref{s1idebut} and $S_{2,i}$ \eqref{s2idebut}.} 
Following the similar argument as for $S_{0,i}$,
we perform the changes of variables given in \eqref{eq:dimfuldimless} so that the dimensions are explicit in computations.
Then, we obtain in terms of dimensionless $\widetilde S_{1,i}$ and $\widetilde S_{2,i}$: 
\bea
&&
S_{1,i}   
=
k^{- 2b} 
k^{d-1}
\widetilde S_{1,i}  
\crcr
&&
S_{2,i} 
=
k^{-2 b}
k^{2 a}
k
\widetilde S_{2,i} 
\,,
\label{eq:S1S2dimfuldimless}
\eea
where in the last equalities, we note 
from Propositions \ref{prop:list+} and \ref{prop:listx} that
\bea
    d-1- 2 b = 2 a - 2b +1
    = 
\begin{cases}
    \frac{1}{2} 
    ,& \text{for the model } +\\
    0 
    ,              & \text{for the model } \times
\end{cases}
\eea




\section{Second order in perturbation in Model $+$.}
\label{app:2ndO+}



\subsection{4-point function in Model $+$.}
\label{app:4pt2+}

\begin{itemize}
    \item The graph $n_{42}^{(c)}$ has degree of divergence $\omega_{n_{42}^{(c)}} = 0$, following the first row of Table \ref{tab:listprim1}, and its amplitude appears in the second order in perturbation theory.

\begin{figure}[H]
\centering
     \begin{minipage}[t]{0.7\textwidth}
      \centering
\def\svgwidth{1\columnwidth}
\tiny{
\input{V4twoloop.pdf_tex}
}
\caption{ {\small  
The two-loop divergent graph, $n_{42}^{(c)}$ that contributes to the renormalization of $4$-pt function in $\Gamma_4^{(c)}(\{p\})$.
Here the rank $d=3$.  
}} 
\label{fig:V4twoloop}
\end{minipage}
\end{figure}




Explicitly, 
at two-loop,
\bea
&&
 K_{n_{42}^{(c)} } = 12 \,, 
 \crcr
 &&
  S_{n_{42}^{(c)}} (\{\mathbf{p}, \mathbf{p}'\}) = \frac{1}{3!}
  \Big(\frac{-\lapc}{2}\Big)^3 
  \sum_{\{q\}} 
  \frac
  {
  \big(\vert q_{c}\vert^{2a}\big)^{3}
  }
  {
  (|\mathbf{p}_{\check c}|^{2b} + \vert q_{c}\vert^{2b} + \mu)
    ( |\mathbf{p}'_{\check c}|^{2b} + \vert q_{c}\vert^{2b} + \mu)^2
    ( \vert \mathbf{q}\vert^{2b} + \mu)^2
    }
    \,.
    \nonumber
\eea
The contribution to the Feynman amplitude is given by
\beq
\sum_c K_{n_{42}^{(c)}}\, S_{n_{42}^{(c)}}\,.
\eeq





    
\end{itemize}








\subsection{$\Gamma_2$ in Model $+$.}
\label{app:Gamma2+}

\begin{itemize}






\item
For $m_{2 e}^{(c \, c')}$,
$\omega_{m_{2 e}^{(c \, c')}} = 0$.
This graph belongs to the class V in Table \ref{tab:listprim1}, and appears in the second order in perturbation theory.
%
\begin{figure}[H]
\centering
     \begin{minipage}[t]{0.7\textwidth}
      \centering
\def\svgwidth{1\columnwidth}
\tiny{
\input{m2e.pdf_tex}
}
\caption{{\small  The graph $m_{2 e}^{(c=1 \, c'=2)}$. For $d=3$ is illustrated. $\{q_{\check c}\} = \{q_{c'}, \; q_{3}\}$, and ${|{\bf q}_{\check c}|}^{2 b} = {|q_{c'}|}^{2b} + {|q_{3}|}^{2 b} $.}}
\label{fig:m2e}
\end{minipage}
\end{figure}
%
The contribution to the Feynman amplitude is then,
\bea
\sum_{c'}
K_{m_{2 e}^{(c \, c')}} S_{m_{2 e}^{(c \, c')}} (\{p \})
& =& 
\sum_{c' \ne c}^{(d-1) \; {\rm terms}}
2 
\Big[
\Big(- \frac{{\lambda_{+}}^{(c)}}{2}\Big) (- Z_a^{(c')}) 
\Big]
\sum_{\{q_{\hat c}\}}
\frac{{\vert q_{c'} \vert}^{2 a} {\vert p_{c}\vert}^{2 a}} {( {|{\bf q}_{{(\check c)}}|}^{2 b}   + \vert{ p_{c}\vert}^{2 b} + \mu)^2}
\nonumber 
\\
&=&
\lambda_{+}^{(c)}  
{\vert p_{c}\vert}^{2 a}
\sum_{c' \ne c}^{(d-1) \; {\rm terms}}
 Z_a^{(c')}
\sum_{\{q_{\check c}\}}
\frac{{\vert q_{c'}\vert}^{2 a}} {( {|{\bf q}_{{(\check c)}}|}^{2 b}   + {\vert p_{c}\vert}^{2 b} + \mu)^2}
\crcr
&=&
\lambda_{+}^{(c)} 
{\vert p_{c}\vert}^{2 a}
\sum_{\{q_{\check c}\}}
\frac{ \sum_{c' \ne c} Z_a^{(c')} {\vert q_{c'}\vert}^{2 a}} {( {|{\bf q}_{{(\check c)}}|}^{2 b}   + {\vert p_{c}\vert}^{2 b} + \mu)^2}
\,,
\eea
where $K_{m_{2 e}^{(c \, c')}} = 2$.

















\item 
For $meme_2^{(c \, c')}$,
$\omega_{meme_2^{(c \, c')}} = \frac{D}{2}$.
This graph belongs to the class II in Table \ref{tab:listprim1}, and appears in the second order in perturbation theory.
%
\begin{figure}[H]
\centering
     \begin{minipage}[t]{0.7\textwidth}
      \centering
\def\svgwidth{1\columnwidth}
\tiny{
\input{meme2.pdf_tex}
}
\caption{  {\small  For $d=3$, $meme_2^{(c=1 \, c'=2)}$ is illustrated. 
$\{q'_{({\check c}{\check {c'}})}\}={q'}_{3}$,
$\{q\}=\{ q_{1}, q_{2}, q_{3}\}$,
${|{\bf q}|}^{2 b}={\vert q_{1}\vert}^{2 b} + {\vert q_{2}\vert}^{2 b} + {\vert q_{3}\vert}^{2 b} $,
${|{\bf q}_{({\check c}{\check {c'}})}|}^{2 b}= {\vert q_{3}\vert}^{2 b}$.}}
\label{fig:meme2}
\end{minipage}
\end{figure}
%
\bea
&&
\sum_{ c' \ne c} 
K_{meme_2^{(c, c')}} S_{meme_2^{(c, c')}} (\{ p\}) 
\crcr
&=&
\sum_{c' \ne c}^{(d-1) \; {\rm terms}}
K_{meme_2^{(c, c')}}
\Big[ 
\Big( - \frac{\lambda^{(c)}}{2}\Big) 
\Big( - \frac{\lambda_{+}^{(c')}}{2}\Big)  
\Big]
{\vert p_{c}\vert}^{2 a} 
\nonumber
\\
 && \qquad \quad
\sum_{\{q'_{({\check c}{\check {c'}})}\}, \{q\}}
\frac{{\vert q_{c'}\vert}^{2 a}}{({|{\bf q}|}^{2 b} +  \mu) ({|{\bf q'}_{({\check c}{\check {c'}})}|}^{2 b}  +{\vert q_{c'} \vert}^{2 b} + {\vert p_{c}\vert}^{2 b} + \mu)^2}
\crcr
&=&
\lambda^{(c)} 
{\vert p_{c}\vert}^{2 a} 
\sum_{\{q'_{({\check c})}\}, \{q\}}
\frac{\sum_{c' \ne c} \lambda_{+}^{(c')}  {\vert q_{c'}\vert}^{2 a}}{({|{\bf q}|}^{2 b} +  \mu) ({|{\bf q'}_{({\check c}{\check {c'}})}|}^{2 b} +{\vert q_{c'} \vert}^{2 b} + {\vert p_{c}\vert}^{2 b} + \mu)^2}
\,,
\quad
\eea
where $K_{meme_2^{(c, c')}}=4$.


\end{itemize}























\subsection{Self energy and mass renormalization in Model $+$.}
\label{app:Sigma+}
\begin{itemize}
\item
For the graph $m_2^{(c, c')}$, the degree of divergence is given by
$\omega_{d;+}(m_2^{(c, c')})= 0$.
This graph belongs to the class VI in Table \ref{tab:listprim1}, and
also, appears in the second order in Taylor expansion of the Feynman amplitude.
%
\begin{figure}[H]
\centering
     \begin{minipage}[t]{0.6\textwidth}
      \centering
\def\svgwidth{0.7\columnwidth}
\tiny{
\input{m2.pdf_tex}
}
\caption{{\small $m_2^{(c, c')}$ in colored and stranded representations.  Here, we consider $d=3$. ${|{\bf q}_{\check {c}}|}^{2 b} = {|q_{c'}|}^{2 b} + {| q_{1}|}^{2b}$ and $\{q_{\check c}\} =\{ q_{c'}, \; q_{1}\}$.}}
\label{fig:m2}
\end{minipage}
\end{figure}
%
The contribution of the graph $m_2^{(c, c')}$ shown in Fig. {\ref{fig:m2}} in the amplitude is
\bea
\sum_{c, c' \ne c} K_{m_2^{(c, c')}}S_{m_2^{(c, c')}} (\{ p\}) 
&=&
\sum_{c, c' \ne c}^{d \times (d-1) \; {\rm terms}}
2
\Big[ 
\Big( - \frac{\lambda^{(c)}}{2}\Big) 
\Big( - {Z_a}^{(c')} \Big) 
\Big]
\sum_{\{q_{\check c}\}}
\frac{\vert q_{c'}\vert^{2 a}}{({|{\bf q}_{\check c}|}^{2 b} + \vert p_{c}\vert^{2 b} + \mu)^2} \nonumber \\
&=&
\sum_{c=1}^{d}
\lambda^{(c)} 
\sum_{\{q_{\check c}\}}
\frac{\sum_{c' \ne c} {Z_a}^{(c')} \vert q_{c'}\vert^{2 a}}{({|{\bf q}_{\check c}|}^{2 b} + \vert p_{c}\vert^{2 b}+ \mu)^2}
\,,
\eea
where $K_{m_2^{(c, c' \ne c)}} = 2$.





\item
For the graph $mme_2^{(c, c')}$, the superfical degree of divergence is
$\omega_{d;+}(mme_2^{(c, c')}) = \frac{D}{2}$.
This graph belongs to the class III in Table \ref{tab:listprim1}, and appears in the 
second order in Taylor expansion.
%
\begin{figure}[H]
\centering
     \begin{minipage}[t]{0.7\textwidth}
      \centering
\def\svgwidth{0.7\columnwidth}
\tiny{
\input{mme2.pdf_tex}
}
\caption{{\small  $mme_2^{(c=1, c'=2)}$. For $d=3$ is shown above in both colored and stranded representations. $\{p\} = \{p_{1}, p_{2}, p_{3}\}$ and 
$\{q'_{({\check c}{\check {c'}})}\}={q'}_{3}$,
$\{q\}=\{ q_{1}, q_{2}, q_{3}\}$,
${|{\bf q}|}^{2 b}={\vert q_{1}\vert}^{2 b} + {\vert q_{2}\vert}^{2 b} + {\vert q_{3}\vert}^{2 b} $,
${|{\bf q}_{({\check c}{\check {c'}})}|}^{2 b}= {\vert q_{3}\vert}^{2 b}$.}}
\label{fig:mme2}
\end{minipage}
\end{figure}
%
The amplitude associated with the graph $mme_2^{(c, c')}$ shown in Fig. {\ref{fig:mme2}} is given by
\bea
&&\sum_{c, c' \ne c} K_{mme_2^{(c, c')}} S_{mme_2^{(c, c')}} (\{ p\}) 
\crcr
&=&
\sum_{c, c' \ne c}^{d \times (d-1) \; {\rm terms}}
4
\Big[ 
\Big( - \frac{\lambda^{(c)}}{2}\Big) 
\Big( - \frac{\lambda_{+}^{(c')}}{2}\Big)  
\Big]
\crcr
&&
\qquad \qquad
\sum_{\{q'_{({\check c}{\check {c'}})}\}, \{q\}}
\frac{{\vert q_{c'}\vert}^{2 a}}{({|{\bf q}|}^{2 b} +  \mu) ({|{\bf q'}_{({\check c}{\check {c'}})}|}^{2 b} +{\vert q_{c'} \vert}^{2 b}+ {\vert p_{c}\vert}^{2 b} + \mu)^2}
\crcr
&=&
\sum_{c=1}^{d}
\lambda^{(c)} 
\sum_{\{q'_{({\check c}{\check {c'}})}\}, \{q\}}
\frac{\sum_{c' \ne c} \lambda_{+}^{(c')}  {\vert q_{c'}\vert}^{2 a}}{({|{\bf q}|}^{2 b} +  \mu) ({|{\bf q'}_{({\check c}{\check {c'}})}|}^{2 b} +{\vert q_{c'} \vert}^{2 b} + {\vert p_{c}\vert}^{2 b} + \mu)^2}
\,,
\quad
\eea
where $K_{mme_2^{(c, c')}} = 4$.








\item
For $n_2^{(c)}$, 
$\omega_{d;+}(n_2^{(c)}) = 0$.
This graph belongs to the class IV in Table \ref{tab:listprim1} and its amplitude appears in 
the second order in Taylor expansion.
%
\begin{figure}[H]
\centering
     \begin{minipage}[t]{0.7\textwidth}
      \centering
\def\svgwidth{0.7\columnwidth}
\tiny{
\input{n2.pdf_tex}
}
\caption{ {\small  The illustration of $n_2^{(c)}$ in $d=3$. ${|{\bf p}_{\check c}|}^{2 b}  = {|p_{1}|}^{2 b} +{|p_{2}|}^{2 b} +{|p_{3}|}^{2 b} $, and $\{p\} = \{p_{1}, \; p_{2}, \; p_{3}\}$
}} 
\label{fig:n2}
\end{minipage}
\end{figure}
%
Now the amplitude of $n_2^{(c)}$ is
\bea
\sum_c K_{n_2^{(c)}} S_{n_2^{(c)}}( \{p\})
&=&
\sum_{c=1}^{d}
\Big [ \Big( - \frac{{\lambda_{+}}^{(c)}}{2}\Big) ( -Z_{a}^{(c)}) \Big ]
\sum_{q_{c}} \frac{(\vert q_{c}\vert ^{2a})^2}{(\vert q_{c}\vert^{2 b} + {|{\bf p}_{\check c}|}^{2 b} + \mu)^2}
\,,
\qquad \quad
\eea
where  
$K_{n_2^{(c)}} = 1$.











\item
For $nme_2^{(c)}$, 
$\omega_{d;+}(nme_2^{(c)}) = \frac{D}{2}$.
This graph belongs to the class I in Table \ref{tab:listprim1} and the
second order in Taylor expansion for the Feynman amplitude.
%
\begin{figure}[H]
\centering
     \begin{minipage}[t]{0.7\textwidth}
      \centering
\def\svgwidth{0.7\columnwidth}
\tiny{
\input{nme2.pdf_tex}
}
\caption{ {\small  For $d=3$, $nme_2^{(c=1)}$ is shown above. $\{p\} = \{p_{1}, p_{2}, p_{3}\}$ 
and ${|{\bf p}_{\check c}|}^{2 b}={|p_{2}|}^{2 b}+{|p_{3}|}^{2 b}$,
${\vert {\bf q} \vert}^{2 b} = {\vert q_1 \vert}^{2 b} + {\vert q_2 \vert}^{2 b} + {\vert q_3 \vert}^{2 b}$.
}} 
\label{fig:nme2}
\end{minipage}
\end{figure}
%
\bea
&&
\sum_c K_{nme_2^{(c)}} S_{nme_2^{(c)}}( \{p\})
\crcr
&=&
\sum_{c=1}^{d}
K_{nme_2^{(c)}} 
\frac{1}{2!}
\Big [ \Big( - \frac{{\lambda_{+}}^{(c)}}{2}\Big)^2 \Big ]
\sum_{\{q\}} \frac{({\vert q_{c}\vert}^{2a})^2}{({\vert {\bf q}\vert}^{2 b} + \mu)({\vert q_{c}\vert}^{2 b} + {|{\bf p}_{\check c}|}^{2 b} + \mu)^2}
\,,
\qquad
\eea
where we set $K_{nme_2^{(c)}} = 4$.






\end{itemize}




\section{Second order in perturbation in Model $\times$.}
\label{app:2ndOx}

\subsection{Self energy and mass renormalization in Model $\times$}
\label{app:Sigmax}

\begin{itemize}




\item
For $mm_{ee}^{(c, c')}$, the superfical degree of divergence is
$\omega_{d;\times}(mm_{ee}^{(c, c')}) = 0$.
This graph belongs to the class III in Table \ref{tab:listprim2}, and appears in the 
second order $\cO (\lambda \, \lambda_\times)$ in Taylor expansion.





\begin{figure}[H]
\centering
     \begin{minipage}[t]{0.7\textwidth}
      \centering
\def\svgwidth{1\columnwidth}
\tiny{
\input{mmee.pdf_tex}
}
\caption{{\small  $mm_{ee}^{(c=1, c'=2)}$. For $d=3$ is shown above in both colored and stranded representations. $\{p\} = \{p_{1}, p_{2}, p_{3}\}$ and 
$\{q'_{({\check c}{\check {c'}})}\}={q'}_{3}$,
$\{q\}=\{ q_{1}, q_{2}, q_{3}\}$,
${|{\bf q}|}^{2 b}={\vert q_{1}\vert}^{2 b} + {\vert q_{2}\vert}^{2 b} + {\vert q_{3}\vert}^{2 b} $,
${|{\bf q}_{({\check c}{\check {c'}})}|}^{2 b}= {\vert q_{3}\vert}^{2 b}$.
}} 
\label{fig:mmee}
\end{minipage}
\end{figure}
%
The amplitude associated with the graph $mm_{ee}^{(c, c')}$ shown in Fig. {\ref{fig:mmee}} is given by
\bea
&&\sum_{c, c' \ne c} K_{mm_{ee}^{(c, c')}} S_{mm_{ee}^{(c, c')}} (\{ p\}) 
\crcr
&=&
\sum_{c, c' \ne c}^{d \times (d-1) \; {\rm terms}}
4
\Big[ 
\Big( - \frac{\lambda^{(c)}}{2}\Big) 
\Big( - \frac{\lambda_{\times}^{(c')}}{2}\Big)  
\Big]
\crcr
&&
\qquad \qquad
\sum_{\{q'_{({\check c}{\check {c'}})}\}, \{q\}}
\frac{{\vert q_{c'}\vert}^{4 a}}{({|{\bf q}|}^{2 b} +  \mu) ({|{\bf q'}_{({\check c}{\check {c'}})}|}^{2 b} +{\vert q_{c'} \vert}^{2 b}+ {\vert p_{c}\vert}^{2 b} + \mu)^2}
\crcr
&=&
\sum_{c=1}^{d}
\lambda^{(c)} 
\sum_{\{q'_{({\check c}{\check {c'}})}\}, \{q\}}
\frac{\sum_{c' \ne c} \lambda_{\times}^{(c')}  {\vert q_{c'}\vert}^{4 a}}{({|{\bf q}|}^{2 b} +  \mu) ({|{\bf q'}_{({\check c}{\check {c'}})}|}^{2 b} +{\vert q_{c'} \vert}^{2 b} + {\vert p_{c}\vert}^{2 b} + \mu)^2}
\,,
\eea
where  $K_{mm_{ee}^{(c, c')}} = 4$.




\end{itemize}





\subsection{$\Gamma_{2;a}$ in Model $\times$. }
\label{app:Gamma2ax}

\begin{itemize}
\item
For $n_{ee}m_{ee}^{(c)}$, 
$\omega_{d; \times}(n_{ee}m_{ee}^{(c)}) = 0$.
This graph belongs to the class I in Table \ref{tab:listprim2} and the
second order in Taylor expansion for the Feynman amplitude.
%
\begin{figure}[H]
\centering
     \begin{minipage}[t]{0.7\textwidth}
      \centering
\def\svgwidth{0.7\columnwidth}
\tiny{
\input{neemee.pdf_tex}
}
\caption{ {\small  For $d=3$, $n_{ee}m_{ee}^{(c=1)}$ is shown above. $\{p\} = \{p_{1}, p_{2}, p_{3}\}$ 
and ${|{\bf p}_{\check c}|}^{2 b}={|p_{2}|}^{2 b}+{|p_{3}|}^{2 b}$,
${\vert {\bf q} \vert}^{2 b} = {\vert q_1 \vert}^{2 b} + {\vert q_2 \vert}^{2 b} + {\vert q_3 \vert}^{2 b}$.
}} 
\label{fig:neemee}
\end{minipage}
\end{figure}
%
\bea
&&
K_{n_{ee}m_{ee}^{(c)}} S_{n_{ee}m_{ee}^{(c)}}( \{p\})
\nonumber
\\
&=&
8 
\,
\frac{1}{2!}
 \Big( - \frac{{\lambda_{\times}}^{(c)}}{2}\Big)^2 
\vert p_{c}\vert^{2a}
\sum_{\{q\}} \frac{({\vert q_{c}\vert}^{2a})^3}{({\vert {\bf q}\vert}^{2 b} + \mu)({\vert q_{c}\vert}^{2 b} + {|{\bf p}_{(\check c)}|}^{2 b} + \mu)^2}
\,,
\eea
where we set $K_{n_{ee}m_{ee}^{(c)}} = 8$.




\end{itemize}


\subsection{$\Gamma_{2; 2a}$ in Model $\times$.}
\label{app:Gamma22ax}


\begin{itemize}

\item 
For $m_{ee}m_{ee}^{(c, c')}$,
$\omega_{d; \times}(m_{ee}m_{ee}^{(c, c')}) = 0$.
This graph belongs to the class II in Table \ref{tab:listprim2}, and appears in the second order in perturbation theory.
%
\begin{figure}[H]
\centering
     \begin{minipage}[t]{0.7\textwidth}
      \centering
\def\svgwidth{1\columnwidth}
\tiny{
\input{meemee.pdf_tex}
}
\caption{ {\small  For $d=3$, $m_{ee}m_{ee}^{(c=1, c'=2)}$ is illustrated. 
$\{q'_{({\check c}{\check {c'}})}\}={q'}_{3}$,
$\{q\}=\{ q_{1}, q_{2}, q_{3}\}$,
${|{\bf q}|}^{2 b}={\vert q_{1}\vert}^{2 b} + {\vert q_{2}\vert}^{2 b} + {\vert q_{3}\vert}^{2 b} $,
${|{\bf q}_{({\check c}{\check {c'}})}|}^{2 b}= {\vert q_{3}\vert}^{2 b}$.}}
\label{fig:meemee}
\end{minipage}
\end{figure}
%
\bea
&&
\sum_{ c' \ne c} 
K_{m_{ee}m_{ee}^{(c, c')}} S_{m_{ee}m_{ee}^{(c, c')}} (\{ p\}) 
\crcr
&=&
\sum_{c' \ne c}^{(d-1) \; {\rm terms}}
K_{meme_2}^{(c, c')}
\Big[ 
\Big( - \frac{\lambda_{\times}^{(c)}}{2}\Big) 
\Big( - \frac{\lambda_{\times}^{(c')}}{2}\Big)  
\Big]
{\vert p_{c}\vert}^{4 a} 
\nonumber
\\
 && \qquad \quad
\sum_{\{q'_{({\check c}{\check {c'}})}\}, \{q\}}
\frac{{\vert q_{c'}\vert}^{4 a}}{({|{\bf q}|}^{2 b} +  \mu) ({|{\bf q'}_{({\check c}{\check {c'}})}|}^{2 b}  +{\vert q_{c'} \vert}^{2 b} + {\vert p_{c}\vert}^{2 b} + \mu)^2}
\crcr
&=&
\lambda_{\times}^{(c)} 
{\vert p_{c}\vert}^{4 a} 
\sum_{\{q'_{({\check c})}\}, \{q\}}
\frac{\sum_{c' \ne c} \lambda_{\times}^{(c')}  {\vert q_{c'}\vert}^{4 a}}{({|{\bf q}|}^{2 b} +  \mu) ({|{\bf q'}_{({\check c}{\check {c'}})}|}^{2 b} +{\vert q_{c'} \vert}^{2 b} + {\vert p_{c}\vert}^{2 b} + \mu)^2}
\,,
\quad
\eea
where  $K_{m_{ee}m{ee}^{(c, c')}}=4$.




\end{itemize}


\begin{thebibliography}{100}

%\bibitem[Abbott et al.(2017)]{Abbott2017} Abbott, B.~P., Abbott, R., Abbott, T.~D., et al., \prl, 119, 161101 (2017). doi:10.1103/PhysRevLett.119.161101

\bibitem[{{Abbott} {et~al.}(2017{\natexlab{c}})Abbott, Abbott, \&
  Abbott}]{Abbott_et_al__2017__PhysicalReviewLetters__GW170817ObservationofGravitationalWavesfromaBinaryNeutronStarInspiral} {Abbott}, B.~P., {Abbott}, R., {Abbott}, T.~D., {et~al.} 2017, Physical Review Letters, 119, 161101. doi:10.1103/PhysRevLett.119.161101

\bibitem[{Abbott {et~al.}(2017{\natexlab{a}})Abbott, Abbott, \&
  Abbott}]{Abbott__2017__TheAstrophysicalJournal__EstimatingtheContributionofDynamicalEjectaintheKilonovaAssociatedwithGW170817} Abbott, B.~P., Abbott, R., \& Abbott, T. D. e.~a. 2017{\natexlab{a}}, The Astrophysical Journal, 850, L39

\bibitem[{Abbott {et~al.}(2017{\natexlab{b}})Abbott, Abbott, \&
  Abbott}]{Abbott__2017__TheAstrophysicalJournal__GravitationalWavesandGammaRaysfromaBinaryNeutronStarMergerGW170817andGRB170817A} Abbott, B.~P., Abbott, R., \& Abbott, T. D. e.~a. 2017{\natexlab{b}}, The Astrophysical Journal, 848, L13

\bibitem[{{Abbott} {et~al.}(2017{\natexlab{d}})Abbott, Abbott, \&
  Abbott}]{Abbott__2017__apjl__MultiMessengerObservationsofaBinaryNeutronStarMerger} {Abbott}, B.~P., {Abbott}, R., \& {Abbott}, T.~D. e.~a. 2017, \apjl, 848, L12

\bibitem[{Abbott {et~al.}(2017{\natexlab{e}})Abbott, Abbott, \&   Abbott}]{Abbott__2017__TheAstrophysicalJournal__OntheProgenitorofBinaryNeutronStarMergerGW170817} Abbott, B.~P., Abbott, R., \& Abbott, T. D. e.~a. 2017, The Astrophysical Journal, 850, L40

\bibitem[{Aloy {et~al.}(2018)Aloy, Cuesta-Mart{\'i}nez, \&
  Obergaulinger}]{Aloy__2018__MonthlyNoticesoftheRoyalAstronomicalSociety__OntheExistenceofaLuminosityThresholdofGRBJetsinMassiveStars}
Aloy, M.~A., Cuesta-Martínez, C., \& Obergaulinger, M. 2018, Monthly Notices of
  the Royal Astronomical Society, 478, 3576

\bibitem[{Aloy \& Obergaulinger(2021)}]{Aloy__2021__MonthlyNoticesoftheRoyalAstronomicalSociety__MagnetorotationalCoreCollapseofPossibleGRBProgenitorsII.FormationofProtomagnetarsandCollapsars} Aloy, M.~Á. \& Obergaulinger, M. 2021, Monthly Notices of the Royal Astronomical Society, 500, 4365. doi:10.1093/mnras/staa3273

%\bibitem[Aloy \& Obergaulinger(2021)]{Aloy2021} Aloy, M. {\'A}. \& Obergaulinger, M., \mnras, 500, 4365 (2021). doi:10.1093/mnras/staa3273

\bibitem[{Alp {et~al.}(2019)Alp, Larsson, Maeda, Fransson, Wongwathanarat,
  Gabler, Janka, Jerkstrand, Heger, \&
  Menon}]{Alp__2019__TheAstrophysicalJournal__XRayandGammaRayEmissionfromCoreCollapseSupernovaeComparisonofThreeDimensionalNeutrinoDrivenExplosionswithSN1987A}
Alp, D., Larsson, J., Maeda, K., {et~al.} 2019, The Astrophysical Journal, 882,
  22, aDS Bibcode: 2019ApJ...882...22A

\bibitem[Arcones \& Thielemann(2013)]{Arcones2013} Arcones, A. \& Thielemann, F.-K., Journal of Physics G Nuclear Physics, 40, 013201 (2013). doi:10.1088/0954-3899/40/1/013201

\bibitem[Arnett(1966)]{Arnett1966} Arnett, W.~D., Canadian Journal of Physics, 44, 2553 (1966). doi:10.1139/p66-210

\bibitem[Arnett(1967)]{Arnett1967} Arnett, D., Canadian Journal of Physics, 45, 1621 (1967). doi:10.1139/p67-126

\bibitem[Arnould(1976)]{Arnould1976} Arnould, M., \aap, 46, 117 (1976)

\bibitem[Argast et al.(2004)]{Argast2004} Argast, D., Samland, M., Thielemann, F.-K., et al., \aap, 416, 997 (2004). doi:10.1051/0004-6361:20034265

\bibitem[Arnould \& Goriely(2003)]{Arnould2003} Arnould, M. \& Goriely, S., \physrep, 384, 1 (2003). doi:10.1016/S0370-1573(03)00242-4

\bibitem[Arnould \& Goriely(2020)]{Arnould2020} Arnould, M. \& Goriely, S., Progress in Particle and Nuclear Physics, 112, 103766 (2020). doi:10.1016/j.ppnp.2020.103766

\bibitem[Arnould et al.(2007)]{Arnould2007} Arnould, M., Goriely, S., \& Takahashi, K., \physrep, 450, 97 (2007). doi:10.1016/j.physrep.2007.06.002

%%\bibitem[Arnould(1976)]{Arnould1976} Arnould, M., \aap, 46, 117 (1976)

\bibitem[Audouze \& Truran(1975)]{Audouze1975} Audouze, J. \& Truran, J.~W., \apj, 202, 204 (1975). doi:10.1086/153965

\bibitem[{{Balbus} \& {Hawley}(1998)}]{Balbus_Hawley__1998__RMP__MRI}
{Balbus}, S.~A. \& {Hawley}, J.~F. 1998, Reviews of Modern Physics, 70, 1

\bibitem[Barbuy et al.(2015)]{Barbuy2015} Barbuy, B., Fria{\c{c}}a, A.~C.~S., da Silveira, C.~R., et al., \aap, 580, A40 (2015). doi:10.1051/0004-6361/201525694

\bibitem[{Begelman(1995)}]{Begelman__1995__ProceedingsoftheNationalAcademyofScience__TheAccelerationandCollimationofJets}
Begelman, M.~C. 1995, Proceedings of the National Academy of Science, 92,
  11442, aDS Bibcode: 1995PNAS...9211442B

\bibitem[Beniamini \& Piran(2019)]{Beniamini2019} Beniamini, P. \& Piran, T.\ 2019, \mnras, 487, 4847. doi:10.1093/mnras/stz1589

\bibitem[Beniamini et al.(2018)]{Beniamini2018} Beniamini, P., Dvorkin, I., \& Silk, J.\ 2018, \mnras, 478, 1994. doi:10.1093/mnras/sty1035

\bibitem[Bergemann et al.(2017)]{Bergemann2017} Bergemann, M., Collet, R., Amarsi, A.~M., et al., \apj, 847, 15 (2017). doi:10.3847/1538-4357/aa88cb

\bibitem[Bergemann et al.(2019)]{Bergemann2019} Bergemann, M., Gallagher, A.~J., Eitner, P., et al., \aap, 631, A80 (2019). doi:10.1051/0004-6361/201935811

\bibitem[Bj{\o} [Bj{\o}rgen \& Leenaarts(2017)]{Bjorgen2017} Bj{\o}rgen, J.~P. \& Leenaarts, J., \aap, 599, A118 (2017). doi:10.1051/0004-6361/201630237

\bibitem[Blake \& Schramm(1976)]{Blake1976} Blake, J.~B. \& Schramm, D.~N., \apj, 209, 846 (1976). doi:10.1086/154782

\bibitem[Blake et al.(1981)]{Blake1981} Blake, J.~B., Woosley, S.~E., Weaver, T.~A., et al., \apj, 248, 315 (1981). doi:10.1086/159156

\bibitem[{{Blandford} \&
  {Znajek}(1977)}]{Blandford_Znajek__1977__mnras__Electromagnetic_extraction_of_energy_from_Kerr_black_holes}
{Blandford}, R.~D. \& {Znajek}, R.~L. 1977, \mnras, 179, 433

\bibitem[Bliss et al.(2018)]{Bliss2018} Bliss, J., Arcones, A., \& Qian, Y.-Z., \apj, 866, 105 (2018). doi:10.3847/1538-4357/aade8d

\bibitem[{Bollig {et~al.}(2021)Bollig, Yadav, Kresse, Janka, Müller, \&
  Heger}]{Bollig__2021__TheAstrophysicalJournal__SelfConsistent3DSupernovaModelsfrom7Minutesto7Sa1BetheExplosionofa19MsolProgenitor}
Bollig, R., Yadav, N., Kresse, D., {et~al.} 2021, The Astrophysical Journal,
  915, 28, aDS Bibcode: 2021ApJ...915...28B

\bibitem[Bovard \& Rezzolla(2017)]{Bovard2017} Bovard, L. \& Rezzolla, L., Classical and Quantum Gravity, 34, 215005 (2017). doi:10.1088/1361-6382/aa8d98

\bibitem[{{Bromberg} {et~al.}(2015){Bromberg}, {Granot}, \&
  {Piran}}]{Bromberg_et_al__2015__mnras__OnthecompositionofGRBsCollapsarjets}
{Bromberg}, O., {Granot}, J., \& {Piran}, T. 2015, \mnras, 450, 1077

\bibitem[{{Bromberg} {et~al.}(2011){Bromberg}, {Nakar}, {Piran}, \&
  {Sari}}]{Bromberg_et_al__2011__apj__ThePropagationofRelativisticJetsinExternalMedia}
{Bromberg}, O., {Nakar}, E., {Piran}, T., \& {Sari}, R. 2011, \apj, 740, 100

\bibitem[Brott et al.(2011)]{Brott2011} Brott, I., de Mink, S.~E., Cantiello, M., et al., \aap, 530, A115 (2011). doi:10.1051/0004-6361/201016113

\bibitem[Brown et al.(2005)]{Brown2005} Brown, E.~F., Calder, A.~C., Plewa, T., et al., \nphysa, 758, 451 (2005). doi:10.1016/j.nuclphysa.2005.05.083

\bibitem[Bruenn(1985)]{Bruenn1985} Bruenn, S.~W., \apjs, 58, 771 (1985). doi:10.1086/191056

\bibitem[Bruenn et al.(2006)]{Bruenn2006} Bruenn, S.~W., Dirk, C.~J., Mezzacappa, A., et al., Journal of Physics Conference Series, 46, 393 (2006). doi:10.1088/1742-6596/46/1/054


\bibitem[Bufano et al.(2012)]{Bufano2012} Bufano, F., Pian, E., Sollerman, J., et al., \apj, 753, 67 (2012). doi:10.1088/0004-637X/753/1/67

\bibitem[{{Bucciantini} {et~al.}(2007){Bucciantini}, {Quataert}, {Arons},
  {Metzger}, \&
  {Thompson}}]{Bucciantini_et_al__2007__mnras__Magnetar-driven_bubbles_and_the_origin_of_collimated_outflows_in_GRBs}
{Bucciantini}, N., {Quataert}, E., {Arons}, J., {Metzger}, B.~D., \&
  {Thompson}, T.~A. 2007, \mnras, 380, 1541

\bibitem[{Bugli {et~al.}(2020)Bugli, Guilet, Obergaulinger, Cerdá-Durán, \&  Aloy}]{Bugli__2020__MonthlyNoticesoftheRoyalAstronomicalSociety__TheImpactofNonDipolarMagneticFieldsinCoreCollapseSupernovae} Bugli, M., Guilet, J., Obergaulinger, M., Cerdá-Durán, P., \& Aloy, M.~A. 2020, Monthly Notices of the Royal Astronomical Society, 492, 58. doi:10.1093/mnras/stz3483

\bibitem[{Bugli {et~al.}(2021)Bugli, Guilet, \& Obergaulinger}]{Bugli__2021__mnras__ThreeDimensionalCoreCollapseSupernovaewithComplexMagneticStructuresI.ExplosionDynamics} Bugli, M., Guilet, J., \& Obergaulinger, M. 2021, \mnras, 507, 443. doi:10.1093/mnras/stab2161

%\bibitem[Bugli et al.(2020)]{Bugli2020} Bugli, M., Guilet, J., Obergaulinger, M., et al., \mnras, 492, 58 (2020). doi:10.1093/mnras/stz3483

%\bibitem[Bugli et al.(2021)]{Bugli2021} Bugli, M., Guilet, J., \& Obergaulinger, M., \mnras, 507, 443 (2021). doi:10.1093/mnras/stab2161

\bibitem[Burbidge et al.(1957)]{Burbidge1957} Burbidge, E.~M., Burbidge, G.~R., Fowler, W.~A., et al., Reviews of Modern Physics, 29, 547 (1957). doi:10.1103/RevModPhys.29.547

\bibitem[{{Burrows} {et~al.}(2007){Burrows}, {Dessart}, {Livne}, {Ott}, \&
  {Murphy}}]{Burrows_etal__2007__ApJ__MHD-SN}
{Burrows}, A., {Dessart}, L., {Livne}, E., {Ott}, C.~D., \& {Murphy}, J. 2007,
  \apj, 664, 416

\bibitem[{Burrows {et~al.}(2020)Burrows, Radice, Vartanyan, Nagakura, Skinner,
  \&
  Dolence}]{Burrows__2020__MonthlyNoticesoftheRoyalAstronomicalSociety__TheOverarchingFrameworkofCoreCollapseSupernovaExplosionsAsRevealedby3DFORNAXSimulations}
Burrows, A., Radice, D., Vartanyan, D., {et~al.} 2020, Monthly Notices of the
  Royal Astronomical Society, 491, 2715

\bibitem[Cabez{\'o}n et al.(2004)]{Cabezon2004} Cabez{\'o}n, R.~M., Garc{\'\i}a-Senz, D., \& Bravo, E., \apjs, 151, 345 (2004). doi:10.1086/382352

\bibitem[Cameron(1957)]{Cameron1957} Cameron, A.~G.~W., \aj, 62, 9 (1957). doi:10.1086/107435

\bibitem[Cayrel et al.(2004)]{Cayrel2004} Cayrel, R., Depagne, E., Spite, M., et al., \aap, 416, 1117 (2004). doi:10.1051/0004-6361:20034074

\bibitem[Cescutti et al.(2015)]{Cescutti2015} Cescutti, G., Romano, D., Matteucci, F., et al., \aap, 577, A139 (2015). doi:10.1051/0004-6361/201525698

\bibitem[Chen et al.(2017)]{Chen2017} Chen, K.-J., Moriya, T.~J., Woosley, S., et al., \apj, 839, 85 (2017). doi:10.3847/1538-4357/aa68a4

\bibitem[{{Choplin} {et~al.}(2020){Choplin}, {Tominaga}, \&  {Meyer}}]{Choplin__2020__aap__AStrongNeutronBurstinJetlikeSupernovaeofSpinstars}{Choplin}, A., {Tominaga}, N., \& {Meyer}, B.~S., \aap, 639, A126 (2020). doi:10.1051/0004-6361/202243331

%\bibitem[Choplin et al.(2022)]{Choplin2022} Choplin, A., Goriely, S., Hirschi, R., et al., \aap, 661, A86 (2022). doi:10.1051/0004-6361/202243331

\bibitem[Chruslinska et al.(2018)]{Chruslinska2018} Chruslinska, M., Belczynski, K., Klencki, J., et al., \mnras, 474, 2937 (2018). doi:10.1093/mnras/stx2923

\bibitem[Clayton(1968)]{Clayton1968} Clayton, D.~D., Principles of stellar evolution and nucleosynthesis (New York, McGraw-Hill, 1968)

\bibitem[Clifford \& Tayler(1965)]{Clifford1965} Clifford, F.~E. \& Tayler, R.~J., \memras, 69, 21 (1965)

\bibitem[Colgate \& White(1966)]{Colgate1966} Colgate, S.~A. \& White, R.~H., \apj, 143, 626 (1966). doi:10.1086/148549

\bibitem[{Corsi \&
  Lazzati(2021)}]{Corsi__2021__NewAstronomyReviews__GammaRayBurstJetsinSupernovae}
Corsi, A. \& Lazzati, D. 2021, New Astronomy Reviews, 92, 101614

\bibitem[{{Couch} \&
  {Ott}(2013)}]{Couch_Ott__2013__apjl__RevivaloftheStalledCore-collapseSupernovaShockTriggeredbyPrecollapseAsphericityintheProgenitorStar}
{Couch}, S.~M. \& {Ott}, C.~D. 2013, \apjl, 778, L7

\bibitem[C{\^o}t{\'e} et al.(2019)]{Cote2019} C{\^o}t{\'e}, B., Eichler, M., Arcones, A., et al., \apj, 875, 106 (2019). doi:10.3847/1538-4357/ab10db

\bibitem[{Cowan {et~al.}(2021)Cowan, Sneden, Lawler, Aprahamian, Wiescher, Langanke, Martínez-Pinedo, \&  Thielemann}]{Cowan__2021__ReviewsofModernPhysics__OriginoftheHeaviestElementstheRapidNeutronCaptureProcess}Cowan, J.~J., Sneden, C., Lawler, J.~E., {et~al.} 2021, Reviews of Modern Physics, 93, 015002, aDS Bibcode: 2021RvMP...93a5002C. doi:10.1103/RevModPhys.93.015002

%\bibitem[Cowan et al.(2021)]{Cowan2021} Cowan, J.~J., Sneden, C., Lawler, J.~E., et al., Reviews of Modern Physics, 93, 015002 (2021). doi:10.1103/RevModPhys.93.015002

\bibitem[Cox \& Giuli(1968)]{Cox1968} Cox, J.~P. \& Giuli, R.~T., Principles of stellar structure. (New York, Gordon and Breach, 1968)

\bibitem[D'Avanzo(2015)]{DAvanzo2015} D'Avanzo, P., Journal of High Energy Astrophysics, 7, 73 (2015). doi:10.1016/j.jheap.2015.07.002

\bibitem[D'Elia et al.(2015)]{DElia2015} D'Elia, V., Pian, E., Melandri, A., et al., \aap, 577, A116 (2015). doi:10.1051/0004-6361/201425381

\bibitem[Denissenkov \& VandenBerg(2003)]{Denissenkov2003} Denissenkov, P.~A. \& VandenBerg, D.~A., \apj, 593, 509 (2003). doi:10.1086/376410

\bibitem[Duffau et al.(2017)]{Duffau2017} Duffau, S., Caffau, E., Sbordone, L., et al., \aap, 604, A128 (2017). doi:10.1051/0004-6361/201730477

\bibitem[{{Duncan} \&
  {Thompson}(1992)}]{Duncan_Thompson__1992__ApJL__Magnetars}
{Duncan}, R.~C. \& {Thompson}, C. 1992, \apjl, 392, L9

\bibitem[Eichler et al.(2018)]{Eichler2018} Eichler, M., Nakamura, K., Takiwaki, T., et al., Journal of Physics G Nuclear Physics, 45, 014001 (2018). doi:10.1088/1361-6471/aa8891

\bibitem[{Ekanger {et~al.}(2022)Ekanger, Bhattacharya, \&
  Horiuchi}]{Ekanger__2022__MonthlyNoticesoftheRoyalAstronomicalSociety__SystematicExplorationofHeavyElementNucleosynthesisinProtomagnetarOutflows}
Ekanger, N., Bhattacharya, M., \& Horiuchi, S. 2022, Monthly Notices of the
  Royal Astronomical Society, 513, 405, aDS Bibcode: 2022MNRAS.513..405E

\bibitem[Ezzeddine et al.(2019)]{Ezzeddine2019} Ezzeddine, R., Frebel, A., Roederer, I.~U., et al., \apj, 876, 97 (2019). doi:10.3847/1538-4357/ab14e7

\bibitem[Farouqi et al.(2021)]{Farouqi2021} Farouqi, K., Thielemann, F.-K., Rosswog, S., et al.\ (2021), arXiv:2107.03486

\bibitem[Fong et al.(2017)]{Fong2017} Fong, W., Berger, E., Blanchard, P.~K., et al., \apjl, 848, L23 (2017). doi:10.3847/2041-8213/aa9018

\bibitem[Freiburghaus et al.(1999)]{Freiburghaus1999} Freiburghaus, C., Rembges, J.-F., Rauscher, T., et al., \apj, 516, 381 (1999). doi:10.1086/307072

\bibitem[Fr{\"o}hlich et al.(2006)]{Froehlich2006} Fr{\"o}hlich, C., Mart{\'\i}nez-Pinedo, G., Liebend{\"o}rfer, M., et al., \prl, 96, 142502 (2006). doi:10.1103/PhysRevLett.96.142502

\bibitem[{Fujimoto {et~al.}(2008)Fujimoto, Nishimura, \&  Hashimoto}]{Fujimoto__2008__TheAstrophysicalJournal__NucleosynthesisinMagneticallyDrivenJetsfromCollapsars}Fujimoto, S.-i., Nishimura, N., \& Hashimoto, M.-a. 2008, The Astrophysical Journal, 680, 1350, aDS Bibcode: 2008ApJ...680.1350F. doi:10.1086/529416

%\bibitem[Fujimoto et al.(2008)]{Fujimoto2008} Fujimoto, S.-. ichiro ., Nishimura, N., \& Hashimoto, M.-. aki ., \apj, 680, 1350 (2008). doi:10.1086/529416

\bibitem[Fuller et al.(1985)]{Fuller1985} Fuller, G.~M., Fowler, W.~A., \& Newman, M.~J., \apj, 293, 1 (1985). doi:10.1086/163208

\bibitem[{Gabler {et~al.}(2021)Gabler, Wongwathanarat, \&
  Janka}]{Gabler__2021__MonthlyNoticesoftheRoyalAstronomicalSociety__TheInfancyofCoreCollapseSupernovaRemnants}
Gabler, M., Wongwathanarat, A., \& Janka, H.-T. 2021, Monthly Notices of the
  Royal Astronomical Society, 502, 3264, aDS Bibcode: 2021MNRAS.502.3264G

\bibitem[Gallagher et al.(2020)]{Gallagher2020} Gallagher, A.~J., Bergemann, M., Collet, R., et al., \aap, 634, A55 (2020). doi:10.1051/0004-6361/201936104

\bibitem[Gratton et al.(2000)]{Gratton2000} Gratton, R.~G., Sneden, C., Carretta, E., et al., \aap, 354, 169 (2000)

\bibitem[Graur et al.(2014)]{Graur2014} Graur, O., Rodney, S.~A., Maoz, D., et al., \apj, 783, 28 (2014). doi:10.1088/0004-637X/783/1/28

\bibitem[Grebel et al.(2003)]{Grebel2003} Grebel, E.~K., Gallagher, J.~S., \& Harbeck, D., \aj, 125, 1926 (2003). doi:10.1086/368363

\bibitem[{Grimmett {et~al.}(2021)Grimmett, Müller, Heger, Banerjee, \&  Obergaulinger}]{Grimmett__2021__MonthlyNoticesoftheRoyalAstronomicalSociety__TheChemicalSignatureofJetDrivenHypernovae}Grimmett, J.~J., Müller, B., Heger, A., Banerjee, P., \& Obergaulinger, M. 2021, Monthly Notices of the Royal Astronomical Society, 501, 2764. doi:10.1093/mnras/staa3819

%\bibitem[Grimmett et al.(2021)]{Grimmett2021} Grimmett, J.~J., M{\"u}ller, B., Heger, A., et al., \mnras, 501, 2764 (2021). doi:10.1093/mnras/staa3819

\bibitem[{{Halevi} \& {M{\"o}sta}(2018)}]{Halevi_Moesta__2018__mnras__r-Processnucleosynthesisfromthree-dimensionaljet-drivencore-collapsesupernovaewithmagneticmisalignments}{Halevi}, G. \& {M{\"o}sta}, P. 2018, \mnras, 477, 2366. doi:10.1093/mnras/sty797

\bibitem[Hamuy(2003)]{Hamuy2003} Hamuy, M., \apj, 582, 905 (2003). doi:10.1086/344689
%\bibitem[Halevi \& M{\"o}sta(2018)]{Halevi2018} Halevi, G. \& M{\"o}sta, P., \mnras, 477, 2366 (2018). doi:10.1093/mnras/sty797

\bibitem[Harris et al.(2017)]{Harris2017} Harris, J.~A., Hix, W.~R., Chertkow, M.~A., et al., \apj, 843, 2 (2017). doi:10.3847/1538-4357/aa76de

\bibitem[Hartmann et al.(1985)]{Hartmann1985} Hartmann, D., Woosley, S.~E., \& El Eid, M.~F., \apj, 297, 837 (1985). doi:10.1086/163580

\bibitem[{{Heger} {et~al.}(2005){Heger}, {Woosley}, \& {Spruit}}]{Heger_et_al__2005__apj__Presupernova_Evolution_of_Differentially_Rotating_Massive_Stars_Including_Magnetic_Fields}{Heger}, A., {Woosley}, S.~E., \& {Spruit}, H.~C. 2005, \apj, 626, 350

\bibitem[Henkel et al.(2018)]{Henkel2018} Henkel, K., Karakas, A.~I., Casey, A.~R., et al., \apjl, 863, L5 (2018). doi:10.3847/2041-8213/aad552

\bibitem[Hillebrandt et al.(2013)]{Hillebrandt2013} Hillebrandt, W., Kromer, M., R{\"o}pke, F.~K., et al., Frontiers of Physics, 8, 116 (2013). doi:10.1007/s11467-013-0303-2

\bibitem[Hillebrandt et al.(1981)]{Hillebrandt1981} Hillebrandt, W., Thielemann, F.~K., Klapdor, H.~V., et al., \aap, 99, 195 (1981)

\bibitem[Hirai et al.(2018)]{Hirai2018} Hirai, Y., Saitoh, T.~R., Ishimaru, Y., et al., \apj, 855, 63 (2018). doi:10.3847/1538-4357/aaaabc

\bibitem[Hirai et al.(2022)]{Hirai2022} Hirai, Y., Beers, T.~C., Chiba, M., et al. (2022), arXiv:2206.04060

\bibitem[Hix \& Meyer(2006)]{Hix2006} Hix, W.~R. \& Meyer, B.~S., \nphysa, 777, 188 (2006). doi:10.1016/j.nuclphysa.2004.10.009

\bibitem[{Hix \& Thielemann(1999)}]{Hix__1999__JournalofComputationalandAppliedMathematics__ComputationalMethodsforNucleosynthesisandNuclearEnergyGeneration.}Hix, W.~R. \& Thielemann, F.~K. 1999, Journal of Computational and Applied  Mathematics, 109, 321

%\bibitem[Hix \& Thielemann(1999)]{Hix1999} [Hix \& Thielemann(1999)]{Hix1999} Hix, W.~R. \& Thielemann, F.-K., Journal of Computational and Applied Mathematics, 109, 321 (1999)

\bibitem[Horowitz et al.(2019)]{Horowitz2019} Horowitz, C.~J., Arcones, A., C{\^o}t{\'e}, B., et al., Journal of Physics G Nuclear Physics, 46, 083001 (2019). doi:10.1088/1361-6471/ab0849

\bibitem[Iliadis(2007)]{Iliadis2007} Iliadis, C., Nuclear Physics of Stars. (Wiley-VCH Verlag, Weinheim, Germany, 2007). doi:10.1002/9783527692668

\bibitem[Ishimaru et al.(2015)]{Ishimaru2015} Ishimaru, Y., Wanajo, S., \& Prantzos, N., \apjl, 804, L35 (2015). doi:10.1088/2041-8205/804/2/L35

\bibitem[{{Janka} {et~al.}(2007){Janka}, {Langanke}, {Marek},
  {Mart{\'{\i}}nez-Pinedo}, \& {M{\"u}ller}}]{Janka_etal__2007__PRD__SN_theory}
{Janka}, H.-T., {Langanke}, K., {Marek}, A., {Mart{\'{\i}}nez-Pinedo}, G., \&
  {M{\"u}ller}, B. 2007, \physrep, 442, 38

\bibitem[{{Janka}(2012)}]{Janka__2012__ARNPS__ExplosionMechanismsofCore-CollapseSupernovae}
{Janka}, H.-T. 2012, Annual Review of Nuclear and Particle Science, 62, 407

\bibitem[{{Janka} {et~al.}(2016){Janka}, {Melson}, \&
  {Summa}}]{Janka_et_al__2016__AnnualReviewofNuclearandParticleScience__PhysicsofCore-CollapseSupernovaeinThreeDimensionsASneakPreview}
{Janka}, H.-T., {Melson}, T., \& {Summa}, A. 2016, Annual Review of Nuclear and
  Particle Science, 66, 341

\bibitem[Ji et al.(2016)]{Ji2016} Ji, A.~P., Frebel, A., Simon, J.~D., et al., \apj, 830, 93 (2016). doi:10.3847/0004-637X/830/2/93

\bibitem[Juodagalvis et al.(2010)]{Juodagalvis2010} Juodagalvis, A., Langanke, K., Hix, W.~R., et al., \nphysa, 848, 454 (2010). doi:10.1016/j.nuclphysa.2010.09.012

\bibitem[{Just {et~al.}(2022)Just, Goriely, Janka, Nagataki, \&
  Bauswein}]{Just__2022__MonthlyNoticesoftheRoyalAstronomicalSociety__NeutrinoAbsorptionandOtherPhysicsDependenciesinNeutrinoCooledBlackHoleAccretionDiscs}
Just, O., Goriely, S., Janka, H.~T., Nagataki, S., \& Bauswein, A. 2022,
  Monthly Notices of the Royal Astronomical Society, 509, 1377, aDS Bibcode:
  2022MNRAS.509.1377J

\bibitem[Kasen et al.(2017)]{Kasen2017} Kasen, D., Metzger, B., Barnes, J., et al., \nat, 551, 80 (2017). doi:10.1038/nature24453

\bibitem[Kobayashi et al.(2006)]{Kobayashi2006} Kobayashi, C., Umeda, H., Nomoto, K., et al., \apj, 653, 1145 (2006). doi:10.1086/508914

\bibitem[Kobayashi et al.(2020)]{Kobayashi2020} Kobayashi, C., Karakas, A.~I., \& Lugaro, M., \apj, 900, 179 (2020). doi:10.3847/1538-4357/abae65

\bibitem[{Komissarov \&
  Porth(2021)}]{Komissarov__2021__NewAstronomyReviews__NumericalSimulationsofJets}
Komissarov, S. \& Porth, O. 2021, New Astronomy Reviews, 92, 101610, aDS
  Bibcode: 2021NewAR..9201610K

\bibitem[{{Komissarov} \&
  {Barkov}(2009)}]{Komissarov_Barkov__2009__mnras__ActivationoftheBlandford-Znajekmechanismincollapsingstars}
{Komissarov}, S.~S. \& {Barkov}, M.~V. 2009, \mnras, 397, 1153

\bibitem[Komiya \& Shigeyama(2016)]{Komiya2016} Komiya, Y. \& Shigeyama, T., \apj, 830, 76 (2016). doi:10.3847/0004-637X/830/2/76

\bibitem[Kondratyev(2018)]{Kondratyev2018} Kondratyev, V.~N., \mnras, 480, 5380 (2018). doi:10.1093/mnras/sty2248

\bibitem[Kondratyev(2019)]{Kondratyev2019} Kondratyev, V.~N., Physics of Particles and Nuclei, 50, 576 (2019). doi:10.1134/S1063779619050149

\bibitem[Korobkin et al.(2012)]{Korobkin2012} Korobkin, O., Rosswog, S., Arcones, A., et al., \mnras, 426, 1940 (2012). doi:10.1111/j.1365-2966.2012.21859.x

\bibitem[Kostka et al.(2014)]{Kostka2014} Kostka, M., Koning, N., Shand, Z., et al., \aap, 568, A97 (2014). doi:10.1051/0004-6361/201322887

\bibitem[{{Kuroda} {et~al.}(2020){Kuroda}, {Arcones}, {Takiwaki}, \&  {Kotake}}]{Kuroda_et_al__2020__apj__MagnetorotationalExplosionofaMassiveStarSupportedbyNeutrinoHeatinginGeneralRelativisticThreeDimensionalSimulations}{Kuroda}, T., {Arcones}, A., {Takiwaki}, T., \& {Kotake}, K. 2020, \apj, 896, 102. doi:10.3847/1538-4357/ab9308

%\bibitem[Kuroda et al.(2020)]{Kuroda2020} Kuroda, T., Arcones, A., Takiwaki, T., et al., \apj, 896, 102 (2020). doi:10.3847/1538-4357/ab9308

\bibitem[Lagarde et al.(2012)]{Lagarde2012} Lagarde, N., Decressin, T., Charbonnel, C., et al., \aap, 543, A108 (2012). doi:10.1051/0004-6361/201118331

\bibitem[Lai et al.(2011)]{Lai2011} Lai, D.~K., Lee, Y.~S., Bolte, M., et al.\ 2011, \apj, 738, 51. doi:10.1088/0004-637X/738/1/51

\bibitem[{{Lander}(2021)}]{Lander__2021__mnras__GeneratingNeutronStarMagneticFieldsThreeDynamoPhases}
{Lander}, S.~K. 2021, \mnras, 507, L36

\bibitem[Langanke \& Mart{\'\i} [Langanke \& Mart{\'\i}nez-Pinedo(2000)]{Langanke2000} Langanke, K. \& Mart{\'\i}nez-Pinedo, G., \nphysa, 673, 481 (2000). doi:10.1016/S0375-9474(00)00131-7

\bibitem[Langanke et al.(2021)]{Langanke2021} Langanke, K., Mart{\'\i}nez-Pinedo, G., \& Zegers, R.~G.~T., Reports on Progress in Physics, 84, 066301 (2021). doi:10.1088/1361-6633/abf207

\bibitem[{{Langer}(2012)}]{Langer__2012__ARAA__Presupernova_Evolution_of_Massive_Single_and_Binary_Stars}
{Langer}, N. 2012, \araa, 50, 107

\bibitem[{{Lazzati} {et~al.}(2012){Lazzati}, {Morsony}, {Blackwell}, \&
  {Begelman}}]{Lazzati_et_al__2012__apj__UnifyingtheZooofJet-drivenStellarExplosions}
{Lazzati}, D., {Morsony}, B.~J., {Blackwell}, C.~H., \& {Begelman}, M.~C. 2012,
  \apj, 750, 68

\bibitem[Lentz et al.(2015)]{Lentz2015} Lentz, E.~J., Bruenn, S.~W., Hix, W.~R., et al., \apjl, 807, L31 (2015). doi:10.1088/2041-8205/807/2/L31


\bibitem[Lippuner \& Roberts(2017)]{Lippuner2017} Lippuner, J. \& Roberts, L.~F., \apjs, 233, 18 (2017). doi:10.3847/1538-4365/aa94cb

\bibitem[Longland et al.(2014)]{Longland2014} Longland, R., Martin, D., \& Jos{\'e}, J., \aap, 563, A67 (2014). doi:10.1051/0004-6361/201321958

\bibitem[Lugaro et al.(2016)]{Lugaro2016} Lugaro, M., Pignatari, M., Ott, U., et al., Proceedings of the National Academy of Science, 113, 907 (2016). doi:10.1073/pnas.1519344113

\bibitem[{{Lynden-Bell}(1996)}]{Lynden-Bell__1996__mnras__Magneticcollimationbyaccretiondiscsofquasarsandstars}
{Lynden-Bell}, D. 1996, \mnras, 279, 389

\bibitem[{{Lynden-Bell}(2003)}]{Lynden-Bell__2003__mnras__Onwhydiscsgeneratemagnetictowersandcollimatejets}
{Lynden-Bell}, D. 2003, \mnras, 341, 1360

\bibitem[{Lyubarsky(2009)}]{Lyubarsky__2009__TheAstrophysicalJournal__AsymptoticStructureofPoyntingDominatedJets}
Lyubarsky, Y. 2009, The Astrophysical Journal, 698, 1570, aDS Bibcode:
  2009ApJ...698.1570L

\bibitem[{{MacFadyen} \& {Woosley}(1999)}]{MacFadyen_Woosley__1999__ApJ__Collapsar}{MacFadyen}, A.~I. \& {Woosley}, S.~E. 1999, \apj, 524, 262. doi:10.1086/307790

%\bibitem[MacFadyen \& Woosley(1999)]{MacFadyen1999} MacFadyen, A.~I. \& Woosley, S.~E., \apj, 524, 262 (1999). doi:10.1086/307790

\bibitem[Maeda \& Nomoto(2003)]{Maeda2003} Maeda, K. \& Nomoto, K., \apj, 598, 1163 (2003). doi:10.1086/378948

\bibitem[Magkotsios et al.(2011)]{Magkotsios2011} Magkotsios, G., Timmes, F.~X., \& Wiescher, M., \apj, 741, 78 (2011). doi:10.1088/0004-637X/741/2/78

\bibitem[Mashonkina et al.(2017)]{Mashonkina2017} Mashonkina, L., Jablonka, P., Sitnova, T., et al., \aap, 608, A89 (2017). doi:10.1051/0004-6361/201731582

\bibitem[Mashonkina et al.(2017a)]{Mashonkina2017a} Mashonkina, L., Jablonka, P., Pakhomov, Y., et al., \aap, 604, A129 (2017). doi:10.1051/0004-6361/201730779

\bibitem[Mashonkina et al.(2017b)]{Mashonkina2017b} Mashonkina, L., Jablonka, P., Sitnova, T., et al., \aap, 608, A89 (2017). doi:10.1051/0004-6361/201731582

\bibitem[Matteucci et al.(2014)]{Matteucci2014} Matteucci, F., Romano, D., Arcones, A., et al., \mnras, 438, 2177 (2014). doi:10.1093/mnras/stt2350

\bibitem[Matteucci(2021)]{Matteucci2021} Matteucci, F.\ 2021, \aapr, 29, 5. doi:10.1007/s00159-021-00133-8

\bibitem[{Matzner(2003)}]{Matzner__2003__MonthlyNoticesoftheRoyalAstronomicalSociety__SupernovaHostsforGammaRayBurstJetsDynamicalConstraints}
Matzner, C.~D. 2003, Monthly Notices of the Royal Astronomical Society, 345,
  575, aDS Bibcode: 2003MNRAS.345..575M

\bibitem[{McLaughlin \&
  Surman(2005)}]{McLaughlin__2005__NuclearPhysicsA__ProspectsforObtaininganRProcessfromGammaRayBurstDiskWinds}
McLaughlin, G.~C. \& Surman, R. 2005, Nuclear Physics A, 758, 189, aDS Bibcode:
  2005NuPhA.758..189M

\bibitem[{{Metzger} {et~al.}(2018){Metzger}, {Beniamini}, \&
  {Giannios}}]{Metzger2018a}
{Metzger}, B.~D., {Beniamini}, P., \& {Giannios}, D. 2018, \apj, 857, 95

\bibitem[{{Metzger} {et~al.}(2007){Metzger}, {Thompson}, \&
  {Quataert}}]{Metzger_et_al__2007__apj__Proto-NeutronStarWindswithMagneticFieldsandRotation}
{Metzger}, B.~D., {Thompson}, T.~A., \& {Quataert}, E. 2007, \apj, 659, 561

\bibitem[Meyer et al.(2000)]{Meyer2000} Meyer, B.~S., Clayton, D.~D., \& The, L.-S., \apjl, 540, L49 (2000). doi:10.1086/312865

\bibitem[Meyer(1994)]{Meyer1994} Meyer, B.~S., \araa, 32, 153 (1994). doi:10.1146/annurev.aa.32.090194.001101

\bibitem[{Miller {et~al.}(2020)Miller, Sprouse, Fryer, Ryan, Dolence, Mumpower,
  \& Surman}]{Miller__2020__TheAstrophysicalJournal__FullTransportGeneralRelativisticRadiationMagnetohydrodynamicsforNucleosynthesisinCollapsars}Miller, J.~M., Sprouse, T.~M., Fryer, C.~L., {et~al.} 2020, The Astrophysical Journal, 902, 66. doi:10.3847/1538-4357/abb4e3

%\bibitem[Miller et al.(2020)]{Miller2020} Miller, J.~M., Sprouse, T.~M., Fryer, C.~L., et al., \apj, 902, 66 (2020). doi:10.3847/1538-4357/abb4e3

\bibitem[Minelli et al.(2021)]{Minelli2021} Minelli, A., Mucciarelli, A., Romano, D., et al., \apj, 910, 114 (2021). doi:10.3847/1538-4357/abe3f9

\bibitem[Molero et al.(2021)]{Molero2021} Molero, M., Romano, D., Reichert, M., et al., \mnras, 505, 2913 (2021). doi:10.1093/mnras/stab1429

%\bibitem[{{Moll}(2009)}]{Moll__2009__aap__Decay_of_the_toroidal_field_in_magnetically_driven_jets}
{Moll}, R. 2009, \aap, 507, 1203

\bibitem[{{Moll}(2010)}]{Moll__2010__aap__LargeJetsfromSmallScaleMagneticFields}
{Moll}, R. 2010, \aap, 512, A5

\bibitem[{{Moll} {et~al.}(2008){Moll}, {Spruit}, \&
  {Obergaulinger}}]{Moll_et_al__2008__aap__Kink_instabilities_in_jets_from_rotating_magnetic_fields}
{Moll}, R., {Spruit}, H.~C., \& {Obergaulinger}, M. 2008, \aap, 492, 621

\bibitem[{{M{\"o}sta} {et~al.}(2014){M{\"o}sta}, {Richers}, {Ott}, {Haas},
  {Piro}, {Boydstun}, {Abdikamalov}, {Reisswig}, \&
  {Schnetter}}]{Mosta_et_al__2014__apjl__MagnetorotationalCore-collapseSupernovaeinThreeDimensions}
{M{\"o}sta}, P., {Richers}, S., {Ott}, C.~D., {et~al.} 2014, \apjl, 785, L29

\bibitem[{{M{\"o}sta} {et~al.}(2015){M{\"o}sta}, {Ott}, {Radice}, {Roberts},
  {Schnetter}, \&
  {Haas}}]{Moesta_et_al__2015__nat__Alarge-scaledynamoandmagnetoturbulenceinrapidlyrotatingcore-collapsesupernovae}
{M{\"o}sta}, P., {Ott}, C.~D., {Radice}, D., {et~al.} 2015, \nat, 528, 376

\bibitem[{{M{\"o}sta} {et~al.}(2018){M{\"o}sta}, {Roberts}, {Halevi}, {Ott},  {Lippuner}, {Haas}, \&
  {Schnetter}}]{Moesta_et_al__2018__apj__r-processNucleosynthesisfromThree-dimensionalMagnetorotationalCore-collapseSupernovae}{M{\"o}sta}, P., {Roberts}, L.~F., {Halevi}, G., {et~al.} 2018, \apj, 864, 171. doi:10.3847/1538-4357/aad6ec

%\bibitem[M{\"o}sta et al.(2018)]{Moesta2018} M{\"o}sta, P., Roberts, L.~F., Halevi, G., et al., \apj, 864, 171 (2018). doi:10.3847/1538-4357/aad6ec

\bibitem[M{\"u}ller(1986)]{Mueller1986} M{\"u}ller, E., \aap, 162, 103 (1986)

\bibitem[{{M{\"u}ller}(2020)}]{Mueller__2020__LivingReviewsinComputationalAstrophysics__HydrodynamicsofCoreCollapseSupernovaeandTheirProgenitors}
{M{\"u}ller}, B. 2020, Living Reviews in Computational Astrophysics, 6, 3

\bibitem[Nagataki et al.(1997)]{Nagataki1997} Nagataki, S., Hashimoto, M.-. aki ., Sato, K., et al., \apj, 486, 1026 (1997). doi:10.1086/304565

\bibitem[{{Nagataki} {et~al.}(2007){Nagataki}, {Takahashi}, {Mizuta}, \&
  {Takiwaki}}]{Nagataki_et_al__2007__apj__GRB_Jet_Formation_in_Collapsars}
{Nagataki}, S., {Takahashi}, R., {Mizuta}, A., \& {Takiwaki}, T. 2007, \apj,
  659, 512

\bibitem[{{Nagataki}(2018)}]{Nagataki__2018__ReportsonProgressinPhysics__TheoriesofCentralEngineforLongGammaRayBursts}
{Nagataki}, S. 2018, Reports on Progress in Physics, 81, 026901

\bibitem[{{Nakamura} {et~al.}(2015){Nakamura}, {Kajino}, {Mathews}, {Sato}, \&
  {Harikae}}]{Nakamura__2015__aap__RProcessNucleosynthesisintheMHDneutrinoHeatedCollapsarJet}
{Nakamura}, K., {Kajino}, T., {Mathews}, G.~J., {Sato}, S., \& {Harikae}, S.
  2015, \aap, 582, A34

\bibitem[Nakamura et al.(2014)]{Nakamura2014} Nakamura, K., Takiwaki, T., Kotake, K., et al., \apj, 782, 91 (2014). doi:10.1088/0004-637X/782/2/91


\bibitem[Nishimura et al.(2006)]{Nishimura2006} Nishimura, S., Kotake, K., Hashimoto, M.-. aki ., et al., \apj, 642, 410 (2006). doi:10.1086/500786

\bibitem[{{Nishimura} {et~al.}(2015){Nishimura}, {Takiwaki}, \&  {Thielemann}}]{Nishimura_et_al__2015__apj__Ther-processNucleosynthesisintheVariousJet-likeExplosionsofMagnetorotationalCore-collapseSupernovae} {Nishimura}, N., {Takiwaki}, T., \& {Thielemann}, F.-K. 2015, \apj, 810, 109. doi:10.1088/0004-637X/810/2/109

%\bibitem[Nishimura et al.(2015)]{Nishimura2015} Nishimura, N., Takiwaki, T., \& Thielemann, F.-K., \apj, 810, 109 (2015). doi:10.1088/0004-637X/810/2/109

\bibitem[{{Nishimura} {et~al.}(2017){Nishimura}, {Sawai}, {Takiwaki}, {Yamada},  \&  {Thielemann}}]{Nishimura_et_al__2017__apjl__TheIntermediateRProcessinCoreCollapseSupernovaeDrivenbytheMagnetoRotationalInstability}{Nishimura}, N., {Sawai}, H., {Takiwaki}, T., {Yamada}, S., \& {Thielemann}, F.~K. 2017, \apjl, 836, L21. doi:10.3847/2041-8213/aa5dee

%\bibitem[Nishimura et al.(2017)]{Nishimura2017} Nishimura, N., Sawai, H., Takiwaki, T., et al., \apjl, 836, L21 (2017). doi:10.3847/2041-8213/aa5dee

\bibitem[Nissen \& Schuster(2011)]{Nissen2011} Nissen, P.~E. \& Schuster, W.~J., \aap, 530, A15 (2011). doi:10.1051/0004-6361/201116619

\bibitem[{Nomoto {et~al.}(2006)Nomoto, Tominaga, Umeda, Kobayashi, \&  Maeda}]{Nomoto__2006__NuclearPhysicsA__NucleosynthesisYieldsofCoreCollapseSupernovaeandHypernovaeandGalacticChemicalEvolution}Nomoto, K., Tominaga, N., Umeda, H., Kobayashi, C., \& Maeda, K. 2006, Nuclear Physics A, 777, 424, aDS Bibcode: 2006NuPhA.777..424N

%\bibitem[Nomoto et al.(2013)]{Nomoto2013} Nomoto, K., Kobayashi, C., \& Tominaga, N., \araa, 51, 457 (2013). doi:10.1146/annurev-astro-082812-140956

\bibitem[{{Nomoto} {et~al.}(2013){Nomoto}, {Kobayashi}, \&  {Tominaga}}]{Nomoto_et_al__2013__araa__NucleosynthesisinStarsandtheChemicalEnrichmentofGalaxies}{Nomoto}, K., {Kobayashi}, C., \& {Tominaga}, N. 2013, \araa, 51, 457. doi:10.1146/annurev-astro-082812-140956

\bibitem[Nomoto(2017)]{Nomoto2017} Nomoto, K., Handbook of Supernovae, 1931 (Springer International Publishing AG, 2017). doi:10.1007/978-3-319-21846-5\_86

%\bibitem[{Nomoto(2017)}]{Nomoto__2017__NucleosynthesisinHypernovaeAssociatedwithGammaRayBursts} Nomoto, K. 2017, Nucleosynthesis in {Hypernovae} {Associated} with {Gamma}-{Ray} {Bursts}, 1931

\bibitem[Nordlander et al.(2017)]{Nordlander2017} Nordlander, T., Amarsi, A.~M., Lind, K., et al., \aap, 597, A6 (2017). doi:10.1051/0004-6361/201629202

\bibitem[Norris \& Yong(2019)]{Norris2019} Norris, J.~E. \& Yong, D., \apj, 879, 37 (2019). doi:10.3847/1538-4357/ab1f84

\bibitem[{{Obergaulinger} \&  {Aloy}(2017)}]{Obergaulinger_Aloy__2017__mnras__Protomagnetarandblackholeformationinhigh-massstars}{Obergaulinger}, M. \& {Aloy}, M.~{\'A}. 2017, \mnras, 469, L43. doi:10.1093/mnrasl/slx046

\bibitem[{{Obergaulinger} \&  {Aloy}(2020)}]{Obergaulinger_Aloy__2020__mnras__MagnetorotationalCoreCollapseofPossibleGRBProgenitorsIExplosionMechanisms}{Obergaulinger}, M. \& {Aloy}, M.~{\'A}. 2020, \mnras, 492, 4613

\bibitem[{{Obergaulinger} \&  {Aloy}(2021)}]{Obergaulinger__2021__mnras__MagnetorotationalCoreCollapseofPossibleGRBProgenitorsIII.ThreeDimensionalModels}{Obergaulinger}, M. \& {Aloy}, M.~{\'A}. 2021, \mnras, 503, 4942). doi:10.1093/mnras/stab295

\bibitem[{Obergaulinger \&  Aloy(2022)}]{Obergaulinger__2022__MonthlyNoticesoftheRoyalAstronomicalSociety__MagnetorotationalCoreCollapseofPossibleGammaRayBurstProgenitorsIV.aWiderRangeofProgenitors}Obergaulinger, M. \& Aloy, M.~Á. 2022, Monthly Notices of the Royal Astronomical Society, 512, 2489. doi:10.1093/mnras/stac613

%\bibitem[Obergaulinger \& Aloy(2017)]{Obergaulinger2017} Obergaulinger, M. \& Aloy, M. {\'A}., \mnras, 469, L43 (2017). doi:10.1093/mnrasl/slx046

%\bibitem[Obergaulinger \& Aloy(2021)]{Obergaulinger2021} Obergaulinger, M. \& Aloy, M. {\'A}., \mnras, 503, 4942 (2021). doi:10.1093/mnras/stab295

%\bibitem[Obergaulinger \& Aloy(2022)]{Obergaulinger2022} Obergaulinger, M. \& Aloy, M. {\'A}., \mnras, 512, 2489 (2022). doi:10.1093/mnras/stac613

\bibitem[O'Connor \& Couch(2018)]{Oconnor2018} O'Connor, E.~P. \& Couch, S.~M., \apj, 865, 81 (2018). doi:10.3847/1538-4357/aadcf7


\bibitem[Oda et al.(1994)]{Oda1994} Oda, T., Hino, M., Muto, K., et al., Atomic Data and Nuclear Data Tables, 56, 231 (1994). doi:10.1006/adnd.1994.1007

\bibitem[Ojima et al.(2018)]{Ojima2018} Ojima, T., Ishimaru, Y., Wanajo, S., et al.\ 2018, \apj, 865, 87. doi:10.3847/1538-4357/aada11

\bibitem[{Ono {et~al.}(2009)Ono, Hashimoto, Fujimoto, Kotake, \&
  Yamada}]{Ono__2009__ProgressofTheoreticalPhysics__ExplosiveNucleosynthesisinMagnetohydrodynamicalJetsfromCollapsars}
Ono, M., Hashimoto, M., Fujimoto, S., Kotake, K., \& Yamada, S. 2009, Progress
  of Theoretical Physics, 122, 755, aDS Bibcode: 2009PThPh.122..755O

\bibitem[{Ono {et~al.}(2012)Ono, Hashimoto, Fujimoto, Kotake, \&
  Yamada}]{Ono__2012__ProgressofTheoreticalPhysics__ExplosiveNucleosynthesisinMagnetohydrodynamicalJetsfromCollapsars.IIHeavyElementNucleosynthesisofSPRProcesses}
Ono, M., Hashimoto, M., Fujimoto, S., Kotake, K., \& Yamada, S. 2012, Progress
  of Theoretical Physics, 128, 741, aDS Bibcode: 2012PThPh.128..741O

\bibitem[{{Orlando} {et~al.}(2021){Orlando}, {Wongwathanarat}, {Janka},
  {Miceli}, {Ono}, {Nagataki}, {Bocchino}, \&
  {Peres}}]{Orlando__2021__aap__TheFullyDevelopedRemnantofaNeutrinoDrivenSupernova.EvolutionofEjectaStructureandAsymmetriesinSNRCassiopeiaa}
{Orlando}, S., {Wongwathanarat}, A., {Janka}, H.~T., {et~al.} 2021, \aap, 645,
  A66

\bibitem[{Papish {et~al.}(2015)Papish, Nordhaus, \&
  Soker}]{Papish__2015__MonthlyNoticesoftheRoyalAstronomicalSociety__ACallforaParadigmShiftfromNeutrinoDriventoJetDrivenCoreCollapseSupernovaMechanisms}
Papish, O., Nordhaus, J., \& Soker, N. 2015, Monthly Notices of the Royal
  Astronomical Society, 448, 2362, aDS Bibcode: 2015MNRAS.448.2362P

\bibitem[{Papish \&
  Soker(2014)}]{Papish__2014__MonthlyNoticesoftheRoyalAstronomicalSociety__ExplodingCoreCollapseSupernovaebyJetsDrivenFeedbackMechanism}
Papish, O. \& Soker, N. 2014, Monthly Notices of the Royal Astronomical
  Society, 438, 1027, aDS Bibcode: 2014MNRAS.438.1027P

\bibitem[Paxton et al.(2011)]{Paxton2011} Paxton, B., Bildsten, L., Dotter, A., et al., \apjs, 192, 3 (2011). doi:10.1088/0067-0049/192/1/3

\bibitem[Pian et al.(2017)]{Pian2017} Pian, E., D'Avanzo, P., Benetti, S., et al., \nat, 551, 67 (2017). doi:10.1038/nature24298

\bibitem[Pignatari et al.(2015)]{Pignatari2015} Pignatari, M., Zinner, E., Hoppe, P., et al., \apjl, 808, L43 (2015). doi:10.1088/2041-8205/808/2/L43

\bibitem[Pignatari et al.(2016)]{Pignatari2016} Pignatari, M., G{\"o}bel, K., Reifarth, R., et al., International Journal of Modern Physics E, 25, 1630003-232 (2016). doi:10.1142/S0218301316300034

\bibitem[Piran(1992)]{Piran1992} Piran, T., \apjl, 389, L45 (1992). doi:10.1086/186345

\bibitem[{{Piran}(2004)}]{Piran__2004__RMP__The_physics_of_GRBs}
{Piran}, T. 2004, Reviews of Modern Physics, 76, 1143

\bibitem[{Piran {et~al.}(2019)Piran, Nakar, Mazzali, \&
  Pian}]{Piran__2019__TheAstrophysicalJournal__RelativisticJetsinCoreCollapseSupernovae}
Piran, T., Nakar, E., Mazzali, P., \& Pian, E. 2019, The Astrophysical Journal,
  871, L25, aDS Bibcode: 2019ApJ...871L..25P

\bibitem[Placco et al.(2014)]{Placco2014} Placco, V.~M., Frebel, A., Beers, T.~C., et al., \apj, 797, 21 (2014). doi:10.1088/0004-637X/797/1/21

\bibitem[Pruet \& Fuller(2003)]{Pruet2003} Pruet, J. \& Fuller, G.~M., \apjs, 149, 189 (2003). doi:10.1086/376753

\bibitem[Pruet et al.(2006)]{Pruet2006} Pruet, J., Hoffman, R.~D., Woosley, S.~E., et al., \apj, 644, 1028 (2006). doi:10.1086/503891

\bibitem[{Qian \&  Woosley(1996)}]{Qian__1996__TheAstrophysicalJournal__NucleosynthesisinNeutrinoDrivenWinds.I.thePhysicalConditions}Qian, Y.~Z. \& Woosley, S.~E. 1996, The Astrophysical Journal, 471, 331. doi:10.1086/177973

%\bibitem[Qian \& Woosley(1996)]{Qian1996} Qian, Y.-Z. \& Woosley, S.~E., \apj, 471, 331 (1996). doi:10.1086/177973

\bibitem[Rauscher et al.(2013)]{Rauscher2013} Rauscher, T., Dauphas, N., Dillmann, I., et al., Reports on Progress in Physics, 76, 066201 (2013). doi:10.1088/0034-4885/76/6/066201

\bibitem[Rauscher et al.(2002)]{Rauscher2002} Rauscher, T., Heger, A., Hoffman, R.~D., et al., \apj, 576, 323 (2002). doi:10.1086/341728


\bibitem[Rayet et al.(1995)]{Rayet1995} Rayet, M., Arnould, M., Hashimoto, M., et al., \aap, 298, 517 (1995)

\bibitem[{Raynaud {et~al.}(2022)Raynaud, Cerdá-Durán, \&
  Guilet}]{Raynaud__2022__MonthlyNoticesoftheRoyalAstronomicalSociety__GravitationalWaveSignatureofProtoNeutronStarConvectionI.MHDNumericalSimulations}
Raynaud, R., Cerdá-Durán, P., \& Guilet, J. 2022, Monthly Notices of the Royal
  Astronomical Society, 509, 3410, aDS Bibcode: 2022MNRAS.509.3410R

\bibitem[{Raynaud {et~al.}(2020)Raynaud, Guilet, Janka, \&
  Gastine}]{Raynaud__2020__ScienceAdvances__MagnetarFormationthroughaConvectiveDynamoinProtoneutronStars}
Raynaud, R., Guilet, J., Janka, H.-T., \& Gastine, T. 2020, Science Advances,
  6, eaay2732, aDS Bibcode: 2020SciA....6.2732R

\bibitem[{Reboul-Salze {et~al.}(2021)Reboul-Salze, Guilet, Raynaud, \&
  Bugli}]{ReboulSalze__2021__aap__AGlobalModeloftheMagnetorotationalInstabilityinProtoneutronStars}
Reboul-Salze, A., Guilet, J., Raynaud, R., \& Bugli, M. 2021, \aap, 645, A109

\bibitem[Reichert et al.(2020)]{Reichert2020} Reichert, M., Hansen, C.~J., Hanke, M., et al., \aap, 641, A127 (2020). doi:10.1051/0004-6361/201936930

\bibitem[Reichert et al.(2022)]{Reichert2022} Reichert, M., Obergaulinger, M., Aloy, M.-A., et al. (2022), arXiv:2206.11914

\bibitem[Reichert et al.(2021a)]{Reichert2021b} Reichert, M., Hansen, C.~J., \& Arcones, A., \apj, 912, 157 (2021). doi:10.3847/1538-4357/abefd8

\bibitem[{Reichert {et~al.}(2021b)Reichert, Obergaulinger, Eichler, Aloy, \&  Arcones}]{Reichert__2021__MonthlyNoticesoftheRoyalAstronomicalSociety__NucleosynthesisinMagnetoRotationalSupernovae}Reichert, M., Obergaulinger, M., Eichler, M., Aloy, M.~Á., \& Arcones, A. 2021,  Monthly Notices of the Royal Astronomical Society, 501, 5733. doi:10.1093/mnras/stab029

%\bibitem[Reichert et al.(2021)]{Reichert2021} Reichert, M., Obergaulinger, M., Eichler, M., et al., \mnras, 501, 5733 (2021). doi:10.1093/mnras/stab029

\bibitem[Roberts et al.(2016)]{Roberts2016} Roberts, L.~F., Ott, C.~D., Haas, R., et al., \apj, 831, 98 (2016). doi:10.3847/0004-637X/831/1/98

\bibitem[Roederer et al.(2018)]{Roederer2018} Roederer, I.~U., Hattori, K., \& Valluri, M., \aj, 156, 179 (2018). doi:10.3847/1538-3881/aadd9c

\bibitem[Sandoval et al.(2021)]{Sandoval2021} Sandoval, M.~A., Hix, W.~R., Messer, O.~E.~B., et al., \apj, 921, 113 (2021). doi:10.3847/1538-4357/ac1d49

\bibitem[{{Scheidegger} {et~al.}(2010){Scheidegger}, {K{\"a}ppeli},
  {Whitehouse}, {Fischer}, \&
  {Liebend{\"o}rfer}}]{Scheidegger_et_al__2010__aap__Influence_of_model_parameters_on_GW_from_core_collapse}
{Scheidegger}, S., {K{\"a}ppeli}, R., {Whitehouse}, S.~C., {Fischer}, T., \&
  {Liebend{\"o}rfer}, M. 2010, \aap, 514, A51+

\bibitem[Sch{\"o}nrich \& Weinberg(2019)]{Schoenrich2019} Sch{\"o}nrich, R.~A. \& Weinberg, D.~H., \mnras, 487, 580 (2019). doi:10.1093/mnras/stz1126

\bibitem[Seitenzahl et al.(2008)]{Seitenzahl2008} Seitenzahl, I.~R., Timmes, F.~X., Marin-Lafl{\`e}che, A., et al., \apjl, 685, L129 (2008). doi:10.1086/592501

\bibitem[Seitenzahl et al.(2010)]{Seitenzahl2010} Seitenzahl, I.~R., R{\"o}pke, F.~K., Fink, M., et al., \mnras, 407, 2297 (2010). doi:10.1111/j.1365-2966.2010.17106.x

\bibitem[{{Shakura} \&
  {Sunyaev}(1973)}]{Shakura_Sunyaev__1973__AA__alpha_visco}
{Shakura}, N.~I. \& {Sunyaev}, R.~A. 1973, \aap, 24, 337

\bibitem[Shen et al.(2015)]{Shen2015} Shen, S., Cooke, R.~J., Ramirez-Ruiz, E., et al., \apj, 807, 115 (2015). doi:10.1088/0004-637X/807/2/115

\bibitem[{Siegel(2019)}]{Siegel__2019__EuropeanPhysicalJournalA__GW170817theFirstObservedNeutronStarMergerandItsKilonovaImplicationsfortheAstrophysicalSiteoftheRProcess}Siegel, D.~M. 2019, European Physical Journal A, 55, 203

\bibitem[{Siegel {et~al.}(2019)Siegel, Barnes, \&  Metzger}]{Siegel__2019__Nature__CollapsarsAsaMajorSourceofRProcessElements}Siegel, D.~M., Barnes, J., \& Metzger, B.~D. 2019, Nature, 569, 241. doi:10.1038/s41586-019-1136-0

%\bibitem[Siegel et al.(2019)]{Siegel2019} Siegel, D.~M., Barnes, J., \& Metzger, B.~D., \nat, 569, 241 (2019). doi:10.1038/s41586-019-1136-0

\bibitem[Sieverding et al.(2019)]{Sieverding2019} Sieverding, A., Langanke, K., Mart{\'\i}nez-Pinedo, G., et al., \apj, 876, 151 (2019). doi:10.3847/1538-4357/ab17e2

\bibitem[Simonetti et al.(2019)]{Simonetti2019} Simonetti, P., Matteucci, F., Greggio, L., et al., \mnras, 486, 2896 (2019). doi:10.1093/mnras/stz991

\bibitem[Siqueira Mello et al.(2014)]{SiqueiraMello2014} Siqueira Mello, C., Hill, V., Barbuy, B., et al., \aap, 565, A93 (2014). doi:10.1051/0004-6361/201423826

\bibitem[Sk{\'u}lad{\'o}ttir et al.(2017)]{Skuladottir2017} Sk{\'u}lad{\'o}ttir, {\'A}., Tolstoy, E., Salvadori, S., et al., \aap, 606, A71 (2017). doi:10.1051/0004-6361/201731158

\bibitem[Sk{\'u}lad{\'o}ttir et al.(2019)]{Skuladottir2019} Sk{\'u}lad{\'o}ttir, {\'A}., Hansen, C.~J., Salvadori, S., et al., \aap, 631, A171 (2019). doi:10.1051/0004-6361/201936125

\bibitem[Sk{\'u}lad{\'o}ttir \& Salvadori(2020)]{Skuladottir2020} Sk{\'u}lad{\'o}ttir, {\'A}. \& Salvadori, S., \aap, 634, L2 (2020). doi:10.1051/0004-6361/201937293

\bibitem[{{Smartt}(2009)}]{Smartt__2009__araa__Progenitors_of_CCSNe}
{Smartt}, S.~J. 2009, \araa, 47, 63

\bibitem[Smartt et al.(2017)]{Smartt2017} Smartt, S.~J., Chen, T.-W., Jerkstrand, A., et al., \nat, 551, 75 (2017). doi:10.1038/nature24303

\bibitem[{{Soker} {et~al.}(2013){Soker}, {Akashi}, {Gilkis}, {Hillel},
  {Papish}, {Refaelovich}, \&
  {Tsebrenko}}]{Soker_et_al__2013__AstronomischeNachrichten__Thejetfeedbackmechanism(JFM):Fromsupernovaetoclustersofgalaxies}
{Soker}, N., {Akashi}, M., {Gilkis}, A., {et~al.} 2013, Astronomische
  Nachrichten, 334, 402

\bibitem[{Soker \&
  Gilkis(2017)}]{Soker__2017__TheAstrophysicalJournal__MagnetarPoweredSuperluminousSupernovaeMustFirstBeExplodedbyJets}
Soker, N. \& Gilkis, A. 2017, The Astrophysical Journal, 851, 95, aDS Bibcode:
  2017ApJ...851...95S

\bibitem[Spite et al.(2005)]{Spite2005} Spite, M., Cayrel, R., Plez, B., et al., \aap, 430, 655 (2005). doi:10.1051/0004-6361:20041274

\bibitem[{{Spruit}(2002)}]{Spruit__2002__AA__Dynamo}
{Spruit}, H.~C. 2002, \aap, 381, 923

\bibitem[{Stockinger {et~al.}(2020)Stockinger, Janka, Kresse, Melson, Ertl,
  Gabler, Gessner, Wongwathanarat, Tolstov, Leung, Nomoto, \&
  Heger}]{Stockinger__2020__MonthlyNoticesoftheRoyalAstronomicalSociety__ThreeDimensionalModelsofCoreCollapseSupernovaefromLowMassProgenitorswithImplicationsforCrab}
Stockinger, G., Janka, H.~T., Kresse, D., {et~al.} 2020, Monthly Notices of the
  Royal Astronomical Society, 496, 2039, aDS Bibcode: 2020MNRAS.496.2039S

\bibitem[{Surman \&
  McLaughlin(2004)}]{Surman__2004__TheAstrophysicalJournal__NeutrinosandNucleosynthesisinGammaRayBurstAccretionDisks}
Surman, R. \& McLaughlin, G.~C. 2004, The Astrophysical Journal, 603, 611, aDS
  Bibcode: 2004ApJ...603..611S

\bibitem[{Surman {et~al.}(2006)Surman, McLaughlin, \&
  Hix}]{Surman__2006__TheAstrophysicalJournal__NucleosynthesisintheOutflowfromGammaRayBurstAccretionDisks}
Surman, R., McLaughlin, G.~C., \& Hix, W.~R. 2006, The Astrophysical Journal,
  643, 1057, aDS Bibcode: 2006ApJ...643.1057S

\bibitem[{Suwa \&
  Tominaga(2015)}]{Suwa__2015__MonthlyNoticesoftheRoyalAstronomicalSociety__HowMuchCan56NiBeSynthesizedbytheMagnetarModelforLongGammaRayBurstsandHypernovae}
Suwa, Y. \& Tominaga, N. 2015, Monthly Notices of the Royal Astronomical
  Society, 451, 282, aDS Bibcode: 2015MNRAS.451..282S

\bibitem[Suwa et al.(2019)]{Suwa2019} Suwa, Y., Tominaga, N., \& Maeda, K., \mnras, 483, 3607 (2019). doi:10.1093/mnras/sty3309

\bibitem[Suzuki et al.(2016)]{Suzuki2016} Suzuki, T., Toki, H., \& Nomoto, K., \apj, 817, 163 (2016). doi:10.3847/0004-637X/817/2/163

\bibitem[{{Takiwaki} \&
  {Kotake}(2011)}]{Takiwaki_Kotake__2011__apj__GravitationalWaveSignaturesofMagnetohydrodynamicallyDrivenCore-collapseSupernovaExplosions}
{Takiwaki}, T. \& {Kotake}, K. 2011, \apj, 743, 30

\bibitem[{{Takiwaki} {et~al.}(2009){Takiwaki}, {Kotake}, \&
  {Sato}}]{Takiwaki_Kotake_Sato__2009__apj__Special_Relativistic_Simulations_of_Magnetically_Dominated_Jets_in_Collapsing_Massive_Stars}
{Takiwaki}, T., {Kotake}, K., \& {Sato}, K. 2009, \apj, 691, 1360

\bibitem[Takiwaki et al.(2016)]{Takiwaki2016} Takiwaki, T., Kotake, K., \& Suwa, Y., \mnras, 461, L112 (2016). doi:10.1093/mnrasl/slw105


\bibitem[Tamborra et al.(2012)]{Tamborra2012} Tamborra, I., M{\"u}ller, B., H{\"u}depohl, L., et al., \prd, 86, 125031 (2012). doi:10.1103/PhysRevD.86.125031

\bibitem[{Tanvir {et~al.}(2017)Tanvir, Levan, \&  González-Fernández}]{Tanvir__2017__TheAstrophysicalJournal__TheEmergenceofaLanthanideRichKilonovaFollowingtheMergerofTwoNeutronStars}Tanvir, N.~R., Levan, A.~J., \& González-Fernández, C. e.~a. 2017, The Astrophysical Journal, 848, L27. doi:10.3847/2041-8213/aa90b6

%\bibitem[Tanvir et al.(2017)]{Tanvir2017} Tanvir, N.~R., Levan, A.~J., Gonz{\'a}lez-Fern{\'a}ndez, C., et al., \apjl, 848, L27 (2017). doi:10.3847/2041-8213/aa90b6

\bibitem[Thielemann et al.(1979)]{Thielemann1979} Thielemann, F.-K., Arnould, M., \& Hillebrandt, W., \aap, 74, 175 (1979)

\bibitem[{{Thielemann} {et~al.}(2018){Thielemann}, {Isern}, {Perego}, \& {von
  Ballmoos}}]{Thielemann_et_al__2018__ssr__NucleosynthesisinSupernovae}
{Thielemann}, F.-K., {Isern}, J., {Perego}, A., \& {von Ballmoos}, P. 2018,
  \ssr, 214, 62

\bibitem[{{Thompson} \& {Duncan}(1993)}]{Thompson_Duncan__1993__ApJ__NS-dynamo}
{Thompson}, C. \& {Duncan}, R.~C. 1993, \apj, 408, 194

\bibitem[{Thompson \&
  ud~Doula(2018)}]{Thompson__2018__MonthlyNoticesoftheRoyalAstronomicalSociety__HighEntropyEjectionsfromMagnetizedProtoNeutronStarWindsImplicationsforHeavyElementNucleosynthesis}
Thompson, T.~A. \& ud~Doula, A. 2018, Monthly Notices of the Royal Astronomical
  Society, 476, 5502, aDS Bibcode: 2018MNRAS.476.5502T

\bibitem[Timmes(1999)]{Timmes1999} Timmes, F.~X., \apjs, 124, 241 (1999). doi:10.1086/313257

\bibitem[Tinsley(1980)]{Tinsley1980} Tinsley, B.~M., \fcp, 5, 287 (1980). doi:10.48550/arXiv.2203.02041

\bibitem[Tolstoy et al.(2009)]{Tolstoy2009} Tolstoy, E., Hill, V., \& Tosi, M., \araa, 47, 371 (2009). doi:10.1146/annurev-astro-082708-101650

\bibitem[{Tominaga {et~al.}(2007)Tominaga, Maeda, Umeda, Nomoto, Tanaka, Iwamoto, Suzuki, \&  Mazzali}]{Tominaga__2007__TheAstrophysicalJournal__TheConnectionbetweenGammaRayBurstsandExtremelyMetalPoorStarsBlackHoleFormingSupernovaewithRelativisticJets}Tominaga, N., Maeda, K., Umeda, H., {et~al.} 2007, The Astrophysical Journal,  657, L77. doi:10.1086/513193

%\bibitem[Tominaga et al.(2007)]{Tominaga2007} Tominaga, N., Maeda, K., Umeda, H., et al., \apjl, 657, L77 (2007). doi:10.1086/513193

\bibitem[{{Tominaga}(2009)}]{Tominaga__2009__apj__Aspherical_Properties_of_Hydrodynamics_and_Nucleosynthesis_in_Jet-Induced_Supernovae}{Tominaga}, N. 2009, \apj, 690, 526. doi:10.1088/0004-637X/690/1/526

%\bibitem[Tominaga(2009)]{Tominaga2009} Tominaga, N., \apj, 690, 526 (2009). doi:10.1088/0004-637X/690/1/526

\bibitem[Townsley et al.(2016)]{Townsley2016} Townsley, D.~M., Miles, B.~J., Timmes, F.~X., et al., \apjs, 225, 3 (2016). doi:10.3847/0067-0049/225/1/3

\bibitem[Travaglio et al.(2004)]{Travaglio2004} Travaglio, C., Hillebrandt, W., Reinecke, M., et al., \aap, 425, 1029 (2004). doi:10.1051/0004-6361:20041108

\bibitem[Truran et al.(1978)]{Truran1978} Truran, J.~W., Cowan, J.~J., \& Cameron, A.~G.~W., \apjl, 222, L63 (1978). doi:10.1086/182693

\bibitem[{{Tsujimoto} \&  {Nishimura}(2015)}]{Tsujimoto_Nishimura__2015__apjl__TheRProcessinMagnetorotationalSupernovae}{Tsujimoto}, T. \& {Nishimura}, N. 2015, \apjl, 811, L10. doi:10.1093/pasj/psv035

\bibitem[Tsujimoto \& Nishimura(2018)]{Tsujimoto2018} Tsujimoto, T. \& Nishimura, N., \apjl, 863, L27 (2018). doi:10.3847/2041-8213/aad86b

%\bibitem[Tsujimoto et al.(2015)]{Tsujimoto2015} Tsujimoto, T., Ishigaki, M.~N., Shigeyama, T., et al., \pasj, 67, L3 (2015). doi:10.1093/pasj/psv035

\bibitem[Tsujimoto(2021)]{Tsujimoto2021} Tsujimoto, T., \apjl, 920, L32 (2021). doi:10.3847/2041-8213/ac2c75

\bibitem[Umeda \& Nomoto(2002)]{Umeda2002} Umeda, H. \& Nomoto, K., \apj, 565, 385 (2002). doi:10.1086/323946

\bibitem[{{Uzdensky} \&
  {MacFadyen}(2006)}]{Uzdensky__2006__apj__Stellar_Explosions_by_Magnetic_Towers}
{Uzdensky}, D.~A. \& {MacFadyen}, A.~I. 2006, \apj, 647, 1192

\bibitem[{{Uzdensky} \&
  {MacFadyen}(2007)}]{Uzdensky_MacFadyen__2007__apj__Magnetar-Driven_Magnetic_Tower_as_a_Model_for_Gamma-Ray_Bursts_and_Asymmetric_Supernovae}
{Uzdensky}, D.~A. \& {MacFadyen}, A.~I. 2007, \apj, 669, 546

\bibitem[{Varma \&
  Müller(2021)}]{Varma__2021__MonthlyNoticesoftheRoyalAstronomicalSociety__3DSimulationsofOxygenShellBurningwithandwithoutMagneticFields}
Varma, V. \& Müller, B. 2021, Monthly Notices of the Royal Astronomical
  Society, 504, 636

\bibitem[{Varma {et~al.}(2021)Varma, Müller, \&
  Obergaulinger}]{Varma__2021__MonthlyNoticesoftheRoyalAstronomicalSociety__AComparisonof2DMagnetohydrodynamicSupernovaSimulationswiththeCOCONUTFMTandAENUSALCARCodes}
Varma, V., Müller, B., \& Obergaulinger, M. 2021, Monthly Notices of the Royal
  Astronomical Society, 508, 6033, aDS Bibcode: 2021MNRAS.508.6033V

\bibitem[{Vlasov {et~al.}(2017)Vlasov, Metzger, Lippuner, Roberts, \&
  Thompson}]{Vlasov__2017__MonthlyNoticesoftheRoyalAstronomicalSociety__NeutrinoHeatedWindsfromMillisecondProtomagnetarsAsSourcesoftheWeakRProcess}
Vlasov, A.~D., Metzger, B.~D., Lippuner, J., Roberts, L.~F., \& Thompson, T.~A.
  2017, Monthly Notices of the Royal Astronomical Society, 468, 1522, aDS
  Bibcode: 2017MNRAS.468.1522V

\bibitem[{{Vlasov} {et~al.}(2014){Vlasov}, {Metzger}, \&
  {Thompson}}]{Vlasov_et_al__2014__mnras__Neutrino-heatedwindsfromrotatingprotomagnetars}
{Vlasov}, A.~D., {Metzger}, B.~D., \& {Thompson}, T.~A. 2014, \mnras, 444, 3537

\bibitem[Wanajo(2006)]{Wanajo2006} Wanajo, S., \apj, 647, 1323 (2006). doi:10.1086/505483


\bibitem[Wanajo et al.(2021)]{Wanajo2021} Wanajo, S., Hirai, Y., \& Prantzos, N., \mnras, 505, 5862 (2021). doi:10.1093/mnras/stab1655

\bibitem[Watson et al.(2019)]{Watson2019} Watson, D., Hansen, C.~J., Selsing, J., et al., \nat, 574, 497 (2019). doi:10.1038/s41586-019-1676-3

\bibitem[Weaver et al.(1978)]{Weaver1978} Weaver, T.~A., Zimmerman, G.~B., \& Woosley, S.~E., \apj, 225, 1021 (1978). doi:10.1086/156569

\bibitem[Wehmeyer et al.(2015)]{Wehmeyer2015} Wehmeyer, B., Pignatari, M., \& Thielemann, F.-K., \mnras, 452, 1970 (2015). doi:10.1093/mnras/stv1352

\bibitem[{White {et~al.}(2022)White, Burrows, Coleman, \&
  Vartanyan}]{White__2022__TheAstrophysicalJournal__OntheOriginofPulsarandMagnetarMagneticFields}
White, C.~J., Burrows, A., Coleman, M. S.~B., \& Vartanyan, D. 2022, The
  Astrophysical Journal, 926, 111, aDS Bibcode: 2022ApJ...926..111W

\bibitem[{{Winteler} {et~al.}(2012){Winteler}, {K{\"a}ppeli}, {Perego},  {Arcones}, {Vasset}, {Nishimura}, {Liebend{\"o}rfer}, \&  {Thielemann}}]{Winteler_et_al__2012__apjl__MagnetorotationallyDrivenSupernovaeastheOriginofEarlyGalaxyr-processElements}{Winteler}, C., {K{\"a}ppeli}, R., {Perego}, A., {et~al.} 2012, \apjl, 750, L22. doi:10.1088/2041-8205/750/1/L22

%\bibitem[Winteler et al.(2012)]{Winteler2012} Winteler, C., K{\"a}ppeli, R., Perego, A., et al., \apjl, 750, L22. (2012) doi:10.1088/2041-8205/750/1/L22

\bibitem[{{Wongwathanarat} {et~al.}(2015){Wongwathanarat}, {M{\"u}ller}, \&
  {Janka}}]{Wongwathanarat_et_al__2015__aap__Three-dimensionalsimulationsofcore-collapsesupernovae:fromshockrevivaltoshockbreakout}
{Wongwathanarat}, A., {M{\"u}ller}, E., \& {Janka}, H.-T. 2015, \aap, 577, A48

\bibitem[Woosley et al.(1973)]{Woosley1973} Woosley, S.~E., Arnett, W.~D., \& Clayton, D.~D., \apjs, 26, 231 (1973). doi:10.1086/190282

\bibitem[Woosley \& Howard(1978)]{Woosley1978} Woosley, S.~E. \& Howard, W.~M., \apjs, 36, 285(1978). doi:10.1086/190501

\bibitem[{Woosley(1993)}]{Woosley__1993__TheAstrophysicalJournal__GammaRayBurstsfromStellarMassAccretionDisksaroundBlackHoles}Woosley, S.~E. 1993, The Astrophysical Journal, 405, 273

%\bibitem[Woosley et al.(2002)]{Woosley2002} Woosley, S.~E., Heger, A., \& Weaver, T.~A., Reviews of Modern Physics, 74, 1015 (2002). doi:10.1103/RevModPhys.74.1015

\bibitem[{{Woosley} {et~al.}(2002){Woosley}, {Heger}, \&  {Weaver}}]{Woosley_Heger_Weaver__2002__ReviewsofModernPhysics__The_evolution_and_explosion_of_massive_stars}{Woosley}, S.~E., {Heger}, A., \& {Weaver}, T.~A. 2002, Reviews of Modern Physics, 74, 1015. doi:10.1103/RevModPhys.74.1015

\bibitem[{{Woosley} \&  {Bloom}(2006)}]{Woosley_Bloom__2006__araa__The_Supernova_Gamma-Ray_Burst_Connection}{Woosley}, S.~E. \& {Bloom}, J.~S. 2006, \araa, 44, 507

\bibitem[Yong et al.(2021)]{Yong2021} Yong, D., Kobayashi, C., Da Costa, G.~S., et al.\ 2021, \nat, 595, 223. doi:10.1038/s41586-021-03611-2

\bibitem[{Zenati {et~al.}(2020)Zenati, Siegel, Metzger, \&  Perets}]{Zenati__2020__MonthlyNoticesoftheRoyalAstronomicalSociety__NuclearBurninginCollapsarAccretionDiscs}Zenati, Y., Siegel, D.~M., Metzger, B.~D., \& Perets, H.~B. 2020, Monthly Notices of the Royal Astronomical Society, 499, 4097. doi:10.1093/mnras/staa3002

%\bibitem[Zenati et al.(2020)]{Zenati2020} Zenati, Y., Siegel, D.~M., Metzger, B.~D., et al., \mnras, 499, 4097 (2020). doi:10.1093/mnras/staa3002

\bibitem[Zevin et al.(2019)]{Zevin2019} Zevin, M., Kremer, K., Siegel, D.~M., et al., \apj, 886, 4 (2019). doi:10.3847/1538-4357/ab498b

\bibitem[Zevin et al.(2022)]{Zevin2022} Zevin, M., Nugent, A.~E., Adhikari, S., et al.\ (2022), arXiv:2206.02814

\bibitem[da Silveira et al.(2018)]{daSilveira2018} da Silveira, C.~R., Barbuy, B., Fria{\c{c}}a, A.~C.~S., et al., \aap, 614, A149 (2018). doi:10.1051/0004-6361/201730562

\bibitem[van de Voort et al.(2015)]{vandevoort2015} van de Voort, F., Quataert, E., Hopkins, P.~F., et al., \mnras, 447, 140 (2015). doi:10.1093/mnras/stu2404

\bibitem[van de Voort et al.(2020)]{vandevoort2020} van de Voort, F., Pakmor, R., Grand, R.~J.~J., et al., \mnras, 494, 4867 (2020). doi:10.1093/mnras/staa754


\end{thebibliography}


\end{document}
