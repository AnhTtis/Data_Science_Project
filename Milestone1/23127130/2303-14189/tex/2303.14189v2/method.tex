\section{Architecture} \label{sec:arch}

\begin{figure*}[t]
    \centering
    \includegraphics[width=0.97\linewidth]{images/fastvit_arch.pdf}
    \caption{(a) Overview of FastViT architecture which decouples train-time and inference-time architecture. Stages 1, 2, and 3 have the same architecture and uses RepMixer for token mixing. In stage 4, self attention layers are used for token mixing. (b) Architecture of the convolutional stem. (c) Architecture of convolutional-FFN (d) Overview of \textit{RepMixer} block, which reparameterizes a skip connection at inference. }
    \label{fig:arch}
    \vspace{-4mm}
\end{figure*}

\begin{figure}[t]
    \centering
    \includegraphics[width=1.0\linewidth]{images/benefits_of_reparam.pdf}
    \caption{  
    Latency comparison of a MetaFormer (S12) architecture with Pooling and RepMixer as a choice for token mixing; measured on iPhone 12 Pro for various image resolutions. Both models have $\sim$1.8G FLOPs. Absence of a skip connection in RepMixer lowers the overall memory access cost leading to lower latency. }
    \label{fig:benefits_reparam}
    \vspace{-0.5cm}
\end{figure}





\subsection{Overview} 
\label{sec:overview}
FastViT is a hybrid transformer and has four distinct stages which operate at different scales as shown in Figure~\ref{fig:arch}. We detail all the FastViT variants in Table~\ref{tab:model_arch}.


FastViT uses \textit{RepMixer}, a token mixer that reparameterizes a skip connection, which helps in alleviating memory access cost (see Figure~\ref{fig:arch}d).
To further improve efficiency and performance, we replace dense k$\times$k convolutions commonly found in stem and patch embedding layers with its factorized version that uses train-time overparameterization (see Figure~\ref{fig:arch}a). 


Self-attention~\cite{ViT} token mixers are computationally expensive, especially at higher resolutions~\cite{litv1, marin2021token}. While efficient versions of self-attention layers are explored in~\cite{Cmt, litv2}, we use large kernel convolutions as an efficient alternative to improve receptive field in early stages of the network architecture (see Figure~\ref{fig:arch}c). 

We analyze various architectural choices made in designing FastViT from a PoolFormer~\cite{metaformer} baseline in Table~\ref{tab:ablation_arch_choices} and detail our approach below.

\begin{table}
    \centering
    \scalebox{0.80}{
    \begin{tabular}{l@{\hspace{0.5\tabcolsep}}c@{\hspace{0.5\tabcolsep}}c@{\hspace{0.5\tabcolsep}}c@{\hspace{0.5\tabcolsep}}c@{\hspace{0.5\tabcolsep}}}
    \toprule
    \multirow{2}{*}{Architectural Choices} & Params & FLOPs & Mobile & Top-1 \\
     & (M) & (G) & Latency (ms) & (\%) \\ 
     \midrule
    \multicolumn{1}{l}{PoolFormer-S12 (Baseline)} & 11.9 & 1.8 & 1.50 & 77.2  \\
    \multicolumn{1}{l}{\quad + 224 $\rightarrow$ 256} & 11.9 & 2.4 & 2.25 & 77.6 \\
    \midrule
    Section \ref{sec:reparam_skip} \\
    \multicolumn{1}{l}{\quad + Pooling $\rightarrow$ RepMixer} & 11.9 & 2.4 & 1.58 & 78.5  \\
    \midrule
    Section \ref{sec:train_overparam} \\
    \multicolumn{1}{l}{\quad + Factorized dense conv.} & 8.7 & 1.7 & 1.26 & 78.0 \\
    \multicolumn{1}{l}{\quad + Train-Time Overparam.} & 8.7 & 1.7 & 1.26 & 78.9 \\
    \midrule
    Section \ref{sec:convnext_ffn} \\
    \multicolumn{1}{l}{\quad + LK. conv. FFN } & 8.7 & 1.8 & 1.33 & 79.4 \\
    \multicolumn{1}{l}{\quad + LK. conv. Patch Emb. } & 8.8 & 1.8 & 1.40 & 79.8 \\
    \bottomrule
    \end{tabular}
    
    }
        \caption{Analysis of architectural choices made to obtain FastViT-S12 variant, starting from PoolFormer-S12. ``LK." stands for Large Kernel.}
    \label{tab:ablation_arch_choices}
\end{table}

\begin{table}
\footnotesize
\centering
\setlength{\tabcolsep}{0.8pt}

\begin{figure*}
    \centering
    \subfloat[]{\includegraphics[width=.33\textwidth]{images/melisa_pretrain.png}\label{fig:melisa_pretrain}}
    \hfil
    \subfloat[]{\includegraphics[width=.33\textwidth]{images/finetune_cine_vs_samples.png}\label{fig:finetune_cine}}
    \hfil
    \subfloat[]{\includegraphics[width=.33\textwidth]{images/finetune_tass_vs_samples.png}\label{fig:finetune_tass}}
    
    \subfloat[]{\includegraphics[width=.33\textwidth]{images/melisa_pretrain_bert.png}\label{fig:melisa_pretrain_bert}}
    \hfil
    \subfloat[]{\includegraphics[width=.33\textwidth]{images/finetune_cine_vs_samples_bert.png}\label{fig:finetune_cine_bert}}
    \hfil
    \subfloat[]{\includegraphics[width=.33\textwidth]{images/finetune_tass_vs_samples_bert.png}\label{fig:finetune_tass_bert}}
    \caption{Training curves ($F_1$-score) for different amounts of training samples. Figures (a), (b) and (c) shows the fine-tuning results on the validation split for the biLSTM model and Figures (d), (e) and (f) for BERT. (a) and (d) are the validation scores on MeLiSA, (b) and (e) are the scores on the MuchoCine validation split and (c) and (f) the scores on the TASS validation.}
    \label{fig:finetunning}
\end{figure*}

\subsection{Classification Models}

As mentioned in Section \ref{sec:intro}, we explored the cross-domain analysis using two different neural-based classification models which are shown in Fig. \ref{fig:classification_models}.

\begin{itemize}
    \item \textbf{BiLSTM classifier}. This architecture is based on a bidirectional recurrent neural network with a LSTM unit activation and it is illustrated in Figure \ref{fig:blstm_classifier}. In this model, each word $w_t$ of the input sequence $w_1,\ldots,w_T$ is represented as a continuous vector $\mathbf{x}_t$ trough an embedding layer at the beginning of the network. This vector sequence is then forwarded to two different LSTM networks~\cite{lstm} (one forward and one backward). The outputs at the last step of these layers are then concatenated and forwarded to the linear output layer, which gives the probabilities of each class through a Softmax activation function. 
    \item \textbf{BERT classifier}. Since the emergence of the Transformer architecture~\cite{transformer}, a series of models based on self-attention mechanisms have been proposed to pre-train a language model. One of this models is the Bidirectional Encoder Representations from Transformers (BERT), illustrated in Figure \ref{fig:bert_classifier}, which consists of 12 identical transformer encoder layers. These layers contains a multi-head self-attention layer~\cite{transformer} at the input followed by a linear layer (feed forward) with some residual connections and layer normalization~\cite{layernorm} in between. As in the LSTM classifier, every word at the input of the network is represented as a continuous vector, but additionally some special tokens are added to the sentence. In particular, the \texttt{[CLS]} token is included at the start of every sentence and it is used to extract features of the entire sequence. That is, at the output of the encoder a sequence of the same length as the input is obtained but only the first vector is used as an input of the output layer, which returns the class probability.
\end{itemize}

The key difference in our analysis of these two models is that the biLSTM network was trained from scratch, whereas the BERT classifier used was pre-trained on a Language Modeling task. Specifically, the pre-trained model called BETO~\cite{beto}, which is trained on a 3 billion words corpus called Spanish Unnanotated Corpora\footnote{https://github.com/josecannete/spanish-corpora} (SUC) was used. %The SUC database is a collection of unnanotated documents, most of them extracted from the Spanish portions of the Open Parallel Corpus (OPUS) subcorpora. The OPUS project is intended to provide the community with a publicly available parallel corpus of free online data. 
Documents included in the SUC contains all the data from Spanish Wikipedia available at the time the corpus was released and all of the sources of the OPUS Project~\cite{opus} that had text in Spanish. This sources
includes United Nations and Government journals, TED Talks, Subtitles, News Stories and more. However, none of these sources are review-like documents. The total size of the corpora gathered was comparable with the corpora used in the original BERT.
% @misc{cardelino,
%     title = "Spanish Billion Words Corpus and Embeddings",
%     author = "Cristian Cardellino",
%     year = "2016",
%     month = "March",
%     URL = "https: //crscardellino.github.io/SBWCE/"
% }

\begin{table*}
    \centering
    \caption{Test results ($F_1$-score) in the fine-tuning configuration.}
    \label{tab:melisa_finetunning}
    \begin{tabular}{r|cc|cc|cc}
        & \multicolumn{2}{c|}{Amazon} & \multicolumn{2}{c|}{TASS} & \multicolumn{2}{c}{MuchoCine} \\
        (\%) & biLSTM & BERT & biLSTM & BERT & biLSTM & BERT \\\hline
        0.0 & 0.5594 & 0.5860 & 0.3470 & 0.3850 & 0.2441 & 0.2758 \\
        0.1 & 0.5643(+0.88 \%) & 0.5860(+0.00 \%) & 0.3557(+2.51 \%) & 0.4104(+6.60 \%) & 0.1868(-23.47 \%) & 0.2834(+2.76 \%) \\
        10.0 & 0.5654(+1.07 \%) & 0.5865(+0.09 \%) & 0.3761(+8.39 \%) & 0.3974(+3.22 \%) & 0.2255(-7.62 \%) & 0.3210(+16.39 \%) \\
        100.0 & 0.5662(+1.22 \%) & 0.5845(-0.26 \%) & 0.3503(+0.96 \%) & 0.4045(+5.06 \%) & 0.2616(+7.17 \%) & 0.3125(+13.31 \%) \\
    \end{tabular}
\end{table*}

\subsection{Experimental Set-Up}

The above mentioned models were used to perform CDSC using the fine-tuning and the zero-shot configurations. For the fine-tuning case, the following steps were applied:
\begin{enumerate}
    \item The model was trained using the train split of the source dataset (MeLiSA, for instance) and hyperparameter search was done with its corresponding validation portion.
    \item Once trained, the model was trained again using the training portion of the target dataset (MuchoCine, for instance), and a new hyperparameter search was done with the target validation split.
    \item Once retrained, the model was evaluated on the test split of the target dataset.
\end{enumerate}

The only difference between both configurations is that step 2 is omitted for the zero-shot learning. This means that evaluation on the target dataset was done without using any training sample of that dataset. As a consequence, zero-shot learning is usually more challenging than fine-tuning and tends to show lower performance. However, it can provide a better idea of the model's generalization capability.

In order to test if CDSC can be achieved from product domains to more general domains like movie reviews or tweets, we used our MeLiSA dataset as source domain and MuchoCine and TASS as target domains. We also used the Amazon dataset as target domain to keep track of the model's learning capability, although this domain is, in principle, be very similar to the source domain. 

Experiments were carried on in \texttt{Python}, and the \texttt{Pytorch} module was used to implement the model training algorithm. We also used the \texttt{Huggingface Transformers} library to load the Spanish BERT pre-trained parameters and a NVIDA GTX 1080 GPU to reduce time computation. We followed \cite{random_grid_hyperparms} to perform random grid sample hyperparameter search on the biLSTM model. The best biLSTM model found consisted in a two-layer LSTM cell with hidden dimension of 50 and an embedding matrix of dimension $60,\!000\times 300$. Dropout was used as a regularization technique with a probability of $0.1$ and Adam Optimization with a batch size of 16 and a learning rate of 1e-3 was found to give the best validation results. For the pre-trained BERT model, layer dimensions are fixed in advance (12 layers with inner dimension of 768). We also used Adam Optimization to train this model and fixed the learning rate and the batch size to 5e-5 and 16 respectively, as suggested in \cite{beto}.




\caption{\textbf{Architecture details of FastViT variants.} Models with smaller embedding dimensions, i.e. [64, 128, 256, 512] are prefixed with ``S" and models that contain Self-Attention layers are prefixed with ``SA". Models with bigger embedding dimensions, i.e. [76, 152, 304, 608] are prefixed with ``M". Models with MLP expansion ratio less than 4 are prefixed with ``T". The number in the notation denotes total number of FastViT blocks. FLOP count is computed by \texttt{fvcore}\cite{fvcore} library.
}
\label{tab:model_arch}
\vspace{-4mm}
\end{table}


\subsection{FastViT}\label{sec:FastViT_arch}

\subsubsection{Reparameterizing Skip Connections}\label{sec:reparam_skip}

\paragraph{RepMixer} 
Convolutional mixing was first introduced in ConvMixer\cite{trockman2022convmixer}. For an input tensor $X$, the mixing block in the layer was implemented as,
\begin{equation}
Y = \texttt{BN(}\sigma\texttt{(DWConv(}X\texttt{)))} + X
\end{equation}
where $\sigma$ is a non-linear activation function and \texttt{BN} is Batch Normalization~\cite{Batch_Norm} layer and \texttt{DWConv} is depthwise convolutional layer. While this block was shown to be effective, in \textit{RepMixer}, we simply rearrange the operations and remove the non-linear activation function as shown below,
\begin{equation}
Y = \texttt{DWConv(BN(}X\texttt{)}+ X
\end{equation}\label{equation:repmix_train}
The main benefit of our design is that it can be reparameterized at inference time to a single depthwise convolutional layer as shown below and in Figure~\ref{fig:arch}d.
\begin{equation}
Y = \texttt{DWConv(}X\texttt{)} 
\end{equation}\label{equation:repmix_infer}

\vspace{-5mm}
\paragraph{Positional Encodings} 
We use conditional positional encodings~\cite{Twins, CPE} that is dynamically generated and conditioned on the local neighborhood of the input tokens. These encodings are generated as a result of a depth-wise convolution operator and are added to the patch embeddings. Note the lack of non-linearities in this group of operations, hence this block is reparameterized as shown in Figure~\ref{fig:arch}a.

\paragraph{Empirical Analysis} In order to verify the benefits of reparameterizing skip connections, we ablate over the using one of the most efficient (in terms of latency) token mixer, i.e. Pooling and RepMixer in a MetaFormer S12 architecture. Both the models being ablated have $\sim$1.8G FLOPs. We time the models for various input resolutions starting from 224$\times$224 to 1024$\times$1024 on an iPhone 12 Pro mobile device. From Figure~\ref{fig:benefits_reparam}, we see that RepMixer significantly improves over Pooling, especially at higher resolutions. At 384$\times$384, using RepMixer will lower the latency by 25.1\% and at larger resolutions like 1024$\times$1024, latency is lowered significantly by 43.9\%. 


\vspace{-3.5mm}
\subsubsection{Linear Train-time Overparameterization}\label{sec:train_overparam}

In order to further improve efficiency (parameter count, FLOPs, and latency), we replace all dense k$\times$k convolutions with its factorized version, i.e. k$\times$k depthwise followed by 1$\times$1 pointwise convolutions. However, the lower parameter count from factorization can diminish the capacity of the model. In order to increase capacity of the factorized layers, we perform linear train-time overparameterization as described in MobileOne~\cite{mobileone}.
MobileOne-style overparameterization in stem, patch embedding, and projection layers help in boosting performance. From Table~\ref{tab:ttop_ablation}, we note that train-time overparameterization improves Top-1 accuracy on ImageNet by 0.6\% on FastViT-SA12 model. On a smaller FastViT-S12 variant, Top-1 accuracy improves by 0.9\% as shown in Table~\ref{tab:ablation_arch_choices}.

However, train-time overparameterization results in increased training time due to computational overhead from the added branches. In our architecture, we only overparameterize those layers that replace dense k$\times$k convolutions with its factorized form as described above. These layers are found in the convolutional stem, patch embedding and projection layers. The computational cost incurred in these layers are lower than the rest of the network, hence overparameterizing these layers do not result in significant increases to train time. For example, FastViT-SA12 takes 6.7\% longer and FastViT-SA36 takes 4.4\% longer to train with train-time overparameterization as opposed to training those variants without it under the same settings described in Section~\ref{sec:imagenet_exps}.

\vspace{-3.5mm}
\subsubsection{Large Kernel Convolutions}\label{sec:convnext_ffn}
The receptive field of RepMixer is local compared to self-attention token mixers. However, self-attention based token mixers are computationally expensive. A computationally efficient approach to improve the receptive field of early stages that do not use self-attention is by incorporating depthwise large kernel convolutions. We introduce depthwise large kernel convolutions in FFN and patch embedding layers. 
From Table~\ref{tab:large_kernel_ablations}, we note that variants using depthwise large kernel convolutions can be highly competitive to variants using self-attention layers while incurring a modest increase in latency. When we compare V5 with V3, model size increases by 11.2\%, and latency increases by a factor of 2.3$\times$ for a relatively small gain of 0.4\% in Top-1 accuracy. V2 is larger than V4 by 20\% and has 7.1\% higher latency than V4 while attaining similar Top-1 accuracy on ImageNet. Further ablations on kernel sizes and latency is provided in the supplementary materials. In Table~\ref{tab:ablation_arch_choices}, we ablate over large kernel convolutions in FFN and patch embedding layers. Overall, large kernel convolutions provide 0.9\% improvement in Top-1 accuracy on FastViT-S12. 

The architecture of our FFN and patch embedding layer is shown in Figure~\ref{fig:arch}. Our FFN block has a structure similar to ConvNeXt~\cite{ConvNext} block with a few key differences, see Figure~\ref{fig:arch}c. We use Batch Normalization as opposed to Layer Normalization, as it can be fused with the preceding layer at inference. Also, it does not require additional reshape operations to obtain appropriate tensor layout for LayerNorm as done in the original implementation of ConvNeXt block. 

Along with increased receptive field, large kernel convolutions help in improving model robustness as observed in~\cite{wang2022robustcnn} and convolutional-FFN blocks generally tend to be more robust than vanilla-FFN blocks as observed in~\cite{mao2022robust}. Hence, incorporating large kernel convolutions is an efficient way to improve model performance and robustness. 

\begin{table}
    \centering
    \scalebox{0.68}{
    \begin{tabular}{l|c|c|c}
    \toprule
    Model   & Ablation  & Top-1 (\%) & Train Time (hrs)  \\ 
    \midrule
    \multirow{2}{*}{FastViT-SA12} & Without Train-Time Overparam. & 80.0 & 31.3 \\
                                & With Train-Time Overparam. & \textbf{80.6} & 33.4 \\
    \midrule                                
    \multirow{2}{*}{FastViT-SA36} & Without Train-Time Overparam. & 83.3 & 73.5 \\
                                & With Train-Time Overparam. & \textbf{83.6} & 76.7 \\                                
    \bottomrule
    \end{tabular}
    
    }
        \caption{Comparison of FastViT variants with and without linear train-time overparameterization when trained on ImageNet-1k dataset. Train time is wall clock time elapsed at the end of a training run.}
    \label{tab:ttop_ablation}
    \vspace{-0.4cm}
\end{table}

\begin{table}
    \centering
    \scalebox{0.78}{
    \begin{tabular}{c|c|c|c|c|c|c|c}
    \toprule
    \multirow{2}{*}{Variant} & \multicolumn{4}{c|}{Stages}  & Params & Top-1 & Mobile \\
    \cmidrule{2-5}
    & 1 & 2 & 3 & 4 & (M) & (\%) & Latency (ms) \\ 
    \midrule
    \multicolumn{8}{c}{Standard Setting} \\
    \midrule
    V1 & RM & RM & RM & RM & 8.7  & 78.9 & 1.3 \\
    V2 & RM & RM & RM & SA & 10.8 & 79.9 & 1.5 \\
    V3 & RM & RM & SA & SA & 12.4 & 81.0 & 3.7 \\
    \midrule
    \multicolumn{8}{c}{Large Kernel Convolutions (7$\times$7)} \\
    \midrule
    V4 & RM & RM & RM & RM & 8.8  & 79.8  & 1.4 \\
    V5 & RM & RM & RM & SA & 10.9 & 80.6  & 1.6 \\
    \bottomrule
    \end{tabular}
    
    }
        \caption{Ablation on using large kernel convolutions as a substitute for self-attention layers. ``RM" indicates [RepMixer-FFN] block is used in the stage. ``SA" indicates [Self Attention-FFN] block is used in the stage. Standard setting uses 3x3 factorized convolutions in patch embedding and stem layers and 1$\times$1 convolutions for FFN. In variants V4 and V5, large kernel convolutions (7$\times$7) are used in patch embedding and FFN layers.}
    \label{tab:large_kernel_ablations}
\end{table}