% This is samplepaper.tex, a sample chapter demonstrating the
% LLNCS macro package for Springer Computer Science proceedings;
% Version 2.21 of 2022/01/12
%
\documentclass[runningheads]{llncs}
%
\usepackage{amsmath}
\usepackage{amssymb}
\usepackage[applemac]{inputenc}
\usepackage[colorlinks]{hyperref}
\usepackage{url}
\usepackage{graphicx,epstopdf}
\usepackage{subcaption}
 \usepackage{epstopdf}
\usepackage[applemac]{inputenc}
\usepackage[colorlinks]{hyperref}
\usepackage{url}
\usepackage{subcaption}
%\renewcommand{\Pr}{\field{P}}

\usepackage{MnSymbol}
\usepackage{comment}

\DeclareMathOperator{\Tr}{Tr}
\newcommand{\bA}{\boldsymbol{A}}
\newcommand{\ba}{\boldsymbol{a}}
\newcommand{\bx}{\boldsymbol{x}}
\newcommand{\bxi}{\boldsymbol{\xi}}
\newcommand{\bc}{\boldsymbol{c}}
\newcommand{\bC}{\boldsymbol{C}}
\newcommand{\bq}{\boldsymbol{q}}
\newcommand{\bd}{\boldsymbol{d}}
\newcommand{\bX}{\boldsymbol{X}}
\newcommand{\bM}{\boldsymbol{M}}
\newcommand{\bone}{\boldsymbol{1}}
\newcommand{\bI}{\boldsymbol{I}}
\newcommand{\bu}{\boldsymbol{u}}
\newcommand{\bb}{\boldsymbol{b}}
\newcommand{\by}{\boldsymbol{y}}
\newcommand{\bY}{\boldsymbol{Y}}
\newcommand{\bhatY}{\boldsymbol{\hat{Y}}}
\newcommand{\bbary}{\boldsymbol{\bar{y}}}
\newcommand{\bg}{\boldsymbol{g}}
\newcommand{\bz}{\boldsymbol{z}}
\newcommand{\bZ}{\boldsymbol{Z}}
\newcommand{\bbarZ}{\boldsymbol{\bar{Z}}}
\newcommand{\bbarz}{\boldsymbol{\bar{z}}}
\newcommand{\bhatZ}{\boldsymbol{\hat{Z}}}
\newcommand{\bhatz}{\boldsymbol{\hat{z}}}
\newcommand{\bhatx}{\boldsymbol{\hat{x}}}
\newcommand{\haty}{\hat{y}}
\newcommand{\barz}{\bar{z}}
\newcommand{\bS}{\boldsymbol{S}}
\newcommand{\bbarS}{\boldsymbol{\bar{S}}}
\newcommand{\bw}{\boldsymbol{w}}
\newcommand{\bhatw}{\hat{\boldsymbol{w}}}
\newcommand{\bW}{\boldsymbol{W}}
\newcommand{\bU}{\boldsymbol{U}}
\newcommand{\bv}{\boldsymbol{v}}
\newcommand{\bzero}{\boldsymbol{0}}
\newcommand{\balpha}{\boldsymbol{\alpha}}
\newcommand{\sA}{\mathcal{A}}
\newcommand{\sC}{\mathcal{C}}
\newcommand{\sD}{\mathcal{D}}
\newcommand{\sX}{\mathcal{X}}
\newcommand{\sY}{\mathcal{Y}}
\newcommand{\sS}{\mathcal{S}}
\newcommand{\sT}{\mathcal{T}}
\newcommand{\sZ}{\mathcal{Z}}
\newcommand{\sL}{\mathcal{L}}
\newcommand{\sI}{\mathcal{I}}
\newcommand{\bbO}{\mathbb{O}}
\newcommand{\sbarZ}{\bar{\mathcal{Z}}}
\newcommand{\fbag}{\bold{F}}

\usepackage{hyperref}
%
\usepackage[T1]{fontenc}
% T1 fonts will be used to generate the final print and online PDFs,
% so please use T1 fonts in your manuscript whenever possible.
% Other font encondings may result in incorrect characters.
%
\usepackage{graphicx}
% Used for displaying a sample figure. If possible, figure files should
% be included in EPS format.
%
% If you use the hyperref package, please uncomment the following two lines
% to display URLs in blue roman font according to Springer's eBook style:
%\usepackage{color}
%\renewcommand\UrlFont{\color{blue}\rmfamily}
%
\begin{document}
%
\title{On canonical parameterizations of $2D$-shapes\thanks{Supported by FWF grant I 5015-N, Institut CNRS Pauli, Vienna, Austria, and University of Lille, France}}

%
%\titlerunning{Abbreviated paper title}
% If the paper title is too long for the running head, you can set
% an abbreviated paper title here
%
\author{Alice~Barbora~Tumpach\inst{2, 3}\orcidID{0000-0002-7771-6758}}
%
\authorrunning{Tumpach}
% First names are abbreviated in the running head.
% If there are more than two authors, 'et al.' is used.
%
\institute{
Institut CNRS Pauli, Oskar-Morgenstern-Platz 1, 1090 Vienna, Austria \and
University of Lille, Cit\'e scientifique, 59650 Villeneuve d'Ascq, France
\email{alice-barbora.tumpach@univ-lille.fr}\\
\url{http://math.univ-lille1.fr/~tumpach/Site/home.html}}

\maketitle              % typeset the header of the contribution
%
\begin{abstract}
This paper is devoted to the study of unparameterized simple curves in the plane. 
We propose diverse canonical parameterizations of a 2D-curve. For instance, the arc-length parameterization is canonical, but we consider other natural parameterizations like the parameterization proportionnal to the curvature of the curve. Both aforementionned parameterizations are very natural and correspond to a natural physical movement : the arc-length parameterization corresponds to travelling along the curve at constant speed, whereas parameterization proportionnal to curvature corresponds to a constant-speed moving frame. Since the curvature function of a curve is a geometric invariant of the unparameterized curve, a parameterization using the curvature function is a canonical parameterization.  The main idea is that to any physically meaningful stricktly increasing function is associated a natural parameterization of 2D-curves, which gives an optimal sampling, and which can be used to compare unparameterized curves in a efficient and pertinent way. An application to point correspondance in medical imaging is given.



\keywords{Canonical parameterization  \and Geometric Green Learning \and shape space.}
\end{abstract}
%
%
%
\section{Introduction}


%\subsection{Motivation}
Curves in $\mathbb{R}^2$  appear in many applications: in shape recognition as outline of an object, in radar detection as the signature of a signal, as trajectories of cars etc... There are two main features of the curve : the route and the speed profil. In this paper, we are only interested in the route drawn by the curve and we will called it the unparameterized curve. An unparameterized curve can be parameterized in multiple ways, and the choosen parameterization selects the speed at which the curve is traversed. Hence a curve can be travelled with many different speed profils, like a car can travel with different speeds (not necessarily constant) along a given road. The choice of a speed profil is called a parameterization of the curve. It may be physically meaningful or not. For instance, depending on applications, there may not be any relevant parameterization of the contour of the statue of Liberty depicted in Fig.~\ref{Liberty3}. In this paper, we propose various very natural parameterization of 2D-curves.
%The set of parameterized curves, called the A1CL Desert Heel 9 cm in 37 EU/ US: 6,5 comfortable size space (\cite{TumPre}), has a natural fiber bundle structure : one can group parameterized curves together when they follow the same route. The set of unparameterized curves is called the shape space. An application which, to any unparameterized curve associates a parameterization of it, is called a section. % of the fiber bundle of parameterized curves over the space of unparameterized curves (or routes). 
%Sections of fiber bundles are not linear spaces, but, in the case of the fiber bundle of parameterized curves, there is a linear space associated to any given section~: the linear space of (generalized) curvature functions. It is a complete set of geometric invariants or descriptors of the unparameterized curves. %The choice of the section used to express these geometric invariants is analoguous to the choice of a basis of a vector space used to express the coordinates of a vector. 
They are based on the curvature, which together with the arc-length measure form a  complete set of geometric invariants or descriptors of the unparameterized curves. 
%This means in particular that given a curvature function keeping in mind a given section of the preshape space, one can construct uniquely a curve with this prescribed curvature function and with the parameterization in the chosen section. The aim of this paper is to provide optimal parameterizations  or optimal sampling of curves with a fix number of points.

%\framebox{Optimal sampling}

%\subsection{Past work}
%
%%\todo{Add Mansfield, and Qmap, Peter Olver, Griffiths, Fokas-Gelfand}
%In this section, we look at the geometry of the space of 2D-curves in the plane, and give an overview of past work on the analysis of contours, comprising comparison and interpolation tasks.
%
%
%When we want to interpolate between two 2D-contours like the two red contours of a ballerina depicted in Figure~\ref{Ballet}, there are a couple of things to keep in mind. 
%
%
%First, one should not take two parameterizations of the contours at random and interpolate linearly between them. This gives usually very bad results, even when one starts the parameterization at points that should correspond. In Figure~\ref{Ballet},  we start the parameterization at the top of the head of the ballerina, travel the contour counterclockwise with a speed profil that is illustrated by the sampling of the curves : on portions of the curve where points accumulate the speed is small, whereas on portions of the curve where points are very far apart, the curve is travelled at high speed  (in order to travel between two successive points, the same amount of time is needed). The resulting interpolation is depicted on the first line of Fig.~\ref{Ballet}. One sees that this interpolation procedure does not give good results.%the picture.
%
%
% \begin{figure}[!ht]
% 		\centering
%		%\makebox[\textwidth]{\framebox[0cm]{\rule{0pt}{3cm}}}
% 		%\includegraphics[width = 8cm]{Liberty_with_235points_and_reconstruction.pdf}%{Tumpach_Greenwich_cropped.pdf}%
%		%\includegraphics[width = 16cm]{Ballerinas_cropped2.pdf}%{Tumpach_Greenwich_cropped.pdf}%
%		\includegraphics[width = 9.5cm]{Linear_ballerinas4_best}
%		%\includegraphics[width = 9cm]{Linear1}
%		
%		%\includegraphics[width = 9cm]{Linear2}
%		
%		%\includegraphics[width = 9cm]{Linear3}
%		
%\includegraphics[height = 2.25cm]{pose_initiale2} 
%\includegraphics[height = 2.25cm]{pose_inter12} 
%\includegraphics[height = 2.25cm]{pose_inter2} 
%\includegraphics[height = 2.25cm]{pose_finale2} 
%
% 		\caption{\scriptsize First line : linear interpolation between some parameterized ballerinas,  second line :  linear interpolation between arc-length parameterized ballerinas, third line : linear interpolation between two registered ballerinas using \cite{Novel}, and \cite{Shape}, fourth line: linear interpolation between two registered ballerinas using a function of arc-length and curvature, fifth line : reference movement taken from \cite{Sugano}.}
% 		\label{Ballet}		
%\end{figure}
%
% 
%
%Second, picking up a preferred parameterization of the curves, for instance the arc-length parameterization, and interpolating linearly between the parameterized curves may also lead to bad results. In the second row of Figure~\ref{Ballet}, the two ballerinas are parameterized proportionally to arc-length hence the resulting samplings are uniform. The linear interpolation between the two contours parameterized proportional to arc-length is depicted at the second row in Figure~\ref{Ballet}. One can see that the deformation shrinks the moving leg and therefor appears unnatural.
%%is not natural, mainly because the length of the two contours are different  due to the extension of the leg and the tips of the toe are not in correspondance. 
%However, in some applications, where the routes to compare are very similar, %like the routes of the body of a caterpillar in Figure~\ref{chenilles}, 
%the result may be satisfactory and no fancy shape analysis is needed.
%
%
%One can distinguish two tasks in the comparison of curves~:
%\begin{enumerate}
%\item  the registration or correspondance, which consists in choosing parameterizations of two curves so that features of the curves that should correspond are associated to the same value of the parameter, 
%\item the measurement of the discrepancy between the two curves and the generation of a deformation of one curve into the other. 
%\end{enumerate}
%The recent use of differential geometry in shape analysis has allowed to take on these two tasks in the same framework. A traditional strategy to generate deformations between unparameterized curves is the following~: the  space of parameterized curves is endowed with a parameterization-equivariant Riemannian metric which allows to compute preferred deformations between curves, called geodesics, which are minimal for the corresponding variational problem. Then, given two unparameterized curves, one chooses the parameterization of one curve and, to each parameterization of the second curve, one compute the geodesic (if it exists!) between the two parameterized curves. At last, one has to solve an optimization problem consisting in singeling out the parameterization of the second curve (if it exists!) that achieve the infimum of the cost function among all possible parameterizations of the second curve. The geodesic between the two unparameterized curves is then given by the geodesic between the first curve (with its arbitrarily choosen parameterization) and the second curve with the parameterization minimizing the cost function. The discrepancy between the two curves is measured as the length of this geodesic. This procedure endow the shape space with a Riemannian structure called the quotient Riemannian metric. One can mention the following problems encountered when one pursue this strategy~:
%\begin{itemize}
%\item The choice of a Riemannian metric on the space of parameterized curves is usually not easy. As was first highlighted in \cite{MM}, a badly chosen Riemannian metric can lead to 
%vanishing geodesic distance, ruling out any effort to use geodesic distance to measure discrepancy between curves. For this reason, a large mathematical literature developped in order to propose Riemannian metrics with good mathematical properties: mention Sobolev metrics in \cite{BHM4}, curvature weighted metrics in \cite{BHM3}, almost local metrics in \cite{BHM5},
%metrics mesuring the deformations of the interiors of shapes in \cite{FJSY09}. 
%\item The geodesic between two parameterized curves with respect to a given Riemannian metric are usually hard to compute, 
%and one has to use algorithms like the path-straightening method or the shooting method to approximate them (\cite{Drira}). These algorithms are time-consuming. To overcome this difficulty and speed up the comparison of curves, some metrics have been proposed where the geodesic on the space of parameterized curves are explicit, like in \cite{Younes98}, 
%\cite{Younes2008}, or \cite{Novel} and \cite{Shape}. The framework of \cite{Shape}  has recently been adapted to general manifolds in \cite{Alice} and homogeneous spaces in \cite{Elise}.
%\item The optimization problem over all parameterizations of a given curve raise mathematical as well as practical difficulties~: first the set of all parameterizations of a given curve is an orbit of  an infinite-dimensional Fr\'echet Lie group, the group of diffeomorphisms, with a lot of pathological properties. There is in general no guarantee that this mathematical problem can be solved. Second, the algorithms used to approximate the solution of this optimization problem are based on dynamical programming (see for instance \cite{MioAnuj}) with the drawback that in practise only a finite number of reparameterizations distributed mainly around the identity map are considered. For this reason, a gauge invariant framework has been proposed in \cite{TPAMI} (see also \cite{Notices}) in the context of shape analysis of surfaces, where this optimization step is avoided by the use of a Riemannian metric which degenerates along the orbit of the reparameterization group. Another idea to avoid this minimization problem was proposed in \cite{TumPres}, where the quotient elastic metric introduced in \cite{Shape} is expressed as a metric on the section of arc-length parameterized curves. Nevertheless, since the geodesics on shape space are not explicit in any of the previously mentionned works, shape comparison  is not really efficient.
%\item In \cite{LRK}, the optimization step is solved for piecewise linear curves (polygons) under the elastic metric of \cite{Srivastava2011b}:  the precise matching minimizing the geodesic distance is given between two piecewise linear curves.  The only lack in this work is that it relies on the Euclidean geometry of $\mathbb{R}^n$ and  may not be adapted to general manifolds or homogeneous spaces.
%\end{itemize}

\section{Different parameterizations of 2D-shapes}\label{interpolate}

\subsection{Arc-length parameterization and Signed curvature}

 By 2D-shape, we mean the shape drawn by a parameterized curve in the plane. It is the ordered set of points visited by the curve. The shapes of two curves are identical if one can reparameterize one curve into the other (using a continuous increasing function). Any rectifiable planar curve admits a canonical parameterization, its \textit{arc-length parameterization}, which draws the same shape, but with constant unit speed. The set of 2D-shapes can be therefore identified with the set of arc-length parameterized curves, which is not a vector subspace, but rather an infinite-dimensional submanifold of the space of parameterized curves (see \cite{TumPres}). 
 
\vspace{-1cm} 
 \begin{figure}[!ht]
 		\centering
		\includegraphics[width = 7cm]{three_Liberty.pdf}
 		\caption{\scriptsize{The statue of Liberty (left), a uniform resampling using Matlab function spline (middle), a reconstruction of the statue using its discrete curvature (right).}
		}
 		\label{Liberty3}		
\end{figure}




It may be difficult to compute an explicit formula of the arc-length parameterization of a given rectifiable curve. Fortunately, when working with a computer, one do not need it. One neither need a concrete parameterization of the curve to depict it, a sample of points on the curve suffises. To draw the statue of Liberty as in Fig.~\ref{Liberty3} left, one just need a finite ordered set of points (the red stars). The discrete version of an arc-length parameterized curve is a uniformly sampled curve, i.e. an ordered set of equally distant points (for the euclidean metric). Resampling a curve uniformly is immediate using some appropriate interpolation function like the matlab function \textit{spline} (the second picture in Fig.~\ref{Liberty3} shows a uniform resampling of the statue of liberty). 
 





Consider the set of 2D  simple  closed curves, such as the contour of Elie Cartan's head in Fig.~\ref{Cartans_head}. After the choice of a starting point and a direction, there is a unique way to travel the curve at unit speed. In Fig.~\ref{Cartans_head}, we have drawn the velocity vector near the glasses of  Elie Cartan, as well as the unit normal vector which is obtained from the unit tangent vector by a rotation of $+\frac{\pi}{2}$. These two vectors form an orthonornal basis, i.e. an element (modulo the choice of a basis of $\mathbb{R}^2$) of the Lie group $\textrm{SO}(2)$, which is characterized by a rotation angle. The rate of variation of this rotation angle is called the signed curvature of the curve. For instance, when moving along the external outline of the glasses, this curvature equals the inverse of the radius of the glasses. We have depicted the curvature function $\kappa$ of Elie Cartan's head in Fig.~\ref{Cartan_curvature}, first line, when the parameter $s\in[0; 1]$ on the horizontal axis is proportional to arc-length, and such that the entire contour of Elie Cartan's head is travelled when the parameter reaches 1. Its corresponds to a uniform sampling of the contour. The curvature function is also depicted when parameterized by two other canonical parameters, namely by the curvature-length parameter (second line) and the curvarc-length parameter (third line). 


 \begin{figure}[!ht]
 		\centering
		
 		\includegraphics[width = 8cm]{Elie_Cartan_profil_sequence3}
		
 		\caption{\scriptsize Elie Cartan and the moving frame associated to the contour of  his head.}
 		\label{Cartans_head}	
\end{figure}

%There is a little difference in the construction of the moving frame for 2D curves in comparison to the moving frame for 3D curves. Indeed in the 2D case, we don't need the second and third derivative of the curve to construct the frame. Just the first derivative is enough. In fact we are using the knowledge that the curve stays in the plane to construct the normal at each point of the curve. In other words, we are using additional geometric properties of the ambient space (in this case the complex structure of the plane). The consequence of this is that the moving frame can be defined even for 2D curves with flat pieces (zero curvature sections) like the statue of Liberty which has a long flat piece at its base (see  Figure~\ref{Liberty3}). The corresponding curvature is therefore signed, with positive sign when the moving frame is turning clockwise, negative sign when the moving frame is turning counterclockwise, and zero along flat pieces. 




\vspace{-0.5cm}
 \begin{figure}[!ht]
 		\centering
		
		\includegraphics[width = 0.8\textwidth]{Elie_Cartan_curvature_arclength_L20}\\
		\includegraphics[width = 0.8\textwidth]{Elie_Cartan_Curvature_Curvature_Length_L20}\\
		\includegraphics[width = 0.8\textwidth]{Elie_Cartan_curvature_curv_arc_length_L20}
		 		\caption{\scriptsize Signed curvature of Elie Cartan's head for the  parameterization proportional to arc-length (first line), proportional to the curvature-length (second line), and proportional to the curvarc length (third line).}
 		\label{Cartan_curvature}		
\end{figure}
\vspace{-0.2cm}


 
A discrete version of an arc-length parameterized curve is an equilateral polygon. To draw an equilateral polygon, one just need to know the length of the edges, the position of the first edge, and the angles between two successive edges. The sequence of turning-angles is the discrete version of the curvature and  defines a equilateral polygon modulo rotation and translation. In Fig.~\ref{Liberty3} , right, we have reconstructed the statue of Liberty using the discrete curvature. 


In order to interpolate between two parameterized curves, it is easier when the domains of the parameter coincide. For this reason we will always consider curves parameterized with a parameter in $[0; 1]$. A natural parameterization is then the parameterization proportional to arc-length. It is obtain from the parameterization by arc-length by dividing the arc-length parameter by the length of the curve $L$. The corresponding curvature function is  also defined  on $[0; 1]$ and is obtained from the curvature function parameterized by arc-length by compressing the $x$-axis by a factor $L$. To recover the initial curve from the curvature function associated to the parameter $s\in[0; 1]$ proportional to arc-length, one only need to know the length of the curve.

\subsection{Parameterization proportional to curvature-length}
In the same spirit as the scale space of T. Lindeberg (\cite{Lindeberg}), and the curvature scale space of Mackworth and Farcin Mokhtarian (\cite{Mokhtarian}), we now define another very natural parameterization space of 2D curves. Its relies on the fact that the integral of the absolute value of the curvature $\kappa$ is an increasing function on the interval $[0; 1]$, 
stricktly increasing when there are no flat pieces. In that case the function 
\begin{equation}\label{r}
r(s) = \frac{\int_0^s |\kappa(s)| ds}{\int_0^1 |\kappa(s)| ds}
\end{equation}
(where $\kappa$ denotes the curvature of the curve) belongs to the group of orientation preserving diffeomorphisms of the parameter space $[0; 1]$, denoted by $\textrm{Diff}^+([0; 1])$. Note that its inverse $s(r)$ can be computed graphically using the fact that its graph is the symmetric of the graph of $r(s)$ with respect to $y = x$. 
The contour of Elie Cartan's head can  be reparameterized using the parameter $r \in [0; 1]$ instead of the parameter $s\in [0; 1]$. In Fig.~\ref{Cartan_int_curvature} upper left, we have depicted the graph of the function $s\mapsto r(s)$. A uniform sampling with respect to the parameter $r$ is obtain by uniformly sampling the vertical-axis (this is materialized by the green equidistributed horizontal lines) and resampling Elie Cartan's head at the sequence of values of the $s$-parameter given by the abscissa of the corresponding points on the graph of $r$ (where  a green line hits the graph of $r$ a red vertical line materializes the corresponding abscissa). One sees that this reparameterization naturally increases the number of points where the 2D contour is the most curved, and decreases the number of points on nearly flat pieces of the contour. For a given number of points, it gives an optimal way to store the information contained in the contour.
The quantity
\begin{equation}\label{curvature_length}
C = L \int_0^1 |\kappa(s)| ds,
\end{equation}
where $s\in[0; 1]$ is proportional to arc-length, is called the {\it total curvature-length} of the curve. It is the length of the curve drawn in $\textrm{SO}(2)$ by the moving frame associated with the arc-length parameterized curve. % when the Lie group $\textrm{SO}(2)$ is endowed with its natural the Riemannian structure.
For this reason we call this parameterization  the {\it parameterization proportional to curvature-length}. In the right picture of Fig.~\ref{Cartan_int_curvature}, we show the corresponding resampling of the contour of Elie Cartan's head. 

 \begin{figure}[!ht]
 		\centering
		%\makebox[\textwidth]{\framebox[0cm]{\rule{0pt}{3cm}}}
 		%\includegraphics[width = 5cm]{Courbure_Cartan_b_k.pdf}
		%\includegraphics[width = 5cm]{Cartan_int_curvature.pdf}%\vspace{-2cm}
		%\begin{subfigure}[t!]{0.30\textwidth}
		%\includegraphics[width = 8cm]{Integrale_Curvature_Cartan_L20.pdf} \hspace{0.5cm}
		\includegraphics[width = 4cm]{curvature1.png} \hspace{0.5cm}
		%\end{subfigure}
		%\begin{subfigure}[t!]{0.30\textwidth}
		\includegraphics[width = 4cm]{Cartan_curvature_length_L20.pdf}\\
		%\end{subfigure}\\
		\vspace{-0.5cm}
		%\begin{subfigure}[t!]{0.30\textwidth}
		\includegraphics[width = 4cm]{Integrale_Curv_arc_Cartan.pdf}\hspace{0.5cm}
		%\end{subfigure}
		%\begin{subfigure}[t!]{0.30\textwidth}
		\includegraphics[width = 4cm]{Cartan_Curv_arc_length_L20.pdf}
		%\end{subfigure}
		%\vspace{-2cm}
		%\includegraphics[width = 5cm]{Cartan_parameterization_curvature.pdf}
		%\includegraphics[width = 5cm]{Curvature_Noether.eps}
 		\caption{\scriptsize First line : Integral of the (renormalized) absolute value of the curvature  (left), and corresponding resampling of Elie's Cartan head (right).
		Second line : Integral of the (renormalized) curvarc length  (left), and corresponding resampling of Elie's Cartan head (right).}
 		\label{Cartan_int_curvature}		
\end{figure}

%\begin{figure}[!ht]
% 		\centering
%		%\makebox[\textwidth]{\framebox[0cm]{\rule{0pt}{3cm}}}
% 		%\includegraphics[width = 5cm]{Courbure_Cartan_b_k.pdf}
%		%\includegraphics[width = 5cm]{Cartan_int_curvature.pdf}%\vspace{-2cm}
%		\begin{subfigure}[t!]{0.30\textwidth}
%		\includegraphics[width = 8cm]{Integrale_Curvature_Cartan_L20.pdf} \hspace{0.5cm}\includegraphics[width = 8cm]{Cartan_curvature_length_L20.pdf}
%		\end{subfigure}
%		\vspace{.5cm}
%		\begin{subfigure}[t!]{0.30\textwidth}
%		\includegraphics[width = 8cm]{Integrale_Curv_arc_Cartan.pdf}\hspace{0.5cm}\includegraphics[width = 8cm]{Cartan_Curv_arc_length_L20.pdf}
%		\end{subfigure}
%		%\vspace{-2cm}
%		%\includegraphics[width = 5cm]{Cartan_parameterization_curvature.pdf}
%		%\includegraphics[width = 5cm]{Curvature_Noether.eps}
% 		\caption{\scriptsize First line : Integral of the (renormalized) absolute value of the curvature  (left), and corresponding resampling of Elie's Cartan head (right).
%		Second line : Integral of the (renormalized) curvarc length  (left), and corresponding resampling of Elie's Cartan head (right).}
% 		\label{Cartan_int_curvature}		
%\end{figure}


This resampling can naturally be adapted in the case of flat pieces resulting in a sampling where there is no points between two points on the curve joint by a straight line. In the left picture of Fig.~\ref{Liberty5}, we have depicted a sampling of the  statue of Liberty proportional to curvature-length. Note that there are no points on the base of the statue. The corresponding parameterization has the advantage of concentrating on the pieces of the contour that are very complex, i.e. where there is a lot of curvature, and not distributing points on the flat pieces which are easy to reconstruct (connecting two points by a straight line is easy, but drawing the moustache of Elie Cartan is harder and needs more information).



%As in section~\ref{interpolate}, it is possible to reconstruct a curve from its curvature function parameterized proportionally to curvature-length, provided that we know the length of the curve $L$ and its total curvature-length  $C$, and provided that there is no flat piece.  Indeed, derivating equation~\eqref{r}, one obtains $dr = \frac{|\kappa(s)|}{C} Lds$, where $Lds$ is the arc-length measure of the curve.
%%\textcolor{red}{It follows that a discrete version of a curve parameterized proportionally to the curvature-length $r$ is  a polygon such that the angles between two successive edges are given by the curvature function at $r$ and such that the length of the edges are inverse proportional to the curvature.}


The drawback of using the parameterization proportional to curvature-length is that one can not reconstruct the flat pieces of a shape without knowing their lengths (remember that the parameterization proportional to curvature-length put no point at all on flat pieces). For this reason we propose a parameterization intermediate between arc-length parameterization and curvature-length parameterization. We call it {\it curvarc-length parameterization}.




%
%
% \begin{figure}[!ht]
% 		\centering
%		%\makebox[\textwidth]{\framebox[0cm]{\rule{0pt}{3cm}}}
% 		\includegraphics[width = 8cm]{Liberty_curvature2.pdf}%{Tumpach_Greenwich_cropped.pdf}%
% 		\caption{\scriptsize The statue of Liberty (left), a uniform resampling of it using Matlab function spline (middle), and a resampling proportionnal to its curvature-length (right).}
% 		\label{Liberty}		
%\end{figure}
%
%It is easy to reconstruct a curve using its curvature function parameterized by arc-length (see the {\it About the cover} section in \cite{Notices}).
%For curves parameterized proportionally to arc-length, one need an additional information, namely the length  of the curve
%$$
%L(f) =  \int_0^1 \|f'(t)\| dt.
%$$
%Now we will explain how to reconstruct a curve using its curvature function parameterized by curvature-length, i.e. by the parameter $r$ defined by formula~\eqref{r}, knowing the length $L(f)$ and the curvature length defined by
%\begin{equation}\label{curvature_length}
%C(f) = \int_0^1 |\kappa(s)| ds.
%\end{equation}
%This curvature length is in fact the length of the curve draw in $\textrm{SO}(2)$ by the moving frame associated with the curve $f$ when the Lie group $\textrm{SO}(2)$ is endowed with its natural Riemannian structure. 
% \begin{figure}[!ht]
% 		\centering
%		%\makebox[\textwidth]{\framebox[0cm]{\rule{0pt}{3cm}}}
% 		%\includegraphics[width = 8cm]{Herz_Rainbow.pdf}%{Tumpach_Greenwich_cropped.pdf}%
%		\includegraphics[width = 4cm]{Cartan_Noether_not_closing.pdf}%{Tumpach_Greenwich_cropped.pdf}%
%		\includegraphics[width = 4cm]{Cartan_Noether_int_vitesse3.pdf}%{Tumpach_Greenwich_cropped.pdf}%
% 		\caption{\scriptsize }
% 		\label{}		
%\end{figure}
%

\subsection{Curvarc-length parameterization}

In order to define the curvarc-length parameterization, we consider the triple $(P(s), \vec{v}(s), \vec{n}(s))$, where $P(s)$ is the point of the shape parameterized proportionally to arc-length with $s\in[0; 1]$, $\vec{v}(s)$ and $\vec{n}(s)$ the corresponding unit tangent vector and unit normal vector respectively. It defines an element of the group of rigid motions of $\mathbb{R}^2$, called the special Euclidean group and denoted by $\textrm{SE}(2) := \mathbb{R}^2\rtimes \textrm{SO}(2)$. The point $P(s)$ corresponds to the translation part of the rigid motion, it is the vector of translation needed to move the origin to the point of the curve corresponding to the parameter value $s$. The moving frame $O(s)$ defined by $\vec{v}(s)$ and $\vec{n}(s)$ is the rotation part of the rigid motion. 
One has the following equations~:
%\begin{equation}\label{triedre}
%\frac{d}{ds}\left(\begin{array}{c} P\\ \vec{v}\\ \vec{n} \end{array}\right) = \left(\begin{matrix} 0 & L & 0\\ 0 & 0 & \kappa(s)\\ 0 & -\kappa(s) & 0\end{matrix}\right)\cdot \left(\begin{array}{c} P\\ \vec{v}\\ \vec{n} \end{array}\right),
%\end{equation}
\begin{equation}\label{triedre}
\frac{dP}{ds}  = L \vec{v}(s) \quad\textrm{and} \quad O(s)^{-1}\frac{d}{ds}O(s) = \left(\begin{smallmatrix} 0 & -\kappa(s)\\ \kappa(s) & 0\end{smallmatrix}\right),
\end{equation}
where $L$ is the length of the curve. 
Endow $\textrm{SE}(2) := \mathbb{R}^2\rtimes \textrm{SO}(2)$ with the structure of a  Riemannian manifold, product of the plane and the Lie group $\textrm{SO}(2) \simeq \mathbb{S}^1$. Than the norm of the  tangent vector to the curve $s\mapsto (P(s), \vec{v}(s), \vec{n}(s))$ is $L + |\kappa(s)|$. Therefore the length of the $\textrm{SE}(2)$-valued curve is $L + \int_0^1 |\kappa(s)| ds = L + \frac{C}{L}$. We call it the total curvarc-length. It follows that the following function
\begin{equation}
u(s) = \frac{\int_0^s (L + |\kappa(s)|) ds}{\int_0^1 (L + |\kappa(s)|) ds}
\end{equation}
defines a reparameterization of $[0; 1]$.  
More generally, one can use the following canonical parameter to reparameterize a curve in a canonical way:
\begin{equation}\label{ulambda}
u_\lambda(s) = \frac{\int_0^s L\lambda + |\kappa(s)|) ds}{\int_0^1 L\lambda + |\kappa(s)|) ds},
\end{equation}
where $s$ is the arc-length parameter.
In Fig.~\ref{Liberty5}  we show the resulting sampling of the Statue of Liberty for different values of $\lambda$. Note that for $\lambda = 0$, one recovers the curvature-length parameterization, for $\lambda = 1$ one obtains the curvarc-length parameterization, and when $\lambda\rightarrow+\infty$ the parmeterization tends to the arc-length parameterization.


%The arc-length parameter of the initial shape  is related to the parameter $u$ by
%$$
%L ds = \frac{ L^2 + C}{L + |\kappa(u)|} du.
%$$
%\framebox{curvarc-length parameterization of the statur of liberty}
%\textcolor{red}{Cite Figure 8 and Figure 9}
%
% \begin{figure}[!ht]
% 		\centering
%		%\makebox[\textwidth]{\framebox[0cm]{\rule{0pt}{3cm}}}
% 		\includegraphics[width = 8cm]{reconstruction_liberty_angle.pdf}%{Tumpach_Greenwich_cropped.pdf}%
% 		\caption{\scriptsize \textcolor{red}{Reparameterization of the statue of Liberty proportional to curvarc-length (left), and its reconstruction using  its curvarc-length parameterized curvature (right).}}
% 		%\label{Liberty}		
%\end{figure}

%\subsubsection{Parameterization proportional  to the integral of $\lambda$ + curvature}
%
%
%, for (from left to right)
%		$\lambda = 0; \lambda = 0.3; \lambda = 1; \lambda = 2; \lambda = 100$.
%		
	\vspace{-0.5cm}
	\begin{figure}[!ht]
 		\centering
		%\makebox[\textwidth]{\framebox[0cm]{\rule{0pt}{3cm}}}
 		%\includegraphics[width = 8cm]{Liberty_with_235points_and_reconstruction.pdf}%{Tumpach_Greenwich_cropped.pdf}%
		%\includegraphics[width = 9cm]{Liberty_fois_4_cropped.pdf}%{Tumpach_Greenwich_cropped.pdf}%
		%\begin{subfigure}[t!]{0.30\textwidth}
		\includegraphics[width = 0.9\textwidth]{five_Liberty.pdf}
 		\caption{\scriptsize 
		%The statue of Liberty (left), a uniform resampling using Matlab function spline (middle left), a reconstruction of the statue using its (discrete) curvature (middle right) and a curvature-length uniform resampling (right).
		Resampling of the statue of Liberty proportional to the intergral of $\lambda$ + curvature, for (from left to right)
		$\lambda = 0; \lambda = 0.3; \lambda = 1; \lambda = 2; \lambda = 100$.		}
 		\label{Liberty5}		
\end{figure}

			
%\subsection{Quotient space versus section of a fiber bundle}
%
%The fiber bundle $p~:\mathcal{F}\rightarrow \mathcal{S}$ is a particular example of a general mathematical object attached to a smooth action of a group on a manifold. Another instance of this notion is when $\mathbb{R}^2$ acts by translations on 2D-curves, or when $\mathbb{R}^+$ acts on shapes by scaling. If one is interested only in curves modulo translations, i.e. irrespective of their position in space, than one can either consider the quotient space of the manifold of curves modulo the action of $\mathbb{R}^2$ by translation, or consider only centered curves (see Tabular~\ref{Tabular_group_action}).  When we specify which procedure we follow to center the curves, one is choosing a representant in each orbit under the action by translation.  This preferred choice is called a {\it section} of the corresponding fiber bundle. Analogously, for the action of $\mathbb{R}^+$ by scaling, a section of the manifold of simple closed  curves could by the set of length-one curves or the set of curves enclosing an area equal to one. We give in tabular~\ref{Tabular_group_action} examples of group actions on 2D simple closed curves and, for each case, two possible sections.
%
%\begin{figure*}[!ht]
% 		\centering
%\begin{tabular}{|p{3.5cm}|p{3.5cm}|p{3.5cm}|p{3.5cm}|}
%\hline
%Group $G$& Some elements of one orbit under the group $G$ & a preferred element in the orbit & another choice of preferred element in the orbit\\
%\hline
%\begin{center} $\mathbb{R}^3$ acting by translation \end{center}&\includegraphics[width=3cm]{./Liberty1_cropped} & \includegraphics[width=2.6cm]{./Liberty2_cropped} {\small centered curve : $\int_0^1 \left(\begin{smallmatrix} f_1(s)\\f_2(s)\end{smallmatrix}\right) \|f'(s)\| ds = \left(\begin{smallmatrix}0\\ 0 \end{smallmatrix}\right).$} & \includegraphics[width=2.8cm]{./Liberty3_cropped} {\small curve starting at $\left(\begin{smallmatrix} 0\\ 0 \end{smallmatrix}\right)$.}\\
%\hline
%\begin{center} $\mathrm{SO}(3)$ acting by rotation \end{center} &\begin{center}\includegraphics[width=3.2cm]{./Liberty4_cropped}\end{center} & \begin{center} \begin{tabular}{l} \includegraphics[height=0.8cm]{./Liberty_ellipse}\\ axes of \\ approximating\\ ellipse aligned \end{tabular}\end{center}& \begin{center} \begin{tabular}{l} \includegraphics[width=0.8cm]{./liberty_vector} \\   tangent vector\\ at starting point\\ horizontal \end{tabular}\end{center} \\
%\hline
%%$\mathbb{R}^+$ acting by scaling &\includegraphics[width=3cm]{./Liberty7_cropped} & length = 1 & enclosed area =1 \\
%%\hline
%%$\mathrm{Diff}^+([0; 1])$ acting by reparameterization & &arc-length parameterization & curvature proportional parameterization   \\
%%\hline
%%$\mathbb{S}^1$ acting by change of based point && no global section & no global section \\
%%\hline
%%\end{tabular}
%%\caption{\scriptsize Examples of group actions on 2D simple closed curves and sections of the corresponding fiber bundle.}
%% %		\label{Tabular}	
%%\end{figure}
%%
%%\begin{figure}[!ht]
%% 		\centering
%%\begin{tabular}{|p{3cm}|p{3cm}|p{3cm}|p{3cm}|}
%%\hline
%%Group $G$& Some elements of one orbit under the group $G$ & a preferred element in the orbit & another choice of preferred element in the orbit\\
%%\hline
%%%$\mathbb{R}^3$ acting by translation &\includegraphics[width=3cm]{./Liberty1_cropped} & \includegraphics[width=2.6cm]{./Liberty2_cropped} {\small centered curve : $\int_0^1 \left(\begin{smallmatrix} f_1(s)\\f_2(s)\end{smallmatrix}\right) \|f'(s)\| ds = \left(\begin{smallmatrix}0\\ 0 \end{smallmatrix}\right).$} & \includegraphics[width=2.8cm]{./Liberty3_cropped} {\small curve starting at $\left(\begin{smallmatrix} 0\\ 0 \end{smallmatrix}\right)$.}\\
%%%\hline
%%%$\mathrm{SO}(3)$ acting by rotation &\includegraphics[width=3cm]{./Liberty4_cropped} & axes of approximating ellipse aligned &  first tangent vector horizontal\\
%%\hline
%\begin{center} $\mathbb{R}^+$ acting by scaling \end{center}&\begin{center} \includegraphics[width=3cm]{./Liberty7_cropped} \end{center}& \begin{center} \begin{tabular}{l} \includegraphics[height=1.5cm]{./Liberty_bleue}\\  length = 1 \end{tabular}\end{center} &\begin{center} \begin{tabular}{l} \includegraphics[height=2cm]{./Liberty_bleue}\\ enclosed area =1 \end{tabular}\end{center}\\
%\hline
%\begin{center} $\mathrm{Diff}^+([0; 1])$ acting by reparameterization \end{center}& \begin{center}\includegraphics[height=2cm]{./Liberty2} \includegraphics[height=2cm]{./Liberty1} \end{center} &
%\begin{center}\begin{tabular}{l}\includegraphics[height=2cm]{./Liberty_arclength}\\ arc-length \\ parameterization\end{tabular} \end{center} & 
%\begin{center}\begin{tabular}{l}\includegraphics[height=2cm]{./Liberty1}\\ curvature \\proportional  \\ parameterization\end{tabular} \end{center} \\
%\hline
%%$\mathbb{S}^1$ acting by change of based point && no global section & no global section \\
%%\hline
%\end{tabular}
%\caption{\scriptsize Examples of group actions on 2D simple closed curves and different choices of sections of the corresponding fiber bundle.}
% 		\label{Tabular_group_action}	
%\end{figure*}
%
%
%A global section of the fiber bundle $p~:\mathcal{F}\rightarrow \mathcal{S}$ is a smooth application $S~:\mathcal{S}\rightarrow \mathcal{F}$, such that $p\circ S([f]) = [f]$ for any $[f]\in\mathcal{S}$. There is one-to-one correspondance between the shape space $\mathcal{S}$ and the range of $S$. Defining a global section of $p~:\mathcal{F}\rightarrow \mathcal{S}$ is in fact defining a way to choose a preferred element in the fiber $p^{-1}([f])$ over $[f]$. It consists of singeling out a preferred parameterization of each oriented shape. 
%
%Let us give an example of a  canonical section of the fiber bundle $p~:\mathcal{F}\rightarrow \mathcal{S}$.
%Suppose that the homogeneous space $G/H$ is endowed with a Riemannian structure. Then one can measure the norm of the velocity vector of a given curve in $G/H$, as well as the length of the curve. In this case,
%{\it the parameterization proportional to arc-length with parameter} $s\in[0; 1]$ is a particularly natural global section of $p~:\mathcal{F}\rightarrow \mathcal{S}$. It associates to the oriented shape $[f]$ the parameterization with constant speed such that the curve is travelled for a time parameter ranging in $[0; 1]$. It can be built by hand by just measuring the distances in $G/H$ travelled by the curve, but it can be also recovered using {\it any} parameterization $t\mapsto f(t)$ by the change of parameter $t\mapsto s(t) = \frac{1}{L(f)}\int_0^t \|f'(t)\| dt$, where $L(f) = \int_0^1 \|f'(t)\| dt$ is the length of $f$. The statue of Liberty figuring in the middle of Fig.~\ref{Liberty3} is an example of a curve parameterized proportional to arc-length. 
%%In section~\ref{norm_moving_frame}, we will define another section of $p~:\mathcal{F}\rightarrow \mathcal{S}$ called the {\it parameterization proportional to curvature-length}. 
%
%


\vspace{-1cm}
\section{Application to medical imaging : parameterization of bones}

\begin{figure}[!ht]
\centering
\includegraphics[height = 3cm]{bones2}
\includegraphics[height = 3cm]{joint_space}
\caption{Landmarks on bones used to measure joint space (courtesy of \cite{Langs})}
\label{landmarks}
\end{figure}
In the analysis of diseases like Rheumatoid Arthritis, one uses X-ray scans to evaluate how the disease affectes the bones. One effect of Rheumatoid Arthritis is erosion of bones, another is joint shrinking \cite{Langs}. In order to measure joint space, one has to solve a point correspondance problem. For this, one uses landmarks along the contours of bones as in Fig.\ref{landmarks}. 
These landmarks have to be placed at the same anatomical positions for every patient.  Below they are placed using  a method by Hans Henrik Thodberg \cite{Th}, based on minimum description length which minimizes the description of a PCA model capturing the variability of the landmark positions. 
For instance in Fig.~\ref{bones} left, the landmark number 56 should always be in the middle of the head of the bone because it is used to measure the width between two adjacent bones in order to detect rheumatoid arthritis.
\begin{figure}[!ht]
 		\centering
		\includegraphics[height = 3cm]{bonesN3.png}
		\includegraphics[height = 3cm]{bonesN2.png}
		\includegraphics[height = 3cm]{bonesN1.png}
		\caption{Point correspondance on 3 different bones using the method of \cite{Th}}
		\label{bones}
		\end{figure}

%\begin{figure}[!ht]
%\includegraphics[height = 5cm]{bones1.png}
%\includegraphics[height = 5cm]{bones2.png}
%\caption{Pictures of bones from the lectures of Georg Langs}
%
%\end{figure}
Although the method by Hans Henrik Thodberg gives good results, it is computationally expensive. In this paper we propose to recover similar results with an quicker algorithm. It is based on the fact that any geometrically meaningful parameterization of a contour can be expressed using the arc-length measure and the curvature of the contour, which are the only geometric invariants of a 2D-curve (modulo translation and rotation). It follows that the parameterization calculated by Thodberg's algorithm should be recovered as a parameterization expressed using arc-length and curvature.
We investigate a  2 parameter family of parameterizations defined by 
\begin{equation}\label{u}
u(s) = \frac{\int_0^s (c* L + |\kappa(s)|^{\lambda}) ds}{\int_0^1 (c*L + |\kappa(s)|^\lambda) ds}
\end{equation}where $c$ and $\lambda$ are positive parameters and where $L$ is the length of the curve and $\kappa$ its curvature function.
We recover an analoguous parameterization to the one given by Thodberg's algorithm with $c = 1$ and $\lambda = 7$ at real-time speed (gain of 2 order of magnitude). %The algorithm computes the parameterization of 14 bones in 2.23s on a Mac M1.
 \begin{figure}[!ht]
 		\centering
\includegraphics[width = 10.5cm]{bones_final.png}	
\caption{14 bones parameterized by Thodberg's algorithm on one hand and the parameterization defined by \eqref{u} with $c = 1$ and $\lambda = 7$ on the other hand (the two parameterizations are superposed). The colored points corresponds to points labelled $1$, $48$, $56$, $66$. They overlap for the two methods.}
\end{figure}

%It follows that the obtained parameterization is not the arc-length parameterization but a parameterization which takes the geometry of the bones into account.


\section{Conclusion}


We proposed diverse canonical parameterization of 2D-contours, which are expressed using  arc-length and curvature of  curves. The curvature-length parameterization and the curvarc-length parameterization are very natural examples, since they corresponds to a constant-speed moving frame in $SO(2)$ and $SE(2)$.
We present an application to the problem of point correspondance in medical imaging consisting of labelling automatically keypoints along the contour of bones. We recover an analoguous parameterization to the one proposed by Thodberg at real-time speed. Having a two-parameter family of parameterizations at our disposal, a fine-tuning can be applied on top of our results in order to improve the point correspondance further.
%\framebox{Use deep-learning to learn the best parameterization}
%In this paper we have proposed a novel Riemannian framework for computing geodesic paths between shapes of parameterized surfaces.
%These geodesics are invariant to rigid motion, scaling and most importantly reparameterization of individual surfaces. The novelty lies in defining a Riemannian metric directly on the quotient (shape) space, rather
%than inheriting it from pre-shape space, and in using it to formulate a path energy that measures only the normal components of velocities along the path. The geodesic
%computation is based on a path-straightening technique that iteratively corrects paths between surfaces until geodesics are achieved. 
%We have presented some examples of geodesics between surfaces in shape spaces and utilized the distances between surfaces for classification of some 3D shapes. 
%However, the computational costs of our programs are deemed high and convergence should be accelerated in order to be able to apply this framework in realistic 
%practical scenarios such as, for instance, human body action recognition.
%

%
% ---- Bibliography ----
%
% BibTeX users should specify bibliography style 'splncs04'.
% References will then be sorted and formatted in the correct style.
%
% \bibliographystyle{splncs04}
% \bibliography{mybibliography}
%
\begin{thebibliography}{8}
%\bibitem{Novel}
%S.~H. Joshi, A.~Srivastava, E.~Klassen, and I.~Jermyn, ``A novel representation
%  for computing geodesics between n-dimensional elastic curves,'' in \emph{IEEE
%  Conference on computer Vision and Pattern Recognition (CVPR)}, 2007.
%
%\bibitem{Shape}
%A.~Srivastava, E.~Klassen, S.~Joshi, and I.~Jermyn, ``Shape analysis of elastic
%  curves in euclidean spaces,'' \emph{IEEE Transactions on Pattern Analysis and
%  Machine Intelligence}, vol.~33, no.~7, pp. 1415--1428, 2011.
%
%\bibitem{Sugano}
%A.~Sugano, ``The physics of the hardest move in ballet,'' 2017. [Online].
%  Available:
%  \url{http://ed.ted.com/lessons/the-physics-of-the-hardest-move-in-ballet-arleen-sugano}
%
%\bibitem{MM}
%P.~W. Michor and D.~Mumford, ``Vanishing geodesic distance on spaces of
%  submanifolds and diffeomorphisms,'' \emph{Doc. Math.}, vol.~10, pp. 217--245,
%  2005.
%
%\bibitem{BHM4}
%M.~Bauer, P.~Harms, and P.~W. Michor, ``Sobolev metrics on shape space of
%  surfaces,'' \emph{Journal of Geometric Mechanics}, vol.~3, no.~4, pp.
%  389--438, 2011.
%
%\bibitem{BHM3}
%------, ``Curvature weighted metrics on shape space of hypersurfaces in
%  n-space,'' \emph{Differential Geom. Appl.}, vol.~30, pp. 33--41, 2012.
%
%\bibitem{BHM5}
%------, ``Almost local metrics on shape space of hypersurfaces in n-space,''
%  \emph{SIAM J. Img. Sci.}, vol.~5, no.~1, pp. 244--310, Mar. 2012.
%
%\bibitem{FJSY09}
%M.~Fuchs, B.~J\"uttler, O.~Scherzer, and H.~Yang, ``Shape metrics based on
%  elastic deformations,'' \emph{J. Math. Imaging Vis.}, vol.~35, no.~1, pp.
%  86--102, 2009.
%
%\bibitem{Drira}
%H.~Drira, A.~Tumpach, and M.~Daoudi, ``Gauge invariant framework for
%  trajectories analysis,'' in \emph{Proceedings of the 1st International
%  Workshop on DIFFerential Geometry in Computer Vision for Analysis of Shapes,
%  Images and Trajectories}, 2015.
%
%\bibitem{Younes98}
%L.~Younes, ``Computable elastic distances between shapes,'' \emph{SIAM J. of
%  Applied Math}, pp. 565--586, 1998.
%
%\bibitem{Younes2008}
%L.~Younes, P.~W. Michor, J.~Shah, and D.~Mumford, ``A metric on shape space
%  with explicit geodesics.'' \emph{Matematica E Applicazioni}, no.~19, pp.
%  25--57, 2008.
%
%\bibitem{Alice}
%A.~L. Brigant, ``Computing distances and geodesics between manifold-valued
%  curves in the srv framework,'' 2017. [Online]. Available:
%  \url{arXiv:1601.02358v4}
%
%\bibitem{Elise}
%E.~Celledoni, S.~Eidnes, and A.~Schmeding, ``Shape analysis on homogeneous
%  spaces,'' 2017. [Online]. Available: \url{arXiv:1704.01471v1}
%
\bibitem{Langs}  G. Langs, P. Peloschek, H. Bischof and F. Kainberger, {\it Automatic Quantification of Joint Space Narrowing and Erosions in Rheumatoid Arthritis}, IEEE Transactions on Medical Imaging, Vol. 28, no.1, 2009.

%Medical University of Vienna, \textcolor{blue}{\href{https://radnuk-prod.meduniwien.ac.at/unsere-abteilungen/computational-imaging-research-cir/team/georg-langs/}{https://radnuk-prod.meduniwien.ac.at/unsere-abteilungen/computational-imaging-research-cir/team/georg-langs/}}

%\bibitem{MioAnuj}
%W.~Mio, A.~Srivastava, and S.~H. Joshi, ``On shape of plane elastic curves,''
%  \emph{International Journal of Computer Vision}, vol.~73, no.~3, pp.
%  307--324, 2007.
%
%\bibitem{TPAMI}
%A.~B. Tumpach, H.~Drira, M.~Daoudi, and A.~Srivastava, ``Gauge invariant
%  framework for shape analysis of surfaces,'' \emph{IEEE Transactions on
%  Pattern Analysis and Machine Intelligence}, vol.~38, no.~1, pp. 46--59, 2016.
%
%\bibitem{Notices}
%A.~B. Tumpach, ``Gauge invariance of degenerate {R}iemannian metrics,''
%  \emph{Notices of the American Mathematical Society}, vol.~63, no.~4, pp.
%  342--350, 2016.
%

%\bibitem{LRK}
%S.~Lahiri, D.~Robinson, and E.~Klassen, ``Precise matching of $\textrm{PL}$
%  curves in $\mathbb{R}^n$ in the square root velocity framework,'' pp.
%  133--186, 2015. [Online]. Available: \url{arXiv:1501.00577}
%
%
%\bibitem{Srivastava2011b}
%A.~Srivastava, E.~Klassen, S.~H. Joshi, and I.~H. Jermyn, ``Shape analysis of
%  elastic curves in euclidean spaces,'' \emph{IEEE Trans. PAMI}, no.~33, pp.
%  1415--1428, 2011.
%
\bibitem{Lindeberg}
T.~Lindeberg, ``Image matching using generalized scale-space interest points,''
  \emph{Journal of Mathematical Imaging and Vision}, vol.~52, no.~1, pp. 3--36,
  2015.

\bibitem{Mokhtarian}
F.~Mokhtarian and A.~K. Mackworth, ``A theory of multiscale, curvature-based
  shape representation for planar curves,'' \emph{IEEE Transactions on Pattern
  Analysis and Machine Intelligence}, vol.~14, no.~8, pp. 789--805, 1992.

\bibitem{Th}
H.~Thodberg, ``Minimum description length shape and appearance models,''
  \emph{Proceedings of Information Processing in Medical Imaging ({IPMI}
  2003)}, 2003.
  
  \bibitem{TumPres}
A.~B. Tumpach and S.~C. Preston, ``Quotient elastic metrics on the manifold of
  arc-length parameterized plane curves,'' \emph{Journal of Geometric
  Mechanics}, vol.~9, no.~2, pp. 227--256, 2017.

\end{thebibliography}
\end{document}
