\documentclass[10pt,twocolumn,letterpaper]{article}
\usepackage{cvpr}          

\usepackage{graphicx}
\usepackage{amsmath}
\usepackage{multirow}
\usepackage{multicol}
\newtheorem{theorem}{Theorem}
\usepackage{amssymb}
\usepackage{booktabs}
\usepackage[ruled,vlined]{algorithm2e}


\linespread{0.97}

\usepackage[pagebackref,breaklinks,colorlinks]{hyperref}


\usepackage[capitalize]{cleveref}
\crefname{section}{Sec.}{Secs.}
\Crefname{section}{Section}{Sections}
\Crefname{table}{Table}{Tables}
\crefname{table}{Tab.}{Tabs.}


\def\cvprPaperID{9439}
\def\confName{CVPR}
\def\confYear{2023}


\begin{document}

\title{CFA: Class-wise Calibrated Fair Adversarial Training}

\author{
    Zeming Wei\textsuperscript{1}, 
    Yifei Wang\textsuperscript{1},
    Yiwen Guo\textsuperscript{2},
    Yisen Wang\textsuperscript{3,4}\thanks{Corresponding Author: Yisen Wang (yisen.wang@pku.edu.cn)}\\
    \textsuperscript{1}School of Mathematical Sciences, Peking University 
    \textsuperscript{2}Independent Researcher\\
    \textsuperscript{3}National Key Lab of General Artificial Intelligence \\ School of Intelligence Science and Technology, Peking University\\
\textsuperscript{4}Institute for Artificial Intelligence, Peking University
}


\maketitle

\begin{abstract}
   Adversarial training has been widely acknowledged as the most effective method to improve the adversarial robustness against adversarial examples for Deep Neural Networks (DNNs). So far, most existing works focus on enhancing the overall model robustness, treating each class equally in both the training and testing phases. Although revealing the disparity in robustness among classes, few works try to make adversarial training fair at the class level without sacrificing overall robustness. In this paper, we are the first to theoretically and empirically investigate the preference of different classes for adversarial configurations, including perturbation margin, regularization, and weight averaging. Motivated by this, we further propose a \textbf{C}lass-wise calibrated \textbf{F}air \textbf{A}dversarial training framework, named CFA, which customizes specific training configurations for each class automatically. Experiments on benchmark datasets demonstrate that our proposed CFA can improve both overall robustness and fairness notably over other state-of-the-art methods. Code is available at \url{https://github.com/PKU-ML/CFA}.
\end{abstract}

\section{Introduction}
\label{sec:introduction}
% \begin{itemize}
%     % Diffusion of FL
%     \item {\st{Diffusion of FL}}
%     % Security threats to FL
%     \item {\st{Security threats to FL with particular focus on model poisoning}}
%     % Limitations of existing countermeasures
%     \item {\st{Current countermeasures (e.g., KRUM) and their limitations}}
%     % Proposed method and its advantages
%     \item {\st{Intuitive description of the proposed method and its difference (i.e., advantages) w.r.t. state of the art}}
%     % Main contributions
%     \item {\st{Summary of the main contributions of this work}}
%     % Paper's structure and organization
%     \item {\st{Paper's structure and organization}}
% \end{itemize}

% Diffusion of FL
Recently, {\em federated learning} (FL) has emerged as the leading paradigm for training distributed, large-scale, and privacy-preserving machine learning (ML) systems~\cite{mcmahan2017googleai,mcmahan2017aistats}. 
The core idea of FL is to allow multiple edge clients to collaboratively train a shared, global model without disclosing their local private training data.
%Specifically, an FL system consists of a central server and many edge clients; 
A typical FL round involves the following steps: {\em(i)} the server randomly picks some clients and sends them the current, global model; {\em(ii)} each selected client locally trains its model with its own private data; then, it sends the resulting local model to the server;\footnote{Whenever we refer to global/local model, we mean global/local model {\em parameters}.} {\em(iii)} the server updates the global model by computing an \emph{aggregation function}, usually the average (FedAvg), on the local models received from clients.
% \begin{enumerate}
%     \item[{\em(i)}] the server sends the current, global model to the clients and appoints some of them for training;
%     \item[{\em(ii)}] each selected client locally trains its copy of the global model with its own private data; then, it sends the resulting local model back to the server;\footnote{Whenever we refer to global/local model, we mean global/local model {\em parameters}.}
%     \item[{\em(iii)}] the server updates the global model by computing an \emph{aggregation function} on the local models received from clients (by default, the average, also referred to as FedAvg~\cite{mcmahan2017aistats}).
% \end{enumerate}
This process goes on until the global model converges. %(e.g., after a certain number of rounds or other similar stopping criteria).
%\\
% The advantages of FL over the traditional, centralized learning paradigm are undoubtedly clear in terms of flexibility/scalability (clients can join/disconnect from the FL network dynamically), network communications (only model weights\footnote{We will use \textit{parameters} and \textit{weights} interchangeably.} are exchanged between clients and server), and privacy (each client's private training data is kept local at the client's end and not uploaded to the server).
\\
% Security threats to FL
%However, the growing adoption of FL also raises security concerns~\cite{costa2022covert}, particularly about its confidentiality, integrity, and availability.
Although its advantages over standard ML, FL also raises security concerns~\cite{costa2022covert}. %, particularly about its confidentiality, integrity, and availability~\cite{costa2022covert}.
% OLD, LONG VERSION
% Indeed, some work deals with privacy leakage that may expose the local data of some clients~\cite{melis2019sp}. 
% A large body of work, instead, investigates attacks that usually aim to detriment the predictive accuracy of the learned global model. For instance, \emph{data poisoning} attacks achieve this goal by letting an adversary pollute the training set of some corrupt FL clients with maliciously crafted examples~\cite{jagielski2018sp}.
% Similarly, in \emph{model poisoning} the attacker attempts to tweak the global model weights~\cite{bhagoji2019pmlr} by directly perturbing the local model's weights of some infected FL clients before these are sent to the central server for aggregation, usually via so-called Byzantine attacks. 
% It turns out that Byzantine model poisoning attacks severely impact standard FedAvg; therefore, more robust aggregation functions must be designed to make FL systems secure.
Here, we focus on \emph{untargeted model poisoning} attacks~\cite{bhagoji2019pmlr}, where an adversary attempts to tweak the global model weights %\footnote{We will use the terms \textit{parameters} and \textit{weights} interchangeably.} 
by directly perturbing the local model's parameters of some infected clients before these are sent to the central server for aggregation.
In doing so, the adversary aims to jeopardize the global model \textit{indiscriminately} at inference time.
Such model poisoning attacks severely impact standard FedAvg; therefore, more robust aggregation functions must be designed to secure FL systems.
\\
% In this paper, we focus on designing a novel robust aggregation scheme at the server's end to contrast the effect of Byzantine model poisoning attacks.
%
% Current countermeasures and their limitations
%Several countermeasures have been proposed in the literature to combat model poisoning attacks on FL systems.
% Some methods use simple statistics more robust than plain average to smooth the impact of malicious updates (e.g., Trimmed Mean and FedMedian~\cite{yin2018icml}). 
% Other defenses implement outlier detection techniques to discard malicious updates from the aggregation performed at the server's end. Those are either based on heuristics (e.g., Krum/Multi-Krum~\cite{blanchard2017nips} and Bulyan~\cite{mhamdi2018pmlr}) or data-driven approaches (e.g., K-means clustering~\cite{shen2016acm} or DnC via spectral analysis~\cite{shejwalkar2021ndss}). 
% Finally, some strategies rely on a centralized ``source of trust'' to spot potential malicious updates (e.g., FLTrust~\cite{cao2020fltrust}).
% Several countermeasures have been proposed in the literature to combat model poisoning attacks on FL systems, i.e., to discard possible malicious local updates from the aggregation performed at the server's end. 
% These techniques range from simple statistics more robust than plain average (e.g., Trimmed Mean and FedMedian~\cite{yin2018icml}) to outlier detection heuristics (e.g., Krum/Multi-Krum~\cite{blanchard2017nips} and Bulyan~\cite{mhamdi2018pmlr}) or data-driven approaches (e.g., spectral analysis via K-means clustering~\cite{shen2016acm} or spectral analysis), or methods based on ``source of trust'' (e.g., FLTrust~\cite{cao2020fltrust}).
% OLD, LONG VERSION
%Several countermeasures have been proposed in the literature to combat Byzantine model poisoning attacks on FL systems.
% Descriptive statistics
% For example, Trimmed Mean and FedMedian aggregate local model updates using more robust statistics than standard average~\cite{yin2018icml}.
%
% % Heuristics for outlier detection
% Many existing Byzantine-resilient strategies implement some outlier detection heuristics to discard the model updates sent by potentially malicious clients from the input of the aggregation function.
% One of the most popular heuristics is Krum~\cite{blanchard2017nips}.
% This strategy tries to mitigate the impact of Byzantine attacks by selecting as a global model the local model with the smallest sum of Euclidean distances to {\em all} the other local models.
% Although powerful, Krum requires the server to know (or, at least, estimate) the number of malicious FL clients upfront, which is generally impossible in a realistic attack scenario. %
% Moreover, Krum may become ineffective for complex, high-dimensional model parameter spaces due to the curse of dimensionality.
% Bulyan~\cite{mhamdi2018pmlr} tries to overcome this issue by combining Krum with a variant of Trimmed Mean.
% % Data-driven outlier detection
% Other strategies use data-driven outlier detection techniques -- e.g., via K-means clustering~\cite{shen2016acm} -- to spot potential malicious local model updates. 
% %For instance, Shen et al. propose to cluster local model updates with K-means and thus identify outliers.
%
% % Other techniques
% As far as the server is concerned, any local model received can be from a potential malicious client. 
% FLTrust~\cite{cao2020fltrust} assumes the server acts as a client, i.e., trains a local model on an additional {\em trustworthy} dataset at the server's end and compares it against all the local models from other clients. 
% This way, the server can rely on some ``source of trust'' when discarding potentially malicious clients.
%\\
% Limitations of existing Byzantine-resilient strategies
Unfortunately, existing defense mechanisms either rely on simple heuristics (e.g., Trimmed Mean and FedMedian by~\cite{yin2018icml}) or need strong and unrealistic assumptions to work effectively (e.g., foreknowledge or estimation of the number of malicious clients in the FL system, as for Krum/Multi-Krum~\cite{blanchard2017nips} and Bulyan~\cite{mhamdi2018pmlr}, which, however, cannot exceed a fixed threshold).
Furthermore, outlier detection methods using K-means clustering~\cite{shen2016acm} or spectral analysis like DnC~\cite{shejwalkar2021ndss} do not directly consider the temporal evolution of local model updates received.
Finally, strategies like FLTrust~\cite{cao2020fltrust} require the server to collect its own dataset and act as a proper client, thereby altering the standard FL protocol.
\\
% OLD, LONG VERSION
% Overall, existing Byzantine-resilient strategies are either simple heuristics (e.g., FedMedian) or, if they are more complex, they rely on strong and unrealistic assumptions to work effectively (e.g., knowing the number of malicious clients in the FL system in advance, as for Krum and alike).
% Furthermore, data-driven outlier detection methods do not consider the temporary evolution of local model updates received (e.g., K-means clustering). 
% Finally, strategies like FLTrust requires the server to collect its own dataset and act as a proper client, thereby altering the standard FL protocol.
%
% Description of the proposed method
This work introduces a novel pre-aggregation \textit{filter} robust to untargeted model poisoning attacks. Notably, this filter $(i)$ operates without requiring prior knowledge or constraints on the number of malicious clients and $(ii)$ inherently integrates temporal dependencies. 
The FL server can employ this filter as a preprocessing step before applying \textit{any} aggregation function, be it standard like FedAvg or robust like Krum or Bulyan.
Specifically, we formulate the problem of identifying corrupted updates as a multidimensional (i.e., matrix-valued) time series anomaly detection task. 
The key idea is that legitimate local updates, resulting from well-calibrated iterative procedures like stochastic gradient descent (SGD) with an appropriate learning rate, show \textit{higher predictability} compared to malicious updates. This hypothesis stems from the fact that the sequence of gradients (thus, model parameters) observed during legitimate training exhibit regular patterns, as validated in Section~\ref{subsec:intuition}. %until convergence. 
%This regularity may be more pronounced for smooth convex loss functions, but it can still be captured within an appropriate time window, even for more complex and convoluted loss surfaces. 
%We provide evidence of this claim in Appendix~B, where we show that the average mutual information (i.e., ``predictability''), calculated over pairs of legitimate model updates sent at different FL rounds, is significantly higher than the corresponding computation for a malicious client.
\\
Inspired by the matrix autoregressive (MAR) framework for multidimensional time series forecasting~\cite{chen2021je}, we propose the FLANDERS ({\em \textbf{F}ederated \textbf{L}earning meets \textbf{AN}omaly \textbf{DE}tection for a \textbf{R}obust and \textbf{S}ecure}) filter.
The main advantages of FLANDERS over existing strategies like FLDetector~\cite{zhao2020multivariate} are its resilience to large-scale attacks, where $50\%$ or more FL participants are hostile, and the capability of working under realistic non-iid scenarios.
We attribute such a capability to two key factors: $(i)$ FLANDERS works without knowing a priori the ratio of corrupted clients, and $(ii)$ it embodies temporal dependencies between intra- and inter-client updates, quickly recognizing local model drifts caused by evil players. Below, we summarize our main contributions:

\begin{itemize}
\item[{\em(i)}]
We provide empirical evidence that the sequence of models sent by legitimate clients is more predictable than those of malicious participants performing untargeted model poisoning attacks.
\\
\item[{\em(ii)}] 
We introduce FLANDERS, the first pre-aggregation filter for FL robust to untargeted model poisoning based on multidimensional time series anomaly detection.
\\
\item[{\em(iii)}] 
We integrate FLANDERS into Flower,\footnote{\scriptsize{\url{https://flower.dev/}}} a popular FL simulation framework for reproducibility.
\\
\item[{\em(iv)}] 
We show that FLANDERS improves the robustness of the existing aggregation methods under multiple settings: different datasets, client's data distribution (non-iid), models, and attack scenarios.
\\
\item[{\em(v)}] 
We publicly release all the implementation code of FLANDERS along with our experiments.\footnote{\scriptsize{\url{https://anonymous.4open.science/r/flanders_exp-7EEB}}}
\end{itemize}

% Paper's structure and organization
The remainder of the paper is structured as follows. %some related work and the current state-of-the-art solutions to security issues that FL entails. 
Section~\ref{sec:background} covers background and preliminaries. 
In Section~\ref{sec:related}, we discuss related work.
Section~\ref{sec:problem} and Section~\ref{sec:method} describe the problem formulation and the method proposed. % to tackle it. 
Section~\ref{sec:experiments} gathers experimental results. %, and Section~\ref{sec:limitations} discusses some limitations of this work.
Finally, we conclude in Section~\ref{sec:conclusion}.
 %discusses the limitations of this work and draws future research directions.
%reports conclusions and draws perspectives for future research directions.

%%%%%%% OLD %%%%%%%
%to overcome the resilience of Byzantine failures in distributed Stochastic Gradient Descent computations. 
% The strength of Krum is its time complexity, which is linear in the gradient dimension. 
% However, the robustness of the approach is guaranteed for gradient-based learning applications only when the majority of the clients are not compromised. 
% Besides, the aggregation mechanism of Krum, as well as that of similar methods, is robust from a coarse-grained perspective and does not provide solutions to errors and perturbations that may occur at inference time.
%A related approach to~\cite{blanchard2017nips} is the work of Su et al.~\cite{su2016dc}. Here, the authors propose an iterated approximate agreement to tackle a multi-layer scenario attacked by Byzantine agents. 
%However, the method works efficiently on the sole discrete context and it is inapplicable to continuous state environments.
%\gabri{Maybe, we should just talk about the main limitations of existing countermeasures without digging into their details (or, we can just mention Krum as this is the most popular one). I will move the description of all these methods to the Related Work section.}

\section{Analysis} \label{section_analysis}
In this section, we focus on the torus case, i.e. $\X = \Pi^d$ is a box with the periodic boundary condition.
This is a typical setting considered in the literature as the universal function approximation of NNs only holds over a compact set. Moreover, the boundary integral resulting from integration by parts vanishes in this setting, making it amenable for analysis purposes.
For completeness, we provide a discussion on the unbounded case, i.e. $\X = \sR^d$ in the Appendix \ref{appendix_unbounded}, which requires additional regularity assumptions.
% \red{Mention that we focus on the torus case.}
% We start by defining the modulated (interaction) energy and modulated free energy for two probability densities $\rho, \bar \rho$.  
% Again $\mathcal{X}$ denotes the underlying space \red{which can be the whole space $\mathbb{R}^d$} or the torus $\Pi^d $ which can be identified with $[-L, L]^d$ with the periodic boundary condition and $L>0$. 
Given the MVE (\ref{eqn_MVE}), if $K$ is bounded, it is sufficient to choose the Lyapunov functional $L(\rho_t^f, \bar \rho_t)$ as the KL divergence (please see Theorem \ref{theorem_bounded_K} in the appendix). But for the singular Coulomb kernel, we need also to consider the modulated energy as in \citep{serfaty2020mean}
\begin{equation}
    \text{(Modulated Energy)}\quad 
	\label{DefModEnergy}
	F(\rho, \bar \rho) \defi \frac 1 2 \int_{\mathcal{X}^2} g(x- y) \ud (\rho - \bar \rho)(x) \ud (\rho - \bar \rho )(y), 
\end{equation}
where $g$ is the fundamental solution to the Laplacian equation in $\mathbb{R}^d$, i.e. $- \Delta g = \delta_0$, and the Coulomb interaction reads $K= -\nabla g$ (see its closed form expression in equation (\ref{eqn_coulomb_interaction})). If we are only interested in the deterministic  dynamics with Coulomb interactions, i.e. $\nu =0$ in equation (\ref{eqn_MVE}), it suffices to choose $L(\rho_t^f, \bar \rho) $ as  $F(\rho_t^f, \bar \rho_t)$ (please see Theorem \ref{ThmCoul}). But if we consider the system with  Coulomb interactions and  diffusions, i.e. $\nu >0$, we shall combine the KL divergence and the modulated energy to form the modulated free energy as in \cite{bresch2019modulated}, which reads 
\begin{equation}
    \text{(Modulated Free Energy)}\quad 
	\label{DefModFree}
	E(\rho, \bar \rho) \defi \nu \KL (\rho, \bar \rho) +  F(\rho, \bar \rho). 
\end{equation}
This definition agrees with the physical meaning that ``Free Energy = Temperature $\times $ Entropy + Energy", and we note that the temperature is proportional to the diffusion coefficient $\nu$. We remark also for two probability densities $\rho$ and $\bar \rho$, $F(\rho, \bar \rho) \geq 0$ since by looking in the Fourier domain $F(\rho, \bar \rho) = \int \hat g(\xi) |\widehat{\rho - \bar \rho}(\xi)|^2  \ud \xi \geq 0$ as $\hat g(\xi) \geq 0$. Moreover, $F(\rho, \bar \rho)$ can be regarded as a negative Sobolev norm for $\rho- \bar \rho$, which metricizes weak convergence.\\
To obtain our main stability estimate, we first obtain the time evolution of the KL divergence.  
%%%% Evolution of the KL divergence 
\begin{lemma}[Time Evolution of the KL divergence] \label{TimeEvolKLMV} 
	Given the hypothesis velocity field $f=f(t, x) \in C^1_{t, x}$. Assume that $(\rho_t^f)_{t \in [0, T]}$ and $(\bar \rho_t)_{t \in [0, T]}$ are classical solutions to equation (\ref{eqn_CE}) and equation (\ref{eqn_MVE_CE}) respectively.  It holds that (recall the definition of $\delta_t$ in equation (\ref{eqn_perturbation})) 
	\[
	\frac{\ud }{\ud t} \int_{\mathcal{X}} \rho^f_t \log \frac{\rho^f_t }{\bar \rho_t} = - \nu \int_{ \X}\rho^f_t |\nabla \log \frac{\rho^f_t}{\bar \rho_t}|^2 +  \int_{\X} \rho^f_t K * (\rho^f_t - \bar \rho_t ) \cdot \nabla  \log  \frac{\rho^f_t }{\bar \rho_t} +  \int_{\X} \rho^f_t \delta_t \cdot \nabla \log \frac{\rho^f_t }{\bar \rho_t }, 
	\]
	where $\X$ is the tours $\Pi^d$. All the integrands are evaluated at $\vx$. 
\end{lemma}

We refer the proof of this lemma and all other lemmas and theorems in this section to the appendix \ref{detailed proof}. We remark that to have the existence of classical solution $(\bar \rho_t)_{t \in [0, T]}$, we definitely need the regularity assumptions on $-\nabla V$ and on $K$. But the linear term $- \nabla V $ will not contribute to the evolution of the relative entropy. See \citep{jabin2018quantitative} for detailed discussions. \\
%%%% Evolution of the Modulated Energy 
Similarly, we have the time evolution of the modulated energy as follows.
\begin{lemma}[Time evolution of the modulated energy] \label{ModuEnergyEvo} Under the same assumptions as in Lemma \ref{TimeEvolKLMV}, given the diffusion coefficient $\nu \geq 0$, it holds that (recall the definition of $\delta_t$ in equation (\ref{eqn_perturbation})) 
	\[
	\begin{split}
		\frac{\ud }{\ud t } F(\rho_t^f, \bar \rho_t)&  =  - \int_{\mathcal{X}} \rho_t^f \|K *(\rho_t^f - \bar \rho_t)\|^2 - \int_{\mathcal{X}} \rho_t^f \, \delta_t \cdot K * (\rho_t^f - \bar \rho_t ) + \nu \int_{\mathcal{X}} \rho^f_t \, K * (\rho_t^f - \bar \rho_t )\cdot \nabla \log \frac{\rho_t^f}{\bar \rho_t} \\
		&  - \frac{1}{2} \int_{\mathcal{X}^2} K(x-y) \cdot \Big( \mathcal{A}[\bar \rho_t](x) - \mathcal{A}[\bar \rho_t](y) \Big) \ud (\rho_t^f - \bar \rho_t )^{\otimes 2 }(x, y) \\
	\end{split}
	\]
	where we recall that the operator $\gA$ is defined in equation (\ref{eqn_operator_A}).
\end{lemma}






%%%%%%% The 2D Navier-Stokes case  
By  Lemma \ref{TimeEvolKLMV} and careful analysis, in particular by rewriting the Biot-Savart law in the divergence of a bounded matrix-valued function (\ref{eqn_K_as_divergence}), we obtain the following estimate for the 2D NSE.

\begin{theorem}[Stability estimate of the 2D NSE] \label{NSMainEstimate}	
	% Consider the 2D NSE in the vorticity formulation (\ref{eqn_MVE}) with $V=0$. 
    Notice that when $K$ is the Biot-Savart kernel, $\udiv K =0$. Assume that the initial data $\bar \rho_0 \in C^3(\Pi^d)$ and there exists $c>1$ such that $\frac 1 c \leq \bar \rho_0 \leq c$.  Assume further the hypothesis velocity field $f(t, x) \in C^1_{t, x}$. Then it holds that 
	\[
	\sup_{t \in [0, T]} \int_{\Pi^d} \rho_t^f \log \frac{\rho_t^f}{\bar \rho_t} \ud x \leq \frac{e^C}{\nu}  R(f), 
	\]
	where $C = \int_0^\infty     M(t) \ud t < \infty$ with 
	$M(t) \defi \|\nabla \log \bar \rho_t\|_{L^\infty}^2/2\nu + 2\Big\| {\nabla^2 \bar \rho_t}/{\bar \rho_t} \Big\|_{L^\infty}$. 
	
\end{theorem} 
We remark that given $\bar \rho_0$ is smooth enough and fully supported on $\X$, one can propagate the regularity to finally show the finiteness of $C$.
See detailed computations as in \cite{guillin2021uniform}. 
We give the complete proof in the appendix \ref{detailed proof}. This theorem tells us that as long as $R(f)$ is small, the KL divergence between $\rho_t^f$ and $\bar \rho_t$ is small and the control is uniform in time $t \in [0, T]$ for any $T$. Moreover, we highlight that $C$ is independent of $T$, and our result on the NSE is significantly better than the average-in-time and exponential-in-$T$ results from \citep{deerror}.\\
% We state this theorem on torus for simplicity and \red{one may expect similar result on $\mathbb{R}^d$}. \red{Also the $C^\infty$ condition can be relaxed for instance to $C^2$}. 
%%%%%  Remark on The case with bounded interactions $K \in L^\infty
%%%% Leave in the appendix  
%%%% The deterministic case with Coulomb 
%%%% Lemma of the evolution of the modulated free energy 
To treat the MVE (\ref{eqn_MVE}) with Coulomb interactions, we exploit the time evolution of the modulated free energy $E(\rho_t^f, \bar \rho_t)$. Indeed, combining Lemma \ref{TimeEvolKLMV} and Lemma \ref{ModuEnergyEvo}, we arrive at the following identity.  
\begin{lemma}[Time evolution of the modulated free energy]\label{TimeEvoMFE} 
	Under the same assumptions as in Lemma \ref{TimeEvolKLMV}, one has (recall the definitions of $\delta_t$ and $\gA$ in (\ref{eqn_perturbation}) and (\ref{eqn_operator_A}) respectively) 
	\[
	\begin{split}
		\frac{\ud }{\ud t } E(\rho_t^f, \bar \rho_t)  & = -\int_{\mathcal{X}} \rho_t^f  \Big|K * (\rho_t^f - \bar \rho_t) - \nu \nabla \log \frac{\rho_t^f }{\bar \rho_t}\Big|^2 - \int_{\mathcal{X}} \rho_t^f \, \delta_t \cdot \Big( K * (\rho_t^f - \bar \rho_t ) - \nu \nabla \log \frac{\rho_t^f}{\bar \rho_t }\Big) \\
		&  - \frac{1}{2} \int_{\mathcal{X}^2} K(x-y) \cdot \Big( \mathcal{A}[\bar \rho_t](x) - \mathcal{A}[\bar \rho_t](y) \Big) \ud (\rho_t^f - \bar \rho_t )^{\otimes 2 }(x, y).
	\end{split}
	\]
	% where we recall that the operator $\gA$ is defined in \eqref{eqn_operator_A}.
\end{lemma}
\vspace{-1mm}
%%%%%%%%%%%%%%Main theorem of the Coulomb case 
Inspired by the mean-field convergence results as in \cite{serfaty2020mean} and \cite{bresch2019modulated}, we finally can control the growth of $E(\rho_t^f, \bar \rho_t)$ in the case when $\nu >0$,  and $F(\rho_t^f, \bar \rho_t)$ in the case when $\nu =0$. Note also that $E(\rho_t^f, \bar \rho_t )$ can also control the KL divergence when $\nu >0$. 
\begin{theorem} [Stability estimate of MVE with Coulomb interactions] \label{ThmCoul}
	Assume that for $t \in [0, T]$, the underlying velocity field $\mathcal{A}[\bar \rho_t](x)$ is Lipschitz  in $x$ and
	$\sup_{t \in [0, T]} \|\nabla \mathcal{A}[\bar \rho_t](\cdot)\|_{L^\infty} = C_1  < \infty.$ 
	Then there exists $C>0$ such that 
	\[
	\sup_{t \in [0, T]} \nu\,  \KL (\rho_t^f, \bar \rho_t) \leq \sup_{t \in [0, T]} E(\rho_t^f, \bar \rho_t) \leq \exp(C C_1 T) R(f).  
	\]
	In the deterministic case when $\nu =0$, under the same assumptions, it holds that 
	\[
	\sup_{t \in [0, T]} F( \rho_t^f, \bar \rho_t) \leq \exp(CC_1T ) R(f). 
	\]
\end{theorem}
\vspace{-1mm}
Recall the definition of the operator $\gA$ in \eqref{eqn_operator_A}. Given that $\mathcal{X}= \Pi^d$, and $\bar \rho_0$ is smooth enough and bounded from below, one can propagate regularity to obtain the Lipschitz condition for $\mathcal{A}[\bar \rho_t]$. See the proof and the discussion on the Lipschitz assumptions on $\mathcal{A}[\bar \rho_t](\cdot)$ in the appendix \ref{detailed proof}. 
\paragraph{Approximation Error of Neural Network}
Theorems \ref{NSMainEstimate} and \ref{ThmCoul} provide the error estimation guarantee for the proposed \EINN\ loss (\ref{eqn_self_consistency_potential}).
Suppose that we parameterize the velocity field $f=f_\theta$ with an NN parameterized by $\theta$, as we did in Section \ref{section_NN_parameterization} and let $\tilde f$ be the output of an optimization procedure when $R(f_\theta)$ is used as objective.
In order the explicitly quantify the mismatch between $\rho^{\tilde f}_t$ and $\bar \rho_t$, we need to quantify two errors: (i) Approximation error, reflecting how well the ground truth solution can be approximated among the NN function class of choice; (ii) Optimization error, involving minimization of a highly nonlinear non-convex objective. 
In the following, we show that for a function class $\gF$ with sufficient capacity, there exists at least one element $\hat f\in\gF$ that can reduce the loss function $R(\hat f)$ as much as desired.
We will not discuss how to identify such an element in the function class $\gF$ as it is independent of our research and remains possibly the largest open problem in modern AI research.
To establish our result, we make the following assumptions.
\begin{assumption} \label{ass_appendix_initial}
	$\rho_0$ is sufficiently regular, such that $\nabla \log \rho_0 \in \gL^\infty(\X)$ and $\bar f_t  = \mathcal{A}[\bar \rho_t] \in W^{2,\infty}(\X)$. $\nabla V$ is Lipschitz continuous. Here $W^{2,\infty}(\X)$ stands for the Sobolev norm of order $(2, \infty)$ over $\X$.
\end{assumption}
\vspace{-1mm}
We here again need to propagate the regularity for $f_t$ at least for a time interval $[0, T]$. It is easy to do so for the torus case, but for the unbounded domain, there are some technical issues to be  overcome. Similar assumptions are also needed in some mathematical works for instance in \cite{jabin2018quantitative}. 
% [cite some papers]
% \begin{remark}
%     For the torus case, if $\rho_0$ is bounded from above and below, and chosen to be in $C^3$, we can propagate the regularity of $\rho_0$ to $\bar \rho_t$ for $t \in [0, T]$. We can then obtain $\bar f_t \in \gC^3(\X)$.
% \end{remark}
We also make the following assumption on the capacity of the function class $\gF$, which is satisfied for example by NNs with tanh activation function \citep{DERYCK2021732}.
\begin{assumption} \label{ass_appendix_approximation}
	The function class is sufficiently large, such that there exists $\hat f \in \gF$ satisfying $\hat f_t \in \gC^3(\X)$ and $\|\hat f_t - \bar f_t\|_{W^{2, \infty}(\X)} \leq \epsilon$ for all $t\in[0, T]$.
\end{assumption}
\begin{theorem} \label{thm_approximation_error_NN}
	Consider the case where the domain is the torus. 
	Suppose that Assumptions \ref{ass_appendix_initial} and \ref{ass_appendix_approximation} hold. 
    For both the Coulomb and the Biot-Savart cases, there exists $\hat f\in\gF$ such that $R(\hat f) \leq C(T)\cdot(\epsilon \cdot\ln 1/\epsilon)^2$, where $C(T)$ is some constant independent of $\epsilon$. Here $R$ is the \EINN\ loss (\ref{eqn_self_consistency_potential}).
\end{theorem}
% \red{Maybe mentioning the difficulty in proving this theorem to promote the novelty of the result.}
% \red{Can we derive the results for unbounded domain?}
% \red{This is just the result for the Coulomb case. We should also prove the NSE case.}
The major difficulty to overcome is the lack of Lipschitz continuity due to the singular interaction. We successfully address this challenge by establishing that the contribution of the singular region $(\|\vx\|\leq\epsilon)$ to $R(\hat f)$ can be bounded by $O((\epsilon \log \frac{1}{\epsilon})^2)$.
Please see the detailed proof in Appendix \ref{appendix_approximation_error_NN}.



%1. Previous analysis provides control of the solution quality by $R(f)$
%2. The next question is how small $R(f)$ can be
%3. Two errors: approximation error and optimization error. The latter is an open question. In this section, we focus on the approximation error.
%4. Conclusions on NN approximation in Sobolev norm
%5. List assumptions
%6. describe conclusion.

% Discuss the long-time guarantee (without exponential in T, under additional assumptions)

% Periodic solution to Taylor-Green vortex

% Revise the discussion on "expected"




\section{Observations on Class-wise Robustness}
In this section, we present our empirical observations on the class-wise robustness of models adversarially trained under different configurations. Taking vanilla AT \cite{madry2017towards} and TRADES \cite{zhang2019theoretically} as examples, we compare two key factors in the training configurations: the perturbation margin $\epsilon$ in vanilla AT and the regularization $\beta$ in TRADES.
We also reveal the fluctuation effect of the worst class robustness during the training process, which has a significant impact on the robust fairness in adversarial training.

\subsection{Different Margins}
\label{margin analysis}

Following the vanilla AT~\cite{madry2017towards}, we train 8 models on the CIFAR10 dataset~\cite{krizhevsky2009learning} with $l_\infty$-norm perturbation margin $\epsilon$ from $2/255$ to $16/255$ and analyze their overall and class-wise robustness. 
\begin{figure*}[t]
    \centering
    \begin{tabular}{ccc}
    \includegraphics[width=0.3\textwidth]{fig/compare_epsilon_overall.png} &
    \includegraphics[width=0.3\textwidth]{fig/compare_epsilon_cw_101_120.png} &
    \includegraphics[width=0.3\textwidth]{fig/compare_epsilon_cw_181_200.png}
    \\
    (a) &  
    (b) & 
    (c)
    \end{tabular}
    \vspace{-0.15 in}
    \caption{Comparison of overall and class-wise robustness of models adversarially trained on CIFAR10 with different perturbation margin $\epsilon$. (a): Overall robust accuracy with different perturbation margin $\epsilon$ from 2/255 to 16/255. (b): Average class-wise robust accuracy at epoch $101-120$ (each line represents a class). (c): Average class-wise robust accuracy at epoch $181-200$ (each line represents a class).}
    \label{fig:margin}
\end{figure*}

The comparison of overall robustness is shown in Fig.~\ref{fig:margin}(a). The robustness is evaluated under PGD-10 attack bounded by $\epsilon_0=8/255$, which is commonly used for robustness evaluation.
Intuitively, using a larger margin can lead to better robustness. For $\epsilon<\epsilon_0$, the attack is too weak and hence the robust accuracy of the trained model is not comparable with $\epsilon\ge \epsilon_0$. 
However, for the three models trained with $\epsilon>\epsilon_0$, although their robustness outperforms the case of $\epsilon=\epsilon_0$ at the last epoch, they do not make significant progress on the best-case robustness (around 100-th epoch). 

We take a closer look at this phenomenon by investigating their class-wise robustness in Fig.~\ref{fig:margin}(b) and Fig.~\ref{fig:margin}(c). For each class, we calculate the average class-wise robust accuracy among the 101$-$120-th epochs (where the model performs the best robustness) and 181$-$200-th epochs, respectively. From Fig.~\ref{fig:margin}(b), we can see that a larger training margin $\epsilon$ does not necessarily result in better class-wise robustness. For the \textit{easy} classes which perform higher robustness, their robustness monotonously increase as $\epsilon$  enlarges from $2/255$ to $16/255$. By contrast, for the \textit{hard} classes (especially class $2, 3, 4$), their robustness drop when $\epsilon$ enlarges from $8/255$. However, for the last several checkpoints in Fig.~\ref{fig:margin}(c), we can see a consistent increase on class-wise robustness when the $\epsilon$ enlarges. Revisiting the overall robustness, we can summarize that the class-wise robustness is boosted mainly by reducing the robust over-fitting problem in the last checkpoint.
This can explain why Fair Robust Learning (FRL)~\cite{xu2021robust} can improve robust fairness  by enlarging the margin for the hard classes, since the model reduces the over-fitting problem on these classes. Considering the overall robustness is lower in the last checkpoint (robust fairness is better though), we hope to improve the best-case robust fairness in the situation of a relatively high overall robustness. 

In summary, larger perturbation is harmful to the hard classes in the best case, while can marginally improve the class-wise robustness in the later stage of training. For easy classes, larger perturbation is useful at whatever the best and last checkpoints. Therefore, a specific and proper perturbation margin is needed for each class. 

\subsection{Different Regularizations}
\label{regularizations}
In this section, we also conduct a similar experiment on the selection of \textit{robustness regularization} $\beta$ in TRADES. We compare models trained on CIFAR10 with $\beta$ from 1 to 8, and plot the average class-wise robust and clean accuracy among the $151-170$-th epochs (where TRADES performs the best performance) in Fig.~\ref{fig:trades}. We can see that bias more weight on robustness (use larger $\beta$) cause different influences among classes. Specifically, for \textit{easy classes}, improving $\beta$ can improve their robustness at the cost of little clean accuracy reduction, while for \textit{hard classes} (\textit{e.g.}, classes $2,3,4$), improving $\beta$ can only obtain limited robustness improvement but drop clean accuracy significantly. 

\begin{figure}[h]
    \centering
    \begin{tabular}{cc}
         \includegraphics[width=0.22\textwidth]{fig/compare_trades.png}
         & 
         \includegraphics[width=0.22\textwidth]{fig/compare_trades_clean.png}
         \\
         (a) & (b)
    \end{tabular}
    \vspace{-0.15 in}
    \caption{Comparison of class-wise robustness trained by TRADES with different robustness regularization parameters $\beta$. (a) Class-wise robust accuracy. (b) Class-wise clean accuracy.}
    \label{fig:trades}
\end{figure}

\begin{figure}[!htbp]
    \centering
    \begin{tabular}{cc}
        \includegraphics[width=0.22\textwidth]{fig/worst_and_overall.png} &  
        \includegraphics[width=0.22\textwidth]{fig/EMA_worst_and_overall.png}
        \\
        (a) & (b)
    \end{tabular}
    \vspace{-0.15 in}
    \caption{Comparison of overall robustness, the worst class robustness, and the absolute variation of the worst class robustness between adjacent checkpoints. (a): Vanilla AT. (b): AT with fairness aware weight averaging (FAWA), start from epoch 50.}
    \label{fig:worst and overall}
\end{figure}

This result is consistent with the Theorem \ref{theorem:compare}. Recall that in the toy model, hard class $y=-1$ costs more clean accuracy to exchanges for little robustness improvement than easy class $y=+1$. Therefore, similar to the analysis on perturbation margin $\epsilon$, we also point out that there exists a proper $\beta_y$ for each class.





\subsection{Fluctuation Effect}
\label{fluctuate}
In this section, we reveal an intriguing property regarding the fluctuation of class-wise robustness during adversarial training. In Fig.~\ref{fig:worst and overall}(a), we plot the overall robustness, the worst class robustness, and the variance of the worst robustness between adjacent epochs in vanilla adversarial training. While the overall robustness tends to be more stable between adjacent checkpoints (except when the learning rate decays), the worst class robustness fluctuates significantly. Particularly, many adjacent checkpoints between the $101-120$-th epochs exhibit a nearly 10\% difference in the worst class robustness, while changes in overall robustness are negligible (less than 1\%).
Therefore, previously widely used selecting the best checkpoint based on overall robustness may result in an extremely unfair model. Taking the plotted training process as an example, the model achieves the highest robust accuracy of 53.2\% at the 108-th epoch, which only has 23.5\% robust accuracy on the worst class. In contrast, the checkpoint at epoch 110, which has 52.6\% overall and 28.1\% worst class robust accuracy, is preferred when considering fairness.


\section{Class-wise Calibrated Fair Adversarial Training}
With the above analysis, we introduce our proposed \textbf{C}lass-wise  calibrated \textbf{F}air \textbf{A}dversarial training (CFA) framework in this section.
Overall, the CFA framework consists of three main components: Customized Class-wise perturbation Margin (CCM), Customized Class-wise Regularization (CCR), and Fairness Aware Weight Averaging (FAWA).
The CCM and CCR customize appropriate training configurations for different classes, and FAWA modifies weight averaging to improve and stabilize fairness.

\subsection{Class-wise Calibrated Margin (CCM)}
\label{CCM}
In Sec.~\ref{margin analysis}, we have demonstrated that different classes prefer specific perturbation margin $\epsilon$ in adversarial training. However, it is impractical to directly find the optimal class-wise margin. Inspired by a series of instance-wise adaptive adversarial training approaches~\cite{ding2018mma,wang2019improving,balaji2019instance}, which customize train setting for each instance according to the model performance on current example, we propose to leverage the class-wise training accuracy as the measurement of difficulty.

Suppose the $k$-th class achieved train robust accuracy $t_k\in[0,1]$ in the last training epoch. 
In the next epoch, we aim to update the margin $\epsilon_k$ for class $k$ based on $t_k$.
Based on our analysis in Sec.~\ref{margin analysis}, we consider using a relatively smaller margin for the hard classes which are more vulnerable to attacks, and identify the \textit{difficulty} among classes by the train robust accuracy tracked from the previous epoch.
To avoid $\epsilon_k$ too small, we add a hyper-parameter $\lambda_1$ (called \textit{base perturbation budget}) on all $t_k$ and set the calibrated margin $\epsilon_k$ by multiply the coefficient on primal margin $\epsilon$:
\begin{equation}
    \epsilon_k\gets (\lambda_1 +  t_k)\cdot \epsilon,
    \label{ccm formula}
\end{equation}
where $\epsilon$ is the original perturbation margin, \textit{e.g.}, $8/255$ that is commonly used for CIFAR-10 dataset. 
Note that the calibrated margin $\epsilon_k$ can adaptively converge to find the proper range during the training phase, for example, 
 if the margin is too small for class $k$, the model will perform high train robust accuracy $t_k$ and then increase $\epsilon_k$ by schedule (\ref{ccm formula}). 

\subsection{Class-wise Calibrated Regularization (CCR)}
We further customize different robustness regularization $\beta$ of TRADES for different classes.
Recall the objective function (\ref{TRADES}) of TRADES, we hope the hard classes tend to bias more weight on its clean accuracy. Still, we measure the difficulty by the train robust accuracy $t_k$ for class $k$, 
and propose the following calibrated robustness regularization $\beta_k$:
\begin{equation}
    \beta_k\gets (\lambda_2 + t_k) \cdot \beta.
\end{equation}
where $\beta$ is the originally selected parameter.
The objective function (\ref{TRADES}) can be rewritten as:
\begin{equation}
\label{ccw objective}
    \mathcal L_{\boldsymbol\theta}(\beta;x,y)=
    \frac{\mathcal L(\boldsymbol\theta;x,y) + \beta_y\max\limits_{\|{x'}-{x}\|\le\epsilon} \mathcal{K} (f_{\boldsymbol\theta}({x}), f_{\boldsymbol\theta}({x'}))}{1+\beta_y}.
\end{equation}

To balance the weight between different classes, we add a denominator $1+\beta_y$ since $\beta_y$ is distinct among classes. Therefore, for the hard classes which have lower $\beta_y$ tend to bias higher weight $\frac 1 {1+\beta_y}$ on its natural loss $\mathcal L(\boldsymbol\theta;x,y)$. Note that simply replacing $\epsilon$ in (\ref{ccw objective}) with $\epsilon_k$ can combine the calibrated margin with this calibrated regularization.  On the other hand, for general adversarial training algorithms, our calibrated margin schedule (\ref{ccm formula}) can also be combined.

\subsection{Fairness Aware Weight Average (FAWA)}
\label{sec: fawa}
As plotted in Fig.~\ref{fig:worst and overall}(a), the worst class robustness changes largely, among which part of checkpoints performs extremely poor robust fairness. 
Previously, there are a series of weight averaging methods to make the model training stable, \textit{e.g.}, exponential moving average (EMA) \cite{DBLP:conf/uai/IzmailovPGVW18,wang2022self}, thus we hope to further improve the worst class robustness by fixing the weight average algorithm.

Inspired by the large fluctuation of the robustness fairness among checkpoints, we consider eliminating the \textit{unfair} checkpoints out in the weight averaging process. To this end, we propose a \textit{Fairness Aware Weight Average (FAWA)} approach, which sets a threshold $\delta$ on the worst class robustness of the new checkpoint in the EMA process. Specifically, we extract a validation set from the dataset, and each checkpoint is adopted in the weight average process if and only if its worst class robustness is higher than $\delta$. Fig.~\ref{fig:worst and overall}(b) shows the effect of the proposed FAWA. The difference between adjacent epochs is extremely small (less than 1\%), and the overall robustness also outperforms vanilla AT.

\begin{algorithm}[h]
    \caption{TRADES with CFA}
	\label{alg:cfa}
	\KwIn{A DNN classifier $f_{\boldsymbol\theta}(\cdot)$ with parameter $\boldsymbol\theta$; Train dataset $D=\{(x_i, y_i)\}_{i=1}^N$; Batch size $m$; Initial perturbation margin $\epsilon$ and robustness regularization $\beta$; Train epochs $N$; Batch size $m$; Learning rate $\eta$; Weight average decay rate $\alpha$; Fairness threshold $\delta$}
	\KwOut{A fair and robust DNN classifier $\bar{f}_{\bar{\boldsymbol\theta}}(\cdot)$}
	\tcc{Initialize parameters and datasets}
    Initialize $\boldsymbol\theta\gets\boldsymbol\theta_0, \bar{\boldsymbol\theta}\gets\boldsymbol\theta$\;
    Split $D=D_{\text{train}}\cup D_{\text{valid}}$\;
    \For{$y\in\mathcal Y$}{
    \tcc{Initialize $\epsilon_y$ and $\beta_y$}
    $\epsilon_y\gets\epsilon, \beta_y\gets\beta$\;
    }
    
    \For{$T\gets 1,2,\cdots N$}{
        \For{Every minibatch $(x,y)$ in $D_{\text{train}}$}{
        \tcc{Use $\epsilon_y$ and $\beta_y$ to train}
        $x'\gets \arg\max\limits_{x'\in\mathcal{B}(x,\epsilon_y)}\mathcal K (f_{\boldsymbol\theta}({x}), f_{\boldsymbol\theta}({x'}))$\;
        $\boldsymbol\theta \gets \boldsymbol\theta - \eta\nabla_{\boldsymbol\theta}\mathcal L_{\boldsymbol\theta}(\beta_y;x,y)$\;
        }
        \For{$y\in\mathcal Y$}{
        $t_y\gets Train\_Acc(f_\theta, T)$\;
        \tcc{Update $\epsilon_y,\beta_y$ with $t_y$}
        $\epsilon_y\gets (\lambda_1 + t_k)\cdot\epsilon$\;
        $\beta_y\gets (\lambda_2 + t_k)\cdot \epsilon$\;}
        \tcc{Fairness Aware Weight Average}
        \If{$\min_{y\in\mathcal Y} \mathcal{R}_{y}(f_{\boldsymbol\theta}, D_{\text{valid}})\ge \delta$}{
        \vspace{0.1cm}
        $\bar{\boldsymbol{\theta}}\gets \alpha\bar{\boldsymbol\theta}+(1-\alpha){\boldsymbol\theta}$\;}
    }
    \textbf{return} $\bar{f}_{\bar{\boldsymbol\theta}}$\;

\end{algorithm}

\subsection{Discussion}

Overall, by combining the above components, we accomplish our CFA framework. An illustration of incorporating CFA to TRADES is shown in Alg.~\ref{alg:cfa}. Note that for other methods like AT, we can still incorporate CFA by removing the CCR schedule specified for TRADES. Moreover, we discuss the difference between our proposed CFA and other works. 

\noindent \textbf{Comparison with Fair Robust Learning (FRL)~\cite{xu2021robust}.}
Here we highlight the differences between our CFA framework and Fair Robust Learning (FRL), the only existing adversarial training algorithm designed to improve the fairness of class-wise robustness. The FRL framework consists of two components: remargin and reweight. Initially, a robust model is trained, and a fairness constraint on the difference of robustness among classes is set. When the constraint is violated, the model is fine-tuned persistently by increasing the perturbation bound $\epsilon_k$ and weighting the loss of the hard classes. Although CFA also includes adaptive margin and regularization weight schedules, our work is fundamentally distinct from FRL. Firstly, as discussed in Sec.~\ref{margin analysis}, a larger margin only mitigates the robust over-fitting problem but does not provide higher peak performance. In contrast, our approach aims to customize the proper margin for each class, which boosts the best performance. Secondly, FRL improves robust fairness at the cost of reducing overall robustness, which could be seen as \textit{unfair} to other classes. However, our CFA framework improves both overall and worst class performance. In addition, FRL requires an initial robust model before fairness fine-tuning, resulting in extra computational burden. Finally, the fluctuation effect discussed in Sec.~\ref{fluctuate} is not considered in FRL. 

\noindent \textbf{Comparison with Instance-wise Adversarial Training.}
\label{dis:instance}
Though there exists a series of instance-wise adaptive adversarial training~\cite{ding2018mma,  balaji2019instance,wang2019improving,cheng2020cat, zhang2020attacks, GAIR, cai2018curriculum, wang2021convergence} toward better robust generalization, to the best of our knowledge, we are the first work to pursue this from a class-wise perspective. 
Here we demonstrate several differences between our class-wise and other instance-wise adversarial training algorithms. 
First of all, CFA focuses on improve both overall and the worst class robust accuracy, while all existing instance-wise approaches only focus on overall robustness. Unfortunately, as shown in Sec.~\ref{sec:experiment}, the instance-wise ones are not comparable with our CFA from the perspective of fairness. 
In addition, instance-wise methods can be seen as to find the solution for each individual sample, while class-wise ones are to find the solution for multiple samples. Thus, class-wise methods can alleviate the frequent fluctuation while remaining the specificity (a class of samples) of configurations among training samples. Therefore, our class-wise calibration achieves a better trade-off between flexibility and stability. Finally,  some instance-wise approaches can be well-combined with our CFA framework to further boost their performance. 


\section{Experiment}
\label{sec:experiment}
In this section, we demonstrate the effectiveness of our proposed CFA framework to improve both overall and class-wise robustness. 

\begin{table*}[!t]
    \centering
    \small
        \caption{Overall comparison of our proposed CFA framework with original methods.}
    \begin{tabular}{l|cc|cc}
    \toprule[2pt]
        & \multicolumn{2}{c|}{Best (Avg. / Worst)} & \multicolumn{2}{c}{Last (Avg. / Worst)} 
        \\
        \textbf{Method} & Clean Accuracy  & AA. Accuracy &
        Clean Accuracy  & AA. Accuracy 
        \\ \midrule[1pt]
        AT & 
        \textbf{82.3} {\scriptsize  $\pm 0.8$ } 
        /
        63.9 {\scriptsize  $\pm1.6$ }
        &  
        46.7 {\scriptsize  $\pm0.5$ }
        /
        20.1 {\scriptsize $\pm1.3$}
        &
        84.1 {\scriptsize  $\pm0.2$ }
        /
        65.1{\scriptsize  $\pm 2.4$ }
        &
        43.0 {\scriptsize  $\pm0.4$ }
        /
        15.5 {\scriptsize  $\pm1.8$ }
         \\
         AT + EMA &
        81.9 {\scriptsize  $\pm 0.3$ }
         /
        61.6 {\scriptsize  $\pm0.5$ }
        &
        49.6 {\scriptsize  $\pm0.2$ }
        /
        21.3 {\scriptsize  $\pm0.8$ }
        &
        \textbf{84.8} {\scriptsize  $\pm0.1$ }
        /
        67.7 {\scriptsize  $\pm0.7$ }
        &
        44.3 {\scriptsize  $\pm0.5$ }        
        /
        18.1 {\scriptsize  $\pm0.5$ }
        \\
        \textbf{AT + CFA} &
        80.8 {\scriptsize  $\pm0.3$ }
        /
        \textbf{64.6} {\scriptsize  $\pm0.4$ }
        &
        \textbf{50.1} {\scriptsize  $\pm0.3$ }
        /
        \textbf{24.4} {\scriptsize  $\pm0.3$ }
        &
        83.6 {\scriptsize  $\pm0.2$ }
        /
        \textbf{68.7} {\scriptsize  $\pm0.7$ }
        &
        \textbf{47.7} {\scriptsize  $\pm0.4$ }
        /
        \textbf{20.5} {\scriptsize  $\pm0.4$ }
        \\ \midrule[1pt]
        TRADES &
        \textbf{82.3} {\scriptsize  $\pm0.1$ }
        /
        \textbf{67.8} {\scriptsize  $\pm0.6$ }
        &
        48.3 {\scriptsize  $\pm0.3$ }
        /
        21.7 {\scriptsize  $\pm0.5$ }
        &
        83.9 {\scriptsize  $\pm0.3$ }
        /
        66.9 {\scriptsize  $\pm1.5$ }
        &
        46.9 {\scriptsize  $\pm0.3$ }
        /
        18.5 {\scriptsize  $\pm1.3$ }
        \\
        TRADES + EMA & 
        81.2 {\scriptsize  $\pm0.4$ }
        /
        65.0 {\scriptsize  $\pm0.7$ }
        &
        49.7 {\scriptsize  $\pm0.3$ }
        /
        24.2 {\scriptsize  $\pm0.6$ }
        &
        \textbf{84.5} {\scriptsize  $\pm0.1$ }
        /
        67.9 {\scriptsize  $\pm0.1$ }
        &
        48.3 {\scriptsize  $\pm0.2$ }
        /
        20.7 {\scriptsize  $\pm0.3$ }
        \\
        \textbf{TRADES + CFA} &
        80.4 {\scriptsize  $\pm0.2$ }
        /
        66.2 {\scriptsize  $\pm0.5$ }
        &
        \textbf{50.1} {\scriptsize  $\pm0.2$ }
        /
        \textbf{26.5} {\scriptsize  $\pm0.4$ }
        &
        83.0 {\scriptsize  $\pm0.1$ }
        /
        \textbf{68.1} {\scriptsize  $\pm0.3$ }
        &
        \textbf{49.3} {\scriptsize  $\pm0.1$ }
        /
        \textbf{21.5} {\scriptsize  $\pm0.3$ }
        \\ \midrule[1pt]

        FAT &
        84.6 {\scriptsize  $\pm0.4$ }
        /
        \textbf{69.2} {\scriptsize  $\pm 0.8$ }
        &
        45.7 {\scriptsize  $\pm0.6$ }
        /
        17.2 {\scriptsize  $\pm1.3$ }
        &
        85.4 {\scriptsize  $\pm0.2$ }
        /
        70.8 {\scriptsize  $\pm1.9$ }
        &
        42.1 {\scriptsize  $\pm0.1$ }
        /
        14.8 {\scriptsize  $\pm1.6$ }
        \\
        FAT + EMA & 
        \textbf{85.2} {\scriptsize  $\pm0.2$ }
        /
        66.7 {\scriptsize  $\pm0.6$ }
        &
        48.6 {\scriptsize  $\pm0.1$ }
        /
        18.3 {\scriptsize  $\pm0.5$ }
        &
        \textbf{85.7} {\scriptsize  $\pm0.2$ }
        /
        \textbf{71.2} {\scriptsize  $\pm0.4$ }
        &
        43.2 {\scriptsize  $\pm0.1$ }
        /
        15.7 {\scriptsize  $\pm0.7$ }
        \\
        \textbf{FAT + CFA} & 
        82.1 {\scriptsize  $\pm0.3$ }
        /
        64.7 {\scriptsize  $\pm0.9$ }
        &
        \textbf{49.6} {\scriptsize  $\pm 0.1$ }
        /
        \textbf{20.9} {\scriptsize  $\pm0.8$ }
       &
       84.3 {\scriptsize  $\pm0.1$ }
       /
       69.4 {\scriptsize  $\pm0.3$ }
       &
       \textbf{45.1} {\scriptsize  $\pm0.2$ }
       /
       \textbf{16.7} {\scriptsize  $\pm0.2$ }
        \\ \midrule[1pt]
        FRL & 
        {82.8} {\scriptsize $\pm  0.1 $}
        /
        \textbf{71.4} {\scriptsize $\pm  2.4 $}
        &
        45.9 {\scriptsize $\pm  0.3 $}
        /
        25.4 {\scriptsize $\pm  2.0 $}
        &
        \textbf{82.8} {\scriptsize $\pm  0.2 $}
        /
        72.9 {\scriptsize $\pm  1.5 $}
        &
        44.7 {\scriptsize $\pm  0.2 $}
        /
        23.1 {\scriptsize $\pm  0.8 $}
        \\
        FRL + EMA &
        \textbf{83.6} {\scriptsize  $\pm0.3$ }
        /
        69.5 {\scriptsize  $\pm0.7$ }
        &
        \textbf{46.1} {\scriptsize  $\pm0.2$ }
        /
        \textbf{25.6} {\scriptsize  $\pm0.4$ }
        &
        {81.9} {\scriptsize  $\pm0.2$ }
        /
        \textbf{74.2} {\scriptsize  $\pm0.3$ }
        &
        \textbf{44.9} {\scriptsize  $\pm0.2$ }
        / 
        \textbf{24.5} {\scriptsize  $\pm0.3$ }
        \\
        
     \bottomrule[2pt]
    \end{tabular}

    \label{tab:overall}

\end{table*}

\subsection{Experimental Setup}
We conduct our experiments on the benchmark dataset CIFAR-10~\cite{krizhevsky2009learning} using PreActResNet-18 (PRN-18)~\cite{he2016identity} model. Experiments on Tiny-ImageNet can be found in Appendix~\ref{C1}.

\noindent \textbf{Baselines.} We select vanilla adversarial training (AT)~\cite{madry2017towards} and TRADES~\cite{zhang2019theoretically} as our baselines.
Additionally, since our {Fairness Aware Weight Average (FAWA)} method is a variant of the weight average method with \textit{Exponential Moving Average (EMA)}, we include baselines with EMA as well.
For instance-wise adaptive adversarial training approaches, we include FAT~\cite{zhang2020attacks}, which adaptively adjusts attack strength on each instance. Finally, we compare our approach with FRL~\cite{xu2021robust}, the only existing adversarial training algorithm that focuses on improving the fairness of class-wise robustness.

\noindent \textbf{Training Settings.} Following the best settings in ~\cite{rice2020overfitting}, we train a PRN-18 using SGD with momentum 0.9, weight decay $5\times 10^{-4}$, and initial learning rate 0.1 for 200 epochs. The learning rate is divided by $10$ after epoch 100 and 150. All experiments are conducted by default perturbation margin $\epsilon=8/255$, and for TRADES, we initialize $\beta=6$. For the base attack strength for Class-wise Calibrated Margin (CCM), we set $\lambda_1=0.5$ for AT and $\lambda_1=0.3$ for TRADES since the training robust accuracy of TRADES is higher than AT. For FAT, we set $\lambda_1=0.7$ to avoid the attack being too weak to hard classes. Besides, we set $\lambda_2=0.5$ for Class-wise Calibrated Regularization (CCR) in TRADES. For the weight average methods, the decay rate of FAWA and EMA is set to 0.85, and the weight average processes begin at the 50-th epoch for better initialization. We draw 2\% samples from each class as the validation set for FAWA, and train on the rest of 98\% samples, hence FAWA does not lead to extra computational costs.
The fairness threshold for FAWA is set to 0.2. 

\noindent \textbf{Metrics.} We evaluate the clean and robust accuracy both in average and the worst case among classes. The robustness is evaluated by \textbf{AutoAttack (AA)}~\cite{croce2020reliable}, a well-known reliable attack for robustness evaluation.
To perform the best performance during the training phase, we adopt early stopping in adversarial training~\cite{rice2020overfitting} and present both the best and last results among training checkpoints. Further, as discussed in Sec.~\ref{fluctuate} that the worst class robust accuracy changes drastically, we select the checkpoint that achieves the highest sum of overall and the worst class robustness to report the results for a fair comparison.

\subsection{Robustness and Fairness Performance}
We implement our proposed training configuration schedule on AT, TRADES, and FAT. To evaluate the effectiveness of our approach, we conduct five independent experiments for each method and report the mean result and standard deviation.

As summarized in Table~\ref{tab:overall}, CFA helps each method achieve a significant robustness improvement both in average and the worst class at the best and last checkpoints.
Furthermore, when compared with baselines that use weight average (EMA), our CFA still achieves higher overall and the worst class robustness for each method, especially in the worst class at the best checkpoints, where the improvement exceeds 2\%. 
Note that the vanilla FAT only achieves 17.2\% the worst class robustness at the best checkpoint which is even lower than TRADES, which verifies the discussion in Sec. \ref{dis:instance} that instance-wise adaptive approaches are not helpful for robustness fairness. We also visualize and compare the robustness for each class in Appendix~\ref{C2}, which shows that CFA indeed reduces the difference among class-wise robustness and improves the fairness without harming other classes. 

We also compare our approach with FRL~\cite{xu2021robust}. However, since FRL also applies a remargin schedule, we cannot incorporate our CFA into FRL.
Therefore, we only report results of FRL with and without EMA in Table~\ref{tab:overall}.
As FRL is a variant of TRADES that applies the loss function of TRADES, we compare the results of FRL with TRADES and TRADES+CFA.
From Table~\ref{tab:overall}, we observe that FRL and FRL+EMA show only marginal progress (less than 2\%) in the worst class robustness as compared to TRADES+EMA, but at a expensive cost (about 3\%) of reducing the average performance.
As demonstrated in Sec.~\ref{margin analysis}, larger margin which is adopted in FRL mainly mitigates the robust over-fitting issue but does not bring satisfactory best performance. 
This is further confirmed by the performance of final checkpoints of FRL,
where FRL exhibits better performance in the worst class robustness.
In contrast, we calibrate the appropriate margin for each class rather than simply enlarging them, thus achieving both better robustness and fairness at the best checkpoint, \textit{i.e.}, our TRADES+CFA outperforms FRL+EMA in both average (about 4\%) and the worst class (about 1\%) robustness.


\subsection{Ablation Study}
In this section, we show the usefulness of each component of our CFA framework. Note that we still apply \textbf{AutoAttack (AA)} to evaluate robustness.

\subsubsection{Effectiveness of Calibrated Configuration}
\label{ablation}
First, we compare our calibrated adversarial configuration including CCM $\epsilon_y$ and CCR $\beta_y$ with vanilla ones for AT, TRADES, and FAT. As Table~\ref{tab:calibrated conf} shows, both the average and worst class robust accuracy are improved for all three methods by applying CCM. Besides, CCR, which is customized for TRADES, also improves the performance of vanilla TRADES. All experiments verify that our proposed class-wise adaptive adversarial configurations are effective for robustness and fairness improvement.

We also investigate the influence of base perturbation budget $\lambda_1$ by conducting 5 experiments of AT incorporated CCM with $\lambda_1$ varies from 0.3 to 0.7. The comparison is plotted in Fig.~\ref{fig:ccm analysis}(a). We can see that all models with different $\lambda_1$ show better overall and the worst class robustness than vanilla AT, among which $\lambda_1=0.5$ performs best. 
We can say that CCM has satisfactory adaptive ability on adjusting $\epsilon_k$ and is not heavily rely on the selection of $\lambda_1$. Fig.~\ref{fig:ccm analysis}(b) shows the class-wise margin used in the training phase for $\lambda_1=0.5$. We can see the hard classes (class 2,3,4,5) use smaller $\epsilon_k$ than the original $\epsilon=8/255$, while the easy classes use larger ones, which is consistent with our empirical observation on different margins in Sec.~\ref{margin analysis} and can explain why CCM is helpful to improve performance. We also present a similar comparison experiments on $\lambda_2$ for CCR in Appendix~\ref{C3}.



\subsubsection{FAWA Improves Worst Class Robustness}
Here we present the results of our Fairness Aware Weight Averaging (FAWA) compared with the simple EMA method in Table~\ref{tab:fawa}. 
By eliminating the unfair checkpoints out,  our FAWA achieves significantly better performance than EMA on the worst class robustness (nearly 2\% improvement) with negligible decrease on the overall robustness (less than 0.3\%). This verifies the effectiveness of FAWA on improving robustness fairness.


\begin{table}[!t]
    \centering
        \caption{Comparison of models with/without our class-wise calibrated configurations including margin $\epsilon$ and regularization $\beta$.}
    \begin{tabular}{l|cc}
        \toprule[1.5pt]
        \textbf{Method} & Avg. Robust & Worst Robust \\ 
        \midrule[1pt]
        AT  &  46.7 & 20.1\\
        + CCM &  \textbf{47.6} & \textbf{22.8}\\ \midrule
        TRADES & 48.3 & 21.7\\
        + CCM & 48.4 & 22.5\\
        + CCR & {48.9} & 23.5\\
        + CCM + CCR &  \textbf{49.2} & \textbf{23.8} \\ \midrule
        FAT & 45.7 & 17.2 \\
        + CCM & \textbf{46.8} & \textbf{18.9} \\
        \bottomrule[1.5pt]
    \end{tabular}

    \label{tab:calibrated conf}
\end{table}

\begin{figure}[!t]
    \centering
    \begin{tabular}{cc}
        \includegraphics[width=0.22\textwidth]{fig/compare_ccm.png}  &  \includegraphics[width=0.22\textwidth]{fig/visualize_eps.png}
         \\
        (a) & (b)
    \end{tabular}
    \caption{Analysis on the base perturbation budget $\lambda_1$. (a): Average and the worst class robustness of models trained with different $\lambda_1$ (solid) and vanilla AT (dotted). (b): Class-wise calibrated margin $\epsilon_k$ in the training phase of $\lambda_1=0.5$.}
    \label{fig:ccm analysis}
\end{figure}


\begin{table}[!t]
    \centering
        \caption{Comparison of simple EMA and our FAWA.}
    \begin{tabular}{l|cc}
    \toprule[1.5pt]
     Method & Avg. Robust & Worst Robust
     \\ \midrule[1pt]
        AT + EMA &  \textbf{49.6} & 21.3\\
        AT + FAWA &  49.3 & \textbf{23.1} \\ \midrule
        TRADES + EMA & \textbf{49.7} & 24.2 \\
        TRADES + FAWA & 49.4 & \textbf{25.1} \\ \midrule
        FAT + EMA & \textbf{48.6} & 18.3\\
        FAT + FAWA & 48.5 & \textbf{19.9}\\
    \bottomrule[1.5pt]
    \end{tabular}

    \label{tab:fawa}
\end{table}



\section{Conclusion}
In this paper, we first give a theoretical analysis of how attack strength in adversarial training impacts the performance of different classes. Then, we empirically show the influence of adversarial configurations on class-wise robustness and the fluctuate effect of robustness fairness, and point out there should be some appropriate configurations for each class.
Based on these insights, we propose a \textbf{C}lass-wise calibrated \textbf{F}air \textbf{A}dversarial training (CFA) framework to adaptively customize class-wise train  configurations for improving robustness and fairness.
Experiment shows our CFA outperforms state-of-the-art methods both in overall and fairness metrics, and can be easily incorporated into existing methods to further enhance their performance. 

\section*{Acknowledgement}
Yisen Wang is partially supported by the National Key R\&D Program of China (2022ZD0160304), the National Natural Science Foundation of China (62006153), and Open Research Projects of Zhejiang Lab (No. 2022RC0AB05).


{\small
\bibliographystyle{ieee_fullname}
\bibliography{egbib}
}

\appendix
\section{Appendix for Proofs}

\paragraph{Proof of Theorem \ref{thm:main}.}

\begin{proof}
\label{proof:main}
Our proof has two steps. In Step 1, we will show that SimCLR is equivalent to minimizing the cross entropy loss defined in Eqn.~(\ref{eqn:cross-entropy}). 
In Step 2, we will show  that minimizing the cross-entropy loss 
is equivalent to spectral clustering on $\bfpi$. 
Combining the two steps together, we have proved our theorem. 

\textbf{Step 1: } SimCLR is equivalent to minimizing the cross entropy loss.

The cross-entropy loss takes expectation over 
$\bfW_\bfX\sim \mathbb{P}(\cdot ; \bfpi)$, 
which means $\bfW_\bfX$ has exactly one non-zero entry in each row $i$. By Lemma~\ref{lem:multinomial}, we know every row $i$ of $\bfW_\bfX$ is independent of other rows. Moreover, 
$\bfW_{\bfX,i}\sim \mathcal{M}(1, \bfpi_i/\sum_j \bfpi_{i,j})=\mathcal{M}(1, \bfpi_i)$, because $\bfpi_i$ itself is a probability distribution.
Similarly, we know $\bfW_\bfZ$ also has the row-independent property by sampling over $\mathbb{P}(\cdot;\bfK_\bfZ)$.
Therefore, by Lemma~\ref{lem:cross_split}, we know Eqn.~(\ref{eqn:cross-entropy}) is equivalent to:
\[
 -\sum_{i=1}^n \mathbb{E}_{\bfW_{\bfX,i}}[\log \mathbb{P}(\bfW_{\bfZ,i}=\bfW_{\bfX,i};\bfK_\bfZ)],
\]

This expression takes expectation over $\bfW_{\bfX,i}$ for the given row $i$. Notice that 
$\bfW_{\bfX,i}$ has exactly one non-zero entry, which equals $1$ (same for $\bfW_{\bfZ,i}$). 
As a result
we expand the above expression to be:
\begin{equation}
 -\sum_{i=1}^n \sum_{j\neq i} \Pr(\bfW_{\bfX,i,j}=1)\log \Pr(\bfW_{\bfZ,i,j}=1).
\label{eqn:detailed-expansion}    
\end{equation}


By Lemma~\ref{lem:multinomial}, $\Pr(\bfW_{\bfZ,i,j}=1)=\bfK_{\bfZ,i,j}/\|\bfK_{\bfZ,i}\|_1$ for $j\neq i$. Recall that $\bfK_\bfZ=(k(\bfZ_i-\bfZ_j))_{(i,j)\in[n]^2}$, which means 
$\bfK_{\bfZ,i,j}/\|\bfK_{\bfZ,i}\|_1=\frac{\exp(-\|\bfZ_i-\bfZ_j\|^2/{2\tau})}{\sum_{k\neq i}
\exp(-\|\bfZ_i-\bfZ_k\|^2/{2\tau})
}$ for $j\neq i$, when $k$ is the Gaussian kernel with variance $\tau$. 

Notice that $\bfZ_i=f(\bfX_i)$, so we know
\begin{equation}
-\log \Pr(\bfW_{\bfZ,i,j}=1)=
-\log \frac{\exp(-\|f(\bfX_i)-f(\bfX_j)\|^2/{2\tau})}{\sum_{k\neq i}
\exp(-\|f(\bfX_i)-f(\bfX_k)\|^2/{2\tau}),
}
\label{eqn:infonce-equivalence}    
\end{equation}


The right hand side is exactly the InfoNCE loss defined in Eqn.~(\ref{eqn:infonce}).
Inserting Eqn.~(\ref{eqn:infonce-equivalence}) into Eqn.~(\ref{eqn:detailed-expansion}), we get the SimCLR algorithm, which first samples augmentation pairs $(i,j)$ with $\Pr(\bfW_{\bfX,i,j}=1)$ for each row $i$, and then optimize the InfoNCE loss. 

\textbf{Step 2: } minimizing the cross entropy loss 
is equivalent to spectral clustering on $\bfpi$.


By Lemma~\ref{lem:convert_to_spectral}, we may further convert the loss to 
\begin{equation}
\label{eqn:main-theorem-repul-attr}
\min_{\bfZ}
-\sum_{(i,j)\in [n]^2} \mathbf{P}_{i,j}
\log k (\bfZ_i-\bfZ_j)+\log \mathbf{R}(\bfZ).
\end{equation}
Since $k$ is the Gaussian kernel, this reduces to \[
\min_\bfZ \mathrm{tr}(\bfZ^\top \mathbf{L}(\bfpi) \bfZ)
+\log \mathbf{R}(\bfZ),
\]

where we use the fact that $\mathbb{E}_{\bfW_\bfX\sim \mathbb{P}(\cdot; \bfpi)}[\mathbf{L}(\bfW_\bfX)]
=\mathbf{L}(\bfpi)
$, because the Laplacian operator is linear and $
\mathbb{E}_{\bfW_\bfX\sim \mathbb{P}(\cdot; \bfpi)}(\bfW_\bfX)=\bfpi
$.
\end{proof}

\paragraph{Proof of Theorem \ref{thm:clip}.}
\begin{proof}
Since $\bfW_\bfX\sim \mathbb{P}(\cdot;\bfpi_{\mathbf{A}, \mathbf{B}})$, we know 
$\bfW_\bfX$ has exactly one non-zero entry in each row, denoting the pair that got sampled. 
A notable difference compared to the previous proof is we now have $n_\mathcal{A}+n_\mathcal{B}$ objects in our graph. CLIP deals with this by taking a mini-batch of size $2N$, 
such that $n_\mathcal{A}=n_\mathcal{B}=N$, and adding the $2N$ InfoNCE losses together. We label the objects in $\mathcal{A}$ as $[n_\mathcal{A}]$, and the objects in $\mathcal{B}$ as $\{n_\mathcal{A}+1, \cdots, n_\mathcal{A}+n_\mathcal{B}\}$. 

Notice that $\bfpi_{\mathbf{A}, \mathbf{B}}$ is a bipartite graph, so the edges of objects in $\mathcal{A}$ will only connect to object in $\mathcal{B}$ and vice versa. We can define the similarity matrix in $\cZ$ as $\bfK_\bfZ$, 
where $\bfK_\bfZ(i, j+n_\mathcal{A})=\bfK_\bfZ(j+n_\mathcal{A},i)= k(\bfZ_i-\bfZ_j)$ for $i\in [n_\mathcal{A}], j\in [n_\mathcal{B}]$, and otherwise we set $\bfK_\bfZ(i,j)=0$. 
The rest is same as the previous proof. 
\end{proof}

\paragraph{Proof of Theorem \ref{thm:exponential}.}

\begin{proof}
\label{proof:exponential}
Since the objective function consists of a linear term combined with an entropy regularization, which is a strongly concave function, the maximization problem is a convex optimization problem. Owing to the implicit constraints provided by the entropy function, the problem is equivalent to having only the equality constraint. We then introduce the Lagrangian multiplier $\lambda$ and obtain the following relaxed problem:

$$
\widetilde{E}(\boldsymbol{\alpha})=\psi_{1}-\sum_{i=1}^n \alpha_{i} \psi_{i}+\tau \sum_{i=1}^n \alpha_{i}\log \alpha_{i}+\lambda\left(\boldsymbol{\alpha}^{\top} \mathbf{1}_n-1\right).
$$

As the relaxed problem is unconstrained, taking the derivative with respect to $\alpha_{i}$ yields

$$
\frac{\partial \widetilde{E}(\boldsymbol{\alpha})}{\partial \alpha_{i}}=-\psi_{i}+\tau\left(\log \alpha_{i}+\alpha_{i} \frac{1}{\alpha_{i}}\right)+\lambda=0.
$$

Solving the above equation implies that $\alpha_{i}$ takes the form
$
\alpha_{i}=\exp \left(\frac{1}{\tau} \psi_{i}\right) \exp \left(\frac{-\lambda}{\tau}-1\right).
$ Since $\alpha_{i}$ lies on the probability simplex, the optimal $\alpha_{i}$ is explicitly given by
$
\alpha^{*}_{i}=\frac{\exp \left(\frac{1}{\tau} \psi_{i}\right)}{\sum_{i^{\prime}=1}^n \exp \left(\frac{1}{\tau} \psi_{i^{\prime}}\right)} .
$ Substituting the optimal point into the objective function, we obtain
$$
\begin{aligned}
E\left(\boldsymbol{\alpha}^*\right)  &=\psi_1-\sum_{i=1}^n \frac{\exp \left(\frac{1}{\tau} \psi_{i}\right)}{\sum_{i^{\prime}=1}^n \exp \left(\frac{1}{\tau} \psi_{i^{\prime}}\right)} \psi_{i}+\tau \sum_{i=1}^n \frac{\exp \left(\frac{1}{\tau} \psi_{i}\right)}{\sum_{i^{\prime}=1}^n \exp \left(\frac{1}{\tau} \psi_{i^{\prime}}\right)}\log \frac{\exp \left(\frac{1}{\tau} \psi_{i}\right)}{\sum_{i^{\prime}=1}^n \exp \left(\frac{1}{\tau} \psi_{i^{\prime}}\right)} \\
& =\psi_1 - \tau \log \left(\sum_{i=1}^n \exp \left(\frac{1}{\tau} \psi_{i}\right)\right).
\end{aligned}
$$
Thus, the Lagrangian dual function is given by
\begin{equation*}
-E\left(\boldsymbol{\alpha}^*\right)= -\tau \log \frac{\exp \left(\frac{1}{\tau} \psi_{1}\right)}{\sum_{i=1}^n \exp \left(\frac{1}{\tau} \psi_{i}\right)}.\qedhere
\end{equation*}
\end{proof}



\section{More on Experiments} \label{section: experiment_details}

\paragraph{CIFAR-10 and CIFAR-100} CIFAR-10 ~\citep{krizhevsky2009learning} and CIFAR-100 ~\citep{krizhevsky2009learning} are well-known classic image classification datasets. Both CIFAR-10 and CIFAR-100 contain a total of 60k $32 \times 32$ labeled images of different classes, with 50k for training and 10k for testing. CIFAR-10 is similar to CIFAR-100, except there are 10 different classes in CIFAR-10 and 100 classes in CIFAR-100.

\paragraph{TinyImageNet} TinyImageNet ~\citep{le2015tiny} is a subset of ImageNet ~\citep{deng2009imagenet}. There are 200 different object classes in TinyImageNet, with 500 training images, 50 validation images, and 50 test images for each class. All the images in TinyImageNet are colored and labeled with a size of $64 \times 64$.

\textbf{Pseudo-code.} Algorithm \ref{alg:Training Procedure} presents the pseudo-code for our empirical training procedure.

\begin{algorithm}[!htbp]
\caption{Training Procedure}
\label{alg:Training Procedure}
\begin{algorithmic}[1]
\REQUIRE trainable encoder network $f$, batch size $N$, augmentation strategy \textit{aug}, loss function $L$ with hyperparameters \textit{args}
\FOR {sampled minibatch ${x_i}_{i=1}^N$}
\FORALL{$i \in { 1, ..., N }$}
\STATE draw two augmentations $t_i = \textit{aug}\left(x_i\right) $, $t_i' = \textit{aug}\left(x_i\right) $
\STATE $z_i = f\left(t_i\right)$, $z_i' = f\left(t_i'\right)$
\ENDFOR
\STATE compute loss $\mathcal{L} = L(N, z, z', \textit{args})$
\STATE update encoder network $f$ to minimize $\mathcal{L}$
\ENDFOR
\STATE \textbf{Return} encoder network $f$
\end{algorithmic}
\end{algorithm}

We also provide the pseudo-code for our core loss function used in the training procedure in Algorithm \ref{alg:Core loss}. The pseudo-code is almost identical to SimCLR's loss function, with the exception of an extra parameter $\gamma$.

\begin{algorithm}[!htbp]
\caption{Core loss function $\mathcal{C}$}
\label{alg:Core loss}
\begin{algorithmic}[1]
\REQUIRE batch size $N$, two encoded minibatches $z_1, z_2$, $\gamma$, temperature $\tau$
\STATE $z = \textit{concat}\left(z_1, z_2\right)$
\FOR {$i \in {1, ..., 2N }, j \in {1, ..., 2N}$ }
\STATE $s_{i,j} = \Vert z_i - z_j \Vert_2^{\gamma}$
\ENDFOR
\STATE \textbf{define} $l(i, j)$ \textbf{as} $l(i, j) = - \log \frac{exp\left(s_{i,j}/\tau \right)}{\sum_{k=1}^{2N} \mathbf{1}{[k \ne i]} exp\left(s{i, j} / \tau \right)} $
\STATE \textbf{Return} $\frac{1}{2N} \sum_{k=1}^N\left[l(i, i+N) + l(i+N, i)\right]$
\end{algorithmic}
\end{algorithm}

Utilizing the core loss function $\mathcal{C}$, we can define all kernel loss functions used in our experiments in Table \ref{table: loss definition}. For all $z_i \in z$ with even dimensions $n$, we define $z_{L_i} = z_i\left[0:n/2\right]$ and $z_{R_i} = z_i\left[n/2:n\right]$.

\begin{table}[ht]
\centering
\begin{tabular}{{@{}l|l@{}}}
Kernel  &  Loss function \\ \midrule
Laplacian & $\mathcal{C}\left(N, z, z', \gamma=1, \tau\right)$\\ \midrule
Sum       & $\lambda * \mathcal{C}\left(N, z, z', \gamma=1, \tau_1\right) + (1-\lambda) * \mathcal{C}\left(N, z, z', \gamma=2, \tau_2\right)$  \\ \midrule
Concatenation Sum&$\lambda * \mathcal{C}\left(N, z_L, z'_L, \gamma=1, \tau_1\right) + (1-\lambda) * \mathcal{C}\left(N, z_R, z'_R, \gamma=2, \tau_2\right)$\\ \midrule
$\gamma = 0.5$ & $\mathcal{C}\left(N, z, z', \gamma=0.5, \tau\right)$          \\ 

\end{tabular}

\caption{Definition of kernel loss functions in our experiments}
\label {table: loss definition}
\end{table}

\textbf{Baselines.} We reproduce the SimCLR algorithm using PyTorch Lightning~\citep{PytorchLightning}.

\textbf{Encoder details.}
The encoder $f$ consists of a backbone network and a projection network. We employ ResNet50~\citep{ResNet} as the backbone and a 2-layer MLP (connected by a batch normalization~\citep{ioffe2015batch} layer and a ReLU \cite{nair2010rectified} layer) with hidden dimensions 2048 and output dimensions 128 (or 256 in the concatenation kernel case).

\textbf{Encoder hyperparameter tuning.}
For each encoder training case, we randomly sample 500 hyperparameter groups (sample details are shown in Table \ref{table: Hyperparameter sample}) and train these samples simultaneously using Ray Tune ~\citep{RayTune}, with the ASHA scheduler~\citep{li2018massively}. Ultimately, the hyperparameter group that maximizes the online validation accuracy (integrated in PyTorch Lightning) within 5000 validation steps is chosen for the given encoder training case.

\begin{table}[ht]
\centering

\begin{tabular}{@{}l|l|l@{}}
\midrule
Hyperparameter  & Sample Range & Sample Strategy \\ \midrule
start learning rate & $\left[10^{-2}, 10\right]$ & log uniform \\ \midrule
$\lambda$       & $\left[0, 1\right]$ & uniform \\ \midrule
$\tau$, $\tau_1$, $\tau_2$ & $\left[0, 1\right]$ & log uniform \\ \midrule
\end{tabular}

\caption{Hyperparameters sample strategy}
\label {table: Hyperparameter sample}
\end{table}

\textbf{Encoder training.} 
We train each encoder using the LARS optimizer~\citep{LARSOptimizer}, LambdaLR Scheduler in PyTorch, momentum 0.9, weight decay $10^{-6}$, batch size 256, and the aforementioned hyperparameters for 400 epochs on a single A-100 GPU.

\textbf{Image transformation.} The image transformation strategy, including augmentation, is identical to the default transformation strategy provided by PyTorch Lightning.

\textbf{Linear evaluation.}
The linear head is trained using the SGD optimizer with a cosine learning rate scheduler, batch size 64, and weight decay $10^{-6}$ for 100 epochs. The learning rate starts at $0.3$ and ends at $0$.

\textbf{Moco Experiments.} We also tested our method based on MoCo~\citep{he2019moco}. The results are summarized in Table \ref{tab:results-moco}. Here we choose ResNet18~\citep{ResNet} as the backbone and set a temperature of $0.1$ as default. For our simple sum kernel, we set $\lambda=0.8$. The results show that our method outperforms the original MoCo method.

\begin{table}[thb]
\centering
\caption{MoCo Experiment Results on CIFAR-10 and CIFAR-100.}
\label{tab:results-moco}
\resizebox{\textwidth}{!}{%
\begin{tabular}{@{}c|ccc|ccc@{}}
\toprule
\multirow{3}{*}{Method} & \multicolumn{3}{c|}{CIFAR-10} & \multicolumn{3}{c}{CIFAR-100} \\ \cmidrule(lr){2-4} \cmidrule(lr){5-7} 
                        & 200 epochs & 400 epochs    & 1000 epochs   & 200 epochs & 400 epochs & 1000 epochs         \\ \midrule
MoCo (repro.)         & $76.41 \pm 0.12$    & $80.01 \pm 0.15$          & $84.45 \pm 0.08$    & $\mathbf{47.02 \pm 0.11}$ & $52.50 \pm 0.07$ & $57.62 \pm 0.15$            \\
\midrule
Laplacian Kernel        & ${78.09 \pm 0.10}$    & $\mathbf{83.85 \pm 0.09}$          & $\mathbf{88.34 \pm 0.16}$    & $46.12 \pm 0.22$   & $53.44 \pm 0.17$ & $59.10 \pm 0.14$        \\
Simple Sum Kernel & $\mathbf{78.12 \pm 0.15}$   & $83.23 \pm 0.18$ & $87.50 \pm 0.20$ & $46.65 \pm 0.06$ & $\mathbf{53.62 \pm 0.19}$ & $\mathbf{59.83 \pm 0.12}$\\
\bottomrule
\end{tabular}
}
\end{table}



\section{More Experiments on Synthetic Data}


Consider a scenario with $n$ clusters, each containing $k$ vertices. Let the probability of vertices $u$ and $v$ from the same cluster belonging to $\bfpi$ be $p$. Conversely, for vertices $u$ and $v$ from different clusters, let the probability of belonging to $\pi$ be $q$. We generate the graph $\bfpi$ randomly, based on $p$ and $q$. We experiment with values of $k=100$ and $n=6$ for ease of visualization, embedding all points in a two-dimensional space. Each vertex's initial position originates from a normal distribution. In each iteration, we sample a subgraph of $\bfpi$ uniformly, ensuring each vertex has an out-degree of $1$. We then optimize the corresponding vectors using InfoNCE loss with an SGD optimizer and iterate until convergence. Our experimental setup consists of an SGD learning rate of $1$, an InfoNCE loss temperature of $0.5$, and a batch size of $50$. We evaluate two scenarios with different $p$ and $q$ values: $p=1$, $q=0$, and $p=0.75$, $q=0.2$. The results of these experiments are visualized in Figure \ref{fig:vis-spectral-cluster}. The obtained embeddings exhibit the hallmark pattern of spectral clustering of graph $\bfpi$.

\begin{figure}[!tb]
\centering
\subfigure{
\includegraphics[width=1\textwidth]{Figures/cluster_pi.png}
\label{fig:vis-cluster}
}
\subfigure{
\includegraphics[width=1\textwidth]{Figures/noised_cluster_pi.png}
\label{fig:vis-noised-cluster}
}
\caption{Visualizations of the optimization process using InfoNCE Loss on the vectors corresponding to $\bfpi$. Points of identical color belong to the same cluster within $\bfpi$. To showcase the internal structure of $\bfpi$, we randomly select 10 vertices from each cluster to display the edge distribution of $\bfpi$.}
\label{fig:vis-spectral-cluster}
\end{figure}





\end{document}
