\section{Results and Conclusion}
\label{sec:conclusion}

    Fig. \ref{fig:4}-a shows the impact of various compression ratios on VAE output images. At lower compression ratios, reconstructed images more closely resemble original input images. Importantly, we see a marked improvement in histologic features that are critical for interpretability such as refined cell-to-cell borders and sharper demarcation of cytoplasmic vs. nuclei compartments. Moreover, in Fig. \ref{fig:4}-b, we use UMAP to visualize the latent space vectors learned by our pipeline. The UMAP captures intra-tumor and across-tumor relationships, separating all four tissue types into distinct clusters. Interestingly, clusters of brain and colon cancers share overlapping boundaries whereas the breast cancer cluster is uniquely separated from the brain cancer cluster. Also, the UMAP identifies a distinct sub-cluster of brain tumor samples that does not overlap with any other cancer types. 

     We envision our pipeline being useful to clinicians and researchers across multiple domains. One potential application is more accurate sub-typing and diagnoses of poorly understood cancers. A notable example of this is brain cancer, which contains over 150 different histologic subtypes, many of which are so rare that a pathologist may only encounter a handful of cases in his or her career (\cite{roetzer-pejrimovsky_digital_2022}). In our UMAP visualization of the latent space, there is an unexpected but distinct sub-cluster of brain tumor samples that does not overlap with other cancer types (Fig. \ref{fig:4}-b). Further characterization of this sub-cluster and its unique attributes could provide novel insights into intra-tumor relationships in brain cancer. 

    Our pipeline also facilitates experiments across different tumor types. The latent space separates breast, colon, lung/bronchus, and brain tissue into unique clusters, demonstrating the preservation of important histological features. Interestingly, we see a closer clustering between brain and colon cancer versus brain and breast or lung (Fig. \ref{fig:4}-b). More investigation into these relationships is warranted – one possible explanation of this phenomenon could be due to both brain and colon tissue containing ganglion nerve cells whereas breast and lung tissue do not. In the future, our embedding approach could be deployed to a hospital system and linked to the electronic health record (EHR) to help clinicians diagnose patients with rare disorders: the images closest to that of the input patient in UMAP embedding have the most similarities, and their records could be retrieved to better contextualize a differential diagnosis for the query patient. 
    
    However, our pipeline carries several limitations. To start, we will need to further explore acceptable thresholds of reconstruction loss introduced via our VAE-based architecture. Additionally, our model architecture lacks human interpretable features, which may lead to higher levels of end-user distrust as “peeking under the hood” to audit our model for biases or errors may be more limited. Along these lines, any insights or novel conclusions will still require manual review and interpretation by human pathologists. In future iterations of this work, we intend to improve upon these areas.

    