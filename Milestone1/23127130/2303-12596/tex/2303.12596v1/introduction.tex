\section{Introduction}
\label{intoduction}

The roll-out of 5G has enabled a plethora of use cases and new applications, which generally fall under three broad categories: \gls{emb}, \gls{mtc} and \gls{urllc} \cite{5g_pillars}. Most human-generated traffic is \gls{emb}, which supports higher data rates, mobility and dense connectivity thanks to the capability of connecting to micro- and nano-cells. Other scenarios that can be characterized as \gls{emb} include large-scale distributed machine learning~\cite{dl}, which is expected to generate massive volumes of traffic. On the other hand, \gls{mtc} focuses on high density \gls{iot} device deployments, and specifically aims at supporting longer transmission distances and reducing energy consumption by adopting lower data rates~\cite{nb-iot-2}. Finally, \gls{urllc} includes scenarios that require near perfect reliability (i.e., an error probability below $10^{-5}$) with a maximum latency of $1$~ms. Its use cases extend to cooperation between autonomous vehicles~\cite{vehicles} and the Tactile Internet~\cite{haptic}.

The research on adapting the \gls{ran} to the new requirements of these different classes of traffic is extensive, but the \gls{ran} is not the only factor affecting quality of application's delivery. The presence of cross-traffic can cause significant delays and congestion, and even without it, self-queuing delay is a well-known problem in wireless networks, as probing the connection capacity aggressively can increase the latency~\cite{polese2019survey}~\cite{tcp-asymetry}.

Additionally, different applications and use-cases expect different levels or latency and/or reliability. This can make optimization even more complex: meeting different requirements for several applications, all with independent adaptation mechanisms, is a significant challenge. Even the nature of the metrics themselves might be different: one recent example is \gls{aoi}, which has currently risen in prominence for several \gls{iot} applications~\cite{yates2021age}, as it can measure the freshness of the data available to the receiver, incorporating the data generation process as well as network-related aspects.

An important detail in \gls{aoi} optimization is that not every packet is needed: if new updates are frequent enough, the more recent data supersedes the older information, and a certain level of packet loss is acceptable. This is opposed to the extreme reliability requirements of \gls{urllc} traffic, but there are several use cases and scenarios where it intuitively makes sense, e.g., cooperative autonomous driving: if a car frequently sends information about its position and direction, the loss of any single sensor reading can be recovered from~\cite{quicest}, keeping in mind that urgent safety messages will be sent as \gls{urllc} traffic, ensuring that mission-critical information, such as collision warnings, is reliably transmitted.

These real-time, relatively loss-tolerant information flows may not require end-to-end reliability mechanisms, as retransmissions can increase delay and traffic on the network: in order to better support the concurrent \gls{urllc} and \gls{emb} flows, which need network resources due to their stricter requirements, these applications can send data unreliably. In this paper, we introduce dynamic reliability: a novel idea for partial reliability. Dynamic reliability permits assignment of per-packet reliability status at the transport layer, governed by tailored policies, opposed to the traditional application layer manner. The framework follows a true unreliable transmission philosophy of removing \glspl{ack} and retransmissions. The proposed solution is a modification of the QUIC protocol~\cite{quic}, which can support concurrent flows with different reliability requirements, dynamically setting the reliability of each packet while considering the state of the connection.

The proposed solution is examined in detail with respect to a number of network topologies and links, considering metrics such as session packet volume, backlogged packets and \gls{aoi}, to ascertain the feasibility of dynamic reliability in real-time scenarios. Furthermore, we explore a number of governing reliability policies to show the adaptability of dynamic reliability to different requirements. The contributions of this paper include:
\begin{itemize}
    \item A dynamic reliability framework that allows per-packet assignment of reliability at the transport layer, depending on the governing reliability policies.
    \item Performance evaluation of a number of reliability policies for dynamic reliability. The policies explore different logic and complexity for reliability assignment based on path and network measurements. 
    \item A comparison between the different dynamic reliability logic over wireless communication using different frequencies, including \gls{mmwave}, sub-6GHz and Wi-Fi.
\end{itemize}

The remainder of this paper is structured as follows: Sec. \ref{soa} showcases the \gls{soa}, informing the reader on QUIC and the different implementations of partial reliability. Sec. \ref{methodology} delves into the system model and design of dynamic reliability, while Sec. \ref{results} analyses the result of implementing dynamic reliability in simulated network environments. Lastly, the conclusion and future works are discussed in Sec. \ref{conclusion}.