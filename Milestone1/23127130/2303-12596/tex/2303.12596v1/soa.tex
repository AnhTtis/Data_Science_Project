\section{State of the Art}
\label{soa}

Although the transport layer does not consider any \gls{ran} parameters directly, indirect end-to-end measurements such as \gls{rtt} are used to assess the bandwidth, latency and loss rate. QUIC is one protocol of many that utilises these path properties for end-to-end communication services.
QUIC was designed by Google to circumvent latency issues that were associated with the traditional \gls{tcp}. More specifically, it allowed multiplexing a session into different streams between the two endpoints, to bypass \gls{hol}. Each stream in QUIC is given a Connection ID and treated as a separate flow, this allows each stream to terminate or migrate without affecting the rest of the session. Each stream also has its own packet sequence independently from other streams, maintaining its own in-order delivery system.

Despite the fact that QUIC runs over \gls{udp}, its transmission is reliable and in-order for each stream. That being said, recently a \emph{Datagram Extension} has been added to QUIC to enable unreliable transmission of packets as well \cite{datagram}. A reliable and unreliable stream between the same end-points can share a single handshake and proceed with transmission normally, but it is left for the application layer to differentiate between datagram flows, burdening the application layer with a larger complexity as well as offering no re-ordering for the datagram packets. Even though datagram frames are unreliable, they are still ACK-eliciting and their rate is affected by the congestion window.

In the most general definition, partial reliability allows for a transport protocol to send reliable and unreliable data when the application deems fit. While the traditional fully reliable approach has been challenged by transport protocols such as \gls{sctp}~\cite{pr_sctp}, \gls{dccp}~\cite{pr-dccp} and even \gls{tcp} \cite{pr_tcp}, there is no unified concept of partial reliability, or a shared technical solution. Each protocol implements their distinct definition of partial reliability with disparate design aims. The QUIC datagram extension also goes in this direction, but the decision on which data should be sent reliably is still an open research question.

QUICSilver \cite{quicsilver} uses predicted deadline awareness to guide the decision making for sending frames reliably. The aim of QUICSilver is to reduce the occurrence of stalls in video transmission, which it has managed to achieve compared to vanilla QUIC, albeit only to only low-quality video transmissions. As stated by the author, the freshness checks to determine the staleness of the frames introduce a non-negligible delay which inherently affects the playback of the video transmission.

Furthermore, ClipStream \cite{clipstream} sends Intra-Frames (i.e., frames that contain sufficient data to display the whole image) and end-of-stream markers reliably, while Predicted Frames (i.e., differentially encoded frames based on the previous I-frame) are sent unreliably. Opportunistic retransmission is used, sending new data instead of retransmissions for the unreliable streams. That being said, this work does not take into account the strict playback delays when considering the retransmission of reliable frames.

Alternatively, QUIC-EST \cite{quicest} adopts a multi-sensory use case as their variation of a real-time application. The aim is to reduce the probability of undelivered frames in blocking fresh content transmission. The design goal is to have every object sent on its own stream with a priority allocation based on the value of information that frame induces. When the packets on a single stream are acknowledged, the stream can be reused for consecutive new objects. If a packet is lost or excessively delayed, the whole stream is discarded, opting out of retransmissions. Although the approach can be generalised to many mission-critical and real-time applications, it requires the application layer to assign a priority value to objects.

In this work, we consider a new approach for partial reliability. We focus on per-packet partial reliability at the transport layer piloted by policies, which emphasises reactivity to fluctuations in network measurements, rather than the traditional method of partial reliability assignment at the application layer. We remove the need for ack-elicitation following a true unreliable philosophy, which has not been explored in the literature. Additionally, we incorporate ``timed reliability'' from \gls{sctp} \cite{pr_sctp}, but utilise an occupancy-driven metric rather than time-based: when the buffer reaches a threshold of packets received, all the packets that have not been sent to the application layer are discarded.