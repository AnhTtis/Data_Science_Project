%------------------------------------- preprint  version
\documentclass[a4paper,12pt,reqno]{amsart}

%--------------- packages
\usepackage[T1]{fontenc}
\usepackage[utf8]{inputenc}
\usepackage[english]{babel}
\usepackage[left=25mm,right=25mm,top=35mm,bottom=35mm]{geometry}
\usepackage{amsmath,amssymb,amsthm}
\usepackage{pgfplots}
\usepackage{pgfplotstable}
\usepackage[margin=0pt,font={small}]{caption}
\usepackage{booktabs}
%\usepackage{slashbox} % needed for tables
\usepackage{color,colortbl}
%\usepackage[hidelinks]{hyperref}
\usepackage[colorlinks=true,allcolors=blue,breaklinks]{hyperref}

\usepackage{algpseudocode} 
\usepackage{multirow} 
\usepackage{graphicx} 
\graphicspath{{./fig/}} 
\usepackage{caption} 
\usepackage{subcaption} 

% fonts 
\font\cmmib=cmmib10
\font\msbm=msbm10
\font\msam=msam10
\font\cmc=cmcsc10
\font\eightrm=cmr8

%--------------- math definitions
\def\vfld#1{\vec{\,#1}}
\def\P{\hbox{\msbm P}}
\def\Q{\hbox{\msbm Q}}
\newcommand{\amg}{AMG}
\newcommand{\RE}{\mathcal R}%{\mathrm{Re}}

\newcommand\0{\boldsymbol{0}}
\newcommand{\eps}{\varepsilon}
\newcommand{\x}{\mathbf{x}}
\newcommand{\y}{\mathbf{y}} 
\newcommand{\bfu}{\mathbf{u}} 
\newcommand{\bfp}{\mathbf{p}}
\newcommand{\bfq}{\mathbf{q}}
\newcommand{\bfr}{\mathbf{r}}
\newcommand{\bff}{\mathbf{f}} 
\newcommand{\bfg}{\mathbf{g}} 
\newcommand{\bfz}{\mathbf{z}} 

\newcommand{\g}{\gamma}
\newcommand{\G}{\Gamma}
\newcommand{\N}{\mathbb{N}}
\renewcommand{\P}{\mathbb{P}}
\newcommand{\R}{\mathbb{R}}
\newcommand{\V}{\mathbb{V}}
\newcommand{\X}{\mathbb{X}}
\newcommand{\Y}{\mathbb{Y}}

\renewcommand{\AA}{\mathcal{A}}
\newcommand{\GG}{\mathcal{G}}
\newcommand{\EE}{\mathcal{E}}
\newcommand{\MM}{\mathcal{M}}
\newcommand{\NN}{\mathcal{N}}
\newcommand{\PP}{\mathcal{P}}
\newcommand{\RR}{\mathcal{R}}
\renewcommand{\SS}{\mathcal{S}}
\newcommand{\TT}{\mathcal{T}}
\newcommand\III{\mathfrak{I}}
\newcommand\MMM{\mathfrak{M}}
\newcommand\PPP{\mathfrak{P}}
\newcommand\QQQ{\mathfrak{Q}}

\DeclareMathOperator*{\hull}{span}
\DeclareMathOperator*{\refine}{refine}
\DeclareMathOperator*{\supp}{supp}

\newcommand{\dpi}{\mathrm{d}\pi}
\newcommand{\dx}{\mathrm{d}x}
\newcommand{\dy}{\mathrm{d}\y}
\newcommand{\cost}{\mathsf{cost}}
\newcommand{\tol}{\mathsf{tol}}

\newcommand{\norm}[3][]{#1\|#2#1\|_{#3}}
\newcommand{\reff}[2]{\stackrel{\eqref{#1}}{#2}}	% to stack equations above (in)equalities
\newcommand{\refp}[2]{\stackrel{\phantom{\eqref{#1}}}{#2}}

\newcommand{\npressure}{n_p}
\newcommand{\nvelocity}{n_u}
\newcommand{\nCpressure}{n_{k}}
\newcommand{\nel}{n_{el}}



\newcommand{\bp}{\boldsymbol{p}}
\newcommand{\bq}{\boldsymbol{q}}
\newcommand{\br}{\boldsymbol{r}}
\newcommand{\bu}{\boldsymbol{u}}
\newcommand{\bv}{\boldsymbol{v}}
\newcommand{\bw}{\boldsymbol{w}}
\newcommand{\bz}{\boldsymbol{z}}

\newcommand{\qnull}{\boldsymbol{k}}
\newcommand{\anull}{\boldsymbol{w}}
\newcommand{\zrange}{\overline{\bz}}


\newcommand{\boldA}{\boldsymbol{A}}
\newcommand{\pressuremass}{Q^{\star}}
\newcommand{\Aapprox}{\boldsymbol{M}}
\newcommand{\Qapprox}{M_S}
\newcommand{\Qzero}{Q_0}

\newcommand{\Vspace}{\boldsymbol{V}_h}
\newcommand{\Qspace}{Q_h}
\newcommand{\Qzerospace}{Q_{h}^0}
\newcommand{\VspaceTH}{\boldsymbol{V}_{h}^{TH}}
\newcommand{\QspaceTH}{Q_h^{TH}}
\newcommand{\VspaceETH}{\VspaceTH}
\newcommand{\QspaceETH}{Q_{h}^{\star}}

\DeclareMathOperator*{\diag}{diag}
\DeclareMathOperator*{\nullspace}{null}

%--------------- right hand margin macros
\def\marginnote#1{\setbox0=\vtop{\hsize 5pc
    \rightskip=.5pc plus 1.5pc #1}\leavevmode
     \vadjust{\dimen0=\dp0
      \kern-\ht0\hbox{\kern\hsize\kern1pc\box0}\kern-\dimen0}}
%\def\bookref{\marginnote{$\heartsuit$}}
%\def\backref{\marginnote{$\clubsuit$}}
\def\bookref{}
\def\backref{}


%--------------- figure formatting using tikz
\pgfplotsset{
%every axis/.append style={
%font={\fontsize{8pt}{12pt}\selectfont},  
%},
tick label style={font=\tiny},
legend style={font=\tiny},
%title style={font=\tiny,yshift=-1.5ex},
%xlabel style={font=\tiny,yshift=+1.0ex},
%ylabel style={font=\tiny,yshift=-1.2ex},
xlabel style={yshift=+0.5ex},
ylabel style={yshift=-1.0ex}
}


%--------------- environments
\theoremstyle{plain}
\newtheorem{theorem}{Theorem}
\newtheorem{proposition}[theorem]{Proposition}
\newtheorem{lemma}[theorem]{Lemma}
\newtheorem{corollary}[theorem]{Corollary}
\theoremstyle{definition}
\newtheorem{algorithm}[theorem]{Algorithm}
\newtheorem{remark}[theorem]{Remark}
\newtheorem{examp}{Test problem}
\newtheorem{marking}{Marking criterion}
\renewcommand{\themarking}{\Alph{marking}}


%--------------- subsection title in bold
\makeatletter
\def\@seccntformat#1{%
  \protect\textup{\protect\@secnumfont
    \ifnum\pdfstrcmp{subsection}{#1}=0 \bfseries\fi% subsection # in \bfseries
    \csname the#1\endcsname
    \protect\@secnumpunct
  }%
}
\makeatother


%---------------  linenumbers
%\newcommand*\patchAmsMathEnvironmentForLineno[1]{%
%  \expandafter\let\csname old#1\expandafter\endcsname\csname #1\endcsname
%  \expandafter\let\csname oldend#1\expandafter\endcsname\csname end#1\endcsname
%  \renewenvironment{#1}%
%     {\linenomath\csname old#1\endcsname}%
%     {\csname oldend#1\endcsname\endlinenomath}}% 
%\newcommand*\patchBothAmsMathEnvironmentsForLineno[1]{%
%  \patchAmsMathEnvironmentForLineno{#1}%
%  \patchAmsMathEnvironmentForLineno{#1*}}%
%\AtBeginDocument{%
%\patchBothAmsMathEnvironmentsForLineno{equation}%
%\patchBothAmsMathEnvironmentsForLineno{align}%
%\patchBothAmsMathEnvironmentsForLineno{flalign}%
%\patchBothAmsMathEnvironmentsForLineno{alignat}%
%\patchBothAmsMathEnvironmentsForLineno{gather}%
%\patchBothAmsMathEnvironmentsForLineno{multline}%
%}
%\usepackage[mathlines]{lineno}
%\renewcommand\linenumberfont{\fontsize{6}{7}\selectfont\sf}
%\linenumbers

%---------------  date every page
\usepackage{fancyhdr}
\cfoot{\small\thepage}
\lhead{}
\rhead{}
\renewcommand{\headrulewidth}{0pt}
\advance\footskip0.5cm
\pagestyle{fancy}


%--------------- colors
\newcommand\abrevx[1]{{\color{black}#1}}
\newcommand\rblx[1]{{\color{black}#1}}
\newcommand\rev[1]{{\color{black}#1}}
\newcommand\revx[1]{{\color{black}#1}}


%--------------- editing
\definecolor{otherblue}{rgb}{0,0.3,0.6}
\def\rbl#1{\textcolor{otherblue}{#1}}
\def\todo#1{\textcolor{red}{#1}}
\def\rems#1{\textcolor{cyan}{#1}}

%--------------- header stuff
\title{Fast solution of incompressible flow problems with two-level pressure approximation}
%
\author{Jennifer Pestana}
\address{Department of Mathematics and Statistics, University of Strathclyde, Glasgow, G1 1XH, UK}
\email{jennifer.pestana@strath.ac.uk}

\author{David J. Silvester}
\address{Department of Mathematics, University of Manchester, Oxford Road, Manchester M13 9PL, UK}
\email{d.silvester@manchester.ac.uk}

%--------------- extra stuff
%\subjclass[2010]{35R60, 65C20, 65N12, 65N15, 65N30}
%
%\keywords{finite elements, mixed approximation, preconditioning, fast solvers}

\thanks{{\em Acknowledgements.}
{This work was supported by EPSRC grant EP/W033801/1.}
}
\date{\today}

\begin{document}

\begin{abstract}
This paper develops efficient preconditioned iterative solvers for 
incompressible flow problems discretised by an enriched Taylor--Hood mixed approximation, 
in which the usual pressure space is augmented by a piecewise constant pressure to ensure 
local mass conservation. 
This enrichment process causes over-specification of the pressure, which  
complicates the design and implementation of efficient solvers for the resulting linear systems. 
We first describe the impact of this choice of pressure space on the matrices involved. 
Next, we show how 
to recover effective solvers for Stokes problems, with a preconditioner based on the singular pressure 
mass matrix, and for Oseen systems arising from linearised Navier--Stokes equations, by using 
a two-stage pressure convection--diffusion strategy.
The codes used to generate the numerical results are available online.
\end{abstract}

\maketitle
\thispagestyle{fancy}

%-------------------------------------------------------------
\section{Introduction} \label{sec:intro}

Reliable and efficient iterative solvers for  models of steady incompressible flow emerged in the  
early 1990s.  Strategies based on block preconditioning of the underlying matrix operators 
using (algebraic or geometric) multigrid components have proved to be the key to  realising  mesh 
independent convergence (and optimal complexity) without the need for tuning parameters, particularly
in the context of classical  mixed finite element approximation, see Elman 
et al.~\cite[chap.\,9]{elman14}.  \bookref 
The focus  of this contribution is on efficient solver strategies in cases 
where (an inf--sup)  stable Taylor--Hood mixed approximation is augmented by a piecewise constant 
pressure in order to guarantee local conservation of mass. The augmentation leads to 
over-specification of the pressure solution requiring a redesign of the established solver technology. 

The idea of adding a piecewise constant pressure to the standard rectangular biquadratic velocity, bilinear
pressure  ($\Q_2$--$\Q_1$) approximation was originally suggested during discussion around a
 blackboard at a conference on finite elements in fluids held in Banff in 1980; 
 see Gresho et al.~\cite{gresho81}. The need for local mass conservation was motivated by competition 
 from finite volume methods (such as the MAC scheme) in  the design of effective strategies for   
 modelling buoyancy-driven flow  in the atmosphere. 

The extension of the augmentation idea to Taylor--Hood ($\P_2$--$\P_1$)  triangular approximation was 
proposed in a paper presented at a conference  in Reading  in 1982;  see Griffiths~\cite{griffiths82}.
A  proof of stability of the augmented $\P_2$--$\P_1$  
approximation on triangular meshes  was constructed by Thatcher \& Silvester~\cite{thatcher87} in 1987. 
An extended version of this manuscript included discussion of $\Q_2$--$\Q_1$ hexahedral elements~\cite{thatcher90}. 
A rigorous assessment of the augmentation strategy was undertaken by Boffi et al.\ 
 two decades  later~\cite{boffi12}.
The strategy of augmenting a continuous pressure approximation to give a  locally mass-conserving 
strategy can also be generalised  to higher-order $\Q_k$--$\Q_{k-1}$ and  $\P_k$--$\P_{k-1}$ Taylor--Hood 
approximations. Inf--sup stability is assured for $k\geq 2$ in two dimensions and  for
 $\P_k$--$\P_{k-1}$ with $k\geq 3$ in three  dimensions; see~\cite[p.\,506]{boffi13}.
 
 The novel contribution in this work  lies in the linear algebra aspects of two-field
pressure approximation. The immediate  issue that needs to be dealt with is the fact that the
 mass matrix  that determines the {\it stability} of the resulting mixed approximation is  {\it singular}.
 This aspect is discussed in  section~\ref{sec:mass}. 
The main issue that has  to be dealt with in practical flow simulation is the over-specification 
and associated ill-conditioning of the discrete operators that arise in the 
preconditioning of the linearised Navier--Stokes operator. This is the focus of the 
discussion in section~\ref{sec:oseen}. The conclusion of this
study is that  {\it optimal} complexity convergence rates can be recovered
when using two-field pressure  approximation, but only after  
a careful redesign of  the preconditioning components. 


\section{Two-field pressure mass matrix} \label{sec:mass}

In this section we consider the Stokes problem 
\begin{alignat*}{2}
-\triangle \vfld{u} + \nabla p &= \vfld{f} & \quad & \text{ in } \Omega,\\
\nabla\cdot \vfld{u} & = 0& \quad & \text{ in } \Omega,\\
\vfld{u} &= \vfld{g} & \quad & \text{ on } \partial \Omega,
\end{alignat*}
where $\vfld{u}$ and $p$ are the fluid velocity and pressure, 
respectively, and $\Omega \subset \R^d$, 
$d \in \{2,3\}$, is a polygonal or polyhedral domain.
 
Throughout this section, we assume that
 $\Vspace \subset H_0^1(\Omega)^d$ and 
$\Qspace \subset L^2_0(\Omega):=\{q \in L_2 : \int_\Omega q = 0\}$
are an inf--sup stable pair of finite element spaces. 
Then the corresponding finite element 
approximation problem is to 
find $(\vfld{u}_h, p_h) \in \Vspace \times \Qspace$ such that 
\begin{subequations}
\label{eq:stokes_fe}
\begin{alignat}{2}
a(\vfld{u}_h,\vfld{v}_h) + b(\vfld{v}_h,p_h) 
&= (\vfld{f},\vfld{v}_h) & \quad & \text{ for all } \vfld{v}_h \in \Vspace,\label{eq:stokes_fe_momentum}\\
b(\vfld{u}_h,q_h) & = 0& \quad & \text{ for all } q_h \in \Qspace,\label{eq:stokes_fe_mass}
\end{alignat}
\end{subequations}
where $(\cdot,\cdot)$ denotes the usual $L^2$ inner product, and 
\begin{align*}
a(\vfld{u},\vfld{v}) = \int_\Omega \nabla \vfld{u} : \nabla \vfld{v},\qquad
b(\vfld{u},p)  = -\int_\Omega p\,\nabla\cdot \vfld{u}.
 \end{align*}
 
Solving the finite element problem \eqref{eq:stokes_fe} 
is then equivalent to solving the  linear system 
\begin{equation}
\label{eq:saddle_point}
\underbrace{\begin{bmatrix}
\boldA & B^T\\
B & 0
\end{bmatrix}}_{\AA}
\begin{bmatrix}
\bu\\
\bp
\end{bmatrix} 
=
\begin{bmatrix}
\boldsymbol{f}\\
\boldsymbol{g}
\end{bmatrix},
\end{equation}
where $\boldA\in\R^{\nvelocity \times \nvelocity}$ is symmetric positive definite 
and $B\in\R^{\npressure\times \nvelocity}$ \cite[chap.\,3]{elman14}.  \bookref
The matrix $\AA$ is of well-known saddle point type, 
and solvers for this sort of system have 
been extensively studied, see, e.g., Benzi et al.~\cite{BGL05}.
Since $\AA$ is large and sparse, the system is typically solved by an iterative method, 
with preconditioned MINRES \cite{PaSa75} a popular choice. 

An ideal preconditioner for $\AA$ is \cite{Kuzn95,MGW00}
\[\PP_{\text{ideal}} = \begin{bmatrix}\boldA & \\ & B\boldA^{-1}B^T\end{bmatrix},\]
since the eigenvalues of the preconditioned matrix are $0$, $1$, and $(1\pm \sqrt{5})/2$, 
with the zero eigenvalue appearing only if the preconditioned matrix is singular. 
This preconditioner is usually too costly to apply, and 
so efficient block diagonal preconditioners for $\AA$ are typically
 based on spectrally equivalent preconditioners for $\boldA$ and 
the negative Schur complement $S = B\boldA^{-1}B^T$. 
For Stokes problems, the solve with $\boldA$ can be replaced by, 
e.g., an algebraic or geometric multgirid solver, 
while for stable discretisations $S$ is spectrally equivalent to $\pressuremass$, 
the pressure mass matrix \cite[chap.\,3]{elman14}, \bookref
i.e., there exist constants $\gamma$ and $\Gamma$, independent of $h$, such that 
\begin{equation}
\label{eq:spec_q}
\gamma^2 \le \frac{\bq^TB\boldA^{-1}B^T\bq}{\bq^T\pressuremass\bq} \le \Gamma^2
\end{equation}
for all vectors $\bq$ except those corresponding to the function $q_h \equiv 1$ on $\Omega$. 
 For certain element pairs, $\pressuremass$ itself is easily inverted, e.g., 
for discontinuous $\P_0$ or $\Q_0$ pressures, the mass matrix 
is diagonal. 
Otherwise, $\pressuremass$ can be 
replaced by its diagonal or by a fixed number of steps of Chebyshev semi-iteration 
when, as is common, $\diag(\pressuremass)^{-1}\pressuremass$ has 
eigenvalues within a small interval 
and this interval lies away from the origin \cite{WaRe09,Wath87}.

To summarise, for inf--sup stable finite element pairs, 
an effective preconditioner for \eqref{eq:saddle_point} is 
\begin{equation}
\label{eq:stokes_pre}
\PP = \begin{bmatrix} \Aapprox & \\ & \Qapprox \end{bmatrix},
\end{equation}
where $\Aapprox \in \R^{\nvelocity \times \nvelocity}$ is $\boldA$ or an approximation, 
and $\Qapprox \in \R^{\npressure\times\npressure}$ 
is $\pressuremass$ or an approximation. 
(See, e.g., Elman et al.~\cite[chap.\,4]{elman14} \bookref 
for results with  $\Q_2$--$\Q_1$ approximation on quadrilaterals).

In this section, our aim is to determine effective preconditioners for the enriched Taylor--Hood element.
As we will see, although enriching the pressure space results in better mass conservation properties, 
it also presents challenges for solving \eqref{eq:saddle_point}. 

\subsection{Augmented Taylor--Hood Elements}
We see from \eqref{eq:stokes_fe} that the mass conservation condition
 $\nabla\cdot \vfld{u} = 0$ is imposed only in a weak sense, 
 and if Taylor--Hood elements are employed 
 we can only guarantee that \eqref{eq:stokes_fe_mass} will hold.  
 However, by augmenting the pressure space by piecewise constant pressures 
 it is possible to obtain local mass conservation, 
 so that  the average of the divergence is zero on each individual element. 
 However, as we will discuss later in this section, 
 a consequence is that constant pressures have multiple 
 representations, 
 and this results in certain challenges for solving \eqref{eq:saddle_point}. 

Let us now describe the enriched Taylor--Hood finite element spaces. 
We first introduce a shape-regular family of simplicial, 
quadrilateral (in 2D) or hexahedral (in 3D) decompositions of the domain $\Omega$. 
We assume that any two elements have at most a common face, 
edge, or vertex and denote  by $h$ the maximum diameter of any element. 
The total number of elements in the resulting mesh is $\nel$. 

We denote the usual Taylor--Hood finite element space by 
$\VspaceTH \times \QspaceTH$, so that $\VspaceTH\times \QspaceTH = (\Q_{k+1})^d \times \Q_{k}$ or 
 $\VspaceTH\times \QspaceTH = (\P_{k+1})^d \times \P_{k}$ with $d\geq 2$. 
In the latter case, we additionally assume that the polynomial degree, $k$, satisfies $k \ge d-1$. 
The corresponding enriched Taylor--Hood space is 
$\VspaceETH \times \QspaceETH$ where 
\begin{align}
\QspaceETH &= \{q = q_k + q_0, q_k \in \QspaceTH, q_0 \in \Qzerospace\}\label{eq:eth_pressure_space}
\end{align}
and 
$\Qzerospace$ is the space of discontinuous pressures that are constant on each element.  
Thus, we see that the velocity approximation space is identical to that of the corresponding 
Taylor--Hood element, 
while $\QspaceETH$ is $\QspaceTH$ augmented with piecewise constant pressures. 
It follows that functions in $\QspaceETH$ may be discontinuous across inter-element boundaries. 
 
From a linear algebra perspective, a critical point is that any constant function on $\Omega$ 
can be  represented by a function in $\QspaceTH$ \emph{or} a function in $\Qzerospace$. 
In other words, any constant function $\beta$ on $\Omega$ can be written as 
$p = p_k + p_0 \in \QspaceETH$, where $p_k \equiv \alpha \in \QspaceTH$ and $p_0 \equiv \beta - \alpha \in \Qzerospace$ on $\Omega$, 
for any $\alpha \in \R$.
This has profound consequences for the linear algebra: 
 the pressure mass matrix $\pressuremass$ that determines $\Qapprox$ in \eqref{eq:stokes_pre} becomes singular, 
and the rank of the matrix $B^T$ is reduced. 
These properties are established in the rest of this section.

We first let 
\begin{equation}
\label{eq:eth_basis}
\QspaceETH = \hull\{\phi_1,\dotsc, \phi_{\nCpressure},\phi_{\nCpressure+1},\dotsc,\phi_{\npressure}\},
\end{equation}
where $\{\phi_k\}_{k = 1}^{\nCpressure}$ and $\{\phi_k\}_{k = \nCpressure+1}^{\npressure}$
are Lagrange bases of $\QspaceTH$ and $\Qzerospace$ (see \eqref{eq:eth_pressure_space}). 
This enables us to relate vectors $\bp \in \R^{\npressure}$ 
and functions $p=p_k+p_0\in \QspaceETH$, where $p_k \in \QspaceTH$, $p_0 \in \Qzerospace$. 
Specifically, 
\begin{equation}
\label{eq:function_basis}
p = \sum_{i = 1}^{\npressure} \bp_i \phi_i, \quad 
p_k = \sum_{i = 1}^{\nCpressure} \bp_i \phi_i \quad \text{ and }\quad  
p_0 = \sum_{i = \nCpressure+1}^{\npressure} \bp_i \phi_i.
\end{equation}

As mentioned above, any constant function has multiple representations in $\QspaceETH$. 
In particular, for any $\alpha \in \R$, setting $p_k = \alpha$ and $p_0 = -\alpha$ 
gives that $p = p_k + p_0 \equiv 0$. 
 We see from \eqref{eq:function_basis} that this representation of the zero function corresponds to the 
 vector $\alpha \qnull$, where 
\begin{equation}
\label{eq:pressure_null_vector}
\qnull = 
\begin{bmatrix}
\boldsymbol{1}_{n_k}\\
-\boldsymbol{1}_{n_0}
\end{bmatrix}
\end{equation}
and $n_0 = n_p-n_k$. 

A direct consequence of the correspondence between $\qnull$ and the zero function on $\Omega$
 is that $\pressuremass\qnull = \boldsymbol{0}$ and $B^T\qnull = \boldsymbol{0}$, 
 as we now show. 

\begin{proposition}
\label{prop:Qnull}
Let 
\[\pressuremass = [q_{ij}]_{i,j = 1}^{\npressure}, \quad q_{ij} = \int_\Omega \phi_i \phi_j,\] 
be the pressure mass matrix for the enriched Taylor--Hood pressure space 
$\QspaceETH$ in \eqref{eq:eth_pressure_space}, with basis functions \eqref{eq:eth_basis}. 
Then, $\nullspace(\pressuremass) = \hull\{\qnull\}$, where $\qnull$ is given in \eqref{eq:pressure_null_vector}. 
\end{proposition}

\begin{proof}
Let $\bp \in \R^{\npressure}$, $\bp\ne \boldsymbol{0}$. Then, using \eqref{eq:function_basis}, we find that  
\begin{align*}
\pressuremass\bp = 0 \quad 
& \Leftrightarrow \quad \bp^T \pressuremass \bp=0\\
& \Leftrightarrow \quad  \int_\Omega\left(\sum_{i = 1}^{\npressure} \bp_i \phi_i\right) 
\left( \sum_{j = 1}^{\npressure}\bp_j \phi_j\right)  = 0\\
& \Leftrightarrow \quad  \int_\Omega p^2  = 0\qquad \Leftrightarrow \quad  p \equiv 0 \text{ in } \Omega,
\end{align*}
since $p$ is continuous on each element. 

As we have already seen in this section, 
$p \equiv 0$ corresponds to the vector $\alpha \qnull$. 
This shows that $\hull\{\qnull\} \subseteq \nullspace(\pressuremass)$. 
We now show that there are no other vectors in the nullspace. 
Since $p = p_k + p_0$, $p_k \in \QspaceTH$, $p_0 \in \Qzerospace$, 
we see that $p_k = -p_0$ everywhere in $\Omega$. 
Moreover, $p_0$ is constant on each element $T$, $T = 1,\dotsc, \nel$, 
so  $p_0 \equiv \alpha_T$ on $T$ for some constant $\alpha_T \in \R$. 
Hence, $p_k \equiv -\alpha_T$ on $T$. 
However, $p_k$ is continuous on $\overline{\Omega}$, and so 
it is straightforward to see that $\alpha_1 = \alpha_2 = \dotsc = \alpha_{\nel}$, 
i.e., that $p_0 \equiv \alpha$ and 
$p_k \equiv -\alpha$ on $\Omega$ for some constant $\alpha\in\R$. 
Such functions correspond to vectors of the form $\alpha \qnull$, 
which shows that  $\nullspace(\pressuremass) = \hull\{\qnull\}$, as required.
\end{proof}

A very similar argument shows that $\qnull \in \nullspace(B^T)$. 
\begin{proposition}
\label{prop:Bnull}
Let 
$B\in \R^{\npressure\times \nvelocity}$ in \eqref{eq:saddle_point} be 
obtained from the enriched Taylor--Hood space
$\VspaceETH\times \QspaceETH$, 
using the basis functions for $\QspaceETH$ in \eqref{eq:eth_basis}. 
Then, 
$\qnull\in\nullspace{(B^T)}$,
where $\qnull$ is given in \eqref{eq:pressure_null_vector}. 
\end{proposition}
\begin{proof}
For any $\bv \in \R^{\nvelocity}$, we have that 
\[  \bv^T B^T \qnull = \qnull^T B \bv = \int_\Omega p \nabla \cdot \vfld{v}\]
with $\vfld{v} \in \VspaceETH$. 
Additionally,  
from \eqref{eq:function_basis} and \eqref{eq:pressure_null_vector} it is clear that 
\begin{align*}
p = \sum_{i = 1}^{\nCpressure} \qnull_i \phi_i + \sum_{i = \nCpressure+1}^{\npressure} \qnull_i \phi_i = 
 \sum_{i = 1}^{\nCpressure} \phi_i - \!\!\! \sum_{i = \nCpressure+1}^{\npressure} \phi_i 
 \end{align*}
 is identically zero, so that
\(\bv^T B^T \qnull = 0.\)
Since $\bv$ was arbitrarily chosen, we conclude that $B^T\qnull = \boldsymbol{0}$.
\end{proof}

What do 
these results mean for the solution of \eqref{eq:saddle_point}
 by preconditioned MINRES? 
 First, 
if $B^T\qnull = \boldsymbol{0}$  then it
 follows that $\AA$ is always singular, with $\AA\anull = \boldsymbol{0}$, 
 where 
\begin{equation}
\label{eq:anull}
\anull = \begin{bmatrix} \boldsymbol{0}_{\nvelocity}\\ \qnull \end{bmatrix}.
\end{equation}
If the linear system \eqref{eq:saddle_point} is consistent then this 
does not pose a problem for preconditioned MINRES.
However, as we shall now see, the proposed block diagonal preconditioner  \eqref{eq:stokes_pre}, 
with $\Qapprox = \pressuremass$,  
is also singular, and this is potentially problematic. 



\begin{algorithm}{Preconditioned MINRES algorithm for solving $\AA \boldsymbol{x} = \boldsymbol{b}$ 
with symmetric positive definite preconditioner $\PP$ \cite[Algorithm 4.1]{elman14}.  \bookref}
\label{alg:minres}
\begin{algorithmic}[1]
\State $\bv^{(0)} = \boldsymbol{0}$, $\bw^{(0)} =  \boldsymbol{0}$, $\bw^{(1)} =  \boldsymbol{0}$, $\gamma_0 = 0$
\State Choose $\boldsymbol{x}^{(0)}$, compute $\bv^{(1)} = \boldsymbol{b} - \AA\boldsymbol{x}^{(0)}$
\State Solve $\PP \bz^{(1)} = \bv^{(1)}$, set $\gamma_1 = \sqrt{\langle \bz^{(1)}, \bv^{(1)}\rangle}$
\State Set $\eta = \gamma_1$, $s_0 = s_1 = 0$, $c_0 = c_1 = 1$
\For {$j=1$ until convergence}
\State $\bz^{(j)} = \bz^{(j)}/\gamma_j$
\State $\delta_j = \langle A\bz^{(j)}, \bz^{(j)}\rangle$
\State $\bv^{(j+1)} = \AA\bz^{(j)} - (\delta_j/\gamma_j)\bv^{(j)}-(\gamma_j/\gamma_{j-1})\bv^{(j-1)}$
\State Solve $\PP \bz^{(j+1)} = \bv^{(j+1)}$
\State $\gamma_{j+1} = \sqrt{\langle \bz^{(j+1)}, \bv^{(j+1)}\rangle}$
\State $\alpha_0 = c_j\delta_j - c_{j-1}s_j\gamma_j$
\State $\alpha_1 = \sqrt{\alpha_0^2 + \gamma_{j+1}^2}$
\State $\alpha_2  = s_j\delta_j + c_{j-1}c_j\gamma_j$
\State $\alpha_3 = s_{j-1}\gamma_j$
\State $c_{j+1} = \alpha_0/\alpha_1$; $s_{j+1} = \gamma_{j+1}/\alpha_1$
\State $\bw^{(j+1)} = (\bz^{(j)} - \alpha_3\bw^{(j-1)} - \alpha_2\bw^{(j)})/\alpha_1$
\State $\boldsymbol{x}^{(j)} = \boldsymbol{x}^{(j-1)} + c_{j+1}\eta \bw^{(j+1)}$
\State $\eta = -s_{j+1}\eta$
\State <Test for convergence>
\EndFor
\end{algorithmic}
\end{algorithm}

\subsection{Dealing with a singular preconditioner}
\label{sec:singular_preconditioner}
Knowing that the enriched Taylor--Hood element is inf--sup stable
implies that the preconditioner \eqref{eq:stokes_pre} will be effective when solving \eqref{eq:saddle_point}. 
However, the matrix $\pressuremass$, and hence the preconditioner $\PP$, are singular, 
since $\PP\anull = \boldsymbol{0}$, where $\anull$ is given in \eqref{eq:anull}. 
(Note that this implies that  $\AA$ and $\PP$ have a common nullspace.) 


To better understand the effect of a singular preconditioner $\PP$, we need to explore
the components of the preconditioned MINRES method that is presented in Algorithm~\ref{alg:minres}. 
Specifically, we note that at each  iteration step we need to 
solve a linear system of the form $\PP\bz^{(j)} = \bv^{(j)}$. 
If this system is consistent there are infinitely many solutions, 
which take the form $\bz^{(j)} = \zrange^{(j)} + \zeta_j \anull$, 
with $\zrange^{(j)}\perp \anull$ and $\zeta_j \in \R$. 

Let us examine the effect of $|\zeta_j|$ on the scalars and vectors computed in 
Algorithm~\ref{alg:minres}. 
First, note that if the linear system $\AA \boldsymbol{x} = \boldsymbol{b}$ is 
consistent then, since $\AA$ is symmetric, $\boldsymbol{b}\perp \anull$
and so $\bv^{(1)} \perp \anull$. 
It then follows by induction that 
$\bv^{(j)}\perp \anull$, $j = 1,2,\dotsc$, and that the systems $\PP\bz^{(j)} = \bv^{(j)}$
are all consistent. 
Furthermore, 
$\gamma_j = \langle \bz^{(j)},\bv^{(j)}\rangle^{\frac{1}{2}} = \langle \zrange^{(j)},\bv^{(j)}\rangle^{\frac{1}{2}}$ because 
of the orthogonality of $\bv^{(j)}$ and $\anull$, so that $\zeta_j$ does not affect $\gamma_j$. 
Similarly, $\delta_j = \langle \AA \bz^{(j)},\bz^{(j)}\rangle =  \langle \AA \zrange^{(j)},\zrange^{(j)}\rangle$, 
which shows that $\delta_j$ is similarly unaffected by $\zeta_j$. 
Indeed, the only quantities that are affected by the nullspace components $\zeta_j \anull$, $j = 1,2,\dotsc$,  
are the vectors $\boldsymbol{w}^{(j)}$ and  $\boldsymbol{x}^{(j)}$.
In exact arithmetic, this is not a problem, since solutions of 
 $\AA \boldsymbol{x} = \boldsymbol{b}$ may certainly contain a component in the direction of $\anull$. 
However, in unlucky cases, the size of the nullspace component of $\boldsymbol{x}^{(j)}$ 
may be so large as to dominate the approximate solution. 
Alternatively, in finite precision $\bv^{(j)}$ and $\bw$ may not be exactly orthogonal. 
Hence, it may be wise to explicitly ensure that $\bz \perp \anull$. 
One option is to orthogonalise $\bz$ against $\anull$ after each preconditioner solve, 
but if $|\zeta_j|$ is large then the result may be inaccurate. 
A more robust approach is to note that solutions of $\PP\bz = \br$ are minimisers of the quadratic form 
\[\frac{1}{2}\bz^T\PP\bz - \bz^T\br,\]
since $\PP$ is positive semidefinite. 
Constraining $\bz$ to be orthogonal to $\anull$ is then equivalent to the following optimisation problem: 
\[\min_{\bz} \frac{1}{2}\bz^T\PP\bz - \bz^T\br \quad \text{s.t.} \quad \anull^T \bz = 0.\]
Applying a Lagrange multiplier approach results in the augmented system 
\[\begin{bmatrix} \PP & \anull\\ \anull^T & 0 \end{bmatrix} 
\begin{bmatrix} \bz\\\lambda\end{bmatrix} 
= 
\begin{bmatrix}\br\\0\end{bmatrix},\]
where $\lambda$ is the Lagrange multiplier, 
and solving this augmented system gives a vector $\bz$ that is orthogonal to $\anull$. 
Moreover, since 
\[
\begin{bmatrix} \PP & \anull\\ \anull^T & 0 \end{bmatrix}
 = 
 \begin{bmatrix} \Aapprox & 0& 0\\0 & \Qapprox & \qnull\\ 0& \qnull^T &0 \end{bmatrix}
 \]
we see that only the solve with $\Qapprox$ needs to be modified. 

\subsection{Approximating the two-level pressure mass matrix}
Now let us consider approximations of the matrix $\pressuremass$ which, 
because of the two-level pressure approximation, has $2\times 2$ block structure: 
\[\pressuremass = \begin{bmatrix} Q_{k} & R^T\\ R & Q_{0}\end{bmatrix},\]
where 
\begin{alignat*}{2}
Q_k &= [q_{k,ij}], \, i,j = 1,\dotsc, \nCpressure, &\qquad  q_{k,ij}& = \int_\Omega \phi_j \, \phi_i,\\
 R &= [r_{ij}], \, i = \nCpressure + 1,\dotsc, \npressure, \, j = 1,\dotsc, \nCpressure, & \qquad  r_{ij} &= \int_\Omega \phi_j \, \phi_i,\\
Q_0 &= [q_{0,ij}], \, i,j = \nCpressure + 1,\dotsc, \npressure, &\qquad  q_{0,ij}& = \int_\Omega \phi_j \, \phi_i.
\end{alignat*}
Note that $Q_k$ is the standard Taylor--Hood pressure mass matrix, 
and $Q_0$ is the standard discontinuous pressure mass matrix, 
while $R$ represents cross terms between the spaces 
$\QspaceTH$ and $\Qzerospace$ (see \eqref{eq:eth_pressure_space}).

As discussed at the start of this section, $Q_0$ is diagonal, 
and hence easy to invert, 
while good approximations of $Q_k$
 based on its diagonal or Chebyshev semi-iteration are known. 
 However, the presence of the matrix $R$ in $\pressuremass$ complicates matters considerably.  
For example, replacing $Q_k$ by its diagonal or Chebyshev semi-iteration results in 
 mesh-dependent convergence rates and high iteration counts. 

Similar issues arise when the whole matrix $\pressuremass$ is approximated 
using Chebyshev semi-iteration. 
We illustrate the difficulty for lowest-order ($\P_1$) elements on triangles. 
First note that $\pressuremass$ is assembled from contributions on each element. 
If we again order the vertex degrees of freedom before the centroid degree of freedom 
then each element mass matrix is of the form 
\[
\pressuremass_{T}
= \frac{\triangle_T}{12}
\begin{bmatrix}
2 & 1 & 1 & 4\\
1 & 2 & 1 & 4\\ 
1 & 1 & 2 & 4\\
4 & 4 & 4 & 12
\end{bmatrix},
\]
where $\triangle_T$, $T = 1,\dotsc, \nel$, is the size of the $T$th triangular element. 
These element pressure mass matrices can be used to bound the eigenvalues of 
$\diag(\pressuremass)^{-1}\pressuremass$ via the method described by Wathen \cite{Wath87}. 
Doing so shows that the eigenvalues of $\diag(\pressuremass)^{-1}\pressuremass$ 
are contained in $[0,3]$, and indeed these bounds are tight. 
The presence of a zero eigenvalue, and a number of additional very small eigenvalues, 
means that any approximation of $\pressuremass$ based on its diagonal, including 
Chebyshev semi-iteration, is unlikely to give good results, and  this is what we see in practice. 
More generally, we have been unable to find a spectrally equivalent approximation to $\pressuremass$ that is efficient, 
and so in the following we use the original sparse matrix $\pressuremass$ in our  preconditioning strategies.

\subsection{Reliable computation of the discrete inf--sup constant}
A typical strategy to estimate the inf--sup constant for \eqref{eq:stokes_fe}
is to find the largest $\gamma$ that satisfies \eqref{eq:spec_q}, 
i.e., to find the smallest nonzero eigenvalue of the generalised eigenvalue problem 
\begin{equation}
\label{eq:inf-sup_eig}
B\boldA^{-1}B^T \bv = \lambda \pressuremass\bv.
\end{equation}
However, in our case, Propositions \ref{prop:Qnull} and \ref{prop:Bnull} 
show that $\qnull$ lies in the nullspaces of both $B^T$ and $\pressuremass$, 
which means that this generalised eigenvalue problem is singular, 
i.e., \emph{any} $\lambda\in\mathbb{R}$ satisfies \eqref{eq:inf-sup_eig} when $\bv = \qnull$. 
It is known that generalised eigenvalue problems with singular pencils are challenging to solve
numerically~\cite{HMP19}, 
and additional checks must be performed to ensure that an estimate of $\gamma$
 is not associated with the eigenvector $\qnull$. 
In practice we find that standard methods for computing eigenvalues of sparse 
 matrices may struggle to accurately compute these eigenvalues, 
 precisely because $B\boldA^{-1}B^T$ and $\pressuremass$ are singular. 

On the other hand, the EST-MINRES approach proposed in \cite{SiSi11} 
is much more robust and gives consistently reliable results. 
The intuition is that by ensuring that any solves with $\pressuremass$ 
are orthogonal to $\qnull$ within the preconditioned MINRES method, 
so that the MINRES basis vectors are orthogonal to $\anull$, 
as described in section~\ref{sec:singular_preconditioner}, we 
instead solve \eqref{eq:inf-sup_eig} for $\bv \perp \qnull$.  

We illustrate the EST-MINRES inf--sup constant estimates using two representative test problems 
that we describe below. 
All numerical results were obtained using T-IFISS~\cite{BRS21} (for triangular elements) and 
a IFISS3D~\cite{PPS22} (for cubic elements). 
The stopping criterion for preconditioned MINRES 
is a reduction of the norm of the preconditioned 
residual by eight orders of magnitude,
 i.e., $\|\boldsymbol{r}_k\|_{\PP^{-1}}/\|\boldsymbol{r}_0\|_{\PP^{-1}} < 10^{-8}$. 
We apply the preconditioner 
\eqref{eq:stokes_pre} with $\Aapprox = \boldA$ and $\Qapprox = \pressuremass$.
Although this preconditioner can be expensive to apply, because it involves exact solves with 
$\boldA$, we have used it here to more clearly illustrate the key findings of this section, namely, 
that \eqref{eq:stokes_pre} with $\Qapprox =\pressuremass$ is an effective preconditioner for 
augmented Taylor--Hood problems, despite the singularity of $\pressuremass$, 
and that EST-MINRES provides reliable approximations to 
$\gamma$, the discrete inf--sup constant in \eqref{eq:spec_q}.
However, we note that it is possible to replace $\boldA$ by, e.g., an algebraic multigrid method 
such as the HSL code MI20 \cite{hsl}. With this AMG approximation, 
we were able to solve a 3D problem with nearly $10^6$ 
degrees of freedom in under 10 seconds 
on a standard desktop machine. 


\begin{examp}[two-dimensional enclosed flow]\label{ex:2d_cavity}
Our first example is a classical driven-cavity flow in the square domain  $D=[-1,1]^2$. 
A Dirichlet no-flow condition is imposed on the 
bottom and side boundaries, while on the lid the nonzero tangential velocity is 
$u_y= 1- x^4$. 
The domain is subdivided uniformly into $n^2$ bisected squares. 
We use both the standard $\P_2$--$\P_1$ Taylor--Hood 
mixed approximation, and the augmented $\P_2$--$\P_1^\ast$ approximation. 
The two components of the pressure solution for the $\P_2$--$\P_1^\ast$ approximation, 
computed  in the case $n=32$, are illustrated
in Fig.~\ref{fig:stokes_testproblem_2d}.   The centroid pressure field is  concentrated in the 
two corners where the pressure is singular, and the centroid pressures are  an  order of magnitude
smaller than  the vertex pressure in all elements. 
Consequently, the overall pressure field is visually identical to
 the $\P_1$ pressure field shown in the left plot.
 \end{examp}
 
 \begin{figure}
       \begin{center}
	\includegraphics[width=0.7\linewidth,clip=true,trim = 2cm 4cm 1cm 2cm]{cavity_stokes_2d_pressure.eps}		 
	\end{center}%\vspace{-4mm}
\caption{Representative $\P_2$--$\P_1^*$ pressure field solution for the cavity flow in 
Example \ref{ex:2d_cavity} computed on a uniform  mesh with 2048 right-angled  triangles and 1089 vertices.}
\label{fig:stokes_testproblem_2d}
\end{figure}
 
 
 \begin{examp}[three-dimensional enclosed flow]\label{ex:3d_cavity}
 Our second problem is a three-dimensional version of driven-cavity flow. 
 The domain is now $D = [-1,1]^3$. 
As in the previous example, the flow is enclosed, but now 
the nonzero tangential velocity 
$u_y= (1- x^4)(1-z^4)$ is specified on the top of the cavity.
The domain is subdivided uniformly into $n^3$ cubic elements, and 
we use $\Q_2$--$\Q_1$ and $\Q_2$--$\Q_1^\ast$ approximations. 
 \end{examp}


Table \ref{tab:2d_infsup} shows preconditioned MINRES iteration counts and EST-MINRES discrete inf--sup constant 
approximations for Example \ref{ex:2d_cavity}. We first note that for both the $\P_2$--$\P_1$ and $\P_2$--$\P_1^\ast$
approximations the iteration counts are quite similar and are mesh independent. 
In both cases the inf--sup constant approximations also appear to be converging from above. 
The approximation for $\P_2$--$\P_1^\ast$ elements appears to rapidly converge to four digits, 
indicating that even a relatively coarse grid is sufficient to obtain an approximation to the 
discrete inf--sup constant. 
However, the approximation for $\P_2$--$\P_1$ elements 
appears to converge more slowly. 
We also note that the two approaches give different inf--sup constant 
estimates, at least for the grids shown here.  
This is not so surprising as the matrices in \eqref{eq:spec_q} 
depend on the choice of finite element spaces. 

Fig.~\ref{fig:2d_infsup} plots the inf--sup approximations at each iteration of preconditioned MINRES 
for Example \ref{ex:2d_cavity}. 
We see that a good approximation of the inf--sup constant is obtained 
after 20--25 iterations. It is again clear that for the enriched Taylor--Hood 
approximations we obtain very similar approximations for all grids. 

\begin{table}
\begin{tabular}{r| r r r r | r r r r}
\hline
 & \multicolumn{4}{c|}{$\P_2$--$\P_1$} & \multicolumn{4}{c}{$\P_2$--$\P_1^\ast$}\\
Grid& Velocity dof & Pressure dof & Iters & $\gamma^2$ & Velocity dof & Pressure dof & Iters & $\gamma^2$\\
\hline
4 & 2178 & 289 & 37 & 0.1947 & 2178 & 801 & 42 & 0.1397\\
5 & 8450 & 1089 & 37 & 0.1926 & 8450 & 3137 & 42 & 0.1396\\
6 & 33282 & 4225 & 39 & 0.1911 & 33282 & 12417 & 40 & 0.1395\\
7 & 132098 & 16641 & 37 & 0.1898 & 132098 & 49409 & 40 & 0.1395\\
8 & 526338 & 66049 & 37 & 0.1888 & 526338 & 197121 & 40 & 0.1395\\
\hline
\end{tabular}
\caption{Preconditioned MINRES iterations and discrete inf--sup constant approximations for Example \ref{ex:2d_cavity} 
and preconditioner \eqref{eq:stokes_pre} with $\Aapprox = \boldA$ and $\Qapprox = \pressuremass$.}
\label{tab:2d_infsup}
\end{table}

\begin{figure}
\begin{center}
\begin{subfigure}[b]{0.4\textwidth}
         \centering
         \includegraphics[width=\textwidth]{Cavity_2D_P2P1}
     \end{subfigure}
     \begin{subfigure}[b]{0.4\textwidth}
         \centering
         \includegraphics[width=\textwidth]{Cavity_2D_P2P1P0}
     \end{subfigure}
     \end{center}
\caption{EST-MINRES estimates of the discrete inf--sup constant $\gamma^2$ at each iteration for Example \ref{ex:2d_cavity} 
with $\P_2$--$\P_1$ (left) and $\P_2$--$\P_1^\ast$ (right) elements for the grids specified in Table \ref{tab:2d_infsup}. 
The preconditioner is \eqref{eq:stokes_pre} with $\Aapprox = \boldA$ and $\Qapprox = \pressuremass$.}
\label{fig:2d_infsup}
\end{figure}

The results for Example \ref{ex:3d_cavity} are broadly similar, as can be seen from Table \ref{tab:3d_infsup} and 
Fig.~\ref{fig:3d_infsup}. The preconditioned MINRES iteration counts are mesh independent and, for 
all but the coarsest mesh, are almost identical for the two element pairs. 
The discrete inf--sup constant approximations again appear to converge from above. Now, at least for the 
grids presented here, the estimates of 
$\gamma^2$  
for Taylor--Hood elements seem to ``track'' the  augmented Taylor--Hood estimates, i.e., the 
Taylor--Hood approximation on grid $j$ is almost the same as the augmented Taylor--Hood estimate on 
grid $j-1$. 
We also see from Fig.~\ref{fig:2d_infsup}, which plots the inf--sup approximations at each iteration of preconditioned MINRES, 
that the discrete inf--sup constant is already reasonably well approximated after 25 iterations. 


\begin{table}
\begin{tabular}{r |r r r r | r r r r}
\hline
& \multicolumn{4}{c|}{$\Q_2$--$\Q_1$} & \multicolumn{4}{c}{$\Q_2$--$\Q_1^\ast$}\\
Grid & Velocity dof & Pressure dof & Iters & $\gamma^2$ & Velocity dof & Pressure dof & Iters & $\gamma^2$\\
\hline
3 & 2187 & 125 & 45 & 0.1128 & 2187 & 189 & 48 & 0.1122\\
4 & 14739 & 729 & 51 & 0.1122 & 14739 & 1241 & 52 & 0.1115\\
5 & 107811 & 4913 & 51 & 0.1116 & 107811 & 9009 & 52 & 0.1110\\
\hline
\end{tabular}
\caption{Preconditioned MINRES iterations and discrete inf--sup constant approximations for Example \ref{ex:3d_cavity} 
and preconditioner \eqref{eq:stokes_pre} with $\Aapprox = \boldA$ and $\Qapprox = \pressuremass$.}
\label{tab:3d_infsup}
\end{table}


\begin{figure}
\begin{center}
\begin{subfigure}[b]{0.4\textwidth}
         \centering
         \includegraphics[width=\textwidth]{Cavity_3D_Q2Q1}
     \end{subfigure}
     \begin{subfigure}[b]{0.4\textwidth}
         \centering
         \includegraphics[width=\textwidth]{Cavity_3D_Q2Q1Q0}
     \end{subfigure}
     \end{center}
\caption{EST-MINRES estimates of the discrete inf--sup constant $\gamma^2$ at each iteration for Example \ref{ex:3d_cavity} 
with $\Q_2$--$\Q_1$ (left) and $\Q_2$--$\Q_1^\ast$ (right) elements for the grids specified in Table \ref{tab:3d_infsup}. 
The preconditioner is \eqref{eq:stokes_pre} with $\Aapprox = \boldA$ and $\Qapprox = \pressuremass$.}
\label{fig:3d_infsup}
\end{figure}



\section{Two-field pressure preconditioning strategies for Oseen flow.} \label{sec:oseen}

The focus of this section is  the discrete matrix system
\begin{align} \label{oseen-system}
\left[
\begin{array}{@{}cc@{}}
\boldsymbol{F}  & B^T \\ B & 0
\end{array}
\right]
\left[
\begin{array}{@{}c@{}}
\bfu \\ \bfp
\end{array}
\right]
=
\left[
\begin{array}{@{}c@{}}
\bff \\ \bfg
\end{array}
\right],
\end{align}
that arises from applying fixed point iteration, see for example 
\cite[equation\,(8.46)]{elman14}, \bookref 
to the spatially discretised Navier--Stokes equations. 
The unknown coefficient vector involves   the discrete velocity vector $\bfu \in \R^{\nvelocity}$ and the
two-field pressure vector $\bfp \in \R^{n_p}$. 
The nonsymmetry of the matrix $\boldsymbol{F}$  means that the
iterative solver of choice is GMRES (see~\cite[section\,9.1]{elman14}) \bookref 
together with a block preconditioning operator of the form
\begin{align} \label{nse-precon-diag}
\MM =
\left[
\begin{array}{@{}cc@{}}
\boldsymbol{M} & B^T \\ 0 & - M_S
\end{array}
\right],
\end{align}
where $\boldsymbol{M}$ is an optimal complexity (multigrid) operator effecting the action 
of the inverse of the matrix  $\boldsymbol{F}$ and 
$M_S$ is an optimal complexity approximation  of the Schur complement matrix
$B\boldsymbol{F}^{-1}B^T$.
We will discuss results for two representative flow problems herein. 

\begin{figure}[!ht]
       \begin{center}
	\includegraphics[width=0.95\linewidth]{steppressure.eps}		 
	\end{center}%\vspace{-4mm}
\caption{Representative $\P_2$--$\P_1^*$ pressure field solution for flow over a step 
($\RE = 100$) computed 
on a uniform   mesh with  5632 right-angled  triangles and 2945 vertices.}
\label{fig.testproblem1}
\end{figure}

\begin{examp}[flow over a step]\label{ex.step}
We consider an inflow--outflow problem
defined  on the domain  $D=[-1,0) \times [0,1]  \cup [0,5]\times[-1,1]$   with the viscosity parameter 
$\nu$ set to $1/50$ and  a parabolic velocity 
$u_x=  4y (1 - y)$ specified on the inflow boundary $x=-1$.  The  construction of  the standard
weak formulation (see \cite[p.\,127]{elman14}) \bookref 
gives rise to a  natural  boundary condition that fixes the hydrostatic pressure
level by weakly enforcing a zero mean pressure at the outflow boundary $x=5$. (The other 
boundary conditions are associated with fixed walls.)  
We consider  the discrete system \eqref{oseen-system}
that results after 5 fixed-point iterations  of the discretised Navier--Stokes system
starting from the corresponding Stokes flow solution.
We generate  solutions using  $\Q_2$--$\Q_1$  or $\P_2$--$\P_1$ augmented Taylor--Hood 
mixed approximation with the domain subdivided uniformly into (bisected) squares. 
The resulting system \eqref{oseen-system} is singular with
the one-dimensional pressure nullspace described in section~2. \backref
The two components of the pressure solution computed on a representative $\P_2$--$\P_1$ mesh 
are illustrated in Fig.~\ref{fig.testproblem1}.  The centroid pressure  values  are  an  order of magnitude
smaller than  the vertex pressure in all the elements---they provide the ``corrections'' to the
vertex pressures that are needed  to ensure  local (elementwise)  conservation of mass.
 \end{examp}
 
 
\begin{examp}[two-dimensional enclosed flow]\label{ex.cavity}
We consider the  classical driven-cavity enclosed flow problem
defined  on the domain  $D=[-1,1]^2$   with the viscosity parameter 
$\nu$ set to $1/100$ and a nonzero tangential velocity 
$u_y= 1- x^4$ specified on the top of the cavity. We take $\P_2$--$\P_1$ augmented Taylor--Hood 
mixed approximation with the domain subdivided uniformly into 
$n^2$  bisected squares. We consider  the discrete system \eqref{oseen-system}
 that arises after 5 fixed-point iterations  starting from the corresponding Stokes flow solution. 
 The discrete system is singular with a  two-dimensional pressure nullspace, corresponding
 to a constant vertex pressure and a constant centroid pressure.
The two components of the pressure solution computed  in the case $n=32$ are illustrated
in Fig.~\ref{fig.testproblem2}.  Since the centroid pressure  values  are  an  order of magnitude
smaller than  the vertex pressure in all elements, the overall pressure field is visually identical to
 the $\P_1$ pressure field shown in the left plot.
 As might be anticipated, the centroid ``correction''  pressure field is   concentrated in the two 
 corners where the pressure is singular.
 \end{examp}
 
 \begin{figure}[!t]
       \begin{center}
	\includegraphics[width=0.7\linewidth]{cavitypressure.eps}		 
	\end{center}%\vspace{-4mm}
\caption{Representative $\P_2$--$\P_1^*$ pressure field solution for cavity flow 
($\RE=200$) computed on a uniform  mesh with 2048 right-angled  triangles and 1089 vertices.}
\label{fig.testproblem2}
\end{figure}
 
 
 
\subsection{Pressure convection-diffusion preconditioning for Oseen flow.} \label{sec:pcd}
There are two alternative ways of approximating the key matrix
$B\boldsymbol{F}^{-1}B^T$ in the case that $B^T$ is generated by a two-field
pressure approximation so that $B^T = [ B_1^T, B_0^T ]$.
 The focus will be on {pressure convection-diffusion} (PCD)
 preconditioning  in this section. 
Both  of the  Schur complement  approximations can be motivated by starting with the Oseen 
matrix operator (\ref{oseen-system}) and observing that the 
diagonal blocks of $\boldsymbol{F}$
are discrete representations of the convection--diffusion operator
\begin{align} \label{generic-conv-diff}
\mathcal{L} = -\nu \nabla^2 + \vfld{w}_h \cdot \nabla ,
\end{align}
defined on the velocity space. In practical calcuations $\nu>0$ is proportional to the 
inverse of the flow Reynolds number  and
$\vfld{w}_h$ is the discrete approximation to the flow 
velocity computed at the most recent nonlinear iteration.
The PCD approximation  supposes that there is an analogous operator
to  (\ref{generic-conv-diff}), namely
\begin{align}\label{pressure-conv-diff}
\mathcal{L}_p = (-\nu \nabla^2 + \vfld{w}_h \cdot \nabla)_p
\end{align}
defined on the two components of the augmented  pressure space. 

To this end,  defining $\{\phi_j\}_{j=1}^{n_{1}}$   to be  the basis for the 
$C^0$ pressure discretization, we construct matrices $Q_1$ and $F_1$ so that
\begin{align*}
Q_1 &=[q_{1,ij}],{\quad}q_{1,ij} = \int_{\Omega}
\phi_j \,  \phi_i \\
F_1 &= [f_{1,ij}], \quad f_{1,ij} = 
\nu \int_{\Omega} \nabla \phi_j \cdot \nabla \phi_i +
\int_{\Omega} (\vfld{w}_h \cdot \nabla \phi_j) \, \phi_i .
\end{align*}
We then note that if the commutator  with the divergence operator
\begin{align*} %\label{commutator}
\mathcal{E} = \nabla \cdot (-\nu \nabla^2 + \vfld{w}_h \cdot \nabla) -
(-\nu \nabla^2 + \vfld{w}_h \cdot \nabla)_p \, \nabla \cdot
\end{align*}
is  small then we have the approximation
\begin{align} \label{discrete-commutator1}
0 \approx (Q_1^{-1}B_1)\, (\boldsymbol{M}^{-1}\boldsymbol{F}) -
(Q_1^{-1}F_1) \, (Q_1^{-1}B_1) 
\end{align}
where $\boldsymbol{M}$ is the diagonal of the mass matrix associated with
the basis representation of the velocity space.\footnote{The inverse of $\boldsymbol{M}$ is
a dense matrix. The diagonal of   $\boldsymbol{M}$ is a spectrally equivalent matrix operator
with a sparse (diagonal)  inverse.} Rearranging  
 (\ref{discrete-commutator1}) gives the first Schur complement approximation
 \begin{align} \label{discrete-commutator1x}
B_1 \, \boldsymbol{F}^{-1} B^T \approx  Q_1 F_1^{-1} (B_1\, \boldsymbol{M}^{-1} B^T) .
\end{align}

A discrete version of  $\mathcal{L}_p$  for
the piecewise constant pressure space can be generated by considering the
jumps in pressure across inter-element boundaries; see 
\cite[pp.\,268--370]{elman14}. \bookref 
To this end, defining $\{\varphi_j\}_{j=1}^{n_{0}}$   to be  the (indicator function)  basis for the 
discontinuous  pressure, we construct matrices $Q_0$ and $F_0$ via
\begin{align*}
Q_0 &=[q_{0,ij}],{\quad}q_{0,ij} = \int_{\Omega}
\varphi_j \,  \varphi_i  =  
\begin{cases} |T_i | & \hbox{if }  i=j , \\
                      \; 0 & \hbox{otherwise},  \end{cases} \\
F_0 &= [f_{0,ij}], \quad f_{0,ij} = 
\nu \sum_{T\in \TT_h} \int_{T} \nabla \varphi_j \cdot \nabla \varphi_i +
\sum_{T\in \TT_h}  \int_{T} (\vfld{w}_h \cdot \nabla \varphi_j) \, \varphi_i 
\end{align*}
and note that if the commutator  with the divergence operator
is  small then we have 
\begin{align} \label{discrete-commutator2}
0 \approx (Q_0^{-1}B_0)\, (\boldsymbol{M}^{-1}\boldsymbol{F}) -
(Q_0^{-1}F_0) \, (Q_0^{-1}B_0) ,
\end{align}
suggesting the second Schur complement approximation
 \begin{align} \label{discrete-commutator2x}
B_0\, \boldsymbol{F}^{-1} B^T \approx  Q_0 F_0^{-1} (B_0 \, \boldsymbol{M}^{-1} B^T) .
\end{align}

Combining \eqref{discrete-commutator1x} with \eqref{discrete-commutator2x} then gives
a two-field PCD approximation
\begin{align} \label{twofield-pcd}
B\boldsymbol{F}^{-1}B^T \approx M_S :=
\left[
\begin{array}{@{}cc@{}}
Q_1 & 0 \\  0 & Q_0 \end{array}
\right]
\left[
\begin{array}{@{}cc@{}}
F_1^{-1} & 0 \\  0 & F_0^{-1}  \end{array}
\right]
 B  \boldsymbol{M}^{-1} B^T .
\end{align}

Two features of the PCD approximation \eqref{twofield-pcd} are worth noting.
The first point is that the coupling  between the pressure components is 
represented by  the $2\times 2$ block matrix  $B \boldsymbol{M}^{-1} B^T$ rather than
by the pressure mass matrix $\pressuremass$ (the coupling term $R$ is 
absent).  The second key point is that the matrices $M_S$ and 
$B  \boldsymbol{F}^{-1} B^T$
have the same nullspace, independent of the nature of underlying flow problem that is 
being solved. 

 \begin{figure}[!th]
       \begin{center}
	\includegraphics[width=0.35\linewidth]{step_p2p1_coarse.eps}		
	\includegraphics[width=0.35\linewidth]{step_p2p1_fine.eps}	 
	\end{center}%\vspace{-4mm}
\caption{Absolute residual reduction  for test problem~3  
when computing $\P_2$--$\P_1^*$   solutions
using preconditioners  $\MM_1$ ({\textcolor{magenta}x}) 
or $\MM_2$ ({\textcolor{blue}o})  on  two nested meshes.}
\label{fig.testproblem1_p2p1}
\end{figure}

 
The PCD approximation  in \eqref{twofield-pcd}  is  imperfect in practice.
To illustrate this, representative convergence histories  that arise in solving the inflow-outflow problem
using $\P_2$--$\P_1^*$  approximation (shown in Fig.~\ref{fig.testproblem1})
 are presented in Fig.~\ref{fig.testproblem1_p2p1}.  Taking the solution from the
 previous Picard iteration as the initial guess, we note  that the initial residual 
 norm of the target matrix system \eqref{oseen-system} is close to $10^{-4}$
 independent of the spatial discretisation.
 Convergence plots are presented for two preconditioning strategies, namely
\begin{align} \label{nse-precon-ref}
\MM_1 =
\left[
\begin{array}{@{}cc@{}}
\boldsymbol{F} & B^T \\ 0 & - {1\over \nu} \pressuremass
\end{array}
\right], \quad
\MM_2 =
\left[
\begin{array}{@{}cc@{}}
\boldsymbol{F} & B^T \\ 0 & - M_S
\end{array}
\right],
\end{align}
with $M_S$ defined in \eqref{twofield-pcd}. 
The first strategy is the  block triangular extension of  the 
Stokes preconditioning strategy discussed in section~2. \backref
We note that the resulting convergence is very slow---around 50 iterations are 
required to reduce the residual norm by an order of magnitude---but is independent of the 
discretisation level.  In contrast we see that the PCD preconditioning  strategy 
has two distinctive phases of convergence behaviour. An initial phase of relatively 
fast convergence is followed by a secondary phase where GMRES 
stagnates. We  hypothesise that this stagnation is a consequence of the 
ill-conditioning of the  eigenvectors of the matrix operator $M_S$. A
notable feature is that the onset of the stagnation is delayed when
solving the same problem on  a finer grid. The two-phase convergence 
behaviour is ubiquitous---the same pattern  is seen  using 
rectangular elements and the stagnation  does not go away when 
the viscosity parameter is increased from $1/50$ to $1/5$. An alternative strategy is
clearly needed!

One way of designing a more robust PCD preconditioning strategy for a two-field pressure 
approximation   is  to exploit the fast convergence of the PCD
approximation for the unaugmented  Taylor-Hood approximation.
The starting point for  such a strategy  is to rewrite the system \eqref{oseen-system} in the form
\begin{align} \label{full-system}
\left[
\begin{array}{@{}ccc@{}}
\boldsymbol{F}  & B_1^T & B_0^T\\ B_1 & 0 & 0 \\ B_0 & 0 & 0
\end{array}
\right]
\left[
\begin{array}{@{}l@{}}
\bfu \\ \bfp_1 \\  \bfp_0
\end{array}
\right]
=
\left[
\begin{array}{@{}l@{}}
\bff \\  \bf{0} \\ \bf{0}%\bfg_0
\end{array}
\right] .
\end{align}
The proposed solution algorithm is then a two-stage process.

\begin{itemize} 
\item{\bfseries Input:}  residual reduction tolerance $\eta$
%
\item[Step I] 

Generate  a PCD  solution  to  the {\it reduced system} 
\begin{align} \label{reduced-system}
\left[
\begin{array}{@{}cc@{}}
\boldsymbol{F}  & B_1^T \\ B_1 & 0
\end{array}
\right]
\left[
\begin{array}{@{}c@{}}
\bfu_1 \\ \bfq_1
\end{array}
\right]
=
\left[
\begin{array}{@{}c@{}}
\bff \\  \bf{0}
\end{array}
\right]
\end{align}
using  the Schur complement approximation \eqref{discrete-commutator1}, stopping
the GMRES iteration when the residual is reduced by a factor of $10\eta$.

\item[Step II] 
Generate  a  solution  to the target system 
\eqref{full-system} with residual tolerance  $\eta$ using 
preconditioning strategy  $\MM_1$ in \eqref{nse-precon-ref} with the refined  initial guess 
$\lbrack \bfu_1^*  , \bfq_1^*, {\bf 0} \rbrack $.

\item
{\bfseries Output:}  refined solution $\lbrack \bfu^*  , \bfp_1^*,  \bfp_0^* \rbrack $
\end{itemize}

\begin{figure}[!ht]
       \begin{center}
	\includegraphics[width=0.35\linewidth]{step_q2q1_coarse.eps}		
	\includegraphics[width=0.35\linewidth]{step_q2q1_fine.eps}	 
	\end{center}%\vspace{-4mm}
\caption{Absolute residual reduction  for test problem~1  
when computing  $\Q_2$--$\Q_1^*$   solutions using 
using preconditioners  $\MM_1$ ({\textcolor{magenta}x}) 
or refined PCD  ({\textcolor{black}o})  on  two nested meshes.}
\label{fig.testproblem1_q2q1}
\end{figure}

\begin{figure}[!ht]
       \begin{center}
	\includegraphics[width=0.35\linewidth]{cavity_p2p1_coarse.eps}		
	\includegraphics[width=0.35\linewidth]{cavity_p2p1_fine.eps}	 
	\end{center}%\vspace{-4mm}
\caption{Absolute residual reduction  for test problem~2  
when computing  $\P_2$--$\P_1^*$   solutions using 
using preconditioners  $\MM_1$ ({\textcolor{magenta}x}) 
or refined PCD  ({\textcolor{black}o})  on  two nested meshes.}
\label{fig.testproblem2_p2p1}
\end{figure}

Sample  results  generated using this strategy with  tolerance 
$\eta$ set to $10^{-4}$ are presented  in Fig.~\ref{fig.testproblem1_q2q1}.   
Sample results for the second test problem
are presented in Fig.~\ref{fig.testproblem2_p2p1}.   
The results in Fig.~\ref{fig.testproblem1_q2q1} and Fig.~\ref{fig.testproblem2_p2p1}
are representative of the two-stage convergence profiles that are generated when solving 
these test problems at other Reynolds numbers.  Our experience is that the level of
residual reduction  is perfectly robust with regards to
the spatial discretisation---typically giving  smaller iteration counts when the mesh resolution 
is increased (a known feature of PCD preconditioning). The convergence rates of both
stages of the algorithm deteriorate slowly when the Reynolds number is increased. 
Our strategy for terminating the first stage of the iteration is  motivated by the following result.
\begin{proposition}\label{pr.residualbound}
The residual error  $\| \bfz^* \|$
associated with the intermediate  solution $\lbrack \bfu^*_1  , \bfq_1^*,  \bf{0} \rbrack $
to the discrete system \eqref{full-system} satisfies the bound
\begin{align} \label{resbound}
\| \bfz^* \|^2  &\leq 100 \eta^2 \,  \| \bff \|^2  + \| B_0 \bfu_1^* \|^2  ,
\end{align}
where  $\| \bff \|$ is the initial residual error associated with  a zero initial vector.

\begin{proof}
The vector  $ \bfz^*$  associated with the intermediate solution is the 
three-component vector
\begin{align} \label{res-system}
\left[
\begin{array}{@{}l@{}}
\bfr^* \\ \bfr_1^* \\ \bfr_0^*
\end{array}
\right]
=
\left[
\begin{array}{@{}l@{}}
\bff \\  \bf{0} \\ \bf{0}
\end{array}
\right] 
-
\left[
\begin{array}{@{}ccc@{}}
\boldsymbol{F}  & B_1^T & B_0^T\\ B_1 & 0 & 0 \\ B_0 & 0 & 0
\end{array}
\right]
\left[
\begin{array}{@{}l@{}}
\bfu_1^* \\ \bfq_1^* \\  \bf{0}
\end{array}
\right] .
\end{align}
The stopping test for solving the reduced system ensures that 
\begin{align}
 \| \bfr^* \|^2   +  \| \bfr_1^* \|^2  \leq 100 \eta^2 \, \| \bff \|^2.
 \end{align}
 Thus we have 
 \begin{align}
\| \bfz^* \|^2 =   \| \bfr^* \|^2   +  \| \bfr_1^* \|^2  
  +   \| \bfr_0^* \|^2  \leq 100 \eta^2  \, \| \bff \|^2 +  \| B_0 \bfu_1^* \|^2.  
 \end{align}
 \end{proof}
\end{proposition}
The bound  \eqref{resbound}  has two terms on the right-hand side. While the first term 
can be controlled by reducing $\eta$,  the second term  measures the  
local incompressibility of the intermediate  Taylor--Hood solution
$ \| B_0 \bfu_1^* \|^2 =  \sum_j \big( \int_{T_j} \! \nabla \cdot {\vec{u}_h}^{\,*} \big)^2$,
where ${\vec{u}_h}^{\,*}$ is the expansion of the coefficient vector $\bfu_1^* $ in the basis of the 
velocity approximation space. Setting  $\eta = 10^{-4}$ we see that the second term saturates 
the residual error and the residual error jumps up  when the switch is made from the first to the  
second step of the algorithm.  This ``transition'' jump in the residual norm is
clearly evident in the convergence plots. The convergence in the second step
is rapid initially but eventually mirrors the rate observed for  $\MM_1$ approximation
with a standard starting guess.

\subsection{Least square commutator approximation for Oseen flow} \label{sec:lsc}
A second way of approximating the key matrix
$B\boldsymbol{F}^{-1}B^T$ in the case that $B^T$ is generated by a two-field
pressure approximation is  given by the  {least-squares commutator} (LSC) preconditioner
 \begin{align} \label{lsc-definition}
B \boldsymbol{F}^{-1} B^T \approx   M_S =
(B  \boldsymbol{H}^{-1} B^T )\, 
(B \boldsymbol{M}^{-1}  \boldsymbol{F} \boldsymbol{H}^{-1} B^T )^{-1}  
\, (B  \boldsymbol{M}^{-1} B^T) .
\end{align}

The attractive feature of LSC  is that the construction of $M_S$ is 
completely algebraic. The only technical issue is the need to make adjustments   
 on rows and columns associated with tangential velocity degree of freedom adjacent to
  inflow and fixed wall boundaries. These adjustments are associated  with
 a diagonal   scaling matrix $\boldsymbol{D}$ so that
$\boldsymbol{H}= \boldsymbol{D}^{-1/2} \boldsymbol{M} \boldsymbol{D}^{-1/2}$.
Full details can be found in \cite[pp.\,376--379]{elman14}). \bookref 


The LSC approximation \eqref{lsc-definition} is also far from perfect  when 
using triangular elements.\footnote{The strategy is designed for
tensor-product approximation spaces. It gives good results in the case of $\Q_2$--$\Q_1^*$.} 
To illustrate this, representative convergence histories  that arise in 
solving the inflow-outflow problem using $\P_2$--$\P_1^*$  approximation
 are presented in Fig.~\ref{fig.testproblem1_lsc}.   
 Convergence plots are presented for two preconditioning strategies, namely
 the refined PCD from the previous section  and 
\begin{align} \label{nse-precon-lsc}
\MM_3 =
\left[
\begin{array}{@{}cc@{}}
\boldsymbol{F} & B^T \\ 0 & - M_S
\end{array}
\right],
\end{align}
with $M_S$ defined in \eqref{lsc-definition}. 

\begin{figure}[!ht]
       \begin{center}
	\includegraphics[width=0.35\linewidth]{step_p2p1_pcd.eps}		
	\includegraphics[width=0.35\linewidth]{step_p2p1_lsc.eps}	 
	\end{center}%\vspace{-4mm}
\caption{Absolute residual reduction  for test problem~1 
when computing  $\P_2$--$\P_1^*$   solutions using 
using  refined PCD  (left) or preconditioner  $\MM_3$ (right)
on  three nested meshes.}
\label{fig.testproblem1_lsc}
\end{figure}

The PCD convergence histories are comparable with those 
generated using square elements (cf. Fig.~\ref{fig.testproblem1_q2q1}).
The associated cpu times for the solution on the intermediate grid  
with  $2\times 11264$ elements  were 13 seconds for the first step (28 iterations) and
26 seconds for the second step (53 iterations). The LSC preconditioning
strategy is not robust---the convergence rate deteriorates with increasing 
grid refinement. The cpu time for generating  the LSC solution on the
intermediate grid  was over 100 seconds  (61 iterations).


\section{Conclusions} \label{sec:conclusions}
Two-level pressure approximation for incompressible flow problems offer the
prospect of accurate approximation with minimal computational overhead.
Derived quantities of practical importance such as the mean wall shear stress 
are likely to be  computed much more precisely if incompressibility is enforced 
locally.\footnote{See
 \url{https://personalpages.manchester.ac.uk/staff/david.silvester/lecture1.18.mp4}
for a comparison of alternative strategies for computing the average shear stress   
for flow over a step at Reynolds number 200.} 
However, the augmented pressure space causes some challenges for the linear algebra, 
 because constant functions can be expressed using either the usual (continuous) 
Taylor--Hood pressure space, or the augmented piecewise constant pressure space.
Specifically, the pressure mass matrix becomes singular, and care should be taken to 
carefully construct and apply preconditioners that involve this matrix. 
Care should also be exercised when approximating the discrete inf--sup constant,  
and we find that naive approaches are not always reliable. 
On the other hand, the approximation implemented in EST-MINRES is robust.  
Our computational experimentation indicates that our two-stage  PCD strategy could be
the best way of iteratively solving  two-level pressure discrete linear algebra systems 
in the sense of  algorithmic reliability  and computational efficiency.


\bibliographystyle{siam}
\bibliography{twolevel}

\end{document} 
