
% Template for Elsevier CRC journal article
% version 1.2 dated 09 May 2011

% This file (c) 2009-2011 Elsevier Ltd.  Modifications may be freely made,
% provided the edited file is saved under a different name

% This file contains modifications for Procedia Computer Science

% Changes since version 1.1
% - added "procedia" option compliant with ecrc.sty version 1.2a
%   (makes the layout approximately the same as the Word CRC template)
% - added example for generating copyright line in abstract

%-----------------------------------------------------------------------------------

%% This template uses the elsarticle.cls document class and the extension package ecrc.sty
%% For full documentation on usage of elsarticle.cls, consult the documentation "elsdoc.pdf"
%% Further resources available at http://www.elsevier.com/latex

%-----------------------------------------------------------------------------------

%%%%%%%%%%%%%%%%%%%%%%%%%%%%%%%%%%%%%%%%%%%%%%%%%%%%%%%%%%%%%%
%%%%%%%%%%%%%%%%%%%%%%%%%%%%%%%%%%%%%%%%%%%%%%%%%%%%%%%%%%%%%%
%%                                                          %%
%% Important note on usage                                  %%
%% -----------------------                                  %%
%% This file should normally be compiled with PDFLaTeX      %%
%% Using standard LaTeX should work but may produce clashes %%
%%                                                          %%
%%%%%%%%%%%%%%%%%%%%%%%%%%%%%%%%%%%%%%%%%%%%%%%%%%%%%%%%%%%%%%
%%%%%%%%%%%%%%%%%%%%%%%%%%%%%%%%%%%%%%%%%%%%%%%%%%%%%%%%%%%%%%

%% The '3p' and 'times' class options of elsarticle are used for Elsevier CRC
%% The 'procedia' option causes ecrc to approximate to the Word template
\documentclass[3p,times,procedia]{elsarticle}

%% The `ecrc' package must be called to make the CRC functionality available
\usepackage{ecrc}

%% The ecrc package defines commands needed for running heads and logos.
%% For running heads, you can set the journal name, the volume, the starting page and the authors

%% set the volume if you know. Otherwise `00'
\volume{00}

%% set the starting page if not 1
\firstpage{1}

%% Give the name of the journal
\journalname{Procedia Computer Science}

%% Give the author list to appear in the running head
%% Example \runauth{C.V. Radhakrishnan et al.}
\runauth{}

%% The choice of journal logo is determined by the \jid and \jnltitlelogo commands.
%% A user-supplied logo with the name <\jid>logo.pdf will be inserted if present.
%% e.g. if \jid{yspmi} the system will look for a file yspmilogo.pdf
%% Otherwise the content of \jnltitlelogo will be set between horizontal lines as a default logo

%% Give the abbreviation of the Journal.
\jid{procs}

%% Give a short journal name for the dummy logo (if needed)
\jnltitlelogo{Procedia Computer Science}

%% Provide the copyright line to appear in the abstract
%% Usage:
%   \CopyrightLine[<text-before-year>]{<year>}{<restt-of-the-copyright-text>}
%   \CopyrightLine[Crown copyright]{2011}{Published by Elsevier Ltd.}
%   \CopyrightLine{2011}{Elsevier Ltd. All rights reserved}
\CopyrightLine{2011}{Published by Elsevier Ltd.}

%% Hereafter the template follows `elsarticle'.
%% For more details see the existing template files elsarticle-template-harv.tex and elsarticle-template-num.tex.

%% Elsevier CRC generally uses a numbered reference style
%% For this, the conventions of elsarticle-template-num.tex should be followed (included below)
%% If using BibTeX, use the style file elsarticle-num.bst

%% End of ecrc-specific commands
%%%%%%%%%%%%%%%%%%%%%%%%%%%%%%%%%%%%%%%%%%%%%%%%%%%%%%%%%%%%%%%%%%%%%%%%%%

\usepackage{amscd,amsmath,amsthm,amssymb,amsfonts,bbm}
\definecolor{purple}{rgb}{1, 0, 1}

\newcommand{\ie}{\emph{i.e.,}\xspace}
\newcommand{\eg}{\emph{e.g.,}\xspace}
\newcommand{\abr}{\emph{abbr.}\xspace}
\newcommand{\ea}{\emph{et al.}\xspace}
\newcommand{\gensync}{\emph{GenSync}\xspace}
\newcommand{\colosseum}{\emph{Colosseum}\xspace}
\newcommand{\srep}{\emph{SREP}\xspace} % Set Reconciliation Enhances
\newcommand{\srepsim}{\emph{SREPSim}\xspace}
% Propagation
\newcommand{\esrep}{\emph{E-SREP}\xspace}
\newcommand{\epsrep}{\emph{EP-SREP}\xspace}
\newcommand{\mesrep}{\emph{ME-SREP}\xspace}
\newcommand{\mempoolsync}{\emph{MempoolSync}}

\newcommand{\fref}[1]{Fig.~\ref{#1}}
\newcommand{\tref}[1]{Table~\ref{#1}}
\newcommand{\aref}[1]{Algorithm~\ref{#1}}
\newcommand{\procref}[1]{Procedure~\ref{#1}}
\newcommand{\sref}[1]{Section~\ref{#1}}
\newcommand{\lineref}[1]{line~\ref{#1}}
\newcommand{\appref}[1]{Appendix~\ref{#1}}

% Change \eqref
\LetLtxMacro{\originaleqref}{\eqref}
\renewcommand{\eqref}{Eq.~\originaleqref}

% Theorems and corollaries
\newcounter{theoremcount}
\setcounter{theoremcount}{0}
\DeclareRobustCommand{\theorem}[1]{%
  \refstepcounter{theoremcount}%
  \noindent\textit{\textbf{Theorem \thetheoremcount\label{theorem:#1}: }}%
}
\DeclareRobustCommand{\theoremref}[1]{Theorem~\ref{theorem:#1}}

\DeclareRobustCommand{\proof}{\emph{Proof:}\xspace}
\DeclareRobustCommand{\qqed}{\hfill$\blacksquare$}

\newcounter{corollcount}
\setcounter{corollcount}{0}
\DeclareRobustCommand{\coroll}[1]{%
  \refstepcounter{corollcount}%
  \noindent\textit{\textbf{Corollary \thecorollcount\label{coroll:#1}: }}%
}
\DeclareRobustCommand{\corollref}[1]{Corollary~\ref{coroll:#1}}

\newcounter{lemmacount}
\setcounter{lemmacount}{0}
\DeclareRobustCommand{\lemma}[1]{%
  \refstepcounter{lemmacount}%
  \noindent\textit{\textbf{Lemma \thelemmacount\label{lemma:#1}: }}%
}
\DeclareRobustCommand{\lemmaref}[1]{Lemma~\ref{lemma:#1}}

\newcounter{definitioncount}
\setcounter{definitioncount}{0}
\DeclareRobustCommand{\definition}[1]{%
  \refstepcounter{definitioncount}%
  \noindent\textit{\textbf{Definition \thedefinitioncount\label{definition:#1}: }}%
}
\DeclareRobustCommand{\defref}[1]{Definition~\ref{definition:#1}}

%notes of different authors
\newif\ifnotes
\notestrue
\notesfalse

\newif\ifdiff
\difftrue
\difffalse

\newcommand{\anote}[1]{\ifnotes $\ll$\textsf{\textcolor{purple}{Ari: {#1}}}$\gg$ \fi}
\newcommand{\nnote}[1]{\ifnotes $\ll$\textsf{\textcolor{orange}{Novak: {#1}}}$\gg$ \fi}
\newcommand{\diff}[1]{\ifdiff\textcolor{orange}{#1}\else#1\fi}

%%% Local Variables:
%%% mode: latex
%%% TeX-master: "main"
%%% End:

% \usepackage[utf8]{inputenc}
\usepackage[algonl,boxed,norelsize,lined]{algorithm2e}
\usepackage{mathtools}
\usepackage{arydshln}
% \usepackage{fullpage}
\usepackage{comment}
\usepackage[all]{xy}
\usepackage{datetime}
\usepackage{subfig}
\usepackage{color,mathtools}
% \usepackage[english]{babel}
\usepackage[T1]{fontenc}
\usepackage{hyperref}
\usepackage{graphicx}
% \usepackage{appendix}
% \usepackage{enumitem}
% \usepackage{theoremref,url}
\usepackage[new]{old-arrows}
\hypersetup{
	colorlinks = true,
	linkcolor = blue,
	anchorcolor = blue,
	citecolor = teal,
	filecolor = blue,
	urlcolor = black
}
% \setlist[itemize]{noitemsep, topsep=0pt}
\usepackage{tocloft}
% \usepackage{enumitem}% http://ctan.org/pkg/enumitem

\usepackage{booktabs}
\usepackage{paralist}
\setdefaultleftmargin{0pt}{0pt}{0pt}{0pt}{0pt}{0pt}

\usepackage{wrapfig}


\usepackage{amsmath}
\usepackage{amssymb}
\usepackage{amsthm}
\usepackage{verbatim}
\usepackage{graphicx}
\usepackage[disable]{todonotes}
\usepackage{floatrow}

\usepackage{tikz,tikz-cd}
\usetikzlibrary{decorations.pathreplacing}
\usetikzlibrary{shapes.multipart}
\usetikzlibrary{calc,cd}

\usepackage{float}

\renewcommand{\paragraph}[1]{\smallskip\noindent\textbf{#1.}}

%%%% Standard Packages
\usepackage{graphicx}
\usepackage{amsmath}
\usepackage{xcolor}
\usepackage{hyperref}
\usepackage{nicefrac}
\usepackage{mathtools}
\usepackage{caption}
% \usepackage{subcaption}
% \usepackage{subfig}
\usepackage{wrapfig}
% \usepackage{mol2chemfig}
\usepackage{chemfig}
\DeclarePairedDelimiter\abs{\lvert}{\rvert}
\DeclarePairedDelimiter\ceil{\lceil}{\rceil}
\DeclarePairedDelimiter\floor{\lfloor}{\rfloor}
\usepackage{paralist}
\usepackage{appendix}


%% The amssymb package provides various useful mathematical symbols
\usepackage{amssymb}
%% The amsthm package provides extended theorem environments
%% \usepackage{amsthm}

%% The lineno packages adds line numbers. Start line numbering with
%% \begin{linenumbers}, end it with \end{linenumbers}. Or switch it on
%% for the whole article with \linenumbers after \end{frontmatter}.
%% \usepackage{lineno}

%% natbib.sty is loaded by default. However, natbib options can be
%% provided with \biboptions{...} command. Following options are
%% valid:

%%   round  -  round parentheses are used (default)
%%   square -  square brackets are used   [option]
%%   curly  -  curly braces are used      {option}
%%   angle  -  angle brackets are used    <option>
%%   semicolon  -  multiple citations separated by semi-colon
%%   colon  - same as semicolon, an earlier confusion
%%   comma  -  separated by comma
%%   numbers-  selects numerical citations
%%   super  -  numerical citations as superscripts
%%   sort   -  sorts multiple citations according to order in ref. list
%%   sort&compress   -  like sort, but also compresses numerical citations
%%   compress - compresses without sorting
%%
%% \biboptions{comma,round}

% \biboptions{}

% if you have landscape tables
\usepackage[figuresright]{rotating}

% put your own definitions here:
%   \newcommand{\cZ}{\cal{Z}}
%   \newtheorem{def}{Definition}[section]
%   ...

% add words to TeX's hyphenation exception list
%\hyphenation{author another created financial paper re-commend-ed Post-Script}

% declarations for front matter

\begin{document}

\begin{frontmatter}

%% Title, authors and addresses

%% use the tnoteref command within \title for footnotes;
%% use the tnotetext command for the associated footnote;
%% use the fnref command within \author or \address for footnotes;
%% use the fntext command for the associated footnote;
%% use the corref command within \author for corresponding author footnotes;
%% use the cortext command for the associated footnote;
%% use the ead command for the email address,
%% and the form \ead[url] for the home page:
%%
%% \title{Title\tnoteref{label1}}
%% \tnotetext[label1]{}
%% \author{Name\corref{cor1}\fnref{label2}}
%% \ead{email address}
%% \ead[url]{home page}
%% \fntext[label2]{}
%% \cortext[cor1]{}
%% \address{Address\fnref{label3}}
%% \fntext[label3]{}

\dochead{XIII Latin American Algorithms, Graphs, and Optimization Symposium}
%% Use \dochead if there is an article header, e.g. \dochead{Short communication}
%% \dochead can also be used to include a conference title, if directed by the editors
%% e.g. \dochead{17th International Conference on Dynamical Processes in Excited States of Solids}

\title{Combinatorial Parameterized Algorithms for Chemical Descriptors based on Molecular Graph Sparsity}

%% use optional labels to link authors explicitly to addresses:
 \author[hkust]{Giovanna Kobus Conrado}
 \author[oxford]{Amir Kafshdar Goharshady}
 \author[mpi]{Harshit Jitendra Motwani}
 \author[hkust]{Sergei Novozhilov}
 \address[hkust]{Hong Kong University of Science and Technology, Clear Water Bay, Hong Kong\\\texttt{\{gkc, snovozhilov\}@connect.ust.hk}}
 \address[oxford]{University of Oxford, Oxford, United Kingdom \\\texttt{amir.goharshady@cs.ox.ac.uk}}
 \address[mpi]{Max Planck Institute for Software Systems, Kaiserslautern, Germany\\\texttt{hmotwani@mpi-sws.org}}

\begin{abstract}
We present efficient combinatorial parameterized algorithms for several classical graph-based counting problems in computational chemistry, including (i)~Kekulé structures, (ii)~the Hosoya index, (iii)~the Merrifield–Simmons index, and (iv)~Graph entropy based on matchings and independent sets. All these problems were known to be $\# P$-complete. Building on the intuition that molecular graphs are often sparse and tree-like, we provide fixed-parameter tractable  (FPT) algorithms using treewidth as our parameter. We also provide extensive experimental results over the entire PubChem database of chemical compounds, containing more than 113 million real-world molecules. In our experiments, we observe that the molecules are indeed sparse and tree-like, with more than $99.9\%$ of them having a treewidth of at most $5.$ This justifies our choice of parameter. Our experiments also illustrate considerable improvements over the previous approaches. Based on these results, we argue that parameterized algorithms, especially based on treewidth, should be adopted as the default approach for problems in computational chemistry that are defined over molecular graphs. 
\end{abstract}

\begin{keyword}
Computational Chemistry \sep Parameterized Algorithms \sep Topological Indices

\end{keyword}

\end{frontmatter}

%%
%% Start line numbering here if you want
%%
% \linenumbers


\section{Introduction}
\label{sec:introduction}
% \begin{itemize}
%     % Diffusion of FL
%     \item {\st{Diffusion of FL}}
%     % Security threats to FL
%     \item {\st{Security threats to FL with particular focus on model poisoning}}
%     % Limitations of existing countermeasures
%     \item {\st{Current countermeasures (e.g., KRUM) and their limitations}}
%     % Proposed method and its advantages
%     \item {\st{Intuitive description of the proposed method and its difference (i.e., advantages) w.r.t. state of the art}}
%     % Main contributions
%     \item {\st{Summary of the main contributions of this work}}
%     % Paper's structure and organization
%     \item {\st{Paper's structure and organization}}
% \end{itemize}

% Diffusion of FL
Recently, {\em federated learning} (FL) has emerged as the leading paradigm for training distributed, large-scale, and privacy-preserving machine learning (ML) systems~\cite{mcmahan2017googleai,mcmahan2017aistats}. 
The core idea of FL is to allow multiple edge clients to collaboratively train a shared, global model without disclosing their local private training data.
%Specifically, an FL system consists of a central server and many edge clients; 
A typical FL round involves the following steps: {\em(i)} the server randomly picks some clients and sends them the current, global model; {\em(ii)} each selected client locally trains its model with its own private data; then, it sends the resulting local model to the server;\footnote{Whenever we refer to global/local model, we mean global/local model {\em parameters}.} {\em(iii)} the server updates the global model by computing an \emph{aggregation function}, usually the average (FedAvg), on the local models received from clients.
% \begin{enumerate}
%     \item[{\em(i)}] the server sends the current, global model to the clients and appoints some of them for training;
%     \item[{\em(ii)}] each selected client locally trains its copy of the global model with its own private data; then, it sends the resulting local model back to the server;\footnote{Whenever we refer to global/local model, we mean global/local model {\em parameters}.}
%     \item[{\em(iii)}] the server updates the global model by computing an \emph{aggregation function} on the local models received from clients (by default, the average, also referred to as FedAvg~\cite{mcmahan2017aistats}).
% \end{enumerate}
This process goes on until the global model converges. %(e.g., after a certain number of rounds or other similar stopping criteria).
%\\
% The advantages of FL over the traditional, centralized learning paradigm are undoubtedly clear in terms of flexibility/scalability (clients can join/disconnect from the FL network dynamically), network communications (only model weights\footnote{We will use \textit{parameters} and \textit{weights} interchangeably.} are exchanged between clients and server), and privacy (each client's private training data is kept local at the client's end and not uploaded to the server).
\\
% Security threats to FL
%However, the growing adoption of FL also raises security concerns~\cite{costa2022covert}, particularly about its confidentiality, integrity, and availability.
Although its advantages over standard ML, FL also raises security concerns~\cite{costa2022covert}. %, particularly about its confidentiality, integrity, and availability~\cite{costa2022covert}.
% OLD, LONG VERSION
% Indeed, some work deals with privacy leakage that may expose the local data of some clients~\cite{melis2019sp}. 
% A large body of work, instead, investigates attacks that usually aim to detriment the predictive accuracy of the learned global model. For instance, \emph{data poisoning} attacks achieve this goal by letting an adversary pollute the training set of some corrupt FL clients with maliciously crafted examples~\cite{jagielski2018sp}.
% Similarly, in \emph{model poisoning} the attacker attempts to tweak the global model weights~\cite{bhagoji2019pmlr} by directly perturbing the local model's weights of some infected FL clients before these are sent to the central server for aggregation, usually via so-called Byzantine attacks. 
% It turns out that Byzantine model poisoning attacks severely impact standard FedAvg; therefore, more robust aggregation functions must be designed to make FL systems secure.
Here, we focus on \emph{untargeted model poisoning} attacks~\cite{bhagoji2019pmlr}, where an adversary attempts to tweak the global model weights %\footnote{We will use the terms \textit{parameters} and \textit{weights} interchangeably.} 
by directly perturbing the local model's parameters of some infected clients before these are sent to the central server for aggregation.
In doing so, the adversary aims to jeopardize the global model \textit{indiscriminately} at inference time.
Such model poisoning attacks severely impact standard FedAvg; therefore, more robust aggregation functions must be designed to secure FL systems.
\\
% In this paper, we focus on designing a novel robust aggregation scheme at the server's end to contrast the effect of Byzantine model poisoning attacks.
%
% Current countermeasures and their limitations
%Several countermeasures have been proposed in the literature to combat model poisoning attacks on FL systems.
% Some methods use simple statistics more robust than plain average to smooth the impact of malicious updates (e.g., Trimmed Mean and FedMedian~\cite{yin2018icml}). 
% Other defenses implement outlier detection techniques to discard malicious updates from the aggregation performed at the server's end. Those are either based on heuristics (e.g., Krum/Multi-Krum~\cite{blanchard2017nips} and Bulyan~\cite{mhamdi2018pmlr}) or data-driven approaches (e.g., K-means clustering~\cite{shen2016acm} or DnC via spectral analysis~\cite{shejwalkar2021ndss}). 
% Finally, some strategies rely on a centralized ``source of trust'' to spot potential malicious updates (e.g., FLTrust~\cite{cao2020fltrust}).
% Several countermeasures have been proposed in the literature to combat model poisoning attacks on FL systems, i.e., to discard possible malicious local updates from the aggregation performed at the server's end. 
% These techniques range from simple statistics more robust than plain average (e.g., Trimmed Mean and FedMedian~\cite{yin2018icml}) to outlier detection heuristics (e.g., Krum/Multi-Krum~\cite{blanchard2017nips} and Bulyan~\cite{mhamdi2018pmlr}) or data-driven approaches (e.g., spectral analysis via K-means clustering~\cite{shen2016acm} or spectral analysis), or methods based on ``source of trust'' (e.g., FLTrust~\cite{cao2020fltrust}).
% OLD, LONG VERSION
%Several countermeasures have been proposed in the literature to combat Byzantine model poisoning attacks on FL systems.
% Descriptive statistics
% For example, Trimmed Mean and FedMedian aggregate local model updates using more robust statistics than standard average~\cite{yin2018icml}.
%
% % Heuristics for outlier detection
% Many existing Byzantine-resilient strategies implement some outlier detection heuristics to discard the model updates sent by potentially malicious clients from the input of the aggregation function.
% One of the most popular heuristics is Krum~\cite{blanchard2017nips}.
% This strategy tries to mitigate the impact of Byzantine attacks by selecting as a global model the local model with the smallest sum of Euclidean distances to {\em all} the other local models.
% Although powerful, Krum requires the server to know (or, at least, estimate) the number of malicious FL clients upfront, which is generally impossible in a realistic attack scenario. %
% Moreover, Krum may become ineffective for complex, high-dimensional model parameter spaces due to the curse of dimensionality.
% Bulyan~\cite{mhamdi2018pmlr} tries to overcome this issue by combining Krum with a variant of Trimmed Mean.
% % Data-driven outlier detection
% Other strategies use data-driven outlier detection techniques -- e.g., via K-means clustering~\cite{shen2016acm} -- to spot potential malicious local model updates. 
% %For instance, Shen et al. propose to cluster local model updates with K-means and thus identify outliers.
%
% % Other techniques
% As far as the server is concerned, any local model received can be from a potential malicious client. 
% FLTrust~\cite{cao2020fltrust} assumes the server acts as a client, i.e., trains a local model on an additional {\em trustworthy} dataset at the server's end and compares it against all the local models from other clients. 
% This way, the server can rely on some ``source of trust'' when discarding potentially malicious clients.
%\\
% Limitations of existing Byzantine-resilient strategies
Unfortunately, existing defense mechanisms either rely on simple heuristics (e.g., Trimmed Mean and FedMedian by~\cite{yin2018icml}) or need strong and unrealistic assumptions to work effectively (e.g., foreknowledge or estimation of the number of malicious clients in the FL system, as for Krum/Multi-Krum~\cite{blanchard2017nips} and Bulyan~\cite{mhamdi2018pmlr}, which, however, cannot exceed a fixed threshold).
Furthermore, outlier detection methods using K-means clustering~\cite{shen2016acm} or spectral analysis like DnC~\cite{shejwalkar2021ndss} do not directly consider the temporal evolution of local model updates received.
Finally, strategies like FLTrust~\cite{cao2020fltrust} require the server to collect its own dataset and act as a proper client, thereby altering the standard FL protocol.
\\
% OLD, LONG VERSION
% Overall, existing Byzantine-resilient strategies are either simple heuristics (e.g., FedMedian) or, if they are more complex, they rely on strong and unrealistic assumptions to work effectively (e.g., knowing the number of malicious clients in the FL system in advance, as for Krum and alike).
% Furthermore, data-driven outlier detection methods do not consider the temporary evolution of local model updates received (e.g., K-means clustering). 
% Finally, strategies like FLTrust requires the server to collect its own dataset and act as a proper client, thereby altering the standard FL protocol.
%
% Description of the proposed method
This work introduces a novel pre-aggregation \textit{filter} robust to untargeted model poisoning attacks. Notably, this filter $(i)$ operates without requiring prior knowledge or constraints on the number of malicious clients and $(ii)$ inherently integrates temporal dependencies. 
The FL server can employ this filter as a preprocessing step before applying \textit{any} aggregation function, be it standard like FedAvg or robust like Krum or Bulyan.
Specifically, we formulate the problem of identifying corrupted updates as a multidimensional (i.e., matrix-valued) time series anomaly detection task. 
The key idea is that legitimate local updates, resulting from well-calibrated iterative procedures like stochastic gradient descent (SGD) with an appropriate learning rate, show \textit{higher predictability} compared to malicious updates. This hypothesis stems from the fact that the sequence of gradients (thus, model parameters) observed during legitimate training exhibit regular patterns, as validated in Section~\ref{subsec:intuition}. %until convergence. 
%This regularity may be more pronounced for smooth convex loss functions, but it can still be captured within an appropriate time window, even for more complex and convoluted loss surfaces. 
%We provide evidence of this claim in Appendix~B, where we show that the average mutual information (i.e., ``predictability''), calculated over pairs of legitimate model updates sent at different FL rounds, is significantly higher than the corresponding computation for a malicious client.
\\
Inspired by the matrix autoregressive (MAR) framework for multidimensional time series forecasting~\cite{chen2021je}, we propose the FLANDERS ({\em \textbf{F}ederated \textbf{L}earning meets \textbf{AN}omaly \textbf{DE}tection for a \textbf{R}obust and \textbf{S}ecure}) filter.
The main advantages of FLANDERS over existing strategies like FLDetector~\cite{zhao2020multivariate} are its resilience to large-scale attacks, where $50\%$ or more FL participants are hostile, and the capability of working under realistic non-iid scenarios.
We attribute such a capability to two key factors: $(i)$ FLANDERS works without knowing a priori the ratio of corrupted clients, and $(ii)$ it embodies temporal dependencies between intra- and inter-client updates, quickly recognizing local model drifts caused by evil players. Below, we summarize our main contributions:

\begin{itemize}
\item[{\em(i)}]
We provide empirical evidence that the sequence of models sent by legitimate clients is more predictable than those of malicious participants performing untargeted model poisoning attacks.
\\
\item[{\em(ii)}] 
We introduce FLANDERS, the first pre-aggregation filter for FL robust to untargeted model poisoning based on multidimensional time series anomaly detection.
\\
\item[{\em(iii)}] 
We integrate FLANDERS into Flower,\footnote{\scriptsize{\url{https://flower.dev/}}} a popular FL simulation framework for reproducibility.
\\
\item[{\em(iv)}] 
We show that FLANDERS improves the robustness of the existing aggregation methods under multiple settings: different datasets, client's data distribution (non-iid), models, and attack scenarios.
\\
\item[{\em(v)}] 
We publicly release all the implementation code of FLANDERS along with our experiments.\footnote{\scriptsize{\url{https://anonymous.4open.science/r/flanders_exp-7EEB}}}
\end{itemize}

% Paper's structure and organization
The remainder of the paper is structured as follows. %some related work and the current state-of-the-art solutions to security issues that FL entails. 
Section~\ref{sec:background} covers background and preliminaries. 
In Section~\ref{sec:related}, we discuss related work.
Section~\ref{sec:problem} and Section~\ref{sec:method} describe the problem formulation and the method proposed. % to tackle it. 
Section~\ref{sec:experiments} gathers experimental results. %, and Section~\ref{sec:limitations} discusses some limitations of this work.
Finally, we conclude in Section~\ref{sec:conclusion}.
 %discusses the limitations of this work and draws future research directions.
%reports conclusions and draws perspectives for future research directions.

%%%%%%% OLD %%%%%%%
%to overcome the resilience of Byzantine failures in distributed Stochastic Gradient Descent computations. 
% The strength of Krum is its time complexity, which is linear in the gradient dimension. 
% However, the robustness of the approach is guaranteed for gradient-based learning applications only when the majority of the clients are not compromised. 
% Besides, the aggregation mechanism of Krum, as well as that of similar methods, is robust from a coarse-grained perspective and does not provide solutions to errors and perturbations that may occur at inference time.
%A related approach to~\cite{blanchard2017nips} is the work of Su et al.~\cite{su2016dc}. Here, the authors propose an iterated approximate agreement to tackle a multi-layer scenario attacked by Byzantine agents. 
%However, the method works efficiently on the sole discrete context and it is inapplicable to continuous state environments.
%\gabri{Maybe, we should just talk about the main limitations of existing countermeasures without digging into their details (or, we can just mention Krum as this is the most popular one). I will move the description of all these methods to the Related Work section.}
% !TEX root = ../main.tex

\section{Preliminaries}\label{sec:prelim}

% \subsection{Path decomposition}

%In this section, we fix our notation and provide an overview of treewidth and pathwidth. See~\cite{cygan2015parameterized} for a much more detailed treatment.

\begin{definition}[Path Decomposition \cite{DBLP:journals/jct/RobertsonS83,cygan2015parameterized}]\label{def:path_dec}
A \textit{path decomposition} of a graph $G = (V, E)$ is a sequence $\mathcal{P} = \{X_1, \dots, X_r\}$ of ``bags'', where each bag $X_i$ is a subset of $V,$ such that following conditions hold:
\begin{compactenum}
	\item For each $v \in V$, there exists a pair of indices 
	$1 \leq l(v) \leq r(v) \leq r$ such that $v \in X_i \Leftrightarrow l(v) \leq i \leq r(v),$ i.e.~each vertex of the graph $G$ appears in a contiguous segment of bags.  
	\item For each $uv \in E$, there exists an index $i$ such that $\{u, v \} \subseteq X_i,$ i.e.~there is a bag that contains both endpoints of the edge.
\end{compactenum}


\begin{definition}[Pathwidth~\cite{DBLP:journals/jct/RobertsonS83}]\label{def:pathwidth}
The \emph{width} of a path decomposition $\mathcal{P} = \{X_1, \dots, X_r\}$ is the size of its largest bag minus one, i.e.~$\max_{1 \leq i \leq r} \lvert X_i \rvert - 1$. The \emph{pathwidth} of a graph $G$, denoted by $\pw(G)$, is the minimum possible width among path decompositions of $G.$
\end{definition}

When designing algorithms, it is often useful to turn decompositions into the following folklore form:

\end{definition}
\begin{definition}[Nice Path Decomposition]\label{def:nice_path_dec}
A \emph{path decomposition} $\mathcal{P} = \{X_1, \dots, X_r\}$ is nice if it satisfies the following additional constraints:
\begin{compactenum}
	\item $X_1 = X_r = \emptyset$.
	\item For every $i \geq 1,$ the bag $X_{i+1}$ is of one of the following types:
	\begin{compactitem}
		\item \emph{Forget Node:} There exists a vertex $v \in X_i$ such that $X_{i + 1} = X_i \setminus \{v\}$. In this case, we say that $X_{i+1}$ \emph{forgets} $v.$
		\item  \emph{Introduce Node}: There exists a vertex $v \in V \setminus X_i$ such that $X_{i + 1} = X_i \cup \{v\}$. We say that $X_{i+1}$ \emph{introduces} $v.$
	\end{compactitem}
\end{compactenum}
It is well-known that every path decomposition can be turned into a nice decomposition of the same width in linear time~\cite{cygan2015parameterized}.
\end{definition}


\begin{definition}[Tree Decomposition \cite{robertson1984graph,cygan2015parameterized}]\label{def:tree_dec} 
A \textit{tree decomposition} of a graph $G$ is a pair $\mathcal{T} = (T, \{X_t\}_{t \in V(T)})$, where $T$ is a rooted tree with root $r$, each bag $X_t$ is a subset of vertices of $G$ and the following conditions hold:
\begin{compactenum}
	\item For every $uv \in E(G)$, there exists a node $t \in V(T)$ such that $\{u, v\} \subseteq X_t$. In other words, every edge is covered by some bag.
	\item For every $v \in V(G)$, the set $T_v := \{t \in V(T): v \in X_t\}$, consisting of all nodes of the tree whose bags contain $v,$ forms a non-empty and connected subtree of $T.$ In other words, every vertex is covered by some bag and the set of bags covering each vertex is a subtree of $T.$
\end{compactenum}
\end{definition}

\begin{definition}[Treewidth \cite{robertson1984graph}]\label{def:treewidth}
	The \emph{width} of a tree decomposition $\mathcal{T} = (T, \{X_t\}_{t \in V(T)})$ is defined as $\max_{t \in V(T)} \lvert X_t \rvert - 1.$ The \emph{treewidth} of a graph $G$, denoted by $\twi(G)$, is the minimum possible width among tree decompositions of $G$.
\end{definition}

Given that every path decomposition is by definition a tree decomposition, too, we always have $\pw(G) \geq \twi(G).$ We consider the last bag $r$ of a path decomposition as its root. Moreover, we can define an analogous notion of niceness for tree decompositions:


\begin{definition}[Nice Tree Decomposition \cite{cygan2015parameterized}]\label{def:nice_tree_dec}
The tree decomposition $\mathcal{T} = (T, \{X_t\})$ is \emph{nice} if it satisfies the following conditions:
\begin{compactenum}
	\item The root bag is empty, i.e.~$X_r = \emptyset.$
	\item If $l$ is a leaf of the $T$, then $X_l = \emptyset.$
	\item Each non-leaf node of tree $T$ is of one of the following three types: 
	\begin{compactitem}
		\item \emph{Forget Node:} If $b$ is a forget node, it has exactly one child $c$ and there is a vertex $v \in X_c$ such that $X_b = X_c \setminus \{v\}$. We say that $b$ forgets $v.$
		\item \emph{Introduce Node:} If $b$ is an introduce node, it has exactly one child $c$ and there is a vertex $v \in V(G) \setminus X_c$ such that $X_b = X_c \cup \{v\}$. We say that $b$ introduces $v.$
		\item \emph{Join Node:} If $b$ is a join node, it has exactly two children ${c_1}$ and ${c_2}$ such that $X_b = X_{c_1} = X_{c_2}.$ 
		\end{compactitem}
\end{compactenum}
It is well-known that every tree decomposition can be turned into a nice tree decomposition of the same width in linear time~\cite{cygan2015parameterized}.
\end{definition}


%\begin{remark}
%Notice that (nice) path decomposition is a special case of a (nice) tree decomposition. For a path decomposition $\mathcal{P} = \{X_1, \dots, X_r\}$ we think that the $X_1$ is a leaf bag and the $X_r$ is a root bag. 
%\end{remark}



\begin{notation}\label{notation:subtree}
We write $T_t$ to denote the subtree of $T$ rooted at $t.$ We also define $G^\downarrow_t := G\left[\cup \{X_t: t \in T_t\}\right].$ In other words, $G^\downarrow_t$ is the subgraph of $G$ induced on vertices that appear in the bags at $t$ or its descendants.
\end{notation}

% \todo{make this use the same numbering as defs and examples}

\begin{example}\label{example:explaining_tree_decomposition}
	Figure~\ref{fig:tree_dec_caffeine} (left) is the caffeine molecule. Figure~\ref{fig:tree_dec_caffeine} (center) is a graph representation of the same molecule and Figure~\ref{fig:tree_dec_caffeine} (right) is a tree decomposition of this graph with width $2.$ This is an optimal decomposition and thus the treewidth of caffeine is $2.$

	\begin{figure}
		\centering
		\subfloat[][Caffeine]{\includegraphics{caffeine-cropped.pdf}}
		\quad
		\subfloat[][Caffeine's Graph]{\resizebox{0.22\textwidth}{!}{% \begin{center}
\begin{tikzpicture}[node distance={12mm}, line width=1pt, main/.style = {draw, circle}] 
\node[main] (1) []{$C_1$}; 
\node[main] (2) [below right of=1]{$C_2$}; 
\node[main] (3) [below of=2]{$C_3$}; 
\node[main] (4) [below left of=3]{$N_1$}; 
\node[main] (5) [above left of=4]{$C_4$}; 
\node[main] (6) [above of=5]{$N_2$};
\node[main] (9) [below right of=3]{$N_4$}; 
\node[main] (8) [above right of=9]{$C_5$}; 
\node[main] (7) [above of=8]{$N_3$}; 
\node[main] (10) [above of=1]{$O_1$}; 
\node[main] (11) [above left of=4]{$C_4$}; 
\node[main] (12) [above of=5]{$N_2$}; 

\node[main] (13) [below of=4]{$C_6$};  

\node[main] (17) [below left of=5]{$O_2$};

\node[main] (18) [above left of=6]{$C_7$}; 

\node[main] (22) [above right of=7]{$C_8$}; 



\draw [] (1) -- (2); 
\draw [] (2) -- (3); 
\draw [] (3) -- (4); 
\draw [] (4) -- (5); 
\draw [] (5) -- (6); 
\draw [] (6) -- (1); 
\draw [] (2) -- (7); 
\draw [] (7) -- (8); 
\draw [] (8) -- (9); 
\draw [] (9) -- (3); 
\draw [] (1) -- (10); 

\draw [] (4) -- (13); 

\draw [] (5) -- (17); 

\draw [] (6) -- (18); 

\draw [] (7) -- (22); 
\end{tikzpicture}
% \end{center}
}}
		\quad
		\subfloat[][A Path Decomposition]{\resizebox{0.16\textwidth}{!}{% !TEX root = ../main.tex
% \begin{center}
\begin{tikzpicture}[node distance={7mm}, thick, main/.style = {draw, rectangle}] 
\node[main, fill=green!20, dashed] at (0,0) (8) {$C_2,C_3$};
\node[main,fill=cyan!20] at (-1,-1) (7){$C_1,C_2,C_3$};
\node[main,fill=cyan!20] (100) [below of =7] {$C_1,C_3,N_1$};
\node[main,fill=cyan!20] (6) [below of =100]{$C_1,N_1,C_2$};
\node[main,fill=cyan!20] (5) [below of =6]{$C_1,N_1,C_6$};
\node[main,fill=cyan!20] (4) [below of =5]{$C_4,C_1,N_1$};
\node[main,fill=cyan!20] (3) [below of =4]{$C_4,C_1,O_1$};
\node[main,fill=cyan!20] (2) [below of =3]{$C_4,C_1,O_2$};
\node[main,fill=cyan!20] (1) [below of =2]{$N_2,C_4,C_1$};
\node[main,fill=cyan!20] (0) [below of =1] {$N_2,C_7$}; 

%\node[main,white] (00) [right = 1cm of 0] {}; 
%\node[main,white] (000) [left = 1cm of 0] {}; 



\node[main,fill=orange!20] at (1, -1) (9){$C_2,C_3,N_4$};
\node[main,fill=orange!20] (10) [below of =9]{$C_2,N_4,N_3$};
\node[main,fill=orange!20] (11) [below of =10]{$N_4,N_3,C_5$};
\node[main,fill=orange!20] (12) [below of =11]{$N_3, C_8$};


\draw [] (0) -- (1); 
\draw [] (1) -- (2); 
\draw [] (2) -- (3); 
\draw [] (3) -- (4); 
\draw [] (4) -- (5); 
\draw [] (5) -- (6); 
\draw [] (6) -- (100); 
\draw [] (7) -- (100); 
\draw [] (7) -- (8); 
\draw [] (8) -- (9); 
\draw [] (9) -- (10); 
\draw [] (10) -- (11); 
\draw [] (11) -- (12); 


\end{tikzpicture}
% \end{center}}}
		\quad
		\subfloat[][A Tree Decomposition]{\resizebox{0.3\textwidth}{!}{% \begin{center}
\begin{tikzpicture}[node distance={15mm}, thick, main/.style = {draw, rectangle}] 
\node[main] (0) []{$C_2,C_3$}; 
\node[main] (00) [below left of =0]{$C_2,C_3,N_1$};
\node[main] (000) [below left of =00]{$N_1,C_6$};
\node[main] (001) [below right of =00]{$C_1,C_2,N_1$};
\node[main] (0011) [below right of =001]{$C_1,O_1$};
\node[main] (0010) [below left of =001]{$C_1,C_4,N_1$};
\node[main] (00101) [below right of =0010]{$C_4,O_2$};
\node[main] (00100) [below left of =0010]{$C_1,C_4,N_2$};
\node[main] (001000) [below of =00100,,node distance={10mm}]{$N_2,C_7$};
\node[main] (01) [below right of =0]{$C_2,C_3,N_4$};
\node[main] (011) [below right of =01]{$C_2,N_4,C_5$};
\node[main] (0111) [below right of =011]{$C_2,N_3,C_5$};
\node[main] (01110) [below of =0111,,node distance={10mm}]{$N_3,C_8$};

\draw [] (0) -- (00); 
\draw [] (00) -- (000); 
\draw [] (00) -- (001); 
\draw [] (001) -- (0010); 
\draw [] (001) -- (0011); 
\draw [] (0010) -- (00100); 
\draw [] (00100) -- (001000);  
\draw [] (0010) -- (00101); 

\draw [] (0) -- (01); 
\draw [] (01) -- (011); 
\draw [] (011) -- (0111); 
\draw [] (0111) -- (01110); 

\end{tikzpicture}
% \end{center}}}
		\caption{A Graph Representation of Caffeine and a path and tree decomposition of this graph. Vertices in the root bag of the tree decomposition are highlighted in green (dashed). Notice how the removal of these nodes in the original molecule separates it into two connected components, each corresponding to one of the sides of the path decomposition, and one of the highlighted subtrees in the tree decomposition.}
		\label{fig:tree_dec_caffeine}
	\end{figure}

\end{example}
% \begin{figure}[H]
% 	\centerin{}g
% 	\chemfig{=^[:270](-[:330]=^[:30]-[:90]=^[:150]-[:210])-[:210]=_[:270](%
-[:330]=_[:30]-[:330]=_[:270]-[:210]=_[:150]-[:90])-[:210](-[:270]=_[:330]%
-[:270]=_[:210]-[:150]=_[:90]-[:30])=_[:150](-[:210]=_[:270]-[:210]=_[:150]%
-[:90]=_[:30]-[:330])-[:90](-[:150]=^[:90]-[:150]=^[:210]-[:270]=^[:330]%
-[:30])=_[:30](-[:330])-[:90]=^[:30]-[:90]=^[:150]-[:210]=^[:270](-[:330])}

% 	\caption{1,2,3,4,5,6-hexakis-phenylbenzene}
% 	\label{fig:tw_2 example}
% \end{figure}

\begin{lemma}[Proof in~\ref{app:proof}]\label{intseplemma}
	If $b$ is an introduce node with a single child $c$ and $X_b = X_c \cup \{v\}$, then $N(v)\cap G_b^\downarrow \subseteq X_c$. Here, $N(v)$ is the set of neighbors of $v.$
\end{lemma}



\begin{lemma}[Proof in~\ref{app:proof}]\label{joinseplemma}
	If $b$ is a join node with two children $c_1$ and $c_2,$ then in $G_b^\downarrow$ there is no edge with one endpoint in $V(G_{c_1}^\downarrow) \setminus X_b$ and the other in $V(G_{c_2}^\downarrow) \setminus X_b.$ Informally, $X_b$ is a cut that separates $V(G_{c_1}^\downarrow)$ from $V(G_{c_2}^\downarrow)$ in $G.$ See Figure~\ref{fig:tree_dec_caffeine}.
\end{lemma}

% \begin{proof}
% Let $T^\prime, T^{\prime\prime}, T^{\prime\prime\prime}$ be the subtrees of $T$ rooted at $b, c_1, c_2$ respectively. Let 
% $\mathcal{T}^\prime, \mathcal{T}^{\prime\prime}, \mathcal{T}^{\prime\prime\prime}$ be the corresponding tree decompositions. We will prove the given statement by contradiction. Therefore, let us assume that $u \in V(G_{c_1}^\downarrow) \setminus X_b$, $v \in V(G_{c_2}^\downarrow) \setminus X_b$, and $uv \in E(G_b^\downarrow)$.

% As $\mathcal{T}^{\prime}$ is a tree decomposition of $G^\downarrow_b$, there exists at least one bag $X_t$, such that $t \in \mathcal{T}^{\prime}$ and $u,v \in X_t$. From our initial assumption, we know that $u, v \notin X_b$, this implies that $t \neq b$. Therefore, either $t \in T^{\prime\prime}$ or $t \in T^{\prime\prime\prime}$. Without loss of generality, let us assume that $t \in T^{\prime\prime}$. As $v \in V(G_{c_2}^\downarrow) \setminus X_b$, there exists at least one bag $X_s$ such that, $ s \in T^{\prime\prime\prime}$ and $v \in X_s$. Observe that $v$ appears in both bags $X_s$ and $X_t$. Now by Definition \ref{def:tree_dec}, we know that $T^\prime_{v}$ is a connected subtree and any path connecting $s$ and $t$ goes through $b$. This implies that $X_b \in T^\prime_v$ and $v \in X_b$. This contradicts our initial assumption that $v \notin X_b$, hence completes the proof.
% \end{proof}
% !TEX root = ../main.tex
\section{Notations and Preliminaries}\label{sec:prelim}

% \subsection{Path decomposition}
\subsection{Basic definitions}
\begin{definition}[Path Decomposition \cite{cygan2015parameterized}]\label{def:path_dec}
A \textit{path decomposition} of a graph $G = (V, E)$ is a sequence of bags $\mathcal{P} = \{X_1, \dots, X_r\}$, where each $X_i \subseteq V$ such that following conditions hold:
\begin{enumerate}
	\item For each $v \in V$, there exists a pair of indices 
	$1 \leq l(v) \leq r(v) \leq r$ such that 
	\[v \in X_i \Longleftrightarrow l(v) \leq i \leq r(v).\]
	In other words each graph vertex maps to continuous subpath of the decomposition.   
	\item For each $uv \in E$, there exists an $i$ such that $\{u, v \} \subseteq X_i$.
\end{enumerate}

\end{definition}
\begin{definition}[Nice Path Decomposition \cite{cygan2015parameterized}]\label{def:nice_path_dec}
A \textit{nice path decomposition} is a path decomposition, where additional conditions hold:
\begin{enumerate}
	\item $X_1, X_r = \varnothing$.
	\item Each bag, except $X_1$, either \textit{introduce} or \textit{forget} node.
	\item If $X_{i + 1}$ is a forget node, then there exists $v \in V$ such that $X_{i + 1} = X_i \setminus \{v\}$.
	\item  If $X_{i + 1}$ is an introduce node, then there exists $v \in V$ such that $X_{i + 1} = X_i \cup \{v\}$.
	
\end{enumerate}
\end{definition}

\begin{definition}[Pathwidth \cite{cygan2015parameterized}]\label{def:pathwidth}
The width of a path decomposition $\mathcal{P} = \{X_1, \dots, X_r\}$ is $max_{1 \leq i \leq r} \lvert X_i \rvert - 1$. The pathwidth of a graph $G$, denoted by pw(G), is the minimum possible width of a path decomposition
of G.
\end{definition}



\begin{definition}[Tree Decomposition \cite{cygan2015parameterized}]\label{def:tree_dec} 
A \textit{tree decomposition} of a graph $G$ is a pair $\mathcal{T} = (T, \{X_t\}_{t \in V(T)})$, where $T$ is a rooted tree with a root $r$, each bag $X_t \subseteq V(G)$ and the following conditions hold:
\begin{enumerate}
	\item For every $uv \in E(G)$, there exists a node $t \in V(T)$ such that $\{u, v\} \subseteq X_t$.
	\item For every $v \in V(G)$, the set $T_v := \{t \in V(T): v \in X_t\}$, induced graph $T[T_v]$ is nonempty subtree of the $T$. 
\end{enumerate}
\end{definition}


\begin{definition}[Nice Tree Decomposition \cite{cygan2015parameterized}]\label{def:nice_tree_dec}
A \textit{nice tree decomposition} is a tree decomposition, where additional conditions hold:

\begin{enumerate}
	\item The root bag is empty: $X_r = \varnothing$.
	\item If $l$ is a leaf of the $T$, then $X_l = \varnothing$.
	\item Each non-leaf node of tree $T$ is of one of the three types: \textit{introduce}, \textit{forget} or \textit{join} node.
	\item If $X_b$ is a forget node, it has exactly one child $X_c$ and there is a vertex $v \in X_c$ such that $X_b = X_c \setminus \{v\}$.
	\item If $X_b$ is an introduce node, it has exactly one child $X_c$ and there is a vertex $v \in V(G) \setminus X_c$ such that $X_b = X_c \cup \{v\}$.
	\item If $X_b$ is a join node, it has exactly two children $X_{c_1}$ and $X_{c_2}$ such that $X_b = X_{c_1} = X_{c_2}$. 
\end{enumerate}
\end{definition}

\begin{definition}[Treewidth \cite{cygan2015parameterized}]\label{def:treewidth}
The width of tree decomposition $\mathcal{T} = (T, \{X_t\}_{t \in V(T)})$ equals $max_{t \in V(T)} \lvert X_t \rvert - 1$, that is, the maximum size of its bag minus 1. The treewidth of a graph $G$, denoted by $tw(G)$, is the minimum possible width of a tree decomposition of $G$.
\end{definition}

\begin{remark}
Notice that (nice) path decomposition is a special case of a (nice) tree decomposition. For a path decomposition $\mathcal{P} = \{X_1, \dots, X_r\}$ we think that the $X_1$ is a leaf bag and the $X_r$ is a root bag. 
\end{remark}



\begin{notation}\label{notation:subtree}
For each $t \in G(T)$ we denote subtree rooted at this vertex as $T_t$, and denote a graph induced by all vertices in all bags correspond to $T_t$:
\[
G^\downarrow_t := G\left[\bigcup \{X_t: t \in T_t\}\right].
\]
\end{notation}

\begin{example}\label{example:explaining_tree_decomposition}
	In this example, we explain the above defintions by explicitly showing tree decomposition for underlying graph of Caffeine molecule (See Figure \ref{fig:caffeine example}).
	\begin{figure}[H]
		\centering
		\chemfig{O=[:270]-[:210]N(-[:150])-[:270](=[:210]O)-[:330]N(-[:270])-[:30]%
-[:342.2,0.994]N=^[:54,0.994]-[:126,0.994]N(-[:71.9])-[:197.8,0.994](%
=^[:270])(-[:150])}

		\caption{Caffeine}
		\label{fig:caffeine example}
	\end{figure}

	We can see from Figure \ref{fig:tree_dec_caffeine}, that Caffeine molecule has treewidth $2$.

	% See Figure \ref{fig:caffeine example} for Caffeine, it has treewidth $2$.

	\begin{figure}[H]
		\centering
		\subfloat[][Graph of Caffeine]{\resizebox{0.2\textwidth}{!}{% \begin{center}
\begin{tikzpicture}[node distance={12mm}, line width=1pt, main/.style = {draw, circle}] 
\node[main] (1) []{$C_1$}; 
\node[main] (2) [below right of=1]{$C_2$}; 
\node[main] (3) [below of=2]{$C_3$}; 
\node[main] (4) [below left of=3]{$N_1$}; 
\node[main] (5) [above left of=4]{$C_4$}; 
\node[main] (6) [above of=5]{$N_2$};
\node[main] (9) [below right of=3]{$N_4$}; 
\node[main] (8) [above right of=9]{$C_5$}; 
\node[main] (7) [above of=8]{$N_3$}; 
\node[main] (10) [above of=1]{$O_1$}; 
\node[main] (11) [above left of=4]{$C_4$}; 
\node[main] (12) [above of=5]{$N_2$}; 

\node[main] (13) [below of=4]{$C_6$};  

\node[main] (17) [below left of=5]{$O_2$};

\node[main] (18) [above left of=6]{$C_7$}; 

\node[main] (22) [above right of=7]{$C_8$}; 



\draw [] (1) -- (2); 
\draw [] (2) -- (3); 
\draw [] (3) -- (4); 
\draw [] (4) -- (5); 
\draw [] (5) -- (6); 
\draw [] (6) -- (1); 
\draw [] (2) -- (7); 
\draw [] (7) -- (8); 
\draw [] (8) -- (9); 
\draw [] (9) -- (3); 
\draw [] (1) -- (10); 

\draw [] (4) -- (13); 

\draw [] (5) -- (17); 

\draw [] (6) -- (18); 

\draw [] (7) -- (22); 
\end{tikzpicture}
% \end{center}
}}
		\qquad
		\qquad
		\qquad
		\qquad
		\subfloat[][Tree decomposition]{\resizebox{0.3\textwidth}{!}{% \begin{center}
\begin{tikzpicture}[node distance={15mm}, thick, main/.style = {draw, rectangle}] 
\node[main] (0) []{$C_2,C_3$}; 
\node[main] (00) [below left of =0]{$C_2,C_3,N_1$};
\node[main] (000) [below left of =00]{$N_1,C_6$};
\node[main] (001) [below right of =00]{$C_1,C_2,N_1$};
\node[main] (0011) [below right of =001]{$C_1,O_1$};
\node[main] (0010) [below left of =001]{$C_1,C_4,N_1$};
\node[main] (00101) [below right of =0010]{$C_4,O_2$};
\node[main] (00100) [below left of =0010]{$C_1,C_4,N_2$};
\node[main] (001000) [below of =00100,,node distance={10mm}]{$N_2,C_7$};
\node[main] (01) [below right of =0]{$C_2,C_3,N_4$};
\node[main] (011) [below right of =01]{$C_2,N_4,C_5$};
\node[main] (0111) [below right of =011]{$C_2,N_3,C_5$};
\node[main] (01110) [below of =0111,,node distance={10mm}]{$N_3,C_8$};

\draw [] (0) -- (00); 
\draw [] (00) -- (000); 
\draw [] (00) -- (001); 
\draw [] (001) -- (0010); 
\draw [] (001) -- (0011); 
\draw [] (0010) -- (00100); 
\draw [] (00100) -- (001000);  
\draw [] (0010) -- (00101); 

\draw [] (0) -- (01); 
\draw [] (01) -- (011); 
\draw [] (011) -- (0111); 
\draw [] (0111) -- (01110); 

\end{tikzpicture}
% \end{center}}}
		\caption{Graph representation and tree decomposition of Caffeine.}
		\label{fig:tree_dec_caffeine}
	\end{figure}


\end{example}
% \begin{figure}[H]
% 	\centering
% 	\chemfig{=^[:270](-[:330]=^[:30]-[:90]=^[:150]-[:210])-[:210]=_[:270](%
-[:330]=_[:30]-[:330]=_[:270]-[:210]=_[:150]-[:90])-[:210](-[:270]=_[:330]%
-[:270]=_[:210]-[:150]=_[:90]-[:30])=_[:150](-[:210]=_[:270]-[:210]=_[:150]%
-[:90]=_[:30]-[:330])-[:90](-[:150]=^[:90]-[:150]=^[:210]-[:270]=^[:330]%
-[:30])=_[:30](-[:330])-[:90]=^[:30]-[:90]=^[:150]-[:210]=^[:270](-[:330])}

% 	\caption{1,2,3,4,5,6-hexakis-phenylbenzene}
% 	\label{fig:tw_2 example}
% \end{figure}


\subsection{Technical Lemmas}
\noindent We now present two technical lemmas that are crucial for development of algorithms in Section \ref{sec:algos}. These lemmas are consequences of separation properties of the tree decomposition.

\begin{lemma}\label{intseplemma}
	If $X_b$ is an introduce nodes, $X_b = X_c \cup \{v\}$, then $N(v)\cap G_b^\downarrow \subseteq X_c$.
\end{lemma}
\begin{proof} 
Let $T^\prime$ be a subtree of the tree $T$ rooted at $b$. Observe that the corresponding tree decomposition
$\mathcal{T}^\prime = \{T^\prime, \{X_t\}_{t\in V(T^\prime)}\}$ is a tree decomposition of $G_b^\downarrow$. Notice that by Definition \ref{def:tree_dec}, $T^\prime_v$ forms an induced subtree of the $T^\prime$. As $v \notin X_c$ and $c$ is the only vertex in the open neighbourhood of $b$, this implies that $T^\prime_v = \{b\}$.

Let $u$ be a vertex in $N(v) \cap G_b^\downarrow$, then by Definition \ref{def:tree_dec}, if there is an edge $uv$ in $G_b^\downarrow$, then there exists a bag containing both $u$ and $v$. But, as we have just shown that the only bag containing $v$ is $X_b$. As $u$ is a vertex in $N(v)$, this implies that $u \in X_c$.
\end{proof}
\begin{lemma}\label{joinseplemma}
	If $X_b$ is a join node with two children $X_b = X_{c_1} = X_{c_2}$, then in $G_b^\downarrow$ there is no edge between $V(G_{c_1}^\downarrow) \setminus X_b$ and $V(G_{c_2}^\downarrow) \setminus X_b$.
\end{lemma}
\begin{proof}
Let $T^\prime, T^{\prime\prime}, T^{\prime\prime\prime}$ be the subtrees of $T$ rooted at $b, c_1, c_2$ respectively. Let 
$\mathcal{T}^\prime, \mathcal{T}^{\prime\prime}, \mathcal{T}^{\prime\prime\prime}$ be the corresponding tree decompositions. We will prove the given statement by contradiction. Therefore, let us assume that $u \in V(G_{c_1}^\downarrow) \setminus X_b$, $v \in V(G_{c_2}^\downarrow) \setminus X_b$, and $uv \in E(G_b^\downarrow)$.

As $\mathcal{T}^{\prime}$ is a tree decomposition of $G^\downarrow_b$, there exists at least one bag $X_t$, such that $t \in \mathcal{T}^{\prime}$ and $u,v \in X_t$. From our initial assumption, we know that $u, v \notin X_b$, this implies that $t \neq b$. Therefore, either $t \in T^{\prime\prime}$ or $t \in T^{\prime\prime\prime}$. Without loss of generality, let us assume that $t \in T^{\prime\prime}$. As $v \in V(G_{c_2}^\downarrow) \setminus X_b$, there exists at least one bag $X_s$ such that, $ s \in T^{\prime\prime\prime}$ and $v \in X_s$. Observe that $v$ appears in both bags $X_s$ and $X_t$. Now by Definition \ref{def:tree_dec}, we know that $T^\prime_{v}$ is a connected subtree and any path connecting $s$ and $t$ goes through $b$. This implies that $X_b \in T^\prime_v$ and $v \in X_b$. This contradicts our initial assumption that $v \notin X_b$, hence completes the proof.
\end{proof}
 
\section{Our Algorithms}\label{sec:algos} 
\subsection{Counting Perfect Matchings}\label{sec:perfect_matchings}
Enumeration of Kekulé structures in organic molecules is equivalent to finding total number of perfect matchings for the underlying chemical graph of the molecule.

We present parametrized algorithms for the cases when the underlying chemical graph have bounded pathwidth and bounded treewidth respectively.



\subsubsection{Bounded Pathwidth}\label{subsec:perfect_pathwidth}
We will use a dynamic programming approach over the nice path decomposition of the given graph $G$. First we will define the dynamic programming state as follows:

\begin{definition}[Respectful Perfect Matchings]\label{def:respectful_perfect_matching}
Let us fix a nice path decomposition $\mathcal{P} = \{X_1, \dots X_r\}$ for the given graph $G$. For each $b \in [r]$ and each $M \subseteq X_b$, we define $\RP(b,M)$ as the set of all perfect matchings $F$ in a $G_b^\downarrow \setminus M$ such that each matching edge $uv \in F$ has at least one point in $G_b^\downarrow \setminus X_b$: $u \notin X_b$ or $v \notin X_b$.
\end{definition}
\begin{definition}[Dynamic Programming State for Perfect Matchings]\label{def:dp_perfect_matching}
Let us fix a nice path decomposition $\mathcal{P} = \{X_1, \dots X_r\}$ for the given graph $G$. For each $b \in [r]$ and each $M \subseteq X_b$, we define $\pma[b, M]$ to be the number of elements in the set $\RP(b,M)$.
\end{definition}

\begin{remark}
Notice that if $X_r = \varnothing$ is the root node, then $G_{r}^\downarrow = G$, and perfect matchings of $G$ are in one-to-one correspondence with respectful perfect matchings $\RP(r,\varnothing)$, which implies $\pma [r, \varnothing]$ is the total number of perfect matchings of $G$.
% If $X_l = \varnothing$ is a leaf bag, only $(X_l, \varnothing)$-respectful matching is an empty 
% matching, which means: 
% \[
% \dpt[X_l, \varnothing] = 1.
% \]
\end{remark}

We now provide a bottom-up approach for filling up the dynamic table. The dynamic table relations depend on the type of the bag. We need to consider three cases when $X_b$ is: leaf node, introduce node and forget node respectively. We refer the reader to follow proof along with Example \ref{example:explaining_dp_perfect_matchings} for better clarity of the dynamic programming approach.


\begin{itemize}
\item \textbf{Leaf node:} If $X_l$ is a leaf bag then $\RP(b,M)$ contains only empty matching as $X_l = \varnothing$. Therefore, $\pma[l,\varnothing] = 1$.
\item \textbf{Introduce node:} Let $X_b$ be an introduce bag such that $X_b = X_c \cup \{v\}$. Then the dynamic programming state corresponding to $(b,M)$ satisfies the following recurrence relations: 
\[
\pma [b,M] =
\begin{cases}
\pma [c,M\setminus v], &v \in M,\\
0, &v \not\in M.
\end{cases}
\]
In order to derive the above recurrence relations, we have to consider two possibilities for $v$.
\begin{enumerate}
\item If $v \in M$, then by Definition \ref{def:respectful_perfect_matching}, all the elements of the set $\RP(b,M)$ are in one-to-one correspondence with all the elements of the set $\RP(c, M \setminus v)$. 
\item If $v \notin M$, then for every matching $F \in \RP(b,M)$, $v$ should be covered by some edge $vw \in F$. By Lemma \ref{intseplemma}, $w \in X_{c} \subset X_b$, which contradicts the Definition \ref{def:respectful_perfect_matching}. Therefore, $\pma[b,m] = 0$ for this case, as there are no matchings satisfying the given conditions.
\end{enumerate}
This transition from $X_c$ to $X_b$ can be computed in $\bigO(1)$ time. 
\item \textbf{Forget node:} Let $X_b$ be the forget node such that $X_b = X_c \setminus \{v\}$. Then the dynamic programming state corresponding to $(b,M)$ satisfies the following recurrence relation: 

\[
\pma[b,M]=
	\pma[c, M] + \sum_{u \in X_b\setminus M:\ uv\in E(G)}\pma[c, M \cup \{v,u\}]
\]
In order to derive the above recurrence relation, let $F$ be a matching from the set $\RP(b,M)$. Note that $v \notin M$, as $M \subseteq X_b$. Therefore, $v$ will be matched by some edge $uv \in F$. Now there are two possible cases i.e., either $u \notin X_b$ or $u \in X_b$. For the former case when $u \notin X_b$, it is clear that there is a bijection between all such matchings $F$ and all elements of $\RP(c,M)$. Therefore, number of such matchings correspond to the first term $\pma[c,M]$ of the recurrence relation. Now let's consider the latter case when $u \in X_b$, then the matching $F \setminus uv \in \RP(c,M M \cup \{u, v\})$, and number of such matchings correspond to the second term $\sum_{u \in X_b\setminus M:\ uv\in E(G)}\pma[c, M \cup \{v,u\}]$ of the recurrence relation.





% To prove that consider any $(X_b, M)$-respectful matching $F$. Since $v \notin M$, $v$ should be matched with some point, say, $u$. Then if $u\notin X_b$, matching $F$ is $(X_{c}, M)$-respectful, and number of such cases corresponds to the first term. If $u \in X_b$ then mathching $F \setminus uv$ is $(X_{c}, M \cup \{u, v\})$ respectful and correspondence is one-to-one.  

This transition to $X_b$ can be computed in $\bigO(\lvert X_b\rvert) = \bigO(\pw(G))$ time.
\end{itemize}


\begin{proposition}\label{prop:comp_pw_perfmat}
The time complexity for finding the total number of perfect matchings for a graph $G = (V,E)$, with $\lvert V \rvert = n$ and pathwidth $\pw$, using the above algorithm is $\bigO(n\cdot \poly(\pw) \cdot2^{\pw})$.
% The final complexity is $\bigO(n\cdot \textup{pw}\cdot2^{\textup{pw}})$
\end{proposition}
\begin{proof}
For each bag, we have at most $2^{\pw}$ dynamic programming states, and for each state we spend at most $\bigO(\poly(\pw))$ time. It is known that there are at most $\bigO(n \cdot \pw)$ bags for any nice path decomposition of $G$ \cite[Chapter 7]{cygan2015parameterized}. Therefore, we spend at most $\bigO(n\cdot \poly(\pw) \cdot2^{\pw})$ time to completely fill the dymamic programming table.
\end{proof}
\subsubsection{Bounded Treewidth}\label{subsec:perfect_treewidth}
We will use a dynamic programming approach over the nice tree decomposition of the given graph $G$. The definitions of respectful perfect matchings and dynamic programming states used here are same as Section \ref{subsec:perfect_pathwidth}. Let $\mathcal{T} = \{T, \{X_t\}_{t \in V(T)}\}$ be a nice tree decomposition of the graph $G = (V,E)$ under consideration.

% We will use a dynamic programming over the nice tree decomposition. Definitions of dynamic programming and respectfulness have only cosmetic differences with path decomposition case. Let us fix a nice tree decomposition $\mathcal{T} = \{T, \{X_t\}_{t \in V(T)}\}$. For each $b \in T$ and each $M \subseteq X_b$ we will define $\dpt[b, M]$ as a number of perfect matchings in a $G_b^\downarrow \setminus M$ such that each matching edge $uv$ has at least on point in $G_b^\downarrow \setminus X_b$: $u \notin X_b$ or $v \notin X_b$. We call such matchings \textit{respectful} to the $(X_b, M)$.


% \begin{remark}

% \end{remark}
% If $X_l = \varnothing$ is a leaf bag, we have only empty $(X_l, \varnothing)$-respectful matching, which means 
% \[
% 	\dpt[X_l, \varnothing] = 1.
% \]

% Also for the root bag $X_r$, $\dpt[X_r, \varnothing]$ contains number of perfect matchings in a whole graph $G$.


% Cases of introduce and forget nodes repet verbatim. We need only consider merge case now. 
We now provide a bottom-up approach for filling up the dynamic table. The dynamic table relations depend on the type of the bag. There are four possible cases for the type of bags, i.e., when $X_b$ is: leaf node, introduce node, forget node and join node respectively. The reccurrence relations for leaf node, introduce node and forget node remains the same as Section \ref{subsec:perfect_pathwidth}. Therefore, we only need to consider the case of join node which is described as follows:
\begin{itemize}
	\item \textbf{Join node:} Let $X_b$ be the join node with $X_{c_1}$ and $X_{c_2}$ as its children. Then the dynamic programming state corresponding to $(b,M)$ will satisfy the following recurrence relation:
\[
\pma[b,M] = \sum_{H_1 \sqcup H_2 = X_b \setminus M}\pma[c_1,M\cup H_2]\cdot \pma[c_2, M\cup H_1]
% \\
% &=\sum_{H \subseteq X_b \setminus M}\pma[c_1,M\cup H]\cdot \pma[c_2, X_b \setminus H],\\
\]

where $H_1 \sqcup H_2 = X_b \setminus M$ means $H_1 \cup H_2 = X_b \setminus M$ and $H_1 \cap H_2 = \varnothing$. 
% This two formulas yields same result, but first easier to analyze and second easier to implement.  
In order to derive the above recurrence relation, let $F$ be a matching from the set $\RP(b,M)$. By Lemma \ref{joinseplemma}, $F$ doesn't have any edges between $G_{c_1}^\downarrow \setminus X_b$ and $G_{c_2}^\downarrow \setminus X_b$. Therefore, it can be split into two matchings, $F_1 := E(G_{c_1}^\downarrow) \cap F$ and $F_2 := E(G_{c_2}^\downarrow) \cap F$. Let $H_1 := V(F_1) \cap X_b$ and $H_2 := V(F_2) \cap X_b$. It is clear from the Definition \ref{def:respectful_perfect_matching}, that $F_1 \in \RP(c_1,M \cup H_2)$ and $F_2 \in \RP(c_2,M \cup H_1)$. If we choose $H_1$ and $H_2$ such that $H_1 \sqcup H_2 = X_b \setminus M$, then for all such matchings $F_1 \in \RP(c_1,M\cup H_2)$ and $F_2 \in \RP(c_2, M\cup H_1)$, we get a matching $ F = F_1 \cup F_2$, such that $F \in \RP(b,M)$. This leads to our recurrence relation.



% Consider arbitrary $(X_b, M)$-respectful matching $F$. By lemma \ref{joinseplemma} $F$ do not have edges between $G_{c_1}^\downarrow \setminus X_b$ and $G_{c_2}^\downarrow \setminus X_b$. Let us split this matching $F$ into $F_1 := E(G_{c_1}^\downarrow) \cap F$, $F_2 := E(G_{c_2}^\downarrow) \cap F$, also let $H_1 := V(F_1) \cap X_b$, $H_2 := V(F_2) \cap X_b$. Observe that $F_1$ is $(X_{c_1}, M \cup H_2)$-respectful and $F_2$ is $(X_{c_2}, M \cup H_1)$-respectful. Notice that if $H_1, H_2$ are chosen, $F_1, F_2$ acts independently, so when $H_1, H_2$ fixed such matching could be counted as $\dpt[c_1, M \cup H_2] \cdot \dpt[c_2, M \cup H_1]$, which leads to our formula. 

\end{itemize}

\begin{proposition}\label{prop:comp_tw_perfmat}
The time complexity for finding the total number of perfect matchings for a graph $G = (V,E)$, with $\lvert V \rvert = n$ and treewidth $\twi$, using the above algorithm is $\bigO(n\cdot \poly(\twi)\cdot3^{\twi})$.
% With this, the final complexity of the algorithm increases to .
\end{proposition}
\begin{proof}
The transition for a join node is the most expensive operation for this algorithm, the rest of the cases are similar to the dynamic programming on nice path decomposition in Section \ref{subsec:perfect_pathwidth}.
Note that the recurrence relation for the dynamic programming state of join node can be rewritten as:
\[
\sum_{H \subseteq X_b \setminus M}\pma[c_1,M\cup H]\cdot \pma[c_2, X_b \setminus H]
\]

For each bag $X_b$ and for each $M \subseteq X_b$, we spend at most $\bigO(\poly(\twi)\cdot 2^{\lvert X_B \rvert - \lvert M \rvert})$ time. By using the binomial theorem, the total time spend for each bag $X_b$ for all possible subsets $M$ is $\bigO(\poly(\twi)\cdot 3^{\lvert X_B \rvert})$. It is known that there are at most $\bigO(n \cdot \twi)$ bags for any nice tree decomposition of $G$ \cite[Chapter 7]{cygan2015parameterized}. Therefore, we spend at most $\bigO(n\cdot \poly(\twi)\cdot3^{\twi})$ time to completely fill the dymamic programming table.
\end{proof}

\begin{example}\label{example:explaining_dp_perfect_matchings}
	In this example, we explain the above algorithm explicitly with figures for all cases of nodes in the tree decomposition. Red color nodes in the following figures are elements of the set $M$ defined in Definition \ref{def:respectful_perfect_matching}.

	\begin{itemize}
		\item Introduce Node:
		\begin{figure}[H]
			\centering
			\resizebox{0.6\textwidth}{!}{\begin{tikzpicture}[line width=1pt,node distance=14mm ,main/.style = {draw, rectangle},scale=1] 
    \node[scale=.8] at (0,5.5){$\textup{PerfMatch}[b,\{x_1,x_2,x_3,x_4\}]$};
   \node[main] at (0,4) (9) {
   \begin{tikzpicture}[line width=1pt,node distance=14mm ,main/.style = {draw, circle},scale=1] 
   \node[main,color=red, densely dashed] at (0,1) (x1) {\texttt{$x_1$}};
   \node[main,color=red,densely dashed] at (1,1) (x2) {\texttt{$x_2$}};
       \node[main,color=red,densely dashed] at (0,0) (x3) {\texttt{$x_3$}};
       \node[main,color=red,densely dashed] at (1,0) (x4) {\texttt{$x_4$}};
      \draw (x1) -- (x2);
       \draw (x1) -- (x3);
      \draw (x2) -- (x3);
      \draw (x3) -- (x4);
   \end{tikzpicture}
       };
   
    \node[scale=.8] at (3.5,5.5){$\textup{PerfMatch}[b,\{x_1,x_4\}]$};
       \node[main] at (3.5,4) (9) {
   \begin{tikzpicture}[line width=1pt,node distance=14mm ,main/.style = {draw, circle},scale=1] 
   \node[main,color=red,densely dashed] at (0,1) (x1) {\texttt{$x_1$}};
   \node[main,color=black] at (1,1) (x2) {\texttt{$x_2$}};
       \node[main,color=black] at (0,0) (x3) {\texttt{$x_3$}};
       \node[main,color=red,densely dashed] at (1,0) (x4) {\texttt{$x_4$}};
      \draw (x1) -- (x2);
       \draw (x1) -- (x3);
      \draw (x2) -- (x3);
      \draw (x3) -- (x4);
   \end{tikzpicture}
       };
   
   \node[scale=.8] at (7,5.5){$\textup{PerfMatch}[b,\{x_1,x_2\}]$};
       \node[main] at (7,4) (9) {
   \begin{tikzpicture}[line width=1pt,node distance=14mm ,main/.style = {draw, circle},scale=1] 
   \node[main,color=red,densely dashed] at (0,1) (x1) {\texttt{$x_1$}};
   \node[main,color=red,densely dashed] at (1,1) (x2) {\texttt{$x_2$}};
       \node[main,color=black] at (0,0) (x3) {\texttt{$x_3$}};
       \node[main,color=black] at (1,0) (x4) {\texttt{$x_4$}};
      \draw (x1) -- (x2);
       \draw (x1) -- (x3);
      \draw (x2) -- (x3);
      \draw (x3) -- (x4);
   \end{tikzpicture}
       };
       
    \node[scale=.8] at (0,1.5){$\textup{PerfMatch}[c,\{x_2,x_3,x_4\}]$};
       \node[main] at (0,0) (8) {
   \begin{tikzpicture}[line width=1pt,node distance=14mm ,main/.style = {draw, circle},scale=1] 
   \node[main,color=red,densely dashed] at (1,1) (x2) {\texttt{$x_2$}};
       \node[main,color=red,densely dashed] at (0,0) (x3) {\texttt{$x_3$}};
       \node[main,color=red,densely dashed] at (1,0) (x4) {\texttt{$x_4$}};
      \draw (x2) -- (x3);
      \draw (x3) -- (x4);
   \end{tikzpicture}
   };
   
    \node[scale=.8] at (3.5,1.5){$\textup{PerfMatch}[c,\{x_4\}]$};
   \node[main] at (3.5,0) (8) {
   \begin{tikzpicture}[line width=1pt,node distance=14mm ,main/.style = {draw, circle},scale=1] 
   \node[main,color=black] at (1,1) (x2) {\texttt{$x_2$}};
       \node[main,color=black] at (0,0) (x3) {\texttt{$x_3$}};
       \node[main,color=red,densely dashed] at (1,0) (x4) {\texttt{$x_4$}};
      \draw (x2) -- (x3);
      \draw (x3) -- (x4);
   \end{tikzpicture}
   };
   
    \node[scale=.8] at (7,1.5){$\textup{PerfMatch}[c,\{x_2\}]$};
   \node[main] at (7,0) (8) {
   \begin{tikzpicture}[line width=1pt,node distance=14mm ,main/.style = {draw, circle},scale=1] 
   \node[main,color=red,densely dashed] at (1,1) (x2) {\texttt{$x_2$}};
       \node[main,color=black] at (0,0) (x3) {\texttt{$x_3$}};
       \node[main,color=black] at (1,0) (x4) {\texttt{$x_4$}};
      \draw (x2) -- (x3);
      \draw (x3) -- (x4);
   \end{tikzpicture}
   };
   
   
   
       \draw [->](0,2) -- (0,2.5);
       \draw [->](3.5,2) -- (3.5,2.5);
       \draw [->](7,2) -- (7,2.5);
   
   
    \node[scale=.8] at (10.5,5.5){$\textup{PerfMatch}[b,\{x_3,x_4\}]$};
       \node[main] at (10.5,4) (9) {
   \begin{tikzpicture}[line width=1pt,node distance=14mm ,main/.style = {draw, circle},scale=1] 
   \node[main,color=black] at (0,1) (x1) {\texttt{$x_1$}};
   \node[main,color=black] at (1,1) (x2) {\texttt{$x_2$}};
       \node[main,color=red,densely dashed] at (0,0) (x3) {\texttt{$x_3$}};
       \node[main,color=red,densely dashed] at (1,0) (x4) {\texttt{$x_4$}};
      \draw (x1) -- (x2);
       \draw (x1) -- (x3);
      \draw (x2) -- (x3);
      \draw (x3) -- (x4);
   \end{tikzpicture}
       };
   
       \draw [->](10.5,1) -- (10.5,2.5);
       \node[scale=3] at (10.5,0){$\varnothing$};
       %\node[text width=6cm] at (2,-2){$\DP[i,c]$};
       %\node[text width=6cm] at (5.5,-2){$\DP[i,c+\{x_4\to \texttt{gray/dotted}]$};
      
   \end{tikzpicture}
   }
			\label{fig:introduce_node}
			\caption{Example for introduce node.}
		\end{figure}
		\item Forget Node:
		\begin{figure}[H]
			\centering
			\resizebox{0.6\textwidth}{!}{% !TEX root = ../main.tex
\begin{tikzpicture}[line width=1pt,node distance=14mm ,main/.style = {draw, rectangle},scale=1] 

    \node[scale=.8] at (3,6){$\textup{PerfMatch}[b,\{x_4\}]$};
       
       \node[main] at (3,4.5) (9) {
   \begin{tikzpicture}[line width=1pt,node distance=14mm ,main/.style = {draw, circle},scale=1] 
   \node[main,color=black] at (1,1) (x2) {\texttt{$x_2$}};
       \node[main,color=black] at (0,0) (x3) {\texttt{$x_3$}};
       \node[main,fill=red!40!white,densely dashed] at (1,0) (x4) {\texttt{$x_4$}};
      \draw (x2) -- (x3);
      \draw (x3) -- (x4);
   \end{tikzpicture}
       };
       
    \node[scale=.8] at (-1,1.5){$\textup{PerfMatch}[c,\{x_4\}]$};
       \node[main] at (-1,0) (8) {
   \begin{tikzpicture}[line width=1pt,node distance=14mm ,main/.style = {draw, circle},scale=1] 
   \node[main,color=black] at (0,1) (x1) {\texttt{$x_1$}};
   \node[main,color=black] at (1,1) (x2) {\texttt{$x_2$}};
       \node[main,color=black] at (0,0) (x3) {\texttt{$x_3$}};
       \node[main,fill=red!40!white,densely dashed] at (1,0) (x4) {\texttt{$x_4$}};
      \draw (x1) -- (x2);
       \draw (x1) -- (x3);
      \draw (x2) -- (x3);
      \draw (x3) -- (x4);
   \end{tikzpicture}
   };
   
    \node[scale=.8] at (3,1.5){$\textup{PerfMatch}[c,\{x_4\}\cup \{x_1,x_2\}]$};
   \node[main] at (3,0) (8) {
   \begin{tikzpicture}[line width=1pt,node distance=14mm ,main/.style = {draw, circle},scale=1] 
   \node[main,fill=red!40!white,densely dashed] at (0,1) (x1) {\texttt{$x_1$}};
   \node[main,fill=red!40!white,densely dashed] at (1,1) (x2) {\texttt{$x_2$}};
       \node[main,color=black] at (0,0) (x3) {\texttt{$x_3$}};
       \node[main,fill=red!40!white,densely dashed] at (1,0) (x4) {\texttt{$x_4$}};
      \draw [line width = 3pt, color=black]  (x1) -- (x2);
       \draw (x1) -- (x3);
      \draw (x2) -- (x3);
      \draw (x3) -- (x4);
   \end{tikzpicture}
   };
   
    \node[scale=.8] at (7,1.5){$\textup{PerfMatch}[c,\{x_4\}\cup \{x_1,x_3\}]$};
   \node[main] at (7,0) (8) {
   \begin{tikzpicture}[line width=1pt,node distance=14mm ,main/.style = {draw, circle},scale=1] 
   \node[main,fill=red!40!white,densely dashed] at (0,1) (x1) {\texttt{$x_1$}};
   \node[main,color=black] at (1,1) (x2) {\texttt{$x_2$}};
       \node[main,fill=red!40!white,densely dashed] at (0,0) (x3) {\texttt{$x_3$}};
       \node[main,fill=red!40!white,densely dashed] at (1,0) (x4) {\texttt{$x_4$}};
      \draw (x1) -- (x2);
       \draw [line width = 3pt, color=black] (x1) -- (x3);
      \draw (x2) -- (x3);
      \draw (x3) -- (x4);
   \end{tikzpicture}
   };
   
       \draw [decorate,decoration={brace,amplitude=10}] (0,2.5) -- (6,2.5) node [black,midway,xshift=-0.6cm] {};
   
       %\node[text width=6cm] at (2,-2){$\DP[i,c]$};
       %\node[text width=6cm] at (5.5,-2){$\DP[i,c+\{x_4\to \texttt{gray/dotted}]$};
      
   \end{tikzpicture}
   }
			\label{fig:forget_node}
			\caption{Example for forget node.}
		\end{figure}
		\item Join Node:
		\begin{figure}[H]
			\centering
			\resizebox{0.6\textwidth}{!}{% !TEX root = ../main.tex
\begin{tikzpicture}[line width=1pt,node distance=14mm ,main/.style = {draw, rectangle},scale=1] 

    \node[scale=.8] at (-4,1){$\textup{PerfMatch}[b,\{x_1,x_4\}]$};
       \node[main] at (-4,-0.5) (9) {
   \begin{tikzpicture}[line width=1pt,node distance=14mm ,main/.style = {draw, circle},scale=1] 
   \node[main,fill=red!40!white,densely dashed] at (0,1) (x1) {\texttt{$x_1$}};
   \node[main,color=black] at (1,1) (x2) {\texttt{$x_2$}};
       \node[main,color=black] at (0,0) (x3) {\texttt{$x_3$}};
       \node[main,fill=red!40!white,densely dashed] at (1,0) (x4) {\texttt{$x_4$}};
      \draw  (x1) -- (x2);
       \draw (x1) -- (x3);
      \draw (x2) -- (x3);
      \draw (x3) -- (x4);
   \end{tikzpicture}
       };
       
    \node[scale=.8] at (0,5.5){$\textup{PerfMatch}[c_1,\{x_1,x_4\}\cup \{x_2,x_3\}]$};
       \node[main] at (0,4) (8) {
   \begin{tikzpicture}[line width=1pt,node distance=14mm ,main/.style = {draw, circle},scale=1] 
   \node[main,fill=red!40!white,densely dashed] at (0,1) (x1) {\texttt{$x_1$}};
   \node[main,fill=red!40!white,densely dashed] at (1,1) (x2) {\texttt{$x_2$}};
       \node[main,fill=red!40!white,densely dashed] at (0,0) (x3) {\texttt{$x_3$}};
       \node[main,fill=red!40!white,densely dashed] at (1,0) (x4) {\texttt{$x_4$}};
      \draw (x1) -- (x2);
       \draw (x1) -- (x3);
      \draw (x2) -- (x3);
      \draw (x3) -- (x4);
   \end{tikzpicture}
   };
   
    \node[scale=3] at (2,4){$\cdot$};
   
    \node[scale=.8] at (4,5.5){$\textup{PerfMatch}[c_2,\{x_1,x_4\}\cup \{\}]$};
   \node[main] at (4,4) (8) {
   \begin{tikzpicture}[line width=1pt,node distance=14mm ,main/.style = {draw, circle},scale=1] 
   \node[main,fill=red!40!white,densely dashed] at (0,1) (x1) {\texttt{$x_1$}};
   \node[main,color=black] at (1,1) (x2) {\texttt{$x_2$}};
       \node[main,color=black] at (0,0) (x3) {\texttt{$x_3$}};
       \node[main,fill=red!40!white,densely dashed] at (1,0) (x4) {\texttt{$x_4$}};
      \draw (x1) -- (x2);
       \draw (x1) -- (x3);
      \draw (x2) -- (x3);
      \draw (x3) -- (x4);
   \end{tikzpicture}
   };
   
    \node[scale=.8] at (0,2.5){$\textup{PerfMatch}[c_1,\{x_1,x_4\}\cup \{x_3\}]$};
   \node[main] at (0,1) (8) {
   \begin{tikzpicture}[line width=1pt,node distance=14mm ,main/.style = {draw, circle},scale=1] 
   \node[main,fill=red!40!white,densely dashed] at (0,1) (x1) {\texttt{$x_1$}};
   \node[main,color=black] at (1,1) (x2) {\texttt{$x_2$}};
       \node[main,fill=red!40!white,densely dashed] at (0,0) (x3) {\texttt{$x_3$}};
       \node[main,fill=red!40!white,densely dashed] at (1,0) (x4) {\texttt{$x_4$}};
      \draw (x1) -- (x2);
       \draw (x1) -- (x3);
      \draw (x2) -- (x3);
      \draw (x3) -- (x4);
   \end{tikzpicture}
   };
   
    \node[scale=3] at (2,1){$\cdot$};
   
    \node[scale=.8] at (4,2.5){$\textup{PerfMatch}[c_2,\{x_1,x_4\}\cup \{x_2\}]$};
   \node[main] at (4,1) (8) {
   \begin{tikzpicture}[line width=1pt,node distance=14mm ,main/.style = {draw, circle},scale=1] 
   \node[main,fill=red!40!white,densely dashed] at (0,1) (x1) {\texttt{$x_1$}};
   \node[main,fill=red!40!white,densely dashed] at (1,1) (x2) {\texttt{$x_2$}};
       \node[main,color=black] at (0,0) (x3) {\texttt{$x_3$}};
       \node[main,fill=red!40!white,densely dashed] at (1,0) (x4) {\texttt{$x_4$}};
      \draw (x1) -- (x2);
       \draw (x1) -- (x3);
      \draw (x2) -- (x3);
      \draw (x3) -- (x4);
   \end{tikzpicture}
   };
   
    \node[scale=.8] at (0,-0.5){$\textup{PerfMatch}[c_1,\{x_1,x_4\}\cup \{x_2\}]$};
   \node[main] at (0,-2) (8) {
   \begin{tikzpicture}[line width=1pt,node distance=14mm ,main/.style = {draw, circle},scale=1] 
   \node[main,fill=red!40!white,densely dashed] at (0,1) (x1) {\texttt{$x_1$}};
   \node[main,fill=red!40!white,densely dashed] at (1,1) (x2) {\texttt{$x_2$}};
       \node[main,color=black] at (0,0) (x3) {\texttt{$x_3$}};
       \node[main,fill=red!40!white,densely dashed] at (1,0) (x4) {\texttt{$x_4$}};
      \draw (x1) -- (x2);
       \draw (x1) -- (x3);
      \draw (x2) -- (x3);
      \draw (x3) -- (x4);
   \end{tikzpicture}
   };
   
    \node[scale=3] at (2,-2){$\cdot$};
   
    \node[scale=.8] at (4,-0.5){$\textup{PerfMatch}[c_2,\{x_1,x_4\}\cup \{x_3\}]$};
   \node[main] at (4,-2) (8) {
   \begin{tikzpicture}[line width=1pt,node distance=14mm ,main/.style = {draw, circle},scale=1] 
   \node[main,fill=red!40!white,densely dashed] at (0,1) (x1) {\texttt{$x_1$}};
   \node[main,color=black] at (1,1) (x2) {\texttt{$x_2$}};
       \node[main,fill=red!40!white,densely dashed] at (0,0) (x3) {\texttt{$x_3$}};
       \node[main,fill=red!40!white,densely dashed] at (1,0) (x4) {\texttt{$x_4$}};
      \draw (x1) -- (x2);
       \draw (x1) -- (x3);
      \draw (x2) -- (x3);
      \draw (x3) -- (x4);
   \end{tikzpicture}
   };
   
    \node[scale=.8] at (0,-3.5){$\textup{PerfMatch}[c_1,\{x_1,x_4\}\cup \{\}]$};
   \node[main] at (0,-5) (8) {
   \begin{tikzpicture}[line width=1pt,node distance=14mm ,main/.style = {draw, circle},scale=1] 
   \node[main,fill=red!40!white,densely dashed] at (0,1) (x1) {\texttt{$x_1$}};
   \node[main,color=black] at (1,1) (x2) {\texttt{$x_2$}};
       \node[main,color=black] at (0,0) (x3) {\texttt{$x_3$}};
       \node[main,fill=red!40!white,densely dashed] at (1,0) (x4) {\texttt{$x_4$}};
      \draw (x1) -- (x2);
       \draw (x1) -- (x3);
      \draw (x2) -- (x3);
      \draw (x3) -- (x4);
   \end{tikzpicture}
   };
   
    \node[scale=3] at (2,-5){$\cdot$};
   
     \node[scale=.8] at (4,-3.5){$\textup{PerfMatch}[c_2,\{x_1,x_4\}\cup \{x_2,x_3\}]$};
   \node[main] at (4,-5) (8) {
   \begin{tikzpicture}[line width=1pt,node distance=14mm ,main/.style = {draw, circle},scale=1] 
   \node[main,fill=red!40!white,densely dashed] at (0,1) (x1) {\texttt{$x_1$}};
   \node[main,fill=red!40!white,densely dashed] at (1,1) (x2) {\texttt{$x_2$}};
       \node[main,fill=red!40!white,densely dashed] at (0,0) (x3) {\texttt{$x_3$}};
       \node[main,fill=red!40!white,densely dashed] at (1,0) (x4) {\texttt{$x_4$}};
      \draw (x1) -- (x2);
       \draw (x1) -- (x3);
      \draw (x2) -- (x3);
      \draw (x3) -- (x4);
   \end{tikzpicture}
   };
   
       \draw [decorate,decoration={brace,amplitude=10}] (-2,-5) -- (-2,4) node [black,midway,xshift=-0.6cm] {};
   
       %\draw [->](-2,3.5) -- (-3,2.5);
       %\draw [->](-2,1) -- (-3,0.5);
       %\draw [->](-2,-2) -- (-3,-1.5);
       %\draw [->](-2,-4.5) -- (-3,-3.5);
       %\draw [->](3.5,-2.5) -- (5,-3);
       %\draw [->](3.5,-5) -- (5,-4.5);
       %\draw [->](3.5,-7.5) -- (5,-6);
       %\node[text width=6cm] at (2,-2){$\DP[i,c]$};
       %\node[text width=6cm] at (5.5,-2){$\DP[i,c+\{x_4\to \texttt{gray/dotted}]$};
      
   \end{tikzpicture}}
			\label{fig:join_node}
			\caption{Example for join node.}
		\end{figure}
	\end{itemize}
	
\end{example}

\subsection{Counting Matchings}\label{sec:matchings}
\subsubsection{Bounded Pathwidth}\label{sec:pathwidth_matchings}
We will use a dynamic programming approach over the nice path decomposition of the given graph $G$. First we will define the dynamic programming state as follows:

\begin{definition}[Respectful Matchings]\label{def:respectful_matching}
Let us fix a nice path decomposition $\mathcal{P} = \{X_1, \dots X_r\}$ for the given graph $G$. For each $b \in [r]$ and each $M \subseteq X_b$, we define $\RM(b,M)$ as the set of all matchings $F$ in a $G_b^\downarrow \setminus M$ such that each matching edge $uv \in F$ has at least one point in $G_b^\downarrow \setminus X_b$: $u \notin X_b$ or $v \notin X_b$ and each point from $X_b \setminus M$ is covered by $F$.
\end{definition}
\begin{definition}[Dynamic Programming State for Matchings]\label{def:dp_matching}
Let us fix a nice path decomposition $\mathcal{P} = \{X_1, \dots X_r\}$ for the given graph $G$. For each $b \in [r]$ and each $M \subseteq X_b$, we define $\ma[b, M]$ to be the number of elements in the set $\RM(b,M)$.
\end{definition}

\begin{remark}
Notice that if $X_r = \varnothing$ is the root node, then $G_{r}^\downarrow = G$, and all matchings of $G$ are in one-to-one correspondence with respectful matchings $\RM(r,\varnothing)$, which implies $\ma [r, \varnothing]$ is the total number of matchings of $G$.
% If $X_l = \varnothing$ is a leaf bag, only $(X_l, \varnothing)$-respectful matching is an empty 
% matching, which means: 
% \[
% \dpt[X_l, \varnothing] = 1.
% \]
\end{remark}

We now provide a bottom-up approach for filling up the dynamic table. The dynamic table relations depend on the type of the bag. We need to consider three cases when $X_b$ is: leaf node, introduce node and forget node respectively.


\begin{itemize}
\item \textbf{Leaf node:} If $X_l$ is a leaf bag then $\RM(b,M)$ contains only empty matching as $X_l = \varnothing$. Therefore, $\ma[l,\varnothing] = 1$.
\item \textbf{Introduce node:} Let $X_b$ be the introduce bag such that $X_b = X_c \cup \{v\}$. Then the dynamic programming state corresponding to $(b,M)$ satisfies the following recurrence relations: 
\[
\ma [b,M] =
\begin{cases}
\ma [c,M\setminus v], &v \in M,\\
0, &v \not\in M.
\end{cases}
\]
In order to derive the above recurrence relations, we have to consider two possibilities for $v$.
\begin{enumerate}
\item If $v \in M$, then by Definition \ref{def:respectful_matching}, all the elements of the set $\RM(b,M)$ are in one-to-one correspondence with all the elements of the set $\RM(c, M \setminus v)$. 
\item If $v \notin M$, then for every matching $F \in \RP(b,M)$, $v$ should be covered by some edge $vw \in F$. By Lemma \ref{intseplemma}, $w \in X_{c} \subset X_b$, which contradicts the Definition \ref{def:respectful_matching}. Therefore, $\ma[b,m] = 0$ for this case, as there are no matchings satisfying the given conditions.
\end{enumerate}
This transition from $X_c$ to $X_b$ can be computed in $\bigO(1)$ time. 
\item \textbf{Forget node:} Let $X_b$ be the forget node such that $X_b = X_c \setminus \{v\}$. Then the dynamic programming state corresponding to $(b,M)$ satisfies the following recurrence relation: 

% \[
% \pma[b,M]=
% 	\pma[c, M] + \sum_{u \in X_b\setminus M:\ uv\in E(G)}\pma[c, M \cup \{v,u\}].
% \]

\begin{align*} 
\ma[b,M]= \ma[c, M] &+  \ma[c, M \cup \{v\}] \\
&+ \sum_{u \in X_b\setminus M:\ uv\in E(G)}\ma[c, M \cup \{v,u\}]
\end{align*}
% \[
% \ma[b,M]=
% \ma[c, M] + \ma[c, M \cup \{v\}] + \sum_{u \in X_b\setminus M:\ uv\in E(G)}\ma[c, M \cup \{v,u\}]
% \]
In order to derive the above recurrence relation, let $F$ be a matching from the set $\RM(b,M)$. Note that $v \notin M$, as $M \subseteq X_b$. There are two possibilities for $v$ i.e., either it will be covered by some edge in $F$ or it remains unmatched. For the latter case, all such matchings $F$ for which $v$ remains unmatched are in one-to-one correpondence with all matchings of the set $\RM(c,M\cup\{v\}$, and number of such matchings correspond to the second term $\ma[c, M \cup \{v\}]$ of the recurrsion. Now let us consider the second case, where $v$ will be matched by some edge $uv \in F$. Now there are again two possible cases to consider i.e., either $u \notin X_b$ or $u \in X_b$. For the former case when $u \notin X_b$, it is clear that there is a bijection between all such matchings $F$ and all elements of $\RM(c,M)$. Therefore, number of such matchings correspond to the first term $\ma[c,M]$ of the recurrence relation. Now let's consider the latter case when $u \in X_b$, then the matching $F \setminus uv \in \RM(c,M M \cup \{u, v\})$, and number of such matchings correspond to the last term $\sum_{u \in X_b\setminus M:\ uv\in E(G)}\ma[c, M \cup \{v,u\}]$ of the recurrence relation.


% To prove that consider any $(X_b, M)$-respectful matching $F$. If $v$ unmached, $v \notin V(F)$ and $F$ is $(X_c, M \cup \{v\})$-respectful. If $v$ is matched with some vertex $u$, we have two cases:  either $w \notin X_b$ $w \in X_b$. In a first case the mathching $F$ is $(X_c, M)$-respectful, and this case corresponds to the term $\dpt[c, M \cup \{v\}]$. If $w \in X_b$, then exist edge $uv$ and matching $F \setminus uv$ is $(X_c, M \cup \{v, u\})$-respectful, which corresponds to the sum of $\dpt[c, M\cup \{v, u\}]$ terms. All correspondences is one-to-one.  


% To prove that consider any $(X_b, M)$-respectful matching $F$. Since $v \notin M$, $v$ should be matched with some point, say, $u$. Then if $u\notin X_b$, matching $F$ is $(X_{c}, M)$-respectful, and number of such cases corresponds to the first term. If $u \in X_b$ then mathching $F \setminus uv$ is $(X_{c}, M \cup \{u, v\})$ respectful and correspondence is one-to-one.  

This transition to $X_b$ can be computed in $\bigO(\lvert X_b\rvert) = \bigO(\pw(G))$ time.
\end{itemize}


\begin{proposition}\label{prop:comp_pw_mat}
The time complexity for finding the total number of matchings for a graph $G = (V,E)$, with $\lvert V \rvert = n$ and pathwidth $\pw$, using the above algorithm is $\bigO(n\cdot \poly(\pw) \cdot2^{\pw})$.
% The final complexity is $\bigO(n\cdot \textup{pw}\cdot2^{\textup{pw}})$
\end{proposition}
\begin{proof}
Same proof as Proposition \ref{prop:comp_pw_perfmat}.
% For each bag, we have atmost $2^{\pw}$ dynamic programming states, and for each state we spend atmost $\bigO(\poly(\pw))$ time. It is known that there are atmost $\bigO(n)$ bags for any nice path decomposition of $G$ \cite[Chapter 7]{cygan2015parameterized}. Therefore, we spend atmost $\bigO(n\cdot \poly(\pw) \cdot2^{\pw})$ time to completely fill the dymamic programming table.
\end{proof}

\subsubsection{Bounded Treewidth}\label{subsec:matching_treewidth}
We will use a dynamic programming approach over the nice tree decomposition of the given graph $G$. The definitions of respectful matchings and dynamic programming states used here are same as Section \ref{sec:pathwidth_matchings}. Let $\mathcal{T} = \{T, \{X_t\}_{t \in V(T)}\}$ be a nice tree decomposition of the graph $G = (V,E)$ under consideration.

% We will use a dynamic programming over the nice tree decomposition. Definitions of dynamic programming and respectfulness have only cosmetic differences with path decomposition case. Let us fix a nice tree decomposition $\mathcal{T} = \{T, \{X_t\}_{t \in V(T)}\}$. For each $b \in T$ and each $M \subseteq X_b$ we will define $\dpt[b, M]$ as a number of perfect matchings in a $G_b^\downarrow \setminus M$ such that each matching edge $uv$ has at least on point in $G_b^\downarrow \setminus X_b$: $u \notin X_b$ or $v \notin X_b$. We call such matchings \textit{respectful} to the $(X_b, M)$.


% \begin{remark}

% \end{remark}
% If $X_l = \varnothing$ is a leaf bag, we have only empty $(X_l, \varnothing)$-respectful matching, which means 
% \[
% 	\dpt[X_l, \varnothing] = 1.
% \]

% Also for the root bag $X_r$, $\dpt[X_r, \varnothing]$ contains number of perfect matchings in a whole graph $G$.


% Cases of introduce and forget nodes repet verbatim. We need only consider merge case now. 
We now provide a bottom-up approach for filling up the dynamic table. The dynamic table relations depend on the type of the bag. There are four possible cases for the type of bags, i.e., when $X_b$ is: leaf node, introduce node, forget node and join node respectively. The reccurrence relations for leaf node, introduce node and forget node remains the same as Section \ref{sec:pathwidth_matchings}. Therefore, we only need to consider the case of join node which is described as follows:
\begin{itemize}
	\item \textbf{Join node:} Let $X_b$ be the join node with $X_{c_1}$ and $X_{c_2}$ as its children. Then the dynamic programming state corresponding to $(b,M)$ will satisfy the following recurrence relation:
\[
\ma[b,M] = \sum_{H_1 \sqcup H_2 = X_b \setminus M}\ma[c_1,M\cup H_2]\cdot \ma[c_2, M\cup H_1]
% \\
% &=\sum_{H \subseteq X_b \setminus M}\pma[c_1,M\cup H]\cdot \pma[c_2, X_b \setminus H],\\
\]

where $H_1 \sqcup H_2 = X_b \setminus M$ means $H_1 \cup H_2 = X_b \setminus M$ and $H_1 \cap H_2 = \varnothing$. 
% This two formulas yields same result, but first easier to analyze and second easier to implement.  
In order to derive the above recurrence relation, let $F$ be a matching from the set $\RM(b,M)$. By Lemma \ref{joinseplemma}, $F$ doesn't have any edges between $G_{c_1}^\downarrow \setminus X_b$ and $G_{c_2}^\downarrow \setminus X_b$. Therefore, it can be split into two matchings, $F_1 := E(G_{c_1}^\downarrow) \cap F$ and $F_2 := E(G_{c_2}^\downarrow) \cap F$. Let $H_1 := V(F_1) \cap X_b$ and $H_2 := V(F_2) \cap X_b$. It is clear from the Definition \ref{def:respectful_matching}, that $F_1 \in \RM(c_1,M \cup H_2)$ and $F_2 \in \RM(c_2,M \cup H_1)$. If we choose $H_1$ and $H_2$ such that $H_1 \sqcup H_2 = X_b \setminus M$, then for all such matchings $F_1 \in \RM(c_1,M\cup H_2)$ and $F_2 \in \RM(c_2, M\cup H_1)$, we get a matching $ F = F_1 \cup F_2$, such that $F \in \RM(b,M)$. This leads to our recurrence relation.


% Consider arbitrary $(X_b, M)$-respectful matching $F$. By lemma \ref{joinseplemma} $F$ do not have edges between $G_{c_1}^\downarrow \setminus X_b$ and $G_{c_2}^\downarrow \setminus X_b$. Let us split this matching $F$ into $F_1 := E(G_{c_1}^\downarrow) \cap F$, $F_2 := E(G_{c_2}^\downarrow) \cap F$, also let $H_1 := V(F_1) \cap X_b$, $H_2 := V(F_2) \cap X_b$. Observe that $F_1$ is $(X_{c_1}, M \cup H_2)$-respectful and $F_2$ is $(X_{c_2}, M \cup H_1)$-respectful. Notice that if $H_1, H_2$ are chosen, $F_1, F_2$ acts independently, so when $H_1, H_2$ fixed such matching could be counted as $\dpt[c_1, M \cup H_2] \cdot \dpt[c_2, M \cup H_1]$, which leads to our formula. 

\end{itemize}

\begin{proposition}\label{prop:comp_tw_match}
The time complexity for finding the total number of matchings for a graph $G = (V,E)$, with $\lvert V \rvert = n$ and treewidth $\twi$, using the above algorithm is $\bigO(n\cdot \poly(\twi)\cdot3^{\twi})$.
% With this, the final complexity of the algorithm increases to .
\end{proposition}
\begin{proof}
Same proof as Proposition \ref{prop:comp_tw_perfmat}.
% The transition for a join node is the most expensive operation for this algorithm, the rest of the cases are similar to the dynamic programming on nice path decomposition in Section \ref{subsec:perfect_pathwidth}.
% Note that the recurrence relation for the dynamic programming state of join node can be rewritten as:
% \[
% \sum_{H \subseteq X_b \setminus M}\pma[c_1,M\cup H]\cdot \pma[c_2, X_b \setminus H]
% \]

% For each bag $X_b$ and for each $M \subseteq X_b$, we spend atmost $\bigO(\poly(\twi)\cdot 2^{\lvert X_B \rvert - \lvert M \rvert})$ time. By using binomial theorem, the total time spend for each bag $X_b$ for all possible subsets $M$ is $\bigO(\poly(\twi)\cdot 3^{\lvert X_B \rvert})$. It is known that there are atmost $\bigO(n)$ bags for any nice path decomposition of $G$ \cite[Chapter 7]{cygan2015parameterized}. Therefore, we spend atmost $\bigO(n\cdot \poly(\twi)\cdot3^{\twi})$ time to completely fill the dymamic programming table.
\end{proof}


% \subsection{Bounded Treewidth}

% Again we will use a dynamic programming over the nice tree decomposition in a bottom-up manner. Let us fix a nice tree decomposition $\mathcal{T} = \{T, \{X_t\}_{t \in V(T)}\}$. For each node $b$ and each subset of a bag $M \subset X_b$ we define $\dpt[b, M]$ as a number of matchings in a $G_b^\downarrow \setminus M$ such that each matching edge $uv$ has at least on point in $G_b^\downarrow \setminus X_b$: $u \notin X_b$ or $v \notin X_b$ and each point from $X_b \setminus M$ is covered. We call such matchings \textit{respectful} to the $(X_b, M)$.


% % If $X_l = \varnothing$ is a leaf bag, we have only empty $(X_l, \varnothing)$-respectful matching, which means 
% % \[
% % \dpt[X_l, \varnothing] = 1.
% % \]

% % Also for the root bag $X_r$, $\dpt[X_r, \varnothing]$ contains number of matchings in a whole graph $G$.

% % Now we need to consider three cases of tree decomposition nodes: introduce node, forget node and merge node. 

% % \textbf{Introduce case.} Let $X_b$ is an introduce bag which introduces vertex $v$, let its child is a bag $X_c$, $X_b = X_c \cup \{v\}$. Then  
% % \[
% % \dpt[b,M] =
% % \begin{cases}
% % \dpt[c,M\setminus v], &v \in M,\\
% % 0, &v \not\in M.
% % \end{cases}
% % \]
% % Indeed, if $v \in M$, we need to count respectful matchings to the $(X_b, M)$, but that matchings in one-to-one correspondance with respectful matchings to $(X_c, M \setminus v)$. On the other hand, if $v \notin M$, consider any $(X_b, M)$-respectful matching. In that matching $v$ should be covered by some edge $vw$. By lemma \ref{intseplemma}, $w \in X_{c} \subset X_b$, which contradicts with definition of of being respectful. 

% % This transition can be calculated in $\bigO(1)$ time. 

% % \textbf{Forget case.} Consider now a node $X_b$. Let $v$ be the vertex erased in $X_b$ and $X_c$ is a child of the $X_b$. Then

% % \[
% % \dpt[b,M]=
% % \dpt[c, M] + \dpt[c, M \cup \{v\}] + \sum_{u \in X_b\setminus M:\ uv\in E(G)}\dpt[c, M \cup \{v,u\}].
% % \]

% % To prove that consider any $(X_b, M)$-respectful matching $F$. If $v$ unmached, $v \notin V(F)$ and $F$ is $(X_c, M \cup \{v\})$-respectful. If $v$ is matched with some vertex $u$, we have two cases:  either $w \notin X_b$ $w \in X_b$. In a first case the mathching $F$ is $(X_c, M)$-respectful, and this case corresponds to the term $\dpt[c, M \cup \{v\}]$. If $w \in X_b$, then exist edge $uv$ and matching $F \setminus uv$ is $(X_c, M \cup \{v, u\})$-respectful, which corresponds to the sum of $\dpt[c, M\cup \{v, u\}]$ terms. All correspondences is one-to-one.  

% % This transition can be calculated in $\bigO(\lvert X_b\rvert) = \bigO(\textup{tw}(G))$ time.

% \textbf{Merge case}. Consinder merge node $X_b$ and two its children $X_{c_1}$ and $X_{c_2}$, $X_b = X_{c_1} = X_{c_2}$.
% We claim that
% \[
% \begin{split}
% \dpt[b,M] &= \sum_{H_1 \sqcup H_2 = X_b \setminus M}\dpt[c_1,M\cup H_2]\cdot \dpt[c_2, M\cup H_1]\\
% &=\sum_{H \subseteq X_b \setminus M}\dpt[c_1,M\cup H]\cdot \dpt[c_2, X_b \setminus H],\\
% \end{split}
% \]

% where $H_1 \sqcup H_2 = X_b \setminus M$ means $H_1 \cup H_2 = X_b \setminus M$ and $H_1 \cap H_2 = \varnothing$.  

% Consider arbitrary $(X_b, M)$-respectful matching $F$. By lemma \ref{joinseplemma} $F$ do not have edges between $G_{c_1}^\downarrow \setminus X_b$ and $G_{c_2}^\downarrow \setminus X_b$. Let us split this matching $F$ into $F_1 := E(G_{c_1}^\downarrow)\cap F$, $F_2 := E(G_{c_2}^\downarrow) \cap F$, also let $H_1 := V(F_1) \cap X_b$, $H_2 := V(F_2) \cap X_b$. Observe that $F_1$ is $(X_{c_1}, M \cup H_2)$-respectful and $F_2$ is $(X_{c_2}, M \cup H_1)$-respectful. Notice that if $H_1, H_2$ are chosen, $F_1, F_2$ acts independently, so when $H_1, H_2$ fixed, such matchings could be counted as $\dpt[c_1, M \cup H_2] \cdot \dpt[c_2, M \cup H_1]$, which leads to our formula. 


% \begin{proposition}
% The final complexity is $\bigO(n\cdot \textup{tw}\cdot3^{\textup{tw}})$

% \end{proposition} 

\subsection{Counting Independent Sets}\label{sec:independent_sets}
\subsubsection{Bounded Pathwidth}\label{sec:pw_indep} 
We will use a dynamic programming approach over the nice path decomposition of the given graph $G$. First we will define the dynamic programming state as follows:

\begin{definition}[Respectful Independent Sets]\label{def:respectful_independent_sets}
Let us fix a nice path decomposition $\mathcal{P} = \{X_1, \dots X_r\}$ for the given graph $G$. For each $b \in [r]$ and each $M \subseteq X_b$, we define $\RI(b,M)$ as the set of all independent sets $I$ in a $G_b^\downarrow $ such that each $I \cap X_b = M$.
\end{definition}
\begin{definition}[Dynamic Programming State for Independent Sets]\label{def:dp_indpendent_set}
Let us fix a nice path decomposition $\mathcal{P} = \{X_1, \dots X_r\}$ for the given graph $G$. For each $b \in [r]$ and each $M \subseteq X_b$, we define $\ind[b, M]$ to be the number of elements in the set $\RI(b,M)$.
\end{definition}

\begin{remark}
Notice that if $X_r = \varnothing$ is the root node, then $G_{r}^\downarrow = G$, and independent sets of $G$ are in one-to-one correspondence with respectful independent sets $\RP(r,\varnothing)$, which implies $\ind [r, \varnothing]$ is the total number of independent sets of $G$.
% If $X_l = \varnothing$ is a leaf bag, only $(X_l, \varnothing)$-respectful matching is an empty 
% matching, which means: 
% \[
% \dpt[X_l, \varnothing] = 1.
% \]
\end{remark}

We now provide a bottom-up approach for filling up the dynamic table. The dynamic table relations depend on the type of the bag. We need to consider three cases when $X_b$ is: leaf node, introduce node and forget node respectively.


\begin{itemize}
\item \textbf{Leaf node:} If $X_l$ is a leaf bag then $\RI(b,M)$ contains only empty independent set as $X_l = \varnothing$. Therefore, $\ind[l,\varnothing] = 1$.
\item \textbf{Introduce node:} Let $X_b$ be the introduce bag such that $X_b = X_c \cup \{v\}$. Then the dynamic programming state corresponding to $(b,M)$ satisfies the following recurrence relations: 
\[
\ind [b,M] =
\begin{cases}
\ind [c,M], &v \notin M,\\
\ind [c,M\setminus v], &v \in M \textup{ and } N(v) \cap M \ne \varnothing, \\
0, &v \in M \textup{ and } N(v) \cap M = \varnothing.
\end{cases}
\]
In order to derive the above recurrence relations, we have to consider two possibilities for $v$.
\begin{enumerate}
\item If $v \in M$, now there are again two subcases to consider i.e., $N(v)\cap M \ne \varnothing$ and $N(v)\cap M = \varnothing$. For the former subcase, there is a one-to-one correspondence with all such independent sets $I$ with elements of the set $\RI(c,M\setminus v)$. Therefore, $\ind[b,M] = \ind[c,M\setminus v]$ for this subcase. For the latter subcase, there cannot exist any independent set $I$ in $\RI(b,M)$, such that it contains both $v$ and its neighbours. Therefore, $\ind[b,M] = 0$ for this subcase. 

% then by Definition \ref{def:respectful_independent_sets}, all the elements of the set $\RI(b,M)$ are in one-to-one correspondence with all the elements of the set $\RP(c, M \setminus v)$. 
\item If $v \notin M$, then there is a bijection between $\RI(b,M)$ and $\RI(c,M)$. Therefore, number of indepedent sets $\ind[b,M]$ is equal to $\ind[c,M]$. 
\end{enumerate}
This transition from $X_c$ to $X_b$ can be computed in $\bigO(1)$ time. 
\item \textbf{Forget node:} Let $X_b$ be the forget node such that $X_b = X_c \setminus \{v\}$. Then the dynamic programming state corresponding to $(b,M)$ satisfies the following recurrence relation: 
\[
\ind [b,M] = \ind[c,M] + \ind[c,M\cup \{v\}] 
\]

In order to derive the above recurrence relation, let $I$ be an independent set from the set $\RI(b,M)$. Now there are two possible cases i.e., either $v \notin I$ or $v \in I$. For the former case when $v \notin I$, there is a bijection for all such independent sets $I$ to all the elements of the set $\RI(c,M)$. Therefore, number of such independent sets is equal to $\ind[c,M]$, and this corresponds to the first term of the recurrence relation. For the latter case when $v \in I$, there is a bijection between all such independent sets $I$ to all the elements of the set $\RI[c,M\cup \{v\}]$. Therefore, number of such independent sets is equal to $\ind[c,M\cup \{v\}]$, which is the second term of the recurrence relation.

This transition to $X_b$ can be computed in $\bigO(1)$ time.
\end{itemize}



\begin{proposition}\label{prop:comp_pw_ind}
The time complexity for finding the total number of independent sets for a graph $G = (V,E)$, with $\lvert V \rvert = n$ and pathwidth $\pw$, using the above algorithm is $\bigO(n\cdot \poly(\pw) \cdot2^{\pw})$.
% The final complexity is $\bigO(n\cdot \textup{pw}\cdot2^{\textup{pw}})$
\end{proposition}
\begin{proof}
Proof same as Proposition \ref{prop:comp_pw_perfmat}.
% For each bag, we have atmost $2^{\pw}$ dynamic programming states, and for each state we spend atmost $\bigO(\poly(\pw))$ time. It is known that there are atmost $\bigO(n)$ bags for any nice path decomposition of $G$ \cite[Chapter 7]{cygan2015parameterized}. Therefore, we spend atmost $\bigO(n\cdot \poly(\pw) \cdot2^{\pw})$ time to completely fill the dymamic programming table.
\end{proof}

\subsubsection{Bounded Treewidth}\label{sec:tw_indep}
We will use a dynamic programming approach over the nice tree decomposition of the given graph $G$. The definitions of respectful matchings and dynamic programming states used here are same as Section \ref{sec:pw_indep}. Let $\mathcal{T} = \{T, \{X_t\}_{t \in V(T)}\}$ be a nice tree decomposition of the graph $G = (V,E)$ under consideration.

Before we procced with the dynamic progamming algorithm, we will first prove the following technical lemma which will be required for the derivation of recurrence relations. 
\begin{lemma}\label{lem:technical_lemma_ind_join_node}
Let $X_b$ be the join node with $X_{c_1}$ and $X_{c_2}$ as its children, then $\lvert \RI(b,M)\rvert  = \lvert \RI(c_1,M) \rvert \cdot \lvert \RI(c_2,M) \rvert$.
\end{lemma}

\begin{proof}
Let $f$ be the map from $\RI(b,M)$ to $\RI(c_1,M) \times \RI(c_2,M)$ such that $f$ is defined as follows:

\[
	f(I) \mapsto (I_1,I_2)
\]
where $I_1 = I \cap V(G_{c_1}^\downarrow)$ and $I_2 = I \cap V(G_{c_2}^\downarrow)$ then we will show that $f$ is a bijection.

First, we will show that $f$ is injective. Let $I,J \in \RI(b,M)$ such that $f(I) = f(J)$, then we need to show that $I = J$. Let us assume that $I \ne J$, this implies that there exists $u \in I$, such that $u \notin J$. As $f(I) = f(J)$, we get $(I_1,I_2) = (J_1,J_2)$, this implies $I = J$, as $I = I_1 \cup I_2$ and $J = J_1 \cup J_2$. Therefore, we get a contradiction to our initial assumption that $I \ne J$, this implies that $f$ is injective.

Now, we will show that $f$ is surjective. Let $(J,K) \in \RI(c_1,M) \times \RI(c_2,M)$ then we need to show that there exists an $I \in \RI(b,M)$ such that $f(I) = (J,K)$. Let $I$ be $J \cup K$. In order to complete the proof, we need to show that $I \in \RI(b,M)$ and $f(I) = (J,K)$. For all $u,v \in I$, we need to show that there doesn't exist any edge between $u$ and $v$. We need to consider the following cases for $u$ and $v$:
\begin{itemize}
	\item $u,v \in J$ or $u,v \in K$: In both the cases, there doesn't exist any edge between $u$ and $v$ as both $J$ and $K$ are independent sets of $G_{c_1}^\downarrow$ and $G_{c_2}^\downarrow$ respectively.
	\item $u \in J \setminus J \cap K$ and $v \in K \setminus J \cap K$: We know using Lemma \ref{joinseplemma}, that there is no edge between $V(G_{c_1}^\downarrow) \setminus X_b$ and $V(G_{c_2}^\downarrow) \setminus X_b$. Therefore, there is no edge between $u$ and $v$.
\end{itemize}
This completes the proof that $I$ is an independent set of $G_{b}^\downarrow$. As $I = J \cup K$ and $J\cap X_b = K \cap X_b = M$, this implies that $I \cap X_b = M$. Therefore, this completes the proof that $I \in \RI(b,M)$ and shows that $f$ is a surjective map.
\end{proof}

We now provide a bottom-up approach for filling up the dynamic table. The dynamic table relations depend on the type of the bag. There are four possible cases for the type of bags, i.e., when $X_b$ is: leaf node, introduce node, forget node and join node respectively. The reccurrence relations for leaf node, introduce node and forget node remains the same as Section \ref{sec:pw_indep}. Therefore, we only need to consider the case of join node which is described as follows:
\begin{itemize}
	\item \textbf{Join node:} Let $X_b$ be the join node with $X_{c_1}$ and $X_{c_2}$ as its children. Then the dynamic programming state corresponding to $(b,M)$ will satisfy the following recurrence relation:
	\[
		\ind[b,M] = \ind[c_1,M] \cdot \ind[c_2,M]
	\]
This recurrence relation follows from Definition \ref{def:dp_indpendent_set} and Lemma \ref{lem:technical_lemma_ind_join_node}.
\end{itemize}
This transition to $X_b$ can be computed in $\bigO(1)$ time.


\begin{proposition}\label{prop:ind_tw_match}
The time complexity for finding the total number of independent sets for a graph $G = (V,E)$, with $\lvert V \rvert = n$ and treewidth $\twi$, using the above algorithm is $\bigO(n\cdot \poly(\twi)\cdot2^{\twi})$.
% With this, the final complexity of the algorithm increases to .
\end{proposition}
\begin{proof}
Same proof as Proposition \ref{prop:comp_pw_perfmat}.
\end{proof}




\subsection{Graph Entropy based on Matchings}\label{sec:entropy_matchings}
In this section, we will introduce dynamic programming algorithms for computing matchings of all sizes for the given graph $G$ for bounded pathwidth and bounded treewidth respectively. We need to slightly adapt the dynamic programming states defined in Section \ref{sec:matchings}, in order to count matchings of all sizes. The derivation of the corresponding recurrence relations of dynamic programming states remain almost the same as Section \ref{sec:matchings}. So, we will provide proofs only for the cases which are significantly different from the earlier sections.




\subsubsection{Bounded Pathwidth}\label{sec:entropy_matchings_pw}
\begin{definition}[Respectful Matchings of size $k$]\label{def:respectful_matching_size_k}
Let us fix a nice path decomposition $\mathcal{P} = \{X_1, \dots X_r\}$ for the given graph $G$. For each $b \in [r]$ and each $M \subseteq X_b$, we define $\RM(b,k,M)$ as the set of all matchings $F$ in a $G_b^\downarrow \setminus M$ such that $\lvert F \rvert = k$ each matching edge $uv \in F$ has at least one point in $G_b^\downarrow \setminus X_b$: $u \notin X_b$ or $v \notin X_b$ and each point from $X_b \setminus M$ is covered by $F$.
\end{definition}
\begin{definition}[Dynamic Programming State for Matchings of size $k$]\label{def:dp_matching_size_k}
Let us fix a nice path decomposition $\mathcal{P} = \{X_1, \dots X_r\}$ for the given graph $G$. For each $b \in [r]$ and each $M \subseteq X_b$, we define $\ma[b,k,M]$ to be the number of elements in the set $\RM(b,k,M)$.
\end{definition}

\begin{remark}
Notice that if $X_r = \varnothing$ is the root node, then $G_{r}^\downarrow = G$, and all matchings of $G$ are in one-to-one correspondence with respectful matchings $\RM(r,k,\varnothing)$, which implies $\ma [r, k,\varnothing]$ is the total number of matchings of size $k$ of $G$.
% If $X_l = \varnothing$ is a leaf bag, only $(X_l, \varnothing)$-respectful matching is an empty 
% matching, which means: 
% \[
% \dpt[X_l, \varnothing] = 1.
% \]
\end{remark}

We now provide a bottom-up approach for filling up the dynamic table. The dynamic table relations depend on the type of the bag. We need to consider three cases when $X_b$ is: leaf node, introduce node and forget node respectively.


\begin{itemize}
\item \textbf{Leaf node:} If $X_l$ is a leaf bag then $\RM(b,k,M)$ contains only empty matching as $X_l = \varnothing$. Therefore, $\ma[l,0,\varnothing] = 1$ and for any other $k > 0$ $\ma[l,k,\varnothing] = 0$.
\item \textbf{Introduce node:} Let $X_b$ be the introduce bag such that $X_b = X_c \cup \{v\}$. Then the dynamic programming state corresponding to $(b,k,M)$ satisfies the following recurrence relations: 
\[
\ma [b,k,M] =
\begin{cases}
\ma [c,k,M\setminus v], &v \in M,\\
0, &v \not\in M.
\end{cases}
\]

This transition from $X_c$ to $X_b$ can be computed in $\bigO(1)$ time. 
\item \textbf{Forget node:} Let $X_b$ be the forget node such that $X_b = X_c \setminus \{v\}$. Then the dynamic programming state corresponding to $(b,k,M)$ satisfies the following recurrence relation: 

\begin{align*} 
\ma[b,k,M]= \ma[c,k,M] &+  \ma[c,k,M \cup \{v\}] \\
&+ \sum_{u \in X_b\setminus M:\ uv\in E(G)}\ma[c,k-1, M \cup \{v,u\}]
\end{align*}

This transition to $X_b$ can be computed in $\bigO(\lvert X_b\rvert) = \bigO(\pw(G))$ time.
\end{itemize}

\begin{proposition}\label{prop:comp_pw_matchings_size_k}
The time complexity for finding the total number of matchings of all possible sizes for a graph $G = (V,E)$, with $\lvert V \rvert = n$ and pathwidth $\pw$, using the above algorithm is $\bigO(n^2\cdot \poly(\pw) \cdot2^{\pw})$.
% The final complexity is $\bigO(n\cdot \textup{pw}\cdot2^{\textup{pw}})$
\end{proposition}
\begin{proof}
Proof same as Proposition \ref{prop:comp_pw_perfmat}.
\end{proof}
\subsubsection{Bounded Treewidth}\label{sec:entropy_matchings_tw}
We will use a dynamic programming approach over the nice tree decomposition of the given graph $G$. The definitions of respectful matchings of fixed size and dynamic programming states used here are same as Section \ref{sec:entropy_matchings_pw}. Let $\mathcal{T} = \{T, \{X_t\}_{t \in V(T)}\}$ be a nice tree decomposition of the graph $G = (V,E)$ under consideration.

We now provide a bottom-up approach for filling up the dynamic table. The dynamic table relations depend on the type of the bag. There are four possible cases for the type of bags, i.e., when $X_b$ is: leaf node, introduce node, forget node and join node respectively. The reccurrence relations for leaf node, introduce node and forget node remains the same as Section \ref{sec:entropy_matchings_pw}. Therefore, we only need to consider the case of join node which is described as follows:
\begin{itemize}
	\item \textbf{Join node:} Let $X_b$ be the join node with $X_{c_1}$ and $X_{c_2}$ as its children. Let us define $n_b = \lvert V(G_b^{\downarrow}) \rvert$ and also assume without loss of generality that $n_{c_1} \leq n_{c_2}$. 
	Then the dynamic programming state corresponding to $(b,k,M)$ will satisfy the following recurrence relation:
\begin{align*}
&\ma[b,k,M] = \\
&\sum_{0\leq k_{c_1}\leq n_{c_1}} \sum_{H_1 \sqcup H_2 = X_b \setminus M}\ma[c_1,k_{c_1},M\cup H_2]\cdot \ma[c_2,k-k_{c_1}, M\cup H_1]
\end{align*}
where $H_1 \sqcup H_2 = X_b \setminus M$ means $H_1 \cup H_2 = X_b \setminus M$ and $H_1 \cap H_2 = \varnothing$. 
\end{itemize}
In order to determine the time taken to compute the above transition from $X_c$ to $X_b$, we need the following definition and lemma.


\begin{definition}[Smaller Subtrees]\label{def:smaller_subtree}
Let $\mathcal{T} = \{T, \{X_t\}_{t \in V(T)}\}$ be a nice tree decomposition of the graph $G = (V,E)$. For all join nodes $X_b$ of $\mathcal{T}$, let $X_{c_1}$ and $X_{c_2}$ be its children, 
and without loss of generality assume $n_{c_1} \leq n_{c_2}$. We define the set of smaller subtrees as the set of all such $T_{c_1}$'s. More precisely, it can be defined as follows:

\[
\SSt(\mathcal{T}) = \{ \Schild(X_b): \forall X_b \in \JoinN(\mathcal{T})\}
\]
where $\Schild$ for any join node $X_b \in \mathcal{T}$ with $X_{c_1}$ and $X_{c_2}$ as its children, is defined as follows:

\[
\Schild(X_b) = \{T_{c_1}: n_{c_1} \leq n_{c_2}    \}
\]
\end{definition}
\begin{lemma}\label{lem:technical_lemma_for_matching_entropy}
Let $\mathcal{T} = \{T, \{X_t\}_{t \in V(T)}\}$ be a nice tree decomposition of the graph $G = (V,E)$. Then $\sum n_a$ for all smaller subtrees $T_a$ in $\SSt(\mathcal{T})$ is at most $\bigO(n\cdot \log n)$.

% Then $ \sum_{T_a \in \SSt(\mathcal{T})} n_a$ is atmost $\bigO(n\cdot log(n))$.
\end{lemma}
\begin{proof}
Let $x_v$ be the number of times any vertex $v$ appears in the smallest subtree of some node with more than one child.

\[\sum_{T_a \in \SSt(\mathcal{T})}n_a = \sum_{v\in T} x_v \]

Observe that $v$ can have at most $\log n$ ancestors with degree more than two in which it is part of the smallest subtree.

This can be shown by keeping track of the current subtree and iterating through the ancestors of $v$. Every time $v$ is in the smallest subtree of a node with more than one child, the size of such subtree doubles, and this can happen at most $\log n$ times.

Therefore, we get the following bound on number of vertices for all smaller subtrees:
\[
\sum_{T_a \in \SSt(\mathcal{T})}n_a = \sum_{v\in T} x_v \leq \sum_{v\in T} \log n \leq n \log n
\]
\end{proof}
\begin{proposition}\label{prop:comp_tw_matchings_size_k}
The time complexity for finding the total number of matchings of all possible sizes for a graph $G = (V,E)$, with $\lvert V \rvert = n$ and treewidth $\twi$, using the above algorithm is $\bigO(n^2 \cdot \log n  \cdot \poly(\twi)\cdot3^{\twi})$.
% With this, the final complexity of the algorithm increases to .
\end{proposition}
\begin{proof}
	The transition for join node is the most expensive operation for this algortihm, rest of the cases are similar to the dynamic programming on nice path decomposition in Section \ref{sec:entropy_matchings_pw}.
	We will fill the dynamic programming table in the bottom-up manner such that we compute all states for $k$ in increasing order.
	For each $k$, the time complexity for computing  $\ma[b, k, M]$ for all $M$ is $\bigO(n_{c_1} 2^{\twi})$, where $n_{c_1}$ is size of smallest child of node $b$. Now using the above Lemma \ref{lem:technical_lemma_for_matching_entropy} we get the following bound:

	\[
	\sum\limits_{b \in \text{Join nodes}} \bigO(n_{c_1} \cdot 2^{\twi}) = \bigO(2^{\twi}n\log n).	
	\]

	Now we need to compute this for all $1 \leq k \leq n$. Therefore, the total time complexity of the algorithms is $\bigO(n^2 \cdot \log n  \cdot \poly(\twi)\cdot2^{\twi})$.

	
	% If $k$ is fixed, complexity of finding an $\ma[b, k, M]$ for all $M$ is $\bigO(n_1(b) 2^{\twi})$, where $n_1(b)$ is size of smallest chld of node $b$. Therefore,
	% using lemma \ref{lem:technical_lemma_for_matching_entropy} we get 
	
	% \[
	% \sum\limits_{b \text{ is Joint}} \bigO(n_1(b) \cdot 2^{\twi}) = \bigO(2^{\twi}n\log n).	
	% \]
	
	% As we need to do that for all $k \leq \abs{V(G)}$, overall complexity of filling all
	% dymamic programming tables at all joint nodes is $\bigO(n^2 \cdot \log n  \cdot \poly(\twi)\cdot2^{\twi})$.
\end{proof}


\subsection{Graph Entropy Based on Independent Sets}
\label{sec:entropy_independent_sets}

Similarly to the previous section, in this section we will introduce dynamic programming algorithms for computing independent sets of all sizes for the given graph $G$ for bounded pathwidth and bounded treewidth respectively. We need to slightly adapt the dynamic programming states defined in Section \ref{sec:independent_sets}, in order to count independent sets of all sizes. The derivation of the corresponding recurrence relations of dynamic programming states remain almost the same as Section \ref{sec:independent_sets}. So, we will provide proofs only for the cases which are significantly different from the earlier sections.

\subsubsection{Bounded Pathwidth}\label{sec:pw_indep_k} 

\begin{definition}[Respectful Independent Sets of size $k$]
\label{def:respectful_independent_sets_k}
Let us fix a nice path decomposition $\mathcal{P} = \{X_1, \dots X_r\}$ for the given graph $G$. For each $b \in [r]$ and each $M \subseteq X_b$, we define $\RI(b,M)$ as the set of all independent sets $I$ in a $G_b^\downarrow $ such that each $I \cap X_b = M$ and $\lvert I \rvert=k$.
\end{definition}

\begin{definition}[Dynamic Programming State for Independent Sets of size $k$]
\label{def:dp_indpendent_set_k}
Let us fix a nice path decomposition $\mathcal{P} = \{X_1, \dots X_r\}$ for the given graph $G$. For each $b \in [r]$ and each $M \subseteq X_b$, we define $\ind[b, k, M]$ to be the number of elements in the set $\RI(b,k,M)$.
\end{definition}

\begin{remark}
Notice that if $X_r = \varnothing$ is the root node, then $G_{r}^\downarrow = G$, and independent sets of $G$ are in one-to-one correspondence with respectful independent sets $\RP(r,k,\varnothing)$, which implies $\ind [r, k, \varnothing]$ is the total number of independent sets of size $k$ in $G$.
\end{remark}

We now provide a bottom-up approach for filling up the dynamic table. The dynamic table relations depend on the type of the bag. We need to consider three cases when $X_b$ is: leaf node, introduce node and forget node respectively.


\begin{itemize}
\item \textbf{Leaf node:} If $X_l$ is a leaf bag then $\RI(b,k,M)$ contains only empty independent set as $X_l = \varnothing$. Therefore, $\ind[l,0,\varnothing] = 1$ and for any other $k > 0$ $\ind[l,k,\varnothing] = 0$.


\item \textbf{Introduce node:} Let $X_b$ be the introduce bag such that $X_b = X_c \cup \{v\}$. Then the dynamic programming state corresponding to $(b,k,M)$ satisfies the following recurrence relations: 
\[
\ind [b,k,M] =
\begin{cases}
\ind [c,k,M], &v \notin M,\\
\ind [c,k-1,M\setminus v], &v \in M \textup{ and } N(v) \cap M \ne \varnothing, \\
0, &v \in M \textup{ and } N(v) \cap M = \varnothing.
\end{cases}
\]
\item \textbf{Forget node:} Let $X_b$ be the forget node such that $X_b = X_c \setminus \{v\}$. Then the dynamic programming state corresponding to $(b,k,M)$ satisfies the following recurrence relation: 
\[
\ind [b,k,M] = \ind[c,k,M] + \ind[c,k-1,M\cup \{v\}] 
\]

This transition to $X_b$ can be computed in $\bigO(1)$ time.
\end{itemize}



\begin{proposition}\label{prop:comp_pw_ind_ent}
The time complexity for finding the total number of independent sets for a graph $G = (V,E)$, with $\lvert V \rvert = n$ and pathwidth $\pw$, using the above algorithm is $\bigO(n^2 \cdot \poly(\pw) \cdot2^{\pw})$.
\end{proposition}
\begin{proof}
Proof same as Proposition \ref{prop:comp_pw_perfmat}.
\end{proof}

\subsubsection{Bounded Treewidth}
\label{sec:tw_indep_k}

We will use a dynamic programming approach over the nice tree decomposition of the given graph $G$. The definitions of respectful matchings and dynamic programming states used here are same as Section \ref{sec:pw_indep_k}. Let $\mathcal{T} = \{T, \{X_t\}_{t \in V(T)}\}$ be a nice tree decomposition of the graph $G = (V,E)$ under consideration.

We now provide a bottom-up approach for filling up the dynamic table. The dynamic table relations depend on the type of the bag. There are four possible cases for the type of bags, i.e., when $X_b$ is: leaf node, introduce node, forget node and join node respectively. The reccurrence relations for leaf node, introduce node and forget node remains the same as Section \ref{sec:pw_indep_k}. Therefore, we only need to consider the case of join node which is described as follows:

\begin{itemize}
	\item \textbf{Join node:} Let $X_b$ be the join node with $X_{c_1}$ and $X_{c_2}$ as its children. Let us define $n_b = \lvert V(G_b^{\downarrow}) \rvert$ and also assume without loss of generality that $n_{c_1} \leq n_{c_2}$. 
	Then the dynamic programming state corresponding to $(b,k,M)$ will satisfy the following recurrence relation:
	\[
		\ind[b, k, M] = \sum_{0\leq k_{c_1} \leq n_{c_1}} \ind[c_1,k_{c_1}, M] \cdot \ind[c_2,k-k_{c_1},M]
	\]
\end{itemize}


% Using \ref{def:smaller_subtree} to bound the sum of $n_{c_1}$ over all join nodes:



\begin{proposition}\label{prop:comp_tw_independent_sets_size_k}
The time complexity for finding the total number of independent sets of all possible sizes for a graph $G = (V,E)$, with $\lvert V \rvert = n$ and treewidth $\twi$, using the above algorithm is $\bigO(n^2 \cdot \log n  \cdot \poly(\twi)\cdot2^{\twi})$.
% With this, the final complexity of the algorithm increases to .
\end{proposition}
\begin{proof}
Proof same as Proposition \ref{prop:comp_tw_matchings_size_k}.
\end{proof}



\subsection{Naive Baseline Algorithms}
We now provide a brief treatment for naive baseline algorithms for computing total number of perfect matchings, matchings and independent set. We will give a complextiy analysis for the naive baseline algorithms. We used these baseline algorithms as a benchmark for our experiments and give a detailed comparision with our algorithms in Section \ref{sec:Experiments}.

\begin{notation}
	We denote number of perfect matchings, matchings and indpendent sets of $G$ by $\pma(G), \ma(G)$ and $\ind(G)$ respectively.
\end{notation}
% \begin{notation}
% Let $W$ be a set of vertices $G$, by $G \setminus W$ we denote a graph $G \setminus W = G[V(G) \setminus W]$. If $e \in E(G)$ by
% 	$G - e$ we demote a new graph $G - e := (V(G), E(G)\setminus e)$. By $N[v]$ we denote closed neighbourhood of a vertex $v$.
% \end{notation}

% Naive algorithms are based on recursive formulas for the corresponding graph structures. 
% For a graph $G$, denote by $\pma(G)$, $\ma(G)$, $\ind(G)$ the number of perfect matchings, the number of matchings, and the number of independent subsets, 
% respectively.
We will now present the baseline algorithms for computing perfect matchings, matchings and independent sets:

\begin{itemize}
	\item Perfect matchings:
		\[
		\pma(G) =
		\begin{cases}
		\pma(G \setminus \{u, v\}) + \\
		\pma(G - (u, v)), & E(G) \neq \varnothing, \\
		1, & E(G) = \varnothing, V(G) = \varnothing, \\
		0, & E(G) = \varnothing, V(G) \neq \varnothing,
		\end{cases}
		\]
	
	\item Matchings:
		\[
			\ma(G) =
			\begin{cases}
			\ma(G \setminus \{u, v\}) + \ma(G - (u, v)), & E(G) \neq \varnothing, \\
			1, & E(G) = \varnothing.\\
			\end{cases}
		\]
		
	\item Independent sets:
		\[
			\ind(G) =
			\begin{cases}
			\ind(G \setminus N[v]) + \ind(G \setminus v), & E(G) \neq \varnothing, \\
			2^{\abs{E(G)}}, & E(G) = \varnothing,
			\end{cases}
		\]
			
\end{itemize}
	

\begin{theorem}\label{thm:naive_baseline}
	% Let $E(G) \neq \varnothing$, then $(u, v) \in E(G)$.
Let $G$ be the graph with $\abs{E(G)} = m$ and $\abs{V(G)} = n$, then the time complexity for computing all the above recurrence relations is $\bigO(2^{m} \cdot \textup{poly}(n))$.
\end{theorem}
\begin{proof}
	The above mentioned baseline algorithms implement the following branching strategy: Consider an arbitrary edge or vertex of the graph $G$. Now we get a choice for it i.e., whether it belongs to the structure we are counting or not. This implies that branching for the above mentioned recurrence formulas have degree at most two. Also, observe that after execution of each branch the number of edges of graph decrease by at least one. Therefore, the depth of the recursion trees is at most $m$, and the number of leaves of the recursion trees are at most $2 ^ {m}$. Hence, completing the proof that the time complexity for all the above mentioned baseline algorithms is $\bigO(2^{m} \cdot \textup{poly}(n))$.
\end{proof}

We present in section~\ref{ssec:faces} an application of PnP-HVAE on face images, using a pretrained state-of-the-art hierarchical VAE. 
Next, we study the application of our framework to natural images. To that end, we introduce  in section~\ref{ssec:patchVDVAE}  a patch hierachical VAE architecture, that is able to model natural images of different resolutions. In section~\ref{ssec:app_nat}, we provide deblurring, super-resolution and inpainting experiments to demonstrate the relevance of the proposed method.

Additional results are presented in Appendix~\ref{app:add}. All experiments can be reproduced using the code available at \url{https://github.com/jprost76/PnP-HVAE}.



\subsection{Face Image restoration (FFHQ)}\label{ssec:faces}
We first demonstrate the effectiveness of PnP-HVAE on highly structured data, by performing face image restoration.
Latent variable generative models can accurately model structured images such as face images \cite{karras2019style,vahdat2020nvae,child2021very,kingma2018glow}, and then be used to produce high quality restoration of such data. 
In our experiments, we use the VDVAE model of~\cite{child2021very}, pre-trained on the FFHQ dataset~\cite{karras2019style}, as our hierarchical VAE prior.
VDVAE has $L=66$ latent variable groups in its hierarchy and generates images at resolution $256\times256$.

We compare PnP-HVAE with the intermediate layer optimization algorithm (ILO)~\cite{daras2021intermediate} that is based on a different class of generative models than HVAE. ILO is a GAN inversion method which optimizes the image latent code along with the intermediate layer representation of a StyleGAN to generate an image consistent with a degraded observation.
We use the official implementation of ILO, along with a StyleGAN2 model~\cite{karras2020analyzing, stylegan2pytorch}, that was trained for 550k iterations on images of resolution $256\times256$ from FFHQ.  
As VDVAE and StyleGAN models are not trained on the same train-test split of FFHQ, we chose to evaluate the methods on a subset of 100 images from the CelebA dataset~\cite{liu2018large}. 
For super-resolution, the degradation model corresponds to the application of a gaussian low-pass filter followed by a $\times 4$ sub-sampling, and the addition of a gaussian white noise with $\sigma=3$.
For the deblurring, we considered motion blur and  gaussian kernels, both with a noise level $\sigma=8$. %

We provide quantitative comparisons in table~\ref{table:comp_ILO}, along with a visual comparison of the results in figure~\ref{fig:face_restoration}.
PnP-HVAE has the best  PSNR and SSIM results for all the considered restoration tasks, while ILO provides better results  for the perceptual distance.
By jointly optimizing the image and its latent variable, PnP-HVAE provides  results that are both realistic and consistent with the degraded observation.
On the other hand,  ILO  only optimizes on an extended latent space. This method generates  sharp and realistic images with better LPIPS scores,   
but the results lack  of consistency with respect to the observation, which explains the overall lower PSNR performance. 






\subsection{PatchVDVAE: a HVAE for natural images}\label{ssec:patchVDVAE}
Available generative models in the literature operate on images of  fixed resolutions and
are either restrained to datasets of limited diversity, or even to registered face images~\cite{kingma2018glow,child2021very, vahdat2020nvae, karras2019style}, or requiring additional class information~\cite{brock2018large, dhariwal2021diffusion, song2020score, luhman2022optimizing}.
Fitting an unconditional model on natural images appears to be a more difficult task, as their resolution can change, and their content is highly diverse.
The complexity of the problem can be reduced by learning a prior model on patches of reduced dimension. 
For image restoration problems, the patch model can be reused on images of higher dimensions~\cite{zoran2011learning,prost2021learning,altekruger2022patchnr}. When the model is a full CNN, the prior on the set of the  patches can  be computed efficiently by applying the network on the full image~\cite{prost2021learning}.

We thus introduce  patchVDVAE, a fully convolutional hierarchical VAE.
Contrary to existing HVAE models whose resolution is constrained by the constant tensor at the input of the top-down block, patchVDVAE can generate images of different resolutions by controlling the dimension of the input latent. 
This amounts to defining a prior on patches whose dimension corresponds to the receptive field of the VAE. A similar model is used for image denoising in~\cite{prakash2021interpretable}.

 
For PatchVDVAE architecture, we use the same bottom-up and top-down blocks as VDVAE~\cite{child2021very}, and replace the constant trainable input in the first top-down block by a latent variable, to make the model fully convolutional (details on the  architecture are given in Appendix~\ref{app:details}). 
The training dataset is composed of $128\times 128$ patches extracted from a combination of DIV2K~\cite{agustsson2017ntire} and Flickr2K~\cite{Lim_2017_CVPR_workshops} datasets.
We perform data augmentation by extracting  patches at $3$ resolutions: HR-images and $\times 2$ and $\times 4$ downscaled images. 
The model is trained for $7.10^5$ iterations with a batch size of $64$. Following the recommendation of~\cite{hazami2022efficient}, we use Adamax optimizer with an exponential moving average and gradient smoothing of the variance.
We set the decoder model to be a gaussian with diagonal covariance, as in~\cite{luhman2022optimizing}.
PatchVDVAE is fully convolutional and can generate images of dimension that are multiples of $64$ as illustrated by
figure~\ref{fig:vdvae}.

\newlength{\patchwidth}
\setlength{\patchwidth}{0.135\columnwidth}
\begin{figure}[!ht]
    \centering
    \begin{subfigure}[t]{.34\columnwidth}\hspace{0.1cm}
        \setlength{\tabcolsep}{0.02pt}
\renewcommand{\arraystretch}{0}
        \begin{tabular}{*{2}{p{1.03\patchwidth}}}
            \includegraphics[width=\patchwidth]{figures_arxiv/patchVDVAE/samples/generated/64x64/setup-5-image-0018.png} &
            \includegraphics[width=\patchwidth]{figures_arxiv/patchVDVAE/samples/generated/64x64/setup-5-image-0016.png} \\
            \includegraphics[width=\patchwidth]{figures_arxiv/patchVDVAE/samples/generated/64x64/setup-5-image-0008.png} &
            \includegraphics[width=\patchwidth]{figures_arxiv/patchVDVAE/samples/generated/64x64/setup-5-image-0019.png}   
        \end{tabular}
    \end{subfigure}\hspace{-0.15cm}
    \begin{subfigure}[t]{.64\columnwidth}
\begin{tabular}{cc}\vspace{-0.1cm}
\includegraphics[width=2\patchwidth]{figures_arxiv/patchVDVAE/samples/generated/256x256/setup-2-image-0009.png}&
        \includegraphics[width=2\patchwidth]{figures_arxiv/patchVDVAE/samples/generated/256x256/setup-2-image-0002.png}\end{tabular}

    \end{subfigure}
    \caption{\label{fig:vdvae} Left: $64\times64$ patches samples from our patchVDVAE model trained on patches from natural images.
    Right: PatchVDVAE is fully convolutional and it can generate images of higher resolution (here: $128\times128$).\vspace{-0.2cm}}
\end{figure}

\subsection{Natural images restoration}\label{ssec:app_nat}
We  evaluate PnP-HVAE on natural image restoration.
For each task, we report the average value of the PSNR, the SSIM, and the LPIPS metrics on $20$ images from the test set of the BSD dataset~\cite{MartinFTM01}.\\


\noindent
{\bf Image deblurring.}
In the experiments, we consider $2$ gaussian kernels and $2$ motion blur kernels from~\cite{levin2009understanding}, with $3$ different noise levels 
$\sigma \in \{2.55, 7.65, 12.75\}$.
As a baseline we consider  EPLL~\cite{zoran2011learning}, which learns a prior on image patches with a gaussian mixture model.
We also compare PnP-HVAE  with PnP-MMO and GS-PnP, $2$ competing convergent Plug-and-Play methods based on CNN denoisers.
PnP-MMO~\cite{pesquet2021learning} restricts the denoiser to be contraction in order to guarantee the convergence of the PnP forward-backard algorithm. GS-PnP~\cite{hurault2022gradient} considers a gradient step denoiser and reaches state-of-the-art performances of non converging methods~\cite{zhang2021plug}.
We set the temperature $\tau$  in our method as $0.95$, $0.8$ and $0.6$ for noise levels $2.55$, $7.65$ and $12.75$ respectively, and we let it run for a maximum of $50$ iterations. 
For the three compared methods we use the official implementations and pre-trained models provided by the respective authors. 
Details on the choice of hyperparameters for the concurrent methods are provided in the Appendix~\ref{app:details}
Figure~\ref{fig:deblurring_bsd} illustrates that our method provides correct deblurring results. 

According to table~\ref{tab:deb}, the performance of PnP-HVAE is between those of EPLL and GS-PnP and it outperforms PnP-MMO for large noise levels.\\

\begin{table}
\begin{center}\footnotesize
    \begin{tabular}{>{\centering}m{.3cm}*{5}{c}}
    $\sigma$ &Method & PSNR$\uparrow$ & SSIM$\uparrow$ & LPIPS$\downarrow$  \\ 
    \hline
    \multirow{4}{*}{\vcell{$2.55$}}
    & PnP-HVAE & $27.75$ & $0.79$ & $0.31$\\
    & GS-PNP \cite{hurault2022gradient} & $\mathbf{29.59}$ & $\mathbf{0.84}$ & $\mathbf{0.22}$\\
    & EPLL \cite{zoran2011learning} & $26.49$ & $0.71$ & $0.36$\\ 
    & PnP-MMO \cite{pesquet2021learning} & $\underbar{29.50}$ & $\underbar{0.83}$ & $\underbar{0.20}$ \\ \hline
    \multirow{4}{*}{\vcell{$7.65$}}
    & PnP-HVAE & $\underbar{26.36}$ & $\underbar{0.72}$ & $\underbar{0.40}$\\
    & GS-PNP \cite{hurault2022gradient} & $\mathbf{27.33}$ & $\mathbf{0.77}$ & $\mathbf{0.31}$\\
    & EPLL \cite{zoran2011learning} & $24.04$ & $0.66$ & $0.45$ \\ 
    & PnP-MMO \cite{pesquet2021learning} & $25.34$ & $0.69$ & $0.34$\\
    \hline
    \multirow{4}{*}{\vcell{$12.75$}}
    & PnP-HVAE & $\underbar{25.12}$ & $\mathbf{0.73}$ & $\underbar{0.47}$\\
    & GS-PNP \cite{hurault2022gradient} & $\mathbf{26.32}$ & $\mathbf{0.73}$ & $\mathbf{0.37}$\\
    & EPLL \cite{zoran2011learning} & $23.28$ & $0.61$ & $0.51$ \\ 
    & PnP-MMO \cite{pesquet2021learning} & $22.42$ & $0.53$& $0.54$ \\
    \hline
    &\vspace*{-.3cm}\\
            \multicolumn{2}{c}{Blur and motion kernels}& \multicolumn{3}{c}{
        \includegraphics*[scale=1]{figures_arxiv/kernels/4.png}\;\includegraphics*[scale=1]{figures_arxiv/kernels/7.png}\;\includegraphics*[scale=1]{figures_arxiv/kernels/9.png}\;\includegraphics*[scale=1]{figures_arxiv/kernels/11.png}} 
    \end{tabular}
        \caption{\label{tab:deb}Comparison  of PnP-HVAE  and other restoration methods on deblurring. Results are averaged on $4$ kernels.\vspace{-0.2cm}}% on image deblurring.}
    \end{center}
\end{table}

\begin{figure}
    
    \begin{subfigure}[h]{\linewidth}
        \centering
        \includegraphics*[width=\columnwidth]{figures_arxiv/deb_s255_k7.pdf}\vspace{-0.1cm}
        \caption{Gaussian blur, $\sigma=2.55$}
    \end{subfigure}
    \begin{subfigure}[h]{\linewidth}
        \centering
        \includegraphics*[width=\columnwidth]{figures_arxiv/deb_s765_k11.pdf}\vspace{-0.1cm}
        \caption{Motion blur, $\sigma=7.65$}
    \end{subfigure}\vspace*{-0.1cm}
    \caption{\label{fig:deblurring_bsd} Natural image deblurring\vspace{-0.1cm}}
\end{figure}

\noindent {\bf Effect of the temperature.}
PnP-HVAE gives control on the temperature of the prior over the latent space.
In figure~\ref{fig:temp_effect}, we illustrate that reducing the temperature increases the strength of the regularization prior. In this example the tuning $\tau=0.7$ produces the best performance.\\
\begin{figure}[!ht]
   
    \includegraphics[width=\columnwidth]{figures_arxiv/demo_temp.pdf}\vspace{-0.15cm}
    \caption{ \label{fig:temp_effect} Effect of the temperature in PnP-VAE on a deblurring problem, with $\sigma=7.65$.\vspace{-0.15cm}}
\end{figure}


\noindent
{\bf Image inpainting.}
Next we consider the task of noisy image inpainting. 
We compose a test-set of 10 images from the validation set of BSD~\cite{MartinFTM01} and we create masks
  by occluding diverse objects of small size in the images. 
A gaussian white noise with $\sigma=3$ is added to the images.
As a comparaison, we still consider GS-PnP and EPLL.
For PnP-HVAE, the temperature is set to $\tau=0.6$, and the algorithm is run for a maximum of $200$ iterations, unless the residual $||\x_{k+1}-\x_k||$ is on a plateau.
We provide on Table~\ref{tab:inpainting_bsd} the distortion metrics with the ground truth, as well as a visual
\begin{table}



\begin{center}
    \begin{tabular}{cccc}
        & PSNR$\uparrow$ & SSIM$\uparrow$ &LPIPS$\downarrow$ \\\hline
        PnP-HVAE  & $\mathbf{29.54}$ & $\mathbf{0.93}$ & $\mathbf{0.06}$\\
        GS-PNP & $28.52$ & $\mathbf{0.93}$ & $0.09$\\
        EPLL & $\underline{29.16}$ & $\mathbf{0.93}$ & $\mathbf{0.06}$\\
    \end{tabular}
    \caption{\label{tab:inpainting_bsd}Quantitative evaluation for inpainting on BSD.}
    \end{center}
\end{table}
comparison on figure~\ref{fig:inpainting_bsd}. 
With its hierarchical structure,  PnP-HVAE outperforms the compared methods. \vspace{0.05cm}



\begin{figure}[!h]
    \includegraphics[width=\columnwidth]{figures_arxiv/demo_inp_bsd2.pdf}\vspace{-0.1cm}
    \caption{\label{fig:inpainting_bsd}Natural image inpainting\vspace{-0.3cm}}
\end{figure}












\bibliographystyle{elsarticle-num}
\bibliography{refs.bib}

\section{Appendix for Proofs}

\paragraph{Proof of Theorem \ref{thm:main}.}

\begin{proof}
\label{proof:main}
Our proof has two steps. In Step 1, we will show that SimCLR is equivalent to minimizing the cross entropy loss defined in Eqn.~(\ref{eqn:cross-entropy}). 
In Step 2, we will show  that minimizing the cross-entropy loss 
is equivalent to spectral clustering on $\bfpi$. 
Combining the two steps together, we have proved our theorem. 

\textbf{Step 1: } SimCLR is equivalent to minimizing the cross entropy loss.

The cross-entropy loss takes expectation over 
$\bfW_\bfX\sim \mathbb{P}(\cdot ; \bfpi)$, 
which means $\bfW_\bfX$ has exactly one non-zero entry in each row $i$. By Lemma~\ref{lem:multinomial}, we know every row $i$ of $\bfW_\bfX$ is independent of other rows. Moreover, 
$\bfW_{\bfX,i}\sim \mathcal{M}(1, \bfpi_i/\sum_j \bfpi_{i,j})=\mathcal{M}(1, \bfpi_i)$, because $\bfpi_i$ itself is a probability distribution.
Similarly, we know $\bfW_\bfZ$ also has the row-independent property by sampling over $\mathbb{P}(\cdot;\bfK_\bfZ)$.
Therefore, by Lemma~\ref{lem:cross_split}, we know Eqn.~(\ref{eqn:cross-entropy}) is equivalent to:
\[
 -\sum_{i=1}^n \mathbb{E}_{\bfW_{\bfX,i}}[\log \mathbb{P}(\bfW_{\bfZ,i}=\bfW_{\bfX,i};\bfK_\bfZ)],
\]

This expression takes expectation over $\bfW_{\bfX,i}$ for the given row $i$. Notice that 
$\bfW_{\bfX,i}$ has exactly one non-zero entry, which equals $1$ (same for $\bfW_{\bfZ,i}$). 
As a result
we expand the above expression to be:
\begin{equation}
 -\sum_{i=1}^n \sum_{j\neq i} \Pr(\bfW_{\bfX,i,j}=1)\log \Pr(\bfW_{\bfZ,i,j}=1).
\label{eqn:detailed-expansion}    
\end{equation}


By Lemma~\ref{lem:multinomial}, $\Pr(\bfW_{\bfZ,i,j}=1)=\bfK_{\bfZ,i,j}/\|\bfK_{\bfZ,i}\|_1$ for $j\neq i$. Recall that $\bfK_\bfZ=(k(\bfZ_i-\bfZ_j))_{(i,j)\in[n]^2}$, which means 
$\bfK_{\bfZ,i,j}/\|\bfK_{\bfZ,i}\|_1=\frac{\exp(-\|\bfZ_i-\bfZ_j\|^2/{2\tau})}{\sum_{k\neq i}
\exp(-\|\bfZ_i-\bfZ_k\|^2/{2\tau})
}$ for $j\neq i$, when $k$ is the Gaussian kernel with variance $\tau$. 

Notice that $\bfZ_i=f(\bfX_i)$, so we know
\begin{equation}
-\log \Pr(\bfW_{\bfZ,i,j}=1)=
-\log \frac{\exp(-\|f(\bfX_i)-f(\bfX_j)\|^2/{2\tau})}{\sum_{k\neq i}
\exp(-\|f(\bfX_i)-f(\bfX_k)\|^2/{2\tau}),
}
\label{eqn:infonce-equivalence}    
\end{equation}


The right hand side is exactly the InfoNCE loss defined in Eqn.~(\ref{eqn:infonce}).
Inserting Eqn.~(\ref{eqn:infonce-equivalence}) into Eqn.~(\ref{eqn:detailed-expansion}), we get the SimCLR algorithm, which first samples augmentation pairs $(i,j)$ with $\Pr(\bfW_{\bfX,i,j}=1)$ for each row $i$, and then optimize the InfoNCE loss. 

\textbf{Step 2: } minimizing the cross entropy loss 
is equivalent to spectral clustering on $\bfpi$.


By Lemma~\ref{lem:convert_to_spectral}, we may further convert the loss to 
\begin{equation}
\label{eqn:main-theorem-repul-attr}
\min_{\bfZ}
-\sum_{(i,j)\in [n]^2} \mathbf{P}_{i,j}
\log k (\bfZ_i-\bfZ_j)+\log \mathbf{R}(\bfZ).
\end{equation}
Since $k$ is the Gaussian kernel, this reduces to \[
\min_\bfZ \mathrm{tr}(\bfZ^\top \mathbf{L}(\bfpi) \bfZ)
+\log \mathbf{R}(\bfZ),
\]

where we use the fact that $\mathbb{E}_{\bfW_\bfX\sim \mathbb{P}(\cdot; \bfpi)}[\mathbf{L}(\bfW_\bfX)]
=\mathbf{L}(\bfpi)
$, because the Laplacian operator is linear and $
\mathbb{E}_{\bfW_\bfX\sim \mathbb{P}(\cdot; \bfpi)}(\bfW_\bfX)=\bfpi
$.
\end{proof}

\paragraph{Proof of Theorem \ref{thm:clip}.}
\begin{proof}
Since $\bfW_\bfX\sim \mathbb{P}(\cdot;\bfpi_{\mathbf{A}, \mathbf{B}})$, we know 
$\bfW_\bfX$ has exactly one non-zero entry in each row, denoting the pair that got sampled. 
A notable difference compared to the previous proof is we now have $n_\mathcal{A}+n_\mathcal{B}$ objects in our graph. CLIP deals with this by taking a mini-batch of size $2N$, 
such that $n_\mathcal{A}=n_\mathcal{B}=N$, and adding the $2N$ InfoNCE losses together. We label the objects in $\mathcal{A}$ as $[n_\mathcal{A}]$, and the objects in $\mathcal{B}$ as $\{n_\mathcal{A}+1, \cdots, n_\mathcal{A}+n_\mathcal{B}\}$. 

Notice that $\bfpi_{\mathbf{A}, \mathbf{B}}$ is a bipartite graph, so the edges of objects in $\mathcal{A}$ will only connect to object in $\mathcal{B}$ and vice versa. We can define the similarity matrix in $\cZ$ as $\bfK_\bfZ$, 
where $\bfK_\bfZ(i, j+n_\mathcal{A})=\bfK_\bfZ(j+n_\mathcal{A},i)= k(\bfZ_i-\bfZ_j)$ for $i\in [n_\mathcal{A}], j\in [n_\mathcal{B}]$, and otherwise we set $\bfK_\bfZ(i,j)=0$. 
The rest is same as the previous proof. 
\end{proof}

\paragraph{Proof of Theorem \ref{thm:exponential}.}

\begin{proof}
\label{proof:exponential}
Since the objective function consists of a linear term combined with an entropy regularization, which is a strongly concave function, the maximization problem is a convex optimization problem. Owing to the implicit constraints provided by the entropy function, the problem is equivalent to having only the equality constraint. We then introduce the Lagrangian multiplier $\lambda$ and obtain the following relaxed problem:

$$
\widetilde{E}(\boldsymbol{\alpha})=\psi_{1}-\sum_{i=1}^n \alpha_{i} \psi_{i}+\tau \sum_{i=1}^n \alpha_{i}\log \alpha_{i}+\lambda\left(\boldsymbol{\alpha}^{\top} \mathbf{1}_n-1\right).
$$

As the relaxed problem is unconstrained, taking the derivative with respect to $\alpha_{i}$ yields

$$
\frac{\partial \widetilde{E}(\boldsymbol{\alpha})}{\partial \alpha_{i}}=-\psi_{i}+\tau\left(\log \alpha_{i}+\alpha_{i} \frac{1}{\alpha_{i}}\right)+\lambda=0.
$$

Solving the above equation implies that $\alpha_{i}$ takes the form
$
\alpha_{i}=\exp \left(\frac{1}{\tau} \psi_{i}\right) \exp \left(\frac{-\lambda}{\tau}-1\right).
$ Since $\alpha_{i}$ lies on the probability simplex, the optimal $\alpha_{i}$ is explicitly given by
$
\alpha^{*}_{i}=\frac{\exp \left(\frac{1}{\tau} \psi_{i}\right)}{\sum_{i^{\prime}=1}^n \exp \left(\frac{1}{\tau} \psi_{i^{\prime}}\right)} .
$ Substituting the optimal point into the objective function, we obtain
$$
\begin{aligned}
E\left(\boldsymbol{\alpha}^*\right)  &=\psi_1-\sum_{i=1}^n \frac{\exp \left(\frac{1}{\tau} \psi_{i}\right)}{\sum_{i^{\prime}=1}^n \exp \left(\frac{1}{\tau} \psi_{i^{\prime}}\right)} \psi_{i}+\tau \sum_{i=1}^n \frac{\exp \left(\frac{1}{\tau} \psi_{i}\right)}{\sum_{i^{\prime}=1}^n \exp \left(\frac{1}{\tau} \psi_{i^{\prime}}\right)}\log \frac{\exp \left(\frac{1}{\tau} \psi_{i}\right)}{\sum_{i^{\prime}=1}^n \exp \left(\frac{1}{\tau} \psi_{i^{\prime}}\right)} \\
& =\psi_1 - \tau \log \left(\sum_{i=1}^n \exp \left(\frac{1}{\tau} \psi_{i}\right)\right).
\end{aligned}
$$
Thus, the Lagrangian dual function is given by
\begin{equation*}
-E\left(\boldsymbol{\alpha}^*\right)= -\tau \log \frac{\exp \left(\frac{1}{\tau} \psi_{1}\right)}{\sum_{i=1}^n \exp \left(\frac{1}{\tau} \psi_{i}\right)}.\qedhere
\end{equation*}
\end{proof}



\section{More on Experiments} \label{section: experiment_details}

\paragraph{CIFAR-10 and CIFAR-100} CIFAR-10 ~\citep{krizhevsky2009learning} and CIFAR-100 ~\citep{krizhevsky2009learning} are well-known classic image classification datasets. Both CIFAR-10 and CIFAR-100 contain a total of 60k $32 \times 32$ labeled images of different classes, with 50k for training and 10k for testing. CIFAR-10 is similar to CIFAR-100, except there are 10 different classes in CIFAR-10 and 100 classes in CIFAR-100.

\paragraph{TinyImageNet} TinyImageNet ~\citep{le2015tiny} is a subset of ImageNet ~\citep{deng2009imagenet}. There are 200 different object classes in TinyImageNet, with 500 training images, 50 validation images, and 50 test images for each class. All the images in TinyImageNet are colored and labeled with a size of $64 \times 64$.

\textbf{Pseudo-code.} Algorithm \ref{alg:Training Procedure} presents the pseudo-code for our empirical training procedure.

\begin{algorithm}[!htbp]
\caption{Training Procedure}
\label{alg:Training Procedure}
\begin{algorithmic}[1]
\REQUIRE trainable encoder network $f$, batch size $N$, augmentation strategy \textit{aug}, loss function $L$ with hyperparameters \textit{args}
\FOR {sampled minibatch ${x_i}_{i=1}^N$}
\FORALL{$i \in { 1, ..., N }$}
\STATE draw two augmentations $t_i = \textit{aug}\left(x_i\right) $, $t_i' = \textit{aug}\left(x_i\right) $
\STATE $z_i = f\left(t_i\right)$, $z_i' = f\left(t_i'\right)$
\ENDFOR
\STATE compute loss $\mathcal{L} = L(N, z, z', \textit{args})$
\STATE update encoder network $f$ to minimize $\mathcal{L}$
\ENDFOR
\STATE \textbf{Return} encoder network $f$
\end{algorithmic}
\end{algorithm}

We also provide the pseudo-code for our core loss function used in the training procedure in Algorithm \ref{alg:Core loss}. The pseudo-code is almost identical to SimCLR's loss function, with the exception of an extra parameter $\gamma$.

\begin{algorithm}[!htbp]
\caption{Core loss function $\mathcal{C}$}
\label{alg:Core loss}
\begin{algorithmic}[1]
\REQUIRE batch size $N$, two encoded minibatches $z_1, z_2$, $\gamma$, temperature $\tau$
\STATE $z = \textit{concat}\left(z_1, z_2\right)$
\FOR {$i \in {1, ..., 2N }, j \in {1, ..., 2N}$ }
\STATE $s_{i,j} = \Vert z_i - z_j \Vert_2^{\gamma}$
\ENDFOR
\STATE \textbf{define} $l(i, j)$ \textbf{as} $l(i, j) = - \log \frac{exp\left(s_{i,j}/\tau \right)}{\sum_{k=1}^{2N} \mathbf{1}{[k \ne i]} exp\left(s{i, j} / \tau \right)} $
\STATE \textbf{Return} $\frac{1}{2N} \sum_{k=1}^N\left[l(i, i+N) + l(i+N, i)\right]$
\end{algorithmic}
\end{algorithm}

Utilizing the core loss function $\mathcal{C}$, we can define all kernel loss functions used in our experiments in Table \ref{table: loss definition}. For all $z_i \in z$ with even dimensions $n$, we define $z_{L_i} = z_i\left[0:n/2\right]$ and $z_{R_i} = z_i\left[n/2:n\right]$.

\begin{table}[ht]
\centering
\begin{tabular}{{@{}l|l@{}}}
Kernel  &  Loss function \\ \midrule
Laplacian & $\mathcal{C}\left(N, z, z', \gamma=1, \tau\right)$\\ \midrule
Sum       & $\lambda * \mathcal{C}\left(N, z, z', \gamma=1, \tau_1\right) + (1-\lambda) * \mathcal{C}\left(N, z, z', \gamma=2, \tau_2\right)$  \\ \midrule
Concatenation Sum&$\lambda * \mathcal{C}\left(N, z_L, z'_L, \gamma=1, \tau_1\right) + (1-\lambda) * \mathcal{C}\left(N, z_R, z'_R, \gamma=2, \tau_2\right)$\\ \midrule
$\gamma = 0.5$ & $\mathcal{C}\left(N, z, z', \gamma=0.5, \tau\right)$          \\ 

\end{tabular}

\caption{Definition of kernel loss functions in our experiments}
\label {table: loss definition}
\end{table}

\textbf{Baselines.} We reproduce the SimCLR algorithm using PyTorch Lightning~\citep{PytorchLightning}.

\textbf{Encoder details.}
The encoder $f$ consists of a backbone network and a projection network. We employ ResNet50~\citep{ResNet} as the backbone and a 2-layer MLP (connected by a batch normalization~\citep{ioffe2015batch} layer and a ReLU \cite{nair2010rectified} layer) with hidden dimensions 2048 and output dimensions 128 (or 256 in the concatenation kernel case).

\textbf{Encoder hyperparameter tuning.}
For each encoder training case, we randomly sample 500 hyperparameter groups (sample details are shown in Table \ref{table: Hyperparameter sample}) and train these samples simultaneously using Ray Tune ~\citep{RayTune}, with the ASHA scheduler~\citep{li2018massively}. Ultimately, the hyperparameter group that maximizes the online validation accuracy (integrated in PyTorch Lightning) within 5000 validation steps is chosen for the given encoder training case.

\begin{table}[ht]
\centering

\begin{tabular}{@{}l|l|l@{}}
\midrule
Hyperparameter  & Sample Range & Sample Strategy \\ \midrule
start learning rate & $\left[10^{-2}, 10\right]$ & log uniform \\ \midrule
$\lambda$       & $\left[0, 1\right]$ & uniform \\ \midrule
$\tau$, $\tau_1$, $\tau_2$ & $\left[0, 1\right]$ & log uniform \\ \midrule
\end{tabular}

\caption{Hyperparameters sample strategy}
\label {table: Hyperparameter sample}
\end{table}

\textbf{Encoder training.} 
We train each encoder using the LARS optimizer~\citep{LARSOptimizer}, LambdaLR Scheduler in PyTorch, momentum 0.9, weight decay $10^{-6}$, batch size 256, and the aforementioned hyperparameters for 400 epochs on a single A-100 GPU.

\textbf{Image transformation.} The image transformation strategy, including augmentation, is identical to the default transformation strategy provided by PyTorch Lightning.

\textbf{Linear evaluation.}
The linear head is trained using the SGD optimizer with a cosine learning rate scheduler, batch size 64, and weight decay $10^{-6}$ for 100 epochs. The learning rate starts at $0.3$ and ends at $0$.

\textbf{Moco Experiments.} We also tested our method based on MoCo~\citep{he2019moco}. The results are summarized in Table \ref{tab:results-moco}. Here we choose ResNet18~\citep{ResNet} as the backbone and set a temperature of $0.1$ as default. For our simple sum kernel, we set $\lambda=0.8$. The results show that our method outperforms the original MoCo method.

\begin{table}[thb]
\centering
\caption{MoCo Experiment Results on CIFAR-10 and CIFAR-100.}
\label{tab:results-moco}
\resizebox{\textwidth}{!}{%
\begin{tabular}{@{}c|ccc|ccc@{}}
\toprule
\multirow{3}{*}{Method} & \multicolumn{3}{c|}{CIFAR-10} & \multicolumn{3}{c}{CIFAR-100} \\ \cmidrule(lr){2-4} \cmidrule(lr){5-7} 
                        & 200 epochs & 400 epochs    & 1000 epochs   & 200 epochs & 400 epochs & 1000 epochs         \\ \midrule
MoCo (repro.)         & $76.41 \pm 0.12$    & $80.01 \pm 0.15$          & $84.45 \pm 0.08$    & $\mathbf{47.02 \pm 0.11}$ & $52.50 \pm 0.07$ & $57.62 \pm 0.15$            \\
\midrule
Laplacian Kernel        & ${78.09 \pm 0.10}$    & $\mathbf{83.85 \pm 0.09}$          & $\mathbf{88.34 \pm 0.16}$    & $46.12 \pm 0.22$   & $53.44 \pm 0.17$ & $59.10 \pm 0.14$        \\
Simple Sum Kernel & $\mathbf{78.12 \pm 0.15}$   & $83.23 \pm 0.18$ & $87.50 \pm 0.20$ & $46.65 \pm 0.06$ & $\mathbf{53.62 \pm 0.19}$ & $\mathbf{59.83 \pm 0.12}$\\
\bottomrule
\end{tabular}
}
\end{table}



\section{More Experiments on Synthetic Data}


Consider a scenario with $n$ clusters, each containing $k$ vertices. Let the probability of vertices $u$ and $v$ from the same cluster belonging to $\bfpi$ be $p$. Conversely, for vertices $u$ and $v$ from different clusters, let the probability of belonging to $\pi$ be $q$. We generate the graph $\bfpi$ randomly, based on $p$ and $q$. We experiment with values of $k=100$ and $n=6$ for ease of visualization, embedding all points in a two-dimensional space. Each vertex's initial position originates from a normal distribution. In each iteration, we sample a subgraph of $\bfpi$ uniformly, ensuring each vertex has an out-degree of $1$. We then optimize the corresponding vectors using InfoNCE loss with an SGD optimizer and iterate until convergence. Our experimental setup consists of an SGD learning rate of $1$, an InfoNCE loss temperature of $0.5$, and a batch size of $50$. We evaluate two scenarios with different $p$ and $q$ values: $p=1$, $q=0$, and $p=0.75$, $q=0.2$. The results of these experiments are visualized in Figure \ref{fig:vis-spectral-cluster}. The obtained embeddings exhibit the hallmark pattern of spectral clustering of graph $\bfpi$.

\begin{figure}[!tb]
\centering
\subfigure{
\includegraphics[width=1\textwidth]{Figures/cluster_pi.png}
\label{fig:vis-cluster}
}
\subfigure{
\includegraphics[width=1\textwidth]{Figures/noised_cluster_pi.png}
\label{fig:vis-noised-cluster}
}
\caption{Visualizations of the optimization process using InfoNCE Loss on the vectors corresponding to $\bfpi$. Points of identical color belong to the same cluster within $\bfpi$. To showcase the internal structure of $\bfpi$, we randomly select 10 vertices from each cluster to display the edge distribution of $\bfpi$.}
\label{fig:vis-spectral-cluster}
\end{figure}






%% Authors are advised to use a BibTeX database file for their reference list.
%% The provided style file elsarticle-num.bst formats references in the required Procedia style

%% For references without a BibTeX database:

% \begin{thebibliography}{00}

%% \bibitem must have the following form:
%%   \bibitem{key}...
%%

% \bibitem{}

% \end{thebibliography}

\end{document}

%%
%% End of file `procs-template.tex'.
