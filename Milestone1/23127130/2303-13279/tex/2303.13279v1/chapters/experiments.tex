% !TEX root = ../main.tex
\section{Experimental Results}{\label{sec:Experiments}}
\subsection{Statistical analysis of molecules}\label
{sec:stats_molecules}
In this section, we provide a detailed a distribution of treewidth for all the PubChem compounds. We observed that more than $99.9 \%$ compunds have treewidth less than $5$. We also provide distirbution of treewidths for handpicked data sources in PubChem and observed the same trend again. In particular, molecules with treewidth greater than $ 15$ are rare in the entire PubChem dataset.
\begin{figure}[H]
	\centering
	\subfloat[]{\includegraphics[width=0.44\textwidth]{./chapters/images/treewidth_histogram.pdf}}
	% \qquad
	\subfloat[]{\includegraphics[width=0.5\textwidth]{./chapters/images/treewidth_violinplot.pdf}}

	% \begin{subfigure}{0.5\textwidth}
	% 	\centering
	% 	\includegraphics[width=0.4\linewidth]{./chapters/images/treewidth_histogram.pdf}
	% 	% \caption{Distribution of Treewidth in PubChem database.}
	% 	% \label{fig:treewidth_histogram}
	% \end{subfigure}
	% \begin{subfigure}{0.5\textwidth}
	% 	\centering
	% 	\includegraphics[width=0.4\linewidth]{./chapters/images/treewidth_violinplot.pdf}
	% 	% \caption{Distribution of Treewidths across data sources from PubChem.}
	% 	% \label{fig:treewidth_violinplot}
	% \end{subfigure}
	\caption{Distribution of treewidth for molecules in PubChem database.}
\end{figure}

% \begin{figure}[h!]
% 	\captionsetup{justification=centering}
% 	\centering
% 	\includegraphics[scale = 0.6]{./chapters/images/treewidth_histogram.pdf}
% 	\caption{Distribution of Treewidth in PubChem database.}
% 	\label{fig:treewidth_histogram}
% \end{figure}
% \begin{figure}[h]
% 	\captionsetup{justification=centering}
% 	\centering
% 	\includegraphics[scale = 0.6]{./chapters/images/vertices_histogram.png}
% 	\caption{Distribution of Vertices in PubChem database.}
% 	\label{fig:vertices_histogram}
% \end{figure}

% \begin{figure}[h!]
% 	\captionsetup{justification=centering}
% 	\centering
% 	\includegraphics[scale = 0.6]{./chapters/images/treewidth_violinplot.pdf}
% 	\caption{Distribution of Treewidths across data sources from PubChem.}
% 	\label{fig:treewidth_violinplot}
% \end{figure}

% \begin{figure}[h]
% 	\captionsetup{justification=centering}
% 	\centering
% 	\includegraphics[scale = 0.6]{./chapters/images/vertices_violinplot.png}
% 	\caption{Distribution of Vertices across data sources from PubChem.}
% 	\label{fig:vertices_violinplot}
% \end{figure}

% \begin{figure}[h]
% 	\captionsetup{justification=centering}
% 	\centering
% 	\includegraphics[scale = 0.6]{./chapters/images/edges_violinplot.png}
% 	\caption{Distribution of Edges across data sources from PubChem.}
% 	\label{fig:edges_violinplot}
% \end{figure}
% \begin{figure}[h]
% 	\captionsetup{justification=centering}
% 	\centering
% 	\includegraphics[scale = 0.6]{./chapters/images/tw_26_example.png}
% 	\caption{Example with Treewidth $26$.}
% 	\label{fig:tw_26_example}
% \end{figure}
\subsection{Performance of our Algorithms}\label{sec:performance_algo}
% \subsubsection{Naive Algotihms}
We now present the time vs treewidth characteristics of our algoritms for computing number of perfect matchings, matchings and independet sets (See Figure \ref{fig:charact_time_tw}). The following characteristics are for selected datasets from PubChem. We have also conducted our experiments on the entire PubChem dataset. The experimental dataset will be available in the Zenedo repository.
\begin{figure}[H]
	\centering
	\subfloat[]{\includegraphics[width=0.3\textwidth]{./chapters/images/treewidth_time__perfect_matchings.pdf}}
	\subfloat[]{\includegraphics[width=0.3\textwidth]{./chapters/images/treewidth_time_matchings.pdf}} 
	\\ 
	\subfloat[]{\includegraphics[width=0.3\textwidth]{./chapters/images/treewidth_time_independent_set.pdf}} 
	\caption{Time vs Treewidth characteristics for algortihms}
	\label{fig:charact_time_tw}
\end{figure}
% \begin{figure}[h!]
% 	\captionsetup{justification=centering}
% 	\centering
% 	\includegraphics[scale = 0.6]{./chapters/images/treewidth_time_independent_set.pdf}
% 	\caption{Time vs Treewidth for computing Independent Sets}
% 	\label{fig:time_treewidth}
% \end{figure}
% \begin{figure}[h!]
% 	\captionsetup{justification=centering}
% 	\centering
% 	\includegraphics[scale = 0.6]{./chapters/images/treewidth_time_matchings.pdf}
% 	\caption{Time vs Treewidth for computing Matchings}
% 	\label{fig:time_matchings}
% \end{figure}
% \begin{figure}[h!]
% 	\captionsetup{justification=centering}
% 	\centering
% 	\includegraphics[scale = 0.6]{./chapters/images/treewidth_time__perfect_matchings.pdf}
% 	\caption{Time vs Treewidth for computing Perfect Matchings}
% 	\label{fig:time_perfect_matchings}
% \end{figure}
\subsection{Comparision with baseline algorithms}
In this sectiom, we present a comparision of our parameterized algorithms with naive baseline algorithms (See Figure \ref{fig:baseline_cmp}). We present time vs edges characteristics for graphs selected uniformly at random from selected PubChem datasources. For conducting this comparision, we limited ourselves to graphs with at most $40$ edges, and for each edge size we sampled $50$ graphs from the datasources, and reported the average time for each edge size. 
\begin{figure}
	\centering
	\subfloat[]{\includegraphics[width=0.4\textwidth]{./chapters/images/perfect_matchings_comparision.pdf}} 
	\subfloat[]{\includegraphics[width=0.4\textwidth]{./chapters/images/matchings_comparision.pdf}} 
	\\ 
	\subfloat[]{\includegraphics[width=0.4\textwidth]{./chapters/images/independent_set_comparision.pdf}}
	\caption{Comparision with baseline algorithms}
	\label{fig:baseline_cmp}
\end{figure}
% \begin{figure}[h!]
% 	\captionsetup{justification=centering}
% 	\centering
% 	\includegraphics[scale = 0.6]{./chapters/images/perfect_matchings_comparision.pdf}
% 	\caption{Comparision of algorithms for perfect matchings}
% 	\label{fig:perfect_matchings_comparision}
% \end{figure}
% \begin{figure}[h!]
% 	\captionsetup{justification=centering}
% 	\centering
% 	\includegraphics[scale = 0.6]{./chapters/images/matchings_comparision.pdf}
% 	\caption{Comparision of algorithms for matchings}
% 	\label{fig:matchings_comparision}
% \end{figure}

% \begin{figure}[h!]
% 	\captionsetup{justification=centering}
% 	\centering
% 	\includegraphics[scale = 0.6]{./chapters/images/independent_set_comparision.pdf}
% 	\caption{Comparision of algorithms for independet set}
% 	\label{fig:independent_set_comparision}
% \end{figure}
\subsection{Implementation Details}\label{sec:implementation}
\subsubsection{Workflow}
We first downloaded SMILES \cite{weininger1988smiles} for all PubChem compounds available using FTP bulk download option. We used open source libraries (RDKit \cite{RDKit} and pysmiles \cite{pymsiles}) to parse these SMILES to Networkx graphs \cite{hagberg2008exploring} in Python3. Later, we used tree decomposition solver (FlowCutter) to obtain the tree decomposition for these graphs. Finally, we implemented our dynamic programming algorithms in C++ using these tree decompositions. All the statistical analysis for experiments are performed in Python3 and a detailed notebook will be provided in the resepective Zenedo repository.
\subsubsection{System details}
It took us nearly $3$ complete days to finish all the computations using a Linux based laptop with the following specifications: System Ubuntu 20.04.5 LTS and CPU AMD Ryzen 5 4600H with Radeon Graphics. Our entire experimental datasets and code repository will be available in Zenedo and GitHub.
% \begin{itemize}
% 	\item g++ 9.4.0
% 	\item System Ubuntu 20.04.5 LTS
% 	\item CPU AMD Ryzen 5 4600H with Radeon Graphics \url{https://www.amd.com/en/products/apu/amd-ryzen-5-4600h}
% 	\item Tree decomposition solver \url{https://github.com/kit-algo/flow-cutter} at commit ebb2427fdb69839c149f3e4ff184fb7112e324ba
% \end{itemize}

\section{Conclusion}\label{sec:Conclusion}
In summary, we presented parameterized algorithms for $\#P$-complete problems corresponding to topological indices and graph entropies. Our algorithms, beat the known state-of-the-art methods in both computer science and chemistry literature. We also provide a first-of-its kind statistical distribution of treewidths along with a comparision of our algorithms with the baseline algorithms for the entire PubChem database of chemical compounds. We hope that the techniques and analysis in this paper would serve as an invitation to both chemists and computer scientists to use parameterized methods for computationally challenging problems in chemistry. 

