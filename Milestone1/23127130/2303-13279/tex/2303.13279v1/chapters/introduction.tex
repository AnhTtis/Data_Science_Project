% !TEX root = ../main.tex
\section{Introduction}\label{sec:intro}

\subsection{Enumeration problems in chemistry}
We studied the enumeration problems on the molecular graphs in chemistry. We focused on the following problems:

\begin{itemize}
	\item \textbf{Enumeration of Kekulé structures} - The Kekulé structure for a molecule is essentially a perfect matching of the underlying graph. Therefore, enumeration of Kekulé structures is equal to computing total number of perfect matchings of the graph \cite{trinajstic2018chemical}.
	\item \textbf{Hosoya index} - The Hosoya index, also known as Z index, of a graph is the total number of matchings of the graph \cite{hosoya1971topological}. 
	\item \textbf{Merrifield–Simmons index} - The Merrifield–Simmons index of a graph is the total number of independent sets of the graph \cite{merrifield1980structures}.
	\item \textbf{Graph entropy based on matchings and independent sets} - The computation of graph entropy based on matchings and independent sets is equivalent to determining all matchings and independent sets for every possible size \cite{cao2017network}.
\end{itemize}

All these problems are known to be $\# P$-complete. The existing theoretical results for these problems are not useful for real-world applications in chemistry where graphs may have a large number of vertices and edges.

\subsection{Importance}
 Topological indices are numerical graph invariants that characterizes the topology of graph \cite{trinajstic2018chemical}. These indices play a vital role in computational chemistry. In particular, they are used as molecular descriptors for QSAR (Quantitative structure-activity relationship) and QSPR (Quantitative structure property relationships) \cite{dearden2017use}. In other words, the topology of hydrogen-supressed graphs of molecule can be used to quantitatively describe the physical and chemical properties. The problems under considerations are some of the well studied examples of molecular descriptors over the years in chemistry literature.
\subsection{Our contribution}
We provide new algorithms for the above mentioned problems that works in polynomial time for graphs that have small treewidth and small pathwidth. We can handle more than $99.9 \%$ of molecules in the PubChem database. Moreover, we also provide a statistical distribution of treewidth for the entire PubChem database \cite{kim2023pubchem}. Hence, we think this result would serve as an invitaion to both chemists and computer scientists to use parameterized algorithms more often for the problems arising in computional chemistry and biology.

\subsubsection{Parameterized Algorithms}
An efficient parameterized algorithm solves a problem in polynomial time with respect to the size of input,
but possibly with non-polynomial dependence on a specific aspect of the input’s structure, which is called a “parameter”. A problem that can be solved by an efficient parameterized algorithm is called fixed-parameter tractable (FPT).
\subsubsection{Pathwidth and Treewidth}
Pathwidth and treewidth are well studied parameters for graphs. Informally, pathwidth is the measure of path-likeness of a graph and treewidth is a measure of tree-likeness of a graph. Many hard graph problems in computer science are shown to have efficient solutions when restricted to graphs with bounded treewidth and bounded pathwidth. In this work we show that more than $99.9 \%$ molecules in the enitre PubChem database have treewidht less than $5$, less than $100$ molecules have treewidth greater than $20$, in the enitre database of more than $100$ million compounds. For more detailed statistics for the entire database refer to Section \ref{sec:Experiments}. Therefore, supporting our argument of developing parameterized algorithms for bounded treewidth and pathwidth for the above mentioned problems.



\subsection{Related works}
\begin{itemize}
	\item In theory, there are known FPT algorithms using bounded treewidth as parameter for computing matchings and independent using monadic second order logic \cite{courcelle2001fixed}. But it is impractical to implement and use them for graphs with large number of vertices and edges arising in chemistry.
	\item In \cite{wan2018computing}, the authors have provided explicit parameterized algorihtms for computing graph entropies corresponding to Hosoya Index and Merrifield–Simmons Index. But our algorithms, are better than their approach by a linear factor for computing the graph entropies and by a quadratic factor for computing the corresponding indices. We also provide a practical implementaion for all the graphs obtained from PubChem database.
\end{itemize}

\subsection{Structure of the paper}
The paper is divided into three main sections. We first start with basic notations and preliminaries in Section \ref{sec:prelim}. Our main algorithms and theoretical reults are described in Section \ref{sec:algos}. We provide all the experimental results, statistics and implementation details in Section \ref{sec:Experiments}. Finally, we end the paper with conclusion in Section \ref{sec:Conclusion}. 