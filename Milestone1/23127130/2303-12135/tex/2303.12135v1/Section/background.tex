\section{Background}

\subsection{Sequential Sentence Classification}

Sequential sentence classification is a natural language processing (NLP) technique that involves classifying a sequence of sentences into one or more categories or labels~\cite{hassan2017deep,cohan2019pretrained,jin2018hierarchical,brack2022cross}. 
The objective of this technique is to analyze and classify the content of a given text document based on the semantics and context of the sentences~\cite{jin2022understanding}.
It is often used in text classification tasks, such as sentiment analysis, spam detection, and topic classification, where the classification of a single sentence can depend on the preceding or succeeding sentences~\cite{hassan2017deep}. This technique can be implemented using various algorithms, such as recurrent neural networks (RNNs) or long-short-term memory (LSTM) networks, which are capable of processing sequential data~\cite{lipton2015critical}.

In sequential sentence classification, the input is a sequence of sentences and the output is one or more labels that describe the content of the document. The classification can be performed at the sentence level or at the document level, depending on the specific use case~\cite{cohan2019pretrained}. Sequential sentence classification is an important technique in NLP because it enables machines to understand and analyze the meaning and context of human language, which is crucial for many applications such as automated text summarization and question-answering systems~\cite{qiu2020pre}.

\subsection{Legal Named Entity Recognition}

The objective of named entity recognition in the legal domain is to detect and label all instances of specific legally relevant named entities within unstructured legal reports~\cite{nadeau2007survey,kalamkar2022named,li2020survey}. 
Using this named entity information, one can analyze, aggregate, and mine data to uncover insightful patterns. 
Furthermore, the ultimate goal of legal document analysis is to automate the process of information retrieval or mapping a legal document to one or more nodes of a hierarchical taxonomy or legal cases, in which legal NER plays a significant role.~\looseness=-1 

Given that legal reports contain a large number of complex medical terms and terminologies, such as statute and precedent, identifying expressions referring to anatomies, findings, and anatomical locations is a crucial aspect in organizing information and knowledge within the legal domain~\cite{kalamkar2022named}. 
Therefore, automating this identification process has been recognized as a key target for automation. 
Moreover, automatic named entity recognition (NER) helps exhaustively extract semantic information and misspelling checking~\cite{kalamkar2022named,legaleval-2023}.