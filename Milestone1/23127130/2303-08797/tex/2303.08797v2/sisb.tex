\label{sec:sisb}


Recently, there has been a surge of interest in the construction of generative models through diffusive bridge processes~\cite{bortoli2021diffusion,peluchetti2022nondenoising,liu2022let, somnath2023aligned}. In this section, we connect these past approaches with our own, highlighting that stochastic interpolants allow us to apply bridge processes in a simpler and more direct manner.

\paragraph{Bridge processes.}  Consider a bridge process $B^{x_0,x_1}_{[0,1]} \equiv \{B^{x_0,x_1}_t:t\in [0,1]\}$ between $x_0\in \RR^d$ and $x_1\in \RR^d$. Here, we define such a bridge process as an element of the sample space $(C^0([0,1]))^d$ (endowed with its canonical $\sigma$-algebra) that satisfies $B^{x_0,x_1}_{t=0}=x_0$ and $B^{x_0,x_1}_{t=1}=x_1$ almost surely. For bridges built on SDEs, this can be achieved via Doob's $h$-transform~\cite{doob1984potential}, as discussed below. We can define the stochastic interpolant associated with any such bridge by randomizing its end points independently from the realization of the bridge itself, i.e. by setting
\begin{equation}
    \label{eq:sisb}
   \forall t \in (0,1) \quad : \quad  x^\brdg_t = B_t^{x_0,x_1} \quad \text{with} \ x_0\sim \rho_0, \ x_1\sim \rho_1 \quad \text{independent of \ $B_{[0,1]}^{x_0,x_1}$}.
\end{equation}

\paragraph{Generality.} The construction in~\eqref{eq:sisb} may appear more general than Definition~\ref{def:interp}, but it fits within our framework because only the single-time statistics of $x^\brdg_t$ are required to derive generative models. Specifically, let $\mu^{x_0,x_1}_t(dx)$ denote the probability distribution of $B^{x_0,x_1}_t$ at any fixed time $t\in[0,1]$, and assume that  $\mu^{x_0,x_1}_t(dx)$ is absolutely continuous with respect to the Lebesgue measure for all $t\in(0,1)$ with density $\rho^{x_0,x_1}(t,x)$\footnote{By construction, we  have $\mu^{x_0,x_1}_{t=0}(dx) = \delta_{x_0} (dx)$  and $\mu^{x_0,x_1}_{t=1}(dx) = \delta_{x_1} (dx)$, so this distribution has no density at the end points.}. Then the bridge process defined in~\eqref{eq:sisb} is such that  $x^\brdg_{t=0} \sim \rho_0$, $x^\brdg_{t=1}\sim \rho_1$, and
\begin{equation}
    \label{eq:sisb:pdf}
   \forall t \in (0,1) \quad : \quad x^\brdg_t \sim \rho(t,\cdot) \quad \text{with} \quad \rho(t,x) = \int_{\RR^d\times\RR^d} \rho^{x_0,x_1}(t,x) \rho_0(x_0)\rho_1(x_1) dx_0 dx_1.
\end{equation}
Importantly, \textit{any} process with the same density $\rho(t)$ leads to the same generative model. Rather than starting from~\eqref{eq:sisb}, it is therefore equivalent to use a stochastic interpolant of the form~\eqref{eq:stochinterp} with the same $\rho(t)$. However,~\eqref{eq:stochinterp} can often be simpler, because it provides a computationally efficient way to sample $\rho(t)$ at any $t$ only using samples from $\rho_0$ and $\rho_1$ that does not require simulating bridge processes with temporal correlations.

\paragraph{The Brownian bridge.} To elaborate on this point, we now consider the simplest instantiation of a stochastic bridge. Assume that  $B^{x_0,x_1}_t $ is the (scaled and shifted) Wiener process $x_0+\sqrt{2} W_t$ conditioned on ending at $x_1$. In this case, $B^{x_0,x_1}_t $ can be realized in terms of $W_t$ as $B^{x_0,x_1}_t =  (1-t) x_0 + t x_1 + \sqrt{2}(W_t - t W_{t=1})$, meaning that~\eqref{eq:sisb} becomes
\begin{equation}
    \label{eq:sisb:2}
    x^\brdg_t =  (1-t) x_0 + t x_1 + \sqrt{2}(W_t -  t W_{t=1})\quad \text{with} \ x_0\sim \rho_0, \ x_1\sim \rho_1.
\end{equation}
In this example, we have
\begin{equation}
    \label{eq:bbrgo}
    \mu^{x_0,x_1}(t)  = {\sf N}((1-t)x_0+ tx_1, 2t(1-t)),
\end{equation}
which implies that the stochastic interpolant~\eqref{eq:sisb} has the same single-time statistics and time-dependent density as
\begin{equation}
\label{eq:bb}
x_t = (1-t) x_0+ t x_1+ \sqrt{2t(1-t)} z \quad \text{with} \ x_0\sim \rho_0, \ x_1\sim \rho_1, \ z \sim {\sf N}(0,\Id),
\end{equation}
which we previously considered in~\eqref{eq:lin:a:b:c}. As a result,~\eqref{eq:sisb:2} and \eqref{eq:bb} lead to the same generative models.
However, it is easier to work with~\eqref{eq:bb} than it is to work with~\eqref{eq:sisb:2}, because it avoids the use of It\^o calculus and enables direct sampling of $x_t$. For completeness, we now re-derive the transport equation for the density $\rho(t)$ starting from~\eqref{eq:sisb:2}, highlighting that it leads to the same result. 

Recall that the Brownian Bridge $B_t = W_t - tW_{t=1}$ satisfies the SDE (this is the SDE obtained by conditioning on $B_{t=1}=0$ via Doob's $h$-transform~\cite{doob1984potential}):
\begin{equation}
    \label{eq:dobb}
    dB_t = - \frac{B_t} {1-t} dt + dW_t, \qquad B_{t=0}=0.
\end{equation}
Since $x^\brdg_t =  (1-t) x_0 + t x_1+ \sqrt{2} B_t$ from~\eqref{eq:sisb:2}, a direct application of It\^o's formula implies that
\begin{equation}
    \label{eq:dobb:2}
    de^{ik\cdot x_t^\brdg} = ik \cdot \Big( -x_0 + x_1 - \frac{\sqrt{2}B_t} {1-t} \Big) e^{ik\cdot x_t^\brdg}  dt - |k|^2 e^{ik\cdot x_t^\brdg}  dt + \sqrt{2}ik\cdot dW_t e^{ik\cdot x_t^\brdg}.
\end{equation}
Taking the expectation of this expression and using the independence between $x_0$, $x_1$, and $B_t$, we deduce that
\begin{equation}
    \label{eq:dobb:3}
    \partial_t \EE e^{ik\cdot x_t^\brdg} = ik \cdot \EE \Big(\Big( -x_0 + x_1 - \frac{\sqrt{2}B_t} {1-t} \Big) e^{ik\cdot x_t^\brdg} \Big)  - |k|^2 \EE e^{ik\cdot x_t^\brdg}.
\end{equation}
Since for all fixed $t\in [0,1]$ we have $B_t \stackrel{d}{=} \sqrt{t(1-t)}z$ and $x^\brdg_t \stackrel{d}{=} x_t = (1-t) x_0+ t x_1+ \sqrt{2t(1-t)} z $, \eqref{eq:dobb:3} can also be written as
\begin{equation}
    \label{eq:dobb:4}
    \partial_t \EE e^{ik\cdot x_t} = ik \cdot \EE \Big(\Big( -x_0 + x_1 - \frac{\sqrt{2t}\, z} {\sqrt{1-t} }\Big) e^{ik\cdot x_t} \Big)  - |k|^2 \EE e^{ik\cdot x_t}.
\end{equation}
Moreover, since $\EE e^{ik\cdot x_t^\brdg}=\EE e^{ik\cdot x_t} = \int_{\RR^d} e^{ik\cdot x}\rho(t,x)dx$, we can deduce from~\eqref{eq:dobb:4} that $\rho(t)$ satisfies
\begin{equation}
    \label{eq:dobb:5}
    \partial_t \rho+\nabla \cdot(u \rho) = \Delta \rho,
\end{equation}
where we defined
\begin{equation}
    \label{eq:u:def}
    u(t,x) = \EE\big( -x_0 + x_1 - \frac{\sqrt{2t} \, z} {\sqrt{1-t}}| x_t = x\big).
\end{equation}
Since we know  from the definition of $b$ and $s$ in~\eqref{eq:b:ode:def} and \eqref{eq:s:def} with $I(t,x_0,x_1) = (1-t) x_0 + t x_1$ and $\gamma(t) = \sqrt{2t(1-t)} $ that, for the interpolant $x_t$ in~\eqref{eq:bb}, we have
\begin{equation}
    \label{eq:sbdoob}
    \begin{aligned} 
    b(t,x) &= \EE\big( -x_0 + x_1 + \frac{(1-2t) z} {\sqrt{2t(1-t)}}| x_t = x\big),\\
    s(t,x) &= \nabla \log\rho(t,x) = - \frac1{\sqrt{2t(1-t)}}\EE(z| x_t = x),
    \end{aligned}
\end{equation}
it follows that $u-s=b$. As a result, it is easy to see that~\eqref{eq:dobb:5} can also be written as the TE~\eqref{eq:transport} using $\Delta \rho = \nabla \cdot(s \rho)$.

\paragraph{Extensions.} In principle, the above approach can be generalized to any stochastic bridge $B_t^{x_0,x_1}$, which can be obtained from any SDE by conditioning with the help of Doob's $h$-transform. In general, however, this construction cannot be made explicit, because the $h$-transform is typically not available analytically. One approach would be to learn it, as proposed in~\cite{peluchetti2022nondenoising,bortoli2021diffusion}, but this adds an additional layer of difficulty that we feel is unnecessary.

If we restrict ourselves to Gaussian bridges, a more practical way to proceed, in line with the approach proposed in this paper, is to generalize the definition of the stochastic interpolant to allow for a time- and state-dependent latent \textit{tensor}
\begin{equation}
\label{eq:gb}
x_t = I(t,x_0,x_1)  + \sigma(t,x_0,x_1) z,
\end{equation}
where $I$ is as in~\eqref{eq:stochinterp}, $z\sim {\sf N}(0,\Id)$, and $\sigma \in C^2([0,1],(C^2(\RR^d\times\RR^d))^{d\times d})$ satisfies the boundary conditions $\sigma(0,x_0,x_1) = 0$ and $\sigma(1,x_0,x_1) = 0$. This defines a Gaussian bridge with
\begin{equation}
    \label{eq:bbrgaus}
    \rho^{x_0,x_1}(t)  = {\sf N}(I(t,x_0,x_1), K(t,x_0,x_1)),
\end{equation}
where $K(t,x_0,x_1)= \sigma(t,x_0,x_1)\sigma^\T(t,x_0,x_1)$. The generalized stochastic interpolants defined in~\eqref{eq:bbrgaus} offers some additional design flexibility that may be useful for some problem settings, and all theoretical results presented in the preceding sections can be readily extended to handle the extra level of generality.