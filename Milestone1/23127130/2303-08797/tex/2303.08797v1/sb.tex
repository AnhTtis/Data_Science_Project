\label{sec:sb}

In the section, we show how our framework can be used to solve the Schro\"dinger bridge problem. For background material on this problem, we refer the reader e.g. to~\cite{leonard2014survey}. Consistent with the  overall viewpoint of this paper, we consider  the fluid dynamics formulation of  the Schro\"dinger bridge problem in which one focus on finding a pair PDF/velocity, $(\rho,u)$, that solve the following optimization problem (here $\eps>0$)
 \begin{equation}
     \label{eq:min:sb}
     \begin{aligned}
         &\min \int_0^1 \int_{\RR^d} |\nabla \hat u(t,x)|^2 \hat \rho(t,x) dx dt\\
         \text{subject to:} \quad & \partial_t \hat \rho + \nabla \cdot \left(\hat u \hat\rho\right) = \eps  \Delta \hat \rho, \quad \hat \rho(0) = \rho_0\quad \hat \rho(1) = \rho_1
     \end{aligned}
 \end{equation}
 Under our assumptions on $\rho_0$, $\rho_1$ listed in Assumption~\ref{as:rho:I}, it is known (see e.g. Proposition 4.1 in~\cite{leonard2014survey}) that problem \eqref{eq:min:sb} has a unique minimizer $(\rho,u=\nabla\lambda)$, with $(\rho,\lambda)$ classical solutions of the Euler-Lagrange equations:
 \begin{equation}
    \label{eq:max:min:rho:j:el:2}
    \begin{aligned}
        & \partial_t \rho + \nabla \cdot \left(\nabla \lambda \rho\right) = \eps  \Delta \rho, \quad \rho(0) = \rho_0,\quad \rho(1) = \rho_1,\\
        & \partial_t \lambda +\tfrac12 |\nabla \lambda|^2=- \eps  \Delta \lambda.
    \end{aligned}
\end{equation}
To proceed we will make the additional assumption that the solution $\rho$ to \eqref{eq:max:min:rho:j:el:2} can be reversibly mapped to a standard Gaussian: 
\begin{assumption}
 \label{as:sb:2}
There exists of a reversible map $T:[0,1]\times \RR^d \to \RR^d$ with $T,T^{-1}\in C^1([0,1], (C^d(\RR^d))^d)$ such that:
\begin{equation}
    \label{eq:interpol}
    \forall t\in [0,1] \quad : \quad z \sim {\sf N}(0,\Id) \ \Rightarrow \ T(t,z) \sim \rho(t); \quad x_t \sim \rho(t) \ \Rightarrow \ T^{-1}(t,x_t) \sim {\sf N}(0,\Id),
\end{equation}
where $\rho$ is the solution to~\eqref{eq:max:min:rho:j:el:2}.
\end{assumption}
We stress that the actual form of the map $T$ is not important for the arguments below.
Assumption~\ref{as:sb:2}  can be used to show the existence of a stochastic interpolant whose PDF solves~\eqref{eq:max:min:rho:j:el:2}:
\begin{restatable}{lemma}{interppdf}
\label{lem:interp}
If Assumption~\eqref{as:sb:2}  holds, then the solution $\rho(t)$ to~\eqref{eq:max:min:rho:j:el:2} is the PDF of the stochastic interpolant
\begin{equation}
    \label{eq:stoch:interpolable}
    x_t = T(t, \alpha(t) T^{-1}(0,x_0) + \beta(t) T^{-1}(1,x_1)) + \gamma(t) z,
\end{equation}
as long as $\alpha^2(t)+ \beta^2(t) + \gamma^2(t) =1$.
\end{restatable}

This lemma is proven in Appendix~\ref{app:sb}: \eqref{eq:stoch:interpolable} corresponds to choosing $I(t,x_0,x_1)= T(t, \alpha(t) T^{-1}(t,x_0) + \beta(t) T^{-1}(t,x_1))$ in~\eqref{eq:stochinterp}. With the help of Lemma~\ref{lem:interp}, we can establish  the following result showing how to optimze over the function $I$ to solve the Schr\"odinger bridge problem:

\begin{restatable}{theorem}{sb}
    \label{prop:sb}
    Pick some $\gamma:[0,1]\to [0,1)$ such that $\gamma(0)=\gamma(1) = 0$, $\gamma(t)>0$ for $t\in(0,1)$, $\gamma \in C^2((0,1))$ and $\gamma^2\in C^1([0,1])$, and let $\hat x_t = \hat I(t,x_0,x_1)+\gamma(t) z$, with $x_0\sim\rho_0$, $x_1\sim\rho_1$, $z\sim {\sf N}(0,\Id)$, all independent.  Consider the max-min problem over $\hat I\in C^1([0,1],(C^1(\RR^d\times\RR^d))^d)$ and $\hat u\in C^0([0,1],(C^1(\RR^d))^d)$: 
\begin{equation}
    \label{eq:max:min}
    \max_{\hat I} \min_{\hat u} \int_0^1 \EE \left( \tfrac12|\hat u(t,\hat x_t)|^2 - \left(\partial_t \hat I(t,x_0,x_1) + (\dot \gamma(t) -\eps \gamma^{-1}(t)) z \right) 
    \cdot \hat u(t,\hat x_t) \right) dt,
\end{equation}
If Assumption~\ref{as:sb:2} holds, then all the optimizers $(I,u)$ of~\eqref{eq:max:min} are such that: the PDF of the associated $x_t = I(t,x_0,x_1)+\gamma(t) z$ is the solution $\rho$ to~\eqref{eq:max:min:rho:j:el:2};
and $u = \nabla \lambda$, with $\lambda$  solution to~\eqref{eq:max:min:rho:j:el:2}.
\end{restatable} 

This theorem is also proven in Appendix~\ref{app:sb}. Note that if we fix $\hat I$, the velocity $u$ minimizing this objective is the forward drift $b_\fwd$ defined in~\eqref{eq:b:def}. Note also that of we set $\eps\to0$, this minimizing velocity field is the velocity field~$b$ defined in~\eqref{eq:b:ode:def}, and the max-min problem formally reduces to solving the optimal transport problem: in this case however Assumption~\ref{as:sb:2} becomes more stringent as we need to assume that \eqref{eq:max:min:rho:j:el:2} with $\eps=0$ (i.e. in the absence of the diffusive terms) have  a classical solution. Theorem~\eqref{prop:sb} gives us a practical route towards solving the Schr\"odinger bridge problem with stochastic interpolant. We will leave the numerical investigation of this formulation to future work. 
