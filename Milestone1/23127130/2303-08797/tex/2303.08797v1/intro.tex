\subsection{Background and motivation}
\label{sec:back:mot}

Dynamical approaches for deterministic and stochastic transport have become a central theme in contemporary generative modeling research. At the heart of progress is the idea to apply ordinary or stochastic differential equations (ODE/SDE) to continuously transform samples from a base probability density function (PDF) $\rho_0$ into samples from a target density $\rho_1$ (or vice-versa), and the realization that inference over the velocity field in these equations can be formulated as empirical risk minimization over a parametric class of functions \cite{grathwohl2018scalable,Song2019,ho2020,song2021scorebased,benhamu2022,albergo2023building,liu2022,lipman2022}.

A major milestone was the introduction of score-based diffusion methods (SBDM) \cite{song2021scorebased}, which map an arbitrary density into a Gaussian by passing samples through an Ornstein-Uhlenbeck (OU) process. The key insight of SBDM is that this process can be reversed by introducing a backwards SDE whose drift coefficient depends on the score of the time-dependent density of the process. By learning this score -- which can be done by minimization of a quadratic objective function known as the denoising loss~\cite{vincent_connection_2011} -- the backwards SDE can be used as a generative model that maps Gaussian noise into data from the target. Though theoretically exact, the mapping takes infinite time in both directions, and hence must be truncated in practice.

While diffusion-based methods have become state-of-the-art for tasks such as image generation, there remains considerable interest in developing methods that bridge two \textit{arbitrary} densities (rather than requiring one to be Gaussian), that accomplish the transport \textit{exactly}, and that do so on a \textit{finite} time interval.
%
Moreover, while highest quality results were originally obtained using stochastic sampling techniques for score-based diffusion methods based on SDEs~\cite{song2021scorebased}, this has been challenged by recent works that find equivalent or better performance with deterministic sampling techniques based on ODEs if the score is learned sufficiently well~\cite{Karras2022edm}.
%
If made to match the performance of their stochastic counterparts, ODE-based methods exhibit a number of desirable characteristics that are absent for SDEs, such as an exact, computationally tractable formula for the likelihood and easy application of well-developed adaptive integration schemes for sampling. 
%
It is an open question of significant practical importance to understand if there exists a separation in sample quality between generative models based on deterministic dynamics and those based on stochastic dynamics. 



In order to satisfy the desirable characteristics outlined in the previous paragraph, here we develop a framework for generative modeling based on  the method proposed in~\cite{albergo2023building}, with two  modifications -- one that improves the baseline performance, and one that enables us to quantify explicitly and explore empirically the tradeoffs between stochastic and deterministic models.
The approach is built on the notion of a \textit{stochastic interpolant}~$x_t$ used to bridge  two arbitrary densities $\rho_0$ and $\rho_1$.  We will consider more general designs below, but to fix idea the reader can keep in mind:
\begin{equation}
    \label{eq:stoch:interp:lin}
    x_t = (1-t) x_0 + t x_1 + \sqrt{2t(1-t)} z, \quad t \in [0,1],
\end{equation}
where $x_0$, $x_1$, and $z$ are random variables drawn independently from $\rho_0$, $\rho_1$, and the standard Gaussian density $\mathsf{N}(0,\Id)$, respectively. The stochastic interpolant~$x_t$ defined in~\eqref{eq:stoch:interp:lin} is a continuous-time stochastic process which, by construction, satisfies $x_{t=0} = x_1\sim \rho_1$ and $x_{t=1} = x_1\sim \rho_1$. Its paths therefore exactly bridge between samples  from $\rho_0$ at $t=0$ and from $\rho_1$ at $t=1$, without any bias. A key observation is that:
\begin{quote}
    \textit{The law of the interpolant $x_t$ at any time $t\in[0,1]$ can be realized by many different processes, including an ODE and forward and backward SDEs whose drift coefficients can be learned from data.}
\end{quote} 
To see why this is the case, one must consider the probability distribution of the interpolant $x_t$; as shown below, for a large class of densities $\rho_0$ and $\rho_1$ supported on $\RR^d$, this distribution is absolutely continuous with respect to the Lebesgue measure, and its time-dependent density~$\rho(t)$ satisfies a first-order transport equation, as well as forward and backward Fokker-Planck equations in which the diffusion coefficient can be varied at will. Out of these equations, we can readily derive deterministic and stochastic processes satisfying ODEs and SDEs, respectively, whose densities at time~$t$ are given by $\rho(t)$ and hence coincide in law with the original~$x_t$.

\begin{figure}[t]
    \centering
    \includegraphics[width=\linewidth]{figs/front_figure-final.pdf}
    \caption{A generative model based on the proposed stochastic interpolant framework connecting two densities with no analytic form. Designing the time-dependent probability density bridging these densities  and learning the drift coefficients in its evolution equations is independent of choosing how to sample this density via deterministic or stochastic generative models. \textit{Left panel:} sampling with the probability flow ODE. \textit{Right panel:} Sampling with the SDE with arbitrary noise amplitude, using the same drift as the ODE.}
    \label{fig:my_label}
\end{figure}

Interestingly, the drift coefficients entering these ODE/SDE are the unique minimizers of qua\-dra\-tic objective functions that can be  estimated empirically using data from $\rho_0$, $\rho_1$, and $\mathsf{N}(0,\Id)$. The resulting least-squares regression problem allows us to estimate the drift coefficients of the ODE/SDE, which can then be used to push samples from $\rho_0$ onto new samples from $\rho_1$ and vice-versa. 

\subsection{Main contributions and organization}
Overall, the approach introduced in this paper is a versatile way to build generative models, with many attractive features that we now summarize.

%The method proposed in this paper has several appealing theoretical and practical properties:
\begin{itemize}[leftmargin=0.2in]
    \item Due to the inclusion of the latent variable, the stochastic interpolant defined in Section~\ref{sec:si:gm} has a probability distribution that is absolutely continuous with respect to the Lesbegue measure, with a density that satisfy a first order transport equation (TE) as well as forward and backward Fokker-Planck equations (FPE) with  tunable diffusion coefficients. These equations are given in Section~\ref{sec:cont:eq}.
    \item The drift coefficients entering the TE and the FPE are smoothed spatially by the presence of the latent variable. These coefficients are also the unique minimizers of quadratic objective functions, given in Section~\ref{sec:cont:eq}, which are readily amenable to empirical estimation using the available data.
    \item Due to the inclusion of the latent variable, our approach gives a new loss for the score of the time-dependent PDF of the interpolant, which we give in Section~\ref{sec:cont:eq}. This score enters as one component of the drifts in the FPE.
    \item We can readily derive ordinary differential equations (ODE) as well as forward and backward stochastic differential equations (SDE) associated with the TE and the forward and backward FPE, respectively. These ODE/SDE are given in Section~\ref{sec:generative} and can be used as deterministic or stochastic generative models, with the possibility to again tune their level of diffusivity. 
    \item We show that the approach controls the likelihood of the SDE-based models, generalizing the ScoreFlow approach from score-based diffusion~\cite{song2021mle}. By contrast, regressing the drift alone is insufficient in general to bound the  likelihood  with ODE-based models, which require more advanced learning schemes to ensure that the Fisher divergence is also minimized. This is discussed in Section~\ref{sec:likelihood_bounds}, where we also show how to optimally tune the level of diffusivity as a function of the error of two components of the drifts.
    \item We develop a general formula for likelihood evaluation of generative models based on stochastic differential equations that serves as a natural counterpart to the continuous change-of-variables formula that is commonly used to compute the likelihood of a deterministic flow. This is discussed in Section~\ref{sec:density}, where we also show how to estimate the cross-entropy.
    \item The flexibility of design of the stochastic interpolant is highlighted in Section~\ref{sec:practical}, where we show how the latent variable can be adapted to various tasks (Section~\ref{sec:reg:noise}) and how the diffusivity can be tuned for better accuracy (Section~\ref{sec:sde:ode}), confirming the picture from the likelihood bounds derived earlier.
    \item Stochastic interpolants admit a simplified, one-sided version in the special case when $\rho_1$ is a Gaussian density. This is discussed in Section~\ref{sec:onesided}, and these one-sided stochastic interpolants allows us to compare our approach with score-based diffusion models (SBDM) in Section~\ref{sec:SBDM}. 
    \item Our approach is amenable to a bias-free variant of the rectification procedure proposed in~\cite{liu2022}, as discussed in Section~\ref{sec:rect}.
    \item Our approach solves the Schr\"odinger bridge problem between two densities when maximizing the loss over the interpolant, as discussed in Section~\ref{sec:sb}.
    \item We highlight the performance of the method on synthetic examples throughout the paper, in particular using Gaussian mixture models for which the drifts are available analytically (as shown in Appendix~\ref{app:Gauss:mixt}).
\end{itemize}

The above list of features and contributions shows that the proposed method conveniently houses many modeling goals under one roof: it forms a connection between arbitrary densities (allowing for the incorporation of prior knowledge and to directly perform data-to-data translation), reaches the target density on a finite time interval, is versatile in the way it can be adapted to various tasks by exploiting the inherent flexibility in the choice of interpolant, and remains bias-free for any choice of the latent variable amplitude and the noise strength, which can both be tuned as model hyper-parameter after training. The method therefore allows us to theoretically and empirically explore the best design choices for learnable  diffusive processes, as well as the resulting trade-off between ODE and SDE methods for generative models. 

\subsection{Related work}
\section{Related work}
\noindent \textbf{Video foundation models.}
With sufficient computational power and an abundant source of data, there have been attempts to build a single large-scale foundation model that can be adapted to diverse downstream tasks.
Along with the success of foundations models in the natural language processing domain~\cite{brown2020language,chen2021evaluating,devlin2019bert} and in computer vision~\cite{bertasius2021space,jia2021scaling,radford2021learning}, video data has become another data type of interest, as it has grown in scale due to numerous internet video-sharing platforms.
Accordingly, several methods to train a video foundation model have been proposed.
Due to the innate multi-modality of video data, \textit{i.e.}, a combination of visual $\cdot$ vocal $\cdot$ textual context, most works have centered around the variations of the cross-modal attention mechanism \cite{akbari2021vatt,bertasius2021space,gabeur2020multi,luo2020univl,neimark2021video,tan2021look,wei2020multi,yang2021taco}.
In addition, as most video data lack proper labels or descriptions, contrastive learning methods were studied to learn meaningful feature representations or enhance video-text alignment in a self-supervised manner \cite{akbari2021vatt,kuang2021video,luo2020univl,yang2021taco}.

More specifically, MERLOT \cite{zellers2021merlot} proposed a multi-modal representation learning method for visual commonsense reasoning, which also performed well in twelve video reasoning tasks.
VATT \cite{akbari2021vatt} introduced a multi-modal learning method via contrastive learning. 
The pre-trained model performed well in a variety of vision tasks from image classification to video action recognition and zero-shot video retrieval.
Another representative work, UniVL \cite{luo2020univl} proposed a straightforward pre-training method with auxiliary loss functions. 
After fine-tuning on a specific task, the pre-trained model performed outstandingly in a wide range of tasks of text-to-video retrieval, action segmentation, action step localization, video sentiment analysis, and video captioning.
Other foundation models for multiple video tasks include \cite{li2020hero,sun2019learning,sun2019videobert,zhu2020actbert,fu2021violet,wang2022all}. 

\noindent \textbf{Auxiliary learning.}
In order to enhance the performance of one or a multitude of primary tasks, auxiliary learning methods can be incorporated.
\cite{ruder2017overview} introduced Multi-task learning (MTL) to the deep neural networks by training a single model with multiple task losses to assist learning on the main task.
Such a method is generally adapted to pre-train the foundation models in the self-supervised manner~\cite{li2020hero,sun2019learning,sun2019videobert,zhu2020actbert,fu2021violet,wang2022all}.
However, these various pretext task losses used in the pre-training phase are ignored in the fine-tuning phase, and only the primary task loss is minimized.

Recently, meta-learning methods have been introduced for auxiliary learning.
\cite{liu2019self,navon2020auxiliary,shu2019meta} proposed a meta-learning method in which the model learns auxiliary tasks to generalize well to unseen data. 
In these settings, a separate subset of data is held out as the primary task, while the others are used as auxiliary tasks that aid the primary task's performance.
Similar methods were adopted for computer vision tasks such as semantic segmentation \cite{xu2021leveraging}.
Other domain applications include navigation tasks with reinforcement learning \cite{ye2021auxiliary}, or self-supervised learning methods on graph data \cite{hwang2020self}.

\subsection{Notations}
\label{sec:notations}

Thorough we denote probability density functions as $\rho_0(x)$, $\rho_1(x)$, $\rho(t,x)$, with $t\in[0,1]$ and $x\in\RR^d$, omitting the function arguments when this leads to no confusion. We proceed similarly for other functions of time and space, such as $b(t,x)$ or $I(t,x_0,x_1)$. We use the subscript $t$ to denote the time-dependency of stochastic processes, like e.g. the stochastic interpolant $x_t$ or the Wiener process $W_t$. To specify that the random variable $x_0$ is drawn for the probability distribution with density $\rho_0$, say, with a slight abuse of notations we use $x_0\sim\rho_0$. Similarly, we use ${\sf N}(0,\Id)$ to denote both the density and the distribution of the Gaussian random variable with mean zero and covariance identity. We denote expectation by $\EE$, and usually specify what are the random variables this expectation is taken over. With a slight abuse of terminology, we say that the law of the process $x_t$ is $\rho(t)$ if $\rho(t)$ is the density of the probability distribution of $x_t$ at time $t$.

We use standard notations for function spaces, e.g. $C^1([0,1])$ is the space of continuously differentiable functions from $[0,1]$ to $\RR$, $(C^2([0,1])^d$ the space of twice continuously differentiable functions from $\RR^d$ to $\RR^d$, and $C^\infty_0(\RR^d)$ the space of compactly supported, infinitely differentiable function from $\RR^d$ to $\RR$. Given a function $b:[0,1]\times \RR^d \to \RR^d$ with value $b(t,x)$ at $(t,x)$, we  use e.g. $b \in C^1([0,1]; (C^2(\RR^d))^d)$ to indicate that $b$  is continuously differentiable in $t$ for all $(t,x)\in[0,1]\times \RR^d$, and that $b(t,\cdot)$ is an element of $(C^2(\RR^d))^d$ for all $t\in[0,1]$. 
