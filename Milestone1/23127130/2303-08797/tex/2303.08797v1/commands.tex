% \newcommand{\draftnote}[1]{{\color{red} #1}}
% \newcommand{\todo}[1]{{\color{red} [todo: #1]}}
\newcommand{\eric}[1]{{\color{orange}{[eve:} #1]}}
\newcommand{\michael}[1]{{\color{blue} [Michael: #1]}}
\newcommand{\nb}[1]{{\color{olive}{[nb: #1}]}}
\newcommand{\fwd}{\mathsf{F}}
\newcommand{\rev}{\mathsf{B}}
\newcommand{\ODE}{{}}%{\mathsf{D}}
\newcommand{\OS}{\mathsf{os}}
\newcommand{\T}{\mathsf{T}}
\newcommand{\KL}[2]{\mathsf{KL}(#1\: \Vert \: #2)}
\newcommand{\fisher}[2]{\mathsf{FI}(#1\: \Vert \: #2)}
\newcommand{\norm}[1]{\left|#1\right|}
\newcommand{\Law}{\mathsf{Law}}
%\newcommand{\Id}{\mathsf{Id}}

\newcommand{\Id}{\text{\it Id}}


\DeclareMathOperator{\supp}{supp}
% \DeclareMathOperator{\Id}{Id}


\def\CC{{\mathbb{C}}}
\def\EE{{\mathbb{E}}}
\def\NN{{\mathbb{N}}}
\def\PP{{\mathbb{P}}}
\def\RR{{\mathbb{R}}}
\def\eps{\epsilon}
\def\seps{\sqrt{2\epsilon}}


%\newtheorem{theorem}{Theorem}
%\newtheorem{corollary}[theorem]{Corollary}
%\newtheorem{proposition}[theorem]{Proposition}
%\newtheorem{assumption}{Assumption}
%\newtheorem{definition}{Definition}
%\newtheorem{lemma}[theorem]{Lemma}
%\newtheorem{claim}[theorem]{Claim}
%\newtheorem{remark}[theorem]{Remark}
%\newtheorem{example}{Example}
%\newtheorem{ansatz}{Ansatz}

\declaretheorem[numberwithin=section]{theorem}
\declaretheorem[numberlike=theorem]{definition}
\declaretheorem[numberlike=theorem]{corollary}
\declaretheorem[numberlike=theorem]{proposition}
\declaretheorem[numberlike=theorem]{assumption}
\declaretheorem[numberlike=theorem]{lemma}
\declaretheorem[numberlike=theorem, style=remark]{remark}

\numberwithin{equation}{section}