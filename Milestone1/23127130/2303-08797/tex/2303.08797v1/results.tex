\label{sec:numerics}
In the following, we benchmark the proposed method. We split these tests into two parts: (1) one for which the analytical solutions are known for $v,s$, allowing us to more directly probe the ODE/SDE tradeoff, and (2) another in which we demonstrate the learnability of $v,s$ in a parametric class of neural networks for tasks such as density estimation and image generation. 

In the latter, because the formulation introduces density estimation techniques for the SDE itself in \ref{eq:fk}, we first demonstrate the validity of the setup on synthetic problems. In the process, we illustrate the ability to learn stochastic processes between any densities, and compare the transports given by the probability flow in \eqref{eq:transport} and by the SDES (\ref{eq:sde:1},~\ref{eq:sde:R}). Following that, we evaluate the generative modeling capabilities of the framework on standard image generation and tabular density estimation tasks, contextualizing against analogous score-based \cite{song2021scorebased, kingma2021on, ho2020} and flow-based \cite{albergo2023building, lipman2022, liu2022, grathwohl2018scalable} techniques.

\michael{Don't forget to discuss sampling tradeoffs: fast ode samplers, etc.}

\begin{figure}[ht]
\centering
  \includegraphics[width=0.99\linewidth]{figs/ode_vs_sde_with_metrics.pdf}
  \caption{Comparing the sampled checkerboard densities with and without noise. Left) a stochastic interpolant with $\epsilon=0$. Center Left) a stochastic interpolant learned with $\epsilon \neq 0$, sampled under the ODE probability flow, and center left) the same stochastic interpolant with $\epsilon \neq 0$, now sampled according to the SDE. The above hints that controlling the score, as done when $\epsilon \neq 0$, suggested by Section~\ref{sec:likelihood_bounds}, improves learning. \michael{maybe comment that we observe a hierarchy in sampling quality. This tradeoff important when ODE sampling is fast (but slightly worse than SDE) and SDE is very good samples but has less efficient sampling algs}.}
  \label{fig:ode-v-sde}
\endnfigure}




\subsection{Density Estimation}



\subsection{Synthetic examples}

\begin{itemize}
    \item Gaussian or mixture of Gaussians? In this case, we should directly benchmark the cross entropy for both the sde and the ode.
    \item How many sample paths are necessary to accurately estimate the likelihood for the SDE? Dimensional dependence? 
    \item Controlled addition of, e.g. sawtooth mode?
\end{itemize}

% \subsection{High-dimensional examples (real datasets)}
% \begin{itemize}
%     \item Mostly just images. Want to:
%     \begin{itemize}
%         \item Generate one (very) high-res image, e.g. $512 \times 512$.
%         \item Compute FID scores.
%         \item ``Flashy'' examples, like in-painting, super-resolution, de-noising.
%     \end{itemize}
% \end{itemize}