\section{Experiment}
\subsection{Experimental setup}
\label{sec:set_up}
\paragraph{Cityscapes~\cite{cordts2016cityscapes}}
provides pixel-level annotations for 19 object categories in urban scene images at a high resolution of $2048 \times 1024$.
It contains 5000 finely annotated images, split into 2975, 500 and 1525 images for training, validation and testing respectively.
\paragraph{ADE20K~\cite{zhou2019semantic}}
is a challenging benchmark for scene parsing with 150 fine-grained semantic categories.
It has 20210, 2000 and 3352 images for training, validation and testing.

\paragraph{MSeg~\cite{lambert2020mseg}} is a composite dataset that unifies multiple semantic segmentation datasets from different domains. In particular, the taxonomy and pixel-level annotations are aligned by relabeling more than 220,000 object masks in over 80,000 images. We follow the standard setting: the train split~\cite{lin2014microsoft, caesar2018coco, chen2014semantic, cordts2016cityscapes, neuhold2017mapillary, varma2019idd, yu2020bdd100k, song2015sun} for training a unified semantic segmentation model,
the test split (unseen to model training)~\cite{everingham2010pascal, mottaghi2014role,brostow2009semantic, geiger2013vision, dai2017scannet, zendel2018wilddash} for cross-domain validation.

\paragraph*{Evaluation metrics}
The mean Intersection over Union (mIoU) and pixel-level accuracy (mAcc) are reported for all categories, following the standard evaluation protocol~\cite{cordts2016cityscapes}.


\paragraph{Implementation details}
We operate all experiments on {\em mmsegmentation} \cite{contributors2020openmmlab} with 8 NVIDIA A6000 cores. \emph{(i)~Data augmentation}: 
Images are resized to $1024 \times 2048$ on Cityscapes, $512 \times 2048$ on ADE20K and MSeg, and random cropped ($768 \times 768$ on cityscapes and $512\times512$ on ADE20K and MSeg) and random horizontal flipped during training.
\textbf{No test time augmentation} is applied. \emph{(ii)~Training schedule for latent prior learning}: 
The batch size is 16 on Cityscapes and MSeg and 32 on ADE20K. The total number of iterations is 80,000 on Cityscapes and 160,000 on ADE20K and MSeg.

\begin{figure*}[tb]
	\def \imwidth {4.2cm}
	\def \adeheight {2.0cm}
	\def \cityheight {2.3cm}
	\centering
% 	\\
	\includegraphics[height=\cityheight, width=\imwidth, align=b]{figures/unlable/frankfurt_000000_011074_leftImg8bit.png}
	\includegraphics[height=\cityheight, width=\imwidth, align=b]{figures/unlable/gt_bbox.png}
	\includegraphics[height=\cityheight, width=\imwidth, align=b]{figures/unlable/wo_bbox.png}
	\includegraphics[height=\cityheight, width=\imwidth, align=b]{figures/unlable/w_bbox.png}
	\\
    \rotatebox{0}{\textcolor{white}{--------------}Image}
	\rotatebox{0}{\textcolor{white}{------------------------}Ground Truth}
	\rotatebox{0}{\textcolor{white}{------------------}without auxiliary}
	\rotatebox{0}{\textcolor{white}{-----------------}with auxiliary\textcolor{white}{---------}}
\vspace{-1em}
	\caption{Qualitative results of {\em unlabeled area auxiliary} on Cityscapes~\cite{cordts2016cityscapes} dataset.
	}
	\vspace{-1em}
	\label{fig:unlabel_area_auxiliary}
\end{figure*}
\begin{table}[tb]
\tablestyle{1.8pt}{1.05}
\renewcommand\tabcolsep{20pt}
\renewcommand\arraystretch{1.2}
\small
\begin{center}
 
\begin{tabular}{l|cc}
\hline

\hline

\hline

\hline
GSS variants & {mIoU} & Training time \\
\hline

\hline

GSS-\gssa{} & 62.83 & ~~~~~~\textbf{0}\\
\textbf{GSS-\gssb{}} & 84.31 & ~~~~~~\textbf{0}\\
GSS-\gssc{} & 86.10 & ~~~$\leq$20\\
GSS-\gssd{} & 84.37 & ~~~$\leq$5\\
GSS-\gsse{} & 36.11 & ~~~$\leq$5\\
\rowcolor[gray]{.9} 
\textbf{GSS-\gssc{}-W} & \textbf{87.73} & ~~~~~~~~$\leq$350\\
\hline

\hline

\hline

\hline
\end{tabular}
\end{center}
\vspace{-1.5em}
\caption{
\textbf{Ablation on the variants of latent posterior learning} on the {\tt val} set of ADE20K.
Metrics: The \texttt{maskige} reconstruction performance in mIoU,
as well as the training time in \texttt{GPU hours} (\ie, effective single-core hours).
}
\vspace{-0.5em}
\label{tab:vqvae}
\end{table}
\begin{table}[tb]
\tablestyle{1.8pt}{1.05}
\renewcommand\tabcolsep{3.2pt}
\renewcommand\arraystretch{1.2}
\small
\begin{center}
 
\begin{tabular}{lc|ccl}
\hline

\hline

\hline

\hline
{Design} & {\texttt{Maskige}?} & {Cityscapes} & {ADE20K} & Train time \\
\hline

\hline
VQGAN~\cite{esser2021taming}& \XSolidBrush & 82.16 & 81.89 & $\leq$500\\
VQGAN~\cite{esser2021taming} &\checkmark & 75.09 & 42.70 & $\leq$100\\
UViM~\cite{kolesnikov2022uvim}\ &  \XSolidBrush & 89.14 & 78.98 & $\leq$2,000 \\
\rowcolor[gray]{.9} 
DALL$\cdot$E~\cite{ramesh2021zero} &\checkmark & \textbf{95.17} & \textbf{87.73} & $\leq$350\\

\hline

\hline

\hline

\hline
\end{tabular}
\end{center}
\vspace{-1.5em}
\caption{
\textbf{Ablation on \texttt{maskige} reconstruction by different VQVAE designs} 
on Cityscapes semantic {\tt val} split and ADE20k {\tt val} split.
In  case of \textit{no \texttt{maskige}}, we directly reconstruct the segmentation mask with $K$ (the number of classes) channels.
Unit for \textit{training time} is \texttt{GPU hour} (\ie\ the effective single-core hour).
}
\vspace{-0.5em}
\label{tab:vqvae_struct}
\end{table}
\begin{table}[tb]
\tablestyle{1.8pt}{1.05}

\renewcommand\tabcolsep{12.3pt}
\renewcommand\arraystretch{1.2}
\small
\begin{center}
 
\begin{tabular}{lcc|cc}
\hline

\hline

\hline

\hline
$d$&Unlabel& MLA &  ~{mIoU}~ & ~{mAcc}~ \\
\hline

\hline

1/8 &&&  {40.64} & {52.55} \\ 
1/8 &\checkmark&&  {43.72} & {56.08}\\ 
1/4&\checkmark& &   {43.98} & {56.11} \\ 
\rowcolor[gray]{.9}
1/4&\checkmark& \checkmark &  \textbf{46.29} & \textbf{57.84} \\ % 160k

\hline

\hline

\hline

\hline
\end{tabular}
\end{center}
\vspace{-1.5em}
\caption{
\textbf{Ablation on latent prior learning} 
on the {\tt val} split of ADE20K. ``Unlabel'' denotes unlabeled area auxiliary, and ``MLA" denotes Multi-Level Aggregation. ``$d$'' is the downsample ratio of discrete mask representation size between input image size.
}
\vspace{-1.5em}
\label{tab:transformer}
\end{table}
\begin{table}[tb]
\tablestyle{1.8pt}{1.1}
\renewcommand\tabcolsep{7.1pt}
\renewcommand\arraystretch{1.2}
\begin{tabular}{x{44}x{14}x{44}x{30}|c}
\hline

\hline

\hline

\hline
Method & Pretrain & Backbone & Iteration&mIoU \\
\shline
\multicolumn{4}{l}{\emph{- Discriminative modeling:}} \\
\shline
\multicolumn{1}{l}{FCN~\cite{long2015fully}} & 1K & ResNet-101 & {80k} &{77.02}\\ 
\multicolumn{1}{l}{PSPNet~\cite{zhao2017pyramid}} & 1K& ResNet-101 &{80k}  & {79.77}\\
\multicolumn{1}{l}{DeepLab-v3+~\cite{chen2018encoder}}& 1K & ResNet-101 &{80k} & {80.65}\\

\multicolumn{1}{l}{NonLocal~\cite{wang2018nonlocal}} & 1K& ResNet-101 &{80k} & {79.40}\\
\multicolumn{1}{l}{CCNet~\cite{huang2019ccnet}}& 1K & ResNet-101 &{80k} &{79.45} \\
% \multicolumn{1}{l}{GCNet~\cite{cao2019gcnet}} & 1K& ResNet-101 &{80k} & {} \\
\multicolumn{1}{l}{Maskformer~\cite{cheng2021per}} & 1K& ResNet-101 & {90k} & {78.50} \\ % resnet-101 miou 78.5, resnet-101-c miou 79.70
\multicolumn{1}{l}{Mask2former~\cite{cheng2022masked}} & 1K& ResNet-101 & {90k} & {80.10} \\ 
\multicolumn{1}{l}{SETR~\cite{zheng2021rethinking}} & 22K& ViT-Large & {80k} & {78.10} \\
\multicolumn{1}{l}{UperNet~\cite{xiao2018unified}}& 22K & Swin-Large & {80k}  & {82.89}\\ % 81.0
\multicolumn{1}{l}{Maskformer~\cite{cheng2021per}}& 22K & Swin-Large & {90k}  & {78.50}\\
\multicolumn{1}{l}{Mask2former~\cite{cheng2022masked}}& 22K & Swin-Large & {90k}  & \textbf{83.30}\\
\multicolumn{1}{l}{SegFormer~\cite{xie2021segformer}} & 1K& MiT-B5 & {160k} & {82.25} \\
\shline
\multicolumn{3}{l}{\emph{- Generative modeling:}} \\
\shline
\multicolumn{1}{l}{UViM$^\dag$~\cite{kolesnikov2022uvim}}& 22K & Swin-Large & 160k &70.77 \\
\rowcolor[gray]{.9}
\multicolumn{1}{l}{\model-\gssb{}~(Ours)} & 1K & ResNet-101 & 80k & {77.76} \\
\rowcolor[gray]{.9}
\multicolumn{1}{l}{\model-\gssc{}-W~(Ours)}& 1K & ResNet-101 & 80k & {78.46} \\
\rowcolor[gray]{.9}
\multicolumn{1}{l}{\model-\gssb{}~(Ours)} & 22K& Swin-Large & 80k & {78.90} \\
\rowcolor[gray]{.9}
\multicolumn{1}{l}{\model-\gssc{}-W~(Ours)}& 22K & Swin-Large & 80k & \textbf{80.05}\\

\hline

\hline

\hline

\hline
\end{tabular}
\vspace{-0.5em}
\caption{\textbf{Performance comparison on the Cityscapes {\tt val} split:} UViM$^\dag$~\cite{kolesnikov2022uvim} is reproduced by us on PyTorch. ``1K" means pretrained on ImageNet 1K~\cite{deng2009imagenet} while ``22K" means pretrained on ImageNet 22K~\cite{deng2009imagenet}.}
\label{tab:cityscapes_val}
\vspace{-0.5em}
\end{table}
\begin{table*}[t]
\begin{center}
\scalebox{0.8}{
\begin{tabular}{ll|ccccc|ccccc}
\toprule[1pt]
\multicolumn{2}{l|}{} & \multicolumn{5}{c|}{COCO}
& \multicolumn{5}{c}{ADE20K} \\
%\cline{1-9}
 & Method & PQ & PQ$^{th}$ & PQ$^{st}$ & SQ & RQ & PQ & PQ$^{th}$ & PQ$^{st}$ & SQ & RQ \\
\hline
1 & Mask2Former~\cite{mask2former} & 51.9 & 57.7 & 43.0 & 83.1  & 61.6 & - & - & - & - & - \\

2 & CAG-Seg + CLIP Embeds  & 12.5 & 17.7  & 4.6 & 68.1 & 15.3 & 4.9 & 5.2 & 4.2 & 45.5 & 6.2  \\
3 & CAG-Seg + CLIP Embeds + Mask Filter & 22.7 & 26.9 & 16.3  & 82.1 & 26.7 & 10.7 & 9.5 & 13.3 & 66.6 & 13.1  \\
4 & CAG-Seg + Query Embeds & 51.5 & 57.3 & 42.8 & 83.2 & 61.1 & 13.6 & 11.3  & 18.0 & 29.8 & 16.8 \\
5 & CAG-Seg + Query Embeds + Mask Filter & 51.9 & 57.4 & 43.4 & 83.3 &  61.5 & 14.5 & 12.4  & 19.3 & 37.7 & 17.6  \\

6 & CAG-Seg + Query Embeds$^{\dagger}$ + Mask Filter & 52.4 & 58.0 & 44.0  & 83.5 & 62.1 & 14.6 & 13.2  & 17.6 & 33.8 & 17.1 \\

\rowcolor{gray!20} 
7 & \textbf{OPSNet} (CAG-Seg + Modulated Embeds + Mask Filter )  & \bf 52.4 & \bf 58.0 &  \bf44.0  &  \bf83.5 &  \bf62.1 &  \bf17.7 & \bf 15.6  &  \bf21.9 &  \bf54.9 & \bf21.6 \\

\bottomrule
\end{tabular}
}
\end{center}

\vspace{-2mm}
\caption{ Ablation study for the roadmap towards open-world panoptic segmentation. All experiments use ResNet-50 backbone, and are trained on COCO for 50 epochs. `CAG-Seg' denotes the class-agnostic segmentation model. `Query Embeds$^{\dagger}$' means adopting the cross attention layer to gather information from the CLIP features.   }
\label{tab:crossade}
\vspace{-2mm}
\end{table*}
\begin{table*}[htb]
\tablestyle{3.8pt}{1.05}
\vspace{-2mm}
\centering
\renewcommand\tabcolsep{2.7pt}
\renewcommand\arraystretch{1.2}
\vspace{-2mm}
\small
\begin{tabular}{lllcccccc|c}
\hline

\hline

\hline

\hline
Method 
& Backbone
& Iteration 
&VOC~\cite{everingham2010pascal} & Context~\cite{mottaghi2014role} & CamVid~\cite{brostow2009semantic} & WildDash~\cite{zendel2018wilddash}   & KITTI~\cite{geiger2013vision}  & ScanNet~\cite{dai2017scannet} & \textit{h.\ mean} \\
\hline

\hline
\multicolumn{3}{l}{\emph{- Discriminative modeling:}}\\
\shline
CCSA~\cite{motiian2017unified} & HRNet-W48 &500k &48.9 & - & 52.4 & 36.0 & - & 27.0 & 39.7 \\
MGDA~\cite{sener2018multi} & HRNet-W48 & 500k &69.4 & - & 57.5 & 39.9 & - & 33.5 & 46.1 \\
MSeg~\cite{lambert2020mseg}  & HRNet-W48 &  500k & 70.7 & 42.7  & \textbf{83.3} & 62.0 & 67.0 & 48.2 & 59.2\\
MSeg$^\dag$~\cite{lambert2020mseg} & HRNet-W48 & 160k & 63.8& 39.6 & 73.9 & 60.9 & 65.1 & 43.5 & 54.9\\
 MSeg$^\dag$~\cite{lambert2020mseg}& Swin-Large &160k & 78.7 & 47.5 & 75.1 & \textbf{66.1} & \textbf{68.1} & 49.0 & 61.7 \\
\hline

\hline
 \multicolumn{3}{l}{\emph{- Generative modeling:}}\\
 \shline
  \rowcolor[gray]{.9}
 \model-\gssb{}~(Ours) & HRNet-W48 & 160k &64.1&37.1&72.3&59.3&62.0&40.6& 52.6\\
 \rowcolor[gray]{.9}
 \model-\gssc{}-W~(Ours) & HRNet-W48 & 160k & 65.2&  38.8& 75.2 & 62.5 & 66.2 &43.1& 55.2 \\
\rowcolor[gray]{.9}
\model-\gssb{}~(Ours) & Swin-Large & 160k & 78.7 & 45.8 & 74.2 & 61.8 & 65.4 & 46.9 & 59.5\\
\rowcolor[gray]{.9}
\model-\gssc{}-W~(Ours) & Swin-Large & 160k & \textbf{79.5} & \textbf{47.7} & 75.9 & 65.3 & 68.0 &  \textbf{49.7} & \textbf{61.9} \\
\hline

\hline

\hline

\hline
\end{tabular}
\vspace{-0.5em}
\caption{\textbf{Cross-domain semantic segmentation performance on MSeg dataset {\tt test} split.} ``\textit{h. mean}" is the harmonic mean~\cite{lambert2020mseg}.
MSeg$^\dag$~\cite{lambert2020mseg} is reproduced by us on MMSegmentation~\cite{contributors2020openmmlab}}
\label{tab:mseg}
\vspace{-1em}
\end{table*}
\begin{table}[tb]
\tablestyle{1.8pt}{1.05}
\renewcommand\tabcolsep{6.6pt}
\renewcommand\arraystretch{1.2}
\begin{tabular}{x{44}x{74}|cc}
\hline

\hline

\hline

\hline
Sharing $\mathcal{I}_\psi$ & Sharing {\tt maskiage} & \model{}-\gssb & \model{}-\gssc{}-W \\
\shline
 &  &\textbf{78.9} &\textbf{80.5}\\
 & \checkmark & 78.0& 79.5\\
 \rowcolor[gray]{.9}
\checkmark & \checkmark & 76.6& 78.4\\

\hline

\hline

\hline

\hline
\end{tabular}
\vspace{-1em}
\caption{\textbf{Transferring the {\tt maskiage} and image encoder $\mathcal{I}_\psi$ from MSeg to Cityscapes ({\tt val} split).} 
Metric: mIoU.
% on Cityscapes {\tt val} split.
}
\vspace{-2em}
\label{tab:share_maskiage_between_cityscapes_and_mseg}
\end{table}
\subsection{Ablation studies}
\label{sec:ablation}
{\noindent\bf Latent posterior learning}
We evaluate the variants of latent posterior learning as described in Section \ref{sec:maskige}.
We observe from Table~\ref{tab:vqvae} that: 
\textbf{(i)} \model{}-\gssb{} comes with no extra training cost. 
Whilst UViM consumes nearly 2K GPU hours for training a VQVAE with similar performance achieved~\cite{kolesnikov2022uvim}.
\textbf{(ii)} The randomly initialized $\beta$ (\ie\ \model{}-\gssa{}) leads to considerable degradation on the least square optimization.
\textbf{(iii)} With our {\em maximal distance assumption}, the regularized $\beta$ of \model{}-\gssb{} brings clear improvement,
suggesting the significance of initialization and its efficacy of our strategy.
\textbf{(iv)} With three-layer conv network with activation function for $\fxpi$, \model{}-\gssc{} achieves good reconstruction.
Further equipping with a two-layer Shifted Window Transformer block~\cite{liu2021swin} for $\mathcal{X}^{-1}$ (\ie\ {\model{}-\gssc{}-W}) leads to the best result at a cost of extra 329.5 GPU hours.
%
This is due to more accurate translation from predicted {\tt maskige} to segmentation mask.
%
\textbf{(v)} Interestingly, with automatic $\beta$ optimization,
\model{}-\gssd{} brings no benefit over \model{}-\gssb{}.
\textbf{(vi)} 
Further, joint optimization of both $\fx$ and $\fxpi$ (\ie\ \model{}-\gsse{}) fails to achieve the best performance.
\textbf{(vii)} In conclusion, \model{}-\gssb{} is most efficient with reasonable accuracy,
whilst {\model{}-\gssc{}-W} is strongest with good efficiency.

{\noindent\bf VQVAE design}
We examine the effect of VQVAE in the context of \texttt{maskige}.
%
We compare three designs:
{\bf(1)} {\em UViM-style} \cite{kolesnikov2022uvim}: Using images as auxiliary input to reconstruct a segmentation mask in form of $K$-channels ($K$ is the class number) in a ViT architecture.
In this no {\tt maskige} case, the size of segmentation mask 
may vary across different datasets, leading to a need for dataset-specific training.
%
This scheme is thus more expensive in compute.
%
{\bf (2)} {\em VQGAN-style} \cite{esser2021taming}: 
Using a CNN model for reconstructing natural images ({\tt maskige} needed for segmentation mask reconstruction) or $K$-channel segmentation masks (no {\tt maskige} case) separately,
both optimized in 
generative adversarial training manner with a smaller codebook.
%
{\bf (3)} {\em DALL$\cdot$E-style} \cite{ramesh2021zero}:
The one we adopt, as discussed earlier.
%
We observe from Table~\ref{tab:vqvae_struct} that:
\textbf{(i)} 
Due to the need for dataset specific training,
UViM-style is indeed more costly than the others.
%
This issue can be well mitigated by our \texttt{maskige} with 
the first stage training cost compressed dramatically,
as evidenced by DALL$\cdot$E-style and VQGAN-style.
%
Further, the inferiority of UViM over DALL$\cdot$E
suggests that our \texttt{maskige} is a favored strategy 
than feeding image as auxiliary input.
%
\textbf{(ii)} 
In conclusion, using our \texttt{maskige} and DALL$\cdot$E pretrained VQVAE yields the best performance in terms of both accuracy and efficiency.

{\noindent\bf Latent prior learning}
We ablate the second training stage for learning latent joint prior.
The baseline is 
% Our baseline is chosen as 
{\model{}-\gssb{}} without the unlabeled area auxiliary and Multi-Level Aggregation (MLA, including a 2-layer Swin block~\cite{liu2021swin}), under 1/8 downsample ratio.
We observe from Table~\ref{tab:transformer} that: 
\textbf{(i)} 
Our unlabeled area auxiliary boosts the accuracy by $3.1\%$,
 suggesting the importance of complete labeling which however is extremely costly
 in semantic segmentation.
%
\textbf{(ii)} 
Increasing the discrete mask representation resolution is slightly useful. 
\textbf{(iii)} 
The MLA plays another important role,
\eg\ giving a gain of $2.3\%$.

\begin{figure*}[htb]
% \vspace{-1em}
	\def \imwidth {3.3cm}
	\def \imheight {2.3cm}
	\centering
	\includegraphics[height=\imheight, width=\imwidth, align=b]{figures/stage1/image/ADE_val_00000021_bbox.jpg}
	\includegraphics[height=\imheight, width=\imwidth, align=b]{figures/stage1/gt/ADE_val_00000021_gt_bbox.png}
	\includegraphics[height=\imheight, width=\imwidth, align=b]{figures/stage1/taming/ADE_val_00000021_pred_bbox.png}
	\includegraphics[height=\imheight, width=\imwidth, align=b]{figures/stage1/uvim/ADE_val_00000021_pred_bbox.png}
	\includegraphics[height=\imheight, width=\imwidth, align=b]{figures/stage1/dalle/ADE_val_00000021_pred_bbox.png}\\
 	\includegraphics[height=\imheight, width=\imwidth, align=b]{figures/stage1/image/ADE_val_00000013_bbox.jpg}
	\includegraphics[height=\imheight, width=\imwidth, align=b]{figures/stage1/gt/ADE_val_00000013_gt_bbox.png}
	\includegraphics[height=\imheight, width=\imwidth, align=b]{figures/stage1/taming/ADE_val_00000013_pred_bbox.png}
	\includegraphics[height=\imheight, width=\imwidth, align=b]{figures/stage1/uvim/ADE_val_00000013_pred_bbox.png}
	\includegraphics[height=\imheight, width=\imwidth, align=b]{figures/stage1/dalle/ADE_val_00000013_pred_bbox.png} \\

	\rotatebox{0}{\textcolor{white}{---}Image}
	\rotatebox{0}{\textcolor{white}{----------------}Ground Truth}
	\rotatebox{0}{\textcolor{white}{--------------}VQGAN~\cite{esser2021taming}}
	\rotatebox{0}{\textcolor{white}{--------------}UViM$^\dag$~\cite{kolesnikov2022uvim}}
	\rotatebox{0}{\textcolor{white}{-------------}\model{} (ours)}
	\vspace{-0.5em}
	\caption{Qualitative results of {\texttt maskige} reconstruction on ADE20K~\cite{zhou2019semantic} dataset. Note that the black areas in the Ground Truth correspond to unlabeled regions, and thus {\em no impact} on mIoU measurement.
	}
	\vspace{-0.3em}
	\label{fig:visualization_stage_1}
\end{figure*}
\begin{figure*}[bth]
	\def \imwidth {2.8cm}
	\def \adeheight {2.0cm}
	\def \cityheight {1.5cm}
	\centering
	\includegraphics[height=\cityheight, width=\imwidth, align=b]{figures/stage2/image/frankfurt_000000_000294_leftImg8bit.png}
	\includegraphics[height=\cityheight, width=\imwidth, align=b]{figures/stage2/gt/frankfurt_000000_000294_leftImg8bit_rec.png}
	\includegraphics[height=\cityheight, width=\imwidth, align=b]{figures/stage2/pred/frankfurt_000000_000294_leftImg8bit_pred.png}
	\includegraphics[height=\cityheight, width=\imwidth, align=b]{figures/stage2/image/frankfurt_000000_001236_leftImg8bit.png}
	\includegraphics[height=\cityheight, width=\imwidth, align=b]{figures/stage2/gt/frankfurt_000000_001236_leftImg8bit_rec.png}
	\includegraphics[height=\cityheight, width=\imwidth, align=b]{figures/stage2/pred/frankfurt_000000_001236_leftImg8bit_pred.png}
	\\
	\includegraphics[height=\adeheight, width=\imwidth, align=b]{figures/stage2/image/ADE_val_00000135.jpg}
	\includegraphics[height=\adeheight, width=\imwidth, align=b]{figures/stage2/gt/ADE_val_00000135_gt.png}
	\includegraphics[height=\adeheight, width=\imwidth, align=b]{figures/stage2/pred/ADE_val_00000135_learnable_pred_our_color.png}
	\includegraphics[height=\adeheight, width=\imwidth, align=b]{figures/stage2/image/ADE_val_00000118.jpg}
	\includegraphics[height=\adeheight, width=\imwidth, align=b]{figures/stage2/gt/ADE_val_00000118_gt.png}
	\includegraphics[height=\adeheight, width=\imwidth, align=b]{figures/stage2/pred/ADE_val_00000118_learnable_pred_our_color.png}
	\\
	\rotatebox{0}{\textcolor{white}{-----------}Image}
	\rotatebox{0}{\textcolor{white}{------------}Ground Truth}
	\rotatebox{0}{\textcolor{white}{---------}Prediction}
	\rotatebox{0}{\textcolor{white}{-------------}Image}
	\rotatebox{0}{\textcolor{white}{------------}Ground Truth}
	\rotatebox{0}{\textcolor{white}{---------}Prediction\textcolor{white}{-------}}
\vspace{-0.5em}
	\caption{Qualitative results of semantic segmentation on  Cityscapes~\cite{cordts2016cityscapes} and ADE20K~\cite{zhou2019semantic} datasets.
	}
	\vspace{-0.8em}
	\label{fig:visualization_stage_2}
\end{figure*}
% \vspace{-0.2em}
\subsection{Single-domain semantic segmentation}
% \vspace{-0.1em}
We compare our \model{} with prior art discriminative methods and the latest generative model (UViM~\cite{kolesnikov2022uvim}, a replicated version for semantic segmentation task).
We report the results in Table~\ref{tab:cityscapes_val}
for Cityscapes~\cite{cordts2016cityscapes} and
Table~\ref{tab:ade20k_val} for ADE20K~\cite{zhou2019semantic}. 
\emph{\textbf{(i)} In comparison to discriminative methods}:
Our \model{} yields competitive performance with either Transformers (Swin) or CNNs (\eg\ ResNet-101).
For example, under the same setting, \model{} matches the result of Maskformer~\cite{cheng2021per}.
Also, {\model{}-\gssc{}-W} is competitive to the Transformer-based SETR~\cite{zheng2021rethinking} on both datasets.
\emph{\textbf{(ii)} In comparison to generative methods}: 
% Under fair comparison, 
\model{}-\gssb{} surpasses UViM~\cite{kolesnikov2022uvim} by a large margin
whilst enjoying higher training efficiency.
Specifically, UViM takes 1,900 TPU-v3 hours for the first training stage and 900 TPU-v3 hours for the second stage.
While the first stage takes only 329.5 GPU hours with \model{}-\gssc{}-W, and zero time with \model{}-\gssb{}. The second stage of \model{}-\gssb{} requires approximately 680 GPU hours. 
%
This achievement is 
due to our \texttt{maskige} mechanism for enabling the use of pretrained data representation and a series of novel designs for joint probability distribution modeling.

\subsection{Cross-domain semantic segmentation}
We evaluate cross-domain zero-shot benchmark~\cite{lambert2020mseg}. 
We compare the proposed \model{} with MSeg~\cite{lambert2020mseg}, a domain generalization algorithm (CCSA)~\cite{motiian2017unified} and a multi-task learning algorithm (MGDA)~\cite{sener2018multi}.
We test both HRNet-W48~\cite{sun2019high} and Swin-Large~\cite{liu2021swin} as backbone. 
As shown in Table~\ref{tab:mseg},
our \model{} is superior to all competitors using either backbone.
%
This suggests that generative learning could achieve more domain-generic representation than conventional discriminative learning counterparts.

\paragraph{Domain generic \texttt{maskige}}
Being independent to the visual appearance of images, \texttt{maskige} is intrinsically domain generic. 
To evaluate this, we transfer the {\tt maskige} from MSeg to Cityscapes.
As shown in Table~\ref{tab:share_maskiage_between_cityscapes_and_mseg}, \model{} can still achieve 79.5 mIoU (1\% drop).
%
In comparison, image representation transfer would double the performance decrease.
\subsection{Qualitative evaluation}
We evaluate the first-stage reconstruction quality of our \model{}, UViM~\cite{kolesnikov2022uvim} and VQGAN~\cite{esser2021taming}. 
As shown in Figure~\ref{fig:visualization_stage_1}, \model{} produces almost error-free reconstruction with clear and precise edges, and UViM fails to recognize some small objects while yielding distorted segmentation. VQGAN~\cite{esser2021taming} achieves better classification accuracy but produces more ambiguous edge segmentation.
As shown in Figure~\ref{fig:visualization_stage_2}, \model{} produces fine edge segmentation for interior furniture divisions on ADE20K~\cite{zhou2019semantic} and accurately segments distant pedestrians and slender poles on Cityscapes~\cite{cordts2016cityscapes}.
