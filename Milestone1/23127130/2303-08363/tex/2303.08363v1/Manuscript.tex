\documentclass[aps,pre,twocolumn,showpacs,superscriptaddress,groupedaddress]{revtex4-2}
\usepackage{amsmath}
\usepackage{amsfonts}
\usepackage{amssymb}
\usepackage{graphicx}
\usepackage{dcolumn}
\usepackage{sidecap}
\usepackage{parskip}
%\usepackage{grffile}
\usepackage{color}
\usepackage{hyperref}
\usepackage{hhline}
\usepackage{mathtools}
\usepackage{multirow}
\usepackage{verbatim}
\usepackage{rotating}
\usepackage{setspace}
\usepackage{epsfig}
\usepackage{epstopdf}
%\usepackage{float}
%\usepackage{subfloat}
\usepackage{subfigure}
%\usepackage{booktabs}
\usepackage[normalem]{ulem}

\usepackage{bm}
%\usepackage{comment}
%\usepackage{cleveref}

%\usepackage[ backend=biber,
%    style=numeric, sorting = none
%]{biblatex}

%\addbibresource{reference.bib}

\newcommand{\n}{\noindent}
\newcommand{\A}{\textbf{A}}
\newcommand{\I}{\textbf{I}}
\newcommand{\M}{\cal{M}}
\newcommand{\mk}{\langle k \rangle}
\newcommand{\iu}{\mathrm{i}\mkern1mu}

\def \hfillx {\hspace*{-\textwidth} \hfill}

\makeatletter
\def\@eqnnum{{\normalsize \normalcolor (\theequation)}}
 \makeatother

\hyphenation{ALPGEN}
\hyphenation{EVTGEN}
\hyphenation{PYTHIA}

\newcommand\filledcirc{{\color{black}\bullet}\mathllap{\circ}}


\begin{document}
%------------------------------
\title{Self-consistent method for Kuramoto oscillators with inertia having higher-order interactions}

\author{Narayan G. Sabhahit, Akanksha S. Khurd, Sarika Jalan}\email{sarika@iiti.ac.in}

\affiliation{Complex Systems Lab, Department of Physics, Indian Institute of Technology Indore, Khandwa Road, Simrol, Indore-453552, India}

\date{\today}

\begin{abstract}
The impact of inertia on the Kuramoto model has been long reported to change the nature of phase transition in the system. Yet, a general analytical framework to decipher the role of inertia on coupled oscillator systems interacting via polyadic interactions still remains a challenge. To that end, in this Letter, we develop a set of self-consistent equations to describe the steady-state behavior of coupled oscillator systems with inertia on simplicial complexes, which reveal a prolonged hysteresis in the synchronization profile. We find that triadic interaction strength and inertia exhibit isolated influence on system dynamics. The forward (backward) critical coupling strength  exclusively depends on inertia (strength of higher-order interactions). This work sets a paradigm to deepen our understanding of real-world complex systems such as power grids modeled as the coupled Kuramoto model with inertia.
 
\end{abstract}

\maketitle
 {\bf{Introduction:}}
The emergence of collective behavior in complex real-world systems has been a long-standing research interest \cite{boccaletti2006complex}. It was initially in the landmark paper \cite{kuramoto1975self} that Kuramoto modeled the phenomenon of synchronization using a system of network-coupled oscillators in an analytically tractable setting, illustrating that the system underwent a second-order phase transition from incoherent to a coherent state. Since then, numerous works on various extensions of the Kuramoto Model have been done, revealing several phenomena \cite{childs2008stability,sethia2008clustered,omel2012nonuniversal,olmi2015chimera}. Of particular interest to us is the Kuramoto Model with inertia  (also referred to as the second-order Kuramoto model). Inspired by the modeling of synchronized flashing in \textit{Pteroptix malaccae} by Ermentrout et al. \cite{ermentrout1991adaptive}, a second-order extension of the Kuramoto model was first proposed by \textit{Tanaka et al.} \cite{tanaka1997first,tanaka1997self}. They showed that the system underwent a first-order phase transition when inertia was introduced rather than the smooth second-order phase transition observed in the Kuramoto model. They put forth a self-consistent method akin to the one proposed by Kuramoto to study the steady-state behavior of the coupled oscillator system. Additionally, the Kuramoto model with inertia on scale-free networks with degree-frequency correlations was shown to depict cluster explosive synchronization \cite{ji2013cluster}. The second-order Kuramoto model has been studied extensively in various systems like the brain \cite{sakyte2011self}, Josephson junctions \cite{watanabe1994constants} and power grids \cite{rohden2012self, dorfler2013synchronization}. In\cite{filatrella2008analysis}, \textit{Filatrella et al.} explained how the second-order Kuramoto model originates in power grids by simply accounting for power conservation at each node of the grid.

However, all these results were obtained by focusing on the interactions to be purely dyadic in nature. Recent research highlights that such a reductionistic view might not reveal the complete picture of the underlying mechanism of exotic phenomena observed in some real-world complex systems where the interactions between agents are inherently polyadic in nature \cite{benson2018simplicial,majhi2022dynamics}. We focus on the results presented by \cite{skardal2020higher} where higher-order interactions were incorporated into the first-order Kuramoto model inducing abrupt synchronization transition. It was naturally inquisitive to observe that adding inertia to the Kuramoto model or incorporating polyadic interactions independently led to a first-order phase transition in the system. Unsurprisingly we were interested in understanding how the interplay of inertia and polyadic interactions manifests itself in the system. 

To that end, in this Letter, we study the Kuramoto model with inertia on simplicial complexes. We develop a self-consistent system of equations 
to characterize the steady-state behavior of the system. We find that inertia and higher-order interactions independently influence the phenomenon of synchronization, due to which there is a prolonged hysteresis in the system. The forward transition point is found to be exclusively dependent on the mass, shifting to the right with an increase in the mass of the oscillators, whereas the backward transition point depends solely upon the strength of higher-order interaction and brings about a shift to the left with an increase in the magnitude of higher order coupling. Hence a prolonged hysteresis is observed where the macroscopic system exists in two stable states depending on the initial condition when both the mass and the strength of the higher-order coupling are increased simultaneously. Numerical simulations are in good agreement with analytical predictions. Such prolonged existence of the coherent branch can prove beneficial for the dynamics of real-world systems like power grids which are explicitly modeled using the Kuramoto model with inertia.


{\bf{Model:}}
 We study the Kuramoto Model with inertia considering the simultaneous presence of  dyadic and triadic interactions. To develop a mean-field approach, we consider a globally coupled network setting. Phases of $N$-coupled oscillators, each with mass $m$, evolve based on the following coupled nonlinear equations.
\begin{equation}\label{Gen_KM_hoi}
    \begin{split}
    m\ddot{\theta_i} =  -\dot{\theta_i} + \omega_i + \frac{K_{1}}{N}\sum_{j = 1}^{N}\sin(\theta_{j} - \theta_{i}) \\
    +\frac{K_{2}}{N^{2}} \sum_{j = 1}^{N}\sum_{k = 1}^{N}\sin(2\theta_{j} - \theta_{k}- \theta_{i})
    \end{split}
\end{equation}
In Eq.~\ref{Gen_KM_hoi}, $\theta_{i}$ and $\dot\theta_{i}$ refer to the  instantaneous phase and angular velocity of  the $i^{th}$ oscillator, respectively. $\omega_{i}$ refers to the intrinsic frequency of the $i^{th}$ oscillator derived from a unimodal symmetric probability distribution $g(\omega)$ with mean $\Omega$.  The coupling constants $K_1 \geq 0$ and $K_2$ are the dyadic and triadic coupling strengths, respectively.


{\bf{Results:}}
We start by decoupling the differential equations in Eq.~\ref{Gen_KM_hoi} by writing them in terms of mean-field quantities by introducing the following general order parameter for $p \in \{1,2\}$. 
\begin{equation}\label{Order Parameter}
     r_{p}e^{i\psi_{p}} = \frac{1}{N}\sum_{j = 1}^{N}e^{ip\theta_{j}}
\end{equation}
$r_{1}$ measures the global phase coherence and can be interpreted as the centroid of phases of oscillators in the unit circle on the complex plane, and $\psi_{1}$ measures the average phase of the oscillators. $r_{2}$, referred to as the Daido order parameter \cite{xu2021spectrum} captures cluster synchronization. We are interested in the steady state behavior of the system hence, we omit the time dependence in the definition of the general order parameter. In the incoherent state, the phases of the oscillators are scattered uniformly on the unit circle and hence $r_{1} \approx r_{2} \approx 0$. Meanwhile, in the coherent state, a single group of oscillators is formed attached to the mean phase $\psi_{1}$ rotating uniformly at angular velocity $\Omega$, hence $r_{1} \approx r_{2} \approx 1$. Using Eq.~\ref{Order Parameter}, Eq.~\ref{Gen_KM_hoi} can be written as,
\begin{equation}\label{mean field equation}
    \begin{split}
        m\ddot{\theta}_{i} =  -\dot{\theta}_{i} + \omega_i + K_{1}r_{1}\sin(\psi_{1} - \theta_{i}) \\  +K_{2}r_{1}r_{2}\sin(\psi_{2} - \psi_{1}- \theta_{i})
    \end{split}
\end{equation}
Because of the rotational symmetry in the model, we can take the mean of $g(\omega)$ distribution to be zero by moving into the rotating frame at the frequency $\Omega$. This can be facilitated by making the transformation $\theta_{i} \rightarrow \theta_{i} + \Omega t$ in Eq.~\ref{Gen_KM_hoi}. Once in the rotating frame, by choosing appropriate initial conditions,  $\psi_{1}$ and $\psi_{2}$ can be set to zero. The mean field equation, Eq.~\ref{mean field equation}, now takes the following form,
\begin{equation}\label{mean field equation 2}
    m\ddot{\theta}_{i} =  -\dot{\theta}_{i} + \omega_i - q\sin(\theta_{i})
\end{equation}
Where, for the ease of notation, $q= r_{1}(K_{1} +K_{2}r_{2})$. Note that for a fixed $K_{2}$, Eq.~\ref{mean field equation 2} has parameters $K_{1}$, $r_{1}$ and $r_{2}$. Hence, following \cite{tanaka1997first}, to chalk up the steady state behavior of the system, we develop a system of self-consistency equations and seek the values of $(K_{1},r_{1},r_{2})$ which simultaneously satisfy the self-consistency equations. We start by taking the thermodynamic limit ($N\rightarrow\infty$); the coupled oscillator system in the steady state is then described by a probability density $\rho(\theta,\omega)$ where for a given intrinsic frequency $\omega$, $\rho(\theta,\omega)d\theta$ represents the fraction of oscillators with their phase between $\theta$ and $\theta + d\theta$. The general order parameter in Eq~\ref{Order Parameter} in the continuum limit takes the  following form,
\begin{equation}\label{Order Parameter Thermodynamic}
     r_{p}e^{i\psi_{p}} = \int_{-\pi}^{\pi}\int_{-\infty}^{\infty} e^{ip\theta}\rho(\theta,\omega)g(\omega)d\omega d\theta
\end{equation}
In the steady state, the oscillator population splits up into two groups depending on their intrinsic frequency. One group of oscillators is locked to the mean phase; meanwhile, the other oscillators drift over the locked oscillators. Hence the overall phase coherence ($r_{p}$) can be split into contributions from the locked ($r_{p}^{l}$) and drifting ($r_{p}^{d}$) oscillators, i.e, $r_{p} = r_{p}^{l} + r_{p}^{d}$.
Before calculating $r_p^l$ and $r_p^d$, we point out that systems whose motion is governed by Eq.~\ref{mean field equation 2} are known to depict hysteresis and have been well studied in \cite{tanaka1997first,tanaka1997self,strogatz2018nonlinear}. For the sake of completion, we briefly summarise the reason for the hysteresis behavior here. Dropping the subscript $i$ and by introducing a new timescale as $\tau = \sqrt{\frac{q}{m}}t$, Eq.~\ref{mean field equation 2} is transformed to a second order differential equation with just two parameters.
\begin{equation}\label{dimensionless mean field equation}
    \ddot{\theta} =  -\alpha \dot{\theta} + \beta -\sin(\theta)
\end{equation}
where $\alpha = \frac{1}{\sqrt{qm}}$, the damping term and $\beta = \frac{\omega}{q}$. Eq.~\ref{dimensionless mean field equation} has two fixed points, a saddle, and a sink for $\beta<1$, obtained by setting $\dot\theta = 0$ and $\ddot\theta = 0$. The sink is a stable fixed point if $\alpha$ is large enough or if $\beta$ is close to one; otherwise, it is a stable spiral. At $\beta = 1$, the system undergoes a saddle-node bifurcation annihilating the two fixed point solutions and admitting a unique stable limit cycle solution for all $\beta >1$ \cite{levi1978dynamics}. However, it so happens that as we decrease the value of $\beta$ to be less than one, the limit cycle persists for some small values of $\alpha$. Hence, bistability exists in the system (Fig.~\ref{paramspace-mainresult}a), where a stable limit cycle and a sink coexist. A further decrease in $\beta$ will result in the disintegration of the limit cycle via a homoclinic bifurcation. For small values of the damping term $\alpha$, ensured by keeping finite inertia, the homoclinic bifurcation curve can be approximated to be a straight line. Upon implementing Melnikov's method, \cite{strogatz2018nonlinear,guckenheimer2013nonlinear} the equation of the straight line comes out to be $\beta = \frac{4}{\pi} \alpha$. In conclusion, we see the presence of three different dynamical regimes for Eq.~\ref{dimensionless mean field equation}, namely a fixed point ($\beta < \frac{4}{\pi} \alpha$), bi-stable region ($\frac{4}{\pi} \alpha < \beta < 1$), and a limit-cycle ($\beta > 1$).

\begin{figure*}[t]
    \centering
    \includegraphics[width=12cm]{Figure3.eps}
    \caption{(Color online) (a) $\beta$ vs $\alpha$ parameter space depicting different dynamical regimes of Eq.~\ref{dimensionless mean field equation}  (b) $r_1$ versus $K_1$ plots for (i) $K_2 = 1$ and $m = 1$ (blue)and (ii) $K_2 = 7$ and $m = 3$ (red). Circles and squares represent the simulation results for the forward and backward cases, respectively. The dashed and continuous curves represent the forward and backward analytical values, respectively. Prolonged hysteresis is observed as a result of an increase in $m$ and $K_2$.}
    \label{paramspace-mainresult}
\end{figure*}

 The bi-stable region turns out to be responsible for hysteresis in systems governed by equations like Eq.~\ref{mean field equation 2}. Hence, following \cite{tanaka1997first}, instead of studying the system in its full generality, we break down the self-consistency analysis for our model Eq.~\ref{mean field equation 2} into forward $\left(f\right)$ and backward $\left(b\right)$ processes. In the forward process, we start from a small $K_1$ value, and therefore the system is in an incoherent state $\left(r_{1}\approx 0\right)$. This leads to high $\alpha$ and $\beta$ values, indicating that the oscillators are in the limit cycle regime. As we adiabatically increase $K_1$, the oscillators stay in the basin of attraction of the stable limit-cycle even after crossing $\beta = 1 (q = \omega)$ and fall into locked clusters only after $\beta = \frac{4}{\pi}$ $\alpha (\omega = \frac{4}{\pi} \sqrt{\frac{q}{m}})$, below which the limit cycle vanishes. For the backward process, we start from a high $K_1$ value and hence the oscillators exist in the fixed-point state, i.e., the oscillators are locked in a cluster $\left(0<< r_1 < 1\right)$. As we adiabatically decrease $K_1$ from this state, the oscillators remain in the basin of attraction of the sink until  $\beta = 1$, when the fixed points vanish via a saddle node bifurcation. Thus, in the backward process, oscillators having $|\omega| \le q = \omega_{b}$ contribute to the locked oscillators, while in the forward process, only those with $|\omega| \le \frac{4}{\pi} \sqrt{\frac{q}{m}} = \omega_{f}$ are in a locked state and all the oscillators with $\omega \geq \omega_{f,b}$ drift around the locked cluster. The contribution of the locked oscillator($r_{p}^{l}$) can now be calculated as $r_{p}^{l} = \int_{-\infty}^{\infty} \int_{-\pi/2}^{\pi/2}e^{ip\theta}\delta(\theta - \sin^{-1}(\frac{\omega}{q}))g(\omega)d\omega d\theta\\$. The imaginary part of $r_{p}^{l}$ goes to zero as $g(-\omega) = g(\omega)$. Hence taking only the real part and noting that  $\theta_{f,b} = \sin^{-1}(\omega_{f,b}/q)$, we arrive at the expression for $r_{p}^{l}$ as follows,
\begin{equation}\label{locked term}
    \begin{split}
    r_{p}^{l} = q\int_{-\theta_{f,b}}^{\theta_{f,b}}\cos(\theta)\cos(p\theta)g(q\sin(\theta))d\theta
    \end{split}
\end{equation}
The contribution to overall coherence from the drifting oscillators can be accounted for by calculating  $r_{p}^{d} = \int_{|\omega|>\omega_{f,b}}\int_{-\pi}^{\pi} e^{ip\theta}\rho_{d}(\theta,\omega)g(\omega)d\omega d\theta$ where $\rho_{d}(\theta,\omega)$ is the density of drifting oscillator which satisfies $\rho_{d}(\theta,\omega) \propto 1/\dot\theta$ \cite{tanaka1997first}. The normalization condition for $\rho_{d}(\theta,\omega)$ gives, $ \int_{-\pi}^{\pi}\rho_{d}(\theta,\omega) d\theta= \int_{0}^{T}\rho_{d}(\theta,\omega)\dot{\theta}dt = 1$ (for a given $\omega$), where $T$ is the time period of the whirling limit cycle solution. Hence we end up with the relation $\rho_{d}(\theta,\omega) = \frac{1}{\dot{\theta}T}$, which when plugged into the form of $r_{p}^{d}$ gives us,
\begin{equation}\label{drift1}
    \begin{split}
     r_{p}^{d}  = \int_{|\omega|> \omega_{f,b}} \left[\frac{1}{T}\int_{0}^{T}e^{ip\theta}dt\right]g(\omega)d\omega
     \end{split}
\end{equation}
To obtain the expression for $r_p^d$, we first need to obtain an approximate analytic expression for the whirling limit cycle solution. We follow the method specified in \cite{gao2018self} of writing $\dot{\theta}$ as a Fourier series in  $\theta$ by only considering the first harmonics $(\dot{\theta} = A_{0} + A_{1}\cos(\theta)+B_{1}\sin(\theta))$. On substituting $\dot{\theta}$ in Eq.~\ref{dimensionless mean field equation}, we find the expression of the coefficients in terms of $\alpha$($=\frac{1}{\sqrt{qm}}$) and $\beta$($=\frac{\omega}{q}$) such that the first harmonic vanishes. This gives us $\dot{\theta} = \frac{\beta}{\alpha} + \frac{\alpha^{2}}{\alpha^{4} + \beta^{2}}\left[\frac{\beta}{\alpha}\cos(\theta) - \alpha \sin(\theta)\right]$ and upon integrating $\dot\theta$ with time, and choosing the constant of integration such that $\theta(0) = 0$, we end up with $\theta = \frac{\beta t}{\alpha} + \frac{\alpha^2}{\alpha^4 + \beta^2}\left[\frac{\alpha^2}{\beta} (\cos(\frac{\beta t}{\alpha}) -1) + \sin(\frac{\beta t}{\alpha})\right]$\cite{gao2018self}. Notice that as $\theta(t,-\omega) = -\theta(t,\omega) $ and $g(-\omega) = g(\omega)$, the imaginary part in Eq.~\ref{drift1} goes to zero. Thus,
\begin{equation}\label{drift term}
     r_{p}^{d} = \int_{|\omega|> \omega_{f,b}}\left<\cos(p\theta)\right>g(\omega)d\omega 
\end{equation}
 The expression for $\left<\cos(p\theta)\right>$ (for $p \in \{1,2\}$) can now be readily calculated by noting that $\left<\cos(p\theta)\right> = \frac{1}{T}\int_{0}^{T}\cos(p\theta)dt = \int_{0}^{2\pi}\frac{\cos(p\theta)}{\dot{\theta}}d\theta \text{\LARGE $/$} \int_{0}^{2\pi}\frac{1}{\dot{\theta}}d\theta$
\begin{subequations}\label{average cos}
    \begin{equation} 
    \left<\cos(\theta)\right> = \frac{\beta}{\alpha} \left[ \sqrt{\frac{\beta^2}{\alpha^2} - \frac{\alpha^2}{\beta^2 + \alpha^4}} - \frac{\beta}{\alpha} \right]
    \end{equation}
    \begin{equation} 
    \begin{split}
    \left<\cos(2\theta)\right> &= \left[\frac{\beta^2 - \alpha^4}{\beta^2 + \alpha^4}\right] \times \\
   & \left[ \frac{2\beta(\beta^2 + \alpha^4)}{\alpha^3} \left(\frac{\beta}{\alpha} - \sqrt{\frac{\beta^2}{\alpha^2} - \frac{\alpha^2}{\beta^2 + \alpha^4}} \right) - 1\right]
    \end{split}
\end{equation}
\end{subequations}
We are finally ready to write down a set of self-consistent equations that lets us describe the steady state of the coupled oscillator system governed by Eq.~\ref{mean field equation 2}. For the remainder of the work, we consider the intrinsic frequency to be derived from Lorentzian distribution, $g(\omega) = \frac{1}{\pi}\frac{1}{1 + \omega^{2}}$ with mean zero. Noting that the integrands in Eqs.~\ref{locked term} and \ref{drift term} for $p \in \{1,2\}$ are even, we arrive at,
\begin{subequations}\label{total self consistent}
     \begin{equation} \label{r1 self consistent final}
     \begin{split}
     r_{1} = 2q\int_{0}^{\theta_{f,b}}\cos^{2}(\theta)g(q\sin(\theta))d\theta \\ 
      + 2\int_{\omega_{f,b}}^{\infty}\left<\cos(\theta)\right>g(\omega)d\omega
     \end{split}
     \end{equation}
     \begin{equation}\label{r2 self consistent final}
     \begin{split}
      r_{2} = 2q\int_{0}^{\theta_{f,b}}\cos(\theta)\cos(2\theta)g(q\sin(\theta))d\theta \\ 
      + 2\int_{\omega_{f,b}}^{\infty}\left<\cos(2\theta)\right>g(\omega)d\omega
     \end{split}
     \end{equation}
\end{subequations}

These two equations together describe the steady state of the system.  $r_{1}=0$ is a trivial solution, but to find another branch of nontrivial solutions (for both forward and backward processes), we numerically solve the above set of self-consistent equations. Fig.~\ref{paramspace-mainresult}b presents analytical $r_{1}$ vs. $K_{1}$ curve and, to back up our analytical formalism, simulation results for $(m, K_{2}) = (1,1)$ and $(m, K_{2}) = (3,7)$. As for the simulation protocol, we simulate Eq.~\ref{mean field equation} on a network of $N = 10^{4}$ nodes by splitting Eq.~\ref{mean field equation} into a pair of first-order differential equations and integrating them using the Runge-Kutta 4 algorithm (time-step 0.1). For a chosen value of $m$ and $K_2$, we start with random initial conditions for $\theta (\in [0,2\pi))$ and $\dot\theta (\in [-1,1])$ and $K_1 = 0$. We increase $K_1$ in steps of $\Delta K_1$ ($=0.1$, unless specified otherwise) till $K_{1, max} = 12$ is reached (forward), followed by a decrease till $K_1 = 0$ is reached (backward). For each $K_1$ value, the order parameter values are calculated after discarding transients by averaging over the steady state. For each $K_1$ except the first, the initial conditions are taken as the final state of the previous $K_1$ (commonly referred to as adiabatic change).

Fig.~\ref{paramspace-mainresult}b displays a good agreement with simulation and analytical results. For the forward process, as $K_{1}$ is increased from zero, the system undergoes a first-order phase transition from incoherent to coherent state at a finite critical coupling value($K_{1}^{f}$). However, for the backward process, the system undergoes abrupt desynchronization at a value($K_{1}^{b}$), which is less than $K_{1}^{f}$. Hence hysteresis is observed where the system stays in two different states depending on the initial condition. We point out that when $m$ and $K_{2}$ values are both increased, $K_{1}^{f}$ shifts to the right while $K_{1}^{b}$ shifts to the left, revealing a prolonged hysteresis region. 

A natural question would then be to address the dependency of the forward and backward critical points on  $m$ and $K_{2}$. To analytically obtain the expression for the forward transition point($K_{1}^{f}$), we need to evaluate the limit $q\rightarrow 0^{+}$($r_{1}\rightarrow 0^{+}$) in Eq.~\ref{r1 self consistent final}. 

\begin{figure}[t]
    \centering
    \includegraphics[width=9cm]{Not_req.eps}
    \caption{(Color online) Global order parameter ($r_1$) versus dyadic coupling strength $K_1$ for a) $m = 1$ and different values of triadic coupling $K_2 = $ 0 (orange), 3 (blue), and 5 (red) and b) $K_2 = 2$ and different values of mass $m = $ 3 (orange), 1 (blue), and 0 (red).
    In both subplots, the circles and squares indicate the numerical values for the forward and backward cases, respectively. The dashed and continuous curves represent the analytically calculated values for the forward and backward. cases.}
    \label{Varying K2 and mass}
\end{figure}
 As we take the limit $q\rightarrow 0^{+}$, we see that in Eq.~\ref{average cos}a $\beta/\alpha$ tends to  very high value as compared to $\alpha^2/(\beta^2 + \alpha^4)$. This allows us to perform a Taylor series expansion of Eq.~\ref{average cos}a for $\epsilon = \alpha^2/(\beta^2 + \alpha^4) << 1$ which gives, $\left<\cos(\theta)\right> = \frac{-\alpha^2}{2(\beta^2 + \alpha^4)} + \mathcal{O}(\epsilon^4) \approx \frac{-mq}{2(1+m^2\omega^2)}$. However note that $r_{2}\rightarrow 0^{+}$ in the limit $r_{1}\rightarrow 0^{+}$. When $r_{1} \approx 0$, the parameter $\alpha \rightarrow \infty$ implies that the limit of the integrals for the forward and backward process becomes the same as there exists no bistability in the system. Taking the $\theta_{f,b} = \frac{\pi}{2}$, on dividing both sides of Eq.~\ref{r1 self consistent final} by $q$, and evaluating the limit we have, $\frac{1}{K_{1}^{c}}= \frac{\pi}{2}g(0) - m\int_{0}^{\infty}\frac{1}{1+m^{2}\omega^{2}}g(\omega)d\omega$. After evaluating the integral and rearranging the terms, we end up with $K_{1}^{f} = 2(m+1)$. We see that the forward critical coupling constant purely depends on the mass of the oscillators. Obtaining such a clean analytical expression for $K_{1}^{b}$ is difficult because of the complexity of the integrand of the drift oscillator contribution in Eq.~\ref{r2 self consistent final}. Hence we resort to simulation results to decipher the dependency of $K_{1}^{b}$ on $m$ and $K_{2}$. 
 
 Fig.~\ref{Varying K2 and mass}a illustrates the effect of varying $K_2 (0.0,3.0,5.0)$ for the case of fixed mass $m = 1.0$. The forward critical coupling ($K_1^f$) remains the same for all three cases validating our analytical result that $K_1^f$ is only dependent on $m$. At this $K_1^f (= 4)$, the magnitude of the first-order jump for fixed $m$ is seen to increase with the value of $K_2$. For the backward process, the coherent branch is seen to persist till increasingly smaller values of $K_1$ with an increase in the $K_2$ value, after which the system undergoes an abrupt transition to asynchrony at $K_1^b$. In Fig.~\ref{Varying K2 and mass}b, we study the effect of varying mass ($0.0, 1.0, 3.0$) for the case of fixed $K_2 = 2.0$. As inertia increases, we see that $K_1^f$ shifts to higher values. However, the backward transition point is seen to be constant for all the mass values. In summary, $K_1^f$ shifts to the right with increasing mass irrespective of $K_2$, and $K_1^b$ shifts to the left on increasing $K_2$, irrespective of $m$. Fig.~\ref{Varying K2 and mass}b also has the plots for $m = 0$ case whose analytics is derived in Supplementary Material \cite{SM}. However, we point out that the analytically calculated values of $K_1^f$ do not match exactly with numerical simulations owing to the finite size effects. A detailed study on the same has been done in \cite{sb2014}. 


{\bf{Conclusion:}}
We put forward a generalized analytical framework to study the steady-state behavior of coupled oscillator systems with inertia interacting via polyadic interactions. The analytical predictions, which are backed up by numerical simulation, show a prolonged hysteretic first-order phase transition to a synchronized state. Surprisingly, we find that the effects of $m$ and $K_{2}$ manifest independently on synchronization phenomena with the forward critical coupling constant explicitly depending on mass while the backward critical coupling constant depends exclusively on $K_2$. We have presented the results for triadic interactions; however, it is easy to see that upon adding other power of higher-order interactions, say $\sin(\theta_j + \theta_k - \theta_l -\theta_i)$, as long as the sinusoidal coupling function contains $\theta_i$ term only, nature of mean field will not change  (here, $q=r_1(K_1 + r_2K_2 + r_1^2K_3)$), and rest follows the same and hence is deferred to SM \cite{SM}. Further note that developing the self-consistent method for other choices of higher-order coupling functions, such as $\sin(\theta_j + \theta_k - 2\theta_i)$ along with pairwise coupling proves to be complicated because of the existence of higher order harmonics in the mean-field equation. Our analytical formulation can be further extended to diluted simplicial complexes, which can provide fundamental insights into the dynamics of various real-world complex systems such as power grids.

\begin{acknowledgments}
SJ gratefully acknowledges SERB Power grant SPF/2021/000136. The work is supported by the computational facility received from the Department of Science and Technology (DST), Government of India under FIST scheme (Grant No. SR/FST/PSI-225/2016)
\end{acknowledgments}

\begin{thebibliography}{99}

    \bibitem{boccaletti2006complex} Boccaletti, S., Latora, V., Moreno, Y., Chavez, M., \& Hwang, D. U. {Complex networks: Structure and dynamics.} {\em Physics reports} {\bf 424} (2006).

    \bibitem{kuramoto1975self} Kuramoto, Y. Self-entrainment of a population of coupled non-linear oscillators.( Springer. Berlin Heidelberg. 1975). {\em International Symposium on Mathematical Problems in Theoretical Physics: January 23–29, 1975, Kyoto University (pp. 420-422)}

    \bibitem{sethia2008clustered} Sethia, Gautam C., Abhijit Sen, and Fatihcan M. Atay. {Clustered chimera states in delay-coupled oscillator systems.} {\em Physical review letters} {\bf 100}, 144102 (2008).

    \bibitem{childs2008stability} Childs, L. M., \& Strogatz, S. H. {Stability diagram for the forced Kuramoto model}. {\em Chaos: An Interdisciplinary Journal of Nonlinear Science}, {\bf 18} (4), 043128 (2008).

    \bibitem{omel2012nonuniversal} Omel’Chenko, E., \& Wolfrum, M. Nonuniversal transitions to synchrony in the Sakaguchi-Kuramoto model. {\em Physical review letters}, {\bf 109} (16), 164101 (2012).

    \bibitem{olmi2015chimera} Olmi, S. {Chimera states in coupled Kuramoto oscillators with inertia}. {\em Chaos: An Interdisciplinary Journal of Nonlinear Science}, {\bf 25} (12), 123125 (2015).

    \bibitem{ermentrout1991adaptive} Ermentrout, B. An adaptive model for synchrony in the firefly Pteroptyx malaccae. {\em Journal of Mathematical Biology}, {\bf 29} (6) (1991).

    \bibitem{tanaka1997first} Tanaka, H. A., Lichtenberg, A. J., \& Oishi, S. I. {First order phase transition resulting from finite inertia in coupled oscillator systems}. {\em Physical review letters}, {\bf 78} (11), 2104 (1997). 

    \bibitem{tanaka1997self} Tanaka, H. A., Lichtenberg, A. J., \& Oishi, S. I. Self-synchronization of coupled oscillators with hysteretic responses. {\em Physica D: Nonlinear Phenomena}, {\bf 100} (3-4) (1997).

    \bibitem{ji2013cluster} Ji, P., Peron, T. K. D., Menck, P. J., Rodrigues, F. A., \& Kurths, J. Cluster explosive synchronization in complex networks. {\em Physical review letters}, {\bf 110} (21), 218701 (2013).

    \bibitem{sakyte2011self} Sakyte, E., \& Ragulskis, M. Self-calming of a random network of dendritic neurons. {\em Neurocomputing}, {\bf 74 }(18) (2011).

    \bibitem{watanabe1994constants} Watanabe, S., \& Strogatz, S. H. Constants of motion for superconducting Josephson arrays. {\em Physica D: Nonlinear Phenomena}, {\bf 74} (1994).

    \bibitem{rohden2012self} Rohden, M., Sorge, A., Timme, M., \& Witthaut, D. Self-organized synchronization in decentralized power grids. {\em Physical review letters}, {\bf 109} (6), 064101 (2012).

    \bibitem{dorfler2013synchronization} Dörfler, F., Chertkov, M., \& Bullo, F. Synchronization in complex oscillator networks and smart grids. {\em Proceedings of the National Academy of Sciences}, {\bf 110} (6) (2013).

    \bibitem{filatrella2008analysis} Filatrella, G., Nielsen, A. H., \& Pedersen, N. F. Analysis of a power grid using a Kuramoto-like model. {\em The European Physical Journal B}, {\bf 61} (2008).

    \bibitem{benson2018simplicial} Benson, A. R., Abebe, R., Schaub, M. T., Jadbabaie, A., \& Kleinberg, J. Simplicial closure and higher-order link prediction. {\em Proceedings of the National Academy of Sciences}, {\bf 115} (48) (2018).

    \bibitem{majhi2022dynamics} Majhi, S., Perc, M., \& Ghosh, D. Dynamics on higher-order networks: A review. {\em Journal of the Royal Society Interface}, {\bf 19} (188), 20220043 (2022).

    \bibitem{skardal2020higher} Skardal, P. S., \& Arenas, A. Higher order interactions in complex networks of phase oscillators promote abrupt synchronization switching. {\em Communications Physics}, {\bf 3} (1), 218 (2020).

    \bibitem{xu2021spectrum} Xu, C., \& Skardal, P. S. Spectrum of extensive multiclusters in the Kuramoto model with higher-order interactions. {\em Physical Review Research}, {\bf 3} (1), 013013 (2021).

    \bibitem{strogatz2018nonlinear} Strogatz, S. H. Nonlinear dynamics and chaos with student solutions manual: With applications to physics, biology, chemistry, and engineering.(CRC press. 2018).

    \bibitem{levi1978dynamics} Levi, M., Hoppensteadt, F. C., \& Miranker, W. L. Dynamics of the Josephson junction. {\em  Quarterly of Applied Mathematics}, {\bf 36} (2) (1978).

    \bibitem{guckenheimer2013nonlinear} Guckenheimer, J., \& Holmes, P. Nonlinear oscillations, dynamical systems, and bifurcations of vector fields. (Springer Science \& Business Media, 2013) Vol. 42.

    \bibitem{gao2018self} Gao, J., \& Efstathiou, K. Self-consistent method and steady states of second-order oscillators. {\em Physical Review E}, {\bf 98} (4), 042201 (2018).

    \bibitem{SM} Supplementary Material (SM) at \url{} contains self-consistency derivation for $m=0$ case, and intermediate steps of $m\n 0$ case. It also contains detailed results for inclusion of other higher-order terms. 

    \bibitem{sb2014} Olmi, S., Navas, A., Boccaletti, S., \& Torcini, A. Hysteretic transitions in the Kuramoto model with inertia. {\em Physical Review E}, {\bf 90} (4), 042905 (2014).

\end{thebibliography}

%\printbibliography

\end{document}
